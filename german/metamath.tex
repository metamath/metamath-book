% metamath.tex - Version of 15-Dec-2023
% If you change the date above, also change the "Printed date" below.
% SPDX-License-Identifier: CC0-1.0
%
%                              PUBLIC DOMAIN
%
% This file (specifically, the version of this file with the above date)
% has been released into the Public Domain per the
% Creative Commons CC0 1.0 Universal (CC0 1.0) Public Domain Dedication
% https://creativecommons.org/publicdomain/zero/1.0/
%
% The public domain release applies worldwide.  In case this is not
% legally possible, the right is granted to use the work for any purpose,
% without any conditions, unless such conditions are required by law.
%
% Several short, attributed quotations from copyrighted works
% appear in this file under the "`fair use"' provision of Section 107 of
% the United States Copyright Act (Title 17 of the {\em United States
% Code}).  The public-domain status of this file is not applicable to
% those quotations.
%
% Norman Megill - email: nm(at)alum(dot)mit(dot)edu
%
% David A. Wheeler also donates his improvements to this file to the
% public domain per the CC0.  He works at the Institute for Defense Analyses
% (IDA), but IDA has agreed that this Metamath work is outside its "lane"
% and is not a work by IDA.  This was specifically confirmed by
% Margaret E. Myers (Division Director of the Information Technology
% and Systems Division) on 2019-05-24 and by Ben Lindorf (General Counsel)
% on 2019-05-22.
%
% Georg M. van der Vekens and Alexander W. van der Vekens also donate their
% German translation of this file to the public domain per the CC0.

% This file, 'metamath.tex', is self-contained with everything needed to
% generate the the PDF file 'metamath.pdf' (the _Metamath_ book) on
% standard LaTeX 2e installations.  The auxiliary files are embedded with
% "filecontents" commands.  To generate metamath.pdf file, run these
% commands under Linux or Cygwin in the directory that contains
% 'metamath.tex':
%
%   rm -f realref.sty metamath.bib
%   touch metamath.ind
%   pdflatex metamath
%   pdflatex metamath
%   bibtex metamath
%   makeindex metamath
%   pdflatex metamath
%   pdflatex metamath
%
% The warnings that occur in the initial runs of pdflatex can be ignored.
% For the final run,
%
%   egrep -i 'error|warn' metamath.log
%
% should show exactly these 5 warnings:
%
%   LaTeX Warning: File `realref.sty' already exists on the system.
%   LaTeX Warning: File `metamath.bib' already exists on the system.
%   LaTeX Font Warning: Font shape `OMS/cmtt/m/n' undefined
%   LaTeX Font Warning: Font shape `OMS/cmtt/bx/n' undefined
%   LaTeX Font Warning: Some font shapes were not available, defaults
%       substituted.
%
% Search for "Uncomment" below if you want to suppress hyperlink boxes
% in the PDF output file
%
% TYPOGRAPHICAL NOTES:
% * It is customary to use an en dash (--) to "connect" names of different
%   people (and to denote ranges), and use a hyphen (-) for a
%   single compound name. Examples of connected multiple people are
%   Zermelo--Fraenkel, Schr\"{o}der--Bernstein, Tarski--Grothendieck,
%   Hewlett--Packard, and Backus--Naur.  Examples of a single person with
%   a compound name include Levi-Civita, Mittag-Leffler, and Burali-Forti.
% * Use non-breaking spaces after page abbreviations, e.g.,
%   p.~\pageref{note2002}.
%
% --------------------------- Start of realref.sty -----------------------------
\begin{filecontents}{realref.sty}
% Save the following as realref.sty.
% You can then use it with \usepackage{realref}
%
% This has \pageref jumping to the page on which the ref appears,
% \ref jumping to the point of the anchor, and \sectionref
% jumping to the start of section.
%
% Author:  Anthony Williams
%          Software Engineer
%          Nortel Networks Optical Components Ltd
% Date:    9 Nov 2001 (posted to comp.text.tex)
%
% The following declaration was made by Anthony Williams on
% 24 Jul 2006 (private email to Norman Megill):
%
%   "`I hereby donate the code for realref.sty posted on the
%   comp.text.tex newsgroup on 9th November 2001, accessible from
%   http://groups.google.com/group/comp.text.tex/msg/5a0e1cc13ea7fbb2
%   to the public domain."'
%
\ProvidesPackage{realref}
\RequirePackage[plainpages=false,pdfpagelabels=true]{hyperref}
\def\realref@anchorname{}
\AtBeginDocument{%
% ensure every label is a possible hyperlink target
\let\realref@oldrefstepcounter\refstepcounter%
\DeclareRobustCommand{\refstepcounter}[1]{\realref@oldrefstepcounter{#1}
\edef\realref@anchorname{\string #1.\@currentlabel}%
}%
\let\realref@oldlabel\label%
\DeclareRobustCommand{\label}[1]{\realref@oldlabel{#1}\hypertarget{#1}{}%
\@bsphack\protected@write\@auxout{}{%
    \string\expandafter\gdef\protect\csname
    page@num.#1\string\endcsname{\thepage}%
    \string\expandafter\gdef\protect\csname
    ref@num.#1\string\endcsname{\@currentlabel}%
    \string\expandafter\gdef\protect\csname
    sectionref@name.#1\string\endcsname{\realref@anchorname}%
}\@esphack}%
\DeclareRobustCommand\pageref[1]{{\edef\a{\csname
            page@num.#1\endcsname}\expandafter\hyperlink{page.\a}{\a}}}%
\DeclareRobustCommand\ref[1]{{\edef\a{\csname
            ref@num.#1\endcsname}\hyperlink{#1}{\a}}}%
\DeclareRobustCommand\sectionref[1]{{\edef\a{\csname
            ref@num.#1\endcsname}\edef\b{\csname
            sectionref@name.#1\endcsname}\hyperlink{\b}{\a}}}%
}
\end{filecontents}
% ---------------------------- End of realref.sty ------------------------------

% --------------------------- Start of metamath.bib -----------------------------
\begin{filecontents}{metamath.bib}
@book{Albers, editor = "Donald J. Albers and G. L. Alexanderson",
  title = "Mathematical People",
  publisher = "Contemporary Books, Inc.",
  address = "Chicago",
  note = "[QA28.M37]",
  year = 1985 }
@book{Anderson, author = "Alan Ross Anderson and Nuel D. Belnap",
  title = "Entailment",
  publisher = "Princeton University Press",
  address = "Princeton",
  volume = 1,
  note = "[QA9.A634 1975 v.1]",
  year = 1975}
@book{Barrow, author = "John D. Barrow",
  title = "Theories of Everything:  The Quest for Ultimate Explanation",
  publisher = "Oxford University Press",
  address = "Oxford",
  note = "[Q175.B225]",
  year = 1991 }
@book{Behnke,
  editor = "H. Behnke and F. Backmann and K. Fladt and W. S{\"{u}}ss",
  title = "Fundamentals of Mathematics",
  volume = "I",
  publisher = "The MIT Press",
  address = "Cambridge, Massachusetts",
  note = "[QA37.2.B413]",
  year = 1974 }
@book{Bell, author = "J. L. Bell and M. Machover",
  title = "A Course in Mathematical Logic",
  publisher = "North-Holland",
  address = "Amsterdam",
  note = "[QA9.B3953]",
  year = 1977 }
@inproceedings{Blass, author = "Andrea Blass",
  title = "The Interaction Between Category Theory and Set Theory",
  pages = "5--29",
  booktitle = "Mathematical Applications of Category Theory (Proceedings
     of the Special Session on Mathematical Applications
     Category Theory, 89th Annual Meeting of the American Mathematical
     Society, held in Denver, Colorado January 5--9, 1983)",
  editor = "John Walter Gray",
  year = 1983,
  note = "[QA169.A47 1983]",
  publisher = "American Mathematical Society",
  address = "Providence, Rhode Island"}
@proceedings{Bledsoe, editor = "W. W. Bledsoe and D. W. Loveland",
  title = "Automated Theorem Proving:  After 25 Years (Proceedings
     of the Special Session on Automatic Theorem Proving,
     89th Annual Meeting of the American Mathematical
     Society, held in Denver, Colorado January 5--9, 1983)",
  year = 1983,
  note = "[QA76.9.A96.S64 1983]",
  publisher = "American Mathematical Society",
  address = "Providence, Rhode Island" }
@book{Boolos, author = "George S. Boolos and Richard C. Jeffrey",
  title = "Computability and Log\-ic",
  publisher = "Cambridge University Press",
  edition = "third",
  address = "Cambridge",
  note = "[QA9.59.B66 1989]",
  year = 1989 }
@book{Campbell, author = "John Campbell",
  title = "Programmer's Progress",
  publisher = "White Star Software",
  address = "Box 51623, Palo Alto, CA 94303",
  year = 1991 }
@article{DBLP:journals/corr/Carneiro14,
  author    = {Mario Carneiro},
  title     = {Conversion of {HOL} Light proofs into Metamath},
  journal   = {CoRR},
  volume    = {abs/1412.8091},
  year      = {2014},
  url       = {http://arxiv.org/abs/1412.8091},
  archivePrefix = {arXiv},
  eprint    = {1412.8091},
  timestamp = {Mon, 13 Aug 2018 16:47:05 +0200},
  biburl    = {https://dblp.org/rec/bib/journals/corr/Carneiro14},
  bibsource = {dblp computer science bibliography, https://dblp.org}
}
@article{CarneiroND,
  author    = {Mario Carneiro},
  title     = {Natural Deductions in the Metamath Proof Language},
  url       = {http://us.metamath.org/ocat/natded.pdf},
  year      = 2014
}
@inproceedings{Chou, author = "Shang-Ching Chou",
  title = "Proving Elementary Geometry Theorems Using {W}u's Algorithm",
  pages = "243--286",
  booktitle = "Automated Theorem Proving:  After 25 Years (Proceedings
     of the Special Session on Automatic Theorem Proving,
     89th Annual Meeting of the American Mathematical
     Society, held in Denver, Colorado January 5--9, 1983)",
  editor = "W. W. Bledsoe and D. W. Loveland",
  year = 1983,
  note = "[QA76.9.A96.S64 1983]",
  publisher = "American Mathematical Society",
  address = "Providence, Rhode Island" }
@book{Clemente, author = "Daniel Clemente Laboreo",
  title = "Introduction to natural deduction",
  year = 2014,
  url = "http://www.danielclemente.com/logica/dn.en.pdf" }
@incollection{Courant, author = "Richard Courant and Herbert Robbins",
  title = "Topology",
  pages = "573--590",
  booktitle = "The World of Mathematics, Volume One",
  editor = "James R. Newman",
  publisher = "Simon and Schuster",
  address = "New York",
  note = "[QA3.W67 1988]",
  year = 1956 }
@book{Curry, author = "Haskell B. Curry",
  title = "Foundations of Mathematical Logic",
  publisher = "Dover Publications, Inc.",
  address = "New York",
  note = "[QA9.C976 1977]",
  year = 1977 }
@book{Davis, author = "Philip J. Davis and Reuben Hersh",
  title = "The Mathematical Experience",
  publisher = "Birkh{\"{a}}user Boston",
  address = "Boston",
  note = "[QA8.4.D37 1982]",
  year = 1981 }
@incollection{deMillo,
  author = "Richard de Millo and Richard Lipton and Alan Perlis",
  title = "Social Processes and Proofs of Theorems and Programs",
  pages = "267--285",
  booktitle = "New Directions in the Philosophy of Mathematics",
  editor = "Thomas Tymoczko",
  publisher = "Birkh{\"{a}}user Boston, Inc.",
  address = "Boston",
  note = "[QA8.6.N48 1986]",
  year = 1986 }
@book{Edwards, author = "Robert E. Edwards",
  title = "A Formal Background to Mathematics",
  publisher = "Springer-Verlag",
  address = "New York",
  note = "[QA37.2.E38 v.1a]",
  year = 1979 }
@book{Enderton, author = "Herbert B. Enderton",
  title = "Elements of Set Theory",
  publisher = "Academic Press, Inc.",
  address = "San Diego",
  note = "[QA248.E5]",
  year = 1977 }
@book{Goodstein, author = "R. L. Goodstein",
  title = "Development of Mathematical Logic",
  publisher = "Springer-Verlag New York Inc.",
  address = "New York",
  note = "[QA9.G6554]",
  year = 1971 }
@book{Guillen, author = "Michael Guillen",
  title = "Bridges to Infinity",
  publisher = "Jeremy P. Tarcher, Inc.",
  address = "Los Angeles",
  note = "[QA93.G8]",
  year = 1983 }
@book{Hamilton, author = "Alan G. Hamilton",
  title = "Logic for Mathematicians",
  edition = "revised",
  publisher = "Cambridge University Press",
  address = "Cambridge",
  note = "[QA9.H298]",
  year = 1988 }
@unpublished{Harrison, author = "John Robert Harrison",
  title = "Metatheory and Reflection in Theorem Proving:
    A Survey and Critique",
  note = "Technical Report
    CRC-053.
    SRI Cambridge,
    Millers Yard, Cambridge, UK,
    1995.
    Available on the Web as
{\verb+http:+}\-{\verb+//www.cl.cam.ac.uk/users/jrh/papers/reflect.html+}"}
@TECHREPORT{Harrison-thesis,
        author          = "John Robert Harrison",
        title           = "Theorem Proving with the Real Numbers",
        institution   = "University of Cambridge Computer
                         Lab\-o\-ra\-to\-ry",
        address         = "New Museums Site, Pembroke Street, Cambridge,
                           CB2 3QG, UK",
        year            = 1996,
        number          = 408,
        type            = "Technical Report",
        note            = "Author's PhD thesis,
   available on the Web at
{\verb+http:+}\-{\verb+//www.cl.cam.ac.uk+}\-{\verb+/users+}\-{\verb+/jrh+}%
\-{\verb+/papers+}\-{\verb+/thesis.html+}"}
@book{Herrlich, author = "Horst Herrlich and George E. Strecker",
  title = "Category Theory:  An Introduction",
  publisher = "Allyn and Bacon Inc.",
  address = "Boston",
  note = "[QA169.H567]",
  year = 1973 }
@article{Hindley, author = "J. Roger Hindley and David Meredith",
  title = "Principal Type-Schemes and Condensed Detachment",
  journal = "The Journal of Symbolic Logic",
  volume = 55,
  year = 1990,
  note = "[QA.J87]",
  pages = "90--105" }
@book{Hofstadter, author = "Douglas R. Hofstadter",
  title = "G{\"{o}}del, Escher, Bach",
  publisher = "Basic Books, Inc.",
  address = "New York",
  note = "[QA9.H63 1980]",
  year = 1979 }
@article{Indrzejczak, author= "Andrzej Indrzejczak",
  title = "Natural Deduction, Hybrid Systems and Modal Logic",
  journal = "Trends in Logic",
  volume = 30,
  publisher = "Springer",
  year = 2010 }
@article{Kalish, author = "D. Kalish and R. Montague",
  title = "On {T}arski's Formalization of Predicate Logic with Identity",
  journal = "Archiv f{\"{u}}r Mathematische Logik und Grundlagenfor\-schung",
  volume = 7,
  year = 1965,
  note = "[QA.A673]",
  pages = "81--101" }
@article{Kalman, author = "J. A. Kalman",
  title = "Condensed Detachment as a Rule of Inference",
  journal = "Studia Logica",
  volume = 42,
  number = 4,
  year = 1983,
  note = "[B18.P6.S933]",
  pages = "443-451" }
@book{Kline, author = "Morris Kline",
  title = "Mathematical Thought from Ancient to Modern Times",
  publisher = "Oxford University Press",
  address = "New York",
  note = "[QA21.K516 1990 v.3]",
  year = 1972 }
@book{Klinel, author = "Morris Kline",
  title = "Mathematics, The Loss of Certainty",
  publisher = "Oxford University Press",
  address = "New York",
  note = "[QA21.K525]",
  year = 1980 }
@book{Kramer, author = "Edna E. Kramer",
  title = "The Nature and Growth of Modern Mathematics",
  publisher = "Princeton University Press",
  address = "Princeton, New Jersey",
  note = "[QA93.K89 1981]",
  year = 1981 }
@article{Knill, author = "Oliver Knill",
  title = "Some Fundamental Theorems in Mathematics",
  year = "2018",
  url = "https://arxiv.org/abs/1807.08416" }
@book{Landau, author = "Edmund Landau",
  title = "Foundations of Analysis",
  publisher = "Chelsea Publishing Company",
  address = "New York",
  edition = "second",
  note = "[QA241.L2541 1960]",
  year = 1960 }
@article{Leblanc, author = "Hugues Leblanc",
  title = "On {M}eyer and {L}ambert's Quantificational Calculus {FQ}",
  journal = "The Journal of Symbolic Logic",
  volume = 33,
  year = 1968,
  note = "[QA.J87]",
  pages = "275--280" }
@article{Lejewski, author = "Czeslaw Lejewski",
  title = "On Implicational Definitions",
  journal = "Studia Logica",
  volume = 8,
  year = 1958,
  note = "[B18.P6.S933]",
  pages = "189--208" }
@book{Levy, author = "Azriel Levy",
  title = "Basic Set Theory",
  publisher = "Dover Publications",
  address = "Mineola, NY",
  year = "2002"
}
@book{Margaris, author = "Angelo Margaris",
  title = "First Order Mathematical Logic",
  publisher = "Blaisdell Publishing Company",
  address = "Waltham, Massachusetts",
  note = "[QA9.M327]",
  year = 1967}
@book{Manin, author = "Yu I. Manin",
  title = "A Course in Mathematical Logic",
  publisher = "Springer-Verlag",
  address = "New York",
  note = "[QA9.M29613]",
  year = "1977" }
@article{Mathias, author = "Adrian R. D. Mathias",
  title = "A Term of Length 4,523,659,424,929",
  journal = "Synthese",
  volume = 133,
  year = 2002,
  note = "[Q.S993]",
  pages = "75--86" }
@article{Megill, author = "Norman D. Megill",
  title = "A Finitely Axiomatized Formalization of Predicate Calculus
     with Equality",
  journal = "Notre Dame Journal of Formal Logic",
  volume = 36,
  year = 1995,
  note = "[QA.N914]",
  pages = "435--453" }
@unpublished{Megillc, author = "Norman D. Megill",
  title = "A Shorter Equivalent of the Axiom of Choice",
  month = "June",
  note = "Unpublished",
  year = 1991 }
@article{MegillBunder, author = "Norman D. Megill and Martin W.
    Bunder",
  title = "Weaker {D}-Complete Logics",
  journal = "Journal of the IGPL",
  volume = 4,
  year = 1996,
  pages = "215--225",
  note = "Available on the Web at
{\verb+http:+}\-{\verb+//www.mpi-sb.mpg.de+}\-{\verb+/igpl+}%
\-{\verb+/Journal+}\-{\verb+/V4-2+}\-{\verb+/#Megill+}"}
}
@book{Mendelson, author = "Elliott Mendelson",
  title = "Introduction to Mathematical Logic",
  edition = "second",
  publisher = "D. Van Nostrand Company, Inc.",
  address = "New York",
  note = "[QA9.M537 1979]",
  year = 1979 }
@article{Meredith, author = "David Meredith",
  title = "In Memoriam {C}arew {A}rthur {M}eredith (1904-1976)",
  journal = "Notre Dame Journal of Formal Logic",
  volume = 18,
  year = 1977,
  note = "[QA.N914]",
  pages = "513--516" }
@article{CAMeredith, author = "C. A. Meredith",
  title = "Single Axioms for the Systems ({C},{N}), ({C},{O}) and ({A},{N})
      of the Two-Valued Propositional Calculus",
  journal = "The Journal of Computing Systems",
  volume = 3,
  year = 1953,
  pages = "155--164" }
@article{Monk, author = "J. Donald Monk",
  title = "Provability With Finitely Many Variables",
  journal = "The Journal of Symbolic Logic",
  volume = 27,
  year = 1971,
  note = "[QA.J87]",
  pages = "353--358" }
@article{Monks, author = "J. Donald Monk",
  title = "Substitutionless Predicate Logic With Identity",
  journal = "Archiv f{\"{u}}r Mathematische Logik und Grundlagenfor\-schung",
  volume = 7,
  year = 1965,
  pages = "103--121" }
  %% Took out this from above to prevent LaTeX underfull warning:
  % note = "[QA.A673]",
@book{Moore, author = "A. W. Moore",
  title = "The Infinite",
  publisher = "Routledge",
  address = "New York",
  note = "[BD411.M59]",
  year = 1989}
@book{Munkres, author = "James R. Munkres",
  title = "Topology: A First Course",
  publisher = "Prentice-Hall, Inc.",
  address = "Englewood Cliffs, New Jersey",
  note = "[QA611.M82]",
  year = 1975}
@article{Nemesszeghy, author = "E. Z. Nemesszeghy and E. A. Nemesszeghy",
  title = "On Strongly Creative Definitions:  A Reply to {V}. {F}. {R}ickey",
  journal = "Logique et Analyse (N.\ S.)",
  year = 1977,
  volume = 20,
  note = "[BC.L832]",
  pages = "111--115" }
@unpublished{Nemeti, author = "N{\'{e}}meti, I.",
  title = "Algebraizations of Quantifier Logics, an Overview",
  note = "Version 11.4, preprint, Mathematical Institute, Budapest,
    1994.  A shortened version without proofs appeared in
    {\glqq }Algebraizations of quantifier logics, an introductory overview{\grqq },
    {\em Studia Logica}, 50:485--569, 1991 [B18.P6.S933]"}
@article{Pavicic, author = "M. Pavi{\v{c}}i{\'{c}}",
  title = "A New Axiomatization of Unified Quantum Logic",
  journal = "International Journal of Theoretical Physics",
  year = 1992,
  volume = 31,
  note = "[QC.I626]",
  pages = "1753 --1766" }
@book{Penrose, author = "Roger Penrose",
  title = "The Emperor's New Mind",
  publisher = "Oxford University Press",
  address = "New York",
  note = "[Q335.P415]",
  year = 1989 }
@book{PetersonI, author = "Ivars Peterson",
  title = "The Mathematical Tourist",
  publisher = "W. H. Freeman and Company",
  address = "New York",
  note = "[QA93.P475]",
  year = 1988 }
@article{Peterson, author = "Jeremy George Peterson",
  title = "An automatic theorem prover for substitution and detachment systems",
  journal = "Notre Dame Journal of Formal Logic",
  volume = 19,
  year = 1978,
  note = "[QA.N914]",
  pages = "119--122" }
@book{Quine, author = "Willard Van Orman Quine",
  title = "Set Theory and Its Logic",
  edition = "revised",
  publisher = "The Belknap Press of Harvard University Press",
  address = "Cambridge, Massachusetts",
  note = "[QA248.Q7 1969]",
  year = 1969 }
@article{Robinson, author = "J. A. Robinson",
  title = "A Machine-Oriented Logic Based on the Resolution Principle",
  journal = "Journal of the Association for Computing Machinery",
  year = 1965,
  volume = 12,
  pages = "23--41" }
@article{RobinsonT, author = "T. Thacher Robinson",
  title = "Independence of Two Nice Sets of Axioms for the Propositional
    Calculus",
  journal = "The Journal of Symbolic Logic",
  volume = 33,
  year = 1968,
  note = "[QA.J87]",
  pages = "265--270" }
@book{Rucker, author = "Rudy Rucker",
  title = "Infinity and the Mind:  The Science and Philosophy of the
    Infinite",
  publisher = "Bantam Books, Inc.",
  address = "New York",
  note = "[QA9.R79 1982]",
  year = 1982 }
@book{Russell, author = "Bertrand Russell",
  title = "Mysticism and Logic, and Other Essays",
  publisher = "Barnes \& Noble Books",
  address = "Totowa, New Jersey",
  note = "[B1649.R963.M9 1981]",
  year = 1981 }
@article{Russell2, author = "Bertrand Russell",
  title = "Recent Work on the Principles of Mathematics",
  journal = "International Monthly",
  volume = 4,
  year = 1901,
  pages = "84"}
@article{Schmidt, author = "Eric Schmidt",
  title = "Reductions in Norman Megill's axiom system for complex numbers",
  url = "http://us.metamath.org/downloads/schmidt-cnaxioms.pdf",
  year = "2012" }
@book{Shoenfield, author = "Joseph R. Shoenfield",
  title = "Mathematical Logic",
  publisher = "Addison-Wesley Publishing Company, Inc.",
  address = "Reading, Massachusetts",
  year = 1967,
  note = "[QA9.S52]" }
@book{Smullyan, author = "Raymond M. Smullyan",
  title = "Theory of Formal Systems",
  publisher = "Princeton University Press",
  address = "Princeton, New Jersey",
  year = 1961,
  note = "[QA248.5.S55]" }
@book{Solow, author = "Daniel Solow",
  title = "How to Read and Do Proofs:  An Introduction to Mathematical
    Thought Process",
  publisher = "John Wiley \& Sons",
  address = "New York",
  year = 1982,
  note = "[QA9.S577]" }
@book{Stark, author = "Harold M. Stark",
  title = "An Introduction to Number Theory",
  publisher = "Markham Publishing Company",
  address = "Chicago",
  note = "[QA241.S72 1978]",
  year = 1970 }
@article{Swart, author = "E. R. Swart",
  title = "The Philosophical Implications of the Four-Color Problem",
  journal = "American Mathematical Monthly",
  year = 1980,
  volume = 87,
  month = "November",
  note = "[QA.A5125]",
  pages = "697--707" }
@book{Szpiro, author = "George G. Szpiro",
  title = "Poincar{\'{e}}'s Prize: The Hundred-Year Quest to Solve One
    of Math's Greatest Puzzles",
  publisher = "Penguin Books Ltd",
  address = "London",
  note = "[QA43.S985 2007]",
  year = 2007}
@book{Takeuti, author = "Gaisi Takeuti and Wilson M. Zaring",
  title = "Introduction to Axiomatic Set Theory",
  edition = "second",
  publisher = "Springer-Verlag New York Inc.",
  address = "New York",
  note = "[QA248.T136 1982]",
  year = 1982}
@inproceedings{Tarski, author = "Alfred Tarski",
  title = "What is Elementary Geometry",
  pages = "16--29",
  booktitle = "The Axiomatic Method, with Special Reference to Geometry and
     Physics (Proceedings of an International Symposium held at the University
     of California, Berkeley, December 26, 1957 --- January 4, 1958)",
  editor = "Leon Henkin and Patrick Suppes and Alfred Tarski",
  year = 1959,
  publisher = "North-Holland Publishing Company",
  address = "Amsterdam"}
@article{Tarski1965, author = "Alfred Tarski",
  title = "A Simplified Formalization of Predicate Logic with Identity",
  journal = "Archiv f{\"{u}}r Mathematische Logik und Grundlagenforschung",
  volume = 7,
  year = 1965,
  note = "[QA.A673]",
  pages = "61--79" }
@book{Tymoczko,
  title = "New Directions in the Philosophy of Mathematics",
  editor = "Thomas Tymoczko",
  publisher = "Birkh{\"{a}}user Boston, Inc.",
  address = "Boston",
  note = "[QA8.6.N48 1986]",
  year = 1986 }
@incollection{Wang,
  author = "Hao Wang",
  title = "Theory and Practice in Mathematics",
  pages = "129--152",
  booktitle = "New Directions in the Philosophy of Mathematics",
  editor = "Thomas Tymoczko",
  publisher = "Birkh{\"{a}}user Boston, Inc.",
  address = "Boston",
  note = "[QA8.6.N48 1986]",
  year = 1986 }
@manual{Webster,
  title = "Webster's New Collegiate Dictionary",
  organization = "G. \& C. Merriam Co.",
  address = "Springfield, Massachusetts",
  note = "[PE1628.W4M4 1977]",
  year = 1977 }
@manual{Whitehead, author = "Alfred North Whitehead",
  title = "An Introduction to Mathematics",
  year = 1911 }
@book{PM, author = "Alfred North Whitehead and Bertrand Russell",
  title = "Principia Mathematica",
  edition = "second",
  publisher = "Cambridge University Press",
  address = "Cambridge",
  year = "1927",
  note = "(3 vols.) [QA9.W592 1927]" }
@article{DBLP:journals/corr/Whalen16,
  author    = {Daniel Whalen},
  title     = {Holophrasm: a neural Automated Theorem Prover for higher-order logic},
  journal   = {CoRR},
  volume    = {abs/1608.02644},
  year      = {2016},
  url       = {http://arxiv.org/abs/1608.02644},
  archivePrefix = {arXiv},
  eprint    = {1608.02644},
  timestamp = {Mon, 13 Aug 2018 16:46:19 +0200},
  biburl    = {https://dblp.org/rec/bib/journals/corr/Whalen16},
  bibsource = {dblp computer science bibliography, https://dblp.org} }
@article{Wiedijk-revisited,
  author = {Freek Wiedijk},
  title = {The QED Manifesto Revisited},
  year = {2007},
  url = {http://mizar.org/trybulec65/8.pdf} }
@book{Wolfram,
  author = "Stephen Wolfram",
  title = "Mathematica:  A System for Doing Mathematics by Computer",
  edition = "second",
  publisher = "Addison-Wesley Publishing Co.",
  address = "Redwood City, California",
  note = "[QA76.95.W65 1991]",
  year = 1991 }
@book{Wos, author = "Larry Wos and Ross Overbeek and Ewing Lusk and Jim Boyle",
  title = "Automated Reasoning:  Introduction and Applications",
  edition = "second",
  publisher = "McGraw-Hill, Inc.",
  address = "New York",
  note = "[QA76.9.A96.A93 1992]",
  year = 1992 }

%
%
%[1] Church, Alonzo, Introduction to Mathematical Logic,
% Volume 1, Princeton University Press, Princeton, N. J., 1956.
%
%[2] Cohen, Paul J., Set Theory and the Continuum Hypothesis,
% W. A. Benjamin, Inc., Reading, Mass., 1966.
%
%[3] Hamilton, Alan G., Logic for Mathematicians, Cambridge
% University Press,
% Cambridge, 1988.

%[6] Kleene, Stephen Cole, Introduction to Metamathematics, D.  Van
% Nostrand Company, Inc., Princeton (1952).

%[13] Tarski, Alfred, "A simplified formalization of predicate
% logic with identity," Archiv fur Mathematische Logik und
% Grundlagenforschung, vol. 7 (1965), pp. 61-79.

%[14] Tarski, Alfred and Steven Givant, A Formalization of Set
% Theory Without Variables, American Mathematical Society Colloquium
% Publications, vol. 41, American Mathematical Society,
% Providence, R. I., 1987.

%[15] Zeman, J. J., Modal Logic, Oxford University Press, Oxford, 1973.
\end{filecontents}
% --------------------------- End of metamath.bib -----------------------------


%Book: Metamath
%Author:  Norman Megill Email:  nm at alum.mit.edu
%Author:  David A. Wheeler Email:  dwheeler at dwheeler.com

% A book template example
% http://www.stsci.edu/ftp/software/tex/bookstuff/book.template

\documentclass[leqno]{book} % LaTeX 2e. 10pt. Use [leqno,12pt] for 12pt
% hyperref 2002/05/27 v6.72r  (couldn't get pagebackref to work)
\usepackage[plainpages=false,pdfpagelabels=true]{hyperref}

\usepackage{needspace}     % Enable control over page breaks
%% AV (nicht verfügbar, nicht benötigt?): \usepackage{breqn}         % automatic equation breaking
\usepackage{microtype}     % microtypography, reduces hyphenation

% Packages for flexible tables.  We need to be able to
% wrap text within a cell (with automatically-determined widths) AND
% split a table automatically across multiple pages.
% * "tabularx" wraps text in cells but only 1 page
% * "longtable" goes across pages but by itself is incompatible with tabularx
% * "ltxtable" combines longtable and tabularx, but table contents
%    must be in a separate file.
% * "ltablex" combines tabularx and longtable - must install specially
% * "booktabs" is recommended as a way to improve the look of tables,
%   but doesn't add these capabilities.
% * "tabu" much more capable and seems to be recommended. So use that.

\usepackage{makecell}      % Enable forced line splits within a table cell
% v4.13 needed for tabu: https://tex.stackexchange.com/questions/600724/dimension-too-large-after-recent-longtable-update
\usepackage{longtable}[=v4.13] % Enable multi-page tables  
\usepackage{tabu}          % Multi-page tables with wrapped text in a cell

% You can find more Tex packages using commands like:
% tlmgr search --file tabu.sty
% find /usr/share/texmf-dist/ -name '*tab*'
%
%%%%%%%%%%%%%%%%%%%%%%%%%%%%%%%%%%%%%%%%%%%%%%%%%%%%%%%%%%%%%%%%%%%%%%%%%%%%
% Uncomment the next 3 lines to suppress boxes and colors on the hyperlinks
%%%%%%%%%%%%%%%%%%%%%%%%%%%%%%%%%%%%%%%%%%%%%%%%%%%%%%%%%%%%%%%%%%%%%%%%%%%%
%\hypersetup{
%colorlinks,citecolor=black,filecolor=black,linkcolor=black,urlcolor=black
%}
%
\usepackage{realref}

% Restarting page numbers: try?
%   \printglossary
%   \cleardoublepage
%   \pagenumbering{arabic}
%   \setcounter{page}{1}    ???needed
%   \include{chap1}

% not used:
% \def\R2Lurl#1#2{\mbox{\href{#1}\texttt{#2}}}

\usepackage{amssymb}

% Version 1 of book: margins: t=.4, b=.2, ll=.4, rr=.55
% \usepackage{anysize}
% % \papersize{<height>}{<width>}
% % \marginsize{<left>}{<right>}{<top>}{<bottom>}
% \papersize{9in}{6in}
% % l/r 0.6124-0.6170 works t/b 0.2418-0.3411 = 192pp. 0.2926-03118=exact
% \marginsize{0.7147in}{0.5147in}{0.4012in}{0.2012in}

\usepackage{anysize}
% \papersize{<height>}{<width>}
% \marginsize{<left>}{<right>}{<top>}{<bottom>}
\papersize{9in}{6in}
% l/r 0.85in&0.6431-0.6539 works t/b ?-?
%\marginsize{0.85in}{0.6485in}{0.55in}{0.35in}
\marginsize{0.8in}{0.65in}{0.5in}{0.3in}

% \usepackage[papersize={3.6in,4.8in},hmargin=0.1in,vmargin={0.1in,0.1in}]{geometry}  % page geometry
\usepackage{special-settings}

\usepackage{ngerman}
%\addto
\captionsngerman{% Replace "ngerman" with the language you use
%not needed
%  \renewcommand{\contentsname}
%    {Inhalt}
% \renewcommand{\chaptername}
%   {Kapitel}
% \renewcommand{\appendixname}
%   {Anhang}
%  \renewcommand{\bibliographyname}
%    {Bibliographie}
% \renewcommand{\indexname}
%   {Register}
}

\raggedbottom
\makeindex

\begin{document}
% Discourage page widows and orphans:
\clubpenalty=300
\widowpenalty=300

%%%%%%% load in AMS fonts %%%%%%% % LaTeX 2.09 - obsolete in LaTeX 2e
%\input{amssym.def}
%\input{amssym.tex}
%\input{c:/texmf/tex/plain/amsfonts/amssym.def}
%\input{c:/texmf/tex/plain/amsfonts/amssym.tex}

\bibliographystyle{plain}
\pagenumbering{roman}
\pagestyle{headings}

\thispagestyle{empty}

\hfill
\vfill

\begin{center}
{\LARGE\bf Metamath} \\
\vspace{1ex}
{\large Eine Computersprache für mathematische Beweise} \\
\vspace{7ex}
{\large Norman Megill} \\
\vspace{7ex}
mit umfangreichen Überarbeitungen durch \\
\vspace{1ex}
{\large David A. Wheeler} \\
\vspace{7ex}
Deutsche Übersetzung von \\
\vspace{1ex}
{\large Georg M. van der Vekens und Alexander W. van der Vekens} \\
\vspace{7ex}
% Printed date. If changing the date below, also fix the date at the beginning.
15.12.2023
\end{center}

\vfill
\hfill

\newpage
\thispagestyle{empty}

\hfill
\vfill

\begin{center}
$\sim$\ {\sc Public Domain}\ $\sim$

\vspace{2ex}
Dieses Buch (einschließlich seiner späteren Überarbeitungen und Übersetzungen) wurde von Norman Megill gemäß der 'Creative Commons CC0 1.0 Universal (CC0 1.0) Public Domain Dedication' in die Public Domain veröffentlicht. 
David A. Wheeler, Georg M. van der Vekens und Alexander W. van der Vekens haben selbiges getan. Diese Public Domain Veröffentlichung gilt weltweit. Für den Fall, dass dies rechtlich nicht möglich ist, wird das Recht eingeräumt, das Werk für jeden Zweck zu nutzen, ohne irgendwelche Bedingungen, es sei denn, solche Bedingungen sind gesetzlich vorgeschrieben.
Siehe \url{https://creativecommons.org/publicdomain/zero/1.0/}.

\vspace{3ex}
In diesem Buch erscheinen mehrere kurze, gekennzeichnete Zitate aus urheberrechtlich geschützten Werken gemäß der "`fair use"'-Bestimmung von Abschnitt 107 des United States Copyright Act (Titel 17 des {\em United States Code}). Der Status der Public Domain ist auf diese Zitate nicht anwendbar.

\vspace{3ex}
Alle in diesem Buch verwendeten Warenzeichen sind Eigentum der jeweiligen Inhaber.

% QA76.9.L63.M??

% \vspace{1ex}
%
% \vspace{1ex}
% {\small Permission is granted to make and distribute verbatim copies of this
% book
% provided the copyright notice and this
% permission notice are preserved on all copies.}
%
% \vspace{1ex}
% {\small Permission is granted to copy and distribute modified versions of this
% book under the conditions for verbatim copying, provided that the
% entire
% resulting derived work is distributed under the terms of a permission
% notice
% identical to this one.}
%
% \vspace{1ex}
% {\small Permission is granted to copy and distribute translations of this
% book into another language, under the above conditions for modified
% versions,
% except that this permission notice may be stated in a translation
% approved by the
% author.}
%
% \vspace{1ex}
% %{\small   For a copy of the \LaTeX\ source files for this book, contact
% %the author.} \\
% \ \\
% \ \\

\vspace{7ex}
% ISBN: 1-4116-3724-0 \\
% ISBN: 978-1-4116-3724-5 \\
ISBN: 978-0-359-70223-7 \\
{\ } \\
Lulu Press \\
Morrisville, North Carolina\\
USA


\hfill
\vfill

Norman Megill\\ 93 Bridge St., Lexington, MA 02421 \\
E-Mail Adresse: \texttt{nm{\char`\@}alum.mit.edu} \\
\vspace{7ex}
David A. Wheeler \\
E-Mail Adresse: \texttt{dwheeler{\char`\@}dwheeler.com} \\
% See notes added at end of Preface for revision history. \\
% For current information on the Metamath software see \\
\vspace{7ex}
\url{http://metamath.org}
\end{center}

\hfill
\vfill

{\parindent0pt%
\footnotesize{%
Titelseite: Aleph Null ($\aleph_0$) ist das Symbol für die erste unendliche Kardinalzahl, entdeckt von Georg Cantor im Jahr 1873. Wir verwenden ein rotes Aleph Null (mit dunklem Umriss und goldenem Schimmer) als das Logo für Metamath.
Urheber: Norman Megill (1994) und Giovanni Mascellani (2019),
Public Domain.%
\index{aleph null}%
\index{Metamath!Logo}\index{Cantor, Georg}\index{Mascellani, Giovanni}}}

\newpage
\thispagestyle{empty}

\hfill
\vfill

\begin{center}
{\it Diese Übersetzung ist Norman Dwight Megill,\\
	 dem Erfinder von Metamath, gewidmet.}
\end{center}

\vfill
\hfill

\newpage

\tableofcontents
%\listoftables

\chapter*{Vorwort}
\markboth{Vorwort}{Vorwort}
\addcontentsline{toc}{section}{Vorwort}


% (For current information, see the notes added at the
% end of this preface on p.~\pageref{note2002}.)

\subsubsection{Übersicht}

Metamath\index{Metamath} ist eine Computersprache und ein zugehöriges Computerprogramm zur Archivierung, Verifikation und Untersuchung mathematischer Beweise auf einer sehr detaillierten Ebene. Die Metamath-Sprache enthält keine Mathematik an sich, sondern betrachtet alle mathematischen Aussagen als reine Folgen von Symbolen. Bei der Nutzung von Metamath werden bestimmte, spezielle Symbolsequenzen (Axiome) vorgegeben, die Metamath sagen, welche Schlussfolgerungsregeln erlaubt sind. Metamath ist nicht auf ein bestimmtes mathematisches Gebiet beschränkt.  Die Metamath-Sprache ist einfach und robust, sie besitzt so gut wie keine fest verdrahtete Syntax. 
Wir\footnote{Sofern nicht anders angegeben, beziehen sich die Worte "`Ich"', "`mich"' und "`mein"' auf Norman Megill\index{Megill, Norman}, während "`wir"', "`uns"' und "`unser"' auf Norman Megill und David A. Wheeler\index{Wheeler, David A.} beziehen.}
glauben, dass sie vielleicht den einfachst möglichen Ansatz bietet, mit dem im Wesentlichen die gesamte Mathematik mit absoluter Strenge ausgedrückt werden kann.

% index test
%\newcommand{\nn}[1]{#1n}
%\index{aaa@bbb}
%\index{abc!def}
%\index{abd|see{qqq}}
%\index{abe|nn}
%\index{abf|emph}
%\index{abg|(}
%\index{abg|)}

Mit der Metamath-Sprache können formale oder mathematische Systeme\index{formales System}\footnote{Ein formales oder mathematisches System besteht aus einer Sammlung von Symbolen (solche wie $2$, $4$, $+$ und $=$), Syntaxregeln, die beschreiben, wie Symbole kombiniert werden können um einen gültigen Ausdruck zu formen (solch ein Ausdruck wird auch 'wohlgeformte Formel', engl. 'well formed formula', genannt, oder kurz {\em wff}, gesprochen "`whiff"'), einige wffs (Axiome genannt), mit denen begonnen wird, und Schlussfolgerungsregeln, die beschreiben, wie Theoreme aus den Axiomen abgeleitet (bewiesen) werden können. Ein Theorem ist eine mathematische Tatsache so wie $2+2=4$. Streng genommen muss selbst solch eine offensichtliche Tatsache anhand von Axiomen bewiesen werden, um von einem Mathematiker formal akzeptiert werden zu können.}\index{Theorem}\index{Axiom}\index{Regel}\index{wohlgeformte Formel (wff)} aufgebaut werden, die Schlussfolgerungen aus Axiomen umfassen.  Obwohl zusammen mit Metamath eine Datenbasis bereitgestellt wird, die einen empfohlenen Satz von Axiomen für die Standardmathematik enthält, könnte man nach Belieben eigene Symbole, Syntax, Axiome, Regeln und Definitionen vorgeben.

Der Name "`Metamath"' wurde gewählt um anzudeuten, dass die Sprache ein Mittel zur {\em Beschreibung} der Mathematik und nicht die Mathematik {\em selbst} ist. Tatsächlich ist in gewissem Sinne jede mathematische Sprache metamathematisch. Auf Papier geschriebene oder in einem Computer gespeicherte Symbole sind nicht die Mathematik selbst, sondern vielmehr eine Möglichkeit, mathematische Gegebenheiten und Zusammenhänge auszudrücken. Beispielsweise sind "`7"' und "`VII"' Symbole um die Zahl sieben in arabischen und römischen Ziffern zu bezeichnen; keines von beiden Symbolen {\em ist} die Zahl sieben.

Wenn Sie in der Lage sind, Computerprogramme zu verstehen und zu schreiben, sollten Sie in der Lage sein, abstrakte Mathematik mit Hilfe von Metamath nachzuvollziehen. In Verbindung mit Standard-Lehrbüchern kann Metamath Sie Schritt für Schritt zu einem Verständnis der abstrakten Mathematik von einem sehr rigorosen Standpunkt aus führen, auch wenn Sie keine formale Ausbildung in abstrakter Mathematik haben. Sobald Sie die grundlegenden Konzepte verstanden haben, bietet Metamath Ihnen durch die Verwendung einer einzigen, konsistenten Notation zur Darstellung von Beweisen die Möglichkeit, allen Beweisschritten sofort zu folgen und im Detail zu verstehen, und das selbst in Ihnen völlig unbekannten Bereichen.

Natürlich führt die bloße Fähigkeit einem Beweis zu folgen nicht unbedingt zu einer intuitive Vertrautheit mit der Mathematik. Das Auswendiglernen der Schachregeln gibt Ihnen nicht die Fähigkeit, die Spielweise eines Meisters zu würdigen, und zu wissen wie die Noten einer Partitur den Klaviertasten zugeordnet sind, gibt Ihnen nicht die Fähigkeit, in Ihrem Kopf zu hören, wie sich diese anhören würde. Aber jede dieser Tätigkeiten kann ein erster Schritt für den Einstieg in ein neues Themengebiet sein.

Metamath erlaubt es Ihnen, Beweise so zu erforschen, dass Sie den Beweis jedes Theorems, auf das in einem Beweisschritt verwiesen wird, wiederum im Detail betrachten können, und das immer weiter bis hin zu den zugrunde liegenden Axiomen der Logik und der Mengenlehre (im Falle der mitgelieferten Datenbasis für die Mengenlehre). Während Metamath nicht das nur durch Übung und harte Arbeit zu erlangende Verständnis der Mathematik auf höherer Ebene ersetzen kann, so hilft die Möglichkeit zu sehen, wie Lücken in einem Beweis geschlossen werden, den Lernprozess zu beschleunigen und Ihnen Zeit zu sparen, wenn Sie nicht selbst bei einem Beweis weiterkommen.

Die Metamath-Sprache zerlegt einen mathematischen Beweis in seine kleinstmöglichen Teile. Diese können wie Puzzleteile zusammengesetzt werden, und zwar so, dass korrekte und absolut strenge Mathematik entsteht.

Die Natur von Metamath erzwingt ein sehr präzises mathematisches Denken, ähnlich dem, das beim Schreiben eines Computerprogramms erforderlich ist. Ein entscheidender Unterschied ist jedoch, dass ein Beweis, sobald er (durch das Metamath-Programm) als korrekt verifiziert wurde, definitiv korrekt ist; er kann niemals einen versteckten "`Fehler"' haben. Nachdem Sie sich an die Strenge und Genauigkeit von Metamath gewöhnt haben, könnten Sie sogar versucht sein die Haltung einzunehmen, dass ein Beweis niemals als korrekt angesehen werden sollte, bevor er nicht von einem Computer verifiziert wurde, genauso wie Sie einem Ausrechnen im Kopf oder auf Papier nicht völlig vertrauen, bis Sie die Berechnung auf einem Taschenrechner nachgeprüft haben.

Meine Zielvorstellung für Metamath war ein System zur Beschreibung und Verifikation der Mathematik, das vollständig universell und dennoch konzeptionell so einfach wie möglich ist. Da ich die Mathematik von einem axiomatischen, formalen Standpunkt aus betrachte, wollte ich erreichen, dass Metamath in der Lage ist, mit fast jedem mathematischen System umzugehen. Das ist nicht gerade einfach, aber zumindest im Prinzip möglich und hoffentlich anwendungstauglich. Ich wollte, dass es Beweise mit absoluter Strenge verifiziert, und aus diesem Grund ist Metamath eher eine "`nur-kompilierbare"' Sprache als eine algorithmische oder Turing-Maschinensprache (wie Pascal, C, Prolog, Mathematica, usw.).  Mit anderen Worten, eine in der Metamath-Sprache geschriebene Datenbasis "`macht"' nichts; sie stellt lediglich mathematisches Wissen dar und erlaubt es, dieses Wissen als korrekt zu verifizieren.  Ein Programm, das in einer algorithmischen Sprache geschrieben ist, kann potentiell versteckte Programmfehler(Bugs)\index{Programmfehler}\index{Bug} haben und möglicherweise auch schwer zu verstehen sein.  Aber jedes Eintrag in einer Metamath-Datenbasis muss konsistent mit den vorherigen Inhalt der Datenbasis sein, nach einfachen, festen Regeln.
Wenn eine Datenbasis als korrekt verifiziert wurde,\footnote{Dies beinhaltet die Verifikation, dass eine sequentielle Liste von Beweisschritten zu dem angegebenen Theorem führt.} dann ist der mathematische Inhalt korrekt, wenn der Verifizierer korrekt ist und die Axiome korrekt sind.
Das Verifikationsprogramm könnte zwar prinzipiell falsch sein, der verwendete Verifikationsalgorithmus ist jedoch relativ einfach, so dass es unwahrscheinlich ist, dass er in einem Metamath-Programm falsch implementiert wird. Außerdem gibt es mehr als ein Dutzend Verifizierer für die Metamath-Datenbasen, geschrieben von verschiedenen Programmierern in unterschiedlichen Programmiersprachen, so dass diese unterschiedlichen Verifizierer als unabhängige Prüfer einer Datenbasis fungieren können.
Die meistgenutzte Metamath-Datenbasis, der Metamath Proof Explorer (auch bekannt als \texttt{set.mm}\index{Mengenlehre-Datenbasis (\texttt{set.mm})}%
\index{Metamath Proof Explorer}), wird derzeit von vier verschiedenen Metamath-Verifikationsprogrammen verifiziert, die von vier unterschiedlichen Personen in vier verschiedenen Programmiersprachen geschrieben wurden, einschließlich des originalen Metamath-Programms, das in diesem Buch beschrieben wird.
Deshalb könnten die einzig möglichen "`Fehler"' in der Formulierung der Axiome liegen, zum Beispiel, wenn die Axiome inkonsistent sind (ein berühmtes Problem, das durch den Gödelschen Unvollständigkeitssatz\index{Gödelscher Unvollständigkeitssatz} als unlösbar aufgezeigt wurde).
Reale mathematische Systeme haben jedoch nur sehr wenige Axiome, und diese können sorgfältig studiert werden.
All dies bietet eine außerordentlich hohe Sicherheit, dass die geprüfte Datenbasis in der Tat korrekt ist.

Das Metamath-Programm beweist Theoreme nicht automatisch, sondern ist darauf ausgelegt, Beweise zu überprüfen, die ihm bereitgestellt werden.
Die Metamath zugrunde liegende Sprache ist völlig allgemein und hat keine eingebaute, vorgefasste Vorstellung über Ihr formales System, seine Logik oder seine Syntax.
Für die Konstruktion von Beweisen verfügt das Metamath-Programm über einen Beweis-Assistenten\index{Beweis-Assistent}, der Sie beim Ausfüllen einiger Details eines Beweisschritts unterstützt, und Ihnen die Möglichkeiten bei jedem Schritt zeigt.
Weiterhin verifiziert er den Beweis, während Sie ihn erstellen; den Beweis führen müssen Sie aber immer noch selbst.

Es gibt viele andere Programme, die Informationen in der Metamath-Sprache verarbeiten oder erzeugen können, und es werden immer mehr geschrieben. Das liegt zum Teil daran, dass die Metamath-Sprache selbst sehr einfach ist und mit Absicht leicht automatisch verarbeitet werden kann.
Einige Programme, wie z. B. \texttt{mmj2}\index{mmj2}, enthalten einen Beweis-Assistenten, der einige Beweisschritte automatisiert erstellen kann, die über das hinausgehen, was das Metamath-Programm kann.
Mario Carneiro hat einen Algorithmus entwickelt, der Beweise im OpenTheory-Austauschformat, welches aus und in jede einzelne Beweissprache der HOL-Familie (HOL4, HOL Light, ProofPower, und Isabelle) übersetzt werden kann, in die Metamath Sprache \cite{DBLP:journals/corr/Carneiro14}\index{Carneiro, Mario} überführt.
Daniel Whalen hat Holophrasm entwickelt, das automatisch viele Metamath-Beweise mithilfe von Ansätzen des maschinellen Lernens\index{maschinelles Lernen}\index{künstliche Intelligenz} (einschließlich mehrerer neuronaler Netze) beweisen kann\cite{DBLP:journals/corr/Whalen16}\index{Whalen, Daniel}.
Eine ausführliche Besprechung dieser anderen Programme würde jedoch den Rahmen dieses Buches sprengen.

Wie die meisten Computersprachen verwendet die Metamath\index{Metamath}-Sprache den
Standardzeichensatz ({\sc ascii}), der auf jeder Computertastatur zur Verfügung steht. Daher kann sie viele der speziellen Symbole, die Mathematiker verwenden, nicht direkt darstellen. 
Eine nützliche Eigenschaft des Metamath-Programms ist sein Fähigkeit, die von ihm verwendete Notation in die Textsatzsprache \LaTeX\ umzuwandeln.\index{latex@{\LaTeX}}
Mit dieser Funktion können Sie die von Ihnen definierten ASCII-Tokens in standardmäßig verwendete mathematische Symbole umwandeln, so dass Sie am Ende Symbole und Formeln erhalten, mit denen Sie vertraut sind, anstelle der etwas kryptischen {\sc ascii}-Darstellungen davon.
Das Metamath-Programm kann auch HTML\index{HTML} generieren, was die Veröffentlichung von Ergebnissen im Internet vereinfacht und die Bereitstellung weiterer Informationen zu einem Thema über Hypertext-Links ermöglicht.

Metamath ist wahrscheinlich konzeptionell anders als alles, was Sie bisher gesehen haben, und einige Aspekte sind vielleicht etwas gewöhnungsbedürftig. Dieses Buch wird Ihnen bei der Entscheidung helfen, ob Metamath Ihren speziellen Bedürfnissen entspricht.


\subsubsection{Was Sie erwartet}

Es ist wichtig, dass Sie verstehen, was Metamath\index{Metamath} ist und was nicht. Wie bereits erwähnt, ist das Metamath-Programm {\em kein} automatischer Theorembeweiser, sondern vielmehr ein Beweisprüfer. Die Entwicklung einer Datenbasis kann eine langwierige, harte Arbeit sein, vor allem wenn man die Beweise so kurz wie möglich halten möchte. Aber es wird einfacher, wenn man bereits eine Sammlung nützlicher Theoreme aufgebaut hat. Der Zweck von Metamath ist es einfach, die bestehende Mathematik in einer absolut strengen, durch Computer überprüfbare Weise zu dokumentieren, nicht um direkt bei der Schaffung neuer mathematischer Sätze oder Erkenntnisse zu helfen.  Es ist auch keine magische Lösung für das Erlernen abstrakter Mathematik, obwohl es hilfreich sein kann, die implizite Strenge hinter dem zu sehen, was man aus den Lehrbüchern lernt. Außerdem erhält man Hinweise um Beweise auszuarbeiten, bei denen man sonst nicht weiterkommt.

Bis zum Zeitpunkt der Erstellung dieses Buches wurde bereits eine umfangreiche Datenbasis für die Mengenlehre entwickelt, die eine Grundlage für viele Bereiche der Mathematik bietet. Aber es ist noch viel mehr Arbeit nötig, um nützliche Datenbasen für andere Bereiche zu entwickeln.

Metamath\index{Metamath} "`kennt keine Mathematik"'; es bietet lediglich Rahmenbedingungen zur Formulierung von Mathematik. Sein Sprachumfang ist sehr klein: man kann zwei Arten von Symbolen definieren, nämlich Konstanten\index{Konstante} und Variablen\index{Variable}.
Das Einzige, was Metamath beherrscht, ist das Ersetzen von Symbolen durch Zeichenketten für die Variablen\index{Substitution!Variable}\index{Variablensubstitution} in einem Ausdruck, der auf Anweisungen basiert, die man in einem Beweis angibt, vorbehaltlich bestimmter Einschränkungen, die man für die Variablen festlegt.  Sogar die dezimale Darstellung einer Zahl ist lediglich eine Folge von bestimmten Konstanten (Ziffern), die zusammengenommen einem beliebigen mathematischen Objekt entsprechen, das man für sie in einem bestimmten Kontext definiert. Im Gegensatz zu anderen Computersprachen wird tatsächlich keine Zahl explizit im Computer gespeichert.
In einem Beweis gibt man Metamath vor, welche Symbol-Substitutionen in vorherigen Axiomen oder Theoremen vorzunehmen sind, und fügt eine Folge von solchen Substitutionen zusammen, um das gewünschte Theorem zu erhalten.  Diese Art der Symbolmanipulation erfasst das Wesentliche der Mathematik auf einer präaxiomatischen Ebene.


\subsubsection{Metamath und mathematische Literatur}

In der Literatur für höhere Mathematik werden die Beweise gewöhnlich in Form von kurzen Skizzen dargestellt, denen oft nur ein Experte folgen kann.  Dies resultiert zum Teil aus dem Wunsch nach Kürze, aber es wäre auch unklug (selbst wenn es praktisch möglich wäre) Beweise mit allen formalen Details zu präsentieren, da das Gesamtbild verloren gehen würde.\index{formaler Beweis}

Eine Lösung\label{envision}, so wie ich sie mir vorstelle, besteht aus einer Kombination des traditionellen kurzen, informellen Beweises in gedruckter Form, begleitet von einem vollständigen formalen Beweis, der in einer Computerdatenbasis gespeichert ist. Dies würde es ermöglichen, dass die Mathematik für den Experten akzeptabel bleibt, aber auch dem Nichtspezialisten zugänglich ist.
In Analogie zu einem Computerprogramm kann man den informellen Beweis als Pseudocode ansehen, der die übergreifenden Schlussfolgerungen und den Inhalt des Beweises beschreibt, während die Computerdatenbasis mit dem tatsächlichen Programmcode verglichen werden kann, die jedem, auch einem Laien, die Möglichkeit bietet, den Beweis so detailliert wie gewünscht nachzuvollziehen, indem man in immer tiefere Schichten von Theoremen bis hin zu den Axiomen der Theorie vordringt (wie bei Unterprogrammen, die wiederum andere Unterprogramme aufrufen).  Darüber hinaus hätte die Computerdatenbasis den Vorteil, dass sie die absolute Sicherheit bietet, dass der Beweis korrekt ist, da jeder Schritt automatisch überprüft werden kann.

Neben Metamath gibt es mehrere andere Ansätze für ein Projekt wie dieses.  Abschnitt~\ref{proofverifiers} erörtert einige davon.

Ein hehres Ziel wäre für uns eine Datenbasis mit Hunderttausenden von Theoremen und ihren durch Computer überprüfbaren Beweisen, die einen bedeutenden Teil der bekannten Mathematik umfasst und für den sofortigen Zugriff zur Verfügung stellt.
Diese würden von mehreren unabhängig voneinander implementierten Verifizierern vollständig geprüft werden, um ein extrem hohes Maß an Vertrauen in die vollständige Korrektheit der Beweise herzustellen.
Die Datenbasis würde es allen ermöglichen, jegliche interessierende Details zu untersuchen, so dass man jeden gewünschten Teil eines Beweises bestätigen kann.
Ob Metamath die richtige Wahl ist, bleibt abzuwarten, aber im Prinzip glauben wir, dass sie hinreichend angemessen ist.


\subsubsection{Formalismus}

In den letzten fünfzig Jahren hat eine Gruppe französischer Mathematiker, die unter dem Pseudonym Bourbaki\index{Bourbaki, Nicolas} zusammenarbeiten, eine Reihe von Monographien verfasst, die versuchen, große Teile der Mathematik konsequent von den Grundlagen her zu formalisieren.  Einerseits hat ein solcher Versuch sicherlich seine Vorzüge; andererseits ist das Bourbaki-Projekt wegen seiner "`Scholastik"' und "`Hyperaxiomatik"', welche die intuitiven, zu den Ergebnissen führenden Schritte verbergen, kritisiert worden \cite[S.~191]{Barrow}\index{Barrow, John D.}.

Metamath treibt diese Philosophie ungeniert auf die Spitze und ist zweifellos der gleichen Art von Kritik ausgesetzt.  Nichtsdestotrotz denke ich, dass in Verbindung mit konventionellen Ansätzen der Mathematik Metamath einen nützlichen Zweck erfüllen kann.  Der Ansatz von Bourbaki ist im Wesentlichen pädagogisch und verlangt vom Leser, sich mit jedem Detail in einer sehr großen Hierarchie vertraut zu machen, bevor er oder sie zum nächsten Schritt übergehen kann.  Der Unterschied zu Metamath besteht darin, dass der "`Leser"' (Benutzer) weiß, dass alle Details in seiner Computerdatenbasis enthalten sind, die bei Bedarf abgerufen werden können; es wird nicht verlangt, dass der Benutzer alles weiß, sondern es werden ihm komfortabel die Teile zur Verfügung gestellt, die von Interesse sind.  Da der Umfang des gesamten mathematischen Wissens immer größer wird, kann kein Einzelner seine Gesamtheit in voller Tiefe überblicken. Metamath kann alle Fragen über die Gültigkeit eines beliebigen Teils des Wissens abschließend klären und Zweifel ausräumen, und kann im Prinzip jeden Teil davon für einen Nicht-Spezialisten zugänglich machen.


\subsubsection{Eine persönliche Anmerkung}
Warum habe ich Metamath\index{Metamath} entwickelt?  Ich mag abstrakte Mathematik, aber manchmal verirre ich mich in einer Flut von Definitionen und verliere das Vertrauen in die Korrektheit meiner Beweise.  Oder ich erreiche einen Punkt, an dem ich aus den Augen verliere, wie alles, was ich tue, mit den Axiomen zusammenhängt, auf denen eine Theorie beruht. Ich habe manchmal den Verdacht, dass unterwegs ein übersehenes implizites Axiom versehentlich eingebracht wurde (wie es historisch mit der euklidischen Geometrie\index{euklidische Geometrie}, deren Auslassung des Pasch'schen Axioms\index{Axiom von Pasch} für 2000 Jahre unbemerkt blieb \cite[p.~160]{Davis}!).
Ich bin auch etwas faul und möchte den Aufwand vermeiden Lücken in informellen Beweisen, die "`dem Leser überlassen"' werden, erneut selbst nachzuprüfen. Ich ziehe es vor, sie nur einmal herauszufinden und mich nicht ein Jahr später erneut durch dieselbe Frustration zu kämpfen, wenn ich vergesse, was ich getan habe.  Metamath bietet eine bessere Möglichkeit zur Wiederherstellung meiner Bemühungen als Papierfetzen, die ich nicht mehr entziffern kann.  Aber vor allem finde ich die Idee sehr reizvoll, mathematisches Wissen in einer Computerdatenbasis zu archivieren, die Präzision, Gewissheit und die Eliminierung menschlicher Fehler gewährleistet.


\subsubsection{Anmerkung zu Bibliographie und Index}

Die Bibliographie enthält in der Regel die Library of Congress-Klassifikation für ein Werk, damit Sie es in einem Regal einer Universitätsbibliothek leichter finden.  Der Index enthält Verweise auf Seiten, auf denen die Werke der Autoren zitiert werden, auch wenn die Namen der Autoren vielleicht nicht auf diesen Seiten erscheinen.


\subsubsection{Danksagungen}

Dank gebührt zunächst meiner Frau Deborah (die am 4. September 1998 verstorben ist), für die Kritik am Manuskript, aber vor allem für ihre Geduld und Unterstützung.  Ich möchte auch Joe Wright, Richard Becker, Clarke Evans, Buddha Buck, und Jeremy Henty für hilfreiche Kommentare danken.  Etwaige Fehler, Auslassungen und andere Unzulänglichkeiten liegen natürlich in meiner Verantwortung.


\subsubsection{Notiz hinzugefügt am 22. Juni 2005}\label{note2002}

Die ursprüngliche, unveröffentlichte Version dieses Buches wurde 1997 geschrieben und über das Internet verbreitet.  Die vorliegende Ausgabe wurde aktualisiert, um das aktuelle Metamath-Programm und die Datenbasen sowie aktuellere {\sc url}s für Internet-Seiten wiederzugeben.  Dank an Josh Purinton\index{Purinton, Josh}, One Hand Clapping, Mel L.\ O'Cat und Roy F. Longton für das Aufzeigen von typografischen und anderen Fehler.  Ich habe auch von zahlreichen Diskussionen mit Raph Levien\index{Levien, Raph} profitiert, der Metamaths Philosophie der Strenge erweitert hat, was in seine Beweissprache {\em Ghilbert}\index{Ghilbert} (\url{http://ghilbert.org}) resultierte.

Robert (Bob) Solovay\index{Solovay, Robert} teilte ein neues Ergebnis von A.~R.~D.~Mathias über das System von Bourbaki mit, und der Text wurde entsprechend aktualisiert (S.~\pageref{bourbaki}).

Bob wies auch auf eine Klärung der Literatur bezüglich der Kategorientheorie und unzugängliche Kardinalzahlen\index{Kategorientheorie}\index{Kardinalzahl, unzugänglich} (S.~\pageref{categoryth}) hin, und eine missverständliche Aussage wurde aus dem Text entfernt.  Genauer gesagt ist es im Gegensatz zu einer Aussage in früheren Ausgaben möglich, "`Es gibt eine eigene Klasse von unzugänglichen Kardinalen"' in der Sprache von ZFC auszudrücken.  Dies lässt sich wie folgt bewerkstelligen:  "`Für jede Menge $x$ gibt es eine unzugängliche Kardinalzahl $\kappa$, so dass $\kappa$ nicht in $x$ liegt"'. Bob schreibt:\footnote{Private Korrespondenz, 30. November 2002.}
\begin{quotation}
  
  Dieses Axiom ist die Art und Weise, wie Grothendieck die Kategorientheorie darstellt.  Jedem unzugänglichen Kardinal $\kappa$ ordnet man ein Grothendieck-Universum \index{Grothendieck, Alexander} $U(\kappa)$ zu. $U(\kappa)$ besteht aus denjenigen Mengen, die in einer transitiven Menge der Kardinalität kleiner als $\kappa$ liegen.  Anstelle der "`Kategorie aller Gruppen"' arbeitet man relativ zu einem Universum [unter Berücksichtigung der Kategorie der Gruppen mit Kardinalität kleiner als $\kappa$].  Nun ist die Kategorie, deren Objekte alle Kategorien "`relativ zum Universum $U(\kappa)$"' sind, eine Kategorie nicht relativ zu diesem Universum, sondern zum nächsten Universum.
  
  All die Dinge, die Kategorientheoretiker gerne tun, können in diesem Rahmen getan werden.  Der einzige strittige Punkt ist, ob das Grothen-Dieck-Axiom für die Bedürfnisse der Kategorientheoretiker zu stark ist.  Mac Lane \index{Mac Lane, Saunders} argumentiert, dass "`ein Universum ausreicht"' und Feferman\index{Feferman, Solomon} hat argumentiert, dass man mit der gewöhnlichen ZFC auskommen kann.  Ich finde die Argumente von Feferman nicht überzeugend.  Mac Lane mag recht haben, aber wenn ich über Kategorientheorie nachdenke, tue ich das \`{a} la Grothendieck.
  
  Übrigens fügt Mizar\index{Mizar} das Axiom "`Es gibt eine eigene Klasse von Unzugänglichkeiten"' hinzu, genau um Kategorientheorie zu betreiben.

\end{quotation}

Die aktuellsten Informationen über das Metamath-Programm und die Datenbasen sind immer unter \url{http://metamath.org} zu finden.


\subsubsection{Notiz hinzugefügt am 24. Juni 2006}\label{note2006}

Die Metamath-Spezifikation wurde leicht eingeschränkt, um das Schreiben von Parsern zu erleichtern.  Siehe die Fußnote auf S.~\pageref{namespace}.


%\subsubsection{Note Added July 24, 2006}\label{note2006b}
\subsubsection{Notiz hinzugefügt 10 März 2007}\label{note2006b}

Ich bin Anthony Williams\index{Williams, Anthony} dankbar für das Schreiben des \LaTeX-Pakets namens {\tt realref.sty} und dafür, dass er es der Public Domain zur Verfügung gestellt hat.  Mit diesem Paket können die internen Hyperlinks in einer {\sc pdf}-Datei auf bestimmte Seitenzahlen verankert werden, anstatt nur auf Abschnittsüberschriften, was die Navigation in der {\sc pdf}-Datei für dieses Buch viel angenehmer und "`logischer"' macht.

Ein von Martin Kiselkov gefundener Druckfehler wurde korrigiert.
Eine verwirrende Bemerkung über die Vereinheitlichung wurde auf Anregung von Mel O'Cat entfernt.


\subsubsection{Notiz hinzugefügt am 27. Mai 2009}\label{note2009}

Mehrere von Kim Sparre gefundene Tippfehler wurden korrigiert.  Es wurde ein Hinweis hinzugefügt, dass die Poincar'{e}-Vermutung bewiesen wurde (S.~\pageref{poincare}).


\subsubsection{Notiz hinzugefügt am 17 Nov. 2014}\label{note2014}

Die Aussage des Schröder-Bernstein-Theorems in Abschnitt~\ref{trust} wurde korrigiert.  Dank an Bob Solovay für den Hinweis auf den Fehler.


\subsubsection{Notiz hinzugefügt am 25. Mai 2016}\label{note2016}

Dank an Jerry James für die Korrektur von 16 Tippfehlern.


\subsubsection{Notiz hinzugefügt Februar 25, 2019}\label{note201902}

David A. Wheeler\index{Wheeler, David A.} hat, in Zusammenarbeit mit mir, eine große Anzahl von Verbesserungen und Aktualisierungen vorgenommen.
Die Axiome der Prädikatenlogik wurden neu nummeriert, und der Text macht nun deutlich, dass sie auf Tarskis System S2 beruhen; die einzige geringfügige Abweichung im Axiom ax-6 wird erklärt und begründet.
Die Axiome für reelle und komplexe Zahlen wurden geändert, damit sie mit \texttt{set.mm}\index{Mengenlehre-Datenbasis (\texttt{set.mm})}%
\index{Metamath Proof Explorer} konsistent sind.
Lang erwartete Änderungen der Spezifikation "`1--8"' wurden vorgenommen, was zur Klärung von zuvor mehrdeutigen Punkten führte.
Einige Fehler im Text, welche die \texttt{\$f}- und \texttt{\$d}-Anweisungen betreffen, wurden korrigiert (die Spezifikation war korrekt, aber die Erklärungen im Buch widersprachen versehentlich der Spezifikation).
Wir haben jetzt ein System zur automatischen Erzeugung von schmalen PDFs, damit alle, die ein Smartphone besitzen, einen einfachen Zugang zur aktuellen Version des Dokuments haben.
Ein neuer Abschnitt über Deduktion wurde hinzugefügt; er behandelt das Standard-Deduktionstheorem, das Theorem der schwachen Deduktion, den Deduktionsstil und die natürliche Deduktion.
Viele kleinere Korrekturen (zu zahlreich, um sie hier aufzulisten) wurden ebenfalls vorgenommen.


\subsubsection{Notiz hinzugefügt am 7. März 2019}\label{note201903}

Eine Beschreibung der Metamath-Syntax in erweiterter Backus--Naur-Form (EBNF)\index{erweiterte Backus--Naur-Form}\index{EBNF} wurde im Anhang \ref{BNF} ergänzt, eine kurze Erklärung über Typcodes hinzugefügt, weitere Beispiele im Abschnitt über Deduktion eingefügt, und eine Vielzahl kleinerer Verbesserungen durchgeführt.


\subsubsection{Notiz hinzugefügt am 7. April 2019}\label{note201904}

In dieser Version des Buches wird die Notation für die "`echte Substitution"' geklärt, die
Erläuterungen zu dem Theorem der schwachen Deduktion und zu der natürlichen Deduktion verbessert,
der Befehl \texttt{undo} dokumentiert, die Informationen zu \texttt{write source} aktualisiert,  der Typcode von \texttt{set} in \texttt{setvar} geändert, um mit der aktuellen Version von \texttt{set.mm} übereinzustimmen, weitere Erläuterungen über Kommentarauszeichnungen hinzugefügt (z.B. wurde dokumentiert, wie man Überschriften erstellt), und die Unterschiede zwischen den verschiedenen Behauptungsformen (insbesondere der Ableitungsform) klargestellt.


\subsubsection{Notiz hinzugefügt am 2. Juni 2019}\label{note201906}

Diese Version behebt eine große Anzahl kleinerer Probleme, die von Beno\^{i}t Jubin\index{Jubin, Beno\^{i}t} gemeldet wurden, wie z.B. redaktionelle Probleme und die Notwendigkeit, \texttt{verify markup} zu dokumentieren (Vielen Dank!).
Außerdem enthält diese Version nun konkrete Beispiele für die Formen von Theoremen (Deduktionsform, Inferenzform und geschlossene Form).
Wir nennen diese Version die "`zweite Auflage"'; die vorherige Ausgabe, die 2007 offiziell veröffentlicht wurde, hatte einen etwas anderen Titel (\textit{Metamath: Eine Computersprache für die reine Mathematik}).


\chapter{Einleitung}
\pagenumbering{arabic}

\begin{quotation}

  {\em {\em I.M.:}  Nein, nein.  Da ist nichts Subjektives dran!  Jeder weiß, was ein Beweis ist.  Lesen Sie einfach ein paar Bücher, besuchen Sie Kurse bei einem kompetenten Mathematiker, und Sie werden es verstehen.     
  
  {\em Schüler:}  Sind Sie sicher?
  
  {\em I.M.:}  Nun - es ist möglich, dass Sie es nicht verstehen, wenn Sie keine Begabung dafür haben.  Das kann auch passieren.
  
  {\em Schüler:}  Dann entscheiden Sie, was ein Beweis ist, und wenn ich nicht lerne auf dieselbe Art und Weise zu entscheiden, dann bestimmen Sie, dass ich keine Begabung habe.
  
  {\em I.M.:}  Wenn nicht ich, wer dann?}       
  
      \flushright\sc "`Der ideale Mathematiker"'
      \index{Davis, Phillip J.}
      \footnote{Frei übersetzt nach \cite{Davis}, \sc  "`The Ideal Mathematician"' S.~40.}\\
\end{quotation}

Brillante Mathematiker haben nahezu unvorstellbar tiefgreifende Ergebnisse erzielt, die zu den krönenden intellektuellen Errungenschaften der Menschheit zählen.  Allerdings hinkt die moderne abstrakte Mathematik in gewisser Weise der Zeit hinterher, und ist in einer Ära vor der Existenz von Computern stecken geblieben.
Zwar bestreitet niemand die bemerkenswerten Ergebnisse, die erzielt wurden.  Jedoch ist es praktisch unmöglich, diese Ergebnisse einem Uneingeweihten in präziser Weise zu vermitteln.  Um diese Ergebnisse zu beschreiben, wird eine knappe, informelle Sprache verwendet, die trotz ihrer Eleganz sehr schwer zu erlernen ist.  Diese informelle Sprache ist nicht unpräzise, ganz im Gegenteil, jedoch werden Details häufig einfach ausgelassen, und es werden Symbole mit verborgenem Kontext verwendet, die implizit von einem Experten verstanden werden, aber nur von wenigen anderen.
Äußerst komplexe technische Bedeutungen werden verbunden mit harmlos klingenden Wörtern wie "`kompakt"' und "`messbar"', die kaum einen Hinweis darauf geben, was eigentlich damit ausgesagt wird.  Wer sich die genaue technische Bedeutung nicht ständig vor Augen hält scheitert, und die Fähigkeit dazu kann nur durch viel Übung und harte Arbeit erworben werden.  Nur die wenigen, welche diese notwendige, schmerzhafte Lernerfahrung machen, können der kleinen, geschlossenen Gruppe der reinen Mathematiker beitreten.  Die informelle Sprache schottet die wahre Natur ihres Wissens von den meisten anderen ab.

Metamath\index{Metamath} macht abstrakte Mathematik konkreter.  Es ermöglicht einem Computer, die mit jedem Wort oder Symbol verbundene Komplexität mit absoluter Strenge zu erfassen.  Man kann diese Komplexität in aller Ruhe erforschen, und zwar bis zum gewünschten Detaillierungsgrad.  Ob Sie nun glauben, dass Konzepte wie Unendlichkeit tatsächlich außerhalb des Verstandes "`existieren"' oder nicht, mit Metamath können Sie das ergründen, was damit tatsächlich ausgesagt werden soll.

Metamath ermöglicht auch eine völlig rigorose und gründliche Überprüfung von Beweisen.
Seine Sprache ist so einfach, dass man sich nicht auf die Autorität von Experten verlassen muss, sondern die Ergebnisse Schritt für Schritt selbst überprüfen kann.  Wenn Sie versuchen wollen, Ihre eigenen Ergebnisse abzuleiten, lässt Metamath Sie keinen Denkfehler machen.
Auch professionelle Mathematiker machen Fehler; Metamath dagegen ermöglicht es, die Korrektheit von Beweisen gründlich zu überprüfen.

Metamath\index{Metamath} ist eine Computersprache und ein zugehöriges Computerprogramm für das Archivieren, Überprüfen und Studieren mathematischer Beweise auf einer sehr detaillierten Ebene.
Mit der Metamath-Sprache können formale mathematische Systeme\index{formales System} beschrieben und Beweise für Theoreme in diesen Systemen formuliert werden.  Eine solche Sprache wird von Mathematikern als Metasprache bezeichnet. Das Metamath-Programm ist ein Computerprogramm zur Überprüfung von Beweisen, die in der Metamath-Sprache geschrieben sind.  Das Metamath-Programm verfügt nicht über die eingebaute Fähigkeit, logische Schlüsse zu ziehen; es führt lediglich eine Reihe von Ersetzungen von Symbolen gemäß den Anweisungen durch, die ihm in einem Beweis gegeben werden, und prüft, ob das Ergebnis mit dem erwarteten Theorem übereinstimmt.  Es macht logische  Schlussfolgerungen nur auf der Grundlage von Regeln der Logik, welche in einer Menge von Axiomen\index{Axiom} oder ersten Prinzipien enthalten sind, die ihm als Ausgangspunkt für Beweise vorgegeben werden.
	
Die vollständige Spezifikation der Metamath-Sprache ist nur vier Seiten lang (Abschnitt~\ref{spec}, S.~\pageref{spec}).  Ihre Einfachheit mag Sie zunächst fragen lassen, was man damit überhaupt erreichen kann.  Aber in der Tat sind die verwendeten Symbolmanipulationen diejenigen, die in allen mathematischen Systemen auf der untersten Ebene implizit durchgeführt werden.  Man kann sie relativ schnell lernen und volles Vertrauen in jeden mathematischen Beweis haben, den Metamath verifiziert.  Andererseits ist die Metamath-Sprache leistungsfähig und allgemein genug, um mit ihr  praktisch jede mathematische Theorie, von der einfachsten bis zur abstraktesten, zu beschreiben.

Obwohl Metamath im Prinzip für jede Art von Mathematik verwendet werden kann, ist es am besten für abstrakte oder "`reine"' Mathematik geeignet, die sich hauptsächlich mit Theoremen und deren Beweisen befasst - im Gegensatz zu der Art von Mathematik, die sich mit der praktischen Handhabung von Zahlen beschäftigt.
Beispiele für Teilgebiete der reinen Mathematik sind die Logik\footnote{Logik ist die Lehre von den Aussagen, die unabhängig von den Objekten, die sie beschreiben, universell wahr sind. Ein Beispiel ist die Aussage: "`Wenn $P$	$Q$ impliziert, dann ist entweder $P$ falsch oder $Q$ wahr."'}, Mengenlehre\index{Mengenlehre}\footnote{Die Mengenlehre ist die Lehre von allgemeinen mathematischen Objekten, den so genannten "`Mengen"'. Von ihr lässt sich im Wesentlichen die gesamte Mathematik ableiten. Zahlen können beispielsweise als spezifische Mengen definiert werden, und ihre Eigenschaften können mit den Werkzeugen der Mengenlehre erforscht werden.}, Zahlentheorie\index{Zahlentheorie}\footnote{Die Zahlentheorie befasst sich mit den Eigenschaften von positiven und negativen ganzen Zahlen.}, Gruppentheorie\index{Gruppentheorie}\footnote{Die Gruppentheorie untersucht die Eigenschaften von mathematischen Objekten, die als "`Gruppen"' bezeichnet werden: Sie gehorchen einem einfachen Satz von Axiomen und haben Symmetrieeigenschaften, die sie für viele andere Bereiche nützlich machen.}, abstrakte Algebra\index{abstrakte Algebra}\footnote{Abstrakte Algebra umfasst die Gruppentheorie und untersucht auch Gruppen mit zusätzlichen Eigenschaften, die sie als "`Ringe"' und "`Körper"' qualifizieren.  Die Menge der reellen Zahlen ist ein bekanntes Beispiel für einen Körper.}, Analysis\index{Analysis}\index{reelle Zahl}\index{komplexe Zahl}\footnote{Analysis ist die Lehre von den reellen und komplexen Zahlen.} und Topologie\index{Topologie}\footnote{Ein Bereich, der von der Topologie untersucht wird, sind Eigenschaften geometrischer Objekte, die unverändert bleiben, wenn sie Verformungen unterzogen werden.  Zum Beispiel haben ein Donut und eine Kaffeetasse jeweils ein Loch (die Tasse hat ein Loch im Henkel) und werden daher als topologisch äquivalent betrachtet.  Im Allgemeinen ist die Topologie jedoch die Lehre von abstrakten mathematischen Objekten, die einer bestimmten (erstaunlich einfachen) Reihe von Axiomen gehorchen. Siehe z. B. Munkres \cite{Munkres}\index{Munkres, James R.}.}. Auch in der Physik könnte Metamath auf bestimmte Zweige angewandt werden, die sich der abstrakten Mathematik, wie z.B. der Quantenlogik (die zur Untersuchung von Aspekten der Quantenmechanik genutzt wird), bedienen.

Andererseits ist Metamath\index{Metamath} weniger geeignet für Anwendungen, die sich hauptsächlich mit intensiven numerischen Berechnungen befassen.  Metamath hat keine eingebaute Darstellung von Zahlen\index{Metamath!Zahlendarstellung}; stattdessen muss eine bestimmte Folge von Symbolen (Ziffern) syntaktisch als Teil eines Beweises konstruiert werden, in dem eine gewöhnliche Zahl verwendet wird.  Aus diesem Grund sind Zahlen in Metamath am besten auf spezifische Konstanten beschränkt, die im Verlauf eines Theorems oder seines Beweises auftauchen.  Zahlen sind nur ein winziger Teil der Welt der abstrakten Mathematik.  Der Ausschluss von eingebauten Zahlen war eine bewusste Entscheidung, um die Einfachheit von Metamath sicherzustellen. Es gibt andere Software-Tools für andere mathematische Fragestellungen. Wenn Sie schnell algebraische Probleme lösen möchten, sind die Computeralgebrasysteme\index{Computeralgebrasystem} Macsyma\index{Macsyma}, Mathematica\index{Mathematica} und Maple\index{Maple} besonders gut geeignet, um mit Zahlen und Algebra effizient umzugehen.  Wenn Sie einfach nur numerische Ausdrücke oder Matrixausdrücke bequem berechnen möchten, dürften Tools wie Octave\index{Octave} eine bessere Wahl sein.

Nach dem Erlernen der grundlegenden Anweisungstypen von Metamath sollte jeder technisch versierte Mensch, ob Mathematiker oder nicht, sofort in der Lage sein, jedes in der Metamath-Sprache bewiesene Theorem so weit wie gewünscht zurückzuverfolgen - bis hin zu den Axiomen, auf denen das Theorem beruht.  Diese Möglichkeit erlaubt eine nicht-traditionelle Art und Weise des Lernens der reine Mathematik.  In Verbindung mit traditionellen Methoden könnte Metamath die reine Mathematik für Menschen zugänglich machen, die nicht ausreichend qualifiziert sind, um die impliziten Details in gewöhnlichen Lehrbuchbeweisen zu verstehen.  Sobald man die Axiome einer Theorie kennt, kann man sich darauf verlassen, dass alles vorhanden ist, was man zum Verstehen eines vorliegenden Beweises braucht.  Somit kann man sich auf jeden Beweisschritt konzentrieren, den man nicht versteht - und zwar so tief wie nötig, ohne sich Sorgen machen zu müssen, dass man bei einem Schritt, den man sich nicht erklären kann, nicht mehr weiter kommt.\footnote{Andererseits ist das Schreiben von Beweisen in der Metamath-Sprache anspruchsvoll und erfordert einen Grad an Strenge, der weit über das hinausgeht, was Schülern oder Studenten normalerweise beigebracht wird.  Ich bezweifle, dass im Mathematikunterricht das Schreiben von Metamath-Beweisen jemals die traditionellen Hausaufgaben mit informellen Beweisen ersetzen wird, denn aufgrund der Zeit, die für die Ausarbeitung der Details benötigt wird, könnten im Unterricht nur wenige Themen behandelt werden.  Schülern mit Schwierigkeiten, die implizite Strenge im vorliegenden, traditionell verfassten Material zu verstehen, kann das Schreiben einiger einfacher Beweise in der Metamath-Sprache jedoch helfen, ungenaue Gedankengänge zu klären.  Obwohl es anfangs etwas schwierig ist, macht dies aufgrund der sofortigen Rückmeldung durch den Computer sogar Spaß, wie das Lösen eines Rätsels.}

Metamath ist wahrscheinlich anders als alles, was Ihnen bisher begegnet ist.  In diesem ersten Kapitel werden wir uns mit der Philosophie und dem Einsatz von Computern in der Mathematik beschäftigen, um die Motivation hinter Metamath besser zu verstehen. Das Material in diesem Kapitel ist nicht erforderlich, um Metamath zu benutzen.  Sie können es überspringen, wenn Sie ungeduldig sind, aber ich hoffe, Sie werden es lehrreich und unterhaltsam finden.  Wenn Sie gleich mit dem Experimentieren mit dem Metamath-Programm beginnen wollen, gehen Sie direkt zu Kapitel~\ref{using}
(S.~\pageref{using}).  Um die Metamath-Sprache zu lernen, überfliegen Sie Kapitel~\ref{using} und fahren direkt fort mit Kapitel~\ref{languagespec} (S.~\pageref{languagespec}).


\section{Mathematik als eine Computersprache}

\begin{quote}
  
  {\em Das Studium der Mathematik beginnt oft mit einer Enttäuschung. \ldots \\
  	Uns wird gesagt, dass mit ihrer Hilfe die Sterne gewogen und die Milliarden von Molekülen in einem Wassertropfen gezählt werden.  Doch wie der Geist von Hamlets Vater entzieht sich diese große Wissenschaft den Bemühungen unserer geistigen Waffen, sie zu begreifen.}
  \flushright\sc  Alfred North Whitehead\footnote{Frei übersetzt nach \cite{Whitehead}, Kap.\ 1.}\\
\end{quote}\index{Whitehead, Alfred North}

\subsection{Ist die Mathematik "`benutzerfreundlich"'?}

Angenommen, Sie haben keine formale Ausbildung in abstrakter Mathematik erhalten.  Aber populäre Bücher, die Sie gelesen haben, bieten verlockende Einblicke in diese Welt voller tiefgründiger Ideen, die den menschlichen Geist aufgewühlt haben.  Sie sind nicht zufrieden mit den informellen, verwässerten Beschreibungen, die Sie gelesen haben, sondern halten es für wichtig, die zugrundeliegende Mathematik selbst zu begreifen, um ihre wahre Bedeutung zu verstehen.  Es ist aber nicht sinnvoll, wieder zur Schule zu gehen, um sie zu lernen; Sie wollen nicht Jahre Ihres Lebens damit verbringen.  Es gibt viele wichtige Dinge im Leben, und man muss Prioritäten für das setzen, was einem wichtig ist.  Was würde passieren, wenn Sie versuchen würden, dieses Vorhaben allein in Ihrer Freizeit zu verfolgen?

Immerhin waren Sie in der Lage, eine Computer-Programmiersprache wie Pascal ohne große Schwierigkeiten selbst zu erlernen, obwohl Sie keine formale Ausbildung für Computer hatten.  Sie behaupten nicht, ein Experte für Software-Design zu sein, aber Sie können ein passables Programm schreiben, das Ihren Bedürfnissen entspricht.  Wichtiger ist sogar noch, dass Sie wissen, dass Sie sich das Pascal-Programm eines anderen ansehen können, egal wie komplex es ist, und mit genügend Geduld herausfinden können, wie es genau funktioniert, auch wenn Sie kein Spezialist sind.  Mit Pascal können Sie alles tun, was ein Computer tun kann, zumindest im Prinzip.  Sie wissen also, dass Sie die Fähigkeit haben, im Prinzip alles zu tun, was ein Computerprogramm tun kann: Sie müssen es nur in genügend kleine Stücke zerlegen.

Das folgende imaginäre Szenario könnte eintreten, wenn Sie sich unbedarft der gleichen Sichtweise der abstrakten Mathematik annehmen und versuchen würden, sie selbst zu lernen.  Und zwar in einem Zeitraum, der vergleichbar ist mit dem Erlernen einer Programmiersprache für Computer.


\subsubsection{Die Suche eines Nicht-Mathematikers nach der Wahrheit}

\begin{quote}
  {\em \ldots meine Töchter studieren schon seit mehreren Semestern (Chemie) und meinen, in der Schule Differential- und Integralrechnung gelernt zu haben, aber wissen doch bis heute nicht, warum $x\cdot y=y\cdot x$ gilt.}
    \flushright\sc Edmund Landau\footnote{Frei übersetzt nach \cite{Landau}, S.~vi.}\
\end{quote}\index{Landau, Edmund}

\begin{quote}
  {\em Minus mal minus ergibt plus,\\
   warum das so ist müssen wir nicht diskutieren.}
    \flushright\sc W.\ H.\ Auden\footnote{Frei übersetzt nach dem Zitat in \cite{Guillen}, S.~64.}\\
\end{quote}\index{Auden, W.\ H.}\index{Guillen, Michael}

Nehmen wir an, Sie sind ein technisch orientierter Fachmann, vielleicht ein Ingenieur, ein Computerprogrammierer oder ein Physiker, aber eben kein Mathematiker.  Sie halten sich für einigermaßen intelligent.  Sie waren gut in der Schule und lernten eine Vielzahl von Methoden und Techniken der praktischen Mathematik, wie z.~B. der Infinitesimalrechnung und das Lösen von Differentialgleichungen.  Aber in Ihrem Unterricht ging es selten um moderne abstrakte Mathematik, und Beweise tauchten nur gelegentlich in Ihren Lehrbüchern auf - eine Art notwendiges Übel, das Sie von einem bestimmten Schlüsselergebnis überzeugen sollte.  Die meisten Ihrer Hausaufgaben bestanden aus Übungen, in denen die Techniken geübt wurden, und Sie wurden kaum jemals aufgefordert, einen eigenen Beweis zu erstellen.

Sie sind neugierig auf fortgeschrittene, abstrakte Mathematik.  Sie sind von der inneren Überzeugung getrieben, dass es wichtig ist, einige der tiefgreifendsten Erkenntnisse der Menschheit zu verstehen und zu schätzen.  Aber es scheint sehr schwer erlernbar zu sein, etwas, das nur bestimmte begabte Fachidioten begreifen und verstehen können.  Sie sind frustriert, dass Ihnen solche Erkenntnisse scheinbar für immer verschlossen bleiben.

Schließlich treibt Sie Ihre Neugierde dazu, etwas dagegen zu tun.
Sie setzen sich das Ziel, Mathematik "`wirklich"' zu verstehen: nicht nur wie man Gleichungen in Algebra oder Infinitesimalrechnung nach Kochbuchregeln manipuliert, sondern vielmehr ein tiefes Verständnis dafür zu erlangen, woher diese Regeln kommen.
In Wirklichkeit geht es Ihnen gar nicht um diese Art von gewöhnlicher Mathematik,
sondern über einen viel abstrakteren, feinstofflichen Bereich der reinen Mathematik, zu denen berühmte Ergebnisse wie der Gödelsche Unvollständigkeitssatz\index{Gödelscher Unvollständigkeitssatz} und Cantors verschiedene Arten von Unendlichkeiten gehören.

Sie haben wahrscheinlich eine Reihe populärer Bücher mit Titeln wie {\em Infinity and the Mind} \cite{Rucker}\index{Rucker, Rudy} zu Themen wie diesen gelesen.  Sie fanden sie inspirierend, aber gleichzeitig auch etwas unbefriedigend.  Sie gaben Ihnen eine allgemeine Vorstellung davon, worum es bei diesen Ergebnissen geht, aber wenn jemand Sie bitten würde, sie zu beweisen, wüssten Sie nicht, wo man anfangen soll.  Sicher, Sie könnten denselben allgemeinen Überblick geben, den Sie aus den populären Büchern gelernt haben, und in gewisser Weise haben Sie auch ein Verständnis.  Aber tief in Ihrem Inneren wissen Sie, dass eine gewisse Strenge fehlt, dass es auf dem Weg dorthin wahrscheinlich viele subtile Schritte und Fallstricke gibt, und dass man sich letztlich auf die Experten auf diesem Fachgebiet verlassen muss.  Das gefällt Ihnen nicht; Sie möchten diese Ergebnisse selbst überprüfen können.

Was tun Sie also als Nächstes?  Als ersten Schritt beschließen Sie, in einigen der Originalarbeiten zu den Sie interessierenden Theoremen nachzuschlagen, oder besser noch, sich einige Standardlehrbücher auf diesem Gebiet zu beschaffen.  Sie schlagen ein Theorem nach, das Sie verstehen wollen.  Natürlich steht es da, aber es wird mit seltsamen Begriffen und merkwürdigen Symbolen ausgedrückt, die für Sie absolut nichts bedeuten.  Es könnte genauso gut in einer Fremdsprache geschrieben sein, die Sie noch nie gesehen haben und deren Symbole Ihnen völlig fremd sind.
Sie sehen sich den Beweis an und haben nicht die leiseste Ahnung, was die einzelnen Schritte bedeuten, geschweige denn, wie ein Schritt auf den anderen folgt.  Nun, offensichtlich müssen Sie eine Menge lernen, wenn Sie diese Dinge verstehen wollen.

Sie denken, dass Sie es wahrscheinlich verstehen könnten, wenn Sie noch einmal drei bis sechs Jahre zur Uni gehen und ein Mathematikstudium absolvieren.  Aber das passt nicht zu Ihrer Karriere und den anderen Dingen in Ihrem Leben und würde keinen praktischen Nutzen bringen.  Sie beschließen, einen schnelleren Weg zu suchen.  Sie denken sich, Sie gehen einfach zurück zum Anfang, Schritt für Schritt, wie bei einem Computerprogramm, bis Sie es verstehen.  Aber Sie stellen schnell fest, dass dies nicht möglich ist, da Sie nicht einmal genug verstehen, um zu wissen, wohin man zurückgehen muss.

Vielleicht ist ein anderer Ansatz angebracht - vielleicht sollte man am Anfang beginnen und sich hocharbeiten.  Zuerst lesen Sie die Einleitung des Buches um herauszufinden, was die Voraussetzungen dafür sind.  Auf ähnliche Art und Weise verfolgen Sie Ihren Weg durch zwei oder drei weitere Bücher zurück und stoßen schließlich auf eines, das am Anfang zu stehen scheint: Es listet die Axiome der Arithmetik auf.  "`Aha!"', denken Sie naiv, "`Das muss der Ausgangspunkt sein, die Quelle allen mathematischen Wissens"'. Oder zumindest der Ausgangspunkt für die Mathematik, die sich mit Zahlen befasst; irgendwo muss man ja anfangen, und man hat keine Ahnung, wo der Ausgangspunkt für eine andere Mathematik sein könnte.  Aber das Wort "`Axiome"' sieht vielversprechend aus.  Also liest man eifrig weiter und arbeitet sich durch einige elementare Übungen am Anfang des Buches durch.  Sie fühlen sich vage beunruhigt: Diese scheinen überhaupt nicht wie Axiome zu sein, zumindest nicht in dem Sinne, den Sie sich vorstellen, wenn Sie an Axiome denken.  Axiome implizieren einen Ausgangspunkt, von dem aus alles andere nach genauen, im Axiomensystem festgelegten Regeln aufgebaut werden kann.  Auch wenn Sie die ersten Beweise auf informelle Weise verstehen können, und in der Lage sind, einige der Übungen zu erledigen, ist es schwer, die Regeln genau auszumachen.   Sicher, jeder Schritt scheint logisch aus den anderen zu folgen, aber was bedeutet das genau?  Ist die "`Logik"' nur eine Frage des gesunden Menschenverstands, etwas Unbestimmtes, das wir alle verstehen, aber nie ganz genau benennen können?

Sie haben einige Jahre - mit Unterbrechungen - damit verbracht Computer zu programmieren, und Sie wissen, dass es bei Computersprachen keine Frage nach den Regeln gibt - sie sind präzise und kristallklar.  Wenn Sie sie befolgen, wird Ihr Programm funktionieren, und wenn Sie es nicht tun, wird es nicht funktionieren.  Egal wie komplex ein Programm ist, es kann jederzeit in immer einfachere Teile zerlegt werden, bis man schließlich die Bits identifizieren kann, die herumgeschoben werden, um eine bestimmte Funktion auszuführen.  Einige Programme erfordern vielleicht viel Ausdauer, um sie zu schreiben, aber wenn Sie sich auf einen bestimmten Teil konzentrieren, müssen Sie nicht einmal unbedingt wissen, wie der Rest des Programms funktioniert. Sollte es nicht eine Analogie in der Mathematik geben?

Sie beschließen, den ultimativen Test durchzuführen: Sie fragen sich, wie ein Computer überprüfen oder sicherstellen könnte, dass die Schritte in diesen Beweisen aufeinander aufbauen.
Sicherlich muss die Mathematik mindestens genauso genau definiert sein wie eine Computersprache, wenn nicht sogar noch präziser; schließlich basiert die Informatik selbst auf ihr.
Wenn man einen Computer dazu bringen kann, diese Beweise zu überprüfen, dann sollte man im Grunde ebenfalls in der Lage sein, sie im Prinzip auch selbst kristallklar, in einer präzisen Weise zu verstehen.

Sie werden überrascht sein: Sie können sich keine Vorgehensweise vorstellen, wie Sie die Beweise, die in deutscher oder englischer Sprache abgefasst sind, in eine Form bringen können, die der Computer versteht.
Die Beweise sind voll von Sätzen wie "`Angenommen, es gibt ein eindeutiges $x$\ldots"' und "`Bei einem beliebigen $y$ sei $z$ eine solche Zahl, dass\ldots"' Das ist nicht die Art von Logik, die man aus der Computerprogrammierung kennt, wo sich alles, sogar die Arithmetik, auf boolesche Einsen und Nullen reduziert, wenn man sie nur ausreichend aufschlüsselt.  Auch wenn Sie glauben, dass Sie die Beweise verstehen, scheint eine Art höheres Denken notwendig zu sein und nicht präzise Regeln, die festlegen, wie man die Symbole in den Axiomen manipuliert.  Was auch immer es ist, es ist einfach nicht offensichtlich, wie man es einem Computer gegenüber ausdrücken würde, und je mehr man darüber nachdenkt, desto verwirrter wird man. Schließlich gelangt man zu einem Punkt, an dem man sich sogar fragt, ob man es wirklich versteht.  Es steckt viel mehr hinter diesen Axiomen der Arithmetik, als man auf den ersten Blick sieht.

Darüber hat in der Schule in den naturwissenschaftlichen Fächern nie jemand gesprochen.  Sie haben nur die Regeln gelernt, die man Ihnen vorgab, ohne ganz zu verstehen, wie oder warum sie funktionierten.  Manchmal waren Sie vage misstrauisch oder unsicher, und haben durch Hausaufgaben und Übertragung von dem Präsentierten gelernt, wie man Lösungen präsentiert, die den Dozenten zufriedenstellten und Ihnen eine "`1"' einbrachten.  Selten hat man tatsächlich etwas auf rigorose Weise "`bewiesen"', und die Mathe-Studenten, die so etwas taten, schienen in einer anderen Welt zu leben.

Natürlich gibt es Computeralgebra-Programme, die Mathematik betreiben können, und zwar ziemlich beeindruckend.  Sie können im Handumdrehen die Integrale lösen, mit denen man in dem Fach Infinitesimalrechnung zu kämpfen hatte, und können noch viel, viel mehr.  Aber wenn man diese Programme anschaut, sieht man eine große Sammlung von Algorithmen und Techniken, die im Laufe der Zeit weiterentwickelt und ergänzt wurden, zusammen mit grundlegender Software, die Symbole manipuliert.  Jeder eingebaute Algorithmus ist das Ergebnis eines Theorems, dessen Beweis weggelassen wurde; man muss nur derjenigen Person vertrauen, die ihn bewiesen hat, und der Person, die ihn einprogrammiert hat, und hoffen, dass es keine Bugs gibt.\index{Programmfehler}\index{Bug} Irgendwie scheint dies nicht die Essenz der Mathematik zu sein.  Obwohl Computeralgebrasysteme Theoreme mit erstaunlicher Geschwindigkeit generieren können, können sie nicht ein einziges davon wirklich beweisen.

Nach einigem Grübeln schauen Sie sich einige populäre Bücher darüber an, worum es in der Mathematik geht.  Irgendwo liest man, dass die gesamte Mathematik eigentlich von etwas abgeleitet ist, das sich "`Mengenlehre"' nennt.  Das ist ein wenig verwirrend, denn in dem Buch, in dem die Axiome der Arithmetik vorgestellt werden, wird nirgends die Mengenlehre erwähnt, oder wenn, dann nur als ein Werkzeug, das hilft, Dinge besser zu beschreiben - die Menge der geraden Zahlen und so weiter.  Wenn Mengenlehre die Grundlage für die gesamte Mathematik ist, warum werden dann zusätzliche Axiome für die Arithmetik benötigt?

Irgendetwas stimmt nicht, aber Sie sind sich nicht sicher, was.  Einer Ihrer Freunde ist ein reiner Mathematiker.  Er weiß, dass er nicht in der Lage ist, Ihnen mitzuteilen, womit er sein Geld verdient und er scheint wenig Interesse daran zu haben, es zu versuchen.  Sie wissen aber, dass für ihn Beweise das sind, worum es in der Mathematik geht. Sie fragen ihn, was ein Beweis ist, und er sagt Ihnen, dass er zwar auf Logik basiert, aber dass das Beweisen eigentlich etwas ist, das man lernt, indem man es immer und immer wieder macht, bis man es kapiert hat.  Er verweist auf ein Buch, {\em How to Read and Do Proofs} \cite{Solow}.\index{Solow, Daniel}  Obwohl dieses Buch Ihnen hilft, traditionelle informelle Beweise zu verstehen, gibt es immer noch etwas, das Sie noch nicht ganz begreifen können.

Sie fragen Ihren Freund, wie Sie einen Beweis von einem Computer überprüfen lassen würden.
Zuerst scheint er über die Frage verwirrt zu sein; warum sollte man das tun?
Dann sagt er, dass es keinen Sinn ergeben würde, aber er hat gehört, dass man den Beweis in Tausende oder sogar Millionen von Einzelschritten zerlegen müsste, um so etwas zu tun, weil die damit verbundenen Überlegungen auf einer so hohen Abstraktionsebene liege.  Er sagt, dass man so etwas vielleicht bis zu einem gewissen Punkt machen könne, aber dass der Computer völlig unpraktisch wäre, sobald man in eine sinnvolle Mathematik einsteigt.  Dort kann man einen Beweis nur noch von Hand verifizieren, und die Fähigkeit dazu kann man nur erwerben, wenn man sich ein paar Jahre lang in der Universität auf das Gebiet spezialisiert.  Wie auch immer, er glaubt, dass das alles mit der Mengenlehre zu tun hat, obwohl er nie einen formalen Kurs im Fach Mengentheorie belegt hat, sondern sich das, was er brauchte, einfach nach und nach angeeignet hat.

Sie sind fasziniert und erstaunt.  Offenbar kann ein Mathematiker mit einem einzigen Konzept etwas erfassen, für das ein Computer tausend oder eine Million Schritte benötigen würde, um es zu verifizieren, und er hat volles Vertrauen darin.  Jeder einzelne dieser Schritte muss absolut korrekt sein, sonst ist der ganze Beweis sinnlos.  Wenn Sie eine Million Zahlen von Hand addieren würden, würden Sie dem Ergebnis vertrauen?  Woher wissen Sie wirklich, dass all diese Schritte korrekt sind, dass es in einem dieser Millionen Schritte nicht irgendeinen subtilen Fallstrick gibt, wie einen Bug in einem Computerprogramm?\index{Programmfehler}\index{Bug}  Immerhin haben Sie gelesen, dass berühmte Mathematiker gelegentlich Fehler gemacht haben, und Sie wissen sicherlich, dass Sie bei Ihren Mathehausaufgaben in der Schule auch schon Fehler gemacht haben.

Sie erinnern sich an die Analogie mit einem Computerprogramm.  Sicher, Sie können verstehen, was ein großes Computerprogramm (z.~B. ein Textverarbeitungsprogramm) tut, als ein einziges hochrangiges Konzept oder eine kleine Menge solcher Konzepte, aber Ihre Fähigkeit, es zu verstehen, garantiert keineswegs, dass das Programm korrekt ist und keine versteckten Fehler enthält.  Sogar wenn Sie das Programm selbst geschrieben haben, können Sie das nicht wirklich wissen; die meisten großen Programme, die Sie geschrieben haben, hatten Fehler, die zu einem späteren Zeitpunkt auftauchten, egal wie sorgfältig Sie beim Schreiben waren.

Also gut, es scheint nun also, dass der Grund dafür ist, warum Sie nicht herausfinden können, wie man einen Computer Beweise verifizieren lassen kann, dass jeder Schritt in Wirklichkeit einer Million kleiner Schritte entspricht.  Nun, Sie mögen sagen, ein Computer könne eine Million Berechnungen in einer Sekunde durchführen, also wäre es vielleicht trotzdem praktisch.  Jetzt ist es also eine Rätselaufgabe, wie man die eine Million Schritte, die jedem der Schritte in deutscher oder englischer Sprache entsprechen, herausfinden kann.  Ihr mathematischer Freund hat keine Ahnung, schlägt aber vor, dass man die Antwort vielleicht durch das Studium der Mengenlehre fände.  Eigentlich findet Ihr Freund, dass Sie ein bisschen verrückt sind, dass Sie sich so etwas überhaupt fragen.  Für ihn ist das nicht das, worum es in der Mathematik geht.

Das Thema Mengenlehre taucht immer wieder auf, also beschließen Sie, dass es an der Zeit ist, es sich genauer anzuschauen.

Sie beschließen, vorsichtig anzufangen, und beginnen mit der Lektüre einiger sehr elementarer Bücher über Mengenlehre.  Vieles davon scheint ziemlich offensichtlich zu sein, wie Schnittmengen, Teilmengen und Venn-Diagramme.  Sie blättern durch eines der Bücher; nirgends werden Axiome erwähnt. Ein anderes Buch verweist auf einen Anhang, eine kurze Diskussion, in der eine Reihe von Axiomen erwähnt wird, die "`Zermelo--Fraenkel-Mengenlehre"'\index{Zermelo--Fraenkel-Mengenlehre} und gibt sie auf Deutsch oder Englisch wieder.  Man sieht sie sich an und hat keine Ahnung, was sie wirklich bedeuten oder was man mit ihnen anfangen kann.  Die Kommentare in diesem Anhang besagen, dass ihre Erwähnung dazu diene, Sie mit der Idee vertraut zu machen, dass sie aber für das grundlegende Verständnis nicht notwendig seien und dass sie eigentlich Gegenstand fortgeschrittener Abhandlungen sind, in denen Feinheiten wie ein bestimmtes Paradoxon (Russells Paradoxon\index{Russells Paradoxon}\footnote{Russells Paradoxon setzt voraus, dass es eine Menge $S$ gibt, die eine Sammlung aller Mengen ist, die sich selbst nicht enthalten.  Also, entweder enthält $S$ sich selbst oder nicht.  Enthält diese Menge sich selbst, widerspricht sie ihrer Definition.  Enthält sie sich aber nicht selbst, widerspricht sie ebenso ihrer Definition.  Das Russellsche Paradoxon wird in der ZF-Mengenlehre Theorie aufgelöst, indem man ausschließt, dass eine solche Menge $S$ existiert.}) gelöst werden.  Moment mal - sollten die Axiome nicht ein Ausgangspunkt und kein Endpunkt sein?  Wenn es Paradoxien gibt, welche ohne die Axiome entstehen, woher weiß man dann, dass man nicht zufällig über eines stolpert, wenn man den informellen Ansatz verwendet?

Und nirgends wird in diesen Büchern beschrieben, wie sich "`die gesamte Mathematik aus der Mengenlehre ableiten lässt"', was Sie inzwischen schon ein paar Mal gehört haben.

Sie finden ein fortgeschritteneres Buch über Mengenlehre.  Dieses Buch listet die
Axiome der ZF-Mengentheorie in einfachem Deutsch oder Englisch auf Seite eins auf.  {\em Jetzt} denken Sie, Ihre Suche sei zu Ende und Sie haben endlich die Quelle allen mathematischen Wissens gefunden; Sie müssen nur noch verstehen, was sie bedeuten.  Hier, an einem einzigen Ort, ist die Grundlage für die gesamte Mathematik!  Sie starren voller Ehrfurcht auf die Axiome, rätseln über sie, lernen sie auswendig und hoffen, dass sie Ihnen klar werden, wenn Sie nur lange genug über sie nachdenken.  Natürlich haben Sie nicht die geringste Ahnung, wie der Rest der Mathematik von ihnen "`abgeleitet"' ist; insbesondere, wenn dies die Axiome der Mathematik sind, warum brauchen dann Arithmetik, Gruppentheorie und so weiter ihre eigenen Axiome?
	
Sie fangen an, dieses fortgeschrittene Buch sorgfältig zu lesen, und denken über die Bedeutung jedes Wortes nach, denn Sie wollen der Sache unbedingt auf den Grund gehen.
Das erste, was das Buch tut, ist zu erklären, wie die Axiome zustande gekommen sind, nämlich um das Russellsche Paradoxon zu lösen.  In der Tat scheint das der Hauptzweck ihrer Existenz zu sein; dass sie angeblich dazu verwendet werden können, die gesamte Mathematik abzuleiten, scheint irrelevant zu sein und wird nicht einmal erwähnt.  Wie dem auch sei, Sie fahren fort.  Sie hoffen, dass das Buch Ihnen klar und deutlich, Schritt für Schritt, erklären wird, wie man die Dinge aus den Axiomen ableitet.  Schließlich ist dies der Ausgangspunkt der Mathematik, so wie ein Buch, das die Grundlagen einer Programmiersprache erklärt.  Aber irgendetwas fehlt.  Sie können nicht einmal den ersten Beweis verstehen oder die erste Übung machen.  Symbole wie $\exists$ und $\forall$ durchziehen die Seite, ohne dass erwähnt wird, woher sie kommen oder wie man sie manipuliert. Der Autor geht davon aus, dass man mit ihnen völlig vertraut ist, und sagt Ihnen nicht einmal, was sie bedeuten.  Inzwischen wissen Sie, dass $\exists$ "`es gibt"' bedeutet und dass $\forall$ "`für alle"' bedeutet, aber sollten nicht die Regeln für die Manipulation dieser Symbole Teil der Axiome sein?  Sie haben immer noch keine Idee, wie man die Axiome überhaupt einem Computer beschreiben könnte.
	
Sicherlich gibt es hier etwas ganz anderes als die technische Literatur, die Sie zu lesen gewohnt sind.  In einem Handbuch für Computersprachen wird fast immer sehr deutlich, was alle Symbole bedeuten, was sie genau machen und welche Regeln es gibt, nach denen sie kombiniert werden, und man arbeitet sich von dort aus weiter vor.

Nach einem Blick in vier oder fünf andere Bücher dieser Art kommt man zu der Erkenntnis, dass es noch ein ganzes Studienfach gibt, das man braucht, um die Axiome der Mengenlehre zu verstehen.  Dieses Gebiet wird "`Logik"' genannt.  In der Tat wurde es in einigen Büchern als Voraussetzung empfohlen, aber man hat es einfach nicht realisiert.  Man nahm an, Logik sei, nun ja, einfach nur Logik, etwas, das ein Mensch mit gesundem Menschenverstand intuitiv versteht.  Warum Ihre Zeit mit der Lektüre langweiliger Abhandlungen über symbolische Logik verschwenden, die Manipulation von 1en und 0en, die Computer machen, wenn man das schon weiß?  Aber dies ist eine andere Art von Logik, die Ihnen völlig fremd ist.  Das Thema von {\sc nand} und {\sc nor}-Gattern wird nicht einmal berührt oder hat ohnehin nur mit einem sehr kleinen Teil dieses Bereichs zu tun.

Ihre Suche geht also weiter.  Wenn Sie die ersten einführenden Bücher überfliegen, bekommen Sie eine allgemeine Vorstellung davon, worum es in der Logik geht und was Quantoren ("`Für alle"', "`Es gibt"') bedeuten, aber Sie finden die Beispiele etwas trivial und leicht nervig ("`Alle Hunde sind Tiere"', "`Einige Tiere sind Hunde"' und so weiter).  Aber alles, was Sie wissen wollen, sind die Regeln für die Manipulation der Symbole, damit man sie in der Mengenlehre anwenden kann.  Einige Formeln, die die Beziehungen zwischen den Quantoren ($\exists$ und $\forall$) beschreiben, sind in Tabellen aufgelistet, zusammen mit einigen verbalen Begründungen, um sie zu rechtfertigen.
Wenn Sie herausfinden wollen, ob eine Formel richtig ist, durchlaufen Sie vermutlich die gleiche Art von Denkprozessen, möglicherweise mit Bildern von Hunden und Tieren.  Intuitiv scheinen die Formeln einen Sinn zu ergeben.  Aber wenn Sie sich fragen "`Was sind die Regeln, die ich brauche, um einen Computer dazu zu bringen, herauszufinden, ob diese Formel richtig ist?"', wissen Sie es immer noch nicht.  Man bittet den Computer ja auch nicht, sich Hunde und Tiere vorzustellen.

Sie sehen sich einige fortgeschrittenere Logikbücher an.  Viele von ihnen haben ein einführendes Kapitel mit einer Zusammenfassung der Mengenlehre, was sich als Voraussetzung erweist.  Sie brauchen Logik, um die Mengenlehre zu verstehen, aber anscheinend braucht man auch die Mengenlehre, um Logik zu verstehen!  Diese Bücher stürzen sich gleich auf den Beweis von ziemlich fortgeschrittenen Theoremen über Logik, ohne den geringsten Hinweis darauf zu geben, woher die Logik kommt, mit der sie diese Theoreme beweisen können.

Glücklicherweise stoßen Sie auf ein elementares Buch über Logik, das nach der Hälfte der Lektüre, nach den üblichen Wahrheitstabellen und Metaphern, auf klare und präzise Weise präsentiert, wonach Sie die ganze Zeit gesucht haben: die Axiome!  Sie sind unterteilt in Aussagenlogik (auch Satzlogik genannt) und Prädikatenlogik (auch Logik erster Ordnung genannt),\index{Logik erster Ordnung} mit Regeln, die so einfach und kristallklar sind, dass man jetzt endlich einen Computer programmieren kann, um sie zu verstehen.  Sie sind in der Tat nicht schwieriger als das Erlernen einer Schachpartie.
Soweit es um das geht, was Sie zu brauchen scheinen, hätte man das ganze Buch auf fünf Seiten schreiben können!

{\em Jetzt} glauben Sie, die ultimative Quelle der mathematischen Wahrheit gefunden zu haben.  Also - die Axiome der Mathematik bestehen aus den Axiomen der Logik, zusammen mit den Axiomen der ZF-Mengentheorie. (Inzwischen haben Sie auch herausgefunden wie man die ZF-Axiome aus dem Deutschen oder Englischen in die eigentlichen Symbole der Logik übersetzen kann, die Sie nun nach präzisen, leicht verständlichen Regeln manipulieren können.)

Natürlich verstehen Sie immer noch nicht, wie "`die gesamte Mathematik aus der Mengenlehre abgeleitet werden kann"', aber vielleicht wird sich das zu gegebener Zeit offenbaren.
	
Sie machen sich eifrig daran, die Axiome und Regeln in einen Computer zu programmieren und beginnen sich mit den Theoremen zu befassen, die Sie beweisen müssen, während die Logik entwickelt wird.  Alle Arten von wichtigen Theoremen tauchen auf: das Deduktionstheorem\index{Deduktionstheorem}, das Substitutionstheorem\index{Substitutionstheorem}, der Vollständigkeitssatz der Aussagenlogik\index{Vollständigkeitssatz der Aussagenlogik}, der Vollständigkeitssatz der Prädikatenlogik.
Oh-oh, da scheint es Probleme zu geben.  Sie werden alle schwieriger und schwieriger, und nicht eine davon kann mit den Axiomen und Regeln der Logik, die Sie gerade erhalten haben, abgeleitet werden.  Stattdessen benötigen sie alle eine "`Metalogik"' für ihre Beweise, eine Art Mischung aus Logik und Mengenlehre, die es erlaubt, Dinge {\em über} die Axiome und Theoreme der Logik zu beweisen, anstatt {\em mit} ihnen.

Sie machen trotzdem weiter.  Einen Monat später haben Sie einen Großteil Ihrer Freizeit damit verbracht, den Computer dazu zu bringen, Beweise in der Aussagenlogik zu verifizieren.
Sie haben die Axiome einprogrammiert, aber Sie mussten auch den Deduktionssatz, den Substitutionssatz und den Satz von der Vollständigkeit der Aussagenlogik einprogrammieren, da Sie sich inzwischen damit abgefunden haben, diese Sätze als ziemlich komplexe Zusatzaxiome zu betrachten, da sie nicht aus den Axiomen bewiesen werden können.  Sie können nun den Computer dazu bringen, vollständige, strenge, formale Beweise\index{formaler Beweis} zu überprüfen und sogar zu erzeugen.  Es macht nichts, dass die Beweise 100.000 Schritte haben können - zumindest können Sie jetzt vollständiges, absolutes Vertrauen in sie haben.  Leider sind die einzigen Theoreme, die Sie bewiesen haben, ziemlich trivial, und Sie können sie in wenigen Minuten mit Wahrheitstabellen überprüfen, wenn nicht sogar durch direkte Überprüfung.

Es sieht so aus, als hätte Ihr Freund, der Mathematiker, recht gehabt.  Den Computer dazu zu bringen, ernsthafte Mathematik mit dieser Art von Strenge zu betreiben, scheint fast aussichtslos.  Sogar schlimmer noch, je weiter man kommt, desto mehr "`Axiome"' muss man hinzufügen, da jedes neue Theorem zusätzliche "`metamathematische"' Argumentation zu beinhalten scheint, die nicht formalisiert wurde, und nichts davon lässt sich aus den Axiomen der Logik ableiten.  Nicht nur, dass die Beweise exponentiell wachsen, je weiter man kommt, sondern auch das Programm zu ihrer Verifizierung wird immer größer, je mehr "`Metatheoreme"'\index{Metatheorem}\footnote{Ein Metatheorem ist normalerweise eine Aussage, die zu allgemein ist, um direkt in einer Theorie bewiesen werden zu können.  Zum Beispiel: "`Wenn $n_1$, $n_2$ und $n_3$ ganze Zahlen sind, dann ist $n_1+n_2+n_3$ eine ganze Zahl"' ist ein Satz der Zahlentheorie.  Aber "`für jede ganze Zahl $k > 1$, wenn $n_1, \ldots, n_k$ ganze Zahlen sind, dann ist $n_1+\ldots +n_k$ eine ganze Zahl"' ist ein Metatheorem, das heißt eine Familie von Theoremen, eines für jedes $k$.  Warum dies kein Theorem ist, ist darin begründet, dass die allgemeine Summe $n_1+\ldots +n_k$ (als Funktion von $k$) keine Operation ist, die direkt in der Zahlentheorie definiert werden kann.} man einprogrammiert. Die bisher aufgetauchten Fehler in Ihrem Computerprogramm haben Sie bereits dazu gebracht, den Glauben an die Strenge zu verlieren, die Sie scheinbar erreicht haben, und Sie wissen, dass es nur noch schlimmer wird, je größer Ihr Programm wird.

\subsection{Mathematik und der Nichtfachmann}

\begin{quote}
  {\em Ein echter Beweis ist nicht von einer Maschine überprüfbar, und sogar nicht von einem Mathematiker, wenn er nicht in die Gestalt, die Denkweise des speziellen Bereichs der Mathematik eingeweiht ist, in dem der Beweis angesiedelt ist.}
  \flushright\sc  Davis und Hersh\index{Davis, Phillip J.}
  \footnote{Frei übersetzt nach \cite{Davis}, S.~354.}\\
\end{quote}

Der größte Teil der abstrakten oder theoretischen Mathematik liegt in der Regel außerhalb der Reichweite aller, mit Ausnahme von nur wenigen Spezialisten auf dem jeweiligen Gebiet, die eine notwendige, schwierige Ausbildung absolviert haben, um in diesen Kreis aufgenommen zu werden.  Der typisch intelligente Laie hat keine begründete Hoffnung, viel davon zu verstehen, und auch der spezialisierte Mathematiker hat keine Chance, andere Bereiche zu verstehen.
Es ist wie eine Fremdsprache, für die es kein Wörterbuch gibt, um die Übersetzung nachzuschlagen; die einzige Möglichkeit sie zu lernen besteht darin, ein paar Jahre in dem jeweiligen Land zu leben.
Der Aufwand, der mit dem Erlernen eines Fachgebiets verbunden ist, wird als ein notwendiger Prozess angesehen, um ein tiefes Verständnis zu erlangen.  Natürlich gilt dies sicherlich, wenn man einen bedeutenden Beitrag zu einem Fachgebiet leisten will; insbesondere Beweise zu "`führen"', was wahrscheinlich der wichtigste Teil der Ausbildung eines Mathematikers ist.  Aber ist es auch notwendig, Außenstehenden den Zugang dazu zu verwehren?  Ist es notwendig, dass abstrakte Mathematik für einen Laien so schwer zu begreifen ist?

Ein Computer ist normalerweise überhaupt keine Hilfe.  Die meisten veröffentlichten Beweise sind eigentlich nur eine Reihe von Hinweisen, die in einem informellen Stil geschrieben sind, der ein beträchtliches Wissen auf dem Gebiet erfordert, um sie zu verstehen.  Dies sind die "`echten Beweise"', auf die Davis und Hersh hinweisen\index{informeller Beweis}. Es gibt eine implizite Übereinkunft darüber, dass ein solcher Beweis im Prinzip in einen vollständigen formalen Beweis umgewandelt werden kann.
Es heißt jedoch, dass niemand eine solche Umwandlung jemals versuchen würde, selbst wenn er es könnte, denn das würde vermutlich Millionen von Schritten erfordern (Abschnitt~\ref{dream}).  Leider schließt der informelle Stil automatisch das Verstehen des Beweises für jeden aus, der nicht die notwendige Ausbildung durchlaufen hat.  Es bleibt einem intelligenten Laien nichts anderes übrig als populäre Bücher über tiefe und berühmte Ergebnisse zu lesen; dies kann zwar hilfreich, aber auch irreführend sein, und der Mangel an Details lässt den Leser in der Regel ohne jegliche Möglichkeit zurück, einen Aspekt des beschriebenen Bereichs zu erforschen.

Die Aussagen der Theoreme verwenden oft eine komplizierte Notation, die sie für den Nichtfachmann unzugänglich machen.  Für einen Nichtfachmann, der einen Beweises genauer verstehen möchte, wird der Prozess der Rückverfolgung von Definitionen und Lemmata durch ihre Hierarchie schnell verwirrend und entmutigend.  Lehrbücher werden normalerweise geschrieben, um Mathematiker auszubilden oder um Mathematikern neues Wissen zu vermitteln, und große Lücken in Beweisen werden oft dem Leser als Aufgabe überlassen, der in eine Sackgasse gerät, wenn er oder sie damit nicht weiterkommt.

Ich glaube, dass Computer es irgendwann auch Nichtfachleuten und sogar intelligenten Laien ermöglichen, fast jeden mathematischen Beweis auf jedem Gebiet nachvollziehen zu können.
Metamath ist ein Ansatz in diese Richtung.  Wäre die gesamte Mathematik so leicht zugänglich wie eine Computerprogrammiersprache, könnte ich mir vorstellen, dass es Computerprogrammierern und Hobbyisten, denen es sonst an mathematischer Raffinesse fehlt, die Welt der Theoreme und Beweise in obskuren Fachgebieten zu erkunden und darüber zu staunen.  Vielleicht gelingt es ihnen dann sogar, eigene Ergebnisse zu finden.  Ein enormer Vorteil wäre, dass jeder mit Vermutungen auf jedem Gebiet experimentieren könnte - der Computer würde eine sofortige Rückmeldung darüber geben, ob ein Schritt in einer Schlussfolgerung korrekt ist.

Mathematiker müssen manchmal das Ärgernis über Spinner\index{Spinner} ertragen, denen es an einem grundlegenden Verständnis der Mathematik mangelt, die aber darauf bestehen, dass ihre Beweise für, sagen wir, den Großen Fermatschen Satz\index{Großer Fermatscher Satz}, ernst genommen werden.  Ich denke, ein Teil des Problems ist, dass diese Leute von der informellen mathematischen Sprache in die Irre geführt werden und sie so behandeln, als ob sie gewöhnliches, erklärendes Deutsch oder Englisch lesen würden, und nicht die implizit zugrunde liegende Strenge beachten.  Solche Verrückten sind im Bereich der Computer selten, denn Computersprachen sind viel expliziter, und der Beweis liegt letztlich darin, ob ein Computerprogramm funktioniert oder nicht.  Mit leicht zugänglicher, computergestützter abstrakter Mathematik könnte ein Mathematiker zu einem Verrückten sagen, "`Belästigen Sie mich nicht, bis Sie Ihre Behauptung auf dem Computer bewiesen haben!"'

% 22-May-04 nm
% Attempt to move De Millo quote so it doesn't separate from attribution
% CHANGE THIS NUMBER (AND ELIMINATE IF POSSIBLE) WHEN ABOVE TEXT CHANGES
\vspace{-0.5em}

\subsection{Ein unmöglicher Traum?}\label{dream}

\begin{quote}
  {\em Die Darstellung der Beweise selbst ganz einfacher Theoreme würde unglaublich umfangreiche Bücher füllen.}
    \flushright\sc  Robert E. Edwards\footnote{Frei übersetzt nach \cite{Edwards}, S.~68.}\\
\end{quote}\index{Edwards, Robert E.}

\begin{quote}
  {\em Oh, natürlich {\em macht} das nie jemand wirklich.  Das würde ewig dauern!  Du zeigst nur, dass du es könntest, das reicht aus.}
    \flushright\sc  "`Der ideale Mathematiker"'
    \index{Davis, Phillip J.}\footnote{Frei übersetzt nach \cite{Davis},
\sc  "`The Ideal Mathematician"' S.~40.}\\
\end{quote}

\begin{quote}
  {\em Es gibt ein Theorem in der elementaren Notation der Mengenlehre, das dem arithmetischen Theorem `$1000+2000=3000$' entspricht.  Die Formel wäre furchtbar lang, denn selbst wenn [man] die Definitionen kennt und gebeten wird, die lange Formel entsprechend zu vereinfachen, wird man wahrscheinlich Fehler machen und zu einem falschen Ergebnis kommen.}
    \flushright\sc  Hao Wang\footnote{Frei übersetzt nach \cite{Wang}, S.~140.}\\
\end{quote}\index{Wang, Hao}

% 22-May-04 nm
% Attempt to move De Millo quote so it doesn't separate from attribution
% CHANGE THIS NUMBER (AND ELIMINATE IF POSSIBLE) WHEN ABOVE TEXT CHANGES
\vspace{-0.5em}

\begin{quote}
  {\em {\em Die Principia Mathematica} war die krönende Leistung der Formalisten.  Es war auch der Todesstoß für die formalistische Sichtweise. \ldots {[Rus\-sell]} gelang es in drei riesigen Bänden nicht, über die elementaren Fakten der Arithmetik hinauszukommen.  Er zeigte, was prinzipiell möglich ist und was in der Praxis nicht möglich ist.  Wäre der mathematische Prozess wirklich eine streng logische Abfolge, dann würden wir immer noch unsere Finger zählen. \ldots Ein Theoretiker schätzt\ldots, dass die Darstellung einer von Ramanujans Vermutungen mithilfe der Mengenlehre und elementarer Analysis etwa zweitausend Seiten umfassen würde; die Länge einer Ableitung aus ersten Prinzipien ist fast unvorstellbar \ldots Die Wahrscheinlichkeitstheoretiker argumentieren, dass \ldots jeder sehr lange Beweis bestenfalls als "`wahrscheinlich richtig"' angesehen werden kann. \ldots}
  \flushright\sc Richard de Millo et. al.\footnote{Frei übersetzt nach \cite{deMillo}, S.~269,
  271.}\\
\end{quote}\index{de Millo, Richard}

Einige Autoren haben den Eindruck erweckt, dass die Art von absoluter Strenge, wie sie Metamath bietet, ein unmöglicher Traum ist und suggeriert, dass ein vollständiger, formaler Beweis eines typischen Theorems Millionen von Schritten in unzähligen Bänden von Büchern erfordern würde.  Selbst wenn es möglich wäre, dann wird manchmal angenommen, dass jegliche Bedeutung bei einer solchen monströsen, langwierigen Überprüfung verloren gehen würde.\index{informeller Beweis}\index{Beweislänge}

Diese Autoren gehen jedoch davon aus, dass man für eine solche Art von vollständiger, formaler Verifikation einen Beweis in einzelne primitive Schritte herunterbrechen müsse, die sich direkt auf die Axiome beziehen.  Dies ist nicht notwendig.  Es gibt keinen Grund, nicht auf bereits bewiesene Theoreme zurückzugreifen, anstatt sie immer wieder aufs Neue zu beweisen.

Genauso wichtig ist es, dass Definitionen\index{Definition} auf dem Beweisweg eingeführt werden können, so dass sehr komplexe Formeln mit wenigen Symbolen dargestellt werden können. Wird dies nicht getan, können absurd lange Formeln entstehen.  Zum Beispiel würde allein das Formulieren des Gödelschen Unvollständigkeitssatzes\index{Gödelscher Unvollständigkeitssatz}, das mit einer überschaubaren Anzahl definierter Symbole ausgedrückt werden kann, etwa 20.000 primitive Symbole erfordern\index{Boolos, George S.}\footnote{George S.\ Boolos, Vortrag am Massachusetts Institute of Technology, Frühling 1990.}.
Ein extremes Beispiel ist Bourbakis\label{bourbaki} Sprache für die Mengenlehre, die \\ 4.523.659.424.929 Symbole plus 1.179.618.517.981 "`unterscheidende Verbin\-dun\-gen"' (Linien, die Symbolpaare miteinander verbinden, die normalerweise unterhalb oder oberhalb der Formel gezogen werden) erfordert, um die Zahl "`eins"' auszudrücken \cite{Mathias}\index{Mathias, Adrian R. D.}\index{Bourbaki, Nicolas}.
% http://www.dpmms.cam.ac.uk/~ardm/

Eine Hierarchie\index{Hierarchie} von Theoremen und Definitionen ermöglicht es, dass exponentiell wachsende Formelgrößen und primitive Beweisschritte mit nur einer linear wachsenden Anzahl von verwendeten Symbole beschrieben werden können.  Natürlich ist dies die Art und Weise, wie sie auch in der gewöhnlichen, informellen Mathematik normalerweise praktiziert wird, aber mit Metamath\index{Metamath} kann dies mit absoluter Strenge und Präzision geschehen.

\subsection{Schönheit}


\begin{quote}
  {\em Aus dem Paradies, das Cantor uns geschaffen, soll uns niemand vertreiben können.}
   \flushright\sc  David Hilbert\footnote{"`Über das Unendliche"' in {\em Mathematische Annalen}, 95. Band, Verlag von Julius Springer, Berlin 1926, S. 170. Außerdem zitiert in \cite{Moore}, S.~131.}\\
\end{quote}\index{Hilbert, David}

\needspace{3\baselineskip}
\begin{quote}
  {\em Die Mathematik hat nicht nur Wahrheit inne, sondern auch eine höchste Schönheit --- eine kalte und strenge Schönheit, wie die einer Skulptur.}
    \flushright\sc  Bertrand Russell\footnote{Frei übersetzt nach \cite{Russell}.}\\
\end{quote}\index{Russell, Bertrand}

\begin{quote}
  {\em Euklid allein hat die schiere Schönheit gesehen.}
  \flushright\sc Edna Millay\footnote{Frei übersetzt nach dem Zitat in \cite{Davis}, S.~150.}\\
\end{quote}\index{Millay, Edna}

\begin{quote}
	{\em Das Wissen eines Menschen macht seine Miene strahlend, und seine strengen Züge lösen sich.}
	\flushright\sc Kohelet/Prediger 8:1\footnote{Ergänzung der Übersetzer, aus {\em Die Bibel - Einheitsübersetzung}, Katholische Bibelanstalt, 1980, Stuttgart.}\\
\end{quote}\index{Bibel}

Für die meisten Menschen ist die abstrakte Mathematik weit weg, fremd und unverständlich.  Viele populäre Bücher haben versucht, etwas von dem Sinn der Schönheit berühmter Theoreme zu vermitteln.  Aber selbst ein intelligenter Laie hat nur eine allgemeine Vorstellung davon, worum es in einem Theorem geht, und erhält kaum die Werkzeuge, die er braucht um sie zu nutzen.  Traditionell erfordert es erst ein jahrelanges, mühsames Studium, um die für ein tiefes Verständnis erforderlichen Konzepte zu erfassen.
Metamath\index{Metamath} ermöglicht es, den Beweis des Satzes aus einer ganz anderen Perspektive anzugehen, indem man die Formeln und Definitionen Schicht für Schicht auseinandernimmt, bis man ein völlig anderes Verständnis gewonnen hat.
Jeder Schritt des Beweises ist nachvollziehbar, mit absoluter Präzision zusammengesetzt, und kann sofort wie durch ein Mikroskop mit einer beliebigen Vergrößerung, so stark ist wie Sie es wünschen, weiter inspiziert werden.

Ein Beweis an sich kann als ein Kunstobjekt angesehen werden.  Das Konstruieren eines eleganten Beweis ist eine Kunst.  Sobald ein berühmtes Theorem bewiesen ist, werden oft beträchtliche Anstrengungen unternommen, um einfachere und leichter verständliche Beweise zu finden.  Das Erstellen und Vermitteln eleganter Beweise ist ein Hauptanliegen von Mathematikern.  Metamath ist eine Möglichkeit, eine gemeinsame Sprache für die Archivierung und Bewahrung dieser Informationen zu bieten.

Die Länge eines Beweises kann bis zu einem gewissen Grad als objektives Maß für seine "`Schönheit"' sein, da kürzere Beweise gewöhnlich als eleganter gelten.  In der Mengenlehre-Datenbasis\texttt{set.mm}\index{Mengenlehre-Datenbasis (\texttt{set.mm})}%
\index{Metamath Proof Explorer}, die zusammen mit dem Metamath-Programm bereitgestellt wird, war und ist es ein Ziel, alle Beweise so kurz wie möglich zu halten.

\needspace{4\baselineskip}
\subsection{Einfachheit}

\begin{quote}
  {\em Gott hat die Menschen einfach und aufrichtig erschaffen, aber manche wollen alles kompliziert haben.}
    \flushright\sc Kohelet/Prediger 7:29\footnote{{\em Die Bibel im heutigem Deutsch}, Deutsche Bibelgesellschaft, 1982, Stuttgart. (Anm. der Übersetzer: Das hier wiedergegebene Zitat entspricht dem Zitat aus dem englischen Original; in anderen Ausgaben der Bibel wird diese Stelle häufig etwas anders übersetzt, z.B. {\em Gott hat die Menschen rechtschaffen gemacht, aber sie haben sich in allen möglichen Berechnungen versucht} in {\em Die Bibel - Einheitsübersetzung}, Katholische Bibelanstalt, 1980, Stuttgart.)}\\
\end{quote}\index{Bibel}

\needspace{3\baselineskip}
\begin{quote}
  {\em Die ganzen Zahlen hat der liebe Gott gemacht, alles andere ist Menschenwerk.}
    \flushright\sc Leopold Kronecker\footnote{{\em Jahresbericht
	der Deutschen Mathematiker-Vereinigung }, Ausg. 2, S. 19.}\\
\end{quote}\index{Kronecker, Leopold}

\needspace{3\baselineskip}
\begin{quote}
  {\em Denn das Klare und leicht Faßliche zieht uns an, das Verwickelte schreckt uns ab.}
    \flushright\sc David Hilbert\footnote{{\em Mathematische Probleme} (1900), S. 254.}\\
\end{quote}\index{Hilbert, David}

Die Metamath\index{Metamath}-Sprache ist einfach und spartanisch.  Metamath betrachtet alle mathematischen Ausdrücke als einfache Abfolgen von Symbolen, die an sich keine Bedeutung haben.
Die übergeordneten oder "`metamathematischen"' Grundsätze, auf denen Metamath basiert, sind so einfach wie möglich gehalten.  Jeder einzelne Schritt in einem Beweis basiert auf einem einzigen Grundkonzept, nämlich der Ersetzung einer Variablen durch einen Ausdruck, so dass im Prinzip fast jeder, ob Mathematiker oder nicht, komplett verstehen kann, wie er zustande gekommen ist.

Eine der grundlegendsten Anwendungen von Metamath\index{Metamath} ist die Entwicklung der Grundlagen der Mathematik\index{Grundlagen der Mathematik} von Anfang an.  Dies geschieht in der Mengenlehre-Datenbasis, die in dem  Metamath-Paket mitgeliefert wird und Gegenstand von Kapitel~\ref{fol} ist. Jede Sprache (eine Metasprache\index{Metasprache}) die zur Beschreibung der Mathematik verwendet wird (eine Objektsprache\index{Objektsprache}), muss selbst mathematische Konzepte beinhalten.  Aber es ist wünschenswert, diese Konzepte auf ein absolutes Minimum zu beschränken, nämlich auf die für die Anwendung der axiomatisch spezifizierten Inferenzregeln benötigten.  Für jede Metasprache gibt es ein "`Henne-Ei'-Problem, ähnlich wie bei einem Zirkelschluss: Man muss die Gültigkeit der Mathematik der Metasprache voraussetzen, um die Gültigkeit der Mathematik der Objektsprache zu beweisen.  Der mathematische Anteil von Metamath selbst ist recht begrenzt.  Wie die Regeln eines Schachspiels sind die  wesentlichen Konzepte so einfach, dass praktisch jeder in der Lage sein sollte, sie zu verstehen (auch wenn das an sich nicht dazu führt, dass man wie ein Meister spielt).  Die Symbole, mit denen Metamath arbeitet, haben an sich keine eigene Bedeutung.  Die Interpretation der Axiome, die Sie für Metamath festlegen, durch Sie selbst ist es, was ihnen Bedeutung verleiht.  Metamath ist ein Versuch, das mathematisches Denken auf seine pure Essenz zu reduzieren und Ihnen genau zu zeigen, wie die Symbole manipuliert werden.

Philosophen und Logiker haben es aus verschiedenen Gründen oft für wichtig gehalten, "`schwache"' Teilbereiche der Logik\index{schwache Logik}\cite{Anderson}\index{Anderson, Alan Ross} \cite{MegillBunder}\index{Megill, Norman}\index{Bunder, Martin}, andere unkonventionelle Systeme der Logik (wie z.B. "`modale"' Logik\index{modale Logik} \cite[Kap.\ 27]{Boolos}\index{Boolos, George S.}), und Quantenlogik\index{Quantenlogik} in der Physik\cite{Pavicic}\index{Pavi{\v{c}}i{\'{c}}, M.} zu untersuchen.  Metamath\index{Metamath} bietet einen Rahmen, in dem solche Systeme mit einer absoluten Präzision ausgedrückt werden können, die alle zugrundeliegenden metamathematischen Annahmen präzise und kristallklar darlegt.

Einige philosophische Denkschulen wie der Intuitionismus\index{Intuitionismus} und der Konstruktivismus\index{Konstruktivismus} fordern, dass die jedem mathematischen System zugrunde liegenden Begriffe so einfach und konkret wie möglich seien.  Metamath sollte den Anforderungen dieser Philosophien entsprechen.  Metamath muss die Symbole, Axiome\index{Axiom}, und Regeln\index{Regel} für eine bestimmte Theorie, vom Skeptischen (wie beim Intuitionismus\index{Intuitionismus}
\footnote{Der Intuitionismus akzeptiert nicht das Gesetz des ausgeschlossenen Dritten ("`entweder ist etwas wahr oder es ist nicht wahr"').  Siehe \cite[S.~xi]{Tymoczko}\index{Tymoczko, Thomas} für Diskussionen und Referenzen zu diesem Thema.  Betrachten Sie das Theorem "`Es gibt irrationale Zahlen $a$ und $b$, so dass $a^b$ rational ist"'.  Ein Intuitionist würde den folgenden Beweis ablehnen:  Wenn $\sqrt{2}^{\sqrt{2}}$ rational ist, sind wir fertig.  Ansonsten gilt $a=\sqrt{2}^{\sqrt{2}}$ und $b=\sqrt{2}$. Dann ist $a^b=2$, was eine rational Zahl ist.})
bis zum Kühnen (wie zum Beispiel das Auswahlaxiom in der Mengenlehre
\footnote{Das Auswahlaxiom\index{Auswahlaxiom} besagt, dass bei einer beliebigen Sammlung von paarweise disjunkten, nicht leeren Mengen eine Menge existiert, die mit jeder Menge der Sammlung genau ein Element gemeinsam hat.  Es wird verwendet, um viele wichtige Theoreme in der Standardmathematik zu beweisen.  Einige Philosophen lehnen es ab, weil die Existenz einer Menge behauptet wird, ohne ihre Elemente benennen zu können\cite[S.~154]{Enderton}\index{Enderton, Herbert B.}.  In einer Grundlage für die Mathematik, die auf Quine\index{Quine, Willard Van Orman} zurückgeht und sich nicht als inkonsistent erwiesen hat, erweist sich das Auswahlaxiom als falsch\cite[S.~23]{Curry}\index{Curry, Haskell B.}.  Der \texttt{show trace{\char`\_}back}-Befehl des Metamath-Programms ermöglicht es Ihnen herauszufinden, ob das Auswahlaxiom oder ein anderes Axiom in einem Beweis angenommen wurde.}\index{\texttt{show trace{\char`\_}back}-Befehl}) beigebracht werden.
	
Die Einfachheit der Metamath-Sprache erlaubt es dem Algorithmus (Computerprogramm),
der die Gültigkeit eines Metamath-Beweises verifiziert, einfach und
robust zu sein.  Sie können darauf vertrauen, dass die Theoreme, die er verifiziert, wirklich aus Ihren Axiomen abgeleitet werden können.

\subsection{Strenge}

\begin{quote}
  {\em Rigorosität wurde bei den Griechen zu einem Ziel \ldots Aber die Bemühungen, die Strenge bis zum Äußersten zu treiben, haben in eine Sackgasse geführt, in der es keine Einigung mehr darüber gibt, was sie wirklich bedeutet.  Die Mathematik bleibt lebendig und vital, aber nur auf einer pragmatischen Basis.}
    \flushright\sc  Morris Kline\footnote{Frei übersetzt nach \cite{Kline}, S.~1209.}\\
\end{quote}\index{Kline, Morris}

Kline bezieht sich auf eine viel tiefere Art von Strenge als die, die wir in diesem Abschnitt erörtern werden.  Der Gödelsche Unvollständigkeitssatz\index{Gödelscher Unvollständigkeitssatz} zeigte, dass es unmöglich ist, in der Standardmathematik absolute Strenge zu erreichen, weil wir nie beweisen können, dass die Mathematik konsistent (frei von Widersprüchen) ist.\index{konsistente Theorie}  Wenn die Mathematik konsistent ist, werden wir es nie wissen, sondern müssen uns auf den Glauben darauf verlassen.  Wenn die Mathematik inkonsistent ist, können wir bestenfalls darauf hoffen, dass in der Zukunft ein kluger Mathematiker die Inkonsistenz entdecken wird.  In diesem Fall würden die Axiome wahrscheinlich leicht überarbeitet werden, um die Inkonsistenz zu beseitigen, wie im Fall des Russell'schen Paradoxons geschehen ist. Aber der Großteil der Mathematik wäre von einer solchen Entdeckung wahrscheinlich nicht betroffen.  Das Russell-Paradoxon hatte zum Beispiel keinen Einfluss auf die meisten bemerkenswerten Ergebnisse, die von Mathematikern des 19. Jahrhunderts und früher erzielt wurden.  Es widerlegte hauptsächlich einige der Arbeiten von Gottlob Frege über die Grundlagen der Mathematik in den späten 1800er Jahren; tatsächlich inspirierte Freges Arbeit Russells Entdeckung.  Trotz dieses Paradoxons enthält Freges Werk wichtige Konzepte, die die moderne Logik maßgeblich beeinflusst haben.  Kline's {\em Mathematics, The Loss of Certainty} \cite{Klinel}\index{Kline, Morris} enthält eine interessante Diskussion zu diesem Thema.

Was mit absoluter Gewissheit\index{Gewissheit} erreicht werden {\em kann} ist das Wissen, dass unter der Annahme, dass die Axiome konsistent und wahr sind, die Ergebnisse, die aus ihnen abgeleitet werden, wahr sind.  Ein Teil der Schönheit der Mathematik ist, dass sie der einzige Bereich menschlichen Strebens ist, in dem absolute Gewissheit in diesem Sinne erreicht werden kann.  Eine mathematische Wahrheit wird bis in alle Ewigkeit wahr bleiben.  Dennoch ist unser tatsächliches Wissen darüber, ob eine bestimmte Aussage eine mathematische Wahrheit darstellt, jedoch nur so sicher wie die Korrektheit des Beweises, der dieser Aussage zugrunde liegt.
Wenn der Beweis einer Aussage fragwürdig oder vage ist, können wir nicht absolutes Vertrauen in die von der Aussage behaupteten Wahrheit haben.

Schauen wir uns einige traditionelle Arten an, Beweise auszudrücken.

Außer auf dem Gebiet der formalen Logik\index{formale Logik} sind fast alle traditionellen Beweise in der Mathematik eigentlich gar keine Beweise, sondern eher Beweisskizzen oder Hinweise darauf, wie der Beweis zu konstruieren ist.  Viele Lücken\index{Lücken in Beweisen} werden dem Leser zum Ausfüllen überlassen. Dafür gibt es mehrere Gründe.  Erstens wird in der mathematischen Literatur gewöhnlich angenommen, dass die Person, die den Beweis liest, ein Mathematiker ist, der mit dem beschriebenen Spezialgebiet vertraut ist, und dass die fehlenden Schritte für einen solchen Leser offensichtlich sind oder er zumindest in der Lage ist, sie zu ergänzen.
Diese Haltung ist für professionelle Mathematiker des Fachgebiets in Ordnung, aber leider hat sie oft den Nachteil, dass sie den Rest der Welt, einschließlich der Mathematiker anderer Fachgebiete, vom Verstehen des Beweises ausschließt.
Wir haben eine mögliche Lösung für dieses Problem auf S.~\pageref{envision} diskutiert.
Zweitens wird oft angenommen, dass ein vollständig formaler Beweis {index{formaler Beweis} unzählige Millionen von Symbolen erfordern würde (Abschnitt~\ref{dream}). Dies könnte richtig sein, wenn der Beweis direkt durch die Axiome der Logik und der Mengenlehre ausgedrückt würde.  Aber es ist normalerweise nicht richtig, wenn wir uns
eine Hierarchie von Definitionen und Theoremen erlauben, auf denen wir aufbauen können, und eine Notation verwenden, die es uns erlaubt, neue Symbole, Definitionen und Theoreme auf eine genau spezifizierte Weise einzuführen.

Selbst in der formalen Logik\index{formale Logik} enthalten formale Beweise\index{formaler Beweis}, die als vollständig angesehen werden, immer noch versteckte oder implizite Informationen.
Zum Beispiel wird ein "`Beweis"' gewöhnlich als eine Folge von wffs,\index{wohlgeformte Formel (wff)}
\footnote{Eine {\em wff} oder eine wohlgeformte Formel ist ein mathematischer Ausdruck (eine Kette von Symbolen), der nach einigen präzisen Regeln aufgebaut ist.  Ein formales mathematisches System\index{formales System} enthält (1) die Regeln für die Konstruktion von syntaktisch korrekten wffs,\index{Syntax-Regeln} (2) eine Liste von Ausgangs-wffs, genannt Axiome, \index{Axiom} und (3) eine oder mehrere Regeln, die vorschreiben, wie man neue wffs, genannt Theoreme\index{Theorem}, aus den Axiomen oder vorher abgeleiteten Theoremen ableitet.  Ein Beispiel für ein solches System ist in Metamaths\index{Metamath} Mengenlehre-Datenbasis enthalten, die ein formales System definiert, aus dem die gesamte Standardmathematik abgeleitet werden kann.
Abschnitt~\ref{startf} führt Sie durch ein vollständiges Beispiel für ein formales Systems, und Sie sollten ihn überfliegen, wenn Sie mit dem Konzept nicht vertraut sind.}
definiert, von denen jedes ein Axiom ist oder aus einer Regel folgt, die auf vorherige wffs in der Sequenz angewendet wird.  Der implizite Teil des Beweises ist der Algorithmus, mit dem eine Folge von Symbolen als gültige wff verifiziert wird, wenn die Definition einer wff gegeben ist.  Der Algorithmus ist in diesem Fall recht einfach, aber damit ein Computer den Beweis verifizieren kann, muss er den Algorithmus in sein Verifikationsprogramm eingebaut haben.
\footnote{Es ist natürlich möglich, die Syntax der wff-Konstruktion außerhalb des Programms mit einer geeigneten Eingabesprache selbst zu spezifizieren (die Metamath-Sprache ist ein Beispiel dafür), aber einige Programme zur Beweisverifikation oder zum Beweisen von Theoremen sind nicht in der Lage, die wff-Syntax auf eine solche Weise zu erweitern.}
Wenn man sich ausschließlich mit Axiomen und elementaren wffs beschäftigt, ist es einfach, einen solchen Algorithmus zu implementieren.  Aber wenn mehr und mehr Definitionen zur Theorie hinzugefügt werden, um diese kompakter zu machen, wird der Algorithmus immer komplizierter.
Ein Computerprogramm, das den Algorithmus implementiert, wird größer und mit jeder neuen Definition schwieriger zu verstehen und damit anfälliger für Bugs.  Je größer das Programm, desto misstrauischer wird der Mathematiker gegenüber der Gültigkeit seiner Algorithmen.  Dies gilt insbesondere, weil Computerprogramme von Natur aus schwer nachzuvollziehen sind, und nur wenige Menschen Spaß daran haben, sie manuell im Detail zu überprüfen.

Metamath\index{Metamath} verfolgt einen anderen Ansatz.  Das "`Wissen"' von Metamath beschränkt sich auf die Fähigkeit, Variablen durch Ausdrücke zu ersetzen, und zwar unter einigen einfachen Einschränkungen.  Sobald der grundlegende Algorithmus von Metamath als getestet und vielleicht unabhängig bestätigt wurde, kann man ihm ein für alle Mal vertrauen.  Die Informationen, die Metamath benötigt, um Mathematik zu verstehen, ist vollständig in dem Wissensbestand enthalten, der Metamath zugrunde gelegt wird.  Jegliche Fehler in der Argumentation können nur Fehler in den Axiomen oder Definitionen sein, die in diesem Wissensbestand enthalten sind.
Als eine "`konstruktive"' Sprache\index{konstruktive Sprache} hat Metamath keine bedingte Verzweigungen oder Schleifen, wie sie Computerprogramme schwer entschlüsselbar machen; stattdessen kann die Sprache nur neue Folgen von Symbolen aus vorherigen Folgen von Symbolen aufbauen.

Die Einfachheit der Regeln, die Metamath zugrunde liegen, macht Metamath nicht nur leicht zu erlernen, sondern verleiht Metamath auch ein hohes Maß an Flexibilität. Metamath ist zum Beispiel nicht auf die Beschreibung der Standardlogik erster Ordnung\index{Logik erster Ordnung} beschränkt; Logik höherer Ordnung\index{Logik höherer Ordnung (HOL)}} und Teilbereiche der Logik\index{schwache Logik} können ebenso einfach beschrieben werden.
Metamath gibt Ihnen die Freiheit, jede beliebige wff-Notation zu definieren, die Sie bevorzugen; es hat keine eingebaute Interpretation der Syntax einer wff.\index{wohlgeformte Formel (wff)} Mit geeigneten Axiomen und Definitionen kann Metamath sogar Dinge über sich selbst beschreiben und beweisen.\index{Metamath!Selbstbeschreibung} (John Harrison\index{Harrison, John} erörtert das "`Reflexions"'-Prinzip\index{Reflexionsprinzip} in selbstbeschreibenden Systemen in \cite{Harrison}.)

Die Flexibilität von Metamath erfordert, dass die Beweise sehr viele Details enthalten, viel mehr als in einem gewöhnlichen "`formalen"' Beweis.\index{formaler Beweis}
Zum Beispiel besteht in einem gewöhnlichen formalen Beweis ein einzelner Schritt aus der Anzeige der wff, die diesen Schritt darstellt.  Damit ein Computerprogramm verifizieren kann dass der Schritt gültig ist, muss es zunächst prüfen, ob die angezeigte Symbolfolge eine gültige wff\index{automatisierte Beweisverifizierung} ist. Die meisten Prüfprogramme haben zumindest eine grundlegende wff-Syntax eingebaut.
Metamath hat kein fest verdrahtetes Wissen darüber eingebaut, was eine wff ist; stattdessen muss jede wff explizit konstruiert werden, basierend auf Regeln, die in einer Datenbasis vorhanden wffs definieren.
Daher kann ein einzelner Schritt in einem gewöhnlichen formalen Beweis vielen Schritten in einem Metamath-Beweis entsprechen. Trotz der größeren Anzahl von Schritten bedeutet dies jedoch nicht, dass ein Metamath-Beweis wesentlich größer sein muss als ein gewöhnlicher formaler Beweis. Der Grund dafür ist, dass wir wissen, da wir die wff von Grund auf neu konstruiert haben, was die wff ist - also gibt es keinen Grund, sie anzuzeigen.  Wir müssen nur auf eine Folge von Aussagen verweisen, die sie konstruieren.  In gewissem Sinne ist die Darstellung der wff in einem gewöhnlichen formalen Beweis ein impliziter Beweis für seine eigene Gültigkeit als wff; Metamath macht den Beweis einfach explizit. (Abschnitt~\ref{proof} beschreibt die Beweisnotation von Metamath.)

\section{Computer und Mathematiker}

\begin{quote}
  {\em Der Computer ist wichtig, aber nicht für die Mathematik.}
  \flushright\sc  Paul Halmos\footnote{Frei übersetzt nach dem Zitat in \cite{Albers}, S.~121.}\\
\end{quote}\index{Halmos, Paul}

Reine Mathematiker stehen Computern seit jeher gleichgültig gegenüber, bis hin zur Verachtung.\index{Computer und reine Mathematik}  Die Computerwissenschaft/Informatik selbst wird manchmal in den banalen Bereich der "`angewandten"' Mathematik eingeordnet, die vielleicht für die reale Welt wichtig, aber für diejenigen, die nach den tiefsten Wahrheiten der Mathematik suchen, intellektuell wenig aufregend ist.  Vielleicht liegt ein Grund für diese Einstellung gegenüber Computern darin, dass es wenig oder gar keine Computersoftware gibt, die ihren Bedürfnissen gerecht wird, und es mag die allgemeine Meinung vorherrschen, dass eine solche Software gar nicht existieren kann.  Auf der einen Seite gibt es die praktischen Computeralgebrasysteme, die erstaunliche Symbolmanipulationen in den Bereichen Algebra und Infinitesimalrechnung\index{Computeralgebrasystem} durchführen können, aber nicht einmal den einfachsten Existenzsatz beweisen, wenn die Vorstellung eines Beweises überhaupt vorhanden ist.  Andererseits gibt es spezialisierte automatische Theorembeweiser,\index{automatisches Theorembeweisen} die technisch gesehen korrekte Beweise generieren können.  Aber manchmal sind ihre speziellen Eingabenotationen sehr kryptisch, und ihre Ausgaben werden als lange, unelegant wirkende, unverständliche Beweise wahrgenommen. 
Die Ausgabe kann mit Misstrauen betrachtet werden, da das Programm, das sie erzeugt, in der Regel sehr groß ist und seine Größe das Potenzial für Bugs\index{Programmfehler}\index{Bug} erhöht.
Ein solcher Beweis kann nur dann als vertrauenswürdig angesehen werden, wenn er von einem Menschen unabhängig verifiziert und "`verstanden"' wurde, aber niemand möchte seine Zeit mit einer solch langweiligen, undankbaren Aufgabe verschwenden.



\needspace{4\baselineskip}
\subsection{Dem Computer vertrauen}

\begin{quote}
  {\em \ldots ich halte die quasi-empirische Interpretation von Computerbeweisen nach wie vor für die plausiblere.\ldots Da nicht alles, was angeblich ein Computerbeweis ist, als gültig anerkannt werden kann, was sind die mathematischen Kriterien für akzeptable Computerbeweise?}
    \flushright\sc  Thomas Tymoczko\footnote{Frei übersetzt nach \cite{Tymoczko}, S.~245.}\\
\end{quote}\index{Tymoczko, Thomas}

In einigen Fällen waren Computer unverzichtbare Werkzeuge für den Beweis berühmter Theoreme.  Aber wenn ein Beweis so lang und obskur ist, dass er praktisch nur mit einem Computer überprüft werden kann, wird er vage als verdächtig empfunden.
Zum Beispiel der Beweis des berühmten Vier-Farben-Satzes\index{Vier-Farben-Satz}\index{Beweislänge} ("`es werden nicht mehr als vier Farben benötigt, um eine Landkarte so einzufärben, dass zwei benachbarte Länder nicht die gleiche Farbe haben"'), kann derzeit nur mit Hilfe eines sehr komplexen Computerprogramms durchgeführt werden, das ursprünglich 1200 Stunden Rechenzeit erforderte. Es hat eine beträchtliche Debatte darüber gegeben, ob man einem solchen Beweis trauen kann und ob ein solcher Beweis "`echte"' Mathematik ist \cite{Swart}\index{Swart, E. R.}.\index{Computern vertrauen}

Unter normalen Umständen würde jedoch selbst ein skeptischer Mathematiker ein sehr großes Vertrauen in das Ergebnis der Multiplikation zweier Zahlen auf einem Taschenrechner haben, auch wenn die genauen Einzelheiten des Vorgangs vor dem Benutzer verborgen sind.
Selbst der Überprüfung, dass eine große Zahl eine Primzahl ist, durch einen Supercomputer wird vertraut, vor allem, wenn es eine unabhängige Überprüfung gibt; niemand macht sich die Mühe über die philosophische Bedeutung des "`Beweises"' zu diskutieren, obwohl der eigentliche Beweis so umfangreich wäre, dass es völlig unpraktisch wäre, ihn jemals auf Papier zu bringen.  Es scheint, dass man dann dem Computer vertrauen kann, wenn der vom Computer verwendete Algorithmus einfach genug ist, um leicht verstanden zu werden.

Metamath\index{Metamath} macht sich diese Philosophie zu eigen.  Die Einfachheit der Sprache macht es leicht sie zu erlernen, und aufgrund seiner Einfachheit kann man prinzipiell absolut sicher sein, dass ein Beweis korrekt ist. Alle Axiome, Regeln und Definitionen sind jederzeit einsehbar, da sie vom Benutzer selbst definiert werden.
Es gibt keine versteckten oder eingebauten Regeln, die anfällig für subtile Bugs\index{Programmfehler}\index{Bug} sein könnten.
Der grundlegende Algorithmus, im Herzstück von Metamath ist einfach und feststehend, und es kann mit einem an Gewissheit grenzenden Grad an Vertrauen davon ausgegangen werden, dass er fehlerfrei und robust ist.
Unabhängig geschriebene Implementierungen des Metamath-Verifizierers können den Restzweifel eines Skeptikers noch weiter reduzieren.
Es gibt inzwischen mehr als ein Dutzend solcher Implementierungen, die von vielen verschiedenen Personen geschrieben wurden.

\subsection{Dem Mathematiker vetrauen}\label{trust}

\begin{quote}
  {\em Es gibt keinen Algebraiker oder Mathematiker, der in seiner Wissenschaft so bewandert ist, dass er sofort volles Vertrauen in eine von ihm entdeckte Gültigkeit setzt oder sie als etwas anderes als eine bloße Wahrscheinlichkeit betrachtet.  Jedes Mal, wenn er seine Beweise durchgeht, wächst sein Vertrauen; aber noch mehr durch die Zustimmung seiner Freunde; und wird durch die allgemeine Zustimmung und den Beifall der gelehrten Welt zu seiner höchsten Vollkommenheit erhoben.}
  \flushright\sc David Hume\footnote{{\em A Treatise of Human Nature}, Frei übersetzt nach dem Zitat in \cite{deMillo}, S.~267.}\\
\end{quote}\index{Hume, David}

\begin{quote}
  {\em Stanislaw Ulam schätzt, dass Mathematiker jedes Jahr 200.000 Theoreme veröffentlichen.  Einige davon werden später widerlegt oder anderweitig nicht anerkannt, andere werden angezweifelt, und die meisten werden ignoriert.}
  \flushright\sc Richard de Millo et. al.\footnote{Frei übersetzt nach \cite{deMillo}, S.~269.}\\
\end{quote}\index{Ulam, Stanislaw}

Unabhängig davon, ob man dem Computer vertrauen kann oder nicht, werden sich Menschen natürlich gelegentlich irren. Nur die denkwürdigsten Beweise werden von unabhängiger Seite verifiziert, und von diesen erlangen nur eine Handvoll wirklich großartiger Beweise den Status von "`bekannten"' mathematischen Wahrheiten, die ohne einen zweiten Gedanken an ihre Richtigkeit verwendet werden.

Es gibt viele berühmte Beispiele für falsche Theoreme und Beweise in der mathematischen Literatur.\index{Fehler in Beweisen}

\begin{itemize}
\item Es gibt Tausende von angeblichen Beweisen für den Großen Fermatschen Satz\index{Großer Fermatscher Satz} ("`keine ganzzahligen Lösungen existieren zu $x^n + y^n = z^n$ für $n > 2$"'), von Amateuren, Spinnern und angesehenen Mathematikern \cite[S.~5]{Stark}\index{Stark, Harold M}.  Fermat schrieb eine Notiz in sein Exemplar von Bachets {\em Diophantus}, dass er "`hierfür einen wahrhaft wunderbaren Beweis entdeckt [hat], doch ist dieser Rand hier zu schmal, um ihn zu fassen."'\cite[S.~507]{Kramer}.  Ein neuerer, viel beachteter Beweis von Yoichi Miyaoka\index{Miyaoka, Yoichi} wurde als falsch erwiesen ({\em Science News}, April 9, 1988, S.~230).  Das Theorem wurde schließlich von Andrew Wiles\index{Wiles, Andrew} ({\em Science News}, 3. Juli 1993, S.~5) bewiesen, aber der Beweis wies anfangs einige Lücken auf und es dauerte über ein Jahr nach seiner Bekanntgabe, bis er gründlich von Experten überprüft wurde.  Am 25. Oktober 1994 gab Wiles bekannt, dass die letzte Lücke in seinem Beweis geschlossen worden sei.
\item 1882 entdeckte M. Pasch, dass in Euklids Formulierung der Geometrie\index{euklidische Geometrie} ein Axiom ausgelassen wurde; ohne dieses Axiom sind die Beweise der wichtigen Theoreme von Euklid nicht gültig.  Das Axiom von Pasch\index{Axiom von Pasch} besagt, dass eine Linie, die eine Seite eines Dreiecks schneidet, auch eine andere Seite schneiden muss, vorausgesetzt, sie berührt keine der Scheitelpunkte.  Das Fehlen des Axioms von Pasch blieb, trotz der (vermutlich) Tausenden von Studenten, Lehrern und Mathematikern, die Euklid studierten, 2000 Jahre lang unbemerkt.
\item Der erste veröffentlichte Beweis des berühmten Schr\"{o}der--Bernstein-Theorems\index{Schr\"{o}der--Bernstein-Theorem} in der Mengenlehre war unkorrekt\cite[p.~148]{Enderton}\index{Enderton, Herbert B.}.  Dieses Theorem besagt, dass wenn es eine 1-zu-1-Funktion
\footnote{Eine {\em Menge}\index{Menge} ist eine beliebige Sammlung von Objekten. Eine {\em Funktion}\index{Funktion} oder {\em Zuordnung}\index{Zuordnung} ist eine Regel, die jedem Element einer Menge (dem sogenannten {\em Definitionsbereich}\index{Definitionsbereich}) ein Element aus einer anderen Menge zuordnet.} 
von der Menge $A$ in die Menge $B$ und umgekehrt gibt, dann haben die Mengen $A$ und $B$ eine 1-zu-1-Entsprechung.  Obwohl es einfach und offensichtlich klingt, ist der Standardbeweis recht lang und komplex.
\item In den frühen 1900er Jahren veröffentlichte Hilbert\index{Hilbert, David} einen angeblichen Beweis für die Kontinuumshypothese\index{Kontinuumshypothese}, die schließlich 1963 von Cohen\index{Cohen, Paul} als unbeweisbar festgestellt wurde\cite[S.~166]{Enderton}.  Die Kontinuumshypothese besagt, dass keine Unendlichkeit\index{Unendlichkeit} ("`transfinite Kardinalzahl"')\index{Kardinalzahl, transfinit} existiert, deren Größe (oder "`Kardinalität"')\index{Kardinalität} zwischen der Größe der Menge der ganzen Zahlen und der Größe der Menge der reellen Zahlen liegt.  Diese Hypothese geht auf den deutschen Mathematiker Georg
Cantor\index{Cantor, Georg} in den späten 1800er Jahren zurück, und seine Unfähigkeit, sie zu beweisen soll zu seiner Geisteskrankheit beigetragen haben, die ihn in seinen späteren
Jahren plagte.
\item Ein falscher Beweis für den Vier-Farben-Satz\index{Vier-Farben-Satz} wurde von Kempe\index{Kempe, A. B.} 1879 veröffentlicht\cite[S.~582]{Courant}\index{Courant, Richard}; er bestand 11 Jahre lang, bevor sein Fehler entdeckt wurde.  Dieses Theorem besagt, dass jede Karte mit nur vier Farben gefärbt werden kann, so dass keine zwei benachbarten Länder die gleiche Farbe haben.
Im Jahr 1976 wurde das Theorem schließlich durch den berühmten computergestützten Beweis von Haken, Appel und Koch bewiesen \cite{Swart}\index{Appel, K.}\index{Haken, W.}\index{Koch, K.}.  Zumindest scheint es so zu sein.  Der Mathematiker H.~S.~M.~Coxeter\index{Coxeter, H. S. M.} hat Zweifel \cite[S.~58]{Davis}:  "`Ich habe das Gefühl, dass das eine unsaubere Art der Nutzung der Computer ist, und je mehr man mit Haken und Appel korrespondiert, desto wackeliger scheint sie zu sein."'
\item Viele falsche "`Beweise"' der Poincar\'{e}-Vermutung\index{Poincar\'{e}-Vermutung} sind im Laufe der Jahre vorgeschlagen worden.  Diese Vermutung besagt, dass jedes Objekt, das sich mathematisch wie eine dreidimensionale Kugel verhält, topologisch eine dreidimensionale Kugel ist, unabhängig davon, wie sehr es verzerrt ist.  Im März 1986 haben die Mathematiker Colin Rourke\index{Rourke, Colin} und Eduardo R\^{e}go\index{R\^{e}go, Eduardo} Aufsehen in der mathematischen Gemeinschaft erregt, als sie verkündeten, einen Beweis gefunden zu haben; im November desselben Jahres stellte sich heraus, dass der Beweis falsch war \cite[S.218]{PetersonI}.  Sie wurde schließlich 2003 von Grigory Perelman bewiesen \label{poincare}\index{Szpiro, George}\index{Perelman, Grigory}\cite{Szpiro}.
 \end{itemize}

Viele Gegenbeispiele zu "`Theoremen"' in der neueren mathematischen Literatur im Zusammenhang mit Clifford-Algebren\index{Clifford-Algebren} wurden von Pertti Lounesto (der 2002 verstorben ist) gefunden.\index{Lounesto, Pertti} Siehe dazu \url{http://mathforum.org/library/view/4933.html}.
% http://users.tkk.fi/~ppuska/mirror/Lounesto/counterexamples.htm

Einer der Verwendungszwecke von Metamath\index{Metamath} ist es, Beweise mit absoluter Präzision zu formulieren.  Die Entwicklung eines Beweises in der Metamath-Sprache kann eine Herausforderung sein, weil Metamath nicht einmal den kleinsten Fehler zulässt.\index{Fehler in Beweisen}  Ist der Beweis jedoch einmal erstellt, kann man sich sofort auf seine Korrektheit verlassen, ohne dass man zur Bestätigung auf den Prozess der Peer Reviews angewiesen ist.

\section{Der Einsatz von Computern in der Mathe\-matik}

\subsection{Computeralgebrasysteme}

In den meisten Fällen werden Sie feststellen, dass Metamath\index{Metamath} kein praktisches Werkzeug zum Rechnen mit Zahlen ist.  (Selbst der Beweis, dass $2 + 2 = 4$, kann ziemlich komplex sein, wenn Sie mit der Mengenlehre beginnen).  Mehrere kommerzielle Softwareprogramme für Mathematik sind ziemlich gut in Arithmetik, Algebra und Infinitesimalrechnung, und als praktische Werkzeuge sind sie von unschätzbarem Wert.\index{Computeralgebrasystem} Aber sie haben keine Möglichkeiten mit Beweisen umzugehen, und können keine Aussagen verstehen, die mit "`Es gibt solche und solche..."' beginnen.

Softwareprogramme wie Mathematica\cite{Wolfram}\index{Mathematica} beschäftigen sich nicht mit Beweisen, sondern arbeiten direkt mit bekannten Ergebnissen.
Diese Programme setzen vor allem auf heuristische Regeln wie die Ersetzung von Gleichen für Gleiches, um einfachere Ausdrücke oder Ausdrücke in einer anderen Form zu erhalten.  Ausgehend von einer reichhaltigen Sammlung eingebauter Regeln und Algorithmen können die Benutzer mit Hilfe einer leistungsfähigen Programmiersprache weitere solche ergänzen.
Allerdings können Ergebnisse wie z. B. die Existenz eines bestimmten abstrakten Objekts ohne dass das tatsächliche Objekt angegeben wird in ihren Sprachen nicht (direkt) ausgedrückt werden.
Die Möglichkeit, einen Beweis aus einer kleinen Menge von Axiomen zu erstellen, fehlt.
Stattdessen geht diese Software einfach davon aus, dass jedes Faktum oder jede Regel, die Sie der eingebauten Sammlung von Algorithmen hinzufügen, gültig ist.  Somit stellt solch eine Software eine große Sammlung von Axiomen dar, aus der die Software unter gegebenen Zielsetzungen versucht, neue Theoreme abzuleiten, z. B. die Gleichheit eines komplexen Ausdrucks mit einem einfacheren Äquivalent. Aber die Begriffe "`Theorem"'\index{Theorem} und "`Beweis"'\index{Beweis} werden zum Beispiel nicht einmal im Index des Benutzerhandbuchs für Mathematica erwähnt.\index{Mathematica und Beweise}
Aus philosophischer Sicht unbefriedigend ist auch, dass es keine Möglichkeit gibt, die Gültigkeit der Ergebnisse zu gewährleisten, als dem Autor jedes einzelnen Anwendungsmoduls zu vertrauen oder jedes Modul mühsam von Hand zu überprüfen, ähnlich wie ein Computerprogramm auf Fehler zu überprüfen.\index{Programmfehler}\index{Bug}
\footnote{Zwei Beispiele verdeutlichen, warum die Wissensdatenbasis von Computeralgebrasystemen manchmal mit einer gewissen Vorsicht zu betrachten ist.  Wenn Sie Mathematica (Version 3.0) auffordern: \texttt{Solve[x\^{ }n + y\^{ }n == z\^{ }n , n]}, dann wird es antworten: \texttt{\{\{n-\char`\>-2\}, \{n-\char`\>-1\}, \{n-\char`\>1\}, \{n-\char`\>2\}\}}. Mit anderen Worten, Mathematica scheint zu "`wissen"', dass der Große Fermatsche Satz\index{Großer Fermatscher Satz} wahr ist!  Zum Zeitpunkt als diese Version von Mathematica veröffentlicht wurde, war diese Tatsache jedoch noch nicht bekannt.
Wenn Sie Maple\index{Maple} auffordern: \texttt{solve(x\^{ }2 = 2\^{ }x)}, danach: \texttt{simplify(\{"{}\})}, gibt es die Lösungsmenge \texttt{\{2, 4\}} zurück, offenbar nicht wissend, dass $-0.7666647$\ldots auch eine Lösung ist.}
Obwohl sie in der angewandten Mathematik natürlich äußerst hilfreich sind, sind Computeralgebrasysteme für den theoretischen Mathematiker eher von geringem Interesse, außer als Hilfsmittel zur Erforschung bestimmter spezifischer Probleme.

Wegen möglicher Fehler würde ein bloßes Vertrauen in die Ausgabe eines Computeralgebrasystems zur Verwendung als Theorem in einem Beweisverifizierer dessen Ziel der Strenge verfehlen.
Andererseits ist zwar eine Tatsache, dass eine bestimmte relativ große Zahl eine Primzahl ist, für ein Computeralgebrasystem leicht herzuleiten, könnte aber einen langen, mühsamen Beweis haben, der einen Beweisverifizierer überfordern könnte. Ein Ansatz zur Verknüpfung von
Computeralgebrasystemen mit einem Beweisverifizierer unter Beibehaltung der Vorteile beider Systeme besteht darin, jedem dieser Theoreme eine Hypothese hinzuzufügen, die seine Quelle angibt.
Zum Beispiel könnte eine Konstante {\sc maple} anzeigen, dass das Theorem aus dem Maple
Paket stammt, und würde bedeuten: "`Angenommen Maple ist konsistent, dann\ldots"' Dieses und
viele andere Themen, welche die Formalisierung der Mathematik betreffen, werden in John Harrisons\index{Harrison, John} sehr interessanter Dissertation~\cite{Harrison-thesis} diskutiert.

\subsection{Automatische Theorembeweiser}\label{theoremprovers}\index{automatisches Theorembeweisen} 

Eine mathematische Theorie ist "`entscheidbar"'\index{entscheidbare Theorie}, wenn eine mechanische Methode oder ein Algorithmus existiert, der garantiert feststellen kann, ob eine bestimmte Formel ein Theorem ist.  Zu den wenigen Theorien, die entscheidbar sind, gehört die Elementargeometrie, wie ein klassisches Ergebnis des Logikers Alfred Tarski\index{Tarski, Alfred} im Jahr 1948
\cite{Tarski} zeigt.
\footnote{Tarskis Ergebnis gilt eigentlich für eine Teilmenge der Geometrie, die in elementaren Lehrbüchern behandelt wird.  Diese Untermenge umfasst das meiste von dem, was man als elementare Geometrie bezeichnen würde, aber sie ist nicht stark genug, um unter anderem die Begriffe Umfang und Fläche eines Kreises auszudrücken.  Die Theorie so zu erweitern, dass sie Begriffe wie diese enthält, macht die Theorie unentscheidbar, wie auch von Tarski gezeigt wurde.  Tarskis Algorithmus ist viel zu ineffizient, um ihn praktisch auf einem Computer zu implementieren.  Ein praktischer Algorithmus zum Beweisen einer kleineren Untermenge von Geometrie-Theoremen (die nicht die Konzepte der "`Ordnung"' oder "`Kontinuität"' beinhalten) wurde von dem chinesischen Mathematiker Wu Wen-ts\"{u}n im Jahr 1977 \cite{Chou}\index{Chou,
Shang-Ching} entdeckt.}\index{Wen-ts{\"{u}}n, Wu} 
Aber die meisten Theorien, einschließlich der elementaren Arithmetik, sind unentscheidbar.  Diese Tatsache trägt dazu bei, die Mathematik am Leben zu erhalten, da viele Mathematiker glauben, dass sie niemals
durch Computer ersetzt werden (wenn sie Roger Penroses Argument glauben, dass ein Computer niemals das Gehirn ersetzen kann \cite{Penrose}\index{Penrose, Roger}).
In der Tat wird die Elementargeometrie oft als "`toter"' Bereich betrachtet, und zwar aus dem einfachen Grund, dass sie entscheidbar ist.

Andererseits bedeutet die Unentscheidbarkeit einer Theorie nicht, dass man nicht einen Computer für die Suche nach Beweisen verwenden kann. Man muss jedoch bereit sein, die Suche nach einer angemessenen Zeitspanne abzubrechen, wenn bis dahin kein Beweis gefunden wurde.  Das Gebiet des automatischen Theorembeweisens\index{automatisches Theorembeweisen} ist spezialisiert auf die Durchführung solcher Suchen mittels Computern.  Zu den bisher erfolgreichsten Ergebnissen gehören diejenigen, die auf einem Algorithmus, der als Robinsons Resolutionsprinzip bekannt ist\cite{Robinson}\index{Robinsons Resolutionsprinzip} basieren.

Automatische Theorembeweiser können hervorragende Werkzeuge sein, wenn man bereit ist zu lernen, wie man sie benutzt.  Aber sie sind in der reinen Mathematik nicht weit verbreitet, obwohl sie wahrscheinlich in vielen Bereichen nützlich sein könnten.  Es gibt dafür mehrere Gründe.  Der wohl wichtigste ist, dass das Hauptziel in der reinen Mathematik ist, zu Ergebnissen zu gelangen, die als tiefgründig oder wichtig wahrgenommen werden; sie zu beweisen ist zwar wichtig, aber zweitrangig.  Normalerweise kann ein automatischer Theorembeweiser bei diesem Hauptziel nicht helfen, und wenn das Hauptziel erreicht ist, hat der Mathematiker vielleicht schon den Beweis als Nebenprodukt.  Außerdem gibt es ein Problem mit der Notation.  Mathematiker sind daran gewöhnt eine sehr kompakte Syntax zu verwenden, bei der ein oder zwei Symbole (die stark vom Kontext abhängen) sehr komplexe Konzepte darstellen können; dies ist Teil der
Hierarchie\index{Hierarchie}, die sie aufgebaut haben, um schwierige Probleme zu bewältigen.  Ein Theorembeweiser hingegen könnte voraussetzen, dass ein Theorem in "`Logik erster Ordnung"'\index{Logik erster Ordnung} ausgedrückt wird, der Logik, auf die der größte Teil der Mathematik beruht, die aber normalerweise nicht direkt verwendet wird, da die Ausdrücke sehr lang werden können.  Einige automatische Theorembeweiser sind experimentelle Programme, die in ihrer Anwendung auf sehr spezielle Bereiche beschränkt sind, und das Ziel vieler Programme ist einfach die Erforschung der Natur des automatischen Theorembeweisens selbst.  Schließlich muss noch viel Forschung betrieben werden, um sehr tiefgreifende Theoreme zu beweisen.  Ein wichtiges Ergebnis war ein Computerbeweis von Larry Wos\index{Wos, Larry} und Kollegen, dass jede Robbins-Algebra\index{Robbins-Algebra} eine Boolesche Algebra\index{Boolesche Algebra} ist ({\em New York Times}, Dec. 10, 1996).
\footnote{Im Jahr 1933 präsentierte, E.~V.\
Huntington\index{Huntington, E. V.}  das folgende Axiomensystem für die Boolesche Algebra mit einer unären Operation $n$ und einer binären Operation $+$:
\begin{center}
    $x + y = y + x$ \\
    $(x + y) + z = x + (y + z)$ \\
    $n(n(x) + y) + n(n(x) + n(y)) = x$
\end{center}
Herbert Robbins\index{Robbins, Herbert}, ein Schüler von Huntington, vermutete, dass die letzte Gleichung durch eine einfachere Gleichung ersetzt werden kann:
\begin{center}
    $n(n(x + y) + n(x + n(y))) = x$
\end{center}
Robbins und Huntington konnten keinen Beweis finden.  Das Problem wurde später erfolglos von Tarski\index{Tarski, Alfred} und seinen Schülern untersucht, und es blieb ein ungelöstes Problem, bis ein
Computer im Jahr 1996 den Beweis fand.  Weitere Informationen zum Robbins-Algebra-Problem finden Sie unter \cite{Wos}.}

Wie verhält sich Metamath\index{Metamath} zu automatischen Theorembeweisern?  Ein Theorembeweiser befasst sich in erster Linie mit jeweils einem Theorem (vielleicht einer kleinen Datenbasis mit bekannten Theoremen), während Metamath eher wie ein Theoremarchivierungssystem, das sowohl das Theorem als auch seinen Beweis in einer Datenbasis für den Zugriff und die Überprüfung speichert.  Metamath ist eine Antwort auf die Frage "`Was macht man mit der Ausgabe eines Theorembeweisers?"' und könnte als der nächste Schritt in diesem Prozess betrachtet werden.  Automatische Theorembeweiser könnten nützliche Werkzeuge sein, um bei der Entwicklung seiner Datenbasis zu helfen.
Beachten Sie, dass sehr lange, automatisch generierte Beweise die Datenbasis fett und hässlich machen und dazu führen, dass Metamaths Beweisverifikation sehr viel Zeit in Anspruch nimmt.  Wenn Sie nicht ein besonders gutes Programm haben, das sehr prägnante Beweise erzeugt, ist es vielleicht am besten, die Verwendung automatisch generierter Beweise als eine unsaubere Schnelllösung zu betrachten, die zu einem späteren Zeitpunkt manuell neu zu schreiben wäre.

Das Programm {\sc otter}\index{otter@{\sc otter}}\footnote{\url{http://www.cs.unm.edu/\~mccune/otter/}.}, später ersetzt durch prover9\index{prover9}\footnote{\url{https://www.cs.unm.edu/~mccune/mace4/}.}, hat einen historischen Einfluss gehabt.
Der E-Prover\index{E-Prover}\footnote{\url{https://github.com/eprover/eprover}.} ist ein immer noch gepflegter automatischer Theorembeweiser für vollständige Logik erster Ordnung mit Gleichheit.
DAneben gibt es noch viele andere automatische Theorembeweiser.

Wenn Sie automatische Theorembeweiser mit Metamath kombinieren wollen, sollten Sie sich mit dem Buch {\em Automated Reasoning:  Introduction and Applications}\cite{Wos}\index{Wos, Larry} beschäftigen.  In diesem Buch geht es darum wie man {\sc otter} in einer solchen Weise einsetzt, dass es nicht nur in der Lage ist relativ effiziente Beweise zu erzeugen, sondern sogar angewiesen werden kann nach kürzeren Beweisen zu suchen.  Die effektive Verwendung von {\sc otter} (und ähnlicher Werkzeuge) erfordert jedoch ein gewisses Maß an Erfahrung, Sachkenntnis und Geduld.  Das Axiomensystem, das in der \texttt{set.mm}\index{Mengenlehre-Datenbasis (\texttt{set.mm})} Mengenlehre-Datenbasis verwendet wird, kann mit Hilfe einer Methode für {\sc otter} ausgedrückt werden, die in \cite{Megill}\index{Megill, Norman}
\footnote{Um diese Axiome mit {\sc otter} zu verwenden, müssen sie in einer Weise umformuliert werden, die die Notwendigkeit von "`Dummy Variablen"' beseitigt.\index{Dummy-Variable!eliminieren} Siehe den Kommentar
auf S.~\pageref{nodd}.}
beschrieben wird. Bei Erfolg neigt diese Methode in der Regel dazu, kurze und clevere Beweise zu generieren, aber meine Experimente mit ihr zeigen, dass die Methode in angemessener Zeit Beweise nur für relativ einfache Theoreme findet.  Es macht trotzdem Spaß, damit zu experimentieren.

Referenz \cite{Bledsoe}\index{Bledsoe, W. W.} gibt einen Überblick über eine Reihe von Ansätzen,
die auf dem Gebiet des automatischen Theorembeweisens\index{automatisches Theorembeweisen} erforscht wurden.

\subsection{Interaktive Theorembeweiser}\label{interactivetheoremprovers}

Das vollautomatische Finden von Beweisen ist schwierig, daher gibt es einige interaktive Theorembeweiser, bei denen ein Mensch den Computer bei der Suche nach einem Beweis anleitet.
Beispiele hierfür sind HOL Light\index{HOL light}%
\footnote{\url{https://www.cl.cam.ac.uk/~jrh13/hol-light/}.},
Isabelle\index{Isabelle}%
\footnote{\url{http://www.cl.cam.ac.uk/Research/HVG/Isabelle}.},
{\sc hol}\index{hol@{\sc hol}}%
\footnote{\url{https://hol-theorem-prover.org/}.},
und
Coq\index{Coq}\footnote{\url{https://coq.inria.fr/}.}.

Ein wesentlicher Unterschied zwischen den meisten dieser Werkzeuge und Metamath ist, dass die "`Beweise"' eigentlich Programme sind, die das Programm anleiten, einen Beweis zu finden, und nicht der Beweis selbst.
Zum Beispiel könnte ein Isabelle/HOL-Beweis einen Schritt \texttt{apply (blast dest: rearrange reduction)} anwenden. Die \texttt{blast} Anweisung verwendet einen automatischen Tableaux-Beweiser und wird beendet, wenn eine Folge von gültigen Beweisschritten gefunden wurde... aber diese Folge wird nicht als Teil des Beweises betrachtet.

Ein guter Überblick zu übergeordneten Sprachen für Beweisverifikationen (wie {\sc lcf}\index{lcf@{\sc lcf}} und {\sc hol}\index{hol@{\sc hol}}) ist in \cite{Harrison} angegeben.  Alle diese Sprachen unterscheiden sich grundlegend von Metamath dadurch, dass ein Großteil des mathematischen Grundwissens in das zugrundeliegende Beweisprogramm und nicht direkt in die zu prüfende Datenbasis eingebettet ist.
Diese Sprachen können eine steile Lernkurve für diejenigen erfordern, die keinen mathematischen Hintergrund haben.  Zum Beispiel muss man in der Regel ein gewisses Verständnis für mathematische Logik haben, um ihren Beweisen folgen zu können.

\subsection{Beweisverifizierer}\label{proofverifiers}

Ein Beweisverifizierer ist ein Programm, das keine Beweise erzeugt, sondern Beweise die im vorgegeben werden verifiziert.  Viele Beweisverifizierer haben eingeschränkte eingebaute automatisierte Beweisfähigkeiten, wie z. B. das Herausfinden einfacher logischer Schlüsse (wobei sie immer noch von einer Person angeleitet werden, die den Gesamtbeweis liefert).  Metamath hat keine eingebauten automatischen Beweisfähigkeiten, außer der begrenzten Fähigkeit seines Beweis-Assistenten.

Sprachen zur Beweisverifikation werden nicht so häufig verwendet, wie sie es könnten.
Reine Mathematiker sind mehr damit beschäftigt, neue Ergebnisse zu erzielen, und solche Detailgenauigkeit und Strenge würden diesem Ziel im Wege stehen.  
Der Einsatz von Computern in der reinen Mathematik konzentriert sich in erster Linie auf automatische Theorembeweiser (nicht auf Verifizierer), wiederum mit dem Ziel, die Schaffung neuer Mathematik zu unterstützen.
Automatische Theorembeweiser befassen sich in der Regel damit, jeweils ein Theorem anzugehen, anstatt dem Benutzer eine große, organisierte Datenbasis zur Verfügung zu stellen.  Metamath ist eine Möglichkeit, diese Lücke zu schließen.

An sich ist Metamath hauptsächlich ein Beweisverifizierer.
Das bedeutet nicht, dass andere Ansätze nicht verwendet werden können; der Unterschied ist, dass in Metamath die Ergebnisse der verschiedenen Beweise Schritt für Schritt aufgezeichnet werden müssen, damit sie verifiziert werden können.

Eine weitere Sprache zur Überprüfung von Beweisen ist Mizar,\index{Mizar}, die ihre Beweise in der informellen Sprache darstellen kann, an die Mathematiker gewöhnt sind. Informationen über die Mizar-Sprache finden Sie unter \url{http://mizar.org}.

Für den tätigen Mathematiker ist Mizar ein ausgezeichnetes Werkzeug um Beweise streng zu dokumentieren. Mizar stellt seine Beweise in dem informellen Englisch dar, das von Mathematikern verwendet wird (und, obwohl sie für diese ausreichen, für Laien genauso unverständlich sind!). Der Preis für Mizar ist eine relativ steile Lernkurve von einigen Wochen.  Mehrere Mathematiker arbeiten aktiv an der Formalisierung verschiedener Bereiche der Mathematik mit Mizar und veröffentlichen die Beweise in einer speziellen Zeitschrift. Leider wird die Aufgabe, Mathematik zu formalisieren, immer noch in gewissem Maße geringgeschätzt, da sie keine "`neue"' Mathematik hervorbringt.

Das System, das Metamath am nächsten kommt, ist die {\em Ghilbert}\index{Ghilbert}-Beweis\-spra\-che (\url{http://ghilbert.org}), entwickelt von Raph Levien\index{Levien, Raph}.
Ghilbert ist ein formaler Beweisprüfer, der stark von Metamath inspiriert ist.
Ghilbert-Anweisungen sind s-Ausdrücke (à la Lisp), die für Computer leicht zu analysieren sind, aber viele Menschen finden sie schwer zu lesen.
Es gibt eine Reihe von Unterschieden in ihren spezifischen Konstrukten, aber es gibt zumindest ein Werkzeug, um einige Metamath-Daten in Ghilbert zu übersetzen.
Nach dem Stand von 2019 ist die Ghilbert-Gemeinschaft kleiner und weniger aktiv als die Metamath-Gemeinschaft.
Dennoch gibt es Überschneidungen zwischen der Metamath- und der Ghilbert-Gemeinschaft, und im Laufe der Jahre haben viele Male fruchtbare Gespräche zwischen ihnen  stattgefunden.

\subsection{Erstellung einer Datenbasis für formalisierte\texorpdfstring{\\}{} Mathematik}\label{mathdatabase}

Neben Metamath gibt es mehrere andere laufende Projekte mit dem Ziel, die Mathematik in durch Computer verifizierbare Datenbasen zu formalisieren.  Um sie zu verstehen hilft ein Rückblick auf deren Historie.

Das {\sc qed}\index{qed project@{\sc qed} Projekt}%
\footnote{\url{http://www-unix.mcs.anl.gov/qed}.}
Projekt wurde 1993 ins Leben gerufen und seine Ziele wurden in dem {\sc qed} Manifest umrissen.
Das {\sc qed} Manifest war ein Vorschlag für eine computergestützte Datenbasis für das gesamte mathematische Wissen, streng formalisiert und mit allen Beweisen, die automatisch überprüft wurden.
Zu diesem Projekt fand eine Konferenz im Jahr 1994 und eine weitere im Jahr 1995 statt. Außerdem gab es einen Workshop "`twenty years of the {\sc qed} manifesto"' im Jahr 2014.
Seine Ideale werden regelmäßig wieder aufgegriffen.

In einem Papier aus dem Jahr 2007 nennt Freek Wiedijk zwei Gründe für das Scheitern des {\sc qed} Projekts in der ursprünglich geplanten Form:%
\cite{Wiedijk-revisited}\index{Wiedijk, Freek}

\begin{itemize}
\item Nur sehr wenige Menschen arbeiten an der Formalisierung der Mathematik. Es gibt keine zwingende Anwendung für vollständig mechanisierte Mathematik.
\item Formalisierte Mathematik entspricht noch nicht der traditionellen Mathematik. Das liegt zum einen an der Komplexität der mathematischen Notation und zum anderen an den Grenzen der vorhandenen Theorembeweiser und Beweis-Assistenten.
\end{itemize}

Doch damit war der Traum von der Formalisierung der Mathematik in durch Computer verifizierbaren Datenbasen noch nicht beendet.
Die Probleme, die zum {\sc qed} Manifest führten, sind immer noch aktuell, auch wenn die Herausforderungen schwieriger waren als ursprünglich angenommen.
Stattdessen sind verschiedene unabhängige Projekte entstanden, die an der Formalisierung der Mathematik in durch Computer verifizierbaren Datenbasen gearbeitet haben, die gleichzeitig miteinander konkurrieren und kooperieren.

Eine konkrete Möglichkeit, diese Projekte zu vergleichen, ist
Freek Wiedijks Liste "`Formalisierung von 100 Theoremen"',%
\footnote{\url{http://www.cs.ru.nl/\%7Efreek/100/}.}
die zeigt, welche Fortschritte verschiedene Systeme beim Finden von Beweisen für eine Liste von 100 mathematischen Theoremen gemacht haben.%
\footnote{ Dies ist nicht die einzige Liste von "`interessanten"' Theoremen. Eine weitere interessante Liste wurde von Oliver Knill veröffentlicht \cite{Knill}\index{Knill, Oliver}.}
Die Top-Systeme im Februar 2019 (in der Reihenfolge der Anzahl der abgeschlossenen Beweise) sind HOL Light, Isabelle, Metamath, Coq und Mizar\footnote{Stand 16.10.2023: Isabelle(89), HOL Light(87), Coq(79), Lean(76), Metamath(74) und Mizar(69)}.

Die Metamath 100%
\footnote{\url{http://us.metamath.org/mm\_100.html}} Seite (betreut von David A. Wheeler\index{Wheeler, David A.})
zeigt den Fortschritt von Metamath (insbesondere der \texttt{set.mm} Datenbasis) im Vergleich zu dieser von Freek Wiedijk geführten Challenge-Liste.
Die Metamath-\texttt{set.mm}Datenbasis hat im Laufe der Jahre große Fortschritte gemacht, zum Teil deshalb, weil die Arbeit an den Beweisen dieser Theoreme
die Definition verschiedener Begriffe und die Beweise ihrer Eigenschaften als Voraussetzung erforderlich machten.
Hier sind nur einige der vielen Sätze, die formal mit Metamath bewiesen wurden\footnote{Anm. der Übersetzer: Eine vollständige Liste befindet sich in Anhang \ref{Metamath100}}:

% The entries of this cause the narrow display to break poorly,
% since the short amount of text means LaTeX doesn't get a lot to work with
% and the itemize format gives it even *less* margin than usual.
% No one will mind if we make just this list flushleft, since this list
% will be internally consistent.
\begin{flushleft}
\begin{itemize}
\item 1. Die Irrationalität der Quadratwurzel aus 2
  (\texttt{sqr2irr}, von Norman Megill, 20. August 2001)
\item 2. Der Fundamentalsatz der Algebra
  (\texttt{fta}, von Mario Carneiro, 15. September 2014)
\item 22. Die Nicht-Abzählbarkeit des Kontinuums
  (\texttt{ruc}, von Norman Megill, 13. August 2004)
\item 54. Das Königsberger Brückenproblem
  (\texttt{konigsberg}, von Mario Carneiro, 16. April 2015)
\item 83. Der Freundschaftssatz
  (\texttt{friendship}, von Alexander W. van der Vekens, 9. Oktober 2018)
\end{itemize}
\end{flushleft}

Wir danken all denen, die mindestens einen der Metamath 100-Beweise entwickelt haben, und wir danken insbesondere Mario Carneiro\index{Carneiro, Mario}, der im Jahr 2019 die meisten Metamath 100-Beweise beigesteuert hat.
Die Metamath 100-Seite zeigt die Liste aller Personen, die einen Beweis beigetragen haben, sowie Links zu Grafiken und Diagrammen, die den Fortschritt im Laufe der Zeit zeigen.
Wir ermutigen andere, an Beweisen für Theoreme zu arbeiten, die noch nicht in Metamath bewiesen wurden, da dies das Werk insgesamt verbessert.

Jedes der mathematischen Formalisierungssysteme (einschließlich Metamath) hat unterschiedliche Stärken und Schwächen, je nachdem, worauf Sie Wert legen.
Die wichtigsten Aspekte, die Metamath von den anderen Top-Systemen unterscheiden, sind:

\begin{itemize}
\item Metamath ist nicht an einen bestimmten Satz von Axiomen gebunden.
\item Metamath kann jeden Schritt von jedem Beweis anzeigen, ohne Ausnahmen.
  Die meisten anderen Beweiser behaupten nur, dass ein Beweis gefunden werden kann, und zeigen nicht jeden Schritt. Das macht auch die Verifikation schnell, da
  das System die Details eines Beweises nicht neu ermitteln muss.
\item Der Metamath-Prüfer wurde in vielen verschiedenen Programmiersprachen erneut implementiert, so dass die Verifikation durch mehrere Implementierungen durchgeführt werden kann.  Insbesondere die
  \texttt{set.mm}\index{Mengenlehre-Datenbasis (\texttt{set.mm})}%
  \index{Metamath Proof Explorer} Datenbasis wird von
  vier verschiedenen Verifizierern verifiziert, die in vier verschiedenen Sprachen von vier verschiedenen Autoren geschrieben wurden\footnote{Anm. der Übersetzer: Damit sind die vier Verifizierer gemeint, die bei jedem Git-Merge für Änderungen an \texttt{set.mm} durchgeführt werden: der originale C-Verifizierer von Norman Megill, der Rust-Verifizierer von Stefan O'Rear, der Java-Verifizierer von Mel L. O'Cat und Mario Carneiro und der Python-Verifizierer von Raph Levin. Es gibt daneben noch viele andere Verifizierer, siehe \url{http://us.metamath.org/other.html}.} .
  Dadurch wird das Risiko der Annahme eines ungültigen Beweises aufgrund eines Fehlers im Verifizierer reduziert.
\item Beweise bleiben bewiesen.  In einigen Systemen können Änderungen an der Syntax des Systems oder an der Funktionsweise einer Taktik dazu führen, dass Beweise in späteren Versionen nicht mehr funktionieren, wodurch ältere Arbeiten im Grunde verloren gehen.
  Die Metamath-Sprache ist extrem klein und stabil, so dass sobald ein Beweis einmal zu einer Datenbasis hinzugefügt wurde,
  kann die Datenbasis mit späteren Versionen des Metamath-Programms und mit anderen Verifizierern von Metamath-Datenbasen überprüft werden.
  Wenn ein Axiom oder eine Schlüsseldefinition geändert werden muss, ist es einfach, die Datenbasis als Ganzes zu manipulieren, um die Änderung zu verarbeiten ohne den zugrunde liegenden Verifizierer zu verändern.
  Da die erneute Verifizierung einer gesamten Datenbasis nur Sekunden dauert, gibt es nie einen Grund, die vollständige Überprüfung hinauszuzögern.
  Dieser Aspekt ist besonders überzeugend, wenn man eine langfristig nutzbare Datenbasis mit Beweisen haben möchte.
\item Die Lizenzierung ist großzügig.  Die wichtigsten Metamath-Datenbasen sind öffentlich verfügbar, und das Metamath-Hauptprogramm ist Open-Source-Software unter einer standardisierten, weit verbreiteten Lizenz.
\item Substitutionen sind leicht zu verstehen, auch für diejenigen, die keine
  professionellen Mathematiker sind.
\end{itemize}

Natürlich können andere Systeme Vorteile gegenüber Metamath haben, die geeigneter sind, je nachdem, worauf Sie Wert legen.
In jedem Fall hoffen wir, dass Ihnen diese Ausführungen helfen, Metamath in einem größeren Kontext zu verstehen.

\subsection{Zusammenfassung}\label{computers-summary}

Unsere Diskussion über Computer und Mathematik kann wie folgt zusammengefasst werden: Computeralgebrasysteme können als Theoremgeneratoren betrachtet werden, die sich auf einen eingeschränkten Bereich der Mathematik (Zahlen und ihre Eigenschaften) konzentrieren, automatische Theorembeweiser als Beweisgeneratoren für spezifische Theoreme in einem viel breiteren Bereich, der durch ein eingebautes formales System wie die Logik erster Ordnung abgedeckt wird, interaktive Theorembeweiser erfordern menschliche Anleitung, Beweisverifizierer verifizieren Beweise, aber historisch gesehen waren sie auf die Logik erster Ordnung beschränkt.
Im Gegensatz dazu ist Metamath ein Beweisverifizierer und Dokumentierer, dessen Anwendungsbereich im Wesentlichen unbegrenzt ist.

\section{Die Mathematik und Metamath}

\subsection{Standardmathematik}

Es gibt eine Reihe von Möglichkeiten, Metamath\index{Metamath} für die Standardmathematik zu benutzen.  Der philosophisch befriedigendste Weg ist, ganz am Anfang zu beginnen und die gewünschte Mathematik aus den Axiomen der Logik und Mengenlehre zu entwickeln.  Dies ist der Ansatz, der in der
\texttt{set.mm}\index{Mengenlehre-Datenbasis (\texttt{set.mm})}%
\index{Metamath Proof Explorer}
Datenbasis (auch bekannt als Metamath Proof Explorer) angewandt wird.
Diese Datenbasis baut unter anderem auch auf den Axiomen der reellen und komplexen Zahlen\index{Analysis}\index{reelle Zahl}\index{komplexe Zahl} auf (siehe Abschnitt~\ref{real}), und eine Standardentwicklung der Analysis könnte beispielsweise an diesem Punkt ansetzen, basierend auf den übrigen Theoremen und Sätzen dieser Datenbasis.
Neben diesem philosophischen Vorteil gibt es auch praktische Vorteile, alle Werkzeuge der Mengenlehre in der unterstützenden Infrastruktur zur Verfügung zu haben.

Andererseits möchte man vielleicht mit den Standardaxiomen einer mathematischen Theorie beginnen, ohne die mengentheoretischen Beweise dieser Axiome berücksichtigen zu müssen.  Sie werden die mathematische Logik benötigen, um Schlussfolgerungen zu ziehen, aber wenn Sie möchten können Sie einfach die Theoreme der Logik\index{Theorem} wo immer man sie braucht als "`Axiome"'\index{Axiom} einführen, unter der impliziten Annahme, dass sie im Prinzip bewiesen werden können, wenn sie für Sie offensichtlich sind.  Wenn Sie diesen Ansatz wählen, werden Sie wahrscheinlich die in
\texttt{set.mm}\index{Mengenlehre-Datenbasis (\texttt{set.mm})} verwendete Notation überprüfen, damit sie mit Ihrer eigenen Notation zusammenpasst.

\subsection{Andere formale Systeme}
\index{formales System}

Im Gegensatz zu anderen Programmen ist Metamath\index{Metamath} weder auf einen bestimmten Bereich der Mathematik beschränkt, noch einer bestimmten mathematischen Philosophie verpflichtet wie z.B. der klassischen oder intuitionistischer Logik, noch beschränkt auf Ausdrücke der Logik erster Ordnung.  Obwohl die Datenbasis \texttt{set.mm} die Standardlogik und Mengenlehre beschreibt, ist Metamath eigentlich eine Allzwecksprache zur Beschreibung einer Vielzahl formaler Systeme.\index{formales System}  Nicht-Standard-Systeme wie die modale Logik\index{modale Logik}, intuitionistische Logik\index{Intuitionismus}, Logik höherer Ordnung\index{Logik höherer Ordnung (HOL)}, Quantenlogik\index{Quantenlogik}, und Kategorientheorie\index{Kategorientheorie} können alle mit der Metamath-Sprache beschrieben werden.  Sie definieren die von Ihnen bevorzugten Symbole und teilen Metamath die Axiome und Regeln mit, von denen Sie ausgehen wollen, und Metamath verifiziert alle Schlüsse, die Sie aus diesen Axiomen und Regeln ziehen.
Ein einfaches Beispiel für ein nicht standardisiertes formales System ist Hofstadters\index{Hofstadter, Douglas R.} MIU-System,\index{MIU-System} dessen Metamath-Version im Anhang~\ref{MIU} dargestellt wird.

Diese Nutzungsmöglichkeiten von Metamatrh sind nicht nur hypothetisch.
Die größte Metamath-Datenbasis ist
\texttt{set.mm}\index{Mengenlehre-Datenbasis (\texttt{set.mm})}%
\index{Metamath Proof Explorer}, auch bekannt als der Metamath Proof Explorer, der die gebräuchlichsten Axiome für mathematische Grundlagen verwendet
(insbesondere die klassische Logik kombiniert mit der Zermelo--Fraenkel
Mengenlehre\index{Zermelo--Fraenkel-Mengenlehre} zusammen mit dem Auswahlaxiom).
Es sind aber auch andere Metamath-Datenbasen verfügbar:

\begin{itemize}
\item Die Datenbasis
  \texttt{iset.mm}\index{intuitionistische Logik-Datenbasis (\texttt{iset.mm})},
  auch bekannt als Intuitionistic Logic Explorer\index{Intuitionistic Logic Explorer},
  verwendet intuitionistische Logik (eine konstruktivistische Sichtweise)
  anstelle der klassischen Logik.
\item Die Datenbasis
  \texttt{nf.mm}\index{New Foundations Datenbasis (\texttt{nf.mm})},
  auch bekannt als New Foundations Explorer\index{New Foundations Explorer},
  konstruiert die Mathematik von Grund auf neu,
  ausgehend von Quines New Foundations (NF) Axiomen der Mengenlehre.
\item Die Datenbasis
  \texttt{hol.mm}\index{Datenbasis für Logik höherer Ordnung (\texttt{hol.mm})},
  auch bekannt als Higher-Order Logic (HOL) Explorer\index{Higher-Order Logic (HOL) Explorer},
  beginnt mit HOL (auch einfache Typentheorie genannt), leitet daraus
  Äquivalente zu den ZFC-Axiomen ab und verbindet so die beiden Ansätze.
\end{itemize}

Seit der Zeit von David Hilbert haben sich die Mathematiker darüber Gedanken gemacht, ob die Metasprache, die zur Beschreibung der Mathematik verwendet wird, mächtiger sein muss als die Mathematik, die beschrieben wird.
Die finitistische\index{finitistischer Beweis}, konstruktive Natur von Metamath\index{Metamath}  bietet eine gute philosophische Grundlage für das Studium selbst der schwächsten Logiken.\index{schwache Logik}

Die Beschreibung von vielen, nicht-standardisierten formaler Systeme\index{formales System} wird die Modelltheorie\index{Modelltheorie} oder die Beweistheorie\index{Beweistheorie} verwendet; diese Theorien basieren ihrerseits auf der Standard-Mengentheorie.  Mit anderen Worten: Ein nicht-standardisiertes formales System ist definiert als eine Menge mit bestimmten Eigenschaften, und die Standard-Mengentheorie wird zur Herleitung zusätzliche Eigenschaften dieser Menge genutzt.  Die Datenbasis der Standard-Mengentheorie, die mit Metamath zur Verfügung gestellt wird, kann für diesen Zweck verwendet werden. Und wenn sie auf diese Weise genutzt wird, ist die Entwicklung eines speziellen Axiomensystems für das nicht-standardisierte formale System überflüssig.  Der modell- oder beweistheoretische Ansatz erlaubt es oft, viel tiefgreifendere Ergebnisse mit weniger Aufwand zu beweisen.

Metamath unterstützt beide Ansätze.  Sie können das nicht-standardisierte formale System direkt definieren, oder das nicht-standardisierte formale System als eine Menge mit bestimmten Eigenschaften definieren, je nachdem, was Sie am geeignetsten finden.

%\section{Additional Remarks}

\subsection{Metamath und seine Philosophie}

Metamath steht im Zusammenhang mit einer bestimmten Philosophie oder Betrachtungsweise
der Mathematik. Diese Philosophie ist verwandt mit der formalistischen
Philosophie\index{Formalismus} von Hilbert\index{Hilbert, David} und seinen Anhängern
\cite[S.~1203--1208]{Kline}\index{Kline, Morris}
\cite[S.~6]{Behnke}\index{Behnke, H.}. In dieser Philosophie ist die Mathematik nichts weiter als eine Reihe von Regeln zur Handhabung von Symbolen, zusammen mit den Folgerungen aus der Anwendung dieser Regeln.  Während die beschriebene Mathematik komplex sein kann, sollten die Regeln, mit denen sie beschrieben wird (die "`Metamathematik"'\index{Metamathematik}) so einfach wie möglich sein.
Insbesondere sollten sich die Beweise auf konkrete Objekte beziehen (die Symbole, die wir auf Papier schreiben, und nicht die abstrakten Konzepte, die sie repräsentieren) und mit ihnen in einer konstruktiven Weise umgehen; solche Beweise werden "`finitistisch"' genannt\index{finitistischer Beweis} \cite[S.~2--3]{Shoenfield}\index{Shoenfield,
Joseph R.}.

Ob Sie Metamath interessant oder nützlich finden, hängt zum Teil von der Anziehungskraft ab, die seine Philosophie auf Sie ausübt, und diese Anziehungskraft hängt wahrscheinlich von Ihren besonderen Zielen in Bezug auf die Mathematik ab.  Wenn Sie zum Beispiel ein reiner Mathematiker sind, der an vorderster Front bei der Entdeckung neuer mathematischer Erkenntnisse steht, werden Sie wahrscheinlich Ihre
Kreativität durch den starren Formalismus von Metamath als einschränkt erachten.  Andererseits würden wir argumentieren, dass es vorteilhaft ist, solches Wissen in einem Standardformat zu dokumentieren, sobald es entdeckt ist, damit es daurch für andere zugänglich ist.  In sechzig Jahren mag Ihr Wissensgebiet vielleicht eingeschlafen sein, und dann würden Ihre "`Schriften noch schwieriger übersetzbar sein als die der Maya"', wie Davis und Hersh es ausdrücken\cite[S.~37]{Davis}\index{Davis, Phillip J.}.

\subsection{Die Hintergründe des Metamath-Ansatzes}

Den wahrscheinlich stärksten Einfluss auf Metamath\index{Metamath} hatte Whiteheads und Russells monumentales Werk {\em Principia Mathematica} \cite{PM}\index{Whitehead, Alfred North}\index{Russell, Bertrand}\index{principia mathematica@{\em Principia Mathematica}}, dessen Ziel es war, die gesamte Mathematik aus einer kleinen Anzahl von primitiven Konzepten abzuleiten, und zwar auf eine sehr expliziten Weise, die im Prinzip jeder verstehen und nachvollziehen konnte.  Während dieses Werk zu seiner Zeit sehr einflussreich war, hat es aus heutiger Sicht mehrere Nachteile.  Sowohl die Notation als auch die zugrunde liegenden Axiome gelten heute als veraltet und werden nicht mehr verwendet.  Von unserem Standpunkt aus ist ihre Entwicklung nicht wirklich so zugänglich, wie wir es gerne hätten; aus praktischen Gründen werden die Beweise mit fortschreitender mathematischer Tiefe immer skizzenhafter, und ihre Ausarbeitung im Detail erfordert ein Maß an mathematischem Geschick und Geduld, über die viele Menschen nicht verfügen.  Es gibt zahlreiche kleine Fehler, was angesichts der langwierigen, technischen Natur der Beweise und dem Fehlen eines Computers zur Überprüfung der Details verständlich ist.
Dennoch ist {\em Principia Mathematica} auch heute noch das Werk, das dem Geist von Metamath am nächsten kommt.  Es bleibt ein verblüffendes Werk, und man kann nicht anders als staunen, wenn man sieht, dass "`$1+1=2$"' schließlich auf Seite 83 von Band II erscheint (Theorem *110.643).

Der Ursprung der von Metamath verwendeten Beweisnotation geht auf die 1950er Jahre zurück, als der Logiker C.~A.~Meredith seine Beweise in einer kompakten Notation ausdrückte, die "`kondensierte Ablösung"'\index{kondensierte Ablösung} (engl. "`condensed detachment"`)
\cite{Hindley}\index{Hindley, J. Roger} \cite{Kalman}\index{Kalman, J. A.}
\cite{Meredith}\index{Meredith, C. A.} \cite{Peterson}\index{Peterson, Jeremy
George} genannt wird.  Diese Notation ermöglicht die eindeutige Wiedergabe von Beweisen durch bloße Bezugnahme auf das Axiom\index{Axiom}, die Regel\index{Regel} oder das Theorem\index{Theorem}, das bei jedem Schritt verwendet wird, ohne eine explizite Angabe der Substitutionen\index{Substitution!Variable}\index{Variablensubstitution}, die für die Variablen in diesem Axiom, dieser Regel oder diesem Theorem vorgenommen werden müssen.  Gewöhnlich ist die kondensierte Ablösung mehr oder weniger auf die Aussagenlogik\index{Aussagenlogik} beschränkt.  Das Konzept ist in \cite{Megill}\index{Megill, Norman} auf Logik erster Ordnung\index{Logik erster Ordnung} erweitert worden, wodurch das Schreiben eines kleinen Computerprogramms zur Überprüfung von Beweisen einfacher Theoreme der Logik erster Ordnung vereinfacht wird.\index{kondensierte Ablösung!und Logik erster Ordnung}

Ein Schlüsselkonzept hinter der Notation der kondensierte Ablösung ist die sogenannte "`Vereinheitlichung"' (engl. "`unification"'), ein Algorithmus zur Bestimmung der Substitutionen\index{Substitution!Variable}\index{Variablensubstitution}, die für Variablen vorgenommen werden müssen, damit zwei Ausdrücke miteinander übereinstimmen.
Die Vereinheitlichung wurde erstmals von dem Logiker J.~A.~Robinson genau definiert, der sie bei der Entwicklung einer leistungsfähigen Theorem-Beweis-Technik namens "`Auflösungsprinzip"' (engl. "`resolution principle"') verwendete\cite{Robinson}\index{Robinsons Resolutionsprinzip}. Metamath macht keinen Gebrauch von diesem Auflösungsprinzip, das für Systeme der Logik erster Ordnung\index{Logik erster Ordnung} gedacht ist.  Die Verwendung von Metamath ist nicht auf die Logik erster Ordnung beschränkt, und wie wir bereits erwähnt haben, findet es nicht automatisch Beweise.  Allerdings ist die Vereinheitlichung eine Schlüsselidee hinter Metamaths Beweisnotation, und Metamath macht von einer sehr einfachen Version davon Gebrauch (Abschnitt~\ref{unify}).

\subsection{Metamath und die Logik erster Ordnung}

Die Logik erster Ordnung\index{Logik erster Ordnung} ist die grundlegende Struktur für die Standardmathematik.  Darauf aufbauend gibt es die Mengenlehre mit den Axiomen, aus denen sich praktisch die gesamte Mathematik ableiten lässt --- eine bemerkenswerte Tatsache.\index{Kategorientheorie}\index{Kardinalzahl, unzugänglich}\label{categoryth}\footnote{Eine Ausnahme scheint die Kategorientheorie zu sein.  Es gibt verschiedene Denkschulen darüber, ob die Kategorientheorie aus der Mengenlehre ableitbar ist.  Zumindest scheint es so, dass ein zusätzliches Axiom erforderlich ist, das die Existenz eines "`unzugänglichen Kardinals"' (eine Art von Unendlichkeit, die so groß ist, dass die Standard-Mengenlehren-Theorie ihre Existenz weder beweisen noch leugnen kann).

%
%%%% (I took this out that was in previous editions:)
% But it is also argued that not just one but a "`proper class"' of them
% is needed, and the existence of proper classes is impossible in standard
% set theory.  (A proper class is a collection of sets so huge that no set
% can contain it as an element.  Proper classes can lead to
% inconsistencies such as "`Russell's paradox."'  The axioms of standard
% set theory are devised so as to deny the existence of proper classes.)
%
Für weitere Informationen siehe\cite[pp.~328--331]{Herrlich}\index{Herrlich, Horst} und
\cite{Blass}\index{Blass, Andrea}.}

Einer der Aspekte, die Metamath\index{Metamath} für Theorien erster Ordnung praktikabler machen, ist ein Satz von Axiomen für die Logik erster Ordnung, der speziell für den Metamath-Ansatz entwickelt wurde.  Diese sind enthalten in der Datenbasis \texttt{set.mm}\index{Mengenlehre-Datenbasis (\texttt{set.mm})}.
Siehe Kapitel~\ref{fol} für eine detaillierte Beschreibung; die Axiome sind in Abschnitt~\ref{metaaxioms} aufgeführt.  Während es logisch äquivalent zu Standard-Axiomensystemen ist, bricht unser Axiomensystem die Standardaxiome in kleinere Teile auf, so dass man aus ihnen direkt ableiten kann, was in anderen Systemen nur als übergeordnete "`Metatheoreme"'\index{Metatheorem} abgeleitet werden kann.
Mit anderen Worten, es ist mächtiger als die Standardaxiome vom metalogischen Standpunkt aus gesehen.  Eine strenge Rechtfertigung für dieses System und seine "`metalogische Vollständigkeit"'\index{metalogische Vollständigkeit} findet sich in
\cite{Megill}\index{Megill, Norman}.  Das System ist eng verwandt mit einem System, das von Monk\index{Monk, J. Donald} und Tarski\index{Tarski, Alfred} im Jahr 1965 \cite{Monks} entwickelt wurde.

Zum Beispiel ist die Formel $\exists x \, x = y $ (zu einem gegebenen $y$ existiert ein
$x$, das diesem gleicht) ist ein Satz der Logik,\footnote{Speziell ist es ein Satz
derjenigen Systeme der Logik, die nicht-leere Wertebereiche voraussetzen.  Es ist kein Theorem von allgemeineren Systemen, die den leeren Wertebereich einschließen, in dem nichts existiert, Punkt!  Solche Systeme nennt man "`freie Logiken."'\index{freie Logik} Für eine Diskussion dieser Systeme, siehe \cite{Leblanc}\index{Leblanc, Hugues}.  Da die Logik eine Grundlage für die Mengenlehre ist, die einen nicht leeren Wertebereich hat, ist es bequemer (und traditioneller), ein weniger allgemeines System zu verwenden.  Eine interessante Kuriosität bei der Verwendung einer freie Logik als Grundlage für die Zermelo--Fraenkel-Mengenlehre\index{Zermelo--Fraenkel-Mengenlehre} (wobei das redundante Axiom der Existenz einer leeren Menge weggelassen wird) ist, dass nicht einmal die Existenz einer einzigen Menge bewiesen werden kann, ohne das Axiom der Unendlichkeit anzunehmen!\index{Axiom der Unendlichkeit}}, egal ob $x$ und $y$ verschiedene Variablen\index{unterschiedliche Variablen} sind oder nicht.
In vielen Systemen der Logik müsste man zwei Theoreme beweisen, um zu diesem Ergebnis zu kommen.  Zuerst würden wir beweisen, dass "`$\exists x \, x = x $"', dann würden wir separat beweisen: "`$\exists x \, x = y $, wobei $x$ und $y$ verschiedene Variablen sind"'.  Wir würden dann diese beiden Spezialfälle "`außerhalb des Systems"' (d.h. in unseren Köpfen) kombinieren, um behaupten zu können: "`$\exists x \, x = y $, unabhängig davon, ob $x$ und $y$ verschieden sind"'.  Mit anderen Worten, die Kombination der beiden Spezialfälle ist ein Metatheorem.  In dem System der Logik, das in der Mengenlehre\index{Mengenlehre-Datenbasis (\texttt{set.mm})} von Metamath verwendet wird, sind die Axiome der Logik in kleine Teile zerlegt, die es erlauben, sie so zusammenzusetzen, dass Theoreme wie diese direkt bewiesen werden können.

Wenn man die Axiome auf diese Weise aufschlüsselt, sehen sie auf den ersten Blick seltsam und nicht sehr intuitiv aus, aber seien Sie versichert, dass sie korrekt und vollständig sind.  Ihre Korrektheit ist gewährleistet, weil es sich um Theoremschemata der Standardlogik erster Ordnung handelt (was Sie leicht überprüfen können, wenn Sie Logiker sind).  Ihre Vollständigkeit folgt aus der Tatsache, dass wir die Standardaxiome der Logik erster Ordnung explizit als Theoreme ableiten.  Die Ableitung der Standardaxiome ist etwas schwierig, aber wenn wir es geschafft haben, haben wir ein System zur Verfügung, das weniger unbequem für die Arbeit mit formalen Beweisen\index{formaler Beweis} ist.
Ausgedrückt in technischen Begriffen der Logiker eliminieren wir die umständlichen Konzepte der "`freien Variable"'\index{freie Variable}, der "`gebundenen Variable"'\index{gebundene Variable}, und der "`echten Substitution"'\index{echte Substitution}\index{Substitution!echte} als primitive Begriffe.  Diese Begriffe sind in unserem System vorhanden, werden aber durch konzeptionelle Begriffe definiert, die durch die Axiome ausgedrückt werden, und können im Prinzip eliminiert werden.  In Standardsystemen werden diese Begriffe tatsächlich wie zusätzliche, implizite Axiome\index{implizites Axiom} verwendet, die etwas komplex sind und nicht eliminiert werden können.

Die traditionelle Herangehensweise an die Logik, bei der freie Variable und die echte Substitution definiert ist, kann auch direkt in der Metamath-Sprache nachgebildet werden.
Allerdings ist die Notation eher umständlich, und es gibt Nachteile: zum Beispiel ist die Erweiterung der Definition einer wff mit einer Definition umständlich, weil die Definition der Konzepte der freien Variablen und der echten Substitution ebenfalls erweitert werden müssen.  Unsere Wahl der Axiome für \texttt{set.mm} ist bis zu einem gewissen Grad eine Frage des Stil bei dem Versuch, eine übergreifende Einfachheit zu gewährleisten mit dem Bewusstsein, dass auch der traditionelle Ansatz, falls gewünscht, möglich ist.

\chapter{Verwendung des Metamath-Programms}
\label{using}

\section{Installation}

Die Art und Weise, wie Sie Metamath auf Ihrem Computersystem installieren\index{Metamath!Installation}, ist von Computer zu Computer verschieden.  Aktuelle Anweisungen werden mit dem Download des Metamath-Programms bereitgestellt unter \url{http://metamath.org}.  Im Allgemeinen ist die Installation einfach.
Es gibt eine Datei, die das Metamath-Programm selbst enthält.  
Diese Datei heißt normalerweise \texttt{metamath} oder \texttt{metamath.}{\em xxx}, wobei {\em xxx} der Konvention (wie \texttt{exe}) für ein ausführbares Programm auf Ihrem Betriebssystem entspricht.  Es gibt mehrere zusätzliche Dateien mit Beispielen für die Metamath-Sprache, die alle mit \texttt{.mm} enden.  Die Datei \texttt{set.mm}\index{Mengenlehre-Datenbasis (\texttt{set.mm})} enthält Logik und Mengenlehre und kann als Ausgangspunkt für andere Bereiche der Mathematik verwendet werden.

Sie benötigen außerdem einen Texteditor\index{Texteditor}, mit dem Sie einfachen {\sc ascii}\footnote{American Standard Code for Information Interchange.} Text bearbeiten können, um Ihre Eingabedateien zu erzeugen.\index{ascii@{\sc ascii}}  Auf den meisten Computern stehen solche Texteditoren standardmäßig zur Verfügung.  Beachten Sie, dass einfacher Text nicht unbedingt der Standard für einige Textverarbeitungsprogramme ist, insbesondere wenn sie mit verschiedenen Schriftarten umgehen können; zum Beispiel müssen Sie bei Microsoft Word\index{Word (Microsoft)} die Datei im Format "`Nur Text"' abspeichern, um eine einfache Textdatei\index{einfacher Text} zu erhalten.\footnote{Es wird empfohlen, dass alle Zeilen in einer Metamath-Quelldatei eine Länge von höchstens 79 Zeichen haben, um die Kompatibilität zwischen verschiedenen Computerterminals zu gewährleisten.  Beim Erstellen einer Quelldatei in einem Editor wie Word, wählen Sie eine nichtproportionale Schriftart\index{nichtproportionale Schriftart} wie Courier\index{Courier Schriftart} oder Monaco\index{Monaco Schriftart}, um dies zu erleichtern.  Oder noch besser, verwenden Sie einfach einen einfachen Texteditor wie Notepad.}

Auf einigen Computersystemen kann Metamath seine Ausgaben nicht direkt ausdrucken; stattdessen senden Sie die Ausgabe in eine Datei (mit den \texttt{open}-Befehlen, die später beschrieben werden).  Die Art und Weise, wie Sie diese Ausgabedatei ausdrucken, hängt von Ihrem Computer ab.\index{Drucker} Einige Computer haben einen Druckbefehl, während Sie bei anderen Computern die Datei möglicherweise in einen Editor einlesen und von dort aus drucken müssen.

Wenn Sie Ihre Metamath-Quelldateien mit professionell formatierten Formeln, die mathematische Standardsymbole enthalten, drucken möchten, benötigen Sie das \LaTeX\
Satzprogramm\index{latex@{\LaTeX}}, das für die meisten Betriebssysteme frei verfügbar ist.  Es läuft von Haus aus auf Unix und Linux und kann unter Windows als Teil des freien Cygwin-Pakets installiert werden (\url{http://cygwin.com}).

Sie können auch {\sc html}-Webseiten\footnote{HyperText Markup Language.} erstellen.
Der Befehl {\tt help html} im Metamath-Programm unterstützt Sie bei dieser Funktion.

\section{Ihr erstes formales System}\label{start}
\subsection{Vom Nichts zur Zahl Null}\label{startf}

Um Ihnen ein Gefühl dafür zu vermitteln, wie die Metamath\index{Metamath}-Sprache aussieht, sehen wir uns ein sehr einfaches Beispiel aus der formalen Zahlentheorie\index{Zahlentheorie} an.  Dieses Beispiel stammt aus Mendelson\index{Mendelson, Elliot} \cite[S. 123]{Mendelson}.\footnote{Um das Beispiel einfach zu halten, haben wir den Formalismus leicht verändert, und was wir als
Axiome\index{Axiom} bezeichnen, sind streng genommen Theoreme\index{Theorem} in
\cite{Mendelson}.}  Wir werden eine kleine Teilmenge dieser Theorie betrachten, nämlich den Teil, der für den ersten in \cite{Mendelson} bewiesenen Satz der Zahlentheorie benötigt wird.

Zuerst werden wir uns einen standardmäßigen formalen Beweis\index{formaler Beweis} für das ausgewählte Beispiel anschauen, und dann einen Blick auf die Metamath-Version werfen.  Wenn Sie noch nie mit formalen Beweisen in Berührung gekommen sind, mag Ihnen die Notation als ein Overkill für die Darstellung so einfacher Begriffe vorkommen, so dass Sie sich vielleicht fragen, ob Sie etwas übersehen haben.  Das haben Sie nicht.  Die verwendeten Konzepte sind in der Tat sehr einfach, und eine detaillierte Aufschlüsselung auf diese Art ist notwendig, um den damit formulierten Beweis mechanisch verifizieren zu können.  Und wie Sie sehen werden, zerlegt Metamath den Beweis in noch feinere Teile, so dass der mechanische Verifikationsprozess so einfach wie möglich gehalten werden kann.

Bevor wir die Axiome\index{Axiom} der Theorie einführen können, müssen wir die Syntaxregeln\index{Syntax-Regeln} zur Bildung legaler Ausdrücke\index{Syntax-Regeln} (Kombinationen von Symbolen) definieren, mit denen diese Axiome verwendet werden können. Die Zahl 0 ist ein {\bf Term}\index{Term}; und wenn $ t$ und $r$ Terme sind, so ist auch $(t+r)$ ein solcher. Hier sind $ t$ und $r$ "`Metavariablen"'\index{Metavariable}, die sich über Terme erstrecken; sie selbst erscheinen nicht als Symbole in einem eigentlichen Term.  Einige Beispiele für konkrete Terme sind $(0 + 0)$ und $((0+0)+0)$. (Beachten Sie, dass unsere Theorie nur die Zahl Null und Summen von Nullen beschreiben kann.  Natürlich kann man mit einer so trivialen Theorie nicht viel anfangen, aber wir haben uns für unser Beispiel eine sehr kleine Teilmenge der kompletten Zahlentheorie ausgewählt.  Das Wichtigste, worauf Sie sich konzentrieren sollten, sind unsere Definitionen, die beschreiben, wie Symbole kombiniert werden, um gültige Ausdrücke zu bilden, und nicht auf den Inhalt oder die Bedeutung dieser Ausdrücke). Wenn $ t$ und $r$ Terme sind, ist ein Ausdruck der Form $ t=r$ eine {\bf wff} (wohlgeformte Formel)\index{wohlgeformte Formel (wff)}; und wenn $P$ und $Q$ wffs sind, so ist dies auch $(P\rightarrow Q)$ (was "`$P$ impliziert $Q$"'\index{Implikation ($\rightarrow$)} oder "`wenn $P$ dann $Q$"' bedeutet).
Hier sind $P$ und $Q$ Metavariablen, die sich über wffs erstrecken.  Beispiele für konkrete wffs sind $0=0$, $(0+0)=0$, $(0=0 \rightarrow (0+0)=0)$, und $(0=0\rightarrow
(0=0\rightarrow 0=(0+0)))$. (Unsere Notation verwendet mehr Klammern als üblich, aber die Verhinderung von Mehrdeutigkeiten auf diese Weise vereinfacht unser Beispiel, da die Notwendigkeit entfällt, die Vorrangigkeit von Operatoren\index{Vorrangigkeit eines Operators} zu definieren.)

Die {\bf Axiome}\index{Axiom} unserer Theorie sind alle wffs der folgenden Form, wobei $ t$, $r$ und $s$ beliebige Terme sind:

%Latex S. 92
\renewcommand{\theequation}{A\arabic{equation}}

\begin{equation}
(t=r\rightarrow (t=s\rightarrow r=s))
\end{equation}
\begin{equation}
(t+0)=t
\end{equation}

Man beachte, dass es eine unendliche Anzahl von Axiomen gibt, da es eine unendliche Anzahl von möglichen Termen gibt.  A1 und A2 müssten korrekterweise "`Axiomenschemata"'\index{Axiomenschema} genannt werden, aber der Kürze halber werden sie als "`Axiome"' bezeichnet.

Ein Axiom ist ein {\bf Theorem}; und wenn $P$ und $(P\rightarrow Q)$ Theoreme sind (wobei $P$ und $Q$ wffs sind), dann ist $Q$ auch ein Theorem.\index{Theorem} Der zweite Teil dieser Definition wird als Modus ponens-Schlussregel (MP)\index{Inferenzregel}\index{Modus ponens} bezeichnet.  Sie erlaubt es uns, neue Theoreme aus alten Theoremen zu gewinnen.

Der {\bf Beweis}\index{Beweis} eines Satzes ist eine Folge von einem oder mehreren
Theoremen, von denen jedes entweder ein Axiom oder das Ergebnis des Modus ponens ist, das auf zwei vorhergehende Theoreme in der Folge angewandt wird, und das letzte davon das zu beweisende Theorem ist.

Das Theorem, das wir für unser Beispiel beweisen wollen, ist sehr einfach: $ t=t$.  Der Beweis unseres Theorems folgt.  Studieren Sie ihn sorgfältig, bis Sie sicher sind, dass Sie ihn verstehen.\label{zeroproof}

% Use tabu so that lines will wrap automatically as needed.
\begin{tabu} { l X X }
1. & $(t+0)=t$ & (nach Axiom A2) \\
2. & $(t+0)=t$ & (nach Axiom A2) \\
3. & $((t+0)=t \rightarrow ((t+0)=t\rightarrow t=t))$ & (nach Axiom A1) \\
4. & $((t+0)=t\rightarrow t=t)$ & (nach MP angewandt auf Schritte 2 und 3) \\
5. & $t=t$ & (nach MP angewandt auf Schritte 1 und 4) \\
\end{tabu}

(Sie fragen sich vielleicht, warum Schritt 1 zweimal wiederholt wird.  Dies ist nicht notwendig in der formalen Sprache, die wir definiert haben. Aber wegen der in Metamath verwendeten "`umgekehrten polnischen Notation"' für Beweise kann auf einen vorherigen Schritt nur einmal verwiesen werden.  Die Wiederholung von Schritt~1 wird es Ihnen ermöglichen, diesen Beweis mit der Metamath\index{Metamath}-Version auf S.~\pageref{demoproof} besser abgleichen zu können).

Unser Theorem müsste richtigerweise  als "`Theoremschema"'\index{Theoremschema} bezeichnet werden, denn es stellt eine unendliche Anzahl von Theoremen dar, eines für jeden möglichen Term $ t$.  Zwei Beispiele für konkrete Theoreme wären $0=0$ und $(0+0)=(0+0)$.  Selten beweisen wir eigentliche Theoreme, da wir durch den Nachweis von Schemata eine unendliche Anzahl von Sätzen auf einen Schlag beweisen können.  In ähnlicher Weise sollte unser Beweis eigentlich als "`Beweisschema"'\index{Beweisschema} bezeichnet werden.  Um einen konkreten Beweis zu erhalten, wählen Sie einen konkreten Term, den Sie anstelle von $ t$ verwenden, und setzen ihn im gesamten Beweis für $ t$ ein.

Lassen Sie uns darüber diskutieren, was wir gerade getan haben.  Die Axiome\index{Axiom} unserer Theorie, A1 und A2, sind trivial und offensichtlich.  Jeder weiß, dass die Addition von Null zu etwas nichts verändert, und auch, dass, wenn zwei Dinge gleich einem dritten sind, sie einander gleichen. Die Feststellung des Trivialen und Offensichtlichen ist ein Ziel, das in jedem axiomatischen System angestrebt werden sollte.  Aus trivialen und offensichtlichen Wahrheiten, über die sich alle einig sind, können wir Ergebnisse beweisen, die nicht so offensichtlich sind, und dennoch absolutes Vertrauen in sie haben.  Wenn wir den Axiomen und den Regeln vertrauen, müssen wir per Definition auch den Konsequenzen dieser Axiome und Regeln vertrauen, wenn Logik überhaupt eine Bedeutung haben soll.

Unsere Schlussregel\index{Regel}, der Modus ponens\index{Modus ponens}, ist ebenfalls ziemlich offensichtlich, sobald man versteht, was sie bedeutet.  Wenn wir eine Tatsache $P$ beweisen, und wir auch beweisen, dass $P$ $Q$ impliziert, dann folgt $Q$ notwendigerweise als eine neue Tatsache.  Die Regel gibt uns ein Mittel an die Hand, um neue Fakten zu erhalten (d.h. Theoreme\index{Theorem}) aus alten Fakten zu gewinnen.

Das Theorem $ t=t$, das wir bewiesen haben, ist so grundlegend, dass man sich fragt warum es nicht zu den Axiomen gehört.  In einigen Axiomensystemen der Arithmetik {\em ist} es ein Axiom.  Die Wahl der Axiome in einer Theorie ist bis zu einem gewissen Grad willkürlich und sogar eine Art Kunst, die nur durch die Anforderung eingeschränkt wird, dass zwei gleichwertige Axiomensysteme in der Lage sein müssen, sich gegenseitig als Theoreme ableiten zu können.  Wir könnten uns vorstellen, dass der Erfinder unseres Axiomensystems ursprünglich $ t=t$ als Axiom aufnahm und dann entdeckte, dass es sich als Theorem aus den anderen Axiomen ableiten lässt.  Aus diesem Grund war es nicht notwendig, es als Axiom beizubehalten.  Durch die Streichung dieses Axioms wurde der endgültige Satz von Axiomen sehr viel einfacher.

Wenn Sie noch nie mit formalen Beweisen\index{formaler Beweis} gearbeitet haben, war es Ihnen wahrscheinlich nicht klar, dass $ t=t$ aus unseren beiden Axiomen abgeleitet werden kann, bis Sie den Beweis gesehen haben. Obwohl Sie sicherlich glauben, dass $ t=t$ wahr ist, könnten Sie einen imaginären Skeptiker, der nur an unsere beiden Axiome glaubt, nicht überzeugen, solange Sie nicht den Beweis erbringen.  Formale Beweise wie dieser sind schwer zu finden, wenn man anfängt, mit ihnen zu arbeiten. Aber wenn man sich an sie gewöhnt hat, können sie interessant werden und Spaß machen.  Sobald Sie die Idee hinter formalen Beweisen verstanden haben, haben Sie das grundlegende Prinzip, das der gesamten Mathematik zugrunde liegt, begriffen.  Je anspruchsvoller die Mathematik wird, desto anspruchsvoller werden auch die Beweise, aber letztlich lassen sie sich alle in einzelne Schritte zerlegen, die so einfach sind wie die in unserem obigen Beweis.

Das Buch von Mendelson\index{Mendelson, Elliot}, dem unser Beispiel entnommen wurde, enthält eine Reihe detaillierter formaler Beweise wie diese, und es könnte Sie interessieren, es dort nachzuschlagen.  Das Buch ist für Mathematiker gedacht, und das meiste davon hat ein ziemlich fortgeschrittenes Niveau.  Zur populären Literatur, die sich mit der Beschreibung von formalen Beweisen befasst, gehören unter anderem \cite[S.~296]{Rucker}\index{Rucker, Rudy} und \cite[S.~204--230]{Hofstadter}\index{Hofstadter, Douglas R.}.

\subsection{Konvertierung des Beweises nach Metamath}\label{convert}

Formale Beweise\index{formaler Beweis}, wie der in unserem Beispiel, zerlegen die logischen Schlussfolgerungen in kleine, präzise Schritte, die keinen Zweifel daran lassen, dass die ihre Ergebnisse aus den Axiomen\index{Axiom} folgen.  Man könnte meinen, dass dies die feinste Aufschlüsselung ist, die wir in der Mathematik erreichen können.  Es steckt jedoch mehr hinter dem Beweis, als man auf den ersten Blick vermuten würde. Obwohl unsere Axiome recht einfach waren, war eine Menge Wortklauberei nötig, bevor wir sie überhaupt formulieren konnten: Wir mussten Begriffe wie "`Term"', "`wff"' und so weiter definieren.  Darüber hinaus gibt es eine Reihe von impliziten Regeln, die wir noch nicht einmal erwähnt haben. Woher wissen wir zum Beispiel, dass Schritt 3 unseres Beweises aus dem Axiom A1 folgt? Es gibt einige versteckte Argumentationen, um dies zu bestimmt sagen zu können.  Axiom A1 hat zwei Vorkommen des Buchstabens $ t$.  Eine der impliziten Regeln besagt, dass alles, was wir für $ t$ ersetzen, ein legaler Term\index{Term}\footnote{Einige Autoren machen diese implizite Regel explizit, indem sie nach der Definition von "`Term"' sagen: "`Nur Ausdrücke der obigen Form sind Terme"'.} sein muss.  Der Ausdruck $ t+0$ ist ganz offensichtlich ein legaler Term, wenn $ t$ ein solcher ist. Aber nehmen wir an, wir wollten solche Ersetzungen in einem riesigen Term mit Tausenden von Symbolen durchführen?  Sicherlich wäre eine Menge Arbeit damit verbunden festzustellen, dass es sich wirklich bei dem Ergebnis um einen Term handelt, aber in gewöhnlichen formalen Beweisen würde man all diese Arbeit als einen einzigen "`Schritt"' betrachten.

Um unser Axiomensystem in der Metamath\index{Metamath}-Sprache auszudrücken, müssen wir diese Zusatzinformationen zusätzlich zu den eigentlichen Axiomen be\-schrei\-ben.
Metamath weiß nicht, was ein "`Term"' oder eine "`wff"'\index{wohlgeformte Formel (wff)} ist.  In Metamath ist die Spezifikation der Art und Weise, in der wir Symbole kombinieren können, um Terme und wffs zu erhalten, selbst wie kleine Axiome zu verstehen.  Diese Hilfsaxiome werden in der gleichen Notation ausgedrückt wie die "`echten"' Axiome\index{Axiom}, und Metamath unterscheidet nicht zwischen diesen beiden.  Die Unterscheidung wird von Ihnen getroffen, d.h. durch die Art und Weise, wie Sie die Notation interpretieren, die Sie gewählt haben, um diese beiden Arten von Axiomen auszudrücken.

Die Metamath-Sprache zerlegt mathematische Beweise in winzige Teile, viel mehr als in gewöhnlichen formalen Beweisen\index{formaler Beweis}.  Wenn ein einzelner Schritt\index{Beweisschritt} die Substitution\index{Substitution!Variable}\index{Variablensubstitution} eines
komplexen Terms für eine seiner Variablen beinhaltet, muss Metamath diesen einzelnen Schritt in viele kleine Schritte unterteilt behandeln.  Diese feinkörnige Aufschlüsselung ist es, die Metamath Allgemeingültigkeit und Flexibilität gibt, da es sich nicht auf eine bestimmte mathematische Notation beschränkt.

Die Beweisnotation von Metamath ist nicht dazu gedacht, von Menschen gelesen zu werden, sondern ihr kompaktes Format ist für eine Maschine gedacht.  Das Metamath-Programm konvertiert diese Notation in eine Form, die Sie verstehen können, indem es den \texttt{show proof}\index{\texttt{show proof}-Befehl} Befehl nutzt.  Sie können dem Programm mitteilen, wie detailliert Sie den Beweis betrachten möchten.  Vielleicht möchten Sie sich nur die logischen Folgerungsschritte ansehen, die den normalen formalen Beweisschritten entsprechen, oder Sie möchten die feingranularen Schritte sehen, die beweisen, dass ein Ausdruck ein Term ist.

Hier ist schließlich, ohne weitere Erläuterungen, unser Beispiel, das in die Metamath\index{Metamath}-Sprache konvertiert wurde:\index{Metavariable}\label{demo0}

\begin{verbatim}
$( Festlegen der konstanten Symbole, die wir nutzen wollen $)
    $c 0 + = -> ( ) term wff |- $.
$( Festlegen der Metavariablen, die wir nutzen wollen $)
    $v t r s P Q $.
$( Spezifizieren der Eigenschaften der Metavariablen $)
    tt $f term t $.
    tr $f term r $.
    ts $f term s $.
    wp $f wff P $.
    wq $f wff Q $.
$( Definieren von "Term" und "wff" $)
    tze $a term 0 $.
    tpl $a term ( t + r ) $.
    weq $a wff t = r $.
    wim $a wff ( P -> Q ) $.
$( Festlegen der Axiome $)
    a1 $a |- ( t = r -> ( t = s -> r = s ) ) $.
    a2 $a |- ( t + 0 ) = t $.
$( Definieren der Schlussregel Modus ponens $)
    ${
       min $e |- P $.
       maj $e |- ( P -> Q ) $.
       mp  $a |- Q $.
    $}
$( Beweisen eines Theorems $)
    th1 $p |- t = t $=
  $( Hier ist sein Beweis: $)
       tt tze tpl tt weq tt tt weq tt a2 tt tze tpl
       tt weq tt tze tpl tt weq tt tt weq wim tt a2
       tt tze tpl tt tt a1 mp mp
     $.
\end{verbatim}\index{Metavariable}

Eine "`Datenbasis"'\index{Datenbasis} ist ein Satz von einer oder mehreren {\sc ascii} Quelldateien.  Es folgt eine kurze Beschreibung solch einer Metamath\index{Metamath}-Datenbasis (die aus einer einzigen Quelldatei besteht), damit Sie allgemein verstehen können, was vor sich geht.  Um die Quelldatei im Detail zu verstehen, sollten Sie Kapitel~\ref{languagespec} lesen.

Die Datenbasis ist eine Folge von "`Token"'\index{Token}, die normalerweise durch Leerzeichen oder Zeilenumbrüche voneinander getrennt sind.  Die einzigen Token, die in der Metamath-Sprache fest vorgegeben sind, sind diejenigen die mit \texttt{\$} beginnen.  Diese Token werden "`Schlüsselwörter"'\index{Schlüsselwort} genannt.  Alle anderen Token sind benutzerdefiniert, und ihre Namen sind beliebig.

Wie Sie vielleicht schon vermutet haben, beginnt mit dem Metamath-Token \texttt{\$(}\index{\texttt{\$(} und
\texttt{\$)} Hilfsschlüsselwörter} ein Kommentar und wird mit \texttt{\$)} beendet.

Die Metamath-Token \texttt{\$c}\index{\texttt{\$c}-Anweisung},
\texttt{\$v}\index{\texttt{\$v}-Anweisung},
\texttt{\$e}\index{\texttt{\$e}-Anweisung},
\texttt{\$f}\index{\texttt{\$f}-Anweisung},
\texttt{\$a}\index{\texttt{\$a}-Anweisung} und
\texttt{\$p}\index{\texttt{\$p}-Anweisung} spezifizieren "`Anweisungen"', die
mit \texttt{\$.}\,\index{\texttt{\$.} Schlüsselwort} enden.

Die Metamath-Token \texttt{\$c} und \texttt{\$v} deklarieren\index{Konstantendeklaration}\index{Variablendeklaration} jeweils eine Liste von benutzerdefinierten Token, genannt "`Mathematische Symbole"',\index{mathematisches Symbol} die später in der Datenbasis verwendet werden.  Alle mathematischen Symbole für unser Beispiel werden so definiert, außer dem Drehkreuzsymbol \texttt{|-} ($\vdash$)\index{Drehkreuz ({$\,\vdash$})}, das von Logikern üblicherweise verwendet wird um zu sagen: "`Ein Beweis existiert für"'.  Für uns ist das Drehkreuz nur ein praktisches Symbol zur Unterscheidung zwischen Ausdrücken, die Axiome\index{Axiom} oder Theoreme\index{Theorem} darstellen, und Ausdrücken, die Terme oder wffs sind.

Die Anweisung \texttt{\$c} deklariert "`Konstanten"'\index{Konstante}\index{Konstantendeklaration}, und die Anweisung \texttt{\$v} deklariert "`Variablen"'\index{Variable}\index{Variablendeklaration} (oder genauer gesagt, Metavariablen\index{Metavariable}).
Eine Variable kann durch Folgen von mathematischen Symbolen ersetzt\index{Substitution!Variable}\index{Variablensubstitution} werden, während eine Konstante nicht ersetzt werden kann.

Es mag redundant erscheinen, dass sowohl \texttt{\$c}\index{\texttt{\$c}-Anweisung} als auch \texttt{\$v}\index{\texttt{\$v}-Anweisung} erforderlich sind (da jedes mathematische Symbol, das nicht mit einer \texttt{\$c}-Anweisung spezifiziert wurde, als Variable angesehen werden könnte). Aber dies ermöglicht eine bessere Prüfung auf Fehler und erlaubt es auch, mathematische Symbole neu zu  deklarieren\index{Umdeklarierung von Symbolen} (Abschnitt~\ref{scoping}).

Das Token \texttt{\$f}\index{\texttt{\$f}-Anweisung} spezifiziert eine Anweisung, die als "`Hypothese vom Variablentyp"' (auch als "`fließende Hypothese"' bekannt) und \texttt{\$e}\index{\texttt{\$e}-Anweisung} spezifiziert eine "`logische Hypothese"' (auch "`essentielle Hypothese"').\index{Hypothese}\index{Variablentyp-Hypothese}\index{logische Hypothese}\index{fließende Hypothese}\index{essentielle Hypothese}
Das Token \texttt{\$a}\index{\texttt{\$a}-Anweisung} spezifiziert eine "`axiomatische
Behauptung"'\index{axiomatische Behauptung}, und \texttt{\$p}\index{\texttt{\$p}-Anweisung} spezifiziert eine "`beweisbare Behauptung"'\index{beweisbare Behauptung}. Links von jedem Vorkommen dieser vier Token befindet sich ein "`Label"' \index{Label}, das die Hypothese oder Behauptung für eine spätere Bezugnahme identifiziert.  Zum Beispiel ist das Label der ersten axiomatischen Behauptung \texttt{tze}.  Eine \texttt{\$f}-Anweisung muss genau zwei mathematische Symbole enthalten, eine Konstante gefolgt von einer Variable.  Die Anweisungen \texttt{\$e}, \texttt{\$a}, und \texttt{\$p} beginnen jeweils mit einer Konstanten, auf die im Allgemeinen eine beliebige Folge von mathematischen Symbolen folgt.

Jeder Behauptung\index{Behauptung} ist ein Satz von Hypothesen zugeordnet die erfüllt sein müssen, damit die Behauptung in einem Beweis verwendet werden kann.
Diese werden "`obligatorische Hypothesen"' \index{obligatorische Hypothese} der Behauptung genannt.  Zu den Hypothesen, deren "`Gültigkeitsbereich"' (siehe unten) die Behauptung umfasst, sind \texttt{\$e} Hypothesen immer obligatorisch und \texttt{\$f}\index{\texttt{\$f}-Anweisung} Hypothesen sind obligatorisch, wenn sie ihre Variable mit der Behauptung oder ihren \texttt{\$e}-Hypothesen teilen.  Die genauen Regeln zur Bestimmung, welche Hypothesen obligatorisch sind, werden in den Abschnitten~\\ref{frames} und \ref{scoping} im Detail beschrieben.  Zum Beispiel sind \texttt{tt} und \texttt{tr} die obligatorischen Hypothesen der Behauptung \texttt{tpl}, während die Behauptung \texttt{tze} keine obligatorischen Hypothesen hat, weil sie keine Variablen enthält und keine \texttt{\$e}\index{\texttt{\$e}-Anweisung} Hypothese hat.  Metamaths \texttt{show statement}-Befehl\index{\texttt{show statement}-Befehl}, der im nächsten Abschnitt beschrieben wird, zeigt Ihnen die obligatorischen Hypothesen einer Behauptung.

Manchmal soll eine Hypothese nur für bestimmte Behauptungen relevant sein.
Die Menge der Behauptungen, für die eine Hypothese relevant ist, wird ihr "`Gültigkeitsbereich"' genannt.  Die Metamath-Klammern, \texttt{\$\char`\{}\index{\texttt{\$\char`\{} und \texttt{\$\char`\}} Schlüsselwörter} und \texttt{\$char`\}}, definieren einen "`Block"'\index{Block}, der den Gültigkeitsbereich jeder dazwischen liegenden Hypothese abgrenzt.  Die Behauptung \texttt{mp} hat obligatorische Hypothesen \texttt{wp}, \texttt{wq}, \texttt{min} und \texttt{maj}.  Die einzige obligatorische Hypothese von \texttt{th1} ist dagegen \texttt{tt}, da \texttt{th1} außerhalb des Blocks auftritt, der \texttt{min} und \texttt{maj} enthält.

Beachten Sie, dass \texttt{\$\char`\{} und \texttt{\$\char`\}} keine Auswirkungen auf den Gültigkeitsbereich von Behauptungen (\texttt{\$a} und \texttt{\$p}) haben.  Behauptung sind immer verfügbar, um von jedem späteren Beweis in der Quelldatei referenziert zu werden.

Jede beweisbare Behauptung (\texttt{\$p}\index{\texttt{\$p}-Anweisung} Behauptung) besteht aus zwei Teilen.  Der erste Teil ist die Behauptung\index{Behauptung} selbst: eine Folge von mathematischen Symbolen\index{mathematisches Symbol}, die zwischen dem \texttt{\$p}-Token und einem \texttt{\$=}\index{\texttt{\$=} Schlüsselwort}-Token platziert ist.  Der zweite Teil ist ein "`Beweis"': eine Liste von Label-Token, die zwischen dem \texttt{\$=}-Token und dem \texttt{\$.}-Token\index{\texttt{\$.} Schlüsselwort}, das die Behauptung beendet, liegt.\footnote{Wenn Sie die \texttt{set.mm}-Datenbasis angesehen haben, ist Ihnen vielleicht eine andere Notation aufgefallen, die für
Beweise genutzt wird.  Die andere Notation wird "`komprimiert"'\index{komprimierter Beweis}\index{Beweis!komprimiert} genannt. Sie reduziert den Platzbedarf zur effizienten Speicherung eines Beweises in der Datenbasis, und wird in Anhang~\ref{compressed} beschrieben.  In dem obigen Beispiel verwenden wir die "`normale"'\index{normaler Beweis}\index{Beweis!normal} Notation.} 
Der Beweis fungiert als eine Reihe von Anweisungen für das Metamath-Programm, die ihm sagen, wie die Abfolge der mathematischen Symbole, die im Behauptungsteil der Anweisung enthalten ist, unter Verwendung der Hypothesen der \texttt{\$p}-Anweisung und der vorherigen Behauptungen.  Die Konstruktion erfolgt nach genauen Regeln.  Wenn die Liste der Label im Beweis gegen diese Regeln verstößt oder wenn die sich ergebende Endsequenz nicht mit der Behauptung übereinstimmt, gibt das Metamath-Programm eine Fehlermeldung aus.

Wenn Sie mit der umgekehrten polnischen Notation (engl. "`reverse Polish notation"'; RPN) vertraut sind, die manchmal auf Taschenrechnern verwendet wird, wissen Sie, wie ein Beweis in Kurzform funktioniert.  Jedes Hypothesenlabel\index{Hypothesenlabel} im Beweis wird auf den RPN-Stapel\index{Stapel}\index{RPN-Stapel} geschoben\index{schieben} sobald es auftaucht. Jedes Label einer Behauptung\index{Behauptungslabel} entfernt\index{entfernen} so viele
Einträge vom Stapel, wie die referenzierte Behauptung zwingende Hypothesen hat.  Variablenersetzungen\index{Substitution!Variable}\index{Variablensubstitution} werden
aus den obligatorischen Hypothesen der referenzierten Behauptung so berechnet, dass diese Hypothesen mit den Stack-Einträgen übereinstimmen. Die gleichen Substitutionen
werden dann an den Variablen in der referenzierten Aussage selbst vorgenommen, die dann auf den Stapel abgelegt wird.  Am Ende des Beweises sollte es nur noch einen Eintrag im Stapel geben, nämlich die zu beweisende Behauptung.  Dieser Vorgang wird im Abschnitt~\ref{proof} ausführlich erläutert.

Die Beweisnotation von Metamath ist für Menschen nicht einfach lesbar, aber sie erlaubt es, den Beweis kompakt in einer Datei zu speichern.  Das Programm Metamath\index{Metamath} hat Funktionen zur Anzeige von Beweisen in einer lesbareren Art, mit der man besser sehen kann, was vor sich geht, wie Sie im nächsten Abschnitt sehen werden.

Die Regeln, die bei der Überprüfung eines Beweises verwendet werden, basieren weder auf einer eingebauten Syntax der Symbolsequenz in einer Behauptung noch auf irgendwelchen eingebauten Bedeutungen, die mit bestimmten Symbolnamen verbunden sind.  Sie basieren ausschließlich auf Symbolübereinstimmungen: Konstanten\index{Konstante} müssen mit sich selbst übereinstimmen, und Variablen\index{Variable} können durch alles ersetzt werden, was eine Übereinstimmung ermöglicht.  Anstelle von \texttt{term}, \texttt{0} und \verb$|-$ könnten wir zum Beispiel ebenso gut \texttt{gelb}, \texttt{Null} und \texttt{beweisbar} verwenden, solange wir dies in der gesamten Datenbasis einheitlich tun.  Wir hätten auch \texttt{ist beweisbar} (zwei Token) anstelle von \verb$|-$ (ein Token) in der gesamten Datenbasis verwenden können.  In jedem dieser Fälle wäre der Beweis genau derselbe.  Die Unabhängigkeit von Beweisen und Notation bedeutet, dass Sie viele Möglichkeiten haben die verwendete Notation zu ändern, ohne die Beweise ändern zu müssen.

\section{Ein Probelauf}\label{trialrun}

Jetzt sind Sie bereit, das Metamath\index{Metamath}-Programm auszuprobieren.

Metamath verfügt auf allen Computersystemen über eine standardmäßige "`Befehlszeilen Schnittstelle"' (command line interface; CLI)\index{Befehlszeilenschnittstelle (CLI)}, die es Ihnen ermöglicht, mit dem Programm zu interagieren.
Sie geben Befehle in die CLI ein, indem Sie sie auf der Tastatur eintippen und nach jeder Zeile die {\em Eingabe}-Taste Ihrer Tastatur drücken.
Die CLI ist einfach zu bedienen und verfügt über integrierte Hilfe-Funktionen.

Als erstes sollten Sie einen Texteditor verwenden, um eine Datei mit dem Namen \texttt{demo0.mm} zu erstellen und in diese den Metamath-Quelltext einzugeben, der auf S.~\pageref{demo0} gezeigt wird.  Eigentlich ist diese Datei im Metamath Software-Paket enthalten, also überprüfen Sie das zuerst.  Wenn Sie den Quelltext eintippen, stellen Sie sicher, dass Sie ihn in Form von "`reinem {\sc ascii} Text mit Zeilenumbrüchen"' speichern.  Die meisten Textverarbeitungsprogramme verfügen über diese Funktion.

Als nächstes müssen Sie das Metamath-Programm ausführen.  Abhängig von Ihrem Computer
System und der Art der Installation von Metamath kann dies von einem Mausklick auf das Metamath-Symbol, die Eingabe von \texttt{run metamath} oder durch die einfache Eingabe von \texttt{metamath} reichen.  (Der Befehl {\tt help invoke} von Metamath beschreibt
alternative Möglichkeiten, das Metamath-Programm aufzurufen.)

Wenn Sie Metamath\index{Metamath} zum ersten Mal eingeben, befindet es sich in der CLI und wartet auf Ihre Eingabe. Auf Ihrem Bildschirm sehen Sie dann etwas wie das Folgende:
\begin{verbatim}
Metamath - Version 0.177 27-Apr-2019
Type HELP for help, EXIT to exit.
MM>
\end{verbatim}
Die Eingabeaufforderung \texttt{MM>} bedeutet, dass Metamath auf einen Befehl wartet.
Bei Befehlsschlüsselwörtern wird nicht zwischen Groß- und Kleinschreibung unterschieden; wir werden in unseren Beispielen Befehle in Kleinbuchstaben verwenden.
Die Versionsnummer und das Veröffentlichungsdatum werden auf Ihrem System wahrscheinlich von der oben gezeigten Version abweichen.

Als erstes müssen Sie Ihre Datenbasis einlesen:\index{\texttt{read}-Befehl}\footnote{Wenn in Unix\index{Unix-Dateinamen}\index{Dateinamen!Unix} ein Verzeichnispfad benötigt wird, sollten Sie den Pfad/Dateinamen in Anführungszeichen setzen, damit Metamath nicht denkt, dass das \texttt{/} im Pfadnamen ein Befehlszeilenparameter ist, z.B., \texttt{read \char`\"`db/set.mm\char`\"'}.  Anführungszeichen sind optional, wenn keine Zweideutigkeit besteht.}
\begin{verbatim}
MM> read demo0.mm
\end{verbatim}
Denken Sie daran, nach der Eingabe dieses Befehls die {\em Eingabe}taste zu drücken.  Wenn Sie den Dateinamen weglassen, wird Metamath Sie nach einem Namen fragen.   Die Syntax für die Angabe eines Macintosh-Dateipfades ist in einer Fußnote auf
S.~\pageref{includef}.\index{Macintosh-Dateinamen}\index{Dateinamen!Macintosh} zu finden.

Wenn es in der Datenbasis Syntaxfehler gibt, wird Metamath Sie beim Einlesen der Datei darauf hinweisen.  Das Einzige, was Metamath beim Einlesen einer Datenbasis nicht prüft, ist die Korrektheit aller Beweise, denn das würde es zu sehr verlangsamen.
Es ist jedoch ratsam, die Beweise in einer Datenbasis, an der Sie Änderungen vornehmen, regelmäßig zu überprüfen.
Verwenden Sie dazu den folgenden Befehl (und führen Sie ihn jetzt für Ihre Datei \texttt{demo0.mm} aus).  Beachten Sie, dass \texttt{*} ein `Platzhalter"' ist, der alle Beweise in der Datei repräsentiert.\index{\texttt{verify proof}-Befehl}
\begin{verbatim}
MM> verify proof *
\end{verbatim}
Metamath meldet alle ungültigen Beweise.

Es ist oft nützlich, die Informationen zu speichern, die das Metamath-Programm auf dem Bildschirm anzeigt. Sie können alles speichern, was auf dem Bildschirm passiert, indem Sie eine Protokolldatei öffnen. Sie sollten dies tun, bevor Sie eine Datenbasis einlesen, damit Sie später eventuelle Fehler untersuchen können.  Um eine Protokolldatei zu öffnen, geben Sie Folgendes ein:
\begin{verbatim}
MM> open log abc.log
\end{verbatim}
Dadurch wird eine Datei namens \texttt{abc.log} geöffnet, und alles, was ab diesem Zeitpunkt auf dem Bildschirm erscheint, wird in dieser Datei gespeichert.  Der Name der Protokolldatei ist frei wählbar. Um die Protokolldatei zu schließen, geben Sie Folgendes ein:
\begin{verbatim}
MM> close log
\end{verbatim}

Sie können mit mehreren Befehlen untersuchen, was in Ihrer Datenbasis enthalten ist.
Abschnitt~\ref{exploring} enthält eine Übersicht über einige nützliche Befehle.  Der Befehl \texttt{show labels} lässt Sie sehen, welche Label\index{Label} vorhanden sind.  Ein \texttt{*} entspricht einer beliebigen Kombination von Zeichen, und \texttt{t*} bezieht sich auf alle Labels, die mit dem Buchstaben \texttt{t}\index{\texttt{show labels}-Befehl} beginnen. Das \texttt{/all} ist ein "`Befehlszeilenparameter"' \index{Befehlszeilenparameter}, der Metamath anweist, die Beschriftungen der Hypothesen einzuschließen.  (Um die Syntax erklärt zu bekommen, geben Sie\texttt{help show labels} ein .)  

Geben Sie Folgendes ein:
\begin{verbatim}
MM> show labels t* /all
\end{verbatim}

Metamath antwortet mit
\begin{verbatim}
The statement number, label, and type are shown.
3 tt $f       4 tr $f       5 ts $f       8 tze $a
9 tpl $a      19 th1 $p
\end{verbatim}

Sie können den Befehl \texttt{show statement} verwenden, um Informationen über eine bestimmte Anweisung\index{\texttt{show statement}-Befehl} zu erhalten.
Sie können zum Beispiel Informationen über die Anweisung mit der Bezeichnung \texttt{mp} bekommen, wenn Sie Folgendes eingeben:
\begin{verbatim}
MM> show statement mp /full
\end{verbatim}
Metamath antwortet mit
\begin{verbatim}
  Statement 17 is located on line 43 of the file
  "demo0.mm".
  "Define the modus ponens inference rule"
  17 mp $a |- Q $.
  Its mandatory hypotheses in RPN order are:
    wp $f wff P $.
    wq $f wff Q $.
    min $e |- P $.
    maj $e |- ( P -> Q ) $.
  The statement and its hypotheses require the
        variables:  Q P
  The variables it contains are:  Q P
\end{verbatim}
Die obligatorischen Hypothesen\index{obligatorische Hypothese} und ihre Reihenfolge\index{RPN-Reihenfolge} sind nützliche Informationen, wenn Sie versuchen, einen Beweis zu verstehen oder zu debuggen.

Jetzt sind Sie bereit, sich anzusehen, was wirklich in unserem Beweis enthalten ist.  Zunächst wird hier gezeigt, wie man sich jeden Schritt des Beweises ansieht - nicht nur die Schritte eines gewöhnlichen formalen Beweises\index{formaler Beweis}, sondern auch diejenigen, die die Formeln aufbauen, die in jedem Schritt eines gewöhnlichen formalen Beweises auftauchen.\index{\texttt{show proof}-Befehl}
\begin{verbatim}
MM> show proof th1 /lemmon /all
\end{verbatim}

Dadurch wird der Beweis in folgendem Format auf dem Bildschirm angezeigt:
\begin{verbatim}
 1 tt            $f term t
 2 tze           $a term 0
 3 1,2 tpl       $a term ( t + 0 )
 4 tt            $f term t
 5 3,4 weq       $a wff ( t + 0 ) = t
 6 tt            $f term t
 7 tt            $f term t
 8 6,7 weq       $a wff t = t
 9 tt            $f term t
10 9 a2          $a |- ( t + 0 ) = t
11 tt            $f term t
12 tze           $a term 0
13 11,12 tpl     $a term ( t + 0 )
14 tt            $f term t
15 13,14 weq     $a wff ( t + 0 ) = t
16 tt            $f term t
17 tze           $a term 0
18 16,17 tpl     $a term ( t + 0 )
19 tt            $f term t
20 18,19 weq     $a wff ( t + 0 ) = t
21 tt            $f term t
22 tt            $f term t
23 21,22 weq     $a wff t = t
24 20,23 wim     $a wff ( ( t + 0 ) = t -> t = t )
25 tt            $f term t
26 25 a2         $a |- ( t + 0 ) = t
27 tt            $f term t
28 tze           $a term 0
29 27,28 tpl     $a term ( t + 0 )
30 tt            $f term t
31 tt            $f term t
32 29,30,31 a1   $a |- ( ( t + 0 ) = t -> ( ( t + 0 )
                                     = t -> t = t ) )
33 15,24,26,32 mp  $a |- ( ( t + 0 ) = t -> t = t )
34 5,8,10,33 mp  $a |- t = t
\end{verbatim}

Der Befehlszeilenparameter \texttt{/lemmon} spezifiziert eine Anzeige im sogenannten Lemmon-Stil\index{Lemmon-Stil Beweis}\index{Beweis!Lemmon-Stil}.  Das Weglassen der Option \texttt{/lemmon} führt zu einer Baumdarstellung des Beweises (siehe S.~\pageref{treeproof} für ein Beispiel), der etwas weniger eindeutig ist, aber leichter zu folgen ist, wenn man sich daran gewöhnt hat.\index{Baumdarstellung eines Beweises}\index{Beweis!Baumdarstellung}

Die erste Zahl in jeder Zeile ist die Schrittnummer des Beweises.  Alle folgenden Zahlen sind Schrittnummern, die den Hypothesen der Aussage zugeordnet sind, auf die dieser Schritt verweist.  Die nächste Zahl ist die Bezeichnung der, auf die der Schritt verweist.  Der Typ der Aussage, auf die verwiesen wird, kommt als nächstes, gefolgt von der Folge mathematischer Symbole\index{mathematisches Symbol}, die durch den Beweis bis zu diesem Schritt konstruiert wurde.

Der letzte Schritt, 34, enthält die Aussage, die bewiesen wird.

Betrachtet man einen kleinen Teil des Beweises, so stellt man fest, dass die Schritte 3 und 4 aussagen, dass \texttt{( t + 0 )} und \texttt{t} \texttt{term}\,e sind, und Schritt 5 nutzt die Schritte 3 und 4, um festzustellen, dass \texttt{( t + 0 ) = t} eine \texttt{wff} ist.  Lassen Sie Metamath selbst im Detail erklären, was in Schritt 5 geschieht.  Beachten Sie, dass die "`Zielhypothese"' sich darauf bezieht, wo Schritt 5 letztendlich verwendet wird, d.h. in Schritt 34.
\begin{verbatim}
MM> show proof th1 /detailed_step 5
Proof step 5:  wp=weq $a wff ( t + 0 ) = t
This step assigns source "weq" ($a) to target "wp"
($f).  The source assertion requires the hypotheses
"tt" ($f, step 3) and "tr" ($f, step 4).  The parent
assertion of the target hypothesis is "mp" ($a,
step 34).
The source assertion before substitution was:
    weq $a wff t = r
The following substitutions were made to the source
assertion:
    Variable  Substituted with
     t         ( t + 0 )
     r         t
The target hypothesis before substitution was:
    wp $f wff P
The following substitution was made to the target
hypothesis:
    Variable  Substituted with
     P         ( t + 0 ) = t
\end{verbatim}

Der soeben gezeigte vollständige Beweis ist nützlich für das Verständnis, was im Detail vor sich geht.
Die meiste Zeit werden Sie jedoch nur an den "`wesentlichen"' oder logischen Schritten eines Beweises, d.h. die Schritte, die einem gewöhnlichen formalen Beweisindex\index{formaler Beweis} entsprechen, interessiert sein.  Wenn Sie Folgendes eingeben
\begin{verbatim}
MM> show proof th1 /lemmon /renumber
\end{verbatim}
dann sehen Sie\label{demoproof}
\begin{verbatim}
1 a2             $a |- ( t + 0 ) = t
2 a2             $a |- ( t + 0 ) = t
3 a1             $a |- ( ( t + 0 ) = t -> ( ( t + 0 )
                                     = t -> t = t ) )
4 2,3 mp         $a |- ( ( t + 0 ) = t -> t = t )
5 1,4 mp         $a |- t = t
\end{verbatim}
Vergleichen Sie dies mit dem formalen Beweis auf S.~\pageref{zeroproof} und beachten Sie die Ähnlichkeit.
Standardmäßig zeigt Metamath für einen Beweis nicht \texttt{\$f}\index{\texttt{\$f}-Anweisung}-Hypothesen und alles, was von ihnen im Beweisbaum abzweigt; dadurch sieht der Beweis eher wie ein gewöhnlicher mathematischen Beweis aus, der normalerweise keine explizite Konstruktion von Ausdrücken beinhaltet.
Dies wird die "`essentielle"' Ansicht genannt (früher musste man den Parameter \texttt{/essential} im Befehl \texttt{show proof} hinzufügen, um diese Ansicht zu erhalten, aber das ist jetzt die Standardeinstellung).
Sie können den Parameter \texttt{/all} im Befehl \texttt{show proof} verwenden, um auch die explizite Konstruktion von Ausdrücken anzuzeigen.
Durch den Parameter \texttt{/renumber} werden die Schritte neu nummeriert, damit sie dem entsprechen, was angezeigt wird.\index{\texttt{show proof}-Befehl}

Um Metamath zu verlassen, geben Sie Folgendes ein:\index{\texttt{exit}-Befehl}
\begin{verbatim}
MM> exit
\end{verbatim}

\subsection{Einige Hinweise zur Verwendung der Befehlszeilenschnittstelle}

Wir schließen diese kurze Einführung in Metamath\index{Metamath} mit einigen hilfreichen Hinweisen, wie Sie sich durch die Befehle navigieren können.
\index{Befehlszeilenschnittstelle (CLI)}

Wenn Sie Befehle in Metamaths CLI eingeben, müssen Sie nur so viele Zei\-chen eines Befehlsschlüsselworts\index{Befehlsschlüsselwort} eingeben, die für die Eindeutigkeit notwendig sind.  Wenn Sie zu wenige Zeichen eingeben, wird Metamath Ihnen mitteilen, welche die Auswahlmöglichkeiten sind.  Im Fall des Befehls \texttt{read} ist nur das \texttt{r} nötigt, um ihn eindeutig zu spezifizieren. Sie hätten also Folgendes eingeben können\index{\texttt{read}-Befehl}

\begin{verbatim}
MM> r demo0.mm
\end{verbatim}

anstelle von

\begin{verbatim}
MM> read demo0.mm
\end{verbatim}

In unserer Beschreibung geben wir immer die vollständigen Befehlswörter an.  Bei Verwendung der Metamath CLI-Befehle in einer Befehlsdatei (zum Einlesen mit dem Befehl \texttt{submit})\index{\texttt{submit}-Befehl}, ist es ratsam, den ungekürzten Befehl zu verwenden, um sicherzustellen, dass Ihre Anweisungen nicht mehrdeutig werden, wenn dem Metamath-Programm in Zukunft weitere Befehle hinzugefügt werden.

Die Befehlsschlüsselwörter\index{Befehlsschlüsselwort} unterscheiden nicht zwischen Groß- und Kleinschreibung; Sie können entweder \texttt{read} oder \texttt{ReAd} eingeben.  Bei Dateinamen wird je nach Betriebssystem Ihres Computers zwischen Groß- und Kleinschreibung unterschieden.
Metamath Label\index{Label} und mathematische Symbole\index{mathematisches Symbol}-Tokens\index{Token} unterscheiden zwischen Groß- und Kleinschreibung.

Der Befehl \texttt{help}\index{\texttt{help}-Befehl} bietet Ihnen eine Liste der Themen, zu denen Sie Hilfe erhalten können.  Sie können dann \texttt{help} {\em topic} eingeben, um Hilfe zu diesem Thema zu erhalten.

Wenn Sie sich über die Schreibweise eines Befehls nicht sicher sind, geben Sie einfach die Zeichen des Befehls ein, an die Sie sich erinnern.
Wenn Sie nicht genug Zeichen eingetippt haben, um ihn eindeutig zu spezifizieren, wird Metamath Ihnen sagen, welche Auswahlmöglichkeiten Sie haben.

\begin{verbatim}
MM> show s
         ^
?Ambiguous keyword - please specify SETTINGS,
STATEMENT, or SOURCE.
\end{verbatim}

Wenn Sie nicht wissen, welches Argument Sie als Teil eines Befehls verwenden sollen, geben Sie ein \texttt{?}\index{\texttt{]}@\texttt{?}\ in Befehlszeilen}\ an der
Position des Arguments ein.  Metamath wird Ihnen sagen, was es dort erwartet.

\begin{verbatim}
MM> show ?
         ^
?Expected SETTINGS, LABELS, STATEMENT, SOURCE, PROOF,
MEMORY, TRACE_BACK, or USAGE.
\end{verbatim}

Schließlich können Sie auch nur das erste Wort oder die ersten Wörter eines Befehls eingeben, gefolgt von {\em return}.  Metamath fragt Sie nach dem restlichen Teil des Befehls und zeigt Ihnen bei jedem Schritt die Auswahlmöglichkeiten an. Sie könnten anstelle der Eingabe von \texttt{show statement th1 /full} beispielsweise folgendermaßen vorgehen:
\begin{verbatim}
MM> show
SETTINGS, LABELS, STATEMENT, SOURCE, PROOF,
MEMORY, TRACE_BACK, or USAGE <SETTINGS>? st
What is the statement label <th1>?
/ or nothing <nothing>? /
TEX, COMMENT_ONLY, or FULL <TEX>? f
/ or nothing <nothing>?
19 th1 $p |- t = t $= ... $.
\end{verbatim}

Nach jedem \texttt{?}\ in diesem Modus müssen Sie Metamath die Informationen geben, die es anfordert.  Manchmal gibt Metamath Ihnen eine Liste von Auswahlmöglichkeiten an, wobei die Standardauswahl durch Klammern \texttt{< > } angezeigt wird. Durch das Drücken von {\em return} nach dem \texttt{?}\, wird die Standardauswahl ausgewählt.
Wenn Sie eine andere Antwort geben, wird die Standardauswahl außer Kraft gesetzt.  Beachten Sie, dass der \texttt{/} in Befehlszeilenparametern als ein separates Token\index{Token} betrachtet wird und deshalb separat abgefragt wird.

\section{Ihr erster Beweis}\label{frstprf}

Beweise werden mit Hilfe des Beweis-Assistenten\index{Beweis-Assistent} erstellt.
Wir werden Ihnen nun zeigen, wie der Beweis des Satzes \texttt{th1} aufgebaut wurde.  Damit Sie diese Schritte selbst nachvollziehen können, lassen wir zunächst den Beweis-Assistenten den Beweis im Quellpuffer\index{Quellpuffer} von Metamath löschen und ihn dann rekonstruieren.  (Der Quellpuffer ist die Stelle im Speicher wo Metamath die Informationen aus der Datenbasis speichert, wenn sie eingelesen werden (mit dem \texttt{read} Befehl\index{\texttt{read}-Befehl}).  Neue oder geänderte Beweise
werden im Quellpuffer gehalten, bis ein \texttt{write source} Befehl\index{\texttt{write source}-Befehl} erteilt wird).  In der Praxis würde man ein \texttt{?}\index{\texttt{]}@\texttt{?}\ innerhalb von Beweisen} zwischen \texttt{\$=}\index{\texttt{\$=} Schlüsselwort} und
\texttt{\$.}\index{\texttt{\$.} Schlüsselwort}\ in der Datenbasis platzieren, um Metamath\index{Metamath} mitzuteilen, dass der Beweis unbekannt ist, und das wäre Ihr Ausgangspunkt.  Wann immer der Befehl \texttt{verify proof} auf einen Beweis mit einem \texttt{?}\ anstelle eines Beweisschritts trifft, wird die Aussage als nicht bewiesen gekennzeichnet.

Als ich anfing, Metamath-Beweise zu erstellen, habe ich mir auf ein Stück Papier den vollständigen formalen Beweis, wie er mit einem \texttt{show proof} Befehl\index{\texttt{show proof}-Befehl} ausgegeben würde (siehe die Anzeige von \texttt{show proof th1 /lemmon /re\-num\-ber} oben als ein Beispiel).  Nachdem Sie sich an den Umgang mit dem Beweis-Assistenten\index{Beweis-Assistent} gewöhnt haben, können Sie den Beweis in Ihrem Kopf "`sehen"' und sich beim Ausfüllen der Details vom Beweis-Assistenten leiten lassen, zumindest bei einfacheren Beweisen. Aber bis Sie diese Erfahrung gesammelt haben, können es für die sehr hilfreich sein alle Details vorher aufzuschreiben.
Andernfalls könnten viel Zeit verschwendet werden, wenn Sie sich vom Assistenten auf einen falschen Weg führen lassen.
Andere finden diesen Ansatz jedoch nicht so hilfreich.
Zum Beispiel findet Thomas Brendan Leahy\index{Leahy, Thomas Brendan}, dass es für ihn hilfreicher ist, interaktiv rückwärts von einer maschinenlesbaren Aussage zu arbeiten.
David A. Wheeler\index{Wheeler, David A.} schreibt sich zuerst einen allgemeinen Ansatz auf, entwickelt den Beweis aber interaktiv durch Umschalten zwischen
vorwärts (ausgehend von Hypothesen und Fakten, die nützlich sein könnten) und rückwärts (vom Ziel ausgehend), bis sich die vorwärts- und rückwärtsgerichteten Ansätze treffen.
Am Ende sollten Sie den Ansatz wählen, der für Sie am besten geeignet ist.

Ein Beweis wird mit dem Beweis-Assistenten entwickelt, indem rückwärts gearbeitet wird, beginnend mit dem zu beweisenden Theorem\index{Theorem}. Danach wird jeder unbekannte Schritt einem Theorem oder einer Hypothese zugeordnet, bis keine unbekannten Schritte mehr übrig bleiben.  Der Beweis-Assistent lässt Sie nur dann eine Zuordnung vornehmen, wenn diese mit dem unbekannten Schritt "`vereinheitlicht"' werden kann.  Das bedeutet, dass eine Substitution\index{Substitution!Variable}\index{Variablensubstitution} von Variablen existiert, die die Zuordnung mit dem unbekannten Schritt übereinstimmen lässt.  Andererseits ist in der Mitte eines Beweises, wenn man rückwärts arbeitet, oft mehr als eine Vereinheitlichung\index{Vereinheitlichung} (Menge von Substitutionen) möglich, da zu diesem Zeitpunkt nicht genügend Informationen vorhanden sind, um sie eindeutig zu bestimmen.  In diesem Fall kann man Metamath mitteilen, welche Vereinheitlichung zu wählen ist, oder man kann weiterhin unbekannte Schritte zuweisen, bis genügend Informationen verfügbar sind, um die Vereinheitlichung eindeutig zu machen.

Wir gehen davon aus, dass Sie Metamath gestartet und die Datenbasis wie oben beschrieben eingelesen haben.  Der folgende Dialog zeigt, wie der Beweis erstellt wird.  Weitere Einzelheiten zu den Funktionen einiger Befehle finden Sie in Abschnitt~\ref{pfcommands}. \index{\texttt{prove}-Befehl}

\begin{verbatim}
MM> prove th1
Entering the Proof Assistant.  Type HELP for help, EXIT
to exit.  You will be working on the proof of statement th1:
  $p |- t = t
Note:  The proof you are starting with is already complete.
MM-PA>
\end{verbatim}

Die Eingabeaufforderung \verb/MM-PA>/ bedeutet, dass wir uns innerhalb des Beweis-Assistenten\index{Beweis-Assistent} befinden. Die meisten der regulären Metamath-Befehle (\texttt{show statement}, etc.) sind weiterhin verfügbar, falls Sie sie benötigen.

\begin{verbatim}
MM-PA> delete all
The entire proof was deleted.
\end{verbatim}

Wir haben den gesamten Beweis gelöscht, damit wir von vorne beginnen können.

\begin{verbatim}
MM-PA> show new_proof/lemmon/all
1 ?              $? |- t = t
\end{verbatim}

Der Befehl \texttt{show new{\char`\_}proof}\index{\texttt{show new{\char`\_}proof}-Befehl} verhält sich wie \texttt{show proof}, außer dass wir keine Aussage angeben; stattdessen wird der Beweis, an dem wir gerade arbeiten, angezeigt.

\begin{verbatim}
MM-PA> assign 1 mp
To undo the assignment, DELETE STEP 5 and INITIALIZE, UNIFY
if needed.
3   min=?  $? |- $2
4   maj=?  $? |- ( $2 -> t = t )
\end{verbatim}

Der obige \texttt{assign}-Befehl\index{\texttt{assign}-Befehl} bedeutet "`Weise die Aussage mit der Bezeichnung \texttt{mp} dem Schritt 1 zu."'  Beachten Sie, dass die Schritte ständig neu nummeriert werden, wenn Sie Schritte in der Mitte eines Beweises zuweisen; im Allgemeinen werden alle Schritte von dem Schritt, den Sie zuweisen, bis zum Ende des Beweises nach oben verschoben.  In diesem Fall ist der frühere Schritt 1 jetzt Schritt 5, weil der (Teil-)Beweis jetzt fünf Schritte hat: die vier Hypothesen der \texttt{mp}-Aussage und die \texttt{mp} Aussage selbst.  Schauen wir uns alle Schritte in unserem Teilbeweis an:

\begin{verbatim}
MM-PA> show new_proof/lemmon/all
1 ?              $? wff $2
2 ?              $? wff t = t
3 ?              $? |- $2
4 ?              $? |- ( $2 -> t = t )
5 1,2,3,4 mp     $a |- t = t
\end{verbatim}

Das Symbol \texttt{\$2} ist eine temporäre Variable\index{temporäre Variable}, die eine noch nicht bekannte Symbolfolge darstellt.  In dem endgültigen Beweis müssen alle temporären Variablen eliminiert sein.  Das allgemeine Format für eine temporäre Variable ist \texttt{\$} gefolgt von einer ganzen Zahl.  Beachten Sie, dass \texttt{\$} kein zulässiges Zeichen in einem mathematischen Symbol ist (siehe Abschnitt~\ref{dollardollar}, S.~\pageref{dollardollar}), also wird es niemals einen Namenskonflikt zwischen realen Symbolen und temporären Variablen geben.

Die unbekannten Schritte 1 und 2 sind Konstruktionen der beiden wffs, die von dem Modus ponens verwendet werden.  Wie Sie am Ende dieses Abschnitts sehen werden, kann der Beweis-Assistent\index{Beweis-Assistent} diese Schritte normalerweise selbst herausfinden, und wir müssen uns nicht um sie kümmern.  Deshalb werden wir von hier an nur noch die "`essentiellen"' Hypothesen anzeigen, d.h. die Schritte, die den traditionellen formalen Beweisen\index{formaler Beweis} entsprechen.

\begin{verbatim}
MM-PA> show new_proof/lemmon
3 ?              $? |- $2
4 ?              $? |- ( $2 -> t = t )
5 3,4 mp         $a |- t = t
\end{verbatim}

Die unbekannten Schritte 3 und 4 sind die, auf die wir uns konzentrieren müssen.  Sie entsprechen den Neben- und Hauptprämissen des Modus ponens.  Wir werden sie wie folgt zuordnen.  Beachten Sie, dass es wegen der Neunummerierung der Schritte nach einer Zuordnung vorteilhaft ist, die unbekannten Schritte in umgekehrter Reihenfolge zuzuordnen, da frühere Schritte nicht neu nummeriert werden.

\begin{verbatim}
MM-PA> assign 4 mp
To undo the assignment, DELETE STEP 8 and INITIALIZE, UNIFY
if needed.
3   min=?  $? |- $2
6     min=?  $? |- $4
7     maj=?  $? |- ( $4 -> ( $2 -> t = t ) )
\end{verbatim}

Wir werden jetzt eine obskure Funktionalität beschreiben, die Sie wahr\-schein\-lich nie benutzen werden, die Sie aber kennen sollten.  Die Metamath-Sprache erlaubt es, dass Variable durch leere Symbolsequenzen ersetzt werden können, aber in den meisten formalen Systemen wird dies nie eingesetzt.  Eines der wenigen Beispiele, in denen sie verwendet wird, ist das MIU-System\index{MIU-System}, das in Anhang~\ref{MIU} beschrieben wird.  Aber solche Systeme sind selten, und standardmäßig ist diese Funktion im Beweis-Assistenten ausgeschaltet (sie ist immer erlaubt für {\tt verify proof}).  Schalten wir sie ein und schauen, was was passiert.\index{\texttt{set empty{\char`\_}substitution}-Befehl}

\begin{verbatim}
MM-PA> set empty_substitution on
Substitutions with empty symbol sequences is now allowed.
\end{verbatim}

Wenn diese Funktionalität aktiviert ist, werden mehr Vereinheitlichungen in der Mitte eines Beweises mehrdeutig\index{mehrdeutige Vereinheitlichung}\index{Vereinheitlichung!mehrdeutig} sein, weil die Substitution\index{Substitution!Variable}\index{Variablensubstitution} von Variablen mit leeren Symbolfolgen eine zusätzliche Möglichkeit darstellt.  Schauen wir uns an, was passiert, wenn wir unsere nächste Zuweisung vornehmen.

\begin{verbatim}
MM-PA> assign 3 a2
There are 2 possible unifications.  Please select the correct
    one or Q if you want to UNIFY later.
Unify:  |- $6
 with:  |- ( $9 + 0 ) = $9
Unification #1 of 2 (weight = 7):
  Replace "$6" with "( + 0 ) ="
  Replace "$9" with ""
  Accept (A), reject (R), or quit (Q) <A>? r
\end{verbatim}

Die erste der vorgestellten Möglichkeiten ist die falsche.  Hätten wir sie gewählt, wäre der temporären Variablen \texttt{\$6} eine abgeschnittene wff, und der temporären Variablen \texttt{\$9} eine leere Folge zugewiesen worden (was in unserem System nicht zulässig ist).  Bei dieser Wahl kämen wir irgendwann an einen Punkt, an dem wir nicht mehr weiterkämen, weil wir mit Schritten enden würden, die nicht mehr zu beweisen sind.  (Probieren Sie es aus.) Wir haben \texttt{r} eingegeben, um die Wahl zu verwerfen.

\begin{verbatim}
Unification #2 of 2 (weight = 21):
  Replace "$6" with "( $9 + 0 ) = $9"
  Accept (A), reject (R), or quit (Q) <A>? q
To undo the assignment, DELETE STEP 4 and INITIALIZE, UNIFY
if needed.
 7     min=?  $? |- $8
 8     maj=?  $? |- ( $8 -> ( $6 -> t = t ) )
\end{verbatim}

Die zweite Wahlmöglichkeit ist richtig, und normalerweise würden wir \texttt{a} eingeben, um sie zu akzeptieren.  Stattdessen haben wir \texttt{q} eingegeben, um zu zeigen, was passieren wird: Der Schritt wird mit einer unbekannten Vereinheitlichung verlassen, die wie folgt aussehen kann:

\begin{verbatim}
MM-PA> show new_proof/not_unified
 4   min    $a |- $6
        =a2  = |- ( $9 + 0 ) = $9
\end{verbatim}

Später können wir diesen Schritt mit dem Befehl \texttt{unify} \texttt{all/interactive} vereinheitlichen.

Es ist wichtig, sich daran zu erinnern, dass Ihnen bei der Eingabe eines Beweises gelegentlich mehrere Vereinheitlichungsmöglichkeiten angeboten werden, wenn das Programm feststellt, dass noch nicht genügend Informationen vorhanden sind, um automatisch eine eindeutige Wahl zu treffen (und das kann sogar bei ausgeschalteter \texttt{set empty{\char`\_}substitution} passieren).  Normalerweise ist es während der Inspektion der Auswahlmöglichkeiten offensichtlich, welche Wahl die richtige ist, da die falsche Wahl zu sinnlosen Fragmenten von wffs führt.  Außerdem ist die richtige Wahl normalerweise die erste, die präsentiert wird, im Gegensatz zu unserem obigen Beispiel.

Genug der Abschweifung.  Kehren wir zur Standardeinstellung zurück.

\begin{verbatim}
MM-PA> set empty_substitution off
The ability to substitute empty expressions for variables
has been turned off.  Note that this may make the Proof
Assistant too restrictive in some cases.
\end{verbatim}

Wenn wir den Beweis löschen, neu beginnen und zu dem Punkt gelangen, an dem wir oben abgeschweift sind, gibt es keine mehrdeutige Vereinheitlichung mehr.

\begin{verbatim}
MM-PA> assign 3 a2
To undo the assignment, DELETE STEP 4 and INITIALIZE, UNIFY
if needed.
 7     min=?  $? |- $4
 8     maj=?  $? |- ( $4 -> ( ( $5 + 0 ) = $5 -> t = t ) )
\end{verbatim}

Schauen wir uns unseren bisherigen Beweis an und fahren wir fort.

\begin{verbatim}
MM-PA> show new_proof/lemmon
 4 a2            $a |- ( $5 + 0 ) = $5
 7 ?             $? |- $4
 8 ?             $? |- ( $4 -> ( ( $5 + 0 ) = $5 -> t = t ) )
 9 7,8 mp        $a |- ( ( $5 + 0 ) = $5 -> t = t )
10 4,9 mp        $a |- t = t
MM-PA> assign 8 a1
To undo the assignment, DELETE STEP 11 and INITIALIZE, UNIFY
if needed.
 7     min=?  $? |- ( t + 0 ) = t
MM-PA> assign 7 a2
To undo the assignment, DELETE STEP 8 and INITIALIZE, UNIFY
if needed.
MM-PA> show new_proof/lemmon
 4 a2            $a |- ( t + 0 ) = t
 8 a2            $a |- ( t + 0 ) = t
12 a1            $a |- ( ( t + 0 ) = t -> ( ( t + 0 ) = t ->
                                                    t = t ) )
13 8,12 mp       $a |- ( ( t + 0 ) = t -> t = t )
14 4,13 mp       $a |- t = t
\end{verbatim}

Nun sind alle temporären Variablen und unbekannten Schritte aus dem "`wesentlichen"' Teil des Beweises entfernt worden.  Wenn dieser Zustand erreicht ist, kann der Beweis-Assistent\index{Beweis-Assistent} den Rest des Beweises in der Regel automatisch bestimmen (angestoßen durch den Befehl \texttt{improve} - beachten Sie, dass der Befehl \texttt{improve} gelegentlich auch zum Ausfüllen wesentlicher Schritte genutzt werden kann. Es wird dabei aber nur versucht Aussagen zu verwenden, die keine neuen Variablen in ihren Hypothesen einführen, was bei \texttt{mp} nicht der Fall ist. Es wird auch nicht versucht, Schritte zu verbessern, die temporäre Variablen enthalten).  Schauen wir uns den vollständigen Beweis nochmals an, führen dann den Befehl \texttt{improve} aus und sehen ihn uns dann noch einmal an.

\begin{verbatim}
MM-PA> show new_proof/lemmon/all
 1 ?             $? wff ( t + 0 ) = t
 2 ?             $? wff t = t
 3 ?             $? term t
 4 3 a2          $a |- ( t + 0 ) = t
 5 ?             $? wff ( t + 0 ) = t
 6 ?             $? wff ( ( t + 0 ) = t -> t = t )
 7 ?             $? term t
 8 7 a2          $a |- ( t + 0 ) = t
 9 ?             $? term ( t + 0 )
10 ?             $? term t
11 ?             $? term t
12 9,10,11 a1    $a |- ( ( t + 0 ) = t -> ( ( t + 0 ) = t ->
                                                    t = t ) )
13 5,6,8,12 mp   $a |- ( ( t + 0 ) = t -> t = t )
14 1,2,4,13 mp   $a |- t = t
\end{verbatim}

\begin{verbatim}
MM-PA> improve all
A proof of length 1 was found for step 11.
A proof of length 1 was found for step 10.
A proof of length 3 was found for step 9.
A proof of length 1 was found for step 7.
A proof of length 9 was found for step 6.
A proof of length 5 was found for step 5.
A proof of length 1 was found for step 3.
A proof of length 3 was found for step 2.
A proof of length 5 was found for step 1.
Steps 1 and above have been renumbered.
CONGRATULATIONS!  The proof is complete.  Use SAVE
NEW_PROOF to save it.  Note:  The Proof Assistant does
not detect $d violations.  After saving the proof, you
should verify it with VERIFY PROOF.
\end{verbatim}

Der \texttt{save new{\char`\_}proof}\index{\texttt{save new{\char`\_}proof}-Befehl} speichert den Beweis in der Datenbasis.  Hier wird er nur in einer Form angezeigt, die aus einer Protokolldatei ausgeschnitten und mit einem Texteditor manuell in die Quelldatei der Datenbasis eingefügt werden kann.\index{normaler Beweis}\index{Beweis!normal}

\begin{verbatim}
MM-PA> show new_proof/normal
---------Clip out the proof below this line:
      tt tze tpl tt weq tt tt weq tt a2 tt tze tpl tt weq
      tt tze tpl tt weq tt tt weq wim tt a2 tt tze tpl tt
      tt a1 mp mp $.
---------The proof of 'th1' to clip out ends above this line.
\end{verbatim}

Es gibt ein weiteres Beweis-Format, der "`komprimierte"' Beweis\index{komprimierter Beweis}\index{Beweis!komprimiert}, das Sie in Datenbasen sehen werden.  Es ist nicht wichtig zu verstehen, wie es kodiert ist, sondern nur es zu erkennen, wenn Sie es sehen.  Sein einziger Zweck ist es, den Speicherbedarf für große Beweise zu verringern.  Ein komprimierter Beweis kann immer in einen normalen umgewandelt werden und umgekehrt, und der Metamath-Befehl \texttt{show proof} funktioniert genauso gut mit komprimierten Beweisen.  Das komprimierte Beweisformat wird im Anhang~\ref{compressed} beschrieben.

\begin{verbatim}
MM-PA> show new_proof/compressed
---------Clip out the proof below this line:
      ( tze tpl weq a2 wim a1 mp ) ABCZADZAADZAEZJJKFLIA
      AGHH $.
---------The proof of 'th1' to clip out ends above this line.
\end{verbatim}

Nun beenden wir den Beweis-Assistenten.  Da wir Änderungen an dem Beweis vorgenommen haben, werden wir gewarnt, dass wir ihn nicht gespeichert haben.  In diesem Fall ist uns das egal.

\begin{verbatim}
MM-PA> exit
Warning:  You have not saved changes to the proof.
Do you want to EXIT anyway (Y, N) <N>? y
Exiting the Proof Assistant.
Type EXIT again to exit Metamath.
\end{verbatim}

Der Beweis-Assistent verfügt über verschiedene andere Befehle, die Ihnen bei der Erstellung von Beweisen helfen können.  Siehe Abschnitt~\ref{pfcommands} für eine Liste dieser Befehle.

Ein oft nützlicher Befehl ist \texttt{minimize{\char`\_}with*/brief}, der versucht, den Beweis zu verkürzen.  Er kann den Prozess des Beweisens effizienter machen, indem er Sie einen etwas "`schlampigen"' Beweis schreiben lässt und ihn dann durch einige feine Optimierungsdetails für Sie bereinigt (obwohl er keine Wunder vollbringen kann, wie z.B. die Umstrukturierung des gesamten Beweises).

\section{Hinweise zur Bearbeitung einer Daten\-basis}

Sobald die Quelldatei ihrer Datenbasis Beweise enthält, gibt es einige Einschränkungen für deren Bearbeitung, damit die Beweise gültig bleiben.
Diese Regeln sollten Sie besonders beachten, da Sie sonst mühsam erzielte Ergebnisse verlieren können. Es ist sinnvoll, alle Beweise regelmäßig mit \texttt{verify proof *} zu überprüfen, um ihre Integrität sicherzustellen.

Wenn Ihre Datei nur normale (im Gegensatz zu komprimierten) Beweise enthält, besteht die Hauptregel darin, dass Sie die Reihenfolge der obligatorischen Hypothesen\index{obligatorische Hypothese} im Index einer Aussage, auf die in einem späteren Beweis verwiesen wird, nicht ändern dürfen.  Wenn Sie zum Beispiel die Reihenfolge der Haupt- und Nebenhypothese in dem Modus ponens vertauschen, werden alle Beweise, die diese Regel verwenden, falsch.  Der Befehl \texttt{show statement} \index{\texttt{show statement}-Befehl} zeigt Ihnen die obligatorischen Hypothesen einer Aussage und ihre Reihenfolge.

Wenn eine Aussage einen komprimierten Beweis hat, dürfen Sie auch nicht die Reihenfolge {\em ihrer} obligatorischen Hypothesen ändern.
Das komprimierte Beweisformat verwendet diese Information als Teil der Komprimierungstechnik. Beachten Sie, dass das Vertauschen der Namen zweier Variablen in einem Theorem die Reihenfolge der zwingenden Hypothesen ändert.

Der sicherste Weg, eine Aussage, z. B. \texttt{mytheorem}, zu bearbeiten besteht darin, sie zu duplizieren und das Original in der gesamten Datenbasis in \texttt{mytheoremOLD} umzubenennen.
Sobald die bearbeitete Version erneut bewiesen ist, können alle Aussagen, die auf \texttt{mytheoremOLD} verweisen, im Beweis-Assistenten mit dem Befehl \texttt{minimize{\char`\_}with mytheorem/allow{\char`\_}growth}\index{\texttt{minimize{\char`\_}with}-Befehl} aktualisiert werden.
% 3/10/07 Note: line-breaking the above results in duplicate index entries

\chapter{Abstrakte Mathematik enthüllt}\label{fol}

\section{Logik und Mengenlehre}\label{logicandsettheory}

\begin{quote}
  {\em Die Mengenlehre kann als eine Form einer exakten Theologie betrachtet werden.}
  \flushright\sc  Rudy Rucker\footnote{Frei übersetzt nach \cite{Barrow}, S.~31.}\\
\end{quote}\index{Rucker, Rudy}

Trotz ihrer vermeintlichen Komplexität lässt sich die gesamte Standardmathematik, egal wie tief oder abstrakt sie ist, erstaunlicherweise aus einem relativ kleinen Satz von Axiomen\index{Axiom} oder ersten Prinzipien ableiten. Die Entwicklung dieser Axiome gehört zu den beeindruckendsten und wichtigsten Errungenschaften der Mathematik im 20. Jahrhundert. Letztlich lassen sich diese Axiome in eine Reihe von Regeln für die Handhabung von Symbolen herunterbrechen, denen jeder technisch orientierte Mensch folgen kann.

Wir werden nicht viel Zeit darauf verwenden, ein tiefes, übergeordnetes Verständnis der Bedeutung der Axiome zu vermitteln. Diese Art von Verständnis erfordert ein gewisses Maß an mathematischer Raffinesse sowie ein Verständnis der Philosophie, die den Grundlagen der Mathematik zugrunde liegt, und entwickelt sich in der Regel im Laufe der Zeit, wenn Sie mit Mathematik arbeiten.  Unser Ziel ist es stattdessen, Ihnen die unmittelbare Fähigkeit zu vermitteln zu verstehen, wie Theoreme\index{Theorem} aus den Axiomen und aus anderen Theoremen abgeleitet werden.  Dies ist vergleichbar mit dem Erlernen der Syntax einer Computersprache, die es Ihnen ermöglicht, die Details eines Programms zu verstehen, Ihnen aber nicht unbedingt die Fähigkeit verleiht, nicht-triviale Programme selbst zu schreiben - eine Fähigkeit, die sich erst mit der Zeit entwickelt. Lassen Sie sich vorerst nicht von den abstrakt klingenden Namen der Axiome beunruhigen, sondern konzentrieren Sie sich auf die Regeln zur Manipulation der Symbole, die den einfachen Konventionen der Metamath\index{Metamath}-Sprache folgen.

Die Axiome, die der gesamten Standardmathematik zugrunde liegen, bestehen aus Axiomen der Logik und Axiomen der Mengenlehre. Die Axiome der Logik sind in zwei Unterkategorien unterteilt, die Aussagenlogik\index{Aussagenlogik} (manchmal auch Satzlogik\index{Satzlogik} genannt) und die Prädikatenlogik (manchmal auch Logik erster Ordnung\index{Logik erster Ordnung}\index{Quantifizierungstheorie}\index{Prädikatenlogik} oder Quantentheorie genannt).  Die Aussagenlogik ist eine Voraussetzung für die Prädikatenlogik, und die Prädikatenlogik ist eine Voraussetzung für die Mengenlehre.  Die am häufigsten verwendete Version der Mengenlehre ist die Zermelo--Fraenkel-Mengenlehre\index{Zermelo--Fraenkel-Mengenlehre}\index{Mengenlehre} mit dem Auswahlaxiom (engl. "`axiom of choice"'), oft abgekürzt als ZFC\index{ZFC}.

Hier ist in aller Kürze dargestellt, worum es bei den Axiomen geht, und zwar auf informelle Art und Weise. Die Verbindung zwischen dieser Beschreibung und den Symbolen, die wir Ihnen zeigen werden, wird nicht sofort offensichtlich sein und muss es im Prinzip auch nicht.  Unsere Beschreibung versucht lediglich zusammenzufassen, worüber Mathematiker nachdenken, wenn sie mit den Axiomen arbeiten.

Logik ist eine Reihe von Regeln, die es uns ermöglichen, Wahrheiten aus anderen Wahrheiten abzuleiten. Anders ausgedrückt, ist Logik mehr oder weniger die Übersetzung dessen, was wir als gesunden Menschenverstand betrachten würden, in einen strengen Satz von Axiomen.\index{Axiome der Logik}  Angenommen, $\varphi$, $\psi$ und $\chi$ (die griechischen Buchstaben phi, psi und chi) stehen für Aussagen, die entweder wahr oder falsch sind, und $x$ ist eine Variable\index{Variable!in der Prädikatenlogik}, die sich über eine Gruppe mathematischer Objekte (Mengen, ganze Zahlen, reelle Zahlen usw.) erstreckt. In der Mathematik ist eine "`Aussage"' eigentlich eine Formel, und $\psi$ könnte z.B. "`$x = 2$"' sein. Die Aussagenlogik\index{Aussagenlogik} erlaubt uns, Variablen zu verwenden, die entweder wahr oder falsch sind, und Schlüsse zu ziehen wie:
"`Wenn $\psi$ aus $\varphi$ und $\chi$  aus $\psi$ folgt, dann folgt $\chi$ aus $\varphi$."' Die Prädikatenlogik\index{Prädikatenlogik} erweitert die Aussagenlogik, indem sie auch Aussagen über Objekte (nicht nur über Wahrheitswerte) erlaubt, einschließlich Aussagen über "`alle"' Objekte oder "`wenigstens ein"' Objekt. Die Prädikatenlogik erlaubt es zum Beispiel zu sagen: "`Wenn $\varphi$ für alle $x$ wahr ist, dann ist $\varphi$ für einige $x$ wahr."' Die in \texttt{set.mm} verwendete Logik ist die klassische Standardlogik (im Gegensatz zu anderen Logiksystemen wie der intuitionistischen Logik).

Die Mengenlehre\index{Mengenlehre} befasst sich mit der Handhabung von Objekten und Sammlungen von Objekten, insbesondere mit den abstrakten, imaginären Objekten, mit denen sich die Mathematik beschäftigt, wie z. B. Zahlen. Alles, was es in der Mathematik geben soll, wird als Menge betrachtet.  Eine Menge, die als leere Menge\index{leere Menge} bezeichnet wird, enthält nichts.  Wir stellen die leere Menge durch $\varnothing$ dar.  Viele Mengen können aus der leeren Menge aufgebaut werden.  Es gibt eine durch $\{\varnothing\}$ dargestellte Menge, die die leere Menge enthält, eine weitere durch $\{\varnothing,\{\varnothing\}\}$ dargestellte Menge, die diese Menge sowie die leere Menge enthält, eine weitere durch $\{\{\varnothing\}\}$ dargestellte Menge, die nur die Menge enthält, die die leere Menge enthält, und so weiter ad infinitum. Alle mathematischen Objekte, egal wie komplex sie sind, werden als identisch mit bestimmten Mengen definiert: Die ganze Zahl\index{ganze Zahl} 0 ist definiert als die leere Menge, die ganze Zahl 1 ist definiert als $\{\varnothing\}$, die ganze Zahl 2 ist definiert als $\{\varnothing,\{\varnothing\}\}$.  (Wie diese Definitionen gewählt wurden, spielt jetzt keine Rolle: die Idee dahinter ist, dass diese Mengen die Eigenschaften haben, die wir von ganzen Zahlen erwarten, sobald geeignete Operationen definiert sind.)  Mathematische Operationen, wie z. B. die Addition, werden in Form von Operationen auf Mengen definiert - ihre Vereinigung\index{Vereinigungsmenge}, ihre Schnittmenge\index{Schnittmenge} usw. - Operationen, die Sie vielleicht schon in der Grundschule verwendet haben, als Sie mit Gruppen von Äpfeln und Orangen gearbeitet haben.

Die Axiome postulieren auch die Existenz unendlicher Mengen\index{unendliche Menge}, wie z.B. die Menge aller nichtnegativen ganzen Zahlen ($0, 1,2,\ldots$, auch "`natürliche Zahlen"'\index{natürliche Zahlen} genannt).  Diese Menge kann nicht mit der soeben gezeigten Klammerschreibweise\index{Klammerschreibweise} dargestellt werden, sondern erfordert eine kompliziertere Schreibweise, die "`Klassenabstraktion"'\index{Klassenabstraktion}\index{Abstraktionsklasse}.  Zum Beispiel bedeutet die unendliche Menge $\{ x |\mbox{"`$x$ ist eine natürliche Zahl"'} \} $ die "`Menge aller Objekte $x$, so dass $x$ eine natürliche Zahl ist"', d.h. die Menge der natürlichen Zahlen; dabei ist "`$x$ eine natürliche Zahl"' eine ziemlich komplizierte Formel, wenn man sie mit den primitiven Symbole ausdrückt. \label{expandom}\footnote{Die Aussage "`$x$ ist eine natürliche Zahl"' wird formal ausgedrückt als "`$x \in \omega$"', wobei $\in$ (stilisiertes Epsilon) bedeutet "`ist in"' oder "`ist ein Element von"' und $\omega$ (omega) bedeutet "`die Menge der natürlichen Zahlen"'.  Wenn "`$x\in\omega$"' vollständig durch die primitiven Symbole der Mengenlehre ausgedrückt wird, ist das Ergebnis $\lnot$ $($ $\lnot$ $($ $\forall$ $z$ $($ $\lnot$ $\forall$ $w$ $($$z$ $\in$ $w$ $\rightarrow$ $\lnot$ $w$ $\in$ $x$ $)$ $\rightarrow$ $z$ $\in$$x$ $)$ $\rightarrow$ $($ $\forall$ $z$ $($ $\lnot$ $($ $\forall$ $w$ $($ $w$$\in$ $z$ $\rightarrow$ $w$ $\in$ $x$ $)$ $\rightarrow$ $\forall$ $w$ $\lnot$$w$ $\in$ $z$ $)$ $\rightarrow$ $\lnot$ $\forall$ $w$ $($ $w$ $\in$ $z$$\rightarrow$ $\lnot$ $\forall$ $v$ $($ $v$ $\in$ $z$ $\rightarrow$ $\lnot$$v$ $\in$ $w$ $)$ $)$ $)$ $\rightarrow$ $\lnot$ $\forall$ $z$ $\forall$ $w$$($ $\lnot$ $($ $z$ $\in$ $x$ $\rightarrow$ $\lnot$ $w$ $\in$ $x$ $)$$\rightarrow$ $($ $\lnot$ $z$ $\in$ $w$ $\rightarrow$ $($ $\lnot$ $z$ $=$ $w$$\rightarrow$ $w$ $\in$ $z$ $)$ $)$ $)$ $)$ $)$ $\rightarrow$ $\lnot$$\forall$ $y$ $($ $\lnot$ $($ $\lnot$ $($ $\forall$ $z$ $($ $\lnot$ $\forall$$w$ $($ $z$ $\in$ $w$ $\rightarrow$ $\lnot$ $w$ $\in$ $y$ $)$ $\rightarrow$$z$ $\in$ $y$ $)$ $\rightarrow$ $($ $\forall$ $z$ $($ $\lnot$ $($ $\forall$$w$ $($ $w$ $\in$ $z$ $\rightarrow$ $w$ $\in$ $y$ $)$ $\rightarrow$ $\forall$$w$ $\lnot$ $w$ $\in$ $z$ $)$ $\rightarrow$ $\lnot$ $\forall$ $w$ $($ $w$$\in$ $z$ $\rightarrow$ $\lnot$ $\forall$ $v$ $($ $v$ $\in$ $z$ $\rightarrow$$\lnot$ $v$ $\in$ $w$ $)$ $)$ $)$ $\rightarrow$ $\lnot$ $\forall$ $z$$\forall$ $w$ $($ $\lnot$ $($ $z$ $\in$ $y$ $\rightarrow$ $\lnot$ $w$ $\in$$y$ $)$ $\rightarrow$ $($ $\lnot$ $z$ $\in$ $w$ $\rightarrow$ $($ $\lnot$ $z$$=$ $w$ $\rightarrow$ $w$ $\in$ $z$ $)$ $)$ $)$ $)$ $\rightarrow$ $($$\forall$ $z$ $\lnot$ $z$ $\in$ $y$ $\rightarrow$ $\lnot$ $\forall$ $w$ $($$\lnot$ $($ $w$ $\in$ $y$ $\rightarrow$ $\lnot$ $\forall$ $z$ $($ $w$ $\in$$z$ $\rightarrow$ $\lnot$ $z$ $\in$ $y$ $)$ $)$ $\rightarrow$ $\lnot$ $($$\lnot$ $\forall$ $z$ $($ $w$ $\in$ $z$ $\rightarrow$ $\lnot$ $z$ $\in$ $y$$)$ $\rightarrow$ $w$ $\in$ $y$ $)$ $)$ $)$ $)$ $\rightarrow$ $x$ $\in$ $y$$)$ $)$ $)$. Abschnitt~\ref{hierarchy} zeigt die Hierarchie der Definitionen, die zu diesem Ausdruck führt.}\index{stilisiertes Epsilon ($\in$)}\index{Omega ($\omega$)}  Die primitiven Symbole enthalten tatsächlich noch nicht einmal die Klammerschreibweise.  Die Klammerschreibweise ist eine übergeordnete Definition, die Sie im Abschnitt~\ref{hierarchy} finden können.

Interessanterweise können die Arithmetik der ganzen Zahlen\index{ganze Zahl} und die Arithmetik der rationalen Zahlen\index{rationale Zahl} entwickelt werden, ohne sich auf die Existenz einer unendlichen Menge zu berufen, während die Arithmetik der reellen Zahlen\index{reelle Zahl} dies erfordert.

Jede Variable\index{Variable!in der Mengenlehre} in den Axiomen der Mengenlehre stellt eine beliebige Menge dar, und die Axiome geben auf einer sehr primitiven Ebene die zulässigen Möglichkeiten an, die man mit diesen Variablen tun kann.

Sie denken jetzt vielleicht, dass Zahlen und Arithmetik viel intuitiver und grundlegender sind als Mengen und daher die Grundlage der Mathematik sein sollten. Aber in Wirklichkeit haben Sie Ihr ganzes Leben lang mit Zahlen zu tun gehabt und sind mit einigen Regeln für ihre Handhabung vertraut, wie z. B. Addition und Multiplikation.  Diese Regeln decken nur einen kleinen Teil dessen ab, was man mit Zahlen tun kann, und nur einen winzigen Teil der übrigen Mathematik.  Wenn Sie sich ein beliebiges Buch über Zahlentheorie ansehen, werden Sie schnell verloren sein, wenn dies die einzigen Regeln sind, die Sie kennen.  Auch wenn solche Bücher eine Liste von "`Axiomen"'\index{Axiom} für die Arithmetik enthalten, erfordert die Fähigkeit, die Axiome zu verwenden und Beweise von Theoremen\index{Theorem} (Fakten) über Zahlen zu verstehen - ein implizites mathematisches Talent, das viele Menschen davon abhält, abstrakte Mathematik zu studieren.  Die Art von Mathematik, die die meisten Menschen kennen, beschränkt sie auf den praktischen, alltäglichen Gebrauch der blinden Manipulation von Zahlen und Formeln, ohne dass sie verstehen, warum diese Regeln richtig sind, und ohne weiter darüber nachzudenken.  Wissen Sie zum Beispiel, warum die Multiplikation von zwei negativen Zahlen eine positive Zahl ergibt?  Wenn Sie mit der Mengenlehre beginnen, werden Sie ebenfalls damit beginnen, Symbole blind nach den von uns vorgegebenen Regeln zu manipulieren, allerdings mit dem Vorteil, dass diese Regeln Ihnen im Prinzip den Zugang zur {\em gesamten} Mathematik ermöglichen, nicht nur zu einem winzigen Teil davon.

Natürlich sind konkrete Beispiele oft hilfreich für den Lernprozess. So kann man zum Beispiel überprüfen, dass $2\cdot 3=3 \cdot 2$ ist, indem man Objekte gruppiert, und man kann leicht "`sehen"', wie dies zu $x\cdot y = y\cdot x$ verallgemeinert wird, auch wenn man nicht in der Lage ist, es rigoros zu beweisen.  In ähnlicher Weise kann es in der Mengenlehre hilfreich sein zu verstehen, wie die Axiome der Mengenlehre auf kleine endliche Sammlungen von Objekten anwendbar (und korrekt) sind.  Sie sollten sich darüber im Klaren sein, dass die Intuition in der Mengenlehre für unendliche Sammlungen irreführend sein kann, und dass deshalb strenge Beweise wichtiger werden.  Zum Beispiel ist $x\cdot y = y\cdot x$ zwar für endliche Ordinalzahlen (die natürlichen Zahlen) richtig, aber nicht für unendliche Ordinalzahlen.

\section{Die Axiome für die gesamte Mathematik}

In diesem Abschnitt\index{Axiome für die Mathematik} zeigen wir Ihnen die Axiome für die gesamte Standardmathematik (d.h.\ Logik und Mengenlehre), wie sie traditionell dargestellt werden.  Die traditionelle Darstellung ist nützlich für jemanden, der über die nötige mathematische Erfahrung verfügt, um abstrakte Konzepte auf hohem Niveau korrekt zu handhaben.  Für jemanden, der nicht über dieses Talent verfügt, kann es schwierig sein zu wissen, wie man diese Axiome tatsächlich anwendet.  Der Zweck dieses Abschnitts ist es, Ihnen zu zeigen, wie sich die Version der Axiome, die in der Standard-Metamath\index{Metamath}-Datenbasis \texttt{set.mm}\index{Mengenlehre-Datenbasis (\texttt{set.mm})} verwendet wird, zu der typischen Version in Lehrbüchern verhält, damit Sieein informelles Gespür dafür entwickeln.

\subsection{Aussagenlogik}

Die Aussagenlogik\index{Aussagenlogik} beschäftigt sich mit Aussagen, die entweder als wahr oder falsch interpretiert werden können.  Einige Beispiele für entweder wahre oder falsche Aussagen (außerhalb der Mathematik) sind "`Es regnet heute"' und "`Die Vereinigten Staaten haben einen weiblichen Präsidenten"'. In der Mathematik sind Aussagen, wie wir bereits erwähnt haben, eigentlich Formeln.

In der Aussagenlogik ist uns der Inhalt der Aussagen egal.  Auch eine logische Kombination von Aussagen, wie "`Es regnet heute und die Vereinigten Staaten haben eine weibliche Präsidentin"', wird nicht anders behandelt als eine Einzelaussage.  Aussagen und ihre Kombinationen werden wohlgeformte Formeln (wffs)\index{wohlgeformte Formel (wff)} genannt.  Wir definieren wffs nur in Bezug auf andere wffs und definieren nicht, was eine "`initiale"' wff ist.  Wie in der Literatur üblich, verwenden wir kleine  griechische Buchstaben zur Darstellung von wffs.

Nehmen wir konkret an, dass $\varphi$ und $\psi$ wffs sind.  Dann sind die Kombinationen $\varphi\rightarrow\psi$ ("`$\varphi$ impliziert $\psi$"', auch gelesen als "`wenn $\varphi$ dann $\psi$"' oder "`aus $\varphi$ folgt $\psi$"')\index{Implikation ($\rightarrow$)} und $\lnot\varphi$ ("`nicht $\varphi$"')\index{Negation ($\lnot$)} ebenfalls wffs.

Die drei Axiome der Aussagenlogik\index{Axiome der Aussagenlogik} sind alle wffs der folgenden Form:\footnote{Ein bemerkenswertes Ergebnis von C.~A.~Meredith\index{Meredith, C. A.} quetscht diese drei Axiome in ein einziges Axiom $((((\varphi\rightarrow \psi)\rightarrow(\neg \chi\rightarrow\neg\theta))\rightarrow \chi)\rightarrow \tau)\rightarrow((\tau\rightarrow\varphi)\rightarrow(\theta\rightarrow \varphi))$ \cite{CAMeredith}, das als das kürzest mögliche gilt. }
\begin{center}
     $\varphi\rightarrow(\psi\rightarrow \varphi)$\\

     $(\varphi\rightarrow (\psi\rightarrow \chi))\rightarrow
((\varphi\rightarrow  \psi)\rightarrow (\varphi\rightarrow \chi))$\\

     $(\neg \varphi\rightarrow \neg\psi)\rightarrow (\psi\rightarrow
\varphi)$
\end{center}

Diese drei Axiome sind weit verbreitet. Sie werden Jan {\L}ukasiewicz (sprich: wu-kah-schei-witsch) zugeschrieben und wurden von Alonzo Church veröffentlicht, der sie als System P2 bezeichnete. (Vielen Dank an Ted Ulrich für diese Information.)

Es gibt eine unendliche Anzahl von Axiomen, eines für jede mögliche wff\index{wohlgeformte Formel (wff)} der obigen Form.  (Aus diesem Grund werden Axiome wie die obigen oft als "`Axiomenschema"'\index{Axiomenschema}) bezeichnet.)  Jeder griechische Buchstabe in den Axiomen kann durch eine komplexeres wff ersetzt werden, um ein anderes Axiom zu erhalten.  Ersetzt man beispielsweise $\varphi$ im ersten Axiom durch $\neg(\varphi\rightarrow\chi)$, erhält man $\neg(\varphi\rightarrow\chi)\rightarrow(\psi\rightarrow\neg(\varphi\rightarrow\chi))$, was immer noch ein Axiom ist.

Um aus den Axiomen neue wahre Aussagen (Theoreme\index{Theorem}) abzuleiten, wird eine Schlussregel\index{Regel} namens "`Modus ponens"'\index{Modus ponens} verwendet.  Diese Regel besagt, dass, wenn die wff $\varphi$ ein Axiom oder ein Theorem ist, und die wff $\varphi\rightarrow\psi$ ein Axiom oder ein Theorem ist, dann ist die wff $\psi$ auch ein Theorem\index{Theorem}.

Ein nicht-mathematisches Beispiel für den Modus ponens: Nehmen wir an, wir haben bewiesen (oder als Axiom genommen): "`Bob ist ein Mann"', und haben separat bewiesen (oder als Axiom genommen): "`Wenn Bob ein Mann ist, dann ist Bob ein Mensch."'  Mit Hilfe des Modus ponens können wir logisch folgern: "`Bob ist ein Mensch."'

Aus der Sicht von Metamath\index{Metamath} definieren die Axiome und die Schlussregel Modus ponens lediglich ein mechanisches Mittel zur Ableitung neuer wahrer Aussagen aus bestehenden wahren Aussagen, und das ist der vollständige Inhalt der Aussagenlogik, soweit es Metamath betrifft.  Sie können ein Logik-Lehrbuch lesen, um ein besseres Verständnis ihrer Bedeutung zu erlangen, oder Sie können sich ihre Bedeutung einfach langsam erschließen, nachdem Sie sie eine Weile benutzt haben.

Es ist eigentlich recht einfach zu prüfen, ob eine Formel ein Satz (oder Theorem) der Aussagenlogik ist.  Theoreme der Aussagenlogik werden auch als "`Tautologien"'\index{Tautologie} bezeichnet.  Die Technik, mit der man überprüfen kann, ob eine Formel eine Tautologie ist, wird als "`Wahrheitstabellenmethode"'\index{Wahrheitstabelle} bezeichnet und funktioniert folgendermaßen.  Eine wff $\varphi\rightarrow\psi$ ist falsch, wenn $\varphi$ wahr und $\psi$ falsch ist.  Andernfalls ist sie wahr.  Eine wff $\lnot\varphi$ ist immer dann falsch, wenn $\varphi$ wahr ist und ansonsten falsch. Um eine Tautologie wie $\varphi\rightarrow(\psi\rightarrow \varphi)$ zu verifizieren, zerlegt man sie in Teil-wffs und konstruiert eine Wahrheitstabelle, die alle möglichen Kombinationen von wahr ($W$) und falsch ($F$) berücksichtigt, die den wff-Metavariablen zugeordnet sind:
\begin{center}\begin{tabular}{|c|c|c|c|}\hline
\mbox{$\varphi$} & \mbox{$\psi$} & \mbox{$\psi\rightarrow\varphi$}
    & \mbox{$\varphi\rightarrow(\psi\rightarrow \varphi)$} \\ \hline \hline
              W   &  W    &      W       &        W    \\ \hline
              W   &  F    &      W       &        W    \\ \hline
              F   &  W    &      F       &        W    \\ \hline
              F   &  F    &      W       &        W    \\ \hline
\end{tabular}\end{center}
Wenn alle Einträge in der letzten Spalte wahr ($W$) sind, ist die Formel eine Tautologie.

Die Wahrheitstabellen-Methode sagt Ihnen nicht, wie Sie die Tautologie aus den Axiomen beweisen können, sondern nur, dass ein Beweis existiert.  Einen tatsächlichen Beweis zu finden (insbesondere einen, der kurz und elegant ist), kann eine Herausforderung sein.  Es gibt zwar Methoden zur automatischen Generierung von Beweisen in der Aussagenlogik, aber die daraus resultierenden Beweise können manchmal sehr lang sein.  In der Metamath-Datenbasis \texttt{set.mm}\index{Mengenlehre-Datenbasis (\texttt{set.mm})} wurden die meisten oder sogar alle Beweise manuell erstellt. 

In Abschnitt \ref{metadefprop} werden verschiedene Definitionen erörtert, die die Verwendung der Aussagenlogik erleichtern. Wir definieren zum Beispiel:

\begin{itemize}
\item $\varphi \vee \psi$ ist wahr, wenn entweder $\varphi$ oder $\psi$ (oder beide) wahr sind (dies ist die Disjunktion\index{Disjunktion ($\vee$)} alias logisches {\sc oder}\index{logisches {\sc oder} ($\vee$)}).

\item $\varphi \wedge \psi$ ist wahr, wenn sowohl $\varphi$ als auch $\psi$ wahr sind (dies ist die Konjunktion\index{Konjunktion ($\wedge$)} alias logisches {\sc und}\index{logisches {\sc und} ($\wedge$)}).

\item $\varphi \leftrightarrow \psi$ ist wahr, wenn $\varphi$ und $\psi$ denselben Wert haben, d. h. beide wahr oder beide falsch sind (dies ist die Äquivalenz oder das Bikonditional\index{Bikonditional ($\leftrightarrow$)}).
\end{itemize}

\subsection{Prädikatenlogik}

Die Prädikatenlogik\index{Prädikatenlogik} führt das Konzept der "`individuellen Variablen"'\index{Variable!in der Prädikatenlogik}\index{individuelle Variable} ein, die wir in der Regel einfach "`Variablen"' nennen werden. Diese Variablen können etwas anderes als wahr oder falsch (wffs) darstellen und werden immer Mengen repräsentieren, sobald wir zur Mengenlehre kommen.  Es gibt auch drei neue Symbole $\forall$\index{Allquantor ($\forall$)}, $=$\index{Gleichheit ($=$)} und $\in$\index{stilisiertes Epsilon ($\in$)}, die jeweils "`für alle"', "`gleich"' und "`ist ein Element von"' bedeuten.  Wir werden Variablen mit lateinischen Kleinbuchstaben wie  $x$, $y$, $z$ und $w$ darstellen, wie es in der Literatur üblich ist. Zum Beispiel bedeutet $\forall x \varphi$: "`Für alle möglichen Werte von $x$ ist $\varphi$ wahr."'

In der Prädikatenlogik erweitern wir die Definition einer wff\index{wohlgeformte Formel (wff)}.  Wenn $\varphi$ eine wff ist und $x$ und $y$ Variablen sind, dann sind $\forall x \, \varphi$, $x=y$, und $x\in y$ wffs. Man beachte, dass diese drei neuen Arten von wffs als "`initiale wffs"' betrachtet werden können, aus denen man andere wffs mit $\rightarrow$ und $\neg$ aufbauen kann.  Das Konzept einer initialen wff war in der Aussagenlogik nicht vorhanden.  Aber egal ob initiale wff oder nicht, uns interessiert wirklich nur, ob unsere wffs korrekt nach diesen mechanischen Regeln konstruiert sind.

Eine kurze Anmerkung:
Um Verwirrung zu vermeiden, ist es an dieser Stelle vielleicht am besten, sich die Variablen von Metamath\index{Metamath} als "`Metavariablen"'\index{Metavariable} vorzustellen, weil sie nicht ganz dasselbe sind wie die Variablen, die wir hier vorstellen.  Eine (Meta-)Variable in Metamath kann ein wff oder eine individuelle Variable sein, sowie viele andere Dinge; im Allgemeinen stellt sie eine Art Platzhalter für eine nicht spezifizierte Folge von mathematischen Symbolen\index{mathematisches Symbol} dar.

Anders als in der Aussagenlogik gibt es kein Entscheidungsverfahren\index{Entscheidungsverfahren} analog zur Wahrheitstabellenmethode (und kann es theoretisch auch nicht geben), mit dem man sicher feststellen kann, ob eine Formel ein Satz der Prädikatenlogik ist.  Ein großer Teil der Arbeit auf dem Gebiet des automatischen Theorembeweisens\index{automatisches Theorembeweisen} wurde der Entwicklung cleverer Heuristiken zum Beweisen von Theoremen der Prädikatenlogik gewidmet, aber es kann nie garantiert werden, dass sie immer funktionieren.

In Abschnitt \ref{metadefpred} werden verschiedene Definitionen erörtert, die die Anwendung der Prädikatenlogik erleichtern. Zum Beispiel definieren wir $\exists x \varphi$ als "`Es gibt mindestens einen möglichen Wert von $x$, bei dem $\varphi$ wahr ist."'

Wir wenden uns nun der Frage zu, wie die Prädikatenklogik formal aufgebaut werden kann.

\subsubsection{Gängige Axiome}

Es gibt eine neue Schlussregel in der Prädikatenlogik: Wenn $\varphi$ ein Axiom oder ein Theorem ist, dann ist $\forall x \,\varphi$ auch ein Theorem\index{Theorem}.  Dies nennt man die "`Regel der Verallgemeinerung"'.\index{Regel der Verallgemeinerung} Dies lässt sich in Metamath leicht darstellen.

In Standardtexten der Logik gibt es oft zwei Axiome der Prädikatenlogik\index{Axiome der Prädikatenlogik}:
\begin{center}
    $\forall x \,\varphi ( x ) \rightarrow \varphi ( y )$, wobei "`$y$ echt durch $x$ ersetzt wird"'.\\\ $\forall x ( \varphi \rightarrow \psi )\rightarrow ( \varphi \rightarrow \forall x\, \psi )$, wobei "`$x$ in $\varphi$ nicht frei ist"'.
\end{center}

Auf den ersten Blick erscheint dies einfach: nur zwei Axiome.  Allerdings sind an jedes Axiom Bedingungsklauseln angehängt, die rätselhaft erscheinende Anforderungen beschreiben.  Außerdem setzt das erste Axiom nach jeder wff ein variables Symbol in Klammern, was offensichtlich gegen unsere Definition einer wff\index{wohlgeformte Formel (wff)} verstößt; dies ist nur eine informelle Art, auf eine beliebige Variable zu verweisen, die in der wff vorkommen kann.  Die Konditionalklauseln haben natürlich eine genaue Bedeutung, aber wie sich herausstellt, ist die genaue Bedeutung etwas kompliziert und schwer in einer Weise zu formalisieren, die ein Computer leicht handhaben kann.  Anders als bei der Aussagenlogik ist ein gewisses Maß an mathematischer Raffinesse und Übung erforderlich, um diese Konzepte leicht zu erfassen und korrekt zu handhaben.

Die Prädikatenlogik kann mit oder ohne Gleichheitsaxiome\index{Gleichheitsaxiome}\index{Gleichheit ($=$)} dargestellt werden. Wir werden die Gleichheitsaxiome als Voraussetzung für die von uns verwendete Version der Mengenlehre benötigen.  Die Gleichheitsaxiome werden, wenn sie enthalten sind, oft durch diese beiden Axiome dargestellt:
\begin{center}
$x=x$\\ \
$x=y\rightarrow (\varphi(x,x)\rightarrow\varphi(x,y))$,\\ wobei sich "`$\varphi(x,y)$ aus $\varphi(x,x)$ ergibt, indem einige, aber nicht notwendigerweise alle Vorkommen von $x$ durch $y$ ersetzt werden, vorausgesetzt, dass $y$ für $x$ in $\varphi(x,x)$ frei ist"'.
\end{center}
% (Mendelson p. 95)
Das erste Gleichheitsaxiom ist einfach, aber auch hier ist die Bedingung für das zweite Axiom etwas umständlich auf einem Computer umzusetzen.

\subsubsection{Tarski-System S2}

Natürlich sind wir nicht die Ersten, die die Komplikationen dieser Axiome der Prädikatenlogik bemerken, wenn man es genau nimmt.

Der bekannte Logiker Alfred Tarski veröffentlichte 1965 ein System, das er als System S2 bezeichnete \cite[S.~77]{Tarski1965}. Tarskis System ist \textit{exakt äquivalent} zu der traditionellen Lehrbuchformalisierung, aber (durch geschickten Gebrauch von Gleichheitsaxiomen) eliminiert es die primitiven Begriffe "`echte Substitution"' und "`freie Variable"' und ersetzt sie durch eine direkte Substitution und den Begriff einer Variable, die nicht in einer Formel vorkommt (was wir mit Nebenbedingungen für verschiedene Variablen ausdrücken).

Als er für sein System plädierte, schrieb Tarski: "`Der relativ komplizierte Charakter von [freien Variablen und echter Substitution] ist eine Quelle gewisser Unannehmlichkeiten sowohl praktischer als auch theoretischer Natur; dies zeigt sich deutlich sowohl beim Unterrichten eines Grundkurses der mathematischen Logik als auch bei der Formalisierung der Syntax der Prädikatenlogik für einige theoretische Zwecke"'\cite[S.~61]{Tarski1965}\index{Tarski, Alfred}.

\subsubsection{Entwicklung einer Metamath-Darstellung}

Die Standard-Lehrbuch-Axiome der Prädikatenlogik sind aufgrund der komplexen Begriffe der "`freien Variablen"'\index{freie Variable} und der "`echten Substitution"'\index{echte Substitution}\index{Substitution!echte} etwas umständlich auf einem Computer zu implementieren. Obwohl es möglich ist, diese Konzepte mit der Metamath\index{Metamath}-Sprache umzusetzen, haben wir uns dafür entschieden, sie nicht als primitive Konstrukte in der \texttt{set.mm}-Datenbasis der Mengenlehre zu implementieren.  Stattdessen haben wir sie aus den Axiomen eliminiert, indem wir, auf Tarskis System S2 aufbauend, die Axiome sorgfältig umformuliert haben, dass sie vermieden werden.  Dies macht es einem Anfänger leicht, den Schritten eines Beweises zu folgen, ohne irgendwelche fortgeschrittenen Konzepte zu kennen, außer dem einfachen Konzept der Ersetzung\index{Substitution!Variable}\index{Variablensubstitution} von Variablen durch Ausdrücke.

Um die Konzepte der freien Variablen und der echten Substitution aus den Axiomen zu entwickeln, verwenden wir einen zusätzlichen Metamath-Anweisungstyp namens "`disjunkte Variableneinschränkung"'\index{disjunkte Variablen}\index{disjunkte Variableneinschränkung}, der uns bisher noch nicht begegnet ist.  Im Zusammenhang mit den Axiomen bedeutet die Aussage \texttt{\$d} $ x\, y$\index{\texttt{\$d}-Anweisung} einfach, dass $x$ und $y$ verschiedene\index{verschiedene Variablen} sein müssen, d.h.\ sie dürfen nicht gleichzeitig durch dieselbe Variable substituiert werden\index{Substitution!Variable}\index{Variablensubstitution}.  Die Aussage \texttt{\$d} $ x\, \varphi$ bedeutet, dass die Variable $x$ nicht in der wff $\varphi$ vorkommen darf.  Für die genaue Definition von \texttt{\$d} siehe Abschnitt~\ref{dollard}.

\subsubsection{Metamath-Darstellung}

Das in set.mm definierte Metamath-Axiomensystem für die Prädikatenlogik verwendet das System S2 von Tarski. Wie oben erwähnt, hat dieses eine andere Darstellung als die traditionelle Lehrbuchformalisierung, aber es ist \textit{exakt äquivalent} zur Lehrbuchformalisierung, und es ist \textit{viel} einfacher damit zu arbeiten. Dies wird als System S3 in Abschnitt 6 von Megills Formalisierung \cite{Megill}\index{Megill, Norman} wiedergegeben.

Es gibt eine Ausnahme, nämlich Tarskis Axiom der Existenz, das wir als Axiom ax-6 bezeichnen. Im Fall von ax-6 ist Tarskis Version schwächer, weil sie eine disjunkte Variableneinschränkung enthält. Wenn wir wollen, können wir auch unsere Version auf diese Weise abschwächen und haben trotzdem ein metalogisch vollständiges System. Theorem ax-6 zeigt dies, indem es bei Vorhandensein der anderen Axiome unser ax-6 aus Tarskis schwächerer Version ax6v ableitet. Wir haben jedoch die stärkere Version für unser System gewählt, weil sie einfacher zu formulieren und leichter zu benutzen ist.

Tarskis System war eher für den Beweis spezifischer Theoreme als für allgemeinere Theoremschemata konzipiert. Theoremschemata sind jedoch sehr viel effizienter als spezifische Theoreme, wenn es darum geht, einen Bestand an mathematischem Wissen aufzubauen, da sie je nach Bedarf als verschiedene Instanzen wiederverwendet werden können. Tarski leitet zwar einige Theoremschemata aus seinen Axiomen ab, aber ihre Beweise erfordern Konzepte, die "`außerhalb"' des Systems liegen, wie z.B. die Induktion über die Länge von Formeln. Die Verifikation solcher Beweise lässt sich nur schwer mit einem Beweisverifizierer automatisieren. (Konkret behandelt Tarski die Formeln seines Systems als mengentheoretische Objekte. Um die Beweise seiner Theoremschemata zu verifizieren, müsste ein Beweisverifizierer eine beträchtliche Menge an Mengenlehre fest implementiert haben).

Das Metamath-Axiomensystem für die Prädikatenlogik erweitert Tarskis System, um diese Schwierigkeit zu beseitigen. Die zusätzlichen "`unter\-stüt\-zen\-den"' Axiomenschemata (wie wir sie in diesem Abschnitt nennen werden; siehe unten) verleihen Tarskis System eine nette Eigenschaft, die wir metalogische Vollständigkeit nennen \cite[Remark 9.6]{Megill}\index{Megill, Norman}. Infolgedessen können wir jedes Theoremschema beweisen, das in der "`einfachen Metalogik"' des Tarski-Systems ausgedrückt werden kann, indem wir nur die direkte Substitutionsregel von Metamath verwenden, die auf das Axiomensystem angewendet wird (und keine anderen metalogischen oder mengentheoretischen Begriffe "`außerhalb"' des Systems). Einfache Metalogik besteht aus Schemata, die wff-Metavariablen (ohne Argumente) und/oder Mengenmetavariablen (auch "`individuelle Variablen"' genannt) enthalten, begleitet von optionalen Vorschriften, die jeweils besagen, dass zwei spezifizierte Mengenmetavariablen verschieden sein müssen oder dass eine spezifizierte Mengenmetavariable nicht in einer spezifizierten wff-Metavariable vorkommen darf. Die Axiom- und Regelschemata der Metamath-Logik und der Mengenlehre sind allesamt Beispiele für eine einfache Metalogik. Die Schemata der traditionellen Prädikatenlogik mit Gleichheit sind Beispiele, die keine einfache Metalogik sind, weil sie wff-Metavariablen mit Argumenten verwenden und "`frei für"' und "`nicht frei in"' als Nebenbedingungen haben.

Eine strenge Begründung für dieses System, das einen älteren, aber genau gleichwertigen Satz von Axiomen verwendet, ist in \cite{Megill}\index{Megill, Norman} zu finden.

Dies ermöglicht es uns, in der Metamath\index{Metamath}-Datenbasis \texttt{set.mm}\index{Mengenlehre-Datenbasis (\texttt{set.mm})} einen anderen Ansatz zu wählen.  Wir verwenden keinen der primitiven Begriffe "`freie Variable"'\index{freie Variable} und "`echte Substitution"'\index{echte Substitution}\index{Substitution!echte} als primitive Konstrukte. Stattdessen verwenden wir eine Reihe von Axiomen, die fast so einfach zu handhaben sind wie die der Aussagenlogik.  Unser Axiomensystem vermeidet komplexe primitive Begriffe, indem es die Komplexität wirksam in die Axiome selbst einbettet.  Das Ergebnis ist eine größere Anzahl von Axiomen, die jedoch ideal für eine Computersprache wie Metamath geeignet sind. (Abschnitt~\ref{metaaxioms} zeigt diese Axiome.)

Wir werden hier nicht weiter auf die Begriffe "`freie Variable"' und "`echte Substitution"' eingehen.  Eine genaue Erklärung dieser Konzepte finden Sie in \cite[ch.\ 3--4]{Hamilton}\index{Hamilton, Alan G.} (sowie in vielen anderen Büchenr).  Wenn Sie beabsichtigen, ernsthaft mathematisch zu arbeiten, ist es ratsam, sich mit dem traditionellen Lehrbuchansatz vertraut zu machen; auch wenn die in den Axiomen enthaltenen Konzepte ein höheres Maß an Raffinesse erfordern, können diese für den täglichen, informellen Umgang praktikabler sein.  Selbst wenn Sie nur Metamath-Beweise entwickeln, kann Ihnen die Vertrautheit mit dem traditionellen Ansatz helfen, viel schneller zu einem Beweisentwurf zu gelangen, den Sie dann in die von Metamath geforderten Details umsetzen können.

Wir entwickeln später eigene Substitutionsregeln, aber in set.mm sind sie als abgeleitete Konstrukte definiert; sie sind keine Primitive.

Sie sollten auch beachten, dass unser System der Prädikatenlogik speziell auf die Mengenlehre zugeschnitten ist; daher gibt es nur zwei spezifische Prädikate $=$ und $\in$ und keine Funktionen\index{Funktion!in der Prädikatenlogik} oder Konstanten\index{Konstante!in der Prädikatenlogik} im Gegensatz zu allgemeineren Systemen. Wir fügen diese später hinzu.

\subsection{Mengenlehre}

Die traditionelle Zermelo--Fraenkel-Mengenlehre\index{Zermelo--Fraenkel-Mengenlehre}\index{Mengenlehre} mit dem Auswahlaxiom hat 10 Axiome, die in der Sprache der Prädikatenlogik ausgedrückt werden können.  In diesem Abschnitt werden wir nur die Namen und kurze deutsche (und englische) Beschreibungen dieser Axiome aufführen, da wir Ihnen später die genauen Formeln der Metamath\index{Metamath}-Mengenlehre-Datenbasis \texttt{set.mm} vorstellen werden.

In den Beschreibungen der Axiome gehen wir davon aus, dass $x$, $y$, $z$, $w$ und $v$ Mengen darstellen.  Dies sind die gleichen Variablen\index{Variable!in der Mengenlehre} wie in unserem obigen System der Prädikatenlogik mit dem Unterschied, dass wir uns die Variablen jetzt informell als Platzhalter für Mengen vorstellen.  Beachten Sie, dass die Begriffe "`Objekt"'\index{Objekt}, "`Menge"'\index{Menge}, "`Element"'\index{Element}, "`Sammlung"'\index{Sammlung} und "`Familie"'\index{Familie} synonym sind, ebenso wie "`ist ein Element von"', "`ist ein Mitglied von"'\index{Mitglied}, "`ist enthalten in"' und "`gehört zu"'.  Die verschiedenen Begriffe werden der Einfachheit halber verwendet; zum Beispiel ist "`eine Sammlung von Mengen"' weniger verwirrend als "`eine Menge von Mengen"'. Eine Menge $x$ ist eine "`Teilmenge"'\index{Teilmenge} von $y$, wenn jedes Element von $x$ auch ein Element von $y$ ist; wir sagen auch, dass $x$ "`in $y$ enthalten"' ist.

Die Axiome sind sehr allgemein und gelten für fast alle denkbaren mathematischen Objekte, und diese Abstraktionsebene kann zunächst erdrückend sein.  Um ein intuitives Gefühl für das Konzept zu bekommen, kann es hilfreich sein, ein Bild zu zeichnen, welches das Konzept veranschaulicht; ein Kreis mit Punkten könnte beispielsweise eine Sammlung von Mengen darstellen, und ein kleinerer Kreis, der innerhalb des Kreises gezeichnet wird, könnte eine Teilmenge darstellen. Sich überschneidende Kreise können Schnittmenge und Vereinigung veranschaulichen.  Kreise, die die Konzepte der Mengenlehre veranschaulichen, werden häufig in Grundschulbüchern verwendet und als Venn-Diagramme\index{Venn-Diagramm} bezeichnet.\index{Axiome der Mengenlehre}

1. Extensionalitätsaxiom (engl. Axiom of Extensionality):  Zwei Mengen sind identisch, wenn sie dieselben Elemente enthalten.\index{Extensionalitätsaxiom}

2. Paarmengenaxiom (engl. Axiom of Pairing):  Die Menge $\{ x , y \}$ existiert.\index{Paarmengenaxiom}

3. Potenzmengenaxiom (engl. Axiom of Power Sets):  Die Potenzmenge einer Menge (die Sammlung aller ihrer Teilmengen) existiert.  Zum Beispiel ist die Potenzmenge von $\{x,y\}$ $\{\varnothing,\{x\},\{y\},\{x,y\}\}$, und sie existiert.\index{Potenzmengenaxiom}

4. Leermengenaxiom (engl. Axiom of the Null Set):  Die leere Menge $\varnothing$ existiert.\index{Leermengenaxiom}

5. Vereinigungsaxiom (engl. Axiom of Union):  Die Vereinigung einer Menge (die Menge, welche die Elemente ihrer Mitglieder enthält) existiert.  Zum Beispiel ist die Vereinigung von $\{x,y\},\{z\}$ die existierende Menge $\{x,y,z\}$.\index{Vereinigungsaxiom}

6. Fundierungsaxiom (engl. Axiom of Regularity):  Grob gesagt kann keine Menge sich selbst enthalten, noch kann es zyklische Zugehörigkeiten geben, wie z.B. dass eine Menge ein Element eines ihrer Mitglieder ist.\index{Fundierungsaxiom}

7. Unendlichkeitsaxiom (engl. Axiom of Infinity):  Eine unendliche Menge existiert.  Ein Beispiel für eine unendliche Menge ist die Menge aller ganzen Zahlen.\index{Unendlichkeitsaxiom}

8. Aussonderungsaxiom (engl. Axiom of Separation):  Es existiert die Menge, die man erhält, wenn man $x$ mit einer bestimmten Eigenschaft einschränkt.  Wenn zum Beispiel die Menge aller ganzen Zahlen existiert, dann existiert auch die Menge aller geraden ganzen Zahlen.\index{Aussonderungsaxiom}

9. Ersetzungsaxiom (engl. Axiom of Replacement):  Der Wertebereich einer Funktion, deren Definitionsbereich auf die Elemente einer Menge $x$ beschränkt ist, ist ebenfalls eine Menge.  Zum Beispiel gibt es eine Funktion von den ganzen Zahlen (der Definitionsbereich der Funktion) zu ihren Quadraten (ihr Wertebereich).  Schränkt man den Definitionsbereich auf gerade Zahlen ein, so wird ihr Wertebereich zur Menge der Quadrate der geraden Zahlen, so dass dieses Axiom besagt, dass die Menge der Quadrate der geraden Zahlen existiert.  Technische Anmerkung: Im Allgemeinen muss die "`Funktion"' keine Menge sein, sondern kann eine echte Klasse sein.\index{Ersetzungsaxiom}

10. Auswahlaxiom:  Sei $x$ eine Menge, deren Mitglieder paarweise disjunkte Mengen\index{disjunkte Mengen} sind. (d.h. deren Mitglieder keine gemeinsamen Elemente enthalten).  Dann gibt es eine andere Menge, die ein Element von jedem Mitglied von $x$ enthält.  Wenn $x$ beispielsweise $\{\{y,z\},\{w,v\}\}$ ist, wobei $y$, $z$, $w$ und $v$ verschiedene Mengen sind, dann existiert eine Menge wie $\{z,w\}$ (das Axiom sagt uns aber nicht welche).  (Eigentlich ist das Auswahlaxiom überflüssig, wenn die Menge $x$, wie in diesem Beispiel, eine endliche Anzahl von Elementen hat.)\index{Auswahlaxiom}

Das Auswahlaxiom wird in der Regel als Erweiterung der ZF-Mengenlehre betrachtet und nicht als deren originärer Bestandteil.  Es wird manchmal als philosophisch umstritten angesehen, weil es die Existenz einer Menge festlegt, ohne sie zu spezifizieren. Konstruktive Logiken, einschließlich der intuitionistischen Logik, akzeptieren das Auswahlaxiom nicht. Da die Kontroverse darüber anhält, bevorzugen wir oft Beweise, die das Auswahlaxiom nicht verwenden (wenn es eine bekannte Alternative gibt), und in einigen Fällen werden wir schwächere Axiome als das vollständige Auswahlaxiom verwenden. Dennoch ist das Auswahlaxiom ein mächtiges und weithin akzeptiertes Werkzeug, so dass wir es bei Bedarf verwenden. Die ZF-Mengenlehre, die das Auswahlaxiom enthält, wird Zermelo--Fraenkel-Mengenlehre mit Auswahlaxiom (ZFC\index{ZFC-Mengenlehre}) genannt.

Symbolisch ausgedrückt enthalten das Aussonderungsaxiom und das Ersetzungsaxiom wff-Symbole und stellen daher jeweils unendlich viele Axiome dar, eines für jede mögliche wff. Aus diesem Grund werden sie oft als Axiomenschemata\index{Axiomenschema}\index{wohlgeformte Formel (wff)} bezeichnet.

Es stellt sich heraus, dass das Leermengenaxiom, das Paarmengenaxiom und das Aussonderungsaxiom aus den anderen Axiomen abgeleitet werden können und daher unnötig sind, obwohl sie aus verschiedenen Gründen (historisch, philosophisch und möglicherweise, weil einige Autoren dies nicht wissen) in Standardtexten enthalten sind.  In der Metamath\index{Metamath}-Mengenlehre-Datenbasis werden diese überflüssigen Axiome von den anderen abgeleitet, anstatt wirklich als Axiome betrachtet zu werden. Dies entspricht unserem allgemeinen Ziel, die Anzahl der Axiome, von denen wir abhängig sind, zu minimieren.

\subsection{Andere Axiome}

Oben haben wir die Formulierung "`die gesamte Mathematik"' mit "`im Wesentlichen"' relativiert. Das wichtigste fehlende Element ist die Fähigkeit zur Kategorientheorie, die riesige Mengen (unzugängliche Kardinalzahlen) erfordert, die größer sind als die, die von den ZFC-Axiomen abgeleitet werden können. Das Tarski--Grothendieck-Axiom postuliert die Existenz solcher Mengen. Man beachte, dass dies dasselbe Axiom ist, das von Mizar zur Unterstützung der Kategorientheorie verwendet wird. Das Tarski--Grothendieck-Axiom kann als ein sehr starker Ersatz für das Unendlichkeitsaxiom, das Auswahlaxiom und das Potenzmengenaxiom angesehen werden. Die Datenbasis \texttt{set.mm} enthält dieses Axiom; Einzelheiten dazu finden Sie in der Datenbasis. Auch dieses Axiom wird nur verwendet, wenn es absolut notwendig ist. Sie werden diesem Axiom wahrscheinlich nur begegnen oder es verwenden, wenn Sie sich mit der Kategorientheorie beschäftigen, da seine Verwendung hochspezialisiert ist. Daher werden wir das Tarski--Grothendieck-Axiom in der folgenden kurzen Liste von Axiomen nicht auf\-führen.

Kann es noch mehr Axiome geben? Ja, natürlich. G\"{o}del hat gezeigt, dass keine endliche Menge von Axiomen oder Axiomenschemata eine konsistente Theorie, die stark genug ist, um die Arithmetik einzuschließen, vollständig beschreiben kann. Aber praktisch gesehen sind die oben genannten Axiome die anerkannte Grundlage, auf der fast alle Mathematiker explizit oder implizit ihre Arbeit aufbauen.

\section{Die Axiome in der Metamath-Sprache}\label{metaaxioms}

Hier führen wir die Axiome so auf, wie sie in der Datenbasis der Mengenlehre \texttt{set.mm}\index{Mengenlehre-Datenbasis (\texttt{set.mm})} erscheinen, damit Sie diese dort leicht nachschlagen können.  Übrigens wurde der Befehl \texttt{show statement /tex}\index{\texttt{show statement}-Befehl} verwendet, um sie darzustellen.

%macros from show statement /tex
\newbox\mlinebox
\newbox\mtrialbox
\newbox\startprefix  % Prefix for first line of a formula
\newbox\contprefix  % Prefix for continuation line of a formula
\def\startm{  % Initialize formula line
  \setbox\mlinebox=\hbox{\unhcopy\startprefix}
}
\def\m#1{  % Add a symbol to the formula
  \setbox\mtrialbox=\hbox{\unhcopy\mlinebox $\,#1$}
  \ifdim\wd\mtrialbox>\hsize
    \box\mlinebox
    \setbox\mlinebox=\hbox{\unhcopy\contprefix $\,#1$}
  \else
    \setbox\mlinebox=\hbox{\unhbox\mtrialbox}
  \fi
}
\def\endm{  % Output the last line of a formula
  \box\mlinebox
}

% \SLASH for \ , \TOR for \/ (text OR), \TAND for /\ (text and)
% This embeds a following forced space to force the space.
\newcommand\SLASH{\char`\\~}
\newcommand\TOR{\char`\\/~}
\newcommand\TAND{/\char`\\~}
%
% Macro to output metamath raw text.
% This assumes \startprefix and \contprefix are set.
% NOTE: "\" is tricky to escape, use \SLASH, \TOR, and \TAND inside.
% Any use of "$ { ~ ^" must be escaped; ~ and ^ must be escaped specially.
% We escape { and } for consistency.
% For more about how this macro written, see:
% https://stackoverflow.com/questions/4073674/
% how-to-disable-indentation-in-particular-section-in-latex/4075706
% Use frenchspacing, or "e." will get an extra space after it.
\newlength\mystoreparindent
\newlength\mystorehangindent
\newenvironment{mmraw}{%
\setlength{\mystoreparindent}{\the\parindent}
\setlength{\mystorehangindent}{\the\hangindent}
\setlength{\parindent}{0pt} % TODO - we'll put in the \startprefix instead
\setlength{\hangindent}{\wd\the\contprefix}
\begin{flushleft}
\begin{frenchspacing}
\begin{tt}
{\unhcopy\startprefix}%
}{%
\end{tt}
\end{frenchspacing}
\end{flushleft}
\setlength{\parindent}{\mystoreparindent}
\setlength{\hangindent}{\mystorehangindent}
\vskip 1ex
}

\needspace{5\baselineskip}
\subsection{Aussagenlogik}\label{propcalc}\index{Axiome der Aussagenlogik}

\needspace{2\baselineskip}
Axiom der Vereinfachung.\label{ax1}

\setbox\startprefix=\hbox{\tt \ \ ax-1\ \$a\ }
\setbox\contprefix=\hbox{\tt \ \ \ \ \ \ \ \ \ \ }
\startm
\m{\vdash}\m{(}\m{\varphi}\m{\rightarrow}\m{(}\m{\psi}\m{\rightarrow}\m{\varphi}\m{)}
\m{)}
\endm

\needspace{3\baselineskip}
\noindent Axiom der Verteilung.

\setbox\startprefix=\hbox{\tt \ \ ax-2\ \$a\ }
\setbox\contprefix=\hbox{\tt \ \ \ \ \ \ \ \ \ \ }
\startm
\m{\vdash}\m{(}\m{(}\m{\varphi}\m{\rightarrow}\m{(}\m{\psi}\m{\rightarrow}\m{\chi}
\m{)}\m{)}\m{\rightarrow}\m{(}\m{(}\m{\varphi}\m{\rightarrow}\m{\psi}\m{)}\m{
\rightarrow}\m{(}\m{\varphi}\m{\rightarrow}\m{\chi}\m{)}\m{)}\m{)}
\endm

\needspace{2\baselineskip}
\noindent Axiom der Kontraposition.

\setbox\startprefix=\hbox{\tt \ \ ax-3\ \$a\ }
\setbox\contprefix=\hbox{\tt \ \ \ \ \ \ \ \ \ \ }
\startm
\m{\vdash}\m{(}\m{(}\m{\lnot}\m{\varphi}\m{\rightarrow}\m{\lnot}\m{\psi}\m{)}\m{
\rightarrow}\m{(}\m{\psi}\m{\rightarrow}\m{\varphi}\m{)}\m{)}
\endm


\needspace{4\baselineskip}
\noindent Die Schlussregel Modus ponens.\label{axmp}\index{Modus ponens}

\setbox\startprefix=\hbox{\tt \ \ min\ \$e\ }
\setbox\contprefix=\hbox{\tt \ \ \ \ \ \ \ \ \ }
\startm
\m{\vdash}\m{\varphi}
\endm

\setbox\startprefix=\hbox{\tt \ \ maj\ \$e\ }
\setbox\contprefix=\hbox{\tt \ \ \ \ \ \ \ \ \ }
\startm
\m{\vdash}\m{(}\m{\varphi}\m{\rightarrow}\m{\psi}\m{)}
\endm

\setbox\startprefix=\hbox{\tt \ \ ax-mp\ \$a\ }
\setbox\contprefix=\hbox{\tt \ \ \ \ \ \ \ \ \ \ \ }
\startm
\m{\vdash}\m{\psi}
\endm


\needspace{7\baselineskip}
\subsection{Axiome der Prädikatenlogik mit Gleichheit\texorpdfstring{\\---}{ ---} Tarskis S2}\index{Axiome der Prädikatenlogik}

\needspace{3\baselineskip}
\noindent Regel der Verallgemeinerung.\index{Regel der Verallgemeinerung}

\setbox\startprefix=\hbox{\tt \ \ ax-g.1\ \$e\ }
\setbox\contprefix=\hbox{\tt \ \ \ \ \ \ \ \ \ \ \ \ }
\startm
\m{\vdash}\m{\varphi}
\endm

\setbox\startprefix=\hbox{\tt \ \ ax-gen\ \$a\ }
\setbox\contprefix=\hbox{\tt \ \ \ \ \ \ \ \ \ \ \ \ }
\startm
\m{\vdash}\m{\forall}\m{x}\m{\varphi}
\endm

\needspace{2\baselineskip}
\noindent Axiom der quantifizierten Implikation.

\setbox\startprefix=\hbox{\tt \ \ ax-4\ \$a\ }
\setbox\contprefix=\hbox{\tt \ \ \ \ \ \ \ \ \ \ }
\startm
\m{\vdash}\m{(}\m{\forall}\m{x}\m{(}\m{\forall}\m{x}\m{\varphi}\m{\rightarrow}\m{
\psi}\m{)}\m{\rightarrow}\m{(}\m{\forall}\m{x}\m{\varphi}\m{\rightarrow}\m{
\forall}\m{x}\m{\psi}\m{)}\m{)}
\endm

\needspace{3\baselineskip}
\noindent Axiom der Unterscheidbarkeit.

% Aka: Add $d x ph $.
\setbox\startprefix=\hbox{\tt \ \ ax-5\ \$a\ }
\setbox\contprefix=\hbox{\tt \ \ \ \ \ \ \ \ \ \ }
\startm
\m{\vdash}\m{(}\m{\varphi}\m{\rightarrow}\m{\forall}\m{x}\m{\varphi}\m{)}
\m{mit}\m{ }\m{\$d}\m{ }\m{x}\m{ }\m{\varphi}\m{ }\m{(}\m{x}\m{ }\m{kommt}\m{ }\m{in}\m{ }\m{\varphi}
\m{ }\m{nicht}\m{ }\m{vor}\m{)}
\endm

\needspace{2\baselineskip}
\noindent Axiom der Existenz.

\setbox\startprefix=\hbox{\tt \ \ ax-6\ \$a\ }
\setbox\contprefix=\hbox{\tt \ \ \ \ \ \ \ \ \ \ }
\startm
\m{\vdash}\m{(}\m{\forall}\m{x}\m{(}\m{x}\m{=}\m{y}\m{\rightarrow}\m{\forall}
\m{x}\m{\varphi}\m{)}\m{\rightarrow}\m{\varphi}\m{)}
\endm

\needspace{2\baselineskip}
\noindent Axiom der Gleichheit.

\setbox\startprefix=\hbox{\tt \ \ ax-7\ \$a\ }
\setbox\contprefix=\hbox{\tt \ \ \ \ \ \ \ \ \ \ }
\startm
\m{\vdash}\m{(}\m{x}\m{=}\m{y}\m{\rightarrow}\m{(}\m{x}\m{=}\m{z}\m{
\rightarrow}\m{y}\m{=}\m{z}\m{)}\m{)}
\endm

\needspace{2\baselineskip}
\noindent Axiom der Linksgleichheit für binäre Prädikate.

\setbox\startprefix=\hbox{\tt \ \ ax-8\ \$a\ }
\setbox\contprefix=\hbox{\tt \ \ \ \ \ \ \ \ \ \ \ }
\startm
\m{\vdash}\m{(}\m{x}\m{=}\m{y}\m{\rightarrow}\m{(}\m{x}\m{\in}\m{z}\m{
\rightarrow}\m{y}\m{\in}\m{z}\m{)}\m{)}
\endm

\needspace{2\baselineskip}
\noindent Axiom der Rechtsgleichheit für binäre Prädikate.

\setbox\startprefix=\hbox{\tt \ \ ax-9\ \$a\ }
\setbox\contprefix=\hbox{\tt \ \ \ \ \ \ \ \ \ \ \ }
\startm
\m{\vdash}\m{(}\m{x}\m{=}\m{y}\m{\rightarrow}\m{(}\m{z}\m{\in}\m{x}\m{
\rightarrow}\m{z}\m{\in}\m{y}\m{)}\m{)}
\endm


\needspace{4\baselineskip}
\subsection{Axiome der Prädikatenlogik mit Gleichheit\texorpdfstring{\\---}{ ---} Hilfsaxiome}\index{Axiome der Prädikatenlogik - Hilfsaxiome}

\needspace{2\baselineskip}
\noindent Axiom der quantifizierten Negation.

\setbox\startprefix=\hbox{\tt \ \ ax-10\ \$a\ }
\setbox\contprefix=\hbox{\tt \ \ \ \ \ \ \ \ \ \ }
\startm
\m{\vdash}\m{(}\m{\lnot}\m{\forall}\m{x}\m{\lnot}\m{\forall}\m{x}\m{\varphi}\m{
\rightarrow}\m{\varphi}\m{)}
\endm

\needspace{2\baselineskip}
\noindent Axiom der Quantifizierungskommutativität.

\setbox\startprefix=\hbox{\tt \ \ ax-11\ \$a\ }
\setbox\contprefix=\hbox{\tt \ \ \ \ \ \ \ \ \ \ }
\startm
\m{\vdash}\m{(}\m{\forall}\m{x}\m{\forall}\m{y}\m{\varphi}\m{\rightarrow}\m{
\forall}\m{y}\m{\forall}\m{x}\m{\varphi}\m{)}
\endm

\needspace{3\baselineskip}
\noindent Axiom der Substitution.

\setbox\startprefix=\hbox{\tt \ \ ax-12\ \$a\ }
\setbox\contprefix=\hbox{\tt \ \ \ \ \ \ \ \ \ \ \ }
\startm
\m{\vdash}\m{(}\m{\lnot}\m{\forall}\m{x}\m{\,x}\m{=}\m{y}\m{\rightarrow}\m{(}
\m{x}\m{=}\m{y}\m{\rightarrow}\m{(}\m{\varphi}\m{\rightarrow}\m{\forall}\m{x}\m{(}
\m{x}\m{=}\m{y}\m{\rightarrow}\m{\varphi}\m{)}\m{)}\m{)}\m{)}
\endm

\needspace{3\baselineskip}
\noindent Axiom der quantifizierten Gleichheit.

\setbox\startprefix=\hbox{\tt \ \ ax-13\ \$a\ }
\setbox\contprefix=\hbox{\tt \ \ \ \ \ \ \ \ \ \ \ }
\startm
\m{\vdash}\m{(}\m{\lnot}\m{\forall}\m{z}\m{\,z}\m{=}\m{x}\m{\rightarrow}\m{(}
\m{\lnot}\m{\forall}\m{z}\m{\,z}\m{=}\m{y}\m{\rightarrow}\m{(}\m{x}\m{=}\m{y}
\m{\rightarrow}\m{\forall}\m{z}\m{\,x}\m{=}\m{y}\m{)}\m{)}\m{)}
\endm

% \noindent Axiom of Quantifier Substitution
%
% \setbox\startprefix=\hbox{\tt \ \ ax-c11n\ \$a\ }
% \setbox\contprefix=\hbox{\tt \ \ \ \ \ \ \ \ \ \ \ }
% \startm
% \m{\vdash}\m{(}\m{\forall}\m{x}\m{\,x}\m{=}\m{y}\m{\rightarrow}\m{(}\m{\forall}
% \m{x}\m{\varphi}\m{\rightarrow}\m{\forall}\m{y}\m{\varphi}\m{)}\m{)}
% \endm
%
% \noindent Axiom of Distinct Variables. (This axiom requires
% that two individual variables
% be distinct\index{\texttt{\$d}-Anweisung}\index{distinct
% variables}.)
%
% \setbox\startprefix=\hbox{\tt \ \ \ \ \ \ \ \ \$d\ }
% \setbox\contprefix=\hbox{\tt \ \ \ \ \ \ \ \ \ \ \ }
% \startm
% \m{x}\m{\,}\m{y}
% \endm
%
% \setbox\startprefix=\hbox{\tt \ \ ax-c16\ \$a\ }
% \setbox\contprefix=\hbox{\tt \ \ \ \ \ \ \ \ \ \ \ }
% \startm
% \m{\vdash}\m{(}\m{\forall}\m{x}\m{\,x}\m{=}\m{y}\m{\rightarrow}\m{(}\m{\varphi}\m{
% \rightarrow}\m{\forall}\m{x}\m{\varphi}\m{)}\m{)}
% \endm

% \noindent Axiom of Quantifier Introduction (2).  (This axiom requires
% that the individual variable not occur in the
% wff\index{\texttt{\$d}-Anweisung}\index{unterschiedliche Variablen}.)
%
% \setbox\startprefix=\hbox{\tt \ \ \ \ \ \ \ \ \$d\ }
% \setbox\contprefix=\hbox{\tt \ \ \ \ \ \ \ \ \ \ \ }
% \startm
% \m{x}\m{\,}\m{\varphi}
% \endm
% \setbox\startprefix=\hbox{\tt \ \ ax-5\ \$a\ }
% \setbox\contprefix=\hbox{\tt \ \ \ \ \ \ \ \ \ \ \ }
% \startm
% \m{\vdash}\m{(}\m{\varphi}\m{\rightarrow}\m{\forall}\m{x}\m{\varphi}\m{)}
% \endm

\subsection{Mengenlehre}\label{mmsettheoryaxioms}

Um die Axiome der Mengenlehre\index{Axiome der Mengenlehre} etwas kompakter zu gestalten, gibt es einige Definitionen aus der Logik, die wir implizit verwenden, nämlich "`logisches {\sc und}"', \index{Konjunktion ($\wedge$)}\index{logisches {\sc und} ($\wedge$)} "`logische Äquivalenz"',\index{logische Äquivalenz ($\leftrightarrow$)}\index{Bikonditional ($\leftrightarrow$)} und "`Es gibt"'\index{Existenzquantor ($\exists$)}.

\begin{center}\begin{tabular}{rcl}
  $( \varphi \wedge \psi )$ &\mbox{steht für}& $\neg ( \varphi
     \rightarrow \neg \psi )$\\
  $( \varphi \leftrightarrow \psi )$& \mbox{steht für}& $( ( \varphi \rightarrow \psi ) \wedge
     ( \psi \rightarrow \varphi ) )$\\
  $\exists x \,\varphi$ &\mbox{steht für}& $\neg \forall x \neg \varphi$
\end{tabular}\end{center}

Darüber hinaus verlangen die Axiome der Mengenlehre, dass alle Variablen unterscheidbar sind,\index{unterschiedliche Variablen}\footnote{Die Axiome der Mengenlehre können so entwickelt werden, dass {\em keine} Variablen unterscheidbar sein müssen, vorausgesetzt, wir ersetzen \texttt{ax-c16} durch ein Axiom, das besagt, dass "`mindestens zwei Dinge existieren"', wodurch \texttt{ax-5} das einzige andere Axiom wird, das die Anweisung \texttt{\$d} erfordert.  Diese Axiome sind unkonventionell und werden hier nicht vorgestellt, aber sie können auf der Website \url{http://metamath.org} gefunden werden.  Siehe auch den Kommentar zu S.~\pageref{nodd}.}\index{\texttt{\$d}-Anweisung} also nehmen wir auch an:
\begin{center}
  \texttt{\$d }$x\,y\,z\,w$
\end{center}

\needspace{2\baselineskip}
\noindent Extensionalitätsaxiom.\index{Extensionalitätsaxiom}

\setbox\startprefix=\hbox{\tt \ \ ax-ext\ \$a\ }
\setbox\contprefix=\hbox{\tt \ \ \ \ \ \ \ \ \ \ \ \ }
\startm
\m{\vdash}\m{(}\m{\forall}\m{x}\m{(}\m{x}\m{\in}\m{y}\m{\leftrightarrow}\m{x}
\m{\in}\m{z}\m{)}\m{\rightarrow}\m{y}\m{=}\m{z}\m{)}
\endm

\needspace{3\baselineskip}
\noindent Ersetzungsaxiom.\index{Ersetzungsaxiom}

\setbox\startprefix=\hbox{\tt \ \ ax-rep\ \$a\ }
\setbox\contprefix=\hbox{\tt \ \ \ \ \ \ \ \ \ \ \ \ }
\startm
\m{\vdash}\m{(}\m{\forall}\m{w}\m{\exists}\m{y}\m{\forall}\m{z}\m{(}\m{%
\forall}\m{y}\m{\varphi}\m{\rightarrow}\m{z}\m{=}\m{y}\m{)}\m{\rightarrow}\m{%
\exists}\m{y}\m{\forall}\m{z}\m{(}\m{z}\m{\in}\m{y}\m{\leftrightarrow}\m{%
\exists}\m{w}\m{(}\m{w}\m{\in}\m{x}\m{\wedge}\m{\forall}\m{y}\m{\varphi}\m{)}%
\m{)}\m{)}
\endm

\needspace{2\baselineskip}
\noindent Vereinigungsaxiom.\index{Vereinigungsaxiom}

\setbox\startprefix=\hbox{\tt \ \ ax-un\ \$a\ }
\setbox\contprefix=\hbox{\tt \ \ \ \ \ \ \ \ \ \ \ }
\startm
\m{\vdash}\m{\exists}\m{x}\m{\forall}\m{y}\m{(}\m{\exists}\m{x}\m{(}\m{y}\m{
\in}\m{x}\m{\wedge}\m{x}\m{\in}\m{z}\m{)}\m{\rightarrow}\m{y}\m{\in}\m{x}\m{)}
\endm

\needspace{2\baselineskip}
\noindent Potenzmengenaxiom.\index{Potenzmengenaxiom}

\setbox\startprefix=\hbox{\tt \ \ ax-pow\ \$a\ }
\setbox\contprefix=\hbox{\tt \ \ \ \ \ \ \ \ \ \ \ \ }
\startm
\m{\vdash}\m{\exists}\m{x}\m{\forall}\m{y}\m{(}\m{\forall}\m{x}\m{(}\m{x}\m{
\in}\m{y}\m{\rightarrow}\m{x}\m{\in}\m{z}\m{)}\m{\rightarrow}\m{y}\m{\in}\m{x}
\m{)}
\endm

\needspace{3\baselineskip}
\noindent Fundierungsaxiom.\index{Fundierungsaxiom}

\setbox\startprefix=\hbox{\tt \ \ ax-reg\ \$a\ }
\setbox\contprefix=\hbox{\tt \ \ \ \ \ \ \ \ \ \ \ \ }
\startm
\m{\vdash}\m{(}\m{\exists}\m{x}\m{\,x}\m{\in}\m{y}\m{\rightarrow}\m{\exists}
\m{x}\m{(}\m{x}\m{\in}\m{y}\m{\wedge}\m{\forall}\m{z}\m{(}\m{z}\m{\in}\m{x}\m{
\rightarrow}\m{\lnot}\m{z}\m{\in}\m{y}\m{)}\m{)}\m{)}
\endm

\needspace{3\baselineskip}
\noindent Unendlichkeitsaxiom.\index{Unendlichkeitsaxiom}

\setbox\startprefix=\hbox{\tt \ \ ax-inf\ \$a\ }
\setbox\contprefix=\hbox{\tt \ \ \ \ \ \ \ \ \ \ \ \ \ \ \ }
\startm
\m{\vdash}\m{\exists}\m{x}\m{(}\m{y}\m{\in}\m{x}\m{\wedge}\m{\forall}\m{y}%
\m{(}\m{y}\m{\in}\m{x}\m{\rightarrow}\m{\exists}\m{z}\m{(}\m{y}\m{\in}\m{z}\m{%
\wedge}\m{z}\m{\in}\m{x}\m{)}\m{)}\m{)}
\endm

\needspace{4\baselineskip}
\noindent Auswahlaxiom.\index{Auswahlaxiom}

\setbox\startprefix=\hbox{\tt \ \ ax-ac\ \$a\ }
\setbox\contprefix=\hbox{\tt \ \ \ \ \ \ \ \ \ \ \ \ \ \ }
\startm
\m{\vdash}\m{\exists}\m{x}\m{\forall}\m{y}\m{\forall}\m{z}\m{(}\m{(}\m{y}\m{%
\in}\m{z}\m{\wedge}\m{z}\m{\in}\m{w}\m{)}\m{\rightarrow}\m{\exists}\m{w}\m{%
\forall}\m{y}\m{(}\m{\exists}\m{w}\m{(}\m{(}\m{y}\m{\in}\m{z}\m{\wedge}\m{z}%
\m{\in}\m{w}\m{)}\m{\wedge}\m{(}\m{y}\m{\in}\m{w}\m{\wedge}\m{w}\m{\in}\m{x}%
\m{)}\m{)}\m{\leftrightarrow}\m{y}\m{=}\m{w}\m{)}\m{)}
\endm

\subsection{Das war's}

Das waren sie, die Axiome für (im Wesentlichen) die gesamte Mathematik! Bestaunen Sie sie und schauen Sie sie ehrfürchtig an.  Stecken Sie ein Exemplar in Ihre Brieftasche und Sie werden die Kodierung aller Theoreme, die jemals bewiesen wurden und die jemals bewiesen werden, in Ihrer Tasche tragen - von den banalsten bis zu den tiefgründigsten.

\section{Eine Hierarchie von Definitionen}\label{hierarchy}

Die Axiome im vorigen Abschnitt ermöglichen im Prinzip alles, was man in der Standardmathematik erreichen kann.  Allerdings ist es in den meisten Fällen unpraktisch, mit ihnen direkt zu arbeiten, da selbst einfache Konzepte (aus menschlicher Sicht) nur mit extrem langen, unverständlichen Formeln ausgedrückt werden können. Die Mathematik wird deshalb erst durch die Einführung von Definitionen\index{Definition} praktikabel. Definitionen führen in der Regel neue Symbole oder zumindest neue Beziehungen zwischen bestehenden Symbolen ein, um komplexere Formeln abzukürzen.  Eine wichtige Voraussetzung für eine Definition ist, dass es eine einfache (algorithmische) Methode gibt, um die Abkürzung zu eliminieren, indem sie durch die primitivere Symbolkette, die sie repräsentiert, ersetzt wird.  Einige wichtige Definitionen, die in der Datei \texttt{set.mm} enthalten sind, werden in diesem Abschnitt als Referenz aufgeführt, und auch, um Ihnen ein Gefühl dafür zu geben, warum etwas wie $\omega$\index{Omega ($\omega$)} (die Menge der natürlichen Zahlen\index{natürliche Zahlen} 0, 1, 2,\ldots) sehr kompliziert wird, wenn es vollständig in primitiven Symbolen ausgedrückt wird.

Was ist die Motivation für Definitionen, abgesehen davon, dass komplizierte Ausdrücke einfacher ausgedrückt werden können?  Im Falle von $\omega$ besteht ein Ziel darin, eine Grundlage für die Theorie der natürlichen Zahlen\index{natürliche Zahlen} zu schaffen. Vor der Erfindung der Mengenlehre wurde eine Reihe von Axiomen für die Arithmetik, die so genannten Peano-Postulate\ index{Peanos Postulate}, entwickelt und gezeigt, dass sie die erwarteten Eigenschaften für natürliche Zahlen haben.  Nun kann jeder einen Satz von Axiomen postulieren, aber wenn die Axiome inkonsistent sind, können daraus Widersprüche abgeleitet werden.  Sobald ein Widerspruch abgeleitet ist, kann alles trivialerweise bewiesen werden, einschließlich aller Fakten der Arithmetik und ihrer Negationen.  Um sicherzustellen, dass ein Axiomensystem mindestens so zuverlässig ist wie die Axiome der Mengenlehre, können wir Mengen und Operationen auf diesen Mengen definieren, die die neuen Axiome erfüllen. In der \texttt{set.mm} Metamath-Datenbasis beweisen wir, dass die Elemente von $\omega$ die Peano-Postulate erfüllen, und es ist ein langer und harter Weg, um von den Axiomen der Mengenlehre direkt dorthin zu gelangen.  Aber das Ergebnis ist das Vertrauen in die Grundlagen der Arithmetik.  Und es gibt noch einen weiteren Vorteil: Wir haben jetzt alle Werkzeuge der Mengenlehre zur Verfügung, um Objekte zu manipulieren, die den Axiomen der Arithmetik gehorchen.

Was sind die Kriterien für unsere Definitionen?  Erstens, und das ist von größter Bedeutung, sollte die Definition nicht {\em kreativ}\index{kreative Definition}\index{Definition!kreativ} sein, d.h. sie sollte nicht zulassen, dass ein Ausdruck, der als wff klassifiziert, aber nicht beweisbar war, beweisbar wird.   Zweitens sollte die Definition {\em eliminierbar}\index{Definition!Eliminierbarkeit} sein, d.h. es sollte eine algorithmische Methode geben, um jeden als wff klasifizierten Ausdruck, der die Definition verwendet, in einen logisch äquivalenten Ausdruck umzuwandeln.

In fast allen folgenden Fällen verbinden die Definitionen zwei Ausdrücke entweder mit $\leftrightarrow$ oder $=$.  Die Eliminierung\footnote{Hier ist die Eliminierung gemeint, die ein Mensch in seinem Kopf durchführen könnte.  Um sie als Teil eines Metamath-Beweises zu eliminieren, würden wir uns auf eines der Theoreme berufen, die sich mit der Transitivität von Äquivalenz oder Gleichheit befassen; es gibt viele solcher Beispiele in den Beweisen in \texttt{set.mm}.} einer solchen Definition ist ein einfaches Ersetzen des Ausdrucks auf der linken Seite ({\em Definiendum}\index{Definiendum} oder Sache, die definiert wird) durch den äquivalenten, primitiveren Ausdruck auf der rechten Seite ({\em Definiens}\index{Definiens} oder Definition).

Häufig enthält eine Definition auf der rechten Seite Variablen, die auf der linken Seite nicht vorkommen; diese werden als {\em Dummy-Variable} bezeichnet.\index{Dummy-Variable!in Definitionen}  In diesem Fall kann jede zulässige Substitution (z. B. eine neue, unterschiedliche Variable) verwendet werden, wenn die Definition eliminiert wird.  Dummy-Variablen dürfen nur verwendet werden, wenn sie {\em effektiv gebunden}\index{effektiv gebundene Variable} sind, was bedeutet, dass die Definition bei jeder Ersetzung einer Dummy-Variablen durch einen anderen {\em qualifizierenden Ausdruck}\index{qualifizierender Ausdruck} logisch äquivalent bleibt, d.h. eine beliebige Zeichenkette (z.B. eine andere Variable), die den Beschränkungen für die Dummy-Variable durch die Anweisungen \texttt{\$d} und \texttt{\$f} entspricht.  Wir könnten zum Beispiel eine Konstante $\perp$ (invertiertes T, d.h. logisch "`falsch"') als $( \varphi \wedge \lnot \varphi )$ definieren, d.h. \ "`phi und nicht phi"'.  Hier ist $\varphi$ effektiv gebunden, weil die Definition logisch äquivalent bleibt, wenn wir $\varphi$ durch irgendeine andere wff ersetzen.  (Eigentlich wird $\perp$ in \texttt{set.mm} durch \texttt{df-fal} definiert.)

Es gibt zwei Fälle, in denen die Eliminierung von Definitionen ein wenig komplexer ist.  Diese Fälle sind die Definitionen \texttt{df-bi} und \texttt{df-cleq}.  Der erste Fall fasst das Konzept einer Definition ein wenig weiter, da er in der Tat eine "`Definition definiert"'; diese Definition erfüllt jedoch unsere Anforderungen an eine Definition, da sie eliminierbar ist und die Mächtigkeit der Sprache nicht erhöht.  Theorem \texttt{bii} zeigt die notwendige Substitution, um das Symbol $\leftrightarrow$\index{logische Äquivalenz ($\leftrightarrow$)}\index{Bikonditional ($\leftrightarrow$)} zu eliminieren.

Definition \texttt{df-cleq}\index{Gleichheit ($=$)} erweitert die Verwendung des Gleichheitssymbols auf "`Klassen"'\index{Klasse} in der Mengenlehre.  Dies könnte potenziell problematisch sein, weil es zu Aussagen führen kann, die nicht allein aus der Logik folgen, sondern das Extensionalitätsaxiom\index{Extensionalitätsaxiom} voraussetzen. Deshalb nehmen wir dieses Axiom als Hypothese in die Definition mit auf.  Wir hätten \texttt{df-cleq} direkt eliminierbar machen können, indem wir ein neues Gleichheitssymbol eingeführt hätten. Wir haben uns aber entschieden, dies nicht zu tun, um der üblichen Lehrbuchpraxis zu entsprechen.  Definitionen wie \texttt{df-cleq}, die die Bedeutung bestehender Symbole erweitern, müssen sorgfältig eingeführt werden, damit sie nicht zu Widersprüchen führen.  Die Definition \texttt{df-clel} erweitert ebenfalls die Bedeutung eines bestehenden Symbols ($\in$); sie erhöht zwar nicht die Mächtigkeit der Sprache so wie \texttt{df-cleq}, aber das ist nicht offensichtlich, weshalb sie ebenfalls einer sorgfältigen Prüfung unterzogen werden muss.

Übung:  Untersuchen Sie, wie die wff $x\in\omega$, die besagt, dass "`$x$ eine natürliche Zahl ist"', in Form von primitiven Symbolen ausgedrückt werden könnte, indem Sie mit den Definitionen \texttt{df-clel} auf S.~\pageref{dfclel} und \texttt{df-om} auf S.~\pageref{dfom} beginnen und sich dann rückwärts vorarbeiten.  Machen Sie sich nicht die Mühe, die Details auszuarbeiten; stellen Sie nur sicher, dass Sie verstehen, wie Sie es im Prinzip tun könnten. Die Antwort finden Sie in der Fußnote auf S.~\pageref{expandom}.  Wenn Sie dies tatsächlich durchführen, werden Sie nicht genau die gleiche Antwort erhalten, weil wir einige Vereinfachungen verwendet haben, wie z.B. das Weglassen von $\lnot\lnot$ (doppelte Negation).

In den untenstehenden Definitionen haben wir die {\sc ascii} Metamath-Quelle unter jede der Formeln gesetzt, damit Sie sich mit der Notation in der Datenbasis vertraut machen können.  Der Einfachheit halber werden die notwendigen \texttt{\$f}- und \texttt{\$d}-Anweisungen nicht gezeigt.  Im Zweifelsfall sollten Sie den Befehl \texttt{show statement}\index{\texttt{show statement}-Befehl} im Metamath-Programm verwenden, um die vollständige Aussage zu sehen. Eine Auswahl dieser Notation ist im Anhang~\ref{ASCII} zusammengefasst.

Um die Motivation für diese Definitionen zu verstehen, sollten Sie die angegebenen Referenzen konsultieren:  Takeuti und Zaring \cite{Takeuti}\index{Takeuti, Gaisi}, Quine \cite{Quine}\index{Quine, Willard Van Orman}, Bell und Machover \cite{Bell}\index{Bell, J. L.}, und Enderton \cite{Enderton}\index{Enderton, Herbert B.}.  Unsere Liste der Definitionen dient eher als Referenz denn als Lernhilfe.  Anhand einiger Definitionen können Sie jedoch ein Gefühl dafür bekommen, wie die Hierarchie aufgebaut ist.  Die Definitionen sind eine repräsentative Auswahl der vielen Definitionen in \texttt{set.mm}, aber sie sind vollständig in Bezug auf die Beispieltheoreme, die wir in Abschnitt~\ref{sometheorems} vorstellen werden.  Außerdem unterscheiden sich einige Definitionen geringfügig von denen in \texttt{set.mm}, sind aber logisch äquivalent zu denen in \texttt{set.mm} (von denen einige im Laufe der Zeit überarbeitet wurden, um sie z. B. zu kürzen).

\subsection{Definitionen für die Aussagenlogik}\label{metadefprop}

Die Symbole $\varphi$, $\psi$ und $\chi$ stehen für wffs.

Unsere erste Definition führt den bikonditionalen Junktor ein\footnote{Der Begriff "`Junktor"' wird informell verwendet, um ein Symbol zu bezeichnen, das zwischen zwei Variablen oder neben einer Variable platziert ist, während eine mathematische "`Konstante"' normalerweise ein Symbol wie die Zahl 0 bezeichnet, das eine Variable oder Metavariable ersetzen kann.  Aus der Sicht von Metamath gibt es keine Unterscheidung zwischen einem Junktor und einer Konstante; beide sind in der Metamath-Sprache Konstanten.}\index{Junktor}\index{Konstante} (auch logische Äquivalenz genannt)\index{logische Äquivalenz ($\leftrightarrow$)}\index{Bikonditional ($\leftrightarrow$)}.  Im Gegensatz zu den meisten traditionellen Vorgehensweisen haben wir uns entschieden, kein separates Symbol wie "`Df."' für "`ist definiert als"' zu verwenden.  Stattdessen werden wir den bikonditionalen Junktor für diesen Zweck verwenden, da er uns erlaubt, Definitionen direkt mithilfe der Logik zu manipulieren.  Hier geben wir die Eigenschaften des bikonditionalen Junktors mit einer sorgfältig formulierten \texttt{\$a}-Anweisung an, die den bikonditionalen Junktor probat dazu verwendet, sich selbst zu definieren.  Das Symbol $\leftrightarrow$ kann mit Hilfe des Theorems \texttt{bii}, das später hergeleitet wird, aus einer Formel entfernt werden.

\vskip 2ex
\noindent Definition des {\bf bikonditionalen Junktors}.\label{df-bi}

\vskip 0.5ex
\setbox\startprefix=\hbox{\tt \ \ df-bi\ \$a\ }
\setbox\contprefix=\hbox{\tt \ \ \ \ \ \ \ \ \ \ \ }
\startm
\m{\vdash}\m{\lnot}\m{(}\m{(}\m{(}\m{\varphi}\m{\leftrightarrow}\m{\psi}\m{)}%
\m{\rightarrow}\m{\lnot}\m{(}\m{(}\m{\varphi}\m{\rightarrow}\m{\psi}\m{)}\m{%
\rightarrow}\m{\lnot}\m{(}\m{\psi}\m{\rightarrow}\m{\varphi}\m{)}\m{)}\m{)}\m{%
\rightarrow}\m{\lnot}\m{(}\m{\lnot}\m{(}\m{(}\m{\varphi}\m{\rightarrow}\m{%
\psi}\m{)}\m{\rightarrow}\m{\lnot}\m{(}\m{\psi}\m{\rightarrow}\m{\varphi}\m{)}%
\m{)}\m{\rightarrow}\m{(}\m{\varphi}\m{\leftrightarrow}\m{\psi}\m{)}\m{)}\m{)}
\endm
\begin{mmraw}%
|- -. ( ( ( ph <-> ps ) -> -. ( ( ph -> ps ) ->
-. ( ps -> ph ) ) ) -> -. ( -. ( ( ph -> ps ) -> -. (
ps -> ph ) ) -> ( ph <-> ps ) ) ) \$.
\end{mmraw}

\noindent Das folgende Theorem stellt eine Beziehung zwischen dem bikonditionalen Junktor und den primitiven Junktoren her und kann verwendet werden, um das $\leftrightarrow$-Symbol aus jeder wff zu eliminieren.

\vskip 0.5ex
\setbox\startprefix=\hbox{\tt \ \ bii\ \$p\ }
\setbox\contprefix=\hbox{\tt \ \ \ \ \ \ \ \ \ }
\startm
\m{\vdash}\m{(}\m{(}\m{\varphi}\m{\leftrightarrow}\m{\psi}\m{)}\m{\leftrightarrow}
\m{\lnot}\m{(}\m{(}\m{\varphi}\m{\rightarrow}\m{\psi}\m{)}\m{\rightarrow}\m{\lnot}
\m{(}\m{\psi}\m{\rightarrow}\m{\varphi}\m{)}\m{)}\m{)}
\endm
\begin{mmraw}%
|- ( ( ph <-> ps ) <-> -. ( ( ph -> ps ) -> -. ( ps -> ph ) ) ) \$= ... \$.
\end{mmraw}

\noindent Definition der {\bf Disjunktion} ({\sc oder}).\index{Disjunktion ($\vee$)}%
\index{logisches {\sc oder} ($\vee$)}%
\index{df-or@\texttt{df-or}}\label{df-or}

\vskip 0.5ex
\setbox\startprefix=\hbox{\tt \ \ df-or\ \$a\ }
\setbox\contprefix=\hbox{\tt \ \ \ \ \ \ \ \ \ \ \ }
\startm
\m{\vdash}\m{(}\m{(}\m{\varphi}\m{\vee}\m{\psi}\m{)}\m{\leftrightarrow}\m{(}\m{
\lnot}\m{\varphi}\m{\rightarrow}\m{\psi}\m{)}\m{)}
\endm
\begin{mmraw}%
|- ( ( ph \TOR ps ) <-> ( -. ph -> ps ) ) \$.
\end{mmraw}

\noindent Definition der {\bf Konjunktion} ({\sc und}).\index{Konjunktion ($\wedge$)}%
\index{logisches {\sc und} ($\wedge$)}%
\index{df-an@\texttt{df-an}}\label{df-an}

\vskip 0.5ex
\setbox\startprefix=\hbox{\tt \ \ df-an\ \$a\ }
\setbox\contprefix=\hbox{\tt \ \ \ \ \ \ \ \ \ \ \ }
\startm
\m{\vdash}\m{(}\m{(}\m{\varphi}\m{\wedge}\m{\psi}\m{)}\m{\leftrightarrow}\m{\lnot}
\m{(}\m{\varphi}\m{\rightarrow}\m{\lnot}\m{\psi}\m{)}\m{)}
\endm
\begin{mmraw}%
|- ( ( ph \TAND ps ) <-> -. ( ph -> -. ps ) ) \$.
\end{mmraw}

\noindent Definition der {\bf Disjunktion ({\sc oder}) von 3 wffs}.%
\index{df-3or@\texttt{df-3or}}\label{df-3or}

\vskip 0.5ex
\setbox\startprefix=\hbox{\tt \ \ df-3or\ \$a\ }
\setbox\contprefix=\hbox{\tt \ \ \ \ \ \ \ \ \ \ \ \ }
\startm
\m{\vdash}\m{(}\m{(}\m{\varphi}\m{\vee}\m{\psi}\m{\vee}\m{\chi}\m{)}\m{
\leftrightarrow}\m{(}\m{(}\m{\varphi}\m{\vee}\m{\psi}\m{)}\m{\vee}\m{\chi}\m{)}
\m{)}
\endm
\begin{mmraw}%
|- ( ( ph \TOR ps \TOR ch ) <-> ( ( ph \TOR ps ) \TOR ch ) ) \$.
\end{mmraw}

\noindent Definition der {\bf Konjunktion ({\sc und}) von 3 wffs}.%
\index{df-3an@\texttt{df-3an}}\label{df-3an}

\vskip 0.5ex
\setbox\startprefix=\hbox{\tt \ \ df-3an\ \$a\ }
\setbox\contprefix=\hbox{\tt \ \ \ \ \ \ \ \ \ \ \ \ }
\startm
\m{\vdash}\m{(}\m{(}\m{\varphi}\m{\wedge}\m{\psi}\m{\wedge}\m{\chi}\m{)}\m{
\leftrightarrow}\m{(}\m{(}\m{\varphi}\m{\wedge}\m{\psi}\m{)}\m{\wedge}\m{\chi}
\m{)}\m{)}
\endm

\begin{mmraw}%
|- ( ( ph \TAND ps \TAND ch ) <-> ( ( ph \TAND ps ) \TAND ch ) ) \$.
\end{mmraw}

\subsection{Definitionen für die Prädikatenlogik}\label{metadefpred}

Die Symbole $x$, $y$ und $z$ stehen für individuelle Variablen der Prädikatenlogik.  In diesem Abschnitt sind sie nicht notwendigerweise verschieden, es sei denn, es wird ausdrücklich erwähnt.

\vskip 2ex
\noindent Definition der {\bf existentiellen Quantifizierung}. 

Der Ausdruck $\exists x \varphi$ bedeutet "`Es existiert ein $x$, bei dem $\varphi$ wahr ist."'\index{Existenzquantor ($\exists$)}\label{df-ex}

\vskip 0.5ex
\setbox\startprefix=\hbox{\tt \ \ df-ex\ \$a\ }
\setbox\contprefix=\hbox{\tt \ \ \ \ \ \ \ \ \ \ \ }
\startm
\m{\vdash}\m{(}\m{\exists}\m{x}\m{\varphi}\m{\leftrightarrow}\m{\lnot}\m{\forall}
\m{x}\m{\lnot}\m{\varphi}\m{)}
\endm
\begin{mmraw}%
|- ( E. x ph <-> -. A. x -. ph ) \$.
\end{mmraw}

\noindent Definition der {\bf echten Substitution}.\index{echte Substitution}\index{Substitution!echte}\label{df-sb}

In unserer Notation verwenden wir $[ y / x ] \varphi$, um "`die wff zu bezeichnen, die sich ergibt, wenn $y$ in der wff $\varphi$ echt durch $x$ ersetzt wird"'.\footnote{Dies kann auch so beschrieben werden, dass $x$ durch $y$ ersetzt wird, $y$ $x$ echt ersetzt, oder $x$ echt durch $y$ ersetzt wird.}
% This is elsb4, though it currently says: ( [ x / y ] z e. y <-> z e. x )
Zum Beispiel ist $[ y / x ] z \in x$ das gleiche wie $z \in y$. Man kann sich diese Notation leicht merken, wenn man sie mit einer Division vergleicht, bei der $( y / x ) \cdot x $ $y$ ist (wenn $x \neq 0$). Die Notation unterscheidet sich von der Notation $\varphi ( x | y )$, die manchmal verwendet wird, weil letztere Notation für uns mehrdeutig ist: Wir wissen zum Beispiel nicht, ob $\lnot \varphi ( x | y )$ als $\lnot ( \varphi ( x | y )$ oder $( \lnot \varphi ) ( x | y )$ zu interpretieren ist.\footnote{Aufgrund der Art und Weise, wie wir wffs ursprünglich definiert haben, ist dies der Fall bei jedem Postfix-Konnektor\index{Postfix-Konnektor} (einer, der nach den zu verbindenden Symbolen auftritt) oder einem Infix-Konnektor\index{Infix-Konnektor} (einer, der zwischen den zu verbindenden Symbolen vorkommt).  Metamath hat keine eingebaute Regel für die Vorrangigkeit einer Operatorausführung, die die Mehrdeutigkeit beseitigen könnte.  Die öffnende Klammer stellt einen effektiven Präfix-Konnektor\index{Präfix-Konnektor} dar, um die Mehrdeutigkeit zu beseitigen.  Einige Konventionen, wie z. B. die polnische Notation\index{polnische Notation}, die in den 1930er und 1940er Jahren von polnischen Logikern verwendet wurde, verwenden nur Präfix-Konnektoren und ermöglichen so den vollständigen Verzicht auf Klammern, was allerdings auf Kosten der Lesbarkeit geht.  In Metamath könnten wir, wenn wir wollten, die gesamte Notation auf die polnische Notation umstellen, ohne irgendwelche Beweise ändern zu müssen!}  In anderen Texten wird oft $\varphi(y)$ verwendet, um unser $[ y / x ] \varphi$ zu bezeichnen, aber diese Schreibweise ist noch mehrdeutiger, da es keinen ausdrücklichen Hinweis darauf gibt, was ersetzt wird. 
Man beachte, dass unsere Definition auch dann gültig ist, wenn $x$ und $y$ die gleiche Variable repräsentieren.  Die erste Konjunktion in der folgenden formalen Definition ist ein "`Trick"', um diese Eigenschaft zu erreichen, was die Definition auf den ersten Blick etwas merkwürdig erscheinen lässt.

\vskip 0.5ex
\setbox\startprefix=\hbox{\tt \ \ df-sb\ \$a\ }
\setbox\contprefix=\hbox{\tt \ \ \ \ \ \ \ \ \ \ \ }
\startm
\m{\vdash}\m{(}\m{[}\m{y}\m{/}\m{x}\m{]}\m{\varphi}\m{\leftrightarrow}\m{(}%
\m{(}\m{x}\m{=}\m{y}\m{\rightarrow}\m{\varphi}\m{)}\m{\wedge}\m{\exists}\m{x}%
\m{(}\m{x}\m{=}\m{y}\m{\wedge}\m{\varphi}\m{)}\m{)}\m{)}
\endm
\begin{mmraw}%
|- ( [ y / x ] ph <-> ( ( x = y -> ph ) \TAND E. x ( x = y \TAND ph ) ) ) \$.
\end{mmraw}


\noindent Definition der {\bf existentiellen Eindeutigkeit}\index{Eindeutigkeitsquantor ($\exists "!$)} ("`Es existiert genau einer"').

Man beachte, dass $y$ eine Variable ist, die sich von $x$ unterscheidet und nicht in $\varphi$ vorkommt.\label{df-eu}

\vskip 0.5ex
\setbox\startprefix=\hbox{\tt \ \ df-eu\ \$a\ }
\setbox\contprefix=\hbox{\tt \ \ \ \ \ \ \ \ \ \ \ }
\startm
\m{\vdash}\m{(}\m{\exists}\m{{!}}\m{x}\m{\varphi}\m{\leftrightarrow}\m{\exists}
\m{y}\m{\forall}\m{x}\m{(}\m{\varphi}\m{\leftrightarrow}\m{x}\m{=}\m{y}\m{)}\m{)}
\endm

\begin{mmraw}%
|- ( E! x ph <-> E. y A. x ( ph <-> x = y ) ) \$.
\end{mmraw}

\subsection{Definitionen für die Mengenlehre}\label{setdefinitions}

Die Symbole $x$, $y$, $z$ und $w$ stellen individuelle Variablen der Prädikatenlogik dar, die in der Mengenlehre als Mengen verstanden werden. Allerdings wäre es sehr unpraktisch, nur die bisher gezeigten Konstrukte zu verwenden.

Um die Mengenlehre praktikabler zu machen, führen wir den Begriff der "`Klasse"' ein. Eine Klasse\index{Klasse} ist entweder eine Mengenvariable (wie $x$) oder ein Ausdruck der Form $\{ x | \varphi\}$ (genannt eine "`Abstraktionsklasse"'\index{Abstraktionsklasse}\index{Klassenabstraktion}).  Man beachte, dass Mengen (d.h. \ individuelle Variablen) immer existieren (dies ist ein Satz der Logik, nämlich $\exists y \, y = x$ für jede beliebige Menge $x$), während Klassen existieren können oder nicht (d.h. \ $\exists y \, y = A$ kann wahr sein oder nicht). Wenn eine Klasse nicht existiert, wird sie als "`echte Klasse"'\index{echte Klasse}\index{Klasse!echte} bezeichnet. Die Definitionen \texttt{df-clab}, \texttt{df-cleq} und \texttt{df-clel} können verwendet werden, um einen Klassen enthaltenden Ausdruck in einen Ausdruck umzuwandeln, der nur Mengenvariablen und wff-Metavariablen enthält.

Die Symbole $A$, $B$, $C$, $D$, $F$, $G$ und $R$ sind Metavariablen, die sich über Klassen erstrecken.  Eine Klassenmetavariable $A$ kann aus einer wff eliminiert werden, indem sie durch $\{ x|\varphi\}$ ersetzt wird, wobei weder $x$ noch $\varphi$ in der wff vorkommen.

Die Klassentheorie erweist sich als eliminierbare und konservative Erweiterung der Mengenlehre. Die Eigenschaft der \textbf{Eliminierbarkeit} bedeutet, dass wir für jede Formel in der erweiterten Sprache eine logisch äquivalente Formel in der Basissprache bilden können. Auch wenn die erweiterte Sprache die Vermittlung und Formulierung mathematischer Ideen für die Mengenlehre erleichtert, stärkt ihre Ausdruckskraft nicht die Ausdruckskraft der Basissprache. Die Eigenschaft der \textbf{Konservativität} bedeutet, dass wir für jeden Beweis einer Formel der Basissprache, der im erweiterten System geführt wird, einen anderen Beweis derselben Formel, der ausschließlich im Basissystem geführt wird, konstruieren können; so dass die deduktiven Möglichkeiten des erweiterten Systems und des Basissystems identisch sind, wenn es nur um Theoreme über Mengen geht. Zusammen bedeuten diese Eigenschaften, dass die erweiterte Sprache als eine definitorische Erweiterung behandelt werden kann, die \textbf{gesund} ist.

Eine strenge Begründung, die wir hier nicht geben werden, findet sich bei Levy \cite[pp.~357-366]{Levy}, der seine informelle Einführung in die Klassentheorie auf S.~7-17 ergänzt. Zwei weitere gute Abhandlungen der Klassentheorie finden sich bei Quine \cite[pp.~15-21]{Quine}\index{Quine, Willard Van Orman} und auch bei \cite[pp.~10-14]{Takeuti}\index{Takeuti, Gaisi}. Quines Ausführungen (er nennt sie virtuelle Klassen) sind elegant geschrieben und sehr lesenswert.

Im weiteren Verlauf dieses Abschnitts wird immer davon ausgegangen, dass die einzelnen Variablen voneinander verschieden sind, sofern nicht anders angegeben.  Darüber hinaus kommen Dummy-Variablen auf der rechten Seite einer Definition nicht in den zu definierenden Metavariablen für Klassen und wffs vor.

Die hier vorgestellten Definitionen sind eine unvollständige, aber in sich geschlossene Auswahl aus mehreren hundert Definitionen, die in der aktuellen Datenbasis \texttt{set.mm} enthalten sind.  Sie sind ausreichend für eine grundlegende Herleitung der elementaren Mengenlehre.

\vskip 2ex
\noindent Definition einer {\bf Abstraktionsklasse}.\index{Abstraktionsklasse}\index{Klassenabstraktion}\label{df-clab}  $x$ und $y$ müssen nicht verschieden sein.  Definition 2.1 von Quine, S.~16.  Diese Definition mag rätselhaft erscheinen, da sie kürzer ist als der zu definierende Ausdruck und uns in Bezug auf die Kürze keinen Vorteil bringt.  Warum wir diese Definition einführen, ist dadurch begründet, dass sie gut zu der von \texttt{df-clel} bereitgestellten Erweiterung des $\in$-Symbols passt.

\vskip 0.5ex
\setbox\startprefix=\hbox{\tt \ \ df-clab\ \$a\ }
\setbox\contprefix=\hbox{\tt \ \ \ \ \ \ \ \ \ \ \ \ \ }
\startm
\m{\vdash}\m{(}\m{x}\m{\in}\m{\{}\m{y}\m{|}\m{\varphi}\m{\}}\m{%
\leftrightarrow}\m{[}\m{x}\m{/}\m{y}\m{]}\m{\varphi}\m{)}
\endm
\begin{mmraw}%
|- ( x e. \{ y | ph \} <-> [ x / y ] ph ) \$.
\end{mmraw}

\noindent Definition des {\bf Gleichheitszeichen zwischen Klassen}\index{Klassengleichheit}\label{df-cleq}.  Siehe Quine oder Kapitel 4 von Takeuti und Zaring für die Rechtfertigung und die Methoden zu ihrer Eliminierung.  Dies ist ein Beispiel für eine etwas "`gefährliche"' Definition, denn sie erweitert die Verwendung des bestehenden Gleichheitssymbols, anstatt ein neues Symbol einzuführen, und erlaubt uns, in der ursprünglichen Sprache Aussagen zu machen, die möglicherweise nicht wahr sind. Zum Beispiel erlaubt sie uns, $y = z \leftrightarrow \forall x ( x \in y \leftrightarrow x \in z )$ abzuleiten, was kein Satz der Logik ist, sondern das Extensionalitätsaxiom\index{Extensionalitätsaxiom} voraussetzt, das wir als Hypothese einfügen, damit wir wissen, wann dieses Axiom in einem Beweis vorausgesetzt wird (mit dem Befehl \texttt{show trace{\char`\_}back}).  Wir könnten die Gefahr vermeiden, indem wir ein anderes Symbol, sagen wir $\eqcirc$, anstelle von $=$ einführen; dies hätte auch den Vorteil, dass die Definition einfach zu eliminieren wäre und die Notwendigkeit der Extensionalität als Hypothese entfiele.  Wir hätten dann auch den Vorteil, dass wir genau feststellen könnten, wo die Extensionalität wirklich ins Spiel kommt.  Eines unserer Theoreme wäre $x \eqcirc y \leftrightarrow x = y$, indem wir uns auf die Extensionalität berufen.  In Übereinstimmung mit der üblichen Praxis behalten wir jedoch die "`gefährliche"' Definition bei.

\vskip 0.5ex
\setbox\startprefix=\hbox{\tt \ \ df-cleq.1\ \$e\ }
\setbox\contprefix=\hbox{\tt \ \ \ \ \ \ \ \ \ \ \ \ \ \ \ }
\startm
\m{\vdash}\m{(}\m{\forall}\m{x}\m{(}\m{x}\m{\in}\m{y}\m{\leftrightarrow}\m{x}
\m{\in}\m{z}\m{)}\m{\rightarrow}\m{y}\m{=}\m{z}\m{)}
\endm
\setbox\startprefix=\hbox{\tt \ \ df-cleq\ \$a\ }
\setbox\contprefix=\hbox{\tt \ \ \ \ \ \ \ \ \ \ \ \ \ }
\startm
\m{\vdash}\m{(}\m{A}\m{=}\m{B}\m{\leftrightarrow}\m{\forall}\m{x}\m{(}\m{x}\m{
\in}\m{A}\m{\leftrightarrow}\m{x}\m{\in}\m{B}\m{)}\m{)}
\endm
% We need to reset the startprefix and contprefix.
\setbox\startprefix=\hbox{\tt \ \ df-cleq.1\ \$e\ }
\setbox\contprefix=\hbox{\tt \ \ \ \ \ \ \ \ \ \ \ \ \ \ \ }
\begin{mmraw}%
|- ( A. x ( x e. y <-> x e. z ) -> y = z ) \$.
\end{mmraw}
\setbox\startprefix=\hbox{\tt \ \ df-cleq\ \$a\ }
\setbox\contprefix=\hbox{\tt \ \ \ \ \ \ \ \ \ \ \ \ \ }
\begin{mmraw}%
|- ( A = B <-> A. x ( x e. A <-> x e. B ) ) \$.
\end{mmraw}

\noindent Definition des {\bf Elementprädikates zwischen Klassen}\index{Klassenzugehörigkeit}.  Theorem 6.3 von Quine, S.~41, das wir als Definition übernehmen. Man beachte, dass er die Verwendung des bestehenden Zugehörigkeitssymbols erweitert, aber im Gegensatz zu {\tt df-cleq} nicht die Menge der gültigen wffs der Logik erweitert, wenn die Klassenmetavariablen durch Mengenvariablen ersetzt werden.\label{dfclel}\label{df-clel}

\vskip 0.5ex
\setbox\startprefix=\hbox{\tt \ \ df-clel\ \$a\ }
\setbox\contprefix=\hbox{\tt \ \ \ \ \ \ \ \ \ \ \ \ \ }
\startm
\m{\vdash}\m{(}\m{A}\m{\in}\m{B}\m{\leftrightarrow}\m{\exists}\m{x}\m{(}\m{x}
\m{=}\m{A}\m{\wedge}\m{x}\m{\in}\m{B}\m{)}\m{)}
\endm
\begin{mmraw}%
|- ( A e. B <-> E. x ( x = A \TAND x e. B ) ) \$.?
\end{mmraw}

\noindent Definition der {\bf Ungleichheit}.

\vskip 0.5ex
\setbox\startprefix=\hbox{\tt \ \ df-ne\ \$a\ }
\setbox\contprefix=\hbox{\tt \ \ \ \ \ \ \ \ \ \ \ }
\startm
\m{\vdash}\m{(}\m{A}\m{\ne}\m{B}\m{\leftrightarrow}\m{\lnot}\m{A}\m{=}\m{B}%
\m{)}
\endm
\begin{mmraw}%
|- ( A =/= B <-> -. A = B ) \$.
\end{mmraw}

\noindent Definition der {\bf eingeschränkten Allquantifizierung}.\index{Allquantor ($\forall$)!eingeschränkt}  Enderton, S.~22.

\vskip 0.5ex
\setbox\startprefix=\hbox{\tt \ \ df-ral\ \$a\ }
\setbox\contprefix=\hbox{\tt \ \ \ \ \ \ \ \ \ \ \ \ }
\startm
\m{\vdash}\m{(}\m{\forall}\m{x}\m{\in}\m{A}\m{\varphi}\m{\leftrightarrow}\m{%
\forall}\m{x}\m{(}\m{x}\m{\in}\m{A}\m{\rightarrow}\m{\varphi}\m{)}\m{)}
\endm
\begin{mmraw}%
|- ( A. x e. A ph <-> A. x ( x e. A -> ph ) ) \$.
\end{mmraw}

\noindent Definition der {\bf eingeschränkten Existenzquantifizierung}.\index{Existenzquantor ($\exists$)!eingeschränkt}  Enderton, S.~22.

\vskip 0.5ex
\setbox\startprefix=\hbox{\tt \ \ df-rex\ \$a\ }
\setbox\contprefix=\hbox{\tt \ \ \ \ \ \ \ \ \ \ \ \ }
\startm
\m{\vdash}\m{(}\m{\exists}\m{x}\m{\in}\m{A}\m{\varphi}\m{\leftrightarrow}\m{%
\exists}\m{x}\m{(}\m{x}\m{\in}\m{A}\m{\wedge}\m{\varphi}\m{)}\m{)}
\endm
\begin{mmraw}%
|- ( E. x e. A ph <-> E. x ( x e. A \TAND ph ) ) \$.
\end{mmraw}

\noindent Definition der {\bf universellen Klasse}\index{universelle Klasse ($V$)}.  Definition 5.20, S.~21, von Takeuti und Zaring.\label{df-v}

\vskip 0.5ex
\setbox\startprefix=\hbox{\tt \ \ df-v\ \$a\ }
\setbox\contprefix=\hbox{\tt \ \ \ \ \ \ \ \ \ \ }
\startm
\m{\vdash}\m{{\rm V}}\m{=}\m{\{}\m{x}\m{|}\m{x}\m{=}\m{x}\m{\}}
\endm
\begin{mmraw}%
|- {\char`\_}V = \{ x | x = x \} \$.
\end{mmraw}

\noindent Definition der {\bf Unterklassen\index{Unterklasse}\index{Teilmenge}-Beziehung zwischen zwei Klassen} (die so genannte Untermengen-Beziehung, wenn die Klassen Mengen sind, d. h.\ keine echten Klassen).
Definition 5.9 von Takeuti und Zaring, S.~17.\label{df-ss}

\vskip 0.5ex
\setbox\startprefix=\hbox{\tt \ \ df-ss\ \$a\ }
\setbox\contprefix=\hbox{\tt \ \ \ \ \ \ \ \ \ \ \ }
\startm
\m{\vdash}\m{(}\m{A}\m{\subseteq}\m{B}\m{\leftrightarrow}\m{\forall}\m{x}\m{(}
\m{x}\m{\in}\m{A}\m{\rightarrow}\m{x}\m{\in}\m{B}\m{)}\m{)}
\endm
\begin{mmraw}%
|- ( A C\_ B <-> A. x ( x e. A -> x e. B ) ) \$.
\end{mmraw}

\noindent Definition der {\bf Vereinigung\index{Vereinigung} von zwei Klassen}.  Definition 5.6 von Takeuti und Zaring, S.~16.\label{df-un}

\vskip 0.5ex
\setbox\startprefix=\hbox{\tt \ \ df-un\ \$a\ }
\setbox\contprefix=\hbox{\tt \ \ \ \ \ \ \ \ \ \ \ }
\startm
\m{\vdash}\m{(}\m{A}\m{\cup}\m{B}\m{)}\m{=}\m{\{}\m{x}\m{|}\m{(}\m{x}\m{\in}
\m{A}\m{\vee}\m{x}\m{\in}\m{B}\m{)}\m{\}}
\endm
\begin{mmraw}%
( A u. B ) = \{ x | ( x e. A \TOR x e. B ) \} \$.
\end{mmraw}

\noindent Definition des {\bf Schnitts\index{Schnittmenge} zwischen zwei Klassen}.  Definition 5.6 von Takeuti und Zaring, S.~16.\label{df-in}

\vskip 0.5ex
\setbox\startprefix=\hbox{\tt \ \ df-in\ \$a\ }
\setbox\contprefix=\hbox{\tt \ \ \ \ \ \ \ \ \ \ \ }
\startm
\m{\vdash}\m{(}\m{A}\m{\cap}\m{B}\m{)}\m{=}\m{\{}\m{x}\m{|}\m{(}\m{x}\m{\in}
\m{A}\m{\wedge}\m{x}\m{\in}\m{B}\m{)}\m{\}}
\endm
% Caret ^ requires special treatment
\begin{mmraw}%
|- ( A i\^{}i B ) = \{ x | ( x e. A \TAND x e. B ) \} \$.
\end{mmraw}

\noindent Definition der {\bf Klassendifferenz}\index{Klassendifferenz}\index{Mengendifferenz}. Definition 5.12 von Takeuti und Zaring, S.~20.  In der Literatur werden verschiedene Schreibweisen verwendet; wir haben die Konvention $\setminus$ anstelle eines Minuszeichens gewählt, um letzteres für die spätere Verwendung z.B. in der Arithmetik zu reservieren.\label{df-dif}

\vskip 0.5ex
\setbox\startprefix=\hbox{\tt \ \ df-dif\ \$a\ }
\setbox\contprefix=\hbox{\tt \ \ \ \ \ \ \ \ \ \ \ \ }
\startm
\m{\vdash}\m{(}\m{A}\m{\setminus}\m{B}\m{)}\m{=}\m{\{}\m{x}\m{|}\m{(}\m{x}\m{
\in}\m{A}\m{\wedge}\m{\lnot}\m{x}\m{\in}\m{B}\m{)}\m{\}}
\endm
\begin{mmraw}%
( A \SLASH B ) = \{ x | ( x e. A \TAND -. x e. B ) \} \$.
\end{mmraw}

\noindent Definition der {\bf leeren Menge}\index{leere Menge}. Vergleiche Definition 5.14 von Takeuti und Zaring, S.~20.\label{df-nul}

\vskip 0.5ex
\setbox\startprefix=\hbox{\tt \ \ df-nul\ \$a\ }
\setbox\contprefix=\hbox{\tt \ \ \ \ \ \ \ \ \ \ }
\startm
\m{\vdash}\m{\varnothing}\m{=}\m{(}\m{{\rm V}}\m{\setminus}\m{{\rm V}}\m{)}
\endm
\begin{mmraw}%
|- (/) = ( {\char`\_}V \SLASH {\char`\_}V ) \$.
\end{mmraw}

\noindent Definition der {\bf Potenzklasse}\index{Potenzmenge}\index{Potenzklasse}.  Definition 5.10 von Takeuti und Zaring, S.~17, aber wir lassen sie auch für echte Klassen gelten.  (Beachten Sie, dass \verb$~P$ das Symbol für das kalligraphische P ist, wobei die Tilde für "`lockig"' steht; siehe Anhang~\ref{ASCII}.)\label{df-pw}

\vskip 0.5ex
\setbox\startprefix=\hbox{\tt \ \ df-pw\ \$a\ }
\setbox\contprefix=\hbox{\tt \ \ \ \ \ \ \ \ \ \ \ }
\startm
\m{\vdash}\m{{\cal P}}\m{A}\m{=}\m{\{}\m{x}\m{|}\m{x}\m{\subseteq}\m{A}\m{\}}
\endm
% Special incantation required to put ~ into the text
\begin{mmraw}%
|- \char`\~P~A = \{ x | x C\_ A \} \$.
\end{mmraw}

\noindent Definition einer {\bf einelementigen Klasse (Singleton)}\index{Singleton}.  Definition 7.1 von Quine, S.~48.  Sie ist auch für echte Klassen wohldefiniert, obwohl sie in diesem Fall nicht sehr aussagekräftig ist, da sie zur leeren Menge ausgewertet wird.

\vskip 0.5ex
\setbox\startprefix=\hbox{\tt \ \ df-sn\ \$a\ }
\setbox\contprefix=\hbox{\tt \ \ \ \ \ \ \ \ \ \ \ }
\startm
\m{\vdash}\m{\{}\m{A}\m{\}}\m{=}\m{\{}\m{x}\m{|}\m{x}\m{=}\m{A}\m{\}}
\endm
\begin{mmraw}%
|- \{ A \} = \{ x | x = A \} \$.
\end{mmraw}%

\noindent Definition eines {\bf ungeordneten Klassenpaares}\index{ungeordnetes Paar}\index{Paar}.  Definition 7.1 von Quine, S.~48.

\vskip 0.5ex
\setbox\startprefix=\hbox{\tt \ \ df-pr\ \$a\ }
\setbox\contprefix=\hbox{\tt \ \ \ \ \ \ \ \ \ \ \ }
\startm
\m{\vdash}\m{\{}\m{A}\m{,}\m{B}\m{\}}\m{=}\m{(}\m{\{}\m{A}\m{\}}\m{\cup}\m{\{}
\m{B}\m{\}}\m{)}
\endm
\begin{mmraw}%
|- \{ A , B \} = ( \{ A \} u. \{ B \} ) \$.
\end{mmraw}

\noindent Definition eines {\bf ungeordneten Klassentripels}\index{ungeordnetes Tripel}.  Definition von Enderton, S.~19.

\vskip 0.5ex
\setbox\startprefix=\hbox{\tt \ \ df-tp\ \$a\ }
\setbox\contprefix=\hbox{\tt \ \ \ \ \ \ \ \ \ \ \ }
\startm
\m{\vdash}\m{\{}\m{A}\m{,}\m{B}\m{,}\m{C}\m{\}}\m{=}\m{(}\m{\{}\m{A}\m{,}\m{B}
\m{\}}\m{\cup}\m{\{}\m{C}\m{\}}\m{)}
\endm
\begin{mmraw}%
|- \{ A , B , C \} = ( \{ A , B \} u. \{ C \} ) \$.
\end{mmraw}%

\noindent Definition von Kuratowskis\index{Kuratowski, Kazimierz} {\bf geordneten Klassenpaares}\index{geordnetes Paar}-Definition.  Definition 9.1 von Quine, S.~58. Für echte Klassen ist sie nicht sinnvoll, aber der Einfachheit halber wohldefiniert.  (Man beachte, dass \verb$<.$ für $\langle$ steht, während \verb$<$ für $<$ steht.)\label{df-op}

\vskip 0.5ex
\setbox\startprefix=\hbox{\tt \ \ df-op\ \$a\ }
\setbox\contprefix=\hbox{\tt \ \ \ \ \ \ \ \ \ \ \ }
\startm
\m{\vdash}\m{\langle}\m{A}\m{,}\m{B}\m{\rangle}\m{=}\m{\{}\m{\{}\m{A}\m{\}}
\m{,}\m{\{}\m{A}\m{,}\m{B}\m{\}}\m{\}}
\endm
\begin{mmraw}%
|- <. A , B >. = \{ \{ A \} , \{ A , B \} \} \$.
\end{mmraw}

\noindent Definition der {\bf Vereinigung einer Klasse}\index{Vereinigung}.  Definition 5.5, S.~16, von Takeuti und Zaring.

\vskip 0.5ex
\setbox\startprefix=\hbox{\tt \ \ df-uni\ \$a\ }
\setbox\contprefix=\hbox{\tt \ \ \ \ \ \ \ \ \ \ \ \ }
\startm
\m{\vdash}\m{\bigcup}\m{A}\m{=}\m{\{}\m{x}\m{|}\m{\exists}\m{y}\m{(}\m{x}\m{
\in}\m{y}\m{\wedge}\m{y}\m{\in}\m{A}\m{)}\m{\}}
\endm
\begin{mmraw}%
|- U. A = \{ x | E. y ( x e. y \TAND y e. A ) \} \$.
\end{mmraw}

\noindent Definition des {\bf Schnittes\index{Schnittmenge} einer Klasse}.  Definition 7.35, S.~44, von Takeuti und Zaring.

\vskip 0.5ex
\setbox\startprefix=\hbox{\tt \ \ df-int\ \$a\ }
\setbox\contprefix=\hbox{\tt \ \ \ \ \ \ \ \ \ \ \ \ }
\startm
\m{\vdash}\m{\bigcap}\m{A}\m{=}\m{\{}\m{x}\m{|}\m{\forall}\m{y}\m{(}\m{y}\m{
\in}\m{A}\m{\rightarrow}\m{x}\m{\in}\m{y}\m{)}\m{\}}
\endm
\begin{mmraw}%
|- |\^{}| A = \{ x | A. y ( y e. A -> x e. y ) \} \$.
\end{mmraw}

\noindent Definition einer {\bf transitiven Klasse}\index{transitive Klasse}\index{transitive Menge}.  Dies sollte nicht mit einer transitiven Beziehung verwechselt werden, die ein anderes Konzept ist.  Definition aus S.~71 von Enderton, erweitert auf Klassen.

\vskip 0.5ex
\setbox\startprefix=\hbox{\tt \ \ df-tr\ \$a\ }
\setbox\contprefix=\hbox{\tt \ \ \ \ \ \ \ \ \ \ \ }
\startm
\m{\vdash}\m{(}\m{\mbox{\rm Tr}}\m{A}\m{\leftrightarrow}\m{\bigcup}\m{A}\m{
\subseteq}\m{A}\m{)}
\endm
\begin{mmraw}%
|- ( Tr A <-> 
U. A C\_ A ) \$.
\end{mmraw}
\noindent Definition einer Notation für eine {\bf allgemeine binäre Relation}\index{binäre Relation}.  Definition 6.18, S.~29, von Takeuti und Zaring, verallgemeinert auf beliebige Klassen.  Diese Definition ist wohldefiniert, wenn auch nicht sehr aussagekräftig, wenn die Klassen $A$ und/oder $B$ echte Klassen sind.\label{dfbr}  Das Fehlen von Klammern (oder eines anderen Konnektors) erzeugt keine Mehrdeutigkeit, da wir eine atomare wff definieren.

\vskip 0.5ex
\setbox\startprefix=\hbox{\tt \ \ df-br\ \$a\ }
\setbox\contprefix=\hbox{\tt \ \ \ \ \ \ \ \ \ \ \ }
\startm
\m{\vdash}\m{(}\m{A}\m{\,R}\m{\,B}\m{\leftrightarrow}\m{\langle}\m{A}\m{,}\m{B}
\m{\rangle}\m{\in}\m{R}\m{)}
\endm
\begin{mmraw}%
|- ( A R B <-> <. A , B >. e. R ) \$.
\end{mmraw}

\noindent Definition einer {\bf Abstraktionsklasse von geordneten Paaren} \index{Abstraktionsklasse!von geordneten Paaren}.  Ein Spezialfall der Definition 4.16, S.~14, von Takeuti und Zaring. Man beachte, dass $ z $ von $ x $ und $ y $ verschieden sein muss und $ z $ nicht in $\varphi$ vorkommen darf, aber $ x $ und $ y $ können identisch sein und in $\varphi$ vorkommen.

\vskip 0.5ex
\setbox\startprefix=\hbox{\tt \ \ df-opab\ \$a\ }
\setbox\contprefix=\hbox{\tt \ \ \ \ \ \ \ \ \ \ \ \ \ }
\startm
\m{\vdash}\m{\{}\m{\langle}\m{x}\m{,}\m{y}\m{\rangle}\m{|}\m{\varphi}\m{\}}\m{=}
\m{\{}\m{z}\m{|}\m{\exists}\m{x}\m{\exists}\m{y}\m{(}\m{z}\m{=}\m{\langle}\m{x}
\m{,}\m{y}\m{\rangle}\m{\wedge}\m{\varphi}\m{)}\m{\}}
\endm

\begin{mmraw}%
|- \{ <. x , y >. | ph \} = \{ z | E. x E. y ( z =
<. x , y >. \TAND ph ) \} \$.
\end{mmraw}

\noindent Definition der {\bf Epsilon-Relation}\index{Epsilon-Relation}.  Ähnlich der Definition 6.22, S.~30, von Takeuti und Zaring.

\vskip 0.5ex
\setbox\startprefix=\hbox{\tt \ \ df-eprel\ \$a\ }
\setbox\contprefix=\hbox{\tt \ \ \ \ \ \ \ \ \ \ \ \ \ \ }
\startm
\m{\vdash}\m{{\rm E}}\m{=}\m{\{}\m{\langle}\m{x}\m{,}\m{y}\m{\rangle}\m{|}\m{x}\m{
\in}\m{y}\m{\}}
\endm
\begin{mmraw}%
|- \_E = \{ <. x , y >. | x e. y \} \$.
\end{mmraw}

\noindent Definition einer {\b fundierten Relation}\index{fundierte Relation}.  $R$ ist eine fundierte Relation auf $A$, genau dann, wenn (wenn und nur dann, wenn) jede nichtleere Teilmenge von $A$ ein "`$R$-minimales Element"' hat.  Ähnlich der Definition 6.21, S.~30, von Takeuti und Zaring.

\vskip 0.5ex
\setbox\startprefix=\hbox{\tt \ \ df-fr\ \$a\ }
\setbox\contprefix=\hbox{\tt \ \ \ \ \ \ \ \ \ \ \ }
\startm
\m{\vdash}\m{(}\m{R}\m{\,\mbox{\rm Fr}}\m{\,A}\m{\leftrightarrow}\m{\forall}\m{x}
\m{(}\m{(}\m{x}\m{\subseteq}\m{A}\m{\wedge}\m{\lnot}\m{x}\m{=}\m{\varnothing}
\m{)}\m{\rightarrow}\m{\exists}\m{y}\m{(}\m{y}\m{\in}\m{x}\m{\wedge}\m{(}\m{x}
\m{\cap}\m{\{}\m{z}\m{|}\m{z}\m{\,R}\m{\,y}\m{\}}\m{)}\m{=}\m{\varnothing}\m{)}
\m{)}\m{)}
\endm
\begin{mmraw}%
|- ( R Fr A <-> A. x ( ( x C\_ A \TAND -. x = (/) ) ->
E. y ( y e. x \TAND ( x i\^{}i \{ z | z R y \} ) = (/) ) ) ) \$.
\end{mmraw}

\noindent Definition einer {\bf Wohlordnung}\index{Wohlordnung}.  $R$ ist eine Wohlordnung von $A$ genau dann, wenn sie auf $A$ fundiert ist und die Elemente von $A$ paarweise $R$-vergleichbar sind. Ähnlich der Definition 6.24(2), S.~30, von Takeuti und Zaring.

\vskip 0.5ex
\setbox\startprefix=\hbox{\tt \ \ df-we\ \$a\ }
\setbox\contprefix=\hbox{\tt \ \ \ \ \ \ \ \ \ \ \ }
\startm
\m{\vdash}\m{(}\m{R}\m{\,\mbox{\rm We}}\m{\,A}\m{\leftrightarrow}\m{(}\m{R}\m{\,
\mbox{\rm Fr}}\m{\,A}\m{\wedge}\m{\forall}\m{x}\m{\forall}\m{y}\m{(}\m{(}\m{x}\m{
\in}\m{A}\m{\wedge}\m{y}\m{\in}\m{A}\m{)}\m{\rightarrow}\m{(}\m{x}\m{\,R}\m{\,y}
\m{\vee}\m{x}\m{=}\m{y}\m{\vee}\m{y}\m{\,R}\m{\,x}\m{)}\m{)}\m{)}\m{)}
\endm
\begin{mmraw}%
( R We A <-> ( R Fr A \TAND A. x A. y ( ( x e.
A \TAND y e. A ) -> ( x R y \TOR x = y \TOR y R x ) ) ) ) \$.
\end{mmraw}

\noindent Definition des {\bf Ordinalprädikats}\index{Ordinalprädikat}, das für eine Klasse gilt, die transitiv ist und durch die Epsilon-Relation wohlgeordnet ist.  Ähnlich der Definition auf S.~468, Bell und Machover.

\vskip 0.5ex
\setbox\startprefix=\hbox{\tt \ \ df-ord\ \$a\ }
\setbox\contprefix=\hbox{\tt \ \ \ \ \ \ \ \ \ \ \ \ }
\startm
\m{\vdash}\m{(}\m{\mbox{\rm Ord}}\m{\,A}\m{\leftrightarrow}\m{(}
\m{\mbox{\rm Tr}}\m{\,A}\m{\wedge}\m{E}\m{\,\mbox{\rm We}}\m{\,A}\m{)}\m{)}
\endm
\begin{mmraw}%
|- ( Ord A <-> ( Tr A \TAND E We A ) ) \$.
\end{mmraw}

\noindent Definition der {\bf Klasse aller Ordinalzahlen}\index{Ordinalzahl}.  Eine Ordinalzahl ist eine Menge, die das Ordinalprädikat erfüllt.  Definition 7.11 von Takeuti und Zaring, S.~38.

\vskip 0.5ex
\setbox\startprefix=\hbox{\tt \ \ df-on\ \$a\ }
\setbox\contprefix=\hbox{\tt \ \ \ \ \ \ \ \ \ \ \ }
\startm
\m{\vdash}\m{\,\mbox{\rm On}}\m{=}\m{\{}\m{x}\m{|}\m{\mbox{\rm Ord}}\m{\,x}
\m{\}}
\endm
\begin{mmraw}%
|- On = \{ x | Ord x \} \$.
\end{mmraw}

\noindent Definition des {\bf Limes-Prädikats}, das für eine nicht leere Ordinalzahl gilt, die kein Nachfolger ist (d.h. \ die die Vereinigung ihrer selbst ist)\index{Grenzzahl}. Vergleiche Bell und Machover, S.~471 und Übung (1), S.~42 von Takeuti und Zaring.

\vskip 0.5ex
\setbox\startprefix=\hbox{\tt \ \ df-lim\ \$a\ }
\setbox\contprefix=\hbox{\tt \ \ \ \ \ \ \ \ \ \ \ \ }
\startm
\m{\vdash}\m{(}\m{\mbox{\rm Lim}}\m{\,A}\m{\leftrightarrow}\m{(}\m{\mbox{
\rm Ord}}\m{\,A}\m{\wedge}\m{\lnot}\m{A}\m{=}\m{\varnothing}\m{\wedge}\m{A}
\m{=}\m{\bigcup}\m{A}\m{)}\m{)}
\endm
\begin{mmraw}%
|- ( Lim A <-> ( Ord A \TAND -. A = (/) \TAND A = U. A ) ) \$.
\end{mmraw}

\noindent Definition eines {\bf Nachfolgers}\index{Nachfolger} einer Klasse.  Definition 7.22 von Takeuti und Zaring, S.~41.  Unsere Definition ist eine Verallgemeinerung auf Klassen, obwohl sie für echte Klassen bedeutungslos ist.

\vskip 0.5ex
\setbox\startprefix=\hbox{\tt \ \ df-suc\ \$a\ }
\setbox\contprefix=\hbox{\tt \ \ \ \ \ \ \ \ \ \ \ \ }
\startm
\m{\vdash}\m{\,\mbox{\rm suc}}\m{\,A}\m{=}\m{(}\m{A}\m{\cup}\m{\{}\m{A}\m{\}}
\m{)}
\endm
\begin{mmraw}%
|- suc A = ( A u. \{ A \} ) \$.
\end{mmraw}

\noindent Definition der {\bf Klasse der natürlichen Zahlen}\index{natürliche Zahlen}\index{Omega ($\omega$)}.  Vergleiche Bell und Machover, S.~471.\label{dfom}

\vskip 0.5ex
\setbox\startprefix=\hbox{\tt \ \ df-om\ \$a\ }
\setbox\contprefix=\hbox{\tt \ \ \ \ \ \ \ \ \ \ \ }
\startm
\m{\vdash}\m{\omega}\m{=}\m{\{}\m{x}\m{|}\m{(}\m{\mbox{\rm Ord}}\m{\,x}\m{
\wedge}\m{\forall}\m{y}\m{(}\m{\mbox{\rm Lim}}\m{\,y}\m{\rightarrow}\m{x}\m{
\in}\m{y}\m{)}\m{)}\m{\}}
\endm
\begin{mmraw}%
|- om = \{ x | ( Ord x \TAND A. y ( Lim y -> x e. y ) ) \} \$.
\end{mmraw}

\noindent Definition eines {\bf kartesischen Produkts} (auch {\bf Kreuzprodukt}\index{Kartesisches Produkt}\index{Kreuzprodukt} genannt) von zwei Klassen.  Definition 9.11 von Quine, S.~64. 

\vskip 0.5ex
\setbox\startprefix=\hbox{\tt \ \ df-xp\ \$a\ }
\setbox\contprefix=\hbox{\tt \ \ \ \ \ \ \ \ \ \ \ }
\startm
\m{\vdash}\m{(}\m{A}\m{\times}\m{B}\m{)}\m{=}\m{\{}\m{\langle}\m{x}\m{,}\m{y}
\m{\rangle}\m{|}\m{(}\m{x}\m{\in}\m{A}\m{\wedge}\m{y}\m{\in}\m{B}\m{)}\m{\}}
\endm
\begin{mmraw}%
|- ( A X. B ) = \{ <. x , y >. | ( x e. A \TAND y e. B) \} \$.
\end{mmraw}

\noindent Definition einer {\bf Relation}\index{Relation}.  Definition 6.4(1) von Takeuti und Zaring, S.~23.

\vskip 0.5ex
\setbox\startprefix=\hbox{\tt \ \ df-rel\ \$a\ }
\setbox\contprefix=\hbox{\tt \ \ \ \ \ \ \ \ \ \ \ \ }
\startm
\m{\vdash}\m{(}\m{\mbox{\rm Rel}}\m{\,A}\m{\leftrightarrow}\m{A}\m{\subseteq}
\m{(}\m{{\rm V}}\m{\times}\m{{\rm V}}\m{)}\m{)}
\endm
\begin{mmraw}%
|- ( Rel A <-> A C\_ ( {\char`\_}V X. {\char`\_}V ) ) \$.
\end{mmraw}

\noindent Definition eines {\bf Definitionsbereichs}\index{Definitionsbereich} einer Klasse\footnote{Anm. der Übersetzer: Der Begriff "`Definitionsbereich"' und die folgenden Begriffe wie "`Wertebereich"' etc. werden üblicherweise für Funktionen oder zumindest Relationen verwendet, können aber so wie hier für beliebige Klassen definiert werden.}.  Definition 6.5(1) von Takeuti und Zaring, S.~24.

\vskip 0.5ex
\setbox\startprefix=\hbox{\tt \ \ df-dm\ \$a\ }
\setbox\contprefix=\hbox{\tt \ \ \ \ \ \ \ \ \ \ \ }
\startm
\m{\vdash}\m{\,\mbox{\rm dom}}\m{A}\m{=}\m{\{}\m{x}\m{|}\m{\exists}\m{y}\m{
\langle}\m{x}\m{,}\m{y}\m{\rangle}\m{\in}\m{A}\m{\}}
\endm
\begin{mmraw}%
|- dom A = \{ x | E. y <. x , y >. e. A \} \$.
\end{mmraw}

\noindent Definition des {\bf Wertebereichs}\index{Wertebereich} einer Klasse.  Definition 6.5(2) von Takeuti und Zaring, S.~24.

\vskip 0.5ex
\setbox\startprefix=\hbox{\tt \ \ df-rn\ \$a\ }
\setbox\contprefix=\hbox{\tt \ \ \ \ \ \ \ \ \ \ \ }
\startm
\m{\vdash}\m{\,\mbox{\rm ran}}\m{A}\m{=}\m{\{}\m{y}\m{|}\m{\exists}\m{x}\m{
\langle}\m{x}\m{,}\m{y}\m{\rangle}\m{\in}\m{A}\m{\}}
\endm
\begin{mmraw}%
|- ran A = \{ y | E. x <. x , y >. e. A \} \$.
\end{mmraw}

\noindent Definition einer {\bf Einschränkung}\index{Einschränkung} einer Klasse.  Definition 6.6(1) von Takeuti und Zaring, S.~24.

\vskip 0.5ex
\setbox\startprefix=\hbox{\tt \ \ df-res\ \$a\ }
\setbox\contprefix=\hbox{\tt \ \ \ \ \ \ \ \ \ \ \ \ }
\startm
\m{\vdash}\m{(}\m{A}\m{\restriction}\m{B}\m{)}\m{=}\m{(}\m{A}\m{\cap}\m{(}\m{B}
\m{\times}\m{{\rm V}}\m{)}\m{)}
\endm
\begin{mmraw}%
|- ( A |` B ) = ( A i\^{}i ( B X. {\char`\_}V ) ) \$.
\end{mmraw}

\noindent Definition des {\bf Bildes}\index{Bild} einer Klasse.  Definition 6.6(2) von Takeuti und Zaring, S.~24.

\vskip 0.5ex
\setbox\startprefix=\hbox{\tt \ \ df-ima\ \$a\ }
\setbox\contprefix=\hbox{\tt \ \ \ \ \ \ \ \ \ \ \ \ }
\startm
\m{\vdash}\m{(}\m{A}\m{``}\m{B}\m{)}\m{=}\m{\,\mbox{\rm ran}}\m{\,(}\m{A}\m{
\restriction}\m{B}\m{)}
\endm
\begin{mmraw}%
|- ( A "{} B ) = ran ( A |` B ) \$.
\end{mmraw}

\noindent Definition der {\bf Komposition}\index{Komposition} zweier Klassen.  Definition 6.6(3) von Takeuti und Zaring, S.~24.

\vskip 0.5ex
\setbox\startprefix=\hbox{\tt \ \ df-co\ \$a\ }
\setbox\contprefix=\hbox{\tt \ \ \ \ \ \ \ \ \ \ \ \ }
\startm
\m{\vdash}\m{(}\m{A}\m{\circ}\m{B}\m{)}\m{=}\m{\{}\m{\langle}\m{x}\m{,}\m{y}\m{
\rangle}\m{|}\m{\exists}\m{z}\m{(}\m{\langle}\m{x}\m{,}\m{z}\m{\rangle}\m{\in}
\m{B}\m{\wedge}\m{\langle}\m{z}\m{,}\m{y}\m{\rangle}\m{\in}\m{A}\m{)}\m{\}}
\endm
\begin{mmraw}%
|- ( A o. B ) = \{ <. x , y >. | E. z ( <. x , z
>. e. B \TAND <. z , y >. e. A ) \} \$.
\end{mmraw}

\noindent Definition einer {\bf Funktion}\index{Funktion}.  Definition 6.4(4) von Takeuti und Zaring, S.~24.

\vskip 0.5ex
\setbox\startprefix=\hbox{\tt \ \ df-fun\ \$a\ }
\setbox\contprefix=\hbox{\tt \ \ \ \ \ \ \ \ \ \ \ \ }
\startm
\m{\vdash}\m{(}\m{\mbox{\rm Fun}}\m{\,A}\m{\leftrightarrow}\m{(}
\m{\mbox{\rm Rel}}\m{\,A}\m{\wedge}
\m{\forall}\m{x}\m{\exists}\m{z}\m{\forall}\m{y}\m{(}
\m{\langle}\m{x}\m{,}\m{y}\m{\rangle}\m{\in}\m{A}\m{\rightarrow}\m{y}\m{=}\m{z}
\m{)}\m{)}\m{)}
\endm
\begin{mmraw}%
|- ( Fun A <-> ( Rel A \TAND A. x E. z A. y ( <. x
   , y >. e. A -> y = z ) ) ) \$.
\end{mmraw}

\noindent Definition einer {\bf Funktion mit Definitionsbereich}.  Definition 6.15(1) von Takeuti und Zaring, S.~27.

\vskip 0.5ex
\setbox\startprefix=\hbox{\tt \ \ df-fn\ \$a\ }
\setbox\contprefix=\hbox{\tt \ \ \ \ \ \ \ \ \ \ \ }
\startm
\m{\vdash}\m{(}\m{A}\m{\,\mbox{\rm Fn}}\m{\,B}\m{\leftrightarrow}\m{(}
\m{\mbox{\rm Fun}}\m{\,A}\m{\wedge}\m{\mbox{\rm dom}}\m{\,A}\m{=}\m{B}\m{)}
\m{)}
\endm
\begin{mmraw}%
|- ( A Fn B <-> ( Fun A \TAND dom A = B ) ) \$.
\end{mmraw}

\noindent Definition einer {\bf Funktion mit Definitionsbereich und Zielbereich}.  Definition 6.15(3) von Takeuti und Zaring, S.~27.

\vskip 0.5ex
\setbox\startprefix=\hbox{\tt \ \ df-f\ \$a\ }
\setbox\contprefix=\hbox{\tt \ \ \ \ \ \ \ \ \ \ }
\startm
\m{\vdash}\m{(}\m{F}\m{:}\m{A}\m{\longrightarrow}\m{B}\m{
\leftrightarrow}\m{(}\m{F}\m{\,\mbox{\rm Fn}}\m{\,A}\m{\wedge}\m{
\mbox{\rm ran}}\m{\,F}\m{\subseteq}\m{B}\m{)}\m{)}
\endm
\begin{mmraw}%
|- ( F : A --> B <-> ( F Fn A \TAND ran F C\_ B ) ) \$.
\end{mmraw}

\noindent Definition einer {\bf injektiven} oder {\bf Eins-zu-eins-Funktion}\index{injektive Funktion}.  Vergleiche Definition 6.15(5) von Takeuti und Zaring, S.~27.

\vskip 0.5ex
\setbox\startprefix=\hbox{\tt \ \ df-f1\ \$a\ }
\setbox\contprefix=\hbox{\tt \ \ \ \ \ \ \ \ \ \ \ }
\startm
\m{\vdash}\m{(}\m{F}\m{:}\m{A}\m{
\raisebox{.5ex}{${\textstyle{\:}_{\mbox{\footnotesize\rm
1\tt -\rm 1}}}\atop{\textstyle{
\longrightarrow}\atop{\textstyle{}^{\mbox{\footnotesize\rm {\ }}}}}$}
}\m{B}
\m{\leftrightarrow}\m{(}\m{F}\m{:}\m{A}\m{\longrightarrow}\m{B}
\m{\wedge}\m{\forall}\m{y}\m{\exists}\m{z}\m{\forall}\m{x}\m{(}\m{\langle}\m{x}
\m{,}\m{y}\m{\rangle}\m{\in}\m{F}\m{\rightarrow}\m{x}\m{=}\m{z}\m{)}\m{)}\m{)}
\endm
\begin{mmraw}%
|- ( F : A -1-1-> B <-> ( F : A --> B \TAND
   A. y E. z A. x ( <. x , y >. e. F -> x = z ) ) ) \$.
\end{mmraw}

\noindent Definition einer {\bf surjektiven} oder {\bf rechtstotalen Funktion}\index{surjektive Funktion}\footnote{Im Englischen {\bf onto function}.}.  Definition 6.15(4) von Takeuti und Zaring, S.~27.

\vskip 0.5ex
\setbox\startprefix=\hbox{\tt \ \ df-fo\ \$a\ }
\setbox\contprefix=\hbox{\tt \ \ \ \ \ \ \ \ \ \ \ }
\startm
\m{\vdash}\m{(}\m{F}\m{:}\m{A}\m{
\raisebox{.5ex}{${\textstyle{\:}_{\mbox{\footnotesize\rm
{\ }}}}\atop{\textstyle{
\longrightarrow}\atop{\textstyle{}^{\mbox{\footnotesize\rm onto}}}}$}
}\m{B}
\m{\leftrightarrow}\m{(}\m{F}\m{\,\mbox{\rm Fn}}\m{\,A}\m{\wedge}
\m{\mbox{\rm ran}}\m{\,F}\m{=}\m{B}\m{)}\m{)}
\endm
\begin{mmraw}%
|- ( F : A -onto-> B\linebreak
<-> ( F Fn A \TAND ran F = B ) ) \$.
\end{mmraw}

\noindent Definition einer {\bf bijektiven Funktion}\index{bijektive Funktion}.  Vergleiche Definition 6.15(6) von Takeuti und Zaring, S.~27.

\vskip 0.5ex
\setbox\startprefix=\hbox{\tt \ \ df-f1o\ \$a\ }
\setbox\contprefix=\hbox{\tt \ \ \ \ \ \ \ \ \ \ \ \ }
\startm
\m{\vdash}\m{(}\m{F}\m{:}\m{A}
\m{
\raisebox{.5ex}{${\textstyle{\:}_{\mbox{\footnotesize\rm
1\tt -\rm 1}}}\atop{\textstyle{
\longrightarrow}\atop{\textstyle{}^{\mbox{\footnotesize\rm onto}}}}$}
}
\m{B}
\m{\leftrightarrow}\m{(}\m{F}\m{:}\m{A}
\m{
\raisebox{.5ex}{${\textstyle{\:}_{\mbox{\footnotesize\rm
1\tt -\rm 1}}}\atop{\textstyle{
\longrightarrow}\atop{\textstyle{}^{\mbox{\footnotesize\rm {\ }}}}}$}
}
\m{B}\m{\wedge}\m{F}\m{:}\m{A}
\m{
\raisebox{.5ex}{${\textstyle{\:}_{\mbox{\footnotesize\rm
{\ }}}}\atop{\textstyle{
\longrightarrow}\atop{\textstyle{}^{\mbox{\footnotesize\rm onto}}}}$}
}
\m{B}\m{)}\m{)}
\endm
\begin{mmraw}%
|- ( F : A -1-1-onto-> B \linebreak
<-> ( F : A -1-1-> B \TAND F : A -onto-> B ) ) \$.
\end{mmraw}

\noindent Definition eines {\bf Funktionswertes}\index{Funktionswert}.  Diese Definition gilt für jede Klasse und wird zur leeren Menge ausgewertet, wenn sie nicht sinnvoll ist. Beachten Sie, dass $ F`A$ dasselbe bedeutet wie die bekanntere Notation $ F(A)$ für den Wert einer Funktion an der Stelle $A$.  Die Notation $ F`A$ ist in der formalen Mengenlehre gebräuchlich.\label{df-fv}

\vskip 0.5ex
\setbox\startprefix=\hbox{\tt \ \ df-fv\ \$a\ }
\setbox\contprefix=\hbox{\tt \ \ \ \ \ \ \ \ \ \ \ }
\startm
\m{\vdash}\m{(}\m{F}\m{`}\m{A}\m{)}\m{=}\m{\bigcup}\m{\{}\m{x}\m{|}\m{(}\m{F}%
\m{``}\m{\{}\m{A}\m{\}}\m{)}\m{=}\m{\{}\m{x}\m{\}}\m{\}}
\endm
\begin{mmraw}%
|- ( F ` A ) = U. \{ x | ( F "{} \{ A \} ) = \{ x \} \} \$.
\end{mmraw}

\noindent Definition des {\bf Ergebnisses einer Operation}.\index{Operation} Hier ist $F$ eine Operation für zwei Operanden (z. B. $+$ für reelle Zahlen).   Dies ist auch für echte Klassen $A$ und $B$ definiert, auch wenn es in diesem Fall nicht sinnvoll ist\footnote{Anm. der Übersetzer: das Ergebnis ist in diesem Fall wie bei einem Funktionswert die leere Menge.}.  Die Definition kann jedoch zu einem sinnvollen Ergebnis führen, wenn $F$ eine echte Klasse ist.\label{dfopr}

\vskip 0.5ex
\setbox\startprefix=\hbox{\tt \ \ df-opr\ \$a\ }
\setbox\contprefix=\hbox{\tt \ \ \ \ \ \ \ \ \ \ \ \ }
\startm
\m{\vdash}\m{(}\m{A}\m{\,F}\m{\,B}\m{)}\m{=}\m{(}\m{F}\m{`}\m{\langle}\m{A}%
\m{,}\m{B}\m{\rangle}\m{)}
\endm
\begin{mmraw}%
|- ( A F B ) = ( F ` <. A , B >. ) \$.
\end{mmraw}

\section{Tricks des Verfahrens}\label{tricks}

 In der Regel war es unser Ziel, in der Datenbasis \texttt{set.mm}\index{Mengenlehre-Datenbasis (\texttt{set.mm})} die moderne Notation zu verwenden.  In einigen Fällen wurde aber in unkonventioneller Weise von der in den Standardlehrbüchern verwendeten Sprache abgewichen, um deren Weiterentwicklung zu vereinfachen und die Vorteile der Metamath-Sprache besser zu nutzen.  In diesem Abschnitt werden wir einige allgemeine, in \texttt{set.mm} verwendete  Konventionen beschreiben.

\begin{itemize}
\item
Das Drehkreuzsymbol $\vdash$, das "`es ist beweisbar, dass"' bedeutet, ist das erste Token aller Behauptungen und Hypothesen, die keine Syntaxkonstruktionen sind.  Dies ist eine Standardkonvention in der Logik.  (Wir haben dies bereits erwähnt, aber dieses Symbol ist für manche Menschen ohne Logikkenntnisse etwas verstörend.  Es hat keine tiefere Bedeutung, sondern dient nur dazu, Syntaxkonstruktionen von gewöhnlichen mathematischen Aussagen zu unterscheiden).

\item
Eine Annahme der Form

\vskip 1ex
\setbox\startprefix=\hbox{\tt \ \ \ \ \ \ \ \ \ \$e\ }
\setbox\contprefix=\hbox{\tt \ \ \ \ \ \ \ \ \ \ }
\startm
\m{\vdash}\m{(}\m{\varphi}\m{\rightarrow}\m{\forall}\m{x}\m{\varphi}\m{)}
\endm
\vskip 1ex

sollte als "`unter der Annahme, dass die Variable $x$ in wff $\varphi$ (effektiv) nicht frei ist"'\index{effektiv nicht frei} verstanden werden. Wörtlich heißt das: "`Angenommen, es ist beweisbar, dass $\varphi \rightarrow \forall x\, \varphi$."'  Auf diese Weise können wir die Komplexität vermeiden, die mit der Standardbehandlung von freien und gebundenen Variablen verbunden ist. 
%Uncomment this when uncommenting section {formalspec} below
In der Fußnote auf S.~\pageref{effectivelybound} wird dies näher erläutert.

\item
Eine Aussage in einer der Formen

\vskip 1ex
\setbox\startprefix=\hbox{\tt \ \ \ \ \ \ \ \ \ \$a\ }
\setbox\contprefix=\hbox{\tt \ \ \ \ \ \ \ \ \ \ }
\startm
\m{\vdash}\m{(}\m{\lnot}\m{\forall}\m{x}\m{\,x}\m{=}\m{y}\m{\rightarrow}
\m{\ldots}\m{)}
\endm
\setbox\startprefix=\hbox{\tt \ \ \ \ \ \ \ \ \ \$p\ }
\setbox\contprefix=\hbox{\tt \ \ \ \ \ \ \ \ \ \ }
\startm
\m{\vdash}\m{(}\m{\lnot}\m{\forall}\m{x}\m{\,x}\m{=}\m{y}\m{\rightarrow}
\m{\ldots}\m{)}
\endm
\vskip 1ex

sollte als "`Wenn $x$ und $y$ verschiedene Variablen sind, dann..."' verstanden werden.  Mit solch einer Voraussetzung können wir in der frühen Entwicklung der Prädikatenlogik auf die \texttt{\$d}-Anweisung verzichten, so dass Symbolmanipulationen konzeptionell so einfach sind wie in der Aussagenlogik. Sobald die \texttt{\$d}-Anweisung jedoch mehr und mehr zum Einsatz gekommen ist, wird dieses Konstrukt nur noch selten verwendet.

\item
Die Aussage

\vskip 1ex
\setbox\startprefix=\hbox{\tt \ \ \ \ \ \ \ \ \ \$d\ }
\setbox\contprefix=\hbox{\tt \ \ \ \ \ \ \ \ \ \ }
\startm
\m{x}\m{\,y}
\endm
\vskip 1ex

sollte als "`Angenommen $x$ und $y$ sind unterschiedliche Variablen"' verstanden werden.

\item
Die Anweisung

\vskip 1ex
\setbox\startprefix=\hbox{\tt \ \ \ \ \ \ \ \ \ \$d\ }
\setbox\contprefix=\hbox{\tt \ \ \ \ \ \ \ \ \ \ }
\startm
\m{x}\m{\,\varphi}
\endm
\vskip 1ex

sollte als "`angenommen $x$ kommt in $\varphi$ nicht vor"' verstanden werden.

\item
Die Anweisung

\vskip 1ex
\setbox\startprefix=\hbox{\tt \ \ \ \ \ \ \ \ \ \$d\ }
\setbox\contprefix=\hbox{\tt \ \ \ \ \ \ \ \ \ \ }
\startm
\m{x}\m{\,A}
\endm
\vskip 1ex

sollte als "`angenommen, die Variable $x$ kommt in der Klasse $A$ nicht vor"' verstanden werden.

\item
Die folgende Gruppe von Variableneinschränkungen und Hypothesen

\vskip 1ex
\setbox\startprefix=\hbox{\tt \ \ \ \ \ \ \ \ \ \$d\ }
\setbox\contprefix=\hbox{\tt \ \ \ \ \ \ \ \ \ \ }
\startm
\m{x}\m{\,A}
\endm
\setbox\startprefix=\hbox{\tt \ \ \ \ \ \ \ \ \ \$d\ }
\setbox\contprefix=\hbox{\tt \ \ \ \ \ \ \ \ \ \ }
\startm
\m{x}\m{\,\psi}
\endm
\setbox\startprefix=\hbox{\tt \ \ \ \ \ \ \ \ \ \$e\ }
\setbox\contprefix=\hbox{\tt \ \ \ \ \ \ \ \ \ \ }
\startm
\m{\vdash}\m{(}\m{x}\m{=}\m{A}\m{\rightarrow}\m{(}\m{\varphi}\m{\leftrightarrow}
\m{\psi}\m{)}\m{)}
\endm
\vskip 1ex

wird häufig anstelle der expliziten Substitution verwendet, was bedeutet: "`angenommen, $\psi$ ergibt sich aus der echten Substitution von $A$ für $x$ in $\varphi$."'\footnote{Anm. der Übersetzer: dies wird dann eine "`implizite Substitution"' genannt.}  Manchmal wird "`\texttt{\$e} $\vdash ( \psi \rightarrow \forall x \, \psi )$"' anstelle von "`\texttt{\$d} $x\, \psi $"'\footnote{Anm. der Übersetzer: Solche Hypothesen werden neuerdings durch  "`\texttt{\$e} $\vdash F/ \, x \, \psi $"' ersetzt.} verwendet, was nur voraussetzt, dass $x$ effektiv nicht frei in $\varphi$ ist, aber nicht notwendigerweise nicht darin vorkommt.  Die Verwendung der impliziten Substitution\index{Substitution!implizite} ist zum Teil eine Frage des persönlichen Stils, obwohl sie Beweise etwas kürzer machen kann, als es bei expliziter Substitution der Fall wäre.

\item
Die Annahme


\vskip 1ex
\setbox\startprefix=\hbox{\tt \ \ \ \ \ \ \ \ \ \$e\ }
\setbox\contprefix=\hbox{\tt \ \ \ \ \ \ \ \ \ \ }
\startm
\m{\vdash}\m{A}\m{\in}\m{{\rm V}}
\endm
\vskip 1ex

sollte als "`angenommen, die Klasse $A$ ist eine Menge (d.h. sie \ existiert)"' verstanden werden. Dies ist eine praktische Konvention, die von Quine verwendet wurde.

\item
Die Bariablenbeschränkung und die Annahme

\vskip 1ex
\setbox\startprefix=\hbox{\tt \ \ \ \ \ \ \ \ \ \$d\ }
\setbox\contprefix=\hbox{\tt \ \ \ \ \ \ \ \ \ \ }
\startm
\m{x}\m{\,y}
\endm
\setbox\startprefix=\hbox{\tt \ \ \ \ \ \ \ \ \ \$e\ }
\setbox\contprefix=\hbox{\tt \ \ \ \ \ \ \ \ \ \ }
\startm
\m{\vdash}\m{(}\m{y}\m{\in}\m{A}\m{\rightarrow}\m{\forall}\m{x}\m{\,y}
\m{\in}\m{A}\m{)}
\endm
\vskip 1ex

sollte als "`angenommen, die Variable $x$ ist (effektiv) nicht frei in der Klasse $A$"' verstanden werden.

\end{itemize}

\section{Einige Beispiele für Theoreme}\label{sometheorems}

In diesem Abschnitt werden einige der wichtigsten Theoreme aufgelistet, die in der Datenbasis \texttt{set.mm} bewiesen werden, und sie veranschaulichen, was man mit Metamath alles machen kann.  Während alle diese Fakten bekannte Ergebnisse sind, bietet Metamath den Vorteil, dass man ihre Herleitung leicht zu den Axiomen zurückverfolgen kann.  Wir wollen hier nicht versuchen, die Details oder die Motivation zu erklären; dafür verweisen wir auf die Lehrbücher, die in den Beschreibungen erwähnt werden.  (Die Datei \texttt{set.mm} enthält bibliografische Angaben zu den Textverweisen.)  Ihre Beweise enthalten oft wichtige Konzepte, die Sie vielleicht mit dem Programm Metamath untersuchen möchten (siehe Abschnitt~\ref{exploring}).  Alle Symbole, die hier verwendet werden, sind in Abschnitt~\ref{hierarchy} definiert.  Der Kürze halber haben wir die \texttt{\$d}-Beschränkungen oder \texttt{\$f}-Hypothesen für diese Theoreme nicht aufgenommen; wenn Sie unsicher sind, konsultieren Sie die \texttt{set.mm}-Datenbasis.

Wir beginnen mit \texttt{syl} (dem Prinzip des Syllogismus). In \textit{Principia Mathe\-matica} nennen Whitehead und Russell dies "`das Prinzip des Syllogismus... weil... der Syllogismus in Barbara von ihnen abgeleitet ist"' \cite[Zitat nach Theorem *2.06 S.~101]{PM}. Einige Autoren nennen dieses Gesetz einen "`hypothetischen Syllogismus"'. Ab 2019 ist \texttt{syl} die am häufigsten referenzierte bewiesene Behauptung in der \texttt{set.mm}-Datenbasis.\footnote{ Der Metamath-Programmbefehl \texttt{show usage} zeigt die Anzahl der Verwendungen. Am 29.04.2019 (commit 71cbbbdb387e [im GitHub-Repository metamath/set.mm]) wurde \texttt{syl} 10.819 Mal direkt referenziert. Die am zweithäufigsten referenzierte bewiesene Assertion war \texttt{eqid}, die 7.738 Mal direkt referenziert wurde. }

\vskip 2ex
\noindent Theorem syl (das Prinzip des Syllogismus)\index{Syllogismus}%
\index{\texttt{syl}}\label{syl}.

\vskip 0.5ex
\setbox\startprefix=\hbox{\tt \ \ syl.1\ \$e\ }
\setbox\contprefix=\hbox{\tt \ \ \ \ \ \ \ \ \ \ \ }
\startm
\m{\vdash}\m{(}\m{\varphi}\m{ \rightarrow }\m{\psi}\m{)}
\endm
\setbox\startprefix=\hbox{\tt \ \ syl.2\ \$e\ }
\setbox\contprefix=\hbox{\tt \ \ \ \ \ \ \ \ \ \ \ }
\startm
\m{\vdash}\m{(}\m{\psi}\m{ \rightarrow }\m{\chi}\m{)}
\endm
\setbox\startprefix=\hbox{\tt \ \ syl\ \$p\ }
\setbox\contprefix=\hbox{\tt \ \ \ \ \ \ \ \ \ }
\startm
\m{\vdash}\m{(}\m{\varphi}\m{ \rightarrow }\m{\chi}\m{)}
\endm
\vskip 2ex

Das folgende Theorem ist nicht sehr tiefgründig, bietet uns aber eine häufig verwendete Notationshilfe.  Es erlaubt uns, den Ausdruck "`$A \in V$"' als eine kompakte Art zu sagen, dass die Klasse $A$ existiert, d.h. eine Menge ist.

\vskip 2ex
\noindent Es gibt zwei Möglichkeiten zu sagen, dass $A$ eine Menge ist: $A$ ist ein Element des Universums $V$ genau dann, wenn $A$ existiert (d.h. wenn es eine Menge gibt, die $A$ entspricht). Theorem 6.9 von Quine, S. 43.

\vskip 0.5ex
\setbox\startprefix=\hbox{\tt \ \ isset\ \$p\ }
\setbox\contprefix=\hbox{\tt \ \ \ \ \ \ \ \ \ \ \ }
\startm
\m{\vdash}\m{(}\m{A}\m{\in}\m{{\rm V}}\m{\leftrightarrow}\m{\exists}\m{x}\m{\,x}\m{=}
\m{A}\m{)}
\endm
\vskip 1ex

Als nächstes beweisen wir die Axiome der Standard-ZF-Mengenlehre, die in unserem Axiomensystem fehlen.  Aus unserer Sicht sind sie Theoreme, da sie aus den anderen Axiomen abgeleitet werden können.

\vskip 2ex
\noindent Das Aussonderungsaxiom\index{Aussonderungsaxiom}, bewiesen aus den anderen Axiomen der ZF-Mengenlehre.  Vergleiche Übung 4 von Takeuti und Zaring, S.~22.

\vskip 0.5ex
\setbox\startprefix=\hbox{\tt \ \ inex1.1\ \$e\ }
\setbox\contprefix=\hbox{\tt \ \ \ \ \ \ \ \ \ \ \ \ \ \ \ }
\startm
\m{\vdash}\m{A}\m{\in}\m{{\rm V}}
\endm
\setbox\startprefix=\hbox{\tt \ \ inex\ \$p\ }
\setbox\contprefix=\hbox{\tt \ \ \ \ \ \ \ \ \ \ \ \ \ }
\startm
\m{\vdash}\m{(}\m{A}\m{\cap}\m{B}\m{)}\m{\in}\m{{\rm V}}
\endm
\vskip 1ex

\noindent Das Leermengenaxiom\index{Leermengenaxiom}, bewiesen aus den anderen Axiomen der ZF-Mengenlehre. Korollar 5.16 von Takeuti und Zaring, S.~20.

\vskip 0.5ex
\setbox\startprefix=\hbox{\tt \ \ 0ex\ \$p\ }
\setbox\contprefix=\hbox{\tt \ \ \ \ \ \ \ \ \ \ \ \ }
\startm
\m{\vdash}\m{\varnothing}\m{\in}\m{{\rm V}}
\endm
\vskip 1ex

\noindent Das Paarmengenaxiom\index{Paarmengenaxiom}, bewiesen aus den anderen Axiomen der ZF-Mengenlehre.  Theorem 7.13 von Quine, S. 51.
\vskip 0.5ex
\setbox\startprefix=\hbox{\tt \ \ prex\ \$p\ }
\setbox\contprefix=\hbox{\tt \ \ \ \ \ \ \ \ \ \ \ \ \ \ }
\startm
\m{\vdash}\m{\{}\m{A}\m{,}\m{B}\m{\}}\m{\in}\m{{\rm V}}
\endm
\vskip 2ex

Als nächstes werden wir einige berühmte oder wichtige Theoreme auflisten, die in der Datenbasis \texttt{set.mm} bewiesen sind.  Keines von ihnen außer \texttt{omex} erfordert das Unendlichkeitsaxiom, wie Sie mit dem Metamath-Befehl \texttt{show trace{\char`\_}back} überprüfen können.

\vskip 2ex
\noindent Die Auflösung des Russell'schen Paradoxons\index{Russells Paradoxon}.  Es gibt keine Menge, die der Klasse aller Mengen, die nicht Mitglieder ihrer selbst sind, entspricht.  Proposition 4.14 von Takeuti und Zaring, S.~14.

\vskip 0.5ex
\setbox\startprefix=\hbox{\tt \ \ ru\ \$p\ }
\setbox\contprefix=\hbox{\tt \ \ \ \ \ \ \ \ }
\startm
\m{\vdash}\m{\lnot}\m{\exists}\m{x}\m{\,x}\m{=}\m{\{}\m{y}\m{|}\m{\lnot}\m{y}
\m{\in}\m{y}\m{\}}
\endm
\vskip 1ex

\noindent Satz von Cantor\index{Satz von Cantor}.  Keine Menge kann auf ihre Potenzmenge abgebildet werden.  Vergleiche Theorem 6B(b) von Enderton, S.~132.

\vskip 0.5ex
\setbox\startprefix=\hbox{\tt \ \ canth.1\ \$e\ }
\setbox\contprefix=\hbox{\tt \ \ \ \ \ \ \ \ \ \ \ \ \ }
\startm
\m{\vdash}\m{A}\m{\in}\m{{\rm V}}
\endm
\setbox\startprefix=\hbox{\tt \ \ canth\ \$p\ }
\setbox\contprefix=\hbox{\tt \ \ \ \ \ \ \ \ \ \ \ }
\startm
\m{\vdash}\m{\lnot}\m{F}\m{:}\m{A}\m{\raisebox{.5ex}{${\textstyle{\:}_{
\mbox{\footnotesize\rm {\ }}}}\atop{\textstyle{\longrightarrow}\atop{
\textstyle{}^{\mbox{\footnotesize\rm onto}}}}$}}\m{{\cal P}}\m{A}
\endm
\vskip 1ex

\noindent Das Burali-Forti-Paradoxon\index{Burali-Forti Paradoxon}.  Keine Menge enthält alle Ordinalzahlen. Enderton, S.~194.  (Burali-Forti war eine Person, nicht zwei.)

\vskip 0.5ex
\setbox\startprefix=\hbox{\tt \ \ onprc\ \$p\ }
\setbox\contprefix=\hbox{\tt \ \ \ \ \ \ \ \ \ \ \ \ }
\startm
\m{\vdash}\m{\lnot}\m{\mbox{\rm On}}\m{\in}\m{{\rm V}}
\endm
\vskip 1ex

\noindent Peano-Postulate\index{Peano-Postulate} für die Arithmetik. Satz 7.30 von Takeuti und Zaring, S.~42--43.  Die zu beschreibenden Objekte sind die Elemente von $\omega$, d.h. die natürlichen Zahlen 0, 1, 2, \ldots.  Die Nachfolger\index{Nachfolger}-Operation suc bedeutet "`plus eins"'.  \texttt{peano1} besagt, dass 0 (die als leere Menge definiert ist) eine natürliche Zahl ist.  \texttt{peano2} besagt, dass wenn $A$ eine natürliche Zahl ist, $A+1$ auch eine natürliche Zahl ist.  \texttt{peano3} besagt, dass 0 nicht der Nachfolger einer natürlichen Zahl ist.  \texttt{peano4} besagt, dass zwei natürliche Zahlen genau dann gleich sind, wenn ihre Nachfolger gleich sind.  \texttt{peano5} ist im Wesentlichen dasselbe wie die vollständige Induktion.

\vskip 1ex
\setbox\startprefix=\hbox{\tt \ \ peano1\ \$p\ }
\setbox\contprefix=\hbox{\tt \ \ \ \ \ \ \ \ \ \ \ \ }
\startm
\m{\vdash}\m{\varnothing}\m{\in}\m{\omega}
\endm
\vskip 1.5ex

\setbox\startprefix=\hbox{\tt \ \ peano2\ \$p\ }
\setbox\contprefix=\hbox{\tt \ \ \ \ \ \ \ \ \ \ \ \ }
\startm
\m{\vdash}\m{(}\m{A}\m{\in}\m{\omega}\m{\rightarrow}\m{{\rm suc}}\m{A}\m{\in}%
\m{\omega}\m{)}
\endm
\vskip 1.5ex

\setbox\startprefix=\hbox{\tt \ \ peano3\ \$p\ }
\setbox\contprefix=\hbox{\tt \ \ \ \ \ \ \ \ \ \ \ \ }
\startm
\m{\vdash}\m{(}\m{A}\m{\in}\m{\omega}\m{\rightarrow}\m{\lnot}\m{{\rm suc}}%
\m{A}\m{=}\m{\varnothing}\m{)}
\endm
\vskip 1.5ex

\setbox\startprefix=\hbox{\tt \ \ peano4\ \$p\ }
\setbox\contprefix=\hbox{\tt \ \ \ \ \ \ \ \ \ \ \ \ }
\startm
\m{\vdash}\m{(}\m{(}\m{A}\m{\in}\m{\omega}\m{\wedge}\m{B}\m{\in}\m{\omega}%
\m{)}\m{\rightarrow}\m{(}\m{{\rm suc}}\m{A}\m{=}\m{{\rm suc}}\m{B}\m{%
\leftrightarrow}\m{A}\m{=}\m{B}\m{)}\m{)}
\endm
\vskip 1.5ex

\setbox\startprefix=\hbox{\tt \ \ peano5\ \$p\ }
\setbox\contprefix=\hbox{\tt \ \ \ \ \ \ \ \ \ \ \ \ }
\startm
\m{\vdash}\m{(}\m{(}\m{\varnothing}\m{\in}\m{A}\m{\wedge}\m{\forall}\m{x}\m{%
\in}\m{\omega}\m{(}\m{x}\m{\in}\m{A}\m{\rightarrow}\m{{\rm suc}}\m{x}\m{\in}%
\m{A}\m{)}\m{)}\m{\rightarrow}\m{\omega}\m{\subseteq}\m{A}\m{)}
\endm
\vskip 1.5ex

\noindent Finite Induktion (vollständige Induktion).\index{finite Induktion}\index{vollständige Induktion} Die erste Hypothese ist der Induktionsanfang und die zweite ist der Induktionsschritt.  Theorem Schema 22 von Suppes, S.~136.

\vskip 0.5ex
\setbox\startprefix=\hbox{\tt \ \ findes.1\ \$e\ }
\setbox\contprefix=\hbox{\tt \ \ \ \ \ \ \ \ \ \ \ \ \ \ }
\startm
\m{\vdash}\m{[}\m{\varnothing}\m{/}\m{x}\m{]}\m{\varphi}
\endm
\setbox\startprefix=\hbox{\tt \ \ findes.2\ \$e\ }
\setbox\contprefix=\hbox{\tt \ \ \ \ \ \ \ \ \ \ \ \ \ \ }
\startm
\m{\vdash}\m{(}\m{x}\m{\in}\m{\omega}\m{\rightarrow}\m{(}\m{\varphi}\m{%
\rightarrow}\m{[}\m{{\rm suc}}\m{x}\m{/}\m{x}\m{]}\m{\varphi}\m{)}\m{)}
\endm
\setbox\startprefix=\hbox{\tt \ \ findes\ \$p\ }
\setbox\contprefix=\hbox{\tt \ \ \ \ \ \ \ \ \ \ \ \ }
\startm
\m{\vdash}\m{(}\m{x}\m{\in}\m{\omega}\m{\rightarrow}\m{\varphi}\m{)}
\endm
\vskip 1ex

\noindent Transfinite Induktion mit expliziter Substitution.  Die erste Hypothese ist der Induktionsanfang, die zweite ist der Induktionsschritt für Nachfolger und die dritte ist der Induktionsschritt für Grenzzahlen.  Theorem Schema 4 von Suppes, S. 197.

\vskip 0.5ex
\setbox\startprefix=\hbox{\tt \ \ tfindes.1\ \$e\ }
\setbox\contprefix=\hbox{\tt \ \ \ \ \ \ \ \ \ \ \ \ \ \ \ }
\startm
\m{\vdash}\m{[}\m{\varnothing}\m{/}\m{x}\m{]}\m{\varphi}
\endm
\setbox\startprefix=\hbox{\tt \ \ tfindes.2\ \$e\ }
\setbox\contprefix=\hbox{\tt \ \ \ \ \ \ \ \ \ \ \ \ \ \ \ }
\startm
\m{\vdash}\m{(}\m{x}\m{\in}\m{{\rm On}}\m{\rightarrow}\m{(}\m{\varphi}\m{%
\rightarrow}\m{[}\m{{\rm suc}}\m{x}\m{/}\m{x}\m{]}\m{\varphi}\m{)}\m{)}
\endm
\setbox\startprefix=\hbox{\tt \ \ tfindes.3\ \$e\ }
\setbox\contprefix=\hbox{\tt \ \ \ \ \ \ \ \ \ \ \ \ \ \ \ }
\startm
\m{\vdash}\m{(}\m{{\rm Lim}}\m{y}\m{\rightarrow}\m{(}\m{\forall}\m{x}\m{\in}%
\m{y}\m{\varphi}\m{\rightarrow}\m{[}\m{y}\m{/}\m{x}\m{]}\m{\varphi}\m{)}\m{)}
\endm
\setbox\startprefix=\hbox{\tt \ \ tfindes\ \$p\ }
\setbox\contprefix=\hbox{\tt \ \ \ \ \ \ \ \ \ \ \ \ \ }
\startm
\m{\vdash}\m{(}\m{x}\m{\in}\m{{\rm On}}\m{\rightarrow}\m{\varphi}\m{)}
\endm
\vskip 1ex

\noindent Prinzip der transfiniten Rekursion.\index{transfinite Rekursion} Theorem 7.41 von Takeuti und Zaring, S.~47.  Die transfinite Rekursion ist der grundlegende Satz für eine strenge Definition der Arithmetik von Ordinalzahlen, und hat auch viele andere wichtige Anwendungen. Die Annahmen \texttt{tfr.1} und \texttt{tfr.2} spezifizieren eine bestimmte (echte) Klasse $ F$.  Die komplizierte Definition von $ F$ ist an sich nicht wichtig; wichtig ist, dass es ein solches $ F$ mit den erforderlichen Eigenschaften gibt, und wir zeigen dies, indem wir $ F$ explizit angeben. \texttt{tfr1} besagt, dass $ F$ eine Funktion ist, deren Definitionsbereich die Menge der Ordnungszahlen ist.  \texttt{tfr2} besagt, dass jeder Wert von $ F$ vollständig durch seine vorherigen Werte und die Werte einer Hilfsfunktion, $G$, bestimmt ist.  \texttt{tfr3} besagt, dass $ F$ eindeutig ist, d.h. es ist die einzige Funktion, die \texttt{tfr1} und \texttt{tfr2} erfüllt.  Beachten Sie, dass $ f$ eine individuelle Variable wie $x$ und $y$ ist; es ist nur eine Gedächtnisstütze, um uns daran zu erinnern, dass $A$ eine Sammlung von Funktionen ist.

\vskip 0.5ex
\setbox\startprefix=\hbox{\tt \ \ tfr.1\ \$e\ }
\setbox\contprefix=\hbox{\tt \ \ \ \ \ \ \ \ \ \ \ }
\startm
\m{\vdash}\m{A}\m{=}\m{\{}\m{f}\m{|}\m{\exists}\m{x}\m{\in}\m{{\rm On}}\m{(}%
\m{f}\m{{\rm Fn}}\m{x}\m{\wedge}\m{\forall}\m{y}\m{\in}\m{x}\m{(}\m{f}\m{`}%
\m{y}\m{)}\m{=}\m{(}\m{G}\m{`}\m{(}\m{f}\m{\restriction}\m{y}\m{)}\m{)}\m{)}%
\m{\}}
\endm
\setbox\startprefix=\hbox{\tt \ \ tfr.2\ \$e\ }
\setbox\contprefix=\hbox{\tt \ \ \ \ \ \ \ \ \ \ \ }
\startm
\m{\vdash}\m{F}\m{=}\m{\bigcup}\m{A}
\endm
\setbox\startprefix=\hbox{\tt \ \ tfr1\ \$p\ }
\setbox\contprefix=\hbox{\tt \ \ \ \ \ \ \ \ \ \ }
\startm
\m{\vdash}\m{F}\m{{\rm Fn}}\m{{\rm On}}
\endm
\setbox\startprefix=\hbox{\tt \ \ tfr2\ \$p\ }
\setbox\contprefix=\hbox{\tt \ \ \ \ \ \ \ \ \ \ }
\startm
\m{\vdash}\m{(}\m{z}\m{\in}\m{{\rm On}}\m{\rightarrow}\m{(}\m{F}\m{`}\m{z}%
\m{)}\m{=}\m{(}\m{G}\m{`}\m{(}\m{F}\m{\restriction}\m{z}\m{)}\m{)}\m{)}
\endm
\setbox\startprefix=\hbox{\tt \ \ tfr3\ \$p\ }
\setbox\contprefix=\hbox{\tt \ \ \ \ \ \ \ \ \ \ }
\startm
\m{\vdash}\m{(}\m{(}\m{B}\m{{\rm Fn}}\m{{\rm On}}\m{\wedge}\m{\forall}\m{x}\m{%
\in}\m{{\rm On}}\m{(}\m{B}\m{`}\m{x}\m{)}\m{=}\m{(}\m{G}\m{`}\m{(}\m{B}\m{%
\restriction}\m{x}\m{)}\m{)}\m{)}\m{\rightarrow}\m{B}\m{=}\m{F}\m{)}
\endm
\vskip 1ex

\noindent Die Existenz von omega (die Klasse der natürlichen Zahlen).\index{natürliche Zahlen}\index{Omega ($\omega$)}\index{Unendlichkeitsaxiom}  Axiom 7 von Takeuti und Zaring, S.~43.  (Dies ist das einzige Theorem in diesem Abschnitt, das das Unendlichkeitsaxiom erfordert).

\vskip 0.5ex
\setbox\startprefix=\hbox{\tt \
\ omex\ \$p\ }
\setbox\contprefix=\hbox{\tt \ \ \ \ \ \ \ \ \ \ }
\startm
\m{\vdash}\m{\omega}\m{\in}\m{{\rm V}}
\endm
%\vskip 2ex


\section{Axiome für reelle und komplexe Zahlen}\label{real}
\index{reelle Zahl}\index{komplexe Zahl}

In diesem Abschnitt werden die Axiome für reelle und komplexe Zahlen vorgestellt und kommentiert.  Analysis-Lehrbücher verwenden implizit oder explizit diese Axiome oder ihre Entsprechungen als Ausgangspunkt.  In der Datenbasis \texttt{set.mm} definieren wir reelle und komplexe Zahlen als (ziemlich komplizierte) spezifische Mengen und leiten diese Axiome als {\em Theoreme} aus den Axiomen der ZF-Mengenlehre ab, indem wir eine Methode der Dedekindschen Schnitte verwenden.  Wir lassen die Details dieser Konstruktion weg, die Sie bei Bedarf mit Hilfe der Datenbasis \texttt{set.mm} in Verbindung mit den darin referenzierten Lehrbüchern nachvollziehen können.

Sobald wir diese Theoreme bewiesen haben, formulieren wir die bewiesenen Theoreme als Axiome neu. Auf diese Weise können wir leicht erkennen, welche Axiome für einen bestimmten Beweis komplexer Zahlen benötigt werden, ohne durch die Komplexität ihrer Herleitung durch die Mengenlehre abgelenkt zu werden. Infolgedessen ist die Konstruktion eigentlich unwichtig, außer um zu zeigen, dass es Mengen gibt, die den Axiomen genügen, und dass die Axiome folglich konsistent sind, wenn die Mengenlehre konsistent ist.  Wenn man mit reellen Zahlen arbeitet, kann man sie tatsächlich als die Mengen betrachten, die sich aus der Konstruktion ergeben (für die Definitheit), oder man kann sie als nicht weiter spezifizierte Mengen betrachten, die zufälligerweise die Axiome erfüllen. Die Herleitung ist nicht einfach, aber die Tatsache, dass sie funktioniert, ist bemerkenswert und unterstützt die Idee, dass die ZFC-Mengenlehre alles ist, was wir brauchen, um eine Grundlage für die gesamte Mathematik zu schaffen.

\needspace{3\baselineskip}
\subsection{Die Axiome für reelle und komplexe Zahlen selbst}\label{realactual}

Für die Axiome werden uns 8 Klassen vorgegeben (oder vorausgesetzt):  $\mathbb{C}$ (die Menge der komplexen Zahlen), $\mathbb{R}$ (die Menge der reellen Zahlen, eine Teilmenge von $\mathbb{C}$), $0$ (Null), $1$ (Eins), $i$ (Quadratwurzel aus $-1$), $+$ (plus), $\cdot$ (mal) und $<_{\mathbb{R}}$ (kleiner als, nur für die reellen Zahlen). Subtraktion und Division sind definierte Begriffe und werden nicht in den Axiomen verwendet. Für ihre Definitionen siehe \texttt{set.mm}.

Man beachte, dass die Notation $(A+B)$ (und ähnlich $(A\cdot B)$) eine Klasse bezeichnet, die als {\em Operation},\index{Operation} bezeichnet wird und den Funktionswert der Klasse $+$ für das geordnete Paar $\langle A,B \rangle$ darstellt.  Eine Operation ist durch die Aussage \texttt{df-opr} auf Seite ~\pageref{dfopr} definiert. Die Notation $A <_{\mathbb{R}} B$ bezeichnet eine wff, die als {\em binäre Relation}\index{binäre Relation} bezeichnet wird und $\langle A,B \rangle \in \,<_{\mathbb{R}}$ bedeutet, wie durch \texttt{df-br} auf Seite ~\pageref{dfbr} definiert. 
 
Wir gehen davon aus, dass die 8 vorgegebenen Klassen die folgenden 22 Axiome erfüllen (in den unten aufgeführten Axiomen wird kurz $<$ statt $<_{\mathbb{R}}$ verwendet).

\vskip 2ex

\noindent 1. Die reellen Zahlen sind eine Teilmenge der komplexen Zahlen.

%\vskip 0.5ex
\setbox\startprefix=\hbox{\tt \ \ ax-resscn\ \$p\ }
\setbox\contprefix=\hbox{\tt \ \ \ \ \ \ \ \ \ \ \ \ \ \ }
\startm
\m{\vdash}\m{\mathbb{R}}\m{\subseteq}\m{\mathbb{C}}
\endm
%\vskip 1ex

\noindent 2. Eins ist eine komplexe Zahl.

%\vskip 0.5ex
\setbox\startprefix=\hbox{\tt \ \ ax-1cn\ \$p\ }
\setbox\contprefix=\hbox{\tt \ \ \ \ \ \ \ \ \ \ \ }
\startm
\m{\vdash}\m{1}\m{\in}\m{\mathbb{C}}
\endm
%\vskip 1ex

\noindent 3. Die imaginäre Einheit $i$ ist eine komplexe Zahl.

%\vskip 0.5ex
\setbox\startprefix=\hbox{\tt \ \ ax-icn\ \$p\ }
\setbox\contprefix=\hbox{\tt \ \ \ \ \ \ \ \ \ \ \ }
\startm
\m{\vdash}\m{i}\m{\in}\m{\mathbb{C}}
\endm
%\vskip 1ex

\noindent 4. Komplexe Zahlen sind bzgl. der Addition abgeschlossen.

%\vskip 0.5ex
\setbox\startprefix=\hbox{\tt \ \ ax-addcl\ \$p\ }
\setbox\contprefix=\hbox{\tt \ \ \ \ \ \ \ \ \ \ \ \ \ }
\startm
\m{\vdash}\m{(}\m{(}\m{A}\m{\in}\m{\mathbb{C}}\m{\wedge}\m{B}\m{\in}\m{\mathbb{C}}%
\m{)}\m{\rightarrow}\m{(}\m{A}\m{+}\m{B}\m{)}\m{\in}\m{\mathbb{C}}\m{)}
\endm
%\vskip 1ex

\noindent 5. Reelle Zahlen sind bzgl. der Addition abgeschlossen.

%\vskip 0.5ex
\setbox\startprefix=\hbox{\tt \ \ ax-addrcl\ \$p\ }
\setbox\contprefix=\hbox{\tt \ \ \ \ \ \ \ \ \ \ \ \ \ \ }
\startm
\m{\vdash}\m{(}\m{(}\m{A}\m{\in}\m{\mathbb{R}}\m{\wedge}\m{B}\m{\in}\m{\mathbb{R}}%
\m{)}\m{\rightarrow}\m{(}\m{A}\m{+}\m{B}\m{)}\m{\in}\m{\mathbb{R}}\m{)}
\endm
%\vskip 1ex

\noindent 6. Komplexe Zahlen sind bzgl. der Multiplikation abgeschlossen.

%\vskip 0.5ex
\setbox\startprefix=\hbox{\tt \ \ ax-mulcl\ \$p\ }
\setbox\contprefix=\hbox{\tt \ \ \ \ \ \ \ \ \ \ \ \ \ }
\startm
\m{\vdash}\m{(}\m{(}\m{A}\m{\in}\m{\mathbb{C}}\m{\wedge}\m{B}\m{\in}\m{\mathbb{C}}%
\m{)}\m{\rightarrow}\m{(}\m{A}\m{\cdot}\m{B}\m{)}\m{\in}\m{\mathbb{C}}\m{)}
\endm
%\vskip 1ex

\noindent 7. Reelle Zahlen sind bzgl. der Multiplikation abgeschlossen.

%\vskip 0.5ex
\setbox\startprefix=\hbox{\tt \ \ ax-mulrcl\ \$p\ }
\setbox\contprefix=\hbox{\tt \ \ \ \ \ \ \ \ \ \ \ \ \ \ }
\startm
\m{\vdash}\m{(}\m{(}\m{A}\m{\in}\m{\mathbb{R}}\m{\wedge}\m{B}\m{\in}\m{\mathbb{R}}%
\m{)}\m{\rightarrow}\m{(}\m{A}\m{\cdot}\m{B}\m{)}\m{\in}\m{\mathbb{R}}\m{)}
\endm
%\vskip 1ex

\noindent 8. Die Multiplikation von komplexen Zahlen ist kommutativ.

%\vskip 0.5ex
\setbox\startprefix=\hbox{\tt \ \ ax-mulcom\ \$p\ }
\setbox\contprefix=\hbox{\tt \ \ \ \ \ \ \ \ \ \ \ \ \ \ }
\startm
\m{\vdash}\m{(}\m{(}\m{A}\m{\in}\m{\mathbb{C}}\m{\wedge}\m{B}\m{\in}\m{\mathbb{C}}%
\m{)}\m{\rightarrow}\m{(}\m{A}\m{\cdot}\m{B}\m{)}\m{=}\m{(}\m{B}\m{\cdot}\m{A}%
\m{)}\m{)}
\endm
%\vskip 1ex

\noindent 9. Die Addition von komplexen Zahlen ist assoziativ.

%\vskip 0.5ex
\setbox\startprefix=\hbox{\tt \ \ ax-addass\ \$p\ }
\setbox\contprefix=\hbox{\tt \ \ \ \ \ \ \ \ \ \ \ \ \ \ }
\startm
\m{\vdash}\m{(}\m{(}\m{A}\m{\in}\m{\mathbb{C}}\m{\wedge}\m{B}\m{\in}\m{\mathbb{C}}%
\m{\wedge}\m{C}\m{\in}\m{\mathbb{C}}\m{)}\m{\rightarrow}\m{(}\m{(}\m{A}\m{+}%
\m{B}\m{)}\m{+}\m{C}\m{)}\m{=}\m{(}\m{A}\m{+}\m{(}\m{B}\m{+}\m{C}\m{)}\m{)}%
\m{)}
\endm
%\vskip 1ex

\noindent 10. Die Multiplikation von komplexen Zahlen ist assoziativ.

%\vskip 0.5ex
\setbox\startprefix=\hbox{\tt \ \ ax-mulass\ \$p\ }
\setbox\contprefix=\hbox{\tt \ \ \ \ \ \ \ \ \ \ \ \ \ \ }
\startm
\m{\vdash}\m{(}\m{(}\m{A}\m{\in}\m{\mathbb{C}}\m{\wedge}\m{B}\m{\in}\m{\mathbb{C}}%
\m{\wedge}\m{C}\m{\in}\m{\mathbb{C}}\m{)}\m{\rightarrow}\m{(}\m{(}\m{A}\m{\cdot}%
\m{B}\m{)}\m{\cdot}\m{C}\m{)}\m{=}\m{(}\m{A}\m{\cdot}\m{(}\m{B}\m{\cdot}\m{C}%
\m{)}\m{)}\m{)}
\endm
%\vskip 1ex

\noindent 11. Die Multiplikation der komplexen Zahlen ist distributiv bzgl. der Addition.

%\vskip 0.5ex
\setbox\startprefix=\hbox{\tt \ \ ax-distr\ \$p\ }
\setbox\contprefix=\hbox{\tt \ \ \ \ \ \ \ \ \ \ \ \ \ }
\startm
\m{\vdash}\m{(}\m{(}\m{A}\m{\in}\m{\mathbb{C}}\m{\wedge}\m{B}\m{\in}\m{\mathbb{C}}%
\m{\wedge}\m{C}\m{\in}\m{\mathbb{C}}\m{)}\m{\rightarrow}\m{(}\m{A}\m{\cdot}\m{(}%
\m{B}\m{+}\m{C}\m{)}\m{)}\m{=}\m{(}\m{(}\m{A}\m{\cdot}\m{B}\m{)}\m{+}\m{(}%
\m{A}\m{\cdot}\m{C}\m{)}\m{)}\m{)}
\endm
%\vskip 1ex

\noindent 12. Das Quadrat von $i$ ist gleich $-1$ (ausgedrückt als $i$-Quadrat plus Eins ist Null).

%\vskip 0.5ex
\setbox\startprefix=\hbox{\tt \ \ ax-i2m1\ \$p\ }
\setbox\contprefix=\hbox{\tt \ \ \ \ \ \ \ \ \ \ \ \ }
\startm
\m{\vdash}\m{(}\m{(}\m{i}\m{\cdot}\m{i}\m{)}\m{+}\m{1}\m{)}\m{=}\m{0}
\endm
%\vskip 1ex

\noindent 13. Eins und Null sind verschieden.

%\vskip 0.5ex
\setbox\startprefix=\hbox{\tt \ \ ax-1ne0\ \$p\ }
\setbox\contprefix=\hbox{\tt \ \ \ \ \ \ \ \ \ \ \ \ }
\startm
\m{\vdash}\m{1}\m{\ne}\m{0}
\endm
%\vskip 1ex

\noindent 14. Eines ist ein neutrales Element für die reelle Multiplikation.

%\vskip 0.5ex
\setbox\startprefix=\hbox{\tt \ \ ax-1rid\ \$p\ }
\setbox\contprefix=\hbox{\tt \ \ \ \ \ \ \ \ \ \ \ }
\startm
\m{\vdash}\m{(}\m{A}\m{\in}\m{\mathbb{R}}\m{\rightarrow}\m{(}\m{A}\m{\cdot}\m{1}%
\m{)}\m{=}\m{A}\m{)}
\endm
%\vskip 1ex

\noindent 15. Zu jeder reellen Zahl gibt es eine entsprechende negative Zahl (additives Inverses).

%\vskip 0.5ex
\setbox\startprefix=\hbox{\tt \ \ ax-rnegex\ \$p\ }
\setbox\contprefix=\hbox{\tt \ \ \ \ \ \ \ \ \ \ \ \ \ \ }
\startm
\m{\vdash}\m{(}\m{A}\m{\in}\m{\mathbb{R}}\m{\rightarrow}\m{\exists}\m{x}\m{\in}%
\m{\mathbb{R}}\m{(}\m{A}\m{+}\m{x}\m{)}\m{=}\m{0}\m{)}
\endm
%\vskip 1ex

\noindent 16. Jede reelle Zahl ungleich Null hat einen Kehrwert.

%\vskip 0.5ex
\setbox\startprefix=\hbox{\tt \ \ ax-rrecex\ \$p\ }
\setbox\contprefix=\hbox{\tt \ \ \ \ \ \ \ \ \ \ \ \ \ \ }
\startm
\m{\vdash}\m{(}\m{A}\m{\in}\m{\mathbb{R}}\m{\rightarrow}\m{(}\m{A}\m{\ne}\m{0}%
\m{\rightarrow}\m{\exists}\m{x}\m{\in}\m{\mathbb{R}}\m{(}\m{A}\m{\cdot}%
\m{x}\m{)}\m{=}\m{1}\m{)}\m{)}
\endm
%\vskip 1ex

\noindent 17. Eine komplexe Zahl kann durch zwei reelle Zahlen ausgedrückt werden.

%\vskip 0.5ex
\setbox\startprefix=\hbox{\tt \ \ ax-cnre\ \$p\ }
\setbox\contprefix=\hbox{\tt \ \ \ \ \ \ \ \ \ \ \ \ }
\startm
\m{\vdash}\m{(}\m{A}\m{\in}\m{\mathbb{C}}\m{\rightarrow}\m{\exists}\m{x}\m{\in}%
\m{\mathbb{R}}\m{\exists}\m{y}\m{\in}\m{\mathbb{R}}\m{A}\m{=}\m{(}\m{x}\m{+}\m{(}%
\m{y}\m{\cdot}\m{i}\m{)}\m{)}\m{)}
\endm
%\vskip 1ex

\noindent 18. Die Ordnung der reellen Zahlen erfüllt die strenge Trichotomie.

%\vskip 0.5ex
\setbox\startprefix=\hbox{\tt \ \ ax-pre-lttri\ \$p\ }
\setbox\contprefix=\hbox{\tt \ \ \ \ \ \ \ \ \ \ \ \ \ }
\startm
\m{\vdash}\m{(}\m{(}\m{A}\m{\in}\m{\mathbb{R}}\m{\wedge}\m{B}\m{\in}\m{\mathbb{R}}%
\m{)}\m{\rightarrow}\m{(}\m{A}\m{<}\m{B}\m{\leftrightarrow}\m{\lnot}\m{(}\m{A}%
\m{=}\m{B}\m{\vee}\m{B}\m{<}\m{A}\m{)}\m{)}\m{)}
\endm
%\vskip 1ex

\noindent 19. Die Ordnung der reellen Zahlen ist transitiv.

%\vskip 0.5ex
\setbox\startprefix=\hbox{\tt \ \ ax-pre-lttrn\ \$p\ }
\setbox\contprefix=\hbox{\tt \ \ \ \ \ \ \ \ \ \ \ \ \ }
\startm
\m{\vdash}\m{(}\m{(}\m{A}\m{\in}\m{\mathbb{R}}\m{\wedge}\m{B}\m{\in}\m{\mathbb{R}}%
\m{\wedge}\m{C}\m{\in}\m{\mathbb{R}}\m{)}\m{\rightarrow}\m{(}\m{(}\m{A}\m{<}%
\m{B}\m{\wedge}\m{B}\m{<}\m{C}\m{)}\m{\rightarrow}\m{A}\m{<}\m{C}\m{)}\m{)}
\endm
%\vskip 1ex

\noindent 20. Die Ordnung der reellen Zahlen ist invariant bzgl. der Addition.

%\vskip 0.5ex
\setbox\startprefix=\hbox{\tt \ \ ax-pre-ltadd\ \$p\ }
\setbox\contprefix=\hbox{\tt \ \ \ \ \ \ \ \ \ \ \ \ \ }
\startm
\m{\vdash}\m{(}\m{(}\m{A}\m{\in}\m{\mathbb{R}}\m{\wedge}\m{B}\m{\in}\m{\mathbb{R}}%
\m{\wedge}\m{C}\m{\in}\m{\mathbb{R}}\m{)}\m{\rightarrow}\m{(}\m{A}\m{<}\m{B}\m{%
\rightarrow}\m{(}\m{C}\m{+}\m{A}\m{)}\m{<}\m{(}\m{C}\m{+}\m{B}\m{)}\m{)}\m{)}
\endm
%\vskip 1ex

\noindent 21. Das Produkt zweier positiver reeller Zahlen ist positiv.

%\vskip 0.5ex
\setbox\startprefix=\hbox{\tt \ \ ax-pre-mulgt0\ \$p\ }
\setbox\contprefix=\hbox{\tt \ \ \ \ \ \ \ \ \ \ \ \ \ \ }
\startm
\m{\vdash}\m{(}\m{(}\m{A}\m{\in}\m{\mathbb{R}}\m{\wedge}\m{B}\m{\in}\m{\mathbb{R}}%
\m{)}\m{\rightarrow}\m{(}\m{(}\m{0}\m{<}\m{A}\m{\wedge}\m{0}%
\m{<}\m{B}\m{)}\m{\rightarrow}\m{0}\m{<}\m{(}\m{A}\m{\cdot}\m{B}\m{)}%
\m{)}\m{)}
\endm
%\vskip 1ex

\noindent 22. Eine nicht leere, nach oben begrenzte Menge von reellen Zahlen hat ein Supremum.

%\vskip 0.5ex
\setbox\startprefix=\hbox{\tt \ \ ax-pre-sup\ \$p\ }
\setbox\contprefix=\hbox{\tt \ \ \ \ \ \ \ \ \ \ \ }
\startm
\m{\vdash}\m{(}\m{(}\m{A}\m{\subseteq}\m{\mathbb{R}}\m{\wedge}\m{A}\m{\ne}\m{%
\varnothing}\m{\wedge}\m{\exists}\m{x}\m{\in}\m{\mathbb{R}}\m{\forall}\m{y}\m{%
\in}\m{A}\m{\,y}\m{<}\m{x}\m{)}\m{\rightarrow}\m{\exists}\m{x}\m{\in}\m{%
\mathbb{R}}\m{(}\m{\forall}\m{y}\m{\in}\m{A}\m{\lnot}\m{x}\m{<}\m{y}\m{\wedge}\m{%
\forall}\m{y}\m{\in}\m{\mathbb{R}}\m{(}\m{y}\m{<}\m{x}\m{\rightarrow}\m{\exists}%
\m{z}\m{\in}\m{A}\m{\,y}\m{<}\m{z}\m{)}\m{)}\m{)}
\endm

% NOTE: The \m{...} expressions above could be represented as
% $ \vdash ( ( A \subseteq \mathbb{R} \wedge A \ne \varnothing \wedge \exists x \in \mathbb{R} \forall y \in A \,y < x ) \rightarrow \exists x \in \mathbb{R} ( \forall y \in A \lnot x < y \wedge \forall y \in \mathbb{R} ( y < x \rightarrow \exists z \in A \,y < z ) ) ) $

\vskip 2ex

Dies ist der vollständige Satz der Axiome für reelle und komplexe Zahlen.  Sehen Sie sich an, wie Subtraktion, Division und Dezimalzahlen in \texttt{set.mm} definiert sind, und schauen Sie sich zum Spaß den Beweis von $2+ 2 = 4$ (Theorem \texttt{2p2e4} in \texttt{set.mm}) an, wie in Abschnitt \ref{2p2e4} besprochen.

In \texttt{set.mm} definieren wir die positiven ganzen Zahlen $\mathbb{N}$, die nichtnegativen ganzen Zahlen $\mathbb{N}_0$, die ganzen Zahlen $\mathbb{Z}$ und die rationalen Zahlen $\mathbb{Q}$ als Teilmengen von $\mathbb{R}$.  Dies führt zu der schönen Teilmengenkette $\mathbb{N} \subseteq \mathbb{N}_0 \subseteq \mathbb{Z} \subseteq \mathbb{Q} \subseteq \mathbb{R} \subseteq \mathbb{C}$, was uns einen einheitlichen Rahmen für die Arithmetik gibt, in dem zum Beispiel eine Eigenschaft wie die Kommutativität der Addition komplexer Zahlen automatisch für ganze Zahlen gilt.  Die natürlichen Zahlen $\mathbb{N}$\footnote{Anm. der Übersetzer: sowohl im Deutschen als auch im Englischen ist nicht eindeutig festgelegt, ob mit dem Begriff "`natürliche Zahlen"' die positiven ganzen Zahlen $\mathbb{N}$ oder die nichtnegativen ganzen Zahlen $\mathbb{N}_0$ gemeint werden.} unterscheiden sich von der zuvor definierten Menge $\omega$, aber beide erfüllen die Peanoschen Postulate.

\subsection{Axiome für komplexe Zahlen in Texten zur \texorpdfstring{\\}{}Analysis}

Die meisten Texte zur Analysis konstruieren komplexe Zahlen als geordnete Paare von reellen Zahlen, was zu konstruktionsabhängigen Eigenschaften führt, die diese Axiome erfüllen, aber nicht in ihrer reinen Form angegeben werden.  (Dies geschieht auch in \texttt{set.mm}, aber unsere Axiome abstrahieren von dieser Konstruktion.) In anderen Texten heißt es einfach, dass $\mathbb{R}$ ein "`komplettes geordnetes Teilfeld von $\mathbb{C}$ ist"', was zu redundanten Axiomen führt, wenn man diese Phrase vollständig ausformuliert.  Tatsächlich habe ich noch keinen Text gesehen, der die Axiome in der obigen expliziten Form enthält. Keines dieser Axiome ist individuell einzigartig, aber diese sorgfältig ausgearbeitete Sammlung von Axiomen ist das Ergebnis jahrelanger Arbeit der Metamath-Gemeinschaft.

\subsection{Beseitigung unnötiger Axiome für komplexe \texorpdfstring{\\}{}Zahlen}

Metamath hatte ursprünglich mehr Axiome für reelle und komplexe Zahlen, aber im Laufe der Zeit haben wir (die Metamath-Gemeinschaft) Wege gefunden, unnötige Axiome zu eliminieren (indem wir sie anhand anderer Axiome bewiesen haben) oder sie abzuschwächen (indem wir schwächere Behauptungen aufgestellt haben, ohne die Beweisbarkeit der auf sie aufbauenden Theoreme zu reduzieren). Es folgen einige Aussagen, die früher Axiome für komplexe Zahlen waren, die aber inzwischen (mit Metamath) formal als überflüssig nachgewiesen wurden:

\begin{itemize}
\item
  $\mathbb{C} \in V$.
  Früher wurde dies als "`Axiom der komplexen Zahlen"' aufgeführt.   Es handelt sich jedoch eigentlich nicht um ein Axiom der komplexen Zahlen, und sein Beweis verwendet in jedem Fall Axiome der Mengenlehre.   Von Mario Carneiro\index{Carneiro, Mario} am 17-Nov-2014 als redundant bewiesen (siehe \texttt{axcnex}).
\item
  $((A \in \mathbb{C} \land B \in \mathbb{C}$) $\rightarrow$
  $(A + B) = (B + A))$.
  Von Eric Schmidt\index{Schmidt, Eric} am 19-Jun-2012 als redundant bewiesen und von Scott Fenton\index{Fenton, Scott} am 3-Jan-2013 formalisiert (siehe \texttt{addcom}).
\item
  $(A \in \mathbb{C} \rightarrow (A + 0) = A)$.
  Von Eric Schmidt am 19. Juni 2012 als überflüssig bewiesen und von Scott Fenton am 3. Januar 2013 formalisiert (siehe \texttt{addid1}).
\item
  $(A \in \mathbb{C} \rightarrow \exists x \in \mathbb{C} (A + x) = 0)$.
  Von Eric Schmidt für überflüssig bewiesen und am 21. Mai 2007 formalisiert (siehe \texttt{cnegex}).
\item
  $((A \in \mathbb{C} \land A \ne 0) \rightarrow \exists x \in \mathbb{C} (A \cdot x) = 1)$.
  Von Eric Schmidt für überflüssig bewiesen und am 22. Mai 2007 formalisiert (siehe \texttt{recex}).
\item
  $0 \in \mathbb{R}$.
  Von Eric Schmidt am 19-Feb-2005 als überflüssig bewiesen und am 21-Mai-2007 formalisiert (siehe \texttt{0re}).
\end{itemize}

Wir könnten 0 als axiomatisches Objekt eliminieren, indem wir es als $( ( i \cdot i ) + 1 )$ definieren und es in den Axiomen durch diesen Ausdruck ersetzen. In diesem Fall wird das Axiom ax-i2m1 überflüssig. Die übrigen Axiome würden jedoch länger und weniger intuitiv werden.

Eric Schmidts Arbeit, in der er dieses Axiomensystem\cite{Schmidt} analysiert, enthält einen Beweis dafür, dass die verbleibenden Axiome, mit der eventuellen Ausnahme von ax-mulcom, unabhängig von den anderen sind. Es ist derzeit eine offene Frage, ob ax-mulcom unabhängig von den anderen Axiomen ist.

\section{Zwei plus zwei ist gleich vier}\label{2p2e4}

Es folgt ein Beweis, dass $2 + 2 = 4$, wie im Theorem \texttt{2p2e4} in der Datenbasis \texttt{set.mm} bewiesen wird. Damit wird anschaulich demonstriert, wie ein Metamath-Beweis aussehen kann. Dieser Beweis hat vielleicht mehr Schritte, als Sie gewohnt sind, aber jeder Schritt ist streng bewiesen, bis hin zu den Axiomen der Logik und Mengenlehre. Diese Darstellung wurde ursprünglich vom Metamath-Programm als {\sc HTML}-Datei erzeugt (siehe \url{https://us.metamath.org/mpeuni/2p2e4.html}).

In der Tabelle, die den Beweis zeigt, ist "`Schritt"' die sequentielle Nummer des entsprechenden Schritts, während der zugehörige "`Ausdruck"' ein Ausdruck ist, den wir bewiesen haben. Unter "`Ref"' (Referenz) ist der Name eines Theorems oder Axioms, das diesen Ausdruck rechtfertigt, und "`Hyp"' bezieht sich auf vorangegangene Schritte (falls vorhanden), die das Theorem oder Axiom benötigt, damit wir es verwenden können.  Ausdrücke werden weiter eingerückt als die von ihnen abhängigen Ausdrücke, um ihre Abhängigkeiten zu verdeutlichen.

\begin{table}[!htbp]
\caption{Zwei plus zwei ist gleich vier}
\begin{tabular}{lllll}
\textbf{Step} & \textbf{Hyp} & \textbf{Ref} & \textbf{Expression} & \\
1  &       & df-2    & $ \hspace*{10mm} \vdash 2 = 1 + 1$  & \\
2  & 1     & oveq2i  & $ \hspace*{5mm} \vdash (2 + 2) = (2 + (1 + 1))$ & \\
3  &       & df-4    & $ \hspace*{10mm} \vdash 4 = (3 + 1)$ & \\
4  &       & df-3    & $ \hspace*{15mm} \vdash 3 = (2 + 1)$ & \\
5  & 4     & oveq1i  & $ \hspace*{10mm} \vdash (3 + 1) = ((2 + 1) + 1)$ & \\
6  &       & 2cn     & $ \hspace*{15mm} \vdash 2 \in \mathbb{C}$ & \\
7  &       & ax-1cn  & $ \hspace*{15mm} \vdash 1 \in \mathbb{C}$ & \\
8  & 6,7,7 & addassi & $ \hspace*{10mm} \vdash ((2 + 1) + 1) = (2 + (1 + 1))$ & \\
9  & 3,5,8 & 3eqtri  & $ \hspace*{5mm} \vdash 4 = (2 + (1 + 1))$ & \\
10 & 2,9   & eqtr4i  & $ \vdash (2 + 2) = 4$ & \\
\end{tabular}
\end{table}

Schritt 1 besagt, dass wir behaupten können, dass $2 = 1 + 1$ ist, weil es durch \texttt{df-2} gerechtfertigt ist. Was ist \texttt{df-2}? Es ist einfach die Definition von $2$, die in unserem System als gleich $1 + 1$ definiert ist.  Dies zeigt, wie wir Definitionen in Beweisen verwenden können.

Sehen Sie sich Schritt 2 des Beweises an. In der Spalte "`Ref"' sehen wir, dass er sich auf ein zuvor bewiesenes Theorem, \texttt{oveq2i}, bezieht. Es stellt sich heraus, dass das Theorem \texttt{oveq2i} eine Annahme erfordert, und in der Spalte Hyp von Schritt 2 geben wir an, dass Schritt 1 diese Annahme erfüllt (entspricht). Wenn wir uns \texttt{oveq2i} ansehen, stellen wir fest, dass es beweist, dass wir bei einer Annahme $A = B$ beweisen können, dass $( C F A ) = ( C F B )$. Wenn wir \texttt{oveq2i} benutzen und das Ergebnis von Schritt 1 als Annahme verwenden, bedeutet das, dass $A = 2$ und $B = ( 1 + 1 )$ innerhalb dieser Verwendung von \texttt{oveq2i} gesetzt wird. Für $C$ und $F$ können wir beliebige Werte einsetzen (vorbehaltlich der syntaktischen Einschränkungen), also können wir $C = 2$ und $F = +$ wählen, was zu unserem gewünschten Ergebnis $ (2 + 2) = (2 + (1 + 1))$ führt.

Schritt 2 ist ein Beispiel für eine Substitution. Letztendlich verwendet jeder Schritt in jedem Beweis nur diese eine Substitutionsregel. Alle Regeln der Logik und alle Axiome sind so ausgedrückt, dass sie mittels dieser einen Substitutionsregel verwendet werden können. Wenn Sie also einmal die Substitution beherrschen, können Sie jeden Metamath-Beweis beherrschen, ohne Ausnahmen.

Jeder Schritt ist klar und kann sofort überprüft werden. In der {\sc HTML}-Anzeige können Sie sogar auf jeden Verweis klicken, um zu sehen, warum er gerechtfertigt ist, so dass Sie leicht erkennen können, warum der Beweis funktioniert.

\section{Deduktion}\label{deduction}

Streng genommen ist eine Deduktion (auch Inferenz genannt) eine Art von Aussage, bei der einige Annahmen wahr sein müssen, damit ihre Schlussfolgerung wahr ist. Ein Theorem hingegen hat keine Hypothesen. Informell werden beide Arten von Aussagen oft als Theoreme bezeichnet, aber in diesem Abschnitt werden wir uns an die strengen Definitionen halten.

Es kommt manchmal vor, dass wir bereits eine Deduktion der Form $\varphi \Rightarrow \psi$\index{$\Rightarrow$} bewiesen haben (bei gegebener Annahme $\varphi$ können wir $\psi$ beweisen) und wir wollen dann ein Theorem der Form $\varphi \rightarrow \psi$ beweisen.

Die Umwandlung einer Deduktion (die eine Annahme verwendet) in ein Theorem (das dies nicht tut) ist nicht so einfach, wie man vielleicht denkt. Die Deduktion besagt: "`Wenn wir $\varphi$ beweisen können, dann können wir auch $\psi$ beweisen"', was in gewisser Weise schwächer ist als die Aussage "`$\varphi$ impliziert $\psi$"'. Es gibt kein Axiom der Logik, das uns erlaubt, das Theorem direkt aus der Deduktion zu erhalten.\footnote{Die Umwandlung einer Deduktion in ein Theorem gilt nicht einmal allgemein für die Quantenlogik, die eine schwache Untermenge der klassischen Aussagenlogik ist. Es wurde gezeigt, dass das Hinzufügen des Standard-Deduktionstheorems (siehe unten) zur Quantenlogik diese zur klassischen Aussagenlogik macht!}

Dies steht im Gegensatz zum umgekehrten Weg. Wenn wir das Theorem ($\varphi \rightarrow \psi$) haben, ist es einfach, die Deduktion ($\varphi \Rightarrow \psi$) mit Hilfe des Modus ponens\index{Modus ponens} (\texttt{ax-mp}; siehe Abschnitt \ref{axmp}) wiederherzustellen.

In den folgenden Unterabschnitten besprechen wir zunächst das Standard-Deduktionstheorem (die traditionelle, aber umständliche Art, Deduktionen in Theoreme umzuwandeln) und das Theorem der schwachen Deduktion (eine eingeschränkte Version des Standard-Deduktionstheorems, die einfacher zu handhaben ist und früher in der Mengenlehre-Datenbasis \texttt{set.mm}\index{Mengenlehre-Datenbasis (\texttt{set.mm})} weit verbreitet war. In Abschnitt \ref{deductionstyle} besprechen wir den Deduktionsstil, den neueren Ansatz, den wir jetzt in den meisten Fällen empfehlen. Der Deduktionsstil verwendet die "`Deduktionsform"', eine Form, bei der jeder Annahme (außer Definitionen) und der Schlussfolgerung eine universelle Prämisse vorangestellt wird ("`$\varphi \rightarrow$"'). Der Deduktionsstil ist in \texttt{set.mm} weit verbreitet, so dass es nützlich ist, ihn zu verstehen und zu begreifen, warum er weit verbreitet ist. In Abschnitt \ref{naturaldeduction} wird kurz unser Ansatz zur Verwendung der natürlichen Deduktion in \texttt{set.mm} erörtert, da dieser Ansatz eng mit dem Deduktionsstil verbunden ist. Wir schließen mit einer Zusammenfassung der Stärken unseres Ansatzes, die wir für überzeugend halten.

\subsection{Das Standard-Deduktionstheorem}\label{standarddeductiontheorem}

Die Informationen, die in der Deduktion oder ihrem Beweis enthalten sind, können genutzt werden, um uns beim Beweis des zugehörigen Theorems zu helfen. In traditionellen Logikbüchern gibt es ein Metatheorem, das sogenannte Deduktionstheorem\index{Deduktionstheorem}\index{Standard-Deduktionstheorem}, das von Herbrand und Tarski um 1930 unabhängig voneinander entdeckt wurde. Das Deduktionstheorem, das wir oft als Standard-Deduktionstheorem bezeichnen, liefert einen Algorithmus für die Konstruktion eines Beweises eines Theorems aus dem Beweis seiner entsprechenden Deduktion. Siehe z. B. \cite[S.~56]{Margaris}\index{Margaris, Angelo}. Um einen Beweis für ein Theorem zu konstruieren, betrachtet der Algorithmus jeden Schritt im Beweis der ursprünglichen Deduktion und ersetzt den Schritt durch mehrere Schritte, wobei die Annahme eliminiert und zu einer Prämisse wird.

In der gewöhnlichen Mathematik führt niemand den Algorithmus tatsäch\-lich aus, weil er (in seiner einfachsten Form) eine exponentielle Explosion der Anzahl der Beweisschritte mit sich bringt, je mehr Annahmen eliminiert werden. Stattdessen beruft man sich auf das Standard-Deduktionstheorem, um zu behaupten, dass der Algorithmus prinzipiell durchführbar ist, ohne ihn tatsächlich auszuführen. Außerdem ist der Algorithmus nicht so einfach, wie er auf den ersten Blick erscheinen mag, wenn man ihn rigoros anwendet. Es gibt eine subtile Einschränkung des Standard-Deduktionstheorems, die bei der Arbeit mit der Prädikatenlogik berücksichtigt werden muss, nämlich das Axiom der Verallgemeinerung (weitere Einzelheiten finden Sie in der Literatur).

Eines der Ziele von Metamath ist es, mit möglichst wenigen zugrundeliegenden Konzepten deutlich zu machen, wie Mathematik direkt aus den Axiomen abgeleitet werden kann, und nicht indirekt nach irgendwelchen versteckten Regeln, die in einem Programm vergraben sind oder nur von Logikern verstanden werden. Wenn wir das Standard-Deduktionstheorem zur Sprache und zum Beweisverifizierer hinzufügen würden, würde das beides stark verkomplizieren und Metamaths Ziel der Einfachheit weitgehend zunichte machen. Im Prinzip könnten wir direkt Beweise erstellen, indem wir die vom Algorithmus des Standard-Deduktionstheorems generierten Beweisschritte erweitern.  Aber das ist in der Praxis kaum machbar, weil die Anzahl der Beweisschritte schnell riesig, ja sogar astronomisch groß wird. Da der Algorithmus des Standard-Deduktionstheorems durch den Beweis gesteuert wird, müssten wir diesen Beweis noch einmal von vorne durchgehen - ausgehend von den Axiomen -, um das entsprechende Theorem zu erhalten. In Bezug auf die Länge des Beweises würde es keine Einsparungen geben, wenn man das Theorem direkt beweist, anstatt zuerst die Deduktionsform zu beweisen.

\subsection{Das Theorem der schwachen Deduktion}\label{weakdeductiontheorem}

Wir haben eine effizientere Methode entwickelt, um ein Theorem aus einer Deduktion zu beweisen, die in vielen (aber nicht allen) Fällen anstelle des Standard-Deduktionstheorems verwendet werden kann. Wir nennen diese effizientere Methode das Theorem der schwachen Deduktion\index{Theorem der schwachen Deduktion}.\footnote{Es gibt auch ein davon unabhängiges "`Theorem der schwachen Deduktion"' im Bereich der Relevanzlogik, so dass wir, um Verwirrung zu vermeiden, unser Theorem "`Theorem der schwachen Deduktion für die klassische Logik"' nennen könnten.} Im Gegensatz zum Standard-Deduktionstheorem erzeugt das Theorem der schwachen Deduktion das Theorem direkt aus einer speziellen Substitutionsinstanz der Deduktion, wobei eine kleine, feste Anzahl von Schritten verwendet wird, die ungefähr proportional zur Länge des endgültigen Theorems ist.

Wenn Sie auf einen Beweis stoßen, der auf das Theorem der schwachen Deduktion \texttt{dedth} (oder eine seiner Varianten \texttt{dedthxx}) verweist, können Sie dem Beweis folgendermaßen folgen, ohne sich in die Details zu vertiefen: Klicken Sie einfach auf das Theorem, auf das in dem Schritt direkt vor dem Verweis auf \texttt{dedth} verwiesen wird, und ignorieren Sie alles andere. Das Theorem \texttt{dedth} verwandelt einfach eine Annahme in eine Prämisse (d.h. die Annahme, gefolgt von $\rightarrow$, wird vor die Behauptung gestellt, und die Annahme selbst wird eliminiert), wenn bestimmte Bedingungen erfüllt sind.

Das Theorem der schwachen Deduktion eliminiert eine Annahme $\varphi$ und macht sie zu einer Prämisse. Dies geschieht durch den Beweis eines Ausdrucks $ \varphi \rightarrow \psi $ bei zwei Annahmen:
(1)
$ ( A = {\rm if} ( \varphi , A , B ) \rightarrow ( \varphi \leftrightarrow \chi ) ) $
und
(2) $\chi$.
Man beachte, dass es für $\varphi$ einen Beweis geben muss, wenn die Klassenvariable $A$ durch eine bestimmte Klasse $B$ ersetzt wird. Die Annahme $\chi$ sollte der Inferenz zugewiesen werden. Die Details des Beweises des Theorems der schwachen Deduktion können Sie im Theorem \texttt{dedth} sehen.

Das Theorem der schwachen Deduktion ist wahrscheinlich einfacher zu verstehen, wenn man Beweise studiert, die es verwenden. Sehen wir uns zum Beispiel den Beweis von \texttt{renegcl} an, der beweist, dass $ \vdash ( A \in \mathbb{R} \rightarrow - A \in \mathbb{R} )$:

\needspace{4\baselineskip}
\begin{longtabu} {l l l X}
\textbf{Step} & \textbf{Hyp} & \textbf{Ref} & \textbf{Expression} \\
  1 &  & negeq &
  $\vdash$ $($ $A$ $=$ ${\rm if}$ $($ $A$ $\in$ $\mathbb{R}$ $,$ $A$ $,$ $1$ $)$ $\rightarrow$
  $\textrm{-}$ $A$ $=$ $\textrm{-}$ ${\rm if}$ $($ $A$ $\in$ $\mathbb{R}$
  $,$ $A$ $,$ $1$ $)$ $)$ \\
 2 & 1 & eleq1d &
    $\vdash$ $($ $A$ $=$ ${\rm if}$ $($ $A$ $\in$ $\mathbb{R}$ $,$ $A$ $,$ $1$ $)$ $\rightarrow$ $($
    $\textrm{-}$ $A$ $\in$ $\mathbb{R}$ $\leftrightarrow$
    $\textrm{-}$ ${\rm if}$ $($ $A$ $\in$ $\mathbb{R}$ $,$ $A$ $,$ $1$ $)$ $\in$
    $\mathbb{R}$ $)$ $)$ \\
 3 &  & 1re & $\vdash 1 \in \mathbb{R}$ \\
 4 & 3 & elimel &
   $\vdash {\rm if} ( A \in \mathbb{R} , A , 1 ) \in \mathbb{R}$ \\
 5 & 4 & renegcli &
   $\vdash \textrm{-} {\rm if} ( A \in \mathbb{R} , A , 1 ) \in \mathbb{R}$ \\
 6 & 2,5 & dedth &
   $\vdash ( A \in \mathbb{R} \rightarrow \textrm{-} A \in \mathbb{R}$ ) \\
\end{longtabu}

Die etwas seltsam aussehenden Schritte in \texttt{renegcl} vor Schritt 5 sind technischer Natur, die dafür sorgen, dass dieser Zauber funktioniert, und sie können für einen schnellen Überblick über den Beweis ignoriert werden. Um den "`wichtigen"' Teil des Beweises von \texttt{renegcl} weiter zu verfolgen, können Sie sich den Verweis auf \texttt{renegcli} in Schritt 5 ansehen.

Nachdem dies geklärt ist, wollen wir uns kurz ansehen, wie für \texttt{renegcl} das Theorem der schwachen Deduktion (\texttt{dedth}) verwendet wird, um seine Aufgabe zu erfüllen, falls Sie etwas Ähnliches tun oder es besser verstehen wollen. Lassen Sie uns im Beweis von \texttt{renegcl} rückwärts arbeiten. Schritt 6 wendet \texttt{dedth} an, um unser Zielergebnis $ \vdash ( A \in \mathbb{R} \rightarrow\, - A \in \mathbb{R} )$ zu erzeugen. Dies erfordert zum einen die (substituierte) Deduktion \texttt{renegcli} in Schritt 5. Von sich aus beweist \texttt{renegcli} die Deduktion $ \vdash A \in \mathbb{R} \Rightarrow\, \vdash - A \in \mathbb{R}$; dies ist die Deduktionsform, die wir in Theoremform zu bringen versuchen, und somit hat \texttt{renegcli} eine eigene Annahme, die erfüllt werden muss. Um die Annahme des Aufrufs von \texttt{renegcli} in Schritt 5 zu erfüllen, wird sie schließlich auf das bereits bewiesene Theorem $1 \in \mathbb{R}$ in Schritt 3 reduziert. Schritt 4 verbindet die Schritte 3 und 5; Schritt 4 ruft \texttt{elimel} auf, einen Spezialfall von \texttt{elimhyp}, der eine Ist-Element-Von-Hypothese für das Theorem der schwachen Deduktion eliminiert\footnote{Anm. der Übersetzer: Das Theorem \texttt{elimel} besagt $ B \in C \Rightarrow\, {\rm if} ( A \in C , A , B ) \in C $, also dass entweder der "`then"'-Teil des if-Statements in $ C $ enthalten ist, was wegen der "`if"'-Bedingungung der Fall ist, oder der "`else"'-Teil, was aus der Annahme $ B \in C $ folgt.}.

Andererseits muss die Äquivalenz der Schlussfolgerung von \texttt{renegcl} $( - A \in \mathbb{R} )$ und der substituierten Schlussfolgerung von \texttt{renegcli} bewiesen werden, was in Schritt 2 und 1 geschieht.

Das Theorem der schwachen Deduktion hat seine Grenzen. Insbesondere müssen wir in der Lage sein, einen Spezialfall der Annahme der Deduktion als eigenständiges Theorem zu beweisen. Wir haben zum Beispiel $1 \in \mathbb{R}$ in Schritt 3 von \texttt{renegcl} verwendet.

Früher haben wir das Theorem der schwachen Deduktion ausgiebig in \texttt{set.mm} verwendet. Inzwischen empfehlen wir jedoch in den meisten Fällen die Anwendung des "`Deduktionsstils"', da der Deduktionsstil oft eine einfachere und klarere Vorgehensweise ermöglicht. Daher werden wir nun den Deduktionsstil beschreiben.

\subsection{Deduktionsstil}\label{deductionstyle}

Wir ziehen es jetzt vor, Behauptungen in "`Deduktionsform"' zu schreiben, um die Verwendung des Standard-Deduktionstheorems oder des Theorems der schwachen Deduktion in Beweisen zu vermeiden. Wir nennen diesen Ansatz "`Deduktionsstil"'.\index{Deduktionsstil}

Es wird einfacher sein, dies zu erklären, wenn man zunächst einige Begriffe definiert:

\begin{itemize}
\item \textbf{geschlossene Form}\index{geschlossene Form}\index{Formen!geschlossen}:
Behauptungen (Theoreme) ohne Hypothesen. Normalerweise hat ihre Bezeichnung kein spezielles Suffix. Ein Beispiel ist \texttt{unss}, das besagt:
$\vdash ( ( A \subseteq C \wedge B \subseteq C ) \leftrightarrow ( A \cup B )
\subseteq C )\label{eq:unss}$
\item \textbf{Deduktionsform}\index{Deduktionsform}\index{Formen!Deduktion}:
Behauptungen mit einer oder mehreren Hypothesen, bei der die Schlussfolgerung eine Implikation mit einer wff-Variablen als Prämisse (normalerweise $\varphi$) ist und jede Hypothese (\$e Aussage) entweder (1) eine Implikation mit derselben Prämisse wie die der Schlussfolgerung oder (2) eine Definition ist. Eine Definition kann für eine Klassenvariable (dies ist eine Klassenvariable, gefolgt von "`="') oder eine wff-Variable (dies ist eine wff-Variable, gefolgt von $\leftrightarrow$) erfolgen; Klassenvariablendefinitionen sind häufiger. In der Praxis wird ein Beweis in Deduktionsform auch viele Schritte enthalten, die Implikationen sind, bei denen die Prämisse entweder diese wff-Variable ist (normalerweise $\varphi$) oder eine Konjunktion (...$\land$...), die diese wff-Variable ($\varphi$) enthält. Wenn eine Behauptung in Deduktionsform vorliegt, und auch andere Formen möglich sind, dann fügen wir ihrer Bezeichnung den Suffix "`d"' hinzu. Ein Beispiel dafür ist \texttt{unssd}, das besagt\footnote{Der Kürze halber zeigen wir hier (und an anderen Stellen) ein $\&$\index{$\&$} zwischen Hypothesen\index{Hypothese} und ein $\Rightarrow$\index{$\Rightarrow$}\index{Schlussfolgerung} zwischen den Hypothesen und der Schlussfolgerung. Diese Notation ist technisch gesehen nicht Teil der Metamath-Sprache, sondern eine bequeme Abkürzung, um sowohl die Hypothesen als auch die Schlussfolgerung darzustellen.}:
$\vdash ( \varphi \rightarrow A \subseteq C )\quad\&\quad \vdash ( \varphi
    \rightarrow B \subseteq C )\quad\Rightarrow\quad \vdash ( \varphi
    \rightarrow ( A \cup B ) \subseteq C )\label{eq:unssd}$
\item \textbf{Inferenzform}\index{Inferenzform}\index{Formen!Inferenz}:
Behauptungen mit einer oder mehreren Annahmen, die nicht in Deduktionsform vorliegen (z.B. gibt es keine gemeinsame Prämisse). Liegt eine Behauptung in der Inferenzform vor und sind auch andere Formen möglich, so fügen wir der Bezeichnung ein "`i"' hinzu. Ein Beispiel ist \texttt{unssi}, das besagt:
$\vdash A \subseteq C\quad\&\quad \vdash B \subseteq C\quad\Rightarrow\quad
    \vdash ( A \cup B ) \subseteq C\label{eq:unssi}$
\end{itemize}

Wenn wir den Deduktionsstil verwenden, drücken wir eine Behauptung in der Deduktionsform aus. In dieser Form wird jeder Annahme (mit Ausnahme von Definitionen) und der Schlussfolgerung eine universelle Prämisse ("`$\varphi \rightarrow$"') vorangestellt. Die Prämisse (z.B. $\varphi$) ahmt den Kontext nach, der im Deduktionstheorem behandelt wird, so dass es nicht notwendig ist, das Deduktionstheorem direkt zu verwenden.

Sobald Sie eine Behauptung in Deduktionsform haben, können Sie sie leicht in die Inferenzform oder geschlossene Form umwandeln:

\begin{itemize}
\item Um eine Behauptung Ti in Inferenzform zu beweisen, wenn die Behauptung Td in Deduktionsform vorliegt, gibt es einen einfachen mechanischen Prozess, den man anwenden kann. Zuerst nimmt man jede Annahme Ti und fügt ein \texttt{T.} $\rightarrow$-Präfix ("`wahr impliziert"') unter Verwendung von \texttt{a1i} ein. Sie können dann die vorhandene Behauptung Td verwenden, um die resultierende Schlussfolgerung mit einem \texttt{T.} $\rightarrow$-Präfix beweisen. Schließlich können Sie dieses Präfix mit \texttt{mptru} entfernen, was zu der Schlussfolgerung führt, die Sie beweisen wollten\footnote{Anm. der Übersetzer: Siehe zum Beispiel \texttt{hadbi123i} oder \texttt{abeq2i}.}. 
\item Um eine Behauptung T in geschlossener Form zu beweisen, wenn die Behauptung Td in Deduktionsform vorliegt, gibt es ein weiteres einfaches mechanisches Verfahren, das Sie anwenden können. Wählen Sie zunächst einen Ausdruck, der die Konjunktion (...$\land$...) aller Folgerungen jeder Annahme von Td ist. Beweisen Sie dann, dass dieser Ausdruck jede der einzelnen Annahmen von Td impliziert, indem Sie die Konjunktionen eliminieren (es gibt eine Reihe von bewiesenen Behauptungen, um dies zu tun, einschließlich
\texttt{simpl},
\texttt{simpr},
\texttt{3simpa},
\texttt{3simpb},
\texttt{3simpc},
\texttt{simp1},
\texttt{simp2},
und
\texttt{simp3}).
Wenn der Ausdruck verschachtelte Konjunktionen hat, können die inneren Konjunktionen durch Verkettung der obigen Theoreme mit \texttt{syl} herausgebrochen werden (siehe Abschnitt \ref{syl}).\footnote{Es gibt tatsächlich viele Theoreme (mit simp* gekennzeichnet, wie z.B. \texttt{simp333}), die innere Konjunktionen in einem Schritt auflösen. Aber anstatt sie alle zu lernen, können Sie einfach die gerade beschriebene Verkettung für den Beweis verwenden, und dann den Metamath-Programmbefehl \texttt{minimize{\char`\_}with}\index{\texttt{minimize{\char`\_}with}-Befehl} die richtigen speziellen simp*-Theoreme herausfinden lassen, die die Verkettungen aufzulösen.} Als letzten Schritt können Sie dann die bereits bewiesene Behauptung Td (die in Deduktionsform vorliegt) anwenden und die Behauptung T in geschlossener Form beweisen.
\end{itemize}

Wir können auch jede Behauptung T in geschlossener Form leicht in die zugehörige Behauptung Ti in Inferenzform umwandeln, indem wir den Modus ponens\index{Modus ponens} anwenden (siehe Abschnitt \ref{axmp}) \footnote{Anm. der Übersetzer: Eine Behauptung T in geschlossener Form kann auch leicht in die zugehörige Behauptung Td in Deduktionsform umgewandelt werden, indem die Prämisse von T durch $\varphi$ ersetzt wird und für jedes Konjunkt aus der ursprünglichen Prämisse eine Hypothese bestehend aus dem Konjunkt mit vorangestelltem  $\varphi \rightarrow $ ergänzt wird.}.

Die gemeinsame Prämisse in der Deduktionsform kann auch verwendet werden, um den Kontext darzustellen, der für die Unterstützung von Systemen des natürlichen Schließens notwendig ist. Daher werden wir nun die natürliche Deduktion diskutieren.

\subsection{Natürliche Deduktion}\label{naturaldeduction}

Systeme des natürlichen Schließens oder der natürlichen Deduktion\index{natürliche Deduktion} (ND) als solche wurden ursprünglich 1934 von zwei unabhängig voneinander arbeitenden Logikern eingeführt: Ja\'skowski und Gentzen. ND-Systeme sollen auf formal ordentliche Weise traditionelle Methoden des mathematischen Schließens (wie den bedingten Beweis, den indirekten Beweis und den Beweis durch Fallunterscheidung) rekonstruieren. Als Rekonstruktionen wurden sie natürlich durch frühere Arbeiten beeinflusst, und viele spezifische ND-Systeme und Notationen wurden seit ihrem Ursprung entwickelt.

Es gibt viele ND-Varianten, aber Indrzejczak\cite[S.~31-32]{Indrzejczak}\index{Indrzejczak, Andrzej} schlägt vor, dass jedes System des natürlichen Schließens zumindest diese drei Kriterien erfüllen muss:

\begin{itemize}
\item "`Es gibt Möglichkeiten, um Annahmen in einen Beweis einzufügen und auch um sie zu eliminieren. Gewöhnlich bedarf es einiger buchhalterischer Hilfsmittel, um den Gültigkeitsbereich einer Annahme anzugeben und zu zeigen, dass ein Teil eines Beweises, der von einer eliminierten Annahme abhängt, entlastet wird.
\item Es gibt keine (oder zumindest eine sehr begrenzte Menge von) Axiomen, weil ihre Rolle von der Menge der primitiven Regeln für die Einführung und Eliminierung logischer Konstanten übernommen wird, was bedeutet, dass elementare Schlussfolgerungen anstelle von Formeln als primitiv angesehen werden.
\item (Ein echtes) ND-System erlaubt eine große Freiheit bei der Konstruktion von Beweisen und die Möglichkeit, verschiedene Strategien der Beweissuche anzuwenden, wie den bedingten Beweis, den Beweis durch Fallunterscheidungen, den Beweis durch reductio ad absurdum usw."'
\end{itemize}

Der Metamath Proof Explorer (MPE), wie er in \texttt{set.mm} definiert ist, ist im Grunde ein System im Hilbert-Stil. Das heißt, MPE basiert auf einer größeren Anzahl von Axiomen (im Vergleich zu Systemen der natürlichen Deduktion), einer sehr kleinen Menge von Schlussregeln (Modus ponens), und der Kontext wird nicht durch die Inferenzregeln in der Mitte eines Beweises verändert. Abgesehen davon können MPE-Beweise mit dem Ansatz der natürlichen Deduktion (ND), wie er ursprünglich von Ja\'skowski und Gentzen entwickelt wurde, erstellt werden.

Der gebräuchlichste und empfohlene Ansatz für die Anwendung von ND in MPE ist die Verwendung der Deduktionsform \index{Deduktionsform}%
\index{Formen!Deduktion} und die Anwendung der in MPE bewiesenen Behauptungen, die den ND-Regeln entsprechen. Zum Beispiel ist MPE's \texttt{jca} äquivalent zur ND-Regel $\land$-I (and-insertion). Wir haben eine Liste von Äquivalenzen erstellt, die Sie einsehen können. Dieser Ansatz für die Anwendung eines ND-Ansatzes innerhalb von MPE stützt sich im Wesentlichen auf Metamaths wff-Metavariablen und wird in der Präsentation "`Natural Deductions in the Metamath Proof Language"' von Mario Carneiro \cite{CarneiroND}\index{Carneiro, Mario} näher beschrieben.

In diesem Stil sind viele Schritte eine Implikation, deren Prämissen den Kontext ($\Gamma$) der meisten ND-Systeme nachahmt. Um eine Annahme hinzuzufügen, fügen Sie sie einfach der Implikationsprämisse hinzu (typischerweise unter Verwendung von \texttt{simpr}) und verwenden diese neue Prämisse für alle späteren Behauptungen im selben Bereich. Wenn Sie eine Behauptung in einem ND-Hypothesenbereich verwenden wollen, der außerhalb des aktuellen ND-Hypothesenbereichs liegt, ändern Sie die Behauptung so, dass die ND-Hypothesenannahme zu ihrer Prämisse hinzugefügt wird (typischerweise mit \texttt{adantr}). Die meisten Beweisschritte werden mit Hilfe von Regeln bewiesen, die Hypothesen und Ergebnisse der Form $\varphi \rightarrow$ ... haben.

Ein Beispiel mag dies deutlicher machen. Schauen wir uns Theorem 5.5 von \cite[S.~18]{Clemente}\index{Clemente Laboreo, Daniel} zusammen mit einer zeilenweisen Übersetzung unter Verwendung der üblichen Übersetzung der natürlichen Deduktion (ND) in die Metamath Proof Explorer (MPE) Notation an (dies ist Beweis \texttt{ex-natded5.5}). Das ursprüngliche Ziel des Beweises war der Beweis von $ \lnot \psi$ unter zwei Annahmen, $( \psi \rightarrow \chi )$ und $ \lnot \chi$. Wir werden diese Aussagen in die MPE-Deduktionsform übersetzen, indem wir ihnen allen das Präfix $\varphi \rightarrow$ voranstellen. In MPE lautet das Ziel also $( \varphi \rightarrow \lnot \psi )$, und die beiden Hypothesen lauten $( \varphi \rightarrow ( \psi \rightarrow \chi )$ und $( \varphi \rightarrow \lnot \chi )$.

Die folgende Tabelle zeigt den Beweis im Stil der natürlichen Deduktion von Fitch und seine MPE-Entsprechung. Die Spalte \textit{\#} zeigt die ursprüngliche Nummerierung, \textit{MPE\#} zeigt die Nummer im äquivalenten MPE-Beweis (den wir später zeigen werden), \textit{ND-Ausdruck} zeigt die ursprüngliche Beweisbehauptung in ND-Notation, und \textit{MPE-Übersetzung} zeigt ihre Übersetzung in MPE, wie in diesem Abschnitt diskutiert. Die letzten Spalten zeigen die Begründung in ND bzw. MPE.

\needspace{4\baselineskip}
{\setlength{\extrarowsep}{4pt} % Keep rows from being too close together
\begin{longtabu}   { @{} c c X X X X }
\textbf{\#} & \textbf{MPE\#} & \textbf{ND-Ausdruck} &
\textbf{MPE-Über\-setzung} & \textbf{ND-Begrün\-dung} &
\textbf{MPE-Begrün\-dung} \\
\endhead

1 & 2;3 &
$( \psi \rightarrow \chi )$ &
$( \varphi \rightarrow ( \psi \rightarrow \chi ) )$ &
gegeben &
\$e; \texttt{adantr} um die ND-Hypothese einzufügen \\

2 & 5 &
$ \lnot \chi$ &
$( \varphi \rightarrow \lnot \chi )$ &
gegeben &
\$e; \texttt{adantr} um die ND-Hypothese einzufügen \\

3 & 1 &
... $\vert$ $\psi$ &
$( \varphi \rightarrow \psi )$ &
Annahme der ND-Hypothese &
\texttt{simpr} \\

4 & 4 &
... $\chi$ &
$( ( \varphi \land \psi ) \rightarrow \chi )$ &
$\rightarrow$\,E 1,3 &
\texttt{mpd} 1,3 \\

5 & 6 &
... $\lnot \chi$ &
$( ( \varphi \land \psi ) \rightarrow \lnot \chi )$ &
IT 2 &
\texttt{adantr} 5 \\

6 & 7 &
$\lnot \psi$ &
$( \varphi \rightarrow \lnot \psi )$ &
$\land$\,I 3,4,5 &
\texttt{pm2.65da} 4,6 \\

\end{longtabu}
}


Im Original wurden lateinische Buchstaben verwendet; wir haben sie durch griechische Buchstaben ersetzt, um den Metamath-Namenskonventionen zu folgen und um die Metamath-Übersetzung leichter nachvollziehen zu können. Die zeilenweise Metamath-Übersetzung dieses Ansatzes der natürlichen Deduktion stellt jeder Zeile eine Prämisse mit $\varphi$ voran und verwendet die Metamath-Äquivalente der natürlichen Deduktionsregeln. Um eine Annahme hinzuzufügen, wird die Prämisse so modifiziert, dass sie sie enthält (typischerweise durch Verwendung von \texttt{adantr}; \texttt{simpr} ist nützlich, wenn man eine direkte Abhängigkeit von der neuen Annahme haben möchte, wie hier gezeigt).

In Metamath können wir die beiden gegebenen Aussagen als folgende Hypothesen darstellen:

\needspace{2\baselineskip}
\begin{itemize}
\item ex-natded5.5.1 $\vdash ( \varphi \rightarrow ( \psi \rightarrow \chi ) )$
\item ex-natded5.5.2 $\vdash ( \varphi \rightarrow \lnot \chi )$
\end{itemize}

\needspace{4\baselineskip}
Hier ist der Beweis in Metamath als zeilenweise Übersetzung:

\begin{longtabu}   { l l l X }
\textbf{Step} & \textbf{Hyp} & \textbf{Ref} & \textbf{Ex\-pres\-sion} \\
\endhead
1 & & simpr & $\vdash ( ( \varphi \land \psi ) \rightarrow \psi )$ \\
2 & & ex-natded5.5.1 &
  $\vdash ( \varphi \rightarrow ( \psi \rightarrow \chi ) )$ \\
3 & 2 & adantr &
 $\vdash ( ( \varphi \land \psi ) \rightarrow ( \psi \rightarrow \chi ) )$ \\
4 & 1, 3 & mpd &
 $\vdash ( ( \varphi \land \psi ) \rightarrow \chi ) $ \\
5 & & ex-natded5.5.2 &
 $\vdash ( \varphi \rightarrow \lnot \chi )$ \\
6 & 5 & adantr &
 $\vdash ( ( \varphi \land \psi ) \rightarrow \lnot \chi )$ \\
7 & 4, 6 & pm2.65da &
 $\vdash ( \varphi \rightarrow \lnot \psi )$ \\
\end{longtabu}

Die direkte Verwendung spezifischer Regeln für die natürliche Deduktion kann zu sehr langen Beweisen führen, und zwar aus genau demselben Grund, aus dem die direkte Verwendung von Axiomen in Beweisen im Hilbert-Stil zu sehr langen Beweisen führen kann. Wenn das Ziel kurze und klare Beweise sind, dann ist es besser, bereits bewiesene Behauptungen in Deduktionsform wiederzuverwenden, als jedes Mal von vorne anzufangen und nur grundlegende natürliche Deduktionsregeln zu verwenden.

\subsection{Die Stärken unseres Ansatzes}

Soweit wir wissen, gibt es in der Literatur weder das Theorem der schwachen Deduktion noch die natürliche Deduktionsmethode von Mario Carneiro\index{Carneiro, Mario}. Um eine Annahme in eine Prämisse umzuwandeln, benötigt das in der Literatur übliche "`Deduktionstheorem"'\index{Deduktionstheorem}\index{Standard-Deduktionstheorem} eine Metalogik außerhalb der vom Axiomensystem bereitgestellten Begriffe. Stattdessen bevorzugen wir im Allgemeinen die Methode der natürlichen Deduktion von Mario Carneiro, verwenden dann das schwache Deduktionstheorem in Fällen, in denen es schwierig ist, es anzuwenden, und verwenden erst dann das vollständige Standard-Deduktionstheorem als letzten Ausweg.

Das Theorem der schwachen Deduktion\index{Theorem der schwachen Deduktion} erfordert keine zusätzliche Metalogik, sondern wandelt eine Schlussfolgerung direkt in ein Theorem in geschlossener Form um, mit einem strengen Beweis, der nur das Axiomensystem verwendet. Im Gegensatz zum Standard-Deduktionstheorem gibt es keine implizite externe Rechtfertigung, auf die wir vertrauen müssen, um es anzuwenden.

Die Methode der natürlichen Deduktion\index{natürliche Deduktion} von Mario Carneiro erfordert ebenfalls keine neuen metalogischen Begriffe. Sie umgeht die Metalogik des Deduktionstheorems, indem sie den Annahmen und Schlussfolgerungen jeder möglichen Schlussfolgerung von Anfang an eine universelle Prämisse ("`$\varphi \rightarrow$"') voranstellt.

Wir finden es beeindruckend und befriedigend, dass wir so viel im praktischen Sinne tun können, ohne unser Hilbert-artiges Axiomensystem zu verlassen. Natürlich enthält unsere Axiomatisierung, die in Form von Schemata vorliegt, eine eigene Metalogik, die wir ausnutzen. Aber diese Metalogik ist relativ einfach, und für unsere Alternativen zum Deduktionstheorem verwenden wir in erster Linie nur die direkte Substitution von Ausdrücken für Metavariablen.

\begin{sloppy}
\section{Erforschung der Mengenlehre-Datenbasis}\label{exploring}
\end{sloppy}
% NOTE: All examples performed in this section are
% recorded wtih "set width 61" % on set.mm as of 2019-05-28
% commit c1e7849557661260f77cfdf0f97ac4354fbb4f4d.

An dieser Stelle möchten Sie vielleicht die Datei \texttt{set.mm}\index{Mengenlehre-Datenbasis (\texttt{set.mm})} genauer studieren.  Achten Sie insbesondere auf die Annahmen, die zur Definition von wffs\index{wohlgeformte Formel (wff)} (die oben nicht enthalten sind) benötigt werden, auf die Variablentypen (\texttt{\$f}\index{\texttt{\$f}-Anweisung}-Anweisungen) und auf die eingeführten Definitionen.  Beginnen Sie mit einigen einfachen Theoremen der Aussagenlogik und stellen Sie sicher, dass Sie jeden Schritt eines Beweises im Detail verstehen.  Sobald Sie die ersten paar Beweise hinter sich gebracht haben und mit der Metamath-Sprache vertraut sind, wird jeder Teil der \texttt{set.mm}-Datenbasis Schritt für Schritt genauso einfach zu verstehen sein wie jeder andere Teil - Sie müssen keinen "`Quantensprung"' bezüglich der mathematischen Raffinesse durchmachen, um einem tiefgehenden Beweis in der Mengenlehre folgen zu können.

Als Nächstes möchten Sie vielleicht untersuchen, wie Konzepte wie die natürlichen Zahlen definiert und beschrieben werden.  Dies geschieht wahrscheinlich am besten in Verbindung mit Standard-Lehrbüchern der Mengenlehre, die Ihnen ein besseres Verständnis vermitteln können.  Die Datenbasis \texttt{set.mm} bietet Referenzen, mit denen Sie beginnen können.  Von dort aus beginnt Ihr Weg zu einem sehr tiefgehenden, rigorosen Verständnis der abstrakten Mathematik.

Das Programm Metamath\index{Metamath} kann Ihnen dabei helfen, eine Metamath-Datenbasis durchzugehen, sei es um herauszufinden, wie ein bestimmter Schritt in einem Beweis erfolgt, oder nur aus allgemeiner Neugier.  Wir werden einige Beispiele für die Befehle durchgehen und dabei die Datenbasis der Mengenlehre \texttt{set.mm}\index{Mengenlehre-Datenbasis (\texttt{set.mm})} verwenden, die mit der Metamath-Software geliefert wird.  Diese sollten Ihnen den Einstieg erleichtern.  Siehe Kapitel~\ref{commands} für eine detailliertere Beschreibung der Befehle.  Beachten Sie, dass wir die vollständige Schreibweise aller Befehle angegeben haben, um Mehrdeutigkeiten bei zukünftigen Befehlen zu vermeiden.  In der praktischen Arbeit brauchen Sie nur die Zeichen eingeben, die erforderlich sind, um jeden Befehl keyword\index{Befehlsschlüsselwort} eindeutig zu identifizieren, oft nur ein oder zwei Zeichen pro Schlüsselwort, und Sie brauchen sie nicht in Großbuchstaben zu schreiben.

Führen Sie zunächst das Programm Metamath wie oben beschrieben aus.  Sie sollten die Eingabeaufforderung \verb/MM>/ sehen.  Lesen Sie die Datei \texttt{set.mm} ein:\index{\texttt{read}-Befehl}

\begin{verbatim}
MM> read set.mm
Reading source file "set.mm"... 34554442 bytes
34554442 bytes were read into the source buffer.
The source has 155711 statements; 2254 are $a and 32250 are $p.
No errors were found.  However, proofs were not checked.
Type VERIFY PROOF * if you want to check them.
\end{verbatim}

Wie bei den meisten Beispielen in diesem Buch wird das, was Sie sehen werden, leicht von den dargestellten Inhalten abweichen, da wir unsere Datenbasen (einschließlich \texttt{set.mm}) ständig verbessern.

Prüfen wir die Integrität der Datenbasis.  Dieser Vorgang kann ein oder zwei Minuten dauern, wenn Ihr Computer langsam ist.

\begin{verbatim}
MM> verify proof *
0 10%  20%  30%  40%  50%  60%  70%  80%  90% 100%
..................................................
All proofs in the database were verified in 2.84 s.
\end{verbatim}

Es wurden keine Fehler gemeldet, so dass jeder Beweis korrekt ist.

Sie müssen die Namen (Bezeichnungen) der Theoreme kennen, bevor Sie sich diese ansehen können. Oft ist das Durchsuchen der Datenbasisdatei(en) mit einem Texteditor die beste Vorgehensweise.  In \texttt{set.mm} gibt es viele detaillierte Kommentare, vor allem am Anfang, die Ihnen helfen können. Der Befehl \texttt{search} im Programm Metamath ist ebenfalls sehr nützlich.  Die Option \texttt{comments} listet die Aussagen auf, deren zugehöriger Kommentar (der unmittelbar vor der Aussage steht) eine von Ihnen angegebene Zeichenfolge enthält.  Wenn Sie zum Beispiel Endertons {\em Elements of Set Theory} studieren \cite{Enderton}\index{Enderton, Herbert B.}, möchten Sie vielleicht die Verweise darauf in der Datenbasis finden.  Bei der zu suchenden Zeichenfolge \texttt{enderton} wird nicht zwischen Groß- und Kleinschreibung unterschieden.  (Auf diese Weise werden nicht alle Theoreme in der Datenbasis angezeigt, die in Endertons Buch enthalten sind, da es für ein bestimmtes Theorem, das in mehreren Lehrbüchern vorkommen kann, in der Regel nur ein einziges Zitat gibt.)\index{\texttt{search}-Befehl}

\begin{verbatim}
MM> search * "enderton" / comments
12067 unineq $p "... Exercise 20 of [Enderton] p. 32 and ..."
12459 undif2 $p "...Corollary 6K of [Enderton] p. 144. (C..."
12953 df-tp $a "...s. Definition of [Enderton] p. 19. (Co..."
13689 unissb $p ".... Exercise 5 of [Enderton] p. 26 and ..."
\end{verbatim}
\begin{center}
(etc.)
\end{center}

Oder Sie möchten nachsehen, welche Theoreme etwas mit Konjunktionen (logisches {\sc und}) zu tun haben.  Die Anführungszeichen um den Suchstring sind optional, wenn es keine Mehrdeutigkeit gibt.\index{\texttt{search}-Befehl}

\begin{verbatim}
MM> search * conjunction / comments
120 a1d $p "...be replaced with a conjunction ( ~ df-an )..."
662 df-bi $a "...viated form after conjunction is introdu..."
1319 wa $a "...ff definition to include conjunction ('and')."
1321 df-an $a "Define conjunction (logical 'and'). Defini..."
1420 imnan $p "...tion in terms of conjunction. (Contribu..."
\end{verbatim}
\begin{center}
(etc.)
\end{center}

Nun werden wir uns mit einigen Details befassen.  Schauen wir uns das erste Axiom der Aussagenlogik an (wir könnten \texttt{sh st} als Abkürzung für \texttt{show statement} verwenden).\index{\texttt{show statement}-Befehl}

\begin{verbatim}
MM> show statement ax-1/full
Statement 49 is located on line 11182 of the file "set.mm".
Its statement number for HTML pages is 6.
"Axiom _Simp_.  Axiom A1 of [Margaris] p. 49.  One of the 3
axioms of propositional calculus.  The 3 axioms are also
given as Definition 2.1 of [Hamilton] p. 28.
..."
49 ax-1 $a |- ( ph -> ( ps -> ph ) ) $.
Its mandatory hypotheses in RPN order are:
  wph $f wff ph $.
  wps $f wff ps $.
The statement and its hypotheses require the variables:
  ph ps
The variables it contains are:  ph ps
\end{verbatim}

Vergleichen Sie dies mit \texttt{ax-1} auf Seite~\pageref{ax1}.  Sie sehen, dass zum Beispiel das Symbol \texttt{ph} die {\sc ascii}-Notation für $\varphi$ ist.  Um die mathematischen Symbole für einen beliebigen Ausdruck zu sehen, können Sie ihn in \LaTeX\ umsetzen (geben Sie \texttt{help tex} für Anweisungen dazu ein)\index{latex@{\LaTeX}} oder, was einfacher ist, verwenden Sie einfach einen Texteditor, um sich die Kommentare anzusehen, in denen die Symbole zuerst in \texttt{set.mm} eingeführt werden.  Die Annahmen \texttt{wph} und \texttt{wps}, die von \texttt{ax-1} verlangt werden, bedeuten, dass die Variablen \texttt{ph} und \texttt{ps} wffs sein müssen.

Als nächstes wählen wir ein einfaches Theorem der Aussagenlogik, das Gesetz der Selbstimplikation\footnote{Anm. der Übersetzer: Im Originaltext wird dieses Gesetz, wie in {\em Principia Mathematica} \cite{PM}, "`principle of identity"' genannt, was aber oft mit dem "`law of identity"' verwechselt wird, das im Deutschen "`Identitätsprinzip"' genannt wird.}, das direkt aus den Axiomen bewiesen wird.  Wir werden uns die Aussage und dann ihren Beweis ansehen.\index{\texttt{show statement}-Befehl}

\begin{verbatim}
MM> show statement id1/full
Statement 116 is located on line 11371 of the file "set.mm".
Its statement number for HTML pages is 22.
"Principle of identity.  Theorem *2.08 of [WhiteheadRussell]
p. 101.  This version is proved directly from the axioms for
demonstration purposes.
..."
116 id1 $p |- ( ph -> ph ) $= ... $.
Its mandatory hypotheses in RPN order are:
  wph $f wff ph $.
Its optional hypotheses are:  wps wch wth wta wet
      wze wsi wrh wmu wla wka
The statement and its hypotheses require the variables:  ph
These additional variables are allowed in its proof:
      ps ch th ta et ze si rh mu la ka
The variables it contains are:  ph
\end{verbatim}

Die optionalen Variablen \index{optionale Variable} \texttt{ps}, \texttt{ch}, etc.\ stehen bei Bedarf zur Verwendung in einem Beweis dieser Aussage zur Verfügung, und wenn diese verwendet würden, würden die entsprechenden optionalen Hypothesen \texttt{wps}, \texttt{wch}, etc. ebenfalls verwendet werden. (Siehe Abschnitt~\ref{dollaref} für die Bedeutung von "`optionaler Hypothese"'. \index{optionale Hypothese}) Der Grund dafür, dass diese in der Anzeige der Aussage auftauchen, ist, dass die Aussage \texttt{id1} zufällig in ihrem Gültigkeitsbereich liegt (siehe Abschnitt~\ref{scoping} für die Definition von "`Gültigkeitsbereich"'\index{Gültigkeitsbereich}), aber tatsächlich werden wir in der Aussagenlogik niemals optionale Hypothesen oder Variablen verwenden.  Dies wird wichtig, nachdem Quantoren eingeführt wurden, wo "`Dummy"'-Variablen oft in der Mitte eines Beweises benötigt werden.

Schauen wir uns den Beweis der Aussage \texttt{id1} an.  Wir verwenden den Befehl \texttt{show proof}, der standardmäßig die "`unwesentlichen"' Schritte unterdrückt, die die wffs konstruieren.\index{\texttt{show proof}-Befehl} Wir zeigen den Beweis im "`lemmon"'-Format an (ein Format ohne Einrückungen mit expliziten Verweisen auf vorherige Schrittnummern) und nummerieren die angezeigten Schritte neu:

\begin{verbatim}
MM> show proof id1 /lemmon/renumber
1 ax-1           $a |- ( ph -> ( ph -> ph ) )
2 ax-1           $a |- ( ph -> ( ( ph -> ph ) -> ph ) )
3 ax-2           $a |- ( ( ph -> ( ( ph -> ph ) -> ph ) ) ->
                     ( ( ph -> ( ph -> ph ) ) -> ( ph -> ph )
                                                          ) )
4 2,3 ax-mp      $a |- ( ( ph -> ( ph -> ph ) ) -> ( ph -> ph
                                                          ) )
5 1,4 ax-mp      $a |- ( ph -> ph )
\end{verbatim}

Wenn Sie Abschnitt~\ref{trialrun} gelesen haben, werden Sie wissen, wie Sie diesen Beweis interpretieren können.  Schritt~2 zum Beispiel ist eine Anwendung des Axioms \texttt{ax-1}.  Dieser Beweis ist identisch mit demjenigen in Hamiltons {\em Logic for Mathematicians} \cite[S.~32]{Hamilton}\index{Hamilton, Alan G.}.

Vielleicht möchten Sie sich ansehen, welche Substitutionen in \texttt{ax-1} vorgenommen werden, um zu Schritt~2 zu gelangen. Der entsprechende Befehl muss die "`echte"' Schrittnummer kennen, also zeigen wir den Beweis noch einmal ohne die Option \texttt{renumber} an.\index{\texttt{show proof}-Befehl}

\begin{verbatim}
MM> show proof id1 /lemmon
 9 ax-1          $a |- ( ph -> ( ph -> ph ) )
20 ax-1          $a |- ( ph -> ( ( ph -> ph ) -> ph ) )
24 ax-2          $a |- ( ( ph -> ( ( ph -> ph ) -> ph ) ) ->
                     ( ( ph -> ( ph -> ph ) ) -> ( ph -> ph )
                                                          ) )
25 20,24 ax-mp   $a |- ( ( ph -> ( ph -> ph ) ) -> ( ph -> ph
                                                          ) )
26 9,25 ax-mp    $a |- ( ph -> ph )
\end{verbatim}

Die "`echte"' Nummer des zu betrachtenden Schrittes ist 20.  Schauen wir uns die Details an.

\begin{verbatim}
MM> show proof id1 /detailed_step 20
Proof step 20:  min=ax-1 $a |- ( ph -> ( ( ph -> ph ) -> ph )
  )
This step assigns source "ax-1" ($a) to target "min" ($e).
The source assertion requires the hypotheses "wph" ($f, step
18) and "wps" ($f, step 19).  The parent assertion of the
target hypothesis is "ax-mp" ($a, step 25).
The source assertion before substitution was:
    ax-1 $a |- ( ph -> ( ps -> ph ) )
The following substitutions were made to the source
assertion:
    Variable  Substituted with
     ph        ph
     ps        ( ph -> ph )
The target hypothesis before substitution was:
    min $e |- ph
The following substitution was made to the target hypothesis:
    Variable  Substituted with
     ph        ( ph -> ( ( ph -> ph ) -> ph ) )
\end{verbatim}

Dies zeigt die Substitutionen\index{Substitution!Variable}\index{Variablensubstitution}, die an den Variablen in \texttt{ax-1} vorgenommen wurden.  Es wird auf die Schritte 18 und 19 verwiesen, die in unserer Beweisdarstellung nicht gezeigt werden.  Um diese Schritte zu sehen, können Sie den Beweis mit der Option \texttt{all} anzeigen. 

Sehen wir uns nun einen etwas fortgeschritteneren Beweis der Aussagenlogik an.  Beachten Sie, dass \verb+/\+ das Symbol für $\wedge$ (logisches {\sc und}, auch Konjunktion genannt) ist. \index{Konjunktion ($\wedge$)} \index{logisches {\sc und} ($\wedge$)}

\begin{verbatim}
MM> show statement prth/full
Statement 1791 is located on line 15503 of the file "set.mm".
Its statement number for HTML pages is 559.
"Conjoin antecedents and consequents of two premises.  This
is the closed theorem form of ~ anim12d .  Theorem *3.47 of
[WhiteheadRussell] p. 113.  It was proved by Leibniz,
and it evidently pleased him enough to call it
_praeclarum theorema_ (splendid theorem).
..."
1791 prth $p |- ( ( ( ph -> ps ) /\ ( ch -> th ) ) -> ( ( ph
      /\ ch ) -> ( ps /\ th ) ) ) $= ... $.
Its mandatory hypotheses in RPN order are:
  wph $f wff ph $.
  wps $f wff ps $.
  wch $f wff ch $.
  wth $f wff th $.
Its optional hypotheses are:  wta wet wze wsi wrh wmu wla wka
The statement and its hypotheses require the variables:  ph
      ps ch th
These additional variables are allowed in its proof:  ta et
      ze si rh mu la ka
The variables it contains are:  ph ps ch th


MM> show proof prth /lemmon/renumber
1 simpl          $p |- ( ( ( ph -> ps ) /\ ( ch -> th ) ) ->
                                               ( ph -> ps ) )
2 simpr          $p |- ( ( ( ph -> ps ) /\ ( ch -> th ) ) ->
                                               ( ch -> th ) )
3 1,2 anim12d    $p |- ( ( ( ph -> ps ) /\ ( ch -> th ) ) ->
                           ( ( ph /\ ch ) -> ( ps /\ th ) ) )
\end{verbatim}

Es gibt Verweise auf eine Reihe unbekannter Aussagen.  Um zu sehen, um welche es sich handelt, können Sie Folgendes eingeben:

\begin{verbatim}
MM> show proof prth /statement_summary
Summary of statements used in the proof of "prth":

Statement simpl is located on line 14748 of the file
"set.mm".
"Elimination of a conjunct.  Theorem *3.26 (Simp) of
[WhiteheadRussell] p. 112. ..."
  simpl $p |- ( ( ph /\ ps ) -> ph ) $= ... $.

Statement simpr is located on line 14777 of the file
"set.mm".
"Elimination of a conjunct.  Theorem *3.27 (Simp) of
[WhiteheadRussell] ..."
  simpr $p |- ( ( ph /\ ps ) -> ps ) $= ... $.

Statement anim12d is located on line 15445 of the file
"set.mm".
"Conjoin antecedents and consequents in a deduction.
..."
  anim12d.1 $e |- ( ph -> ( ps -> ch ) ) $.
  anim12d.2 $e |- ( ph -> ( th -> ta ) ) $.
  anim12d $p |- ( ph -> ( ( ps /\ th ) -> ( ch /\ ta ) ) )
      $= ... $.
\end{verbatim}
\begin{center}
(etc.)
\end{center}

Natürlich können Sie jede dieser Aussagen und ihre Beweise und so weiter bis zu den Axiomen der Aussagenlogik zurückverfolgen, wenn Sie möchten.

Der Befehl \texttt{search} ist nützlich, um Aussagen zu finden, deren Inhalt Sie ganz oder teilweise kennen.  Das folgende Beispiel findet alle Aussagen, die \verb@ph -> ps@ gefolgt von \verb@ch -> th@ enthalten.  Das \verb@$*@ ist ein Platzhalter, der auf alles passt; das \texttt{\$} vor dem \verb$*$ verhindert Konflikte mit Token-Namen für mathematische Symbole.  Das \verb@*@ nach \texttt{SEARCH} ist ebenfalls ein Platzhalter, der in diesem Fall "`passt auf jede Bezeichnung"' bedeutet. \index{\texttt{search}-Befehl}

% I'm omitting this one, since readers are unlikely to see it:
% 1096 bisymOLD $p |- ( ( ( ph -> ps ) -> ( ch -> th ) ) -> ( (
%   ( ps -> ph ) -> ( th -> ch ) ) -> ( ( ph <-> ps ) -> ( ch
%    <-> th ) ) ) )
\begin{verbatim}
MM> search * "ph -> ps $* ch -> th"
1791 prth $p |- ( ( ( ph -> ps ) /\ ( ch -> th ) ) -> ( ( ph
    /\ ch ) -> ( ps /\ th ) ) )
2455 pm3.48 $p |- ( ( ( ph -> ps ) /\ ( ch -> th ) ) -> ( (
    ph \/ ch ) -> ( ps \/ th ) ) )
117859 pm11.71 $p |- ( ( E. x ph /\ E. y ch ) -> ( ( A. x (
    ph -> ps ) /\ A. y ( ch -> th ) ) <-> A. x A. y ( ( ph /\
    ch ) -> ( ps /\ th ) ) ) )
\end{verbatim}

Drei Aussagen, \texttt{prth}, \texttt{pm3.48} und \texttt{pm11.71}, wurden als übereinstimmend befunden.

Um zu sehen, von welchen Axiomen\index{Axiom} und Definitionen\index{Definition} der Beweis von \texttt{prth} letztlich abhängt, können Sie das Programm die Hierarchie der Theoreme und Definitionen zurückverfolgen lassen.\index{\texttt{show trace{\char`\_}back}-Befehl}

\begin{verbatim}
MM> show trace_back prth /essential/axioms
Statement "prth" assumes the following axioms ($a
statements):
  ax-1 ax-2 ax-3 ax-mp df-bi df-an
\end{verbatim}

Beachten Sie, dass die 3 Axiome der Aussagenlogik und der Modus ponens benötigt werden (wie erwartet); außerdem gibt es eine Reihe von Definitionen, die unterwegs verwendet werden.  Beachten Sie, dass Metamath keinen Unterschied\index{Axiom vs. Definition} zwischen Axiomen\index{Axiom} und Definitionen\index{Definition} macht.  In \texttt{set.mm} wurden sie künstlich unterschieden, indem man ihren Bezeichnungen\index{Label in \texttt{set.mm}} jeweils \texttt{ax-} und \texttt{df-} voranstellte.  Zum Beispiel definiert \texttt{df-an} die Konjunktion (logisch {\sc und}), die durch das Symbol \verb+/\+ dargestellt wird. In Abschnitt~\ref{definitions} wird die Philosophie von Definitionen erörtert, und die Metamath-Sprache verfolgt einen besonders einfachen, konservativen Ansatz, indem sie die \texttt{\$a}\index{\texttt{\$a}-Anweisung}-Anweisungen sowohl für Axiome als auch für Definitionen verwendet.

Sie können das Programm auch berechnen lassen, wie viele Schritte ein Beweis hat,\index{Beweislänge} wenn wir ihn bis zu den \texttt{\$a}-Anweisungen zurückverfolgen würden.

\begin{verbatim}
MM> show trace_back prth /essential/count_steps
The statement's actual proof has 3 steps.  Backtracking, a
total of 79 different subtheorems are used.  The statement
and subtheorems have a total of 274 actual steps.  If
subtheorems used only once were eliminated, there would be a
total of 38 subtheorems, and the statement and subtheorems
would have a total of 185 steps.  The proof would have 28349
steps if fully expanded back to axiom references.  The
maximum path length is 38.  A longest path is:  prth <-
anim12d <- syl2and <- sylan2d <- ancomsd <- ancom <- pm3.22
<- pm3.21 <- pm3.2 <- ex <- sylbir <- biimpri <- bicomi <-
bicom1 <- bi2 <- dfbi1 <- impbii <- bi3 <- simprim <- impi <-
con1i <- nsyl2 <- mt3d <- con1d <- notnot1 <- con2i <- nsyl3
<- mt2d <- con2d <- notnot2 <- pm2.18d <- pm2.18 <- pm2.21 <-
pm2.21d <- a1d <- syl <- mpd <- a2i <- a2i.1 .
\end{verbatim}

Daraus ergibt sich, dass wir 274 Schritte überprüfen müssten, wenn wir den Beweis ausgehend von den Axiomen vollständig verifizieren wollen.  Es werden auch einige weitere Statistiken angezeigt.  Es gibt einen oder mehrere Pfade zurück zu den Axiomen, die am längsten sind; dieser Befehl sucht einen von ihnen heraus und zeigt ihn an.  Die Länge des längsten Pfades kann in gewisser Weise ausdrücken, wie "`tief"' das Theorem ist.

Wir könnten uns auch fragen, welche Beweise von dem Theorem \texttt{prth} abhängen.  Wenn er später nie verwendet wird, könnten wir ihn als überflüssig streichen, wenn er an sich nicht von Interesse ist.\index{\texttt{show usage}-Befehl}

% I decided to show the OLD values here.
\begin{verbatim}
MM> show usage prth
Statement "prth" is directly referenced in the proofs of 18
statements:
  mo3 moOLD 2mo 2moOLD euind reuind reuss2 reusv3i opelopabt
  wemaplem2 rexanre rlimcn2 o1of2 o1rlimmul 2sqlem6 spanuni
  heicant pm11.71
\end{verbatim}

Somit wird \texttt{prth} von 18 Beweisen direkt verwendet. Wir können die Option \texttt{/recursive} verwenden, um die indirekte Verwendung einzuschließen:

\begin{verbatim}
MM> show usage prth /recursive
Statement "prth" directly or indirectly affects the proofs of
24214 statements:
  mo3 mo mo3OLD eu2 moOLD eu2OLD eu3OLD mo4f mo4 eu4 mopick
...
\end{verbatim}

\subsection{Eine Anmerkung zum "`kompakten"'\texorpdfstring{\\}{} Beweis\-format}

Das Programm Metamath zeigt Beweise in einem "`kompakten"'\index{kompakter Beweis} Format an, wenn der Beweis in komprimiertem Format in der Datenbasis gespeichert ist.  Dies kann etwas verwirrend sein, wenn man nicht weiß, wie dies zu interpretieren ist. Wenn Sie zum Beispiel den vollständigen Beweis des Theorems \texttt{id1} anzeigen lassen, wird er wie folgt beginnen:

\begin{verbatim}
MM> show proof id1 /lemmon/all
 1 wph           $f wff ph
 2 wph           $f wff ph
 3 wph           $f wff ph
 4 2,3 wi    @4: $a wff ( ph -> ph )
 5 1,4 wi    @5: $a wff ( ph -> ( ph -> ph ) )
 6 @4            $a wff ( ph -> ph )
\end{verbatim}

\begin{center}
{etc.}
\end{center}

Schritt 4 wird ein "`lokales Label"', \index{lokales Label } \texttt{@4}, zugewiesen. Später, bei Schritt 6, wird auf dieses Label \texttt{@4} verwiesen, anstatt den expliziten Beweis für diesen Schritt anzuzeigen.  Diese Technik macht sich die Tatsache zunutze, dass sich Schritte in einem Beweis häufig wiederholen, insbesondere bei der Konstruktion von wffs.  Das kompakte Format reduziert die Anzahl der Schritte in der Darstellung des Beweises und wird von manchen Benutzern bevorzugt.

Wenn Sie das normale Format mit den "`wahren"' Schrittzahlen sehen wollen, können Sie folgende Abhilfe verwenden:\index{\texttt{save proof}-Befehl}

\begin{verbatim}
MM> save proof id1 /normal
The proof of "id1" has been reformatted and saved internally.
Remember to use WRITE SOURCE to save it permanently.
MM> show proof id1 /lemmon/all
 1 wph           $f wff ph
 2 wph           $f wff ph
 3 wph           $f wff ph
 4 2,3 wi        $a wff ( ph -> ph )
 5 1,4 wi        $a wff ( ph -> ( ph -> ph ) )
 6 wph           $f wff ph
 7 wph           $f wff ph
 8 6,7 wi        $a wff ( ph -> ph )
\end{verbatim}

\begin{center}
{etc.}
\end{center}

Beachten Sie, dass aus den ursprünglichen 6 Schritten nun 8 Schritte geworden sind.  Das Format ist jetzt jedoch dasselbe wie in Kapitel~\ref{using} beschrieben.

\chapter{Die Metamath-Sprache}
\label{languagespec}

\begin{quote}
  {\em So kann die Mathematik als das Fach definiert werden, in dem wir nie wissen, wovon wir sprechen, noch ob das, was wir sagen, wahr ist.}
      \flushright\sc  Bertrand Russell\footnote{Frei übersetzt nach \cite[S.~84]{Russell2}.}\\
\end{quote}\index{Russell, Bertrand}

Das wohl auffälligste Merkmal der Metamath-Sprache ist das fast voll\-stän\-dige Fehlen einer fest verdrahteten Syntax. Metamath\index{Metamath} versteht keine andere Mathematik oder Logik als die, die für die Konstruktion endlicher Symbolfolgen nach einer kleinen Menge einfacher, eingebauter Regeln erforderlich ist.  Die einzige Regel für Beweise ist die Ersetzung einer Variablen durch einen Ausdruck (Symbolfolge) mit einer einfachen Variableneinschränkung, um Kollisionen zwischen gebundenen Variablen zu verhindern.  Die primitiven Konzepte, die in Metamath eingebaut sind, beinhalten die einfache Manipulation von endlichen Objekten (Symbolen), die wir uns als Menschen leicht vergegenwärtigen können und mit denen Computer leicht umgehen können.  Sie scheinen so ziemlich die einfachsten Konzepte zu sein, die für die Standardmathematik erforderlich sind.

Dieses Kapitel dient als Nachschlagewerk für die Sprache Metamath\index{Metamath}. Es behandelt die langwierigen technischen Details der Sprache, von denen Sie einige beim ersten Lesen vielleicht überfliegen möchten.  Andererseits sollten Sie den definierten Begriffen in {\bf Fettschrift} große Aufmerksamkeit schenken; sie haben präzise Bedeutungen, die Sie sich für das spätere Verständnis merken sollten.  Am besten machen Sie sich zunächst mit den Beispielen in Kapitel~\ref{using} vertraut, um eine gewisse Motivation für die Sprache zu erhalten.

%% Uncomment this when uncommenting section {formalspec} below
Wenn Sie eine gewisse Kenntnis über die Mengenlehre haben, sollten Sie dieses Kapitel in Verbindung mit der formalen mengentheoretischen Beschreibung der Metamath-Sprache im Anhang~\ref{formalspec} studieren.

Wir werden den Namen "`Metamath"'\index{Metamath} verwenden, um entweder die Meta\-math-Computersprache oder die mit der Computersprache verbundene Meta\-math-Software zu bezeichnen.  Wir werden nicht zwischen diesen beiden unterscheiden, wenn der Kontext klar ist.

Der nächste Abschnitt enthält die vollständige Spezifikation der Metamath-Sprache. Sie dient als maßgebliche Referenz und stellt die Syntax detailliert genug dar, um einen Parser\index{Metamath parsen} und einen Beweisverifizierer zu schreiben.  Die Spezifikation ist knapp und es ist wahrscheinlich schwer, die Sprache direkt aus ihr heraus zu lernen. Aber wir nehmen sie hier für die ungeduldigen Leute auf, die lieber alles im Voraus sehen wollen, bevor sie sich mit ausführlichen Erklärungen beschäftigen wollen.  Spätere Abschnitte erläutern dieses Material und liefern Beispiele. Wir werden die Definitionen in diesen Abschnitten wiederholen, und Sie können den nächsten Abschnitt beim ersten Lesen überspringen und mit Abschnitt~\ref{tut1} (S.~\pageref{tut1}) fortfahren.

\section{Spezifikation der Metamath-Sprache}\label{spec}
\index{Metamath!Spezifikation}

\begin{quote}
  {\em Manchmal muss man schwierige Dinge sagen, aber man sollte sie so einfach wie möglich sagen.}
    \flushright\sc  G. H. Hardy\footnote{Frei übersetzt nach dem Zitat in \cite{deMillo}, S.~273.}\\
\end{quote}\index{Hardy, G. H.}

\subsection{Vorbereitungen}\label{spec1}

% Space is technically a printable character, so we'll word things
% carefully so it's unambiguous.
Eine Metamath-{\bf Datenbasis}\index{Datenbasis} wird aus einer übergeordneten Quelldatei zusammen mit allen Quelldateien aufgebaut, die durch Anweisungen zum Einbinden von Dateien (siehe unten) eingebunden werden.  Die einzigen Zeichen, die in einer Metamath-Quelldatei vorkommen dürfen, sind die 94 druckbaren {\sc ascii}\index{ascii@{\sc ascii}}-Zeichen ohne Whitespace-Zeichen, d.h. Ziffern, Groß- und Kleinbuchstaben, und die folgenden 32 Sonderzeichen\index{Sonderzeichen}:\label{spec1chars}

\begin{verbatim}
! '' # $ % & ' ( ) * + , - . / :
; < = > ? @ [ \ ] ^ _ ` { | } ~
\end{verbatim}
plus die folgenden Zeichen, die als "`Whitespace"'-Zeichen\label{whitespace} bezeichnet werden: Leerzeichen (ein druckbares Zeichen) und die nicht druckbaren Steuerzeichen Tabulator, Wagenrücklauf, Zeilenvorschub und Seitenvorschub.\footnote{Anm. der Übersetzer: Im Deutschen wird ein "`Whitespace"'-Zeichen manchmal auch als "`Leerraum"' oder "`Weißraum"'oder noch sperriger "`Zwischenraumzeichen"' übersetzt. Wir benutzen hier aber durchgängig die auch im Deutschen üblicherweise verwendete englische Bezeichnung "`Whitespace"'.} Wir verwenden die Schriftart \texttt{typewriter}, um die druckbaren Zeichen anzuzeigen.

Eine Metamath-Datenbasis besteht aus einer Folge von drei Arten von {\bf Token}\index{Token}, die durch {\bf Whitespace}\index{Whitespace} (eine beliebige Folge von einem oder mehreren "`Whitespace"'-Zeichen ) getrennt sind.  Die Menge der {\bf Schlüsselwort}\index{Schlüsselwort}-Token ist \texttt{\$\char`\{}, \texttt{\$\char`\}}, \texttt{\$c}, \texttt{\$v}, \texttt{\$f}, \texttt{\$e}, \texttt{\$d}, \texttt{\$a}, \texttt{\$p}, \texttt{\$.}, \texttt{\$=}, \texttt{\$(}, \texttt{\$)}, \texttt{\$[}, und \texttt{\$]}. Die letzten vier werden als {\bf Hilfs-}\index{Hilfsschlüsselwort} oder vorverarbeitende Schlüsselwörter bezeichnet.  Ein {\bf Label}\index{Label}-Token besteht aus einer beliebigen Kombination von Buchstaben, Ziffern und den Zeichen Bindestrich, Unterstrich und Punkt.  Ein Token für ein {\bf mathematisches Symbol}\index{mathematisches Symbol} kann aus einer beliebigen Kombination der 93 druckbaren Standard-{\sc ascii}-Zeichen außer Leerzeichen und \texttt{\$}~ bestehen. Bei allen Token wird zwischen Groß- und Kleinschreibung unterschieden.

\subsection{Vorverarbeitung}

Mit dem Token \texttt{\$(} beginnt ein {\bf Kommentar} und \texttt{\$)} beendet einen Kommentar.\index{\texttt{\$(} und \texttt{\$)} Hilfsschlüsselwörter}\index{Kommentar} Kommentare können jedes der 94 druckbaren Nicht-Whitespace-Zeichen und Whitespace enthalten, mit der Ausnahme, dass sie nicht die 2-Zeichen-Sequenzen \texttt{\$(} oder \texttt{\$)} enthalten dürfen (Kommentare lassen sich nicht verschachteln). Kommentare werden beim Parsen ignoriert (wie Whitespace behandelt), z. B. ist \texttt{\$( \$[ \$)} ein Kommentar. Siehe S.~\pageref{mathcomments} für die Konventionen zur Darstellung von Kommentaren; diese Konventionen können für die Zwecke des Parsens ignoriert werden.

Eine {\bf Anweisung zum Einbinden von Dateien} besteht aus \texttt{\$[}, gefolgt von einem Dateinamen, gefolgt von \texttt{\$]}.\index{\texttt{\$[} und \texttt{\$]} Hilfsschlüsselwörter}\index{eingebundene Datei}\index{Dateieinbindung} Es ist nur im äußersten Bereich zulässig (d. h. nicht zwischen \texttt{\$\char`\{} und \texttt{\$\char`\}}) und darf nicht innerhalb einer Aussage stehen (z. B. darf es nicht zwischen dem Label einer \texttt{\$a}-Anweisung und ihrem abschließenden \texttt{\$.} vorkommen). Der Dateiname darf weder ein \texttt{\$} noch ein Whitespace enthalten.  Die Datei muss existieren. Die Unterscheidung zwischen Groß- und Kleinschreibung des Namens folgt den Konventionen des Betriebssystems.  Der Inhalt der Datei ersetzt die Anweisung zum Einbinden von Dateien beim Einlesen der übergeordneten Datei. Eingebundene Dateien können andere Dateien einbinden. Nur der erste Verweis auf eine bestimmte Datei wird eingebunden; alle späteren Verweise auf dieselbe Datei (ob in der Datei der obersten Ebene oder in eingebundene Dateien) führen dazu, dass die Anweisung zum Einbinden von Dateien ignoriert wird (wie Whitespace behandelt wird). Ein Prüfprogramm kann davon ausgehen, dass Dateinamen bestehend aus unterschiedlichen Zeichenketten sich auf unterschiedliche Dateien beziehen, um spätere Verweise zu ignorieren. Ein Selbstverweis auf eine Datei wird ignoriert, ebenso wie jeder Verweis auf die übergeordnete Datei (um Schleifen zu vermeiden). Eingebundene Dateien dürfen kein \texttt{\$(} ohne ein passendes \texttt{\$)}, kein \texttt{\$[} ohne ein passendes \texttt{\$]} und keine unvollständigen Aussagen (z. B. ein \texttt{\$a} ohne ein passendes \texttt{\$.}) enthalten. Es ist derzeit nicht spezifiziert, ob Pfadreferenzen relativ zum aktuellen Verzeichnis des Prozesses oder zum Verzeichnis, das die Datei enthält, zu verstehen sind, so dass Datenbasen die Verwendung von Pfadnamen-Trennzeichen (z.B. "`/"') in Dateinamen vermeiden sollten.

Wie alle Token müssen die Schlüsselwörter \texttt{\$(}, \texttt{\$)}, \texttt{\$[} und \texttt{\$]}
von Whitespace umgeben sein.

\subsection{Grundlegende Syntax}

Nach der Vorverarbeitung besteht eine Datenbasis aus einer Folge von {\bf Anweisungen}. Dies sind die Gültigkeitsbereichsanweisungen \texttt{\$\char`\{} und \texttt{\$\char`\}} sowie die Anweisungen \texttt{\$c}, \texttt{\$v}, \texttt{\$f}, \texttt{\$e}, \texttt{\$d}, \texttt{\$a} und \texttt{\$p}.

Eine {\bf Gültigkeitsbereichsanweisung}\index{Gültigkeitsbereichsanweisung} besteht nur aus ihrem Schlüsselwort: \texttt{\$\char`\{} oder \texttt{\$\char`\}}. Mit \texttt{\$\char`\{} beginnt ein {\bf Block}\index{Block}, und ein zugehöriges \texttt{\$\char`\}} beendet den Block. Jedes \texttt{\$\char`\{} muss ein zugehöriges \texttt{\$char`\}} haben. Rekursiv definiert ist ein Block eine Folge von keinem, einem oder mehreren Token außer \texttt{\$\char`\{} und \texttt{\$\char`\}} und möglicherweise anderen enthaltenen Blöcken.  Es gibt einen {\bf äußersten Block}\index{Block!äußerster}, der nicht von \texttt{\$\char`\{} \texttt{\$\char`\}} eingeklammert ist; das Ende des äußersten Blocks ist das Ende der Datenbasis.


% LaTeX bug? can't do \bf\tt

Eine {\bf \$v}- oder {\bf \$c-Anweisung}\index{\texttt{\$v}-Anweisung}\index{\texttt{\$c}-Anweisung} besteht aus dem Schlüsselwort-Token \texttt{\$v} bzw. \texttt{\$c}, gefolgt von einem oder mehreren mathematischen Symbolen,
% The word "token" is used to distinguish "$." from the sentence-ending period.
gefolgt von dem \texttt{\$.}-Token. Diese Anweisungen {\bf erklären}\index{Erklärung} die mathematischen Symbole zu {\bf Variablen}\index{Variable!Metamath} bzw. {\bf Konstanten}\index{Konstanten}. Dasselbe mathematische Symbol darf nicht zweimal in ein und derselben \texttt{\$v}- oder \texttt{\$c}-Anweisung vorkommen.

%c%A math symbol becomes an {\bf active}\index{aktives mathematisches Symbol}
%c%when declared and stays active until the end of the block in which it is
%c%declared.  A math symbol may not be declared a second time while it is active,
%c%but it may be declared again after it becomes inactive.

Ein mathematisches Symbol wird {\bf aktiv}\index{aktives mathematisches Symbol}, sobald es deklariert wird, und bleibt es bis zum Ende des Blocks, in dem es deklariert wird.  Eine Variable kann nicht ein zweites Mal deklariert werden, solange sie aktiv ist, aber sie kann erneut deklariert werden (als Variable, aber nicht als Konstante), nachdem sie inaktiv geworden ist.  Eine Konstante muss im äußersten Block deklariert werden und darf nicht ein zweites Mal deklariert werden.\index{Umdeklarierung von Symbolen}

Eine {\bf \$f-Anweisung}\index{\texttt{\$f}-Anweisung} besteht aus einem Label, gefolgt von \texttt{\$f}, gefolgt von seinem Typcode (eine aktive Konstante), gefolgt von einer aktiven Variablen, gefolgt von dem \texttt{\$.}-Token.  Eine {\bf \$e-Anweisung}\index{\texttt{\$e}-Anweisung} besteht aus einem Label, gefolgt von \texttt{\$e}, gefolgt von seinem Typcode (einer aktiven Konstante), gefolgt von keinem, einem oder mehreren aktiven mathematischen Symbolen, gefolgt von dem \texttt{\$.}-Token.  Eine {\bf Hypothese}\index{Hypothese} ist eine \texttt{\$f}- oder \texttt{\$e}-Anweisung. Der durch eine \texttt{\$f}-Anweisung für ein bestimmtes Label deklarierte Typ ist global, auch wenn die Variable es nicht ist (z. B. kann eine Datenbasis nicht \texttt{wff P} in einem lokalen Bereich und \texttt{class P} in einem anderen haben).

Eine {\bf einfache \$d-Anweisung}\index{\texttt{\$d}-Anweisung!einfach}besteht aus \texttt{\$d}, gefolgt von zwei verschiedenen aktiven Variablen, gefolgt von dem Token \texttt{\$.}\.  Eine zusammengesetzte Anweisung besteht aus \texttt{\$d}, gefolgt von drei oder mehr Variablen (alle unterschiedlich), gefolgt von dem \texttt{\$.}\ Token.  Die Reihenfolge der Variablen in einer \texttt{\$d}-Anweisung ist unerheblich.  Eine zusammengesetzte \texttt{\$d}-Anweisung ist äquivalent zu einer Reihe von einfachen \texttt{\$d}-Anweisungen, eine für jedes mögliche Paar von Variablen, die in der zusammengesetzten \texttt{\$d}-Anweisung vorkommen.  In dieser Spezifikation nehmen wir an, dass alle \texttt{\$d}-Anweisungen einfach sind.  Eine \texttt{\$d}-Anweisung wird auch als {\bf disjunkte} {\bf Variableneinschränkung}\index{disjunkte Variableneinschränkung} bezeichnet.

Eine {\bf \$a-Anweisung}\index{\texttt{\$a}-Anweisung} besteht aus einem Label, gefolgt von \texttt{\$a}, gefolgt von seinem Typcode (einer aktiven Konstante), gefolgt von keinem, einem oder mehreren aktiven mathematischen Symbolen, gefolgt von dem \texttt{\$.}\-Token.  Eine {\bf \$p-Anweisung}\index{\texttt{\$p}-Anweisung} besteht aus einem Label, gefolgt von \texttt{\$p}, gefolgt von seinem Typcode (einer aktiven Konstanten), gefolgt von keinem, einem oder mehreren aktiven mathematischen Symbolen, gefolgt von \texttt{\$=}, gefolgt von einer Folge von Labels, gefolgt von dem \texttt{\$.}\-Token.  Eine {\bf Behauptung}\index{Behauptung} ist eine \texttt{\$a}- oder \texttt{\$p}-Anweisung.

Eine \texttt{\$f}-, \texttt{\$e}- oder \texttt{\$d}-Anweisung ist von der Stelle, an der sie auftritt, bis zum Ende des Blocks, in dem sie auftritt, {\bf aktiv}. Eine \texttt{\$a}- oder \texttt{\$p}-Anweisung ist {\bf aktiv} von der Stelle, an der sie auftritt, bis zum Ende der Datenbasis. Es darf nicht zwei aktive \texttt{\$f}-Anweisungen geben, die die gleiche Variable enthalten.  Jede Variable in einer \texttt{\$e}-, \texttt{\$a}- oder \texttt{\$p}-Anweisung muss in einer aktiven \texttt{\$f}-Anweisung vorkommen.\footnote{Diese Anforderung kann den Vereinheitlichungsalgorithmus (Berechnung der Substitutionen), der für die Beweisüberprüfung erforderlich ist, erheblich vereinfachen.}

%The label that begins each \texttt{\$f}, \texttt{\$e}, \texttt{\$a}, and
%\texttt{\$p} statement must be unique.
Jedes Label-Token muss eindeutig sein, und kein Label-Token darf mit einem Token eines mathematischen Symbols übereinstimmen.\label{namespace}\footnote{Diese Einschränkung wurde am 24. Juni 2006 hinzugefügt. Sie ist theoretisch nicht notwendig, wird aber eingeführt, um das Schreiben bestimmter Parser zu erleichtern.}

Die Menge der {\bf obligatorischen Variablen}\index{obligatorische Variable}, die mit einer Behauptung verbunden ist, ist die (möglicherweise leere) Menge der Variablen in der Behauptung und in allen aktiven \texttt{\$e}-Anweisungen.  Die (möglicherweise leere) Menge der {\bf obligatorischen Hypothesen}\index{obligatorische Hypothese} ist die Menge aller aktiven \texttt{\$f}-Anweisungen, die obligatorische Variablen enthalten, zusammen mit allen aktiven \texttt{\$e}-Anweisungen. Die Menge der {\bf obligatorischen {\bf \$d}-Anweisungen}\index{obligatorische disjunkte Variableneinschränkung}, die mit einer Behauptung verbunden ist, sind die aktiven \texttt{\$d}-Anweisungen, deren Variablen beide zu den obligatorischen Variablen der Behauptung gehören.

\subsection{Beweisverifizierung}\label{spec4}

Die Folge von Labeln zwischen den Token in einer \texttt{\$p}-Anweisung ist ein {\bf Beweis}\index{Beweis!Metamath}. Jedes Label in einem Beweis muss das Label einer anderen aktiven Aussage als die der \texttt{\$p}-Anweisung selbst sein; eine Kennzeichnung muss sich also entweder auf eine aktive Hypothese der \texttt{\$p}-Anweisung oder auf eine frühere Behauptung beziehen.

Ein {\bf Ausdruck}\index{Ausdruck} ist eine Folge von mathematischen Symbolen. Eine {\bf Substitutionsabbildung}\index{Substitutionsabbildung} assoziiert eine Menge von Variablen mit einer Menge von Ausdrücken.  Es ist zulässig, dass eine Variable auf einen Ausdruck abgebildet wird, der sie enthält.  Eine {\bf Substitution}\index{Substitution!Variable}\index{Variablensubstitution} ist die gleichzeitige Ersetzung aller Variablen in einem oder mehreren Ausdrücken durch die Ausdrücke, denen die Variablen zugeordnet sind.

Ein Beweis wird in der Reihenfolge seiner Labelfolge gescannt.  Wenn sich das Label auf eine aktive Hypothese bezieht, wird der Ausdruck in der Hypothese auf einen Stapel\index{Stapel}\index{RPN-Stapel} geschoben.  Bezieht sich das Label auf eine Behauptung, so muss eine (eindeutige) Substitution vorhanden sein, die, wenn sie an den obligatorischen Hypothesen der referenzierten Behauptung vorgenommen wird, bewirkt, dass diese mit den obersten (d. h. jüngsten) Einträgen des Stapels in der Reihenfolge des Auftretens der obligatorischen Hypothesen übereinstimmen, wobei der oberste Stapel-Eintrag mit der letzten obligatorischen Hypothese der referenzierten Behauptung übereinstimmt.  Anschließend werden so viele Stapel-Einträge wie obligatorischen Hypothesen vorhanden sind, vom Stapel entfernt.  Die gleiche Ersetzung wird an der referenzierten Behauptung vorgenommen, und das Ergebnis wird auf den Stapel geschoben. Nachdem das letzte Label im Beweis verarbeitet wurde, muss der Stapel einen einzigen Eintrag enthalten, der mit dem Ausdruck in der \texttt{\$p}-Anweisung übereinstimmt, die den Beweis enthält.

%c%{\footnotesize\begin{quotation}\index{Umdeklarierung von Symbolen}
%c%{{\em Comment.}\label{spec4comment} Whenever a math symbol token occurs in a
%c%{\texttt{\$c} or \texttt{\$v} statement, it is considered to designate a distinct new
%c%{symbol, even if the same token was previously declared (and is now inactive).
%c%{Thus a math token declared as a constant in two different blocks is considered
%c%{to designate two distinct constants (even though they have the same name).
%c%{The two constants will not match in a proof that references both blocks.
%c%{However, a proof referencing both blocks is acceptable as long as it doesn't
%c%{require that the constants match.  Similarly, a token declared to be a
%c%{constant for a referenced assertion will not match the same token declared to
%c%{be a variable for the \texttt{\$p} statement containing the proof.  In the case
%c%{of a token declared to be a variable for a referenced assertion, this is not
%c%{an issue since the variable can be substituted with whatever expression is
%c%{needed to achieve the required match.
%c%{\end{quotation}}
%c2%A proof may reference an assertion that contains or whose hypotheses contain a
%c2%constant that is not active for the \texttt{\$p} statement containing the proof.
%c2%However, the final result of the proof may not contain that constant. A proof
%c2%may also reference an assertion that contains or whose hypotheses contain a
%c2%variable that is not active for the \texttt{\$p} statement containing the proof.
%c2%That variable, of course, will be substituted with whatever expression is
%c2%needed to achieve the required match.

Ein Beweis kann ein \texttt{?}\ anstelle eines Labels enthalten, um einen unbekannten Schritt anzuzeigen (Abschnitt~\ref{unknown}).  Ein Beweisverifizierer kann jeden Beweis ignorieren, der \texttt{?}\ enthält, sollte aber den Benutzer warnen, dass der Beweis unvollständig ist.

Ein {\bf komprimierter Beweis}\index{komprimierter Beweis}\index{Beweis!komprimiert} ist eine alternative Beweis-Notation, die im Anhang\ref{compressed} beschrieben wird; siehe auch Verweise auf "`komprimierte Beweise"' im Index.  Komprimierte Beweise sind eine Erweiterung der Metamath-Sprache, die ein vollständiger Beweisverifizierer analysieren und verifizieren können sollte.

\subsubsection{Überprüfung von disjunkten Variableneinschränkungen}

Jede in einem Beweis vorgenommene Substitution muss wie folgt überprüft werden, um sicherzustellen, dass alle disjunkten Variableneinschränkungen erfüllt sind.

Wenn zwei durch eine Substitution ersetzte Variablen in einer obligatorischen \texttt{\$d}-Anweisung\index{\texttt{\$d}-Anweisung} der referenzierten Behauptung vorhanden sind, müssen die beiden aus der Substitution resultierenden Ausdrücke folgende Bedingungen erfüllen.  Erstens dürfen die beiden Ausdrücke keine Variablen gemeinsam haben. Zweitens muss jedes mögliche Paar von Variablen, eine von jedem Ausdruck, in einer aktiven \texttt{\$d}-Anweisung der \texttt{\$p}-Anweisung, die den Beweis enthält, vorhanden sein.

\vskip 1ex

Damit ist die Spezifikation der Metamath-Sprache abgeschlossen; siehe Anhang \ref{BNF} für ihre Syntax in erweiterter Backus--Naur-Form (EBNF)\index{erweiterte Backus--Naur-Form}\index{EBNF}.

\section{Die grundlegenden Schlüsselwörter}\label{tut1}

Im folgenden werden die im letzten Abschnitt vorgestellten Konstrukte der Metamath-Sprache genauer erläutert.

Wie die meisten Computersprachen bezieht Metamath\index{Metamath} seine Eingaben aus einer oder mehreren {\bf-Quelldateien}\index{Quelldatei}, die Zeichen im Standardcode {\sc ascii} (American Standard Code for Information Interchange)\index{ascii@{\sc ascii}} für Computer enthalten.  Eine Quelldatei besteht aus einer Reihe von {\bf Tokens}\index{Token}, d.h. Zeichenketten aus druckbaren Zeichen ohne Whitespace (aus dem auf S.~\pageref{spec1chars} gezeigten Satz von 94 Zeichen), die durch {\bf Whitespace}\index{Whitespace} (Leerzeichen, Tabulatoren, Wagenrücklauf, Zeilenvorschub und Seitenvorschub) getrennt sind. Jede Zeichenkette, die nur aus diesen Zeichen besteht, wird wie ein einzelnes Leerzeichen behandelt.  Die druckbaren Zeichen ohne Whitespace {\index{druckbares Zeichen}}, die Metamath erkennt, sind die 94 Zeichen auf Standard-{\sc ascii}-Tastaturen.

Metamath hat die Möglichkeit, mehrere Dateien zusammenzufügen, um die Eingabe zu bilden (Abschnitt~\ref{include}).  Wir nennen den Gesamtinhalt aller Dateien, nachdem sie zusammengefügt wurden, eine {\bf database}\index{Datenbasis}, um sie von einer einzelnen Quelldatei zu unterscheiden.  Die Tokens in einer Datenbasis bestehen aus {\bf Schlüsselwörter}\index{Schlüsselwort}, die in die Sprache eingebaut sind, sowie aus zwei Arten von benutzerdefinierten Tokens, den {\bf Labels}\index{Label} und die {\bf mathematischen Symbole}\index{mathematisches Symbol}.  (Der Kürze halber werden wir oft einfach {\bf symbol}\index{Symbol} anstelle von mathematischem Symbol sagen).  Die Menge der {\bf grundlegenden Schlüsselwörter}\index{grundlegendes Schlüsselwort} ist
\texttt{\$c}\index{\texttt{\$c}-Anweisung},
\texttt{\$v}\index{\texttt{\$v}-Anweisung},
\texttt{\$e}\index{\texttt{\$e}-Anweisung},
\texttt{\$f}\index{\texttt{\$f}-Anweisung},
\texttt{\$d}\index{\texttt{\$d}-Anweisung},
\texttt{\$a}\index{\texttt{\$a}-Anweisung},
\texttt{\$p}\index{\texttt{\$p}-Anweisung},
\texttt{\$=}\index{\texttt{\$=} Schlüsselwort},
\texttt{\$.}\index{\texttt{\$.} Schlüsselwort},
\texttt{\$\char`\{}\index{\texttt{\$\char`\{} und \texttt{\$\char`\}} Schlüsselwörter} und \texttt{\$\char`\}}.  Dies ist der vollständige Satz der syntaktischen Elemente der sogenannten {\bf Grundsprache}\index{grundlegende Sprache} von Metamath, und mit ihnen können Sie die gesamte Mathematik ausdrücken, so wie es bei der Entwicklung von Metamath beabsichtigt war.  Sie sollten sich unbedingt mit ihnen vertraut machen. Tabelle~\ref{basickeywords} listet die grundlegenden Schlüsselwörter zusammen mit einer kurzen Beschreibung ihrer Funktionen auf.  Für den Moment wird diese Beschreibung Ihnen nur eine vage Vorstellung von der Bedeutung der Schlüsselwörter geben; später werden wir die Schlüsselwörter im Detail beschreiben.


\begin{table}[htp] \caption{Zusammenfassung der grundlegenden Schlüsselwörter von Metamath} \label{basickeywords}
\begin{center}
\begin{tabular}{|p{5pc}|l|}
\hline
\em \centering Schlüsselwort&\em Beschreibung\\
\hline
\hline
\centering
   \texttt{\$c}&Konstantensymbol-Deklaration\\
\hline
\centering
   \texttt{\$v}&Variablensymbol-Deklaration\\
\hline
\centering
   \texttt{\$d}&disjunkte Variableneinschränkung\\
\hline
\centering
   \texttt{\$f}&("`Fließende"') Hypothese vom Variablentyp \\
\hline
\centering
   \texttt{\$e}&Logische ("`wesentliche"') Hypothese\\
\hline
\centering
   \texttt{\$a}&Axiomatische Behauptung\\
\hline
\centering
   \texttt{\$p}&Beweisbare Behauptung\\
\hline
\centering
   \texttt{\$=}&Beginn des Beweises in einer \texttt{\$p}-Anweisung \\
\hline
\centering
   \texttt{\$.}&Ende einer der obigen Anweisungen\\
\hline
\centering
   \texttt{\$\char`\{}&Beginn eines Blocks\\
\hline
\centering
   \texttt{\$\char`\}}&Ende eines Blocks\\
\hline
\end{tabular}
\end{center}
\end{table}

%For LaTeX bug(?) where it puts tables on blank page instead of btwn text
%May have to adjust if text changes
%\newpage

Es gibt einige zusätzliche Schlüsselwörter, {\bf Hilfsschlüsselwörter}\index{Hilfsschlüsselwort} genannt, die hilfreich für das Arbeiten mit Metamath\index{Metamath} sind. Diese sind Teil der {\bf erweiterten Sprache}\index{erweiterte Sprache}. Sie bieten Ihnen die Möglichkeit, Kommentare in eine Metamath-Quelldatei\index{Quelldatei} einzufügen und auf andere Quelldateien zu verweisen.  Wir werden diese in späteren Abschnitten vorstellen. Tabelle~\ref{otherkeywords} fasst sie zusammen, so dass Sie diese beim Durchsehen einiger Quelldateien zum Lernen der grundlegenden Schlüsselwörter erkennen können.

\begin{table}[htp] \caption{Hilfsschlüsselwörter in Methamath} \label{otherkeywords}
\begin{center}
\begin{tabular}{|p{5pc}|l|}
\hline
\em \centering Schlüsselwort&\em Beschreibung\\
\hline
\hline
\centering
   \texttt{\$(}&Beginn eines Kommentars\\
\hline
\centering
   \texttt{\$)}&Ende eines Kommentars\\
\hline
\centering
   \texttt{\$[}&Anfang des Namens einer eingeschlossenen Quelldatei\\
\hline
\centering
   \texttt{\$]}&Ende des Namens einer eingeschlossenen Quelldatei\\
\hline
\end{tabular}
\end{center}
\end{table}
\index{\texttt{\$(} und \texttt{\$)} Hilfsschlüsselwörter}
\index{\texttt{\$[} und \texttt{\$]} Hilfsschlüsselwörter}


Im Gegensatz zu den Schlüsselwörtern\index{Schlüsselwort} in einigen Computersprachen handelt es sich bei den Schlüsselwörtern nicht um englische Wörter, sondern um kurze zweistellige Zeichenfolgen.  Dadurch sind sie zwar anfangs etwas schwerer zu merken, aber durch ihre Kürze fügen sie sich in die beschriebene Mathematik ein und lenken nicht davon ab, so wie es etwa Satzzeichen tun.


\subsection{Benutzerdefinierte Tokens}\label{dollardollar}\index{Token}

Wie Sie vielleicht bemerkt haben, beginnen alle Schlüsselwörter mit dem Zeichen \texttt{\$}.  Dieses banale Währungssymbol wird normalerweise nicht in der höheren Mathematik verwendet (abgesehen von Förderanträgen), daher haben wir es ausgewählt, um die Metamath\index{Metamath}-Schlüsselwörter von gewöhnlichen mathematischen Symbolen zu unterscheiden. Das Zeichen \texttt{\$} wird daher als Sonderzeichen betrachtet und darf nicht als Zeichen in einem benutzer\-defi\-nier\-ten Token verwendet werden.  Bei allen Tokens und Schlüsselwörtern wird zwischen Groß- und Kleinschreibung unterschieden; so wird beispiels\-weise \texttt{n} als ein anderes Zeichen als \texttt{N} betrachtet.  Durch die Unterscheidung zwischen Groß- und Kleinschreibung wird der verfügbare {\sc ascii}-Zeichensatz so umfangreich wie möglich genutzt.

\subsubsection{Mathemathische Symbole}\index{Token}\index{mathematisches Symbol}

Mathematische Symbole sind Tokens, die zur Darstellung von Symbolen verwendet werden, die in gewöhnlichen mathematischen Formeln vorkommen.  Sie können aus einer beliebigen Kombination der 93 druckbaren {\sc ascii}-Zeichen (das Leerzeichen ausgeschlossen) außer \texttt{\$}~ bestehen. Einige Beispiele sind \texttt{x}, \texttt{+}, \texttt{(}, \texttt{|-}, \verb$!%@?&$, und \texttt{bounded}.  Aus Gründen der Lesbarkeit ist es am besten, wenn diese Symbole den tatsächlichen mathematischen Symbolen im Rahmen des {\sc ascii}-Zeichensatzes so ähnlich wie möglich sind, um die Lesbarkeit der resultierenden mathematischen Ausdrücke zu verbessern.

In der Metamath-Sprache drückt man gewöhnliche mathematische Formeln und Aussagen als Sequenzen von mathematischen Symbolen aus, wie z.B. \texttt{2 + 2 = 4} (fünf Symbole, alles Konstanten).\footnote{Um Mehrdeutigkeiten mit anderen Ausdrücken zu vermeiden, wird dies in der Mengenlehre-Datenbasis \texttt{set.mm} als \texttt{|- ( 2 + 2 ) = 4 } ausgedrückt, dessen \LaTeX-Äquivalent $\vdash (2+2)=4$ ist.  Das \,$\vdash$ bedeutet "`ist ein Satz"' und die Klammern ermöglichen eine explizite assoziative Gruppierung.}\index{Drehkreuz ({$\,\vdash$})} Es kann sich sogar um englische oder auch deutsche Sätze handeln, wie in \texttt{E ist geschlossen und begrenzt} (fünf Symbole) - hier wäre \texttt{E} eine Variable und die anderen vier Konstantensymbole.  Im Prinzip könnte eine Metamath-Datenbasis so konstruiert werden, dass sie mit fast jeder eindeutigen mathematischen Aussage in englischer Sprache arbeitet. In der Praxis wären die Definitionen, die erforderlich wären, um alle möglichen Syntax-Variationen zu berücksichtigen, jedoch umständlich und verwirrend und hätten möglicherweise subtile, zufällig eingebaute Tücken.  Wir empfehlen generell, mathematische Aussagen, wann immer möglich, mit kompakten mathematischen Standardsymbolen auszudrücken und ihre englischsprachigen Beschreibungen in Kommentare zu setzen.
Axiome\index{Axiom} und Definitionen\index{Definition} (\texttt{\$a}\index{\texttt{\$a}-Anweisung}-Anweisungen) sind die einzigen Stellen, an denen Metamath keinen Fehler erkennt, und dies hilft, die Anzahl der benötigten Definitionen zu reduzieren.

Es steht Ihnen frei, beliebige Tokens\index{Token} für mathematische Symbole\index{mathematisches Symbol} zu verwenden.  Im Anhang~\ref{ASCII} werden Token-Namen für Symbole in der Mengenlehre empfohlen, und wir schlagen vor, dass Sie diese übernehmen, um die Mengenlehre-Datenbasis \texttt{set.mm}  in Ihre Datenbasis aufnehmen zu können.  Für Ausdrucke können Sie die Tokens in einer Datenbasis mit dem Satzprogramm \LaTeX\ in mathematische Standardsymbole umwandeln.  Der Metamath-Befehl \texttt{open tex} {\em Dateinamen}\index{\texttt{open tex}-Befehl} erzeugt eine Ausgabe, die von \LaTeX.\index{latex@{\LaTeX}} gelesen werden kann. Die Korrespondenz zwischen den Token und den tatsächlichen Symbolen wird durch \texttt{latexdef}-Anweisungen innerhalb eines speziellen Datenbasiskommentars hergestellt, der mit \texttt{\$t} gekennzeichnet ist.\index{\texttt{\$t}-Anweisung}\index{Schriftsatzanweisung}   Sie können diesen Kommentar bearbeiten, um die Definitionen zu ändern oder neue hinzuzufügen. Im Anhang~\ref{ASCII} wird genauer beschrieben, wie das funktioniert.

% White space\index{Whitespace} is normally used to separate math
% symbol\index{mathematisches Symbol} tokens, but they may be juxtaposed without white
% space in \texttt{\$d}\index{\texttt{\$d}-Anweisung}, \texttt{\$e}\index{\texttt{\$e}
% statement}, \texttt{\$f}\index{\texttt{\$f}-Anweisung}, \texttt{\$a}\index{\texttt{\$a}
% statement}, and \texttt{\$p}\index{\texttt{\$p}-Anweisung} statements when no
% ambiguity will result.  Specifically, Metamath parses the math symbol sequence
% in one of these statements in the following manner:  when the math symbol
% sequence has been broken up into tokens\index{Token} up to a given character,
% the next token is the longest string of characters that could constitute a
% math symbol that is active\index{active
% math symbol} at that point.  (See Section~\ref{scoping} for the
% definition of an active math symbol.)  For example, if \texttt{-}, \texttt{>}, and
% \texttt{->} are the only active math symbols, the juxtaposition \texttt{>-} will be
% interpreted as the two symbols \texttt{>} and \texttt{-}, whereas \texttt{->} will
% always be interpreted as that single symbol.\footnote{For better readability we
% recommend a white space between each token.  This also makes searching for a
% symbol easier to do with an editor.  Omission of optional white space is useful
% for reducing typing when assigning an expression to a temporary
% variable\index{temporäre Variable} with the \texttt{let variable} Metamath
% program command.}\index{\texttt{let variable}-Befehl}
%
% Schlüsselwörter\index{Schlüsselwort} may be placed next to math symbols without white
% space\index{Whitespace} between them.\footnote{Again, we do not recommend
% this for readability.}
%
% The math symbols\index{mathematisches Symbol} in \texttt{\$c}\index{\texttt{\$c}-Anweisung}
% and \texttt{\$v}\index{\texttt{\$v}-Anweisung} statements must always be separated
% by white space\index{white
% space}, for the obvious reason that these statements define the names
% of the symbols.
%
% Math symbols referred to in comments (see Section~\ref{comments}) must also be
% separated by white space.  This allows you to make comments about symbols that
% are not yet active\index{active
% math symbol}.  (The "`math mode"' feature of comments is also a quick and
% easy way to obtain word processing text with embedded mathematical symbols,
% independently of the main purpose of Metamath; the way to do this is described
% in Section~\ref{comments})

\subsubsection{Labels}\index{Token}\index{Label}

Label-Token werden verwendet, um Metamath\index{Metamath}-Anweisungen für eine spätere Referenz zu identifizieren\footnote{Anm. der Übersetzer: deshalb werden "`Labels"' im Deutschen oft auch als "`Sprungmarken"' bezeichnet.}. Label-Token dürfen nur Buchstaben, Ziffern und die drei Zeichen Punkt, Bindestrich und Unterstrich enthalten:
\begin{verbatim}
. - _
\end{verbatim}

Ein Label wird deklariert\index{Label-Deklaration}, indem man es unmittelbar vor das Schlüsselwort der Anweisung setzt, die es kennzeichnet.  Zum Beispiel könnte das Label \texttt{axiom.1} wie folgt deklariert werden:
\begin{verbatim}
axiom.1 $a |- x = x $.
\end{verbatim}

Jede \texttt{\$e}\index{\texttt{\$e}-Anweisung},
\texttt{\$f}\index{\texttt{\$f}-Anweisung},
\texttt{\$a}\index{\texttt{\$a}-Anweisung}, und
\texttt{\$p}\index{\texttt{\$p}-Anweisung}-Anweisung in einer Datenbasis muss ein Label für sie deklariert haben.  Keine anderen Anweisungstypen dürfen Label-Deklarationen haben.  Jedes Label muss eindeutig sein.

Auf ein Label (und die damit bezeichnete Anweisung) wird {\bf referenziert}\index{Label-Referenz}, indem das Label zwischen den \texttt{\$=}\index{\texttt{\$=} Schlüsselwort} und \texttt{\$.}\index{\texttt{\$.} Schlüsselwort}\ Schlüsselwörtern in eine \texttt{\$p}-Anweisung aufgenommen wird.  Die Folge von Labels\index{Label-Sequenz} zwischen diesen beiden Schlüsselwörtern wird als {\bf Beweis}\index{Beweis} bezeichnet.  Ein Beispiel für eine Anweisung mit einem Beweis, dem wir später begegnen werden (Abschnitt~\ref{proof}), ist
\begin{verbatim}
wnew $p wff ( s -> ( r -> p ) )
     $= ws wr wp w2 w2 $.
\end{verbatim}

Sie müssen noch nicht wissen, was das bedeutet, aber Sie sollten wissen, dass das Label \texttt{wnew} durch diese \texttt{\$p}-Anweisung deklariert wird und dass davon ausgegangen wird, dass die Labels \texttt{ws}, \texttt{wr}, \texttt{wp} und \texttt{w2} früher in der Datenbasis deklariert wurden und hier referenziert werden.

\subsection{Konstanten und Variablen}
\index{Konstante}
\index{Variable}

Ein {\bf Ausdruck}\index{Ausdruck} ist eine beliebige Folge von mathematischen Symbolen, die auch leer sein kann.

Die grundlegende Metamath\index{Metamath}-Sprache\index{grundlegende Sprache} hat zwei Arten von mathematischen Symbolen\index{mathematisches Symbol}:  {\bf Konstanten}\index{Konstante} und {\bf Variablen}\index{Variable}.  In einem Metamath-Beweis kann eine Konstante nicht durch einen beliebigen Ausdruck ersetzt werden.  Eine Variable kann durch einen beliebigen Ausdruck ersetzt\index{Substitution!Variable}\index{Variablensubstitution} werden.  Diese Sequenz kann andere Variablen und sogar die zu ersetzende Variable einschließen.  Diese Substitution findet statt, wenn Beweise verifiziert werden, und wird in Abschnitt~\ref{proof} beschrieben.  Die Anweisung \texttt{\$f} (später in Abschnitt~\ref{dollaref} beschrieben) wird verwendet, um den {\bf Typ} einer Variablen anzugeben (d.h. um welche Art von Variable es sich handelt)\index{Variablentyp}\index{Typ} zu spezifizieren und ihr eine Bedeutung zu geben, die typischerweise mit einer "`Metavariablen"' assoziiert wird. \index{Metavariable}\footnote{Eine Metavariable ist eine Variable, die sich über die syntaktischen Elemente der diskutierten Objektsprache erstreckt; zum Beispiel könnte eine Metavariable eine Variable der Objektsprache und eine andere Metavariable eine Formel in der Objektsprache repräsentieren. } in der gewöhnlichen Mathematik; eine Variable kann z. B. als wff oder wohlgeformte Formel (in der Logik), als Menge (in der Mengenlehre) oder als nichtnegative ganze Zahl (in der Zahlentheorie) angegeben werden.

\subsection{Die \texttt{\$c}- und \texttt{\$v}-Deklarationsanweisungen}
\index{\texttt{\$c}-Anweisung}
\index{Konstantendeklaration}
\index{\texttt{\$v}-Anweisung}
\index{Variablendeklaration}

Konstanten werden mit \texttt{\$c}\index{\texttt{\$c}-Anweisung}-Anweisungen eingeführt oder {\bf deklariert}\index{Konstantendeklaration}, und Variablen werden mit \texttt{\$v}\index{\texttt{\$v}-Anweisung}-Anweisungen deklariert.  Eine {\bf einfache} Deklarationsanweisung\index{einfache Deklaration} führt eine einzelne Konstante oder Variable ein.  Ihre Syntax ist eine der folgenden:
\begin{center}
  \texttt{\$c} {\em math-symbol} \texttt{\$.}\\
  \texttt{\$v} {\em math-symbol} \texttt{\$.}
\end{center}
Die Notation {\em math-symbol} bedeutet ein beliebiges  Token\index{Token} für ein mathe\-ma\-tisches Symbol.

Einige Beispiele für einfache Anweisungen sind:
\begin{center}
  \texttt{\$c + \$.}\\
  \texttt{\$c -> \$.}\\
  \texttt{\$c ( \$.}\\
  \texttt{\$v x \$.}\\
  \texttt{\$v y2 \$.}
\end{center}

Die Zeichen in einem mathematischen Symbol\index{mathematisches Symbol}, das deklariert wird, sind für Meta\-math irrelevant; zum Beispiel könnten wir eine rechte Klammer als Variable deklarieren,
\begin{center}
  \texttt{\$v ) \$.}\\
\end{center}
obwohl dies unkonventionell wäre.

Die Anweisung {\bf zusammengesetzte} Deklaration\index{zusammengesetzte Deklaration} ist eine Kurzform für die gleichzeitige Deklaration mehrerer Symbole.  Ihre Syntax ist eine der folgenden:
\begin{center}
  \texttt{\$c} {\em math-symbol}\ \,$\cdots$\ {\em math-symbol} \texttt{\$.}\\
  \texttt{\$v} {\em math-symbol}\ \,$\cdots$\ {\em math-symbol} \texttt{\$.}
\end{center}\index{\texttt{\$c}-Anweisung}
Hier bedeutet die Ellipse (\ldots) eine beliebige Anzahl von {\em math-symbol}\,en.

Ein Beispiel für eine zusammengesetzte Deklaration ist:
\begin{center}
  \texttt{\$v x y mu \$.}\\
\end{center}
Dies entspricht den drei einfachen Deklarationsanweisung
\begin{center}
  \texttt{\$v x \$.}\\
  \texttt{\$v y \$.}\\
  \texttt{\$v mu \$.}\\
\end{center}
\index{\texttt{\$v}-Anweisung}

Es gibt bestimmte Regeln dafür, wo in der Datenbasis mathematische Symbole deklariert werden dürfen, welche Abschnitte der Datenbasis von ihnen Kenntnis haben (d.h. wo sie "`aktiv"' sind) und wann sie mehr als einmal deklariert werden dürfen.  Diese werden in Abschnitt~\ref{scoping} und speziell auf S.~\pageref{redeclaration} diskutiert.

\subsection{Die \texttt{\$d}-Anweisung}\label{dollard}
\index{\texttt{\$d}-Anweisung}

Die Anweisung \texttt{\$d} wird als eine {\bf disjunkte Variableneinschränkung} be\-zeich\-net.  Die Syntax der {\bf einfachen} Version dieser Anweisung lautet
\begin{center}
  \texttt{\$d} {\em variable variable} \texttt{\$.},
\end{center}
wobei jede {\em variable} eine zuvor deklarierte Variable ist und die beiden {\em variable}\,n unterschiedlich sind.  (Genauer gesagt muss jede {\em variable} eine {\bf aktive} Variable sein, was bedeutet, dass es eine vorherige \texttt{\$v}-Anweisung geben muss, deren {\bf Gültigkeitsbereich}\index{Gültigkeitsbereich} die \texttt{\$d}-Anweisung einschließt.  Diese Begriffe werden definiert, wenn wir in Abschnitt~\ref{scoping} über Anweisungen mit Gültigkeitsbereich sprechen).

In der gewöhnlichen Mathematik können Formeln auftreten, die wahr sind, wenn die Variablen in ihnen unterschiedlich\index{unterschiedliche Variablen} sind, aber falsch werden, wenn diese Variablen identisch gemacht werden. Zum Beispiel ist die prädikatenlogische Formel $\exists x\,x \neq y$, was bedeutet: "`Für ein gegebenes $y$ existiert ein $x$, das nicht gleich $y$ ist"', in den meisten mathematischen Theorien wahr (nämlich in allen nicht-trivialen Theorien\index{nicht-triviale Theorie}, d.h. solchen mit mehr als einem Individuum, wie z.B. die Arithmetik).  Wenn wir jedoch $y$ durch $x$ ersetzen, erhalten wir $\exists x\,x \neq x$, was immer falsch ist, da es bedeutet, dass "`etwas existiert, das nicht gleich sich selbst ist."'\footnote{Wenn Sie Logiker sind, werden Sie dies als die unzulässige Substitution\index{echte Substitution}\index{Substitution!echte} einer freien Variablen\index{freie Variable} mit einer gebundenen Variablen\index{gebundene Variable} erkennen.  Metamath macht keinen inhärenten Unterschied zwischen freien und gebundenen Variablen; stattdessen lässt man Metamath wissen, welche Substitutionen zulässig sind, indem man \texttt{\$d}-Anweisungen in der richtigen Weise in seinem Axiomensystem verwendet.}\index{freie vs.\ gebundene Variable}  Mit der Anweisung \texttt{\$d} können Sie eine Beschränkung angeben, die die Ersetzung einer Variablen durch eine andere verbietet.  In diesem Fall würden wir die Anweisung
\begin{center}
  \texttt{\$d x y \$.}
\end{center}\index{\texttt{\$d}-Anweisung}
verwenden, um diese Beschränkung zu spezifizieren.

Die Reihenfolge, in der die Variablen in einer \texttt{\$d}-Anweisung erscheinen, ist nicht wichtig.  Wir könnten auch
\begin{center}
  \texttt{\$d y x \$.}
\end{center}
verwenden.

Die Anweisung \texttt{\$d} ist eigentlich noch allgemeiner, wie das "`disjunkt"'\index{disjunkte Variablen} in ihrem Namen vermuten lässt.  Die volle Bedeutung besteht darin, dass bei einer Substitution der beiden Variablen (im Verlauf eines Beweises, der sich auf eine \texttt{\$a}- oder \texttt{\$p}-Anweisung bezieht, die mit der \texttt{\$d}-Anweisung verknüpft ist), die beiden Ausdrücke, die sich aus der Substitution ergeben, keine gemeinsamen Variablen haben dürfen.  Außerdem muss jedes mögliche Paar von Variablen, eine von jedem Ausdruck, in einer \texttt{\$d}-Anweisung enthalten sein, die mit der zu beweisenden Anweisung verbunden ist.  (Diese Anforderung zwingt die zu beweisende Anweisung dazu, die ursprüngliche disjunkte Variableneinschränkung zu "`vererben"').

Nehmen wir zum Beispiel an, dass \texttt{u} eine Variable ist.  Wenn die Beschränkung
\begin{center}
  \texttt{\$d A B \$.}
\end{center}
für ein Theorem angegeben wurde, auf das in einem
Beweis verwendet wird, dürfen wir nicht \texttt{A} durch \mbox{\tt a + u} und
\texttt{B} durch \mbox{\tt b + u} ersetzen, da diese beiden Symbolfolgen die
Variable \texttt{u} gemeinsam haben.  Außerdem, wenn \texttt{a} und \texttt{b} Variablen sind, können wir nicht \texttt{A} durch \texttt{a} und \texttt{B} durch \texttt{b} ersetzen, es sei denn, wir haben auch \texttt{\$d a b} für das zu beweisende Theorem angegeben. Mit anderen Worten, die Eigenschaft von \texttt{\$d}, die mit einem Paar von Variablen verbunden ist, muss nach der Substitution wirksam erhalten bleiben.

Die \texttt{\$d}\index{\texttt{\$d}-Anweisung}-Anweisung bedeutet {\em nicht}, dass die beiden Variablen nicht durch dasselbe ersetzt werden dürfen, wie man zunächst denken könnte.  Wenn man beispielsweise \texttt{A} und \texttt{B} im obigen Beispiel durch identische Symbolsequenzen ersetzt, die nur aus Konstanten bestehen, entsteht kein disjunkter Variablenkonflikt, da die beiden Symbolsequenzen keine Variablen gemeinsam haben (da sie überhaupt keine Variablen haben).  Ebenso tritt kein Konflikt auf, wenn die beiden Variablen in einer Anweisung \texttt{\$d} durch die leere Symbolfolge\index{leere Substitution} ersetzt werden.

Die \texttt{\$d}-Anweisung hat keine direkte Entsprechung in der gewöhnlichen Mathematik: zum Teil deshalb, weil die Variablen\index{Variable} von Metamath nicht wirklich dasselbe sind wie die Variablen\index{Variable!in der gewöhnlichen Mathematik} der gewöhnlichen Mathematik, sondern vielmehr Metavariablen\index{Metavariable}, die sich über gewöhnliche Variablen erstrecken (sowie über andere Arten von Symbolen und Gruppen von Symbolen).  Je nach Situation können wir die Anweisung \texttt{\$d} informell auf verschiedene Weise interpretieren.  Nehmen wir zum Beispiel an, dass \texttt{x} und \texttt{y} Variablen sind, die sich über Zahlen erstrecken (genauer gesagt, dass \texttt{x} und \texttt{y} Metavariablen sind, die sich über Variablen erstrecken, die beliebige Zahlen repräsentieren), und dass \texttt{ph} ($\varphi$) und \texttt{ps} ($\psi$) Variablen (genauer gesagt, Metavariablen) sind, die sich über Formeln erstrecken.  Wir können die folgenden Interpretationen durchführen, die der informellen Sprache der gewöhnlichen Mathematik entsprechen:
\begin{quote}
\begin{tabbing}
\texttt{\$d x y \$.} bedeutet "`\=unter der Voraussetzung, dass $x$ und $y$\\ \>unterschiedliche Variablen sind."'\\
\texttt{\$d x ph \$.} bedeutet "`\=unter der Voraussetzung, dass $x$ nicht in\\ \>$\varphi$ vorkommt."'\\
\texttt{\$d ph ps \$.} bedeutet "`\=unter der Voraussetzung, dass $\varphi$ und $\psi$ \\ \>keine gemeinsamen Variablen haben."'
\end{tabbing}
\end{quote}\index{\texttt{\$d}-Anweisung}

\subsubsection{Zusammengesetzte \texttt{\$d} Anweisungen}

Die {\bf zusammengesetzte} Version der \texttt{\$d}-Anweisung ist eine Kurzform für die Angabe mehrerer Variablen, deren Ersetzungen paarweise disjunkt sein müssen. Ihre Syntax lautet:
\begin{center}
  \texttt{\$d} {\em variable}\ \,$\cdots$\ {\em variable} \texttt{\$.}
\end{center}\index{\texttt{\$d}-Anweisung}
Hier steht {\em variable} für das Token einer zuvor deklarierten Variable (genauer gesagt, einer aktiven Variable) und alle {\em variable}\,n sind unterschiedlich.  Die zusammengesetzte \texttt{\$d}-Anweisung wird von Metamath intern in eine einfache \texttt{\$d}-Anweisung für jedes mögliche Paar von Variablen in der ursprünglichen \texttt{\$d}-Anweisung zerlegt.  Zum Beispiel ist
\begin{center}
  \texttt{\$d w x y z \$.}
\end{center}
gleichbedeutend mit
\begin{center}
  \texttt{\$d w x \$.}\\
  \texttt{\$d w y \$.}\\
  \texttt{\$d w z \$.}\\
  \texttt{\$d x y \$.}\\
  \texttt{\$d x z \$.}\\
  \texttt{\$d y z \$.}
\end{center}

Zwei oder mehr einfache \texttt{\$d}-Anweisungen, die das gleiche Variablenpaar angeben, werden intern zu einer einzigen \texttt{\$d}-Anweisung zusammengefasst.  So ist die Menge der drei Anweisungen
\begin{center}
  \texttt{\$d x y \$.}
  \texttt{\$d x y \$.}
  \texttt{\$d y x \$.}
\end{center}
gleichbedeutend mit
\begin{center}
  \texttt{\$d x y \$.}
\end{center}

In ähnlicher Weise werden bei zusammengesetzten \texttt{\$d}-Anweisungen nach der internen Zerlegung die gemeinsamen Variablenpaare kombiniert.  Zum Beispiel ist die Menge der Anweisungen
\begin{center}
  \texttt{\$d x y A \$.}
  \texttt{\$d x y B \$.}
\end{center}
gleichbedeutend mit
\begin{center}
  \texttt{\$d x y \$.}
  \texttt{\$d x A \$.}
  \texttt{\$d y A \$.}
  \texttt{\$d x y \$.}
  \texttt{\$d x B \$.}
  \texttt{\$d y B \$.}
\end{center}
welches gleichbedeutend ist mit
\begin{center}
  \texttt{\$d x y \$.}
  \texttt{\$d x A \$.}
  \texttt{\$d y A \$.}
  \texttt{\$d x B \$.}
  \texttt{\$d y B \$.}
\end{center}

Metamath\index{Metamath} verifiziert beim Verifizieen eines Beweises automatisch, dass alle \texttt{\$d}-Beschränkungen erfüllt sind.  \texttt{\$d}-Anweisungen werden in Beweisen nie direkt referenziert (deshalb haben sie keine Labels\index{Label}), aber Metamath weiß immer, welche erfüllt sein müssen (d.h. \ aktiv sind) und meldet einen Fehler, wenn eine Verletzung auftritt.

Zur Veranschaulichung, wie Metamath eine fehlende \texttt{\$d}-Anweisung erkennt, betrachten wir das folgende Beispiel aus der Datenbasis \texttt{set.mm}.

\begin{verbatim}
$d x z $.  $d y z $.
$( Theorem to add distinct quantifier to atomic formula. $)
ax5eq $p |- ( x = y -> A. z x = y ) $=...
\end{verbatim}

Diese Aussage benötigt offensichtlich die Bedingung, dass $z$ von $x$ verschieden sein muss in dem Theorem \texttt{ax5eq}, das  $x=y \rightarrow \forall z \, x=y$ besagt (das ist offensichtlich, wenn man Logiker ist, denn sonst könnte man $x=y \rightarrow \forall x \, x=y$ folgern, was falsch ist, wenn die freien Variablen $x$ und $y$ gleich sind).

Schauen wir uns an, was passiert, wenn wir in der Datenbasis diese \texttt{\$d}-Anweisung auszukommentieren.

\begin{verbatim}
$( $d x z $. $) $d y z $.
$( Theorem to add distinct quantifier to atomic formula. $)
ax5eq $p |- ( x = y -> A. z x = y ) $=...
\end{verbatim}

Wenn Metamath versucht, den Beweis zu verifizieren, wird es Ihnen sagen, dass \texttt{x} und \texttt{z} disjunkt sein müssen, weil einer seiner Schritte auf ein Axiom oder Theorem verweist, das diese Bedingung erfüllt.

\begin{verbatim}
MM> verify proof ax5eq
ax5eq ?Error at statement 1918, label "ax5eq", type "$p":
      vz wal wi vx vy vz ax-13 vx vy weq vz vx ax-c16 vx vy
                                               ^^^^^
There is a disjoint variable ($d) violation at proof step 29.
Assertion "ax-c16" requires that variables "x" and "y" be
disjoint.  But "x" was substituted with "z" and "y" was
substituted with "x".  The assertion being proved, "ax5eq",
does not require that variables "z" and "x" be disjoint.
\end{verbatim}

Wir können die Ersetzungen in \texttt{ax-c16} mit dem folgenden Befehl sehen.

\begin{verbatim}
MM> show proof ax5eq / detailed_step 29
Proof step 29:  pm2.61dd.2=ax-c16 $a |- ( A. z z = x -> ( x =
  y -> A. z x = y ) )
This step assigns source "ax-c16" ($a) to target "pm2.61dd.2"
($e).  The source assertion requires the hypotheses "wph"
($f, step 26), "vx" ($f, step 27), and "vy" ($f, step 28).
The parent assertion of the target hypothesis is "pm2.61dd"
($p, step 36).
The source assertion before substitution was:
    ax-c16 $a |- ( A. x x = y -> ( ph -> A. x ph ) )
The following substitutions were made to the source
assertion:
    Variable  Substituted with
     x         z
     y         x
     ph        x = y
The target hypothesis before substitution was:
    pm2.61dd.2 $e |- ( ph -> ch )
The following substitutions were made to the target
hypothesis:
    Variable  Substituted with
     ph        A. z z = x
     ch        ( x = y -> A. z x = y )
\end{verbatim}

Die disjunkten Variableneinschränkungen von \texttt{ax-c16} können mit dem Befehl \texttt{show state\-ment} ermittelt werden.  Die Zeile, die mit "`\texttt{Its mandatory
dis\-joint var\-i\-able pairs are:}\ldots"' beginnt, listet alle Variablenpaare in spitzen Klammern auf.

\begin{verbatim}
MM> show statement ax-c16/full
Statement 3033 is located on line 9338 of the file "set.mm".
"Axiom of Distinct Variables. ..."
  ax-c16 $a |- ( A. x x = y -> ( ph -> A. x ph ) ) $.
Its mandatory hypotheses in RPN order are:
  wph $f wff ph $.
  vx $f setvar x $.
  vy $f setvar y $.
Its mandatory disjoint variable pairs are:  <x,y>
The statement and its hypotheses require the variables:  x y
      ph
The variables it contains are:  x y ph
\end{verbatim}

Da Metamath immer erkennt, wenn \texttt{\$d}\index{\texttt{\$d}-Anweisung}-Anweisungen für einen Beweis benötigt werden, brauchen Sie sich keine Sorgen zu machen, dass Sie vergessen haben, eine entsprechende Anweisung einzufügen; sie kann immer hinzugefügt werden, wenn Sie die obige Fehlermeldung sehen.  Wenn Sie unnötige \texttt{\$d}-Anweisungen einfügen, kann es schlimmstenfalls passieren, dass Ihr Theorem nicht so allgemein ist wie es sein könnte, und dies kann seine spätere Verwendung einschränken.

Andererseits muss man bei der Einführung von Axiomen (\texttt{\$a}\index{\texttt{\$a}-Anweisung}-Anweisun\-gen) sehr vorsichtig sein, um die notwendigen zugehörigen \texttt{\$d}-Anweisungen richtig anzugeben, da Metamath keine Möglichkeit hat zu prüfen, ob die Axiome korrekt sind.  Metamath hätte zum Beispiel keine Kenntnis darüber, dass \texttt{ax-c16}, das wir als ein Axiom der Logik betrachten, zu Widersprüchen führen würde, wenn wir seine zugehörige \texttt{\$d}-Anweisung weglassen.

% This was previously a comment in footnote-sized type, but it can be
% hard to read this much text in a small size.
% As a result, it's been changed to normally-sized text.
\label{nodd}
Sie fragen sich vielleicht, ob es möglich ist, Standardmathematik in der Metamath-Sprache ohne die Anweisung \texttt{\$d}\index{\texttt{\$d}-Anweisung} zu entwickeln, da sie wie ein Ärgernis erscheint, das die Überprüfung von Beweisen erschwert. Die Anweisung \texttt{\$d} wird in bestimmten Teilbereichen der Mathematik wie der Aussagenlogik nicht benötigt.  Dummy-Variablen\index{Dummy-Variable!eliminieren} und die damit verbundenen \texttt{\$d}-Anweisungen lassen sich jedoch in Beweisen in der Standardlogik erster Ordnung sowie in der in \texttt{set.mm} verwendeten Variante nicht vermeiden.  Tatsächlich gibt es keine Obergrenze für die Anzahl der Dummy-Variablen, die in einem Beweis eines Satzes der Logik erster Ordnung mit 3 oder mehr Variablen benötigt werden, wie H.\ Andr{e}ka\index{Andr{\'{e}}ka, H.} \cite{Nemeti} gezeigt hat.  Ein System erster Ordnung, das sie vollständig vermeidet, findet sich in \cite{Megill}\index{Megill, Norman}; der Trick dort besteht einfach darin, die notwendigen Dummy-Variablen bedenkenlos in ein zu beweisendes Theorem einzubetten, so dass sie nicht mehr "`dummy"' sind, und dann das resultierende längere Theorem so zu interpretieren, dass die eingebetteten Dummy-Variablen ignoriert werden.  Falls Sie das interessiert, das System in \texttt{set.mm}, das sich aus \texttt{ax-1} bis \texttt{ax-c14} in \texttt{set.mm} ergibt, und das Entfernen von \texttt{ax-c16} und \texttt{ax-5} erfordert keine \texttt{\$d}-Anweisungen, ist aber logisch vollständig in dem in \cite{Megill} beschriebenen Sinne.  Das bedeutet, dass damit jeder Satz der Logik erster Ordnung bewiesen werden kann, solange wir dem Satz eine Prämisse hinzufügen, das Dummy- und alle anderen Variablen, die unterschiedlich sein müssen, einschließt.  In ähnlicher Weise können Axiome für die Mengenlehre entwickelt werden, die keine disjunkten Variableneinschränkungen erfordern\index{Mengenlehre ohne disjunkte Variableneinschränkungen}, wie unter \url{http://us.metamath.org/mpeuni/mmzfcnd.html} erläutert. Zusammen ermöglichen diese im Prinzip die Entwicklung der gesamten Mathematik unter Metamath ohne eine \texttt{\$d}-Anweisung, während die Länge der resultierenden Theoreme wachsen würde, je mehr Dummy-Variablen in ihren Beweisen benötigt werden.

\subsection{Die \texttt{\$f}- und \texttt{\$e}-Anweisungen}\label{dollaref}
\index{\texttt{\$e}-Anweisung}
\index{\texttt{\$f}-Anweisung}
\index{fließende Hypothese}
\index{essentielle Hypothese}
\index{Variablentyp-Hypothese}
\index{logische Hypothese}
\index{Hypothese}

Metamath kennt zwei Arten von Hypothesen, die \texttt{\$f}\index{\texttt{\$f}-Anweisung}- oder {\bf Variablentyp}-Hypothese und die \texttt{\$e}- oder {\bf logische} Hypothese. \index{\texttt{\$d}-Anweisung}\footnote{Streng genommen handelt es sich bei der \texttt{\$d}-Anweisung auch um eine Hypothese, aber sie wird in einem Beweis nie direkt referenziert, weshalb wir sie eher als Beschränkung denn als Hypothese bezeichnen, um Verwirrung zu vermeiden.  Die Überprüfung auf Verletzungen von \texttt{\$d}-Restriktionen erfolgt automatisch und ist in den Algorithmus zur Überprüfung von Beweisen in Metamath integriert.} Die Buchstaben \texttt{f} und \texttt{e} stehen für "`fließend"' \index{fließende Hypothese} (was in etwa bedeutet, dass sie nur verwendet wird, wenn sie relevant ist) bzw. "`essentiell"' \index{essentielle Hypothese} (was bedeutet, dass sie immer verwendet wird), aus Gründen, die deutlich werden, wenn wir Frames in Abschnitt~\ref{frames} und Gültigkeitsbereiche in Abschnitt~\ref{scoping} besprechen. Die Syntax dieser Anweisungen ist wie folgt:
\begin{center}
  {\em label} \texttt{\$f} {\em typecode} {\em variable} \texttt{\$.}\\
  {\em label} \texttt{\$e} {\em typecode}
      {\em math-symbol}\ \,$\cdots$\ {\em math-symbol} \texttt{\$.}\\
\end{center}
\index{\texttt{\$e}-Anweisung}
\index{\texttt{\$f}-Anweisung}
Eine Hypothese muss ein {\em Label}\index{Label} haben.  Der Ausdruck in einer \texttt{\$e}-Hypothese besteht aus einem Typcode (einem aktiven konstanten mathematischen Symbol), gefolgt von einer Folge von keinem, einem oder mehreren mathematischen Symbolen. Jedes mathematische Symbol (einschließlich {\em constant} und {\em variable}) muss eine zuvor deklarierte Konstante oder Variable sein.  (Außerdem muss jedes mathematische Symbol aktiv sein, was bei der Besprechung von Gültigkeitsbereichsanweisungen in Abschnitt~\ref{scoping} behandelt wird).  Eine \texttt{\$f}-Hypothese wird verwendet, um die Art oder den {\bf Typ}\index{Variablentyp}\index{Typ} einer Variablen zu spezifizieren (wie z.B. "`sei $x$ eine ganze Zahl"'), und eine \texttt{\$e}-Hypothese wird verwendet, um eine logische Wahrheit auszudrücken (wie z.B. "`angenommen $x$ ist prim"'), die festgelegt sein muss, damit eine Behauptung, die sie erfordert, auch wahr ist.

Der Typ einer Variablen muss in einer \texttt{\$f}-Anweisung angegeben sein, bevor sie in einer \texttt{\$e}-, \texttt{\$a}- oder \texttt{\$p}-Anweisung verwendet werden kann.  Es darf nur eine (aktive) \texttt{\$f}-Anweisung für eine bestimmte Variable geben.  ("`Aktiv"' ist in Abschnitt~\ref{scoping} definiert.)

In der gewöhnlichen Mathematik werden Theoreme oft in der Form "`Annahme $P$; dann $Q$"' ausgedrückt, wobei $Q$ eine Aussage ist, die man ableiten kann, wenn man von der Aussage $P$ ausgeht.\index{freie Variable}
\footnote{Eine stärkere Version eines Theorems wie dieses wäre die {\em einzelne} Formel $P\rightarrow Q$ ($P$ impliziert $Q$), aus der die schwächere Version oben durch den Modus ponens in der Logik folgt.  
Wir diskutieren diese stärkere Form hier nicht.  In der schwächeren Form sagen wir nur, dass, wenn wir $P$ beweisen können, wir auch $Q$ beweisen können.  
Wenn $x$ die einzige freie Variable in $P$ und $Q$ ist, ist in der Sprache der Logiker die stärkere Form äquivalent zu $\forall x ( P \rightarrow Q)$ (für alle $x$ impliziert $P$ $Q$), während die schwächere Form äquivalent zu $\forall x P \rightarrow \forall x Q$ ist. Die stärkere Form impliziert die schwächere, aber nicht umgekehrt.  
Um genau zu sein, wird die schwächere Form des Satzes richtigerweise als "`Inferenz"' und nicht als Satz bezeichnet.}
In der Metamath-Sprache würde man die mathematische Aussage $P$ als Hypothese (in diesem Fall eine \texttt{\$e}-Anweisung in der Metamath-Sprache) und die Aussage $Q$ als beweisbare Behauptung (eine \texttt{\$p}-Anweisung in der Metamath-Sprache) ausdrücken.

Hier einige Beispiele für Hypothesen, denen Sie in der Logik und Mengenlehre begegnen könnten:
\begin{center}
	\texttt{stmt1 \$f wff P \$.}\\
	\texttt{stmt2 \$f setvar x \$.}\\
	\texttt{stmt3 \$e |- ( P -> Q ) \$.}
\end{center}
\index{\texttt{\$e}-Anweisung}
\index{\texttt{\$f}-Anweisung}

Informell würde dies lauten: "`Angenommen $P$ sei eine wohlgeformte Formel"', "`Angenommen $x$ sei eine (individuelle) Variable"' und "`Angenommen, wir haben $P \rightarrow Q$ bewiesen."'  Das Drehkreuz-Symbol \,$\vdash$\index{Drehkreuz ({$\,\vdash$})} wird in Logik-Texten häufig verwendet, um anzuzeigen, dass "`ein Beweis existiert für"'.

Zusammengefasst:
\begin{itemize}
\item Eine \texttt{\$f}-Hypothese teilt Metamath den Typ oder die Art seiner Variablen mit. Sie ist vergleichbar mit einer Variablendeklaration in einer Computersprache, die dem Compiler mitteilt, dass eine Variable eine Ganzzahl oder eine Gleitkommazahl ist.
\item Die \texttt{\$e}-Hypothese entspricht dem, was man in der gewöhnlichen Mathematik eine "`Hypothese"' nennt.
\end{itemize}

Bevor eine Behauptung\index{Behauptung} (\texttt{\$a}- oder \texttt{\$p}-Aussage) in einem Beweis referenziert werden kann, müssen alle zugehörigen \texttt{\$f}- und \texttt{\$e}-Hypothesen (d.h. \ jene \texttt{\$e}-Hypothesen, die aktiv sind) erfüllt sein (d.h. durch den Beweis nachgewiesen werden).  Die Bedeutung des Begriffs "`assoziiert"' (den wir in Abschnitt~\ref{frames} als {\bf obligatorisch} bezeichnen werden) wird klar, wenn wir später über Gültigkeitsbereiche sprechen.

Beachten Sie, dass nach jedem \texttt{\$f}-, \texttt{\$e}-, \texttt{\$a}- oder \texttt{\$p}-Token ein \textit{typecode}\index{Typcode} stehen muss. Der Typcode ist eine Konstante, die verwendet wird, um die Typen von Ausdrücken festzulegen. Dies wird deutlicher, wenn wir mehr über Behauptungen (\texttt{\$a}- und \texttt{\$p}-Anweisungen) erfahren. Ein Beispiel kann auch ihren Zweck verdeutlichen. In der Datenbasis \texttt{set.mm}\index{Mengenlehre-Datenbasis (\texttt{set.mm})}%
\index{Metamath Proof Explorer} werden die folgenden Typcodes verwendet:

\begin{itemize}
\item \texttt{wff} :
  Symbol einer wohlgeformten Formel (wff)
  (sprich: "`die folgende Symbolfolge ist eine wff"').
% The *textual* typecode for turnstile is "|-", but when read it's a little
% confusing, so I intentionally display the mathematical symbol here instead
% (I think it's clearer in this context).
\item \texttt{$\vdash$} :
  Drehkreuz (sprich: "`die folgende Symbolfolge ist beweisbar"' oder "`ein Beweis existiert für"').
\item \texttt{setvar} :
  Typ der individuellen Mengenvariable (sprich: "`das Folgende ist eine individuelle Mengenvariable"').   Beachten Sie, dass dies \textit{nicht} der Typ eines beliebigen Mengenausdrucks ist, sondern dazu dient, sicherzustellen, dass nur ein einziges Symbol nach Quantoren wie "`für alle"' ($\forall$) und "`Es gibt ein"' ($\exists$) verwendet wird.
\item \texttt{class} :
  Ein Ausdruck, der ein syntaktisch gültiger Klassenausdruck ist.   Alle gültigen Mengenausdrücke sind auch gültige Klassenausdrücke, daher haben Mengenausdrücke normalerweise den Typcode \texttt{class}.   Verwenden Sie den Typcode \texttt{class} und \textit{nicht} den \texttt{setvar}-Typcode für den Typ von Mengenausdrücken, es sei denn, Sie wollen gezielt eine einzelne Mengenvariable bestimmen.
\end{itemize}

\subsection{Behauptungen (\texttt{\$a}- und \texttt{\$p}-Anweisungen)}
\index{\texttt{\$a}-Anweisung}\index{\texttt{\$p}-Anweisung}\index{Behauptung}\index{axiomatische Behauptung}\index{beweisbare Behauptung}

Es gibt zwei Arten von Behauptungen, \texttt{\$a}\index{\texttt{\$a}-Anweisung}-Anweisungen ({\bf axiomatische Behauptungen}) und \texttt{\$p}-Anweisungen ({\bf beweisbare Behauptungen}).  Ihre Syntax lautet wie folgt:
\begin{center}
  {\em label} \texttt{\$a} {\em typecode} {\em math-symbol} \ldots
         {\em math-symbol} \texttt{\$.}\\
  {\em label} \texttt{\$p} {\em typecode} {\em math-symbol} \ldots
        {\em math-symbol} \texttt{\$=} {\em proof} \texttt{\$.}
\end{center}
\index{\texttt{\$a}-Anweisung}
\index{\texttt{\$p}-Anweisung}
\index{\texttt{\$=} Schlüsselwort}
Eine Behauptung erfordert immer ein {\em Label}\index{Label}. Der Ausdruck in einer Behauptung besteht aus einem Typcode (einer aktiven Konstante), gefolgt von einer Folge von keinem, einem oder mehreren mathematischen Symbolen.  Jedes mathematische Symbol, einschließlich aller {\em Konstanten}, muss eine zuvor deklarierte Konstante oder Variable sein.  (Außerdem muss jedes mathematische Symbol aktiv sein, was bei der Besprechung von Gültigkeitsbereichsanweisungen in Abschnitt~\ref{scoping} behandelt wird).

Eine \texttt{\$a}-Anweisung ist in der Regel eine syntaktische Definition (z. B. wenn $P$ und $Q$ wffs sind, dann ist $(P\to Q)$ auch eine wwf), ein Axiom\index{Axiom} der gewöhnlichen Mathematik (z. B. $x=x$) oder eine Definition\index{Definition} der gewöhnlichen Mathematik (z. B. $x\ne y$ bedeutet $\lnot x=y$). Eine \texttt{\$p}-Anweisung ist eine Behauptung, dass eine bestimmte Kombination von mathematischen Symbolen aus früheren Behauptungen folgt, und wird durch einen Beweis ergänzt, der dies zeigt.

Behauptungen können auch in (späteren) Beweisen referenziert werden, um neue Behauptungen aus ihnen abzuleiten. Das Label einer Behauptung wird verwendet, um in einem Beweis auf sie zu verweisen. In Abschnitt~\ref{proof} wird der Beweis im Detail beschrieben.

Behauptungen sind auch das wichtigste Mittel, um Menschen die mathematischen Ergebnisse in der Datenbasis mitzuteilen.  Beweise (wenn sie in geeigneter Weise angezeigt werden) vermitteln Menschen, wie die Ergebnisse zustande gekommen sind.

\subsubsection{Die \texttt{\$a}-Anweisung}
\index{\texttt{\$a}-Anweisung}

Axiomatische Behauptungen (\texttt{\$a}-Anweisungen) stellen die Ausgangspunkte dar, von denen andere Behauptungen (\texttt{\$p}\index{\texttt{\$p}-Anweisung}-Anweisungen) abgeleitet werden.  Ihr offensichtlichster Verwendungszweck ist die Spezifizierung gewöhnlicher mathematischer Axiome\index{Axiom}, aber sie werden auch für zwei andere Zwecke verwendet.

Zunächst muss Metamath\index{Metamath} die Syntax der Symbolfolgen kennen, die gültige mathematische Aussagen darstellen.  Ein Metamath-Beweis muss viel detaillierter aufgeschlüsselt werden als gewöhnliche mathematische Beweise, an die Sie vielleicht gewöhnt sind (sogar die "`vollständigen"' Beweise der formalen Logik\index{formale Logik}).  Dies ist einer der Faktoren, die Metamath zu einer Allzwecksprache machen, die unabhängig von jedem Logiksystem oder sogar irgendeiner Syntax ist.  Wenn Sie einen Substitution für eine Behauptung als Schritt in einem Beweis durchführen wollen, müssen Sie zuerst beweisen, dass die Substitution syntaktisch korrekt ist (oder, wenn Sie es vorziehen, müssen Sie sie "`konstruieren"'), indem Sie zum Beispiel zeigen, dass der Ausdruck, den Sie für eine wff-Metavariable ersetzen, eine gültige wff ist. Die Anweisung \texttt{\$a}\index{\texttt{\$a}-Anweisung} wird verwendet, um die Kombinationen von Symbolen anzugeben, die als syntaktisch gültig angesehen werden, wie z.B. die legalen Formen von wffs.

Zweitens werden \texttt{\$a}-Anweisungen verwendet, um das zu spezifizieren, was man sich normalerweise als Definitionen vorstellt, d.h. \ neue Kombinationen von Symbolen, die andere Kombinationen von Symbolen abkürzen.  Metamath unterscheidet nicht zwischen Axiomen\index{Axiom} und Definitionen\index{Definition}. In der Tat wurde argumentiert, dass eine solche Unterscheidung nicht einmal in der gewöhnlichen Mathematik gemacht werden sollte; siehe Abschnitt~\ref{definitions}, der die Philosophie der Definitionen diskutiert.  In Abschnitt~\ref{hierarchy} werden einige technische Anforderungen an Definitionen erörtert.  In \texttt{set.mm} übernehmen wir die Konvention, den Labels für Axiome \texttt{ax-} und den Labels für Definitionen \texttt{df-}\index{Label} voranzustellen.

Die Ergebnisse, die mit der Metamath-Sprache abgeleitet werden können, sind nur so gut wie die \texttt{\$a}\index{\texttt{\$a}-Anweisung}-Anweisungen, die als deren Ausgangspunkt verwendet werden.  Wir können dies nicht genug betonen.  Metamath wird Sie zum Beispiel nicht daran hindern, $x\neq x$ als ein Axiom der Logik anzugeben.  Es ist wichtig, dass Sie alle \texttt{\$a}-Anweisungen mit großer Sorgfalt prüfen. Da sie eine Quelle potenzieller Fallstricke sind, ist es am besten, keine neuen (normalerweise neue Definitionen) beiläufig hinzuzufügen; vielmehr sollten Sie die Notwendigkeit und die Vorteile jeder einzelnen sorgfältig bewerten.

Wenn Sie alle grundlegenden Axiome\index{Axiom} und Regeln\index{Regel} einer mathematischen Theorie aufgestellt haben, sind die einzigen \texttt{\$a}-Anweisungen, die Sie hinzufügen werden, die so genannten Definitionen.  Im Prinzip sollten Definitionen in gewissem Sinne aus der Sprache einer Theorie eliminierbar sein, und zwar gemäß einer bestimmten Konvention (die normalerweise logische Äquivalenz oder Gleichheit beinhaltet).  Die gebräuchlichste Konvention ist, dass jede Formel, die vor der Einführung der Definition syntaktisch gültig, aber nicht beweisbar war, nach der Einführung der Definition nicht beweisbar wird.  In einer idealen Welt sollten Definitionen überhaupt nicht vorhanden sein, wenn man absolutes Vertrauen in ein mathematisches Ergebnis haben will.  Sie sind jedoch notwendig, um die Mathematik praktikabel zu machen, da die resultierenden Formeln sonst extrem lang und unverständlich wären.  Da es die Natur von Definitionen (im allgemeinsten Sinne) nicht gestattet, dass sie automatisch als "`zulässig"'\index{zulässige Definition}\index{Definition!zulässig} überprüft werden können, ist das Urteil des Mathematikers erforderlich, um dies sicherzustellen.  (In \texttt{set.mm} wurden Anstrengungen unternommen, um fast alle Definitionen direkt eliminierbar zu machen und so die Notwendigkeit eines solchen Urteils zu minimieren.)

Wenn Sie kein Mathematiker sind, ist es vielleicht am besten, keine \texttt{\$a}\index{\texttt{\$a}-Anweisung}-Anweisungen hinzuzufügen oder zu ändern, sondern stattdessen die mathematische Sprache zu verwenden, die bereits in Standarddatenbasen zur Verfügung gestellt wird.  Auf diese Weise wird Metamath nicht zulassen, dass Sie einen Fehler machen (d.h. ein falsches Ergebnis beweisen).

\subsection{Frames}\label{frames}

Wir führen nun das Konzept einer Sammlung zusammengehöriger Metamath-Anweisungen ein, die als Frame bezeichnet wird.  Jede Behauptung (\texttt{\$a}- oder \texttt{\$p}-Anweisung) in der Datenbasis hat einen zugehörigen Frame.

Ein {\bf Frame}\index{Frame} ist eine Folge von \texttt{\$d}-, \texttt{\$f}- und \texttt{\$e}-Anweisungen (keine, eine oder mehrere von jeder), gefolgt von einer \texttt{\$a}- oder \texttt{\$p}-Anweisung, vorbehaltlich bestimmter, im Folgenden beschriebenen Bedingungen.  Der Einfachheit halber nehmen wir an, dass alle verwendeten Token, die mathematischen Symbole darstellen, am Anfang der Datenbasis mit den Anweisungen \texttt{\$c} und \texttt{\$v} deklariert werden (die eigentlich nicht Teil eines Frames sind).  Der Einfachheit halber nehmen wir auch an, dass es nur einfache \texttt{\$d}-Anweisungen gibt (solche mit nur zwei Variablen) und stellen uns alle zusammengesetzten \texttt{\$d}-Anweisungen (solche mit mehr als zwei Variablen) als in einfache zerlegt vor.

Ein Frame fasst die Hypothesen (und \texttt{\$d}-Anweisungen) zusammen, die für eine Behauptung (\texttt{\$a}- oder \texttt{\$p}-Anweisung) relevant sind.  Die Anweisungen in einem Frame können in einer Datenbasis physisch benachbart sein, müssen es aber nicht; wir werden dies in unserer Diskussion über Gültigkeitsbereichsanweisungen in Abschnitt~\ref{scoping} behandeln.

Ein Frame hat die folgenden Eigenschaften:
\begin{enumerate}
 \item Die Menge der in den Anweisungen von \texttt{\$f} enthaltenen Variablen muss mit der Menge der in seinen \texttt{\$e}-, \texttt{\$a}- und/oder \texttt{\$p}-Anweisungen enthaltenen Variablen identisch sein.  Mit anderen Worten: Für jede Variable in einer \texttt{\$e}-, \texttt{\$a}- oder \texttt{\$p}-Anweisung muss in einer \texttt{\$f}-Anweisung ein zugehöriger "`Variablentyp"' definiert sein.
  \item Keine zwei \texttt{\$f}-Anweisungen dürfen die gleiche Variable enthalten.
  \item Jede \texttt{\$f}-Anweisung muss vor einer \texttt{\$e}-Anweisung stehen, in der ihre Variable vorkommt.
\end{enumerate}

Die erste Eigenschaft bestimmt die Menge der in einem Frame vorkommenden Variablen. Dies sind die {\bf obligatorischen Variablen}\index{obligatorische Variable} des Frames.  Die zweite Eigenschaft besagt, dass für eine Variable nur ein Typ angegeben werden darf. Die letzte Eigenschaft ist keine theoretische Anforderung, aber sie erleichtert das Parsen der Datenbasis.

Für unsere Beispiele gehen wir davon aus, dass unsere Datenbasis die folgenden Deklarationen hat:

\begin{verbatim}
$v P Q R $.
$c -> ( ) |- wff $.
\end{verbatim}

Die folgende Folge von Anweisungen, die die Modus ponens Schlussregel beschreibt, ist ein Beispiel für ein Frame: 

\begin{verbatim}
wp  $f wff P $.
wq  $f wff Q $.
maj $e |- ( P -> Q ) $.
min $e |- P $.
mp  $a |- Q $.
\end{verbatim}

Die folgende Folge von Anweisungen ist kein Frame, da \texttt{R} weder in den \texttt{\$e}'s noch in den \texttt{\$a}'s vorkommt:

\begin{verbatim}
wp  $f wff P $.
wq  $f wff Q $.
wr  $f wff R $.
maj $e |- ( P -> Q ) $.
min $e |- P $.
mp  $a |- Q $.
\end{verbatim}

Die folgende Folge von Anweisungen ist kein Frame, da \texttt{Q} nicht in einem \texttt{\$f} vorkommt:

\begin{verbatim}
wp  $f wff P $.
maj $e |- ( P -> Q ) $.
min $e |- P $.
mp  $a |- Q $.
\end{verbatim}

Die folgende Folge von Anweisungen ist kein Frame, da die Anweisung \texttt{\$a} nicht die letzte ist:

\begin{verbatim}
wp  $f wff P $.
wq  $f wff Q $.
maj $e |- ( P -> Q ) $.
mp  $a |- Q $.
min $e |- P $.
\end{verbatim}

Einem Frame ist eine Folge von {\bf obligatorischen Hypothesen}\index{obligatorische Hypothese} zugeordnet.  Dies ist einfach die Menge aller \texttt{\$f}- und \texttt{\$e}-Anweisungen im Frame, in der Reihenfolge, in der sie angegeben werden.  Auf einen Frame kann in einem späteren Beweis über das Label der \texttt{\$a}- oder \texttt{\$p}-Aussage Bezug genommen werden, und der Beweis nimmt eine Zuordnung zu jeder zwingenden Hypothese in der Reihenfolge vor, in der sie angegeben wurde.  Das bedeutet, dass die einmal gewählte Reihenfolge der Hypothesen nicht mehr geändert werden darf, um spätere Beweise, die sich auf die Anweisung des Frames beziehen, nicht zu beeinflussen.  (Der Beweisverifizierer von Metamath wird natürlich einen Fehler anzeigen, wenn ein Beweis dadurch falsch wird.)  Da Beweise die "`umgekehrte Polnische Notation"' verwenden, die in Abschnitt~\ref{proof} beschrieben wird, nennen wir diese Reihenfolge die {\bf RPN-Reihenfolge}\index{RPN-Reihenfolge} der Hypothesen.

Beachten Sie, dass \texttt{\$d}-Anweisungen nicht zur Menge der obligatorischen Hypothesen gehören und ihre Reihenfolge keine Rolle spielt (solange sie die oben beschriebene vierte Eigenschaft für einen Frame erfüllen).  Die \texttt{\$d}-Anweisungen geben Beschränkungen für Variablen an, die erfüllt sein müssen (und vom Beweisverifizierer überprüft werden), wenn Ausdrücke in einem Beweis für sie ersetzt werden, und die \texttt{\$d}-Anweisungen selbst werden in einem Beweis nie direkt referenziert.

Ein Frame mit einer (beweisbaren) \texttt{\$p}-Anweisung erfordert einen Beweis als Teil der \texttt{\$p}-Anweisung.  Manchmal wollen wir in einem Beweis temporäre Variablen oder Dummy-Variablen verwenden, die nicht in der \texttt{\$p}-Anweisung oder ihren obligatorischen Hypothesen vorkommen.  Um dies zu ermöglichen, definieren wir einen {\bf erweiterten Frame}\index{erweiterter Frame} als einen Frame mit keinem, einem oder mehreren \texttt{\$d}- und \texttt{\$f}-Anweisungen, die auf Variablen verweisen, die nicht zu den obligatorischen Variablen des Frames gehören.  Alle neuen Variablen, auf die verwiesen wird, werden als {\bf optionale Variablen}\index{optionale Variable} des erweiterten Frames bezeichnet. Wenn eine \texttt{\$f}-Anweisung auf eine optionale Variable verweist, wird sie als {\bf optionale Hypothese}\index{optionale Hypothese} bezeichnet, und wenn eine oder beide der Variablen in einer \texttt{\$d}-Anweisung optionale Variablen sind, wird sie als {\bf optionale disjunkte Variableneinschränkung}\index{optionale disjunkte Variableneinschränkung} bezeichnet.  Die Eigenschaften 2 und 3 für einen Frame gelten auch für einen erweiterten Frame.

Das Konzept der optionalen Variablen ist für Frames mit \texttt{\$a}-Anweisungen nicht sinnvoll, da diese Anweisungen keine Beweise haben, die von ihnen Gebrauch machen könnten. Es gibt kein Verbot für die Aufnahme optionaler Hypothesen in den erweiterten Frame für eine \texttt{\$a}-Anweisung, aber sie haben keinen Zweck.

Der folgende Satz von Anweisungen ist ein Beispiel für einen erweiterten Frame, das eine optionale Variable \texttt{R} und eine optionale Hypothese \texttt{wr} enthält.  In diesem Beispiel nehmen wir an, dass der Modus ponens kein Axiom ist, sondern als Theorem aus früheren Aussagen abgeleitet wurde (wir lassen den angenommenen Beweis weg). Die Variable \texttt{R} kann, falls gewünscht, in ihrem Beweis verwendet werden (obwohl dies in der Aussagenlogik wahrscheinlich keinen Vorteil hätte).  Beachten Sie, dass die Reihenfolge der obligatorischen Hypothesen in der RPN-Reihenfolge immer noch \texttt{wp}, \texttt{wq}, \texttt{maj}, \texttt{min} lautet (d.h. \texttt{wr} wird weggelassen), und diese Reihenfolge wird immer noch angenommen, wenn in einem nachfolgenden Beweis auf die Behauptung \texttt{mp} verwiesen wird.

\begin{verbatim}
wp  $f wff P $.
wq  $f wff Q $.
wr  $f wff R $.
maj $e |- ( P -> Q ) $.
min $e |- P $.
mp  $p |- Q $= ... $.
\end{verbatim}

Jeder Frame ist ein erweiterter Frame, aber nicht jeder erweiterte Frame ist ein Frame, wie dieses Beispiel zeigt.  Den zugrundeliegenden Frame für einen erweiterten Frame erhält man, indem man einfach alle Anweisungen entfernt, die optionale Variablen enthalten. Jeder Beweis, der sich auf eine Behauptung bezieht, ignoriert alle Erweiterungen ihres Frames, was bedeutet, dass wir optionale Hypothesen nach Belieben hinzufügen oder löschen können, ohne dass dies Auswirkungen auf nachfolgende Beweise hat. 

Die konzeptionell einfachste Art, eine Metamath-Datenbasis zu organisieren, ist eine Folge von erweiterten Frames.  Die Gültigkeitsbereichsanweisungen \texttt{\$\char`\{}\index{\texttt{\$\char`\{} und \texttt{\$\char`\}} Schlüsselwörter} und \texttt{\$\char`\}} können verwendet werden, um den Anfang und das Ende eines erweiterten Frames abzugrenzen, was zu der folgenden möglichen Struktur für eine Datenbasis führt.  \label{framelist} 

\vskip 2ex
\setbox\startprefix=\hbox{\tt \ \ \ \ \ \ \ \ }
\setbox\contprefix=\hbox{}
\startm
\m{\mbox{(\texttt{\$v-} {\em und} \texttt{\$c-}{\em Anweisungen})}}
\endm
\startm
\m{\mbox{\texttt{\$\char`\{}}}
\endm
\startm
\m{\mbox{\texttt{\ \ } {\em erweiterter Frame}}}
\endm
\startm
\m{\mbox{\texttt{\$\char`\}}}}
\endm
\startm
\m{\mbox{\texttt{\$\char`\{}}}
\endm
\startm
\m{\mbox{\texttt{\ \ } {\em erweiterter Frame}}}
\endm
\startm
\m{\mbox{\texttt{\$\char`\}}}}
\endm
\startm
\m{\mbox{\texttt{\ \ \ \ \ \ \ \ \ }}\vdots}
\endm
\vskip 2ex

In der Praxis ist diese Struktur unzweckmäßig, weil wir alle \texttt{\$f}-, \texttt{\$e}- und \texttt{\$d}-Anweisungen immer und immer wieder wiederholen müssen, anstatt sie einmal für die Verwendung durch mehrere Annahmen anzugeben. Die Gültigkeitsbereichsanweisungen, die wir als nächstes besprechen werden, ermöglichen dies.  Im Prinzip kann jede Metamath-Datenbasis in das obige Format konvertiert werden, und das obige Format ist am bequemsten zu verwenden, wenn man eine Metamath-Datenbasis als formales System betrachtet %
%% Uncomment this when uncommenting section {formalspec} below
   (Appendix \ref{formalspec})%
.
Tatsächlich konvertiert Metamath die Datenbasis intern in das oben genannte Format. Der Befehl \texttt{show statement} im Metamath-Programm zeigt Ihnen den Inhalt des Frames für jede \texttt{\$a}- oder \texttt{\$p}-Anweisung sowie die Erweiterung im Falle einer \texttt{\$p}-Anweisung.

%c%(provided that all "`local"' variables and constants with limited scope have
%c%unique names),

Bei der Diskussion von Anweisungen mit Gültigkeitsbereich kann es hilfreich sein, sich die äquivalente Folge von Frames vorzustellen, die beim Parsen der Datenbasis entsteht.  Gültigkeitsbereiche sind (abgesehen von der oben erwähnten begrenzten Verwendung zur Abgrenzung von Frames) keine theoretische Voraussetzung für Metamath, macht den Umgang damit aber bequemer.

\subsection{Gültigkeitsbereichsanweisungen (\texttt{\$\{} und \texttt{\$\}})}\label{scoping}
\index{\texttt{\$\char`\{} und \texttt{\$\char`\}} Schlüsselwörter}\index{Gültigkeitsbereichsanweisung}

%c%Some Metamath statements may be needed only temporarily to
%c%serve a specific purpose, and after we're done with them we would like to
%c%disregard or ignore them.  For example, when we're finished using a variable,
%c%we might want to
%c%we might want to free up the token\index{Token} used to name it so that the
%c%token can be used for other purposes later on, such as a different kind of
%c%variable or even a constant.  In the terminology of computer programming, we
%c%might want to let some symbol declarations be "`local"' rather than "`global."'
%c%\index{local symbol}\index{global symbol}

Die {\bf Gültigkeitsbereich}-Anweisungen, \texttt{\$\char`\{} ({\bf beginn eines Blocks}) und \texttt{\$\char`\}} ({\bf Ende eines Blocks})\index{Block}, bieten ein Mittel zur Steuerung des Teils einer Datenbasis, in dem bestimmte Anweisungstypen erkannt werden.  Die Syntax einer Gültigkeitsbereichsanweisung ist sehr einfach; sie besteht lediglich aus dem Schlüsselwort der Anweisung: 
\begin{center}
\texttt{\$\char`\{}\\
\texttt{\$\char`\}}
\end{center}
\index{\texttt{\$\char`\{} und \texttt{\$\char`\}} Schlüsselwörter}

Betrachten wir zum Beispiel die folgende Datenbasis, in der wir alle Token außer den Schlüsselwörtern der Anweisung entfernt haben.  Für die folgende Diskussion haben wir die Anweisungen mit Indizes (tiefgestellten Ziffern) versehen; diese Indizes erscheinen nicht in der eigentlichen Datenbasis.
\[
 \mbox{\tt \ \$\char`\{}_1
 \mbox{\tt \ \$\char`\{}_2
 \mbox{\tt \ \$\char`\}}_2
 \mbox{\tt \ \$\char`\{}_3
 \mbox{\tt \ \$\char`\{}_4
 \mbox{\tt \ \$\char`\}}_4
 \mbox{\tt \ \$\char`\}}_3
 \mbox{\tt \ \$\char`\}}_1
\]
Zu jeder \texttt{\$\char`\{}-Anweisung in diesem Beispiel {\bf gehört} die \texttt{\$\char`\}}-Anweisung mit demselben Index.  Jedes Paar von so zusammengehörenden Anweisungen definiert einen Bereich der Datenbasis, der als {\bf Block}\index{Block} bezeichnet wird. Blöcke können innerhalb anderer Blöcke {\bf verschachtelt} werden; im Beispiel, ist der Block, der durch $\mbox{\tt \$\char`\{}_4$ und $\mbox{\tt \$\char`\}}_4$ definiert ist, innerhalb des Blocks verschachtelt, der durch $\mbox{\tt \$\char`\{}_3$ und $\mbox{\tt \$\char`\}}_3$ sowie innerhalb des durch $\mbox{\tt \$\char`\{}_1$ und $\mbox{\tt \$\char`\}}_1$ definierten Blocks.  Im Allgemeinen kann ein Block leer sein, er kann nur Anweisungen enthalten, die keine Gültigkeitsbereichsanweisungen sind\footnote{Die Anweisungen, die nicht \texttt{\$\char`\{} und \texttt{\$\char`\}} sind.} oder er kann eine beliebige Mischung aus anderen Blöcken und Anweisungen, die keine Gültigkeitsbereichsanweisungen sind, enthalten.  (Dies wird eine "`rekursive"' Definition\index{rekursive Definition} eines Blocks genannt.)

Jedem Block ist eine Nummer zugeordnet, die als seine {\bf Verschachtelungstiefe}\index{Verschachtelungstiefe} bezeichnet wird und angibt, wie tief der Block verschachtelt ist. Die Verschachtelungstiefe der Blöcke in unserem Beispiel sind wie folgt:
\[
  \underbrace{
    \mbox{\tt \ }
    \underbrace{
     \mbox{\tt \$\char`\{\ }
     \underbrace{
       \mbox{\tt \$\char`\{\ }
       \mbox{\tt \$\char`\}}
     }_{2}
     \mbox{\tt \ }
     \underbrace{
       \mbox{\tt \$\char`\{\ }
       \underbrace{
         \mbox{\tt \$\char`\{\ }
         \mbox{\tt \$\char`\}}
       }_{3}
       \mbox{\tt \ \$\char`\}}
     }_{2}
     \mbox{\tt \ \$\char`\}}
   }_{1}
   \mbox{\tt \ }
 }_{0}
\]
\index{\texttt{\$\char`\{} und \texttt{\$\char`\}} Schlüsselwörter}
Die gesamte Datenbasis wird als ein einziger großer Block (der äußerste Block) mit einer Verschachtelungstiefe von 0 betrachtet. Der äußerste Block wird nicht durch Gültigkeitsbereichsanweisungen eingeklammert.\footnote{Die Sprache wurde deshalb so konzipiert, damit mehrere Quelldateien leichter zusammengefügt werden können.}\index{äußerster Block}

Alle Metamath-Anweisungen, die unabhängig von einem Gültigkeitsbereich sind, werden ab der Stelle, an der sie angegeben werden, gültig oder {\bf aktiv}\index{aktive Anweisung}.\footnote{Um die Dinge etwas zu vereinfachen, machen wir uns nicht die Mühe, den Begriff "`aktiv"' für die Gültigkeitsbereichsanweisungen zu definieren.}  Bestimmte dieser Anweisungstypen werden am Ende des Blocks, in dem sie enthalten sind, inaktiv; diese Anweisungstypen sind:

\begin{center}
  \texttt{\$c}, \texttt{\$v}, \texttt{\$d}, \texttt{\$e}, und \texttt{\$f}.
%  \texttt{\$v}, \texttt{\$f}, \texttt{\$e}, und \texttt{\$d}.
\end{center}
\index{\texttt{\$c}-Anweisung}
\index{\texttt{\$d}-Anweisung}
\index{\texttt{\$e}-Anweisung}
\index{\texttt{\$f}-Anweisung}
\index{\texttt{\$v}-Anweisung}

Die anderen Anweisungstypen bleiben für immer aktiv (d.h. bis zum Ende der Datenbasis); dies sind:
\begin{center}
  \texttt{\$a} und \texttt{\$p}.
%  \texttt{\$c}, \texttt{\$a}, und \texttt{\$p}.
\end{center}
\index{\texttt{\$a}-Anweisung}
\index{\texttt{\$p}-Anweisung}
Jede Anweisung (dieser 7 Typen), die sich im äußersten Block\index{äußerster Block} befindet, bleibt bis zum Ende der Datenbasis aktiv und ist somit effektiv eine "`globale"' Anweisung.\index{globale Anweisung}

Alle \texttt{\$c}-Anweisungen müssen im äußersten Block platziert werden.  Da sie also immer global sind, könnte man sie als zu beiden oben genannten Kategorien gehörig betrachten.

Der {\bf Gültigkeitsbereich}\index{Gültigkeitsbereich} einer Anweisung ist die Menge der Anweisungen, die sie als aktiv erkennen.

%c%The concept of "`active"' is also defined for math symbols\index{math
%c%symbol}.  Math symbols (constants\index{Konstante} and
%c%variables\index{Variable}) become {\bf active}\index{active
%c%math symbol} in the \texttt{\$c}\index{\texttt{\$c}
%c%statement} and \texttt{\$v}\index{\texttt{\$v}-Anweisung} statements that
%c%declare them.  They become inactive when their declaration statements become
%c%inactive.

Der Begriff "`aktiv"' ist auch für mathematische Symbole\index{mathematisches Symbol} definiert.  Mathematische Symbole (Konstanten\index{Konstante} und Variablen\index{Variable}) werden {\bf aktiv}\index{aktives mathematisches Symbol} in den \texttt{\$c}\index{\texttt{\$c}-Anweisung}-Anweisungen und \texttt{\$v}\index{\texttt{\$v}-Anweisung}-Anweisungen, in den sie deklariert werden. Eine Variable wird inaktiv, wenn ihre Deklarationsanweisung inaktiv wird.  Da alle \texttt{\$c}-Anweisungen im äußersten Block stehen müssen, wird eine Konstante niemals inaktiv, nachdem sie deklariert wurde.

\subsubsection{Umdeklarierung von mathematischen Symbolen}
\index{Umdeklarierung von Symbolen}\label{redeclaration}

%c%A math symbol may not be declared a second time while it is active, but it may
%c%be declared again after it becomes inactive.

Eine Variable kann nicht ein zweites Mal deklariert werden, solange sie aktiv ist, aber sie kann erneut deklariert werden, nachdem sie inaktiv geworden ist.  Auf diese Weise lassen sich auf bequeme Weise "`lokale"' Variablen\index{lokale Variable} einführen, d.h. temporäre Variablen, die im Rahmen einer Behauptung oder eines Beweises verwendet werden können, ohne dass man sie für immer beibehalten muss.  Eine zuvor deklarierte Variable kann nicht neu als Konstante deklariert werden.

Eine Konstante darf nicht neu deklariert werden.  Und, wie oben erwähnt, müssen Konstanten im äußersten Block deklariert werden.

Der Grund dafür, dass Variablen einen begrenzten Gültigkeitsbereich haben können, Konstanten jedoch nicht, liegt darin, dass eine Behauptung (\texttt{\$a}- oder \texttt{\$p}-Anweisung) bis zum Ende der Datenbasis für die Verwendung in Beweisen verfügbar bleibt.  Variablen im Rahmen einer Behauptung können in einem Beweisschritt, der sich auf die Behauptung bezieht, durch alles ersetzt werden, was benötigt wird, während Konstanten fest bleiben und durch nichts ersetzt werden können.  Das spezielle Token, das für eine Variable im Rahmen einer Behauptung verwendet wird, ist irrelevant, wenn die Behauptung in einem Beweis referenziert wird, und es spielt keine Rolle, wenn dieses Token außerhalb des Frames der referenzierten Behauptung nicht verfügbar ist. Konstanten hingegen müssen global festgelegt werden.

Theoretisch ist es nicht notwendig, dass Variablen für begrenzte Bereiche aktiv sind, anstatt global zu sein. Es handelt sich lediglich um eine Annehmlichkeit, die es beispielsweise ermöglicht, sie lokal mit den entsprechenden Deklarationen vom \texttt{\$f} Variablentyp zu gruppieren.

%c%If you declare a math symbol more than once, internally Metamath considers it a
%c%new distinct symbol, even though it has the same name.  If you are unaware of
%c%this, you may find that what you think are correct proofs are incorrectly
%c%rejected as invalid, because Metamath may tell you that a constant you
%c%previously declared does not match a newly declared math symbol with the same
%c%name.  For details on this subtle point, see the Comment on
%c%p.~\pageref{spec4comment}.  This is done purposely to allow temporary
%c%constants to be introduced while developing a subtheory, then allow their math
%c%symbol tokens to be reused later on; in general they will not refer to the
%c%same thing.  In practice, you would not ordinarily reuse the names of
%c%constants because it would tend to be confusing to the reader.  The reuse of
%c%names of variables, on the other hand, is something that is often useful to do
%c%(for example it is done frequently in \texttt{set.mm}).  Since variables in an
%c%assertion referenced in a proof can be substituted as needed to achieve a
%c%symbol match, this is not an issue.

% (This section covers a somewhat advanced topic you may want to skip
% at first reading.)
%
% Under certain circumstances, math symbol\index{mathematisches Symbol}
% tokens\index{Token} may be redeclared (i.e.\ the token
% may appear in more than
% one \texttt{\$c}\index{\texttt{\$c}-Anweisung} or \texttt{\$v}\index{\texttt{\$v}
% statement} statement).  You might want to do this say, to make temporary use
% of a variable name without having to worry about its affect elsewhere,
% somewhat analogous to declaring a local variable in a standard computer
% language.  Understanding what goes on when math symbol tokens are redeclared
% is a little tricky to understand at first, since it requires that we
% distinguish the token itself from the math symbol that it names.  It will help
% if we first take a peek at the internal workings of the
% Metamath\index{Metamath} program.
%
% Metamath reserves a memory location for each occurrence of a
% token\index{Token} in a declaration statement (\texttt{\$c}\index{\texttt{\$c}
% statement} or \texttt{\$v}\index{\texttt{\$v}-Anweisung}).  If a given token appears
% in more than one declaration statement, it will refer to more than one memory
% locations.  A math symbol\index{mathematisches Symbol} may be thought of as being one of
% these memory locations rather than as the token itself.  Only one of the
% memory locations associated with a given token may be active at any one time.
% The math symbol (memory location) that gets looked up when the token appears
% in a non-declaration statement is the one that happens to be active at that
% time.
%
% We now look at the rules for the redeclaration\index{Umdeklarierung von Symbolen}
% of math symbol tokens.
% \begin{itemize}
% \item A math symbol token may not be declared twice in the
% same block.\footnote{While there is no theoretical reason for disallowing
% this, it was decided in the design of Metamath that allowing it would offer no
% advantage and might cause confusion.}
% \item An inactive math symbol may always be
% redeclared.
% \item  An active math symbol may be redeclared in a different (i.e.\
% inner) block\index{Block} from the one it became active in.
% \end{itemize}
%
% When a math symbol token is redeclared, it conceptually refers to a different
% math symbol, just as it would be if it were called a different name.  In
% addition, the original math symbol that it referred to, if it was active,
% temporarily becomes inactive.  At the end of the block in which the
% redeclaration occurred, the new math symbol\index{mathematisches Symbol} becomes
% inactive and the original symbol becomes active again.  This concept is
% illustrated in the following example, where the symbol \texttt{e} is
% ordinarily a constant (say Euler's constant, 2.71828...) but
% temporarily we want to use it as a "`local"' variable, say as a coefficient
% in the equation $a x^4 + b x^3 + c x^2 + d x + e$:
% \[
%   \mbox{\tt \$\char`\{\ \$c e \$.}
%   \underbrace{
%     \ \ldots\ %
%     \mbox{\tt \$\char`\{}\ \ldots\ %
%   }_{\mbox{\rm region A}}
%   \mbox{\tt \$v e \$.}
%   \underbrace{
%     \mbox{\ \ \ \ldots\ \ \ }
%   }_{\mbox{\rm region B}}
%   \mbox{\tt \$\char`\}}
%   \underbrace{
%     \mbox{\ \ \ \ldots\ \ \ }
%   }_{\mbox{\rm region C}}
%   \mbox{\tt \$\char`\}}
% \]
% \index{\texttt{\$\char`\{} and \texttt{\$\char`\}} Schlüsselwörter}
% In region A, the token \texttt{e} refers to a constant.  It is redeclared as a
% variable in region B, and any reference to it in this region will refer to this
% variable.  In region C, the redeclaration becomes inactive, and the original
% declaration becomes active again.  In region C, the token \texttt{x} refers to the
% original constant.
%
% As a practical matter, overuse of math symbol\index{mathematisches Symbol}
% redeclarations\index{Umdeklarierung von Symbolen} can be confusing (even though
% it is well-defined) and is best avoided when possible.  Here are some good
% general guidelines you can follow.  Usually, you should declare all
% constants\index{Konstante} in the outermost block\index{äußerster Block},
% especially if they are general-purpose (such as the token \verb$A.$, meaning
% $\forall$ or "`for all"').  This will make them "`globally"' active (although
% as in the example above local redeclarations will temporarily make them
% inactive.)  Most or all variables\index{Variable}, on the other hand, could be
% declared in inner blocks, so that the token for them can be used later for a
% different type of variable or a constant.  (The names of the variables you
% choose are not used when you refer to an assertion\index{Behauptung} in a
% proof, whereas constants must match exactly.  A locally declared constant will
% not match a globally declared constant in a proof, even if they use the same
% token, because Metamath internally considers them to be different math
% symbols.)  To avoid confusion, you should generally avoid redeclaring active
% variables.  If you must redeclare them, do so at the beginning of a block.
% The temporary declaration of constants in inner blocks might be occasionally
% appropriate when you make use of a temporary definition to prove lemmas
% leading to a main result that does not make direct use of the definition.
% This way, you will not clutter up your database with a large number of
% seldom-used global constant symbols.  You might want to note that while
% inactive constants may not appear directly in an assertion (a \texttt{\$a}\index{\texttt{\$a}
% statement} or \texttt{\$p}\index{\texttt{\$p}-Anweisung}
% statement), they may be indirectly used in the proof of a \texttt{\$p} statement
% so long as they do not appear in the final math symbol sequence constructed by
% the proof.  In the end, you will have to use your best judgment, taking into
% account standard mathematical usage of the symbols as well as consideration
% for the reader of your work.
%
% \subsubsection{Reuse of Labels}\index{reuse of labels}\index{Label}
%
% The \texttt{\$e}\index{\texttt{\$e}-Anweisung}, \texttt{\$f}\index{\texttt{\$f}
% statement}, \texttt{\$a}\index{\texttt{\$a}-Anweisung}, and
% \texttt{\$p}\index{\texttt{\$p}
% statement} statement types require labels, which allow them to be
% referenced later inside proofs.  A label is considered {\bf
% active}\index{active label} when the statement it is associated with is
% active.  The token\index{Token} for a label may be reused
% (redeclared)\index{redeclaration of labels} provided that it is not being used
% for a currently active label.  (Unlike the tokens for math symbols, active
% label tokens may not be redeclared in an inner scope.)  Note that the labels
% of \texttt{\$a} and \texttt{\$p} statements can never be reused after these
% statements appear, because these statements remain active through the end of
% the database.
%
% You might find the reuse of labels a convenient way to have standard names for
% temporary hypotheses, such as \texttt{h1}, \texttt{h2}, etc.  This way you don't have
% to invent unique names for each of them, and in some cases it may be less
% confusing to the reader (although in other cases it might be more confusing, if
% the hypothesis is located far away from the assertion that uses
% it).\footnote{The current implementation requires that all labels, even
% inactive ones, be unique.}

\subsubsection{Zurück zu Frames}\index{Frame- und Gültigkeitsbereichanweisungen}

Nachdem wir nun die Gültigkeitsbereiche behandelt haben, werden wir uns ansehen, wie eine beliebige Metamath-Datenbasis in die einfache Folge von erweiterten Frames konvertiert werden kann, die auf S.~\pageref{framelist} beschrieben ist.  Dies ist auch die Art und Weise, wie Metamath die Datenbasis intern speichert, wenn es die Datenbasisquelle einliest.\label{frameconvert} Die Methode ist einfach.  Zunächst werden alle Konstanten- und Variablendeklarationen (\texttt{\$c} und \texttt{\$v}) in der Datenbasis gesammelt, wobei doppelte Deklarationen derselben Variable in verschiedenen Bereichen ignoriert werden.  Dann setzen wir unsere gesammelten \texttt{\$c}- und \texttt{\$v}-Deklarationen an den Anfang der Datenbasis, so dass ihr Gültigkeitsbereich die gesamte Datenbasis ist.  Als Nächstes bestimmen wir für jede Behauptung in der Datenbasis ihren Frame und ihren erweiterten Frame.  Der erweiterte Frame ist einfach die Sammlung der \texttt{\$f}, \texttt{\$e} und \texttt{\$d}-Anweisungen, die aktiv sind.  Der Frame ist der erweiterte Frame, aus dem alle optionalen Hypothesen entfernt wurden.

Eine äquivalente Formulierung ist, dass der erweiterte Frame einer Behauptung die Sammlung aller \texttt{\$f}-, \texttt{\$e}- und \texttt{\$d}-Anweisungen ist, deren Gültigkeitsbereich die Behauptung umfasst. Die Anweisungen \texttt{\$f} und \texttt{\$e} treten in der Reihenfolge auf, in der sie angegeben werden (die Reihenfolge ist für \texttt{\$d}-Anweisungen irrelevant).

%c%, renaming any
%c%redeclared variables as needed so that all of them have unique names.  (The
%c%exact renaming convention is unimportant.  You might imagine renaming
%c%different declarations of math symbol \texttt{a} as \texttt{a\$1}, \texttt{a\$2}, etc.\
%c%which would prevent any conflicts since \texttt{\$} is not a legal character in a
%c%math symbol token.)

\section{Die Anatomie eines Beweises} \label{proof}
\index{Beweis!Metamath, Beschreibung von}

Jede beweisbare Behauptung (\texttt{\$p}\index{\texttt{\$p}-Anweisung}-Anweisung) in einer Datenbasis muss einen {\bf Beweis}\index{Beweis} enthalten.  Der Beweis befindet sich zwischen den Schlüsselwörtern \texttt{\$=}\index{\texttt{\$=} Schlüsselwort} und \texttt{\$.}\ in der \texttt{\$p}-Anweisung.

In der Metamath-Basissprache\index{grundlegende Sprache} ist ein Beweis eine Folge von Anweisungs-Labels.  Diese Label-Sequenz\index{Label-Sequenz} dient als eine Menge von Anweisungen, die das Metamath-Programm verwendet, um eine Reihe von mathematischen Symbolsequenzen zu konstruieren.  Die Konstruktion muss letztlich zu der Foilge mathematischer Symbole führen, die zwischen den Schlüsselwörtern \texttt{\$p}\index{\texttt{\$p}-Anweisung} und \texttt{\$=}\index{\texttt{\$=} Schlüsselwort} der \texttt{\$p}-Anweisung enthalten ist.  Andernfalls wird das Metamath-Programm beim Verifizieren des Beweises diesen als falsch betrachten und Sie mit einer entsprechenden Fehlermeldung darauf hinweisen.\footnote{Um das Laden zu beschleunigen, verifiziert das Metamath-Programm Beweise nicht automatisch, wenn Sie mit \texttt{read} eine Datenbasis einlesen, es sei denn, Sie verwenden die Option \texttt{/verify}.  Nachdem eine Datenbasis eingelesen wurde, können Sie den Befehl \texttt{verify proof *} verwenden, um Beweise zu verifizieren.}\index{\texttt{verify proof}-Befehl} Jedes Label in einem Beweis {\bf referenziert} seine entsprechende Anweisung\index{Label-Referenz}.

Zu jeder Behauptung \index{Behauptung} (\texttt{\$p}- oder \texttt{\$a}-\index{\texttt{\$a}-Anweisung}Anweisung) gehört eine Reihe von Hypothesen (\texttt{\$f}-\index{\texttt{\$f}-Anweisung} oder \texttt{\$e}-\index{\texttt{\$e}-Anweisung}Anweisungen), die in Bezug auf diese Behauptung aktiv sind.  Einige sind obligatorisch, die anderen sind optional.  Sie sollten sich diese Konzepte bei Bedarf nochmals in Erinnerung rufen.

Jedes Label\index{Label} in einem Beweis muss entweder das Label einer vorhergehenden Behauptung (\texttt{\$a}-\index{\texttt{\$a}-Anweisung} oder \texttt{\$p}-\index{\texttt{\$p}-Anweisung}Anweisung) oder das Label einer aktiven Hypothese (\texttt{\$e}- oder \texttt{\$f}\index{\texttt{\$f}-Anweisung}-Anweisung) der \texttt{\$p}-Anweisung sein, die den Beweis enthält.  Labels für Hypothesen können sowohl auf die obligatorischen als auch auf die optionalen Hypothesen der Anweisung \texttt{\$p} verweisen.

Die Label-Sequenz in einem Beweis spezifiziert eine Konstruktion in {\bf umgekehrter polnischer Notation}\index{umgekehrte polnische Notation (RPN)} (RPN).  Sie sind vielleicht mit RPN vertraut, wenn Sie ältere Hewlett-Packard- oder ähnliche Taschenrechner benutzt haben. In der Analogie zum Taschenrechner ist ein Label einer Hypothese\index{Hypothesenlabel} wie eine Zahl und ein Label für eine Behauptung\index{Behauptungslabel} wie eine Operation (genauer gesagt, eine $n$-äre Operation, wenn die Behauptung $n$ \texttt{\$e}-Hypothesen hat) zu verstehen. Auf einem RPN-Rechner nimmt eine Operation eine oder mehrere vorherige Zahlen in einer Eingabefolge, führt eine Berechnung mit ihnen durch und ersetzt diese Zahlen und sich selbst durch das Ergebnis der Berechnung.  Zum Beispiel ergibt die Eingabefolge $2,3,+$ auf einem RPN-Rechner $5$, und die Eingabefolge $2,3,5,{\times},+$ ergibt $2,15,+$, was $17$ letztendlich ergibt.

Zum Verständnis der RPN-Verarbeitung gehört das Konzept eines {\bf Stapels}\index{Stapel}\index{RPN-Stapel}, den man sich als eine Reihe von temporären Speicherplätzen vorstellen kann, die Zwischenergebnisse enthalten.  Wenn Metamath auf ein Label für eine Hypothese stößt, wird die mathematische Symbolfolge der Hypothese auf den Stapel gelegt oder {bf geschoben}\index{schieben}.  Wenn Metamath auf ein Label für eine Behauptung stößt, ordnet es die obersten Stapel-Einträge den {\em obligatorischen} Hypothesen\index{obligatorische Hypothese} der Behauptung zu, und zwar in der Reihenfolge, in der der oberste Stapel-Eintrag der letzten obligatorischen Hypothese der Behauptung zugeordnet ist.  Anschließend wird ermittelt, welche Substitutionen\index{Substitution!Variable}\index{Variablensubstitution} in den Variablen der obligatorischen Hypothesen der Behauptung vorgenommen werden müssen, damit sie mit den zugehörigen Stapel-Einträgen übereinstimmen.  Dann werden die gleichen Substitutionen in der Behauptung selbst vorgenommen.  Schließlich entfernt Metamath die übereinstimmenden Hypothesen vom Stapel und schiebt die substituierte Behauptung auf den Stapel.

Für den Zweck des Abgleichs der obligatorischen Hypothese mit den obersten Stapel-Einträgen ist es unerheblich, ob eine Hypothese eine \texttt{\$e}- oder \texttt{\$f}-Anweisung ist.  Wichtig ist nur, dass es eine Menge von Substitutionen\footnote{In der Metamath-Spezifikation (Abschnitt~\ref{spec}) wird der Begriff "`Substitution"' im Singular verwendet, um sich auf die Menge der Substitutionen zu beziehen, über die wir hier sprechen.} gibt, die eine Übereinstimmung ermöglichen (und wenn nicht, wird der Beweisverifizierer eine Fehlermeldung ausgeben).  Die Metamath-Sprache ist so spezifiziert, dass, wenn eine Menge von Substitutionen existiert, diese eindeutig ist. Insbesondere die Anforderung, dass jede Variable einen Typ hat, der mit einer Anweisung \texttt{\$f} spezifiziert ist, stellt die Eindeutigkeit sicher.

Wir werden dies anhand eines Beispiels veranschaulichen. Betrachten Sie die folgende Metamath-Quelldatei:
\begin{verbatim}
$c ( ) -> wff $.
$v p q r s $.
wp $f wff p $.
wq $f wff q $.
wr $f wff r $.
ws $f wff s $.
w2 $a wff ( p -> q ) $.
wnew $p wff ( s -> ( r -> p ) ) $= ws wr wp w2 w2 $.
\end{verbatim}
Dieses Metamath-Beispiel zeigt die Definition und den Beweis (d.h. die Konstruktion) einer wohlgeformten Formel (wff)\index{wohlgeformte Formel (wff)} in der Aussagenlogik.  (Vielleicht möchten Sie dieses Beispiel in eine Datei eingeben, um mit dem Metamath-Programm zu experimentieren.)  Die ersten beiden Anweisungen deklarieren (führen die Namen ein von) vier Konstanten und vier Variablen.  Die nächsten vier Anweisungen spezifizieren die Variablentypen, nämlich dass jede Variable für eine wff steht.  Die Anweisung \texttt{w2} definiert (postuliert) eine Möglichkeit, um aus zwei gegebenen wffs \texttt{p} und \texttt{q} eine neue wff, \texttt{( p -> q )}, zu erzeugen. Die obligatorischen Hypothesen von \texttt{w2} sind \texttt{wp} und \texttt{wq}. Die Anweisung \texttt{wnew} behauptet, dass \texttt{( s -> ( r -> p ) )} eine wff ist, wenn drei wff \texttt{s}, \texttt{r} und \texttt{p} gegeben sind.  Genauer gesagt behauptet \texttt{wnew}, dass die Folge von zehn Symbolen \texttt{wff ( s -> ( r -> p ) )} aus früheren Behauptungen und den Hypothesen von \texttt{wnew} beweisbar ist.  Metamath weiß nicht oder kümmert sich nicht darum, was eine wff ist, und soweit es sie betrifft, ist der Typcode \texttt{wff} einfach ein beliebiges konstantes Symbol in einer mathematischen Symbolfolge.  Die obligatorischen Hypothesen von \texttt{wnew} sind \texttt{wp}, \texttt{wr}, und \texttt{ws}; \texttt{wq} ist eine optionale Hypothese.  In unserem speziellen Beweis wird die optionale Hypothese nicht referenziert, aber im Allgemeinen könnte jede Kombination von aktiven (d.h. optionalen und obligatorischen) Hypothesen referenziert werden.  Der Beweis der Anweisung \texttt{wnew} ist die Folge von fünf Labels, die mit \texttt{ws} (Schritt~1) beginnt und mit \texttt{w2} (Schritt~5) endet.

Wenn Metamath den Beweis verifiziert, scannt es den Beweis von links nach rechts.  Wir werden untersuchen, was bei jedem Schritt des Beweises passiert.  Der Stapel ist zu Beginn leer.  In Schritt 1 sucht Metamath nach dem Label \texttt{ws} und stellt fest, dass es sich um eine Hypothese handelt, also schiebt es die Symbolfolge der Anweisung \texttt{ws} auf den Stapel:

\begin{center}\begin{tabular}{|l|l|}\hline
{Stapelposition} & {Inhalt} \\ \hline \hline
1 & \texttt{wff s} \\ \hline
\end{tabular}\end{center}

Metamath sieht, dass die Labels \texttt{wr} und \texttt{wp} in den Schritten~2 und 3 ebenfalls Hypothesen sind, also schiebt es sie auf den Stapel.  Nach Schritt~3 sieht der Stapel wie folgt aus:

\begin{center}\begin{tabular}{|l|l|}\hline
{Stapelposition} & {Inhalt} \\ \hline \hline
3 & \texttt{wff p} \\ \hline
2 & \texttt{wff r} \\ \hline
1 & \texttt{wff s} \\ \hline
\end{tabular}\end{center}

In Schritt 4 sieht Metamath, dass das Label \texttt{w2} eine Behauptung ist, also muss es etwas verarbeiten.  Zunächst werden die obligatorischen Hypothesen von \texttt{w2}, nämlich \texttt{wp} und \texttt{wq}, mit den Stapelpositionen~2 und 3 verknüpft, {\em in dieser Reihenfolge}. Metamath stellt fest, dass die einzige Möglichkeit, die Hypothese \texttt{wp} mit dem Inhalt auf Stapelposition~2 und \texttt{wq} mit dem Inhalt auf Stapelposition 3 übereinstimmen zu lassen, darin besteht, die Variable \texttt{p} durch \texttt{r} und \texttt{q} durch \texttt{p} zu ersetzen.  Metamath nimmt diese Substitutionen in \texttt{w2} vor und erhält die Symbolfolge \texttt{wff ( r -> p )}.  Es entfernt die Hypothesen von den Stapelpositionen 2 und 3 und legt das Ergebnis auf der Stapelposition 2 ab:

\begin{center}\begin{tabular}{|l|l|}\hline
{Stapelposition} & {Inhalt} \\ \hline \hline
2 & \texttt{wff ( r -> p )} \\ \hline
1 & \texttt{wff s} \\ \hline
\end{tabular}\end{center}

In Schritt 5 sieht Metamath, dass das Label \texttt{w2} eine Behauptung ist, also muss wieder eine Verarbeitung stattfinden.  Zunächst werden die obligatorischen Hypothesen von \texttt{w2}, d.h. \texttt{wp} und \texttt{wq}, den Stapelpositionen 1 und 2 zugeordnet. Metamath stellt fest, dass die einzige Möglichkeit, die Hypothesen zur Übereinstimmung zu bringen, darin besteht, die Variable \texttt{p} durch \texttt{s} und \texttt{q} durch \texttt{( r -> p )} zu ersetzen.  Metamath führt diese Substitutionen in \texttt{w2} durch und erhält die Symbolfolge \texttt{wff ( s -> ( r -> p ) )}.  Es entfernt die Stapelposition 1 und 2 und legt das Ergebnis auf Stapelposition~1 ab:

\begin{center}\begin{tabular}{|l|l|}\hline
{Stapelposition} & {Inhalt} \\ \hline \hline
1 & \texttt{wff ( s -> ( r -> p ) )} \\ \hline
\end{tabular}\end{center}

Nachdem Metamath die Verarbeitung des Beweises abgeschlossen hat\footnote{Anm. der Übersetzer: Da der Beweis kein weiteres Label enthält.}, prüft es, ob der Stapel genau ein Element enthält und ob dieses Element mit der mathematischen Symbolfolge in der \texttt{\$p}\index{\texttt{\$p}-Anweisung}-Anweisung übereinstimmt.  Dies ist bei unserem Beweis von \texttt{wnew} der Fall, also haben wir \texttt{wnew} erfolgreich bewiesen.  Wenn das Ergebnis davon abweicht, wird Metamath Sie mit einer Fehlermeldung informieren.  Eine Fehlermeldung wird auch ausgegeben, wenn der Stapel am Ende des Beweises mehr als einen Eintrag enthält, oder wenn der Stapel an irgendeiner Stelle des Beweises nicht genügend Einträge enthält, um alle obligatorischen Hypothesen\index{obligatorische Hypothese} einer Behauptung zu erfüllen.  Schließlich wird Metamath Sie mit einer Fehlermeldung benachrichtigen, wenn keine Substitution möglich ist, die die Hypothese einer referenzierten Behauptung mit den Stapeleinträgen übereinstimmen lässt.  Sie können mit den verschiedenen Arten von Fehlern, die Metamath erkennt, experimentieren, indem Sie einige kleine Änderungen am Beweis unseres Beispiels vornehmen. 

Die Notation von Metamath für Beweise wurde in erster Linie entwickelt, um Beweise in einer relativ kompakten Weise auszudrücken, nicht um sie für Menschen lesbar zu machen.  Metamath kann Beweise mit dem Befehl \texttt{show proof}\index{\texttt{show proof}-Befehl} auf verschiedene Arten anzeigen.  Die Option \texttt{/lemmon} zeigt sie in einem Format an, das leichter zu lesen ist, wenn die Beweise kurz sind, und Sie haben Beispiele für seine Verwendung in Kapitel~\ref{using} gesehen.  Bei längeren Beweisen ist es nützlich, die Baumstruktur des Beweises zu sehen.  Eine Baumstruktur wird angezeigt, wenn die Option \texttt{/lemmon} weggelassen wird.  Wenn Sie sich an diese Darstellung gewöhnt haben, werden Sie sie wahrscheinlich bequemer finden. Die Baumdarstellung des Beweises in unserem Beispiel sieht wie folgt aus:\label{treeproof}\index{Baumdarstellung eines Beweises}\index{Beweis!Baumdarstellung}

\begin{verbatim}
1     wp=ws    $f wff s
2        wp=wr    $f wff r
3        wq=wp    $f wff p
4     wq=w2    $a wff ( r -> p )
5  wnew=w2  $a wff ( s -> ( r -> p ) )
\end{verbatim}

Die Zahl links von jeder Zeile ist die Schrittnummer.  Es folgt eine {\bf Hypothesenzuordnung}\index{Hypothesenzuordnung}, bestehend aus zwei Labels\index{Label}, die durch \texttt{=} getrennt sind.  Links von \texttt{=} steht (außer im letzten Schritt) das Label einer Hypothese einer Behauptung, auf die später im Beweis Bezug genommen wird; hier sind die Schritte 1 und 4 die Hypothesenzuordnungen für die Behauptung \texttt{w2}, auf die in Schritt 5 Bezug genommen wird.  Eine Hypothesenzuordnung ist eine Stufe weiter eingerückt als die Behauptung, die sie verwendet. Dadurch ist es einfach, die entsprechende Behauptung zu finden, indem man sich direkt nach unten bewegt, bis die Einrückungsstufe um eine Stufe niedriger ist als die, von der aus man begonnen hat.  Rechts von jedem \texttt{=} befindet sich das Label des Beweisschritts für diesen Schritt.  Das Schlüsselwort der Anweisung im Label des Beweisschritts wird als nächstes aufgeführt, gefolgt vom Inhalt des obersten Stapeleintrags (dem neusten Stapeleintrag), wie er nach der Verarbeitung dieses Beweisschritts vorliegt.  Mit ein wenig Übung sollten Sie keine Probleme haben, Beweise in diesem Format zu lesen.

Metamath-Beweise beinhalten die Syntaxkonstruktion einer Formel. In der Standardmathematik wird diese Art der Konstruktion nicht als Teil des Beweises angesehen, und sie wird nach einer Weile sicherlich ziemlich langweilig. Daher zeigt der Befehl \texttt{show proof}\index{\texttt{show proof}-Befehl} standardmäßig die Syntaxkonstruktion nicht an.
Früher \textit{zeigte} der Befehl \texttt{show proof} die Syntaxkonstruktionen an, und man musste die Option 
\texttt{/essential} hinzufügen, um sie auszublenden, aber heute ist 
\texttt{/essential} der Standard, und man muss 
\texttt{/all} verwenden, um die Syntaxkonstruktionen zu sehen.

Bei der Überprüfung eines Beweises prüft Metamath, dass keine obligatorische \texttt{\$d}\index{\texttt{\$d}-Anweisung}\index{obligatorische \texttt{\$d}-Anweisung}-Anweisung einer Behauptung, auf die in einem Beweis verwiesen wird, verletzt wird, wenn Substitutionen\index{Substitution!Variable}\index{Variablensubstitution} an den Variablen in der Behauptung vorgenommen werden.  Für Einzelheiten siehe Abschnitt~\ref{spec4} oder \ref{dollard}.

\subsection{Das Konzept der Vereinheitlichung} \label{unify}

Wenn Metamath\index{Metamath} während der Überprüfung eines Beweises auf ein Label einer Behauptung\index{Behauptungslabel} stößt, assoziiert es die obligatorischen Hypothesen\index{obligatorische Hypothese} der Behauptung mit den obersten Einträgen des RPN-Stapels\index{Stapel}\index{RPN-Stapel}.  Metamath bestimmt dann, welche Substitutionen\index{Substitution!Variable}\index{Variablensubstitution} es an den Variablen in den obligatorischen Hypothesen der Behauptung vornehmen muss, damit diese Hypothesen mit ihren entsprechenden Stapeleinträgen korrespondieren.  Dieser Vorgang wird als {\bf Vereinheitlichung}\index{Vereinheitlichung} bezeichnet.  (Wir verwenden den Begriff "`Vereinheitlichung"' auch informell, um eine Reihe von Substitutionen zu bezeichnen, die sich aus diesem Prozess ergeben, wie in "`zwei Vereinheitlichungen sind möglich"').  Nachdem die Substitutionen vorgenommen wurden, werden die Hypothesen als {\bf vereinheitlicht} bezeichnet.

Ist eine solche Ersetzung nicht möglich, hält Metamath den Beweis für fehlerhaft und gibt eine Fehlermeldung aus.
% (deleted 3/10/07, per suggestion of Mel O'Cat:)
% The syntax of the
% Metamath language ensures that if a set of substitutions exists, it
% will be unique.

Der in der Literatur beschriebene allgemeine Algorithmus für eine Vereinheitlichung ist etwas komplex. Im Fall von Metamath ist er jedoch absichtlich einfach gehalten. Obligatorische Hypothesen müssen in der Reihenfolge ihres Auftretens auf den Beweisstapel geschoben werden. Darüber hinaus muss der Typ jeder Variablen mit einer \texttt{\$f}-Hypothese spezifiziert werden, bevor sie verwendet wird, und jede \texttt{\$f}-Hypothese muss die eingeschränkte Syntax eines Typcodes (einer Konstanten) gefolgt von einer Variablen aufweisen. Der Typcode in der \texttt{\$f}-Hypothese muss mit dem ersten Symbol des entsprechenden RPN-Stack-Eintrags übereinstimmen (der ebenfalls eine Konstante ist), so dass die einzige mögliche Übereinstimmung für die Variable in der \texttt{\$f}-Hypothese die Folge von Symbolen im Stapeleintrag nach der anfänglichen Konstanten ist. 

Im Beweis-Assistenten\index{Beweis-Assistent} wird ein allgemeinerer Vereinheitlichungsalgorithmus verwendet.  Während ein Beweis entwickelt wird, sind manchmal nicht genügend Informationen verfügbar, um eine eindeutige Vereinheitlichung zu bestimmen.  In diesem Fall bittet Metamath Sie, die richtige zu wählen.\index{mehrdeutige Vereinheitlichung}\index{Vereinheitlichung!mehrdeutig} 

\section{Erweiterungen der Metamath-Sprache}\index{erweiterte Sprache}

\subsection{Kommentare in der Metamath-Sprache}\label{comments}
\index{Auszeichnungsnotation}
\index{Kommentar!Auszeichnungsnotation}

Die Kommentarfunktion ermöglicht es Ihnen, den Inhalt einer Datenbasis mit Anmerkungen zu versehen.  Wie bei den meisten Computersprachen werden Kommentare bei der Interpretation des Inhalts der Datenbasis ignoriert. Kommentare fungieren beim Parsen einer Datenbasis effektiv als zusätzlicher Whitespace\index{Whitespace} zwischen den Token.

Ein Kommentar kann am Anfang, am Ende oder zwischen zwei beliebigen Token\index{Token} in einer Quelldatei stehen.

Kommentare haben die folgende Syntax:
\begin{center}
 \texttt{\$(} {\em text} \texttt{\$)}
\end{center}
Hier ist \index{\texttt{\$(} und \texttt{\$)} Hilfsschlüsselwörter}\index{Kommentar} {\em text} eine, möglicherweise leere, Zeichenkette aus beliebigen Zeichen des Metamath-Zeichensatzes (p.~\pageref{spec1chars}), mit der Ausnahme, dass die Zeichenketten \texttt{\$(} und \texttt{\$)} nicht in {\em text} vorkommen dürfen.  Daher sind verschachtelte Kommentare nicht erlaubt:\footnote{Computersprachen haben unterschiedliche Standards für verschachtelte Kommentare, und anstatt sich für einen zu entscheiden, ist es am einfachsten, sie überhaupt nicht zu erlauben, zumindest in der aktuellen Version (0.177) von Metamath\index{Metamath!Limitationen der Version 0.177}.} Metamath wird sich beschweren, wenn
\begin{center}
  \texttt{\$( This is a \$( nested \$) comment.\ \$)}
 \end{center} 
 in einer Datenbasis vorkommt. Um diese fehlende Verschachtelungsmöglichkeit zu kompensieren, ändere ich oft alle \texttt{\$}'s in \texttt{@}'s in Abschnitten des Metamath-Codes, die ich auskommentieren möchte.

Das Metamath-Programm unterstützt eine Reihe von Auszeichnungsmechanismen und Konventionen, um gut aussehende Ergebnisse in \LaTeX\ und {\sc html} zu generieren, wie unten beschrieben wird. Diese Auszeichnungsfunktionen haben ausschließlich damit zu tun, wie die Kommentare ausgegeben werden, und haben keinen Einfluss darauf, wie Metamath die Beweise in der Datenbasis verifiziert. Ihre unsachgemäße Verwendung kann zu einer falsch angezeigten Ausgabe führen, aber es werden keine Metamath-Fehlermeldungen bei den Befehlen \texttt{read} und \texttt{verify proof} ausgegeben.  (Der Befehl \texttt{write theorem\texttt{\char`\_}list} prüft jedoch als Nebeneffekt seiner {\sc html}-Generierung auf Auszeichnungsfehler.) Abschnitt~\ref{texout} enthält Anweisungen zur Erstellung von \LaTeX-Ausgaben, und Abschnitt~\ref{htmlout} enthält Anweisungen zur Erstellung von {\sc html}\index{HTML}-Ausgaben.

\subsubsection{Überschriften}\label{commentheadings}

Wenn unmittelbar nach dem \texttt{\$(} eine neue Zeile folgt, die mit einer Kennzeichnung für Überschriften beginnt, handelt es sich um eine Überschrift. Diese kann beginnen mit:

\begin{itemize}
 \item[] \texttt{\#\#\#\#} - Hauptteilüberschrift
 \item[] \texttt{\#*\#*} - Abschnittsüberschrift
 \item[] \texttt{=-=-} - Unterabschnittsüberschrift
 \item[] \texttt{-.-.} - Unterunterabschnittsüberschrift
\end{itemize}

Die auf die Zeile mit der Kennzeichnung folgende Zeile wird nach dem Abschneiden der Leerzeichen für den Eintrag im Inhaltsverzeichnisses verwendet. Die nächste Zeile sollte eine weitere Zeile mit einer (abschließenden) passenden Kennzeichnung sein. Jeglicher Text danach, aber vor dem abschließenden \texttt{\$}, wird in die Seite \texttt{mmtheoremsNNN.html} aufgenommen. Dies kann  z. B. eine ausführliche Beschreibung des Abschnitts sein.

Weitere Informationen erhalten Sie, wenn Sie \texttt{help write theorem\char`\_list} ausführen.

\subsubsection{Mathe-Modus}
\label{mathcomments}
\index{\texttt{`} innerhalb von Kommentaren}
\index{\texttt{\char`\~} innerhalb von Kommentaren}
\index{Mathe-Modus}

Innerhalb von Kommentaren wird eine Zeichenkette von Token\index{Token}, die von einfachen Anführungszeichen\index{einfache Anführungszeichen (\texttt{`})} (\texttt{`}) eingeschlossen ist, während des {\sc HTML}\index{HTML} oder \LaTeX-Ausgabesatzes in standardmäßige mathematische Symbole umgewandelt, \index{latex@{\LaTeX}} entsprechend den Informationen in der speziellen \texttt{\$t}\index{\texttt{\$t}-Anweisung}\index{Schriftsatzanweisung}  in der Datenbasis (siehe Abschnitt~\ref{tcomment} für Informationen über den Schriftsatzkommentar und Anhang~\ref{ASCII} für Beispiele seiner Ergebnisse).

Das erste einfache Anführungszeichen \index{einfache Anführungszeichen (\texttt{`})} \texttt{`} veranlasst den Ausgabeprozessor, in den {\bf Mathe-Modus}\index{Mathe-Modus} einzutreten, und der zweite verlässt ihn. In diesem Modus werden die auf \texttt{`} folgenden Zeichen als eine Folge von mathematischen Symbolen interpretiert, die durch Whitespace getrennt sind.  Die Zeichen werden in dem \texttt{\$t}-Kommentar \index{\texttt{\$t}-Anweisung}\index{Schriftsatzanweisung} gesucht und, wenn sie gefunden werden, durch die mathematischen Standardsymbole ersetzt, denen sie entsprechen, bevor sie in die Ausgabedatei geschrieben werden.  Werden sie nicht gefunden, wird das Symbol so ausgegeben, wie es ist, und es wird eine Warnung ausgegeben. Die Token müssen nicht in der Datenbasis aktiv sein, obwohl eine Warnung ausgegeben wird, wenn sie nicht mit den Anweisungen \texttt{\$c} oder \texttt{\$v} deklariert sind.

Zwei aufeinanderfolgende einfache Anführungszeichen \texttt{"`} werden als ein einziges einfaches Anführungszeichen ausgegeben (sowohl innerhalb als auch außerhalb des Mathe-Modus) und führen nicht dazu, dass der Ausgabeprozessor in den Mathe-Modus eintritt oder diesen verlässt.

Hier ist ein Beispiel für seine Anwendung\index{Pierces Axiom}:
\begin{center}
\texttt{\$( Pierce's axiom, ` ( ( ph -> ps ) -> ph ) -> ph ` ,\\
         is not very intuitive. \$)}
\end{center}
wird ausgegeben als
\begin{center}
   \texttt{\$(} Pierce's axiom, $((\varphi \rightarrow \psi)\rightarrow
\varphi)\rightarrow \varphi$, is not very intuitive. \texttt{\$)}
\end{center}

Beachten Sie, dass das mathematische Symbol\index{Token} von Whitespace\index{Whitespace} umgeben sein muss.
%, since there is no context that allows ambiguity to be
%resolved, as is the case with math symbol sequences in some of the Metamath
%statements.
Whitespace sollte auch die Begrenzungszeichen \texttt{`} umgeben.

Die Funktion "`Mathe-Modus"' bietet Ihnen auch eine schnelle und einfache Möglichkeit, Text mit mathematischen Symbolen zu erzeugen, unabhängig vom Verwendungszweck von Metamath.\index{Metamath!Verwendung als Mathe-Editor} Dazu erstellen Sie einfach Ihren Text mit einfachen Anführungszeichen, die Ihre Formeln umgeben, nachdem Sie sichergestellt haben, dass Ihre mathematischen Symbole auf \LaTeX-Symbole abgebildet sind, wie in Anhang~\ref{ASCII} beschrieben.  Es ist einfacher, wenn Sie mit einer Datenbasis mit vordefinierten Symbolen wie \texttt{set.mm} beginnen.  Ersetzen Sie einen vorhandenen Kommentar durch eine mathematische Zeichenkette in einfachen Anführungszeichen und setzen Sie dann die diesem Kommentar entsprechende Anweisung gemäß den Anweisungen des Befehls \texttt{help tex} im Metamath-Programm.  Sie werden dann wahrscheinlich die resultierende Datei mit einem Texteditor bearbeiten wollen, um sie genau auf Ihre Bedürfnisse abzustimmen.

\subsubsection{Label-Modus}\index{Label-Modus}

Außerhalb des Mathe-Modus zeigt eine Tilde\index{Tilde (\texttt{\char`\~})} \verb/~/ dem Ausgabeprozessor von Metamath\index{Metamath} an, dass das folgende Token\index{Token} (d.h. die Zeichen bis zum nächsten Whitespace\index{Whitespace}) ein Label oder eine URL darstellt. Dieser Formatierungsmodus wird als {\bf Label-Modus}\index{Label-Modus} bezeichnet. Wenn anstelle des Label-Modus tatsächlich eine Tilde ausgegeben werden soll (außerhalb des Mathe-Modus), verwenden Sie zwei Tilden in einer Reihe, um sie darzustellen.

Bei der Erzeugung einer \LaTeX-Ausgabedatei wird das folgende Token in der Schriftart \texttt{typewriter} formatiert und die Tilde entfernt, damit es sich vom restlichen Text abhebt. Diese Formatierung wird auf alle Zeichen nach der Tilde bis zum ersten Whitespace\index{Whitespace} angewendet. Es wird nicht geprüft, ob es sich bei dem Token um ein Label für eine Anweisung handelt oder nicht, und das Token muss nicht die korrekte Syntax für ein Label haben; es werden keine Fehlermeldungen ausgegeben.  Die einzige Auswirkung des Label-Modus auf die Ausgabe ist, dass für die Token, die in die Ausgabedatei \LaTeX\ eingefügt werden, eine Schreibmaschinenschrift verwendet wird.

Bei der Erzeugung von {\sc html} {\em müssen} die Token nach der Tilde eine URL (entweder http: oder https:) oder ein gültiges Label sein. Ist dies nicht der Fall, werden bei der Ausgabe Fehlermeldungen ausgegeben. Es wird ein Hyperlink zu dieser URL oder diesem Label erzeugt.

\subsubsection{Querverweis zum Literaturverzeichnis}\index{Zitierungen}%
\index{Querverweis zu Literaturangaben}

Querverweis zum Literaturverzeichnis werden bei der Erstellung von {\sc html} besonders behandelt, wenn sie speziell formatiert sind. Text in der Form \texttt{[}{\em author}\texttt{]} wird als Querverweis zum Literaturverzeichnis betrachtet. Siehe \texttt{help html} und \texttt{help write bibliography} im Metamath-Programm für weitere Informationen. 
% \index{\texttt{\char`\[}\ldots\texttt{]} innerhalb von Kommentaren}
Siehe auch Abschnitte~\ref{tcomment} und \ref{wrbib}.

Die Notation \texttt{[}{\em author}\texttt{]} erzeugt auch einen Eintrag in der Datei für das Literaturverzeichnis, die von \texttt{write bibliography} (Abschnitt~\ref{wrbib}) für {\sc HTML} erzeugt wird. Damit dies richtig funktioniert, muss der umgebende Kommentar wie folgt formatiert sein:
\begin{quote}
    {\em keyword} {\em label} {\em noise-word}
     \texttt{[}{\em author}\texttt{] p.} {\em number}
\end{quote}
zum Beispiel
\begin{verbatim}
     Theorem 5.2 von [Monk] S. 223
\end{verbatim}
Beim {\em keyword} wird nicht zwischen Groß- und Kleinschreibung unterschieden, und es muss einer der folgenden Begriffe verwendet werden\footnote{Anm. der Übersetzer: angepasst an Metamath - Version 0.198 7-Aug-2021}:
\begin{verbatim}    
     Axiom Chapter Claim Compare Conclusion Condition
     Conjecture Corollary Definition Equation Example
     Exercise Fact Figure Introduction Item Lemma Lemmas
     Line Lines Notation Note Observation Paragraph Part
     Postulate Problem Proof Property Proposition Remark
     Result Rule Scheme Scolia Scolion Section Statement
     Subsection Table Theorem
\end{verbatim}
Das optionale {\em label} kann aus mehr als einem Wort (nicht {\em keyword} und nicht {\em noise-word}) bestehen. Das optionale {\em noise-word} ist eines der folgenden:
\begin{verbatim}
     from in of on
\end{verbatim}
und wird bei der Erstellung der Datei für das Literaturverzeichnis ignoriert.  Der Befehl \texttt{write biblio\-graphy} führt eine Fehlerprüfung durch, um das obige Format zu überprüfen.\index{Fehlerprüfung}\footnote{Anm. der Übersetzer: oft soll ein Wort/Text in eckigen Klammern kein Querverweis sein. Dann sollte statt der einfachen öffnenden eckigen Klammer eine doppelte öffnende eckige Klammer verwendet werden, z.B. \texttt{[}\texttt{[}{\em wieder}\texttt{]}. Ansonsten wird ein Fehler gemeldet, wenn zwischen den eckigen Klammern kein Autor (aus dem Literaturverzeichnis) steht.}

\subsubsection{Klammerausdrücke}\label{parentheticals}

Das Ende eines Kommentars kann eine oder mehrere Klammerausdrücke enthalten, d.h. spezielle Beschreibungen, die in Klammern eingeschlossen sind. Das Metamath-Programm sucht nach bestimmten Klammerausdrücken und kann daraufhin Warnungen ausgeben. Diese sind:

\begin{itemize}
 \item[] \texttt{(Contributed by }
   \textit{NAME}\texttt{,} \textit{DATE}\texttt{.)} -
   dokumentiert den Namen des ursprünglichen Verfassers und das Erstellungsdatum.
 \item[] \texttt{(Revised by }
   \textit{NAME}\texttt{,} \textit{DATE}\texttt{.)} -
   dokumentiert den Namen des Mitwirkenden und das Erstellungsdatum, das zu einer signifikanten Überarbeitung geführt hat (nicht nur zu einer automatischen Minimierung oder Kürzung eines Beweises).
 \item[] \texttt{(Proof shortened by }
   \textit{NAME}\texttt{,} \textit{DATE}\texttt{.)} -
   dokumentiert den Namen und das Datum desjenigen, der eine erhebliche Verkürzung des Beweises erzielt hat (nicht nur eine automatische Minimierung).
 \item[] \texttt{(Proof modification is discouraged.)} -
   Hinweis, dass dieser Beweis normalerweise nicht geändert werden sollte.
 \item[] \texttt{(New usage is discouraged.)} -
   Hinweis, dass diese Behauptung normalerweise nicht verwendet werden sollte.
\end{itemize}

Das Datum \textit{DATE} muss in der Form (D)D-MMM-YYYY angegeben werden, wobei MMM die englische Abkürzung für den Monat ist.\footnote{Anm. der Übersetzer: Und (D)D der ein- oder zweistellige Tag im Monat ist.}

\subsubsection{Sonstige Textauszeichnungen}\label{othermarkup}
\index{Auszeichnungsnotation}

Neben dem Mathe-Modus und dem Label-Modus gibt es noch weitere Textauszeichnungen, um ansprechende Ergebnisse zu erzeugen:


 {\em space}\texttt{\char`\_}{\em non-space} (i.e.\ \texttt{\char`\_}
 
\begin{itemize}
 \item[]
         \texttt{\char`\_} (Unterstrich)\index{\texttt{\char`\_} innerhalb von Kommentaren} -
             Kursive Darstellung von Text ab {\em space}\texttt{\char`\_}{\em non-space} (d.h. \texttt{\char`\_} mit einem Leerzeichen davor und einem Nicht-Leerzeichen danach) bis zum nächsten {\em non-space}\texttt{\char`\_}{\em space}.  Normale Interpunktion (z.B. ein nachgestelltes Komma oder ein Punkt) wird bei der Bestimmung von {\em space} ignoriert.
 \item[]
         \texttt{\char`\_} (Unterstrich) - {\em
         non-space}\texttt{\char`\_}{\em non-space-string}, wobei
          {\em non-space-string} eine Zeichenkette aus Nicht-Leerzeichen ist, stellt {\em non-space-string} tiefer.
 \item[]
         \texttt{<HTML>}...\texttt{</HTML>} - konvertiert nicht "`\texttt{<}"' und "`\texttt{>}"' im enthaltenden Text, wenn {\sc HTML} generiert wird. Ansonsten wird die Textauszeichnungen normal verarbeitet. Dies ermöglicht das direkte Einfügen von {\sc html}-Befehlen.
 \item[]
       "`\texttt{\&}ref\texttt{;}"' - fügt eine {\sc HTML}-Zeichenreferenz ein. So können Sie beliebige Unicode-Zeichen einfügen (z. B. Zeichen mit Akzent).  Derzeit nur direkt unterstützt, wenn {\sc HTML} erzeugt wird.
\end{itemize}

Es wird empfohlen, alle \texttt{\char`\~}- und \texttt{`}-Token im Kommentar mit Leerzeichen zu umgeben und ein Leerzeichen an das einem \texttt{\char`\~}-Token folgenden {\em Label} anzuhängen.  Dadurch werden globale Ersetzungen zum Ändern von Labels und Symbolnamen viel einfacher, und auch die Gefahr von Mehrdeutigkeiten wird in Zukunft ausgeschlossen.  Leerzeichen um diese Zeichen herum werden in der endgültigen Ausgabe automatisch entfernt, um den normalen Interpunktionsregeln zu entsprechen; zum Beispiel wird ein Leerzeichen zwischen einem nachgestellten \texttt{`} und einer linken Klammer entfernt.

Eine gute Möglichkeit, sich mit den Textauszeichnungen vertraut zu machen, ist die Betrachtung der umfangreichen Beispiele in der Datenbasis \texttt{set.mm}.

\subsection{Der Schriftsatz-Kommentar (\texttt{\$t})}\label{tcomment}

Der Schriftsatzkommentar \texttt{\$t} in der Datenbasisdatei liefert die erforderlichen Informationen, um gut aussehende Ausgaben zu erzeugen. Er bietet \LaTeX- und {\sc html}-Definitionen für mathematische Symbole sowie Unterstützung und Anpassungen bei der Generierung von Web-Seite. Wenn Sie ein neues Token zu einer Datenbasis hinzufügen und später eine Ausgabe in \LaTeX\ oder {\sc HTML} erstellen möchten, dann sollten Sie auch die \texttt{\$t}-Kommentarinformationen aktualisieren. In der Datenbasisdatei \texttt{set.mm}\index{Mengenlehre-Datenbasis (\texttt{set.mm})} finden Sie ein ausführliches Beispiel für einen \texttt{\$t}-Kommentar, das viele der unten beschriebenen Funktionen verdeutlicht.

Programme, die keine gut aussehenden Ausgaben für eine Präsentation erzeugen müssen, wie z. B. Programme, die nur Metamath-Datenbasen überprüfen, können Schriftsatzkommentare vollständig ignorieren und sie einfach als normale Kommentare behandeln. Selbst das Metamath-Programm konsultiert die \texttt{\$t}-Kommentarinformationen nur dann, wenn es eine gesetzte Ausgabe in \LaTeX\ oder {\sc HTML} erzeugen muss (z. B. wenn Sie eine \LaTeX-Ausgabedatei mit dem Befehl \texttt{open tex} öffnen).

Wir werden zunächst die Syntax von Schriftsatzkommentaren besprechen und dann kurz darauf eingehen, wie diese innerhalb des Metamath-Programms verwendet werden können.

\subsubsection{Übersicht über die Syntax von Schriftsatzkommentaren}

Der Schriftsatzkommentar wird durch das Token \texttt{\$t}\index{\texttt{\$t}-Anweisung}\index{Schriftsatzanweisung} im Kommentar identifiziert, und der Schriftsatzkommentar endet an dem passenden \texttt{\$)}:
\[
  \mbox{\tt \$(\ }
  \mbox{\tt \$t\ }
  \underbrace{
    \mbox{\tt \ \ \ \ \ \ \ \ \ \ \ }
    \cdots
    \mbox{\tt \ \ \ \ \ \ \ \ \ \ \ }
  }_{\mbox{hier befinden sich Schriftsatzdefinitionen}}
  \mbox{\tt \ \$)}
\]

Zwischen dem \texttt{\$(}, mit dem der Kommentar beginnt, und dem Symbol \texttt{\$t} muss ein Whitespace enthalten sein, und zwar ausschließlich, und auf \texttt{\$t} muss wieder ein Whitespace folgen (siehe Abschnitt \ref{whitespace} zur Definition von Whitespace). Der Schriftsatzkommentar wird bis zum Kommentarende-Token \texttt{\$)} fortgesetzt (dem ein Whitespace vorausgehen muss).

In der Version 0.177\index{Metamath!Limitationen der Version 0.177} des Metamath-Programms darf es nur einen \texttt{\$t}-Kommentar in einer Datenbasis geben.  Diese Einschränkung kann in Zukunft aufgehoben werden, um mehrere \texttt{\$t}-Kommentare in einer Datenbasis zu ermöglichen.

Zwischen dem Symbol \texttt{\$t} (und dem folgenden Whitespace) und dem Kommentarende-Token \texttt{\$)} (und dem vorangehenden Whitespace) befindet sich eine Folge von einer oder mehreren Schriftsatzdefinitionen, wobei jede Definition die Form \textit{definition-type arg arg ... ;} hat. Jeder der null oder mehr \textit{arg}-Werte kann entweder ein Schriftsatzdatenwert oder ein Schlüsselwort sein (welche Schlüsselwörter wo erlaubt sind, hängt vom spezifischen \textit{definition-type} ab). Der \textit{definition-type} und jedes Argument \textit{arg} werden durch einen Whitespace getrennt. Jede Definition endet mit einem Semikolon; ein Whitespace ist vor dem abschließenden Semikolon einer Definition nicht erforderlich. Jede Definition sollte in einer neuen Zeile beginnen.\footnote{Diese Einschränkung der aktuellen Version von Metamath (0.177)\index{Metamath!Limitationen der Version 0.177} wird vielleicht in einer zukünftigen Version entfernt, aber man sollte es aus Gründen der Lesbarkeit trotzdem tun.}

Als Beispiel für solch eine Schriftsatzdefinition definiert
\begin{center}
 \verb$latexdef "C_" as "\subseteq";$
\end{center}
für das Token \verb$C_$ das \LaTeX-Symbol $\subseteq$ (was "`Teilmenge"' bedeutet).

Schriftsatzdaten sind eine Folge von einer oder mehreren Zeichenkette in Anführungszeichen (bei mehreren Zeichenkette in Anführungszeichen werden sie durch \texttt{\char`\+} verbunden). Häufig wird eine einzelne Zeichenkette in Anführungszeichen verwendet, um Daten für eine Definition bereitzustellen, wobei entweder doppelte (\texttt{\char`\"}) oder einfache (\texttt{'}) Anführungszeichen verwendet werden. {\em Eine in Anführungszeichen eingeschlossene Zeichenkette darf jedoch keine Zeilenumbrüche enthalten.} Eine in Anführungszeichen gesetzte Zeichenkette kann ein Anführungszeichen enthalten, das mit den eingeschlossenen Anführungszeichen übereinstimmt, indem das Anführungszeichen zweimal wiederholt wird.  Hier sind einige Beispiele:
\\
\\
\begin{tabu}   { l l }
\textbf{Beispiel} & \textbf{Bedeutung} \\
\texttt{\char`\"a\char`\"\char`\"b\char`\"} & \texttt{a\char`\"b} \\
\texttt{'c''d'} & \texttt{c'd} \\
\texttt{\char`\"e''f\char`\"} & \texttt{e''f} \\
\texttt{'g\char`\"\char`\"h'} & \texttt{g\char`\"\char`\"h} \\
\end{tabu}
\\
\\
\\
Schließlich kann eine lange in Anführungszeichen gesetzte Zeichenkette in mehrere in Anführungszeichen gesetzte Zeichenketten aufgeteilt werden (die als Ganzes als eine einzige in Anführungszeichen gesetzte Zeichenkette betrachtet werden) und mit \texttt{\char`\+} verbunden werden. Sie können sogar mehrere Zeilen verwenden, solange sich am Ende jeder Zeile außer der letzten ein '+' befindet. Vor und nach \texttt{\char`\+} sollte ein Whitespace stehen. Also zum Beispiel,
\begin{center}
 \texttt{\char`\"ab\char`\"\ \char`\+\ \char`\"cd\char`\"
    \ \char`\+\ \\ 'ef'}
\end{center}
ist dasselbe wie
\begin{center}
 \texttt{\char`\"abcdef\char`\"}
\end{center}

Kommentare im {\sc c}-Stil \texttt{/*}\ldots\texttt{*/} werden ebenso unterstützt.

In der Praxis werden Sie oft Satzdefinitionen mit \texttt{latexdef}, \texttt{htmldef} und \texttt{althtmldef} wie unten beschrieben hinzufügen wollen, wenn Sie ein neues mathematisches Token ergänzen. Auf diese Weise werden sie alle auf dem neuesten Stand sein. Ob Sie alle drei Definitionen verwenden wollen oder nicht, hängt natürlich davon ab, wie die Datenbasis verwendet werden soll.

Im Folgenden werden die verschiedenen möglichen \textit{definition-type}-Optio\-nen erörtert. Wir zeigen die Daten in doppelten Anführungszeichen (in der Praxis können sie auch in einfachen Anführungszeichen stehen und/oder eine durch \texttt{+}s verbundene Sequenz sein). Wir werden spezifische Namen für die \textit{data} verwenden, um zu verdeutlichen, wofür die Daten verwendet werden, z. B. {\em Mathe-Token} (für ein Metamath-Mathematik-Token), {\em Latex-string} (für eine Zeichenkette, die in einen \LaTeX-Text eingefügt werden soll), {\em {\sc html}-code} (für {\sc html}-Code) und {\em filename} (für einen Dateinamen).

\subsubsection{Schriftsatzkommentar - \LaTeX}

Die Syntax für eine \LaTeX-Definition lautet:
\begin{center}
 \texttt{latexdef \char`\"}{\em math-token}\texttt{\char`\"\ as \char`\"}{\em latex-string}\texttt{\char`\";}
\end{center}
\index{latex definitions@\LaTeX\ Definitionen}%
\index{\texttt{latexdef}-Anweisung}

{\em token-string} und {\em latex-string} sind die Daten (Zeichenketten) für das Token bzw. die \LaTeX-Definition des Tokens.

Diese \LaTeX-Definitionen werden vom Metamath-Programm verwendet, wenn es mit dem Befehl \texttt{write tex} eine \LaTeX-Ausgabe erzeugen soll.

\subsubsection{Schriftsatzkommentar - {\sc html}}

Die wichtigsten Arten von {\sc HTML}-Definitionen haben die folgende Syntax:

\vskip 1ex
    \texttt{htmldef \char`\"}{\em math-token}\texttt{\char`\"\ as \char`\"}{\em
    {\sc html}-code}\texttt{\char`\";}\index{\texttt{htmldef}-Anweisung}

    \texttt{althtmldef \char`\"}{\em math-token}\texttt{\char`\"\ as \char`\"}{\em
{\sc html}-code}\texttt{\char`\";}\index{\texttt{althtmldef}-Anweisung}

\vskip 1ex

Beachten Sie, dass es in {\sc HTML} zwei mögliche Definitionen für mathematische Token gibt. Diese Funktionalität ist nützlich, wenn eine alternative Darstellung von Symbolen gewünscht wird, zum Beispiel eine, die Unicode-Entities verwendet, und eine andere, die {\sc gif}-Bilder verwendet. 

Es gibt viele andere Schrifsatzdefinitionen, die {\sc HTML} steuern können. Dazu gehören:

\vskip 1ex

    \texttt{htmltitle \char`\"}{\em {\sc html}-code}\texttt{\char`\";}%
\index{\texttt{htmltitle}-Anweisung}

    \texttt{htmlhome \char`\"}{\em {\sc html}-code}\texttt{\char`\";}%
\index{\texttt{htmlhome}-Anweisung}

    \texttt{htmlvarcolor \char`\"}{\em {\sc html}-code}\texttt{\char`\";}%
\index{\texttt{htmlvarcolor}-Anweisung}

    \texttt{htmlbibliography \char`\"}{\em filename}\texttt{\char`\";}%
\index{\texttt{htmlbibliography}-Anweisung}

\vskip 1ex

\noindent Der \texttt{htmltitle} ist der {\sc html} Code für einen allgemeinen Titel, wie z.B. "`Metamath Proof Explorer"'.  Das Feld \texttt{htmlhome} ist der Code für einen Link zurück zur Startseite.  \texttt{htmlvarcolor} ist der Code für einen Farbschlüssel, der am unteren Rand jedes Beweises erscheint.  Die durch {\em filename} angegebene Datei ist eine {\sc html}-Datei, die ein \texttt{<A NAME=}\ldots\texttt{>}-Tag für jeden Verweis in das Literaturverzeichnis in den Datenbasiskommentaren enthäten sollte.  Wenn zum Beispiel \texttt{[Monk]}\index{\texttt{\char`\[}\ldots\texttt{]} innerhalb von Kommentaren} im Kommentar für ein Theorem vorkommt, muss \texttt{<A NAME='Monk'>} in der Datei vorhanden sein; andernfalls wird eine Warnmeldung ausgegeben.

Mit \texttt{htmldef} und  \texttt{althtmldef} verbunden sind die Anweisungen

\vskip 1ex

    \texttt{htmldir \char`\"}{\em
      directoryname}\texttt{\char`\";}\index{\texttt{htmldir}-Anweisung}

    \texttt{althtmldir \char`\"}{\em
     directoryname}\texttt{\char`\";}\index{\texttt{althtmldir}-Anweisung}

\vskip 1ex
\noindent geben die Verzeichnisse der {\sc gif}- bzw. Unicode-Versionen an; ihr Zweck ist es, Querverbindungen zwischen den beiden Versionen in den erzeugten Webseiten herzustellen.

Wenn zwei verschiedene Arten von Seiten aus einer einzigen Datenbasis erzeugt werden müssen, wie z.B. der Hilbert Space Explorer, der den Metamath Proof Explorer erweitert, können "`erweiterte"' Variablen in dem \texttt{\$t}-Kommentar deklariert werden:
\vskip 1ex

    \texttt{exthtmltitle \char`\"}{\em {\sc html}-code}\texttt{\char`\";}%
\index{\texttt{exthtmltitle}-Anweisung}

    \texttt{exthtmlhome \char`\"}{\em {\sc html}-code}\texttt{\char`\";}%
\index{\texttt{exthtmlhome}-Anweisung}

    \texttt{exthtmlbibliography \char`\"}{\em filename}\texttt{\char`\";}%
\index{\texttt{exthtmlbibliography}-Anweisung}

\vskip 1ex

\noindent Wenn diese deklariert werden, müssen Sie auch Folgendes deklarieren

\vskip 1ex

    \texttt{exthtmllabel \char`\"}{\em label}\texttt{\char`\";}%
\index{\texttt{exthtmllabel}-Anweisung}

\vskip 1ex 

\noindent welches die Anweisung der Datenbasis identifiziert, mit der der "`erweiterte"' Abschnitt der Datenbasis (in unserem Beispiel der Hilbert Space Explorer) beginnt.  Bei der Generierung von Webseiten für diese erste Anweisung und die nachfolgenden Anweisungen wird der {\sc html}-Code, der \texttt{exthtmltitle} und \texttt{exthtmlhome} zugewiesen ist, anstelle des Codes verwendet, der \texttt{htmltitle} bzw. \texttt{htmlhome} zugewiesen ist.

\begin{sloppy}
\subsection{Zusatzinformationskommentar (\texttt{\$j})} \label{jcomment}
\end{sloppy}

Der Zusatzinformationskommentar, auch bekannt als \texttt{\$j}-Kommentar\index{\texttt{\$j}-Kommentar}\index{zusätzliche Informationen-Kommentar}, bietet eine Möglichkeit, zusätzliche strukturierte Informationen hinzuzufügen, die optional von Systemen geparst werden können.

Der Zusatzinformationskommentar wird auf die gleiche Weise geparst wie der Schriftsatzkommentar (\texttt{\$t}) (siehe Abschnitt \ref{tcomment}). Das heißt, derZusatzinformationskommentar beginnt mit dem Token \texttt{\$j} innerhalb eines Kommentars und wird bis zum Kommentarschluss \texttt{\$)} fortgesetzt. Innerhalb eines Zusatzinformationskommentars befindet sich eine Folge von einem oder mehreren Befehlen der Form \texttt{command arg arg ... ;}, wobei jeder der null oder mehr \texttt{arg}-Werte entweder eine Zeichenkette in Anführungszeichen oder ein Schlüsselwort sein kann. Beachten Sie, dass jeder Befehl mit einem nicht in Anführungszeichen gesetztes Semikolon endet. Wenn ein Prüfprogramm einen Zusatzinformationskommentar parst, aber einen bestimmten Befehl nicht erkennt, muss er den Befehl überspringen, indem er das Ende des Befehls (ein nicht in Anführungszeichen gesetztes Semikolon) sucht.

Eine Datenbasis kann keine, eine oder mehr Zusatzinformationskommentare haben. Beachten Sie jedoch, dass ein Prüfprogramm diese Kommentare vollständig ignorieren oder nur bestimmte Befehle in einem Zusatzinformationskommentar verarbeiten kann. Der \texttt{mmj2} Verifier unterstützt viele Befehle in Zusatzinformationskommentaren. Wir empfehlen Systemen, die Zusatzinformationskommentare verarbeiten, sich abzustimmen, so dass sie denselben Befehl für denselben Effekt verwenden. 

Beispiele für Zusatzinformationskommentare mit verschiedenen Befehlen (aus der Datenbasis \texttt{set.mm}) sind: 

\begin{itemize}
   \item Definition der Syntax und der logischen Typcodes und Deklaration, dass unsere Grammatik eindeutig ist (überprüfbar mit dem KLR-Parser, mit Kompositionstiefe 5).
\begin{verbatim}
  $( $j
    syntax 'wff';
    syntax '|-' as 'wff';
    unambiguous 'klr 5';
  $)
\end{verbatim}

   \item Registrierung von $\lnot$ und $\rightarrow$ als primitive Ausdrücke (ohne Definitionen).
\begin{verbatim}
  $( $j primitive 'wn' 'wi'; $)
\end{verbatim}

   \item Für \texttt{df-bi} gibt es eine besondere Rechtfertigung.
\begin{verbatim}
  $( $j justification 'bijust' for 'df-bi'; $)
\end{verbatim}

   \item Registrierung von $\leftrightarrow$ als eine Gleichheit für seinen Typ (wff).
\begin{verbatim}
  $( $j
    equality 'wb' from 'biid' 'bicomi' 'bitri';
    definition 'dfbi1' for 'wb';
  $)
\end{verbatim}

   \item Theorem \texttt{notbii} ist das Kongruenzgesetz für die Negation.
\begin{verbatim}
  $( $j congruence 'notbii'; $)
\end{verbatim}

   \item Ergänzung von \texttt{setvar} als Typcode.
\begin{verbatim}
  $( $j syntax 'setvar'; $)
\end{verbatim}

   \item Registrierung von $=$ als Gleichheit für seinen Typ (\texttt{class}).
\begin{verbatim}
  $( $j equality 'wceq' from 'eqid' 'eqcomi' 'eqtri'; $)
\end{verbatim}

\end{itemize}


\subsection{Einbindung anderer Dateien in eine Metamath-Quelldatei} \label{include}
\index{\texttt{\$(} und \texttt{\$)} Hilfsschlüsselwörter}

Die Schlüsselwörter \texttt{\$[} und \texttt{\$]} spezifizieren eine einzubindende Datei\index{eingebundene Datei}\index{Dateieinbindung} an dieser Stelle in einer Metamath\index{Metamath}-Quelldatei\index{Quelldatei}.  Die Syntax für die Einbindung einer Datei lautet wie folgt:
\begin{center}
\texttt{\$[} {\em file-name} \texttt{\$]}
\end{center}

Der Dateiname {\em file-name} sollte ein einzelnes Token mit der gleichen Syntax wie ein mathematisches Symbol sein (d. h. alle 93 druckbaren Zeichen ohne Leerzeichen außer \texttt{\$} sind zulässig, vorbehaltlich der Dateinamensbeschränkungen Ihres Betriebssystems). Zwischen den Schlüsselwörtern \texttt{\$[} und \texttt{\$]} können Kommentare stehen.  Eingebundene Dateien können andere Dateien einbinden, die wiederum andere Dateien einbinden können, und so weiter.

Nehmen wir zum Beispiel an, Sie möchten die Datenbasis der Mengenlehre als Ausgangspunkt für Ihre eigene Theorie verwenden.  Die erste Zeile in Ihrer Datei könnte lauten: 
\begin{center} 
  \texttt{\$[ set.mm \$]} 
\end{center} 
Alle Informationen (Axiome, Theoreme usw.) in \texttt{set.mm} und alle Dateien, die {\em sie selbst wiederum} einschließt, stehen dann Ihnen zur Verfügung, damit Sie in Ihrer Datei darauf verweisen können. Dies ermöglicht eine modularen Aufbau Ihrer eigenen Arbeit. Ein Nachteil des Einbebindens von Dateien ist, dass Sie, wenn Sie den Namen eines Symbols oder das Label einer Anweisung ändern, auch daran denken müssen, alle Verweise in jeder Datei, die sie einbindet, zu aktualisieren.

Die Namenskonventionen für eingebundene Dateien entsprechen denen Ihres Betriebssystems.\footnote{Auf dem Macintosh, vor Mac OS X, wird ein Doppelpunkt verwendet, um Laufwerke- und Ordnernamen von Ihrem Dateinamen zu trennen.  Zum Beispiel bezieht sich {\em volume}\texttt{:}{\em file-name} auf das Stammverzeichnis, {\em volume}\texttt{:}{\em folder-name}\texttt{:}{\em file-name} auf einen Ordner im Stammverzeichnis, und {\em volume}\texttt{:}{\em folder-name}\texttt{:}\ldots\texttt{:}{\em file-name} auf einen Unterordner.  Ein einfacher {\em Dateiname} verweist auf eine Datei in dem Ordner, aus dem Sie die Metamath-Anwendung starten.  Unter Mac OS X und später wird das Metamath-Programm unter der Terminal-Anwendung ausgeführt, die den Unix-Namenskonventionen entspricht.}\index{Macintosh-Dateinamen}\index{Dateinamen!Macintosh}\label{includef} Um die Kompatibilität zwischen verschiedenen Betriebssystemen zu gewährleisten, sollten Sie die Dateinamen so einfach wie möglich halten.  Eine gute Konvention ist {\em file}\texttt{.mm}, wobei {\em file} aus acht Zeichen oder weniger in Kleinbuchstaben besteht. 

Es gibt keine Begrenzung für die Verschachtelungstiefe von eingebundenen Dateien.  Sie sollten jedoch beachten, dass, wenn zwei eingebundene Dateien selbst eine gemeinsame dritte Datei einbinden, diese nur beim {\em ersten} Verweis auf diese gemeinsame Datei eingelesen wird.  Dies ermöglicht es Ihnen, zwei oder mehr Dateien einzubinden, die auf einer gemeinsamen Ausgangsdatei aufbauen, ohne sich um Label- und Symbolkonflikte sorgen zu müssen, die auftreten würden, wenn die gemeinsame Datei mehr als einmal eingelesen würde.  (Wenn eine Datei sich selbst einbindet, wobei das natürlich nicht sinnvoll wäre, wird die Selbstreferenz ignoriert).  Dieses Vorgehen bedeutet jedoch auch, dass das Ergebnis möglicherweise nicht das ist, was Sie erwarten, wenn Sie versuchen, eine gemeinsame Datei in mehreren inneren Blöcken einzubinden, da nur die erste Referenz durch die eingebundene Datei ersetzt wird (im Gegensatz zur include-Anweisung in den meisten anderen Computersprachen).  Daher würden Sie normalerweise gemeinsame Dateien nur im äußersten Block einbinden\index{äußerster Block}. 

\subsection{Komprimiertes Beweisformat}\label{compressed1}\index{komprimierter Beweis}\index{Beweis!komprimiert}

Die in Abschnitt~\ref{proof} vorgestellte Beweisschreibweise wird als {\bf normaler Beweis}\index{normaler Beweis}\index{Beweis!normal} bezeichnet und ist im Prinzip ausreichend, um jeden Beweis vollständig zu beschreiben.  Beweise enthalten jedoch oft Schritte und Unterbeweise, die identisch sind.  Dies gilt insbesondere für typische Metamath\index{Metamath}-Anwendungen, da Metamath verlangt, dass die mathematische Symbolfolge (die in der Regel eine Formel enthält) bei jedem Schritt separat konstruiert, d. h. Stück für Stück aufgebaut wird. Daraus ergibt sich oft eine große Anzahl von Wiederholungen.  Das {\bf komprimierte Beweisformat} ermöglicht es Metamath, diese Redundanz zu nutzen, um Beweise zu verkürzen.

Die Spezifikation für das komprimierte Format der Beweise ist in Anhang~\ref{compressed} enthalten. 

Normalerweise brauchen Sie sich nicht mit den Details des komprimierten Beweisformats zu befassen, da das Programm Metamath eine bequeme Konvertierung vom normalen Format in das komprimierte Format ermöglicht und auch automatisch vom komprimierten Format in das normale Format konvertiert, wenn Beweise angezeigt werden. Die allgemeine Struktur des komprimierten Formats ist wie folgt:
\begin{center}
  \texttt{\$= ( } {\em label-list} \texttt{) } {\em compressed-proof\ }\ \texttt{\$.}
\end{center}
\index{\texttt{\$=} Schlüsselwort}

Die  erste Klammer \texttt{(} dient als Kennzeichen für Metamath, dass ein komprimierter Beweis folgt.  Die {\em label-list} enthält alle Anweisungen, auf die sich der Beweis bezieht, mit Ausnahme der obligatorischen Hypothesen\index{obligatorische Hypothese}.  Das {\em compressed-proof} ist eine kompakte Kodierung des Beweises unter Verwendung von Großbuchstaben und kann als eine große ganze Zahl zur Basis 26 angesehen werden.  Das Whitespace\index{Whitespace} innerhalb eines {\em compressed-proof} ist optional und wird ignoriert.

Es ist wichtig zu beachten, dass die Reihenfolge der obligatorischen Hypothesen der zu beweisenden Anweisung nicht geändert werden darf, wenn das komprimierte Beweisformat verwendet wird, da der Beweis sonst falsch wird.  Der Grund dafür ist, dass die obligatorischen Hypothesen im komprimierten Beweis nicht explizit erwähnt werden, um die Komprimierung effizienter zu gestalten. Wenn Sie die Reihenfolge der obligatorischen Hypothesen ändern möchten, müssen Sie den Beweis zunächst wieder in das normale Format konvertieren, indem Sie die Anweisung \texttt{save proof} \texttt{{\em statement} /normal}\index{\texttt{save proof}-Befehl} ausführen. Später können Sie mit dem Befehl \texttt{save proof} \texttt{{\em statement} /compressed} wieder in das komprimierte Format wechseln. 

Bei der Fehlerprüfung mit dem Befehl \texttt{verify proof} kann ein in einem komprimierten Beweis gefundener Fehler auf ein Zeichen in {\em compressed-proof} verweisen, das für Sie möglicherweise nicht sehr aussagekräftig ist.  Versuchen Sie in diesem Fall zuerst \texttt{save proof /normal} und führen Sie dann den Befehl \texttt{verify proof} erneut aus.  Im Allgemeinen ist es am besten sich zu vergewissern, dass ein Beweis korrekt ist, bevor man ihn im komprimierten Format speichert, da schwere Fehler mit geringerer Wahrscheinlichkeit wiederhergestellt werden können als im normalen Format. 

\subsection{Unbekannte Beweise oder Teilbeweise}\label{unknown}

In einem in Entwicklung befindlichen Beweis kann jeder Schritt oder Teilbeweis, der noch nicht bekannt ist, mit einem einzelnen \texttt{?} dargestellt werden.  Beim Parsen des Beweises wird bei einem \texttt{?}\index{\texttt{]}@\texttt{?}\ innerhalb von Beweisen} ein einzelner Eintrag auf den RPN-Stapel geschoben, als wäre es eine Hypothese.  Während der Entwicklung eines Beweises mit dem Beweis-Assistenten\index{Beweis-Assistent} kann ein teilweise entwickelter Beweis mit dem Befehl \texttt{save new{\char`\_}proof}\index{\texttt{save new{\char`\_}proof}-Befehl} gespeichert werden, und die \texttt{?}'s werden an die entsprechenden Stellen gesetzt. 

Für alle \texttt{\$p}\index{\texttt{\$p}-Anweisung}-Anweisungen müssen Beweise vorliegen, auch wenn sie völlig unbekannt sind.  Bevor Sie einen Beweis mit dem Beweis-Assistenten erstellen, sollten Sie einen völlig unbekannten Beweis wie folgt angeben: 
\begin{center}
  {\em label} \texttt{\$p} {\em statement} \texttt{\$= ?\ \$.}
\end{center}
\index{\texttt{\$=} Schlüsselwort}
\index{\texttt{]}@\texttt{?}\ innerhalb von Beweisen}

Der Befehl \texttt{verify proof}\index{\texttt{verify proof}-Befehl} prüft die bekannten Teile eines Teilbeweises auf Fehler, warnt Sie aber, dass die Anweisung nicht bewiesen ist. 

Beachten Sie, dass teilweise entwickelte Beweise auf Wunsch im komprimierten Format gespeichert werden können.  In diesem Fall sehen Sie einen oder mehrere \texttt{?}'s im Teil {\em compressed-proof}\index{komprimierter Beweis}\index{Beweis!komprimiert}.

\section{Axiome vs. Definitionen}\label{definitions}

Die Metamath \textit{zugrunde liegende} Sprache und das Metamath\index{Metamath}-Programm unterscheiden nicht zwischen Axiomen\index{Axiom} und Definitionen.\index{Definition} Die \texttt{\$a}\index{\texttt{\$a}-Anweisung}-Anweisung wird für beides verwendet.  Auf den ersten Blick mag dies seltsam erscheinen.  In den Augen vieler Mathematiker ist die Unterscheidung klar, sogar offensichtlich, und kaum eine Diskussion wert.  Eine Definition wird lediglich als eine Abkürzung betrachtet, die durch den Ausdruck, für den sie steht, ersetzt werden kann; wenn man dies jedoch nicht tut, sollte man sagen, dass ein Theorem\index{Theorem} eine Folge der Axiome {\em und} der Definitionen ist, die bei der Formulierung des Theorems verwendet werden \cite[S.~20]{Behnke}.\index{Behnke, H.}

\subsection{Was ist eine Definition?}

Was ist eine Definition?  In ihrer einfachsten Form führt eine Definition ein neues Symbol ein und liefert eine eindeutige Regel zur Umwandlung eines Ausdrucks, der das neue Symbol enthält, in einen Ausdruck ohne dieses Symbol.  Das Konzept einer "`zulässigen Definition"'\index{zulässige Definition}\index{Definition!zulässig} (im Gegensatz zu einer kreativen Definition)\index{kreative Definition}\index{Definition!kreativ}, auf das man sich in der Regel einigt, ist, dass (1) die Definition die Sprache nicht mächtiger machen sollte und (2) alle durch die Definition eingeführten Symbole aus der Sprache eliminierbar sein sollten \cite{Nemesszeghy}\index{Nemesszeghy, E. Z.}.  Mit anderen Worten, sie sind bloße typografische Bequemlichkeiten, die nicht zum System gehören und theoretisch überflüssig sind.  Dies mag offensichtlich erscheinen, aber in der Tat kann die Natur von Definitionen subtil sein und erfordert manchmal komplexe Metatheoreme, um zu beweisen, dass sie nicht kreativ sind. 

Eine konservativere Haltung vertrat der Logiker S. Le\'{s}niewski.\index{Le\'{s}niewski, S.} 
\begin{quote}
  Le\'{s}niewski betrachtet Definitionen als Thesen des Systems.  In dieser Hinsicht unterscheiden sie sich weder von den Axiomen noch von den Theoremen, d.h. von den Thesen, die dem System auf der Grundlage der Substitutionsregel oder der Abtrennregel [Modus ponens] hinzugefügt werden.  Sobald Definitionen als Thesen des Systems akzeptiert wurden, müssen sie als wahre Sätze in demselben Sinne betrachtet werden, in dem auch Axiome wahr sind
  \cite{Lejewski}.
\end{quote}\index{Lejewski, Czeslaw} 

Sehen wir uns einige einfache Beispiele für Definitionen in der Aussagenlogik an.  Betrachten wir die Definition des logischen {\sc or} (Disjunktion):\index{Disjunktion ($\vee$)} "`$P\vee Q$ steht für $\neg P \rightarrow Q$ (nicht $P$ impliziert $Q$)"'.  Eine Anweisung, die von dieser Definition Gebrauch macht, ist sehr leicht zu erkennen, weil sie das neue Symbol $\vee$ verwendet, das es vorher in der Sprache nicht gab.  Es ist leicht zu erkennen, dass sich aus dieser Definition keine neuen Theoreme der ursprünglichen Sprache ergeben. 

Betrachten wir nun eine Definition, bei der die Klammern entfallen:  "`$P \rightarrow Q \rightarrow R$ bedeutet $P\rightarrow (Q \rightarrow R)$."' Dies ist subtiler, da keine neuen Symbole eingeführt werden.  Der Grund, warum diese Definition als richtig angesehen wird, ist, dass sich aus der Definition keine neuen Symbolfolgen ergeben, die in der ursprünglichen Sprache gültige wffs (wohlgeformte Formeln)\index{wohlgeformte Formel (wff)} sind, da "`$P \rightarrow Q\rightarrow R$"' keine wff in der ursprünglichen Sprache ist.  Wir machen hier implizit Gebrauch von der Tatsache, dass es ein Entscheidungsverfahren gibt, mit dem wir feststellen können, ob eine Symbolfolge eine wff ist oder nicht, und diese Tatsache erlaubt es uns, Symbolfolgen, die keine wffs sind, zu verwenden, um andere Dinge (wie wffs) mit Hilfe der Definition darzustellen.  Um jedoch zu rechtfertigen, dass die Definition nicht kreativ ist, müssen wir beweisen, dass "`$P \rightarrow Q\rightarrow R$"' in der ursprünglichen Sprache tatsächlich keine wff ist, und das ist schwieriger als in dem Fall, in dem wir einfach ein neues Symbol einführen. 

%Now let's take this reasoning to an extreme.  Propositional calculus is a
%decidable theory,\footnote{This means that a mechanical algorithm exists to
%determine whether or not a wff is a theorem.} so in principle we could make use
%of symbol sequences that are not theorems to represent other things (say, to
%encode actual theorems in a more compact way).  For example, let us extend the
%language by defining a wff "`$P$"' in the extended language as the theorem
%"`$P\rightarrow P$"'\footnote{This is one of the first theorems proved in the
%Metamath database \texttt{set.mm}.}\index{set
%theory database (\texttt{set.mm})} in the original language whenever "`$P$"' is
%not a theorem in the original language.  In the extended language, any wff
%"`$Q$"' thus represents a theorem; to find out what theorem (in the original
%language) "`$Q$"' represents, we determine whether "`$Q$"' is a theorem in the
%original language (before the definition was introduced).  If so, we're done; if
%not, we replace "`$Q$"' by "`$Q\rightarrow Q$"' to eliminate the definition.
%This definition is therefore eliminable, and it does not "`strengthen"' the
%language because any wff that is not a theorem is not in the set of statements
%provable in the original language and thus is available for use by definitions.
%
%Of course, a definition such as this would render practically useless the
%communication of theorems of propositional calculus; but
%this is just a human shortcoming, since we can't always easily discern what is
%and is not a theorem by inspection.  In fact, the extended theory with this
%definition has no more and no less information than the original theory; it just
%expresses certain theorems of the form "`$P\rightarrow P$"'
%in a more compact way.
%
%The point here is that what constitutes a zulässigen Definitionnition is a matter of
%judgment about whether a symbol sequence can easily be recognized by a human
%as invalid in some sense (for example, not a wff); if so, the symbol sequence
%can be appropriated for use by a definition in order to make the extended
%language more compact.  Metamath\index{Metamath} lacks the ability to make this
%judgment, since as far as Metamath is concerned the definition of a wff, for
%example, is arbitrary.  You define for Metamath how wffs\index{wohlgeformte Formel (wff)} are constructed according to your own preferred style.  The
%concept of a wff may not even exist in a given formal system\index{formal
%system}.  Metamath treats all definitions as if they were new axioms, and it
%is up to the human mathematician to judge whether the definition is "`proper"'
%'\index{zulässige Definition}\index{Definition!zulässig} in some agreed-upon way.

Was eine Definition\index{Definition} von einem Axiom\index{Axiom}\index{Axiom vs. Definition} unterscheidet ist in der mathe\-ma\-tischen Literatur manchmal willkürlich.  Zum Beispiel werden die Junktoren $\vee$ ({\sc oder}), $\wedge$ ({\sc und}) und $\leftrightarrow$ (äquivalent zu) in der Aussagenlogik gewöhnlich als definierte Symbole betrachtet, die als Abkürzungen für Ausdrücke verwendet werden können, die die "`primitiven"' Junktoren $\rightarrow$ und $\neg$ enthalten.  So werden sie auch in der Standard-Datenbasis für Logik und Mengenlehre \texttt{set.mm}\index{Mengenlehre-Datenbasis (\texttt{set.mm})} behandelt.  Die ersten drei Junktoren können jedoch auch als "`primitiv"' betrachtet werden, und es wurden Axiomensysteme entwickelt, die alle diese Junktoren als solche behandeln.  So enthält z. B. \cite[S.~35]{Goodstein}\index{Goodstein, R. L.} 15 Axiome, von denen einige mit dem übereinstimmen, was wir in \texttt{set.mm} als Definitionen bezeichnet haben.  In bestimmten Teilmengen der klassischen Aussagenlogik, wie dem intuitionistischen Fragment\index{Intuitionismus}, kann man zeigen, dass man nicht nur mit $\rightarrow$ und $\neg$ auskommen kann, sondern zusätzliche Junktoren als primitiv behandeln muss, damit das System einen Sinn ergibt.\footnote{Zwei schöne Systeme, die den Übergang von dem intuitionistischen und anderen schwachen Fragmenten zur klassischen Logik nur durch Hinzufügen von Axiomen schaffen, sind in \cite{Robinsont}\index{Robinson, T. Thacher} angegeben.}

\subsection{Der Ansatz für Definitionen in \texttt{set.mm}}

In der Mengenlehre definieren rekursive Definitionen ein neu eingeführtes Symbol in Bezug auf sich selbst. Die Begründung rekursiver Definitionen mit Hilfe mehrerer "`Rekursionstheoreme"' ist normalerweise einer der ersten anspruchsvollen Beweise, mit denen ein Student beim Erlernen der Mengenlehre konfrontiert wird, und hinter einer rekursiven Definition steckt eine beträchtliche Menge impliziter Metalogik, obwohl die Definition selbst normalerweise einfach zu formulieren ist.

Metamath selbst hat keine eingebauten technischen Einschränkungen, die mehrteilige rekursive Definitionen im traditionellen Lehrbuchstil verhindern. Da die rekursive Definition jedoch eine fortgeschrittene Metalogik erfordert, um sie zu begründen, ist die Eliminierung einer rekursiven Definition sehr schwierig und wird in Lehrbüchern oft nicht einmal gezeigt. 

\subsubsection{Direkte Definitionen anstelle von rekursiven Definitionen}

Es ist jedoch möglich, eine Art von Komplexität durch eine andere zu ersetzen.  Wir können die Notwendigkeit einer metalogischen Begründung vermeiden, indem wir die Operation direkt mit einem expliziten (aber komplizierten) Ausdruck definieren und dann die rekursive Definition direkt als Theorem ableiten, indem wir ein Rekursionstheorem "`in die andere Richtung"' verwenden. Die Eliminierung einer direkten Definition erfolgt diurch eine einfache mechanische Substitution. Wir tun dies in \texttt{set.mm} wie folgt.

In \texttt{set.mm} war es unser Ziel, fast alle Definitionen in Form von zwei Ausdrücken einzuführen, die entweder durch $\leftrightarrow$ oder $=$ verbunden sind, wobei das Definierte nicht auf der rechten Seite erscheint.  Quine nennt diese Form "`eine echte oder direkte Definition"' \cite[S. 174]{Quine}\index{Quine, Willard Van Orman}, wodurch die Definitionen sehr leicht zu eliminieren sind und die Metalogik\index{Metalogik}, die zu ihrer Rechtfertigung erforderlich ist, so einfach wie möglich ist. Anders ausgedrückt: Unser Ziel war es, alle Definitionen durch direkte mechanische Substitutionen zu eliminieren und die Fundiertheit der Definitionen leicht zu überprüfen. 

\subsubsection{Beispiel für direkte Definitionen}

Wir haben dieses Ziel in fast allen Fällen in \texttt{set.mm} erreicht. Manchmal macht dies die Definitionen komplexer und weniger intuitiv. Die traditionelle Art, die Addition natürlicher Zahlen zu definieren, besteht zum Beispiel darin, eine Operation namens {\em Nachfolger}\index{Nachfolger} zu definieren (was "`plus eins"' bedeutet und mit "`${\rm suc}$"' bezeichnet wird), dann die Addition rekursiv zu definieren\index{rekursive Definition} mit den beiden Definitionen $n + 0 = n$ und $m + {\rm suc}\,n = {\rm suc} (m + n)$.  Obwohl diese Definition einfach und offensichtlich erscheint, ist die Methode zur Eliminierung der Definition nicht offensichtlich: Im zweiten Teil der Definition wird die Addition in Bezug auf sich selbst definiert.  Mit der Eliminierung der Definition ist nicht gemeint, dass man sie wiederholt auf bestimmte $m$ und $n$ anwendet, sondern dass man den expliziten, in sich geschlossenen mengentheoretischen Ausdruck angibt, den $m + n$ darstellt, der für beliebige $m$ und $n$ gilt und der kein $+$-Zeichen auf der rechten Seite hat.  Damit eine rekursive Definition wie diese nicht zirkulär (kreativ) ist, müssen wir einige versteckte, zugrundeliegende Annahmen machen, zum Beispiel, dass die natürlichen Zahlen eine bestimmte Ordnung haben. 

In \texttt{set.mm} haben wir uns entschieden, mit der direkten (wenn auch komplexen und nicht intuitiven) Definition zu beginnen und daraus die rekursive Standarddefinition abzuleiten. Die geschlossene Definition, die in \texttt{set.mm} für die Additionsoperation von Ordinalzahlen\index{ordinale Addition}\index{Addition!von Ordinalen} (von denen die natürlichen Zahlen eine Teilmenge sind) verwendet wird, lautet beispielsweise 

\setbox\startprefix=\hbox{\tt \ \ df-oadd\ \$a\ }
\setbox\contprefix=\hbox{\tt \ \ \ \ \ \ \ \ \ \ \ \ \ }
\startm
\m{\vdash}\m{+_o}\m{=}\m{(}\m{x}\m{\in}\m{{\rm On}}\m{,}\m{y}\m{\in}\m{{\rm
On}}\m{\mapsto}\m{(}\m{{\rm rec}}\m{(}\m{(}\m{z}\m{\in}\m{{\rm
V}}\m{\mapsto}\m{{\rm suc}}\m{z}\m{)}\m{,}\m{x}\m{)}\m{`}\m{y}\m{)}\m{)}
\endm
\noindent welche vom Operator ${\rm rec}$ (siehe folgenden Abschnitt) abhängt.

\subsubsection{Rekursionsoperatoren}

Die obige Definition von \texttt{df-oadd} hängt von der Definition von ${\rm rec}$ ab, einem "`Rekursionsoperator"'\index{Rekursionsoperator} mit der Definition \texttt{df-rdg}: 

\setbox\startprefix=\hbox{\tt \ \ df-rdg\ \$a\ }
\setbox\contprefix=\hbox{\tt \ \ \ \ \ \ \ \ \ \ \ \ }
\startm
\m{\vdash}\m{{\rm
rec}}\m{(}\m{F}\m{,}\m{I}\m{)}\m{=}\m{\mathrm{recs}}\m{(}\m{(}\m{g}\m{\in}\m{{\rm
V}}\m{\mapsto}\m{{\rm if}}\m{(}\m{g}\m{=}\m{\varnothing}\m{,}\m{I}\m{,}\m{{\rm
if}}\m{(}\m{{\rm Lim}}\m{{\rm dom}}\m{g}\m{,}\m{\bigcup}\m{{\rm
ran}}\m{g}\m{,}\m{(}\m{F}\m{`}\m{(}\m{g}\m{`}\m{\bigcup}\m{{\rm
dom}}\m{g}\m{)}\m{)}\m{)}\m{)}\m{)}\m{)}
\endm

\noindent die anhand der Definitionen in Abschnitt~\ref{setdefinitions} weiter heruntergebrochen werden könnte.

Diese Definition von ${\rm rec}$ definiert einen rekursiven Definitionsgenerator auf ${\rm On}$ (der Klasse der Ordinalzahlen) mit charakteristischer Funktion $F$ und Anfangswert $I$. Diese Operation erlaubt es uns, mit kompakten direkten Definitionen Funktionen zu definieren, die normalerweise in Lehrbüchern mit rekursiven Definitionen definiert werden. Der Preis, den wir mit unserem Ansatz zahlen, ist die Komplexität unserer ${\rm rec}$-Operation (insbesondere, wenn {\tt df-recs}, auf dem sie aufbaut, ebenfalls eliminiert werden soll). Aber sobald wir diese Hürde überwunden haben, werden Definitionen, die sonst rekursiv wären, relativ einfach, wie zum Beispiel {\tt oav}, aus dem wir die rekursive Lehrbuchdefinition als Theoreme {\tt oa0}, {\tt oasuc} und {\tt oalim} beweisen (mit Hilfe der Theoreme {\tt rdg0}, {\tt rdgsuc} und {\tt rdglim2a}).  Wir können die ${\rm rec}$-Operation auch einschränken, um rekursive Funktionen auf den natürlichen Zahlen $\omega$ zu definieren; siehe {\tt fr0g} und {\tt frsuc}.  Unsere ${\rm rec}$-Operation taucht in der veröffentlichten Literatur offenbar nicht auf, obwohl sie eng mit der Definition 25.2 von [Quine] S. 177 verwandt ist, die er verwendet, "`um eine Rekursion in eine echte oder direkte Definition zu verwandeln"' (S. 174).  Man beachte, dass die ${\rm if}$-Operationen (siehe {\tt df-if}) Fälle danach auswählen, ob der Definitionsbereich von $g$ die leere Menge, ein Nachfolger oder eine Grenzordinale ist. 

Eine wichtige Anwendung dieser Definition ${\rm rec}$ ist der rekursive Sequenzgenerator {\tt df-seq} auf den natürlichen Zahlen (als Teilmenge der komplexen unendlichen Sequenzen, wie der Fakultätsfunktion {\tt df-fac} und den ganzzahligen Potenzen {\tt df-exp}). 

Die Definition von ${\rm rec}$ hängt von ${\rm recs}$ ab. Von der direkten Verwendung des mächtigeren (und primitiveren) ${\rm recs}$-Konstrukts wird abgeraten, es ist aber bei Bedarf verfügbar. Dies definiert eine Funktion $\mathrm{recs} ( F )$ auf ${\rm On}$, der Klasse der Ordinalzahlen, durch transfinite Rekursion unter der Voraussetzung einer Regel $F$, die den nächsten Wert unter Berücksichtigung aller bisherigen Werte bestimmt. Im Gegensatz zu {\tt df-rdg}, das die Aktualisierungsregel darauf beschränkt, nur den vorherigen Wert zu verwenden, erlaubt diese Version der Aktualisierungsregel, alle vorherigen Werte zu verwenden, weshalb sie als "`stark"' bezeichnet wird, obwohl sie eigentlich primitiver ist.  Siehe {\tt recsfnon} und {\tt recsval} für die primären Eigenschaften dieser Definition. Sie ist definiert als: 

\setbox\startprefix=\hbox{\tt \ \ df-recs\ \$a\ }
\setbox\contprefix=\hbox{\tt \ \ \ \ \ \ \ \ \ \ \ \ \ }
\startm
\m{\vdash}\m{\mathrm{recs}}\m{(}\m{F}\m{)}\m{=}\m{\bigcup}\m{\{}\m{f}\m{|}\m{\exists}\m{x}\m{\in}\m{{\rm
On}}\m{(}\m{f}\m{{\rm
Fn}}\m{x}\m{\wedge}\m{\forall}\m{y}\m{\in}\m{x}\m{(}\m{f}\m{`}\m{y}\m{)}\m{=}\m{(}\m{F}\m{`}\m{(}\m{f}\m{\restriction}\m{y}\m{)}\m{)}\m{)}\m{\}}
\endm

\subsubsection{Abschließende Bemerkungen zu direkten Definitionen}

Aus diesen direkten Definitionen wird die einfachere, intuitivere rekursive Definition als eine Reihe von Theoremen abgeleitet.\index{natürliche Zahlen}\index{Addition}\index{rekursive Definition}\index{ordinale Addition} Das Endergebnis ist dasselbe, aber wir können dadurch vollständig auf die recht komplexe Metalogik verzichten, die die rekursive Definition rechtfertigt. 

Rekursive Definitionen werden oft als effizienter und intuitiver angesehen als direkte Definitionen, sobald die Metalogik erlernt oder möglicherweise einfach als korrekt akzeptiert wurde.  Man war jedoch der Ansicht, dass die direkte Definition in \texttt{set.mm} die Strenge maximiert, indem sie die Metalogik minimiert.  Eine solche Definition kann mühelos eliminiert werden, was bei einer rekursiven Definition nur schwer möglich ist. 

Auch hier gilt, dass Metamath selbst keine eingebauten technischen Einschränkungen hat, die mehrteilige rekursive Definitionen im traditionellen Lehrbuchstil verhindern. Stattdessen ist es unser Ziel, alle Definitionen mit direkter mechanischer Substitution zu eliminieren und die Fundiertheit der Definitionen leicht zu überprüfen. 

\subsection{Hinzufügen von Einschränkungen für Definitionen}

Die Metamath-Basissprache und das Metamath-Programm haben keine eingebauten einschrämkungen für Definitionen, da sie nur \texttt{\$a}-Anweisungen sind. 

Nichts hindert jedoch ein Verifikationssystem daran, zusätzliche Regeln zu verifizieren, um weitere Einschränkungen für Definitionen vorzunehmen. Das \texttt{mmj2}\index{mmj2}-Programm unterstützt zum Beispiel verschiedene Arten von zusätzlichen Informationskommentaren (siehe Abschnitt \ref{jcomment}). Eine ihrer Verwendungen ist die optionale Überprüfung zusätzlicher Einschränkungen, einschließlich der Überprüfung, ob Definitionen bestimmte Anforderungen erfüllen. Diese zusätzlichen Prüfungen werden von der kontinuierlichen Integration (CI)\index{kontinuierliche Integration (CI)} der \texttt{set.mm}\index{Mengenlehre-Datenbasis (\texttt{set.mm})}\index{Metamath Proof Explorer} Datenbasis benötigt. Dieser Ansatz ermöglicht es uns, optional zusätzliche Anforderungen an Definitionen zu stellen, wenn wir dies wünschen, ohne dass diese Regeln notwendigerweise für alle Datenbasen gelten oder von allen Verifikationssystemen verlangen zu müssen diese zu prüfen. Außerdem können wir auf diese Weise spezielle, auf eine Datenbasis zugeschnittene Einschränkungen vornehmen, ohne dass andere Systeme diese speziellen Einschränkungen berücksichtigen müssen. 

Es gibt zwei solcher Einschränkungen in der Datenbasis \texttt{set.mm}\index{Mengenlehre-Datenbasis (\texttt{set.mm})}\index{Metamath Proof Explorer}, die meit dem Programm \texttt{mmj2}\index{mmj2} geprüft werden, die es wert sind, hier besprochen zu werden: eine Syntax-Prüfung und eine Prüfung der Fundiertheit von Definitionen.

% On February 11, 2019 8:32:32 PM EST, saueran@oregonstate.edu wrote:
% The following addition to the end of set.mm is accepted by the mmj2
% parser and definition checker and the metamath verifier(at least it was
% when I checked, you should check it too), and creates a contradiction by
% proving the theorem |- ph.
% ${
% wleftp $a wff ( ( ph ) $.
% wbothp $a wff ( ph ) $.
% df-leftp $a |- ( ( ( ph ) <-> -. ph ) $.
% df-bothp $a |- ( ( ph ) <-> ph ) $.
% anything $p |- ph $=
%   ( wbothp wn wi wleftp df-leftp biimpi df-bothp mpbir mpbi simplim ax-mp)
%   ABZAMACZDZCZMOEZOCQAEZNDZRNAFGSHIOFJMNKLAHJ $.
% $}
%
% This particular problem is countered by enabling, within mmj2,
% SetParser,mmj.verify.LRParser

Erstens aktivieren wir in \texttt{mmj2} (über den Befehl \texttt{SetParser}) eine Syntax-Prüfung, die vorschreibt, dass alle neuen Definitionen für einen KLR(5)-Parser kein mehrdeutiges Ergebnis erzeugen dürfen. Dies verhindert einige Fehler wie z. B. Definitionen mit unausgeglichenen Klammern. 

Zweitens führen wir eine Definitionsprüfung durch, die spezifisch für \texttt{set.mm} oder ähnliche Datenbasen ist (über das Makro \texttt{definitionCheck}). Einige \texttt{\$a}-Anweisungen (einschließlich aller ax-*-Statements) sind von diesen Prüfungen ausgenommen, da sie diese einfache Prüfung immer nicht bestehen werden, aber sie sind für die meisten Definitionen geeignet. Diese Prüfung erzwingt eine Reihe von zusätzlichen Regeln: 

\begin{enumerate}

\item Neue Definitionen müssen mit $=$ oder $\leftrightarrow$ eingeführt werden.

\item Keine vor dieser Anweisung eingeführte \texttt{\$a}-Anweisung darf das in dieser Definition definierte Symbol verwenden, und die Definition darf sich selbst nicht verwenden (außer einmal im Definiendum).

\item Die im Definiens verwendeten Variablen müssen distinkt sein.

\item Alle Dummy-Variablen im Definiendum müssen untereinander und mit den Variablen im Definiendum distinkt sein. Um dies festzustellen, sucht das System in der Datenbasis nach einem "`Rechtfertigungssatz"'\footnote{Anm. der Übersetzer: Ein "`Rechtfertigungssatz"' ("`justification theorem"') ist ein Theorem, dass die Korrektheit der Definition rechtfertigt. Der Name des Theorems ist der Teil des Namens der Definition nach "`df-"' ergänzt um "`just"', siehe z.B. {\tt df-mo} und {\tt mojust}.}. Wenn dieser nicht vorhanden ist, versucht es intern, für jede Dummy-Variable x $( \varphi \rightarrow \forall x \varphi )$ zu beweisen.

\item Jede Dummy-Variable sollte eine Mengenvariable sein, es sei denn, es gibt einen Rechtfertigungssatz.

\item Jede Dummy-Variable muss gebunden sein (wenn das System dies nicht bestimmen kann, muss ein Rechtfertigungssatz angegeben werden).

\end{enumerate}

\subsection{Zusammenfassung des Metamath-Ansatzes für\texorpdfstring{\\}{} Definitionen}

Kurz gesagt, bei einem rigorosen Vorgehen stellt sich heraus, dass Definitionen subtil sein können und manchmal schwierige Metatheoreme erfordern, um zu beweisen, dass sie nicht kreativ sind. 

Anstatt solche Komplikationen in die Metamath-Sprache selbst einzubauen, behandeln die grundlegende Metamath-Sprache und das Programm die traditionellen Axiome und Definitionen einheitlich als \texttt{\$a}-Anweisungen. Wir haben dann verschiedene Werkzeuge entwickelt, die es jedem ermöglichen, für seine spezifischen, selbst angelegten Datenbasen zusätzliche, selbst festgelegte Bedingungen zu überprüfen, ohne die grundlegenden Eigenschaften und Funktionalitäten von Metamath zu verkomplizieren. 

\chapter{Das Metamath-Programm}\label{commands}

Dieses Kapitel ist als ein Referenzhandbuch für das Metamath-Programm zu verstehen.\index{Metamath!Befehle}

Aktuelle Anweisungen für den Bezug und die Installation des Metamath-Programms finden Sie auf der Website \url{http://metamath.org}. Für Windows gibt es eine vorkompilierte Version namens \texttt{metamath.exe}.  Für Unix, Linux und Mac OS X (die wir zusammenfassend als "`Unix"' bezeichnen) kann das Metamath-Programm aus seinem Quellcode mit dem Befehl 
\begin{verbatim} 
gcc *.c -o metamath 
\end{verbatim} 
unter Verwendung des auf diesen Systemen verfügbaren \texttt{gcc} {\sc c}-Compilers kompiliert werden. 

In den nachfolgenden Beschreibungen der Befehlssyntax sind die in eckige Klammern [\ ] eingeschlossenen Angaben optional.  Dateinamen können optional in einfache oder doppelte Anführungszeichen gesetzt werden.  Dies ist nützlich, wenn der Dateiname Leerzeichen oder Schrägstriche (\texttt{/}) enthält, wie z.B. in Unix-Pfadnamen, \index{Unix-Dateinamen}\index{Dateinamen!Unix}, die mit Metamath-Befehlszeilenparameter verwechselt werden könnten.\index{Unix-Dateinamen}\index{Dateinamen!Unix}

\section{Aufruf von Metamath}

Unix, Linux und Mac OS X verfügen über eine Befehlszeilenschnittstelle, die  {\em Bash Shell}.  (Unter Mac OS X wählen Sie das Programm Terminal aus Anwendungen/Dienstprogramme.) Um Metamath von der Bash-Shell-Eingabeaufforderung aus aufzurufen, geben Sie Folgendes ein, vorausgesetzt das Metamath-Programm befindet sich im aktuellen Verzeichnis: 
\begin{verbatim} 
bash$ ./metamath 
\end{verbatim} 

Um Metamath von einem Windows-Konsolenfenster aus aufzurufen, geben Sie Folgendes ein, vorausgesetzt das Metamath-Programm befindet sich im aktuellen Verzeichnis (oder in einem Verzeichnis, das in der Systemumgebungsvariablen Path enthalten ist): 
\begin{verbatim} 
C:\metamath>metamath 
\end{verbatim} 

Um Befehlszeilenargumente beim Aufruf zu verwenden, sollten die Befehlszeilenargumente eine Liste von Metamath-Befehlen sein, die von Anfüh\-rungszeichen umgeben sind, wenn sie Leerzeichen enthalten.  Unter Windows müssen die umgebenden Anführungszeichen doppelte (nicht einfache) Anführungszeichen sein.  Um zum Beispiel die Datenbasisdatei \texttt{set.mm} zu lesen, alle Beweise zu überprüfen und das Programm zu beenden, geben Sie (unter Unix) Folgendes ein:
\begin{verbatim} 
bash$ ./metamath 'read set.mm' 'verify proof *' exit 
\end{verbatim} Beachten Sie, dass unter Unix jeder Verzeichnispfad mit \texttt{/}'s von Anführungszeichen umgeben sein muss, damit Metamath das \texttt{/} nicht als Befehlszeilenparameter interpretiert.  Wenn also \texttt{set.mm} im Verzeichnis \texttt{/tmp} liegt, verwenden Sie für das obige Beispiel 
\begin{verbatim} 
bash$ ./metamath 'read "/tmp/set.mm"' 'verify proof *' exit 
\end{verbatim} 

Wenn die Befehlszeile nur ein Argument und keine Leerzeichen enthält, wird implizit angenommen, dass der Befehl \texttt{read} lautet.  In diesem einen Sonderfall werden \texttt{/} nicht als Befehlszeilenparameter interpretiert, so dass Sie keine Anführungszeichen um einen Unix-Dateinamen herum benötigen.  Also 
\begin{verbatim} 
bash$ ./metamath /tmp/set.mm 
\end{verbatim} und 
\begin{verbatim} 
bash$ ./metamath "read '/tmp/set.mm'" 
\end{verbatim} sind gleichwertig. 


\section{Steuerung von Metamath}

Das Metamath-Programm wurde zuerst auf einem {\sc vax/vms}-System entwickelt, und einige Aspekte seines Befehlszeilenverhaltens spiegeln dieses Erbe wider. Wir hoffen, dass Sie es einigermaßen benutzerfreundlich finden, sobald Sie sich daran gewöhnt haben. 

Jede Befehlszeile besteht aus einer Folge englischsprachiger Wörter, die durch Leerzeichen getrennt sind, wie in \texttt{show settings}.  Bei Befehlswörtern wird nicht zwischen Groß- und Kleinschreibung unterschieden, und es werden nur so viele Buchstaben benötigt, wie nötig sind, um Mehrdeutigkeiten auszuschließen; so würde beispielsweise bereits \texttt{sh se} für die Ausführung des Befehls \texttt{show settings} ausreichen.  In einigen Fällen sind Argumente wie Dateinamen, Label von Anweisungen oder Symbolnamen erforderlich; bei diesen wird zwischen Groß- und Kleinschreibung unterschieden (obwohl Dateinamen auf einigen Betriebssystemen möglicherweise nicht unterschieden werden). 

Eine Befehlszeile wird eingegeben, indem man sie eintippt und dann die {\em Return-} ({\em Enter-})Taste drückt.  Um herauszufinden, welche Befehle verfügbar sind, geben Sie \texttt{?} an der Eingabeaufforderung \texttt{MM>} ein.  Um herauszufinden, welche Möglichkeiten Sie an einer beliebigen Stelle eines Befehls haben, drücken Sie {\em return} und Sie werden dazu aufgefordert eine angegebene Möglichkeit auszuwählen.  Die Standardauswahl (diejenige, die ausgewählt wird, wenn Sie nur {\em Return} drücken) wird in Klammern angezeigt (\texttt{<>}). 

Sie können auch \texttt{?} anstelle eines Befehlsworts eingeben, um Metamath dazu zu bringen, Ihnen die verfügbaren Möglichkeiten mitzuteilen.  Die Methode \texttt{?} funktioniert allerdings nicht, wenn an dieser Stelle ein Argument erwartet wird, das kein Schlüsselwort ist, wie z. B. ein Dateiname, weil das Programm dann das \texttt{?} als Wert des Arguments interpretiert. 

Einige Befehle haben einen oder mehrere optionale Parameter, die das Verhalten des Befehls beeinflussen.  Parameter werden mit einem Schrägstrich (\texttt{/}) eingeleitet, wie z. B. in \texttt{read set.mm / verify}.  Leerzeichen um das \texttt{/} sind optional.  Wenn Sie ein Leerzeichen oder einen Schrägstrich in einem Befehlsargument verwenden müssen, wie in einem Unix-Dateinamen, setzen Sie das Befehlsargument in einfache oder doppelte Anführungszeichen. 

Der Befehl \texttt{open log} speichert alles, was Sie auf dem Bildschirm sehen, und ist nützlich, wenn Sie etwas wiederherstellen wollen, falls bei einem Beweis etwas schief geht, oder wenn Sie einen Fehler dokumentieren wollen. 

Wenn ein Befehl mit mehr Zeilen als auf einen Bildschirm passen antwortet, werden Sie aufgefordert, \texttt{<return> zum Fortfahren, Q zum Beenden oder S zum Scrollen zum Ende} einzugeben.  \texttt{Q} oder \texttt{q} (Groß- und Kleinschreibung wird nicht beachtet) führt den Befehl intern zu Ende, unterdrückt aber weitere Ausgaben bis zur nächsten Eingabeaufforderung \texttt{MM>}.  \texttt{s} unterdrückt weitere Pausen bis zur nächsten Eingabeaufforderung \texttt{MM>}.  Nach der ersten Bildschirmseite haben Sie auch die Möglichkeit, mit \texttt{b} eine Bildschirmseite zurückzugehen.  Beachten Sie, dass \texttt{b} auch an der Eingabeaufforderung \texttt{MM>} unmittelbar nach einem Befehl eingegeben werden kann, um durch die Ausgabe dieses Befehls zurückzublättern. 

Eine in Anführungszeichen eingeschlossene Befehlszeile wird von Ihrem Betriebssystem ausgeführt. Siehe Abschnitt~\ref{oscmd}. 

{Warnung:} Wenn Sie {\sc ctrl-c} drücken, wird das Metamath-Programm sofort abgebrochen.  Das bedeutet, dass alle nicht gespeicherten Daten verloren gehen. 


\subsection{\texttt{exit}-Befehl}\index{\texttt{exit}-Befehl}

Syntax:  \texttt{exit} [\texttt{/force}]

Mit diesem Befehl verlässt man Metamath.  Wenn mit den Befehlen \texttt{proof} oder \texttt{save new{\char`\_}proof} Änderungen am Quelltext vorgenommen wurden, erhalten Sie die Möglichkeit, mit \texttt{write source} die Änderungen dauerhaft zu speichern. 

Im Modus "`Beweis-Assistent"'\index{Beweis-Assistent} kehrt man mit dem Befehl \texttt{exit} zur Eingabeaufforderung \verb/MM>/ zurück. Wenn Änderungen am Beweis vorgenommen wurden, erhalten Sie die Möglichkeit, den neuen Beweis mit \texttt{save new{\char`\_}proof} zu speichern. 

Der Befehl \texttt{quit} ist ein Synonym für \texttt{exit}.

Optionaler Befehlszeilenparameter:
\texttt{/force} - Keine Eingabeaufforderung, wenn Änderungen nicht gespeichert wurden.  Dieser Qualifizierer ist in \texttt{submit}-Befehlsdateien (siehe Abschnitt~\ref{sbmt}) nützlich, um ein vorhersehbares Verhalten zu gewährleisten. 


\subsection{\texttt{open log}-Befehl}\index{\texttt{open log}-Befehl}

Syntax:  \texttt{open log} {\em Dateiname}

Mit diesem Befehl wird eine Protokolldatei geöffnet, in der alles gespeichert wird, was Sie auf dem Bildschirm sehen.  Sie ist nützlich, um einen Fehler während einer langen Sitzung mit dem Beweis-Assistenten zu beheben oder um Fehler zu dokumentieren.\index{Metamath!Bug} 

Die Protokolldatei kann mit \texttt{close log} geschlossen werden.  Sie wird automatisch beim Beenden von Metamath geschlossen. 


\subsection{\texttt{close log}-Befehl}\index{\texttt{close log}-Befehl}

Syntax:  \texttt{close log}

Der Befehl \texttt{close log} schließt eine Protokolldatei, falls eine geöffnet ist.  Siehe auch \texttt{open log}. 


\subsection{\texttt{submit}-Befehl}\index{\texttt{submit}-Befehl}\label{sbmt}

Syntax:  \texttt{submit} {\em Dateiname}

Dieser Befehl bewirkt, dass weitere Befehlszeilen aus der angegebenen Datei entnommen und ausgeführt werden.  Beachten Sie, dass jede Zeile, die mit einem Ausrufezeichen (\texttt{!}) beginnt, als Kommentar behandelt (d.h. ignoriert) wird.  Beachten Sie auch, dass die Bildschirmausgabe kontinuierlich durchläuft, so dass Sie eventuell eine Protokolldatei öffnen sollten (siehe \texttt{open log}), um die auf dem Bildschirm vorbeiziehenden Ergebnisse aufzuzeichnen. Nachdem alle Befehlszeilen in der Datei ausgeführt wurden, kehrt Metamath in den normalen Modus der Benutzeroberfläche zurück. 

Der Befehl \texttt{submit} kann rekursiv aufgerufen werden (d.h. ebenfalls innerhalb einer \texttt{submit} Befehlsdatei). 

Optionaler Befehlszeilenparameter:
\texttt{/silent} - unterdrückt die Bildschirmausgabe, zeichnet die Ausgabe aber dennoch in einer Protokolldatei auf, falls eine solche geöffnet ist.  


\subsection{\texttt{erase}-Befehl}\index{\texttt{erase}-Befehl}

Syntax:  \texttt{erase}

Dieser Befehl setzt Metamath auf seinen Ausgangszustand zurück und löscht alle Datenbasen, die mit \texttt{read} eingelesen wurden.  Wenn mit den Befehlen \texttt{save proof} oder \texttt{save new{\char`\_}proof} Änderungen an der Quelldatei vorgenommen wurden, wird Ihnen die Möglichkeit gegeben, \texttt{write source} zu verwenden, um die Änderungen dauerhaft zu speichern. 


\subsection{\texttt{set echo}-Befehl}\index{\texttt{set echo}-Befehl}

Syntax:  \texttt{set echo on} or \texttt{set echo off}

Der Befehl \texttt{set echo on} bewirkt, dass die Befehlszeilen mit erweiterten Abkürzungen wiedergegeben werden.  Beim Erlernen der Metamath-Befehle zeigt Ihnen diese Funktion genau den Befehl an, dem Ihre abgekürzte Eingabe entspricht. 


\subsection{\texttt{set scroll}-Befehl}\index{\texttt{set scroll}-Befehl}

Syntax:  \texttt{set scroll prompted} oder \texttt{set scroll continuous}

Die Metamath-Befehlszeilenschnittstelle startet im Modus \texttt{prompted}, was bedeutet, dass Sie nach jedem Vollbild in einer langen Auflistung zum Fortfahren oder Beenden aufgefordert werden.  Im Modus \texttt{continuous} werden lange Auflistungen ohne Pause durchlaufen. 


\subsection{\texttt{set width}-Befehl}\index{\texttt{set
width}-Befehl}

Syntax:  \texttt{set width} {\em Zahl}

Metamath geht davon aus, dass die Breite Ihres Bildschirms 79 Zeichen beträgt (dies wurde gewählt, weil die Eingabeaufforderung in Windows XP einen Umbruchfehler bei Spalte 80 aufweist).  Wenn Ihr Bildschirm breiter oder schmaler ist, können Sie mit diesem Befehl die Standardbreite des Bildschirms ändern.  Eine größere Breite ist vorteilhaft für die Protokollierung von Beweisen in einer Ausgabedatei, die auf einem breiten Drucker gedruckt werden soll.  Auf manchen Terminals kann eine geringere Breite erforderlich sein; in diesem Fall kann der Umbruch der Informationsmeldungen jedoch manchmal etwas unnatürlich wirken.  In \LaTeX\index{latex@{\LaTeX}!Zeichen pro Zeile} gibt es normalerweise maximal 61 Zeichen pro Zeile mit Schreibmaschinenschrift (die Beispiele in diesem Buch wurden mit 61 Zeichen pro Zeile erstellt). 


\subsection{\texttt{set height}-Befehl}\index{\texttt{set
height}-Befehl}

Syntax:  \texttt{set height} {\em Zahl}

Metamath geht davon aus, dass Ihre Bildschirmhöhe 24 Zeilen an Zeichen beträgt.  Wenn Ihr Bildschirm größer oder kleiner ist, können Sie mit diesem Befehl die Anzahl der Zeilen ändern, bei denen die Anzeige pausiert und Sie zum Fortfahren auffordert. 


\subsection{\texttt{beep}-Befehl}\index{\texttt{beep}-Befehl}

Syntax:  \texttt{beep}

Bei diesem Befehl ertönt ein Piepton.  Wenn Sie ihn vor dem Start eines lang laufenden Befehls eingeben, werden Sie darauf hingewiesen, dass der Befehl beendet ist.  Der Einfachheit halber ist \texttt{b} eine Abkürzung für \texttt{beep}. 

Hinweis: Wenn \texttt{b} an der Eingabeaufforderung \texttt{MM>} unmittelbar nach dem Ende einer mehrseitigen Anzeige mit der Aufforderung "`\texttt{Press <return> for more}..."' eingegeben wird, kehrt der \texttt{b} zur vorherigen Seite zurück, anstatt den Befehl \texttt{beep} auszuführen. In diesem Fall müssen Sie die ungekürzte \texttt{beep}-Form des Befehls eingeben. 


\subsection{\texttt{more}-Befehl}\index{\texttt{more}-Befehl}

Syntax:  \texttt{more} {\em Dateiname}

Dieser Befehl zeigt den Inhalt einer {\sc ascii}-Datei auf dem Bildschirm an.  

(Dieser Befehl dient der Bequemlichkeit, ist aber nicht sehr leistungsfähig.  Siehe Abschnitt~\ref{oscmd}, um den entsprechenden Befehl Ihres Betriebssystems aufzurufen, z. B. den Befehl \texttt{more} unter Unix). 


\subsection{Betriebssystem-Befehle}\index{Betriebssystem-Befehl}\label{oscmd}

Eine in einfache oder doppelte Anführungszeichen eingeschlossene Zeile wird vom Betriebssystem Ihres Computers ausgeführt, wenn dieses über eine Befehlszeilenschnittstelle verfügt.  Auf einem {\sc vax/vms}-System wird beispielsweise \verb/MM> 'dir'/ den Inhalt des Festplattenverzeichnisses ausgeben.  Beachten Sie, dass diese Funktion auf Macintosh-Systemen vor Mac OS X, die keine Befehlszeilenschnittstelle haben, nicht funktioniert.  

Der Einfachheit halber ist das Anführungszeichen am Ende optional. 


\subsection{Größenbeschränkungen in Metamath}

Im Allgemeinen gibt es keine festen, vordefinierten Grenzen\index{Metamath!Memory-Limits} dafür, wie viele Labels, Token\index{Token}, Anweisungen usw. Sie in einer Datenbasisdatei verwenden können.  Das Metamath-Programm verwendet 32-Bit-Variablen (64-Bit auf 64-Bit-CPUs) als Indizes für fast alle internen Arrays, die bei Bedarf dynamisch zugewiesen werden.  


\section{Lesen und Schreiben von Dateien}

Die folgenden Befehle erstellen neue Dateien: die \texttt{open}-Befehle; die \texttt{write}-Befehle; die Optionen \texttt{/html}, \texttt{/alt{\char`\_}html}, \texttt{/brief{\char`\_}html}, \texttt{/brief{\char`\_}alt{\char`\_}html} von \texttt{show statement}, und \texttt{midi}.  Die folgenden Befehle werden an zuvor geöffnete Dateien angehängt: die Option \texttt{/tex} von \texttt{show proof} und \texttt{show new{\char`\_}proof}; die Optionen \texttt{/tex} und \texttt{/simple{\char`\_}tex} von \texttt{show statement}; die Befehle \texttt{close}; und alle Bildschirmdialoge zwischen \texttt{open log} und \texttt{close log}. 

Die Befehle, die neue Dateien erstellen, überschreiben keine vorhandenen {\em Dateinamen}, sondern benennen die vorhandene Datei in {\em Dateiname}\texttt{{\char`\~}1} um.  Ein vorhandener {\em Dateiname}\texttt{{\char`\~}1} wird umbenannt in {\em Dateiname}\texttt{{\char`\~}2}, usw.\ bis zu {\em Dateiname}\texttt{{\char`\~}9}.  Ein vorhandener {\em Dateiname}\texttt{{\char`\~}9} wird gelöscht.  Dies erleichtert die Wiederherstellung nach Fehlern, bringt aber auch Unordnung in Ihr Verzeichnis, so dass Sie gelegentlich diese \texttt{{\char`\~}}$n$-Dateien bereinigen (löschen) sollten. 


\subsection{\texttt{read}-Befehl}\index{\texttt{read}-Befehl}

Syntax:  \texttt{read} {\em Dateiname} [\texttt{/verify}]

Mit diesem Befehl wird eine Quelldatei in der Metamath-Sprache und alle darin referenzierten Dateien eingelesen.  Normalerweise ist es das erste, was Sie tun, wenn Sie mit Metamath beginnen. Die Syntax der Anweisungen wird überprüft, die Syntax der Beweise jedoch nicht. Beachten Sie, dass der Dateiname in einfache oder doppelte Anführungszeichen eingeschlossen werden kann; dies ist nützlich, wenn der Dateiname Schrägstriche enthält, wie es unter Unix der Fall sein kann. 

Falls Sie eine "`\texttt{?Expected VERIFY}"' Fehlermeldung erhalten, wenn Sie versuchen einen Unix-Dateinamen mit Schrägstrichen zu lesen, haben Sie ihn wahrscheinlich nicht in Anführungszeichen gesetzt.\index{Unix-Dateinamen}\index{Dateinamen!Unix}

Wenn Sie zur Eingabe des Dateinamens aufgefordert werden (durch Drücken von {\em Enter} nach \texttt{read}), sollten Sie ihn {\em nicht} in Anführungszeichen setzen, auch wenn es sich um einen Unix-Dateinamen mit Schrägstrichen handelt. 

Optionaler Befehlszeilenparameter:

    \texttt{/verify} - Überprüft alle Beweise beim Einlesen der Datenbasis.  Diese Option verlangsamt das Einlesen der Datei.  Siehe \texttt{verify proof} für weitere Informationen zur Fehlerprüfung von Dateien.

Siehe auch \texttt{erase}.


\subsection{\texttt{write source}-Befehl}\index{\texttt{write source}-Befehl}

Syntax:  \texttt{write source} {\em filename}
[\texttt{/rewrap}]
[\texttt{/split}]
% TeX doesn't handle this long line with tt text very well,
% so force a line break here.
[\texttt{/keep\_includes}] {\\}
[\texttt{/no\_versioning}]

Mit diesem Befehl wird der Inhalt einer Metamath-Datenbasis\index{Datenbasis} in eine Datei\index{Quelldatei} geschrieben.

Optionale Befehlszeilenparameter:

\texttt{/rewrap} - Formatiert Anweisungen und Kommentare entsprechend der in der set.mm-Datenbasis verwendeten Konvention um. Es hebt den Zeilenumbruch im Kommentar vor jeder \texttt{\$a}- und \texttt{\$p}-Anweisung auf und bricht die Zeile dann neu um.  Sie sollten die Ausgabe mit dem Original vergleichen, um sicherzustellen, dass der gewünschte Effekt erzielt wird; falls nicht, gehen Sie zurück zum Original.  Die Länge der umgebrochenen Zeile berücksichtigt den aktuell gültigen Parameter \texttt{set width}.  

Hinweis: Text, der in \texttt{<HTML>}...\texttt{</HTML>}-Tags eingeschlossen ist, wird durch den Qualifier \texttt{/rewrap} nicht verändert. Beweise selbst werden nicht umformatiert; verwenden Sie dazu \texttt{save proof * / compressed}. Eine isolierte Tilde (\~{}) wird immer in derselben Zeile wie das folgende Symbol gehalten, so dass Sie alle Kommentarverweise auf ein Symbol finden können, indem Sie nach \~{}, gefolgt von einem Leerzeichen und dem Symbol, suchen (dies ist nützlich, um Querverweise zu finden). Übrigens, \texttt{save proof} beachtet auch den derzeit gültigen Parameter \texttt{set width}.

\texttt{/split} - 
Dateien, die mit dem Ausdruck \$[ \textit{inclfile} \$] in den Quelltext inkludiert werden, werden in separate Dateien geschrieben, anstatt in eine einzige Ausgabedatei aufgenommen zu werden.  Der Name jeder separat geschriebenen Datei ist das Argument \textit{inclfile} des Include-Befehls. 

\texttt{/keep\_includes} - 
Wenn eine Quelldatei inkludierte Dateien hat, aber durch Weglassen von \texttt{/split} als einzelne Datei geschrieben wird, werden die inkludierten Dateien standardmäßig gelöscht (eigentlich nur mit einem Suffix \char`\~1 umbenannt, es sei denn, \texttt{/no\_versioning} ist gesetzt), um die möglicherweise verwirrende Quelldateiduplikation sowohl in der Ausgabedatei als auch in der inkludierten Datei zu verhindern. Die Option \texttt{/keep\_includes} verhindert diese Löschung. 

\texttt{/no\_versioning} - 
Sicherungsdateien mit dem Suffix \char`\~1 werden nicht erstellt.


\section{Anzeige von Status und Anweisungen}

\subsection{\texttt{show settings}-Befehl}\index{\texttt{show settings}-Befehl}

Syntax:  \texttt{show settings}

Dieser Befehl zeigt den Zustand verschiedener Parameter an.

\subsection{\texttt{show memory}-Befehl}\index{\texttt{show memory}-Befehl}

Syntax:  \texttt{show memory}

Dieser Befehl zeigt den noch verfügbaren Speicher an.  Er ist auf den meisten modernen Betriebssystemen, die über virtuellen Speicher verfügen, nicht aussagekräftig.\index{Metamath!Memory-Nutzung} 

\subsection{\texttt{show labels}-Befehl}\index{\texttt{show labels}-Befehl}

Syntax:  \texttt{show labels} {\em label-match} [\texttt{/all}]
   [\texttt{/linear}]

Dieser Befehl zeigt die Labels von \texttt{\$a}- und \texttt{\$p}-Anweisungen an, die auf {\em label-match} passen.  Ein \verb$*$ in {label-match} ist ein Platzhalter für null oder mehr beliebige Zeichen.  Zum Beispiel passt \verb$*abc*def$ auf alle Labels, die \verb$abc$ enthalten und mit \verb$def$ enden. 

Optionale Befehlszeilenparameter:

   \texttt{/all} - 
    Übereinstimmungen für \texttt{\$e}- und \texttt{\$f}-Anweisungslabels einschließen.

   \texttt{/linear} - 
    Nur ein Label pro Zeile anzeigen.  Dies kann für die Erstellung von Skripten in Verbindung mit den Dienstprogrammen unter dem Befehl \texttt{tools}\index{\texttt{tools}-Befehl} nützlich sein. 


\subsection{\texttt{show statement}-Befehl}\index{\texttt{show statement}-Befehl}

Syntax:  \texttt{show statement} {\em label-match} [{\em qualifiers} (siehe unten)]

Dieser Befehl liefert Informationen über eine Anweisung.  Es können nur Anweisungen mit Labels (\texttt{\$f}\index{\texttt{\$f}-Anweisung}, 
\texttt{\$e}\index{\texttt{\$e}-Anweisung}, 
\texttt{\$a}\index{\texttt{\$a}-Anweisung}, und 
\texttt{\$p}\index{\texttt{\$p}-Anweisung}) angegeben werden. 
Wenn {\em label-match} Platzhalterzeichen (\verb$*$) enthält, werden alle übereinstimmenden Anweisungen in der Reihenfolge angezeigt, in der sie in der Datenbasis vorkommen. 

Optionale Befehlszeilenparameter ({\em qualifiers}, es ist jeweils nur ein Parameter zulässig):

    \texttt{/comment} - 
    Diese Option schließt den Kommentar ein, der der Anweisung unmittelbar vorausgeht.

    \texttt{/full} - 
    Zeigt vollständige Informationen zu jeder Anweisung an, und zwar für alle Anweisungen, die mit {\em label} übereinstimmen (einschließlich der \texttt{\$e}- und \texttt{\$f}-Anweisungen).

    \texttt{/tex} - 
    Diese Option schreibt die Anweisungsinformationen in die \LaTeX-Datei, die zuvor mit \texttt{open tex} geöffnet wurde.  Siehe Abschnitt~\ref{texout}.

    \texttt{/simple{\char`\_}tex} - 
    Wie \texttt{/tex} mit dem Unterschied, dass \LaTeX-Makros nicht für die Formatierung von Gleichungen verwendet werden, was eine einfachere manuelle Bearbeitung der Ausgabe für Folienpräsentationen usw. ermöglicht.

    \texttt{/html}\index{HTML generierung@{\sc html} generation},
    \texttt{/alt{\char`\_}html}, \texttt{/brief{\char`\_}html},
    \texttt{/brief{\char`\_}alt{\char`\_}html} -
    Diese Optionen aktivieren einen speziellen Modus von \texttt{show statement}, der eine Webseite für die Anweisung erstellt.  Sie dürfen nicht zusammen mit einem anderen Qualifizierer verwendet werden.  Siehe Abschnitt~\ref{htmlout} oder \texttt{help html} im Programm. 


\subsection{\texttt{search}-Befehl}\index{\texttt{search}-Befehl}

Syntax:  search {\em label-match}
\texttt{\char`\"}{\em symbol-match}\texttt{\char`\"} [\texttt{/all}] [\texttt{/comments}]
[\texttt{/join}]

Dieser Befehl durchsucht alle \texttt{\$a}- und \texttt{\$p}-Anweisungen, die mit {\em label-match} übereinstimmen, nach Vorkommen von {\em symbol-match}.  Ein \verb@*@ in {\em label-match} entspricht einem beliebigen Label-Zeichen.  Ein \verb@$*@ in {\em symbol-match} passt auf eine beliebige Folge von Symbolen.  Die Symbole in {\em symbol-match} müssen durch einen Whitespace getrennt sein.  Die Anführungszeichen, die {\em symbol-match} umgeben, können einfache oder doppelte Anführungszeichen sein.  Zum Beispiel listet \texttt{search b}\verb@* "-> $* ch"@ alle Anweisungen auf, deren Label mit \texttt{b} beginnen und die Symbole \verb@->@ und \texttt{ch} enthalten, die eine beliebige Symbolfolge umgeben (einschließlich keiner Symbolfolge).  Die Platzhalter \texttt{?} und \texttt{\$?} sind auch verfügbar, um einzelne Zeichen in Labels bzw. Symbolen zu finden; siehe \texttt{help search} im Metamath-Programm für Details zu ihrer Verwendung. 

Optionale Befehlszeilenparameter:

    \texttt{/all} - Suche auch nach \texttt{\$e}- und \texttt{\$f}-Anweisungen.

    \texttt{/comments} - Sucht auch in dem Kommentar, der jeder Anweisung mit Label-Matching unmittelbar vorausgeht, nach {\em symbol-match}.  In diesem Fall ist {\em symbol-match} eine beliebige, nicht von Groß- und Kleinschreibung abhängige Zeichenkette.  Anführungszeichen um {\em symbol-match} sind optional, wenn es keine Mehrdeutigkeit gibt.

    \texttt{/join} - Im Falle einer \texttt{\$a}- oder \texttt{\$p}-Anweisung werden deren \texttt{\$e}-Hypothesen für die Suche vorangestellt. Die Option \texttt{/join} hat im Modus \texttt{/comments} keine Wirkung.


\section{Anzeigen und Verifizieren von Beweisen}

\subsection{\texttt{show proof}-Befehl}\index{\texttt{show proof}-Befehl}

Syntax:  \texttt{show proof} {\em label-match} [{\em qualifiers} (siehe unten)]

Mit diesem Befehl wird der Beweis der angegebenen \texttt{\$p}-Anweisung\index{\texttt{\$p}-Anweisung} in verschiedenen Formaten angezeigt. Der Parameter {\em label-match} kann Platzhalterzeichen (\verb@$*@) enthalten, um mehrere Anweisungen anzuzeigen.  Ohne Optionen ({\em qualifiers}) werden nur die logischen Schritte in einem eingerückten Format angezeigt (d.h. die Syntaxkonstruktionsschritte werden weggelassen). 

In den meisten Fällen werden Sie \texttt{show proof} {\em label} verwenden, um nur die Beweisschritte zu sehen, die logischen Schlussfolgerungen entsprechen. 

Optionale Befehlszeilenparameter:

    \texttt{/essential} - Der Beweisbaum wird vor der Anzeige um alle Hypothesen (\texttt{\$f}-Anweisungen\index{\texttt{\$f}-Anweisung}) bereinigt.  (Dies ist die Voreinstellung und muss deshalb nicht angegeben werden).

    \texttt{/all} - Der Beweisbaum wird vor der Anzeige nicht von allen \texttt{\$f}-Hypothesen bereinigt.  \texttt{/essential} und \texttt{/all} schließen sich gegenseitig aus.

    \texttt{/from{\char`\_}step} {\em step} - Die Anzeige beginnt mit dem angegebenen Schritt.  Wird diese Option weggelassen, beginnt die Anzeige beim ersten Schritt.

    \texttt{/to{\char`\_}step} {\em step} - Die Anzeige endet mit dem angegebenen Schritt.  Wird diese Option weggelassen, endet die Anzeige beim letzten Schritt.

    \texttt{/tree{\char`\_}depth} {\em number} - Es werden nur Schritte angezeigt, die weniger als die angegebene Tiefe des Beweisbaums haben.  Manchmal nützlich, um einen Überblick über den Beweis zu erhalten.

    \texttt{/reverse} - Die Schritte werden in umgekehrter Reihenfolge angezeigt.

    \texttt{/renumber} - Bei Verwendung mit \texttt{/essential} werden die Schritte neu nummeriert und entsprechen nur den wesentlichen Schritten.

    \texttt{/tex} - Der Beweis wird in \LaTeX\ konvertiert und\index{latex@{\LaTeX}} in der mit \texttt{open tex} geöffneten Datei gespeichert.  Siehe Abschnitt~\ref{texout} oder \texttt{help tex} im Programm.

    \texttt{/lemmon} - Der Beweis wird in einem nicht eingerückten Format angezeigt, das als Lemmon-Stil bekannt ist, mit expliziten Verweisen auf vorherige Schrittnummern. Wird diese Option weggelassen, werden die Schritte in einem Baumformat eingerückt.

    \texttt{/start{\char`\_}column} {\em number} - Setzt die Standardspalte (16) außer Kraft, bei der die Formelanzeige in einer Lemmon-Anzeige beginnt.  Kann nur in Verbindung mit \texttt{/lemmon} verwendet werden.

    \texttt{/normal} - Der Beweis wird in einem normalen Format angezeigt, das sich zur Aufnahme in eine Metamath-Quelldatei eignet.  Darf nicht mit einer anderen Option verwendet werden.

    \texttt{/compressed} - Der Beweis wird in einem komprimierten Format angezeigt, das sich zur Aufnahme in eine Metamath-Quelldatei eignet.  Darf nicht mit einer anderen Option verwendet werden.

    \texttt{/statement{\char`\_}summary} - Gibt einen Überblick über alle Anweisungen (wie mit \texttt{show statement}), die im Beweis verwendet werden.  Er darf nicht mit einer anderen Option außer \texttt{/essential} verwendet werden.

    \texttt{/detailed{\char`\_}step} {\em step} - Zeigt an, was in einem bestimmten Schritt des Beweises im Einzelnen geschieht.  Darf nicht mit einer anderen Option verwendet werden.  Der Schritt {\em step} ist die Schrittnummer, die angezeigt wird, wenn ein Beweis ohne die Option \texttt{/renumber} dargestellt wird.


\subsection{\texttt{show usage}-Befehl}\index{\texttt{show usage}-Befehl}

Syntax:  \texttt{show usage} {\em label-match} [\texttt{/recursive}]

Dieser Befehl listet die Anweisungen auf, deren Beweise sich direkt auf die angegebene Anweisung beziehen.

Optionaler Befehlszeilenparameter:

    \texttt{/recursive} - Dazu gehören auch Anweisungen, deren Beweise letztlich von der angegebenen Anweisung abhängen. 


\subsection{\texttt{show trace\_back}-Befehl}\index{\texttt{show
       trace{\char`\_}back}-Befehl}
   
Syntax:  \texttt{show trace{\char`\_}back} {\em label-match} [\texttt{/essential}] [\texttt{/axioms}]
    [\texttt{/tree}] {\\} [\texttt{/depth} {\em number}]

Dieser Befehl listet alle Anweisungen auf, von denen der Beweis der durch {\em label-match} angegebenen Anweisung(en) abhängt. 
    
Optionale Befehlszeilenparameter:

    \texttt{/essential} - Beschränkt die Rückverfolgung auf \texttt{\$e}\index{\texttt{\$e}-Anweisung}-Hypothesen von Beweisbäumen.

    \texttt{/axioms} - Führt nur die Axiome auf, von denen der Beweis letztlich abhängt.

    \texttt{/tree} - Anzeige der Rückverfolgung in einem eingerückten Baumformat.

    \texttt{/depth} {\em number} - Schränkt die \texttt{/tree}-Rückverfolgung auf die angegebene Einrückungstiefe ein.

    \texttt{/count{\char`\_}steps} - Zählt die Anzahl der Schritte, die der Beweis bis zu den Axiomen zurückführt.  Wenn \texttt{/essential} angegeben ist, werden Expansionen von Hypothesen vom Variablentyp (Syntaxkonstruktionen) nicht gezählt. 


\subsection{\texttt{verify proof}-Befehl}\index{\texttt{verify proof}-Befehl}

Syntax:  \texttt{verify proof} {\em label-match} [\texttt{/syntax{\char`\_}only}]

Mit diesem Befehl werden die Beweise der angegebenen Aussagen überprüft.  Die Option {\em label-match} kann Platzhalterzeichen (\texttt{*}) enthalten, um mehr als einen Beweis zu überprüfen; zum Beispiel wird \verb/*abc*def/ auf alle Labels passen, die \texttt{abc} enthalten und mit \texttt{def} enden. Der Befehl \texttt{verify proof *} prüft alle Beweise in der Datenbasis. 

Optionaler Befehlszeilenparameter:

    \texttt{/syntax{\char`\_}only} - Mit dieser Option wird nur auf Syntax- und RPN-Stack-Verletzungen geprüft.  Es wird nicht geprüft, ob der Beweis korrekt ist.  Diese Option ist nützlich um schnell festzustellen, welche Beweise unvollständig sind (d.h. in der Entwicklung sind und \texttt{?}'s in ihnen enthalten sind).

{\em Anmerkung:} \texttt{read}, gefolgt von \texttt{verify proof *}, stellt sicher, dass die Datenbasis frei von Fehlern in der Metamath-Sprache ist, überprüft aber nicht die Auszeichnungsnotation in Kommentaren. Sie können die Auszeichnungsnotation auch überprüfen, indem Sie \texttt{verify markup *} ausführen, wie in Abschnitt~\ref{verifymarkup} beschrieben; siehe auch die Diskussion über die Erzeugung von {\sc HTML} in Abschnitt~\ref{htmlout}. 


\subsection{\texttt{verify markup}-Befehl}\index{\texttt{verify markup}-Befehl}\label{verifymarkup}

Syntax:  \texttt{verify markup} {\em label-match}
[\texttt{/date{\char`\_}skip}]
[\texttt{/top{\char`\_}date{\char`\_}skip}] {\\}
[\texttt{/file{\char`\_}skip}]
[\texttt{/verbose}]

Dieser Befehl überprüft Kommentarauszeichnungen und andere informelle Konventionen, die wir festgesetzt haben.  Er prüft die latexdef-, htmldef- und althtmldef-Anweisungen in der \texttt{\$t}-Anweisung einer Metamath-Quelldatei auf Fehler. Es prüft alle \texttt{`}...\texttt{`}, \texttt{\char`\~}~\textit{Label} und bibliographischen Markierungen in Anweisungsbeschreibungen auf Fehler. Es wird geprüft, ob \texttt{\$p}- und \texttt{\$a}-Anweisungen den gleichen Inhalt haben, wenn ihre Labels mit "`ax"' bzw. "`ax-"' beginnen, aber ansonsten identisch sind, zum Beispiel ax4 und ax-4. Er überprüft auch die Datumsübereinstimmung von "`(Contributed by...)"', "`(Revised by...)"' und "`(Proof shortened by...)"' in dem Kommentar über jeder \texttt{\$a}- und \texttt{\$p}-Anweisung. 

Optionale Befehlszeilenparameter:

    \texttt{/date{\char`\_}skip} - Mit dieser Option wird die Prüfung der Datums\-über\-ein\-stimmung übersprungen, die normalerweise für andere Datenbasen als \texttt{set.mm} nicht erforderlich ist.

    \texttt{/top{\char`\_}date{\char`\_}skip} - Mit dieser Option wird die Datumsübereinstimmung geprüft, mit der Ausnahme, dass das Versionsdatum am Anfang der Datenbasisdatei nicht geprüft wird.  Es kann nur eine der beiden Optionen \texttt{/date{\char`\_}skip} und \texttt{/top{\char`\_}date{\char`\_}skip} angegeben werden.

    \texttt{/file{\char`\_}skip} - Mit dieser Option werden Prüfungen übersprungen, die das Vorhandensein externer Dateien voraussetzen, wie z. B. die Überprüfung des Vorhandenseins von GIFs und bibliografischen Links zu mmset.html oder Ähnlichem.  Dies ist nützlich, um eine schnelle Prüfung aus einem Verzeichnis ohne diese Dateien durchzuführen.

    \texttt{/verbose} - Liefert mehr Informationen.  Gegenwärtig wird eine Liste der Übereinstimmungen zwischen axXXX und ax-XXX angezeigt.


\subsection{\texttt{save proof}-Befehl}\index{\texttt{save proof}-Befehl}

Syntax:  \texttt{save proof} {\em label-match} [\texttt{/normal}]
   [\texttt{/compressed}]

   Der Befehl \texttt{save proof} formatiert einen Beweis in einem von zwei Formaten neu und ersetzt den vorhandenen Beweis im Quellpuffer\index{Quellpuffer}.  Er ist nützlich, um zwischen verschiedenen Formaten von Beweisen zu konvertieren.  Beachten Sie, dass ein Beweis erst dann dauerhaft gespeichert wird, wenn Sie den Befehl \texttt{write source} ausführen. 

Optionale Befehlszeilenparameter:

    \texttt{/normal} - Der Beweis wird im normalen Format gespeichert (d. h. als eine Folge von Labels, was dem definierten Format der Metamath-Basissprache entspricht).\index{grundlegende Sprache}  Dies ist das Standardformat, das verwendet wird, wenn keine Option angegeben wird.

    \texttt{/compressed} - Der Beweis wird im komprimierten Format gespeichert, was den Speicherbedarf für eine Datenbasis reduziert. Siehe Anhang~\ref{compressed}.


\section{Beweise erstellen}\label{pfcommands}\index{Beweis-Assistent}

Bevor Sie den Beweis-Assistenten verwenden, müssen Sie (mit einem Texteditor) eine \texttt{\$p}-Anweisung in Ihre Quelldatei einfügen, der die zu beweisende Aussage enthält.  Der Beweis sollte aus einem einzigen \texttt{?} bestehen, was "`unbekannter Schritt"' bedeutet.  Beispiel:
\begin{verbatim}
equid $p x = x $= ? $.
\end{verbatim}

Um den Beweis-Assistenten aufzurufen, geben Sie \texttt{prove} {\em label} ein, z.B. \texttt{prove equid}.  Metamath antwortet mit der Eingabeaufforderung \texttt{MM-PA>}.

Beweise werden ausgehend von der zu beweisenden Anweisung rückwärts erstellt, wobei hauptsächlich eine Reihe von \texttt{assign}-Befehlen verwendet wird.  Ein Beweis ist vollständig, wenn allen Schritten Anweisungen zugewiesen sind und alle Schritte vereinheitlicht wurden und vollständig bekannt sind.  Während der Erstellung eines Beweises lässt Metamath nur Operationen zu, die aufgrund der bis dahin bekannten Daten zulässig sind.  So wird zum Beispiel kein \texttt{assign} mit einer Anweisung zugelassen, die nicht in den unbekannten Beweisschritt, für den die Zuweisung erfolgen soll, eingesetzt werden kann. 

{\em Wichtig:} Der Beweis-Assistent ist kein Werkzeug, das Ihnen hilft, Beweise zu finden.  Er ist nur ein Hilfsmittel, das Ihnen hilft, Beweise zur Datenbasis hinzuzufügen.  Eine Anleitung dazu finden Sie in Abschnitt~\ref{frstprf}. Um die Verwendung des Beweis-Assistenten zu üben, können Sie ein bestehendes Theorem mit  \texttt{prove} bearbeiten, dann alle Schritte mit \texttt{delete all} löschen und den Bewqeis dann mit dem Beweis-Assistenten neu erstellen, während Sie den (vor dem Löschen) angezeigten Beweises betrachten. Es kann sinnvoll sein, die ersten Beweise vollständig selbst zu erarbeiten und von Hand aufzuschreiben, bevor Sie den Beweis-Assistenten benutzen, auch wenn das nicht für jeden geeignet ist. 

{\em Wichtig:} Der Befehl \texttt{undo} ist sehr hilfreich bei der Eingabe eines Beweises, da Sie damit einen zuvor eingegebenen Schritt rückgängig machen können. Außerdem empfehlen wir Ihnen, Ihre Arbeit in einer Protokolldatei (\texttt{open log}) festzuhalten und sie regelmäßig zu speichern (\texttt{save new{\char`\_}proof}, \texttt{write source}). Sie können \texttt{delete} verwenden, um ein \texttt{assign} rückgängig zu machen. Sie können auch \texttt{delete floating{\char`\_}hypotheses}, dann \texttt{initialize all} und dann \texttt{unify all /interactive} verwenden, um ungewollte Vereinheitlichungen, die versehentlich oder durch unpassende \texttt{assign}s gemacht wurden, zu reinitialisieren.  Sie können ein \texttt{delete} nicht rückgängig machen, es sei denn, Sie verwenden ein entsprechendes \texttt{undo} oder \texttt{exit /force} und rufen dann den Beweis-Assistenten erneut auf, um den letzten \texttt{save new{\char`\_}proof} wiederherzustellen.

Die folgenden Befehle stehen im Proof-Assistenten (an der Eingabeaufforderung \texttt{MM-PA>}) zur Verfügung, um Sie bei der Erstellung Ihres Beweises zu unterstützen.  Siehe die einzelnen Befehle für weitere Details. 

\begin{itemize}
\item[]
    \texttt{show new{\char`\_}proof} [\texttt{/all},...] - Zeigt den aktuellen Beweis an.  Sie werden diesen Befehl häufig verwenden; siehe \texttt{help show new{\char`\_}proof}, um sich mit seinen Optionen vertraut zu machen.  Die Optionen \texttt{/unknown} und \texttt{/not{\char`\_}unified} sind nützlich, um die noch zu erledigende Arbeit anzuzeigen.  Die Kombination \texttt{/all/unknown} ist nützlich, um Dummy-Variablen zu identifizieren, die zugewiesen werden müssen, oder Versuche, ungültige Syntax zu verwenden, wenn \texttt{improve all} nicht in der Lage ist, die Syntaxkonstruktionen abzuschließen.  Unbekannte Variablen werden als \texttt{\$1}, \texttt{\$2}, ... angezeigt.
\item[]
    \texttt{assign} {\em step} {\em label} - Weist einem noch nicht zugeordneten Beweisschritt mit der {\em step}-Nummer die durch {\em label} angegebene Anweisung zu.
\item[]
    \texttt{let variable} {\em variable}
        \texttt{= \char`\"}{\em symbol sequence}\texttt{\char`\"}
          - Erzwingt die Ersetzung einer unbekannten Variablen (z. B. \texttt{\$1}) in einem Beweis durch eine Symbolfolge. Dies ist nützlich, um schwierige Vereinheitlichungen zu erleichtern, und es ist notwendig, wenn Sie Dummy-Variablen nutzen, denen schließlich ein Name zugewiesen werden muss.
\item[]
    \texttt{let step} {\em step} \texttt{= \char`\"}{\em symbol sequence}\texttt{\char`\"} - Erzwingt, dass eine Symbolsequenz den Inhalt eines Beweisschritts ersetzt, sofern sie mit dem vorhandenen Schrittinhalt vereinheitlicht werden kann.  (Ich verwende dies selten.)
\item[]
    \texttt{unify step} {\em step} (oder \texttt{unify all}) - Vereinheitlicht die Quelle und das Ziel eines Schrittes.  Wenn Sie einen bestimmten Schritt angeben, werden Sie aufgefordert, eine der möglichen Vereinheitlichungen auszuwählen.  Wenn Sie \texttt{all} angeben, werden alle Schritte mit eindeutigen Vereinheitlichungen vereinheitlicht, aber nur diese Schritte.  \texttt{unify all /interactive} geht durch alle nicht vereinheitlichten Schritte.
\item[]
    \texttt{initialize} {\em step} (oder \texttt{all}) - Entkoppelt das Ziel und die Quelle eines Schritts (oder aller Schritte) sowie die Hypothesen der Quelle und macht alle Variablen in der Quelle unbekannt.  Nützlich, um einen \texttt{assign}- oder \texttt{let}-Fehler zu beheben, der zu falschen Vereinheitlichungen führte.
\item[]
    \texttt{delete} {\em step} (oder \texttt{all} oder \texttt{floating{\char`\_}hypotheses}) - Löscht den/die angegebenen Schritt(e).  \texttt{delete floating{\char`\_}hypotheses}, dann\linebreak
    \texttt{initialize all}, dann \texttt{unify all /interactive} ist nützlich, um Fehler zu beheben, bei denen falsche Vereinheitlichungen den Variablen falsche mathematische Symbolfolgen zugewiesen haben.
\item[]
    \texttt{improve} {\em step} (oder \texttt{all}) - Erstellt automatisch einen Beweis für Schritte (ohne unbekannte Variablen), deren Beweis keine Anweisungen mit \texttt{\$e}-Hypothesen erfordert.  Nützlich zum Ausfüllen von Beweisen für \texttt{\$f}-Hypothesen.  Die Option \texttt{/depth} versucht dies auch mit Anweisungen, deren \texttt{\$e}-Hypothesen keine neuen Variablen enthalten.  {Warnung:} Speichern Sie Ihre Arbeit (mit \texttt{save new{\char`\_}proof} und dann \texttt{write source}), bevor Sie \texttt{/depth = 2} oder größer verwenden, da die Suchzeit exponentiell ansteigt und möglicherweise nie in einer angemessenen Zeit beendet wird, und Sie können die Suche nicht unterbrechen.  Ich habe festgestellt, dass \texttt{/depth = 3} oder größer nur selten nützlich ist.
 \item[]
    \texttt{save new{\char`\_}proof} - Speichert den laufenden Beweis im internen Datenbasispuffer des Programms.  Um ihn dauerhaft in der Datenbasisdatei zu speichern, verwenden Sie \texttt{write source} nach \texttt{save new{\char`\_}proof}.  Um zum letzten \texttt{save new{\char`\_}proof} zurückzukehren, beenden Sie mit \texttt{exit /force} den Beweis-Assistenten und starten Sie ihn dann erneut.
 \item[]
    \texttt{match step} {\em step} (oder \texttt{match all}) - Zeigt an, welche Anweisungen für die Anweisung \texttt{assign} möglich sind. (Dieser Befehl ist in seiner jetzigen Form nicht sehr nützlich und wird hoffentlich in Zukunft verbessert werden.  Verwenden Sie in der Zwischenzeit die Anweisung \texttt{search} für Kandidaten, die auf bestimmte Kombinatiuonen von mathematische Token passen.)
 \item[]
 \texttt{minimize{\char`\_}with}\index{\texttt{minimize{\char`\_}with}-Befehl}
% 3/10/07 Note: line-breaking the above results in duplicate index entries
     - Nachdem ein Beweis abgeschlossen ist, können mit diesem Befehl andere Datenbasistheoreme mit dem Beweis abgeglichen werden, um damit ggf. die Größe des Beweises zu verringern.  Siehe \texttt{help minimize{\char`\_}with} im Metamath-Programm für seine Verwendung.
 \item[]
 \texttt{undo}\index{\texttt{undo}-Befehl}
    - Macht die Wirkung eines den Beweis verändernden Befehls rückgängig (alle Befehle außer den oben genannten \texttt{show} und \texttt{save}).
 \item[]
 \texttt{redo}\index{\texttt{redo}-Befehl}
    - Macht ein vorheriges \texttt{undo} wieder rückgängig.
\end{itemize}

Die folgenden Befehle setzen Parameter, die für Ihren Beweis relevant sein können. Konsultieren Sie die einzelnen \texttt{help set}... Befehle. \begin{itemize}
   \item[] \texttt{set unification{\char`\_}timeout}
 \item[]
    \texttt{set search{\char`\_}limit}
  \item[]
    \texttt{set empty{\char`\_}substitution} - Beachten Sie, dass der Standardwert \texttt{off} ist.
\end{itemize}

Geben Sie \texttt{exit} ein, um die Eingabeaufforderung \texttt{MM-PA>} zu verlassen und zur Eingabeaufforderung \texttt{MM>} zurückzukehren. Ein weiteres \texttt{exit} beendet dann Metamath komplett. 


\subsection{\texttt{prove}-Befehl}\index{\texttt{prove}-Befehl}

Syntax:  \texttt{prove} {\em label}

Mit diesem Befehl wird der Beweis-Assistent aufgerufen, mit dem Sie den Beweis der angegebenen Aussage erstellen oder bearbeiten können. Die Eingabeaufforderung in der Befehlszeile ändert sich von \texttt{MM>} zu \texttt{MM-PA>}. 

Hinweis: In der aktuellen Version (0.177) von Metamath\index{Metamath!Limitationen der Version 0.177} prüft der Beweis-Assistent nicht, ob die Einschränkungen für \texttt{\$d}\index{\texttt{\$d}-Anweisung}-Anweisungen eingehalten werden, während ein Beweis erstellt wird.  Nachdem Sie einen Beweis abgeschlossen haben, sollten Sie \texttt{save new{\char`\_}proof} gefolgt von \texttt{verify proof} {\em label} (wobei {\em label} die Anweisung ist, die Sie mit dem Befehl \texttt{prove} beweisen) eingeben, um die \texttt{\$d}-Einschränkungen zu überprüfen. 

Siehe auch: \texttt{exit}


\subsection{\texttt{set unification\_timeout}-Befehl}\index{\texttt{set
unification{\char`\_}timeout}-Befehl}

Syntax:  \texttt{set unification{\char`\_}timeout} {\em number}

(Dieser Befehl ist auch außerhalb des Beweis-Assistenten verfügbar, wirkt sich aber nur auf den Beweis-Assistenten\index{Beweis-Assistent} aus). 

Manchmal meldet der Beweis-Assistent, dass eine Zeitüberschreitung beim Vereinheitlichen aufgetreten ist.  Dies kann passieren, wenn Sie versuchen, Formeln mit vielen temporären Variablen\index{temporäre Variable} (\texttt{\$1}, \texttt{\$2}, usw.) zu vereinheitlichen, da die Zeit für die Berechnung aller möglichen Vereinheitlichungen exponentiell mit der Anzahl der Variablen wachsen kann.  Wenn Sie möchten, dass Metamath sich mehr Mühe gibt (und Sie bereit sind, länger zu warten), können Sie diesen Parameter erhöhen.  \texttt{show settings} zeigt Ihnen den aktuellen Wert an. 


\subsection{\texttt{set empty\_substitution}-Befehl}\index{\texttt{set
empty{\char`\_}substitution}-Befehl}
% These long names can't break well in narrow mode, and even "`sloppy"'
% is not enough. Work around this by not demanding justification.

\begin{flushleft}
Syntax:  \texttt{set empty{\char`\_}substitution on} oder \texttt{set
empty{\char`\_}substitution off}
\end{flushleft}

(Dieser Befehl ist auch außerhalb des Proof-Assistenten verfügbar, wirkt sich aber nur auf den Beweis-Assistenten\index{Beweis-Assistent} aus). 

Die Metamath-Sprache erlaubt es, Variablen durch leere Symbolfolgen\index{leere Substitution} zu ersetzen\index{Substitution!Variable}\index{Variablensubstitution}.  In vielen formalen Systemen\index{formales System} wird dies jedoch nie in einem gültigen Beweis vorkommen.  Die Berücksichtigung dieser Möglichkeit erhöht die Wahrscheinlichkeit mehrdeutiger Vereinheitlichungen\index{mehrdeutige Vereinheitlichung}\index{Vereinheitlichung!mehrdeutig} während der Beweiserstellung. Standardmäßig sind leere Substitutionen nicht erlaubt; für formale Systeme, die sie erfordern, müssen Sie \texttt{set empty{\char`\_}substitution on} setzen. Ein Beispiel, bei dem dieser Parameter \texttt{on} sein muss, wäre ein System, das eine Deduktionsregel implementiert und in dem Deduktionen von leeren Annahmelisten zulässig wären.  Das im Anhang~\ref{MIU} beschriebene MIU-System\index{MIU-System} ist ein weiteres Beispiel.

Es ist besser, diesen Befehl auszuschalten (auf \texttt{off} zu setzen oder zu belassen), wenn Sie mit \texttt{set.mm} arbeiten. Beachten Sie, dass dieser Befehl keinen Einfluss darauf hat, wie Beweise mit dem Befehl \texttt{verify proof} überprüft werden.  Außerhalb des Beweis-Assistenten ist die Ersetzung von leeren Sequenzen für mathematische Symbole immer erlaubt. 


\subsection{\texttt{set search\_limit}-Befehl}\index{\texttt{set
search{\char`\_}limit}-Befehl} 

Syntax:  \texttt{set search{\char`\_}limit} {\em number}

(Dieser Befehl ist auch außerhalb des Proof-Assistenten verfügbar, wirkt sich aber nur auf den Beweis-Assistenten\index{Beweis-Assistent} aus). 

Dieser Befehl legt einen Parameter fest, der bestimmt, wann der Befehl \texttt{improve} im Modus Beweis-Assistent seine Suche nach Vereinheitlichungen beendet.  Wenn Sie möchten, dass \texttt{improve} intensiver sucht, können Sie den Wert erhöhen.  Der Befehl \texttt{show settings} zeigt Ihnen den aktuellen Wert an. 


\subsection{\texttt{show new\_proof}-Befehl}\index{\texttt{show new{\char`\_}proof}-Befehl}

Syntax:  \texttt{show new{\char`\_}proof} [{\em Optionen} (siehe unten)]

Dieser Befehl (nur im Modus Beweis-Assistent verfügbar) zeigt den aktuellen Beweis an.  Er ist identisch mit dem Befehl \texttt{show proof} mit dem Unterschied, dass es kein Argument für die Aussage gibt (da es sich um die zu beweisende Aussage handelt). Außerdem sind die folgenden Optionen nicht verfügbar: 

    \texttt{/statement{\char`\_}summary}

    \texttt{/detailed{\char`\_}step}

Es sind aber die folgenden zusätzlichen Optionen verfügbar:

    \texttt{/unknown} - Zeigt nur Schritte an, denen keine Anweisung zugewiesen ist.

    \texttt{/not{\char`\_}unified} - Zeigt nur Schritte an, die noch nicht vereinheitlicht wurden.

Beachten Sie, dass \texttt{/essential}, \texttt{/depth}, \texttt{/unknown} und \texttt{/not{\char`\_}unified} in jeder beliebigen Kombination verwendet werden können; jede von ihnen filtert effektiv zusätzliche Schritte aus der Beweisanzeige heraus.

Siehe auch:  \texttt{show proof}


\subsection{\texttt{assign}-Befehl}\index{\texttt{assign}-Befehl}

Syntax:   \texttt{assign} {\em step} {\em label} [\texttt{/no{\char`\_}unify}]

   und:   \texttt{assign first} {\em label}

   und:   \texttt{assign last} {\em label}

Dieser Befehl, der nur im Beweis-Assistenten verfügbar ist, ordnet einem unbekannten (d.h. noch nicht zugeordneten) Schritt (einen mit \texttt{?} in der Auflistung \texttt{show new{\char`\_}proof}) die durch {\em label} angegebene Anweisung zu.  Die Zuordnung wird nicht zugelassen, wenn die Anweisung nicht mit dem Schritt vereinheitlicht werden kann. 

Wenn \texttt{last} anstelle der {\em step}-Nummer angegeben wird, wird der letzte Schritt, der von \texttt{show new{\char`\_}proof /unknown} angezeigt wird, verwendet.  Dies kann für die Erstellung eines Beweises mit einer Befehlsdatei nützlich sein (siehe \texttt{help submit}).  Es beschleunigt auch das Erstellen von Beweisen, wenn Sie die Zuordnung für den letzten Schritt kennen. 

Wenn \texttt{first} anstelle der {\em step}-Nummer angegeben wird, wird der erste Schritt verwendet, der durch \texttt{show new{\char`\_}proof /unknown} angezeigt wird. 

Wenn {\em step} 0 oder negativ ist, wird der -{\em step}-te von dem letzten unbekannten Schritt, wie durch \texttt{show new{\char`\_}proof /unknown} gezeigt, verwendet.  \texttt{assign -1} {\em label} weist den vorletzten unbekannten Schritt zu, \texttt{assign -2} {\em label} den vorvorletzten, und \texttt{assign 0} {\em label} ist dasselbe wie \texttt{assign last} {\em label}. 

Optionaler Befehlszeilenparameter:

    \texttt{/no{\char`\_}unify} - der Benutzer wird nicht aufgefordert, eine Vereinheitlichung zu wählen, wenn es mehr als eine Möglichkeit gibt.  Dies ist nützlich für nicht-interaktive Befehlsdateien.  Später kann der Benutzer \texttt{unify all /interactive} wählen. (Die Zuweisung wird immer noch automatisch vereinheitlicht, wenn es nur eine Möglichkeit gibt, und wird abgelehnt, wenn eine Vereinheitlichung nicht möglich ist).


\subsection{\texttt{match}-Befehl}\index{\texttt{match}-Befehl}

Syntax:  \texttt{match step} {\em step} [\texttt{/max{\char`\_}essential{\char`\_}hyp}
{\em number}]

    und:  \texttt{match all} [\texttt{/essential}]
          [\texttt{/max{\char`\_}essential{\char`\_}hyp} {\em number}]

Dieser Befehl, der nur im Beweis-Assistenten verfügbar ist, zeigt an, welche Anweisungen mit dem/den angegebenen Schritt(en) vereinigt werden können.  {\em Hinweis:} In seiner jetzigen Form ist dieser Befehl nicht sehr nützlich, da er eine große Anzahl von Übereinstimmungen anzeigt. Er kann in Zukunft verbessert werden.  In der Zwischenzeit kann der Befehl \texttt{search} oft eine bessere Möglichkeit für das Auffinden von Theoremen von Interesse bieten. 

Optionale Befehlszeilenparameter:

    \texttt{/max{\char`\_}essential{\char`\_}hyp} {\em number} - filtert aus der Liste alle Anweisungen mit mehr als der angegebenen Anzahl von \texttt{\$e}\index{\texttt{\$e}-Anweisung}-Anweisungshypothesen heraus.

    \texttt{/essential{\char`\_}only} - in der Anweisung \texttt{match all} werden nur die Schritte abgeglichen, die in der Anzeige \texttt{show new{\char`\_}proof /essential} aufgelistet wären.


\subsection{\texttt{let}-Befehl}\index{\texttt{let}-Befehl}

Syntax: \texttt{let variable} {\em variable} = \texttt{\char`\"}{\em symbol-sequence}\texttt{\char`\"}

  und: \texttt{let step} {\em step} = \texttt{\char`\"}{\em symbol-sequence}\texttt{\char`\"}

Diese Befehle, die nur im Beweis-Assistenten\index{Beweis-Assistent} verfügbar sind, weisen einer temporären Variable\index{temporäre Variable} oder einem unbekannten Schritt eine bestimmte Symbolfolge zu.  Sie sind während der Erstellung eines Beweises nützlich, wenn Sie wissen, was in dem Beweisschritt enthalten sein soll, der Vereinheitlichungsalgorithmus aber noch nicht genügend Informationen hat, um die temporären Variablen vollständig zu ermitteln.  Eine "`temporäre Variable"' ist eine Variable, die in der Beweisanzeige die Form \texttt{\$}{\em nn} hat, wie z.B. \texttt{\$1}, \texttt{\$2}, usw.  Die {\em Symbolfolge} kann auch andere unbekannte Variablen enthalten, falls gewünscht.  Beispiele: 

    \verb/let variable $32 = "A = B"/

    \verb/let variable $32 = "A = $35"/

    \verb/let step 10 = '|- x = x'/

    \verb/let step -2 = "|- ( $7 = ph )"/

Für den Befehl \texttt{let variable} wird jede beliebige Symbolfolge akzeptiert.  Für \texttt{let step} werden nur solche Symbolfolgen akzeptiert, die mit dem Schritt vereinheitlicht werden können. 

Die \texttt{let}-Befehle "`knallen"' Informationen in den Beweis, die nur verifiziert werden können, wenn der Beweis weiter aufgebaut wird.  Wenn Sie einen \mbox{Fehler} machen, kann die Befehlssequenz \texttt{delete floating{\char`\_}hypotheses},
\texttt{initialize all} und \texttt{unify all /interactive} eine falsche \texttt{let}-Zuweisung rückgängig machen. 

Wenn {\em step} 0 oder negativ ist, wird der -{\em step}-te vom letzten unbekannten Schritt, wie durch \texttt{show new{\char`\_}proof /unknown} gezeigt, verwendet.  Der Befehl \texttt{let step 0} = \texttt{\char`\"}{\em symbol-sequence}\texttt{\char`\"} verwendet den letzten unbekannten Schritt, \texttt{let step -1} = \texttt{\char`\"}{\em symbol-sequence}\texttt{\char`\"} den vorletzten, usw. Wenn {\em step} positiv ist, kann \texttt{let step} verwendet werden, um sowohl bekannte (im Sinne von zuvor mit \texttt{assign} ein Label zugewiesen) als auch unbekannte Schritte zuzuweisen. 

Einfache oder doppelte Anführungszeichen können {\em symbol-sequence} um\-ge\-ben, solange sie sich von allen Anführungszeichen innerhalb einer {\em symbol-sequence} unterscheiden.  Wenn {\em symbol-sequence} beide Arten von Anführungs\-zeichen enthält: siehe die Anweisungen am Ende von \texttt{help let} im Metamath-Programm. 


\subsection{\texttt{unify}-Befehl}\index{\texttt{unify}-Befehl}

Syntax:  \texttt{unify step} {\em step}

      und:   \texttt{unify all} [\texttt{/interactive}]

Diese Befehle, die nur im Beweis-Assistenten verfügbar sind, vereinheitlichen die Quelle und das Ziel des/der angegebenen Schrittes/Schritte. Wenn Sie einen bestimmten Schritt angeben, werden Sie aufgefordert, eine der möglichen Vereinheitlichungen auszuwählen.  Wenn Sie \texttt{all} angeben, werden nur die Schritte mit eindeutigen Vereinheitlichungen vereinheitlicht. 

Optionaler Befehlszeilenparameter für \texttt{unify all}:

    \texttt{/interactive} - Sie werden aufgefordert, eine der möglichen Vereinheitlichungen für alle Schritte auszuwählen, die keine eindeutigen Vereinheitlichungen haben.  (Andernfalls wird \texttt{unify all} diese Schritte übergehen).

Siehe auch \texttt{set unification{\char`\_}timeout}.  Der Standardwert ist 100000, aber eine Erhöhung auf 1000000 kann in problematischen Fällen helfen.  Die manuelle Zuweisung einiger oder aller unbekannten Variablen mit dem Befehl \texttt{let variable} hilft ebenfalls in schwierigen Fällen. 


\subsection{\texttt{initialize}-Befehl}\index{\texttt{initialize}-Befehl}

Syntax:  \texttt{initialize step} {\em step}

    und: \texttt{initialize all}

Diese Befehle, die nur im Beweis-Assistenten\index{Beweis-Assistent} verfügbar sind, "`ent-unifizie\-ren"' das Ziel und die Quelle eines Schrittes (oder aller Schritte), sowie die Hypothesen der Quelle, und machen alle Variablen in der Quelle und die Hypothesen der Quelle unbekannt.  Dieser Befehl ist nützlich, um einen Beweis nach einer falsche Vereinheitlichungen, die durch ein falsches \texttt{assign}, ein falsches \texttt{let} oder einer falschen automatischen Zuordnung entstanden sind, wiederherzustellen.  Ein Teil oder die gesamte Befehlssequenz \texttt{delete floating{\char`\_}hypotheses}, \texttt{initialize all} und \texttt{unify all /interactive}\linebreak
wird den Beweis nach falschen Vereinheitlichungen wiederherstellen.

Siehe auch:  \texttt{unify} und \texttt{delete}


\subsection{\texttt{delete}-Befehl}\index{\texttt{delete}-Befehl}
Syntax:  \texttt{delete step} {\em step}

   und:      \texttt{delete all} -- {\em Achtung: Gefährlich!}

   und:      \texttt{delete floating{\char`\_}hypotheses}

Diese Befehle sind nur im Beweis-Assistenten verfügbar.

Der Befehl \texttt{delete step} löscht den Abschnitt des Beweisbaums, der von dem angegebenen Schritt abzweigt, und lässt den Schritt unbekannt werden. \texttt{delete all} ist äquivalent zu \texttt{delete step} {\em step}, wobei {\em step} der letzte Schritt im Beweis ist (d.h. \ der Anfang des Beweisbaums). 

In den meisten Fällen ist der Befehl \texttt{undo} die beste Methode, um einen vorherigen Schritt rückgängig zu machen. Eine Alternative ist, den letzten Beweis zu speichern, indem Sie den Beweis-Assistenten verlassen und erneut aufrufen. Damit dies funktioniert, sollten Sie eine Protokolldatei öffnen, um Ihre Arbeit zu protokollieren, und den Befehl \texttt{save new{\char`\_}proof} häufig ausführen, insbesondere vor \texttt{delete}. 

\texttt{delete floating{\char`\_}hypotheses} löscht alle Abschnitte des Beweises, die von \texttt{\$f}\index{\texttt{\$f}-Anweisung}-Anweisungen abzweigen.  Es ist manchmal nützlich, dies vor einem \texttt{initialize}-Befehl zu tun, um einen Fehler zu beheben.  Sobald ein Beweis\-schritt mit einer \texttt{\$f}-Hypothese als Ziel vollständig bekannt ist, kann der Befehl \texttt{improve} normalerweise den Beweis für diesen Schritt ausfüllen.  Im Gegensatz zum Löschen von logischen Schritten ist \texttt{delete} \texttt{floating{\char`\_}hypotheses} ein relativ sicherer Befehl, nach dem der Beweis normalerweise leicht wiederhergestellt werden kann. 


\subsection{\texttt{improve}-Befehl}\index{\texttt{improve}-Befehl}
\label{improve}

Syntax:  \texttt{improve} {\em step} [\texttt{/depth} {\em number}]
                                               [\texttt{/no{\char`\_}distinct}]

   und:   \texttt{improve first} [\texttt{/depth} {\em number}]
                                              [\texttt{/no{\char`\_}distinct}]

   und:   \texttt{improve last} [\texttt{/depth} {\em number}]
                                              [\texttt{/no{\char`\_}distinct}]

   und:   \texttt{improve all} [\texttt{/depth} {\em number}]
                                              [\texttt{/no{\char`\_}distinct}]

Diese Befehle, die nur im Beweis-Assistenten\index{Beweis-Assistent} verfügbar sind, versuchen, automatisch Beweise für unbekannte Schritte zu finden, deren Symbolfolgen vollständig bekannt sind.  Sie sind in erster Linie zum Ausfüllen von Beweisen für \texttt{\$f}\index{\texttt{\$f}-Anweisung}-Hypothesen nützlich.  Die Suche wird auf Anweisungen beschränkt, die keine \texttt{\$e}\index{\texttt{\$e}-Anweisung}-Hypothesen enthalten. 

\begin{sloppypar} % narrow
Hinweis: Wenn der Speicher begrenzt ist, kann \texttt{improve all} bei einem großen Beweis den Speicher überlaufen lassen.  Wenn Sie \texttt{set unification{\char`\_}timeout 1} vor \texttt{improve all} verwenden, wird in der Regel eine ausreichende Verbesserung erzielt, um den Beweis später auf einem größeren Computer leicht wiederherzustellen und mittels \texttt{improve} zu vervollständigen.  Warnung:  Wenn der Speicher einmal übergelaufen ist, gibt es keine Möglichkeit für eine Wiederherstellung mehr.  Speichern Sie im Zweifelsfall den Zwischenbeweis (\texttt{save new{\char`\_}proof} und danach \texttt{write source}) vor \texttt{improve all}. 
\end{sloppypar}

Wenn \texttt{last} anstelle von {\em step} number angegeben wird, wird der letzte Schritt, der durch \texttt{show new{\char`\_}proof /unknown} angezeigt wird, verwendet. 

Wenn \texttt{first} anstelle der {\em step} Nummer angegeben wird, wird der erste Schritt verwendet, der durch \texttt{show new{\char`\_}proof /unknown} angezeigt wird. 

Wenn {\em step} 0 oder negativ ist, wird der -{\em step}-te von dem letzten unbekannten Schritt, wie durch \texttt{show new{\char`\_}proof /unknown} gezeigt, verwendet.  \texttt{improve -1} verwendet den vorletzten unbekannten Schritt, \texttt{improve -2} {\em label} den vorvorletzten, und \texttt{improve 0} ist dasselbe wie \texttt{improve last}. 

Optionaler Befehlszeilenparameter:

    \texttt{/depth} {\em number} - Diese Option bewirkt, dass bei der Suche auch Anweisungen mit \texttt{\$e}-Hypothesen (aber keine neuen Variablen in den \texttt{\$e}-Hypothesen) berücksichtigt werden, sofern das Backtracking die angegebene Tiefe nicht überschritten hat. {\em Warnung:}  Versuchen Sie \texttt{/depth 1}, dann \texttt{2}, dann \texttt{3} usw. der Reihe nach wegen möglicher exponentieller Blowups.  Speichern Sie Ihre Arbeit, bevor Sie \texttt{/depth} größer als \texttt{1} ausprobieren!

    \texttt{/no{\char`\_}distinct} - Überspringt Anweisungen, welche \texttt{\$d}\index{\texttt{\$d}-Anweisung}-Anforderungen haben. Diese Option verhindert Zuweisungen, die gegen \texttt{\$d}-Anforderungen verstoßen könnten, aber er könnten dann auch mögliche legale Zuweisungen übersehen werden.

Siehe auch: \texttt{set search{\char`\_}limit}

\subsection{\texttt{save new\_proof}-Befehl}\index{\texttt{save
new{\char`\_}proof}-Befehl}

Syntax:  \texttt{save new{\char`\_}proof} {\em label} [\texttt{/normal}]
   [\texttt{/compressed}]

Der Befehl \texttt{save new{\char`\_}proof} ist nur im Beweis-Assistenten verfügbar.  Er speichert den laufenden Beweis in den Quellpuffer\index{Quellpuffer}.  \texttt{save new{\char`\_}proof} kann verwendet werden, um einen fertigen Beweis zu speichern, oder um einen sich in Bearbeitung befindenden Beweis zu speichern, um ihn später weiter zu bearbeiten.  Wenn ein unvollständiger Beweis gespeichert wird, gehen alle Benutzerzuweisungen mit \texttt{let step} oder \texttt{let variable} verloren, ebenso wie alle mehrdeutigen Vereinheitlichungen\index{mehrdeutige Vereinheitlichung}\index{Vereinheitlichung!mehrdeutig}, die manuell aufgelöst wurden. Um die Wiederherstellung zu erleichtern, kann es hilfreich sein, \texttt{improve all} vor \texttt{save new{\char`\_}proof} zu verwenden, damit der unvollständige Beweis so viele Informationen wie möglich enthält. 

Beachten Sie, dass der Beweis erst dann dauerhaft gespeichert wird, wenn der Befehl \texttt{write source} aufgwerufen wird. 

Optionale Befehlszeilenparameter:

    \texttt{/normal} - Der Beweis wird im normalen Format gespeichert (d. h. als eine Folge von Labels, was dem definierten Format der Metamath-Basissprache entspricht)\index{grundlegende Sprache}.  Dies ist das Standardformat, das verwendet wird, wenn ein Qualifier weggelassen wird.

    \texttt{/compressed} - Der Beweis wird im komprimierten Format gespeichert, was den Speicherbedarf für eine Datenbasis reduziert. (Siehe Anhang~\ref{compressed}.)


\section{Erstellen von \LaTeX-Ausgaben}\label{texout}\index{latex@{\LaTeX}}

Sie können \LaTeX-Ausgaben anhand der Informationen in einer Datenbasis erzeugen. Die Datenbasis muss bereits die erforderlichen Schriftsatzinformationen enthalten (siehe Abschnitt \ref{tcomment} für die Bereitstellung dieser Informationen). 

Die Befehle \texttt{show statement} und \texttt{show proof} haben jeweils einen spe\-zi\-el\-len \texttt{/tex} Befehlszeilenparameter, der eine \LaTeX-Ausgabe erzeugt (Der Befehl \texttt{show statement} verfügt auch über die Option \texttt{/simple{\char`\_}tex} für eine Ausgabe, die leichter von Hand zu bearbeiten ist).  Bevor Sie diese Befehle verwenden können, müssen Sie eine \LaTeX-Datei öffnen, an die Sie ihre Ausgabe senden können.  Eine typische vollständige Sitzung verwendet diese Abfolge von Metamath-Befehlen: 

\begin{verbatim}
read set.mm
open tex example.tex
show statement a1i /tex
show proof a1i /all/lemmon/renumber/tex
show statement uneq2 /tex
show proof uneq2 /all/lemmon/renumber/tex
close tex
\end{verbatim}

Siehe Abschnitt~\ref{mathcomments} für Informationen über Kommentarauszeichnungen und Anhang~\ref{ASCII} für Informationen darüber, wie die Übersetzung von mathematischen Symbolen angegeben wird. 

Um den \LaTeX-Quelltext zu formatieren und zu drucken, benötigen Sie das Programm \LaTeX\, das auf den meisten Linux-Installationen standardmäßig vorhanden und für Windows verfügbar ist.  Um unter Linux eine pdf-Datei zu erstellen, geben Sie normalerweise an der Shell-Eingabeaufforderung Folgendes ein 
\begin{verbatim}
$ pdflatex example.tex
\end{verbatim}

\subsection{\texttt{open tex}-Befehl}\index{\texttt{open tex}-Befehl}

Syntax:  \texttt{open tex} {\em Dateiname} [\texttt{/no{\char`\_}header}]

Dieser Befehl öffnet eine Datei zum Schreiben von \LaTeX-Quelltext\index{latex@{\LaTeX}} und schreibt einen \LaTeX-Header in die Datei. \LaTeX-Quelltext kann mit den Befehlen \texttt{show proof}, \texttt{show new{\char`\_}proof} und \texttt{show statement} unter Verwendung der Option \texttt{/tex} geschrieben werden. 

Die Zuordnung zu \LaTeX-Symbolen wird in einem speziellen Kommentar definiert, der ein \texttt{\$t}-Token enthält, so wie im Anhang~\ref{ASCII} beschrieben. 

Es gibt einen optionalen Befehlszeilenparameter:

    \texttt{/no{\char`\_}header} - Diese Option verhindert, dass ein standardmäßiger \LaTeX-Header und -Trailer in den ausgegebenen \LaTeX-Code aufgenommen wird.


\subsection{\texttt{close tex}-Befehl}\index{\texttt{close tex}-Befehl}

Syntax:  \texttt{close tex}

Dieser Befehl schreibt einen Trailer in jede \LaTeX-Datei\index{latex@{\LaTeX}}, die mit \texttt{open tex} geöffnet wurde (sofern nicht \texttt{/no{\char`\_}header} mit \texttt{open tex} verwendet wurde) und schließt die \LaTeX-Datei. 


\section{Erstellen von {\sc HTML}-Ausgaben}\label{htmlout}

Sie können anhand der Informationen in einer Datenbasis Webseiten generieren. Die Datenbasis muss bereits die notwendigen Schriftsatzinformationen enthalten (siehe Abschnitt \ref{tcomment}, wie man diese Informationen bereitstellt). Die Fähigkeit, {\sc html}-Webseiten zu erzeugen, wurde in Metamath Version 0.07.30 hinzugefügt. 

Um (eine) {\sc html} Ausgabedatei(en) für \texttt{\$a}- oder \texttt{\$p}-Anweisung(en) zu erstellen, verwenden Sie
\begin{quote}
\texttt{show statement} {\em label-match} \texttt{/html}
\end{quote}
Die Ausgabedatei erhält für jede Übereinstimmung den Namen\linebreak
{\em label-match}\texttt{.html}.  Wenn {\em label-match} Platzhalterzeichen (\texttt{*}) enthält, werden für alle Anweisungen mit übereinstimmenden Labels {\sc html}-Dateien erzeugt.  Wenn {\em label-match} einen Platzhalter (\texttt{*}) enthält, werden außerdem zwei zusätzliche Dateien, \texttt{mmdefinitions.html} und \texttt{mmascii.html}, erzeugt.  Um nur diese beiden zusätzlichen Dateien zu erzeugen, können Sie anstelle von {\em label-match} \texttt{?*} verwenden, das mit keinem Label einer Anweisung übereinstimmt. 

Es gibt drei weitere Optionen für \texttt{show statement}, die ebenfalls {\sc HTML}-Code erzeugen.  Diese sind \texttt{/alt{\char`\_}html}, \texttt{/brief{\char`\_}html} und \texttt{/brief{\char`\_}alt{\char`\_}html}, welche im nächsten Abschnitt beschrieben werden. 

Der Befehl 
\begin{quote}
\texttt{show statement} {\em label-match} \texttt{/alt{\char`\_}html}
\end{quote}
bewirkt das Gleiche wie \texttt{show statement} {\em label-match} \texttt{/html}, außer dass der {\sc html}-Code für die Symbole aus \texttt{althtmldef}-Anweisungen anstelle von \texttt{htmldef}-Anweisungen im \texttt{\$t}-Kommentar entnommen wird. 

Der Befehl 
\begin{verbatim} 
show statement * /brief_html 
\end{verbatim} 
ruft einen speziellen Modus auf, der nur Definitions- und Theoremlisten zusammen mit den zugehörigen Symbolen in einem Format ausgibt, das sich zum Kopieren und Einfügen in eine andere Webseite eignet (z. B. in die Tutorial-Seiten auf der Metamath-Website). 

Der Befehl 
\begin{verbatim}
show statement * /brief_alt_html 
\end{verbatim}
bewirkt schließlich dasselbe wie \texttt{show statement * / brief{\char`\_}html} für die alternative {\sc html} Darstellung des Symbols. 

Der Kommentar einer Anweisung kann eine spezielle Notation enthalten, die eine gewisse Kontrolle über die {\sc HTML}-Version des Kommentars bietet.  Siehe Abschnitt~\ref{mathcomments} (p.~\pageref{mathcomments}) für die Kommentarauszeichnungsfunktionen. 

Die Befehle \texttt{write theorem{\char`\_}list} und \texttt{write bibliography}, die im Folgenden beschrieben werden, bieten als Nebeneffekt eine vollständige Fehlerprüfung für alle in diesem Abschnitt beschriebenen Funktionen.\index{Fehlerprüfung} 

\subsection{\texttt{write theorem\_list}
-Befehl}\index{\texttt{write theorem{\char`\_}list}-Befehl}

Syntax:  \texttt{write theorem{\char`\_}list}
[\texttt{/theorems{\char`\_}per{\char`\_}page} {\em number}]

Dieser Befehl schreibt eine Liste aller \texttt{\$a}- und \texttt{\$p}-Anweisungen in der Datenbasis in eine Webseitendatei namens \texttt{mmtheorems.html}. Wenn weitere Dateien benötigt werden, heißen sie \texttt{mmtheorems2.html}, \texttt{mmtheorems3.html}, usw. 

Optionaler Befehlszeilenparameter:

    \texttt{/theorems{\char`\_}per{\char`\_}page} {\em number} - Diese Option gibt die Anzahl der Anweisungen an, die pro Webseite geschrieben werden sollen.  Der Standardwert ist 100.

{\em Anmerkung:} In Version 0.177\index{Metamath!Limitationen der Version 0.177} von Metamath setzen die "`Nearby theorems"'-Links auf den einzelnen Webseiten 100 Theoreme pro Seite voraus, wenn sie auf eine Seite mit einer Theoremliste verweisen.  Daher führt die Option \texttt{/theorems{\char`\_}per{\char`\_}page}, wenn sie eine andere Zahl als 100 angibt, dazu, dass die einzelnen Webseiten nicht mehr synchron sind, und sollte nicht verwendet werden, um die Haupttheoremliste für die Website zu erzeugen.  Dieses Problem wird möglicherweise in einer zukünftigen Version behoben. 


\subsection{\texttt{write bibliography}\label{wrbib}
-Befehl}\index{\texttt{write bibliography}-Befehl}

Syntax:  \texttt{write bibliography} {\em filename}

Dieser Befehl liest eine bestehende bibliografische Querverweisdatei ein, die normalerweise \texttt{mmbiblio.html} heißt, und aktualisiert sie anhand der bibliografischen Links in den Datenbasiskommentaren.  Die Datei wird zwischen den {\sc html} Kommentarzeilen \texttt{<!-- {\char`\#}START{\char`\#} -->} und \texttt{<!-- {\char`\#}END{\char`\#} -->} aktualisiert.  Die ursprüngliche Eingabedatei wird in {\em Dateiname}\texttt{{\char`\~}1} umbenannt. 

Ein bibliografischer Verweis wird mit dem Namen der Referenz in Klammern angegeben, z. B. \texttt{Theorem 3.1 aus [Monk] S.\ 22}. Siehe Abschnitt~\ref{htmlout} (p.~\pageref{htmlout}) für Einzelheiten zur Syntax. 


\subsection{\texttt{write recent\_additions}
-Befehl}\index{\texttt{write recent{\char`\_}additions}-Befehl}

Syntax:  \texttt{write recent{\char`\_}additions} {\em filename}
[\texttt{/limit} {\em number}]

Dieser Befehl liest eine vorhandene Datei "`Recent Additions"' {\sc html} ein, die normalerweise \texttt{mmrecent.html} heißt, und aktualisiert sie mit den Beschreibungen der Theoreme, die der Datenbasis zuletzt hinzugefügt wurden.  Die Datei wird zwischen den {\sc html} Kommentarzeilen \texttt{<!-- {\char`\#}START{\char`\#} -->} und \texttt{<!-- {\char`\#}END{\char`\#} -->} aktualisiert.  Die ursprüngliche Eingabedatei wird in {\em Dateiname}\texttt{{\char`\~}1} umbenannt. 

Optionaler Befehlszeilenparameter:

    \texttt{/limit} {\em number} - Dieser Qualifier gibt die Anzahl der neuesten Theoreme an, die in die Ausgabedatei geschrieben werden sollen.  Der Standardwert ist 100.


\section{Text File Utilities}

\subsection{\texttt{tools}-Befehl}\index{\texttt{tools}-Befehl}

Syntax:  \texttt{tools}

Dieser Befehl ruft ein einfach zu bedienendes, universelles Dienstprogramm zur Bearbeitung des Inhalts von Textdateien auf.  Nach der Eingabe von \texttt{tools} ändert sich die Eingabeaufforderung in \texttt{TOOLS>}, bis Sie \texttt{exit} eingeben.  Mit den \texttt{tools}-Befehlen können Sie einfache, globale Änderungen an einer Ein-/Ausgabedatei vornehmen, z. B. eine Zeichenkettenersetzung in jeder Zeile vornehmen, eine Zeichenkette zu jeder Zeile hinzufügen usw.  Eine typische Verwendung dieses Dienstprogramms ist die Erstellung einer \texttt{submit}-Eingabedatei, um eine allgemeine Operation an einer Liste von Anweisungen durchzuführen, die mittels \texttt{show label} oder \texttt{show usage} erzeugt wurde. 

Die Aktionen der meisten \texttt{tools}-Befehle können auch mit äquivalenten (und leistungsfähigeren) Unix-Shell-Befehlen ausgeführt werden, und manche Benutzer finden diese vielleicht effizienter.  Aber für Windows-Benutzer oder Benutzer, die mit Unix nicht vertraut sind, bietet \texttt{tools} eine leicht zu erlernende Alternative, die für die meisten Aufgaben der Skripterstellung ausreicht, die für die effektive Nutzung des Metamath-Programms erforderlich sind. 


\subsection{\texttt{help}-Befehl (in \texttt{tools})}

Syntax:  \texttt{help}

Der Befehl \texttt{help} listet die im Dienstprogramm \texttt{tools} verfügbaren Befehle zusammen mit einer kurzen Beschreibung auf.  Für jeden Befehl gibt es wiederum eine eigene Hilfe, wie z. B. \texttt{help add}.  Wie bei der \texttt{MM>}-Eingabeaufforderung von Metamath kann ein kompletter Befehl auf einmal eingegeben werden, oder es kann nur das Befehlswort eingegeben werden, woraufhin das Programm nach jedem Argument fragt. 

\vskip 1ex
\noindent Befehle für eine zeilenweise Bearbeitung:

  \texttt{add} - Fügt eine angegebene Zeichenkette zu jeder Zeile einer Datei hinzu.

  \texttt{clean} - Schneidet Leerzeichen und Tabulatoren in jeder Zeile einer Datei ab; konvertiert Zeichen.

  \texttt{delete} - Löschen eines Abschnitts jeder Zeile in einer Datei.

  \texttt{insert} - Fügt eine Zeichenkette in einer bestimmten Spalte in jeder Zeile einer Datei ein.

  \texttt{substitute} - Führt eine einfache Ersetzung in jeder Zeile der Datei durch.

  \texttt{tag} - Wie \texttt{add}, aber beschränkt auf einen Bereich von Zeilen.

  \texttt{swap} - Vertauscht die beiden Hälften jeder Zeile in einer Datei.

\vskip 1ex
\noindent Andere Befehle zur Dateiverarbeitung:

  \texttt{break} - Zerlegt (tokenisiert) eine Datei in eine Liste von Token (eines pro Zeile).

  \texttt{build} - Erzeugt eine Datei mit mehreren Token pro Zeile aus einer Liste.

  \texttt{count} - Zählt die Vorkommen einer bestimmten Zeichenfolge in einer Datei.

  \texttt{number} - Erstellt eine Liste von Zahlen.

  \texttt{parallel} - Schaltet zwei Dateien nebeneinander.

  \texttt{reverse} - Kehrt die Reihenfolge der Zeilen in einer Datei um.

  \texttt{right} - Richtet Zeilen in einer Datei rechtsbündig aus (nützlich vor dem Sortieren von Zahlen).

%  \texttt{tag} - Tag edit updates in a program for revision control.

  \texttt{sort} - Sortiert die Zeilen in einer Datei nach dem Schlüssel, der mit der angegebenen Zeichenfolge beginnt.

  \texttt{match} - Extrahiert Zeilen, die eine bestimmte Zeichenfolge enthalten (oder nicht).

  \texttt{unduplicate} - Eliminiert doppelt vorkommende Zeilen in einer Datei.

  \texttt{duplicate} - Extrahiert das erste Vorkommen einer Zeile, die mehr als \ \ \ einmal in einer Datei vorkommt, und verwirft Zeilen, die genau einmal vorkommen.

  \texttt{unique} - Extrahiert Zeilen, die genau einmal in einer Datei vorkommen.

  \texttt{type} (10 Zeilen) - Zeigt die ersten paar Zeilen einer Datei an. Ähnlich wie bei Unix \texttt{head}.

  \texttt{copy} - Ähnlich wie Unix \texttt{cat}, aber sicher (gleiche Eingabe- und Ausgabedatei erlaubt).

  \texttt{submit} - Führt ein Skript aus, das \texttt{tools} Befehle enthält.

\vskip 1ex

\noindent Hinweis:
\texttt{unduplicate}, \texttt{duplicate} und \texttt{unique} sortieren als Nebeneffekt auch die Zeilen.


\subsection{Verwendung von \texttt{tools} zur Erstellung von Metamath \texttt{submit}-Skripten}

Der Befehl \texttt{break} Befehl wird normalerweise verwendet, um eine Reihe von Anweisungs-Labels, wie z.B. die Ausgabe von Metamaths \texttt{show usage}, in ein Label pro Zeile aufzubrechen.  Die anderen \texttt{tools}-Befehle können dann verwendet werden, um Zeichenketten vor und nach jedem Label einer Anweisung hinzuzufügen, um Befehle anzugeben, die für die Anweisung ausgeführt werden sollen.  Der Befehl \texttt{parallel} ist nützlich, wenn ein Label für eine Anweisung mehr als einmal in einer Zeile erwähnt werden muss. 

Sehr oft erfordert ein Skript für Metamath mehrere Befehlszeilen für jede zu verarbeitende Anweisung.  Sie möchten zum Beispiel den Beweis-Assistenten starten, \texttt{minimize{\char`\_}with} für Ihren zuletzt bearbeitetes Theorem ausführen, mit \texttt{save} den neuen Beweis speichern und mit \texttt{exit} den Beweis-Assistenten beenden.  Um dies zu erreichen, können Sie eine Datei mit diesen vier Befehlen für jede Anweisung in einer einzigen Zeile erstellen und jeden Befehl durch ein bestimmtes Zeichen wie \texttt{@} abtrennen.  Am Ende können Sie dann jedes \texttt{@} mit \texttt{{\char`\\}n} ersetzen, um die Zeilen in einzelne Befehlszeilen aufzulösen (siehe \texttt{help substitute}). 


\subsection{Beispiel für eine \texttt{tools}-Sitzung}

Um Ihnen ein schnelles Gefühl für das Dienstprogramm \texttt{tools} zu vermitteln, zeigen wir eine einfache Sitzung, in der wir eine Datei \texttt{n.txt} mit 3 Zeilen erstellen, Zeichenketten vor und nach jeder Zeile hinzufügen und die Zeilen auf dem Bildschirm anzeigen. Sie können mit den verschiedenen Befehlen experimentieren, um Erfahrungen mit dem Dienstprogramm \texttt{tools} zu sammeln. 

\begin{verbatim}
MM> tools
Entering the Text Tools utilities.
Type HELP for help, EXIT to exit.
TOOLS> number
Output file <n.tmp>? n.txt
First number <1>?
Last number <10>? 3
Increment <1>?
TOOLS> add
Input/output file? n.txt
String to add to beginning of each line <>? This is line
String to add to end of each line <>? .
The file n.txt has 3 lines; 3 were changed.
First change is on line 1:
This is line 1.
TOOLS> type n.txt
This is line 1.
This is line 2.
This is line 3.
TOOLS> exit
Exiting the Text Tools.
Type EXIT again to exit Metamath.
MM>
\end{verbatim}



\appendix
\chapter{Beispielhafte Darstellungen}
\label{ASCII}

Dieser Anhang enthält eine Auswahl von {\sc ASCII}-Darstellungen, die ent\-spre\-chen\-den traditionellen mathematischen Symbole und eine Erläuterung ihrer Bedeutungen in der Datenbasis \texttt{set.mm}. Die Symbole sind in der Reihenfolge ihres Auftretens aufgeführt. Dies ist nur eine unvollständige Liste, und es werden laufend neue Definitionen hinzugefügt. Eine vollständige Liste finden Sie unter \url{http://metamath.org}. 

Diese {\sc ASCII}-Darstellungen sowie Informationen zu ihrer Anzeige werden in der Datenbasisdatei \texttt{set.mm} in einem speziellen Kommentar namens \texttt{\$t} {\em comment}\index{\texttt{\$t}-Anweisung} oder {\em Schriftsatzkommentar} definiert. Ein Schriftsatzkommentar ist durch die zweistellige Zeichenfolge \texttt{\$t} am Anfang des Kommentars gekennzeichnet. Weitere Informationen finden Sie in Abschnitt~\ref{tcomment}, S.~\pageref{tcomment}. 

In der folgenden Tabelle zeigt die Spalte "`{\sc ASCII}"' die {\sc ASCII}-Darstellung, "`Symbol"' die mathematisch-symbolische Darstellung, die dieser {\sc ASCII}-Dar\-stel\-lung entspricht, "`Labels"' die Schlüssel-Labels, die die Darstellung de\-fi\-nie\-ren, und "`Beschreibung"' liefert eine Beschreibung des Symbols. Wie üblich ist "`gdw"' die Abkürzung für "`genau dann, wenn"' oder "`dann und nur dann, wenn"'. \index{gdw} In den meisten Fällen zeigt die Spalte "`{\sc ASCII}"' nur das Schlüssel-Token an, aber manchmal wird auch eine Folge von Token angezeigt, wenn dies für die Verständlichkeit notwendig ist. 

{\setlength{\extrarowsep}{4pt} % Keep rows from being too close together
\begin{longtabu}   { @{} c c l X }
\textbf{ASCII} & \textbf{Symbol} & \textbf{Labels} & \textbf{Beschreibung} \\
\endhead
\texttt{|-} & $\vdash$ & &
"`Es ist beweisbar, dass..."' \\
\texttt{ph} & $\varphi$ & \texttt{wph} &
Die (boolesche) wff-Variable Phi, üblicherweise die erste wff-Variable. \\
\texttt{ps} & $\psi$ & \texttt{wps} &
Die (boolesche) wff-Variable Psi, üblicherweise die zweite wff-Variable. \\  
\texttt{ch} & $\chi$ & \texttt{wch} &
Die (boolesche) wff-Variable Chi, üblicherweise die dritte wff-Variable. \\  
\texttt{-.} & $\lnot$ & \texttt{wn} &
Logisch nicht. Wenn z.B. $\varphi$ wahr ist, dann ist $\lnot \varphi$ falsch. \\
\texttt{->} & $\rightarrow$ & \texttt{wi} &
Impliziert, auch als materielle Implikation bezeichnet.   In der klassischen Logik ist der Ausdruck $\varphi \rightarrow \psi$ wahr, wenn entweder $\varphi$ falsch oder $\psi$ wahr ist (oder beides), d.h. $\varphi \rightarrow \psi$ hat die gleiche Bedeutung wie $\lnot \varphi \lor \psi$ (wie in Theorem \texttt{imor} bewiesen). \\
\texttt{<->} & $\leftrightarrow$ &
\hyperref[df-bi]{\texttt{df-bi}} &
Bikonditional (auch bekannt als ist-gleich für boolesche Werte). $\varphi \leftrightarrow \psi$ ist wahr, wenn und nur dann wenn $\varphi$ und $\psi$ den gleichen Wert haben.\\
\texttt{\char`\\/} & $\lor$ &
\makecell[tl]{{\hyperref[df-or]{\texttt{df-or}}}, \\
	\hyperref[df-3or]{\texttt{df-3or}}} &
Disjunktion (logisches "`oder"'). $\varphi \lor \psi$ ist wahr, wenn $\varphi$, $\psi$, oder beide wahr sind. \\
\texttt{/\char`\\} & $\land$ &
\makecell[tl]{{\hyperref[df-an]{\texttt{df-an}}}, \\
	\hyperref[df-3an]{\texttt{df-3an}}} &
Konjunktion (logisches "`und"'). $\varphi \land \psi$ ist wahr, wenn sowohl $\varphi$ als auch $\psi$ wahr sind. \\
\texttt{A.} & $\forall$ &
\texttt{wal} &
Für alle; die wff $\forall x \varphi$ ist wahr, wenn $\varphi$ für alle Werte von $x$ wahr ist. \\
\texttt{E.} & $\exists$ &
\hyperref[df-ex]{\texttt{df-ex}} &
Es existiert; die wff $\exists x \varphi$ ist wahr, wenn es mindestens ein $x$ gibt, für das $\varphi$ wahr ist. \\
\texttt{[ y / x ]} & $[ y / x ]$ &
\hyperref[df-sb]{\texttt{df-sb}} &
Die wff $[ y / x ] \varphi$ ist das Ergebnis, wenn $y$ in $\varphi$ echt durch $x$ ersetzt wird ($y$ ersetzt $x$). 
% Dies ist elsb4 
% ( [ x / y ] z e. y <-> z e. x ) 
Zum Beispiel ist $[ x / y ] z \in y$ das Gleiche wie $z \in x$. \\
\texttt{E!} & $\exists !$ &
\hyperref[df-eu]{\texttt{df-eu}} &
Es gibt genau ein; $\exists ! x \varphi$ ist wahr, wenn es genau ein $x$ gibt, bei dem $\varphi$ wahr ist. \\
\texttt{\{ y | phi \}}  & $ \{ y | \varphi \}$ &
\hyperref[df-clab]{\texttt{df-clab}} &
Die Klasse aller Mengen, in denen $\varphi$ wahr ist. \\
\texttt{=} & $ = $ &
\hyperref[df-cleq]{\texttt{df-cleq}} &
Klassengleichheit; $A = B$ gdw $A$ gleich $B$ ist. \\
\texttt{e.} & $ \in $ &
\hyperref[df-clel]{\texttt{df-clel}} &
Klassenzugehörigkeit; $A \in B$ gdw $A$ ein Element von $B$ ist. \\
\texttt{{\char`\_}V} & {\rm V} &
\hyperref[df-v]{\texttt{df-v}} &
Klasse aller Mengen (selbst keine Menge). \\
\texttt{C\_} & $ \subseteq $ &
\hyperref[df-ss]{\texttt{df-ss}} &
Unterklasse (Untermenge); $A \subseteq B$ ist wahr gdw $A$ eine Unterklasse von $B$ ist. \\
\texttt{u.} & $ \cup $ &
\hyperref[df-un]{\texttt{df-un}} &
$A \cup B$ ist die Vereinigung der Klassen $A$ und $B$. \\
\texttt{i\char`\^i} & $ \cap $ &
\hyperref[df-in]{\texttt{df-in}} &
$A \cap B$ ist die Schnittmenge der Klassen $A$ und $B$. \\
\texttt{\char`\\} & $ \setminus $ &
\hyperref[df-dif]{\texttt{df-dif}} &
$A \setminus B$ (Klassendifferenz) ist die Klasse aller Mengen in $A$ mit Ausnahme derjenigen in $B$. \\
\texttt{(/)} & $ \varnothing $ &
\hyperref[df-nul]{\texttt{df-nul}} &
$ \varnothing $ ist die leere Menge. \\
\texttt{\char`\~P} & $ \cal P $ &
\hyperref[df-pw]{\texttt{df-pw}} &
Potenzklasse. \\
\texttt{<.\ A , B >.} & $\langle A , B \rangle$ &
\hyperref[df-op]{\texttt{df-op}} &
Das geordnete Paar $\langle A , B \rangle$. \\
\texttt{( F ` A )} & $ ( F ` A ) $ &
\hyperref[df-fv]{\texttt{df-fv}} &
Der Wert der Funktion $F$ an der Stelle $A$. \\
\texttt{\_i} & $ i $ &
\texttt{df-i} &
Die Quadratwurzel von minus eins. \\
\texttt{x.} & $ \cdot $ &
\texttt{df-mul} &
Multiplikation komplexer Zahlen; $2~\cdot~3~=~6$. \\
\texttt{CC} & $ \mathbb{C} $ &
\texttt{df-c} &
Die Menge der komplexen Zahlen. \\
\texttt{RR} & $ \mathbb{R} $ &
\texttt{df-r} &
Die Menge der reellen Zahlen. \\
\end{longtabu}
} % end of extrarowsep}}% }}entfernen

\chapter{Komprimierte Beweise}
\label{compressed}\index{komprimierter Beweis}\index{Beweis!komprimiert}

Die Beweise in der Mengenlehre-Datenbasis \texttt{set.mm} werden aus Effizienzgründen in einem komprimierten Format gespeichert.  Normalerweise brau\-chen Sie sich nicht um das komprimierte Format zu kümmern, da Sie es mit den üblichen Werkzeugen zur Anzeige von Beweisen im Metamath-Programm (\texttt{show proof}\ldots) anzeigen oder in das normale RPN-Beweisformat konvertieren können, so wie in Abschnitt~\ref{proof} beschrieben (mit \texttt{save proof} {\em label} \texttt{/normal}).  Der Vollständigkeit halber beschreiben wir hier jedoch das Format und zeigen, wie es auf das normale RPN-Beweisformat abgebildet wird. 

Ein komprimierter Beweis, der sich zwischen den Schlüsselwörtern \texttt{\$=} und \texttt{\$.}\ befindet, besteht aus einer linken Klammer, einer Folge von Anweisungs-Labels, einer rechten Klammer und einer Folge von Großbuchstaben \texttt{A} bis \texttt{Z} (mit optionalem Whitespace dazwischen).  Die Klammern und die Labels müssen von Whitespace umgeben sein.  Die linke Klammer sagt Metamath, dass ein komprimierter Beweis folgt (Ein normaler RPN-Beweis besteht nur aus einer Folge von Labels, und eine Klammer ist kein zulässiges Zeichen in einem Label).

Die Folge von Großbuchstaben entspricht einer Folge von ganzen Zahlen mit der folgenden Zuordnung.  Jede ganze Zahl entspricht einem Beweisschritt, wie später beschrieben. 
\begin{center}
  \texttt{A} = 1 \\
  \texttt{B} = 2 \\
   \ldots \\
  \texttt{T} = 20 \\
  \texttt{UA} = 21 \\
  \texttt{UB} = 22 \\
   \ldots \\
  \texttt{UT} = 40 \\
  \texttt{VA} = 41 \\
  \texttt{VB} = 42 \\
   \ldots \\
  \texttt{YT} = 120 \\
  \texttt{UUA} = 121 \\
   \ldots \\
  \texttt{YYT} = 620 \\
  \texttt{UUUA} = 621 \\
   etc.
\end{center}

Mit anderen Worten: \texttt{A} bis \texttt{T} stehen für die niedrigstwertige Ziffer zur Basis 20, und \texttt{U} bis \texttt{Y} stehen für null oder mehr höchstwertige Ziffern zur Basis 5, wobei die Ziffern bei 1 anstelle der üblichen 0 beginnen. Bei diesem Schema brauchen wir keinen Whitespace zwischen diesen "`Ziffern"'. 

(Beim Entwurf des komprimierten Beweisformats wurden nur Großbuchstaben gewählt, im Gegensatz zu allen druckbaren nicht-Whitespace {\sc ascii}-Zeichen außer
%\texttt{\$}, was chosen to make the compressed proof a little less
%displeasing to the eye, at the expense of a typical 20\% compression
\texttt{\$}, um nicht mit der Suchfunktion der meisten Texteditorn in Konflikt zu geraten. Dadurch wird ein Kompressionsverlust von typischerweise 20\% i Kauf genommen.  Die Gruppierung Basis 5/Basis 20 wurde gewählt, z.B. statt Basis 6/Basis 19, nachdem experimentell die Gruppierung ermittelt wurde, die in typischen Fällen die beste Kompression ergab). 

Der Buchstabe \texttt{Z} kennzeichnet ("`taggt"') einen Beweisschritt, der mit einem später im Beweis vorkommenden Schritt übereinstimmt; dadurch wird ein Beweis verkürzt, indem identische Beweisschritte nicht immer wieder erneut bewiesen werden müssen (was bei der Erstellung von wff's häufig vorkommt).  Das \texttt{Z} wird unmittelbar nach der niedrigstwertigen Ziffer (Buchstaben \texttt{A} bis \texttt{T}) platziert, die die ganze Zahl beendet, die dem Schritt entspricht, auf den später verwiesen werden soll. 

Die ganzen Zahlen, denen die Großbuchstaben entsprechen, werden wie folgt auf Labels abgebildet.  Wenn die zu beweisende Aussage $m$ zwingende Hypothesen hat, entsprechen die ganzen Zahlen 1 bis $m$ den Labels dieser Hypothesen in der Reihenfolge, die durch den Befehl \texttt{show statement ... / full} angezeigt werden, d.h. der RPN-Reihenfolge\index{RPN-Reihenfolge} der zwingenden Hypothesen.  Die ganzen Zahlen $m+1$ bis $m+n$ entsprechen den in den Klammern des komprimierten Beweises eingeschlossenen Labels, und zwar in der Reihenfolge ihres Auftretens, wobei $n$ die Anzahl dieser Labels ist.  Ganze Zahlen ab $m+n+1$ entsprechen nicht direkt den Labels der Anweisung, sondern verweisen auf Beweisschritte, die mit dem Buchstaben \texttt{Z} gekennzeichnet sind, so dass auf diese Beweisschritte später im Beweis Bezug genommen werden kann.  Die ganze Zahl $m+n+1$ entspricht dem ersten Schritt, der mit einem \texttt{Z} gekennzeichnet ist, $m+n+2$ dem zweiten Schritt, der mit einem \texttt{Z} gekennzeichnet ist, usw.  Wenn der komprimierte Beweis in einen normalen Beweis umgewandelt wird, ersetzt der gesamte Teilbeweis eines mit \texttt{Z} gekennzeichneten Schritts die Referenz auf diesen Schritt. 

Aus Effizienzgründen arbeitet Metamath direkt mit komprimierten Beweisen, ohne sie intern in normale Beweise umzuwandeln.  Zusätzlich zur üblichen Fehlerprüfung wird eine Fehlermeldung ausgegeben, wenn (1) ein Label in der Label-Liste in Klammern nicht auf eine vorherige \texttt{\$p}- oder \texttt{\$a}-Anweisung oder eine nicht zwingende Hypothese der zu beweisenden Anweisung verweist und (2) ein mit \texttt{Z} markierter Beweisschritt vor dem mit \texttt{Z} markierten Schritt referenziert wird. 

Wie bei einem normalen Beweis in der Entwicklung (Abschnitt~\ref{unknown}) kann jeder Schritt oder Teilbeweis, der noch nicht bekannt ist, mit einem einzigen \texttt{?} dargestellt werden. Zwischen dem \texttt{?}\ und den Großbuchstaben (oder anderen \texttt{?}'s), die den Rest des Beweises darstellen, muss kein Whitespace eingefügt sein. 

% April 1, 2004 Appendix C has been added back in with corrections.
%
% May 20, 2003 Appendix C was removed for now because there was a problem found
% by Bob Solovay
%
% Also, removed earlier \ref{formalspec} 's (3 cases above)
%
% Bob Solovay wrote on 30 Nov 2002:
%%%%%%%%%%%%% (start of email comment )
%      3. My next set of comments concern appendix C. I read this before I
% read Chapter 4. So I first noted that the system as presented in the
% Appendix lacked a certain formal property that I thought desirable. I
% then came up with a revised formal system that had this property. Upon
% reading Chapter 4, I noticed that the revised system was closer to the
% treatment in Chapter 4 than the system in Appendix C.
%
%         First a very minor correction:
%
%         On page 142 line 2: The condition that V(e) != V(f) should only be
% required of e, f in T such that e != f.
%
%         Here is a natural property [transitivity] that one would like
% the formal system to have:
%
%         Let Gamma be a set of statements. Suppose that the statement Phi
% is provable from Gamma and that the statement Psi is provable from Gamma
% \cup {Phi}. Then Psi is provable from Gamma.
%
%         I shall present an example to show that this property does not
% hold for the formal systems of Appendix C:
%
%         I write the example in metamath style:
%
% $c A B C D E $.
% $v x y
%
% ${
% tx $f A x $.
% ty $f B y $.
% ax1 $a C x y $.
% $}
%
% ${
% tx $f A x $.
% ty $f B y $.
% ax2-h1 $e C x y $.
% ax2 $a D y $.
% $}
%
% ${
% ty $f B y $.
% ax3-h1 $e D y $.
% ax3 $a E y $.
% $}
%
% $(These three axioms are Gamma $)
%
% ${
% tx $f A x $.
% ty $f B y $.
% Phi $p D y $=
% tx ty tx ty ax1 ax2 $.
% $}
%
% ${
% ty $f B y $.
% Psi $p E y $=
% ty ty Phi ax3 $.
% $}
%
%
% I omit the formal proofs of the following claims. [I will be glad to
% supply them upon request.]
%
% 1) Psi is not provable from Gamma;
%
% 2) Psi is provable from Gamma + Phi.
%
% Here "provable" refers to the formalism of Appendix C.
%
% The trouble of course is that Psi is lacking the variable declaration
%
% $f Ax $.
%
% In the Metamath system there is no trouble proving Psi. I attach a
% metamath file that shows this and which has been checked by the
% metamath program.
%
% I next want to indicate how I think the treatment in Appendix C should
% be revised so as to conform more closely to the metamath system of the
% main text. The revised system *does* have the transitivity property.
%
% We want to give revised definitions of "statement" and
% "provable". [cf. sections C.2.4. and C.2.5] Our new definitions will
% use the definitions given in Appendix C. So we take the following
% tack. We refer to the original notions as o-statement and o-provable. And
% we refer to the notions we are defining as n-statement and n-provable.
%
%         A n-statement is an o-statement in which the only variables
% that appear in the T component are mandatory.
%
%         To any o-statement we can associate its reduct which is a
% n-statement by dropping all the elements of T or D which contain
% non-mandatory variables.
%
%         An n-statement gamma is n-provable if there is an o-statement
% gamma' which has gamma as its reduct andf such that gamma' is
% o-provable.
%
%         It seems to me [though I am not completely sure on this point]
% that n-provability corresponds to metamath provability as discussed
% say in Chapter 4.
%
%         Attached to this letter is the metamath proof of Phi and Psi
% from Gamma discussed above.
%
%         I am still brooding over the question of whether metamath
% correctly formalizes set-theory. No doubt I will have some questions
% re this after my thoughts become clearer.
%%%%%%%%%%%%%%%% (end of email comment)

%%%%%%%%%%%%%%%% (start of 2nd email comment from Bob Solovay 1-Apr-04)
%
%         I hope that Appendix C is the one that gives a "formal" treatment
% of Metamath. At any rate, thats the appendix I want to comment on.
%
%         I'm going to suggest two changes in the definition.
%
%         First change (in the definition of statement): Require that the
% sets D, T, and E be finite.
%
%         Probably things are fine as you give them. But in the applications
% to the main metamath system they will always be finite, and its useful in
% thinking about things [at least for me] to stick to the finite case.
%
%         Second change:
%
%         First let me give an approximate description. Remove the dummy
% variables from the statement. Instead, include them in the proof.
%
%         More formally: Require that T consists of type declarations only
% for mandatory variables. Require that all the pairs in D consist of
% mandatory variables.
%
%         At the start of a proof we are allowed to declare a finite number
% of dummy variables [provided that none of them appear in any of the
% statements in E \cup {A}. We have to supply type declarations for all the
% dummy variables. We are allowed to add new $d statements referring to
% either the mandatory or dummy variables. But we require that no new $d
% statement references only mandatory variables.
%
%         I find this way of doing things more conceptual than the treatment
% in Appendix C. But the change [which I will use implicitly in later
% letters about doing Peano] is mainly aesthetic. I definitely claim that my
% results on doing Peano all apply to Metamath as it is presented in your
% book.
%
%         --Bob
%
%%%%%%%%%%%%%%%% (end of 2nd email comment)

%%
%% When uncommenting the below, also uncomment references above to {formalspec}
%%
\chapter{Das formale System von Metamath}\label{formalspec}\index{Metamath!als ein formales System}

\section{Einführung}

\begin{quote}
  {\em Vollkommenheit ist, wenn es nichts mehr wegzunehmen gibt.}
    \flushright\sc Antoine de Saint-Exupery\footnote{Nach \cite[S.~3-25]{Campbell}.}\\
\end{quote}\index{de Saint-Exupery, Antoine}

Dieser Anhang beschreibt die Theorie hinter der Metamath-Sprache in einer abstrakten Form, die für Mathematiker gedacht ist.  Konkret konstruieren wir zwei Mengen-theoretische Objekte: ein "`formales System"' (grob gesagt, eine Menge von Syntaxregeln, Axiomen und logischen Regeln) und sein "`Universum"' (grob gesagt, die Menge der Theoreme, die im formalen System ableitbar sind).  Die Computersprache Metamath bietet uns eine Möglichkeit, bestimmte formale Systeme zu beschreiben und mit Hilfe eines vom Benutzer bereitgestellten Beweises zu überprüfen, ob gegebene Theoreme zu ihren Universen gehören. 

Um diesen Anhang zu verstehen, benötigen Sie Grundkenntnisse der informellen Mengenlehre. Es sollte ausreichen, z.B. Kap.\ 1 von Munkres' {\em Topology} zu verstehen\cite{Munkres}\index{Munkres, James R.} oder das einführende Kapitel zur Mengenlehre in vielen Lehrbüchern, die in die abstrakte Mathematik einführen. (Beachten Sie, dass es zwischen den Autoren kleinere Unterschiede in der Schreibweise gibt; z.B. verwendet Munkres $\subset$ anstelle unseres $\subseteq$ für "`subset"'.  Wir verwenden "`enthalten in"' für "`eine Teilmenge von"' und "`gehört zu"' oder "`ist enthalten in"' für "`ist ein Element von"'). Was wir hier als "`formale"' Beschreibung bezeichnen, ist anders als früher, eigentlich eine informelle Beschreibung in der gewöhnlichen Sprache der Mathematiker.  Wir geben jedoch genügend Details an, so dass ein Mathematiker sie leicht formalisieren kann, sogar in der Sprache von Metamath selbst, falls gewünscht.  Um die Logikbeispiele am Ende dieses Anhangs zu verstehen, wäre die Kenntnis eines einführenden Buches über mathematische Logik hilfreich. 

\section{Die formale Beschreibung}

\subsection[Vorbereitende Maßnahmen]{Vorbereitende Maßnahmen\protect\footnotemark}%
\footnotetext{Dieser Abschnitt ist größtenteils wörtlich von Tarski\cite[p.~63]{Tarski1965}\index{Tarski, Alfred} übernommen und frei übersetzt.}

Mit $\omega$ bezeichnen wir die Menge aller natürlichen Zahlen (nichtnegative ganze Zahlen). Jede natürliche Zahl $n$ wird mit der Menge aller kleineren Zahlen identifiziert: $n = \{ m | m < n \}$.  Die Formel $m < n$ ist also äquivalent zu der Bedingung: $m \in n$ und $m,n \in \omega$. Insbesondere ist 0 die Zahl Null und zugleich die leere Menge $\varnothing$, $1=\{0\}$, $2=\{0,1\}$ usw. ${}^B A$ bezeichnet die Menge aller Funktionen von  $B$ nach $A$ (d.h. \ mit dem Definitionsbereich $B$ und einem in $A$ enthaltenen Wertebereich).  Die Elemente von ${}^\omega A$ sind so genannte {\em einfache unendliche Folgen},\index{einfache unendliche Folge} mit allen {\em Gliedern}\index{Glied} in $A$.  Für den Fall $n \in \omega$ werden die Elemente von ${}^n A$ als {\em endliche $n$-gliedrige Folgen},\index{endliche $n$-gliedrige Folge} bezeichnet, wiederum mit Gliedern in $A$.  Die aufeinanderfolgenden Glieder (Funktionswerte) einer endlichen oder unendlichen Folge $f$ werden mit $f_0, f_1, \ldots ,f_n,\ldots$ bezeichnet.  Jede endliche Folge $f \in \bigcup _{n \in \omega} {}^n A$ bestimmt eindeutig die Zahl $n$, so dass $f \in {}^n A$; $n$ heißt die {\em Länge}\index{Länge einer Folge ({$"|\"|$})} von $f$ und wird mit $|f|$ bezeichnet.  $\langle a \rangle$ ist die Folge $f$ mit $|f|=1$ und $f_0=a$; $\langle a,b \rangle$ ist die Folge $f$ mit $|f|=2$, $f_0=a$, $f_1=b$; usw.  Für zwei gegebene endliche Folgen $f$ und $g$ bezeichnen wir mit $f\frown g$ ihre {\em Verkettung},\index{Verkettung} d.h. die endliche Folge $h$, die durch die folgende Bedingungen bestimmt ist: 
\begin{eqnarray*}
& |h| = |f|+|g|;&  \\
& h_n = f_n & \mbox{\ for\ } n < |f|;  \\
& h_{|f|+n} = g_n & \mbox{\ for\ } n < |g|.
\end{eqnarray*}

\subsection{Konstanten, Variablen und Ausdrücke}

Ein formales System hat eine Menge von {\em Symbolen}\index{Symbol!in einem formalen System}, die mit $\mbox{\em SM}$ bezeichnet wird.  Eine genaue mengentheoretische Definition dieser Menge ist unwichtig; ein Symbol kann als primitives oder atomares Element betrachtet werden, wenn man will.  Wir nehmen an, dass diese Menge in zwei disjunkte Teilmengen unterteilt ist: eine Menge $\mbox{\em CN}$ von {\em Konstanten}\index{Konstante!in einem formalen System} und eine Menge $\mbox{\em VR}$ von {\em Variablen}\index{Variable!in einem formalen System}. $\mbox{\em CN}$ und $\mbox{\em VR}$ bestehen jeweils aus abzählbar vielen Symbolen, die in endlichen oder einfachen unendlichen Folgen $c_0, c_1, \ldots$ bzw. $v_0, v_1, \ldots$ ohne Wiederholumgen angeordnet werden können.  Beliebige Symbole werden wir durch Metavariablen $\alpha$, $\beta$ usw. darstellen. 

{\footnotesize\begin{quotation}
{\em Kommentar:} Die Variablen $v_0, v_1, \ldots$ unseres formalen Systems ent\-spre\-chen dem, was in der Literatur zu spezifischen formalen Systemen gewöhnlich als "`Metavariablen"' bezeichnet wird.  Typischerweise wird bei der Beschreibung eines bestimmten formalen Systems in einem Buch eine Reihe von primitiven Objekten postuliert, die Variablen genannt werden, und dann werden deren Eigenschaften mit Hilfe von Metavariablen beschrieben, die sich über diese erstrecken, wobei die eigentlichen Variablen selbst nie wieder erwähnt werden.  Unser formales System erwähnt diese primitiven, Variablen genannte Objekte überhaupt nicht, sondern befasst sich von Anfang an direkt mit Metavariablen als seine primitiven Objekten.  Dies ist ein subtiler, aber wichtiger Unterschied, den man im Auge behalten sollte. Denn dadurch unterscheidet sich unsere Definition von "`formalem System"' etwas von denen, die man normalerweise in der Literatur findet.  (So sind die oben genannten Metavariablen $\alpha$, $\beta$ usw.\ eigentlich "`Metametavariablen"', wenn sie zur Darstellung von $v_0, v_1, \ldots$ verwendet werden.)
\end{quotation}}

Endliche Folgen, bei denen alle Glieder Symbole sind, heißen {\em-Ausdrücke}.\index{Ausdruck!in einem formalen System} $\mbox{\em EX}$ ist die Menge aller Ausdrücke; also 
\begin{displaymath}
\mbox{\em EX} = \bigcup _{n \in \omega} {}^n \mbox{\em SM}.
\end{displaymath}

Ein {\em konstant-gepräfixter Ausdruck}\index{konstant-gepräfixter Ausdruck} ist ein Ausdruck der Länge ungleich Null, dessen erstes Glied eine Konstante ist.  Wir bezeichnen die Menge aller konstant-gepräfixter Ausdrücke mit $\mbox{\em EX}_C = \{ e \in \mbox{\em EX} | ( |e| > 0 \wedge e_0 \in \mbox{\em CN} ) \}$. 

Ein {\em Konstante-Variable-Paar}\index{Konstante-Variable-Paar} ist ein Ausdruck der Länge 2, dessen erstes Glied eine Konstante ist und dessen zweites Glied eine Variable ist.  Wir bezeichnen die Menge aller Konstanten-Variablen-Paare mit $\mbox{\em EX}_2 = \{ e \in \mbox{\em EX}_C | ( |e| = 2 \wedge e_1 \in \mbox{\em VR} ) \}$. 


{\footnotesize\begin{quotation}
{\em Beziehung zu Metamath:} Im Allgemeinen entspricht die Menge $\mbox{\em SM}$ der Menge der deklarierten mathematischen Symbole in einer Metamath-Datenbasis, die Menge $\mbox{\em CN}$ denjenigen Symbolen, die mit \texttt{\$c}-Anweisungen deklariert sind, und die Menge $\mbox{\em VR}$ denjenigen Symbolen, die mit \texttt{\$v}-Anweisungen deklariert sind.  Natürlich kann eine Metamath-Datenbasis nur eine endliche Anzahl von mathematischen Symbolen haben, während formale Systeme im Allgemeinen eine unendliche Anzahl haben können, obwohl die Anzahl der in Metamath verfügbaren mathematischen Symbole im Prinzip unbegrenzt ist.  Die Menge $\mbox{\em EX}_C$ entspricht der Menge der zulässigen Ausdrücke für \texttt{\$e}-, \texttt{\$a}- und \texttt{\$p}-Anweisungen.  Die Menge $\mbox{\em EX}_2$ entspricht der Menge der zulässigen Ausdrücke für \texttt{\$f}-Anweisungen.
\end{quotation}}

Wir bezeichnen mit ${\cal V}(e)$ die Menge aller Variablen in einem Ausdruck $e \in \mbox{\em EX}$, d.h. die Menge aller $\alpha \in \mbox{\em VR}$, so dass $\alpha = e_n$ für mindestens ein $n < |e|$.  Wir bezeichnen auch (unter Missbrauch der Notation) mit ${\cal V}(E)$ die Menge aller Variablen in einer Sammlung von Ausdrücken $E \subseteq \mbox{\em EX}$, d.h.\ $\bigcup _{e \in E} {\cal V}(e)$. 


\subsection{Substitution}

Bei einer Funktion $F$ von $\mbox{\em VR}$ nach $\mbox{\em EX}$ bezeichnen wir mit $\sigma_{F}$ oder einfach $\sigma$ die Funktion von $\mbox{\em EX}$ nach $\mbox{\em EX}$, die rekursiv für nichtleere Folgen durch 
\begin{eqnarray*}
& \sigma(<\alpha>) = F(\alpha) & \mbox{mit\ } \alpha \in \mbox{\em VR}; \\
& \sigma(<\alpha>) = <\alpha> & \mbox{mit\ } \alpha \not\in \mbox{\em VR}; \\
& \sigma(g \frown h) = \sigma(g) \frown
    \sigma(h) & \mbox{mit\ } g,h \in \mbox{\em EX}.
\end{eqnarray*}
definiert ist.

Wir definieren außerdem $\sigma(\varnothing)=\varnothing$.  Wir nennen $\sigma$ eine {\em simultane Substitution}\index{Substitution!Variable}\index{Variablensubstitution} (oder einfach {\em Substitution}) mit {\em Substitutionsabbildung}\index{Substitutionsabbildung} $F$. 

Mit $\sigma(E)$ bezeichnen wir (unter Missbrauch der Notation) auch eine Substitution auf einer Sammlung von Ausdrücken $E \subseteq \mbox{\em EX}$, d.h. die Menge $\{ \sigma(e) | e \in E \}$.  Die Sammlung $\sigma(E)$ kann natürlich weniger Ausdrücke als $E$ enthalten, weil durch die Substitution doppelte Ausdrücke entstehen könnten. 


\subsection{Aussagen}

Wir bezeichnen mit $\mbox{\em DV}$ die Menge aller ungeordneten Paare $\{\alpha, \beta \} \subseteq \mbox{\em VR}$, so dass $\alpha \neq \beta$. $\mbox{\em DV}$ steht für "`unterscheidbare Variablen"'. 

Eine {\em Prä-Aussage}\index{Prä-Aussage!in einem formalen System} ist ein Quadrupel $\langle D,T,H,A \rangle$ derart, dass $D\subseteq \mbox{\em DV}$, $T\subseteq \mbox{\em EX}_2$, $H\subseteq \mbox{\em EX}_C$ und $H$ endlich ist, $A\in \mbox{\em EX}_C$, ${\cal V}(H\cup\{A\}) \subseteq {\cal V}(T)$, und $\forall e,f\in T {\ } {\cal V}(e) \neq {\cal V}(f)$ (oder äquivalent, $e_1 \ne f_1$), wann immer $e \neq f$. Die Terme des Quadrupels werden respektive
{\em disjunkte Variableneinschränkungen},\index{disjunkte Variableneinschränkung!in einem formalen System},
{\em Variablentyp-Hypothesen},\index{Variablentyp-Hypothese!in einem formalen System},
{\em logische Hypothesen},\index{logische Hypothese!in einem formalen System} und die 
{\em Behauptung}\index{Behauptung!in einem formalen System} genannt.
Wir bezeichnen mit $T_M$ ({\em obligatorische Variablentyp-Hypothesen}\index{obligatorische Variablentyp-Hypothese!in einem formalen System}) die Teilmenge von $T$, so dass ${\cal V}(T_M) ={\cal V}(H \cup \{A\})$.  Wir bezeichnen mit $D_M=\{\{\alpha,\beta\}\in D|\{\alpha,\beta\}\subseteq {\cal V}(T_M)\}$ die {\em obligatorischen disjunkte Variableneinschränkung}\index{obligatorische disjunkte Variableneinschränkung!in einem formalen System} der Prä-Aussage. Die Menge der {\em obligatorischen Hypothesen}\index{obligatorische Hypothese!in einem formalen System} ist $T_M\cup H$.  Wir nennen das Quadrupel $\langle D_M,T_M,H,A \rangle$ das {\em Redukt}\index{Redukt!in einem formalen System} der Prä-Aussage $\langle D,T,H,A \rangle$.  

Eine {\em Aussage} ist das Redukt einer Prä-Aussage\index{Aussage!in einem formalen System}.  Eine Aussage ist also eine besondere Art von Prä-Aussage; insbesondere ist eine Aussage das Redukt ihrer selbst. 

{\footnotesize\begin{quotation}
{\em Kommentar:}  $T$ ist eine Menge von Ausdrücken der Länge 2, die eine Menge von Konstanten ("`Variablentypen"') mit einer Menge von Variablen verknüpfen.  Die Bedingung ${\cal V}(H\cup\{A\}) \subseteq {\cal V}(T) $ bedeutet, dass jede Variable, die in den logischen Hypothesen oder Behauptungen einer Aussage vorkommt, eine zugehörige Variablentyp-Hypothese oder eine "`Typendeklaration"' haben muss, in Analogie zu einer Programmiersprache für Computer, in der eine Variable beispielsweise als String oder Integer deklariert werden muss.  Die Anforderung, dass $\forall e,f\in T \, e_1 \ne f_1$ für $e\neq f$ bedeutet, dass jede Variable eindeutig einer Konstanten zugeordnet sein muss, die ihren Variablentyp bezeichnet; z.B. kann eine Variable ein "`wff"' oder ein "`set"' sein, aber nicht beides.

Disjunkte Variableneinschränkungen werden verwendet, um anzugeben, welche Variablensubstitutionen zulässig sind, damit die Anweisung gültig bleibt.  In dem Theoremschema der Mengenlehre $\lnot\forall x\,x=y$ dürfen wir zum Beispiel nicht dieselbe Variable sowohl für $x$ als auch für $y$ einsetzen.  Andererseits verlangt das Theoremschema $x=y\to y=x$ nicht, dass $x$ und $y$ verschieden sein müssen, so dass wir keine disjunkte Variableneinschränkungen brauchen, obwohl eine solche Einschränkung nur dazu führen würde, dass das Schema weniger allgemein wäre.  

Eine obligatorische Variablentyp-Hypothese ist eine, deren Variable in einer logischen Hypothese oder der Behauptung vorkommt.  Eine beweisbare Prä-Aussage (siehe unten) kann nicht-obligatorische Variablentyp-Hypothe\-sen erfordern, die im Endeffekt "`Dummy"'-Variablen zur Verwendung in ihrem Beweis einführen.  Jede mögliche Anzahl von Dummy-Variablen kann für einen bestimmten Beweis erforderlich sein; tatsächlich wurde von H.\ Andr\'{e}ka\index{Andr{\'{e}}ka, H.} \cite{Nemeti} gezeigt, dass es keine endliche Obergrenze für die Anzahl der Dummy-Variablen gibt, die benötigt werden, um einen beliebiges Theorem in der Logik erster Ordnung (mit Gleichheit) zu beweisen, der eine feste Anzahl $n>2$ von individuellen Variablen hat.  (Siehe auch den Kommentar zu S.~\pageref{nodd}.) Aus diesem Grund setzen wir keine endliche Größenbeschränkung für die Sammlungen $D$ und $T$, obwohl diese in einer tatsächlichen Anwendung (Metamath-Datenbasis) natürlich endlich sein werden und deren Anzahl wenn nötig vergrößert werden muss, wenn mehr Beweise hinzugefügt werden.
\end{quotation}}

{\footnotesize\begin{quotation}
{\em Beziehung zu Metamath:} Eine Prä-Aussage eines formalen Systems entspricht einem erweiterten Rahmen in einer Metamath-Datenbasis (Abschnitt~\ref{frames}).  Die Sammlungen $D$, $T$ und $H$ entsprechen den Sammlungen der Anweisungen \texttt{\$d}, \texttt{\$f} und \texttt{\$e} in einem erweiterten Rahmen.  Der Ausdruck $A$ entspricht der Anweisung \texttt{\$a} (oder \texttt{\$p}) in einem erweiterten Rahmen.  Eine Ausage eines formalen Systems entspricht einem Frame in einer Metamath-Datenbasis.
\end{quotation}}

\subsection{Formale Systeme}

Ein {\em formales System}\index{formales System} ist ein Tripel $\langle \mbox{\em CN},\mbox{\em VR},\Gamma\rangle$ wobei $\Gamma$ eine Menge von Aussagen ist.  Die Elemente von $\Gamma$ werden {\em axiomatische Aussagen} genannt\index{axiomatische Aussagen!in einem formalen System}.  Manchmal wird ein formales System nur mit $\Gamma$ bezeichnet, wenn $\mbox{\em CN}$ und $\mbox{\em VR}$ klar sind.\footnote{Anm. der Übersetzer: Üblicherweise ist ein {\em formales System}\index{formales System} ein Quadrupel $\langle \mbox{\em A},\mbox{\em B},\Gamma,\mbox{\em R}\rangle$
wobei $\mbox{\em A}$ ein Alphabet (hier also $\mbox{\em A} = \mbox{\em SM} = \mbox{\em CN}\cup \mbox{\em VR}$),
$\mbox{\em B}$ eine Teilmenge aller Wörter, die sich über dem Alphabet $\mbox{\em A}$ bilden lassen, also die Menge aller "`wohlgeformten Formeln"' oder eine "`formale Sprache"' über dem Alphabet $\mbox{\em A}$,
$\Gamma$ eine Menge von Aussagen, die als "`Axiome"' aufgefasst werden (es gilt $\Gamma\subseteq\mbox{\em B}$) und 
$\mbox{\em R}$ eine Menge von zwei- oder mehrstelligen Relationen über Wörtern aus $\mbox{\em B}$ (hier also die Substitution $\sigma$ als Funktion/zweistellige Releation) ist.}

In einem formalen System $\Gamma$ ist der {\em Abschluss}\index{Abschluss}\footnote{Diese Definition des Abschlusses enthält eine Vereinfachung, die Josh Purinton zu verdanken ist.\index{Purinton, Josh}.} einer Prä-Aussage\linebreak
$\langle D,T,H,A \rangle$ die kleinste Menge $C$ von Ausdrücken, die so beschaffen ist, dass: 
\begin{list}{}{\itemsep 0.0pt}
  \item[1.] $T\cup H\subseteq C$ und
  \item[2.] Wenn für eine axiomatische Aussage
    $\langle D_M',T_M',H',A' \rangle \in
       \Gamma$ und eine Substitution $\sigma$ gilt
    \begin{enumerate}
       \item[a.] $\sigma(T_M' \cup H') \subseteq C$ und
       \item[b.] für alle $\{\alpha,\beta\}\in D_M'$, für alle $\gamma\in
         {\cal V}(\sigma(\langle \alpha
         \rangle))$ und für alle $\delta\in  {\cal V}(\sigma(\langle \beta
         \rangle))$ gilt $\{\gamma, \delta\} \in D$,
   \end{enumerate}
   dann gilt $\sigma(A') \in C$.
\end{list}
Eine Prä-Aussage $\langle D,T,H,A \rangle$ ist {\em beweisbar}\index{beweisbare Aussage!in einem formalen System}, wenn $A\in C$ d.h.\ wenn ihre Behauptung zu ihrem Abschluss gehört.  Eine Aussage ist {\em beweisbar}, wenn sie das Redukt einer beweisbaren Prä-Aussage ist. Das {\em Universum}\index{Universum eines formalen Systems} eines formalen Systems ist die Sammlung aller seiner beweisbaren Aussagen.  Man beachte, dass die Menge der axiomatischen Aussagen $\Gamma$ in einem formalen System eine Teilmenge seines Universums ist. 

{\footnotesize\begin{quotation}
{\em Kommentar:} Die erste Bedingung in der Definition des Abschlusses besagt einfach, dass die Hypothesen der Prä-Aussage in ihrem Abschluss enthalten sind.  

Bedingung 2(a) besagt, dass es eine Substitution gibt, die dafür sorgt, dass die obligatorischen Hypothesen einer axiomatischen Aussage genau mit mindestens einem Element des Abschlusses übereinstimmt.  Dies zeigen wir explizit in einem Beweis mittels der Metamath-Sprache.

%Conditions 2(a) and 2(b) say that a substitution exists that makes the
%(mandatory) hypotheses of an axiomatic statement exactly match some members of
%the closure.  This is what we explicitly demonstrate with a Metamath language
%proof.
%
%The set of expressions $F$ in condition 2(b) excludes the variable-type
%hypotheses; this is done because non-mandatory variable-type hypotheses are
%effectively "`dropped"' as irrelevant whereas logical hypotheses must be
%retained to achieve a consistent logical system.

Bedingung 2(b) beschreibt, wie die disjunkte Variableneinschränkung in der axiomatischen Aussage erfüllt werden müssen.  Sie besagt, dass nach einer Substitution für zwei Variablen, die verschieden sein müssen, die beiden resultierenden Ausdrücke entweder keine Variablen enthalten dürfen oder, falls doch, keine Variablen gemeinsam haben dürfen, und dass jedes Paar von Variablen, die sie haben, mit einer Variablen aus jedem Ausdruck, in der ursprünglichen Anweisung als distinkt angegeben werden muss.
\end{quotation}}

{\footnotesize\begin{quotation}
{\em Beziehung zu Metamath:} Axiomatische Aussagen und beweisbare Aussagen in einem formalen System entsprechen den Frames für \texttt{\$a}- bzw. \texttt{\$p}-Aussagen in einer Metamath-Datenbasis.  Die Menge der axiomatischen Aussagen ist eine Teilmenge der Menge der beweisbaren Aussagen in einem formalen System, obwohl in einer Metamath-Datenbasis eine \texttt{\$a}-Aussage dadurch gekennzeichnet ist, dass sie keinen Beweis hat.  Ein Beweis in der Metamath-Sprache für eine \texttt{\$p}-Anweisung sagt dem Computer, wie er explizit eine Folge von Elementen des Abschlusses konstruieren soll, was schließlich zu dem Nachweis führt, dass die zu beweisende Behauptung in dem Abschluss enthalten ist.  Der tatsächliche Abschluss enthält normalerweise eine unendliche Anzahl von Ausdrücken.  Ein formales System selbst hat kein explizites Objekt, das als "`Beweis"' bezeichnet wird, sondern die Existenz eines Beweises wird indirekt durch die Zugehörigkeit einer Behauptung zum Abschluss einer beweisbaren Ausage impliziert.  Wir tun dies, um das formale System leichter in der Sprache der Mengenlehre beschreiben zu können.  

Wir weisen auch darauf hin, dass eine einmal als beweisbar nachgewiesene Aussage denselben Status wie eine axiomatische Aussage erhält, denn wenn die Menge der axiomatischen Aussagen um eine beweisbare Aussage erweitert wird, bleibt das Universum des formalen Systems unverändert (vorausgesetzt, dass $\mbox{\em VR}$ unendlich ist). In der Praxis bedeutet dies, dass wir eine Hierarchie von beweisbaren Aussagen aufbauen können, um effizienter weitere beweisbare Aussagen zu ermitteln.  Genau das tun wir in Metamath, wenn wir zulassen, dass Beweise auf vorherige \texttt{\$p}-Anweisungen sowie auf vorherige \texttt{\$a}-Anweisungen verweisen.
\end{quotation}}


\section{Beispiele für formale Systeme}

{\footnotesize\begin{quotation}
{\em Beziehung zu Metamath:} Die Beispiele in diesem Abschnitt, mit Ausnahme von Beispiel~2, entsprechen größtenteils exakt dem Vorgehen in der Mengenlehre-Datenbasis \texttt{set.mm}.  Vergleichen Sie die Beispiele~1, 3 und 5 mit Abschnitt~\ref{metaaxioms}, Beispiel 4 mit den Abschnitten~\ref{metadefprop} und \ref{metadefpred} und Beispiel 6 mit Abschnitt~\ref{setdefinitions}.\label{exampleref}
\end{quotation}}

\subsection{Beispiel~1 --- Aussagenlogik}\index{Aussagenlogik}

Die klassische Aussagenlogik kann durch das folgende formale System beschrieben werden.  Wir nehmen an, dass die Menge der Variablen unendlich ist.  Anstatt die Konstanten und Variablen mit $c_0, c_1, \ldots$ und $v_0, v_1, \ldots$ zu bezeichnen, werden wir aus Gründen der Lesbarkeit stattdessen gängigere Symbole verwenden, wobei wir natürlich davon ausgehen, dass sie unterschiedliche primitive Objekte bezeichnen. Der Lesbarkeit halber können wir auch Kommas zwischen aufeinanderfolgenden Glieder einer Folge weglassen; so steht $\langle \mbox{wff\ } \varphi\rangle$ für $\langle \mbox{wff}, \varphi\rangle$. 

Sei
\begin{itemize}
  \item[] $\mbox{\em CN}=\{\mbox{wff}, \vdash, \to, \lnot, (,)\}$
  \item[] $\mbox{\em VR}=\{\varphi,\psi,\chi,\ldots\}$
  \item[] $T = \{\langle \mbox{wff\ } \varphi\rangle,
             \langle \mbox{wff\ } \psi\rangle,
             \langle \mbox{wff\ } \chi\rangle,\ldots\}$, d.h. diejenigen Ausdrücke der Länge 2, deren erstes Glied $\mbox{\rm wff}$ ist und deren zweites Glied zu $\mbox{\em VR}$ gehört.\footnote{ Der Einfachheit halber lassen wir $T$ eine unendliche Menge sein; die Definition einer Aussage erlaubt dies im Prinzip.  Da eine Metamath-Quelldatei eine endliche Größe hat, müssen wir in der Praxis natürlich geeignete endliche Teilmengen dieses $T$ verwenden, und zwar solche, die zumindest die obligatorischen Variablentyp-Hypothesen enthalten.  In ähnlicher Weise führen wir in der Quelldatei nach Bedarf neue Variablen ein, wobei wir davon ausgehen, dass eine potenziell unendliche Anzahl von ihnen verfügbar ist.}
\end{itemize}

\noindent Dann besteht $\Gamma$ aus den axiomatischen Aussagen, die die Redukte der folgenden Prä-Aussagen sind:
    \begin{itemize}
      \item[] $\langle\varnothing,T,\varnothing,
               \langle \mbox{wff\ }(\varphi\to\psi)\rangle\rangle$
      \item[] $\langle\varnothing,T,\varnothing,
               \langle \mbox{wff\ }\lnot\varphi\rangle\rangle$
      \item[] $\langle\varnothing,T,\varnothing,
               \langle \vdash(\varphi\to(\psi\to\varphi))
               \rangle\rangle$
      \item[] $\langle\varnothing,T,
               \varnothing,
               \langle \vdash((\varphi\to(\psi\to\chi))\to
               ((\varphi\to\psi)\to(\varphi\to\chi)))
               \rangle\rangle$
      \item[] $\langle\varnothing,T,
               \varnothing,
               \langle \vdash((\lnot\varphi\to\lnot\psi)\to
               (\psi\to\varphi))\rangle\rangle$
      \item[] $\langle\varnothing,T,
               \{\langle\vdash(\varphi\to\psi)\rangle,
                 \langle\vdash\varphi\rangle\},
               \langle\vdash\psi\rangle\rangle$
    \end{itemize}

\noindent Zum Beispiel ist das Redukt von $\langle\varnothing,T,\varnothing, \langle \mbox{wff\ }(\varphi\to\psi)\rangle\rangle$ 
\begin{itemize}
\item[] $\langle\varnothing,
\{\langle \mbox{wff\ } \varphi\rangle,
             \langle \mbox{wff\ } \psi\rangle\},
             \varnothing,
               \langle \mbox{wff\ }(\varphi\to\psi)\rangle\rangle$,
\end{itemize}

und von $\langle\varnothing,T,\varnothing,\langle \vdash((\varphi\to(\psi\to\chi))\to((\varphi\to\psi)\to(\varphi\to\chi)))
\rangle\rangle$
\begin{itemize}
\item[] $\langle\varnothing,\{\langle \mbox{wff\ } \varphi\rangle,
\langle \mbox{wff\ } \psi\rangle, \langle \mbox{wff\ } \chi\rangle\},\varnothing,\langle \vdash((\varphi\to(\psi\to\chi))\to((\varphi\to\psi)\to(\varphi\to\chi)))
\rangle\rangle$,
\end{itemize}
welches die erste und die vierte axiomatischen Aussagen sind.

Wir nennen die Elemente von $\mbox{\em VR}$ {\em wff-Variablen} oder (im Kontext der Logik erster Ordnung, die wir gleich beschreiben werden) {\em wff-Metavariablen}. Man beachte, dass die Symbole $\phi$, $\psi$ usw.\ tatsächliche spezifische Elemente der Menge $\mbox{\em VR}$ bezeichnen; sie sind keine Metavariablen unserer Beschreibungssprache (die wir mit $\alpha$, $\beta$ usw. bezeichnen), sondern sind stattdessen (meta)konstante Symbole (Elemente der Menge $\mbox{\em SM}$) aus der Sicht unserer Beschreibungssprache.  Das in \cite{Tarski1965} beschriebene äquivalente System der Aussagenlogik verwendet ebenfalls die Symbole $\phi$, $\psi$ usw.\, um wff-Metavariablen zu bezeichnen, aber in \cite{Tarski1965} sind dies im Gegensatz zu hier Metavariablen der Beschreibungssprache und keine primitiven Symbole des formalen Systems. 

Die ersten beiden Aussagen definieren wffs: wenn $\varphi$ und $\psi$ wffs sind, dann ist $(\varphi \to \psi)$ auch eine wff; wenn $\varphi$ eine wff ist, dann ist $\lnot\varphi$ auch eine wff. Die nächsten drei sind die Axiome der Aussagenlogik: Wenn $\varphi$ und $\psi$ wffs sind, dann ist $\vdash (\varphi \to (\psi \to \varphi))$ ein (axiomatisches) Theorem; usw. Die letzte ist der Modus ponens: wenn $\varphi$ und $\psi$ wffs sind, und $\vdash (\varphi\to\psi)$ und $\vdash \varphi$ Theoreme sind, dann ist $\vdash \psi$ ein Theorem. 

Die Entsprechung zur gewöhnlichen Aussagenlogik ist wie folgt.  Wir betrachten nur beweisbare Aussagen der Form $\langle\varnothing, T,\varnothing,A\rangle$ mit $T$ definiert wie oben.  Der erste Term der Behauptung $A$ einer solchen Aussage ist entweder "`wff"' oder "`$\vdash$"'.  Eine Aussage, bei der der erste Term "`wff"' ist, ist eine {\em wff} der Aussagenlogik, und eine, bei der der erste Term "`$\vdash$"' ist, ist ein {\em Theorem (Schema)} der Aussagenlogik. 

Das Universum dieses formalen Systems enthält auch viele andere beweisbare Aussagen.  Diejenigen mit Beschränkungen für unterschiedliche Variablen sind irrelevant, da die Aussagenlogik keine Beschränkungen für Substitutionen kennt.  Diejenigen, die logische Hypothesen haben, nennen wir {\em Inferenzen}\index{Inferenz}, wenn die logischen Hypothesen von der Form $\langle\vdash\rangle\frown w$ sind, wobei $w$ eine wff ist (wobei der führende konstante Term "`wff"' entfernt wurde).  Inferenzen (mit Ausnahme des Modus ponens) sind kein eigentlicher Bestandteil der Aussagenlogik, lassen sich aber beim Aufbau einer Hierarchie von beweisbaren Aussagen gut verwenden.  Eine beweisbare Aussage mit einer unsinnigen Hypothese wie $\langle \to,\vdash,\lnot\rangle$ und demselben Ausdruck als Behauptung betrachten wir als irrelevant; sie kann beim Beweis von Theoremen nicht verwendet werden, da es keine Möglichkeit gibt, die unsinnige Hypothese zu eliminieren. 

{\footnotesize\begin{quotation}
{\em Kommentar:} Unsere Verwendung von Klammern in der Definition einer wff zeigt, dass axiomatische Aussagegen sorgfältig so formuliert werden sollten, dass sie eindeutig mit den vom formalen System erlaubten Substitutionen zusammenpassen. Es gibt viele Möglichkeiten, wffs zu definieren - die polnische Präfix-Notation hätte es uns beispielsweise erlaubt, die Klammern ganz wegzulassen, was allerdings zu Lasten der Lesbarkeit gegangen wäre -, aber wir müssen sie auf eine Weise definieren, die eindeutig ist.  Hätten wir z.B. die Klammern in der Definition von $(\varphi\to \psi)$ weggelassen, hätte die wff $\lnot\varphi\to \psi$ entweder als $\lnot(\varphi\to\psi)$ oder $(\lnot\varphi\to\psi)$ interpretiert werden können und hätte uns erlaubt, Unsinn zu beweisen.  Es ist zu beachten, dass unser formales System kein Konzept der Vorrangigkeit von Operatoren enthält.
\end{quotation}}

\begin{sloppy}
\subsection{Beispiel~2 --- Prädikatenlogik mit Gleichheit}\index{Prädikatenlogik}
\end{sloppy}

Hier erweitern wir Beispiel~1 um die Prädikatenlogik mit Gleichheit zu erhalten und veranschaulichen damit die Verwendung von disjunkte Variableneinschränkungen.  Dieses System ist dasselbe wie Tarskis System $\mathfrak{S}_2$ in \cite{Tarski1965} (mit der Ausnahme, dass die Axiome der Aussagenlogik unterschiedlich, aber äquivalent sind, und dass ein redundantes Axiom weggelassen wird).  Wir erweitern $\mbox{\em CN}$ um die Konstanten $\{\mbox{var},\forall,=\}$ und $\mbox{\em VR}$ um eine unendliche Menge von {\em individuelle Metavariablen}\index{individuelle Metavariable} $\{x,y,z,\ldots\}$ und bezeichnen diese Teilmenge als $\mbox{\em Vr}$. 

Wir erweitern $\mbox{\em CN}$ auch um eine möglicherweise unendliche Menge $\mbox{\em Pr}$ von {\em Prädikaten} $\{R,S,\ldots\}$.  Wir assoziieren mit $\mbox{\em Pr}$ eine Funktion $\mbox{rnk}$ von $\mbox{\em Pr}$ nach $\omega$, und für $\alpha\in \mbox{\em Pr}$ nennen wir $\mbox{rnk}(\alpha)$ den {\em Rang} des Prädikats $\alpha$, der einfach die Anzahl der "`Argumente"' ist, die das Prädikat hat.  (Die meisten Anwendungen der Prädikatenlogik haben eine endliche Anzahl von Prädikaten; in der Mengenlehre gibt es z.B. ein einziges Prädikat mit zwei Argumenten (auch binäres Prädikat genannt) $\in$, das üblicherweise mit seinen Argumenten um das Prädikatssymbol herum geschrieben wird und nicht mit der Präfix-Notation, die wir für den allgemeinen Fall verwenden).
Um unsere Diskussion zu erleichtern sei $\mbox{\em Vs}$ eine beliebige feste injektive Funktion von $\omega$ nach $\mbox{\em Vr}$; somit ist $\mbox{\em Vs}$ eine beliebige einfache unendliche Folge von einzelnen Metavariablen ohne Wiederholungen. 

In diesem Beispiel verzichten wir auf die Funktionssymbole, die häufig Teil von Formalisierungen der Prädikatenlogik sind.  Unter Verwendung metalogischer Argumente, die den Rahmen unserer Diskussion sprengen würden, kann gezeigt werden, dass unsere Formalisierung äquivalent ist, wenn Funktionen über geeignete Definitionen eingeführt werden. 

Wir erweitern die in Beispiel~1 definierte Menge $T$ um die Ausdrücke $\{\langle \mbox{var\ } x\rangle,$ $ \langle \mbox{var\ } y\rangle, \langle \mbox{var\ } z\rangle,\ldots\}$ und das obige $\Gamma$ um die axiomatischen Aussagen, die die Redukte der folgenden Prä-Aussagen sind: 
\begin{list}{}{\itemsep 0.0pt}
      \item[] $\langle\varnothing,T,\varnothing,
               \langle \mbox{wff\ }\forall x\,\varphi\rangle\rangle$
      \item[] $\langle\varnothing,T,\varnothing,
               \langle \mbox{wff\ }x=y\rangle\rangle$
      \item[] $\langle\varnothing,T,
               \{\langle\vdash\varphi\rangle\},
               \langle\vdash\forall x\,\varphi\rangle\rangle$
      \item[] $\langle\varnothing,T,\varnothing,
               \langle \vdash((\forall x(\varphi\to\psi)
                  \to(\forall x\,\varphi\to\forall x\,\psi))
               \rangle\rangle$
      \item[] $\langle\{\{x,\varphi\}\},T,\varnothing,
               \langle \vdash(\varphi\to\forall x\,\varphi)
               \rangle\rangle$
      \item[] $\langle\{\{x,y\}\},T,\varnothing,
               \langle \vdash\lnot\forall x\lnot x=y
               \rangle\rangle$
      \item[] $\langle\varnothing,T,\varnothing,
               \langle \vdash(x=z
                  \to(x=y\to z=y))
               \rangle\rangle$
      \item[] $\langle\varnothing,T,\varnothing,
               \langle \vdash(y=z
                  \to(x=y\to x=z))
               \rangle\rangle$
\end{list}
Dies sind die Axiome, die keine Prädikatssymbole beinhalten. Die ersten beiden Anweisungen erweitern die Definition einer wff.  Die dritte ist die Regel der Verallgemeinerung.  Die fünfte besagt: "`Für eine wff $\varphi$ und die Variable $x$ gilt: $\vdash(\varphi\to\forall x\,\varphi)$, sofern $x$ nicht in $\varphi$ vorkommt."'  Die sechste lautet: "`Für die Variablen $x$ und $y$ gilt $\vdash\lnot\forall x\lnot x = y$, sofern $x$ und $y$ verschieden sind."' (Dieser Vorbehalt ist nicht notwendig, wurde aber von Tarski eingefügt, um das Axiom abzuschwächen und trotzdem zu zeigen, dass das System logisch vollständig ist.) 

Schließlich fügen wir für jedes Prädikatssymbol $\alpha\in \mbox{\em Pr}$ eine axiomatische Aussage zu $\Gamma$ hinzu, die die Definition von wff erweitert und die das Redukt der folgenden Prä-Ausage ist: 
\begin{displaymath}
    \langle\varnothing,T,\varnothing,
            \langle \mbox{wff},\alpha\rangle\
            \frown \mbox{\em Vs}\restriction\mbox{rnk}(\alpha)\rangle
\end{displaymath}
und für jedes $\alpha\in \mbox{\em Pr}$ und jedes $n < \mbox{rnk}(\alpha)$ fügen wir zu $\Gamma$ ein Gleichheitsaxiom hinzu, das das Redukt der folgenden Prä-Aussage ist:
\begin{eqnarray*}
    \lefteqn{\langle\varnothing,T,\varnothing,
            \langle
      \vdash,(,\mbox{\em Vs}_n,=,\mbox{\em Vs}_{\mbox{rnk}(\alpha)},\to,
            (,\alpha\rangle\frown \mbox{\em Vs}\restriction\mbox{rnk}(\alpha)} \\
  & & \frown
            \langle\to,\alpha\rangle\frown \mbox{\em Vs}\restriction n\frown
            \langle \mbox{\em Vs}_{\mbox{rnk}(\alpha)}\rangle \\
 & & \frown
            \mbox{\em Vs}\restriction(\mbox{rnk}(\alpha)\setminus(n+1))\frown
            \langle),)\rangle\rangle
\end{eqnarray*}
wobei $\restriction$ die Einschränkung des Definitionsbereichs und $\setminus$ die Mengendifferenz bezeichnet.  Erinnern Sie sich daran, dass ein tiefgestellter Index in $\mbox{\em Vs}$ einen seiner Terme kennzeichnet.  (In den beiden obigen axiomatischen Aussagen werden Kommas zwischen aufeinanderfolgende Terme von Sequenzen gesetzt, um Mehrdeutigkeit zu vermeiden, und wenn Sie sie genau betrachten, werden Sie in der Lage sein, die Klammern, die konstante Symbole bezeichnen, von den Klammern unserer Beschreibungssprache, die Funktionsargumente abgrenzen, zu unterscheiden.  Es wäre vielleicht besser gewesen, unsere primitiven Symbole fett zu schreiben, aber leider waren nicht für alle Zeichen in dem \LaTeX-System, das für den Schriftsatz dieses Textes verwendet wurde, Fettschrift verfügbar).  Diese scheinbar verbotenen Axiome lassen sich in Analogie zur Verkettung von Teilzeichenfolgen in einer Computersprache verstehen.  Tatsächlich sind sie für jeden spezifischen Fall relativ einfach und werden deutlicher, wenn man den Spezialfall eines binären Prädikats $\alpha = R$ betrachtet, bei dem $\mbox{rnk}(R)=2$ ist.  Wenn $\mbox{\em Vs}$ die Folge $\langle x,y,z,\ldots\rangle$ ist, wären die Axiome, die wir für diesen Fall zu $\Gamma$ hinzufügen würden, die wff-Erweiterung und zwei Gleichheitsaxiome, die die Redukte der folgenden Aussagen sind: 
\begin{list}{}{\itemsep 0.0pt}
      \item[] $\langle\varnothing,T,\varnothing,
               \langle \mbox{wff\ }R x y\rangle\rangle$
      \item[] $\langle\varnothing,T,\varnothing,
               \langle \vdash(x=z
                  \to(R x y \to R z y))
               \rangle\rangle$
      \item[] $\langle\varnothing,T,\varnothing,
               \langle \vdash(y=z
                  \to(R x y \to R x z))
               \rangle\rangle$
\end{list}
Studieren Sie diese sorgfältig, um zu sehen, wie sie aus den obigen allgemeinen Axiome entstehen.  In der Praxis werden typischerweise nur wenige Spezialfälle wie dieser benötigt, und in jedem Fall erlaubt uns die Metamath-Sprache nur die Beschreibung einer endlichen Anzahl von Prädikaten, im Gegensatz zu der unendlichen Anzahl, die das formale System erlaubt.  (Sollte aus irgendeinem Grund eine unendliche Anzahl benötigt werden, könnten wir das formale System nicht direkt in der Metamath-Sprache definieren, sondern könnten es stattdessen metalogisch unter der Mengenlehre definieren, wie wir es in diesem Anhang tun, und nur die zugrunde liegende Mengenlehre mit ihrem einzigen binären Prädikat würde direkt in der Metamath-Sprache definiert). 


{\footnotesize\begin{quotation}
{\em Kommentar:}  Wie bereits erwähnt, handelt es sich bei den spezifischen Variablen, die durch die Symbole $x,y,z,\ldots\in \mbox{\em Vr}\subseteq \mbox{\em VR}\subseteq \mbox{\em SM}$ in Beispiel~2 dargestellt werden, nicht um die eigentlichen Variablen der gewöhnlichen Prädikatenlogik, sondern sie sind als Metavariablen zu betrachten, die sich über diese erstrecken.  Zum Beispiel wäre eine disjunkte Variableneinschränkung für eigentliche Variablen der gewöhnlichen Prädikatenlogik bedeutungslos, da zwei verschiedene eigentliche Variablen per Definition unterschiedlich sind.  Und wenn wir über einen beliebigen Repräsentanten $\alpha\in \mbox{\em Vr}$ sprechen, ist $\alpha$ eine Metavariable (in unserer Erklärungssprache), die sich über Metavariablen erstreckt (die Primitive unseres formalen Systems sind), von denen sich jede über die einzelnen eigentlichen Variablen der Prädikatenlogik erstreckt (die in unserem formalen System nie erwähnt werden).  

Die oben genannte Konstante "`var"' heißt in der \texttt{set.mm}-Datenbasis \texttt{setvar}, aber sie bedeutet dasselbe.  Ich war der Meinung, dass "`var"' im Kontext der Prädikatenlogik, deren Verwendung nicht auf die Mengenlehre beschränkt ist, ein sinnvollerer Name ist.  Aus Gründen der Konsistenz bleiben wir in diesem Anhang bei dem Namen "`var"', auch nachdem die Mengenlehre eingeführt wurde.
\end{quotation}}

\subsection{Freie Variablen und echte Substitution}\index{freie Variable}
\index{echte Substitution}\index{Substitution!echte}

Typische Darstellungen mathematischer Axiome verwenden Konzepte wie "`freie Variable"', "`gebundene Variable"' und "`echte Substitution"' als primitive Begriffe. Eine freie Variable ist eine Variable, die kein Parameter eines Containerausdrucks ist. Eine gebundene Variable ist das Gegenteil einer freien Variable; sie ist eine Variable, die in einem Containerausdruck gebunden wurde. Zum Beispiel ist in dem Ausdruck $\forall x \varphi$ (für alle $x$ ist $\varphi$ wahr) die Variable $x$ in dem "`für alle"'-Ausdruck ($\forall$) gebunden. Es ist möglich, eine Variable durch eine andere zu ersetzen, und diesen Vorgang nennt man "`echte Substitution"'. In den meisten Büchern hat die echte Substitution eine etwas komplizierte rekursive Definition mit mehreren Fällen, die auf dem Vorkommen von freien und gebundenen Variablen basieren.
Sie können in \cite[ch.\ 3--4]{Hamilton}\index{Hamilton, Alan G.} (sowie in vielen anderen Texten) für weitere formale Details zu diesen Begriffen nachschauen. 

Die Verwendung dieser Konzepte als \texttt{primitives} schafft Komplikationen für Computerimplementierungen. 

In dem System von Beispiel~2 gibt es keine primitiven Begriffe für freie Variablen und die echten Substitution.  Tarski \cite{Tarski1965} zeigt, dass dieses System logisch äquivalent zu den typischeren Lehrbuchsystemen ist, die diese primitiven Begriffe haben, wenn wir diese Begriffe mit geeigneten Definitionen und Metalogik einführen.  Wir könnten auch direkt Axiome für solche Systeme definieren, obwohl die rekursiven Definitionen der freien Variablen und der echten Substitution unübersichtlich und umständlich zu handhaben wären.  Stattdessen weisen wir auf zwei Hilfsmittel hin, die in der Praxis verwendet werden können, um diese Begriffe zu imitieren.  (1) Anstatt eine spezielle Notation einzuführen, um (als logische Hypothese) "`wobei $x$ nicht frei in $\varphi$ ist"' auszudrücken, können wir die logische Hypothese $\vdash(\varphi\to\forall x\,\varphi)$ verwenden.\label{effectivelybound}\index{effektiv nicht frei}\footnote{Dies ist eine etwas schwächere Anforderung als "`wobei $x$ nicht frei in $\varphi$ ist"'.  Ersetzen wir $\varphi$ durch $x=x$, so haben wir den Satz $(x=x\to\forall x\,x=x)$, der die Hypothese erfüllt, obwohl $x$ in $x=x$ frei ist. In einem solchen Fall sagen wir, dass $x$ {\em effektiv nicht frei}\index{effektiv nicht frei} in $x=x$ ist, da $x=x$ logisch äquivalent zu $\forall x\,x=x$ ist, in dem $x$ gebunden ist.} (2) Es kann gezeigt werden, dass die wff $((x=y\to\varphi)\wedge\exists x(x=y\wedge\varphi))$ (mit den üblichen Definitionen von $\wedge$ und $\exists$; siehe Beispiel~4 unten) logisch äquivalent ist zu "`die wff, die sich aus der echten Substitutiong von $y$ für $x$ in $\varphi$ ergibt"'.  Das funktioniert unabhängig davon, ob $x$ und $y$ verschieden sind oder nicht. 

\subsection{Metalogische Vollständigkeit}\index{metalogische Vollständigkeit}

In dem System von Beispiel~2 sind die folgenden Prä-Aussagen beweisbar (und ihre Redukte sind beweisbare Aussagen): 
\begin{eqnarray*}
      & \langle\{\{x,y\}\},T,\varnothing,
               \langle \vdash\lnot\forall x\lnot x=y
               \rangle\rangle & \\
     &  \langle\varnothing,T,\varnothing,
               \langle \vdash\lnot\forall x\lnot x=x
               \rangle\rangle &
\end{eqnarray*}
wohingegen die folgende Prä-Aussage meines Wissens nach nicht beweisbar ist (aber wir werden in der folgenden Diskussion so tun, als ob sie es nicht wäre: 
\begin{eqnarray*}
     &  \langle\varnothing,T,\varnothing,
               \langle \vdash\lnot\forall x\lnot x=y
               \rangle\rangle &
\end{eqnarray*}
Mit anderen Worten, wir können "`$\lnot\forall x\lnot x=y$, wobei $x$ und $y$ verschieden sind"' und separat "`$\lnot\forall x\lnot x=x$"' beweisen, aber wir können den kombinierten allgemeinen Fall "`$\lnot\forall x\lnot x=y$"' nicht beweisen, der keine zusätzlich Bedingung hat.  Dies beeinträchtigt jedoch nicht die logische Vollständigkeit, da die Variablen wirklich Metavariablen sind und die beiden beweisbaren Fälle zusammen alle möglichen Fälle abdecken.  Der dritte Fall kann als ein Metatheorem betrachtet werden, dessen direkter Beweis mit dem System von Beispiel~2 außerhalb der Möglichkeiten des formalen Systems liegt. 

Außerdem ist im System von Beispiel~2 die folgende Anweisung meines Wissens nicht beweisbar (wiederum eine Vermutung, die wir als wahr unterstellen werden): 
\begin{eqnarray*}
     & \langle\varnothing,T,\varnothing,
               \langle \vdash(\forall x\, \varphi\to\varphi)
               \rangle\rangle &
\end{eqnarray*}
Stattdessen können wir nur spezielle Fälle von $\varphi$ mit individuellen Metavariablen beweisen und durch Induktion über die Formellänge die obige allgemeine Anweisung als Metatheorem außerhalb unseres formalen Systems beweisen.  Die Einzelheiten dieses Beweises finden sich in \cite{Kalish}. 

Es gibt jedoch ein System der Prädikatenlogik, in dem alle derartigen "`einfachen Metatheoreme"' wie die obigen direkt bewiesen werden können, und wir stellen es in Beispiel~3 vor. Ein {\em einfaches Metatheorem}\index{einfaches Metatheorem} ist jede Aussage des formalen Systems aus Beispiel~2, in dem alle disjukte Variableneinschränkung entweder aus zwei individuellen Metavariablen oder einer individuellen Metavariablen und einer wff-Metavariablen bestehen, und die durch Kombination von Fällen außerhalb des Systems wie oben bewiesen werden kann.  Ein System ist {\em metalogisch vollständig}\index{metalogische Vollständigkeit}, wenn alle seine einfachen Metatheoreme (direkt) beweisbare Aussagen sind. Die genaue Definition von "`einfachem Metatheorem"' und der Beweis der "`metalogischen Vollständigkeit"' von Beispiel~3 findet sich in Bemerkung 9.6 und Theorem 9.7 von \cite{Megill}.\index{Megill, Norman} 

\begin{sloppy}
\subsection{Beispiel~3 --- Metalogisch vollständige Prädikatenlogik mit Gleichheit}
\end{sloppy}

Der Einfachheit halber nehmen wir an, dass es ein binäres Prädikat $R$ gibt; dieses System reicht für die Mengenlehre aus, wobei das $R$ natürlich das Prädikat $\in$ ist.  Wir enennen die Axiome so, wie sie in \cite{Megill} vorkommen.  Dieses System ist logisch äquivalent zu dem in Beispiel~2 (wenn letzteres auf dieses eine binäre Prädikat beschränkt wird), ist aber auch metalogisch vollständig\index{metalogische Vollständigkeit}.

Angenommen
\begin{itemize}
  \item[] $\mbox{\em CN}=\{\mbox{wff}, \mbox{var}, \vdash, \to, \lnot, (,),\forall,=,R\}$.
  \item[] $\mbox{\em VR}=\{\varphi,\psi,\chi,\ldots\}\cup\{x,y,z,\ldots\}$.
  \item[] $T = \{\langle \mbox{wff\ } \varphi\rangle,
             \langle \mbox{wff\ } \psi\rangle,
             \langle \mbox{wff\ } \chi\rangle,\ldots\}\cup
       \{\langle \mbox{var\ } x\rangle, \langle \mbox{var\ } y\rangle, \langle
       \mbox{var\ }z\rangle,\ldots\}$.

\noindent Dann besteht $\Gamma$ aus den Redukten der folgenden Prä-Aussagen:
    \begin{itemize}
      \item[] $\langle\varnothing,T,\varnothing,
               \langle \mbox{wff\ }(\varphi\to\psi)\rangle\rangle$
      \item[] $\langle\varnothing,T,\varnothing,
               \langle \mbox{wff\ }\lnot\varphi\rangle\rangle$
      \item[] $\langle\varnothing,T,\varnothing,
               \langle \mbox{wff\ }\forall x\,\varphi\rangle\rangle$
      \item[] $\langle\varnothing,T,\varnothing,
               \langle \mbox{wff\ }x=y\rangle\rangle$
      \item[] $\langle\varnothing,T,\varnothing,
               \langle \mbox{wff\ }Rxy\rangle\rangle$
      \item[(C1$'$)] $\langle\varnothing,T,\varnothing,
               \langle \vdash(\varphi\to(\psi\to\varphi))
               \rangle\rangle$
      \item[(C2$'$)] $\langle\varnothing,T,
               \varnothing,
               \langle \vdash((\varphi\to(\psi\to\chi))\to
               ((\varphi\to\psi)\to(\varphi\to\chi)))
               \rangle\rangle$
      \item[(C3$'$)] $\langle\varnothing,T,
               \varnothing,
               \langle \vdash((\lnot\varphi\to\lnot\psi)\to
               (\psi\to\varphi))\rangle\rangle$
      \item[(C4$'$)] $\langle\varnothing,T,
               \varnothing,
               \langle \vdash(\forall x(\forall x\,\varphi\to\psi)\to
                 (\forall x\,\varphi\to\forall x\,\psi))\rangle\rangle$
      \item[(C5$'$)] $\langle\varnothing,T,
               \varnothing,
               \langle \vdash(\forall x\,\varphi\to\varphi)\rangle\rangle$
      \item[(C6$'$)] $\langle\varnothing,T,
               \varnothing,
               \langle \vdash(\forall x\forall y\,\varphi\to
                 \forall y\forall x\,\varphi)\rangle\rangle$
      \item[(C7$'$)] $\langle\varnothing,T,
               \varnothing,
               \langle \vdash(\lnot\varphi\to\forall x\lnot\forall x\,\varphi
                 )\rangle\rangle$
      \item[(C8$'$)] $\langle\varnothing,T,
               \varnothing,
               \langle \vdash(x=y\to(x=z\to y=z))\rangle\rangle$
      \item[(C9$'$)] $\langle\varnothing,T,
               \varnothing,
               \langle \vdash(\lnot\forall x\, x=y\to(\lnot\forall x\, x=z\to
                 (y=z\to\forall x\, y=z)))\rangle\rangle$
      \item[(C10$'$)] $\langle\varnothing,T,
               \varnothing,
               \langle \vdash(\forall x(x=y\to\forall x\,\varphi)\to
                 \varphi))\rangle\rangle$
      \item[(C11$'$)] $\langle\varnothing,T,
               \varnothing,
               \langle \vdash(\forall x\, x=y\to(\forall x\,\varphi
               \to\forall y\,\varphi))\rangle\rangle$
      \item[(C12$'$)] $\langle\varnothing,T,
               \varnothing,
               \langle \vdash(x=y\to(Rxz\to Ryz))\rangle\rangle$
      \item[(C13$'$)] $\langle\varnothing,T,
               \varnothing,
               \langle \vdash(x=y\to(Rzx\to Rzy))\rangle\rangle$
      \item[(C15$'$)] $\langle\varnothing,T,
               \varnothing,
               \langle \vdash(\lnot\forall x\, x=y\to(x=y\to(\varphi
                 \to\forall x(x=y\to\varphi))))\rangle\rangle$
      \item[(C16$'$)] $\langle\{\{x,y\}\},T,
               \varnothing,
               \langle \vdash(\forall x\, x=y\to(\varphi\to\forall x\,\varphi)
                 )\rangle\rangle$
      \item[(C5)] $\langle\{\{x,\varphi\}\},T,\varnothing,
               \langle \vdash(\varphi\to\forall x\,\varphi)
               \rangle\rangle$
      \item[(MP)] $\langle\varnothing,T,
               \{\langle\vdash(\varphi\to\psi)\rangle,
                 \langle\vdash\varphi\rangle\},
               \langle\vdash\psi\rangle\rangle$
      \item[(Gen)] $\langle\varnothing,T,
               \{\langle\vdash\varphi\rangle\},
               \langle\vdash\forall x\,\varphi\rangle\rangle$
    \end{itemize}
\end{itemize}

Es ist zwar bekannt, dass diese Axiome "`metalogisch vollständig"' sind, aber es ist nicht bekannt, ob sie im metalogischen Sinne unabhängig sind (d.h. keines ist redundant); insbesondere, ob irgendein Axiom (möglicherweise mit zusätzlichen optionalen disjunkte Variableneinschränkungen zur Verwendung von beliebigen Dummy-Variablen in seinem Beweis) aus den anderen beweisbar ist.  Beachten Sie, dass metalogische Unabhängigkeit eine schwächere Anforderung ist als Unabhängigkeit im üblichen logischen Sinne.  Nicht alle der oben genannten Axiome sind logisch unabhängig: beispielsweise kann C9$'$ als Metatheorem aus den anderen bewiesen werden, und zwar außerhalb des formalen Systems, indem die möglichen Fälle von unterscheidbaren Variablen kombiniert werden. 

\subsection{Beispiel~4 --- Hinzufügen von Definitionen}\index{Definition}

Es gibt mehrere Möglichkeiten, einem formalen System Definitionen hinzuzu\-fügen.  Der wahrscheinlich beste Weg ist, Definitionen überhaupt nicht als Teil des formalen Systems zu betrachten, sondern als Abkürzungen, die Teil der erklärenden Metalogik außerhalb des formalen Systems sind.  Der Einfachheit halber können wir jedoch das formale System selbst verwenden, um Definitionen einzubeziehen, indem wir sie als axiomatische Erweiterungen zum System hinzufügen.  Dies könnte durch das Hinzufügen einer Konstante geschehen, die den Begriff "`ist definiert als"' zusammen mit Axiomen für diesen Begriff repräsentiert. Aber es gibt einen schöneren Weg, zumindest meiner Meinung nach, der Definitionen als direkte Erweiterungen der Sprache und nicht als extralogische primitive Begriffe einführt.  Wir führen zusätzliche logische Junktoren ein und stellen Axiome für sie bereit.  Für Logiksysteme wie die Beispiele 1 bis 3 müssen die zusätzlichen Axiome in dem Sinne konservativ sein, dass keine wff des ursprünglichen Systems, das kein Theorem war (wenn der ursprüngliche Begriff "`wff"' natürlich durch "`$\vdash$"' ersetzt wird), zu einem Theorem des erweiterten Systems wird.  In diesem Beispiel erweitern wir Beispiel~3 (oder 2) mit Standardabkürzungen der Logik. 

Wir erweitern $\mbox{\em CN}$ aus Beispiel~3 um neue Konstanten $\{\leftrightarrow, \wedge,\vee,\exists\}$, die der logischen Äquivalenz\index{logische Äquivalenz ($\leftrightarrow$)}\index{Bikonditional ($\leftrightarrow$)}, Konjunktion\index{Konjunktion ($\wedge$)}, Disjunktion\index{Disjunktion ($\vee$)} und dem Existenzquantor\index{Existenzquantor ($\exists$)} entsprechen. Wir erweitern $\Gamma$ um die axiomatischen Ausagen, die die Redukte der folgenden Prä-Aussagen sind: 
\begin{list}{}{\itemsep 0.0pt}
      \item[] $\langle\varnothing,T,\varnothing,
               \langle \mbox{wff\ }(\varphi\leftrightarrow\psi)\rangle\rangle$
      \item[] $\langle\varnothing,T,\varnothing,
               \langle \mbox{wff\ }(\varphi\vee\psi)\rangle\rangle$
      \item[] $\langle\varnothing,T,\varnothing,
               \langle \mbox{wff\ }(\varphi\wedge\psi)\rangle\rangle$
      \item[] $\langle\varnothing,T,\varnothing,
               \langle \mbox{wff\ }\exists x\, \varphi\rangle\rangle$
  \item[] $\langle\varnothing,T,\varnothing,
     \langle\vdash ( ( \varphi \leftrightarrow \psi ) \to
     ( \varphi \to \psi ) )\rangle\rangle$
  \item[] $\langle\varnothing,T,\varnothing,
     \langle\vdash ((\varphi\leftrightarrow\psi)\to
    (\psi\to\varphi))\rangle\rangle$
  \item[] $\langle\varnothing,T,\varnothing,
     \langle\vdash ((\varphi\to\psi)\to(
     (\psi\to\varphi)\to(\varphi
     \leftrightarrow\psi)))\rangle\rangle$
  \item[] $\langle\varnothing,T,\varnothing,
     \langle\vdash (( \varphi \wedge \psi ) \leftrightarrow\neg ( \varphi
     \to \neg \psi )) \rangle\rangle$
  \item[] $\langle\varnothing,T,\varnothing,
     \langle\vdash (( \varphi \vee \psi ) \leftrightarrow (\neg \varphi
     \to \psi )) \rangle\rangle$
  \item[] $\langle\varnothing,T,\varnothing,
     \langle\vdash (\exists x \,\varphi\leftrightarrow
     \lnot \forall x \lnot \varphi)\rangle\rangle$
\end{list}
Die ersten drei logischen Axiome (Aussagen, die "`$\vdash$"' enthalten) führen die logische Äquivalenz, "`$\leftrightarrow$"', ein und definieren sie effektiv.  Die letzten drei verwenden "`$\leftrightarrow$"' effektiv in der bedeutung von "`ist definiert als"'. 


\subsection{Beispiel~5 --- ZFC Mengenlehre}\index{ZFC-Mengenlehre}

Hier fügen wir dem System in Beispiel~4 die Axiome der Zermelo--Fraenkel-Mengenlehre mit Auswahlaxionm hinzu.  Der Einfachheit halber verwenden wir die Definitionen aus Beispiel~4. 

In der $\mbox{\em CN}$ aus Beispiel~4 (die Beispiel~3 erweitert), ersetzen wir das Symbol $R$ durch das Symbol $\in$. Genauer gesagt, wir entfernen aus $\Gamma$ aus Beispiel~4 die drei axiomatischen Aussagegen, die $R$ enthalten, und ersetzen sie durch die Redukte der folgenden: 
\begin{list}{}{\itemsep 0.0pt}
      \item[] $\langle\varnothing,T,\varnothing,
               \langle \mbox{wff\ }x\in y\rangle\rangle$
      \item[] $\langle\varnothing,T,
               \varnothing,
               \langle \vdash(x=y\to(x\in z\to y\in z))\rangle\rangle$
      \item[] $\langle\varnothing,T,
               \varnothing,
               \langle \vdash(x=y\to(z\in x\to z\in y))\rangle\rangle$
\end{list}
Unter der Annahme, dass $D=\{\{\alpha,\beta\}\in \mbox{\em DV}\,|\alpha,\beta\in \mbox{\em Vr}\}$ (mit anderen Worten müssen alle einzelnen Variablen verschieden sein), erweitern wir $\Gamma$ um die ZFC-Axiome, genannt 
Extensionalität\index{Extensionalitätsaxiom},
Ersetzung\index{Ersetzungsaxiom},
Vereinigung\index{Vereinigungsaxiom},
Potenzmenge\index{Potenzmengenaxiom},
Regelmäßigkeit\index{Fundierungsaxiom},
Unendlichkeit\index{Unendlichkeitsaxiom} und
dem Auswahlaxiom\index{Auswahlaxiom}, die die Redukte der folgenden Prä-Aussagen sind: 
\begin{list}{}{\itemsep 0.0pt}
      \item[Ext] $\langle D,T,
               \varnothing,
               \langle\vdash (\forall x(x\in y\leftrightarrow x \in z)\to y
               =z) \rangle\rangle$
      \item[Rep] $\langle D,T,
               \varnothing,
               \langle\vdash\exists x ( \exists y \forall z (\varphi \to z = y
                        ) \to
                        \forall z ( z \in x \leftrightarrow \exists x ( x \in
                        y \wedge \forall y\,\varphi ) ) )\rangle\rangle$
      \item[Un] $\langle D,T,
               \varnothing,
               \langle\vdash \exists x \forall y ( \exists x ( y \in x \wedge
               x \in z ) \to y \in x ) \rangle\rangle$
      \item[Pow] $\langle D,T,
               \varnothing,
               \langle\vdash \exists x \forall y ( \forall x ( x \in y \to x
               \in z ) \to y \in x ) \rangle\rangle$
      \item[Reg] $\langle D,T,
               \varnothing,
               \langle\vdash (  x \in y \to
                 \exists x ( x \in y \wedge \forall z ( z \in x \to \lnot z
                \in y ) ) ) \rangle\rangle$
      \item[Inf] $\langle D,T,
               \varnothing,
               \langle\vdash \exists x(y\in x\wedge\forall y(y\in
               x\to
               \exists z(y \in z\wedge z\in x))) \rangle\rangle$
      \item[AC] $\langle D,T,
               \varnothing,
               \langle\vdash \exists x \forall y \forall z ( ( y \in z
               \wedge z \in w ) \to \exists w \forall y ( \exists w
              ( ( y \in z \wedge z \in w ) \wedge ( y \in w \wedge w \in x
              ) ) \leftrightarrow y = w ) ) \rangle\rangle$
\end{list}

\subsection{Beispiel~6 --- Begriff der Klasse in der Mengenlehre}\label{class}

Ein leistungsfähiges Hilfsmittel, das die Mengenlehre vereinfacht (und das wir die ganze Zeit in unserer informellen Beschreibungssprache verwendet haben), ist die Notation der {\em Klassenabstraktion}\index{Klassenabstraktion}\index{Abstraktionsklasse}.  Die von uns eingeführten Definitionen werden von Takeuti und Zaring \cite{Takeuti}\index{Takeuti, Gaisi} oder Quine \cite{Quine}\index{Quine, Willard Van Orman} rigoros als konservativ nachgewiesen.  Die Schlüsselidee ist die Einführung der Notation $\{x|\mbox{---}\}$ für Abstraktionsklassen, was "`die Klasse aller $x$, so dass ---"' bedeutet, und die Einführung von (Meta-)Variablen, die sich über sie erstrecken.  Eine Abstraktionsklasse kann eine Menge sein oder auch nicht, je nachdem, ob sie (als Menge) existiert.  Eine Klasse, die nicht (als Menge) existiert, nennt man eine {\em echte Klasse}\index{echte Klasse}\index{Klasse!echte}.

Zur Veranschaulichung der Verwendung von Abstraktionsklassen geben wir einige Beispiele für Definitionen, die von ihnen Gebrauch machen: die leere Menge, die Klassenvereinigung und das ungeordnete Paar.  Viele weitere derartige Definitionen finden sich in der Metamath-Datenbasis für Mengenlehre, \texttt{set.mm}\index{Mengenlehre-Datenbasis (\texttt{set.mm})}. 

% We intentionally break up the sequence of math symbols here
% because otherwise the overlong line goes beyond the page in narrow mode.
Wir erweitern $\mbox{\em CN}$ aus Beispiel~5 um neue Symbole $\{$ $\mbox{class},$ $\{,$ $|,$ $\},$ $\varnothing,$ $\cup,$ $,$ $\}$, wobei die inneren Klammern und das letzte Komma konstante Symbole sind. (Wie zuvor sollte unsere doppelte Verwendung einiger mathematischer Symbole sowohl für unsere Beschreibungssprache als auch als Primitive des formalen Systems aus dem Kontext heraus klar sein). 

Wir erweitern $\mbox{\em VR}$ aus Beispiel~5 mit einer Menge von {\em Klassenvariablen}\index{Klassenvariable} $\{A,B,C,\ldots\}$. Wir erweitern das $T$ aus Beispiel~5 mit $\{\langle \mbox{class\ } A\rangle, \langle \mbox{class\ }B\rangle,$ $\langle \mbox{class\ } C\rangle, \ldots\}$. 

Um unsere Definitionen einzuführen, fügen wir zu $\Gamma$ aus Beispiel~5 die axiomatischen Aussagen hinzu, die die Redukte der folgenden Prä-Aussagen sind: 
\begin{list}{}{\itemsep 0.0pt}
      \item[] $\langle\varnothing,T,\varnothing,
               \langle \mbox{class\ }x\rangle\rangle$
      \item[] $\langle\varnothing,T,\varnothing,
               \langle \mbox{class\ }\{x|\varphi\}\rangle\rangle$
      \item[] $\langle\varnothing,T,\varnothing,
               \langle \mbox{wff\ }A=B\rangle\rangle$
      \item[] $\langle\varnothing,T,\varnothing,
               \langle \mbox{wff\ }A\in B\rangle\rangle$
      \item[Ab] $\langle\varnothing,T,\varnothing,
               \langle \vdash ( y \in \{ x |\varphi\} \leftrightarrow
                  ( ( x = y \to\varphi) \wedge \exists x ( x = y
                  \wedge\varphi) ))
               \rangle\rangle$
      \item[Eq] $\langle\{\{x,A\},\{x,B\}\},T,\varnothing,
               \langle \vdash ( A = B \leftrightarrow
               \forall x ( x \in A \leftrightarrow x \in B ) )
               \rangle\rangle$
      \item[El] $\langle\{\{x,A\},\{x,B\}\},T,\varnothing,
               \langle \vdash ( A \in B \leftrightarrow \exists x
               ( x = A \wedge x \in B ) )
               \rangle\rangle$
\end{list}
Hier sagen wir, dass eine individuelle Variable eine Klasse ist; $\{x|\varphi\}$ ist eine Klasse; und wir erweitern die Definition einer wff, um Klassengleichheit und -zugehörigkeit einzuschließen.  Axiom Ab definiert die Zugehörigkeit einer Variablen zu einer Klassenabstraktion; die rechte Seite kann gelesen werden als "`die wff, die sich aus der echten Substitution von $y$ für $x$ in $\varphi$ ergibt."'\footnote{Anmerkung: Diese Definition macht die Einführung einer separaten Notation ähnlich $\varphi(x|y)$ für die echte Substitution überflüssig, obwohl wir dies aus Gründen der Konvention tun könnten.  Übrigens wäre $\varphi(x|y)$ in seiner jetzigen Form in den formalen Systemen unserer Beispiele mehrdeutig, da wir nicht wissen würden, ob $\lnot(\varphi(x|y)$ entweder $\lnot(\varphi(x|y))$ oder $(\lnot\varphi)(x|y)$ bedeutet. Stattdessen müssten wir eine eindeutige Variante wie $(\varphi\, x|y)$ verwenden.}  Die Axiome Eq und El erweitern die Bedeutung des bestehenden Gleichheitszeichens und des Elementprädikats.  Dies ist potenziell gefährlich und muss sorgfältig begründet werden.  Zum Beispiel können wir aus Eq das Extensionalitätsaxiom allein mit Prädikatenlogik ableiten; daher sollten wir das Extensionalitätsaxiom im Prinzip als logische Hypothese aufnehmen.  Wir machen uns jedoch nicht die Mühe, dies zu tun, da wir dieses Axiom bereits vorher vorausgesetzt haben. Die disjunkte Variableneinschränkungen sollten verstanden werden als: "`wobei $x$ nicht in $A$ oder $B$ vorkommt."'  Wir tun dies typischerweise, wenn die rechte Seite einer Definition eine individuelle Variable beinhaltet, die nicht in dem zu definierenden Ausdruck vorkommt; dies geschieht, damit die rechte Seite unabhängig von der speziellen "`Dummy'-Variable bleibt, die wir verwenden. 

Wir fügen $\Gamma$ weiterhin die folgenden Definitionen (d.h. die Reduktionen der folgenden Prä-Aussagen) für die leere Menge,\index{leere Menge}, die Klassenvereinigung,\index{Vereinigung} und das ungeordnete Paar\index{ungeordnetes Paar} hinzu.  Sie sollten selbsterklärend sein.  Analog zu unserer Verwendung von "`$\leftrightarrow$"' zur Definition neuer wffs in Beispiel~4, verwenden wir "`$=$"' zur Definition neuer Abstraktionsbegriffe, und beide können in diesem Zusammenhang informell als "`ist definiert als"' gelesen werden. 
\begin{list}{}{\itemsep 0.0pt}
      \item[] $\langle\varnothing,T,\varnothing,
               \langle \mbox{class\ }\varnothing\rangle\rangle$
      \item[] $\langle\varnothing,T,\varnothing,
               \langle \vdash \varnothing = \{ x | \lnot x = x \}
               \rangle\rangle$
      \item[] $\langle\varnothing,T,\varnothing,
               \langle \mbox{class\ }(A\cup B)\rangle\rangle$
      \item[] $\langle\{\{x,A\},\{x,B\}\},T,\varnothing,
               \langle \vdash ( A \cup B ) = \{ x | ( x \in A \vee x \in B ) \}
               \rangle\rangle$
      \item[] $\langle\varnothing,T,\varnothing,
               \langle \mbox{class\ }\{A,B\}\rangle\rangle$
      \item[] $\langle\{\{x,A\},\{x,B\}\},T,\varnothing,
               \langle \vdash \{ A , B \} = \{ x | ( x = A \vee x = B ) \}
               \rangle\rangle$
\end{list}

\section{Metamath als formales System}\label{theorymm}

Dieser Abschnitt setzt die Kenntnis der Computersprache Metamath voraus.

Unsere Theorie beschreibt formale Systeme und ihre Universen.  Die Metamath-Sprache bietet eine Möglichkeit, diese mengentheoretischen Objekte auf einem Computer darzustellen.  Eine Metamath-Datenbasis, die aus einer endlichen Menge von {\sc ascii}-Zeichen besteht, kann in der Regel nur eine Teilmenge eines formalen Systems und seines Universums, die normalerweise unendlich sind, beschreiben.  Allerdings kann die Datenbasis eine beliebig große endliche Teilmenge des formalen Systems und seines Universums enthalten.  (Natürlich kann eine Metamath-Mengenlehre-Datenbasis im Prinzip indirekt ein ganzes unendliches formales System beschreiben, indem sie die Beschreibungssprache in diesem Anhang formalisiert). 

Für unsere Diskussion gehen wir davon aus, dass die Metamath-Datenbasis die auf S.~\pageref{framelist} beschriebene einfache Form hat, die aus allen Konstanten- und Variablendeklarationen am Anfang besteht, gefolgt von einer Folge erweiterter Frames, die jeweils durch \texttt{\$\char`\{} und \texttt{\$\char`\}} begrenzt sind.  Jede Metamath-Datenbasis kann in diese Form konvertiert werden, wie auf S.~\pageref{frameconvert} beschrieben. 

Die mathematischen Symbol-Token einer Metamath-Quelldatei, die mit den Anweisungen \texttt{\$c} und \texttt{\$v} deklariert werden, sind Namen, die wir den Repräsentanten von $\mbox{\em CN}$ und $\mbox{\em VR}$ zuweisen.  Der Eindeutigkeit halber könnten wir annehmen, dass das erste mathematische Symbol, das als Variable deklariert wird, $v_0$ entspricht, das zweite $v_1$ usw., obwohl die gewählte Zuordnung nicht von bedeutung ist. 

In der Metamath-Sprache entspricht jede \texttt{\$d}-, \texttt{\$f}- und \texttt{\$e}-Quellanweisung in einem erweiterten Rahmen (Abschnitt~\ref{frames}) jeweils einem Element der Sammlungen $D$, $T$ und $H$ in einer Aussage  $\langle D_M,T_M,H,A\rangle$ des formalen Systems.  Die auf diese Metamath-Schlüsselwörter folgenden Zeichenketten mit mathematischen Symbolen entsprechen einem Variablenpaar (im Fall von \texttt{\$d}) oder einem Ausdruck (für die beiden anderen Schlüsselwörter). Die mathematische Symbolkette nach einer \texttt{\$a}-Quellanweisung entspricht dem Ausdruck $A$ in einer axiomatischen Aussage des formalen Systems; die nach einer \texttt{\$p}-Quellanweisung entspricht $A$ in einer beweisbaren Aussage, die nicht axiomatisch ist.  Mit anderen Worten: Jeder erweiterte Rahmen in einer Metamath-Datenbasis entspricht einer Prä-Aussage des formalen Systems, und ein Rahmen entspricht einer Aussage des formalen Systems\footnote{Anm. der Übersetzer: Im englischen Originaltext folgt hier ein Hinweis auf die doppelte Bedeutung des Wortes "`statement"', was wir durch die unterschiedliche Übersetzung ("`Aussage"' im formalen System, "`Anweisung"' in der Metamath-Datenbasis) vermieden haben.}.  

Damit der Computer überprüfen kann, ob eine Aussage des formalen Systems beweisbar ist, wird jede entsprechende Anweisung von einem Beweis begleitet. Der Beweis hat jedoch keine Entsprechung im formalen System, sondern ist lediglich eine Möglichkeit, dem Computer die für seine Verifikation benötigten Informationen mitzuteilen.  Der Beweis sagt dem Computer, {\em wie} er bestimmte Glieder des Abschlusses der Prä-Aussage des formalen Systems konstruieren soll, die dem erweiterten Rahmen der \texttt{\$p}-Anweisung entspricht.  Das Endergebnis der Konstruktion ist das Element des Abschlusses, das mit der Anweisung \texttt{\$p} übereinstimmt.  Das abstrakte formale System hingegen befasst sich nur mit der {\em Existenz} von Elementen des Abschlusses. 

Wie auf S.~\pageref{exampleref} erwähnt, entsprechen die Beispiele 1 und 3--6 im vorigen Abschnitt der Entwicklung der Logik und Mengenlehre in der Metamath-Datenbasis \texttt{set.mm}.\index{Mengenlehre-Datenbasis (\texttt{set.mm})} Es ist sicherlich aufschlussreich, sie zu vergleichen. 


\chapter{Das MIU-System}
\label{MIU}
\index{formales System}
\index{MIU-System}

Es folgt eine (übersetzte\footnote{Anm. der Übersetzer: Übersetzung angelehnt an die deutschen Übersetzung von \textit{Gödel, Escher, Bach: ein Endloses Geflochtenes Band} (Klett-Cotta, Stuttgart, 1986), S. 37ff.}) Auflistung der Datei \texttt{miu.mm}.  Sie ist selbsterklärend.

%%%%%%%%%%%%%%%%%%%%%%%%%%%%%%%%%%%%%%%%%%%%%%%%%%%%%%%%%%%%

\begin{verbatim}
$( Das MIU-System: Ein einfaches formales System $)

$( Hinweis: Dieses formale System ist insofern ungewöhnlich,
als es leere wffs zulässt.  Um mit einem Beweis zu arbeiten,
müssen Sie SET EMPTY_SUBSTITUTION ON eingeben, bevor Sie den
Befehl PROVE verwenden. Standardwert ist OFF, um die Anzahl
der mehrdeutigen Vereinheitlichungsmöglichkeiten zu reduzieren,
die während der Konstruktion eines Beweises bereitgestellt
werden müssen.  $) 

$( Hofstadters MIU-System ist ein einfaches Beispiel für ein
formales System, das einige Konzepte von Metamath illustriert.
Siehe Douglas R. Hofstadter, _Goedel, Escher, Bach: An Eternal
Golden Braid_ (Vintage Books, New York, 1979), S. 33ff. für
eine Beschreibung des MIU-Systems.

Das System hat 3 konstante Symbole, M, I, und U. Das einzige
Axiom des Systems ist MI. Es gibt 4 Regeln:
      Regel I:  Wenn Sie eine Zeichenkette besitzen, deren
      letzter Buchstabe I ist, können Sie am Schluß ein U
      zufügen.
      Regel II: Angenommen Sie haben Mx.  Dann können Sie
      Ihrer Sammlung Mxx zufügen.
      Regel III: Wenn in einer der Zeichenketten Ihrer
      Sammlung III vorkommt, können Sie eine neue Kette mit
      U anstelle von III bilden.
      Regel IV: Wenn UU in einer Ihrer Ketten vorkommt,
      kann man es streichen.
Leider haben die Regeln III und IV keine eindeutigen
Ergebnisse: (Zeichen-)Ketten können mehr als ein Vorkommen
von III oder UU haben. Daher müssen wir das Konzept der
"wohlgeformten MIU-Formel" oder wff einführen, das es uns
ermöglicht, eindeutige Symbolfolgen zu konstruieren, auf
die die Regeln III und IV angewendet werden können. $)

$( Zuerst deklarieren wir die konstanten Symbole der Sprache.
Man beachte, dass wir zwei Symbole brauchen, um die Behauptung,
dass eine Folge eine wff ist, von der Behauptung, dass sie ein
Theorem ist, zu unterscheiden; wir haben willkürlich "wff" und
"|-" gewählt. $)
 $c M I U |- wff $. $( Konstanten deklarieren $) 

$( Als nächstes deklarieren wir einige Variablen. $)
 $v x y $. 

$( In unserer gesamten Theorie gehen wir davon aus, dass diese
Variablen wffs darstellen. $) 
 wx   $f wff x $.
 wy   $f wff y $.

$( Definition von MIU-wffs. Wir erlauben, dass die leere Folge
eine wff ist. $) 

$( Die leere Folge ist eine wff. $)
 we $a wff $.
$( "M" nach einer beliebigen wff ist eine wff. $)
 wM $a wff x M $.
$( "I" nach einer beliebigen wff ist eine wff. $)
 wI $a wff x I $.
$( "U" nach einer beliebigen wff ist eine wff. $)
 wU $a wff x U $.

$( Festlegung des Axioms. $)
 ax   $a |- M I $.

$( Festlegung der Regeln. $)
 ${
   Ia   $e |- x I $.
$( Ein beliebiger Satz, der mit "I" endet, bleibt auch dann ein
Satz, wenn ein "U" angehängt wird.  (Wir unterscheiden das Label
I_ vom mathematischen Symbol I, um der Metamath-Spezifikation
vom 24. Juni 2006 zu entsprechen.) $) 
   I_    $a |- x I U $.
 $}
 ${
IIa  $e |- M x $.
$( Jeder Satz, der mit "M" beginnt, bleibt ein Satz, wenn der
Teil nach dem "M" nochmals hinzugefügt wird. $) 
   II   $a |- M x x $.
 $}
 ${
   IIIa $e |- x I I I y $.
$( Jeder Satz mit "III" in der Mitte bleibt ein Satz, wenn das
 "III" durch "U" ersetzt wird. $) 
   III  $a |- x U y $.
 $}
 ${
   IVa  $e |- x U U y $.
$( Ein beliebiger Satz mit "UU" in der Mitte bleibt ein Satz,
wenn das "UU" gelöscht wird. $) 
   IV   $a |- x y $.
  $}

$( Nun beweisen wir das Theorem MUIIU.  Vielleicht ist es
für Sie interessant, diesen Beweis mit dem von Hofstadter
(S. 35 - 36 bzw. S. 40 in der deutschen Ausgabe) zu
vergleichen. $) 
 theorem1  $p |- M U I I U $=
      we wM wU wI we wI wU we wU wI wU we wM we wI wU we wM
      wI wI wI we wI wI we wI ax II II I_ III II IV $.
\end{verbatim}\index{wohlgeformte Formel (wff)}

Der Befehl \texttt{show proof /lemmon/renumber} erzeugt die folgende Anzeige.  Sie ist derjenigen in \cite[S.~35--36]{Hofstadter}\index{Hofstadter, Douglas R.}\footnote{Anm. der Übersetzer: S. 40 in der deutschen Ausgabe} sehr ähnlich. 

\begin{verbatim}
1 ax             $a |- M I
2 1 II           $a |- M I I
3 2 II           $a |- M I I I I
4 3 I_           $a |- M I I I I U
5 4 III          $a |- M U I U
6 5 II           $a |- M U I U U I U
7 6 IV           $a |- M U I I U
\end{verbatim}

Wir stellen fest, dass Hofstadters "`MU-Rätsel"', indem es um die Frage geht, ob MU ein Satz des MIU-Systems ist, nicht mit dem obigen System beantwortet werden kann, weil das MU-Rätsel eine Frage {\em über} das System ist.  Um die Antwort auf das MU-Rätsel zu beweisen, ist ein viel ausgefeilteres System erforderlich, nämlich eines, das das MIU-System innerhalb der Mengenlehre modelliert.  (Die Antwort auf das MU-Rätsel ist übrigens nein.) 


\chapter{Metamath-Sprache EBNF}%
\label{BNF}%
\index{Metamath-Sprache EBNF}

Dieser Anhang enthält eine formale Beschreibung der grundlegenden Syntax der Metamath-Sprache (mit komprimierten Beweisen und Unterstützung für unbekannte Beweisschritte). Sie ist definiert unter Verwendung der Erweiterten Backus--Naur-Form (EBNF)\index{erweiterte Backus--Naur-Form}\index{EBNF}, eine Notation, so wie sie in W3C\index{W3C} \textit{Extensible Markup Language (XML) 1.0 (Fifth Edition)} (W3C Recommendation 26 November 2008) unter \url{https://www.w3.org/TR/xml/#sec-notation} beschrieben und verwendet wird. 

Die Regel \texttt{database} wird bis zum Ende der Datei (\texttt{EOF}) verarbeitet. Die Regeln erfordern schließlich das Lesen von Token, die durch ein Whitespace getrennt sind. Ein Token hat eine Großbuchstaben-Definition (siehe unten) oder ist eine String-Konstante in einem Nicht-Token (wie \texttt{'\$a'}). Wir hoffen, dass dies korrekt ist, aber wenn es einen Konflikt gibt, gelten die Regeln des Abschnitts \ref{spec}. In diesem Abschnitt werden auch nicht-syntaktische Einschränkungen erörtert, die hier nicht gezeigt werden (z. B. dass jedes neue Label-Token, das in einem \texttt{hypothesis-stmt} oder \texttt{assert-stmt} definiert wird, eindeutig sein muss). 

\begin{verbatim}
database ::= outermost-scope-stmt*

outermost-scope-stmt ::=
  include-stmt | constant-stmt | stmt

/* Anweisung zum Einbinden von Dateien; behandelt eine Datei als
   (einen Teil einer) Datenbasis. Innerhalb des Dateinamens darf
   sich KEIN Kommentar befinden. */
include-stmt ::= '$[' filename '$]'

/* Deklaration von Symbolen für Konstanten. */
constant-stmt ::= '$c' constant+ '$.'

/* Eine normale Anweisung kann sich in jedem Scope befinden. */
stmt ::= block | variable-stmt | disjoint-stmt |
  hypothesis-stmt | assert-stmt

/* Ein Block, der auch leer sein kann. */
block ::= '${' stmt* '$}'

/* Deklaration von Symbolen für Variablen. */
variable-stmt ::= '$v' variable+ '$.'

/* Disjunkte Variablen. Einfache disjunkte Variableneinschränkung
   bestehen aus 2 Variables, d.h. "variable*" ist in diesem Fall 
   leer. */
disjoint-stmt ::= '$d' variable variable variable* '$.'

hypothesis-stmt ::= floating-stmt | essential-stmt

/* Fließende (Variablentyp-)Hypothese. */
floating-stmt ::= LABEL '$f' typecode variable '$.'

/* Essenzielle (logische) Hypothese. */
essential-stmt ::= LABEL '$e' typecode MATH-SYMBOL* '$.'

assert-stmt ::= axiom-stmt | provable-stmt

/* Axiomatische Behauptung. */
axiom-stmt ::= LABEL '$a' typecode MATH-SYMBOL* '$.'

/* Beweisbare Behauptung. */
provable-stmt ::= LABEL '$p' typecode MATH-SYMBOL*
  '$=' proof '$.'

/* Ein Beweis. Innerhalb von Beweisen können sich auch
   Kommentare befinden. Wenn ein '?' in dem Beweis 
   enthalten ist, dann handelt es sich um einen
   "unvollständigen" Beweis. */
proof ::= uncompressed-proof | compressed-proof
uncompressed-proof ::= (LABEL | '?')+
compressed-proof ::= '(' LABEL* ')' COMPRESSED-PROOF-BLOCK+

typecode ::= constant

filename ::= MATH-SYMBOL /* enthält kein Whitespace oder '$' */
constant ::= MATH-SYMBOL
variable ::= MATH-SYMBOL
\end{verbatim}

\needspace{2\baselineskip}
Ein \texttt{frame} ist eine Folge von keinem, einem oder mehreren \texttt{disjoint-{\allowbreak}stmt}- und \texttt{hypothe\-ses-{\allowbreak}stmt}-Anweisungen (möglicherweise verschachtelt mit anderen Anweisungen, die keine \texttt{assert-stmt}-Anweisungen sind), gefolgt von einem \texttt{assert-stmt}.

\needspace{3\baselineskip}
Hier sind die Regeln für die lexikalische Verarbeitung (Tokenisierung) über die oben gezeigten konstanten Token hinaus. Konventionell werden diese Tokenisierungsregeln in Großbuchstaben geschrieben. Jedes Token wird so lang wie möglich gelesen. Durch ein Whitespace getrennte Token werden nacheinander gelesen; beachten Sie, dass der trennende Whitespace und die Kommentare \texttt{\$(} ... \texttt{\$)} übersprungen werden. 

Wenn eine Token-Definition eine andere Token-Definition verwendet, wird das Ganze als ein einziges Token betrachtet. Ein Muster, das nur Teil eines vollständigen Tokens ist, hat einen Namen, der mit einem Unterstrich ("`\_"') beginnt. Eine Implementierung könnte viele Token als \texttt{PRINTABLE-SEQUENCE} tokenisieren und dann prüfen, ob sie die hier dargestellte spezifischere Regel erfüllen. 

Kommentare lassen sich nicht verschachteln, und sowohl \texttt{\$(} als auch \texttt{\$)} müssen von mindestens einem Whitespace-Zeichen (\texttt{\_WHITECHAR}) umgeben sein. Technisch gesehen enden Kommentare vor dem abschließenden \texttt{\_WHITECHAR}, aber das abschließende \texttt{\_WHITECHAR} wird sowieso ignoriert, so dass wir dieses Detail hier ignorieren. Metamath-Sprachprozessoren müssen kein \texttt{\$)} unterstützen, das unmittelbar vor dem Dateiende steht, da auf das abschließende Kommentarsymbol ein \texttt{\_WHITECHAR}, wie ein Zeilenumbruch, folgen muss. 

\begin{verbatim}
PRINTABLE-SEQUENCE ::= _PRINTABLE-CHARACTER+

MATH-SYMBOL ::= (_PRINTABLE-CHARACTER - '$')+

/* druckbare ASCII-Zeichen ohne Whitespace-Zeichen */
_PRINTABLE-CHARACTER ::= [#x21-#x7e]

LABEL ::= ( _LETTER-OR-DIGIT | '.' | '-' | '_' )+

_LETTER-OR-DIGIT ::= [A-Za-z0-9]

COMPRESSED-PROOF-BLOCK ::= ([A-Z] | '?')+

/* Definition von Whitespace zwischen Token. Das -> SKIP
   bedeutet, dass ein auftretendes Whitespace übersprungen
   und mit dem Lesen weiterer Zeichen fortgefahren wird. */
WHITESPACE ::= (_WHITECHAR+ | _COMMENT) -> SKIP

/* Kommentare. $( ... $), die nicht verschachtelt sind. */
_COMMENT ::= '$(' (_WHITECHAR+ (PRINTABLE-SEQUENCE - '$)'))*
  _WHITECHAR+ '$)' _WHITECHAR

/* Whitespace: (' ' | '\t' | '\r' | '\n' | '\f') */
_WHITECHAR ::= [#x20#x09#x0d#x0a#x0c]
\end{verbatim}
% This EBNF was developed as a collaboration between
% David A. Wheeler\index{Wheeler, David A.},
% Mario Carneiro\index{Carneiro, Mario}, and
% Benoit Jubin\index{Jubin, Benoit}, inspired by a request
% (and a lot of initial work) by Benoit Jubin.
%

\chapter{Metamath 100}%
\label{Metamath100}%

Die folgende Tabelle enthält alle Theoreme der Liste 
"`Formalisierung von 100 Theoremen"' ("`Formalizing 100 Theorems"', siehe \url{http://www.cs.ru.nl/\%7Efreek/100/}) von Freek Wiedijk, die bereits mit Metamath bewiesen wurden (Stand 12.11.2023). Die Nr. entspricht der Nummer in der Liste von Freek Wiedijk, die Bezeichnung wurde ins Deutsche übersetzt\footnote{Anm. der Übersetzer: Dieser Anhang ist im Original nicht vorhanden}

\begin{longtabu}   to \textwidth {
	X[1,c]
	X[10,l]
	X[5,l]
	X[5,l]
	X[5,l]}
\textbf{Nr.} & \textbf{Bezeichnung des Theorems} & \textbf{Label in set.mm} & \textbf{Autor} & \textbf{Datum}\\
& & & & \\
\endhead
1.  & Die Quadratwurzel aus 2 ist irrational        & sqrt2irr     & Norman Megill     & 2001-08-20 \\
2.  & Der Fundamentalsatz der Algebra               & fta          & Mario Carneiro    & 2014-09-15 \\
3.  & Die Abzählbarkeit der rationalen Zahlen       & qnnen        & Norman Megill     & 2004-07-31 \\
4.  & Der Satz von Pythagoras                       & pythi        & Norman Megill     & 2008-04-17 \\
5.  & Der Primzahlensatz                            & pnt          & Mario Carneiro    & 2016-06-01 \\
7.  & Das quadratische Reziprozitätsgesetz          & lgsquad      & Mario Carneiro    & 2015-06-19 \\
9.  & Die Kreisfläche                               & areacirc     & Brendan Leahy     & 2017-08-31 \\
10. & Der Satz von Euler (Verallgemeinerung  
      des kleinen Fermatschen Satzes)               & eulerth      & Mario Carneiro    & 2014-02-28 \\
11. & Der Satz von Euklid
      (Existenz unendlich vieler Primzahlen)        & infpn2       & Norman Megill     & 2005-05-05 \\
14. & Das Baseler Problem 
      (Summe der reziproken Quadratzahlen)          & basel        & Mario Carneiro    & 2014-07-30 \\
15. & Der Fundamentalsatz der Analysis              & ftc1, ftc2   & Mario Carneiro    & 2014-09-03 \\
17. & Der Moivresche Satz                           & demoivreALT  & Steve Rodriguez   & 2006-11-10 \\
18. & Der Approximationssatz von Liouville          & aaliou       & Stefan O'Rear     & 2014-11-22 \\
19. & Der Vier-Quadrate-Satz von Lagrange           & 4sq          & Mario Carneiro    & 2014-07-16 \\
20. & Der Zwei-Quadrate-Satz von Fermat             & 2sq          & Mario Carneiro    & 2015-06-20 \\
22. & Die Überabzählbarkeit des Kontinuums
      (der reelen Zahlen)                           & ruc          & Norman Megill     & 2004-08-13 \\
23. & Formel zur Bildung der pythagoreischen Tripel & pythagtrip   & Scott Fenton      & 2014-04-19 \\
25. & Der Cantor--Bernstein--Schrödersche
      Äquivalenzssatz                               & sbth         & Norman Megill     & 1998-06-08 \\
26. & Die Leibniz-Reihe für Pi                      & leibpi       & Mario Carneiro    & 2015-04-16 \\
27. & Der Innenwinkelsatz für Dreiecke              & ang180       & Mario Carneiro    & 2014-09-24 \\
30. & Bertrand's Ballot Problem                     & ballotth     & Thierry Arnoux    & 2016-12-07 \\
31. & Der Satz von Ramsey (Kombinatorik)            & ramsey       & Mario Carneiro    & 2015-04-23 \\
34. & Die Divergenz der harmonischen Reihe          & harmonic     & Mario Carneiro    & 2014-07-11 \\
35. & Der Satz von Taylor                           & taylth       & Mario Carneiro    & 2017-01-01 \\
37. & Die Lösungsformel für kubische Gleichungen    & cubic        & Mario Carneiro    & 2015-04-26 \\
38. & Die Ungleichung vom arithmetischen
      und geometrischen Mittel                      & amgm         & Mario Carneiro    & 2015-06-21 \\
39. & Lösungen der Pellschen Gleichung              & rmxycomplete & Stefan O'Rear     & 2014-11-22 \\
42. & Die Summe der Kehrwerte aller Dreieckszahlen  & trirecip     & Scott Fenton      & 2014-05-02 \\
44. & Der binomische Lehrsatz                       & binom        & Norman Megill     & 2005-12-07 \\
45. & Der Partitionssatz von Euler
      (als Spezialfall des Satzes von Glaisher)     & eulerpart    & Thierry Arnoux    & 2018-08-30 \\
46. & Die Lösungsformel für allgemeine quartische
      Gleichungen                                   & quart        & Mario Carneiro    & 2015-05-06 \\
48. & Der Dirichletsche Primzahlsatz                & dirith       & Mario Carneiro    & 2016-05-12 \\
49. & Der Satz von Cayley--Hamilton                  & cayleyhamilton & Alexander van der Vekens
                                                                                       & 2019-11-25 \\
51. & Der Satz von Wilson                           & wilth        & Mario Carneiro    & 2015-01-28 \\
52. & Die Anzahl der Teilmengen von Mengen          & pw2en        & Norman Megill     & 2004-01-29 \\
54. & Das Königsberger Brückenproblem               & konigsberg   & Mario Carneiro    & 2015-04-16 \\
55. & Der Sehnensatz                                & chordthm     & David Moews       & 2017-02-28 \\
57. & Der Satz von Heron                            & heron        & Mario Carneiro
                                                         (Jon Pennant, Thierry Arnoux) & 2019-03-10 \\
58. & Die Formel für die Anzahl von Kombinationen   & hashbc       & Mario Carneiro    & 2014-07-13 \\
60. & Das Lemma von Bézout                          & bezout       & Mario Carneiro    & 2014-02-22 \\
61. & Der Satz von Ceva                             & cevath       & Saveliy Skresanov & 2017-09-24 \\
63. & Der Satz von Cantor                           & canth2       & Norman Megill     & 1994-08-07 \\
64. & Die Regel von de L’Hospital                   & lhop         & Mario Carneiro    & 2016-12-30 \\
65. & Der Basiswinkelsatz in gleichschenkligen
      Dreiecken                                     & isosctr      & Saveliy Skresanov & 2017-01-01 \\
66. & Die (endlichen) Partialsummen einer
      geometrischen Reihe                           & geoser       & Norman Megill     & 2006-05-09 \\
67. & Die Eulersche Zahl e ist transzendent         & etransc      & Glauco Siliprandi & 2020-04-05 \\
68. & Die Gaußsche Summenformel                     & arisum       & Frédéric Liné     & 2006-11-16 \\
69. & Der Euklidische Algorithmus                   & eucalg       & Paul Chapman      & 2011-03-31 \\
70. & Der Euklid--Euler-Satz über vollkommene Zahlen & perfect      & Mario Carneiro    & 2016-05-17 \\
71. & Der Satz von Lagrange                         & lagsubg, lagsubg2 
                                                                   & Mario Carneiro    & 2014-07-11 \\ 
72. & Die Sylow-Sätze                               & sylow1, sylow2, sylow2b, sylow3
                                                                   & Mario Carneiro    & 2015-01-19 \\
73. & Der Satz von Erdős und Szekeres               & erdsze, erdsze2 & Mario Carneiro & 2015-01-28 \\
74. & Das Prinzip der vollständigen Induktion       & finds        & Norman Megill     & 1995-04-14 \\
75. & Der Mittelwertsatz der Differentialrechnung   & mvth         & Mario Carneiro    & 2014-09-14 \\
76. & Fourierreihen                                 & fourier      & Glauco Siliprandi & 2019-12-11 \\
77. & Die Faulhabersche Formel für Potenzsummen     & fsumkthpow   & Scott Fenton      & 2014-05-16 \\
78. & Die Cauchy--Schwarz-Ungleichung                & sii          & Norman Megill     & 2008-01-12 \\
79. & Der Zwischenwertsatz (der reellen Analysis)   & ivth         & Paul Chapman      & 2008-01-22 \\
80. & Der Fundamentalsatz der Arithmetik            & 1arith2      & Paul Chapman      & 2012-11-17 \\
81. & Erdős Beweis der Divergenz der Reihe der
      Kehrwerte der Primzahlen (Satz von Euler)     & prmrec       & Mario Carneiro    & 2014-08-10 \\
83. & Der Freundschaftssatz                         & friendship   & Alexander van der Vekens
                                                                                       & 2018-10-09 \\
85. & Quersummenregel für die Teilbarkeit durch 3   & 3dvds        & Mario Carneiro    & 2014-07-14 \\
86. & Lebesgue-Maß und -Integral                    & itgcl        & Mario Carneiro    & 2014-06-29 \\
87. & Der Satz von Desargues                        & dath         & Norman Megill     & 2012-08-20 \\
88. & Die Anzahl der fixpunktfreien Permutationen   & derangfmla, subfaclim 
                                                                   & Mario Carneiro    & 2015-01-28 \\
89. & Der Restpolynom-Satz                          & facth, plyrem & Mario Carneiro   & 2014-07-26 \\
90. & Die Stirling-Formel                           & stirling     & Glauco Siliprandi & 2017-06-29 \\
91. & Die Dreiecksungleichung                       & abstrii      & Norman Megill     & 1999-10-02 \\
93. & Das Geburtstagsproblem                        & birthday     & Mario Carneiro    & 2015-04-17 \\
94. & Der Kosinussatz                               & lawcos       & David A. Wheeler  & 2015-06-12 \\
95. & Der Satz des Ptolemäus                        & ptolemy      & David A. Wheeler  & 2015-05-31 \\
96. & Das Prinzip von Inklusion und Exklusion       & incexc       & Mario Carneiro    & 2017-08-07 \\
97. & Die Cramersche Regel                          & cramer       & Alexander van der Vekens
                                                                       (Stefan O'Rear) & 2019-02-21 \\
98. & Das Bertrandsche Postulat                     & bpos         & Mario Carneiro    & 2014-03-15 \\
\end{longtabu}
\pagebreak 

\chapter{Glossar}%
\section{Deutsch - Englisch}

\begin{longtabu}   { @{} X[l] X[l] }
\textbf{Deutsch} & \textbf{Englisch}\\
 & \\
\endhead
    Abschluss & closure \\
    abstrakte Algebra & abstract algebra \\
    Abstraktionsklasse & abstraction class \\
    Addition & addition \\
    aktive Anweisung & active statement \\
    aktives mathematisches Symbol & active math symbol \\
    Allquantor & universal quantifier \\
    Analysis & analysis \\
    Anweisung & statement   \\
    äußerster Block & outermost block \\
    Ausdruck & expression \\
    Aussage & statement \\
    Aussagenlogik & propositional calculus \\
    Aussonderungssaxiom & Axiom of Separation \\
    Auswahlaxiom & Axiom of Choice \\
    Auszeichnungsnotation & markup notation \\
    automatisches Theorembeweisen & automated theorem proving \\
    automatisierte Beweisverifizierung & automated proof verification \\
    Axiom & axiom \\
    axiomatische Aussagen & axiomatic statement \\
    axiomatische Behauptung & axiomatic assertion \\
    Axiome der Aussagenlogik & axioms of propositional calculus \\
    Axiome der Logik & axioms of logic \\
    Axiome der Mengenlehre & axioms of set theory \\
    Axiome der Prädikatenlogik & axioms of predicate calculus \\
    Axiome für die Gleichheit & axioms for equality  \\
    Axiome für die Mathematik & axioms for mathematics \\
    Axiomenschema & axiom scheme \\
    Axiom von Pasch & Pasch's axiom \\
     & \\
    Baumdarstellung eines Beweises & tree-style proof \\
    Befehlsschlüsselwort & command keyword \\
    Befehlszeilenparameter & command qualifier \\
    Befehlszeilenschnittstelle (CLI) & command line interface \\
    Behauptung & assertion \\
    Behauptungslabel & assertion label \\
    Betriebssystem-Befehl & operating system command \\
    Beweis & proof \\
    Beweis-Assistent & Proof Assistant \\
    beweisbare Aussage & provable statement \\
    beweisbare Behauptung & provable assertion \\
    Beweislänge & proof length \\
    Beweisschema & proof scheme \\
    Beweisschritt & proof step \\
    Beweistheorie & proof theory \\
    bijektive Funktion & one-to-one, onto function \\
    Bikonditional & biconditional \\
    Bild & image \\
    binäre Relation & binary relation \\
    Block & block \\
    Boolesche Algebra & Boolean algebra \\
    Bug & software bug \\
    Burali-{\allowbreak}Forti Paradoxon & Burali-{\allowbreak}Forti paradox \\
     & \\
    Clifford-{\allowbreak}Algebren & Clifford algebras \\
    Computeralgebrasystem & computer algebra system \\
    Courier Schriftart & Courier font \\
     & \\
    Dateieinbindung & file inclusion \\
    Dateiname & file name \\
    Datenbasis & database \\
    Deduktionsform & deduction form \\
    Deduktionsstil & deduction style \\
    Deduktionstheorem & deduction theorem \\
    Definiendum & definiendum \\
    Definiens & definiens \\
    Definition & definition \\
    Definitionsbereich & domain \\
    Deklaration & declaration \\
    disjunkte Mengen & disjoint sets \\
    disjunkte Variablen & disjoint variables \\
    disjunkte Variableneinschränkung & disjoint-{\allowbreak}variable restriction \\
    Disjunktion & disjunction \\
    Drehkreuz & turnstile \\
    druckbares Zeichen & printable character \\
    Drucker & printer \\
    Dummy-Variable & dummy variable \\
     & \\
    echte Klasse & proper class \\
    echte Substitution & proper substitution \\
    effektiv gebundene Variable & effectively bound variable \\
    effektiv nicht frei & effectively not free \\
    einelementige Menge & singleton \\
    Eindeutigkeitsquantor & existential uniqueness quantifier \\
    einfache Deklaration & simple declaration \\
    einfache unendliche Folge & simple infinite sequence \\
    einfacher Text & plain text \\
    einfaches Anführungszeichen & grave accent \\
    einfaches Metatheorem & simple metatheorem \\
    eingeschränkter Allquantor & restricted universal quantifier \\
    eingeschränkter Existenzquantor & restricted existential quantifier \\
    eingebundene Datei & included file \\
    Einschränkung & restriction \\
    Element & element \\
    endliche n-gliedrige Folge & finite n-termed sequence \\
    entfernen & pop \\
    entscheidbare Theorie & decidable theory \\
    Entscheidungsverfahren & decision procedure \\
    Epsilon-Relation & epsilon relation \\
    Ersetzung & substitution \\
    Ersetzungsaxiom & Axiom of Replacement \\
    erweiterte Backus--Naur-Form & Extended Backus--Naur Form \\
    erweiterte Sprache & extended language \\
    erweiterter Frame & extended frame \\
    essentielle Hypothese & essential hypothesis \\
    euklidische Geometrie & Euclidean geometry \\
    Existenzquantor & existential quantifier \\
    Extensionalitätsaxiom & Axiom of Extensionality \\
     & \\
    Familie & family \\
    Fehler in Beweisen & errors in proofs \\
    Fehlerprüfung & error checking \\
    finite Induktion & finite induction \\
    finitistischer Beweis & finitary proof \\
    fließende Hypothese & floating hypothesis \\
    formale Logik & formal logic \\
    formaler Beweis & formal proof \\
    formales System & formal system \\
    Formalismus & formalism \\
    Formen & forms \\
    Frame & frame \\
    freie Logik & free logic \\
    freie Variable & free variable \\
    fundierte Relation & founded relation \\
    Fundierungsaxiom & Axiom of Regularity \\
    Funktion & function \\
    Funktionswert & function value \\
     & \\
    ganze Zahl & integer \\
    gebundene Variable & bound variable \\
    genau dann, wenn & iff \\
    geordnetes Paar & ordered pair \\
    geschlossene Form & closed form \\
    Gleichheit & equality \\
    Glied & member \\
    globale Anweisung & global statement \\
    Gödelscher Unvollständigkeitssatz & Gödel's incompleteness theorem \\
    Grenzzahl & limit ordinal \\
    Großer Fermatscher Satz & Fermat's Last Therorem \\
    Grundlagen der Mathematik & foundations of mathematics \\
    grundlegende Sprache & basic language \\
    grundlegendes Schlüsselwort & basic keyword \\
    Gruppentheorie & group theory \\
    Gültigkeitsbereich & scope \\
    Gültigkeitsbereichsanweisung & scoping statement \\
     & \\
    Hierarchie & hierarchy \\
    Hilfsschlüsselwort & auxiliary keyword \\
    Hypothese & hypothesis \\
    Hypothesenlabel & hypothesis label \\
    Hypothesenzuordnung & hypothesis association \\
     & \\
    Implikation & implication \\
    implizites Axiom & implicit axiom \\
    implizite Substitution & implicit substitution \\
    individuelle Metavariable & individual metavariable \\
    individuelle Variable & individual variable \\
    Inferenz & inference \\
    Inferenzform & inference form \\
    Inferenzregel & inference rule \\
    Infix-Konnektor & inflix connective \\
    informeller Beweis & informal proof \\
    injektive Funktion & one-to-one function \\
    Intuitionismus & intuitionism \\
     & \\
    Junktor & connective \\
     & \\       
    Kardinalität & cardinality \\
    Kardinalzahl & cardinal \\
    Kartesisches Produkt & Cartesian product \\
    Kategorientheorie & category theory \\
    Klammerschreibweise & brace notation \\
    Klasse & class \\
    Klassenabstraktion & class abstraction \\
    Klassendifferenz & class difference \\
    Klassengleichheit & class equality \\
    Klassenvariable & class variable \\
    Klassenzugehörigkeit & class membership \\
    Kommentar & comment \\
    \hangindent=0.5cm Kommentar mit zusätzlichen Informationen\vspace{3pt} & additional information comment \\
    kompakter Beweis & compact proof \\
    komplexe Zahl & complex number \\
    Komposition & composition \\
    komprimierter Beweis & compressed proof \\
    kondensierte Ablösung & condensed detachment \\
    Konjunktion & conjunction \\
    Konstante & constant \\
    Konstantendeklaration & constant declaration \\
    Konstante-Variable-Paar & constant-variable pair \\
    konstant-gepräfixter Ausdruck & constant-prefixed expression \\
    konstruktive Sprache & constructive language \\
    Konstruktivismus & constructivism \\
    kontinuierliche Integration & continuous integration \\
    Kontinuumshypothese & continuum hypothesis \\
    kreative Definintion & creative definition \\
    Kreuzprodukt & cross product \\
    künstliche Intelligenz & artificial intelligence \\
     & \\
    Label & label \\
    Label-Deklaration & label declaration \\
    Label-Modus & label mode \\
    Label-Referenz & label reference \\
    Label-Sequenz & label sequence \\
    leere Menge & empty set, null set \\
    leerer Definitionsbereich & empty domain \\
    leere Substitution & empty substitution \\
    Leermengenaxiom & Axiom of the Null Set \\
    Leerraum & white space \\
    Lemmon-Stil Beweis & Lemmon-style proof \\
    Logik & logic \\
    Logik erster Ordnung & first-order logic (FOL) \\
    Logik höherer Ordnung & higher-order logic (HOL) \\
    logische Äquivalenz & logical equivalenz \\
    logische Hypothese & logical hypothesis \\
    logisches ODER & logical OR \\
    logisches UND & logical AND \\
    lokale Variable & local variable \\
    lokales Label & local label \\
    Lücken in Beweisen & gaps in proofs \\
     & \\
    Macintosh-Dateiname & Macintosh file name \\
    maschinelles Lernen & machine learning \\
    mathematisches Symbol & math symbol \\
    Mathe-Modus & math mode \\
    mehrdeutige Vereinheitlichung & ambiguous unification \\
    Menge & set \\
    Mengendifferenz & set difference \\
    Mengenlehre & set theory \\
    Metalogik & metalogic \\
    metalogische Vollständigkeit & metalogical completeness \\
    Metamath & Metamath \\
    Metamath Proof Explorer & Metamath Proof Explorer \\
    Metamathematik & metamathematics \\
    Metamath-Sprache EBNF & Metamath Language EBNF \\
    Metasprache & metalanguage \\
    Metatheorem & metatheorem \\
    Metavariable & metavariable \\
    Mitglied & member \\
    MIU-System & MIU-system \\
    modale Logik & modal logic \\
    Modelltheorie & model theory \\
    Modus ponens & modus ponens \\
    Monaco Schriftart & Monaco font \\
     & \\
    Nachfolger & successor \\
    natürliche Deduktion & natural deduction \\
    natürliche Zahl & natural number \\
    Negation & negation \\
    nichtproportionale Schriftart & monospaced font \\
    nicht-triviale Theorie & non-trivial theory \\
    normaler Beweis & normal proof \\
     & \\
    Objekt & object \\
    Objektsprache & object language \\
    obligatorische \texttt{\$d}-Anweisung & mandatory \texttt{\$d} statement \\
    \hangindent=0.5cm obligatorische disjunkte Variableneinschränkung & \hangindent=0.5cm mandatory distinct-/disjoint-{\allowbreak}variable restriction \\
    obligatorische Hypothese & mandatory hypothesis \\
    obligatorische Variable & mandatory variable \\
    \hangindent=0.5cm obligatorische Variablentyp-{\allowbreak}Hypothese\vspace{2pt} & mandatory variable-{\allowbreak}type hypothesis \\
    Omega & omega \\
    Operation & operation \\
    \hangindent=0.5cm optionale disjunkte Variableneinschränkung\vspace{2pt} & optional disjoint-{\allowbreak}variable restriction \\
    optionale Hypothese & optional hypothesis \\
    optionale Variable & optional variable \\
    ordinale Addition & ordinal addition \\
    Ordinalprädikat & ordinal predicate \\
    Ordinalzahl & ordinal number \\
     & \\
    Paar & pair \\
    Paarmengenaxiom & Axiom of Pairing \\
    Parser & parser \\
    Peano-Postulate & Peano's postulates \\
    Pierces Axiom & Pierce's axiom \\
    Poincaré-Vermutung & Poincaré conjecture \\
    polnische Notation & Polish notation \\
    Postfix-Konnektor & postfix connective \\
    Potenzklasse & power class \\
    Potenzmenge & power set \\
    Potenzmengenaxiom & Axiom of Power Sets  \\
    Prä-Aussage & pre-statement \\
    Prädikatenlogik & predicate calculus \\
    Präfix-Konnektor & prefix connective \\
    Principia Mathematica & Principia Mathematica \\
    Programmfehler & software bug \\
     & \\
    qualifizierender Ausdruck & qualifying expression \\
    Quantenlogik & quantum logic \\
    Quantenmechanik & quantum mechanics \\
    Quantifizierungstheorie & quantifier theory \\
    Quelldatei & source file \\
    Quellpuffer & source buffer \\
    Querverweis zu Literaturangaben & bibliographical references \\
     & \\
    rationale Zahl & rational number \\
    Redukt & reduct \\
    reelle Zahl & real number \\
    Reflexionsprinzip & reflection principle \\
    Regel & rule \\
    Regel der Verallgemeinerung & rule of generalization \\
    reine Mathematik & pure mathematics \\
    Rekursionsoperator & recursion operator \\
    rekursive Definition & recursive definition \\
    Relation & relation \\
    Robbins-Algebra & Robbins algebra \\
    Robinsons Resolutionsprinzip & Robinson's resolution principle \\
    RPN-Reihenfolge & RPN order \\
    RPN-Stapel & RPN stack \\
    Russells Paradoxon & Russell's paradox \\
     & \\
    Sammlung & collection \\
    Satz von Cantor & Cantor's theorem \\
    Satzlogik & sentential logic \\
    schieben & push \\
    Schlüsselwort & keyword \\
    Schlussfolgerung  & conclusion \\
    Schnittmenge & intersection \\
    Schriftsatzanweisung & typesetting comment \\
    Schröder--Bernstein-Theorem & Schröder--Bernstein theorem \\
    schwache Logik & weak logic \\
    Sonderzeichen & special characters \\
    Spinner & crank \\
    Standard-{\allowbreak}Deduktionstheorem & Standard Deduction Theorem \\
    Stapel & stack \\
    stilisiertes Epsilon & stylized epsilon \\
    Substitution & substitution \\
    Substitutionsabbildung & substitution map \\
    Substitutionstheorem & substitution theorem \\
    surjektive Funktion & onto function \\
    Syllogismus & Syllogism \\
    Symbol & symbol \\
    Syntax-Regeln & syntax rules \\
     & \\
    Tautologie & tautology \\
    Teilmenge & subset \\
    temporäre Variable & temporary variable \\
    Term & term \\
    Texteditor & text editor \\
    Textverarbeitung & word processing \\
    Theorem & theorem \\
    Theorem der schwachen Deduktion & Weak Deduction Theorem \\
    Theoremschema & theorem scheme \\
    Tilde & tilde \\
    Token & token \\
    Topologie & topology \\
    transfinite Kardinalzahl & transfinite cardinal \\
    transfinite Rekursion & transfinite recoursion \\
    transitive Klasse & transitive class \\
    transitive Menge & transitive set \\
    Typ & type \\
    Typcode & typecode \\
     & \\
    Umdeklarierung von Symbolen & redeclaration of symbols \\
    umgekehrte polnische Notation & reverse polish notation (RPN) \\
    unendliche Menge & infinite set \\
    Unendlichkeit & infinity \\
    Unendlichkeitsaxiom & Axiom of Infinity  \\
    ungeordnetes Paar & unordered pair \\
    ungeordnetes Tripel & unordered tripel \\
    universelle Klasse & universal class \\
    Universum eines formalen Systems & universe of a formal system \\
    Unix-Dateiname & Unix file name \\
    Unterklasse & subclass \\
    unterschiedliche Variablen & distinct variables \\
    unzugängliche Kardinalzahl & inaccessible cardinal \\
     & \\
    Variable & variable \\
    Variablendeklaration & variable declaration \\
    Variablensubstitution & variable substitution \\
    Variablentyp & variable type \\
    Variablentyp-Hypothese & variable-type hypothesis \\
    Venn-Diagramm & Venn diagram \\
    Vereinheitlichung & unification \\
    Vereinigung & union \\
    Vereinigungsaxiom & Axiom of Union \\
    Vereinigungsmenge & set union \\
    Verkettung & concatenation \\
    verschachtelter Block & nested block \\
    Verschachtelungstiefe & nesting level \\
    verschiedene Variablen & distinct variables \\
    Vier-Farben-Satz & four-color theorem \\
    vollständige Induktion & mathematical induction \\
    \hangindent=0.5cm Vollständigkeitssatz der Aussagenlogik\vspace{2pt} & completeness theorem of propositional calculus \\
    Vorrangigkeit eines Operators & operator precedence \\
     & \\
    Wahrheitstabelle & truth table \\
    Wertebereich & range \\
    Whitespace & white space \\
    widerspruchsfreie Theorie & consistent theory \\
    wohlgeformte Formel (wff) & well-formed formula \\
    Wohlordnung & well-ordering \\
     & \\
    Zahlentheorie & number theory \\
    Zermelo--Fraenkel-{\allowbreak}Mengenlehre & Zermelo--Fraenkel set theory \\
    ZFC-Mengenlehre & ZFC set theory \\
    Zitat & citation \\
    zulässige Definition & proper definition \\
    Zuordnung & mapping \\
    zusammengesetzte Deklaration & compound declaration \\ 
  \end{longtabu}
  \pagebreak


\section{Englisch - Deutsch}

\begin{longtabu}   { @{} X[l] X[l] }
\textbf{Englisch} & \textbf{Deutsch}\\
& \\
\endhead
    abstract algebra & abstrakte Algebra \\
    abstraction class & Abstraktionsklasse \\
    active math symbol & aktives mathematisches Symbol \\
    active statement & aktive Anweisung \\
    addition & Addition \\
    additional information comment & \hangindent=0.5cm Kommentar mit zusätzlichen Informationen\vspace{3pt} \\
    ambiguous unification & mehrdeutige Vereinheitlichung \\
    artificial intelligence & künstliche Intelligenz \\
    analysis & Analysis \\
    assertion & Behauptung \\
    assertion label & Behauptungslabel \\
    automated proof verification & automatisierte Beweisverifizierung \\
    automated theorem proving & automatisches Theorembeweisen \\
    auxiliary keyword & Hilfsschlüsselwort \\
    Axiom & axiom \\
    Axiom of Choice & Auswahlaxiom \\
    Axiom of Extensionality & Extensionalitätsaxiom \\
    Axiom of Infinity & Unendlichkeitsaxiom \\
    Axiom of Pairing & Paarmengenaxiom \\
    Axiom of Power Sets & Potenzmengenaxiom \\
    Axiom of Regularity & Fundierungsaxiom \\
    Axiom of Replacement & Ersetzungsaxiom \\
    Axiom of Separation & Aussonderungssaxiom \\
    Axiom of the Null Set & Leermengenaxiom \\
    Axiom of Union & Vereinigungsaxiom \\
    axiom scheme & Axiomenschema \\
    axiomatic assertion & axiomatische Behauptung \\
    axiomatic statement & axiomatische Aussagen \\
    axioms for equality & Axiome für die Gleichheit \\
    axioms for mathematics & Axiome für die Mathematik \\
    axioms of logic & Axiome der Logik \\
    axioms of predicate calculus & Axiome der Prädikatenlogik \\
    axioms of propositional calculus & Axiome der Aussagenlogik \\
    axioms of set theory & Axiome der Mengenlehre \\
     & \\
    basic keyword & grundlegendes Schlüsselwort \\
    basic language & grundlegende Sprache \\
    bibliographical references & Querverweis zu Literaturangaben \\
    biconditional & Bikonditional \\
    binary relation & binäre Relation \\
    block & Block \\
    Boolean algebra & Boolesche Algebra \\
    bound variable & gebundene Variable \\
    brace notation & Klammerschreibweise \\
    Burali-Forti paradox & Burali-Forti Paradoxon \\
     & \\
    Cantor's theorem & Satz von Cantor \\
    cardinal & Kardinalzahl \\
    cardinality & Kardinalität \\
    Cartesian product & Kartesisches Produkt \\
    category theory & Kategorientheorie \\
    citation & Zitat \\
    class & Klasse \\
    class abstraction & Klassenabstraktion \\
    class difference & Klassendifferenz \\
    class equality & Klassengleichheit \\
    class membership & Klassenzugehörigkeit \\
    class variable & Klassenvariable \\
    Clifford algebras & Clifford-Algebren \\
    closed form & geschlossene Form \\
    closure & Abschluss \\
    collection & Sammlung \\
    command keyword & Befehlsschlüsselwort \\
    command line interface & Befehlszeilenschnittstelle (CLI) \\
    command qualifier & Befehlszeilenparameter \\
    comment & Kommentar \\
    compact proof & kompakter Beweis \\
    completeness theorem of propositional calculus & Vollständigkeitssatz der Aussagenlogik \\
    complex number & komplexe Zahl \\
    composition & Komposition \\
    compound declaration & zusammengesetzte Deklaration \\
    compressed proof & komprimierter Beweis \\
    computer algebra system & Computeralgebrasystem \\
    concatenation & Verkettung \\
    conclusion & Schlussfolgerung \\
    condensed detachment & kondensierte Ablösung \\
    conjunction & Konjunktion \\
    connective & Junktor \\
    consistent theory & widerspruchsfreie Theorie \\
    constant & Konstante \\
    constant declaration & Konstantendeklaration \\
    constant-prefixed expression & konstant-gepräfixter Ausdruck \\
    constant-variable pair & Konstante-Variable-{\allowbreak}Paar \\
    constructive language & konstruktive Sprache \\
    constructivism & Konstruktivismus \\
    continuous integration & kontinuierliche Integration \\
    continuum hypothesis & Kontinuumshypothese \\
    Courier font & Courier Schriftart \\
    crank & Spinner \\
    creative definition & kreative Definintion \\
    cross product & Kreuzprodukt \\
     & \\
    database & Datenbasis \\
    decidable theory & entscheidbare Theorie \\
    decision procedure & Entscheidungsverfahren \\
    declaration & Deklaration \\
    deduction form & Deduktionsform \\
    deduction style & Deduktionsstil \\
    deduction theorem & Deduktionstheorem \\
    definiendum & Definiendum \\
    definiens & Definiens \\
    definition & Definition \\
    disjoint sets & disjunkte Mengen \\
    disjoint variables & disjunkte Variablen \\
    disjoint-variable restriction & disjunkte Variableneinschränkung \\
    disjunction & Disjunktion \\
    distinct variables & \hangindent=0.5cm unterschiedliche/verschiedene Variablen\vspace{3pt} \\
    domain & Definitionsbereich \\
    dummy variable & Dummy-Variable \\
     & \\
    effectively bound variable & effektiv gebundene Variable \\
    effectively not free & effektiv nicht frei \\
    element & Element \\
    empty domain & leerer Definitionsbereich \\
    empty set & leere Menge \\
    empty substitution & leere Substitution \\
    epsilon relation & Epsilon-Relation \\
    equality & Gleichheit \\
    error checking & Fehlerprüfung \\
    errors in proofs & Fehler in Beweisen \\
    essential hypothesis & essentielle Hypothese \\
    Euclidean geometry & euklidische Geometrie \\
    existential quantifier & Existenzquantor \\
    existential uniqueness quantifier & Eindeutigkeitsquantor \\
    expression & Ausdruck \\
    Extended Backus--Naur Form & erweiterte Backus--Naur-Form \\
    extended frame & erweiterter Frame \\
    extended language & erweiterte Sprache \\
     & \\
    family & Familie \\
    Fermat's Last Therorem & Großer Fermatscher Satz \\
    file inclusion & Dateieinbindung \\
    file name & Dateiname \\
    finitary proof & finitistischer Beweis \\
    finite induction & finite Induktion \\
    finite n-termed sequence & endliche n-gliedrige Folge \\
    first-order logic (FOL) & Logik erster Ordnung \\
    floating hypothesis & fließende Hypothese \\
    formal logic & formale Logik \\
    formal proof & formaler Beweis \\
    formal system & formales System \\
    formalism & Formalismus \\
    forms & Formen \\
    foundations of mathematics & Grundlagen der Mathematik \\
    founded relation & fundierte Relation \\
    four-color theorem & Vier-Farben-Satz \\
    frame & Frame \\
    free logic & freie Logik \\
    free variable & freie Variable \\
    function & Funktion \\
    function value & Funktionswert \\
     & \\
    gaps in proofs & Lücken in Beweisen \\
    global statement & globale Anweisung \\
    Gödel's incompleteness theorem &  Gödelscher Unvollständigkeitssatz \\
    grave accent & einfaches Anführungszeichen \\
    group theory & Gruppentheorie \\
     & \\
    hierarchy & Hierarchie \\
    higher-order logic (HOL) & Logik höherer Ordnung \\
    hypothesis & Hypothese \\
    hypothesis association & Hypothesenzuordnung \\
    hypothesis label & Hypothesenlabel \\
     & \\
    iff & genau dann, wenn \\
    image & Bild \\
    implication & Implikation \\
    implicit axiom & implizites Axiom \\
    implicit substitution & implizite Substitution \\
    inaccessible cardinal & unzugängliche Kardinalzahl \\
    included file & eingebundene Datei \\
    individual metavariable & individuelle Metavariable \\
    individual variable & individuelle Variable \\
    inference & Inferenz \\
    inference form & Inferenzform \\
    inference rule & Inferenzregel \\
    infinite set & unendliche Menge \\
    infinity & Unendlichkeit \\
    inflix connective & Infix-Konnektor \\
    informal proof & informeller Beweis \\
    integer & ganze Zahl \\
    intersection & Schnittmenge \\
    intuitionism & Intuitionismus \\
     & \\
    keyword & Schlüsselwort \\
     & \\
    label & Label \\
    label declaration & Label-Deklaration \\
    label mode & Label-Modus \\
    label reference & Label-Referenz \\
    label sequence & Label-Sequenz \\
    Lemmon-style proof & Lemmon-Stil Beweis \\
    limit ordinal & Grenzzahl \\
    local label & lokales Label \\
    local variable & lokale Variable \\
    logical AND & logisches UND \\
    logical equivalenz & logische Äquivalenz \\
    logical hypothesis & logische Hypothese \\
    logical OR & logisches ODER \\
     & \\
    machine learning & maschinelles Lernen \\
    Macintosh file name & Macintosh-Dateinamen \\
    mandatory \texttt{\$d} statement & obligatorische \texttt{\$d}-Anweisung \\
    \hangindent=0.5cm mandatory disjoint-{\allowbreak}variable restriction & \hangindent=0.5cm obligatorische disjunkte Variableneinschränkung \\
    \hangindent=0.5cm mandatory distinct-{\allowbreak}variable restriction & \hangindent=0.5cm obligatorische disjunkte Variableneinschränkung \\
    mandatory hypothesis & obligatorische Hypothese \\
    mandatory variable & obligatorische Variable \\
    mandatory variable-type hypothesis & \hangindent=0.5cm obligatorische Variablentyp-{\allowbreak}Hypothese\vspace{2pt} \\
    mapping & Zuordnung \\
    markup notation & Auszeichnungsnotation \\
    math mode & Mathe-Modus \\
    math symbol & mathematisches Symbol \\
    mathematical induction & vollständige Induktion \\
    member & Glied \\
    member & Mitglied \\
    metalanguage & Metasprache \\
    metalogic & Metalogik \\
    metalogical completeness & metalogische Vollständigkeit \\
    Metamath & Metamath \\
    Metamath Language EBNF & Metamath-Sprache EBNF \\
    Metamath Proof Explorer & Metamath Proof Explorer \\
    metamathematics & Metamathematik \\
    metatheorem & Metatheorem \\
    metavariable & Metavariable \\
    MIU-system & MIU-System \\
    modal logic & modale Logik \\
    model theory & Modelltheorie \\
    modus ponens & Modus ponens \\
    Monaco font & Monaco Schriftart \\
    monospaced font & nichtproportionale Schriftart \\
     & \\
    natural deduction & natürliche Deduktion \\
    natural number & natürliche Zahl \\
    negation & Negation \\
    nested block & verschachtelter Block \\
    nesting level & Verschachtelungstiefe \\
    non-trivial theory & nicht-triviale Theorie \\
    normal proof & normaler Beweis \\
    null set & leere Menge \\
    number theory & Zahlentheorie \\
     & \\
    object & Objekt \\
    object language & Objektsprache \\
    omega & Omega \\
    one-to-one function & injektive Funktion \\
    one-to-one, onto function & bijektive Funktion \\
    onto function & surjektive Funktion \\
    operating system command & Betriebssystem-Befehl \\
    operation & Operation \\
    operator precedence & Vorrangigkeit eines Operators \\
    optional disjoint-{\allowbreak}variable restriction & \hangindent=0.5cm optionale disjunkte Variableneinschränkung\vspace{3pt} \\
    optional hypothesis & optionale Hypothese \\
    optional variable & optionale Variable \\
    ordered pair & geordnetes Paar \\
    ordinal addition & ordinale Addition \\
    ordinal number & Ordinalzahl \\
    ordinal predicate & Ordinalprädikat \\
    outermost block & äußerster Block \\
     & \\
    pair & Paar \\
    parser & Parser \\
    Pasch's axiom & Axiom von Pasch \\
    Peano's postulates & Peano-Postulate \\
    Pierce's axiom & Pierces Axiom \\
    plain text & einfacher Text \\
    Poincaré conjecture & Poincaré-Vermutung \\
    Polish notation & polnische Notation \\
    pop & entfernen \\
    postfix connective & Postfix-Konnektor \\
    power class & Potenzklasse \\
    power set & Potenzmenge \\
    predicate calculus & Prädikatenlogik \\
    prefix connective & Präfix-Konnektor \\
    pre-statement & Prä-Aussage \\
    Principia Mathematica & Principia Mathematica \\
    printable character & druckbares Zeichen \\
    printer & Drucker \\
    proof & Beweis \\
    Proof Assistant & Beweis-Assistent \\
    proof length & Beweislänge \\
    proof scheme & Beweisschema \\
    proof step & Beweisschritt \\
    proof theory & Beweistheorie \\
    proper class & echte Klasse \\
    proper definition & zulässige Definition \\
    proper substitution & echte Substitution \\
    propositional calculus & Aussagenlogik \\
    provable assertion & beweisbare Behauptung \\
    provable statement & beweisbare Aussage \\
    pure mathematics & reine Mathematik \\
    push & schieben \\
     & \\
    qualifying expression & qualifizierender Ausdruck \\
    quantifier theory & Quantifizierungstheorie \\
    quantum logic & Quantenlogik \\
    quantum mechanics & Quantenmechanik \\
     & \\
    range & Wertebereich \\
    rational number & rationale Zahl \\
    real number & reelle Zahl \\
    recursion operator & Rekursionsoperator \\
    recursive definition & rekursive Definition \\
    redeclaration of symbols & Umdeklarierung von Symbolen \\
    reduct & Redukt \\
    reflection principle & Reflexionsprinzip \\
    relation & Relation \\
    restricted universal quantifier & eingeschränkter Allquantor \\
    restricted existential quantifier & eingeschränkter Existenzquantor \\
    restriction & Einschränkung \\
    reverse polish notation (RPN) & umgekehrte polnische Notation \\
    Robbins algebra & Robbins-Algebra \\
    Robinson's resolution principle & Robinsons Resolutionsprinzip \\
    RPN order & RPN-Reihenfolge \\
    RPN stack & RPN-Stapel \\
    rule & Regel \\
    rule of generalization & Regel der Verallgemeinerung \\
    Russell's paradox & Russells Paradoxon \\
     & \\
    Schröder--Bernstein theorem & Schröder--Bernstein-{\allowbreak}Theorem \\
    scope & Gültigkeitsbereich \\
    scoping statement & Gültigkeitsbereichsanweisung \\
    sentential logic & Satzlogik \\
    set & Menge \\
    set difference & Mengendifferenz \\
    set theory & Mengenlehre \\
    set union & Vereinigungsmenge \\
    simple declaration & einfache Deklaration \\
    simple infinite sequence & einfache unendliche Folge \\
    simple metatheorem & einfaches Metatheorem \\
    singleton & einelementige Menge \\
    software bug & Programmfehler, Bug \\
    source buffer & Quellpuffer \\
    source file & Quelldatei \\
    special characters & Sonderzeichen \\
    stack & Stapel \\
    Standard Deduction Theorem & Standard-Deduktionstheorem \\
    statement & Anweisung, Aussage \\
    stylized epsilon & stilisiertes Epsilon \\
    subclass & Unterklasse \\
    subset & Teilmenge \\
    substitution & Substitution, Ersetzung \\
    substitution map & Substitutionsabbildung \\
    substitution theorem & Substitutionstheorem \\
    successor & Nachfolger \\
    Syllogism & Syllogismus \\
    symbol & Symbol \\
    syntax rules & Syntax-Regeln \\
     & \\
    tautology & Tautologie \\
    temporary variable & temporäre Variable \\
    term & Term \\
    text editor & Texteditor \\
    theorem & Theorem \\
    theorem scheme & Theoremschema \\
    tilde & Tilde \\
    token & Token \\
    topology & Topologie \\
    transfinite cardinal & transfinite Kardinalzahl \\
    transfinite recoursion & transfinite Rekursion \\
    transitive class & transitive Klasse \\
    transitive set & transitive Menge \\
    tree-style proof & Baumdarstellung eines Beweises \\
    truth table & Wahrheitstabelle \\
    turnstile & Drehkreuz \\
    type & Typ \\
    typecode & Typcode \\
    typesetting comment & Schriftsatzanweisung \\
     & \\
    unification & Vereinheitlichung \\
    union & Vereinigung \\
    universal class & universelle Klasse \\
    universal quantifier & Allquantor \\
    universe of a formal system & Universum eines formalen Systems \\
    Unix file name & Unix-Dateiname \\
    unordered pair & ungeordnetes Paar \\
    unordered tripel & ungeordnetes Tripel \\
     & \\
    variable & Variable \\
    variable declaration & Variablendeklaration \\
    variable substitution & Variablensubstitution \\
    variable type & Variablentyp \\
    variable-type hypothesis & Variablentyp-Hypothese \\
    Venn diagram & Venn-Diagramm \\
     & \\
    Weak Deduction Theorem & Theorem der schwachen Deduktion \\
    weak logic & schwache Logik \\
    well-formed formula & wohlgeformte Formel (wff) \\
    well-ordering & Wohlordnung \\
    white space & Whitespace, Leerraum \\
    word processing & Textverarbeitung \\
     & \\
    Zermelo--Fraenkel set theory & Zermelo--Fraenkel-{\allowbreak}Mengenlehre \\
    ZFC set theory & ZFC-Mengenlehre \\    
\end{longtabu}

% \chapter{Disclaimer and Trademarks}
%
% Information in this document is subject to change without notice and does not
% represent a commitment on the part of Norman Megill.
% \vspace{2ex}
%
% \noindent Norman D. Megill makes no warranties, either express or implied,
% regarding the Metamath computer software package.
%
% \vspace{2ex}
%
% \noindent Any trademarks mentioned in this book are the property of
% their respective owners.  The name "`Metamath"' is a trademark of
% Norman Megill.
%

\cleardoublepage
\phantomsection  % fixes the link anchor
\addcontentsline{toc}{chapter}{\bibname}

\bibliography{metamath}
%% metamath.tex - Version of 2-Jun-2019
% If you change the date above, also change the "Printed date" below.
% SPDX-License-Identifier: CC0-1.0
%
%                              PUBLIC DOMAIN
%
% This file (specifically, the version of this file with the above date)
% has been released into the Public Domain per the
% Creative Commons CC0 1.0 Universal (CC0 1.0) Public Domain Dedication
% https://creativecommons.org/publicdomain/zero/1.0/
%
% The public domain release applies worldwide.  In case this is not
% legally possible, the right is granted to use the work for any purpose,
% without any conditions, unless such conditions are required by law.
%
% Several short, attributed quotations from copyrighted works
% appear in this file under the ``fair use'' provision of Section 107 of
% the United States Copyright Act (Title 17 of the {\em United States
% Code}).  The public-domain status of this file is not applicable to
% those quotations.
%
% Norman Megill - email: nm(at)alum(dot)mit(dot)edu
%
% David A. Wheeler also donates his improvements to this file to the
% public domain per the CC0.  He works at the Institute for Defense Analyses
% (IDA), but IDA has agreed that this Metamath work is outside its "lane"
% and is not a work by IDA.  This was specifically confirmed by
% Margaret E. Myers (Division Director of the Information Technology
% and Systems Division) on 2019-05-24 and by Ben Lindorf (General Counsel)
% on 2019-05-22.

% This file, 'metamath.tex', is self-contained with everything needed to
% generate the the PDF file 'metamath.pdf' (the _Metamath_ book) on
% standard LaTeX 2e installations.  The auxiliary files are embedded with
% "filecontents" commands.  To generate metamath.pdf file, run these
% commands under Linux or Cygwin in the directory that contains
% 'metamath.tex':
%
%   rm -f realref.sty metamath.bib
%   touch metamath.ind
%   pdflatex metamath
%   pdflatex metamath
%   bibtex metamath
%   makeindex metamath
%   pdflatex metamath
%   pdflatex metamath
%
% The warnings that occur in the initial runs of pdflatex can be ignored.
% For the final run,
%
%   egrep -i 'error|warn' metamath.log
%
% should show exactly these 5 warnings:
%
%   LaTeX Warning: File `realref.sty' already exists on the system.
%   LaTeX Warning: File `metamath.bib' already exists on the system.
%   LaTeX Font Warning: Font shape `OMS/cmtt/m/n' undefined
%   LaTeX Font Warning: Font shape `OMS/cmtt/bx/n' undefined
%   LaTeX Font Warning: Some font shapes were not available, defaults
%       substituted.
%
% Search for "Uncomment" below if you want to suppress hyperlink boxes
% in the PDF output file
%
% TYPOGRAPHICAL NOTES:
% * It is customary to use an en dash (--) to "connect" names of different
%   people (and to denote ranges), and use a hyphen (-) for a
%   single compound name. Examples of connected multiple people are
%   Zermelo--Fraenkel, Schr\"{o}der--Bernstein, Tarski--Grothendieck,
%   Hewlett--Packard, and Backus--Naur.  Examples of a single person with
%   a compound name include Levi-Civita, Mittag-Leffler, and Burali-Forti.
% * Use non-breaking spaces after page abbreviations, e.g.,
%   p.~\pageref{note2002}.
%
% --------------------------- Start of realref.sty -----------------------------
\begin{filecontents}{realref.sty}
% Save the following as realref.sty.
% You can then use it with \usepackage{realref}
%
% This has \pageref jumping to the page on which the ref appears,
% \ref jumping to the point of the anchor, and \sectionref
% jumping to the start of section.
%
% Author:  Anthony Williams
%          Software Engineer
%          Nortel Networks Optical Components Ltd
% Date:    9 Nov 2001 (posted to comp.text.tex)
%
% The following declaration was made by Anthony Williams on
% 24 Jul 2006 (private email to Norman Megill):
%
%   ``I hereby donate the code for realref.sty posted on the
%   comp.text.tex newsgroup on 9th November 2001, accessible from
%   http://groups.google.com/group/comp.text.tex/msg/5a0e1cc13ea7fbb2
%   to the public domain.''
%
\ProvidesPackage{realref}
\RequirePackage[plainpages=false,pdfpagelabels=true]{hyperref}
\def\realref@anchorname{}
\AtBeginDocument{%
% ensure every label is a possible hyperlink target
\let\realref@oldrefstepcounter\refstepcounter%
\DeclareRobustCommand{\refstepcounter}[1]{\realref@oldrefstepcounter{#1}
\edef\realref@anchorname{\string #1.\@currentlabel}%
}%
\let\realref@oldlabel\label%
\DeclareRobustCommand{\label}[1]{\realref@oldlabel{#1}\hypertarget{#1}{}%
\@bsphack\protected@write\@auxout{}{%
    \string\expandafter\gdef\protect\csname
    page@num.#1\string\endcsname{\thepage}%
    \string\expandafter\gdef\protect\csname
    ref@num.#1\string\endcsname{\@currentlabel}%
    \string\expandafter\gdef\protect\csname
    sectionref@name.#1\string\endcsname{\realref@anchorname}%
}\@esphack}%
\DeclareRobustCommand\pageref[1]{{\edef\a{\csname
            page@num.#1\endcsname}\expandafter\hyperlink{page.\a}{\a}}}%
\DeclareRobustCommand\ref[1]{{\edef\a{\csname
            ref@num.#1\endcsname}\hyperlink{#1}{\a}}}%
\DeclareRobustCommand\sectionref[1]{{\edef\a{\csname
            ref@num.#1\endcsname}\edef\b{\csname
            sectionref@name.#1\endcsname}\hyperlink{\b}{\a}}}%
}
\end{filecontents}
% ---------------------------- End of realref.sty ------------------------------

% --------------------------- Start of metamath.bib -----------------------------
\begin{filecontents}{metamath.bib}
@book{Albers, editor = "Donald J. Albers and G. L. Alexanderson",
  title = "Mathematical People",
  publisher = "Contemporary Books, Inc.",
  address = "Chicago",
  note = "[QA28.M37]",
  year = 1985 }
@book{Anderson, author = "Alan Ross Anderson and Nuel D. Belnap",
  title = "Entailment",
  publisher = "Princeton University Press",
  address = "Princeton",
  volume = 1,
  note = "[QA9.A634 1975 v.1]",
  year = 1975}
@book{Barrow, author = "John D. Barrow",
  title = "Theories of Everything:  The Quest for Ultimate Explanation",
  publisher = "Oxford University Press",
  address = "Oxford",
  note = "[Q175.B225]",
  year = 1991 }
@book{Behnke,
  editor = "H. Behnke and F. Backmann and K. Fladt and W. S{\"{u}}ss",
  title = "Fundamentals of Mathematics",
  volume = "I",
  publisher = "The MIT Press",
  address = "Cambridge, Massachusetts",
  note = "[QA37.2.B413]",
  year = 1974 }
@book{Bell, author = "J. L. Bell and M. Machover",
  title = "A Course in Mathematical Logic",
  publisher = "North-Holland",
  address = "Amsterdam",
  note = "[QA9.B3953]",
  year = 1977 }
@inproceedings{Blass, author = "Andrea Blass",
  title = "The Interaction Between Category Theory and Set Theory",
  pages = "5--29",
  booktitle = "Mathematical Applications of Category Theory (Proceedings
     of the Special Session on Mathematical Applications
     Category Theory, 89th Annual Meeting of the American Mathematical
     Society, held in Denver, Colorado January 5--9, 1983)",
  editor = "John Walter Gray",
  year = 1983,
  note = "[QA169.A47 1983]",
  publisher = "American Mathematical Society",
  address = "Providence, Rhode Island"}
@proceedings{Bledsoe, editor = "W. W. Bledsoe and D. W. Loveland",
  title = "Automated Theorem Proving:  After 25 Years (Proceedings
     of the Special Session on Automatic Theorem Proving,
     89th Annual Meeting of the American Mathematical
     Society, held in Denver, Colorado January 5--9, 1983)",
  year = 1983,
  note = "[QA76.9.A96.S64 1983]",
  publisher = "American Mathematical Society",
  address = "Providence, Rhode Island" }
@book{Boolos, author = "George S. Boolos and Richard C. Jeffrey",
  title = "Computability and Log\-ic",
  publisher = "Cambridge University Press",
  edition = "third",
  address = "Cambridge",
  note = "[QA9.59.B66 1989]",
  year = 1989 }
@book{Campbell, author = "John Campbell",
  title = "Programmer's Progress",
  publisher = "White Star Software",
  address = "Box 51623, Palo Alto, CA 94303",
  year = 1991 }
@article{DBLP:journals/corr/Carneiro14,
  author    = {Mario Carneiro},
  title     = {Conversion of {HOL} Light proofs into Metamath},
  journal   = {CoRR},
  volume    = {abs/1412.8091},
  year      = {2014},
  url       = {http://arxiv.org/abs/1412.8091},
  archivePrefix = {arXiv},
  eprint    = {1412.8091},
  timestamp = {Mon, 13 Aug 2018 16:47:05 +0200},
  biburl    = {https://dblp.org/rec/bib/journals/corr/Carneiro14},
  bibsource = {dblp computer science bibliography, https://dblp.org}
}
@article{CarneiroND,
  author    = {Mario Carneiro},
  title     = {Natural Deductions in the Metamath Proof Language},
  url       = {http://us.metamath.org/ocat/natded.pdf},
  year      = 2014
}
@inproceedings{Chou, author = "Shang-Ching Chou",
  title = "Proving Elementary Geometry Theorems Using {W}u's Algorithm",
  pages = "243--286",
  booktitle = "Automated Theorem Proving:  After 25 Years (Proceedings
     of the Special Session on Automatic Theorem Proving,
     89th Annual Meeting of the American Mathematical
     Society, held in Denver, Colorado January 5--9, 1983)",
  editor = "W. W. Bledsoe and D. W. Loveland",
  year = 1983,
  note = "[QA76.9.A96.S64 1983]",
  publisher = "American Mathematical Society",
  address = "Providence, Rhode Island" }
@book{Clemente, author = "Daniel Clemente Laboreo",
  title = "Introduction to natural deduction",
  year = 2014,
  url = "http://www.danielclemente.com/logica/dn.en.pdf" }
@incollection{Courant, author = "Richard Courant and Herbert Robbins",
  title = "Topology",
  pages = "573--590",
  booktitle = "The World of Mathematics, Volume One",
  editor = "James R. Newman",
  publisher = "Simon and Schuster",
  address = "New York",
  note = "[QA3.W67 1988]",
  year = 1956 }
@book{Curry, author = "Haskell B. Curry",
  title = "Foundations of Mathematical Logic",
  publisher = "Dover Publications, Inc.",
  address = "New York",
  note = "[QA9.C976 1977]",
  year = 1977 }
@book{Davis, author = "Philip J. Davis and Reuben Hersh",
  title = "The Mathematical Experience",
  publisher = "Birkh{\"{a}}user Boston",
  address = "Boston",
  note = "[QA8.4.D37 1982]",
  year = 1981 }
@incollection{deMillo,
  author = "Richard de Millo and Richard Lipton and Alan Perlis",
  title = "Social Processes and Proofs of Theorems and Programs",
  pages = "267--285",
  booktitle = "New Directions in the Philosophy of Mathematics",
  editor = "Thomas Tymoczko",
  publisher = "Birkh{\"{a}}user Boston, Inc.",
  address = "Boston",
  note = "[QA8.6.N48 1986]",
  year = 1986 }
@book{Edwards, author = "Robert E. Edwards",
  title = "A Formal Background to Mathematics",
  publisher = "Springer-Verlag",
  address = "New York",
  note = "[QA37.2.E38 v.1a]",
  year = 1979 }
@book{Enderton, author = "Herbert B. Enderton",
  title = "Elements of Set Theory",
  publisher = "Academic Press, Inc.",
  address = "San Diego",
  note = "[QA248.E5]",
  year = 1977 }
@book{Goodstein, author = "R. L. Goodstein",
  title = "Development of Mathematical Logic",
  publisher = "Springer-Verlag New York Inc.",
  address = "New York",
  note = "[QA9.G6554]",
  year = 1971 }
@book{Guillen, author = "Michael Guillen",
  title = "Bridges to Infinity",
  publisher = "Jeremy P. Tarcher, Inc.",
  address = "Los Angeles",
  note = "[QA93.G8]",
  year = 1983 }
@book{Hamilton, author = "Alan G. Hamilton",
  title = "Logic for Mathematicians",
  edition = "revised",
  publisher = "Cambridge University Press",
  address = "Cambridge",
  note = "[QA9.H298]",
  year = 1988 }
@unpublished{Harrison, author = "John Robert Harrison",
  title = "Metatheory and Reflection in Theorem Proving:
    A Survey and Critique",
  note = "Technical Report
    CRC-053.
    SRI Cambridge,
    Millers Yard, Cambridge, UK,
    1995.
    Available on the Web as
{\verb+http:+}\-{\verb+//www.cl.cam.ac.uk/users/jrh/papers/reflect.html+}"}
@TECHREPORT{Harrison-thesis,
        author          = "John Robert Harrison",
        title           = "Theorem Proving with the Real Numbers",
        institution   = "University of Cambridge Computer
                         Lab\-o\-ra\-to\-ry",
        address         = "New Museums Site, Pembroke Street, Cambridge,
                           CB2 3QG, UK",
        year            = 1996,
        number          = 408,
        type            = "Technical Report",
        note            = "Author's PhD thesis,
   available on the Web at
{\verb+http:+}\-{\verb+//www.cl.cam.ac.uk+}\-{\verb+/users+}\-{\verb+/jrh+}%
\-{\verb+/papers+}\-{\verb+/thesis.html+}"}
@book{Herrlich, author = "Horst Herrlich and George E. Strecker",
  title = "Category Theory:  An Introduction",
  publisher = "Allyn and Bacon Inc.",
  address = "Boston",
  note = "[QA169.H567]",
  year = 1973 }
@article{Hindley, author = "J. Roger Hindley and David Meredith",
  title = "Principal Type-Schemes and Condensed Detachment",
  journal = "The Journal of Symbolic Logic",
  volume = 55,
  year = 1990,
  note = "[QA.J87]",
  pages = "90--105" }
@book{Hofstadter, author = "Douglas R. Hofstadter",
  title = "G{\"{o}}del, Escher, Bach",
  publisher = "Basic Books, Inc.",
  address = "New York",
  note = "[QA9.H63 1980]",
  year = 1979 }
@article{Indrzejczak, author= "Andrzej Indrzejczak",
  title = "Natural Deduction, Hybrid Systems and Modal Logic",
  journal = "Trends in Logic",
  volume = 30,
  publisher = "Springer",
  year = 2010 }
@article{Kalish, author = "D. Kalish and R. Montague",
  title = "On {T}arski's Formalization of Predicate Logic with Identity",
  journal = "Archiv f{\"{u}}r Mathematische Logik und Grundlagenfor\-schung",
  volume = 7,
  year = 1965,
  note = "[QA.A673]",
  pages = "81--101" }
@article{Kalman, author = "J. A. Kalman",
  title = "Condensed Detachment as a Rule of Inference",
  journal = "Studia Logica",
  volume = 42,
  number = 4,
  year = 1983,
  note = "[B18.P6.S933]",
  pages = "443-451" }
@book{Kline, author = "Morris Kline",
  title = "Mathematical Thought from Ancient to Modern Times",
  publisher = "Oxford University Press",
  address = "New York",
  note = "[QA21.K516 1990 v.3]",
  year = 1972 }
@book{Klinel, author = "Morris Kline",
  title = "Mathematics, The Loss of Certainty",
  publisher = "Oxford University Press",
  address = "New York",
  note = "[QA21.K525]",
  year = 1980 }
@book{Kramer, author = "Edna E. Kramer",
  title = "The Nature and Growth of Modern Mathematics",
  publisher = "Princeton University Press",
  address = "Princeton, New Jersey",
  note = "[QA93.K89 1981]",
  year = 1981 }
@article{Knill, author = "Oliver Knill",
  title = "Some Fundamental Theorems in Mathematics",
  year = "2018",
  url = "https://arxiv.org/abs/1807.08416" }
@book{Landau, author = "Edmund Landau",
  title = "Foundations of Analysis",
  publisher = "Chelsea Publishing Company",
  address = "New York",
  edition = "second",
  note = "[QA241.L2541 1960]",
  year = 1960 }
@article{Leblanc, author = "Hugues Leblanc",
  title = "On {M}eyer and {L}ambert's Quantificational Calculus {FQ}",
  journal = "The Journal of Symbolic Logic",
  volume = 33,
  year = 1968,
  note = "[QA.J87]",
  pages = "275--280" }
@article{Lejewski, author = "Czeslaw Lejewski",
  title = "On Implicational Definitions",
  journal = "Studia Logica",
  volume = 8,
  year = 1958,
  note = "[B18.P6.S933]",
  pages = "189--208" }
@book{Levy, author = "Azriel Levy",
  title = "Basic Set Theory",
  publisher = "Dover Publications",
  address = "Mineola, NY",
  year = "2002"
}
@book{Margaris, author = "Angelo Margaris",
  title = "First Order Mathematical Logic",
  publisher = "Blaisdell Publishing Company",
  address = "Waltham, Massachusetts",
  note = "[QA9.M327]",
  year = 1967}
@book{Manin, author = "Yu I. Manin",
  title = "A Course in Mathematical Logic",
  publisher = "Springer-Verlag",
  address = "New York",
  note = "[QA9.M29613]",
  year = "1977" }
@article{Mathias, author = "Adrian R. D. Mathias",
  title = "A Term of Length 4,523,659,424,929",
  journal = "Synthese",
  volume = 133,
  year = 2002,
  note = "[Q.S993]",
  pages = "75--86" }
@article{Megill, author = "Norman D. Megill",
  title = "A Finitely Axiomatized Formalization of Predicate Calculus
     with Equality",
  journal = "Notre Dame Journal of Formal Logic",
  volume = 36,
  year = 1995,
  note = "[QA.N914]",
  pages = "435--453" }
@unpublished{Megillc, author = "Norman D. Megill",
  title = "A Shorter Equivalent of the Axiom of Choice",
  month = "June",
  note = "Unpublished",
  year = 1991 }
@article{MegillBunder, author = "Norman D. Megill and Martin W.
    Bunder",
  title = "Weaker {D}-Complete Logics",
  journal = "Journal of the IGPL",
  volume = 4,
  year = 1996,
  pages = "215--225",
  note = "Available on the Web at
{\verb+http:+}\-{\verb+//www.mpi-sb.mpg.de+}\-{\verb+/igpl+}%
\-{\verb+/Journal+}\-{\verb+/V4-2+}\-{\verb+/#Megill+}"}
}
@book{Mendelson, author = "Elliott Mendelson",
  title = "Introduction to Mathematical Logic",
  edition = "second",
  publisher = "D. Van Nostrand Company, Inc.",
  address = "New York",
  note = "[QA9.M537 1979]",
  year = 1979 }
@article{Meredith, author = "David Meredith",
  title = "In Memoriam {C}arew {A}rthur {M}eredith (1904-1976)",
  journal = "Notre Dame Journal of Formal Logic",
  volume = 18,
  year = 1977,
  note = "[QA.N914]",
  pages = "513--516" }
@article{CAMeredith, author = "C. A. Meredith",
  title = "Single Axioms for the Systems ({C},{N}), ({C},{O}) and ({A},{N})
      of the Two-Valued Propositional Calculus",
  journal = "The Journal of Computing Systems",
  volume = 3,
  year = 1953,
  pages = "155--164" }
@article{Monk, author = "J. Donald Monk",
  title = "Provability With Finitely Many Variables",
  journal = "The Journal of Symbolic Logic",
  volume = 27,
  year = 1971,
  note = "[QA.J87]",
  pages = "353--358" }
@article{Monks, author = "J. Donald Monk",
  title = "Substitutionless Predicate Logic With Identity",
  journal = "Archiv f{\"{u}}r Mathematische Logik und Grundlagenfor\-schung",
  volume = 7,
  year = 1965,
  pages = "103--121" }
  %% Took out this from above to prevent LaTeX underfull warning:
  % note = "[QA.A673]",
@book{Moore, author = "A. W. Moore",
  title = "The Infinite",
  publisher = "Routledge",
  address = "New York",
  note = "[BD411.M59]",
  year = 1989}
@book{Munkres, author = "James R. Munkres",
  title = "Topology: A First Course",
  publisher = "Prentice-Hall, Inc.",
  address = "Englewood Cliffs, New Jersey",
  note = "[QA611.M82]",
  year = 1975}
@article{Nemesszeghy, author = "E. Z. Nemesszeghy and E. A. Nemesszeghy",
  title = "On Strongly Creative Definitions:  A Reply to {V}. {F}. {R}ickey",
  journal = "Logique et Analyse (N.\ S.)",
  year = 1977,
  volume = 20,
  note = "[BC.L832]",
  pages = "111--115" }
@unpublished{Nemeti, author = "N{\'{e}}meti, I.",
  title = "Algebraizations of Quantifier Logics, an Overview",
  note = "Version 11.4, preprint, Mathematical Institute, Budapest,
    1994.  A shortened version without proofs appeared in
    ``Algebraizations of quantifier logics, an introductory overview,''
   {\em Studia Logica}, 50:485--569, 1991 [B18.P6.S933]"}
@article{Pavicic, author = "M. Pavi{\v{c}}i{\'{c}}",
  title = "A New Axiomatization of Unified Quantum Logic",
  journal = "International Journal of Theoretical Physics",
  year = 1992,
  volume = 31,
  note = "[QC.I626]",
  pages = "1753 --1766" }
@book{Penrose, author = "Roger Penrose",
  title = "The Emperor's New Mind",
  publisher = "Oxford University Press",
  address = "New York",
  note = "[Q335.P415]",
  year = 1989 }
@book{PetersonI, author = "Ivars Peterson",
  title = "The Mathematical Tourist",
  publisher = "W. H. Freeman and Company",
  address = "New York",
  note = "[QA93.P475]",
  year = 1988 }
@article{Peterson, author = "Jeremy George Peterson",
  title = "An automatic theorem prover for substitution and detachment systems",
  journal = "Notre Dame Journal of Formal Logic",
  volume = 19,
  year = 1978,
  note = "[QA.N914]",
  pages = "119--122" }
@book{Quine, author = "Willard Van Orman Quine",
  title = "Set Theory and Its Logic",
  edition = "revised",
  publisher = "The Belknap Press of Harvard University Press",
  address = "Cambridge, Massachusetts",
  note = "[QA248.Q7 1969]",
  year = 1969 }
@article{Robinson, author = "J. A. Robinson",
  title = "A Machine-Oriented Logic Based on the Resolution Principle",
  journal = "Journal of the Association for Computing Machinery",
  year = 1965,
  volume = 12,
  pages = "23--41" }
@article{RobinsonT, author = "T. Thacher Robinson",
  title = "Independence of Two Nice Sets of Axioms for the Propositional
    Calculus",
  journal = "The Journal of Symbolic Logic",
  volume = 33,
  year = 1968,
  note = "[QA.J87]",
  pages = "265--270" }
@book{Rucker, author = "Rudy Rucker",
  title = "Infinity and the Mind:  The Science and Philosophy of the
    Infinite",
  publisher = "Bantam Books, Inc.",
  address = "New York",
  note = "[QA9.R79 1982]",
  year = 1982 }
@book{Russell, author = "Bertrand Russell",
  title = "Mysticism and Logic, and Other Essays",
  publisher = "Barnes \& Noble Books",
  address = "Totowa, New Jersey",
  note = "[B1649.R963.M9 1981]",
  year = 1981 }
@article{Russell2, author = "Bertrand Russell",
  title = "Recent Work on the Principles of Mathematics",
  journal = "International Monthly",
  volume = 4,
  year = 1901,
  pages = "84"}
@article{Schmidt, author = "Eric Schmidt",
  title = "Reductions in Norman Megill's axiom system for complex numbers",
  url = "http://us.metamath.org/downloads/schmidt-cnaxioms.pdf",
  year = "2012" }
@book{Shoenfield, author = "Joseph R. Shoenfield",
  title = "Mathematical Logic",
  publisher = "Addison-Wesley Publishing Company, Inc.",
  address = "Reading, Massachusetts",
  year = 1967,
  note = "[QA9.S52]" }
@book{Smullyan, author = "Raymond M. Smullyan",
  title = "Theory of Formal Systems",
  publisher = "Princeton University Press",
  address = "Princeton, New Jersey",
  year = 1961,
  note = "[QA248.5.S55]" }
@book{Solow, author = "Daniel Solow",
  title = "How to Read and Do Proofs:  An Introduction to Mathematical
    Thought Process",
  publisher = "John Wiley \& Sons",
  address = "New York",
  year = 1982,
  note = "[QA9.S577]" }
@book{Stark, author = "Harold M. Stark",
  title = "An Introduction to Number Theory",
  publisher = "Markham Publishing Company",
  address = "Chicago",
  note = "[QA241.S72 1978]",
  year = 1970 }
@article{Swart, author = "E. R. Swart",
  title = "The Philosophical Implications of the Four-Color Problem",
  journal = "American Mathematical Monthly",
  year = 1980,
  volume = 87,
  month = "November",
  note = "[QA.A5125]",
  pages = "697--707" }
@book{Szpiro, author = "George G. Szpiro",
  title = "Poincar{\'{e}}'s Prize: The Hundred-Year Quest to Solve One
    of Math's Greatest Puzzles",
  publisher = "Penguin Books Ltd",
  address = "London",
  note = "[QA43.S985 2007]",
  year = 2007}
@book{Takeuti, author = "Gaisi Takeuti and Wilson M. Zaring",
  title = "Introduction to Axiomatic Set Theory",
  edition = "second",
  publisher = "Springer-Verlag New York Inc.",
  address = "New York",
  note = "[QA248.T136 1982]",
  year = 1982}
@inproceedings{Tarski, author = "Alfred Tarski",
  title = "What is Elementary Geometry",
  pages = "16--29",
  booktitle = "The Axiomatic Method, with Special Reference to Geometry and
     Physics (Proceedings of an International Symposium held at the University
     of California, Berkeley, December 26, 1957 --- January 4, 1958)",
  editor = "Leon Henkin and Patrick Suppes and Alfred Tarski",
  year = 1959,
  publisher = "North-Holland Publishing Company",
  address = "Amsterdam"}
@article{Tarski1965, author = "Alfred Tarski",
  title = "A Simplified Formalization of Predicate Logic with Identity",
  journal = "Archiv f{\"{u}}r Mathematische Logik und Grundlagenforschung",
  volume = 7,
  year = 1965,
  note = "[QA.A673]",
  pages = "61--79" }
@book{Tymoczko,
  title = "New Directions in the Philosophy of Mathematics",
  editor = "Thomas Tymoczko",
  publisher = "Birkh{\"{a}}user Boston, Inc.",
  address = "Boston",
  note = "[QA8.6.N48 1986]",
  year = 1986 }
@incollection{Wang,
  author = "Hao Wang",
  title = "Theory and Practice in Mathematics",
  pages = "129--152",
  booktitle = "New Directions in the Philosophy of Mathematics",
  editor = "Thomas Tymoczko",
  publisher = "Birkh{\"{a}}user Boston, Inc.",
  address = "Boston",
  note = "[QA8.6.N48 1986]",
  year = 1986 }
@manual{Webster,
  title = "Webster's New Collegiate Dictionary",
  organization = "G. \& C. Merriam Co.",
  address = "Springfield, Massachusetts",
  note = "[PE1628.W4M4 1977]",
  year = 1977 }
@manual{Whitehead, author = "Alfred North Whitehead",
  title = "An Introduction to Mathematics",
  year = 1911 }
@book{PM, author = "Alfred North Whitehead and Bertrand Russell",
  title = "Principia Mathematica",
  edition = "second",
  publisher = "Cambridge University Press",
  address = "Cambridge",
  year = "1927",
  note = "(3 vols.) [QA9.W592 1927]" }
@article{DBLP:journals/corr/Whalen16,
  author    = {Daniel Whalen},
  title     = {Holophrasm: a neural Automated Theorem Prover for higher-order logic},
  journal   = {CoRR},
  volume    = {abs/1608.02644},
  year      = {2016},
  url       = {http://arxiv.org/abs/1608.02644},
  archivePrefix = {arXiv},
  eprint    = {1608.02644},
  timestamp = {Mon, 13 Aug 2018 16:46:19 +0200},
  biburl    = {https://dblp.org/rec/bib/journals/corr/Whalen16},
  bibsource = {dblp computer science bibliography, https://dblp.org} }
@article{Wiedijk-revisited,
  author = {Freek Wiedijk},
  title = {The QED Manifesto Revisited},
  year = {2007},
  url = {http://mizar.org/trybulec65/8.pdf} }
@book{Wolfram,
  author = "Stephen Wolfram",
  title = "Mathematica:  A System for Doing Mathematics by Computer",
  edition = "second",
  publisher = "Addison-Wesley Publishing Co.",
  address = "Redwood City, California",
  note = "[QA76.95.W65 1991]",
  year = 1991 }
@book{Wos, author = "Larry Wos and Ross Overbeek and Ewing Lusk and Jim Boyle",
  title = "Automated Reasoning:  Introduction and Applications",
  edition = "second",
  publisher = "McGraw-Hill, Inc.",
  address = "New York",
  note = "[QA76.9.A96.A93 1992]",
  year = 1992 }

%
%
%[1] Church, Alonzo, Introduction to Mathematical Logic,
% Volume 1, Princeton University Press, Princeton, N. J., 1956.
%
%[2] Cohen, Paul J., Set Theory and the Continuum Hypothesis,
% W. A. Benjamin, Inc., Reading, Mass., 1966.
%
%[3] Hamilton, Alan G., Logic for Mathematicians, Cambridge
% University Press,
% Cambridge, 1988.

%[6] Kleene, Stephen Cole, Introduction to Metamathematics, D.  Van
% Nostrand Company, Inc., Princeton (1952).

%[13] Tarski, Alfred, "A simplified formalization of predicate
% logic with identity," Archiv fur Mathematische Logik und
% Grundlagenforschung, vol. 7 (1965), pp. 61-79.

%[14] Tarski, Alfred and Steven Givant, A Formalization of Set
% Theory Without Variables, American Mathematical Society Colloquium
% Publications, vol. 41, American Mathematical Society,
% Providence, R. I., 1987.

%[15] Zeman, J. J., Modal Logic, Oxford University Press, Oxford, 1973.
\end{filecontents}
% --------------------------- End of metamath.bib -----------------------------


%Book: Metamath
%Author:  Norman Megill Email:  nm at alum.mit.edu
%Author:  David A. Wheeler Email:  dwheeler at dwheeler.com

% A book template example
% http://www.stsci.edu/ftp/software/tex/bookstuff/book.template

\documentclass[leqno]{book} % LaTeX 2e. 10pt. Use [leqno,12pt] for 12pt
% hyperref 2002/05/27 v6.72r  (couldn't get pagebackref to work)
\usepackage[plainpages=false,pdfpagelabels=true]{hyperref}

\usepackage{needspace}     % Enable control over page breaks
\usepackage{breqn}         % automatic equation breaking
\usepackage{microtype}     % microtypography, reduces hyphenation

% Packages for flexible tables.  We need to be able to
% wrap text within a cell (with automatically-determined widths) AND
% split a table automatically across multiple pages.
% * "tabularx" wraps text in cells but only 1 page
% * "longtable" goes across pages but by itself is incompatible with tabularx
% * "ltxtable" combines longtable and tabularx, but table contents
%    must be in a separate file.
% * "ltablex" combines tabularx and longtable - must install specially
% * "booktabs" is recommended as a way to improve the look of tables,
%   but doesn't add these capabilities.
% * "tabu" much more capable and seems to be recommended. So use that.

\usepackage{makecell}      % Enable forced line splits within a table cell
% v4.13 needed for tabu: https://tex.stackexchange.com/questions/600724/dimension-too-large-after-recent-longtable-update
\usepackage{longtable}[=v4.13] % Enable multi-page tables  
\usepackage{tabu}          % Multi-page tables with wrapped text in a cell

% You can find more Tex packages using commands like:
% tlmgr search --file tabu.sty
% find /usr/share/texmf-dist/ -name '*tab*'
%
%%%%%%%%%%%%%%%%%%%%%%%%%%%%%%%%%%%%%%%%%%%%%%%%%%%%%%%%%%%%%%%%%%%%%%%%%%%%
% Uncomment the next 3 lines to suppress boxes and colors on the hyperlinks
%%%%%%%%%%%%%%%%%%%%%%%%%%%%%%%%%%%%%%%%%%%%%%%%%%%%%%%%%%%%%%%%%%%%%%%%%%%%
%\hypersetup{
%colorlinks,citecolor=black,filecolor=black,linkcolor=black,urlcolor=black
%}
%
\usepackage{realref}

% Restarting page numbers: try?
%   \printglossary
%   \cleardoublepage
%   \pagenumbering{arabic}
%   \setcounter{page}{1}    ???needed
%   \include{chap1}

% not used:
% \def\R2Lurl#1#2{\mbox{\href{#1}\texttt{#2}}}

\usepackage{amssymb}

% Version 1 of book: margins: t=.4, b=.2, ll=.4, rr=.55
% \usepackage{anysize}
% % \papersize{<height>}{<width>}
% % \marginsize{<left>}{<right>}{<top>}{<bottom>}
% \papersize{9in}{6in}
% % l/r 0.6124-0.6170 works t/b 0.2418-0.3411 = 192pp. 0.2926-03118=exact
% \marginsize{0.7147in}{0.5147in}{0.4012in}{0.2012in}

\usepackage{anysize}
% \papersize{<height>}{<width>}
% \marginsize{<left>}{<right>}{<top>}{<bottom>}
\papersize{9in}{6in}
% l/r 0.85in&0.6431-0.6539 works t/b ?-?
%\marginsize{0.85in}{0.6485in}{0.55in}{0.35in}
\marginsize{0.8in}{0.65in}{0.5in}{0.3in}

% \usepackage[papersize={3.6in,4.8in},hmargin=0.1in,vmargin={0.1in,0.1in}]{geometry}  % page geometry
\usepackage{special-settings}

\raggedbottom
\makeindex

\begin{document}
% Discourage page widows and orphans:
\clubpenalty=300
\widowpenalty=300

%%%%%%% load in AMS fonts %%%%%%% % LaTeX 2.09 - obsolete in LaTeX 2e
%\input{amssym.def}
%\input{amssym.tex}
%\input{c:/texmf/tex/plain/amsfonts/amssym.def}
%\input{c:/texmf/tex/plain/amsfonts/amssym.tex}

\bibliographystyle{plain}
\pagenumbering{roman}
\pagestyle{headings}

\thispagestyle{empty}

\hfill
\vfill

\begin{center}
{\LARGE\bf Metamath} \\
\vspace{1ex}
{\large A Computer Language for Mathematical Proofs} \\
\vspace{7ex}
{\large Norman Megill} \\
\vspace{7ex}
with extensive revisions by \\
\vspace{1ex}
{\large David A. Wheeler} \\
\vspace{7ex}
% Printed date. If changing the date below, also fix the date at the beginning.
2019-06-02
\end{center}

\vfill
\hfill

\newpage
\thispagestyle{empty}

\hfill
\vfill

\begin{center}
$\sim$\ {\sc Public Domain}\ $\sim$

\vspace{2ex}
This book (including its later revisions)
has been released into the Public Domain by Norman Megill per the
Creative Commons CC0 1.0 Universal (CC0 1.0) Public Domain Dedication.
David A. Wheeler has done the same.
This public domain release applies worldwide.  In case this is not
legally possible, the right is granted to use the work for any purpose,
without any conditions, unless such conditions are required by law.
See \url{https://creativecommons.org/publicdomain/zero/1.0/}.

\vspace{3ex}
Several short, attributed quotations from copyrighted works
appear in this book under the ``fair use'' provision of Section 107 of
the United States Copyright Act (Title 17 of the {\em United States
Code}).  The public-domain status of this book is not applicable to
those quotations.

\vspace{3ex}
Any trademarks used in this book are the property of their owners.

% QA76.9.L63.M??

% \vspace{1ex}
%
% \vspace{1ex}
% {\small Permission is granted to make and distribute verbatim copies of this
% book
% provided the copyright notice and this
% permission notice are preserved on all copies.}
%
% \vspace{1ex}
% {\small Permission is granted to copy and distribute modified versions of this
% book under the conditions for verbatim copying, provided that the
% entire
% resulting derived work is distributed under the terms of a permission
% notice
% identical to this one.}
%
% \vspace{1ex}
% {\small Permission is granted to copy and distribute translations of this
% book into another language, under the above conditions for modified
% versions,
% except that this permission notice may be stated in a translation
% approved by the
% author.}
%
% \vspace{1ex}
% %{\small   For a copy of the \LaTeX\ source files for this book, contact
% %the author.} \\
% \ \\
% \ \\

\vspace{7ex}
% ISBN: 1-4116-3724-0 \\
% ISBN: 978-1-4116-3724-5 \\
ISBN: 978-0-359-70223-7 \\
{\ } \\
Lulu Press \\
Morrisville, North Carolina\\
USA


\hfill
\vfill

Norman Megill\\ 93 Bridge St., Lexington, MA 02421 \\
E-mail address: \texttt{nm{\char`\@}alum.mit.edu} \\
\vspace{7ex}
David A. Wheeler \\
E-mail address: \texttt{dwheeler{\char`\@}dwheeler.com} \\
% See notes added at end of Preface for revision history. \\
% For current information on the Metamath software see \\
\vspace{7ex}
\url{http://metamath.org}
\end{center}

\hfill
\vfill

{\parindent0pt%
\footnotesize{%
Cover: Aleph null ($\aleph_0$) is the symbol for the
first infinite cardinal number, discovered by Georg Cantor in 1873.
We use a red aleph null (with dark outline and gold glow) as the Metamath logo.
Credit: Norman Megill (1994) and Giovanni Mascellani (2019),
public domain.%
\index{aleph null}%
\index{Metamath!logo}\index{Cantor, Georg}\index{Mascellani, Giovanni}}}

% \newpage
% \thispagestyle{empty}
%
% \hfill
% \vfill
%
% \begin{center}
% {\it To my son Robin Dwight Megill}
% \end{center}
%
% \vfill
% \hfill
%
% \newpage

\tableofcontents
%\listoftables

\chapter*{Preface}
\markboth{PREFACE}{PREFACE}
\addcontentsline{toc}{section}{Preface}


% (For current information, see the notes added at the
% end of this preface on p.~\pageref{note2002}.)

\subsubsection{Overview}

Metamath\index{Metamath} is a computer language and an associated computer
program for archiving, verifying, and studying mathematical proofs at a very
detailed level.  The Metamath language incorporates no mathematics per se but
treats all mathematical statements as mere sequences of symbols.  You provide
Metamath with certain special sequences (axioms) that tell it what rules
of inference are allowed.  Metamath is not limited to any specific field of
mathematics.  The Metamath language is simple and robust, with an
almost total absence of hard-wired syntax, and
we\footnote{Unless otherwise noted, the words
``I,'' ``me,'' and ``my'' refer to Norman Megill\index{Megill, Norman}, while
``we,'' ``us,'' and ``our'' refer to Norman Megill and
David A. Wheeler\index{Wheeler, David A.}.}
believe that it
provides about the simplest possible framework that allows essentially all of
mathematics to be expressed with absolute rigor.

% index test
%\newcommand{\nn}[1]{#1n}
%\index{aaa@bbb}
%\index{abc!def}
%\index{abd|see{qqq}}
%\index{abe|nn}
%\index{abf|emph}
%\index{abg|(}
%\index{abg|)}

Using the Metamath language, you can build formal or mathematical
systems\index{formal system}\footnote{A formal or mathematical system consists
of a collection of symbols (such as $2$, $4$, $+$ and $=$), syntax rules that
describe how symbols may be combined to form a legal expression (called a
well-formed formula or {\em wff}, pronounced ``whiff''), some starting wffs
called axioms, and inference rules that describe how theorems may be derived
(proved) from the axioms.  A theorem is a mathematical fact such as $2+2=4$.
Strictly speaking, even an obvious fact such as this must be proved from
axioms to be formally acceptable to a mathematician.}\index{theorem}
\index{axiom}\index{rule}\index{well-formed formula (wff)} that involve
inferences from axioms.  Although a database is provided
that includes a recommended set of axioms for standard mathematics, if you
wish you can supply your own symbols, syntax, axioms, rules, and definitions.

The name ``Metamath'' was chosen to suggest that the language provides a
means for {\em describing} mathematics rather than {\em being} the
mathematics itself.  Actually in some sense any mathematical language is
metamathematical.  Symbols written on paper, or stored in a computer,
are not mathematics itself but rather a way of expressing mathematics.
For example ``7'' and ``VII'' are symbols for denoting the number seven
in Arabic and Roman numerals; neither {\em is} the number seven.

If you are able to understand and write computer programs, you should be able
to follow abstract mathematics with the aid of Metamath.  Used in conjunction
with standard textbooks, Metamath can guide you step-by-step towards an
understanding of abstract mathematics from a very rigorous viewpoint, even if
you have no formal abstract mathematics background.  By using a single,
consistent notation to express proofs, once you grasp its basic concepts
Metamath provides you with the ability to immediately follow and dissect
proofs even in totally unfamiliar areas.

Of course, just being able follow a proof will not necessarily give you an
intuitive familiarity with mathematics.  Memorizing the rules of chess does not
give you the ability to appreciate the game of a master, and knowing how the
notes on a musical score map to piano keys does not give you the ability to
hear in your head how it would sound.  But each of these can be a first step.

Metamath allows you to explore proofs in the sense that you can see the
theorem referenced at any step expanded in as much detail as you want, right
down to the underlying axioms of logic and set theory (in the case of the set
theory database provided).  While Metamath will not replace the higher-level
understanding that can only be acquired through exercises and hard work, being
able to see how gaps in a proof are filled in can give you increased
confidence that can speed up the learning process and save you time when you
get stuck.

The Metamath language breaks down a mathematical proof into its tiniest
possible parts.  These can be pieced together, like interlocking
pieces in a puzzle, only in a way that produces correct and absolutely rigorous
mathematics.

The nature of Metamath\index{Metamath} enforces very precise mathematical
thinking, similar to that involved in writing a computer program.  A crucial
difference, though, is that once a proof is verified (by the Metamath program)
to be correct, it is definitely correct; it can never have a hidden
``bug.''\index{computer program bugs}  After getting used to the kind of rigor
and accuracy provided by Metamath, you might even be tempted to
adopt the attitude that a proof should never be considered correct until it
has been verified by a computer, just as you would not completely trust a
manual calculation until you have verified it on a
calculator.

My goal
for Metamath was a system for describing and verifying
mathematics that is completely universal yet conceptually as simple as
possible.  In approaching mathematics from an axiomatic, formal viewpoint, I
wanted Metamath to be able to handle almost any mathematical system, not
necessarily with ease, but at least in principle and hopefully in practice. I
wanted it to verify proofs with absolute rigor, and for this reason Metamath
is what might be thought of as a ``compile-only'' language rather than an
algorithmic or Turing-machine language (Pascal, C, Prolog, Mathematica,
etc.).  In other words, a database written in the Metamath
language doesn't ``do'' anything; it merely exhibits mathematical knowledge
and permits this knowledge to be verified as being correct.  A program in an
algorithmic language can potentially have hidden bugs\index{computer program
bugs} as well as possibly being hard to understand.  But each token in a
Metamath database must be consistent with the database's earlier
contents according to simple, fixed rules.
If a database is verified
to be correct,\footnote{This includes
verification that a sequential list of proof steps results in the specified
theorem.} then the mathematical content is correct if the
verifier is correct and the axioms are correct.
The verification program could be incorrect, but the verification algorithm
is relatively simple (making it unlikely to be implemented incorrectly
by the Metamath program),
and there are over a dozen Metamath database verifiers
written by different people in different programming languages
(so these different verifiers can act as multiple reviewers of a database).
The most-used Metamath database, the Metamath Proof Explorer
(aka \texttt{set.mm}\index{set theory database (\texttt{set.mm})}%
\index{Metamath Proof Explorer}),
is currently verified by four different Metamath verifiers written by
four different people in four different languages, including the
original Metamath program described in this book.
The only ``bugs'' that can exist are in the statement of the axioms,
for example if the axioms are inconsistent (a famous problem shown to be
unsolvable by G\"{o}del's incompleteness theorem\index{G\"{o}del's
incompleteness theorem}).
However, real mathematical systems have very few axioms, and these can
be carefully studied.
All of this provides extraordinarily high confidence that the verified database
is in fact correct.

The Metamath program
doesn't prove theorems automatically but is designed to verify proofs
that you supply to it.
The underlying Metamath language is completely general and has no built-in,
preconceived notions about your formal system\index{formal system}, its logic
or its syntax.
For constructing proofs, the Metamath program has a Proof Assistant\index{Proof
Assistant} which helps you fill in some of a proof step's details, shows you
what choices you have at any step, and verifies the proof as you build it; but
you are still expected to provide the proof.

There are many other programs that can process or generate information
in the Metamath language, and more continue to be written.
This is in part because the Metamath language itself is very simple
and intentionally easy to automatically process.
Some programs, such as \texttt{mmj2}\index{mmj2}, include a proof assistant
that can automate some steps beyond what the Metamath program can do.
Mario Carneiro has developed an algorithm for converting proofs from
the OpenTheory interchange format, which can be translated to and from
any of the HOL family of proof languages (HOL4, HOL Light, ProofPower,
and Isabelle), into the
Metamath language \cite{DBLP:journals/corr/Carneiro14}\index{Carneiro, Mario}.
Daniel Whalen has developed Holophrasm, which can automatically
prove many Metamath proofs using
machine learning\index{machine learning}\index{artificial intelligence}
approaches
(including multiple neural networks\index{neural networks})\cite{DBLP:journals/corr/Whalen16}\index{Whalen, Daniel}.
However,
a discussion of these other programs is beyond the scope of this book.

Like most computer languages, the Metamath\index{Metamath} language uses the
standard ({\sc ascii}) characters on a computer keyboard, so it cannot
directly represent many of the special symbols that mathematicians use.  A
useful feature of the Metamath program is its ability to convert its notation
into the \LaTeX\ typesetting language.\index{latex@{\LaTeX}}  This feature
lets you convert the {\sc ascii} tokens you've defined into standard
mathematical symbols, so you end up with symbols and formulas you are familiar
with instead of somewhat cryptic {\sc ascii} representations of them.
The Metamath program can also generate HTML\index{HTML}, making it easy
to view results on the web and to see related information by using
hypertext links.

Metamath is probably conceptually different from anything you've seen
before and some aspects may take some getting used to.  This book will
help you decide whether Metamath suits your specific needs.

\subsubsection{Setting Your Expectations}
It is important for you to understand what Metamath\index{Metamath} is and is
not.  As mentioned, the Metamath program
is {\em not} an automated theorem prover but
rather a proof verifier.  Developing a database can be tedious, hard work,
especially if you want to make the proofs as short as possible, but it becomes
easier as you build up a collection of useful theorems.  The purpose of
Metamath is simply to document existing mathematics in an absolutely rigorous,
computer-verifiable way, not to aid directly in the creation of new
mathematics.  It also is not a magic solution for learning abstract
mathematics, although it may be helpful to be able to actually see the implied
rigor behind what you are learning from textbooks, as well as providing hints
to work out proofs that you are stumped on.

As of this writing, a sizable set theory database has been developed to
provide a foundation for many fields of mathematics, but much more work would
be required to develop useful databases for specific fields.

Metamath\index{Metamath} ``knows no math;'' it just provides a framework in
which to express mathematics.  Its language is very small.  You can define two
kinds of symbols, constants\index{constant} and variables\index{variable}.
The only thing Metamath knows how to do is to substitute strings of symbols
for the variables\index{substitution!variable}\index{variable substitution} in
an expression based on instructions you provide it in a proof, subject to
certain constraints you specify for the variables.  Even the decimal
representation of a number is merely a string of certain constants (digits)
which together, in a specific context, correspond to whatever mathematical
object you choose to define for it; unlike other computer languages, there is
no actual number stored inside the computer.  In a proof, you in effect
instruct Metamath what symbol substitutions to make in previous axioms or
theorems and join a sequence of them together to result in the desired
theorem.  This kind of symbol manipulation captures the essence of mathematics
at a preaxiomatic level.

\subsubsection{Metamath and Mathematical Literature}

In advanced mathematical literature, proofs are usually presented in the form
of short outlines that often only an expert can follow.  This is partly out of
a desire for brevity, but it would also be unwise (even if it were practical)
to present proofs in complete formal detail, since the overall picture would
be lost.\index{formal proof}

A solution I envision\label{envision} that would allow mathematics to remain
acceptable to the expert, yet increase its accessibility to non-specialists,
consists of a combination of the traditional short, informal proof in print
accompanied by a complete formal proof stored in a computer database.  In an
analogy with a computer program, the informal proof is like a set of comments
that describe the overall reasoning and content of the proof, whereas the
computer database is like the actual program and provides a means for anyone,
even a non-expert, to follow the proof in as much detail as desired, exploring
it back through layers of theorems (like subroutines that call other
subroutines) all the way back to the axioms of the theory.  In addition, the
computer database would have the advantage of providing absolute assurance
that the proof is correct, since each step can be verified automatically.

There are several other approaches besides Metamath to a project such
as this.  Section~\ref{proofverifiers} discusses some of these.

To us, a noble goal would be a database with hundreds of thousands of
theorems and their computer-verifiable proofs, encompassing a significant
fraction of known mathematics and available for instant access.
These would be fully verified by multiple independently-implemented verifiers,
to provide extremely high confidence that the proofs are completely correct.
The database would allow people to investigate whatever details they were
interested in, so that they could confirm whatever portions they wished.
Whether or not Metamath is an appropriate choice remains to be seen, but in
principle we believe it is sufficient.

\subsubsection{Formalism}

Over the past fifty years, a group of French mathematicians working
collectively under the pseudonym of Bourbaki\index{Bourbaki, Nicolas} have
co-authored a series of monographs that attempt to rigorously and
consistently formalize large bodies of mathematics from foundations.  On the
one hand, certainly such an effort has its merits; on the other hand, the
Bourbaki project has been criticized for its ``scholasticism'' and
``hyperaxiomatics'' that hide the intuitive steps that lead to the results
\cite[p.~191]{Barrow}\index{Barrow, John D.}.

Metamath unabashedly carries this philosophy to its extreme and no doubt is
subject to the same kind of criticism.  Nonetheless I think that in
conjunction with conventional approaches to mathematics Metamath can serve a
useful purpose.  The Bourbaki approach is essentially pedagogic, requiring the
reader to become intimately familiar with each detail in a very large
hierarchy before he or she can proceed to the next step.  The difference with
Metamath is that the ``reader'' (user) knows that all details are contained in
its computer database, available as needed; it does not demand that the user
know everything but conveniently makes available those portions that are of
interest.  As the body of all mathematical knowledge grows larger and larger,
no one individual can have a thorough grasp of its entirety.  Metamath
can finalize and put to rest any questions about the validity of any part of it
and can make any part of it accessible, in principle, to a non-specialist.

\subsubsection{A Personal Note}
Why did I develop Metamath\index{Metamath}?  I enjoy abstract mathematics, but
I sometimes get lost in a barrage of definitions and start to lose confidence
that my proofs are correct.  Or I reach a point where I lose sight of how
anything I'm doing relates to the axioms that a theory is based on and am
sometimes suspicious that there may be some overlooked implicit axiom
accidentally introduced along the way (as happened historically with Euclidean
geometry\index{Euclidean geometry}, whose omission of Pasch's
axiom\index{Pasch's axiom} went unnoticed for 2000 years
\cite[p.~160]{Davis}!). I'm also somewhat lazy and wish to avoid the effort
involved in re-verifying the gaps in informal proofs ``left to the reader;'' I
prefer to figure them out just once and not have to go through the same
frustration a year from now when I've forgotten what I did.  Metamath provides
better recovery of my efforts than scraps of paper that I can't
decipher anymore.  But mostly I find very appealing the idea of rigorously
archiving mathematical knowledge in a computer database, providing precision,
certainty, and elimination of human error.

\subsubsection{Note on Bibliography and Index}

The Bibliography usually includes the Library of Congress classification
for a work to make it easier for you to find it in on a university
library shelf.  The Index has author references to pages where their works
are cited, even though the authors' names may not appear on those pages.

\subsubsection{Acknowledgments}

Acknowledgments are first due to my wife, Deborah (who passed away on
September 4, 1998), for critiquing the manu\-script but most of all for
her patience and support.  I also wish to thank Joe Wright, Richard
Becker, Clarke Evans, Buddha Buck, and Jeremy Henty for helpful
comments.  Any errors, omissions, and other shortcomings are of course
my responsibility.

\subsubsection{Note Added June 22, 2005}\label{note2002}

The original, unpublished version of this book was written in 1997 and
distributed via the web.  The present edition has been updated to
reflect the current Metamath program and databases, as well as more
current {\sc url}s for Internet sites.  Thanks to Josh
Purinton\index{Purinton, Josh}, One Hand
Clapping, Mel L.\ O'Cat, and Roy F. Longton for pointing out
typographical and other errors.  I have also benefitted from numerous
discussions with Raph Levien\index{Levien, Raph}, who has extended
Metamath's philosophy of rigor to result in his {\em
Ghilbert}\index{Ghilbert} proof language (\url{http://ghilbert.org}).

Robert (Bob) Solovay\index{Solovay, Robert} communicated a new result of
A.~R.~D.~Mathias on the system of Bourbaki, and the text has been
updated accordingly (p.~\pageref{bourbaki}).

Bob also pointed out a clarification of the literature regarding
category theory and inaccessible cardinals\index{category
theory}\index{cardinal, inaccessible} (p.~\pageref{categoryth}),
and a misleading statement was removed from the text.  Specifically,
contrary to a statement in previous editions, it is possible to express
``There is a proper class of inaccessible cardinals'' in the language of
ZFC.  This can be done as follows:  ``For every set $x$ there is an
inaccessible cardinal $\kappa$ such that $\kappa$ is not in $x$.''
Bob writes:\footnote{Private communication, Nov.~30, 2002.}
\begin{quotation}
     This axiom is how Grothendieck presents category theory.  To each
inaccessible cardinal $\kappa$ one associates a Grothendieck universe
\index{Grothendieck, Alexander} $U(\kappa)$.  $U(\kappa)$ consists of
those sets which lie in a transitive set of cardinality less than
$\kappa$.  Instead of the ``category of all groups,'' one works relative
to a universe [considering the category of groups of cardinality less
than $\kappa$].  Now the category whose objects are all categories
``relative to the universe $U(\kappa)$'' will be a category not
relative to this universe but to the next universe.

     All of the things category theorists like to do can be done in this
framework.  The only controversial point is whether the Grothen\-dieck
axiom is too strong for the needs of category theorists.  Mac Lane
\index{Mac Lane, Saunders} argues that ``one universe is enough'' and
Feferman\index{Feferman, Solomon} has argued that one can get by with
ordinary ZFC.  I don't find Feferman's arguments persuasive.  Mac Lane
may be right, but when I think about category theory I do it \`{a} la
Grothendieck.

        By the way Mizar\index{Mizar} adds the axiom ``there is a proper
class of inaccessibles'' precisely so as to do category theory.
\end{quotation}

The most current information on the Metamath program and databases can
always be found at \url{http://metamath.org}.


\subsubsection{Note Added June 24, 2006}\label{note2006}

The Metamath spec was restricted slightly to make parsers easier to
write.  See the footnote on p.~\pageref{namespace}.

%\subsubsection{Note Added July 24, 2006}\label{note2006b}
\subsubsection{Note Added March 10, 2007}\label{note2006b}

I am grateful to Anthony Williams\index{Williams, Anthony} for writing
the \LaTeX\ package called {\tt realref.sty} and contributing it to the
public domain.  This package allows the internal hyperlinks in a {\sc
pdf} file to anchor to specific page numbers instead of just section
titles, making the navigation of the {\sc pdf} file for this book much
more pleasant and ``logical.''

A typographical error found by Martin Kiselkov was corrected.
A confusing remark about unification was deleted per suggestion of
Mel O'Cat.

\subsubsection{Note Added May 27, 2009}\label{note2009}

Several typos found by Kim Sparre were corrected.  A note was added that
the Poincar\'{e} conjecture has been proved (p.~\pageref{poincare}).

\subsubsection{Note Added Nov. 17, 2014}\label{note2014}

The statement of the Schr\"{o}der--Bernstein theorem was corrected in
Section~\ref{trust}.  Thanks to Bob Solovay for pointing out the error.

\subsubsection{Note Added May 25, 2016}\label{note2016}

Thanks to Jerry James for correcting 16 typos.

\subsubsection{Note Added February 25, 2019}\label{note201902}

David A. Wheeler\index{Wheeler, David A.}
made a large number of improvements and updates,
in coordination with Norman Megill.
The predicate calculus axioms were renumbered, and the text makes
it clear that they are based on Tarski's system S2;
the one slight deviation in axiom ax-6 is explained and justified.
The real and complex number axioms were modified to be consistent with
\texttt{set.mm}\index{set theory database (\texttt{set.mm})}%
\index{Metamath Proof Explorer}.
Long-awaited specification changes ``1--8'' were made,
which clarified previously ambiguous points.
Some errors in the text involving \texttt{\$f} and
\texttt{\$d} statements were corrected (the spec was correct, but
the in-book explanations unintentionally contradicted the spec).
We now have a system for automatically generating narrow PDFs,
so that those with smartphones can have easy access to the current
version of this document.
A new section on deduction was added;
it discusses the standard deduction theorem,
the weak deduction theorem,
deduction style, and natural deduction.
Many minor corrections (too numerous to list here) were also made.

\subsubsection{Note Added March 7, 2019}\label{note201903}

This added a description of the Matamath language syntax in
Extended Backus--Naur Form (EBNF)\index{Extended Backus--Naur Form}\index{EBNF}
in Appendix \ref{BNF}, added a brief explanation about typecodes,
inserted more examples in the deduction section,
and added a variety of smaller improvements.

\subsubsection{Note Added April 7, 2019}\label{note201904}

This version clarified the proper substitution notation, improved the
discussion on the weak deduction theorem and natural deduction,
documented the \texttt{undo} command, updated the information on
\texttt{write source}, changed the typecode
from \texttt{set} to \texttt{setvar} to be consistent with the current
version of \texttt{set.mm}, added more documentation about comment markup
(e.g., documented how to create headings), and clarified the
differences between various assertion forms (in particular deduction form).

\subsubsection{Note Added June 2, 2019}\label{note201906}

This version fixes a large number of small issues reported by
Beno\^{i}t Jubin\index{Jubin, Beno\^{i}t}, such as editorial issues
and the need to document \texttt{verify markup} (thank you!).
This version also includes specific examples
of forms (deduction form, inference form, and closed form).
We call this version the ``second edition'';
the previous edition formally published in 2007 had a slightly different title
(\textit{Metamath: A Computer Language for Pure Mathematics}).

\chapter{Introduction}
\pagenumbering{arabic}

\begin{quotation}
  {\em {\em I.M.:}  No, no.  There's nothing subjective about it!  Everybody
knows what a proof is.  Just read some books, take courses from a competent
mathematician, and you'll catch on.

{\em Student:}  Are you sure?

{\em I.M.:}  Well---it is possible that you won't, if you don't have any
aptitude for it.  That can happen, too.

{\em Student:}  Then {\em you} decide what a proof is, and if I don't learn
to decide in the same way, you decide I don't have any aptitude.

{\em I.M.:}  If not me, then who?}
    \flushright\sc  ``The Ideal Mathematician''
    \index{Davis, Phillip J.}
    \footnote{\cite{Davis}, p.~40.}\\
\end{quotation}

Brilliant mathematicians have discovered almost
unimaginably profound results that rank among the crowning intellectual
achievements of mankind.  However, there is a sense in which modern abstract
mathematics is behind the times, stuck in an era before computers existed.
While no one disputes the remarkable results that have been achieved,
communicating these results in a precise way to the uninitiated is virtually
impossible.  To describe these results, a terse informal language is used which
despite its elegance is very difficult to learn.  This informal language is not
imprecise, far from it, but rather it often has omitted detail
and symbols with hidden context that are
implicitly understood by an expert but few others.  Extremely complex technical
meanings are associated with innocent-sounding English words such as
``compact'' and ``measurable'' that barely hint at what is actually being
said.  Anyone who does not keep the precise technical meaning constantly in
mind is bound to fail, and acquiring the ability to do this can be achieved
only through much practice and hard work.  Only the few who complete the
painful learning experience can join the small in-group of pure
mathematicians.  The informal language effectively cuts off the true nature of
their knowledge from most everyone else.

Metamath\index{Metamath} makes abstract mathematics more concrete.  It allows
a computer to keep track of the complexity associated with each word or symbol
with absolute rigor.  You can explore this complexity at your leisure, to
whatever degree you desire.  Whether or not you believe that concepts such as
infinity actually ``exist'' outside of the mind, Metamath lets you get to the
foundation for what's really being said.

Metamath also enables completely rigorous and thorough proof verification.
Its language is simple enough so that you
don't have to rely on the authority of experts but can verify the results
yourself, step by step.  If you want to attempt to derive your own results,
Metamath will not let you make a mistake in reasoning.
Even professional mathematicians make mistakes; Metamath makes it possible
to thoroughly verify that proofs are correct.

Metamath\index{Metamath} is a computer language and an associated computer
program for archiving, verifying, and studying mathematical proofs at a very
detailed level.
The Metamath language
describes formal\index{formal system} mathematical
systems and expresses proofs of theorems in those systems.  Such a language
is called a metalanguage\index{metalanguage} by mathematicians.
The Metamath program is a computer program that verifies
proofs expressed in the Metamath language.
The Metamath program does not have the built-in
ability to make logical inferences; it just makes a series of symbol
substitutions according to instructions given to it in a proof
and verifies that the result matches the expected theorem.  It makes logical
inferences based only on rules of logic that are contained in a set of
axioms\index{axiom}, or first principles, that you provide to it as the
starting point for proofs.

The complete specification of the Metamath language is only four pages long
(Section~\ref{spec}, p.~\pageref{spec}).  Its simplicity may at first make you
wonder how it can do much of anything at all.  But in fact the kinds of
symbol manipulations it performs are the ones that are implicitly done in all
mathematical systems at the lowest level.  You can learn it relatively quickly
and have complete confidence in any mathematical proof that it verifies.  On
the other hand, it is powerful and general enough so that virtually any
mathematical theory, from the most basic to the deeply abstract, can be
described with it.

Although in principle Metamath can be used with any
kind of mathematics, it is best suited for abstract or ``pure'' mathematics
that is mostly concerned with theorems and their proofs, as opposed to the
kind of mathematics that deals with the practical manipulation of numbers.
Examples of branches of pure mathematics are logic\index{logic},\footnote{Logic
is the study of statements that are universally true regardless of the objects
being described by the statements.  An example is the statement, ``if $P$
implies $Q$, then either $P$ is false or $Q$ is true.''} set theory\index{set
theory},\footnote{Set theory is the study of general-purpose mathematical objects called
``sets,'' and from it essentially all of mathematics can be derived.  For
example, numbers can be defined as specific sets, and their properties
can be explored using the tools of set theory.} number theory\index{number
theory},\footnote{Number theory deals with the properties of positive and
negative integers (whole numbers).} group theory\index{group
theory},\footnote{Group theory studies the properties of mathematical objects
called groups that obey a simple set of axioms and have properties of symmetry
that make them useful in many other fields.} abstract algebra\index{abstract
algebra},\footnote{Abstract algebra includes group theory and also studies
groups with additional properties that qualify them as ``rings'' and
``fields.''  The set of real numbers is a familiar example of a field.},
analysis\index{analysis} \index{real and complex numbers}\footnote{Analysis is
the study of real and complex numbers.} and
topology\index{topology}.\footnote{One area studied by topology are properties
that remain unchanged when geometrical objects undergo stretching
deformations; for example a doughnut and a coffee cup each have one hole (the
cup's hole is in its handle) and are thus considered topologically
equivalent.  In general, though, topology is the study of abstract
mathematical objects that obey a certain (surprisingly simple) set of axioms.
See, for example, Munkres \cite{Munkres}\index{Munkres, James R.}.} Even in
physics, Metamath could be applied to certain branches that make use of
abstract mathematics, such as quantum logic\index{quantum logic} (used to study
aspects of quantum mechanics\index{quantum mechanics}).

On the other hand, Metamath\index{Metamath} is less suited to applications
that deal primarily with intensive numeric computations.  Metamath does not
have any built-in representation of numbers\index{Metamath!representation of
numbers}; instead, a specific string of symbols (digits) must be syntactically
constructed as part of any proof in which an ordinary number is used.  For
this reason, numbers in Metamath are best limited to specific constants that
arise during the course of a theorem or its proof.  Numbers are only a tiny
part of the world of abstract mathematics.  The exclusion of built-in numbers
was a conscious decision to help achieve Metamath's simplicity, and there are
other software tools if you have different mathematical needs.
If you wish to quickly solve algebraic problems, the computer algebra
programs\index{computer algebra system} {\sc
macsyma}\index{macsyma@{\sc macsyma}}, Mathematica\index{Mathematica}, and
Maple\index{Maple} are specifically suited to handling numbers and
algebra efficiently.
If you wish to simply calculate numeric or matrix expressions easily,
tools such as Octave\index{Octave} may be a better choice.

After learning Metamath's basic statement types, any
tech\-ni\-cal\-ly ori\-ent\-ed person, mathematician or not, can
immediately trace
any theorem proved in the language as far back as he or she wants, all the way
to the axioms on which the theorem is based.  This ability suggests a
non-traditional way of learning about pure mathematics.  Used in conjunction
with traditional methods, Metamath could make pure mathematics accessible to
people who are not sufficiently skilled to figure out the implicit detail in
ordinary textbook proofs.  Once you learn the axioms of a theory, you can have
complete confidence that everything you need to understand a proof you are
studying is all there, at your beck and call, allowing you to focus in on any
proof step you don't understand in as much depth as you need, without worrying
about getting stuck on a step you can't figure out.\footnote{On the other
hand, writing proofs in the Metamath language is challenging, requiring
a degree of rigor far in excess of that normally taught to students.  In a
classroom setting, I doubt that writing Metamath proofs would ever replace
traditional homework exercises involving informal proofs, because the time
needed to work out the details would not allow a course to
cover much material.  For students who have trouble grasping the implied rigor
in traditional material, writing a few simple proofs in the Metamath language
might help clarify fuzzy thought processes.  Although somewhat difficult at
first, it eventually becomes fun to do, like solving a puzzle, because of the
instant feedback provided by the computer.}

Metamath\index{Metamath} is probably unlike anything you have
encountered before.  In this first chapter we will look at the philosophy and
use of computers in mathematics in order to better understand the motivation
behind Metamath.  The material in this chapter is not required in order to use
Metamath.  You may skip it if you are impatient, but I hope you will find it
educational and enjoyable.  If you want to start experimenting with the
Metamath program right away, proceed directly to Chapter~\ref{using}
(p.~\pageref{using}).  To
learn the Metamath language, skim Chapter~\ref{using} then proceed to
Chapter~\ref{languagespec} (p.~\pageref{languagespec}).

\section{Mathematics as a Computer Language}

\begin{quote}
  {\em The study of mathematics is apt to commence in
dis\-ap\-point\-ment.\ldots \\
We are told that by its aid the stars are weighted
and the billions of molecules in a drop of water are counted.  Yet, like the
ghost of Hamlet's father, this great science eludes the efforts of our mental
weapons to grasp it.}
  \flushright\sc  Alfred North Whitehead\footnote{\cite{Whitehead}, ch.\ 1.}\\
\end{quote}\index{Whitehead, Alfred North}

\subsection{Is Mathematics ``User-Friendly''?}

Suppose you have no formal training in abstract mathematics.  But popular
books you've read offer tempting glimpses of this world filled with profound
ideas that have stirred the human spirit.  You are not satisfied with the
informal, watered-down descriptions you've read but feel it is important to
grasp the underlying mathematics itself to understand its true meaning. It's
not practical to go back to school to learn it, though; you don't want to
dedicate years of your life to it.  There are many important things in life,
and you have to set priorities for what's important to you.  What would happen
if you tried to pursue it on your own, in your spare time?

After all, you were able to learn a computer programming language such as
Pascal on your own without too much difficulty, even though you had no formal
training in computers.  You don't claim to be an expert in software design,
but you can write a passable program when necessary to suit your needs.  Even
more important, you know that you can look at anyone else's Pascal program, no
matter how complex, and with enough patience figure out exactly how it works,
even though you are not a specialist.  Pascal allows you do anything that a
computer can do, at least in principle.  Thus you know you have the ability,
in principle, to follow anything that a computer program can do:  you just
have to break it down into small enough pieces.

Here's an imaginary scenario of what might happen if you na\-ive\-ly a\-dopted
this same view of abstract mathematics and tried to pick it up on your own, in
a period of time comparable to, say, learning a computer programming
language.

\subsubsection{A Non-Mathematician's Quest for Truth}

\begin{quote}
  {\em \ldots my daughters have been studying (chemistry) for several
se\-mes\-ters, think they have learned differential and integral calculus in
school, and yet even today don't know why $x\cdot y=y\cdot x$ is true.}
  \flushright\sc  Edmund Landau\footnote{\cite{Landau}, p.~vi.}\\
\end{quote}\index{Landau, Edmund}

\begin{quote}
  {\em Minus times minus is plus,\\
The reason for this we need not discuss.}
  \flushright\sc W.\ H.\ Auden\footnote{As quoted in \cite{Guillen}, p.~64.}\\
\end{quote}\index{Auden, W.\ H.}\index{Guillen, Michael}

We'll suppose you are a technically oriented professional, perhaps an engineer, a
computer programmer, or a physicist, but probably not a mathematician.  You
consider yourself reasonably intelligent.  You did well in school, learning a
variety of methods and techniques in practical mathematics such as calculus and
differential equations.  But rarely did your courses get into anything
resembling modern abstract mathematics, and proofs were something that appeared
only occasionally in your textbooks, a kind of necessary evil that was
supposed to convince you of a certain key result.  Most of your
homework consisted of exercises that gave you practice in the techniques, and
you were hardly ever asked to come up with a proof of your own.

You find yourself curious about advanced, abstract mathematics.  You are
driven by an inner conviction that it is important to understand and
appreciate some of the most profound knowledge discovered by mankind.  But it
seems very hard to learn, something that only certain gifted longhairs can
access and understand.  You are frustrated that it seems forever cut off from
you.

Eventually your curiosity drives you to do something about it.
You set for yourself a goal of ``really'' understanding mathematics:  not just
how to manipulate equations in algebra or calculus according to cookbook
rules, but rather to gain a deep understanding of where those rules come from.
In fact, you're not thinking about this kind of ordinary mathematics at all,
but about a much more abstract, ethereal realm of pure mathematics, where
famous results such as G\"{o}del's incompleteness theorem\index{G\"{o}del's
incompleteness theorem} and Cantor's different kinds of infinities
reside.

You have probably read a number of popular books, with titles like {\em
Infinity and the Mind} \cite{Rucker}\index{Rucker, Rudy}, on topics such as
these.  You found them inspiring but at the same time somewhat
unsatisfactory.  They gave you a general idea of what these results are about,
but if someone asked you to prove them, you wouldn't have the faintest idea of
where to begin.   Sure, you could give the same overall outline that you
learned from the popular books; and in a general sort of way, you do have an
understanding.  But deep down inside, you know that there is a rigor that is
missing, that probably there are many subtle steps and pitfalls along the way,
and ultimately it seems you have to place your trust in the experts in the
field.  You don't like this; you want to be able to verify these results for
yourself.

So where do you go next?  As a first step, you decide to look up some of the
original papers on the theorems you are curious about, or better, obtain some
standard textbooks in the field.  You look up a theorem you want to
understand.  Sure enough, it's there, but it's expressed with strange
terms and odd symbols that mean absolutely nothing to you.  It might as well be written in
a foreign language you've never seen before, whose symbols are totally alien.
You look at the proof, and you haven't the foggiest notion what each step
means, much less how one step follows from another.  Well, obviously you have
a lot to learn if you want to understand this stuff.

You feel that you could probably understand it by
going back to college for another three to six years and getting a math
degree.  But that does not fit in with your career and the other things in
your life and would serve no practical purpose.  You decide to seek a quicker
path.  You figure you'll just trace your way back to the beginning, step by
step, as you would do with a computer program, until you understand it.  But
you quickly find that this is not possible, since you can't even understand
enough to know what you have to trace back to.

Maybe a different approach is in order---maybe you should start at the
beginning and work your way up.  First, you read the introduction to the book
to find out what the prerequisites are.  In a similar fashion, you trace your
way back through two or three more books, finally arriving at one that seems
to start at a beginning:  it lists the axioms of arithmetic.  ``Aha!'' you
naively think, ``This must be the starting point, the source of all mathematical
knowledge.'' Or at least the starting point for mathematics dealing with
numbers; you have to start somewhere and have no idea what the starting point
for other mathematics would be.  But the word ``axioms'' looks promising.  So
you eagerly read along and work through some elementary exercises at the
beginning of the book.  You feel vaguely bothered:  these
don't seem like axioms at all, at least not in the sense that you want to
think of axioms.  Axioms imply a starting point from which everything else can
be built up, according to precise rules specified in the axiom system.  Even
though you can understand the first few proofs in an informal way,
and are able to do some of the
exercises, it's hard to pin down precisely what the
rules are.   Sure, each step seems to follow logically from the others, but
exactly what does that mean?  Is the ``logic'' just a matter of common sense,
something vague that we all understand but can never quite state precisely?

You've spent a number of years, off and on, programming computers, and you
know that in the case of computer languages there is no question of what the
rules are---they are precise and crystal clear.  If you follow them, your
program will work, and if you don't, it won't.  No matter how complex a
program, it can always be broken down into simpler and simpler pieces, until
you can ultimately identify the bits that are moved around to perform a
specific function.  Some programs might require a lot of perseverance to
accomplish this, but if you focus on a specific portion of it, you don't even
necessarily have to know how the rest of it works. Shouldn't there be an
analogy in mathematics?

You decide to apply the ultimate test:  you ask yourself how a computer could
verify or ensure that the steps in these proofs follow from one another.
Certainly mathematics must be at least as precisely defined as a computer
language, if not more so; after all, computer science itself is based on it.
If you can get a computer to verify these proofs, then you should also be
able, in principle, to understand them yourself in a very crystal clear,
precise way.

You're in for a surprise:  you can conceive of no way to convert the
proofs, which are in English, to a form that the computer can understand.
The proofs are filled with phrases such as ``assume there exists a unique
$x$\ldots'' and ``given any $y$, let $z$ be the number such that\ldots''  This
isn't the kind of logic you are used to in computer programming, where
everything, even arithmetic, reduces to Boolean ones and zeroes if you care to
break it down sufficiently.  Even though you think you understand the proofs,
there seems to be some kind of higher reasoning involved rather than precise
rules that define how you manipulate the symbols in the axioms.  Whatever it
is, it just isn't obvious how you would express it to a computer, and the more
you think about it, the more puzzled and confused you get, to the point where
you even wonder whether {\em you} really understand it.  There's a lot more to
these axioms of arithmetic than meets the eye.

Nobody ever talked about this in school in your applied math and engineering
courses.  You just learned the rules they gave you, not quite understanding
how or why they worked, sometimes vaguely suspicious or uncertain of them, and
through homework problems and osmosis learned how to present solutions that
satisfied the instructor and earned you an ``A.''  Rarely did you actually
``prove'' anything in a rigorous way, and the math majors who did do stuff
like that seemed to be in a different world.

Of course, there are computer algebra programs that can do mathematics, and
rather impressively.  They can instantly solve the integrals that you
struggled with in freshman calculus, and do much, much more.  But when you
look at these programs, what you see is a big collection of algorithms and
techniques that evolved and were added to over time, along with some basic
software that manipulates symbols.  Each algorithm that is built in is the
result of someone's theorem whose proof is omitted; you just have to trust the
person who proved it and the person who programmed it in and hope there are no
bugs.\index{computer program bugs}  Somehow this doesn't seem to be the
essence of mathematics.  Although computer algebra systems can generate
theorems with amazing speed, they can't actually prove a single one of them.

After some puzzlement, you revisit some popular books on what mathematics is
all about.  Somewhere you read that all of mathematics is actually derived
from something called ``set theory.''  This is a little confusing, because
nowhere in the book that presented the axioms of arithmetic was there any
mention of set theory, or if there was, it seemed to be just a tool that helps
you describe things better---the set of even numbers, that sort of thing.  If
set theory is the basis for all mathematics, then why are additional axioms
needed for arithmetic?

Something is wrong but you're not sure what.  One of your friends is a pure
mathematician.  He knows he is unable to communicate to you what he does for a
living and seems to have little interest in trying.  You do know that for him,
proofs are what mathematics is all about. You ask him what a proof is, and he
essentially tells you that, while of course it's based on logic, really it's
something you learn by doing it over and over until you pick it up.  He refers
you to a book, {\em How to Read and Do Proofs} \cite{Solow}.\index{Solow,
Daniel}  Although this book helps you understand traditional informal proofs,
there is still something missing you can't seem to pin down yet.

You ask your friend how you would go about having a computer verify a proof.
At first he seems puzzled by the question; why would you want to do that?
Then he says it's not something that would make any sense to do, but he's
heard that you'd have to break the proof down into thousands or even millions
of individual steps to do such a thing, because the reasoning involved is at
such a high level of abstraction.  He says that maybe it's something you could
do up to a point, but the computer would be completely impractical once you
get into any meaningful mathematics.  There, the only way you can verify a
proof is by hand, and you can only acquire the ability to do this by
specializing in the field for a couple of years in grad school.  Anyway, he
thinks it all has to do with set theory, although he has never taken a formal
course in set theory but just learned what he needed as he went along.

You are intrigued and amazed.  Apparently a mathematician can grasp as a
single concept something that would take a computer a thousand or a million
steps to verify, and have complete confidence in it.  Each one of these
thousand or million steps must be absolutely correct, or else the whole proof
is meaningless.  If you added a million numbers by hand, would you trust the
result?  How do you really know that all these steps are correct, that there
isn't some subtle pitfall in one of these million steps, like a bug in a
computer program?\index{computer program bugs}  After all, you've read that
famous mathematicians have occasionally made mistakes, and you certainly know
you've made your share on your math homework problems in school.

You recall the analogy with a computer program.  Sure, you can understand what
a large computer program such as a word processor does, as a single high-level
concept or a small set of such concepts, but your ability to understand it in
no way ensures that the program is correct and doesn't have hidden bugs.  Even
if you wrote the program yourself you can't really know this; most large
programs that you've written have had bugs that crop up at some later date, no
matter how careful you tried to be while writing them.

OK, so now it seems the reason you can't figure out how to make a
computer verify proofs is because each step really corresponds to a
million small steps.  Well, you say, a computer can do a million
calculations in a second, so maybe it's still practical to do.  Now the
puzzle becomes how to figure out what the million steps are that each
English-language step corresponds to.  Your mathematician friend hasn't
a clue, but suggests that maybe you would find the answer by studying
set theory.  Actually, your friend thinks you're a little off the wall
for even wondering such a thing.  For him, this is not what mathematics
is all about.

The subject of set theory keeps popping up, so you decide it's
time to look it up.

You decide to start off on a careful footing, so you start reading a couple of
very elementary books on set theory.  A lot of it seems pretty obvious, like
intersections, subsets, and Venn diagrams.  You thumb through one of the
books; nowhere is anything about axioms mentioned. The other book relegates to
an appendix a brief discussion that mentions a set of axioms called
``Zermelo--Fraenkel set theory''\index{Zermelo--Fraenkel set theory} and states
them in English.  You look at them and have no idea what they really mean or
what you can do with them.  The comments in this appendix say that the purpose
of mentioning them is to expose you to the idea, but imply that they are not
necessary for basic understanding and that they are really the subject matter
of advanced treatments where fine points such as a certain paradox (Russell's
paradox\index{Russell's paradox}\footnote{Russell's paradox assumes that there
exists a set $S$ that is a collection of all sets that don't contain
themselves.  Now, either $S$ contains itself or it doesn't.  If it contains
itself, it contradicts its definition.  But if it doesn't contain itself, it
also contradicts its definition.  Russell's paradox is resolved in ZF set
theory by denying that such a set $S$ exists.}) are resolved.  Wait a
minute---shouldn't the axioms be a starting point, not an ending point?  If
there are paradoxes that arise without the axioms, how do you know you won't
stumble across one accidentally when using the informal approach?

And nowhere do these books describe how ``all of mathematics can be
derived from set theory'' which by now you've heard a few times.

You find a more advanced book on set theory.  This one actually lists the
axioms of ZF set theory in plain English on page one.  {\em Now} you think
your quest has ended and you've finally found the source of all mathematical
knowledge; you just have to understand what it means.  Here, in one place, is
the basis for all of mathematics!  You stare at the axioms in awe, puzzle over
them, memorize them, hoping that if you just meditate on them long enough they
will become clear.  Of course, you haven't the slightest idea how the rest of
mathematics is ``derived'' from them; in particular, if these are the axioms
of mathematics, then why do arithmetic, group theory, and so on need their own
axioms?

You start reading this advanced book carefully, pondering the meaning of every
word, because by now you're really determined to get to the bottom of this.
The first thing the book does is explain how the axioms came about, which was
to resolve Russell's paradox.\index{Russell's paradox}  In fact that seems to
be the main purpose of their existence; that they supposedly can be used to
derive all of mathematics seems irrelevant and is not even mentioned.  Well,
you go on.  You hope the book will explain to you clearly, step by step, how
to derive things from the axioms.  After all, this is the starting point of
mathematics, like a book that explains the basics of a computer programming
language.  But something is missing.  You find you can't even understand the
first proof or do the first exercise.  Symbols such as $\exists$ and $\forall$
permeate the page without any mention of where they came from or how to
manipulate them; the author assumes you are totally familiar with them and
doesn't even tell you what they mean.  By now you know that $\exists$ means
``there exists'' and $\forall$ means ``for all,'' but shouldn't the rules for
manipulating these symbols be part of the axioms?  You still have no idea
how you could even describe the axioms to a computer.

Certainly there is something much different here from the technical
literature you're used to reading.  A computer language manual almost
always explains very clearly what all the symbols mean, precisely what
they do, and the rules used for combining them, and you work your way up
from there.

After glancing at four or five other such books, you come to the realization
that there is another whole field of study that you need just to get to the
point at which you can understand the axioms of set theory.  The field is
called ``logic.''  In fact, some of the books did recommend it as a
prerequisite, but it just didn't sink in.  You assumed logic was, well, just
logic, something that a person with common sense intuitively understood.  Why
waste your time reading boring treatises on symbolic logic, the manipulation
of 1's and 0's that computers do, when you already know that?  But this is a
different kind of logic, quite alien to you.  The subject of {\sc nand} and
{\sc nor} gates is not even touched upon or in any case has to do with only a
very small part of this field.

So your quest continues.  Skimming through the first couple of introductory
books, you get a general idea of what logic is about and what quantifiers
(``for all,'' ``there exists'') mean, but you find their examples somewhat
trivial and mildly annoying (``all dogs are animals,'' ``some animals are
dogs,'' and such).  But all you want to know is what the rules are for
manipulating the symbols so you can apply them to set theory.  Some formulas
describing the relationships among quantifiers ($\exists$ and $\forall$) are
listed in tables, along with some verbal reasoning to justify them.
Presumably, if you want to find out if a formula is correct, you go through
this same kind of mental reasoning process, possibly using images of dogs and
animals. Intuitively, the formulas seem to make sense.  But when you ask
yourself, ``What are the rules I need to get a computer to figure out whether
this formula is correct?'', you still don't know.  Certainly you don't ask the
computer to imagine dogs and animals.

You look at some more advanced logic books.  Many of them have an introductory
chapter summarizing set theory, which turns out to be a prerequisite.  You
need logic to understand set theory, but it seems you also need set theory to
understand logic!  These books jump right into proving rather advanced
theorems about logic, without offering the faintest clue about where the logic
came from that allows them to prove these theorems.

Luckily, you come across an elementary book of logic that, halfway through,
after the usual truth tables and metaphors, presents in a clear, precise way
what you've been looking for all along: the axioms!  They're divided into
propositional calculus (also called sentential logic) and predicate calculus
(also called first-order logic),\index{first-order logic} with rules so simple
and crystal clear that now you can finally program a computer to understand
them.  Indeed, they're no harder than learning how to play a game of chess.
As far as what you seem to need is concerned, the whole book could have been
written in five pages!

{\em Now} you think you've found the ultimate source of mathematical
truth.  So---the axioms of mathematics consist of these axioms of logic,
together with the axioms of ZF set theory. (By now you've also been able to
figure out how to translate the ZF axioms from English into the
actual symbols of logic which you can now manipulate according to
precise, easy-to-understand rules.)

Of course, you still don't understand how ``all of mathematics can be
derived from set theory,'' but maybe this will reveal itself in due
course.

You eagerly set out to program the axioms and rules into a computer and start
to look at the theorems you will have to prove as the logic is developed.  All
sorts of important theorems start popping up:  the deduction
theorem,\index{deduction theorem} the substitution theorem,\index{substitution
theorem} the completeness theorem of propositional calculus,\index{first-order
logic!completeness} the completeness theorem of predicate calculus.  Uh-oh,
there seems to be trouble.  They all get harder and harder, and not one of
them can be derived with the axioms and rules of logic you've just been
handed.  Instead, they all require ``metalogic'' for their proofs, a kind of
mixture of logic and set theory that allows you to prove things {\em about}
the axioms and theorems of logic rather than {\em with} them.

You plow ahead anyway.  A month later, you've spent much of your
free time getting the computer to verify proofs in propositional calculus.
You've programmed in the axioms, but you've also had to program in the
deduction theorem, the substitution theorem, and the completeness theorem of
propositional calculus, which by now you've resigned yourself to treating as
rather complex additional axioms, since they can't be proved from the axioms
you were given.  You can now get the computer to verify and even generate
complete, rigorous, formal proofs\index{formal proof}.  Never mind that they
may have 100,000 steps---at least now you can have complete, absolute
confidence in them.  Unfortunately, the only theorems you have proved are
pretty trivial and you can easily verify them in a few minutes with truth
tables, if not by inspection.

It looks like your mathematician friend was right.  Getting the computer to do
serious mathematics with this kind of rigor seems almost hopeless.  Even
worse, it seems that the further along you get, the more ``axioms'' you have
to add, as each new theorem seems to involve additional ``metamathematical''
reasoning that hasn't been formalized, and none of it can be derived from the
axioms of logic.  Not only do the proofs keep growing exponentially as you get
further along, but the program to verify them keeps getting bigger and bigger
as you program in more ``metatheorems.''\index{metatheorem}\footnote{A
metatheorem is usually a statement that is too general to be directly provable
in a theory.  For example, ``if $n_1$, $n_2$, and $n_3$ are integers, then
$n_1+n_2+n_3$ is an integer'' is a theorem of number theory.  But ``for any
integer $k > 1$, if $n_1, \ldots, n_k$ are integers, then $n_1+\ldots +n_k$ is
an integer'' is a metatheorem, in other words a family of theorems, one for
every $k$.  The reason it is not a theorem is that the general sum $n_1+\ldots
+n_k$ (as a function of $k$) is not an operation that can be defined directly
in number theory.} The bugs\index{computer program bugs} that have cropped up
so far have already made you start to lose faith in the rigor you seem to have
achieved, and you know it's just going to get worse as your program gets larger.

\subsection{Mathematics and the Non-Specialist}

\begin{quote}
  {\em A real proof is not checkable by a machine, or even by any mathematician
not privy to the gestalt, the mode of thought of the particular field of
mathematics in which the proof is located.}
  \flushright\sc  Davis and Hersh\index{Davis, Phillip J.}
  \footnote{\cite{Davis}, p.~354.}\\
\end{quote}

The bulk of abstract or theoretical mathematics is ordinarily outside
the reach of anyone but a few specialists in each field who have completed
the necessary difficult internship in order to enter its coterie.  The
typical intelligent layperson has no reasonable hope of understanding much of
it, nor even the specialist mathematician of understanding other fields.  It
is like a foreign language that has no dictionary to look up the translation;
the only way you can learn it is by living in the country for a few years.  It
is argued that the effort involved in learning a specialty is a necessary
process for acquiring a deep understanding.  Of course, this is almost certainly
true if one is to make significant contributions to a field; in particular,
``doing'' proofs is probably the most important part of a mathematician's
training.  But is it also necessary to deny outsiders access to it?  Is it
necessary that abstract mathematics be so hard for a layperson to grasp?

A computer normally is of no help whatsoever.  Most published proofs are
actually just series of hints written in an informal style that requires
considerable knowledge of the field to understand.  These are the ``real
proofs'' referred to by Davis and Hersh.\index{informal proof}  There is an
implicit understanding that, in principle, such a proof could be converted to
a complete formal proof\index{formal proof}.  However, it is said that no one
would ever attempt such a conversion, even if they could, because that would
presumably require millions of steps (Section~\ref{dream}).  Unfortunately the
informal style automatically excludes the understanding of the proof
by anyone who hasn't gone through the necessary apprenticeship. The
best that the intelligent layperson can do is to read popular books about deep
and famous results; while this can be helpful, it can also be misleading, and
the lack of detail usually leaves the reader with no ability whatsoever to
explore any aspect of the field being described.

The statements of theorems often use sophisticated notation that makes them
inaccessible to the non-specialist.  For a non-specialist who wants to achieve
a deeper understanding of a proof, the process of tracing definitions and
lemmas back through their hierarchy\index{hierarchy} quickly becomes confusing
and discouraging.  Textbooks are usually written to train mathematicians or to
communicate to people who are already mathematicians, and large gaps in proofs
are often left as exercises to the reader who is left at an impasse if he or
she becomes stuck.

I believe that eventually computers will enable non-specialists and even
intelligent laypersons to follow almost any mathematical proof in any field.
Metamath is an attempt in that direction.  If all of mathematics were as
easily accessible as a computer programming language, I could envision
computer programmers and hobbyists who otherwise lack mathematical
sophistication exploring and being amazed by the world of theorems and proofs
in obscure specialties, perhaps even coming up with results of their own.  A
tremendous advantage would be that anyone could experiment with conjectures in
any field---the computer would offer instant feedback as to whether
an inference step was correct.

Mathematicians sometimes have to put up with the annoyance of
cranks\index{cranks} who lack a fundamental understanding of mathematics but
insist that their ``proofs'' of, say, Fermat's Last Theorem\index{Fermat's
Last Theorem} be taken seriously.  I think part of the problem is that these
people are misled by informal mathematical language, treating it as if they
were reading ordinary expository English and failing to appreciate the
implicit underlying rigor.  Such cranks are rare in the field of computers,
because computer languages are much more explicit, and ultimately the proof is
in whether a computer program works or not.  With easily accessible
computer-based abstract mathematics, a mathematician could say to a crank,
``don't bother me until you've demonstrated your claim on the computer!''

% 22-May-04 nm
% Attempt to move De Millo quote so it doesn't separate from attribution
% CHANGE THIS NUMBER (AND ELIMINATE IF POSSIBLE) WHEN ABOVE TEXT CHANGES
\vspace{-0.5em}

\subsection{An Impossible Dream?}\label{dream}

\begin{quote}
  {\em Even quite basic theorems would demand almost unbelievably vast
  books to display their proofs.}
    \flushright\sc  Robert E. Edwards\footnote{\cite{Edwards}, p.~68.}\\
\end{quote}\index{Edwards, Robert E.}

\begin{quote}
  {\em Oh, of course no one ever really {\em does} it.  It would take
  forever!  You just show that you could do it, that's sufficient.}
    \flushright\sc  ``The Ideal Mathematician''
    \index{Davis, Phillip J.}\footnote{\cite{Davis},
p.~40.}\\
\end{quote}

\begin{quote}
  {\em There is a theorem in the primitive notation of set theory that
  corresponds to the arithmetic theorem `$1000+2000=3000$'.  The formula
  would be forbiddingly long\ldots even if [one] knows the definitions
  and is asked to simplify the long formula according to them, chances are
  he will make errors and arrive at some incorrect result.}
    \flushright\sc  Hao Wang\footnote{\cite{Wang}, p.~140.}\\
\end{quote}\index{Wang, Hao}

% 22-May-04 nm
% Attempt to move De Millo quote so it doesn't separate from attribution
% CHANGE THIS NUMBER (AND ELIMINATE IF POSSIBLE) WHEN ABOVE TEXT CHANGES
\vspace{-0.5em}

\begin{quote}
  {\em The {\em Principia Mathematica} was the crowning achievement of the
  formalists.  It was also the deathblow of the formalist view.\ldots
  {[Rus\-sell]} failed, in three enormous volumes, to get beyond the elementary
  facts of arithmetic.  He showed what can be done in principle and what
  cannot be done in practice.  If the mathematical process were really
  one of strict, logical progression, we would still be counting our
  fingers.\ldots
  One theoretician estimates\ldots that a demonstration of one of
  Ramanujan's conjectures assuming set theory and elementary analysis would
  take about two thousand pages; the length of a deduction from first principles
  is nearly in\-con\-ceiv\-a\-ble\ldots The probabilists argue that\ldots any
  very long proof can at best be viewed as only probably correct\ldots}
  \flushright\sc Richard de Millo et. al.\footnote{\cite{deMillo}, pp.~269,
  271.}\\
\end{quote}\index{de Millo, Richard}

A number of writers have conveyed the impression that the kind of absolute
rigor provided by Metamath\index{Metamath} is an impossible dream, suggesting
that a complete, formal verification\index{formal proof} of a typical theorem
would take millions of steps in untold volumes of books.  Even if it could be
done, the thinking sometimes goes, all meaning would be lost in such a
monstrous, tedious verification.\index{informal proof}\index{proof length}

These writers assume, however, that in order to achieve the kind of complete
formal verification they desire one must break down a proof into individual
primitive steps that make direct reference to the axioms.  This is
not necessary.  There is no reason not to make use of previously proved
theorems rather than proving them over and over.

Just as important, definitions\index{definition} can be introduced along
the way, allowing very complex formulas to be represented with few
symbols.  Not doing this can lead to absurdly long formulas.  For
example, the mere statement of
G\"{o}del's incompleteness theorem\index{G\"{o}del's
incompleteness theorem}, which can be expressed with a small number of
defined symbols, would require about 20,000 primitive symbols to express
it.\index{Boolos, George S.}\footnote{George S.\ Boolos, lecture at
Massachusetts Institute of Technology, spring 1990.} An extreme example
is Bourbaki's\label{bourbaki} language for set theory, which requires
4,523,659,424,929 symbols plus 1,179,618,517,981 disambiguatory links
(lines connecting symbol pairs, usually drawn below or above the
formula) to express the number
``one'' \cite{Mathias}.\index{Mathias, Adrian R. D.}\index{Bourbaki,
Nicolas}
% http://www.dpmms.cam.ac.uk/~ardm/

A hierarchy\index{hierarchy} of theorems and definitions permits an
exponential growth in the formula sizes and primitive proof steps to be
described with only a linear growth in the number of symbols used.  Of course,
this is how ordinary informal mathematics is normally done anyway, but with
Metamath\index{Metamath} it can be done with absolute rigor and precision.

\subsection{Beauty}


\begin{quote}
  {\em No one shall be able to drive us from the paradise that Cantor has
created for us.}
   \flushright\sc  David Hilbert\footnote{As quoted in \cite{Moore}, p.~131.}\\
\end{quote}\index{Hilbert, David}

\needspace{3\baselineskip}
\begin{quote}
  {\em Mathematics possesses not only truth, but some supreme beauty ---a
  beauty cold and austere, like that of a sculpture.}
    \flushright\sc  Bertrand
    Russell\footnote{\cite{Russell}.}\\
\end{quote}\index{Russell, Bertrand}

\begin{quote}
  {\em Euclid alone has looked on Beauty bare.}
  \flushright\sc Edna Millay\footnote{As quoted in \cite{Davis}, p.~150.}\\
\end{quote}\index{Millay, Edna}

For most people, abstract mathematics is distant, strange, and
incomprehensible.  Many popular books have tried to convey some of the sense
of beauty in famous theorems.  But even an intelligent layperson is left with
only a general idea of what a theorem is about and is hardly given the tools
needed to make use of it.  Traditionally, it is only after years of arduous
study that one can grasp the concepts needed for deep understanding.
Metamath\index{Metamath} allows you to approach the proof of the theorem from
a quite different perspective, peeling apart the formulas and definitions
layer by layer until an entirely different kind of understanding is achieved.
Every step of the proof is there, pieced together with absolute precision and
instantly available for inspection through a microscope with a magnification
as powerful as you desire.

A proof in itself can be considered an object of beauty.  Constructing an
elegant proof is an art.  Once a famous theorem has been proved, often
considerable effort is made to find simpler and more easily understood
proofs.  Creating and communicating elegant proofs is a major concern of
mathematicians.  Metamath is one way of providing a common language for
archiving and preserving this information.

The length of a proof can, to a certain extent, be considered an
objective measure of its ``beauty,'' since shorter proofs are usually
considered more elegant.  In the set theory database
\texttt{set.mm}\index{set theory database (\texttt{set.mm})}%
\index{Metamath Proof Explorer}
provided with Metamath, one goal was to make all proofs as short as possible.

\needspace{4\baselineskip}
\subsection{Simplicity}

\begin{quote}
  {\em God made man simple; man's complex problems are of his own
  devising.}
    \flushright\sc Eccles. 7:29\footnote{Jerusalem Bible.}\\
\end{quote}\index{Bible}

\needspace{3\baselineskip}
\begin{quote}
  {\em God made integers, all else is the work of man.}
    \flushright\sc Leopold Kronecker\footnote{{\em Jahresbericht
	der Deutschen Mathematiker-Vereinigung }, vol. 2, p. 19.}\\
\end{quote}\index{Kronecker, Leopold}

\needspace{3\baselineskip}
\begin{quote}
  {\em For what is clear and easily comprehended attracts; the
  complicated repels.}
    \flushright\sc David Hilbert\footnote{As quoted in \cite{deMillo},
p.~273.}\\
\end{quote}\index{Hilbert, David}

The Metamath\index{Metamath} language is simple and Spartan.  Metamath treats
all mathematical expressions as simple sequences of symbols, devoid of meaning.
The higher-level or ``metamathematical'' notions underlying Metamath are about
as simple as they could possibly be.  Each individual step in a proof involves
a single basic concept, the substitution of an expression for a variable, so
that in principle almost anyone, whether mathematician or not, can
completely understand how it was arrived at.

In one of its most basic applications, Metamath\index{Metamath} can be used to
develop the foundations of mathematics\index{foundations of mathematics} from
the very beginning.  This is done in the set theory database that is provided
with the Metamath package and is the subject matter
of Chapter~\ref{fol}. Any language (a metalanguage\index{metalanguage})
used to describe mathematics (an object language\index{object language}) must
have a mathematical content of its own, but it is desirable to keep this
content down to a bare minimum, namely that needed to make use of the
inference rules specified by the axioms.  With any metalanguage there is a
``chicken and egg'' problem somewhat like circular reasoning:  you must assume
the validity of the mathematics of the metalanguage in order to prove the
validity of the mathematics of the object language.  The mathematical content
of Metamath itself is quite limited.  Like the rules of a game of chess, the
essential concepts are simple enough so that virtually anyone should be able to
understand them (although that in itself will not let you play like
a master).  The symbols that Metamath manipulates do not in themselves
have any intrinsic meaning.  Your interpretation of the axioms that you supply
to Metamath is what gives them meaning.  Metamath is an attempt to strip down
mathematical thought to its bare essence and show you exactly how the symbols
are manipulated.

Philosophers and logicians, with various motivations, have often thought it
important to study ``weak'' fragments of logic\index{weak logic}
\cite{Anderson}\index{Anderson, Alan Ross} \cite{MegillBunder}\index{Megill,
Norman}\index{Bunder, Martin}, other unconventional systems of logic (such as
``modal'' logic\index{modal logic} \cite[ch.\ 27]{Boolos}\index{Boolos, George
S.}), and quantum logic\index{quantum logic} in physics
\cite{Pavicic}\index{Pavi{\v{c}}i{\'{c}}, M.}.  Metamath\index{Metamath}
provides a framework in which such systems can be expressed, with an absolute
precision that makes all underlying metamathematical assumptions rigorous and
crystal clear.

Some schools of philosophical thought, for example
intuitionism\index{intuitionism} and constructivism\index{constructivism},
demand that the notions underlying any mathematical system be as simple and
concrete as possible.  Metamath should meet the requirements of these
philosophies.  Metamath must be taught the symbols, axioms\index{axiom}, and
rules\index{rule} for a specific theory, from the skeptical (such as
intuitionism\index{intuitionism}\footnote{Intuitionism does not accept the law
of excluded middle (``either something is true or it is not true'').  See
\cite[p.~xi]{Tymoczko}\index{Tymoczko, Thomas} for discussion and references
on this topic.  Consider the theorem, ``There exist irrational numbers $a$ and
$b$ such that $a^b$ is rational.''  An intuitionist would reject the following
proof:  If $\sqrt{2}^{\sqrt{2}}$ is rational, we are done.  Otherwise, let
$a=\sqrt{2}^{\sqrt{2}}$ and $b=\sqrt{2}$. Then $a^b=2$, which is rational.})
to the bold (such as the axiom of choice in set theory\footnote{The axiom of
choice\index{Axiom of Choice} asserts that given any collection of pairwise
disjoint nonempty sets, there exists a set that has exactly one element in
common with each set of the collection.  It is used to prove many important
theorems in standard mathematics.  Some philosophers object to it because it
asserts the existence of a set without specifying what the set contains
\cite[p.~154]{Enderton}\index{Enderton, Herbert B.}.  In one foundation for
mathematics due to Quine\index{Quine, Willard Van Orman}, that has not been
otherwise shown to be inconsistent, the axiom of choice turns out to be false
\cite[p.~23]{Curry}\index{Curry, Haskell B.}.  The \texttt{show
trace{\char`\_}back} command of the Metamath program allows you to find out
whether the axiom of choice, or any other axiom, was assumed by a
proof.}\index{\texttt{show trace{\char`\_}back} command}).

The simplicity of the Metamath language lets the algorithm (computer program)
that verifies the validity of a Metamath proof be straightforward and
robust.  You can have confidence that the theorems it verifies really can be
derived from your axioms.

\subsection{Rigor}

\begin{quote}
  {\em Rigor became a goal with the Greeks\ldots But the efforts to
  pursue rigor to the utmost have led to an impasse in which there is
  no longer any agreement on what it really means.  Mathematics remains
  alive and vital, but only on a pragmatic basis.}
    \flushright\sc  Morris Kline\footnote{\cite{Kline}, p.~1209.}\\
\end{quote}\index{Kline, Morris}

Kline refers to a much deeper kind of rigor than that which we will discuss in
this section.  G\"{o}del's incompleteness theorem\index{G\"{o}del's
incompleteness theorem} showed that it is impossible to achieve absolute rigor
in standard mathematics because we can never prove that mathematics is
consistent (free from contradictions).\index{consistent theory}  If
mathematics is consistent, we will never know it, but must rely on faith.  If
mathematics is inconsistent, the best we can hope for is that some clever
future mathematician will discover the inconsistency.  In this case, the
axioms would probably be revised slightly to eliminate the inconsistency, as
was done in the case of Russell's paradox,\index{Russell's paradox} but the
bulk of mathematics would probably not be affected by such a discovery.
Russell's paradox, for example, did not affect most of the remarkable results
achieved by 19th-century and earlier mathematicians.  It mainly invalidated
some of Gottlob Frege's\index{Frege, Gottlob} work on the foundations of
mathematics in the late 1800's; in fact Frege's work inspired Russell's
discovery.  Despite the paradox, Frege's work contains important concepts that
have significantly influenced modern logic.  Kline's {\em Mathematics, The
Loss of Certainty} \cite{Klinel}\index{Kline, Morris} has an interesting
discussion of this topic.

What {\em can} be achieved with absolute certainty\index{certainty} is the
knowledge that if we assume the axioms are consistent and true, then the
results derived from them are true.  Part of the beauty of mathematics is that
it is the one area of human endeavor where absolute certainty can be achieved
in this sense.  A mathematical truth will remain such for eternity.  However,
our actual knowledge of whether a particular statement is a mathematical truth
is only as certain as the correctness of the proof that establishes it.  If
the proof of a statement is questionable or vague, we can't have absolute
confidence in the truth that the statement claims.

Let us look at some traditional ways of expressing proofs.

Except in the field of formal logic\index{formal logic}, almost all
traditional proofs in mathematics are really not proofs at all, but rather
proof outlines or hints as to how to go about constructing the proof.  Many
gaps\index{gaps in proofs} are left for the reader to fill in. There are
several reasons for this.  First, it is usually assumed in mathematical
literature that the person reading the proof is a mathematician familiar with
the specialty being described, and that the missing steps are obvious to such
a reader or at least that the reader is capable of filling them in.  This
attitude is fine for professional mathematicians in the specialty, but
unfortunately it often has the drawback of cutting off the rest of the world,
including mathematicians in other specialties, from understanding the proof.
We discussed one possible resolution to this on p.~\pageref{envision}.
Second, it is often assumed that a complete formal proof\index{formal proof}
would require countless millions of symbols (Section~\ref{dream}). This might
be true if the proof were to be expressed directly in terms of the axioms of
logic and set theory,\index{set theory} but it is usually not true if we allow
ourselves a hierarchy\index{hierarchy} of definitions and theorems to build
upon, using a notation that allows us to introduce new symbols, definitions,
and theorems in a precisely specified way.

Even in formal logic,\index{formal logic} formal proofs\index{formal proof}
that are considered complete still contain hidden or implicit information.
For example, a ``proof'' is usually defined as a sequence of
wffs,\index{well-formed formula (wff)}\footnote{A {\em wff} or well-formed
formula is a mathematical expression (string of symbols) constructed according
to some precise rules.  A formal mathematical system\index{formal system}
contains (1) the rules for constructing syntactically correct
wffs,\index{syntax rules} (2) a list of starting wffs called
axioms,\index{axiom} and (3) one or more rules prescribing how to derive new
wffs, called theorems\index{theorem}, from the axioms or previously derived
theorems.  An example of such a system is contained in
Metamath's\index{Metamath} set theory database, which defines a formal
system\index{formal system} from which all of standard mathematics can be
derived.  Section~\ref{startf} steps you through a complete example of a formal
system, and you may want to skim it now if you are unfamiliar with the
concept.} each of which is an axiom or follows from a rule applied to previous
wffs in the sequence.  The implicit part of the proof is the algorithm by
which a sequence of symbols is verified to be a valid wff, given the
definition of a wff.  The algorithm in this case is rather simple, but for a
computer to verify the proof,\index{automated proof verification} it must have
the algorithm built into its verification program.\footnote{It is possible, of
course, to specify wff construction syntax outside of the program itself
with a suitable input language (the Metamath language being an example), but
some proof-verification or theorem-proving programs lack the ability to extend
wff syntax in such a fashion.} If one deals exclusively with axioms and
elementary wffs, it is straightforward to implement such an algorithm.  But as
more and more definitions are added to the theory in order to make the
expression of wffs more compact, the algorithm becomes more and more
complicated.  A computer program that implements the algorithm becomes larger
and harder to understand as each definition is introduced, and thus more prone
to bugs.\index{computer program bugs}  The larger the program, the
more suspicious the mathematician may be about
the validity of its algorithms.  This is especially true because
computer programs are inherently hard to follow to begin with, and few people
enjoy verifying them manually in detail.

Metamath\index{Metamath} takes a different approach.  Metamath's ``knowledge''
is limited to the ability to substitute variables for expressions, subject to
some simple constraints.  Once the basic algorithm of Metamath is assumed to
be debugged, and perhaps independently confirmed, it
can be trusted once and for all.  The information that Metamath needs to
``understand'' mathematics is contained entirely in the body of knowledge
presented to Metamath.  Any errors in reasoning can only be errors in the
axioms or definitions contained in this body of knowledge.  As a
``constructive'' language\index{constructive language} Metamath has no
conditional branches or loops like the ones that make computer programs hard
to decipher; instead, the language can only build new sequences of symbols
from earlier sequences  of symbols.

The simplicity of the rules that underlie Metamath not only makes Metamath
easy to learn but also gives Metamath a great deal of flexibility. For
example, Metamath is not limited to describing standard first-order
logic\index{first-order logic}; higher-order logics\index{higher-order logic}
and fragments of logic\index{weak logic} can be described just as easily.
Metamath gives you the freedom to define whatever wff notation you prefer; it
has no built-in conception of the syntax of a wff.\index{well-formed formula
(wff)}  With suitable axioms and definitions, Metamath can even describe and
prove things about itself.\index{Metamath!self-description}  (John
Harrison\index{Harrison, John} discusses the ``reflection''
principle\index{reflection principle} involved in self-descriptive systems in
\cite{Harrison}.)

The flexibility of Metamath requires that its proofs specify a lot of detail,
much more than in an ordinary ``formal'' proof.\index{formal proof}  For
example, in an ordinary formal proof, a single step consists of displaying the
wff that constitutes that step.  In order for a computer program to verify
that the step is acceptable, it first must verify that the symbol sequence
being displayed is an acceptable wff.\index{automated proof verification} Most
proof verifiers have at least basic wff syntax built into their programs.
Metamath has no hard-wired knowledge of what constitutes a wff built into it;
instead every wff must be explicitly constructed based on rules defining wffs
that are present in a database.  Thus a single step in an ordinary formal
proof may be correspond to many steps in a Metamath proof. Despite the larger
number of steps, though, this does not mean that a Metamath proof must be
significantly larger than an ordinary formal proof. The reason is that since
we have constructed the wff from scratch, we know what the wff is, so there is
no reason to display it.  We only need to refer to a sequence of statements
that construct it.  In a sense, the display of the wff in an ordinary formal
proof is an implicit proof of its own validity as a wff; Metamath just makes
the proof explicit. (Section~\ref{proof} describes Metamath's proof notation.)

\section{Computers and Mathematicians}

\begin{quote}
  {\em The computer is important, but not to mathematics.}
    \flushright\sc  Paul Halmos\footnote{As quoted in \cite{Albers}, p.~121.}\\
\end{quote}\index{Halmos, Paul}

Pure mathematicians have traditionally been indifferent to computers, even to
the point of disdain.\index{computers and pure mathematics}  Computer science
itself is sometimes considered to fall in the mundane realm of ``applied''
mathematics, perhaps essential for the real world but intellectually unexciting
to those who seek the deepest truths in mathematics.  Perhaps a reason for this
attitude towards computers is that there is little or no computer software that
meets their needs, and there may be a general feeling that such software could
not even exist.  On the one hand, there are the practical computer algebra
systems, which can perform amazing symbolic manipulations in algebra and
calculus,\index{computer algebra system} yet can't prove the simplest
existence theorem, if the idea of a proof is present at all.  On the other
hand, there are specialized automated theorem provers that technically speaking
may generate correct proofs.\index{automated theorem proving}  But sometimes
their specialized input notation may be cryptic and their output perceived to
be long, inelegant, incomprehensible proofs.    The output
may be viewed with suspicion, since the program that generates it tends to be
very large, and its size increases the potential for bugs\index{computer
program bugs}.  Such a proof may be considered trustworthy only if
independently verified and ``understood'' by a human, but no one wants to
waste their time on such a boring, unrewarding chore.



\needspace{4\baselineskip}
\subsection{Trusting the Computer}

\begin{quote}
  {\em \ldots I continue to find the quasi-empirical interpretation of
  computer proofs to be the more plausible.\ldots Since not
  everything that claims to be a computer proof can be
  accepted as valid, what are the mathematical criteria for acceptable
  computer proofs?}
    \flushright\sc  Thomas Tymoczko\footnote{\cite{Tymoczko}, p.~245.}\\
\end{quote}\index{Tymoczko, Thomas}

In some cases, computers have been essential tools for proving famous
theorems.  But if a proof is so long and obscure that it can be verified in a
practical way only with a computer, it is vaguely felt to be suspicious.  For
example, proving the famous four-color theorem\index{four-color
theorem}\index{proof length} (``a map needs no more than four colors to
prevent any two adjacent countries from having the same color'') can presently
only be done with the aid of a very complex computer program which originally
required 1200 hours of computer time. There has been considerable debate about
whether such a proof can be trusted and whether such a proof is ``real''
mathematics \cite{Swart}\index{Swart, E. R.}.\index{trusting computers}

However, under normal circumstances even a skeptical mathematician would have a
great deal of confidence in the result of multiplying two numbers on a pocket
calculator, even though the precise details of what goes on are hidden from its
user.  Even the verification on a supercomputer that a huge number is prime is
trusted, especially if there is independent verification; no one bothers to
debate the philosophical significance of its ``proof,'' even though the actual
proof would be so large that it would be completely impractical to ever write
it down on paper.  It seems that if the algorithm used by the computer is
simple enough to be readily understood, then the computer can be trusted.

Metamath\index{Metamath} adopts this philosophy.  The simplicity of its
language makes it easy to learn, and because of its simplicity one can have
essentially absolute confidence that a proof is correct. All axioms, rules, and
definitions are available for inspection at any time because they are defined
by the user; there are no hidden or built-in rules that may be prone to subtle
bugs\index{computer program bugs}.  The basic algorithm at the heart of
Metamath is simple and fixed, and it can be assumed to be bug-free and robust
with a degree of confidence approaching certainty.
Independently written implementations of the Metamath verifier
can reduce any residual doubt on the part of a skeptic even further;
there are now over a dozen such implementations, written by many people.

\subsection{Trusting the Mathematician}\label{trust}

\begin{quote}
  {\em There is no Algebraist nor Mathematician so expert in his science, as
  to place entire confidence in any truth immediately upon his discovery of it,
  or regard it as any thing, but a mere probability.  Every time he runs over
  his proofs, his confidence encreases; but still more by the approbation of
  his friends; and is rais'd to its utmost perfection by the universal assent
  and applauses of the learned world.}
  \flushright\sc David Hume\footnote{{\em A Treatise of Human Nature}, as
  quoted in \cite{deMillo}, p.~267.}\\
\end{quote}\index{Hume, David}

\begin{quote}
  {\em Stanislaw Ulam estimates that mathematicians publish 200,000 theorems
  every year.  A number of these are subsequently contradicted or otherwise
  disallowed, others are thrown into doubt, and most are ignored.}
  \flushright\sc Richard de Millo et. al.\footnote{\cite{deMillo}, p.~269.}\\
\end{quote}\index{Ulam, Stanislaw}

Whether or not the computer can be trusted, humans  of course will occasionally
err. Only the most memorable proofs get independently verified, and of these
only a handful of truly great ones achieve the status of being ``known''
mathematical truths that are used without giving a second thought to their
correctness.

There are many famous examples of incorrect theorems and proofs in
mathematical literature.\index{errors in proofs}

\begin{itemize}
\item There have been thousands of purported proofs of Fermat's Last
Theorem\index{Fermat's Last Theorem} (``no integer solutions exist to $x^n +
y^n = z^n$ for $n > 2$''), by amateurs, cranks, and well-regarded
mathematicians \cite[p.~5]{Stark}\index{Stark, Harold M}.  Fermat wrote a note
in his copy of Bachet's {\em Diophantus} that he found ``a truly marvelous
proof of this theorem but this margin is too narrow to contain it''
\cite[p.~507]{Kramer}.  A recent, much publicized proof by Yoichi
Miyaoka\index{Miyaoka, Yoichi} was shown to be incorrect ({\em Science News},
April 9, 1988, p.~230).  The theorem was finally proved by Andrew
Wiles\index{Wiles, Andrew} ({\em Science News}, July 3, 1993, p.~5), but it
initially had some gaps and took over a year after its announcement to be
checked thoroughly by experts.  On Oct. 25, 1994, Wiles announced that the last
gap found in his proof had been filled in.
  \item In 1882, M. Pasch discovered that an axiom was omitted from Euclid's
formulation of geometry\index{Euclidean geometry}; without it, the proofs of
important theorems of Euclid are not valid.  Pasch's axiom\index{Pasch's
axiom} states that a line that intersects one side of a triangle must also
intersect another side, provided that it does not touch any of the triangle's
vertices.  The omission of Pasch's axiom went unnoticed for 2000
years \cite[p.~160]{Davis}, in spite of (one presumes) the thousands of
students, instructors, and mathematicians who studied Euclid.
  \item The first published proof of the famous Schr\"{o}der--Bernstein
theorem\index{Schr\"{o}der--Bernstein theorem} in set theory was incorrect
\cite[p.~148]{Enderton}\index{Enderton, Herbert B.}.  This theorem states
that if there exists a 1-to-1 function\footnote{A {\em set}\index{set} is any
collection of objects. A {\em function}\index{function} or {\em
mapping}\index{mapping} is a rule that assigns to each element of one set
(called the function's {\em domain}\index{domain}) an element from another
set.} from set $A$ into set $B$ and vice-versa, then sets $A$ and $B$ have
a 1-to-1 correspondence.  Although it sounds simple and obvious,
the standard proof is quite long and complex.
  \item In the early 1900's, Hilbert\index{Hilbert, David} published a
purported proof of the continuum hypothesis\index{continuum hypothesis}, which
was eventually established as unprovable by Cohen\index{Cohen, Paul} in 1963
\cite[p.~166]{Enderton}.  The continuum hypothesis states that no
infinity\index{infinity} (``transfinite cardinal number'')\index{cardinal,
transfinite} exists whose size (or ``cardinality''\index{cardinality}) is
between the size of the set of integers and the size of the set of real
numbers.  This hypothesis originated with German mathematician Georg
Cantor\index{Cantor, Georg} in the late 1800's, and his inability to prove it
is said to have contributed to mental illness that afflicted him in his later
years.
  \item An incorrect proof of the four-color theorem\index{four-color theorem}
was published by Kempe\index{Kempe, A. B.} in 1879
\cite[p.~582]{Courant}\index{Courant, Richard}; it stood for 11 years before
its flaw was discovered.  This theorem states that any map can be colored
using only four colors, so that no two adjacent countries have the same
color.  In 1976 the theorem was finally proved by the famous computer-assisted
proof of Haken, Appel, and Koch \cite{Swart}\index{Appel, K.}\index{Haken,
W.}\index{Koch, K.}.  Or at least it seems that way.  Mathematician
H.~S.~M.~Coxeter\index{Coxeter, H. S. M.} has doubts \cite[p.~58]{Davis}:  ``I
have a feeling that that is an untidy kind of use of the computers, and the more
you correspond with Haken and Appel, the more shaky you seem to be.''
  \item Many false ``proofs'' of the Poincar\'{e}
conjecture\index{Poincar\'{e} conjecture} have been proposed over the years.
This conjecture states that any object that mathematically behaves like a
three-dimensional sphere is a three-dimensional sphere topologically,
regardless of how it is distorted.  In March 1986, mathematicians Colin
Rourke\index{Rourke, Colin} and Eduardo R\^{e}go\index{R\^{e}go, Eduardo}
caused  a stir in the mathematical community by announcing that they had found
a proof; in November of that year the proof was found to be false \cite[p.
218]{PetersonI}.  It was finally proved in 2003 by Grigory Perelman
\label{poincare}\index{Szpiro, George}\index{Perelman, Grigory}\cite{Szpiro}.
 \end{itemize}

Many counterexamples to ``theorems'' in recent mathematical
literature related to Clifford algebras\index{Clifford algebras}
 have been found by Pertti
Lounesto (who passed away in 2002).\index{Lounesto, Pertti}
See the web page \url{http://mathforum.org/library/view/4933.html}.
% http://users.tkk.fi/~ppuska/mirror/Lounesto/counterexamples.htm

One of the purposes of Metamath\index{Metamath} is to allow proofs to be
expressed with absolute precision.  Developing a proof in the Metamath
language can be challenging, because Metamath will not permit even the
tiniest mistake.\index{errors in proofs}  But once the proof is created, its
correctness can be trusted immediately, without having to depend on the
process of peer review for confirmation.

\section{The Use of Computers in Mathematics}

\subsection{Computer Algebra Systems}

For the most part, you will find that Metamath\index{Metamath} is not a
practical tool for manipulating numbers.  (Even proving that $2 + 2 = 4$, if
you start with set theory, can be quite complex!)  Several commercial
mathematics packages are quite good at arithmetic, algebra, and calculus, and
as practical tools they are invaluable.\index{computer algebra system} But
they have no notion of proof, and cannot understand statements starting with
``there exists such and such...''.

Software packages such as Mathematica \cite{Wolfram}\index{Mathematica} do not
concern themselves with proofs but instead work directly with known results.
These packages primarily emphasize heuristic rules such as the substitution of
equals for equals to achieve simpler expressions or expressions in a different
form.  Starting with a rich collection of built-in rules and algorithms, users
can add to the collection by means of a powerful programming language.
However, results such as, say, the existence of a certain abstract object
without displaying the actual object cannot be expressed (directly) in their
languages.  The idea of a proof from a small set of axioms is absent.  Instead
this software simply assumes that each fact or rule you add to the built-in
collection of algorithms is valid.  One way to view the software is as a large
collection of axioms from which the software, with certain goals, attempts to
derive new theorems, for example equating a complex expression with a simpler
equivalent. But the terms ``theorem''\index{theorem} and
``proof,''\index{proof} for example, are not even mentioned in the index of
the user's manual for Mathematica.\index{Mathematica and proofs}  What is also
unsatisfactory from a philosophical point of view is that there is no way to
ensure the validity of the results other than by trusting the writer of each
application module or tediously checking each module by hand, similar to
checking a computer program for bugs.\index{computer program
bugs}\footnote{Two examples illustrate why the knowledge database of computer
algebra systems should sometimes be regarded with a certain caution.  If you
ask Mathematica (version 3.0) to \texttt{Solve[x\^{ }n + y\^{ }n == z\^{ }n , n]}
it will respond with \texttt{\{\{n-\char`\>-2\}, \{n-\char`\>-1\},
\{n-\char`\>1\}, \{n-\char`\>2\}\}}. In other words, Mathematica seems to
``know'' that Fermat's Last Theorem\index{Fermat's Last Theorem} is true!  (At
the time this version of Mathematica was released this fact was unknown.)  If
you ask Maple\index{Maple} to \texttt{solve(x\^{ }2 = 2\^{ }x)} then
\texttt{simplify(\{"\})}, it returns the solution set \texttt{\{2, 4\}}, apparently
unaware that $-0.7666647$\ldots is also a solution.} While of course extremely
valuable in applied mathematics, computer algebra systems tend to be of little
interest to the theoretical mathematician except as aids for exploring certain
specific problems.

Because of possible bugs, trusting the output of a computer algebra system for
use as theorems in a proof-verifier would defeat the latter's goal of rigor.
On the other hand, a fact such that a certain relatively large number is
prime, while easy for a computer algebra system to derive, might have a long,
tedious proof that could overwhelm a proof-verifier. One approach for linking
computer algebra systems to a proof-verifier while retaining the advantages of
both is to add a hypothesis to each such theorem indicating its source.  For
example, a constant {\sc maple} could indicate the theorem came from the Maple
package, and would mean ``assuming Maple is consistent, then\ldots''  This and
many other topics concerning the formalization of mathematics are discussed in
John Harrison's\index{Harrison, John} very interesting
PhD thesis~\cite{Harrison-thesis}.

\subsection{Automated Theorem Provers}\label{theoremprovers}

A mathematical theory is ``decidable''\index{decidable theory} if a mechanical
method or algorithm exists that is guaranteed to determine whether or not a
particular formula is a theorem.  Among the few theories that are decidable is
elementary geometry,\index{Euclidean geometry} as was shown by a classic
result of logician Alfred Tarski\index{Tarski, Alfred} in 1948
\cite{Tarski}.\footnote{Tarski's result actually applies to a subset of the
geometry discussed in elementary textbooks.  This subset includes most of what
would be considered elementary geometry but it is not powerful enough to
express, among other things, the notions of the circumference and area of a
circle.  Extending the theory in a way that includes notions such as these
makes the theory undecidable, as was also shown by Tarski.  Tarski's algorithm
is far too inefficient to implement practically on a computer.  A practical
algorithm for proving a smaller subset of geometry theorems (those not
involving concepts of ``order'' or ``continuity'') was discovered by Chinese
mathematician Wu Wen-ts\"{u}n in 1977 \cite{Chou}\index{Chou,
Shang-Ching}.}\index{Wen-ts{\"{u}}n, Wu}  But most theories, including
elementary arithmetic, are undecidable.  This fact contributes to keeping
mathematics alive and well, since many mathematicians believe
that they will never be
replaced by computers (if they believe Roger Penrose's argument that a
computer can never replace the brain \cite{Penrose}\index{Penrose, Roger}).
In fact,  elementary geometry is often considered a ``dead'' field
for the simple reason that it is decidable.

On the other hand, the undecidability of a theory does not mean that one cannot
use a computer to search for proofs, providing one is willing to give up if a
proof is not found after a reasonable amount of time.  The field of automated
theorem proving\index{automated theorem proving} specializes in pursuing such
computer searches.  Among the more successful results to date are those based
on an algorithm known as Robinson's resolution principle
\cite{Robinson}\index{Robinson's resolution principle}.

Automated theorem provers can be excellent tools for those willing to learn
how to use them.  But they are not widely used in mainstream pure
mathematics, even though they could probably be useful in many areas.  There
are several reasons for this.  Probably most important, the main goal in pure
mathematics is to arrive at results that are considered to be deep or
important; proving them is essential but secondary.  Usually, an automated
theorem prover cannot assist in this main goal, and by the time the main goal
is achieved, the mathematician may have already figured out the proof as a
by-product.  There is also a notational problem.  Mathematicians are used to
using very compact syntax where one or two symbols (heavily dependent on
context) can represent very complex concepts; this is part of the
hierarchy\index{hierarchy} they have built up to tackle difficult problems.  A
theorem prover on the other hand might require that a theorem be expressed in
``first-order logic,''\index{first-order logic} which is the logic on which
most of mathematics is ultimately based but which is not ordinarily used
directly because expressions can become very long.  Some automated theorem
provers are experimental programs, limited in their use to very specialized
areas, and the goal of many is simply research into the nature of automated
theorem proving itself.  Finally, much research remains to be done to enable
them to prove very deep theorems.  One significant result was a
computer proof by Larry Wos\index{Wos, Larry} and colleagues that every Robbins
algebra\index{Robbins algebra} is a Boolean  algebra\index{Boolean algebra}
({\em New York Times}, Dec. 10, 1996).\footnote{In 1933, E.~V.\
Huntington\index{Huntington, E. V.}
presented the following axiom system for
Boolean algebra with a unary operation $n$ and a binary operation $+$:
\begin{center}
    $x + y = y + x$ \\
    $(x + y) + z = x + (y + z)$ \\
    $n(n(x) + y) + n(n(x) + n(y)) = x$
\end{center}
Herbert Robbins\index{Robbins, Herbert}, a student of Huntington, conjectured
that the last equation can be replaced with a simpler one:
\begin{center}
    $n(n(x + y) + n(x + n(y))) = x$
\end{center}
Robbins and Huntington could not find a proof.  The problem was
later studied unsuccessfully by Tarski\index{Tarski, Alfred} and his
students, and it remained an unsolved problem until a
computer found the proof in 1996.  For more information on
the Robbins algebra problem see \cite{Wos}.}

How does Metamath\index{Metamath} relate to automated theorem provers?  A
theorem prover is primarily concerned with one theorem at a time (perhaps
tapping into a small database of known theorems) whereas Metamath is more like
a theorem archiving system, storing both the theorem and its proof in a
database for access and verification.  Metamath is one answer to ``what do you
do with the output of a theorem prover?''  and could be viewed as the
next step in the process.  Automated theorem provers could be useful tools for
helping develop its database.
Note that very long, automatically
generated proofs can make your database fat and ugly and cause Metamath's proof
verification to take a long time to run.  Unless you have a particularly good
program that generates very concise proofs, it might be best to consider the
use of automatically generated proofs as a quick-and-dirty approach, to be
manually rewritten at some later date.

The program {\sc otter}\index{otter@{\sc otter}}\footnote{\url{http://www.cs.unm.edu/\~mccune/otter/}.}, later succeeded by
prover9\index{prover9}\footnote{\url{https://www.cs.unm.edu/~mccune/mace4/}.},
have been historically influential.
The E prover\index{E prover}\footnote{\url{https://github.com/eprover/eprover}.}
is a maintained automated theorem prover
for full first-order logic with equality.
There are many other automated theorem provers as well.

If you want to combine automated theorem provers with Metamath
consider investigating
the book {\em Automated Reasoning:  Introduction and Applications}
\cite{Wos}\index{Wos, Larry}.  This book discusses
how to use {\sc otter} in a way that can
not only able to generate
relatively efficient proofs, it can even be instructed to search for
shorter proofs.  The effective use of {\sc otter} (and similar tools)
does require a certain
amount of experience, skill, and patience.  The axiom system used in the
\texttt{set.mm}\index{set theory database (\texttt{set.mm})} set theory
database can be expressed to {\sc otter} using a method described in
\cite{Megill}.\index{Megill, Norman}\footnote{To use those axioms with
{\sc otter}, they must be restated in a way that eliminates the need for
``dummy variables.''\index{dummy variable!eliminating} See the Comment
on p.~\pageref{nodd}.} When successful, this method tends to generate
short and clever proofs, but my experiments with it indicate that the
method will find proofs within a reasonable time only for relatively
easy theorems.  It is still fun to experiment with.

Reference \cite{Bledsoe}\index{Bledsoe, W. W.} surveys a number of approaches
people have explored in the field of automated theorem proving\index{automated
theorem proving}.

\subsection{Interactive Theorem Provers}\label{interactivetheoremprovers}

Finding proofs completely automatically is difficult, so there
are some interactive theorem provers that allow a human to guide the
computer to find a proof.
Examples include
HOL Light\index{HOL light}%
\footnote{\url{https://www.cl.cam.ac.uk/~jrh13/hol-light/}.},
Isabelle\index{Isabelle}%
\footnote{\url{http://www.cl.cam.ac.uk/Research/HVG/Isabelle}.},
{\sc hol}\index{hol@{\sc hol}}%
\footnote{\url{https://hol-theorem-prover.org/}.},
and
Coq\index{Coq}\footnote{\url{https://coq.inria.fr/}.}.

A major difference between most of these tools and Metamath is that the
``proofs'' are actually programs that guide the program to find a proof,
and not the proof itself.
For example, an Isabelle/HOL proof might apply a step
\texttt{apply (blast dest: rearrange reduction)}. The \texttt{blast}
instruction applies
an automatic tableux prover and returns if it found a sequence of proof
steps that work... but the sequence is not considered part of the proof.

A good overview of
higher-level proof verification languages (such as {\sc lcf}\index{lcf@{\sc
lcf}} and {\sc hol}\index{hol@{\sc hol}})
is given in \cite{Harrison}.  All of these languages are fundamentally
different from Metamath in that much of the mathematical foundational
knowledge is embedded in the underlying proof-verification program, rather
than placed directly in the database that is being verified.
These can have a steep learning curve for those without a mathematical
background.  For example, one usually must have a fair understanding of
mathematical logic in order to follow their proofs.

\subsection{Proof Verifiers}\label{proofverifiers}

A proof verifier is a program that doesn't generate proofs but instead
verifies proofs that you give it.  Many proof verifiers have limited built-in
automated proof capabilities, such as figuring out simple logical inferences
(while still being guided by a person who provides the overall proof).  Metamath
has no built-in automated proof capability other than the limited
capability of its Proof Assistant.

Proof-verification languages are not used as frequently as they might be.
Pure mathematicians are more concerned with producing new results, and such
detail and rigor would interfere with that goal.  The use of computers in pure
mathematics is primarily focused on automated theorem provers (not verifiers),
again with the ultimate goal of aiding the creation of new mathematics.
Automated theorem provers are usually concerned with attacking one theorem at
time rather than making a large, organized database easily available to the
user.  Metamath is one way to help close this gap.

By itself Metamath is a mostly a proof verifier.
This does not mean that other approaches can't be used; the difference
is that in Metamath, the results of various provers must be recorded
step-by-step so that they can be verified.

Another proof-verification language is Mizar,\index{Mizar} which can display
its proofs in the informal language that mathematicians are accustomed to.
Information on the Mizar language is available at \url{http://mizar.org}.

For the working mathematician, Mizar is an excellent tool for rigorously
documenting proofs. Mizar typesets its proofs in the informal English used by
mathematicians (and, while fine for them, are just as inscrutable by
laypersons!). A price paid for Mizar is a relatively steep learning curve of a
couple of weeks.  Several mathematicians are actively formalizing different
areas of mathematics using Mizar and publishing the proofs in a dedicated
journal. Unfortunately the task of formalizing mathematics is still looked
down upon to a certain extent since it doesn't involve the creation of ``new''
mathematics.

The closest system to Metamath is
the {\em Ghilbert}\index{Ghilbert} proof language (\url{http://ghilbert.org})
system developed by
Raph Levien\index{Levien, Raph}.
Ghilbert is a formal proof checker heavily inspired by Metamath.
Ghilbert statements are s-expressions (a la Lisp), which is easy
for computers to parse but many people find them hard to read.
There are a number of differences in their specific constructs, but
there is at least one tool to translate some Metamath materials into Ghilbert.
As of 2019 the Ghilbert community is smaller and less active than the
Metamath community.
That said, the Metamath and Ghilbert communities overlap, and fruitful
conversations between them have occurred many times over the years.

\subsection{Creating a Database of Formalized Mathematics}\label{mathdatabase}

Besides Metamath, there are several other ongoing projects with the goal of
formalizing mathematics into computer-verifiable databases.
Understanding some history will help.

The {\sc qed}\index{qed project@{\sc qed} project}%
\footnote{\url{http://www-unix.mcs.anl.gov/qed}.}
project arose in 1993 and its goals were outlined in the
{\sc qed} manifesto.
The {\sc qed} manifesto was
a proposal for a computer-based database of all mathematical knowledge,
strictly formalized and with all proofs having been checked automatically.
The project had a conference in 1994 and another in 1995;
there was also a ``twenty years of the {\sc qed} manifesto'' workshop
in 2014.
Its ideals are regularly reraised.

In a 2007 paper, Freek Wiedijk identified two reasons
for the failure of the {\sc qed} project as originally envisioned:%
\cite{Wiedijk-revisited}\index{Wiedijk, Freek}

\begin{itemize}
\item Very few people are working on formalization of mathematics. There is no compelling application for fully mechanized mathematics.
\item Formalized mathematics does not yet resemble traditional mathematics. This is partly due to the complexity of mathematical notation, and partly to the limitations of existing theorem provers and proof assistants.
\end{itemize}

But this did not end the dream of
formalizing mathematics into computer-verifiable databases.
The problems that led to the {\sc qed} manifesto are still with us,
even though the challenges were harder than originally considered.
What has happened instead is that various independent projects have
worked towards formalizing mathematics into computer-verifiable databases,
each simultaneously competing and cooperating with each other.

A concrete way to see this is
Freek Wiedijk's ``Formalizing 100 Theorems'' list%
\footnote{\url{http://www.cs.ru.nl/\%7Efreek/100/}.}
which shows the progress different systems have made on a challenge list
of 100 mathematical theorems.%
\footnote{ This is not the only list of ``interesting'' theorems.
Another interesting list was posted by Oliver Knill's list
\cite{Knill}\index{Knill, Oliver}.}
The top systems as of February 2019
(in order of the number of challenges completed) are
HOL Light, Isabelle, Metamath, Coq, and Mizar.

The Metamath 100%
\footnote{\url{http://us.metamath.org/mm\_100.html}}
page (maintained by David A. Wheeler\index{Wheeler, David A.})
shows the progress of Metamath (specifically its \texttt{set.mm} database)
against this challenge list maintained by Freek Wiedijk.
The Metamath \texttt{set.mm} database
has made a lot of progress over the years,
in part because working to prove those challenge theorems required
defining various terms and proving their properties as a prerequisite.
Here are just a few of the many statements that have been
formally proven with Metamath:

% The entries of this cause the narrow display to break poorly,
% since the short amount of text means LaTeX doesn't get a lot to work with
% and the itemize format gives it even *less* margin than usual.
% No one will mind if we make just this list flushleft, since this list
% will be internally consistent.
\begin{flushleft}
\begin{itemize}
\item 1. The Irrationality of the Square Root of 2
  (\texttt{sqr2irr}, by Norman Megill, 2001-08-20)
\item 2. The Fundamental Theorem of Algebra
  (\texttt{fta}, by Mario Carneiro, 2014-09-15)
\item 22. The Non-Denumerability of the Continuum
  (\texttt{ruc}, by Norman Megill, 2004-08-13)
\item 54. The Konigsberg Bridge Problem
  (\texttt{konigsberg}, by Mario Carneiro, 2015-04-16)
\item 83. The Friendship Theorem
  (\texttt{friendship}, by Alexander W. van der Vekens, 2018-10-09)
\end{itemize}
\end{flushleft}

We thank all of those who have developed at least one of the Metamath 100
proofs, and we particularly thank
Mario Carneiro\index{Carneiro, Mario}
who has contributed the most Metamath 100 proofs as of 2019.
The Metamath 100 page shows the list of all people who have contributed a
proof, and links to graphs and charts showing progress over time.
We encourage others to work on proving theorems not yet proven in Metamath,
since doing so improves the work as a whole.

Each of the math formalization systems (including Metamath)
has different strengths and weaknesses, depending on what you value.
Key aspects that differentiate Metamath from the other top systems are:

\begin{itemize}
\item Metamath is not tied to any particular set of axioms.
\item Metamath can show every step of every proof, no exceptions.
  Most other provers only assert that a proof can be found, and do not
  show every step. This also makes verification fast, because
  the system does not need to rediscover proof details.
\item The Metamath verifier has been re-implemented in many different
  programming languages, so verification can be done by multiple
  implementations.  In particular, the
  \texttt{set.mm}\index{set theory database (\texttt{set.mm})}%
  \index{Metamath Proof Explorer} database is verified by
  four different verifiers
  written in four different languages by four different authors.
  This greatly reduces the risk of accepting an invalid
  proof due to an error in the verifier.
\item Proofs stay proven.  In some systems, changes to the system's
  syntax or how a tactic works causes proofs to fail in later versions,
  causing older work to become essentially lost.
  Metamath's language is
  extremely small and fixed, so once a proof is added to a database,
  the database can be rechecked with later versions of the Metamath program
  and with other verifiers of Metamath databases.
  If an axiom or key definition needs to be changed, it is easy to
  manipulate the database as a whole to handle the change
  without touching the underlying verifier.
  Since re-verification of an entire database takes seconds, there
  is never a reason to delay complete verification.
  This aspect is especially compelling if your
  goal is to have a long-term database of proofs.
\item Licensing is generous.  The main Metamath databases are released to
  the public domain, and the main Metamath program is open source software
  under a standard, widely-used license.
\item Substitutions are easy to understand, even by those who are not
  professional mathematicians.
\end{itemize}

Of course, other systems may have advantages over Metamath
that are more compelling, depending on what you value.
In any case, we hope this helps you understand Metamath
within a wider context.

\subsection{In Summary}\label{computers-summary}

To summarize our discussions of computers and mathematics, computer algebra
systems can be viewed as theorem generators focusing on a narrow realm of
mathematics (numbers and their properties), automated theorem provers as proof
generators for specific theorems in a much broader realm covered by a built-in
formal system such as first-order logic, interactive theorem
provers require human guidance, proof verifiers verify proofs but
historically they have been
restricted to first-order logic.
Metamath, in contrast,
is a proof verifier and documenter whose realm is essentially unlimited.

\section{Mathematics and Metamath}

\subsection{Standard Mathematics}

There are a number of ways that Metamath\index{Metamath} can be used with
standard mathematics.  The most satisfying way philosophically is to start at
the very beginning, and develop the desired mathematics from the axioms of
logic and set theory.\index{set theory}  This is the approach taken in the
\texttt{set.mm}\index{set theory database (\texttt{set.mm})}%
\index{Metamath Proof Explorer}
database (also known as the Metamath Proof Explorer).
Among other things, this database builds up to the
axioms of real and complex numbers\index{analysis}\index{real and complex
numbers} (see Section~\ref{real}), and a standard development of analysis, for
example, could start at that point, using it as a basis.   Besides this
philosophical advantage, there are practical advantages to having all of the
tools of set theory available in the supporting infrastructure.

On the other hand, you may wish to start with the standard axioms of a
mathematical theory without going through the set theoretical proofs of those
axioms.  You will need mathematical logic to make inferences, but if you wish
you can simply introduce theorems\index{theorem} of logic as
``axioms''\index{axiom} wherever you need them, with the implicit assumption
that in principle they can be proved, if they are obvious to you.  If you
choose this approach, you will probably want to review the notation used in
\texttt{set.mm}\index{set theory database (\texttt{set.mm})} so that your own
notation will be consistent with it.

\subsection{Other Formal Systems}
\index{formal system}

Unlike some programs, Metamath\index{Metamath} is not limited to any specific
area of mathematics, nor committed to any particular mathematical philosophy
such as classical logic versus intuitionism, nor limited, say, to expressions
in first-order logic.  Although the database \texttt{set.mm}
describes standard logic and set theory, Meta\-math
is actually a general-purpose language for describing a wide variety of formal
systems.\index{formal system}  Non-standard systems such as modal
logic,\index{modal logic} intuitionist logic\index{intuitionism}, higher-order
logic\index{higher-order logic}, quantum logic\index{quantum logic}, and
category theory\index{category theory} can all be described with the Metamath
language.  You define the symbols you prefer and tell Metamath the axioms and
rules you want to start from, and Metamath will verify any inferences you make
from those axioms and rules.  A simple example of a non-standard formal system
is Hofstadter's\index{Hofstadter, Douglas R.} MIU system,\index{MIU-system}
whose Metamath description is presented in Appendix~\ref{MIU}.

This is not hypothetical.
The largest Metamath database is
\texttt{set.mm}\index{set theory database (\texttt{set.mm}}%
\index{Metamath Proof Explorer}), aka the Metamath Proof Explorer,
which uses the most common axioms for mathematical foundations
(specifically classical logic combined with Zermelo--Fraenkel
set theory\index{Zermelo--Fraenkel set theory} with the Axiom of Choice).
But other Metamath databases are available:

\begin{itemize}
\item The database
  \texttt{iset.mm}\index{intuitionistic logic database (\texttt{iset.mm})},
  aka the
  Intuitionistic Logic Explorer\index{Intuitionistic Logic Explorer},
  uses intuitionistic logic (a constructivist point of view)
  instead of classical logic.
\item The database
  \texttt{nf.mm}\index{New Foundations database (\texttt{nf.mm})},
  aka the
  New Foundations Explorer\index{New Foundations Explorer},
  constructs mathematics from scratch,
  starting from Quine's New Foundations (NF) set theory axioms.
\item The database
  \texttt{hol.mm}\index{Higher-order Logic database (\texttt{hol.mm})},
  aka the
  Higher-Order Logic (HOL) Explorer\index{Higher-Order Logic (HOL) Explorer},
  starts with HOL (also called simple type theory) and derives
  equivalents to ZFC axioms, connecting the two approaches.
\end{itemize}

Since the days of David Hilbert,\index{Hilbert, David} mathematicians have
been concerned with the fact that the metalanguage\index{metalanguage} used to
describe mathematics may be stronger than the mathematics being described.
Metamath\index{Metamath}'s underlying finitary\index{finitary proof},
constructive nature provides a good philosophical basis for studying even the
weakest logics.\index{weak logic}

The usual treatment of many non-standard formal systems\index{formal
system} uses model theory\index{model theory} or proof theory\index{proof
theory} to describe these systems; these theories, in turn, are based on
standard set theory.  In other words, a non-standard formal system is defined
as a set with certain properties, and standard set theory is used to derive
additional properties of this set.  The standard set theory database provided
with Metamath can be used for this purpose, and when used this way
the development of a special
axiom system for the non-standard formal system becomes unnecessary.  The
model- or proof-theoretic approach often allows you to prove much deeper
results with less effort.

Metamath supports both approaches.  You can define the non-standard
formal system directly, or define the non-standard formal system as
a set with certain properties, whichever you find most helpful.

%\section{Additional Remarks}

\subsection{Metamath and Its Philosophy}

Closely related to Metamath\index{Metamath} is a philosophy or way of looking
at mathematics. This philosophy is related to the formalist
philosophy\index{formalism} of Hilbert\index{Hilbert, David} and his followers
\cite[pp.~1203--1208]{Kline}\index{Kline, Morris}
\cite[p.~6]{Behnke}\index{Behnke, H.}. In this philosophy, mathematics is
viewed as nothing more than a set of rules that manipulate symbols, together
with the consequences of those rules.  While the mathematics being described
may be complex, the rules used to describe it (the
``metamathematics''\index{metamathematics}) should be as simple as possible.
In particular, proofs should be restricted to dealing with concrete objects
(the symbols we write on paper rather than the abstract concepts they
represent) in a constructive manner; these are called ``finitary''
proofs\index{finitary proof} \cite[pp.~2--3]{Shoenfield}\index{Shoenfield,
Joseph R.}.

Whether or not you find Metamath interesting or useful will in part depend on
the appeal you find in its philosophy, and this appeal will probably depend on
your particular goals with respect to mathematics.  For example, if you are a
pure mathematician at the forefront of discovering new mathematical knowledge,
you will probably find that the rigid formality of Metamath stifles your
creativity.  On the other hand, we would argue that once this knowledge is
discovered, there are advantages to documenting it in a standard format that
will make it accessible to others.  Sixty years from now, your field may be
dormant, and as Davis and Hersh put it, your ``writings would become less
translatable than those of the Maya'' \cite[p.~37]{Davis}\index{Davis, Phillip
J.}.


\subsection{A History of the Approach Behind Metamath}

Probably the one work that has had the most motivating influence on
Metamath\index{Metamath} is Whitehead and Russell's monumental {\em Principia
Mathematica} \cite{PM}\index{Whitehead, Alfred North}\index{Russell,
Bertrand}\index{principia mathematica@{\em Principia Mathematica}}, whose aim
was to deduce all of mathematics from a small number of primitive ideas, in a
very explicit way that in principle anyone could understand and follow.  While
this work was tremendously influential in its time, from a modern perspective
it suffers from several drawbacks.  Both its notation and its underlying
axioms are now considered dated and are no longer used.  From our point of
view, its development is not really as accessible as we would like to see; for
practical reasons, proofs become more and more sketchy as its mathematics
progresses, and working them out in fine detail requires a degree of
mathematical skill and patience that many people don't have.  There are
numerous small errors, which is understandable given the tedious, technical
nature of the proofs and the lack of a computer to verify the details.
However, even today {\em Principia Mathematica} stands out as the work closest
in spirit to Metamath.  It remains a mind-boggling work, and one can't help
but be amazed at seeing ``$1+1=2$'' finally appear on page 83 of Volume II
(Theorem *110.643).

The origin of the proof notation used by Metamath dates back to the 1950's,
when the logician C.~A.~Meredith expressed his proofs in a compact notation
called ``condensed detachment''\index{condensed detachment}
\cite{Hindley}\index{Hindley, J. Roger} \cite{Kalman}\index{Kalman, J. A.}
\cite{Meredith}\index{Meredith, C. A.} \cite{Peterson}\index{Peterson, Jeremy
George}.  This notation allows proofs to be communicated unambiguously by
merely referencing the axiom\index{axiom}, rule\index{rule}, or
theorem\index{theorem} used at each step, without explicitly indicating the
substitutions\index{substitution!variable}\index{variable substitution} that
have to be made to the variables in that axiom, rule, or theorem.  Ordinarily,
condensed detachment is more or less limited to propositional
calculus\index{propositional calculus}.  The concept has been extended to
first-order logic\index{first-order logic} in \cite{Megill}\index{Megill,
Norman}, making it is easy to write a small computer program to verify proofs
of simple first-order logic theorems.\index{condensed detachment!and
first-order logic}

A key concept behind the notation of condensed detachment is called
``unification,''\index{unification} which is an algorithm for determining what
substitutions\index{substitution!variable}\index{variable substitution} to
variables have to be made to make two expressions match each other.
Unification was first precisely defined by the logician J.~A.~Robinson, who
used it in the development of a powerful
theorem-proving technique called the ``resolution principle''
\cite{Robinson}\index{Robinson's resolution principle}. Metamath does not make
use of the resolution principle, which is intended for systems of first-order
logic.\index{first-order logic}  Metamath's use is not restricted to
first-order logic, and as we have mentioned it does not automatically discover
proofs.  However, unification is a key idea behind Metamath's proof
notation, and Metamath makes use of a very simple version of it
(Section~\ref{unify}).

\subsection{Metamath and First-Order Logic}

First-order logic\index{first-order logic} is the supporting structure
for standard mathematics.  On top of it is set theory, which contains
the axioms from which virtually all of mathematics can be derived---a
remarkable fact.\index{category
theory}\index{cardinal, inaccessible}\label{categoryth}\footnote{An exception seems
to be category theory.  There are several schools of thought on whether
category theory is derivable from set theory.  At a minimum, it appears
that an additional axiom is needed that asserts the existence of an
``inaccessible cardinal'' (a type of infinity so large that standard set
theory can't prove or deny that it exists).
%
%%%% (I took this out that was in previous editions:)
% But it is also argued that not just one but a ``proper class'' of them
% is needed, and the existence of proper classes is impossible in standard
% set theory.  (A proper class is a collection of sets so huge that no set
% can contain it as an element.  Proper classes can lead to
% inconsistencies such as ``Russell's paradox.''  The axioms of standard
% set theory are devised so as to deny the existence of proper classes.)
%
For more information, see
\cite[pp.~328--331]{Herrlich}\index{Herrlich, Horst} and
\cite{Blass}\index{Blass, Andrea}.}

One of the things that makes Metamath\index{Metamath} more practical for
first-order theories is a set of axioms for first-order logic designed
specifically with Metamath's approach in mind.  These are included in
the database \texttt{set.mm}\index{set theory database (\texttt{set.mm})}.
See Chapter~\ref{fol} for a detailed
description; the axioms are shown in Section~\ref{metaaxioms}.  While
logically equivalent to standard axiom systems, our axiom system breaks
up the standard axioms into smaller pieces such that from them, you can
directly derive what in other systems can only be derived as higher-level
``metatheorems.''\index{metatheorem}  In other words, it is more powerful than
the standard axioms from a metalogical point of view.  A rigorous
justification for this system and its ``metalogical
completeness''\index{metalogical completeness} is found in
\cite{Megill}\index{Megill, Norman}.  The system is closely related to a
system developed by Monk\index{Monk, J. Donald} and Tarski\index{Tarski,
Alfred} in 1965 \cite{Monks}.

For example, the formula $\exists x \, x = y $ (given $y$, there exists some
$x$ equal to it) is a theorem of logic,\footnote{Specifically, it is a theorem
of those systems of logic that assume non-empty domains.  It is not a theorem
of more general systems that include the empty domain\index{empty domain}, in
which nothing exists, period!  Such systems are called ``free
logics.''\index{free logic} For a discussion of these systems, see
\cite{Leblanc}\index{Leblanc, Hugues}.  Since our use for logic is as a basis
for set theory, which has a non-empty domain, it is more convenient (and more
traditional) to use a less general system.  An interesting curiosity is that,
using a free logic as a basis for Zermelo--Fraenkel set
theory\index{Zermelo--Fraenkel set theory} (with the redundant Axiom of the
Null Set omitted),\index{Axiom of the Null Set} we cannot even prove the
existence of a single set without assuming the axiom of infinity!\index{Axiom
of Infinity}} whether or not $x$ and $y$ are distinct variables\index{distinct
variables}.  In many systems of logic, we would have to prove two theorems to
arrive at this result.  First we would prove ``$\exists x \, x = x $,'' then
we would separately prove ``$\exists x \, x = y $, where $x$ and $y$ are
distinct variables.''  We would then combine these two special cases ``outside
of the system'' (i.e.\ in our heads) to be able to claim, ``$\exists x \, x =
y $, regardless of whether $x$ and $y$ are distinct.''  In other words, the
combination of the two special cases is a
metatheorem.  In the system of logic
used in Metamath's set theory\index{set theory database (\texttt{set.mm})}
database, the axioms of logic are broken down into small pieces that allow
them to be reassembled in such a way that theorems such as these can be proved
directly.

Breaking down the axioms in this way makes them look peculiar and not very
intuitive at first, but rest assured that they are correct and complete.  Their
correctness is ensured because they are theorem schemes of standard first-order
logic (which you can easily verify if you are a logician).  Their completeness
follows from the fact that we explicitly derive the standard axioms of
first-order logic as theorems.  Deriving the standard axioms is somewhat
tricky, but once we're there, we have at our disposal a system that is less
awkward to work with in formal proofs\index{formal proof}.  In technical terms
that logicians understand, we eliminate the cumbersome concepts of ``free
variable,''\index{free variable} ``bound variable,''\index{bound variable} and
``proper substitution''\index{proper substitution}\index{substitution!proper}
as primitive notions.  These concepts are present in our system but are
defined in terms of concepts expressed by the axioms and can be eliminated in
principle.  In standard systems, these concepts are really like additional,
implicit axioms\index{implicit axiom} that are somewhat complex and cannot be
eliminated.

The traditional approach to logic, wherein free variables and proper
substitution is defined, is also possible to do directly in the Metamath
language.  However, the notation tends to become awkward, and there are
disadvantages:  for example, extending the definition of a wff with a
definition is awkward, because the free variable and proper substitution
concepts have to have their definitions also extended.  Our choice of
axioms for \texttt{set.mm} is to a certain extent a matter of style, in
an attempt to achieve overall simplicity, but you should also be aware
that the traditional approach is possible as well if you should choose
it.

\chapter{Using the Metamath Program}
\label{using}

\section{Installation}

The way that you install Metamath\index{Metamath!installation} on your
computer system will vary for different computers.  Current
instructions are provided with the Metamath program download at
\url{http://metamath.org}.  In general, the installation is simple.
There is one file containing the Metamath program itself.  This file is
usually called \texttt{metamath} or \texttt{metamath.}{\em xxx} where
{\em xxx} is the convention (such as \texttt{exe}) for an executable
program on your operating system.  There are several additional files
containing samples of the Metamath language, all ending with
\texttt{.mm}.  The file \texttt{set.mm}\index{set theory database
(\texttt{set.mm})} contains logic and set theory and can be used as a
starting point for other areas of mathematics.

You will also need a text editor\index{text editor} capable of editing plain
{\sc ascii}\footnote{American Standard Code for Information Interchange.} text
in order to prepare your input files.\index{ascii@{\sc ascii}}  Most computers
have this capability built in.  Note that plain text is not necessarily the
default for some word processing programs\index{word processor}, especially if
they can handle different fonts; for example, with Microsoft Word\index{Word
(Microsoft)}, you must save the file in the format ``Text Only With Line
Breaks'' to get a plain text\index{plain text} file.\footnote{It is
recommended that all lines in a Metamath source file be 79 characters or less
in length for compatibility among different computer terminals.  When creating
a source file on an editor such as Word, select a monospaced
font\index{monospaced font} such as Courier\index{Courier font} or
Monaco\index{Monaco font} to make this easier to achieve.  Better yet,
just use a plain text editor such as Notepad.}

On some computer systems, Metamath does not have the capability to print
its output directly; instead, you send its output to a file (using the
\texttt{open} commands described later).  The way you print this output
file depends on your computer.\index{printers} Some computers have a
print command, whereas with others, you may have to read the file into
an editor and print it from there.

If you want to print your Metamath source files with typeset formulas
containing standard mathematical symbols, you will need the \LaTeX\
typesetting program\index{latex@{\LaTeX}}, which is widely and freely
available for most operating systems.  It runs natively on Unix and
Linux, and can be installed on Windows as part of the free Cygwin
package (\url{http://cygwin.com}).

You can also produce {\sc html}\footnote{HyperText Markup Language.}
web pages.  The {\tt help html} command in the Metamath program will
assist you with this feature.

\section{Your First Formal System}\label{start}
\subsection{From Nothing to Zero}\label{startf}

To give you a feel for what the Metamath\index{Metamath} language looks like,
we will take a look at a very simple example from formal number
theory\index{number theory}.  This example is taken from
Mendelson\index{Mendelson, Elliot} \cite[p. 123]{Mendelson}.\footnote{To keep
the example simple, we have changed the formalism slightly, and what we call
axioms\index{axiom} are strictly speaking theorems\index{theorem} in
\cite{Mendelson}.}  We will look at a small subset of this theory, namely that
part needed for the first number theory theorem proved in \cite{Mendelson}.

First we will look at a standard formal proof\index{formal proof} for the
example we have picked, then we will look at the Metamath version.  If you
have never been exposed to formal proofs, the notation may seem to be such
overkill to express such simple notions that you may wonder if you are missing
something.  You aren't.  The concepts involved are in fact very simple, and a
detailed breakdown in this fashion is necessary to express the proof in a way
that can be verified mechanically.  And as you will see, Metamath breaks the
proof down into even finer pieces so that the mechanical verification process
can be about as simple as possible.

Before we can introduce the axioms\index{axiom} of the theory, we must define
the syntax rules for forming legal expressions\index{syntax rules}
(combinations of symbols) with which those axioms can be used. The number 0 is
a {\bf term}\index{term}; and if $ t$ and $r$ are terms, so is $(t+r)$. Here,
$ t$ and $r$ are ``metavariables''\index{metavariable} ranging over terms; they
themselves do not appear as symbols in an actual term.  Some examples of
actual terms are $(0 + 0)$ and $((0+0)+0)$.  (Note that our theory describes
only the number zero and sums of zeroes.  Of course, not much can be done with
such a trivial theory, but remember that we have picked a very small subset of
complete number theory for our example.  The important thing for you to focus
on is our definitions that describe how symbols are combined to form valid
expressions, and not on the content or meaning of those expressions.) If $ t$
and $r$ are terms, an expression of the form $ t=r$ is a {\bf wff}
(well-formed formula)\index{well-formed formula (wff)}; and if $P$ and $Q$ are
wffs, so is $(P\rightarrow Q)$ (which means ``$P$ implies
$Q$''\index{implication ($\rightarrow$)} or ``if $P$ then $Q$'').
Here $P$ and $Q$ are metavariables ranging over wffs.  Examples of actual
wffs are $0=0$, $(0+0)=0$, $(0=0 \rightarrow (0+0)=0)$, and $(0=0\rightarrow
(0=0\rightarrow 0=(0+0)))$.  (Our notation makes use of more parentheses than
are customary, but the elimination of ambiguity this way simplifies our
example by avoiding the need to define operator precedence\index{operator
precedence}.)

The {\bf axioms}\index{axiom} of our theory are all wffs of the following
form, where $ t$, $r$, and $s$ are any terms:

%Latex p. 92
\renewcommand{\theequation}{A\arabic{equation}}

\begin{equation}
(t=r\rightarrow (t=s\rightarrow r=s))
\end{equation}
\begin{equation}
(t+0)=t
\end{equation}

Note that there are an infinite number of axioms since there are an infinite
number of possible terms.  A1 and A2 are properly called ``axiom
schemes,''\index{axiom scheme} but we will refer to them as ``axioms'' for
brevity.

An axiom is a {\bf theorem}; and if $P$ and $(P\rightarrow Q)$ are theorems
(where $P$ and $Q$ are wffs), then $Q$ is also a theorem.\index{theorem}  The
second part of this definition is called the modus ponens (MP) rule of
inference\index{inference rule}\index{modus ponens}.  It allows us to obtain
new theorems from old ones.

The {\bf proof}\index{proof} of a theorem is a sequence of one or more
theorems, each of which is either an axiom or the result of modus ponens
applied to two previous theorems in the sequence, and the last of which is the
theorem being proved.

The theorem we will prove for our example is very simple:  $ t=t$.  The proof of
our theorem follows.  Study it carefully until you feel sure you
understand it.\label{zeroproof}

% Use tabu so that lines will wrap automatically as needed.
\begin{tabu} { l X X }
1. & $(t+0)=t$ & (by axiom A2) \\
2. & $(t+0)=t$ & (by axiom A2) \\
3. & $((t+0)=t \rightarrow ((t+0)=t\rightarrow t=t))$ & (by axiom A1) \\
4. & $((t+0)=t\rightarrow t=t)$ & (by MP applied to steps 2 and 3) \\
5. & $t=t$ & (by MP applied to steps 1 and 4) \\
\end{tabu}

(You may wonder why step 1 is repeated twice.  This is not necessary in the
formal language we have defined, but in Metamath's ``reverse Polish
notation''\index{reverse Polish notation (RPN)} for proofs, a previous step
can be referred to only once.  The repetition of step~1 here will enable you
to see more clearly the correspondence of this proof with the
Metamath\index{Metamath} version on p.~\pageref{demoproof}.)

Our theorem is more properly called a ``theorem scheme,''\index{theorem
scheme} for it represents an infinite number of theorems, one for each
possible term $ t$.  Two examples of actual theorems would be $0=0$ and
$(0+0)=(0+0)$.  Rarely do we prove actual theorems, since by proving schemes
we can prove an infinite number of theorems in one fell swoop.  Similarly, our
proof should really be called a ``proof scheme.''\index{proof scheme}  To
obtain an actual proof, pick an actual term to use in place of $ t$, and
substitute it for $ t$ throughout the proof.

Let's discuss what we have done here.  The axioms\index{axiom} of our theory,
A1 and A2, are trivial and obvious.  Everyone knows that adding zero to
something doesn't change it, and also that if two things are equal to a third,
then they are equal to each other. In fact, stating the trivial and obvious is
a goal to strive for in any axiomatic system.  From trivial and obvious truths
that everyone agrees upon, we can prove results that are not so obvious yet
have absolute faith in them.  If we trust the axioms and the rules, we must,
by definition, trust the consequences of those axioms and rules, if logic is
to mean anything at all.

Our rule of inference\index{rule}, modus ponens\index{modus ponens}, is also
pretty obvious once you understand what it means.  If we prove a fact $P$, and
we also prove that $P$ implies $Q$, then $Q$ necessarily follows as a new
fact.  The rule provides us with a means for obtaining new facts (i.e.\
theorems\index{theorem}) from old ones.

The theorem that we have proved, $ t=t$, is so fundamental that you may wonder
why it isn't one of the axioms\index{axiom}.  In some axiom systems of
arithmetic, it {\em is} an axiom.  The choice of axioms in a theory is to some
extent arbitrary and even an art form, constrained only by the requirement
that any two equivalent axiom systems be able to derive each other as
theorems.  We could imagine that the inventor of our axiom system originally
included $ t=t$ as an axiom, then discovered that it could be derived as a
theorem from the other axioms.  Because of this, it was not necessary to
keep it as an axiom.  By eliminating it, the final set of axioms became
that much simpler.

Unless you have worked with formal proofs\index{formal proof} before, it
probably wasn't apparent to you that $ t=t$ could be derived from our two
axioms until you saw the proof. While you certainly believe that $ t=t$ is
true, you might not be able to convince an imaginary skeptic who believes only
in our two axioms until you produce the proof.  Formal proofs such as this are
hard to come up with when you first start working with them, but after you get
used to them they can become interesting and fun.  Once you understand the
idea behind formal proofs you will have grasped the fundamental principle that
underlies all of mathematics.  As the mathematics becomes more sophisticated,
its proofs become more challenging, but ultimately they all can be broken down
into individual steps as simple as the ones in our proof above.

Mendelson's\index{Mendelson, Elliot} book, from which our example was taken,
contains a number of detailed formal proofs such as these, and you may be
interested in looking it up.  The book is intended for mathematicians,
however, and most of it is rather advanced.  Popular literature describing
formal proofs\index{formal proof} include \cite[p.~296]{Rucker}\index{Rucker,
Rudy} and \cite[pp.~204--230]{Hofstadter}\index{Hofstadter, Douglas R.}.

\subsection{Converting It to Metamath}\label{convert}

Formal proofs\index{formal proof} such as the one in our example break down
logical reasoning into small, precise steps that leave little doubt that the
results follow from the axioms\index{axiom}.  You might think that this would
be the finest breakdown we can achieve in mathematics.  However, there is more
to the proof than meets the eye. Although our axioms were rather simple, a lot
of verbiage was needed before we could even state them:  we needed to define
``term,'' ``wff,'' and so on.  In addition, there are a number of implied
rules that we haven't even mentioned. For example, how do we know that step 3
of our proof follows from axiom A1? There is some hidden reasoning involved in
determining this.  Axiom A1 has two occurrences of the letter $ t$.  One of
the implied rules states that whatever we substitute for $ t$ must be a legal
term\index{term}.\footnote{Some authors make this implied rule explicit by
stating, ``only expressions of the above form are terms,'' after defining
``term.''}  The expression $ t+0$ is pretty obviously a legal term whenever $
t$ is, but suppose we wanted to substitute a huge term with thousands of
symbols?  Certainly a lot of work would be involved in determining that it
really is a term, but in ordinary formal proofs all of this work would be
considered a single ``step.''

To express our axiom system in the Metamath\index{Metamath} language, we must
describe this auxiliary information in addition to the axioms themselves.
Metamath does not know what a ``term'' or a ``wff''\index{well-formed formula
(wff)} is.  In Metamath, the specification of the ways in which we can combine
symbols to obtain terms and wffs are like little axioms in themselves.  These
auxiliary axioms are expressed in the same notation as the ``real''
axioms\index{axiom}, and Metamath does not distinguish between the two.  The
distinction is made by you, i.e.\ by the way in which you interpret the
notation you have chosen to express these two kinds of axioms.

The Metamath language breaks down mathematical proofs into tiny pieces, much
more so than in ordinary formal proofs\index{formal proof}.  If a single
step\index{proof step} involves the
substitution\index{substitution!variable}\index{variable substitution} of a
complex term for one of its variables, Metamath must see this single step
broken down into many small steps.  This fine-grained breakdown is what gives
Metamath generality and flexibility as it lets it not be limited to any
particular mathematical notation.

Metamath's proof notation is not, in itself, intended to be read by humans but
rather is in a compact format intended for a machine.  The Metamath program
will convert this notation to a form you can understand, using the \texttt{show
proof}\index{\texttt{show proof} command} command.  You can tell the program what
level of detail of the proof you want to look at.  You may want to look at
just the logical inference steps that correspond
to ordinary formal proof steps,
or you may want to see the fine-grained steps that prove that an expression is
a term.

Here, without further ado, is our example converted to the
Metamath\index{Metamath} language:\index{metavariable}\label{demo0}

\begin{verbatim}
$( Declare the constant symbols we will use $)
    $c 0 + = -> ( ) term wff |- $.
$( Declare the metavariables we will use $)
    $v t r s P Q $.
$( Specify properties of the metavariables $)
    tt $f term t $.
    tr $f term r $.
    ts $f term s $.
    wp $f wff P $.
    wq $f wff Q $.
$( Define "term" and "wff" $)
    tze $a term 0 $.
    tpl $a term ( t + r ) $.
    weq $a wff t = r $.
    wim $a wff ( P -> Q ) $.
$( State the axioms $)
    a1 $a |- ( t = r -> ( t = s -> r = s ) ) $.
    a2 $a |- ( t + 0 ) = t $.
$( Define the modus ponens inference rule $)
    ${
       min $e |- P $.
       maj $e |- ( P -> Q ) $.
       mp  $a |- Q $.
    $}
$( Prove a theorem $)
    th1 $p |- t = t $=
  $( Here is its proof: $)
       tt tze tpl tt weq tt tt weq tt a2 tt tze tpl
       tt weq tt tze tpl tt weq tt tt weq wim tt a2
       tt tze tpl tt tt a1 mp mp
     $.
\end{verbatim}\index{metavariable}

A ``database''\index{database} is a set of one or more {\sc ascii} source
files.  Here's a brief description of this Metamath\index{Metamath} database
(which consists of this single source file), so that you can understand in
general terms what is going on.  To understand the source file in detail, you
should read Chapter~\ref{languagespec}.

The database is a sequence of ``tokens,''\index{token} which are normally
separated by spaces or line breaks.  The only tokens that are built into
the Metamath language are those beginning with \texttt{\$}.  These tokens
are called ``keywords.''\index{keyword}  All other tokens are
user-defined, and their names are arbitrary.

As you might have guessed, the Metamath token \texttt{\$(}\index{\texttt{\$(} and
\texttt{\$)} auxiliary keywords} starts a comment and \texttt{\$)} ends a comment.

The Metamath tokens \texttt{\$c}\index{\texttt{\$c} statement},
\texttt{\$v}\index{\texttt{\$v} statement},
\texttt{\$e}\index{\texttt{\$e} statement},
\texttt{\$f}\index{\texttt{\$f} statement},
\texttt{\$a}\index{\texttt{\$a} statement}, and
\texttt{\$p}\index{\texttt{\$p} statement} specify ``statements'' that
end with \texttt{\$.}\,.\index{\texttt{\$.}\ keyword}

The Metamath tokens \texttt{\$c} and \texttt{\$v} each declare\index{constant
declaration}\index{variable declaration} a list of user-defined tokens, called
``math symbols,''\index{math symbol} that the database will reference later
on.  All of the math symbols they define you have seen earlier except the
turnstile symbol \texttt{|-} ($\vdash$)\index{turnstile ({$\,\vdash$})}, which is
commonly used by logicians to mean ``a proof exists for.''  For us
the turnstile is just a
convenient symbol that distinguishes expressions that are axioms\index{axiom}
or theorems\index{theorem} from expressions that are terms or wffs.

The \texttt{\$c} statement declares ``constants''\index{constant} and
the \texttt{\$v} statement declares
``variables''\index{variable}\index{constant declaration}\index{variable
declaration} (or more precisely, metavariables\index{metavariable}).  A
variable may be substituted\index{substitution!variable}\index{variable
substitution} with sequences of math symbols whereas a constant may not
be substituted with anything.

It may seem redundant to require both \texttt{\$c}\index{\texttt{\$c} statement} and
\texttt{\$v}\index{\texttt{\$v} statement} statements (since any math
symbol\index{math symbol} not specified with a \texttt{\$c} statement could be
presumed to be a variable), but this provides for better error checking and
also allows math symbols to be redeclared\index{redeclaration of symbols}
(Section~\ref{scoping}).

The token \texttt{\$f}\index{\texttt{\$f} statement} specifies a
statement called a ``variable-type hypothesis'' (also called a
``floating hypothesis'') and \texttt{\$e}\index{\texttt{\$e} statement}
specifies a ``logical hypothesis'' (also called an ``essential
hypothesis'').\index{hypothesis}\index{variable-type
hypothesis}\index{logical hypothesis}\index{floating
hypothesis}\index{essential hypothesis} The token
\texttt{\$a}\index{\texttt{\$a} statement} specifies an ``axiomatic
assertion,''\index{axiomatic assertion} and
\texttt{\$p}\index{\texttt{\$p} statement} specifies a ``provable
assertion.''\index{provable assertion} To the left of each occurrence of
these four tokens is a ``label''\index{label} that identifies the
hypothesis or assertion for later reference.  For example, the label of
the first axiomatic assertion is \texttt{tze}.  A \texttt{\$f} statement
must contain exactly two math symbols, a constant followed by a
variable.  The \texttt{\$e}, \texttt{\$a}, and \texttt{\$p} statements
each start with a constant followed by, in general, an arbitrary
sequence of math symbols.

Associated with each assertion\index{assertion} is a set of hypotheses
that must be satisfied in order for the assertion to be used in a proof.
These are called the ``mandatory hypotheses''\index{mandatory
hypothesis} of the assertion.  Among those hypotheses whose ``scope''
(described below) includes the assertion, \texttt{\$e} hypotheses are
always mandatory and \texttt{\$f}\index{\texttt{\$f} statement}
hypotheses are mandatory when they share their variable with the
assertion or its \texttt{\$e} hypotheses.  The exact rules for
determining which hypotheses are mandatory are described in detail in
Sections~\ref{frames} and \ref{scoping}.  For example, the mandatory
hypotheses of assertion \texttt{tpl} are \texttt{tt} and \texttt{tr},
whereas assertion \texttt{tze} has no mandatory hypotheses because it
contains no variables and has no \texttt{\$e}\index{\texttt{\$e}
statement} hypothesis.  Metamath's \texttt{show statement}
command\index{\texttt{show statement} command}, described in the next
section, will show you a statement's mandatory hypotheses.

Sometimes we need to make a hypothesis relevant to only certain
assertions.  The set of statements to which a hypothesis is relevant is
called its ``scope.''  The Metamath brackets,
\texttt{\$\char`\{}\index{\texttt{\$\char`\{} and \texttt{\$\char`\}}
keywords} and \texttt{\$\char`\}}, define a ``block''\index{block} that
delimits the scope of any hypothesis contained between them.  The
assertion \texttt{mp} has mandatory hypotheses \texttt{wp}, \texttt{wq},
\texttt{min}, and \texttt{maj}.  The only mandatory hypothesis of
\texttt{th1}, on the other hand, is \texttt{tt}, since \texttt{th1}
occurs outside of the block containing \texttt{min} and \texttt{maj}.

Note that \texttt{\$\char`\{} and \texttt{\$\char`\}} do not affect the
scope of assertions (\texttt{\$a} and \texttt{\$p}).  Assertions are always
available to be referenced by any later proof in the source file.

Each provable assertion (\texttt{\$p}\index{\texttt{\$p} statement}
statement) has two parts.  The first part is the
assertion\index{assertion} itself, which is a sequence of math
symbol\index{math symbol} tokens placed between the \texttt{\$p} token
and a \texttt{\$=}\index{\texttt{\$=} keyword} token.  The second part
is a ``proof,'' which is a list of label tokens placed between the
\texttt{\$=} token and the \texttt{\$.}\index{\texttt{\$.}\ keyword}\
token that ends the statement.\footnote{If you've looked at the
\texttt{set.mm} database, you may have noticed another notation used for
proofs.  The other notation is called ``compressed.''\index{compressed
proof}\index{proof!compressed} It reduces the amount of space needed to
store a proof in the database and is described in
Appendix~\ref{compressed}.  In the example above, we use
``normal''\index{normal proof}\index{proof!normal} notation.} The proof
acts as a series of instructions to the Metamath program, telling it how
to build up the sequence of math symbols contained in the assertion part of
the \texttt{\$p} statement, making use of the hypotheses of the
\texttt{\$p} statement and previous assertions.  The construction takes
place according to precise rules.  If the list of labels in the proof
causes these rules to be violated, or if the final sequence that results
does not match the assertion, the Metamath program will notify you with
an error message.

If you are familiar with reverse Polish notation (RPN), which is sometimes used
on pocket calculators, here in a nutshell is how a proof works.  Each
hypothesis label\index{hypothesis label} in the proof is pushed\index{push}
onto the RPN stack\index{stack}\index{RPN stack} as it is encountered. Each
assertion label\index{assertion label} pops\index{pop} off the stack as many
entries as the referenced assertion has mandatory hypotheses.  Variable
substitutions\index{substitution!variable}\index{variable substitution} are
computed which, when made to the referenced assertion's mandatory hypotheses,
cause these hypotheses to match the stack entries. These same substitutions
are then made to the variables in the referenced assertion itself, which is
then pushed onto the stack.  At the end of the proof, there should be one
stack entry, namely the assertion being proved.  This process is explained in
detail in Section~\ref{proof}.

Metamath's proof notation is not very readable for humans, but it allows the
proof to be stored compactly in a file.  The Metamath\index{Metamath} program
has proof display features that let you see what's going on in a more
readable way, as you will see in the next section.

The rules used in verifying a proof are not based on any built-in syntax of the
symbol sequence in an assertion\index{assertion} nor on any built-in meanings
attached to specific symbol names.  They are based strictly on symbol
matching:  constants\index{constant} must match themselves, and
variables\index{variable} may be replaced with anything that allows a match to
occur.  For example, instead of \texttt{term}, \texttt{0}, and \verb$|-$ we could
have just as well used \texttt{yellow}, \texttt{zero}, and \texttt{provable}, as long
as we did so consistently throughout the database.  Also, we could have used
\texttt{is provable} (two tokens) instead of \verb$|-$ (one token) throughout the
database.  In each of these cases, the proof would be exactly the same.  The
independence of proofs and notation means that you have a lot of flexibility to
change the notation you use without having to change any proofs.

\section{A Trial Run}\label{trialrun}

Now you are ready to try out the Metamath\index{Metamath} program.

On all computer systems, Metamath has a standard ``command line
interface'' (CLI)\index{command line interface (CLI)} that allows you to
interact with it.  You supply commands to the CLI by typing them on the
keyboard and pressing your keyboard's {\em return} key after each line
you enter.  The CLI is designed to be easy to use and has built-in help
features.

The first thing you should do is to use a text editor to create a file
called \texttt{demo0.mm} and type into it the Metamath source shown on
p.~\pageref{demo0}.  Actually, this file is included with your Metamath
software package, so check that first.  If you type it in, make sure
that you save it in the form of ``plain {\sc ascii} text with line
breaks.''  Most word processors will have this feature.

Next you must run the Metamath program.  Depending on your computer
system and how Metamath is installed, this could range from clicking the
mouse on the Metamath icon to typing \texttt{run metamath} to typing
simply \texttt{metamath}.  (Metamath's {\tt help invoke} command describes
alternate ways of invoking the Metamath program.)

When you first enter Metamath\index{Metamath}, it will be at the CLI, waiting
for your input. You will see something like the following on your screen:
\begin{verbatim}
Metamath - Version 0.177 27-Apr-2019
Type HELP for help, EXIT to exit.
MM>
\end{verbatim}
The \texttt{MM>} prompt means that Metamath is waiting for a command.
Command keywords\index{command keyword} are not case sensitive;
we will use lower-case commands in our examples.
The version number and its release date will probably be different on your
system from the one we show above.

The first thing that you need to do is to read in your
database:\index{\texttt{read} command}\footnote{If a directory path is
needed on Unix,\index{Unix file names}\index{file names!Unix} you should
enclose the path/file name in quotes to prevent Metamath from thinking
that the \texttt{/} in the path name is a command qualifier, e.g.,
\texttt{read \char`\"db/set.mm\char`\"}.  Quotes are optional when there
is no ambiguity.}
\begin{verbatim}
MM> read demo0.mm
\end{verbatim}
Remember to press the {\em return} key after entering this command.  If
you omit the file name, Metamath will prompt you for one.   The syntax for
specifying a Macintosh file name path is given in a footnote on
p.~\pageref{includef}.\index{Macintosh file names}\index{file
names!Macintosh}

If there are any syntax errors in the database, Metamath will let you know
when it reads in the file.  The one thing that Metamath does not check when
reading in a database is that all proofs are correct, because this would
slow it down too much.  It is a good idea to periodically verify the proofs in
a database you are making changes to.  To do this, use the following command
(and do it for your \texttt{demo0.mm} file now).  Note that the \texttt{*} is a
``wild card'' meaning all proofs in the file.\index{\texttt{verify proof} command}
\begin{verbatim}
MM> verify proof *
\end{verbatim}
Metamath will report any proofs that are incorrect.

It is often useful to save the information that the Metamath program displays
on the screen. You can save everything that happens on the screen by opening a
log file. You may want to do this before you read in a database so that you
can examine any errors later on.  To open a log file, type
\begin{verbatim}
MM> open log abc.log
\end{verbatim}
This will open a file called \texttt{abc.log}, and everything that appears on the
screen from this point on will be stored in this file.  The name of the log file
is arbitrary. To close the log file, type
\begin{verbatim}
MM> close log
\end{verbatim}

Several commands let you examine what's inside your database.
Section~\ref{exploring} has an overview of some useful ones.  The
\texttt{show labels} command lets you see what statement
labels\index{label} exist.  A \texttt{*} matches any combination of
characters, and \texttt{t*} refers to all labels starting with the
letter \texttt{t}.\index{\texttt{show labels} command} The \texttt{/all}
is a ``command qualifier''\index{command qualifier} that tells Metamath
to include labels of hypotheses.  (To see the syntax explained, type
\texttt{help show labels}.)  Type
\begin{verbatim}
MM> show labels t* /all
\end{verbatim}
Metamath will respond with
\begin{verbatim}
The statement number, label, and type are shown.
3 tt $f       4 tr $f       5 ts $f       8 tze $a
9 tpl $a      19 th1 $p
\end{verbatim}

You can use the \texttt{show statement} command to get information about a
particular statement.\index{\texttt{show statement} command}
For example, you can get information about the statement with label \texttt{mp}
by typing
\begin{verbatim}
MM> show statement mp /full
\end{verbatim}
Metamath will respond with
\begin{verbatim}
Statement 17 is located on line 43 of the file
"demo0.mm".
"Define the modus ponens inference rule"
17 mp $a |- Q $.
Its mandatory hypotheses in RPN order are:
  wp $f wff P $.
  wq $f wff Q $.
  min $e |- P $.
  maj $e |- ( P -> Q ) $.
The statement and its hypotheses require the
      variables:  Q P
The variables it contains are:  Q P
\end{verbatim}
The mandatory hypotheses\index{mandatory hypothesis} and their
order\index{RPN order} are
useful to know when you are trying to understand or debug a proof.

Now you are ready to look at what's really inside our proof.  First, here is
how to look at every step in the proof---not just the ones corresponding to an
ordinary formal proof\index{formal proof}, but also the ones that build up the
formulas that appear in each ordinary formal proof step.\index{\texttt{show
proof} command}
\begin{verbatim}
MM> show proof th1 /lemmon /all
\end{verbatim}

This will display the proof on the screen in the following format:
\begin{verbatim}
 1 tt            $f term t
 2 tze           $a term 0
 3 1,2 tpl       $a term ( t + 0 )
 4 tt            $f term t
 5 3,4 weq       $a wff ( t + 0 ) = t
 6 tt            $f term t
 7 tt            $f term t
 8 6,7 weq       $a wff t = t
 9 tt            $f term t
10 9 a2          $a |- ( t + 0 ) = t
11 tt            $f term t
12 tze           $a term 0
13 11,12 tpl     $a term ( t + 0 )
14 tt            $f term t
15 13,14 weq     $a wff ( t + 0 ) = t
16 tt            $f term t
17 tze           $a term 0
18 16,17 tpl     $a term ( t + 0 )
19 tt            $f term t
20 18,19 weq     $a wff ( t + 0 ) = t
21 tt            $f term t
22 tt            $f term t
23 21,22 weq     $a wff t = t
24 20,23 wim     $a wff ( ( t + 0 ) = t -> t = t )
25 tt            $f term t
26 25 a2         $a |- ( t + 0 ) = t
27 tt            $f term t
28 tze           $a term 0
29 27,28 tpl     $a term ( t + 0 )
30 tt            $f term t
31 tt            $f term t
32 29,30,31 a1   $a |- ( ( t + 0 ) = t -> ( ( t + 0 )
                                     = t -> t = t ) )
33 15,24,26,32 mp  $a |- ( ( t + 0 ) = t -> t = t )
34 5,8,10,33 mp  $a |- t = t
\end{verbatim}

The \texttt{/lemmon} command qualifier specifies what is known as a Lemmon-style
display\index{Lemmon-style proof}\index{proof!Lemmon-style}.  Omitting the
\texttt{/lemmon} qualifier results in a tree-style proof (see
p.~\pageref{treeproof} for an example) that is somewhat less explicit but
easier to follow once you get used to it.\index{tree-style
proof}\index{proof!tree-style}

The first number on each line is the step
number of the proof.  Any numbers that follow are step numbers assigned to the
hypotheses of the statement referenced by that step.  Next is the label of
the statement referenced by the step.  The statement type of the statement
referenced comes next, followed by the math symbol\index{math symbol} string
constructed by the proof up to that step.

The last step, 34, contains the statement that is being proved.

Looking at a small piece of the proof, notice that steps 3 and 4 have
established that
\texttt{( t + 0 )} and \texttt{t} are \texttt{term}\,s, and step 5 makes use of steps 3 and
4 to establish that \texttt{( t + 0 ) = t} is a \texttt{wff}.  Let Metamath
itself tell us in detail what is happening in step 5.  Note that the
``target hypothesis'' refers to where step 5 is eventually used, i.e., in step
34.
\begin{verbatim}
MM> show proof th1 /detailed_step 5
Proof step 5:  wp=weq $a wff ( t + 0 ) = t
This step assigns source "weq" ($a) to target "wp"
($f).  The source assertion requires the hypotheses
"tt" ($f, step 3) and "tr" ($f, step 4).  The parent
assertion of the target hypothesis is "mp" ($a,
step 34).
The source assertion before substitution was:
    weq $a wff t = r
The following substitutions were made to the source
assertion:
    Variable  Substituted with
     t         ( t + 0 )
     r         t
The target hypothesis before substitution was:
    wp $f wff P
The following substitution was made to the target
hypothesis:
    Variable  Substituted with
     P         ( t + 0 ) = t
\end{verbatim}

The full proof just shown is useful to understand what is going on in detail.
However, most of the time you will just be interested in
the ``essential'' or logical steps of a proof, i.e.\ those steps
that correspond to an
ordinary formal proof\index{formal proof}.  If you type
\begin{verbatim}
MM> show proof th1 /lemmon /renumber
\end{verbatim}
you will see\label{demoproof}
\begin{verbatim}
1 a2             $a |- ( t + 0 ) = t
2 a2             $a |- ( t + 0 ) = t
3 a1             $a |- ( ( t + 0 ) = t -> ( ( t + 0 )
                                     = t -> t = t ) )
4 2,3 mp         $a |- ( ( t + 0 ) = t -> t = t )
5 1,4 mp         $a |- t = t
\end{verbatim}
Compare this to the formal proof on p.~\pageref{zeroproof} and
notice the resemblance.
By default Metamath
does not show \texttt{\$f}\index{\texttt{\$f}
statement} hypotheses and everything branching off of them in the proof tree
when the proof is displayed; this makes the proof look more like an ordinary
mathematical proof, which does not normally incorporate the explicit
construction of expressions.
This is called the ``essential'' view
(at one time you had to add the
\texttt{/essential} qualifier in the \texttt{show proof}
command to get this view, but this is now the default).
You can could use the \texttt{/all} qualifier in the \texttt{show
proof} command to also show the explicit construction of expressions.
The \texttt{/renumber} qualifier means to renumber
the steps to correspond only to what is displayed.\index{\texttt{show proof}
command}

To exit Metamath, type\index{\texttt{exit} command}
\begin{verbatim}
MM> exit
\end{verbatim}

\subsection{Some Hints for Using the Command Line Interface}

We will conclude this quick introduction to Metamath\index{Metamath} with some
helpful hints on how to navigate your way through the commands.
\index{command line interface (CLI)}

When you type commands into Metamath's CLI, you only have to type as many
characters of a command keyword\index{command keyword} as are needed to make
it unambiguous.  If you type too few characters, Metamath will tell you what
the choices are.  In the case of the \texttt{read} command, only the \texttt{r} is
needed to specify it unambiguously, so you could have typed\index{\texttt{read}
command}
\begin{verbatim}
MM> r demo0.mm
\end{verbatim}
instead of
\begin{verbatim}
MM> read demo0.mm
\end{verbatim}
In our description, we always show the full command words.  When using the
Metamath CLI commands in a command file (to be read with the \texttt{submit}
command)\index{\texttt{submit} command}, it is good practice to use
the unabbreviated command to ensure your instructions will not become ambiguous
if more commands are added to the Metamath program in the future.

The command keywords\index{command
keyword} are not case sensitive; you may type either \texttt{read} or
\texttt{ReAd}.  File names may or may not be case sensitive, depending on your
computer's operating system.  Metamath label\index{label} and math
symbol\index{math symbol} tokens\index{token} are case-sensitive.

The \texttt{help} command\index{\texttt{help} command} will provide you
with a list of topics you can get help on.  You can then type
\texttt{help} {\em topic} to get help on that topic.

If you are uncertain of a command's spelling, just type as many characters
as you remember of the command.  If you have not typed enough characters to
specify it unambiguously, Metamath will tell you what choices you have.

\begin{verbatim}
MM> show s
         ^
?Ambiguous keyword - please specify SETTINGS,
STATEMENT, or SOURCE.
\end{verbatim}

If you don't know what argument to use as part of a command, type a
\texttt{?}\index{\texttt{]}@\texttt{?}\ in command lines}\ at the
argument position.  Metamath will tell you what it expected there.

\begin{verbatim}
MM> show ?
         ^
?Expected SETTINGS, LABELS, STATEMENT, SOURCE, PROOF,
MEMORY, TRACE_BACK, or USAGE.
\end{verbatim}

Finally, you may type just the first word or words of a command followed
by {\em return}.  Metamath will prompt you for the remaining part of the
command, showing you the choices at each step.  For example, instead of
typing \texttt{show statement th1 /full} you could interact in the
following manner:
\begin{verbatim}
MM> show
SETTINGS, LABELS, STATEMENT, SOURCE, PROOF,
MEMORY, TRACE_BACK, or USAGE <SETTINGS>? st
What is the statement label <th1>?
/ or nothing <nothing>? /
TEX, COMMENT_ONLY, or FULL <TEX>? f
/ or nothing <nothing>?
19 th1 $p |- t = t $= ... $.
\end{verbatim}
After each \texttt{?}\ in this mode, you must give Metamath the
information it requests.  Sometimes Metamath gives you a list of choices
with the default choice indicated by brackets \texttt{< > }. Pressing
{\em return} after the \texttt{?}\ will select the default choice.
Answering anything else will override the default.  Note that the
\texttt{/} in command qualifiers is considered a separate
token\index{token} by the parser, and this is why it is asked for
separately.

\section{Your First Proof}\label{frstprf}

Proofs are developed with the aid of the Proof Assistant\index{Proof
Assistant}.  We will now show you how the proof of theorem \texttt{th1}
was built.  So that you can repeat these steps, we will first have the
Proof Assistant erase the proof in Metamath's source buffer\index{source
buffer}, then reconstruct it.  (The source buffer is the place in memory
where Metamath stores the information in the database when it is
\texttt{read}\index{\texttt{read} command} in.  New or modified proofs
are kept in the source buffer until a \texttt{write source}
command\index{\texttt{write source} command} is issued.)  In practice, you
would place a \texttt{?}\index{\texttt{]}@\texttt{?}\ inside proofs}\
between \texttt{\$=}\index{\texttt{\$=} keyword} and
\texttt{\$.}\index{\texttt{\$.}\ keyword}\ in the database to indicate
to Metamath\index{Metamath} that the proof is unknown, and that would be
your starting point.  Whenever the \texttt{verify proof} command encounters
a proof with a \texttt{?}\ in place of a proof step, the statement is
identified as not proved.

When I first started creating Metamath proofs, I would write down
on a piece of paper the complete
formal proof\index{formal proof} as it would appear
in a \texttt{show proof} command\index{\texttt{show proof} command}; see
the display of \texttt{show proof th1 /lemmon /re\-num\-ber} above as an
example.  After you get used to using the Proof Assistant\index{Proof
Assistant} you may get to a point where you can ``see'' the proof in your mind
and let the Proof Assistant guide you in filling in the details, at least for
simpler proofs, but until you gain that experience you may find it very useful
to write down all the details in advance.
Otherwise you may waste a lot of time as you let it take you down a wrong path.
However, others do not find this approach as helpful.
For example, Thomas Brendan Leahy\index{Leahy, Thomas Brendan}
finds that it is more helpful to him to interactively
work backward from a machine-readable statement.
David A. Wheeler\index{Wheeler, David A.}
writes down a general approach, but develops the proof
interactively by switching between
working forwards (from hypotheses and facts likely to be useful) and
backwards (from the goal) until the forwards and backwards approaches meet.
In the end, use whatever approach works for you.

A proof is developed with the Proof Assistant by working backwards, starting
with the theorem\index{theorem} to be proved, and assigning each unknown step
with a theorem or hypothesis until no more unknown steps remain.  The Proof
Assistant will not let you make an assignment unless it can be ``unified''
with the unknown step.  This means that a
substitution\index{substitution!variable}\index{variable substitution} of
variables exists that will make the assignment match the unknown step.  On the
other hand, in the middle of a proof, when working backwards, often more than
one unification\index{unification} (set of substitutions) is possible, since
there is not enough information available at that point to uniquely establish
it.  In this case you can tell Metamath which unification to choose, or you
can continue to assign unknown steps until enough information is available to
make the unification unique.

We will assume you have entered Metamath and read in the database as described
above.  The following dialog shows how the proof was developed.  For more
details on what some of the commands do, refer to Section~\ref{pfcommands}.
\index{\texttt{prove} command}

\begin{verbatim}
MM> prove th1
Entering the Proof Assistant.  Type HELP for help, EXIT
to exit.  You will be working on the proof of statement th1:
  $p |- t = t
Note:  The proof you are starting with is already complete.
MM-PA>
\end{verbatim}

The \verb/MM-PA>/ prompt means we are inside the Proof
Assistant.\index{Proof Assistant} Most of the regular Metamath commands
(\texttt{show statement}, etc.) are still available if you need them.

\begin{verbatim}
MM-PA> delete all
The entire proof was deleted.
\end{verbatim}

We have deleted the whole proof so we can start from scratch.

\begin{verbatim}
MM-PA> show new_proof/lemmon/all
1 ?              $? |- t = t
\end{verbatim}

The \texttt{show new{\char`\_}proof} command\index{\texttt{show
new{\char`\_}proof} command} is like \texttt{show proof} except that we
don't specify a statement; instead, the proof we're working on is
displayed.

\begin{verbatim}
MM-PA> assign 1 mp
To undo the assignment, DELETE STEP 5 and INITIALIZE, UNIFY
if needed.
3   min=?  $? |- $2
4   maj=?  $? |- ( $2 -> t = t )
\end{verbatim}

The \texttt{assign} command\index{\texttt{assign} command} above means
``assign step 1 with the statement whose label is \texttt{mp}.''  Note
that step renumbering will constantly occur as you assign steps in the
middle of a proof; in general all steps from the step you assign until
the end of the proof will get moved up.  In this case, what used to be
step 1 is now step 5, because the (partial) proof now has five steps:
the four hypotheses of the \texttt{mp} statement and the \texttt{mp}
statement itself.  Let's look at all the steps in our partial proof:

\begin{verbatim}
MM-PA> show new_proof/lemmon/all
1 ?              $? wff $2
2 ?              $? wff t = t
3 ?              $? |- $2
4 ?              $? |- ( $2 -> t = t )
5 1,2,3,4 mp     $a |- t = t
\end{verbatim}

The symbol \texttt{\$2} is a temporary variable\index{temporary
variable} that represents a symbol sequence not yet known.  In the final
proof, all temporary variables will be eliminated.  The general format
for a temporary variable is \texttt{\$} followed by an integer.  Note
that \texttt{\$} is not a legal character in a math symbol (see
Section~\ref{dollardollar}, p.~\pageref{dollardollar}), so there will
never be a naming conflict between real symbols and temporary variables.

Unknown steps 1 and 2 are constructions of the two wffs used by the
modus ponens rule.  As you will see at the end of this section, the
Proof Assistant\index{Proof Assistant} can usually figure these steps
out by itself, and we will not have to worry about them.  Therefore from
here on we will display only the ``essential'' hypotheses, i.e.\ those
steps that correspond to traditional formal proofs\index{formal proof}.

\begin{verbatim}
MM-PA> show new_proof/lemmon
3 ?              $? |- $2
4 ?              $? |- ( $2 -> t = t )
5 3,4 mp         $a |- t = t
\end{verbatim}

Unknown steps 3 and 4 are the ones we must focus on.  They correspond to the
minor and major premises of the modus ponens rule.  We will assign them as
follows.  Notice that because of the step renumbering that takes place
after an assignment, it is advantageous to assign unknown steps in reverse
order, because earlier steps will not get renumbered.

\begin{verbatim}
MM-PA> assign 4 mp
To undo the assignment, DELETE STEP 8 and INITIALIZE, UNIFY
if needed.
3   min=?  $? |- $2
6     min=?  $? |- $4
7     maj=?  $? |- ( $4 -> ( $2 -> t = t ) )
\end{verbatim}

We are now going to describe an obscure feature that you will probably
never use but should be aware of.  The Metamath language allows empty
symbol sequences to be substituted for variables, but in most formal
systems this feature is never used.  One of the few examples where is it
used is the MIU-system\index{MIU-system} described in
Appendix~\ref{MIU}.  But such systems are rare, and by default this
feature is turned off in the Proof Assistant.  (It is always allowed for
{\tt verify proof}.)  Let us turn it on and see what
happens.\index{\texttt{set empty{\char`\_}substitution} command}

\begin{verbatim}
MM-PA> set empty_substitution on
Substitutions with empty symbol sequences is now allowed.
\end{verbatim}

With this feature enabled, more unifications will be
ambiguous\index{ambiguous unification}\index{unification!ambiguous} in
the middle of a proof, because
substitution\index{substitution!variable}\index{variable substitution}
of variables with empty symbol sequences will become an additional
possibility.  Let's see what happens when we make our next assignment.

\begin{verbatim}
MM-PA> assign 3 a2
There are 2 possible unifications.  Please select the correct
    one or Q if you want to UNIFY later.
Unify:  |- $6
 with:  |- ( $9 + 0 ) = $9
Unification #1 of 2 (weight = 7):
  Replace "$6" with "( + 0 ) ="
  Replace "$9" with ""
  Accept (A), reject (R), or quit (Q) <A>? r
\end{verbatim}

The first choice presented is the wrong one.  If we had selected it,
temporary variable \texttt{\$6} would have been assigned a truncated
wff, and temporary variable \texttt{\$9} would have been assigned an
empty sequence (which is not allowed in our system).  With this choice,
eventually we would reach a point where we would get stuck because
we would end up with steps impossible to prove.  (You may want to
try it.)  We typed \texttt{r} to reject the choice.

\begin{verbatim}
Unification #2 of 2 (weight = 21):
  Replace "$6" with "( $9 + 0 ) = $9"
  Accept (A), reject (R), or quit (Q) <A>? q
To undo the assignment, DELETE STEP 4 and INITIALIZE, UNIFY
if needed.
 7     min=?  $? |- $8
 8     maj=?  $? |- ( $8 -> ( $6 -> t = t ) )
\end{verbatim}

The second choice is correct, and normally we would type \texttt{a}
to accept it.  But instead we typed \texttt{q} to show what will happen:
it will leave the step with an unknown unification, which can be
seen as follows:

\begin{verbatim}
MM-PA> show new_proof/not_unified
 4   min    $a |- $6
        =a2  = |- ( $9 + 0 ) = $9
\end{verbatim}

Later we can unify this with the \texttt{unify}
\texttt{all/interactive} command.

The important point to remember is that occasionally you will be
presented with several unification choices while entering a proof, when
the program determines that there is not enough information yet to make
an unambiguous choice automatically (and this can happen even with
\texttt{set empty{\char`\_}substitution} turned off).  Usually it is
obvious by inspection which choice is correct, since incorrect ones will
tend to be meaningless fragments of wffs.  In addition, the correct
choice will usually be the first one presented, unlike our example
above.

Enough of this digression.  Let us go back to the default setting.

\begin{verbatim}
MM-PA> set empty_substitution off
The ability to substitute empty expressions for variables
has been turned off.  Note that this may make the Proof
Assistant too restrictive in some cases.
\end{verbatim}

If we delete the proof, start over, and get to the point where
we digressed above, there will no longer be an ambiguous unification.

\begin{verbatim}
MM-PA> assign 3 a2
To undo the assignment, DELETE STEP 4 and INITIALIZE, UNIFY
if needed.
 7     min=?  $? |- $4
 8     maj=?  $? |- ( $4 -> ( ( $5 + 0 ) = $5 -> t = t ) )
\end{verbatim}

Let us look at our proof so far, and continue.

\begin{verbatim}
MM-PA> show new_proof/lemmon
 4 a2            $a |- ( $5 + 0 ) = $5
 7 ?             $? |- $4
 8 ?             $? |- ( $4 -> ( ( $5 + 0 ) = $5 -> t = t ) )
 9 7,8 mp        $a |- ( ( $5 + 0 ) = $5 -> t = t )
10 4,9 mp        $a |- t = t
MM-PA> assign 8 a1
To undo the assignment, DELETE STEP 11 and INITIALIZE, UNIFY
if needed.
 7     min=?  $? |- ( t + 0 ) = t
MM-PA> assign 7 a2
To undo the assignment, DELETE STEP 8 and INITIALIZE, UNIFY
if needed.
MM-PA> show new_proof/lemmon
 4 a2            $a |- ( t + 0 ) = t
 8 a2            $a |- ( t + 0 ) = t
12 a1            $a |- ( ( t + 0 ) = t -> ( ( t + 0 ) = t ->
                                                    t = t ) )
13 8,12 mp       $a |- ( ( t + 0 ) = t -> t = t )
14 4,13 mp       $a |- t = t
\end{verbatim}

Now all temporary variables and unknown steps have been eliminated from the
``essential'' part of the proof.  When this is achieved, the Proof
Assistant\index{Proof Assistant} can usually figure out the rest of the proof
automatically.  (Note that the \texttt{improve} command can occasionally be
useful for filling in essential steps as well, but it only tries to make use
of statements that introduce no new variables in their hypotheses, which is
not the case for \texttt{mp}. Also it will not try to improve steps containing
temporary variables.)  Let's look at the complete proof, then run
the \texttt{improve} command, then look at it again.

\begin{verbatim}
MM-PA> show new_proof/lemmon/all
 1 ?             $? wff ( t + 0 ) = t
 2 ?             $? wff t = t
 3 ?             $? term t
 4 3 a2          $a |- ( t + 0 ) = t
 5 ?             $? wff ( t + 0 ) = t
 6 ?             $? wff ( ( t + 0 ) = t -> t = t )
 7 ?             $? term t
 8 7 a2          $a |- ( t + 0 ) = t
 9 ?             $? term ( t + 0 )
10 ?             $? term t
11 ?             $? term t
12 9,10,11 a1    $a |- ( ( t + 0 ) = t -> ( ( t + 0 ) = t ->
                                                    t = t ) )
13 5,6,8,12 mp   $a |- ( ( t + 0 ) = t -> t = t )
14 1,2,4,13 mp   $a |- t = t
\end{verbatim}

\begin{verbatim}
MM-PA> improve all
A proof of length 1 was found for step 11.
A proof of length 1 was found for step 10.
A proof of length 3 was found for step 9.
A proof of length 1 was found for step 7.
A proof of length 9 was found for step 6.
A proof of length 5 was found for step 5.
A proof of length 1 was found for step 3.
A proof of length 3 was found for step 2.
A proof of length 5 was found for step 1.
Steps 1 and above have been renumbered.
CONGRATULATIONS!  The proof is complete.  Use SAVE
NEW_PROOF to save it.  Note:  The Proof Assistant does
not detect $d violations.  After saving the proof, you
should verify it with VERIFY PROOF.
\end{verbatim}

The \texttt{save new{\char`\_}proof} command\index{\texttt{save
new{\char`\_}proof} command} will save the proof in the database.  Here
we will just display it in a form that can be clipped out of a log file
and inserted manually into the database source file with a text
editor.\index{normal proof}\index{proof!normal}

\begin{verbatim}
MM-PA> show new_proof/normal
---------Clip out the proof below this line:
      tt tze tpl tt weq tt tt weq tt a2 tt tze tpl tt weq
      tt tze tpl tt weq tt tt weq wim tt a2 tt tze tpl tt
      tt a1 mp mp $.
---------The proof of 'th1' to clip out ends above this line.
\end{verbatim}

There is another proof format called ``compressed''\index{compressed
proof}\index{proof!compressed} that you will see in databases.  It is
not important to understand how it is encoded but only to recognize it
when you see it.  Its only purpose is to reduce storage requirements for
large proofs.  A compressed proof can always be converted to a normal
one and vice-versa, and the Metamath \texttt{show proof}
commands\index{\texttt{show proof} command} work equally well with
compressed proofs.  The compressed proof format is described in
Appendix~\ref{compressed}.

\begin{verbatim}
MM-PA> show new_proof/compressed
---------Clip out the proof below this line:
      ( tze tpl weq a2 wim a1 mp ) ABCZADZAADZAEZJJKFLIA
      AGHH $.
---------The proof of 'th1' to clip out ends above this line.
\end{verbatim}

Now we will exit the Proof Assistant.  Since we made changes to the proof,
it will warn us that we have not saved it.  In this case, we don't care.

\begin{verbatim}
MM-PA> exit
Warning:  You have not saved changes to the proof.
Do you want to EXIT anyway (Y, N) <N>? y
Exiting the Proof Assistant.
Type EXIT again to exit Metamath.
\end{verbatim}

The Proof Assistant\index{Proof Assistant} has several other commands
that can help you while creating proofs.  See Section~\ref{pfcommands}
for a list of them.

A command that is often useful is \texttt{minimize{\char`\_}with
*/brief}, which tries to shorten the proof.  It can make the process
more efficient by letting you write a somewhat ``sloppy'' proof then
clean up some of the fine details of optimization for you (although it
can't perform miracles such as restructuring the overall proof).

\section{A Note About Editing a Data\-base File}

Once your source file contains proofs, there are some restrictions on
how you can edit it so that the proofs remain valid.  Pay particular
attention to these rules, since otherwise you can lose a lot of work.
It is a good idea to periodically verify all proofs with \texttt{verify
proof *} to ensure their integrity.

If your file contains only normal (as opposed to compressed) proofs, the
main rule is that you may not change the order of the mandatory
hypotheses\index{mandatory hypothesis} of any statement referenced in a
later proof.  For example, if you swap the order of the major and minor
premise in the modus ponens rule, all proofs making use of that rule
will become incorrect.  The \texttt{show statement}
command\index{\texttt{show statement} command} will show you the
mandatory hypotheses of a statement and their order.

If a statement has a compressed proof, you also must not change the
order of {\em its} mandatory hypotheses.  The compressed proof format
makes use of this information as part of the compression technique.
Note that swapping the names of two variables in a theorem will change
the order of its mandatory hypotheses.

The safest way to edit a statement, say \texttt{mytheorem}, is to
duplicate it then rename the original to \texttt{mytheoremOLD}
throughout the database.  Once the edited version is re-proved, all
statements referencing \texttt{mytheoremOLD} can be updated in the Proof
Assistant using \texttt{minimize{\char`\_}with
mytheorem
/allow{\char`\_}growth}.\index{\texttt{minimize{\char`\_}with} command}
% 3/10/07 Note: line-breaking the above results in duplicate index entries

\chapter{Abstract Mathematics Revealed}\label{fol}

\section{Logic and Set Theory}\label{logicandsettheory}

\begin{quote}
  {\em Set theory can be viewed as a form of exact theology.}
  \flushright\sc  Rudy Rucker\footnote{\cite{Barrow}, p.~31.}\\
\end{quote}\index{Rucker, Rudy}

Despite its seeming complexity, all of standard mathematics, no matter how
deep or abstract, can amazingly enough be derived from a relatively small set
of axioms\index{axiom} or first principles. The development of these axioms is
among the most impressive and important accomplishments of mathematics in the
20th century. Ultimately, these axioms can be broken down into a set of rules
for manipulating symbols that any technically oriented person can follow.

We will not spend much time trying to convey a deep, higher-level
understanding of the meaning of the axioms. This kind of understanding
requires some mathematical sophistication as well as an understanding of the
philosophy underlying the foundations of mathematics and typically develops
over time as you work with mathematics.  Our goal, instead, is to give you the
immediate ability to follow how theorems\index{theorem} are derived from the
axioms and from other theorems.  This will be similar to learning the syntax
of a computer language, which lets you follow the details in a program but
does not necessarily give you the ability to write non-trivial programs on
your own, an ability that comes with practice. For now don't be alarmed by
abstract-sounding names of the axioms; just focus on the rules for
manipulating the symbols, which follow the simple conventions of the
Metamath\index{Metamath} language.

The axioms that underlie all of standard mathematics consist of axioms of logic
and axioms of set theory. The axioms of logic are divided into two
subcategories, propositional calculus\index{propositional calculus} (sometimes
called sentential logic\index{sentential logic}) and predicate calculus
(sometimes called first-order logic\index{first-order logic}\index{quantifier
theory}\index{predicate calculus} or quantifier theory).  Propositional
calculus is a prerequisite for predicate calculus, and predicate calculus is a
prerequisite for set theory.  The version of set theory most commonly used is
Zermelo--Fraenkel set theory\index{Zermelo--Fraenkel set theory}\index{set theory}
with the axiom of choice,
often abbreviated as ZFC\index{ZFC}.

Here in a nutshell is what the axioms are all about in an informal way. The
connection between this description and symbols we will show you won't be
immediately apparent and in principle needn't ever be.  Our description just
tries to summarize what mathematicians think about when they work with the
axioms.

Logic is a set of rules that allow us determine truths given other truths.
Put another way,
logic is more or less the translation of what we would consider common sense
into a rigorous set of axioms.\index{axioms of logic}  Suppose $\varphi$,
$\psi$, and $\chi$ (the Greek letters phi, psi, and chi) represent statements
that are either true or false, and $x$ is a variable\index{variable!in predicate
calculus} ranging over some group of mathematical objects (sets, integers,
real numbers, etc.). In mathematics, a ``statement'' really means a formula,
and $\psi$ could be for example ``$x = 2$.''
Propositional calculus\index{propositional calculus}
allows us to use variables that are either true or false
and make deductions such as
``if $\varphi$ implies $\psi$ and $\psi$ implies $\chi$, then $\varphi$
implies $\chi$.''
Predicate calculus\index{predicate calculus}
extends propositional calculus by also allowing us
to discuss statements about objects (not just true and false values), including
statements about ``all'' or ``at least one'' object.
For example, predicate calculus allows to say,
``if $\varphi$ is true for all $x$, then $\varphi$ is true for some $x$.''
The logic used in \texttt{set.mm} is standard classical logic
(as opposed to other logic systems like intuitionistic logic).

Set theory\index{set theory} has to do with the manipulation of objects and
collections of objects, specifically the abstract, imaginary objects that
mathematics deals with, such as numbers. Everything that is claimed to exist
in mathematics is considered to be a set.  A set called the empty
set\index{empty set} contains nothing.  We represent the empty set by
$\varnothing$.  Many sets can be built up from the empty set.  There is a set
represented by $\{\varnothing\}$ that contains the empty set, another set
represented by $\{\varnothing,\{\varnothing\}\}$ that contains this set as
well as the empty set, another set represented by $\{\{\varnothing\}\}$ that
contains just the set that contains the empty set, and so on ad infinitum. All
mathematical objects, no matter how complex, are defined as being identical to
certain sets: the integer\index{integer} 0 is defined as the empty set, the
integer 1 is defined as $\{\varnothing\}$, the integer 2 is defined as
$\{\varnothing,\{\varnothing\}\}$.  (How these definitions were chosen doesn't
matter now, but the idea behind it is that these sets have the properties we
expect of integers once suitable operations are defined.)  Mathematical
operations, such as addition, are defined in terms of operations on
sets---their union\index{set union}, intersection\index{set intersection}, and
so on---operations you may have used in elementary school when you worked
with groups of apples and oranges.

With a leap of faith, the axioms also postulate the existence of infinite
sets\index{infinite set}, such as the set of all non-negative integers ($0, 1,
2,\ldots$, also called ``natural numbers''\index{natural number}).  This set
can't be represented with the brace notation\index{brace notation} we just
showed you, but requires a more complicated notation called ``class
abstraction.''\index{class abstraction}\index{abstraction class}  For
example, the infinite set $\{ x |
\mbox{``$x$ is a natural number''} \} $ means the ``set of all objects $x$
such that $x$ is a natural number'' i.e.\ the set of natural numbers; here,
``$x$ is a natural number'' is a rather complicated formula when broken down
into the primitive symbols.\label{expandom}\footnote{The statement ``$x$ is a
natural number'' is formally expressed as ``$x \in \omega$,'' where $\in$
(stylized epsilon) means ``is in'' or ``is an element of'' and $\omega$
(omega) means ``the set of natural numbers.''  When ``$x\in\omega$'' is
completely expanded in terms of the primitive symbols of set theory, the
result is  $\lnot$ $($ $\lnot$ $($ $\forall$ $z$ $($ $\lnot$ $\forall$ $w$ $($
$z$ $\in$ $w$ $\rightarrow$ $\lnot$ $w$ $\in$ $x$ $)$ $\rightarrow$ $z$ $\in$
$x$ $)$ $\rightarrow$ $($ $\forall$ $z$ $($ $\lnot$ $($ $\forall$ $w$ $($ $w$
$\in$ $z$ $\rightarrow$ $w$ $\in$ $x$ $)$ $\rightarrow$ $\forall$ $w$ $\lnot$
$w$ $\in$ $z$ $)$ $\rightarrow$ $\lnot$ $\forall$ $w$ $($ $w$ $\in$ $z$
$\rightarrow$ $\lnot$ $\forall$ $v$ $($ $v$ $\in$ $z$ $\rightarrow$ $\lnot$
$v$ $\in$ $w$ $)$ $)$ $)$ $\rightarrow$ $\lnot$ $\forall$ $z$ $\forall$ $w$
$($ $\lnot$ $($ $z$ $\in$ $x$ $\rightarrow$ $\lnot$ $w$ $\in$ $x$ $)$
$\rightarrow$ $($ $\lnot$ $z$ $\in$ $w$ $\rightarrow$ $($ $\lnot$ $z$ $=$ $w$
$\rightarrow$ $w$ $\in$ $z$ $)$ $)$ $)$ $)$ $)$ $\rightarrow$ $\lnot$
$\forall$ $y$ $($ $\lnot$ $($ $\lnot$ $($ $\forall$ $z$ $($ $\lnot$ $\forall$
$w$ $($ $z$ $\in$ $w$ $\rightarrow$ $\lnot$ $w$ $\in$ $y$ $)$ $\rightarrow$
$z$ $\in$ $y$ $)$ $\rightarrow$ $($ $\forall$ $z$ $($ $\lnot$ $($ $\forall$
$w$ $($ $w$ $\in$ $z$ $\rightarrow$ $w$ $\in$ $y$ $)$ $\rightarrow$ $\forall$
$w$ $\lnot$ $w$ $\in$ $z$ $)$ $\rightarrow$ $\lnot$ $\forall$ $w$ $($ $w$
$\in$ $z$ $\rightarrow$ $\lnot$ $\forall$ $v$ $($ $v$ $\in$ $z$ $\rightarrow$
$\lnot$ $v$ $\in$ $w$ $)$ $)$ $)$ $\rightarrow$ $\lnot$ $\forall$ $z$
$\forall$ $w$ $($ $\lnot$ $($ $z$ $\in$ $y$ $\rightarrow$ $\lnot$ $w$ $\in$
$y$ $)$ $\rightarrow$ $($ $\lnot$ $z$ $\in$ $w$ $\rightarrow$ $($ $\lnot$ $z$
$=$ $w$ $\rightarrow$ $w$ $\in$ $z$ $)$ $)$ $)$ $)$ $\rightarrow$ $($
$\forall$ $z$ $\lnot$ $z$ $\in$ $y$ $\rightarrow$ $\lnot$ $\forall$ $w$ $($
$\lnot$ $($ $w$ $\in$ $y$ $\rightarrow$ $\lnot$ $\forall$ $z$ $($ $w$ $\in$
$z$ $\rightarrow$ $\lnot$ $z$ $\in$ $y$ $)$ $)$ $\rightarrow$ $\lnot$ $($
$\lnot$ $\forall$ $z$ $($ $w$ $\in$ $z$ $\rightarrow$ $\lnot$ $z$ $\in$ $y$
$)$ $\rightarrow$ $w$ $\in$ $y$ $)$ $)$ $)$ $)$ $\rightarrow$ $x$ $\in$ $y$
$)$ $)$ $)$. Section~\ref{hierarchy} shows the hierarchy of definitions that
leads up to this expression.}\index{stylized epsilon ($\in$)}\index{omega
($\omega$)}  Actually, the primitive symbols don't even include the brace
notation.  The brace notation is a high-level definition, which you can find in
Section~\ref{hierarchy}.

Interestingly, the arithmetic of integers\index{integer} and
rationals\index{rational number} can be developed without appealing to the
existence of an infinite set, whereas the arithmetic of real
numbers\index{real number} requires it.

Each variable\index{variable!in set theory} in the axioms of set theory
represents an arbitrary set, and the axioms specify the legal kinds of things
you can do with these variables at a very primitive level.

Now, you may think that numbers and arithmetic are a lot more intuitive and
fundamental than sets and therefore should be the foundation of mathematics.
What is really the case is that you've dealt with numbers all your life and
are comfortable with a few rules for manipulating them such as addition and
multiplication.  Those rules only cover a small portion of what can be done
with numbers and only a very tiny fraction of the rest of mathematics.  If you
look at any elementary book on number theory, you will quickly become lost if
these are the only rules that you know.  Even though such books may present a
list of ``axioms''\index{axiom} for arithmetic, the ability to use the axioms
and to understand proofs of theorems\index{theorem} (facts) about numbers
requires an implicit mathematical talent that frustrates many people
from studying abstract mathematics.  The kind of mathematics that most people
know limits them to the practical, everyday usage of blindly manipulating
numbers and formulas, without any understanding of why those rules are correct
nor any ability to go any further.  For example, do you know why multiplying
two negative numbers yields a positive number?  Starting with set theory, you
will also start off blindly manipulating symbols according to the rules we give
you, but with the advantage that these rules will allow you, in principle, to
access {\em all} of mathematics, not just a tiny part of it.

Of course, concrete examples are often helpful in the learning process. For
example, you can verify that $2\cdot 3=3 \cdot 2$ by actually grouping
objects and can easily ``see'' how it generalizes to $x\cdot y = y\cdot x$,
even though you might not be able to rigorously prove it.  Similarly, in set
theory it can be helpful to understand how the axioms of set theory apply to
(and are correct for) small finite collections of objects.  You should be aware
that in set theory intuition can be misleading for infinite collections, and
rigorous proofs become more important.  For example, while $x\cdot y = y\cdot
x$ is correct for finite ordinals (which are the natural numbers), it is not
usually true for infinite ordinals.

\section{The Axioms for All of Mathematics}

In this section\index{axioms for mathematics}, we will show you the axioms
for all of standard mathematics (i.e.\ logic and set theory) as they are
traditionally presented.  The traditional presentation is useful for someone
with the mathematical experience needed to correctly manipulate high-level
abstract concepts.  For someone without this talent, knowing how to actually
make use of these axioms can be difficult.  The purpose of this section is to
allow you to see how the version of the axioms used in the standard
Metamath\index{Metamath} database \texttt{set.mm}\index{set
theory database (\texttt{set.mm})} relates to  the typical version
in textbooks, and also to give you an informal feel for them.

\subsection{Propositional Calculus}

Propositional calculus\index{propositional calculus} concerns itself with
statements that can be interpreted as either true or false.  Some examples of
statements (outside of mathematics) that are either true or false are ``It is
raining today'' and ``The United States has a female president.'' In
mathematics, as we mentioned, statements are really formulas.

In propositional calculus, we don't care what the statements are.  We also
treat a logical combination of statements, such as ``It is raining today and
the United States has a female president,'' no differently from a single
statement.  Statements and their combinations are called well-formed formulas
(wffs)\index{well-formed formula (wff)}.  We define wffs only in terms of
other wffs and don't define what a ``starting'' wff is.  As is common practice
in the literature, we use Greek letters to represent wffs.

Specifically, suppose $\varphi$ and $\psi$ are wffs.  Then the combinations
$\varphi\rightarrow\psi$ (``$\varphi$ implies $\psi$,'' also read ``if
$\varphi$ then $\psi$'')\index{implication ($\rightarrow$)} and $\lnot\varphi$
(``not $\varphi$'')\index{negation ($\lnot$)} are also wffs.

The three axioms of propositional calculus\index{axioms of propositional
calculus} are all wffs of the following form:\footnote{A remarkable result of
C.~A.~Meredith\index{Meredith, C. A.} squeezes these three axioms into the
single axiom $((((\varphi\rightarrow \psi)\rightarrow(\neg \chi\rightarrow\neg
\theta))\rightarrow \chi )\rightarrow \tau)\rightarrow((\tau\rightarrow
\varphi)\rightarrow(\theta\rightarrow \varphi))$ \cite{CAMeredith},
which is believed to be the shortest possible.}
\begin{center}
     $\varphi\rightarrow(\psi\rightarrow \varphi)$\\

     $(\varphi\rightarrow (\psi\rightarrow \chi))\rightarrow
((\varphi\rightarrow  \psi)\rightarrow (\varphi\rightarrow \chi))$\\

     $(\neg \varphi\rightarrow \neg\psi)\rightarrow (\psi\rightarrow
\varphi)$
\end{center}

These three axioms are widely used.
They are attributed to Jan {\L}ukasiewicz
(pronounced woo-kah-SHAY-vitch) and was popularized by Alonzo Church,
who called it system P2. (Thanks to Ted Ulrich for this information.)

There are an infinite number of axioms, one for each possible
wff\index{well-formed formula (wff)} of the above form.  (For this reason,
axioms such as the above are often called ``axiom schemes.''\index{axiom
scheme})  Each Greek letter in the axioms may be substituted with a more
complex wff to result in another axiom.  For example, substituting
$\neg(\varphi\rightarrow\chi)$ for $\varphi$ in the first axiom yields
$\neg(\varphi\rightarrow\chi)\rightarrow(\psi\rightarrow
\neg(\varphi\rightarrow\chi))$, which is still an axiom.

To deduce new true statements (theorems\index{theorem}) from the axioms, a
rule\index{rule} called ``modus ponens''\index{modus ponens} is used.  This
rule states that if the wff $\varphi$ is an axiom or a theorem, and the wff
$\varphi\rightarrow\psi$ is an axiom or a theorem, then the wff $\psi$ is also
a theorem\index{theorem}.

As a non-mathematical example of modus ponens, suppose we have proved (or
taken as an axiom) ``Bob is a man'' and separately have proved (or taken as
an axiom) ``If Bob is a man, then Bob is a human.''  Using the rule of modus
ponens, we can logically deduce, ``Bob is a human.''

From Metamath's\index{Metamath} point of view, the axioms and the rule of
modus ponens just define a mechanical means for deducing new true statements
from existing true statements, and that is the complete content of
propositional calculus as far as Metamath is concerned.  You can read a logic
textbook to gain a better understanding of their meaning, or you can just let
their meaning slowly become apparent to you after you use them for a while.

It is actually rather easy to check to see if a formula is a theorem of
propositional calculus.  Theorems of propositional calculus are also called
``tautologies.''\index{tautology}  The technique to check whether a formula is
a tautology is called the ``truth table method,''\index{truth table} and it
works like this.  A wff $\varphi\rightarrow\psi$ is false whenever $\varphi$ is true
and $\psi$ is false.  Otherwise it is true.  A wff $\lnot\varphi$ is false
whenever $\varphi$ is true and false otherwise. To verify a tautology such as
$\varphi\rightarrow(\psi\rightarrow \varphi)$, you break it down into sub-wffs and
construct a truth table that accounts for all possible combinations of true
and false assigned to the wff metavariables:
\begin{center}\begin{tabular}{|c|c|c|c|}\hline
\mbox{$\varphi$} & \mbox{$\psi$} & \mbox{$\psi\rightarrow\varphi$}
    & \mbox{$\varphi\rightarrow(\psi\rightarrow \varphi)$} \\ \hline \hline
              T   &  T    &      T       &        T    \\ \hline
              T   &  F    &      T       &        T    \\ \hline
              F   &  T    &      F       &        T    \\ \hline
              F   &  F    &      T       &        T    \\ \hline
\end{tabular}\end{center}
If all entries in the last column are true, the formula is a tautology.

Now, the truth table method doesn't tell you how to prove the tautology from
the axioms, but only that a proof exists.  Finding an actual proof (especially
one that is short and elegant) can be challenging.  Methods do exist for
automatically generating proofs in propositional calculus, but the proofs that
result can sometimes be very long.  In the Metamath \texttt{set.mm}\index{set
theory database (\texttt{set.mm})} database, most
or all proofs were created manually.

Section \ref{metadefprop} discusses various definitions
that make propositional calculus easier to use.
For example, we define:

\begin{itemize}
\item $\varphi \vee \psi$
  is true if either $\varphi$ or $\psi$ (or both) are true
  (this is disjunction\index{disjunction ($\vee$)}
  aka logical {\sc or}\index{logical {\sc or} ($\vee$)}).

\item $\varphi \wedge \psi$
  is true if both $\varphi$ and $\psi$ are true
  (this is conjunction\index{conjunction ($\wedge$)}
  aka logical {\sc and}\index{logical {\sc and} ($\wedge$)}).

\item $\varphi \leftrightarrow \psi$
  is true if $\varphi$ and $\psi$ have the same value, that is,
  they are both true or both false
  (this is the biconditional\index{biconditional ($\leftrightarrow$)}).
\end{itemize}

\subsection{Predicate Calculus}

Predicate calculus\index{predicate calculus} introduces the concept of
``individual variables,''\index{variable!in predicate calculus}\index{individual
variable} which
we will usually just call ``variables.''
These variables can represent something other than true or false (wffs),
and will always represent sets when we get to set theory.  There are also
three new symbols $\forall$\index{universal quantifier ($\forall$)},
$=$\index{equality ($=$)}, and $\in$\index{stylized epsilon ($\in$)},
read ``for all,'' ``equals,'' and ``is an element of''
respectively.  We will represent variables with the letters $x$, $y$, $z$, and
$w$, as is common practice in the literature.
For example, $\forall x \varphi$ means ``for all possible values of
$x$, $\varphi$ is true.''

In predicate calculus, we extend the definition of a wff\index{well-formed
formula (wff)}.  If $\varphi$ is a wff and $x$ and $y$ are variables, then
$\forall x \, \varphi$, $x=y$, and $x\in y$ are wffs. Note that these three new
types of wffs can be considered ``starting'' wffs from which we can build
other wffs with $\rightarrow$ and $\neg$ .  The concept of a starting wff was
absent in propositional calculus.  But starting wff or not, all we are really
concerned with is whether our wffs are correctly constructed according to
these mechanical rules.

A quick aside:
To prevent confusion, it might be best at this point to think of the variables
of Metamath\index{Metamath} as ``metavariables,''\index{metavariable} because
they are not quite the same as the variables we are introducing here.  A
(meta)variable in Metamath can be a wff or an individual variable, as well
as many other things; in general, it represents a kind of place holder for an
unspecified sequence of math symbols\index{math symbol}.

Unlike propositional calculus, no decision procedure\index{decision procedure}
analogous to the truth table method exists (nor theoretically can exist) that
will definitely determine whether a formula is a theorem of predicate
calculus.  Much of the work in the field of automated theorem
proving\index{automated theorem proving} has been dedicated to coming up with
clever heuristics for proving theorems of predicate calculus, but they can
never be guaranteed to work always.

Section \ref{metadefpred} discusses various definitions
that make predicate calculus easier to use.
For example, we define
$\exists x \varphi$ to mean
``there exists at least one possible value of $x$ where $\varphi$ is true.''

We now turn to looking at how predicate calculus can be formally
represented.

\subsubsection{Common Axioms}

There is a new rule of inference in predicate calculus:  if $\varphi$ is
an axiom or a theorem, then $\forall x \,\varphi$ is also a
theorem\index{theorem}.  This is called the rule of
``generalization.''\index{rule of generalization}
This is easily represented in Metamath.

In standard texts of logic, there are often two axioms of predicate
calculus\index{axioms of predicate calculus}:
\begin{center}
  $\forall x \,\varphi ( x ) \rightarrow \varphi ( y )$,
      where ``$y$ is properly substituted for $x$.''\\
  $\forall x ( \varphi \rightarrow \psi )\rightarrow ( \varphi \rightarrow
    \forall x\, \psi )$,
    where ``$x$ is not free in $\varphi$.''
\end{center}

Now at first glance, this seems simple:  just two axioms.  However,
conditional clauses are attached to each axiom describing requirements that
may seem puzzling to you.  In addition, the first axiom puts a variable symbol
in parentheses after each wff, seemingly violating our definition of a
wff\index{well-formed formula (wff)}; this is just an informal way of
referring to some arbitrary variable that may occur in the wff.  The
conditional clauses do, of course, have a precise meaning, but as it turns out
the precise meaning is somewhat complicated and awkward to formalize in a
way that a computer can handle easily.  Unlike propositional calculus, a
certain amount of mathematical sophistication and practice is needed to be
able to easily grasp and manipulate these concepts correctly.

Predicate calculus may be presented with or without axioms for
equality\index{axioms of equality}\index{equality ($=$)}. We will require the
axioms of equality as a prerequisite for the version of set theory we will
use.  The axioms for equality, when included, are often represented using these
two axioms:
\begin{center}
$x=x$\\ \ \\
$x=y\rightarrow (\varphi(x,x)\rightarrow\varphi(x,y))$ where ``$\varphi(x,y)$
   arises from $\varphi(x,x)$ by replacing some, but not necessarily all,
   free\index{free variable}
   occurrences of $x$ by $y$,\\ provided that $y$ is free for $x$
   in $\varphi(x,x)$.'' \end{center}
% (Mendelson p. 95)
The first equality axiom is simple, but again,
the condition on the second one is
somewhat awkward to implement on a computer.

\subsubsection{Tarski System S2}

Of course, we are not the first to notice the complications of these
predicate calculus axioms when being rigorous.

Well-known logician Alfred Tarski published in 1965
a system he called system S2\cite[p.~77]{Tarski1965}.
Tarski's system is \textit{exactly equivalent} to the traditional textbook
formalization, but (by clever use of equality axioms) it eliminates the
latter's primitive notions of ``proper substitution'' and ``free variable,''
replacing them with direct substitution and the notion of a variable
not occurring in a formula (which we express with distinct variable
constraints).

In advocating his system, Tarski wrote, ``The relatively complicated
character of [free variables and proper substitution] is a source
of certain inconveniences of both practical and theoretical nature;
this is clearly experienced both in teaching an elementary course of
mathematical logic and in formalizing the syntax of predicate logic for
some theoretical purposes''\cite[p.~61]{Tarski1965}\index{Tarski, Alfred}.

\subsubsection{Developing a Metamath Representation}

The standard textbook axioms of predicate calculus are somewhat
cumbersome to implement on a computer because of the complex notions of
``free variable''\index{free variable} and ``proper
substitution.''\index{proper substitution}\index{substitution!proper}
While it is possible to use the Metamath\index{Metamath} language to
implement these concepts, we have chosen not to implement them
as primitive constructs in the
\texttt{set.mm} set theory database.  Instead, we have eliminated them
within the axioms
by carefully crafting the axioms so as to avoid them,
building on Tarski's system S2.  This makes it
easy for a beginner to follow the steps in a proof without knowing any
advanced concepts other than the simple concept of
replacing\index{substitution!variable}\index{variable substitution}
variables with expressions.

In order to develop the concepts of free variable and proper
substitution from the axioms, we use an additional
Metamath statement type called ``disjoint variable
restriction''\index{disjoint variables} that we have not encountered
before.  In the context of the axioms, the statement \texttt{\$d} $ x\,
y$\index{\texttt{\$d} statement} simply means that $x$ and $y$ must be
distinct\index{distinct variables}, i.e.\ they may not be simultaneously
substituted\index{substitution!variable}\index{variable substitution}
with the same variable.  The statement \texttt{\$d} $ x\, \varphi$ means
variable $x$ must not occur in wff $\varphi$.  For the precise
definition of \texttt{\$d}, see Section~\ref{dollard}.

\subsubsection{Metamath representation}

The Metamath axiom system for predicate calculus
defined in set.mm uses Tarski's system S2.
As noted above, this has a different representation
than the traditional textbook formalization,
but it is \textit{exactly equivalent} to the textbook formalization,
and it is \textit{much} easier to work with.
This is reproduced as system S3 in Section 6 of
Megill's formalization \cite{Megill}\index{Megill, Norman}.

There is one exception, Tarski's axiom of existence,
which we label as axiom ax-6.
In the case of ax-6, Tarski's version is weaker because it includes a
distinct variable proviso. If we wish, we can also weaken our version
in this way and still have a metalogically complete system. Theorem
ax6 shows this by deriving, in the presence of the other axioms, our
ax-6 from Tarski's weaker version ax6v. However, we chose the stronger
version for our system because it is simpler to state and easier to use.

Tarski's system was designed for proving specific theorems rather than
more general theorem schemes. However, theorem schemes are much more
efficient than specific theorems for building a body of mathematical
knowledge, since they can be reused with different instances as
needed. While Tarski does derive some theorem schemes from his axioms,
their proofs require concepts that are ``outside'' of the system, such as
induction on formula length. The verification of such proofs is difficult
to automate in a proof verifier. (Specifically, Tarski treats the formulas
of his system as set-theoretical objects. In order to verify the proofs
of his theorem schemes, a proof verifier would need a significant amount
of set theory built into it.)

The Metamath axiom system for predicate calculus extends
Tarski's system to eliminate this difficulty. The additional
``auxilliary'' axiom
schemes (as we will call them in this section; see below) endow Tarski's
system with a nice property we call
metalogical completeness \cite[Remark 9.6]{Megill}\index{Megill, Norman}.
As a result, we can prove any theorem scheme
expressable in the ``simple metalogic'' of Tarski's system by using
only Metamath's direct substitution rule applied to the axiom system
(and no other metalogical or set-theoretical notions ``outside'' of the
system). Simple metalogic consists of schemes containing wff metavariables
(with no arguments) and/or set (also called ``individual'') metavariables,
accompanied by optional provisos each stating that two specified set
metavariables must be distinct or that a specified set metavariable may
not occur in a specified wff metavariable. Metamath's logic and set theory
axiom and rule schemes are all examples of simple metalogic. The schemes
of traditional predicate calculus with equality are examples which are
not simple metalogic, because they use wff metavariables with arguments
and have ``free for'' and ``not free in'' side conditions.

A rigorous justification for this system, using an older but
exactly equivalent set of axioms, can be
found in \cite{Megill}\index{Megill, Norman}.

This allows us to
take a different approach in the Metamath\index{Metamath} database
\texttt{set.mm}\index{set theory database (\texttt{set.mm})}.  We do not
directly use the primitive notions of ``free variable''\index{free variable}
and ``proper substitution''\index{proper
substitution}\index{substitution!proper} at all as primitive constructs.
Instead, we use a set
of axioms that are almost as simple to manipulate as those of
propositional calculus.  Our axiom system avoids complex primitive
notions by effectively embedding the complexity into the axioms
themselves.  As a result, we will end up with a larger number of axioms,
but they are ideally suited for a computer language such as Metamath.
(Section~\ref{metaaxioms} shows these axioms.)

We will not elaborate further
on the ``free variable'' and ``proper substitution''
concepts here.  You may consult
\cite[ch.\ 3--4]{Hamilton}\index{Hamilton, Alan G.} (as well as
many other books) for a precise explanation
of these concepts.  If you intend to do serious mathematical work, it is wise
to become familiar with the traditional textbook approach; even though the
concepts embedded in their axioms require a higher level of sophistication,
they can be more practical to deal with on an everyday, informal basis.  Even
if you are just developing Metamath proofs, familiarity with the traditional
approach can help you arrive at a proof outline much faster, which you can
then convert to the detail required by Metamath.

We do develop proper substitution rules later on, but in set.mm
they are defined as derived constructs; they are not primitives.

You should also note that our system of predicate calculus is specifically
tailored for set theory; thus there are only two specific predicates $=$ and
$\in$ and no functions\index{function!in predicate calculus}
or constants\index{constant!in predicate calculus} unlike more general systems.
We later add these.

\subsection{Set Theory}

Traditional Zermelo--Fraenkel set theory\index{Zermelo--Fraenkel set
theory}\index{set theory} with the Axiom of Choice
has 10 axioms, which can be expressed in the
language of predicate calculus.  In this section, we will list only the
names and brief English descriptions of these axioms, since we will give
you the precise formulas used by the Metamath\index{Metamath} set theory
database \texttt{set.mm} later on.

In the descriptions of the axioms, we assume that $x$, $y$, $z$, $w$, and $v$
represent sets.  These are the same as the variables\index{variable!in set
theory} in our predicate calculus system above, except that now we informally
think of the variables as ranging over sets.  Note that the terms
``object,''\index{object} ``set,''\index{set} ``element,''\index{element}
``collection,''\index{collection} and ``family''\index{family} are synonymous,
as are ``is an element of,'' ``is a member of,''\index{member} ``is contained
in,'' and ``belongs to.''  The different terms are used for convenience; for
example, ``a collection of sets'' is less confusing than ``a set of sets.''
A set $x$ is said to be a ``subset''\index{subset} of $y$ if every element of
$x$ is also an element of $y$; we also say $x$ is ``included in''
$y$.

The axioms are very general and apply to almost any conceivable mathematical
object, and this level of abstraction can be overwhelming at first.  To gain an
intuitive feel, it can be helpful to draw a picture illustrating the concept;
for example, a circle containing dots could represent a collection of sets,
and a smaller circle drawn inside the circle could represent a subset.
Overlapping circles can illustrate intersection and union.  Circles that
illustrate the concepts of set theory are frequently used in elementary
textbooks and are called Venn diagrams\index{Venn diagram}.\index{axioms of
set theory}

1. Axiom of Extensionality:  Two sets are identical if they contain the same
   elements.\index{Axiom of Extensionality}

2. Axiom of Pairing:  The set $\{ x , y \}$ exists.\index{Axiom of Pairing}

3. Axiom of Power Sets:  The power set of a set (the collection of all of
   its subsets) exists.  For example, the power set of $\{x,y\}$ is
   $\{\varnothing,\{x\},\{y\},\{x,y\}\}$ and it exists.\index{Axiom
of Power Sets}

4. Axiom of the Null Set:  The empty set $\varnothing$ exists.\index{Axiom of
the Null Set}

5. Axiom of Union:  The union of a set (the set containing the elements of
   its members) exists.  For example, the union of $\{\{x,y\},\{z\}\}$ is
 $\{x,y,z\}$ and
   it exists.\index{Axiom of Union}

6. Axiom of Regularity:  Roughly, no set can contain itself, nor can there
   be membership ``loops,'' such as a set being an
   element of one of its members.\index{Axiom of Regularity}

7. Axiom of Infinity:  An infinite set exists.  An example of an infinite
   set is the set of all
   integers.\index{Axiom of Infinity}

8. Axiom of Separation:  The set exists that is obtained by restricting $x$
   with some property.  For example, if the set of all integers exists,
   then the set of all even integers exists.\index{Axiom of Separation}

9. Axiom of Replacement:  The range of a function whose domain is restricted
   to the elements of a set $x$, is also a set.  For example, there
   is a function
   from integers (the function's domain) to their squares (its
   range).  If we
   restrict the domain to even integers, its range will become the set of
   squares of even integers, so this axiom asserts that the set of
    squares of even numbers exists.  Technical note:  In general, the
   ``function'' need not be a set but can be a proper class.
   \index{Axiom of Replacement}

10. Axiom of Choice:  Let $x$ be a set whose members are pairwise
  disjoint\index{disjoint sets} (i.e,
  whose members contain no elements in common).  Then there exists another
  set containing one element from each member of $x$.  For
  example, if $x$ is
  $\{\{y,z\},\{w,v\}\}$, where $y$, $z$, $w$, and $v$ are
  different sets, then a set such as $\{z,w\}$
  exists (but the axiom doesn't tell
  us which one).  (Actually the Axiom
  of Choice is redundant if the set $x$, as in this example, has a finite
  number of elements.)\index{Axiom of Choice}

The Axiom of Choice is usually considered an extension of ZF set theory rather
than a proper part of it.  It is sometimes considered philosophically
controversial because it specifies the existence of a set without specifying
what the set is. Constructive logics, including intuitionistic logic,
do not accept the axiom of choice.
Since there is some lingering controversy, we often prefer proofs that do
not use the axiom of choice (where there is a known alternative), and
in some cases we will use weaker axioms than the full axiom of choice.
That said, the axiom of choice is a powerful and widely-accepted tool,
so we do use it when needed.
ZF set theory that includes the Axiom of Choice is
called Zermelo--Fraenkel set theory with choice (ZFC\index{ZFC set theory}).

When expressed symbolically, the Axiom of Separation and the Axiom of
Replacement contain wff symbols and therefore each represent infinitely many
axioms, one for each possible wff. For this reason, they are often called
axiom schemes\index{axiom scheme}\index{well-formed formula (wff)}.

It turns out that the Axiom of the Null Set, the Axiom of Pairing, and the
Axiom of Separation can be derived from the other axioms and are therefore
unnecessary, although they tend to be included in standard texts for various
reasons (historical, philosophical, and possibly because some authors may not
know this).  In the Metamath\index{Metamath} set theory database, these
redundant axioms are derived from the other ones instead of truly
being considered axioms.
This is in keeping with our general goal of minimizing the number of
axioms we must depend on.

\subsection{Other Axioms}

Above we qualified the phrase ``all of mathematics'' with ``essentially.''
The main important missing piece is the ability to do category theory,
which requires huge sets (inaccessible cardinals) larger than those
postulated by the ZFC axioms. The Tarski--Grothendieck Axiom postulates
the existence of such sets.
Note that this is the same axiom used by Mizar for supporting
category theory.
The Tarski--Grothendieck axiom
can be viewed as a very strong replacement of the Axiom of Infinity,
the Axiom of Choice, and the Axiom of Power Sets.
The \texttt{set.mm} database includes this axiom; see the database
for details about it.
Again, we only use this axiom when we need to.
You are only likely to encounter or use this axiom if you are doing
category theory, since its use is highly specialized,
so we will not list the Tarsky-Grothendieck axiom
in the short list of axioms below.

Can there be even more axioms?
Of course.
G\"{o}del showed that no finite set of axioms or axiom schemes can completely
describe any consistent theory strong enough to include arithmetic.
But practically speaking, the ones above are the accepted foundation that
almost all mathematicians explicitly or implicitly base their work on.

\section{The Axioms in the Metamath Language}\label{metaaxioms}

Here we list the axioms as they appear in
\texttt{set.mm}\index{set theory database (\texttt{set.mm})} so you can
look them up there easily.  Incidentally, the \texttt{show statement
/tex} command\index{\texttt{show statement} command} was used to
typeset them.

%macros from show statement /tex
\newbox\mlinebox
\newbox\mtrialbox
\newbox\startprefix  % Prefix for first line of a formula
\newbox\contprefix  % Prefix for continuation line of a formula
\def\startm{  % Initialize formula line
  \setbox\mlinebox=\hbox{\unhcopy\startprefix}
}
\def\m#1{  % Add a symbol to the formula
  \setbox\mtrialbox=\hbox{\unhcopy\mlinebox $\,#1$}
  \ifdim\wd\mtrialbox>\hsize
    \box\mlinebox
    \setbox\mlinebox=\hbox{\unhcopy\contprefix $\,#1$}
  \else
    \setbox\mlinebox=\hbox{\unhbox\mtrialbox}
  \fi
}
\def\endm{  % Output the last line of a formula
  \box\mlinebox
}

% \SLASH for \ , \TOR for \/ (text OR), \TAND for /\ (text and)
% This embeds a following forced space to force the space.
\newcommand\SLASH{\char`\\~}
\newcommand\TOR{\char`\\/~}
\newcommand\TAND{/\char`\\~}
%
% Macro to output metamath raw text.
% This assumes \startprefix and \contprefix are set.
% NOTE: "\" is tricky to escape, use \SLASH, \TOR, and \TAND inside.
% Any use of "$ { ~ ^" must be escaped; ~ and ^ must be escaped specially.
% We escape { and } for consistency.
% For more about how this macro written, see:
% https://stackoverflow.com/questions/4073674/
% how-to-disable-indentation-in-particular-section-in-latex/4075706
% Use frenchspacing, or "e." will get an extra space after it.
\newlength\mystoreparindent
\newlength\mystorehangindent
\newenvironment{mmraw}{%
\setlength{\mystoreparindent}{\the\parindent}
\setlength{\mystorehangindent}{\the\hangindent}
\setlength{\parindent}{0pt} % TODO - we'll put in the \startprefix instead
\setlength{\hangindent}{\wd\the\contprefix}
\begin{flushleft}
\begin{frenchspacing}
\begin{tt}
{\unhcopy\startprefix}%
}{%
\end{tt}
\end{frenchspacing}
\end{flushleft}
\setlength{\parindent}{\mystoreparindent}
\setlength{\hangindent}{\mystorehangindent}
\vskip 1ex
}

\needspace{5\baselineskip}
\subsection{Propositional Calculus}\label{propcalc}\index{axioms of
propositional calculus}

\needspace{2\baselineskip}
Axiom of Simplification.\label{ax1}

\setbox\startprefix=\hbox{\tt \ \ ax-1\ \$a\ }
\setbox\contprefix=\hbox{\tt \ \ \ \ \ \ \ \ \ \ }
\startm
\m{\vdash}\m{(}\m{\varphi}\m{\rightarrow}\m{(}\m{\psi}\m{\rightarrow}\m{\varphi}\m{)}
\m{)}
\endm

\needspace{3\baselineskip}
\noindent Axiom of Distribution.

\setbox\startprefix=\hbox{\tt \ \ ax-2\ \$a\ }
\setbox\contprefix=\hbox{\tt \ \ \ \ \ \ \ \ \ \ }
\startm
\m{\vdash}\m{(}\m{(}\m{\varphi}\m{\rightarrow}\m{(}\m{\psi}\m{\rightarrow}\m{\chi}
\m{)}\m{)}\m{\rightarrow}\m{(}\m{(}\m{\varphi}\m{\rightarrow}\m{\psi}\m{)}\m{
\rightarrow}\m{(}\m{\varphi}\m{\rightarrow}\m{\chi}\m{)}\m{)}\m{)}
\endm

\needspace{2\baselineskip}
\noindent Axiom of Contraposition.

\setbox\startprefix=\hbox{\tt \ \ ax-3\ \$a\ }
\setbox\contprefix=\hbox{\tt \ \ \ \ \ \ \ \ \ \ }
\startm
\m{\vdash}\m{(}\m{(}\m{\lnot}\m{\varphi}\m{\rightarrow}\m{\lnot}\m{\psi}\m{)}\m{
\rightarrow}\m{(}\m{\psi}\m{\rightarrow}\m{\varphi}\m{)}\m{)}
\endm


\needspace{4\baselineskip}
\noindent Rule of Modus Ponens.\label{axmp}\index{modus ponens}

\setbox\startprefix=\hbox{\tt \ \ min\ \$e\ }
\setbox\contprefix=\hbox{\tt \ \ \ \ \ \ \ \ \ }
\startm
\m{\vdash}\m{\varphi}
\endm

\setbox\startprefix=\hbox{\tt \ \ maj\ \$e\ }
\setbox\contprefix=\hbox{\tt \ \ \ \ \ \ \ \ \ }
\startm
\m{\vdash}\m{(}\m{\varphi}\m{\rightarrow}\m{\psi}\m{)}
\endm

\setbox\startprefix=\hbox{\tt \ \ ax-mp\ \$a\ }
\setbox\contprefix=\hbox{\tt \ \ \ \ \ \ \ \ \ \ \ }
\startm
\m{\vdash}\m{\psi}
\endm


\needspace{7\baselineskip}
\subsection{Axioms of Predicate Calculus with Equality---Tarski's S2}\index{axioms of predicate calculus}

\needspace{3\baselineskip}
\noindent Rule of Generalization.\index{rule of generalization}

\setbox\startprefix=\hbox{\tt \ \ ax-g.1\ \$e\ }
\setbox\contprefix=\hbox{\tt \ \ \ \ \ \ \ \ \ \ \ \ }
\startm
\m{\vdash}\m{\varphi}
\endm

\setbox\startprefix=\hbox{\tt \ \ ax-gen\ \$a\ }
\setbox\contprefix=\hbox{\tt \ \ \ \ \ \ \ \ \ \ \ \ }
\startm
\m{\vdash}\m{\forall}\m{x}\m{\varphi}
\endm

\needspace{2\baselineskip}
\noindent Axiom of Quantified Implication.

\setbox\startprefix=\hbox{\tt \ \ ax-4\ \$a\ }
\setbox\contprefix=\hbox{\tt \ \ \ \ \ \ \ \ \ \ }
\startm
\m{\vdash}\m{(}\m{\forall}\m{x}\m{(}\m{\forall}\m{x}\m{\varphi}\m{\rightarrow}\m{
\psi}\m{)}\m{\rightarrow}\m{(}\m{\forall}\m{x}\m{\varphi}\m{\rightarrow}\m{
\forall}\m{x}\m{\psi}\m{)}\m{)}
\endm

\needspace{3\baselineskip}
\noindent Axiom of Distinctness.

% Aka: Add $d x ph $.
\setbox\startprefix=\hbox{\tt \ \ ax-5\ \$a\ }
\setbox\contprefix=\hbox{\tt \ \ \ \ \ \ \ \ \ \ }
\startm
\m{\vdash}\m{(}\m{\varphi}\m{\rightarrow}\m{\forall}\m{x}\m{\varphi}\m{)}\m{where}\m{ }\m{\$d}\m{ }\m{x}\m{ }\m{\varphi}\m{ }\m{(}\m{x}\m{ }\m{does}\m{ }\m{not}\m{ }\m{occur}\m{ }\m{in}\m{ }\m{\varphi}\m{)}
\endm

\needspace{2\baselineskip}
\noindent Axiom of Existence.

\setbox\startprefix=\hbox{\tt \ \ ax-6\ \$a\ }
\setbox\contprefix=\hbox{\tt \ \ \ \ \ \ \ \ \ \ }
\startm
\m{\vdash}\m{(}\m{\forall}\m{x}\m{(}\m{x}\m{=}\m{y}\m{\rightarrow}\m{\forall}
\m{x}\m{\varphi}\m{)}\m{\rightarrow}\m{\varphi}\m{)}
\endm

\needspace{2\baselineskip}
\noindent Axiom of Equality.

\setbox\startprefix=\hbox{\tt \ \ ax-7\ \$a\ }
\setbox\contprefix=\hbox{\tt \ \ \ \ \ \ \ \ \ \ }
\startm
\m{\vdash}\m{(}\m{x}\m{=}\m{y}\m{\rightarrow}\m{(}\m{x}\m{=}\m{z}\m{
\rightarrow}\m{y}\m{=}\m{z}\m{)}\m{)}
\endm

\needspace{2\baselineskip}
\noindent Axiom of Left Equality for Binary Predicate.

\setbox\startprefix=\hbox{\tt \ \ ax-8\ \$a\ }
\setbox\contprefix=\hbox{\tt \ \ \ \ \ \ \ \ \ \ \ }
\startm
\m{\vdash}\m{(}\m{x}\m{=}\m{y}\m{\rightarrow}\m{(}\m{x}\m{\in}\m{z}\m{
\rightarrow}\m{y}\m{\in}\m{z}\m{)}\m{)}
\endm

\needspace{2\baselineskip}
\noindent Axiom of Right Equality for Binary Predicate.

\setbox\startprefix=\hbox{\tt \ \ ax-9\ \$a\ }
\setbox\contprefix=\hbox{\tt \ \ \ \ \ \ \ \ \ \ \ }
\startm
\m{\vdash}\m{(}\m{x}\m{=}\m{y}\m{\rightarrow}\m{(}\m{z}\m{\in}\m{x}\m{
\rightarrow}\m{z}\m{\in}\m{y}\m{)}\m{)}
\endm


\needspace{4\baselineskip}
\subsection{Axioms of Predicate Calculus with Equality---Auxiliary}\index{axioms of predicate calculus - auxiliary}

\needspace{2\baselineskip}
\noindent Axiom of Quantified Negation.

\setbox\startprefix=\hbox{\tt \ \ ax-10\ \$a\ }
\setbox\contprefix=\hbox{\tt \ \ \ \ \ \ \ \ \ \ }
\startm
\m{\vdash}\m{(}\m{\lnot}\m{\forall}\m{x}\m{\lnot}\m{\forall}\m{x}\m{\varphi}\m{
\rightarrow}\m{\varphi}\m{)}
\endm

\needspace{2\baselineskip}
\noindent Axiom of Quantifier Commutation.

\setbox\startprefix=\hbox{\tt \ \ ax-11\ \$a\ }
\setbox\contprefix=\hbox{\tt \ \ \ \ \ \ \ \ \ \ }
\startm
\m{\vdash}\m{(}\m{\forall}\m{x}\m{\forall}\m{y}\m{\varphi}\m{\rightarrow}\m{
\forall}\m{y}\m{\forall}\m{x}\m{\varphi}\m{)}
\endm

\needspace{3\baselineskip}
\noindent Axiom of Substitution.

\setbox\startprefix=\hbox{\tt \ \ ax-12\ \$a\ }
\setbox\contprefix=\hbox{\tt \ \ \ \ \ \ \ \ \ \ \ }
\startm
\m{\vdash}\m{(}\m{\lnot}\m{\forall}\m{x}\m{\,x}\m{=}\m{y}\m{\rightarrow}\m{(}
\m{x}\m{=}\m{y}\m{\rightarrow}\m{(}\m{\varphi}\m{\rightarrow}\m{\forall}\m{x}\m{(}
\m{x}\m{=}\m{y}\m{\rightarrow}\m{\varphi}\m{)}\m{)}\m{)}\m{)}
\endm

\needspace{3\baselineskip}
\noindent Axiom of Quantified Equality.

\setbox\startprefix=\hbox{\tt \ \ ax-13\ \$a\ }
\setbox\contprefix=\hbox{\tt \ \ \ \ \ \ \ \ \ \ \ }
\startm
\m{\vdash}\m{(}\m{\lnot}\m{\forall}\m{z}\m{\,z}\m{=}\m{x}\m{\rightarrow}\m{(}
\m{\lnot}\m{\forall}\m{z}\m{\,z}\m{=}\m{y}\m{\rightarrow}\m{(}\m{x}\m{=}\m{y}
\m{\rightarrow}\m{\forall}\m{z}\m{\,x}\m{=}\m{y}\m{)}\m{)}\m{)}
\endm

% \noindent Axiom of Quantifier Substitution
%
% \setbox\startprefix=\hbox{\tt \ \ ax-c11n\ \$a\ }
% \setbox\contprefix=\hbox{\tt \ \ \ \ \ \ \ \ \ \ \ }
% \startm
% \m{\vdash}\m{(}\m{\forall}\m{x}\m{\,x}\m{=}\m{y}\m{\rightarrow}\m{(}\m{\forall}
% \m{x}\m{\varphi}\m{\rightarrow}\m{\forall}\m{y}\m{\varphi}\m{)}\m{)}
% \endm
%
% \noindent Axiom of Distinct Variables. (This axiom requires
% that two individual variables
% be distinct\index{\texttt{\$d} statement}\index{distinct
% variables}.)
%
% \setbox\startprefix=\hbox{\tt \ \ \ \ \ \ \ \ \$d\ }
% \setbox\contprefix=\hbox{\tt \ \ \ \ \ \ \ \ \ \ \ }
% \startm
% \m{x}\m{\,}\m{y}
% \endm
%
% \setbox\startprefix=\hbox{\tt \ \ ax-c16\ \$a\ }
% \setbox\contprefix=\hbox{\tt \ \ \ \ \ \ \ \ \ \ \ }
% \startm
% \m{\vdash}\m{(}\m{\forall}\m{x}\m{\,x}\m{=}\m{y}\m{\rightarrow}\m{(}\m{\varphi}\m{
% \rightarrow}\m{\forall}\m{x}\m{\varphi}\m{)}\m{)}
% \endm

% \noindent Axiom of Quantifier Introduction (2).  (This axiom requires
% that the individual variable not occur in the
% wff\index{\texttt{\$d} statement}\index{distinct variables}.)
%
% \setbox\startprefix=\hbox{\tt \ \ \ \ \ \ \ \ \$d\ }
% \setbox\contprefix=\hbox{\tt \ \ \ \ \ \ \ \ \ \ \ }
% \startm
% \m{x}\m{\,}\m{\varphi}
% \endm
% \setbox\startprefix=\hbox{\tt \ \ ax-5\ \$a\ }
% \setbox\contprefix=\hbox{\tt \ \ \ \ \ \ \ \ \ \ \ }
% \startm
% \m{\vdash}\m{(}\m{\varphi}\m{\rightarrow}\m{\forall}\m{x}\m{\varphi}\m{)}
% \endm

\subsection{Set Theory}\label{mmsettheoryaxioms}

In order to make the axioms of set theory\index{axioms of set theory} a little
more compact, there are several definitions from logic that we make use of
implicitly, namely, ``logical {\sc and},''\index{conjunction ($\wedge$)}
\index{logical {\sc and} ($\wedge$)} ``logical equivalence,''\index{logical
equivalence ($\leftrightarrow$)}\index{biconditional ($\leftrightarrow$)} and
``there exists.''\index{existential quantifier ($\exists$)}

\begin{center}\begin{tabular}{rcl}
  $( \varphi \wedge \psi )$ &\mbox{stands for}& $\neg ( \varphi
     \rightarrow \neg \psi )$\\
  $( \varphi \leftrightarrow \psi )$& \mbox{stands
     for}& $( ( \varphi \rightarrow \psi ) \wedge
     ( \psi \rightarrow \varphi ) )$\\
  $\exists x \,\varphi$ &\mbox{stands for}& $\neg \forall x \neg \varphi$
\end{tabular}\end{center}

In addition, the axioms of set theory require that all variables be
dis\-tinct,\index{distinct variables}\footnote{Set theory axioms can be
devised so that {\em no} variables are required to be distinct,
provided we replace \texttt{ax-c16} with an axiom stating that ``at
least two things exist,'' thus
making \texttt{ax-5} the only other axiom requiring the
\texttt{\$d} statement.  These axioms are unconventional and are not
presented here, but they can be found on the \url{http://metamath.org}
web site.  See also the Comment on
p.~\pageref{nodd}.}\index{\texttt{\$d} statement} thus we also assume:
\begin{center}
  \texttt{\$d }$x\,y\,z\,w$
\end{center}

\needspace{2\baselineskip}
\noindent Axiom of Extensionality.\index{Axiom of Extensionality}

\setbox\startprefix=\hbox{\tt \ \ ax-ext\ \$a\ }
\setbox\contprefix=\hbox{\tt \ \ \ \ \ \ \ \ \ \ \ \ }
\startm
\m{\vdash}\m{(}\m{\forall}\m{x}\m{(}\m{x}\m{\in}\m{y}\m{\leftrightarrow}\m{x}
\m{\in}\m{z}\m{)}\m{\rightarrow}\m{y}\m{=}\m{z}\m{)}
\endm

\needspace{3\baselineskip}
\noindent Axiom of Replacement.\index{Axiom of Replacement}

\setbox\startprefix=\hbox{\tt \ \ ax-rep\ \$a\ }
\setbox\contprefix=\hbox{\tt \ \ \ \ \ \ \ \ \ \ \ \ }
\startm
\m{\vdash}\m{(}\m{\forall}\m{w}\m{\exists}\m{y}\m{\forall}\m{z}\m{(}\m{%
\forall}\m{y}\m{\varphi}\m{\rightarrow}\m{z}\m{=}\m{y}\m{)}\m{\rightarrow}\m{%
\exists}\m{y}\m{\forall}\m{z}\m{(}\m{z}\m{\in}\m{y}\m{\leftrightarrow}\m{%
\exists}\m{w}\m{(}\m{w}\m{\in}\m{x}\m{\wedge}\m{\forall}\m{y}\m{\varphi}\m{)}%
\m{)}\m{)}
\endm

\needspace{2\baselineskip}
\noindent Axiom of Union.\index{Axiom of Union}

\setbox\startprefix=\hbox{\tt \ \ ax-un\ \$a\ }
\setbox\contprefix=\hbox{\tt \ \ \ \ \ \ \ \ \ \ \ }
\startm
\m{\vdash}\m{\exists}\m{x}\m{\forall}\m{y}\m{(}\m{\exists}\m{x}\m{(}\m{y}\m{
\in}\m{x}\m{\wedge}\m{x}\m{\in}\m{z}\m{)}\m{\rightarrow}\m{y}\m{\in}\m{x}\m{)}
\endm

\needspace{2\baselineskip}
\noindent Axiom of Power Sets.\index{Axiom of Power Sets}

\setbox\startprefix=\hbox{\tt \ \ ax-pow\ \$a\ }
\setbox\contprefix=\hbox{\tt \ \ \ \ \ \ \ \ \ \ \ \ }
\startm
\m{\vdash}\m{\exists}\m{x}\m{\forall}\m{y}\m{(}\m{\forall}\m{x}\m{(}\m{x}\m{
\in}\m{y}\m{\rightarrow}\m{x}\m{\in}\m{z}\m{)}\m{\rightarrow}\m{y}\m{\in}\m{x}
\m{)}
\endm

\needspace{3\baselineskip}
\noindent Axiom of Regularity.\index{Axiom of Regularity}

\setbox\startprefix=\hbox{\tt \ \ ax-reg\ \$a\ }
\setbox\contprefix=\hbox{\tt \ \ \ \ \ \ \ \ \ \ \ \ }
\startm
\m{\vdash}\m{(}\m{\exists}\m{x}\m{\,x}\m{\in}\m{y}\m{\rightarrow}\m{\exists}
\m{x}\m{(}\m{x}\m{\in}\m{y}\m{\wedge}\m{\forall}\m{z}\m{(}\m{z}\m{\in}\m{x}\m{
\rightarrow}\m{\lnot}\m{z}\m{\in}\m{y}\m{)}\m{)}\m{)}
\endm

\needspace{3\baselineskip}
\noindent Axiom of Infinity.\index{Axiom of Infinity}

\setbox\startprefix=\hbox{\tt \ \ ax-inf\ \$a\ }
\setbox\contprefix=\hbox{\tt \ \ \ \ \ \ \ \ \ \ \ \ \ \ \ }
\startm
\m{\vdash}\m{\exists}\m{x}\m{(}\m{y}\m{\in}\m{x}\m{\wedge}\m{\forall}\m{y}%
\m{(}\m{y}\m{\in}\m{x}\m{\rightarrow}\m{\exists}\m{z}\m{(}\m{y}\m{\in}\m{z}\m{%
\wedge}\m{z}\m{\in}\m{x}\m{)}\m{)}\m{)}
\endm

\needspace{4\baselineskip}
\noindent Axiom of Choice.\index{Axiom of Choice}

\setbox\startprefix=\hbox{\tt \ \ ax-ac\ \$a\ }
\setbox\contprefix=\hbox{\tt \ \ \ \ \ \ \ \ \ \ \ \ \ \ }
\startm
\m{\vdash}\m{\exists}\m{x}\m{\forall}\m{y}\m{\forall}\m{z}\m{(}\m{(}\m{y}\m{%
\in}\m{z}\m{\wedge}\m{z}\m{\in}\m{w}\m{)}\m{\rightarrow}\m{\exists}\m{w}\m{%
\forall}\m{y}\m{(}\m{\exists}\m{w}\m{(}\m{(}\m{y}\m{\in}\m{z}\m{\wedge}\m{z}%
\m{\in}\m{w}\m{)}\m{\wedge}\m{(}\m{y}\m{\in}\m{w}\m{\wedge}\m{w}\m{\in}\m{x}%
\m{)}\m{)}\m{\leftrightarrow}\m{y}\m{=}\m{w}\m{)}\m{)}
\endm

\subsection{That's It}

There you have it, the axioms for (essentially) all of mathematics!
Wonder at them and stare at them in awe.  Put a copy in your wallet, and
you will carry in your pocket the encoding for all theorems ever proved
and that ever will be proved, from the most mundane to the most
profound.

\section{A Hierarchy of Definitions}\label{hierarchy}

The axioms in the previous section in principle embody everything that can be
done within standard mathematics.  However, it is impractical to accomplish
very much by using them directly, for even simple concepts (from a human
perspective) can involve extremely long, incomprehensible formulas.
Mathematics is made practical by introducing definitions\index{definition}.
Definitions usually introduce new symbols, or at least new relationships among
existing symbols, to abbreviate more complex formulas.  An important
requirement for a definition is that there exist a straightforward
(algorithmic) method for eliminating the abbreviation by expanding it into the
more primitive symbol string that it represents.  Some
important definitions included in
the file \texttt{set.mm} are listed in this section for reference, and also to
give you a feel for why something like $\omega$\index{omega ($\omega$)} (the
set of natural numbers\index{natural number} 0, 1, 2,\ldots) becomes very
complicated when completely expanded into primitive symbols.

What is the motivation for definitions, aside from allowing complicated
expressions to be expressed more simply?  In the case of  $\omega$, one goal is
to provide a basis for the theory of natural numbers.\index{natural number}
Before set theory was invented, a set of axioms for arithmetic, called Peano's
postulates\index{Peano's postulates}, was devised and shown to have the
properties one expects for natural numbers.  Now anyone can postulate a
set of axioms, but if the axioms are inconsistent contradictions can be derived
from them.  Once a contradiction is derived, anything can be trivially
proved, including
all the facts of arithmetic and their negations.  To ensure that an
axiom system is at least as reliable as the axioms for set theory, we can
define sets and operations on those sets that satisfy the new axioms. In the
\texttt{set.mm} Metamath database, we prove that the elements of $\omega$ satisfy
Peano's postulates, and it's a long and hard journey to get there directly
from the axioms of set theory.  But the result is confidence in the
foundations of arithmetic.  And there is another advantage:  we now have all
the tools of set theory at our disposal for manipulating objects that obey the
axioms for arithmetic.

What are the criteria we use for definitions?  First, and of utmost importance,
the definition should not be {\em creative}\index{creative
definition}\index{definition!creative}, that
is it should not allow an expression that previously qualified as a wff but
was not provable, to become provable.   Second, the definition should be {\em
eliminable}\index{definition!eliminability}, that is, there should exist an
algorithmic method for converting any expression using the definition into
a logically equivalent expression that previously qualified as a wff.

In almost all cases below, definitions connect two expressions with either
$\leftrightarrow$ or $=$.  Eliminating\footnote{Here we mean the
elimination that a human might do in his or her head.  To eliminate them as
part of a Metamath proof we would invoke one of a number of
theorems that deal with transitivity of equivalence or equality; there are
many such examples in the proofs in \texttt{set.mm}.} such a definition is a
simple matter of substituting the expression on the left-hand side ({\em
definiendum}\index{definiendum} or thing being defined) with the equivalent,
more primitive expression on the right-hand side ({\em
definiens}\index{definiens} or definition).

Often a definition has variables on the right-hand side which do not appear on
the left-hand side; these are called {\em dummy variables}.\index{dummy
variable!in definitions}  In this case, any
allowable substitution (such as a new, distinct
variable) can be used when the definition is eliminated.  Dummy variables may
be used only if they are {\em effectively bound}\index{effectively bound
variable}, meaning that the definition will remain logically equivalent upon
any substitution of a dummy variable with any other {\em qualifying
expression}\index{qualifying expression}, i.e.\ any symbol string (such as
another variable) that
meets the restrictions on the dummy variable imposed by \texttt{\$d} and
\texttt{\$f} statements.  For example, we could define a constant $\perp$
(inverted tee, meaning logical ``false'') as $( \varphi \wedge \lnot \varphi
)$, i.e.\ ``phi and not phi.''  Here $\varphi$ is effectively bound because the
definition remains logically equivalent when we replace $\varphi$ with any
other wff.  (It is actually \texttt{df-fal}
in \texttt{set.mm}, which defines $\perp$.)

There are two cases where eliminating definitions is a little more
complex.  These cases are the definitions \texttt{df-bi} and
\texttt{df-cleq}.  The first stretches the concept of a definition a
little, as in effect it ``defines a definition;'' however, it meets our
requirements for a definition in that it is eliminable and does not
strengthen the language.  Theorem \texttt{bii} shows the substitution
needed to eliminate the $\leftrightarrow$\index{logical equivalence
($\leftrightarrow$)}\index{biconditional ($\leftrightarrow$)} symbol.

Definition \texttt{df-cleq}\index{equality ($=$)} extends the usage of
the equality symbol to include ``classes''\index{class} in set theory.  The
reason it is potentially problematic is that it can lead to statements which
do not follow from logic alone but presuppose the Axiom of
Extensionality\index{Axiom of Extensionality}, so we include this axiom
as a hypothesis for the definition.  We could have made \texttt{df-cleq} directly
eliminable by introducing a new equality symbol, but have chosen not to do so
in keeping with standard textbook practice.  Definitions such as \texttt{df-cleq}
that extend the meaning of existing symbols must be introduced carefully so
that they do not lead to contradictions.  Definition \texttt{df-clel} also
extends the meaning of an existing symbol ($\in$); while it doesn't strengthen
the language like \texttt{df-cleq}, this is not obvious and it must also be
subject to the same scrutiny.

Exercise:  Study how the wff $x\in\omega$, meaning ``$x$ is a natural
number,'' could be expanded in terms of primitive symbols, starting with the
definitions \texttt{df-clel} on p.~\pageref{dfclel} and \texttt{df-om} on
p.~\pageref{dfom} and working your way back.  Don't bother to work out the
details; just make sure that you understand how you could do it in principle.
The answer is shown in the footnote on p.~\pageref{expandom}.  If you
actually do work it out, you won't get exactly the same answer because we used
a few simplifications such as discarding occurrences of $\lnot\lnot$ (double
negation).

In the definitions below, we have placed the {\sc ascii} Metamath source
below each of the formulas to help you become familiar with the
notation in the database.  For simplicity, the necessary \texttt{\$f}
and \texttt{\$d} statements are not shown.  If you are in doubt, use the
\texttt{show statement}\index{\texttt{show statement} command} command
in the Metamath program to see the full statement.
A selection of this notation is summarized in Appendix~\ref{ASCII}.

To understand the motivation for these definitions, you should consult the
references indicated:  Takeuti and Zaring \cite{Takeuti}\index{Takeuti, Gaisi},
Quine \cite{Quine}\index{Quine, Willard Van Orman}, Bell and Machover
\cite{Bell}\index{Bell, J. L.}, and Enderton \cite{Enderton}\index{Enderton,
Herbert B.}.  Our list of definitions is provided more for reference than as a
learning aid.  However, by looking at a few of them you can gain a feel for
how the hierarchy is built up.  The definitions are a representative sample of
the many definitions
in \texttt{set.mm}, but they are complete with respect to the
theorem examples we will present in Section~\ref{sometheorems}.  Also, some are
slightly different from, but logically equivalent to, the ones in \texttt{set.mm}
(some of which have been revised over time to shorten them, for example).

\subsection{Definitions for Propositional Calculus}\label{metadefprop}

The symbols $\varphi$, $\psi$, and $\chi$ represent wffs.

Our first definition introduces the biconditional
connective\footnote{The term ``connective'' is informally used to mean a
symbol that is placed between two variables or adjacent to a variable,
whereas a mathematical ``constant'' usually indicates a symbol such as
the number 0 that may replace a variable or metavariable.  From
Metamath's point of view, there is no distinction between a connective
and a constant; both are constants in the Metamath
language.}\index{connective}\index{constant} (also called logical
equivalence)\index{logical equivalence
($\leftrightarrow$)}\index{biconditional ($\leftrightarrow$)}.  Unlike
most traditional developments, we have chosen not to have a separate
symbol such as ``Df.'' to mean ``is defined as.''  Instead, we will use
the biconditional connective for this purpose, as it lets us use
logic to manipulate definitions directly.  Here we state the properties
of the biconditional connective with a carefully crafted \texttt{\$a}
statement, which effectively uses the biconditional connective to define
itself.  The $\leftrightarrow$ symbol can be eliminated from a formula
using theorem \texttt{bii}, which is derived later.

\vskip 2ex
\noindent Define the biconditional connective.\label{df-bi}

\vskip 0.5ex
\setbox\startprefix=\hbox{\tt \ \ df-bi\ \$a\ }
\setbox\contprefix=\hbox{\tt \ \ \ \ \ \ \ \ \ \ \ }
\startm
\m{\vdash}\m{\lnot}\m{(}\m{(}\m{(}\m{\varphi}\m{\leftrightarrow}\m{\psi}\m{)}%
\m{\rightarrow}\m{\lnot}\m{(}\m{(}\m{\varphi}\m{\rightarrow}\m{\psi}\m{)}\m{%
\rightarrow}\m{\lnot}\m{(}\m{\psi}\m{\rightarrow}\m{\varphi}\m{)}\m{)}\m{)}\m{%
\rightarrow}\m{\lnot}\m{(}\m{\lnot}\m{(}\m{(}\m{\varphi}\m{\rightarrow}\m{%
\psi}\m{)}\m{\rightarrow}\m{\lnot}\m{(}\m{\psi}\m{\rightarrow}\m{\varphi}\m{)}%
\m{)}\m{\rightarrow}\m{(}\m{\varphi}\m{\leftrightarrow}\m{\psi}\m{)}\m{)}\m{)}
\endm
\begin{mmraw}%
|- -. ( ( ( ph <-> ps ) -> -. ( ( ph -> ps ) ->
-. ( ps -> ph ) ) ) -> -. ( -. ( ( ph -> ps ) -> -. (
ps -> ph ) ) -> ( ph <-> ps ) ) ) \$.
\end{mmraw}

\noindent This theorem relates the biconditional connective to primitive
connectives and can be used to eliminate the $\leftrightarrow$ symbol from any
wff.

\vskip 0.5ex
\setbox\startprefix=\hbox{\tt \ \ bii\ \$p\ }
\setbox\contprefix=\hbox{\tt \ \ \ \ \ \ \ \ \ }
\startm
\m{\vdash}\m{(}\m{(}\m{\varphi}\m{\leftrightarrow}\m{\psi}\m{)}\m{\leftrightarrow}
\m{\lnot}\m{(}\m{(}\m{\varphi}\m{\rightarrow}\m{\psi}\m{)}\m{\rightarrow}\m{\lnot}
\m{(}\m{\psi}\m{\rightarrow}\m{\varphi}\m{)}\m{)}\m{)}
\endm
\begin{mmraw}%
|- ( ( ph <-> ps ) <-> -. ( ( ph -> ps ) -> -. ( ps -> ph ) ) ) \$= ... \$.
\end{mmraw}

\noindent Define disjunction ({\sc or}).\index{disjunction ($\vee$)}%
\index{logical or (vee)@logical {\sc or} ($\vee$)}%
\index{df-or@\texttt{df-or}}\label{df-or}

\vskip 0.5ex
\setbox\startprefix=\hbox{\tt \ \ df-or\ \$a\ }
\setbox\contprefix=\hbox{\tt \ \ \ \ \ \ \ \ \ \ \ }
\startm
\m{\vdash}\m{(}\m{(}\m{\varphi}\m{\vee}\m{\psi}\m{)}\m{\leftrightarrow}\m{(}\m{
\lnot}\m{\varphi}\m{\rightarrow}\m{\psi}\m{)}\m{)}
\endm
\begin{mmraw}%
|- ( ( ph \TOR ps ) <-> ( -. ph -> ps ) ) \$.
\end{mmraw}

\noindent Define conjunction ({\sc and}).\index{conjunction ($\wedge$)}%
\index{logical {\sc and} ($\wedge$)}%
\index{df-an@\texttt{df-an}}\label{df-an}

\vskip 0.5ex
\setbox\startprefix=\hbox{\tt \ \ df-an\ \$a\ }
\setbox\contprefix=\hbox{\tt \ \ \ \ \ \ \ \ \ \ \ }
\startm
\m{\vdash}\m{(}\m{(}\m{\varphi}\m{\wedge}\m{\psi}\m{)}\m{\leftrightarrow}\m{\lnot}
\m{(}\m{\varphi}\m{\rightarrow}\m{\lnot}\m{\psi}\m{)}\m{)}
\endm
\begin{mmraw}%
|- ( ( ph \TAND ps ) <-> -. ( ph -> -. ps ) ) \$.
\end{mmraw}

\noindent Define disjunction ({\sc or}) of 3 wffs.%
\index{df-3or@\texttt{df-3or}}\label{df-3or}

\vskip 0.5ex
\setbox\startprefix=\hbox{\tt \ \ df-3or\ \$a\ }
\setbox\contprefix=\hbox{\tt \ \ \ \ \ \ \ \ \ \ \ \ }
\startm
\m{\vdash}\m{(}\m{(}\m{\varphi}\m{\vee}\m{\psi}\m{\vee}\m{\chi}\m{)}\m{
\leftrightarrow}\m{(}\m{(}\m{\varphi}\m{\vee}\m{\psi}\m{)}\m{\vee}\m{\chi}\m{)}
\m{)}
\endm
\begin{mmraw}%
|- ( ( ph \TOR ps \TOR ch ) <-> ( ( ph \TOR ps ) \TOR ch ) ) \$.
\end{mmraw}

\noindent Define conjunction ({\sc and}) of 3 wffs.%
\index{df-3an}\label{df-3an}

\vskip 0.5ex
\setbox\startprefix=\hbox{\tt \ \ df-3an\ \$a\ }
\setbox\contprefix=\hbox{\tt \ \ \ \ \ \ \ \ \ \ \ \ }
\startm
\m{\vdash}\m{(}\m{(}\m{\varphi}\m{\wedge}\m{\psi}\m{\wedge}\m{\chi}\m{)}\m{
\leftrightarrow}\m{(}\m{(}\m{\varphi}\m{\wedge}\m{\psi}\m{)}\m{\wedge}\m{\chi}
\m{)}\m{)}
\endm

\begin{mmraw}%
|- ( ( ph \TAND ps \TAND ch ) <-> ( ( ph \TAND ps ) \TAND ch ) ) \$.
\end{mmraw}

\subsection{Definitions for Predicate Calculus}\label{metadefpred}

The symbols $x$, $y$, and $z$ represent individual variables of predicate
calculus.  In this section, they are not necessarily distinct unless it is
explicitly
mentioned.

\vskip 2ex
\noindent Define existential quantification.
The expression $\exists x \varphi$ means
``there exists an $x$ where $\varphi$ is true.''\index{existential quantifier
($\exists$)}\label{df-ex}

\vskip 0.5ex
\setbox\startprefix=\hbox{\tt \ \ df-ex\ \$a\ }
\setbox\contprefix=\hbox{\tt \ \ \ \ \ \ \ \ \ \ \ }
\startm
\m{\vdash}\m{(}\m{\exists}\m{x}\m{\varphi}\m{\leftrightarrow}\m{\lnot}\m{\forall}
\m{x}\m{\lnot}\m{\varphi}\m{)}
\endm
\begin{mmraw}%
|- ( E. x ph <-> -. A. x -. ph ) \$.
\end{mmraw}

\noindent Define proper substitution.\index{proper
substitution}\index{substitution!proper}\label{df-sb}
In our notation, we use $[ y / x ] \varphi$ to mean ``the wff that
results when $y$ is properly substituted for $x$ in the wff
$\varphi$.''\footnote{
This can also be described
as substituting $x$ with $y$, $y$ properly replaces $x$, or
$x$ is properly replaced by $y$.}
% This is elsb4, though it currently says: ( [ x / y ] z e. y <-> z e. x )
For example,
$[ y / x ] z \in x$ is the same as $z \in y$.
One way to remember this notation is to notice that it looks like division
and recall that $( y / x ) \cdot x $ is $y$ (when $x \neq 0$).
The notation is different from the notation $\varphi ( x | y )$
that is sometimes used, because the latter notation is ambiguous for us:
for example, we don't know whether $\lnot \varphi ( x | y )$ is to be
interpreted as $\lnot ( \varphi ( x | y ) )$ or
$( \lnot \varphi ) ( x | y )$.\footnote{Because of the way
we initially defined wffs, this is the case
with any postfix connective\index{postfix connective} (one occurring after the
symbols being connected) or infix connective\index{infix connective} (one
occurring between the symbols being connected).  Metamath does not have a
built-in notion of operator binding strength that could eliminate the
ambiguity.  The initial parenthesis effectively provides a prefix
connective\index{prefix connective} to eliminate ambiguity.  Some conventions,
such as Polish notation\index{Polish notation} used in the 1930's and 1940's
by Polish logicians, use only prefix connectives and thus allow the total
elimination of parentheses, at the expense of readability.  In Metamath we
could actually redefine all notation to be Polish if we wanted to without
having to change any proofs!}  Other texts often use $\varphi(y)$ to indicate
our $[ y / x ] \varphi$, but this notation is even more ambiguous since there is
no explicit indication of what is being substituted.
Note that this
definition is valid even when
$x$ and $y$ are the same variable.  The first conjunct is a ``trick'' used to
achieve this property, making the definition look somewhat peculiar at
first.

\vskip 0.5ex
\setbox\startprefix=\hbox{\tt \ \ df-sb\ \$a\ }
\setbox\contprefix=\hbox{\tt \ \ \ \ \ \ \ \ \ \ \ }
\startm
\m{\vdash}\m{(}\m{[}\m{y}\m{/}\m{x}\m{]}\m{\varphi}\m{\leftrightarrow}\m{(}%
\m{(}\m{x}\m{=}\m{y}\m{\rightarrow}\m{\varphi}\m{)}\m{\wedge}\m{\exists}\m{x}%
\m{(}\m{x}\m{=}\m{y}\m{\wedge}\m{\varphi}\m{)}\m{)}\m{)}
\endm
\begin{mmraw}%
|- ( [ y / x ] ph <-> ( ( x = y -> ph ) \TAND E. x ( x = y \TAND ph ) ) ) \$.
\end{mmraw}


\noindent Define existential uniqueness\index{existential uniqueness
quantifier ($\exists "!$)} (``there exists exactly one'').  Note that $y$ is a
variable distinct from $x$ and not occurring in $\varphi$.\label{df-eu}

\vskip 0.5ex
\setbox\startprefix=\hbox{\tt \ \ df-eu\ \$a\ }
\setbox\contprefix=\hbox{\tt \ \ \ \ \ \ \ \ \ \ \ }
\startm
\m{\vdash}\m{(}\m{\exists}\m{{!}}\m{x}\m{\varphi}\m{\leftrightarrow}\m{\exists}
\m{y}\m{\forall}\m{x}\m{(}\m{\varphi}\m{\leftrightarrow}\m{x}\m{=}\m{y}\m{)}\m{)}
\endm

\begin{mmraw}%
|- ( E! x ph <-> E. y A. x ( ph <-> x = y ) ) \$.
\end{mmraw}

\subsection{Definitions for Set Theory}\label{setdefinitions}

The symbols $x$, $y$, $z$, and $w$ represent individual variables of
predicate calculus, which in set theory are understood to be sets.
However, using only the constructs shown so far would be very inconvenient.

To make set theory more practical, we introduce the notion of a ``class.''
A class\index{class} is either a set variable (such as $x$) or an
expression of the form $\{ x | \varphi\}$ (called an ``abstraction
class''\index{abstraction class}\index{class abstraction}).  Note that
sets (i.e.\ individual variables) always exist (this is a theorem of
logic, namely $\exists y \, y = x$ for any set $x$), whereas classes may
or may not exist (i.e.\ $\exists y \, y = A$ may or may not be true).
If a class does not exist it is called a ``proper class.''\index{proper
class}\index{class!proper} Definitions \texttt{df-clab},
\texttt{df-cleq}, and \texttt{df-clel} can be used to convert an
expression containing classes into one containing only set variables and
wff metavariables.

The symbols $A$, $B$, $C$, $D$, $ F$, $G$, and $R$ are metavariables that range
over classes.  A class metavariable $A$ may be eliminated from a wff by
replacing it with $\{ x|\varphi\}$ where neither $x$ nor $\varphi$ occur in
the wff.

The theory of classes can be shown to be an eliminable and conservative
extension of set theory. The \textbf{eliminability}
property shows that for every
formula in the extended language we can build a logically equivalent
formula in the basic language; so that even if the extended language
provides more ease to convey and formulate mathematical ideas for set
theory, its expressive power does not in fact strengthen the basic
language's expressive power.
The \textbf{conservation} property shows that for
every proof of a formula of the basic language in the extended system
we can build another proof of the same formula in the basic system;
so that, concerning theorems on sets only, the deductive powers of
the extended system and of the basic system are identical. Together,
these properties mean that the extended language can be treated as a
definitional extension that is \textbf{sound}.

A rigorous justification, which we will not give here, can be found in
Levy \cite[pp.~357-366]{Levy} supplementing his informal introduction to class
theory on pp.~7-17. Two other good treatments of class theory are provided
by Quine \cite[pp.~15-21]{Quine}\index{Quine, Willard Van Orman}
and also \cite[pp.~10-14]{Takeuti}\index{Takeuti, Gaisi}.
Quine's exposition (he calls them virtual classes)
is nicely written and very readable.

In the rest of this
section, individual variables are always assumed to be distinct from
each other unless otherwise indicated.  In addition, dummy variables on the
right-hand side of a definition do not occur in the class and wff
metavariables in the definition.

The definitions we present here are a partial but self-contained
collection selected from several hundred that appear in the current
\texttt{set.mm} database.  They are adequate for a basic development of
elementary set theory.

\vskip 2ex
\noindent Define the abstraction class.\index{abstraction class}\index{class
abstraction}\label{df-clab}  $x$ and $y$
need not be distinct.  Definition 2.1 of Quine, p.~16.  This definition may
seem puzzling since it is shorter than the expression being defined and does not
buy us anything in terms of brevity.  The reason we introduce this definition
is because it fits in neatly with the extension of the $\in$ connective
provided by \texttt{df-clel}.

\vskip 0.5ex
\setbox\startprefix=\hbox{\tt \ \ df-clab\ \$a\ }
\setbox\contprefix=\hbox{\tt \ \ \ \ \ \ \ \ \ \ \ \ \ }
\startm
\m{\vdash}\m{(}\m{x}\m{\in}\m{\{}\m{y}\m{|}\m{\varphi}\m{\}}\m{%
\leftrightarrow}\m{[}\m{x}\m{/}\m{y}\m{]}\m{\varphi}\m{)}
\endm
\begin{mmraw}%
|- ( x e. \{ y | ph \} <-> [ x / y ] ph ) \$.
\end{mmraw}

\noindent Define the equality connective between classes\index{class
equality}\label{df-cleq}.  See Quine or Chapter 4 of Takeuti and Zaring for its
justification and methods for eliminating it.  This is an example of a
somewhat ``dangerous'' definition, because it extends the use of the
existing equality symbol rather than introducing a new symbol, allowing
us to make statements in the original language that may not be true.
For example, it permits us to deduce $y = z \leftrightarrow \forall x (
x \in y \leftrightarrow x \in z )$ which is not a theorem of logic but
rather presupposes the Axiom of Extensionality,\index{Axiom of
Extensionality} which we include as a hypothesis so that we can know
when this axiom is assumed in a proof (with the \texttt{show
trace{\char`\_}back} command).  We could avoid the danger by introducing
another symbol, say $\eqcirc$, in place of $=$; this
would also have the advantage of making elimination of the definition
straightforward and would eliminate the need for Extensionality as a
hypothesis.  We would then also have the advantage of being able to
identify exactly where Extensionality truly comes into play.  One of our
theorems would be $x \eqcirc y \leftrightarrow x = y$
by invoking Extensionality.  However in keeping with standard practice
we retain the ``dangerous'' definition.

\vskip 0.5ex
\setbox\startprefix=\hbox{\tt \ \ df-cleq.1\ \$e\ }
\setbox\contprefix=\hbox{\tt \ \ \ \ \ \ \ \ \ \ \ \ \ \ \ }
\startm
\m{\vdash}\m{(}\m{\forall}\m{x}\m{(}\m{x}\m{\in}\m{y}\m{\leftrightarrow}\m{x}
\m{\in}\m{z}\m{)}\m{\rightarrow}\m{y}\m{=}\m{z}\m{)}
\endm
\setbox\startprefix=\hbox{\tt \ \ df-cleq\ \$a\ }
\setbox\contprefix=\hbox{\tt \ \ \ \ \ \ \ \ \ \ \ \ \ }
\startm
\m{\vdash}\m{(}\m{A}\m{=}\m{B}\m{\leftrightarrow}\m{\forall}\m{x}\m{(}\m{x}\m{
\in}\m{A}\m{\leftrightarrow}\m{x}\m{\in}\m{B}\m{)}\m{)}
\endm
% We need to reset the startprefix and contprefix.
\setbox\startprefix=\hbox{\tt \ \ df-cleq.1\ \$e\ }
\setbox\contprefix=\hbox{\tt \ \ \ \ \ \ \ \ \ \ \ \ \ \ \ }
\begin{mmraw}%
|- ( A. x ( x e. y <-> x e. z ) -> y = z ) \$.
\end{mmraw}
\setbox\startprefix=\hbox{\tt \ \ df-cleq\ \$a\ }
\setbox\contprefix=\hbox{\tt \ \ \ \ \ \ \ \ \ \ \ \ \ }
\begin{mmraw}%
|- ( A = B <-> A. x ( x e. A <-> x e. B ) ) \$.
\end{mmraw}

\noindent Define the membership connective between classes\index{class
membership}.  Theorem 6.3 of Quine, p.~41, which we adopt as a definition.
Note that it extends the use of the existing membership symbol, but unlike
{\tt df-cleq} it does not extend the set of valid wffs of logic when the class
metavariables are replaced with set variables.\label{dfclel}\label{df-clel}

\vskip 0.5ex
\setbox\startprefix=\hbox{\tt \ \ df-clel\ \$a\ }
\setbox\contprefix=\hbox{\tt \ \ \ \ \ \ \ \ \ \ \ \ \ }
\startm
\m{\vdash}\m{(}\m{A}\m{\in}\m{B}\m{\leftrightarrow}\m{\exists}\m{x}\m{(}\m{x}
\m{=}\m{A}\m{\wedge}\m{x}\m{\in}\m{B}\m{)}\m{)}
\endm
\begin{mmraw}%
|- ( A e. B <-> E. x ( x = A \TAND x e. B ) ) \$.?
\end{mmraw}

\noindent Define inequality.

\vskip 0.5ex
\setbox\startprefix=\hbox{\tt \ \ df-ne\ \$a\ }
\setbox\contprefix=\hbox{\tt \ \ \ \ \ \ \ \ \ \ \ }
\startm
\m{\vdash}\m{(}\m{A}\m{\ne}\m{B}\m{\leftrightarrow}\m{\lnot}\m{A}\m{=}\m{B}%
\m{)}
\endm
\begin{mmraw}%
|- ( A =/= B <-> -. A = B ) \$.
\end{mmraw}

\noindent Define restricted universal quantification.\index{universal
quantifier ($\forall$)!restricted}  Enderton, p.~22.

\vskip 0.5ex
\setbox\startprefix=\hbox{\tt \ \ df-ral\ \$a\ }
\setbox\contprefix=\hbox{\tt \ \ \ \ \ \ \ \ \ \ \ \ }
\startm
\m{\vdash}\m{(}\m{\forall}\m{x}\m{\in}\m{A}\m{\varphi}\m{\leftrightarrow}\m{%
\forall}\m{x}\m{(}\m{x}\m{\in}\m{A}\m{\rightarrow}\m{\varphi}\m{)}\m{)}
\endm
\begin{mmraw}%
|- ( A. x e. A ph <-> A. x ( x e. A -> ph ) ) \$.
\end{mmraw}

\noindent Define restricted existential quantification.\index{existential
quantifier ($\exists$)!restricted}  Enderton, p.~22.

\vskip 0.5ex
\setbox\startprefix=\hbox{\tt \ \ df-rex\ \$a\ }
\setbox\contprefix=\hbox{\tt \ \ \ \ \ \ \ \ \ \ \ \ }
\startm
\m{\vdash}\m{(}\m{\exists}\m{x}\m{\in}\m{A}\m{\varphi}\m{\leftrightarrow}\m{%
\exists}\m{x}\m{(}\m{x}\m{\in}\m{A}\m{\wedge}\m{\varphi}\m{)}\m{)}
\endm
\begin{mmraw}%
|- ( E. x e. A ph <-> E. x ( x e. A \TAND ph ) ) \$.
\end{mmraw}

\noindent Define the universal class\index{universal class ($V$)}.  Definition
5.20, p.~21, of Takeuti and Zaring.\label{df-v}

\vskip 0.5ex
\setbox\startprefix=\hbox{\tt \ \ df-v\ \$a\ }
\setbox\contprefix=\hbox{\tt \ \ \ \ \ \ \ \ \ \ }
\startm
\m{\vdash}\m{{\rm V}}\m{=}\m{\{}\m{x}\m{|}\m{x}\m{=}\m{x}\m{\}}
\endm
\begin{mmraw}%
|- {\char`\_}V = \{ x | x = x \} \$.
\end{mmraw}

\noindent Define the subclass\index{subclass}\index{subset} relationship
between two classes (called the subset relation if the classes are sets i.e.\
are not proper).  Definition 5.9 of Takeuti and Zaring, p.~17.\label{df-ss}

\vskip 0.5ex
\setbox\startprefix=\hbox{\tt \ \ df-ss\ \$a\ }
\setbox\contprefix=\hbox{\tt \ \ \ \ \ \ \ \ \ \ \ }
\startm
\m{\vdash}\m{(}\m{A}\m{\subseteq}\m{B}\m{\leftrightarrow}\m{\forall}\m{x}\m{(}
\m{x}\m{\in}\m{A}\m{\rightarrow}\m{x}\m{\in}\m{B}\m{)}\m{)}
\endm
\begin{mmraw}%
|- ( A C\_ B <-> A. x ( x e. A -> x e. B ) ) \$.
\end{mmraw}

\noindent Define the union\index{union} of two classes.  Definition 5.6 of Takeuti and Zaring,
p.~16.\label{df-un}

\vskip 0.5ex
\setbox\startprefix=\hbox{\tt \ \ df-un\ \$a\ }
\setbox\contprefix=\hbox{\tt \ \ \ \ \ \ \ \ \ \ \ }
\startm
\m{\vdash}\m{(}\m{A}\m{\cup}\m{B}\m{)}\m{=}\m{\{}\m{x}\m{|}\m{(}\m{x}\m{\in}
\m{A}\m{\vee}\m{x}\m{\in}\m{B}\m{)}\m{\}}
\endm
\begin{mmraw}%
( A u. B ) = \{ x | ( x e. A \TOR x e. B ) \} \$.
\end{mmraw}

\noindent Define the intersection\index{intersection} of two classes.  Definition 5.6 of
Takeuti and Zaring, p.~16.\label{df-in}

\vskip 0.5ex
\setbox\startprefix=\hbox{\tt \ \ df-in\ \$a\ }
\setbox\contprefix=\hbox{\tt \ \ \ \ \ \ \ \ \ \ \ }
\startm
\m{\vdash}\m{(}\m{A}\m{\cap}\m{B}\m{)}\m{=}\m{\{}\m{x}\m{|}\m{(}\m{x}\m{\in}
\m{A}\m{\wedge}\m{x}\m{\in}\m{B}\m{)}\m{\}}
\endm
% Caret ^ requires special treatment
\begin{mmraw}%
|- ( A i\^{}i B ) = \{ x | ( x e. A \TAND x e. B ) \} \$.
\end{mmraw}

\noindent Define class difference\index{class difference}\index{set difference}.
Definition 5.12 of Takeuti and Zaring, p.~20.  Several notations are used in
the literature; we chose the $\setminus$ convention instead of a minus sign to
reserve the latter for later use in, e.g., arithmetic.\label{df-dif}

\vskip 0.5ex
\setbox\startprefix=\hbox{\tt \ \ df-dif\ \$a\ }
\setbox\contprefix=\hbox{\tt \ \ \ \ \ \ \ \ \ \ \ \ }
\startm
\m{\vdash}\m{(}\m{A}\m{\setminus}\m{B}\m{)}\m{=}\m{\{}\m{x}\m{|}\m{(}\m{x}\m{
\in}\m{A}\m{\wedge}\m{\lnot}\m{x}\m{\in}\m{B}\m{)}\m{\}}
\endm
\begin{mmraw}%
( A \SLASH B ) = \{ x | ( x e. A \TAND -. x e. B ) \} \$.
\end{mmraw}

\noindent Define the empty or null set\index{empty set}\index{null set}.
Compare  Definition 5.14 of Takeuti and Zaring, p.~20.\label{df-nul}

\vskip 0.5ex
\setbox\startprefix=\hbox{\tt \ \ df-nul\ \$a\ }
\setbox\contprefix=\hbox{\tt \ \ \ \ \ \ \ \ \ \ }
\startm
\m{\vdash}\m{\varnothing}\m{=}\m{(}\m{{\rm V}}\m{\setminus}\m{{\rm V}}\m{)}
\endm
\begin{mmraw}%
|- (/) = ( {\char`\_}V \SLASH {\char`\_}V ) \$.
\end{mmraw}

\noindent Define power class\index{power set}\index{power class}.  Definition 5.10 of
Takeuti and Zaring, p.~17, but we also let it apply to proper classes.  (Note
that \verb$~P$ is the symbol for calligraphic P, the tilde
suggesting ``curly;'' see Appendix~\ref{ASCII}.)\label{df-pw}

\vskip 0.5ex
\setbox\startprefix=\hbox{\tt \ \ df-pw\ \$a\ }
\setbox\contprefix=\hbox{\tt \ \ \ \ \ \ \ \ \ \ \ }
\startm
\m{\vdash}\m{{\cal P}}\m{A}\m{=}\m{\{}\m{x}\m{|}\m{x}\m{\subseteq}\m{A}\m{\}}
\endm
% Special incantation required to put ~ into the text
\begin{mmraw}%
|- \char`\~P~A = \{ x | x C\_ A \} \$.
\end{mmraw}

\noindent Define the singleton of a class\index{singleton}.  Definition 7.1 of
Quine, p.~48.  It is well-defined for proper classes, although
it is not very meaningful in this case, where it evaluates to the empty
set.

\vskip 0.5ex
\setbox\startprefix=\hbox{\tt \ \ df-sn\ \$a\ }
\setbox\contprefix=\hbox{\tt \ \ \ \ \ \ \ \ \ \ \ }
\startm
\m{\vdash}\m{\{}\m{A}\m{\}}\m{=}\m{\{}\m{x}\m{|}\m{x}\m{=}\m{A}\m{\}}
\endm
\begin{mmraw}%
|- \{ A \} = \{ x | x = A \} \$.
\end{mmraw}%

\noindent Define an unordered pair of classes\index{unordered pair}\index{pair}.  Definition
7.1 of Quine, p.~48.

\vskip 0.5ex
\setbox\startprefix=\hbox{\tt \ \ df-pr\ \$a\ }
\setbox\contprefix=\hbox{\tt \ \ \ \ \ \ \ \ \ \ \ }
\startm
\m{\vdash}\m{\{}\m{A}\m{,}\m{B}\m{\}}\m{=}\m{(}\m{\{}\m{A}\m{\}}\m{\cup}\m{\{}
\m{B}\m{\}}\m{)}
\endm
\begin{mmraw}%
|- \{ A , B \} = ( \{ A \} u. \{ B \} ) \$.
\end{mmraw}

\noindent Define an unordered triple of classes\index{unordered triple}.  Definition of
Enderton, p.~19.

\vskip 0.5ex
\setbox\startprefix=\hbox{\tt \ \ df-tp\ \$a\ }
\setbox\contprefix=\hbox{\tt \ \ \ \ \ \ \ \ \ \ \ }
\startm
\m{\vdash}\m{\{}\m{A}\m{,}\m{B}\m{,}\m{C}\m{\}}\m{=}\m{(}\m{\{}\m{A}\m{,}\m{B}
\m{\}}\m{\cup}\m{\{}\m{C}\m{\}}\m{)}
\endm
\begin{mmraw}%
|- \{ A , B , C \} = ( \{ A , B \} u. \{ C \} ) \$.
\end{mmraw}%

\noindent Kuratowski's\index{Kuratowski, Kazimierz} ordered pair\index{ordered
pair} definition.  Definition 9.1 of Quine, p.~58. For proper classes it is
not meaningful but is well-defined for convenience.  (Note that \verb$<.$
stands for $\langle$ whereas \verb$<$ stands for $<$, and similarly for
\verb$>.$\,.)\label{df-op}

\vskip 0.5ex
\setbox\startprefix=\hbox{\tt \ \ df-op\ \$a\ }
\setbox\contprefix=\hbox{\tt \ \ \ \ \ \ \ \ \ \ \ }
\startm
\m{\vdash}\m{\langle}\m{A}\m{,}\m{B}\m{\rangle}\m{=}\m{\{}\m{\{}\m{A}\m{\}}
\m{,}\m{\{}\m{A}\m{,}\m{B}\m{\}}\m{\}}
\endm
\begin{mmraw}%
|- <. A , B >. = \{ \{ A \} , \{ A , B \} \} \$.
\end{mmraw}

\noindent Define the union of a class\index{union}.  Definition 5.5, p.~16,
of Takeuti and Zaring.

\vskip 0.5ex
\setbox\startprefix=\hbox{\tt \ \ df-uni\ \$a\ }
\setbox\contprefix=\hbox{\tt \ \ \ \ \ \ \ \ \ \ \ \ }
\startm
\m{\vdash}\m{\bigcup}\m{A}\m{=}\m{\{}\m{x}\m{|}\m{\exists}\m{y}\m{(}\m{x}\m{
\in}\m{y}\m{\wedge}\m{y}\m{\in}\m{A}\m{)}\m{\}}
\endm
\begin{mmraw}%
|- U. A = \{ x | E. y ( x e. y \TAND y e. A ) \} \$.
\end{mmraw}

\noindent Define the intersection\index{intersection} of a class.  Definition 7.35,
p.~44, of Takeuti and Zaring.

\vskip 0.5ex
\setbox\startprefix=\hbox{\tt \ \ df-int\ \$a\ }
\setbox\contprefix=\hbox{\tt \ \ \ \ \ \ \ \ \ \ \ \ }
\startm
\m{\vdash}\m{\bigcap}\m{A}\m{=}\m{\{}\m{x}\m{|}\m{\forall}\m{y}\m{(}\m{y}\m{
\in}\m{A}\m{\rightarrow}\m{x}\m{\in}\m{y}\m{)}\m{\}}
\endm
\begin{mmraw}%
|- |\^{}| A = \{ x | A. y ( y e. A -> x e. y ) \} \$.
\end{mmraw}

\noindent Define a transitive class\index{transitive class}\index{transitive
set}.  This should not be confused with a transitive relation, which is a different
concept.  Definition from p.~71 of Enderton, extended to classes.

\vskip 0.5ex
\setbox\startprefix=\hbox{\tt \ \ df-tr\ \$a\ }
\setbox\contprefix=\hbox{\tt \ \ \ \ \ \ \ \ \ \ \ }
\startm
\m{\vdash}\m{(}\m{\mbox{\rm Tr}}\m{A}\m{\leftrightarrow}\m{\bigcup}\m{A}\m{
\subseteq}\m{A}\m{)}
\endm
\begin{mmraw}%
|- ( Tr A <-> U. A C\_ A ) \$.
\end{mmraw}

\noindent Define a notation for a general binary relation\index{binary
relation}.  Definition 6.18, p.~29, of Takeuti and Zaring, generalized to
arbitrary classes.  This definition is well-defined, although not very
meaningful, when classes $A$ and/or $B$ are proper.\label{dfbr}  The lack of
parentheses (or any other connective) creates no ambiguity since we are defining
an atomic wff.

\vskip 0.5ex
\setbox\startprefix=\hbox{\tt \ \ df-br\ \$a\ }
\setbox\contprefix=\hbox{\tt \ \ \ \ \ \ \ \ \ \ \ }
\startm
\m{\vdash}\m{(}\m{A}\m{\,R}\m{\,B}\m{\leftrightarrow}\m{\langle}\m{A}\m{,}\m{B}
\m{\rangle}\m{\in}\m{R}\m{)}
\endm
\begin{mmraw}%
|- ( A R B <-> <. A , B >. e. R ) \$.
\end{mmraw}

\noindent Define an abstraction class of ordered pairs\index{abstraction
class!of ordered
pairs}.  A special case of Definition 4.16, p.~14, of Takeuti and Zaring.
Note that $ z $ must be distinct from $ x $ and $ y $,
and $ z $ must not occur in $\varphi$, but $ x $ and $ y $ may be
identical and may appear in $\varphi$.

\vskip 0.5ex
\setbox\startprefix=\hbox{\tt \ \ df-opab\ \$a\ }
\setbox\contprefix=\hbox{\tt \ \ \ \ \ \ \ \ \ \ \ \ \ }
\startm
\m{\vdash}\m{\{}\m{\langle}\m{x}\m{,}\m{y}\m{\rangle}\m{|}\m{\varphi}\m{\}}\m{=}
\m{\{}\m{z}\m{|}\m{\exists}\m{x}\m{\exists}\m{y}\m{(}\m{z}\m{=}\m{\langle}\m{x}
\m{,}\m{y}\m{\rangle}\m{\wedge}\m{\varphi}\m{)}\m{\}}
\endm

\begin{mmraw}%
|- \{ <. x , y >. | ph \} = \{ z | E. x E. y ( z =
<. x , y >. /\ ph ) \} \$.
\end{mmraw}

\noindent Define the epsilon relation\index{epsilon relation}.  Similar to Definition
6.22, p.~30, of Takeuti and Zaring.

\vskip 0.5ex
\setbox\startprefix=\hbox{\tt \ \ df-eprel\ \$a\ }
\setbox\contprefix=\hbox{\tt \ \ \ \ \ \ \ \ \ \ \ \ \ \ }
\startm
\m{\vdash}\m{{\rm E}}\m{=}\m{\{}\m{\langle}\m{x}\m{,}\m{y}\m{\rangle}\m{|}\m{x}\m{
\in}\m{y}\m{\}}
\endm
\begin{mmraw}%
|- \_E = \{ <. x , y >. | x e. y \} \$.
\end{mmraw}

\noindent Define a founded relation\index{founded relation}.  $R$ is a founded
relation on $A$ iff\index{iff} (if and only if) each nonempty subset of $A$
has an ``$R$-minimal element.''  Similar to Definition 6.21, p.~30, of
Takeuti and Zaring.

\vskip 0.5ex
\setbox\startprefix=\hbox{\tt \ \ df-fr\ \$a\ }
\setbox\contprefix=\hbox{\tt \ \ \ \ \ \ \ \ \ \ \ }
\startm
\m{\vdash}\m{(}\m{R}\m{\,\mbox{\rm Fr}}\m{\,A}\m{\leftrightarrow}\m{\forall}\m{x}
\m{(}\m{(}\m{x}\m{\subseteq}\m{A}\m{\wedge}\m{\lnot}\m{x}\m{=}\m{\varnothing}
\m{)}\m{\rightarrow}\m{\exists}\m{y}\m{(}\m{y}\m{\in}\m{x}\m{\wedge}\m{(}\m{x}
\m{\cap}\m{\{}\m{z}\m{|}\m{z}\m{\,R}\m{\,y}\m{\}}\m{)}\m{=}\m{\varnothing}\m{)}
\m{)}\m{)}
\endm
\begin{mmraw}%
|- ( R Fr A <-> A. x ( ( x C\_ A \TAND -. x = (/) ) ->
E. y ( y e. x \TAND ( x i\^{}i \{ z | z R y \} ) = (/) ) ) ) \$.
\end{mmraw}

\noindent Define a well-ordering\index{well-ordering}.  $R$ is a well-ordering of $A$ iff
it is founded on $A$ and the elements of $A$ are pairwise $R$-comparable.
Similar to Definition 6.24(2), p.~30, of Takeuti and Zaring.

\vskip 0.5ex
\setbox\startprefix=\hbox{\tt \ \ df-we\ \$a\ }
\setbox\contprefix=\hbox{\tt \ \ \ \ \ \ \ \ \ \ \ }
\startm
\m{\vdash}\m{(}\m{R}\m{\,\mbox{\rm We}}\m{\,A}\m{\leftrightarrow}\m{(}\m{R}\m{\,
\mbox{\rm Fr}}\m{\,A}\m{\wedge}\m{\forall}\m{x}\m{\forall}\m{y}\m{(}\m{(}\m{x}\m{
\in}\m{A}\m{\wedge}\m{y}\m{\in}\m{A}\m{)}\m{\rightarrow}\m{(}\m{x}\m{\,R}\m{\,y}
\m{\vee}\m{x}\m{=}\m{y}\m{\vee}\m{y}\m{\,R}\m{\,x}\m{)}\m{)}\m{)}\m{)}
\endm
\begin{mmraw}%
( R We A <-> ( R Fr A \TAND A. x A. y ( ( x e.
A \TAND y e. A ) -> ( x R y \TOR x = y \TOR y R x ) ) ) ) \$.
\end{mmraw}

\noindent Define the ordinal predicate\index{ordinal predicate}, which is true for a class
that is transitive and is well-ordered by the epsilon relation.  Similar to
definition on p.~468, Bell and Machover.

\vskip 0.5ex
\setbox\startprefix=\hbox{\tt \ \ df-ord\ \$a\ }
\setbox\contprefix=\hbox{\tt \ \ \ \ \ \ \ \ \ \ \ \ }
\startm
\m{\vdash}\m{(}\m{\mbox{\rm Ord}}\m{\,A}\m{\leftrightarrow}\m{(}
\m{\mbox{\rm Tr}}\m{\,A}\m{\wedge}\m{E}\m{\,\mbox{\rm We}}\m{\,A}\m{)}\m{)}
\endm
\begin{mmraw}%
|- ( Ord A <-> ( Tr A \TAND E We A ) ) \$.
\end{mmraw}

\noindent Define the class of all ordinal numbers\index{ordinal number}.  An ordinal number is
a set that satisfies the ordinal predicate.  Definition 7.11 of Takeuti and
Zaring, p.~38.

\vskip 0.5ex
\setbox\startprefix=\hbox{\tt \ \ df-on\ \$a\ }
\setbox\contprefix=\hbox{\tt \ \ \ \ \ \ \ \ \ \ \ }
\startm
\m{\vdash}\m{\,\mbox{\rm On}}\m{=}\m{\{}\m{x}\m{|}\m{\mbox{\rm Ord}}\m{\,x}
\m{\}}
\endm
\begin{mmraw}%
|- On = \{ x | Ord x \} \$.
\end{mmraw}

\noindent Define the limit ordinal predicate\index{limit ordinal}, which is true for a
non-empty ordinal that is not a successor (i.e.\ that is the union of itself).
Compare Bell and Machover, p.~471 and Exercise (1), p.~42 of Takeuti and
Zaring.

\vskip 0.5ex
\setbox\startprefix=\hbox{\tt \ \ df-lim\ \$a\ }
\setbox\contprefix=\hbox{\tt \ \ \ \ \ \ \ \ \ \ \ \ }
\startm
\m{\vdash}\m{(}\m{\mbox{\rm Lim}}\m{\,A}\m{\leftrightarrow}\m{(}\m{\mbox{
\rm Ord}}\m{\,A}\m{\wedge}\m{\lnot}\m{A}\m{=}\m{\varnothing}\m{\wedge}\m{A}
\m{=}\m{\bigcup}\m{A}\m{)}\m{)}
\endm
\begin{mmraw}%
|- ( Lim A <-> ( Ord A \TAND -. A = (/) \TAND A = U. A ) ) \$.
\end{mmraw}

\noindent Define the successor\index{successor} of a class.  Definition 7.22 of Takeuti
and Zaring, p.~41.  Our definition is a generalization to classes, although it
is meaningless when classes are proper.

\vskip 0.5ex
\setbox\startprefix=\hbox{\tt \ \ df-suc\ \$a\ }
\setbox\contprefix=\hbox{\tt \ \ \ \ \ \ \ \ \ \ \ \ }
\startm
\m{\vdash}\m{\,\mbox{\rm suc}}\m{\,A}\m{=}\m{(}\m{A}\m{\cup}\m{\{}\m{A}\m{\}}
\m{)}
\endm
\begin{mmraw}%
|- suc A = ( A u. \{ A \} ) \$.
\end{mmraw}

\noindent Define the class of natural numbers\index{natural number}\index{omega
($\omega$)}.  Compare Bell and Machover, p.~471.\label{dfom}

\vskip 0.5ex
\setbox\startprefix=\hbox{\tt \ \ df-om\ \$a\ }
\setbox\contprefix=\hbox{\tt \ \ \ \ \ \ \ \ \ \ \ }
\startm
\m{\vdash}\m{\omega}\m{=}\m{\{}\m{x}\m{|}\m{(}\m{\mbox{\rm Ord}}\m{\,x}\m{
\wedge}\m{\forall}\m{y}\m{(}\m{\mbox{\rm Lim}}\m{\,y}\m{\rightarrow}\m{x}\m{
\in}\m{y}\m{)}\m{)}\m{\}}
\endm
\begin{mmraw}%
|- om = \{ x | ( Ord x \TAND A. y ( Lim y -> x e. y ) ) \} \$.
\end{mmraw}

\noindent Define the Cartesian product (also called the
cross product)\index{Cartesian product}\index{cross product}
of two classes.  Definition 9.11 of Quine, p.~64.

\vskip 0.5ex
\setbox\startprefix=\hbox{\tt \ \ df-xp\ \$a\ }
\setbox\contprefix=\hbox{\tt \ \ \ \ \ \ \ \ \ \ \ }
\startm
\m{\vdash}\m{(}\m{A}\m{\times}\m{B}\m{)}\m{=}\m{\{}\m{\langle}\m{x}\m{,}\m{y}
\m{\rangle}\m{|}\m{(}\m{x}\m{\in}\m{A}\m{\wedge}\m{y}\m{\in}\m{B}\m{)}\m{\}}
\endm
\begin{mmraw}%
|- ( A X. B ) = \{ <. x , y >. | ( x e. A \TAND y e. B) \} \$.
\end{mmraw}

\noindent Define a relation\index{relation}.  Definition 6.4(1) of Takeuti and
Zaring, p.~23.

\vskip 0.5ex
\setbox\startprefix=\hbox{\tt \ \ df-rel\ \$a\ }
\setbox\contprefix=\hbox{\tt \ \ \ \ \ \ \ \ \ \ \ \ }
\startm
\m{\vdash}\m{(}\m{\mbox{\rm Rel}}\m{\,A}\m{\leftrightarrow}\m{A}\m{\subseteq}
\m{(}\m{{\rm V}}\m{\times}\m{{\rm V}}\m{)}\m{)}
\endm
\begin{mmraw}%
|- ( Rel A <-> A C\_ ( {\char`\_}V X. {\char`\_}V ) ) \$.
\end{mmraw}

\noindent Define the domain\index{domain} of a class.  Definition 6.5(1) of
Takeuti and Zaring, p.~24.

\vskip 0.5ex
\setbox\startprefix=\hbox{\tt \ \ df-dm\ \$a\ }
\setbox\contprefix=\hbox{\tt \ \ \ \ \ \ \ \ \ \ \ }
\startm
\m{\vdash}\m{\,\mbox{\rm dom}}\m{A}\m{=}\m{\{}\m{x}\m{|}\m{\exists}\m{y}\m{
\langle}\m{x}\m{,}\m{y}\m{\rangle}\m{\in}\m{A}\m{\}}
\endm
\begin{mmraw}%
|- dom A = \{ x | E. y <. x , y >. e. A \} \$.
\end{mmraw}

\noindent Define the range\index{range} of a class.  Definition 6.5(2) of
Takeuti and Zaring, p.~24.

\vskip 0.5ex
\setbox\startprefix=\hbox{\tt \ \ df-rn\ \$a\ }
\setbox\contprefix=\hbox{\tt \ \ \ \ \ \ \ \ \ \ \ }
\startm
\m{\vdash}\m{\,\mbox{\rm ran}}\m{A}\m{=}\m{\{}\m{y}\m{|}\m{\exists}\m{x}\m{
\langle}\m{x}\m{,}\m{y}\m{\rangle}\m{\in}\m{A}\m{\}}
\endm
\begin{mmraw}%
|- ran A = \{ y | E. x <. x , y >. e. A \} \$.
\end{mmraw}

\noindent Define the restriction\index{restriction} of a class.  Definition
6.6(1) of Takeuti and Zaring, p.~24.

\vskip 0.5ex
\setbox\startprefix=\hbox{\tt \ \ df-res\ \$a\ }
\setbox\contprefix=\hbox{\tt \ \ \ \ \ \ \ \ \ \ \ \ }
\startm
\m{\vdash}\m{(}\m{A}\m{\restriction}\m{B}\m{)}\m{=}\m{(}\m{A}\m{\cap}\m{(}\m{B}
\m{\times}\m{{\rm V}}\m{)}\m{)}
\endm
\begin{mmraw}%
|- ( A |` B ) = ( A i\^{}i ( B X. {\char`\_}V ) ) \$.
\end{mmraw}

\noindent Define the image\index{image} of a class.  Definition 6.6(2) of
Takeuti and Zaring, p.~24.

\vskip 0.5ex
\setbox\startprefix=\hbox{\tt \ \ df-ima\ \$a\ }
\setbox\contprefix=\hbox{\tt \ \ \ \ \ \ \ \ \ \ \ \ }
\startm
\m{\vdash}\m{(}\m{A}\m{``}\m{B}\m{)}\m{=}\m{\,\mbox{\rm ran}}\m{\,(}\m{A}\m{
\restriction}\m{B}\m{)}
\endm
\begin{mmraw}%
|- ( A " B ) = ran ( A |` B ) \$.
\end{mmraw}

\noindent Define the composition\index{composition} of two classes.  Definition 6.6(3) of
Takeuti and Zaring, p.~24.

\vskip 0.5ex
\setbox\startprefix=\hbox{\tt \ \ df-co\ \$a\ }
\setbox\contprefix=\hbox{\tt \ \ \ \ \ \ \ \ \ \ \ \ }
\startm
\m{\vdash}\m{(}\m{A}\m{\circ}\m{B}\m{)}\m{=}\m{\{}\m{\langle}\m{x}\m{,}\m{y}\m{
\rangle}\m{|}\m{\exists}\m{z}\m{(}\m{\langle}\m{x}\m{,}\m{z}\m{\rangle}\m{\in}
\m{B}\m{\wedge}\m{\langle}\m{z}\m{,}\m{y}\m{\rangle}\m{\in}\m{A}\m{)}\m{\}}
\endm
\begin{mmraw}%
|- ( A o. B ) = \{ <. x , y >. | E. z ( <. x , z
>. e. B \TAND <. z , y >. e. A ) \} \$.
\end{mmraw}

\noindent Define a function\index{function}.  Definition 6.4(4) of Takeuti and
Zaring, p.~24.

\vskip 0.5ex
\setbox\startprefix=\hbox{\tt \ \ df-fun\ \$a\ }
\setbox\contprefix=\hbox{\tt \ \ \ \ \ \ \ \ \ \ \ \ }
\startm
\m{\vdash}\m{(}\m{\mbox{\rm Fun}}\m{\,A}\m{\leftrightarrow}\m{(}
\m{\mbox{\rm Rel}}\m{\,A}\m{\wedge}
\m{\forall}\m{x}\m{\exists}\m{z}\m{\forall}\m{y}\m{(}
\m{\langle}\m{x}\m{,}\m{y}\m{\rangle}\m{\in}\m{A}\m{\rightarrow}\m{y}\m{=}\m{z}
\m{)}\m{)}\m{)}
\endm
\begin{mmraw}%
|- ( Fun A <-> ( Rel A /\ A. x E. z A. y ( <. x
   , y >. e. A -> y = z ) ) ) \$.
\end{mmraw}

\noindent Define a function with domain.  Definition 6.15(1) of Takeuti and
Zaring, p.~27.

\vskip 0.5ex
\setbox\startprefix=\hbox{\tt \ \ df-fn\ \$a\ }
\setbox\contprefix=\hbox{\tt \ \ \ \ \ \ \ \ \ \ \ }
\startm
\m{\vdash}\m{(}\m{A}\m{\,\mbox{\rm Fn}}\m{\,B}\m{\leftrightarrow}\m{(}
\m{\mbox{\rm Fun}}\m{\,A}\m{\wedge}\m{\mbox{\rm dom}}\m{\,A}\m{=}\m{B}\m{)}
\m{)}
\endm
\begin{mmraw}%
|- ( A Fn B <-> ( Fun A \TAND dom A = B ) ) \$.
\end{mmraw}

\noindent Define a function with domain and co-domain.  Definition 6.15(3)
of Takeuti and Zaring, p.~27.

\vskip 0.5ex
\setbox\startprefix=\hbox{\tt \ \ df-f\ \$a\ }
\setbox\contprefix=\hbox{\tt \ \ \ \ \ \ \ \ \ \ }
\startm
\m{\vdash}\m{(}\m{F}\m{:}\m{A}\m{\longrightarrow}\m{B}\m{
\leftrightarrow}\m{(}\m{F}\m{\,\mbox{\rm Fn}}\m{\,A}\m{\wedge}\m{
\mbox{\rm ran}}\m{\,F}\m{\subseteq}\m{B}\m{)}\m{)}
\endm
\begin{mmraw}%
|- ( F : A --> B <-> ( F Fn A \TAND ran F C\_ B ) ) \$.
\end{mmraw}

\noindent Define a one-to-one function\index{one-to-one function}.  Compare
Definition 6.15(5) of Takeuti and Zaring, p.~27.

\vskip 0.5ex
\setbox\startprefix=\hbox{\tt \ \ df-f1\ \$a\ }
\setbox\contprefix=\hbox{\tt \ \ \ \ \ \ \ \ \ \ \ }
\startm
\m{\vdash}\m{(}\m{F}\m{:}\m{A}\m{
\raisebox{.5ex}{${\textstyle{\:}_{\mbox{\footnotesize\rm
1\tt -\rm 1}}}\atop{\textstyle{
\longrightarrow}\atop{\textstyle{}^{\mbox{\footnotesize\rm {\ }}}}}$}
}\m{B}
\m{\leftrightarrow}\m{(}\m{F}\m{:}\m{A}\m{\longrightarrow}\m{B}
\m{\wedge}\m{\forall}\m{y}\m{\exists}\m{z}\m{\forall}\m{x}\m{(}\m{\langle}\m{x}
\m{,}\m{y}\m{\rangle}\m{\in}\m{F}\m{\rightarrow}\m{x}\m{=}\m{z}\m{)}\m{)}\m{)}
\endm
\begin{mmraw}%
|- ( F : A -1-1-> B <-> ( F : A --> B \TAND
   A. y E. z A. x ( <. x , y >. e. F -> x = z ) ) ) \$.
\end{mmraw}

\noindent Define an onto function\index{onto function}.  Definition 6.15(4) of Takeuti and
Zaring, p.~27.

\vskip 0.5ex
\setbox\startprefix=\hbox{\tt \ \ df-fo\ \$a\ }
\setbox\contprefix=\hbox{\tt \ \ \ \ \ \ \ \ \ \ \ }
\startm
\m{\vdash}\m{(}\m{F}\m{:}\m{A}\m{
\raisebox{.5ex}{${\textstyle{\:}_{\mbox{\footnotesize\rm
{\ }}}}\atop{\textstyle{
\longrightarrow}\atop{\textstyle{}^{\mbox{\footnotesize\rm onto}}}}$}
}\m{B}
\m{\leftrightarrow}\m{(}\m{F}\m{\,\mbox{\rm Fn}}\m{\,A}\m{\wedge}
\m{\mbox{\rm ran}}\m{\,F}\m{=}\m{B}\m{)}\m{)}
\endm
\begin{mmraw}%
|- ( F : A -onto-> B <-> ( F Fn A /\ ran F = B ) ) \$.
\end{mmraw}

\noindent Define a one-to-one, onto function.  Compare Definition 6.15(6) of
Takeuti and Zaring, p.~27.

\vskip 0.5ex
\setbox\startprefix=\hbox{\tt \ \ df-f1o\ \$a\ }
\setbox\contprefix=\hbox{\tt \ \ \ \ \ \ \ \ \ \ \ \ }
\startm
\m{\vdash}\m{(}\m{F}\m{:}\m{A}
\m{
\raisebox{.5ex}{${\textstyle{\:}_{\mbox{\footnotesize\rm
1\tt -\rm 1}}}\atop{\textstyle{
\longrightarrow}\atop{\textstyle{}^{\mbox{\footnotesize\rm onto}}}}$}
}
\m{B}
\m{\leftrightarrow}\m{(}\m{F}\m{:}\m{A}
\m{
\raisebox{.5ex}{${\textstyle{\:}_{\mbox{\footnotesize\rm
1\tt -\rm 1}}}\atop{\textstyle{
\longrightarrow}\atop{\textstyle{}^{\mbox{\footnotesize\rm {\ }}}}}$}
}
\m{B}\m{\wedge}\m{F}\m{:}\m{A}
\m{
\raisebox{.5ex}{${\textstyle{\:}_{\mbox{\footnotesize\rm
{\ }}}}\atop{\textstyle{
\longrightarrow}\atop{\textstyle{}^{\mbox{\footnotesize\rm onto}}}}$}
}
\m{B}\m{)}\m{)}
\endm
\begin{mmraw}%
|- ( F : A -1-1-onto-> B <-> ( F : A -1-1-> B? \TAND F : A -onto-> B ) ) \$.?
\end{mmraw}

\noindent Define the value of a function\index{function value}.  This
definition applies to any class and evaluates to the empty set when it is not
meaningful. Note that $ F`A$ means the same thing as the more familiar $ F(A)$
notation for a function's value at $A$.  The $ F`A$ notation is common in
formal set theory.\label{df-fv}

\vskip 0.5ex
\setbox\startprefix=\hbox{\tt \ \ df-fv\ \$a\ }
\setbox\contprefix=\hbox{\tt \ \ \ \ \ \ \ \ \ \ \ }
\startm
\m{\vdash}\m{(}\m{F}\m{`}\m{A}\m{)}\m{=}\m{\bigcup}\m{\{}\m{x}\m{|}\m{(}\m{F}%
\m{``}\m{\{}\m{A}\m{\}}\m{)}\m{=}\m{\{}\m{x}\m{\}}\m{\}}
\endm
\begin{mmraw}%
|- ( F ` A ) = U. \{ x | ( F " \{ A \} ) = \{ x \} \} \$.
\end{mmraw}

\noindent Define the result of an operation.\index{operation}  Here, $F$ is
     an operation on two
     values (such as $+$ for real numbers).   This is defined for proper
     classes $A$ and $B$ even though not meaningful in that case.  However,
     the definition can be meaningful when $F$ is a proper class.\label{dfopr}

\vskip 0.5ex
\setbox\startprefix=\hbox{\tt \ \ df-opr\ \$a\ }
\setbox\contprefix=\hbox{\tt \ \ \ \ \ \ \ \ \ \ \ \ }
\startm
\m{\vdash}\m{(}\m{A}\m{\,F}\m{\,B}\m{)}\m{=}\m{(}\m{F}\m{`}\m{\langle}\m{A}%
\m{,}\m{B}\m{\rangle}\m{)}
\endm
\begin{mmraw}%
|- ( A F B ) = ( F ` <. A , B >. ) \$.
\end{mmraw}

\section{Tricks of the Trade}\label{tricks}

In the \texttt{set.mm}\index{set theory database (\texttt{set.mm})} database our goal
was usually to conform to modern notation.  However in some cases the
relationship to standard textbook language may be obscured by several
unconventional devices we used to simplify the development and to take
advantage of the Metamath language.  In this section we will describe some
common conventions used in \texttt{set.mm}.

\begin{itemize}
\item
The turnstile\index{turnstile ({$\,\vdash$})} symbol, $\vdash$, meaning ``it
is provable that,'' is the first token of all assertions and hypotheses that
aren't syntax constructions.  This is a standard convention in logic.  (We
mentioned this earlier, but this symbol is bothersome to some people without a
logic background.  It has no deeper meaning but just provides us with a way to
distinguish syntax constructions from ordinary mathematical statements.)

\item
A hypothesis of the form

\vskip 1ex
\setbox\startprefix=\hbox{\tt \ \ \ \ \ \ \ \ \ \$e\ }
\setbox\contprefix=\hbox{\tt \ \ \ \ \ \ \ \ \ \ }
\startm
\m{\vdash}\m{(}\m{\varphi}\m{\rightarrow}\m{\forall}\m{x}\m{\varphi}\m{)}
\endm
\vskip 1ex

should be read ``assume variable $x$ is (effectively) not free in wff
$\varphi$.''\index{effectively not free}
Literally, this says ``assume it is provable that $\varphi \rightarrow \forall
x\, \varphi$.''  This device lets us avoid the complexities associated with
the standard treatment of free and bound variables.
%% Uncomment this when uncommenting section {formalspec} below
The footnote on p.~\pageref{effectivelybound} discusses this further.

\item
A statement of one of the forms

\vskip 1ex
\setbox\startprefix=\hbox{\tt \ \ \ \ \ \ \ \ \ \$a\ }
\setbox\contprefix=\hbox{\tt \ \ \ \ \ \ \ \ \ \ }
\startm
\m{\vdash}\m{(}\m{\lnot}\m{\forall}\m{x}\m{\,x}\m{=}\m{y}\m{\rightarrow}
\m{\ldots}\m{)}
\endm
\setbox\startprefix=\hbox{\tt \ \ \ \ \ \ \ \ \ \$p\ }
\setbox\contprefix=\hbox{\tt \ \ \ \ \ \ \ \ \ \ }
\startm
\m{\vdash}\m{(}\m{\lnot}\m{\forall}\m{x}\m{\,x}\m{=}\m{y}\m{\rightarrow}
\m{\ldots}\m{)}
\endm
\vskip 1ex

should be read ``if $x$ and $y$ are distinct variables, then...''  This
antecedent provides us with a technical device to avoid the need for the
\texttt{\$d} statement early in our development of predicate calculus,
permitting symbol manipulations to be as conceptually simple as those in
propositional calculus.  However, the \texttt{\$d} statement eventually
becomes a requirement, and after that this device is rarely used.

\item
The statement

\vskip 1ex
\setbox\startprefix=\hbox{\tt \ \ \ \ \ \ \ \ \ \$d\ }
\setbox\contprefix=\hbox{\tt \ \ \ \ \ \ \ \ \ \ }
\startm
\m{x}\m{\,y}
\endm
\vskip 1ex

should be read ``assume $x$ and $y$ are distinct variables.''

\item
The statement

\vskip 1ex
\setbox\startprefix=\hbox{\tt \ \ \ \ \ \ \ \ \ \$d\ }
\setbox\contprefix=\hbox{\tt \ \ \ \ \ \ \ \ \ \ }
\startm
\m{x}\m{\,\varphi}
\endm
\vskip 1ex

should be read ``assume $x$ does not occur in $\varphi$.''

\item
The statement

\vskip 1ex
\setbox\startprefix=\hbox{\tt \ \ \ \ \ \ \ \ \ \$d\ }
\setbox\contprefix=\hbox{\tt \ \ \ \ \ \ \ \ \ \ }
\startm
\m{x}\m{\,A}
\endm
\vskip 1ex

should be read ``assume variable $x$ does not occur in class $A$.''

\item
The restriction and hypothesis group

\vskip 1ex
\setbox\startprefix=\hbox{\tt \ \ \ \ \ \ \ \ \ \$d\ }
\setbox\contprefix=\hbox{\tt \ \ \ \ \ \ \ \ \ \ }
\startm
\m{x}\m{\,A}
\endm
\setbox\startprefix=\hbox{\tt \ \ \ \ \ \ \ \ \ \$d\ }
\setbox\contprefix=\hbox{\tt \ \ \ \ \ \ \ \ \ \ }
\startm
\m{x}\m{\,\psi}
\endm
\setbox\startprefix=\hbox{\tt \ \ \ \ \ \ \ \ \ \$e\ }
\setbox\contprefix=\hbox{\tt \ \ \ \ \ \ \ \ \ \ }
\startm
\m{\vdash}\m{(}\m{x}\m{=}\m{A}\m{\rightarrow}\m{(}\m{\varphi}\m{\leftrightarrow}
\m{\psi}\m{)}\m{)}
\endm
\vskip 1ex

is frequently used in place of explicit substitution, meaning ``assume
$\psi$ results from the proper substitution of $A$ for $x$ in
$\varphi$.''  Sometimes ``\texttt{\$e} $\vdash ( \psi \rightarrow
\forall x \, \psi )$'' is used instead of ``\texttt{\$d} $x\, \psi $,''
which requires only that $x$ be effectively not free in $\varphi$ but
not necessarily absent from it.  The use of implicit
substitution\index{substitution!implicit} is partly a matter of personal
style, although it may make proofs somewhat shorter than would be the
case with explicit substitution.

\item
The hypothesis


\vskip 1ex
\setbox\startprefix=\hbox{\tt \ \ \ \ \ \ \ \ \ \$e\ }
\setbox\contprefix=\hbox{\tt \ \ \ \ \ \ \ \ \ \ }
\startm
\m{\vdash}\m{A}\m{\in}\m{{\rm V}}
\endm
\vskip 1ex

should be read ``assume class $A$ is a set (i.e.\ exists).''
This is a convenient convention used by Quine.

\item
The restriction and hypothesis

\vskip 1ex
\setbox\startprefix=\hbox{\tt \ \ \ \ \ \ \ \ \ \$d\ }
\setbox\contprefix=\hbox{\tt \ \ \ \ \ \ \ \ \ \ }
\startm
\m{x}\m{\,y}
\endm
\setbox\startprefix=\hbox{\tt \ \ \ \ \ \ \ \ \ \$e\ }
\setbox\contprefix=\hbox{\tt \ \ \ \ \ \ \ \ \ \ }
\startm
\m{\vdash}\m{(}\m{y}\m{\in}\m{A}\m{\rightarrow}\m{\forall}\m{x}\m{\,y}
\m{\in}\m{A}\m{)}
\endm
\vskip 1ex

should be read ``assume variable $x$ is
(effectively) not free in class $A$.''

\end{itemize}

\section{A Theorem Sampler}\label{sometheorems}

In this section we list some of the more important theorems that are proved in
the \texttt{set.mm} database, and they illustrate the kinds of things that can be
done with Metamath.  While all of these facts are well-known results,
Metamath offers the advantage of easily allowing you to trace their
derivation back to axioms.  Our intent here is not to try to explain the
details or motivation; for this we refer you to the textbooks that are
mentioned in the descriptions.  (The \texttt{set.mm} file has bibliographic
references for the text references.)  Their proofs often embody important
concepts you may wish to explore with the Metamath program (see
Section~\ref{exploring}).  All the symbols that are used here are defined in
Section~\ref{hierarchy}.  For brevity we haven't included the \texttt{\$d}
restrictions or \texttt{\$f} hypotheses for these theorems; when you are
uncertain consult the \texttt{set.mm} database.

We start with \texttt{syl} (principle of the syllogism).
In \textit{Principia Mathematica}
Whitehead and Russell call this ``the principle of the
syllogism... because... the syllogism in Barbara is derived from them''
\cite[quote after Theorem *2.06 p.~101]{PM}.
Some authors call this law a ``hypothetical syllogism.''
As of 2019 \texttt{syl} is the most commonly referenced proven
assertion in the \texttt{set.mm} database.\footnote{
The Metamath program command \texttt{show usage}
shows the number of references.
On 2019-04-29 (commit 71cbbbdb387e) \texttt{syl} was directly referenced
10,819 times. The second most commonly referenced proven assertion
was \texttt{eqid}, which was directly referenced 7,738 times.
}

\vskip 2ex
\noindent Theorem syl (principle of the syllogism)\index{Syllogism}%
\index{\texttt{syl}}\label{syl}.

\vskip 0.5ex
\setbox\startprefix=\hbox{\tt \ \ syl.1\ \$e\ }
\setbox\contprefix=\hbox{\tt \ \ \ \ \ \ \ \ \ \ \ }
\startm
\m{\vdash}\m{(}\m{\varphi}\m{ \rightarrow }\m{\psi}\m{)}
\endm
\setbox\startprefix=\hbox{\tt \ \ syl.2\ \$e\ }
\setbox\contprefix=\hbox{\tt \ \ \ \ \ \ \ \ \ \ \ }
\startm
\m{\vdash}\m{(}\m{\psi}\m{ \rightarrow }\m{\chi}\m{)}
\endm
\setbox\startprefix=\hbox{\tt \ \ syl\ \$p\ }
\setbox\contprefix=\hbox{\tt \ \ \ \ \ \ \ \ \ }
\startm
\m{\vdash}\m{(}\m{\varphi}\m{ \rightarrow }\m{\chi}\m{)}
\endm
\vskip 2ex

The following theorem is not very deep but provides us with a notational device
that is frequently used.  It allows us to use the expression ``$A \in V$'' as
a compact way of saying that class $A$ exists, i.e.\ is a set.

\vskip 2ex
\noindent Two ways to say ``$A$ is a set'':  $A$ is a member of the universe
$V$ if and only if $A$ exists (i.e.\ there exists a set equal to $A$).
Theorem 6.9 of Quine, p. 43.

\vskip 0.5ex
\setbox\startprefix=\hbox{\tt \ \ isset\ \$p\ }
\setbox\contprefix=\hbox{\tt \ \ \ \ \ \ \ \ \ \ \ }
\startm
\m{\vdash}\m{(}\m{A}\m{\in}\m{{\rm V}}\m{\leftrightarrow}\m{\exists}\m{x}\m{\,x}\m{=}
\m{A}\m{)}
\endm
\vskip 1ex

Next we prove the axioms of standard ZF set theory that were missing from our
axiom system.  From our point of view they are theorems since they
can be derived from the other axioms.

\vskip 2ex
\noindent Axiom of Separation\index{Axiom of Separation}
(Aussonderung)\index{Aussonderung} proved from the other axioms of ZF set
theory.  Compare Exercise 4 of Takeuti and Zaring, p.~22.

\vskip 0.5ex
\setbox\startprefix=\hbox{\tt \ \ inex1.1\ \$e\ }
\setbox\contprefix=\hbox{\tt \ \ \ \ \ \ \ \ \ \ \ \ \ \ \ }
\startm
\m{\vdash}\m{A}\m{\in}\m{{\rm V}}
\endm
\setbox\startprefix=\hbox{\tt \ \ inex\ \$p\ }
\setbox\contprefix=\hbox{\tt \ \ \ \ \ \ \ \ \ \ \ \ \ }
\startm
\m{\vdash}\m{(}\m{A}\m{\cap}\m{B}\m{)}\m{\in}\m{{\rm V}}
\endm
\vskip 1ex

\noindent Axiom of the Null Set\index{Axiom of the Null Set} proved from the
other axioms of ZF set theory. Corollary 5.16 of Takeuti and Zaring, p.~20.

\vskip 0.5ex
\setbox\startprefix=\hbox{\tt \ \ 0ex\ \$p\ }
\setbox\contprefix=\hbox{\tt \ \ \ \ \ \ \ \ \ \ \ \ }
\startm
\m{\vdash}\m{\varnothing}\m{\in}\m{{\rm V}}
\endm
\vskip 1ex

\noindent The Axiom of Pairing\index{Axiom of Pairing} proved from the other
axioms of ZF set theory.  Theorem 7.13 of Quine, p.~51.
\vskip 0.5ex
\setbox\startprefix=\hbox{\tt \ \ prex\ \$p\ }
\setbox\contprefix=\hbox{\tt \ \ \ \ \ \ \ \ \ \ \ \ \ \ }
\startm
\m{\vdash}\m{\{}\m{A}\m{,}\m{B}\m{\}}\m{\in}\m{{\rm V}}
\endm
\vskip 2ex

Next we will list some famous or important theorems that are proved in
the \texttt{set.mm} database.  None of them except \texttt{omex}
require the Axiom of Infinity, as you can verify with the \texttt{show
trace{\char`\_}back} Metamath command.

\vskip 2ex
\noindent The resolution of Russell's paradox\index{Russell's paradox}.  There
exists no set containing the set of all sets which are not members of
themselves.  Proposition 4.14 of Takeuti and Zaring, p.~14.

\vskip 0.5ex
\setbox\startprefix=\hbox{\tt \ \ ru\ \$p\ }
\setbox\contprefix=\hbox{\tt \ \ \ \ \ \ \ \ }
\startm
\m{\vdash}\m{\lnot}\m{\exists}\m{x}\m{\,x}\m{=}\m{\{}\m{y}\m{|}\m{\lnot}\m{y}
\m{\in}\m{y}\m{\}}
\endm
\vskip 1ex

\noindent Cantor's theorem\index{Cantor's theorem}.  No set can be mapped onto
its power set.  Compare Theorem 6B(b) of Enderton, p.~132.

\vskip 0.5ex
\setbox\startprefix=\hbox{\tt \ \ canth.1\ \$e\ }
\setbox\contprefix=\hbox{\tt \ \ \ \ \ \ \ \ \ \ \ \ \ }
\startm
\m{\vdash}\m{A}\m{\in}\m{{\rm V}}
\endm
\setbox\startprefix=\hbox{\tt \ \ canth\ \$p\ }
\setbox\contprefix=\hbox{\tt \ \ \ \ \ \ \ \ \ \ \ }
\startm
\m{\vdash}\m{\lnot}\m{F}\m{:}\m{A}\m{\raisebox{.5ex}{${\textstyle{\:}_{
\mbox{\footnotesize\rm {\ }}}}\atop{\textstyle{\longrightarrow}\atop{
\textstyle{}^{\mbox{\footnotesize\rm onto}}}}$}}\m{{\cal P}}\m{A}
\endm
\vskip 1ex

\noindent The Burali-Forti paradox\index{Burali-Forti paradox}.  No set
contains all ordinal numbers. Enderton, p.~194.  (Burali-Forti was one person,
not two.)

\vskip 0.5ex
\setbox\startprefix=\hbox{\tt \ \ onprc\ \$p\ }
\setbox\contprefix=\hbox{\tt \ \ \ \ \ \ \ \ \ \ \ \ }
\startm
\m{\vdash}\m{\lnot}\m{\mbox{\rm On}}\m{\in}\m{{\rm V}}
\endm
\vskip 1ex

\noindent Peano's postulates\index{Peano's postulates} for arithmetic.
Proposition 7.30 of Takeuti and Zaring, pp.~42--43.  The objects being
described are the members of $\omega$ i.e.\ the natural numbers 0, 1,
2,\ldots.  The successor\index{successor} operation suc means ``plus
one.''  \texttt{peano1} says that 0 (which is defined as the empty set)
is a natural number.  \texttt{peano2} says that if $A$ is a natural
number, so is $A+1$.  \texttt{peano3} says that 0 is not the successor
of any natural number.  \texttt{peano4} says that two natural numbers
are equal if and only if their successors are equal.  \texttt{peano5} is
essentially the same as mathematical induction.

\vskip 1ex
\setbox\startprefix=\hbox{\tt \ \ peano1\ \$p\ }
\setbox\contprefix=\hbox{\tt \ \ \ \ \ \ \ \ \ \ \ \ }
\startm
\m{\vdash}\m{\varnothing}\m{\in}\m{\omega}
\endm
\vskip 1.5ex

\setbox\startprefix=\hbox{\tt \ \ peano2\ \$p\ }
\setbox\contprefix=\hbox{\tt \ \ \ \ \ \ \ \ \ \ \ \ }
\startm
\m{\vdash}\m{(}\m{A}\m{\in}\m{\omega}\m{\rightarrow}\m{{\rm suc}}\m{A}\m{\in}%
\m{\omega}\m{)}
\endm
\vskip 1.5ex

\setbox\startprefix=\hbox{\tt \ \ peano3\ \$p\ }
\setbox\contprefix=\hbox{\tt \ \ \ \ \ \ \ \ \ \ \ \ }
\startm
\m{\vdash}\m{(}\m{A}\m{\in}\m{\omega}\m{\rightarrow}\m{\lnot}\m{{\rm suc}}%
\m{A}\m{=}\m{\varnothing}\m{)}
\endm
\vskip 1.5ex

\setbox\startprefix=\hbox{\tt \ \ peano4\ \$p\ }
\setbox\contprefix=\hbox{\tt \ \ \ \ \ \ \ \ \ \ \ \ }
\startm
\m{\vdash}\m{(}\m{(}\m{A}\m{\in}\m{\omega}\m{\wedge}\m{B}\m{\in}\m{\omega}%
\m{)}\m{\rightarrow}\m{(}\m{{\rm suc}}\m{A}\m{=}\m{{\rm suc}}\m{B}\m{%
\leftrightarrow}\m{A}\m{=}\m{B}\m{)}\m{)}
\endm
\vskip 1.5ex

\setbox\startprefix=\hbox{\tt \ \ peano5\ \$p\ }
\setbox\contprefix=\hbox{\tt \ \ \ \ \ \ \ \ \ \ \ \ }
\startm
\m{\vdash}\m{(}\m{(}\m{\varnothing}\m{\in}\m{A}\m{\wedge}\m{\forall}\m{x}\m{%
\in}\m{\omega}\m{(}\m{x}\m{\in}\m{A}\m{\rightarrow}\m{{\rm suc}}\m{x}\m{\in}%
\m{A}\m{)}\m{)}\m{\rightarrow}\m{\omega}\m{\subseteq}\m{A}\m{)}
\endm
\vskip 1.5ex

\noindent Finite Induction (mathematical induction).\index{finite
induction}\index{mathematical induction} The first hypothesis is the
basis and the second is the induction hypothesis.  Theorem Schema 22 of
Suppes, p.~136.

\vskip 0.5ex
\setbox\startprefix=\hbox{\tt \ \ findes.1\ \$e\ }
\setbox\contprefix=\hbox{\tt \ \ \ \ \ \ \ \ \ \ \ \ \ \ }
\startm
\m{\vdash}\m{[}\m{\varnothing}\m{/}\m{x}\m{]}\m{\varphi}
\endm
\setbox\startprefix=\hbox{\tt \ \ findes.2\ \$e\ }
\setbox\contprefix=\hbox{\tt \ \ \ \ \ \ \ \ \ \ \ \ \ \ }
\startm
\m{\vdash}\m{(}\m{x}\m{\in}\m{\omega}\m{\rightarrow}\m{(}\m{\varphi}\m{%
\rightarrow}\m{[}\m{{\rm suc}}\m{x}\m{/}\m{x}\m{]}\m{\varphi}\m{)}\m{)}
\endm
\setbox\startprefix=\hbox{\tt \ \ findes\ \$p\ }
\setbox\contprefix=\hbox{\tt \ \ \ \ \ \ \ \ \ \ \ \ }
\startm
\m{\vdash}\m{(}\m{x}\m{\in}\m{\omega}\m{\rightarrow}\m{\varphi}\m{)}
\endm
\vskip 1ex

\noindent Transfinite Induction with explicit substitution.  The first
hypothesis is the basis, the second is the induction hypothesis for
successors, and the third is the induction hypothesis for limit
ordinals.  Theorem Schema 4 of Suppes, p. 197.

\vskip 0.5ex
\setbox\startprefix=\hbox{\tt \ \ tfindes.1\ \$e\ }
\setbox\contprefix=\hbox{\tt \ \ \ \ \ \ \ \ \ \ \ \ \ \ \ }
\startm
\m{\vdash}\m{[}\m{\varnothing}\m{/}\m{x}\m{]}\m{\varphi}
\endm
\setbox\startprefix=\hbox{\tt \ \ tfindes.2\ \$e\ }
\setbox\contprefix=\hbox{\tt \ \ \ \ \ \ \ \ \ \ \ \ \ \ \ }
\startm
\m{\vdash}\m{(}\m{x}\m{\in}\m{{\rm On}}\m{\rightarrow}\m{(}\m{\varphi}\m{%
\rightarrow}\m{[}\m{{\rm suc}}\m{x}\m{/}\m{x}\m{]}\m{\varphi}\m{)}\m{)}
\endm
\setbox\startprefix=\hbox{\tt \ \ tfindes.3\ \$e\ }
\setbox\contprefix=\hbox{\tt \ \ \ \ \ \ \ \ \ \ \ \ \ \ \ }
\startm
\m{\vdash}\m{(}\m{{\rm Lim}}\m{y}\m{\rightarrow}\m{(}\m{\forall}\m{x}\m{\in}%
\m{y}\m{\varphi}\m{\rightarrow}\m{[}\m{y}\m{/}\m{x}\m{]}\m{\varphi}\m{)}\m{)}
\endm
\setbox\startprefix=\hbox{\tt \ \ tfindes\ \$p\ }
\setbox\contprefix=\hbox{\tt \ \ \ \ \ \ \ \ \ \ \ \ \ }
\startm
\m{\vdash}\m{(}\m{x}\m{\in}\m{{\rm On}}\m{\rightarrow}\m{\varphi}\m{)}
\endm
\vskip 1ex

\noindent Principle of Transfinite Recursion.\index{transfinite
recursion} Theorem 7.41 of Takeuti and Zaring, p.~47.  Transfinite
recursion is the key theorem that allows arithmetic of ordinals to be
rigorously defined, and has many other important uses as well.
Hypotheses \texttt{tfr.1} and \texttt{tfr.2} specify a certain (proper)
class $ F$.  The complicated definition of $ F$ is not important in
itself; what is important is that there be such an $ F$ with the
required properties, and we show this by displaying $ F$ explicitly.
\texttt{tfr1} states that $ F$ is a function whose domain is the set of
ordinal numbers.  \texttt{tfr2} states that any value of $ F$ is
completely determined by its previous values and the values of an
auxiliary function, $G$.  \texttt{tfr3} states that $ F$ is unique,
i.e.\ it is the only function that satisfies \texttt{tfr1} and
\texttt{tfr2}.  Note that $ f$ is an individual variable like $x$ and
$y$; it is just a mnemonic to remind us that $A$ is a collection of
functions.

\vskip 0.5ex
\setbox\startprefix=\hbox{\tt \ \ tfr.1\ \$e\ }
\setbox\contprefix=\hbox{\tt \ \ \ \ \ \ \ \ \ \ \ }
\startm
\m{\vdash}\m{A}\m{=}\m{\{}\m{f}\m{|}\m{\exists}\m{x}\m{\in}\m{{\rm On}}\m{(}%
\m{f}\m{{\rm Fn}}\m{x}\m{\wedge}\m{\forall}\m{y}\m{\in}\m{x}\m{(}\m{f}\m{`}%
\m{y}\m{)}\m{=}\m{(}\m{G}\m{`}\m{(}\m{f}\m{\restriction}\m{y}\m{)}\m{)}\m{)}%
\m{\}}
\endm
\setbox\startprefix=\hbox{\tt \ \ tfr.2\ \$e\ }
\setbox\contprefix=\hbox{\tt \ \ \ \ \ \ \ \ \ \ \ }
\startm
\m{\vdash}\m{F}\m{=}\m{\bigcup}\m{A}
\endm
\setbox\startprefix=\hbox{\tt \ \ tfr1\ \$p\ }
\setbox\contprefix=\hbox{\tt \ \ \ \ \ \ \ \ \ \ }
\startm
\m{\vdash}\m{F}\m{{\rm Fn}}\m{{\rm On}}
\endm
\setbox\startprefix=\hbox{\tt \ \ tfr2\ \$p\ }
\setbox\contprefix=\hbox{\tt \ \ \ \ \ \ \ \ \ \ }
\startm
\m{\vdash}\m{(}\m{z}\m{\in}\m{{\rm On}}\m{\rightarrow}\m{(}\m{F}\m{`}\m{z}%
\m{)}\m{=}\m{(}\m{G}\m{`}\m{(}\m{F}\m{\restriction}\m{z}\m{)}\m{)}\m{)}
\endm
\setbox\startprefix=\hbox{\tt \ \ tfr3\ \$p\ }
\setbox\contprefix=\hbox{\tt \ \ \ \ \ \ \ \ \ \ }
\startm
\m{\vdash}\m{(}\m{(}\m{B}\m{{\rm Fn}}\m{{\rm On}}\m{\wedge}\m{\forall}\m{x}\m{%
\in}\m{{\rm On}}\m{(}\m{B}\m{`}\m{x}\m{)}\m{=}\m{(}\m{G}\m{`}\m{(}\m{B}\m{%
\restriction}\m{x}\m{)}\m{)}\m{)}\m{\rightarrow}\m{B}\m{=}\m{F}\m{)}
\endm
\vskip 1ex

\noindent The existence of omega (the class of natural numbers).\index{natural
number}\index{omega ($\omega$)}\index{Axiom of Infinity}  Axiom 7 of Takeuti
and Zaring, p.~43.  (This is the only theorem in this section requiring the
Axiom of Infinity.)

\vskip 0.5ex
\setbox\startprefix=\hbox{\tt \
\ omex\ \$p\ }
\setbox\contprefix=\hbox{\tt \ \ \ \ \ \ \ \ \ \ }
\startm
\m{\vdash}\m{\omega}\m{\in}\m{{\rm V}}
\endm
%\vskip 2ex


\section{Axioms for Real and Complex Numbers}\label{real}
\index{real and complex numbers!axioms for}

This section presents the axioms for real and complex numbers, along
with some commentary about them.  Analysis
textbooks implicitly or explicitly use these axioms or their equivalents
as their starting point.  In the database \texttt{set.mm}, we define real
and complex numbers as (rather complicated) specific sets and derive these
axioms as {\em theorems} from the axioms of ZF set theory, using a method
called Dedekind cuts.  We omit the details of this construction, which you can
follow if you wish using the \texttt{set.mm} database in conjunction with the
textbooks referenced therein.

Once we prove those theorems, we then restate these proven theorems as axioms.
This lets us easily identify which axioms are needed for a particular complex number proof, without the obfuscation of the set theory used to derive them.
As a result,
the construction is actually unimportant other
than to show that sets exist that satisfy the axioms, and thus that the axioms
are consistent if ZF set theory is consistent.  When working with real numbers
you can think of them as being the actual sets resulting
from the construction (for definiteness), or you can
think of them as otherwise unspecified sets that happen to satisfy the axioms.
The derivation is not easy, but the fact that it works is quite remarkable
and lends support to the idea that ZFC set theory is all we need to
provide a foundation for essentially all of mathematics.

\needspace{3\baselineskip}
\subsection{The Axioms for Real and Complex Numbers Themselves}\label{realactual}

For the axioms we are given (or postulate) 8 classes:  $\mathbb{C}$ (the
set of complex numbers), $\mathbb{R}$ (the set of real numbers, a subset
of $\mathbb{C}$), $0$ (zero), $1$ (one), $i$ (square root of
$-1$), $+$ (plus), $\cdot$ (times), and
$<_{\mathbb{R}}$ (less than for just the real numbers).
Subtraction and division are defined terms and are not part of the
axioms; for their definitions see \texttt{set.mm}.

Note that the notation $(A+B)$ (and similarly $(A\cdot B)$) specifies a class
called an {\em operation},\index{operation} and is the function value of the
class $+$ at ordered pair $\langle A,B \rangle$.  An operation is defined by
statement \texttt{df-opr} on p.~\pageref{dfopr}.
The notation $A <_{\mathbb{R}} B$ specifies a
wff called a {\em binary relation}\index{binary relation} and means $\langle A,B \rangle \in \,<_{\mathbb{R}}$, as defined by statement \texttt{df-br} on p.~\pageref{dfbr}.

Our set of 8 given classes is assumed to satisfy the following 22 axioms
(in the axioms listed below, $<$ really means $<_{\mathbb{R}}$).

\vskip 2ex

\noindent 1. The real numbers are a subset of the complex numbers.

%\vskip 0.5ex
\setbox\startprefix=\hbox{\tt \ \ ax-resscn\ \$p\ }
\setbox\contprefix=\hbox{\tt \ \ \ \ \ \ \ \ \ \ \ \ \ \ }
\startm
\m{\vdash}\m{\mathbb{R}}\m{\subseteq}\m{\mathbb{C}}
\endm
%\vskip 1ex

\noindent 2. One is a complex number.

%\vskip 0.5ex
\setbox\startprefix=\hbox{\tt \ \ ax-1cn\ \$p\ }
\setbox\contprefix=\hbox{\tt \ \ \ \ \ \ \ \ \ \ \ }
\startm
\m{\vdash}\m{1}\m{\in}\m{\mathbb{C}}
\endm
%\vskip 1ex

\noindent 3. The imaginary number $i$ is a complex number.

%\vskip 0.5ex
\setbox\startprefix=\hbox{\tt \ \ ax-icn\ \$p\ }
\setbox\contprefix=\hbox{\tt \ \ \ \ \ \ \ \ \ \ \ }
\startm
\m{\vdash}\m{i}\m{\in}\m{\mathbb{C}}
\endm
%\vskip 1ex

\noindent 4. Complex numbers are closed under addition.

%\vskip 0.5ex
\setbox\startprefix=\hbox{\tt \ \ ax-addcl\ \$p\ }
\setbox\contprefix=\hbox{\tt \ \ \ \ \ \ \ \ \ \ \ \ \ }
\startm
\m{\vdash}\m{(}\m{(}\m{A}\m{\in}\m{\mathbb{C}}\m{\wedge}\m{B}\m{\in}\m{\mathbb{C}}%
\m{)}\m{\rightarrow}\m{(}\m{A}\m{+}\m{B}\m{)}\m{\in}\m{\mathbb{C}}\m{)}
\endm
%\vskip 1ex

\noindent 5. Real numbers are closed under addition.

%\vskip 0.5ex
\setbox\startprefix=\hbox{\tt \ \ ax-addrcl\ \$p\ }
\setbox\contprefix=\hbox{\tt \ \ \ \ \ \ \ \ \ \ \ \ \ \ }
\startm
\m{\vdash}\m{(}\m{(}\m{A}\m{\in}\m{\mathbb{R}}\m{\wedge}\m{B}\m{\in}\m{\mathbb{R}}%
\m{)}\m{\rightarrow}\m{(}\m{A}\m{+}\m{B}\m{)}\m{\in}\m{\mathbb{R}}\m{)}
\endm
%\vskip 1ex

\noindent 6. Complex numbers are closed under multiplication.

%\vskip 0.5ex
\setbox\startprefix=\hbox{\tt \ \ ax-mulcl\ \$p\ }
\setbox\contprefix=\hbox{\tt \ \ \ \ \ \ \ \ \ \ \ \ \ }
\startm
\m{\vdash}\m{(}\m{(}\m{A}\m{\in}\m{\mathbb{C}}\m{\wedge}\m{B}\m{\in}\m{\mathbb{C}}%
\m{)}\m{\rightarrow}\m{(}\m{A}\m{\cdot}\m{B}\m{)}\m{\in}\m{\mathbb{C}}\m{)}
\endm
%\vskip 1ex

\noindent 7. Real numbers are closed under multiplication.

%\vskip 0.5ex
\setbox\startprefix=\hbox{\tt \ \ ax-mulrcl\ \$p\ }
\setbox\contprefix=\hbox{\tt \ \ \ \ \ \ \ \ \ \ \ \ \ \ }
\startm
\m{\vdash}\m{(}\m{(}\m{A}\m{\in}\m{\mathbb{R}}\m{\wedge}\m{B}\m{\in}\m{\mathbb{R}}%
\m{)}\m{\rightarrow}\m{(}\m{A}\m{\cdot}\m{B}\m{)}\m{\in}\m{\mathbb{R}}\m{)}
\endm
%\vskip 1ex

\noindent 8. Multiplication of complex numbers is commutative.

%\vskip 0.5ex
\setbox\startprefix=\hbox{\tt \ \ ax-mulcom\ \$p\ }
\setbox\contprefix=\hbox{\tt \ \ \ \ \ \ \ \ \ \ \ \ \ \ }
\startm
\m{\vdash}\m{(}\m{(}\m{A}\m{\in}\m{\mathbb{C}}\m{\wedge}\m{B}\m{\in}\m{\mathbb{C}}%
\m{)}\m{\rightarrow}\m{(}\m{A}\m{\cdot}\m{B}\m{)}\m{=}\m{(}\m{B}\m{\cdot}\m{A}%
\m{)}\m{)}
\endm
%\vskip 1ex

\noindent 9. Addition of complex numbers is associative.

%\vskip 0.5ex
\setbox\startprefix=\hbox{\tt \ \ ax-addass\ \$p\ }
\setbox\contprefix=\hbox{\tt \ \ \ \ \ \ \ \ \ \ \ \ \ \ }
\startm
\m{\vdash}\m{(}\m{(}\m{A}\m{\in}\m{\mathbb{C}}\m{\wedge}\m{B}\m{\in}\m{\mathbb{C}}%
\m{\wedge}\m{C}\m{\in}\m{\mathbb{C}}\m{)}\m{\rightarrow}\m{(}\m{(}\m{A}\m{+}%
\m{B}\m{)}\m{+}\m{C}\m{)}\m{=}\m{(}\m{A}\m{+}\m{(}\m{B}\m{+}\m{C}\m{)}\m{)}%
\m{)}
\endm
%\vskip 1ex

\noindent 10. Multiplication of complex numbers is associative.

%\vskip 0.5ex
\setbox\startprefix=\hbox{\tt \ \ ax-mulass\ \$p\ }
\setbox\contprefix=\hbox{\tt \ \ \ \ \ \ \ \ \ \ \ \ \ \ }
\startm
\m{\vdash}\m{(}\m{(}\m{A}\m{\in}\m{\mathbb{C}}\m{\wedge}\m{B}\m{\in}\m{\mathbb{C}}%
\m{\wedge}\m{C}\m{\in}\m{\mathbb{C}}\m{)}\m{\rightarrow}\m{(}\m{(}\m{A}\m{\cdot}%
\m{B}\m{)}\m{\cdot}\m{C}\m{)}\m{=}\m{(}\m{A}\m{\cdot}\m{(}\m{B}\m{\cdot}\m{C}%
\m{)}\m{)}\m{)}
\endm
%\vskip 1ex

\noindent 11. Multiplication distributes over addition for complex numbers.

%\vskip 0.5ex
\setbox\startprefix=\hbox{\tt \ \ ax-distr\ \$p\ }
\setbox\contprefix=\hbox{\tt \ \ \ \ \ \ \ \ \ \ \ \ \ }
\startm
\m{\vdash}\m{(}\m{(}\m{A}\m{\in}\m{\mathbb{C}}\m{\wedge}\m{B}\m{\in}\m{\mathbb{C}}%
\m{\wedge}\m{C}\m{\in}\m{\mathbb{C}}\m{)}\m{\rightarrow}\m{(}\m{A}\m{\cdot}\m{(}%
\m{B}\m{+}\m{C}\m{)}\m{)}\m{=}\m{(}\m{(}\m{A}\m{\cdot}\m{B}\m{)}\m{+}\m{(}%
\m{A}\m{\cdot}\m{C}\m{)}\m{)}\m{)}
\endm
%\vskip 1ex

\noindent 12. The square of $i$ equals $-1$ (expressed as $i$-squared plus 1 is
0).

%\vskip 0.5ex
\setbox\startprefix=\hbox{\tt \ \ ax-i2m1\ \$p\ }
\setbox\contprefix=\hbox{\tt \ \ \ \ \ \ \ \ \ \ \ \ }
\startm
\m{\vdash}\m{(}\m{(}\m{i}\m{\cdot}\m{i}\m{)}\m{+}\m{1}\m{)}\m{=}\m{0}
\endm
%\vskip 1ex

\noindent 13. One and zero are distinct.

%\vskip 0.5ex
\setbox\startprefix=\hbox{\tt \ \ ax-1ne0\ \$p\ }
\setbox\contprefix=\hbox{\tt \ \ \ \ \ \ \ \ \ \ \ \ }
\startm
\m{\vdash}\m{1}\m{\ne}\m{0}
\endm
%\vskip 1ex

\noindent 14. One is an identity element for real multiplication.

%\vskip 0.5ex
\setbox\startprefix=\hbox{\tt \ \ ax-1rid\ \$p\ }
\setbox\contprefix=\hbox{\tt \ \ \ \ \ \ \ \ \ \ \ }
\startm
\m{\vdash}\m{(}\m{A}\m{\in}\m{\mathbb{R}}\m{\rightarrow}\m{(}\m{A}\m{\cdot}\m{1}%
\m{)}\m{=}\m{A}\m{)}
\endm
%\vskip 1ex

\noindent 15. Every real number has a negative.

%\vskip 0.5ex
\setbox\startprefix=\hbox{\tt \ \ ax-rnegex\ \$p\ }
\setbox\contprefix=\hbox{\tt \ \ \ \ \ \ \ \ \ \ \ \ \ \ }
\startm
\m{\vdash}\m{(}\m{A}\m{\in}\m{\mathbb{R}}\m{\rightarrow}\m{\exists}\m{x}\m{\in}%
\m{\mathbb{R}}\m{(}\m{A}\m{+}\m{x}\m{)}\m{=}\m{0}\m{)}
\endm
%\vskip 1ex

\noindent 16. Every nonzero real number has a reciprocal.

%\vskip 0.5ex
\setbox\startprefix=\hbox{\tt \ \ ax-rrecex\ \$p\ }
\setbox\contprefix=\hbox{\tt \ \ \ \ \ \ \ \ \ \ \ \ \ \ }
\startm
\m{\vdash}\m{(}\m{A}\m{\in}\m{\mathbb{R}}\m{\rightarrow}\m{(}\m{A}\m{\ne}\m{0}%
\m{\rightarrow}\m{\exists}\m{x}\m{\in}\m{\mathbb{R}}\m{(}\m{A}\m{\cdot}%
\m{x}\m{)}\m{=}\m{1}\m{)}\m{)}
\endm
%\vskip 1ex

\noindent 17. A complex number can be expressed in terms of two reals.

%\vskip 0.5ex
\setbox\startprefix=\hbox{\tt \ \ ax-cnre\ \$p\ }
\setbox\contprefix=\hbox{\tt \ \ \ \ \ \ \ \ \ \ \ \ }
\startm
\m{\vdash}\m{(}\m{A}\m{\in}\m{\mathbb{C}}\m{\rightarrow}\m{\exists}\m{x}\m{\in}%
\m{\mathbb{R}}\m{\exists}\m{y}\m{\in}\m{\mathbb{R}}\m{A}\m{=}\m{(}\m{x}\m{+}\m{(}%
\m{y}\m{\cdot}\m{i}\m{)}\m{)}\m{)}
\endm
%\vskip 1ex

\noindent 18. Ordering on reals satisfies strict trichotomy.

%\vskip 0.5ex
\setbox\startprefix=\hbox{\tt \ \ ax-pre-lttri\ \$p\ }
\setbox\contprefix=\hbox{\tt \ \ \ \ \ \ \ \ \ \ \ \ \ }
\startm
\m{\vdash}\m{(}\m{(}\m{A}\m{\in}\m{\mathbb{R}}\m{\wedge}\m{B}\m{\in}\m{\mathbb{R}}%
\m{)}\m{\rightarrow}\m{(}\m{A}\m{<}\m{B}\m{\leftrightarrow}\m{\lnot}\m{(}\m{A}%
\m{=}\m{B}\m{\vee}\m{B}\m{<}\m{A}\m{)}\m{)}\m{)}
\endm
%\vskip 1ex

\noindent 19. Ordering on reals is transitive.

%\vskip 0.5ex
\setbox\startprefix=\hbox{\tt \ \ ax-pre-lttrn\ \$p\ }
\setbox\contprefix=\hbox{\tt \ \ \ \ \ \ \ \ \ \ \ \ \ }
\startm
\m{\vdash}\m{(}\m{(}\m{A}\m{\in}\m{\mathbb{R}}\m{\wedge}\m{B}\m{\in}\m{\mathbb{R}}%
\m{\wedge}\m{C}\m{\in}\m{\mathbb{R}}\m{)}\m{\rightarrow}\m{(}\m{(}\m{A}\m{<}%
\m{B}\m{\wedge}\m{B}\m{<}\m{C}\m{)}\m{\rightarrow}\m{A}\m{<}\m{C}\m{)}\m{)}
\endm
%\vskip 1ex

\noindent 20. Ordering on reals is preserved after addition to both sides.

%\vskip 0.5ex
\setbox\startprefix=\hbox{\tt \ \ ax-pre-ltadd\ \$p\ }
\setbox\contprefix=\hbox{\tt \ \ \ \ \ \ \ \ \ \ \ \ \ }
\startm
\m{\vdash}\m{(}\m{(}\m{A}\m{\in}\m{\mathbb{R}}\m{\wedge}\m{B}\m{\in}\m{\mathbb{R}}%
\m{\wedge}\m{C}\m{\in}\m{\mathbb{R}}\m{)}\m{\rightarrow}\m{(}\m{A}\m{<}\m{B}\m{%
\rightarrow}\m{(}\m{C}\m{+}\m{A}\m{)}\m{<}\m{(}\m{C}\m{+}\m{B}\m{)}\m{)}\m{)}
\endm
%\vskip 1ex

\noindent 21. The product of two positive reals is positive.

%\vskip 0.5ex
\setbox\startprefix=\hbox{\tt \ \ ax-pre-mulgt0\ \$p\ }
\setbox\contprefix=\hbox{\tt \ \ \ \ \ \ \ \ \ \ \ \ \ \ }
\startm
\m{\vdash}\m{(}\m{(}\m{A}\m{\in}\m{\mathbb{R}}\m{\wedge}\m{B}\m{\in}\m{\mathbb{R}}%
\m{)}\m{\rightarrow}\m{(}\m{(}\m{0}\m{<}\m{A}\m{\wedge}\m{0}%
\m{<}\m{B}\m{)}\m{\rightarrow}\m{0}\m{<}\m{(}\m{A}\m{\cdot}\m{B}\m{)}%
\m{)}\m{)}
\endm
%\vskip 1ex

\noindent 22. A non-empty, bounded-above set of reals has a supremum.

%\vskip 0.5ex
\setbox\startprefix=\hbox{\tt \ \ ax-pre-sup\ \$p\ }
\setbox\contprefix=\hbox{\tt \ \ \ \ \ \ \ \ \ \ \ }
\startm
\m{\vdash}\m{(}\m{(}\m{A}\m{\subseteq}\m{\mathbb{R}}\m{\wedge}\m{A}\m{\ne}\m{%
\varnothing}\m{\wedge}\m{\exists}\m{x}\m{\in}\m{\mathbb{R}}\m{\forall}\m{y}\m{%
\in}\m{A}\m{\,y}\m{<}\m{x}\m{)}\m{\rightarrow}\m{\exists}\m{x}\m{\in}\m{%
\mathbb{R}}\m{(}\m{\forall}\m{y}\m{\in}\m{A}\m{\lnot}\m{x}\m{<}\m{y}\m{\wedge}\m{%
\forall}\m{y}\m{\in}\m{\mathbb{R}}\m{(}\m{y}\m{<}\m{x}\m{\rightarrow}\m{\exists}%
\m{z}\m{\in}\m{A}\m{\,y}\m{<}\m{z}\m{)}\m{)}\m{)}
\endm

% NOTE: The \m{...} expressions above could be represented as
% $ \vdash ( ( A \subseteq \mathbb{R} \wedge A \ne \varnothing \wedge \exists x \in \mathbb{R} \forall y \in A \,y < x ) \rightarrow \exists x \in \mathbb{R} ( \forall y \in A \lnot x < y \wedge \forall y \in \mathbb{R} ( y < x \rightarrow \exists z \in A \,y < z ) ) ) $

\vskip 2ex

This completes the set of axioms for real and complex numbers.  You may
wish to look at how subtraction, division, and decimal numbers
are defined in \texttt{set.mm}, and for fun look at the proof of $2+
2 = 4$ (theorem \texttt{2p2e4} in \texttt{set.mm})
as discussed in section \ref{2p2e4}.

In \texttt{set.mm} we define the non-negative integers $\mathbb{N}$, the integers
$\mathbb{Z}$, and the rationals $\mathbb{Q}$ as subsets of $\mathbb{R}$.  This leads
to the nice inclusion $\mathbb{N} \subseteq \mathbb{Z} \subseteq \mathbb{Q} \subseteq
\mathbb{R} \subseteq \mathbb{C}$, giving us a uniform framework in which, for
example, a property such as commutativity of complex number addition
automatically applies to integers.  The natural numbers $\mathbb{N}$
are different from the set $\omega$ we defined earlier, but both satisfy
Peano's postulates.

\subsection{Complex Number Axioms in Analysis Texts}

Most analysis texts construct complex numbers as ordered pairs of reals,
leading to construction-dependent properties that satisfy these axioms
but are not stated in their pure form.  (This is also done in
\texttt{set.mm} but our axioms are extracted from that construction.)
Other texts will simply state that $\mathbb{R}$ is a ``complete ordered
subfield of $\mathbb{C}$,'' leading to redundant axioms when this phrase
is completely expanded out.  In fact I have not seen a text with the
axioms in the explicit form above.
None of these axioms is unique individually, but this carefully worked out
collection of axioms is the result of years of work
by the Metamath community.

\subsection{Eliminating Unnecessary Complex Number Axioms}

We once had more axioms for real and complex numbers, but over years of time
we (the Metamath community)
have found ways to eliminate them (by proving them from other axioms)
or weaken them (by making weaker claims without reducing what
can be proved).
In particular, here are statements that used to be complex number
axioms but have since been formally proven (with Metamath) to be redundant:

\begin{itemize}
\item
  $\mathbb{C} \in V$.
  At one time this was listed as a ``complex number axiom.''
  However, this is not properly speaking a complex number axiom,
  and in any case its proof uses axioms of set theory.
  Proven redundant by Mario Carneiro\index{Carneiro, Mario} on
  17-Nov-2014 (see \texttt{axcnex}).
\item
  $((A \in \mathbb{C} \land B \in \mathbb{C}$) $\rightarrow$
  $(A + B) = (B + A))$.
  Proved redundant by Eric Schmidt\index{Schmidt, Eric} on 19-Jun-2012,
  and formalized by Scott Fenton\index{Fenton, Scott} on 3-Jan-2013
  (see \texttt{addcom}).
\item
  $(A \in \mathbb{C} \rightarrow (A + 0) = A)$.
  Proved redundant by Eric Schmidt on 19-Jun-2012,
  and formalized by Scott Fenton on 3-Jan-2013
  (see \texttt{addid1}).
\item
  $(A \in \mathbb{C} \rightarrow \exists x \in \mathbb{C} (A + x) = 0)$.
  Proved redundant by Eric Schmidt and formalized on 21-May-2007
  (see \texttt{cnegex}).
\item
  $((A \in \mathbb{C} \land A \ne 0) \rightarrow \exists x \in \mathbb{C} (A \cdot x) = 1)$.
  Proved redundant by Eric Schmidt and formalized on 22-May-2007
  (see \texttt{recex}).
\item
  $0 \in \mathbb{R}$.
  Proved redundant by Eric Schmidt on 19-Feb-2005 and formalized 21-May-2007
  (see \texttt{0re}).
\end{itemize}

We could eliminate 0 as an axiomatic object by defining it as
$( ( i \cdot i ) + 1 )$
and replacing it with this expression throughout the axioms. If this
is done, axiom ax-i2m1 becomes redundant. However, the remaining axioms
would become longer and less intuitive.

Eric Schmidt's paper analyzing this axiom system \cite{Schmidt}
presented a proof that these remaining axioms,
with the possible exception of ax-mulcom, are independent of the others.
It is currently an open question if ax-mulcom is independent of the others.

\section{Two Plus Two Equals Four}\label{2p2e4}

Here is a proof that $2 + 2 = 4$, as proven in the theorem \texttt{2p2e4}
in the database \texttt{set.mm}.
This is a useful demonstration of what a Metamath proof can look like.
This proof may have more steps than you're used to, but each step is rigorously
proven all the way back to the axioms of logic and set theory.
This display was originally generated by the Metamath program
as an {\sc HTML} file.

In the table showing the proof ``Step'' is the sequential step number,
while its associated ``Expression'' is an expression that we have proved.
``Ref'' is the name of a theorem or axiom that justifies that expression,
and ``Hyp'' refers to previous steps (if any) that the theorem or axiom
needs so that we can use it.  Expressions are indented further than
the expressions that depend on them to show their interdependencies.

\begin{table}[!htbp]
\caption{Two plus two equals four}
\begin{tabular}{lllll}
\textbf{Step} & \textbf{Hyp} & \textbf{Ref} & \textbf{Expression} & \\
1  &       & df-2    & $ \; \; \vdash 2 = 1 + 1$  & \\
2  & 1     & oveq2i  & $ \; \vdash (2 + 2) = (2 + (1 + 1))$ & \\
3  &       & df-4    & $ \; \; \vdash 4 = (3 + 1)$ & \\
4  &       & df-3    & $ \; \; \; \vdash 3 = (2 + 1)$ & \\
5  & 4     & oveq1i  & $ \; \; \vdash (3 + 1) = ((2 + 1) + 1)$ & \\
6  &       & 2cn     & $ \; \; \; \vdash 2 \in \mathbb{C}$ & \\
7  &       & ax-1cn  & $ \; \; \; \vdash 1 \in \mathbb{C}$ & \\
8  & 6,7,7 & addassi & $ \; \; \vdash ((2 + 1) + 1) = (2 + (1 + 1))$ & \\
9  & 3,5,8 & 3eqtri  & $ \; \vdash 4 = (2 + (1 + 1))$ & \\
10 & 2,9   & eqtr4i  & $ \vdash (2 + 2) = 4$ & \\
\end{tabular}
\end{table}

Step 1 says that we can assert that $2 = 1 + 1$ because it is
justified by \texttt{df-2}.
What is \texttt{df-2}?
It is simply the definition of $2$, which in our system is defined as being
equal to $1 + 1$.  This shows how we can use definitions in proofs.

Look at Step 2 of the proof. In the Ref column, we see that it references
a previously proved theorem, \texttt{oveq2i}.
It turns out that
theorem \texttt{oveq2i} requires a
hypothesis, and in the Hyp column of Step 2 we indicate that Step 1 will
satisfy (match) this hypothesis.
If we looked at \texttt{oveq2i}
we would find that it proves that given some hypothesis
$A = B$, we can prove that $( C F A ) = ( C F B )$.
If we use \texttt{oveq2i} and apply step 1's result as the hypothesis,
that will mean that $A = 2$ and $B = ( 1 + 1 )$ within this use of
\texttt{oveq2i}.
We are free to select any value of $C$ and $F$ (subject to syntax constraints),
so we are free to select $C = 2$ and $F = +$,
producing our desired result,
$ (2 + 2) = (2 + (1 + 1))$.

Step 2 is an example of substitution.
In the end, every step in every proof uses only this one substitution rule.
All the rules of logic, and all the axioms, are expressed so that
they can be used via this one substitution rule.
So once you master substitution, you can master every Metamath proof,
no exceptions.

Each step is clear and can be immediately checked.
In the {\sc HTML} display you can even click on each reference to see why it is
justified, making it easy to see why the proof works.

\section{Deduction}\label{deduction}

Strictly speaking,
a deduction (also called an inference) is a kind of statement that needs
some hypotheses to be true in order for its conclusion to be true.
A theorem, on the other hand, has no hypotheses.
Informally we often call both of them theorems, but in this section we
will stick to the strict definitions.

It sometimes happens that we have proved a deduction of the form
$\varphi \Rightarrow \psi$\index{$\Rightarrow$}
(given hypothesis $\varphi$ we can prove $\psi$)
and we want to then prove a theorem of the form
$\varphi \rightarrow \psi$.

Converting a deduction (which uses a hypothesis) into a theorem
(which does not) is not as simple as you might think.
The deduction says, ``if we can prove $\varphi$ then we can prove $\psi$,''
which is in some sense weaker than saying
``$\varphi$ implies $\psi$.''
There is no axiom of logic that permits us to directly obtain the theorem
given the deduction.\footnote{
The conversion of a deduction to a theorem does not even hold in general
for quantum propositional calculus,
which is a weak subset of classical propositional calculus.
It has been shown that adding the Standard Deduction Theorem (discussed below)
to quantum propositional calculus turns it into classical
propositional calculus!
}

This is in contrast to going the other way.
If we have the theorem ($\varphi \rightarrow \psi$),
it is easy to recover the deduction
($\varphi \Rightarrow \psi$)
using modus ponens\index{modus ponens}
(\texttt{ax-mp}; see section \ref{axmp}).

In the following subsections we first discuss the standard deduction theorem
(the traditional but awkward way to convert deductions into theorems) and
the weak deduction theorem (a limited version of the standard deduction
theorem that is easier to use and was once widely used in
\texttt{set.mm}\index{set theory database (\texttt{set.mm})}).
In section \ref{deductionstyle} we discuss
deduction style, the newer approach we now recommend in most cases.
Deduction style uses ``deduction form,'' a form that
prefixes each hypothesis (other than definitions) and the
conclusion with a universal antecedent (``$\varphi \rightarrow$'').
Deduction style is widely used in \texttt{set.mm},
so it is useful to understand it and \textit{why} it is widely used.
Section \ref{naturaldeduction}
briefly discusses our approach for using natural deduction
within \texttt{set.mm},
as that approach is deeply related to deduction style.
We conclude with a summary of the strengths of
our approach, which we believe are compelling.

\subsection{The Standard Deduction Theorem}\label{standarddeductiontheorem}

It is possible to make use of information
contained in the deduction or its proof to assist us with the proof of
the related theorem.
In traditional logic books, there is a metatheorem called the
Deduction Theorem\index{Deduction Theorem}\index{Standard Deduction Theorem},
discovered independently by Herbrand and Tarski around 1930.
The Deduction Theorem, which we often call the Standard Deduction Theorem,
provides an algorithm for constructing a proof of a theorem from the
proof of its corresponding deduction. See, for example,
\cite[p.~56]{Margaris}\index{Margaris, Angelo}.
To construct a proof for a theorem, the
algorithm looks at each step in the proof of the original deduction and
rewrites the step with several steps wherein the hypothesis is eliminated
and becomes an antecedent.

In ordinary mathematics, no one actually carries out the algorithm,
because (in its most basic form) it involves an exponential explosion of
the number of proof steps as more hypotheses are eliminated. Instead,
the Standard Deduction Theorem is invoked simply to claim that it can
be done in principle, without actually doing it.
What's more, the algorithm is not as simple as it might first appear
when applying it rigorously.
There is a subtle restriction on the Standard Deduction Theorem
that must be taken into account involving the axiom of generalization
when working with predicate calculus (see the literature for more detail).

One of the goals of Metamath is to let you plainly see, with as few
underlying concepts as possible, how mathematics can be derived directly
from the axioms, and not indirectly according to some hidden rules
buried inside a program or understood only by logicians. If we added
the Standard Deduction Theorem to the language and proof verifier,
that would greatly complicate both and largely defeat Metamath's goal
of simplicity. In principle, we could show direct proofs by expanding
out the proof steps generated by the algorithm of the Standard Deduction
Theorem, but that is not feasible in practice because the number of proof
steps quickly becomes huge, even astronomical.
Since the algorithm of the Standard Deduction Theorem is driven by the proof,
we would have to go through that proof
all over again---starting from axioms---in order to obtain the theorem form.
In terms of proof length, there would be no savings over just
proving the theorem directly instead of first proving the deduction form.

\subsection{Weak Deduction Theorem}\label{weakdeductiontheorem}

We have developed
a more efficient method for proving a theorem from a deduction
that can be used instead of the Standard Deduction Theorem
in many (but not all) cases.
We call this more efficient method the
Weak Deduction Theorem\index{Weak Deduction Theorem}.\footnote{
There is also an unrelated ``Weak Deduction Theorem''
in the field of relevance logic, so to avoid confusion we could call
ours the ``Weak Deduction Theorem for Classical Logic.''}
Unlike the Standard Deduction Theorem, the Weak Deduction Theorem produces the
theorem directly from a special substitution instance of the deduction,
using a small, fixed number of steps roughly proportional to the length
of the final theorem.

If you come to a proof referencing the Weak Deduction Theorem
\texttt{dedth} (or one of its variants \texttt{dedthxx}),
here is how to follow the proof without getting into the details:
just click on the theorem referenced in the step
just before the reference to \texttt{dedth} and ignore everything else.
Theorem \texttt{dedth} simply turns a hypothesis into an antecedent
(i.e. the hypothesis followed by $\rightarrow$
is placed in front of the assertion, and the hypothesis
itself is eliminated) given certain conditions.

The Weak Deduction Theorem
eliminates a hypothesis $\varphi$, making it become an antecedent.
It does this by proving an expression
$ \varphi \rightarrow \psi $ given two hypotheses:
(1)
$ ( A = {\rm if} ( \varphi , A , B ) \rightarrow ( \varphi \leftrightarrow \chi ) ) $
and
(2) $\chi$.
Note that it requires that a proof exists for $\varphi$ when the class variable
$A$ is replaced with a specific class $B$. The hypothesis $\chi$
should be assigned to the inference.
You can see the details of the proof of the Weak Deduction Theorem
in theorem \texttt{dedth}.

The Weak Deduction Theorem
is probably easier to understand by studying proofs that make use of it.
For example, let's look at the proof of \texttt{renegcl}, which proves that
$ \vdash ( A \in \mathbb{R} \rightarrow - A \in \mathbb{R} )$:

\needspace{4\baselineskip}
\begin{longtabu} {l l l X}
\textbf{Step} & \textbf{Hyp} & \textbf{Ref} & \textbf{Expression} \\
  1 &  & negeq &
  $\vdash$ $($ $A$ $=$ ${\rm if}$ $($ $A$ $\in$ $\mathbb{R}$ $,$ $A$ $,$ $1$ $)$ $\rightarrow$
  $\textrm{-}$ $A$ $=$ $\textrm{-}$ ${\rm if}$ $($ $A$ $\in$ $\mathbb{R}$
  $,$ $A$ $,$ $1$ $)$ $)$ \\
 2 & 1 & eleq1d &
    $\vdash$ $($ $A$ $=$ ${\rm if}$ $($ $A$ $\in$ $\mathbb{R}$ $,$ $A$ $,$ $1$ $)$ $\rightarrow$ $($
    $\textrm{-}$ $A$ $\in$ $\mathbb{R}$ $\leftrightarrow$
    $\textrm{-}$ ${\rm if}$ $($ $A$ $\in$ $\mathbb{R}$ $,$ $A$ $,$ $1$ $)$ $\in$
    $\mathbb{R}$ $)$ $)$ \\
 3 &  & 1re & $\vdash 1 \in \mathbb{R}$ \\
 4 & 3 & elimel &
   $\vdash {\rm if} ( A \in \mathbb{R} , A , 1 ) \in \mathbb{R}$ \\
 5 & 4 & renegcli &
   $\vdash \textrm{-} {\rm if} ( A \in \mathbb{R} , A , 1 ) \in \mathbb{R}$ \\
 6 & 2,5 & dedth &
   $\vdash ( A \in \mathbb{R} \rightarrow \textrm{-} A \in \mathbb{R}$ ) \\
\end{longtabu}

The somewhat strange-looking steps in \texttt{renegcl} before step 5 are
technical stuff that makes this magic work, and they can be ignored
for a quick overview of the proof. To continue following the ``important''
part of the proof of \texttt{renegcl},
you can look at the reference to \texttt{renegcli} at step 5.

That said, let's briefly look at how
\texttt{renegcl} uses the
Weak Deduction Theorem (\texttt{dedth}) to do its job,
in case you want to do something similar or want understand it more deeply.
Let's work backwards in the proof of \texttt{renegcl}.
Step 6 applies \texttt{dedth} to produce our goal result
$ \vdash ( A \in \mathbb{R} \rightarrow\, - A \in \mathbb{R} )$.
This requires on the one hand the (substituted) deduction
\texttt{renegcli} in step 5.
By itself \texttt{renegcli} proves the deduction
$ \vdash A \in \mathbb{R} \Rightarrow\, \vdash - A \in \mathbb{R}$;
this is the deduction form we are trying to turn into theorem form,
and thus
\texttt{renegcli} has a separate hypothesis that must be fulfilled.
To fulfill the hypothesis of the invocation of
\texttt{renegcli} in step 5, it is eventually
reduced to the already proven theorem $1 \in \mathbb{R}$ in step 3.
Step 4 connects steps 3 and 5; step 4 invokes
\texttt{elimel}, a special case of \texttt{elimhyp} that eliminates
a membership hypothesis for the weak deduction theorem.
On the other hand, the equivalence of the conclusion of
\texttt{renegcl}
$( - A \in \mathbb{R} )$ and the substituted conclusion of
\texttt{renegcli} must be proven, which is done in steps 2 and 1.

The weak deduction theorem has limitations.
In particular, we must be able to prove a special case of the deduction's
hypothesis as a stand-alone theorem.
For example, we used $1 \in \mathbb{R}$ in step 3 of \texttt{renegcl}.

We used to use the weak deduction theorem
extensively within \texttt{set.mm}.
However, we now recommend applying ``deduction style''
instead in most cases, as deduction style is
often an easier and clearer approach.
Therefore, we will now describe deduction style.

\subsection{Deduction Style}\label{deductionstyle}

We now prefer to write assertions in ``deduction form''
instead of writing a proof that would require use of the standard or
weak deduction theorem.
We call this appraoch
``deduction style.''\index{deduction style}

It will be easier to explain this by first defining some terms:

\begin{itemize}
\item \textbf{closed form}\index{closed form}\index{forms!closed}:
A kind of assertion (theorem) with no hypotheses.
Typically its label has no special suffix.
An example is \texttt{unss}, which states:
$\vdash ( ( A \subseteq C \wedge B \subseteq C ) \leftrightarrow ( A \cup B )
\subseteq C )\label{eq:unss}$
\item \textbf{deduction form}\index{deduction form}\index{forms!deduction}:
A kind of assertion with one or more hypotheses
where the conclusion is an implication with
a wff variable as the antecedent (usually $\varphi$), and every hypothesis
(\$e statement)
is either (1) an implication with the same antecedent as the conclusion or
(2) a definition.
A definition
can be for a class variable (this is a class variable followed by ``='')
or a wff variable (this is a wff variable followed by $\leftrightarrow$);
class variable definitions are more common.
In practice, a proof
in deduction form will also contain many steps that are implications
where the antecedent is either that wff variable (normally $\varphi$)
or is
a conjunction (...$\land$...) including that wff variable ($\varphi$).
If an assertion is in deduction form, and other forms are also available,
then we suffix its label with ``d.''
An example is \texttt{unssd}, which states\footnote{
For brevity we show here (and in other places)
a $\&$\index{$\&$} between hypotheses\index{hypotheses}
and a $\Rightarrow$\index{$\Rightarrow$}\index{conclusion}
between the hypotheses and the conclusion.
This notation is technically not part of the Metamath language, but is
instead a convenient abbreviation to show both the hypotheses and conclusion.}:
$\vdash ( \varphi \rightarrow A \subseteq C )\quad\&\quad \vdash ( \varphi
    \rightarrow B \subseteq C )\quad\Rightarrow\quad \vdash ( \varphi
    \rightarrow ( A \cup B ) \subseteq C )\label{eq:unssd}$
\item \textbf{inference form}\index{inference form}\index{forms!inference}:
A kind of assertion with one or more hypotheses that is not in deduction form
(e.g., there is no common antecedent).
If an assertion is in inference form, and other forms are also available,
then we suffix its label with ``i.''
An example is \texttt{unssi}, which states:
$\vdash A \subseteq C\quad\&\quad \vdash B \subseteq C\quad\Rightarrow\quad
    \vdash ( A \cup B ) \subseteq C\label{eq:unssi}$
\end{itemize}

When using deduction style we express an assertion in deduction form.
This form prefixes each hypothesis (other than definitions) and the
conclusion with a universal antecedent (``$\varphi \rightarrow$'').
The antecedent (e.g., $\varphi$)
mimics the context handled in the deduction theorem, eliminating
the need to directly use the deduction theorem.

Once you have an assertion in deduction form, you can easily convert it
to inference form or closed form:

\begin{itemize}
\item To
prove some assertion Ti in inference form, given assertion Td in deduction
form, there is a simple mechanical process you can use. First take each
Ti hypothesis and insert a \texttt{T.} $\rightarrow$ prefix (``true implies'')
using \texttt{a1i}. You
can then use the existing assertion Td to prove the resulting conclusion
with a \texttt{T.} $\rightarrow$ prefix.
Finally, you can remove that prefix using \texttt{mptru},
resulting in the conclusion you wanted to prove.
\item To
prove some assertion T in closed form, given assertion Td in deduction
form, there is another simple mechanical process you can use. First,
select an expression that is the conjunction (...$\land$...) of all of the
consequents of every hypothesis of Td. Next, prove that this expression
implies each of the separate hypotheses of Td in turn by eliminating
conjuncts (there are a variety of proven assertions to do this, including
\texttt{simpl},
\texttt{simpr},
\texttt{3simpa},
\texttt{3simpb},
\texttt{3simpc},
\texttt{simp1},
\texttt{simp2},
and
\texttt{simp3}).
If the
expression has nested conjunctions, inner conjuncts can be broken out by
chaining the above theorems with \texttt{syl}
(see section \ref{syl}).\footnote{
There are actually many theorems
(labeled simp* such as \texttt{simp333}) that break out inner conjuncts in one
step, but rather than learning them you can just use the chaining we
just described to prove them, and then let the Metamath program command
\texttt{minimize{\char`\_}with}\index{\texttt{minimize{\char`\_}with} command}
figure out the right ones needed to collapse them.}
As your final step, you can then apply the already-proven assertion Td
(which is in deduction form), proving assertion T in closed form.
\end{itemize}

We can also easily convert any assertion T in closed form to its related
assertion Ti in inference form by applying
modus ponens\index{modus ponens} (see section \ref{axmp}).

The deduction form antecedent can also be used to represent the context
necessary to support natural deduction systems, so we will now
discuss natural deduction.

\subsection{Natural Deduction}\label{naturaldeduction}

Natural deduction\index{natural deduction}
(ND) systems, as such, were originally introduced in
1934 by two logicians working independently: Ja\'skowski and Gentzen. ND
systems are supposed to reconstruct, in a formally proper way, traditional
ways of mathematical reasoning (such as conditional proof, indirect proof,
and proof by cases). As reconstructions they were naturally influenced
by previous work, and many specific ND systems and notations have been
developed since their original work.

There are many ND variants, but
Indrzejczak \cite[p.~31-32]{Indrzejczak}\index{Indrzejczak, Andrzej}
suggests that any natural deductive system must satisfy at
least these three criteria:

\begin{itemize}
\item ``There are some means for entering assumptions into a proof and
also for eliminating them. Usually it requires some bookkeeping devices
for indicating the scope of an assumption, and showing that a part of
a proof depending on eliminated assumption is discharged.
\item There are no (or, at least, a very limited set of) axioms, because
their role is taken over by the set of primitive rules for introduction
and elimination of logical constants which means that elementary
inferences instead of formulae are taken as primitive.
\item (A genuine) ND system admits a lot of freedom in proof construction
and possibility of applying several proof search strategies, like
conditional proof, proof by cases, proof by reductio ad absurdum etc.''
\end{itemize}

The Metamath Proof Explorer (MPE) as defined in \texttt{set.mm}
is fundamentally a Hilbert-style system.
That is, MPE is based on a larger number of axioms (compared
to natural deduction systems), a very small set of rules of inference
(modus ponens), and the context is not changed by the rules of inference
in the middle of a proof. That said, MPE proofs can be developed using
the natural deduction (ND) approach as originally developed by Ja\'skowski
and Gentzen.

The most common and recommended approach for applying ND in MPE is to use
deduction form\index{deduction form}%
\index{forms!deduction}
and apply the MPE proven assertions that are equivalent to ND rules.
For example, MPE's \texttt{jca} is equivalent to ND rule $\land$-I
(and-insertion).
We maintain a list of equivalences that you may consult.
This approach for applying an ND approach within MPE relies on Metamath's
wff metavariables in an essential way, and is described in more detail
in the presentation ``Natural Deductions in the Metamath Proof Language''
by Mario Carneiro \cite{CarneiroND}\index{Carneiro, Mario}.

In this style many steps are an implication, whose antecedent mimics
the context ($\Gamma$) of most ND systems. To add an assumption, simply add
it to the implication antecedent (typically using
\texttt{simpr}),
and use that
new antecedent for all later claims in the same scope. If you wish to
use an assertion in an ND hypothesis scope that is outside the current
ND hypothesis scope, modify the assertion so that the ND hypothesis
assumption is added to its antecedent (typically using \texttt{adantr}). Most
proof steps will be proved using rules that have hypotheses and results
of the form $\varphi \rightarrow$ ...

An example may make this clearer.
Let's show theorem 5.5 of
\cite[p.~18]{Clemente}\index{Clemente Laboreo, Daniel}
along with a line by line translation using the usual
translation of natural deduction (ND) in the Metamath Proof Explorer
(MPE) notation (this is proof \texttt{ex-natded5.5}).
The proof's original goal was to prove
$\lnot \psi$ given two hypotheses,
$( \psi \rightarrow \chi )$ and $ \lnot \chi$.
We will translate these statements into MPE deduction form
by prefixing them all with $\varphi \rightarrow$.
As a result, in MPE the goal is stated as
$( \varphi \rightarrow \lnot \psi )$, and the two hypotheses are stated as
$( \varphi \rightarrow ( \psi \rightarrow \chi ) )$ and
$( \varphi \rightarrow \lnot \chi )$.

The following table shows the proof in Fitch natural deduction style
and its MPE equivalent.
The \textit{\#} column shows the original numbering,
\textit{MPE\#} shows the number in the equivalent MPE proof
(which we will show later),
\textit{ND Expression} shows the original proof claim in ND notation,
and \textit{MPE Translation} shows its translation into MPE
as discussed in this section.
The final columns show the rationale in ND and MPE respectively.

\needspace{4\baselineskip}
{\setlength{\extrarowsep}{4pt} % Keep rows from being too close together
\begin{longtabu}   { @{} c c X X X X }
\textbf{\#} & \textbf{MPE\#} & \textbf{ND Ex\-pres\-sion} &
\textbf{MPE Trans\-lation} & \textbf{ND Ration\-ale} &
\textbf{MPE Ra\-tio\-nale} \\
\endhead

1 & 2;3 &
$( \psi \rightarrow \chi )$ &
$( \varphi \rightarrow ( \psi \rightarrow \chi ) )$ &
Given &
\$e; \texttt{adantr} to put in ND hypothesis \\

2 & 5 &
$ \lnot \chi$ &
$( \varphi \rightarrow \lnot \chi )$ &
Given &
\$e; \texttt{adantr} to put in ND hypothesis \\

3 & 1 &
... $\vert$ $\psi$ &
$( \varphi \rightarrow \psi )$ &
ND hypothesis assumption &
\texttt{simpr} \\

4 & 4 &
... $\chi$ &
$( ( \varphi \land \psi ) \rightarrow \chi )$ &
$\rightarrow$\,E 1,3 &
\texttt{mpd} 1,3 \\

5 & 6 &
... $\lnot \chi$ &
$( ( \varphi \land \psi ) \rightarrow \lnot \chi )$ &
IT 2 &
\texttt{adantr} 5 \\

6 & 7 &
$\lnot \psi$ &
$( \varphi \rightarrow \lnot \psi )$ &
$\land$\,I 3,4,5 &
\texttt{pm2.65da} 4,6 \\

\end{longtabu}
}


The original used Latin letters; we have replaced them with Greek letters
to follow Metamath naming conventions and so that it is easier to follow
the Metamath translation. The Metamath line-for-line translation of
this natural deduction approach precedes every line with an antecedent
including $\varphi$ and uses the Metamath equivalents of the natural deduction
rules. To add an assumption, the antecedent is modified to include it
(typically by using \texttt{adantr};
\texttt{simpr} is useful when you want to
depend directly on the new assumption, as is shown here).

In Metamath we can represent the two given statements as these hypotheses:

\needspace{2\baselineskip}
\begin{itemize}
\item ex-natded5.5.1 $\vdash ( \varphi \rightarrow ( \psi \rightarrow \chi ) )$
\item ex-natded5.5.2 $\vdash ( \varphi \rightarrow \lnot \chi )$
\end{itemize}

\needspace{4\baselineskip}
Here is the proof in Metamath as a line-by-line translation:

\begin{longtabu}   { l l l X }
\textbf{Step} & \textbf{Hyp} & \textbf{Ref} & \textbf{Ex\-pres\-sion} \\
\endhead
1 & & simpr & $\vdash ( ( \varphi \land \psi ) \rightarrow \psi )$ \\
2 & & ex-natded5.5.1 &
  $\vdash ( \varphi \rightarrow ( \psi \rightarrow \chi ) )$ \\
3 & 2 & adantr &
 $\vdash ( ( \varphi \land \psi ) \rightarrow ( \psi \rightarrow \chi ) )$ \\
4 & 1, 3 & mpd &
 $\vdash ( ( \varphi \land \psi ) \rightarrow \chi ) $ \\
5 & & ex-natded5.5.2 &
 $\vdash ( \varphi \rightarrow \lnot \chi )$ \\
6 & 5 & adantr &
 $\vdash ( ( \varphi \land \psi ) \rightarrow \lnot \chi )$ \\
7 & 4, 6 & pm2.65da &
 $\vdash ( \varphi \rightarrow \lnot \psi )$ \\
\end{longtabu}

Only using specific natural deduction rules directly can lead to very
long proofs, for exactly the same reason that only using axioms directly
in Hilbert-style proofs can lead to very long proofs.
If the goal is short and clear proofs,
then it is better to reuse already-proven assertions
in deduction form than to start from scratch each time
and using only basic natural deduction rules.

\subsection{Strengths of Our Approach}

As far as we know there is nothing else in the literature like either the
weak deduction theorem or Mario Carneiro\index{Carneiro, Mario}'s
natural deduction method.
In order to
transform a hypothesis into an antecedent, the literature's standard
``Deduction Theorem''\index{Deduction Theorem}\index{Standard Deduction Theorem}
requires metalogic outside of the notions provided
by the axiom system. We instead generally prefer to use Mario Carneiro's
natural deduction method, then use the weak deduction theorem in cases
where that is difficult to apply, and only then use the full standard
deduction theorem as a last resort.

The weak deduction theorem\index{Weak Deduction Theorem}
does not require any additional metalogic
but converts an inference directly into a closed form theorem, with
a rigorous proof that uses only the axiom system. Unlike the standard
Deduction Theorem, there is no implicit external justification that we
have to trust in order to use it.

Mario Carneiro's natural deduction\index{natural deduction}
method also does not require any new metalogical
notions. It avoids the Deduction Theorem's metalogic by prefixing the
hypotheses and conclusion of every would-be inference with a universal
antecedent (``$\varphi \rightarrow$'') from the very start.

We think it is impressive and satisfying that we can do so much in a
practical sense without stepping outside of our Hilbert-style axiom system.
Of course our axiomatization, which is in the form of schemes,
contains a metalogic of its own that we exploit. But this metalogic
is relatively simple, and for our Deduction Theorem alternatives,
we primarily use just the direct substitution of expressions for
metavariables.

\begin{sloppy}
\section{Exploring the Set The\-o\-ry Data\-base}\label{exploring}
\end{sloppy}
% NOTE: All examples performed in this section are
% recorded wtih "set width 61" % on set.mm as of 2019-05-28
% commit c1e7849557661260f77cfdf0f97ac4354fbb4f4d.

At this point you may wish to study the \texttt{set.mm}\index{set theory
database (\texttt{set.mm})} file in more detail.  Pay particular
attention to the assumptions needed to define wffs\index{well-formed
formula (wff)} (which are not included above), the variable types
(\texttt{\$f}\index{\texttt{\$f} statement} statements), and the
definitions that are introduced.  Start with some simple theorems in
propositional calculus, making sure you understand in detail each step
of a proof.  Once you get past the first few proofs and become familiar
with the Metamath language, any part of the \texttt{set.mm} database
will be as easy to follow, step by step, as any other part---you won't
have to undergo a ``quantum leap'' in mathematical sophistication to be
able to follow a deep proof in set theory.

Next, you may want to explore how concepts such as natural numbers are
defined and described.  This is probably best done in conjunction with
standard set theory textbooks, which can help give you a higher-level
understanding.  The \texttt{set.mm} database provides references that will get
you started.  From there, you will be on your way towards a very deep,
rigorous understanding of abstract mathematics.

The Metamath\index{Metamath} program can help you peruse a Metamath data\-base,
wheth\-er you are trying to figure out how a certain step follows in a proof or
just have a general curiosity.  We will go through some examples of the
commands, using the \texttt{set.mm}\index{set theory database (\texttt{set.mm})}
database provided with the Metamath software.  These should help get you
started.  See Chapter~\ref{commands} for a more detailed description of
the commands.  Note that we have included the full spelling of all commands to
prevent ambiguity with future commands.  In practice you may type just the
characters needed to specify each command keyword\index{command keyword}
unambiguously, often just one or two characters per keyword, and you don't
need to type them in upper case.

First run the Metamath program as described earlier.  You should see the
\verb/MM>/ prompt.  Read in the \texttt{set.mm} file:\index{\texttt{read}
command}

\begin{verbatim}
MM> read set.mm
Reading source file "set.mm"... 34554442 bytes
34554442 bytes were read into the source buffer.
The source has 155711 statements; 2254 are $a and 32250 are $p.
No errors were found.  However, proofs were not checked.
Type VERIFY PROOF * if you want to check them.
\end{verbatim}

As with most examples in this book, what you will see
will be slightly different because we are continuously
improving our databases (including \texttt{set.mm}).

Let's check the database integrity.  This may take a minute or two to run if
your computer is slow.

\begin{verbatim}
MM> verify proof *
0 10%  20%  30%  40%  50%  60%  70%  80%  90% 100%
..................................................
All proofs in the database were verified in 2.84 s.
\end{verbatim}

No errors were reported, so every proof is correct.

You need to know the names (labels) of theorems before you can look at them.
Often just examining the database file(s) with a text editor is the best
approach.  In \texttt{set.mm} there are many detailed comments, especially near
the beginning, that can help guide you. The \texttt{search} command in the
Metamath program is also handy.  The \texttt{comments} qualifier will list the
statements whose associated comment (the one immediately before it) contain a
string you give it.  For example, if you are studying Enderton's {\em Elements
of Set Theory} \cite{Enderton}\index{Enderton, Herbert B.} you may want to see
the references to it in the database.  The search string \texttt{enderton} is not
case sensitive.  (This will not show you all the database theorems that are in
Enderton's book because there is usually only one citation for a given
theorem, which may appear in several textbooks.)\index{\texttt{search}
command}

\begin{verbatim}
MM> search * "enderton" / comments
12067 unineq $p "... Exercise 20 of [Enderton] p. 32 and ..."
12459 undif2 $p "...Corollary 6K of [Enderton] p. 144. (C..."
12953 df-tp $a "...s. Definition of [Enderton] p. 19. (Co..."
13689 unissb $p ".... Exercise 5 of [Enderton] p. 26 and ..."
\end{verbatim}
\begin{center}
(etc.)
\end{center}

Or you may want to see what theorems have something to do with
conjunction (logical {\sc and}).  The quotes around the search
string are optional when there's no ambiguity.\index{\texttt{search}
command}

\begin{verbatim}
MM> search * conjunction / comments
120 a1d $p "...be replaced with a conjunction ( ~ df-an )..."
662 df-bi $a "...viated form after conjunction is introdu..."
1319 wa $a "...ff definition to include conjunction ('and')."
1321 df-an $a "Define conjunction (logical 'and'). Defini..."
1420 imnan $p "...tion in terms of conjunction. (Contribu..."
\end{verbatim}
\begin{center}
(etc.)
\end{center}


Now we will start to look at some details.  Let's look at the first
axiom of propositional calculus
(we could use \texttt{sh st} to abbreviate
\texttt{show statement}).\index{\texttt{show statement} command}

\begin{verbatim}
MM> show statement ax-1/full
Statement 19 is located on line 881 of the file "set.mm".
"Axiom _Simp_.  Axiom A1 of [Margaris] p. 49.  One of the 3
axioms of propositional calculus.  The 3 axioms are also
        ...
19 ax-1 $a |- ( ph -> ( ps -> ph ) ) $.
Its mandatory hypotheses in RPN order are:
  wph $f wff ph $.
  wps $f wff ps $.
The statement and its hypotheses require the variables:  ph
      ps
The variables it contains are:  ph ps


Statement 49 is located on line 11182 of the file "set.mm".
Its statement number for HTML pages is 6.
"Axiom _Simp_.  Axiom A1 of [Margaris] p. 49.  One of the 3
axioms of propositional calculus.  The 3 axioms are also
given as Definition 2.1 of [Hamilton] p. 28.
...
49 ax-1 $a |- ( ph -> ( ps -> ph ) ) $.
Its mandatory hypotheses in RPN order are:
  wph $f wff ph $.
  wps $f wff ps $.
The statement and its hypotheses require the variables:
  ph ps
The variables it contains are:  ph ps
\end{verbatim}

Compare this to \texttt{ax-1} on p.~\pageref{ax1}.  You can see that the
symbol \texttt{ph} is the {\sc ascii} notation for $\varphi$, etc.  To
see the mathematical symbols for any expression you may typeset it in
\LaTeX\ (type \texttt{help tex} for instructions)\index{latex@{\LaTeX}}
or, easier, just use a text editor to look at the comments where symbols
are first introduced in \texttt{set.mm}.  The hypotheses \texttt{wph}
and \texttt{wps} required by \texttt{ax-1} mean that variables
\texttt{ph} and \texttt{ps} must be wffs.

Next we'll pick a simple theorem of propositional calculus, the Principle of
Identity, which is proved directly from the axioms.  We'll look at the
statement then its proof.\index{\texttt{show statement}
command}

\begin{verbatim}
MM> show statement id1/full
Statement 116 is located on line 11371 of the file "set.mm".
Its statement number for HTML pages is 22.
"Principle of identity.  Theorem *2.08 of [WhiteheadRussell]
p. 101.  This version is proved directly from the axioms for
demonstration purposes.
...
116 id1 $p |- ( ph -> ph ) $= ... $.
Its mandatory hypotheses in RPN order are:
  wph $f wff ph $.
Its optional hypotheses are:  wps wch wth wta wet
      wze wsi wrh wmu wla wka
The statement and its hypotheses require the variables:  ph
These additional variables are allowed in its proof:
      ps ch th ta et ze si rh mu la ka
The variables it contains are:  ph
\end{verbatim}

The optional variables\index{optional variable} \texttt{ps}, \texttt{ch}, etc.\ are
available for use in a proof of this statement if we wish, and were we to do
so we would make use of optional hypotheses \texttt{wps}, \texttt{wch}, etc.  (See
Section~\ref{dollaref} for the meaning of ``optional
hypothesis.''\index{optional hypothesis}) The reason these show up in the
statement display is that statement \texttt{id1} happens to be in their scope
(see Section~\ref{scoping} for the definition of ``scope''\index{scope}), but
in fact in propositional calculus we will never make use of optional
hypotheses or variables.  This becomes important after quantifiers are
introduced, where ``dummy'' variables\index{dummy variable} are often needed
in the middle of a proof.

Let's look at the proof of statement \texttt{id1}.  We'll use the
\texttt{show proof} command, which by default suppresses the
``non-essential'' steps that construct the wffs.\index{\texttt{show proof}
command}
We will display the proof in ``lemmon' format (a non-indented format
with explicit previous step number references) and renumber the
displayed steps:

\begin{verbatim}
MM> show proof id1 /lemmon/renumber
1 ax-1           $a |- ( ph -> ( ph -> ph ) )
2 ax-1           $a |- ( ph -> ( ( ph -> ph ) -> ph ) )
3 ax-2           $a |- ( ( ph -> ( ( ph -> ph ) -> ph ) ) ->
                     ( ( ph -> ( ph -> ph ) ) -> ( ph -> ph )
                                                          ) )
4 2,3 ax-mp      $a |- ( ( ph -> ( ph -> ph ) ) -> ( ph -> ph
                                                          ) )
5 1,4 ax-mp      $a |- ( ph -> ph )
\end{verbatim}

If you have read Section~\ref{trialrun}, you'll know how to interpret this
proof.  Step~2, for example, is an application of axiom \texttt{ax-1}.  This
proof is identical to the one in Hamilton's {\em Logic for Mathematicians}
\cite[p.~32]{Hamilton}\index{Hamilton, Alan G.}.

You may want to look at what
substitutions\index{substitution!variable}\index{variable substitution} are
made into \texttt{ax-1} to arrive at step~2.  The command to do this needs to
know the ``real'' step number, so we'll display the proof again without
the \texttt{renumber} qualifier.\index{\texttt{show proof}
command}

\begin{verbatim}
MM> show proof id1 /lemmon
 9 ax-1          $a |- ( ph -> ( ph -> ph ) )
20 ax-1          $a |- ( ph -> ( ( ph -> ph ) -> ph ) )
24 ax-2          $a |- ( ( ph -> ( ( ph -> ph ) -> ph ) ) ->
                     ( ( ph -> ( ph -> ph ) ) -> ( ph -> ph )
                                                          ) )
25 20,24 ax-mp   $a |- ( ( ph -> ( ph -> ph ) ) -> ( ph -> ph
                                                          ) )
26 9,25 ax-mp    $a |- ( ph -> ph )
\end{verbatim}

The ``real'' step number is 20.  Let's look at its details.

\begin{verbatim}
MM> show proof id1 /detailed_step 20
Proof step 20:  min=ax-1 $a |- ( ph -> ( ( ph -> ph ) -> ph )
  )
This step assigns source "ax-1" ($a) to target "min" ($e).
The source assertion requires the hypotheses "wph" ($f, step
18) and "wps" ($f, step 19).  The parent assertion of the
target hypothesis is "ax-mp" ($a, step 25).
The source assertion before substitution was:
    ax-1 $a |- ( ph -> ( ps -> ph ) )
The following substitutions were made to the source
assertion:
    Variable  Substituted with
     ph        ph
     ps        ( ph -> ph )
The target hypothesis before substitution was:
    min $e |- ph
The following substitution was made to the target hypothesis:
    Variable  Substituted with
     ph        ( ph -> ( ( ph -> ph ) -> ph ) )
\end{verbatim}

This shows the substitutions\index{substitution!variable}\index{variable
substitution} made to the variables in \texttt{ax-1}.  References are made to
steps 18 and 19 which are not shown in our proof display.  To see these steps,
you can display the proof with the \texttt{all} qualifier.

Let's look at a slightly more advanced proof of propositional calculus.  Note
that \verb+/\+ is the symbol for $\wedge$ (logical {\sc and}, also
called conjunction).\index{conjunction ($\wedge$)}
\index{logical {\sc and} ($\wedge$)}

\begin{verbatim}
MM> show statement prth/full
Statement 1791 is located on line 15503 of the file "set.mm".
Its statement number for HTML pages is 559.
"Conjoin antecedents and consequents of two premises.  This
is the closed theorem form of ~ anim12d .  Theorem *3.47 of
[WhiteheadRussell] p. 113.  It was proved by Leibniz,
and it evidently pleased him enough to call it
_praeclarum theorema_ (splendid theorem).
...
1791 prth $p |- ( ( ( ph -> ps ) /\ ( ch -> th ) ) -> ( ( ph
      /\ ch ) -> ( ps /\ th ) ) ) $= ... $.
Its mandatory hypotheses in RPN order are:
  wph $f wff ph $.
  wps $f wff ps $.
  wch $f wff ch $.
  wth $f wff th $.
Its optional hypotheses are:  wta wet wze wsi wrh wmu wla wka
The statement and its hypotheses require the variables:  ph
      ps ch th
These additional variables are allowed in its proof:  ta et
      ze si rh mu la ka
The variables it contains are:  ph ps ch th


MM> show proof prth /lemmon/renumber
1 simpl          $p |- ( ( ( ph -> ps ) /\ ( ch -> th ) ) ->
                                               ( ph -> ps ) )
2 simpr          $p |- ( ( ( ph -> ps ) /\ ( ch -> th ) ) ->
                                               ( ch -> th ) )
3 1,2 anim12d    $p |- ( ( ( ph -> ps ) /\ ( ch -> th ) ) ->
                           ( ( ph /\ ch ) -> ( ps /\ th ) ) )
\end{verbatim}

There are references to a lot of unfamiliar statements.  To see what they are,
you may type the following:

\begin{verbatim}
MM> show proof prth /statement_summary
Summary of statements used in the proof of "prth":

Statement simpl is located on line 14748 of the file
"set.mm".
"Elimination of a conjunct.  Theorem *3.26 (Simp) of
[WhiteheadRussell] p. 112. ..."
  simpl $p |- ( ( ph /\ ps ) -> ph ) $= ... $.

Statement simpr is located on line 14777 of the file
"set.mm".
"Elimination of a conjunct.  Theorem *3.27 (Simp) of
[WhiteheadRussell] ..."
  simpr $p |- ( ( ph /\ ps ) -> ps ) $= ... $.

Statement anim12d is located on line 15445 of the file
"set.mm".
"Conjoin antecedents and consequents in a deduction.
..."
  anim12d.1 $e |- ( ph -> ( ps -> ch ) ) $.
  anim12d.2 $e |- ( ph -> ( th -> ta ) ) $.
  anim12d $p |- ( ph -> ( ( ps /\ th ) -> ( ch /\ ta ) ) )
      $= ... $.
\end{verbatim}
\begin{center}
(etc.)
\end{center}

Of course you can look at each of these statements and their proofs, and
so on, back to the axioms of propositional calculus if you wish.

The \texttt{search} command is useful for finding statements when you
know all or part of their contents.  The following example finds all
statements containing \verb@ph -> ps@ followed by \verb@ch -> th@.  The
\verb@$*@ is a wildcard that matches anything; the \texttt{\$} before the
\verb$*$ prevents conflicts with math symbol token names.  The \verb@*@ after
\texttt{SEARCH} is also a wildcard that in this case means ``match any label.''
\index{\texttt{search} command}

% I'm omitting this one, since readers are unlikely to see it:
% 1096 bisymOLD $p |- ( ( ( ph -> ps ) -> ( ch -> th ) ) -> ( (
%   ( ps -> ph ) -> ( th -> ch ) ) -> ( ( ph <-> ps ) -> ( ch
%    <-> th ) ) ) )
\begin{verbatim}
MM> search * "ph -> ps $* ch -> th"
1791 prth $p |- ( ( ( ph -> ps ) /\ ( ch -> th ) ) -> ( ( ph
    /\ ch ) -> ( ps /\ th ) ) )
2455 pm3.48 $p |- ( ( ( ph -> ps ) /\ ( ch -> th ) ) -> ( (
    ph \/ ch ) -> ( ps \/ th ) ) )
117859 pm11.71 $p |- ( ( E. x ph /\ E. y ch ) -> ( ( A. x (
    ph -> ps ) /\ A. y ( ch -> th ) ) <-> A. x A. y ( ( ph /\
    ch ) -> ( ps /\ th ) ) ) )
\end{verbatim}

Three statements, \texttt{prth}, \texttt{pm3.48},
 and \texttt{pm11.71}, were found to match.

To see what axioms\index{axiom} and definitions\index{definition}
\texttt{prth} ultimately depends on for its proof, you can have the
program backtrack through the hierarchy\index{hierarchy} of theorems and
definitions.\index{\texttt{show trace{\char`\_}back} command}

\begin{verbatim}
MM> show trace_back prth /essential/axioms
Statement "prth" assumes the following axioms ($a
statements):
  ax-1 ax-2 ax-3 ax-mp df-bi df-an
\end{verbatim}

Note that the 3 axioms of propositional calculus and the modus ponens rule are
needed (as expected); in addition, there are a couple of definitions that are used
along the way.  Note that Metamath makes no distinction\index{axiom vs.\
definition} between axioms\index{axiom} and definitions\index{definition}.  In
\texttt{set.mm} they have been distinguished artificially by prefixing their
labels\index{labels in \texttt{set.mm}} with \texttt{ax-} and \texttt{df-}
respectively.  For example, \texttt{df-an} defines conjunction (logical {\sc
and}), which is represented by the symbol \verb+/\+.
Section~\ref{definitions} discusses the philosophy of definitions, and the
Metamath language takes a particularly simple, conservative approach by using
the \texttt{\$a}\index{\texttt{\$a} statement} statement for both axioms and
definitions.

You can also have the program compute how many steps a proof
has\index{proof length} if we were to follow it all the way back to
\texttt{\$a} statements.

\begin{verbatim}
MM> show trace_back prth /essential/count_steps
The statement's actual proof has 3 steps.  Backtracking, a
total of 79 different subtheorems are used.  The statement
and subtheorems have a total of 274 actual steps.  If
subtheorems used only once were eliminated, there would be a
total of 38 subtheorems, and the statement and subtheorems
would have a total of 185 steps.  The proof would have 28349
steps if fully expanded back to axiom references.  The
maximum path length is 38.  A longest path is:  prth <-
anim12d <- syl2and <- sylan2d <- ancomsd <- ancom <- pm3.22
<- pm3.21 <- pm3.2 <- ex <- sylbir <- biimpri <- bicomi <-
bicom1 <- bi2 <- dfbi1 <- impbii <- bi3 <- simprim <- impi <-
con1i <- nsyl2 <- mt3d <- con1d <- notnot1 <- con2i <- nsyl3
<- mt2d <- con2d <- notnot2 <- pm2.18d <- pm2.18 <- pm2.21 <-
pm2.21d <- a1d <- syl <- mpd <- a2i <- a2i.1 .
\end{verbatim}

This tells us that we would have to inspect 274 steps if we want to
verify the proof completely starting from the axioms.  A few more
statistics are also shown.  There are one or more paths back to axioms
that are the longest; this command ferrets out one of them and shows it
to you.  There may be a sense in which the longest path length is
related to how ``deep'' the theorem is.

We might also be curious about what proofs depend on the theorem
\texttt{prth}.  If it is never used later on, we could eliminate it as
redundant if it has no intrinsic interest by itself.\index{\texttt{show
usage} command}

% I decided to show the OLD values here.
\begin{verbatim}
MM> show usage prth
Statement "prth" is directly referenced in the proofs of 18
statements:
  mo3 moOLD 2mo 2moOLD euind reuind reuss2 reusv3i opelopabt
  wemaplem2 rexanre rlimcn2 o1of2 o1rlimmul 2sqlem6 spanuni
  heicant pm11.71
\end{verbatim}

Thus \texttt{prth} is directly used by 18 proofs.
We can use the \texttt{/recursive} qualifier to include indirect use:

\begin{verbatim}
MM> show usage prth /recursive
Statement "prth" directly or indirectly affects the proofs of
24214 statements:
  mo3 mo mo3OLD eu2 moOLD eu2OLD eu3OLD mo4f mo4 eu4 mopick
...
\end{verbatim}

\subsection{A Note on the ``Compact'' Proof Format}

The Metamath program will display proofs in a ``compact''\index{compact proof}
format whenever the proof is stored in compressed format in the database.  It
may be be slightly confusing unless you know how to interpret it.
For example,
if you display the complete proof of theorem \texttt{id1} it will start
off as follows:

\begin{verbatim}
MM> show proof id1 /lemmon/all
 1 wph           $f wff ph
 2 wph           $f wff ph
 3 wph           $f wff ph
 4 2,3 wi    @4: $a wff ( ph -> ph )
 5 1,4 wi    @5: $a wff ( ph -> ( ph -> ph ) )
 6 @4            $a wff ( ph -> ph )
\end{verbatim}

\begin{center}
{etc.}
\end{center}

Step 4 has a ``local label,''\index{local label} \texttt{@4}, assigned to it.
Later on, at step 6, the label \texttt{@1} is referenced instead of
displaying the explicit proof for that step.  This technique takes advantage
of the fact that steps in a proof often repeat, especially during the
construction of wffs.  The compact format reduces the number of steps in the
proof display and may be preferred by some people.

If you want to see the normal format with the ``true'' step numbers, you can
use the following workaround:\index{\texttt{save proof} command}

\begin{verbatim}
MM> save proof id1 /normal
The proof of "id1" has been reformatted and saved internally.
Remember to use WRITE SOURCE to save it permanently.
MM> show proof id1 /lemmon/all
 1 wph           $f wff ph
 2 wph           $f wff ph
 3 wph           $f wff ph
 4 2,3 wi        $a wff ( ph -> ph )
 5 1,4 wi        $a wff ( ph -> ( ph -> ph ) )
 6 wph           $f wff ph
 7 wph           $f wff ph
 8 6,7 wi        $a wff ( ph -> ph )
\end{verbatim}

\begin{center}
{etc.}
\end{center}

Note that the original 6 steps are now 8 steps.  However, the format is
now the same as that described in Chapter~\ref{using}.

\chapter{The Metamath Language}
\label{languagespec}

\begin{quote}
  {\em Thus mathematics may be defined as the subject in which we never know
what we are talking about, nor whether what we are saying is true.}
    \flushright\sc  Bertrand Russell\footnote{\cite[p.~84]{Russell2}.}\\
\end{quote}\index{Russell, Bertrand}

Probably the most striking feature of the Metamath language is its almost
complete absence of hard-wired syntax. Metamath\index{Metamath} does not
understand any mathematics or logic other than that needed to construct finite
sequences of symbols according to a small set of simple, built-in rules.  The
only rule it uses in a proof is the substitution of an expression (symbol
sequence) for a variable, subject to a simple constraint to prevent
bound-variable clashes.  The primitive notions built into Metamath involve the
simple manipulation of finite objects (symbols) that we as humans can easily
visualize and that computers can easily deal with.  They seem to be just
about the simplest notions possible that are required to do standard
mathematics.

This chapter serves as a reference manual for the Metamath\index{Metamath}
language. It covers the tedious technical details of the language, some of
which you may wish to skim in a first reading.  On the other hand, you should
pay close attention to the defined terms in {\bf boldface}; they have precise
meanings that are important to keep in mind for later understanding.  It may
be best to first become familiar with the examples in Chapter~\ref{using} to
gain some motivation for the language.

%% Uncomment this when uncommenting section {formalspec} below
If you have some knowledge of set theory, you may wish to study this
chapter in conjunction with the formal set-theoretical description of the
Metamath language in Appendix~\ref{formalspec}.

We will use the name ``Metamath''\index{Metamath} to mean either the Metamath
computer language or the Metamath software associated with the computer
language.  We will not distinguish these two when the context is clear.

The next section contains the complete specification of the Metamath
language.
It serves as an
authoritative reference and presents the syntax in enough detail to
write a parser\index{parsing Metamath} and proof verifier.  The
specification is terse and it is probably hard to learn the language
directly from it, but we include it here for those impatient people who
prefer to see everything up front before looking at verbose expository
material.  Later sections explain this material and provide examples.
We will repeat the definitions in those sections, and you may skip the
next section at first reading and proceed to Section~\ref{tut1}
(p.~\pageref{tut1}).

\section{Specification of the Metamath Language}\label{spec}
\index{Metamath!specification}

\begin{quote}
  {\em Sometimes one has to say difficult things, but one ought to say
them as simply as one knows how.}
    \flushright\sc  G. H. Hardy\footnote{As quoted in
    \cite{deMillo}, p.~273.}\\
\end{quote}\index{Hardy, G. H.}

\subsection{Preliminaries}\label{spec1}

% Space is technically a printable character, so we'll word things
% carefully so it's unambiguous.
A Metamath {\bf database}\index{database} is built up from a top-level source
file together with any source files that are brought in through file inclusion
commands (see below).  The only characters that are allowed to appear in a
Metamath source file are the 94 non-whitespace printable {\sc
ascii}\index{ascii@{\sc ascii}} characters, which are digits, upper and lower
case letters, and the following 32 special
characters\index{special characters}:\label{spec1chars}

\begin{verbatim}
! " # $ % & ' ( ) * + , - . / :
; < = > ? @ [ \ ] ^ _ ` { | } ~
\end{verbatim}
plus the following characters which are the ``white space'' characters:
space (a printable character),
tab, carriage return, line feed, and form feed.\label{whitespace}
We will use \texttt{typewriter}
font to display the printable characters.

A Metamath database consists of a sequence of three kinds of {\bf
tokens}\index{token} separated by {\bf white space}\index{white space}
(which is any sequence of one or more white space characters).  The set
of {\bf keyword}\index{keyword} tokens is \texttt{\$\char`\{},
\texttt{\$\char`\}}, \texttt{\$c}, \texttt{\$v}, \texttt{\$f},
\texttt{\$e}, \texttt{\$d}, \texttt{\$a}, \texttt{\$p}, \texttt{\$.},
\texttt{\$=}, \texttt{\$(}, \texttt{\$)}, \texttt{\$[}, and
\texttt{\$]}.  The last four are called {\bf auxiliary}\index{auxiliary
keyword} or preprocessing keywords.  A {\bf label}\index{label} token
consists of any combination of letters, digits, and the characters
hyphen, underscore, and period.  A {\bf math symbol}\index{math symbol}
token may consist of any combination of the 93 printable standard {\sc
ascii} characters other than space or \texttt{\$}~. All tokens are
case-sensitive.

\subsection{Preprocessing}

The token \texttt{\$(} begins a {\bf comment} and
\texttt{\$)} ends a comment.\index{\texttt{\$(}
and \texttt{\$)} auxiliary keywords}\index{comment}
Comments may contain any of
the 94 non-whitespace printable characters and white space,
except they may not contain the
2-character sequences \texttt{\$(} or \texttt{\$)} (comments do not nest).
Comments are ignored (treated
like white space) for the purpose of parsing, e.g.,
\texttt{\$( \$[ \$)} is a comment.
See p.~\pageref{mathcomments} for comment typesetting conventions; these
conventions may be ignored for the purpose of parsing.

A {\bf file inclusion command} consists of \texttt{\$[} followed by a file name
followed by \texttt{\$]}.\index{\texttt{\$[} and \texttt{\$]} auxiliary
keywords}\index{included file}\index{file inclusion}
It is only allowed in the outermost scope (i.e., not between
\texttt{\$\char`\{} and \texttt{\$\char`\}})
and must not be inside a statement (e.g., it may not occur
between the label of a \texttt{\$a} statement and its \texttt{\$.}).
The file name may not
contain a \texttt{\$} or white space.  The file must exist.
The case-sensitivity
of its name follows the conventions of the operating system.  The contents of
the file replace the inclusion command.
Included files may include other files.
Only the first reference to a given file is included; any later
references to the same file (whether in the top-level file or in included
files) cause the inclusion command to be ignored (treated like white space).
A verifier may assume that file names with different strings
refer to different files for the purpose of ignoring later references.
A file self-reference is ignored, as is any reference to the top-level file
(to avoid loops).
Included files may not include a \texttt{\$(} without a matching \texttt{\$)},
may not include a \texttt{\$[} without a matching \texttt{\$]}, and may
not include incomplete statements (e.g., a \texttt{\$a} without a matching
\texttt{\$.}).
It is currently unspecified if path references are relative to the process'
current directory or the file's containing directory, so databases should
avoid using pathname separators (e.g., ``/'') in file names.

Like all tokens, the \texttt{\$(}, \texttt{\$)}, \texttt{\$[}, and \texttt{\$]} keywords
must be surrounded by white space.

\subsection{Basic Syntax}

After preprocessing, a database will consist of a sequence of {\bf
statements}.
These are the scoping statements \texttt{\$\char`\{} and
\texttt{\$\char`\}}, along with the \texttt{\$c}, \texttt{\$v},
\texttt{\$f}, \texttt{\$e}, \texttt{\$d}, \texttt{\$a}, and \texttt{\$p}
statements.

A {\bf scoping statement}\index{scoping statement} consists only of its
keyword, \texttt{\$\char`\{} or \texttt{\$\char`\}}.
A \texttt{\$\char`\{} begins a {\bf
block}\index{block} and a matching \texttt{\$\char`\}} ends the block.
Every \texttt{\$\char`\{}
must have a matching \texttt{\$\char`\}}.
Defining it recursively, we say a block
contains a sequence of zero or more tokens other
than \texttt{\$\char`\{} and \texttt{\$\char`\}} and
possibly other blocks.  There is an {\bf outermost
block}\index{block!outermost} not bracketed by \texttt{\$\char`\{} \texttt{\$\char`\}}; the end
of the outermost block is the end of the database.

% LaTeX bug? can't do \bf\tt

A {\bf \$v} or {\bf \$c statement}\index{\texttt{\$v} statement}\index{\texttt{\$c}
statement} consists of the keyword token \texttt{\$v} or \texttt{\$c} respectively,
followed by one or more math symbols,
% The word "token" is used to distinguish "$." from the sentence-ending period.
followed by the \texttt{\$.}\ token.
These
statements {\bf declare}\index{declaration} the math symbols to be {\bf
variables}\index{variable!Metamath} or {\bf constants}\index{constant}
respectively. The same math symbol may not occur twice in a given \texttt{\$v} or
\texttt{\$c} statement.

%c%A math symbol becomes an {\bf active}\index{active math symbol}
%c%when declared and stays active until the end of the block in which it is
%c%declared.  A math symbol may not be declared a second time while it is active,
%c%but it may be declared again after it becomes inactive.

A math symbol becomes {\bf active}\index{active math symbol} when declared
and stays active until the end of the block in which it is declared.  A
variable may not be declared a second time while it is active, but it
may be declared again (as a variable, but not as a constant) after it
becomes inactive.  A constant must be declared in the outermost block and may
not be declared a second time.\index{redeclaration of symbols}

A {\bf \$f statement}\index{\texttt{\$f} statement} consists of a label,
followed by \texttt{\$f}, followed by its typecode (an active constant),
followed by an
active variable, followed by the \texttt{\$.}\ token.  A {\bf \$e
statement}\index{\texttt{\$e} statement} consists of a label, followed
by \texttt{\$e}, followed by its typecode (an active constant),
followed by zero or more
active math symbols, followed by the \texttt{\$.}\ token.  A {\bf
hypothesis}\index{hypothesis} is a \texttt{\$f} or \texttt{\$e}
statement.
The type declared by a \texttt{\$f} statement for a given label
is global even if the variable is not
(e.g., a database may not have \texttt{wff P} in one local scope
and \texttt{class P} in another).

A {\bf simple \$d statement}\index{\texttt{\$d} statement!simple}
consists of \texttt{\$d}, followed by two different active variables,
followed by the \texttt{\$.}\ token.  A {\bf compound \$d
statement}\index{\texttt{\$d} statement!compound} consists of
\texttt{\$d}, followed by three or more variables (all different),
followed by the \texttt{\$.}\ token.  The order of the variables in a
\texttt{\$d} statement is unimportant.  A compound \texttt{\$d}
statement is equivalent to a set of simple \texttt{\$d} statements, one
for each possible pair of variables occurring in the compound
\texttt{\$d} statement.  Henceforth in this specification we shall
assume all \texttt{\$d} statements are simple.  A \texttt{\$d} statement
is also called a {\bf disjoint} (or {\bf distinct}) {\bf variable
restriction}.\index{disjoint-variable restriction}

A {\bf \$a statement}\index{\texttt{\$a} statement} consists of a label,
followed by \texttt{\$a}, followed by its typecode (an active constant),
followed by
zero or more active math symbols, followed by the \texttt{\$.}\ token.  A {\bf
\$p statement}\index{\texttt{\$p} statement} consists of a label,
followed by \texttt{\$p}, followed by its typecode (an active constant),
followed by
zero or more active math symbols, followed by \texttt{\$=}, followed by
a sequence of labels, followed by the \texttt{\$.}\ token.  An {\bf
assertion}\index{assertion} is a \texttt{\$a} or \texttt{\$p} statement.

A \texttt{\$f}, \texttt{\$e}, or \texttt{\$d} statement is {\bf active}\index{active
statement} from the place it occurs until the end of the block it occurs in.
A \texttt{\$a} or \texttt{\$p} statement is {\bf active} from the place it occurs
through the end of the database.
There may not be two active \texttt{\$f} statements containing the same
variable.  Each variable in a \texttt{\$e}, \texttt{\$a}, or
\texttt{\$p} statement must exist in an active \texttt{\$f}
statement.\footnote{This requirement can greatly simplify the
unification algorithm (substitution calculation) required by proof
verification.}

%The label that begins each \texttt{\$f}, \texttt{\$e}, \texttt{\$a}, and
%\texttt{\$p} statement must be unique.
Each label token must be unique, and
no label token may match any math symbol
token.\label{namespace}\footnote{This
restriction was added on June 24, 2006.
It is not theoretically necessary but is imposed to make it easier to
write certain parsers.}

The set of {\bf mandatory variables}\index{mandatory variable} associated with
an assertion is the set of (zero or more) variables in the assertion and in any
active \texttt{\$e} statements.  The (possibly empty) set of {\bf mandatory
hypotheses}\index{mandatory hypothesis} is the set of all active \texttt{\$f}
statements containing mandatory variables, together with all active \texttt{\$e}
statements.
The set of {\bf mandatory {\bf \$d} statements}\index{mandatory
disjoint-variable restriction} associated with an assertion are those active
\texttt{\$d} statements whose variables are both among the assertion's
mandatory variables.

\subsection{Proof Verification}\label{spec4}

The sequence of labels between the \texttt{\$=} and \texttt{\$.}\ tokens
in a \texttt{\$p} statement is a {\bf proof}.\index{proof!Metamath} Each
label in a proof must be the label of an active statement other than the
\texttt{\$p} statement itself; thus a label must refer either to an
active hypothesis of the \texttt{\$p} statement or to an earlier
assertion.

An {\bf expression}\index{expression} is a sequence of math symbols. A {\bf
substitution map}\index{substitution map} associates a set of variables with a
set of expressions.  It is acceptable for a variable to be mapped to an
expression containing it.  A {\bf
substitution}\index{substitution!variable}\index{variable substitution} is the
simultaneous replacement of all variables in one or more expressions with the
expressions that the variables map to.

A proof is scanned in order of its label sequence.  If the label refers to an
active hypothesis, the expression in the hypothesis is pushed onto a
stack.\index{stack}\index{RPN stack}  If the label refers to an assertion, a
(unique) substitution must exist that, when made to the mandatory hypotheses
of the referenced assertion, causes them to match the topmost (i.e.\ most
recent) entries of the stack, in order of occurrence of the mandatory
hypotheses, with the topmost stack entry matching the last mandatory
hypothesis of the referenced assertion.  As many stack entries as there are
mandatory hypotheses are then popped from the stack.  The same substitution is
made to the referenced assertion, and the result is pushed onto the stack.
After the last label in the proof is processed, the stack must have a single
entry that matches the expression in the \texttt{\$p} statement containing the
proof.

%c%{\footnotesize\begin{quotation}\index{redeclaration of symbols}
%c%{{\em Comment.}\label{spec4comment} Whenever a math symbol token occurs in a
%c%{\texttt{\$c} or \texttt{\$v} statement, it is considered to designate a distinct new
%c%{symbol, even if the same token was previously declared (and is now inactive).
%c%{Thus a math token declared as a constant in two different blocks is considered
%c%{to designate two distinct constants (even though they have the same name).
%c%{The two constants will not match in a proof that references both blocks.
%c%{However, a proof referencing both blocks is acceptable as long as it doesn't
%c%{require that the constants match.  Similarly, a token declared to be a
%c%{constant for a referenced assertion will not match the same token declared to
%c%{be a variable for the \texttt{\$p} statement containing the proof.  In the case
%c%{of a token declared to be a variable for a referenced assertion, this is not
%c%{an issue since the variable can be substituted with whatever expression is
%c%{needed to achieve the required match.
%c%{\end{quotation}}
%c2%A proof may reference an assertion that contains or whose hypotheses contain a
%c2%constant that is not active for the \texttt{\$p} statement containing the proof.
%c2%However, the final result of the proof may not contain that constant. A proof
%c2%may also reference an assertion that contains or whose hypotheses contain a
%c2%variable that is not active for the \texttt{\$p} statement containing the proof.
%c2%That variable, of course, will be substituted with whatever expression is
%c2%needed to achieve the required match.

A proof may contain a \texttt{?}\ in place of a label to indicate an unknown step
(Section~\ref{unknown}).  A proof verifier may ignore any proof containing
\texttt{?}\ but should warn the user that the proof is incomplete.

A {\bf compressed proof}\index{compressed proof}\index{proof!compressed} is an
alternate proof notation described in Appen\-dix~\ref{compressed}; also see
references to ``compressed proof'' in the Index.  Compressed proofs are a
Metamath language extension which a complete proof verifier should be able to
parse and verify.

\subsubsection{Verifying Disjoint Variable Restrictions}

Each substitution made in a proof must be checked to verify that any
disjoint variable restrictions are satisfied, as follows.

If two variables replaced by a substitution exist in a mandatory \texttt{\$d}
statement\index{\texttt{\$d} statement} of the assertion referenced, the two
expressions resulting from the substitution must satisfy the following
conditions.  First, the two expressions must have no variables in common.
Second, each possible pair of variables, one from each expression, must exist
in an active \texttt{\$d} statement of the \texttt{\$p} statement containing the
proof.

\vskip 1ex

This ends the specification of the Metamath language;
see Appendix \ref{BNF} for its syntax in
Extended Backus--Naur Form (EBNF)\index{Extended Backus--Naur Form}\index{EBNF}.

\section{The Basic Keywords}\label{tut1}

Our expository material begins here.

Like most computer languages, Metamath\index{Metamath} takes its input from
one or more {\bf source files}\index{source file} which contain characters
expressed in the standard {\sc ascii} (American Standard Code for Information
Interchange)\index{ascii@{\sc ascii}} code for computers.  A source file
consists of a series of {\bf tokens}\index{token}, which are strings of
non-whitespace
printable characters (from the set of 94 shown on p.~\pageref{spec1chars})
separated by {\bf white space}\index{white space} (spaces, tabs, carriage
returns, line feeds, and form feeds). Any string consisting only of these
characters is treated the same as a single space.  The non-whitespace printable
characters\index{printable character} that Metamath recognizes are the 94
characters on standard {\sc ascii} keyboards.

Metamath has the ability to join several files together to form its
input (Section~\ref{include}).  We call the aggregate contents of all
the files after they have been joined together a {\bf
database}\index{database} to distinguish it from an individual source
file.  The tokens in a database consist of {\bf
keywords}\index{keyword}, which are built into the language, together
with two kinds of user-defined tokens called {\bf labels}\index{label}
and {\bf math symbols}\index{math symbol}.  (Often we will simply say
{\bf symbol}\index{symbol} instead of math symbol for brevity).  The set
of {\bf basic keywords}\index{basic keyword} is
\texttt{\$c}\index{\texttt{\$c} statement},
\texttt{\$v}\index{\texttt{\$v} statement},
\texttt{\$e}\index{\texttt{\$e} statement},
\texttt{\$f}\index{\texttt{\$f} statement},
\texttt{\$d}\index{\texttt{\$d} statement},
\texttt{\$a}\index{\texttt{\$a} statement},
\texttt{\$p}\index{\texttt{\$p} statement},
\texttt{\$=}\index{\texttt{\$=} keyword},
\texttt{\$.}\index{\texttt{\$.}\ keyword},
\texttt{\$\char`\{}\index{\texttt{\$\char`\{} and \texttt{\$\char`\}}
keywords}, and \texttt{\$\char`\}}.  This is the complete set of
syntactical elements of what we call the {\bf basic
language}\index{basic language} of Metamath, and with them you can
express all of the mathematics that were intended by the design of
Metamath.  You should make it a point to become very familiar with them.
Table~\ref{basickeywords} lists the basic keywords along with a brief
description of their functions.  For now, this description will give you
only a vague notion of what the keywords are for; later we will describe
the keywords in detail.


\begin{table}[htp] \caption{Summary of the basic Metamath
keywords} \label{basickeywords}
\begin{center}
\begin{tabular}{|p{4pc}|l|}
\hline
\em \centering Keyword&\em Description\\
\hline
\hline
\centering
   \texttt{\$c}&Constant symbol declaration\\
\hline
\centering
   \texttt{\$v}&Variable symbol declaration\\
\hline
\centering
   \texttt{\$d}&Disjoint variable restriction\\
\hline
\centering
   \texttt{\$f}&Variable-type (``floating'') hypothesis\\
\hline
\centering
   \texttt{\$e}&Logical (``essential'') hypothesis\\
\hline
\centering
   \texttt{\$a}&Axiomatic assertion\\
\hline
\centering
   \texttt{\$p}&Provable assertion\\
\hline
\centering
   \texttt{\$=}&Start of proof in \texttt{\$p} statement\\
\hline
\centering
   \texttt{\$.}&End of the above statement types\\
\hline
\centering
   \texttt{\$\char`\{}&Start of block\\
\hline
\centering
   \texttt{\$\char`\}}&End of block\\
\hline
\end{tabular}
\end{center}
\end{table}

%For LaTeX bug(?) where it puts tables on blank page instead of btwn text
%May have to adjust if text changes
%\newpage

There are some additional keywords, called {\bf auxiliary
keywords}\index{auxiliary keyword} that help make Metamath\index{Metamath}
more practical. These are part of the {\bf extended language}\index{extended
language}. They provide you with a means to put comments into a Metamath
source file\index{source file} and reference other source files.  We will
introduce these in later sections. Table~\ref{otherkeywords} summarizes them
so that you can recognize them now if you want to peruse some source
files while learning the basic keywords.


\begin{table}[htp] \caption{Auxiliary Metamath
keywords} \label{otherkeywords}
\begin{center}
\begin{tabular}{|p{4pc}|l|}
\hline
\em \centering Keyword&\em Description\\
\hline
\hline
\centering
   \texttt{\$(}&Start of comment\\
\hline
\centering
   \texttt{\$)}&End of comment\\
\hline
\centering
   \texttt{\$[}&Start of included source file name\\
\hline
\centering
   \texttt{\$]}&End of included source file name\\
\hline
\end{tabular}
\end{center}
\end{table}
\index{\texttt{\$(} and \texttt{\$)} auxiliary keywords}
\index{\texttt{\$[} and \texttt{\$]} auxiliary keywords}


Unlike those in some computer languages, the keywords\index{keyword} are short
two-character sequences rather than English-like words.  While this may make
them slightly more difficult to remember at first, their brevity allows
them to blend in with the mathematics being described, not
distract from it, like punctuation marks.


\subsection{User-Defined Tokens}\label{dollardollar}\index{token}

As you may have noticed, all keywords\index{keyword} begin with the \texttt{\$}
character.  This mundane monetary symbol is not ordinarily used in higher
mathematics (outside of grant proposals), so we have appropriated it to
distinguish the Metamath\index{Metamath} keywords from ordinary mathematical
symbols. The \texttt{\$} character is thus considered special and may not be
used as a character in a user-defined token.  All tokens and keywords are
case-sensitive; for example, \texttt{n} is considered to be a different character
from \texttt{N}.  Case-sensitivity makes the available {\sc ascii} character set
as rich as possible.

\subsubsection{Math Symbol Tokens}\index{token}

Math symbols\index{math symbol} are tokens used to represent the symbols
that appear in ordinary mathematical formulas.  They may consist of any
combination of the 93 non-whitespace printable {\sc ascii} characters other than
\texttt{\$}~. Some examples are \texttt{x}, \texttt{+}, \texttt{(},
\texttt{|-}, \verb$!%@?&$, and \texttt{bounded}.  For readability, it is
best to try to make these look as similar to actual mathematical symbols
as possible, within the constraints of the {\sc ascii} character set, in
order to make the resulting mathematical expressions more readable.

In the Metamath\index{Metamath} language, you express ordinary
mathematical formulas and statements as sequences of math symbols such
as \texttt{2 + 2 = 4} (five symbols, all constants).\footnote{To
eliminate ambiguity with other expressions, this is expressed in the set
theory database \texttt{set.mm} as \texttt{|- ( 2 + 2
 ) = 4 }, whose \LaTeX\ equivalent is $\vdash
(2+2)=4$.  The \,$\vdash$ means ``is a theorem'' and the
parentheses allow explicit associative grouping.}\index{turnstile
({$\,\vdash$})} They may even be English
sentences, as in \texttt{E is closed and bounded} (five symbols)---here
\texttt{E} would be a variable and the other four symbols constants.  In
principle, a Metamath database could be constructed to work with almost
any unambiguous English-language mathematical statement, but as a
practical matter the definitions needed to provide for all possible
syntax variations would be cumbersome and distracting and possibly have
subtle pitfalls accidentally built in.  We generally recommend that you
express mathematical statements with compact standard mathematical
symbols whenever possible and put their English-language descriptions in
comments.  Axioms\index{axiom} and definitions\index{definition}
(\texttt{\$a}\index{\texttt{\$a} statement} statements) are the only
places where Metamath will not detect an error, and doing this will help
reduce the number of definitions needed.

You are free to use any tokens\index{token} you like for math
symbols\index{math symbol}.  Appendix~\ref{ASCII} recommends token names to
use for symbols in set theory, and we suggest you adopt these in order to be
able to include the \texttt{set.mm} set theory database in your database.  For
printouts, you can convert the tokens in a database
to standard mathematical symbols with the \LaTeX\ typesetting program.  The
Metamath command \texttt{open tex} {\em filename}\index{\texttt{open tex} command}
produces output that can be read by \LaTeX.\index{latex@{\LaTeX}}
The correspondence
between tokens and the actual symbols is made by \texttt{latexdef}
statements inside a special database comment tagged
with \texttt{\$t}.\index{\texttt{\$t} comment}\index{typesetting comment}
  You can edit
this comment to change the definitions or add new ones.
Appendix~\ref{ASCII} describes how to do this in more detail.

% White space\index{white space} is normally used to separate math
% symbol\index{math symbol} tokens, but they may be juxtaposed without white
% space in \texttt{\$d}\index{\texttt{\$d} statement}, \texttt{\$e}\index{\texttt{\$e}
% statement}, \texttt{\$f}\index{\texttt{\$f} statement}, \texttt{\$a}\index{\texttt{\$a}
% statement}, and \texttt{\$p}\index{\texttt{\$p} statement} statements when no
% ambiguity will result.  Specifically, Metamath parses the math symbol sequence
% in one of these statements in the following manner:  when the math symbol
% sequence has been broken up into tokens\index{token} up to a given character,
% the next token is the longest string of characters that could constitute a
% math symbol that is active\index{active
% math symbol} at that point.  (See Section~\ref{scoping} for the
% definition of an active math symbol.)  For example, if \texttt{-}, \texttt{>}, and
% \texttt{->} are the only active math symbols, the juxtaposition \texttt{>-} will be
% interpreted as the two symbols \texttt{>} and \texttt{-}, whereas \texttt{->} will
% always be interpreted as that single symbol.\footnote{For better readability we
% recommend a white space between each token.  This also makes searching for a
% symbol easier to do with an editor.  Omission of optional white space is useful
% for reducing typing when assigning an expression to a temporary
% variable\index{temporary variable} with the \texttt{let variable} Metamath
% program command.}\index{\texttt{let variable} command}
%
% Keywords\index{keyword} may be placed next to math symbols without white
% space\index{white space} between them.\footnote{Again, we do not recommend
% this for readability.}
%
% The math symbols\index{math symbol} in \texttt{\$c}\index{\texttt{\$c} statement}
% and \texttt{\$v}\index{\texttt{\$v} statement} statements must always be separated
% by white space\index{white
% space}, for the obvious reason that these statements define the names
% of the symbols.
%
% Math symbols referred to in comments (see Section~\ref{comments}) must also be
% separated by white space.  This allows you to make comments about symbols that
% are not yet active\index{active
% math symbol}.  (The ``math mode'' feature of comments is also a quick and
% easy way to obtain word processing text with embedded mathematical symbols,
% independently of the main purpose of Metamath; the way to do this is described
% in Section~\ref{comments})

\subsubsection{Label Tokens}\index{token}\index{label}

Label tokens are used to identify Metamath\index{Metamath} statements for
later reference. Label tokens may contain only letters, digits, and the three
characters period, hyphen, and underscore:
\begin{verbatim}
. - _
\end{verbatim}

A label is {\bf declared}\index{label declaration} by placing it immediately
before the keyword of the statement it identifies.  For example, the label
\texttt{axiom.1} might be declared as follows:
\begin{verbatim}
axiom.1 $a |- x = x $.
\end{verbatim}

Each \texttt{\$e}\index{\texttt{\$e} statement},
\texttt{\$f}\index{\texttt{\$f} statement},
\texttt{\$a}\index{\texttt{\$a} statement}, and
\texttt{\$p}\index{\texttt{\$p} statement} statement in a database must
have a label declared for it.  No other statement types may have label
declarations.  Every label must be unique.

A label (and the statement it identifies) is {\bf referenced}\index{label
reference} by including the label between the \texttt{\$=}\index{\texttt{\$=}
keyword} and \texttt{\$.}\index{\texttt{\$.}\ keyword}\ keywords in a \texttt{\$p}
statement.  The sequence of labels\index{label sequence} between these two
keywords is called a {\bf proof}\index{proof}.  An example of a statement with
a proof that we will encounter later (Section~\ref{proof}) is
\begin{verbatim}
wnew $p wff ( s -> ( r -> p ) )
     $= ws wr wp w2 w2 $.
\end{verbatim}

You don't have to know what this means just yet, but you should know that the
label \texttt{wnew} is declared by this \texttt{\$p} statement and that the labels
\texttt{ws}, \texttt{wr}, \texttt{wp}, and \texttt{w2} are assumed to have been declared
earlier in the database and are referenced here.

\subsection{Constants and Variables}
\index{constant}
\index{variable}

An {\bf expression}\index{expression} is any sequence of math
symbols, possibly empty.

The basic Metamath\index{Metamath} language\index{basic language} has two
kinds of math symbols\index{math symbol}:  {\bf constants}\index{constant} and
{\bf variables}\index{variable}.  In a Metamath proof, a constant may not be
substituted with any expression.  A variable can be
substituted\index{substitution!variable}\index{variable substitution} with any
expression.  This sequence may include other variables and may even include
the variable being substituted.  This substitution takes place when proofs are
verified, and it will be described in Section~\ref{proof}.  The \texttt{\$f}
statement (described later in Section~\ref{dollaref}) is used to specify the
{\bf type} of a variable (i.e.\ what kind of
variable it is)\index{variable type}\index{type} and
give it a meaning typically
associated with a ``metavariable''\index{metavariable}\footnote{A metavariable
is a variable that ranges over the syntactical elements of the object language
being discussed; for example, one metavariable might represent a variable of
the object language and another metavariable might represent a formula in the
object language.} in ordinary mathematics; for example, a variable may be
specified to be a wff or well-formed formula (in logic), a set (in set
theory), or a non-negative integer (in number theory).

%\subsection{The \texttt{\$c} and \texttt{\$v} Declaration Statements}
\subsection{The \texttt{\$c} and \texttt{\$v} Declaration Statements}
\index{\texttt{\$c} statement}
\index{constant declaration}
\index{\texttt{\$v} statement}
\index{variable declaration}

Constants are introduced or {\bf declared}\index{constant declaration}
with \texttt{\$c}\index{\texttt{\$c} statement} statements, and
variables are declared\index{variable declaration} with
\texttt{\$v}\index{\texttt{\$v} statement} statements.  A {\bf simple}
declaration\index{simple declaration} statement introduces a single
constant or variable.  Its syntax is one of the following:
\begin{center}
  \texttt{\$c} {\em math-symbol} \texttt{\$.}\\
  \texttt{\$v} {\em math-symbol} \texttt{\$.}
\end{center}
The notation {\em math-symbol} means any math symbol token\index{token}.

Some examples of simple declaration statements are:
\begin{center}
  \texttt{\$c + \$.}\\
  \texttt{\$c -> \$.}\\
  \texttt{\$c ( \$.}\\
  \texttt{\$v x \$.}\\
  \texttt{\$v y2 \$.}
\end{center}

The characters in a math symbol\index{math symbol} being declared are
irrelevant to Meta\-math; for example, we could declare a right parenthesis to
be a variable,
\begin{center}
  \texttt{\$v ) \$.}\\
\end{center}
although this would be unconventional.

A {\bf compound} declaration\index{compound declaration} statement is a
shorthand for declaring several symbols at once.  Its syntax is one of the
following:
\begin{center}
  \texttt{\$c} {\em math-symbol}\ \,$\cdots$\ {\em math-symbol} \texttt{\$.}\\
  \texttt{\$v} {\em math-symbol}\ \,$\cdots$\ {\em math-symbol} \texttt{\$.}
\end{center}\index{\texttt{\$c} statement}
Here, the ellipsis (\ldots) means any number of {\em math-symbol}\,s.

An example of a compound declaration statement is:
\begin{center}
  \texttt{\$v x y mu \$.}\\
\end{center}
This is equivalent to the three simple declaration statements
\begin{center}
  \texttt{\$v x \$.}\\
  \texttt{\$v y \$.}\\
  \texttt{\$v mu \$.}\\
\end{center}
\index{\texttt{\$v} statement}

There are certain rules on where in the database math symbols may be declared,
what sections of the database are aware of them (i.e.\ where they are
``active''), and when they may be declared more than once.  These will be
discussed in Section~\ref{scoping} and specifically on
p.~\pageref{redeclaration}.

\subsection{The \texttt{\$d} Statement}\label{dollard}
\index{\texttt{\$d} statement}

The \texttt{\$d} statement is called a {\bf disjoint-variable restriction}.  The
syntax of the {\bf simple} version of this statement is
\begin{center}
  \texttt{\$d} {\em variable variable} \texttt{\$.}
\end{center}
where each {\em variable} is a previously declared variable and the two {\em
variable}\,s are different.  (More specifically, each  {\em variable} must be
an {\bf active} variable\index{active math symbol}, which means there must be
a previous \texttt{\$v} statement whose {\bf scope}\index{scope} includes the
\texttt{\$d} statement.  These terms will be defined when we discuss scoping
statements in Section~\ref{scoping}.)

In ordinary mathematics, formulas may arise that are true if the variables in
them are distinct\index{distinct variables}, but become false when those
variables are made identical. For example, the formula in logic $\exists x\,x
\neq y$, which means ``for a given $y$, there exists an $x$ that is not equal
to $y$,'' is true in most mathematical theories (namely all non-trivial
theories\index{non-trivial theory}, i.e.\ those that describe more than one
individual, such as arithmetic).  However, if we substitute $y$ with $x$, we
obtain $\exists x\,x \neq x$, which is always false, as it means ``there
exists something that is not equal to itself.''\footnote{If you are a
logician, you will recognize this as the improper substitution\index{proper
substitution}\index{substitution!proper} of a free variable\index{free
variable} with a bound variable\index{bound variable}.  Metamath makes no
inherent distinction between free and bound variables; instead, you let
Metamath know what substitutions are permissible by using \texttt{\$d} statements
in the right way in your axiom system.}\index{free vs.\ bound variable}  The
\texttt{\$d} statement allows you to specify a restriction that forbids the
substitution of one variable with another.  In
this case, we would use the statement
\begin{center}
  \texttt{\$d x y \$.}
\end{center}\index{\texttt{\$d} statement}
to specify this restriction.

The order in which the variables appear in a \texttt{\$d} statement is not
important.  We could also use
\begin{center}
  \texttt{\$d y x \$.}
\end{center}

The \texttt{\$d} statement is actually more general than this, as the
``disjoint''\index{disjoint variables} in its name suggests.  The full meaning
is that if any substitution is made to its two variables (during the
course of a proof that references a \texttt{\$a} or \texttt{\$p} statement
associated with the \texttt{\$d}), the two expressions that result from the
substitution must have no variables in common.  In addition, each possible
pair of variables, one from each expression, must be in a \texttt{\$d} statement
associated with the statement being proved.  (This requirement forces the
statement being proved to ``inherit'' the original disjoint variable
restriction.)

For example, suppose \texttt{u} is a variable.  If the restriction
\begin{center}
  \texttt{\$d A B \$.}
\end{center}
has been specified for a theorem referenced in a
proof, we may not substitute \texttt{A} with \mbox{\tt a + u} and
\texttt{B} with \mbox{\tt b + u} because these two symbol sequences have the
variable \texttt{u} in common.  Furthermore, if \texttt{a} and \texttt{b} are
variables, we may not substitute \texttt{A} with \texttt{a} and \texttt{B} with \texttt{b}
unless we have also specified \texttt{\$d a b} for the theorem being proved; in
other words, the \texttt{\$d} property associated with a pair of variables must
be effectively preserved after substitution.

The \texttt{\$d}\index{\texttt{\$d} statement} statement does {\em not} mean ``the
two variables may not be substituted with the same thing,'' as you might think
at first.  For example, substituting each of \texttt{A} and \texttt{B} in the above
example with identical symbol sequences consisting only of constants does not
cause a disjoint variable conflict, because two symbol sequences have no
variables in common (since they have no variables, period).  Similarly, a
conflict will not occur by substituting the two variables in a \texttt{\$d}
statement with the empty symbol sequence\index{empty substitution}.

The \texttt{\$d} statement does not have a direct counterpart in
ordinary mathematics, partly because the variables\index{variable} of
Metamath are not really the same as the variables\index{variable!in
ordinary mathematics} of ordinary mathematics but rather are
metavariables\index{metavariable} ranging over them (as well as over
other kinds of symbols and groups of symbols).  Depending on the
situation, we may informally interpret the \texttt{\$d} statement in
different ways.  Suppose, for example, that \texttt{x} and \texttt{y}
are variables ranging over numbers (more precisely, that \texttt{x} and
\texttt{y} are metavariables ranging over variables that range over
numbers), and that \texttt{ph} ($\varphi$) and \texttt{ps} ($\psi$) are
variables (more precisely, metavariables) ranging over formulas.  We can
make the following interpretations that correspond to the informal
language of ordinary mathematics:
\begin{quote}
\begin{tabbing}
\texttt{\$d x y \$.} means ``assume $x$ and $y$ are
distinct variables.''\\
\texttt{\$d x ph \$.} means ``assume $x$ does not
occur in $\varphi$.''\\
\texttt{\$d ph ps \$.} \=means ``assume $\varphi$ and
$\psi$ have no variables\\ \>in common.''
\end{tabbing}
\end{quote}\index{\texttt{\$d} statement}

\subsubsection{Compound \texttt{\$d} Statements}

The {\bf compound} version of the \texttt{\$d} statement is a shorthand for
specifying several variables whose substitutions must be pairwise disjoint.
Its syntax is:
\begin{center}
  \texttt{\$d} {\em variable}\ \,$\cdots$\ {\em variable} \texttt{\$.}
\end{center}\index{\texttt{\$d} statement}
Here, {\em variable} represents the token of a previously declared
variable (specifically, an active variable) and all {\em variable}\,s are
different.  The compound \texttt{\$d}
statement is internally broken up by Metamath into one simple \texttt{\$d}
statement for each possible pair of variables in the original \texttt{\$d}
statement.  For example,
\begin{center}
  \texttt{\$d w x y z \$.}
\end{center}
is equivalent to
\begin{center}
  \texttt{\$d w x \$.}\\
  \texttt{\$d w y \$.}\\
  \texttt{\$d w z \$.}\\
  \texttt{\$d x y \$.}\\
  \texttt{\$d x z \$.}\\
  \texttt{\$d y z \$.}
\end{center}

Two or more simple \texttt{\$d} statements specifying the same variable pair are
internally combined into a single \texttt{\$d} statement.  Thus the set of three
statements
\begin{center}
  \texttt{\$d x y \$.}
  \texttt{\$d x y \$.}
  \texttt{\$d y x \$.}
\end{center}
is equivalent to
\begin{center}
  \texttt{\$d x y \$.}
\end{center}

Similarly, compound \texttt{\$d} statements, after being internally broken up,
internally have their common variable pairs combined.  For example the
set of statements
\begin{center}
  \texttt{\$d x y A \$.}
  \texttt{\$d x y B \$.}
\end{center}
is equivalent to
\begin{center}
  \texttt{\$d x y \$.}
  \texttt{\$d x A \$.}
  \texttt{\$d y A \$.}
  \texttt{\$d x y \$.}
  \texttt{\$d x B \$.}
  \texttt{\$d y B \$.}
\end{center}
which is equivalent to
\begin{center}
  \texttt{\$d x y \$.}
  \texttt{\$d x A \$.}
  \texttt{\$d y A \$.}
  \texttt{\$d x B \$.}
  \texttt{\$d y B \$.}
\end{center}

Metamath\index{Metamath} automatically verifies that all \texttt{\$d}
restrictions are met whenever it verifies proofs.  \texttt{\$d} statements are
never referenced directly in proofs (this is why they do not have
labels\index{label}), but Metamath is always aware of which ones must be
satisfied (i.e.\ are active) and will notify you with an error message if any
violation occurs.

To illustrate how Metamath detects a missing \texttt{\$d}
statement, we will look at the following example from the
\texttt{set.mm} database.

\begin{verbatim}
$d x z $.  $d y z $.
$( Theorem to add distinct quantifier to atomic formula. $)
ax17eq $p |- ( x = y -> A. z x = y ) $=...
\end{verbatim}

This statement has the obvious requirement that $z$ must be
distinct\index{distinct variables} from $x$ in theorem \texttt{ax17eq} that
states $x=y \rightarrow \forall z \, x=y$ (well, obvious if you're a logician,
for otherwise we could conclude  $x=y \rightarrow \forall x \, x=y$, which is
false when the free variables $x$ and $y$ are equal).

Let's look at what happens if we edit the database to comment out this
requirement.

\begin{verbatim}
$( $d x z $. $) $d y z $.
$( Theorem to add distinct quantifier to atomic formula. $)
ax17eq $p |- ( x = y -> A. z x = y ) $=...
\end{verbatim}

When it tries to verify the proof, Metamath will tell you that \texttt{x} and
\texttt{z} must be disjoint, because one of its steps references an axiom or
theorem that has this requirement.

\begin{verbatim}
MM> verify proof ax17eq
ax17eq ?Error at statement 1918, label "ax17eq", type "$p":
      vz wal wi vx vy vz ax-13 vx vy weq vz vx ax-c16 vx vy
                                               ^^^^^
There is a disjoint variable ($d) violation at proof step 29.
Assertion "ax-c16" requires that variables "x" and "y" be
disjoint.  But "x" was substituted with "z" and "y" was
substituted with "x".  The assertion being proved, "ax17eq",
does not require that variables "z" and "x" be disjoint.
\end{verbatim}

We can see the substitutions into \texttt{ax-c16} with the following command.

\begin{verbatim}
MM> show proof ax17eq / detailed_step 29
Proof step 29:  pm2.61dd.2=ax-c16 $a |- ( A. z z = x -> ( x =
  y -> A. z x = y ) )
This step assigns source "ax-c16" ($a) to target "pm2.61dd.2"
($e).  The source assertion requires the hypotheses "wph"
($f, step 26), "vx" ($f, step 27), and "vy" ($f, step 28).
The parent assertion of the target hypothesis is "pm2.61dd"
($p, step 36).
The source assertion before substitution was:
    ax-c16 $a |- ( A. x x = y -> ( ph -> A. x ph ) )
The following substitutions were made to the source
assertion:
    Variable  Substituted with
     x         z
     y         x
     ph        x = y
The target hypothesis before substitution was:
    pm2.61dd.2 $e |- ( ph -> ch )
The following substitutions were made to the target
hypothesis:
    Variable  Substituted with
     ph        A. z z = x
     ch        ( x = y -> A. z x = y )
\end{verbatim}

The disjoint variable restrictions of \texttt{ax-c16} can be seen from the
\texttt{show state\-ment} command.  The line that begins ``\texttt{Its mandatory
dis\-joint var\-i\-able pairs are:}\ldots'' lists any \texttt{\$d} variable
pairs in brackets.

\begin{verbatim}
MM> show statement ax-c16/full
Statement 3033 is located on line 9338 of the file "set.mm".
"Axiom of Distinct Variables. ..."
  ax-c16 $a |- ( A. x x = y -> ( ph -> A. x ph ) ) $.
Its mandatory hypotheses in RPN order are:
  wph $f wff ph $.
  vx $f setvar x $.
  vy $f setvar y $.
Its mandatory disjoint variable pairs are:  <x,y>
The statement and its hypotheses require the variables:  x y
      ph
The variables it contains are:  x y ph
\end{verbatim}

Since Metamath will always detect when \texttt{\$d}\index{\texttt{\$d} statement}
statements are needed for a proof, you don't have to worry too much about
forgetting to put one in; it can always be added if you see the error message
above.  If you put in unnecessary \texttt{\$d} statements, the worst that could
happen is that your theorem might not be as general as it could be, and this
may limit its use later on.

On the other hand, when you introduce axioms (\texttt{\$a}\index{\texttt{\$a}
statement} statements), you must be very careful to properly specify the
necessary associated \texttt{\$d} statements since Metamath has no way of knowing
whether your axioms are correct.  For example, Metamath would have no idea
that \texttt{ax-c16}, which we are telling it is an axiom of logic, would lead to
contradictions if we omitted its associated \texttt{\$d} statement.

% This was previously a comment in footnote-sized type, but it can be
% hard to read this much text in a small size.
% As a result, it's been changed to normally-sized text.
\label{nodd}
You may wonder if it is possible to develop standard
mathematics in the Metamath language without the \texttt{\$d}\index{\texttt{\$d}
statement} statement, since it seems like a nuisance that complicates proof
verification. The \texttt{\$d} statement is not needed in certain subsets of
mathematics such as propositional calculus.  However, dummy
variables\index{dummy variable!eliminating} and their associated \texttt{\$d}
statements are impossible to avoid in proofs in standard first-order logic as
well as in the variant used in \texttt{set.mm}.  In fact, there is no upper bound to
the number of dummy variables that might be needed in a proof of a theorem of
first-order logic containing 3 or more variables, as shown by H.\
Andr\'{e}ka\index{Andr{\'{e}}ka, H.} \cite{Nemeti}.  A first-order system that
avoids them entirely is given in \cite{Megill}\index{Megill, Norman}; the
trick there is simply to embed harmlessly the necessary dummy variables into a
theorem being proved so that they aren't ``dummy'' anymore, then interpret the
resulting longer theorem so as to ignore the embedded dummy variables.  If
this interests you, the system in \texttt{set.mm} obtained from \texttt{ax-1}
through \texttt{ax-c14} in \texttt{set.mm}, and deleting \texttt{ax-c16} and \texttt{ax-5},
requires no \texttt{\$d} statements but is logically complete in the sense
described in \cite{Megill}.  This means it can prove any theorem of
first-order logic as long as we add to the theorem an antecedent that embeds
dummy and any other variables that must be distinct.  In a similar fashion,
axioms for set theory can be devised that
do not require distinct variable
provisos\index{Set theory without distinct variable provisos},
as explained at
\url{http://us.metamath.org/mpeuni/mmzfcnd.html}.
Together, these in principle allow all of
mathematics to be developed under Metamath without a \texttt{\$d} statement,
although the length of the resulting theorems will grow as more and
more dummy variables become required in their proofs.

\subsection{The \texttt{\$f}
and \texttt{\$e} Statements}\label{dollaref}
\index{\texttt{\$e} statement}
\index{\texttt{\$f} statement}
\index{floating hypothesis}
\index{essential hypothesis}
\index{variable-type hypothesis}
\index{logical hypothesis}
\index{hypothesis}

Metamath has two kinds of hypo\-theses, the \texttt{\$f}\index{\texttt{\$f}
statement} or {\bf variable-type} hypothesis and the \texttt{\$e} or {\bf logical}
hypo\-the\-sis.\index{\texttt{\$d} statement}\footnote{Strictly speaking, the
\texttt{\$d} statement is also a hypothesis, but it is never directly referenced
in a proof, so we call it a restriction rather than a hypothesis to lessen
confusion.  The checking for violations of \texttt{\$d} restrictions is automatic
and built into Metamath's proof-checking algorithm.} The letters \texttt{f} and
\texttt{e} stand for ``floating''\index{floating hypothesis} (roughly meaning
used only if relevant) and ``essential''\index{essential hypothesis} (meaning
always used) respectively, for reasons that will become apparent
when we discuss frames in
Section~\ref{frames} and scoping in Section~\ref{scoping}. The syntax of these
are as follows:
\begin{center}
  {\em label} \texttt{\$f} {\em typecode} {\em variable} \texttt{\$.}\\
  {\em label} \texttt{\$e} {\em typecode}
      {\em math-symbol}\ \,$\cdots$\ {\em math-symbol} \texttt{\$.}\\
\end{center}
\index{\texttt{\$e} statement}
\index{\texttt{\$f} statement}
A hypothesis must have a {\em label}\index{label}.  The expression in a
\texttt{\$e} hypothesis consists of a typecode (an active constant math symbol)
followed by a sequence
of zero or more math symbols. Each math symbol (including {\em constant}
and {\em variable}) must be a previously declared constant or variable.  (In
addition, each math symbol must be active, which will be covered when we
discuss scoping statements in Section~\ref{scoping}.)  You use a \texttt{\$f}
hypothesis to specify the
nature or {\bf type}\index{variable type}\index{type} of a variable (such as ``let $x$ be an
integer'') and use a \texttt{\$e} hypothesis to express a logical truth (such as
``assume $x$ is prime'') that must be established in order for an assertion
requiring it to also be true.

A variable must have its type specified in a \texttt{\$f} statement before
it may be used in a \texttt{\$e}, \texttt{\$a}, or \texttt{\$p}
statement.  There may be only one (active) \texttt{\$f} statement for a
given variable.  (``Active'' is defined in Section~\ref{scoping}.)

In ordinary mathematics, theorems\index{theorem} are often expressed in the
form ``Assume $P$; then $Q$,'' where $Q$ is a statement that you can derive
if you start with statement $P$.\index{free variable}\footnote{A stronger
version of a theorem like this would be the {\em single} formula $P\rightarrow
Q$ ($P$ implies $Q$) from which the weaker version above follows by the rule
of modus ponens in logic.  We are not discussing this stronger form here.  In
the weaker form, we are saying only that if we can {\em prove} $P$, then we can
{\em prove} $Q$.  In a logician's language, if $x$ is the only free variable
in $P$ and $Q$, the stronger form is equivalent to $\forall x ( P \rightarrow
Q)$ (for all $x$, $P$ implies $Q$), whereas the weaker form is equivalent to
$\forall x P \rightarrow \forall x Q$. The stronger form implies the weaker,
but not vice-versa.  To be precise, the weaker form of the theorem is more
properly called an ``inference'' rather than a theorem.}\index{inference}
In the
Metamath\index{Metamath} language, you would express mathematical statement
$P$ as a hypothesis (a \texttt{\$e} Metamath language statement in this case) and
statement $Q$ as a provable assertion (a \texttt{\$p}\index{\texttt{\$p} statement}
statement).

Some examples of hypotheses you might encounter in logic and set theory are
\begin{center}
  \texttt{stmt1 \$f wff P \$.}\\
  \texttt{stmt2 \$f setvar x \$.}\\
  \texttt{stmt3 \$e |- ( P -> Q ) \$.}
\end{center}
\index{\texttt{\$e} statement}
\index{\texttt{\$f} statement}
Informally, these would be read, ``Let $P$ be a well-formed-formula,'' ``Let
$x$ be an (individual) variable,'' and ``Assume we have proved $P \rightarrow
Q$.''  The turnstile symbol \,$\vdash$\index{turnstile ({$\,\vdash$})} is
commonly used in logic texts to mean ``a proof exists for.''

To summarize:
\begin{itemize}
\item A \texttt{\$f} hypothesis tells Metamath the type or kind of its variable.
It is analogous to a variable declaration in a computer language that
tells the compiler that a variable is an integer or a floating-point
number.
\item The \texttt{\$e} hypothesis corresponds to what you would usually call a
``hypothesis'' in ordinary mathematics.
\end{itemize}

Before an assertion\index{assertion} (\texttt{\$a} or \texttt{\$p} statement) can be
referenced in a proof, all of its associated \texttt{\$f} and \texttt{\$e} hypotheses
(i.e.\ those \texttt{\$e} hypotheses that are active) must be satisfied (i.e.
established by the proof).  The meaning of ``associated'' (which we will call
{\bf mandatory} in Section~\ref{frames}) will become clear when we discuss
scoping later.

Note that after any \texttt{\$f}, \texttt{\$e},
\texttt{\$a}, or \texttt{\$p} token there is a required
\textit{typecode}\index{typecode}.
The typecode is a constant used to enforce types of expressions.
This will become clearer once we learn more about
assertions (\texttt{\$a} and \texttt{\$p} statements).
An example may also clarify their purpose.
In the
\texttt{set.mm}\index{set theory database (\texttt{set.mm})}%
\index{Metamath Proof Explorer}
database,
the following typecodes are used:

\begin{itemize}
\item \texttt{wff} :
  Well-formed formula (wff) symbol
  (read: ``the following symbol sequence is a wff'').
% The *textual* typecode for turnstile is "|-", but when read it's a little
% confusing, so I intentionally display the mathematical symbol here instead
% (I think it's clearer in this context).
\item \texttt{$\vdash$} :
  Turnstile (read: ``the following symbol sequence is provable'' or
  ``a proof exists for'').
\item \texttt{setvar} :
  Individual set variable type (read: ``the following is an
  individual set variable'').
  Note that this is \textit{not} the type of an arbitrary set expression,
  instead, it is used to ensure that there is only a single symbol used
  after quantifiers like for-all ($\forall$) and there-exists ($\exists$).
\item \texttt{class} :
  An expression that is a syntactically valid class expression.
  All valid set expressions are also valid class expression, so expressions
  of sets normally have the \texttt{class} typecode.
  Use the \texttt{class} typecode,
  \textit{not} the \texttt{setvar} typecode,
  for the type of set expressions unless you are specifically identifying
  a single set variable.
\end{itemize}

\subsection{Assertions (\texttt{\$a} and \texttt{\$p} Statements)}
\index{\texttt{\$a} statement}
\index{\texttt{\$p} statement}\index{assertion}\index{axiomatic assertion}
\index{provable assertion}

There are two types of assertions, \texttt{\$a}\index{\texttt{\$a} statement}
statements ({\bf axiomatic assertions}) and \texttt{\$p} statements ({\bf
provable assertions}).  Their syntax is as follows:
\begin{center}
  {\em label} \texttt{\$a} {\em typecode} {\em math-symbol} \ldots
         {\em math-symbol} \texttt{\$.}\\
  {\em label} \texttt{\$p} {\em typecode} {\em math-symbol} \ldots
        {\em math-symbol} \texttt{\$=} {\em proof} \texttt{\$.}
\end{center}
\index{\texttt{\$a} statement}
\index{\texttt{\$p} statement}
\index{\texttt{\$=} keyword}
An assertion always requires a {\em label}\index{label}. The expression in an
assertion consists of a typecode (an active constant)
followed by a sequence of zero
or more math symbols.  Each math symbol, including any {\em constant}, must be a
previously declared constant or variable.  (In addition, each math symbol
must be active, which will be covered when we discuss scoping statements in
Section~\ref{scoping}.)

A \texttt{\$a} statement is usually a definition of syntax (for example, if $P$
and $Q$ are wffs then so is $(P\to Q)$), an axiom\index{axiom} of ordinary
mathematics (for example, $x=x$), or a definition\index{definition} of
ordinary mathematics (for example, $x\ne y$ means $\lnot x=y$). A \texttt{\$p}
statement is a claim that a certain combination of math symbols follows from
previous assertions and is accompanied by a proof that demonstrates it.

Assertions can also be referenced in (later) proofs in order to derive new
assertions from them. The label of an assertion is used to refer to it in a
proof. Section~\ref{proof} will describe the proof in detail.

Assertions also provide the primary means for communicating the mathematical
results in the database to people.  Proofs (when conveniently displayed)
communicate to people how the results were arrived at.

\subsubsection{The \texttt{\$a} Statement}
\index{\texttt{\$a} statement}

Axiomatic assertions (\texttt{\$a} statements) represent the starting points from
which other assertions (\texttt{\$p}\index{\texttt{\$p} statement} statements) are
derived.  Their most obvious use is for specifying ordinary mathematical
axioms\index{axiom}, but they are also used for two other purposes.

First, Metamath\index{Metamath} needs to know the syntax of symbol
sequences that constitute valid mathematical statements.  A Metamath
proof must be broken down into much more detail than ordinary
mathematical proofs that you may be used to thinking of (even the
``complete'' proofs of formal logic\index{formal logic}).  This is one
of the things that makes Metamath a general-purpose language,
independent of any system of logic or even syntax.  If you want to use a
substitution instance of an assertion as a step in a proof, you must
first prove that the substitution is syntactically correct (or if you
prefer, you must ``construct'' it), showing for example that the
expression you are substituting for a wff metavariable is a valid wff.
The \texttt{\$a}\index{\texttt{\$a} statement} statement is used to
specify those combinations of symbols that are considered syntactically
valid, such as the legal forms of wffs.

Second, \texttt{\$a} statements are used to specify what are ordinarily thought of
as definitions, i.e.\ new combinations of symbols that abbreviate other
combinations of symbols.  Metamath makes no distinction\index{axiom vs.\
definition} between axioms\index{axiom} and definitions\index{definition}.
Indeed, it has been argued that such distinction should not be made even in
ordinary mathematics; see Section~\ref{definitions}, which discusses the
philosophy of definitions.  Section~\ref{hierarchy} discusses some
technical requirements for definitions.  In \texttt{set.mm} we adopt the
convention of prefixing axiom labels with \texttt{ax-} and definition labels with
\texttt{df-}\index{label}.

The results that can be derived with the Metamath language are only as good as
the \texttt{\$a}\index{\texttt{\$a} statement} statements used as their starting
point.  We cannot stress this too strongly.  For example, Metamath will
not prevent you from specifying $x\neq x$ as an axiom of logic.  It is
essential that you scrutinize all \texttt{\$a} statements with great care.
Because they are a source of potential pitfalls, it is best not to add new
ones (usually new definitions) casually; rather you should carefully evaluate
each one's necessity and advantages.

Once you have in place all of the basic axioms\index{axiom} and
rules\index{rule} of a mathematical theory, the only \texttt{\$a} statements that
you will be adding will be what are ordinarily called definitions.  In
principle, definitions should be in some sense eliminable from the language of
a theory according to some convention (usually involving logical equivalence
or equality).  The most common convention is that any formula that was
syntactically valid but not provable before the definition was introduced will
not become provable after the definition is introduced.  In an ideal world,
definitions should not be present at all if one is to have absolute confidence
in a mathematical result.  However, they are necessary to make
mathematics practical, for otherwise the resulting formulas would be
extremely long and incomprehensible.  Since the nature of definitions (in the
most general sense) does not permit them to automatically be verified as
``proper,''\index{proper definition}\index{definition!proper} the judgment of
the mathematician is required to ensure it.  (In \texttt{set.mm} effort was made
to make almost all definitions directly eliminable and thus minimize the need
for such judgment.)

If you are not a mathematician, it may be best not to add or change any
\texttt{\$a}\index{\texttt{\$a} statement} statements but instead use
the mathematical language already provided in standard databases.  This
way Metamath will not allow you to make a mistake (i.e.\ prove a false
result).


\subsection{Frames}\label{frames}

We now introduce the concept of a collection of related Metamath statements
called a frame.  Every assertion (\texttt{\$a} or \texttt{\$p} statement) in the database has
an associated frame.

A {\bf frame}\index{frame} is a sequence of \texttt{\$d}, \texttt{\$f},
and \texttt{\$e} statements (zero or more of each) followed by one
\texttt{\$a} or \texttt{\$p} statement, subject to certain conditions we
will describe.  For simplicity we will assume that all math symbol
tokens used are declared at the beginning of the database with
\texttt{\$c} and \texttt{\$v} statements (which are not properly part of
a frame).  Also for simplicity we will assume there are only simple
\texttt{\$d} statements (those with only two variables) and imagine any
compound \texttt{\$d} statements (those with more than two variables) as
broken up into simple ones.

A frame groups together those hypotheses (and \texttt{\$d} statements) relevant
to an assertion (\texttt{\$a} or \texttt{\$p} statement).  The statements in a frame
may or may not be physically adjacent in a database; we will cover
this in our discussion of scoping statements
in Section~\ref{scoping}.

A frame has the following properties:
\begin{enumerate}
 \item The set of variables contained in its \texttt{\$f} statements must
be identical to the set of variables contained in its \texttt{\$e},
\texttt{\$a}, and/or \texttt{\$p} statements.  In other words, each
variable in a \texttt{\$e}, \texttt{\$a}, or \texttt{\$p} statement must
have an associated ``variable type'' defined for it in a \texttt{\$f}
statement.
  \item No two \texttt{\$f} statements may contain the same variable.
  \item Any \texttt{\$f} statement
must occur before a \texttt{\$e} statement in which its variable occurs.
\end{enumerate}

The first property determines the set of variables occurring in a frame.
These are the {\bf mandatory
variables}\index{mandatory variable} of the frame.  The second property
tells us there must be only one type specified for a variable.
The last property is not a theoretical requirement but it
makes parsing of the database easier.

For our examples, we assume our database has the following declarations:

\begin{verbatim}
$v P Q R $.
$c -> ( ) |- wff $.
\end{verbatim}

The following sequence of statements, describing the modus ponens inference
rule, is an example of a frame:

\begin{verbatim}
wp  $f wff P $.
wq  $f wff Q $.
maj $e |- ( P -> Q ) $.
min $e |- P $.
mp  $a |- Q $.
\end{verbatim}

The following sequence of statements is not a frame because \texttt{R} does not
occur in the \texttt{\$e}'s or the \texttt{\$a}:

\begin{verbatim}
wp  $f wff P $.
wq  $f wff Q $.
wr  $f wff R $.
maj $e |- ( P -> Q ) $.
min $e |- P $.
mp  $a |- Q $.
\end{verbatim}

The following sequence of statements is not a frame because \texttt{Q} does not
occur in a \texttt{\$f}:

\begin{verbatim}
wp  $f wff P $.
maj $e |- ( P -> Q ) $.
min $e |- P $.
mp  $a |- Q $.
\end{verbatim}

The following sequence of statements is not a frame because the \texttt{\$a} statement is
not the last one:

\begin{verbatim}
wp  $f wff P $.
wq  $f wff Q $.
maj $e |- ( P -> Q ) $.
mp  $a |- Q $.
min $e |- P $.
\end{verbatim}

Associated with a frame is a sequence of {\bf mandatory
hypotheses}\index{mandatory hypothesis}.  This is simply the set of all
\texttt{\$f} and \texttt{\$e} statements in the frame, in the order they
appear.  A frame can be referenced in a later proof using the label of
the \texttt{\$a} or \texttt{\$p} assertion statement, and the proof
makes an assignment to each mandatory hypothesis in the order in which
it appears.  This means the order of the hypotheses, once chosen, must
not be changed so as not to affect later proofs referencing the frame's
assertion statement.  (The Metamath proof verifier will, of course, flag
an error if a proof becomes incorrect by doing this.)  Since proofs make
use of ``Reverse Polish notation,'' described in Section~\ref{proof}, we
call this order the {\bf RPN order}\index{RPN order} of the hypotheses.

Note that \texttt{\$d} statements are not part of the set of mandatory
hypotheses, and their order doesn't matter (as long as they satisfy the
fourth property for a frame described above).  The \texttt{\$d}
statements specify restrictions on variables that must be satisfied (and
are checked by the proof verifier) when expressions are substituted for
them in a proof, and the \texttt{\$d} statements themselves are never
referenced directly in a proof.

A frame with a \texttt{\$p} (provable) statement requires a proof as part of the
\texttt{\$p} statement.  Sometimes in a proof we want to make use of temporary or
dummy variables\index{dummy variable} that do not occur in the \texttt{\$p}
statement or its mandatory hypotheses.  To accommodate this we define an {\bf
extended frame}\index{extended frame} as a frame together with zero or more
\texttt{\$d} and \texttt{\$f} statements that reference variables not among the
mandatory variables of the frame.  Any new variables referenced are called the
{\bf optional variables}\index{optional variable} of the extended frame. If a
\texttt{\$f} statement references an optional variable it is called an {\bf
optional hypothesis}\index{optional hypothesis}, and if one or both of the
variables in a \texttt{\$d} statement are optional variables it is called an {\bf
optional disjoint-variable restriction}\index{optional disjoint-variable
restriction}.  Properties 2 and 3 for a frame also apply to an extended
frame.

The concept of optional variables is not meaningful for frames with \texttt{\$a}
statements, since those statements have no proofs that might make use of them.
There is no restriction on including optional hypotheses in the extended frame
for a \texttt{\$a} statement, but they serve no purpose.

The following set of statements is an example of an extended frame, which
contains an optional variable \texttt{R} and an optional hypothesis \texttt{wr}.  In
this example, we suppose the rule of modus ponens is not an axiom but is
derived as a theorem from earlier statements (we omit its presumed proof).
Variable \texttt{R} may be used in its proof if desired (although this would
probably have no advantage in propositional calculus).  Note that the sequence
of mandatory hypotheses in RPN order is still \texttt{wp}, \texttt{wq}, \texttt{maj},
\texttt{min} (i.e.\ \texttt{wr} is omitted), and this sequence is still assumed
whenever the assertion \texttt{mp} is referenced in a subsequent proof.

\begin{verbatim}
wp  $f wff P $.
wq  $f wff Q $.
wr  $f wff R $.
maj $e |- ( P -> Q ) $.
min $e |- P $.
mp  $p |- Q $= ... $.
\end{verbatim}

Every frame is an extended frame, but not every extended frame is a frame, as
this example shows.  The underlying frame for an extended frame is
obtained by simply removing all statements containing optional variables.
Any proof referencing an assertion will ignore any extensions to its
frame, which means we may add or delete optional hypotheses at will without
affecting subsequent proofs.

The conceptually simplest way of organizing a Metamath database is as a
sequence of extended frames.  The scoping statements
\texttt{\$\char`\{}\index{\texttt{\$\char`\{} and \texttt{\$\char`\}}
keywords} and \texttt{\$\char`\}} can be used to delimit the start and
end of an extended frame, leading to the following possible structure for a
database.  \label{framelist}

\vskip 2ex
\setbox\startprefix=\hbox{\tt \ \ \ \ \ \ \ \ }
\setbox\contprefix=\hbox{}
\startm
\m{\mbox{(\texttt{\$v} {\em and} \texttt{\$c}\,{\em statements})}}
\endm
\startm
\m{\mbox{\texttt{\$\char`\{}}}
\endm
\startm
\m{\mbox{\texttt{\ \ } {\em extended frame}}}
\endm
\startm
\m{\mbox{\texttt{\$\char`\}}}}
\endm
\startm
\m{\mbox{\texttt{\$\char`\{}}}
\endm
\startm
\m{\mbox{\texttt{\ \ } {\em extended frame}}}
\endm
\startm
\m{\mbox{\texttt{\$\char`\}}}}
\endm
\startm
\m{\mbox{\texttt{\ \ \ \ \ \ \ \ \ }}\vdots}
\endm
\vskip 2ex

In practice, this structure is inconvenient because we have to repeat
any \texttt{\$f}, \texttt{\$e}, and \texttt{\$d} statements over and
over again rather than stating them once for use by several assertions.
The scoping statements, which we will discuss next, allow this to be
done.  In principle, any Metamath database can be converted to the above
format, and the above format is the most convenient to use when studying
a Metamath database as a formal system%
%% Uncomment this when uncommenting section {formalspec} below
   (Appendix \ref{formalspec})%
.
In fact, Metamath internally converts the database to the above format.
The command \texttt{show statement} in the Metamath program will show
you the contents of the frame for any \texttt{\$a} or \texttt{\$p}
statement, as well as its extension in the case of a \texttt{\$p}
statement.

%c%(provided that all ``local'' variables and constants with limited scope have
%c%unique names),

During our discussion of scoping statements, it may be helpful to
think in terms of the equivalent sequence of frames that will result when
the database is parsed.  Scoping (other than the limited
use above to delimit frames) is not a theoretical requirement for
Metamath but makes it more convenient.


\subsection{Scoping Statements (\texttt{\$\{} and \texttt{\$\}})}\label{scoping}
\index{\texttt{\$\char`\{} and \texttt{\$\char`\}} keywords}\index{scoping statement}

%c%Some Metamath statements may be needed only temporarily to
%c%serve a specific purpose, and after we're done with them we would like to
%c%disregard or ignore them.  For example, when we're finished using a variable,
%c%we might want to
%c%we might want to free up the token\index{token} used to name it so that the
%c%token can be used for other purposes later on, such as a different kind of
%c%variable or even a constant.  In the terminology of computer programming, we
%c%might want to let some symbol declarations be ``local'' rather than ``global.''
%c%\index{local symbol}\index{global symbol}

The {\bf scoping} statements, \texttt{\$\char`\{} ({\bf start of block}) and \texttt{\$\char`\}}
({\bf end of block})\index{block}, provide a means for controlling the portion
of a database over which certain statement types are recognized.  The
syntax of a scoping statement is very simple; it just consists of the
statement's keyword:
\begin{center}
\texttt{\$\char`\{}\\
\texttt{\$\char`\}}
\end{center}
\index{\texttt{\$\char`\{} and \texttt{\$\char`\}} keywords}

For example, consider the following database where we have stripped out
all tokens except the scoping statement keywords.  For the purpose of the
discussion, we have added subscripts to the scoping statements; these subscripts
do not appear in the actual database.
\[
 \mbox{\tt \ \$\char`\{}_1
 \mbox{\tt \ \$\char`\{}_2
 \mbox{\tt \ \$\char`\}}_2
 \mbox{\tt \ \$\char`\{}_3
 \mbox{\tt \ \$\char`\{}_4
 \mbox{\tt \ \$\char`\}}_4
 \mbox{\tt \ \$\char`\}}_3
 \mbox{\tt \ \$\char`\}}_1
\]
Each \texttt{\$\char`\{} statement in this example is said to be {\bf
matched} with the \texttt{\$\char`\}} statement that has the same
subscript.  Each pair of matched scoping statements defines a region of
the database called a {\bf block}.\index{block} Blocks can be {\bf
nested}\index{nested block} inside other blocks; in the example, the
block defined by $\mbox{\tt \$\char`\{}_4$ and $\mbox{\tt \$\char`\}}_4$
is nested inside the block defined by $\mbox{\tt \$\char`\{}_3$ and
$\mbox{\tt \$\char`\}}_3$ as well as inside the block defined by
$\mbox{\tt \$\char`\{}_1$ and $\mbox{\tt \$\char`\}}_1$.  In general, a
block may be empty, it may contain only non-scoping
statements,\footnote{Those statements other than \texttt{\$\char`\{} and
\texttt{\$\char`\}}.}\index{non-scoping statement} or it may contain any
mixture of other blocks and non-scoping statements.  (This is called a
``recursive'' definition\index{recursive definition} of a block.)

Associated with each block is a number called its {\bf nesting
level}\index{nesting level} that indicates how deeply the block is nested.
The nesting levels of the blocks in our example are as follows:
\[
  \underbrace{
    \mbox{\tt \ }
    \underbrace{
     \mbox{\tt \$\char`\{\ }
     \underbrace{
       \mbox{\tt \$\char`\{\ }
       \mbox{\tt \$\char`\}}
     }_{2}
     \mbox{\tt \ }
     \underbrace{
       \mbox{\tt \$\char`\{\ }
       \underbrace{
         \mbox{\tt \$\char`\{\ }
         \mbox{\tt \$\char`\}}
       }_{3}
       \mbox{\tt \ \$\char`\}}
     }_{2}
     \mbox{\tt \ \$\char`\}}
   }_{1}
   \mbox{\tt \ }
 }_{0}
\]
\index{\texttt{\$\char`\{} and \texttt{\$\char`\}} keywords}
The entire database is considered to be one big block (the {\bf outermost}
block) with a nesting level of 0.  The outermost block is {\em not} bracketed
by scoping statements.\footnote{The language was designed this way so that
several source files can be joined together more easily.}\index{outermost
block}

All non-scoping Metamath statements become recognized or {\bf
active}\index{active statement} at the place where they appear.\footnote{To
keep things slightly simpler, we do not bother to define the concept of
``active'' for the scoping statements.}  Certain of these statement types
become inactive at the end of the block in which they appear; these statement
types are:
\begin{center}
  \texttt{\$c}, \texttt{\$v}, \texttt{\$d}, \texttt{\$e}, and \texttt{\$f}.
%  \texttt{\$v}, \texttt{\$f}, \texttt{\$e}, and \texttt{\$d}.
\end{center}
\index{\texttt{\$c} statement}
\index{\texttt{\$d} statement}
\index{\texttt{\$e} statement}
\index{\texttt{\$f} statement}
\index{\texttt{\$v} statement}
The other statement types remain active forever (i.e.\ through the end of the
database); they are:
\begin{center}
  \texttt{\$a} and \texttt{\$p}.
%  \texttt{\$c}, \texttt{\$a}, and \texttt{\$p}.
\end{center}
\index{\texttt{\$a} statement}
\index{\texttt{\$p} statement}
Any statement (of these 7 types) located in the outermost
block\index{outermost block} will remain active through the end of the
database and thus are effectively ``global'' statements.\index{global
statement}

All \texttt{\$c} statements must be placed in the outermost block.  Since they are
therefore always global, they could be considered as belonging to both of the
above categories.

The {\bf scope}\index{scope} of a statement is the set of statements that
recognize it as active.

%c%The concept of ``active'' is also defined for math symbols\index{math
%c%symbol}.  Math symbols (constants\index{constant} and
%c%variables\index{variable}) become {\bf active}\index{active
%c%math symbol} in the \texttt{\$c}\index{\texttt{\$c}
%c%statement} and \texttt{\$v}\index{\texttt{\$v} statement} statements that
%c%declare them.  They become inactive when their declaration statements become
%c%inactive.

The concept of ``active'' is also defined for math symbols\index{math
symbol}.  Math symbols (constants\index{constant} and
variables\index{variable}) become {\bf active}\index{active math symbol}
in the \texttt{\$c}\index{\texttt{\$c} statement} and
\texttt{\$v}\index{\texttt{\$v} statement} statements that declare them.
A variable becomes inactive when its declaration statement becomes
inactive.  Because all \texttt{\$c} statements must be in the outermost
block, a constant will never become inactive after it is declared.

\subsubsection{Redeclaration of Math Symbols}
\index{redeclaration of symbols}\label{redeclaration}

%c%A math symbol may not be declared a second time while it is active, but it may
%c%be declared again after it becomes inactive.

A variable may not be declared a second time while it is active, but it may be
declared again after it becomes inactive.  This provides a convenient way to
introduce ``local'' variables,\index{local variable} i.e.\ temporary variables
for use in the frame of an assertion or in a proof without keeping them around
forever.  A previously declared variable may not be redeclared as a constant.

A constant may not be redeclared.  And, as mentioned above, constants must be
declared in the outermost block.

The reason variables may have limited scope but not constants is that an
assertion (\texttt{\$a} or \texttt{\$p} statement) remains available for use in
proofs through the end of the database.  Variables in an assertion's frame may
be substituted with whatever is needed in a proof step that references the
assertion, whereas constants remain fixed and may not be substituted with
anything.  The particular token used for a variable in an assertion's frame is
irrelevant when the assertion is referenced in a proof, and it doesn't matter
if that token is not available outside of the referenced assertion's frame.
Constants, however, must be globally fixed.

There is no theoretical
benefit for the feature allowing variables to be active for limited scopes
rather than global. It is just a convenience that allows them, for example, to
be locally grouped together with their corresponding \texttt{\$f} variable-type
declarations.

%c%If you declare a math symbol more than once, internally Metamath considers it a
%c%new distinct symbol, even though it has the same name.  If you are unaware of
%c%this, you may find that what you think are correct proofs are incorrectly
%c%rejected as invalid, because Metamath may tell you that a constant you
%c%previously declared does not match a newly declared math symbol with the same
%c%name.  For details on this subtle point, see the Comment on
%c%p.~\pageref{spec4comment}.  This is done purposely to allow temporary
%c%constants to be introduced while developing a subtheory, then allow their math
%c%symbol tokens to be reused later on; in general they will not refer to the
%c%same thing.  In practice, you would not ordinarily reuse the names of
%c%constants because it would tend to be confusing to the reader.  The reuse of
%c%names of variables, on the other hand, is something that is often useful to do
%c%(for example it is done frequently in \texttt{set.mm}).  Since variables in an
%c%assertion referenced in a proof can be substituted as needed to achieve a
%c%symbol match, this is not an issue.

% (This section covers a somewhat advanced topic you may want to skip
% at first reading.)
%
% Under certain circumstances, math symbol\index{math symbol}
% tokens\index{token} may be redeclared (i.e.\ the token
% may appear in more than
% one \texttt{\$c}\index{\texttt{\$c} statement} or \texttt{\$v}\index{\texttt{\$v}
% statement} statement).  You might want to do this say, to make temporary use
% of a variable name without having to worry about its affect elsewhere,
% somewhat analogous to declaring a local variable in a standard computer
% language.  Understanding what goes on when math symbol tokens are redeclared
% is a little tricky to understand at first, since it requires that we
% distinguish the token itself from the math symbol that it names.  It will help
% if we first take a peek at the internal workings of the
% Metamath\index{Metamath} program.
%
% Metamath reserves a memory location for each occurrence of a
% token\index{token} in a declaration statement (\texttt{\$c}\index{\texttt{\$c}
% statement} or \texttt{\$v}\index{\texttt{\$v} statement}).  If a given token appears
% in more than one declaration statement, it will refer to more than one memory
% locations.  A math symbol\index{math symbol} may be thought of as being one of
% these memory locations rather than as the token itself.  Only one of the
% memory locations associated with a given token may be active at any one time.
% The math symbol (memory location) that gets looked up when the token appears
% in a non-declaration statement is the one that happens to be active at that
% time.
%
% We now look at the rules for the redeclaration\index{redeclaration of symbols}
% of math symbol tokens.
% \begin{itemize}
% \item A math symbol token may not be declared twice in the
% same block.\footnote{While there is no theoretical reason for disallowing
% this, it was decided in the design of Metamath that allowing it would offer no
% advantage and might cause confusion.}
% \item An inactive math symbol may always be
% redeclared.
% \item  An active math symbol may be redeclared in a different (i.e.\
% inner) block\index{block} from the one it became active in.
% \end{itemize}
%
% When a math symbol token is redeclared, it conceptually refers to a different
% math symbol, just as it would be if it were called a different name.  In
% addition, the original math symbol that it referred to, if it was active,
% temporarily becomes inactive.  At the end of the block in which the
% redeclaration occurred, the new math symbol\index{math symbol} becomes
% inactive and the original symbol becomes active again.  This concept is
% illustrated in the following example, where the symbol \texttt{e} is
% ordinarily a constant (say Euler's constant, 2.71828...) but
% temporarily we want to use it as a ``local'' variable, say as a coefficient
% in the equation $a x^4 + b x^3 + c x^2 + d x + e$:
% \[
%   \mbox{\tt \$\char`\{\ \$c e \$.}
%   \underbrace{
%     \ \ldots\ %
%     \mbox{\tt \$\char`\{}\ \ldots\ %
%   }_{\mbox{\rm region A}}
%   \mbox{\tt \$v e \$.}
%   \underbrace{
%     \mbox{\ \ \ \ldots\ \ \ }
%   }_{\mbox{\rm region B}}
%   \mbox{\tt \$\char`\}}
%   \underbrace{
%     \mbox{\ \ \ \ldots\ \ \ }
%   }_{\mbox{\rm region C}}
%   \mbox{\tt \$\char`\}}
% \]
% \index{\texttt{\$\char`\{} and \texttt{\$\char`\}} keywords}
% In region A, the token \texttt{e} refers to a constant.  It is redeclared as a
% variable in region B, and any reference to it in this region will refer to this
% variable.  In region C, the redeclaration becomes inactive, and the original
% declaration becomes active again.  In region C, the token \texttt{x} refers to the
% original constant.
%
% As a practical matter, overuse of math symbol\index{math symbol}
% redeclarations\index{redeclaration of symbols} can be confusing (even though
% it is well-defined) and is best avoided when possible.  Here are some good
% general guidelines you can follow.  Usually, you should declare all
% constants\index{constant} in the outermost block\index{outermost block},
% especially if they are general-purpose (such as the token \verb$A.$, meaning
% $\forall$ or ``for all'').  This will make them ``globally'' active (although
% as in the example above local redeclarations will temporarily make them
% inactive.)  Most or all variables\index{variable}, on the other hand, could be
% declared in inner blocks, so that the token for them can be used later for a
% different type of variable or a constant.  (The names of the variables you
% choose are not used when you refer to an assertion\index{assertion} in a
% proof, whereas constants must match exactly.  A locally declared constant will
% not match a globally declared constant in a proof, even if they use the same
% token, because Metamath internally considers them to be different math
% symbols.)  To avoid confusion, you should generally avoid redeclaring active
% variables.  If you must redeclare them, do so at the beginning of a block.
% The temporary declaration of constants in inner blocks might be occasionally
% appropriate when you make use of a temporary definition to prove lemmas
% leading to a main result that does not make direct use of the definition.
% This way, you will not clutter up your database with a large number of
% seldom-used global constant symbols.  You might want to note that while
% inactive constants may not appear directly in an assertion (a \texttt{\$a}\index{\texttt{\$a}
% statement} or \texttt{\$p}\index{\texttt{\$p} statement}
% statement), they may be indirectly used in the proof of a \texttt{\$p} statement
% so long as they do not appear in the final math symbol sequence constructed by
% the proof.  In the end, you will have to use your best judgment, taking into
% account standard mathematical usage of the symbols as well as consideration
% for the reader of your work.
%
% \subsubsection{Reuse of Labels}\index{reuse of labels}\index{label}
%
% The \texttt{\$e}\index{\texttt{\$e} statement}, \texttt{\$f}\index{\texttt{\$f}
% statement}, \texttt{\$a}\index{\texttt{\$a} statement}, and
% \texttt{\$p}\index{\texttt{\$p}
% statement} statement types require labels, which allow them to be
% referenced later inside proofs.  A label is considered {\bf
% active}\index{active label} when the statement it is associated with is
% active.  The token\index{token} for a label may be reused
% (redeclared)\index{redeclaration of labels} provided that it is not being used
% for a currently active label.  (Unlike the tokens for math symbols, active
% label tokens may not be redeclared in an inner scope.)  Note that the labels
% of \texttt{\$a} and \texttt{\$p} statements can never be reused after these
% statements appear, because these statements remain active through the end of
% the database.
%
% You might find the reuse of labels a convenient way to have standard names for
% temporary hypotheses, such as \texttt{h1}, \texttt{h2}, etc.  This way you don't have
% to invent unique names for each of them, and in some cases it may be less
% confusing to the reader (although in other cases it might be more confusing, if
% the hypothesis is located far away from the assertion that uses
% it).\footnote{The current implementation requires that all labels, even
% inactive ones, be unique.}

\subsubsection{Frames Revisited}\index{frames and scoping statements}

Now that we have covered scoping, we will look at how an arbitrary
Metamath database can be converted to the simple sequence of extended
frames described on p.~\pageref{framelist}.  This is also how Metamath
stores the database internally when it reads in the database
source.\label{frameconvert} The method is simple.  First, we collect all
constant and variable (\texttt{\$c} and \texttt{\$v}) declarations in
the database, ignoring duplicate declarations of the same variable in
different scopes.  We then put our collected \texttt{\$c} and
\texttt{\$v} declarations at the beginning of the database, so that
their scope is the entire database.  Next, for each assertion in the
database, we determine its frame and extended frame.  The extended frame
is simply the \texttt{\$f}, \texttt{\$e}, and \texttt{\$d} statements
that are active.  The frame is the extended frame with all optional
hypotheses removed.

An equivalent way of saying this is that the extended frame of an assertion
is the collection of all \texttt{\$f}, \texttt{\$e}, and \texttt{\$d} statements
whose scope includes the assertion.
The \texttt{\$f} and \texttt{\$e} statements
occur in the order they appear
(order is irrelevant for \texttt{\$d} statements).

%c%, renaming any
%c%redeclared variables as needed so that all of them have unique names.  (The
%c%exact renaming convention is unimportant.  You might imagine renaming
%c%different declarations of math symbol \texttt{a} as \texttt{a\$1}, \texttt{a\$2}, etc.\
%c%which would prevent any conflicts since \texttt{\$} is not a legal character in a
%c%math symbol token.)

\section{The Anatomy of a Proof} \label{proof}
\index{proof!Metamath, description of}

Each provable assertion (\texttt{\$p}\index{\texttt{\$p} statement} statement) in a
database must include a {\bf proof}\index{proof}.  The proof is located
between the \texttt{\$=}\index{\texttt{\$=} keyword} and \texttt{\$.}\ keywords in the
\texttt{\$p} statement.

In the basic Metamath language\index{basic language}, a proof is a
sequence of statement labels.  This label sequence\index{label sequence}
serves as a set of instructions that the Metamath program uses to
construct a series of math symbol sequences.  The construction must
ultimately result in the math symbol sequence contained between the
\texttt{\$p}\index{\texttt{\$p} statement} and
\texttt{\$=}\index{\texttt{\$=} keyword} keywords of the \texttt{\$p}
statement.  Otherwise, the Metamath program will consider the proof
incorrect, and it will notify you with an appropriate error message when
you ask it to verify the proof.\footnote{To make the loading faster, the
Metamath program does not automatically verify proofs when you
\texttt{read} in a database unless you use the \texttt{/verify}
qualifier.  After a database has been read in, you may use the
\texttt{verify proof *} command to verify proofs.}\index{\texttt{verify
proof} command} Each label in a proof is said to {\bf
reference}\index{label reference} its corresponding statement.

Associated with any assertion\index{assertion} (\texttt{\$p} or
\texttt{\$a}\index{\texttt{\$a} statement} statement) is a set of
hypotheses (\texttt{\$f}\index{\texttt{\$f} statement} or
\texttt{\$e}\index{\texttt{\$e} statement} statements) that are active
with respect to that assertion.  Some are mandatory and the others are
optional.  You should review these concepts if necessary.

Each label\index{label} in a proof must be either the label of a
previous assertion (\texttt{\$a}\index{\texttt{\$a} statement} or
\texttt{\$p}\index{\texttt{\$p} statement} statement) or the label of an
active hypothesis (\texttt{\$e} or \texttt{\$f}\index{\texttt{\$f}
statement} statement) of the \texttt{\$p} statement containing the
proof.  Hypothesis labels may reference both the
mandatory\index{mandatory hypothesis} and the optional hypotheses of the
\texttt{\$p} statement.

The label sequence in a proof specifies a construction in {\bf reverse Polish
notation}\index{reverse Polish notation (RPN)} (RPN).  You may be familiar
with RPN if you have used older
Hewlett--Packard or similar hand-held calculators.
In the calculator analogy, a hypothesis label\index{hypothesis label} is like
a number and an assertion label\index{assertion label} is like an operation
(more precisely, an $n$-ary operation when the
assertion has $n$ \texttt{\$e}-hypotheses).
On an RPN calculator, an operation takes one or more previous numbers in an
input sequence, performs a calculation on them, and replaces those numbers and
itself with the result of the calculation.  For example, the input sequence
$2,3,+$ on an RPN calculator results in $5$, and the input sequence
$2,3,5,{\times},+$ results in $2,15,+$ which results in $17$.

Understanding how RPN is processed involves the concept of a {\bf
stack}\index{stack}\index{RPN stack}, which can be thought of as a set of
temporary memory locations that hold intermediate results.  When Metamath
encounters a hypothesis label it places or {\bf pushes}\index{push} the math
symbol sequence of the hypothesis onto the stack.  When Metamath encounters an
assertion label, it associates the most recent stack entries with the {\em
mandatory} hypotheses\index{mandatory hypothesis} of the assertion, in the
order where the most recent stack entry is associated with the last mandatory
hypothesis of the assertion.  It then determines what
substitutions\index{substitution!variable}\index{variable substitution} have
to be made into the variables of the assertion's mandatory hypotheses to make
them identical to the associated stack entries.  It then makes those same
substitutions into the assertion itself.  Finally, Metamath removes or {\bf
pops}\index{pop} the matched hypotheses from the stack and pushes the
substituted assertion onto the stack.

For the purpose of matching the mandatory hypothesis to the most recent stack
entries, whether a hypothesis is a \texttt{\$e} or \texttt{\$f} statement is
irrelevant.  The only important thing is that a set of
substitutions\footnote{In the Metamath spec (Section~\ref{spec}), we use the
singular term ``substitution'' to refer to the set of substitutions we talk
about here.} exist that allow a match (and if they don't, the proof verifier
will let you know with an error message).  The Metamath language is specified
in such a way that if a set of substitutions exists, it will be unique.
Specifically, the requirement that each variable have a type specified for it
with a \texttt{\$f} statement ensures the uniqueness.

We will illustrate this with an example.
Consider the following Metamath source file:
\begin{verbatim}
$c ( ) -> wff $.
$v p q r s $.
wp $f wff p $.
wq $f wff q $.
wr $f wff r $.
ws $f wff s $.
w2 $a wff ( p -> q ) $.
wnew $p wff ( s -> ( r -> p ) ) $= ws wr wp w2 w2 $.
\end{verbatim}
This Metamath source example shows the definition and ``proof'' (i.e.,
construction) of a well-formed formula (wff)\index{well-formed formula (wff)}
in propositional calculus.  (You may wish to type this example into a file to
experiment with the Metamath program.)  The first two statements declare
(introduce the names of) four constants and four variables.  The next four
statements specify the variable types, namely that
each variable is assumed to be a wff.  Statement \texttt{w2} defines (postulates)
a way to produce a new wff, \texttt{( p -> q )}, from two given wffs \texttt{p} and
\texttt{q}. The mandatory hypotheses of \texttt{w2} are \texttt{wp} and \texttt{wq}.
Statement \texttt{wnew} claims that \texttt{( s -> ( r -> p ) )} is a wff given
three wffs \texttt{s}, \texttt{r}, and \texttt{p}.  More precisely, \texttt{wnew} claims
that the sequence of ten symbols \texttt{wff ( s -> ( r -> p ) )} is provable from
previous assertions and the hypotheses of \texttt{wnew}.  Metamath does not know
or care what a wff is, and as far as it is concerned
the typecode \texttt{wff} is just an
arbitrary constant symbol in a math symbol sequence.  The mandatory hypotheses
of \texttt{wnew} are \texttt{wp}, \texttt{wr}, and \texttt{ws}; \texttt{wq} is an optional
hypothesis.  In our particular proof, the optional hypothesis is not
referenced, but in general, any combination of active (i.e.\ optional and
mandatory) hypotheses could be referenced.  The proof of statement \texttt{wnew}
is the sequence of five labels starting with \texttt{ws} (step~1) and ending with
\texttt{w2} (step~5).

When Metamath verifies the proof, it scans the proof from left to right.  We
will examine what happens at each step of the proof.  The stack starts off
empty.  At step 1, Metamath looks up label \texttt{ws} and determines that it is a
hypothesis, so it pushes the symbol sequence of statement \texttt{ws} onto the
stack:

\begin{center}\begin{tabular}{|l|l|}\hline
{Stack location} & {Contents} \\ \hline \hline
1 & \texttt{wff s} \\ \hline
\end{tabular}\end{center}

Metamath sees that the labels \texttt{wr} and \texttt{wp} in steps~2 and 3 are also
hypotheses, so it pushes them onto the stack.  After step~3, the stack looks
like
this:

\begin{center}\begin{tabular}{|l|l|}\hline
{Stack location} & {Contents} \\ \hline \hline
3 & \texttt{wff p} \\ \hline
2 & \texttt{wff r} \\ \hline
1 & \texttt{wff s} \\ \hline
\end{tabular}\end{center}

At step 4, Metamath sees that label \texttt{w2} is an assertion, so it must do
some processing.  First, it associates the mandatory hypotheses of \texttt{w2},
which are \texttt{wp} and \texttt{wq}, with stack locations~2 and 3, {\em in that
order}. Metamath determines that the only possible way
to make hypothesis \texttt{wp} match (become identical to) stack location~2 and
\texttt{wq} match stack location 3 is to substitute variable \texttt{p} with \texttt{r}
and \texttt{q} with \texttt{p}.  Metamath makes these substitutions into \texttt{w2} and
obtains the symbol sequence \texttt{wff ( r -> p )}.  It removes the hypotheses
from stack locations~2 and 3, then places the result into stack location~2:

\begin{center}\begin{tabular}{|l|l|}\hline
{Stack location} & {Contents} \\ \hline \hline
2 & \texttt{wff ( r -> p )} \\ \hline
1 & \texttt{wff s} \\ \hline
\end{tabular}\end{center}

At step 5, Metamath sees that label \texttt{w2} is an assertion, so it must again
do some processing.  First, it matches the mandatory hypotheses of \texttt{w2},
which are \texttt{wp} and \texttt{wq}, to stack locations 1 and 2.
Metamath determines that the only possible way to make the
hypotheses match is to substitute variable \texttt{p} with \texttt{s} and \texttt{q} with
\texttt{( r -> p )}.  Metamath makes these substitutions into \texttt{w2} and obtains
the symbol
sequence \texttt{wff ( s -> ( r -> p ) )}.  It removes stack
locations 1 and 2, then places the result into stack location~1:

\begin{center}\begin{tabular}{|l|l|}\hline
{Stack location} & {Contents} \\ \hline \hline
1 & \texttt{wff ( s -> ( r -> p ) )} \\ \hline
\end{tabular}\end{center}

After Metamath finishes processing the proof, it checks to see that the
stack contains exactly one element and that this element is
the same as the math symbol sequence in the
\texttt{\$p}\index{\texttt{\$p} statement} statement.  This is the case for our
proof of \texttt{wnew},
so we have proved \texttt{wnew} successfully.  If the result
differs, Metamath will notify you with an error message.  An error message
will also result if the stack contains more than one entry at the end of the
proof, or if the stack did not contain enough entries at any point in the
proof to match all of the mandatory hypotheses\index{mandatory hypothesis} of
an assertion.  Finally, Metamath will notify you with an error message if no
substitution is possible that will make a referenced assertion's hypothesis
match the
stack entries.  You may want to experiment with the different kinds of errors
that Metamath will detect by making some small changes in the proof of our
example.

Metamath's proof notation was designed primarily to express proofs in a
relatively compact manner, not for readability by humans.  Metamath can display
proofs in a number of different ways with the \texttt{show proof}\index{\texttt{show
proof} command} command.  The
\texttt{/lemmon} qualifier displays it in a format that is easier to read when the
proofs are short, and you saw examples of its use in Chapter~\ref{using}.  For
longer proofs, it is useful to see the tree structure of the proof.  A tree
structure is displayed when the \texttt{/lemmon} qualifier is omitted.  You will
probably find this display more convenient as you get used to it. The tree
display of the proof in our example looks like
this:\label{treeproof}\index{tree-style proof}\index{proof!tree-style}
\begin{verbatim}
1     wp=ws    $f wff s
2        wp=wr    $f wff r
3        wq=wp    $f wff p
4     wq=w2    $a wff ( r -> p )
5  wnew=w2  $a wff ( s -> ( r -> p ) )
\end{verbatim}
The number to the left of each line is the step number.  Following it is a
{\bf hypothesis association}\index{hypothesis association}, consisting of two
labels\index{label} separated by \texttt{=}.  To the left of the \texttt{=} (except
in the last step) is the label of a hypothesis of an assertion referenced
later in the proof; here, steps 1 and 4 are the hypothesis associations for
the assertion \texttt{w2} that is referenced in step 5.  A hypothesis association
is indented one level more than the assertion that uses it, so it is easy to
find the corresponding assertion by moving directly down until the indentation
level decreases to one less than where you started from.  To the right of each
\texttt{=} is the proof step label for that proof step.  The statement keyword of
the proof step label is listed next, followed by the content of the top of the
stack (the most recent stack entry) as it exists after that proof step is
processed.  With a little practice, you should have no trouble reading proofs
displayed in this format.

Metamath proofs include the syntax construction of a formula.
In standard mathematics, this kind of
construction is not considered a proper part of the proof at all, and it
certainly becomes rather boring after a while.
Therefore,
by default the \texttt{show proof}\index{\texttt{show proof}
command} command does not show the syntax construction.
Historically \texttt{show proof} command
\textit{did} show the syntax construction, and you needed to add the
\texttt{/essential} option to hide, them, but today
\texttt{/essential} is the default and you need to use
\texttt{/all} to see the syntax constructions.

When verifying a proof, Metamath will check that no mandatory
\texttt{\$d}\index{\texttt{\$d} statement}\index{mandatory \texttt{\$d}
statement} statement of an assertion referenced in a proof is violated
when substitutions\index{substitution!variable}\index{variable
substitution} are made to the variables in the assertion.  For details
see Section~\ref{spec4} or \ref{dollard}.

\subsection{The Concept of Unification} \label{unify}

During the course of verifying a proof, when Metamath\index{Metamath}
encounters an assertion label\index{assertion label}, it associates the
mandatory hypotheses\index{mandatory hypothesis} of the assertion with the top
entries of the RPN stack\index{stack}\index{RPN stack}.  Metamath then
determines what substitutions\index{substitution!variable}\index{variable
substitution} it must make to the variables in the assertion's mandatory
hypotheses in order for these hypotheses to become identical to their
corresponding stack entries.  This process is called {\bf
unification}\index{unification}.  (We also informally use the term
``unification'' to refer to a set of substitutions that results from the
process, as in ``two unifications are possible.'')  After the substitutions
are made, the hypotheses are said to be {\bf unified}.

If no such substitutions are possible, Metamath will consider the proof
incorrect and notify you with an error message.
% (deleted 3/10/07, per suggestion of Mel O'Cat:)
% The syntax of the
% Metamath language ensures that if a set of substitutions exists, it
% will be unique.

The general algorithm for unification described in the literature is
somewhat complex.
However, in the case of Metamath it is intentionally trivial.
Mandatory hypotheses must be
pushed on the proof stack in the order in which they appear.
In addition, each variable must have its type specified
with a \texttt{\$f} hypothesis before it is used
and that each \texttt{\$f} hypothesis
have the restricted syntax of a typecode (a constant) followed by a variable.
The typecode in the \texttt{\$f} hypothesis must match the first symbol of
the corresponding RPN stack entry (which will also be a constant), so
the only possible match for the variable in the \texttt{\$f} hypothesis is
the sequence of symbols in the stack entry after the initial constant.

In the Proof Assistant\index{Proof Assistant}, a more general unification
algorithm is used.  While a proof is being developed, sometimes not enough
information is available to determine a unique unification.  In this case
Metamath will ask you to pick the correct one.\index{ambiguous
unification}\index{unification!ambiguous}

\section{Extensions to the Metamath Language}\index{extended
language}

\subsection{Comments in the Metamath Language}\label{comments}
\index{markup notation}
\index{comments!markup notation}

The commenting feature allows you to annotate the contents of
a database.  Just as with most
computer languages, comments are ignored for the purpose of interpreting the
contents of the database. Comments effectively act as
additional white space\index{white
space} between tokens
when a database is parsed.

A comment may be placed at the beginning, end, or
between any two tokens\index{token} in a source file.

Comments have the following syntax:
\begin{center}
 \texttt{\$(} {\em text} \texttt{\$)}
\end{center}
Here,\index{\texttt{\$(} and \texttt{\$)} auxiliary
keywords}\index{comment} {\em text} is a string, possibly empty, of any
characters in Metamath's character set (p.~\pageref{spec1chars}), except
that the character strings \texttt{\$(} and \texttt{\$)} may not appear
in {\em text}.  Thus nested comments are not
permitted:\footnote{Computer languages have differing standards for
nested comments, and rather than picking one it was felt simplest not to
allow them at all, at least in the current version (0.177) of
Metamath\index{Metamath!limitations of version 0.177}.} Metamath will
complain if you give it
\begin{center}
 \texttt{\$( This is a \$( nested \$) comment.\ \$)}
\end{center}
To compensate for this non-nesting behavior, I often change all \texttt{\$}'s
to \texttt{@}'s in sections of Metamath code I wish to comment out.

The Metamath program supports a number of markup mechanisms and conventions
to generate good-looking results in \LaTeX\ and {\sc html},
as discussed below.
These markup features have to do only with how the comments are typeset,
and have no effect on how Metamath verifies the proofs in the database.
The improper
use of them may result in incorrectly typeset output, but no Metamath
error messages will result during the \texttt{read} and \texttt{verify
proof} commands.  (However, the \texttt{write
theorem\texttt{\char`\_}list} command
will check for markup errors as a side-effect of its
{\sc html} generation.)
Section~\ref{texout} has instructions for creating \LaTeX\ output, and
section~\ref{htmlout} has instructions for creating
{\sc html}\index{HTML} output.

\subsubsection{Headings}\label{commentheadings}

If the \texttt{\$(} is immediately followed by a new line
starting with a heading marker, it is a header.
This can start with:

\begin{itemize}
 \item[] \texttt{\#\#\#\#} - major part header
 \item[] \texttt{\#*\#*} - section header
 \item[] \texttt{=-=-} - subsection header
 \item[] \texttt{-.-.} - subsubsection header
\end{itemize}

The line following the marker line
will be used for the table of contents entry, after trimming spaces.
The next line should be another (closing) matching marker line.
Any text after that
but before the closing \texttt{\$}, such as an extended description of the
section, will be included on the \texttt{mmtheoremsNNN.html} page.

For more information, run
\texttt{help write theorem\char`\_list}.

\subsubsection{Math mode}
\label{mathcomments}
\index{\texttt{`} inside comments}
\index{\texttt{\char`\~} inside comments}
\index{math mode}

Inside of comments, a string of tokens\index{token} enclosed in
grave accents\index{grave accent (\texttt{`})} (\texttt{`}) will be converted
to standard mathematical symbols during
{\sc HTML}\index{HTML} or \LaTeX\ output
typesetting,\index{latex@{\LaTeX}} according to the information in the
special \texttt{\$t}\index{\texttt{\$t} comment}\index{typesetting
comment} comment in the database
(see section~\ref{tcomment} for information about the typesetting
comment, and Appendix~\ref{ASCII} to see examples of its results).

The first grave accent\index{grave accent (\texttt{`})} \texttt{`}
causes the output processor to enter {\bf math mode}\index{math mode}
and the second one exits it.
In this
mode, the characters following the \texttt{`} are interpreted as a
sequence of math symbol tokens separated by white space\index{white
space}.  The tokens are looked up in the \texttt{\$t}
comment\index{\texttt{\$t} comment}\index{typesetting comment} and if
found, they will be replaced by the standard mathematical symbols that
they correspond to before being placed in the typeset output file.  If
not found, the symbol will be output as is and a warning will be issued.
The tokens do not have to be active in the database, although a warning
will be issued if they are not declared with \texttt{\$c} or
\texttt{\$v} statements.

Two consecutive
grave accents \texttt{``} are treated as a single actual grave accent
(both inside and outside of math mode) and will not cause the output
processor to enter or exit math mode.

Here is an example of its use\index{Pierce's axiom}:
\begin{center}
\texttt{\$( Pierce's axiom, ` ( ( ph -> ps ) -> ph ) -> ph ` ,\\
         is not very intuitive. \$)}
\end{center}
becomes
\begin{center}
   \texttt{\$(} Pierce's axiom, $((\varphi \rightarrow \psi)\rightarrow
\varphi)\rightarrow \varphi$, is not very intuitive. \texttt{\$)}
\end{center}

Note that the math symbol tokens\index{token} must be surrounded by white
space\index{white space}.
%, since there is no context that allows ambiguity to be
%resolved, as is the case with math symbol sequences in some of the Metamath
%statements.
White space should also surround the \texttt{`}
delimiters.

The math mode feature also gives you a quick and easy way to generate
text containing mathematical symbols, independently of the intended
purpose of Metamath.\index{Metamath!using as a math editor} To do this,
simply create your text with grave accents surrounding your formulas,
after making sure that your math symbols are mapped to \LaTeX\ symbols
as described in Appendix~\ref{ASCII}.  It is easier if you start with a
database with predefined symbols such as \texttt{set.mm}.  Use your
grave-quoted math string to replace an existing comment, then typeset
the statement corresponding to that comment following the instructions
from the \texttt{help tex} command in the Metamath program.  You will
then probably want to edit the resulting file with a text editor to fine
tune it to your exact needs.

\subsubsection{Label Mode}\index{label mode}

Outside of math mode, a tilde\index{tilde (\texttt{\char`\~})} \verb/~/
indicates to Metamath's\index{Metamath} output processor that the
token\index{token} that follows (i.e.\ the characters up to the next
white space\index{white space}) represents a statement label or URL.
This formatting mode is called {\bf label mode}\index{label mode}.
If a literal tilde
is desired (outside of math mode) instead of label mode,
use two tildes in a row to represent it.

When generating a \LaTeX\ output file,
the following token will be formatted in \texttt{typewriter}
font, and the tilde removed, to make it stand out from the rest of the text.
This formatting will be applied to all characters after the
tilde up to the first white space\index{white space}.
Whether
or not the token is an actual statement label is not checked, and the
token does not have to have the correct syntax for a label; no error
messages will be produced.  The only effect of the label mode on the
output is that typewriter font will be used for the tokens that are
placed in the \LaTeX\ output file.

When generating {\sc html},
the tokens after the tilde {\em must} be a URL (either http: or https:)
or a valid label.
Error messages will be issued during that output if they aren't.
A hyperlink will be generated to that URL or label.

\subsubsection{Link to bibliographical reference}\index{citation}%
\index{link to bibliographical reference}

Bibliographical references are handled specially when generating
{\sc html} if formatted specially.
Text in the form \texttt{[}{\em author}\texttt{]}
is considered a link to a bibliographical reference.
See \texttt{help html} and \texttt{help write
bibliography} in the Metamath program for more
information.
% \index{\texttt{\char`\[}\ldots\texttt{]} inside comments}
See also Sections~\ref{tcomment} and \ref{wrbib}.

The \texttt{[}{\em author}\texttt{]} notation will also create an entry in
the bibliography cross-reference file generated by \texttt{write
bibliography} (Section~\ref{wrbib}) for {\sc HTML}.
For this to work properly, the
surrounding comment must be formatted as follows:
\begin{quote}
    {\em keyword} {\em label} {\em noise-word}
     \texttt{[}{\em author}\texttt{] p.} {\em number}
\end{quote}
for example
\begin{verbatim}
     Theorem 5.2 of [Monk] p. 223
\end{verbatim}
The {\em keyword} is not case sensitive and must be one of the following:
\begin{verbatim}
     theorem lemma definition compare proposition corollary
     axiom rule remark exercise problem notation example
     property figure postulate equation scheme chapter
\end{verbatim}
The optional {\em label} may consist of more than one
(non-{\em keyword} and non-{\em noise-word}) word.
The optional {\em noise-word} is one of:
\begin{verbatim}
     of in from on
\end{verbatim}
and is  ignored when the cross-reference file is created.  The
\texttt{write
biblio\-graphy} command will perform error checking to verify the
above format.\index{error checking}

\subsubsection{Parentheticals}\label{parentheticals}

The end of a comment may include one or more parenthicals, that is,
statements enclosed in parentheses.
The Metamath program looks for certain parentheticals and can issue
warnings based on them.
They are:

\begin{itemize}
 \item[] \texttt{(Contributed by }
   \textit{NAME}\texttt{,} \textit{DATE}\texttt{.)} -
   document the original contributor's name and the date it was created.
 \item[] \texttt{(Revised by }
   \textit{NAME}\texttt{,} \textit{DATE}\texttt{.)} -
   document the contributor's name and creation date
   that resulted in significant revision
   (not just an automated minimization or shortening).
 \item[] \texttt{(Proof shortened by }
   \textit{NAME}\texttt{,} \textit{DATE}\texttt{.)} -
   document the contributor's name and date that developed a significant
   shortening of the proof (not just an automated minimization).
 \item[] \texttt{(Proof modification is discouraged.)} -
   Note that this proof should normally not be modified.
 \item[] \texttt{(New usage is discouraged.)} -
   Note that this assertion should normally not be used.
\end{itemize}

The \textit{DATE} must be in form YYYY-MMM-DD, where MMM is the
English abbreviation of that month.

\subsubsection{Other markup}\label{othermarkup}
\index{markup notation}

There are other markup notations for generating good-looking results
beyond math mode and label mode:

\begin{itemize}
 \item[]
         \texttt{\char`\_} (underscore)\index{\texttt{\char`\_} inside comments} -
             Italicize text starting from
              {\em space}\texttt{\char`\_}{\em non-space} (i.e.\ \texttt{\char`\_}
              with a space before it and a non-space character after it) until
             the next
             {\em non-space}\texttt{\char`\_}{\em space}.  Normal
             punctuation (e.g.\ a trailing
             comma or period) is ignored when determining {\em space}.
 \item[]
         \texttt{\char`\_} (underscore) - {\em
         non-space}\texttt{\char`\_}{\em non-space-string}, where
          {\em non-space-string} is a string of non-space characters,
         will make {\em non-space-string} become a subscript.
 \item[]
         \texttt{<HTML>}...\texttt{</HTML>} - do not convert
         ``\texttt{<}'' and ``\texttt{>}''
         in the enclosed text when generating {\sc HTML},
         otherwise process markup normally. This allows direct insertion
         of {\sc html} commands.
 \item[]
       ``\texttt{\&}ref\texttt{;}'' - insert an {\sc HTML}
         character reference.
         This is how to insert arbitrary Unicode characters
         (such as accented characters).  Currently only directly supported
         when generating {\sc HTML}.
\end{itemize}

It is recommended that spaces surround any \texttt{\char`\~} and
\texttt{`} tokens in the comment and that a space follow the {\em label}
after a \texttt{\char`\~} token.  This will make global substitutions
to change labels and symbol names much easier and also eliminate any
future chance of ambiguity.  Spaces around these tokens are automatically
removed in the final output to conform with normal rules of punctuation;
for example, a space between a trailing \texttt{`} and a left parenthesis
will be removed.

A good way to become familiar with the markup notation is to look at
the extensive examples in the \texttt{set.mm} database.

\subsection{The Typesetting Comment (\texttt{\$t})}\label{tcomment}

The typesetting comment \texttt{\$t} in the input database file
provides the information necessary to produce good-looking results.
It provides \LaTeX\ and {\sc html}
definitions for math symbols,
as well supporting as some
customization of the generated web page.
If you add a new token to a database, you should also
update the \texttt{\$t} comment information if you want to eventually
create output in \LaTeX\ or {\sc HTML}.
See the
\texttt{set.mm}\index{set theory database (\texttt{set.mm})} database
file for an extensive example of a \texttt{\$t} comment illustrating
many of the features described below.

Programs that do not need to generate good-looking presentation results,
such as programs that only verify Metamath databases,
can completely ignore typesetting comments
and just treat them as normal comments.
Even the Metamath program only consults the
\texttt{\$t} comment information when it needs to generate typeset output
in \LaTeX\ or {\sc HTML}
(e.g., when you open a \LaTeX\ output file with the \texttt{open tex} command).

We will first discuss the syntax of typesetting comments, and then
briefly discuss how this can be used within the Metamath program.

\subsubsection{Typesetting Comment Syntax Overview}

The typesetting comment is identified by the token
\texttt{\$t}\index{\texttt{\$t} comment}\index{typesetting comment} in
the comment, and the typesetting comment ends at the matching
\texttt{\$)}:
\[
  \mbox{\tt \$(\ }
  \mbox{\tt \$t\ }
  \underbrace{
    \mbox{\tt \ \ \ \ \ \ \ \ \ \ \ }
    \cdots
    \mbox{\tt \ \ \ \ \ \ \ \ \ \ \ }
  }_{\mbox{Typesetting definitions go here}}
  \mbox{\tt \ \$)}
\]

There must be one or more white space characters, and only white space
characters, between the \texttt{\$(} that starts the comment
and the \texttt{\$t} symbol,
and the \texttt{\$t} must be followed by one
or more white space characters
(see section \ref{whitespace} for the definition of white space characters).
The typesetting comment continues until the comment end token \texttt{\$)}
(which must be preceded by one or more white space characters).

In version 0.177\index{Metamath!limitations of version 0.177} of the
Metamath program, there may be only one \texttt{\$t} comment in a
database.  This restriction may be lifted in the future to allow
many \texttt{\$t} comments in a database.

Between the \texttt{\$t} symbol (and its following white space) and the
comment end token \texttt{\$)} (and its preceding white space)
is a sequence of one or more typesetting definitions, where
each definition has the form
\textit{definition-type arg arg ... ;}.
Each of the zero or more \textit{arg} values
can be either a typesetting data or a keyword
(what keywords are allowed, and where, depends on the specific
\textit{definition-type}).
The \textit{definition-type}, and each argument \textit{arg},
are separated by one or more white space characters.
Every definition ends in an unquoted semicolon;
white space is not required before the terminating semicolon of a definition.
Each definition should start on a new line.\footnote{This
restriction of the current version of Metamath
(0.177)\index{Metamath!limitations of version 0.177} may be removed
in a future version, but you should do it anyway for readability.}

For example, this typesetting definition:
\begin{center}
 \verb$latexdef "C_" as "\subseteq";$
\end{center}
defines the token \verb$C_$ as the \LaTeX\ symbol $\subseteq$ (which means
``subset'').

Typesetting data is a sequence of one or more quoted strings
(if there is more than one, they are connected by \texttt{\char`\+}).
Often a single quoted string is used to provide data for a definition, using
either double (\texttt{\char`\"}) or single (\texttt{'}) quotation marks.
However,
{\em a quoted string (enclosed in quotation marks) may not include
line breaks.}
A quoted string
may include a quotation mark that matches the enclosing quotes by repeating
the quotation mark twice.  Here are some examples:

\begin{tabu}   { l l }
\textbf{Example} & \textbf{Meaning} \\
\texttt{\char`\"a\char`\"\char`\"b\char`\"} & \texttt{a\char`\"b} \\
\texttt{'c''d'} & \texttt{c'd} \\
\texttt{\char`\"e''f\char`\"} & \texttt{e''f} \\
\texttt{'g\char`\"\char`\"h'} & \texttt{g\char`\"\char`\"h} \\
\end{tabu}

Finally, a long quoted string
may be broken up into multiple quoted strings (considered, as a whole,
a single quoted string) and joined with \texttt{\char`\+}.
You can even use multiple lines as long as a
'+' is at the end of every line except the last one.
The \texttt{\char`\+} should be preceded and followed by at least one
white space character.
Thus, for example,
\begin{center}
 \texttt{\char`\"ab\char`\"\ \char`\+\ \char`\"cd\char`\"
    \ \char`\+\ \\ 'ef'}
\end{center}
is the same as
\begin{center}
 \texttt{\char`\"abcdef\char`\"}
\end{center}

{\sc c}-style comments \texttt{/*}\ldots\texttt{*/} are also supported.

In practice, whenever you add a new math token you will often want to add
typesetting definitions using
\texttt{latexdef}, \texttt{htmldef}, and
\texttt{althtmldef}, as described below.
That way, they will all be up to date.
Of course, whether or not you want to use all three definitions will
depend on how the database is intended to be used.

Below we discuss the different possible \textit{definition-kind} options.
We will show data surrounded by double quotes (in practice they can also use
single quotes and/or be a sequence joined by \texttt{+}s).
We will use specific names for the \textit{data} to make clear what
the data is used for, such as
{\em math-token} (for a Metamath math token,
{\em latex-string} (for string to be placed in a \LaTeX\ stream),
{\em {\sc html}-code} (for {\sc html} code),
and {\em filename} (for a filename).

\subsubsection{Typesetting Comment - \LaTeX}

The syntax for a \LaTeX\ definition is:
\begin{center}
 \texttt{latexdef "}{\em math-token}\texttt{" as "}{\em latex-string}\texttt{";}
\end{center}
\index{latex definitions@\LaTeX\ definitions}%
\index{\texttt{latexdef} statement}

The {\em token-string} and {\em latex-string} are the data
(character strings) for
the token and the \LaTeX\ definition of the token, respectively,

These \LaTeX\ definitions are used by the Metamath program
when it is asked to product \LaTeX output using
the \texttt{write tex} command.

\subsubsection{Typesetting Comment - {\sc html}}

The key kinds of {\sc HTML} definitions have the following syntax:

\vskip 1ex
    \texttt{htmldef "}{\em math-token}\texttt{" as "}{\em
    {\sc html}-code}\texttt{";}\index{\texttt{htmldef} statement}
                    \ \ \ \ \ \ldots

    \texttt{althtmldef "}{\em math-token}\texttt{" as "}{\em
{\sc html}-code}\texttt{";}\index{\texttt{althtmldef} statement}

                    \ \ \ \ \ \ldots

Note that in {\sc HTML} there are two possible definitions for math tokens.
This feature is useful when
an alternate representation of symbols is desired, for example one that
uses Unicode entities and another uses {\sc gif} images.

There are many other typesetting definitions that can control {\sc HTML}.
These include:

\vskip 1ex

    \texttt{htmldef "}{\em math-token}\texttt{" as "}{\em {\sc
    html}-code}\texttt{";}

    \texttt{htmltitle "}{\em {\sc html}-code}\texttt{";}%
\index{\texttt{htmltitle} statement}

    \texttt{htmlhome "}{\em {\sc html}-code}\texttt{";}%
\index{\texttt{htmlhome} statement}

    \texttt{htmlvarcolor "}{\em {\sc html}-code}\texttt{";}%
\index{\texttt{htmlvarcolor} statement}

    \texttt{htmlbibliography "}{\em filename}\texttt{";}%
\index{\texttt{htmlbibliography} statement}

\vskip 1ex

\noindent The \texttt{htmltitle} is the {\sc html} code for a common
title, such as ``Metamath Proof Explorer.''  The \texttt{htmlhome} is
code for a link back to the home page.  The \texttt{htmlvarcolor} is
code for a color key that appears at the bottom of each proof.  The file
specified by {\em filename} is an {\sc html} file that is assumed to
have a \texttt{<A NAME=}\ldots\texttt{>} tag for each bibiographic
reference in the database comments.  For example, if
\texttt{[Monk]}\index{\texttt{\char`\[}\ldots\texttt{]} inside comments}
occurs in the comment for a theorem, then \texttt{<A NAME='Monk'>} must
be present in the file; if not, a warning message is given.

Associated with
\texttt{althtmldef}
are the statements
\vskip 1ex

    \texttt{htmldir "}{\em
      directoryname}\texttt{";}\index{\texttt{htmldir} statement}

    \texttt{althtmldir "}{\em
     directoryname}\texttt{";}\index{\texttt{althtmldir} statement}

\vskip 1ex
\noindent giving the directories of the {\sc gif} and Unicode versions
respectively; their purpose is to provide cross-linking between the
two versions in the generated web pages.

When two different types of pages need to be produced from a single
database, such as the Hilbert Space Explorer that extends the Metamath
Proof Explorer, ``extended'' variables may be declared in the
\texttt{\$t} comment:
\vskip 1ex

    \texttt{exthtmltitle "}{\em {\sc html}-code}\texttt{";}%
\index{\texttt{exthtmltitle} statement}

    \texttt{exthtmlhome "}{\em {\sc html}-code}\texttt{";}%
\index{\texttt{exthtmlhome} statement}

    \texttt{exthtmlbibliography "}{\em filename}\texttt{";}%
\index{\texttt{exthtmlbibliography} statement}

\vskip 1ex
\noindent When these are declared, you also must declare
\vskip 1ex

    \texttt{exthtmllabel "}{\em label}\texttt{";}%
\index{\texttt{exthtmllabel} statement}

\vskip 1ex \noindent that identifies the database statement where the
``extended'' section of the database starts (in our example, where the
Hilbert Space Explorer starts).  During the generation of web pages for
that starting statement and the statements after it, the {\sc html} code
assigned to \texttt{exthtmltitle} and \texttt{exthtmlhome} is used
instead of that assigned to \texttt{htmltitle} and \texttt{htmlhome},
respectively.

\begin{sloppy}
\subsection{Additional Information Com\-ment (\texttt{\$j})} \label{jcomment}
\end{sloppy}

The additional information comment, aka the
\texttt{\$j}\index{\texttt{\$j} comment}\index{additional information comment}
comment,
provides a way to add additional structured information that can
be optionally parsed by systems.

The additional information comment is parsed the same way as the
typesetting comment (\texttt{\$t}) (see section \ref{tcomment}).
That is,
the additional information comment begins with the token
\texttt{\$j} within a comment,
and continues until the comment close \texttt{\$)}.
Within an additional information comment is a sequence of one or more
commands of the form \texttt{command arg arg ... ;}
where each of the zero or more \texttt{arg} values
can be either a quoted string or a keyword.
Note that every command ends in an unquoted semicolon.
If a verifier is parsing an additional information comment, but
doesn't recognize a particular command, it must skip the command
by finding the end of the command (an unquoted semicolon).

A database may have 0 or more additional information comments.
Note, however, that a verifier may ignore these comments entirely or only
process certain commands in an additional information comment.
The \texttt{mmj2} verifier supports many commands in additional information
comments.
We encourage systems that process additional information comments
to coordinate so that they will use the same command for the same effect.

Examples of additional information comments with various commands
(from the \texttt{set.mm} database) are:

\begin{itemize}
   \item Define the syntax and logical typecodes,
     and declare that our grammar is
     unambiguous (verifiable using the KLR parser, with compositing depth 5).
\begin{verbatim}
  $( $j
    syntax 'wff';
    syntax '|-' as 'wff';
    unambiguous 'klr 5';
  $)
\end{verbatim}

   \item Register $\lnot$ and $\rightarrow$ as primitive expressions
           (lacking definitions).
\begin{verbatim}
  $( $j primitive 'wn' 'wi'; $)
\end{verbatim}

   \item There is a special justification for \texttt{df-bi}.
\begin{verbatim}
  $( $j justification 'bijust' for 'df-bi'; $)
\end{verbatim}

   \item Register $\leftrightarrow$ as an equality for its type (wff).
\begin{verbatim}
  $( $j
    equality 'wb' from 'biid' 'bicomi' 'bitri';
    definition 'dfbi1' for 'wb';
  $)
\end{verbatim}

   \item Theorem \texttt{notbii} is the congruence law for negation.
\begin{verbatim}
  $( $j congruence 'notbii'; $)
\end{verbatim}

   \item Add \texttt{setvar} as a typecode.
\begin{verbatim}
  $( $j syntax 'setvar'; $)
\end{verbatim}

   \item Register $=$ as an equality for its type (\texttt{class}).
\begin{verbatim}
  $( $j equality 'wceq' from 'eqid' 'eqcomi' 'eqtri'; $)
\end{verbatim}

\end{itemize}


\subsection{Including Other Files in a Metamath Source File} \label{include}
\index{\texttt{\$[} and \texttt{\$]} auxiliary keywords}

The keywords \texttt{\$[} and \texttt{\$]} specify a file to be
included\index{included file}\index{file inclusion} at that point in a
Metamath\index{Metamath} source file\index{source file}.  The syntax for
including a file is as follows:
\begin{center}
\texttt{\$[} {\em file-name} \texttt{\$]}
\end{center}

The {\em file-name} should be a single token\index{token} with the same syntax
as a math symbol (i.e., all 93 non-whitespace
printable characters other than \texttt{\$} are
allowed, subject to the file-naming limitations of your operating system).
Comments may appear between the \texttt{\$[} and \texttt{\$]} keywords.  Included
files may include other files, which may in turn include other files, and so
on.

For example, suppose you want to use the set theory database as the starting
point for your own theory.  The first line in your file could be
\begin{center}
\texttt{\$[ set.mm \$]}
\end{center} All of the information (axioms, theorems,
etc.) in \texttt{set.mm} and any files that {\em it} includes will become
available for you to reference in your file. This can help make your work more
modular. A drawback to including files is that if you change the name of a
symbol or the label of a statement, you must also remember to update any
references in any file that includes it.


The naming conventions for included files are the same as those of your
operating system.\footnote{On the Macintosh, prior to Mac OS X,
 a colon is used to separate disk
and folder names from your file name.  For example, {\em volume}\texttt{:}{\em
file-name} refers to the root directory, {\em volume}\texttt{:}{\em
folder-name}\texttt{:}{\em file-name} refers to a folder in root, and {\em
volume}\texttt{:}{\em folder-name}\texttt{:}\ldots\texttt{:}{\em file-name} refers to a
deeper folder.  A simple {\em file-name} refers to a file in the folder from
which you launch the Metamath application.  Under Mac OS X and later,
the Metamath program is run under the Terminal application, which
conforms to Unix naming conventions.}\index{Macintosh file
names}\index{file names!Macintosh}\label{includef} For compatibility among
operating systems, you should keep the file names as simple as possible.  A
good convention to use is {\em file}\texttt{.mm} where {\em file} is eight
characters or less, in lower case.

There is no limit to the nesting depth of included files.  One thing that you
should be aware of is that if two included files themselves include a common
third file, only the {\em first} reference to this common file will be read
in.  This allows you to include two or more files that build on a common
starting file without having to worry about label and symbol conflicts that
would occur if the common file were read in more than once.  (In fact, if a
file includes itself, the self-reference will be ignored, although of course
it would not make any sense to do that.)  This feature also means, however,
that if you try to include a common file in several inner blocks, the result
might not be what you expect, since only the first reference will be replaced
with the included file (unlike the include statement in most other computer
languages).  Thus you would normally include common files only in the
outermost block\index{outermost block}.

\subsection{Compressed Proof Format}\label{compressed1}\index{compressed
proof}\index{proof!compressed}

The proof notation presented in Section~\ref{proof} is called a
{\bf normal proof}\index{normal proof}\index{proof!normal} and in principle is
sufficient to express any proof.  However, proofs often contain steps and
subproofs that are identical.  This is particularly true in typical
Metamath\index{Metamath} applications, because Metamath requires that the math
symbol sequence (usually containing a formula) at each step be separately
constructed, that is, built up piece by piece. As a result, a lot of
repetition often results.  The {\bf compressed proof} format allows Metamath
to take advantage of this redundancy to shorten proofs.

The specification for the compressed proof format is given in
Appen\-dix~\ref{compressed}.

Normally you need not concern yourself with the details of the compressed
proof format, since the Metamath program will allow you to convert from
the normal format to the compressed format with ease, and will also
automatically convert from the compressed format when proofs are displayed.
The overall structure of the compressed format is as follows:
\begin{center}
  \texttt{\$= ( } {\em label-list} \texttt{) } {\em compressed-proof\ }\ \texttt{\$.}
\end{center}
\index{\texttt{\$=} keyword}
The first \texttt{(} serves as a flag to Metamath that a compressed proof
follows.  The {\em label-list} includes all statements referred to by the
proof except the mandatory hypotheses\index{mandatory hypothesis}.  The {\em
compressed-proof} is a compact encoding of the proof, using upper-case
letters, and can be thought of as a large integer in base 26.  White
space\index{white space} inside a {\em compressed-proof} is
optional and is ignored.

It is important to note that the order of the mandatory hypotheses of
the statement being proved must not be changed if the compressed proof
format is used, otherwise the proof will become incorrect.  The reason
for this is that the mandatory hypotheses are not mentioned explicitly
in the compressed proof in order to make the compression more efficient.
If you wish to change the order of mandatory hypotheses, you must first
convert the proof back to normal format using the \texttt{save proof
{\em statement} /normal}\index{\texttt{save proof} command} command.
Later, you can go back to compressed format with \texttt{save proof {\em
statement} /compressed}.

During error checking with the \texttt{verify proof} command, an error
found in a compressed proof may point to a character in {\em
compressed-proof}, which may not be very meaningful to you.  In this
case, try to \texttt{save proof /normal} first, then do the
\texttt{verify proof} again.  In general, it is best to make sure a
proof is correct before saving it in compressed format, because severe
errors are less likely to be recoverable than in normal format.

\subsection{Specifying Unknown Proofs or Subproofs}\label{unknown}

In a proof under development, any step or subproof that is not yet known
may be represented with a single \texttt{?}.  For the purposes of
parsing the proof, the \texttt{?}\ \index{\texttt{]}@\texttt{?}\ inside
proofs} will push a single entry onto the RPN stack just as if it were a
hypothesis.  While developing a proof with the Proof
Assistant\index{Proof Assistant}, a partially developed proof may be
saved with the \texttt{save new{\char`\_}proof}\index{\texttt{save
new{\char`\_}proof} command} command, and \texttt{?}'s will be placed at
the appropriate places.

All \texttt{\$p}\index{\texttt{\$p} statement} statements must have
proofs, even if they are entirely unknown.  Before creating a proof with
the Proof Assistant, you should specify a completely unknown proof as
follows:
\begin{center}
  {\em label} \texttt{\$p} {\em statement} \texttt{\$= ?\ \$.}
\end{center}
\index{\texttt{\$=} keyword}
\index{\texttt{]}@\texttt{?}\ inside proofs}

The \texttt{verify proof}\index{\texttt{verify proof} command} command
will check the known portions of a partial proof for errors, but will
warn you that the statement has not been proved.

Note that partially developed proofs may be saved in compressed format
if desired.  In this case, you will see one or more \texttt{?}'s in the
{\em compressed-proof} part.\index{compressed
proof}\index{proof!compressed}

\section{Axioms vs.\ Definitions}\label{definitions}

The \textit{basic}
Metamath\index{Metamath} language and program
make no distinction\index{axiom vs.\
definition} between axioms\index{axiom} and
definitions.\index{definition} The \texttt{\$a}\index{\texttt{\$a}
statement} statement is used for both.  At first, this may seem
puzzling.  In the minds of many mathematicians, the distinction is
clear, even obvious, and hardly worth discussing.  A definition is
considered to be merely an abbreviation that can be replaced by the
expression for which it stands; although unless one actually does this,
to be precise then one should say that a theorem\index{theorem} is a
consequence of the axioms {\em and} the definitions that are used in the
formulation of the theorem \cite[p.~20]{Behnke}.\index{Behnke, H.}

\subsection{What is a Definition?}

What is a definition?  In its simplest form, a definition introduces a new
symbol and provides an unambiguous rule to transform an expression containing
the new symbol to one without it.  The concept of a ``proper
definition''\index{proper definition}\index{definition!proper} (as opposed to
a creative definition)\index{creative definition}\index{definition!creative}
that is usually agreed upon is (1) the definition should not strengthen the
language and (2) any symbols introduced by the definition should be eliminable
from the language \cite{Nemesszeghy}\index{Nemesszeghy, E. Z.}.  In other
words, they are mere typographical conveniences that do not belong to the
system and are theoretically superfluous.  This may seem obvious, but in fact
the nature of definitions can be subtle, sometimes requiring difficult
metatheorems to establish that they are not creative.

A more conservative stance was taken by logician S.
Le\'{s}niewski.\index{Le\'{s}niewski, S.}
\begin{quote}
Le\'{s}niewski
regards definitions as theses of the system.  In this respect they do
not differ either from the axioms or from theorems, i.e.\ from the
theses added to the system on the basis of the rule of substitution or
the rule of detachment [modus ponens].  Once definitions have been
accepted as theses of the system, it becomes necessary to consider them
as true propositions in the same sense in which axioms are true
\cite{Lejewski}.
\end{quote}\index{Lejewski, Czeslaw}

Let us look at some simple examples of definitions in propositional
calculus.  Consider the definition of logical {\sc or}
(disjunction):\index{disjunction ($\vee$)} ``$P\vee Q$ denotes $\neg P
\rightarrow Q$ (not $P$ implies $Q$).''  It is very easy to recognize a
statement making use of this definition, because it introduces the new
symbol $\vee$ that did not previously exist in the language.  It is easy
to see that no new theorems of the original language will result from
this definition.

Next, consider a definition that eliminates parentheses:  ``$P
\rightarrow Q\rightarrow R$ denotes $P\rightarrow (Q \rightarrow R)$.''
This is more subtle, because no new symbols are introduced.  The reason
this definition is considered proper is that no new symbol sequences
that are valid wffs (well-formed formulas)\index{well-formed formula
(wff)} in the original language will result from the definition, since
``$P \rightarrow Q\rightarrow R$'' is not a wff in the original
language.  Here, we implicitly make use of the fact that there is a
decision procedure that allows us to determine whether or not a symbol
sequence is a wff, and this fact allows us to use symbol sequences that
are not wffs to represent other things (such as wffs) by means of the
definition.  However, to justify the definition as not being creative we
need to prove that ``$P \rightarrow Q\rightarrow R$'' is in fact not a
wff in the original language, and this is more difficult than in the
case where we simply introduce a new symbol.

%Now let's take this reasoning to an extreme.  Propositional calculus is a
%decidable theory,\footnote{This means that a mechanical algorithm exists to
%determine whether or not a wff is a theorem.} so in principle we could make use
%of symbol sequences that are not theorems to represent other things (say, to
%encode actual theorems in a more compact way).  For example, let us extend the
%language by defining a wff ``$P$'' in the extended language as the theorem
%``$P\rightarrow P$''\footnote{This is one of the first theorems proved in the
%Metamath database \texttt{set.mm}.}\index{set
%theory database (\texttt{set.mm})} in the original language whenever ``$P$'' is
%not a theorem in the original language.  In the extended language, any wff
%``$Q$'' thus represents a theorem; to find out what theorem (in the original
%language) ``$Q$'' represents, we determine whether ``$Q$'' is a theorem in the
%original language (before the definition was introduced).  If so, we're done; if
%not, we replace ``$Q$'' by ``$Q\rightarrow Q$'' to eliminate the definition.
%This definition is therefore eliminable, and it does not ``strengthen'' the
%language because any wff that is not a theorem is not in the set of statements
%provable in the original language and thus is available for use by definitions.
%
%Of course, a definition such as this would render practically useless the
%communication of theorems of propositional calculus; but
%this is just a human shortcoming, since we can't always easily discern what is
%and is not a theorem by inspection.  In fact, the extended theory with this
%definition has no more and no less information than the original theory; it just
%expresses certain theorems of the form ``$P\rightarrow P$''
%in a more compact way.
%
%The point here is that what constitutes a proper definition is a matter of
%judgment about whether a symbol sequence can easily be recognized by a human
%as invalid in some sense (for example, not a wff); if so, the symbol sequence
%can be appropriated for use by a definition in order to make the extended
%language more compact.  Metamath\index{Metamath} lacks the ability to make this
%judgment, since as far as Metamath is concerned the definition of a wff, for
%example, is arbitrary.  You define for Metamath how wffs\index{well-formed
%formula (wff)} are constructed according to your own preferred style.  The
%concept of a wff may not even exist in a given formal system\index{formal
%system}.  Metamath treats all definitions as if they were new axioms, and it
%is up to the human mathematician to judge whether the definition is ``proper''
%'\index{proper definition}\index{definition!proper} in some agreed-upon way.

What constitutes a definition\index{definition} versus\index{axiom vs.\
definition} an axiom\index{axiom} is sometimes arbitrary in mathematical
literature.  For example, the connectives $\vee$ ({\sc or}), $\wedge$
({\sc and}), and $\leftrightarrow$ (equivalent to) in propositional
calculus are usually considered defined symbols that can be used as
abbreviations for expressions containing the ``primitive'' connectives
$\rightarrow$ and $\neg$.  This is the way we treat them in the standard
logic and set theory database \texttt{set.mm}\index{set theory database
(\texttt{set.mm})}.  However, the first three connectives can also be
considered ``primitive,'' and axiom systems have been devised that treat
all of them as such.  For example,
\cite[p.~35]{Goodstein}\index{Goodstein, R. L.} presents one with 15
axioms, some of which in fact coincide with what we have chosen to call
definitions in \texttt{set.mm}.  In certain subsets of classical
propositional calculus, such as the intuitionist
fragment\index{intuitionism}, it can be shown that one cannot make do
with just $\rightarrow$ and $\neg$ but must treat additional connectives
as primitive in order for the system to make sense.\footnote{Two nice
systems that make the transition from intuitionistic and other weak
fragments to classical logic just by adding axioms are given in
\cite{Robinsont}\index{Robinson, T. Thacher}.}

\subsection{The Approach to Definitions in \texttt{set.mm}}

In set theory, recursive definitions define a newly introduced symbol in
terms of itself.
The justification of recursive definitions, using
several ``recursion theorems,'' is usually one of the first
sophisticated proofs a student encounters when learning set theory, and
there is a significant amount of implicit metalogic behind a recursive
definition even though the definition itself is typically simple to
state.

Metamath itself has no built-in technical limitation that prevents
multiple-part recursive definitions in the traditional textbook style.
However, because the recursive definition requires advanced metalogic
to justify, eliminating a recursive definition is very difficult and
often not even shown in textbooks.

\subsubsection{Direct definitions instead of recursive definitions}

It is, however, possible to substitute one kind of complexity
for another.  We can eliminate the need for metalogical justification by
defining the operation directly with an explicit (but complicated)
expression, then deriving the recursive definition directly as a
theorem, using a recursion theorem ``in reverse.''
The elimination
of a direct definition is a matter of simple mechanical substitution.
We do this in
\texttt{set.mm}, as follows.

In \texttt{set.mm} our goal was to introduce almost all definitions in
the form of two expressions connected by either $\leftrightarrow$ or
$=$, where the thing being defined does not appear on the right hand
side.  Quine calls this form ``a genuine or direct definition'' \cite[p.
174]{Quine}\index{Quine, Willard Van Orman}, which makes the definitions
very easy to eliminate and the metalogic\index{metalogic} needed to
justify them as simple as possible.
Put another way, we had a goal of being able to
eliminate all definitions with direct mechanical substitution and to
verify easily the soundness of the definitions.

\subsubsection{Example of direct definitions}

We achieved this goal in almost all cases in \texttt{set.mm}.
Sometimes this makes the definitions more complex and less
intuitive.
For example, the traditional way to define addition of
natural numbers is to define an operation called {\em
successor}\index{successor} (which means ``plus one'' and is denoted by
``${\rm suc}$''), then define addition recursively\index{recursive
definition} with the two definitions $n + 0 = n$ and $m + {\rm suc}\,n =
{\rm suc} (m + n)$.  Although this definition seems simple and obvious,
the method to eliminate the definition is not obvious:  in the second
part of the definition, addition is defined in terms of itself.  By
eliminating the definition, we don't mean repeatedly applying it to
specific $m$ and $n$ but rather showing the explicit, closed-form
set-theoretical expression that $m + n$ represents, that will work for
any $m$ and $n$ and that does not have a $+$ sign on its right-hand
side.  For a recursive definition like this not to be circular
(creative), there are some hidden, underlying assumptions we must make,
for example that the natural numbers have a certain kind of order.

In \texttt{set.mm} we chose to start with the direct (though complex and
nonintuitive) definition then derive from it the standard recursive
definition.
For example, the closed-form definition used in \texttt{set.mm}
for the addition operation on ordinals\index{ordinal
addition}\index{addition!of ordinals} (of which natural numbers are a
subset) is

\setbox\startprefix=\hbox{\tt \ \ df-oadd\ \$a\ }
\setbox\contprefix=\hbox{\tt \ \ \ \ \ \ \ \ \ \ \ \ \ }
\startm
\m{\vdash}\m{+_o}\m{=}\m{(}\m{x}\m{\in}\m{{\rm On}}\m{,}\m{y}\m{\in}\m{{\rm
On}}\m{\mapsto}\m{(}\m{{\rm rec}}\m{(}\m{(}\m{z}\m{\in}\m{{\rm
V}}\m{\mapsto}\m{{\rm suc}}\m{z}\m{)}\m{,}\m{x}\m{)}\m{`}\m{y}\m{)}\m{)}
\endm
\noindent which depends on ${\rm rec}$.

\subsubsection{Recursion operators}

The above definition of \texttt{df-oadd} depends on the definition of
${\rm rec}$, a ``recursion operator''\index{recursion operator} with
the definition \texttt{df-rdg}:

\setbox\startprefix=\hbox{\tt \ \ df-rdg\ \$a\ }
\setbox\contprefix=\hbox{\tt \ \ \ \ \ \ \ \ \ \ \ \ }
\startm
\m{\vdash}\m{{\rm
rec}}\m{(}\m{F}\m{,}\m{I}\m{)}\m{=}\m{\mathrm{recs}}\m{(}\m{(}\m{g}\m{\in}\m{{\rm
V}}\m{\mapsto}\m{{\rm if}}\m{(}\m{g}\m{=}\m{\varnothing}\m{,}\m{I}\m{,}\m{{\rm
if}}\m{(}\m{{\rm Lim}}\m{{\rm dom}}\m{g}\m{,}\m{\bigcup}\m{{\rm
ran}}\m{g}\m{,}\m{(}\m{F}\m{`}\m{(}\m{g}\m{`}\m{\bigcup}\m{{\rm
dom}}\m{g}\m{)}\m{)}\m{)}\m{)}\m{)}\m{)}
\endm

\noindent which can be further broken down with definitions shown in
Section~\ref{setdefinitions}.

This definition of ${\rm rec}$
defines a recursive definition generator on ${\rm On}$ (the class of ordinal
numbers) with characteristic function $F$ and initial value $I$.
This operation allows us to define, with
compact direct definitions, functions that are usually defined in
textbooks with recursive definitions.
The price paid with our approach
is the complexity of our ${\rm rec}$ operation
(especially when {\tt df-recs} that it is built on is also eliminated).
But once we get past this hurdle, definitions that would otherwise be
recursive become relatively simple, as in for example {\tt oav}, from
which we prove the recursive textbook definition as theorems {\tt oa0}, {\tt
oasuc}, and {\tt oalim} (with the help of theorems {\tt rdg0}, {\tt rdgsuc},
and {\tt rdglim2a}).  We can also restrict the ${\rm rec}$ operation to
define otherwise recursive functions on the natural numbers $\omega$; see {\tt
fr0g} and {\tt frsuc}.  Our ${\rm rec}$ operation apparently does not appear
in published literature, although closely related is Definition 25.2 of
[Quine] p. 177, which he uses to ``turn...a recursion into a genuine or
direct definition" (p. 174).  Note that the ${\rm if}$ operations (see
{\tt df-if}) select cases based on whether the domain of $g$ is zero, a
successor, or a limit ordinal.

An important use of this definition ${\rm rec}$ is in the recursive sequence
generator {\tt df-seq} on the natural numbers (as a subset of the
complex infinite sequences such as the factorial function {\tt df-fac} and
integer powers {\tt df-exp}).

The definition of ${\rm rec}$ depends on ${\rm recs}$.
New direct usage of the more powerful (and more primitive) ${\rm recs}$
construct is discouraged, but it is available when needed.
This
defines a function $\mathrm{recs} ( F )$ on ${\rm On}$, the class of ordinal
numbers, by transfinite recursion given a rule $F$ which sets the next
value given all values so far.
Unlike {\tt df-rdg} which restricts the
update rule to use only the previous value, this version allows the
update rule to use all previous values, which is why it is described
as ``strong,'' although it is actually more primitive.  See {\tt
recsfnon} and {\tt recsval} for the primary contract of this definition.
It is defined as:

\setbox\startprefix=\hbox{\tt \ \ df-recs\ \$a\ }
\setbox\contprefix=\hbox{\tt \ \ \ \ \ \ \ \ \ \ \ \ \ }
\startm
\m{\vdash}\m{\mathrm{recs}}\m{(}\m{F}\m{)}\m{=}\m{\bigcup}\m{\{}\m{f}\m{|}\m{\exists}\m{x}\m{\in}\m{{\rm
On}}\m{(}\m{f}\m{{\rm
Fn}}\m{x}\m{\wedge}\m{\forall}\m{y}\m{\in}\m{x}\m{(}\m{f}\m{`}\m{y}\m{)}\m{=}\m{(}\m{F}\m{`}\m{(}\m{f}\m{\restriction}\m{y}\m{)}\m{)}\m{)}\m{\}}
\endm

\subsubsection{Closing comments on direct definitions}

From these direct definitions the simpler, more
intuitive recursive definition is derived as a set of theorems.\index{natural
number}\index{addition}\index{recursive definition}\index{ordinal addition}
The end result is the same, but we completely eliminate the rather complex
metalogic that justifies the recursive definition.

Recursive definitions are often considered more efficient and intuitive than
direct ones once the metalogic has been learned or possibly just accepted as
correct.  However, it was felt that direct definition in \texttt{set.mm}
maximizes rigor by minimizing metalogic.  It can be eliminated effortlessly,
something that is difficult to do with a recursive definition.

Again, Metamath itself has no built-in technical limitation that prevents
multiple-part recursive definitions in the traditional textbook style.
Instead, our goal is to eliminate all definitions with
direct mechanical substitution and to verify easily the soundness of
definitions.

\subsection{Adding Constraints on Definitions}

The basic Metamath language and the Metamath program do
not have any built-in constraints on definitions, since they are just
\$a statements.

However, nothing prevents a verification system from verifying
additional rules to impose further limitations on definitions.
For example, the \texttt{mmj2}\index{mmj2} program
supports various kinds of
additional information comments (see section \ref{jcomment}).
One of their uses is to optionally verify additional constraints,
including constraints to verify that definitions meet certain
requirements.
These additional checks are required by the
continuous integration (CI)\index{continuous integration (CI)}
checks of the
\texttt{set.mm}\index{set theory database (\texttt{set.mm})}%
\index{Metamath Proof Explorer}
database.
This approach enables us to optionally impose additional requirements
on definitions if we wish, without necessarily imposing those rules on
all databases or requiring all verification systems to implement them.
In addition, this allows us to impose specialized constraints tailored
to one database while not requiring other systems to implement
those specialized constraints.

We impose two constraints on the
\texttt{set.mm}\index{set theory database (\texttt{set.mm})}%
\index{Metamath Proof Explorer} database
via the \texttt{mmj2}\index{mmj2} program that are worth discussing here:
a parse check and a definition soundness check.

% On February 11, 2019 8:32:32 PM EST, saueran@oregonstate.edu wrote:
% The following addition to the end of set.mm is accepted by the mmj2
% parser and definition checker and the metamath verifier(at least it was
% when I checked, you should check it too), and creates a contradiction by
% proving the theorem |- ph.
% ${
% wleftp $a wff ( ( ph ) $.
% wbothp $a wff ( ph ) $.
% df-leftp $a |- ( ( ( ph ) <-> -. ph ) $.
% df-bothp $a |- ( ( ph ) <-> ph ) $.
% anything $p |- ph $=
%   ( wbothp wn wi wleftp df-leftp biimpi df-bothp mpbir mpbi simplim ax-mp)
%   ABZAMACZDZCZMOEZOCQAEZNDZRNAFGSHIOFJMNKLAHJ $.
% $}
%
% This particular problem is countered by enabling, within mmj2,
% SetParser,mmj.verify.LRParser

First,
we enable a parse check in \texttt{mmj2} (through its
\texttt{SetParser} command) that requires that all new definitions
must \textit{not} create an ambiguous parse for a KLR(5) parser.
This prevents some errors such as definitions with imbalanced parentheses.

Second, we run a definition soundness check specific to
\texttt{set.mm} or databases similar to it.
(through the \texttt{definitionCheck} macro).
Some \texttt{\$a} statements (including all ax-* statemnets)
are excluded from these checks, as they will
always fail this simple check,
but they are appropriate for most definitions.
This check imposes a set of additional rules:

\begin{enumerate}

\item New definitions must be introduced using $=$ or $\leftrightarrow$.

\item No \texttt{\$a} statement introduced before this one is allowed to use the
symbol being defined in this definition, and the definition is not
allowed to use itself (except once, in the definiendum).

\item Every variable in the definiens should not be distinct

\item Every dummy variable in the definiendum
are required to be distinct from each other and from variables in
the definiendum.
To determine this, the system will look for a "justification" theorem
in the database, and if it is not there, attempt to briefly prove
$( \varphi \rightarrow \forall x \varphi )$  for each dummy variable x.

\item Every dummy variable should be a set variable,
unless there is a justification theorem available.

\item Every dummy variable must be bound
(if the system cannot determine this a justification theorem must be
provided).

\end{enumerate}

\subsection{Summary of Approach to Definitions}

In short, when being rigorous it turns out that
definitions can be subtle, sometimes requiring difficult
metatheorems to establish that they are not creative.

Instead of building such complications into the Metamath language itself,
the basic Metmath language and program simply treat traditional
axioms and definitions as the same kind of \texttt{\$a} statement.
We have then built various tools to enable people to
verify additional conditions as their creators believe is appropriate
for those specific databases, without complicating the Metamath foundations.

\chapter{The Metamath Program}\label{commands}

This chapter provides a reference manual for the
Metamath program.\index{Metamath!commands}

Current instructions for obtaining and installing the Metamath program
can be found at the \url{http://metamath.org} web site.
For Windows, there is a pre-compiled version called
\texttt{metamath.exe}.  For Unix, Linux, and Mac OS X
(which we will refer to collectively as ``Unix''), the Metamath program
can be compiled from its source code with the command
\begin{verbatim}
gcc *.c -o metamath
\end{verbatim}
using the \texttt{gcc} {\sc c} compiler available on those systems.

In the command syntax descriptions below, fields enclosed in square
brackets [\ ] are optional.  File names may be optionally enclosed in
single or double quotes.  This is useful if the file name contains
spaces or
slashes (\texttt{/}), such as in Unix path names, \index{Unix file
names}\index{file names!Unix} that might be confused with Metamath
command qualifiers.\index{Unix file names}\index{file names!Unix}

\section{Invoking Metamath}

Unix, Linux, and Mac OS X
have a command-line interface called the {\em
bash shell}.  (In Mac OS X, select the Terminal application from
Applications/Utilities.)  To invoke Metamath from the bash shell prompt,
assuming that the Metamath program is in the current directory, type
\begin{verbatim}
bash$ ./metamath
\end{verbatim}

To invoke Metamath from a Windows DOS or Command Prompt, assuming that
the Metamath program is in the current directory (or in a directory
included in the Path system environment variable), type
\begin{verbatim}
C:\metamath>metamath
\end{verbatim}

To use command-line arguments at invocation, the command-line arguments
should be a list of Metamath commands, surrounded by quotes if they
contain spaces.  In Windows, the surrounding quotes must be double (not
single) quotes.  For example, to read the database file \texttt{set.mm},
verify all proofs, and exit the program, type (under Unix)
\begin{verbatim}
bash$ ./metamath 'read set.mm' 'verify proof *' exit
\end{verbatim}
Note that in Unix, any directory path with \texttt{/}'s must be
surrounded by quotes so Metamath will not interpret the \texttt{/} as a
command qualifier.  So if \texttt{set.mm} is in the \texttt{/tmp}
directory, use for the above example
\begin{verbatim}
bash$ ./metamath 'read "/tmp/set.mm"' 'verify proof *' exit
\end{verbatim}

For convenience, if the command-line has one argument and no spaces in
the argument, the command is implicitly assumed to be \texttt{read}.  In
this one special case, \texttt{/}'s are not interpreted as command
qualifiers, so you don't need quotes around a Unix file name.  Thus
\begin{verbatim}
bash$ ./metamath /tmp/set.mm
\end{verbatim}
and
\begin{verbatim}
bash$ ./metamath "read '/tmp/set.mm'"
\end{verbatim}
are equivalent.


\section{Controlling Metamath}

The Metamath program was first developed on a {\sc vax/vms} system, and
some aspects of its command line behavior reflect this heritage.
Hopefully you will find it reasonably user-friendly once you get used to
it.

Each command line is a sequence of English-like words separated by
spaces, as in \texttt{show settings}.  Command words are not case
sensitive, and only as many letters are needed as are necessary to
eliminate ambiguity; for example, \texttt{sh se} would work for the
command \texttt{show settings}.  In some cases arguments such as file
names, statement labels, or symbol names are required; these are
case-sensitive (although file names may not be on some operating
systems).

A command line is entered by typing it in then pressing the {\em return}
({\em enter}) key.  To find out what commands are available, type
\texttt{?} at the \texttt{MM>} prompt.  To find out the choices at any
point in a command, press {\em return} and you will be prompted for
them.  The default choice (the one selected if you just press {\em
return}) is shown in brackets (\texttt{<>}).

You may also type \texttt{?} in place of a command word to force
Metamath to tell you what the choices are.  The \texttt{?} method won't
work, though, if a non-keyword argument such as a file name is expected
at that point, because the program will think that \texttt{?} is the
value of the argument.

Some commands have one or more optional qualifiers which modify the
behavior of the command.  Qualifiers are preceded by a slash
(\texttt{/}), such as in \texttt{read set.mm / verify}.  Spaces are
optional around the \texttt{/}.  If you need to use a space or
slash in a command
argument, as in a Unix file name, put single or double quotes around the
command argument.

The \texttt{open log} command will save everything you see on the
screen and is useful to help you recover should something go wrong in a
proof, or if you want to document a bug.

If a command responds with more than a screenful, you will be
prompted to \texttt{<return> to continue, Q to quit, or S to scroll to
end}.  \texttt{Q} or \texttt{q} (not case-sensitive) will complete the
command internally but will suppress further output until the next
\texttt{MM>} prompt.  \texttt{s} will suppress further pausing until the
next \texttt{MM>} prompt.  After the first screen, you are also
presented with the choice of \texttt{b} to go back a screenful.  Note
that \texttt{b} may also be entered at the \texttt{MM>} prompt
immediately after a command to scroll back through the output of that
command.

A command line enclosed in quotes is executed by your operating system.
See Section~\ref{oscmd}.

{\em Warning:} Pressing {\sc ctrl-c} will abort the Metamath program
unconditionally.  This means any unsaved work will be lost.


\subsection{\texttt{exit} Command}\index{\texttt{exit} command}

Syntax:  \texttt{exit} [\texttt{/force}]

This command exits from Metamath.  If there have been changes to the
source with the \texttt{save proof} or \texttt{save new{\char`\_}proof}
commands, you will be given an opportunity to \texttt{write source} to
permanently save the changes.

In Proof Assistant\index{Proof Assistant} mode, the \texttt{exit} command will
return to the \verb/MM>/ prompt. If there were changes to the proof, you will
be given an opportunity to \texttt{save new{\char`\_}proof}.

The \texttt{quit} command is a synonym for \texttt{exit}.

Optional qualifier:
    \texttt{/force} - Do not prompt if changes were not saved.  This qualifier is
        useful in \texttt{submit} command files (Section~\ref{sbmt})
        to ensure predictable behavior.





\subsection{\texttt{open log} Command}\index{\texttt{open log} command}
Syntax:  \texttt{open log} {\em file-name}

This command will open a log file that will store everything you see on
the screen.  It is useful to help recovery from a mistake in a long Proof
Assistant session, or to document bugs.\index{Metamath!bugs}

The log file can be closed with \texttt{close log}.  It will automatically be
closed upon exiting Metamath.



\subsection{\texttt{close log} Command}\index{\texttt{close log} command}
Syntax:  \texttt{close log}

The \texttt{close log} command closes a log file if one is open.  See
also \texttt{open log}.




\subsection{\texttt{submit} Command}\index{\texttt{submit} command}\label{sbmt}
Syntax:  \texttt{submit} {\em filename}

This command causes further command lines to be taken from the specified
file.  Note that any line beginning with an exclamation point (\texttt{!}) is
treated as a comment (i.e.\ ignored).  Also note that the scrolling
of the screen output is continuous, so you may want to open a log file
(see \texttt{open log}) to record the results that fly by on the screen.
After the lines in the file are exhausted, Metamath returns to its
normal user interface mode.

The \texttt{submit} command can be called recursively (i.e. from inside
of a \texttt{submit} command file).


Optional command qualifier:

    \texttt{/silent} - suppresses the screen output but still
        records the output in a log file if one is open.


\subsection{\texttt{erase} Command}\index{\texttt{erase} command}
Syntax:  \texttt{erase}

This command will reset Metamath to its starting state, deleting any
data\-base that was \texttt{read} in.
 If there have been changes to the
source with the \texttt{save proof} or \texttt{save new{\char`\_}proof}
commands, you will be given an opportunity to \texttt{write source} to
permanently save the changes.



\subsection{\texttt{set echo} Command}\index{\texttt{set echo} command}
Syntax:  \texttt{set echo on} or \texttt{set echo off}

The \texttt{set echo on} command will cause command lines to be echoed with any
abbreviations expanded.  While learning the Metamath commands, this
feature will show you the exact command that your abbreviated input
corresponds to.



\subsection{\texttt{set scroll} Command}\index{\texttt{set scroll} command}
Syntax:  \texttt{set scroll prompted} or \texttt{set scroll continuous}

The Metamath command line interface starts off in the \texttt{prompted} mode,
which means that you will be prompted to continue or quit after each
full screen in a long listing.  In \texttt{continuous} mode, long listings will be
scrolled without pausing.

% LaTeX bug? (1) \texttt{\_} puts out different character than
% \texttt{{\char`\_}}
%  = \verb$_$  (2) \texttt{{\char`\_}} puts out garbage in \subsection
%  argument
\subsection{\texttt{set width} Command}\index{\texttt{set
width} command}
Syntax:  \texttt{set width} {\em number}

Metamath assumes the width of your screen is 79 characters (chosen
because the Command Prompt in Windows XP has a wrapping bug at column
80).  If your screen is wider or narrower, this command allows you to
change this default screen width.  A larger width is advantageous for
logging proofs to an output file to be printed on a wide printer.  A
smaller width may be necessary on some terminals; in this case, the
wrapping of the information messages may sometimes seem somewhat
unnatural, however.  In \LaTeX\index{latex@{\LaTeX}!characters per line},
there is normally a maximum of 61 characters per line with typewriter
font.  (The examples in this book were produced with 61 characters per
line.)

\subsection{\texttt{set height} Command}\index{\texttt{set
height} command}
Syntax:  \texttt{set height} {\em number}

Metamath assumes your screen height is 24 lines of characters.  If your
screen is taller or shorter, this command lets you to change the number
of lines at which the display pauses and prompts you to continue.

\subsection{\texttt{beep} Command}\index{\texttt{beep} command}

Syntax:  \texttt{beep}

This command will produce a beep.  By typing it ahead after a
long-running command has started, it will alert you that the command is
finished.  For convenience, \texttt{b} is an abbreviation for
\texttt{beep}.

Note:  If \texttt{b} is typed at the \texttt{MM>} prompt immediately
after the end of a multiple-page display paged with ``\texttt{Press
<return> for more}...'' prompts, then the \texttt{b} will back up to the
previous page rather than perform the \texttt{beep} command.
In that case you must type the unabbreviated \texttt{beep} form
of the command.

\subsection{\texttt{more} Command}\index{\texttt{more} command}

Syntax:  \texttt{more} {\em filename}

This command will display the contents of an {\sc ascii} file on your
screen.  (This command is provided for convenience but is not very
powerful.  See Section~\ref{oscmd} to invoke your operating system's
command to do this, such as the \texttt{more} command in Unix.)

\subsection{Operating System Commands}\index{operating system
command}\label{oscmd}

A line enclosed in single or double quotes will be executed by your
computer's operating system if it has a command line interface.  For
example, on a {\sc vax/vms} system,
\verb/MM> 'dir'/
will print disk directory contents.  Note that this feature will not
work on the Macintosh prior to Mac OS X, which does not have a command
line interface.

For your convenience, the trailing quote is optional.

\subsection{Size Limitations in Metamath}

In general, there are no fixed, predefined limits\index{Metamath!memory
limits} on how many labels, tokens\index{token}, statements, etc.\ that
you may have in a database file.  The Metamath program uses 32-bit
variables (64-bit on 64-bit CPUs) as indices for almost all internal
arrays, which are allocated dynamically as needed.



\section{Reading and Writing Files}

The following commands create new files:  the \texttt{open} commands;
the \texttt{write} commands; the \texttt{/html},
\texttt{/alt{\char`\_}html}, \texttt{/brief{\char`\_}html},
\texttt{/brief{\char`\_}alt{\char`\_}html} qualifiers of \texttt{show
statement}, and \texttt{midi}.  The following commands append to files
previously opened:  the \texttt{/tex} qualifier of \texttt{show proof}
and \texttt{show new{\char`\_}proof}; the \texttt{/tex} and
\texttt{/simple{\char`\_}tex} qualifiers of \texttt{show statement}; the
\texttt{close} commands; and all screen dialog between \texttt{open log}
and \texttt{close log}.

The commands that create new files will not overwrite an existing {\em
filename} but will rename the existing one to {\em
filename}\texttt{{\char`\~}1}.  An existing {\em
filename}\texttt{{\char`\~}1} is renamed {\em
filename}\texttt{{\char`\~}2}, etc.\ up to {\em
filename}\texttt{{\char`\~}9}.  An existing {\em
filename}\texttt{{\char`\~}9} is deleted.  This makes recovery from
mistakes easier but also will clutter up your directory, so occasionally
you may want to clean up (delete) these \texttt{{\char`\~}}$n$ files.


\subsection{\texttt{read} Command}\index{\texttt{read} command}
Syntax:  \texttt{read} {\em file-name} [\texttt{/verify}]

This command will read in a Metamath language source file and any included
files.  Normally it will be the first thing you do when entering Metamath.
Statement syntax is checked, but proof syntax is not checked.
Note that the file name may be enclosed in single or double quotes;
this is useful if the file name contains slashes, as might be the case
under Unix.

If you are getting an ``\texttt{?Expected VERIFY}'' error
when trying to read a Unix file name with slashes, you probably haven't
quoted it.\index{Unix file names}\index{file names!Unix}

If you are prompted for the file name (by pressing {\em return}
 after \texttt{read})
you should {\em not} put quotes around it, even if it is a Unix file name
with slashes.

Optional command qualifier:

    \texttt{/verify} - Verify all proofs as the database is read in.  This
         qualifier will slow down reading in the file.  See \texttt{verify
         proof} for more information on file error-checking.

See also \texttt{erase}.



\subsection{\texttt{write source} Command}\index{\texttt{write source} command}
Syntax:  \texttt{write source} {\em filename}
[\texttt{/rewrap}]
[\texttt{/split}]
% TeX doesn't handle this long line with tt text very well,
% so force a line break here.
[\texttt{/keep\_includes}] {\\}
[\texttt{/no\_versioning}]

This command will write the contents of a Metamath\index{database}
database into a file.\index{source file}

Optional command qualifiers:

\texttt{/rewrap} -
Reformats statements and comments according to the
convention used in the set.mm database.
It unwraps the
lines in the comment before each \texttt{\$a} and \texttt{\$p} statement,
then it
rewraps the line.  You should compare the output to the original
to make sure that the desired effect results; if not, go back to
the original source.  The wrapped line length honors the
\texttt{set width}
parameter currently in effect.  Note:  Text
enclosed in \texttt{<HTML>}...\texttt{</HTML>} tags is not modified by the
\texttt{/rewrap} qualifier.
Proofs themselves are not reformatted;
use \texttt{save proof * / compressed} to do that.
An isolated tilde (\~{}) is always kept on the same line as the following
symbol, so you can find all comment references to a symbol by
searching for \~{} followed by a space and that symbol
(this is useful for finding cross-references).
Incidentally, \texttt{save proof} also honors the \texttt{set width}
parameter currently in effect.

\texttt{/split} - Files included in the source using the expression
\$[ \textit{inclfile} \$] will be
written into separate files instead of being included in a single output
file.  The name of each separately written included file will be the
\textit{inclfile} argument of its inclusion command.

\texttt{/keep\_includes} - If a source file has includes but is written as a
single file by omitting \texttt{/split}, by default the included files will
be deleted (actually just renamed with a \char`\~1 suffix unless
\texttt{/no\_versioning} is specified) to prevent the possibly confusing
source duplication in both the output file and the included file.
The \texttt{/keep\_includes} qualifier will prevent this deletion.

\texttt{/no\_versioning} - Backup files suffixed with \char`\~1 are not created.


\section{Showing Status and Statements}



\subsection{\texttt{show settings} Command}\index{\texttt{show settings} command}
Syntax:  \texttt{show settings}

This command shows the state of various parameters.

\subsection{\texttt{show memory} Command}\index{\texttt{show memory} command}
Syntax:  \texttt{show memory}

This command shows the available memory left.  It is not meaningful
on most modern operating systems,
which have virtual memory.\index{Metamath!memory usage}


\subsection{\texttt{show labels} Command}\index{\texttt{show labels} command}
Syntax:  \texttt{show labels} {\em label-match} [\texttt{/all}]
   [\texttt{/linear}]

This command shows the labels of \texttt{\$a} and \texttt{\$p}
statements that match {\em label-match}.  A \verb$*$ in {label-match}
matches zero or more characters.  For example, \verb$*abc*def$ will match all
labels containing \verb$abc$ and ending with \verb$def$.

Optional command qualifiers:

   \texttt{/all} - Include matches for \texttt{\$e} and \texttt{\$f}
   statement labels.

   \texttt{/linear} - Display only one label per line.  This can be useful for
       building scripts in conjunction with the utilities under the
       \texttt{tools}\index{\texttt{tools} command} command.



\subsection{\texttt{show statement} Command}\index{\texttt{show statement} command}
Syntax:  \texttt{show statement} {\em label-match} [{\em qualifiers} (see below)]

This command provides information about a statement.  Only statements
that have labels (\texttt{\$f}\index{\texttt{\$f} statement},
\texttt{\$e}\index{\texttt{\$e} statement},
\texttt{\$a}\index{\texttt{\$a} statement}, and
\texttt{\$p}\index{\texttt{\$p} statement}) may be specified.
If {\em label-match}
contains wildcard (\verb$*$) characters, all matching statements will be
displayed in the order they occur in the database.

Optional qualifiers (only one qualifier at a time is allowed):

    \texttt{/comment} - This qualifier includes the comment that immediately
       precedes the statement.

    \texttt{/full} - Show complete information about each statement,
       and show all
       statements matching {\em label} (including \texttt{\$e}
       and \texttt{\$f} statements).

    \texttt{/tex} - This qualifier will write the statement information to the
       \LaTeX\ file previously opened with \texttt{open tex}.  See
       Section~\ref{texout}.

    \texttt{/simple{\char`\_}tex} - The same as \texttt{/tex}, except that
       \LaTeX\ macros are not used for formatting equations, allowing easier
       manual edits of the output for slide presentations, etc.

    \texttt{/html}\index{html generation@{\sc html} generation},
       \texttt{/alt{\char`\_}html}, \texttt{/brief{\char`\_}html},
       \texttt{/brief{\char`\_}alt{\char`\_}html} -
       These qualifiers invoke a special mode of
       \texttt{show statement} that
       creates a web page for the statement.  They may not be used with
       any other qualifier.  See Section~\ref{htmlout} or
       \texttt{help html} in the program.


\subsection{\texttt{search} Command}\index{\texttt{search} command}
Syntax:  search {\em label-match}
\texttt{"}{\em symbol-match}{\tt}" [\texttt{/all}] [\texttt{/comments}]
[\texttt{/join}]

This command searches all \texttt{\$a} and \texttt{\$p} statements
matching {\em label-match} for occurrences of {\em symbol-match}.  A
\verb@*@ in {\em label-match} matches any label character.  A \verb@$*@
in {\em symbol-match} matches any sequence of symbols.  The symbols in
{\em symbol-match} must be separated by white space.  The quotes
surrounding {\em symbol-match} may be single or double quotes.  For
example, \texttt{search b}\verb@* "-> $* ch"@ will list all statements
whose labels begin with \texttt{b} and contain the symbols \verb@->@ and
\texttt{ch} surrounding any symbol sequence (including no symbol
sequence).  The wildcards \texttt{?} and \texttt{\$?} are also available
to match individual characters in labels and symbols respectively; see
\texttt{help search} in the Metamath program for details on their usage.

Optional command qualifiers:

    \texttt{/all} - Also search \texttt{\$e} and \texttt{\$f} statements.

    \texttt{/comments} - Search the comment that immediately precedes each
        label-matched statement for {\em symbol-match}.  In this case
        {\em symbol-match} is an arbitrary, non-case-sensitive character
        string.  Quotes around {\em symbol-match} are optional if there
        is no ambiguity.

    \texttt{/join} - In the case of a \texttt{\$a} or \texttt{\$p} statement,
	prepend its \texttt{\$e}
	hypotheses for searching. The
	\texttt{/join} qualifier has no effect in \texttt{/comments} mode.

\section{Displaying and Verifying Proofs}


\subsection{\texttt{show proof} Command}\index{\texttt{show proof} command}
Syntax:  \texttt{show proof} {\em label-match} [{\em qualifiers} (see below)]

This command displays the proof of the specified
\texttt{\$p}\index{\texttt{\$p} statement} statement in various formats.
The {\em label-match} may contain wildcard (\verb@$*@) characters to match
multiple statements.  Without any qualifiers, only the logical steps
will be shown (i.e.\ syntax construction steps will be omitted), in an
indented format.

Most of the time, you will use
    \texttt{show proof} {\em label}
to see just the proof steps corresponding to logical inferences.

Optional command qualifiers:

    \texttt{/essential} - The proof tree is trimmed of all
        \texttt{\$f}\index{\texttt{\$f} statement} hypotheses before
        being displayed.  (This is the default, and it is redundant to
        specify it.)

    \texttt{/all} - the proof tree is not trimmed of all \texttt{\$f} hypotheses before
        being displayed.  \texttt{/essential} and \texttt{/all} are mutually exclusive.

    \texttt{/from{\char`\_}step} {\em step} - The display starts at the specified
        step.  If
        this qualifier is omitted, the display starts at the first step.

    \texttt{/to{\char`\_}step} {\em step} - The display ends at the specified
        step.  If this
        qualifier is omitted, the display ends at the last step.

    \texttt{/tree{\char`\_}depth} {\em number} - Only
         steps at less than the specified proof
        tree depth are displayed.  Sometimes useful for obtaining an overview of
        the proof.

    \texttt{/reverse} - The steps are displayed in reverse order.

    \texttt{/renumber} - When used with \texttt{/essential}, the steps are renumbered
        to correspond only to the essential steps.

    \texttt{/tex} - The proof is converted to \LaTeX\ and\index{latex@{\LaTeX}}
        stored in the file opened
        with \texttt{open tex}.  See Section~\ref{texout} or
        \texttt{help tex} in the program.

    \texttt{/lemmon} - The proof is displayed in a non-indented format known
        as Lemmon style, with explicit previous step number references.
        If this qualifier is omitted, steps are indented in a tree format.

    \texttt{/start{\char`\_}column} {\em number} - Overrides the default column
        (16)
        at which the formula display starts in a Lemmon-style display.  May be
        used only in conjunction with \texttt{/lemmon}.

    \texttt{/normal} - The proof is displayed in normal format suitable for
        inclusion in a Metamath source file.  May not be used with any other
        qualifier.

    \texttt{/compressed} - The proof is displayed in compressed format
        suitable for inclusion in a Metamath source file.  May not be used with
        any other qualifier.

    \texttt{/statement{\char`\_}summary} - Summarizes all statements (like a
        brief \texttt{show statement})
        used by the proof.  It may not be used with any other qualifier
        except \texttt{/essential}.

    \texttt{/detailed{\char`\_}step} {\em step} - Shows the details of what is
        happening at
        a specific proof step.  May not be used with any other qualifier.
        The {\em step} is the step number shown when displaying a
        proof without the \texttt{/renumber} qualifier.


\subsection{\texttt{show usage} Command}\index{\texttt{show usage} command}
Syntax:  \texttt{show usage} {\em label-match} [\texttt{/recursive}]

This command lists the statements whose proofs make direct reference to
the statement specified.

Optional command qualifier:

    \texttt{/recursive} - Also include statements whose proofs ultimately
        depend on the statement specified.



\subsection{\texttt{show trace\_back} Command}\index{\texttt{show
       trace{\char`\_}back} command}
Syntax:  \texttt{show trace{\char`\_}back} {\em label-match} [\texttt{/essential}] [\texttt{/axioms}]
    [\texttt{/tree}] [\texttt{/depth} {\em number}]

This command lists all statements that the proof of the \texttt{\$p}
statement(s) specified by {\em label-match} depends on.

Optional command qualifiers:

    \texttt{/essential} - Restrict the trace-back to \texttt{\$e}
        \index{\texttt{\$e} statement} hypotheses of proof trees.

    \texttt{/axioms} - List only the axioms that the proof ultimately depends on.

    \texttt{/tree} - Display the trace-back in an indented tree format.

    \texttt{/depth} {\em number} - Restrict the \texttt{/tree} trace-back to the
        specified indentation depth.

    \texttt{/count{\char`\_}steps} - Count the number of steps the proof has
       all the way back to axioms.  If \texttt{/essential} is specified,
       expansions of variable-type hypotheses (syntax constructions) are not counted.

\subsection{\texttt{verify proof} Command}\index{\texttt{verify proof} command}
Syntax:  \texttt{verify proof} {\em label-match} [\texttt{/syntax{\char`\_}only}]

This command verifies the proofs of the specified statements.  {\em
label-match} may contain wild card characters (\texttt{*}) to verify
more than one proof; for example \verb/*abc*def/ will match all labels
containing \texttt{abc} and ending with \texttt{def}.
The command \texttt{verify proof *} will verify all proofs in the database.

Optional command qualifier:

    \texttt{/syntax{\char`\_}only} - This qualifier will perform a check of syntax
        and RPN
        stack violations only.  It will not verify that the proof is
        correct.  This qualifier is useful for quickly determining which
        proofs are incomplete (i.e.\ are under development and have \texttt{?}'s
        in them).

{\em Note:} \texttt{read}, followed by \texttt{verify proof *}, will ensure
 the database is free
from errors in the Metamath language but will not check the markup notation
in comments.
You can also check the markup notation by running \texttt{verify markup *}
as discussed in Section~\ref{verifymarkup}; see also the discussion
on generating {\sc HTML} in Section~\ref{htmlout}.

\subsection{\texttt{verify markup} Command}\index{\texttt{verify markup} command}\label{verifymarkup}
Syntax:  \texttt{verify markup} {\em label-match}
[\texttt{/date{\char`\_}skip}]
[\texttt{/top{\char`\_}date{\char`\_}skip}] {\\}
[\texttt{/file{\char`\_}skip}]
[\texttt{/verbose}]

This command checks comment markup and other informal conventions we have
adopted.  It error-checks the latexdef, htmldef, and althtmldef statements
in the \texttt{\$t} statement of a Metamath source file.\index{error checking}
It error-checks any \texttt{`}...\texttt{`},
\texttt{\char`\~}~\textit{label},
and bibliographic markups in statement descriptions.
It checks that
\texttt{\$p} and \texttt{\$a} statements
have the same content when their labels start with
``ax'' and ``ax-'' respectively but are otherwise identical, for example
ax4 and ax-4.
It also verifies the date consistency of ``(Contributed by...),''
``(Revised by...),'' and ``(Proof shortened by...)'' tags in the comment
above each \texttt{\$a} and \texttt{\$p} statement.

Optional command qualifiers:

    \texttt{/date{\char`\_}skip} - This qualifier will
        skip date consistency checking,
        which is usually not required for databases other than
	\texttt{set.mm}.

    \texttt{/top{\char`\_}date{\char`\_}skip} - This qualifier will check date consistency except
        that the version date at the top of the database file will not
        be checked.  Only one of
        \texttt{/date{\char`\_}skip} and
        \texttt{/top{\char`\_}date{\char`\_}skip} may be
        specified.

    \texttt{/file{\char`\_}skip} - This qualifier will skip checks that require
        external files to be present, such as checking GIF existence and
        bibliographic links to mmset.html or equivalent.  It is useful
        for doing a quick check from a directory without these files.

    \texttt{/verbose} - Provides more information.  Currently it provides a list
        of axXXX vs. ax-XXX matches.

\subsection{\texttt{save proof} Command}\index{\texttt{save proof} command}
Syntax:  \texttt{save proof} {\em label-match} [\texttt{/normal}]
   [\texttt{/compressed}]

The \texttt{save proof} command will reformat a proof in one of two formats and
replace the existing proof in the source buffer\index{source
buffer}.  It is useful for
converting between proof formats.  Note that a proof will not be
permanently saved until a \texttt{write source} command is issued.

Optional command qualifiers:

    \texttt{/normal} - The proof is saved in the normal format (i.e., as a
        sequence
        of labels, which is the defined format of the basic Metamath
        language).\index{basic language}  This is the default format that
        is used if a qualifier
        is omitted.

    \texttt{/compressed} - The proof is saved in the compressed format which
        reduces storage requirements for a database.
        See Appendix~\ref{compressed}.




\section{Creating Proofs}\label{pfcommands}\index{Proof Assistant}

Before using the Proof Assistant, you must add a \texttt{\$p} to your
source file (using a text editor) containing the statement you want to
prove.  Its proof should consist of a single \texttt{?}, meaning
``unknown step.''  Example:
\begin{verbatim}
equid $p x = x $= ? $.
\end{verbatim}

To enter the Proof assistant, type \texttt{prove} {\em label}, e.g.
\texttt{prove equid}.  Metamath will respond with the \texttt{MM-PA>}
prompt.

Proofs are created working backwards from the statement being proved,
primarily using a series of \texttt{assign} commands.  A proof is
complete when all steps are assigned to statements and all steps are
unified and completely known.  During the creation of a proof, Metamath
will allow only operations that are legal based on what is known up to
that point.  For example, it will not allow an \texttt{assign} of a
statement that cannot be unified with the unknown proof step being
assigned.

{\em Important:}
The Proof Assistant is
{\em not} a tool to help you discover proofs.  It is just a tool to help
you add them to the database.  For a tutorial read
Section~\ref{frstprf}.
To practice using the Proof Assistant, you may
want to \texttt{prove} an existing theorem, then delete all steps with
\texttt{delete all}, then re-create it with the Proof Assistant while
looking at its proof display (before deletion).
You might want to figure out your first few proofs completely
and write them down by hand, before using the Proof Assistant, though
not everyone finds that effective.

{\em Important:}
The \texttt{undo} command if very helpful when entering a proof, because
it allows you to undo a previously-entered step.
In addition, we suggest that you
keep track of your work with a log file (\texttt{open
log}) and save it frequently (\texttt{save new{\char`\_}proof},
\texttt{write source}).
You can use \texttt{delete} to reverse an \texttt{assign}.
You can also do \texttt{delete floating{\char`\_}hypotheses}, then
\texttt{initialize all}, then \texttt{unify all /interactive} to
reinitialize bad unifications made accidentally or by bad
\texttt{assign}s.  You cannot reverse a \texttt{delete} except by
a relevant \texttt{undo} or using
\texttt{exit /force} then reentering the Proof Assistant to recover from
the last \texttt{save new{\char`\_}proof}.

The following commands are available in the Proof Assistant (at the
\texttt{MM-PA>} prompt) to help you create your proof.  See the
individual commands for more detail.

\begin{itemize}
\item[]
    \texttt{show new{\char`\_}proof} [\texttt{/all},...] - Displays the
        proof in progress.  You will use this command a lot; see \texttt{help
        show new{\char`\_}proof} to become familiar with its qualifiers.  The
        qualifiers \texttt{/unknown} and \texttt{/not{\char`\_}unified} are
        useful for seeing the work remaining to be done.  The combination
        \texttt{/all/unknown} is useful for identifying dummy variables that must be
        assigned, or attempts to use illegal syntax, when \texttt{improve all}
        is unable to complete the syntax constructions.  Unknown variables are
        shown as \texttt{\$1}, \texttt{\$2},...
\item[]
    \texttt{assign} {\em step} {\em label} - Assigns an unknown {\em step}
        number with the statement
        specified by {\em label}.
\item[]
    \texttt{let variable} {\em variable}
        \texttt{= "}{\em symbol sequence}\texttt{"}
          - Forces a symbol
        sequence to replace an unknown variable (such as \texttt{\$1}) in a proof.
        It is useful
        for helping difficult unifications, and it is necessary when you have
        dummy variables that eventually must be assigned a name.
\item[]
    \texttt{let step} {\em step} \texttt{= "}{\em symbol sequence}\texttt{"} -
          Forces a symbol sequence
        to replace the contents of a proof step, provided it can be
        unified with the existing step contents.  (I rarely use this.)
\item[]
    \texttt{unify step} {\em step} (or \texttt{unify all}) - Unifies
        the source and target of
        a step.  If you specify a specific step, you will be prompted
        to select among the unifications that are possible.  If you
        specify \texttt{all}, all steps with unique unifications, but only
        those steps, will be
        unified.  \texttt{unify all /interactive} goes through all non-unified
        steps.
\item[]
    \texttt{initialize} {\em step} (or \texttt{all}) - De-unifies the target and source of
        a step (or all steps), as well as the hypotheses of the source,
        and makes all variables in the source unknown.  Useful to recover from
        an \texttt{assign} or \texttt{let} mistake that
        resulted in incorrect unifications.
\item[]
    \texttt{delete} {\em step} (or \texttt{all} or \texttt{floating{\char`\_}hypotheses}) -
        Deletes the specified
        step(s).  \texttt{delete floating{\char`\_}hypotheses}, then \texttt{initialize all}, then
        \texttt{unify all /interactive} is useful for recovering from mistakes
        where incorrect unifications assigned wrong math symbol strings to
        variables.
\item[]
    \texttt{improve} {\em step} (or \texttt{all}) -
      Automatically creates a proof for steps (with no unknown variables)
      whose proof requires no statements with \texttt{\$e} hypotheses.  Useful
      for filling in proofs of \texttt{\$f} hypotheses.  The \texttt{/depth}
      qualifier will also try statements whose \texttt{\$e} hypotheses contain
      no new variables.  {\em Warning:} Save your work (with \texttt{save
      new{\char`\_}proof} then \texttt{write source}) before using
      \texttt{/depth = 2} or greater, since the search time grows
      exponentially and may never terminate in a reasonable time, and you
      cannot interrupt the search.  I have found that it is rare for
      \texttt{/depth = 3} or greater to be useful.
 \item[]
    \texttt{save new{\char`\_}proof} - Saves the proof in progress in the program's
        internal database buffer.  To save it permanently into the database file,
        use \texttt{write source} after
        \texttt{save new{\char`\_}proof}.  To revert to the last
        \texttt{save new{\char`\_}proof},
        \texttt{exit /force} from the Proof Assistant then re-enter the Proof
        Assistant.
 \item[]
    \texttt{match step} {\em step} (or \texttt{match all}) - Shows what
        statements are
        possibilities for the \texttt{assign} statement. (This command
        is not very
        useful in its present form and hopefully will be improved
        eventually.  In the meantime, use the \texttt{search} statement for
        candidates matching specific math token combinations.)
 \item[]
 \texttt{minimize{\char`\_}with}\index{\texttt{minimize{\char`\_}with} command}
% 3/10/07 Note: line-breaking the above results in duplicate index entries
     - After a proof is complete, this command will attempt
        to match other database theorems to the proof to see if the proof
        size can be reduced as a result.  See \texttt{help
        minimize{\char`\_}with} in the
        Metamath program for its usage.
 \item[]
 \texttt{undo}\index{\texttt{undo} command}
    - Undo the effect of a proof-changing command (all but the
      \texttt{show} and \texttt{save} commands above).
 \item[]
 \texttt{redo}\index{\texttt{redo} command}
    - Reverse the previous \texttt{undo}.
\end{itemize}

The following commands set parameters that may be relevant to your proof.
Consult the individual \texttt{help set}... commands.
\begin{itemize}
   \item[] \texttt{set unification{\char`\_}timeout}
 \item[]
    \texttt{set search{\char`\_}limit}
  \item[]
    \texttt{set empty{\char`\_}substitution} - note that default is \texttt{off}
\end{itemize}

Type \texttt{exit} to exit the \texttt{MM-PA>}
 prompt and get back to the \texttt{MM>} prompt.
Another \texttt{exit} will then get you out of Metamath.



\subsection{\texttt{prove} Command}\index{\texttt{prove} command}
Syntax:  \texttt{prove} {\em label}

This command will enter the Proof Assistant\index{Proof Assistant}, which will
allow you to create or edit the proof of the specified statement.
The command-line prompt will change from \texttt{MM>} to \texttt{MM-PA>}.

Note:  In the present version (0.177) of
Metamath\index{Metamath!limitations of version 0.177}, the Proof
Assistant does not verify that \texttt{\$d}\index{\texttt{\$d}
statement} restrictions are met as a proof is being built.  After you
have completed a proof, you should type \texttt{save new{\char`\_}proof}
followed by \texttt{verify proof} {\em label} (where {\em label} is the
statement you are proving with the \texttt{prove} command) to verify the
\texttt{\$d} restrictions.

See also: \texttt{exit}

\subsection{\texttt{set unification\_timeout} Command}\index{\texttt{set
unification{\char`\_}timeout} command}
Syntax:  \texttt{set unification{\char`\_}timeout} {\em number}

(This command is available outside the Proof Assistant but affects the
Proof Assistant\index{Proof Assistant} only.)

Sometimes the Proof Assistant will inform you that a unification
time-out occurred.  This may happen when you try to \texttt{unify}
formulas with many temporary variables\index{temporary variable}
(\texttt{\$1}, \texttt{\$2}, etc.), since the time to compute all possible
unifications may grow exponentially with the number of variables.  If
you want Metamath to try harder (and you're willing to wait longer) you
may increase this parameter.  \texttt{show settings} will show you the
current value.

\subsection{\texttt{set empty\_substitution} Command}\index{\texttt{set
empty{\char`\_}substitution} command}
% These long names can't break well in narrow mode, and even "sloppy"
% is not enough. Work around this by not demanding justification.
\begin{flushleft}
Syntax:  \texttt{set empty{\char`\_}substitution on} or \texttt{set
empty{\char`\_}substitution off}
\end{flushleft}

(This command is available outside the Proof Assistant but affects the
Proof Assistant\index{Proof Assistant} only.)

The Metamath language allows variables to be
substituted\index{substitution!variable}\index{variable substitution}
with empty symbol sequences\index{empty substitution}.  However, in many
formal systems\index{formal system} this will never happen in a valid
proof.  Allowing for this possibility increases the likelihood of
ambiguous unifications\index{ambiguous
unification}\index{unification!ambiguous} during proof creation.
The default is that
empty substitutions are not allowed; for formal systems requiring them,
you must \texttt{set empty{\char`\_}substitution on}.
(An example where this must be \texttt{on}
would be a system that implements a Deduction Rule and in
which deductions from empty assumption lists would be permissible.  The
MIU-system\index{MIU-system} described in Appendix~\ref{MIU} is another
example.)
Note that empty substitutions are
always permissible in proof verification (VERIFY PROOF...) outside the
Proof Assistant.  (See the MIU system in the Metamath book for an example
of a system needing empty substitutions; another example would be a
system that implements a Deduction Rule and in which deductions from
empty assumption lists would be permissible.)

It is better to leave this \texttt{off} when working with \texttt{set.mm}.
Note that this command does not affect the way proofs are verified with
the \texttt{verify proof} command.  Outside of the Proof Assistant,
substitution of empty sequences for math symbols is always allowed.

\subsection{\texttt{set search\_limit} Command}\index{\texttt{set
search{\char`\_}limit} command} Syntax:  \texttt{set search{\char`\_}limit} {\em
number}

(This command is available outside the Proof Assistant but affects the
Proof Assistant\index{Proof Assistant} only.)

This command sets a parameter that determines when the \texttt{improve} command
in Proof Assistant mode gives up.  If you want \texttt{improve} to search harder,
you may increase it.  The \texttt{show settings} command tells you its current
value.


\subsection{\texttt{show new\_proof} Command}\index{\texttt{show
new{\char`\_}proof} command}
Syntax:  \texttt{show new{\char`\_}proof} [{\em
qualifiers} (see below)]

This command (available only in Proof Assistant mode) displays the proof
in progress.  It is identical to the \texttt{show proof} command, except that
there is no statement argument (since it is the statement being proved) and
the following qualifiers are not available:

    \texttt{/statement{\char`\_}summary}

    \texttt{/detailed{\char`\_}step}

Also, the following additional qualifiers are available:

    \texttt{/unknown} - Shows only steps that have no statement assigned.

    \texttt{/not{\char`\_}unified} - Shows only steps that have not been unified.

Note that \texttt{/essential}, \texttt{/depth}, \texttt{/unknown}, and
\texttt{/not{\char`\_}unified} may be
used in any combination; each of them effectively filters out additional
steps from the proof display.

See also:  \texttt{show proof}






\subsection{\texttt{assign} Command}\index{\texttt{assign} command}
Syntax:   \texttt{assign} {\em step} {\em label} [\texttt{/no{\char`\_}unify}]

   and:   \texttt{assign first} {\em label}

   and:   \texttt{assign last} {\em label}


This command, available in the Proof Assistant only, assigns an unknown
step (one with \texttt{?} in the \texttt{show new{\char`\_}proof}
listing) with the statement specified by {\em label}.  The assignment
will not be allowed if the statement cannot be unified with the step.

If \texttt{last} is specified instead of {\em step} number, the last
step that is shown by \texttt{show new{\char`\_}proof /unknown} will be
used.  This can be useful for building a proof with a command file (see
\texttt{help submit}).  It also makes building proofs faster when you know
the assignment for the last step.

If \texttt{first} is specified instead of {\em step} number, the first
step that is shown by \texttt{show new{\char`\_}proof /unknown} will be
used.

If {\em step} is zero or negative, the -{\em step}th from last unknown
step, as shown by \texttt{show new{\char`\_}proof /unknown}, will be
used.  \texttt{assign -1} {\em label} will assign the penultimate
unknown step, \texttt{assign -2} {\em label} the antepenultimate, and
\texttt{assign 0} {\em label} is the same as \texttt{assign last} {\em
label}.

Optional command qualifier:

    \texttt{/no{\char`\_}unify} - do not prompt user to select a unification if there is
        more than one possibility.  This is useful for noninteractive
        command files.  Later, the user can \texttt{unify all /interactive}.
        (The assignment will still be automatically unified if there is only
        one possibility and will be refused if unification is not possible.)



\subsection{\texttt{match} Command}\index{\texttt{match} command}
Syntax:  \texttt{match step} {\em step} [\texttt{/max{\char`\_}essential{\char`\_}hyp}
{\em number}]

    and:  \texttt{match all} [\texttt{/essential}]
          [\texttt{/max{\char`\_}essential{\char`\_}hyp} {\em number}]

This command, available in the Proof Assistant only, shows what
statements can be unified with the specified step(s).  {\em Note:} In
its current form, this command is not very useful because of the large
number of matches it reports.
It may be enhanced in the future.  In the meantime, the \texttt{search}
command can often provide finer control over locating theorems of interest.

Optional command qualifiers:

    \texttt{/max{\char`\_}essential{\char`\_}hyp} {\em number} - filters out
        of the list any statements
        with more than the specified number of
        \texttt{\$e}\index{\texttt{\$e} statement} hypotheses.

    \texttt{/essential{\char`\_}only} - in the \texttt{match all} statement, only
        the steps that
        would be listed in the \texttt{show new{\char`\_}proof /essential} display are
        matched.



\subsection{\texttt{let} Command}\index{\texttt{let} command}
Syntax: \texttt{let variable} {\em variable} = \verb/"/{\em symbol-sequence}\verb/"/

  and: \texttt{let step} {\em step} = \verb/"/{\em symbol-sequence}\verb/"/

These commands, available in the Proof Assistant\index{Proof Assistant}
only, assign a temporary variable\index{temporary variable} or unknown
step with a specific symbol sequence.  They are useful in the middle of
creating a proof, when you know what should be in the proof step but the
unification algorithm doesn't yet have enough information to completely
specify the temporary variables.  A ``temporary variable'' is one that
has the form \texttt{\$}{\em nn} in the proof display, such as
\texttt{\$1}, \texttt{\$2}, etc.  The {\em symbol-sequence} may contain
other unknown variables if desired.  Examples:

    \verb/let variable $32 = "A = B"/

    \verb/let variable $32 = "A = $35"/

    \verb/let step 10 = '|- x = x'/

    \verb/let step -2 = "|- ( $7 = ph )"/

Any symbol sequence will be accepted for the \texttt{let variable}
command.  Only those symbol sequences that can be unified with the step
will be accepted for \texttt{let step}.

The \texttt{let} commands ``zap'' the proof with information that can
only be verified when the proof is built up further.  If you make an
error, the command sequence \texttt{delete
floating{\char`\_}hypotheses}, \texttt{initialize all}, and
\texttt{unify all /interactive} will undo a bad \texttt{let} assignment.

If {\em step} is zero or negative, the -{\em step}th from last unknown
step, as shown by \texttt{show new{\char`\_}proof /unknown}, will be
used.  The command \texttt{let step 0} = \verb/"/{\em
symbol-sequence}\verb/"/ will use the last unknown step, \texttt{let
step -1} = \verb/"/{\em symbol-sequence}\verb/"/ the penultimate, etc.
If {\em step} is positive, \texttt{let step} may be used to assign known
(in the sense of having previously been assigned a label with
\texttt{assign}) as well as unknown steps.

Either single or double quotes can surround the {\em symbol-sequence} as
long as they are different from any quotes inside a {\em
symbol-sequence}.  If {\em symbol-sequence} contains both kinds of
quotes, see the instructions at the end of \texttt{help let} in the
Metamath program.


\subsection{\texttt{unify} Command}\index{\texttt{unify} command}
Syntax:  \texttt{unify step} {\em step}

      and:   \texttt{unify all} [\texttt{/interactive}]

These commands, available in the Proof Assistant only, unify the source
and target of the specified step(s). If you specify a specific step, you
will be prompted to select among the unifications that are possible.  If
you specify \texttt{all}, only those steps with unique unifications will be
unified.

Optional command qualifier for \texttt{unify all}:

    \texttt{/interactive} - You will be prompted to select among the
        unifications
        that are possible for any steps that do not have unique
        unifications.  (Otherwise \texttt{unify all} will bypass these.)

See also \texttt{set unification{\char`\_}timeout}.  The default is
100000, but increasing it to 1000000 can help difficult cases.  Manually
assigning some or all of the unknown variables with the \texttt{let
variable} command also helps difficult cases.



\subsection{\texttt{initialize} Command}\index{\texttt{initialize} command}
Syntax:  \texttt{initialize step} {\em step}

    and: \texttt{initialize all}

These commands, available in the Proof Assistant\index{Proof Assistant}
only, ``de-unify'' the target and source of a step (or all steps), as
well as the hypotheses of the source, and makes all variables in the
source and the source's hypotheses unknown.  This command is useful to
help recover from incorrect unifications that resulted from an incorrect
\texttt{assign}, \texttt{let}, or unification choice.  Part or all of
the command sequence \texttt{delete floating{\char`\_}hypotheses},
\texttt{initialize all}, and \texttt{unify all /interactive} will recover
from incorrect unifications.

See also:  \texttt{unify} and \texttt{delete}



\subsection{\texttt{delete} Command}\index{\texttt{delete} command}
Syntax:  \texttt{delete step} {\em step}

   and:      \texttt{delete all} -- {\em Warning: dangerous!}

   and:      \texttt{delete floating{\char`\_}hypotheses}

These commands are available in the Proof Assistant only.  The
\texttt{delete step} command deletes the proof tree section that
branches off of the specified step and makes the step become unknown.
\texttt{delete all} is equivalent to \texttt{delete step} {\em step}
where {\em step} is the last step in the proof (i.e.\ the beginning of
the proof tree).

In most cases the \texttt{undo} command is the best way to undo
a previous step.
An alternative is to salvage your last \texttt{save
new{\char`\_}proof} by exiting and reentering the Proof Assistant.
For this to work, keep a log file open to record your work
and to do \texttt{save new{\char`\_}proof} frequently, especially before
\texttt{delete}.

\texttt{delete floating{\char`\_}hypotheses} will delete all sections of
the proof that branch off of \texttt{\$f}\index{\texttt{\$f} statement}
statements.  It is sometimes useful to do this before an
\texttt{initialize} command to recover from an error.  Note that once a
proof step with a \texttt{\$f} hypothesis as the target is completely
known, the \texttt{improve} command can usually fill in the proof for
that step.  Unlike the deletion of logical steps, \texttt{delete
floating{\char`\_}hypotheses} is a relatively safe command that is
usually easy to recover from.



\subsection{\texttt{improve} Command}\index{\texttt{improve} command}
\label{improve}
Syntax:  \texttt{improve} {\em step} [\texttt{/depth} {\em number}]
                                               [\texttt{/no{\char`\_}distinct}]

   and:   \texttt{improve first} [\texttt{/depth} {\em number}]
                                              [\texttt{/no{\char`\_}distinct}]

   and:   \texttt{improve last} [\texttt{/depth} {\em number}]
                                              [\texttt{/no{\char`\_}distinct}]

   and:   \texttt{improve all} [\texttt{/depth} {\em number}]
                                              [\texttt{/no{\char`\_}distinct}]

These commands, available in the Proof Assistant\index{Proof Assistant}
only, try to find proofs automatically for unknown steps whose symbol
sequences are completely known.  They are primarily useful for filling in
proofs of \texttt{\$f}\index{\texttt{\$f} statement} hypotheses.  The
search will be restricted to statements having no
\texttt{\$e}\index{\texttt{\$e} statement} hypotheses.

\begin{sloppypar} % narrow
Note:  If memory is limited, \texttt{improve all} on a large proof may
overflow memory.  If you use \texttt{set unification{\char`\_}timeout 1}
before \texttt{improve all}, there will usually be sufficient
improvement to easily recover and completely \texttt{improve} the proof
later on a larger computer.  Warning:  Once memory has overflowed, there
is no recovery.  If in doubt, save the intermediate proof (\texttt{save
new{\char`\_}proof} then \texttt{write source}) before \texttt{improve
all}.
\end{sloppypar}

If \texttt{last} is specified instead of {\em step} number, the last
step that is shown by \texttt{show new{\char`\_}proof /unknown} will be
used.

If \texttt{first} is specified instead of {\em step} number, the first
step that is shown by \texttt{show new{\char`\_}proof /unknown} will be
used.

If {\em step} is zero or negative, the -{\em step}th from last unknown
step, as shown by \texttt{show new{\char`\_}proof /unknown}, will be
used.  \texttt{improve -1} will use the penultimate
unknown step, \texttt{improve -2} {\em label} the antepenultimate, and
\texttt{improve 0} is the same as \texttt{improve last}.

Optional command qualifier:

    \texttt{/depth} {\em number} - This qualifier will cause the search
        to include
        statements with \texttt{\$e} hypotheses (but no new variables in
        the \texttt{\$e}
        hypotheses), provided that the backtracking has not exceeded the
        specified depth. {\em Warning:}  Try \texttt{/depth 1},
        then \texttt{2}, then \texttt{3}, etc.
        in sequence because of possible exponential blowups.  Save your
        work before trying \texttt{/depth} greater than \texttt{1}!

    \texttt{/no{\char`\_}distinct} - Skip trial statements that have
        \texttt{\$d}\index{\texttt{\$d} statement} requirements.
        This qualifier will prevent assignments that might violate \texttt{\$d}
        requirements but it also could miss possible legal assignments.

See also: \texttt{set search{\char`\_}limit}

\subsection{\texttt{save new\_proof} Command}\index{\texttt{save
new{\char`\_}proof} command}
Syntax:  \texttt{save new{\char`\_}proof} {\em label} [\texttt{/normal}]
   [\texttt{/compressed}]

The \texttt{save new{\char`\_}proof} command is available in the Proof
Assistant only.  It saves the proof in progress in the source
buffer\index{source buffer}.  \texttt{save new{\char`\_}proof} may be
used to save a completed proof, or it may be used to save a proof in
progress in order to work on it later.  If an incomplete proof is saved,
any user assignments with \texttt{let step} or \texttt{let variable}
will be lost, as will any ambiguous unifications\index{ambiguous
unification}\index{unification!ambiguous} that were resolved manually.
To help make recovery easier, it can be helpful to \texttt{improve all}
before \texttt{save new{\char`\_}proof} so that the incomplete proof
will have as much information as possible.

Note that the proof will not be permanently saved until a \texttt{write
source} command is issued.

Optional command qualifiers:

    \texttt{/normal} - The proof is saved in the normal format (i.e., as a
        sequence of labels, which is the defined format of the basic Metamath
        language).\index{basic language}  This is the default format that
        is used if a qualifier is omitted.

    \texttt{/compressed} - The proof is saved in the compressed format, which
        reduces storage requirements for a database.
        (See Appendix~\ref{compressed}.)


\section{Creating \LaTeX\ Output}\label{texout}\index{latex@{\LaTeX}}

You can generate \LaTeX\ output given the
information in a database.
The database must already include the necessary typesetting information
(see section \ref{tcomment} for how to provide this information).

The \texttt{show statement} and \texttt{show proof} commands each have a
special \texttt{/tex} command qualifier that produces \LaTeX\ output.
(The \texttt{show statement} command also has the
\texttt{/simple{\char`\_}tex} qualifier for output that is easier to
edit by hand.)  Before you can use them, you must open a \LaTeX\ file to
which to send their output.  A typical complete session will use this
sequence of Metamath commands:

\begin{verbatim}
read set.mm
open tex example.tex
show statement a1i /tex
show proof a1i /all/lemmon/renumber/tex
show statement uneq2 /tex
show proof uneq2 /all/lemmon/renumber/tex
close tex
\end{verbatim}

See Section~\ref{mathcomments} for information on comment markup and
Appendix~\ref{ASCII} for information on how math symbol translation is
specified.

To format and print the \LaTeX\ source, you will need the \LaTeX\
program, which is standard on most Linux installations and available for
Windows.  On Linux, in order to create a {\sc pdf} file, you will
typically type at the shell prompt
\begin{verbatim}
$ pdflatex example.tex
\end{verbatim}

\subsection{\texttt{open tex} Command}\index{\texttt{open tex} command}
Syntax:  \texttt{open tex} {\em file-name} [\texttt{/no{\char`\_}header}]

This command opens a file for writing \LaTeX\
source\index{latex@{\LaTeX}} and writes a \LaTeX\ header to the file.
\LaTeX\ source can be written with the \texttt{show proof}, \texttt{show
new{\char`\_}proof}, and \texttt{show statement} commands using the
\texttt{/tex} qualifier.

The mapping to \LaTeX\ symbols is defined in a special comment
containing a \texttt{\$t} token, described in Appendix~\ref{ASCII}.

There is an optional command qualifier:

    \texttt{/no{\char`\_}header} - This qualifier prevents a standard
        \LaTeX\ header and trailer
        from being included with the output \LaTeX\ code.


\subsection{\texttt{close tex} Command}\index{\texttt{close tex} command}
Syntax:  \texttt{close tex}

This command writes a trailer to any \LaTeX\ file\index{latex@{\LaTeX}}
that was opened with \texttt{open tex} (unless
\texttt{/no{\char`\_}header} was used with \texttt{open tex}) and closes
the \LaTeX\ file.


\section{Creating {\sc HTML} Output}\label{htmlout}

You can generate {\sc html} web pages given the
information in a database.
The database must already include the necessary typesetting information
(see section \ref{tcomment} for how to provide this information).
The ability to produce {\sc html} web pages was added in Metamath version
0.07.30.

To create an {\sc html} output file(s) for \texttt{\$a} or \texttt{\$p}
statement(s), use
\begin{quote}
    \texttt{show statement} {\em label-match} \texttt{/html}
\end{quote}
The output file will be named {\em label-match}\texttt{.html}
for each match.  When {\em
label-match} has wildcard (\texttt{*}) characters, all statements with
matching labels will have {\sc html} files produced for them.  Also,
when {\em label-match} has a wildcard (\texttt{*}) character, two additional
files, \texttt{mmdefinitions.html} and \texttt{mmascii.html} will be
produced.  To produce {\em only} these two additional files, you can use
\texttt{?*}, which will not match any statement label, in place of {\em
label-match}.

There are three other qualifiers for \texttt{show statement} that also
generate {\sc HTML} code.  These are \texttt{/alt{\char`\_}html},
\texttt{/brief{\char`\_}html}, and
\texttt{/brief{\char`\_}alt{\char`\_}html}, and are described in the
next section.

The command
\begin{quote}
    \texttt{show statement} {\em label-match} \texttt{/alt{\char`\_}html}
\end{quote}
does the same as \texttt{show statement} {\em label-match} \texttt{/html},
except that the {\sc html} code for the symbols is taken from
\texttt{althtmldef} statements instead of \texttt{htmldef} statements in
the \texttt{\$t} comment.

The command
\begin{verbatim}
show statement * /brief_html
\end{verbatim}
invokes a special mode that just produces definition and theorem lists
accompanied by their symbol strings, in a format suitable for copying and
pasting into another web page (such as the tutorial pages on the
Metamath web site).

Finally, the command
\begin{verbatim}
show statement * /brief_alt_html
\end{verbatim}
does the same as \texttt{show statement * / brief{\char`\_}html}
for the alternate {\sc html}
symbol representation.

A statement's comment can include a special notation that provides a
certain amount of control over the {\sc HTML} version of the comment.  See
Section~\ref{mathcomments} (p.~\pageref{mathcomments}) for the comment
markup features.

The \texttt{write theorem{\char`\_}list} and \texttt{write bibliography}
commands, which are described below, provide as a side effect complete
error checking for all of the features described in this
section.\index{error checking}

\subsection{\texttt{write theorem\_list}
Command}\index{\texttt{write theorem{\char`\_}list} command}
Syntax:  \texttt{write theorem{\char`\_}list}
[\texttt{/theorems{\char`\_}per{\char`\_}page} {\em number}]

This command writes a list of all of the \texttt{\$a} and \texttt{\$p}
statements in the database into a web page file
 called \texttt{mmtheorems.html}.
When additional files are needed, they are called
\texttt{mmtheorems2.html}, \texttt{mmtheorems3.html}, etc.

Optional command qualifier:

    \texttt{/theorems{\char`\_}per{\char`\_}page} {\em number} -
 This qualifier specifies the number of statements to
        write per web page.  The default is 100.

{\em Note:} In version 0.177\index{Metamath!limitations of version
0.177} of Metamath, the ``Nearby theorems'' links on the individual
web pages presuppose 100 theorems per page when linking to the theorem
list pages.  Therefore the \texttt{/theorems{\char`\_}per{\char`\_}page}
qualifier, if it specifies a number other than 100, will cause the
individual web pages to be out of sync and should not be used to
generate the main theorem list for the web site.  This may be
fixed in a future version.


\subsection{\texttt{write bibliography}\label{wrbib}
Command}\index{\texttt{write bibliography} command}
Syntax:  \texttt{write bibliography} {\em filename}

This command reads an existing {\sc html} bibliographic cross-reference
file, normally called \texttt{mmbiblio.html}, and updates it per the
bibliographic links in the database comments.  The file is updated
between the {\sc html} comment lines \texttt{<!--
{\char`\#}START{\char`\#} -->} and \texttt{<!-- {\char`\#}END{\char`\#}
-->}.  The original input file is renamed to {\em
filename}\texttt{{\char`\~}1}.

A bibliographic reference is indicated with the reference name
in brackets, such as  \texttt{Theorem 3.1 of
[Monk] p.\ 22}.
See Section~\ref{htmlout} (p.~\pageref{htmlout}) for
syntax details.


\subsection{\texttt{write recent\_additions}
Command}\index{\texttt{write recent{\char`\_}additions} command}
Syntax:  \texttt{write recent{\char`\_}additions} {\em filename}
[\texttt{/limit} {\em number}]

This command reads an existing ``Recent Additions'' {\sc html} file,
normally called \texttt{mmrecent.html}, and updates it with the
descriptions of the most recently added theorems to the database.
 The file is updated between
the {\sc html} comment lines \texttt{<!-- {\char`\#}START{\char`\#} -->}
and \texttt{<!-- {\char`\#}END{\char`\#} -->}.  The original input file
is renamed to {\em filename}\texttt{{\char`\~}1}.

Optional command qualifier:

    \texttt{/limit} {\em number} -
 This qualifier specifies the number of most recent theorems to
   write to the output file.  The default is 100.


\section{Text File Utilities}

\subsection{\texttt{tools} Command}\index{\texttt{tools} command}
Syntax:  \texttt{tools}

This command invokes an easy-to-use, general purpose utility for
manipulating the contents of {\sc ascii} text files.  Upon typing
\texttt{tools}, the command-line prompt will change to \texttt{TOOLS>}
until you type \texttt{exit}.  The \texttt{tools} commands can be used
to perform simple, global edits on an input/output file,
such as making a character string substitution on each line, adding a
string to each line, and so on.  A typical use of this utility is
to build a \texttt{submit} input file to perform a common operation on a
list of statements obtained from \texttt{show label} or \texttt{show
usage}.

The actions of most of the \texttt{tools} commands can also be
performed with equivalent (and more powerful) Unix shell commands, and
some users may find those more efficient.  But for Windows users or
users not comfortable with Unix, \texttt{tools} provides an
easy-to-learn alternative that is adequate for most of the
script-building tasks needed to use the Metamath program effectively.

\subsection{\texttt{help} Command (in \texttt{tools})}
Syntax:  \texttt{help}

The \texttt{help} command lists the commands available in the
\texttt{tools} utility, along with a brief description.  Each command,
in turn, has its own help, such as \texttt{help add}.  As with
Metamath's \texttt{MM>} prompt, a complete command can be entered at
once, or just the command word can be typed, causing the program to
prompt for each argument.

\vskip 1ex
\noindent Line-by-line editing commands:

  \texttt{add} - Add a specified string to each line in a file.

  \texttt{clean} - Trim spaces and tabs on each line in a file; convert
         characters.

  \texttt{delete} - Delete a section of each line in a file.

  \texttt{insert} - Insert a string at a specified column in each line of
        a file.

  \texttt{substitute} - Make a simple substitution on each line of the file.

  \texttt{tag} - Like \texttt{add}, but restricted to a range of lines.

  \texttt{swap} - Swap the two halves of each line in a file.

\vskip 1ex
\noindent Other file-processing commands:

  \texttt{break} - Break up (tokenize) a file into a list of tokens (one per
        line).

  \texttt{build} - Build a file with multiple tokens per line from a list.

  \texttt{count} - Count the occurrences in a file of a specified string.

  \texttt{number} - Create a list of numbers.

  \texttt{parallel} - Put two files in parallel.

  \texttt{reverse} - Reverse the order of the lines in a file.

  \texttt{right} - Right-justify lines in a file (useful before sorting
         numbers).

%  \texttt{tag} - Tag edit updates in a program for revision control.

  \texttt{sort} - Sort the lines in a file with key starting at
         specified string.

  \texttt{match} - Extract lines containing (or not) a specified string.

  \texttt{unduplicate} - Eliminate duplicate occurrences of lines in a file.

  \texttt{duplicate} - Extract first occurrence of any line occurring
         more than

   \ \ \    once in a file, discarding lines occurring exactly once.

  \texttt{unique} - Extract lines occurring exactly once in a file.

  \texttt{type} (10 lines) - Display the first few lines in a file.
                  Similar to Unix \texttt{head}.

  \texttt{copy} - Similar to Unix \texttt{cat} but safe (same input
         and output file allowed).

  \texttt{submit} - Run a script containing \texttt{tools} commands.

\vskip 1ex

\noindent Note:
  \texttt{unduplicate}, \texttt{duplicate}, and \texttt{unique} also
 sort the lines as a side effect.


\subsection{Using \texttt{tools} to Build Metamath \texttt{submit}
Scripts}

The \texttt{break} command is typically used to break up a series of
statement labels, such as the output of Metamath's \texttt{show usage},
into one label per line.  The other \texttt{tools} commands can then be
used to add strings before and after each statement label to specify
commands to be performed on the statement.  The \texttt{parallel}
command is useful when a statement label must be mentioned more than
once on a line.

Very often a \texttt{submit} script for Metamath will require multiple
command lines for each statement being processed.  For example, you may
want to enter the Proof Assistant, \texttt{minimize{\char`\_}with} your
latest theorem, \texttt{save} the new proof, and \texttt{exit} the Proof
Assistant.  To accomplish this, you can build a file with these four
commands for each statement on a single line, separating each command
with a designated character such as \texttt{@}.  Then at the end you can
\texttt{substitute} each \texttt{@} with \texttt{{\char`\\}n} to break
up the lines into individual command lines (see \texttt{help
substitute}).


\subsection{Example of a \texttt{tools} Session}

To give you a quick feel for the \texttt{tools} utility, we show a
simple session where we create a file \texttt{n.txt} with 3 lines, add
strings before and after each line, and display the lines on the screen.
You can experiment with the various commands to gain experience with the
\texttt{tools} utility.

\begin{verbatim}
MM> tools
Entering the Text Tools utilities.
Type HELP for help, EXIT to exit.
TOOLS> number
Output file <n.tmp>? n.txt
First number <1>?
Last number <10>? 3
Increment <1>?
TOOLS> add
Input/output file? n.txt
String to add to beginning of each line <>? This is line
String to add to end of each line <>? .
The file n.txt has 3 lines; 3 were changed.
First change is on line 1:
This is line 1.
TOOLS> type n.txt
This is line 1.
This is line 2.
This is line 3.
TOOLS> exit
Exiting the Text Tools.
Type EXIT again to exit Metamath.
MM>
\end{verbatim}



\appendix
\chapter{Sample Representations}
\label{ASCII}

This Appendix provides a sample of {\sc ASCII} representations,
their corresponding traditional mathematical symbols,
and a discussion of their meanings
in the \texttt{set.mm} database.
These are provided in order of appearance.
This is only a partial list, and new definitions are routinely added.
A complete list is available at \url{http://metamath.org}.

These {\sc ASCII} representations, along
with information on how to display them,
are defined in the \texttt{set.mm} database file inside
a special comment called a \texttt{\$t} {\em
comment}\index{\texttt{\$t} comment} or {\em typesetting
comment.}\index{typesetting comment}
A typesetting comment
is indicated by the appearance of the
two-character string \texttt{\$t} at the beginning of the comment.
For more information,
see Section~\ref{tcomment}, p.~\pageref{tcomment}.

In the following table the ``{\sc ASCII}'' column shows the {\sc ASCII}
representation,
``Symbol'' shows the mathematical symbolic display
that corresponds to that {\sc ASCII} representation, ``Labels'' shows
the key label(s) that define the representation, and
``Description'' provides a description about the symbol.
As usual, ``iff'' is short for ``if and only if.''\index{iff}
In most cases the ``{\sc ASCII}'' column only shows
the key token, but it will sometimes show a sequence of tokens
if that is necessary for clarity.

{\setlength{\extrarowsep}{4pt} % Keep rows from being too close together
\begin{longtabu}   { @{} c c l X }
\textbf{ASCII} & \textbf{Symbol} & \textbf{Labels} & \textbf{Description} \\
\endhead
\texttt{|-} & $\vdash$ & &
  ``It is provable that...'' \\
\texttt{ph} & $\varphi$ & \texttt{wph} &
  The wff (boolean) variable phi,
  conventionally the first wff variable. \\
\texttt{ps} & $\psi$ & \texttt{wps} &
  The wff (boolean) variable psi,
  conventionally the second wff variable. \\
\texttt{ch} & $\chi$ & \texttt{wch} &
  The wff (boolean) variable chi,
  conventionally the third wff variable. \\
\texttt{-.} & $\lnot$ & \texttt{wn} &
  Logical not. E.g., if $\varphi$ is true, then $\lnot \varphi$ is false. \\
\texttt{->} & $\rightarrow$ & \texttt{wi} &
  Implies, also known as material implication.
  In classical logic the expression $\varphi \rightarrow \psi$ is true
  if either $\varphi$ is false or $\psi$ is true (or both), that is,
  $\varphi \rightarrow \psi$ has the same meaning as
  $\lnot \varphi \lor \psi$ (as proven in theorem \texttt{imor}). \\
\texttt{<->} & $\leftrightarrow$ &
  \hyperref[df-bi]{\texttt{df-bi}} &
  Biconditional (aka is-equals for boolean values).
  $\varphi \leftrightarrow \psi$ is true iff
  $\varphi$ and $\psi$ have the same value. \\
\texttt{\char`\\/} & $\lor$ &
  \makecell[tl]{{\hyperref[df-or]{\texttt{df-or}}}, \\
	         \hyperref[df-3or]{\texttt{df-3or}}} &
  Disjunction (logical ``or''). $\varphi \lor \psi$ is true iff
  $\varphi$, $\psi$, or both are true. \\
\texttt{/\char`\\} & $\land$ &
  \makecell[tl]{{\hyperref[df-an]{\texttt{df-an}}}, \\
                 \hyperref[df-3an]{\texttt{df-3an}}} &
  Conjunction (logical ``and''). $\varphi \land \psi$ is true iff
  both $\varphi$ and $\psi$ are true. \\
\texttt{A.} & $\forall$ &
  \texttt{wal} &
  For all; the wff $\forall x \varphi$ is true iff
  $\varphi$ is true for all values of $x$. \\
\texttt{E.} & $\exists$ &
  \hyperref[df-ex]{\texttt{df-ex}} &
  There exists; the wff
  $\exists x \varphi$ is true iff
  there is at least one $x$ where $\varphi$ is true. \\
\texttt{[ y / x ]} & $[ y / x ]$ &
  \hyperref[df-sb]{\texttt{df-sb}} &
  The wff $[ y / x ] \varphi$ produces
  the result when $y$ is properly substituted for $x$ in $\varphi$
  ($y$ replaces $x$).
  % This is elsb4
  % ( [ x / y ] z e. y <-> z e. x )
  For example,
  $[ x / y ] z \in y$ is the same as $z \in x$. \\
\texttt{E!} & $\exists !$ &
  \hyperref[df-eu]{\texttt{df-eu}} &
  There exists exactly one;
  $\exists ! x \varphi$ is true iff
  there is at least one $x$ where $\varphi$ is true. \\
\texttt{\{ y | phi \}}  & $ \{ y | \varphi \}$ &
  \hyperref[df-clab]{\texttt{df-clab}} &
  The class of all sets where $\varphi$ is true. \\
\texttt{=} & $ = $ &
  \hyperref[df-cleq]{\texttt{df-cleq}} &
  Class equality; $A = B$ iff $A$ equals $B$. \\
\texttt{e.} & $ \in $ &
  \hyperref[df-clel]{\texttt{df-clel}} &
  Class membership; $A \in B$ if $A$ is a member of $B$. \\
\texttt{{\char`\_}V} & {\rm V} &
  \hyperref[df-v]{\texttt{df-v}} &
  Class of all sets (not itself a set). \\
\texttt{C\_} & $ \subseteq $ &
  \hyperref[df-ss]{\texttt{df-ss}} &
  Subclass (subset); $A \subseteq B$ is true iff
  $A$ is a subclass of $B$. \\
\texttt{u.} & $ \cup $ &
  \hyperref[df-un]{\texttt{df-un}} &
  $A \cup B$ is the union of classes $A$ and $B$. \\
\texttt{i^i} & $ \cap $ &
  \hyperref[df-in]{\texttt{df-in}} &
  $A \cap B$ is the intersection of classes $A$ and $B$. \\
\texttt{\char`\\} & $ \setminus $ &
  \hyperref[df-dif]{\texttt{df-dif}} &
  $A \setminus B$ (set difference)
  is the class of all sets in $A$ except for those in $B$. \\
\texttt{(/)} & $ \varnothing $ &
  \hyperref[df-nul]{\texttt{df-nul}} &
  $ \varnothing $ is the empty set (aka null set). \\
\texttt{\char`\~P} & $ \cal P $ &
  \hyperref[df-pw]{\texttt{df-pw}} &
  Power class. \\
\texttt{<.\ A , B >.} & $\langle A , B \rangle$ &
  \hyperref[df-op]{\texttt{df-op}} &
  The ordered pair $\langle A , B \rangle$. \\
\texttt{( F ` A )} & $ ( F ` A ) $ &
  \hyperref[df-fv]{\texttt{df-fv}} &
  The value of function $F$ when applied to $A$. \\
\texttt{\_i} & $ i $ &
  \texttt{df-i} &
  The square root of negative one. \\
\texttt{x.} & $ \cdot $ &
  \texttt{df-mul} &
  Complex number multiplication; $2~\cdot~3~=~6$. \\
\texttt{CC} & $ \mathbb{C} $ &
  \texttt{df-c} &
  The set of complex numbers. \\
\texttt{RR} & $ \mathbb{R} $ &
  \texttt{df-r} &
  The set of real numbers. \\
\end{longtabu}
} % end of extrarowsep

\chapter{Compressed Proofs}
\label{compressed}\index{compressed proof}\index{proof!compressed}

The proofs in the \texttt{set.mm} set theory database are stored in compressed
format for efficiency.  Normally you needn't concern yourself with the
compressed format, since you can display it with the usual proof display tools
in the Metamath program (\texttt{show proof}\ldots) or convert it to the normal
RPN proof format described in Section~\ref{proof} (with \texttt{save proof}
{\em label} \texttt{/normal}).  However for sake of completeness we describe the
format here and show how it maps to the normal RPN proof format.

A compressed proof, located between \texttt{\$=} and \texttt{\$.}\ keywords, consists
of a left parenthesis, a sequence of statement labels, a right parenthesis,
and a sequence of upper-case letters \texttt{A} through \texttt{Z} (with optional
white space between them).  White space must surround the parentheses
and the labels.  The left parenthesis tells Metamath that a
compressed proof follows.  (A normal RPN proof consists of just a sequence of
labels, and a parenthesis is not a legal character in a label.)

The sequence of upper-case letters corresponds to a sequence of integers
with the following mapping.  Each integer corresponds to a proof step as
described later.
\begin{center}
  \texttt{A} = 1 \\
  \texttt{B} = 2 \\
   \ldots \\
  \texttt{T} = 20 \\
  \texttt{UA} = 21 \\
  \texttt{UB} = 22 \\
   \ldots \\
  \texttt{UT} = 40 \\
  \texttt{VA} = 41 \\
  \texttt{VB} = 42 \\
   \ldots \\
  \texttt{YT} = 120 \\
  \texttt{UUA} = 121 \\
   \ldots \\
  \texttt{YYT} = 620 \\
  \texttt{UUUA} = 621 \\
   etc.
\end{center}

In other words, \texttt{A} through \texttt{T} represent the
least-significant digit in base 20, and \texttt{U} through \texttt{Y}
represent zero or more most-significant digits in base 5, where the
digits start counting at 1 instead of the usual 0. With this scheme, we
don't need white space between these ``numbers.''

(In the design of the compressed proof format, only upper-case letters,
as opposed to say all non-whitespace printable {\sc ascii} characters other than
%\texttt{\$}, was chosen to make the compressed proof a little less
%displeasing to the eye, at the expense of a typical 20\% compression
\texttt{\$}, were chosen so as not to collide with most text editor
searches, at the expense of a typical 20\% compression
loss.  The base 5/base 20 grouping, as opposed to say base 6/base 19,
was chosen by experimentally determining the grouping that resulted in
best typical compression.)

The letter \texttt{Z} identifies (tags) a proof step that is identical to one
that occurs later on in the proof; it helps shorten the proof by not requiring
that identical proof steps be proved over and over again (which happens often
when building wff's).  The \texttt{Z} is placed immediately after the
least-significant digit (letters \texttt{A} through \texttt{T}) that ends the integer
corresponding to the step to later be referenced.

The integers that the upper-case letters correspond to are mapped to labels as
follows.  If the statement being proved has $m$ mandatory hypotheses, integers
1 through $m$ correspond to the labels of these hypotheses in the order shown
by the \texttt{show statement ... / full} command, i.e., the RPN order\index{RPN
order} of the mandatory
hypotheses.  Integers $m+1$ through $m+n$ correspond to the labels enclosed in
the parentheses of the compressed proof, in the order that they appear, where
$n$ is the number of those labels.  Integers $m+n+1$ on up don't directly
correspond to statement labels but point to proof steps identified with the
letter \texttt{Z}, so that these proof steps can be referenced later in the
proof.  Integer $m+n+1$ corresponds to the first step tagged with a \texttt{Z},
$m+n+2$ to the second step tagged with a \texttt{Z}, etc.  When the compressed
proof is converted to a normal proof, the entire subproof of a step tagged
with \texttt{Z} replaces the reference to that step.

For efficiency, Metamath works with compressed proofs directly, without
converting them internally to normal proofs.  In addition to the usual
error-checking, an error message is given if (1) a label in the label list in
parentheses does not refer to a previous \texttt{\$p} or \texttt{\$a} statement or a
non-mandatory hypothesis of the statement being proved and (2) a proof step
tagged with \texttt{Z} is referenced before the step tagged with the \texttt{Z}.

Just as in a normal proof under development (Section~\ref{unknown}), any step
or subproof that is not yet known may be represented with a single \texttt{?}.
White space does not have to appear between the \texttt{?}\ and the upper-case
letters (or other \texttt{?}'s) representing the remainder of the proof.

% April 1, 2004 Appendix C has been added back in with corrections.
%
% May 20, 2003 Appendix C was removed for now because there was a problem found
% by Bob Solovay
%
% Also, removed earlier \ref{formalspec} 's (3 cases above)
%
% Bob Solovay wrote on 30 Nov 2002:
%%%%%%%%%%%%% (start of email comment )
%      3. My next set of comments concern appendix C. I read this before I
% read Chapter 4. So I first noted that the system as presented in the
% Appendix lacked a certain formal property that I thought desirable. I
% then came up with a revised formal system that had this property. Upon
% reading Chapter 4, I noticed that the revised system was closer to the
% treatment in Chapter 4 than the system in Appendix C.
%
%         First a very minor correction:
%
%         On page 142 line 2: The condition that V(e) != V(f) should only be
% required of e, f in T such that e != f.
%
%         Here is a natural property [transitivity] that one would like
% the formal system to have:
%
%         Let Gamma be a set of statements. Suppose that the statement Phi
% is provable from Gamma and that the statement Psi is provable from Gamma
% \cup {Phi}. Then Psi is provable from Gamma.
%
%         I shall present an example to show that this property does not
% hold for the formal systems of Appendix C:
%
%         I write the example in metamath style:
%
% $c A B C D E $.
% $v x y
%
% ${
% tx $f A x $.
% ty $f B y $.
% ax1 $a C x y $.
% $}
%
% ${
% tx $f A x $.
% ty $f B y $.
% ax2-h1 $e C x y $.
% ax2 $a D y $.
% $}
%
% ${
% ty $f B y $.
% ax3-h1 $e D y $.
% ax3 $a E y $.
% $}
%
% $(These three axioms are Gamma $)
%
% ${
% tx $f A x $.
% ty $f B y $.
% Phi $p D y $=
% tx ty tx ty ax1 ax2 $.
% $}
%
% ${
% ty $f B y $.
% Psi $p E y $=
% ty ty Phi ax3 $.
% $}
%
%
% I omit the formal proofs of the following claims. [I will be glad to
% supply them upon request.]
%
% 1) Psi is not provable from Gamma;
%
% 2) Psi is provable from Gamma + Phi.
%
% Here "provable" refers to the formalism of Appendix C.
%
% The trouble of course is that Psi is lacking the variable declaration
%
% $f Ax $.
%
% In the Metamath system there is no trouble proving Psi. I attach a
% metamath file that shows this and which has been checked by the
% metamath program.
%
% I next want to indicate how I think the treatment in Appendix C should
% be revised so as to conform more closely to the metamath system of the
% main text. The revised system *does* have the transitivity property.
%
% We want to give revised definitions of "statement" and
% "provable". [cf. sections C.2.4. and C.2.5] Our new definitions will
% use the definitions given in Appendix C. So we take the following
% tack. We refer to the original notions as o-statement and o-provable. And
% we refer to the notions we are defining as n-statement and n-provable.
%
%         A n-statement is an o-statement in which the only variables
% that appear in the T component are mandatory.
%
%         To any o-statement we can associate its reduct which is a
% n-statement by dropping all the elements of T or D which contain
% non-mandatory variables.
%
%         An n-statement gamma is n-provable if there is an o-statement
% gamma' which has gamma as its reduct andf such that gamma' is
% o-provable.
%
%         It seems to me [though I am not completely sure on this point]
% that n-provability corresponds to metamath provability as discussed
% say in Chapter 4.
%
%         Attached to this letter is the metamath proof of Phi and Psi
% from Gamma discussed above.
%
%         I am still brooding over the question of whether metamath
% correctly formalizes set-theory. No doubt I will have some questions
% re this after my thoughts become clearer.
%%%%%%%%%%%%%%%% (end of email comment)

%%%%%%%%%%%%%%%% (start of 2nd email comment from Bob Solovay 1-Apr-04)
%
%         I hope that Appendix C is the one that gives a "formal" treatment
% of Metamath. At any rate, thats the appendix I want to comment on.
%
%         I'm going to suggest two changes in the definition.
%
%         First change (in the definition of statement): Require that the
% sets D, T, and E be finite.
%
%         Probably things are fine as you give them. But in the applications
% to the main metamath system they will always be finite, and its useful in
% thinking about things [at least for me] to stick to the finite case.
%
%         Second change:
%
%         First let me give an approximate description. Remove the dummy
% variables from the statement. Instead, include them in the proof.
%
%         More formally: Require that T consists of type declarations only
% for mandatory variables. Require that all the pairs in D consist of
% mandatory variables.
%
%         At the start of a proof we are allowed to declare a finite number
% of dummy variables [provided that none of them appear in any of the
% statements in E \cup {A}. We have to supply type declarations for all the
% dummy variables. We are allowed to add new $d statements referring to
% either the mandatory or dummy variables. But we require that no new $d
% statement references only mandatory variables.
%
%         I find this way of doing things more conceptual than the treatment
% in Appendix C. But the change [which I will use implicitly in later
% letters about doing Peano] is mainly aesthetic. I definitely claim that my
% results on doing Peano all apply to Metamath as it is presented in your
% book.
%
%         --Bob
%
%%%%%%%%%%%%%%%% (end of 2nd email comment)

%%
%% When uncommenting the below, also uncomment references above to {formalspec}
%%
\chapter{Metamath's Formal System}\label{formalspec}\index{Metamath!as a formal
system}

\section{Introduction}

\begin{quote}
  {\em Perfection is when there is no longer anything more to take away.}
    \flushright\sc Antoine de
     Saint-Exupery\footnote{\cite[p.~3-25]{Campbell}.}\\
\end{quote}\index{de Saint-Exupery, Antoine}

This appendix describes the theory behind the Metamath language in an abstract
way intended for mathematicians.  Specifically, we construct two
set-theo\-ret\-i\-cal objects:  a ``formal system'' (roughly, a set of syntax
rules, axioms, and logical rules) and its ``universe'' (roughly, the set of
theorems derivable in the formal system).  The Metamath computer language
provides us with a way to describe specific formal systems and, with the aid of
a proof provided by the user, to verify that given theorems
belong to their universes.

To understand this appendix, you need a basic knowledge of informal set theory.
It should be sufficient to understand, for example, Ch.\ 1 of Munkres' {\em
Topology} \cite{Munkres}\index{Munkres, James R.} or the
introductory set theory chapter
in many textbooks that introduce abstract mathematics. (Note that there are
minor notational differences among authors; e.g.\ Munkres uses $\subset$ instead
of our $\subseteq$ for ``subset.''  We use ``included in'' to mean ``a subset
of,'' and ``belongs to'' or ``is contained in'' to mean ``is an element of.'')
What we call a ``formal'' description here, unlike earlier, is actually an
informal description in the ordinary language of mathematicians.  However we
provide sufficient detail so that a mathematician could easily formalize it,
even in the language of Metamath itself if desired.  To understand the logic
examples at the end of this appendix, familiarity with an introductory book on
mathematical logic would be helpful.

\section{The Formal Description}

\subsection[Preliminaries]{Preliminaries\protect\footnotemark}%
\footnotetext{This section is taken mostly verbatim
from Tarski \cite[p.~63]{Tarski1965}\index{Tarski, Alfred}.}

By $\omega$ we denote the set of all natural numbers (non-negative integers).
Each natural number $n$ is identified with the set of all smaller numbers: $n =
\{ m | m < n \}$.  The formula $m < n$ is thus equivalent to the condition: $m
\in n$ and $m,n \in \omega$. In particular, 0 is the number zero and at the
same time the empty set $\varnothing$, $1=\{0\}$, $2=\{0,1\}$, etc. ${}^B A$
denotes the set of all functions on $B$ to $A$ (i.e.\ with domain $B$ and range
included in $A$).  The members of ${}^\omega A$ are what are called {\em simple
infinite sequences},\index{simple infinite sequence}
with all {\em terms}\index{term} in $A$.  In case $n \in \omega$, the
members of ${}^n A$ are referred to as {\em finite $n$-termed
sequences},\index{finite $n$-termed
sequence} again
with terms in $A$.  The consecutive terms (function values) of a finite or
infinite sequence $f$ are denoted by $f_0, f_1, \ldots ,f_n,\ldots$.  Every
finite sequence $f \in \bigcup _{n \in \omega} {}^n A$ uniquely determines the
number $n$ such that $f \in {}^n A$; $n$ is called the {\em
length}\index{length of a sequence ({$"|\ "|$})} of $f$ and
is denoted by $|f|$.  $\langle a \rangle$ is the sequence $f$ with $|f|=1$ and
$f_0=a$; $\langle a,b \rangle$ is the sequence $f$ with $|f|=2$, $f_0=a$,
$f_1=b$; etc.  Given two finite sequences $f$ and $g$, we denote by $f\frown g$
their {\em concatenation},\index{concatenation} i.e., the
finite sequence $h$ determined by the
conditions:
\begin{eqnarray*}
& |h| = |f|+|g|;&  \\
& h_n = f_n & \mbox{\ for\ } n < |f|;  \\
& h_{|f|+n} = g_n & \mbox{\ for\ } n < |g|.
\end{eqnarray*}

\subsection{Constants, Variables, and Expressions}

A formal system has a set of {\em symbols}\index{symbol!in
a formal system} denoted
by $\mbox{\em SM}$.  A
precise set-theo\-ret\-i\-cal definition of this set is unimportant; a symbol
could be considered a primitive or atomic element if we wish.  We assume this
set is divided into two disjoint subsets:  a set $\mbox{\em CN}$ of {\em
constants}\index{constant!in a formal system} and a set $\mbox{\em VR}$ of
{\em variables}.\index{variable!in a formal system}  $\mbox{\em CN}$ and
$\mbox{\em VR}$ are each assumed to consist of countably many symbols which
may be arranged in finite or simple infinite sequences $c_0, c_1, \ldots$ and
$v_0, v_1, \ldots$ respectively, without repeating terms.  We will represent
arbitrary symbols by metavariables $\alpha$, $\beta$, etc.

{\footnotesize\begin{quotation}
{\em Comment.} The variables $v_0, v_1, \ldots$ of our formal system
correspond to what are usually considered ``metavariables'' in
descriptions of specific formal systems in the literature.  Typically,
when describing a specific formal system a book will postulate a set of
primitive objects called variables, then proceed to describe their
properties using metavariables that range over them, never mentioning
again the actual variables themselves.  Our formal system does not
mention these primitive variable objects at all but deals directly with
metavariables, as its primitive objects, from the start.  This is a
subtle but key distinction you should keep in mind, and it makes our
definition of ``formal system'' somewhat different from that typically
found in the literature.  (So, the $\alpha$, $\beta$, etc.\ above are
actually ``metametavariables'' when used to represent $v_0, v_1,
\ldots$.)
\end{quotation}}

Finite sequences all terms of which are symbols are called {\em
expressions}.\index{expression!in a formal system}  $\mbox{\em EX}$ is
the set of all expressions; thus
\begin{displaymath}
\mbox{\em EX} = \bigcup _{n \in \omega} {}^n \mbox{\em SM}.
\end{displaymath}

A {\em constant-prefixed expression}\index{constant-prefixed expression}
is an expression of non-zero length
whose first term is a constant.  We denote the set of all constant-prefixed
expressions by $\mbox{\em EX}_C = \{ e \in \mbox{\em EX} | ( |e| > 0 \wedge
e_0 \in \mbox{\em CN} ) \}$.

A {\em constant-variable pair}\index{constant-variable pair}
is an expression of length 2 whose first term
is a constant and whose second term is a variable.  We denote the set of all
constant-variable pairs by $\mbox{\em EX}_2 = \{ e \in \mbox{\em EX}_C | ( |e|
= 2 \wedge e_1 \in \mbox{\em VR} ) \}$.


{\footnotesize\begin{quotation}
{\em Relationship to Metamath.} In general, the set $\mbox{\em SM}$
corresponds to the set of declared math symbols in a Metamath database, the
set $\mbox{\em CN}$ to those declared with \texttt{\$c} statements, and the set
$\mbox{\em VR}$ to those declared with \texttt{\$v} statements.  Of course a
Metamath database can only have a finite number of math symbols, whereas
formal systems in general can have an infinite number, although the number of
Metamath math symbols available is in principle unlimited.

The set $\mbox{\em EX}_C$ corresponds to the set of permissible expressions
for \texttt{\$e}, \texttt{\$a}, and \texttt{\$p} statements.  The set $\mbox{\em EX}_2$
corresponds to the set of permissible expressions for \texttt{\$f} statements.
\end{quotation}}

We denote by ${\cal V}(e)$ the set of all variables in an expression $e \in
\mbox{\em EX}$, i.e.\ the set of all $\alpha \in \mbox{\em VR}$ such that
$\alpha = e_n$ for some $n < |e|$.  We also denote (with abuse of notation) by
${\cal V}(E)$ the set of all variables in a collection of expressions $E
\subseteq \mbox{\em EX}$, i.e.\ $\bigcup _{e \in E} {\cal V}(e)$.


\subsection{Substitution}

Given a function $F$ from $\mbox{\em VR}$ to
$\mbox{\em EX}$, we
denote by $\sigma_{F}$ or just $\sigma$ the function from $\mbox{\em EX}$ to
$\mbox{\em EX}$ defined recursively for nonempty sequences by
\begin{eqnarray*}
& \sigma(<\alpha>) = F(\alpha) & \mbox{for\ } \alpha \in \mbox{\em VR}; \\
& \sigma(<\alpha>) = <\alpha> & \mbox{for\ } \alpha \not\in \mbox{\em VR}; \\
& \sigma(g \frown h) = \sigma(g) \frown
    \sigma(h) & \mbox{for\ } g,h \in \mbox{\em EX}.
\end{eqnarray*}
We also define $\sigma(\varnothing)=\varnothing$.  We call $\sigma$ a {\em
simultaneous substitution}\index{substitution!variable}\index{variable
substitution} (or just {\em substitution}) with {\em substitution
map}\index{substitution map} $F$.

We also denote (with abuse of notation) by $\sigma(E)$ a substitution on a
collection of expressions $E \subseteq \mbox{\em EX}$, i.e.\ the set $\{
\sigma(e) | e \in E \}$.  The collection $\sigma(E)$ may of course contain
fewer expressions than $E$ because duplicate expressions could result from the
substitution.

\subsection{Statements}

We denote by $\mbox{\em DV}$ the set of all
unordered pairs $\{\alpha, \beta \} \subseteq \mbox{\em VR}$ such that $\alpha
\neq \beta$.  $\mbox{\em DV}$ stands for ``distinct variables.''

A {\em pre-statement}\index{pre-statement!in a formal system} is a
quadruple $\langle D,T,H,A \rangle$ such that
$D\subseteq \mbox{\em DV}$, $T\subseteq \mbox{\em EX}_2$, $H\subseteq
\mbox{\em EX}_C$ and $H$ is finite,
$A\in \mbox{\em EX}_C$, ${\cal V}(H\cup\{A\}) \subseteq
{\cal V}(T)$, and $\forall e,f\in T {\ } {\cal V}(e) \neq {\cal V}(f)$ (or
equivalently, $e_1 \ne f_1$) whenever $e \neq f$. The terms of the quadruple are called {\em
distinct-variable restrictions},\index{disjoint-variable restriction!in a
formal system} {\em variable-type hypotheses},\index{variable-type
hypothesis!in a formal system} {\em logical hypotheses},\index{logical
hypothesis!in a formal system} and the {\em assertion}\index{assertion!in a
formal system} respectively.  We denote by $T_M$ ({\em mandatory variable-type
hypotheses}\index{mandatory variable-type hypothesis!in a formal system}) the
subset of $T$ such that ${\cal V}(T_M) ={\cal V}(H \cup \{A\})$.  We denote by
$D_M=\{\{\alpha,\beta\}\in D|\{\alpha,\beta\}\subseteq {\cal V}(T_M)\}$ the
{\em mandatory distinct-variable restrictions}\index{mandatory
disjoint-variable restriction!in a formal system} of the pre-statement.
The set
of {\em mandatory hypotheses}\index{mandatory hypothesis!in a formal system}
is $T_M\cup H$.  We call the quadruple $\langle D_M,T_M,H,A \rangle$
the {\em reduct}\index{reduct!in a formal system} of
the pre-statement $\langle D,T,H,A \rangle$.

A {\em statement} is the reduct of some pre-statement\index{statement!in a
formal system}.  A statement is therefore a special kind of pre-statement;
in particular, a statement is the reduct of itself.

{\footnotesize\begin{quotation}
{\em Comment.}  $T$ is a set of expressions, each of length 2, that associate
a set of constants (``variable types'') with a set of variables.  The
condition ${\cal V}(H\cup\{A\}) \subseteq {\cal V}(T) $
means that each variable occurring in a statement's logical
hypotheses or assertion must have an associated variable-type hypothesis or
``type declaration,'' in  analogy to a computer programming language, where a
variable must be declared to be say, a string or an integer.  The requirement
that $\forall e,f\in T \, e_1 \ne f_1$ for $e\neq f$
means that each variable must be
associated with a unique constant designating its variable type; e.g., a
variable might be a ``wff'' or a ``set'' but not both.

Distinct-variable restrictions are used to specify what variable substitutions
are permissible to make for the statement to remain valid.  For example, in
the theorem scheme of set theory $\lnot\forall x\,x=y$ we may not substitute
the same variable for both $x$ and $y$.  On the other hand, the theorem scheme
$x=y\to y=x$ does not require that $x$ and $y$ be distinct, so we do not
require a distinct-variable restriction, although having one
would cause no harm other than making the scheme less general.

A mandatory variable-type hypothesis is one whose variable exists in a logical
hypothesis or the assertion.  A provable pre-statement
(defined below) may require
non-mandatory variable-type hypotheses that effectively introduce ``dummy''
variables for use in its proof.  Any number of dummy variables might
be required by a specific proof; indeed, it has been shown by H.\
Andr\'{e}ka\index{Andr{\'{e}}ka, H.} \cite{Nemeti} that there is no finite
upper bound to the number of dummy variables needed to prove an arbitrary
theorem in first-order logic (with equality) having a fixed number $n>2$ of
individual variables.  (See also the Comment on p.~\pageref{nodd}.)
For this reason we do not set a finite size bound on the collections $D$ and
$T$, although in an actual application (Metamath database) these will of
course be finite, increased to whatever size is necessary as more
proofs are added.
\end{quotation}}

{\footnotesize\begin{quotation}
{\em Relationship to Metamath.} A pre-statement of a formal system
corresponds to an extended frame in a Metamath database
(Section~\ref{frames}).  The collections $D$, $T$, and $H$ correspond
respectively to the \texttt{\$d}, \texttt{\$f}, and \texttt{\$e}
statement collections in an extended frame.  The expression $A$
corresponds to the \texttt{\$a} (or \texttt{\$p}) statement in an
extended frame.

A statement of a formal system corresponds to a frame in a Metamath
database.
\end{quotation}}

\subsection{Formal Systems}

A {\em formal system}\index{formal system} is a
triple $\langle \mbox{\em CN},\mbox{\em
VR},\Gamma\rangle$ where $\Gamma$ is a set of statements.  The members of
$\Gamma$ are called {\em axiomatic statements}.\index{axiomatic
statement!in a formal system}  Sometimes we will refer to a
formal system by just $\Gamma$ when $\mbox{\em CN}$ and $\mbox{\em VR}$ are
understood.

Given a formal system $\Gamma$, the {\em closure}\index{closure}\footnote{This
definition of closure incorporates a simplification due to
Josh Purinton.\index{Purinton, Josh}.} of a
pre-statement
$\langle D,T,H,A \rangle$ is the smallest set $C$ of expressions
such that:
%\begin{enumerate}
%  \item $T\cup H\subseteq C$; and
%  \item If for some axiomatic statement
%    $\langle D_M',T_M',H',A' \rangle \in \Gamma_A$, for
%    some $E \subseteq C$, some $F \subseteq C-T$ (where ``-'' denotes
%    set difference), and some substitution
%    $\sigma$ we have
%    \begin{enumerate}
%       \item $\sigma(T_M') = E$ (where, as above, the $M$ denotes the
%           mandatory variable-type hypotheses of $T^A$);
%       \item $\sigma(H') = F$;
%       \item for all $\{\alpha,\beta\}\in D^A$ and $\subseteq
%         {\cal V}(T_M')$, for all $\gamma\in {\cal V}(\sigma(\langle \alpha
%         \rangle))$, and for all $\delta\in  {\cal V}(\sigma(\langle \beta
%         \rangle))$, we have $\{\gamma, \delta\} \in D$;
%   \end{enumerate}
%   then $\sigma(A') \in C$.
%\end{enumerate}
\begin{list}{}{\itemsep 0.0pt}
  \item[1.] $T\cup H\subseteq C$; and
  \item[2.] If for some axiomatic statement
    $\langle D_M',T_M',H',A' \rangle \in
       \Gamma$ and for some substitution
    $\sigma$ we have
    \begin{enumerate}
       \item[a.] $\sigma(T_M' \cup H') \subseteq C$; and
       \item[b.] for all $\{\alpha,\beta\}\in D_M'$, for all $\gamma\in
         {\cal V}(\sigma(\langle \alpha
         \rangle))$, and for all $\delta\in  {\cal V}(\sigma(\langle \beta
         \rangle))$, we have $\{\gamma, \delta\} \in D$;
   \end{enumerate}
   then $\sigma(A') \in C$.
\end{list}
A pre-statement $\langle D,T,H,A
\rangle$ is {\em provable}\index{provable statement!in a formal
system} if $A\in C$ i.e.\ if its assertion belongs to its
closure.  A statement is {\em provable} if it is
the reduct of a provable pre-statement.
The {\em universe}\index{universe of a formal system}
of a formal system is
the collection of all of its provable statements.  Note that the
set of axiomatic statements $\Gamma$ in a formal system is a subset of its
universe.

{\footnotesize\begin{quotation}
{\em Comment.} The first condition in the definition of closure simply says
that the hypotheses of the pre-statement are in its closure.

Condition 2(a) says that a substitution exists that makes the
mandatory hypotheses of an axiomatic statement exactly match some members of
the closure.  This is what we explicitly demonstrate in a Metamath language
proof.

%Conditions 2(a) and 2(b) say that a substitution exists that makes the
%(mandatory) hypotheses of an axiomatic statement exactly match some members of
%the closure.  This is what we explicitly demonstrate with a Metamath language
%proof.
%
%The set of expressions $F$ in condition 2(b) excludes the variable-type
%hypotheses; this is done because non-mandatory variable-type hypotheses are
%effectively ``dropped'' as irrelevant whereas logical hypotheses must be
%retained to achieve a consistent logical system.

Condition 2(b) describes how distinct-variable restrictions in the axiomatic
statement must be met.  It means that after a substitution for two variables
that must be distinct, the resulting two expressions must either contain no
variables, or if they do, they may not have variables in common, and each pair
of any variables they do have, with one variable from each expression, must be
specified as distinct in the original statement.
\end{quotation}}

{\footnotesize\begin{quotation}
{\em Relationship to Metamath.} Axiomatic statements
 and provable statements in a formal
system correspond to the frames for \texttt{\$a} and \texttt{\$p} statements
respectively in a Metamath database.  The set of axiomatic statements is a
subset of the set of provable statements in a formal system, although in a
Metamath database a \texttt{\$a} statement is distinguished by not having a
proof.  A Metamath language proof for a \texttt{\$p} statement tells the computer
how to explicitly construct a series of members of the closure ultimately
leading to a demonstration that the assertion
being proved is in the closure.  The actual closure typically contains
an infinite number of expressions.  A formal system itself does not have
an explicit object called a ``proof'' but rather the existence of a proof
is implied indirectly by membership of an assertion in a provable
statement's closure.  We do this to make the formal system easier
to describe in the language of set theory.

We also note that once established as provable, a statement may be considered
to acquire the same status as an axiomatic statement, because if the set of
axiomatic statements is extended with a provable statement, the universe of
the formal system remains unchanged (provided that $\mbox{\em VR}$ is
infinite).
In practice, this means we can build a hierarchy of provable statements to
more efficiently establish additional provable statements.  This is
what we do in Metamath when we allow proofs to reference previous
\texttt{\$p} statements as well as previous \texttt{\$a} statements.
\end{quotation}}

\section{Examples of Formal Systems}

{\footnotesize\begin{quotation}
{\em Relationship to Metamath.} The examples in this section, except Example~2,
are for the most part exact equivalents of the development in the set
theory database \texttt{set.mm}.  You may want to compare Examples~1, 3, and 5
to Section~\ref{metaaxioms}, Example 4 to Sections~\ref{metadefprop} and
\ref{metadefpred}, and Example 6 to
Section~\ref{setdefinitions}.\label{exampleref}
\end{quotation}}

\subsection{Example~1---Propositional Calculus}\index{propositional calculus}

Classical propositional calculus can be described by the following formal
system.  We assume the set of variables is infinite.  Rather than denoting the
constants and variables by $c_0, c_1, \ldots$ and $v_0, v_1, \ldots$, for
readability we will instead use more conventional symbols, with the
understanding of course that they denote distinct primitive objects.
Also for readability we may omit commas between successive terms of a
sequence; thus $\langle \mbox{wff\ } \varphi\rangle$ denotes
$\langle \mbox{wff}, \varphi\rangle$.

Let
\begin{itemize}
  \item[] $\mbox{\em CN}=\{\mbox{wff}, \vdash, \to, \lnot, (,)\}$
  \item[] $\mbox{\em VR}=\{\varphi,\psi,\chi,\ldots\}$
  \item[] $T = \{\langle \mbox{wff\ } \varphi\rangle,
             \langle \mbox{wff\ } \psi\rangle,
             \langle \mbox{wff\ } \chi\rangle,\ldots\}$, i.e.\ those
             expressions of length 2 whose first member is $\mbox{\rm wff}$
             and whose second member belongs to $\mbox{\em VR}$.\footnote{For
convenience we let $T$ be an infinite set; the definition of a statement
permits this in principle.  Since a Metamath source file has a finite size, in
practice we must of course use appropriate finite subsets of this $T$,
specifically ones containing at least the mandatory variable-type
hypotheses.  Similarly, in the source file we introduce new variables as
required, with the understanding that a potentially infinite number of
them are available.}
\noindent Then $\Gamma$ consists of the axiomatic statements that
are the reducts of the following pre-statements:
    \begin{itemize}
      \item[] $\langle\varnothing,T,\varnothing,
               \langle \mbox{wff\ }(\varphi\to\psi)\rangle\rangle$
      \item[] $\langle\varnothing,T,\varnothing,
               \langle \mbox{wff\ }\lnot\varphi\rangle\rangle$
      \item[] $\langle\varnothing,T,\varnothing,
               \langle \vdash(\varphi\to(\psi\to\varphi))
               \rangle\rangle$
      \item[] $\langle\varnothing,T,
               \varnothing,
               \langle \vdash((\varphi\to(\psi\to\chi))\to
               ((\varphi\to\psi)\to(\varphi\to\chi)))
               \rangle\rangle$
      \item[] $\langle\varnothing,T,
               \varnothing,
               \langle \vdash((\lnot\varphi\to\lnot\psi)\to
               (\psi\to\varphi))\rangle\rangle$
      \item[] $\langle\varnothing,T,
               \{\langle\vdash(\varphi\to\psi)\rangle,
                 \langle\vdash\varphi\rangle\},
               \langle\vdash\psi\rangle\rangle$
    \end{itemize}
\end{itemize}

(For example, the reduct of $\langle\varnothing,T,\varnothing,
               \langle \mbox{wff\ }(\varphi\to\psi)\rangle\rangle$
is
\begin{itemize}
\item[] $\langle\varnothing,
\{\langle \mbox{wff\ } \varphi\rangle,
             \langle \mbox{wff\ } \psi\rangle\},
             \varnothing,
               \langle \mbox{wff\ }(\varphi\to\psi)\rangle\rangle$,
\end{itemize}
which is the first axiomatic statement.)

We call the members of $\mbox{\em VR}$ {\em wff variables} or (in the context
of first-order logic which we will describe shortly) {\em wff metavariables}.
Note that the symbols $\phi$, $\psi$, etc.\ denote actual specific members of
$\mbox{\em VR}$; they are not metavariables of our expository language (which
we denote with $\alpha$, $\beta$, etc.) but are instead (meta)constant symbols
(members of $\mbox{\em SM}$) from the point of view of our expository
language.  The equivalent system of propositional calculus described in
\cite{Tarski1965} also uses the symbols $\phi$, $\psi$, etc.\ to denote wff
metavariables, but in \cite{Tarski1965} unlike here those are metavariables of
the expository language and not primitive symbols of the formal system.

The first two statements define wffs: if $\varphi$ and $\psi$ are wffs, so is
$(\varphi \to \psi)$; if $\varphi$ is a wff, so is $\lnot\varphi$. The next
three are the axioms of propositional calculus: if $\varphi$ and $\psi$ are
wffs, then $\vdash (\varphi \to (\psi \to \varphi))$ is an (axiomatic)
theorem; etc. The
last is the rule of modus ponens: if $\varphi$ and $\psi$ are wffs, and
$\vdash (\varphi\to\psi)$ and $\vdash \varphi$ are theorems, then $\vdash
\psi$ is a theorem.

The correspondence to ordinary propositional calculus is as follows.  We
consider only provable statements of the form $\langle\varnothing,
T,\varnothing,A\rangle$ with $T$ defined as above.  The first term of the
assertion $A$ of any such statement is either ``wff'' or ``$\vdash$''.  A
statement for which the first term is ``wff'' is a {\em wff} of propositional
calculus, and one where the first term is ``$\vdash$'' is a {\em
theorem (scheme)} of propositional calculus.

The universe of this formal system also contains many other provable
statements.  Those with distinct-variable restrictions are irrelevant because
propositional calculus has no constraints on substitutions.  Those that have
logical hypotheses we call {\em inferences}\index{inference} when
the logical hypotheses are of the form
$\langle\vdash\rangle\frown w$ where $w$ is a wff (with the leading constant
term ``wff'' removed).  Inferences (other than the modus ponens rule) are not a
proper part of propositional calculus but are convenient to use when building a
hierarchy of provable statements.  A provable statement with a nonsense
hypothesis such as $\langle \to,\vdash,\lnot\rangle$, and this same expression
as its assertion, we consider irrelevant; no use can be made of it in
proving theorems, since there is no way to eliminate the nonsense hypothesis.

{\footnotesize\begin{quotation}
{\em Comment.} Our use of parentheses in the definition of a wff illustrates
how axiomatic statements should be carefully stated in a way that
ties in unambiguously with the substitutions allowed by the formal system.
There are many ways we could have defined wffs---for example, Polish
prefix notation would have allowed us to omit parentheses entirely, at
the expense of readability---but we must define them in a way that is
unambiguous.  For example, if we had omitted parentheses from the
definition of $(\varphi\to \psi)$, the wff $\lnot\varphi\to \psi$ could
be interpreted as either $\lnot(\varphi\to\psi)$ or $(\lnot\varphi\to\psi)$
and would have allowed us to prove nonsense.  Note that there is no
concept of operator binding precedence built into our formal system.
\end{quotation}}

\begin{sloppy}
\subsection{Example~2---Predicate Calculus with Equality}\index{predicate
calculus}
\end{sloppy}

Here we extend Example~1 to include predicate calculus with equality,
illustrating the use of distinct-variable restrictions.  This system is the
same as Tarski's system $\mathfrak{S}_2$ in \cite{Tarski1965} (except that the
axioms of propositional calculus are different but equivalent, and a redundant
axiom is omitted).  We extend $\mbox{\em CN}$ with the constants
$\{\mbox{var},\forall,=\}$.  We extend $\mbox{\em VR}$ with an infinite set of
{\em individual metavariables}\index{individual
metavariable} $\{x,y,z,\ldots\}$ and denote this subset
$\mbox{\em Vr}$.

We also join to $\mbox{\em CN}$ a possibly infinite set $\mbox{\em Pr}$ of {\em
predicates} $\{R,S,\ldots\}$.  We associate with $\mbox{\em Pr}$ a function
$\mbox{rnk}$ from $\mbox{\em Pr}$ to $\omega$, and for $\alpha\in \mbox{\em
Pr}$ we call $\mbox{rnk}(\alpha)$ the {\em rank} of the predicate $\alpha$,
which is simply the number of ``arguments'' that the predicate has.  (Most
applications of predicate calculus will have a finite number of predicates;
for example, set theory has the single two-argument or binary predicate $\in$,
which is usually written with its arguments surrounding the predicate symbol
rather than with the prefix notation we will use for the general case.)  As a
device to facilitate our discussion, we will let $\mbox{\em Vs}$ be any fixed
one-to-one function from $\omega$ to $\mbox{\em Vr}$; thus $\mbox{\em Vs}$ is
any simple infinite sequence of individual metavariables with no repeating
terms.

In this example we will not include the function symbols that are often part of
formalizations of predicate calculus.  Using metalogical arguments that are
beyond the scope of our discussion, it can be shown that our formalization is
equivalent when functions are introduced via appropriate definitions.

We extend the set $T$ defined in Example~1 with the expressions
$\{\langle \mbox{var\ } x\rangle,$ $ \langle \mbox{var\ } y\rangle, \langle
\mbox{var\ } z\rangle,\ldots\}$.  We extend the $\Gamma$ above
with the axiomatic statements that are the reducts of the following
pre-statements:
\begin{list}{}{\itemsep 0.0pt}
      \item[] $\langle\varnothing,T,\varnothing,
               \langle \mbox{wff\ }\forall x\,\varphi\rangle\rangle$
      \item[] $\langle\varnothing,T,\varnothing,
               \langle \mbox{wff\ }x=y\rangle\rangle$
      \item[] $\langle\varnothing,T,
               \{\langle\vdash\varphi\rangle\},
               \langle\vdash\forall x\,\varphi\rangle\rangle$
      \item[] $\langle\varnothing,T,\varnothing,
               \langle \vdash((\forall x(\varphi\to\psi)
                  \to(\forall x\,\varphi\to\forall x\,\psi))
               \rangle\rangle$
      \item[] $\langle\{\{x,\varphi\}\},T,\varnothing,
               \langle \vdash(\varphi\to\forall x\,\varphi)
               \rangle\rangle$
      \item[] $\langle\{\{x,y\}\},T,\varnothing,
               \langle \vdash\lnot\forall x\lnot x=y
               \rangle\rangle$
      \item[] $\langle\varnothing,T,\varnothing,
               \langle \vdash(x=z
                  \to(x=y\to z=y))
               \rangle\rangle$
      \item[] $\langle\varnothing,T,\varnothing,
               \langle \vdash(y=z
                  \to(x=y\to x=z))
               \rangle\rangle$
\end{list}
These are the axioms not involving predicate symbols. The first two statements
extend the definition of a wff.  The third is the rule of generalization.  The
fifth states, in effect, ``For a wff $\varphi$ and variable $x$,
$\vdash(\varphi\to\forall x\,\varphi)$, provided that $x$ does not occur in
$\varphi$.''  The sixth states ``For variables $x$ and $y$,
$\vdash\lnot\forall x\lnot x = y$, provided that $x$ and $y$ are distinct.''
(This proviso is not necessary but was included by Tarski to
weaken the axiom and still show that the system is logically complete.)

Finally, for each predicate symbol $\alpha\in \mbox{\em Pr}$, we add to
$\Gamma$ an axiomatic statement, extending the definition of wff,
that is the reduct of the following pre-statement:
\begin{displaymath}
    \langle\varnothing,T,\varnothing,
            \langle \mbox{wff},\alpha\rangle\
            \frown \mbox{\em Vs}\restriction\mbox{rnk}(\alpha)\rangle
\end{displaymath}
and for each $\alpha\in \mbox{\em Pr}$ and each $n < \mbox{rnk}(\alpha)$
we add to $\Gamma$ an equality axiom that is the reduct of the
following pre-statement:
\begin{eqnarray*}
    \lefteqn{\langle\varnothing,T,\varnothing,
            \langle
      \vdash,(,\mbox{\em Vs}_n,=,\mbox{\em Vs}_{\mbox{rnk}(\alpha)},\to,
            (,\alpha\rangle\frown \mbox{\em Vs}\restriction\mbox{rnk}(\alpha)} \\
  & & \frown
            \langle\to,\alpha\rangle\frown \mbox{\em Vs}\restriction n\frown
            \langle \mbox{\em Vs}_{\mbox{rnk}(\alpha)}\rangle \\
 & & \frown
            \mbox{\em Vs}\restriction(\mbox{rnk}(\alpha)\setminus(n+1))\frown
            \langle),)\rangle\rangle
\end{eqnarray*}
where $\restriction$ denotes function domain restriction and $\setminus$
denotes set difference.  Recall that a subscript on $\mbox{\em Vs}$
denotes one of its terms.  (In the above two axiom sets commas are placed
between successive terms of sequences to prevent ambiguity, and if you examine
them with care you will be able to distinguish those parentheses that denote
constant symbols from those of our expository language that delimit function
arguments.  Although it might have been better to use boldface for our
primitive symbols, unfortunately boldface was not available for all characters
on the \LaTeX\ system used to typeset this text.)  These seemingly forbidding
axioms can be understood by analogy to concatenation of substrings in a
computer language.  They are actually relatively simple for each specific case
and will become clearer by looking at the special case of a binary predicate
$\alpha = R$ where $\mbox{rnk}(R)=2$.  Letting $\mbox{\em Vs}$ be the sequence
$\langle x,y,z,\ldots\rangle$, the axioms we would add to $\Gamma$ for this
case would be the wff extension and two equality axioms that are the
reducts of the pre-statements:
\begin{list}{}{\itemsep 0.0pt}
      \item[] $\langle\varnothing,T,\varnothing,
               \langle \mbox{wff\ }R x y\rangle\rangle$
      \item[] $\langle\varnothing,T,\varnothing,
               \langle \vdash(x=z
                  \to(R x y \to R z y))
               \rangle\rangle$
      \item[] $\langle\varnothing,T,\varnothing,
               \langle \vdash(y=z
                  \to(R x y \to R x z))
               \rangle\rangle$
\end{list}
Study these carefully to see how the general axioms above evaluate to
them.  In practice, typically only a few special cases such as this would be
needed, and in any case the Metamath language will only permit us to describe
a finite number of predicates, as opposed to the infinite number permitted by
the formal system.  (If an infinite number should be needed for some reason,
we could not define the formal system directly in the Metamath language but
could instead define it metalogically under set theory as we
do in this appendix, and only the underlying set theory, with its single
binary predicate, would be defined directly in the Metamath language.)


{\footnotesize\begin{quotation}
{\em Comment.}  As we noted earlier, the specific variables denoted by the
symbols $x,y,z,\ldots\in \mbox{\em Vr}\subseteq \mbox{\em VR}\subseteq
\mbox{\em SM}$ in Example~2 are not the actual variables of ordinary predicate
calculus but should be thought of as metavariables ranging over them.  For
example, a distinct-variable restriction would be meaningless for actual
variables of ordinary predicate calculus since two different actual variables
are by definition distinct.  And when we talk about an arbitrary
representative $\alpha\in \mbox{\em Vr}$, $\alpha$ is a metavariable (in our
expository language) that ranges over metavariables (which are primitives of
our formal system) each of which ranges over the actual individual variables
of predicate calculus (which are never mentioned in our formal system).

The constant called ``var'' above is called \texttt{setvar} in the
\texttt{set.mm} database file, but it means the same thing.  I felt
that ``var'' is a more meaningful name in the context of predicate
calculus, whose use is not limited to set theory.  For consistency we
stick with the name ``var'' throughout this Appendix, even after set
theory is introduced.
\end{quotation}}

\subsection{Free Variables and Proper Substitution}\index{free variable}
\index{proper substitution}\index{substitution!proper}

Typical representations of mathematical axioms use concepts such
as ``free variable,'' ``bound variable,'' and ``proper substitution''
as primitive notions.
A free variable is a variable that
is not a parameter of any container expression.
A bound variable is the opposite of a free variable; it is a
a variable that has been bound in a container expression.
For example, in the expression $\forall x \varphi$ (for all $x$, $\varphi$
is true), the variable $x$
is bound within the for-all ($\forall$) expression.
It is possible to change one variable to another, and that process is called
``proper substitution.''
In most books, proper substitution has a somewhat complicated recursive
definition with multiple cases based on the occurrences of free and
bound variables.
You may consult
\cite[ch.\ 3--4]{Hamilton}\index{Hamilton, Alan G.} (as well as
many other texts) for more formal details about these terms.

Using these concepts as \texttt{primitives} creates complications
for computer implementations.

In the system of Example~2, there are no primitive notions of free variable
and proper substitution.  Tarski \cite{Tarski1965} shows that this system is
logically equivalent to the more typical textbook systems that do have these
primitive notions, if we introduce these notions with appropriate definitions
and metalogic.  We could also define axioms for such systems directly,
although the recursive definitions of free variable and proper substitution
would be messy and awkward to work with.  Instead, we mention two devices that
can be used in practice to mimic these notions.  (1) Instead of introducing
special notation to express (as a logical hypothesis) ``where $x$ is not free
in $\varphi$'' we can use the logical hypothesis $\vdash(\varphi\to\forall
x\,\varphi)$.\label{effectivelybound}\index{effectively
not free}\footnote{This is a slightly weaker requirement than ``where $x$ is
not free in $\varphi$.''  If we let $\varphi$ be $x=x$, we have the theorem
$(x=x\to\forall x\,x=x)$ which satisfies the hypothesis, even though $x$ is
free in $x=x$ .  In a case like this we say that $x$ is {\em effectively not
free}\index{effectively not free} in $x=x$, since $x=x$ is logically
equivalent to $\forall x\,x=x$ in which $x$ is bound.} (2) It can be shown
that the wff $((x=y\to\varphi)\wedge\exists x(x=y\wedge\varphi))$ (with the
usual definitions of $\wedge$ and $\exists$; see Example~4 below) is logically
equivalent to ``the wff that results from proper substitution of $y$ for $x$
in $\varphi$.''  This works whether or not $x$ and $y$ are distinct.

\subsection{Metalogical Completeness}\index{metalogical completeness}

In the system of Example~2, the
following are provable pre-statements (and their reducts are
provable statements):
\begin{eqnarray*}
      & \langle\{\{x,y\}\},T,\varnothing,
               \langle \vdash\lnot\forall x\lnot x=y
               \rangle\rangle & \\
     &  \langle\varnothing,T,\varnothing,
               \langle \vdash\lnot\forall x\lnot x=x
               \rangle\rangle &
\end{eqnarray*}
whereas the following pre-statement is not to my knowledge provable (but
in any case we will pretend it's not for sake of illustration):
\begin{eqnarray*}
     &  \langle\varnothing,T,\varnothing,
               \langle \vdash\lnot\forall x\lnot x=y
               \rangle\rangle &
\end{eqnarray*}
In other words, we can prove ``$\lnot\forall x\lnot x=y$ where $x$ and $y$ are
distinct'' and separately prove ``$\lnot\forall x\lnot x=x$'', but we can't
prove the combined general case ``$\lnot\forall x\lnot x=y$'' that has no
proviso.  Now this does not compromise logical completeness, because the
variables are really metavariables and the two provable cases together cover
all possible cases.  The third case can be considered a metatheorem whose
direct proof, using the system of Example~2, lies outside the capability of the
formal system.

Also, in the system of Example~2 the following pre-statement is not to my
knowledge provable (again, a conjecture that we will pretend to be the case):
\begin{eqnarray*}
     & \langle\varnothing,T,\varnothing,
               \langle \vdash(\forall x\, \varphi\to\varphi)
               \rangle\rangle &
\end{eqnarray*}
Instead, we can only prove specific cases of $\varphi$ involving individual
metavariables, and by induction on formula length, prove as a metatheorem
outside of our formal system the general statement above.  The details of this
proof are found in \cite{Kalish}.

There does, however, exist a system of predicate calculus in which all such
``simple metatheorems'' as those above can be proved directly, and we present
it in Example~3. A {\em simple metatheorem}\index{simple metatheorem}
is any statement of the formal
system of Example~2 where all distinct variable restrictions consist of either
two individual metavariables or an individual metavariable and a wff
metavariable, and which is provable by combining cases outside the system as
above.  A system is {\em metalogically complete}\index{metalogical
completeness} if all of its simple
metatheorems are (directly) provable statements. The precise definition of
``simple metatheorem'' and the proof of the ``metalogical completeness'' of
Example~3 is found in Remark 9.6 and Theorem 9.7 of \cite{Megill}.\index{Megill,
Norman}

\begin{sloppy}
\subsection{Example~3---Metalogically Complete Predicate
Calculus with
Equality}
\end{sloppy}

For simplicity we will assume there is one binary predicate $R$;
this system suffices for set theory, where the $R$ is of course the $\in$
predicate.  We label the axioms as they appear in \cite{Megill}.  This
system is logically equivalent to that of Example~2 (when the latter is
restricted to this single binary predicate) but is also metalogically
complete.\index{metalogical completeness}

Let
\begin{itemize}
  \item[] $\mbox{\em CN}=\{\mbox{wff}, \mbox{var}, \vdash, \to, \lnot, (,),\forall,=,R\}$.
  \item[] $\mbox{\em VR}=\{\varphi,\psi,\chi,\ldots\}\cup\{x,y,z,\ldots\}$.
  \item[] $T = \{\langle \mbox{wff\ } \varphi\rangle,
             \langle \mbox{wff\ } \psi\rangle,
             \langle \mbox{wff\ } \chi\rangle,\ldots\}\cup
       \{\langle \mbox{var\ } x\rangle, \langle \mbox{var\ } y\rangle, \langle
       \mbox{var\ }z\rangle,\ldots\}$.

\noindent Then
  $\Gamma$ consists of the reducts of the following pre-statements:
    \begin{itemize}
      \item[] $\langle\varnothing,T,\varnothing,
               \langle \mbox{wff\ }(\varphi\to\psi)\rangle\rangle$
      \item[] $\langle\varnothing,T,\varnothing,
               \langle \mbox{wff\ }\lnot\varphi\rangle\rangle$
      \item[] $\langle\varnothing,T,\varnothing,
               \langle \mbox{wff\ }\forall x\,\varphi\rangle\rangle$
      \item[] $\langle\varnothing,T,\varnothing,
               \langle \mbox{wff\ }x=y\rangle\rangle$
      \item[] $\langle\varnothing,T,\varnothing,
               \langle \mbox{wff\ }Rxy\rangle\rangle$
      \item[(C1$'$)] $\langle\varnothing,T,\varnothing,
               \langle \vdash(\varphi\to(\psi\to\varphi))
               \rangle\rangle$
      \item[(C2$'$)] $\langle\varnothing,T,
               \varnothing,
               \langle \vdash((\varphi\to(\psi\to\chi))\to
               ((\varphi\to\psi)\to(\varphi\to\chi)))
               \rangle\rangle$
      \item[(C3$'$)] $\langle\varnothing,T,
               \varnothing,
               \langle \vdash((\lnot\varphi\to\lnot\psi)\to
               (\psi\to\varphi))\rangle\rangle$
      \item[(C4$'$)] $\langle\varnothing,T,
               \varnothing,
               \langle \vdash(\forall x(\forall x\,\varphi\to\psi)\to
                 (\forall x\,\varphi\to\forall x\,\psi))\rangle\rangle$
      \item[(C5$'$)] $\langle\varnothing,T,
               \varnothing,
               \langle \vdash(\forall x\,\varphi\to\varphi)\rangle\rangle$
      \item[(C6$'$)] $\langle\varnothing,T,
               \varnothing,
               \langle \vdash(\forall x\forall y\,\varphi\to
                 \forall y\forall x\,\varphi)\rangle\rangle$
      \item[(C7$'$)] $\langle\varnothing,T,
               \varnothing,
               \langle \vdash(\lnot\varphi\to\forall x\lnot\forall x\,\varphi
                 )\rangle\rangle$
      \item[(C8$'$)] $\langle\varnothing,T,
               \varnothing,
               \langle \vdash(x=y\to(x=z\to y=z))\rangle\rangle$
      \item[(C9$'$)] $\langle\varnothing,T,
               \varnothing,
               \langle \vdash(\lnot\forall x\, x=y\to(\lnot\forall x\, x=z\to
                 (y=z\to\forall x\, y=z)))\rangle\rangle$
      \item[(C10$'$)] $\langle\varnothing,T,
               \varnothing,
               \langle \vdash(\forall x(x=y\to\forall x\,\varphi)\to
                 \varphi))\rangle\rangle$
      \item[(C11$'$)] $\langle\varnothing,T,
               \varnothing,
               \langle \vdash(\forall x\, x=y\to(\forall x\,\varphi
               \to\forall y\,\varphi))\rangle\rangle$
      \item[(C12$'$)] $\langle\varnothing,T,
               \varnothing,
               \langle \vdash(x=y\to(Rxz\to Ryz))\rangle\rangle$
      \item[(C13$'$)] $\langle\varnothing,T,
               \varnothing,
               \langle \vdash(x=y\to(Rzx\to Rzy))\rangle\rangle$
      \item[(C15$'$)] $\langle\varnothing,T,
               \varnothing,
               \langle \vdash(\lnot\forall x\, x=y\to(x=y\to(\varphi
                 \to\forall x(x=y\to\varphi))))\rangle\rangle$
      \item[(C16$'$)] $\langle\{\{x,y\}\},T,
               \varnothing,
               \langle \vdash(\forall x\, x=y\to(\varphi\to\forall x\,\varphi)
                 )\rangle\rangle$
      \item[(C5)] $\langle\{\{x,\varphi\}\},T,\varnothing,
               \langle \vdash(\varphi\to\forall x\,\varphi)
               \rangle\rangle$
      \item[(MP)] $\langle\varnothing,T,
               \{\langle\vdash(\varphi\to\psi)\rangle,
                 \langle\vdash\varphi\rangle\},
               \langle\vdash\psi\rangle\rangle$
      \item[(Gen)] $\langle\varnothing,T,
               \{\langle\vdash\varphi\rangle\},
               \langle\vdash\forall x\,\varphi\rangle\rangle$
    \end{itemize}
\end{itemize}

While it is known that these axioms are ``metalogically complete,'' it is
not known whether they are independent (i.e.\ none is
redundant) in the metalogical sense; specifically, whether any axiom (possibly
with additional non-mandatory distinct-variable restrictions, for use with any
dummy variables in its proof) is provable from the others.  Note that
metalogical independence is a weaker requirement than independence in the
usual logical sense.  Not all of the above axioms are logically independent:
for example, C9$'$ can be proved as a metatheorem from the others, outside the
formal system, by combining the possible cases of distinct variables.

\subsection{Example~4---Adding Definitions}\index{definition}
There are several ways to add definitions to a formal system.  Probably the
most proper way is to consider definitions not as part of the formal system at
all but rather as abbreviations that are part of the expository metalogic
outside the formal system.  For convenience, though, we may use the formal
system itself to incorporate definitions, adding them as axiomatic extensions
to the system.  This could be done by adding a constant representing the
concept ``is defined as'' along with axioms for it. But there is a nicer way,
at least in this writer's opinion, that introduces definitions as direct
extensions to the language rather than as extralogical primitive notions.  We
introduce additional logical connectives and provide axioms for them.  For
systems of logic such as Examples 1 through 3, the additional axioms must be
conservative in the sense that no wff of the original system that was not a
theorem (when the initial term ``wff'' is replaced by ``$\vdash$'' of course)
becomes a theorem of the extended system.  In this example we extend Example~3
(or 2) with standard abbreviations of logic.

We extend $\mbox{\em CN}$ of Example~3 with new constants $\{\leftrightarrow,
\wedge,\vee,\exists\}$, corresponding to logical equivalence,\index{logical
equivalence ($\leftrightarrow$)}\index{biconditional ($\leftrightarrow$)}
conjunction,\index{conjunction ($\wedge$)} disjunction,\index{disjunction
($\vee$)} and the existential quantifier.\index{existential quantifier
($\exists$)}  We extend $\Gamma$ with the axiomatic statements that are
the reducts of the following pre-statements:
\begin{list}{}{\itemsep 0.0pt}
      \item[] $\langle\varnothing,T,\varnothing,
               \langle \mbox{wff\ }(\varphi\leftrightarrow\psi)\rangle\rangle$
      \item[] $\langle\varnothing,T,\varnothing,
               \langle \mbox{wff\ }(\varphi\vee\psi)\rangle\rangle$
      \item[] $\langle\varnothing,T,\varnothing,
               \langle \mbox{wff\ }(\varphi\wedge\psi)\rangle\rangle$
      \item[] $\langle\varnothing,T,\varnothing,
               \langle \mbox{wff\ }\exists x\, \varphi\rangle\rangle$
  \item[] $\langle\varnothing,T,\varnothing,
     \langle\vdash ( ( \varphi \leftrightarrow \psi ) \to
     ( \varphi \to \psi ) )\rangle\rangle$
  \item[] $\langle\varnothing,T,\varnothing,
     \langle\vdash ((\varphi\leftrightarrow\psi)\to
    (\psi\to\varphi))\rangle\rangle$
  \item[] $\langle\varnothing,T,\varnothing,
     \langle\vdash ((\varphi\to\psi)\to(
     (\psi\to\varphi)\to(\varphi
     \leftrightarrow\psi)))\rangle\rangle$
  \item[] $\langle\varnothing,T,\varnothing,
     \langle\vdash (( \varphi \wedge \psi ) \leftrightarrow\neg ( \varphi
     \to \neg \psi )) \rangle\rangle$
  \item[] $\langle\varnothing,T,\varnothing,
     \langle\vdash (( \varphi \vee \psi ) \leftrightarrow (\neg \varphi
     \to \psi )) \rangle\rangle$
  \item[] $\langle\varnothing,T,\varnothing,
     \langle\vdash (\exists x \,\varphi\leftrightarrow
     \lnot \forall x \lnot \varphi)\rangle\rangle$
\end{list}
The first three logical axioms (statements containing ``$\vdash$'') introduce
and effectively define logical equivalence, ``$\leftrightarrow$''.  The last
three use ``$\leftrightarrow$'' to effectively mean ``is defined as.''

\subsection{Example~5---ZFC Set Theory}\index{ZFC set theory}

Here we add to the system of Example~4 the axioms of Zermelo--Fraenkel set
theory with Choice.  For convenience we make use of the
definitions in Example~4.

In the $\mbox{\em CN}$ of Example~4 (which extends Example~3), we replace the symbol $R$
with the symbol $\in$.
More explicitly, we remove from $\Gamma$ of Example~4 the three
axiomatic statements containing $R$ and replace them with the
reducts of the following:
\begin{list}{}{\itemsep 0.0pt}
      \item[] $\langle\varnothing,T,\varnothing,
               \langle \mbox{wff\ }x\in y\rangle\rangle$
      \item[] $\langle\varnothing,T,
               \varnothing,
               \langle \vdash(x=y\to(x\in z\to y\in z))\rangle\rangle$
      \item[] $\langle\varnothing,T,
               \varnothing,
               \langle \vdash(x=y\to(z\in x\to z\in y))\rangle\rangle$
\end{list}
Letting $D=\{\{\alpha,\beta\}\in \mbox{\em DV}\,|\alpha,\beta\in \mbox{\em
Vr}\}$ (in other words all individual variables must be distinct), we extend
$\Gamma$ with the ZFC axioms, called
\index{Axiom of Extensionality}
\index{Axiom of Replacement}
\index{Axiom of Union}
\index{Axiom of Power Sets}
\index{Axiom of Regularity}
\index{Axiom of Infinity}
\index{Axiom of Choice}
Extensionality, Replacement, Union, Power
Set, Regularity, Infinity, and Choice, that are the reducts of:
\begin{list}{}{\itemsep 0.0pt}
      \item[Ext] $\langle D,T,
               \varnothing,
               \langle\vdash (\forall x(x\in y\leftrightarrow x \in z)\to y
               =z) \rangle\rangle$
      \item[Rep] $\langle D,T,
               \varnothing,
               \langle\vdash\exists x ( \exists y \forall z (\varphi \to z = y
                        ) \to
                        \forall z ( z \in x \leftrightarrow \exists x ( x \in
                        y \wedge \forall y\,\varphi ) ) )\rangle\rangle$
      \item[Un] $\langle D,T,
               \varnothing,
               \langle\vdash \exists x \forall y ( \exists x ( y \in x \wedge
               x \in z ) \to y \in x ) \rangle\rangle$
      \item[Pow] $\langle D,T,
               \varnothing,
               \langle\vdash \exists x \forall y ( \forall x ( x \in y \to x
               \in z ) \to y \in x ) \rangle\rangle$
      \item[Reg] $\langle D,T,
               \varnothing,
               \langle\vdash (  x \in y \to
                 \exists x ( x \in y \wedge \forall z ( z \in x \to \lnot z
                \in y ) ) ) \rangle\rangle$
      \item[Inf] $\langle D,T,
               \varnothing,
               \langle\vdash \exists x(y\in x\wedge\forall y(y\in
               x\to
               \exists z(y \in z\wedge z\in x))) \rangle\rangle$
      \item[AC] $\langle D,T,
               \varnothing,
               \langle\vdash \exists x \forall y \forall z ( ( y \in z
               \wedge z \in w ) \to \exists w \forall y ( \exists w
              ( ( y \in z \wedge z \in w ) \wedge ( y \in w \wedge w \in x
              ) ) \leftrightarrow y = w ) ) \rangle\rangle$
\end{list}

\subsection{Example~6---Class Notation in Set Theory}\label{class}

A powerful device that makes set theory easier (and that we have
been using all along in our informal expository language) is {\em class
abstraction notation}.\index{class abstraction}\index{abstraction class}  The
definitions we introduce are rigorously justified
as conservative by Takeuti and Zaring \cite{Takeuti}\index{Takeuti, Gaisi} or
Quine \cite{Quine}\index{Quine, Willard Van Orman}.  The key idea is to
introduce the notation $\{x|\mbox{---}\}$ which means ``the class of all $x$
such that ---'' for abstraction classes and introduce (meta)variables that
range over them.  An abstraction class may or may not be a set, depending on
whether it exists (as a set).  A class that does not exist is
called a {\em proper class}.\index{proper class}\index{class!proper}

To illustrate the use of abstraction classes we will provide some examples
of definitions that make use of them:  the empty set, class union, and
unordered pair.  Many other such definitions can be found in the
Metamath set theory database,
\texttt{set.mm}.\index{set theory database (\texttt{set.mm})}

% We intentionally break up the sequence of math symbols here
% because otherwise the overlong line goes beyond the page in narrow mode.
We extend $\mbox{\em CN}$ of Example~5 with new symbols $\{$
$\mbox{class},$ $\{,$ $|,$ $\},$ $\varnothing,$ $\cup,$ $,$ $\}$
where the inner braces and last comma are
constant symbols. (As before,
our dual use of some mathematical symbols for both our expository
language and as primitives of the formal system should be clear from context.)

We extend $\mbox{\em VR}$ of Example~5 with a set of {\em class
variables}\index{class variable}
$\{A,B,C,\ldots\}$. We extend the $T$ of Example~5 with $\{\langle
\mbox{class\ } A\rangle, \langle \mbox{class\ }B\rangle, \langle \mbox{class\ }
C\rangle,\ldots\}$.

To
introduce our definitions,
we add to $\Gamma$ of Example~5 the axiomatic statements
that are the reducts of the following pre-statements:
\begin{list}{}{\itemsep 0.0pt}
      \item[] $\langle\varnothing,T,\varnothing,
               \langle \mbox{class\ }x\rangle\rangle$
      \item[] $\langle\varnothing,T,\varnothing,
               \langle \mbox{class\ }\{x|\varphi\}\rangle\rangle$
      \item[] $\langle\varnothing,T,\varnothing,
               \langle \mbox{wff\ }A=B\rangle\rangle$
      \item[] $\langle\varnothing,T,\varnothing,
               \langle \mbox{wff\ }A\in B\rangle\rangle$
      \item[Ab] $\langle\varnothing,T,\varnothing,
               \langle \vdash ( y \in \{ x |\varphi\} \leftrightarrow
                  ( ( x = y \to\varphi) \wedge \exists x ( x = y
                  \wedge\varphi) ))
               \rangle\rangle$
      \item[Eq] $\langle\{\{x,A\},\{x,B\}\},T,\varnothing,
               \langle \vdash ( A = B \leftrightarrow
               \forall x ( x \in A \leftrightarrow x \in B ) )
               \rangle\rangle$
      \item[El] $\langle\{\{x,A\},\{x,B\}\},T,\varnothing,
               \langle \vdash ( A \in B \leftrightarrow \exists x
               ( x = A \wedge x \in B ) )
               \rangle\rangle$
\end{list}
Here we say that an individual variable is a class; $\{x|\varphi\}$ is a
class; and we extend the definition of a wff to include class equality and
membership.  Axiom Ab defines membership of a variable in a class abstraction;
the right-hand side can be read as ``the wff that results from proper
substitution of $y$ for $x$ in $\varphi$.''\footnote{Note that this definition
makes unnecessary the introduction of a separate notation similar to
$\varphi(x|y)$ for proper substitution, although we may choose to do so to be
conventional.  Incidentally, $\varphi(x|y)$ as it stands would be ambiguous in
the formal systems of our examples, since we wouldn't know whether
$\lnot\varphi(x|y)$ meant $\lnot(\varphi(x|y))$ or $(\lnot\varphi)(x|y)$.
Instead, we would have to use an unambiguous variant such as $(\varphi\,
x|y)$.}  Axioms Eq and El extend the meaning of the existing equality and
membership connectives.  This is potentially dangerous and requires careful
justification.  For example, from Eq we can derive the Axiom of Extensionality
with predicate logic alone; thus in principle we should include the Axiom of
Extensionality as a logical hypothesis.  However we do not bother to do this
since we have already presupposed that axiom earlier. The distinct variable
restrictions should be read ``where $x$ does not occur in $A$ or $B$.''  We
typically do this when the right-hand side of a definition involves an
individual variable not in the expression being defined; it is done so that
the right-hand side remains independent of the particular ``dummy'' variable
we use.

We continue to add to $\Gamma$ the following definitions
(i.e. the reducts of the following pre-statements) for empty
set,\index{empty set} class union,\index{union} and unordered
pair.\index{unordered pair}  They should be self-explanatory.  Analogous to our
use of ``$\leftrightarrow$'' to define new wffs in Example~4, we use ``$=$''
to define new abstraction terms, and both may be read informally as ``is
defined as'' in this context.
\begin{list}{}{\itemsep 0.0pt}
      \item[] $\langle\varnothing,T,\varnothing,
               \langle \mbox{class\ }\varnothing\rangle\rangle$
      \item[] $\langle\varnothing,T,\varnothing,
               \langle \vdash \varnothing = \{ x | \lnot x = x \}
               \rangle\rangle$
      \item[] $\langle\varnothing,T,\varnothing,
               \langle \mbox{class\ }(A\cup B)\rangle\rangle$
      \item[] $\langle\{\{x,A\},\{x,B\}\},T,\varnothing,
               \langle \vdash ( A \cup B ) = \{ x | ( x \in A \vee x \in B ) \}
               \rangle\rangle$
      \item[] $\langle\varnothing,T,\varnothing,
               \langle \mbox{class\ }\{A,B\}\rangle\rangle$
      \item[] $\langle\{\{x,A\},\{x,B\}\},T,\varnothing,
               \langle \vdash \{ A , B \} = \{ x | ( x = A \vee x = B ) \}
               \rangle\rangle$
\end{list}

\section{Metamath as a Formal System}\label{theorymm}

This section presupposes a familiarity with the Metamath computer language.

Our theory describes formal systems and their universes.  The Metamath
language provides a way of representing these set-theoretical objects to
a computer.  A Metamath database, being a finite set of {\sc ascii}
characters, can usually describe only a subset of a formal system and
its universe, which are typically infinite.  However the database can
contain as large a finite subset of the formal system and its universe
as we wish.  (Of course a Metamath set theory database can, in
principle, indirectly describe an entire infinite formal system by
formalizing the expository language in this Appendix.)

For purpose of our discussion, we assume the Metamath database
is in the simple form described on p.~\pageref{framelist},
consisting of all constant and variable declarations at the beginning,
followed by a sequence of extended frames each
delimited by \texttt{\$\char`\{} and \texttt{\$\char`\}}.  Any Metamath database can
be converted to this form, as described on p.~\pageref{frameconvert}.

The math symbol tokens of a Metamath source file, which are declared
with \texttt{\$c} and \texttt{\$v} statements, are names we assign to
representatives of $\mbox{\em CN}$ and $\mbox{\em VR}$.  For
definiteness we could assume that the first math symbol declared as a
variable corresponds to $v_0$, the second to $v_1$, etc., although the
exact correspondence we choose is not important.

In the Metamath language, each \texttt{\$d}, \texttt{\$f}, and
 \texttt{\$e} source
statement in an extended frame (Section~\ref{frames})
corresponds respectively to a member of the
collections $D$, $T$, and $H$ in a formal system statement $\langle
D_M,T_M,H,A\rangle$.  The math symbol strings following these Metamath keywords
correspond to a variable pair (in the case of \texttt{\$d}) or an expression (for
the other two keywords). The math symbol string following a \texttt{\$a} source
statement corresponds to expression $A$ in an axiomatic statement of the
formal system; the one following a \texttt{\$p} source statement corresponds to
$A$ in a provable statement that is not axiomatic.  In other words, each
extended frame in a Metamath database corresponds to
a pre-statement of the formal system, and a frame corresponds to
a statement of the formal system.  (Don't confuse the two meanings of
``statement'' here.  A statement of the formal system corresponds to the
several statements in a Metamath database that may constitute a
frame.)

In order for the computer to verify that a formal system statement is
provable, each \texttt{\$p} source statement is accompanied by a proof.
However, the proof does not correspond to anything in the formal system
but is simply a way of communicating to the computer the information
needed for its verification.  The proof tells the computer {\em how to
construct} specific members of closure of the formal system
pre-statement corresponding to the extended frame of the \texttt{\$p}
statement.  The final result of the construction is the member of the
closure that matches the \texttt{\$p} statement.  The abstract formal
system, on the other hand, is concerned only with the {\em existence} of
members of the closure.

As mentioned on p.~\pageref{exampleref}, Examples 1 and 3--6 in the
previous Section parallel the development of logic and set theory in the
Metamath database
\texttt{set.mm}.\index{set theory database (\texttt{set.mm})} You may
find it instructive to compare them.


\chapter{The MIU System}
\label{MIU}
\index{formal system}
\index{MIU-system}

The following is a listing of the file \texttt{miu.mm}.  It is self-explanatory.

%%%%%%%%%%%%%%%%%%%%%%%%%%%%%%%%%%%%%%%%%%%%%%%%%%%%%%%%%%%%

\begin{verbatim}
$( The MIU-system:  A simple formal system $)

$( Note:  This formal system is unusual in that it allows
empty wffs.  To work with a proof, you must type
SET EMPTY_SUBSTITUTION ON before using the PROVE command.
By default, this is OFF in order to reduce the number of
ambiguous unification possibilities that have to be selected
during the construction of a proof.  $)

$(
Hofstadter's MIU-system is a simple example of a formal
system that illustrates some concepts of Metamath.  See
Douglas R. Hofstadter, _Goedel, Escher, Bach:  An Eternal
Golden Braid_ (Vintage Books, New York, 1979), pp. 33ff. for
a description of the MIU-system.

The system has 3 constant symbols, M, I, and U.  The sole
axiom of the system is MI. There are 4 rules:
     Rule I:  If you possess a string whose last letter is I,
     you can add on a U at the end.
     Rule II:  Suppose you have Mx.  Then you may add Mxx to
     your collection.
     Rule III:  If III occurs in one of the strings in your
     collection, you may make a new string with U in place
     of III.
     Rule IV:  If UU occurs inside one of your strings, you
     can drop it.
Unfortunately, Rules III and IV do not have unique results:
strings could have more than one occurrence of III or UU.
This requires that we introduce the concept of an "MIU
well-formed formula" or wff, which allows us to construct
unique symbol sequences to which Rules III and IV can be
applied.
$)

$( First, we declare the constant symbols of the language.
Note that we need two symbols to distinguish the assertion
that a sequence is a wff from the assertion that it is a
theorem; we have arbitrarily chosen "wff" and "|-". $)
      $c M I U |- wff $. $( Declare constants $)

$( Next, we declare some variables. $)
     $v x y $.

$( Throughout our theory, we shall assume that these
variables represent wffs. $)
 wx   $f wff x $.
 wy   $f wff y $.

$( Define MIU-wffs.  We allow the empty sequence to be a
wff. $)

$( The empty sequence is a wff. $)
 we   $a wff $.
$( "M" after any wff is a wff. $)
 wM   $a wff x M $.
$( "I" after any wff is a wff. $)
 wI   $a wff x I $.
$( "U" after any wff is a wff. $)
 wU   $a wff x U $.

$( Assert the axiom. $)
 ax   $a |- M I $.

$( Assert the rules. $)
 ${
   Ia   $e |- x I $.
$( Given any theorem ending with "I", it remains a theorem
if "U" is added after it.  (We distinguish the label I_
from the math symbol I to conform to the 24-Jun-2006
Metamath spec.) $)
   I_    $a |- x I U $.
 $}
 ${
IIa  $e |- M x $.
$( Given any theorem starting with "M", it remains a theorem
if the part after the "M" is added again after it. $)
   II   $a |- M x x $.
 $}
 ${
   IIIa $e |- x I I I y $.
$( Given any theorem with "III" in the middle, it remains a
theorem if the "III" is replaced with "U". $)
   III  $a |- x U y $.
 $}
 ${
   IVa  $e |- x U U y $.
$( Given any theorem with "UU" in the middle, it remains a
theorem if the "UU" is deleted. $)
   IV   $a |- x y $.
  $}

$( Now we prove the theorem MUIIU.  You may be interested in
comparing this proof with that of Hofstadter (pp. 35 - 36).
$)
 theorem1  $p |- M U I I U $=
      we wM wU wI we wI wU we wU wI wU we wM we wI wU we wM
      wI wI wI we wI wI we wI ax II II I_ III II IV $.
\end{verbatim}\index{well-formed formula (wff)}

The \texttt{show proof /lemmon/renumber} command
yields the following display.  It is very similar
to the one in \cite[pp.~35--36]{Hofstadter}.\index{Hofstadter, Douglas R.}

\begin{verbatim}
1 ax             $a |- M I
2 1 II           $a |- M I I
3 2 II           $a |- M I I I I
4 3 I_           $a |- M I I I I U
5 4 III          $a |- M U I U
6 5 II           $a |- M U I U U I U
7 6 IV           $a |- M U I I U
\end{verbatim}

We note that Hofstadter's ``MU-puzzle,'' which asks whether
MU is a theorem of the MIU-system, cannot be answered using
the system above because the MU-puzzle is a question {\em
about} the system.  To prove the answer to the MU-puzzle,
a much more elaborate system is needed, namely one that
models the MIU-system within set theory.  (Incidentally, the
answer to the MU-puzzle is no.)

\chapter{Metamath Language EBNF}%
\label{BNF}%
\index{Metamath Language EBNF}

The following is a formal description of the basic Metamath language syntax
(with compressed proofs and support for unknown proof steps).
It is defined using the
Extended Backus--Naur Form (EBNF)\index{Extended Backus--Naur Form}\index{EBNF}
notation from W3C\index{W3C}
\textit{Extensible Markup Language (XML) 1.0 (Fifth Edition)}
(W3C Recommendation 26 November 2008) at
\url{https://www.w3.org/TR/xml/#sec-notation}.

The \texttt{database}
rule is processed until the end of the file (\texttt{EOF}).
The rules eventually require reading whitespace-separated tokens.
A token has an upper-case definition (see below)
or is a string constant in a non-token (such as \texttt{'\$a'}).
We intend for this to be correct, but if there is a conflict the
rules of section \ref{spec} govern. That section also discusses
non-syntax restrictions not shown here
(e.g., that each new label token
defined in a \texttt{hypothesis-stmt} or \texttt{assert-stmt}
must be unique).

\begin{verbatim}
database ::= outermost-scope-stmt*

outermost-scope-stmt ::=
  include-stmt | constant-stmt | stmt

/* File inclusion command; process file as a database.
   Databases should NOT have a comment in the filename. */
include-stmt ::= '$[' filename '$]'

/* Constant symbols declaration. */
constant-stmt ::= '$c' constant+ '$.'

/* A normal statement can occur in any scope. */
stmt ::= block | variable-stmt | disjoint-stmt |
  hypothesis-stmt | assert-stmt

/* A block. You can have 0 statements in a block. */
block ::= '${' stmt* '$}'

/* Variable symbols declaration. */
variable-stmt ::= '$v' variable+ '$.'

/* Disjoint variables. Simple disjoint statements have
   2 variables, i.e., "variable*" is empty for them. */
disjoint-stmt ::= '$d' variable variable variable* '$.'

hypothesis-stmt ::= floating-stmt | essential-stmt

/* Floating (variable-type) hypothesis. */
floating-stmt ::= LABEL '$f' typecode variable '$.'

/* Essential (logical) hypothesis. */
essential-stmt ::= LABEL '$e' typecode MATH-SYMBOL* '$.'

assert-stmt ::= axiom-stmt | provable-stmt

/* Axiomatic assertion. */
axiom-stmt ::= LABEL '$a' typecode MATH-SYMBOL* '$.'

/* Provable assertion. */
provable-stmt ::= LABEL '$p' typecode MATH-SYMBOL*
  '$=' proof '$.'

/* A proof. Proofs may be interspersed by comments.
   If '?' is in a proof it's an "incomplete" proof. */
proof ::= uncompressed-proof | compressed-proof
uncompressed-proof ::= (LABEL | '?')+
compressed-proof ::= '(' LABEL* ')' COMPRESSED-PROOF-BLOCK+

typecode ::= constant

filename ::= MATH-SYMBOL /* No whitespace or '$' */
constant ::= MATH-SYMBOL
variable ::= MATH-SYMBOL
\end{verbatim}

\needspace{2\baselineskip}
A \texttt{frame} is a sequence of 0 or more
\texttt{disjoint-{\allowbreak}stmt} and
\texttt{hypotheses-{\allowbreak}stmt} statements
(possibly interleaved with other non-\texttt{assert-stmt} statements)
followed by one \texttt{assert-stmt}.

\needspace{3\baselineskip}
Here are the rules for lexical processing (tokenization) beyond
the constant tokens shown above.
By convention these tokenization rules have upper-case names.
Every token is read for the longest possible length.
Whitespace-separated tokens are read sequentially;
note that the separating whitespace and \texttt{\$(} ... \texttt{\$)}
comments are skipped.

If a token definition uses another token definition, the whole thing
is considered a single token.
A pattern that is only part of a full token has a name beginning
with an underscore (``\_'').
An implementation could tokenize many tokens as a
\texttt{PRINTABLE-SEQUENCE}
and then check if it meets the more specific rule shown here.

Comments do not nest, and both \texttt{\$(} and \texttt{\$)}
have to be surrounded
by at least one whitespace character (\texttt{\_WHITECHAR}).
Technically comments end without consuming the trailing
\texttt{\_WHITECHAR}, but the trailing
\texttt{\_WHITECHAR} gets ignored anyway so we ignore that detail here.
Metamath language processors
are not required to support \texttt{\$)} followed
immediately by a bare end-of-file, because the closing
comment symbol is supposed to be followed by a
\texttt{\_WHITECHAR} such as a newline.

\begin{verbatim}
PRINTABLE-SEQUENCE ::= _PRINTABLE-CHARACTER+

MATH-SYMBOL ::= (_PRINTABLE-CHARACTER - '$')+

/* ASCII non-whitespace printable characters */
_PRINTABLE-CHARACTER ::= [#x21-#x7e]

LABEL ::= ( _LETTER-OR-DIGIT | '.' | '-' | '_' )+

_LETTER-OR-DIGIT ::= [A-Za-z0-9]

COMPRESSED-PROOF-BLOCK ::= ([A-Z] | '?')+

/* Define whitespace between tokens. The -> SKIP
   means that when whitespace is seen, it is
   skipped and we simply read again. */
WHITESPACE ::= (_WHITECHAR+ | _COMMENT) -> SKIP

/* Comments. $( ... $) and do not nest. */
_COMMENT ::= '$(' (_WHITECHAR+ (PRINTABLE-SEQUENCE - '$)'))*
  _WHITECHAR+ '$)' _WHITECHAR

/* Whitespace: (' ' | '\t' | '\r' | '\n' | '\f') */
_WHITECHAR ::= [#x20#x09#x0d#x0a#x0c]
\end{verbatim}
% This EBNF was developed as a collaboration between
% David A. Wheeler\index{Wheeler, David A.},
% Mario Carneiro\index{Carneiro, Mario}, and
% Benoit Jubin\index{Jubin, Benoit}, inspired by a request
% (and a lot of initial work) by Benoit Jubin.
%
% \chapter{Disclaimer and Trademarks}
%
% Information in this document is subject to change without notice and does not
% represent a commitment on the part of Norman Megill.
% \vspace{2ex}
%
% \noindent Norman D. Megill makes no warranties, either express or implied,
% regarding the Metamath computer software package.
%
% \vspace{2ex}
%
% \noindent Any trademarks mentioned in this book are the property of
% their respective owners.  The name ``Metamath'' is a trademark of
% Norman Megill.
%
\cleardoublepage
\phantomsection  % fixes the link anchor
\addcontentsline{toc}{chapter}{\bibname}

\bibliography{metamath}
%% metamath.tex - Version of 2-Jun-2019
% If you change the date above, also change the "Printed date" below.
% SPDX-License-Identifier: CC0-1.0
%
%                              PUBLIC DOMAIN
%
% This file (specifically, the version of this file with the above date)
% has been released into the Public Domain per the
% Creative Commons CC0 1.0 Universal (CC0 1.0) Public Domain Dedication
% https://creativecommons.org/publicdomain/zero/1.0/
%
% The public domain release applies worldwide.  In case this is not
% legally possible, the right is granted to use the work for any purpose,
% without any conditions, unless such conditions are required by law.
%
% Several short, attributed quotations from copyrighted works
% appear in this file under the ``fair use'' provision of Section 107 of
% the United States Copyright Act (Title 17 of the {\em United States
% Code}).  The public-domain status of this file is not applicable to
% those quotations.
%
% Norman Megill - email: nm(at)alum(dot)mit(dot)edu
%
% David A. Wheeler also donates his improvements to this file to the
% public domain per the CC0.  He works at the Institute for Defense Analyses
% (IDA), but IDA has agreed that this Metamath work is outside its "lane"
% and is not a work by IDA.  This was specifically confirmed by
% Margaret E. Myers (Division Director of the Information Technology
% and Systems Division) on 2019-05-24 and by Ben Lindorf (General Counsel)
% on 2019-05-22.

% This file, 'metamath.tex', is self-contained with everything needed to
% generate the the PDF file 'metamath.pdf' (the _Metamath_ book) on
% standard LaTeX 2e installations.  The auxiliary files are embedded with
% "filecontents" commands.  To generate metamath.pdf file, run these
% commands under Linux or Cygwin in the directory that contains
% 'metamath.tex':
%
%   rm -f realref.sty metamath.bib
%   touch metamath.ind
%   pdflatex metamath
%   pdflatex metamath
%   bibtex metamath
%   makeindex metamath
%   pdflatex metamath
%   pdflatex metamath
%
% The warnings that occur in the initial runs of pdflatex can be ignored.
% For the final run,
%
%   egrep -i 'error|warn' metamath.log
%
% should show exactly these 5 warnings:
%
%   LaTeX Warning: File `realref.sty' already exists on the system.
%   LaTeX Warning: File `metamath.bib' already exists on the system.
%   LaTeX Font Warning: Font shape `OMS/cmtt/m/n' undefined
%   LaTeX Font Warning: Font shape `OMS/cmtt/bx/n' undefined
%   LaTeX Font Warning: Some font shapes were not available, defaults
%       substituted.
%
% Search for "Uncomment" below if you want to suppress hyperlink boxes
% in the PDF output file
%
% TYPOGRAPHICAL NOTES:
% * It is customary to use an en dash (--) to "connect" names of different
%   people (and to denote ranges), and use a hyphen (-) for a
%   single compound name. Examples of connected multiple people are
%   Zermelo--Fraenkel, Schr\"{o}der--Bernstein, Tarski--Grothendieck,
%   Hewlett--Packard, and Backus--Naur.  Examples of a single person with
%   a compound name include Levi-Civita, Mittag-Leffler, and Burali-Forti.
% * Use non-breaking spaces after page abbreviations, e.g.,
%   p.~\pageref{note2002}.
%
% --------------------------- Start of realref.sty -----------------------------
\begin{filecontents}{realref.sty}
% Save the following as realref.sty.
% You can then use it with \usepackage{realref}
%
% This has \pageref jumping to the page on which the ref appears,
% \ref jumping to the point of the anchor, and \sectionref
% jumping to the start of section.
%
% Author:  Anthony Williams
%          Software Engineer
%          Nortel Networks Optical Components Ltd
% Date:    9 Nov 2001 (posted to comp.text.tex)
%
% The following declaration was made by Anthony Williams on
% 24 Jul 2006 (private email to Norman Megill):
%
%   ``I hereby donate the code for realref.sty posted on the
%   comp.text.tex newsgroup on 9th November 2001, accessible from
%   http://groups.google.com/group/comp.text.tex/msg/5a0e1cc13ea7fbb2
%   to the public domain.''
%
\ProvidesPackage{realref}
\RequirePackage[plainpages=false,pdfpagelabels=true]{hyperref}
\def\realref@anchorname{}
\AtBeginDocument{%
% ensure every label is a possible hyperlink target
\let\realref@oldrefstepcounter\refstepcounter%
\DeclareRobustCommand{\refstepcounter}[1]{\realref@oldrefstepcounter{#1}
\edef\realref@anchorname{\string #1.\@currentlabel}%
}%
\let\realref@oldlabel\label%
\DeclareRobustCommand{\label}[1]{\realref@oldlabel{#1}\hypertarget{#1}{}%
\@bsphack\protected@write\@auxout{}{%
    \string\expandafter\gdef\protect\csname
    page@num.#1\string\endcsname{\thepage}%
    \string\expandafter\gdef\protect\csname
    ref@num.#1\string\endcsname{\@currentlabel}%
    \string\expandafter\gdef\protect\csname
    sectionref@name.#1\string\endcsname{\realref@anchorname}%
}\@esphack}%
\DeclareRobustCommand\pageref[1]{{\edef\a{\csname
            page@num.#1\endcsname}\expandafter\hyperlink{page.\a}{\a}}}%
\DeclareRobustCommand\ref[1]{{\edef\a{\csname
            ref@num.#1\endcsname}\hyperlink{#1}{\a}}}%
\DeclareRobustCommand\sectionref[1]{{\edef\a{\csname
            ref@num.#1\endcsname}\edef\b{\csname
            sectionref@name.#1\endcsname}\hyperlink{\b}{\a}}}%
}
\end{filecontents}
% ---------------------------- End of realref.sty ------------------------------

% --------------------------- Start of metamath.bib -----------------------------
\begin{filecontents}{metamath.bib}
@book{Albers, editor = "Donald J. Albers and G. L. Alexanderson",
  title = "Mathematical People",
  publisher = "Contemporary Books, Inc.",
  address = "Chicago",
  note = "[QA28.M37]",
  year = 1985 }
@book{Anderson, author = "Alan Ross Anderson and Nuel D. Belnap",
  title = "Entailment",
  publisher = "Princeton University Press",
  address = "Princeton",
  volume = 1,
  note = "[QA9.A634 1975 v.1]",
  year = 1975}
@book{Barrow, author = "John D. Barrow",
  title = "Theories of Everything:  The Quest for Ultimate Explanation",
  publisher = "Oxford University Press",
  address = "Oxford",
  note = "[Q175.B225]",
  year = 1991 }
@book{Behnke,
  editor = "H. Behnke and F. Backmann and K. Fladt and W. S{\"{u}}ss",
  title = "Fundamentals of Mathematics",
  volume = "I",
  publisher = "The MIT Press",
  address = "Cambridge, Massachusetts",
  note = "[QA37.2.B413]",
  year = 1974 }
@book{Bell, author = "J. L. Bell and M. Machover",
  title = "A Course in Mathematical Logic",
  publisher = "North-Holland",
  address = "Amsterdam",
  note = "[QA9.B3953]",
  year = 1977 }
@inproceedings{Blass, author = "Andrea Blass",
  title = "The Interaction Between Category Theory and Set Theory",
  pages = "5--29",
  booktitle = "Mathematical Applications of Category Theory (Proceedings
     of the Special Session on Mathematical Applications
     Category Theory, 89th Annual Meeting of the American Mathematical
     Society, held in Denver, Colorado January 5--9, 1983)",
  editor = "John Walter Gray",
  year = 1983,
  note = "[QA169.A47 1983]",
  publisher = "American Mathematical Society",
  address = "Providence, Rhode Island"}
@proceedings{Bledsoe, editor = "W. W. Bledsoe and D. W. Loveland",
  title = "Automated Theorem Proving:  After 25 Years (Proceedings
     of the Special Session on Automatic Theorem Proving,
     89th Annual Meeting of the American Mathematical
     Society, held in Denver, Colorado January 5--9, 1983)",
  year = 1983,
  note = "[QA76.9.A96.S64 1983]",
  publisher = "American Mathematical Society",
  address = "Providence, Rhode Island" }
@book{Boolos, author = "George S. Boolos and Richard C. Jeffrey",
  title = "Computability and Log\-ic",
  publisher = "Cambridge University Press",
  edition = "third",
  address = "Cambridge",
  note = "[QA9.59.B66 1989]",
  year = 1989 }
@book{Campbell, author = "John Campbell",
  title = "Programmer's Progress",
  publisher = "White Star Software",
  address = "Box 51623, Palo Alto, CA 94303",
  year = 1991 }
@article{DBLP:journals/corr/Carneiro14,
  author    = {Mario Carneiro},
  title     = {Conversion of {HOL} Light proofs into Metamath},
  journal   = {CoRR},
  volume    = {abs/1412.8091},
  year      = {2014},
  url       = {http://arxiv.org/abs/1412.8091},
  archivePrefix = {arXiv},
  eprint    = {1412.8091},
  timestamp = {Mon, 13 Aug 2018 16:47:05 +0200},
  biburl    = {https://dblp.org/rec/bib/journals/corr/Carneiro14},
  bibsource = {dblp computer science bibliography, https://dblp.org}
}
@article{CarneiroND,
  author    = {Mario Carneiro},
  title     = {Natural Deductions in the Metamath Proof Language},
  url       = {http://us.metamath.org/ocat/natded.pdf},
  year      = 2014
}
@inproceedings{Chou, author = "Shang-Ching Chou",
  title = "Proving Elementary Geometry Theorems Using {W}u's Algorithm",
  pages = "243--286",
  booktitle = "Automated Theorem Proving:  After 25 Years (Proceedings
     of the Special Session on Automatic Theorem Proving,
     89th Annual Meeting of the American Mathematical
     Society, held in Denver, Colorado January 5--9, 1983)",
  editor = "W. W. Bledsoe and D. W. Loveland",
  year = 1983,
  note = "[QA76.9.A96.S64 1983]",
  publisher = "American Mathematical Society",
  address = "Providence, Rhode Island" }
@book{Clemente, author = "Daniel Clemente Laboreo",
  title = "Introduction to natural deduction",
  year = 2014,
  url = "http://www.danielclemente.com/logica/dn.en.pdf" }
@incollection{Courant, author = "Richard Courant and Herbert Robbins",
  title = "Topology",
  pages = "573--590",
  booktitle = "The World of Mathematics, Volume One",
  editor = "James R. Newman",
  publisher = "Simon and Schuster",
  address = "New York",
  note = "[QA3.W67 1988]",
  year = 1956 }
@book{Curry, author = "Haskell B. Curry",
  title = "Foundations of Mathematical Logic",
  publisher = "Dover Publications, Inc.",
  address = "New York",
  note = "[QA9.C976 1977]",
  year = 1977 }
@book{Davis, author = "Philip J. Davis and Reuben Hersh",
  title = "The Mathematical Experience",
  publisher = "Birkh{\"{a}}user Boston",
  address = "Boston",
  note = "[QA8.4.D37 1982]",
  year = 1981 }
@incollection{deMillo,
  author = "Richard de Millo and Richard Lipton and Alan Perlis",
  title = "Social Processes and Proofs of Theorems and Programs",
  pages = "267--285",
  booktitle = "New Directions in the Philosophy of Mathematics",
  editor = "Thomas Tymoczko",
  publisher = "Birkh{\"{a}}user Boston, Inc.",
  address = "Boston",
  note = "[QA8.6.N48 1986]",
  year = 1986 }
@book{Edwards, author = "Robert E. Edwards",
  title = "A Formal Background to Mathematics",
  publisher = "Springer-Verlag",
  address = "New York",
  note = "[QA37.2.E38 v.1a]",
  year = 1979 }
@book{Enderton, author = "Herbert B. Enderton",
  title = "Elements of Set Theory",
  publisher = "Academic Press, Inc.",
  address = "San Diego",
  note = "[QA248.E5]",
  year = 1977 }
@book{Goodstein, author = "R. L. Goodstein",
  title = "Development of Mathematical Logic",
  publisher = "Springer-Verlag New York Inc.",
  address = "New York",
  note = "[QA9.G6554]",
  year = 1971 }
@book{Guillen, author = "Michael Guillen",
  title = "Bridges to Infinity",
  publisher = "Jeremy P. Tarcher, Inc.",
  address = "Los Angeles",
  note = "[QA93.G8]",
  year = 1983 }
@book{Hamilton, author = "Alan G. Hamilton",
  title = "Logic for Mathematicians",
  edition = "revised",
  publisher = "Cambridge University Press",
  address = "Cambridge",
  note = "[QA9.H298]",
  year = 1988 }
@unpublished{Harrison, author = "John Robert Harrison",
  title = "Metatheory and Reflection in Theorem Proving:
    A Survey and Critique",
  note = "Technical Report
    CRC-053.
    SRI Cambridge,
    Millers Yard, Cambridge, UK,
    1995.
    Available on the Web as
{\verb+http:+}\-{\verb+//www.cl.cam.ac.uk/users/jrh/papers/reflect.html+}"}
@TECHREPORT{Harrison-thesis,
        author          = "John Robert Harrison",
        title           = "Theorem Proving with the Real Numbers",
        institution   = "University of Cambridge Computer
                         Lab\-o\-ra\-to\-ry",
        address         = "New Museums Site, Pembroke Street, Cambridge,
                           CB2 3QG, UK",
        year            = 1996,
        number          = 408,
        type            = "Technical Report",
        note            = "Author's PhD thesis,
   available on the Web at
{\verb+http:+}\-{\verb+//www.cl.cam.ac.uk+}\-{\verb+/users+}\-{\verb+/jrh+}%
\-{\verb+/papers+}\-{\verb+/thesis.html+}"}
@book{Herrlich, author = "Horst Herrlich and George E. Strecker",
  title = "Category Theory:  An Introduction",
  publisher = "Allyn and Bacon Inc.",
  address = "Boston",
  note = "[QA169.H567]",
  year = 1973 }
@article{Hindley, author = "J. Roger Hindley and David Meredith",
  title = "Principal Type-Schemes and Condensed Detachment",
  journal = "The Journal of Symbolic Logic",
  volume = 55,
  year = 1990,
  note = "[QA.J87]",
  pages = "90--105" }
@book{Hofstadter, author = "Douglas R. Hofstadter",
  title = "G{\"{o}}del, Escher, Bach",
  publisher = "Basic Books, Inc.",
  address = "New York",
  note = "[QA9.H63 1980]",
  year = 1979 }
@article{Indrzejczak, author= "Andrzej Indrzejczak",
  title = "Natural Deduction, Hybrid Systems and Modal Logic",
  journal = "Trends in Logic",
  volume = 30,
  publisher = "Springer",
  year = 2010 }
@article{Kalish, author = "D. Kalish and R. Montague",
  title = "On {T}arski's Formalization of Predicate Logic with Identity",
  journal = "Archiv f{\"{u}}r Mathematische Logik und Grundlagenfor\-schung",
  volume = 7,
  year = 1965,
  note = "[QA.A673]",
  pages = "81--101" }
@article{Kalman, author = "J. A. Kalman",
  title = "Condensed Detachment as a Rule of Inference",
  journal = "Studia Logica",
  volume = 42,
  number = 4,
  year = 1983,
  note = "[B18.P6.S933]",
  pages = "443-451" }
@book{Kline, author = "Morris Kline",
  title = "Mathematical Thought from Ancient to Modern Times",
  publisher = "Oxford University Press",
  address = "New York",
  note = "[QA21.K516 1990 v.3]",
  year = 1972 }
@book{Klinel, author = "Morris Kline",
  title = "Mathematics, The Loss of Certainty",
  publisher = "Oxford University Press",
  address = "New York",
  note = "[QA21.K525]",
  year = 1980 }
@book{Kramer, author = "Edna E. Kramer",
  title = "The Nature and Growth of Modern Mathematics",
  publisher = "Princeton University Press",
  address = "Princeton, New Jersey",
  note = "[QA93.K89 1981]",
  year = 1981 }
@article{Knill, author = "Oliver Knill",
  title = "Some Fundamental Theorems in Mathematics",
  year = "2018",
  url = "https://arxiv.org/abs/1807.08416" }
@book{Landau, author = "Edmund Landau",
  title = "Foundations of Analysis",
  publisher = "Chelsea Publishing Company",
  address = "New York",
  edition = "second",
  note = "[QA241.L2541 1960]",
  year = 1960 }
@article{Leblanc, author = "Hugues Leblanc",
  title = "On {M}eyer and {L}ambert's Quantificational Calculus {FQ}",
  journal = "The Journal of Symbolic Logic",
  volume = 33,
  year = 1968,
  note = "[QA.J87]",
  pages = "275--280" }
@article{Lejewski, author = "Czeslaw Lejewski",
  title = "On Implicational Definitions",
  journal = "Studia Logica",
  volume = 8,
  year = 1958,
  note = "[B18.P6.S933]",
  pages = "189--208" }
@book{Levy, author = "Azriel Levy",
  title = "Basic Set Theory",
  publisher = "Dover Publications",
  address = "Mineola, NY",
  year = "2002"
}
@book{Margaris, author = "Angelo Margaris",
  title = "First Order Mathematical Logic",
  publisher = "Blaisdell Publishing Company",
  address = "Waltham, Massachusetts",
  note = "[QA9.M327]",
  year = 1967}
@book{Manin, author = "Yu I. Manin",
  title = "A Course in Mathematical Logic",
  publisher = "Springer-Verlag",
  address = "New York",
  note = "[QA9.M29613]",
  year = "1977" }
@article{Mathias, author = "Adrian R. D. Mathias",
  title = "A Term of Length 4,523,659,424,929",
  journal = "Synthese",
  volume = 133,
  year = 2002,
  note = "[Q.S993]",
  pages = "75--86" }
@article{Megill, author = "Norman D. Megill",
  title = "A Finitely Axiomatized Formalization of Predicate Calculus
     with Equality",
  journal = "Notre Dame Journal of Formal Logic",
  volume = 36,
  year = 1995,
  note = "[QA.N914]",
  pages = "435--453" }
@unpublished{Megillc, author = "Norman D. Megill",
  title = "A Shorter Equivalent of the Axiom of Choice",
  month = "June",
  note = "Unpublished",
  year = 1991 }
@article{MegillBunder, author = "Norman D. Megill and Martin W.
    Bunder",
  title = "Weaker {D}-Complete Logics",
  journal = "Journal of the IGPL",
  volume = 4,
  year = 1996,
  pages = "215--225",
  note = "Available on the Web at
{\verb+http:+}\-{\verb+//www.mpi-sb.mpg.de+}\-{\verb+/igpl+}%
\-{\verb+/Journal+}\-{\verb+/V4-2+}\-{\verb+/#Megill+}"}
}
@book{Mendelson, author = "Elliott Mendelson",
  title = "Introduction to Mathematical Logic",
  edition = "second",
  publisher = "D. Van Nostrand Company, Inc.",
  address = "New York",
  note = "[QA9.M537 1979]",
  year = 1979 }
@article{Meredith, author = "David Meredith",
  title = "In Memoriam {C}arew {A}rthur {M}eredith (1904-1976)",
  journal = "Notre Dame Journal of Formal Logic",
  volume = 18,
  year = 1977,
  note = "[QA.N914]",
  pages = "513--516" }
@article{CAMeredith, author = "C. A. Meredith",
  title = "Single Axioms for the Systems ({C},{N}), ({C},{O}) and ({A},{N})
      of the Two-Valued Propositional Calculus",
  journal = "The Journal of Computing Systems",
  volume = 3,
  year = 1953,
  pages = "155--164" }
@article{Monk, author = "J. Donald Monk",
  title = "Provability With Finitely Many Variables",
  journal = "The Journal of Symbolic Logic",
  volume = 27,
  year = 1971,
  note = "[QA.J87]",
  pages = "353--358" }
@article{Monks, author = "J. Donald Monk",
  title = "Substitutionless Predicate Logic With Identity",
  journal = "Archiv f{\"{u}}r Mathematische Logik und Grundlagenfor\-schung",
  volume = 7,
  year = 1965,
  pages = "103--121" }
  %% Took out this from above to prevent LaTeX underfull warning:
  % note = "[QA.A673]",
@book{Moore, author = "A. W. Moore",
  title = "The Infinite",
  publisher = "Routledge",
  address = "New York",
  note = "[BD411.M59]",
  year = 1989}
@book{Munkres, author = "James R. Munkres",
  title = "Topology: A First Course",
  publisher = "Prentice-Hall, Inc.",
  address = "Englewood Cliffs, New Jersey",
  note = "[QA611.M82]",
  year = 1975}
@article{Nemesszeghy, author = "E. Z. Nemesszeghy and E. A. Nemesszeghy",
  title = "On Strongly Creative Definitions:  A Reply to {V}. {F}. {R}ickey",
  journal = "Logique et Analyse (N.\ S.)",
  year = 1977,
  volume = 20,
  note = "[BC.L832]",
  pages = "111--115" }
@unpublished{Nemeti, author = "N{\'{e}}meti, I.",
  title = "Algebraizations of Quantifier Logics, an Overview",
  note = "Version 11.4, preprint, Mathematical Institute, Budapest,
    1994.  A shortened version without proofs appeared in
    ``Algebraizations of quantifier logics, an introductory overview,''
   {\em Studia Logica}, 50:485--569, 1991 [B18.P6.S933]"}
@article{Pavicic, author = "M. Pavi{\v{c}}i{\'{c}}",
  title = "A New Axiomatization of Unified Quantum Logic",
  journal = "International Journal of Theoretical Physics",
  year = 1992,
  volume = 31,
  note = "[QC.I626]",
  pages = "1753 --1766" }
@book{Penrose, author = "Roger Penrose",
  title = "The Emperor's New Mind",
  publisher = "Oxford University Press",
  address = "New York",
  note = "[Q335.P415]",
  year = 1989 }
@book{PetersonI, author = "Ivars Peterson",
  title = "The Mathematical Tourist",
  publisher = "W. H. Freeman and Company",
  address = "New York",
  note = "[QA93.P475]",
  year = 1988 }
@article{Peterson, author = "Jeremy George Peterson",
  title = "An automatic theorem prover for substitution and detachment systems",
  journal = "Notre Dame Journal of Formal Logic",
  volume = 19,
  year = 1978,
  note = "[QA.N914]",
  pages = "119--122" }
@book{Quine, author = "Willard Van Orman Quine",
  title = "Set Theory and Its Logic",
  edition = "revised",
  publisher = "The Belknap Press of Harvard University Press",
  address = "Cambridge, Massachusetts",
  note = "[QA248.Q7 1969]",
  year = 1969 }
@article{Robinson, author = "J. A. Robinson",
  title = "A Machine-Oriented Logic Based on the Resolution Principle",
  journal = "Journal of the Association for Computing Machinery",
  year = 1965,
  volume = 12,
  pages = "23--41" }
@article{RobinsonT, author = "T. Thacher Robinson",
  title = "Independence of Two Nice Sets of Axioms for the Propositional
    Calculus",
  journal = "The Journal of Symbolic Logic",
  volume = 33,
  year = 1968,
  note = "[QA.J87]",
  pages = "265--270" }
@book{Rucker, author = "Rudy Rucker",
  title = "Infinity and the Mind:  The Science and Philosophy of the
    Infinite",
  publisher = "Bantam Books, Inc.",
  address = "New York",
  note = "[QA9.R79 1982]",
  year = 1982 }
@book{Russell, author = "Bertrand Russell",
  title = "Mysticism and Logic, and Other Essays",
  publisher = "Barnes \& Noble Books",
  address = "Totowa, New Jersey",
  note = "[B1649.R963.M9 1981]",
  year = 1981 }
@article{Russell2, author = "Bertrand Russell",
  title = "Recent Work on the Principles of Mathematics",
  journal = "International Monthly",
  volume = 4,
  year = 1901,
  pages = "84"}
@article{Schmidt, author = "Eric Schmidt",
  title = "Reductions in Norman Megill's axiom system for complex numbers",
  url = "http://us.metamath.org/downloads/schmidt-cnaxioms.pdf",
  year = "2012" }
@book{Shoenfield, author = "Joseph R. Shoenfield",
  title = "Mathematical Logic",
  publisher = "Addison-Wesley Publishing Company, Inc.",
  address = "Reading, Massachusetts",
  year = 1967,
  note = "[QA9.S52]" }
@book{Smullyan, author = "Raymond M. Smullyan",
  title = "Theory of Formal Systems",
  publisher = "Princeton University Press",
  address = "Princeton, New Jersey",
  year = 1961,
  note = "[QA248.5.S55]" }
@book{Solow, author = "Daniel Solow",
  title = "How to Read and Do Proofs:  An Introduction to Mathematical
    Thought Process",
  publisher = "John Wiley \& Sons",
  address = "New York",
  year = 1982,
  note = "[QA9.S577]" }
@book{Stark, author = "Harold M. Stark",
  title = "An Introduction to Number Theory",
  publisher = "Markham Publishing Company",
  address = "Chicago",
  note = "[QA241.S72 1978]",
  year = 1970 }
@article{Swart, author = "E. R. Swart",
  title = "The Philosophical Implications of the Four-Color Problem",
  journal = "American Mathematical Monthly",
  year = 1980,
  volume = 87,
  month = "November",
  note = "[QA.A5125]",
  pages = "697--707" }
@book{Szpiro, author = "George G. Szpiro",
  title = "Poincar{\'{e}}'s Prize: The Hundred-Year Quest to Solve One
    of Math's Greatest Puzzles",
  publisher = "Penguin Books Ltd",
  address = "London",
  note = "[QA43.S985 2007]",
  year = 2007}
@book{Takeuti, author = "Gaisi Takeuti and Wilson M. Zaring",
  title = "Introduction to Axiomatic Set Theory",
  edition = "second",
  publisher = "Springer-Verlag New York Inc.",
  address = "New York",
  note = "[QA248.T136 1982]",
  year = 1982}
@inproceedings{Tarski, author = "Alfred Tarski",
  title = "What is Elementary Geometry",
  pages = "16--29",
  booktitle = "The Axiomatic Method, with Special Reference to Geometry and
     Physics (Proceedings of an International Symposium held at the University
     of California, Berkeley, December 26, 1957 --- January 4, 1958)",
  editor = "Leon Henkin and Patrick Suppes and Alfred Tarski",
  year = 1959,
  publisher = "North-Holland Publishing Company",
  address = "Amsterdam"}
@article{Tarski1965, author = "Alfred Tarski",
  title = "A Simplified Formalization of Predicate Logic with Identity",
  journal = "Archiv f{\"{u}}r Mathematische Logik und Grundlagenforschung",
  volume = 7,
  year = 1965,
  note = "[QA.A673]",
  pages = "61--79" }
@book{Tymoczko,
  title = "New Directions in the Philosophy of Mathematics",
  editor = "Thomas Tymoczko",
  publisher = "Birkh{\"{a}}user Boston, Inc.",
  address = "Boston",
  note = "[QA8.6.N48 1986]",
  year = 1986 }
@incollection{Wang,
  author = "Hao Wang",
  title = "Theory and Practice in Mathematics",
  pages = "129--152",
  booktitle = "New Directions in the Philosophy of Mathematics",
  editor = "Thomas Tymoczko",
  publisher = "Birkh{\"{a}}user Boston, Inc.",
  address = "Boston",
  note = "[QA8.6.N48 1986]",
  year = 1986 }
@manual{Webster,
  title = "Webster's New Collegiate Dictionary",
  organization = "G. \& C. Merriam Co.",
  address = "Springfield, Massachusetts",
  note = "[PE1628.W4M4 1977]",
  year = 1977 }
@manual{Whitehead, author = "Alfred North Whitehead",
  title = "An Introduction to Mathematics",
  year = 1911 }
@book{PM, author = "Alfred North Whitehead and Bertrand Russell",
  title = "Principia Mathematica",
  edition = "second",
  publisher = "Cambridge University Press",
  address = "Cambridge",
  year = "1927",
  note = "(3 vols.) [QA9.W592 1927]" }
@article{DBLP:journals/corr/Whalen16,
  author    = {Daniel Whalen},
  title     = {Holophrasm: a neural Automated Theorem Prover for higher-order logic},
  journal   = {CoRR},
  volume    = {abs/1608.02644},
  year      = {2016},
  url       = {http://arxiv.org/abs/1608.02644},
  archivePrefix = {arXiv},
  eprint    = {1608.02644},
  timestamp = {Mon, 13 Aug 2018 16:46:19 +0200},
  biburl    = {https://dblp.org/rec/bib/journals/corr/Whalen16},
  bibsource = {dblp computer science bibliography, https://dblp.org} }
@article{Wiedijk-revisited,
  author = {Freek Wiedijk},
  title = {The QED Manifesto Revisited},
  year = {2007},
  url = {http://mizar.org/trybulec65/8.pdf} }
@book{Wolfram,
  author = "Stephen Wolfram",
  title = "Mathematica:  A System for Doing Mathematics by Computer",
  edition = "second",
  publisher = "Addison-Wesley Publishing Co.",
  address = "Redwood City, California",
  note = "[QA76.95.W65 1991]",
  year = 1991 }
@book{Wos, author = "Larry Wos and Ross Overbeek and Ewing Lusk and Jim Boyle",
  title = "Automated Reasoning:  Introduction and Applications",
  edition = "second",
  publisher = "McGraw-Hill, Inc.",
  address = "New York",
  note = "[QA76.9.A96.A93 1992]",
  year = 1992 }

%
%
%[1] Church, Alonzo, Introduction to Mathematical Logic,
% Volume 1, Princeton University Press, Princeton, N. J., 1956.
%
%[2] Cohen, Paul J., Set Theory and the Continuum Hypothesis,
% W. A. Benjamin, Inc., Reading, Mass., 1966.
%
%[3] Hamilton, Alan G., Logic for Mathematicians, Cambridge
% University Press,
% Cambridge, 1988.

%[6] Kleene, Stephen Cole, Introduction to Metamathematics, D.  Van
% Nostrand Company, Inc., Princeton (1952).

%[13] Tarski, Alfred, "A simplified formalization of predicate
% logic with identity," Archiv fur Mathematische Logik und
% Grundlagenforschung, vol. 7 (1965), pp. 61-79.

%[14] Tarski, Alfred and Steven Givant, A Formalization of Set
% Theory Without Variables, American Mathematical Society Colloquium
% Publications, vol. 41, American Mathematical Society,
% Providence, R. I., 1987.

%[15] Zeman, J. J., Modal Logic, Oxford University Press, Oxford, 1973.
\end{filecontents}
% --------------------------- End of metamath.bib -----------------------------


%Book: Metamath
%Author:  Norman Megill Email:  nm at alum.mit.edu
%Author:  David A. Wheeler Email:  dwheeler at dwheeler.com

% A book template example
% http://www.stsci.edu/ftp/software/tex/bookstuff/book.template

\documentclass[leqno]{book} % LaTeX 2e. 10pt. Use [leqno,12pt] for 12pt
% hyperref 2002/05/27 v6.72r  (couldn't get pagebackref to work)
\usepackage[plainpages=false,pdfpagelabels=true]{hyperref}

\usepackage{needspace}     % Enable control over page breaks
\usepackage{breqn}         % automatic equation breaking
\usepackage{microtype}     % microtypography, reduces hyphenation

% Packages for flexible tables.  We need to be able to
% wrap text within a cell (with automatically-determined widths) AND
% split a table automatically across multiple pages.
% * "tabularx" wraps text in cells but only 1 page
% * "longtable" goes across pages but by itself is incompatible with tabularx
% * "ltxtable" combines longtable and tabularx, but table contents
%    must be in a separate file.
% * "ltablex" combines tabularx and longtable - must install specially
% * "booktabs" is recommended as a way to improve the look of tables,
%   but doesn't add these capabilities.
% * "tabu" much more capable and seems to be recommended. So use that.

\usepackage{makecell}      % Enable forced line splits within a table cell
% v4.13 needed for tabu: https://tex.stackexchange.com/questions/600724/dimension-too-large-after-recent-longtable-update
\usepackage{longtable}[=v4.13] % Enable multi-page tables  
\usepackage{tabu}          % Multi-page tables with wrapped text in a cell

% You can find more Tex packages using commands like:
% tlmgr search --file tabu.sty
% find /usr/share/texmf-dist/ -name '*tab*'
%
%%%%%%%%%%%%%%%%%%%%%%%%%%%%%%%%%%%%%%%%%%%%%%%%%%%%%%%%%%%%%%%%%%%%%%%%%%%%
% Uncomment the next 3 lines to suppress boxes and colors on the hyperlinks
%%%%%%%%%%%%%%%%%%%%%%%%%%%%%%%%%%%%%%%%%%%%%%%%%%%%%%%%%%%%%%%%%%%%%%%%%%%%
%\hypersetup{
%colorlinks,citecolor=black,filecolor=black,linkcolor=black,urlcolor=black
%}
%
\usepackage{realref}

% Restarting page numbers: try?
%   \printglossary
%   \cleardoublepage
%   \pagenumbering{arabic}
%   \setcounter{page}{1}    ???needed
%   \include{chap1}

% not used:
% \def\R2Lurl#1#2{\mbox{\href{#1}\texttt{#2}}}

\usepackage{amssymb}

% Version 1 of book: margins: t=.4, b=.2, ll=.4, rr=.55
% \usepackage{anysize}
% % \papersize{<height>}{<width>}
% % \marginsize{<left>}{<right>}{<top>}{<bottom>}
% \papersize{9in}{6in}
% % l/r 0.6124-0.6170 works t/b 0.2418-0.3411 = 192pp. 0.2926-03118=exact
% \marginsize{0.7147in}{0.5147in}{0.4012in}{0.2012in}

\usepackage{anysize}
% \papersize{<height>}{<width>}
% \marginsize{<left>}{<right>}{<top>}{<bottom>}
\papersize{9in}{6in}
% l/r 0.85in&0.6431-0.6539 works t/b ?-?
%\marginsize{0.85in}{0.6485in}{0.55in}{0.35in}
\marginsize{0.8in}{0.65in}{0.5in}{0.3in}

% \usepackage[papersize={3.6in,4.8in},hmargin=0.1in,vmargin={0.1in,0.1in}]{geometry}  % page geometry
\usepackage{special-settings}

\raggedbottom
\makeindex

\begin{document}
% Discourage page widows and orphans:
\clubpenalty=300
\widowpenalty=300

%%%%%%% load in AMS fonts %%%%%%% % LaTeX 2.09 - obsolete in LaTeX 2e
%\input{amssym.def}
%\input{amssym.tex}
%\input{c:/texmf/tex/plain/amsfonts/amssym.def}
%\input{c:/texmf/tex/plain/amsfonts/amssym.tex}

\bibliographystyle{plain}
\pagenumbering{roman}
\pagestyle{headings}

\thispagestyle{empty}

\hfill
\vfill

\begin{center}
{\LARGE\bf Metamath} \\
\vspace{1ex}
{\large A Computer Language for Mathematical Proofs} \\
\vspace{7ex}
{\large Norman Megill} \\
\vspace{7ex}
with extensive revisions by \\
\vspace{1ex}
{\large David A. Wheeler} \\
\vspace{7ex}
% Printed date. If changing the date below, also fix the date at the beginning.
2019-06-02
\end{center}

\vfill
\hfill

\newpage
\thispagestyle{empty}

\hfill
\vfill

\begin{center}
$\sim$\ {\sc Public Domain}\ $\sim$

\vspace{2ex}
This book (including its later revisions)
has been released into the Public Domain by Norman Megill per the
Creative Commons CC0 1.0 Universal (CC0 1.0) Public Domain Dedication.
David A. Wheeler has done the same.
This public domain release applies worldwide.  In case this is not
legally possible, the right is granted to use the work for any purpose,
without any conditions, unless such conditions are required by law.
See \url{https://creativecommons.org/publicdomain/zero/1.0/}.

\vspace{3ex}
Several short, attributed quotations from copyrighted works
appear in this book under the ``fair use'' provision of Section 107 of
the United States Copyright Act (Title 17 of the {\em United States
Code}).  The public-domain status of this book is not applicable to
those quotations.

\vspace{3ex}
Any trademarks used in this book are the property of their owners.

% QA76.9.L63.M??

% \vspace{1ex}
%
% \vspace{1ex}
% {\small Permission is granted to make and distribute verbatim copies of this
% book
% provided the copyright notice and this
% permission notice are preserved on all copies.}
%
% \vspace{1ex}
% {\small Permission is granted to copy and distribute modified versions of this
% book under the conditions for verbatim copying, provided that the
% entire
% resulting derived work is distributed under the terms of a permission
% notice
% identical to this one.}
%
% \vspace{1ex}
% {\small Permission is granted to copy and distribute translations of this
% book into another language, under the above conditions for modified
% versions,
% except that this permission notice may be stated in a translation
% approved by the
% author.}
%
% \vspace{1ex}
% %{\small   For a copy of the \LaTeX\ source files for this book, contact
% %the author.} \\
% \ \\
% \ \\

\vspace{7ex}
% ISBN: 1-4116-3724-0 \\
% ISBN: 978-1-4116-3724-5 \\
ISBN: 978-0-359-70223-7 \\
{\ } \\
Lulu Press \\
Morrisville, North Carolina\\
USA


\hfill
\vfill

Norman Megill\\ 93 Bridge St., Lexington, MA 02421 \\
E-mail address: \texttt{nm{\char`\@}alum.mit.edu} \\
\vspace{7ex}
David A. Wheeler \\
E-mail address: \texttt{dwheeler{\char`\@}dwheeler.com} \\
% See notes added at end of Preface for revision history. \\
% For current information on the Metamath software see \\
\vspace{7ex}
\url{http://metamath.org}
\end{center}

\hfill
\vfill

{\parindent0pt%
\footnotesize{%
Cover: Aleph null ($\aleph_0$) is the symbol for the
first infinite cardinal number, discovered by Georg Cantor in 1873.
We use a red aleph null (with dark outline and gold glow) as the Metamath logo.
Credit: Norman Megill (1994) and Giovanni Mascellani (2019),
public domain.%
\index{aleph null}%
\index{Metamath!logo}\index{Cantor, Georg}\index{Mascellani, Giovanni}}}

% \newpage
% \thispagestyle{empty}
%
% \hfill
% \vfill
%
% \begin{center}
% {\it To my son Robin Dwight Megill}
% \end{center}
%
% \vfill
% \hfill
%
% \newpage

\tableofcontents
%\listoftables

\chapter*{Preface}
\markboth{PREFACE}{PREFACE}
\addcontentsline{toc}{section}{Preface}


% (For current information, see the notes added at the
% end of this preface on p.~\pageref{note2002}.)

\subsubsection{Overview}

Metamath\index{Metamath} is a computer language and an associated computer
program for archiving, verifying, and studying mathematical proofs at a very
detailed level.  The Metamath language incorporates no mathematics per se but
treats all mathematical statements as mere sequences of symbols.  You provide
Metamath with certain special sequences (axioms) that tell it what rules
of inference are allowed.  Metamath is not limited to any specific field of
mathematics.  The Metamath language is simple and robust, with an
almost total absence of hard-wired syntax, and
we\footnote{Unless otherwise noted, the words
``I,'' ``me,'' and ``my'' refer to Norman Megill\index{Megill, Norman}, while
``we,'' ``us,'' and ``our'' refer to Norman Megill and
David A. Wheeler\index{Wheeler, David A.}.}
believe that it
provides about the simplest possible framework that allows essentially all of
mathematics to be expressed with absolute rigor.

% index test
%\newcommand{\nn}[1]{#1n}
%\index{aaa@bbb}
%\index{abc!def}
%\index{abd|see{qqq}}
%\index{abe|nn}
%\index{abf|emph}
%\index{abg|(}
%\index{abg|)}

Using the Metamath language, you can build formal or mathematical
systems\index{formal system}\footnote{A formal or mathematical system consists
of a collection of symbols (such as $2$, $4$, $+$ and $=$), syntax rules that
describe how symbols may be combined to form a legal expression (called a
well-formed formula or {\em wff}, pronounced ``whiff''), some starting wffs
called axioms, and inference rules that describe how theorems may be derived
(proved) from the axioms.  A theorem is a mathematical fact such as $2+2=4$.
Strictly speaking, even an obvious fact such as this must be proved from
axioms to be formally acceptable to a mathematician.}\index{theorem}
\index{axiom}\index{rule}\index{well-formed formula (wff)} that involve
inferences from axioms.  Although a database is provided
that includes a recommended set of axioms for standard mathematics, if you
wish you can supply your own symbols, syntax, axioms, rules, and definitions.

The name ``Metamath'' was chosen to suggest that the language provides a
means for {\em describing} mathematics rather than {\em being} the
mathematics itself.  Actually in some sense any mathematical language is
metamathematical.  Symbols written on paper, or stored in a computer,
are not mathematics itself but rather a way of expressing mathematics.
For example ``7'' and ``VII'' are symbols for denoting the number seven
in Arabic and Roman numerals; neither {\em is} the number seven.

If you are able to understand and write computer programs, you should be able
to follow abstract mathematics with the aid of Metamath.  Used in conjunction
with standard textbooks, Metamath can guide you step-by-step towards an
understanding of abstract mathematics from a very rigorous viewpoint, even if
you have no formal abstract mathematics background.  By using a single,
consistent notation to express proofs, once you grasp its basic concepts
Metamath provides you with the ability to immediately follow and dissect
proofs even in totally unfamiliar areas.

Of course, just being able follow a proof will not necessarily give you an
intuitive familiarity with mathematics.  Memorizing the rules of chess does not
give you the ability to appreciate the game of a master, and knowing how the
notes on a musical score map to piano keys does not give you the ability to
hear in your head how it would sound.  But each of these can be a first step.

Metamath allows you to explore proofs in the sense that you can see the
theorem referenced at any step expanded in as much detail as you want, right
down to the underlying axioms of logic and set theory (in the case of the set
theory database provided).  While Metamath will not replace the higher-level
understanding that can only be acquired through exercises and hard work, being
able to see how gaps in a proof are filled in can give you increased
confidence that can speed up the learning process and save you time when you
get stuck.

The Metamath language breaks down a mathematical proof into its tiniest
possible parts.  These can be pieced together, like interlocking
pieces in a puzzle, only in a way that produces correct and absolutely rigorous
mathematics.

The nature of Metamath\index{Metamath} enforces very precise mathematical
thinking, similar to that involved in writing a computer program.  A crucial
difference, though, is that once a proof is verified (by the Metamath program)
to be correct, it is definitely correct; it can never have a hidden
``bug.''\index{computer program bugs}  After getting used to the kind of rigor
and accuracy provided by Metamath, you might even be tempted to
adopt the attitude that a proof should never be considered correct until it
has been verified by a computer, just as you would not completely trust a
manual calculation until you have verified it on a
calculator.

My goal
for Metamath was a system for describing and verifying
mathematics that is completely universal yet conceptually as simple as
possible.  In approaching mathematics from an axiomatic, formal viewpoint, I
wanted Metamath to be able to handle almost any mathematical system, not
necessarily with ease, but at least in principle and hopefully in practice. I
wanted it to verify proofs with absolute rigor, and for this reason Metamath
is what might be thought of as a ``compile-only'' language rather than an
algorithmic or Turing-machine language (Pascal, C, Prolog, Mathematica,
etc.).  In other words, a database written in the Metamath
language doesn't ``do'' anything; it merely exhibits mathematical knowledge
and permits this knowledge to be verified as being correct.  A program in an
algorithmic language can potentially have hidden bugs\index{computer program
bugs} as well as possibly being hard to understand.  But each token in a
Metamath database must be consistent with the database's earlier
contents according to simple, fixed rules.
If a database is verified
to be correct,\footnote{This includes
verification that a sequential list of proof steps results in the specified
theorem.} then the mathematical content is correct if the
verifier is correct and the axioms are correct.
The verification program could be incorrect, but the verification algorithm
is relatively simple (making it unlikely to be implemented incorrectly
by the Metamath program),
and there are over a dozen Metamath database verifiers
written by different people in different programming languages
(so these different verifiers can act as multiple reviewers of a database).
The most-used Metamath database, the Metamath Proof Explorer
(aka \texttt{set.mm}\index{set theory database (\texttt{set.mm})}%
\index{Metamath Proof Explorer}),
is currently verified by four different Metamath verifiers written by
four different people in four different languages, including the
original Metamath program described in this book.
The only ``bugs'' that can exist are in the statement of the axioms,
for example if the axioms are inconsistent (a famous problem shown to be
unsolvable by G\"{o}del's incompleteness theorem\index{G\"{o}del's
incompleteness theorem}).
However, real mathematical systems have very few axioms, and these can
be carefully studied.
All of this provides extraordinarily high confidence that the verified database
is in fact correct.

The Metamath program
doesn't prove theorems automatically but is designed to verify proofs
that you supply to it.
The underlying Metamath language is completely general and has no built-in,
preconceived notions about your formal system\index{formal system}, its logic
or its syntax.
For constructing proofs, the Metamath program has a Proof Assistant\index{Proof
Assistant} which helps you fill in some of a proof step's details, shows you
what choices you have at any step, and verifies the proof as you build it; but
you are still expected to provide the proof.

There are many other programs that can process or generate information
in the Metamath language, and more continue to be written.
This is in part because the Metamath language itself is very simple
and intentionally easy to automatically process.
Some programs, such as \texttt{mmj2}\index{mmj2}, include a proof assistant
that can automate some steps beyond what the Metamath program can do.
Mario Carneiro has developed an algorithm for converting proofs from
the OpenTheory interchange format, which can be translated to and from
any of the HOL family of proof languages (HOL4, HOL Light, ProofPower,
and Isabelle), into the
Metamath language \cite{DBLP:journals/corr/Carneiro14}\index{Carneiro, Mario}.
Daniel Whalen has developed Holophrasm, which can automatically
prove many Metamath proofs using
machine learning\index{machine learning}\index{artificial intelligence}
approaches
(including multiple neural networks\index{neural networks})\cite{DBLP:journals/corr/Whalen16}\index{Whalen, Daniel}.
However,
a discussion of these other programs is beyond the scope of this book.

Like most computer languages, the Metamath\index{Metamath} language uses the
standard ({\sc ascii}) characters on a computer keyboard, so it cannot
directly represent many of the special symbols that mathematicians use.  A
useful feature of the Metamath program is its ability to convert its notation
into the \LaTeX\ typesetting language.\index{latex@{\LaTeX}}  This feature
lets you convert the {\sc ascii} tokens you've defined into standard
mathematical symbols, so you end up with symbols and formulas you are familiar
with instead of somewhat cryptic {\sc ascii} representations of them.
The Metamath program can also generate HTML\index{HTML}, making it easy
to view results on the web and to see related information by using
hypertext links.

Metamath is probably conceptually different from anything you've seen
before and some aspects may take some getting used to.  This book will
help you decide whether Metamath suits your specific needs.

\subsubsection{Setting Your Expectations}
It is important for you to understand what Metamath\index{Metamath} is and is
not.  As mentioned, the Metamath program
is {\em not} an automated theorem prover but
rather a proof verifier.  Developing a database can be tedious, hard work,
especially if you want to make the proofs as short as possible, but it becomes
easier as you build up a collection of useful theorems.  The purpose of
Metamath is simply to document existing mathematics in an absolutely rigorous,
computer-verifiable way, not to aid directly in the creation of new
mathematics.  It also is not a magic solution for learning abstract
mathematics, although it may be helpful to be able to actually see the implied
rigor behind what you are learning from textbooks, as well as providing hints
to work out proofs that you are stumped on.

As of this writing, a sizable set theory database has been developed to
provide a foundation for many fields of mathematics, but much more work would
be required to develop useful databases for specific fields.

Metamath\index{Metamath} ``knows no math;'' it just provides a framework in
which to express mathematics.  Its language is very small.  You can define two
kinds of symbols, constants\index{constant} and variables\index{variable}.
The only thing Metamath knows how to do is to substitute strings of symbols
for the variables\index{substitution!variable}\index{variable substitution} in
an expression based on instructions you provide it in a proof, subject to
certain constraints you specify for the variables.  Even the decimal
representation of a number is merely a string of certain constants (digits)
which together, in a specific context, correspond to whatever mathematical
object you choose to define for it; unlike other computer languages, there is
no actual number stored inside the computer.  In a proof, you in effect
instruct Metamath what symbol substitutions to make in previous axioms or
theorems and join a sequence of them together to result in the desired
theorem.  This kind of symbol manipulation captures the essence of mathematics
at a preaxiomatic level.

\subsubsection{Metamath and Mathematical Literature}

In advanced mathematical literature, proofs are usually presented in the form
of short outlines that often only an expert can follow.  This is partly out of
a desire for brevity, but it would also be unwise (even if it were practical)
to present proofs in complete formal detail, since the overall picture would
be lost.\index{formal proof}

A solution I envision\label{envision} that would allow mathematics to remain
acceptable to the expert, yet increase its accessibility to non-specialists,
consists of a combination of the traditional short, informal proof in print
accompanied by a complete formal proof stored in a computer database.  In an
analogy with a computer program, the informal proof is like a set of comments
that describe the overall reasoning and content of the proof, whereas the
computer database is like the actual program and provides a means for anyone,
even a non-expert, to follow the proof in as much detail as desired, exploring
it back through layers of theorems (like subroutines that call other
subroutines) all the way back to the axioms of the theory.  In addition, the
computer database would have the advantage of providing absolute assurance
that the proof is correct, since each step can be verified automatically.

There are several other approaches besides Metamath to a project such
as this.  Section~\ref{proofverifiers} discusses some of these.

To us, a noble goal would be a database with hundreds of thousands of
theorems and their computer-verifiable proofs, encompassing a significant
fraction of known mathematics and available for instant access.
These would be fully verified by multiple independently-implemented verifiers,
to provide extremely high confidence that the proofs are completely correct.
The database would allow people to investigate whatever details they were
interested in, so that they could confirm whatever portions they wished.
Whether or not Metamath is an appropriate choice remains to be seen, but in
principle we believe it is sufficient.

\subsubsection{Formalism}

Over the past fifty years, a group of French mathematicians working
collectively under the pseudonym of Bourbaki\index{Bourbaki, Nicolas} have
co-authored a series of monographs that attempt to rigorously and
consistently formalize large bodies of mathematics from foundations.  On the
one hand, certainly such an effort has its merits; on the other hand, the
Bourbaki project has been criticized for its ``scholasticism'' and
``hyperaxiomatics'' that hide the intuitive steps that lead to the results
\cite[p.~191]{Barrow}\index{Barrow, John D.}.

Metamath unabashedly carries this philosophy to its extreme and no doubt is
subject to the same kind of criticism.  Nonetheless I think that in
conjunction with conventional approaches to mathematics Metamath can serve a
useful purpose.  The Bourbaki approach is essentially pedagogic, requiring the
reader to become intimately familiar with each detail in a very large
hierarchy before he or she can proceed to the next step.  The difference with
Metamath is that the ``reader'' (user) knows that all details are contained in
its computer database, available as needed; it does not demand that the user
know everything but conveniently makes available those portions that are of
interest.  As the body of all mathematical knowledge grows larger and larger,
no one individual can have a thorough grasp of its entirety.  Metamath
can finalize and put to rest any questions about the validity of any part of it
and can make any part of it accessible, in principle, to a non-specialist.

\subsubsection{A Personal Note}
Why did I develop Metamath\index{Metamath}?  I enjoy abstract mathematics, but
I sometimes get lost in a barrage of definitions and start to lose confidence
that my proofs are correct.  Or I reach a point where I lose sight of how
anything I'm doing relates to the axioms that a theory is based on and am
sometimes suspicious that there may be some overlooked implicit axiom
accidentally introduced along the way (as happened historically with Euclidean
geometry\index{Euclidean geometry}, whose omission of Pasch's
axiom\index{Pasch's axiom} went unnoticed for 2000 years
\cite[p.~160]{Davis}!). I'm also somewhat lazy and wish to avoid the effort
involved in re-verifying the gaps in informal proofs ``left to the reader;'' I
prefer to figure them out just once and not have to go through the same
frustration a year from now when I've forgotten what I did.  Metamath provides
better recovery of my efforts than scraps of paper that I can't
decipher anymore.  But mostly I find very appealing the idea of rigorously
archiving mathematical knowledge in a computer database, providing precision,
certainty, and elimination of human error.

\subsubsection{Note on Bibliography and Index}

The Bibliography usually includes the Library of Congress classification
for a work to make it easier for you to find it in on a university
library shelf.  The Index has author references to pages where their works
are cited, even though the authors' names may not appear on those pages.

\subsubsection{Acknowledgments}

Acknowledgments are first due to my wife, Deborah (who passed away on
September 4, 1998), for critiquing the manu\-script but most of all for
her patience and support.  I also wish to thank Joe Wright, Richard
Becker, Clarke Evans, Buddha Buck, and Jeremy Henty for helpful
comments.  Any errors, omissions, and other shortcomings are of course
my responsibility.

\subsubsection{Note Added June 22, 2005}\label{note2002}

The original, unpublished version of this book was written in 1997 and
distributed via the web.  The present edition has been updated to
reflect the current Metamath program and databases, as well as more
current {\sc url}s for Internet sites.  Thanks to Josh
Purinton\index{Purinton, Josh}, One Hand
Clapping, Mel L.\ O'Cat, and Roy F. Longton for pointing out
typographical and other errors.  I have also benefitted from numerous
discussions with Raph Levien\index{Levien, Raph}, who has extended
Metamath's philosophy of rigor to result in his {\em
Ghilbert}\index{Ghilbert} proof language (\url{http://ghilbert.org}).

Robert (Bob) Solovay\index{Solovay, Robert} communicated a new result of
A.~R.~D.~Mathias on the system of Bourbaki, and the text has been
updated accordingly (p.~\pageref{bourbaki}).

Bob also pointed out a clarification of the literature regarding
category theory and inaccessible cardinals\index{category
theory}\index{cardinal, inaccessible} (p.~\pageref{categoryth}),
and a misleading statement was removed from the text.  Specifically,
contrary to a statement in previous editions, it is possible to express
``There is a proper class of inaccessible cardinals'' in the language of
ZFC.  This can be done as follows:  ``For every set $x$ there is an
inaccessible cardinal $\kappa$ such that $\kappa$ is not in $x$.''
Bob writes:\footnote{Private communication, Nov.~30, 2002.}
\begin{quotation}
     This axiom is how Grothendieck presents category theory.  To each
inaccessible cardinal $\kappa$ one associates a Grothendieck universe
\index{Grothendieck, Alexander} $U(\kappa)$.  $U(\kappa)$ consists of
those sets which lie in a transitive set of cardinality less than
$\kappa$.  Instead of the ``category of all groups,'' one works relative
to a universe [considering the category of groups of cardinality less
than $\kappa$].  Now the category whose objects are all categories
``relative to the universe $U(\kappa)$'' will be a category not
relative to this universe but to the next universe.

     All of the things category theorists like to do can be done in this
framework.  The only controversial point is whether the Grothen\-dieck
axiom is too strong for the needs of category theorists.  Mac Lane
\index{Mac Lane, Saunders} argues that ``one universe is enough'' and
Feferman\index{Feferman, Solomon} has argued that one can get by with
ordinary ZFC.  I don't find Feferman's arguments persuasive.  Mac Lane
may be right, but when I think about category theory I do it \`{a} la
Grothendieck.

        By the way Mizar\index{Mizar} adds the axiom ``there is a proper
class of inaccessibles'' precisely so as to do category theory.
\end{quotation}

The most current information on the Metamath program and databases can
always be found at \url{http://metamath.org}.


\subsubsection{Note Added June 24, 2006}\label{note2006}

The Metamath spec was restricted slightly to make parsers easier to
write.  See the footnote on p.~\pageref{namespace}.

%\subsubsection{Note Added July 24, 2006}\label{note2006b}
\subsubsection{Note Added March 10, 2007}\label{note2006b}

I am grateful to Anthony Williams\index{Williams, Anthony} for writing
the \LaTeX\ package called {\tt realref.sty} and contributing it to the
public domain.  This package allows the internal hyperlinks in a {\sc
pdf} file to anchor to specific page numbers instead of just section
titles, making the navigation of the {\sc pdf} file for this book much
more pleasant and ``logical.''

A typographical error found by Martin Kiselkov was corrected.
A confusing remark about unification was deleted per suggestion of
Mel O'Cat.

\subsubsection{Note Added May 27, 2009}\label{note2009}

Several typos found by Kim Sparre were corrected.  A note was added that
the Poincar\'{e} conjecture has been proved (p.~\pageref{poincare}).

\subsubsection{Note Added Nov. 17, 2014}\label{note2014}

The statement of the Schr\"{o}der--Bernstein theorem was corrected in
Section~\ref{trust}.  Thanks to Bob Solovay for pointing out the error.

\subsubsection{Note Added May 25, 2016}\label{note2016}

Thanks to Jerry James for correcting 16 typos.

\subsubsection{Note Added February 25, 2019}\label{note201902}

David A. Wheeler\index{Wheeler, David A.}
made a large number of improvements and updates,
in coordination with Norman Megill.
The predicate calculus axioms were renumbered, and the text makes
it clear that they are based on Tarski's system S2;
the one slight deviation in axiom ax-6 is explained and justified.
The real and complex number axioms were modified to be consistent with
\texttt{set.mm}\index{set theory database (\texttt{set.mm})}%
\index{Metamath Proof Explorer}.
Long-awaited specification changes ``1--8'' were made,
which clarified previously ambiguous points.
Some errors in the text involving \texttt{\$f} and
\texttt{\$d} statements were corrected (the spec was correct, but
the in-book explanations unintentionally contradicted the spec).
We now have a system for automatically generating narrow PDFs,
so that those with smartphones can have easy access to the current
version of this document.
A new section on deduction was added;
it discusses the standard deduction theorem,
the weak deduction theorem,
deduction style, and natural deduction.
Many minor corrections (too numerous to list here) were also made.

\subsubsection{Note Added March 7, 2019}\label{note201903}

This added a description of the Matamath language syntax in
Extended Backus--Naur Form (EBNF)\index{Extended Backus--Naur Form}\index{EBNF}
in Appendix \ref{BNF}, added a brief explanation about typecodes,
inserted more examples in the deduction section,
and added a variety of smaller improvements.

\subsubsection{Note Added April 7, 2019}\label{note201904}

This version clarified the proper substitution notation, improved the
discussion on the weak deduction theorem and natural deduction,
documented the \texttt{undo} command, updated the information on
\texttt{write source}, changed the typecode
from \texttt{set} to \texttt{setvar} to be consistent with the current
version of \texttt{set.mm}, added more documentation about comment markup
(e.g., documented how to create headings), and clarified the
differences between various assertion forms (in particular deduction form).

\subsubsection{Note Added June 2, 2019}\label{note201906}

This version fixes a large number of small issues reported by
Beno\^{i}t Jubin\index{Jubin, Beno\^{i}t}, such as editorial issues
and the need to document \texttt{verify markup} (thank you!).
This version also includes specific examples
of forms (deduction form, inference form, and closed form).
We call this version the ``second edition'';
the previous edition formally published in 2007 had a slightly different title
(\textit{Metamath: A Computer Language for Pure Mathematics}).

\chapter{Introduction}
\pagenumbering{arabic}

\begin{quotation}
  {\em {\em I.M.:}  No, no.  There's nothing subjective about it!  Everybody
knows what a proof is.  Just read some books, take courses from a competent
mathematician, and you'll catch on.

{\em Student:}  Are you sure?

{\em I.M.:}  Well---it is possible that you won't, if you don't have any
aptitude for it.  That can happen, too.

{\em Student:}  Then {\em you} decide what a proof is, and if I don't learn
to decide in the same way, you decide I don't have any aptitude.

{\em I.M.:}  If not me, then who?}
    \flushright\sc  ``The Ideal Mathematician''
    \index{Davis, Phillip J.}
    \footnote{\cite{Davis}, p.~40.}\\
\end{quotation}

Brilliant mathematicians have discovered almost
unimaginably profound results that rank among the crowning intellectual
achievements of mankind.  However, there is a sense in which modern abstract
mathematics is behind the times, stuck in an era before computers existed.
While no one disputes the remarkable results that have been achieved,
communicating these results in a precise way to the uninitiated is virtually
impossible.  To describe these results, a terse informal language is used which
despite its elegance is very difficult to learn.  This informal language is not
imprecise, far from it, but rather it often has omitted detail
and symbols with hidden context that are
implicitly understood by an expert but few others.  Extremely complex technical
meanings are associated with innocent-sounding English words such as
``compact'' and ``measurable'' that barely hint at what is actually being
said.  Anyone who does not keep the precise technical meaning constantly in
mind is bound to fail, and acquiring the ability to do this can be achieved
only through much practice and hard work.  Only the few who complete the
painful learning experience can join the small in-group of pure
mathematicians.  The informal language effectively cuts off the true nature of
their knowledge from most everyone else.

Metamath\index{Metamath} makes abstract mathematics more concrete.  It allows
a computer to keep track of the complexity associated with each word or symbol
with absolute rigor.  You can explore this complexity at your leisure, to
whatever degree you desire.  Whether or not you believe that concepts such as
infinity actually ``exist'' outside of the mind, Metamath lets you get to the
foundation for what's really being said.

Metamath also enables completely rigorous and thorough proof verification.
Its language is simple enough so that you
don't have to rely on the authority of experts but can verify the results
yourself, step by step.  If you want to attempt to derive your own results,
Metamath will not let you make a mistake in reasoning.
Even professional mathematicians make mistakes; Metamath makes it possible
to thoroughly verify that proofs are correct.

Metamath\index{Metamath} is a computer language and an associated computer
program for archiving, verifying, and studying mathematical proofs at a very
detailed level.
The Metamath language
describes formal\index{formal system} mathematical
systems and expresses proofs of theorems in those systems.  Such a language
is called a metalanguage\index{metalanguage} by mathematicians.
The Metamath program is a computer program that verifies
proofs expressed in the Metamath language.
The Metamath program does not have the built-in
ability to make logical inferences; it just makes a series of symbol
substitutions according to instructions given to it in a proof
and verifies that the result matches the expected theorem.  It makes logical
inferences based only on rules of logic that are contained in a set of
axioms\index{axiom}, or first principles, that you provide to it as the
starting point for proofs.

The complete specification of the Metamath language is only four pages long
(Section~\ref{spec}, p.~\pageref{spec}).  Its simplicity may at first make you
wonder how it can do much of anything at all.  But in fact the kinds of
symbol manipulations it performs are the ones that are implicitly done in all
mathematical systems at the lowest level.  You can learn it relatively quickly
and have complete confidence in any mathematical proof that it verifies.  On
the other hand, it is powerful and general enough so that virtually any
mathematical theory, from the most basic to the deeply abstract, can be
described with it.

Although in principle Metamath can be used with any
kind of mathematics, it is best suited for abstract or ``pure'' mathematics
that is mostly concerned with theorems and their proofs, as opposed to the
kind of mathematics that deals with the practical manipulation of numbers.
Examples of branches of pure mathematics are logic\index{logic},\footnote{Logic
is the study of statements that are universally true regardless of the objects
being described by the statements.  An example is the statement, ``if $P$
implies $Q$, then either $P$ is false or $Q$ is true.''} set theory\index{set
theory},\footnote{Set theory is the study of general-purpose mathematical objects called
``sets,'' and from it essentially all of mathematics can be derived.  For
example, numbers can be defined as specific sets, and their properties
can be explored using the tools of set theory.} number theory\index{number
theory},\footnote{Number theory deals with the properties of positive and
negative integers (whole numbers).} group theory\index{group
theory},\footnote{Group theory studies the properties of mathematical objects
called groups that obey a simple set of axioms and have properties of symmetry
that make them useful in many other fields.} abstract algebra\index{abstract
algebra},\footnote{Abstract algebra includes group theory and also studies
groups with additional properties that qualify them as ``rings'' and
``fields.''  The set of real numbers is a familiar example of a field.},
analysis\index{analysis} \index{real and complex numbers}\footnote{Analysis is
the study of real and complex numbers.} and
topology\index{topology}.\footnote{One area studied by topology are properties
that remain unchanged when geometrical objects undergo stretching
deformations; for example a doughnut and a coffee cup each have one hole (the
cup's hole is in its handle) and are thus considered topologically
equivalent.  In general, though, topology is the study of abstract
mathematical objects that obey a certain (surprisingly simple) set of axioms.
See, for example, Munkres \cite{Munkres}\index{Munkres, James R.}.} Even in
physics, Metamath could be applied to certain branches that make use of
abstract mathematics, such as quantum logic\index{quantum logic} (used to study
aspects of quantum mechanics\index{quantum mechanics}).

On the other hand, Metamath\index{Metamath} is less suited to applications
that deal primarily with intensive numeric computations.  Metamath does not
have any built-in representation of numbers\index{Metamath!representation of
numbers}; instead, a specific string of symbols (digits) must be syntactically
constructed as part of any proof in which an ordinary number is used.  For
this reason, numbers in Metamath are best limited to specific constants that
arise during the course of a theorem or its proof.  Numbers are only a tiny
part of the world of abstract mathematics.  The exclusion of built-in numbers
was a conscious decision to help achieve Metamath's simplicity, and there are
other software tools if you have different mathematical needs.
If you wish to quickly solve algebraic problems, the computer algebra
programs\index{computer algebra system} {\sc
macsyma}\index{macsyma@{\sc macsyma}}, Mathematica\index{Mathematica}, and
Maple\index{Maple} are specifically suited to handling numbers and
algebra efficiently.
If you wish to simply calculate numeric or matrix expressions easily,
tools such as Octave\index{Octave} may be a better choice.

After learning Metamath's basic statement types, any
tech\-ni\-cal\-ly ori\-ent\-ed person, mathematician or not, can
immediately trace
any theorem proved in the language as far back as he or she wants, all the way
to the axioms on which the theorem is based.  This ability suggests a
non-traditional way of learning about pure mathematics.  Used in conjunction
with traditional methods, Metamath could make pure mathematics accessible to
people who are not sufficiently skilled to figure out the implicit detail in
ordinary textbook proofs.  Once you learn the axioms of a theory, you can have
complete confidence that everything you need to understand a proof you are
studying is all there, at your beck and call, allowing you to focus in on any
proof step you don't understand in as much depth as you need, without worrying
about getting stuck on a step you can't figure out.\footnote{On the other
hand, writing proofs in the Metamath language is challenging, requiring
a degree of rigor far in excess of that normally taught to students.  In a
classroom setting, I doubt that writing Metamath proofs would ever replace
traditional homework exercises involving informal proofs, because the time
needed to work out the details would not allow a course to
cover much material.  For students who have trouble grasping the implied rigor
in traditional material, writing a few simple proofs in the Metamath language
might help clarify fuzzy thought processes.  Although somewhat difficult at
first, it eventually becomes fun to do, like solving a puzzle, because of the
instant feedback provided by the computer.}

Metamath\index{Metamath} is probably unlike anything you have
encountered before.  In this first chapter we will look at the philosophy and
use of computers in mathematics in order to better understand the motivation
behind Metamath.  The material in this chapter is not required in order to use
Metamath.  You may skip it if you are impatient, but I hope you will find it
educational and enjoyable.  If you want to start experimenting with the
Metamath program right away, proceed directly to Chapter~\ref{using}
(p.~\pageref{using}).  To
learn the Metamath language, skim Chapter~\ref{using} then proceed to
Chapter~\ref{languagespec} (p.~\pageref{languagespec}).

\section{Mathematics as a Computer Language}

\begin{quote}
  {\em The study of mathematics is apt to commence in
dis\-ap\-point\-ment.\ldots \\
We are told that by its aid the stars are weighted
and the billions of molecules in a drop of water are counted.  Yet, like the
ghost of Hamlet's father, this great science eludes the efforts of our mental
weapons to grasp it.}
  \flushright\sc  Alfred North Whitehead\footnote{\cite{Whitehead}, ch.\ 1.}\\
\end{quote}\index{Whitehead, Alfred North}

\subsection{Is Mathematics ``User-Friendly''?}

Suppose you have no formal training in abstract mathematics.  But popular
books you've read offer tempting glimpses of this world filled with profound
ideas that have stirred the human spirit.  You are not satisfied with the
informal, watered-down descriptions you've read but feel it is important to
grasp the underlying mathematics itself to understand its true meaning. It's
not practical to go back to school to learn it, though; you don't want to
dedicate years of your life to it.  There are many important things in life,
and you have to set priorities for what's important to you.  What would happen
if you tried to pursue it on your own, in your spare time?

After all, you were able to learn a computer programming language such as
Pascal on your own without too much difficulty, even though you had no formal
training in computers.  You don't claim to be an expert in software design,
but you can write a passable program when necessary to suit your needs.  Even
more important, you know that you can look at anyone else's Pascal program, no
matter how complex, and with enough patience figure out exactly how it works,
even though you are not a specialist.  Pascal allows you do anything that a
computer can do, at least in principle.  Thus you know you have the ability,
in principle, to follow anything that a computer program can do:  you just
have to break it down into small enough pieces.

Here's an imaginary scenario of what might happen if you na\-ive\-ly a\-dopted
this same view of abstract mathematics and tried to pick it up on your own, in
a period of time comparable to, say, learning a computer programming
language.

\subsubsection{A Non-Mathematician's Quest for Truth}

\begin{quote}
  {\em \ldots my daughters have been studying (chemistry) for several
se\-mes\-ters, think they have learned differential and integral calculus in
school, and yet even today don't know why $x\cdot y=y\cdot x$ is true.}
  \flushright\sc  Edmund Landau\footnote{\cite{Landau}, p.~vi.}\\
\end{quote}\index{Landau, Edmund}

\begin{quote}
  {\em Minus times minus is plus,\\
The reason for this we need not discuss.}
  \flushright\sc W.\ H.\ Auden\footnote{As quoted in \cite{Guillen}, p.~64.}\\
\end{quote}\index{Auden, W.\ H.}\index{Guillen, Michael}

We'll suppose you are a technically oriented professional, perhaps an engineer, a
computer programmer, or a physicist, but probably not a mathematician.  You
consider yourself reasonably intelligent.  You did well in school, learning a
variety of methods and techniques in practical mathematics such as calculus and
differential equations.  But rarely did your courses get into anything
resembling modern abstract mathematics, and proofs were something that appeared
only occasionally in your textbooks, a kind of necessary evil that was
supposed to convince you of a certain key result.  Most of your
homework consisted of exercises that gave you practice in the techniques, and
you were hardly ever asked to come up with a proof of your own.

You find yourself curious about advanced, abstract mathematics.  You are
driven by an inner conviction that it is important to understand and
appreciate some of the most profound knowledge discovered by mankind.  But it
seems very hard to learn, something that only certain gifted longhairs can
access and understand.  You are frustrated that it seems forever cut off from
you.

Eventually your curiosity drives you to do something about it.
You set for yourself a goal of ``really'' understanding mathematics:  not just
how to manipulate equations in algebra or calculus according to cookbook
rules, but rather to gain a deep understanding of where those rules come from.
In fact, you're not thinking about this kind of ordinary mathematics at all,
but about a much more abstract, ethereal realm of pure mathematics, where
famous results such as G\"{o}del's incompleteness theorem\index{G\"{o}del's
incompleteness theorem} and Cantor's different kinds of infinities
reside.

You have probably read a number of popular books, with titles like {\em
Infinity and the Mind} \cite{Rucker}\index{Rucker, Rudy}, on topics such as
these.  You found them inspiring but at the same time somewhat
unsatisfactory.  They gave you a general idea of what these results are about,
but if someone asked you to prove them, you wouldn't have the faintest idea of
where to begin.   Sure, you could give the same overall outline that you
learned from the popular books; and in a general sort of way, you do have an
understanding.  But deep down inside, you know that there is a rigor that is
missing, that probably there are many subtle steps and pitfalls along the way,
and ultimately it seems you have to place your trust in the experts in the
field.  You don't like this; you want to be able to verify these results for
yourself.

So where do you go next?  As a first step, you decide to look up some of the
original papers on the theorems you are curious about, or better, obtain some
standard textbooks in the field.  You look up a theorem you want to
understand.  Sure enough, it's there, but it's expressed with strange
terms and odd symbols that mean absolutely nothing to you.  It might as well be written in
a foreign language you've never seen before, whose symbols are totally alien.
You look at the proof, and you haven't the foggiest notion what each step
means, much less how one step follows from another.  Well, obviously you have
a lot to learn if you want to understand this stuff.

You feel that you could probably understand it by
going back to college for another three to six years and getting a math
degree.  But that does not fit in with your career and the other things in
your life and would serve no practical purpose.  You decide to seek a quicker
path.  You figure you'll just trace your way back to the beginning, step by
step, as you would do with a computer program, until you understand it.  But
you quickly find that this is not possible, since you can't even understand
enough to know what you have to trace back to.

Maybe a different approach is in order---maybe you should start at the
beginning and work your way up.  First, you read the introduction to the book
to find out what the prerequisites are.  In a similar fashion, you trace your
way back through two or three more books, finally arriving at one that seems
to start at a beginning:  it lists the axioms of arithmetic.  ``Aha!'' you
naively think, ``This must be the starting point, the source of all mathematical
knowledge.'' Or at least the starting point for mathematics dealing with
numbers; you have to start somewhere and have no idea what the starting point
for other mathematics would be.  But the word ``axioms'' looks promising.  So
you eagerly read along and work through some elementary exercises at the
beginning of the book.  You feel vaguely bothered:  these
don't seem like axioms at all, at least not in the sense that you want to
think of axioms.  Axioms imply a starting point from which everything else can
be built up, according to precise rules specified in the axiom system.  Even
though you can understand the first few proofs in an informal way,
and are able to do some of the
exercises, it's hard to pin down precisely what the
rules are.   Sure, each step seems to follow logically from the others, but
exactly what does that mean?  Is the ``logic'' just a matter of common sense,
something vague that we all understand but can never quite state precisely?

You've spent a number of years, off and on, programming computers, and you
know that in the case of computer languages there is no question of what the
rules are---they are precise and crystal clear.  If you follow them, your
program will work, and if you don't, it won't.  No matter how complex a
program, it can always be broken down into simpler and simpler pieces, until
you can ultimately identify the bits that are moved around to perform a
specific function.  Some programs might require a lot of perseverance to
accomplish this, but if you focus on a specific portion of it, you don't even
necessarily have to know how the rest of it works. Shouldn't there be an
analogy in mathematics?

You decide to apply the ultimate test:  you ask yourself how a computer could
verify or ensure that the steps in these proofs follow from one another.
Certainly mathematics must be at least as precisely defined as a computer
language, if not more so; after all, computer science itself is based on it.
If you can get a computer to verify these proofs, then you should also be
able, in principle, to understand them yourself in a very crystal clear,
precise way.

You're in for a surprise:  you can conceive of no way to convert the
proofs, which are in English, to a form that the computer can understand.
The proofs are filled with phrases such as ``assume there exists a unique
$x$\ldots'' and ``given any $y$, let $z$ be the number such that\ldots''  This
isn't the kind of logic you are used to in computer programming, where
everything, even arithmetic, reduces to Boolean ones and zeroes if you care to
break it down sufficiently.  Even though you think you understand the proofs,
there seems to be some kind of higher reasoning involved rather than precise
rules that define how you manipulate the symbols in the axioms.  Whatever it
is, it just isn't obvious how you would express it to a computer, and the more
you think about it, the more puzzled and confused you get, to the point where
you even wonder whether {\em you} really understand it.  There's a lot more to
these axioms of arithmetic than meets the eye.

Nobody ever talked about this in school in your applied math and engineering
courses.  You just learned the rules they gave you, not quite understanding
how or why they worked, sometimes vaguely suspicious or uncertain of them, and
through homework problems and osmosis learned how to present solutions that
satisfied the instructor and earned you an ``A.''  Rarely did you actually
``prove'' anything in a rigorous way, and the math majors who did do stuff
like that seemed to be in a different world.

Of course, there are computer algebra programs that can do mathematics, and
rather impressively.  They can instantly solve the integrals that you
struggled with in freshman calculus, and do much, much more.  But when you
look at these programs, what you see is a big collection of algorithms and
techniques that evolved and were added to over time, along with some basic
software that manipulates symbols.  Each algorithm that is built in is the
result of someone's theorem whose proof is omitted; you just have to trust the
person who proved it and the person who programmed it in and hope there are no
bugs.\index{computer program bugs}  Somehow this doesn't seem to be the
essence of mathematics.  Although computer algebra systems can generate
theorems with amazing speed, they can't actually prove a single one of them.

After some puzzlement, you revisit some popular books on what mathematics is
all about.  Somewhere you read that all of mathematics is actually derived
from something called ``set theory.''  This is a little confusing, because
nowhere in the book that presented the axioms of arithmetic was there any
mention of set theory, or if there was, it seemed to be just a tool that helps
you describe things better---the set of even numbers, that sort of thing.  If
set theory is the basis for all mathematics, then why are additional axioms
needed for arithmetic?

Something is wrong but you're not sure what.  One of your friends is a pure
mathematician.  He knows he is unable to communicate to you what he does for a
living and seems to have little interest in trying.  You do know that for him,
proofs are what mathematics is all about. You ask him what a proof is, and he
essentially tells you that, while of course it's based on logic, really it's
something you learn by doing it over and over until you pick it up.  He refers
you to a book, {\em How to Read and Do Proofs} \cite{Solow}.\index{Solow,
Daniel}  Although this book helps you understand traditional informal proofs,
there is still something missing you can't seem to pin down yet.

You ask your friend how you would go about having a computer verify a proof.
At first he seems puzzled by the question; why would you want to do that?
Then he says it's not something that would make any sense to do, but he's
heard that you'd have to break the proof down into thousands or even millions
of individual steps to do such a thing, because the reasoning involved is at
such a high level of abstraction.  He says that maybe it's something you could
do up to a point, but the computer would be completely impractical once you
get into any meaningful mathematics.  There, the only way you can verify a
proof is by hand, and you can only acquire the ability to do this by
specializing in the field for a couple of years in grad school.  Anyway, he
thinks it all has to do with set theory, although he has never taken a formal
course in set theory but just learned what he needed as he went along.

You are intrigued and amazed.  Apparently a mathematician can grasp as a
single concept something that would take a computer a thousand or a million
steps to verify, and have complete confidence in it.  Each one of these
thousand or million steps must be absolutely correct, or else the whole proof
is meaningless.  If you added a million numbers by hand, would you trust the
result?  How do you really know that all these steps are correct, that there
isn't some subtle pitfall in one of these million steps, like a bug in a
computer program?\index{computer program bugs}  After all, you've read that
famous mathematicians have occasionally made mistakes, and you certainly know
you've made your share on your math homework problems in school.

You recall the analogy with a computer program.  Sure, you can understand what
a large computer program such as a word processor does, as a single high-level
concept or a small set of such concepts, but your ability to understand it in
no way ensures that the program is correct and doesn't have hidden bugs.  Even
if you wrote the program yourself you can't really know this; most large
programs that you've written have had bugs that crop up at some later date, no
matter how careful you tried to be while writing them.

OK, so now it seems the reason you can't figure out how to make a
computer verify proofs is because each step really corresponds to a
million small steps.  Well, you say, a computer can do a million
calculations in a second, so maybe it's still practical to do.  Now the
puzzle becomes how to figure out what the million steps are that each
English-language step corresponds to.  Your mathematician friend hasn't
a clue, but suggests that maybe you would find the answer by studying
set theory.  Actually, your friend thinks you're a little off the wall
for even wondering such a thing.  For him, this is not what mathematics
is all about.

The subject of set theory keeps popping up, so you decide it's
time to look it up.

You decide to start off on a careful footing, so you start reading a couple of
very elementary books on set theory.  A lot of it seems pretty obvious, like
intersections, subsets, and Venn diagrams.  You thumb through one of the
books; nowhere is anything about axioms mentioned. The other book relegates to
an appendix a brief discussion that mentions a set of axioms called
``Zermelo--Fraenkel set theory''\index{Zermelo--Fraenkel set theory} and states
them in English.  You look at them and have no idea what they really mean or
what you can do with them.  The comments in this appendix say that the purpose
of mentioning them is to expose you to the idea, but imply that they are not
necessary for basic understanding and that they are really the subject matter
of advanced treatments where fine points such as a certain paradox (Russell's
paradox\index{Russell's paradox}\footnote{Russell's paradox assumes that there
exists a set $S$ that is a collection of all sets that don't contain
themselves.  Now, either $S$ contains itself or it doesn't.  If it contains
itself, it contradicts its definition.  But if it doesn't contain itself, it
also contradicts its definition.  Russell's paradox is resolved in ZF set
theory by denying that such a set $S$ exists.}) are resolved.  Wait a
minute---shouldn't the axioms be a starting point, not an ending point?  If
there are paradoxes that arise without the axioms, how do you know you won't
stumble across one accidentally when using the informal approach?

And nowhere do these books describe how ``all of mathematics can be
derived from set theory'' which by now you've heard a few times.

You find a more advanced book on set theory.  This one actually lists the
axioms of ZF set theory in plain English on page one.  {\em Now} you think
your quest has ended and you've finally found the source of all mathematical
knowledge; you just have to understand what it means.  Here, in one place, is
the basis for all of mathematics!  You stare at the axioms in awe, puzzle over
them, memorize them, hoping that if you just meditate on them long enough they
will become clear.  Of course, you haven't the slightest idea how the rest of
mathematics is ``derived'' from them; in particular, if these are the axioms
of mathematics, then why do arithmetic, group theory, and so on need their own
axioms?

You start reading this advanced book carefully, pondering the meaning of every
word, because by now you're really determined to get to the bottom of this.
The first thing the book does is explain how the axioms came about, which was
to resolve Russell's paradox.\index{Russell's paradox}  In fact that seems to
be the main purpose of their existence; that they supposedly can be used to
derive all of mathematics seems irrelevant and is not even mentioned.  Well,
you go on.  You hope the book will explain to you clearly, step by step, how
to derive things from the axioms.  After all, this is the starting point of
mathematics, like a book that explains the basics of a computer programming
language.  But something is missing.  You find you can't even understand the
first proof or do the first exercise.  Symbols such as $\exists$ and $\forall$
permeate the page without any mention of where they came from or how to
manipulate them; the author assumes you are totally familiar with them and
doesn't even tell you what they mean.  By now you know that $\exists$ means
``there exists'' and $\forall$ means ``for all,'' but shouldn't the rules for
manipulating these symbols be part of the axioms?  You still have no idea
how you could even describe the axioms to a computer.

Certainly there is something much different here from the technical
literature you're used to reading.  A computer language manual almost
always explains very clearly what all the symbols mean, precisely what
they do, and the rules used for combining them, and you work your way up
from there.

After glancing at four or five other such books, you come to the realization
that there is another whole field of study that you need just to get to the
point at which you can understand the axioms of set theory.  The field is
called ``logic.''  In fact, some of the books did recommend it as a
prerequisite, but it just didn't sink in.  You assumed logic was, well, just
logic, something that a person with common sense intuitively understood.  Why
waste your time reading boring treatises on symbolic logic, the manipulation
of 1's and 0's that computers do, when you already know that?  But this is a
different kind of logic, quite alien to you.  The subject of {\sc nand} and
{\sc nor} gates is not even touched upon or in any case has to do with only a
very small part of this field.

So your quest continues.  Skimming through the first couple of introductory
books, you get a general idea of what logic is about and what quantifiers
(``for all,'' ``there exists'') mean, but you find their examples somewhat
trivial and mildly annoying (``all dogs are animals,'' ``some animals are
dogs,'' and such).  But all you want to know is what the rules are for
manipulating the symbols so you can apply them to set theory.  Some formulas
describing the relationships among quantifiers ($\exists$ and $\forall$) are
listed in tables, along with some verbal reasoning to justify them.
Presumably, if you want to find out if a formula is correct, you go through
this same kind of mental reasoning process, possibly using images of dogs and
animals. Intuitively, the formulas seem to make sense.  But when you ask
yourself, ``What are the rules I need to get a computer to figure out whether
this formula is correct?'', you still don't know.  Certainly you don't ask the
computer to imagine dogs and animals.

You look at some more advanced logic books.  Many of them have an introductory
chapter summarizing set theory, which turns out to be a prerequisite.  You
need logic to understand set theory, but it seems you also need set theory to
understand logic!  These books jump right into proving rather advanced
theorems about logic, without offering the faintest clue about where the logic
came from that allows them to prove these theorems.

Luckily, you come across an elementary book of logic that, halfway through,
after the usual truth tables and metaphors, presents in a clear, precise way
what you've been looking for all along: the axioms!  They're divided into
propositional calculus (also called sentential logic) and predicate calculus
(also called first-order logic),\index{first-order logic} with rules so simple
and crystal clear that now you can finally program a computer to understand
them.  Indeed, they're no harder than learning how to play a game of chess.
As far as what you seem to need is concerned, the whole book could have been
written in five pages!

{\em Now} you think you've found the ultimate source of mathematical
truth.  So---the axioms of mathematics consist of these axioms of logic,
together with the axioms of ZF set theory. (By now you've also been able to
figure out how to translate the ZF axioms from English into the
actual symbols of logic which you can now manipulate according to
precise, easy-to-understand rules.)

Of course, you still don't understand how ``all of mathematics can be
derived from set theory,'' but maybe this will reveal itself in due
course.

You eagerly set out to program the axioms and rules into a computer and start
to look at the theorems you will have to prove as the logic is developed.  All
sorts of important theorems start popping up:  the deduction
theorem,\index{deduction theorem} the substitution theorem,\index{substitution
theorem} the completeness theorem of propositional calculus,\index{first-order
logic!completeness} the completeness theorem of predicate calculus.  Uh-oh,
there seems to be trouble.  They all get harder and harder, and not one of
them can be derived with the axioms and rules of logic you've just been
handed.  Instead, they all require ``metalogic'' for their proofs, a kind of
mixture of logic and set theory that allows you to prove things {\em about}
the axioms and theorems of logic rather than {\em with} them.

You plow ahead anyway.  A month later, you've spent much of your
free time getting the computer to verify proofs in propositional calculus.
You've programmed in the axioms, but you've also had to program in the
deduction theorem, the substitution theorem, and the completeness theorem of
propositional calculus, which by now you've resigned yourself to treating as
rather complex additional axioms, since they can't be proved from the axioms
you were given.  You can now get the computer to verify and even generate
complete, rigorous, formal proofs\index{formal proof}.  Never mind that they
may have 100,000 steps---at least now you can have complete, absolute
confidence in them.  Unfortunately, the only theorems you have proved are
pretty trivial and you can easily verify them in a few minutes with truth
tables, if not by inspection.

It looks like your mathematician friend was right.  Getting the computer to do
serious mathematics with this kind of rigor seems almost hopeless.  Even
worse, it seems that the further along you get, the more ``axioms'' you have
to add, as each new theorem seems to involve additional ``metamathematical''
reasoning that hasn't been formalized, and none of it can be derived from the
axioms of logic.  Not only do the proofs keep growing exponentially as you get
further along, but the program to verify them keeps getting bigger and bigger
as you program in more ``metatheorems.''\index{metatheorem}\footnote{A
metatheorem is usually a statement that is too general to be directly provable
in a theory.  For example, ``if $n_1$, $n_2$, and $n_3$ are integers, then
$n_1+n_2+n_3$ is an integer'' is a theorem of number theory.  But ``for any
integer $k > 1$, if $n_1, \ldots, n_k$ are integers, then $n_1+\ldots +n_k$ is
an integer'' is a metatheorem, in other words a family of theorems, one for
every $k$.  The reason it is not a theorem is that the general sum $n_1+\ldots
+n_k$ (as a function of $k$) is not an operation that can be defined directly
in number theory.} The bugs\index{computer program bugs} that have cropped up
so far have already made you start to lose faith in the rigor you seem to have
achieved, and you know it's just going to get worse as your program gets larger.

\subsection{Mathematics and the Non-Specialist}

\begin{quote}
  {\em A real proof is not checkable by a machine, or even by any mathematician
not privy to the gestalt, the mode of thought of the particular field of
mathematics in which the proof is located.}
  \flushright\sc  Davis and Hersh\index{Davis, Phillip J.}
  \footnote{\cite{Davis}, p.~354.}\\
\end{quote}

The bulk of abstract or theoretical mathematics is ordinarily outside
the reach of anyone but a few specialists in each field who have completed
the necessary difficult internship in order to enter its coterie.  The
typical intelligent layperson has no reasonable hope of understanding much of
it, nor even the specialist mathematician of understanding other fields.  It
is like a foreign language that has no dictionary to look up the translation;
the only way you can learn it is by living in the country for a few years.  It
is argued that the effort involved in learning a specialty is a necessary
process for acquiring a deep understanding.  Of course, this is almost certainly
true if one is to make significant contributions to a field; in particular,
``doing'' proofs is probably the most important part of a mathematician's
training.  But is it also necessary to deny outsiders access to it?  Is it
necessary that abstract mathematics be so hard for a layperson to grasp?

A computer normally is of no help whatsoever.  Most published proofs are
actually just series of hints written in an informal style that requires
considerable knowledge of the field to understand.  These are the ``real
proofs'' referred to by Davis and Hersh.\index{informal proof}  There is an
implicit understanding that, in principle, such a proof could be converted to
a complete formal proof\index{formal proof}.  However, it is said that no one
would ever attempt such a conversion, even if they could, because that would
presumably require millions of steps (Section~\ref{dream}).  Unfortunately the
informal style automatically excludes the understanding of the proof
by anyone who hasn't gone through the necessary apprenticeship. The
best that the intelligent layperson can do is to read popular books about deep
and famous results; while this can be helpful, it can also be misleading, and
the lack of detail usually leaves the reader with no ability whatsoever to
explore any aspect of the field being described.

The statements of theorems often use sophisticated notation that makes them
inaccessible to the non-specialist.  For a non-specialist who wants to achieve
a deeper understanding of a proof, the process of tracing definitions and
lemmas back through their hierarchy\index{hierarchy} quickly becomes confusing
and discouraging.  Textbooks are usually written to train mathematicians or to
communicate to people who are already mathematicians, and large gaps in proofs
are often left as exercises to the reader who is left at an impasse if he or
she becomes stuck.

I believe that eventually computers will enable non-specialists and even
intelligent laypersons to follow almost any mathematical proof in any field.
Metamath is an attempt in that direction.  If all of mathematics were as
easily accessible as a computer programming language, I could envision
computer programmers and hobbyists who otherwise lack mathematical
sophistication exploring and being amazed by the world of theorems and proofs
in obscure specialties, perhaps even coming up with results of their own.  A
tremendous advantage would be that anyone could experiment with conjectures in
any field---the computer would offer instant feedback as to whether
an inference step was correct.

Mathematicians sometimes have to put up with the annoyance of
cranks\index{cranks} who lack a fundamental understanding of mathematics but
insist that their ``proofs'' of, say, Fermat's Last Theorem\index{Fermat's
Last Theorem} be taken seriously.  I think part of the problem is that these
people are misled by informal mathematical language, treating it as if they
were reading ordinary expository English and failing to appreciate the
implicit underlying rigor.  Such cranks are rare in the field of computers,
because computer languages are much more explicit, and ultimately the proof is
in whether a computer program works or not.  With easily accessible
computer-based abstract mathematics, a mathematician could say to a crank,
``don't bother me until you've demonstrated your claim on the computer!''

% 22-May-04 nm
% Attempt to move De Millo quote so it doesn't separate from attribution
% CHANGE THIS NUMBER (AND ELIMINATE IF POSSIBLE) WHEN ABOVE TEXT CHANGES
\vspace{-0.5em}

\subsection{An Impossible Dream?}\label{dream}

\begin{quote}
  {\em Even quite basic theorems would demand almost unbelievably vast
  books to display their proofs.}
    \flushright\sc  Robert E. Edwards\footnote{\cite{Edwards}, p.~68.}\\
\end{quote}\index{Edwards, Robert E.}

\begin{quote}
  {\em Oh, of course no one ever really {\em does} it.  It would take
  forever!  You just show that you could do it, that's sufficient.}
    \flushright\sc  ``The Ideal Mathematician''
    \index{Davis, Phillip J.}\footnote{\cite{Davis},
p.~40.}\\
\end{quote}

\begin{quote}
  {\em There is a theorem in the primitive notation of set theory that
  corresponds to the arithmetic theorem `$1000+2000=3000$'.  The formula
  would be forbiddingly long\ldots even if [one] knows the definitions
  and is asked to simplify the long formula according to them, chances are
  he will make errors and arrive at some incorrect result.}
    \flushright\sc  Hao Wang\footnote{\cite{Wang}, p.~140.}\\
\end{quote}\index{Wang, Hao}

% 22-May-04 nm
% Attempt to move De Millo quote so it doesn't separate from attribution
% CHANGE THIS NUMBER (AND ELIMINATE IF POSSIBLE) WHEN ABOVE TEXT CHANGES
\vspace{-0.5em}

\begin{quote}
  {\em The {\em Principia Mathematica} was the crowning achievement of the
  formalists.  It was also the deathblow of the formalist view.\ldots
  {[Rus\-sell]} failed, in three enormous volumes, to get beyond the elementary
  facts of arithmetic.  He showed what can be done in principle and what
  cannot be done in practice.  If the mathematical process were really
  one of strict, logical progression, we would still be counting our
  fingers.\ldots
  One theoretician estimates\ldots that a demonstration of one of
  Ramanujan's conjectures assuming set theory and elementary analysis would
  take about two thousand pages; the length of a deduction from first principles
  is nearly in\-con\-ceiv\-a\-ble\ldots The probabilists argue that\ldots any
  very long proof can at best be viewed as only probably correct\ldots}
  \flushright\sc Richard de Millo et. al.\footnote{\cite{deMillo}, pp.~269,
  271.}\\
\end{quote}\index{de Millo, Richard}

A number of writers have conveyed the impression that the kind of absolute
rigor provided by Metamath\index{Metamath} is an impossible dream, suggesting
that a complete, formal verification\index{formal proof} of a typical theorem
would take millions of steps in untold volumes of books.  Even if it could be
done, the thinking sometimes goes, all meaning would be lost in such a
monstrous, tedious verification.\index{informal proof}\index{proof length}

These writers assume, however, that in order to achieve the kind of complete
formal verification they desire one must break down a proof into individual
primitive steps that make direct reference to the axioms.  This is
not necessary.  There is no reason not to make use of previously proved
theorems rather than proving them over and over.

Just as important, definitions\index{definition} can be introduced along
the way, allowing very complex formulas to be represented with few
symbols.  Not doing this can lead to absurdly long formulas.  For
example, the mere statement of
G\"{o}del's incompleteness theorem\index{G\"{o}del's
incompleteness theorem}, which can be expressed with a small number of
defined symbols, would require about 20,000 primitive symbols to express
it.\index{Boolos, George S.}\footnote{George S.\ Boolos, lecture at
Massachusetts Institute of Technology, spring 1990.} An extreme example
is Bourbaki's\label{bourbaki} language for set theory, which requires
4,523,659,424,929 symbols plus 1,179,618,517,981 disambiguatory links
(lines connecting symbol pairs, usually drawn below or above the
formula) to express the number
``one'' \cite{Mathias}.\index{Mathias, Adrian R. D.}\index{Bourbaki,
Nicolas}
% http://www.dpmms.cam.ac.uk/~ardm/

A hierarchy\index{hierarchy} of theorems and definitions permits an
exponential growth in the formula sizes and primitive proof steps to be
described with only a linear growth in the number of symbols used.  Of course,
this is how ordinary informal mathematics is normally done anyway, but with
Metamath\index{Metamath} it can be done with absolute rigor and precision.

\subsection{Beauty}


\begin{quote}
  {\em No one shall be able to drive us from the paradise that Cantor has
created for us.}
   \flushright\sc  David Hilbert\footnote{As quoted in \cite{Moore}, p.~131.}\\
\end{quote}\index{Hilbert, David}

\needspace{3\baselineskip}
\begin{quote}
  {\em Mathematics possesses not only truth, but some supreme beauty ---a
  beauty cold and austere, like that of a sculpture.}
    \flushright\sc  Bertrand
    Russell\footnote{\cite{Russell}.}\\
\end{quote}\index{Russell, Bertrand}

\begin{quote}
  {\em Euclid alone has looked on Beauty bare.}
  \flushright\sc Edna Millay\footnote{As quoted in \cite{Davis}, p.~150.}\\
\end{quote}\index{Millay, Edna}

For most people, abstract mathematics is distant, strange, and
incomprehensible.  Many popular books have tried to convey some of the sense
of beauty in famous theorems.  But even an intelligent layperson is left with
only a general idea of what a theorem is about and is hardly given the tools
needed to make use of it.  Traditionally, it is only after years of arduous
study that one can grasp the concepts needed for deep understanding.
Metamath\index{Metamath} allows you to approach the proof of the theorem from
a quite different perspective, peeling apart the formulas and definitions
layer by layer until an entirely different kind of understanding is achieved.
Every step of the proof is there, pieced together with absolute precision and
instantly available for inspection through a microscope with a magnification
as powerful as you desire.

A proof in itself can be considered an object of beauty.  Constructing an
elegant proof is an art.  Once a famous theorem has been proved, often
considerable effort is made to find simpler and more easily understood
proofs.  Creating and communicating elegant proofs is a major concern of
mathematicians.  Metamath is one way of providing a common language for
archiving and preserving this information.

The length of a proof can, to a certain extent, be considered an
objective measure of its ``beauty,'' since shorter proofs are usually
considered more elegant.  In the set theory database
\texttt{set.mm}\index{set theory database (\texttt{set.mm})}%
\index{Metamath Proof Explorer}
provided with Metamath, one goal was to make all proofs as short as possible.

\needspace{4\baselineskip}
\subsection{Simplicity}

\begin{quote}
  {\em God made man simple; man's complex problems are of his own
  devising.}
    \flushright\sc Eccles. 7:29\footnote{Jerusalem Bible.}\\
\end{quote}\index{Bible}

\needspace{3\baselineskip}
\begin{quote}
  {\em God made integers, all else is the work of man.}
    \flushright\sc Leopold Kronecker\footnote{{\em Jahresbericht
	der Deutschen Mathematiker-Vereinigung }, vol. 2, p. 19.}\\
\end{quote}\index{Kronecker, Leopold}

\needspace{3\baselineskip}
\begin{quote}
  {\em For what is clear and easily comprehended attracts; the
  complicated repels.}
    \flushright\sc David Hilbert\footnote{As quoted in \cite{deMillo},
p.~273.}\\
\end{quote}\index{Hilbert, David}

The Metamath\index{Metamath} language is simple and Spartan.  Metamath treats
all mathematical expressions as simple sequences of symbols, devoid of meaning.
The higher-level or ``metamathematical'' notions underlying Metamath are about
as simple as they could possibly be.  Each individual step in a proof involves
a single basic concept, the substitution of an expression for a variable, so
that in principle almost anyone, whether mathematician or not, can
completely understand how it was arrived at.

In one of its most basic applications, Metamath\index{Metamath} can be used to
develop the foundations of mathematics\index{foundations of mathematics} from
the very beginning.  This is done in the set theory database that is provided
with the Metamath package and is the subject matter
of Chapter~\ref{fol}. Any language (a metalanguage\index{metalanguage})
used to describe mathematics (an object language\index{object language}) must
have a mathematical content of its own, but it is desirable to keep this
content down to a bare minimum, namely that needed to make use of the
inference rules specified by the axioms.  With any metalanguage there is a
``chicken and egg'' problem somewhat like circular reasoning:  you must assume
the validity of the mathematics of the metalanguage in order to prove the
validity of the mathematics of the object language.  The mathematical content
of Metamath itself is quite limited.  Like the rules of a game of chess, the
essential concepts are simple enough so that virtually anyone should be able to
understand them (although that in itself will not let you play like
a master).  The symbols that Metamath manipulates do not in themselves
have any intrinsic meaning.  Your interpretation of the axioms that you supply
to Metamath is what gives them meaning.  Metamath is an attempt to strip down
mathematical thought to its bare essence and show you exactly how the symbols
are manipulated.

Philosophers and logicians, with various motivations, have often thought it
important to study ``weak'' fragments of logic\index{weak logic}
\cite{Anderson}\index{Anderson, Alan Ross} \cite{MegillBunder}\index{Megill,
Norman}\index{Bunder, Martin}, other unconventional systems of logic (such as
``modal'' logic\index{modal logic} \cite[ch.\ 27]{Boolos}\index{Boolos, George
S.}), and quantum logic\index{quantum logic} in physics
\cite{Pavicic}\index{Pavi{\v{c}}i{\'{c}}, M.}.  Metamath\index{Metamath}
provides a framework in which such systems can be expressed, with an absolute
precision that makes all underlying metamathematical assumptions rigorous and
crystal clear.

Some schools of philosophical thought, for example
intuitionism\index{intuitionism} and constructivism\index{constructivism},
demand that the notions underlying any mathematical system be as simple and
concrete as possible.  Metamath should meet the requirements of these
philosophies.  Metamath must be taught the symbols, axioms\index{axiom}, and
rules\index{rule} for a specific theory, from the skeptical (such as
intuitionism\index{intuitionism}\footnote{Intuitionism does not accept the law
of excluded middle (``either something is true or it is not true'').  See
\cite[p.~xi]{Tymoczko}\index{Tymoczko, Thomas} for discussion and references
on this topic.  Consider the theorem, ``There exist irrational numbers $a$ and
$b$ such that $a^b$ is rational.''  An intuitionist would reject the following
proof:  If $\sqrt{2}^{\sqrt{2}}$ is rational, we are done.  Otherwise, let
$a=\sqrt{2}^{\sqrt{2}}$ and $b=\sqrt{2}$. Then $a^b=2$, which is rational.})
to the bold (such as the axiom of choice in set theory\footnote{The axiom of
choice\index{Axiom of Choice} asserts that given any collection of pairwise
disjoint nonempty sets, there exists a set that has exactly one element in
common with each set of the collection.  It is used to prove many important
theorems in standard mathematics.  Some philosophers object to it because it
asserts the existence of a set without specifying what the set contains
\cite[p.~154]{Enderton}\index{Enderton, Herbert B.}.  In one foundation for
mathematics due to Quine\index{Quine, Willard Van Orman}, that has not been
otherwise shown to be inconsistent, the axiom of choice turns out to be false
\cite[p.~23]{Curry}\index{Curry, Haskell B.}.  The \texttt{show
trace{\char`\_}back} command of the Metamath program allows you to find out
whether the axiom of choice, or any other axiom, was assumed by a
proof.}\index{\texttt{show trace{\char`\_}back} command}).

The simplicity of the Metamath language lets the algorithm (computer program)
that verifies the validity of a Metamath proof be straightforward and
robust.  You can have confidence that the theorems it verifies really can be
derived from your axioms.

\subsection{Rigor}

\begin{quote}
  {\em Rigor became a goal with the Greeks\ldots But the efforts to
  pursue rigor to the utmost have led to an impasse in which there is
  no longer any agreement on what it really means.  Mathematics remains
  alive and vital, but only on a pragmatic basis.}
    \flushright\sc  Morris Kline\footnote{\cite{Kline}, p.~1209.}\\
\end{quote}\index{Kline, Morris}

Kline refers to a much deeper kind of rigor than that which we will discuss in
this section.  G\"{o}del's incompleteness theorem\index{G\"{o}del's
incompleteness theorem} showed that it is impossible to achieve absolute rigor
in standard mathematics because we can never prove that mathematics is
consistent (free from contradictions).\index{consistent theory}  If
mathematics is consistent, we will never know it, but must rely on faith.  If
mathematics is inconsistent, the best we can hope for is that some clever
future mathematician will discover the inconsistency.  In this case, the
axioms would probably be revised slightly to eliminate the inconsistency, as
was done in the case of Russell's paradox,\index{Russell's paradox} but the
bulk of mathematics would probably not be affected by such a discovery.
Russell's paradox, for example, did not affect most of the remarkable results
achieved by 19th-century and earlier mathematicians.  It mainly invalidated
some of Gottlob Frege's\index{Frege, Gottlob} work on the foundations of
mathematics in the late 1800's; in fact Frege's work inspired Russell's
discovery.  Despite the paradox, Frege's work contains important concepts that
have significantly influenced modern logic.  Kline's {\em Mathematics, The
Loss of Certainty} \cite{Klinel}\index{Kline, Morris} has an interesting
discussion of this topic.

What {\em can} be achieved with absolute certainty\index{certainty} is the
knowledge that if we assume the axioms are consistent and true, then the
results derived from them are true.  Part of the beauty of mathematics is that
it is the one area of human endeavor where absolute certainty can be achieved
in this sense.  A mathematical truth will remain such for eternity.  However,
our actual knowledge of whether a particular statement is a mathematical truth
is only as certain as the correctness of the proof that establishes it.  If
the proof of a statement is questionable or vague, we can't have absolute
confidence in the truth that the statement claims.

Let us look at some traditional ways of expressing proofs.

Except in the field of formal logic\index{formal logic}, almost all
traditional proofs in mathematics are really not proofs at all, but rather
proof outlines or hints as to how to go about constructing the proof.  Many
gaps\index{gaps in proofs} are left for the reader to fill in. There are
several reasons for this.  First, it is usually assumed in mathematical
literature that the person reading the proof is a mathematician familiar with
the specialty being described, and that the missing steps are obvious to such
a reader or at least that the reader is capable of filling them in.  This
attitude is fine for professional mathematicians in the specialty, but
unfortunately it often has the drawback of cutting off the rest of the world,
including mathematicians in other specialties, from understanding the proof.
We discussed one possible resolution to this on p.~\pageref{envision}.
Second, it is often assumed that a complete formal proof\index{formal proof}
would require countless millions of symbols (Section~\ref{dream}). This might
be true if the proof were to be expressed directly in terms of the axioms of
logic and set theory,\index{set theory} but it is usually not true if we allow
ourselves a hierarchy\index{hierarchy} of definitions and theorems to build
upon, using a notation that allows us to introduce new symbols, definitions,
and theorems in a precisely specified way.

Even in formal logic,\index{formal logic} formal proofs\index{formal proof}
that are considered complete still contain hidden or implicit information.
For example, a ``proof'' is usually defined as a sequence of
wffs,\index{well-formed formula (wff)}\footnote{A {\em wff} or well-formed
formula is a mathematical expression (string of symbols) constructed according
to some precise rules.  A formal mathematical system\index{formal system}
contains (1) the rules for constructing syntactically correct
wffs,\index{syntax rules} (2) a list of starting wffs called
axioms,\index{axiom} and (3) one or more rules prescribing how to derive new
wffs, called theorems\index{theorem}, from the axioms or previously derived
theorems.  An example of such a system is contained in
Metamath's\index{Metamath} set theory database, which defines a formal
system\index{formal system} from which all of standard mathematics can be
derived.  Section~\ref{startf} steps you through a complete example of a formal
system, and you may want to skim it now if you are unfamiliar with the
concept.} each of which is an axiom or follows from a rule applied to previous
wffs in the sequence.  The implicit part of the proof is the algorithm by
which a sequence of symbols is verified to be a valid wff, given the
definition of a wff.  The algorithm in this case is rather simple, but for a
computer to verify the proof,\index{automated proof verification} it must have
the algorithm built into its verification program.\footnote{It is possible, of
course, to specify wff construction syntax outside of the program itself
with a suitable input language (the Metamath language being an example), but
some proof-verification or theorem-proving programs lack the ability to extend
wff syntax in such a fashion.} If one deals exclusively with axioms and
elementary wffs, it is straightforward to implement such an algorithm.  But as
more and more definitions are added to the theory in order to make the
expression of wffs more compact, the algorithm becomes more and more
complicated.  A computer program that implements the algorithm becomes larger
and harder to understand as each definition is introduced, and thus more prone
to bugs.\index{computer program bugs}  The larger the program, the
more suspicious the mathematician may be about
the validity of its algorithms.  This is especially true because
computer programs are inherently hard to follow to begin with, and few people
enjoy verifying them manually in detail.

Metamath\index{Metamath} takes a different approach.  Metamath's ``knowledge''
is limited to the ability to substitute variables for expressions, subject to
some simple constraints.  Once the basic algorithm of Metamath is assumed to
be debugged, and perhaps independently confirmed, it
can be trusted once and for all.  The information that Metamath needs to
``understand'' mathematics is contained entirely in the body of knowledge
presented to Metamath.  Any errors in reasoning can only be errors in the
axioms or definitions contained in this body of knowledge.  As a
``constructive'' language\index{constructive language} Metamath has no
conditional branches or loops like the ones that make computer programs hard
to decipher; instead, the language can only build new sequences of symbols
from earlier sequences  of symbols.

The simplicity of the rules that underlie Metamath not only makes Metamath
easy to learn but also gives Metamath a great deal of flexibility. For
example, Metamath is not limited to describing standard first-order
logic\index{first-order logic}; higher-order logics\index{higher-order logic}
and fragments of logic\index{weak logic} can be described just as easily.
Metamath gives you the freedom to define whatever wff notation you prefer; it
has no built-in conception of the syntax of a wff.\index{well-formed formula
(wff)}  With suitable axioms and definitions, Metamath can even describe and
prove things about itself.\index{Metamath!self-description}  (John
Harrison\index{Harrison, John} discusses the ``reflection''
principle\index{reflection principle} involved in self-descriptive systems in
\cite{Harrison}.)

The flexibility of Metamath requires that its proofs specify a lot of detail,
much more than in an ordinary ``formal'' proof.\index{formal proof}  For
example, in an ordinary formal proof, a single step consists of displaying the
wff that constitutes that step.  In order for a computer program to verify
that the step is acceptable, it first must verify that the symbol sequence
being displayed is an acceptable wff.\index{automated proof verification} Most
proof verifiers have at least basic wff syntax built into their programs.
Metamath has no hard-wired knowledge of what constitutes a wff built into it;
instead every wff must be explicitly constructed based on rules defining wffs
that are present in a database.  Thus a single step in an ordinary formal
proof may be correspond to many steps in a Metamath proof. Despite the larger
number of steps, though, this does not mean that a Metamath proof must be
significantly larger than an ordinary formal proof. The reason is that since
we have constructed the wff from scratch, we know what the wff is, so there is
no reason to display it.  We only need to refer to a sequence of statements
that construct it.  In a sense, the display of the wff in an ordinary formal
proof is an implicit proof of its own validity as a wff; Metamath just makes
the proof explicit. (Section~\ref{proof} describes Metamath's proof notation.)

\section{Computers and Mathematicians}

\begin{quote}
  {\em The computer is important, but not to mathematics.}
    \flushright\sc  Paul Halmos\footnote{As quoted in \cite{Albers}, p.~121.}\\
\end{quote}\index{Halmos, Paul}

Pure mathematicians have traditionally been indifferent to computers, even to
the point of disdain.\index{computers and pure mathematics}  Computer science
itself is sometimes considered to fall in the mundane realm of ``applied''
mathematics, perhaps essential for the real world but intellectually unexciting
to those who seek the deepest truths in mathematics.  Perhaps a reason for this
attitude towards computers is that there is little or no computer software that
meets their needs, and there may be a general feeling that such software could
not even exist.  On the one hand, there are the practical computer algebra
systems, which can perform amazing symbolic manipulations in algebra and
calculus,\index{computer algebra system} yet can't prove the simplest
existence theorem, if the idea of a proof is present at all.  On the other
hand, there are specialized automated theorem provers that technically speaking
may generate correct proofs.\index{automated theorem proving}  But sometimes
their specialized input notation may be cryptic and their output perceived to
be long, inelegant, incomprehensible proofs.    The output
may be viewed with suspicion, since the program that generates it tends to be
very large, and its size increases the potential for bugs\index{computer
program bugs}.  Such a proof may be considered trustworthy only if
independently verified and ``understood'' by a human, but no one wants to
waste their time on such a boring, unrewarding chore.



\needspace{4\baselineskip}
\subsection{Trusting the Computer}

\begin{quote}
  {\em \ldots I continue to find the quasi-empirical interpretation of
  computer proofs to be the more plausible.\ldots Since not
  everything that claims to be a computer proof can be
  accepted as valid, what are the mathematical criteria for acceptable
  computer proofs?}
    \flushright\sc  Thomas Tymoczko\footnote{\cite{Tymoczko}, p.~245.}\\
\end{quote}\index{Tymoczko, Thomas}

In some cases, computers have been essential tools for proving famous
theorems.  But if a proof is so long and obscure that it can be verified in a
practical way only with a computer, it is vaguely felt to be suspicious.  For
example, proving the famous four-color theorem\index{four-color
theorem}\index{proof length} (``a map needs no more than four colors to
prevent any two adjacent countries from having the same color'') can presently
only be done with the aid of a very complex computer program which originally
required 1200 hours of computer time. There has been considerable debate about
whether such a proof can be trusted and whether such a proof is ``real''
mathematics \cite{Swart}\index{Swart, E. R.}.\index{trusting computers}

However, under normal circumstances even a skeptical mathematician would have a
great deal of confidence in the result of multiplying two numbers on a pocket
calculator, even though the precise details of what goes on are hidden from its
user.  Even the verification on a supercomputer that a huge number is prime is
trusted, especially if there is independent verification; no one bothers to
debate the philosophical significance of its ``proof,'' even though the actual
proof would be so large that it would be completely impractical to ever write
it down on paper.  It seems that if the algorithm used by the computer is
simple enough to be readily understood, then the computer can be trusted.

Metamath\index{Metamath} adopts this philosophy.  The simplicity of its
language makes it easy to learn, and because of its simplicity one can have
essentially absolute confidence that a proof is correct. All axioms, rules, and
definitions are available for inspection at any time because they are defined
by the user; there are no hidden or built-in rules that may be prone to subtle
bugs\index{computer program bugs}.  The basic algorithm at the heart of
Metamath is simple and fixed, and it can be assumed to be bug-free and robust
with a degree of confidence approaching certainty.
Independently written implementations of the Metamath verifier
can reduce any residual doubt on the part of a skeptic even further;
there are now over a dozen such implementations, written by many people.

\subsection{Trusting the Mathematician}\label{trust}

\begin{quote}
  {\em There is no Algebraist nor Mathematician so expert in his science, as
  to place entire confidence in any truth immediately upon his discovery of it,
  or regard it as any thing, but a mere probability.  Every time he runs over
  his proofs, his confidence encreases; but still more by the approbation of
  his friends; and is rais'd to its utmost perfection by the universal assent
  and applauses of the learned world.}
  \flushright\sc David Hume\footnote{{\em A Treatise of Human Nature}, as
  quoted in \cite{deMillo}, p.~267.}\\
\end{quote}\index{Hume, David}

\begin{quote}
  {\em Stanislaw Ulam estimates that mathematicians publish 200,000 theorems
  every year.  A number of these are subsequently contradicted or otherwise
  disallowed, others are thrown into doubt, and most are ignored.}
  \flushright\sc Richard de Millo et. al.\footnote{\cite{deMillo}, p.~269.}\\
\end{quote}\index{Ulam, Stanislaw}

Whether or not the computer can be trusted, humans  of course will occasionally
err. Only the most memorable proofs get independently verified, and of these
only a handful of truly great ones achieve the status of being ``known''
mathematical truths that are used without giving a second thought to their
correctness.

There are many famous examples of incorrect theorems and proofs in
mathematical literature.\index{errors in proofs}

\begin{itemize}
\item There have been thousands of purported proofs of Fermat's Last
Theorem\index{Fermat's Last Theorem} (``no integer solutions exist to $x^n +
y^n = z^n$ for $n > 2$''), by amateurs, cranks, and well-regarded
mathematicians \cite[p.~5]{Stark}\index{Stark, Harold M}.  Fermat wrote a note
in his copy of Bachet's {\em Diophantus} that he found ``a truly marvelous
proof of this theorem but this margin is too narrow to contain it''
\cite[p.~507]{Kramer}.  A recent, much publicized proof by Yoichi
Miyaoka\index{Miyaoka, Yoichi} was shown to be incorrect ({\em Science News},
April 9, 1988, p.~230).  The theorem was finally proved by Andrew
Wiles\index{Wiles, Andrew} ({\em Science News}, July 3, 1993, p.~5), but it
initially had some gaps and took over a year after its announcement to be
checked thoroughly by experts.  On Oct. 25, 1994, Wiles announced that the last
gap found in his proof had been filled in.
  \item In 1882, M. Pasch discovered that an axiom was omitted from Euclid's
formulation of geometry\index{Euclidean geometry}; without it, the proofs of
important theorems of Euclid are not valid.  Pasch's axiom\index{Pasch's
axiom} states that a line that intersects one side of a triangle must also
intersect another side, provided that it does not touch any of the triangle's
vertices.  The omission of Pasch's axiom went unnoticed for 2000
years \cite[p.~160]{Davis}, in spite of (one presumes) the thousands of
students, instructors, and mathematicians who studied Euclid.
  \item The first published proof of the famous Schr\"{o}der--Bernstein
theorem\index{Schr\"{o}der--Bernstein theorem} in set theory was incorrect
\cite[p.~148]{Enderton}\index{Enderton, Herbert B.}.  This theorem states
that if there exists a 1-to-1 function\footnote{A {\em set}\index{set} is any
collection of objects. A {\em function}\index{function} or {\em
mapping}\index{mapping} is a rule that assigns to each element of one set
(called the function's {\em domain}\index{domain}) an element from another
set.} from set $A$ into set $B$ and vice-versa, then sets $A$ and $B$ have
a 1-to-1 correspondence.  Although it sounds simple and obvious,
the standard proof is quite long and complex.
  \item In the early 1900's, Hilbert\index{Hilbert, David} published a
purported proof of the continuum hypothesis\index{continuum hypothesis}, which
was eventually established as unprovable by Cohen\index{Cohen, Paul} in 1963
\cite[p.~166]{Enderton}.  The continuum hypothesis states that no
infinity\index{infinity} (``transfinite cardinal number'')\index{cardinal,
transfinite} exists whose size (or ``cardinality''\index{cardinality}) is
between the size of the set of integers and the size of the set of real
numbers.  This hypothesis originated with German mathematician Georg
Cantor\index{Cantor, Georg} in the late 1800's, and his inability to prove it
is said to have contributed to mental illness that afflicted him in his later
years.
  \item An incorrect proof of the four-color theorem\index{four-color theorem}
was published by Kempe\index{Kempe, A. B.} in 1879
\cite[p.~582]{Courant}\index{Courant, Richard}; it stood for 11 years before
its flaw was discovered.  This theorem states that any map can be colored
using only four colors, so that no two adjacent countries have the same
color.  In 1976 the theorem was finally proved by the famous computer-assisted
proof of Haken, Appel, and Koch \cite{Swart}\index{Appel, K.}\index{Haken,
W.}\index{Koch, K.}.  Or at least it seems that way.  Mathematician
H.~S.~M.~Coxeter\index{Coxeter, H. S. M.} has doubts \cite[p.~58]{Davis}:  ``I
have a feeling that that is an untidy kind of use of the computers, and the more
you correspond with Haken and Appel, the more shaky you seem to be.''
  \item Many false ``proofs'' of the Poincar\'{e}
conjecture\index{Poincar\'{e} conjecture} have been proposed over the years.
This conjecture states that any object that mathematically behaves like a
three-dimensional sphere is a three-dimensional sphere topologically,
regardless of how it is distorted.  In March 1986, mathematicians Colin
Rourke\index{Rourke, Colin} and Eduardo R\^{e}go\index{R\^{e}go, Eduardo}
caused  a stir in the mathematical community by announcing that they had found
a proof; in November of that year the proof was found to be false \cite[p.
218]{PetersonI}.  It was finally proved in 2003 by Grigory Perelman
\label{poincare}\index{Szpiro, George}\index{Perelman, Grigory}\cite{Szpiro}.
 \end{itemize}

Many counterexamples to ``theorems'' in recent mathematical
literature related to Clifford algebras\index{Clifford algebras}
 have been found by Pertti
Lounesto (who passed away in 2002).\index{Lounesto, Pertti}
See the web page \url{http://mathforum.org/library/view/4933.html}.
% http://users.tkk.fi/~ppuska/mirror/Lounesto/counterexamples.htm

One of the purposes of Metamath\index{Metamath} is to allow proofs to be
expressed with absolute precision.  Developing a proof in the Metamath
language can be challenging, because Metamath will not permit even the
tiniest mistake.\index{errors in proofs}  But once the proof is created, its
correctness can be trusted immediately, without having to depend on the
process of peer review for confirmation.

\section{The Use of Computers in Mathematics}

\subsection{Computer Algebra Systems}

For the most part, you will find that Metamath\index{Metamath} is not a
practical tool for manipulating numbers.  (Even proving that $2 + 2 = 4$, if
you start with set theory, can be quite complex!)  Several commercial
mathematics packages are quite good at arithmetic, algebra, and calculus, and
as practical tools they are invaluable.\index{computer algebra system} But
they have no notion of proof, and cannot understand statements starting with
``there exists such and such...''.

Software packages such as Mathematica \cite{Wolfram}\index{Mathematica} do not
concern themselves with proofs but instead work directly with known results.
These packages primarily emphasize heuristic rules such as the substitution of
equals for equals to achieve simpler expressions or expressions in a different
form.  Starting with a rich collection of built-in rules and algorithms, users
can add to the collection by means of a powerful programming language.
However, results such as, say, the existence of a certain abstract object
without displaying the actual object cannot be expressed (directly) in their
languages.  The idea of a proof from a small set of axioms is absent.  Instead
this software simply assumes that each fact or rule you add to the built-in
collection of algorithms is valid.  One way to view the software is as a large
collection of axioms from which the software, with certain goals, attempts to
derive new theorems, for example equating a complex expression with a simpler
equivalent. But the terms ``theorem''\index{theorem} and
``proof,''\index{proof} for example, are not even mentioned in the index of
the user's manual for Mathematica.\index{Mathematica and proofs}  What is also
unsatisfactory from a philosophical point of view is that there is no way to
ensure the validity of the results other than by trusting the writer of each
application module or tediously checking each module by hand, similar to
checking a computer program for bugs.\index{computer program
bugs}\footnote{Two examples illustrate why the knowledge database of computer
algebra systems should sometimes be regarded with a certain caution.  If you
ask Mathematica (version 3.0) to \texttt{Solve[x\^{ }n + y\^{ }n == z\^{ }n , n]}
it will respond with \texttt{\{\{n-\char`\>-2\}, \{n-\char`\>-1\},
\{n-\char`\>1\}, \{n-\char`\>2\}\}}. In other words, Mathematica seems to
``know'' that Fermat's Last Theorem\index{Fermat's Last Theorem} is true!  (At
the time this version of Mathematica was released this fact was unknown.)  If
you ask Maple\index{Maple} to \texttt{solve(x\^{ }2 = 2\^{ }x)} then
\texttt{simplify(\{"\})}, it returns the solution set \texttt{\{2, 4\}}, apparently
unaware that $-0.7666647$\ldots is also a solution.} While of course extremely
valuable in applied mathematics, computer algebra systems tend to be of little
interest to the theoretical mathematician except as aids for exploring certain
specific problems.

Because of possible bugs, trusting the output of a computer algebra system for
use as theorems in a proof-verifier would defeat the latter's goal of rigor.
On the other hand, a fact such that a certain relatively large number is
prime, while easy for a computer algebra system to derive, might have a long,
tedious proof that could overwhelm a proof-verifier. One approach for linking
computer algebra systems to a proof-verifier while retaining the advantages of
both is to add a hypothesis to each such theorem indicating its source.  For
example, a constant {\sc maple} could indicate the theorem came from the Maple
package, and would mean ``assuming Maple is consistent, then\ldots''  This and
many other topics concerning the formalization of mathematics are discussed in
John Harrison's\index{Harrison, John} very interesting
PhD thesis~\cite{Harrison-thesis}.

\subsection{Automated Theorem Provers}\label{theoremprovers}

A mathematical theory is ``decidable''\index{decidable theory} if a mechanical
method or algorithm exists that is guaranteed to determine whether or not a
particular formula is a theorem.  Among the few theories that are decidable is
elementary geometry,\index{Euclidean geometry} as was shown by a classic
result of logician Alfred Tarski\index{Tarski, Alfred} in 1948
\cite{Tarski}.\footnote{Tarski's result actually applies to a subset of the
geometry discussed in elementary textbooks.  This subset includes most of what
would be considered elementary geometry but it is not powerful enough to
express, among other things, the notions of the circumference and area of a
circle.  Extending the theory in a way that includes notions such as these
makes the theory undecidable, as was also shown by Tarski.  Tarski's algorithm
is far too inefficient to implement practically on a computer.  A practical
algorithm for proving a smaller subset of geometry theorems (those not
involving concepts of ``order'' or ``continuity'') was discovered by Chinese
mathematician Wu Wen-ts\"{u}n in 1977 \cite{Chou}\index{Chou,
Shang-Ching}.}\index{Wen-ts{\"{u}}n, Wu}  But most theories, including
elementary arithmetic, are undecidable.  This fact contributes to keeping
mathematics alive and well, since many mathematicians believe
that they will never be
replaced by computers (if they believe Roger Penrose's argument that a
computer can never replace the brain \cite{Penrose}\index{Penrose, Roger}).
In fact,  elementary geometry is often considered a ``dead'' field
for the simple reason that it is decidable.

On the other hand, the undecidability of a theory does not mean that one cannot
use a computer to search for proofs, providing one is willing to give up if a
proof is not found after a reasonable amount of time.  The field of automated
theorem proving\index{automated theorem proving} specializes in pursuing such
computer searches.  Among the more successful results to date are those based
on an algorithm known as Robinson's resolution principle
\cite{Robinson}\index{Robinson's resolution principle}.

Automated theorem provers can be excellent tools for those willing to learn
how to use them.  But they are not widely used in mainstream pure
mathematics, even though they could probably be useful in many areas.  There
are several reasons for this.  Probably most important, the main goal in pure
mathematics is to arrive at results that are considered to be deep or
important; proving them is essential but secondary.  Usually, an automated
theorem prover cannot assist in this main goal, and by the time the main goal
is achieved, the mathematician may have already figured out the proof as a
by-product.  There is also a notational problem.  Mathematicians are used to
using very compact syntax where one or two symbols (heavily dependent on
context) can represent very complex concepts; this is part of the
hierarchy\index{hierarchy} they have built up to tackle difficult problems.  A
theorem prover on the other hand might require that a theorem be expressed in
``first-order logic,''\index{first-order logic} which is the logic on which
most of mathematics is ultimately based but which is not ordinarily used
directly because expressions can become very long.  Some automated theorem
provers are experimental programs, limited in their use to very specialized
areas, and the goal of many is simply research into the nature of automated
theorem proving itself.  Finally, much research remains to be done to enable
them to prove very deep theorems.  One significant result was a
computer proof by Larry Wos\index{Wos, Larry} and colleagues that every Robbins
algebra\index{Robbins algebra} is a Boolean  algebra\index{Boolean algebra}
({\em New York Times}, Dec. 10, 1996).\footnote{In 1933, E.~V.\
Huntington\index{Huntington, E. V.}
presented the following axiom system for
Boolean algebra with a unary operation $n$ and a binary operation $+$:
\begin{center}
    $x + y = y + x$ \\
    $(x + y) + z = x + (y + z)$ \\
    $n(n(x) + y) + n(n(x) + n(y)) = x$
\end{center}
Herbert Robbins\index{Robbins, Herbert}, a student of Huntington, conjectured
that the last equation can be replaced with a simpler one:
\begin{center}
    $n(n(x + y) + n(x + n(y))) = x$
\end{center}
Robbins and Huntington could not find a proof.  The problem was
later studied unsuccessfully by Tarski\index{Tarski, Alfred} and his
students, and it remained an unsolved problem until a
computer found the proof in 1996.  For more information on
the Robbins algebra problem see \cite{Wos}.}

How does Metamath\index{Metamath} relate to automated theorem provers?  A
theorem prover is primarily concerned with one theorem at a time (perhaps
tapping into a small database of known theorems) whereas Metamath is more like
a theorem archiving system, storing both the theorem and its proof in a
database for access and verification.  Metamath is one answer to ``what do you
do with the output of a theorem prover?''  and could be viewed as the
next step in the process.  Automated theorem provers could be useful tools for
helping develop its database.
Note that very long, automatically
generated proofs can make your database fat and ugly and cause Metamath's proof
verification to take a long time to run.  Unless you have a particularly good
program that generates very concise proofs, it might be best to consider the
use of automatically generated proofs as a quick-and-dirty approach, to be
manually rewritten at some later date.

The program {\sc otter}\index{otter@{\sc otter}}\footnote{\url{http://www.cs.unm.edu/\~mccune/otter/}.}, later succeeded by
prover9\index{prover9}\footnote{\url{https://www.cs.unm.edu/~mccune/mace4/}.},
have been historically influential.
The E prover\index{E prover}\footnote{\url{https://github.com/eprover/eprover}.}
is a maintained automated theorem prover
for full first-order logic with equality.
There are many other automated theorem provers as well.

If you want to combine automated theorem provers with Metamath
consider investigating
the book {\em Automated Reasoning:  Introduction and Applications}
\cite{Wos}\index{Wos, Larry}.  This book discusses
how to use {\sc otter} in a way that can
not only able to generate
relatively efficient proofs, it can even be instructed to search for
shorter proofs.  The effective use of {\sc otter} (and similar tools)
does require a certain
amount of experience, skill, and patience.  The axiom system used in the
\texttt{set.mm}\index{set theory database (\texttt{set.mm})} set theory
database can be expressed to {\sc otter} using a method described in
\cite{Megill}.\index{Megill, Norman}\footnote{To use those axioms with
{\sc otter}, they must be restated in a way that eliminates the need for
``dummy variables.''\index{dummy variable!eliminating} See the Comment
on p.~\pageref{nodd}.} When successful, this method tends to generate
short and clever proofs, but my experiments with it indicate that the
method will find proofs within a reasonable time only for relatively
easy theorems.  It is still fun to experiment with.

Reference \cite{Bledsoe}\index{Bledsoe, W. W.} surveys a number of approaches
people have explored in the field of automated theorem proving\index{automated
theorem proving}.

\subsection{Interactive Theorem Provers}\label{interactivetheoremprovers}

Finding proofs completely automatically is difficult, so there
are some interactive theorem provers that allow a human to guide the
computer to find a proof.
Examples include
HOL Light\index{HOL light}%
\footnote{\url{https://www.cl.cam.ac.uk/~jrh13/hol-light/}.},
Isabelle\index{Isabelle}%
\footnote{\url{http://www.cl.cam.ac.uk/Research/HVG/Isabelle}.},
{\sc hol}\index{hol@{\sc hol}}%
\footnote{\url{https://hol-theorem-prover.org/}.},
and
Coq\index{Coq}\footnote{\url{https://coq.inria.fr/}.}.

A major difference between most of these tools and Metamath is that the
``proofs'' are actually programs that guide the program to find a proof,
and not the proof itself.
For example, an Isabelle/HOL proof might apply a step
\texttt{apply (blast dest: rearrange reduction)}. The \texttt{blast}
instruction applies
an automatic tableux prover and returns if it found a sequence of proof
steps that work... but the sequence is not considered part of the proof.

A good overview of
higher-level proof verification languages (such as {\sc lcf}\index{lcf@{\sc
lcf}} and {\sc hol}\index{hol@{\sc hol}})
is given in \cite{Harrison}.  All of these languages are fundamentally
different from Metamath in that much of the mathematical foundational
knowledge is embedded in the underlying proof-verification program, rather
than placed directly in the database that is being verified.
These can have a steep learning curve for those without a mathematical
background.  For example, one usually must have a fair understanding of
mathematical logic in order to follow their proofs.

\subsection{Proof Verifiers}\label{proofverifiers}

A proof verifier is a program that doesn't generate proofs but instead
verifies proofs that you give it.  Many proof verifiers have limited built-in
automated proof capabilities, such as figuring out simple logical inferences
(while still being guided by a person who provides the overall proof).  Metamath
has no built-in automated proof capability other than the limited
capability of its Proof Assistant.

Proof-verification languages are not used as frequently as they might be.
Pure mathematicians are more concerned with producing new results, and such
detail and rigor would interfere with that goal.  The use of computers in pure
mathematics is primarily focused on automated theorem provers (not verifiers),
again with the ultimate goal of aiding the creation of new mathematics.
Automated theorem provers are usually concerned with attacking one theorem at
time rather than making a large, organized database easily available to the
user.  Metamath is one way to help close this gap.

By itself Metamath is a mostly a proof verifier.
This does not mean that other approaches can't be used; the difference
is that in Metamath, the results of various provers must be recorded
step-by-step so that they can be verified.

Another proof-verification language is Mizar,\index{Mizar} which can display
its proofs in the informal language that mathematicians are accustomed to.
Information on the Mizar language is available at \url{http://mizar.org}.

For the working mathematician, Mizar is an excellent tool for rigorously
documenting proofs. Mizar typesets its proofs in the informal English used by
mathematicians (and, while fine for them, are just as inscrutable by
laypersons!). A price paid for Mizar is a relatively steep learning curve of a
couple of weeks.  Several mathematicians are actively formalizing different
areas of mathematics using Mizar and publishing the proofs in a dedicated
journal. Unfortunately the task of formalizing mathematics is still looked
down upon to a certain extent since it doesn't involve the creation of ``new''
mathematics.

The closest system to Metamath is
the {\em Ghilbert}\index{Ghilbert} proof language (\url{http://ghilbert.org})
system developed by
Raph Levien\index{Levien, Raph}.
Ghilbert is a formal proof checker heavily inspired by Metamath.
Ghilbert statements are s-expressions (a la Lisp), which is easy
for computers to parse but many people find them hard to read.
There are a number of differences in their specific constructs, but
there is at least one tool to translate some Metamath materials into Ghilbert.
As of 2019 the Ghilbert community is smaller and less active than the
Metamath community.
That said, the Metamath and Ghilbert communities overlap, and fruitful
conversations between them have occurred many times over the years.

\subsection{Creating a Database of Formalized Mathematics}\label{mathdatabase}

Besides Metamath, there are several other ongoing projects with the goal of
formalizing mathematics into computer-verifiable databases.
Understanding some history will help.

The {\sc qed}\index{qed project@{\sc qed} project}%
\footnote{\url{http://www-unix.mcs.anl.gov/qed}.}
project arose in 1993 and its goals were outlined in the
{\sc qed} manifesto.
The {\sc qed} manifesto was
a proposal for a computer-based database of all mathematical knowledge,
strictly formalized and with all proofs having been checked automatically.
The project had a conference in 1994 and another in 1995;
there was also a ``twenty years of the {\sc qed} manifesto'' workshop
in 2014.
Its ideals are regularly reraised.

In a 2007 paper, Freek Wiedijk identified two reasons
for the failure of the {\sc qed} project as originally envisioned:%
\cite{Wiedijk-revisited}\index{Wiedijk, Freek}

\begin{itemize}
\item Very few people are working on formalization of mathematics. There is no compelling application for fully mechanized mathematics.
\item Formalized mathematics does not yet resemble traditional mathematics. This is partly due to the complexity of mathematical notation, and partly to the limitations of existing theorem provers and proof assistants.
\end{itemize}

But this did not end the dream of
formalizing mathematics into computer-verifiable databases.
The problems that led to the {\sc qed} manifesto are still with us,
even though the challenges were harder than originally considered.
What has happened instead is that various independent projects have
worked towards formalizing mathematics into computer-verifiable databases,
each simultaneously competing and cooperating with each other.

A concrete way to see this is
Freek Wiedijk's ``Formalizing 100 Theorems'' list%
\footnote{\url{http://www.cs.ru.nl/\%7Efreek/100/}.}
which shows the progress different systems have made on a challenge list
of 100 mathematical theorems.%
\footnote{ This is not the only list of ``interesting'' theorems.
Another interesting list was posted by Oliver Knill's list
\cite{Knill}\index{Knill, Oliver}.}
The top systems as of February 2019
(in order of the number of challenges completed) are
HOL Light, Isabelle, Metamath, Coq, and Mizar.

The Metamath 100%
\footnote{\url{http://us.metamath.org/mm\_100.html}}
page (maintained by David A. Wheeler\index{Wheeler, David A.})
shows the progress of Metamath (specifically its \texttt{set.mm} database)
against this challenge list maintained by Freek Wiedijk.
The Metamath \texttt{set.mm} database
has made a lot of progress over the years,
in part because working to prove those challenge theorems required
defining various terms and proving their properties as a prerequisite.
Here are just a few of the many statements that have been
formally proven with Metamath:

% The entries of this cause the narrow display to break poorly,
% since the short amount of text means LaTeX doesn't get a lot to work with
% and the itemize format gives it even *less* margin than usual.
% No one will mind if we make just this list flushleft, since this list
% will be internally consistent.
\begin{flushleft}
\begin{itemize}
\item 1. The Irrationality of the Square Root of 2
  (\texttt{sqr2irr}, by Norman Megill, 2001-08-20)
\item 2. The Fundamental Theorem of Algebra
  (\texttt{fta}, by Mario Carneiro, 2014-09-15)
\item 22. The Non-Denumerability of the Continuum
  (\texttt{ruc}, by Norman Megill, 2004-08-13)
\item 54. The Konigsberg Bridge Problem
  (\texttt{konigsberg}, by Mario Carneiro, 2015-04-16)
\item 83. The Friendship Theorem
  (\texttt{friendship}, by Alexander W. van der Vekens, 2018-10-09)
\end{itemize}
\end{flushleft}

We thank all of those who have developed at least one of the Metamath 100
proofs, and we particularly thank
Mario Carneiro\index{Carneiro, Mario}
who has contributed the most Metamath 100 proofs as of 2019.
The Metamath 100 page shows the list of all people who have contributed a
proof, and links to graphs and charts showing progress over time.
We encourage others to work on proving theorems not yet proven in Metamath,
since doing so improves the work as a whole.

Each of the math formalization systems (including Metamath)
has different strengths and weaknesses, depending on what you value.
Key aspects that differentiate Metamath from the other top systems are:

\begin{itemize}
\item Metamath is not tied to any particular set of axioms.
\item Metamath can show every step of every proof, no exceptions.
  Most other provers only assert that a proof can be found, and do not
  show every step. This also makes verification fast, because
  the system does not need to rediscover proof details.
\item The Metamath verifier has been re-implemented in many different
  programming languages, so verification can be done by multiple
  implementations.  In particular, the
  \texttt{set.mm}\index{set theory database (\texttt{set.mm})}%
  \index{Metamath Proof Explorer} database is verified by
  four different verifiers
  written in four different languages by four different authors.
  This greatly reduces the risk of accepting an invalid
  proof due to an error in the verifier.
\item Proofs stay proven.  In some systems, changes to the system's
  syntax or how a tactic works causes proofs to fail in later versions,
  causing older work to become essentially lost.
  Metamath's language is
  extremely small and fixed, so once a proof is added to a database,
  the database can be rechecked with later versions of the Metamath program
  and with other verifiers of Metamath databases.
  If an axiom or key definition needs to be changed, it is easy to
  manipulate the database as a whole to handle the change
  without touching the underlying verifier.
  Since re-verification of an entire database takes seconds, there
  is never a reason to delay complete verification.
  This aspect is especially compelling if your
  goal is to have a long-term database of proofs.
\item Licensing is generous.  The main Metamath databases are released to
  the public domain, and the main Metamath program is open source software
  under a standard, widely-used license.
\item Substitutions are easy to understand, even by those who are not
  professional mathematicians.
\end{itemize}

Of course, other systems may have advantages over Metamath
that are more compelling, depending on what you value.
In any case, we hope this helps you understand Metamath
within a wider context.

\subsection{In Summary}\label{computers-summary}

To summarize our discussions of computers and mathematics, computer algebra
systems can be viewed as theorem generators focusing on a narrow realm of
mathematics (numbers and their properties), automated theorem provers as proof
generators for specific theorems in a much broader realm covered by a built-in
formal system such as first-order logic, interactive theorem
provers require human guidance, proof verifiers verify proofs but
historically they have been
restricted to first-order logic.
Metamath, in contrast,
is a proof verifier and documenter whose realm is essentially unlimited.

\section{Mathematics and Metamath}

\subsection{Standard Mathematics}

There are a number of ways that Metamath\index{Metamath} can be used with
standard mathematics.  The most satisfying way philosophically is to start at
the very beginning, and develop the desired mathematics from the axioms of
logic and set theory.\index{set theory}  This is the approach taken in the
\texttt{set.mm}\index{set theory database (\texttt{set.mm})}%
\index{Metamath Proof Explorer}
database (also known as the Metamath Proof Explorer).
Among other things, this database builds up to the
axioms of real and complex numbers\index{analysis}\index{real and complex
numbers} (see Section~\ref{real}), and a standard development of analysis, for
example, could start at that point, using it as a basis.   Besides this
philosophical advantage, there are practical advantages to having all of the
tools of set theory available in the supporting infrastructure.

On the other hand, you may wish to start with the standard axioms of a
mathematical theory without going through the set theoretical proofs of those
axioms.  You will need mathematical logic to make inferences, but if you wish
you can simply introduce theorems\index{theorem} of logic as
``axioms''\index{axiom} wherever you need them, with the implicit assumption
that in principle they can be proved, if they are obvious to you.  If you
choose this approach, you will probably want to review the notation used in
\texttt{set.mm}\index{set theory database (\texttt{set.mm})} so that your own
notation will be consistent with it.

\subsection{Other Formal Systems}
\index{formal system}

Unlike some programs, Metamath\index{Metamath} is not limited to any specific
area of mathematics, nor committed to any particular mathematical philosophy
such as classical logic versus intuitionism, nor limited, say, to expressions
in first-order logic.  Although the database \texttt{set.mm}
describes standard logic and set theory, Meta\-math
is actually a general-purpose language for describing a wide variety of formal
systems.\index{formal system}  Non-standard systems such as modal
logic,\index{modal logic} intuitionist logic\index{intuitionism}, higher-order
logic\index{higher-order logic}, quantum logic\index{quantum logic}, and
category theory\index{category theory} can all be described with the Metamath
language.  You define the symbols you prefer and tell Metamath the axioms and
rules you want to start from, and Metamath will verify any inferences you make
from those axioms and rules.  A simple example of a non-standard formal system
is Hofstadter's\index{Hofstadter, Douglas R.} MIU system,\index{MIU-system}
whose Metamath description is presented in Appendix~\ref{MIU}.

This is not hypothetical.
The largest Metamath database is
\texttt{set.mm}\index{set theory database (\texttt{set.mm}}%
\index{Metamath Proof Explorer}), aka the Metamath Proof Explorer,
which uses the most common axioms for mathematical foundations
(specifically classical logic combined with Zermelo--Fraenkel
set theory\index{Zermelo--Fraenkel set theory} with the Axiom of Choice).
But other Metamath databases are available:

\begin{itemize}
\item The database
  \texttt{iset.mm}\index{intuitionistic logic database (\texttt{iset.mm})},
  aka the
  Intuitionistic Logic Explorer\index{Intuitionistic Logic Explorer},
  uses intuitionistic logic (a constructivist point of view)
  instead of classical logic.
\item The database
  \texttt{nf.mm}\index{New Foundations database (\texttt{nf.mm})},
  aka the
  New Foundations Explorer\index{New Foundations Explorer},
  constructs mathematics from scratch,
  starting from Quine's New Foundations (NF) set theory axioms.
\item The database
  \texttt{hol.mm}\index{Higher-order Logic database (\texttt{hol.mm})},
  aka the
  Higher-Order Logic (HOL) Explorer\index{Higher-Order Logic (HOL) Explorer},
  starts with HOL (also called simple type theory) and derives
  equivalents to ZFC axioms, connecting the two approaches.
\end{itemize}

Since the days of David Hilbert,\index{Hilbert, David} mathematicians have
been concerned with the fact that the metalanguage\index{metalanguage} used to
describe mathematics may be stronger than the mathematics being described.
Metamath\index{Metamath}'s underlying finitary\index{finitary proof},
constructive nature provides a good philosophical basis for studying even the
weakest logics.\index{weak logic}

The usual treatment of many non-standard formal systems\index{formal
system} uses model theory\index{model theory} or proof theory\index{proof
theory} to describe these systems; these theories, in turn, are based on
standard set theory.  In other words, a non-standard formal system is defined
as a set with certain properties, and standard set theory is used to derive
additional properties of this set.  The standard set theory database provided
with Metamath can be used for this purpose, and when used this way
the development of a special
axiom system for the non-standard formal system becomes unnecessary.  The
model- or proof-theoretic approach often allows you to prove much deeper
results with less effort.

Metamath supports both approaches.  You can define the non-standard
formal system directly, or define the non-standard formal system as
a set with certain properties, whichever you find most helpful.

%\section{Additional Remarks}

\subsection{Metamath and Its Philosophy}

Closely related to Metamath\index{Metamath} is a philosophy or way of looking
at mathematics. This philosophy is related to the formalist
philosophy\index{formalism} of Hilbert\index{Hilbert, David} and his followers
\cite[pp.~1203--1208]{Kline}\index{Kline, Morris}
\cite[p.~6]{Behnke}\index{Behnke, H.}. In this philosophy, mathematics is
viewed as nothing more than a set of rules that manipulate symbols, together
with the consequences of those rules.  While the mathematics being described
may be complex, the rules used to describe it (the
``metamathematics''\index{metamathematics}) should be as simple as possible.
In particular, proofs should be restricted to dealing with concrete objects
(the symbols we write on paper rather than the abstract concepts they
represent) in a constructive manner; these are called ``finitary''
proofs\index{finitary proof} \cite[pp.~2--3]{Shoenfield}\index{Shoenfield,
Joseph R.}.

Whether or not you find Metamath interesting or useful will in part depend on
the appeal you find in its philosophy, and this appeal will probably depend on
your particular goals with respect to mathematics.  For example, if you are a
pure mathematician at the forefront of discovering new mathematical knowledge,
you will probably find that the rigid formality of Metamath stifles your
creativity.  On the other hand, we would argue that once this knowledge is
discovered, there are advantages to documenting it in a standard format that
will make it accessible to others.  Sixty years from now, your field may be
dormant, and as Davis and Hersh put it, your ``writings would become less
translatable than those of the Maya'' \cite[p.~37]{Davis}\index{Davis, Phillip
J.}.


\subsection{A History of the Approach Behind Metamath}

Probably the one work that has had the most motivating influence on
Metamath\index{Metamath} is Whitehead and Russell's monumental {\em Principia
Mathematica} \cite{PM}\index{Whitehead, Alfred North}\index{Russell,
Bertrand}\index{principia mathematica@{\em Principia Mathematica}}, whose aim
was to deduce all of mathematics from a small number of primitive ideas, in a
very explicit way that in principle anyone could understand and follow.  While
this work was tremendously influential in its time, from a modern perspective
it suffers from several drawbacks.  Both its notation and its underlying
axioms are now considered dated and are no longer used.  From our point of
view, its development is not really as accessible as we would like to see; for
practical reasons, proofs become more and more sketchy as its mathematics
progresses, and working them out in fine detail requires a degree of
mathematical skill and patience that many people don't have.  There are
numerous small errors, which is understandable given the tedious, technical
nature of the proofs and the lack of a computer to verify the details.
However, even today {\em Principia Mathematica} stands out as the work closest
in spirit to Metamath.  It remains a mind-boggling work, and one can't help
but be amazed at seeing ``$1+1=2$'' finally appear on page 83 of Volume II
(Theorem *110.643).

The origin of the proof notation used by Metamath dates back to the 1950's,
when the logician C.~A.~Meredith expressed his proofs in a compact notation
called ``condensed detachment''\index{condensed detachment}
\cite{Hindley}\index{Hindley, J. Roger} \cite{Kalman}\index{Kalman, J. A.}
\cite{Meredith}\index{Meredith, C. A.} \cite{Peterson}\index{Peterson, Jeremy
George}.  This notation allows proofs to be communicated unambiguously by
merely referencing the axiom\index{axiom}, rule\index{rule}, or
theorem\index{theorem} used at each step, without explicitly indicating the
substitutions\index{substitution!variable}\index{variable substitution} that
have to be made to the variables in that axiom, rule, or theorem.  Ordinarily,
condensed detachment is more or less limited to propositional
calculus\index{propositional calculus}.  The concept has been extended to
first-order logic\index{first-order logic} in \cite{Megill}\index{Megill,
Norman}, making it is easy to write a small computer program to verify proofs
of simple first-order logic theorems.\index{condensed detachment!and
first-order logic}

A key concept behind the notation of condensed detachment is called
``unification,''\index{unification} which is an algorithm for determining what
substitutions\index{substitution!variable}\index{variable substitution} to
variables have to be made to make two expressions match each other.
Unification was first precisely defined by the logician J.~A.~Robinson, who
used it in the development of a powerful
theorem-proving technique called the ``resolution principle''
\cite{Robinson}\index{Robinson's resolution principle}. Metamath does not make
use of the resolution principle, which is intended for systems of first-order
logic.\index{first-order logic}  Metamath's use is not restricted to
first-order logic, and as we have mentioned it does not automatically discover
proofs.  However, unification is a key idea behind Metamath's proof
notation, and Metamath makes use of a very simple version of it
(Section~\ref{unify}).

\subsection{Metamath and First-Order Logic}

First-order logic\index{first-order logic} is the supporting structure
for standard mathematics.  On top of it is set theory, which contains
the axioms from which virtually all of mathematics can be derived---a
remarkable fact.\index{category
theory}\index{cardinal, inaccessible}\label{categoryth}\footnote{An exception seems
to be category theory.  There are several schools of thought on whether
category theory is derivable from set theory.  At a minimum, it appears
that an additional axiom is needed that asserts the existence of an
``inaccessible cardinal'' (a type of infinity so large that standard set
theory can't prove or deny that it exists).
%
%%%% (I took this out that was in previous editions:)
% But it is also argued that not just one but a ``proper class'' of them
% is needed, and the existence of proper classes is impossible in standard
% set theory.  (A proper class is a collection of sets so huge that no set
% can contain it as an element.  Proper classes can lead to
% inconsistencies such as ``Russell's paradox.''  The axioms of standard
% set theory are devised so as to deny the existence of proper classes.)
%
For more information, see
\cite[pp.~328--331]{Herrlich}\index{Herrlich, Horst} and
\cite{Blass}\index{Blass, Andrea}.}

One of the things that makes Metamath\index{Metamath} more practical for
first-order theories is a set of axioms for first-order logic designed
specifically with Metamath's approach in mind.  These are included in
the database \texttt{set.mm}\index{set theory database (\texttt{set.mm})}.
See Chapter~\ref{fol} for a detailed
description; the axioms are shown in Section~\ref{metaaxioms}.  While
logically equivalent to standard axiom systems, our axiom system breaks
up the standard axioms into smaller pieces such that from them, you can
directly derive what in other systems can only be derived as higher-level
``metatheorems.''\index{metatheorem}  In other words, it is more powerful than
the standard axioms from a metalogical point of view.  A rigorous
justification for this system and its ``metalogical
completeness''\index{metalogical completeness} is found in
\cite{Megill}\index{Megill, Norman}.  The system is closely related to a
system developed by Monk\index{Monk, J. Donald} and Tarski\index{Tarski,
Alfred} in 1965 \cite{Monks}.

For example, the formula $\exists x \, x = y $ (given $y$, there exists some
$x$ equal to it) is a theorem of logic,\footnote{Specifically, it is a theorem
of those systems of logic that assume non-empty domains.  It is not a theorem
of more general systems that include the empty domain\index{empty domain}, in
which nothing exists, period!  Such systems are called ``free
logics.''\index{free logic} For a discussion of these systems, see
\cite{Leblanc}\index{Leblanc, Hugues}.  Since our use for logic is as a basis
for set theory, which has a non-empty domain, it is more convenient (and more
traditional) to use a less general system.  An interesting curiosity is that,
using a free logic as a basis for Zermelo--Fraenkel set
theory\index{Zermelo--Fraenkel set theory} (with the redundant Axiom of the
Null Set omitted),\index{Axiom of the Null Set} we cannot even prove the
existence of a single set without assuming the axiom of infinity!\index{Axiom
of Infinity}} whether or not $x$ and $y$ are distinct variables\index{distinct
variables}.  In many systems of logic, we would have to prove two theorems to
arrive at this result.  First we would prove ``$\exists x \, x = x $,'' then
we would separately prove ``$\exists x \, x = y $, where $x$ and $y$ are
distinct variables.''  We would then combine these two special cases ``outside
of the system'' (i.e.\ in our heads) to be able to claim, ``$\exists x \, x =
y $, regardless of whether $x$ and $y$ are distinct.''  In other words, the
combination of the two special cases is a
metatheorem.  In the system of logic
used in Metamath's set theory\index{set theory database (\texttt{set.mm})}
database, the axioms of logic are broken down into small pieces that allow
them to be reassembled in such a way that theorems such as these can be proved
directly.

Breaking down the axioms in this way makes them look peculiar and not very
intuitive at first, but rest assured that they are correct and complete.  Their
correctness is ensured because they are theorem schemes of standard first-order
logic (which you can easily verify if you are a logician).  Their completeness
follows from the fact that we explicitly derive the standard axioms of
first-order logic as theorems.  Deriving the standard axioms is somewhat
tricky, but once we're there, we have at our disposal a system that is less
awkward to work with in formal proofs\index{formal proof}.  In technical terms
that logicians understand, we eliminate the cumbersome concepts of ``free
variable,''\index{free variable} ``bound variable,''\index{bound variable} and
``proper substitution''\index{proper substitution}\index{substitution!proper}
as primitive notions.  These concepts are present in our system but are
defined in terms of concepts expressed by the axioms and can be eliminated in
principle.  In standard systems, these concepts are really like additional,
implicit axioms\index{implicit axiom} that are somewhat complex and cannot be
eliminated.

The traditional approach to logic, wherein free variables and proper
substitution is defined, is also possible to do directly in the Metamath
language.  However, the notation tends to become awkward, and there are
disadvantages:  for example, extending the definition of a wff with a
definition is awkward, because the free variable and proper substitution
concepts have to have their definitions also extended.  Our choice of
axioms for \texttt{set.mm} is to a certain extent a matter of style, in
an attempt to achieve overall simplicity, but you should also be aware
that the traditional approach is possible as well if you should choose
it.

\chapter{Using the Metamath Program}
\label{using}

\section{Installation}

The way that you install Metamath\index{Metamath!installation} on your
computer system will vary for different computers.  Current
instructions are provided with the Metamath program download at
\url{http://metamath.org}.  In general, the installation is simple.
There is one file containing the Metamath program itself.  This file is
usually called \texttt{metamath} or \texttt{metamath.}{\em xxx} where
{\em xxx} is the convention (such as \texttt{exe}) for an executable
program on your operating system.  There are several additional files
containing samples of the Metamath language, all ending with
\texttt{.mm}.  The file \texttt{set.mm}\index{set theory database
(\texttt{set.mm})} contains logic and set theory and can be used as a
starting point for other areas of mathematics.

You will also need a text editor\index{text editor} capable of editing plain
{\sc ascii}\footnote{American Standard Code for Information Interchange.} text
in order to prepare your input files.\index{ascii@{\sc ascii}}  Most computers
have this capability built in.  Note that plain text is not necessarily the
default for some word processing programs\index{word processor}, especially if
they can handle different fonts; for example, with Microsoft Word\index{Word
(Microsoft)}, you must save the file in the format ``Text Only With Line
Breaks'' to get a plain text\index{plain text} file.\footnote{It is
recommended that all lines in a Metamath source file be 79 characters or less
in length for compatibility among different computer terminals.  When creating
a source file on an editor such as Word, select a monospaced
font\index{monospaced font} such as Courier\index{Courier font} or
Monaco\index{Monaco font} to make this easier to achieve.  Better yet,
just use a plain text editor such as Notepad.}

On some computer systems, Metamath does not have the capability to print
its output directly; instead, you send its output to a file (using the
\texttt{open} commands described later).  The way you print this output
file depends on your computer.\index{printers} Some computers have a
print command, whereas with others, you may have to read the file into
an editor and print it from there.

If you want to print your Metamath source files with typeset formulas
containing standard mathematical symbols, you will need the \LaTeX\
typesetting program\index{latex@{\LaTeX}}, which is widely and freely
available for most operating systems.  It runs natively on Unix and
Linux, and can be installed on Windows as part of the free Cygwin
package (\url{http://cygwin.com}).

You can also produce {\sc html}\footnote{HyperText Markup Language.}
web pages.  The {\tt help html} command in the Metamath program will
assist you with this feature.

\section{Your First Formal System}\label{start}
\subsection{From Nothing to Zero}\label{startf}

To give you a feel for what the Metamath\index{Metamath} language looks like,
we will take a look at a very simple example from formal number
theory\index{number theory}.  This example is taken from
Mendelson\index{Mendelson, Elliot} \cite[p. 123]{Mendelson}.\footnote{To keep
the example simple, we have changed the formalism slightly, and what we call
axioms\index{axiom} are strictly speaking theorems\index{theorem} in
\cite{Mendelson}.}  We will look at a small subset of this theory, namely that
part needed for the first number theory theorem proved in \cite{Mendelson}.

First we will look at a standard formal proof\index{formal proof} for the
example we have picked, then we will look at the Metamath version.  If you
have never been exposed to formal proofs, the notation may seem to be such
overkill to express such simple notions that you may wonder if you are missing
something.  You aren't.  The concepts involved are in fact very simple, and a
detailed breakdown in this fashion is necessary to express the proof in a way
that can be verified mechanically.  And as you will see, Metamath breaks the
proof down into even finer pieces so that the mechanical verification process
can be about as simple as possible.

Before we can introduce the axioms\index{axiom} of the theory, we must define
the syntax rules for forming legal expressions\index{syntax rules}
(combinations of symbols) with which those axioms can be used. The number 0 is
a {\bf term}\index{term}; and if $ t$ and $r$ are terms, so is $(t+r)$. Here,
$ t$ and $r$ are ``metavariables''\index{metavariable} ranging over terms; they
themselves do not appear as symbols in an actual term.  Some examples of
actual terms are $(0 + 0)$ and $((0+0)+0)$.  (Note that our theory describes
only the number zero and sums of zeroes.  Of course, not much can be done with
such a trivial theory, but remember that we have picked a very small subset of
complete number theory for our example.  The important thing for you to focus
on is our definitions that describe how symbols are combined to form valid
expressions, and not on the content or meaning of those expressions.) If $ t$
and $r$ are terms, an expression of the form $ t=r$ is a {\bf wff}
(well-formed formula)\index{well-formed formula (wff)}; and if $P$ and $Q$ are
wffs, so is $(P\rightarrow Q)$ (which means ``$P$ implies
$Q$''\index{implication ($\rightarrow$)} or ``if $P$ then $Q$'').
Here $P$ and $Q$ are metavariables ranging over wffs.  Examples of actual
wffs are $0=0$, $(0+0)=0$, $(0=0 \rightarrow (0+0)=0)$, and $(0=0\rightarrow
(0=0\rightarrow 0=(0+0)))$.  (Our notation makes use of more parentheses than
are customary, but the elimination of ambiguity this way simplifies our
example by avoiding the need to define operator precedence\index{operator
precedence}.)

The {\bf axioms}\index{axiom} of our theory are all wffs of the following
form, where $ t$, $r$, and $s$ are any terms:

%Latex p. 92
\renewcommand{\theequation}{A\arabic{equation}}

\begin{equation}
(t=r\rightarrow (t=s\rightarrow r=s))
\end{equation}
\begin{equation}
(t+0)=t
\end{equation}

Note that there are an infinite number of axioms since there are an infinite
number of possible terms.  A1 and A2 are properly called ``axiom
schemes,''\index{axiom scheme} but we will refer to them as ``axioms'' for
brevity.

An axiom is a {\bf theorem}; and if $P$ and $(P\rightarrow Q)$ are theorems
(where $P$ and $Q$ are wffs), then $Q$ is also a theorem.\index{theorem}  The
second part of this definition is called the modus ponens (MP) rule of
inference\index{inference rule}\index{modus ponens}.  It allows us to obtain
new theorems from old ones.

The {\bf proof}\index{proof} of a theorem is a sequence of one or more
theorems, each of which is either an axiom or the result of modus ponens
applied to two previous theorems in the sequence, and the last of which is the
theorem being proved.

The theorem we will prove for our example is very simple:  $ t=t$.  The proof of
our theorem follows.  Study it carefully until you feel sure you
understand it.\label{zeroproof}

% Use tabu so that lines will wrap automatically as needed.
\begin{tabu} { l X X }
1. & $(t+0)=t$ & (by axiom A2) \\
2. & $(t+0)=t$ & (by axiom A2) \\
3. & $((t+0)=t \rightarrow ((t+0)=t\rightarrow t=t))$ & (by axiom A1) \\
4. & $((t+0)=t\rightarrow t=t)$ & (by MP applied to steps 2 and 3) \\
5. & $t=t$ & (by MP applied to steps 1 and 4) \\
\end{tabu}

(You may wonder why step 1 is repeated twice.  This is not necessary in the
formal language we have defined, but in Metamath's ``reverse Polish
notation''\index{reverse Polish notation (RPN)} for proofs, a previous step
can be referred to only once.  The repetition of step~1 here will enable you
to see more clearly the correspondence of this proof with the
Metamath\index{Metamath} version on p.~\pageref{demoproof}.)

Our theorem is more properly called a ``theorem scheme,''\index{theorem
scheme} for it represents an infinite number of theorems, one for each
possible term $ t$.  Two examples of actual theorems would be $0=0$ and
$(0+0)=(0+0)$.  Rarely do we prove actual theorems, since by proving schemes
we can prove an infinite number of theorems in one fell swoop.  Similarly, our
proof should really be called a ``proof scheme.''\index{proof scheme}  To
obtain an actual proof, pick an actual term to use in place of $ t$, and
substitute it for $ t$ throughout the proof.

Let's discuss what we have done here.  The axioms\index{axiom} of our theory,
A1 and A2, are trivial and obvious.  Everyone knows that adding zero to
something doesn't change it, and also that if two things are equal to a third,
then they are equal to each other. In fact, stating the trivial and obvious is
a goal to strive for in any axiomatic system.  From trivial and obvious truths
that everyone agrees upon, we can prove results that are not so obvious yet
have absolute faith in them.  If we trust the axioms and the rules, we must,
by definition, trust the consequences of those axioms and rules, if logic is
to mean anything at all.

Our rule of inference\index{rule}, modus ponens\index{modus ponens}, is also
pretty obvious once you understand what it means.  If we prove a fact $P$, and
we also prove that $P$ implies $Q$, then $Q$ necessarily follows as a new
fact.  The rule provides us with a means for obtaining new facts (i.e.\
theorems\index{theorem}) from old ones.

The theorem that we have proved, $ t=t$, is so fundamental that you may wonder
why it isn't one of the axioms\index{axiom}.  In some axiom systems of
arithmetic, it {\em is} an axiom.  The choice of axioms in a theory is to some
extent arbitrary and even an art form, constrained only by the requirement
that any two equivalent axiom systems be able to derive each other as
theorems.  We could imagine that the inventor of our axiom system originally
included $ t=t$ as an axiom, then discovered that it could be derived as a
theorem from the other axioms.  Because of this, it was not necessary to
keep it as an axiom.  By eliminating it, the final set of axioms became
that much simpler.

Unless you have worked with formal proofs\index{formal proof} before, it
probably wasn't apparent to you that $ t=t$ could be derived from our two
axioms until you saw the proof. While you certainly believe that $ t=t$ is
true, you might not be able to convince an imaginary skeptic who believes only
in our two axioms until you produce the proof.  Formal proofs such as this are
hard to come up with when you first start working with them, but after you get
used to them they can become interesting and fun.  Once you understand the
idea behind formal proofs you will have grasped the fundamental principle that
underlies all of mathematics.  As the mathematics becomes more sophisticated,
its proofs become more challenging, but ultimately they all can be broken down
into individual steps as simple as the ones in our proof above.

Mendelson's\index{Mendelson, Elliot} book, from which our example was taken,
contains a number of detailed formal proofs such as these, and you may be
interested in looking it up.  The book is intended for mathematicians,
however, and most of it is rather advanced.  Popular literature describing
formal proofs\index{formal proof} include \cite[p.~296]{Rucker}\index{Rucker,
Rudy} and \cite[pp.~204--230]{Hofstadter}\index{Hofstadter, Douglas R.}.

\subsection{Converting It to Metamath}\label{convert}

Formal proofs\index{formal proof} such as the one in our example break down
logical reasoning into small, precise steps that leave little doubt that the
results follow from the axioms\index{axiom}.  You might think that this would
be the finest breakdown we can achieve in mathematics.  However, there is more
to the proof than meets the eye. Although our axioms were rather simple, a lot
of verbiage was needed before we could even state them:  we needed to define
``term,'' ``wff,'' and so on.  In addition, there are a number of implied
rules that we haven't even mentioned. For example, how do we know that step 3
of our proof follows from axiom A1? There is some hidden reasoning involved in
determining this.  Axiom A1 has two occurrences of the letter $ t$.  One of
the implied rules states that whatever we substitute for $ t$ must be a legal
term\index{term}.\footnote{Some authors make this implied rule explicit by
stating, ``only expressions of the above form are terms,'' after defining
``term.''}  The expression $ t+0$ is pretty obviously a legal term whenever $
t$ is, but suppose we wanted to substitute a huge term with thousands of
symbols?  Certainly a lot of work would be involved in determining that it
really is a term, but in ordinary formal proofs all of this work would be
considered a single ``step.''

To express our axiom system in the Metamath\index{Metamath} language, we must
describe this auxiliary information in addition to the axioms themselves.
Metamath does not know what a ``term'' or a ``wff''\index{well-formed formula
(wff)} is.  In Metamath, the specification of the ways in which we can combine
symbols to obtain terms and wffs are like little axioms in themselves.  These
auxiliary axioms are expressed in the same notation as the ``real''
axioms\index{axiom}, and Metamath does not distinguish between the two.  The
distinction is made by you, i.e.\ by the way in which you interpret the
notation you have chosen to express these two kinds of axioms.

The Metamath language breaks down mathematical proofs into tiny pieces, much
more so than in ordinary formal proofs\index{formal proof}.  If a single
step\index{proof step} involves the
substitution\index{substitution!variable}\index{variable substitution} of a
complex term for one of its variables, Metamath must see this single step
broken down into many small steps.  This fine-grained breakdown is what gives
Metamath generality and flexibility as it lets it not be limited to any
particular mathematical notation.

Metamath's proof notation is not, in itself, intended to be read by humans but
rather is in a compact format intended for a machine.  The Metamath program
will convert this notation to a form you can understand, using the \texttt{show
proof}\index{\texttt{show proof} command} command.  You can tell the program what
level of detail of the proof you want to look at.  You may want to look at
just the logical inference steps that correspond
to ordinary formal proof steps,
or you may want to see the fine-grained steps that prove that an expression is
a term.

Here, without further ado, is our example converted to the
Metamath\index{Metamath} language:\index{metavariable}\label{demo0}

\begin{verbatim}
$( Declare the constant symbols we will use $)
    $c 0 + = -> ( ) term wff |- $.
$( Declare the metavariables we will use $)
    $v t r s P Q $.
$( Specify properties of the metavariables $)
    tt $f term t $.
    tr $f term r $.
    ts $f term s $.
    wp $f wff P $.
    wq $f wff Q $.
$( Define "term" and "wff" $)
    tze $a term 0 $.
    tpl $a term ( t + r ) $.
    weq $a wff t = r $.
    wim $a wff ( P -> Q ) $.
$( State the axioms $)
    a1 $a |- ( t = r -> ( t = s -> r = s ) ) $.
    a2 $a |- ( t + 0 ) = t $.
$( Define the modus ponens inference rule $)
    ${
       min $e |- P $.
       maj $e |- ( P -> Q ) $.
       mp  $a |- Q $.
    $}
$( Prove a theorem $)
    th1 $p |- t = t $=
  $( Here is its proof: $)
       tt tze tpl tt weq tt tt weq tt a2 tt tze tpl
       tt weq tt tze tpl tt weq tt tt weq wim tt a2
       tt tze tpl tt tt a1 mp mp
     $.
\end{verbatim}\index{metavariable}

A ``database''\index{database} is a set of one or more {\sc ascii} source
files.  Here's a brief description of this Metamath\index{Metamath} database
(which consists of this single source file), so that you can understand in
general terms what is going on.  To understand the source file in detail, you
should read Chapter~\ref{languagespec}.

The database is a sequence of ``tokens,''\index{token} which are normally
separated by spaces or line breaks.  The only tokens that are built into
the Metamath language are those beginning with \texttt{\$}.  These tokens
are called ``keywords.''\index{keyword}  All other tokens are
user-defined, and their names are arbitrary.

As you might have guessed, the Metamath token \texttt{\$(}\index{\texttt{\$(} and
\texttt{\$)} auxiliary keywords} starts a comment and \texttt{\$)} ends a comment.

The Metamath tokens \texttt{\$c}\index{\texttt{\$c} statement},
\texttt{\$v}\index{\texttt{\$v} statement},
\texttt{\$e}\index{\texttt{\$e} statement},
\texttt{\$f}\index{\texttt{\$f} statement},
\texttt{\$a}\index{\texttt{\$a} statement}, and
\texttt{\$p}\index{\texttt{\$p} statement} specify ``statements'' that
end with \texttt{\$.}\,.\index{\texttt{\$.}\ keyword}

The Metamath tokens \texttt{\$c} and \texttt{\$v} each declare\index{constant
declaration}\index{variable declaration} a list of user-defined tokens, called
``math symbols,''\index{math symbol} that the database will reference later
on.  All of the math symbols they define you have seen earlier except the
turnstile symbol \texttt{|-} ($\vdash$)\index{turnstile ({$\,\vdash$})}, which is
commonly used by logicians to mean ``a proof exists for.''  For us
the turnstile is just a
convenient symbol that distinguishes expressions that are axioms\index{axiom}
or theorems\index{theorem} from expressions that are terms or wffs.

The \texttt{\$c} statement declares ``constants''\index{constant} and
the \texttt{\$v} statement declares
``variables''\index{variable}\index{constant declaration}\index{variable
declaration} (or more precisely, metavariables\index{metavariable}).  A
variable may be substituted\index{substitution!variable}\index{variable
substitution} with sequences of math symbols whereas a constant may not
be substituted with anything.

It may seem redundant to require both \texttt{\$c}\index{\texttt{\$c} statement} and
\texttt{\$v}\index{\texttt{\$v} statement} statements (since any math
symbol\index{math symbol} not specified with a \texttt{\$c} statement could be
presumed to be a variable), but this provides for better error checking and
also allows math symbols to be redeclared\index{redeclaration of symbols}
(Section~\ref{scoping}).

The token \texttt{\$f}\index{\texttt{\$f} statement} specifies a
statement called a ``variable-type hypothesis'' (also called a
``floating hypothesis'') and \texttt{\$e}\index{\texttt{\$e} statement}
specifies a ``logical hypothesis'' (also called an ``essential
hypothesis'').\index{hypothesis}\index{variable-type
hypothesis}\index{logical hypothesis}\index{floating
hypothesis}\index{essential hypothesis} The token
\texttt{\$a}\index{\texttt{\$a} statement} specifies an ``axiomatic
assertion,''\index{axiomatic assertion} and
\texttt{\$p}\index{\texttt{\$p} statement} specifies a ``provable
assertion.''\index{provable assertion} To the left of each occurrence of
these four tokens is a ``label''\index{label} that identifies the
hypothesis or assertion for later reference.  For example, the label of
the first axiomatic assertion is \texttt{tze}.  A \texttt{\$f} statement
must contain exactly two math symbols, a constant followed by a
variable.  The \texttt{\$e}, \texttt{\$a}, and \texttt{\$p} statements
each start with a constant followed by, in general, an arbitrary
sequence of math symbols.

Associated with each assertion\index{assertion} is a set of hypotheses
that must be satisfied in order for the assertion to be used in a proof.
These are called the ``mandatory hypotheses''\index{mandatory
hypothesis} of the assertion.  Among those hypotheses whose ``scope''
(described below) includes the assertion, \texttt{\$e} hypotheses are
always mandatory and \texttt{\$f}\index{\texttt{\$f} statement}
hypotheses are mandatory when they share their variable with the
assertion or its \texttt{\$e} hypotheses.  The exact rules for
determining which hypotheses are mandatory are described in detail in
Sections~\ref{frames} and \ref{scoping}.  For example, the mandatory
hypotheses of assertion \texttt{tpl} are \texttt{tt} and \texttt{tr},
whereas assertion \texttt{tze} has no mandatory hypotheses because it
contains no variables and has no \texttt{\$e}\index{\texttt{\$e}
statement} hypothesis.  Metamath's \texttt{show statement}
command\index{\texttt{show statement} command}, described in the next
section, will show you a statement's mandatory hypotheses.

Sometimes we need to make a hypothesis relevant to only certain
assertions.  The set of statements to which a hypothesis is relevant is
called its ``scope.''  The Metamath brackets,
\texttt{\$\char`\{}\index{\texttt{\$\char`\{} and \texttt{\$\char`\}}
keywords} and \texttt{\$\char`\}}, define a ``block''\index{block} that
delimits the scope of any hypothesis contained between them.  The
assertion \texttt{mp} has mandatory hypotheses \texttt{wp}, \texttt{wq},
\texttt{min}, and \texttt{maj}.  The only mandatory hypothesis of
\texttt{th1}, on the other hand, is \texttt{tt}, since \texttt{th1}
occurs outside of the block containing \texttt{min} and \texttt{maj}.

Note that \texttt{\$\char`\{} and \texttt{\$\char`\}} do not affect the
scope of assertions (\texttt{\$a} and \texttt{\$p}).  Assertions are always
available to be referenced by any later proof in the source file.

Each provable assertion (\texttt{\$p}\index{\texttt{\$p} statement}
statement) has two parts.  The first part is the
assertion\index{assertion} itself, which is a sequence of math
symbol\index{math symbol} tokens placed between the \texttt{\$p} token
and a \texttt{\$=}\index{\texttt{\$=} keyword} token.  The second part
is a ``proof,'' which is a list of label tokens placed between the
\texttt{\$=} token and the \texttt{\$.}\index{\texttt{\$.}\ keyword}\
token that ends the statement.\footnote{If you've looked at the
\texttt{set.mm} database, you may have noticed another notation used for
proofs.  The other notation is called ``compressed.''\index{compressed
proof}\index{proof!compressed} It reduces the amount of space needed to
store a proof in the database and is described in
Appendix~\ref{compressed}.  In the example above, we use
``normal''\index{normal proof}\index{proof!normal} notation.} The proof
acts as a series of instructions to the Metamath program, telling it how
to build up the sequence of math symbols contained in the assertion part of
the \texttt{\$p} statement, making use of the hypotheses of the
\texttt{\$p} statement and previous assertions.  The construction takes
place according to precise rules.  If the list of labels in the proof
causes these rules to be violated, or if the final sequence that results
does not match the assertion, the Metamath program will notify you with
an error message.

If you are familiar with reverse Polish notation (RPN), which is sometimes used
on pocket calculators, here in a nutshell is how a proof works.  Each
hypothesis label\index{hypothesis label} in the proof is pushed\index{push}
onto the RPN stack\index{stack}\index{RPN stack} as it is encountered. Each
assertion label\index{assertion label} pops\index{pop} off the stack as many
entries as the referenced assertion has mandatory hypotheses.  Variable
substitutions\index{substitution!variable}\index{variable substitution} are
computed which, when made to the referenced assertion's mandatory hypotheses,
cause these hypotheses to match the stack entries. These same substitutions
are then made to the variables in the referenced assertion itself, which is
then pushed onto the stack.  At the end of the proof, there should be one
stack entry, namely the assertion being proved.  This process is explained in
detail in Section~\ref{proof}.

Metamath's proof notation is not very readable for humans, but it allows the
proof to be stored compactly in a file.  The Metamath\index{Metamath} program
has proof display features that let you see what's going on in a more
readable way, as you will see in the next section.

The rules used in verifying a proof are not based on any built-in syntax of the
symbol sequence in an assertion\index{assertion} nor on any built-in meanings
attached to specific symbol names.  They are based strictly on symbol
matching:  constants\index{constant} must match themselves, and
variables\index{variable} may be replaced with anything that allows a match to
occur.  For example, instead of \texttt{term}, \texttt{0}, and \verb$|-$ we could
have just as well used \texttt{yellow}, \texttt{zero}, and \texttt{provable}, as long
as we did so consistently throughout the database.  Also, we could have used
\texttt{is provable} (two tokens) instead of \verb$|-$ (one token) throughout the
database.  In each of these cases, the proof would be exactly the same.  The
independence of proofs and notation means that you have a lot of flexibility to
change the notation you use without having to change any proofs.

\section{A Trial Run}\label{trialrun}

Now you are ready to try out the Metamath\index{Metamath} program.

On all computer systems, Metamath has a standard ``command line
interface'' (CLI)\index{command line interface (CLI)} that allows you to
interact with it.  You supply commands to the CLI by typing them on the
keyboard and pressing your keyboard's {\em return} key after each line
you enter.  The CLI is designed to be easy to use and has built-in help
features.

The first thing you should do is to use a text editor to create a file
called \texttt{demo0.mm} and type into it the Metamath source shown on
p.~\pageref{demo0}.  Actually, this file is included with your Metamath
software package, so check that first.  If you type it in, make sure
that you save it in the form of ``plain {\sc ascii} text with line
breaks.''  Most word processors will have this feature.

Next you must run the Metamath program.  Depending on your computer
system and how Metamath is installed, this could range from clicking the
mouse on the Metamath icon to typing \texttt{run metamath} to typing
simply \texttt{metamath}.  (Metamath's {\tt help invoke} command describes
alternate ways of invoking the Metamath program.)

When you first enter Metamath\index{Metamath}, it will be at the CLI, waiting
for your input. You will see something like the following on your screen:
\begin{verbatim}
Metamath - Version 0.177 27-Apr-2019
Type HELP for help, EXIT to exit.
MM>
\end{verbatim}
The \texttt{MM>} prompt means that Metamath is waiting for a command.
Command keywords\index{command keyword} are not case sensitive;
we will use lower-case commands in our examples.
The version number and its release date will probably be different on your
system from the one we show above.

The first thing that you need to do is to read in your
database:\index{\texttt{read} command}\footnote{If a directory path is
needed on Unix,\index{Unix file names}\index{file names!Unix} you should
enclose the path/file name in quotes to prevent Metamath from thinking
that the \texttt{/} in the path name is a command qualifier, e.g.,
\texttt{read \char`\"db/set.mm\char`\"}.  Quotes are optional when there
is no ambiguity.}
\begin{verbatim}
MM> read demo0.mm
\end{verbatim}
Remember to press the {\em return} key after entering this command.  If
you omit the file name, Metamath will prompt you for one.   The syntax for
specifying a Macintosh file name path is given in a footnote on
p.~\pageref{includef}.\index{Macintosh file names}\index{file
names!Macintosh}

If there are any syntax errors in the database, Metamath will let you know
when it reads in the file.  The one thing that Metamath does not check when
reading in a database is that all proofs are correct, because this would
slow it down too much.  It is a good idea to periodically verify the proofs in
a database you are making changes to.  To do this, use the following command
(and do it for your \texttt{demo0.mm} file now).  Note that the \texttt{*} is a
``wild card'' meaning all proofs in the file.\index{\texttt{verify proof} command}
\begin{verbatim}
MM> verify proof *
\end{verbatim}
Metamath will report any proofs that are incorrect.

It is often useful to save the information that the Metamath program displays
on the screen. You can save everything that happens on the screen by opening a
log file. You may want to do this before you read in a database so that you
can examine any errors later on.  To open a log file, type
\begin{verbatim}
MM> open log abc.log
\end{verbatim}
This will open a file called \texttt{abc.log}, and everything that appears on the
screen from this point on will be stored in this file.  The name of the log file
is arbitrary. To close the log file, type
\begin{verbatim}
MM> close log
\end{verbatim}

Several commands let you examine what's inside your database.
Section~\ref{exploring} has an overview of some useful ones.  The
\texttt{show labels} command lets you see what statement
labels\index{label} exist.  A \texttt{*} matches any combination of
characters, and \texttt{t*} refers to all labels starting with the
letter \texttt{t}.\index{\texttt{show labels} command} The \texttt{/all}
is a ``command qualifier''\index{command qualifier} that tells Metamath
to include labels of hypotheses.  (To see the syntax explained, type
\texttt{help show labels}.)  Type
\begin{verbatim}
MM> show labels t* /all
\end{verbatim}
Metamath will respond with
\begin{verbatim}
The statement number, label, and type are shown.
3 tt $f       4 tr $f       5 ts $f       8 tze $a
9 tpl $a      19 th1 $p
\end{verbatim}

You can use the \texttt{show statement} command to get information about a
particular statement.\index{\texttt{show statement} command}
For example, you can get information about the statement with label \texttt{mp}
by typing
\begin{verbatim}
MM> show statement mp /full
\end{verbatim}
Metamath will respond with
\begin{verbatim}
Statement 17 is located on line 43 of the file
"demo0.mm".
"Define the modus ponens inference rule"
17 mp $a |- Q $.
Its mandatory hypotheses in RPN order are:
  wp $f wff P $.
  wq $f wff Q $.
  min $e |- P $.
  maj $e |- ( P -> Q ) $.
The statement and its hypotheses require the
      variables:  Q P
The variables it contains are:  Q P
\end{verbatim}
The mandatory hypotheses\index{mandatory hypothesis} and their
order\index{RPN order} are
useful to know when you are trying to understand or debug a proof.

Now you are ready to look at what's really inside our proof.  First, here is
how to look at every step in the proof---not just the ones corresponding to an
ordinary formal proof\index{formal proof}, but also the ones that build up the
formulas that appear in each ordinary formal proof step.\index{\texttt{show
proof} command}
\begin{verbatim}
MM> show proof th1 /lemmon /all
\end{verbatim}

This will display the proof on the screen in the following format:
\begin{verbatim}
 1 tt            $f term t
 2 tze           $a term 0
 3 1,2 tpl       $a term ( t + 0 )
 4 tt            $f term t
 5 3,4 weq       $a wff ( t + 0 ) = t
 6 tt            $f term t
 7 tt            $f term t
 8 6,7 weq       $a wff t = t
 9 tt            $f term t
10 9 a2          $a |- ( t + 0 ) = t
11 tt            $f term t
12 tze           $a term 0
13 11,12 tpl     $a term ( t + 0 )
14 tt            $f term t
15 13,14 weq     $a wff ( t + 0 ) = t
16 tt            $f term t
17 tze           $a term 0
18 16,17 tpl     $a term ( t + 0 )
19 tt            $f term t
20 18,19 weq     $a wff ( t + 0 ) = t
21 tt            $f term t
22 tt            $f term t
23 21,22 weq     $a wff t = t
24 20,23 wim     $a wff ( ( t + 0 ) = t -> t = t )
25 tt            $f term t
26 25 a2         $a |- ( t + 0 ) = t
27 tt            $f term t
28 tze           $a term 0
29 27,28 tpl     $a term ( t + 0 )
30 tt            $f term t
31 tt            $f term t
32 29,30,31 a1   $a |- ( ( t + 0 ) = t -> ( ( t + 0 )
                                     = t -> t = t ) )
33 15,24,26,32 mp  $a |- ( ( t + 0 ) = t -> t = t )
34 5,8,10,33 mp  $a |- t = t
\end{verbatim}

The \texttt{/lemmon} command qualifier specifies what is known as a Lemmon-style
display\index{Lemmon-style proof}\index{proof!Lemmon-style}.  Omitting the
\texttt{/lemmon} qualifier results in a tree-style proof (see
p.~\pageref{treeproof} for an example) that is somewhat less explicit but
easier to follow once you get used to it.\index{tree-style
proof}\index{proof!tree-style}

The first number on each line is the step
number of the proof.  Any numbers that follow are step numbers assigned to the
hypotheses of the statement referenced by that step.  Next is the label of
the statement referenced by the step.  The statement type of the statement
referenced comes next, followed by the math symbol\index{math symbol} string
constructed by the proof up to that step.

The last step, 34, contains the statement that is being proved.

Looking at a small piece of the proof, notice that steps 3 and 4 have
established that
\texttt{( t + 0 )} and \texttt{t} are \texttt{term}\,s, and step 5 makes use of steps 3 and
4 to establish that \texttt{( t + 0 ) = t} is a \texttt{wff}.  Let Metamath
itself tell us in detail what is happening in step 5.  Note that the
``target hypothesis'' refers to where step 5 is eventually used, i.e., in step
34.
\begin{verbatim}
MM> show proof th1 /detailed_step 5
Proof step 5:  wp=weq $a wff ( t + 0 ) = t
This step assigns source "weq" ($a) to target "wp"
($f).  The source assertion requires the hypotheses
"tt" ($f, step 3) and "tr" ($f, step 4).  The parent
assertion of the target hypothesis is "mp" ($a,
step 34).
The source assertion before substitution was:
    weq $a wff t = r
The following substitutions were made to the source
assertion:
    Variable  Substituted with
     t         ( t + 0 )
     r         t
The target hypothesis before substitution was:
    wp $f wff P
The following substitution was made to the target
hypothesis:
    Variable  Substituted with
     P         ( t + 0 ) = t
\end{verbatim}

The full proof just shown is useful to understand what is going on in detail.
However, most of the time you will just be interested in
the ``essential'' or logical steps of a proof, i.e.\ those steps
that correspond to an
ordinary formal proof\index{formal proof}.  If you type
\begin{verbatim}
MM> show proof th1 /lemmon /renumber
\end{verbatim}
you will see\label{demoproof}
\begin{verbatim}
1 a2             $a |- ( t + 0 ) = t
2 a2             $a |- ( t + 0 ) = t
3 a1             $a |- ( ( t + 0 ) = t -> ( ( t + 0 )
                                     = t -> t = t ) )
4 2,3 mp         $a |- ( ( t + 0 ) = t -> t = t )
5 1,4 mp         $a |- t = t
\end{verbatim}
Compare this to the formal proof on p.~\pageref{zeroproof} and
notice the resemblance.
By default Metamath
does not show \texttt{\$f}\index{\texttt{\$f}
statement} hypotheses and everything branching off of them in the proof tree
when the proof is displayed; this makes the proof look more like an ordinary
mathematical proof, which does not normally incorporate the explicit
construction of expressions.
This is called the ``essential'' view
(at one time you had to add the
\texttt{/essential} qualifier in the \texttt{show proof}
command to get this view, but this is now the default).
You can could use the \texttt{/all} qualifier in the \texttt{show
proof} command to also show the explicit construction of expressions.
The \texttt{/renumber} qualifier means to renumber
the steps to correspond only to what is displayed.\index{\texttt{show proof}
command}

To exit Metamath, type\index{\texttt{exit} command}
\begin{verbatim}
MM> exit
\end{verbatim}

\subsection{Some Hints for Using the Command Line Interface}

We will conclude this quick introduction to Metamath\index{Metamath} with some
helpful hints on how to navigate your way through the commands.
\index{command line interface (CLI)}

When you type commands into Metamath's CLI, you only have to type as many
characters of a command keyword\index{command keyword} as are needed to make
it unambiguous.  If you type too few characters, Metamath will tell you what
the choices are.  In the case of the \texttt{read} command, only the \texttt{r} is
needed to specify it unambiguously, so you could have typed\index{\texttt{read}
command}
\begin{verbatim}
MM> r demo0.mm
\end{verbatim}
instead of
\begin{verbatim}
MM> read demo0.mm
\end{verbatim}
In our description, we always show the full command words.  When using the
Metamath CLI commands in a command file (to be read with the \texttt{submit}
command)\index{\texttt{submit} command}, it is good practice to use
the unabbreviated command to ensure your instructions will not become ambiguous
if more commands are added to the Metamath program in the future.

The command keywords\index{command
keyword} are not case sensitive; you may type either \texttt{read} or
\texttt{ReAd}.  File names may or may not be case sensitive, depending on your
computer's operating system.  Metamath label\index{label} and math
symbol\index{math symbol} tokens\index{token} are case-sensitive.

The \texttt{help} command\index{\texttt{help} command} will provide you
with a list of topics you can get help on.  You can then type
\texttt{help} {\em topic} to get help on that topic.

If you are uncertain of a command's spelling, just type as many characters
as you remember of the command.  If you have not typed enough characters to
specify it unambiguously, Metamath will tell you what choices you have.

\begin{verbatim}
MM> show s
         ^
?Ambiguous keyword - please specify SETTINGS,
STATEMENT, or SOURCE.
\end{verbatim}

If you don't know what argument to use as part of a command, type a
\texttt{?}\index{\texttt{]}@\texttt{?}\ in command lines}\ at the
argument position.  Metamath will tell you what it expected there.

\begin{verbatim}
MM> show ?
         ^
?Expected SETTINGS, LABELS, STATEMENT, SOURCE, PROOF,
MEMORY, TRACE_BACK, or USAGE.
\end{verbatim}

Finally, you may type just the first word or words of a command followed
by {\em return}.  Metamath will prompt you for the remaining part of the
command, showing you the choices at each step.  For example, instead of
typing \texttt{show statement th1 /full} you could interact in the
following manner:
\begin{verbatim}
MM> show
SETTINGS, LABELS, STATEMENT, SOURCE, PROOF,
MEMORY, TRACE_BACK, or USAGE <SETTINGS>? st
What is the statement label <th1>?
/ or nothing <nothing>? /
TEX, COMMENT_ONLY, or FULL <TEX>? f
/ or nothing <nothing>?
19 th1 $p |- t = t $= ... $.
\end{verbatim}
After each \texttt{?}\ in this mode, you must give Metamath the
information it requests.  Sometimes Metamath gives you a list of choices
with the default choice indicated by brackets \texttt{< > }. Pressing
{\em return} after the \texttt{?}\ will select the default choice.
Answering anything else will override the default.  Note that the
\texttt{/} in command qualifiers is considered a separate
token\index{token} by the parser, and this is why it is asked for
separately.

\section{Your First Proof}\label{frstprf}

Proofs are developed with the aid of the Proof Assistant\index{Proof
Assistant}.  We will now show you how the proof of theorem \texttt{th1}
was built.  So that you can repeat these steps, we will first have the
Proof Assistant erase the proof in Metamath's source buffer\index{source
buffer}, then reconstruct it.  (The source buffer is the place in memory
where Metamath stores the information in the database when it is
\texttt{read}\index{\texttt{read} command} in.  New or modified proofs
are kept in the source buffer until a \texttt{write source}
command\index{\texttt{write source} command} is issued.)  In practice, you
would place a \texttt{?}\index{\texttt{]}@\texttt{?}\ inside proofs}\
between \texttt{\$=}\index{\texttt{\$=} keyword} and
\texttt{\$.}\index{\texttt{\$.}\ keyword}\ in the database to indicate
to Metamath\index{Metamath} that the proof is unknown, and that would be
your starting point.  Whenever the \texttt{verify proof} command encounters
a proof with a \texttt{?}\ in place of a proof step, the statement is
identified as not proved.

When I first started creating Metamath proofs, I would write down
on a piece of paper the complete
formal proof\index{formal proof} as it would appear
in a \texttt{show proof} command\index{\texttt{show proof} command}; see
the display of \texttt{show proof th1 /lemmon /re\-num\-ber} above as an
example.  After you get used to using the Proof Assistant\index{Proof
Assistant} you may get to a point where you can ``see'' the proof in your mind
and let the Proof Assistant guide you in filling in the details, at least for
simpler proofs, but until you gain that experience you may find it very useful
to write down all the details in advance.
Otherwise you may waste a lot of time as you let it take you down a wrong path.
However, others do not find this approach as helpful.
For example, Thomas Brendan Leahy\index{Leahy, Thomas Brendan}
finds that it is more helpful to him to interactively
work backward from a machine-readable statement.
David A. Wheeler\index{Wheeler, David A.}
writes down a general approach, but develops the proof
interactively by switching between
working forwards (from hypotheses and facts likely to be useful) and
backwards (from the goal) until the forwards and backwards approaches meet.
In the end, use whatever approach works for you.

A proof is developed with the Proof Assistant by working backwards, starting
with the theorem\index{theorem} to be proved, and assigning each unknown step
with a theorem or hypothesis until no more unknown steps remain.  The Proof
Assistant will not let you make an assignment unless it can be ``unified''
with the unknown step.  This means that a
substitution\index{substitution!variable}\index{variable substitution} of
variables exists that will make the assignment match the unknown step.  On the
other hand, in the middle of a proof, when working backwards, often more than
one unification\index{unification} (set of substitutions) is possible, since
there is not enough information available at that point to uniquely establish
it.  In this case you can tell Metamath which unification to choose, or you
can continue to assign unknown steps until enough information is available to
make the unification unique.

We will assume you have entered Metamath and read in the database as described
above.  The following dialog shows how the proof was developed.  For more
details on what some of the commands do, refer to Section~\ref{pfcommands}.
\index{\texttt{prove} command}

\begin{verbatim}
MM> prove th1
Entering the Proof Assistant.  Type HELP for help, EXIT
to exit.  You will be working on the proof of statement th1:
  $p |- t = t
Note:  The proof you are starting with is already complete.
MM-PA>
\end{verbatim}

The \verb/MM-PA>/ prompt means we are inside the Proof
Assistant.\index{Proof Assistant} Most of the regular Metamath commands
(\texttt{show statement}, etc.) are still available if you need them.

\begin{verbatim}
MM-PA> delete all
The entire proof was deleted.
\end{verbatim}

We have deleted the whole proof so we can start from scratch.

\begin{verbatim}
MM-PA> show new_proof/lemmon/all
1 ?              $? |- t = t
\end{verbatim}

The \texttt{show new{\char`\_}proof} command\index{\texttt{show
new{\char`\_}proof} command} is like \texttt{show proof} except that we
don't specify a statement; instead, the proof we're working on is
displayed.

\begin{verbatim}
MM-PA> assign 1 mp
To undo the assignment, DELETE STEP 5 and INITIALIZE, UNIFY
if needed.
3   min=?  $? |- $2
4   maj=?  $? |- ( $2 -> t = t )
\end{verbatim}

The \texttt{assign} command\index{\texttt{assign} command} above means
``assign step 1 with the statement whose label is \texttt{mp}.''  Note
that step renumbering will constantly occur as you assign steps in the
middle of a proof; in general all steps from the step you assign until
the end of the proof will get moved up.  In this case, what used to be
step 1 is now step 5, because the (partial) proof now has five steps:
the four hypotheses of the \texttt{mp} statement and the \texttt{mp}
statement itself.  Let's look at all the steps in our partial proof:

\begin{verbatim}
MM-PA> show new_proof/lemmon/all
1 ?              $? wff $2
2 ?              $? wff t = t
3 ?              $? |- $2
4 ?              $? |- ( $2 -> t = t )
5 1,2,3,4 mp     $a |- t = t
\end{verbatim}

The symbol \texttt{\$2} is a temporary variable\index{temporary
variable} that represents a symbol sequence not yet known.  In the final
proof, all temporary variables will be eliminated.  The general format
for a temporary variable is \texttt{\$} followed by an integer.  Note
that \texttt{\$} is not a legal character in a math symbol (see
Section~\ref{dollardollar}, p.~\pageref{dollardollar}), so there will
never be a naming conflict between real symbols and temporary variables.

Unknown steps 1 and 2 are constructions of the two wffs used by the
modus ponens rule.  As you will see at the end of this section, the
Proof Assistant\index{Proof Assistant} can usually figure these steps
out by itself, and we will not have to worry about them.  Therefore from
here on we will display only the ``essential'' hypotheses, i.e.\ those
steps that correspond to traditional formal proofs\index{formal proof}.

\begin{verbatim}
MM-PA> show new_proof/lemmon
3 ?              $? |- $2
4 ?              $? |- ( $2 -> t = t )
5 3,4 mp         $a |- t = t
\end{verbatim}

Unknown steps 3 and 4 are the ones we must focus on.  They correspond to the
minor and major premises of the modus ponens rule.  We will assign them as
follows.  Notice that because of the step renumbering that takes place
after an assignment, it is advantageous to assign unknown steps in reverse
order, because earlier steps will not get renumbered.

\begin{verbatim}
MM-PA> assign 4 mp
To undo the assignment, DELETE STEP 8 and INITIALIZE, UNIFY
if needed.
3   min=?  $? |- $2
6     min=?  $? |- $4
7     maj=?  $? |- ( $4 -> ( $2 -> t = t ) )
\end{verbatim}

We are now going to describe an obscure feature that you will probably
never use but should be aware of.  The Metamath language allows empty
symbol sequences to be substituted for variables, but in most formal
systems this feature is never used.  One of the few examples where is it
used is the MIU-system\index{MIU-system} described in
Appendix~\ref{MIU}.  But such systems are rare, and by default this
feature is turned off in the Proof Assistant.  (It is always allowed for
{\tt verify proof}.)  Let us turn it on and see what
happens.\index{\texttt{set empty{\char`\_}substitution} command}

\begin{verbatim}
MM-PA> set empty_substitution on
Substitutions with empty symbol sequences is now allowed.
\end{verbatim}

With this feature enabled, more unifications will be
ambiguous\index{ambiguous unification}\index{unification!ambiguous} in
the middle of a proof, because
substitution\index{substitution!variable}\index{variable substitution}
of variables with empty symbol sequences will become an additional
possibility.  Let's see what happens when we make our next assignment.

\begin{verbatim}
MM-PA> assign 3 a2
There are 2 possible unifications.  Please select the correct
    one or Q if you want to UNIFY later.
Unify:  |- $6
 with:  |- ( $9 + 0 ) = $9
Unification #1 of 2 (weight = 7):
  Replace "$6" with "( + 0 ) ="
  Replace "$9" with ""
  Accept (A), reject (R), or quit (Q) <A>? r
\end{verbatim}

The first choice presented is the wrong one.  If we had selected it,
temporary variable \texttt{\$6} would have been assigned a truncated
wff, and temporary variable \texttt{\$9} would have been assigned an
empty sequence (which is not allowed in our system).  With this choice,
eventually we would reach a point where we would get stuck because
we would end up with steps impossible to prove.  (You may want to
try it.)  We typed \texttt{r} to reject the choice.

\begin{verbatim}
Unification #2 of 2 (weight = 21):
  Replace "$6" with "( $9 + 0 ) = $9"
  Accept (A), reject (R), or quit (Q) <A>? q
To undo the assignment, DELETE STEP 4 and INITIALIZE, UNIFY
if needed.
 7     min=?  $? |- $8
 8     maj=?  $? |- ( $8 -> ( $6 -> t = t ) )
\end{verbatim}

The second choice is correct, and normally we would type \texttt{a}
to accept it.  But instead we typed \texttt{q} to show what will happen:
it will leave the step with an unknown unification, which can be
seen as follows:

\begin{verbatim}
MM-PA> show new_proof/not_unified
 4   min    $a |- $6
        =a2  = |- ( $9 + 0 ) = $9
\end{verbatim}

Later we can unify this with the \texttt{unify}
\texttt{all/interactive} command.

The important point to remember is that occasionally you will be
presented with several unification choices while entering a proof, when
the program determines that there is not enough information yet to make
an unambiguous choice automatically (and this can happen even with
\texttt{set empty{\char`\_}substitution} turned off).  Usually it is
obvious by inspection which choice is correct, since incorrect ones will
tend to be meaningless fragments of wffs.  In addition, the correct
choice will usually be the first one presented, unlike our example
above.

Enough of this digression.  Let us go back to the default setting.

\begin{verbatim}
MM-PA> set empty_substitution off
The ability to substitute empty expressions for variables
has been turned off.  Note that this may make the Proof
Assistant too restrictive in some cases.
\end{verbatim}

If we delete the proof, start over, and get to the point where
we digressed above, there will no longer be an ambiguous unification.

\begin{verbatim}
MM-PA> assign 3 a2
To undo the assignment, DELETE STEP 4 and INITIALIZE, UNIFY
if needed.
 7     min=?  $? |- $4
 8     maj=?  $? |- ( $4 -> ( ( $5 + 0 ) = $5 -> t = t ) )
\end{verbatim}

Let us look at our proof so far, and continue.

\begin{verbatim}
MM-PA> show new_proof/lemmon
 4 a2            $a |- ( $5 + 0 ) = $5
 7 ?             $? |- $4
 8 ?             $? |- ( $4 -> ( ( $5 + 0 ) = $5 -> t = t ) )
 9 7,8 mp        $a |- ( ( $5 + 0 ) = $5 -> t = t )
10 4,9 mp        $a |- t = t
MM-PA> assign 8 a1
To undo the assignment, DELETE STEP 11 and INITIALIZE, UNIFY
if needed.
 7     min=?  $? |- ( t + 0 ) = t
MM-PA> assign 7 a2
To undo the assignment, DELETE STEP 8 and INITIALIZE, UNIFY
if needed.
MM-PA> show new_proof/lemmon
 4 a2            $a |- ( t + 0 ) = t
 8 a2            $a |- ( t + 0 ) = t
12 a1            $a |- ( ( t + 0 ) = t -> ( ( t + 0 ) = t ->
                                                    t = t ) )
13 8,12 mp       $a |- ( ( t + 0 ) = t -> t = t )
14 4,13 mp       $a |- t = t
\end{verbatim}

Now all temporary variables and unknown steps have been eliminated from the
``essential'' part of the proof.  When this is achieved, the Proof
Assistant\index{Proof Assistant} can usually figure out the rest of the proof
automatically.  (Note that the \texttt{improve} command can occasionally be
useful for filling in essential steps as well, but it only tries to make use
of statements that introduce no new variables in their hypotheses, which is
not the case for \texttt{mp}. Also it will not try to improve steps containing
temporary variables.)  Let's look at the complete proof, then run
the \texttt{improve} command, then look at it again.

\begin{verbatim}
MM-PA> show new_proof/lemmon/all
 1 ?             $? wff ( t + 0 ) = t
 2 ?             $? wff t = t
 3 ?             $? term t
 4 3 a2          $a |- ( t + 0 ) = t
 5 ?             $? wff ( t + 0 ) = t
 6 ?             $? wff ( ( t + 0 ) = t -> t = t )
 7 ?             $? term t
 8 7 a2          $a |- ( t + 0 ) = t
 9 ?             $? term ( t + 0 )
10 ?             $? term t
11 ?             $? term t
12 9,10,11 a1    $a |- ( ( t + 0 ) = t -> ( ( t + 0 ) = t ->
                                                    t = t ) )
13 5,6,8,12 mp   $a |- ( ( t + 0 ) = t -> t = t )
14 1,2,4,13 mp   $a |- t = t
\end{verbatim}

\begin{verbatim}
MM-PA> improve all
A proof of length 1 was found for step 11.
A proof of length 1 was found for step 10.
A proof of length 3 was found for step 9.
A proof of length 1 was found for step 7.
A proof of length 9 was found for step 6.
A proof of length 5 was found for step 5.
A proof of length 1 was found for step 3.
A proof of length 3 was found for step 2.
A proof of length 5 was found for step 1.
Steps 1 and above have been renumbered.
CONGRATULATIONS!  The proof is complete.  Use SAVE
NEW_PROOF to save it.  Note:  The Proof Assistant does
not detect $d violations.  After saving the proof, you
should verify it with VERIFY PROOF.
\end{verbatim}

The \texttt{save new{\char`\_}proof} command\index{\texttt{save
new{\char`\_}proof} command} will save the proof in the database.  Here
we will just display it in a form that can be clipped out of a log file
and inserted manually into the database source file with a text
editor.\index{normal proof}\index{proof!normal}

\begin{verbatim}
MM-PA> show new_proof/normal
---------Clip out the proof below this line:
      tt tze tpl tt weq tt tt weq tt a2 tt tze tpl tt weq
      tt tze tpl tt weq tt tt weq wim tt a2 tt tze tpl tt
      tt a1 mp mp $.
---------The proof of 'th1' to clip out ends above this line.
\end{verbatim}

There is another proof format called ``compressed''\index{compressed
proof}\index{proof!compressed} that you will see in databases.  It is
not important to understand how it is encoded but only to recognize it
when you see it.  Its only purpose is to reduce storage requirements for
large proofs.  A compressed proof can always be converted to a normal
one and vice-versa, and the Metamath \texttt{show proof}
commands\index{\texttt{show proof} command} work equally well with
compressed proofs.  The compressed proof format is described in
Appendix~\ref{compressed}.

\begin{verbatim}
MM-PA> show new_proof/compressed
---------Clip out the proof below this line:
      ( tze tpl weq a2 wim a1 mp ) ABCZADZAADZAEZJJKFLIA
      AGHH $.
---------The proof of 'th1' to clip out ends above this line.
\end{verbatim}

Now we will exit the Proof Assistant.  Since we made changes to the proof,
it will warn us that we have not saved it.  In this case, we don't care.

\begin{verbatim}
MM-PA> exit
Warning:  You have not saved changes to the proof.
Do you want to EXIT anyway (Y, N) <N>? y
Exiting the Proof Assistant.
Type EXIT again to exit Metamath.
\end{verbatim}

The Proof Assistant\index{Proof Assistant} has several other commands
that can help you while creating proofs.  See Section~\ref{pfcommands}
for a list of them.

A command that is often useful is \texttt{minimize{\char`\_}with
*/brief}, which tries to shorten the proof.  It can make the process
more efficient by letting you write a somewhat ``sloppy'' proof then
clean up some of the fine details of optimization for you (although it
can't perform miracles such as restructuring the overall proof).

\section{A Note About Editing a Data\-base File}

Once your source file contains proofs, there are some restrictions on
how you can edit it so that the proofs remain valid.  Pay particular
attention to these rules, since otherwise you can lose a lot of work.
It is a good idea to periodically verify all proofs with \texttt{verify
proof *} to ensure their integrity.

If your file contains only normal (as opposed to compressed) proofs, the
main rule is that you may not change the order of the mandatory
hypotheses\index{mandatory hypothesis} of any statement referenced in a
later proof.  For example, if you swap the order of the major and minor
premise in the modus ponens rule, all proofs making use of that rule
will become incorrect.  The \texttt{show statement}
command\index{\texttt{show statement} command} will show you the
mandatory hypotheses of a statement and their order.

If a statement has a compressed proof, you also must not change the
order of {\em its} mandatory hypotheses.  The compressed proof format
makes use of this information as part of the compression technique.
Note that swapping the names of two variables in a theorem will change
the order of its mandatory hypotheses.

The safest way to edit a statement, say \texttt{mytheorem}, is to
duplicate it then rename the original to \texttt{mytheoremOLD}
throughout the database.  Once the edited version is re-proved, all
statements referencing \texttt{mytheoremOLD} can be updated in the Proof
Assistant using \texttt{minimize{\char`\_}with
mytheorem
/allow{\char`\_}growth}.\index{\texttt{minimize{\char`\_}with} command}
% 3/10/07 Note: line-breaking the above results in duplicate index entries

\chapter{Abstract Mathematics Revealed}\label{fol}

\section{Logic and Set Theory}\label{logicandsettheory}

\begin{quote}
  {\em Set theory can be viewed as a form of exact theology.}
  \flushright\sc  Rudy Rucker\footnote{\cite{Barrow}, p.~31.}\\
\end{quote}\index{Rucker, Rudy}

Despite its seeming complexity, all of standard mathematics, no matter how
deep or abstract, can amazingly enough be derived from a relatively small set
of axioms\index{axiom} or first principles. The development of these axioms is
among the most impressive and important accomplishments of mathematics in the
20th century. Ultimately, these axioms can be broken down into a set of rules
for manipulating symbols that any technically oriented person can follow.

We will not spend much time trying to convey a deep, higher-level
understanding of the meaning of the axioms. This kind of understanding
requires some mathematical sophistication as well as an understanding of the
philosophy underlying the foundations of mathematics and typically develops
over time as you work with mathematics.  Our goal, instead, is to give you the
immediate ability to follow how theorems\index{theorem} are derived from the
axioms and from other theorems.  This will be similar to learning the syntax
of a computer language, which lets you follow the details in a program but
does not necessarily give you the ability to write non-trivial programs on
your own, an ability that comes with practice. For now don't be alarmed by
abstract-sounding names of the axioms; just focus on the rules for
manipulating the symbols, which follow the simple conventions of the
Metamath\index{Metamath} language.

The axioms that underlie all of standard mathematics consist of axioms of logic
and axioms of set theory. The axioms of logic are divided into two
subcategories, propositional calculus\index{propositional calculus} (sometimes
called sentential logic\index{sentential logic}) and predicate calculus
(sometimes called first-order logic\index{first-order logic}\index{quantifier
theory}\index{predicate calculus} or quantifier theory).  Propositional
calculus is a prerequisite for predicate calculus, and predicate calculus is a
prerequisite for set theory.  The version of set theory most commonly used is
Zermelo--Fraenkel set theory\index{Zermelo--Fraenkel set theory}\index{set theory}
with the axiom of choice,
often abbreviated as ZFC\index{ZFC}.

Here in a nutshell is what the axioms are all about in an informal way. The
connection between this description and symbols we will show you won't be
immediately apparent and in principle needn't ever be.  Our description just
tries to summarize what mathematicians think about when they work with the
axioms.

Logic is a set of rules that allow us determine truths given other truths.
Put another way,
logic is more or less the translation of what we would consider common sense
into a rigorous set of axioms.\index{axioms of logic}  Suppose $\varphi$,
$\psi$, and $\chi$ (the Greek letters phi, psi, and chi) represent statements
that are either true or false, and $x$ is a variable\index{variable!in predicate
calculus} ranging over some group of mathematical objects (sets, integers,
real numbers, etc.). In mathematics, a ``statement'' really means a formula,
and $\psi$ could be for example ``$x = 2$.''
Propositional calculus\index{propositional calculus}
allows us to use variables that are either true or false
and make deductions such as
``if $\varphi$ implies $\psi$ and $\psi$ implies $\chi$, then $\varphi$
implies $\chi$.''
Predicate calculus\index{predicate calculus}
extends propositional calculus by also allowing us
to discuss statements about objects (not just true and false values), including
statements about ``all'' or ``at least one'' object.
For example, predicate calculus allows to say,
``if $\varphi$ is true for all $x$, then $\varphi$ is true for some $x$.''
The logic used in \texttt{set.mm} is standard classical logic
(as opposed to other logic systems like intuitionistic logic).

Set theory\index{set theory} has to do with the manipulation of objects and
collections of objects, specifically the abstract, imaginary objects that
mathematics deals with, such as numbers. Everything that is claimed to exist
in mathematics is considered to be a set.  A set called the empty
set\index{empty set} contains nothing.  We represent the empty set by
$\varnothing$.  Many sets can be built up from the empty set.  There is a set
represented by $\{\varnothing\}$ that contains the empty set, another set
represented by $\{\varnothing,\{\varnothing\}\}$ that contains this set as
well as the empty set, another set represented by $\{\{\varnothing\}\}$ that
contains just the set that contains the empty set, and so on ad infinitum. All
mathematical objects, no matter how complex, are defined as being identical to
certain sets: the integer\index{integer} 0 is defined as the empty set, the
integer 1 is defined as $\{\varnothing\}$, the integer 2 is defined as
$\{\varnothing,\{\varnothing\}\}$.  (How these definitions were chosen doesn't
matter now, but the idea behind it is that these sets have the properties we
expect of integers once suitable operations are defined.)  Mathematical
operations, such as addition, are defined in terms of operations on
sets---their union\index{set union}, intersection\index{set intersection}, and
so on---operations you may have used in elementary school when you worked
with groups of apples and oranges.

With a leap of faith, the axioms also postulate the existence of infinite
sets\index{infinite set}, such as the set of all non-negative integers ($0, 1,
2,\ldots$, also called ``natural numbers''\index{natural number}).  This set
can't be represented with the brace notation\index{brace notation} we just
showed you, but requires a more complicated notation called ``class
abstraction.''\index{class abstraction}\index{abstraction class}  For
example, the infinite set $\{ x |
\mbox{``$x$ is a natural number''} \} $ means the ``set of all objects $x$
such that $x$ is a natural number'' i.e.\ the set of natural numbers; here,
``$x$ is a natural number'' is a rather complicated formula when broken down
into the primitive symbols.\label{expandom}\footnote{The statement ``$x$ is a
natural number'' is formally expressed as ``$x \in \omega$,'' where $\in$
(stylized epsilon) means ``is in'' or ``is an element of'' and $\omega$
(omega) means ``the set of natural numbers.''  When ``$x\in\omega$'' is
completely expanded in terms of the primitive symbols of set theory, the
result is  $\lnot$ $($ $\lnot$ $($ $\forall$ $z$ $($ $\lnot$ $\forall$ $w$ $($
$z$ $\in$ $w$ $\rightarrow$ $\lnot$ $w$ $\in$ $x$ $)$ $\rightarrow$ $z$ $\in$
$x$ $)$ $\rightarrow$ $($ $\forall$ $z$ $($ $\lnot$ $($ $\forall$ $w$ $($ $w$
$\in$ $z$ $\rightarrow$ $w$ $\in$ $x$ $)$ $\rightarrow$ $\forall$ $w$ $\lnot$
$w$ $\in$ $z$ $)$ $\rightarrow$ $\lnot$ $\forall$ $w$ $($ $w$ $\in$ $z$
$\rightarrow$ $\lnot$ $\forall$ $v$ $($ $v$ $\in$ $z$ $\rightarrow$ $\lnot$
$v$ $\in$ $w$ $)$ $)$ $)$ $\rightarrow$ $\lnot$ $\forall$ $z$ $\forall$ $w$
$($ $\lnot$ $($ $z$ $\in$ $x$ $\rightarrow$ $\lnot$ $w$ $\in$ $x$ $)$
$\rightarrow$ $($ $\lnot$ $z$ $\in$ $w$ $\rightarrow$ $($ $\lnot$ $z$ $=$ $w$
$\rightarrow$ $w$ $\in$ $z$ $)$ $)$ $)$ $)$ $)$ $\rightarrow$ $\lnot$
$\forall$ $y$ $($ $\lnot$ $($ $\lnot$ $($ $\forall$ $z$ $($ $\lnot$ $\forall$
$w$ $($ $z$ $\in$ $w$ $\rightarrow$ $\lnot$ $w$ $\in$ $y$ $)$ $\rightarrow$
$z$ $\in$ $y$ $)$ $\rightarrow$ $($ $\forall$ $z$ $($ $\lnot$ $($ $\forall$
$w$ $($ $w$ $\in$ $z$ $\rightarrow$ $w$ $\in$ $y$ $)$ $\rightarrow$ $\forall$
$w$ $\lnot$ $w$ $\in$ $z$ $)$ $\rightarrow$ $\lnot$ $\forall$ $w$ $($ $w$
$\in$ $z$ $\rightarrow$ $\lnot$ $\forall$ $v$ $($ $v$ $\in$ $z$ $\rightarrow$
$\lnot$ $v$ $\in$ $w$ $)$ $)$ $)$ $\rightarrow$ $\lnot$ $\forall$ $z$
$\forall$ $w$ $($ $\lnot$ $($ $z$ $\in$ $y$ $\rightarrow$ $\lnot$ $w$ $\in$
$y$ $)$ $\rightarrow$ $($ $\lnot$ $z$ $\in$ $w$ $\rightarrow$ $($ $\lnot$ $z$
$=$ $w$ $\rightarrow$ $w$ $\in$ $z$ $)$ $)$ $)$ $)$ $\rightarrow$ $($
$\forall$ $z$ $\lnot$ $z$ $\in$ $y$ $\rightarrow$ $\lnot$ $\forall$ $w$ $($
$\lnot$ $($ $w$ $\in$ $y$ $\rightarrow$ $\lnot$ $\forall$ $z$ $($ $w$ $\in$
$z$ $\rightarrow$ $\lnot$ $z$ $\in$ $y$ $)$ $)$ $\rightarrow$ $\lnot$ $($
$\lnot$ $\forall$ $z$ $($ $w$ $\in$ $z$ $\rightarrow$ $\lnot$ $z$ $\in$ $y$
$)$ $\rightarrow$ $w$ $\in$ $y$ $)$ $)$ $)$ $)$ $\rightarrow$ $x$ $\in$ $y$
$)$ $)$ $)$. Section~\ref{hierarchy} shows the hierarchy of definitions that
leads up to this expression.}\index{stylized epsilon ($\in$)}\index{omega
($\omega$)}  Actually, the primitive symbols don't even include the brace
notation.  The brace notation is a high-level definition, which you can find in
Section~\ref{hierarchy}.

Interestingly, the arithmetic of integers\index{integer} and
rationals\index{rational number} can be developed without appealing to the
existence of an infinite set, whereas the arithmetic of real
numbers\index{real number} requires it.

Each variable\index{variable!in set theory} in the axioms of set theory
represents an arbitrary set, and the axioms specify the legal kinds of things
you can do with these variables at a very primitive level.

Now, you may think that numbers and arithmetic are a lot more intuitive and
fundamental than sets and therefore should be the foundation of mathematics.
What is really the case is that you've dealt with numbers all your life and
are comfortable with a few rules for manipulating them such as addition and
multiplication.  Those rules only cover a small portion of what can be done
with numbers and only a very tiny fraction of the rest of mathematics.  If you
look at any elementary book on number theory, you will quickly become lost if
these are the only rules that you know.  Even though such books may present a
list of ``axioms''\index{axiom} for arithmetic, the ability to use the axioms
and to understand proofs of theorems\index{theorem} (facts) about numbers
requires an implicit mathematical talent that frustrates many people
from studying abstract mathematics.  The kind of mathematics that most people
know limits them to the practical, everyday usage of blindly manipulating
numbers and formulas, without any understanding of why those rules are correct
nor any ability to go any further.  For example, do you know why multiplying
two negative numbers yields a positive number?  Starting with set theory, you
will also start off blindly manipulating symbols according to the rules we give
you, but with the advantage that these rules will allow you, in principle, to
access {\em all} of mathematics, not just a tiny part of it.

Of course, concrete examples are often helpful in the learning process. For
example, you can verify that $2\cdot 3=3 \cdot 2$ by actually grouping
objects and can easily ``see'' how it generalizes to $x\cdot y = y\cdot x$,
even though you might not be able to rigorously prove it.  Similarly, in set
theory it can be helpful to understand how the axioms of set theory apply to
(and are correct for) small finite collections of objects.  You should be aware
that in set theory intuition can be misleading for infinite collections, and
rigorous proofs become more important.  For example, while $x\cdot y = y\cdot
x$ is correct for finite ordinals (which are the natural numbers), it is not
usually true for infinite ordinals.

\section{The Axioms for All of Mathematics}

In this section\index{axioms for mathematics}, we will show you the axioms
for all of standard mathematics (i.e.\ logic and set theory) as they are
traditionally presented.  The traditional presentation is useful for someone
with the mathematical experience needed to correctly manipulate high-level
abstract concepts.  For someone without this talent, knowing how to actually
make use of these axioms can be difficult.  The purpose of this section is to
allow you to see how the version of the axioms used in the standard
Metamath\index{Metamath} database \texttt{set.mm}\index{set
theory database (\texttt{set.mm})} relates to  the typical version
in textbooks, and also to give you an informal feel for them.

\subsection{Propositional Calculus}

Propositional calculus\index{propositional calculus} concerns itself with
statements that can be interpreted as either true or false.  Some examples of
statements (outside of mathematics) that are either true or false are ``It is
raining today'' and ``The United States has a female president.'' In
mathematics, as we mentioned, statements are really formulas.

In propositional calculus, we don't care what the statements are.  We also
treat a logical combination of statements, such as ``It is raining today and
the United States has a female president,'' no differently from a single
statement.  Statements and their combinations are called well-formed formulas
(wffs)\index{well-formed formula (wff)}.  We define wffs only in terms of
other wffs and don't define what a ``starting'' wff is.  As is common practice
in the literature, we use Greek letters to represent wffs.

Specifically, suppose $\varphi$ and $\psi$ are wffs.  Then the combinations
$\varphi\rightarrow\psi$ (``$\varphi$ implies $\psi$,'' also read ``if
$\varphi$ then $\psi$'')\index{implication ($\rightarrow$)} and $\lnot\varphi$
(``not $\varphi$'')\index{negation ($\lnot$)} are also wffs.

The three axioms of propositional calculus\index{axioms of propositional
calculus} are all wffs of the following form:\footnote{A remarkable result of
C.~A.~Meredith\index{Meredith, C. A.} squeezes these three axioms into the
single axiom $((((\varphi\rightarrow \psi)\rightarrow(\neg \chi\rightarrow\neg
\theta))\rightarrow \chi )\rightarrow \tau)\rightarrow((\tau\rightarrow
\varphi)\rightarrow(\theta\rightarrow \varphi))$ \cite{CAMeredith},
which is believed to be the shortest possible.}
\begin{center}
     $\varphi\rightarrow(\psi\rightarrow \varphi)$\\

     $(\varphi\rightarrow (\psi\rightarrow \chi))\rightarrow
((\varphi\rightarrow  \psi)\rightarrow (\varphi\rightarrow \chi))$\\

     $(\neg \varphi\rightarrow \neg\psi)\rightarrow (\psi\rightarrow
\varphi)$
\end{center}

These three axioms are widely used.
They are attributed to Jan {\L}ukasiewicz
(pronounced woo-kah-SHAY-vitch) and was popularized by Alonzo Church,
who called it system P2. (Thanks to Ted Ulrich for this information.)

There are an infinite number of axioms, one for each possible
wff\index{well-formed formula (wff)} of the above form.  (For this reason,
axioms such as the above are often called ``axiom schemes.''\index{axiom
scheme})  Each Greek letter in the axioms may be substituted with a more
complex wff to result in another axiom.  For example, substituting
$\neg(\varphi\rightarrow\chi)$ for $\varphi$ in the first axiom yields
$\neg(\varphi\rightarrow\chi)\rightarrow(\psi\rightarrow
\neg(\varphi\rightarrow\chi))$, which is still an axiom.

To deduce new true statements (theorems\index{theorem}) from the axioms, a
rule\index{rule} called ``modus ponens''\index{modus ponens} is used.  This
rule states that if the wff $\varphi$ is an axiom or a theorem, and the wff
$\varphi\rightarrow\psi$ is an axiom or a theorem, then the wff $\psi$ is also
a theorem\index{theorem}.

As a non-mathematical example of modus ponens, suppose we have proved (or
taken as an axiom) ``Bob is a man'' and separately have proved (or taken as
an axiom) ``If Bob is a man, then Bob is a human.''  Using the rule of modus
ponens, we can logically deduce, ``Bob is a human.''

From Metamath's\index{Metamath} point of view, the axioms and the rule of
modus ponens just define a mechanical means for deducing new true statements
from existing true statements, and that is the complete content of
propositional calculus as far as Metamath is concerned.  You can read a logic
textbook to gain a better understanding of their meaning, or you can just let
their meaning slowly become apparent to you after you use them for a while.

It is actually rather easy to check to see if a formula is a theorem of
propositional calculus.  Theorems of propositional calculus are also called
``tautologies.''\index{tautology}  The technique to check whether a formula is
a tautology is called the ``truth table method,''\index{truth table} and it
works like this.  A wff $\varphi\rightarrow\psi$ is false whenever $\varphi$ is true
and $\psi$ is false.  Otherwise it is true.  A wff $\lnot\varphi$ is false
whenever $\varphi$ is true and false otherwise. To verify a tautology such as
$\varphi\rightarrow(\psi\rightarrow \varphi)$, you break it down into sub-wffs and
construct a truth table that accounts for all possible combinations of true
and false assigned to the wff metavariables:
\begin{center}\begin{tabular}{|c|c|c|c|}\hline
\mbox{$\varphi$} & \mbox{$\psi$} & \mbox{$\psi\rightarrow\varphi$}
    & \mbox{$\varphi\rightarrow(\psi\rightarrow \varphi)$} \\ \hline \hline
              T   &  T    &      T       &        T    \\ \hline
              T   &  F    &      T       &        T    \\ \hline
              F   &  T    &      F       &        T    \\ \hline
              F   &  F    &      T       &        T    \\ \hline
\end{tabular}\end{center}
If all entries in the last column are true, the formula is a tautology.

Now, the truth table method doesn't tell you how to prove the tautology from
the axioms, but only that a proof exists.  Finding an actual proof (especially
one that is short and elegant) can be challenging.  Methods do exist for
automatically generating proofs in propositional calculus, but the proofs that
result can sometimes be very long.  In the Metamath \texttt{set.mm}\index{set
theory database (\texttt{set.mm})} database, most
or all proofs were created manually.

Section \ref{metadefprop} discusses various definitions
that make propositional calculus easier to use.
For example, we define:

\begin{itemize}
\item $\varphi \vee \psi$
  is true if either $\varphi$ or $\psi$ (or both) are true
  (this is disjunction\index{disjunction ($\vee$)}
  aka logical {\sc or}\index{logical {\sc or} ($\vee$)}).

\item $\varphi \wedge \psi$
  is true if both $\varphi$ and $\psi$ are true
  (this is conjunction\index{conjunction ($\wedge$)}
  aka logical {\sc and}\index{logical {\sc and} ($\wedge$)}).

\item $\varphi \leftrightarrow \psi$
  is true if $\varphi$ and $\psi$ have the same value, that is,
  they are both true or both false
  (this is the biconditional\index{biconditional ($\leftrightarrow$)}).
\end{itemize}

\subsection{Predicate Calculus}

Predicate calculus\index{predicate calculus} introduces the concept of
``individual variables,''\index{variable!in predicate calculus}\index{individual
variable} which
we will usually just call ``variables.''
These variables can represent something other than true or false (wffs),
and will always represent sets when we get to set theory.  There are also
three new symbols $\forall$\index{universal quantifier ($\forall$)},
$=$\index{equality ($=$)}, and $\in$\index{stylized epsilon ($\in$)},
read ``for all,'' ``equals,'' and ``is an element of''
respectively.  We will represent variables with the letters $x$, $y$, $z$, and
$w$, as is common practice in the literature.
For example, $\forall x \varphi$ means ``for all possible values of
$x$, $\varphi$ is true.''

In predicate calculus, we extend the definition of a wff\index{well-formed
formula (wff)}.  If $\varphi$ is a wff and $x$ and $y$ are variables, then
$\forall x \, \varphi$, $x=y$, and $x\in y$ are wffs. Note that these three new
types of wffs can be considered ``starting'' wffs from which we can build
other wffs with $\rightarrow$ and $\neg$ .  The concept of a starting wff was
absent in propositional calculus.  But starting wff or not, all we are really
concerned with is whether our wffs are correctly constructed according to
these mechanical rules.

A quick aside:
To prevent confusion, it might be best at this point to think of the variables
of Metamath\index{Metamath} as ``metavariables,''\index{metavariable} because
they are not quite the same as the variables we are introducing here.  A
(meta)variable in Metamath can be a wff or an individual variable, as well
as many other things; in general, it represents a kind of place holder for an
unspecified sequence of math symbols\index{math symbol}.

Unlike propositional calculus, no decision procedure\index{decision procedure}
analogous to the truth table method exists (nor theoretically can exist) that
will definitely determine whether a formula is a theorem of predicate
calculus.  Much of the work in the field of automated theorem
proving\index{automated theorem proving} has been dedicated to coming up with
clever heuristics for proving theorems of predicate calculus, but they can
never be guaranteed to work always.

Section \ref{metadefpred} discusses various definitions
that make predicate calculus easier to use.
For example, we define
$\exists x \varphi$ to mean
``there exists at least one possible value of $x$ where $\varphi$ is true.''

We now turn to looking at how predicate calculus can be formally
represented.

\subsubsection{Common Axioms}

There is a new rule of inference in predicate calculus:  if $\varphi$ is
an axiom or a theorem, then $\forall x \,\varphi$ is also a
theorem\index{theorem}.  This is called the rule of
``generalization.''\index{rule of generalization}
This is easily represented in Metamath.

In standard texts of logic, there are often two axioms of predicate
calculus\index{axioms of predicate calculus}:
\begin{center}
  $\forall x \,\varphi ( x ) \rightarrow \varphi ( y )$,
      where ``$y$ is properly substituted for $x$.''\\
  $\forall x ( \varphi \rightarrow \psi )\rightarrow ( \varphi \rightarrow
    \forall x\, \psi )$,
    where ``$x$ is not free in $\varphi$.''
\end{center}

Now at first glance, this seems simple:  just two axioms.  However,
conditional clauses are attached to each axiom describing requirements that
may seem puzzling to you.  In addition, the first axiom puts a variable symbol
in parentheses after each wff, seemingly violating our definition of a
wff\index{well-formed formula (wff)}; this is just an informal way of
referring to some arbitrary variable that may occur in the wff.  The
conditional clauses do, of course, have a precise meaning, but as it turns out
the precise meaning is somewhat complicated and awkward to formalize in a
way that a computer can handle easily.  Unlike propositional calculus, a
certain amount of mathematical sophistication and practice is needed to be
able to easily grasp and manipulate these concepts correctly.

Predicate calculus may be presented with or without axioms for
equality\index{axioms of equality}\index{equality ($=$)}. We will require the
axioms of equality as a prerequisite for the version of set theory we will
use.  The axioms for equality, when included, are often represented using these
two axioms:
\begin{center}
$x=x$\\ \ \\
$x=y\rightarrow (\varphi(x,x)\rightarrow\varphi(x,y))$ where ``$\varphi(x,y)$
   arises from $\varphi(x,x)$ by replacing some, but not necessarily all,
   free\index{free variable}
   occurrences of $x$ by $y$,\\ provided that $y$ is free for $x$
   in $\varphi(x,x)$.'' \end{center}
% (Mendelson p. 95)
The first equality axiom is simple, but again,
the condition on the second one is
somewhat awkward to implement on a computer.

\subsubsection{Tarski System S2}

Of course, we are not the first to notice the complications of these
predicate calculus axioms when being rigorous.

Well-known logician Alfred Tarski published in 1965
a system he called system S2\cite[p.~77]{Tarski1965}.
Tarski's system is \textit{exactly equivalent} to the traditional textbook
formalization, but (by clever use of equality axioms) it eliminates the
latter's primitive notions of ``proper substitution'' and ``free variable,''
replacing them with direct substitution and the notion of a variable
not occurring in a formula (which we express with distinct variable
constraints).

In advocating his system, Tarski wrote, ``The relatively complicated
character of [free variables and proper substitution] is a source
of certain inconveniences of both practical and theoretical nature;
this is clearly experienced both in teaching an elementary course of
mathematical logic and in formalizing the syntax of predicate logic for
some theoretical purposes''\cite[p.~61]{Tarski1965}\index{Tarski, Alfred}.

\subsubsection{Developing a Metamath Representation}

The standard textbook axioms of predicate calculus are somewhat
cumbersome to implement on a computer because of the complex notions of
``free variable''\index{free variable} and ``proper
substitution.''\index{proper substitution}\index{substitution!proper}
While it is possible to use the Metamath\index{Metamath} language to
implement these concepts, we have chosen not to implement them
as primitive constructs in the
\texttt{set.mm} set theory database.  Instead, we have eliminated them
within the axioms
by carefully crafting the axioms so as to avoid them,
building on Tarski's system S2.  This makes it
easy for a beginner to follow the steps in a proof without knowing any
advanced concepts other than the simple concept of
replacing\index{substitution!variable}\index{variable substitution}
variables with expressions.

In order to develop the concepts of free variable and proper
substitution from the axioms, we use an additional
Metamath statement type called ``disjoint variable
restriction''\index{disjoint variables} that we have not encountered
before.  In the context of the axioms, the statement \texttt{\$d} $ x\,
y$\index{\texttt{\$d} statement} simply means that $x$ and $y$ must be
distinct\index{distinct variables}, i.e.\ they may not be simultaneously
substituted\index{substitution!variable}\index{variable substitution}
with the same variable.  The statement \texttt{\$d} $ x\, \varphi$ means
variable $x$ must not occur in wff $\varphi$.  For the precise
definition of \texttt{\$d}, see Section~\ref{dollard}.

\subsubsection{Metamath representation}

The Metamath axiom system for predicate calculus
defined in set.mm uses Tarski's system S2.
As noted above, this has a different representation
than the traditional textbook formalization,
but it is \textit{exactly equivalent} to the textbook formalization,
and it is \textit{much} easier to work with.
This is reproduced as system S3 in Section 6 of
Megill's formalization \cite{Megill}\index{Megill, Norman}.

There is one exception, Tarski's axiom of existence,
which we label as axiom ax-6.
In the case of ax-6, Tarski's version is weaker because it includes a
distinct variable proviso. If we wish, we can also weaken our version
in this way and still have a metalogically complete system. Theorem
ax6 shows this by deriving, in the presence of the other axioms, our
ax-6 from Tarski's weaker version ax6v. However, we chose the stronger
version for our system because it is simpler to state and easier to use.

Tarski's system was designed for proving specific theorems rather than
more general theorem schemes. However, theorem schemes are much more
efficient than specific theorems for building a body of mathematical
knowledge, since they can be reused with different instances as
needed. While Tarski does derive some theorem schemes from his axioms,
their proofs require concepts that are ``outside'' of the system, such as
induction on formula length. The verification of such proofs is difficult
to automate in a proof verifier. (Specifically, Tarski treats the formulas
of his system as set-theoretical objects. In order to verify the proofs
of his theorem schemes, a proof verifier would need a significant amount
of set theory built into it.)

The Metamath axiom system for predicate calculus extends
Tarski's system to eliminate this difficulty. The additional
``auxilliary'' axiom
schemes (as we will call them in this section; see below) endow Tarski's
system with a nice property we call
metalogical completeness \cite[Remark 9.6]{Megill}\index{Megill, Norman}.
As a result, we can prove any theorem scheme
expressable in the ``simple metalogic'' of Tarski's system by using
only Metamath's direct substitution rule applied to the axiom system
(and no other metalogical or set-theoretical notions ``outside'' of the
system). Simple metalogic consists of schemes containing wff metavariables
(with no arguments) and/or set (also called ``individual'') metavariables,
accompanied by optional provisos each stating that two specified set
metavariables must be distinct or that a specified set metavariable may
not occur in a specified wff metavariable. Metamath's logic and set theory
axiom and rule schemes are all examples of simple metalogic. The schemes
of traditional predicate calculus with equality are examples which are
not simple metalogic, because they use wff metavariables with arguments
and have ``free for'' and ``not free in'' side conditions.

A rigorous justification for this system, using an older but
exactly equivalent set of axioms, can be
found in \cite{Megill}\index{Megill, Norman}.

This allows us to
take a different approach in the Metamath\index{Metamath} database
\texttt{set.mm}\index{set theory database (\texttt{set.mm})}.  We do not
directly use the primitive notions of ``free variable''\index{free variable}
and ``proper substitution''\index{proper
substitution}\index{substitution!proper} at all as primitive constructs.
Instead, we use a set
of axioms that are almost as simple to manipulate as those of
propositional calculus.  Our axiom system avoids complex primitive
notions by effectively embedding the complexity into the axioms
themselves.  As a result, we will end up with a larger number of axioms,
but they are ideally suited for a computer language such as Metamath.
(Section~\ref{metaaxioms} shows these axioms.)

We will not elaborate further
on the ``free variable'' and ``proper substitution''
concepts here.  You may consult
\cite[ch.\ 3--4]{Hamilton}\index{Hamilton, Alan G.} (as well as
many other books) for a precise explanation
of these concepts.  If you intend to do serious mathematical work, it is wise
to become familiar with the traditional textbook approach; even though the
concepts embedded in their axioms require a higher level of sophistication,
they can be more practical to deal with on an everyday, informal basis.  Even
if you are just developing Metamath proofs, familiarity with the traditional
approach can help you arrive at a proof outline much faster, which you can
then convert to the detail required by Metamath.

We do develop proper substitution rules later on, but in set.mm
they are defined as derived constructs; they are not primitives.

You should also note that our system of predicate calculus is specifically
tailored for set theory; thus there are only two specific predicates $=$ and
$\in$ and no functions\index{function!in predicate calculus}
or constants\index{constant!in predicate calculus} unlike more general systems.
We later add these.

\subsection{Set Theory}

Traditional Zermelo--Fraenkel set theory\index{Zermelo--Fraenkel set
theory}\index{set theory} with the Axiom of Choice
has 10 axioms, which can be expressed in the
language of predicate calculus.  In this section, we will list only the
names and brief English descriptions of these axioms, since we will give
you the precise formulas used by the Metamath\index{Metamath} set theory
database \texttt{set.mm} later on.

In the descriptions of the axioms, we assume that $x$, $y$, $z$, $w$, and $v$
represent sets.  These are the same as the variables\index{variable!in set
theory} in our predicate calculus system above, except that now we informally
think of the variables as ranging over sets.  Note that the terms
``object,''\index{object} ``set,''\index{set} ``element,''\index{element}
``collection,''\index{collection} and ``family''\index{family} are synonymous,
as are ``is an element of,'' ``is a member of,''\index{member} ``is contained
in,'' and ``belongs to.''  The different terms are used for convenience; for
example, ``a collection of sets'' is less confusing than ``a set of sets.''
A set $x$ is said to be a ``subset''\index{subset} of $y$ if every element of
$x$ is also an element of $y$; we also say $x$ is ``included in''
$y$.

The axioms are very general and apply to almost any conceivable mathematical
object, and this level of abstraction can be overwhelming at first.  To gain an
intuitive feel, it can be helpful to draw a picture illustrating the concept;
for example, a circle containing dots could represent a collection of sets,
and a smaller circle drawn inside the circle could represent a subset.
Overlapping circles can illustrate intersection and union.  Circles that
illustrate the concepts of set theory are frequently used in elementary
textbooks and are called Venn diagrams\index{Venn diagram}.\index{axioms of
set theory}

1. Axiom of Extensionality:  Two sets are identical if they contain the same
   elements.\index{Axiom of Extensionality}

2. Axiom of Pairing:  The set $\{ x , y \}$ exists.\index{Axiom of Pairing}

3. Axiom of Power Sets:  The power set of a set (the collection of all of
   its subsets) exists.  For example, the power set of $\{x,y\}$ is
   $\{\varnothing,\{x\},\{y\},\{x,y\}\}$ and it exists.\index{Axiom
of Power Sets}

4. Axiom of the Null Set:  The empty set $\varnothing$ exists.\index{Axiom of
the Null Set}

5. Axiom of Union:  The union of a set (the set containing the elements of
   its members) exists.  For example, the union of $\{\{x,y\},\{z\}\}$ is
 $\{x,y,z\}$ and
   it exists.\index{Axiom of Union}

6. Axiom of Regularity:  Roughly, no set can contain itself, nor can there
   be membership ``loops,'' such as a set being an
   element of one of its members.\index{Axiom of Regularity}

7. Axiom of Infinity:  An infinite set exists.  An example of an infinite
   set is the set of all
   integers.\index{Axiom of Infinity}

8. Axiom of Separation:  The set exists that is obtained by restricting $x$
   with some property.  For example, if the set of all integers exists,
   then the set of all even integers exists.\index{Axiom of Separation}

9. Axiom of Replacement:  The range of a function whose domain is restricted
   to the elements of a set $x$, is also a set.  For example, there
   is a function
   from integers (the function's domain) to their squares (its
   range).  If we
   restrict the domain to even integers, its range will become the set of
   squares of even integers, so this axiom asserts that the set of
    squares of even numbers exists.  Technical note:  In general, the
   ``function'' need not be a set but can be a proper class.
   \index{Axiom of Replacement}

10. Axiom of Choice:  Let $x$ be a set whose members are pairwise
  disjoint\index{disjoint sets} (i.e,
  whose members contain no elements in common).  Then there exists another
  set containing one element from each member of $x$.  For
  example, if $x$ is
  $\{\{y,z\},\{w,v\}\}$, where $y$, $z$, $w$, and $v$ are
  different sets, then a set such as $\{z,w\}$
  exists (but the axiom doesn't tell
  us which one).  (Actually the Axiom
  of Choice is redundant if the set $x$, as in this example, has a finite
  number of elements.)\index{Axiom of Choice}

The Axiom of Choice is usually considered an extension of ZF set theory rather
than a proper part of it.  It is sometimes considered philosophically
controversial because it specifies the existence of a set without specifying
what the set is. Constructive logics, including intuitionistic logic,
do not accept the axiom of choice.
Since there is some lingering controversy, we often prefer proofs that do
not use the axiom of choice (where there is a known alternative), and
in some cases we will use weaker axioms than the full axiom of choice.
That said, the axiom of choice is a powerful and widely-accepted tool,
so we do use it when needed.
ZF set theory that includes the Axiom of Choice is
called Zermelo--Fraenkel set theory with choice (ZFC\index{ZFC set theory}).

When expressed symbolically, the Axiom of Separation and the Axiom of
Replacement contain wff symbols and therefore each represent infinitely many
axioms, one for each possible wff. For this reason, they are often called
axiom schemes\index{axiom scheme}\index{well-formed formula (wff)}.

It turns out that the Axiom of the Null Set, the Axiom of Pairing, and the
Axiom of Separation can be derived from the other axioms and are therefore
unnecessary, although they tend to be included in standard texts for various
reasons (historical, philosophical, and possibly because some authors may not
know this).  In the Metamath\index{Metamath} set theory database, these
redundant axioms are derived from the other ones instead of truly
being considered axioms.
This is in keeping with our general goal of minimizing the number of
axioms we must depend on.

\subsection{Other Axioms}

Above we qualified the phrase ``all of mathematics'' with ``essentially.''
The main important missing piece is the ability to do category theory,
which requires huge sets (inaccessible cardinals) larger than those
postulated by the ZFC axioms. The Tarski--Grothendieck Axiom postulates
the existence of such sets.
Note that this is the same axiom used by Mizar for supporting
category theory.
The Tarski--Grothendieck axiom
can be viewed as a very strong replacement of the Axiom of Infinity,
the Axiom of Choice, and the Axiom of Power Sets.
The \texttt{set.mm} database includes this axiom; see the database
for details about it.
Again, we only use this axiom when we need to.
You are only likely to encounter or use this axiom if you are doing
category theory, since its use is highly specialized,
so we will not list the Tarsky-Grothendieck axiom
in the short list of axioms below.

Can there be even more axioms?
Of course.
G\"{o}del showed that no finite set of axioms or axiom schemes can completely
describe any consistent theory strong enough to include arithmetic.
But practically speaking, the ones above are the accepted foundation that
almost all mathematicians explicitly or implicitly base their work on.

\section{The Axioms in the Metamath Language}\label{metaaxioms}

Here we list the axioms as they appear in
\texttt{set.mm}\index{set theory database (\texttt{set.mm})} so you can
look them up there easily.  Incidentally, the \texttt{show statement
/tex} command\index{\texttt{show statement} command} was used to
typeset them.

%macros from show statement /tex
\newbox\mlinebox
\newbox\mtrialbox
\newbox\startprefix  % Prefix for first line of a formula
\newbox\contprefix  % Prefix for continuation line of a formula
\def\startm{  % Initialize formula line
  \setbox\mlinebox=\hbox{\unhcopy\startprefix}
}
\def\m#1{  % Add a symbol to the formula
  \setbox\mtrialbox=\hbox{\unhcopy\mlinebox $\,#1$}
  \ifdim\wd\mtrialbox>\hsize
    \box\mlinebox
    \setbox\mlinebox=\hbox{\unhcopy\contprefix $\,#1$}
  \else
    \setbox\mlinebox=\hbox{\unhbox\mtrialbox}
  \fi
}
\def\endm{  % Output the last line of a formula
  \box\mlinebox
}

% \SLASH for \ , \TOR for \/ (text OR), \TAND for /\ (text and)
% This embeds a following forced space to force the space.
\newcommand\SLASH{\char`\\~}
\newcommand\TOR{\char`\\/~}
\newcommand\TAND{/\char`\\~}
%
% Macro to output metamath raw text.
% This assumes \startprefix and \contprefix are set.
% NOTE: "\" is tricky to escape, use \SLASH, \TOR, and \TAND inside.
% Any use of "$ { ~ ^" must be escaped; ~ and ^ must be escaped specially.
% We escape { and } for consistency.
% For more about how this macro written, see:
% https://stackoverflow.com/questions/4073674/
% how-to-disable-indentation-in-particular-section-in-latex/4075706
% Use frenchspacing, or "e." will get an extra space after it.
\newlength\mystoreparindent
\newlength\mystorehangindent
\newenvironment{mmraw}{%
\setlength{\mystoreparindent}{\the\parindent}
\setlength{\mystorehangindent}{\the\hangindent}
\setlength{\parindent}{0pt} % TODO - we'll put in the \startprefix instead
\setlength{\hangindent}{\wd\the\contprefix}
\begin{flushleft}
\begin{frenchspacing}
\begin{tt}
{\unhcopy\startprefix}%
}{%
\end{tt}
\end{frenchspacing}
\end{flushleft}
\setlength{\parindent}{\mystoreparindent}
\setlength{\hangindent}{\mystorehangindent}
\vskip 1ex
}

\needspace{5\baselineskip}
\subsection{Propositional Calculus}\label{propcalc}\index{axioms of
propositional calculus}

\needspace{2\baselineskip}
Axiom of Simplification.\label{ax1}

\setbox\startprefix=\hbox{\tt \ \ ax-1\ \$a\ }
\setbox\contprefix=\hbox{\tt \ \ \ \ \ \ \ \ \ \ }
\startm
\m{\vdash}\m{(}\m{\varphi}\m{\rightarrow}\m{(}\m{\psi}\m{\rightarrow}\m{\varphi}\m{)}
\m{)}
\endm

\needspace{3\baselineskip}
\noindent Axiom of Distribution.

\setbox\startprefix=\hbox{\tt \ \ ax-2\ \$a\ }
\setbox\contprefix=\hbox{\tt \ \ \ \ \ \ \ \ \ \ }
\startm
\m{\vdash}\m{(}\m{(}\m{\varphi}\m{\rightarrow}\m{(}\m{\psi}\m{\rightarrow}\m{\chi}
\m{)}\m{)}\m{\rightarrow}\m{(}\m{(}\m{\varphi}\m{\rightarrow}\m{\psi}\m{)}\m{
\rightarrow}\m{(}\m{\varphi}\m{\rightarrow}\m{\chi}\m{)}\m{)}\m{)}
\endm

\needspace{2\baselineskip}
\noindent Axiom of Contraposition.

\setbox\startprefix=\hbox{\tt \ \ ax-3\ \$a\ }
\setbox\contprefix=\hbox{\tt \ \ \ \ \ \ \ \ \ \ }
\startm
\m{\vdash}\m{(}\m{(}\m{\lnot}\m{\varphi}\m{\rightarrow}\m{\lnot}\m{\psi}\m{)}\m{
\rightarrow}\m{(}\m{\psi}\m{\rightarrow}\m{\varphi}\m{)}\m{)}
\endm


\needspace{4\baselineskip}
\noindent Rule of Modus Ponens.\label{axmp}\index{modus ponens}

\setbox\startprefix=\hbox{\tt \ \ min\ \$e\ }
\setbox\contprefix=\hbox{\tt \ \ \ \ \ \ \ \ \ }
\startm
\m{\vdash}\m{\varphi}
\endm

\setbox\startprefix=\hbox{\tt \ \ maj\ \$e\ }
\setbox\contprefix=\hbox{\tt \ \ \ \ \ \ \ \ \ }
\startm
\m{\vdash}\m{(}\m{\varphi}\m{\rightarrow}\m{\psi}\m{)}
\endm

\setbox\startprefix=\hbox{\tt \ \ ax-mp\ \$a\ }
\setbox\contprefix=\hbox{\tt \ \ \ \ \ \ \ \ \ \ \ }
\startm
\m{\vdash}\m{\psi}
\endm


\needspace{7\baselineskip}
\subsection{Axioms of Predicate Calculus with Equality---Tarski's S2}\index{axioms of predicate calculus}

\needspace{3\baselineskip}
\noindent Rule of Generalization.\index{rule of generalization}

\setbox\startprefix=\hbox{\tt \ \ ax-g.1\ \$e\ }
\setbox\contprefix=\hbox{\tt \ \ \ \ \ \ \ \ \ \ \ \ }
\startm
\m{\vdash}\m{\varphi}
\endm

\setbox\startprefix=\hbox{\tt \ \ ax-gen\ \$a\ }
\setbox\contprefix=\hbox{\tt \ \ \ \ \ \ \ \ \ \ \ \ }
\startm
\m{\vdash}\m{\forall}\m{x}\m{\varphi}
\endm

\needspace{2\baselineskip}
\noindent Axiom of Quantified Implication.

\setbox\startprefix=\hbox{\tt \ \ ax-4\ \$a\ }
\setbox\contprefix=\hbox{\tt \ \ \ \ \ \ \ \ \ \ }
\startm
\m{\vdash}\m{(}\m{\forall}\m{x}\m{(}\m{\forall}\m{x}\m{\varphi}\m{\rightarrow}\m{
\psi}\m{)}\m{\rightarrow}\m{(}\m{\forall}\m{x}\m{\varphi}\m{\rightarrow}\m{
\forall}\m{x}\m{\psi}\m{)}\m{)}
\endm

\needspace{3\baselineskip}
\noindent Axiom of Distinctness.

% Aka: Add $d x ph $.
\setbox\startprefix=\hbox{\tt \ \ ax-5\ \$a\ }
\setbox\contprefix=\hbox{\tt \ \ \ \ \ \ \ \ \ \ }
\startm
\m{\vdash}\m{(}\m{\varphi}\m{\rightarrow}\m{\forall}\m{x}\m{\varphi}\m{)}\m{where}\m{ }\m{\$d}\m{ }\m{x}\m{ }\m{\varphi}\m{ }\m{(}\m{x}\m{ }\m{does}\m{ }\m{not}\m{ }\m{occur}\m{ }\m{in}\m{ }\m{\varphi}\m{)}
\endm

\needspace{2\baselineskip}
\noindent Axiom of Existence.

\setbox\startprefix=\hbox{\tt \ \ ax-6\ \$a\ }
\setbox\contprefix=\hbox{\tt \ \ \ \ \ \ \ \ \ \ }
\startm
\m{\vdash}\m{(}\m{\forall}\m{x}\m{(}\m{x}\m{=}\m{y}\m{\rightarrow}\m{\forall}
\m{x}\m{\varphi}\m{)}\m{\rightarrow}\m{\varphi}\m{)}
\endm

\needspace{2\baselineskip}
\noindent Axiom of Equality.

\setbox\startprefix=\hbox{\tt \ \ ax-7\ \$a\ }
\setbox\contprefix=\hbox{\tt \ \ \ \ \ \ \ \ \ \ }
\startm
\m{\vdash}\m{(}\m{x}\m{=}\m{y}\m{\rightarrow}\m{(}\m{x}\m{=}\m{z}\m{
\rightarrow}\m{y}\m{=}\m{z}\m{)}\m{)}
\endm

\needspace{2\baselineskip}
\noindent Axiom of Left Equality for Binary Predicate.

\setbox\startprefix=\hbox{\tt \ \ ax-8\ \$a\ }
\setbox\contprefix=\hbox{\tt \ \ \ \ \ \ \ \ \ \ \ }
\startm
\m{\vdash}\m{(}\m{x}\m{=}\m{y}\m{\rightarrow}\m{(}\m{x}\m{\in}\m{z}\m{
\rightarrow}\m{y}\m{\in}\m{z}\m{)}\m{)}
\endm

\needspace{2\baselineskip}
\noindent Axiom of Right Equality for Binary Predicate.

\setbox\startprefix=\hbox{\tt \ \ ax-9\ \$a\ }
\setbox\contprefix=\hbox{\tt \ \ \ \ \ \ \ \ \ \ \ }
\startm
\m{\vdash}\m{(}\m{x}\m{=}\m{y}\m{\rightarrow}\m{(}\m{z}\m{\in}\m{x}\m{
\rightarrow}\m{z}\m{\in}\m{y}\m{)}\m{)}
\endm


\needspace{4\baselineskip}
\subsection{Axioms of Predicate Calculus with Equality---Auxiliary}\index{axioms of predicate calculus - auxiliary}

\needspace{2\baselineskip}
\noindent Axiom of Quantified Negation.

\setbox\startprefix=\hbox{\tt \ \ ax-10\ \$a\ }
\setbox\contprefix=\hbox{\tt \ \ \ \ \ \ \ \ \ \ }
\startm
\m{\vdash}\m{(}\m{\lnot}\m{\forall}\m{x}\m{\lnot}\m{\forall}\m{x}\m{\varphi}\m{
\rightarrow}\m{\varphi}\m{)}
\endm

\needspace{2\baselineskip}
\noindent Axiom of Quantifier Commutation.

\setbox\startprefix=\hbox{\tt \ \ ax-11\ \$a\ }
\setbox\contprefix=\hbox{\tt \ \ \ \ \ \ \ \ \ \ }
\startm
\m{\vdash}\m{(}\m{\forall}\m{x}\m{\forall}\m{y}\m{\varphi}\m{\rightarrow}\m{
\forall}\m{y}\m{\forall}\m{x}\m{\varphi}\m{)}
\endm

\needspace{3\baselineskip}
\noindent Axiom of Substitution.

\setbox\startprefix=\hbox{\tt \ \ ax-12\ \$a\ }
\setbox\contprefix=\hbox{\tt \ \ \ \ \ \ \ \ \ \ \ }
\startm
\m{\vdash}\m{(}\m{\lnot}\m{\forall}\m{x}\m{\,x}\m{=}\m{y}\m{\rightarrow}\m{(}
\m{x}\m{=}\m{y}\m{\rightarrow}\m{(}\m{\varphi}\m{\rightarrow}\m{\forall}\m{x}\m{(}
\m{x}\m{=}\m{y}\m{\rightarrow}\m{\varphi}\m{)}\m{)}\m{)}\m{)}
\endm

\needspace{3\baselineskip}
\noindent Axiom of Quantified Equality.

\setbox\startprefix=\hbox{\tt \ \ ax-13\ \$a\ }
\setbox\contprefix=\hbox{\tt \ \ \ \ \ \ \ \ \ \ \ }
\startm
\m{\vdash}\m{(}\m{\lnot}\m{\forall}\m{z}\m{\,z}\m{=}\m{x}\m{\rightarrow}\m{(}
\m{\lnot}\m{\forall}\m{z}\m{\,z}\m{=}\m{y}\m{\rightarrow}\m{(}\m{x}\m{=}\m{y}
\m{\rightarrow}\m{\forall}\m{z}\m{\,x}\m{=}\m{y}\m{)}\m{)}\m{)}
\endm

% \noindent Axiom of Quantifier Substitution
%
% \setbox\startprefix=\hbox{\tt \ \ ax-c11n\ \$a\ }
% \setbox\contprefix=\hbox{\tt \ \ \ \ \ \ \ \ \ \ \ }
% \startm
% \m{\vdash}\m{(}\m{\forall}\m{x}\m{\,x}\m{=}\m{y}\m{\rightarrow}\m{(}\m{\forall}
% \m{x}\m{\varphi}\m{\rightarrow}\m{\forall}\m{y}\m{\varphi}\m{)}\m{)}
% \endm
%
% \noindent Axiom of Distinct Variables. (This axiom requires
% that two individual variables
% be distinct\index{\texttt{\$d} statement}\index{distinct
% variables}.)
%
% \setbox\startprefix=\hbox{\tt \ \ \ \ \ \ \ \ \$d\ }
% \setbox\contprefix=\hbox{\tt \ \ \ \ \ \ \ \ \ \ \ }
% \startm
% \m{x}\m{\,}\m{y}
% \endm
%
% \setbox\startprefix=\hbox{\tt \ \ ax-c16\ \$a\ }
% \setbox\contprefix=\hbox{\tt \ \ \ \ \ \ \ \ \ \ \ }
% \startm
% \m{\vdash}\m{(}\m{\forall}\m{x}\m{\,x}\m{=}\m{y}\m{\rightarrow}\m{(}\m{\varphi}\m{
% \rightarrow}\m{\forall}\m{x}\m{\varphi}\m{)}\m{)}
% \endm

% \noindent Axiom of Quantifier Introduction (2).  (This axiom requires
% that the individual variable not occur in the
% wff\index{\texttt{\$d} statement}\index{distinct variables}.)
%
% \setbox\startprefix=\hbox{\tt \ \ \ \ \ \ \ \ \$d\ }
% \setbox\contprefix=\hbox{\tt \ \ \ \ \ \ \ \ \ \ \ }
% \startm
% \m{x}\m{\,}\m{\varphi}
% \endm
% \setbox\startprefix=\hbox{\tt \ \ ax-5\ \$a\ }
% \setbox\contprefix=\hbox{\tt \ \ \ \ \ \ \ \ \ \ \ }
% \startm
% \m{\vdash}\m{(}\m{\varphi}\m{\rightarrow}\m{\forall}\m{x}\m{\varphi}\m{)}
% \endm

\subsection{Set Theory}\label{mmsettheoryaxioms}

In order to make the axioms of set theory\index{axioms of set theory} a little
more compact, there are several definitions from logic that we make use of
implicitly, namely, ``logical {\sc and},''\index{conjunction ($\wedge$)}
\index{logical {\sc and} ($\wedge$)} ``logical equivalence,''\index{logical
equivalence ($\leftrightarrow$)}\index{biconditional ($\leftrightarrow$)} and
``there exists.''\index{existential quantifier ($\exists$)}

\begin{center}\begin{tabular}{rcl}
  $( \varphi \wedge \psi )$ &\mbox{stands for}& $\neg ( \varphi
     \rightarrow \neg \psi )$\\
  $( \varphi \leftrightarrow \psi )$& \mbox{stands
     for}& $( ( \varphi \rightarrow \psi ) \wedge
     ( \psi \rightarrow \varphi ) )$\\
  $\exists x \,\varphi$ &\mbox{stands for}& $\neg \forall x \neg \varphi$
\end{tabular}\end{center}

In addition, the axioms of set theory require that all variables be
dis\-tinct,\index{distinct variables}\footnote{Set theory axioms can be
devised so that {\em no} variables are required to be distinct,
provided we replace \texttt{ax-c16} with an axiom stating that ``at
least two things exist,'' thus
making \texttt{ax-5} the only other axiom requiring the
\texttt{\$d} statement.  These axioms are unconventional and are not
presented here, but they can be found on the \url{http://metamath.org}
web site.  See also the Comment on
p.~\pageref{nodd}.}\index{\texttt{\$d} statement} thus we also assume:
\begin{center}
  \texttt{\$d }$x\,y\,z\,w$
\end{center}

\needspace{2\baselineskip}
\noindent Axiom of Extensionality.\index{Axiom of Extensionality}

\setbox\startprefix=\hbox{\tt \ \ ax-ext\ \$a\ }
\setbox\contprefix=\hbox{\tt \ \ \ \ \ \ \ \ \ \ \ \ }
\startm
\m{\vdash}\m{(}\m{\forall}\m{x}\m{(}\m{x}\m{\in}\m{y}\m{\leftrightarrow}\m{x}
\m{\in}\m{z}\m{)}\m{\rightarrow}\m{y}\m{=}\m{z}\m{)}
\endm

\needspace{3\baselineskip}
\noindent Axiom of Replacement.\index{Axiom of Replacement}

\setbox\startprefix=\hbox{\tt \ \ ax-rep\ \$a\ }
\setbox\contprefix=\hbox{\tt \ \ \ \ \ \ \ \ \ \ \ \ }
\startm
\m{\vdash}\m{(}\m{\forall}\m{w}\m{\exists}\m{y}\m{\forall}\m{z}\m{(}\m{%
\forall}\m{y}\m{\varphi}\m{\rightarrow}\m{z}\m{=}\m{y}\m{)}\m{\rightarrow}\m{%
\exists}\m{y}\m{\forall}\m{z}\m{(}\m{z}\m{\in}\m{y}\m{\leftrightarrow}\m{%
\exists}\m{w}\m{(}\m{w}\m{\in}\m{x}\m{\wedge}\m{\forall}\m{y}\m{\varphi}\m{)}%
\m{)}\m{)}
\endm

\needspace{2\baselineskip}
\noindent Axiom of Union.\index{Axiom of Union}

\setbox\startprefix=\hbox{\tt \ \ ax-un\ \$a\ }
\setbox\contprefix=\hbox{\tt \ \ \ \ \ \ \ \ \ \ \ }
\startm
\m{\vdash}\m{\exists}\m{x}\m{\forall}\m{y}\m{(}\m{\exists}\m{x}\m{(}\m{y}\m{
\in}\m{x}\m{\wedge}\m{x}\m{\in}\m{z}\m{)}\m{\rightarrow}\m{y}\m{\in}\m{x}\m{)}
\endm

\needspace{2\baselineskip}
\noindent Axiom of Power Sets.\index{Axiom of Power Sets}

\setbox\startprefix=\hbox{\tt \ \ ax-pow\ \$a\ }
\setbox\contprefix=\hbox{\tt \ \ \ \ \ \ \ \ \ \ \ \ }
\startm
\m{\vdash}\m{\exists}\m{x}\m{\forall}\m{y}\m{(}\m{\forall}\m{x}\m{(}\m{x}\m{
\in}\m{y}\m{\rightarrow}\m{x}\m{\in}\m{z}\m{)}\m{\rightarrow}\m{y}\m{\in}\m{x}
\m{)}
\endm

\needspace{3\baselineskip}
\noindent Axiom of Regularity.\index{Axiom of Regularity}

\setbox\startprefix=\hbox{\tt \ \ ax-reg\ \$a\ }
\setbox\contprefix=\hbox{\tt \ \ \ \ \ \ \ \ \ \ \ \ }
\startm
\m{\vdash}\m{(}\m{\exists}\m{x}\m{\,x}\m{\in}\m{y}\m{\rightarrow}\m{\exists}
\m{x}\m{(}\m{x}\m{\in}\m{y}\m{\wedge}\m{\forall}\m{z}\m{(}\m{z}\m{\in}\m{x}\m{
\rightarrow}\m{\lnot}\m{z}\m{\in}\m{y}\m{)}\m{)}\m{)}
\endm

\needspace{3\baselineskip}
\noindent Axiom of Infinity.\index{Axiom of Infinity}

\setbox\startprefix=\hbox{\tt \ \ ax-inf\ \$a\ }
\setbox\contprefix=\hbox{\tt \ \ \ \ \ \ \ \ \ \ \ \ \ \ \ }
\startm
\m{\vdash}\m{\exists}\m{x}\m{(}\m{y}\m{\in}\m{x}\m{\wedge}\m{\forall}\m{y}%
\m{(}\m{y}\m{\in}\m{x}\m{\rightarrow}\m{\exists}\m{z}\m{(}\m{y}\m{\in}\m{z}\m{%
\wedge}\m{z}\m{\in}\m{x}\m{)}\m{)}\m{)}
\endm

\needspace{4\baselineskip}
\noindent Axiom of Choice.\index{Axiom of Choice}

\setbox\startprefix=\hbox{\tt \ \ ax-ac\ \$a\ }
\setbox\contprefix=\hbox{\tt \ \ \ \ \ \ \ \ \ \ \ \ \ \ }
\startm
\m{\vdash}\m{\exists}\m{x}\m{\forall}\m{y}\m{\forall}\m{z}\m{(}\m{(}\m{y}\m{%
\in}\m{z}\m{\wedge}\m{z}\m{\in}\m{w}\m{)}\m{\rightarrow}\m{\exists}\m{w}\m{%
\forall}\m{y}\m{(}\m{\exists}\m{w}\m{(}\m{(}\m{y}\m{\in}\m{z}\m{\wedge}\m{z}%
\m{\in}\m{w}\m{)}\m{\wedge}\m{(}\m{y}\m{\in}\m{w}\m{\wedge}\m{w}\m{\in}\m{x}%
\m{)}\m{)}\m{\leftrightarrow}\m{y}\m{=}\m{w}\m{)}\m{)}
\endm

\subsection{That's It}

There you have it, the axioms for (essentially) all of mathematics!
Wonder at them and stare at them in awe.  Put a copy in your wallet, and
you will carry in your pocket the encoding for all theorems ever proved
and that ever will be proved, from the most mundane to the most
profound.

\section{A Hierarchy of Definitions}\label{hierarchy}

The axioms in the previous section in principle embody everything that can be
done within standard mathematics.  However, it is impractical to accomplish
very much by using them directly, for even simple concepts (from a human
perspective) can involve extremely long, incomprehensible formulas.
Mathematics is made practical by introducing definitions\index{definition}.
Definitions usually introduce new symbols, or at least new relationships among
existing symbols, to abbreviate more complex formulas.  An important
requirement for a definition is that there exist a straightforward
(algorithmic) method for eliminating the abbreviation by expanding it into the
more primitive symbol string that it represents.  Some
important definitions included in
the file \texttt{set.mm} are listed in this section for reference, and also to
give you a feel for why something like $\omega$\index{omega ($\omega$)} (the
set of natural numbers\index{natural number} 0, 1, 2,\ldots) becomes very
complicated when completely expanded into primitive symbols.

What is the motivation for definitions, aside from allowing complicated
expressions to be expressed more simply?  In the case of  $\omega$, one goal is
to provide a basis for the theory of natural numbers.\index{natural number}
Before set theory was invented, a set of axioms for arithmetic, called Peano's
postulates\index{Peano's postulates}, was devised and shown to have the
properties one expects for natural numbers.  Now anyone can postulate a
set of axioms, but if the axioms are inconsistent contradictions can be derived
from them.  Once a contradiction is derived, anything can be trivially
proved, including
all the facts of arithmetic and their negations.  To ensure that an
axiom system is at least as reliable as the axioms for set theory, we can
define sets and operations on those sets that satisfy the new axioms. In the
\texttt{set.mm} Metamath database, we prove that the elements of $\omega$ satisfy
Peano's postulates, and it's a long and hard journey to get there directly
from the axioms of set theory.  But the result is confidence in the
foundations of arithmetic.  And there is another advantage:  we now have all
the tools of set theory at our disposal for manipulating objects that obey the
axioms for arithmetic.

What are the criteria we use for definitions?  First, and of utmost importance,
the definition should not be {\em creative}\index{creative
definition}\index{definition!creative}, that
is it should not allow an expression that previously qualified as a wff but
was not provable, to become provable.   Second, the definition should be {\em
eliminable}\index{definition!eliminability}, that is, there should exist an
algorithmic method for converting any expression using the definition into
a logically equivalent expression that previously qualified as a wff.

In almost all cases below, definitions connect two expressions with either
$\leftrightarrow$ or $=$.  Eliminating\footnote{Here we mean the
elimination that a human might do in his or her head.  To eliminate them as
part of a Metamath proof we would invoke one of a number of
theorems that deal with transitivity of equivalence or equality; there are
many such examples in the proofs in \texttt{set.mm}.} such a definition is a
simple matter of substituting the expression on the left-hand side ({\em
definiendum}\index{definiendum} or thing being defined) with the equivalent,
more primitive expression on the right-hand side ({\em
definiens}\index{definiens} or definition).

Often a definition has variables on the right-hand side which do not appear on
the left-hand side; these are called {\em dummy variables}.\index{dummy
variable!in definitions}  In this case, any
allowable substitution (such as a new, distinct
variable) can be used when the definition is eliminated.  Dummy variables may
be used only if they are {\em effectively bound}\index{effectively bound
variable}, meaning that the definition will remain logically equivalent upon
any substitution of a dummy variable with any other {\em qualifying
expression}\index{qualifying expression}, i.e.\ any symbol string (such as
another variable) that
meets the restrictions on the dummy variable imposed by \texttt{\$d} and
\texttt{\$f} statements.  For example, we could define a constant $\perp$
(inverted tee, meaning logical ``false'') as $( \varphi \wedge \lnot \varphi
)$, i.e.\ ``phi and not phi.''  Here $\varphi$ is effectively bound because the
definition remains logically equivalent when we replace $\varphi$ with any
other wff.  (It is actually \texttt{df-fal}
in \texttt{set.mm}, which defines $\perp$.)

There are two cases where eliminating definitions is a little more
complex.  These cases are the definitions \texttt{df-bi} and
\texttt{df-cleq}.  The first stretches the concept of a definition a
little, as in effect it ``defines a definition;'' however, it meets our
requirements for a definition in that it is eliminable and does not
strengthen the language.  Theorem \texttt{bii} shows the substitution
needed to eliminate the $\leftrightarrow$\index{logical equivalence
($\leftrightarrow$)}\index{biconditional ($\leftrightarrow$)} symbol.

Definition \texttt{df-cleq}\index{equality ($=$)} extends the usage of
the equality symbol to include ``classes''\index{class} in set theory.  The
reason it is potentially problematic is that it can lead to statements which
do not follow from logic alone but presuppose the Axiom of
Extensionality\index{Axiom of Extensionality}, so we include this axiom
as a hypothesis for the definition.  We could have made \texttt{df-cleq} directly
eliminable by introducing a new equality symbol, but have chosen not to do so
in keeping with standard textbook practice.  Definitions such as \texttt{df-cleq}
that extend the meaning of existing symbols must be introduced carefully so
that they do not lead to contradictions.  Definition \texttt{df-clel} also
extends the meaning of an existing symbol ($\in$); while it doesn't strengthen
the language like \texttt{df-cleq}, this is not obvious and it must also be
subject to the same scrutiny.

Exercise:  Study how the wff $x\in\omega$, meaning ``$x$ is a natural
number,'' could be expanded in terms of primitive symbols, starting with the
definitions \texttt{df-clel} on p.~\pageref{dfclel} and \texttt{df-om} on
p.~\pageref{dfom} and working your way back.  Don't bother to work out the
details; just make sure that you understand how you could do it in principle.
The answer is shown in the footnote on p.~\pageref{expandom}.  If you
actually do work it out, you won't get exactly the same answer because we used
a few simplifications such as discarding occurrences of $\lnot\lnot$ (double
negation).

In the definitions below, we have placed the {\sc ascii} Metamath source
below each of the formulas to help you become familiar with the
notation in the database.  For simplicity, the necessary \texttt{\$f}
and \texttt{\$d} statements are not shown.  If you are in doubt, use the
\texttt{show statement}\index{\texttt{show statement} command} command
in the Metamath program to see the full statement.
A selection of this notation is summarized in Appendix~\ref{ASCII}.

To understand the motivation for these definitions, you should consult the
references indicated:  Takeuti and Zaring \cite{Takeuti}\index{Takeuti, Gaisi},
Quine \cite{Quine}\index{Quine, Willard Van Orman}, Bell and Machover
\cite{Bell}\index{Bell, J. L.}, and Enderton \cite{Enderton}\index{Enderton,
Herbert B.}.  Our list of definitions is provided more for reference than as a
learning aid.  However, by looking at a few of them you can gain a feel for
how the hierarchy is built up.  The definitions are a representative sample of
the many definitions
in \texttt{set.mm}, but they are complete with respect to the
theorem examples we will present in Section~\ref{sometheorems}.  Also, some are
slightly different from, but logically equivalent to, the ones in \texttt{set.mm}
(some of which have been revised over time to shorten them, for example).

\subsection{Definitions for Propositional Calculus}\label{metadefprop}

The symbols $\varphi$, $\psi$, and $\chi$ represent wffs.

Our first definition introduces the biconditional
connective\footnote{The term ``connective'' is informally used to mean a
symbol that is placed between two variables or adjacent to a variable,
whereas a mathematical ``constant'' usually indicates a symbol such as
the number 0 that may replace a variable or metavariable.  From
Metamath's point of view, there is no distinction between a connective
and a constant; both are constants in the Metamath
language.}\index{connective}\index{constant} (also called logical
equivalence)\index{logical equivalence
($\leftrightarrow$)}\index{biconditional ($\leftrightarrow$)}.  Unlike
most traditional developments, we have chosen not to have a separate
symbol such as ``Df.'' to mean ``is defined as.''  Instead, we will use
the biconditional connective for this purpose, as it lets us use
logic to manipulate definitions directly.  Here we state the properties
of the biconditional connective with a carefully crafted \texttt{\$a}
statement, which effectively uses the biconditional connective to define
itself.  The $\leftrightarrow$ symbol can be eliminated from a formula
using theorem \texttt{bii}, which is derived later.

\vskip 2ex
\noindent Define the biconditional connective.\label{df-bi}

\vskip 0.5ex
\setbox\startprefix=\hbox{\tt \ \ df-bi\ \$a\ }
\setbox\contprefix=\hbox{\tt \ \ \ \ \ \ \ \ \ \ \ }
\startm
\m{\vdash}\m{\lnot}\m{(}\m{(}\m{(}\m{\varphi}\m{\leftrightarrow}\m{\psi}\m{)}%
\m{\rightarrow}\m{\lnot}\m{(}\m{(}\m{\varphi}\m{\rightarrow}\m{\psi}\m{)}\m{%
\rightarrow}\m{\lnot}\m{(}\m{\psi}\m{\rightarrow}\m{\varphi}\m{)}\m{)}\m{)}\m{%
\rightarrow}\m{\lnot}\m{(}\m{\lnot}\m{(}\m{(}\m{\varphi}\m{\rightarrow}\m{%
\psi}\m{)}\m{\rightarrow}\m{\lnot}\m{(}\m{\psi}\m{\rightarrow}\m{\varphi}\m{)}%
\m{)}\m{\rightarrow}\m{(}\m{\varphi}\m{\leftrightarrow}\m{\psi}\m{)}\m{)}\m{)}
\endm
\begin{mmraw}%
|- -. ( ( ( ph <-> ps ) -> -. ( ( ph -> ps ) ->
-. ( ps -> ph ) ) ) -> -. ( -. ( ( ph -> ps ) -> -. (
ps -> ph ) ) -> ( ph <-> ps ) ) ) \$.
\end{mmraw}

\noindent This theorem relates the biconditional connective to primitive
connectives and can be used to eliminate the $\leftrightarrow$ symbol from any
wff.

\vskip 0.5ex
\setbox\startprefix=\hbox{\tt \ \ bii\ \$p\ }
\setbox\contprefix=\hbox{\tt \ \ \ \ \ \ \ \ \ }
\startm
\m{\vdash}\m{(}\m{(}\m{\varphi}\m{\leftrightarrow}\m{\psi}\m{)}\m{\leftrightarrow}
\m{\lnot}\m{(}\m{(}\m{\varphi}\m{\rightarrow}\m{\psi}\m{)}\m{\rightarrow}\m{\lnot}
\m{(}\m{\psi}\m{\rightarrow}\m{\varphi}\m{)}\m{)}\m{)}
\endm
\begin{mmraw}%
|- ( ( ph <-> ps ) <-> -. ( ( ph -> ps ) -> -. ( ps -> ph ) ) ) \$= ... \$.
\end{mmraw}

\noindent Define disjunction ({\sc or}).\index{disjunction ($\vee$)}%
\index{logical or (vee)@logical {\sc or} ($\vee$)}%
\index{df-or@\texttt{df-or}}\label{df-or}

\vskip 0.5ex
\setbox\startprefix=\hbox{\tt \ \ df-or\ \$a\ }
\setbox\contprefix=\hbox{\tt \ \ \ \ \ \ \ \ \ \ \ }
\startm
\m{\vdash}\m{(}\m{(}\m{\varphi}\m{\vee}\m{\psi}\m{)}\m{\leftrightarrow}\m{(}\m{
\lnot}\m{\varphi}\m{\rightarrow}\m{\psi}\m{)}\m{)}
\endm
\begin{mmraw}%
|- ( ( ph \TOR ps ) <-> ( -. ph -> ps ) ) \$.
\end{mmraw}

\noindent Define conjunction ({\sc and}).\index{conjunction ($\wedge$)}%
\index{logical {\sc and} ($\wedge$)}%
\index{df-an@\texttt{df-an}}\label{df-an}

\vskip 0.5ex
\setbox\startprefix=\hbox{\tt \ \ df-an\ \$a\ }
\setbox\contprefix=\hbox{\tt \ \ \ \ \ \ \ \ \ \ \ }
\startm
\m{\vdash}\m{(}\m{(}\m{\varphi}\m{\wedge}\m{\psi}\m{)}\m{\leftrightarrow}\m{\lnot}
\m{(}\m{\varphi}\m{\rightarrow}\m{\lnot}\m{\psi}\m{)}\m{)}
\endm
\begin{mmraw}%
|- ( ( ph \TAND ps ) <-> -. ( ph -> -. ps ) ) \$.
\end{mmraw}

\noindent Define disjunction ({\sc or}) of 3 wffs.%
\index{df-3or@\texttt{df-3or}}\label{df-3or}

\vskip 0.5ex
\setbox\startprefix=\hbox{\tt \ \ df-3or\ \$a\ }
\setbox\contprefix=\hbox{\tt \ \ \ \ \ \ \ \ \ \ \ \ }
\startm
\m{\vdash}\m{(}\m{(}\m{\varphi}\m{\vee}\m{\psi}\m{\vee}\m{\chi}\m{)}\m{
\leftrightarrow}\m{(}\m{(}\m{\varphi}\m{\vee}\m{\psi}\m{)}\m{\vee}\m{\chi}\m{)}
\m{)}
\endm
\begin{mmraw}%
|- ( ( ph \TOR ps \TOR ch ) <-> ( ( ph \TOR ps ) \TOR ch ) ) \$.
\end{mmraw}

\noindent Define conjunction ({\sc and}) of 3 wffs.%
\index{df-3an}\label{df-3an}

\vskip 0.5ex
\setbox\startprefix=\hbox{\tt \ \ df-3an\ \$a\ }
\setbox\contprefix=\hbox{\tt \ \ \ \ \ \ \ \ \ \ \ \ }
\startm
\m{\vdash}\m{(}\m{(}\m{\varphi}\m{\wedge}\m{\psi}\m{\wedge}\m{\chi}\m{)}\m{
\leftrightarrow}\m{(}\m{(}\m{\varphi}\m{\wedge}\m{\psi}\m{)}\m{\wedge}\m{\chi}
\m{)}\m{)}
\endm

\begin{mmraw}%
|- ( ( ph \TAND ps \TAND ch ) <-> ( ( ph \TAND ps ) \TAND ch ) ) \$.
\end{mmraw}

\subsection{Definitions for Predicate Calculus}\label{metadefpred}

The symbols $x$, $y$, and $z$ represent individual variables of predicate
calculus.  In this section, they are not necessarily distinct unless it is
explicitly
mentioned.

\vskip 2ex
\noindent Define existential quantification.
The expression $\exists x \varphi$ means
``there exists an $x$ where $\varphi$ is true.''\index{existential quantifier
($\exists$)}\label{df-ex}

\vskip 0.5ex
\setbox\startprefix=\hbox{\tt \ \ df-ex\ \$a\ }
\setbox\contprefix=\hbox{\tt \ \ \ \ \ \ \ \ \ \ \ }
\startm
\m{\vdash}\m{(}\m{\exists}\m{x}\m{\varphi}\m{\leftrightarrow}\m{\lnot}\m{\forall}
\m{x}\m{\lnot}\m{\varphi}\m{)}
\endm
\begin{mmraw}%
|- ( E. x ph <-> -. A. x -. ph ) \$.
\end{mmraw}

\noindent Define proper substitution.\index{proper
substitution}\index{substitution!proper}\label{df-sb}
In our notation, we use $[ y / x ] \varphi$ to mean ``the wff that
results when $y$ is properly substituted for $x$ in the wff
$\varphi$.''\footnote{
This can also be described
as substituting $x$ with $y$, $y$ properly replaces $x$, or
$x$ is properly replaced by $y$.}
% This is elsb4, though it currently says: ( [ x / y ] z e. y <-> z e. x )
For example,
$[ y / x ] z \in x$ is the same as $z \in y$.
One way to remember this notation is to notice that it looks like division
and recall that $( y / x ) \cdot x $ is $y$ (when $x \neq 0$).
The notation is different from the notation $\varphi ( x | y )$
that is sometimes used, because the latter notation is ambiguous for us:
for example, we don't know whether $\lnot \varphi ( x | y )$ is to be
interpreted as $\lnot ( \varphi ( x | y ) )$ or
$( \lnot \varphi ) ( x | y )$.\footnote{Because of the way
we initially defined wffs, this is the case
with any postfix connective\index{postfix connective} (one occurring after the
symbols being connected) or infix connective\index{infix connective} (one
occurring between the symbols being connected).  Metamath does not have a
built-in notion of operator binding strength that could eliminate the
ambiguity.  The initial parenthesis effectively provides a prefix
connective\index{prefix connective} to eliminate ambiguity.  Some conventions,
such as Polish notation\index{Polish notation} used in the 1930's and 1940's
by Polish logicians, use only prefix connectives and thus allow the total
elimination of parentheses, at the expense of readability.  In Metamath we
could actually redefine all notation to be Polish if we wanted to without
having to change any proofs!}  Other texts often use $\varphi(y)$ to indicate
our $[ y / x ] \varphi$, but this notation is even more ambiguous since there is
no explicit indication of what is being substituted.
Note that this
definition is valid even when
$x$ and $y$ are the same variable.  The first conjunct is a ``trick'' used to
achieve this property, making the definition look somewhat peculiar at
first.

\vskip 0.5ex
\setbox\startprefix=\hbox{\tt \ \ df-sb\ \$a\ }
\setbox\contprefix=\hbox{\tt \ \ \ \ \ \ \ \ \ \ \ }
\startm
\m{\vdash}\m{(}\m{[}\m{y}\m{/}\m{x}\m{]}\m{\varphi}\m{\leftrightarrow}\m{(}%
\m{(}\m{x}\m{=}\m{y}\m{\rightarrow}\m{\varphi}\m{)}\m{\wedge}\m{\exists}\m{x}%
\m{(}\m{x}\m{=}\m{y}\m{\wedge}\m{\varphi}\m{)}\m{)}\m{)}
\endm
\begin{mmraw}%
|- ( [ y / x ] ph <-> ( ( x = y -> ph ) \TAND E. x ( x = y \TAND ph ) ) ) \$.
\end{mmraw}


\noindent Define existential uniqueness\index{existential uniqueness
quantifier ($\exists "!$)} (``there exists exactly one'').  Note that $y$ is a
variable distinct from $x$ and not occurring in $\varphi$.\label{df-eu}

\vskip 0.5ex
\setbox\startprefix=\hbox{\tt \ \ df-eu\ \$a\ }
\setbox\contprefix=\hbox{\tt \ \ \ \ \ \ \ \ \ \ \ }
\startm
\m{\vdash}\m{(}\m{\exists}\m{{!}}\m{x}\m{\varphi}\m{\leftrightarrow}\m{\exists}
\m{y}\m{\forall}\m{x}\m{(}\m{\varphi}\m{\leftrightarrow}\m{x}\m{=}\m{y}\m{)}\m{)}
\endm

\begin{mmraw}%
|- ( E! x ph <-> E. y A. x ( ph <-> x = y ) ) \$.
\end{mmraw}

\subsection{Definitions for Set Theory}\label{setdefinitions}

The symbols $x$, $y$, $z$, and $w$ represent individual variables of
predicate calculus, which in set theory are understood to be sets.
However, using only the constructs shown so far would be very inconvenient.

To make set theory more practical, we introduce the notion of a ``class.''
A class\index{class} is either a set variable (such as $x$) or an
expression of the form $\{ x | \varphi\}$ (called an ``abstraction
class''\index{abstraction class}\index{class abstraction}).  Note that
sets (i.e.\ individual variables) always exist (this is a theorem of
logic, namely $\exists y \, y = x$ for any set $x$), whereas classes may
or may not exist (i.e.\ $\exists y \, y = A$ may or may not be true).
If a class does not exist it is called a ``proper class.''\index{proper
class}\index{class!proper} Definitions \texttt{df-clab},
\texttt{df-cleq}, and \texttt{df-clel} can be used to convert an
expression containing classes into one containing only set variables and
wff metavariables.

The symbols $A$, $B$, $C$, $D$, $ F$, $G$, and $R$ are metavariables that range
over classes.  A class metavariable $A$ may be eliminated from a wff by
replacing it with $\{ x|\varphi\}$ where neither $x$ nor $\varphi$ occur in
the wff.

The theory of classes can be shown to be an eliminable and conservative
extension of set theory. The \textbf{eliminability}
property shows that for every
formula in the extended language we can build a logically equivalent
formula in the basic language; so that even if the extended language
provides more ease to convey and formulate mathematical ideas for set
theory, its expressive power does not in fact strengthen the basic
language's expressive power.
The \textbf{conservation} property shows that for
every proof of a formula of the basic language in the extended system
we can build another proof of the same formula in the basic system;
so that, concerning theorems on sets only, the deductive powers of
the extended system and of the basic system are identical. Together,
these properties mean that the extended language can be treated as a
definitional extension that is \textbf{sound}.

A rigorous justification, which we will not give here, can be found in
Levy \cite[pp.~357-366]{Levy} supplementing his informal introduction to class
theory on pp.~7-17. Two other good treatments of class theory are provided
by Quine \cite[pp.~15-21]{Quine}\index{Quine, Willard Van Orman}
and also \cite[pp.~10-14]{Takeuti}\index{Takeuti, Gaisi}.
Quine's exposition (he calls them virtual classes)
is nicely written and very readable.

In the rest of this
section, individual variables are always assumed to be distinct from
each other unless otherwise indicated.  In addition, dummy variables on the
right-hand side of a definition do not occur in the class and wff
metavariables in the definition.

The definitions we present here are a partial but self-contained
collection selected from several hundred that appear in the current
\texttt{set.mm} database.  They are adequate for a basic development of
elementary set theory.

\vskip 2ex
\noindent Define the abstraction class.\index{abstraction class}\index{class
abstraction}\label{df-clab}  $x$ and $y$
need not be distinct.  Definition 2.1 of Quine, p.~16.  This definition may
seem puzzling since it is shorter than the expression being defined and does not
buy us anything in terms of brevity.  The reason we introduce this definition
is because it fits in neatly with the extension of the $\in$ connective
provided by \texttt{df-clel}.

\vskip 0.5ex
\setbox\startprefix=\hbox{\tt \ \ df-clab\ \$a\ }
\setbox\contprefix=\hbox{\tt \ \ \ \ \ \ \ \ \ \ \ \ \ }
\startm
\m{\vdash}\m{(}\m{x}\m{\in}\m{\{}\m{y}\m{|}\m{\varphi}\m{\}}\m{%
\leftrightarrow}\m{[}\m{x}\m{/}\m{y}\m{]}\m{\varphi}\m{)}
\endm
\begin{mmraw}%
|- ( x e. \{ y | ph \} <-> [ x / y ] ph ) \$.
\end{mmraw}

\noindent Define the equality connective between classes\index{class
equality}\label{df-cleq}.  See Quine or Chapter 4 of Takeuti and Zaring for its
justification and methods for eliminating it.  This is an example of a
somewhat ``dangerous'' definition, because it extends the use of the
existing equality symbol rather than introducing a new symbol, allowing
us to make statements in the original language that may not be true.
For example, it permits us to deduce $y = z \leftrightarrow \forall x (
x \in y \leftrightarrow x \in z )$ which is not a theorem of logic but
rather presupposes the Axiom of Extensionality,\index{Axiom of
Extensionality} which we include as a hypothesis so that we can know
when this axiom is assumed in a proof (with the \texttt{show
trace{\char`\_}back} command).  We could avoid the danger by introducing
another symbol, say $\eqcirc$, in place of $=$; this
would also have the advantage of making elimination of the definition
straightforward and would eliminate the need for Extensionality as a
hypothesis.  We would then also have the advantage of being able to
identify exactly where Extensionality truly comes into play.  One of our
theorems would be $x \eqcirc y \leftrightarrow x = y$
by invoking Extensionality.  However in keeping with standard practice
we retain the ``dangerous'' definition.

\vskip 0.5ex
\setbox\startprefix=\hbox{\tt \ \ df-cleq.1\ \$e\ }
\setbox\contprefix=\hbox{\tt \ \ \ \ \ \ \ \ \ \ \ \ \ \ \ }
\startm
\m{\vdash}\m{(}\m{\forall}\m{x}\m{(}\m{x}\m{\in}\m{y}\m{\leftrightarrow}\m{x}
\m{\in}\m{z}\m{)}\m{\rightarrow}\m{y}\m{=}\m{z}\m{)}
\endm
\setbox\startprefix=\hbox{\tt \ \ df-cleq\ \$a\ }
\setbox\contprefix=\hbox{\tt \ \ \ \ \ \ \ \ \ \ \ \ \ }
\startm
\m{\vdash}\m{(}\m{A}\m{=}\m{B}\m{\leftrightarrow}\m{\forall}\m{x}\m{(}\m{x}\m{
\in}\m{A}\m{\leftrightarrow}\m{x}\m{\in}\m{B}\m{)}\m{)}
\endm
% We need to reset the startprefix and contprefix.
\setbox\startprefix=\hbox{\tt \ \ df-cleq.1\ \$e\ }
\setbox\contprefix=\hbox{\tt \ \ \ \ \ \ \ \ \ \ \ \ \ \ \ }
\begin{mmraw}%
|- ( A. x ( x e. y <-> x e. z ) -> y = z ) \$.
\end{mmraw}
\setbox\startprefix=\hbox{\tt \ \ df-cleq\ \$a\ }
\setbox\contprefix=\hbox{\tt \ \ \ \ \ \ \ \ \ \ \ \ \ }
\begin{mmraw}%
|- ( A = B <-> A. x ( x e. A <-> x e. B ) ) \$.
\end{mmraw}

\noindent Define the membership connective between classes\index{class
membership}.  Theorem 6.3 of Quine, p.~41, which we adopt as a definition.
Note that it extends the use of the existing membership symbol, but unlike
{\tt df-cleq} it does not extend the set of valid wffs of logic when the class
metavariables are replaced with set variables.\label{dfclel}\label{df-clel}

\vskip 0.5ex
\setbox\startprefix=\hbox{\tt \ \ df-clel\ \$a\ }
\setbox\contprefix=\hbox{\tt \ \ \ \ \ \ \ \ \ \ \ \ \ }
\startm
\m{\vdash}\m{(}\m{A}\m{\in}\m{B}\m{\leftrightarrow}\m{\exists}\m{x}\m{(}\m{x}
\m{=}\m{A}\m{\wedge}\m{x}\m{\in}\m{B}\m{)}\m{)}
\endm
\begin{mmraw}%
|- ( A e. B <-> E. x ( x = A \TAND x e. B ) ) \$.?
\end{mmraw}

\noindent Define inequality.

\vskip 0.5ex
\setbox\startprefix=\hbox{\tt \ \ df-ne\ \$a\ }
\setbox\contprefix=\hbox{\tt \ \ \ \ \ \ \ \ \ \ \ }
\startm
\m{\vdash}\m{(}\m{A}\m{\ne}\m{B}\m{\leftrightarrow}\m{\lnot}\m{A}\m{=}\m{B}%
\m{)}
\endm
\begin{mmraw}%
|- ( A =/= B <-> -. A = B ) \$.
\end{mmraw}

\noindent Define restricted universal quantification.\index{universal
quantifier ($\forall$)!restricted}  Enderton, p.~22.

\vskip 0.5ex
\setbox\startprefix=\hbox{\tt \ \ df-ral\ \$a\ }
\setbox\contprefix=\hbox{\tt \ \ \ \ \ \ \ \ \ \ \ \ }
\startm
\m{\vdash}\m{(}\m{\forall}\m{x}\m{\in}\m{A}\m{\varphi}\m{\leftrightarrow}\m{%
\forall}\m{x}\m{(}\m{x}\m{\in}\m{A}\m{\rightarrow}\m{\varphi}\m{)}\m{)}
\endm
\begin{mmraw}%
|- ( A. x e. A ph <-> A. x ( x e. A -> ph ) ) \$.
\end{mmraw}

\noindent Define restricted existential quantification.\index{existential
quantifier ($\exists$)!restricted}  Enderton, p.~22.

\vskip 0.5ex
\setbox\startprefix=\hbox{\tt \ \ df-rex\ \$a\ }
\setbox\contprefix=\hbox{\tt \ \ \ \ \ \ \ \ \ \ \ \ }
\startm
\m{\vdash}\m{(}\m{\exists}\m{x}\m{\in}\m{A}\m{\varphi}\m{\leftrightarrow}\m{%
\exists}\m{x}\m{(}\m{x}\m{\in}\m{A}\m{\wedge}\m{\varphi}\m{)}\m{)}
\endm
\begin{mmraw}%
|- ( E. x e. A ph <-> E. x ( x e. A \TAND ph ) ) \$.
\end{mmraw}

\noindent Define the universal class\index{universal class ($V$)}.  Definition
5.20, p.~21, of Takeuti and Zaring.\label{df-v}

\vskip 0.5ex
\setbox\startprefix=\hbox{\tt \ \ df-v\ \$a\ }
\setbox\contprefix=\hbox{\tt \ \ \ \ \ \ \ \ \ \ }
\startm
\m{\vdash}\m{{\rm V}}\m{=}\m{\{}\m{x}\m{|}\m{x}\m{=}\m{x}\m{\}}
\endm
\begin{mmraw}%
|- {\char`\_}V = \{ x | x = x \} \$.
\end{mmraw}

\noindent Define the subclass\index{subclass}\index{subset} relationship
between two classes (called the subset relation if the classes are sets i.e.\
are not proper).  Definition 5.9 of Takeuti and Zaring, p.~17.\label{df-ss}

\vskip 0.5ex
\setbox\startprefix=\hbox{\tt \ \ df-ss\ \$a\ }
\setbox\contprefix=\hbox{\tt \ \ \ \ \ \ \ \ \ \ \ }
\startm
\m{\vdash}\m{(}\m{A}\m{\subseteq}\m{B}\m{\leftrightarrow}\m{\forall}\m{x}\m{(}
\m{x}\m{\in}\m{A}\m{\rightarrow}\m{x}\m{\in}\m{B}\m{)}\m{)}
\endm
\begin{mmraw}%
|- ( A C\_ B <-> A. x ( x e. A -> x e. B ) ) \$.
\end{mmraw}

\noindent Define the union\index{union} of two classes.  Definition 5.6 of Takeuti and Zaring,
p.~16.\label{df-un}

\vskip 0.5ex
\setbox\startprefix=\hbox{\tt \ \ df-un\ \$a\ }
\setbox\contprefix=\hbox{\tt \ \ \ \ \ \ \ \ \ \ \ }
\startm
\m{\vdash}\m{(}\m{A}\m{\cup}\m{B}\m{)}\m{=}\m{\{}\m{x}\m{|}\m{(}\m{x}\m{\in}
\m{A}\m{\vee}\m{x}\m{\in}\m{B}\m{)}\m{\}}
\endm
\begin{mmraw}%
( A u. B ) = \{ x | ( x e. A \TOR x e. B ) \} \$.
\end{mmraw}

\noindent Define the intersection\index{intersection} of two classes.  Definition 5.6 of
Takeuti and Zaring, p.~16.\label{df-in}

\vskip 0.5ex
\setbox\startprefix=\hbox{\tt \ \ df-in\ \$a\ }
\setbox\contprefix=\hbox{\tt \ \ \ \ \ \ \ \ \ \ \ }
\startm
\m{\vdash}\m{(}\m{A}\m{\cap}\m{B}\m{)}\m{=}\m{\{}\m{x}\m{|}\m{(}\m{x}\m{\in}
\m{A}\m{\wedge}\m{x}\m{\in}\m{B}\m{)}\m{\}}
\endm
% Caret ^ requires special treatment
\begin{mmraw}%
|- ( A i\^{}i B ) = \{ x | ( x e. A \TAND x e. B ) \} \$.
\end{mmraw}

\noindent Define class difference\index{class difference}\index{set difference}.
Definition 5.12 of Takeuti and Zaring, p.~20.  Several notations are used in
the literature; we chose the $\setminus$ convention instead of a minus sign to
reserve the latter for later use in, e.g., arithmetic.\label{df-dif}

\vskip 0.5ex
\setbox\startprefix=\hbox{\tt \ \ df-dif\ \$a\ }
\setbox\contprefix=\hbox{\tt \ \ \ \ \ \ \ \ \ \ \ \ }
\startm
\m{\vdash}\m{(}\m{A}\m{\setminus}\m{B}\m{)}\m{=}\m{\{}\m{x}\m{|}\m{(}\m{x}\m{
\in}\m{A}\m{\wedge}\m{\lnot}\m{x}\m{\in}\m{B}\m{)}\m{\}}
\endm
\begin{mmraw}%
( A \SLASH B ) = \{ x | ( x e. A \TAND -. x e. B ) \} \$.
\end{mmraw}

\noindent Define the empty or null set\index{empty set}\index{null set}.
Compare  Definition 5.14 of Takeuti and Zaring, p.~20.\label{df-nul}

\vskip 0.5ex
\setbox\startprefix=\hbox{\tt \ \ df-nul\ \$a\ }
\setbox\contprefix=\hbox{\tt \ \ \ \ \ \ \ \ \ \ }
\startm
\m{\vdash}\m{\varnothing}\m{=}\m{(}\m{{\rm V}}\m{\setminus}\m{{\rm V}}\m{)}
\endm
\begin{mmraw}%
|- (/) = ( {\char`\_}V \SLASH {\char`\_}V ) \$.
\end{mmraw}

\noindent Define power class\index{power set}\index{power class}.  Definition 5.10 of
Takeuti and Zaring, p.~17, but we also let it apply to proper classes.  (Note
that \verb$~P$ is the symbol for calligraphic P, the tilde
suggesting ``curly;'' see Appendix~\ref{ASCII}.)\label{df-pw}

\vskip 0.5ex
\setbox\startprefix=\hbox{\tt \ \ df-pw\ \$a\ }
\setbox\contprefix=\hbox{\tt \ \ \ \ \ \ \ \ \ \ \ }
\startm
\m{\vdash}\m{{\cal P}}\m{A}\m{=}\m{\{}\m{x}\m{|}\m{x}\m{\subseteq}\m{A}\m{\}}
\endm
% Special incantation required to put ~ into the text
\begin{mmraw}%
|- \char`\~P~A = \{ x | x C\_ A \} \$.
\end{mmraw}

\noindent Define the singleton of a class\index{singleton}.  Definition 7.1 of
Quine, p.~48.  It is well-defined for proper classes, although
it is not very meaningful in this case, where it evaluates to the empty
set.

\vskip 0.5ex
\setbox\startprefix=\hbox{\tt \ \ df-sn\ \$a\ }
\setbox\contprefix=\hbox{\tt \ \ \ \ \ \ \ \ \ \ \ }
\startm
\m{\vdash}\m{\{}\m{A}\m{\}}\m{=}\m{\{}\m{x}\m{|}\m{x}\m{=}\m{A}\m{\}}
\endm
\begin{mmraw}%
|- \{ A \} = \{ x | x = A \} \$.
\end{mmraw}%

\noindent Define an unordered pair of classes\index{unordered pair}\index{pair}.  Definition
7.1 of Quine, p.~48.

\vskip 0.5ex
\setbox\startprefix=\hbox{\tt \ \ df-pr\ \$a\ }
\setbox\contprefix=\hbox{\tt \ \ \ \ \ \ \ \ \ \ \ }
\startm
\m{\vdash}\m{\{}\m{A}\m{,}\m{B}\m{\}}\m{=}\m{(}\m{\{}\m{A}\m{\}}\m{\cup}\m{\{}
\m{B}\m{\}}\m{)}
\endm
\begin{mmraw}%
|- \{ A , B \} = ( \{ A \} u. \{ B \} ) \$.
\end{mmraw}

\noindent Define an unordered triple of classes\index{unordered triple}.  Definition of
Enderton, p.~19.

\vskip 0.5ex
\setbox\startprefix=\hbox{\tt \ \ df-tp\ \$a\ }
\setbox\contprefix=\hbox{\tt \ \ \ \ \ \ \ \ \ \ \ }
\startm
\m{\vdash}\m{\{}\m{A}\m{,}\m{B}\m{,}\m{C}\m{\}}\m{=}\m{(}\m{\{}\m{A}\m{,}\m{B}
\m{\}}\m{\cup}\m{\{}\m{C}\m{\}}\m{)}
\endm
\begin{mmraw}%
|- \{ A , B , C \} = ( \{ A , B \} u. \{ C \} ) \$.
\end{mmraw}%

\noindent Kuratowski's\index{Kuratowski, Kazimierz} ordered pair\index{ordered
pair} definition.  Definition 9.1 of Quine, p.~58. For proper classes it is
not meaningful but is well-defined for convenience.  (Note that \verb$<.$
stands for $\langle$ whereas \verb$<$ stands for $<$, and similarly for
\verb$>.$\,.)\label{df-op}

\vskip 0.5ex
\setbox\startprefix=\hbox{\tt \ \ df-op\ \$a\ }
\setbox\contprefix=\hbox{\tt \ \ \ \ \ \ \ \ \ \ \ }
\startm
\m{\vdash}\m{\langle}\m{A}\m{,}\m{B}\m{\rangle}\m{=}\m{\{}\m{\{}\m{A}\m{\}}
\m{,}\m{\{}\m{A}\m{,}\m{B}\m{\}}\m{\}}
\endm
\begin{mmraw}%
|- <. A , B >. = \{ \{ A \} , \{ A , B \} \} \$.
\end{mmraw}

\noindent Define the union of a class\index{union}.  Definition 5.5, p.~16,
of Takeuti and Zaring.

\vskip 0.5ex
\setbox\startprefix=\hbox{\tt \ \ df-uni\ \$a\ }
\setbox\contprefix=\hbox{\tt \ \ \ \ \ \ \ \ \ \ \ \ }
\startm
\m{\vdash}\m{\bigcup}\m{A}\m{=}\m{\{}\m{x}\m{|}\m{\exists}\m{y}\m{(}\m{x}\m{
\in}\m{y}\m{\wedge}\m{y}\m{\in}\m{A}\m{)}\m{\}}
\endm
\begin{mmraw}%
|- U. A = \{ x | E. y ( x e. y \TAND y e. A ) \} \$.
\end{mmraw}

\noindent Define the intersection\index{intersection} of a class.  Definition 7.35,
p.~44, of Takeuti and Zaring.

\vskip 0.5ex
\setbox\startprefix=\hbox{\tt \ \ df-int\ \$a\ }
\setbox\contprefix=\hbox{\tt \ \ \ \ \ \ \ \ \ \ \ \ }
\startm
\m{\vdash}\m{\bigcap}\m{A}\m{=}\m{\{}\m{x}\m{|}\m{\forall}\m{y}\m{(}\m{y}\m{
\in}\m{A}\m{\rightarrow}\m{x}\m{\in}\m{y}\m{)}\m{\}}
\endm
\begin{mmraw}%
|- |\^{}| A = \{ x | A. y ( y e. A -> x e. y ) \} \$.
\end{mmraw}

\noindent Define a transitive class\index{transitive class}\index{transitive
set}.  This should not be confused with a transitive relation, which is a different
concept.  Definition from p.~71 of Enderton, extended to classes.

\vskip 0.5ex
\setbox\startprefix=\hbox{\tt \ \ df-tr\ \$a\ }
\setbox\contprefix=\hbox{\tt \ \ \ \ \ \ \ \ \ \ \ }
\startm
\m{\vdash}\m{(}\m{\mbox{\rm Tr}}\m{A}\m{\leftrightarrow}\m{\bigcup}\m{A}\m{
\subseteq}\m{A}\m{)}
\endm
\begin{mmraw}%
|- ( Tr A <-> U. A C\_ A ) \$.
\end{mmraw}

\noindent Define a notation for a general binary relation\index{binary
relation}.  Definition 6.18, p.~29, of Takeuti and Zaring, generalized to
arbitrary classes.  This definition is well-defined, although not very
meaningful, when classes $A$ and/or $B$ are proper.\label{dfbr}  The lack of
parentheses (or any other connective) creates no ambiguity since we are defining
an atomic wff.

\vskip 0.5ex
\setbox\startprefix=\hbox{\tt \ \ df-br\ \$a\ }
\setbox\contprefix=\hbox{\tt \ \ \ \ \ \ \ \ \ \ \ }
\startm
\m{\vdash}\m{(}\m{A}\m{\,R}\m{\,B}\m{\leftrightarrow}\m{\langle}\m{A}\m{,}\m{B}
\m{\rangle}\m{\in}\m{R}\m{)}
\endm
\begin{mmraw}%
|- ( A R B <-> <. A , B >. e. R ) \$.
\end{mmraw}

\noindent Define an abstraction class of ordered pairs\index{abstraction
class!of ordered
pairs}.  A special case of Definition 4.16, p.~14, of Takeuti and Zaring.
Note that $ z $ must be distinct from $ x $ and $ y $,
and $ z $ must not occur in $\varphi$, but $ x $ and $ y $ may be
identical and may appear in $\varphi$.

\vskip 0.5ex
\setbox\startprefix=\hbox{\tt \ \ df-opab\ \$a\ }
\setbox\contprefix=\hbox{\tt \ \ \ \ \ \ \ \ \ \ \ \ \ }
\startm
\m{\vdash}\m{\{}\m{\langle}\m{x}\m{,}\m{y}\m{\rangle}\m{|}\m{\varphi}\m{\}}\m{=}
\m{\{}\m{z}\m{|}\m{\exists}\m{x}\m{\exists}\m{y}\m{(}\m{z}\m{=}\m{\langle}\m{x}
\m{,}\m{y}\m{\rangle}\m{\wedge}\m{\varphi}\m{)}\m{\}}
\endm

\begin{mmraw}%
|- \{ <. x , y >. | ph \} = \{ z | E. x E. y ( z =
<. x , y >. /\ ph ) \} \$.
\end{mmraw}

\noindent Define the epsilon relation\index{epsilon relation}.  Similar to Definition
6.22, p.~30, of Takeuti and Zaring.

\vskip 0.5ex
\setbox\startprefix=\hbox{\tt \ \ df-eprel\ \$a\ }
\setbox\contprefix=\hbox{\tt \ \ \ \ \ \ \ \ \ \ \ \ \ \ }
\startm
\m{\vdash}\m{{\rm E}}\m{=}\m{\{}\m{\langle}\m{x}\m{,}\m{y}\m{\rangle}\m{|}\m{x}\m{
\in}\m{y}\m{\}}
\endm
\begin{mmraw}%
|- \_E = \{ <. x , y >. | x e. y \} \$.
\end{mmraw}

\noindent Define a founded relation\index{founded relation}.  $R$ is a founded
relation on $A$ iff\index{iff} (if and only if) each nonempty subset of $A$
has an ``$R$-minimal element.''  Similar to Definition 6.21, p.~30, of
Takeuti and Zaring.

\vskip 0.5ex
\setbox\startprefix=\hbox{\tt \ \ df-fr\ \$a\ }
\setbox\contprefix=\hbox{\tt \ \ \ \ \ \ \ \ \ \ \ }
\startm
\m{\vdash}\m{(}\m{R}\m{\,\mbox{\rm Fr}}\m{\,A}\m{\leftrightarrow}\m{\forall}\m{x}
\m{(}\m{(}\m{x}\m{\subseteq}\m{A}\m{\wedge}\m{\lnot}\m{x}\m{=}\m{\varnothing}
\m{)}\m{\rightarrow}\m{\exists}\m{y}\m{(}\m{y}\m{\in}\m{x}\m{\wedge}\m{(}\m{x}
\m{\cap}\m{\{}\m{z}\m{|}\m{z}\m{\,R}\m{\,y}\m{\}}\m{)}\m{=}\m{\varnothing}\m{)}
\m{)}\m{)}
\endm
\begin{mmraw}%
|- ( R Fr A <-> A. x ( ( x C\_ A \TAND -. x = (/) ) ->
E. y ( y e. x \TAND ( x i\^{}i \{ z | z R y \} ) = (/) ) ) ) \$.
\end{mmraw}

\noindent Define a well-ordering\index{well-ordering}.  $R$ is a well-ordering of $A$ iff
it is founded on $A$ and the elements of $A$ are pairwise $R$-comparable.
Similar to Definition 6.24(2), p.~30, of Takeuti and Zaring.

\vskip 0.5ex
\setbox\startprefix=\hbox{\tt \ \ df-we\ \$a\ }
\setbox\contprefix=\hbox{\tt \ \ \ \ \ \ \ \ \ \ \ }
\startm
\m{\vdash}\m{(}\m{R}\m{\,\mbox{\rm We}}\m{\,A}\m{\leftrightarrow}\m{(}\m{R}\m{\,
\mbox{\rm Fr}}\m{\,A}\m{\wedge}\m{\forall}\m{x}\m{\forall}\m{y}\m{(}\m{(}\m{x}\m{
\in}\m{A}\m{\wedge}\m{y}\m{\in}\m{A}\m{)}\m{\rightarrow}\m{(}\m{x}\m{\,R}\m{\,y}
\m{\vee}\m{x}\m{=}\m{y}\m{\vee}\m{y}\m{\,R}\m{\,x}\m{)}\m{)}\m{)}\m{)}
\endm
\begin{mmraw}%
( R We A <-> ( R Fr A \TAND A. x A. y ( ( x e.
A \TAND y e. A ) -> ( x R y \TOR x = y \TOR y R x ) ) ) ) \$.
\end{mmraw}

\noindent Define the ordinal predicate\index{ordinal predicate}, which is true for a class
that is transitive and is well-ordered by the epsilon relation.  Similar to
definition on p.~468, Bell and Machover.

\vskip 0.5ex
\setbox\startprefix=\hbox{\tt \ \ df-ord\ \$a\ }
\setbox\contprefix=\hbox{\tt \ \ \ \ \ \ \ \ \ \ \ \ }
\startm
\m{\vdash}\m{(}\m{\mbox{\rm Ord}}\m{\,A}\m{\leftrightarrow}\m{(}
\m{\mbox{\rm Tr}}\m{\,A}\m{\wedge}\m{E}\m{\,\mbox{\rm We}}\m{\,A}\m{)}\m{)}
\endm
\begin{mmraw}%
|- ( Ord A <-> ( Tr A \TAND E We A ) ) \$.
\end{mmraw}

\noindent Define the class of all ordinal numbers\index{ordinal number}.  An ordinal number is
a set that satisfies the ordinal predicate.  Definition 7.11 of Takeuti and
Zaring, p.~38.

\vskip 0.5ex
\setbox\startprefix=\hbox{\tt \ \ df-on\ \$a\ }
\setbox\contprefix=\hbox{\tt \ \ \ \ \ \ \ \ \ \ \ }
\startm
\m{\vdash}\m{\,\mbox{\rm On}}\m{=}\m{\{}\m{x}\m{|}\m{\mbox{\rm Ord}}\m{\,x}
\m{\}}
\endm
\begin{mmraw}%
|- On = \{ x | Ord x \} \$.
\end{mmraw}

\noindent Define the limit ordinal predicate\index{limit ordinal}, which is true for a
non-empty ordinal that is not a successor (i.e.\ that is the union of itself).
Compare Bell and Machover, p.~471 and Exercise (1), p.~42 of Takeuti and
Zaring.

\vskip 0.5ex
\setbox\startprefix=\hbox{\tt \ \ df-lim\ \$a\ }
\setbox\contprefix=\hbox{\tt \ \ \ \ \ \ \ \ \ \ \ \ }
\startm
\m{\vdash}\m{(}\m{\mbox{\rm Lim}}\m{\,A}\m{\leftrightarrow}\m{(}\m{\mbox{
\rm Ord}}\m{\,A}\m{\wedge}\m{\lnot}\m{A}\m{=}\m{\varnothing}\m{\wedge}\m{A}
\m{=}\m{\bigcup}\m{A}\m{)}\m{)}
\endm
\begin{mmraw}%
|- ( Lim A <-> ( Ord A \TAND -. A = (/) \TAND A = U. A ) ) \$.
\end{mmraw}

\noindent Define the successor\index{successor} of a class.  Definition 7.22 of Takeuti
and Zaring, p.~41.  Our definition is a generalization to classes, although it
is meaningless when classes are proper.

\vskip 0.5ex
\setbox\startprefix=\hbox{\tt \ \ df-suc\ \$a\ }
\setbox\contprefix=\hbox{\tt \ \ \ \ \ \ \ \ \ \ \ \ }
\startm
\m{\vdash}\m{\,\mbox{\rm suc}}\m{\,A}\m{=}\m{(}\m{A}\m{\cup}\m{\{}\m{A}\m{\}}
\m{)}
\endm
\begin{mmraw}%
|- suc A = ( A u. \{ A \} ) \$.
\end{mmraw}

\noindent Define the class of natural numbers\index{natural number}\index{omega
($\omega$)}.  Compare Bell and Machover, p.~471.\label{dfom}

\vskip 0.5ex
\setbox\startprefix=\hbox{\tt \ \ df-om\ \$a\ }
\setbox\contprefix=\hbox{\tt \ \ \ \ \ \ \ \ \ \ \ }
\startm
\m{\vdash}\m{\omega}\m{=}\m{\{}\m{x}\m{|}\m{(}\m{\mbox{\rm Ord}}\m{\,x}\m{
\wedge}\m{\forall}\m{y}\m{(}\m{\mbox{\rm Lim}}\m{\,y}\m{\rightarrow}\m{x}\m{
\in}\m{y}\m{)}\m{)}\m{\}}
\endm
\begin{mmraw}%
|- om = \{ x | ( Ord x \TAND A. y ( Lim y -> x e. y ) ) \} \$.
\end{mmraw}

\noindent Define the Cartesian product (also called the
cross product)\index{Cartesian product}\index{cross product}
of two classes.  Definition 9.11 of Quine, p.~64.

\vskip 0.5ex
\setbox\startprefix=\hbox{\tt \ \ df-xp\ \$a\ }
\setbox\contprefix=\hbox{\tt \ \ \ \ \ \ \ \ \ \ \ }
\startm
\m{\vdash}\m{(}\m{A}\m{\times}\m{B}\m{)}\m{=}\m{\{}\m{\langle}\m{x}\m{,}\m{y}
\m{\rangle}\m{|}\m{(}\m{x}\m{\in}\m{A}\m{\wedge}\m{y}\m{\in}\m{B}\m{)}\m{\}}
\endm
\begin{mmraw}%
|- ( A X. B ) = \{ <. x , y >. | ( x e. A \TAND y e. B) \} \$.
\end{mmraw}

\noindent Define a relation\index{relation}.  Definition 6.4(1) of Takeuti and
Zaring, p.~23.

\vskip 0.5ex
\setbox\startprefix=\hbox{\tt \ \ df-rel\ \$a\ }
\setbox\contprefix=\hbox{\tt \ \ \ \ \ \ \ \ \ \ \ \ }
\startm
\m{\vdash}\m{(}\m{\mbox{\rm Rel}}\m{\,A}\m{\leftrightarrow}\m{A}\m{\subseteq}
\m{(}\m{{\rm V}}\m{\times}\m{{\rm V}}\m{)}\m{)}
\endm
\begin{mmraw}%
|- ( Rel A <-> A C\_ ( {\char`\_}V X. {\char`\_}V ) ) \$.
\end{mmraw}

\noindent Define the domain\index{domain} of a class.  Definition 6.5(1) of
Takeuti and Zaring, p.~24.

\vskip 0.5ex
\setbox\startprefix=\hbox{\tt \ \ df-dm\ \$a\ }
\setbox\contprefix=\hbox{\tt \ \ \ \ \ \ \ \ \ \ \ }
\startm
\m{\vdash}\m{\,\mbox{\rm dom}}\m{A}\m{=}\m{\{}\m{x}\m{|}\m{\exists}\m{y}\m{
\langle}\m{x}\m{,}\m{y}\m{\rangle}\m{\in}\m{A}\m{\}}
\endm
\begin{mmraw}%
|- dom A = \{ x | E. y <. x , y >. e. A \} \$.
\end{mmraw}

\noindent Define the range\index{range} of a class.  Definition 6.5(2) of
Takeuti and Zaring, p.~24.

\vskip 0.5ex
\setbox\startprefix=\hbox{\tt \ \ df-rn\ \$a\ }
\setbox\contprefix=\hbox{\tt \ \ \ \ \ \ \ \ \ \ \ }
\startm
\m{\vdash}\m{\,\mbox{\rm ran}}\m{A}\m{=}\m{\{}\m{y}\m{|}\m{\exists}\m{x}\m{
\langle}\m{x}\m{,}\m{y}\m{\rangle}\m{\in}\m{A}\m{\}}
\endm
\begin{mmraw}%
|- ran A = \{ y | E. x <. x , y >. e. A \} \$.
\end{mmraw}

\noindent Define the restriction\index{restriction} of a class.  Definition
6.6(1) of Takeuti and Zaring, p.~24.

\vskip 0.5ex
\setbox\startprefix=\hbox{\tt \ \ df-res\ \$a\ }
\setbox\contprefix=\hbox{\tt \ \ \ \ \ \ \ \ \ \ \ \ }
\startm
\m{\vdash}\m{(}\m{A}\m{\restriction}\m{B}\m{)}\m{=}\m{(}\m{A}\m{\cap}\m{(}\m{B}
\m{\times}\m{{\rm V}}\m{)}\m{)}
\endm
\begin{mmraw}%
|- ( A |` B ) = ( A i\^{}i ( B X. {\char`\_}V ) ) \$.
\end{mmraw}

\noindent Define the image\index{image} of a class.  Definition 6.6(2) of
Takeuti and Zaring, p.~24.

\vskip 0.5ex
\setbox\startprefix=\hbox{\tt \ \ df-ima\ \$a\ }
\setbox\contprefix=\hbox{\tt \ \ \ \ \ \ \ \ \ \ \ \ }
\startm
\m{\vdash}\m{(}\m{A}\m{``}\m{B}\m{)}\m{=}\m{\,\mbox{\rm ran}}\m{\,(}\m{A}\m{
\restriction}\m{B}\m{)}
\endm
\begin{mmraw}%
|- ( A " B ) = ran ( A |` B ) \$.
\end{mmraw}

\noindent Define the composition\index{composition} of two classes.  Definition 6.6(3) of
Takeuti and Zaring, p.~24.

\vskip 0.5ex
\setbox\startprefix=\hbox{\tt \ \ df-co\ \$a\ }
\setbox\contprefix=\hbox{\tt \ \ \ \ \ \ \ \ \ \ \ \ }
\startm
\m{\vdash}\m{(}\m{A}\m{\circ}\m{B}\m{)}\m{=}\m{\{}\m{\langle}\m{x}\m{,}\m{y}\m{
\rangle}\m{|}\m{\exists}\m{z}\m{(}\m{\langle}\m{x}\m{,}\m{z}\m{\rangle}\m{\in}
\m{B}\m{\wedge}\m{\langle}\m{z}\m{,}\m{y}\m{\rangle}\m{\in}\m{A}\m{)}\m{\}}
\endm
\begin{mmraw}%
|- ( A o. B ) = \{ <. x , y >. | E. z ( <. x , z
>. e. B \TAND <. z , y >. e. A ) \} \$.
\end{mmraw}

\noindent Define a function\index{function}.  Definition 6.4(4) of Takeuti and
Zaring, p.~24.

\vskip 0.5ex
\setbox\startprefix=\hbox{\tt \ \ df-fun\ \$a\ }
\setbox\contprefix=\hbox{\tt \ \ \ \ \ \ \ \ \ \ \ \ }
\startm
\m{\vdash}\m{(}\m{\mbox{\rm Fun}}\m{\,A}\m{\leftrightarrow}\m{(}
\m{\mbox{\rm Rel}}\m{\,A}\m{\wedge}
\m{\forall}\m{x}\m{\exists}\m{z}\m{\forall}\m{y}\m{(}
\m{\langle}\m{x}\m{,}\m{y}\m{\rangle}\m{\in}\m{A}\m{\rightarrow}\m{y}\m{=}\m{z}
\m{)}\m{)}\m{)}
\endm
\begin{mmraw}%
|- ( Fun A <-> ( Rel A /\ A. x E. z A. y ( <. x
   , y >. e. A -> y = z ) ) ) \$.
\end{mmraw}

\noindent Define a function with domain.  Definition 6.15(1) of Takeuti and
Zaring, p.~27.

\vskip 0.5ex
\setbox\startprefix=\hbox{\tt \ \ df-fn\ \$a\ }
\setbox\contprefix=\hbox{\tt \ \ \ \ \ \ \ \ \ \ \ }
\startm
\m{\vdash}\m{(}\m{A}\m{\,\mbox{\rm Fn}}\m{\,B}\m{\leftrightarrow}\m{(}
\m{\mbox{\rm Fun}}\m{\,A}\m{\wedge}\m{\mbox{\rm dom}}\m{\,A}\m{=}\m{B}\m{)}
\m{)}
\endm
\begin{mmraw}%
|- ( A Fn B <-> ( Fun A \TAND dom A = B ) ) \$.
\end{mmraw}

\noindent Define a function with domain and co-domain.  Definition 6.15(3)
of Takeuti and Zaring, p.~27.

\vskip 0.5ex
\setbox\startprefix=\hbox{\tt \ \ df-f\ \$a\ }
\setbox\contprefix=\hbox{\tt \ \ \ \ \ \ \ \ \ \ }
\startm
\m{\vdash}\m{(}\m{F}\m{:}\m{A}\m{\longrightarrow}\m{B}\m{
\leftrightarrow}\m{(}\m{F}\m{\,\mbox{\rm Fn}}\m{\,A}\m{\wedge}\m{
\mbox{\rm ran}}\m{\,F}\m{\subseteq}\m{B}\m{)}\m{)}
\endm
\begin{mmraw}%
|- ( F : A --> B <-> ( F Fn A \TAND ran F C\_ B ) ) \$.
\end{mmraw}

\noindent Define a one-to-one function\index{one-to-one function}.  Compare
Definition 6.15(5) of Takeuti and Zaring, p.~27.

\vskip 0.5ex
\setbox\startprefix=\hbox{\tt \ \ df-f1\ \$a\ }
\setbox\contprefix=\hbox{\tt \ \ \ \ \ \ \ \ \ \ \ }
\startm
\m{\vdash}\m{(}\m{F}\m{:}\m{A}\m{
\raisebox{.5ex}{${\textstyle{\:}_{\mbox{\footnotesize\rm
1\tt -\rm 1}}}\atop{\textstyle{
\longrightarrow}\atop{\textstyle{}^{\mbox{\footnotesize\rm {\ }}}}}$}
}\m{B}
\m{\leftrightarrow}\m{(}\m{F}\m{:}\m{A}\m{\longrightarrow}\m{B}
\m{\wedge}\m{\forall}\m{y}\m{\exists}\m{z}\m{\forall}\m{x}\m{(}\m{\langle}\m{x}
\m{,}\m{y}\m{\rangle}\m{\in}\m{F}\m{\rightarrow}\m{x}\m{=}\m{z}\m{)}\m{)}\m{)}
\endm
\begin{mmraw}%
|- ( F : A -1-1-> B <-> ( F : A --> B \TAND
   A. y E. z A. x ( <. x , y >. e. F -> x = z ) ) ) \$.
\end{mmraw}

\noindent Define an onto function\index{onto function}.  Definition 6.15(4) of Takeuti and
Zaring, p.~27.

\vskip 0.5ex
\setbox\startprefix=\hbox{\tt \ \ df-fo\ \$a\ }
\setbox\contprefix=\hbox{\tt \ \ \ \ \ \ \ \ \ \ \ }
\startm
\m{\vdash}\m{(}\m{F}\m{:}\m{A}\m{
\raisebox{.5ex}{${\textstyle{\:}_{\mbox{\footnotesize\rm
{\ }}}}\atop{\textstyle{
\longrightarrow}\atop{\textstyle{}^{\mbox{\footnotesize\rm onto}}}}$}
}\m{B}
\m{\leftrightarrow}\m{(}\m{F}\m{\,\mbox{\rm Fn}}\m{\,A}\m{\wedge}
\m{\mbox{\rm ran}}\m{\,F}\m{=}\m{B}\m{)}\m{)}
\endm
\begin{mmraw}%
|- ( F : A -onto-> B <-> ( F Fn A /\ ran F = B ) ) \$.
\end{mmraw}

\noindent Define a one-to-one, onto function.  Compare Definition 6.15(6) of
Takeuti and Zaring, p.~27.

\vskip 0.5ex
\setbox\startprefix=\hbox{\tt \ \ df-f1o\ \$a\ }
\setbox\contprefix=\hbox{\tt \ \ \ \ \ \ \ \ \ \ \ \ }
\startm
\m{\vdash}\m{(}\m{F}\m{:}\m{A}
\m{
\raisebox{.5ex}{${\textstyle{\:}_{\mbox{\footnotesize\rm
1\tt -\rm 1}}}\atop{\textstyle{
\longrightarrow}\atop{\textstyle{}^{\mbox{\footnotesize\rm onto}}}}$}
}
\m{B}
\m{\leftrightarrow}\m{(}\m{F}\m{:}\m{A}
\m{
\raisebox{.5ex}{${\textstyle{\:}_{\mbox{\footnotesize\rm
1\tt -\rm 1}}}\atop{\textstyle{
\longrightarrow}\atop{\textstyle{}^{\mbox{\footnotesize\rm {\ }}}}}$}
}
\m{B}\m{\wedge}\m{F}\m{:}\m{A}
\m{
\raisebox{.5ex}{${\textstyle{\:}_{\mbox{\footnotesize\rm
{\ }}}}\atop{\textstyle{
\longrightarrow}\atop{\textstyle{}^{\mbox{\footnotesize\rm onto}}}}$}
}
\m{B}\m{)}\m{)}
\endm
\begin{mmraw}%
|- ( F : A -1-1-onto-> B <-> ( F : A -1-1-> B? \TAND F : A -onto-> B ) ) \$.?
\end{mmraw}

\noindent Define the value of a function\index{function value}.  This
definition applies to any class and evaluates to the empty set when it is not
meaningful. Note that $ F`A$ means the same thing as the more familiar $ F(A)$
notation for a function's value at $A$.  The $ F`A$ notation is common in
formal set theory.\label{df-fv}

\vskip 0.5ex
\setbox\startprefix=\hbox{\tt \ \ df-fv\ \$a\ }
\setbox\contprefix=\hbox{\tt \ \ \ \ \ \ \ \ \ \ \ }
\startm
\m{\vdash}\m{(}\m{F}\m{`}\m{A}\m{)}\m{=}\m{\bigcup}\m{\{}\m{x}\m{|}\m{(}\m{F}%
\m{``}\m{\{}\m{A}\m{\}}\m{)}\m{=}\m{\{}\m{x}\m{\}}\m{\}}
\endm
\begin{mmraw}%
|- ( F ` A ) = U. \{ x | ( F " \{ A \} ) = \{ x \} \} \$.
\end{mmraw}

\noindent Define the result of an operation.\index{operation}  Here, $F$ is
     an operation on two
     values (such as $+$ for real numbers).   This is defined for proper
     classes $A$ and $B$ even though not meaningful in that case.  However,
     the definition can be meaningful when $F$ is a proper class.\label{dfopr}

\vskip 0.5ex
\setbox\startprefix=\hbox{\tt \ \ df-opr\ \$a\ }
\setbox\contprefix=\hbox{\tt \ \ \ \ \ \ \ \ \ \ \ \ }
\startm
\m{\vdash}\m{(}\m{A}\m{\,F}\m{\,B}\m{)}\m{=}\m{(}\m{F}\m{`}\m{\langle}\m{A}%
\m{,}\m{B}\m{\rangle}\m{)}
\endm
\begin{mmraw}%
|- ( A F B ) = ( F ` <. A , B >. ) \$.
\end{mmraw}

\section{Tricks of the Trade}\label{tricks}

In the \texttt{set.mm}\index{set theory database (\texttt{set.mm})} database our goal
was usually to conform to modern notation.  However in some cases the
relationship to standard textbook language may be obscured by several
unconventional devices we used to simplify the development and to take
advantage of the Metamath language.  In this section we will describe some
common conventions used in \texttt{set.mm}.

\begin{itemize}
\item
The turnstile\index{turnstile ({$\,\vdash$})} symbol, $\vdash$, meaning ``it
is provable that,'' is the first token of all assertions and hypotheses that
aren't syntax constructions.  This is a standard convention in logic.  (We
mentioned this earlier, but this symbol is bothersome to some people without a
logic background.  It has no deeper meaning but just provides us with a way to
distinguish syntax constructions from ordinary mathematical statements.)

\item
A hypothesis of the form

\vskip 1ex
\setbox\startprefix=\hbox{\tt \ \ \ \ \ \ \ \ \ \$e\ }
\setbox\contprefix=\hbox{\tt \ \ \ \ \ \ \ \ \ \ }
\startm
\m{\vdash}\m{(}\m{\varphi}\m{\rightarrow}\m{\forall}\m{x}\m{\varphi}\m{)}
\endm
\vskip 1ex

should be read ``assume variable $x$ is (effectively) not free in wff
$\varphi$.''\index{effectively not free}
Literally, this says ``assume it is provable that $\varphi \rightarrow \forall
x\, \varphi$.''  This device lets us avoid the complexities associated with
the standard treatment of free and bound variables.
%% Uncomment this when uncommenting section {formalspec} below
The footnote on p.~\pageref{effectivelybound} discusses this further.

\item
A statement of one of the forms

\vskip 1ex
\setbox\startprefix=\hbox{\tt \ \ \ \ \ \ \ \ \ \$a\ }
\setbox\contprefix=\hbox{\tt \ \ \ \ \ \ \ \ \ \ }
\startm
\m{\vdash}\m{(}\m{\lnot}\m{\forall}\m{x}\m{\,x}\m{=}\m{y}\m{\rightarrow}
\m{\ldots}\m{)}
\endm
\setbox\startprefix=\hbox{\tt \ \ \ \ \ \ \ \ \ \$p\ }
\setbox\contprefix=\hbox{\tt \ \ \ \ \ \ \ \ \ \ }
\startm
\m{\vdash}\m{(}\m{\lnot}\m{\forall}\m{x}\m{\,x}\m{=}\m{y}\m{\rightarrow}
\m{\ldots}\m{)}
\endm
\vskip 1ex

should be read ``if $x$ and $y$ are distinct variables, then...''  This
antecedent provides us with a technical device to avoid the need for the
\texttt{\$d} statement early in our development of predicate calculus,
permitting symbol manipulations to be as conceptually simple as those in
propositional calculus.  However, the \texttt{\$d} statement eventually
becomes a requirement, and after that this device is rarely used.

\item
The statement

\vskip 1ex
\setbox\startprefix=\hbox{\tt \ \ \ \ \ \ \ \ \ \$d\ }
\setbox\contprefix=\hbox{\tt \ \ \ \ \ \ \ \ \ \ }
\startm
\m{x}\m{\,y}
\endm
\vskip 1ex

should be read ``assume $x$ and $y$ are distinct variables.''

\item
The statement

\vskip 1ex
\setbox\startprefix=\hbox{\tt \ \ \ \ \ \ \ \ \ \$d\ }
\setbox\contprefix=\hbox{\tt \ \ \ \ \ \ \ \ \ \ }
\startm
\m{x}\m{\,\varphi}
\endm
\vskip 1ex

should be read ``assume $x$ does not occur in $\varphi$.''

\item
The statement

\vskip 1ex
\setbox\startprefix=\hbox{\tt \ \ \ \ \ \ \ \ \ \$d\ }
\setbox\contprefix=\hbox{\tt \ \ \ \ \ \ \ \ \ \ }
\startm
\m{x}\m{\,A}
\endm
\vskip 1ex

should be read ``assume variable $x$ does not occur in class $A$.''

\item
The restriction and hypothesis group

\vskip 1ex
\setbox\startprefix=\hbox{\tt \ \ \ \ \ \ \ \ \ \$d\ }
\setbox\contprefix=\hbox{\tt \ \ \ \ \ \ \ \ \ \ }
\startm
\m{x}\m{\,A}
\endm
\setbox\startprefix=\hbox{\tt \ \ \ \ \ \ \ \ \ \$d\ }
\setbox\contprefix=\hbox{\tt \ \ \ \ \ \ \ \ \ \ }
\startm
\m{x}\m{\,\psi}
\endm
\setbox\startprefix=\hbox{\tt \ \ \ \ \ \ \ \ \ \$e\ }
\setbox\contprefix=\hbox{\tt \ \ \ \ \ \ \ \ \ \ }
\startm
\m{\vdash}\m{(}\m{x}\m{=}\m{A}\m{\rightarrow}\m{(}\m{\varphi}\m{\leftrightarrow}
\m{\psi}\m{)}\m{)}
\endm
\vskip 1ex

is frequently used in place of explicit substitution, meaning ``assume
$\psi$ results from the proper substitution of $A$ for $x$ in
$\varphi$.''  Sometimes ``\texttt{\$e} $\vdash ( \psi \rightarrow
\forall x \, \psi )$'' is used instead of ``\texttt{\$d} $x\, \psi $,''
which requires only that $x$ be effectively not free in $\varphi$ but
not necessarily absent from it.  The use of implicit
substitution\index{substitution!implicit} is partly a matter of personal
style, although it may make proofs somewhat shorter than would be the
case with explicit substitution.

\item
The hypothesis


\vskip 1ex
\setbox\startprefix=\hbox{\tt \ \ \ \ \ \ \ \ \ \$e\ }
\setbox\contprefix=\hbox{\tt \ \ \ \ \ \ \ \ \ \ }
\startm
\m{\vdash}\m{A}\m{\in}\m{{\rm V}}
\endm
\vskip 1ex

should be read ``assume class $A$ is a set (i.e.\ exists).''
This is a convenient convention used by Quine.

\item
The restriction and hypothesis

\vskip 1ex
\setbox\startprefix=\hbox{\tt \ \ \ \ \ \ \ \ \ \$d\ }
\setbox\contprefix=\hbox{\tt \ \ \ \ \ \ \ \ \ \ }
\startm
\m{x}\m{\,y}
\endm
\setbox\startprefix=\hbox{\tt \ \ \ \ \ \ \ \ \ \$e\ }
\setbox\contprefix=\hbox{\tt \ \ \ \ \ \ \ \ \ \ }
\startm
\m{\vdash}\m{(}\m{y}\m{\in}\m{A}\m{\rightarrow}\m{\forall}\m{x}\m{\,y}
\m{\in}\m{A}\m{)}
\endm
\vskip 1ex

should be read ``assume variable $x$ is
(effectively) not free in class $A$.''

\end{itemize}

\section{A Theorem Sampler}\label{sometheorems}

In this section we list some of the more important theorems that are proved in
the \texttt{set.mm} database, and they illustrate the kinds of things that can be
done with Metamath.  While all of these facts are well-known results,
Metamath offers the advantage of easily allowing you to trace their
derivation back to axioms.  Our intent here is not to try to explain the
details or motivation; for this we refer you to the textbooks that are
mentioned in the descriptions.  (The \texttt{set.mm} file has bibliographic
references for the text references.)  Their proofs often embody important
concepts you may wish to explore with the Metamath program (see
Section~\ref{exploring}).  All the symbols that are used here are defined in
Section~\ref{hierarchy}.  For brevity we haven't included the \texttt{\$d}
restrictions or \texttt{\$f} hypotheses for these theorems; when you are
uncertain consult the \texttt{set.mm} database.

We start with \texttt{syl} (principle of the syllogism).
In \textit{Principia Mathematica}
Whitehead and Russell call this ``the principle of the
syllogism... because... the syllogism in Barbara is derived from them''
\cite[quote after Theorem *2.06 p.~101]{PM}.
Some authors call this law a ``hypothetical syllogism.''
As of 2019 \texttt{syl} is the most commonly referenced proven
assertion in the \texttt{set.mm} database.\footnote{
The Metamath program command \texttt{show usage}
shows the number of references.
On 2019-04-29 (commit 71cbbbdb387e) \texttt{syl} was directly referenced
10,819 times. The second most commonly referenced proven assertion
was \texttt{eqid}, which was directly referenced 7,738 times.
}

\vskip 2ex
\noindent Theorem syl (principle of the syllogism)\index{Syllogism}%
\index{\texttt{syl}}\label{syl}.

\vskip 0.5ex
\setbox\startprefix=\hbox{\tt \ \ syl.1\ \$e\ }
\setbox\contprefix=\hbox{\tt \ \ \ \ \ \ \ \ \ \ \ }
\startm
\m{\vdash}\m{(}\m{\varphi}\m{ \rightarrow }\m{\psi}\m{)}
\endm
\setbox\startprefix=\hbox{\tt \ \ syl.2\ \$e\ }
\setbox\contprefix=\hbox{\tt \ \ \ \ \ \ \ \ \ \ \ }
\startm
\m{\vdash}\m{(}\m{\psi}\m{ \rightarrow }\m{\chi}\m{)}
\endm
\setbox\startprefix=\hbox{\tt \ \ syl\ \$p\ }
\setbox\contprefix=\hbox{\tt \ \ \ \ \ \ \ \ \ }
\startm
\m{\vdash}\m{(}\m{\varphi}\m{ \rightarrow }\m{\chi}\m{)}
\endm
\vskip 2ex

The following theorem is not very deep but provides us with a notational device
that is frequently used.  It allows us to use the expression ``$A \in V$'' as
a compact way of saying that class $A$ exists, i.e.\ is a set.

\vskip 2ex
\noindent Two ways to say ``$A$ is a set'':  $A$ is a member of the universe
$V$ if and only if $A$ exists (i.e.\ there exists a set equal to $A$).
Theorem 6.9 of Quine, p. 43.

\vskip 0.5ex
\setbox\startprefix=\hbox{\tt \ \ isset\ \$p\ }
\setbox\contprefix=\hbox{\tt \ \ \ \ \ \ \ \ \ \ \ }
\startm
\m{\vdash}\m{(}\m{A}\m{\in}\m{{\rm V}}\m{\leftrightarrow}\m{\exists}\m{x}\m{\,x}\m{=}
\m{A}\m{)}
\endm
\vskip 1ex

Next we prove the axioms of standard ZF set theory that were missing from our
axiom system.  From our point of view they are theorems since they
can be derived from the other axioms.

\vskip 2ex
\noindent Axiom of Separation\index{Axiom of Separation}
(Aussonderung)\index{Aussonderung} proved from the other axioms of ZF set
theory.  Compare Exercise 4 of Takeuti and Zaring, p.~22.

\vskip 0.5ex
\setbox\startprefix=\hbox{\tt \ \ inex1.1\ \$e\ }
\setbox\contprefix=\hbox{\tt \ \ \ \ \ \ \ \ \ \ \ \ \ \ \ }
\startm
\m{\vdash}\m{A}\m{\in}\m{{\rm V}}
\endm
\setbox\startprefix=\hbox{\tt \ \ inex\ \$p\ }
\setbox\contprefix=\hbox{\tt \ \ \ \ \ \ \ \ \ \ \ \ \ }
\startm
\m{\vdash}\m{(}\m{A}\m{\cap}\m{B}\m{)}\m{\in}\m{{\rm V}}
\endm
\vskip 1ex

\noindent Axiom of the Null Set\index{Axiom of the Null Set} proved from the
other axioms of ZF set theory. Corollary 5.16 of Takeuti and Zaring, p.~20.

\vskip 0.5ex
\setbox\startprefix=\hbox{\tt \ \ 0ex\ \$p\ }
\setbox\contprefix=\hbox{\tt \ \ \ \ \ \ \ \ \ \ \ \ }
\startm
\m{\vdash}\m{\varnothing}\m{\in}\m{{\rm V}}
\endm
\vskip 1ex

\noindent The Axiom of Pairing\index{Axiom of Pairing} proved from the other
axioms of ZF set theory.  Theorem 7.13 of Quine, p.~51.
\vskip 0.5ex
\setbox\startprefix=\hbox{\tt \ \ prex\ \$p\ }
\setbox\contprefix=\hbox{\tt \ \ \ \ \ \ \ \ \ \ \ \ \ \ }
\startm
\m{\vdash}\m{\{}\m{A}\m{,}\m{B}\m{\}}\m{\in}\m{{\rm V}}
\endm
\vskip 2ex

Next we will list some famous or important theorems that are proved in
the \texttt{set.mm} database.  None of them except \texttt{omex}
require the Axiom of Infinity, as you can verify with the \texttt{show
trace{\char`\_}back} Metamath command.

\vskip 2ex
\noindent The resolution of Russell's paradox\index{Russell's paradox}.  There
exists no set containing the set of all sets which are not members of
themselves.  Proposition 4.14 of Takeuti and Zaring, p.~14.

\vskip 0.5ex
\setbox\startprefix=\hbox{\tt \ \ ru\ \$p\ }
\setbox\contprefix=\hbox{\tt \ \ \ \ \ \ \ \ }
\startm
\m{\vdash}\m{\lnot}\m{\exists}\m{x}\m{\,x}\m{=}\m{\{}\m{y}\m{|}\m{\lnot}\m{y}
\m{\in}\m{y}\m{\}}
\endm
\vskip 1ex

\noindent Cantor's theorem\index{Cantor's theorem}.  No set can be mapped onto
its power set.  Compare Theorem 6B(b) of Enderton, p.~132.

\vskip 0.5ex
\setbox\startprefix=\hbox{\tt \ \ canth.1\ \$e\ }
\setbox\contprefix=\hbox{\tt \ \ \ \ \ \ \ \ \ \ \ \ \ }
\startm
\m{\vdash}\m{A}\m{\in}\m{{\rm V}}
\endm
\setbox\startprefix=\hbox{\tt \ \ canth\ \$p\ }
\setbox\contprefix=\hbox{\tt \ \ \ \ \ \ \ \ \ \ \ }
\startm
\m{\vdash}\m{\lnot}\m{F}\m{:}\m{A}\m{\raisebox{.5ex}{${\textstyle{\:}_{
\mbox{\footnotesize\rm {\ }}}}\atop{\textstyle{\longrightarrow}\atop{
\textstyle{}^{\mbox{\footnotesize\rm onto}}}}$}}\m{{\cal P}}\m{A}
\endm
\vskip 1ex

\noindent The Burali-Forti paradox\index{Burali-Forti paradox}.  No set
contains all ordinal numbers. Enderton, p.~194.  (Burali-Forti was one person,
not two.)

\vskip 0.5ex
\setbox\startprefix=\hbox{\tt \ \ onprc\ \$p\ }
\setbox\contprefix=\hbox{\tt \ \ \ \ \ \ \ \ \ \ \ \ }
\startm
\m{\vdash}\m{\lnot}\m{\mbox{\rm On}}\m{\in}\m{{\rm V}}
\endm
\vskip 1ex

\noindent Peano's postulates\index{Peano's postulates} for arithmetic.
Proposition 7.30 of Takeuti and Zaring, pp.~42--43.  The objects being
described are the members of $\omega$ i.e.\ the natural numbers 0, 1,
2,\ldots.  The successor\index{successor} operation suc means ``plus
one.''  \texttt{peano1} says that 0 (which is defined as the empty set)
is a natural number.  \texttt{peano2} says that if $A$ is a natural
number, so is $A+1$.  \texttt{peano3} says that 0 is not the successor
of any natural number.  \texttt{peano4} says that two natural numbers
are equal if and only if their successors are equal.  \texttt{peano5} is
essentially the same as mathematical induction.

\vskip 1ex
\setbox\startprefix=\hbox{\tt \ \ peano1\ \$p\ }
\setbox\contprefix=\hbox{\tt \ \ \ \ \ \ \ \ \ \ \ \ }
\startm
\m{\vdash}\m{\varnothing}\m{\in}\m{\omega}
\endm
\vskip 1.5ex

\setbox\startprefix=\hbox{\tt \ \ peano2\ \$p\ }
\setbox\contprefix=\hbox{\tt \ \ \ \ \ \ \ \ \ \ \ \ }
\startm
\m{\vdash}\m{(}\m{A}\m{\in}\m{\omega}\m{\rightarrow}\m{{\rm suc}}\m{A}\m{\in}%
\m{\omega}\m{)}
\endm
\vskip 1.5ex

\setbox\startprefix=\hbox{\tt \ \ peano3\ \$p\ }
\setbox\contprefix=\hbox{\tt \ \ \ \ \ \ \ \ \ \ \ \ }
\startm
\m{\vdash}\m{(}\m{A}\m{\in}\m{\omega}\m{\rightarrow}\m{\lnot}\m{{\rm suc}}%
\m{A}\m{=}\m{\varnothing}\m{)}
\endm
\vskip 1.5ex

\setbox\startprefix=\hbox{\tt \ \ peano4\ \$p\ }
\setbox\contprefix=\hbox{\tt \ \ \ \ \ \ \ \ \ \ \ \ }
\startm
\m{\vdash}\m{(}\m{(}\m{A}\m{\in}\m{\omega}\m{\wedge}\m{B}\m{\in}\m{\omega}%
\m{)}\m{\rightarrow}\m{(}\m{{\rm suc}}\m{A}\m{=}\m{{\rm suc}}\m{B}\m{%
\leftrightarrow}\m{A}\m{=}\m{B}\m{)}\m{)}
\endm
\vskip 1.5ex

\setbox\startprefix=\hbox{\tt \ \ peano5\ \$p\ }
\setbox\contprefix=\hbox{\tt \ \ \ \ \ \ \ \ \ \ \ \ }
\startm
\m{\vdash}\m{(}\m{(}\m{\varnothing}\m{\in}\m{A}\m{\wedge}\m{\forall}\m{x}\m{%
\in}\m{\omega}\m{(}\m{x}\m{\in}\m{A}\m{\rightarrow}\m{{\rm suc}}\m{x}\m{\in}%
\m{A}\m{)}\m{)}\m{\rightarrow}\m{\omega}\m{\subseteq}\m{A}\m{)}
\endm
\vskip 1.5ex

\noindent Finite Induction (mathematical induction).\index{finite
induction}\index{mathematical induction} The first hypothesis is the
basis and the second is the induction hypothesis.  Theorem Schema 22 of
Suppes, p.~136.

\vskip 0.5ex
\setbox\startprefix=\hbox{\tt \ \ findes.1\ \$e\ }
\setbox\contprefix=\hbox{\tt \ \ \ \ \ \ \ \ \ \ \ \ \ \ }
\startm
\m{\vdash}\m{[}\m{\varnothing}\m{/}\m{x}\m{]}\m{\varphi}
\endm
\setbox\startprefix=\hbox{\tt \ \ findes.2\ \$e\ }
\setbox\contprefix=\hbox{\tt \ \ \ \ \ \ \ \ \ \ \ \ \ \ }
\startm
\m{\vdash}\m{(}\m{x}\m{\in}\m{\omega}\m{\rightarrow}\m{(}\m{\varphi}\m{%
\rightarrow}\m{[}\m{{\rm suc}}\m{x}\m{/}\m{x}\m{]}\m{\varphi}\m{)}\m{)}
\endm
\setbox\startprefix=\hbox{\tt \ \ findes\ \$p\ }
\setbox\contprefix=\hbox{\tt \ \ \ \ \ \ \ \ \ \ \ \ }
\startm
\m{\vdash}\m{(}\m{x}\m{\in}\m{\omega}\m{\rightarrow}\m{\varphi}\m{)}
\endm
\vskip 1ex

\noindent Transfinite Induction with explicit substitution.  The first
hypothesis is the basis, the second is the induction hypothesis for
successors, and the third is the induction hypothesis for limit
ordinals.  Theorem Schema 4 of Suppes, p. 197.

\vskip 0.5ex
\setbox\startprefix=\hbox{\tt \ \ tfindes.1\ \$e\ }
\setbox\contprefix=\hbox{\tt \ \ \ \ \ \ \ \ \ \ \ \ \ \ \ }
\startm
\m{\vdash}\m{[}\m{\varnothing}\m{/}\m{x}\m{]}\m{\varphi}
\endm
\setbox\startprefix=\hbox{\tt \ \ tfindes.2\ \$e\ }
\setbox\contprefix=\hbox{\tt \ \ \ \ \ \ \ \ \ \ \ \ \ \ \ }
\startm
\m{\vdash}\m{(}\m{x}\m{\in}\m{{\rm On}}\m{\rightarrow}\m{(}\m{\varphi}\m{%
\rightarrow}\m{[}\m{{\rm suc}}\m{x}\m{/}\m{x}\m{]}\m{\varphi}\m{)}\m{)}
\endm
\setbox\startprefix=\hbox{\tt \ \ tfindes.3\ \$e\ }
\setbox\contprefix=\hbox{\tt \ \ \ \ \ \ \ \ \ \ \ \ \ \ \ }
\startm
\m{\vdash}\m{(}\m{{\rm Lim}}\m{y}\m{\rightarrow}\m{(}\m{\forall}\m{x}\m{\in}%
\m{y}\m{\varphi}\m{\rightarrow}\m{[}\m{y}\m{/}\m{x}\m{]}\m{\varphi}\m{)}\m{)}
\endm
\setbox\startprefix=\hbox{\tt \ \ tfindes\ \$p\ }
\setbox\contprefix=\hbox{\tt \ \ \ \ \ \ \ \ \ \ \ \ \ }
\startm
\m{\vdash}\m{(}\m{x}\m{\in}\m{{\rm On}}\m{\rightarrow}\m{\varphi}\m{)}
\endm
\vskip 1ex

\noindent Principle of Transfinite Recursion.\index{transfinite
recursion} Theorem 7.41 of Takeuti and Zaring, p.~47.  Transfinite
recursion is the key theorem that allows arithmetic of ordinals to be
rigorously defined, and has many other important uses as well.
Hypotheses \texttt{tfr.1} and \texttt{tfr.2} specify a certain (proper)
class $ F$.  The complicated definition of $ F$ is not important in
itself; what is important is that there be such an $ F$ with the
required properties, and we show this by displaying $ F$ explicitly.
\texttt{tfr1} states that $ F$ is a function whose domain is the set of
ordinal numbers.  \texttt{tfr2} states that any value of $ F$ is
completely determined by its previous values and the values of an
auxiliary function, $G$.  \texttt{tfr3} states that $ F$ is unique,
i.e.\ it is the only function that satisfies \texttt{tfr1} and
\texttt{tfr2}.  Note that $ f$ is an individual variable like $x$ and
$y$; it is just a mnemonic to remind us that $A$ is a collection of
functions.

\vskip 0.5ex
\setbox\startprefix=\hbox{\tt \ \ tfr.1\ \$e\ }
\setbox\contprefix=\hbox{\tt \ \ \ \ \ \ \ \ \ \ \ }
\startm
\m{\vdash}\m{A}\m{=}\m{\{}\m{f}\m{|}\m{\exists}\m{x}\m{\in}\m{{\rm On}}\m{(}%
\m{f}\m{{\rm Fn}}\m{x}\m{\wedge}\m{\forall}\m{y}\m{\in}\m{x}\m{(}\m{f}\m{`}%
\m{y}\m{)}\m{=}\m{(}\m{G}\m{`}\m{(}\m{f}\m{\restriction}\m{y}\m{)}\m{)}\m{)}%
\m{\}}
\endm
\setbox\startprefix=\hbox{\tt \ \ tfr.2\ \$e\ }
\setbox\contprefix=\hbox{\tt \ \ \ \ \ \ \ \ \ \ \ }
\startm
\m{\vdash}\m{F}\m{=}\m{\bigcup}\m{A}
\endm
\setbox\startprefix=\hbox{\tt \ \ tfr1\ \$p\ }
\setbox\contprefix=\hbox{\tt \ \ \ \ \ \ \ \ \ \ }
\startm
\m{\vdash}\m{F}\m{{\rm Fn}}\m{{\rm On}}
\endm
\setbox\startprefix=\hbox{\tt \ \ tfr2\ \$p\ }
\setbox\contprefix=\hbox{\tt \ \ \ \ \ \ \ \ \ \ }
\startm
\m{\vdash}\m{(}\m{z}\m{\in}\m{{\rm On}}\m{\rightarrow}\m{(}\m{F}\m{`}\m{z}%
\m{)}\m{=}\m{(}\m{G}\m{`}\m{(}\m{F}\m{\restriction}\m{z}\m{)}\m{)}\m{)}
\endm
\setbox\startprefix=\hbox{\tt \ \ tfr3\ \$p\ }
\setbox\contprefix=\hbox{\tt \ \ \ \ \ \ \ \ \ \ }
\startm
\m{\vdash}\m{(}\m{(}\m{B}\m{{\rm Fn}}\m{{\rm On}}\m{\wedge}\m{\forall}\m{x}\m{%
\in}\m{{\rm On}}\m{(}\m{B}\m{`}\m{x}\m{)}\m{=}\m{(}\m{G}\m{`}\m{(}\m{B}\m{%
\restriction}\m{x}\m{)}\m{)}\m{)}\m{\rightarrow}\m{B}\m{=}\m{F}\m{)}
\endm
\vskip 1ex

\noindent The existence of omega (the class of natural numbers).\index{natural
number}\index{omega ($\omega$)}\index{Axiom of Infinity}  Axiom 7 of Takeuti
and Zaring, p.~43.  (This is the only theorem in this section requiring the
Axiom of Infinity.)

\vskip 0.5ex
\setbox\startprefix=\hbox{\tt \
\ omex\ \$p\ }
\setbox\contprefix=\hbox{\tt \ \ \ \ \ \ \ \ \ \ }
\startm
\m{\vdash}\m{\omega}\m{\in}\m{{\rm V}}
\endm
%\vskip 2ex


\section{Axioms for Real and Complex Numbers}\label{real}
\index{real and complex numbers!axioms for}

This section presents the axioms for real and complex numbers, along
with some commentary about them.  Analysis
textbooks implicitly or explicitly use these axioms or their equivalents
as their starting point.  In the database \texttt{set.mm}, we define real
and complex numbers as (rather complicated) specific sets and derive these
axioms as {\em theorems} from the axioms of ZF set theory, using a method
called Dedekind cuts.  We omit the details of this construction, which you can
follow if you wish using the \texttt{set.mm} database in conjunction with the
textbooks referenced therein.

Once we prove those theorems, we then restate these proven theorems as axioms.
This lets us easily identify which axioms are needed for a particular complex number proof, without the obfuscation of the set theory used to derive them.
As a result,
the construction is actually unimportant other
than to show that sets exist that satisfy the axioms, and thus that the axioms
are consistent if ZF set theory is consistent.  When working with real numbers
you can think of them as being the actual sets resulting
from the construction (for definiteness), or you can
think of them as otherwise unspecified sets that happen to satisfy the axioms.
The derivation is not easy, but the fact that it works is quite remarkable
and lends support to the idea that ZFC set theory is all we need to
provide a foundation for essentially all of mathematics.

\needspace{3\baselineskip}
\subsection{The Axioms for Real and Complex Numbers Themselves}\label{realactual}

For the axioms we are given (or postulate) 8 classes:  $\mathbb{C}$ (the
set of complex numbers), $\mathbb{R}$ (the set of real numbers, a subset
of $\mathbb{C}$), $0$ (zero), $1$ (one), $i$ (square root of
$-1$), $+$ (plus), $\cdot$ (times), and
$<_{\mathbb{R}}$ (less than for just the real numbers).
Subtraction and division are defined terms and are not part of the
axioms; for their definitions see \texttt{set.mm}.

Note that the notation $(A+B)$ (and similarly $(A\cdot B)$) specifies a class
called an {\em operation},\index{operation} and is the function value of the
class $+$ at ordered pair $\langle A,B \rangle$.  An operation is defined by
statement \texttt{df-opr} on p.~\pageref{dfopr}.
The notation $A <_{\mathbb{R}} B$ specifies a
wff called a {\em binary relation}\index{binary relation} and means $\langle A,B \rangle \in \,<_{\mathbb{R}}$, as defined by statement \texttt{df-br} on p.~\pageref{dfbr}.

Our set of 8 given classes is assumed to satisfy the following 22 axioms
(in the axioms listed below, $<$ really means $<_{\mathbb{R}}$).

\vskip 2ex

\noindent 1. The real numbers are a subset of the complex numbers.

%\vskip 0.5ex
\setbox\startprefix=\hbox{\tt \ \ ax-resscn\ \$p\ }
\setbox\contprefix=\hbox{\tt \ \ \ \ \ \ \ \ \ \ \ \ \ \ }
\startm
\m{\vdash}\m{\mathbb{R}}\m{\subseteq}\m{\mathbb{C}}
\endm
%\vskip 1ex

\noindent 2. One is a complex number.

%\vskip 0.5ex
\setbox\startprefix=\hbox{\tt \ \ ax-1cn\ \$p\ }
\setbox\contprefix=\hbox{\tt \ \ \ \ \ \ \ \ \ \ \ }
\startm
\m{\vdash}\m{1}\m{\in}\m{\mathbb{C}}
\endm
%\vskip 1ex

\noindent 3. The imaginary number $i$ is a complex number.

%\vskip 0.5ex
\setbox\startprefix=\hbox{\tt \ \ ax-icn\ \$p\ }
\setbox\contprefix=\hbox{\tt \ \ \ \ \ \ \ \ \ \ \ }
\startm
\m{\vdash}\m{i}\m{\in}\m{\mathbb{C}}
\endm
%\vskip 1ex

\noindent 4. Complex numbers are closed under addition.

%\vskip 0.5ex
\setbox\startprefix=\hbox{\tt \ \ ax-addcl\ \$p\ }
\setbox\contprefix=\hbox{\tt \ \ \ \ \ \ \ \ \ \ \ \ \ }
\startm
\m{\vdash}\m{(}\m{(}\m{A}\m{\in}\m{\mathbb{C}}\m{\wedge}\m{B}\m{\in}\m{\mathbb{C}}%
\m{)}\m{\rightarrow}\m{(}\m{A}\m{+}\m{B}\m{)}\m{\in}\m{\mathbb{C}}\m{)}
\endm
%\vskip 1ex

\noindent 5. Real numbers are closed under addition.

%\vskip 0.5ex
\setbox\startprefix=\hbox{\tt \ \ ax-addrcl\ \$p\ }
\setbox\contprefix=\hbox{\tt \ \ \ \ \ \ \ \ \ \ \ \ \ \ }
\startm
\m{\vdash}\m{(}\m{(}\m{A}\m{\in}\m{\mathbb{R}}\m{\wedge}\m{B}\m{\in}\m{\mathbb{R}}%
\m{)}\m{\rightarrow}\m{(}\m{A}\m{+}\m{B}\m{)}\m{\in}\m{\mathbb{R}}\m{)}
\endm
%\vskip 1ex

\noindent 6. Complex numbers are closed under multiplication.

%\vskip 0.5ex
\setbox\startprefix=\hbox{\tt \ \ ax-mulcl\ \$p\ }
\setbox\contprefix=\hbox{\tt \ \ \ \ \ \ \ \ \ \ \ \ \ }
\startm
\m{\vdash}\m{(}\m{(}\m{A}\m{\in}\m{\mathbb{C}}\m{\wedge}\m{B}\m{\in}\m{\mathbb{C}}%
\m{)}\m{\rightarrow}\m{(}\m{A}\m{\cdot}\m{B}\m{)}\m{\in}\m{\mathbb{C}}\m{)}
\endm
%\vskip 1ex

\noindent 7. Real numbers are closed under multiplication.

%\vskip 0.5ex
\setbox\startprefix=\hbox{\tt \ \ ax-mulrcl\ \$p\ }
\setbox\contprefix=\hbox{\tt \ \ \ \ \ \ \ \ \ \ \ \ \ \ }
\startm
\m{\vdash}\m{(}\m{(}\m{A}\m{\in}\m{\mathbb{R}}\m{\wedge}\m{B}\m{\in}\m{\mathbb{R}}%
\m{)}\m{\rightarrow}\m{(}\m{A}\m{\cdot}\m{B}\m{)}\m{\in}\m{\mathbb{R}}\m{)}
\endm
%\vskip 1ex

\noindent 8. Multiplication of complex numbers is commutative.

%\vskip 0.5ex
\setbox\startprefix=\hbox{\tt \ \ ax-mulcom\ \$p\ }
\setbox\contprefix=\hbox{\tt \ \ \ \ \ \ \ \ \ \ \ \ \ \ }
\startm
\m{\vdash}\m{(}\m{(}\m{A}\m{\in}\m{\mathbb{C}}\m{\wedge}\m{B}\m{\in}\m{\mathbb{C}}%
\m{)}\m{\rightarrow}\m{(}\m{A}\m{\cdot}\m{B}\m{)}\m{=}\m{(}\m{B}\m{\cdot}\m{A}%
\m{)}\m{)}
\endm
%\vskip 1ex

\noindent 9. Addition of complex numbers is associative.

%\vskip 0.5ex
\setbox\startprefix=\hbox{\tt \ \ ax-addass\ \$p\ }
\setbox\contprefix=\hbox{\tt \ \ \ \ \ \ \ \ \ \ \ \ \ \ }
\startm
\m{\vdash}\m{(}\m{(}\m{A}\m{\in}\m{\mathbb{C}}\m{\wedge}\m{B}\m{\in}\m{\mathbb{C}}%
\m{\wedge}\m{C}\m{\in}\m{\mathbb{C}}\m{)}\m{\rightarrow}\m{(}\m{(}\m{A}\m{+}%
\m{B}\m{)}\m{+}\m{C}\m{)}\m{=}\m{(}\m{A}\m{+}\m{(}\m{B}\m{+}\m{C}\m{)}\m{)}%
\m{)}
\endm
%\vskip 1ex

\noindent 10. Multiplication of complex numbers is associative.

%\vskip 0.5ex
\setbox\startprefix=\hbox{\tt \ \ ax-mulass\ \$p\ }
\setbox\contprefix=\hbox{\tt \ \ \ \ \ \ \ \ \ \ \ \ \ \ }
\startm
\m{\vdash}\m{(}\m{(}\m{A}\m{\in}\m{\mathbb{C}}\m{\wedge}\m{B}\m{\in}\m{\mathbb{C}}%
\m{\wedge}\m{C}\m{\in}\m{\mathbb{C}}\m{)}\m{\rightarrow}\m{(}\m{(}\m{A}\m{\cdot}%
\m{B}\m{)}\m{\cdot}\m{C}\m{)}\m{=}\m{(}\m{A}\m{\cdot}\m{(}\m{B}\m{\cdot}\m{C}%
\m{)}\m{)}\m{)}
\endm
%\vskip 1ex

\noindent 11. Multiplication distributes over addition for complex numbers.

%\vskip 0.5ex
\setbox\startprefix=\hbox{\tt \ \ ax-distr\ \$p\ }
\setbox\contprefix=\hbox{\tt \ \ \ \ \ \ \ \ \ \ \ \ \ }
\startm
\m{\vdash}\m{(}\m{(}\m{A}\m{\in}\m{\mathbb{C}}\m{\wedge}\m{B}\m{\in}\m{\mathbb{C}}%
\m{\wedge}\m{C}\m{\in}\m{\mathbb{C}}\m{)}\m{\rightarrow}\m{(}\m{A}\m{\cdot}\m{(}%
\m{B}\m{+}\m{C}\m{)}\m{)}\m{=}\m{(}\m{(}\m{A}\m{\cdot}\m{B}\m{)}\m{+}\m{(}%
\m{A}\m{\cdot}\m{C}\m{)}\m{)}\m{)}
\endm
%\vskip 1ex

\noindent 12. The square of $i$ equals $-1$ (expressed as $i$-squared plus 1 is
0).

%\vskip 0.5ex
\setbox\startprefix=\hbox{\tt \ \ ax-i2m1\ \$p\ }
\setbox\contprefix=\hbox{\tt \ \ \ \ \ \ \ \ \ \ \ \ }
\startm
\m{\vdash}\m{(}\m{(}\m{i}\m{\cdot}\m{i}\m{)}\m{+}\m{1}\m{)}\m{=}\m{0}
\endm
%\vskip 1ex

\noindent 13. One and zero are distinct.

%\vskip 0.5ex
\setbox\startprefix=\hbox{\tt \ \ ax-1ne0\ \$p\ }
\setbox\contprefix=\hbox{\tt \ \ \ \ \ \ \ \ \ \ \ \ }
\startm
\m{\vdash}\m{1}\m{\ne}\m{0}
\endm
%\vskip 1ex

\noindent 14. One is an identity element for real multiplication.

%\vskip 0.5ex
\setbox\startprefix=\hbox{\tt \ \ ax-1rid\ \$p\ }
\setbox\contprefix=\hbox{\tt \ \ \ \ \ \ \ \ \ \ \ }
\startm
\m{\vdash}\m{(}\m{A}\m{\in}\m{\mathbb{R}}\m{\rightarrow}\m{(}\m{A}\m{\cdot}\m{1}%
\m{)}\m{=}\m{A}\m{)}
\endm
%\vskip 1ex

\noindent 15. Every real number has a negative.

%\vskip 0.5ex
\setbox\startprefix=\hbox{\tt \ \ ax-rnegex\ \$p\ }
\setbox\contprefix=\hbox{\tt \ \ \ \ \ \ \ \ \ \ \ \ \ \ }
\startm
\m{\vdash}\m{(}\m{A}\m{\in}\m{\mathbb{R}}\m{\rightarrow}\m{\exists}\m{x}\m{\in}%
\m{\mathbb{R}}\m{(}\m{A}\m{+}\m{x}\m{)}\m{=}\m{0}\m{)}
\endm
%\vskip 1ex

\noindent 16. Every nonzero real number has a reciprocal.

%\vskip 0.5ex
\setbox\startprefix=\hbox{\tt \ \ ax-rrecex\ \$p\ }
\setbox\contprefix=\hbox{\tt \ \ \ \ \ \ \ \ \ \ \ \ \ \ }
\startm
\m{\vdash}\m{(}\m{A}\m{\in}\m{\mathbb{R}}\m{\rightarrow}\m{(}\m{A}\m{\ne}\m{0}%
\m{\rightarrow}\m{\exists}\m{x}\m{\in}\m{\mathbb{R}}\m{(}\m{A}\m{\cdot}%
\m{x}\m{)}\m{=}\m{1}\m{)}\m{)}
\endm
%\vskip 1ex

\noindent 17. A complex number can be expressed in terms of two reals.

%\vskip 0.5ex
\setbox\startprefix=\hbox{\tt \ \ ax-cnre\ \$p\ }
\setbox\contprefix=\hbox{\tt \ \ \ \ \ \ \ \ \ \ \ \ }
\startm
\m{\vdash}\m{(}\m{A}\m{\in}\m{\mathbb{C}}\m{\rightarrow}\m{\exists}\m{x}\m{\in}%
\m{\mathbb{R}}\m{\exists}\m{y}\m{\in}\m{\mathbb{R}}\m{A}\m{=}\m{(}\m{x}\m{+}\m{(}%
\m{y}\m{\cdot}\m{i}\m{)}\m{)}\m{)}
\endm
%\vskip 1ex

\noindent 18. Ordering on reals satisfies strict trichotomy.

%\vskip 0.5ex
\setbox\startprefix=\hbox{\tt \ \ ax-pre-lttri\ \$p\ }
\setbox\contprefix=\hbox{\tt \ \ \ \ \ \ \ \ \ \ \ \ \ }
\startm
\m{\vdash}\m{(}\m{(}\m{A}\m{\in}\m{\mathbb{R}}\m{\wedge}\m{B}\m{\in}\m{\mathbb{R}}%
\m{)}\m{\rightarrow}\m{(}\m{A}\m{<}\m{B}\m{\leftrightarrow}\m{\lnot}\m{(}\m{A}%
\m{=}\m{B}\m{\vee}\m{B}\m{<}\m{A}\m{)}\m{)}\m{)}
\endm
%\vskip 1ex

\noindent 19. Ordering on reals is transitive.

%\vskip 0.5ex
\setbox\startprefix=\hbox{\tt \ \ ax-pre-lttrn\ \$p\ }
\setbox\contprefix=\hbox{\tt \ \ \ \ \ \ \ \ \ \ \ \ \ }
\startm
\m{\vdash}\m{(}\m{(}\m{A}\m{\in}\m{\mathbb{R}}\m{\wedge}\m{B}\m{\in}\m{\mathbb{R}}%
\m{\wedge}\m{C}\m{\in}\m{\mathbb{R}}\m{)}\m{\rightarrow}\m{(}\m{(}\m{A}\m{<}%
\m{B}\m{\wedge}\m{B}\m{<}\m{C}\m{)}\m{\rightarrow}\m{A}\m{<}\m{C}\m{)}\m{)}
\endm
%\vskip 1ex

\noindent 20. Ordering on reals is preserved after addition to both sides.

%\vskip 0.5ex
\setbox\startprefix=\hbox{\tt \ \ ax-pre-ltadd\ \$p\ }
\setbox\contprefix=\hbox{\tt \ \ \ \ \ \ \ \ \ \ \ \ \ }
\startm
\m{\vdash}\m{(}\m{(}\m{A}\m{\in}\m{\mathbb{R}}\m{\wedge}\m{B}\m{\in}\m{\mathbb{R}}%
\m{\wedge}\m{C}\m{\in}\m{\mathbb{R}}\m{)}\m{\rightarrow}\m{(}\m{A}\m{<}\m{B}\m{%
\rightarrow}\m{(}\m{C}\m{+}\m{A}\m{)}\m{<}\m{(}\m{C}\m{+}\m{B}\m{)}\m{)}\m{)}
\endm
%\vskip 1ex

\noindent 21. The product of two positive reals is positive.

%\vskip 0.5ex
\setbox\startprefix=\hbox{\tt \ \ ax-pre-mulgt0\ \$p\ }
\setbox\contprefix=\hbox{\tt \ \ \ \ \ \ \ \ \ \ \ \ \ \ }
\startm
\m{\vdash}\m{(}\m{(}\m{A}\m{\in}\m{\mathbb{R}}\m{\wedge}\m{B}\m{\in}\m{\mathbb{R}}%
\m{)}\m{\rightarrow}\m{(}\m{(}\m{0}\m{<}\m{A}\m{\wedge}\m{0}%
\m{<}\m{B}\m{)}\m{\rightarrow}\m{0}\m{<}\m{(}\m{A}\m{\cdot}\m{B}\m{)}%
\m{)}\m{)}
\endm
%\vskip 1ex

\noindent 22. A non-empty, bounded-above set of reals has a supremum.

%\vskip 0.5ex
\setbox\startprefix=\hbox{\tt \ \ ax-pre-sup\ \$p\ }
\setbox\contprefix=\hbox{\tt \ \ \ \ \ \ \ \ \ \ \ }
\startm
\m{\vdash}\m{(}\m{(}\m{A}\m{\subseteq}\m{\mathbb{R}}\m{\wedge}\m{A}\m{\ne}\m{%
\varnothing}\m{\wedge}\m{\exists}\m{x}\m{\in}\m{\mathbb{R}}\m{\forall}\m{y}\m{%
\in}\m{A}\m{\,y}\m{<}\m{x}\m{)}\m{\rightarrow}\m{\exists}\m{x}\m{\in}\m{%
\mathbb{R}}\m{(}\m{\forall}\m{y}\m{\in}\m{A}\m{\lnot}\m{x}\m{<}\m{y}\m{\wedge}\m{%
\forall}\m{y}\m{\in}\m{\mathbb{R}}\m{(}\m{y}\m{<}\m{x}\m{\rightarrow}\m{\exists}%
\m{z}\m{\in}\m{A}\m{\,y}\m{<}\m{z}\m{)}\m{)}\m{)}
\endm

% NOTE: The \m{...} expressions above could be represented as
% $ \vdash ( ( A \subseteq \mathbb{R} \wedge A \ne \varnothing \wedge \exists x \in \mathbb{R} \forall y \in A \,y < x ) \rightarrow \exists x \in \mathbb{R} ( \forall y \in A \lnot x < y \wedge \forall y \in \mathbb{R} ( y < x \rightarrow \exists z \in A \,y < z ) ) ) $

\vskip 2ex

This completes the set of axioms for real and complex numbers.  You may
wish to look at how subtraction, division, and decimal numbers
are defined in \texttt{set.mm}, and for fun look at the proof of $2+
2 = 4$ (theorem \texttt{2p2e4} in \texttt{set.mm})
as discussed in section \ref{2p2e4}.

In \texttt{set.mm} we define the non-negative integers $\mathbb{N}$, the integers
$\mathbb{Z}$, and the rationals $\mathbb{Q}$ as subsets of $\mathbb{R}$.  This leads
to the nice inclusion $\mathbb{N} \subseteq \mathbb{Z} \subseteq \mathbb{Q} \subseteq
\mathbb{R} \subseteq \mathbb{C}$, giving us a uniform framework in which, for
example, a property such as commutativity of complex number addition
automatically applies to integers.  The natural numbers $\mathbb{N}$
are different from the set $\omega$ we defined earlier, but both satisfy
Peano's postulates.

\subsection{Complex Number Axioms in Analysis Texts}

Most analysis texts construct complex numbers as ordered pairs of reals,
leading to construction-dependent properties that satisfy these axioms
but are not stated in their pure form.  (This is also done in
\texttt{set.mm} but our axioms are extracted from that construction.)
Other texts will simply state that $\mathbb{R}$ is a ``complete ordered
subfield of $\mathbb{C}$,'' leading to redundant axioms when this phrase
is completely expanded out.  In fact I have not seen a text with the
axioms in the explicit form above.
None of these axioms is unique individually, but this carefully worked out
collection of axioms is the result of years of work
by the Metamath community.

\subsection{Eliminating Unnecessary Complex Number Axioms}

We once had more axioms for real and complex numbers, but over years of time
we (the Metamath community)
have found ways to eliminate them (by proving them from other axioms)
or weaken them (by making weaker claims without reducing what
can be proved).
In particular, here are statements that used to be complex number
axioms but have since been formally proven (with Metamath) to be redundant:

\begin{itemize}
\item
  $\mathbb{C} \in V$.
  At one time this was listed as a ``complex number axiom.''
  However, this is not properly speaking a complex number axiom,
  and in any case its proof uses axioms of set theory.
  Proven redundant by Mario Carneiro\index{Carneiro, Mario} on
  17-Nov-2014 (see \texttt{axcnex}).
\item
  $((A \in \mathbb{C} \land B \in \mathbb{C}$) $\rightarrow$
  $(A + B) = (B + A))$.
  Proved redundant by Eric Schmidt\index{Schmidt, Eric} on 19-Jun-2012,
  and formalized by Scott Fenton\index{Fenton, Scott} on 3-Jan-2013
  (see \texttt{addcom}).
\item
  $(A \in \mathbb{C} \rightarrow (A + 0) = A)$.
  Proved redundant by Eric Schmidt on 19-Jun-2012,
  and formalized by Scott Fenton on 3-Jan-2013
  (see \texttt{addid1}).
\item
  $(A \in \mathbb{C} \rightarrow \exists x \in \mathbb{C} (A + x) = 0)$.
  Proved redundant by Eric Schmidt and formalized on 21-May-2007
  (see \texttt{cnegex}).
\item
  $((A \in \mathbb{C} \land A \ne 0) \rightarrow \exists x \in \mathbb{C} (A \cdot x) = 1)$.
  Proved redundant by Eric Schmidt and formalized on 22-May-2007
  (see \texttt{recex}).
\item
  $0 \in \mathbb{R}$.
  Proved redundant by Eric Schmidt on 19-Feb-2005 and formalized 21-May-2007
  (see \texttt{0re}).
\end{itemize}

We could eliminate 0 as an axiomatic object by defining it as
$( ( i \cdot i ) + 1 )$
and replacing it with this expression throughout the axioms. If this
is done, axiom ax-i2m1 becomes redundant. However, the remaining axioms
would become longer and less intuitive.

Eric Schmidt's paper analyzing this axiom system \cite{Schmidt}
presented a proof that these remaining axioms,
with the possible exception of ax-mulcom, are independent of the others.
It is currently an open question if ax-mulcom is independent of the others.

\section{Two Plus Two Equals Four}\label{2p2e4}

Here is a proof that $2 + 2 = 4$, as proven in the theorem \texttt{2p2e4}
in the database \texttt{set.mm}.
This is a useful demonstration of what a Metamath proof can look like.
This proof may have more steps than you're used to, but each step is rigorously
proven all the way back to the axioms of logic and set theory.
This display was originally generated by the Metamath program
as an {\sc HTML} file.

In the table showing the proof ``Step'' is the sequential step number,
while its associated ``Expression'' is an expression that we have proved.
``Ref'' is the name of a theorem or axiom that justifies that expression,
and ``Hyp'' refers to previous steps (if any) that the theorem or axiom
needs so that we can use it.  Expressions are indented further than
the expressions that depend on them to show their interdependencies.

\begin{table}[!htbp]
\caption{Two plus two equals four}
\begin{tabular}{lllll}
\textbf{Step} & \textbf{Hyp} & \textbf{Ref} & \textbf{Expression} & \\
1  &       & df-2    & $ \; \; \vdash 2 = 1 + 1$  & \\
2  & 1     & oveq2i  & $ \; \vdash (2 + 2) = (2 + (1 + 1))$ & \\
3  &       & df-4    & $ \; \; \vdash 4 = (3 + 1)$ & \\
4  &       & df-3    & $ \; \; \; \vdash 3 = (2 + 1)$ & \\
5  & 4     & oveq1i  & $ \; \; \vdash (3 + 1) = ((2 + 1) + 1)$ & \\
6  &       & 2cn     & $ \; \; \; \vdash 2 \in \mathbb{C}$ & \\
7  &       & ax-1cn  & $ \; \; \; \vdash 1 \in \mathbb{C}$ & \\
8  & 6,7,7 & addassi & $ \; \; \vdash ((2 + 1) + 1) = (2 + (1 + 1))$ & \\
9  & 3,5,8 & 3eqtri  & $ \; \vdash 4 = (2 + (1 + 1))$ & \\
10 & 2,9   & eqtr4i  & $ \vdash (2 + 2) = 4$ & \\
\end{tabular}
\end{table}

Step 1 says that we can assert that $2 = 1 + 1$ because it is
justified by \texttt{df-2}.
What is \texttt{df-2}?
It is simply the definition of $2$, which in our system is defined as being
equal to $1 + 1$.  This shows how we can use definitions in proofs.

Look at Step 2 of the proof. In the Ref column, we see that it references
a previously proved theorem, \texttt{oveq2i}.
It turns out that
theorem \texttt{oveq2i} requires a
hypothesis, and in the Hyp column of Step 2 we indicate that Step 1 will
satisfy (match) this hypothesis.
If we looked at \texttt{oveq2i}
we would find that it proves that given some hypothesis
$A = B$, we can prove that $( C F A ) = ( C F B )$.
If we use \texttt{oveq2i} and apply step 1's result as the hypothesis,
that will mean that $A = 2$ and $B = ( 1 + 1 )$ within this use of
\texttt{oveq2i}.
We are free to select any value of $C$ and $F$ (subject to syntax constraints),
so we are free to select $C = 2$ and $F = +$,
producing our desired result,
$ (2 + 2) = (2 + (1 + 1))$.

Step 2 is an example of substitution.
In the end, every step in every proof uses only this one substitution rule.
All the rules of logic, and all the axioms, are expressed so that
they can be used via this one substitution rule.
So once you master substitution, you can master every Metamath proof,
no exceptions.

Each step is clear and can be immediately checked.
In the {\sc HTML} display you can even click on each reference to see why it is
justified, making it easy to see why the proof works.

\section{Deduction}\label{deduction}

Strictly speaking,
a deduction (also called an inference) is a kind of statement that needs
some hypotheses to be true in order for its conclusion to be true.
A theorem, on the other hand, has no hypotheses.
Informally we often call both of them theorems, but in this section we
will stick to the strict definitions.

It sometimes happens that we have proved a deduction of the form
$\varphi \Rightarrow \psi$\index{$\Rightarrow$}
(given hypothesis $\varphi$ we can prove $\psi$)
and we want to then prove a theorem of the form
$\varphi \rightarrow \psi$.

Converting a deduction (which uses a hypothesis) into a theorem
(which does not) is not as simple as you might think.
The deduction says, ``if we can prove $\varphi$ then we can prove $\psi$,''
which is in some sense weaker than saying
``$\varphi$ implies $\psi$.''
There is no axiom of logic that permits us to directly obtain the theorem
given the deduction.\footnote{
The conversion of a deduction to a theorem does not even hold in general
for quantum propositional calculus,
which is a weak subset of classical propositional calculus.
It has been shown that adding the Standard Deduction Theorem (discussed below)
to quantum propositional calculus turns it into classical
propositional calculus!
}

This is in contrast to going the other way.
If we have the theorem ($\varphi \rightarrow \psi$),
it is easy to recover the deduction
($\varphi \Rightarrow \psi$)
using modus ponens\index{modus ponens}
(\texttt{ax-mp}; see section \ref{axmp}).

In the following subsections we first discuss the standard deduction theorem
(the traditional but awkward way to convert deductions into theorems) and
the weak deduction theorem (a limited version of the standard deduction
theorem that is easier to use and was once widely used in
\texttt{set.mm}\index{set theory database (\texttt{set.mm})}).
In section \ref{deductionstyle} we discuss
deduction style, the newer approach we now recommend in most cases.
Deduction style uses ``deduction form,'' a form that
prefixes each hypothesis (other than definitions) and the
conclusion with a universal antecedent (``$\varphi \rightarrow$'').
Deduction style is widely used in \texttt{set.mm},
so it is useful to understand it and \textit{why} it is widely used.
Section \ref{naturaldeduction}
briefly discusses our approach for using natural deduction
within \texttt{set.mm},
as that approach is deeply related to deduction style.
We conclude with a summary of the strengths of
our approach, which we believe are compelling.

\subsection{The Standard Deduction Theorem}\label{standarddeductiontheorem}

It is possible to make use of information
contained in the deduction or its proof to assist us with the proof of
the related theorem.
In traditional logic books, there is a metatheorem called the
Deduction Theorem\index{Deduction Theorem}\index{Standard Deduction Theorem},
discovered independently by Herbrand and Tarski around 1930.
The Deduction Theorem, which we often call the Standard Deduction Theorem,
provides an algorithm for constructing a proof of a theorem from the
proof of its corresponding deduction. See, for example,
\cite[p.~56]{Margaris}\index{Margaris, Angelo}.
To construct a proof for a theorem, the
algorithm looks at each step in the proof of the original deduction and
rewrites the step with several steps wherein the hypothesis is eliminated
and becomes an antecedent.

In ordinary mathematics, no one actually carries out the algorithm,
because (in its most basic form) it involves an exponential explosion of
the number of proof steps as more hypotheses are eliminated. Instead,
the Standard Deduction Theorem is invoked simply to claim that it can
be done in principle, without actually doing it.
What's more, the algorithm is not as simple as it might first appear
when applying it rigorously.
There is a subtle restriction on the Standard Deduction Theorem
that must be taken into account involving the axiom of generalization
when working with predicate calculus (see the literature for more detail).

One of the goals of Metamath is to let you plainly see, with as few
underlying concepts as possible, how mathematics can be derived directly
from the axioms, and not indirectly according to some hidden rules
buried inside a program or understood only by logicians. If we added
the Standard Deduction Theorem to the language and proof verifier,
that would greatly complicate both and largely defeat Metamath's goal
of simplicity. In principle, we could show direct proofs by expanding
out the proof steps generated by the algorithm of the Standard Deduction
Theorem, but that is not feasible in practice because the number of proof
steps quickly becomes huge, even astronomical.
Since the algorithm of the Standard Deduction Theorem is driven by the proof,
we would have to go through that proof
all over again---starting from axioms---in order to obtain the theorem form.
In terms of proof length, there would be no savings over just
proving the theorem directly instead of first proving the deduction form.

\subsection{Weak Deduction Theorem}\label{weakdeductiontheorem}

We have developed
a more efficient method for proving a theorem from a deduction
that can be used instead of the Standard Deduction Theorem
in many (but not all) cases.
We call this more efficient method the
Weak Deduction Theorem\index{Weak Deduction Theorem}.\footnote{
There is also an unrelated ``Weak Deduction Theorem''
in the field of relevance logic, so to avoid confusion we could call
ours the ``Weak Deduction Theorem for Classical Logic.''}
Unlike the Standard Deduction Theorem, the Weak Deduction Theorem produces the
theorem directly from a special substitution instance of the deduction,
using a small, fixed number of steps roughly proportional to the length
of the final theorem.

If you come to a proof referencing the Weak Deduction Theorem
\texttt{dedth} (or one of its variants \texttt{dedthxx}),
here is how to follow the proof without getting into the details:
just click on the theorem referenced in the step
just before the reference to \texttt{dedth} and ignore everything else.
Theorem \texttt{dedth} simply turns a hypothesis into an antecedent
(i.e. the hypothesis followed by $\rightarrow$
is placed in front of the assertion, and the hypothesis
itself is eliminated) given certain conditions.

The Weak Deduction Theorem
eliminates a hypothesis $\varphi$, making it become an antecedent.
It does this by proving an expression
$ \varphi \rightarrow \psi $ given two hypotheses:
(1)
$ ( A = {\rm if} ( \varphi , A , B ) \rightarrow ( \varphi \leftrightarrow \chi ) ) $
and
(2) $\chi$.
Note that it requires that a proof exists for $\varphi$ when the class variable
$A$ is replaced with a specific class $B$. The hypothesis $\chi$
should be assigned to the inference.
You can see the details of the proof of the Weak Deduction Theorem
in theorem \texttt{dedth}.

The Weak Deduction Theorem
is probably easier to understand by studying proofs that make use of it.
For example, let's look at the proof of \texttt{renegcl}, which proves that
$ \vdash ( A \in \mathbb{R} \rightarrow - A \in \mathbb{R} )$:

\needspace{4\baselineskip}
\begin{longtabu} {l l l X}
\textbf{Step} & \textbf{Hyp} & \textbf{Ref} & \textbf{Expression} \\
  1 &  & negeq &
  $\vdash$ $($ $A$ $=$ ${\rm if}$ $($ $A$ $\in$ $\mathbb{R}$ $,$ $A$ $,$ $1$ $)$ $\rightarrow$
  $\textrm{-}$ $A$ $=$ $\textrm{-}$ ${\rm if}$ $($ $A$ $\in$ $\mathbb{R}$
  $,$ $A$ $,$ $1$ $)$ $)$ \\
 2 & 1 & eleq1d &
    $\vdash$ $($ $A$ $=$ ${\rm if}$ $($ $A$ $\in$ $\mathbb{R}$ $,$ $A$ $,$ $1$ $)$ $\rightarrow$ $($
    $\textrm{-}$ $A$ $\in$ $\mathbb{R}$ $\leftrightarrow$
    $\textrm{-}$ ${\rm if}$ $($ $A$ $\in$ $\mathbb{R}$ $,$ $A$ $,$ $1$ $)$ $\in$
    $\mathbb{R}$ $)$ $)$ \\
 3 &  & 1re & $\vdash 1 \in \mathbb{R}$ \\
 4 & 3 & elimel &
   $\vdash {\rm if} ( A \in \mathbb{R} , A , 1 ) \in \mathbb{R}$ \\
 5 & 4 & renegcli &
   $\vdash \textrm{-} {\rm if} ( A \in \mathbb{R} , A , 1 ) \in \mathbb{R}$ \\
 6 & 2,5 & dedth &
   $\vdash ( A \in \mathbb{R} \rightarrow \textrm{-} A \in \mathbb{R}$ ) \\
\end{longtabu}

The somewhat strange-looking steps in \texttt{renegcl} before step 5 are
technical stuff that makes this magic work, and they can be ignored
for a quick overview of the proof. To continue following the ``important''
part of the proof of \texttt{renegcl},
you can look at the reference to \texttt{renegcli} at step 5.

That said, let's briefly look at how
\texttt{renegcl} uses the
Weak Deduction Theorem (\texttt{dedth}) to do its job,
in case you want to do something similar or want understand it more deeply.
Let's work backwards in the proof of \texttt{renegcl}.
Step 6 applies \texttt{dedth} to produce our goal result
$ \vdash ( A \in \mathbb{R} \rightarrow\, - A \in \mathbb{R} )$.
This requires on the one hand the (substituted) deduction
\texttt{renegcli} in step 5.
By itself \texttt{renegcli} proves the deduction
$ \vdash A \in \mathbb{R} \Rightarrow\, \vdash - A \in \mathbb{R}$;
this is the deduction form we are trying to turn into theorem form,
and thus
\texttt{renegcli} has a separate hypothesis that must be fulfilled.
To fulfill the hypothesis of the invocation of
\texttt{renegcli} in step 5, it is eventually
reduced to the already proven theorem $1 \in \mathbb{R}$ in step 3.
Step 4 connects steps 3 and 5; step 4 invokes
\texttt{elimel}, a special case of \texttt{elimhyp} that eliminates
a membership hypothesis for the weak deduction theorem.
On the other hand, the equivalence of the conclusion of
\texttt{renegcl}
$( - A \in \mathbb{R} )$ and the substituted conclusion of
\texttt{renegcli} must be proven, which is done in steps 2 and 1.

The weak deduction theorem has limitations.
In particular, we must be able to prove a special case of the deduction's
hypothesis as a stand-alone theorem.
For example, we used $1 \in \mathbb{R}$ in step 3 of \texttt{renegcl}.

We used to use the weak deduction theorem
extensively within \texttt{set.mm}.
However, we now recommend applying ``deduction style''
instead in most cases, as deduction style is
often an easier and clearer approach.
Therefore, we will now describe deduction style.

\subsection{Deduction Style}\label{deductionstyle}

We now prefer to write assertions in ``deduction form''
instead of writing a proof that would require use of the standard or
weak deduction theorem.
We call this appraoch
``deduction style.''\index{deduction style}

It will be easier to explain this by first defining some terms:

\begin{itemize}
\item \textbf{closed form}\index{closed form}\index{forms!closed}:
A kind of assertion (theorem) with no hypotheses.
Typically its label has no special suffix.
An example is \texttt{unss}, which states:
$\vdash ( ( A \subseteq C \wedge B \subseteq C ) \leftrightarrow ( A \cup B )
\subseteq C )\label{eq:unss}$
\item \textbf{deduction form}\index{deduction form}\index{forms!deduction}:
A kind of assertion with one or more hypotheses
where the conclusion is an implication with
a wff variable as the antecedent (usually $\varphi$), and every hypothesis
(\$e statement)
is either (1) an implication with the same antecedent as the conclusion or
(2) a definition.
A definition
can be for a class variable (this is a class variable followed by ``='')
or a wff variable (this is a wff variable followed by $\leftrightarrow$);
class variable definitions are more common.
In practice, a proof
in deduction form will also contain many steps that are implications
where the antecedent is either that wff variable (normally $\varphi$)
or is
a conjunction (...$\land$...) including that wff variable ($\varphi$).
If an assertion is in deduction form, and other forms are also available,
then we suffix its label with ``d.''
An example is \texttt{unssd}, which states\footnote{
For brevity we show here (and in other places)
a $\&$\index{$\&$} between hypotheses\index{hypotheses}
and a $\Rightarrow$\index{$\Rightarrow$}\index{conclusion}
between the hypotheses and the conclusion.
This notation is technically not part of the Metamath language, but is
instead a convenient abbreviation to show both the hypotheses and conclusion.}:
$\vdash ( \varphi \rightarrow A \subseteq C )\quad\&\quad \vdash ( \varphi
    \rightarrow B \subseteq C )\quad\Rightarrow\quad \vdash ( \varphi
    \rightarrow ( A \cup B ) \subseteq C )\label{eq:unssd}$
\item \textbf{inference form}\index{inference form}\index{forms!inference}:
A kind of assertion with one or more hypotheses that is not in deduction form
(e.g., there is no common antecedent).
If an assertion is in inference form, and other forms are also available,
then we suffix its label with ``i.''
An example is \texttt{unssi}, which states:
$\vdash A \subseteq C\quad\&\quad \vdash B \subseteq C\quad\Rightarrow\quad
    \vdash ( A \cup B ) \subseteq C\label{eq:unssi}$
\end{itemize}

When using deduction style we express an assertion in deduction form.
This form prefixes each hypothesis (other than definitions) and the
conclusion with a universal antecedent (``$\varphi \rightarrow$'').
The antecedent (e.g., $\varphi$)
mimics the context handled in the deduction theorem, eliminating
the need to directly use the deduction theorem.

Once you have an assertion in deduction form, you can easily convert it
to inference form or closed form:

\begin{itemize}
\item To
prove some assertion Ti in inference form, given assertion Td in deduction
form, there is a simple mechanical process you can use. First take each
Ti hypothesis and insert a \texttt{T.} $\rightarrow$ prefix (``true implies'')
using \texttt{a1i}. You
can then use the existing assertion Td to prove the resulting conclusion
with a \texttt{T.} $\rightarrow$ prefix.
Finally, you can remove that prefix using \texttt{mptru},
resulting in the conclusion you wanted to prove.
\item To
prove some assertion T in closed form, given assertion Td in deduction
form, there is another simple mechanical process you can use. First,
select an expression that is the conjunction (...$\land$...) of all of the
consequents of every hypothesis of Td. Next, prove that this expression
implies each of the separate hypotheses of Td in turn by eliminating
conjuncts (there are a variety of proven assertions to do this, including
\texttt{simpl},
\texttt{simpr},
\texttt{3simpa},
\texttt{3simpb},
\texttt{3simpc},
\texttt{simp1},
\texttt{simp2},
and
\texttt{simp3}).
If the
expression has nested conjunctions, inner conjuncts can be broken out by
chaining the above theorems with \texttt{syl}
(see section \ref{syl}).\footnote{
There are actually many theorems
(labeled simp* such as \texttt{simp333}) that break out inner conjuncts in one
step, but rather than learning them you can just use the chaining we
just described to prove them, and then let the Metamath program command
\texttt{minimize{\char`\_}with}\index{\texttt{minimize{\char`\_}with} command}
figure out the right ones needed to collapse them.}
As your final step, you can then apply the already-proven assertion Td
(which is in deduction form), proving assertion T in closed form.
\end{itemize}

We can also easily convert any assertion T in closed form to its related
assertion Ti in inference form by applying
modus ponens\index{modus ponens} (see section \ref{axmp}).

The deduction form antecedent can also be used to represent the context
necessary to support natural deduction systems, so we will now
discuss natural deduction.

\subsection{Natural Deduction}\label{naturaldeduction}

Natural deduction\index{natural deduction}
(ND) systems, as such, were originally introduced in
1934 by two logicians working independently: Ja\'skowski and Gentzen. ND
systems are supposed to reconstruct, in a formally proper way, traditional
ways of mathematical reasoning (such as conditional proof, indirect proof,
and proof by cases). As reconstructions they were naturally influenced
by previous work, and many specific ND systems and notations have been
developed since their original work.

There are many ND variants, but
Indrzejczak \cite[p.~31-32]{Indrzejczak}\index{Indrzejczak, Andrzej}
suggests that any natural deductive system must satisfy at
least these three criteria:

\begin{itemize}
\item ``There are some means for entering assumptions into a proof and
also for eliminating them. Usually it requires some bookkeeping devices
for indicating the scope of an assumption, and showing that a part of
a proof depending on eliminated assumption is discharged.
\item There are no (or, at least, a very limited set of) axioms, because
their role is taken over by the set of primitive rules for introduction
and elimination of logical constants which means that elementary
inferences instead of formulae are taken as primitive.
\item (A genuine) ND system admits a lot of freedom in proof construction
and possibility of applying several proof search strategies, like
conditional proof, proof by cases, proof by reductio ad absurdum etc.''
\end{itemize}

The Metamath Proof Explorer (MPE) as defined in \texttt{set.mm}
is fundamentally a Hilbert-style system.
That is, MPE is based on a larger number of axioms (compared
to natural deduction systems), a very small set of rules of inference
(modus ponens), and the context is not changed by the rules of inference
in the middle of a proof. That said, MPE proofs can be developed using
the natural deduction (ND) approach as originally developed by Ja\'skowski
and Gentzen.

The most common and recommended approach for applying ND in MPE is to use
deduction form\index{deduction form}%
\index{forms!deduction}
and apply the MPE proven assertions that are equivalent to ND rules.
For example, MPE's \texttt{jca} is equivalent to ND rule $\land$-I
(and-insertion).
We maintain a list of equivalences that you may consult.
This approach for applying an ND approach within MPE relies on Metamath's
wff metavariables in an essential way, and is described in more detail
in the presentation ``Natural Deductions in the Metamath Proof Language''
by Mario Carneiro \cite{CarneiroND}\index{Carneiro, Mario}.

In this style many steps are an implication, whose antecedent mimics
the context ($\Gamma$) of most ND systems. To add an assumption, simply add
it to the implication antecedent (typically using
\texttt{simpr}),
and use that
new antecedent for all later claims in the same scope. If you wish to
use an assertion in an ND hypothesis scope that is outside the current
ND hypothesis scope, modify the assertion so that the ND hypothesis
assumption is added to its antecedent (typically using \texttt{adantr}). Most
proof steps will be proved using rules that have hypotheses and results
of the form $\varphi \rightarrow$ ...

An example may make this clearer.
Let's show theorem 5.5 of
\cite[p.~18]{Clemente}\index{Clemente Laboreo, Daniel}
along with a line by line translation using the usual
translation of natural deduction (ND) in the Metamath Proof Explorer
(MPE) notation (this is proof \texttt{ex-natded5.5}).
The proof's original goal was to prove
$\lnot \psi$ given two hypotheses,
$( \psi \rightarrow \chi )$ and $ \lnot \chi$.
We will translate these statements into MPE deduction form
by prefixing them all with $\varphi \rightarrow$.
As a result, in MPE the goal is stated as
$( \varphi \rightarrow \lnot \psi )$, and the two hypotheses are stated as
$( \varphi \rightarrow ( \psi \rightarrow \chi ) )$ and
$( \varphi \rightarrow \lnot \chi )$.

The following table shows the proof in Fitch natural deduction style
and its MPE equivalent.
The \textit{\#} column shows the original numbering,
\textit{MPE\#} shows the number in the equivalent MPE proof
(which we will show later),
\textit{ND Expression} shows the original proof claim in ND notation,
and \textit{MPE Translation} shows its translation into MPE
as discussed in this section.
The final columns show the rationale in ND and MPE respectively.

\needspace{4\baselineskip}
{\setlength{\extrarowsep}{4pt} % Keep rows from being too close together
\begin{longtabu}   { @{} c c X X X X }
\textbf{\#} & \textbf{MPE\#} & \textbf{ND Ex\-pres\-sion} &
\textbf{MPE Trans\-lation} & \textbf{ND Ration\-ale} &
\textbf{MPE Ra\-tio\-nale} \\
\endhead

1 & 2;3 &
$( \psi \rightarrow \chi )$ &
$( \varphi \rightarrow ( \psi \rightarrow \chi ) )$ &
Given &
\$e; \texttt{adantr} to put in ND hypothesis \\

2 & 5 &
$ \lnot \chi$ &
$( \varphi \rightarrow \lnot \chi )$ &
Given &
\$e; \texttt{adantr} to put in ND hypothesis \\

3 & 1 &
... $\vert$ $\psi$ &
$( \varphi \rightarrow \psi )$ &
ND hypothesis assumption &
\texttt{simpr} \\

4 & 4 &
... $\chi$ &
$( ( \varphi \land \psi ) \rightarrow \chi )$ &
$\rightarrow$\,E 1,3 &
\texttt{mpd} 1,3 \\

5 & 6 &
... $\lnot \chi$ &
$( ( \varphi \land \psi ) \rightarrow \lnot \chi )$ &
IT 2 &
\texttt{adantr} 5 \\

6 & 7 &
$\lnot \psi$ &
$( \varphi \rightarrow \lnot \psi )$ &
$\land$\,I 3,4,5 &
\texttt{pm2.65da} 4,6 \\

\end{longtabu}
}


The original used Latin letters; we have replaced them with Greek letters
to follow Metamath naming conventions and so that it is easier to follow
the Metamath translation. The Metamath line-for-line translation of
this natural deduction approach precedes every line with an antecedent
including $\varphi$ and uses the Metamath equivalents of the natural deduction
rules. To add an assumption, the antecedent is modified to include it
(typically by using \texttt{adantr};
\texttt{simpr} is useful when you want to
depend directly on the new assumption, as is shown here).

In Metamath we can represent the two given statements as these hypotheses:

\needspace{2\baselineskip}
\begin{itemize}
\item ex-natded5.5.1 $\vdash ( \varphi \rightarrow ( \psi \rightarrow \chi ) )$
\item ex-natded5.5.2 $\vdash ( \varphi \rightarrow \lnot \chi )$
\end{itemize}

\needspace{4\baselineskip}
Here is the proof in Metamath as a line-by-line translation:

\begin{longtabu}   { l l l X }
\textbf{Step} & \textbf{Hyp} & \textbf{Ref} & \textbf{Ex\-pres\-sion} \\
\endhead
1 & & simpr & $\vdash ( ( \varphi \land \psi ) \rightarrow \psi )$ \\
2 & & ex-natded5.5.1 &
  $\vdash ( \varphi \rightarrow ( \psi \rightarrow \chi ) )$ \\
3 & 2 & adantr &
 $\vdash ( ( \varphi \land \psi ) \rightarrow ( \psi \rightarrow \chi ) )$ \\
4 & 1, 3 & mpd &
 $\vdash ( ( \varphi \land \psi ) \rightarrow \chi ) $ \\
5 & & ex-natded5.5.2 &
 $\vdash ( \varphi \rightarrow \lnot \chi )$ \\
6 & 5 & adantr &
 $\vdash ( ( \varphi \land \psi ) \rightarrow \lnot \chi )$ \\
7 & 4, 6 & pm2.65da &
 $\vdash ( \varphi \rightarrow \lnot \psi )$ \\
\end{longtabu}

Only using specific natural deduction rules directly can lead to very
long proofs, for exactly the same reason that only using axioms directly
in Hilbert-style proofs can lead to very long proofs.
If the goal is short and clear proofs,
then it is better to reuse already-proven assertions
in deduction form than to start from scratch each time
and using only basic natural deduction rules.

\subsection{Strengths of Our Approach}

As far as we know there is nothing else in the literature like either the
weak deduction theorem or Mario Carneiro\index{Carneiro, Mario}'s
natural deduction method.
In order to
transform a hypothesis into an antecedent, the literature's standard
``Deduction Theorem''\index{Deduction Theorem}\index{Standard Deduction Theorem}
requires metalogic outside of the notions provided
by the axiom system. We instead generally prefer to use Mario Carneiro's
natural deduction method, then use the weak deduction theorem in cases
where that is difficult to apply, and only then use the full standard
deduction theorem as a last resort.

The weak deduction theorem\index{Weak Deduction Theorem}
does not require any additional metalogic
but converts an inference directly into a closed form theorem, with
a rigorous proof that uses only the axiom system. Unlike the standard
Deduction Theorem, there is no implicit external justification that we
have to trust in order to use it.

Mario Carneiro's natural deduction\index{natural deduction}
method also does not require any new metalogical
notions. It avoids the Deduction Theorem's metalogic by prefixing the
hypotheses and conclusion of every would-be inference with a universal
antecedent (``$\varphi \rightarrow$'') from the very start.

We think it is impressive and satisfying that we can do so much in a
practical sense without stepping outside of our Hilbert-style axiom system.
Of course our axiomatization, which is in the form of schemes,
contains a metalogic of its own that we exploit. But this metalogic
is relatively simple, and for our Deduction Theorem alternatives,
we primarily use just the direct substitution of expressions for
metavariables.

\begin{sloppy}
\section{Exploring the Set The\-o\-ry Data\-base}\label{exploring}
\end{sloppy}
% NOTE: All examples performed in this section are
% recorded wtih "set width 61" % on set.mm as of 2019-05-28
% commit c1e7849557661260f77cfdf0f97ac4354fbb4f4d.

At this point you may wish to study the \texttt{set.mm}\index{set theory
database (\texttt{set.mm})} file in more detail.  Pay particular
attention to the assumptions needed to define wffs\index{well-formed
formula (wff)} (which are not included above), the variable types
(\texttt{\$f}\index{\texttt{\$f} statement} statements), and the
definitions that are introduced.  Start with some simple theorems in
propositional calculus, making sure you understand in detail each step
of a proof.  Once you get past the first few proofs and become familiar
with the Metamath language, any part of the \texttt{set.mm} database
will be as easy to follow, step by step, as any other part---you won't
have to undergo a ``quantum leap'' in mathematical sophistication to be
able to follow a deep proof in set theory.

Next, you may want to explore how concepts such as natural numbers are
defined and described.  This is probably best done in conjunction with
standard set theory textbooks, which can help give you a higher-level
understanding.  The \texttt{set.mm} database provides references that will get
you started.  From there, you will be on your way towards a very deep,
rigorous understanding of abstract mathematics.

The Metamath\index{Metamath} program can help you peruse a Metamath data\-base,
wheth\-er you are trying to figure out how a certain step follows in a proof or
just have a general curiosity.  We will go through some examples of the
commands, using the \texttt{set.mm}\index{set theory database (\texttt{set.mm})}
database provided with the Metamath software.  These should help get you
started.  See Chapter~\ref{commands} for a more detailed description of
the commands.  Note that we have included the full spelling of all commands to
prevent ambiguity with future commands.  In practice you may type just the
characters needed to specify each command keyword\index{command keyword}
unambiguously, often just one or two characters per keyword, and you don't
need to type them in upper case.

First run the Metamath program as described earlier.  You should see the
\verb/MM>/ prompt.  Read in the \texttt{set.mm} file:\index{\texttt{read}
command}

\begin{verbatim}
MM> read set.mm
Reading source file "set.mm"... 34554442 bytes
34554442 bytes were read into the source buffer.
The source has 155711 statements; 2254 are $a and 32250 are $p.
No errors were found.  However, proofs were not checked.
Type VERIFY PROOF * if you want to check them.
\end{verbatim}

As with most examples in this book, what you will see
will be slightly different because we are continuously
improving our databases (including \texttt{set.mm}).

Let's check the database integrity.  This may take a minute or two to run if
your computer is slow.

\begin{verbatim}
MM> verify proof *
0 10%  20%  30%  40%  50%  60%  70%  80%  90% 100%
..................................................
All proofs in the database were verified in 2.84 s.
\end{verbatim}

No errors were reported, so every proof is correct.

You need to know the names (labels) of theorems before you can look at them.
Often just examining the database file(s) with a text editor is the best
approach.  In \texttt{set.mm} there are many detailed comments, especially near
the beginning, that can help guide you. The \texttt{search} command in the
Metamath program is also handy.  The \texttt{comments} qualifier will list the
statements whose associated comment (the one immediately before it) contain a
string you give it.  For example, if you are studying Enderton's {\em Elements
of Set Theory} \cite{Enderton}\index{Enderton, Herbert B.} you may want to see
the references to it in the database.  The search string \texttt{enderton} is not
case sensitive.  (This will not show you all the database theorems that are in
Enderton's book because there is usually only one citation for a given
theorem, which may appear in several textbooks.)\index{\texttt{search}
command}

\begin{verbatim}
MM> search * "enderton" / comments
12067 unineq $p "... Exercise 20 of [Enderton] p. 32 and ..."
12459 undif2 $p "...Corollary 6K of [Enderton] p. 144. (C..."
12953 df-tp $a "...s. Definition of [Enderton] p. 19. (Co..."
13689 unissb $p ".... Exercise 5 of [Enderton] p. 26 and ..."
\end{verbatim}
\begin{center}
(etc.)
\end{center}

Or you may want to see what theorems have something to do with
conjunction (logical {\sc and}).  The quotes around the search
string are optional when there's no ambiguity.\index{\texttt{search}
command}

\begin{verbatim}
MM> search * conjunction / comments
120 a1d $p "...be replaced with a conjunction ( ~ df-an )..."
662 df-bi $a "...viated form after conjunction is introdu..."
1319 wa $a "...ff definition to include conjunction ('and')."
1321 df-an $a "Define conjunction (logical 'and'). Defini..."
1420 imnan $p "...tion in terms of conjunction. (Contribu..."
\end{verbatim}
\begin{center}
(etc.)
\end{center}


Now we will start to look at some details.  Let's look at the first
axiom of propositional calculus
(we could use \texttt{sh st} to abbreviate
\texttt{show statement}).\index{\texttt{show statement} command}

\begin{verbatim}
MM> show statement ax-1/full
Statement 19 is located on line 881 of the file "set.mm".
"Axiom _Simp_.  Axiom A1 of [Margaris] p. 49.  One of the 3
axioms of propositional calculus.  The 3 axioms are also
        ...
19 ax-1 $a |- ( ph -> ( ps -> ph ) ) $.
Its mandatory hypotheses in RPN order are:
  wph $f wff ph $.
  wps $f wff ps $.
The statement and its hypotheses require the variables:  ph
      ps
The variables it contains are:  ph ps


Statement 49 is located on line 11182 of the file "set.mm".
Its statement number for HTML pages is 6.
"Axiom _Simp_.  Axiom A1 of [Margaris] p. 49.  One of the 3
axioms of propositional calculus.  The 3 axioms are also
given as Definition 2.1 of [Hamilton] p. 28.
...
49 ax-1 $a |- ( ph -> ( ps -> ph ) ) $.
Its mandatory hypotheses in RPN order are:
  wph $f wff ph $.
  wps $f wff ps $.
The statement and its hypotheses require the variables:
  ph ps
The variables it contains are:  ph ps
\end{verbatim}

Compare this to \texttt{ax-1} on p.~\pageref{ax1}.  You can see that the
symbol \texttt{ph} is the {\sc ascii} notation for $\varphi$, etc.  To
see the mathematical symbols for any expression you may typeset it in
\LaTeX\ (type \texttt{help tex} for instructions)\index{latex@{\LaTeX}}
or, easier, just use a text editor to look at the comments where symbols
are first introduced in \texttt{set.mm}.  The hypotheses \texttt{wph}
and \texttt{wps} required by \texttt{ax-1} mean that variables
\texttt{ph} and \texttt{ps} must be wffs.

Next we'll pick a simple theorem of propositional calculus, the Principle of
Identity, which is proved directly from the axioms.  We'll look at the
statement then its proof.\index{\texttt{show statement}
command}

\begin{verbatim}
MM> show statement id1/full
Statement 116 is located on line 11371 of the file "set.mm".
Its statement number for HTML pages is 22.
"Principle of identity.  Theorem *2.08 of [WhiteheadRussell]
p. 101.  This version is proved directly from the axioms for
demonstration purposes.
...
116 id1 $p |- ( ph -> ph ) $= ... $.
Its mandatory hypotheses in RPN order are:
  wph $f wff ph $.
Its optional hypotheses are:  wps wch wth wta wet
      wze wsi wrh wmu wla wka
The statement and its hypotheses require the variables:  ph
These additional variables are allowed in its proof:
      ps ch th ta et ze si rh mu la ka
The variables it contains are:  ph
\end{verbatim}

The optional variables\index{optional variable} \texttt{ps}, \texttt{ch}, etc.\ are
available for use in a proof of this statement if we wish, and were we to do
so we would make use of optional hypotheses \texttt{wps}, \texttt{wch}, etc.  (See
Section~\ref{dollaref} for the meaning of ``optional
hypothesis.''\index{optional hypothesis}) The reason these show up in the
statement display is that statement \texttt{id1} happens to be in their scope
(see Section~\ref{scoping} for the definition of ``scope''\index{scope}), but
in fact in propositional calculus we will never make use of optional
hypotheses or variables.  This becomes important after quantifiers are
introduced, where ``dummy'' variables\index{dummy variable} are often needed
in the middle of a proof.

Let's look at the proof of statement \texttt{id1}.  We'll use the
\texttt{show proof} command, which by default suppresses the
``non-essential'' steps that construct the wffs.\index{\texttt{show proof}
command}
We will display the proof in ``lemmon' format (a non-indented format
with explicit previous step number references) and renumber the
displayed steps:

\begin{verbatim}
MM> show proof id1 /lemmon/renumber
1 ax-1           $a |- ( ph -> ( ph -> ph ) )
2 ax-1           $a |- ( ph -> ( ( ph -> ph ) -> ph ) )
3 ax-2           $a |- ( ( ph -> ( ( ph -> ph ) -> ph ) ) ->
                     ( ( ph -> ( ph -> ph ) ) -> ( ph -> ph )
                                                          ) )
4 2,3 ax-mp      $a |- ( ( ph -> ( ph -> ph ) ) -> ( ph -> ph
                                                          ) )
5 1,4 ax-mp      $a |- ( ph -> ph )
\end{verbatim}

If you have read Section~\ref{trialrun}, you'll know how to interpret this
proof.  Step~2, for example, is an application of axiom \texttt{ax-1}.  This
proof is identical to the one in Hamilton's {\em Logic for Mathematicians}
\cite[p.~32]{Hamilton}\index{Hamilton, Alan G.}.

You may want to look at what
substitutions\index{substitution!variable}\index{variable substitution} are
made into \texttt{ax-1} to arrive at step~2.  The command to do this needs to
know the ``real'' step number, so we'll display the proof again without
the \texttt{renumber} qualifier.\index{\texttt{show proof}
command}

\begin{verbatim}
MM> show proof id1 /lemmon
 9 ax-1          $a |- ( ph -> ( ph -> ph ) )
20 ax-1          $a |- ( ph -> ( ( ph -> ph ) -> ph ) )
24 ax-2          $a |- ( ( ph -> ( ( ph -> ph ) -> ph ) ) ->
                     ( ( ph -> ( ph -> ph ) ) -> ( ph -> ph )
                                                          ) )
25 20,24 ax-mp   $a |- ( ( ph -> ( ph -> ph ) ) -> ( ph -> ph
                                                          ) )
26 9,25 ax-mp    $a |- ( ph -> ph )
\end{verbatim}

The ``real'' step number is 20.  Let's look at its details.

\begin{verbatim}
MM> show proof id1 /detailed_step 20
Proof step 20:  min=ax-1 $a |- ( ph -> ( ( ph -> ph ) -> ph )
  )
This step assigns source "ax-1" ($a) to target "min" ($e).
The source assertion requires the hypotheses "wph" ($f, step
18) and "wps" ($f, step 19).  The parent assertion of the
target hypothesis is "ax-mp" ($a, step 25).
The source assertion before substitution was:
    ax-1 $a |- ( ph -> ( ps -> ph ) )
The following substitutions were made to the source
assertion:
    Variable  Substituted with
     ph        ph
     ps        ( ph -> ph )
The target hypothesis before substitution was:
    min $e |- ph
The following substitution was made to the target hypothesis:
    Variable  Substituted with
     ph        ( ph -> ( ( ph -> ph ) -> ph ) )
\end{verbatim}

This shows the substitutions\index{substitution!variable}\index{variable
substitution} made to the variables in \texttt{ax-1}.  References are made to
steps 18 and 19 which are not shown in our proof display.  To see these steps,
you can display the proof with the \texttt{all} qualifier.

Let's look at a slightly more advanced proof of propositional calculus.  Note
that \verb+/\+ is the symbol for $\wedge$ (logical {\sc and}, also
called conjunction).\index{conjunction ($\wedge$)}
\index{logical {\sc and} ($\wedge$)}

\begin{verbatim}
MM> show statement prth/full
Statement 1791 is located on line 15503 of the file "set.mm".
Its statement number for HTML pages is 559.
"Conjoin antecedents and consequents of two premises.  This
is the closed theorem form of ~ anim12d .  Theorem *3.47 of
[WhiteheadRussell] p. 113.  It was proved by Leibniz,
and it evidently pleased him enough to call it
_praeclarum theorema_ (splendid theorem).
...
1791 prth $p |- ( ( ( ph -> ps ) /\ ( ch -> th ) ) -> ( ( ph
      /\ ch ) -> ( ps /\ th ) ) ) $= ... $.
Its mandatory hypotheses in RPN order are:
  wph $f wff ph $.
  wps $f wff ps $.
  wch $f wff ch $.
  wth $f wff th $.
Its optional hypotheses are:  wta wet wze wsi wrh wmu wla wka
The statement and its hypotheses require the variables:  ph
      ps ch th
These additional variables are allowed in its proof:  ta et
      ze si rh mu la ka
The variables it contains are:  ph ps ch th


MM> show proof prth /lemmon/renumber
1 simpl          $p |- ( ( ( ph -> ps ) /\ ( ch -> th ) ) ->
                                               ( ph -> ps ) )
2 simpr          $p |- ( ( ( ph -> ps ) /\ ( ch -> th ) ) ->
                                               ( ch -> th ) )
3 1,2 anim12d    $p |- ( ( ( ph -> ps ) /\ ( ch -> th ) ) ->
                           ( ( ph /\ ch ) -> ( ps /\ th ) ) )
\end{verbatim}

There are references to a lot of unfamiliar statements.  To see what they are,
you may type the following:

\begin{verbatim}
MM> show proof prth /statement_summary
Summary of statements used in the proof of "prth":

Statement simpl is located on line 14748 of the file
"set.mm".
"Elimination of a conjunct.  Theorem *3.26 (Simp) of
[WhiteheadRussell] p. 112. ..."
  simpl $p |- ( ( ph /\ ps ) -> ph ) $= ... $.

Statement simpr is located on line 14777 of the file
"set.mm".
"Elimination of a conjunct.  Theorem *3.27 (Simp) of
[WhiteheadRussell] ..."
  simpr $p |- ( ( ph /\ ps ) -> ps ) $= ... $.

Statement anim12d is located on line 15445 of the file
"set.mm".
"Conjoin antecedents and consequents in a deduction.
..."
  anim12d.1 $e |- ( ph -> ( ps -> ch ) ) $.
  anim12d.2 $e |- ( ph -> ( th -> ta ) ) $.
  anim12d $p |- ( ph -> ( ( ps /\ th ) -> ( ch /\ ta ) ) )
      $= ... $.
\end{verbatim}
\begin{center}
(etc.)
\end{center}

Of course you can look at each of these statements and their proofs, and
so on, back to the axioms of propositional calculus if you wish.

The \texttt{search} command is useful for finding statements when you
know all or part of their contents.  The following example finds all
statements containing \verb@ph -> ps@ followed by \verb@ch -> th@.  The
\verb@$*@ is a wildcard that matches anything; the \texttt{\$} before the
\verb$*$ prevents conflicts with math symbol token names.  The \verb@*@ after
\texttt{SEARCH} is also a wildcard that in this case means ``match any label.''
\index{\texttt{search} command}

% I'm omitting this one, since readers are unlikely to see it:
% 1096 bisymOLD $p |- ( ( ( ph -> ps ) -> ( ch -> th ) ) -> ( (
%   ( ps -> ph ) -> ( th -> ch ) ) -> ( ( ph <-> ps ) -> ( ch
%    <-> th ) ) ) )
\begin{verbatim}
MM> search * "ph -> ps $* ch -> th"
1791 prth $p |- ( ( ( ph -> ps ) /\ ( ch -> th ) ) -> ( ( ph
    /\ ch ) -> ( ps /\ th ) ) )
2455 pm3.48 $p |- ( ( ( ph -> ps ) /\ ( ch -> th ) ) -> ( (
    ph \/ ch ) -> ( ps \/ th ) ) )
117859 pm11.71 $p |- ( ( E. x ph /\ E. y ch ) -> ( ( A. x (
    ph -> ps ) /\ A. y ( ch -> th ) ) <-> A. x A. y ( ( ph /\
    ch ) -> ( ps /\ th ) ) ) )
\end{verbatim}

Three statements, \texttt{prth}, \texttt{pm3.48},
 and \texttt{pm11.71}, were found to match.

To see what axioms\index{axiom} and definitions\index{definition}
\texttt{prth} ultimately depends on for its proof, you can have the
program backtrack through the hierarchy\index{hierarchy} of theorems and
definitions.\index{\texttt{show trace{\char`\_}back} command}

\begin{verbatim}
MM> show trace_back prth /essential/axioms
Statement "prth" assumes the following axioms ($a
statements):
  ax-1 ax-2 ax-3 ax-mp df-bi df-an
\end{verbatim}

Note that the 3 axioms of propositional calculus and the modus ponens rule are
needed (as expected); in addition, there are a couple of definitions that are used
along the way.  Note that Metamath makes no distinction\index{axiom vs.\
definition} between axioms\index{axiom} and definitions\index{definition}.  In
\texttt{set.mm} they have been distinguished artificially by prefixing their
labels\index{labels in \texttt{set.mm}} with \texttt{ax-} and \texttt{df-}
respectively.  For example, \texttt{df-an} defines conjunction (logical {\sc
and}), which is represented by the symbol \verb+/\+.
Section~\ref{definitions} discusses the philosophy of definitions, and the
Metamath language takes a particularly simple, conservative approach by using
the \texttt{\$a}\index{\texttt{\$a} statement} statement for both axioms and
definitions.

You can also have the program compute how many steps a proof
has\index{proof length} if we were to follow it all the way back to
\texttt{\$a} statements.

\begin{verbatim}
MM> show trace_back prth /essential/count_steps
The statement's actual proof has 3 steps.  Backtracking, a
total of 79 different subtheorems are used.  The statement
and subtheorems have a total of 274 actual steps.  If
subtheorems used only once were eliminated, there would be a
total of 38 subtheorems, and the statement and subtheorems
would have a total of 185 steps.  The proof would have 28349
steps if fully expanded back to axiom references.  The
maximum path length is 38.  A longest path is:  prth <-
anim12d <- syl2and <- sylan2d <- ancomsd <- ancom <- pm3.22
<- pm3.21 <- pm3.2 <- ex <- sylbir <- biimpri <- bicomi <-
bicom1 <- bi2 <- dfbi1 <- impbii <- bi3 <- simprim <- impi <-
con1i <- nsyl2 <- mt3d <- con1d <- notnot1 <- con2i <- nsyl3
<- mt2d <- con2d <- notnot2 <- pm2.18d <- pm2.18 <- pm2.21 <-
pm2.21d <- a1d <- syl <- mpd <- a2i <- a2i.1 .
\end{verbatim}

This tells us that we would have to inspect 274 steps if we want to
verify the proof completely starting from the axioms.  A few more
statistics are also shown.  There are one or more paths back to axioms
that are the longest; this command ferrets out one of them and shows it
to you.  There may be a sense in which the longest path length is
related to how ``deep'' the theorem is.

We might also be curious about what proofs depend on the theorem
\texttt{prth}.  If it is never used later on, we could eliminate it as
redundant if it has no intrinsic interest by itself.\index{\texttt{show
usage} command}

% I decided to show the OLD values here.
\begin{verbatim}
MM> show usage prth
Statement "prth" is directly referenced in the proofs of 18
statements:
  mo3 moOLD 2mo 2moOLD euind reuind reuss2 reusv3i opelopabt
  wemaplem2 rexanre rlimcn2 o1of2 o1rlimmul 2sqlem6 spanuni
  heicant pm11.71
\end{verbatim}

Thus \texttt{prth} is directly used by 18 proofs.
We can use the \texttt{/recursive} qualifier to include indirect use:

\begin{verbatim}
MM> show usage prth /recursive
Statement "prth" directly or indirectly affects the proofs of
24214 statements:
  mo3 mo mo3OLD eu2 moOLD eu2OLD eu3OLD mo4f mo4 eu4 mopick
...
\end{verbatim}

\subsection{A Note on the ``Compact'' Proof Format}

The Metamath program will display proofs in a ``compact''\index{compact proof}
format whenever the proof is stored in compressed format in the database.  It
may be be slightly confusing unless you know how to interpret it.
For example,
if you display the complete proof of theorem \texttt{id1} it will start
off as follows:

\begin{verbatim}
MM> show proof id1 /lemmon/all
 1 wph           $f wff ph
 2 wph           $f wff ph
 3 wph           $f wff ph
 4 2,3 wi    @4: $a wff ( ph -> ph )
 5 1,4 wi    @5: $a wff ( ph -> ( ph -> ph ) )
 6 @4            $a wff ( ph -> ph )
\end{verbatim}

\begin{center}
{etc.}
\end{center}

Step 4 has a ``local label,''\index{local label} \texttt{@4}, assigned to it.
Later on, at step 6, the label \texttt{@1} is referenced instead of
displaying the explicit proof for that step.  This technique takes advantage
of the fact that steps in a proof often repeat, especially during the
construction of wffs.  The compact format reduces the number of steps in the
proof display and may be preferred by some people.

If you want to see the normal format with the ``true'' step numbers, you can
use the following workaround:\index{\texttt{save proof} command}

\begin{verbatim}
MM> save proof id1 /normal
The proof of "id1" has been reformatted and saved internally.
Remember to use WRITE SOURCE to save it permanently.
MM> show proof id1 /lemmon/all
 1 wph           $f wff ph
 2 wph           $f wff ph
 3 wph           $f wff ph
 4 2,3 wi        $a wff ( ph -> ph )
 5 1,4 wi        $a wff ( ph -> ( ph -> ph ) )
 6 wph           $f wff ph
 7 wph           $f wff ph
 8 6,7 wi        $a wff ( ph -> ph )
\end{verbatim}

\begin{center}
{etc.}
\end{center}

Note that the original 6 steps are now 8 steps.  However, the format is
now the same as that described in Chapter~\ref{using}.

\chapter{The Metamath Language}
\label{languagespec}

\begin{quote}
  {\em Thus mathematics may be defined as the subject in which we never know
what we are talking about, nor whether what we are saying is true.}
    \flushright\sc  Bertrand Russell\footnote{\cite[p.~84]{Russell2}.}\\
\end{quote}\index{Russell, Bertrand}

Probably the most striking feature of the Metamath language is its almost
complete absence of hard-wired syntax. Metamath\index{Metamath} does not
understand any mathematics or logic other than that needed to construct finite
sequences of symbols according to a small set of simple, built-in rules.  The
only rule it uses in a proof is the substitution of an expression (symbol
sequence) for a variable, subject to a simple constraint to prevent
bound-variable clashes.  The primitive notions built into Metamath involve the
simple manipulation of finite objects (symbols) that we as humans can easily
visualize and that computers can easily deal with.  They seem to be just
about the simplest notions possible that are required to do standard
mathematics.

This chapter serves as a reference manual for the Metamath\index{Metamath}
language. It covers the tedious technical details of the language, some of
which you may wish to skim in a first reading.  On the other hand, you should
pay close attention to the defined terms in {\bf boldface}; they have precise
meanings that are important to keep in mind for later understanding.  It may
be best to first become familiar with the examples in Chapter~\ref{using} to
gain some motivation for the language.

%% Uncomment this when uncommenting section {formalspec} below
If you have some knowledge of set theory, you may wish to study this
chapter in conjunction with the formal set-theoretical description of the
Metamath language in Appendix~\ref{formalspec}.

We will use the name ``Metamath''\index{Metamath} to mean either the Metamath
computer language or the Metamath software associated with the computer
language.  We will not distinguish these two when the context is clear.

The next section contains the complete specification of the Metamath
language.
It serves as an
authoritative reference and presents the syntax in enough detail to
write a parser\index{parsing Metamath} and proof verifier.  The
specification is terse and it is probably hard to learn the language
directly from it, but we include it here for those impatient people who
prefer to see everything up front before looking at verbose expository
material.  Later sections explain this material and provide examples.
We will repeat the definitions in those sections, and you may skip the
next section at first reading and proceed to Section~\ref{tut1}
(p.~\pageref{tut1}).

\section{Specification of the Metamath Language}\label{spec}
\index{Metamath!specification}

\begin{quote}
  {\em Sometimes one has to say difficult things, but one ought to say
them as simply as one knows how.}
    \flushright\sc  G. H. Hardy\footnote{As quoted in
    \cite{deMillo}, p.~273.}\\
\end{quote}\index{Hardy, G. H.}

\subsection{Preliminaries}\label{spec1}

% Space is technically a printable character, so we'll word things
% carefully so it's unambiguous.
A Metamath {\bf database}\index{database} is built up from a top-level source
file together with any source files that are brought in through file inclusion
commands (see below).  The only characters that are allowed to appear in a
Metamath source file are the 94 non-whitespace printable {\sc
ascii}\index{ascii@{\sc ascii}} characters, which are digits, upper and lower
case letters, and the following 32 special
characters\index{special characters}:\label{spec1chars}

\begin{verbatim}
! " # $ % & ' ( ) * + , - . / :
; < = > ? @ [ \ ] ^ _ ` { | } ~
\end{verbatim}
plus the following characters which are the ``white space'' characters:
space (a printable character),
tab, carriage return, line feed, and form feed.\label{whitespace}
We will use \texttt{typewriter}
font to display the printable characters.

A Metamath database consists of a sequence of three kinds of {\bf
tokens}\index{token} separated by {\bf white space}\index{white space}
(which is any sequence of one or more white space characters).  The set
of {\bf keyword}\index{keyword} tokens is \texttt{\$\char`\{},
\texttt{\$\char`\}}, \texttt{\$c}, \texttt{\$v}, \texttt{\$f},
\texttt{\$e}, \texttt{\$d}, \texttt{\$a}, \texttt{\$p}, \texttt{\$.},
\texttt{\$=}, \texttt{\$(}, \texttt{\$)}, \texttt{\$[}, and
\texttt{\$]}.  The last four are called {\bf auxiliary}\index{auxiliary
keyword} or preprocessing keywords.  A {\bf label}\index{label} token
consists of any combination of letters, digits, and the characters
hyphen, underscore, and period.  A {\bf math symbol}\index{math symbol}
token may consist of any combination of the 93 printable standard {\sc
ascii} characters other than space or \texttt{\$}~. All tokens are
case-sensitive.

\subsection{Preprocessing}

The token \texttt{\$(} begins a {\bf comment} and
\texttt{\$)} ends a comment.\index{\texttt{\$(}
and \texttt{\$)} auxiliary keywords}\index{comment}
Comments may contain any of
the 94 non-whitespace printable characters and white space,
except they may not contain the
2-character sequences \texttt{\$(} or \texttt{\$)} (comments do not nest).
Comments are ignored (treated
like white space) for the purpose of parsing, e.g.,
\texttt{\$( \$[ \$)} is a comment.
See p.~\pageref{mathcomments} for comment typesetting conventions; these
conventions may be ignored for the purpose of parsing.

A {\bf file inclusion command} consists of \texttt{\$[} followed by a file name
followed by \texttt{\$]}.\index{\texttt{\$[} and \texttt{\$]} auxiliary
keywords}\index{included file}\index{file inclusion}
It is only allowed in the outermost scope (i.e., not between
\texttt{\$\char`\{} and \texttt{\$\char`\}})
and must not be inside a statement (e.g., it may not occur
between the label of a \texttt{\$a} statement and its \texttt{\$.}).
The file name may not
contain a \texttt{\$} or white space.  The file must exist.
The case-sensitivity
of its name follows the conventions of the operating system.  The contents of
the file replace the inclusion command.
Included files may include other files.
Only the first reference to a given file is included; any later
references to the same file (whether in the top-level file or in included
files) cause the inclusion command to be ignored (treated like white space).
A verifier may assume that file names with different strings
refer to different files for the purpose of ignoring later references.
A file self-reference is ignored, as is any reference to the top-level file
(to avoid loops).
Included files may not include a \texttt{\$(} without a matching \texttt{\$)},
may not include a \texttt{\$[} without a matching \texttt{\$]}, and may
not include incomplete statements (e.g., a \texttt{\$a} without a matching
\texttt{\$.}).
It is currently unspecified if path references are relative to the process'
current directory or the file's containing directory, so databases should
avoid using pathname separators (e.g., ``/'') in file names.

Like all tokens, the \texttt{\$(}, \texttt{\$)}, \texttt{\$[}, and \texttt{\$]} keywords
must be surrounded by white space.

\subsection{Basic Syntax}

After preprocessing, a database will consist of a sequence of {\bf
statements}.
These are the scoping statements \texttt{\$\char`\{} and
\texttt{\$\char`\}}, along with the \texttt{\$c}, \texttt{\$v},
\texttt{\$f}, \texttt{\$e}, \texttt{\$d}, \texttt{\$a}, and \texttt{\$p}
statements.

A {\bf scoping statement}\index{scoping statement} consists only of its
keyword, \texttt{\$\char`\{} or \texttt{\$\char`\}}.
A \texttt{\$\char`\{} begins a {\bf
block}\index{block} and a matching \texttt{\$\char`\}} ends the block.
Every \texttt{\$\char`\{}
must have a matching \texttt{\$\char`\}}.
Defining it recursively, we say a block
contains a sequence of zero or more tokens other
than \texttt{\$\char`\{} and \texttt{\$\char`\}} and
possibly other blocks.  There is an {\bf outermost
block}\index{block!outermost} not bracketed by \texttt{\$\char`\{} \texttt{\$\char`\}}; the end
of the outermost block is the end of the database.

% LaTeX bug? can't do \bf\tt

A {\bf \$v} or {\bf \$c statement}\index{\texttt{\$v} statement}\index{\texttt{\$c}
statement} consists of the keyword token \texttt{\$v} or \texttt{\$c} respectively,
followed by one or more math symbols,
% The word "token" is used to distinguish "$." from the sentence-ending period.
followed by the \texttt{\$.}\ token.
These
statements {\bf declare}\index{declaration} the math symbols to be {\bf
variables}\index{variable!Metamath} or {\bf constants}\index{constant}
respectively. The same math symbol may not occur twice in a given \texttt{\$v} or
\texttt{\$c} statement.

%c%A math symbol becomes an {\bf active}\index{active math symbol}
%c%when declared and stays active until the end of the block in which it is
%c%declared.  A math symbol may not be declared a second time while it is active,
%c%but it may be declared again after it becomes inactive.

A math symbol becomes {\bf active}\index{active math symbol} when declared
and stays active until the end of the block in which it is declared.  A
variable may not be declared a second time while it is active, but it
may be declared again (as a variable, but not as a constant) after it
becomes inactive.  A constant must be declared in the outermost block and may
not be declared a second time.\index{redeclaration of symbols}

A {\bf \$f statement}\index{\texttt{\$f} statement} consists of a label,
followed by \texttt{\$f}, followed by its typecode (an active constant),
followed by an
active variable, followed by the \texttt{\$.}\ token.  A {\bf \$e
statement}\index{\texttt{\$e} statement} consists of a label, followed
by \texttt{\$e}, followed by its typecode (an active constant),
followed by zero or more
active math symbols, followed by the \texttt{\$.}\ token.  A {\bf
hypothesis}\index{hypothesis} is a \texttt{\$f} or \texttt{\$e}
statement.
The type declared by a \texttt{\$f} statement for a given label
is global even if the variable is not
(e.g., a database may not have \texttt{wff P} in one local scope
and \texttt{class P} in another).

A {\bf simple \$d statement}\index{\texttt{\$d} statement!simple}
consists of \texttt{\$d}, followed by two different active variables,
followed by the \texttt{\$.}\ token.  A {\bf compound \$d
statement}\index{\texttt{\$d} statement!compound} consists of
\texttt{\$d}, followed by three or more variables (all different),
followed by the \texttt{\$.}\ token.  The order of the variables in a
\texttt{\$d} statement is unimportant.  A compound \texttt{\$d}
statement is equivalent to a set of simple \texttt{\$d} statements, one
for each possible pair of variables occurring in the compound
\texttt{\$d} statement.  Henceforth in this specification we shall
assume all \texttt{\$d} statements are simple.  A \texttt{\$d} statement
is also called a {\bf disjoint} (or {\bf distinct}) {\bf variable
restriction}.\index{disjoint-variable restriction}

A {\bf \$a statement}\index{\texttt{\$a} statement} consists of a label,
followed by \texttt{\$a}, followed by its typecode (an active constant),
followed by
zero or more active math symbols, followed by the \texttt{\$.}\ token.  A {\bf
\$p statement}\index{\texttt{\$p} statement} consists of a label,
followed by \texttt{\$p}, followed by its typecode (an active constant),
followed by
zero or more active math symbols, followed by \texttt{\$=}, followed by
a sequence of labels, followed by the \texttt{\$.}\ token.  An {\bf
assertion}\index{assertion} is a \texttt{\$a} or \texttt{\$p} statement.

A \texttt{\$f}, \texttt{\$e}, or \texttt{\$d} statement is {\bf active}\index{active
statement} from the place it occurs until the end of the block it occurs in.
A \texttt{\$a} or \texttt{\$p} statement is {\bf active} from the place it occurs
through the end of the database.
There may not be two active \texttt{\$f} statements containing the same
variable.  Each variable in a \texttt{\$e}, \texttt{\$a}, or
\texttt{\$p} statement must exist in an active \texttt{\$f}
statement.\footnote{This requirement can greatly simplify the
unification algorithm (substitution calculation) required by proof
verification.}

%The label that begins each \texttt{\$f}, \texttt{\$e}, \texttt{\$a}, and
%\texttt{\$p} statement must be unique.
Each label token must be unique, and
no label token may match any math symbol
token.\label{namespace}\footnote{This
restriction was added on June 24, 2006.
It is not theoretically necessary but is imposed to make it easier to
write certain parsers.}

The set of {\bf mandatory variables}\index{mandatory variable} associated with
an assertion is the set of (zero or more) variables in the assertion and in any
active \texttt{\$e} statements.  The (possibly empty) set of {\bf mandatory
hypotheses}\index{mandatory hypothesis} is the set of all active \texttt{\$f}
statements containing mandatory variables, together with all active \texttt{\$e}
statements.
The set of {\bf mandatory {\bf \$d} statements}\index{mandatory
disjoint-variable restriction} associated with an assertion are those active
\texttt{\$d} statements whose variables are both among the assertion's
mandatory variables.

\subsection{Proof Verification}\label{spec4}

The sequence of labels between the \texttt{\$=} and \texttt{\$.}\ tokens
in a \texttt{\$p} statement is a {\bf proof}.\index{proof!Metamath} Each
label in a proof must be the label of an active statement other than the
\texttt{\$p} statement itself; thus a label must refer either to an
active hypothesis of the \texttt{\$p} statement or to an earlier
assertion.

An {\bf expression}\index{expression} is a sequence of math symbols. A {\bf
substitution map}\index{substitution map} associates a set of variables with a
set of expressions.  It is acceptable for a variable to be mapped to an
expression containing it.  A {\bf
substitution}\index{substitution!variable}\index{variable substitution} is the
simultaneous replacement of all variables in one or more expressions with the
expressions that the variables map to.

A proof is scanned in order of its label sequence.  If the label refers to an
active hypothesis, the expression in the hypothesis is pushed onto a
stack.\index{stack}\index{RPN stack}  If the label refers to an assertion, a
(unique) substitution must exist that, when made to the mandatory hypotheses
of the referenced assertion, causes them to match the topmost (i.e.\ most
recent) entries of the stack, in order of occurrence of the mandatory
hypotheses, with the topmost stack entry matching the last mandatory
hypothesis of the referenced assertion.  As many stack entries as there are
mandatory hypotheses are then popped from the stack.  The same substitution is
made to the referenced assertion, and the result is pushed onto the stack.
After the last label in the proof is processed, the stack must have a single
entry that matches the expression in the \texttt{\$p} statement containing the
proof.

%c%{\footnotesize\begin{quotation}\index{redeclaration of symbols}
%c%{{\em Comment.}\label{spec4comment} Whenever a math symbol token occurs in a
%c%{\texttt{\$c} or \texttt{\$v} statement, it is considered to designate a distinct new
%c%{symbol, even if the same token was previously declared (and is now inactive).
%c%{Thus a math token declared as a constant in two different blocks is considered
%c%{to designate two distinct constants (even though they have the same name).
%c%{The two constants will not match in a proof that references both blocks.
%c%{However, a proof referencing both blocks is acceptable as long as it doesn't
%c%{require that the constants match.  Similarly, a token declared to be a
%c%{constant for a referenced assertion will not match the same token declared to
%c%{be a variable for the \texttt{\$p} statement containing the proof.  In the case
%c%{of a token declared to be a variable for a referenced assertion, this is not
%c%{an issue since the variable can be substituted with whatever expression is
%c%{needed to achieve the required match.
%c%{\end{quotation}}
%c2%A proof may reference an assertion that contains or whose hypotheses contain a
%c2%constant that is not active for the \texttt{\$p} statement containing the proof.
%c2%However, the final result of the proof may not contain that constant. A proof
%c2%may also reference an assertion that contains or whose hypotheses contain a
%c2%variable that is not active for the \texttt{\$p} statement containing the proof.
%c2%That variable, of course, will be substituted with whatever expression is
%c2%needed to achieve the required match.

A proof may contain a \texttt{?}\ in place of a label to indicate an unknown step
(Section~\ref{unknown}).  A proof verifier may ignore any proof containing
\texttt{?}\ but should warn the user that the proof is incomplete.

A {\bf compressed proof}\index{compressed proof}\index{proof!compressed} is an
alternate proof notation described in Appen\-dix~\ref{compressed}; also see
references to ``compressed proof'' in the Index.  Compressed proofs are a
Metamath language extension which a complete proof verifier should be able to
parse and verify.

\subsubsection{Verifying Disjoint Variable Restrictions}

Each substitution made in a proof must be checked to verify that any
disjoint variable restrictions are satisfied, as follows.

If two variables replaced by a substitution exist in a mandatory \texttt{\$d}
statement\index{\texttt{\$d} statement} of the assertion referenced, the two
expressions resulting from the substitution must satisfy the following
conditions.  First, the two expressions must have no variables in common.
Second, each possible pair of variables, one from each expression, must exist
in an active \texttt{\$d} statement of the \texttt{\$p} statement containing the
proof.

\vskip 1ex

This ends the specification of the Metamath language;
see Appendix \ref{BNF} for its syntax in
Extended Backus--Naur Form (EBNF)\index{Extended Backus--Naur Form}\index{EBNF}.

\section{The Basic Keywords}\label{tut1}

Our expository material begins here.

Like most computer languages, Metamath\index{Metamath} takes its input from
one or more {\bf source files}\index{source file} which contain characters
expressed in the standard {\sc ascii} (American Standard Code for Information
Interchange)\index{ascii@{\sc ascii}} code for computers.  A source file
consists of a series of {\bf tokens}\index{token}, which are strings of
non-whitespace
printable characters (from the set of 94 shown on p.~\pageref{spec1chars})
separated by {\bf white space}\index{white space} (spaces, tabs, carriage
returns, line feeds, and form feeds). Any string consisting only of these
characters is treated the same as a single space.  The non-whitespace printable
characters\index{printable character} that Metamath recognizes are the 94
characters on standard {\sc ascii} keyboards.

Metamath has the ability to join several files together to form its
input (Section~\ref{include}).  We call the aggregate contents of all
the files after they have been joined together a {\bf
database}\index{database} to distinguish it from an individual source
file.  The tokens in a database consist of {\bf
keywords}\index{keyword}, which are built into the language, together
with two kinds of user-defined tokens called {\bf labels}\index{label}
and {\bf math symbols}\index{math symbol}.  (Often we will simply say
{\bf symbol}\index{symbol} instead of math symbol for brevity).  The set
of {\bf basic keywords}\index{basic keyword} is
\texttt{\$c}\index{\texttt{\$c} statement},
\texttt{\$v}\index{\texttt{\$v} statement},
\texttt{\$e}\index{\texttt{\$e} statement},
\texttt{\$f}\index{\texttt{\$f} statement},
\texttt{\$d}\index{\texttt{\$d} statement},
\texttt{\$a}\index{\texttt{\$a} statement},
\texttt{\$p}\index{\texttt{\$p} statement},
\texttt{\$=}\index{\texttt{\$=} keyword},
\texttt{\$.}\index{\texttt{\$.}\ keyword},
\texttt{\$\char`\{}\index{\texttt{\$\char`\{} and \texttt{\$\char`\}}
keywords}, and \texttt{\$\char`\}}.  This is the complete set of
syntactical elements of what we call the {\bf basic
language}\index{basic language} of Metamath, and with them you can
express all of the mathematics that were intended by the design of
Metamath.  You should make it a point to become very familiar with them.
Table~\ref{basickeywords} lists the basic keywords along with a brief
description of their functions.  For now, this description will give you
only a vague notion of what the keywords are for; later we will describe
the keywords in detail.


\begin{table}[htp] \caption{Summary of the basic Metamath
keywords} \label{basickeywords}
\begin{center}
\begin{tabular}{|p{4pc}|l|}
\hline
\em \centering Keyword&\em Description\\
\hline
\hline
\centering
   \texttt{\$c}&Constant symbol declaration\\
\hline
\centering
   \texttt{\$v}&Variable symbol declaration\\
\hline
\centering
   \texttt{\$d}&Disjoint variable restriction\\
\hline
\centering
   \texttt{\$f}&Variable-type (``floating'') hypothesis\\
\hline
\centering
   \texttt{\$e}&Logical (``essential'') hypothesis\\
\hline
\centering
   \texttt{\$a}&Axiomatic assertion\\
\hline
\centering
   \texttt{\$p}&Provable assertion\\
\hline
\centering
   \texttt{\$=}&Start of proof in \texttt{\$p} statement\\
\hline
\centering
   \texttt{\$.}&End of the above statement types\\
\hline
\centering
   \texttt{\$\char`\{}&Start of block\\
\hline
\centering
   \texttt{\$\char`\}}&End of block\\
\hline
\end{tabular}
\end{center}
\end{table}

%For LaTeX bug(?) where it puts tables on blank page instead of btwn text
%May have to adjust if text changes
%\newpage

There are some additional keywords, called {\bf auxiliary
keywords}\index{auxiliary keyword} that help make Metamath\index{Metamath}
more practical. These are part of the {\bf extended language}\index{extended
language}. They provide you with a means to put comments into a Metamath
source file\index{source file} and reference other source files.  We will
introduce these in later sections. Table~\ref{otherkeywords} summarizes them
so that you can recognize them now if you want to peruse some source
files while learning the basic keywords.


\begin{table}[htp] \caption{Auxiliary Metamath
keywords} \label{otherkeywords}
\begin{center}
\begin{tabular}{|p{4pc}|l|}
\hline
\em \centering Keyword&\em Description\\
\hline
\hline
\centering
   \texttt{\$(}&Start of comment\\
\hline
\centering
   \texttt{\$)}&End of comment\\
\hline
\centering
   \texttt{\$[}&Start of included source file name\\
\hline
\centering
   \texttt{\$]}&End of included source file name\\
\hline
\end{tabular}
\end{center}
\end{table}
\index{\texttt{\$(} and \texttt{\$)} auxiliary keywords}
\index{\texttt{\$[} and \texttt{\$]} auxiliary keywords}


Unlike those in some computer languages, the keywords\index{keyword} are short
two-character sequences rather than English-like words.  While this may make
them slightly more difficult to remember at first, their brevity allows
them to blend in with the mathematics being described, not
distract from it, like punctuation marks.


\subsection{User-Defined Tokens}\label{dollardollar}\index{token}

As you may have noticed, all keywords\index{keyword} begin with the \texttt{\$}
character.  This mundane monetary symbol is not ordinarily used in higher
mathematics (outside of grant proposals), so we have appropriated it to
distinguish the Metamath\index{Metamath} keywords from ordinary mathematical
symbols. The \texttt{\$} character is thus considered special and may not be
used as a character in a user-defined token.  All tokens and keywords are
case-sensitive; for example, \texttt{n} is considered to be a different character
from \texttt{N}.  Case-sensitivity makes the available {\sc ascii} character set
as rich as possible.

\subsubsection{Math Symbol Tokens}\index{token}

Math symbols\index{math symbol} are tokens used to represent the symbols
that appear in ordinary mathematical formulas.  They may consist of any
combination of the 93 non-whitespace printable {\sc ascii} characters other than
\texttt{\$}~. Some examples are \texttt{x}, \texttt{+}, \texttt{(},
\texttt{|-}, \verb$!%@?&$, and \texttt{bounded}.  For readability, it is
best to try to make these look as similar to actual mathematical symbols
as possible, within the constraints of the {\sc ascii} character set, in
order to make the resulting mathematical expressions more readable.

In the Metamath\index{Metamath} language, you express ordinary
mathematical formulas and statements as sequences of math symbols such
as \texttt{2 + 2 = 4} (five symbols, all constants).\footnote{To
eliminate ambiguity with other expressions, this is expressed in the set
theory database \texttt{set.mm} as \texttt{|- ( 2 + 2
 ) = 4 }, whose \LaTeX\ equivalent is $\vdash
(2+2)=4$.  The \,$\vdash$ means ``is a theorem'' and the
parentheses allow explicit associative grouping.}\index{turnstile
({$\,\vdash$})} They may even be English
sentences, as in \texttt{E is closed and bounded} (five symbols)---here
\texttt{E} would be a variable and the other four symbols constants.  In
principle, a Metamath database could be constructed to work with almost
any unambiguous English-language mathematical statement, but as a
practical matter the definitions needed to provide for all possible
syntax variations would be cumbersome and distracting and possibly have
subtle pitfalls accidentally built in.  We generally recommend that you
express mathematical statements with compact standard mathematical
symbols whenever possible and put their English-language descriptions in
comments.  Axioms\index{axiom} and definitions\index{definition}
(\texttt{\$a}\index{\texttt{\$a} statement} statements) are the only
places where Metamath will not detect an error, and doing this will help
reduce the number of definitions needed.

You are free to use any tokens\index{token} you like for math
symbols\index{math symbol}.  Appendix~\ref{ASCII} recommends token names to
use for symbols in set theory, and we suggest you adopt these in order to be
able to include the \texttt{set.mm} set theory database in your database.  For
printouts, you can convert the tokens in a database
to standard mathematical symbols with the \LaTeX\ typesetting program.  The
Metamath command \texttt{open tex} {\em filename}\index{\texttt{open tex} command}
produces output that can be read by \LaTeX.\index{latex@{\LaTeX}}
The correspondence
between tokens and the actual symbols is made by \texttt{latexdef}
statements inside a special database comment tagged
with \texttt{\$t}.\index{\texttt{\$t} comment}\index{typesetting comment}
  You can edit
this comment to change the definitions or add new ones.
Appendix~\ref{ASCII} describes how to do this in more detail.

% White space\index{white space} is normally used to separate math
% symbol\index{math symbol} tokens, but they may be juxtaposed without white
% space in \texttt{\$d}\index{\texttt{\$d} statement}, \texttt{\$e}\index{\texttt{\$e}
% statement}, \texttt{\$f}\index{\texttt{\$f} statement}, \texttt{\$a}\index{\texttt{\$a}
% statement}, and \texttt{\$p}\index{\texttt{\$p} statement} statements when no
% ambiguity will result.  Specifically, Metamath parses the math symbol sequence
% in one of these statements in the following manner:  when the math symbol
% sequence has been broken up into tokens\index{token} up to a given character,
% the next token is the longest string of characters that could constitute a
% math symbol that is active\index{active
% math symbol} at that point.  (See Section~\ref{scoping} for the
% definition of an active math symbol.)  For example, if \texttt{-}, \texttt{>}, and
% \texttt{->} are the only active math symbols, the juxtaposition \texttt{>-} will be
% interpreted as the two symbols \texttt{>} and \texttt{-}, whereas \texttt{->} will
% always be interpreted as that single symbol.\footnote{For better readability we
% recommend a white space between each token.  This also makes searching for a
% symbol easier to do with an editor.  Omission of optional white space is useful
% for reducing typing when assigning an expression to a temporary
% variable\index{temporary variable} with the \texttt{let variable} Metamath
% program command.}\index{\texttt{let variable} command}
%
% Keywords\index{keyword} may be placed next to math symbols without white
% space\index{white space} between them.\footnote{Again, we do not recommend
% this for readability.}
%
% The math symbols\index{math symbol} in \texttt{\$c}\index{\texttt{\$c} statement}
% and \texttt{\$v}\index{\texttt{\$v} statement} statements must always be separated
% by white space\index{white
% space}, for the obvious reason that these statements define the names
% of the symbols.
%
% Math symbols referred to in comments (see Section~\ref{comments}) must also be
% separated by white space.  This allows you to make comments about symbols that
% are not yet active\index{active
% math symbol}.  (The ``math mode'' feature of comments is also a quick and
% easy way to obtain word processing text with embedded mathematical symbols,
% independently of the main purpose of Metamath; the way to do this is described
% in Section~\ref{comments})

\subsubsection{Label Tokens}\index{token}\index{label}

Label tokens are used to identify Metamath\index{Metamath} statements for
later reference. Label tokens may contain only letters, digits, and the three
characters period, hyphen, and underscore:
\begin{verbatim}
. - _
\end{verbatim}

A label is {\bf declared}\index{label declaration} by placing it immediately
before the keyword of the statement it identifies.  For example, the label
\texttt{axiom.1} might be declared as follows:
\begin{verbatim}
axiom.1 $a |- x = x $.
\end{verbatim}

Each \texttt{\$e}\index{\texttt{\$e} statement},
\texttt{\$f}\index{\texttt{\$f} statement},
\texttt{\$a}\index{\texttt{\$a} statement}, and
\texttt{\$p}\index{\texttt{\$p} statement} statement in a database must
have a label declared for it.  No other statement types may have label
declarations.  Every label must be unique.

A label (and the statement it identifies) is {\bf referenced}\index{label
reference} by including the label between the \texttt{\$=}\index{\texttt{\$=}
keyword} and \texttt{\$.}\index{\texttt{\$.}\ keyword}\ keywords in a \texttt{\$p}
statement.  The sequence of labels\index{label sequence} between these two
keywords is called a {\bf proof}\index{proof}.  An example of a statement with
a proof that we will encounter later (Section~\ref{proof}) is
\begin{verbatim}
wnew $p wff ( s -> ( r -> p ) )
     $= ws wr wp w2 w2 $.
\end{verbatim}

You don't have to know what this means just yet, but you should know that the
label \texttt{wnew} is declared by this \texttt{\$p} statement and that the labels
\texttt{ws}, \texttt{wr}, \texttt{wp}, and \texttt{w2} are assumed to have been declared
earlier in the database and are referenced here.

\subsection{Constants and Variables}
\index{constant}
\index{variable}

An {\bf expression}\index{expression} is any sequence of math
symbols, possibly empty.

The basic Metamath\index{Metamath} language\index{basic language} has two
kinds of math symbols\index{math symbol}:  {\bf constants}\index{constant} and
{\bf variables}\index{variable}.  In a Metamath proof, a constant may not be
substituted with any expression.  A variable can be
substituted\index{substitution!variable}\index{variable substitution} with any
expression.  This sequence may include other variables and may even include
the variable being substituted.  This substitution takes place when proofs are
verified, and it will be described in Section~\ref{proof}.  The \texttt{\$f}
statement (described later in Section~\ref{dollaref}) is used to specify the
{\bf type} of a variable (i.e.\ what kind of
variable it is)\index{variable type}\index{type} and
give it a meaning typically
associated with a ``metavariable''\index{metavariable}\footnote{A metavariable
is a variable that ranges over the syntactical elements of the object language
being discussed; for example, one metavariable might represent a variable of
the object language and another metavariable might represent a formula in the
object language.} in ordinary mathematics; for example, a variable may be
specified to be a wff or well-formed formula (in logic), a set (in set
theory), or a non-negative integer (in number theory).

%\subsection{The \texttt{\$c} and \texttt{\$v} Declaration Statements}
\subsection{The \texttt{\$c} and \texttt{\$v} Declaration Statements}
\index{\texttt{\$c} statement}
\index{constant declaration}
\index{\texttt{\$v} statement}
\index{variable declaration}

Constants are introduced or {\bf declared}\index{constant declaration}
with \texttt{\$c}\index{\texttt{\$c} statement} statements, and
variables are declared\index{variable declaration} with
\texttt{\$v}\index{\texttt{\$v} statement} statements.  A {\bf simple}
declaration\index{simple declaration} statement introduces a single
constant or variable.  Its syntax is one of the following:
\begin{center}
  \texttt{\$c} {\em math-symbol} \texttt{\$.}\\
  \texttt{\$v} {\em math-symbol} \texttt{\$.}
\end{center}
The notation {\em math-symbol} means any math symbol token\index{token}.

Some examples of simple declaration statements are:
\begin{center}
  \texttt{\$c + \$.}\\
  \texttt{\$c -> \$.}\\
  \texttt{\$c ( \$.}\\
  \texttt{\$v x \$.}\\
  \texttt{\$v y2 \$.}
\end{center}

The characters in a math symbol\index{math symbol} being declared are
irrelevant to Meta\-math; for example, we could declare a right parenthesis to
be a variable,
\begin{center}
  \texttt{\$v ) \$.}\\
\end{center}
although this would be unconventional.

A {\bf compound} declaration\index{compound declaration} statement is a
shorthand for declaring several symbols at once.  Its syntax is one of the
following:
\begin{center}
  \texttt{\$c} {\em math-symbol}\ \,$\cdots$\ {\em math-symbol} \texttt{\$.}\\
  \texttt{\$v} {\em math-symbol}\ \,$\cdots$\ {\em math-symbol} \texttt{\$.}
\end{center}\index{\texttt{\$c} statement}
Here, the ellipsis (\ldots) means any number of {\em math-symbol}\,s.

An example of a compound declaration statement is:
\begin{center}
  \texttt{\$v x y mu \$.}\\
\end{center}
This is equivalent to the three simple declaration statements
\begin{center}
  \texttt{\$v x \$.}\\
  \texttt{\$v y \$.}\\
  \texttt{\$v mu \$.}\\
\end{center}
\index{\texttt{\$v} statement}

There are certain rules on where in the database math symbols may be declared,
what sections of the database are aware of them (i.e.\ where they are
``active''), and when they may be declared more than once.  These will be
discussed in Section~\ref{scoping} and specifically on
p.~\pageref{redeclaration}.

\subsection{The \texttt{\$d} Statement}\label{dollard}
\index{\texttt{\$d} statement}

The \texttt{\$d} statement is called a {\bf disjoint-variable restriction}.  The
syntax of the {\bf simple} version of this statement is
\begin{center}
  \texttt{\$d} {\em variable variable} \texttt{\$.}
\end{center}
where each {\em variable} is a previously declared variable and the two {\em
variable}\,s are different.  (More specifically, each  {\em variable} must be
an {\bf active} variable\index{active math symbol}, which means there must be
a previous \texttt{\$v} statement whose {\bf scope}\index{scope} includes the
\texttt{\$d} statement.  These terms will be defined when we discuss scoping
statements in Section~\ref{scoping}.)

In ordinary mathematics, formulas may arise that are true if the variables in
them are distinct\index{distinct variables}, but become false when those
variables are made identical. For example, the formula in logic $\exists x\,x
\neq y$, which means ``for a given $y$, there exists an $x$ that is not equal
to $y$,'' is true in most mathematical theories (namely all non-trivial
theories\index{non-trivial theory}, i.e.\ those that describe more than one
individual, such as arithmetic).  However, if we substitute $y$ with $x$, we
obtain $\exists x\,x \neq x$, which is always false, as it means ``there
exists something that is not equal to itself.''\footnote{If you are a
logician, you will recognize this as the improper substitution\index{proper
substitution}\index{substitution!proper} of a free variable\index{free
variable} with a bound variable\index{bound variable}.  Metamath makes no
inherent distinction between free and bound variables; instead, you let
Metamath know what substitutions are permissible by using \texttt{\$d} statements
in the right way in your axiom system.}\index{free vs.\ bound variable}  The
\texttt{\$d} statement allows you to specify a restriction that forbids the
substitution of one variable with another.  In
this case, we would use the statement
\begin{center}
  \texttt{\$d x y \$.}
\end{center}\index{\texttt{\$d} statement}
to specify this restriction.

The order in which the variables appear in a \texttt{\$d} statement is not
important.  We could also use
\begin{center}
  \texttt{\$d y x \$.}
\end{center}

The \texttt{\$d} statement is actually more general than this, as the
``disjoint''\index{disjoint variables} in its name suggests.  The full meaning
is that if any substitution is made to its two variables (during the
course of a proof that references a \texttt{\$a} or \texttt{\$p} statement
associated with the \texttt{\$d}), the two expressions that result from the
substitution must have no variables in common.  In addition, each possible
pair of variables, one from each expression, must be in a \texttt{\$d} statement
associated with the statement being proved.  (This requirement forces the
statement being proved to ``inherit'' the original disjoint variable
restriction.)

For example, suppose \texttt{u} is a variable.  If the restriction
\begin{center}
  \texttt{\$d A B \$.}
\end{center}
has been specified for a theorem referenced in a
proof, we may not substitute \texttt{A} with \mbox{\tt a + u} and
\texttt{B} with \mbox{\tt b + u} because these two symbol sequences have the
variable \texttt{u} in common.  Furthermore, if \texttt{a} and \texttt{b} are
variables, we may not substitute \texttt{A} with \texttt{a} and \texttt{B} with \texttt{b}
unless we have also specified \texttt{\$d a b} for the theorem being proved; in
other words, the \texttt{\$d} property associated with a pair of variables must
be effectively preserved after substitution.

The \texttt{\$d}\index{\texttt{\$d} statement} statement does {\em not} mean ``the
two variables may not be substituted with the same thing,'' as you might think
at first.  For example, substituting each of \texttt{A} and \texttt{B} in the above
example with identical symbol sequences consisting only of constants does not
cause a disjoint variable conflict, because two symbol sequences have no
variables in common (since they have no variables, period).  Similarly, a
conflict will not occur by substituting the two variables in a \texttt{\$d}
statement with the empty symbol sequence\index{empty substitution}.

The \texttt{\$d} statement does not have a direct counterpart in
ordinary mathematics, partly because the variables\index{variable} of
Metamath are not really the same as the variables\index{variable!in
ordinary mathematics} of ordinary mathematics but rather are
metavariables\index{metavariable} ranging over them (as well as over
other kinds of symbols and groups of symbols).  Depending on the
situation, we may informally interpret the \texttt{\$d} statement in
different ways.  Suppose, for example, that \texttt{x} and \texttt{y}
are variables ranging over numbers (more precisely, that \texttt{x} and
\texttt{y} are metavariables ranging over variables that range over
numbers), and that \texttt{ph} ($\varphi$) and \texttt{ps} ($\psi$) are
variables (more precisely, metavariables) ranging over formulas.  We can
make the following interpretations that correspond to the informal
language of ordinary mathematics:
\begin{quote}
\begin{tabbing}
\texttt{\$d x y \$.} means ``assume $x$ and $y$ are
distinct variables.''\\
\texttt{\$d x ph \$.} means ``assume $x$ does not
occur in $\varphi$.''\\
\texttt{\$d ph ps \$.} \=means ``assume $\varphi$ and
$\psi$ have no variables\\ \>in common.''
\end{tabbing}
\end{quote}\index{\texttt{\$d} statement}

\subsubsection{Compound \texttt{\$d} Statements}

The {\bf compound} version of the \texttt{\$d} statement is a shorthand for
specifying several variables whose substitutions must be pairwise disjoint.
Its syntax is:
\begin{center}
  \texttt{\$d} {\em variable}\ \,$\cdots$\ {\em variable} \texttt{\$.}
\end{center}\index{\texttt{\$d} statement}
Here, {\em variable} represents the token of a previously declared
variable (specifically, an active variable) and all {\em variable}\,s are
different.  The compound \texttt{\$d}
statement is internally broken up by Metamath into one simple \texttt{\$d}
statement for each possible pair of variables in the original \texttt{\$d}
statement.  For example,
\begin{center}
  \texttt{\$d w x y z \$.}
\end{center}
is equivalent to
\begin{center}
  \texttt{\$d w x \$.}\\
  \texttt{\$d w y \$.}\\
  \texttt{\$d w z \$.}\\
  \texttt{\$d x y \$.}\\
  \texttt{\$d x z \$.}\\
  \texttt{\$d y z \$.}
\end{center}

Two or more simple \texttt{\$d} statements specifying the same variable pair are
internally combined into a single \texttt{\$d} statement.  Thus the set of three
statements
\begin{center}
  \texttt{\$d x y \$.}
  \texttt{\$d x y \$.}
  \texttt{\$d y x \$.}
\end{center}
is equivalent to
\begin{center}
  \texttt{\$d x y \$.}
\end{center}

Similarly, compound \texttt{\$d} statements, after being internally broken up,
internally have their common variable pairs combined.  For example the
set of statements
\begin{center}
  \texttt{\$d x y A \$.}
  \texttt{\$d x y B \$.}
\end{center}
is equivalent to
\begin{center}
  \texttt{\$d x y \$.}
  \texttt{\$d x A \$.}
  \texttt{\$d y A \$.}
  \texttt{\$d x y \$.}
  \texttt{\$d x B \$.}
  \texttt{\$d y B \$.}
\end{center}
which is equivalent to
\begin{center}
  \texttt{\$d x y \$.}
  \texttt{\$d x A \$.}
  \texttt{\$d y A \$.}
  \texttt{\$d x B \$.}
  \texttt{\$d y B \$.}
\end{center}

Metamath\index{Metamath} automatically verifies that all \texttt{\$d}
restrictions are met whenever it verifies proofs.  \texttt{\$d} statements are
never referenced directly in proofs (this is why they do not have
labels\index{label}), but Metamath is always aware of which ones must be
satisfied (i.e.\ are active) and will notify you with an error message if any
violation occurs.

To illustrate how Metamath detects a missing \texttt{\$d}
statement, we will look at the following example from the
\texttt{set.mm} database.

\begin{verbatim}
$d x z $.  $d y z $.
$( Theorem to add distinct quantifier to atomic formula. $)
ax17eq $p |- ( x = y -> A. z x = y ) $=...
\end{verbatim}

This statement has the obvious requirement that $z$ must be
distinct\index{distinct variables} from $x$ in theorem \texttt{ax17eq} that
states $x=y \rightarrow \forall z \, x=y$ (well, obvious if you're a logician,
for otherwise we could conclude  $x=y \rightarrow \forall x \, x=y$, which is
false when the free variables $x$ and $y$ are equal).

Let's look at what happens if we edit the database to comment out this
requirement.

\begin{verbatim}
$( $d x z $. $) $d y z $.
$( Theorem to add distinct quantifier to atomic formula. $)
ax17eq $p |- ( x = y -> A. z x = y ) $=...
\end{verbatim}

When it tries to verify the proof, Metamath will tell you that \texttt{x} and
\texttt{z} must be disjoint, because one of its steps references an axiom or
theorem that has this requirement.

\begin{verbatim}
MM> verify proof ax17eq
ax17eq ?Error at statement 1918, label "ax17eq", type "$p":
      vz wal wi vx vy vz ax-13 vx vy weq vz vx ax-c16 vx vy
                                               ^^^^^
There is a disjoint variable ($d) violation at proof step 29.
Assertion "ax-c16" requires that variables "x" and "y" be
disjoint.  But "x" was substituted with "z" and "y" was
substituted with "x".  The assertion being proved, "ax17eq",
does not require that variables "z" and "x" be disjoint.
\end{verbatim}

We can see the substitutions into \texttt{ax-c16} with the following command.

\begin{verbatim}
MM> show proof ax17eq / detailed_step 29
Proof step 29:  pm2.61dd.2=ax-c16 $a |- ( A. z z = x -> ( x =
  y -> A. z x = y ) )
This step assigns source "ax-c16" ($a) to target "pm2.61dd.2"
($e).  The source assertion requires the hypotheses "wph"
($f, step 26), "vx" ($f, step 27), and "vy" ($f, step 28).
The parent assertion of the target hypothesis is "pm2.61dd"
($p, step 36).
The source assertion before substitution was:
    ax-c16 $a |- ( A. x x = y -> ( ph -> A. x ph ) )
The following substitutions were made to the source
assertion:
    Variable  Substituted with
     x         z
     y         x
     ph        x = y
The target hypothesis before substitution was:
    pm2.61dd.2 $e |- ( ph -> ch )
The following substitutions were made to the target
hypothesis:
    Variable  Substituted with
     ph        A. z z = x
     ch        ( x = y -> A. z x = y )
\end{verbatim}

The disjoint variable restrictions of \texttt{ax-c16} can be seen from the
\texttt{show state\-ment} command.  The line that begins ``\texttt{Its mandatory
dis\-joint var\-i\-able pairs are:}\ldots'' lists any \texttt{\$d} variable
pairs in brackets.

\begin{verbatim}
MM> show statement ax-c16/full
Statement 3033 is located on line 9338 of the file "set.mm".
"Axiom of Distinct Variables. ..."
  ax-c16 $a |- ( A. x x = y -> ( ph -> A. x ph ) ) $.
Its mandatory hypotheses in RPN order are:
  wph $f wff ph $.
  vx $f setvar x $.
  vy $f setvar y $.
Its mandatory disjoint variable pairs are:  <x,y>
The statement and its hypotheses require the variables:  x y
      ph
The variables it contains are:  x y ph
\end{verbatim}

Since Metamath will always detect when \texttt{\$d}\index{\texttt{\$d} statement}
statements are needed for a proof, you don't have to worry too much about
forgetting to put one in; it can always be added if you see the error message
above.  If you put in unnecessary \texttt{\$d} statements, the worst that could
happen is that your theorem might not be as general as it could be, and this
may limit its use later on.

On the other hand, when you introduce axioms (\texttt{\$a}\index{\texttt{\$a}
statement} statements), you must be very careful to properly specify the
necessary associated \texttt{\$d} statements since Metamath has no way of knowing
whether your axioms are correct.  For example, Metamath would have no idea
that \texttt{ax-c16}, which we are telling it is an axiom of logic, would lead to
contradictions if we omitted its associated \texttt{\$d} statement.

% This was previously a comment in footnote-sized type, but it can be
% hard to read this much text in a small size.
% As a result, it's been changed to normally-sized text.
\label{nodd}
You may wonder if it is possible to develop standard
mathematics in the Metamath language without the \texttt{\$d}\index{\texttt{\$d}
statement} statement, since it seems like a nuisance that complicates proof
verification. The \texttt{\$d} statement is not needed in certain subsets of
mathematics such as propositional calculus.  However, dummy
variables\index{dummy variable!eliminating} and their associated \texttt{\$d}
statements are impossible to avoid in proofs in standard first-order logic as
well as in the variant used in \texttt{set.mm}.  In fact, there is no upper bound to
the number of dummy variables that might be needed in a proof of a theorem of
first-order logic containing 3 or more variables, as shown by H.\
Andr\'{e}ka\index{Andr{\'{e}}ka, H.} \cite{Nemeti}.  A first-order system that
avoids them entirely is given in \cite{Megill}\index{Megill, Norman}; the
trick there is simply to embed harmlessly the necessary dummy variables into a
theorem being proved so that they aren't ``dummy'' anymore, then interpret the
resulting longer theorem so as to ignore the embedded dummy variables.  If
this interests you, the system in \texttt{set.mm} obtained from \texttt{ax-1}
through \texttt{ax-c14} in \texttt{set.mm}, and deleting \texttt{ax-c16} and \texttt{ax-5},
requires no \texttt{\$d} statements but is logically complete in the sense
described in \cite{Megill}.  This means it can prove any theorem of
first-order logic as long as we add to the theorem an antecedent that embeds
dummy and any other variables that must be distinct.  In a similar fashion,
axioms for set theory can be devised that
do not require distinct variable
provisos\index{Set theory without distinct variable provisos},
as explained at
\url{http://us.metamath.org/mpeuni/mmzfcnd.html}.
Together, these in principle allow all of
mathematics to be developed under Metamath without a \texttt{\$d} statement,
although the length of the resulting theorems will grow as more and
more dummy variables become required in their proofs.

\subsection{The \texttt{\$f}
and \texttt{\$e} Statements}\label{dollaref}
\index{\texttt{\$e} statement}
\index{\texttt{\$f} statement}
\index{floating hypothesis}
\index{essential hypothesis}
\index{variable-type hypothesis}
\index{logical hypothesis}
\index{hypothesis}

Metamath has two kinds of hypo\-theses, the \texttt{\$f}\index{\texttt{\$f}
statement} or {\bf variable-type} hypothesis and the \texttt{\$e} or {\bf logical}
hypo\-the\-sis.\index{\texttt{\$d} statement}\footnote{Strictly speaking, the
\texttt{\$d} statement is also a hypothesis, but it is never directly referenced
in a proof, so we call it a restriction rather than a hypothesis to lessen
confusion.  The checking for violations of \texttt{\$d} restrictions is automatic
and built into Metamath's proof-checking algorithm.} The letters \texttt{f} and
\texttt{e} stand for ``floating''\index{floating hypothesis} (roughly meaning
used only if relevant) and ``essential''\index{essential hypothesis} (meaning
always used) respectively, for reasons that will become apparent
when we discuss frames in
Section~\ref{frames} and scoping in Section~\ref{scoping}. The syntax of these
are as follows:
\begin{center}
  {\em label} \texttt{\$f} {\em typecode} {\em variable} \texttt{\$.}\\
  {\em label} \texttt{\$e} {\em typecode}
      {\em math-symbol}\ \,$\cdots$\ {\em math-symbol} \texttt{\$.}\\
\end{center}
\index{\texttt{\$e} statement}
\index{\texttt{\$f} statement}
A hypothesis must have a {\em label}\index{label}.  The expression in a
\texttt{\$e} hypothesis consists of a typecode (an active constant math symbol)
followed by a sequence
of zero or more math symbols. Each math symbol (including {\em constant}
and {\em variable}) must be a previously declared constant or variable.  (In
addition, each math symbol must be active, which will be covered when we
discuss scoping statements in Section~\ref{scoping}.)  You use a \texttt{\$f}
hypothesis to specify the
nature or {\bf type}\index{variable type}\index{type} of a variable (such as ``let $x$ be an
integer'') and use a \texttt{\$e} hypothesis to express a logical truth (such as
``assume $x$ is prime'') that must be established in order for an assertion
requiring it to also be true.

A variable must have its type specified in a \texttt{\$f} statement before
it may be used in a \texttt{\$e}, \texttt{\$a}, or \texttt{\$p}
statement.  There may be only one (active) \texttt{\$f} statement for a
given variable.  (``Active'' is defined in Section~\ref{scoping}.)

In ordinary mathematics, theorems\index{theorem} are often expressed in the
form ``Assume $P$; then $Q$,'' where $Q$ is a statement that you can derive
if you start with statement $P$.\index{free variable}\footnote{A stronger
version of a theorem like this would be the {\em single} formula $P\rightarrow
Q$ ($P$ implies $Q$) from which the weaker version above follows by the rule
of modus ponens in logic.  We are not discussing this stronger form here.  In
the weaker form, we are saying only that if we can {\em prove} $P$, then we can
{\em prove} $Q$.  In a logician's language, if $x$ is the only free variable
in $P$ and $Q$, the stronger form is equivalent to $\forall x ( P \rightarrow
Q)$ (for all $x$, $P$ implies $Q$), whereas the weaker form is equivalent to
$\forall x P \rightarrow \forall x Q$. The stronger form implies the weaker,
but not vice-versa.  To be precise, the weaker form of the theorem is more
properly called an ``inference'' rather than a theorem.}\index{inference}
In the
Metamath\index{Metamath} language, you would express mathematical statement
$P$ as a hypothesis (a \texttt{\$e} Metamath language statement in this case) and
statement $Q$ as a provable assertion (a \texttt{\$p}\index{\texttt{\$p} statement}
statement).

Some examples of hypotheses you might encounter in logic and set theory are
\begin{center}
  \texttt{stmt1 \$f wff P \$.}\\
  \texttt{stmt2 \$f setvar x \$.}\\
  \texttt{stmt3 \$e |- ( P -> Q ) \$.}
\end{center}
\index{\texttt{\$e} statement}
\index{\texttt{\$f} statement}
Informally, these would be read, ``Let $P$ be a well-formed-formula,'' ``Let
$x$ be an (individual) variable,'' and ``Assume we have proved $P \rightarrow
Q$.''  The turnstile symbol \,$\vdash$\index{turnstile ({$\,\vdash$})} is
commonly used in logic texts to mean ``a proof exists for.''

To summarize:
\begin{itemize}
\item A \texttt{\$f} hypothesis tells Metamath the type or kind of its variable.
It is analogous to a variable declaration in a computer language that
tells the compiler that a variable is an integer or a floating-point
number.
\item The \texttt{\$e} hypothesis corresponds to what you would usually call a
``hypothesis'' in ordinary mathematics.
\end{itemize}

Before an assertion\index{assertion} (\texttt{\$a} or \texttt{\$p} statement) can be
referenced in a proof, all of its associated \texttt{\$f} and \texttt{\$e} hypotheses
(i.e.\ those \texttt{\$e} hypotheses that are active) must be satisfied (i.e.
established by the proof).  The meaning of ``associated'' (which we will call
{\bf mandatory} in Section~\ref{frames}) will become clear when we discuss
scoping later.

Note that after any \texttt{\$f}, \texttt{\$e},
\texttt{\$a}, or \texttt{\$p} token there is a required
\textit{typecode}\index{typecode}.
The typecode is a constant used to enforce types of expressions.
This will become clearer once we learn more about
assertions (\texttt{\$a} and \texttt{\$p} statements).
An example may also clarify their purpose.
In the
\texttt{set.mm}\index{set theory database (\texttt{set.mm})}%
\index{Metamath Proof Explorer}
database,
the following typecodes are used:

\begin{itemize}
\item \texttt{wff} :
  Well-formed formula (wff) symbol
  (read: ``the following symbol sequence is a wff'').
% The *textual* typecode for turnstile is "|-", but when read it's a little
% confusing, so I intentionally display the mathematical symbol here instead
% (I think it's clearer in this context).
\item \texttt{$\vdash$} :
  Turnstile (read: ``the following symbol sequence is provable'' or
  ``a proof exists for'').
\item \texttt{setvar} :
  Individual set variable type (read: ``the following is an
  individual set variable'').
  Note that this is \textit{not} the type of an arbitrary set expression,
  instead, it is used to ensure that there is only a single symbol used
  after quantifiers like for-all ($\forall$) and there-exists ($\exists$).
\item \texttt{class} :
  An expression that is a syntactically valid class expression.
  All valid set expressions are also valid class expression, so expressions
  of sets normally have the \texttt{class} typecode.
  Use the \texttt{class} typecode,
  \textit{not} the \texttt{setvar} typecode,
  for the type of set expressions unless you are specifically identifying
  a single set variable.
\end{itemize}

\subsection{Assertions (\texttt{\$a} and \texttt{\$p} Statements)}
\index{\texttt{\$a} statement}
\index{\texttt{\$p} statement}\index{assertion}\index{axiomatic assertion}
\index{provable assertion}

There are two types of assertions, \texttt{\$a}\index{\texttt{\$a} statement}
statements ({\bf axiomatic assertions}) and \texttt{\$p} statements ({\bf
provable assertions}).  Their syntax is as follows:
\begin{center}
  {\em label} \texttt{\$a} {\em typecode} {\em math-symbol} \ldots
         {\em math-symbol} \texttt{\$.}\\
  {\em label} \texttt{\$p} {\em typecode} {\em math-symbol} \ldots
        {\em math-symbol} \texttt{\$=} {\em proof} \texttt{\$.}
\end{center}
\index{\texttt{\$a} statement}
\index{\texttt{\$p} statement}
\index{\texttt{\$=} keyword}
An assertion always requires a {\em label}\index{label}. The expression in an
assertion consists of a typecode (an active constant)
followed by a sequence of zero
or more math symbols.  Each math symbol, including any {\em constant}, must be a
previously declared constant or variable.  (In addition, each math symbol
must be active, which will be covered when we discuss scoping statements in
Section~\ref{scoping}.)

A \texttt{\$a} statement is usually a definition of syntax (for example, if $P$
and $Q$ are wffs then so is $(P\to Q)$), an axiom\index{axiom} of ordinary
mathematics (for example, $x=x$), or a definition\index{definition} of
ordinary mathematics (for example, $x\ne y$ means $\lnot x=y$). A \texttt{\$p}
statement is a claim that a certain combination of math symbols follows from
previous assertions and is accompanied by a proof that demonstrates it.

Assertions can also be referenced in (later) proofs in order to derive new
assertions from them. The label of an assertion is used to refer to it in a
proof. Section~\ref{proof} will describe the proof in detail.

Assertions also provide the primary means for communicating the mathematical
results in the database to people.  Proofs (when conveniently displayed)
communicate to people how the results were arrived at.

\subsubsection{The \texttt{\$a} Statement}
\index{\texttt{\$a} statement}

Axiomatic assertions (\texttt{\$a} statements) represent the starting points from
which other assertions (\texttt{\$p}\index{\texttt{\$p} statement} statements) are
derived.  Their most obvious use is for specifying ordinary mathematical
axioms\index{axiom}, but they are also used for two other purposes.

First, Metamath\index{Metamath} needs to know the syntax of symbol
sequences that constitute valid mathematical statements.  A Metamath
proof must be broken down into much more detail than ordinary
mathematical proofs that you may be used to thinking of (even the
``complete'' proofs of formal logic\index{formal logic}).  This is one
of the things that makes Metamath a general-purpose language,
independent of any system of logic or even syntax.  If you want to use a
substitution instance of an assertion as a step in a proof, you must
first prove that the substitution is syntactically correct (or if you
prefer, you must ``construct'' it), showing for example that the
expression you are substituting for a wff metavariable is a valid wff.
The \texttt{\$a}\index{\texttt{\$a} statement} statement is used to
specify those combinations of symbols that are considered syntactically
valid, such as the legal forms of wffs.

Second, \texttt{\$a} statements are used to specify what are ordinarily thought of
as definitions, i.e.\ new combinations of symbols that abbreviate other
combinations of symbols.  Metamath makes no distinction\index{axiom vs.\
definition} between axioms\index{axiom} and definitions\index{definition}.
Indeed, it has been argued that such distinction should not be made even in
ordinary mathematics; see Section~\ref{definitions}, which discusses the
philosophy of definitions.  Section~\ref{hierarchy} discusses some
technical requirements for definitions.  In \texttt{set.mm} we adopt the
convention of prefixing axiom labels with \texttt{ax-} and definition labels with
\texttt{df-}\index{label}.

The results that can be derived with the Metamath language are only as good as
the \texttt{\$a}\index{\texttt{\$a} statement} statements used as their starting
point.  We cannot stress this too strongly.  For example, Metamath will
not prevent you from specifying $x\neq x$ as an axiom of logic.  It is
essential that you scrutinize all \texttt{\$a} statements with great care.
Because they are a source of potential pitfalls, it is best not to add new
ones (usually new definitions) casually; rather you should carefully evaluate
each one's necessity and advantages.

Once you have in place all of the basic axioms\index{axiom} and
rules\index{rule} of a mathematical theory, the only \texttt{\$a} statements that
you will be adding will be what are ordinarily called definitions.  In
principle, definitions should be in some sense eliminable from the language of
a theory according to some convention (usually involving logical equivalence
or equality).  The most common convention is that any formula that was
syntactically valid but not provable before the definition was introduced will
not become provable after the definition is introduced.  In an ideal world,
definitions should not be present at all if one is to have absolute confidence
in a mathematical result.  However, they are necessary to make
mathematics practical, for otherwise the resulting formulas would be
extremely long and incomprehensible.  Since the nature of definitions (in the
most general sense) does not permit them to automatically be verified as
``proper,''\index{proper definition}\index{definition!proper} the judgment of
the mathematician is required to ensure it.  (In \texttt{set.mm} effort was made
to make almost all definitions directly eliminable and thus minimize the need
for such judgment.)

If you are not a mathematician, it may be best not to add or change any
\texttt{\$a}\index{\texttt{\$a} statement} statements but instead use
the mathematical language already provided in standard databases.  This
way Metamath will not allow you to make a mistake (i.e.\ prove a false
result).


\subsection{Frames}\label{frames}

We now introduce the concept of a collection of related Metamath statements
called a frame.  Every assertion (\texttt{\$a} or \texttt{\$p} statement) in the database has
an associated frame.

A {\bf frame}\index{frame} is a sequence of \texttt{\$d}, \texttt{\$f},
and \texttt{\$e} statements (zero or more of each) followed by one
\texttt{\$a} or \texttt{\$p} statement, subject to certain conditions we
will describe.  For simplicity we will assume that all math symbol
tokens used are declared at the beginning of the database with
\texttt{\$c} and \texttt{\$v} statements (which are not properly part of
a frame).  Also for simplicity we will assume there are only simple
\texttt{\$d} statements (those with only two variables) and imagine any
compound \texttt{\$d} statements (those with more than two variables) as
broken up into simple ones.

A frame groups together those hypotheses (and \texttt{\$d} statements) relevant
to an assertion (\texttt{\$a} or \texttt{\$p} statement).  The statements in a frame
may or may not be physically adjacent in a database; we will cover
this in our discussion of scoping statements
in Section~\ref{scoping}.

A frame has the following properties:
\begin{enumerate}
 \item The set of variables contained in its \texttt{\$f} statements must
be identical to the set of variables contained in its \texttt{\$e},
\texttt{\$a}, and/or \texttt{\$p} statements.  In other words, each
variable in a \texttt{\$e}, \texttt{\$a}, or \texttt{\$p} statement must
have an associated ``variable type'' defined for it in a \texttt{\$f}
statement.
  \item No two \texttt{\$f} statements may contain the same variable.
  \item Any \texttt{\$f} statement
must occur before a \texttt{\$e} statement in which its variable occurs.
\end{enumerate}

The first property determines the set of variables occurring in a frame.
These are the {\bf mandatory
variables}\index{mandatory variable} of the frame.  The second property
tells us there must be only one type specified for a variable.
The last property is not a theoretical requirement but it
makes parsing of the database easier.

For our examples, we assume our database has the following declarations:

\begin{verbatim}
$v P Q R $.
$c -> ( ) |- wff $.
\end{verbatim}

The following sequence of statements, describing the modus ponens inference
rule, is an example of a frame:

\begin{verbatim}
wp  $f wff P $.
wq  $f wff Q $.
maj $e |- ( P -> Q ) $.
min $e |- P $.
mp  $a |- Q $.
\end{verbatim}

The following sequence of statements is not a frame because \texttt{R} does not
occur in the \texttt{\$e}'s or the \texttt{\$a}:

\begin{verbatim}
wp  $f wff P $.
wq  $f wff Q $.
wr  $f wff R $.
maj $e |- ( P -> Q ) $.
min $e |- P $.
mp  $a |- Q $.
\end{verbatim}

The following sequence of statements is not a frame because \texttt{Q} does not
occur in a \texttt{\$f}:

\begin{verbatim}
wp  $f wff P $.
maj $e |- ( P -> Q ) $.
min $e |- P $.
mp  $a |- Q $.
\end{verbatim}

The following sequence of statements is not a frame because the \texttt{\$a} statement is
not the last one:

\begin{verbatim}
wp  $f wff P $.
wq  $f wff Q $.
maj $e |- ( P -> Q ) $.
mp  $a |- Q $.
min $e |- P $.
\end{verbatim}

Associated with a frame is a sequence of {\bf mandatory
hypotheses}\index{mandatory hypothesis}.  This is simply the set of all
\texttt{\$f} and \texttt{\$e} statements in the frame, in the order they
appear.  A frame can be referenced in a later proof using the label of
the \texttt{\$a} or \texttt{\$p} assertion statement, and the proof
makes an assignment to each mandatory hypothesis in the order in which
it appears.  This means the order of the hypotheses, once chosen, must
not be changed so as not to affect later proofs referencing the frame's
assertion statement.  (The Metamath proof verifier will, of course, flag
an error if a proof becomes incorrect by doing this.)  Since proofs make
use of ``Reverse Polish notation,'' described in Section~\ref{proof}, we
call this order the {\bf RPN order}\index{RPN order} of the hypotheses.

Note that \texttt{\$d} statements are not part of the set of mandatory
hypotheses, and their order doesn't matter (as long as they satisfy the
fourth property for a frame described above).  The \texttt{\$d}
statements specify restrictions on variables that must be satisfied (and
are checked by the proof verifier) when expressions are substituted for
them in a proof, and the \texttt{\$d} statements themselves are never
referenced directly in a proof.

A frame with a \texttt{\$p} (provable) statement requires a proof as part of the
\texttt{\$p} statement.  Sometimes in a proof we want to make use of temporary or
dummy variables\index{dummy variable} that do not occur in the \texttt{\$p}
statement or its mandatory hypotheses.  To accommodate this we define an {\bf
extended frame}\index{extended frame} as a frame together with zero or more
\texttt{\$d} and \texttt{\$f} statements that reference variables not among the
mandatory variables of the frame.  Any new variables referenced are called the
{\bf optional variables}\index{optional variable} of the extended frame. If a
\texttt{\$f} statement references an optional variable it is called an {\bf
optional hypothesis}\index{optional hypothesis}, and if one or both of the
variables in a \texttt{\$d} statement are optional variables it is called an {\bf
optional disjoint-variable restriction}\index{optional disjoint-variable
restriction}.  Properties 2 and 3 for a frame also apply to an extended
frame.

The concept of optional variables is not meaningful for frames with \texttt{\$a}
statements, since those statements have no proofs that might make use of them.
There is no restriction on including optional hypotheses in the extended frame
for a \texttt{\$a} statement, but they serve no purpose.

The following set of statements is an example of an extended frame, which
contains an optional variable \texttt{R} and an optional hypothesis \texttt{wr}.  In
this example, we suppose the rule of modus ponens is not an axiom but is
derived as a theorem from earlier statements (we omit its presumed proof).
Variable \texttt{R} may be used in its proof if desired (although this would
probably have no advantage in propositional calculus).  Note that the sequence
of mandatory hypotheses in RPN order is still \texttt{wp}, \texttt{wq}, \texttt{maj},
\texttt{min} (i.e.\ \texttt{wr} is omitted), and this sequence is still assumed
whenever the assertion \texttt{mp} is referenced in a subsequent proof.

\begin{verbatim}
wp  $f wff P $.
wq  $f wff Q $.
wr  $f wff R $.
maj $e |- ( P -> Q ) $.
min $e |- P $.
mp  $p |- Q $= ... $.
\end{verbatim}

Every frame is an extended frame, but not every extended frame is a frame, as
this example shows.  The underlying frame for an extended frame is
obtained by simply removing all statements containing optional variables.
Any proof referencing an assertion will ignore any extensions to its
frame, which means we may add or delete optional hypotheses at will without
affecting subsequent proofs.

The conceptually simplest way of organizing a Metamath database is as a
sequence of extended frames.  The scoping statements
\texttt{\$\char`\{}\index{\texttt{\$\char`\{} and \texttt{\$\char`\}}
keywords} and \texttt{\$\char`\}} can be used to delimit the start and
end of an extended frame, leading to the following possible structure for a
database.  \label{framelist}

\vskip 2ex
\setbox\startprefix=\hbox{\tt \ \ \ \ \ \ \ \ }
\setbox\contprefix=\hbox{}
\startm
\m{\mbox{(\texttt{\$v} {\em and} \texttt{\$c}\,{\em statements})}}
\endm
\startm
\m{\mbox{\texttt{\$\char`\{}}}
\endm
\startm
\m{\mbox{\texttt{\ \ } {\em extended frame}}}
\endm
\startm
\m{\mbox{\texttt{\$\char`\}}}}
\endm
\startm
\m{\mbox{\texttt{\$\char`\{}}}
\endm
\startm
\m{\mbox{\texttt{\ \ } {\em extended frame}}}
\endm
\startm
\m{\mbox{\texttt{\$\char`\}}}}
\endm
\startm
\m{\mbox{\texttt{\ \ \ \ \ \ \ \ \ }}\vdots}
\endm
\vskip 2ex

In practice, this structure is inconvenient because we have to repeat
any \texttt{\$f}, \texttt{\$e}, and \texttt{\$d} statements over and
over again rather than stating them once for use by several assertions.
The scoping statements, which we will discuss next, allow this to be
done.  In principle, any Metamath database can be converted to the above
format, and the above format is the most convenient to use when studying
a Metamath database as a formal system%
%% Uncomment this when uncommenting section {formalspec} below
   (Appendix \ref{formalspec})%
.
In fact, Metamath internally converts the database to the above format.
The command \texttt{show statement} in the Metamath program will show
you the contents of the frame for any \texttt{\$a} or \texttt{\$p}
statement, as well as its extension in the case of a \texttt{\$p}
statement.

%c%(provided that all ``local'' variables and constants with limited scope have
%c%unique names),

During our discussion of scoping statements, it may be helpful to
think in terms of the equivalent sequence of frames that will result when
the database is parsed.  Scoping (other than the limited
use above to delimit frames) is not a theoretical requirement for
Metamath but makes it more convenient.


\subsection{Scoping Statements (\texttt{\$\{} and \texttt{\$\}})}\label{scoping}
\index{\texttt{\$\char`\{} and \texttt{\$\char`\}} keywords}\index{scoping statement}

%c%Some Metamath statements may be needed only temporarily to
%c%serve a specific purpose, and after we're done with them we would like to
%c%disregard or ignore them.  For example, when we're finished using a variable,
%c%we might want to
%c%we might want to free up the token\index{token} used to name it so that the
%c%token can be used for other purposes later on, such as a different kind of
%c%variable or even a constant.  In the terminology of computer programming, we
%c%might want to let some symbol declarations be ``local'' rather than ``global.''
%c%\index{local symbol}\index{global symbol}

The {\bf scoping} statements, \texttt{\$\char`\{} ({\bf start of block}) and \texttt{\$\char`\}}
({\bf end of block})\index{block}, provide a means for controlling the portion
of a database over which certain statement types are recognized.  The
syntax of a scoping statement is very simple; it just consists of the
statement's keyword:
\begin{center}
\texttt{\$\char`\{}\\
\texttt{\$\char`\}}
\end{center}
\index{\texttt{\$\char`\{} and \texttt{\$\char`\}} keywords}

For example, consider the following database where we have stripped out
all tokens except the scoping statement keywords.  For the purpose of the
discussion, we have added subscripts to the scoping statements; these subscripts
do not appear in the actual database.
\[
 \mbox{\tt \ \$\char`\{}_1
 \mbox{\tt \ \$\char`\{}_2
 \mbox{\tt \ \$\char`\}}_2
 \mbox{\tt \ \$\char`\{}_3
 \mbox{\tt \ \$\char`\{}_4
 \mbox{\tt \ \$\char`\}}_4
 \mbox{\tt \ \$\char`\}}_3
 \mbox{\tt \ \$\char`\}}_1
\]
Each \texttt{\$\char`\{} statement in this example is said to be {\bf
matched} with the \texttt{\$\char`\}} statement that has the same
subscript.  Each pair of matched scoping statements defines a region of
the database called a {\bf block}.\index{block} Blocks can be {\bf
nested}\index{nested block} inside other blocks; in the example, the
block defined by $\mbox{\tt \$\char`\{}_4$ and $\mbox{\tt \$\char`\}}_4$
is nested inside the block defined by $\mbox{\tt \$\char`\{}_3$ and
$\mbox{\tt \$\char`\}}_3$ as well as inside the block defined by
$\mbox{\tt \$\char`\{}_1$ and $\mbox{\tt \$\char`\}}_1$.  In general, a
block may be empty, it may contain only non-scoping
statements,\footnote{Those statements other than \texttt{\$\char`\{} and
\texttt{\$\char`\}}.}\index{non-scoping statement} or it may contain any
mixture of other blocks and non-scoping statements.  (This is called a
``recursive'' definition\index{recursive definition} of a block.)

Associated with each block is a number called its {\bf nesting
level}\index{nesting level} that indicates how deeply the block is nested.
The nesting levels of the blocks in our example are as follows:
\[
  \underbrace{
    \mbox{\tt \ }
    \underbrace{
     \mbox{\tt \$\char`\{\ }
     \underbrace{
       \mbox{\tt \$\char`\{\ }
       \mbox{\tt \$\char`\}}
     }_{2}
     \mbox{\tt \ }
     \underbrace{
       \mbox{\tt \$\char`\{\ }
       \underbrace{
         \mbox{\tt \$\char`\{\ }
         \mbox{\tt \$\char`\}}
       }_{3}
       \mbox{\tt \ \$\char`\}}
     }_{2}
     \mbox{\tt \ \$\char`\}}
   }_{1}
   \mbox{\tt \ }
 }_{0}
\]
\index{\texttt{\$\char`\{} and \texttt{\$\char`\}} keywords}
The entire database is considered to be one big block (the {\bf outermost}
block) with a nesting level of 0.  The outermost block is {\em not} bracketed
by scoping statements.\footnote{The language was designed this way so that
several source files can be joined together more easily.}\index{outermost
block}

All non-scoping Metamath statements become recognized or {\bf
active}\index{active statement} at the place where they appear.\footnote{To
keep things slightly simpler, we do not bother to define the concept of
``active'' for the scoping statements.}  Certain of these statement types
become inactive at the end of the block in which they appear; these statement
types are:
\begin{center}
  \texttt{\$c}, \texttt{\$v}, \texttt{\$d}, \texttt{\$e}, and \texttt{\$f}.
%  \texttt{\$v}, \texttt{\$f}, \texttt{\$e}, and \texttt{\$d}.
\end{center}
\index{\texttt{\$c} statement}
\index{\texttt{\$d} statement}
\index{\texttt{\$e} statement}
\index{\texttt{\$f} statement}
\index{\texttt{\$v} statement}
The other statement types remain active forever (i.e.\ through the end of the
database); they are:
\begin{center}
  \texttt{\$a} and \texttt{\$p}.
%  \texttt{\$c}, \texttt{\$a}, and \texttt{\$p}.
\end{center}
\index{\texttt{\$a} statement}
\index{\texttt{\$p} statement}
Any statement (of these 7 types) located in the outermost
block\index{outermost block} will remain active through the end of the
database and thus are effectively ``global'' statements.\index{global
statement}

All \texttt{\$c} statements must be placed in the outermost block.  Since they are
therefore always global, they could be considered as belonging to both of the
above categories.

The {\bf scope}\index{scope} of a statement is the set of statements that
recognize it as active.

%c%The concept of ``active'' is also defined for math symbols\index{math
%c%symbol}.  Math symbols (constants\index{constant} and
%c%variables\index{variable}) become {\bf active}\index{active
%c%math symbol} in the \texttt{\$c}\index{\texttt{\$c}
%c%statement} and \texttt{\$v}\index{\texttt{\$v} statement} statements that
%c%declare them.  They become inactive when their declaration statements become
%c%inactive.

The concept of ``active'' is also defined for math symbols\index{math
symbol}.  Math symbols (constants\index{constant} and
variables\index{variable}) become {\bf active}\index{active math symbol}
in the \texttt{\$c}\index{\texttt{\$c} statement} and
\texttt{\$v}\index{\texttt{\$v} statement} statements that declare them.
A variable becomes inactive when its declaration statement becomes
inactive.  Because all \texttt{\$c} statements must be in the outermost
block, a constant will never become inactive after it is declared.

\subsubsection{Redeclaration of Math Symbols}
\index{redeclaration of symbols}\label{redeclaration}

%c%A math symbol may not be declared a second time while it is active, but it may
%c%be declared again after it becomes inactive.

A variable may not be declared a second time while it is active, but it may be
declared again after it becomes inactive.  This provides a convenient way to
introduce ``local'' variables,\index{local variable} i.e.\ temporary variables
for use in the frame of an assertion or in a proof without keeping them around
forever.  A previously declared variable may not be redeclared as a constant.

A constant may not be redeclared.  And, as mentioned above, constants must be
declared in the outermost block.

The reason variables may have limited scope but not constants is that an
assertion (\texttt{\$a} or \texttt{\$p} statement) remains available for use in
proofs through the end of the database.  Variables in an assertion's frame may
be substituted with whatever is needed in a proof step that references the
assertion, whereas constants remain fixed and may not be substituted with
anything.  The particular token used for a variable in an assertion's frame is
irrelevant when the assertion is referenced in a proof, and it doesn't matter
if that token is not available outside of the referenced assertion's frame.
Constants, however, must be globally fixed.

There is no theoretical
benefit for the feature allowing variables to be active for limited scopes
rather than global. It is just a convenience that allows them, for example, to
be locally grouped together with their corresponding \texttt{\$f} variable-type
declarations.

%c%If you declare a math symbol more than once, internally Metamath considers it a
%c%new distinct symbol, even though it has the same name.  If you are unaware of
%c%this, you may find that what you think are correct proofs are incorrectly
%c%rejected as invalid, because Metamath may tell you that a constant you
%c%previously declared does not match a newly declared math symbol with the same
%c%name.  For details on this subtle point, see the Comment on
%c%p.~\pageref{spec4comment}.  This is done purposely to allow temporary
%c%constants to be introduced while developing a subtheory, then allow their math
%c%symbol tokens to be reused later on; in general they will not refer to the
%c%same thing.  In practice, you would not ordinarily reuse the names of
%c%constants because it would tend to be confusing to the reader.  The reuse of
%c%names of variables, on the other hand, is something that is often useful to do
%c%(for example it is done frequently in \texttt{set.mm}).  Since variables in an
%c%assertion referenced in a proof can be substituted as needed to achieve a
%c%symbol match, this is not an issue.

% (This section covers a somewhat advanced topic you may want to skip
% at first reading.)
%
% Under certain circumstances, math symbol\index{math symbol}
% tokens\index{token} may be redeclared (i.e.\ the token
% may appear in more than
% one \texttt{\$c}\index{\texttt{\$c} statement} or \texttt{\$v}\index{\texttt{\$v}
% statement} statement).  You might want to do this say, to make temporary use
% of a variable name without having to worry about its affect elsewhere,
% somewhat analogous to declaring a local variable in a standard computer
% language.  Understanding what goes on when math symbol tokens are redeclared
% is a little tricky to understand at first, since it requires that we
% distinguish the token itself from the math symbol that it names.  It will help
% if we first take a peek at the internal workings of the
% Metamath\index{Metamath} program.
%
% Metamath reserves a memory location for each occurrence of a
% token\index{token} in a declaration statement (\texttt{\$c}\index{\texttt{\$c}
% statement} or \texttt{\$v}\index{\texttt{\$v} statement}).  If a given token appears
% in more than one declaration statement, it will refer to more than one memory
% locations.  A math symbol\index{math symbol} may be thought of as being one of
% these memory locations rather than as the token itself.  Only one of the
% memory locations associated with a given token may be active at any one time.
% The math symbol (memory location) that gets looked up when the token appears
% in a non-declaration statement is the one that happens to be active at that
% time.
%
% We now look at the rules for the redeclaration\index{redeclaration of symbols}
% of math symbol tokens.
% \begin{itemize}
% \item A math symbol token may not be declared twice in the
% same block.\footnote{While there is no theoretical reason for disallowing
% this, it was decided in the design of Metamath that allowing it would offer no
% advantage and might cause confusion.}
% \item An inactive math symbol may always be
% redeclared.
% \item  An active math symbol may be redeclared in a different (i.e.\
% inner) block\index{block} from the one it became active in.
% \end{itemize}
%
% When a math symbol token is redeclared, it conceptually refers to a different
% math symbol, just as it would be if it were called a different name.  In
% addition, the original math symbol that it referred to, if it was active,
% temporarily becomes inactive.  At the end of the block in which the
% redeclaration occurred, the new math symbol\index{math symbol} becomes
% inactive and the original symbol becomes active again.  This concept is
% illustrated in the following example, where the symbol \texttt{e} is
% ordinarily a constant (say Euler's constant, 2.71828...) but
% temporarily we want to use it as a ``local'' variable, say as a coefficient
% in the equation $a x^4 + b x^3 + c x^2 + d x + e$:
% \[
%   \mbox{\tt \$\char`\{\ \$c e \$.}
%   \underbrace{
%     \ \ldots\ %
%     \mbox{\tt \$\char`\{}\ \ldots\ %
%   }_{\mbox{\rm region A}}
%   \mbox{\tt \$v e \$.}
%   \underbrace{
%     \mbox{\ \ \ \ldots\ \ \ }
%   }_{\mbox{\rm region B}}
%   \mbox{\tt \$\char`\}}
%   \underbrace{
%     \mbox{\ \ \ \ldots\ \ \ }
%   }_{\mbox{\rm region C}}
%   \mbox{\tt \$\char`\}}
% \]
% \index{\texttt{\$\char`\{} and \texttt{\$\char`\}} keywords}
% In region A, the token \texttt{e} refers to a constant.  It is redeclared as a
% variable in region B, and any reference to it in this region will refer to this
% variable.  In region C, the redeclaration becomes inactive, and the original
% declaration becomes active again.  In region C, the token \texttt{x} refers to the
% original constant.
%
% As a practical matter, overuse of math symbol\index{math symbol}
% redeclarations\index{redeclaration of symbols} can be confusing (even though
% it is well-defined) and is best avoided when possible.  Here are some good
% general guidelines you can follow.  Usually, you should declare all
% constants\index{constant} in the outermost block\index{outermost block},
% especially if they are general-purpose (such as the token \verb$A.$, meaning
% $\forall$ or ``for all'').  This will make them ``globally'' active (although
% as in the example above local redeclarations will temporarily make them
% inactive.)  Most or all variables\index{variable}, on the other hand, could be
% declared in inner blocks, so that the token for them can be used later for a
% different type of variable or a constant.  (The names of the variables you
% choose are not used when you refer to an assertion\index{assertion} in a
% proof, whereas constants must match exactly.  A locally declared constant will
% not match a globally declared constant in a proof, even if they use the same
% token, because Metamath internally considers them to be different math
% symbols.)  To avoid confusion, you should generally avoid redeclaring active
% variables.  If you must redeclare them, do so at the beginning of a block.
% The temporary declaration of constants in inner blocks might be occasionally
% appropriate when you make use of a temporary definition to prove lemmas
% leading to a main result that does not make direct use of the definition.
% This way, you will not clutter up your database with a large number of
% seldom-used global constant symbols.  You might want to note that while
% inactive constants may not appear directly in an assertion (a \texttt{\$a}\index{\texttt{\$a}
% statement} or \texttt{\$p}\index{\texttt{\$p} statement}
% statement), they may be indirectly used in the proof of a \texttt{\$p} statement
% so long as they do not appear in the final math symbol sequence constructed by
% the proof.  In the end, you will have to use your best judgment, taking into
% account standard mathematical usage of the symbols as well as consideration
% for the reader of your work.
%
% \subsubsection{Reuse of Labels}\index{reuse of labels}\index{label}
%
% The \texttt{\$e}\index{\texttt{\$e} statement}, \texttt{\$f}\index{\texttt{\$f}
% statement}, \texttt{\$a}\index{\texttt{\$a} statement}, and
% \texttt{\$p}\index{\texttt{\$p}
% statement} statement types require labels, which allow them to be
% referenced later inside proofs.  A label is considered {\bf
% active}\index{active label} when the statement it is associated with is
% active.  The token\index{token} for a label may be reused
% (redeclared)\index{redeclaration of labels} provided that it is not being used
% for a currently active label.  (Unlike the tokens for math symbols, active
% label tokens may not be redeclared in an inner scope.)  Note that the labels
% of \texttt{\$a} and \texttt{\$p} statements can never be reused after these
% statements appear, because these statements remain active through the end of
% the database.
%
% You might find the reuse of labels a convenient way to have standard names for
% temporary hypotheses, such as \texttt{h1}, \texttt{h2}, etc.  This way you don't have
% to invent unique names for each of them, and in some cases it may be less
% confusing to the reader (although in other cases it might be more confusing, if
% the hypothesis is located far away from the assertion that uses
% it).\footnote{The current implementation requires that all labels, even
% inactive ones, be unique.}

\subsubsection{Frames Revisited}\index{frames and scoping statements}

Now that we have covered scoping, we will look at how an arbitrary
Metamath database can be converted to the simple sequence of extended
frames described on p.~\pageref{framelist}.  This is also how Metamath
stores the database internally when it reads in the database
source.\label{frameconvert} The method is simple.  First, we collect all
constant and variable (\texttt{\$c} and \texttt{\$v}) declarations in
the database, ignoring duplicate declarations of the same variable in
different scopes.  We then put our collected \texttt{\$c} and
\texttt{\$v} declarations at the beginning of the database, so that
their scope is the entire database.  Next, for each assertion in the
database, we determine its frame and extended frame.  The extended frame
is simply the \texttt{\$f}, \texttt{\$e}, and \texttt{\$d} statements
that are active.  The frame is the extended frame with all optional
hypotheses removed.

An equivalent way of saying this is that the extended frame of an assertion
is the collection of all \texttt{\$f}, \texttt{\$e}, and \texttt{\$d} statements
whose scope includes the assertion.
The \texttt{\$f} and \texttt{\$e} statements
occur in the order they appear
(order is irrelevant for \texttt{\$d} statements).

%c%, renaming any
%c%redeclared variables as needed so that all of them have unique names.  (The
%c%exact renaming convention is unimportant.  You might imagine renaming
%c%different declarations of math symbol \texttt{a} as \texttt{a\$1}, \texttt{a\$2}, etc.\
%c%which would prevent any conflicts since \texttt{\$} is not a legal character in a
%c%math symbol token.)

\section{The Anatomy of a Proof} \label{proof}
\index{proof!Metamath, description of}

Each provable assertion (\texttt{\$p}\index{\texttt{\$p} statement} statement) in a
database must include a {\bf proof}\index{proof}.  The proof is located
between the \texttt{\$=}\index{\texttt{\$=} keyword} and \texttt{\$.}\ keywords in the
\texttt{\$p} statement.

In the basic Metamath language\index{basic language}, a proof is a
sequence of statement labels.  This label sequence\index{label sequence}
serves as a set of instructions that the Metamath program uses to
construct a series of math symbol sequences.  The construction must
ultimately result in the math symbol sequence contained between the
\texttt{\$p}\index{\texttt{\$p} statement} and
\texttt{\$=}\index{\texttt{\$=} keyword} keywords of the \texttt{\$p}
statement.  Otherwise, the Metamath program will consider the proof
incorrect, and it will notify you with an appropriate error message when
you ask it to verify the proof.\footnote{To make the loading faster, the
Metamath program does not automatically verify proofs when you
\texttt{read} in a database unless you use the \texttt{/verify}
qualifier.  After a database has been read in, you may use the
\texttt{verify proof *} command to verify proofs.}\index{\texttt{verify
proof} command} Each label in a proof is said to {\bf
reference}\index{label reference} its corresponding statement.

Associated with any assertion\index{assertion} (\texttt{\$p} or
\texttt{\$a}\index{\texttt{\$a} statement} statement) is a set of
hypotheses (\texttt{\$f}\index{\texttt{\$f} statement} or
\texttt{\$e}\index{\texttt{\$e} statement} statements) that are active
with respect to that assertion.  Some are mandatory and the others are
optional.  You should review these concepts if necessary.

Each label\index{label} in a proof must be either the label of a
previous assertion (\texttt{\$a}\index{\texttt{\$a} statement} or
\texttt{\$p}\index{\texttt{\$p} statement} statement) or the label of an
active hypothesis (\texttt{\$e} or \texttt{\$f}\index{\texttt{\$f}
statement} statement) of the \texttt{\$p} statement containing the
proof.  Hypothesis labels may reference both the
mandatory\index{mandatory hypothesis} and the optional hypotheses of the
\texttt{\$p} statement.

The label sequence in a proof specifies a construction in {\bf reverse Polish
notation}\index{reverse Polish notation (RPN)} (RPN).  You may be familiar
with RPN if you have used older
Hewlett--Packard or similar hand-held calculators.
In the calculator analogy, a hypothesis label\index{hypothesis label} is like
a number and an assertion label\index{assertion label} is like an operation
(more precisely, an $n$-ary operation when the
assertion has $n$ \texttt{\$e}-hypotheses).
On an RPN calculator, an operation takes one or more previous numbers in an
input sequence, performs a calculation on them, and replaces those numbers and
itself with the result of the calculation.  For example, the input sequence
$2,3,+$ on an RPN calculator results in $5$, and the input sequence
$2,3,5,{\times},+$ results in $2,15,+$ which results in $17$.

Understanding how RPN is processed involves the concept of a {\bf
stack}\index{stack}\index{RPN stack}, which can be thought of as a set of
temporary memory locations that hold intermediate results.  When Metamath
encounters a hypothesis label it places or {\bf pushes}\index{push} the math
symbol sequence of the hypothesis onto the stack.  When Metamath encounters an
assertion label, it associates the most recent stack entries with the {\em
mandatory} hypotheses\index{mandatory hypothesis} of the assertion, in the
order where the most recent stack entry is associated with the last mandatory
hypothesis of the assertion.  It then determines what
substitutions\index{substitution!variable}\index{variable substitution} have
to be made into the variables of the assertion's mandatory hypotheses to make
them identical to the associated stack entries.  It then makes those same
substitutions into the assertion itself.  Finally, Metamath removes or {\bf
pops}\index{pop} the matched hypotheses from the stack and pushes the
substituted assertion onto the stack.

For the purpose of matching the mandatory hypothesis to the most recent stack
entries, whether a hypothesis is a \texttt{\$e} or \texttt{\$f} statement is
irrelevant.  The only important thing is that a set of
substitutions\footnote{In the Metamath spec (Section~\ref{spec}), we use the
singular term ``substitution'' to refer to the set of substitutions we talk
about here.} exist that allow a match (and if they don't, the proof verifier
will let you know with an error message).  The Metamath language is specified
in such a way that if a set of substitutions exists, it will be unique.
Specifically, the requirement that each variable have a type specified for it
with a \texttt{\$f} statement ensures the uniqueness.

We will illustrate this with an example.
Consider the following Metamath source file:
\begin{verbatim}
$c ( ) -> wff $.
$v p q r s $.
wp $f wff p $.
wq $f wff q $.
wr $f wff r $.
ws $f wff s $.
w2 $a wff ( p -> q ) $.
wnew $p wff ( s -> ( r -> p ) ) $= ws wr wp w2 w2 $.
\end{verbatim}
This Metamath source example shows the definition and ``proof'' (i.e.,
construction) of a well-formed formula (wff)\index{well-formed formula (wff)}
in propositional calculus.  (You may wish to type this example into a file to
experiment with the Metamath program.)  The first two statements declare
(introduce the names of) four constants and four variables.  The next four
statements specify the variable types, namely that
each variable is assumed to be a wff.  Statement \texttt{w2} defines (postulates)
a way to produce a new wff, \texttt{( p -> q )}, from two given wffs \texttt{p} and
\texttt{q}. The mandatory hypotheses of \texttt{w2} are \texttt{wp} and \texttt{wq}.
Statement \texttt{wnew} claims that \texttt{( s -> ( r -> p ) )} is a wff given
three wffs \texttt{s}, \texttt{r}, and \texttt{p}.  More precisely, \texttt{wnew} claims
that the sequence of ten symbols \texttt{wff ( s -> ( r -> p ) )} is provable from
previous assertions and the hypotheses of \texttt{wnew}.  Metamath does not know
or care what a wff is, and as far as it is concerned
the typecode \texttt{wff} is just an
arbitrary constant symbol in a math symbol sequence.  The mandatory hypotheses
of \texttt{wnew} are \texttt{wp}, \texttt{wr}, and \texttt{ws}; \texttt{wq} is an optional
hypothesis.  In our particular proof, the optional hypothesis is not
referenced, but in general, any combination of active (i.e.\ optional and
mandatory) hypotheses could be referenced.  The proof of statement \texttt{wnew}
is the sequence of five labels starting with \texttt{ws} (step~1) and ending with
\texttt{w2} (step~5).

When Metamath verifies the proof, it scans the proof from left to right.  We
will examine what happens at each step of the proof.  The stack starts off
empty.  At step 1, Metamath looks up label \texttt{ws} and determines that it is a
hypothesis, so it pushes the symbol sequence of statement \texttt{ws} onto the
stack:

\begin{center}\begin{tabular}{|l|l|}\hline
{Stack location} & {Contents} \\ \hline \hline
1 & \texttt{wff s} \\ \hline
\end{tabular}\end{center}

Metamath sees that the labels \texttt{wr} and \texttt{wp} in steps~2 and 3 are also
hypotheses, so it pushes them onto the stack.  After step~3, the stack looks
like
this:

\begin{center}\begin{tabular}{|l|l|}\hline
{Stack location} & {Contents} \\ \hline \hline
3 & \texttt{wff p} \\ \hline
2 & \texttt{wff r} \\ \hline
1 & \texttt{wff s} \\ \hline
\end{tabular}\end{center}

At step 4, Metamath sees that label \texttt{w2} is an assertion, so it must do
some processing.  First, it associates the mandatory hypotheses of \texttt{w2},
which are \texttt{wp} and \texttt{wq}, with stack locations~2 and 3, {\em in that
order}. Metamath determines that the only possible way
to make hypothesis \texttt{wp} match (become identical to) stack location~2 and
\texttt{wq} match stack location 3 is to substitute variable \texttt{p} with \texttt{r}
and \texttt{q} with \texttt{p}.  Metamath makes these substitutions into \texttt{w2} and
obtains the symbol sequence \texttt{wff ( r -> p )}.  It removes the hypotheses
from stack locations~2 and 3, then places the result into stack location~2:

\begin{center}\begin{tabular}{|l|l|}\hline
{Stack location} & {Contents} \\ \hline \hline
2 & \texttt{wff ( r -> p )} \\ \hline
1 & \texttt{wff s} \\ \hline
\end{tabular}\end{center}

At step 5, Metamath sees that label \texttt{w2} is an assertion, so it must again
do some processing.  First, it matches the mandatory hypotheses of \texttt{w2},
which are \texttt{wp} and \texttt{wq}, to stack locations 1 and 2.
Metamath determines that the only possible way to make the
hypotheses match is to substitute variable \texttt{p} with \texttt{s} and \texttt{q} with
\texttt{( r -> p )}.  Metamath makes these substitutions into \texttt{w2} and obtains
the symbol
sequence \texttt{wff ( s -> ( r -> p ) )}.  It removes stack
locations 1 and 2, then places the result into stack location~1:

\begin{center}\begin{tabular}{|l|l|}\hline
{Stack location} & {Contents} \\ \hline \hline
1 & \texttt{wff ( s -> ( r -> p ) )} \\ \hline
\end{tabular}\end{center}

After Metamath finishes processing the proof, it checks to see that the
stack contains exactly one element and that this element is
the same as the math symbol sequence in the
\texttt{\$p}\index{\texttt{\$p} statement} statement.  This is the case for our
proof of \texttt{wnew},
so we have proved \texttt{wnew} successfully.  If the result
differs, Metamath will notify you with an error message.  An error message
will also result if the stack contains more than one entry at the end of the
proof, or if the stack did not contain enough entries at any point in the
proof to match all of the mandatory hypotheses\index{mandatory hypothesis} of
an assertion.  Finally, Metamath will notify you with an error message if no
substitution is possible that will make a referenced assertion's hypothesis
match the
stack entries.  You may want to experiment with the different kinds of errors
that Metamath will detect by making some small changes in the proof of our
example.

Metamath's proof notation was designed primarily to express proofs in a
relatively compact manner, not for readability by humans.  Metamath can display
proofs in a number of different ways with the \texttt{show proof}\index{\texttt{show
proof} command} command.  The
\texttt{/lemmon} qualifier displays it in a format that is easier to read when the
proofs are short, and you saw examples of its use in Chapter~\ref{using}.  For
longer proofs, it is useful to see the tree structure of the proof.  A tree
structure is displayed when the \texttt{/lemmon} qualifier is omitted.  You will
probably find this display more convenient as you get used to it. The tree
display of the proof in our example looks like
this:\label{treeproof}\index{tree-style proof}\index{proof!tree-style}
\begin{verbatim}
1     wp=ws    $f wff s
2        wp=wr    $f wff r
3        wq=wp    $f wff p
4     wq=w2    $a wff ( r -> p )
5  wnew=w2  $a wff ( s -> ( r -> p ) )
\end{verbatim}
The number to the left of each line is the step number.  Following it is a
{\bf hypothesis association}\index{hypothesis association}, consisting of two
labels\index{label} separated by \texttt{=}.  To the left of the \texttt{=} (except
in the last step) is the label of a hypothesis of an assertion referenced
later in the proof; here, steps 1 and 4 are the hypothesis associations for
the assertion \texttt{w2} that is referenced in step 5.  A hypothesis association
is indented one level more than the assertion that uses it, so it is easy to
find the corresponding assertion by moving directly down until the indentation
level decreases to one less than where you started from.  To the right of each
\texttt{=} is the proof step label for that proof step.  The statement keyword of
the proof step label is listed next, followed by the content of the top of the
stack (the most recent stack entry) as it exists after that proof step is
processed.  With a little practice, you should have no trouble reading proofs
displayed in this format.

Metamath proofs include the syntax construction of a formula.
In standard mathematics, this kind of
construction is not considered a proper part of the proof at all, and it
certainly becomes rather boring after a while.
Therefore,
by default the \texttt{show proof}\index{\texttt{show proof}
command} command does not show the syntax construction.
Historically \texttt{show proof} command
\textit{did} show the syntax construction, and you needed to add the
\texttt{/essential} option to hide, them, but today
\texttt{/essential} is the default and you need to use
\texttt{/all} to see the syntax constructions.

When verifying a proof, Metamath will check that no mandatory
\texttt{\$d}\index{\texttt{\$d} statement}\index{mandatory \texttt{\$d}
statement} statement of an assertion referenced in a proof is violated
when substitutions\index{substitution!variable}\index{variable
substitution} are made to the variables in the assertion.  For details
see Section~\ref{spec4} or \ref{dollard}.

\subsection{The Concept of Unification} \label{unify}

During the course of verifying a proof, when Metamath\index{Metamath}
encounters an assertion label\index{assertion label}, it associates the
mandatory hypotheses\index{mandatory hypothesis} of the assertion with the top
entries of the RPN stack\index{stack}\index{RPN stack}.  Metamath then
determines what substitutions\index{substitution!variable}\index{variable
substitution} it must make to the variables in the assertion's mandatory
hypotheses in order for these hypotheses to become identical to their
corresponding stack entries.  This process is called {\bf
unification}\index{unification}.  (We also informally use the term
``unification'' to refer to a set of substitutions that results from the
process, as in ``two unifications are possible.'')  After the substitutions
are made, the hypotheses are said to be {\bf unified}.

If no such substitutions are possible, Metamath will consider the proof
incorrect and notify you with an error message.
% (deleted 3/10/07, per suggestion of Mel O'Cat:)
% The syntax of the
% Metamath language ensures that if a set of substitutions exists, it
% will be unique.

The general algorithm for unification described in the literature is
somewhat complex.
However, in the case of Metamath it is intentionally trivial.
Mandatory hypotheses must be
pushed on the proof stack in the order in which they appear.
In addition, each variable must have its type specified
with a \texttt{\$f} hypothesis before it is used
and that each \texttt{\$f} hypothesis
have the restricted syntax of a typecode (a constant) followed by a variable.
The typecode in the \texttt{\$f} hypothesis must match the first symbol of
the corresponding RPN stack entry (which will also be a constant), so
the only possible match for the variable in the \texttt{\$f} hypothesis is
the sequence of symbols in the stack entry after the initial constant.

In the Proof Assistant\index{Proof Assistant}, a more general unification
algorithm is used.  While a proof is being developed, sometimes not enough
information is available to determine a unique unification.  In this case
Metamath will ask you to pick the correct one.\index{ambiguous
unification}\index{unification!ambiguous}

\section{Extensions to the Metamath Language}\index{extended
language}

\subsection{Comments in the Metamath Language}\label{comments}
\index{markup notation}
\index{comments!markup notation}

The commenting feature allows you to annotate the contents of
a database.  Just as with most
computer languages, comments are ignored for the purpose of interpreting the
contents of the database. Comments effectively act as
additional white space\index{white
space} between tokens
when a database is parsed.

A comment may be placed at the beginning, end, or
between any two tokens\index{token} in a source file.

Comments have the following syntax:
\begin{center}
 \texttt{\$(} {\em text} \texttt{\$)}
\end{center}
Here,\index{\texttt{\$(} and \texttt{\$)} auxiliary
keywords}\index{comment} {\em text} is a string, possibly empty, of any
characters in Metamath's character set (p.~\pageref{spec1chars}), except
that the character strings \texttt{\$(} and \texttt{\$)} may not appear
in {\em text}.  Thus nested comments are not
permitted:\footnote{Computer languages have differing standards for
nested comments, and rather than picking one it was felt simplest not to
allow them at all, at least in the current version (0.177) of
Metamath\index{Metamath!limitations of version 0.177}.} Metamath will
complain if you give it
\begin{center}
 \texttt{\$( This is a \$( nested \$) comment.\ \$)}
\end{center}
To compensate for this non-nesting behavior, I often change all \texttt{\$}'s
to \texttt{@}'s in sections of Metamath code I wish to comment out.

The Metamath program supports a number of markup mechanisms and conventions
to generate good-looking results in \LaTeX\ and {\sc html},
as discussed below.
These markup features have to do only with how the comments are typeset,
and have no effect on how Metamath verifies the proofs in the database.
The improper
use of them may result in incorrectly typeset output, but no Metamath
error messages will result during the \texttt{read} and \texttt{verify
proof} commands.  (However, the \texttt{write
theorem\texttt{\char`\_}list} command
will check for markup errors as a side-effect of its
{\sc html} generation.)
Section~\ref{texout} has instructions for creating \LaTeX\ output, and
section~\ref{htmlout} has instructions for creating
{\sc html}\index{HTML} output.

\subsubsection{Headings}\label{commentheadings}

If the \texttt{\$(} is immediately followed by a new line
starting with a heading marker, it is a header.
This can start with:

\begin{itemize}
 \item[] \texttt{\#\#\#\#} - major part header
 \item[] \texttt{\#*\#*} - section header
 \item[] \texttt{=-=-} - subsection header
 \item[] \texttt{-.-.} - subsubsection header
\end{itemize}

The line following the marker line
will be used for the table of contents entry, after trimming spaces.
The next line should be another (closing) matching marker line.
Any text after that
but before the closing \texttt{\$}, such as an extended description of the
section, will be included on the \texttt{mmtheoremsNNN.html} page.

For more information, run
\texttt{help write theorem\char`\_list}.

\subsubsection{Math mode}
\label{mathcomments}
\index{\texttt{`} inside comments}
\index{\texttt{\char`\~} inside comments}
\index{math mode}

Inside of comments, a string of tokens\index{token} enclosed in
grave accents\index{grave accent (\texttt{`})} (\texttt{`}) will be converted
to standard mathematical symbols during
{\sc HTML}\index{HTML} or \LaTeX\ output
typesetting,\index{latex@{\LaTeX}} according to the information in the
special \texttt{\$t}\index{\texttt{\$t} comment}\index{typesetting
comment} comment in the database
(see section~\ref{tcomment} for information about the typesetting
comment, and Appendix~\ref{ASCII} to see examples of its results).

The first grave accent\index{grave accent (\texttt{`})} \texttt{`}
causes the output processor to enter {\bf math mode}\index{math mode}
and the second one exits it.
In this
mode, the characters following the \texttt{`} are interpreted as a
sequence of math symbol tokens separated by white space\index{white
space}.  The tokens are looked up in the \texttt{\$t}
comment\index{\texttt{\$t} comment}\index{typesetting comment} and if
found, they will be replaced by the standard mathematical symbols that
they correspond to before being placed in the typeset output file.  If
not found, the symbol will be output as is and a warning will be issued.
The tokens do not have to be active in the database, although a warning
will be issued if they are not declared with \texttt{\$c} or
\texttt{\$v} statements.

Two consecutive
grave accents \texttt{``} are treated as a single actual grave accent
(both inside and outside of math mode) and will not cause the output
processor to enter or exit math mode.

Here is an example of its use\index{Pierce's axiom}:
\begin{center}
\texttt{\$( Pierce's axiom, ` ( ( ph -> ps ) -> ph ) -> ph ` ,\\
         is not very intuitive. \$)}
\end{center}
becomes
\begin{center}
   \texttt{\$(} Pierce's axiom, $((\varphi \rightarrow \psi)\rightarrow
\varphi)\rightarrow \varphi$, is not very intuitive. \texttt{\$)}
\end{center}

Note that the math symbol tokens\index{token} must be surrounded by white
space\index{white space}.
%, since there is no context that allows ambiguity to be
%resolved, as is the case with math symbol sequences in some of the Metamath
%statements.
White space should also surround the \texttt{`}
delimiters.

The math mode feature also gives you a quick and easy way to generate
text containing mathematical symbols, independently of the intended
purpose of Metamath.\index{Metamath!using as a math editor} To do this,
simply create your text with grave accents surrounding your formulas,
after making sure that your math symbols are mapped to \LaTeX\ symbols
as described in Appendix~\ref{ASCII}.  It is easier if you start with a
database with predefined symbols such as \texttt{set.mm}.  Use your
grave-quoted math string to replace an existing comment, then typeset
the statement corresponding to that comment following the instructions
from the \texttt{help tex} command in the Metamath program.  You will
then probably want to edit the resulting file with a text editor to fine
tune it to your exact needs.

\subsubsection{Label Mode}\index{label mode}

Outside of math mode, a tilde\index{tilde (\texttt{\char`\~})} \verb/~/
indicates to Metamath's\index{Metamath} output processor that the
token\index{token} that follows (i.e.\ the characters up to the next
white space\index{white space}) represents a statement label or URL.
This formatting mode is called {\bf label mode}\index{label mode}.
If a literal tilde
is desired (outside of math mode) instead of label mode,
use two tildes in a row to represent it.

When generating a \LaTeX\ output file,
the following token will be formatted in \texttt{typewriter}
font, and the tilde removed, to make it stand out from the rest of the text.
This formatting will be applied to all characters after the
tilde up to the first white space\index{white space}.
Whether
or not the token is an actual statement label is not checked, and the
token does not have to have the correct syntax for a label; no error
messages will be produced.  The only effect of the label mode on the
output is that typewriter font will be used for the tokens that are
placed in the \LaTeX\ output file.

When generating {\sc html},
the tokens after the tilde {\em must} be a URL (either http: or https:)
or a valid label.
Error messages will be issued during that output if they aren't.
A hyperlink will be generated to that URL or label.

\subsubsection{Link to bibliographical reference}\index{citation}%
\index{link to bibliographical reference}

Bibliographical references are handled specially when generating
{\sc html} if formatted specially.
Text in the form \texttt{[}{\em author}\texttt{]}
is considered a link to a bibliographical reference.
See \texttt{help html} and \texttt{help write
bibliography} in the Metamath program for more
information.
% \index{\texttt{\char`\[}\ldots\texttt{]} inside comments}
See also Sections~\ref{tcomment} and \ref{wrbib}.

The \texttt{[}{\em author}\texttt{]} notation will also create an entry in
the bibliography cross-reference file generated by \texttt{write
bibliography} (Section~\ref{wrbib}) for {\sc HTML}.
For this to work properly, the
surrounding comment must be formatted as follows:
\begin{quote}
    {\em keyword} {\em label} {\em noise-word}
     \texttt{[}{\em author}\texttt{] p.} {\em number}
\end{quote}
for example
\begin{verbatim}
     Theorem 5.2 of [Monk] p. 223
\end{verbatim}
The {\em keyword} is not case sensitive and must be one of the following:
\begin{verbatim}
     theorem lemma definition compare proposition corollary
     axiom rule remark exercise problem notation example
     property figure postulate equation scheme chapter
\end{verbatim}
The optional {\em label} may consist of more than one
(non-{\em keyword} and non-{\em noise-word}) word.
The optional {\em noise-word} is one of:
\begin{verbatim}
     of in from on
\end{verbatim}
and is  ignored when the cross-reference file is created.  The
\texttt{write
biblio\-graphy} command will perform error checking to verify the
above format.\index{error checking}

\subsubsection{Parentheticals}\label{parentheticals}

The end of a comment may include one or more parenthicals, that is,
statements enclosed in parentheses.
The Metamath program looks for certain parentheticals and can issue
warnings based on them.
They are:

\begin{itemize}
 \item[] \texttt{(Contributed by }
   \textit{NAME}\texttt{,} \textit{DATE}\texttt{.)} -
   document the original contributor's name and the date it was created.
 \item[] \texttt{(Revised by }
   \textit{NAME}\texttt{,} \textit{DATE}\texttt{.)} -
   document the contributor's name and creation date
   that resulted in significant revision
   (not just an automated minimization or shortening).
 \item[] \texttt{(Proof shortened by }
   \textit{NAME}\texttt{,} \textit{DATE}\texttt{.)} -
   document the contributor's name and date that developed a significant
   shortening of the proof (not just an automated minimization).
 \item[] \texttt{(Proof modification is discouraged.)} -
   Note that this proof should normally not be modified.
 \item[] \texttt{(New usage is discouraged.)} -
   Note that this assertion should normally not be used.
\end{itemize}

The \textit{DATE} must be in form YYYY-MMM-DD, where MMM is the
English abbreviation of that month.

\subsubsection{Other markup}\label{othermarkup}
\index{markup notation}

There are other markup notations for generating good-looking results
beyond math mode and label mode:

\begin{itemize}
 \item[]
         \texttt{\char`\_} (underscore)\index{\texttt{\char`\_} inside comments} -
             Italicize text starting from
              {\em space}\texttt{\char`\_}{\em non-space} (i.e.\ \texttt{\char`\_}
              with a space before it and a non-space character after it) until
             the next
             {\em non-space}\texttt{\char`\_}{\em space}.  Normal
             punctuation (e.g.\ a trailing
             comma or period) is ignored when determining {\em space}.
 \item[]
         \texttt{\char`\_} (underscore) - {\em
         non-space}\texttt{\char`\_}{\em non-space-string}, where
          {\em non-space-string} is a string of non-space characters,
         will make {\em non-space-string} become a subscript.
 \item[]
         \texttt{<HTML>}...\texttt{</HTML>} - do not convert
         ``\texttt{<}'' and ``\texttt{>}''
         in the enclosed text when generating {\sc HTML},
         otherwise process markup normally. This allows direct insertion
         of {\sc html} commands.
 \item[]
       ``\texttt{\&}ref\texttt{;}'' - insert an {\sc HTML}
         character reference.
         This is how to insert arbitrary Unicode characters
         (such as accented characters).  Currently only directly supported
         when generating {\sc HTML}.
\end{itemize}

It is recommended that spaces surround any \texttt{\char`\~} and
\texttt{`} tokens in the comment and that a space follow the {\em label}
after a \texttt{\char`\~} token.  This will make global substitutions
to change labels and symbol names much easier and also eliminate any
future chance of ambiguity.  Spaces around these tokens are automatically
removed in the final output to conform with normal rules of punctuation;
for example, a space between a trailing \texttt{`} and a left parenthesis
will be removed.

A good way to become familiar with the markup notation is to look at
the extensive examples in the \texttt{set.mm} database.

\subsection{The Typesetting Comment (\texttt{\$t})}\label{tcomment}

The typesetting comment \texttt{\$t} in the input database file
provides the information necessary to produce good-looking results.
It provides \LaTeX\ and {\sc html}
definitions for math symbols,
as well supporting as some
customization of the generated web page.
If you add a new token to a database, you should also
update the \texttt{\$t} comment information if you want to eventually
create output in \LaTeX\ or {\sc HTML}.
See the
\texttt{set.mm}\index{set theory database (\texttt{set.mm})} database
file for an extensive example of a \texttt{\$t} comment illustrating
many of the features described below.

Programs that do not need to generate good-looking presentation results,
such as programs that only verify Metamath databases,
can completely ignore typesetting comments
and just treat them as normal comments.
Even the Metamath program only consults the
\texttt{\$t} comment information when it needs to generate typeset output
in \LaTeX\ or {\sc HTML}
(e.g., when you open a \LaTeX\ output file with the \texttt{open tex} command).

We will first discuss the syntax of typesetting comments, and then
briefly discuss how this can be used within the Metamath program.

\subsubsection{Typesetting Comment Syntax Overview}

The typesetting comment is identified by the token
\texttt{\$t}\index{\texttt{\$t} comment}\index{typesetting comment} in
the comment, and the typesetting comment ends at the matching
\texttt{\$)}:
\[
  \mbox{\tt \$(\ }
  \mbox{\tt \$t\ }
  \underbrace{
    \mbox{\tt \ \ \ \ \ \ \ \ \ \ \ }
    \cdots
    \mbox{\tt \ \ \ \ \ \ \ \ \ \ \ }
  }_{\mbox{Typesetting definitions go here}}
  \mbox{\tt \ \$)}
\]

There must be one or more white space characters, and only white space
characters, between the \texttt{\$(} that starts the comment
and the \texttt{\$t} symbol,
and the \texttt{\$t} must be followed by one
or more white space characters
(see section \ref{whitespace} for the definition of white space characters).
The typesetting comment continues until the comment end token \texttt{\$)}
(which must be preceded by one or more white space characters).

In version 0.177\index{Metamath!limitations of version 0.177} of the
Metamath program, there may be only one \texttt{\$t} comment in a
database.  This restriction may be lifted in the future to allow
many \texttt{\$t} comments in a database.

Between the \texttt{\$t} symbol (and its following white space) and the
comment end token \texttt{\$)} (and its preceding white space)
is a sequence of one or more typesetting definitions, where
each definition has the form
\textit{definition-type arg arg ... ;}.
Each of the zero or more \textit{arg} values
can be either a typesetting data or a keyword
(what keywords are allowed, and where, depends on the specific
\textit{definition-type}).
The \textit{definition-type}, and each argument \textit{arg},
are separated by one or more white space characters.
Every definition ends in an unquoted semicolon;
white space is not required before the terminating semicolon of a definition.
Each definition should start on a new line.\footnote{This
restriction of the current version of Metamath
(0.177)\index{Metamath!limitations of version 0.177} may be removed
in a future version, but you should do it anyway for readability.}

For example, this typesetting definition:
\begin{center}
 \verb$latexdef "C_" as "\subseteq";$
\end{center}
defines the token \verb$C_$ as the \LaTeX\ symbol $\subseteq$ (which means
``subset'').

Typesetting data is a sequence of one or more quoted strings
(if there is more than one, they are connected by \texttt{\char`\+}).
Often a single quoted string is used to provide data for a definition, using
either double (\texttt{\char`\"}) or single (\texttt{'}) quotation marks.
However,
{\em a quoted string (enclosed in quotation marks) may not include
line breaks.}
A quoted string
may include a quotation mark that matches the enclosing quotes by repeating
the quotation mark twice.  Here are some examples:

\begin{tabu}   { l l }
\textbf{Example} & \textbf{Meaning} \\
\texttt{\char`\"a\char`\"\char`\"b\char`\"} & \texttt{a\char`\"b} \\
\texttt{'c''d'} & \texttt{c'd} \\
\texttt{\char`\"e''f\char`\"} & \texttt{e''f} \\
\texttt{'g\char`\"\char`\"h'} & \texttt{g\char`\"\char`\"h} \\
\end{tabu}

Finally, a long quoted string
may be broken up into multiple quoted strings (considered, as a whole,
a single quoted string) and joined with \texttt{\char`\+}.
You can even use multiple lines as long as a
'+' is at the end of every line except the last one.
The \texttt{\char`\+} should be preceded and followed by at least one
white space character.
Thus, for example,
\begin{center}
 \texttt{\char`\"ab\char`\"\ \char`\+\ \char`\"cd\char`\"
    \ \char`\+\ \\ 'ef'}
\end{center}
is the same as
\begin{center}
 \texttt{\char`\"abcdef\char`\"}
\end{center}

{\sc c}-style comments \texttt{/*}\ldots\texttt{*/} are also supported.

In practice, whenever you add a new math token you will often want to add
typesetting definitions using
\texttt{latexdef}, \texttt{htmldef}, and
\texttt{althtmldef}, as described below.
That way, they will all be up to date.
Of course, whether or not you want to use all three definitions will
depend on how the database is intended to be used.

Below we discuss the different possible \textit{definition-kind} options.
We will show data surrounded by double quotes (in practice they can also use
single quotes and/or be a sequence joined by \texttt{+}s).
We will use specific names for the \textit{data} to make clear what
the data is used for, such as
{\em math-token} (for a Metamath math token,
{\em latex-string} (for string to be placed in a \LaTeX\ stream),
{\em {\sc html}-code} (for {\sc html} code),
and {\em filename} (for a filename).

\subsubsection{Typesetting Comment - \LaTeX}

The syntax for a \LaTeX\ definition is:
\begin{center}
 \texttt{latexdef "}{\em math-token}\texttt{" as "}{\em latex-string}\texttt{";}
\end{center}
\index{latex definitions@\LaTeX\ definitions}%
\index{\texttt{latexdef} statement}

The {\em token-string} and {\em latex-string} are the data
(character strings) for
the token and the \LaTeX\ definition of the token, respectively,

These \LaTeX\ definitions are used by the Metamath program
when it is asked to product \LaTeX output using
the \texttt{write tex} command.

\subsubsection{Typesetting Comment - {\sc html}}

The key kinds of {\sc HTML} definitions have the following syntax:

\vskip 1ex
    \texttt{htmldef "}{\em math-token}\texttt{" as "}{\em
    {\sc html}-code}\texttt{";}\index{\texttt{htmldef} statement}
                    \ \ \ \ \ \ldots

    \texttt{althtmldef "}{\em math-token}\texttt{" as "}{\em
{\sc html}-code}\texttt{";}\index{\texttt{althtmldef} statement}

                    \ \ \ \ \ \ldots

Note that in {\sc HTML} there are two possible definitions for math tokens.
This feature is useful when
an alternate representation of symbols is desired, for example one that
uses Unicode entities and another uses {\sc gif} images.

There are many other typesetting definitions that can control {\sc HTML}.
These include:

\vskip 1ex

    \texttt{htmldef "}{\em math-token}\texttt{" as "}{\em {\sc
    html}-code}\texttt{";}

    \texttt{htmltitle "}{\em {\sc html}-code}\texttt{";}%
\index{\texttt{htmltitle} statement}

    \texttt{htmlhome "}{\em {\sc html}-code}\texttt{";}%
\index{\texttt{htmlhome} statement}

    \texttt{htmlvarcolor "}{\em {\sc html}-code}\texttt{";}%
\index{\texttt{htmlvarcolor} statement}

    \texttt{htmlbibliography "}{\em filename}\texttt{";}%
\index{\texttt{htmlbibliography} statement}

\vskip 1ex

\noindent The \texttt{htmltitle} is the {\sc html} code for a common
title, such as ``Metamath Proof Explorer.''  The \texttt{htmlhome} is
code for a link back to the home page.  The \texttt{htmlvarcolor} is
code for a color key that appears at the bottom of each proof.  The file
specified by {\em filename} is an {\sc html} file that is assumed to
have a \texttt{<A NAME=}\ldots\texttt{>} tag for each bibiographic
reference in the database comments.  For example, if
\texttt{[Monk]}\index{\texttt{\char`\[}\ldots\texttt{]} inside comments}
occurs in the comment for a theorem, then \texttt{<A NAME='Monk'>} must
be present in the file; if not, a warning message is given.

Associated with
\texttt{althtmldef}
are the statements
\vskip 1ex

    \texttt{htmldir "}{\em
      directoryname}\texttt{";}\index{\texttt{htmldir} statement}

    \texttt{althtmldir "}{\em
     directoryname}\texttt{";}\index{\texttt{althtmldir} statement}

\vskip 1ex
\noindent giving the directories of the {\sc gif} and Unicode versions
respectively; their purpose is to provide cross-linking between the
two versions in the generated web pages.

When two different types of pages need to be produced from a single
database, such as the Hilbert Space Explorer that extends the Metamath
Proof Explorer, ``extended'' variables may be declared in the
\texttt{\$t} comment:
\vskip 1ex

    \texttt{exthtmltitle "}{\em {\sc html}-code}\texttt{";}%
\index{\texttt{exthtmltitle} statement}

    \texttt{exthtmlhome "}{\em {\sc html}-code}\texttt{";}%
\index{\texttt{exthtmlhome} statement}

    \texttt{exthtmlbibliography "}{\em filename}\texttt{";}%
\index{\texttt{exthtmlbibliography} statement}

\vskip 1ex
\noindent When these are declared, you also must declare
\vskip 1ex

    \texttt{exthtmllabel "}{\em label}\texttt{";}%
\index{\texttt{exthtmllabel} statement}

\vskip 1ex \noindent that identifies the database statement where the
``extended'' section of the database starts (in our example, where the
Hilbert Space Explorer starts).  During the generation of web pages for
that starting statement and the statements after it, the {\sc html} code
assigned to \texttt{exthtmltitle} and \texttt{exthtmlhome} is used
instead of that assigned to \texttt{htmltitle} and \texttt{htmlhome},
respectively.

\begin{sloppy}
\subsection{Additional Information Com\-ment (\texttt{\$j})} \label{jcomment}
\end{sloppy}

The additional information comment, aka the
\texttt{\$j}\index{\texttt{\$j} comment}\index{additional information comment}
comment,
provides a way to add additional structured information that can
be optionally parsed by systems.

The additional information comment is parsed the same way as the
typesetting comment (\texttt{\$t}) (see section \ref{tcomment}).
That is,
the additional information comment begins with the token
\texttt{\$j} within a comment,
and continues until the comment close \texttt{\$)}.
Within an additional information comment is a sequence of one or more
commands of the form \texttt{command arg arg ... ;}
where each of the zero or more \texttt{arg} values
can be either a quoted string or a keyword.
Note that every command ends in an unquoted semicolon.
If a verifier is parsing an additional information comment, but
doesn't recognize a particular command, it must skip the command
by finding the end of the command (an unquoted semicolon).

A database may have 0 or more additional information comments.
Note, however, that a verifier may ignore these comments entirely or only
process certain commands in an additional information comment.
The \texttt{mmj2} verifier supports many commands in additional information
comments.
We encourage systems that process additional information comments
to coordinate so that they will use the same command for the same effect.

Examples of additional information comments with various commands
(from the \texttt{set.mm} database) are:

\begin{itemize}
   \item Define the syntax and logical typecodes,
     and declare that our grammar is
     unambiguous (verifiable using the KLR parser, with compositing depth 5).
\begin{verbatim}
  $( $j
    syntax 'wff';
    syntax '|-' as 'wff';
    unambiguous 'klr 5';
  $)
\end{verbatim}

   \item Register $\lnot$ and $\rightarrow$ as primitive expressions
           (lacking definitions).
\begin{verbatim}
  $( $j primitive 'wn' 'wi'; $)
\end{verbatim}

   \item There is a special justification for \texttt{df-bi}.
\begin{verbatim}
  $( $j justification 'bijust' for 'df-bi'; $)
\end{verbatim}

   \item Register $\leftrightarrow$ as an equality for its type (wff).
\begin{verbatim}
  $( $j
    equality 'wb' from 'biid' 'bicomi' 'bitri';
    definition 'dfbi1' for 'wb';
  $)
\end{verbatim}

   \item Theorem \texttt{notbii} is the congruence law for negation.
\begin{verbatim}
  $( $j congruence 'notbii'; $)
\end{verbatim}

   \item Add \texttt{setvar} as a typecode.
\begin{verbatim}
  $( $j syntax 'setvar'; $)
\end{verbatim}

   \item Register $=$ as an equality for its type (\texttt{class}).
\begin{verbatim}
  $( $j equality 'wceq' from 'eqid' 'eqcomi' 'eqtri'; $)
\end{verbatim}

\end{itemize}


\subsection{Including Other Files in a Metamath Source File} \label{include}
\index{\texttt{\$[} and \texttt{\$]} auxiliary keywords}

The keywords \texttt{\$[} and \texttt{\$]} specify a file to be
included\index{included file}\index{file inclusion} at that point in a
Metamath\index{Metamath} source file\index{source file}.  The syntax for
including a file is as follows:
\begin{center}
\texttt{\$[} {\em file-name} \texttt{\$]}
\end{center}

The {\em file-name} should be a single token\index{token} with the same syntax
as a math symbol (i.e., all 93 non-whitespace
printable characters other than \texttt{\$} are
allowed, subject to the file-naming limitations of your operating system).
Comments may appear between the \texttt{\$[} and \texttt{\$]} keywords.  Included
files may include other files, which may in turn include other files, and so
on.

For example, suppose you want to use the set theory database as the starting
point for your own theory.  The first line in your file could be
\begin{center}
\texttt{\$[ set.mm \$]}
\end{center} All of the information (axioms, theorems,
etc.) in \texttt{set.mm} and any files that {\em it} includes will become
available for you to reference in your file. This can help make your work more
modular. A drawback to including files is that if you change the name of a
symbol or the label of a statement, you must also remember to update any
references in any file that includes it.


The naming conventions for included files are the same as those of your
operating system.\footnote{On the Macintosh, prior to Mac OS X,
 a colon is used to separate disk
and folder names from your file name.  For example, {\em volume}\texttt{:}{\em
file-name} refers to the root directory, {\em volume}\texttt{:}{\em
folder-name}\texttt{:}{\em file-name} refers to a folder in root, and {\em
volume}\texttt{:}{\em folder-name}\texttt{:}\ldots\texttt{:}{\em file-name} refers to a
deeper folder.  A simple {\em file-name} refers to a file in the folder from
which you launch the Metamath application.  Under Mac OS X and later,
the Metamath program is run under the Terminal application, which
conforms to Unix naming conventions.}\index{Macintosh file
names}\index{file names!Macintosh}\label{includef} For compatibility among
operating systems, you should keep the file names as simple as possible.  A
good convention to use is {\em file}\texttt{.mm} where {\em file} is eight
characters or less, in lower case.

There is no limit to the nesting depth of included files.  One thing that you
should be aware of is that if two included files themselves include a common
third file, only the {\em first} reference to this common file will be read
in.  This allows you to include two or more files that build on a common
starting file without having to worry about label and symbol conflicts that
would occur if the common file were read in more than once.  (In fact, if a
file includes itself, the self-reference will be ignored, although of course
it would not make any sense to do that.)  This feature also means, however,
that if you try to include a common file in several inner blocks, the result
might not be what you expect, since only the first reference will be replaced
with the included file (unlike the include statement in most other computer
languages).  Thus you would normally include common files only in the
outermost block\index{outermost block}.

\subsection{Compressed Proof Format}\label{compressed1}\index{compressed
proof}\index{proof!compressed}

The proof notation presented in Section~\ref{proof} is called a
{\bf normal proof}\index{normal proof}\index{proof!normal} and in principle is
sufficient to express any proof.  However, proofs often contain steps and
subproofs that are identical.  This is particularly true in typical
Metamath\index{Metamath} applications, because Metamath requires that the math
symbol sequence (usually containing a formula) at each step be separately
constructed, that is, built up piece by piece. As a result, a lot of
repetition often results.  The {\bf compressed proof} format allows Metamath
to take advantage of this redundancy to shorten proofs.

The specification for the compressed proof format is given in
Appen\-dix~\ref{compressed}.

Normally you need not concern yourself with the details of the compressed
proof format, since the Metamath program will allow you to convert from
the normal format to the compressed format with ease, and will also
automatically convert from the compressed format when proofs are displayed.
The overall structure of the compressed format is as follows:
\begin{center}
  \texttt{\$= ( } {\em label-list} \texttt{) } {\em compressed-proof\ }\ \texttt{\$.}
\end{center}
\index{\texttt{\$=} keyword}
The first \texttt{(} serves as a flag to Metamath that a compressed proof
follows.  The {\em label-list} includes all statements referred to by the
proof except the mandatory hypotheses\index{mandatory hypothesis}.  The {\em
compressed-proof} is a compact encoding of the proof, using upper-case
letters, and can be thought of as a large integer in base 26.  White
space\index{white space} inside a {\em compressed-proof} is
optional and is ignored.

It is important to note that the order of the mandatory hypotheses of
the statement being proved must not be changed if the compressed proof
format is used, otherwise the proof will become incorrect.  The reason
for this is that the mandatory hypotheses are not mentioned explicitly
in the compressed proof in order to make the compression more efficient.
If you wish to change the order of mandatory hypotheses, you must first
convert the proof back to normal format using the \texttt{save proof
{\em statement} /normal}\index{\texttt{save proof} command} command.
Later, you can go back to compressed format with \texttt{save proof {\em
statement} /compressed}.

During error checking with the \texttt{verify proof} command, an error
found in a compressed proof may point to a character in {\em
compressed-proof}, which may not be very meaningful to you.  In this
case, try to \texttt{save proof /normal} first, then do the
\texttt{verify proof} again.  In general, it is best to make sure a
proof is correct before saving it in compressed format, because severe
errors are less likely to be recoverable than in normal format.

\subsection{Specifying Unknown Proofs or Subproofs}\label{unknown}

In a proof under development, any step or subproof that is not yet known
may be represented with a single \texttt{?}.  For the purposes of
parsing the proof, the \texttt{?}\ \index{\texttt{]}@\texttt{?}\ inside
proofs} will push a single entry onto the RPN stack just as if it were a
hypothesis.  While developing a proof with the Proof
Assistant\index{Proof Assistant}, a partially developed proof may be
saved with the \texttt{save new{\char`\_}proof}\index{\texttt{save
new{\char`\_}proof} command} command, and \texttt{?}'s will be placed at
the appropriate places.

All \texttt{\$p}\index{\texttt{\$p} statement} statements must have
proofs, even if they are entirely unknown.  Before creating a proof with
the Proof Assistant, you should specify a completely unknown proof as
follows:
\begin{center}
  {\em label} \texttt{\$p} {\em statement} \texttt{\$= ?\ \$.}
\end{center}
\index{\texttt{\$=} keyword}
\index{\texttt{]}@\texttt{?}\ inside proofs}

The \texttt{verify proof}\index{\texttt{verify proof} command} command
will check the known portions of a partial proof for errors, but will
warn you that the statement has not been proved.

Note that partially developed proofs may be saved in compressed format
if desired.  In this case, you will see one or more \texttt{?}'s in the
{\em compressed-proof} part.\index{compressed
proof}\index{proof!compressed}

\section{Axioms vs.\ Definitions}\label{definitions}

The \textit{basic}
Metamath\index{Metamath} language and program
make no distinction\index{axiom vs.\
definition} between axioms\index{axiom} and
definitions.\index{definition} The \texttt{\$a}\index{\texttt{\$a}
statement} statement is used for both.  At first, this may seem
puzzling.  In the minds of many mathematicians, the distinction is
clear, even obvious, and hardly worth discussing.  A definition is
considered to be merely an abbreviation that can be replaced by the
expression for which it stands; although unless one actually does this,
to be precise then one should say that a theorem\index{theorem} is a
consequence of the axioms {\em and} the definitions that are used in the
formulation of the theorem \cite[p.~20]{Behnke}.\index{Behnke, H.}

\subsection{What is a Definition?}

What is a definition?  In its simplest form, a definition introduces a new
symbol and provides an unambiguous rule to transform an expression containing
the new symbol to one without it.  The concept of a ``proper
definition''\index{proper definition}\index{definition!proper} (as opposed to
a creative definition)\index{creative definition}\index{definition!creative}
that is usually agreed upon is (1) the definition should not strengthen the
language and (2) any symbols introduced by the definition should be eliminable
from the language \cite{Nemesszeghy}\index{Nemesszeghy, E. Z.}.  In other
words, they are mere typographical conveniences that do not belong to the
system and are theoretically superfluous.  This may seem obvious, but in fact
the nature of definitions can be subtle, sometimes requiring difficult
metatheorems to establish that they are not creative.

A more conservative stance was taken by logician S.
Le\'{s}niewski.\index{Le\'{s}niewski, S.}
\begin{quote}
Le\'{s}niewski
regards definitions as theses of the system.  In this respect they do
not differ either from the axioms or from theorems, i.e.\ from the
theses added to the system on the basis of the rule of substitution or
the rule of detachment [modus ponens].  Once definitions have been
accepted as theses of the system, it becomes necessary to consider them
as true propositions in the same sense in which axioms are true
\cite{Lejewski}.
\end{quote}\index{Lejewski, Czeslaw}

Let us look at some simple examples of definitions in propositional
calculus.  Consider the definition of logical {\sc or}
(disjunction):\index{disjunction ($\vee$)} ``$P\vee Q$ denotes $\neg P
\rightarrow Q$ (not $P$ implies $Q$).''  It is very easy to recognize a
statement making use of this definition, because it introduces the new
symbol $\vee$ that did not previously exist in the language.  It is easy
to see that no new theorems of the original language will result from
this definition.

Next, consider a definition that eliminates parentheses:  ``$P
\rightarrow Q\rightarrow R$ denotes $P\rightarrow (Q \rightarrow R)$.''
This is more subtle, because no new symbols are introduced.  The reason
this definition is considered proper is that no new symbol sequences
that are valid wffs (well-formed formulas)\index{well-formed formula
(wff)} in the original language will result from the definition, since
``$P \rightarrow Q\rightarrow R$'' is not a wff in the original
language.  Here, we implicitly make use of the fact that there is a
decision procedure that allows us to determine whether or not a symbol
sequence is a wff, and this fact allows us to use symbol sequences that
are not wffs to represent other things (such as wffs) by means of the
definition.  However, to justify the definition as not being creative we
need to prove that ``$P \rightarrow Q\rightarrow R$'' is in fact not a
wff in the original language, and this is more difficult than in the
case where we simply introduce a new symbol.

%Now let's take this reasoning to an extreme.  Propositional calculus is a
%decidable theory,\footnote{This means that a mechanical algorithm exists to
%determine whether or not a wff is a theorem.} so in principle we could make use
%of symbol sequences that are not theorems to represent other things (say, to
%encode actual theorems in a more compact way).  For example, let us extend the
%language by defining a wff ``$P$'' in the extended language as the theorem
%``$P\rightarrow P$''\footnote{This is one of the first theorems proved in the
%Metamath database \texttt{set.mm}.}\index{set
%theory database (\texttt{set.mm})} in the original language whenever ``$P$'' is
%not a theorem in the original language.  In the extended language, any wff
%``$Q$'' thus represents a theorem; to find out what theorem (in the original
%language) ``$Q$'' represents, we determine whether ``$Q$'' is a theorem in the
%original language (before the definition was introduced).  If so, we're done; if
%not, we replace ``$Q$'' by ``$Q\rightarrow Q$'' to eliminate the definition.
%This definition is therefore eliminable, and it does not ``strengthen'' the
%language because any wff that is not a theorem is not in the set of statements
%provable in the original language and thus is available for use by definitions.
%
%Of course, a definition such as this would render practically useless the
%communication of theorems of propositional calculus; but
%this is just a human shortcoming, since we can't always easily discern what is
%and is not a theorem by inspection.  In fact, the extended theory with this
%definition has no more and no less information than the original theory; it just
%expresses certain theorems of the form ``$P\rightarrow P$''
%in a more compact way.
%
%The point here is that what constitutes a proper definition is a matter of
%judgment about whether a symbol sequence can easily be recognized by a human
%as invalid in some sense (for example, not a wff); if so, the symbol sequence
%can be appropriated for use by a definition in order to make the extended
%language more compact.  Metamath\index{Metamath} lacks the ability to make this
%judgment, since as far as Metamath is concerned the definition of a wff, for
%example, is arbitrary.  You define for Metamath how wffs\index{well-formed
%formula (wff)} are constructed according to your own preferred style.  The
%concept of a wff may not even exist in a given formal system\index{formal
%system}.  Metamath treats all definitions as if they were new axioms, and it
%is up to the human mathematician to judge whether the definition is ``proper''
%'\index{proper definition}\index{definition!proper} in some agreed-upon way.

What constitutes a definition\index{definition} versus\index{axiom vs.\
definition} an axiom\index{axiom} is sometimes arbitrary in mathematical
literature.  For example, the connectives $\vee$ ({\sc or}), $\wedge$
({\sc and}), and $\leftrightarrow$ (equivalent to) in propositional
calculus are usually considered defined symbols that can be used as
abbreviations for expressions containing the ``primitive'' connectives
$\rightarrow$ and $\neg$.  This is the way we treat them in the standard
logic and set theory database \texttt{set.mm}\index{set theory database
(\texttt{set.mm})}.  However, the first three connectives can also be
considered ``primitive,'' and axiom systems have been devised that treat
all of them as such.  For example,
\cite[p.~35]{Goodstein}\index{Goodstein, R. L.} presents one with 15
axioms, some of which in fact coincide with what we have chosen to call
definitions in \texttt{set.mm}.  In certain subsets of classical
propositional calculus, such as the intuitionist
fragment\index{intuitionism}, it can be shown that one cannot make do
with just $\rightarrow$ and $\neg$ but must treat additional connectives
as primitive in order for the system to make sense.\footnote{Two nice
systems that make the transition from intuitionistic and other weak
fragments to classical logic just by adding axioms are given in
\cite{Robinsont}\index{Robinson, T. Thacher}.}

\subsection{The Approach to Definitions in \texttt{set.mm}}

In set theory, recursive definitions define a newly introduced symbol in
terms of itself.
The justification of recursive definitions, using
several ``recursion theorems,'' is usually one of the first
sophisticated proofs a student encounters when learning set theory, and
there is a significant amount of implicit metalogic behind a recursive
definition even though the definition itself is typically simple to
state.

Metamath itself has no built-in technical limitation that prevents
multiple-part recursive definitions in the traditional textbook style.
However, because the recursive definition requires advanced metalogic
to justify, eliminating a recursive definition is very difficult and
often not even shown in textbooks.

\subsubsection{Direct definitions instead of recursive definitions}

It is, however, possible to substitute one kind of complexity
for another.  We can eliminate the need for metalogical justification by
defining the operation directly with an explicit (but complicated)
expression, then deriving the recursive definition directly as a
theorem, using a recursion theorem ``in reverse.''
The elimination
of a direct definition is a matter of simple mechanical substitution.
We do this in
\texttt{set.mm}, as follows.

In \texttt{set.mm} our goal was to introduce almost all definitions in
the form of two expressions connected by either $\leftrightarrow$ or
$=$, where the thing being defined does not appear on the right hand
side.  Quine calls this form ``a genuine or direct definition'' \cite[p.
174]{Quine}\index{Quine, Willard Van Orman}, which makes the definitions
very easy to eliminate and the metalogic\index{metalogic} needed to
justify them as simple as possible.
Put another way, we had a goal of being able to
eliminate all definitions with direct mechanical substitution and to
verify easily the soundness of the definitions.

\subsubsection{Example of direct definitions}

We achieved this goal in almost all cases in \texttt{set.mm}.
Sometimes this makes the definitions more complex and less
intuitive.
For example, the traditional way to define addition of
natural numbers is to define an operation called {\em
successor}\index{successor} (which means ``plus one'' and is denoted by
``${\rm suc}$''), then define addition recursively\index{recursive
definition} with the two definitions $n + 0 = n$ and $m + {\rm suc}\,n =
{\rm suc} (m + n)$.  Although this definition seems simple and obvious,
the method to eliminate the definition is not obvious:  in the second
part of the definition, addition is defined in terms of itself.  By
eliminating the definition, we don't mean repeatedly applying it to
specific $m$ and $n$ but rather showing the explicit, closed-form
set-theoretical expression that $m + n$ represents, that will work for
any $m$ and $n$ and that does not have a $+$ sign on its right-hand
side.  For a recursive definition like this not to be circular
(creative), there are some hidden, underlying assumptions we must make,
for example that the natural numbers have a certain kind of order.

In \texttt{set.mm} we chose to start with the direct (though complex and
nonintuitive) definition then derive from it the standard recursive
definition.
For example, the closed-form definition used in \texttt{set.mm}
for the addition operation on ordinals\index{ordinal
addition}\index{addition!of ordinals} (of which natural numbers are a
subset) is

\setbox\startprefix=\hbox{\tt \ \ df-oadd\ \$a\ }
\setbox\contprefix=\hbox{\tt \ \ \ \ \ \ \ \ \ \ \ \ \ }
\startm
\m{\vdash}\m{+_o}\m{=}\m{(}\m{x}\m{\in}\m{{\rm On}}\m{,}\m{y}\m{\in}\m{{\rm
On}}\m{\mapsto}\m{(}\m{{\rm rec}}\m{(}\m{(}\m{z}\m{\in}\m{{\rm
V}}\m{\mapsto}\m{{\rm suc}}\m{z}\m{)}\m{,}\m{x}\m{)}\m{`}\m{y}\m{)}\m{)}
\endm
\noindent which depends on ${\rm rec}$.

\subsubsection{Recursion operators}

The above definition of \texttt{df-oadd} depends on the definition of
${\rm rec}$, a ``recursion operator''\index{recursion operator} with
the definition \texttt{df-rdg}:

\setbox\startprefix=\hbox{\tt \ \ df-rdg\ \$a\ }
\setbox\contprefix=\hbox{\tt \ \ \ \ \ \ \ \ \ \ \ \ }
\startm
\m{\vdash}\m{{\rm
rec}}\m{(}\m{F}\m{,}\m{I}\m{)}\m{=}\m{\mathrm{recs}}\m{(}\m{(}\m{g}\m{\in}\m{{\rm
V}}\m{\mapsto}\m{{\rm if}}\m{(}\m{g}\m{=}\m{\varnothing}\m{,}\m{I}\m{,}\m{{\rm
if}}\m{(}\m{{\rm Lim}}\m{{\rm dom}}\m{g}\m{,}\m{\bigcup}\m{{\rm
ran}}\m{g}\m{,}\m{(}\m{F}\m{`}\m{(}\m{g}\m{`}\m{\bigcup}\m{{\rm
dom}}\m{g}\m{)}\m{)}\m{)}\m{)}\m{)}\m{)}
\endm

\noindent which can be further broken down with definitions shown in
Section~\ref{setdefinitions}.

This definition of ${\rm rec}$
defines a recursive definition generator on ${\rm On}$ (the class of ordinal
numbers) with characteristic function $F$ and initial value $I$.
This operation allows us to define, with
compact direct definitions, functions that are usually defined in
textbooks with recursive definitions.
The price paid with our approach
is the complexity of our ${\rm rec}$ operation
(especially when {\tt df-recs} that it is built on is also eliminated).
But once we get past this hurdle, definitions that would otherwise be
recursive become relatively simple, as in for example {\tt oav}, from
which we prove the recursive textbook definition as theorems {\tt oa0}, {\tt
oasuc}, and {\tt oalim} (with the help of theorems {\tt rdg0}, {\tt rdgsuc},
and {\tt rdglim2a}).  We can also restrict the ${\rm rec}$ operation to
define otherwise recursive functions on the natural numbers $\omega$; see {\tt
fr0g} and {\tt frsuc}.  Our ${\rm rec}$ operation apparently does not appear
in published literature, although closely related is Definition 25.2 of
[Quine] p. 177, which he uses to ``turn...a recursion into a genuine or
direct definition" (p. 174).  Note that the ${\rm if}$ operations (see
{\tt df-if}) select cases based on whether the domain of $g$ is zero, a
successor, or a limit ordinal.

An important use of this definition ${\rm rec}$ is in the recursive sequence
generator {\tt df-seq} on the natural numbers (as a subset of the
complex infinite sequences such as the factorial function {\tt df-fac} and
integer powers {\tt df-exp}).

The definition of ${\rm rec}$ depends on ${\rm recs}$.
New direct usage of the more powerful (and more primitive) ${\rm recs}$
construct is discouraged, but it is available when needed.
This
defines a function $\mathrm{recs} ( F )$ on ${\rm On}$, the class of ordinal
numbers, by transfinite recursion given a rule $F$ which sets the next
value given all values so far.
Unlike {\tt df-rdg} which restricts the
update rule to use only the previous value, this version allows the
update rule to use all previous values, which is why it is described
as ``strong,'' although it is actually more primitive.  See {\tt
recsfnon} and {\tt recsval} for the primary contract of this definition.
It is defined as:

\setbox\startprefix=\hbox{\tt \ \ df-recs\ \$a\ }
\setbox\contprefix=\hbox{\tt \ \ \ \ \ \ \ \ \ \ \ \ \ }
\startm
\m{\vdash}\m{\mathrm{recs}}\m{(}\m{F}\m{)}\m{=}\m{\bigcup}\m{\{}\m{f}\m{|}\m{\exists}\m{x}\m{\in}\m{{\rm
On}}\m{(}\m{f}\m{{\rm
Fn}}\m{x}\m{\wedge}\m{\forall}\m{y}\m{\in}\m{x}\m{(}\m{f}\m{`}\m{y}\m{)}\m{=}\m{(}\m{F}\m{`}\m{(}\m{f}\m{\restriction}\m{y}\m{)}\m{)}\m{)}\m{\}}
\endm

\subsubsection{Closing comments on direct definitions}

From these direct definitions the simpler, more
intuitive recursive definition is derived as a set of theorems.\index{natural
number}\index{addition}\index{recursive definition}\index{ordinal addition}
The end result is the same, but we completely eliminate the rather complex
metalogic that justifies the recursive definition.

Recursive definitions are often considered more efficient and intuitive than
direct ones once the metalogic has been learned or possibly just accepted as
correct.  However, it was felt that direct definition in \texttt{set.mm}
maximizes rigor by minimizing metalogic.  It can be eliminated effortlessly,
something that is difficult to do with a recursive definition.

Again, Metamath itself has no built-in technical limitation that prevents
multiple-part recursive definitions in the traditional textbook style.
Instead, our goal is to eliminate all definitions with
direct mechanical substitution and to verify easily the soundness of
definitions.

\subsection{Adding Constraints on Definitions}

The basic Metamath language and the Metamath program do
not have any built-in constraints on definitions, since they are just
\$a statements.

However, nothing prevents a verification system from verifying
additional rules to impose further limitations on definitions.
For example, the \texttt{mmj2}\index{mmj2} program
supports various kinds of
additional information comments (see section \ref{jcomment}).
One of their uses is to optionally verify additional constraints,
including constraints to verify that definitions meet certain
requirements.
These additional checks are required by the
continuous integration (CI)\index{continuous integration (CI)}
checks of the
\texttt{set.mm}\index{set theory database (\texttt{set.mm})}%
\index{Metamath Proof Explorer}
database.
This approach enables us to optionally impose additional requirements
on definitions if we wish, without necessarily imposing those rules on
all databases or requiring all verification systems to implement them.
In addition, this allows us to impose specialized constraints tailored
to one database while not requiring other systems to implement
those specialized constraints.

We impose two constraints on the
\texttt{set.mm}\index{set theory database (\texttt{set.mm})}%
\index{Metamath Proof Explorer} database
via the \texttt{mmj2}\index{mmj2} program that are worth discussing here:
a parse check and a definition soundness check.

% On February 11, 2019 8:32:32 PM EST, saueran@oregonstate.edu wrote:
% The following addition to the end of set.mm is accepted by the mmj2
% parser and definition checker and the metamath verifier(at least it was
% when I checked, you should check it too), and creates a contradiction by
% proving the theorem |- ph.
% ${
% wleftp $a wff ( ( ph ) $.
% wbothp $a wff ( ph ) $.
% df-leftp $a |- ( ( ( ph ) <-> -. ph ) $.
% df-bothp $a |- ( ( ph ) <-> ph ) $.
% anything $p |- ph $=
%   ( wbothp wn wi wleftp df-leftp biimpi df-bothp mpbir mpbi simplim ax-mp)
%   ABZAMACZDZCZMOEZOCQAEZNDZRNAFGSHIOFJMNKLAHJ $.
% $}
%
% This particular problem is countered by enabling, within mmj2,
% SetParser,mmj.verify.LRParser

First,
we enable a parse check in \texttt{mmj2} (through its
\texttt{SetParser} command) that requires that all new definitions
must \textit{not} create an ambiguous parse for a KLR(5) parser.
This prevents some errors such as definitions with imbalanced parentheses.

Second, we run a definition soundness check specific to
\texttt{set.mm} or databases similar to it.
(through the \texttt{definitionCheck} macro).
Some \texttt{\$a} statements (including all ax-* statemnets)
are excluded from these checks, as they will
always fail this simple check,
but they are appropriate for most definitions.
This check imposes a set of additional rules:

\begin{enumerate}

\item New definitions must be introduced using $=$ or $\leftrightarrow$.

\item No \texttt{\$a} statement introduced before this one is allowed to use the
symbol being defined in this definition, and the definition is not
allowed to use itself (except once, in the definiendum).

\item Every variable in the definiens should not be distinct

\item Every dummy variable in the definiendum
are required to be distinct from each other and from variables in
the definiendum.
To determine this, the system will look for a "justification" theorem
in the database, and if it is not there, attempt to briefly prove
$( \varphi \rightarrow \forall x \varphi )$  for each dummy variable x.

\item Every dummy variable should be a set variable,
unless there is a justification theorem available.

\item Every dummy variable must be bound
(if the system cannot determine this a justification theorem must be
provided).

\end{enumerate}

\subsection{Summary of Approach to Definitions}

In short, when being rigorous it turns out that
definitions can be subtle, sometimes requiring difficult
metatheorems to establish that they are not creative.

Instead of building such complications into the Metamath language itself,
the basic Metmath language and program simply treat traditional
axioms and definitions as the same kind of \texttt{\$a} statement.
We have then built various tools to enable people to
verify additional conditions as their creators believe is appropriate
for those specific databases, without complicating the Metamath foundations.

\chapter{The Metamath Program}\label{commands}

This chapter provides a reference manual for the
Metamath program.\index{Metamath!commands}

Current instructions for obtaining and installing the Metamath program
can be found at the \url{http://metamath.org} web site.
For Windows, there is a pre-compiled version called
\texttt{metamath.exe}.  For Unix, Linux, and Mac OS X
(which we will refer to collectively as ``Unix''), the Metamath program
can be compiled from its source code with the command
\begin{verbatim}
gcc *.c -o metamath
\end{verbatim}
using the \texttt{gcc} {\sc c} compiler available on those systems.

In the command syntax descriptions below, fields enclosed in square
brackets [\ ] are optional.  File names may be optionally enclosed in
single or double quotes.  This is useful if the file name contains
spaces or
slashes (\texttt{/}), such as in Unix path names, \index{Unix file
names}\index{file names!Unix} that might be confused with Metamath
command qualifiers.\index{Unix file names}\index{file names!Unix}

\section{Invoking Metamath}

Unix, Linux, and Mac OS X
have a command-line interface called the {\em
bash shell}.  (In Mac OS X, select the Terminal application from
Applications/Utilities.)  To invoke Metamath from the bash shell prompt,
assuming that the Metamath program is in the current directory, type
\begin{verbatim}
bash$ ./metamath
\end{verbatim}

To invoke Metamath from a Windows DOS or Command Prompt, assuming that
the Metamath program is in the current directory (or in a directory
included in the Path system environment variable), type
\begin{verbatim}
C:\metamath>metamath
\end{verbatim}

To use command-line arguments at invocation, the command-line arguments
should be a list of Metamath commands, surrounded by quotes if they
contain spaces.  In Windows, the surrounding quotes must be double (not
single) quotes.  For example, to read the database file \texttt{set.mm},
verify all proofs, and exit the program, type (under Unix)
\begin{verbatim}
bash$ ./metamath 'read set.mm' 'verify proof *' exit
\end{verbatim}
Note that in Unix, any directory path with \texttt{/}'s must be
surrounded by quotes so Metamath will not interpret the \texttt{/} as a
command qualifier.  So if \texttt{set.mm} is in the \texttt{/tmp}
directory, use for the above example
\begin{verbatim}
bash$ ./metamath 'read "/tmp/set.mm"' 'verify proof *' exit
\end{verbatim}

For convenience, if the command-line has one argument and no spaces in
the argument, the command is implicitly assumed to be \texttt{read}.  In
this one special case, \texttt{/}'s are not interpreted as command
qualifiers, so you don't need quotes around a Unix file name.  Thus
\begin{verbatim}
bash$ ./metamath /tmp/set.mm
\end{verbatim}
and
\begin{verbatim}
bash$ ./metamath "read '/tmp/set.mm'"
\end{verbatim}
are equivalent.


\section{Controlling Metamath}

The Metamath program was first developed on a {\sc vax/vms} system, and
some aspects of its command line behavior reflect this heritage.
Hopefully you will find it reasonably user-friendly once you get used to
it.

Each command line is a sequence of English-like words separated by
spaces, as in \texttt{show settings}.  Command words are not case
sensitive, and only as many letters are needed as are necessary to
eliminate ambiguity; for example, \texttt{sh se} would work for the
command \texttt{show settings}.  In some cases arguments such as file
names, statement labels, or symbol names are required; these are
case-sensitive (although file names may not be on some operating
systems).

A command line is entered by typing it in then pressing the {\em return}
({\em enter}) key.  To find out what commands are available, type
\texttt{?} at the \texttt{MM>} prompt.  To find out the choices at any
point in a command, press {\em return} and you will be prompted for
them.  The default choice (the one selected if you just press {\em
return}) is shown in brackets (\texttt{<>}).

You may also type \texttt{?} in place of a command word to force
Metamath to tell you what the choices are.  The \texttt{?} method won't
work, though, if a non-keyword argument such as a file name is expected
at that point, because the program will think that \texttt{?} is the
value of the argument.

Some commands have one or more optional qualifiers which modify the
behavior of the command.  Qualifiers are preceded by a slash
(\texttt{/}), such as in \texttt{read set.mm / verify}.  Spaces are
optional around the \texttt{/}.  If you need to use a space or
slash in a command
argument, as in a Unix file name, put single or double quotes around the
command argument.

The \texttt{open log} command will save everything you see on the
screen and is useful to help you recover should something go wrong in a
proof, or if you want to document a bug.

If a command responds with more than a screenful, you will be
prompted to \texttt{<return> to continue, Q to quit, or S to scroll to
end}.  \texttt{Q} or \texttt{q} (not case-sensitive) will complete the
command internally but will suppress further output until the next
\texttt{MM>} prompt.  \texttt{s} will suppress further pausing until the
next \texttt{MM>} prompt.  After the first screen, you are also
presented with the choice of \texttt{b} to go back a screenful.  Note
that \texttt{b} may also be entered at the \texttt{MM>} prompt
immediately after a command to scroll back through the output of that
command.

A command line enclosed in quotes is executed by your operating system.
See Section~\ref{oscmd}.

{\em Warning:} Pressing {\sc ctrl-c} will abort the Metamath program
unconditionally.  This means any unsaved work will be lost.


\subsection{\texttt{exit} Command}\index{\texttt{exit} command}

Syntax:  \texttt{exit} [\texttt{/force}]

This command exits from Metamath.  If there have been changes to the
source with the \texttt{save proof} or \texttt{save new{\char`\_}proof}
commands, you will be given an opportunity to \texttt{write source} to
permanently save the changes.

In Proof Assistant\index{Proof Assistant} mode, the \texttt{exit} command will
return to the \verb/MM>/ prompt. If there were changes to the proof, you will
be given an opportunity to \texttt{save new{\char`\_}proof}.

The \texttt{quit} command is a synonym for \texttt{exit}.

Optional qualifier:
    \texttt{/force} - Do not prompt if changes were not saved.  This qualifier is
        useful in \texttt{submit} command files (Section~\ref{sbmt})
        to ensure predictable behavior.





\subsection{\texttt{open log} Command}\index{\texttt{open log} command}
Syntax:  \texttt{open log} {\em file-name}

This command will open a log file that will store everything you see on
the screen.  It is useful to help recovery from a mistake in a long Proof
Assistant session, or to document bugs.\index{Metamath!bugs}

The log file can be closed with \texttt{close log}.  It will automatically be
closed upon exiting Metamath.



\subsection{\texttt{close log} Command}\index{\texttt{close log} command}
Syntax:  \texttt{close log}

The \texttt{close log} command closes a log file if one is open.  See
also \texttt{open log}.




\subsection{\texttt{submit} Command}\index{\texttt{submit} command}\label{sbmt}
Syntax:  \texttt{submit} {\em filename}

This command causes further command lines to be taken from the specified
file.  Note that any line beginning with an exclamation point (\texttt{!}) is
treated as a comment (i.e.\ ignored).  Also note that the scrolling
of the screen output is continuous, so you may want to open a log file
(see \texttt{open log}) to record the results that fly by on the screen.
After the lines in the file are exhausted, Metamath returns to its
normal user interface mode.

The \texttt{submit} command can be called recursively (i.e. from inside
of a \texttt{submit} command file).


Optional command qualifier:

    \texttt{/silent} - suppresses the screen output but still
        records the output in a log file if one is open.


\subsection{\texttt{erase} Command}\index{\texttt{erase} command}
Syntax:  \texttt{erase}

This command will reset Metamath to its starting state, deleting any
data\-base that was \texttt{read} in.
 If there have been changes to the
source with the \texttt{save proof} or \texttt{save new{\char`\_}proof}
commands, you will be given an opportunity to \texttt{write source} to
permanently save the changes.



\subsection{\texttt{set echo} Command}\index{\texttt{set echo} command}
Syntax:  \texttt{set echo on} or \texttt{set echo off}

The \texttt{set echo on} command will cause command lines to be echoed with any
abbreviations expanded.  While learning the Metamath commands, this
feature will show you the exact command that your abbreviated input
corresponds to.



\subsection{\texttt{set scroll} Command}\index{\texttt{set scroll} command}
Syntax:  \texttt{set scroll prompted} or \texttt{set scroll continuous}

The Metamath command line interface starts off in the \texttt{prompted} mode,
which means that you will be prompted to continue or quit after each
full screen in a long listing.  In \texttt{continuous} mode, long listings will be
scrolled without pausing.

% LaTeX bug? (1) \texttt{\_} puts out different character than
% \texttt{{\char`\_}}
%  = \verb$_$  (2) \texttt{{\char`\_}} puts out garbage in \subsection
%  argument
\subsection{\texttt{set width} Command}\index{\texttt{set
width} command}
Syntax:  \texttt{set width} {\em number}

Metamath assumes the width of your screen is 79 characters (chosen
because the Command Prompt in Windows XP has a wrapping bug at column
80).  If your screen is wider or narrower, this command allows you to
change this default screen width.  A larger width is advantageous for
logging proofs to an output file to be printed on a wide printer.  A
smaller width may be necessary on some terminals; in this case, the
wrapping of the information messages may sometimes seem somewhat
unnatural, however.  In \LaTeX\index{latex@{\LaTeX}!characters per line},
there is normally a maximum of 61 characters per line with typewriter
font.  (The examples in this book were produced with 61 characters per
line.)

\subsection{\texttt{set height} Command}\index{\texttt{set
height} command}
Syntax:  \texttt{set height} {\em number}

Metamath assumes your screen height is 24 lines of characters.  If your
screen is taller or shorter, this command lets you to change the number
of lines at which the display pauses and prompts you to continue.

\subsection{\texttt{beep} Command}\index{\texttt{beep} command}

Syntax:  \texttt{beep}

This command will produce a beep.  By typing it ahead after a
long-running command has started, it will alert you that the command is
finished.  For convenience, \texttt{b} is an abbreviation for
\texttt{beep}.

Note:  If \texttt{b} is typed at the \texttt{MM>} prompt immediately
after the end of a multiple-page display paged with ``\texttt{Press
<return> for more}...'' prompts, then the \texttt{b} will back up to the
previous page rather than perform the \texttt{beep} command.
In that case you must type the unabbreviated \texttt{beep} form
of the command.

\subsection{\texttt{more} Command}\index{\texttt{more} command}

Syntax:  \texttt{more} {\em filename}

This command will display the contents of an {\sc ascii} file on your
screen.  (This command is provided for convenience but is not very
powerful.  See Section~\ref{oscmd} to invoke your operating system's
command to do this, such as the \texttt{more} command in Unix.)

\subsection{Operating System Commands}\index{operating system
command}\label{oscmd}

A line enclosed in single or double quotes will be executed by your
computer's operating system if it has a command line interface.  For
example, on a {\sc vax/vms} system,
\verb/MM> 'dir'/
will print disk directory contents.  Note that this feature will not
work on the Macintosh prior to Mac OS X, which does not have a command
line interface.

For your convenience, the trailing quote is optional.

\subsection{Size Limitations in Metamath}

In general, there are no fixed, predefined limits\index{Metamath!memory
limits} on how many labels, tokens\index{token}, statements, etc.\ that
you may have in a database file.  The Metamath program uses 32-bit
variables (64-bit on 64-bit CPUs) as indices for almost all internal
arrays, which are allocated dynamically as needed.



\section{Reading and Writing Files}

The following commands create new files:  the \texttt{open} commands;
the \texttt{write} commands; the \texttt{/html},
\texttt{/alt{\char`\_}html}, \texttt{/brief{\char`\_}html},
\texttt{/brief{\char`\_}alt{\char`\_}html} qualifiers of \texttt{show
statement}, and \texttt{midi}.  The following commands append to files
previously opened:  the \texttt{/tex} qualifier of \texttt{show proof}
and \texttt{show new{\char`\_}proof}; the \texttt{/tex} and
\texttt{/simple{\char`\_}tex} qualifiers of \texttt{show statement}; the
\texttt{close} commands; and all screen dialog between \texttt{open log}
and \texttt{close log}.

The commands that create new files will not overwrite an existing {\em
filename} but will rename the existing one to {\em
filename}\texttt{{\char`\~}1}.  An existing {\em
filename}\texttt{{\char`\~}1} is renamed {\em
filename}\texttt{{\char`\~}2}, etc.\ up to {\em
filename}\texttt{{\char`\~}9}.  An existing {\em
filename}\texttt{{\char`\~}9} is deleted.  This makes recovery from
mistakes easier but also will clutter up your directory, so occasionally
you may want to clean up (delete) these \texttt{{\char`\~}}$n$ files.


\subsection{\texttt{read} Command}\index{\texttt{read} command}
Syntax:  \texttt{read} {\em file-name} [\texttt{/verify}]

This command will read in a Metamath language source file and any included
files.  Normally it will be the first thing you do when entering Metamath.
Statement syntax is checked, but proof syntax is not checked.
Note that the file name may be enclosed in single or double quotes;
this is useful if the file name contains slashes, as might be the case
under Unix.

If you are getting an ``\texttt{?Expected VERIFY}'' error
when trying to read a Unix file name with slashes, you probably haven't
quoted it.\index{Unix file names}\index{file names!Unix}

If you are prompted for the file name (by pressing {\em return}
 after \texttt{read})
you should {\em not} put quotes around it, even if it is a Unix file name
with slashes.

Optional command qualifier:

    \texttt{/verify} - Verify all proofs as the database is read in.  This
         qualifier will slow down reading in the file.  See \texttt{verify
         proof} for more information on file error-checking.

See also \texttt{erase}.



\subsection{\texttt{write source} Command}\index{\texttt{write source} command}
Syntax:  \texttt{write source} {\em filename}
[\texttt{/rewrap}]
[\texttt{/split}]
% TeX doesn't handle this long line with tt text very well,
% so force a line break here.
[\texttt{/keep\_includes}] {\\}
[\texttt{/no\_versioning}]

This command will write the contents of a Metamath\index{database}
database into a file.\index{source file}

Optional command qualifiers:

\texttt{/rewrap} -
Reformats statements and comments according to the
convention used in the set.mm database.
It unwraps the
lines in the comment before each \texttt{\$a} and \texttt{\$p} statement,
then it
rewraps the line.  You should compare the output to the original
to make sure that the desired effect results; if not, go back to
the original source.  The wrapped line length honors the
\texttt{set width}
parameter currently in effect.  Note:  Text
enclosed in \texttt{<HTML>}...\texttt{</HTML>} tags is not modified by the
\texttt{/rewrap} qualifier.
Proofs themselves are not reformatted;
use \texttt{save proof * / compressed} to do that.
An isolated tilde (\~{}) is always kept on the same line as the following
symbol, so you can find all comment references to a symbol by
searching for \~{} followed by a space and that symbol
(this is useful for finding cross-references).
Incidentally, \texttt{save proof} also honors the \texttt{set width}
parameter currently in effect.

\texttt{/split} - Files included in the source using the expression
\$[ \textit{inclfile} \$] will be
written into separate files instead of being included in a single output
file.  The name of each separately written included file will be the
\textit{inclfile} argument of its inclusion command.

\texttt{/keep\_includes} - If a source file has includes but is written as a
single file by omitting \texttt{/split}, by default the included files will
be deleted (actually just renamed with a \char`\~1 suffix unless
\texttt{/no\_versioning} is specified) to prevent the possibly confusing
source duplication in both the output file and the included file.
The \texttt{/keep\_includes} qualifier will prevent this deletion.

\texttt{/no\_versioning} - Backup files suffixed with \char`\~1 are not created.


\section{Showing Status and Statements}



\subsection{\texttt{show settings} Command}\index{\texttt{show settings} command}
Syntax:  \texttt{show settings}

This command shows the state of various parameters.

\subsection{\texttt{show memory} Command}\index{\texttt{show memory} command}
Syntax:  \texttt{show memory}

This command shows the available memory left.  It is not meaningful
on most modern operating systems,
which have virtual memory.\index{Metamath!memory usage}


\subsection{\texttt{show labels} Command}\index{\texttt{show labels} command}
Syntax:  \texttt{show labels} {\em label-match} [\texttt{/all}]
   [\texttt{/linear}]

This command shows the labels of \texttt{\$a} and \texttt{\$p}
statements that match {\em label-match}.  A \verb$*$ in {label-match}
matches zero or more characters.  For example, \verb$*abc*def$ will match all
labels containing \verb$abc$ and ending with \verb$def$.

Optional command qualifiers:

   \texttt{/all} - Include matches for \texttt{\$e} and \texttt{\$f}
   statement labels.

   \texttt{/linear} - Display only one label per line.  This can be useful for
       building scripts in conjunction with the utilities under the
       \texttt{tools}\index{\texttt{tools} command} command.



\subsection{\texttt{show statement} Command}\index{\texttt{show statement} command}
Syntax:  \texttt{show statement} {\em label-match} [{\em qualifiers} (see below)]

This command provides information about a statement.  Only statements
that have labels (\texttt{\$f}\index{\texttt{\$f} statement},
\texttt{\$e}\index{\texttt{\$e} statement},
\texttt{\$a}\index{\texttt{\$a} statement}, and
\texttt{\$p}\index{\texttt{\$p} statement}) may be specified.
If {\em label-match}
contains wildcard (\verb$*$) characters, all matching statements will be
displayed in the order they occur in the database.

Optional qualifiers (only one qualifier at a time is allowed):

    \texttt{/comment} - This qualifier includes the comment that immediately
       precedes the statement.

    \texttt{/full} - Show complete information about each statement,
       and show all
       statements matching {\em label} (including \texttt{\$e}
       and \texttt{\$f} statements).

    \texttt{/tex} - This qualifier will write the statement information to the
       \LaTeX\ file previously opened with \texttt{open tex}.  See
       Section~\ref{texout}.

    \texttt{/simple{\char`\_}tex} - The same as \texttt{/tex}, except that
       \LaTeX\ macros are not used for formatting equations, allowing easier
       manual edits of the output for slide presentations, etc.

    \texttt{/html}\index{html generation@{\sc html} generation},
       \texttt{/alt{\char`\_}html}, \texttt{/brief{\char`\_}html},
       \texttt{/brief{\char`\_}alt{\char`\_}html} -
       These qualifiers invoke a special mode of
       \texttt{show statement} that
       creates a web page for the statement.  They may not be used with
       any other qualifier.  See Section~\ref{htmlout} or
       \texttt{help html} in the program.


\subsection{\texttt{search} Command}\index{\texttt{search} command}
Syntax:  search {\em label-match}
\texttt{"}{\em symbol-match}{\tt}" [\texttt{/all}] [\texttt{/comments}]
[\texttt{/join}]

This command searches all \texttt{\$a} and \texttt{\$p} statements
matching {\em label-match} for occurrences of {\em symbol-match}.  A
\verb@*@ in {\em label-match} matches any label character.  A \verb@$*@
in {\em symbol-match} matches any sequence of symbols.  The symbols in
{\em symbol-match} must be separated by white space.  The quotes
surrounding {\em symbol-match} may be single or double quotes.  For
example, \texttt{search b}\verb@* "-> $* ch"@ will list all statements
whose labels begin with \texttt{b} and contain the symbols \verb@->@ and
\texttt{ch} surrounding any symbol sequence (including no symbol
sequence).  The wildcards \texttt{?} and \texttt{\$?} are also available
to match individual characters in labels and symbols respectively; see
\texttt{help search} in the Metamath program for details on their usage.

Optional command qualifiers:

    \texttt{/all} - Also search \texttt{\$e} and \texttt{\$f} statements.

    \texttt{/comments} - Search the comment that immediately precedes each
        label-matched statement for {\em symbol-match}.  In this case
        {\em symbol-match} is an arbitrary, non-case-sensitive character
        string.  Quotes around {\em symbol-match} are optional if there
        is no ambiguity.

    \texttt{/join} - In the case of a \texttt{\$a} or \texttt{\$p} statement,
	prepend its \texttt{\$e}
	hypotheses for searching. The
	\texttt{/join} qualifier has no effect in \texttt{/comments} mode.

\section{Displaying and Verifying Proofs}


\subsection{\texttt{show proof} Command}\index{\texttt{show proof} command}
Syntax:  \texttt{show proof} {\em label-match} [{\em qualifiers} (see below)]

This command displays the proof of the specified
\texttt{\$p}\index{\texttt{\$p} statement} statement in various formats.
The {\em label-match} may contain wildcard (\verb@$*@) characters to match
multiple statements.  Without any qualifiers, only the logical steps
will be shown (i.e.\ syntax construction steps will be omitted), in an
indented format.

Most of the time, you will use
    \texttt{show proof} {\em label}
to see just the proof steps corresponding to logical inferences.

Optional command qualifiers:

    \texttt{/essential} - The proof tree is trimmed of all
        \texttt{\$f}\index{\texttt{\$f} statement} hypotheses before
        being displayed.  (This is the default, and it is redundant to
        specify it.)

    \texttt{/all} - the proof tree is not trimmed of all \texttt{\$f} hypotheses before
        being displayed.  \texttt{/essential} and \texttt{/all} are mutually exclusive.

    \texttt{/from{\char`\_}step} {\em step} - The display starts at the specified
        step.  If
        this qualifier is omitted, the display starts at the first step.

    \texttt{/to{\char`\_}step} {\em step} - The display ends at the specified
        step.  If this
        qualifier is omitted, the display ends at the last step.

    \texttt{/tree{\char`\_}depth} {\em number} - Only
         steps at less than the specified proof
        tree depth are displayed.  Sometimes useful for obtaining an overview of
        the proof.

    \texttt{/reverse} - The steps are displayed in reverse order.

    \texttt{/renumber} - When used with \texttt{/essential}, the steps are renumbered
        to correspond only to the essential steps.

    \texttt{/tex} - The proof is converted to \LaTeX\ and\index{latex@{\LaTeX}}
        stored in the file opened
        with \texttt{open tex}.  See Section~\ref{texout} or
        \texttt{help tex} in the program.

    \texttt{/lemmon} - The proof is displayed in a non-indented format known
        as Lemmon style, with explicit previous step number references.
        If this qualifier is omitted, steps are indented in a tree format.

    \texttt{/start{\char`\_}column} {\em number} - Overrides the default column
        (16)
        at which the formula display starts in a Lemmon-style display.  May be
        used only in conjunction with \texttt{/lemmon}.

    \texttt{/normal} - The proof is displayed in normal format suitable for
        inclusion in a Metamath source file.  May not be used with any other
        qualifier.

    \texttt{/compressed} - The proof is displayed in compressed format
        suitable for inclusion in a Metamath source file.  May not be used with
        any other qualifier.

    \texttt{/statement{\char`\_}summary} - Summarizes all statements (like a
        brief \texttt{show statement})
        used by the proof.  It may not be used with any other qualifier
        except \texttt{/essential}.

    \texttt{/detailed{\char`\_}step} {\em step} - Shows the details of what is
        happening at
        a specific proof step.  May not be used with any other qualifier.
        The {\em step} is the step number shown when displaying a
        proof without the \texttt{/renumber} qualifier.


\subsection{\texttt{show usage} Command}\index{\texttt{show usage} command}
Syntax:  \texttt{show usage} {\em label-match} [\texttt{/recursive}]

This command lists the statements whose proofs make direct reference to
the statement specified.

Optional command qualifier:

    \texttt{/recursive} - Also include statements whose proofs ultimately
        depend on the statement specified.



\subsection{\texttt{show trace\_back} Command}\index{\texttt{show
       trace{\char`\_}back} command}
Syntax:  \texttt{show trace{\char`\_}back} {\em label-match} [\texttt{/essential}] [\texttt{/axioms}]
    [\texttt{/tree}] [\texttt{/depth} {\em number}]

This command lists all statements that the proof of the \texttt{\$p}
statement(s) specified by {\em label-match} depends on.

Optional command qualifiers:

    \texttt{/essential} - Restrict the trace-back to \texttt{\$e}
        \index{\texttt{\$e} statement} hypotheses of proof trees.

    \texttt{/axioms} - List only the axioms that the proof ultimately depends on.

    \texttt{/tree} - Display the trace-back in an indented tree format.

    \texttt{/depth} {\em number} - Restrict the \texttt{/tree} trace-back to the
        specified indentation depth.

    \texttt{/count{\char`\_}steps} - Count the number of steps the proof has
       all the way back to axioms.  If \texttt{/essential} is specified,
       expansions of variable-type hypotheses (syntax constructions) are not counted.

\subsection{\texttt{verify proof} Command}\index{\texttt{verify proof} command}
Syntax:  \texttt{verify proof} {\em label-match} [\texttt{/syntax{\char`\_}only}]

This command verifies the proofs of the specified statements.  {\em
label-match} may contain wild card characters (\texttt{*}) to verify
more than one proof; for example \verb/*abc*def/ will match all labels
containing \texttt{abc} and ending with \texttt{def}.
The command \texttt{verify proof *} will verify all proofs in the database.

Optional command qualifier:

    \texttt{/syntax{\char`\_}only} - This qualifier will perform a check of syntax
        and RPN
        stack violations only.  It will not verify that the proof is
        correct.  This qualifier is useful for quickly determining which
        proofs are incomplete (i.e.\ are under development and have \texttt{?}'s
        in them).

{\em Note:} \texttt{read}, followed by \texttt{verify proof *}, will ensure
 the database is free
from errors in the Metamath language but will not check the markup notation
in comments.
You can also check the markup notation by running \texttt{verify markup *}
as discussed in Section~\ref{verifymarkup}; see also the discussion
on generating {\sc HTML} in Section~\ref{htmlout}.

\subsection{\texttt{verify markup} Command}\index{\texttt{verify markup} command}\label{verifymarkup}
Syntax:  \texttt{verify markup} {\em label-match}
[\texttt{/date{\char`\_}skip}]
[\texttt{/top{\char`\_}date{\char`\_}skip}] {\\}
[\texttt{/file{\char`\_}skip}]
[\texttt{/verbose}]

This command checks comment markup and other informal conventions we have
adopted.  It error-checks the latexdef, htmldef, and althtmldef statements
in the \texttt{\$t} statement of a Metamath source file.\index{error checking}
It error-checks any \texttt{`}...\texttt{`},
\texttt{\char`\~}~\textit{label},
and bibliographic markups in statement descriptions.
It checks that
\texttt{\$p} and \texttt{\$a} statements
have the same content when their labels start with
``ax'' and ``ax-'' respectively but are otherwise identical, for example
ax4 and ax-4.
It also verifies the date consistency of ``(Contributed by...),''
``(Revised by...),'' and ``(Proof shortened by...)'' tags in the comment
above each \texttt{\$a} and \texttt{\$p} statement.

Optional command qualifiers:

    \texttt{/date{\char`\_}skip} - This qualifier will
        skip date consistency checking,
        which is usually not required for databases other than
	\texttt{set.mm}.

    \texttt{/top{\char`\_}date{\char`\_}skip} - This qualifier will check date consistency except
        that the version date at the top of the database file will not
        be checked.  Only one of
        \texttt{/date{\char`\_}skip} and
        \texttt{/top{\char`\_}date{\char`\_}skip} may be
        specified.

    \texttt{/file{\char`\_}skip} - This qualifier will skip checks that require
        external files to be present, such as checking GIF existence and
        bibliographic links to mmset.html or equivalent.  It is useful
        for doing a quick check from a directory without these files.

    \texttt{/verbose} - Provides more information.  Currently it provides a list
        of axXXX vs. ax-XXX matches.

\subsection{\texttt{save proof} Command}\index{\texttt{save proof} command}
Syntax:  \texttt{save proof} {\em label-match} [\texttt{/normal}]
   [\texttt{/compressed}]

The \texttt{save proof} command will reformat a proof in one of two formats and
replace the existing proof in the source buffer\index{source
buffer}.  It is useful for
converting between proof formats.  Note that a proof will not be
permanently saved until a \texttt{write source} command is issued.

Optional command qualifiers:

    \texttt{/normal} - The proof is saved in the normal format (i.e., as a
        sequence
        of labels, which is the defined format of the basic Metamath
        language).\index{basic language}  This is the default format that
        is used if a qualifier
        is omitted.

    \texttt{/compressed} - The proof is saved in the compressed format which
        reduces storage requirements for a database.
        See Appendix~\ref{compressed}.




\section{Creating Proofs}\label{pfcommands}\index{Proof Assistant}

Before using the Proof Assistant, you must add a \texttt{\$p} to your
source file (using a text editor) containing the statement you want to
prove.  Its proof should consist of a single \texttt{?}, meaning
``unknown step.''  Example:
\begin{verbatim}
equid $p x = x $= ? $.
\end{verbatim}

To enter the Proof assistant, type \texttt{prove} {\em label}, e.g.
\texttt{prove equid}.  Metamath will respond with the \texttt{MM-PA>}
prompt.

Proofs are created working backwards from the statement being proved,
primarily using a series of \texttt{assign} commands.  A proof is
complete when all steps are assigned to statements and all steps are
unified and completely known.  During the creation of a proof, Metamath
will allow only operations that are legal based on what is known up to
that point.  For example, it will not allow an \texttt{assign} of a
statement that cannot be unified with the unknown proof step being
assigned.

{\em Important:}
The Proof Assistant is
{\em not} a tool to help you discover proofs.  It is just a tool to help
you add them to the database.  For a tutorial read
Section~\ref{frstprf}.
To practice using the Proof Assistant, you may
want to \texttt{prove} an existing theorem, then delete all steps with
\texttt{delete all}, then re-create it with the Proof Assistant while
looking at its proof display (before deletion).
You might want to figure out your first few proofs completely
and write them down by hand, before using the Proof Assistant, though
not everyone finds that effective.

{\em Important:}
The \texttt{undo} command if very helpful when entering a proof, because
it allows you to undo a previously-entered step.
In addition, we suggest that you
keep track of your work with a log file (\texttt{open
log}) and save it frequently (\texttt{save new{\char`\_}proof},
\texttt{write source}).
You can use \texttt{delete} to reverse an \texttt{assign}.
You can also do \texttt{delete floating{\char`\_}hypotheses}, then
\texttt{initialize all}, then \texttt{unify all /interactive} to
reinitialize bad unifications made accidentally or by bad
\texttt{assign}s.  You cannot reverse a \texttt{delete} except by
a relevant \texttt{undo} or using
\texttt{exit /force} then reentering the Proof Assistant to recover from
the last \texttt{save new{\char`\_}proof}.

The following commands are available in the Proof Assistant (at the
\texttt{MM-PA>} prompt) to help you create your proof.  See the
individual commands for more detail.

\begin{itemize}
\item[]
    \texttt{show new{\char`\_}proof} [\texttt{/all},...] - Displays the
        proof in progress.  You will use this command a lot; see \texttt{help
        show new{\char`\_}proof} to become familiar with its qualifiers.  The
        qualifiers \texttt{/unknown} and \texttt{/not{\char`\_}unified} are
        useful for seeing the work remaining to be done.  The combination
        \texttt{/all/unknown} is useful for identifying dummy variables that must be
        assigned, or attempts to use illegal syntax, when \texttt{improve all}
        is unable to complete the syntax constructions.  Unknown variables are
        shown as \texttt{\$1}, \texttt{\$2},...
\item[]
    \texttt{assign} {\em step} {\em label} - Assigns an unknown {\em step}
        number with the statement
        specified by {\em label}.
\item[]
    \texttt{let variable} {\em variable}
        \texttt{= "}{\em symbol sequence}\texttt{"}
          - Forces a symbol
        sequence to replace an unknown variable (such as \texttt{\$1}) in a proof.
        It is useful
        for helping difficult unifications, and it is necessary when you have
        dummy variables that eventually must be assigned a name.
\item[]
    \texttt{let step} {\em step} \texttt{= "}{\em symbol sequence}\texttt{"} -
          Forces a symbol sequence
        to replace the contents of a proof step, provided it can be
        unified with the existing step contents.  (I rarely use this.)
\item[]
    \texttt{unify step} {\em step} (or \texttt{unify all}) - Unifies
        the source and target of
        a step.  If you specify a specific step, you will be prompted
        to select among the unifications that are possible.  If you
        specify \texttt{all}, all steps with unique unifications, but only
        those steps, will be
        unified.  \texttt{unify all /interactive} goes through all non-unified
        steps.
\item[]
    \texttt{initialize} {\em step} (or \texttt{all}) - De-unifies the target and source of
        a step (or all steps), as well as the hypotheses of the source,
        and makes all variables in the source unknown.  Useful to recover from
        an \texttt{assign} or \texttt{let} mistake that
        resulted in incorrect unifications.
\item[]
    \texttt{delete} {\em step} (or \texttt{all} or \texttt{floating{\char`\_}hypotheses}) -
        Deletes the specified
        step(s).  \texttt{delete floating{\char`\_}hypotheses}, then \texttt{initialize all}, then
        \texttt{unify all /interactive} is useful for recovering from mistakes
        where incorrect unifications assigned wrong math symbol strings to
        variables.
\item[]
    \texttt{improve} {\em step} (or \texttt{all}) -
      Automatically creates a proof for steps (with no unknown variables)
      whose proof requires no statements with \texttt{\$e} hypotheses.  Useful
      for filling in proofs of \texttt{\$f} hypotheses.  The \texttt{/depth}
      qualifier will also try statements whose \texttt{\$e} hypotheses contain
      no new variables.  {\em Warning:} Save your work (with \texttt{save
      new{\char`\_}proof} then \texttt{write source}) before using
      \texttt{/depth = 2} or greater, since the search time grows
      exponentially and may never terminate in a reasonable time, and you
      cannot interrupt the search.  I have found that it is rare for
      \texttt{/depth = 3} or greater to be useful.
 \item[]
    \texttt{save new{\char`\_}proof} - Saves the proof in progress in the program's
        internal database buffer.  To save it permanently into the database file,
        use \texttt{write source} after
        \texttt{save new{\char`\_}proof}.  To revert to the last
        \texttt{save new{\char`\_}proof},
        \texttt{exit /force} from the Proof Assistant then re-enter the Proof
        Assistant.
 \item[]
    \texttt{match step} {\em step} (or \texttt{match all}) - Shows what
        statements are
        possibilities for the \texttt{assign} statement. (This command
        is not very
        useful in its present form and hopefully will be improved
        eventually.  In the meantime, use the \texttt{search} statement for
        candidates matching specific math token combinations.)
 \item[]
 \texttt{minimize{\char`\_}with}\index{\texttt{minimize{\char`\_}with} command}
% 3/10/07 Note: line-breaking the above results in duplicate index entries
     - After a proof is complete, this command will attempt
        to match other database theorems to the proof to see if the proof
        size can be reduced as a result.  See \texttt{help
        minimize{\char`\_}with} in the
        Metamath program for its usage.
 \item[]
 \texttt{undo}\index{\texttt{undo} command}
    - Undo the effect of a proof-changing command (all but the
      \texttt{show} and \texttt{save} commands above).
 \item[]
 \texttt{redo}\index{\texttt{redo} command}
    - Reverse the previous \texttt{undo}.
\end{itemize}

The following commands set parameters that may be relevant to your proof.
Consult the individual \texttt{help set}... commands.
\begin{itemize}
   \item[] \texttt{set unification{\char`\_}timeout}
 \item[]
    \texttt{set search{\char`\_}limit}
  \item[]
    \texttt{set empty{\char`\_}substitution} - note that default is \texttt{off}
\end{itemize}

Type \texttt{exit} to exit the \texttt{MM-PA>}
 prompt and get back to the \texttt{MM>} prompt.
Another \texttt{exit} will then get you out of Metamath.



\subsection{\texttt{prove} Command}\index{\texttt{prove} command}
Syntax:  \texttt{prove} {\em label}

This command will enter the Proof Assistant\index{Proof Assistant}, which will
allow you to create or edit the proof of the specified statement.
The command-line prompt will change from \texttt{MM>} to \texttt{MM-PA>}.

Note:  In the present version (0.177) of
Metamath\index{Metamath!limitations of version 0.177}, the Proof
Assistant does not verify that \texttt{\$d}\index{\texttt{\$d}
statement} restrictions are met as a proof is being built.  After you
have completed a proof, you should type \texttt{save new{\char`\_}proof}
followed by \texttt{verify proof} {\em label} (where {\em label} is the
statement you are proving with the \texttt{prove} command) to verify the
\texttt{\$d} restrictions.

See also: \texttt{exit}

\subsection{\texttt{set unification\_timeout} Command}\index{\texttt{set
unification{\char`\_}timeout} command}
Syntax:  \texttt{set unification{\char`\_}timeout} {\em number}

(This command is available outside the Proof Assistant but affects the
Proof Assistant\index{Proof Assistant} only.)

Sometimes the Proof Assistant will inform you that a unification
time-out occurred.  This may happen when you try to \texttt{unify}
formulas with many temporary variables\index{temporary variable}
(\texttt{\$1}, \texttt{\$2}, etc.), since the time to compute all possible
unifications may grow exponentially with the number of variables.  If
you want Metamath to try harder (and you're willing to wait longer) you
may increase this parameter.  \texttt{show settings} will show you the
current value.

\subsection{\texttt{set empty\_substitution} Command}\index{\texttt{set
empty{\char`\_}substitution} command}
% These long names can't break well in narrow mode, and even "sloppy"
% is not enough. Work around this by not demanding justification.
\begin{flushleft}
Syntax:  \texttt{set empty{\char`\_}substitution on} or \texttt{set
empty{\char`\_}substitution off}
\end{flushleft}

(This command is available outside the Proof Assistant but affects the
Proof Assistant\index{Proof Assistant} only.)

The Metamath language allows variables to be
substituted\index{substitution!variable}\index{variable substitution}
with empty symbol sequences\index{empty substitution}.  However, in many
formal systems\index{formal system} this will never happen in a valid
proof.  Allowing for this possibility increases the likelihood of
ambiguous unifications\index{ambiguous
unification}\index{unification!ambiguous} during proof creation.
The default is that
empty substitutions are not allowed; for formal systems requiring them,
you must \texttt{set empty{\char`\_}substitution on}.
(An example where this must be \texttt{on}
would be a system that implements a Deduction Rule and in
which deductions from empty assumption lists would be permissible.  The
MIU-system\index{MIU-system} described in Appendix~\ref{MIU} is another
example.)
Note that empty substitutions are
always permissible in proof verification (VERIFY PROOF...) outside the
Proof Assistant.  (See the MIU system in the Metamath book for an example
of a system needing empty substitutions; another example would be a
system that implements a Deduction Rule and in which deductions from
empty assumption lists would be permissible.)

It is better to leave this \texttt{off} when working with \texttt{set.mm}.
Note that this command does not affect the way proofs are verified with
the \texttt{verify proof} command.  Outside of the Proof Assistant,
substitution of empty sequences for math symbols is always allowed.

\subsection{\texttt{set search\_limit} Command}\index{\texttt{set
search{\char`\_}limit} command} Syntax:  \texttt{set search{\char`\_}limit} {\em
number}

(This command is available outside the Proof Assistant but affects the
Proof Assistant\index{Proof Assistant} only.)

This command sets a parameter that determines when the \texttt{improve} command
in Proof Assistant mode gives up.  If you want \texttt{improve} to search harder,
you may increase it.  The \texttt{show settings} command tells you its current
value.


\subsection{\texttt{show new\_proof} Command}\index{\texttt{show
new{\char`\_}proof} command}
Syntax:  \texttt{show new{\char`\_}proof} [{\em
qualifiers} (see below)]

This command (available only in Proof Assistant mode) displays the proof
in progress.  It is identical to the \texttt{show proof} command, except that
there is no statement argument (since it is the statement being proved) and
the following qualifiers are not available:

    \texttt{/statement{\char`\_}summary}

    \texttt{/detailed{\char`\_}step}

Also, the following additional qualifiers are available:

    \texttt{/unknown} - Shows only steps that have no statement assigned.

    \texttt{/not{\char`\_}unified} - Shows only steps that have not been unified.

Note that \texttt{/essential}, \texttt{/depth}, \texttt{/unknown}, and
\texttt{/not{\char`\_}unified} may be
used in any combination; each of them effectively filters out additional
steps from the proof display.

See also:  \texttt{show proof}






\subsection{\texttt{assign} Command}\index{\texttt{assign} command}
Syntax:   \texttt{assign} {\em step} {\em label} [\texttt{/no{\char`\_}unify}]

   and:   \texttt{assign first} {\em label}

   and:   \texttt{assign last} {\em label}


This command, available in the Proof Assistant only, assigns an unknown
step (one with \texttt{?} in the \texttt{show new{\char`\_}proof}
listing) with the statement specified by {\em label}.  The assignment
will not be allowed if the statement cannot be unified with the step.

If \texttt{last} is specified instead of {\em step} number, the last
step that is shown by \texttt{show new{\char`\_}proof /unknown} will be
used.  This can be useful for building a proof with a command file (see
\texttt{help submit}).  It also makes building proofs faster when you know
the assignment for the last step.

If \texttt{first} is specified instead of {\em step} number, the first
step that is shown by \texttt{show new{\char`\_}proof /unknown} will be
used.

If {\em step} is zero or negative, the -{\em step}th from last unknown
step, as shown by \texttt{show new{\char`\_}proof /unknown}, will be
used.  \texttt{assign -1} {\em label} will assign the penultimate
unknown step, \texttt{assign -2} {\em label} the antepenultimate, and
\texttt{assign 0} {\em label} is the same as \texttt{assign last} {\em
label}.

Optional command qualifier:

    \texttt{/no{\char`\_}unify} - do not prompt user to select a unification if there is
        more than one possibility.  This is useful for noninteractive
        command files.  Later, the user can \texttt{unify all /interactive}.
        (The assignment will still be automatically unified if there is only
        one possibility and will be refused if unification is not possible.)



\subsection{\texttt{match} Command}\index{\texttt{match} command}
Syntax:  \texttt{match step} {\em step} [\texttt{/max{\char`\_}essential{\char`\_}hyp}
{\em number}]

    and:  \texttt{match all} [\texttt{/essential}]
          [\texttt{/max{\char`\_}essential{\char`\_}hyp} {\em number}]

This command, available in the Proof Assistant only, shows what
statements can be unified with the specified step(s).  {\em Note:} In
its current form, this command is not very useful because of the large
number of matches it reports.
It may be enhanced in the future.  In the meantime, the \texttt{search}
command can often provide finer control over locating theorems of interest.

Optional command qualifiers:

    \texttt{/max{\char`\_}essential{\char`\_}hyp} {\em number} - filters out
        of the list any statements
        with more than the specified number of
        \texttt{\$e}\index{\texttt{\$e} statement} hypotheses.

    \texttt{/essential{\char`\_}only} - in the \texttt{match all} statement, only
        the steps that
        would be listed in the \texttt{show new{\char`\_}proof /essential} display are
        matched.



\subsection{\texttt{let} Command}\index{\texttt{let} command}
Syntax: \texttt{let variable} {\em variable} = \verb/"/{\em symbol-sequence}\verb/"/

  and: \texttt{let step} {\em step} = \verb/"/{\em symbol-sequence}\verb/"/

These commands, available in the Proof Assistant\index{Proof Assistant}
only, assign a temporary variable\index{temporary variable} or unknown
step with a specific symbol sequence.  They are useful in the middle of
creating a proof, when you know what should be in the proof step but the
unification algorithm doesn't yet have enough information to completely
specify the temporary variables.  A ``temporary variable'' is one that
has the form \texttt{\$}{\em nn} in the proof display, such as
\texttt{\$1}, \texttt{\$2}, etc.  The {\em symbol-sequence} may contain
other unknown variables if desired.  Examples:

    \verb/let variable $32 = "A = B"/

    \verb/let variable $32 = "A = $35"/

    \verb/let step 10 = '|- x = x'/

    \verb/let step -2 = "|- ( $7 = ph )"/

Any symbol sequence will be accepted for the \texttt{let variable}
command.  Only those symbol sequences that can be unified with the step
will be accepted for \texttt{let step}.

The \texttt{let} commands ``zap'' the proof with information that can
only be verified when the proof is built up further.  If you make an
error, the command sequence \texttt{delete
floating{\char`\_}hypotheses}, \texttt{initialize all}, and
\texttt{unify all /interactive} will undo a bad \texttt{let} assignment.

If {\em step} is zero or negative, the -{\em step}th from last unknown
step, as shown by \texttt{show new{\char`\_}proof /unknown}, will be
used.  The command \texttt{let step 0} = \verb/"/{\em
symbol-sequence}\verb/"/ will use the last unknown step, \texttt{let
step -1} = \verb/"/{\em symbol-sequence}\verb/"/ the penultimate, etc.
If {\em step} is positive, \texttt{let step} may be used to assign known
(in the sense of having previously been assigned a label with
\texttt{assign}) as well as unknown steps.

Either single or double quotes can surround the {\em symbol-sequence} as
long as they are different from any quotes inside a {\em
symbol-sequence}.  If {\em symbol-sequence} contains both kinds of
quotes, see the instructions at the end of \texttt{help let} in the
Metamath program.


\subsection{\texttt{unify} Command}\index{\texttt{unify} command}
Syntax:  \texttt{unify step} {\em step}

      and:   \texttt{unify all} [\texttt{/interactive}]

These commands, available in the Proof Assistant only, unify the source
and target of the specified step(s). If you specify a specific step, you
will be prompted to select among the unifications that are possible.  If
you specify \texttt{all}, only those steps with unique unifications will be
unified.

Optional command qualifier for \texttt{unify all}:

    \texttt{/interactive} - You will be prompted to select among the
        unifications
        that are possible for any steps that do not have unique
        unifications.  (Otherwise \texttt{unify all} will bypass these.)

See also \texttt{set unification{\char`\_}timeout}.  The default is
100000, but increasing it to 1000000 can help difficult cases.  Manually
assigning some or all of the unknown variables with the \texttt{let
variable} command also helps difficult cases.



\subsection{\texttt{initialize} Command}\index{\texttt{initialize} command}
Syntax:  \texttt{initialize step} {\em step}

    and: \texttt{initialize all}

These commands, available in the Proof Assistant\index{Proof Assistant}
only, ``de-unify'' the target and source of a step (or all steps), as
well as the hypotheses of the source, and makes all variables in the
source and the source's hypotheses unknown.  This command is useful to
help recover from incorrect unifications that resulted from an incorrect
\texttt{assign}, \texttt{let}, or unification choice.  Part or all of
the command sequence \texttt{delete floating{\char`\_}hypotheses},
\texttt{initialize all}, and \texttt{unify all /interactive} will recover
from incorrect unifications.

See also:  \texttt{unify} and \texttt{delete}



\subsection{\texttt{delete} Command}\index{\texttt{delete} command}
Syntax:  \texttt{delete step} {\em step}

   and:      \texttt{delete all} -- {\em Warning: dangerous!}

   and:      \texttt{delete floating{\char`\_}hypotheses}

These commands are available in the Proof Assistant only.  The
\texttt{delete step} command deletes the proof tree section that
branches off of the specified step and makes the step become unknown.
\texttt{delete all} is equivalent to \texttt{delete step} {\em step}
where {\em step} is the last step in the proof (i.e.\ the beginning of
the proof tree).

In most cases the \texttt{undo} command is the best way to undo
a previous step.
An alternative is to salvage your last \texttt{save
new{\char`\_}proof} by exiting and reentering the Proof Assistant.
For this to work, keep a log file open to record your work
and to do \texttt{save new{\char`\_}proof} frequently, especially before
\texttt{delete}.

\texttt{delete floating{\char`\_}hypotheses} will delete all sections of
the proof that branch off of \texttt{\$f}\index{\texttt{\$f} statement}
statements.  It is sometimes useful to do this before an
\texttt{initialize} command to recover from an error.  Note that once a
proof step with a \texttt{\$f} hypothesis as the target is completely
known, the \texttt{improve} command can usually fill in the proof for
that step.  Unlike the deletion of logical steps, \texttt{delete
floating{\char`\_}hypotheses} is a relatively safe command that is
usually easy to recover from.



\subsection{\texttt{improve} Command}\index{\texttt{improve} command}
\label{improve}
Syntax:  \texttt{improve} {\em step} [\texttt{/depth} {\em number}]
                                               [\texttt{/no{\char`\_}distinct}]

   and:   \texttt{improve first} [\texttt{/depth} {\em number}]
                                              [\texttt{/no{\char`\_}distinct}]

   and:   \texttt{improve last} [\texttt{/depth} {\em number}]
                                              [\texttt{/no{\char`\_}distinct}]

   and:   \texttt{improve all} [\texttt{/depth} {\em number}]
                                              [\texttt{/no{\char`\_}distinct}]

These commands, available in the Proof Assistant\index{Proof Assistant}
only, try to find proofs automatically for unknown steps whose symbol
sequences are completely known.  They are primarily useful for filling in
proofs of \texttt{\$f}\index{\texttt{\$f} statement} hypotheses.  The
search will be restricted to statements having no
\texttt{\$e}\index{\texttt{\$e} statement} hypotheses.

\begin{sloppypar} % narrow
Note:  If memory is limited, \texttt{improve all} on a large proof may
overflow memory.  If you use \texttt{set unification{\char`\_}timeout 1}
before \texttt{improve all}, there will usually be sufficient
improvement to easily recover and completely \texttt{improve} the proof
later on a larger computer.  Warning:  Once memory has overflowed, there
is no recovery.  If in doubt, save the intermediate proof (\texttt{save
new{\char`\_}proof} then \texttt{write source}) before \texttt{improve
all}.
\end{sloppypar}

If \texttt{last} is specified instead of {\em step} number, the last
step that is shown by \texttt{show new{\char`\_}proof /unknown} will be
used.

If \texttt{first} is specified instead of {\em step} number, the first
step that is shown by \texttt{show new{\char`\_}proof /unknown} will be
used.

If {\em step} is zero or negative, the -{\em step}th from last unknown
step, as shown by \texttt{show new{\char`\_}proof /unknown}, will be
used.  \texttt{improve -1} will use the penultimate
unknown step, \texttt{improve -2} {\em label} the antepenultimate, and
\texttt{improve 0} is the same as \texttt{improve last}.

Optional command qualifier:

    \texttt{/depth} {\em number} - This qualifier will cause the search
        to include
        statements with \texttt{\$e} hypotheses (but no new variables in
        the \texttt{\$e}
        hypotheses), provided that the backtracking has not exceeded the
        specified depth. {\em Warning:}  Try \texttt{/depth 1},
        then \texttt{2}, then \texttt{3}, etc.
        in sequence because of possible exponential blowups.  Save your
        work before trying \texttt{/depth} greater than \texttt{1}!

    \texttt{/no{\char`\_}distinct} - Skip trial statements that have
        \texttt{\$d}\index{\texttt{\$d} statement} requirements.
        This qualifier will prevent assignments that might violate \texttt{\$d}
        requirements but it also could miss possible legal assignments.

See also: \texttt{set search{\char`\_}limit}

\subsection{\texttt{save new\_proof} Command}\index{\texttt{save
new{\char`\_}proof} command}
Syntax:  \texttt{save new{\char`\_}proof} {\em label} [\texttt{/normal}]
   [\texttt{/compressed}]

The \texttt{save new{\char`\_}proof} command is available in the Proof
Assistant only.  It saves the proof in progress in the source
buffer\index{source buffer}.  \texttt{save new{\char`\_}proof} may be
used to save a completed proof, or it may be used to save a proof in
progress in order to work on it later.  If an incomplete proof is saved,
any user assignments with \texttt{let step} or \texttt{let variable}
will be lost, as will any ambiguous unifications\index{ambiguous
unification}\index{unification!ambiguous} that were resolved manually.
To help make recovery easier, it can be helpful to \texttt{improve all}
before \texttt{save new{\char`\_}proof} so that the incomplete proof
will have as much information as possible.

Note that the proof will not be permanently saved until a \texttt{write
source} command is issued.

Optional command qualifiers:

    \texttt{/normal} - The proof is saved in the normal format (i.e., as a
        sequence of labels, which is the defined format of the basic Metamath
        language).\index{basic language}  This is the default format that
        is used if a qualifier is omitted.

    \texttt{/compressed} - The proof is saved in the compressed format, which
        reduces storage requirements for a database.
        (See Appendix~\ref{compressed}.)


\section{Creating \LaTeX\ Output}\label{texout}\index{latex@{\LaTeX}}

You can generate \LaTeX\ output given the
information in a database.
The database must already include the necessary typesetting information
(see section \ref{tcomment} for how to provide this information).

The \texttt{show statement} and \texttt{show proof} commands each have a
special \texttt{/tex} command qualifier that produces \LaTeX\ output.
(The \texttt{show statement} command also has the
\texttt{/simple{\char`\_}tex} qualifier for output that is easier to
edit by hand.)  Before you can use them, you must open a \LaTeX\ file to
which to send their output.  A typical complete session will use this
sequence of Metamath commands:

\begin{verbatim}
read set.mm
open tex example.tex
show statement a1i /tex
show proof a1i /all/lemmon/renumber/tex
show statement uneq2 /tex
show proof uneq2 /all/lemmon/renumber/tex
close tex
\end{verbatim}

See Section~\ref{mathcomments} for information on comment markup and
Appendix~\ref{ASCII} for information on how math symbol translation is
specified.

To format and print the \LaTeX\ source, you will need the \LaTeX\
program, which is standard on most Linux installations and available for
Windows.  On Linux, in order to create a {\sc pdf} file, you will
typically type at the shell prompt
\begin{verbatim}
$ pdflatex example.tex
\end{verbatim}

\subsection{\texttt{open tex} Command}\index{\texttt{open tex} command}
Syntax:  \texttt{open tex} {\em file-name} [\texttt{/no{\char`\_}header}]

This command opens a file for writing \LaTeX\
source\index{latex@{\LaTeX}} and writes a \LaTeX\ header to the file.
\LaTeX\ source can be written with the \texttt{show proof}, \texttt{show
new{\char`\_}proof}, and \texttt{show statement} commands using the
\texttt{/tex} qualifier.

The mapping to \LaTeX\ symbols is defined in a special comment
containing a \texttt{\$t} token, described in Appendix~\ref{ASCII}.

There is an optional command qualifier:

    \texttt{/no{\char`\_}header} - This qualifier prevents a standard
        \LaTeX\ header and trailer
        from being included with the output \LaTeX\ code.


\subsection{\texttt{close tex} Command}\index{\texttt{close tex} command}
Syntax:  \texttt{close tex}

This command writes a trailer to any \LaTeX\ file\index{latex@{\LaTeX}}
that was opened with \texttt{open tex} (unless
\texttt{/no{\char`\_}header} was used with \texttt{open tex}) and closes
the \LaTeX\ file.


\section{Creating {\sc HTML} Output}\label{htmlout}

You can generate {\sc html} web pages given the
information in a database.
The database must already include the necessary typesetting information
(see section \ref{tcomment} for how to provide this information).
The ability to produce {\sc html} web pages was added in Metamath version
0.07.30.

To create an {\sc html} output file(s) for \texttt{\$a} or \texttt{\$p}
statement(s), use
\begin{quote}
    \texttt{show statement} {\em label-match} \texttt{/html}
\end{quote}
The output file will be named {\em label-match}\texttt{.html}
for each match.  When {\em
label-match} has wildcard (\texttt{*}) characters, all statements with
matching labels will have {\sc html} files produced for them.  Also,
when {\em label-match} has a wildcard (\texttt{*}) character, two additional
files, \texttt{mmdefinitions.html} and \texttt{mmascii.html} will be
produced.  To produce {\em only} these two additional files, you can use
\texttt{?*}, which will not match any statement label, in place of {\em
label-match}.

There are three other qualifiers for \texttt{show statement} that also
generate {\sc HTML} code.  These are \texttt{/alt{\char`\_}html},
\texttt{/brief{\char`\_}html}, and
\texttt{/brief{\char`\_}alt{\char`\_}html}, and are described in the
next section.

The command
\begin{quote}
    \texttt{show statement} {\em label-match} \texttt{/alt{\char`\_}html}
\end{quote}
does the same as \texttt{show statement} {\em label-match} \texttt{/html},
except that the {\sc html} code for the symbols is taken from
\texttt{althtmldef} statements instead of \texttt{htmldef} statements in
the \texttt{\$t} comment.

The command
\begin{verbatim}
show statement * /brief_html
\end{verbatim}
invokes a special mode that just produces definition and theorem lists
accompanied by their symbol strings, in a format suitable for copying and
pasting into another web page (such as the tutorial pages on the
Metamath web site).

Finally, the command
\begin{verbatim}
show statement * /brief_alt_html
\end{verbatim}
does the same as \texttt{show statement * / brief{\char`\_}html}
for the alternate {\sc html}
symbol representation.

A statement's comment can include a special notation that provides a
certain amount of control over the {\sc HTML} version of the comment.  See
Section~\ref{mathcomments} (p.~\pageref{mathcomments}) for the comment
markup features.

The \texttt{write theorem{\char`\_}list} and \texttt{write bibliography}
commands, which are described below, provide as a side effect complete
error checking for all of the features described in this
section.\index{error checking}

\subsection{\texttt{write theorem\_list}
Command}\index{\texttt{write theorem{\char`\_}list} command}
Syntax:  \texttt{write theorem{\char`\_}list}
[\texttt{/theorems{\char`\_}per{\char`\_}page} {\em number}]

This command writes a list of all of the \texttt{\$a} and \texttt{\$p}
statements in the database into a web page file
 called \texttt{mmtheorems.html}.
When additional files are needed, they are called
\texttt{mmtheorems2.html}, \texttt{mmtheorems3.html}, etc.

Optional command qualifier:

    \texttt{/theorems{\char`\_}per{\char`\_}page} {\em number} -
 This qualifier specifies the number of statements to
        write per web page.  The default is 100.

{\em Note:} In version 0.177\index{Metamath!limitations of version
0.177} of Metamath, the ``Nearby theorems'' links on the individual
web pages presuppose 100 theorems per page when linking to the theorem
list pages.  Therefore the \texttt{/theorems{\char`\_}per{\char`\_}page}
qualifier, if it specifies a number other than 100, will cause the
individual web pages to be out of sync and should not be used to
generate the main theorem list for the web site.  This may be
fixed in a future version.


\subsection{\texttt{write bibliography}\label{wrbib}
Command}\index{\texttt{write bibliography} command}
Syntax:  \texttt{write bibliography} {\em filename}

This command reads an existing {\sc html} bibliographic cross-reference
file, normally called \texttt{mmbiblio.html}, and updates it per the
bibliographic links in the database comments.  The file is updated
between the {\sc html} comment lines \texttt{<!--
{\char`\#}START{\char`\#} -->} and \texttt{<!-- {\char`\#}END{\char`\#}
-->}.  The original input file is renamed to {\em
filename}\texttt{{\char`\~}1}.

A bibliographic reference is indicated with the reference name
in brackets, such as  \texttt{Theorem 3.1 of
[Monk] p.\ 22}.
See Section~\ref{htmlout} (p.~\pageref{htmlout}) for
syntax details.


\subsection{\texttt{write recent\_additions}
Command}\index{\texttt{write recent{\char`\_}additions} command}
Syntax:  \texttt{write recent{\char`\_}additions} {\em filename}
[\texttt{/limit} {\em number}]

This command reads an existing ``Recent Additions'' {\sc html} file,
normally called \texttt{mmrecent.html}, and updates it with the
descriptions of the most recently added theorems to the database.
 The file is updated between
the {\sc html} comment lines \texttt{<!-- {\char`\#}START{\char`\#} -->}
and \texttt{<!-- {\char`\#}END{\char`\#} -->}.  The original input file
is renamed to {\em filename}\texttt{{\char`\~}1}.

Optional command qualifier:

    \texttt{/limit} {\em number} -
 This qualifier specifies the number of most recent theorems to
   write to the output file.  The default is 100.


\section{Text File Utilities}

\subsection{\texttt{tools} Command}\index{\texttt{tools} command}
Syntax:  \texttt{tools}

This command invokes an easy-to-use, general purpose utility for
manipulating the contents of {\sc ascii} text files.  Upon typing
\texttt{tools}, the command-line prompt will change to \texttt{TOOLS>}
until you type \texttt{exit}.  The \texttt{tools} commands can be used
to perform simple, global edits on an input/output file,
such as making a character string substitution on each line, adding a
string to each line, and so on.  A typical use of this utility is
to build a \texttt{submit} input file to perform a common operation on a
list of statements obtained from \texttt{show label} or \texttt{show
usage}.

The actions of most of the \texttt{tools} commands can also be
performed with equivalent (and more powerful) Unix shell commands, and
some users may find those more efficient.  But for Windows users or
users not comfortable with Unix, \texttt{tools} provides an
easy-to-learn alternative that is adequate for most of the
script-building tasks needed to use the Metamath program effectively.

\subsection{\texttt{help} Command (in \texttt{tools})}
Syntax:  \texttt{help}

The \texttt{help} command lists the commands available in the
\texttt{tools} utility, along with a brief description.  Each command,
in turn, has its own help, such as \texttt{help add}.  As with
Metamath's \texttt{MM>} prompt, a complete command can be entered at
once, or just the command word can be typed, causing the program to
prompt for each argument.

\vskip 1ex
\noindent Line-by-line editing commands:

  \texttt{add} - Add a specified string to each line in a file.

  \texttt{clean} - Trim spaces and tabs on each line in a file; convert
         characters.

  \texttt{delete} - Delete a section of each line in a file.

  \texttt{insert} - Insert a string at a specified column in each line of
        a file.

  \texttt{substitute} - Make a simple substitution on each line of the file.

  \texttt{tag} - Like \texttt{add}, but restricted to a range of lines.

  \texttt{swap} - Swap the two halves of each line in a file.

\vskip 1ex
\noindent Other file-processing commands:

  \texttt{break} - Break up (tokenize) a file into a list of tokens (one per
        line).

  \texttt{build} - Build a file with multiple tokens per line from a list.

  \texttt{count} - Count the occurrences in a file of a specified string.

  \texttt{number} - Create a list of numbers.

  \texttt{parallel} - Put two files in parallel.

  \texttt{reverse} - Reverse the order of the lines in a file.

  \texttt{right} - Right-justify lines in a file (useful before sorting
         numbers).

%  \texttt{tag} - Tag edit updates in a program for revision control.

  \texttt{sort} - Sort the lines in a file with key starting at
         specified string.

  \texttt{match} - Extract lines containing (or not) a specified string.

  \texttt{unduplicate} - Eliminate duplicate occurrences of lines in a file.

  \texttt{duplicate} - Extract first occurrence of any line occurring
         more than

   \ \ \    once in a file, discarding lines occurring exactly once.

  \texttt{unique} - Extract lines occurring exactly once in a file.

  \texttt{type} (10 lines) - Display the first few lines in a file.
                  Similar to Unix \texttt{head}.

  \texttt{copy} - Similar to Unix \texttt{cat} but safe (same input
         and output file allowed).

  \texttt{submit} - Run a script containing \texttt{tools} commands.

\vskip 1ex

\noindent Note:
  \texttt{unduplicate}, \texttt{duplicate}, and \texttt{unique} also
 sort the lines as a side effect.


\subsection{Using \texttt{tools} to Build Metamath \texttt{submit}
Scripts}

The \texttt{break} command is typically used to break up a series of
statement labels, such as the output of Metamath's \texttt{show usage},
into one label per line.  The other \texttt{tools} commands can then be
used to add strings before and after each statement label to specify
commands to be performed on the statement.  The \texttt{parallel}
command is useful when a statement label must be mentioned more than
once on a line.

Very often a \texttt{submit} script for Metamath will require multiple
command lines for each statement being processed.  For example, you may
want to enter the Proof Assistant, \texttt{minimize{\char`\_}with} your
latest theorem, \texttt{save} the new proof, and \texttt{exit} the Proof
Assistant.  To accomplish this, you can build a file with these four
commands for each statement on a single line, separating each command
with a designated character such as \texttt{@}.  Then at the end you can
\texttt{substitute} each \texttt{@} with \texttt{{\char`\\}n} to break
up the lines into individual command lines (see \texttt{help
substitute}).


\subsection{Example of a \texttt{tools} Session}

To give you a quick feel for the \texttt{tools} utility, we show a
simple session where we create a file \texttt{n.txt} with 3 lines, add
strings before and after each line, and display the lines on the screen.
You can experiment with the various commands to gain experience with the
\texttt{tools} utility.

\begin{verbatim}
MM> tools
Entering the Text Tools utilities.
Type HELP for help, EXIT to exit.
TOOLS> number
Output file <n.tmp>? n.txt
First number <1>?
Last number <10>? 3
Increment <1>?
TOOLS> add
Input/output file? n.txt
String to add to beginning of each line <>? This is line
String to add to end of each line <>? .
The file n.txt has 3 lines; 3 were changed.
First change is on line 1:
This is line 1.
TOOLS> type n.txt
This is line 1.
This is line 2.
This is line 3.
TOOLS> exit
Exiting the Text Tools.
Type EXIT again to exit Metamath.
MM>
\end{verbatim}



\appendix
\chapter{Sample Representations}
\label{ASCII}

This Appendix provides a sample of {\sc ASCII} representations,
their corresponding traditional mathematical symbols,
and a discussion of their meanings
in the \texttt{set.mm} database.
These are provided in order of appearance.
This is only a partial list, and new definitions are routinely added.
A complete list is available at \url{http://metamath.org}.

These {\sc ASCII} representations, along
with information on how to display them,
are defined in the \texttt{set.mm} database file inside
a special comment called a \texttt{\$t} {\em
comment}\index{\texttt{\$t} comment} or {\em typesetting
comment.}\index{typesetting comment}
A typesetting comment
is indicated by the appearance of the
two-character string \texttt{\$t} at the beginning of the comment.
For more information,
see Section~\ref{tcomment}, p.~\pageref{tcomment}.

In the following table the ``{\sc ASCII}'' column shows the {\sc ASCII}
representation,
``Symbol'' shows the mathematical symbolic display
that corresponds to that {\sc ASCII} representation, ``Labels'' shows
the key label(s) that define the representation, and
``Description'' provides a description about the symbol.
As usual, ``iff'' is short for ``if and only if.''\index{iff}
In most cases the ``{\sc ASCII}'' column only shows
the key token, but it will sometimes show a sequence of tokens
if that is necessary for clarity.

{\setlength{\extrarowsep}{4pt} % Keep rows from being too close together
\begin{longtabu}   { @{} c c l X }
\textbf{ASCII} & \textbf{Symbol} & \textbf{Labels} & \textbf{Description} \\
\endhead
\texttt{|-} & $\vdash$ & &
  ``It is provable that...'' \\
\texttt{ph} & $\varphi$ & \texttt{wph} &
  The wff (boolean) variable phi,
  conventionally the first wff variable. \\
\texttt{ps} & $\psi$ & \texttt{wps} &
  The wff (boolean) variable psi,
  conventionally the second wff variable. \\
\texttt{ch} & $\chi$ & \texttt{wch} &
  The wff (boolean) variable chi,
  conventionally the third wff variable. \\
\texttt{-.} & $\lnot$ & \texttt{wn} &
  Logical not. E.g., if $\varphi$ is true, then $\lnot \varphi$ is false. \\
\texttt{->} & $\rightarrow$ & \texttt{wi} &
  Implies, also known as material implication.
  In classical logic the expression $\varphi \rightarrow \psi$ is true
  if either $\varphi$ is false or $\psi$ is true (or both), that is,
  $\varphi \rightarrow \psi$ has the same meaning as
  $\lnot \varphi \lor \psi$ (as proven in theorem \texttt{imor}). \\
\texttt{<->} & $\leftrightarrow$ &
  \hyperref[df-bi]{\texttt{df-bi}} &
  Biconditional (aka is-equals for boolean values).
  $\varphi \leftrightarrow \psi$ is true iff
  $\varphi$ and $\psi$ have the same value. \\
\texttt{\char`\\/} & $\lor$ &
  \makecell[tl]{{\hyperref[df-or]{\texttt{df-or}}}, \\
	         \hyperref[df-3or]{\texttt{df-3or}}} &
  Disjunction (logical ``or''). $\varphi \lor \psi$ is true iff
  $\varphi$, $\psi$, or both are true. \\
\texttt{/\char`\\} & $\land$ &
  \makecell[tl]{{\hyperref[df-an]{\texttt{df-an}}}, \\
                 \hyperref[df-3an]{\texttt{df-3an}}} &
  Conjunction (logical ``and''). $\varphi \land \psi$ is true iff
  both $\varphi$ and $\psi$ are true. \\
\texttt{A.} & $\forall$ &
  \texttt{wal} &
  For all; the wff $\forall x \varphi$ is true iff
  $\varphi$ is true for all values of $x$. \\
\texttt{E.} & $\exists$ &
  \hyperref[df-ex]{\texttt{df-ex}} &
  There exists; the wff
  $\exists x \varphi$ is true iff
  there is at least one $x$ where $\varphi$ is true. \\
\texttt{[ y / x ]} & $[ y / x ]$ &
  \hyperref[df-sb]{\texttt{df-sb}} &
  The wff $[ y / x ] \varphi$ produces
  the result when $y$ is properly substituted for $x$ in $\varphi$
  ($y$ replaces $x$).
  % This is elsb4
  % ( [ x / y ] z e. y <-> z e. x )
  For example,
  $[ x / y ] z \in y$ is the same as $z \in x$. \\
\texttt{E!} & $\exists !$ &
  \hyperref[df-eu]{\texttt{df-eu}} &
  There exists exactly one;
  $\exists ! x \varphi$ is true iff
  there is at least one $x$ where $\varphi$ is true. \\
\texttt{\{ y | phi \}}  & $ \{ y | \varphi \}$ &
  \hyperref[df-clab]{\texttt{df-clab}} &
  The class of all sets where $\varphi$ is true. \\
\texttt{=} & $ = $ &
  \hyperref[df-cleq]{\texttt{df-cleq}} &
  Class equality; $A = B$ iff $A$ equals $B$. \\
\texttt{e.} & $ \in $ &
  \hyperref[df-clel]{\texttt{df-clel}} &
  Class membership; $A \in B$ if $A$ is a member of $B$. \\
\texttt{{\char`\_}V} & {\rm V} &
  \hyperref[df-v]{\texttt{df-v}} &
  Class of all sets (not itself a set). \\
\texttt{C\_} & $ \subseteq $ &
  \hyperref[df-ss]{\texttt{df-ss}} &
  Subclass (subset); $A \subseteq B$ is true iff
  $A$ is a subclass of $B$. \\
\texttt{u.} & $ \cup $ &
  \hyperref[df-un]{\texttt{df-un}} &
  $A \cup B$ is the union of classes $A$ and $B$. \\
\texttt{i^i} & $ \cap $ &
  \hyperref[df-in]{\texttt{df-in}} &
  $A \cap B$ is the intersection of classes $A$ and $B$. \\
\texttt{\char`\\} & $ \setminus $ &
  \hyperref[df-dif]{\texttt{df-dif}} &
  $A \setminus B$ (set difference)
  is the class of all sets in $A$ except for those in $B$. \\
\texttt{(/)} & $ \varnothing $ &
  \hyperref[df-nul]{\texttt{df-nul}} &
  $ \varnothing $ is the empty set (aka null set). \\
\texttt{\char`\~P} & $ \cal P $ &
  \hyperref[df-pw]{\texttt{df-pw}} &
  Power class. \\
\texttt{<.\ A , B >.} & $\langle A , B \rangle$ &
  \hyperref[df-op]{\texttt{df-op}} &
  The ordered pair $\langle A , B \rangle$. \\
\texttt{( F ` A )} & $ ( F ` A ) $ &
  \hyperref[df-fv]{\texttt{df-fv}} &
  The value of function $F$ when applied to $A$. \\
\texttt{\_i} & $ i $ &
  \texttt{df-i} &
  The square root of negative one. \\
\texttt{x.} & $ \cdot $ &
  \texttt{df-mul} &
  Complex number multiplication; $2~\cdot~3~=~6$. \\
\texttt{CC} & $ \mathbb{C} $ &
  \texttt{df-c} &
  The set of complex numbers. \\
\texttt{RR} & $ \mathbb{R} $ &
  \texttt{df-r} &
  The set of real numbers. \\
\end{longtabu}
} % end of extrarowsep

\chapter{Compressed Proofs}
\label{compressed}\index{compressed proof}\index{proof!compressed}

The proofs in the \texttt{set.mm} set theory database are stored in compressed
format for efficiency.  Normally you needn't concern yourself with the
compressed format, since you can display it with the usual proof display tools
in the Metamath program (\texttt{show proof}\ldots) or convert it to the normal
RPN proof format described in Section~\ref{proof} (with \texttt{save proof}
{\em label} \texttt{/normal}).  However for sake of completeness we describe the
format here and show how it maps to the normal RPN proof format.

A compressed proof, located between \texttt{\$=} and \texttt{\$.}\ keywords, consists
of a left parenthesis, a sequence of statement labels, a right parenthesis,
and a sequence of upper-case letters \texttt{A} through \texttt{Z} (with optional
white space between them).  White space must surround the parentheses
and the labels.  The left parenthesis tells Metamath that a
compressed proof follows.  (A normal RPN proof consists of just a sequence of
labels, and a parenthesis is not a legal character in a label.)

The sequence of upper-case letters corresponds to a sequence of integers
with the following mapping.  Each integer corresponds to a proof step as
described later.
\begin{center}
  \texttt{A} = 1 \\
  \texttt{B} = 2 \\
   \ldots \\
  \texttt{T} = 20 \\
  \texttt{UA} = 21 \\
  \texttt{UB} = 22 \\
   \ldots \\
  \texttt{UT} = 40 \\
  \texttt{VA} = 41 \\
  \texttt{VB} = 42 \\
   \ldots \\
  \texttt{YT} = 120 \\
  \texttt{UUA} = 121 \\
   \ldots \\
  \texttt{YYT} = 620 \\
  \texttt{UUUA} = 621 \\
   etc.
\end{center}

In other words, \texttt{A} through \texttt{T} represent the
least-significant digit in base 20, and \texttt{U} through \texttt{Y}
represent zero or more most-significant digits in base 5, where the
digits start counting at 1 instead of the usual 0. With this scheme, we
don't need white space between these ``numbers.''

(In the design of the compressed proof format, only upper-case letters,
as opposed to say all non-whitespace printable {\sc ascii} characters other than
%\texttt{\$}, was chosen to make the compressed proof a little less
%displeasing to the eye, at the expense of a typical 20\% compression
\texttt{\$}, were chosen so as not to collide with most text editor
searches, at the expense of a typical 20\% compression
loss.  The base 5/base 20 grouping, as opposed to say base 6/base 19,
was chosen by experimentally determining the grouping that resulted in
best typical compression.)

The letter \texttt{Z} identifies (tags) a proof step that is identical to one
that occurs later on in the proof; it helps shorten the proof by not requiring
that identical proof steps be proved over and over again (which happens often
when building wff's).  The \texttt{Z} is placed immediately after the
least-significant digit (letters \texttt{A} through \texttt{T}) that ends the integer
corresponding to the step to later be referenced.

The integers that the upper-case letters correspond to are mapped to labels as
follows.  If the statement being proved has $m$ mandatory hypotheses, integers
1 through $m$ correspond to the labels of these hypotheses in the order shown
by the \texttt{show statement ... / full} command, i.e., the RPN order\index{RPN
order} of the mandatory
hypotheses.  Integers $m+1$ through $m+n$ correspond to the labels enclosed in
the parentheses of the compressed proof, in the order that they appear, where
$n$ is the number of those labels.  Integers $m+n+1$ on up don't directly
correspond to statement labels but point to proof steps identified with the
letter \texttt{Z}, so that these proof steps can be referenced later in the
proof.  Integer $m+n+1$ corresponds to the first step tagged with a \texttt{Z},
$m+n+2$ to the second step tagged with a \texttt{Z}, etc.  When the compressed
proof is converted to a normal proof, the entire subproof of a step tagged
with \texttt{Z} replaces the reference to that step.

For efficiency, Metamath works with compressed proofs directly, without
converting them internally to normal proofs.  In addition to the usual
error-checking, an error message is given if (1) a label in the label list in
parentheses does not refer to a previous \texttt{\$p} or \texttt{\$a} statement or a
non-mandatory hypothesis of the statement being proved and (2) a proof step
tagged with \texttt{Z} is referenced before the step tagged with the \texttt{Z}.

Just as in a normal proof under development (Section~\ref{unknown}), any step
or subproof that is not yet known may be represented with a single \texttt{?}.
White space does not have to appear between the \texttt{?}\ and the upper-case
letters (or other \texttt{?}'s) representing the remainder of the proof.

% April 1, 2004 Appendix C has been added back in with corrections.
%
% May 20, 2003 Appendix C was removed for now because there was a problem found
% by Bob Solovay
%
% Also, removed earlier \ref{formalspec} 's (3 cases above)
%
% Bob Solovay wrote on 30 Nov 2002:
%%%%%%%%%%%%% (start of email comment )
%      3. My next set of comments concern appendix C. I read this before I
% read Chapter 4. So I first noted that the system as presented in the
% Appendix lacked a certain formal property that I thought desirable. I
% then came up with a revised formal system that had this property. Upon
% reading Chapter 4, I noticed that the revised system was closer to the
% treatment in Chapter 4 than the system in Appendix C.
%
%         First a very minor correction:
%
%         On page 142 line 2: The condition that V(e) != V(f) should only be
% required of e, f in T such that e != f.
%
%         Here is a natural property [transitivity] that one would like
% the formal system to have:
%
%         Let Gamma be a set of statements. Suppose that the statement Phi
% is provable from Gamma and that the statement Psi is provable from Gamma
% \cup {Phi}. Then Psi is provable from Gamma.
%
%         I shall present an example to show that this property does not
% hold for the formal systems of Appendix C:
%
%         I write the example in metamath style:
%
% $c A B C D E $.
% $v x y
%
% ${
% tx $f A x $.
% ty $f B y $.
% ax1 $a C x y $.
% $}
%
% ${
% tx $f A x $.
% ty $f B y $.
% ax2-h1 $e C x y $.
% ax2 $a D y $.
% $}
%
% ${
% ty $f B y $.
% ax3-h1 $e D y $.
% ax3 $a E y $.
% $}
%
% $(These three axioms are Gamma $)
%
% ${
% tx $f A x $.
% ty $f B y $.
% Phi $p D y $=
% tx ty tx ty ax1 ax2 $.
% $}
%
% ${
% ty $f B y $.
% Psi $p E y $=
% ty ty Phi ax3 $.
% $}
%
%
% I omit the formal proofs of the following claims. [I will be glad to
% supply them upon request.]
%
% 1) Psi is not provable from Gamma;
%
% 2) Psi is provable from Gamma + Phi.
%
% Here "provable" refers to the formalism of Appendix C.
%
% The trouble of course is that Psi is lacking the variable declaration
%
% $f Ax $.
%
% In the Metamath system there is no trouble proving Psi. I attach a
% metamath file that shows this and which has been checked by the
% metamath program.
%
% I next want to indicate how I think the treatment in Appendix C should
% be revised so as to conform more closely to the metamath system of the
% main text. The revised system *does* have the transitivity property.
%
% We want to give revised definitions of "statement" and
% "provable". [cf. sections C.2.4. and C.2.5] Our new definitions will
% use the definitions given in Appendix C. So we take the following
% tack. We refer to the original notions as o-statement and o-provable. And
% we refer to the notions we are defining as n-statement and n-provable.
%
%         A n-statement is an o-statement in which the only variables
% that appear in the T component are mandatory.
%
%         To any o-statement we can associate its reduct which is a
% n-statement by dropping all the elements of T or D which contain
% non-mandatory variables.
%
%         An n-statement gamma is n-provable if there is an o-statement
% gamma' which has gamma as its reduct andf such that gamma' is
% o-provable.
%
%         It seems to me [though I am not completely sure on this point]
% that n-provability corresponds to metamath provability as discussed
% say in Chapter 4.
%
%         Attached to this letter is the metamath proof of Phi and Psi
% from Gamma discussed above.
%
%         I am still brooding over the question of whether metamath
% correctly formalizes set-theory. No doubt I will have some questions
% re this after my thoughts become clearer.
%%%%%%%%%%%%%%%% (end of email comment)

%%%%%%%%%%%%%%%% (start of 2nd email comment from Bob Solovay 1-Apr-04)
%
%         I hope that Appendix C is the one that gives a "formal" treatment
% of Metamath. At any rate, thats the appendix I want to comment on.
%
%         I'm going to suggest two changes in the definition.
%
%         First change (in the definition of statement): Require that the
% sets D, T, and E be finite.
%
%         Probably things are fine as you give them. But in the applications
% to the main metamath system they will always be finite, and its useful in
% thinking about things [at least for me] to stick to the finite case.
%
%         Second change:
%
%         First let me give an approximate description. Remove the dummy
% variables from the statement. Instead, include them in the proof.
%
%         More formally: Require that T consists of type declarations only
% for mandatory variables. Require that all the pairs in D consist of
% mandatory variables.
%
%         At the start of a proof we are allowed to declare a finite number
% of dummy variables [provided that none of them appear in any of the
% statements in E \cup {A}. We have to supply type declarations for all the
% dummy variables. We are allowed to add new $d statements referring to
% either the mandatory or dummy variables. But we require that no new $d
% statement references only mandatory variables.
%
%         I find this way of doing things more conceptual than the treatment
% in Appendix C. But the change [which I will use implicitly in later
% letters about doing Peano] is mainly aesthetic. I definitely claim that my
% results on doing Peano all apply to Metamath as it is presented in your
% book.
%
%         --Bob
%
%%%%%%%%%%%%%%%% (end of 2nd email comment)

%%
%% When uncommenting the below, also uncomment references above to {formalspec}
%%
\chapter{Metamath's Formal System}\label{formalspec}\index{Metamath!as a formal
system}

\section{Introduction}

\begin{quote}
  {\em Perfection is when there is no longer anything more to take away.}
    \flushright\sc Antoine de
     Saint-Exupery\footnote{\cite[p.~3-25]{Campbell}.}\\
\end{quote}\index{de Saint-Exupery, Antoine}

This appendix describes the theory behind the Metamath language in an abstract
way intended for mathematicians.  Specifically, we construct two
set-theo\-ret\-i\-cal objects:  a ``formal system'' (roughly, a set of syntax
rules, axioms, and logical rules) and its ``universe'' (roughly, the set of
theorems derivable in the formal system).  The Metamath computer language
provides us with a way to describe specific formal systems and, with the aid of
a proof provided by the user, to verify that given theorems
belong to their universes.

To understand this appendix, you need a basic knowledge of informal set theory.
It should be sufficient to understand, for example, Ch.\ 1 of Munkres' {\em
Topology} \cite{Munkres}\index{Munkres, James R.} or the
introductory set theory chapter
in many textbooks that introduce abstract mathematics. (Note that there are
minor notational differences among authors; e.g.\ Munkres uses $\subset$ instead
of our $\subseteq$ for ``subset.''  We use ``included in'' to mean ``a subset
of,'' and ``belongs to'' or ``is contained in'' to mean ``is an element of.'')
What we call a ``formal'' description here, unlike earlier, is actually an
informal description in the ordinary language of mathematicians.  However we
provide sufficient detail so that a mathematician could easily formalize it,
even in the language of Metamath itself if desired.  To understand the logic
examples at the end of this appendix, familiarity with an introductory book on
mathematical logic would be helpful.

\section{The Formal Description}

\subsection[Preliminaries]{Preliminaries\protect\footnotemark}%
\footnotetext{This section is taken mostly verbatim
from Tarski \cite[p.~63]{Tarski1965}\index{Tarski, Alfred}.}

By $\omega$ we denote the set of all natural numbers (non-negative integers).
Each natural number $n$ is identified with the set of all smaller numbers: $n =
\{ m | m < n \}$.  The formula $m < n$ is thus equivalent to the condition: $m
\in n$ and $m,n \in \omega$. In particular, 0 is the number zero and at the
same time the empty set $\varnothing$, $1=\{0\}$, $2=\{0,1\}$, etc. ${}^B A$
denotes the set of all functions on $B$ to $A$ (i.e.\ with domain $B$ and range
included in $A$).  The members of ${}^\omega A$ are what are called {\em simple
infinite sequences},\index{simple infinite sequence}
with all {\em terms}\index{term} in $A$.  In case $n \in \omega$, the
members of ${}^n A$ are referred to as {\em finite $n$-termed
sequences},\index{finite $n$-termed
sequence} again
with terms in $A$.  The consecutive terms (function values) of a finite or
infinite sequence $f$ are denoted by $f_0, f_1, \ldots ,f_n,\ldots$.  Every
finite sequence $f \in \bigcup _{n \in \omega} {}^n A$ uniquely determines the
number $n$ such that $f \in {}^n A$; $n$ is called the {\em
length}\index{length of a sequence ({$"|\ "|$})} of $f$ and
is denoted by $|f|$.  $\langle a \rangle$ is the sequence $f$ with $|f|=1$ and
$f_0=a$; $\langle a,b \rangle$ is the sequence $f$ with $|f|=2$, $f_0=a$,
$f_1=b$; etc.  Given two finite sequences $f$ and $g$, we denote by $f\frown g$
their {\em concatenation},\index{concatenation} i.e., the
finite sequence $h$ determined by the
conditions:
\begin{eqnarray*}
& |h| = |f|+|g|;&  \\
& h_n = f_n & \mbox{\ for\ } n < |f|;  \\
& h_{|f|+n} = g_n & \mbox{\ for\ } n < |g|.
\end{eqnarray*}

\subsection{Constants, Variables, and Expressions}

A formal system has a set of {\em symbols}\index{symbol!in
a formal system} denoted
by $\mbox{\em SM}$.  A
precise set-theo\-ret\-i\-cal definition of this set is unimportant; a symbol
could be considered a primitive or atomic element if we wish.  We assume this
set is divided into two disjoint subsets:  a set $\mbox{\em CN}$ of {\em
constants}\index{constant!in a formal system} and a set $\mbox{\em VR}$ of
{\em variables}.\index{variable!in a formal system}  $\mbox{\em CN}$ and
$\mbox{\em VR}$ are each assumed to consist of countably many symbols which
may be arranged in finite or simple infinite sequences $c_0, c_1, \ldots$ and
$v_0, v_1, \ldots$ respectively, without repeating terms.  We will represent
arbitrary symbols by metavariables $\alpha$, $\beta$, etc.

{\footnotesize\begin{quotation}
{\em Comment.} The variables $v_0, v_1, \ldots$ of our formal system
correspond to what are usually considered ``metavariables'' in
descriptions of specific formal systems in the literature.  Typically,
when describing a specific formal system a book will postulate a set of
primitive objects called variables, then proceed to describe their
properties using metavariables that range over them, never mentioning
again the actual variables themselves.  Our formal system does not
mention these primitive variable objects at all but deals directly with
metavariables, as its primitive objects, from the start.  This is a
subtle but key distinction you should keep in mind, and it makes our
definition of ``formal system'' somewhat different from that typically
found in the literature.  (So, the $\alpha$, $\beta$, etc.\ above are
actually ``metametavariables'' when used to represent $v_0, v_1,
\ldots$.)
\end{quotation}}

Finite sequences all terms of which are symbols are called {\em
expressions}.\index{expression!in a formal system}  $\mbox{\em EX}$ is
the set of all expressions; thus
\begin{displaymath}
\mbox{\em EX} = \bigcup _{n \in \omega} {}^n \mbox{\em SM}.
\end{displaymath}

A {\em constant-prefixed expression}\index{constant-prefixed expression}
is an expression of non-zero length
whose first term is a constant.  We denote the set of all constant-prefixed
expressions by $\mbox{\em EX}_C = \{ e \in \mbox{\em EX} | ( |e| > 0 \wedge
e_0 \in \mbox{\em CN} ) \}$.

A {\em constant-variable pair}\index{constant-variable pair}
is an expression of length 2 whose first term
is a constant and whose second term is a variable.  We denote the set of all
constant-variable pairs by $\mbox{\em EX}_2 = \{ e \in \mbox{\em EX}_C | ( |e|
= 2 \wedge e_1 \in \mbox{\em VR} ) \}$.


{\footnotesize\begin{quotation}
{\em Relationship to Metamath.} In general, the set $\mbox{\em SM}$
corresponds to the set of declared math symbols in a Metamath database, the
set $\mbox{\em CN}$ to those declared with \texttt{\$c} statements, and the set
$\mbox{\em VR}$ to those declared with \texttt{\$v} statements.  Of course a
Metamath database can only have a finite number of math symbols, whereas
formal systems in general can have an infinite number, although the number of
Metamath math symbols available is in principle unlimited.

The set $\mbox{\em EX}_C$ corresponds to the set of permissible expressions
for \texttt{\$e}, \texttt{\$a}, and \texttt{\$p} statements.  The set $\mbox{\em EX}_2$
corresponds to the set of permissible expressions for \texttt{\$f} statements.
\end{quotation}}

We denote by ${\cal V}(e)$ the set of all variables in an expression $e \in
\mbox{\em EX}$, i.e.\ the set of all $\alpha \in \mbox{\em VR}$ such that
$\alpha = e_n$ for some $n < |e|$.  We also denote (with abuse of notation) by
${\cal V}(E)$ the set of all variables in a collection of expressions $E
\subseteq \mbox{\em EX}$, i.e.\ $\bigcup _{e \in E} {\cal V}(e)$.


\subsection{Substitution}

Given a function $F$ from $\mbox{\em VR}$ to
$\mbox{\em EX}$, we
denote by $\sigma_{F}$ or just $\sigma$ the function from $\mbox{\em EX}$ to
$\mbox{\em EX}$ defined recursively for nonempty sequences by
\begin{eqnarray*}
& \sigma(<\alpha>) = F(\alpha) & \mbox{for\ } \alpha \in \mbox{\em VR}; \\
& \sigma(<\alpha>) = <\alpha> & \mbox{for\ } \alpha \not\in \mbox{\em VR}; \\
& \sigma(g \frown h) = \sigma(g) \frown
    \sigma(h) & \mbox{for\ } g,h \in \mbox{\em EX}.
\end{eqnarray*}
We also define $\sigma(\varnothing)=\varnothing$.  We call $\sigma$ a {\em
simultaneous substitution}\index{substitution!variable}\index{variable
substitution} (or just {\em substitution}) with {\em substitution
map}\index{substitution map} $F$.

We also denote (with abuse of notation) by $\sigma(E)$ a substitution on a
collection of expressions $E \subseteq \mbox{\em EX}$, i.e.\ the set $\{
\sigma(e) | e \in E \}$.  The collection $\sigma(E)$ may of course contain
fewer expressions than $E$ because duplicate expressions could result from the
substitution.

\subsection{Statements}

We denote by $\mbox{\em DV}$ the set of all
unordered pairs $\{\alpha, \beta \} \subseteq \mbox{\em VR}$ such that $\alpha
\neq \beta$.  $\mbox{\em DV}$ stands for ``distinct variables.''

A {\em pre-statement}\index{pre-statement!in a formal system} is a
quadruple $\langle D,T,H,A \rangle$ such that
$D\subseteq \mbox{\em DV}$, $T\subseteq \mbox{\em EX}_2$, $H\subseteq
\mbox{\em EX}_C$ and $H$ is finite,
$A\in \mbox{\em EX}_C$, ${\cal V}(H\cup\{A\}) \subseteq
{\cal V}(T)$, and $\forall e,f\in T {\ } {\cal V}(e) \neq {\cal V}(f)$ (or
equivalently, $e_1 \ne f_1$) whenever $e \neq f$. The terms of the quadruple are called {\em
distinct-variable restrictions},\index{disjoint-variable restriction!in a
formal system} {\em variable-type hypotheses},\index{variable-type
hypothesis!in a formal system} {\em logical hypotheses},\index{logical
hypothesis!in a formal system} and the {\em assertion}\index{assertion!in a
formal system} respectively.  We denote by $T_M$ ({\em mandatory variable-type
hypotheses}\index{mandatory variable-type hypothesis!in a formal system}) the
subset of $T$ such that ${\cal V}(T_M) ={\cal V}(H \cup \{A\})$.  We denote by
$D_M=\{\{\alpha,\beta\}\in D|\{\alpha,\beta\}\subseteq {\cal V}(T_M)\}$ the
{\em mandatory distinct-variable restrictions}\index{mandatory
disjoint-variable restriction!in a formal system} of the pre-statement.
The set
of {\em mandatory hypotheses}\index{mandatory hypothesis!in a formal system}
is $T_M\cup H$.  We call the quadruple $\langle D_M,T_M,H,A \rangle$
the {\em reduct}\index{reduct!in a formal system} of
the pre-statement $\langle D,T,H,A \rangle$.

A {\em statement} is the reduct of some pre-statement\index{statement!in a
formal system}.  A statement is therefore a special kind of pre-statement;
in particular, a statement is the reduct of itself.

{\footnotesize\begin{quotation}
{\em Comment.}  $T$ is a set of expressions, each of length 2, that associate
a set of constants (``variable types'') with a set of variables.  The
condition ${\cal V}(H\cup\{A\}) \subseteq {\cal V}(T) $
means that each variable occurring in a statement's logical
hypotheses or assertion must have an associated variable-type hypothesis or
``type declaration,'' in  analogy to a computer programming language, where a
variable must be declared to be say, a string or an integer.  The requirement
that $\forall e,f\in T \, e_1 \ne f_1$ for $e\neq f$
means that each variable must be
associated with a unique constant designating its variable type; e.g., a
variable might be a ``wff'' or a ``set'' but not both.

Distinct-variable restrictions are used to specify what variable substitutions
are permissible to make for the statement to remain valid.  For example, in
the theorem scheme of set theory $\lnot\forall x\,x=y$ we may not substitute
the same variable for both $x$ and $y$.  On the other hand, the theorem scheme
$x=y\to y=x$ does not require that $x$ and $y$ be distinct, so we do not
require a distinct-variable restriction, although having one
would cause no harm other than making the scheme less general.

A mandatory variable-type hypothesis is one whose variable exists in a logical
hypothesis or the assertion.  A provable pre-statement
(defined below) may require
non-mandatory variable-type hypotheses that effectively introduce ``dummy''
variables for use in its proof.  Any number of dummy variables might
be required by a specific proof; indeed, it has been shown by H.\
Andr\'{e}ka\index{Andr{\'{e}}ka, H.} \cite{Nemeti} that there is no finite
upper bound to the number of dummy variables needed to prove an arbitrary
theorem in first-order logic (with equality) having a fixed number $n>2$ of
individual variables.  (See also the Comment on p.~\pageref{nodd}.)
For this reason we do not set a finite size bound on the collections $D$ and
$T$, although in an actual application (Metamath database) these will of
course be finite, increased to whatever size is necessary as more
proofs are added.
\end{quotation}}

{\footnotesize\begin{quotation}
{\em Relationship to Metamath.} A pre-statement of a formal system
corresponds to an extended frame in a Metamath database
(Section~\ref{frames}).  The collections $D$, $T$, and $H$ correspond
respectively to the \texttt{\$d}, \texttt{\$f}, and \texttt{\$e}
statement collections in an extended frame.  The expression $A$
corresponds to the \texttt{\$a} (or \texttt{\$p}) statement in an
extended frame.

A statement of a formal system corresponds to a frame in a Metamath
database.
\end{quotation}}

\subsection{Formal Systems}

A {\em formal system}\index{formal system} is a
triple $\langle \mbox{\em CN},\mbox{\em
VR},\Gamma\rangle$ where $\Gamma$ is a set of statements.  The members of
$\Gamma$ are called {\em axiomatic statements}.\index{axiomatic
statement!in a formal system}  Sometimes we will refer to a
formal system by just $\Gamma$ when $\mbox{\em CN}$ and $\mbox{\em VR}$ are
understood.

Given a formal system $\Gamma$, the {\em closure}\index{closure}\footnote{This
definition of closure incorporates a simplification due to
Josh Purinton.\index{Purinton, Josh}.} of a
pre-statement
$\langle D,T,H,A \rangle$ is the smallest set $C$ of expressions
such that:
%\begin{enumerate}
%  \item $T\cup H\subseteq C$; and
%  \item If for some axiomatic statement
%    $\langle D_M',T_M',H',A' \rangle \in \Gamma_A$, for
%    some $E \subseteq C$, some $F \subseteq C-T$ (where ``-'' denotes
%    set difference), and some substitution
%    $\sigma$ we have
%    \begin{enumerate}
%       \item $\sigma(T_M') = E$ (where, as above, the $M$ denotes the
%           mandatory variable-type hypotheses of $T^A$);
%       \item $\sigma(H') = F$;
%       \item for all $\{\alpha,\beta\}\in D^A$ and $\subseteq
%         {\cal V}(T_M')$, for all $\gamma\in {\cal V}(\sigma(\langle \alpha
%         \rangle))$, and for all $\delta\in  {\cal V}(\sigma(\langle \beta
%         \rangle))$, we have $\{\gamma, \delta\} \in D$;
%   \end{enumerate}
%   then $\sigma(A') \in C$.
%\end{enumerate}
\begin{list}{}{\itemsep 0.0pt}
  \item[1.] $T\cup H\subseteq C$; and
  \item[2.] If for some axiomatic statement
    $\langle D_M',T_M',H',A' \rangle \in
       \Gamma$ and for some substitution
    $\sigma$ we have
    \begin{enumerate}
       \item[a.] $\sigma(T_M' \cup H') \subseteq C$; and
       \item[b.] for all $\{\alpha,\beta\}\in D_M'$, for all $\gamma\in
         {\cal V}(\sigma(\langle \alpha
         \rangle))$, and for all $\delta\in  {\cal V}(\sigma(\langle \beta
         \rangle))$, we have $\{\gamma, \delta\} \in D$;
   \end{enumerate}
   then $\sigma(A') \in C$.
\end{list}
A pre-statement $\langle D,T,H,A
\rangle$ is {\em provable}\index{provable statement!in a formal
system} if $A\in C$ i.e.\ if its assertion belongs to its
closure.  A statement is {\em provable} if it is
the reduct of a provable pre-statement.
The {\em universe}\index{universe of a formal system}
of a formal system is
the collection of all of its provable statements.  Note that the
set of axiomatic statements $\Gamma$ in a formal system is a subset of its
universe.

{\footnotesize\begin{quotation}
{\em Comment.} The first condition in the definition of closure simply says
that the hypotheses of the pre-statement are in its closure.

Condition 2(a) says that a substitution exists that makes the
mandatory hypotheses of an axiomatic statement exactly match some members of
the closure.  This is what we explicitly demonstrate in a Metamath language
proof.

%Conditions 2(a) and 2(b) say that a substitution exists that makes the
%(mandatory) hypotheses of an axiomatic statement exactly match some members of
%the closure.  This is what we explicitly demonstrate with a Metamath language
%proof.
%
%The set of expressions $F$ in condition 2(b) excludes the variable-type
%hypotheses; this is done because non-mandatory variable-type hypotheses are
%effectively ``dropped'' as irrelevant whereas logical hypotheses must be
%retained to achieve a consistent logical system.

Condition 2(b) describes how distinct-variable restrictions in the axiomatic
statement must be met.  It means that after a substitution for two variables
that must be distinct, the resulting two expressions must either contain no
variables, or if they do, they may not have variables in common, and each pair
of any variables they do have, with one variable from each expression, must be
specified as distinct in the original statement.
\end{quotation}}

{\footnotesize\begin{quotation}
{\em Relationship to Metamath.} Axiomatic statements
 and provable statements in a formal
system correspond to the frames for \texttt{\$a} and \texttt{\$p} statements
respectively in a Metamath database.  The set of axiomatic statements is a
subset of the set of provable statements in a formal system, although in a
Metamath database a \texttt{\$a} statement is distinguished by not having a
proof.  A Metamath language proof for a \texttt{\$p} statement tells the computer
how to explicitly construct a series of members of the closure ultimately
leading to a demonstration that the assertion
being proved is in the closure.  The actual closure typically contains
an infinite number of expressions.  A formal system itself does not have
an explicit object called a ``proof'' but rather the existence of a proof
is implied indirectly by membership of an assertion in a provable
statement's closure.  We do this to make the formal system easier
to describe in the language of set theory.

We also note that once established as provable, a statement may be considered
to acquire the same status as an axiomatic statement, because if the set of
axiomatic statements is extended with a provable statement, the universe of
the formal system remains unchanged (provided that $\mbox{\em VR}$ is
infinite).
In practice, this means we can build a hierarchy of provable statements to
more efficiently establish additional provable statements.  This is
what we do in Metamath when we allow proofs to reference previous
\texttt{\$p} statements as well as previous \texttt{\$a} statements.
\end{quotation}}

\section{Examples of Formal Systems}

{\footnotesize\begin{quotation}
{\em Relationship to Metamath.} The examples in this section, except Example~2,
are for the most part exact equivalents of the development in the set
theory database \texttt{set.mm}.  You may want to compare Examples~1, 3, and 5
to Section~\ref{metaaxioms}, Example 4 to Sections~\ref{metadefprop} and
\ref{metadefpred}, and Example 6 to
Section~\ref{setdefinitions}.\label{exampleref}
\end{quotation}}

\subsection{Example~1---Propositional Calculus}\index{propositional calculus}

Classical propositional calculus can be described by the following formal
system.  We assume the set of variables is infinite.  Rather than denoting the
constants and variables by $c_0, c_1, \ldots$ and $v_0, v_1, \ldots$, for
readability we will instead use more conventional symbols, with the
understanding of course that they denote distinct primitive objects.
Also for readability we may omit commas between successive terms of a
sequence; thus $\langle \mbox{wff\ } \varphi\rangle$ denotes
$\langle \mbox{wff}, \varphi\rangle$.

Let
\begin{itemize}
  \item[] $\mbox{\em CN}=\{\mbox{wff}, \vdash, \to, \lnot, (,)\}$
  \item[] $\mbox{\em VR}=\{\varphi,\psi,\chi,\ldots\}$
  \item[] $T = \{\langle \mbox{wff\ } \varphi\rangle,
             \langle \mbox{wff\ } \psi\rangle,
             \langle \mbox{wff\ } \chi\rangle,\ldots\}$, i.e.\ those
             expressions of length 2 whose first member is $\mbox{\rm wff}$
             and whose second member belongs to $\mbox{\em VR}$.\footnote{For
convenience we let $T$ be an infinite set; the definition of a statement
permits this in principle.  Since a Metamath source file has a finite size, in
practice we must of course use appropriate finite subsets of this $T$,
specifically ones containing at least the mandatory variable-type
hypotheses.  Similarly, in the source file we introduce new variables as
required, with the understanding that a potentially infinite number of
them are available.}
\noindent Then $\Gamma$ consists of the axiomatic statements that
are the reducts of the following pre-statements:
    \begin{itemize}
      \item[] $\langle\varnothing,T,\varnothing,
               \langle \mbox{wff\ }(\varphi\to\psi)\rangle\rangle$
      \item[] $\langle\varnothing,T,\varnothing,
               \langle \mbox{wff\ }\lnot\varphi\rangle\rangle$
      \item[] $\langle\varnothing,T,\varnothing,
               \langle \vdash(\varphi\to(\psi\to\varphi))
               \rangle\rangle$
      \item[] $\langle\varnothing,T,
               \varnothing,
               \langle \vdash((\varphi\to(\psi\to\chi))\to
               ((\varphi\to\psi)\to(\varphi\to\chi)))
               \rangle\rangle$
      \item[] $\langle\varnothing,T,
               \varnothing,
               \langle \vdash((\lnot\varphi\to\lnot\psi)\to
               (\psi\to\varphi))\rangle\rangle$
      \item[] $\langle\varnothing,T,
               \{\langle\vdash(\varphi\to\psi)\rangle,
                 \langle\vdash\varphi\rangle\},
               \langle\vdash\psi\rangle\rangle$
    \end{itemize}
\end{itemize}

(For example, the reduct of $\langle\varnothing,T,\varnothing,
               \langle \mbox{wff\ }(\varphi\to\psi)\rangle\rangle$
is
\begin{itemize}
\item[] $\langle\varnothing,
\{\langle \mbox{wff\ } \varphi\rangle,
             \langle \mbox{wff\ } \psi\rangle\},
             \varnothing,
               \langle \mbox{wff\ }(\varphi\to\psi)\rangle\rangle$,
\end{itemize}
which is the first axiomatic statement.)

We call the members of $\mbox{\em VR}$ {\em wff variables} or (in the context
of first-order logic which we will describe shortly) {\em wff metavariables}.
Note that the symbols $\phi$, $\psi$, etc.\ denote actual specific members of
$\mbox{\em VR}$; they are not metavariables of our expository language (which
we denote with $\alpha$, $\beta$, etc.) but are instead (meta)constant symbols
(members of $\mbox{\em SM}$) from the point of view of our expository
language.  The equivalent system of propositional calculus described in
\cite{Tarski1965} also uses the symbols $\phi$, $\psi$, etc.\ to denote wff
metavariables, but in \cite{Tarski1965} unlike here those are metavariables of
the expository language and not primitive symbols of the formal system.

The first two statements define wffs: if $\varphi$ and $\psi$ are wffs, so is
$(\varphi \to \psi)$; if $\varphi$ is a wff, so is $\lnot\varphi$. The next
three are the axioms of propositional calculus: if $\varphi$ and $\psi$ are
wffs, then $\vdash (\varphi \to (\psi \to \varphi))$ is an (axiomatic)
theorem; etc. The
last is the rule of modus ponens: if $\varphi$ and $\psi$ are wffs, and
$\vdash (\varphi\to\psi)$ and $\vdash \varphi$ are theorems, then $\vdash
\psi$ is a theorem.

The correspondence to ordinary propositional calculus is as follows.  We
consider only provable statements of the form $\langle\varnothing,
T,\varnothing,A\rangle$ with $T$ defined as above.  The first term of the
assertion $A$ of any such statement is either ``wff'' or ``$\vdash$''.  A
statement for which the first term is ``wff'' is a {\em wff} of propositional
calculus, and one where the first term is ``$\vdash$'' is a {\em
theorem (scheme)} of propositional calculus.

The universe of this formal system also contains many other provable
statements.  Those with distinct-variable restrictions are irrelevant because
propositional calculus has no constraints on substitutions.  Those that have
logical hypotheses we call {\em inferences}\index{inference} when
the logical hypotheses are of the form
$\langle\vdash\rangle\frown w$ where $w$ is a wff (with the leading constant
term ``wff'' removed).  Inferences (other than the modus ponens rule) are not a
proper part of propositional calculus but are convenient to use when building a
hierarchy of provable statements.  A provable statement with a nonsense
hypothesis such as $\langle \to,\vdash,\lnot\rangle$, and this same expression
as its assertion, we consider irrelevant; no use can be made of it in
proving theorems, since there is no way to eliminate the nonsense hypothesis.

{\footnotesize\begin{quotation}
{\em Comment.} Our use of parentheses in the definition of a wff illustrates
how axiomatic statements should be carefully stated in a way that
ties in unambiguously with the substitutions allowed by the formal system.
There are many ways we could have defined wffs---for example, Polish
prefix notation would have allowed us to omit parentheses entirely, at
the expense of readability---but we must define them in a way that is
unambiguous.  For example, if we had omitted parentheses from the
definition of $(\varphi\to \psi)$, the wff $\lnot\varphi\to \psi$ could
be interpreted as either $\lnot(\varphi\to\psi)$ or $(\lnot\varphi\to\psi)$
and would have allowed us to prove nonsense.  Note that there is no
concept of operator binding precedence built into our formal system.
\end{quotation}}

\begin{sloppy}
\subsection{Example~2---Predicate Calculus with Equality}\index{predicate
calculus}
\end{sloppy}

Here we extend Example~1 to include predicate calculus with equality,
illustrating the use of distinct-variable restrictions.  This system is the
same as Tarski's system $\mathfrak{S}_2$ in \cite{Tarski1965} (except that the
axioms of propositional calculus are different but equivalent, and a redundant
axiom is omitted).  We extend $\mbox{\em CN}$ with the constants
$\{\mbox{var},\forall,=\}$.  We extend $\mbox{\em VR}$ with an infinite set of
{\em individual metavariables}\index{individual
metavariable} $\{x,y,z,\ldots\}$ and denote this subset
$\mbox{\em Vr}$.

We also join to $\mbox{\em CN}$ a possibly infinite set $\mbox{\em Pr}$ of {\em
predicates} $\{R,S,\ldots\}$.  We associate with $\mbox{\em Pr}$ a function
$\mbox{rnk}$ from $\mbox{\em Pr}$ to $\omega$, and for $\alpha\in \mbox{\em
Pr}$ we call $\mbox{rnk}(\alpha)$ the {\em rank} of the predicate $\alpha$,
which is simply the number of ``arguments'' that the predicate has.  (Most
applications of predicate calculus will have a finite number of predicates;
for example, set theory has the single two-argument or binary predicate $\in$,
which is usually written with its arguments surrounding the predicate symbol
rather than with the prefix notation we will use for the general case.)  As a
device to facilitate our discussion, we will let $\mbox{\em Vs}$ be any fixed
one-to-one function from $\omega$ to $\mbox{\em Vr}$; thus $\mbox{\em Vs}$ is
any simple infinite sequence of individual metavariables with no repeating
terms.

In this example we will not include the function symbols that are often part of
formalizations of predicate calculus.  Using metalogical arguments that are
beyond the scope of our discussion, it can be shown that our formalization is
equivalent when functions are introduced via appropriate definitions.

We extend the set $T$ defined in Example~1 with the expressions
$\{\langle \mbox{var\ } x\rangle,$ $ \langle \mbox{var\ } y\rangle, \langle
\mbox{var\ } z\rangle,\ldots\}$.  We extend the $\Gamma$ above
with the axiomatic statements that are the reducts of the following
pre-statements:
\begin{list}{}{\itemsep 0.0pt}
      \item[] $\langle\varnothing,T,\varnothing,
               \langle \mbox{wff\ }\forall x\,\varphi\rangle\rangle$
      \item[] $\langle\varnothing,T,\varnothing,
               \langle \mbox{wff\ }x=y\rangle\rangle$
      \item[] $\langle\varnothing,T,
               \{\langle\vdash\varphi\rangle\},
               \langle\vdash\forall x\,\varphi\rangle\rangle$
      \item[] $\langle\varnothing,T,\varnothing,
               \langle \vdash((\forall x(\varphi\to\psi)
                  \to(\forall x\,\varphi\to\forall x\,\psi))
               \rangle\rangle$
      \item[] $\langle\{\{x,\varphi\}\},T,\varnothing,
               \langle \vdash(\varphi\to\forall x\,\varphi)
               \rangle\rangle$
      \item[] $\langle\{\{x,y\}\},T,\varnothing,
               \langle \vdash\lnot\forall x\lnot x=y
               \rangle\rangle$
      \item[] $\langle\varnothing,T,\varnothing,
               \langle \vdash(x=z
                  \to(x=y\to z=y))
               \rangle\rangle$
      \item[] $\langle\varnothing,T,\varnothing,
               \langle \vdash(y=z
                  \to(x=y\to x=z))
               \rangle\rangle$
\end{list}
These are the axioms not involving predicate symbols. The first two statements
extend the definition of a wff.  The third is the rule of generalization.  The
fifth states, in effect, ``For a wff $\varphi$ and variable $x$,
$\vdash(\varphi\to\forall x\,\varphi)$, provided that $x$ does not occur in
$\varphi$.''  The sixth states ``For variables $x$ and $y$,
$\vdash\lnot\forall x\lnot x = y$, provided that $x$ and $y$ are distinct.''
(This proviso is not necessary but was included by Tarski to
weaken the axiom and still show that the system is logically complete.)

Finally, for each predicate symbol $\alpha\in \mbox{\em Pr}$, we add to
$\Gamma$ an axiomatic statement, extending the definition of wff,
that is the reduct of the following pre-statement:
\begin{displaymath}
    \langle\varnothing,T,\varnothing,
            \langle \mbox{wff},\alpha\rangle\
            \frown \mbox{\em Vs}\restriction\mbox{rnk}(\alpha)\rangle
\end{displaymath}
and for each $\alpha\in \mbox{\em Pr}$ and each $n < \mbox{rnk}(\alpha)$
we add to $\Gamma$ an equality axiom that is the reduct of the
following pre-statement:
\begin{eqnarray*}
    \lefteqn{\langle\varnothing,T,\varnothing,
            \langle
      \vdash,(,\mbox{\em Vs}_n,=,\mbox{\em Vs}_{\mbox{rnk}(\alpha)},\to,
            (,\alpha\rangle\frown \mbox{\em Vs}\restriction\mbox{rnk}(\alpha)} \\
  & & \frown
            \langle\to,\alpha\rangle\frown \mbox{\em Vs}\restriction n\frown
            \langle \mbox{\em Vs}_{\mbox{rnk}(\alpha)}\rangle \\
 & & \frown
            \mbox{\em Vs}\restriction(\mbox{rnk}(\alpha)\setminus(n+1))\frown
            \langle),)\rangle\rangle
\end{eqnarray*}
where $\restriction$ denotes function domain restriction and $\setminus$
denotes set difference.  Recall that a subscript on $\mbox{\em Vs}$
denotes one of its terms.  (In the above two axiom sets commas are placed
between successive terms of sequences to prevent ambiguity, and if you examine
them with care you will be able to distinguish those parentheses that denote
constant symbols from those of our expository language that delimit function
arguments.  Although it might have been better to use boldface for our
primitive symbols, unfortunately boldface was not available for all characters
on the \LaTeX\ system used to typeset this text.)  These seemingly forbidding
axioms can be understood by analogy to concatenation of substrings in a
computer language.  They are actually relatively simple for each specific case
and will become clearer by looking at the special case of a binary predicate
$\alpha = R$ where $\mbox{rnk}(R)=2$.  Letting $\mbox{\em Vs}$ be the sequence
$\langle x,y,z,\ldots\rangle$, the axioms we would add to $\Gamma$ for this
case would be the wff extension and two equality axioms that are the
reducts of the pre-statements:
\begin{list}{}{\itemsep 0.0pt}
      \item[] $\langle\varnothing,T,\varnothing,
               \langle \mbox{wff\ }R x y\rangle\rangle$
      \item[] $\langle\varnothing,T,\varnothing,
               \langle \vdash(x=z
                  \to(R x y \to R z y))
               \rangle\rangle$
      \item[] $\langle\varnothing,T,\varnothing,
               \langle \vdash(y=z
                  \to(R x y \to R x z))
               \rangle\rangle$
\end{list}
Study these carefully to see how the general axioms above evaluate to
them.  In practice, typically only a few special cases such as this would be
needed, and in any case the Metamath language will only permit us to describe
a finite number of predicates, as opposed to the infinite number permitted by
the formal system.  (If an infinite number should be needed for some reason,
we could not define the formal system directly in the Metamath language but
could instead define it metalogically under set theory as we
do in this appendix, and only the underlying set theory, with its single
binary predicate, would be defined directly in the Metamath language.)


{\footnotesize\begin{quotation}
{\em Comment.}  As we noted earlier, the specific variables denoted by the
symbols $x,y,z,\ldots\in \mbox{\em Vr}\subseteq \mbox{\em VR}\subseteq
\mbox{\em SM}$ in Example~2 are not the actual variables of ordinary predicate
calculus but should be thought of as metavariables ranging over them.  For
example, a distinct-variable restriction would be meaningless for actual
variables of ordinary predicate calculus since two different actual variables
are by definition distinct.  And when we talk about an arbitrary
representative $\alpha\in \mbox{\em Vr}$, $\alpha$ is a metavariable (in our
expository language) that ranges over metavariables (which are primitives of
our formal system) each of which ranges over the actual individual variables
of predicate calculus (which are never mentioned in our formal system).

The constant called ``var'' above is called \texttt{setvar} in the
\texttt{set.mm} database file, but it means the same thing.  I felt
that ``var'' is a more meaningful name in the context of predicate
calculus, whose use is not limited to set theory.  For consistency we
stick with the name ``var'' throughout this Appendix, even after set
theory is introduced.
\end{quotation}}

\subsection{Free Variables and Proper Substitution}\index{free variable}
\index{proper substitution}\index{substitution!proper}

Typical representations of mathematical axioms use concepts such
as ``free variable,'' ``bound variable,'' and ``proper substitution''
as primitive notions.
A free variable is a variable that
is not a parameter of any container expression.
A bound variable is the opposite of a free variable; it is a
a variable that has been bound in a container expression.
For example, in the expression $\forall x \varphi$ (for all $x$, $\varphi$
is true), the variable $x$
is bound within the for-all ($\forall$) expression.
It is possible to change one variable to another, and that process is called
``proper substitution.''
In most books, proper substitution has a somewhat complicated recursive
definition with multiple cases based on the occurrences of free and
bound variables.
You may consult
\cite[ch.\ 3--4]{Hamilton}\index{Hamilton, Alan G.} (as well as
many other texts) for more formal details about these terms.

Using these concepts as \texttt{primitives} creates complications
for computer implementations.

In the system of Example~2, there are no primitive notions of free variable
and proper substitution.  Tarski \cite{Tarski1965} shows that this system is
logically equivalent to the more typical textbook systems that do have these
primitive notions, if we introduce these notions with appropriate definitions
and metalogic.  We could also define axioms for such systems directly,
although the recursive definitions of free variable and proper substitution
would be messy and awkward to work with.  Instead, we mention two devices that
can be used in practice to mimic these notions.  (1) Instead of introducing
special notation to express (as a logical hypothesis) ``where $x$ is not free
in $\varphi$'' we can use the logical hypothesis $\vdash(\varphi\to\forall
x\,\varphi)$.\label{effectivelybound}\index{effectively
not free}\footnote{This is a slightly weaker requirement than ``where $x$ is
not free in $\varphi$.''  If we let $\varphi$ be $x=x$, we have the theorem
$(x=x\to\forall x\,x=x)$ which satisfies the hypothesis, even though $x$ is
free in $x=x$ .  In a case like this we say that $x$ is {\em effectively not
free}\index{effectively not free} in $x=x$, since $x=x$ is logically
equivalent to $\forall x\,x=x$ in which $x$ is bound.} (2) It can be shown
that the wff $((x=y\to\varphi)\wedge\exists x(x=y\wedge\varphi))$ (with the
usual definitions of $\wedge$ and $\exists$; see Example~4 below) is logically
equivalent to ``the wff that results from proper substitution of $y$ for $x$
in $\varphi$.''  This works whether or not $x$ and $y$ are distinct.

\subsection{Metalogical Completeness}\index{metalogical completeness}

In the system of Example~2, the
following are provable pre-statements (and their reducts are
provable statements):
\begin{eqnarray*}
      & \langle\{\{x,y\}\},T,\varnothing,
               \langle \vdash\lnot\forall x\lnot x=y
               \rangle\rangle & \\
     &  \langle\varnothing,T,\varnothing,
               \langle \vdash\lnot\forall x\lnot x=x
               \rangle\rangle &
\end{eqnarray*}
whereas the following pre-statement is not to my knowledge provable (but
in any case we will pretend it's not for sake of illustration):
\begin{eqnarray*}
     &  \langle\varnothing,T,\varnothing,
               \langle \vdash\lnot\forall x\lnot x=y
               \rangle\rangle &
\end{eqnarray*}
In other words, we can prove ``$\lnot\forall x\lnot x=y$ where $x$ and $y$ are
distinct'' and separately prove ``$\lnot\forall x\lnot x=x$'', but we can't
prove the combined general case ``$\lnot\forall x\lnot x=y$'' that has no
proviso.  Now this does not compromise logical completeness, because the
variables are really metavariables and the two provable cases together cover
all possible cases.  The third case can be considered a metatheorem whose
direct proof, using the system of Example~2, lies outside the capability of the
formal system.

Also, in the system of Example~2 the following pre-statement is not to my
knowledge provable (again, a conjecture that we will pretend to be the case):
\begin{eqnarray*}
     & \langle\varnothing,T,\varnothing,
               \langle \vdash(\forall x\, \varphi\to\varphi)
               \rangle\rangle &
\end{eqnarray*}
Instead, we can only prove specific cases of $\varphi$ involving individual
metavariables, and by induction on formula length, prove as a metatheorem
outside of our formal system the general statement above.  The details of this
proof are found in \cite{Kalish}.

There does, however, exist a system of predicate calculus in which all such
``simple metatheorems'' as those above can be proved directly, and we present
it in Example~3. A {\em simple metatheorem}\index{simple metatheorem}
is any statement of the formal
system of Example~2 where all distinct variable restrictions consist of either
two individual metavariables or an individual metavariable and a wff
metavariable, and which is provable by combining cases outside the system as
above.  A system is {\em metalogically complete}\index{metalogical
completeness} if all of its simple
metatheorems are (directly) provable statements. The precise definition of
``simple metatheorem'' and the proof of the ``metalogical completeness'' of
Example~3 is found in Remark 9.6 and Theorem 9.7 of \cite{Megill}.\index{Megill,
Norman}

\begin{sloppy}
\subsection{Example~3---Metalogically Complete Predicate
Calculus with
Equality}
\end{sloppy}

For simplicity we will assume there is one binary predicate $R$;
this system suffices for set theory, where the $R$ is of course the $\in$
predicate.  We label the axioms as they appear in \cite{Megill}.  This
system is logically equivalent to that of Example~2 (when the latter is
restricted to this single binary predicate) but is also metalogically
complete.\index{metalogical completeness}

Let
\begin{itemize}
  \item[] $\mbox{\em CN}=\{\mbox{wff}, \mbox{var}, \vdash, \to, \lnot, (,),\forall,=,R\}$.
  \item[] $\mbox{\em VR}=\{\varphi,\psi,\chi,\ldots\}\cup\{x,y,z,\ldots\}$.
  \item[] $T = \{\langle \mbox{wff\ } \varphi\rangle,
             \langle \mbox{wff\ } \psi\rangle,
             \langle \mbox{wff\ } \chi\rangle,\ldots\}\cup
       \{\langle \mbox{var\ } x\rangle, \langle \mbox{var\ } y\rangle, \langle
       \mbox{var\ }z\rangle,\ldots\}$.

\noindent Then
  $\Gamma$ consists of the reducts of the following pre-statements:
    \begin{itemize}
      \item[] $\langle\varnothing,T,\varnothing,
               \langle \mbox{wff\ }(\varphi\to\psi)\rangle\rangle$
      \item[] $\langle\varnothing,T,\varnothing,
               \langle \mbox{wff\ }\lnot\varphi\rangle\rangle$
      \item[] $\langle\varnothing,T,\varnothing,
               \langle \mbox{wff\ }\forall x\,\varphi\rangle\rangle$
      \item[] $\langle\varnothing,T,\varnothing,
               \langle \mbox{wff\ }x=y\rangle\rangle$
      \item[] $\langle\varnothing,T,\varnothing,
               \langle \mbox{wff\ }Rxy\rangle\rangle$
      \item[(C1$'$)] $\langle\varnothing,T,\varnothing,
               \langle \vdash(\varphi\to(\psi\to\varphi))
               \rangle\rangle$
      \item[(C2$'$)] $\langle\varnothing,T,
               \varnothing,
               \langle \vdash((\varphi\to(\psi\to\chi))\to
               ((\varphi\to\psi)\to(\varphi\to\chi)))
               \rangle\rangle$
      \item[(C3$'$)] $\langle\varnothing,T,
               \varnothing,
               \langle \vdash((\lnot\varphi\to\lnot\psi)\to
               (\psi\to\varphi))\rangle\rangle$
      \item[(C4$'$)] $\langle\varnothing,T,
               \varnothing,
               \langle \vdash(\forall x(\forall x\,\varphi\to\psi)\to
                 (\forall x\,\varphi\to\forall x\,\psi))\rangle\rangle$
      \item[(C5$'$)] $\langle\varnothing,T,
               \varnothing,
               \langle \vdash(\forall x\,\varphi\to\varphi)\rangle\rangle$
      \item[(C6$'$)] $\langle\varnothing,T,
               \varnothing,
               \langle \vdash(\forall x\forall y\,\varphi\to
                 \forall y\forall x\,\varphi)\rangle\rangle$
      \item[(C7$'$)] $\langle\varnothing,T,
               \varnothing,
               \langle \vdash(\lnot\varphi\to\forall x\lnot\forall x\,\varphi
                 )\rangle\rangle$
      \item[(C8$'$)] $\langle\varnothing,T,
               \varnothing,
               \langle \vdash(x=y\to(x=z\to y=z))\rangle\rangle$
      \item[(C9$'$)] $\langle\varnothing,T,
               \varnothing,
               \langle \vdash(\lnot\forall x\, x=y\to(\lnot\forall x\, x=z\to
                 (y=z\to\forall x\, y=z)))\rangle\rangle$
      \item[(C10$'$)] $\langle\varnothing,T,
               \varnothing,
               \langle \vdash(\forall x(x=y\to\forall x\,\varphi)\to
                 \varphi))\rangle\rangle$
      \item[(C11$'$)] $\langle\varnothing,T,
               \varnothing,
               \langle \vdash(\forall x\, x=y\to(\forall x\,\varphi
               \to\forall y\,\varphi))\rangle\rangle$
      \item[(C12$'$)] $\langle\varnothing,T,
               \varnothing,
               \langle \vdash(x=y\to(Rxz\to Ryz))\rangle\rangle$
      \item[(C13$'$)] $\langle\varnothing,T,
               \varnothing,
               \langle \vdash(x=y\to(Rzx\to Rzy))\rangle\rangle$
      \item[(C15$'$)] $\langle\varnothing,T,
               \varnothing,
               \langle \vdash(\lnot\forall x\, x=y\to(x=y\to(\varphi
                 \to\forall x(x=y\to\varphi))))\rangle\rangle$
      \item[(C16$'$)] $\langle\{\{x,y\}\},T,
               \varnothing,
               \langle \vdash(\forall x\, x=y\to(\varphi\to\forall x\,\varphi)
                 )\rangle\rangle$
      \item[(C5)] $\langle\{\{x,\varphi\}\},T,\varnothing,
               \langle \vdash(\varphi\to\forall x\,\varphi)
               \rangle\rangle$
      \item[(MP)] $\langle\varnothing,T,
               \{\langle\vdash(\varphi\to\psi)\rangle,
                 \langle\vdash\varphi\rangle\},
               \langle\vdash\psi\rangle\rangle$
      \item[(Gen)] $\langle\varnothing,T,
               \{\langle\vdash\varphi\rangle\},
               \langle\vdash\forall x\,\varphi\rangle\rangle$
    \end{itemize}
\end{itemize}

While it is known that these axioms are ``metalogically complete,'' it is
not known whether they are independent (i.e.\ none is
redundant) in the metalogical sense; specifically, whether any axiom (possibly
with additional non-mandatory distinct-variable restrictions, for use with any
dummy variables in its proof) is provable from the others.  Note that
metalogical independence is a weaker requirement than independence in the
usual logical sense.  Not all of the above axioms are logically independent:
for example, C9$'$ can be proved as a metatheorem from the others, outside the
formal system, by combining the possible cases of distinct variables.

\subsection{Example~4---Adding Definitions}\index{definition}
There are several ways to add definitions to a formal system.  Probably the
most proper way is to consider definitions not as part of the formal system at
all but rather as abbreviations that are part of the expository metalogic
outside the formal system.  For convenience, though, we may use the formal
system itself to incorporate definitions, adding them as axiomatic extensions
to the system.  This could be done by adding a constant representing the
concept ``is defined as'' along with axioms for it. But there is a nicer way,
at least in this writer's opinion, that introduces definitions as direct
extensions to the language rather than as extralogical primitive notions.  We
introduce additional logical connectives and provide axioms for them.  For
systems of logic such as Examples 1 through 3, the additional axioms must be
conservative in the sense that no wff of the original system that was not a
theorem (when the initial term ``wff'' is replaced by ``$\vdash$'' of course)
becomes a theorem of the extended system.  In this example we extend Example~3
(or 2) with standard abbreviations of logic.

We extend $\mbox{\em CN}$ of Example~3 with new constants $\{\leftrightarrow,
\wedge,\vee,\exists\}$, corresponding to logical equivalence,\index{logical
equivalence ($\leftrightarrow$)}\index{biconditional ($\leftrightarrow$)}
conjunction,\index{conjunction ($\wedge$)} disjunction,\index{disjunction
($\vee$)} and the existential quantifier.\index{existential quantifier
($\exists$)}  We extend $\Gamma$ with the axiomatic statements that are
the reducts of the following pre-statements:
\begin{list}{}{\itemsep 0.0pt}
      \item[] $\langle\varnothing,T,\varnothing,
               \langle \mbox{wff\ }(\varphi\leftrightarrow\psi)\rangle\rangle$
      \item[] $\langle\varnothing,T,\varnothing,
               \langle \mbox{wff\ }(\varphi\vee\psi)\rangle\rangle$
      \item[] $\langle\varnothing,T,\varnothing,
               \langle \mbox{wff\ }(\varphi\wedge\psi)\rangle\rangle$
      \item[] $\langle\varnothing,T,\varnothing,
               \langle \mbox{wff\ }\exists x\, \varphi\rangle\rangle$
  \item[] $\langle\varnothing,T,\varnothing,
     \langle\vdash ( ( \varphi \leftrightarrow \psi ) \to
     ( \varphi \to \psi ) )\rangle\rangle$
  \item[] $\langle\varnothing,T,\varnothing,
     \langle\vdash ((\varphi\leftrightarrow\psi)\to
    (\psi\to\varphi))\rangle\rangle$
  \item[] $\langle\varnothing,T,\varnothing,
     \langle\vdash ((\varphi\to\psi)\to(
     (\psi\to\varphi)\to(\varphi
     \leftrightarrow\psi)))\rangle\rangle$
  \item[] $\langle\varnothing,T,\varnothing,
     \langle\vdash (( \varphi \wedge \psi ) \leftrightarrow\neg ( \varphi
     \to \neg \psi )) \rangle\rangle$
  \item[] $\langle\varnothing,T,\varnothing,
     \langle\vdash (( \varphi \vee \psi ) \leftrightarrow (\neg \varphi
     \to \psi )) \rangle\rangle$
  \item[] $\langle\varnothing,T,\varnothing,
     \langle\vdash (\exists x \,\varphi\leftrightarrow
     \lnot \forall x \lnot \varphi)\rangle\rangle$
\end{list}
The first three logical axioms (statements containing ``$\vdash$'') introduce
and effectively define logical equivalence, ``$\leftrightarrow$''.  The last
three use ``$\leftrightarrow$'' to effectively mean ``is defined as.''

\subsection{Example~5---ZFC Set Theory}\index{ZFC set theory}

Here we add to the system of Example~4 the axioms of Zermelo--Fraenkel set
theory with Choice.  For convenience we make use of the
definitions in Example~4.

In the $\mbox{\em CN}$ of Example~4 (which extends Example~3), we replace the symbol $R$
with the symbol $\in$.
More explicitly, we remove from $\Gamma$ of Example~4 the three
axiomatic statements containing $R$ and replace them with the
reducts of the following:
\begin{list}{}{\itemsep 0.0pt}
      \item[] $\langle\varnothing,T,\varnothing,
               \langle \mbox{wff\ }x\in y\rangle\rangle$
      \item[] $\langle\varnothing,T,
               \varnothing,
               \langle \vdash(x=y\to(x\in z\to y\in z))\rangle\rangle$
      \item[] $\langle\varnothing,T,
               \varnothing,
               \langle \vdash(x=y\to(z\in x\to z\in y))\rangle\rangle$
\end{list}
Letting $D=\{\{\alpha,\beta\}\in \mbox{\em DV}\,|\alpha,\beta\in \mbox{\em
Vr}\}$ (in other words all individual variables must be distinct), we extend
$\Gamma$ with the ZFC axioms, called
\index{Axiom of Extensionality}
\index{Axiom of Replacement}
\index{Axiom of Union}
\index{Axiom of Power Sets}
\index{Axiom of Regularity}
\index{Axiom of Infinity}
\index{Axiom of Choice}
Extensionality, Replacement, Union, Power
Set, Regularity, Infinity, and Choice, that are the reducts of:
\begin{list}{}{\itemsep 0.0pt}
      \item[Ext] $\langle D,T,
               \varnothing,
               \langle\vdash (\forall x(x\in y\leftrightarrow x \in z)\to y
               =z) \rangle\rangle$
      \item[Rep] $\langle D,T,
               \varnothing,
               \langle\vdash\exists x ( \exists y \forall z (\varphi \to z = y
                        ) \to
                        \forall z ( z \in x \leftrightarrow \exists x ( x \in
                        y \wedge \forall y\,\varphi ) ) )\rangle\rangle$
      \item[Un] $\langle D,T,
               \varnothing,
               \langle\vdash \exists x \forall y ( \exists x ( y \in x \wedge
               x \in z ) \to y \in x ) \rangle\rangle$
      \item[Pow] $\langle D,T,
               \varnothing,
               \langle\vdash \exists x \forall y ( \forall x ( x \in y \to x
               \in z ) \to y \in x ) \rangle\rangle$
      \item[Reg] $\langle D,T,
               \varnothing,
               \langle\vdash (  x \in y \to
                 \exists x ( x \in y \wedge \forall z ( z \in x \to \lnot z
                \in y ) ) ) \rangle\rangle$
      \item[Inf] $\langle D,T,
               \varnothing,
               \langle\vdash \exists x(y\in x\wedge\forall y(y\in
               x\to
               \exists z(y \in z\wedge z\in x))) \rangle\rangle$
      \item[AC] $\langle D,T,
               \varnothing,
               \langle\vdash \exists x \forall y \forall z ( ( y \in z
               \wedge z \in w ) \to \exists w \forall y ( \exists w
              ( ( y \in z \wedge z \in w ) \wedge ( y \in w \wedge w \in x
              ) ) \leftrightarrow y = w ) ) \rangle\rangle$
\end{list}

\subsection{Example~6---Class Notation in Set Theory}\label{class}

A powerful device that makes set theory easier (and that we have
been using all along in our informal expository language) is {\em class
abstraction notation}.\index{class abstraction}\index{abstraction class}  The
definitions we introduce are rigorously justified
as conservative by Takeuti and Zaring \cite{Takeuti}\index{Takeuti, Gaisi} or
Quine \cite{Quine}\index{Quine, Willard Van Orman}.  The key idea is to
introduce the notation $\{x|\mbox{---}\}$ which means ``the class of all $x$
such that ---'' for abstraction classes and introduce (meta)variables that
range over them.  An abstraction class may or may not be a set, depending on
whether it exists (as a set).  A class that does not exist is
called a {\em proper class}.\index{proper class}\index{class!proper}

To illustrate the use of abstraction classes we will provide some examples
of definitions that make use of them:  the empty set, class union, and
unordered pair.  Many other such definitions can be found in the
Metamath set theory database,
\texttt{set.mm}.\index{set theory database (\texttt{set.mm})}

% We intentionally break up the sequence of math symbols here
% because otherwise the overlong line goes beyond the page in narrow mode.
We extend $\mbox{\em CN}$ of Example~5 with new symbols $\{$
$\mbox{class},$ $\{,$ $|,$ $\},$ $\varnothing,$ $\cup,$ $,$ $\}$
where the inner braces and last comma are
constant symbols. (As before,
our dual use of some mathematical symbols for both our expository
language and as primitives of the formal system should be clear from context.)

We extend $\mbox{\em VR}$ of Example~5 with a set of {\em class
variables}\index{class variable}
$\{A,B,C,\ldots\}$. We extend the $T$ of Example~5 with $\{\langle
\mbox{class\ } A\rangle, \langle \mbox{class\ }B\rangle, \langle \mbox{class\ }
C\rangle,\ldots\}$.

To
introduce our definitions,
we add to $\Gamma$ of Example~5 the axiomatic statements
that are the reducts of the following pre-statements:
\begin{list}{}{\itemsep 0.0pt}
      \item[] $\langle\varnothing,T,\varnothing,
               \langle \mbox{class\ }x\rangle\rangle$
      \item[] $\langle\varnothing,T,\varnothing,
               \langle \mbox{class\ }\{x|\varphi\}\rangle\rangle$
      \item[] $\langle\varnothing,T,\varnothing,
               \langle \mbox{wff\ }A=B\rangle\rangle$
      \item[] $\langle\varnothing,T,\varnothing,
               \langle \mbox{wff\ }A\in B\rangle\rangle$
      \item[Ab] $\langle\varnothing,T,\varnothing,
               \langle \vdash ( y \in \{ x |\varphi\} \leftrightarrow
                  ( ( x = y \to\varphi) \wedge \exists x ( x = y
                  \wedge\varphi) ))
               \rangle\rangle$
      \item[Eq] $\langle\{\{x,A\},\{x,B\}\},T,\varnothing,
               \langle \vdash ( A = B \leftrightarrow
               \forall x ( x \in A \leftrightarrow x \in B ) )
               \rangle\rangle$
      \item[El] $\langle\{\{x,A\},\{x,B\}\},T,\varnothing,
               \langle \vdash ( A \in B \leftrightarrow \exists x
               ( x = A \wedge x \in B ) )
               \rangle\rangle$
\end{list}
Here we say that an individual variable is a class; $\{x|\varphi\}$ is a
class; and we extend the definition of a wff to include class equality and
membership.  Axiom Ab defines membership of a variable in a class abstraction;
the right-hand side can be read as ``the wff that results from proper
substitution of $y$ for $x$ in $\varphi$.''\footnote{Note that this definition
makes unnecessary the introduction of a separate notation similar to
$\varphi(x|y)$ for proper substitution, although we may choose to do so to be
conventional.  Incidentally, $\varphi(x|y)$ as it stands would be ambiguous in
the formal systems of our examples, since we wouldn't know whether
$\lnot\varphi(x|y)$ meant $\lnot(\varphi(x|y))$ or $(\lnot\varphi)(x|y)$.
Instead, we would have to use an unambiguous variant such as $(\varphi\,
x|y)$.}  Axioms Eq and El extend the meaning of the existing equality and
membership connectives.  This is potentially dangerous and requires careful
justification.  For example, from Eq we can derive the Axiom of Extensionality
with predicate logic alone; thus in principle we should include the Axiom of
Extensionality as a logical hypothesis.  However we do not bother to do this
since we have already presupposed that axiom earlier. The distinct variable
restrictions should be read ``where $x$ does not occur in $A$ or $B$.''  We
typically do this when the right-hand side of a definition involves an
individual variable not in the expression being defined; it is done so that
the right-hand side remains independent of the particular ``dummy'' variable
we use.

We continue to add to $\Gamma$ the following definitions
(i.e. the reducts of the following pre-statements) for empty
set,\index{empty set} class union,\index{union} and unordered
pair.\index{unordered pair}  They should be self-explanatory.  Analogous to our
use of ``$\leftrightarrow$'' to define new wffs in Example~4, we use ``$=$''
to define new abstraction terms, and both may be read informally as ``is
defined as'' in this context.
\begin{list}{}{\itemsep 0.0pt}
      \item[] $\langle\varnothing,T,\varnothing,
               \langle \mbox{class\ }\varnothing\rangle\rangle$
      \item[] $\langle\varnothing,T,\varnothing,
               \langle \vdash \varnothing = \{ x | \lnot x = x \}
               \rangle\rangle$
      \item[] $\langle\varnothing,T,\varnothing,
               \langle \mbox{class\ }(A\cup B)\rangle\rangle$
      \item[] $\langle\{\{x,A\},\{x,B\}\},T,\varnothing,
               \langle \vdash ( A \cup B ) = \{ x | ( x \in A \vee x \in B ) \}
               \rangle\rangle$
      \item[] $\langle\varnothing,T,\varnothing,
               \langle \mbox{class\ }\{A,B\}\rangle\rangle$
      \item[] $\langle\{\{x,A\},\{x,B\}\},T,\varnothing,
               \langle \vdash \{ A , B \} = \{ x | ( x = A \vee x = B ) \}
               \rangle\rangle$
\end{list}

\section{Metamath as a Formal System}\label{theorymm}

This section presupposes a familiarity with the Metamath computer language.

Our theory describes formal systems and their universes.  The Metamath
language provides a way of representing these set-theoretical objects to
a computer.  A Metamath database, being a finite set of {\sc ascii}
characters, can usually describe only a subset of a formal system and
its universe, which are typically infinite.  However the database can
contain as large a finite subset of the formal system and its universe
as we wish.  (Of course a Metamath set theory database can, in
principle, indirectly describe an entire infinite formal system by
formalizing the expository language in this Appendix.)

For purpose of our discussion, we assume the Metamath database
is in the simple form described on p.~\pageref{framelist},
consisting of all constant and variable declarations at the beginning,
followed by a sequence of extended frames each
delimited by \texttt{\$\char`\{} and \texttt{\$\char`\}}.  Any Metamath database can
be converted to this form, as described on p.~\pageref{frameconvert}.

The math symbol tokens of a Metamath source file, which are declared
with \texttt{\$c} and \texttt{\$v} statements, are names we assign to
representatives of $\mbox{\em CN}$ and $\mbox{\em VR}$.  For
definiteness we could assume that the first math symbol declared as a
variable corresponds to $v_0$, the second to $v_1$, etc., although the
exact correspondence we choose is not important.

In the Metamath language, each \texttt{\$d}, \texttt{\$f}, and
 \texttt{\$e} source
statement in an extended frame (Section~\ref{frames})
corresponds respectively to a member of the
collections $D$, $T$, and $H$ in a formal system statement $\langle
D_M,T_M,H,A\rangle$.  The math symbol strings following these Metamath keywords
correspond to a variable pair (in the case of \texttt{\$d}) or an expression (for
the other two keywords). The math symbol string following a \texttt{\$a} source
statement corresponds to expression $A$ in an axiomatic statement of the
formal system; the one following a \texttt{\$p} source statement corresponds to
$A$ in a provable statement that is not axiomatic.  In other words, each
extended frame in a Metamath database corresponds to
a pre-statement of the formal system, and a frame corresponds to
a statement of the formal system.  (Don't confuse the two meanings of
``statement'' here.  A statement of the formal system corresponds to the
several statements in a Metamath database that may constitute a
frame.)

In order for the computer to verify that a formal system statement is
provable, each \texttt{\$p} source statement is accompanied by a proof.
However, the proof does not correspond to anything in the formal system
but is simply a way of communicating to the computer the information
needed for its verification.  The proof tells the computer {\em how to
construct} specific members of closure of the formal system
pre-statement corresponding to the extended frame of the \texttt{\$p}
statement.  The final result of the construction is the member of the
closure that matches the \texttt{\$p} statement.  The abstract formal
system, on the other hand, is concerned only with the {\em existence} of
members of the closure.

As mentioned on p.~\pageref{exampleref}, Examples 1 and 3--6 in the
previous Section parallel the development of logic and set theory in the
Metamath database
\texttt{set.mm}.\index{set theory database (\texttt{set.mm})} You may
find it instructive to compare them.


\chapter{The MIU System}
\label{MIU}
\index{formal system}
\index{MIU-system}

The following is a listing of the file \texttt{miu.mm}.  It is self-explanatory.

%%%%%%%%%%%%%%%%%%%%%%%%%%%%%%%%%%%%%%%%%%%%%%%%%%%%%%%%%%%%

\begin{verbatim}
$( The MIU-system:  A simple formal system $)

$( Note:  This formal system is unusual in that it allows
empty wffs.  To work with a proof, you must type
SET EMPTY_SUBSTITUTION ON before using the PROVE command.
By default, this is OFF in order to reduce the number of
ambiguous unification possibilities that have to be selected
during the construction of a proof.  $)

$(
Hofstadter's MIU-system is a simple example of a formal
system that illustrates some concepts of Metamath.  See
Douglas R. Hofstadter, _Goedel, Escher, Bach:  An Eternal
Golden Braid_ (Vintage Books, New York, 1979), pp. 33ff. for
a description of the MIU-system.

The system has 3 constant symbols, M, I, and U.  The sole
axiom of the system is MI. There are 4 rules:
     Rule I:  If you possess a string whose last letter is I,
     you can add on a U at the end.
     Rule II:  Suppose you have Mx.  Then you may add Mxx to
     your collection.
     Rule III:  If III occurs in one of the strings in your
     collection, you may make a new string with U in place
     of III.
     Rule IV:  If UU occurs inside one of your strings, you
     can drop it.
Unfortunately, Rules III and IV do not have unique results:
strings could have more than one occurrence of III or UU.
This requires that we introduce the concept of an "MIU
well-formed formula" or wff, which allows us to construct
unique symbol sequences to which Rules III and IV can be
applied.
$)

$( First, we declare the constant symbols of the language.
Note that we need two symbols to distinguish the assertion
that a sequence is a wff from the assertion that it is a
theorem; we have arbitrarily chosen "wff" and "|-". $)
      $c M I U |- wff $. $( Declare constants $)

$( Next, we declare some variables. $)
     $v x y $.

$( Throughout our theory, we shall assume that these
variables represent wffs. $)
 wx   $f wff x $.
 wy   $f wff y $.

$( Define MIU-wffs.  We allow the empty sequence to be a
wff. $)

$( The empty sequence is a wff. $)
 we   $a wff $.
$( "M" after any wff is a wff. $)
 wM   $a wff x M $.
$( "I" after any wff is a wff. $)
 wI   $a wff x I $.
$( "U" after any wff is a wff. $)
 wU   $a wff x U $.

$( Assert the axiom. $)
 ax   $a |- M I $.

$( Assert the rules. $)
 ${
   Ia   $e |- x I $.
$( Given any theorem ending with "I", it remains a theorem
if "U" is added after it.  (We distinguish the label I_
from the math symbol I to conform to the 24-Jun-2006
Metamath spec.) $)
   I_    $a |- x I U $.
 $}
 ${
IIa  $e |- M x $.
$( Given any theorem starting with "M", it remains a theorem
if the part after the "M" is added again after it. $)
   II   $a |- M x x $.
 $}
 ${
   IIIa $e |- x I I I y $.
$( Given any theorem with "III" in the middle, it remains a
theorem if the "III" is replaced with "U". $)
   III  $a |- x U y $.
 $}
 ${
   IVa  $e |- x U U y $.
$( Given any theorem with "UU" in the middle, it remains a
theorem if the "UU" is deleted. $)
   IV   $a |- x y $.
  $}

$( Now we prove the theorem MUIIU.  You may be interested in
comparing this proof with that of Hofstadter (pp. 35 - 36).
$)
 theorem1  $p |- M U I I U $=
      we wM wU wI we wI wU we wU wI wU we wM we wI wU we wM
      wI wI wI we wI wI we wI ax II II I_ III II IV $.
\end{verbatim}\index{well-formed formula (wff)}

The \texttt{show proof /lemmon/renumber} command
yields the following display.  It is very similar
to the one in \cite[pp.~35--36]{Hofstadter}.\index{Hofstadter, Douglas R.}

\begin{verbatim}
1 ax             $a |- M I
2 1 II           $a |- M I I
3 2 II           $a |- M I I I I
4 3 I_           $a |- M I I I I U
5 4 III          $a |- M U I U
6 5 II           $a |- M U I U U I U
7 6 IV           $a |- M U I I U
\end{verbatim}

We note that Hofstadter's ``MU-puzzle,'' which asks whether
MU is a theorem of the MIU-system, cannot be answered using
the system above because the MU-puzzle is a question {\em
about} the system.  To prove the answer to the MU-puzzle,
a much more elaborate system is needed, namely one that
models the MIU-system within set theory.  (Incidentally, the
answer to the MU-puzzle is no.)

\chapter{Metamath Language EBNF}%
\label{BNF}%
\index{Metamath Language EBNF}

The following is a formal description of the basic Metamath language syntax
(with compressed proofs and support for unknown proof steps).
It is defined using the
Extended Backus--Naur Form (EBNF)\index{Extended Backus--Naur Form}\index{EBNF}
notation from W3C\index{W3C}
\textit{Extensible Markup Language (XML) 1.0 (Fifth Edition)}
(W3C Recommendation 26 November 2008) at
\url{https://www.w3.org/TR/xml/#sec-notation}.

The \texttt{database}
rule is processed until the end of the file (\texttt{EOF}).
The rules eventually require reading whitespace-separated tokens.
A token has an upper-case definition (see below)
or is a string constant in a non-token (such as \texttt{'\$a'}).
We intend for this to be correct, but if there is a conflict the
rules of section \ref{spec} govern. That section also discusses
non-syntax restrictions not shown here
(e.g., that each new label token
defined in a \texttt{hypothesis-stmt} or \texttt{assert-stmt}
must be unique).

\begin{verbatim}
database ::= outermost-scope-stmt*

outermost-scope-stmt ::=
  include-stmt | constant-stmt | stmt

/* File inclusion command; process file as a database.
   Databases should NOT have a comment in the filename. */
include-stmt ::= '$[' filename '$]'

/* Constant symbols declaration. */
constant-stmt ::= '$c' constant+ '$.'

/* A normal statement can occur in any scope. */
stmt ::= block | variable-stmt | disjoint-stmt |
  hypothesis-stmt | assert-stmt

/* A block. You can have 0 statements in a block. */
block ::= '${' stmt* '$}'

/* Variable symbols declaration. */
variable-stmt ::= '$v' variable+ '$.'

/* Disjoint variables. Simple disjoint statements have
   2 variables, i.e., "variable*" is empty for them. */
disjoint-stmt ::= '$d' variable variable variable* '$.'

hypothesis-stmt ::= floating-stmt | essential-stmt

/* Floating (variable-type) hypothesis. */
floating-stmt ::= LABEL '$f' typecode variable '$.'

/* Essential (logical) hypothesis. */
essential-stmt ::= LABEL '$e' typecode MATH-SYMBOL* '$.'

assert-stmt ::= axiom-stmt | provable-stmt

/* Axiomatic assertion. */
axiom-stmt ::= LABEL '$a' typecode MATH-SYMBOL* '$.'

/* Provable assertion. */
provable-stmt ::= LABEL '$p' typecode MATH-SYMBOL*
  '$=' proof '$.'

/* A proof. Proofs may be interspersed by comments.
   If '?' is in a proof it's an "incomplete" proof. */
proof ::= uncompressed-proof | compressed-proof
uncompressed-proof ::= (LABEL | '?')+
compressed-proof ::= '(' LABEL* ')' COMPRESSED-PROOF-BLOCK+

typecode ::= constant

filename ::= MATH-SYMBOL /* No whitespace or '$' */
constant ::= MATH-SYMBOL
variable ::= MATH-SYMBOL
\end{verbatim}

\needspace{2\baselineskip}
A \texttt{frame} is a sequence of 0 or more
\texttt{disjoint-{\allowbreak}stmt} and
\texttt{hypotheses-{\allowbreak}stmt} statements
(possibly interleaved with other non-\texttt{assert-stmt} statements)
followed by one \texttt{assert-stmt}.

\needspace{3\baselineskip}
Here are the rules for lexical processing (tokenization) beyond
the constant tokens shown above.
By convention these tokenization rules have upper-case names.
Every token is read for the longest possible length.
Whitespace-separated tokens are read sequentially;
note that the separating whitespace and \texttt{\$(} ... \texttt{\$)}
comments are skipped.

If a token definition uses another token definition, the whole thing
is considered a single token.
A pattern that is only part of a full token has a name beginning
with an underscore (``\_'').
An implementation could tokenize many tokens as a
\texttt{PRINTABLE-SEQUENCE}
and then check if it meets the more specific rule shown here.

Comments do not nest, and both \texttt{\$(} and \texttt{\$)}
have to be surrounded
by at least one whitespace character (\texttt{\_WHITECHAR}).
Technically comments end without consuming the trailing
\texttt{\_WHITECHAR}, but the trailing
\texttt{\_WHITECHAR} gets ignored anyway so we ignore that detail here.
Metamath language processors
are not required to support \texttt{\$)} followed
immediately by a bare end-of-file, because the closing
comment symbol is supposed to be followed by a
\texttt{\_WHITECHAR} such as a newline.

\begin{verbatim}
PRINTABLE-SEQUENCE ::= _PRINTABLE-CHARACTER+

MATH-SYMBOL ::= (_PRINTABLE-CHARACTER - '$')+

/* ASCII non-whitespace printable characters */
_PRINTABLE-CHARACTER ::= [#x21-#x7e]

LABEL ::= ( _LETTER-OR-DIGIT | '.' | '-' | '_' )+

_LETTER-OR-DIGIT ::= [A-Za-z0-9]

COMPRESSED-PROOF-BLOCK ::= ([A-Z] | '?')+

/* Define whitespace between tokens. The -> SKIP
   means that when whitespace is seen, it is
   skipped and we simply read again. */
WHITESPACE ::= (_WHITECHAR+ | _COMMENT) -> SKIP

/* Comments. $( ... $) and do not nest. */
_COMMENT ::= '$(' (_WHITECHAR+ (PRINTABLE-SEQUENCE - '$)'))*
  _WHITECHAR+ '$)' _WHITECHAR

/* Whitespace: (' ' | '\t' | '\r' | '\n' | '\f') */
_WHITECHAR ::= [#x20#x09#x0d#x0a#x0c]
\end{verbatim}
% This EBNF was developed as a collaboration between
% David A. Wheeler\index{Wheeler, David A.},
% Mario Carneiro\index{Carneiro, Mario}, and
% Benoit Jubin\index{Jubin, Benoit}, inspired by a request
% (and a lot of initial work) by Benoit Jubin.
%
% \chapter{Disclaimer and Trademarks}
%
% Information in this document is subject to change without notice and does not
% represent a commitment on the part of Norman Megill.
% \vspace{2ex}
%
% \noindent Norman D. Megill makes no warranties, either express or implied,
% regarding the Metamath computer software package.
%
% \vspace{2ex}
%
% \noindent Any trademarks mentioned in this book are the property of
% their respective owners.  The name ``Metamath'' is a trademark of
% Norman Megill.
%
\cleardoublepage
\phantomsection  % fixes the link anchor
\addcontentsline{toc}{chapter}{\bibname}

\bibliography{metamath}
%% metamath.tex - Version of 2-Jun-2019
% If you change the date above, also change the "Printed date" below.
% SPDX-License-Identifier: CC0-1.0
%
%                              PUBLIC DOMAIN
%
% This file (specifically, the version of this file with the above date)
% has been released into the Public Domain per the
% Creative Commons CC0 1.0 Universal (CC0 1.0) Public Domain Dedication
% https://creativecommons.org/publicdomain/zero/1.0/
%
% The public domain release applies worldwide.  In case this is not
% legally possible, the right is granted to use the work for any purpose,
% without any conditions, unless such conditions are required by law.
%
% Several short, attributed quotations from copyrighted works
% appear in this file under the ``fair use'' provision of Section 107 of
% the United States Copyright Act (Title 17 of the {\em United States
% Code}).  The public-domain status of this file is not applicable to
% those quotations.
%
% Norman Megill - email: nm(at)alum(dot)mit(dot)edu
%
% David A. Wheeler also donates his improvements to this file to the
% public domain per the CC0.  He works at the Institute for Defense Analyses
% (IDA), but IDA has agreed that this Metamath work is outside its "lane"
% and is not a work by IDA.  This was specifically confirmed by
% Margaret E. Myers (Division Director of the Information Technology
% and Systems Division) on 2019-05-24 and by Ben Lindorf (General Counsel)
% on 2019-05-22.

% This file, 'metamath.tex', is self-contained with everything needed to
% generate the the PDF file 'metamath.pdf' (the _Metamath_ book) on
% standard LaTeX 2e installations.  The auxiliary files are embedded with
% "filecontents" commands.  To generate metamath.pdf file, run these
% commands under Linux or Cygwin in the directory that contains
% 'metamath.tex':
%
%   rm -f realref.sty metamath.bib
%   touch metamath.ind
%   pdflatex metamath
%   pdflatex metamath
%   bibtex metamath
%   makeindex metamath
%   pdflatex metamath
%   pdflatex metamath
%
% The warnings that occur in the initial runs of pdflatex can be ignored.
% For the final run,
%
%   egrep -i 'error|warn' metamath.log
%
% should show exactly these 5 warnings:
%
%   LaTeX Warning: File `realref.sty' already exists on the system.
%   LaTeX Warning: File `metamath.bib' already exists on the system.
%   LaTeX Font Warning: Font shape `OMS/cmtt/m/n' undefined
%   LaTeX Font Warning: Font shape `OMS/cmtt/bx/n' undefined
%   LaTeX Font Warning: Some font shapes were not available, defaults
%       substituted.
%
% Search for "Uncomment" below if you want to suppress hyperlink boxes
% in the PDF output file
%
% TYPOGRAPHICAL NOTES:
% * It is customary to use an en dash (--) to "connect" names of different
%   people (and to denote ranges), and use a hyphen (-) for a
%   single compound name. Examples of connected multiple people are
%   Zermelo--Fraenkel, Schr\"{o}der--Bernstein, Tarski--Grothendieck,
%   Hewlett--Packard, and Backus--Naur.  Examples of a single person with
%   a compound name include Levi-Civita, Mittag-Leffler, and Burali-Forti.
% * Use non-breaking spaces after page abbreviations, e.g.,
%   p.~\pageref{note2002}.
%
% --------------------------- Start of realref.sty -----------------------------
\begin{filecontents}{realref.sty}
% Save the following as realref.sty.
% You can then use it with \usepackage{realref}
%
% This has \pageref jumping to the page on which the ref appears,
% \ref jumping to the point of the anchor, and \sectionref
% jumping to the start of section.
%
% Author:  Anthony Williams
%          Software Engineer
%          Nortel Networks Optical Components Ltd
% Date:    9 Nov 2001 (posted to comp.text.tex)
%
% The following declaration was made by Anthony Williams on
% 24 Jul 2006 (private email to Norman Megill):
%
%   ``I hereby donate the code for realref.sty posted on the
%   comp.text.tex newsgroup on 9th November 2001, accessible from
%   http://groups.google.com/group/comp.text.tex/msg/5a0e1cc13ea7fbb2
%   to the public domain.''
%
\ProvidesPackage{realref}
\RequirePackage[plainpages=false,pdfpagelabels=true]{hyperref}
\def\realref@anchorname{}
\AtBeginDocument{%
% ensure every label is a possible hyperlink target
\let\realref@oldrefstepcounter\refstepcounter%
\DeclareRobustCommand{\refstepcounter}[1]{\realref@oldrefstepcounter{#1}
\edef\realref@anchorname{\string #1.\@currentlabel}%
}%
\let\realref@oldlabel\label%
\DeclareRobustCommand{\label}[1]{\realref@oldlabel{#1}\hypertarget{#1}{}%
\@bsphack\protected@write\@auxout{}{%
    \string\expandafter\gdef\protect\csname
    page@num.#1\string\endcsname{\thepage}%
    \string\expandafter\gdef\protect\csname
    ref@num.#1\string\endcsname{\@currentlabel}%
    \string\expandafter\gdef\protect\csname
    sectionref@name.#1\string\endcsname{\realref@anchorname}%
}\@esphack}%
\DeclareRobustCommand\pageref[1]{{\edef\a{\csname
            page@num.#1\endcsname}\expandafter\hyperlink{page.\a}{\a}}}%
\DeclareRobustCommand\ref[1]{{\edef\a{\csname
            ref@num.#1\endcsname}\hyperlink{#1}{\a}}}%
\DeclareRobustCommand\sectionref[1]{{\edef\a{\csname
            ref@num.#1\endcsname}\edef\b{\csname
            sectionref@name.#1\endcsname}\hyperlink{\b}{\a}}}%
}
\end{filecontents}
% ---------------------------- End of realref.sty ------------------------------

% --------------------------- Start of metamath.bib -----------------------------
\begin{filecontents}{metamath.bib}
@book{Albers, editor = "Donald J. Albers and G. L. Alexanderson",
  title = "Mathematical People",
  publisher = "Contemporary Books, Inc.",
  address = "Chicago",
  note = "[QA28.M37]",
  year = 1985 }
@book{Anderson, author = "Alan Ross Anderson and Nuel D. Belnap",
  title = "Entailment",
  publisher = "Princeton University Press",
  address = "Princeton",
  volume = 1,
  note = "[QA9.A634 1975 v.1]",
  year = 1975}
@book{Barrow, author = "John D. Barrow",
  title = "Theories of Everything:  The Quest for Ultimate Explanation",
  publisher = "Oxford University Press",
  address = "Oxford",
  note = "[Q175.B225]",
  year = 1991 }
@book{Behnke,
  editor = "H. Behnke and F. Backmann and K. Fladt and W. S{\"{u}}ss",
  title = "Fundamentals of Mathematics",
  volume = "I",
  publisher = "The MIT Press",
  address = "Cambridge, Massachusetts",
  note = "[QA37.2.B413]",
  year = 1974 }
@book{Bell, author = "J. L. Bell and M. Machover",
  title = "A Course in Mathematical Logic",
  publisher = "North-Holland",
  address = "Amsterdam",
  note = "[QA9.B3953]",
  year = 1977 }
@inproceedings{Blass, author = "Andrea Blass",
  title = "The Interaction Between Category Theory and Set Theory",
  pages = "5--29",
  booktitle = "Mathematical Applications of Category Theory (Proceedings
     of the Special Session on Mathematical Applications
     Category Theory, 89th Annual Meeting of the American Mathematical
     Society, held in Denver, Colorado January 5--9, 1983)",
  editor = "John Walter Gray",
  year = 1983,
  note = "[QA169.A47 1983]",
  publisher = "American Mathematical Society",
  address = "Providence, Rhode Island"}
@proceedings{Bledsoe, editor = "W. W. Bledsoe and D. W. Loveland",
  title = "Automated Theorem Proving:  After 25 Years (Proceedings
     of the Special Session on Automatic Theorem Proving,
     89th Annual Meeting of the American Mathematical
     Society, held in Denver, Colorado January 5--9, 1983)",
  year = 1983,
  note = "[QA76.9.A96.S64 1983]",
  publisher = "American Mathematical Society",
  address = "Providence, Rhode Island" }
@book{Boolos, author = "George S. Boolos and Richard C. Jeffrey",
  title = "Computability and Log\-ic",
  publisher = "Cambridge University Press",
  edition = "third",
  address = "Cambridge",
  note = "[QA9.59.B66 1989]",
  year = 1989 }
@book{Campbell, author = "John Campbell",
  title = "Programmer's Progress",
  publisher = "White Star Software",
  address = "Box 51623, Palo Alto, CA 94303",
  year = 1991 }
@article{DBLP:journals/corr/Carneiro14,
  author    = {Mario Carneiro},
  title     = {Conversion of {HOL} Light proofs into Metamath},
  journal   = {CoRR},
  volume    = {abs/1412.8091},
  year      = {2014},
  url       = {http://arxiv.org/abs/1412.8091},
  archivePrefix = {arXiv},
  eprint    = {1412.8091},
  timestamp = {Mon, 13 Aug 2018 16:47:05 +0200},
  biburl    = {https://dblp.org/rec/bib/journals/corr/Carneiro14},
  bibsource = {dblp computer science bibliography, https://dblp.org}
}
@article{CarneiroND,
  author    = {Mario Carneiro},
  title     = {Natural Deductions in the Metamath Proof Language},
  url       = {http://us.metamath.org/ocat/natded.pdf},
  year      = 2014
}
@inproceedings{Chou, author = "Shang-Ching Chou",
  title = "Proving Elementary Geometry Theorems Using {W}u's Algorithm",
  pages = "243--286",
  booktitle = "Automated Theorem Proving:  After 25 Years (Proceedings
     of the Special Session on Automatic Theorem Proving,
     89th Annual Meeting of the American Mathematical
     Society, held in Denver, Colorado January 5--9, 1983)",
  editor = "W. W. Bledsoe and D. W. Loveland",
  year = 1983,
  note = "[QA76.9.A96.S64 1983]",
  publisher = "American Mathematical Society",
  address = "Providence, Rhode Island" }
@book{Clemente, author = "Daniel Clemente Laboreo",
  title = "Introduction to natural deduction",
  year = 2014,
  url = "http://www.danielclemente.com/logica/dn.en.pdf" }
@incollection{Courant, author = "Richard Courant and Herbert Robbins",
  title = "Topology",
  pages = "573--590",
  booktitle = "The World of Mathematics, Volume One",
  editor = "James R. Newman",
  publisher = "Simon and Schuster",
  address = "New York",
  note = "[QA3.W67 1988]",
  year = 1956 }
@book{Curry, author = "Haskell B. Curry",
  title = "Foundations of Mathematical Logic",
  publisher = "Dover Publications, Inc.",
  address = "New York",
  note = "[QA9.C976 1977]",
  year = 1977 }
@book{Davis, author = "Philip J. Davis and Reuben Hersh",
  title = "The Mathematical Experience",
  publisher = "Birkh{\"{a}}user Boston",
  address = "Boston",
  note = "[QA8.4.D37 1982]",
  year = 1981 }
@incollection{deMillo,
  author = "Richard de Millo and Richard Lipton and Alan Perlis",
  title = "Social Processes and Proofs of Theorems and Programs",
  pages = "267--285",
  booktitle = "New Directions in the Philosophy of Mathematics",
  editor = "Thomas Tymoczko",
  publisher = "Birkh{\"{a}}user Boston, Inc.",
  address = "Boston",
  note = "[QA8.6.N48 1986]",
  year = 1986 }
@book{Edwards, author = "Robert E. Edwards",
  title = "A Formal Background to Mathematics",
  publisher = "Springer-Verlag",
  address = "New York",
  note = "[QA37.2.E38 v.1a]",
  year = 1979 }
@book{Enderton, author = "Herbert B. Enderton",
  title = "Elements of Set Theory",
  publisher = "Academic Press, Inc.",
  address = "San Diego",
  note = "[QA248.E5]",
  year = 1977 }
@book{Goodstein, author = "R. L. Goodstein",
  title = "Development of Mathematical Logic",
  publisher = "Springer-Verlag New York Inc.",
  address = "New York",
  note = "[QA9.G6554]",
  year = 1971 }
@book{Guillen, author = "Michael Guillen",
  title = "Bridges to Infinity",
  publisher = "Jeremy P. Tarcher, Inc.",
  address = "Los Angeles",
  note = "[QA93.G8]",
  year = 1983 }
@book{Hamilton, author = "Alan G. Hamilton",
  title = "Logic for Mathematicians",
  edition = "revised",
  publisher = "Cambridge University Press",
  address = "Cambridge",
  note = "[QA9.H298]",
  year = 1988 }
@unpublished{Harrison, author = "John Robert Harrison",
  title = "Metatheory and Reflection in Theorem Proving:
    A Survey and Critique",
  note = "Technical Report
    CRC-053.
    SRI Cambridge,
    Millers Yard, Cambridge, UK,
    1995.
    Available on the Web as
{\verb+http:+}\-{\verb+//www.cl.cam.ac.uk/users/jrh/papers/reflect.html+}"}
@TECHREPORT{Harrison-thesis,
        author          = "John Robert Harrison",
        title           = "Theorem Proving with the Real Numbers",
        institution   = "University of Cambridge Computer
                         Lab\-o\-ra\-to\-ry",
        address         = "New Museums Site, Pembroke Street, Cambridge,
                           CB2 3QG, UK",
        year            = 1996,
        number          = 408,
        type            = "Technical Report",
        note            = "Author's PhD thesis,
   available on the Web at
{\verb+http:+}\-{\verb+//www.cl.cam.ac.uk+}\-{\verb+/users+}\-{\verb+/jrh+}%
\-{\verb+/papers+}\-{\verb+/thesis.html+}"}
@book{Herrlich, author = "Horst Herrlich and George E. Strecker",
  title = "Category Theory:  An Introduction",
  publisher = "Allyn and Bacon Inc.",
  address = "Boston",
  note = "[QA169.H567]",
  year = 1973 }
@article{Hindley, author = "J. Roger Hindley and David Meredith",
  title = "Principal Type-Schemes and Condensed Detachment",
  journal = "The Journal of Symbolic Logic",
  volume = 55,
  year = 1990,
  note = "[QA.J87]",
  pages = "90--105" }
@book{Hofstadter, author = "Douglas R. Hofstadter",
  title = "G{\"{o}}del, Escher, Bach",
  publisher = "Basic Books, Inc.",
  address = "New York",
  note = "[QA9.H63 1980]",
  year = 1979 }
@article{Indrzejczak, author= "Andrzej Indrzejczak",
  title = "Natural Deduction, Hybrid Systems and Modal Logic",
  journal = "Trends in Logic",
  volume = 30,
  publisher = "Springer",
  year = 2010 }
@article{Kalish, author = "D. Kalish and R. Montague",
  title = "On {T}arski's Formalization of Predicate Logic with Identity",
  journal = "Archiv f{\"{u}}r Mathematische Logik und Grundlagenfor\-schung",
  volume = 7,
  year = 1965,
  note = "[QA.A673]",
  pages = "81--101" }
@article{Kalman, author = "J. A. Kalman",
  title = "Condensed Detachment as a Rule of Inference",
  journal = "Studia Logica",
  volume = 42,
  number = 4,
  year = 1983,
  note = "[B18.P6.S933]",
  pages = "443-451" }
@book{Kline, author = "Morris Kline",
  title = "Mathematical Thought from Ancient to Modern Times",
  publisher = "Oxford University Press",
  address = "New York",
  note = "[QA21.K516 1990 v.3]",
  year = 1972 }
@book{Klinel, author = "Morris Kline",
  title = "Mathematics, The Loss of Certainty",
  publisher = "Oxford University Press",
  address = "New York",
  note = "[QA21.K525]",
  year = 1980 }
@book{Kramer, author = "Edna E. Kramer",
  title = "The Nature and Growth of Modern Mathematics",
  publisher = "Princeton University Press",
  address = "Princeton, New Jersey",
  note = "[QA93.K89 1981]",
  year = 1981 }
@article{Knill, author = "Oliver Knill",
  title = "Some Fundamental Theorems in Mathematics",
  year = "2018",
  url = "https://arxiv.org/abs/1807.08416" }
@book{Landau, author = "Edmund Landau",
  title = "Foundations of Analysis",
  publisher = "Chelsea Publishing Company",
  address = "New York",
  edition = "second",
  note = "[QA241.L2541 1960]",
  year = 1960 }
@article{Leblanc, author = "Hugues Leblanc",
  title = "On {M}eyer and {L}ambert's Quantificational Calculus {FQ}",
  journal = "The Journal of Symbolic Logic",
  volume = 33,
  year = 1968,
  note = "[QA.J87]",
  pages = "275--280" }
@article{Lejewski, author = "Czeslaw Lejewski",
  title = "On Implicational Definitions",
  journal = "Studia Logica",
  volume = 8,
  year = 1958,
  note = "[B18.P6.S933]",
  pages = "189--208" }
@book{Levy, author = "Azriel Levy",
  title = "Basic Set Theory",
  publisher = "Dover Publications",
  address = "Mineola, NY",
  year = "2002"
}
@book{Margaris, author = "Angelo Margaris",
  title = "First Order Mathematical Logic",
  publisher = "Blaisdell Publishing Company",
  address = "Waltham, Massachusetts",
  note = "[QA9.M327]",
  year = 1967}
@book{Manin, author = "Yu I. Manin",
  title = "A Course in Mathematical Logic",
  publisher = "Springer-Verlag",
  address = "New York",
  note = "[QA9.M29613]",
  year = "1977" }
@article{Mathias, author = "Adrian R. D. Mathias",
  title = "A Term of Length 4,523,659,424,929",
  journal = "Synthese",
  volume = 133,
  year = 2002,
  note = "[Q.S993]",
  pages = "75--86" }
@article{Megill, author = "Norman D. Megill",
  title = "A Finitely Axiomatized Formalization of Predicate Calculus
     with Equality",
  journal = "Notre Dame Journal of Formal Logic",
  volume = 36,
  year = 1995,
  note = "[QA.N914]",
  pages = "435--453" }
@unpublished{Megillc, author = "Norman D. Megill",
  title = "A Shorter Equivalent of the Axiom of Choice",
  month = "June",
  note = "Unpublished",
  year = 1991 }
@article{MegillBunder, author = "Norman D. Megill and Martin W.
    Bunder",
  title = "Weaker {D}-Complete Logics",
  journal = "Journal of the IGPL",
  volume = 4,
  year = 1996,
  pages = "215--225",
  note = "Available on the Web at
{\verb+http:+}\-{\verb+//www.mpi-sb.mpg.de+}\-{\verb+/igpl+}%
\-{\verb+/Journal+}\-{\verb+/V4-2+}\-{\verb+/#Megill+}"}
}
@book{Mendelson, author = "Elliott Mendelson",
  title = "Introduction to Mathematical Logic",
  edition = "second",
  publisher = "D. Van Nostrand Company, Inc.",
  address = "New York",
  note = "[QA9.M537 1979]",
  year = 1979 }
@article{Meredith, author = "David Meredith",
  title = "In Memoriam {C}arew {A}rthur {M}eredith (1904-1976)",
  journal = "Notre Dame Journal of Formal Logic",
  volume = 18,
  year = 1977,
  note = "[QA.N914]",
  pages = "513--516" }
@article{CAMeredith, author = "C. A. Meredith",
  title = "Single Axioms for the Systems ({C},{N}), ({C},{O}) and ({A},{N})
      of the Two-Valued Propositional Calculus",
  journal = "The Journal of Computing Systems",
  volume = 3,
  year = 1953,
  pages = "155--164" }
@article{Monk, author = "J. Donald Monk",
  title = "Provability With Finitely Many Variables",
  journal = "The Journal of Symbolic Logic",
  volume = 27,
  year = 1971,
  note = "[QA.J87]",
  pages = "353--358" }
@article{Monks, author = "J. Donald Monk",
  title = "Substitutionless Predicate Logic With Identity",
  journal = "Archiv f{\"{u}}r Mathematische Logik und Grundlagenfor\-schung",
  volume = 7,
  year = 1965,
  pages = "103--121" }
  %% Took out this from above to prevent LaTeX underfull warning:
  % note = "[QA.A673]",
@book{Moore, author = "A. W. Moore",
  title = "The Infinite",
  publisher = "Routledge",
  address = "New York",
  note = "[BD411.M59]",
  year = 1989}
@book{Munkres, author = "James R. Munkres",
  title = "Topology: A First Course",
  publisher = "Prentice-Hall, Inc.",
  address = "Englewood Cliffs, New Jersey",
  note = "[QA611.M82]",
  year = 1975}
@article{Nemesszeghy, author = "E. Z. Nemesszeghy and E. A. Nemesszeghy",
  title = "On Strongly Creative Definitions:  A Reply to {V}. {F}. {R}ickey",
  journal = "Logique et Analyse (N.\ S.)",
  year = 1977,
  volume = 20,
  note = "[BC.L832]",
  pages = "111--115" }
@unpublished{Nemeti, author = "N{\'{e}}meti, I.",
  title = "Algebraizations of Quantifier Logics, an Overview",
  note = "Version 11.4, preprint, Mathematical Institute, Budapest,
    1994.  A shortened version without proofs appeared in
    ``Algebraizations of quantifier logics, an introductory overview,''
   {\em Studia Logica}, 50:485--569, 1991 [B18.P6.S933]"}
@article{Pavicic, author = "M. Pavi{\v{c}}i{\'{c}}",
  title = "A New Axiomatization of Unified Quantum Logic",
  journal = "International Journal of Theoretical Physics",
  year = 1992,
  volume = 31,
  note = "[QC.I626]",
  pages = "1753 --1766" }
@book{Penrose, author = "Roger Penrose",
  title = "The Emperor's New Mind",
  publisher = "Oxford University Press",
  address = "New York",
  note = "[Q335.P415]",
  year = 1989 }
@book{PetersonI, author = "Ivars Peterson",
  title = "The Mathematical Tourist",
  publisher = "W. H. Freeman and Company",
  address = "New York",
  note = "[QA93.P475]",
  year = 1988 }
@article{Peterson, author = "Jeremy George Peterson",
  title = "An automatic theorem prover for substitution and detachment systems",
  journal = "Notre Dame Journal of Formal Logic",
  volume = 19,
  year = 1978,
  note = "[QA.N914]",
  pages = "119--122" }
@book{Quine, author = "Willard Van Orman Quine",
  title = "Set Theory and Its Logic",
  edition = "revised",
  publisher = "The Belknap Press of Harvard University Press",
  address = "Cambridge, Massachusetts",
  note = "[QA248.Q7 1969]",
  year = 1969 }
@article{Robinson, author = "J. A. Robinson",
  title = "A Machine-Oriented Logic Based on the Resolution Principle",
  journal = "Journal of the Association for Computing Machinery",
  year = 1965,
  volume = 12,
  pages = "23--41" }
@article{RobinsonT, author = "T. Thacher Robinson",
  title = "Independence of Two Nice Sets of Axioms for the Propositional
    Calculus",
  journal = "The Journal of Symbolic Logic",
  volume = 33,
  year = 1968,
  note = "[QA.J87]",
  pages = "265--270" }
@book{Rucker, author = "Rudy Rucker",
  title = "Infinity and the Mind:  The Science and Philosophy of the
    Infinite",
  publisher = "Bantam Books, Inc.",
  address = "New York",
  note = "[QA9.R79 1982]",
  year = 1982 }
@book{Russell, author = "Bertrand Russell",
  title = "Mysticism and Logic, and Other Essays",
  publisher = "Barnes \& Noble Books",
  address = "Totowa, New Jersey",
  note = "[B1649.R963.M9 1981]",
  year = 1981 }
@article{Russell2, author = "Bertrand Russell",
  title = "Recent Work on the Principles of Mathematics",
  journal = "International Monthly",
  volume = 4,
  year = 1901,
  pages = "84"}
@article{Schmidt, author = "Eric Schmidt",
  title = "Reductions in Norman Megill's axiom system for complex numbers",
  url = "http://us.metamath.org/downloads/schmidt-cnaxioms.pdf",
  year = "2012" }
@book{Shoenfield, author = "Joseph R. Shoenfield",
  title = "Mathematical Logic",
  publisher = "Addison-Wesley Publishing Company, Inc.",
  address = "Reading, Massachusetts",
  year = 1967,
  note = "[QA9.S52]" }
@book{Smullyan, author = "Raymond M. Smullyan",
  title = "Theory of Formal Systems",
  publisher = "Princeton University Press",
  address = "Princeton, New Jersey",
  year = 1961,
  note = "[QA248.5.S55]" }
@book{Solow, author = "Daniel Solow",
  title = "How to Read and Do Proofs:  An Introduction to Mathematical
    Thought Process",
  publisher = "John Wiley \& Sons",
  address = "New York",
  year = 1982,
  note = "[QA9.S577]" }
@book{Stark, author = "Harold M. Stark",
  title = "An Introduction to Number Theory",
  publisher = "Markham Publishing Company",
  address = "Chicago",
  note = "[QA241.S72 1978]",
  year = 1970 }
@article{Swart, author = "E. R. Swart",
  title = "The Philosophical Implications of the Four-Color Problem",
  journal = "American Mathematical Monthly",
  year = 1980,
  volume = 87,
  month = "November",
  note = "[QA.A5125]",
  pages = "697--707" }
@book{Szpiro, author = "George G. Szpiro",
  title = "Poincar{\'{e}}'s Prize: The Hundred-Year Quest to Solve One
    of Math's Greatest Puzzles",
  publisher = "Penguin Books Ltd",
  address = "London",
  note = "[QA43.S985 2007]",
  year = 2007}
@book{Takeuti, author = "Gaisi Takeuti and Wilson M. Zaring",
  title = "Introduction to Axiomatic Set Theory",
  edition = "second",
  publisher = "Springer-Verlag New York Inc.",
  address = "New York",
  note = "[QA248.T136 1982]",
  year = 1982}
@inproceedings{Tarski, author = "Alfred Tarski",
  title = "What is Elementary Geometry",
  pages = "16--29",
  booktitle = "The Axiomatic Method, with Special Reference to Geometry and
     Physics (Proceedings of an International Symposium held at the University
     of California, Berkeley, December 26, 1957 --- January 4, 1958)",
  editor = "Leon Henkin and Patrick Suppes and Alfred Tarski",
  year = 1959,
  publisher = "North-Holland Publishing Company",
  address = "Amsterdam"}
@article{Tarski1965, author = "Alfred Tarski",
  title = "A Simplified Formalization of Predicate Logic with Identity",
  journal = "Archiv f{\"{u}}r Mathematische Logik und Grundlagenforschung",
  volume = 7,
  year = 1965,
  note = "[QA.A673]",
  pages = "61--79" }
@book{Tymoczko,
  title = "New Directions in the Philosophy of Mathematics",
  editor = "Thomas Tymoczko",
  publisher = "Birkh{\"{a}}user Boston, Inc.",
  address = "Boston",
  note = "[QA8.6.N48 1986]",
  year = 1986 }
@incollection{Wang,
  author = "Hao Wang",
  title = "Theory and Practice in Mathematics",
  pages = "129--152",
  booktitle = "New Directions in the Philosophy of Mathematics",
  editor = "Thomas Tymoczko",
  publisher = "Birkh{\"{a}}user Boston, Inc.",
  address = "Boston",
  note = "[QA8.6.N48 1986]",
  year = 1986 }
@manual{Webster,
  title = "Webster's New Collegiate Dictionary",
  organization = "G. \& C. Merriam Co.",
  address = "Springfield, Massachusetts",
  note = "[PE1628.W4M4 1977]",
  year = 1977 }
@manual{Whitehead, author = "Alfred North Whitehead",
  title = "An Introduction to Mathematics",
  year = 1911 }
@book{PM, author = "Alfred North Whitehead and Bertrand Russell",
  title = "Principia Mathematica",
  edition = "second",
  publisher = "Cambridge University Press",
  address = "Cambridge",
  year = "1927",
  note = "(3 vols.) [QA9.W592 1927]" }
@article{DBLP:journals/corr/Whalen16,
  author    = {Daniel Whalen},
  title     = {Holophrasm: a neural Automated Theorem Prover for higher-order logic},
  journal   = {CoRR},
  volume    = {abs/1608.02644},
  year      = {2016},
  url       = {http://arxiv.org/abs/1608.02644},
  archivePrefix = {arXiv},
  eprint    = {1608.02644},
  timestamp = {Mon, 13 Aug 2018 16:46:19 +0200},
  biburl    = {https://dblp.org/rec/bib/journals/corr/Whalen16},
  bibsource = {dblp computer science bibliography, https://dblp.org} }
@article{Wiedijk-revisited,
  author = {Freek Wiedijk},
  title = {The QED Manifesto Revisited},
  year = {2007},
  url = {http://mizar.org/trybulec65/8.pdf} }
@book{Wolfram,
  author = "Stephen Wolfram",
  title = "Mathematica:  A System for Doing Mathematics by Computer",
  edition = "second",
  publisher = "Addison-Wesley Publishing Co.",
  address = "Redwood City, California",
  note = "[QA76.95.W65 1991]",
  year = 1991 }
@book{Wos, author = "Larry Wos and Ross Overbeek and Ewing Lusk and Jim Boyle",
  title = "Automated Reasoning:  Introduction and Applications",
  edition = "second",
  publisher = "McGraw-Hill, Inc.",
  address = "New York",
  note = "[QA76.9.A96.A93 1992]",
  year = 1992 }

%
%
%[1] Church, Alonzo, Introduction to Mathematical Logic,
% Volume 1, Princeton University Press, Princeton, N. J., 1956.
%
%[2] Cohen, Paul J., Set Theory and the Continuum Hypothesis,
% W. A. Benjamin, Inc., Reading, Mass., 1966.
%
%[3] Hamilton, Alan G., Logic for Mathematicians, Cambridge
% University Press,
% Cambridge, 1988.

%[6] Kleene, Stephen Cole, Introduction to Metamathematics, D.  Van
% Nostrand Company, Inc., Princeton (1952).

%[13] Tarski, Alfred, "A simplified formalization of predicate
% logic with identity," Archiv fur Mathematische Logik und
% Grundlagenforschung, vol. 7 (1965), pp. 61-79.

%[14] Tarski, Alfred and Steven Givant, A Formalization of Set
% Theory Without Variables, American Mathematical Society Colloquium
% Publications, vol. 41, American Mathematical Society,
% Providence, R. I., 1987.

%[15] Zeman, J. J., Modal Logic, Oxford University Press, Oxford, 1973.
\end{filecontents}
% --------------------------- End of metamath.bib -----------------------------


%Book: Metamath
%Author:  Norman Megill Email:  nm at alum.mit.edu
%Author:  David A. Wheeler Email:  dwheeler at dwheeler.com

% A book template example
% http://www.stsci.edu/ftp/software/tex/bookstuff/book.template

\documentclass[leqno]{book} % LaTeX 2e. 10pt. Use [leqno,12pt] for 12pt
% hyperref 2002/05/27 v6.72r  (couldn't get pagebackref to work)
\usepackage[plainpages=false,pdfpagelabels=true]{hyperref}

\usepackage{needspace}     % Enable control over page breaks
\usepackage{breqn}         % automatic equation breaking
\usepackage{microtype}     % microtypography, reduces hyphenation

% Packages for flexible tables.  We need to be able to
% wrap text within a cell (with automatically-determined widths) AND
% split a table automatically across multiple pages.
% * "tabularx" wraps text in cells but only 1 page
% * "longtable" goes across pages but by itself is incompatible with tabularx
% * "ltxtable" combines longtable and tabularx, but table contents
%    must be in a separate file.
% * "ltablex" combines tabularx and longtable - must install specially
% * "booktabs" is recommended as a way to improve the look of tables,
%   but doesn't add these capabilities.
% * "tabu" much more capable and seems to be recommended. So use that.

\usepackage{makecell}      % Enable forced line splits within a table cell
% v4.13 needed for tabu: https://tex.stackexchange.com/questions/600724/dimension-too-large-after-recent-longtable-update
\usepackage{longtable}[=v4.13] % Enable multi-page tables  
\usepackage{tabu}          % Multi-page tables with wrapped text in a cell

% You can find more Tex packages using commands like:
% tlmgr search --file tabu.sty
% find /usr/share/texmf-dist/ -name '*tab*'
%
%%%%%%%%%%%%%%%%%%%%%%%%%%%%%%%%%%%%%%%%%%%%%%%%%%%%%%%%%%%%%%%%%%%%%%%%%%%%
% Uncomment the next 3 lines to suppress boxes and colors on the hyperlinks
%%%%%%%%%%%%%%%%%%%%%%%%%%%%%%%%%%%%%%%%%%%%%%%%%%%%%%%%%%%%%%%%%%%%%%%%%%%%
%\hypersetup{
%colorlinks,citecolor=black,filecolor=black,linkcolor=black,urlcolor=black
%}
%
\usepackage{realref}

% Restarting page numbers: try?
%   \printglossary
%   \cleardoublepage
%   \pagenumbering{arabic}
%   \setcounter{page}{1}    ???needed
%   \include{chap1}

% not used:
% \def\R2Lurl#1#2{\mbox{\href{#1}\texttt{#2}}}

\usepackage{amssymb}

% Version 1 of book: margins: t=.4, b=.2, ll=.4, rr=.55
% \usepackage{anysize}
% % \papersize{<height>}{<width>}
% % \marginsize{<left>}{<right>}{<top>}{<bottom>}
% \papersize{9in}{6in}
% % l/r 0.6124-0.6170 works t/b 0.2418-0.3411 = 192pp. 0.2926-03118=exact
% \marginsize{0.7147in}{0.5147in}{0.4012in}{0.2012in}

\usepackage{anysize}
% \papersize{<height>}{<width>}
% \marginsize{<left>}{<right>}{<top>}{<bottom>}
\papersize{9in}{6in}
% l/r 0.85in&0.6431-0.6539 works t/b ?-?
%\marginsize{0.85in}{0.6485in}{0.55in}{0.35in}
\marginsize{0.8in}{0.65in}{0.5in}{0.3in}

% \usepackage[papersize={3.6in,4.8in},hmargin=0.1in,vmargin={0.1in,0.1in}]{geometry}  % page geometry
\usepackage{special-settings}

\raggedbottom
\makeindex

\begin{document}
% Discourage page widows and orphans:
\clubpenalty=300
\widowpenalty=300

%%%%%%% load in AMS fonts %%%%%%% % LaTeX 2.09 - obsolete in LaTeX 2e
%\input{amssym.def}
%\input{amssym.tex}
%\input{c:/texmf/tex/plain/amsfonts/amssym.def}
%\input{c:/texmf/tex/plain/amsfonts/amssym.tex}

\bibliographystyle{plain}
\pagenumbering{roman}
\pagestyle{headings}

\thispagestyle{empty}

\hfill
\vfill

\begin{center}
{\LARGE\bf Metamath} \\
\vspace{1ex}
{\large A Computer Language for Mathematical Proofs} \\
\vspace{7ex}
{\large Norman Megill} \\
\vspace{7ex}
with extensive revisions by \\
\vspace{1ex}
{\large David A. Wheeler} \\
\vspace{7ex}
% Printed date. If changing the date below, also fix the date at the beginning.
2019-06-02
\end{center}

\vfill
\hfill

\newpage
\thispagestyle{empty}

\hfill
\vfill

\begin{center}
$\sim$\ {\sc Public Domain}\ $\sim$

\vspace{2ex}
This book (including its later revisions)
has been released into the Public Domain by Norman Megill per the
Creative Commons CC0 1.0 Universal (CC0 1.0) Public Domain Dedication.
David A. Wheeler has done the same.
This public domain release applies worldwide.  In case this is not
legally possible, the right is granted to use the work for any purpose,
without any conditions, unless such conditions are required by law.
See \url{https://creativecommons.org/publicdomain/zero/1.0/}.

\vspace{3ex}
Several short, attributed quotations from copyrighted works
appear in this book under the ``fair use'' provision of Section 107 of
the United States Copyright Act (Title 17 of the {\em United States
Code}).  The public-domain status of this book is not applicable to
those quotations.

\vspace{3ex}
Any trademarks used in this book are the property of their owners.

% QA76.9.L63.M??

% \vspace{1ex}
%
% \vspace{1ex}
% {\small Permission is granted to make and distribute verbatim copies of this
% book
% provided the copyright notice and this
% permission notice are preserved on all copies.}
%
% \vspace{1ex}
% {\small Permission is granted to copy and distribute modified versions of this
% book under the conditions for verbatim copying, provided that the
% entire
% resulting derived work is distributed under the terms of a permission
% notice
% identical to this one.}
%
% \vspace{1ex}
% {\small Permission is granted to copy and distribute translations of this
% book into another language, under the above conditions for modified
% versions,
% except that this permission notice may be stated in a translation
% approved by the
% author.}
%
% \vspace{1ex}
% %{\small   For a copy of the \LaTeX\ source files for this book, contact
% %the author.} \\
% \ \\
% \ \\

\vspace{7ex}
% ISBN: 1-4116-3724-0 \\
% ISBN: 978-1-4116-3724-5 \\
ISBN: 978-0-359-70223-7 \\
{\ } \\
Lulu Press \\
Morrisville, North Carolina\\
USA


\hfill
\vfill

Norman Megill\\ 93 Bridge St., Lexington, MA 02421 \\
E-mail address: \texttt{nm{\char`\@}alum.mit.edu} \\
\vspace{7ex}
David A. Wheeler \\
E-mail address: \texttt{dwheeler{\char`\@}dwheeler.com} \\
% See notes added at end of Preface for revision history. \\
% For current information on the Metamath software see \\
\vspace{7ex}
\url{http://metamath.org}
\end{center}

\hfill
\vfill

{\parindent0pt%
\footnotesize{%
Cover: Aleph null ($\aleph_0$) is the symbol for the
first infinite cardinal number, discovered by Georg Cantor in 1873.
We use a red aleph null (with dark outline and gold glow) as the Metamath logo.
Credit: Norman Megill (1994) and Giovanni Mascellani (2019),
public domain.%
\index{aleph null}%
\index{Metamath!logo}\index{Cantor, Georg}\index{Mascellani, Giovanni}}}

% \newpage
% \thispagestyle{empty}
%
% \hfill
% \vfill
%
% \begin{center}
% {\it To my son Robin Dwight Megill}
% \end{center}
%
% \vfill
% \hfill
%
% \newpage

\tableofcontents
%\listoftables

\chapter*{Preface}
\markboth{PREFACE}{PREFACE}
\addcontentsline{toc}{section}{Preface}


% (For current information, see the notes added at the
% end of this preface on p.~\pageref{note2002}.)

\subsubsection{Overview}

Metamath\index{Metamath} is a computer language and an associated computer
program for archiving, verifying, and studying mathematical proofs at a very
detailed level.  The Metamath language incorporates no mathematics per se but
treats all mathematical statements as mere sequences of symbols.  You provide
Metamath with certain special sequences (axioms) that tell it what rules
of inference are allowed.  Metamath is not limited to any specific field of
mathematics.  The Metamath language is simple and robust, with an
almost total absence of hard-wired syntax, and
we\footnote{Unless otherwise noted, the words
``I,'' ``me,'' and ``my'' refer to Norman Megill\index{Megill, Norman}, while
``we,'' ``us,'' and ``our'' refer to Norman Megill and
David A. Wheeler\index{Wheeler, David A.}.}
believe that it
provides about the simplest possible framework that allows essentially all of
mathematics to be expressed with absolute rigor.

% index test
%\newcommand{\nn}[1]{#1n}
%\index{aaa@bbb}
%\index{abc!def}
%\index{abd|see{qqq}}
%\index{abe|nn}
%\index{abf|emph}
%\index{abg|(}
%\index{abg|)}

Using the Metamath language, you can build formal or mathematical
systems\index{formal system}\footnote{A formal or mathematical system consists
of a collection of symbols (such as $2$, $4$, $+$ and $=$), syntax rules that
describe how symbols may be combined to form a legal expression (called a
well-formed formula or {\em wff}, pronounced ``whiff''), some starting wffs
called axioms, and inference rules that describe how theorems may be derived
(proved) from the axioms.  A theorem is a mathematical fact such as $2+2=4$.
Strictly speaking, even an obvious fact such as this must be proved from
axioms to be formally acceptable to a mathematician.}\index{theorem}
\index{axiom}\index{rule}\index{well-formed formula (wff)} that involve
inferences from axioms.  Although a database is provided
that includes a recommended set of axioms for standard mathematics, if you
wish you can supply your own symbols, syntax, axioms, rules, and definitions.

The name ``Metamath'' was chosen to suggest that the language provides a
means for {\em describing} mathematics rather than {\em being} the
mathematics itself.  Actually in some sense any mathematical language is
metamathematical.  Symbols written on paper, or stored in a computer,
are not mathematics itself but rather a way of expressing mathematics.
For example ``7'' and ``VII'' are symbols for denoting the number seven
in Arabic and Roman numerals; neither {\em is} the number seven.

If you are able to understand and write computer programs, you should be able
to follow abstract mathematics with the aid of Metamath.  Used in conjunction
with standard textbooks, Metamath can guide you step-by-step towards an
understanding of abstract mathematics from a very rigorous viewpoint, even if
you have no formal abstract mathematics background.  By using a single,
consistent notation to express proofs, once you grasp its basic concepts
Metamath provides you with the ability to immediately follow and dissect
proofs even in totally unfamiliar areas.

Of course, just being able follow a proof will not necessarily give you an
intuitive familiarity with mathematics.  Memorizing the rules of chess does not
give you the ability to appreciate the game of a master, and knowing how the
notes on a musical score map to piano keys does not give you the ability to
hear in your head how it would sound.  But each of these can be a first step.

Metamath allows you to explore proofs in the sense that you can see the
theorem referenced at any step expanded in as much detail as you want, right
down to the underlying axioms of logic and set theory (in the case of the set
theory database provided).  While Metamath will not replace the higher-level
understanding that can only be acquired through exercises and hard work, being
able to see how gaps in a proof are filled in can give you increased
confidence that can speed up the learning process and save you time when you
get stuck.

The Metamath language breaks down a mathematical proof into its tiniest
possible parts.  These can be pieced together, like interlocking
pieces in a puzzle, only in a way that produces correct and absolutely rigorous
mathematics.

The nature of Metamath\index{Metamath} enforces very precise mathematical
thinking, similar to that involved in writing a computer program.  A crucial
difference, though, is that once a proof is verified (by the Metamath program)
to be correct, it is definitely correct; it can never have a hidden
``bug.''\index{computer program bugs}  After getting used to the kind of rigor
and accuracy provided by Metamath, you might even be tempted to
adopt the attitude that a proof should never be considered correct until it
has been verified by a computer, just as you would not completely trust a
manual calculation until you have verified it on a
calculator.

My goal
for Metamath was a system for describing and verifying
mathematics that is completely universal yet conceptually as simple as
possible.  In approaching mathematics from an axiomatic, formal viewpoint, I
wanted Metamath to be able to handle almost any mathematical system, not
necessarily with ease, but at least in principle and hopefully in practice. I
wanted it to verify proofs with absolute rigor, and for this reason Metamath
is what might be thought of as a ``compile-only'' language rather than an
algorithmic or Turing-machine language (Pascal, C, Prolog, Mathematica,
etc.).  In other words, a database written in the Metamath
language doesn't ``do'' anything; it merely exhibits mathematical knowledge
and permits this knowledge to be verified as being correct.  A program in an
algorithmic language can potentially have hidden bugs\index{computer program
bugs} as well as possibly being hard to understand.  But each token in a
Metamath database must be consistent with the database's earlier
contents according to simple, fixed rules.
If a database is verified
to be correct,\footnote{This includes
verification that a sequential list of proof steps results in the specified
theorem.} then the mathematical content is correct if the
verifier is correct and the axioms are correct.
The verification program could be incorrect, but the verification algorithm
is relatively simple (making it unlikely to be implemented incorrectly
by the Metamath program),
and there are over a dozen Metamath database verifiers
written by different people in different programming languages
(so these different verifiers can act as multiple reviewers of a database).
The most-used Metamath database, the Metamath Proof Explorer
(aka \texttt{set.mm}\index{set theory database (\texttt{set.mm})}%
\index{Metamath Proof Explorer}),
is currently verified by four different Metamath verifiers written by
four different people in four different languages, including the
original Metamath program described in this book.
The only ``bugs'' that can exist are in the statement of the axioms,
for example if the axioms are inconsistent (a famous problem shown to be
unsolvable by G\"{o}del's incompleteness theorem\index{G\"{o}del's
incompleteness theorem}).
However, real mathematical systems have very few axioms, and these can
be carefully studied.
All of this provides extraordinarily high confidence that the verified database
is in fact correct.

The Metamath program
doesn't prove theorems automatically but is designed to verify proofs
that you supply to it.
The underlying Metamath language is completely general and has no built-in,
preconceived notions about your formal system\index{formal system}, its logic
or its syntax.
For constructing proofs, the Metamath program has a Proof Assistant\index{Proof
Assistant} which helps you fill in some of a proof step's details, shows you
what choices you have at any step, and verifies the proof as you build it; but
you are still expected to provide the proof.

There are many other programs that can process or generate information
in the Metamath language, and more continue to be written.
This is in part because the Metamath language itself is very simple
and intentionally easy to automatically process.
Some programs, such as \texttt{mmj2}\index{mmj2}, include a proof assistant
that can automate some steps beyond what the Metamath program can do.
Mario Carneiro has developed an algorithm for converting proofs from
the OpenTheory interchange format, which can be translated to and from
any of the HOL family of proof languages (HOL4, HOL Light, ProofPower,
and Isabelle), into the
Metamath language \cite{DBLP:journals/corr/Carneiro14}\index{Carneiro, Mario}.
Daniel Whalen has developed Holophrasm, which can automatically
prove many Metamath proofs using
machine learning\index{machine learning}\index{artificial intelligence}
approaches
(including multiple neural networks\index{neural networks})\cite{DBLP:journals/corr/Whalen16}\index{Whalen, Daniel}.
However,
a discussion of these other programs is beyond the scope of this book.

Like most computer languages, the Metamath\index{Metamath} language uses the
standard ({\sc ascii}) characters on a computer keyboard, so it cannot
directly represent many of the special symbols that mathematicians use.  A
useful feature of the Metamath program is its ability to convert its notation
into the \LaTeX\ typesetting language.\index{latex@{\LaTeX}}  This feature
lets you convert the {\sc ascii} tokens you've defined into standard
mathematical symbols, so you end up with symbols and formulas you are familiar
with instead of somewhat cryptic {\sc ascii} representations of them.
The Metamath program can also generate HTML\index{HTML}, making it easy
to view results on the web and to see related information by using
hypertext links.

Metamath is probably conceptually different from anything you've seen
before and some aspects may take some getting used to.  This book will
help you decide whether Metamath suits your specific needs.

\subsubsection{Setting Your Expectations}
It is important for you to understand what Metamath\index{Metamath} is and is
not.  As mentioned, the Metamath program
is {\em not} an automated theorem prover but
rather a proof verifier.  Developing a database can be tedious, hard work,
especially if you want to make the proofs as short as possible, but it becomes
easier as you build up a collection of useful theorems.  The purpose of
Metamath is simply to document existing mathematics in an absolutely rigorous,
computer-verifiable way, not to aid directly in the creation of new
mathematics.  It also is not a magic solution for learning abstract
mathematics, although it may be helpful to be able to actually see the implied
rigor behind what you are learning from textbooks, as well as providing hints
to work out proofs that you are stumped on.

As of this writing, a sizable set theory database has been developed to
provide a foundation for many fields of mathematics, but much more work would
be required to develop useful databases for specific fields.

Metamath\index{Metamath} ``knows no math;'' it just provides a framework in
which to express mathematics.  Its language is very small.  You can define two
kinds of symbols, constants\index{constant} and variables\index{variable}.
The only thing Metamath knows how to do is to substitute strings of symbols
for the variables\index{substitution!variable}\index{variable substitution} in
an expression based on instructions you provide it in a proof, subject to
certain constraints you specify for the variables.  Even the decimal
representation of a number is merely a string of certain constants (digits)
which together, in a specific context, correspond to whatever mathematical
object you choose to define for it; unlike other computer languages, there is
no actual number stored inside the computer.  In a proof, you in effect
instruct Metamath what symbol substitutions to make in previous axioms or
theorems and join a sequence of them together to result in the desired
theorem.  This kind of symbol manipulation captures the essence of mathematics
at a preaxiomatic level.

\subsubsection{Metamath and Mathematical Literature}

In advanced mathematical literature, proofs are usually presented in the form
of short outlines that often only an expert can follow.  This is partly out of
a desire for brevity, but it would also be unwise (even if it were practical)
to present proofs in complete formal detail, since the overall picture would
be lost.\index{formal proof}

A solution I envision\label{envision} that would allow mathematics to remain
acceptable to the expert, yet increase its accessibility to non-specialists,
consists of a combination of the traditional short, informal proof in print
accompanied by a complete formal proof stored in a computer database.  In an
analogy with a computer program, the informal proof is like a set of comments
that describe the overall reasoning and content of the proof, whereas the
computer database is like the actual program and provides a means for anyone,
even a non-expert, to follow the proof in as much detail as desired, exploring
it back through layers of theorems (like subroutines that call other
subroutines) all the way back to the axioms of the theory.  In addition, the
computer database would have the advantage of providing absolute assurance
that the proof is correct, since each step can be verified automatically.

There are several other approaches besides Metamath to a project such
as this.  Section~\ref{proofverifiers} discusses some of these.

To us, a noble goal would be a database with hundreds of thousands of
theorems and their computer-verifiable proofs, encompassing a significant
fraction of known mathematics and available for instant access.
These would be fully verified by multiple independently-implemented verifiers,
to provide extremely high confidence that the proofs are completely correct.
The database would allow people to investigate whatever details they were
interested in, so that they could confirm whatever portions they wished.
Whether or not Metamath is an appropriate choice remains to be seen, but in
principle we believe it is sufficient.

\subsubsection{Formalism}

Over the past fifty years, a group of French mathematicians working
collectively under the pseudonym of Bourbaki\index{Bourbaki, Nicolas} have
co-authored a series of monographs that attempt to rigorously and
consistently formalize large bodies of mathematics from foundations.  On the
one hand, certainly such an effort has its merits; on the other hand, the
Bourbaki project has been criticized for its ``scholasticism'' and
``hyperaxiomatics'' that hide the intuitive steps that lead to the results
\cite[p.~191]{Barrow}\index{Barrow, John D.}.

Metamath unabashedly carries this philosophy to its extreme and no doubt is
subject to the same kind of criticism.  Nonetheless I think that in
conjunction with conventional approaches to mathematics Metamath can serve a
useful purpose.  The Bourbaki approach is essentially pedagogic, requiring the
reader to become intimately familiar with each detail in a very large
hierarchy before he or she can proceed to the next step.  The difference with
Metamath is that the ``reader'' (user) knows that all details are contained in
its computer database, available as needed; it does not demand that the user
know everything but conveniently makes available those portions that are of
interest.  As the body of all mathematical knowledge grows larger and larger,
no one individual can have a thorough grasp of its entirety.  Metamath
can finalize and put to rest any questions about the validity of any part of it
and can make any part of it accessible, in principle, to a non-specialist.

\subsubsection{A Personal Note}
Why did I develop Metamath\index{Metamath}?  I enjoy abstract mathematics, but
I sometimes get lost in a barrage of definitions and start to lose confidence
that my proofs are correct.  Or I reach a point where I lose sight of how
anything I'm doing relates to the axioms that a theory is based on and am
sometimes suspicious that there may be some overlooked implicit axiom
accidentally introduced along the way (as happened historically with Euclidean
geometry\index{Euclidean geometry}, whose omission of Pasch's
axiom\index{Pasch's axiom} went unnoticed for 2000 years
\cite[p.~160]{Davis}!). I'm also somewhat lazy and wish to avoid the effort
involved in re-verifying the gaps in informal proofs ``left to the reader;'' I
prefer to figure them out just once and not have to go through the same
frustration a year from now when I've forgotten what I did.  Metamath provides
better recovery of my efforts than scraps of paper that I can't
decipher anymore.  But mostly I find very appealing the idea of rigorously
archiving mathematical knowledge in a computer database, providing precision,
certainty, and elimination of human error.

\subsubsection{Note on Bibliography and Index}

The Bibliography usually includes the Library of Congress classification
for a work to make it easier for you to find it in on a university
library shelf.  The Index has author references to pages where their works
are cited, even though the authors' names may not appear on those pages.

\subsubsection{Acknowledgments}

Acknowledgments are first due to my wife, Deborah (who passed away on
September 4, 1998), for critiquing the manu\-script but most of all for
her patience and support.  I also wish to thank Joe Wright, Richard
Becker, Clarke Evans, Buddha Buck, and Jeremy Henty for helpful
comments.  Any errors, omissions, and other shortcomings are of course
my responsibility.

\subsubsection{Note Added June 22, 2005}\label{note2002}

The original, unpublished version of this book was written in 1997 and
distributed via the web.  The present edition has been updated to
reflect the current Metamath program and databases, as well as more
current {\sc url}s for Internet sites.  Thanks to Josh
Purinton\index{Purinton, Josh}, One Hand
Clapping, Mel L.\ O'Cat, and Roy F. Longton for pointing out
typographical and other errors.  I have also benefitted from numerous
discussions with Raph Levien\index{Levien, Raph}, who has extended
Metamath's philosophy of rigor to result in his {\em
Ghilbert}\index{Ghilbert} proof language (\url{http://ghilbert.org}).

Robert (Bob) Solovay\index{Solovay, Robert} communicated a new result of
A.~R.~D.~Mathias on the system of Bourbaki, and the text has been
updated accordingly (p.~\pageref{bourbaki}).

Bob also pointed out a clarification of the literature regarding
category theory and inaccessible cardinals\index{category
theory}\index{cardinal, inaccessible} (p.~\pageref{categoryth}),
and a misleading statement was removed from the text.  Specifically,
contrary to a statement in previous editions, it is possible to express
``There is a proper class of inaccessible cardinals'' in the language of
ZFC.  This can be done as follows:  ``For every set $x$ there is an
inaccessible cardinal $\kappa$ such that $\kappa$ is not in $x$.''
Bob writes:\footnote{Private communication, Nov.~30, 2002.}
\begin{quotation}
     This axiom is how Grothendieck presents category theory.  To each
inaccessible cardinal $\kappa$ one associates a Grothendieck universe
\index{Grothendieck, Alexander} $U(\kappa)$.  $U(\kappa)$ consists of
those sets which lie in a transitive set of cardinality less than
$\kappa$.  Instead of the ``category of all groups,'' one works relative
to a universe [considering the category of groups of cardinality less
than $\kappa$].  Now the category whose objects are all categories
``relative to the universe $U(\kappa)$'' will be a category not
relative to this universe but to the next universe.

     All of the things category theorists like to do can be done in this
framework.  The only controversial point is whether the Grothen\-dieck
axiom is too strong for the needs of category theorists.  Mac Lane
\index{Mac Lane, Saunders} argues that ``one universe is enough'' and
Feferman\index{Feferman, Solomon} has argued that one can get by with
ordinary ZFC.  I don't find Feferman's arguments persuasive.  Mac Lane
may be right, but when I think about category theory I do it \`{a} la
Grothendieck.

        By the way Mizar\index{Mizar} adds the axiom ``there is a proper
class of inaccessibles'' precisely so as to do category theory.
\end{quotation}

The most current information on the Metamath program and databases can
always be found at \url{http://metamath.org}.


\subsubsection{Note Added June 24, 2006}\label{note2006}

The Metamath spec was restricted slightly to make parsers easier to
write.  See the footnote on p.~\pageref{namespace}.

%\subsubsection{Note Added July 24, 2006}\label{note2006b}
\subsubsection{Note Added March 10, 2007}\label{note2006b}

I am grateful to Anthony Williams\index{Williams, Anthony} for writing
the \LaTeX\ package called {\tt realref.sty} and contributing it to the
public domain.  This package allows the internal hyperlinks in a {\sc
pdf} file to anchor to specific page numbers instead of just section
titles, making the navigation of the {\sc pdf} file for this book much
more pleasant and ``logical.''

A typographical error found by Martin Kiselkov was corrected.
A confusing remark about unification was deleted per suggestion of
Mel O'Cat.

\subsubsection{Note Added May 27, 2009}\label{note2009}

Several typos found by Kim Sparre were corrected.  A note was added that
the Poincar\'{e} conjecture has been proved (p.~\pageref{poincare}).

\subsubsection{Note Added Nov. 17, 2014}\label{note2014}

The statement of the Schr\"{o}der--Bernstein theorem was corrected in
Section~\ref{trust}.  Thanks to Bob Solovay for pointing out the error.

\subsubsection{Note Added May 25, 2016}\label{note2016}

Thanks to Jerry James for correcting 16 typos.

\subsubsection{Note Added February 25, 2019}\label{note201902}

David A. Wheeler\index{Wheeler, David A.}
made a large number of improvements and updates,
in coordination with Norman Megill.
The predicate calculus axioms were renumbered, and the text makes
it clear that they are based on Tarski's system S2;
the one slight deviation in axiom ax-6 is explained and justified.
The real and complex number axioms were modified to be consistent with
\texttt{set.mm}\index{set theory database (\texttt{set.mm})}%
\index{Metamath Proof Explorer}.
Long-awaited specification changes ``1--8'' were made,
which clarified previously ambiguous points.
Some errors in the text involving \texttt{\$f} and
\texttt{\$d} statements were corrected (the spec was correct, but
the in-book explanations unintentionally contradicted the spec).
We now have a system for automatically generating narrow PDFs,
so that those with smartphones can have easy access to the current
version of this document.
A new section on deduction was added;
it discusses the standard deduction theorem,
the weak deduction theorem,
deduction style, and natural deduction.
Many minor corrections (too numerous to list here) were also made.

\subsubsection{Note Added March 7, 2019}\label{note201903}

This added a description of the Matamath language syntax in
Extended Backus--Naur Form (EBNF)\index{Extended Backus--Naur Form}\index{EBNF}
in Appendix \ref{BNF}, added a brief explanation about typecodes,
inserted more examples in the deduction section,
and added a variety of smaller improvements.

\subsubsection{Note Added April 7, 2019}\label{note201904}

This version clarified the proper substitution notation, improved the
discussion on the weak deduction theorem and natural deduction,
documented the \texttt{undo} command, updated the information on
\texttt{write source}, changed the typecode
from \texttt{set} to \texttt{setvar} to be consistent with the current
version of \texttt{set.mm}, added more documentation about comment markup
(e.g., documented how to create headings), and clarified the
differences between various assertion forms (in particular deduction form).

\subsubsection{Note Added June 2, 2019}\label{note201906}

This version fixes a large number of small issues reported by
Beno\^{i}t Jubin\index{Jubin, Beno\^{i}t}, such as editorial issues
and the need to document \texttt{verify markup} (thank you!).
This version also includes specific examples
of forms (deduction form, inference form, and closed form).
We call this version the ``second edition'';
the previous edition formally published in 2007 had a slightly different title
(\textit{Metamath: A Computer Language for Pure Mathematics}).

\chapter{Introduction}
\pagenumbering{arabic}

\begin{quotation}
  {\em {\em I.M.:}  No, no.  There's nothing subjective about it!  Everybody
knows what a proof is.  Just read some books, take courses from a competent
mathematician, and you'll catch on.

{\em Student:}  Are you sure?

{\em I.M.:}  Well---it is possible that you won't, if you don't have any
aptitude for it.  That can happen, too.

{\em Student:}  Then {\em you} decide what a proof is, and if I don't learn
to decide in the same way, you decide I don't have any aptitude.

{\em I.M.:}  If not me, then who?}
    \flushright\sc  ``The Ideal Mathematician''
    \index{Davis, Phillip J.}
    \footnote{\cite{Davis}, p.~40.}\\
\end{quotation}

Brilliant mathematicians have discovered almost
unimaginably profound results that rank among the crowning intellectual
achievements of mankind.  However, there is a sense in which modern abstract
mathematics is behind the times, stuck in an era before computers existed.
While no one disputes the remarkable results that have been achieved,
communicating these results in a precise way to the uninitiated is virtually
impossible.  To describe these results, a terse informal language is used which
despite its elegance is very difficult to learn.  This informal language is not
imprecise, far from it, but rather it often has omitted detail
and symbols with hidden context that are
implicitly understood by an expert but few others.  Extremely complex technical
meanings are associated with innocent-sounding English words such as
``compact'' and ``measurable'' that barely hint at what is actually being
said.  Anyone who does not keep the precise technical meaning constantly in
mind is bound to fail, and acquiring the ability to do this can be achieved
only through much practice and hard work.  Only the few who complete the
painful learning experience can join the small in-group of pure
mathematicians.  The informal language effectively cuts off the true nature of
their knowledge from most everyone else.

Metamath\index{Metamath} makes abstract mathematics more concrete.  It allows
a computer to keep track of the complexity associated with each word or symbol
with absolute rigor.  You can explore this complexity at your leisure, to
whatever degree you desire.  Whether or not you believe that concepts such as
infinity actually ``exist'' outside of the mind, Metamath lets you get to the
foundation for what's really being said.

Metamath also enables completely rigorous and thorough proof verification.
Its language is simple enough so that you
don't have to rely on the authority of experts but can verify the results
yourself, step by step.  If you want to attempt to derive your own results,
Metamath will not let you make a mistake in reasoning.
Even professional mathematicians make mistakes; Metamath makes it possible
to thoroughly verify that proofs are correct.

Metamath\index{Metamath} is a computer language and an associated computer
program for archiving, verifying, and studying mathematical proofs at a very
detailed level.
The Metamath language
describes formal\index{formal system} mathematical
systems and expresses proofs of theorems in those systems.  Such a language
is called a metalanguage\index{metalanguage} by mathematicians.
The Metamath program is a computer program that verifies
proofs expressed in the Metamath language.
The Metamath program does not have the built-in
ability to make logical inferences; it just makes a series of symbol
substitutions according to instructions given to it in a proof
and verifies that the result matches the expected theorem.  It makes logical
inferences based only on rules of logic that are contained in a set of
axioms\index{axiom}, or first principles, that you provide to it as the
starting point for proofs.

The complete specification of the Metamath language is only four pages long
(Section~\ref{spec}, p.~\pageref{spec}).  Its simplicity may at first make you
wonder how it can do much of anything at all.  But in fact the kinds of
symbol manipulations it performs are the ones that are implicitly done in all
mathematical systems at the lowest level.  You can learn it relatively quickly
and have complete confidence in any mathematical proof that it verifies.  On
the other hand, it is powerful and general enough so that virtually any
mathematical theory, from the most basic to the deeply abstract, can be
described with it.

Although in principle Metamath can be used with any
kind of mathematics, it is best suited for abstract or ``pure'' mathematics
that is mostly concerned with theorems and their proofs, as opposed to the
kind of mathematics that deals with the practical manipulation of numbers.
Examples of branches of pure mathematics are logic\index{logic},\footnote{Logic
is the study of statements that are universally true regardless of the objects
being described by the statements.  An example is the statement, ``if $P$
implies $Q$, then either $P$ is false or $Q$ is true.''} set theory\index{set
theory},\footnote{Set theory is the study of general-purpose mathematical objects called
``sets,'' and from it essentially all of mathematics can be derived.  For
example, numbers can be defined as specific sets, and their properties
can be explored using the tools of set theory.} number theory\index{number
theory},\footnote{Number theory deals with the properties of positive and
negative integers (whole numbers).} group theory\index{group
theory},\footnote{Group theory studies the properties of mathematical objects
called groups that obey a simple set of axioms and have properties of symmetry
that make them useful in many other fields.} abstract algebra\index{abstract
algebra},\footnote{Abstract algebra includes group theory and also studies
groups with additional properties that qualify them as ``rings'' and
``fields.''  The set of real numbers is a familiar example of a field.},
analysis\index{analysis} \index{real and complex numbers}\footnote{Analysis is
the study of real and complex numbers.} and
topology\index{topology}.\footnote{One area studied by topology are properties
that remain unchanged when geometrical objects undergo stretching
deformations; for example a doughnut and a coffee cup each have one hole (the
cup's hole is in its handle) and are thus considered topologically
equivalent.  In general, though, topology is the study of abstract
mathematical objects that obey a certain (surprisingly simple) set of axioms.
See, for example, Munkres \cite{Munkres}\index{Munkres, James R.}.} Even in
physics, Metamath could be applied to certain branches that make use of
abstract mathematics, such as quantum logic\index{quantum logic} (used to study
aspects of quantum mechanics\index{quantum mechanics}).

On the other hand, Metamath\index{Metamath} is less suited to applications
that deal primarily with intensive numeric computations.  Metamath does not
have any built-in representation of numbers\index{Metamath!representation of
numbers}; instead, a specific string of symbols (digits) must be syntactically
constructed as part of any proof in which an ordinary number is used.  For
this reason, numbers in Metamath are best limited to specific constants that
arise during the course of a theorem or its proof.  Numbers are only a tiny
part of the world of abstract mathematics.  The exclusion of built-in numbers
was a conscious decision to help achieve Metamath's simplicity, and there are
other software tools if you have different mathematical needs.
If you wish to quickly solve algebraic problems, the computer algebra
programs\index{computer algebra system} {\sc
macsyma}\index{macsyma@{\sc macsyma}}, Mathematica\index{Mathematica}, and
Maple\index{Maple} are specifically suited to handling numbers and
algebra efficiently.
If you wish to simply calculate numeric or matrix expressions easily,
tools such as Octave\index{Octave} may be a better choice.

After learning Metamath's basic statement types, any
tech\-ni\-cal\-ly ori\-ent\-ed person, mathematician or not, can
immediately trace
any theorem proved in the language as far back as he or she wants, all the way
to the axioms on which the theorem is based.  This ability suggests a
non-traditional way of learning about pure mathematics.  Used in conjunction
with traditional methods, Metamath could make pure mathematics accessible to
people who are not sufficiently skilled to figure out the implicit detail in
ordinary textbook proofs.  Once you learn the axioms of a theory, you can have
complete confidence that everything you need to understand a proof you are
studying is all there, at your beck and call, allowing you to focus in on any
proof step you don't understand in as much depth as you need, without worrying
about getting stuck on a step you can't figure out.\footnote{On the other
hand, writing proofs in the Metamath language is challenging, requiring
a degree of rigor far in excess of that normally taught to students.  In a
classroom setting, I doubt that writing Metamath proofs would ever replace
traditional homework exercises involving informal proofs, because the time
needed to work out the details would not allow a course to
cover much material.  For students who have trouble grasping the implied rigor
in traditional material, writing a few simple proofs in the Metamath language
might help clarify fuzzy thought processes.  Although somewhat difficult at
first, it eventually becomes fun to do, like solving a puzzle, because of the
instant feedback provided by the computer.}

Metamath\index{Metamath} is probably unlike anything you have
encountered before.  In this first chapter we will look at the philosophy and
use of computers in mathematics in order to better understand the motivation
behind Metamath.  The material in this chapter is not required in order to use
Metamath.  You may skip it if you are impatient, but I hope you will find it
educational and enjoyable.  If you want to start experimenting with the
Metamath program right away, proceed directly to Chapter~\ref{using}
(p.~\pageref{using}).  To
learn the Metamath language, skim Chapter~\ref{using} then proceed to
Chapter~\ref{languagespec} (p.~\pageref{languagespec}).

\section{Mathematics as a Computer Language}

\begin{quote}
  {\em The study of mathematics is apt to commence in
dis\-ap\-point\-ment.\ldots \\
We are told that by its aid the stars are weighted
and the billions of molecules in a drop of water are counted.  Yet, like the
ghost of Hamlet's father, this great science eludes the efforts of our mental
weapons to grasp it.}
  \flushright\sc  Alfred North Whitehead\footnote{\cite{Whitehead}, ch.\ 1.}\\
\end{quote}\index{Whitehead, Alfred North}

\subsection{Is Mathematics ``User-Friendly''?}

Suppose you have no formal training in abstract mathematics.  But popular
books you've read offer tempting glimpses of this world filled with profound
ideas that have stirred the human spirit.  You are not satisfied with the
informal, watered-down descriptions you've read but feel it is important to
grasp the underlying mathematics itself to understand its true meaning. It's
not practical to go back to school to learn it, though; you don't want to
dedicate years of your life to it.  There are many important things in life,
and you have to set priorities for what's important to you.  What would happen
if you tried to pursue it on your own, in your spare time?

After all, you were able to learn a computer programming language such as
Pascal on your own without too much difficulty, even though you had no formal
training in computers.  You don't claim to be an expert in software design,
but you can write a passable program when necessary to suit your needs.  Even
more important, you know that you can look at anyone else's Pascal program, no
matter how complex, and with enough patience figure out exactly how it works,
even though you are not a specialist.  Pascal allows you do anything that a
computer can do, at least in principle.  Thus you know you have the ability,
in principle, to follow anything that a computer program can do:  you just
have to break it down into small enough pieces.

Here's an imaginary scenario of what might happen if you na\-ive\-ly a\-dopted
this same view of abstract mathematics and tried to pick it up on your own, in
a period of time comparable to, say, learning a computer programming
language.

\subsubsection{A Non-Mathematician's Quest for Truth}

\begin{quote}
  {\em \ldots my daughters have been studying (chemistry) for several
se\-mes\-ters, think they have learned differential and integral calculus in
school, and yet even today don't know why $x\cdot y=y\cdot x$ is true.}
  \flushright\sc  Edmund Landau\footnote{\cite{Landau}, p.~vi.}\\
\end{quote}\index{Landau, Edmund}

\begin{quote}
  {\em Minus times minus is plus,\\
The reason for this we need not discuss.}
  \flushright\sc W.\ H.\ Auden\footnote{As quoted in \cite{Guillen}, p.~64.}\\
\end{quote}\index{Auden, W.\ H.}\index{Guillen, Michael}

We'll suppose you are a technically oriented professional, perhaps an engineer, a
computer programmer, or a physicist, but probably not a mathematician.  You
consider yourself reasonably intelligent.  You did well in school, learning a
variety of methods and techniques in practical mathematics such as calculus and
differential equations.  But rarely did your courses get into anything
resembling modern abstract mathematics, and proofs were something that appeared
only occasionally in your textbooks, a kind of necessary evil that was
supposed to convince you of a certain key result.  Most of your
homework consisted of exercises that gave you practice in the techniques, and
you were hardly ever asked to come up with a proof of your own.

You find yourself curious about advanced, abstract mathematics.  You are
driven by an inner conviction that it is important to understand and
appreciate some of the most profound knowledge discovered by mankind.  But it
seems very hard to learn, something that only certain gifted longhairs can
access and understand.  You are frustrated that it seems forever cut off from
you.

Eventually your curiosity drives you to do something about it.
You set for yourself a goal of ``really'' understanding mathematics:  not just
how to manipulate equations in algebra or calculus according to cookbook
rules, but rather to gain a deep understanding of where those rules come from.
In fact, you're not thinking about this kind of ordinary mathematics at all,
but about a much more abstract, ethereal realm of pure mathematics, where
famous results such as G\"{o}del's incompleteness theorem\index{G\"{o}del's
incompleteness theorem} and Cantor's different kinds of infinities
reside.

You have probably read a number of popular books, with titles like {\em
Infinity and the Mind} \cite{Rucker}\index{Rucker, Rudy}, on topics such as
these.  You found them inspiring but at the same time somewhat
unsatisfactory.  They gave you a general idea of what these results are about,
but if someone asked you to prove them, you wouldn't have the faintest idea of
where to begin.   Sure, you could give the same overall outline that you
learned from the popular books; and in a general sort of way, you do have an
understanding.  But deep down inside, you know that there is a rigor that is
missing, that probably there are many subtle steps and pitfalls along the way,
and ultimately it seems you have to place your trust in the experts in the
field.  You don't like this; you want to be able to verify these results for
yourself.

So where do you go next?  As a first step, you decide to look up some of the
original papers on the theorems you are curious about, or better, obtain some
standard textbooks in the field.  You look up a theorem you want to
understand.  Sure enough, it's there, but it's expressed with strange
terms and odd symbols that mean absolutely nothing to you.  It might as well be written in
a foreign language you've never seen before, whose symbols are totally alien.
You look at the proof, and you haven't the foggiest notion what each step
means, much less how one step follows from another.  Well, obviously you have
a lot to learn if you want to understand this stuff.

You feel that you could probably understand it by
going back to college for another three to six years and getting a math
degree.  But that does not fit in with your career and the other things in
your life and would serve no practical purpose.  You decide to seek a quicker
path.  You figure you'll just trace your way back to the beginning, step by
step, as you would do with a computer program, until you understand it.  But
you quickly find that this is not possible, since you can't even understand
enough to know what you have to trace back to.

Maybe a different approach is in order---maybe you should start at the
beginning and work your way up.  First, you read the introduction to the book
to find out what the prerequisites are.  In a similar fashion, you trace your
way back through two or three more books, finally arriving at one that seems
to start at a beginning:  it lists the axioms of arithmetic.  ``Aha!'' you
naively think, ``This must be the starting point, the source of all mathematical
knowledge.'' Or at least the starting point for mathematics dealing with
numbers; you have to start somewhere and have no idea what the starting point
for other mathematics would be.  But the word ``axioms'' looks promising.  So
you eagerly read along and work through some elementary exercises at the
beginning of the book.  You feel vaguely bothered:  these
don't seem like axioms at all, at least not in the sense that you want to
think of axioms.  Axioms imply a starting point from which everything else can
be built up, according to precise rules specified in the axiom system.  Even
though you can understand the first few proofs in an informal way,
and are able to do some of the
exercises, it's hard to pin down precisely what the
rules are.   Sure, each step seems to follow logically from the others, but
exactly what does that mean?  Is the ``logic'' just a matter of common sense,
something vague that we all understand but can never quite state precisely?

You've spent a number of years, off and on, programming computers, and you
know that in the case of computer languages there is no question of what the
rules are---they are precise and crystal clear.  If you follow them, your
program will work, and if you don't, it won't.  No matter how complex a
program, it can always be broken down into simpler and simpler pieces, until
you can ultimately identify the bits that are moved around to perform a
specific function.  Some programs might require a lot of perseverance to
accomplish this, but if you focus on a specific portion of it, you don't even
necessarily have to know how the rest of it works. Shouldn't there be an
analogy in mathematics?

You decide to apply the ultimate test:  you ask yourself how a computer could
verify or ensure that the steps in these proofs follow from one another.
Certainly mathematics must be at least as precisely defined as a computer
language, if not more so; after all, computer science itself is based on it.
If you can get a computer to verify these proofs, then you should also be
able, in principle, to understand them yourself in a very crystal clear,
precise way.

You're in for a surprise:  you can conceive of no way to convert the
proofs, which are in English, to a form that the computer can understand.
The proofs are filled with phrases such as ``assume there exists a unique
$x$\ldots'' and ``given any $y$, let $z$ be the number such that\ldots''  This
isn't the kind of logic you are used to in computer programming, where
everything, even arithmetic, reduces to Boolean ones and zeroes if you care to
break it down sufficiently.  Even though you think you understand the proofs,
there seems to be some kind of higher reasoning involved rather than precise
rules that define how you manipulate the symbols in the axioms.  Whatever it
is, it just isn't obvious how you would express it to a computer, and the more
you think about it, the more puzzled and confused you get, to the point where
you even wonder whether {\em you} really understand it.  There's a lot more to
these axioms of arithmetic than meets the eye.

Nobody ever talked about this in school in your applied math and engineering
courses.  You just learned the rules they gave you, not quite understanding
how or why they worked, sometimes vaguely suspicious or uncertain of them, and
through homework problems and osmosis learned how to present solutions that
satisfied the instructor and earned you an ``A.''  Rarely did you actually
``prove'' anything in a rigorous way, and the math majors who did do stuff
like that seemed to be in a different world.

Of course, there are computer algebra programs that can do mathematics, and
rather impressively.  They can instantly solve the integrals that you
struggled with in freshman calculus, and do much, much more.  But when you
look at these programs, what you see is a big collection of algorithms and
techniques that evolved and were added to over time, along with some basic
software that manipulates symbols.  Each algorithm that is built in is the
result of someone's theorem whose proof is omitted; you just have to trust the
person who proved it and the person who programmed it in and hope there are no
bugs.\index{computer program bugs}  Somehow this doesn't seem to be the
essence of mathematics.  Although computer algebra systems can generate
theorems with amazing speed, they can't actually prove a single one of them.

After some puzzlement, you revisit some popular books on what mathematics is
all about.  Somewhere you read that all of mathematics is actually derived
from something called ``set theory.''  This is a little confusing, because
nowhere in the book that presented the axioms of arithmetic was there any
mention of set theory, or if there was, it seemed to be just a tool that helps
you describe things better---the set of even numbers, that sort of thing.  If
set theory is the basis for all mathematics, then why are additional axioms
needed for arithmetic?

Something is wrong but you're not sure what.  One of your friends is a pure
mathematician.  He knows he is unable to communicate to you what he does for a
living and seems to have little interest in trying.  You do know that for him,
proofs are what mathematics is all about. You ask him what a proof is, and he
essentially tells you that, while of course it's based on logic, really it's
something you learn by doing it over and over until you pick it up.  He refers
you to a book, {\em How to Read and Do Proofs} \cite{Solow}.\index{Solow,
Daniel}  Although this book helps you understand traditional informal proofs,
there is still something missing you can't seem to pin down yet.

You ask your friend how you would go about having a computer verify a proof.
At first he seems puzzled by the question; why would you want to do that?
Then he says it's not something that would make any sense to do, but he's
heard that you'd have to break the proof down into thousands or even millions
of individual steps to do such a thing, because the reasoning involved is at
such a high level of abstraction.  He says that maybe it's something you could
do up to a point, but the computer would be completely impractical once you
get into any meaningful mathematics.  There, the only way you can verify a
proof is by hand, and you can only acquire the ability to do this by
specializing in the field for a couple of years in grad school.  Anyway, he
thinks it all has to do with set theory, although he has never taken a formal
course in set theory but just learned what he needed as he went along.

You are intrigued and amazed.  Apparently a mathematician can grasp as a
single concept something that would take a computer a thousand or a million
steps to verify, and have complete confidence in it.  Each one of these
thousand or million steps must be absolutely correct, or else the whole proof
is meaningless.  If you added a million numbers by hand, would you trust the
result?  How do you really know that all these steps are correct, that there
isn't some subtle pitfall in one of these million steps, like a bug in a
computer program?\index{computer program bugs}  After all, you've read that
famous mathematicians have occasionally made mistakes, and you certainly know
you've made your share on your math homework problems in school.

You recall the analogy with a computer program.  Sure, you can understand what
a large computer program such as a word processor does, as a single high-level
concept or a small set of such concepts, but your ability to understand it in
no way ensures that the program is correct and doesn't have hidden bugs.  Even
if you wrote the program yourself you can't really know this; most large
programs that you've written have had bugs that crop up at some later date, no
matter how careful you tried to be while writing them.

OK, so now it seems the reason you can't figure out how to make a
computer verify proofs is because each step really corresponds to a
million small steps.  Well, you say, a computer can do a million
calculations in a second, so maybe it's still practical to do.  Now the
puzzle becomes how to figure out what the million steps are that each
English-language step corresponds to.  Your mathematician friend hasn't
a clue, but suggests that maybe you would find the answer by studying
set theory.  Actually, your friend thinks you're a little off the wall
for even wondering such a thing.  For him, this is not what mathematics
is all about.

The subject of set theory keeps popping up, so you decide it's
time to look it up.

You decide to start off on a careful footing, so you start reading a couple of
very elementary books on set theory.  A lot of it seems pretty obvious, like
intersections, subsets, and Venn diagrams.  You thumb through one of the
books; nowhere is anything about axioms mentioned. The other book relegates to
an appendix a brief discussion that mentions a set of axioms called
``Zermelo--Fraenkel set theory''\index{Zermelo--Fraenkel set theory} and states
them in English.  You look at them and have no idea what they really mean or
what you can do with them.  The comments in this appendix say that the purpose
of mentioning them is to expose you to the idea, but imply that they are not
necessary for basic understanding and that they are really the subject matter
of advanced treatments where fine points such as a certain paradox (Russell's
paradox\index{Russell's paradox}\footnote{Russell's paradox assumes that there
exists a set $S$ that is a collection of all sets that don't contain
themselves.  Now, either $S$ contains itself or it doesn't.  If it contains
itself, it contradicts its definition.  But if it doesn't contain itself, it
also contradicts its definition.  Russell's paradox is resolved in ZF set
theory by denying that such a set $S$ exists.}) are resolved.  Wait a
minute---shouldn't the axioms be a starting point, not an ending point?  If
there are paradoxes that arise without the axioms, how do you know you won't
stumble across one accidentally when using the informal approach?

And nowhere do these books describe how ``all of mathematics can be
derived from set theory'' which by now you've heard a few times.

You find a more advanced book on set theory.  This one actually lists the
axioms of ZF set theory in plain English on page one.  {\em Now} you think
your quest has ended and you've finally found the source of all mathematical
knowledge; you just have to understand what it means.  Here, in one place, is
the basis for all of mathematics!  You stare at the axioms in awe, puzzle over
them, memorize them, hoping that if you just meditate on them long enough they
will become clear.  Of course, you haven't the slightest idea how the rest of
mathematics is ``derived'' from them; in particular, if these are the axioms
of mathematics, then why do arithmetic, group theory, and so on need their own
axioms?

You start reading this advanced book carefully, pondering the meaning of every
word, because by now you're really determined to get to the bottom of this.
The first thing the book does is explain how the axioms came about, which was
to resolve Russell's paradox.\index{Russell's paradox}  In fact that seems to
be the main purpose of their existence; that they supposedly can be used to
derive all of mathematics seems irrelevant and is not even mentioned.  Well,
you go on.  You hope the book will explain to you clearly, step by step, how
to derive things from the axioms.  After all, this is the starting point of
mathematics, like a book that explains the basics of a computer programming
language.  But something is missing.  You find you can't even understand the
first proof or do the first exercise.  Symbols such as $\exists$ and $\forall$
permeate the page without any mention of where they came from or how to
manipulate them; the author assumes you are totally familiar with them and
doesn't even tell you what they mean.  By now you know that $\exists$ means
``there exists'' and $\forall$ means ``for all,'' but shouldn't the rules for
manipulating these symbols be part of the axioms?  You still have no idea
how you could even describe the axioms to a computer.

Certainly there is something much different here from the technical
literature you're used to reading.  A computer language manual almost
always explains very clearly what all the symbols mean, precisely what
they do, and the rules used for combining them, and you work your way up
from there.

After glancing at four or five other such books, you come to the realization
that there is another whole field of study that you need just to get to the
point at which you can understand the axioms of set theory.  The field is
called ``logic.''  In fact, some of the books did recommend it as a
prerequisite, but it just didn't sink in.  You assumed logic was, well, just
logic, something that a person with common sense intuitively understood.  Why
waste your time reading boring treatises on symbolic logic, the manipulation
of 1's and 0's that computers do, when you already know that?  But this is a
different kind of logic, quite alien to you.  The subject of {\sc nand} and
{\sc nor} gates is not even touched upon or in any case has to do with only a
very small part of this field.

So your quest continues.  Skimming through the first couple of introductory
books, you get a general idea of what logic is about and what quantifiers
(``for all,'' ``there exists'') mean, but you find their examples somewhat
trivial and mildly annoying (``all dogs are animals,'' ``some animals are
dogs,'' and such).  But all you want to know is what the rules are for
manipulating the symbols so you can apply them to set theory.  Some formulas
describing the relationships among quantifiers ($\exists$ and $\forall$) are
listed in tables, along with some verbal reasoning to justify them.
Presumably, if you want to find out if a formula is correct, you go through
this same kind of mental reasoning process, possibly using images of dogs and
animals. Intuitively, the formulas seem to make sense.  But when you ask
yourself, ``What are the rules I need to get a computer to figure out whether
this formula is correct?'', you still don't know.  Certainly you don't ask the
computer to imagine dogs and animals.

You look at some more advanced logic books.  Many of them have an introductory
chapter summarizing set theory, which turns out to be a prerequisite.  You
need logic to understand set theory, but it seems you also need set theory to
understand logic!  These books jump right into proving rather advanced
theorems about logic, without offering the faintest clue about where the logic
came from that allows them to prove these theorems.

Luckily, you come across an elementary book of logic that, halfway through,
after the usual truth tables and metaphors, presents in a clear, precise way
what you've been looking for all along: the axioms!  They're divided into
propositional calculus (also called sentential logic) and predicate calculus
(also called first-order logic),\index{first-order logic} with rules so simple
and crystal clear that now you can finally program a computer to understand
them.  Indeed, they're no harder than learning how to play a game of chess.
As far as what you seem to need is concerned, the whole book could have been
written in five pages!

{\em Now} you think you've found the ultimate source of mathematical
truth.  So---the axioms of mathematics consist of these axioms of logic,
together with the axioms of ZF set theory. (By now you've also been able to
figure out how to translate the ZF axioms from English into the
actual symbols of logic which you can now manipulate according to
precise, easy-to-understand rules.)

Of course, you still don't understand how ``all of mathematics can be
derived from set theory,'' but maybe this will reveal itself in due
course.

You eagerly set out to program the axioms and rules into a computer and start
to look at the theorems you will have to prove as the logic is developed.  All
sorts of important theorems start popping up:  the deduction
theorem,\index{deduction theorem} the substitution theorem,\index{substitution
theorem} the completeness theorem of propositional calculus,\index{first-order
logic!completeness} the completeness theorem of predicate calculus.  Uh-oh,
there seems to be trouble.  They all get harder and harder, and not one of
them can be derived with the axioms and rules of logic you've just been
handed.  Instead, they all require ``metalogic'' for their proofs, a kind of
mixture of logic and set theory that allows you to prove things {\em about}
the axioms and theorems of logic rather than {\em with} them.

You plow ahead anyway.  A month later, you've spent much of your
free time getting the computer to verify proofs in propositional calculus.
You've programmed in the axioms, but you've also had to program in the
deduction theorem, the substitution theorem, and the completeness theorem of
propositional calculus, which by now you've resigned yourself to treating as
rather complex additional axioms, since they can't be proved from the axioms
you were given.  You can now get the computer to verify and even generate
complete, rigorous, formal proofs\index{formal proof}.  Never mind that they
may have 100,000 steps---at least now you can have complete, absolute
confidence in them.  Unfortunately, the only theorems you have proved are
pretty trivial and you can easily verify them in a few minutes with truth
tables, if not by inspection.

It looks like your mathematician friend was right.  Getting the computer to do
serious mathematics with this kind of rigor seems almost hopeless.  Even
worse, it seems that the further along you get, the more ``axioms'' you have
to add, as each new theorem seems to involve additional ``metamathematical''
reasoning that hasn't been formalized, and none of it can be derived from the
axioms of logic.  Not only do the proofs keep growing exponentially as you get
further along, but the program to verify them keeps getting bigger and bigger
as you program in more ``metatheorems.''\index{metatheorem}\footnote{A
metatheorem is usually a statement that is too general to be directly provable
in a theory.  For example, ``if $n_1$, $n_2$, and $n_3$ are integers, then
$n_1+n_2+n_3$ is an integer'' is a theorem of number theory.  But ``for any
integer $k > 1$, if $n_1, \ldots, n_k$ are integers, then $n_1+\ldots +n_k$ is
an integer'' is a metatheorem, in other words a family of theorems, one for
every $k$.  The reason it is not a theorem is that the general sum $n_1+\ldots
+n_k$ (as a function of $k$) is not an operation that can be defined directly
in number theory.} The bugs\index{computer program bugs} that have cropped up
so far have already made you start to lose faith in the rigor you seem to have
achieved, and you know it's just going to get worse as your program gets larger.

\subsection{Mathematics and the Non-Specialist}

\begin{quote}
  {\em A real proof is not checkable by a machine, or even by any mathematician
not privy to the gestalt, the mode of thought of the particular field of
mathematics in which the proof is located.}
  \flushright\sc  Davis and Hersh\index{Davis, Phillip J.}
  \footnote{\cite{Davis}, p.~354.}\\
\end{quote}

The bulk of abstract or theoretical mathematics is ordinarily outside
the reach of anyone but a few specialists in each field who have completed
the necessary difficult internship in order to enter its coterie.  The
typical intelligent layperson has no reasonable hope of understanding much of
it, nor even the specialist mathematician of understanding other fields.  It
is like a foreign language that has no dictionary to look up the translation;
the only way you can learn it is by living in the country for a few years.  It
is argued that the effort involved in learning a specialty is a necessary
process for acquiring a deep understanding.  Of course, this is almost certainly
true if one is to make significant contributions to a field; in particular,
``doing'' proofs is probably the most important part of a mathematician's
training.  But is it also necessary to deny outsiders access to it?  Is it
necessary that abstract mathematics be so hard for a layperson to grasp?

A computer normally is of no help whatsoever.  Most published proofs are
actually just series of hints written in an informal style that requires
considerable knowledge of the field to understand.  These are the ``real
proofs'' referred to by Davis and Hersh.\index{informal proof}  There is an
implicit understanding that, in principle, such a proof could be converted to
a complete formal proof\index{formal proof}.  However, it is said that no one
would ever attempt such a conversion, even if they could, because that would
presumably require millions of steps (Section~\ref{dream}).  Unfortunately the
informal style automatically excludes the understanding of the proof
by anyone who hasn't gone through the necessary apprenticeship. The
best that the intelligent layperson can do is to read popular books about deep
and famous results; while this can be helpful, it can also be misleading, and
the lack of detail usually leaves the reader with no ability whatsoever to
explore any aspect of the field being described.

The statements of theorems often use sophisticated notation that makes them
inaccessible to the non-specialist.  For a non-specialist who wants to achieve
a deeper understanding of a proof, the process of tracing definitions and
lemmas back through their hierarchy\index{hierarchy} quickly becomes confusing
and discouraging.  Textbooks are usually written to train mathematicians or to
communicate to people who are already mathematicians, and large gaps in proofs
are often left as exercises to the reader who is left at an impasse if he or
she becomes stuck.

I believe that eventually computers will enable non-specialists and even
intelligent laypersons to follow almost any mathematical proof in any field.
Metamath is an attempt in that direction.  If all of mathematics were as
easily accessible as a computer programming language, I could envision
computer programmers and hobbyists who otherwise lack mathematical
sophistication exploring and being amazed by the world of theorems and proofs
in obscure specialties, perhaps even coming up with results of their own.  A
tremendous advantage would be that anyone could experiment with conjectures in
any field---the computer would offer instant feedback as to whether
an inference step was correct.

Mathematicians sometimes have to put up with the annoyance of
cranks\index{cranks} who lack a fundamental understanding of mathematics but
insist that their ``proofs'' of, say, Fermat's Last Theorem\index{Fermat's
Last Theorem} be taken seriously.  I think part of the problem is that these
people are misled by informal mathematical language, treating it as if they
were reading ordinary expository English and failing to appreciate the
implicit underlying rigor.  Such cranks are rare in the field of computers,
because computer languages are much more explicit, and ultimately the proof is
in whether a computer program works or not.  With easily accessible
computer-based abstract mathematics, a mathematician could say to a crank,
``don't bother me until you've demonstrated your claim on the computer!''

% 22-May-04 nm
% Attempt to move De Millo quote so it doesn't separate from attribution
% CHANGE THIS NUMBER (AND ELIMINATE IF POSSIBLE) WHEN ABOVE TEXT CHANGES
\vspace{-0.5em}

\subsection{An Impossible Dream?}\label{dream}

\begin{quote}
  {\em Even quite basic theorems would demand almost unbelievably vast
  books to display their proofs.}
    \flushright\sc  Robert E. Edwards\footnote{\cite{Edwards}, p.~68.}\\
\end{quote}\index{Edwards, Robert E.}

\begin{quote}
  {\em Oh, of course no one ever really {\em does} it.  It would take
  forever!  You just show that you could do it, that's sufficient.}
    \flushright\sc  ``The Ideal Mathematician''
    \index{Davis, Phillip J.}\footnote{\cite{Davis},
p.~40.}\\
\end{quote}

\begin{quote}
  {\em There is a theorem in the primitive notation of set theory that
  corresponds to the arithmetic theorem `$1000+2000=3000$'.  The formula
  would be forbiddingly long\ldots even if [one] knows the definitions
  and is asked to simplify the long formula according to them, chances are
  he will make errors and arrive at some incorrect result.}
    \flushright\sc  Hao Wang\footnote{\cite{Wang}, p.~140.}\\
\end{quote}\index{Wang, Hao}

% 22-May-04 nm
% Attempt to move De Millo quote so it doesn't separate from attribution
% CHANGE THIS NUMBER (AND ELIMINATE IF POSSIBLE) WHEN ABOVE TEXT CHANGES
\vspace{-0.5em}

\begin{quote}
  {\em The {\em Principia Mathematica} was the crowning achievement of the
  formalists.  It was also the deathblow of the formalist view.\ldots
  {[Rus\-sell]} failed, in three enormous volumes, to get beyond the elementary
  facts of arithmetic.  He showed what can be done in principle and what
  cannot be done in practice.  If the mathematical process were really
  one of strict, logical progression, we would still be counting our
  fingers.\ldots
  One theoretician estimates\ldots that a demonstration of one of
  Ramanujan's conjectures assuming set theory and elementary analysis would
  take about two thousand pages; the length of a deduction from first principles
  is nearly in\-con\-ceiv\-a\-ble\ldots The probabilists argue that\ldots any
  very long proof can at best be viewed as only probably correct\ldots}
  \flushright\sc Richard de Millo et. al.\footnote{\cite{deMillo}, pp.~269,
  271.}\\
\end{quote}\index{de Millo, Richard}

A number of writers have conveyed the impression that the kind of absolute
rigor provided by Metamath\index{Metamath} is an impossible dream, suggesting
that a complete, formal verification\index{formal proof} of a typical theorem
would take millions of steps in untold volumes of books.  Even if it could be
done, the thinking sometimes goes, all meaning would be lost in such a
monstrous, tedious verification.\index{informal proof}\index{proof length}

These writers assume, however, that in order to achieve the kind of complete
formal verification they desire one must break down a proof into individual
primitive steps that make direct reference to the axioms.  This is
not necessary.  There is no reason not to make use of previously proved
theorems rather than proving them over and over.

Just as important, definitions\index{definition} can be introduced along
the way, allowing very complex formulas to be represented with few
symbols.  Not doing this can lead to absurdly long formulas.  For
example, the mere statement of
G\"{o}del's incompleteness theorem\index{G\"{o}del's
incompleteness theorem}, which can be expressed with a small number of
defined symbols, would require about 20,000 primitive symbols to express
it.\index{Boolos, George S.}\footnote{George S.\ Boolos, lecture at
Massachusetts Institute of Technology, spring 1990.} An extreme example
is Bourbaki's\label{bourbaki} language for set theory, which requires
4,523,659,424,929 symbols plus 1,179,618,517,981 disambiguatory links
(lines connecting symbol pairs, usually drawn below or above the
formula) to express the number
``one'' \cite{Mathias}.\index{Mathias, Adrian R. D.}\index{Bourbaki,
Nicolas}
% http://www.dpmms.cam.ac.uk/~ardm/

A hierarchy\index{hierarchy} of theorems and definitions permits an
exponential growth in the formula sizes and primitive proof steps to be
described with only a linear growth in the number of symbols used.  Of course,
this is how ordinary informal mathematics is normally done anyway, but with
Metamath\index{Metamath} it can be done with absolute rigor and precision.

\subsection{Beauty}


\begin{quote}
  {\em No one shall be able to drive us from the paradise that Cantor has
created for us.}
   \flushright\sc  David Hilbert\footnote{As quoted in \cite{Moore}, p.~131.}\\
\end{quote}\index{Hilbert, David}

\needspace{3\baselineskip}
\begin{quote}
  {\em Mathematics possesses not only truth, but some supreme beauty ---a
  beauty cold and austere, like that of a sculpture.}
    \flushright\sc  Bertrand
    Russell\footnote{\cite{Russell}.}\\
\end{quote}\index{Russell, Bertrand}

\begin{quote}
  {\em Euclid alone has looked on Beauty bare.}
  \flushright\sc Edna Millay\footnote{As quoted in \cite{Davis}, p.~150.}\\
\end{quote}\index{Millay, Edna}

For most people, abstract mathematics is distant, strange, and
incomprehensible.  Many popular books have tried to convey some of the sense
of beauty in famous theorems.  But even an intelligent layperson is left with
only a general idea of what a theorem is about and is hardly given the tools
needed to make use of it.  Traditionally, it is only after years of arduous
study that one can grasp the concepts needed for deep understanding.
Metamath\index{Metamath} allows you to approach the proof of the theorem from
a quite different perspective, peeling apart the formulas and definitions
layer by layer until an entirely different kind of understanding is achieved.
Every step of the proof is there, pieced together with absolute precision and
instantly available for inspection through a microscope with a magnification
as powerful as you desire.

A proof in itself can be considered an object of beauty.  Constructing an
elegant proof is an art.  Once a famous theorem has been proved, often
considerable effort is made to find simpler and more easily understood
proofs.  Creating and communicating elegant proofs is a major concern of
mathematicians.  Metamath is one way of providing a common language for
archiving and preserving this information.

The length of a proof can, to a certain extent, be considered an
objective measure of its ``beauty,'' since shorter proofs are usually
considered more elegant.  In the set theory database
\texttt{set.mm}\index{set theory database (\texttt{set.mm})}%
\index{Metamath Proof Explorer}
provided with Metamath, one goal was to make all proofs as short as possible.

\needspace{4\baselineskip}
\subsection{Simplicity}

\begin{quote}
  {\em God made man simple; man's complex problems are of his own
  devising.}
    \flushright\sc Eccles. 7:29\footnote{Jerusalem Bible.}\\
\end{quote}\index{Bible}

\needspace{3\baselineskip}
\begin{quote}
  {\em God made integers, all else is the work of man.}
    \flushright\sc Leopold Kronecker\footnote{{\em Jahresbericht
	der Deutschen Mathematiker-Vereinigung }, vol. 2, p. 19.}\\
\end{quote}\index{Kronecker, Leopold}

\needspace{3\baselineskip}
\begin{quote}
  {\em For what is clear and easily comprehended attracts; the
  complicated repels.}
    \flushright\sc David Hilbert\footnote{As quoted in \cite{deMillo},
p.~273.}\\
\end{quote}\index{Hilbert, David}

The Metamath\index{Metamath} language is simple and Spartan.  Metamath treats
all mathematical expressions as simple sequences of symbols, devoid of meaning.
The higher-level or ``metamathematical'' notions underlying Metamath are about
as simple as they could possibly be.  Each individual step in a proof involves
a single basic concept, the substitution of an expression for a variable, so
that in principle almost anyone, whether mathematician or not, can
completely understand how it was arrived at.

In one of its most basic applications, Metamath\index{Metamath} can be used to
develop the foundations of mathematics\index{foundations of mathematics} from
the very beginning.  This is done in the set theory database that is provided
with the Metamath package and is the subject matter
of Chapter~\ref{fol}. Any language (a metalanguage\index{metalanguage})
used to describe mathematics (an object language\index{object language}) must
have a mathematical content of its own, but it is desirable to keep this
content down to a bare minimum, namely that needed to make use of the
inference rules specified by the axioms.  With any metalanguage there is a
``chicken and egg'' problem somewhat like circular reasoning:  you must assume
the validity of the mathematics of the metalanguage in order to prove the
validity of the mathematics of the object language.  The mathematical content
of Metamath itself is quite limited.  Like the rules of a game of chess, the
essential concepts are simple enough so that virtually anyone should be able to
understand them (although that in itself will not let you play like
a master).  The symbols that Metamath manipulates do not in themselves
have any intrinsic meaning.  Your interpretation of the axioms that you supply
to Metamath is what gives them meaning.  Metamath is an attempt to strip down
mathematical thought to its bare essence and show you exactly how the symbols
are manipulated.

Philosophers and logicians, with various motivations, have often thought it
important to study ``weak'' fragments of logic\index{weak logic}
\cite{Anderson}\index{Anderson, Alan Ross} \cite{MegillBunder}\index{Megill,
Norman}\index{Bunder, Martin}, other unconventional systems of logic (such as
``modal'' logic\index{modal logic} \cite[ch.\ 27]{Boolos}\index{Boolos, George
S.}), and quantum logic\index{quantum logic} in physics
\cite{Pavicic}\index{Pavi{\v{c}}i{\'{c}}, M.}.  Metamath\index{Metamath}
provides a framework in which such systems can be expressed, with an absolute
precision that makes all underlying metamathematical assumptions rigorous and
crystal clear.

Some schools of philosophical thought, for example
intuitionism\index{intuitionism} and constructivism\index{constructivism},
demand that the notions underlying any mathematical system be as simple and
concrete as possible.  Metamath should meet the requirements of these
philosophies.  Metamath must be taught the symbols, axioms\index{axiom}, and
rules\index{rule} for a specific theory, from the skeptical (such as
intuitionism\index{intuitionism}\footnote{Intuitionism does not accept the law
of excluded middle (``either something is true or it is not true'').  See
\cite[p.~xi]{Tymoczko}\index{Tymoczko, Thomas} for discussion and references
on this topic.  Consider the theorem, ``There exist irrational numbers $a$ and
$b$ such that $a^b$ is rational.''  An intuitionist would reject the following
proof:  If $\sqrt{2}^{\sqrt{2}}$ is rational, we are done.  Otherwise, let
$a=\sqrt{2}^{\sqrt{2}}$ and $b=\sqrt{2}$. Then $a^b=2$, which is rational.})
to the bold (such as the axiom of choice in set theory\footnote{The axiom of
choice\index{Axiom of Choice} asserts that given any collection of pairwise
disjoint nonempty sets, there exists a set that has exactly one element in
common with each set of the collection.  It is used to prove many important
theorems in standard mathematics.  Some philosophers object to it because it
asserts the existence of a set without specifying what the set contains
\cite[p.~154]{Enderton}\index{Enderton, Herbert B.}.  In one foundation for
mathematics due to Quine\index{Quine, Willard Van Orman}, that has not been
otherwise shown to be inconsistent, the axiom of choice turns out to be false
\cite[p.~23]{Curry}\index{Curry, Haskell B.}.  The \texttt{show
trace{\char`\_}back} command of the Metamath program allows you to find out
whether the axiom of choice, or any other axiom, was assumed by a
proof.}\index{\texttt{show trace{\char`\_}back} command}).

The simplicity of the Metamath language lets the algorithm (computer program)
that verifies the validity of a Metamath proof be straightforward and
robust.  You can have confidence that the theorems it verifies really can be
derived from your axioms.

\subsection{Rigor}

\begin{quote}
  {\em Rigor became a goal with the Greeks\ldots But the efforts to
  pursue rigor to the utmost have led to an impasse in which there is
  no longer any agreement on what it really means.  Mathematics remains
  alive and vital, but only on a pragmatic basis.}
    \flushright\sc  Morris Kline\footnote{\cite{Kline}, p.~1209.}\\
\end{quote}\index{Kline, Morris}

Kline refers to a much deeper kind of rigor than that which we will discuss in
this section.  G\"{o}del's incompleteness theorem\index{G\"{o}del's
incompleteness theorem} showed that it is impossible to achieve absolute rigor
in standard mathematics because we can never prove that mathematics is
consistent (free from contradictions).\index{consistent theory}  If
mathematics is consistent, we will never know it, but must rely on faith.  If
mathematics is inconsistent, the best we can hope for is that some clever
future mathematician will discover the inconsistency.  In this case, the
axioms would probably be revised slightly to eliminate the inconsistency, as
was done in the case of Russell's paradox,\index{Russell's paradox} but the
bulk of mathematics would probably not be affected by such a discovery.
Russell's paradox, for example, did not affect most of the remarkable results
achieved by 19th-century and earlier mathematicians.  It mainly invalidated
some of Gottlob Frege's\index{Frege, Gottlob} work on the foundations of
mathematics in the late 1800's; in fact Frege's work inspired Russell's
discovery.  Despite the paradox, Frege's work contains important concepts that
have significantly influenced modern logic.  Kline's {\em Mathematics, The
Loss of Certainty} \cite{Klinel}\index{Kline, Morris} has an interesting
discussion of this topic.

What {\em can} be achieved with absolute certainty\index{certainty} is the
knowledge that if we assume the axioms are consistent and true, then the
results derived from them are true.  Part of the beauty of mathematics is that
it is the one area of human endeavor where absolute certainty can be achieved
in this sense.  A mathematical truth will remain such for eternity.  However,
our actual knowledge of whether a particular statement is a mathematical truth
is only as certain as the correctness of the proof that establishes it.  If
the proof of a statement is questionable or vague, we can't have absolute
confidence in the truth that the statement claims.

Let us look at some traditional ways of expressing proofs.

Except in the field of formal logic\index{formal logic}, almost all
traditional proofs in mathematics are really not proofs at all, but rather
proof outlines or hints as to how to go about constructing the proof.  Many
gaps\index{gaps in proofs} are left for the reader to fill in. There are
several reasons for this.  First, it is usually assumed in mathematical
literature that the person reading the proof is a mathematician familiar with
the specialty being described, and that the missing steps are obvious to such
a reader or at least that the reader is capable of filling them in.  This
attitude is fine for professional mathematicians in the specialty, but
unfortunately it often has the drawback of cutting off the rest of the world,
including mathematicians in other specialties, from understanding the proof.
We discussed one possible resolution to this on p.~\pageref{envision}.
Second, it is often assumed that a complete formal proof\index{formal proof}
would require countless millions of symbols (Section~\ref{dream}). This might
be true if the proof were to be expressed directly in terms of the axioms of
logic and set theory,\index{set theory} but it is usually not true if we allow
ourselves a hierarchy\index{hierarchy} of definitions and theorems to build
upon, using a notation that allows us to introduce new symbols, definitions,
and theorems in a precisely specified way.

Even in formal logic,\index{formal logic} formal proofs\index{formal proof}
that are considered complete still contain hidden or implicit information.
For example, a ``proof'' is usually defined as a sequence of
wffs,\index{well-formed formula (wff)}\footnote{A {\em wff} or well-formed
formula is a mathematical expression (string of symbols) constructed according
to some precise rules.  A formal mathematical system\index{formal system}
contains (1) the rules for constructing syntactically correct
wffs,\index{syntax rules} (2) a list of starting wffs called
axioms,\index{axiom} and (3) one or more rules prescribing how to derive new
wffs, called theorems\index{theorem}, from the axioms or previously derived
theorems.  An example of such a system is contained in
Metamath's\index{Metamath} set theory database, which defines a formal
system\index{formal system} from which all of standard mathematics can be
derived.  Section~\ref{startf} steps you through a complete example of a formal
system, and you may want to skim it now if you are unfamiliar with the
concept.} each of which is an axiom or follows from a rule applied to previous
wffs in the sequence.  The implicit part of the proof is the algorithm by
which a sequence of symbols is verified to be a valid wff, given the
definition of a wff.  The algorithm in this case is rather simple, but for a
computer to verify the proof,\index{automated proof verification} it must have
the algorithm built into its verification program.\footnote{It is possible, of
course, to specify wff construction syntax outside of the program itself
with a suitable input language (the Metamath language being an example), but
some proof-verification or theorem-proving programs lack the ability to extend
wff syntax in such a fashion.} If one deals exclusively with axioms and
elementary wffs, it is straightforward to implement such an algorithm.  But as
more and more definitions are added to the theory in order to make the
expression of wffs more compact, the algorithm becomes more and more
complicated.  A computer program that implements the algorithm becomes larger
and harder to understand as each definition is introduced, and thus more prone
to bugs.\index{computer program bugs}  The larger the program, the
more suspicious the mathematician may be about
the validity of its algorithms.  This is especially true because
computer programs are inherently hard to follow to begin with, and few people
enjoy verifying them manually in detail.

Metamath\index{Metamath} takes a different approach.  Metamath's ``knowledge''
is limited to the ability to substitute variables for expressions, subject to
some simple constraints.  Once the basic algorithm of Metamath is assumed to
be debugged, and perhaps independently confirmed, it
can be trusted once and for all.  The information that Metamath needs to
``understand'' mathematics is contained entirely in the body of knowledge
presented to Metamath.  Any errors in reasoning can only be errors in the
axioms or definitions contained in this body of knowledge.  As a
``constructive'' language\index{constructive language} Metamath has no
conditional branches or loops like the ones that make computer programs hard
to decipher; instead, the language can only build new sequences of symbols
from earlier sequences  of symbols.

The simplicity of the rules that underlie Metamath not only makes Metamath
easy to learn but also gives Metamath a great deal of flexibility. For
example, Metamath is not limited to describing standard first-order
logic\index{first-order logic}; higher-order logics\index{higher-order logic}
and fragments of logic\index{weak logic} can be described just as easily.
Metamath gives you the freedom to define whatever wff notation you prefer; it
has no built-in conception of the syntax of a wff.\index{well-formed formula
(wff)}  With suitable axioms and definitions, Metamath can even describe and
prove things about itself.\index{Metamath!self-description}  (John
Harrison\index{Harrison, John} discusses the ``reflection''
principle\index{reflection principle} involved in self-descriptive systems in
\cite{Harrison}.)

The flexibility of Metamath requires that its proofs specify a lot of detail,
much more than in an ordinary ``formal'' proof.\index{formal proof}  For
example, in an ordinary formal proof, a single step consists of displaying the
wff that constitutes that step.  In order for a computer program to verify
that the step is acceptable, it first must verify that the symbol sequence
being displayed is an acceptable wff.\index{automated proof verification} Most
proof verifiers have at least basic wff syntax built into their programs.
Metamath has no hard-wired knowledge of what constitutes a wff built into it;
instead every wff must be explicitly constructed based on rules defining wffs
that are present in a database.  Thus a single step in an ordinary formal
proof may be correspond to many steps in a Metamath proof. Despite the larger
number of steps, though, this does not mean that a Metamath proof must be
significantly larger than an ordinary formal proof. The reason is that since
we have constructed the wff from scratch, we know what the wff is, so there is
no reason to display it.  We only need to refer to a sequence of statements
that construct it.  In a sense, the display of the wff in an ordinary formal
proof is an implicit proof of its own validity as a wff; Metamath just makes
the proof explicit. (Section~\ref{proof} describes Metamath's proof notation.)

\section{Computers and Mathematicians}

\begin{quote}
  {\em The computer is important, but not to mathematics.}
    \flushright\sc  Paul Halmos\footnote{As quoted in \cite{Albers}, p.~121.}\\
\end{quote}\index{Halmos, Paul}

Pure mathematicians have traditionally been indifferent to computers, even to
the point of disdain.\index{computers and pure mathematics}  Computer science
itself is sometimes considered to fall in the mundane realm of ``applied''
mathematics, perhaps essential for the real world but intellectually unexciting
to those who seek the deepest truths in mathematics.  Perhaps a reason for this
attitude towards computers is that there is little or no computer software that
meets their needs, and there may be a general feeling that such software could
not even exist.  On the one hand, there are the practical computer algebra
systems, which can perform amazing symbolic manipulations in algebra and
calculus,\index{computer algebra system} yet can't prove the simplest
existence theorem, if the idea of a proof is present at all.  On the other
hand, there are specialized automated theorem provers that technically speaking
may generate correct proofs.\index{automated theorem proving}  But sometimes
their specialized input notation may be cryptic and their output perceived to
be long, inelegant, incomprehensible proofs.    The output
may be viewed with suspicion, since the program that generates it tends to be
very large, and its size increases the potential for bugs\index{computer
program bugs}.  Such a proof may be considered trustworthy only if
independently verified and ``understood'' by a human, but no one wants to
waste their time on such a boring, unrewarding chore.



\needspace{4\baselineskip}
\subsection{Trusting the Computer}

\begin{quote}
  {\em \ldots I continue to find the quasi-empirical interpretation of
  computer proofs to be the more plausible.\ldots Since not
  everything that claims to be a computer proof can be
  accepted as valid, what are the mathematical criteria for acceptable
  computer proofs?}
    \flushright\sc  Thomas Tymoczko\footnote{\cite{Tymoczko}, p.~245.}\\
\end{quote}\index{Tymoczko, Thomas}

In some cases, computers have been essential tools for proving famous
theorems.  But if a proof is so long and obscure that it can be verified in a
practical way only with a computer, it is vaguely felt to be suspicious.  For
example, proving the famous four-color theorem\index{four-color
theorem}\index{proof length} (``a map needs no more than four colors to
prevent any two adjacent countries from having the same color'') can presently
only be done with the aid of a very complex computer program which originally
required 1200 hours of computer time. There has been considerable debate about
whether such a proof can be trusted and whether such a proof is ``real''
mathematics \cite{Swart}\index{Swart, E. R.}.\index{trusting computers}

However, under normal circumstances even a skeptical mathematician would have a
great deal of confidence in the result of multiplying two numbers on a pocket
calculator, even though the precise details of what goes on are hidden from its
user.  Even the verification on a supercomputer that a huge number is prime is
trusted, especially if there is independent verification; no one bothers to
debate the philosophical significance of its ``proof,'' even though the actual
proof would be so large that it would be completely impractical to ever write
it down on paper.  It seems that if the algorithm used by the computer is
simple enough to be readily understood, then the computer can be trusted.

Metamath\index{Metamath} adopts this philosophy.  The simplicity of its
language makes it easy to learn, and because of its simplicity one can have
essentially absolute confidence that a proof is correct. All axioms, rules, and
definitions are available for inspection at any time because they are defined
by the user; there are no hidden or built-in rules that may be prone to subtle
bugs\index{computer program bugs}.  The basic algorithm at the heart of
Metamath is simple and fixed, and it can be assumed to be bug-free and robust
with a degree of confidence approaching certainty.
Independently written implementations of the Metamath verifier
can reduce any residual doubt on the part of a skeptic even further;
there are now over a dozen such implementations, written by many people.

\subsection{Trusting the Mathematician}\label{trust}

\begin{quote}
  {\em There is no Algebraist nor Mathematician so expert in his science, as
  to place entire confidence in any truth immediately upon his discovery of it,
  or regard it as any thing, but a mere probability.  Every time he runs over
  his proofs, his confidence encreases; but still more by the approbation of
  his friends; and is rais'd to its utmost perfection by the universal assent
  and applauses of the learned world.}
  \flushright\sc David Hume\footnote{{\em A Treatise of Human Nature}, as
  quoted in \cite{deMillo}, p.~267.}\\
\end{quote}\index{Hume, David}

\begin{quote}
  {\em Stanislaw Ulam estimates that mathematicians publish 200,000 theorems
  every year.  A number of these are subsequently contradicted or otherwise
  disallowed, others are thrown into doubt, and most are ignored.}
  \flushright\sc Richard de Millo et. al.\footnote{\cite{deMillo}, p.~269.}\\
\end{quote}\index{Ulam, Stanislaw}

Whether or not the computer can be trusted, humans  of course will occasionally
err. Only the most memorable proofs get independently verified, and of these
only a handful of truly great ones achieve the status of being ``known''
mathematical truths that are used without giving a second thought to their
correctness.

There are many famous examples of incorrect theorems and proofs in
mathematical literature.\index{errors in proofs}

\begin{itemize}
\item There have been thousands of purported proofs of Fermat's Last
Theorem\index{Fermat's Last Theorem} (``no integer solutions exist to $x^n +
y^n = z^n$ for $n > 2$''), by amateurs, cranks, and well-regarded
mathematicians \cite[p.~5]{Stark}\index{Stark, Harold M}.  Fermat wrote a note
in his copy of Bachet's {\em Diophantus} that he found ``a truly marvelous
proof of this theorem but this margin is too narrow to contain it''
\cite[p.~507]{Kramer}.  A recent, much publicized proof by Yoichi
Miyaoka\index{Miyaoka, Yoichi} was shown to be incorrect ({\em Science News},
April 9, 1988, p.~230).  The theorem was finally proved by Andrew
Wiles\index{Wiles, Andrew} ({\em Science News}, July 3, 1993, p.~5), but it
initially had some gaps and took over a year after its announcement to be
checked thoroughly by experts.  On Oct. 25, 1994, Wiles announced that the last
gap found in his proof had been filled in.
  \item In 1882, M. Pasch discovered that an axiom was omitted from Euclid's
formulation of geometry\index{Euclidean geometry}; without it, the proofs of
important theorems of Euclid are not valid.  Pasch's axiom\index{Pasch's
axiom} states that a line that intersects one side of a triangle must also
intersect another side, provided that it does not touch any of the triangle's
vertices.  The omission of Pasch's axiom went unnoticed for 2000
years \cite[p.~160]{Davis}, in spite of (one presumes) the thousands of
students, instructors, and mathematicians who studied Euclid.
  \item The first published proof of the famous Schr\"{o}der--Bernstein
theorem\index{Schr\"{o}der--Bernstein theorem} in set theory was incorrect
\cite[p.~148]{Enderton}\index{Enderton, Herbert B.}.  This theorem states
that if there exists a 1-to-1 function\footnote{A {\em set}\index{set} is any
collection of objects. A {\em function}\index{function} or {\em
mapping}\index{mapping} is a rule that assigns to each element of one set
(called the function's {\em domain}\index{domain}) an element from another
set.} from set $A$ into set $B$ and vice-versa, then sets $A$ and $B$ have
a 1-to-1 correspondence.  Although it sounds simple and obvious,
the standard proof is quite long and complex.
  \item In the early 1900's, Hilbert\index{Hilbert, David} published a
purported proof of the continuum hypothesis\index{continuum hypothesis}, which
was eventually established as unprovable by Cohen\index{Cohen, Paul} in 1963
\cite[p.~166]{Enderton}.  The continuum hypothesis states that no
infinity\index{infinity} (``transfinite cardinal number'')\index{cardinal,
transfinite} exists whose size (or ``cardinality''\index{cardinality}) is
between the size of the set of integers and the size of the set of real
numbers.  This hypothesis originated with German mathematician Georg
Cantor\index{Cantor, Georg} in the late 1800's, and his inability to prove it
is said to have contributed to mental illness that afflicted him in his later
years.
  \item An incorrect proof of the four-color theorem\index{four-color theorem}
was published by Kempe\index{Kempe, A. B.} in 1879
\cite[p.~582]{Courant}\index{Courant, Richard}; it stood for 11 years before
its flaw was discovered.  This theorem states that any map can be colored
using only four colors, so that no two adjacent countries have the same
color.  In 1976 the theorem was finally proved by the famous computer-assisted
proof of Haken, Appel, and Koch \cite{Swart}\index{Appel, K.}\index{Haken,
W.}\index{Koch, K.}.  Or at least it seems that way.  Mathematician
H.~S.~M.~Coxeter\index{Coxeter, H. S. M.} has doubts \cite[p.~58]{Davis}:  ``I
have a feeling that that is an untidy kind of use of the computers, and the more
you correspond with Haken and Appel, the more shaky you seem to be.''
  \item Many false ``proofs'' of the Poincar\'{e}
conjecture\index{Poincar\'{e} conjecture} have been proposed over the years.
This conjecture states that any object that mathematically behaves like a
three-dimensional sphere is a three-dimensional sphere topologically,
regardless of how it is distorted.  In March 1986, mathematicians Colin
Rourke\index{Rourke, Colin} and Eduardo R\^{e}go\index{R\^{e}go, Eduardo}
caused  a stir in the mathematical community by announcing that they had found
a proof; in November of that year the proof was found to be false \cite[p.
218]{PetersonI}.  It was finally proved in 2003 by Grigory Perelman
\label{poincare}\index{Szpiro, George}\index{Perelman, Grigory}\cite{Szpiro}.
 \end{itemize}

Many counterexamples to ``theorems'' in recent mathematical
literature related to Clifford algebras\index{Clifford algebras}
 have been found by Pertti
Lounesto (who passed away in 2002).\index{Lounesto, Pertti}
See the web page \url{http://mathforum.org/library/view/4933.html}.
% http://users.tkk.fi/~ppuska/mirror/Lounesto/counterexamples.htm

One of the purposes of Metamath\index{Metamath} is to allow proofs to be
expressed with absolute precision.  Developing a proof in the Metamath
language can be challenging, because Metamath will not permit even the
tiniest mistake.\index{errors in proofs}  But once the proof is created, its
correctness can be trusted immediately, without having to depend on the
process of peer review for confirmation.

\section{The Use of Computers in Mathematics}

\subsection{Computer Algebra Systems}

For the most part, you will find that Metamath\index{Metamath} is not a
practical tool for manipulating numbers.  (Even proving that $2 + 2 = 4$, if
you start with set theory, can be quite complex!)  Several commercial
mathematics packages are quite good at arithmetic, algebra, and calculus, and
as practical tools they are invaluable.\index{computer algebra system} But
they have no notion of proof, and cannot understand statements starting with
``there exists such and such...''.

Software packages such as Mathematica \cite{Wolfram}\index{Mathematica} do not
concern themselves with proofs but instead work directly with known results.
These packages primarily emphasize heuristic rules such as the substitution of
equals for equals to achieve simpler expressions or expressions in a different
form.  Starting with a rich collection of built-in rules and algorithms, users
can add to the collection by means of a powerful programming language.
However, results such as, say, the existence of a certain abstract object
without displaying the actual object cannot be expressed (directly) in their
languages.  The idea of a proof from a small set of axioms is absent.  Instead
this software simply assumes that each fact or rule you add to the built-in
collection of algorithms is valid.  One way to view the software is as a large
collection of axioms from which the software, with certain goals, attempts to
derive new theorems, for example equating a complex expression with a simpler
equivalent. But the terms ``theorem''\index{theorem} and
``proof,''\index{proof} for example, are not even mentioned in the index of
the user's manual for Mathematica.\index{Mathematica and proofs}  What is also
unsatisfactory from a philosophical point of view is that there is no way to
ensure the validity of the results other than by trusting the writer of each
application module or tediously checking each module by hand, similar to
checking a computer program for bugs.\index{computer program
bugs}\footnote{Two examples illustrate why the knowledge database of computer
algebra systems should sometimes be regarded with a certain caution.  If you
ask Mathematica (version 3.0) to \texttt{Solve[x\^{ }n + y\^{ }n == z\^{ }n , n]}
it will respond with \texttt{\{\{n-\char`\>-2\}, \{n-\char`\>-1\},
\{n-\char`\>1\}, \{n-\char`\>2\}\}}. In other words, Mathematica seems to
``know'' that Fermat's Last Theorem\index{Fermat's Last Theorem} is true!  (At
the time this version of Mathematica was released this fact was unknown.)  If
you ask Maple\index{Maple} to \texttt{solve(x\^{ }2 = 2\^{ }x)} then
\texttt{simplify(\{"\})}, it returns the solution set \texttt{\{2, 4\}}, apparently
unaware that $-0.7666647$\ldots is also a solution.} While of course extremely
valuable in applied mathematics, computer algebra systems tend to be of little
interest to the theoretical mathematician except as aids for exploring certain
specific problems.

Because of possible bugs, trusting the output of a computer algebra system for
use as theorems in a proof-verifier would defeat the latter's goal of rigor.
On the other hand, a fact such that a certain relatively large number is
prime, while easy for a computer algebra system to derive, might have a long,
tedious proof that could overwhelm a proof-verifier. One approach for linking
computer algebra systems to a proof-verifier while retaining the advantages of
both is to add a hypothesis to each such theorem indicating its source.  For
example, a constant {\sc maple} could indicate the theorem came from the Maple
package, and would mean ``assuming Maple is consistent, then\ldots''  This and
many other topics concerning the formalization of mathematics are discussed in
John Harrison's\index{Harrison, John} very interesting
PhD thesis~\cite{Harrison-thesis}.

\subsection{Automated Theorem Provers}\label{theoremprovers}

A mathematical theory is ``decidable''\index{decidable theory} if a mechanical
method or algorithm exists that is guaranteed to determine whether or not a
particular formula is a theorem.  Among the few theories that are decidable is
elementary geometry,\index{Euclidean geometry} as was shown by a classic
result of logician Alfred Tarski\index{Tarski, Alfred} in 1948
\cite{Tarski}.\footnote{Tarski's result actually applies to a subset of the
geometry discussed in elementary textbooks.  This subset includes most of what
would be considered elementary geometry but it is not powerful enough to
express, among other things, the notions of the circumference and area of a
circle.  Extending the theory in a way that includes notions such as these
makes the theory undecidable, as was also shown by Tarski.  Tarski's algorithm
is far too inefficient to implement practically on a computer.  A practical
algorithm for proving a smaller subset of geometry theorems (those not
involving concepts of ``order'' or ``continuity'') was discovered by Chinese
mathematician Wu Wen-ts\"{u}n in 1977 \cite{Chou}\index{Chou,
Shang-Ching}.}\index{Wen-ts{\"{u}}n, Wu}  But most theories, including
elementary arithmetic, are undecidable.  This fact contributes to keeping
mathematics alive and well, since many mathematicians believe
that they will never be
replaced by computers (if they believe Roger Penrose's argument that a
computer can never replace the brain \cite{Penrose}\index{Penrose, Roger}).
In fact,  elementary geometry is often considered a ``dead'' field
for the simple reason that it is decidable.

On the other hand, the undecidability of a theory does not mean that one cannot
use a computer to search for proofs, providing one is willing to give up if a
proof is not found after a reasonable amount of time.  The field of automated
theorem proving\index{automated theorem proving} specializes in pursuing such
computer searches.  Among the more successful results to date are those based
on an algorithm known as Robinson's resolution principle
\cite{Robinson}\index{Robinson's resolution principle}.

Automated theorem provers can be excellent tools for those willing to learn
how to use them.  But they are not widely used in mainstream pure
mathematics, even though they could probably be useful in many areas.  There
are several reasons for this.  Probably most important, the main goal in pure
mathematics is to arrive at results that are considered to be deep or
important; proving them is essential but secondary.  Usually, an automated
theorem prover cannot assist in this main goal, and by the time the main goal
is achieved, the mathematician may have already figured out the proof as a
by-product.  There is also a notational problem.  Mathematicians are used to
using very compact syntax where one or two symbols (heavily dependent on
context) can represent very complex concepts; this is part of the
hierarchy\index{hierarchy} they have built up to tackle difficult problems.  A
theorem prover on the other hand might require that a theorem be expressed in
``first-order logic,''\index{first-order logic} which is the logic on which
most of mathematics is ultimately based but which is not ordinarily used
directly because expressions can become very long.  Some automated theorem
provers are experimental programs, limited in their use to very specialized
areas, and the goal of many is simply research into the nature of automated
theorem proving itself.  Finally, much research remains to be done to enable
them to prove very deep theorems.  One significant result was a
computer proof by Larry Wos\index{Wos, Larry} and colleagues that every Robbins
algebra\index{Robbins algebra} is a Boolean  algebra\index{Boolean algebra}
({\em New York Times}, Dec. 10, 1996).\footnote{In 1933, E.~V.\
Huntington\index{Huntington, E. V.}
presented the following axiom system for
Boolean algebra with a unary operation $n$ and a binary operation $+$:
\begin{center}
    $x + y = y + x$ \\
    $(x + y) + z = x + (y + z)$ \\
    $n(n(x) + y) + n(n(x) + n(y)) = x$
\end{center}
Herbert Robbins\index{Robbins, Herbert}, a student of Huntington, conjectured
that the last equation can be replaced with a simpler one:
\begin{center}
    $n(n(x + y) + n(x + n(y))) = x$
\end{center}
Robbins and Huntington could not find a proof.  The problem was
later studied unsuccessfully by Tarski\index{Tarski, Alfred} and his
students, and it remained an unsolved problem until a
computer found the proof in 1996.  For more information on
the Robbins algebra problem see \cite{Wos}.}

How does Metamath\index{Metamath} relate to automated theorem provers?  A
theorem prover is primarily concerned with one theorem at a time (perhaps
tapping into a small database of known theorems) whereas Metamath is more like
a theorem archiving system, storing both the theorem and its proof in a
database for access and verification.  Metamath is one answer to ``what do you
do with the output of a theorem prover?''  and could be viewed as the
next step in the process.  Automated theorem provers could be useful tools for
helping develop its database.
Note that very long, automatically
generated proofs can make your database fat and ugly and cause Metamath's proof
verification to take a long time to run.  Unless you have a particularly good
program that generates very concise proofs, it might be best to consider the
use of automatically generated proofs as a quick-and-dirty approach, to be
manually rewritten at some later date.

The program {\sc otter}\index{otter@{\sc otter}}\footnote{\url{http://www.cs.unm.edu/\~mccune/otter/}.}, later succeeded by
prover9\index{prover9}\footnote{\url{https://www.cs.unm.edu/~mccune/mace4/}.},
have been historically influential.
The E prover\index{E prover}\footnote{\url{https://github.com/eprover/eprover}.}
is a maintained automated theorem prover
for full first-order logic with equality.
There are many other automated theorem provers as well.

If you want to combine automated theorem provers with Metamath
consider investigating
the book {\em Automated Reasoning:  Introduction and Applications}
\cite{Wos}\index{Wos, Larry}.  This book discusses
how to use {\sc otter} in a way that can
not only able to generate
relatively efficient proofs, it can even be instructed to search for
shorter proofs.  The effective use of {\sc otter} (and similar tools)
does require a certain
amount of experience, skill, and patience.  The axiom system used in the
\texttt{set.mm}\index{set theory database (\texttt{set.mm})} set theory
database can be expressed to {\sc otter} using a method described in
\cite{Megill}.\index{Megill, Norman}\footnote{To use those axioms with
{\sc otter}, they must be restated in a way that eliminates the need for
``dummy variables.''\index{dummy variable!eliminating} See the Comment
on p.~\pageref{nodd}.} When successful, this method tends to generate
short and clever proofs, but my experiments with it indicate that the
method will find proofs within a reasonable time only for relatively
easy theorems.  It is still fun to experiment with.

Reference \cite{Bledsoe}\index{Bledsoe, W. W.} surveys a number of approaches
people have explored in the field of automated theorem proving\index{automated
theorem proving}.

\subsection{Interactive Theorem Provers}\label{interactivetheoremprovers}

Finding proofs completely automatically is difficult, so there
are some interactive theorem provers that allow a human to guide the
computer to find a proof.
Examples include
HOL Light\index{HOL light}%
\footnote{\url{https://www.cl.cam.ac.uk/~jrh13/hol-light/}.},
Isabelle\index{Isabelle}%
\footnote{\url{http://www.cl.cam.ac.uk/Research/HVG/Isabelle}.},
{\sc hol}\index{hol@{\sc hol}}%
\footnote{\url{https://hol-theorem-prover.org/}.},
and
Coq\index{Coq}\footnote{\url{https://coq.inria.fr/}.}.

A major difference between most of these tools and Metamath is that the
``proofs'' are actually programs that guide the program to find a proof,
and not the proof itself.
For example, an Isabelle/HOL proof might apply a step
\texttt{apply (blast dest: rearrange reduction)}. The \texttt{blast}
instruction applies
an automatic tableux prover and returns if it found a sequence of proof
steps that work... but the sequence is not considered part of the proof.

A good overview of
higher-level proof verification languages (such as {\sc lcf}\index{lcf@{\sc
lcf}} and {\sc hol}\index{hol@{\sc hol}})
is given in \cite{Harrison}.  All of these languages are fundamentally
different from Metamath in that much of the mathematical foundational
knowledge is embedded in the underlying proof-verification program, rather
than placed directly in the database that is being verified.
These can have a steep learning curve for those without a mathematical
background.  For example, one usually must have a fair understanding of
mathematical logic in order to follow their proofs.

\subsection{Proof Verifiers}\label{proofverifiers}

A proof verifier is a program that doesn't generate proofs but instead
verifies proofs that you give it.  Many proof verifiers have limited built-in
automated proof capabilities, such as figuring out simple logical inferences
(while still being guided by a person who provides the overall proof).  Metamath
has no built-in automated proof capability other than the limited
capability of its Proof Assistant.

Proof-verification languages are not used as frequently as they might be.
Pure mathematicians are more concerned with producing new results, and such
detail and rigor would interfere with that goal.  The use of computers in pure
mathematics is primarily focused on automated theorem provers (not verifiers),
again with the ultimate goal of aiding the creation of new mathematics.
Automated theorem provers are usually concerned with attacking one theorem at
time rather than making a large, organized database easily available to the
user.  Metamath is one way to help close this gap.

By itself Metamath is a mostly a proof verifier.
This does not mean that other approaches can't be used; the difference
is that in Metamath, the results of various provers must be recorded
step-by-step so that they can be verified.

Another proof-verification language is Mizar,\index{Mizar} which can display
its proofs in the informal language that mathematicians are accustomed to.
Information on the Mizar language is available at \url{http://mizar.org}.

For the working mathematician, Mizar is an excellent tool for rigorously
documenting proofs. Mizar typesets its proofs in the informal English used by
mathematicians (and, while fine for them, are just as inscrutable by
laypersons!). A price paid for Mizar is a relatively steep learning curve of a
couple of weeks.  Several mathematicians are actively formalizing different
areas of mathematics using Mizar and publishing the proofs in a dedicated
journal. Unfortunately the task of formalizing mathematics is still looked
down upon to a certain extent since it doesn't involve the creation of ``new''
mathematics.

The closest system to Metamath is
the {\em Ghilbert}\index{Ghilbert} proof language (\url{http://ghilbert.org})
system developed by
Raph Levien\index{Levien, Raph}.
Ghilbert is a formal proof checker heavily inspired by Metamath.
Ghilbert statements are s-expressions (a la Lisp), which is easy
for computers to parse but many people find them hard to read.
There are a number of differences in their specific constructs, but
there is at least one tool to translate some Metamath materials into Ghilbert.
As of 2019 the Ghilbert community is smaller and less active than the
Metamath community.
That said, the Metamath and Ghilbert communities overlap, and fruitful
conversations between them have occurred many times over the years.

\subsection{Creating a Database of Formalized Mathematics}\label{mathdatabase}

Besides Metamath, there are several other ongoing projects with the goal of
formalizing mathematics into computer-verifiable databases.
Understanding some history will help.

The {\sc qed}\index{qed project@{\sc qed} project}%
\footnote{\url{http://www-unix.mcs.anl.gov/qed}.}
project arose in 1993 and its goals were outlined in the
{\sc qed} manifesto.
The {\sc qed} manifesto was
a proposal for a computer-based database of all mathematical knowledge,
strictly formalized and with all proofs having been checked automatically.
The project had a conference in 1994 and another in 1995;
there was also a ``twenty years of the {\sc qed} manifesto'' workshop
in 2014.
Its ideals are regularly reraised.

In a 2007 paper, Freek Wiedijk identified two reasons
for the failure of the {\sc qed} project as originally envisioned:%
\cite{Wiedijk-revisited}\index{Wiedijk, Freek}

\begin{itemize}
\item Very few people are working on formalization of mathematics. There is no compelling application for fully mechanized mathematics.
\item Formalized mathematics does not yet resemble traditional mathematics. This is partly due to the complexity of mathematical notation, and partly to the limitations of existing theorem provers and proof assistants.
\end{itemize}

But this did not end the dream of
formalizing mathematics into computer-verifiable databases.
The problems that led to the {\sc qed} manifesto are still with us,
even though the challenges were harder than originally considered.
What has happened instead is that various independent projects have
worked towards formalizing mathematics into computer-verifiable databases,
each simultaneously competing and cooperating with each other.

A concrete way to see this is
Freek Wiedijk's ``Formalizing 100 Theorems'' list%
\footnote{\url{http://www.cs.ru.nl/\%7Efreek/100/}.}
which shows the progress different systems have made on a challenge list
of 100 mathematical theorems.%
\footnote{ This is not the only list of ``interesting'' theorems.
Another interesting list was posted by Oliver Knill's list
\cite{Knill}\index{Knill, Oliver}.}
The top systems as of February 2019
(in order of the number of challenges completed) are
HOL Light, Isabelle, Metamath, Coq, and Mizar.

The Metamath 100%
\footnote{\url{http://us.metamath.org/mm\_100.html}}
page (maintained by David A. Wheeler\index{Wheeler, David A.})
shows the progress of Metamath (specifically its \texttt{set.mm} database)
against this challenge list maintained by Freek Wiedijk.
The Metamath \texttt{set.mm} database
has made a lot of progress over the years,
in part because working to prove those challenge theorems required
defining various terms and proving their properties as a prerequisite.
Here are just a few of the many statements that have been
formally proven with Metamath:

% The entries of this cause the narrow display to break poorly,
% since the short amount of text means LaTeX doesn't get a lot to work with
% and the itemize format gives it even *less* margin than usual.
% No one will mind if we make just this list flushleft, since this list
% will be internally consistent.
\begin{flushleft}
\begin{itemize}
\item 1. The Irrationality of the Square Root of 2
  (\texttt{sqr2irr}, by Norman Megill, 2001-08-20)
\item 2. The Fundamental Theorem of Algebra
  (\texttt{fta}, by Mario Carneiro, 2014-09-15)
\item 22. The Non-Denumerability of the Continuum
  (\texttt{ruc}, by Norman Megill, 2004-08-13)
\item 54. The Konigsberg Bridge Problem
  (\texttt{konigsberg}, by Mario Carneiro, 2015-04-16)
\item 83. The Friendship Theorem
  (\texttt{friendship}, by Alexander W. van der Vekens, 2018-10-09)
\end{itemize}
\end{flushleft}

We thank all of those who have developed at least one of the Metamath 100
proofs, and we particularly thank
Mario Carneiro\index{Carneiro, Mario}
who has contributed the most Metamath 100 proofs as of 2019.
The Metamath 100 page shows the list of all people who have contributed a
proof, and links to graphs and charts showing progress over time.
We encourage others to work on proving theorems not yet proven in Metamath,
since doing so improves the work as a whole.

Each of the math formalization systems (including Metamath)
has different strengths and weaknesses, depending on what you value.
Key aspects that differentiate Metamath from the other top systems are:

\begin{itemize}
\item Metamath is not tied to any particular set of axioms.
\item Metamath can show every step of every proof, no exceptions.
  Most other provers only assert that a proof can be found, and do not
  show every step. This also makes verification fast, because
  the system does not need to rediscover proof details.
\item The Metamath verifier has been re-implemented in many different
  programming languages, so verification can be done by multiple
  implementations.  In particular, the
  \texttt{set.mm}\index{set theory database (\texttt{set.mm})}%
  \index{Metamath Proof Explorer} database is verified by
  four different verifiers
  written in four different languages by four different authors.
  This greatly reduces the risk of accepting an invalid
  proof due to an error in the verifier.
\item Proofs stay proven.  In some systems, changes to the system's
  syntax or how a tactic works causes proofs to fail in later versions,
  causing older work to become essentially lost.
  Metamath's language is
  extremely small and fixed, so once a proof is added to a database,
  the database can be rechecked with later versions of the Metamath program
  and with other verifiers of Metamath databases.
  If an axiom or key definition needs to be changed, it is easy to
  manipulate the database as a whole to handle the change
  without touching the underlying verifier.
  Since re-verification of an entire database takes seconds, there
  is never a reason to delay complete verification.
  This aspect is especially compelling if your
  goal is to have a long-term database of proofs.
\item Licensing is generous.  The main Metamath databases are released to
  the public domain, and the main Metamath program is open source software
  under a standard, widely-used license.
\item Substitutions are easy to understand, even by those who are not
  professional mathematicians.
\end{itemize}

Of course, other systems may have advantages over Metamath
that are more compelling, depending on what you value.
In any case, we hope this helps you understand Metamath
within a wider context.

\subsection{In Summary}\label{computers-summary}

To summarize our discussions of computers and mathematics, computer algebra
systems can be viewed as theorem generators focusing on a narrow realm of
mathematics (numbers and their properties), automated theorem provers as proof
generators for specific theorems in a much broader realm covered by a built-in
formal system such as first-order logic, interactive theorem
provers require human guidance, proof verifiers verify proofs but
historically they have been
restricted to first-order logic.
Metamath, in contrast,
is a proof verifier and documenter whose realm is essentially unlimited.

\section{Mathematics and Metamath}

\subsection{Standard Mathematics}

There are a number of ways that Metamath\index{Metamath} can be used with
standard mathematics.  The most satisfying way philosophically is to start at
the very beginning, and develop the desired mathematics from the axioms of
logic and set theory.\index{set theory}  This is the approach taken in the
\texttt{set.mm}\index{set theory database (\texttt{set.mm})}%
\index{Metamath Proof Explorer}
database (also known as the Metamath Proof Explorer).
Among other things, this database builds up to the
axioms of real and complex numbers\index{analysis}\index{real and complex
numbers} (see Section~\ref{real}), and a standard development of analysis, for
example, could start at that point, using it as a basis.   Besides this
philosophical advantage, there are practical advantages to having all of the
tools of set theory available in the supporting infrastructure.

On the other hand, you may wish to start with the standard axioms of a
mathematical theory without going through the set theoretical proofs of those
axioms.  You will need mathematical logic to make inferences, but if you wish
you can simply introduce theorems\index{theorem} of logic as
``axioms''\index{axiom} wherever you need them, with the implicit assumption
that in principle they can be proved, if they are obvious to you.  If you
choose this approach, you will probably want to review the notation used in
\texttt{set.mm}\index{set theory database (\texttt{set.mm})} so that your own
notation will be consistent with it.

\subsection{Other Formal Systems}
\index{formal system}

Unlike some programs, Metamath\index{Metamath} is not limited to any specific
area of mathematics, nor committed to any particular mathematical philosophy
such as classical logic versus intuitionism, nor limited, say, to expressions
in first-order logic.  Although the database \texttt{set.mm}
describes standard logic and set theory, Meta\-math
is actually a general-purpose language for describing a wide variety of formal
systems.\index{formal system}  Non-standard systems such as modal
logic,\index{modal logic} intuitionist logic\index{intuitionism}, higher-order
logic\index{higher-order logic}, quantum logic\index{quantum logic}, and
category theory\index{category theory} can all be described with the Metamath
language.  You define the symbols you prefer and tell Metamath the axioms and
rules you want to start from, and Metamath will verify any inferences you make
from those axioms and rules.  A simple example of a non-standard formal system
is Hofstadter's\index{Hofstadter, Douglas R.} MIU system,\index{MIU-system}
whose Metamath description is presented in Appendix~\ref{MIU}.

This is not hypothetical.
The largest Metamath database is
\texttt{set.mm}\index{set theory database (\texttt{set.mm}}%
\index{Metamath Proof Explorer}), aka the Metamath Proof Explorer,
which uses the most common axioms for mathematical foundations
(specifically classical logic combined with Zermelo--Fraenkel
set theory\index{Zermelo--Fraenkel set theory} with the Axiom of Choice).
But other Metamath databases are available:

\begin{itemize}
\item The database
  \texttt{iset.mm}\index{intuitionistic logic database (\texttt{iset.mm})},
  aka the
  Intuitionistic Logic Explorer\index{Intuitionistic Logic Explorer},
  uses intuitionistic logic (a constructivist point of view)
  instead of classical logic.
\item The database
  \texttt{nf.mm}\index{New Foundations database (\texttt{nf.mm})},
  aka the
  New Foundations Explorer\index{New Foundations Explorer},
  constructs mathematics from scratch,
  starting from Quine's New Foundations (NF) set theory axioms.
\item The database
  \texttt{hol.mm}\index{Higher-order Logic database (\texttt{hol.mm})},
  aka the
  Higher-Order Logic (HOL) Explorer\index{Higher-Order Logic (HOL) Explorer},
  starts with HOL (also called simple type theory) and derives
  equivalents to ZFC axioms, connecting the two approaches.
\end{itemize}

Since the days of David Hilbert,\index{Hilbert, David} mathematicians have
been concerned with the fact that the metalanguage\index{metalanguage} used to
describe mathematics may be stronger than the mathematics being described.
Metamath\index{Metamath}'s underlying finitary\index{finitary proof},
constructive nature provides a good philosophical basis for studying even the
weakest logics.\index{weak logic}

The usual treatment of many non-standard formal systems\index{formal
system} uses model theory\index{model theory} or proof theory\index{proof
theory} to describe these systems; these theories, in turn, are based on
standard set theory.  In other words, a non-standard formal system is defined
as a set with certain properties, and standard set theory is used to derive
additional properties of this set.  The standard set theory database provided
with Metamath can be used for this purpose, and when used this way
the development of a special
axiom system for the non-standard formal system becomes unnecessary.  The
model- or proof-theoretic approach often allows you to prove much deeper
results with less effort.

Metamath supports both approaches.  You can define the non-standard
formal system directly, or define the non-standard formal system as
a set with certain properties, whichever you find most helpful.

%\section{Additional Remarks}

\subsection{Metamath and Its Philosophy}

Closely related to Metamath\index{Metamath} is a philosophy or way of looking
at mathematics. This philosophy is related to the formalist
philosophy\index{formalism} of Hilbert\index{Hilbert, David} and his followers
\cite[pp.~1203--1208]{Kline}\index{Kline, Morris}
\cite[p.~6]{Behnke}\index{Behnke, H.}. In this philosophy, mathematics is
viewed as nothing more than a set of rules that manipulate symbols, together
with the consequences of those rules.  While the mathematics being described
may be complex, the rules used to describe it (the
``metamathematics''\index{metamathematics}) should be as simple as possible.
In particular, proofs should be restricted to dealing with concrete objects
(the symbols we write on paper rather than the abstract concepts they
represent) in a constructive manner; these are called ``finitary''
proofs\index{finitary proof} \cite[pp.~2--3]{Shoenfield}\index{Shoenfield,
Joseph R.}.

Whether or not you find Metamath interesting or useful will in part depend on
the appeal you find in its philosophy, and this appeal will probably depend on
your particular goals with respect to mathematics.  For example, if you are a
pure mathematician at the forefront of discovering new mathematical knowledge,
you will probably find that the rigid formality of Metamath stifles your
creativity.  On the other hand, we would argue that once this knowledge is
discovered, there are advantages to documenting it in a standard format that
will make it accessible to others.  Sixty years from now, your field may be
dormant, and as Davis and Hersh put it, your ``writings would become less
translatable than those of the Maya'' \cite[p.~37]{Davis}\index{Davis, Phillip
J.}.


\subsection{A History of the Approach Behind Metamath}

Probably the one work that has had the most motivating influence on
Metamath\index{Metamath} is Whitehead and Russell's monumental {\em Principia
Mathematica} \cite{PM}\index{Whitehead, Alfred North}\index{Russell,
Bertrand}\index{principia mathematica@{\em Principia Mathematica}}, whose aim
was to deduce all of mathematics from a small number of primitive ideas, in a
very explicit way that in principle anyone could understand and follow.  While
this work was tremendously influential in its time, from a modern perspective
it suffers from several drawbacks.  Both its notation and its underlying
axioms are now considered dated and are no longer used.  From our point of
view, its development is not really as accessible as we would like to see; for
practical reasons, proofs become more and more sketchy as its mathematics
progresses, and working them out in fine detail requires a degree of
mathematical skill and patience that many people don't have.  There are
numerous small errors, which is understandable given the tedious, technical
nature of the proofs and the lack of a computer to verify the details.
However, even today {\em Principia Mathematica} stands out as the work closest
in spirit to Metamath.  It remains a mind-boggling work, and one can't help
but be amazed at seeing ``$1+1=2$'' finally appear on page 83 of Volume II
(Theorem *110.643).

The origin of the proof notation used by Metamath dates back to the 1950's,
when the logician C.~A.~Meredith expressed his proofs in a compact notation
called ``condensed detachment''\index{condensed detachment}
\cite{Hindley}\index{Hindley, J. Roger} \cite{Kalman}\index{Kalman, J. A.}
\cite{Meredith}\index{Meredith, C. A.} \cite{Peterson}\index{Peterson, Jeremy
George}.  This notation allows proofs to be communicated unambiguously by
merely referencing the axiom\index{axiom}, rule\index{rule}, or
theorem\index{theorem} used at each step, without explicitly indicating the
substitutions\index{substitution!variable}\index{variable substitution} that
have to be made to the variables in that axiom, rule, or theorem.  Ordinarily,
condensed detachment is more or less limited to propositional
calculus\index{propositional calculus}.  The concept has been extended to
first-order logic\index{first-order logic} in \cite{Megill}\index{Megill,
Norman}, making it is easy to write a small computer program to verify proofs
of simple first-order logic theorems.\index{condensed detachment!and
first-order logic}

A key concept behind the notation of condensed detachment is called
``unification,''\index{unification} which is an algorithm for determining what
substitutions\index{substitution!variable}\index{variable substitution} to
variables have to be made to make two expressions match each other.
Unification was first precisely defined by the logician J.~A.~Robinson, who
used it in the development of a powerful
theorem-proving technique called the ``resolution principle''
\cite{Robinson}\index{Robinson's resolution principle}. Metamath does not make
use of the resolution principle, which is intended for systems of first-order
logic.\index{first-order logic}  Metamath's use is not restricted to
first-order logic, and as we have mentioned it does not automatically discover
proofs.  However, unification is a key idea behind Metamath's proof
notation, and Metamath makes use of a very simple version of it
(Section~\ref{unify}).

\subsection{Metamath and First-Order Logic}

First-order logic\index{first-order logic} is the supporting structure
for standard mathematics.  On top of it is set theory, which contains
the axioms from which virtually all of mathematics can be derived---a
remarkable fact.\index{category
theory}\index{cardinal, inaccessible}\label{categoryth}\footnote{An exception seems
to be category theory.  There are several schools of thought on whether
category theory is derivable from set theory.  At a minimum, it appears
that an additional axiom is needed that asserts the existence of an
``inaccessible cardinal'' (a type of infinity so large that standard set
theory can't prove or deny that it exists).
%
%%%% (I took this out that was in previous editions:)
% But it is also argued that not just one but a ``proper class'' of them
% is needed, and the existence of proper classes is impossible in standard
% set theory.  (A proper class is a collection of sets so huge that no set
% can contain it as an element.  Proper classes can lead to
% inconsistencies such as ``Russell's paradox.''  The axioms of standard
% set theory are devised so as to deny the existence of proper classes.)
%
For more information, see
\cite[pp.~328--331]{Herrlich}\index{Herrlich, Horst} and
\cite{Blass}\index{Blass, Andrea}.}

One of the things that makes Metamath\index{Metamath} more practical for
first-order theories is a set of axioms for first-order logic designed
specifically with Metamath's approach in mind.  These are included in
the database \texttt{set.mm}\index{set theory database (\texttt{set.mm})}.
See Chapter~\ref{fol} for a detailed
description; the axioms are shown in Section~\ref{metaaxioms}.  While
logically equivalent to standard axiom systems, our axiom system breaks
up the standard axioms into smaller pieces such that from them, you can
directly derive what in other systems can only be derived as higher-level
``metatheorems.''\index{metatheorem}  In other words, it is more powerful than
the standard axioms from a metalogical point of view.  A rigorous
justification for this system and its ``metalogical
completeness''\index{metalogical completeness} is found in
\cite{Megill}\index{Megill, Norman}.  The system is closely related to a
system developed by Monk\index{Monk, J. Donald} and Tarski\index{Tarski,
Alfred} in 1965 \cite{Monks}.

For example, the formula $\exists x \, x = y $ (given $y$, there exists some
$x$ equal to it) is a theorem of logic,\footnote{Specifically, it is a theorem
of those systems of logic that assume non-empty domains.  It is not a theorem
of more general systems that include the empty domain\index{empty domain}, in
which nothing exists, period!  Such systems are called ``free
logics.''\index{free logic} For a discussion of these systems, see
\cite{Leblanc}\index{Leblanc, Hugues}.  Since our use for logic is as a basis
for set theory, which has a non-empty domain, it is more convenient (and more
traditional) to use a less general system.  An interesting curiosity is that,
using a free logic as a basis for Zermelo--Fraenkel set
theory\index{Zermelo--Fraenkel set theory} (with the redundant Axiom of the
Null Set omitted),\index{Axiom of the Null Set} we cannot even prove the
existence of a single set without assuming the axiom of infinity!\index{Axiom
of Infinity}} whether or not $x$ and $y$ are distinct variables\index{distinct
variables}.  In many systems of logic, we would have to prove two theorems to
arrive at this result.  First we would prove ``$\exists x \, x = x $,'' then
we would separately prove ``$\exists x \, x = y $, where $x$ and $y$ are
distinct variables.''  We would then combine these two special cases ``outside
of the system'' (i.e.\ in our heads) to be able to claim, ``$\exists x \, x =
y $, regardless of whether $x$ and $y$ are distinct.''  In other words, the
combination of the two special cases is a
metatheorem.  In the system of logic
used in Metamath's set theory\index{set theory database (\texttt{set.mm})}
database, the axioms of logic are broken down into small pieces that allow
them to be reassembled in such a way that theorems such as these can be proved
directly.

Breaking down the axioms in this way makes them look peculiar and not very
intuitive at first, but rest assured that they are correct and complete.  Their
correctness is ensured because they are theorem schemes of standard first-order
logic (which you can easily verify if you are a logician).  Their completeness
follows from the fact that we explicitly derive the standard axioms of
first-order logic as theorems.  Deriving the standard axioms is somewhat
tricky, but once we're there, we have at our disposal a system that is less
awkward to work with in formal proofs\index{formal proof}.  In technical terms
that logicians understand, we eliminate the cumbersome concepts of ``free
variable,''\index{free variable} ``bound variable,''\index{bound variable} and
``proper substitution''\index{proper substitution}\index{substitution!proper}
as primitive notions.  These concepts are present in our system but are
defined in terms of concepts expressed by the axioms and can be eliminated in
principle.  In standard systems, these concepts are really like additional,
implicit axioms\index{implicit axiom} that are somewhat complex and cannot be
eliminated.

The traditional approach to logic, wherein free variables and proper
substitution is defined, is also possible to do directly in the Metamath
language.  However, the notation tends to become awkward, and there are
disadvantages:  for example, extending the definition of a wff with a
definition is awkward, because the free variable and proper substitution
concepts have to have their definitions also extended.  Our choice of
axioms for \texttt{set.mm} is to a certain extent a matter of style, in
an attempt to achieve overall simplicity, but you should also be aware
that the traditional approach is possible as well if you should choose
it.

\chapter{Using the Metamath Program}
\label{using}

\section{Installation}

The way that you install Metamath\index{Metamath!installation} on your
computer system will vary for different computers.  Current
instructions are provided with the Metamath program download at
\url{http://metamath.org}.  In general, the installation is simple.
There is one file containing the Metamath program itself.  This file is
usually called \texttt{metamath} or \texttt{metamath.}{\em xxx} where
{\em xxx} is the convention (such as \texttt{exe}) for an executable
program on your operating system.  There are several additional files
containing samples of the Metamath language, all ending with
\texttt{.mm}.  The file \texttt{set.mm}\index{set theory database
(\texttt{set.mm})} contains logic and set theory and can be used as a
starting point for other areas of mathematics.

You will also need a text editor\index{text editor} capable of editing plain
{\sc ascii}\footnote{American Standard Code for Information Interchange.} text
in order to prepare your input files.\index{ascii@{\sc ascii}}  Most computers
have this capability built in.  Note that plain text is not necessarily the
default for some word processing programs\index{word processor}, especially if
they can handle different fonts; for example, with Microsoft Word\index{Word
(Microsoft)}, you must save the file in the format ``Text Only With Line
Breaks'' to get a plain text\index{plain text} file.\footnote{It is
recommended that all lines in a Metamath source file be 79 characters or less
in length for compatibility among different computer terminals.  When creating
a source file on an editor such as Word, select a monospaced
font\index{monospaced font} such as Courier\index{Courier font} or
Monaco\index{Monaco font} to make this easier to achieve.  Better yet,
just use a plain text editor such as Notepad.}

On some computer systems, Metamath does not have the capability to print
its output directly; instead, you send its output to a file (using the
\texttt{open} commands described later).  The way you print this output
file depends on your computer.\index{printers} Some computers have a
print command, whereas with others, you may have to read the file into
an editor and print it from there.

If you want to print your Metamath source files with typeset formulas
containing standard mathematical symbols, you will need the \LaTeX\
typesetting program\index{latex@{\LaTeX}}, which is widely and freely
available for most operating systems.  It runs natively on Unix and
Linux, and can be installed on Windows as part of the free Cygwin
package (\url{http://cygwin.com}).

You can also produce {\sc html}\footnote{HyperText Markup Language.}
web pages.  The {\tt help html} command in the Metamath program will
assist you with this feature.

\section{Your First Formal System}\label{start}
\subsection{From Nothing to Zero}\label{startf}

To give you a feel for what the Metamath\index{Metamath} language looks like,
we will take a look at a very simple example from formal number
theory\index{number theory}.  This example is taken from
Mendelson\index{Mendelson, Elliot} \cite[p. 123]{Mendelson}.\footnote{To keep
the example simple, we have changed the formalism slightly, and what we call
axioms\index{axiom} are strictly speaking theorems\index{theorem} in
\cite{Mendelson}.}  We will look at a small subset of this theory, namely that
part needed for the first number theory theorem proved in \cite{Mendelson}.

First we will look at a standard formal proof\index{formal proof} for the
example we have picked, then we will look at the Metamath version.  If you
have never been exposed to formal proofs, the notation may seem to be such
overkill to express such simple notions that you may wonder if you are missing
something.  You aren't.  The concepts involved are in fact very simple, and a
detailed breakdown in this fashion is necessary to express the proof in a way
that can be verified mechanically.  And as you will see, Metamath breaks the
proof down into even finer pieces so that the mechanical verification process
can be about as simple as possible.

Before we can introduce the axioms\index{axiom} of the theory, we must define
the syntax rules for forming legal expressions\index{syntax rules}
(combinations of symbols) with which those axioms can be used. The number 0 is
a {\bf term}\index{term}; and if $ t$ and $r$ are terms, so is $(t+r)$. Here,
$ t$ and $r$ are ``metavariables''\index{metavariable} ranging over terms; they
themselves do not appear as symbols in an actual term.  Some examples of
actual terms are $(0 + 0)$ and $((0+0)+0)$.  (Note that our theory describes
only the number zero and sums of zeroes.  Of course, not much can be done with
such a trivial theory, but remember that we have picked a very small subset of
complete number theory for our example.  The important thing for you to focus
on is our definitions that describe how symbols are combined to form valid
expressions, and not on the content or meaning of those expressions.) If $ t$
and $r$ are terms, an expression of the form $ t=r$ is a {\bf wff}
(well-formed formula)\index{well-formed formula (wff)}; and if $P$ and $Q$ are
wffs, so is $(P\rightarrow Q)$ (which means ``$P$ implies
$Q$''\index{implication ($\rightarrow$)} or ``if $P$ then $Q$'').
Here $P$ and $Q$ are metavariables ranging over wffs.  Examples of actual
wffs are $0=0$, $(0+0)=0$, $(0=0 \rightarrow (0+0)=0)$, and $(0=0\rightarrow
(0=0\rightarrow 0=(0+0)))$.  (Our notation makes use of more parentheses than
are customary, but the elimination of ambiguity this way simplifies our
example by avoiding the need to define operator precedence\index{operator
precedence}.)

The {\bf axioms}\index{axiom} of our theory are all wffs of the following
form, where $ t$, $r$, and $s$ are any terms:

%Latex p. 92
\renewcommand{\theequation}{A\arabic{equation}}

\begin{equation}
(t=r\rightarrow (t=s\rightarrow r=s))
\end{equation}
\begin{equation}
(t+0)=t
\end{equation}

Note that there are an infinite number of axioms since there are an infinite
number of possible terms.  A1 and A2 are properly called ``axiom
schemes,''\index{axiom scheme} but we will refer to them as ``axioms'' for
brevity.

An axiom is a {\bf theorem}; and if $P$ and $(P\rightarrow Q)$ are theorems
(where $P$ and $Q$ are wffs), then $Q$ is also a theorem.\index{theorem}  The
second part of this definition is called the modus ponens (MP) rule of
inference\index{inference rule}\index{modus ponens}.  It allows us to obtain
new theorems from old ones.

The {\bf proof}\index{proof} of a theorem is a sequence of one or more
theorems, each of which is either an axiom or the result of modus ponens
applied to two previous theorems in the sequence, and the last of which is the
theorem being proved.

The theorem we will prove for our example is very simple:  $ t=t$.  The proof of
our theorem follows.  Study it carefully until you feel sure you
understand it.\label{zeroproof}

% Use tabu so that lines will wrap automatically as needed.
\begin{tabu} { l X X }
1. & $(t+0)=t$ & (by axiom A2) \\
2. & $(t+0)=t$ & (by axiom A2) \\
3. & $((t+0)=t \rightarrow ((t+0)=t\rightarrow t=t))$ & (by axiom A1) \\
4. & $((t+0)=t\rightarrow t=t)$ & (by MP applied to steps 2 and 3) \\
5. & $t=t$ & (by MP applied to steps 1 and 4) \\
\end{tabu}

(You may wonder why step 1 is repeated twice.  This is not necessary in the
formal language we have defined, but in Metamath's ``reverse Polish
notation''\index{reverse Polish notation (RPN)} for proofs, a previous step
can be referred to only once.  The repetition of step~1 here will enable you
to see more clearly the correspondence of this proof with the
Metamath\index{Metamath} version on p.~\pageref{demoproof}.)

Our theorem is more properly called a ``theorem scheme,''\index{theorem
scheme} for it represents an infinite number of theorems, one for each
possible term $ t$.  Two examples of actual theorems would be $0=0$ and
$(0+0)=(0+0)$.  Rarely do we prove actual theorems, since by proving schemes
we can prove an infinite number of theorems in one fell swoop.  Similarly, our
proof should really be called a ``proof scheme.''\index{proof scheme}  To
obtain an actual proof, pick an actual term to use in place of $ t$, and
substitute it for $ t$ throughout the proof.

Let's discuss what we have done here.  The axioms\index{axiom} of our theory,
A1 and A2, are trivial and obvious.  Everyone knows that adding zero to
something doesn't change it, and also that if two things are equal to a third,
then they are equal to each other. In fact, stating the trivial and obvious is
a goal to strive for in any axiomatic system.  From trivial and obvious truths
that everyone agrees upon, we can prove results that are not so obvious yet
have absolute faith in them.  If we trust the axioms and the rules, we must,
by definition, trust the consequences of those axioms and rules, if logic is
to mean anything at all.

Our rule of inference\index{rule}, modus ponens\index{modus ponens}, is also
pretty obvious once you understand what it means.  If we prove a fact $P$, and
we also prove that $P$ implies $Q$, then $Q$ necessarily follows as a new
fact.  The rule provides us with a means for obtaining new facts (i.e.\
theorems\index{theorem}) from old ones.

The theorem that we have proved, $ t=t$, is so fundamental that you may wonder
why it isn't one of the axioms\index{axiom}.  In some axiom systems of
arithmetic, it {\em is} an axiom.  The choice of axioms in a theory is to some
extent arbitrary and even an art form, constrained only by the requirement
that any two equivalent axiom systems be able to derive each other as
theorems.  We could imagine that the inventor of our axiom system originally
included $ t=t$ as an axiom, then discovered that it could be derived as a
theorem from the other axioms.  Because of this, it was not necessary to
keep it as an axiom.  By eliminating it, the final set of axioms became
that much simpler.

Unless you have worked with formal proofs\index{formal proof} before, it
probably wasn't apparent to you that $ t=t$ could be derived from our two
axioms until you saw the proof. While you certainly believe that $ t=t$ is
true, you might not be able to convince an imaginary skeptic who believes only
in our two axioms until you produce the proof.  Formal proofs such as this are
hard to come up with when you first start working with them, but after you get
used to them they can become interesting and fun.  Once you understand the
idea behind formal proofs you will have grasped the fundamental principle that
underlies all of mathematics.  As the mathematics becomes more sophisticated,
its proofs become more challenging, but ultimately they all can be broken down
into individual steps as simple as the ones in our proof above.

Mendelson's\index{Mendelson, Elliot} book, from which our example was taken,
contains a number of detailed formal proofs such as these, and you may be
interested in looking it up.  The book is intended for mathematicians,
however, and most of it is rather advanced.  Popular literature describing
formal proofs\index{formal proof} include \cite[p.~296]{Rucker}\index{Rucker,
Rudy} and \cite[pp.~204--230]{Hofstadter}\index{Hofstadter, Douglas R.}.

\subsection{Converting It to Metamath}\label{convert}

Formal proofs\index{formal proof} such as the one in our example break down
logical reasoning into small, precise steps that leave little doubt that the
results follow from the axioms\index{axiom}.  You might think that this would
be the finest breakdown we can achieve in mathematics.  However, there is more
to the proof than meets the eye. Although our axioms were rather simple, a lot
of verbiage was needed before we could even state them:  we needed to define
``term,'' ``wff,'' and so on.  In addition, there are a number of implied
rules that we haven't even mentioned. For example, how do we know that step 3
of our proof follows from axiom A1? There is some hidden reasoning involved in
determining this.  Axiom A1 has two occurrences of the letter $ t$.  One of
the implied rules states that whatever we substitute for $ t$ must be a legal
term\index{term}.\footnote{Some authors make this implied rule explicit by
stating, ``only expressions of the above form are terms,'' after defining
``term.''}  The expression $ t+0$ is pretty obviously a legal term whenever $
t$ is, but suppose we wanted to substitute a huge term with thousands of
symbols?  Certainly a lot of work would be involved in determining that it
really is a term, but in ordinary formal proofs all of this work would be
considered a single ``step.''

To express our axiom system in the Metamath\index{Metamath} language, we must
describe this auxiliary information in addition to the axioms themselves.
Metamath does not know what a ``term'' or a ``wff''\index{well-formed formula
(wff)} is.  In Metamath, the specification of the ways in which we can combine
symbols to obtain terms and wffs are like little axioms in themselves.  These
auxiliary axioms are expressed in the same notation as the ``real''
axioms\index{axiom}, and Metamath does not distinguish between the two.  The
distinction is made by you, i.e.\ by the way in which you interpret the
notation you have chosen to express these two kinds of axioms.

The Metamath language breaks down mathematical proofs into tiny pieces, much
more so than in ordinary formal proofs\index{formal proof}.  If a single
step\index{proof step} involves the
substitution\index{substitution!variable}\index{variable substitution} of a
complex term for one of its variables, Metamath must see this single step
broken down into many small steps.  This fine-grained breakdown is what gives
Metamath generality and flexibility as it lets it not be limited to any
particular mathematical notation.

Metamath's proof notation is not, in itself, intended to be read by humans but
rather is in a compact format intended for a machine.  The Metamath program
will convert this notation to a form you can understand, using the \texttt{show
proof}\index{\texttt{show proof} command} command.  You can tell the program what
level of detail of the proof you want to look at.  You may want to look at
just the logical inference steps that correspond
to ordinary formal proof steps,
or you may want to see the fine-grained steps that prove that an expression is
a term.

Here, without further ado, is our example converted to the
Metamath\index{Metamath} language:\index{metavariable}\label{demo0}

\begin{verbatim}
$( Declare the constant symbols we will use $)
    $c 0 + = -> ( ) term wff |- $.
$( Declare the metavariables we will use $)
    $v t r s P Q $.
$( Specify properties of the metavariables $)
    tt $f term t $.
    tr $f term r $.
    ts $f term s $.
    wp $f wff P $.
    wq $f wff Q $.
$( Define "term" and "wff" $)
    tze $a term 0 $.
    tpl $a term ( t + r ) $.
    weq $a wff t = r $.
    wim $a wff ( P -> Q ) $.
$( State the axioms $)
    a1 $a |- ( t = r -> ( t = s -> r = s ) ) $.
    a2 $a |- ( t + 0 ) = t $.
$( Define the modus ponens inference rule $)
    ${
       min $e |- P $.
       maj $e |- ( P -> Q ) $.
       mp  $a |- Q $.
    $}
$( Prove a theorem $)
    th1 $p |- t = t $=
  $( Here is its proof: $)
       tt tze tpl tt weq tt tt weq tt a2 tt tze tpl
       tt weq tt tze tpl tt weq tt tt weq wim tt a2
       tt tze tpl tt tt a1 mp mp
     $.
\end{verbatim}\index{metavariable}

A ``database''\index{database} is a set of one or more {\sc ascii} source
files.  Here's a brief description of this Metamath\index{Metamath} database
(which consists of this single source file), so that you can understand in
general terms what is going on.  To understand the source file in detail, you
should read Chapter~\ref{languagespec}.

The database is a sequence of ``tokens,''\index{token} which are normally
separated by spaces or line breaks.  The only tokens that are built into
the Metamath language are those beginning with \texttt{\$}.  These tokens
are called ``keywords.''\index{keyword}  All other tokens are
user-defined, and their names are arbitrary.

As you might have guessed, the Metamath token \texttt{\$(}\index{\texttt{\$(} and
\texttt{\$)} auxiliary keywords} starts a comment and \texttt{\$)} ends a comment.

The Metamath tokens \texttt{\$c}\index{\texttt{\$c} statement},
\texttt{\$v}\index{\texttt{\$v} statement},
\texttt{\$e}\index{\texttt{\$e} statement},
\texttt{\$f}\index{\texttt{\$f} statement},
\texttt{\$a}\index{\texttt{\$a} statement}, and
\texttt{\$p}\index{\texttt{\$p} statement} specify ``statements'' that
end with \texttt{\$.}\,.\index{\texttt{\$.}\ keyword}

The Metamath tokens \texttt{\$c} and \texttt{\$v} each declare\index{constant
declaration}\index{variable declaration} a list of user-defined tokens, called
``math symbols,''\index{math symbol} that the database will reference later
on.  All of the math symbols they define you have seen earlier except the
turnstile symbol \texttt{|-} ($\vdash$)\index{turnstile ({$\,\vdash$})}, which is
commonly used by logicians to mean ``a proof exists for.''  For us
the turnstile is just a
convenient symbol that distinguishes expressions that are axioms\index{axiom}
or theorems\index{theorem} from expressions that are terms or wffs.

The \texttt{\$c} statement declares ``constants''\index{constant} and
the \texttt{\$v} statement declares
``variables''\index{variable}\index{constant declaration}\index{variable
declaration} (or more precisely, metavariables\index{metavariable}).  A
variable may be substituted\index{substitution!variable}\index{variable
substitution} with sequences of math symbols whereas a constant may not
be substituted with anything.

It may seem redundant to require both \texttt{\$c}\index{\texttt{\$c} statement} and
\texttt{\$v}\index{\texttt{\$v} statement} statements (since any math
symbol\index{math symbol} not specified with a \texttt{\$c} statement could be
presumed to be a variable), but this provides for better error checking and
also allows math symbols to be redeclared\index{redeclaration of symbols}
(Section~\ref{scoping}).

The token \texttt{\$f}\index{\texttt{\$f} statement} specifies a
statement called a ``variable-type hypothesis'' (also called a
``floating hypothesis'') and \texttt{\$e}\index{\texttt{\$e} statement}
specifies a ``logical hypothesis'' (also called an ``essential
hypothesis'').\index{hypothesis}\index{variable-type
hypothesis}\index{logical hypothesis}\index{floating
hypothesis}\index{essential hypothesis} The token
\texttt{\$a}\index{\texttt{\$a} statement} specifies an ``axiomatic
assertion,''\index{axiomatic assertion} and
\texttt{\$p}\index{\texttt{\$p} statement} specifies a ``provable
assertion.''\index{provable assertion} To the left of each occurrence of
these four tokens is a ``label''\index{label} that identifies the
hypothesis or assertion for later reference.  For example, the label of
the first axiomatic assertion is \texttt{tze}.  A \texttt{\$f} statement
must contain exactly two math symbols, a constant followed by a
variable.  The \texttt{\$e}, \texttt{\$a}, and \texttt{\$p} statements
each start with a constant followed by, in general, an arbitrary
sequence of math symbols.

Associated with each assertion\index{assertion} is a set of hypotheses
that must be satisfied in order for the assertion to be used in a proof.
These are called the ``mandatory hypotheses''\index{mandatory
hypothesis} of the assertion.  Among those hypotheses whose ``scope''
(described below) includes the assertion, \texttt{\$e} hypotheses are
always mandatory and \texttt{\$f}\index{\texttt{\$f} statement}
hypotheses are mandatory when they share their variable with the
assertion or its \texttt{\$e} hypotheses.  The exact rules for
determining which hypotheses are mandatory are described in detail in
Sections~\ref{frames} and \ref{scoping}.  For example, the mandatory
hypotheses of assertion \texttt{tpl} are \texttt{tt} and \texttt{tr},
whereas assertion \texttt{tze} has no mandatory hypotheses because it
contains no variables and has no \texttt{\$e}\index{\texttt{\$e}
statement} hypothesis.  Metamath's \texttt{show statement}
command\index{\texttt{show statement} command}, described in the next
section, will show you a statement's mandatory hypotheses.

Sometimes we need to make a hypothesis relevant to only certain
assertions.  The set of statements to which a hypothesis is relevant is
called its ``scope.''  The Metamath brackets,
\texttt{\$\char`\{}\index{\texttt{\$\char`\{} and \texttt{\$\char`\}}
keywords} and \texttt{\$\char`\}}, define a ``block''\index{block} that
delimits the scope of any hypothesis contained between them.  The
assertion \texttt{mp} has mandatory hypotheses \texttt{wp}, \texttt{wq},
\texttt{min}, and \texttt{maj}.  The only mandatory hypothesis of
\texttt{th1}, on the other hand, is \texttt{tt}, since \texttt{th1}
occurs outside of the block containing \texttt{min} and \texttt{maj}.

Note that \texttt{\$\char`\{} and \texttt{\$\char`\}} do not affect the
scope of assertions (\texttt{\$a} and \texttt{\$p}).  Assertions are always
available to be referenced by any later proof in the source file.

Each provable assertion (\texttt{\$p}\index{\texttt{\$p} statement}
statement) has two parts.  The first part is the
assertion\index{assertion} itself, which is a sequence of math
symbol\index{math symbol} tokens placed between the \texttt{\$p} token
and a \texttt{\$=}\index{\texttt{\$=} keyword} token.  The second part
is a ``proof,'' which is a list of label tokens placed between the
\texttt{\$=} token and the \texttt{\$.}\index{\texttt{\$.}\ keyword}\
token that ends the statement.\footnote{If you've looked at the
\texttt{set.mm} database, you may have noticed another notation used for
proofs.  The other notation is called ``compressed.''\index{compressed
proof}\index{proof!compressed} It reduces the amount of space needed to
store a proof in the database and is described in
Appendix~\ref{compressed}.  In the example above, we use
``normal''\index{normal proof}\index{proof!normal} notation.} The proof
acts as a series of instructions to the Metamath program, telling it how
to build up the sequence of math symbols contained in the assertion part of
the \texttt{\$p} statement, making use of the hypotheses of the
\texttt{\$p} statement and previous assertions.  The construction takes
place according to precise rules.  If the list of labels in the proof
causes these rules to be violated, or if the final sequence that results
does not match the assertion, the Metamath program will notify you with
an error message.

If you are familiar with reverse Polish notation (RPN), which is sometimes used
on pocket calculators, here in a nutshell is how a proof works.  Each
hypothesis label\index{hypothesis label} in the proof is pushed\index{push}
onto the RPN stack\index{stack}\index{RPN stack} as it is encountered. Each
assertion label\index{assertion label} pops\index{pop} off the stack as many
entries as the referenced assertion has mandatory hypotheses.  Variable
substitutions\index{substitution!variable}\index{variable substitution} are
computed which, when made to the referenced assertion's mandatory hypotheses,
cause these hypotheses to match the stack entries. These same substitutions
are then made to the variables in the referenced assertion itself, which is
then pushed onto the stack.  At the end of the proof, there should be one
stack entry, namely the assertion being proved.  This process is explained in
detail in Section~\ref{proof}.

Metamath's proof notation is not very readable for humans, but it allows the
proof to be stored compactly in a file.  The Metamath\index{Metamath} program
has proof display features that let you see what's going on in a more
readable way, as you will see in the next section.

The rules used in verifying a proof are not based on any built-in syntax of the
symbol sequence in an assertion\index{assertion} nor on any built-in meanings
attached to specific symbol names.  They are based strictly on symbol
matching:  constants\index{constant} must match themselves, and
variables\index{variable} may be replaced with anything that allows a match to
occur.  For example, instead of \texttt{term}, \texttt{0}, and \verb$|-$ we could
have just as well used \texttt{yellow}, \texttt{zero}, and \texttt{provable}, as long
as we did so consistently throughout the database.  Also, we could have used
\texttt{is provable} (two tokens) instead of \verb$|-$ (one token) throughout the
database.  In each of these cases, the proof would be exactly the same.  The
independence of proofs and notation means that you have a lot of flexibility to
change the notation you use without having to change any proofs.

\section{A Trial Run}\label{trialrun}

Now you are ready to try out the Metamath\index{Metamath} program.

On all computer systems, Metamath has a standard ``command line
interface'' (CLI)\index{command line interface (CLI)} that allows you to
interact with it.  You supply commands to the CLI by typing them on the
keyboard and pressing your keyboard's {\em return} key after each line
you enter.  The CLI is designed to be easy to use and has built-in help
features.

The first thing you should do is to use a text editor to create a file
called \texttt{demo0.mm} and type into it the Metamath source shown on
p.~\pageref{demo0}.  Actually, this file is included with your Metamath
software package, so check that first.  If you type it in, make sure
that you save it in the form of ``plain {\sc ascii} text with line
breaks.''  Most word processors will have this feature.

Next you must run the Metamath program.  Depending on your computer
system and how Metamath is installed, this could range from clicking the
mouse on the Metamath icon to typing \texttt{run metamath} to typing
simply \texttt{metamath}.  (Metamath's {\tt help invoke} command describes
alternate ways of invoking the Metamath program.)

When you first enter Metamath\index{Metamath}, it will be at the CLI, waiting
for your input. You will see something like the following on your screen:
\begin{verbatim}
Metamath - Version 0.177 27-Apr-2019
Type HELP for help, EXIT to exit.
MM>
\end{verbatim}
The \texttt{MM>} prompt means that Metamath is waiting for a command.
Command keywords\index{command keyword} are not case sensitive;
we will use lower-case commands in our examples.
The version number and its release date will probably be different on your
system from the one we show above.

The first thing that you need to do is to read in your
database:\index{\texttt{read} command}\footnote{If a directory path is
needed on Unix,\index{Unix file names}\index{file names!Unix} you should
enclose the path/file name in quotes to prevent Metamath from thinking
that the \texttt{/} in the path name is a command qualifier, e.g.,
\texttt{read \char`\"db/set.mm\char`\"}.  Quotes are optional when there
is no ambiguity.}
\begin{verbatim}
MM> read demo0.mm
\end{verbatim}
Remember to press the {\em return} key after entering this command.  If
you omit the file name, Metamath will prompt you for one.   The syntax for
specifying a Macintosh file name path is given in a footnote on
p.~\pageref{includef}.\index{Macintosh file names}\index{file
names!Macintosh}

If there are any syntax errors in the database, Metamath will let you know
when it reads in the file.  The one thing that Metamath does not check when
reading in a database is that all proofs are correct, because this would
slow it down too much.  It is a good idea to periodically verify the proofs in
a database you are making changes to.  To do this, use the following command
(and do it for your \texttt{demo0.mm} file now).  Note that the \texttt{*} is a
``wild card'' meaning all proofs in the file.\index{\texttt{verify proof} command}
\begin{verbatim}
MM> verify proof *
\end{verbatim}
Metamath will report any proofs that are incorrect.

It is often useful to save the information that the Metamath program displays
on the screen. You can save everything that happens on the screen by opening a
log file. You may want to do this before you read in a database so that you
can examine any errors later on.  To open a log file, type
\begin{verbatim}
MM> open log abc.log
\end{verbatim}
This will open a file called \texttt{abc.log}, and everything that appears on the
screen from this point on will be stored in this file.  The name of the log file
is arbitrary. To close the log file, type
\begin{verbatim}
MM> close log
\end{verbatim}

Several commands let you examine what's inside your database.
Section~\ref{exploring} has an overview of some useful ones.  The
\texttt{show labels} command lets you see what statement
labels\index{label} exist.  A \texttt{*} matches any combination of
characters, and \texttt{t*} refers to all labels starting with the
letter \texttt{t}.\index{\texttt{show labels} command} The \texttt{/all}
is a ``command qualifier''\index{command qualifier} that tells Metamath
to include labels of hypotheses.  (To see the syntax explained, type
\texttt{help show labels}.)  Type
\begin{verbatim}
MM> show labels t* /all
\end{verbatim}
Metamath will respond with
\begin{verbatim}
The statement number, label, and type are shown.
3 tt $f       4 tr $f       5 ts $f       8 tze $a
9 tpl $a      19 th1 $p
\end{verbatim}

You can use the \texttt{show statement} command to get information about a
particular statement.\index{\texttt{show statement} command}
For example, you can get information about the statement with label \texttt{mp}
by typing
\begin{verbatim}
MM> show statement mp /full
\end{verbatim}
Metamath will respond with
\begin{verbatim}
Statement 17 is located on line 43 of the file
"demo0.mm".
"Define the modus ponens inference rule"
17 mp $a |- Q $.
Its mandatory hypotheses in RPN order are:
  wp $f wff P $.
  wq $f wff Q $.
  min $e |- P $.
  maj $e |- ( P -> Q ) $.
The statement and its hypotheses require the
      variables:  Q P
The variables it contains are:  Q P
\end{verbatim}
The mandatory hypotheses\index{mandatory hypothesis} and their
order\index{RPN order} are
useful to know when you are trying to understand or debug a proof.

Now you are ready to look at what's really inside our proof.  First, here is
how to look at every step in the proof---not just the ones corresponding to an
ordinary formal proof\index{formal proof}, but also the ones that build up the
formulas that appear in each ordinary formal proof step.\index{\texttt{show
proof} command}
\begin{verbatim}
MM> show proof th1 /lemmon /all
\end{verbatim}

This will display the proof on the screen in the following format:
\begin{verbatim}
 1 tt            $f term t
 2 tze           $a term 0
 3 1,2 tpl       $a term ( t + 0 )
 4 tt            $f term t
 5 3,4 weq       $a wff ( t + 0 ) = t
 6 tt            $f term t
 7 tt            $f term t
 8 6,7 weq       $a wff t = t
 9 tt            $f term t
10 9 a2          $a |- ( t + 0 ) = t
11 tt            $f term t
12 tze           $a term 0
13 11,12 tpl     $a term ( t + 0 )
14 tt            $f term t
15 13,14 weq     $a wff ( t + 0 ) = t
16 tt            $f term t
17 tze           $a term 0
18 16,17 tpl     $a term ( t + 0 )
19 tt            $f term t
20 18,19 weq     $a wff ( t + 0 ) = t
21 tt            $f term t
22 tt            $f term t
23 21,22 weq     $a wff t = t
24 20,23 wim     $a wff ( ( t + 0 ) = t -> t = t )
25 tt            $f term t
26 25 a2         $a |- ( t + 0 ) = t
27 tt            $f term t
28 tze           $a term 0
29 27,28 tpl     $a term ( t + 0 )
30 tt            $f term t
31 tt            $f term t
32 29,30,31 a1   $a |- ( ( t + 0 ) = t -> ( ( t + 0 )
                                     = t -> t = t ) )
33 15,24,26,32 mp  $a |- ( ( t + 0 ) = t -> t = t )
34 5,8,10,33 mp  $a |- t = t
\end{verbatim}

The \texttt{/lemmon} command qualifier specifies what is known as a Lemmon-style
display\index{Lemmon-style proof}\index{proof!Lemmon-style}.  Omitting the
\texttt{/lemmon} qualifier results in a tree-style proof (see
p.~\pageref{treeproof} for an example) that is somewhat less explicit but
easier to follow once you get used to it.\index{tree-style
proof}\index{proof!tree-style}

The first number on each line is the step
number of the proof.  Any numbers that follow are step numbers assigned to the
hypotheses of the statement referenced by that step.  Next is the label of
the statement referenced by the step.  The statement type of the statement
referenced comes next, followed by the math symbol\index{math symbol} string
constructed by the proof up to that step.

The last step, 34, contains the statement that is being proved.

Looking at a small piece of the proof, notice that steps 3 and 4 have
established that
\texttt{( t + 0 )} and \texttt{t} are \texttt{term}\,s, and step 5 makes use of steps 3 and
4 to establish that \texttt{( t + 0 ) = t} is a \texttt{wff}.  Let Metamath
itself tell us in detail what is happening in step 5.  Note that the
``target hypothesis'' refers to where step 5 is eventually used, i.e., in step
34.
\begin{verbatim}
MM> show proof th1 /detailed_step 5
Proof step 5:  wp=weq $a wff ( t + 0 ) = t
This step assigns source "weq" ($a) to target "wp"
($f).  The source assertion requires the hypotheses
"tt" ($f, step 3) and "tr" ($f, step 4).  The parent
assertion of the target hypothesis is "mp" ($a,
step 34).
The source assertion before substitution was:
    weq $a wff t = r
The following substitutions were made to the source
assertion:
    Variable  Substituted with
     t         ( t + 0 )
     r         t
The target hypothesis before substitution was:
    wp $f wff P
The following substitution was made to the target
hypothesis:
    Variable  Substituted with
     P         ( t + 0 ) = t
\end{verbatim}

The full proof just shown is useful to understand what is going on in detail.
However, most of the time you will just be interested in
the ``essential'' or logical steps of a proof, i.e.\ those steps
that correspond to an
ordinary formal proof\index{formal proof}.  If you type
\begin{verbatim}
MM> show proof th1 /lemmon /renumber
\end{verbatim}
you will see\label{demoproof}
\begin{verbatim}
1 a2             $a |- ( t + 0 ) = t
2 a2             $a |- ( t + 0 ) = t
3 a1             $a |- ( ( t + 0 ) = t -> ( ( t + 0 )
                                     = t -> t = t ) )
4 2,3 mp         $a |- ( ( t + 0 ) = t -> t = t )
5 1,4 mp         $a |- t = t
\end{verbatim}
Compare this to the formal proof on p.~\pageref{zeroproof} and
notice the resemblance.
By default Metamath
does not show \texttt{\$f}\index{\texttt{\$f}
statement} hypotheses and everything branching off of them in the proof tree
when the proof is displayed; this makes the proof look more like an ordinary
mathematical proof, which does not normally incorporate the explicit
construction of expressions.
This is called the ``essential'' view
(at one time you had to add the
\texttt{/essential} qualifier in the \texttt{show proof}
command to get this view, but this is now the default).
You can could use the \texttt{/all} qualifier in the \texttt{show
proof} command to also show the explicit construction of expressions.
The \texttt{/renumber} qualifier means to renumber
the steps to correspond only to what is displayed.\index{\texttt{show proof}
command}

To exit Metamath, type\index{\texttt{exit} command}
\begin{verbatim}
MM> exit
\end{verbatim}

\subsection{Some Hints for Using the Command Line Interface}

We will conclude this quick introduction to Metamath\index{Metamath} with some
helpful hints on how to navigate your way through the commands.
\index{command line interface (CLI)}

When you type commands into Metamath's CLI, you only have to type as many
characters of a command keyword\index{command keyword} as are needed to make
it unambiguous.  If you type too few characters, Metamath will tell you what
the choices are.  In the case of the \texttt{read} command, only the \texttt{r} is
needed to specify it unambiguously, so you could have typed\index{\texttt{read}
command}
\begin{verbatim}
MM> r demo0.mm
\end{verbatim}
instead of
\begin{verbatim}
MM> read demo0.mm
\end{verbatim}
In our description, we always show the full command words.  When using the
Metamath CLI commands in a command file (to be read with the \texttt{submit}
command)\index{\texttt{submit} command}, it is good practice to use
the unabbreviated command to ensure your instructions will not become ambiguous
if more commands are added to the Metamath program in the future.

The command keywords\index{command
keyword} are not case sensitive; you may type either \texttt{read} or
\texttt{ReAd}.  File names may or may not be case sensitive, depending on your
computer's operating system.  Metamath label\index{label} and math
symbol\index{math symbol} tokens\index{token} are case-sensitive.

The \texttt{help} command\index{\texttt{help} command} will provide you
with a list of topics you can get help on.  You can then type
\texttt{help} {\em topic} to get help on that topic.

If you are uncertain of a command's spelling, just type as many characters
as you remember of the command.  If you have not typed enough characters to
specify it unambiguously, Metamath will tell you what choices you have.

\begin{verbatim}
MM> show s
         ^
?Ambiguous keyword - please specify SETTINGS,
STATEMENT, or SOURCE.
\end{verbatim}

If you don't know what argument to use as part of a command, type a
\texttt{?}\index{\texttt{]}@\texttt{?}\ in command lines}\ at the
argument position.  Metamath will tell you what it expected there.

\begin{verbatim}
MM> show ?
         ^
?Expected SETTINGS, LABELS, STATEMENT, SOURCE, PROOF,
MEMORY, TRACE_BACK, or USAGE.
\end{verbatim}

Finally, you may type just the first word or words of a command followed
by {\em return}.  Metamath will prompt you for the remaining part of the
command, showing you the choices at each step.  For example, instead of
typing \texttt{show statement th1 /full} you could interact in the
following manner:
\begin{verbatim}
MM> show
SETTINGS, LABELS, STATEMENT, SOURCE, PROOF,
MEMORY, TRACE_BACK, or USAGE <SETTINGS>? st
What is the statement label <th1>?
/ or nothing <nothing>? /
TEX, COMMENT_ONLY, or FULL <TEX>? f
/ or nothing <nothing>?
19 th1 $p |- t = t $= ... $.
\end{verbatim}
After each \texttt{?}\ in this mode, you must give Metamath the
information it requests.  Sometimes Metamath gives you a list of choices
with the default choice indicated by brackets \texttt{< > }. Pressing
{\em return} after the \texttt{?}\ will select the default choice.
Answering anything else will override the default.  Note that the
\texttt{/} in command qualifiers is considered a separate
token\index{token} by the parser, and this is why it is asked for
separately.

\section{Your First Proof}\label{frstprf}

Proofs are developed with the aid of the Proof Assistant\index{Proof
Assistant}.  We will now show you how the proof of theorem \texttt{th1}
was built.  So that you can repeat these steps, we will first have the
Proof Assistant erase the proof in Metamath's source buffer\index{source
buffer}, then reconstruct it.  (The source buffer is the place in memory
where Metamath stores the information in the database when it is
\texttt{read}\index{\texttt{read} command} in.  New or modified proofs
are kept in the source buffer until a \texttt{write source}
command\index{\texttt{write source} command} is issued.)  In practice, you
would place a \texttt{?}\index{\texttt{]}@\texttt{?}\ inside proofs}\
between \texttt{\$=}\index{\texttt{\$=} keyword} and
\texttt{\$.}\index{\texttt{\$.}\ keyword}\ in the database to indicate
to Metamath\index{Metamath} that the proof is unknown, and that would be
your starting point.  Whenever the \texttt{verify proof} command encounters
a proof with a \texttt{?}\ in place of a proof step, the statement is
identified as not proved.

When I first started creating Metamath proofs, I would write down
on a piece of paper the complete
formal proof\index{formal proof} as it would appear
in a \texttt{show proof} command\index{\texttt{show proof} command}; see
the display of \texttt{show proof th1 /lemmon /re\-num\-ber} above as an
example.  After you get used to using the Proof Assistant\index{Proof
Assistant} you may get to a point where you can ``see'' the proof in your mind
and let the Proof Assistant guide you in filling in the details, at least for
simpler proofs, but until you gain that experience you may find it very useful
to write down all the details in advance.
Otherwise you may waste a lot of time as you let it take you down a wrong path.
However, others do not find this approach as helpful.
For example, Thomas Brendan Leahy\index{Leahy, Thomas Brendan}
finds that it is more helpful to him to interactively
work backward from a machine-readable statement.
David A. Wheeler\index{Wheeler, David A.}
writes down a general approach, but develops the proof
interactively by switching between
working forwards (from hypotheses and facts likely to be useful) and
backwards (from the goal) until the forwards and backwards approaches meet.
In the end, use whatever approach works for you.

A proof is developed with the Proof Assistant by working backwards, starting
with the theorem\index{theorem} to be proved, and assigning each unknown step
with a theorem or hypothesis until no more unknown steps remain.  The Proof
Assistant will not let you make an assignment unless it can be ``unified''
with the unknown step.  This means that a
substitution\index{substitution!variable}\index{variable substitution} of
variables exists that will make the assignment match the unknown step.  On the
other hand, in the middle of a proof, when working backwards, often more than
one unification\index{unification} (set of substitutions) is possible, since
there is not enough information available at that point to uniquely establish
it.  In this case you can tell Metamath which unification to choose, or you
can continue to assign unknown steps until enough information is available to
make the unification unique.

We will assume you have entered Metamath and read in the database as described
above.  The following dialog shows how the proof was developed.  For more
details on what some of the commands do, refer to Section~\ref{pfcommands}.
\index{\texttt{prove} command}

\begin{verbatim}
MM> prove th1
Entering the Proof Assistant.  Type HELP for help, EXIT
to exit.  You will be working on the proof of statement th1:
  $p |- t = t
Note:  The proof you are starting with is already complete.
MM-PA>
\end{verbatim}

The \verb/MM-PA>/ prompt means we are inside the Proof
Assistant.\index{Proof Assistant} Most of the regular Metamath commands
(\texttt{show statement}, etc.) are still available if you need them.

\begin{verbatim}
MM-PA> delete all
The entire proof was deleted.
\end{verbatim}

We have deleted the whole proof so we can start from scratch.

\begin{verbatim}
MM-PA> show new_proof/lemmon/all
1 ?              $? |- t = t
\end{verbatim}

The \texttt{show new{\char`\_}proof} command\index{\texttt{show
new{\char`\_}proof} command} is like \texttt{show proof} except that we
don't specify a statement; instead, the proof we're working on is
displayed.

\begin{verbatim}
MM-PA> assign 1 mp
To undo the assignment, DELETE STEP 5 and INITIALIZE, UNIFY
if needed.
3   min=?  $? |- $2
4   maj=?  $? |- ( $2 -> t = t )
\end{verbatim}

The \texttt{assign} command\index{\texttt{assign} command} above means
``assign step 1 with the statement whose label is \texttt{mp}.''  Note
that step renumbering will constantly occur as you assign steps in the
middle of a proof; in general all steps from the step you assign until
the end of the proof will get moved up.  In this case, what used to be
step 1 is now step 5, because the (partial) proof now has five steps:
the four hypotheses of the \texttt{mp} statement and the \texttt{mp}
statement itself.  Let's look at all the steps in our partial proof:

\begin{verbatim}
MM-PA> show new_proof/lemmon/all
1 ?              $? wff $2
2 ?              $? wff t = t
3 ?              $? |- $2
4 ?              $? |- ( $2 -> t = t )
5 1,2,3,4 mp     $a |- t = t
\end{verbatim}

The symbol \texttt{\$2} is a temporary variable\index{temporary
variable} that represents a symbol sequence not yet known.  In the final
proof, all temporary variables will be eliminated.  The general format
for a temporary variable is \texttt{\$} followed by an integer.  Note
that \texttt{\$} is not a legal character in a math symbol (see
Section~\ref{dollardollar}, p.~\pageref{dollardollar}), so there will
never be a naming conflict between real symbols and temporary variables.

Unknown steps 1 and 2 are constructions of the two wffs used by the
modus ponens rule.  As you will see at the end of this section, the
Proof Assistant\index{Proof Assistant} can usually figure these steps
out by itself, and we will not have to worry about them.  Therefore from
here on we will display only the ``essential'' hypotheses, i.e.\ those
steps that correspond to traditional formal proofs\index{formal proof}.

\begin{verbatim}
MM-PA> show new_proof/lemmon
3 ?              $? |- $2
4 ?              $? |- ( $2 -> t = t )
5 3,4 mp         $a |- t = t
\end{verbatim}

Unknown steps 3 and 4 are the ones we must focus on.  They correspond to the
minor and major premises of the modus ponens rule.  We will assign them as
follows.  Notice that because of the step renumbering that takes place
after an assignment, it is advantageous to assign unknown steps in reverse
order, because earlier steps will not get renumbered.

\begin{verbatim}
MM-PA> assign 4 mp
To undo the assignment, DELETE STEP 8 and INITIALIZE, UNIFY
if needed.
3   min=?  $? |- $2
6     min=?  $? |- $4
7     maj=?  $? |- ( $4 -> ( $2 -> t = t ) )
\end{verbatim}

We are now going to describe an obscure feature that you will probably
never use but should be aware of.  The Metamath language allows empty
symbol sequences to be substituted for variables, but in most formal
systems this feature is never used.  One of the few examples where is it
used is the MIU-system\index{MIU-system} described in
Appendix~\ref{MIU}.  But such systems are rare, and by default this
feature is turned off in the Proof Assistant.  (It is always allowed for
{\tt verify proof}.)  Let us turn it on and see what
happens.\index{\texttt{set empty{\char`\_}substitution} command}

\begin{verbatim}
MM-PA> set empty_substitution on
Substitutions with empty symbol sequences is now allowed.
\end{verbatim}

With this feature enabled, more unifications will be
ambiguous\index{ambiguous unification}\index{unification!ambiguous} in
the middle of a proof, because
substitution\index{substitution!variable}\index{variable substitution}
of variables with empty symbol sequences will become an additional
possibility.  Let's see what happens when we make our next assignment.

\begin{verbatim}
MM-PA> assign 3 a2
There are 2 possible unifications.  Please select the correct
    one or Q if you want to UNIFY later.
Unify:  |- $6
 with:  |- ( $9 + 0 ) = $9
Unification #1 of 2 (weight = 7):
  Replace "$6" with "( + 0 ) ="
  Replace "$9" with ""
  Accept (A), reject (R), or quit (Q) <A>? r
\end{verbatim}

The first choice presented is the wrong one.  If we had selected it,
temporary variable \texttt{\$6} would have been assigned a truncated
wff, and temporary variable \texttt{\$9} would have been assigned an
empty sequence (which is not allowed in our system).  With this choice,
eventually we would reach a point where we would get stuck because
we would end up with steps impossible to prove.  (You may want to
try it.)  We typed \texttt{r} to reject the choice.

\begin{verbatim}
Unification #2 of 2 (weight = 21):
  Replace "$6" with "( $9 + 0 ) = $9"
  Accept (A), reject (R), or quit (Q) <A>? q
To undo the assignment, DELETE STEP 4 and INITIALIZE, UNIFY
if needed.
 7     min=?  $? |- $8
 8     maj=?  $? |- ( $8 -> ( $6 -> t = t ) )
\end{verbatim}

The second choice is correct, and normally we would type \texttt{a}
to accept it.  But instead we typed \texttt{q} to show what will happen:
it will leave the step with an unknown unification, which can be
seen as follows:

\begin{verbatim}
MM-PA> show new_proof/not_unified
 4   min    $a |- $6
        =a2  = |- ( $9 + 0 ) = $9
\end{verbatim}

Later we can unify this with the \texttt{unify}
\texttt{all/interactive} command.

The important point to remember is that occasionally you will be
presented with several unification choices while entering a proof, when
the program determines that there is not enough information yet to make
an unambiguous choice automatically (and this can happen even with
\texttt{set empty{\char`\_}substitution} turned off).  Usually it is
obvious by inspection which choice is correct, since incorrect ones will
tend to be meaningless fragments of wffs.  In addition, the correct
choice will usually be the first one presented, unlike our example
above.

Enough of this digression.  Let us go back to the default setting.

\begin{verbatim}
MM-PA> set empty_substitution off
The ability to substitute empty expressions for variables
has been turned off.  Note that this may make the Proof
Assistant too restrictive in some cases.
\end{verbatim}

If we delete the proof, start over, and get to the point where
we digressed above, there will no longer be an ambiguous unification.

\begin{verbatim}
MM-PA> assign 3 a2
To undo the assignment, DELETE STEP 4 and INITIALIZE, UNIFY
if needed.
 7     min=?  $? |- $4
 8     maj=?  $? |- ( $4 -> ( ( $5 + 0 ) = $5 -> t = t ) )
\end{verbatim}

Let us look at our proof so far, and continue.

\begin{verbatim}
MM-PA> show new_proof/lemmon
 4 a2            $a |- ( $5 + 0 ) = $5
 7 ?             $? |- $4
 8 ?             $? |- ( $4 -> ( ( $5 + 0 ) = $5 -> t = t ) )
 9 7,8 mp        $a |- ( ( $5 + 0 ) = $5 -> t = t )
10 4,9 mp        $a |- t = t
MM-PA> assign 8 a1
To undo the assignment, DELETE STEP 11 and INITIALIZE, UNIFY
if needed.
 7     min=?  $? |- ( t + 0 ) = t
MM-PA> assign 7 a2
To undo the assignment, DELETE STEP 8 and INITIALIZE, UNIFY
if needed.
MM-PA> show new_proof/lemmon
 4 a2            $a |- ( t + 0 ) = t
 8 a2            $a |- ( t + 0 ) = t
12 a1            $a |- ( ( t + 0 ) = t -> ( ( t + 0 ) = t ->
                                                    t = t ) )
13 8,12 mp       $a |- ( ( t + 0 ) = t -> t = t )
14 4,13 mp       $a |- t = t
\end{verbatim}

Now all temporary variables and unknown steps have been eliminated from the
``essential'' part of the proof.  When this is achieved, the Proof
Assistant\index{Proof Assistant} can usually figure out the rest of the proof
automatically.  (Note that the \texttt{improve} command can occasionally be
useful for filling in essential steps as well, but it only tries to make use
of statements that introduce no new variables in their hypotheses, which is
not the case for \texttt{mp}. Also it will not try to improve steps containing
temporary variables.)  Let's look at the complete proof, then run
the \texttt{improve} command, then look at it again.

\begin{verbatim}
MM-PA> show new_proof/lemmon/all
 1 ?             $? wff ( t + 0 ) = t
 2 ?             $? wff t = t
 3 ?             $? term t
 4 3 a2          $a |- ( t + 0 ) = t
 5 ?             $? wff ( t + 0 ) = t
 6 ?             $? wff ( ( t + 0 ) = t -> t = t )
 7 ?             $? term t
 8 7 a2          $a |- ( t + 0 ) = t
 9 ?             $? term ( t + 0 )
10 ?             $? term t
11 ?             $? term t
12 9,10,11 a1    $a |- ( ( t + 0 ) = t -> ( ( t + 0 ) = t ->
                                                    t = t ) )
13 5,6,8,12 mp   $a |- ( ( t + 0 ) = t -> t = t )
14 1,2,4,13 mp   $a |- t = t
\end{verbatim}

\begin{verbatim}
MM-PA> improve all
A proof of length 1 was found for step 11.
A proof of length 1 was found for step 10.
A proof of length 3 was found for step 9.
A proof of length 1 was found for step 7.
A proof of length 9 was found for step 6.
A proof of length 5 was found for step 5.
A proof of length 1 was found for step 3.
A proof of length 3 was found for step 2.
A proof of length 5 was found for step 1.
Steps 1 and above have been renumbered.
CONGRATULATIONS!  The proof is complete.  Use SAVE
NEW_PROOF to save it.  Note:  The Proof Assistant does
not detect $d violations.  After saving the proof, you
should verify it with VERIFY PROOF.
\end{verbatim}

The \texttt{save new{\char`\_}proof} command\index{\texttt{save
new{\char`\_}proof} command} will save the proof in the database.  Here
we will just display it in a form that can be clipped out of a log file
and inserted manually into the database source file with a text
editor.\index{normal proof}\index{proof!normal}

\begin{verbatim}
MM-PA> show new_proof/normal
---------Clip out the proof below this line:
      tt tze tpl tt weq tt tt weq tt a2 tt tze tpl tt weq
      tt tze tpl tt weq tt tt weq wim tt a2 tt tze tpl tt
      tt a1 mp mp $.
---------The proof of 'th1' to clip out ends above this line.
\end{verbatim}

There is another proof format called ``compressed''\index{compressed
proof}\index{proof!compressed} that you will see in databases.  It is
not important to understand how it is encoded but only to recognize it
when you see it.  Its only purpose is to reduce storage requirements for
large proofs.  A compressed proof can always be converted to a normal
one and vice-versa, and the Metamath \texttt{show proof}
commands\index{\texttt{show proof} command} work equally well with
compressed proofs.  The compressed proof format is described in
Appendix~\ref{compressed}.

\begin{verbatim}
MM-PA> show new_proof/compressed
---------Clip out the proof below this line:
      ( tze tpl weq a2 wim a1 mp ) ABCZADZAADZAEZJJKFLIA
      AGHH $.
---------The proof of 'th1' to clip out ends above this line.
\end{verbatim}

Now we will exit the Proof Assistant.  Since we made changes to the proof,
it will warn us that we have not saved it.  In this case, we don't care.

\begin{verbatim}
MM-PA> exit
Warning:  You have not saved changes to the proof.
Do you want to EXIT anyway (Y, N) <N>? y
Exiting the Proof Assistant.
Type EXIT again to exit Metamath.
\end{verbatim}

The Proof Assistant\index{Proof Assistant} has several other commands
that can help you while creating proofs.  See Section~\ref{pfcommands}
for a list of them.

A command that is often useful is \texttt{minimize{\char`\_}with
*/brief}, which tries to shorten the proof.  It can make the process
more efficient by letting you write a somewhat ``sloppy'' proof then
clean up some of the fine details of optimization for you (although it
can't perform miracles such as restructuring the overall proof).

\section{A Note About Editing a Data\-base File}

Once your source file contains proofs, there are some restrictions on
how you can edit it so that the proofs remain valid.  Pay particular
attention to these rules, since otherwise you can lose a lot of work.
It is a good idea to periodically verify all proofs with \texttt{verify
proof *} to ensure their integrity.

If your file contains only normal (as opposed to compressed) proofs, the
main rule is that you may not change the order of the mandatory
hypotheses\index{mandatory hypothesis} of any statement referenced in a
later proof.  For example, if you swap the order of the major and minor
premise in the modus ponens rule, all proofs making use of that rule
will become incorrect.  The \texttt{show statement}
command\index{\texttt{show statement} command} will show you the
mandatory hypotheses of a statement and their order.

If a statement has a compressed proof, you also must not change the
order of {\em its} mandatory hypotheses.  The compressed proof format
makes use of this information as part of the compression technique.
Note that swapping the names of two variables in a theorem will change
the order of its mandatory hypotheses.

The safest way to edit a statement, say \texttt{mytheorem}, is to
duplicate it then rename the original to \texttt{mytheoremOLD}
throughout the database.  Once the edited version is re-proved, all
statements referencing \texttt{mytheoremOLD} can be updated in the Proof
Assistant using \texttt{minimize{\char`\_}with
mytheorem
/allow{\char`\_}growth}.\index{\texttt{minimize{\char`\_}with} command}
% 3/10/07 Note: line-breaking the above results in duplicate index entries

\chapter{Abstract Mathematics Revealed}\label{fol}

\section{Logic and Set Theory}\label{logicandsettheory}

\begin{quote}
  {\em Set theory can be viewed as a form of exact theology.}
  \flushright\sc  Rudy Rucker\footnote{\cite{Barrow}, p.~31.}\\
\end{quote}\index{Rucker, Rudy}

Despite its seeming complexity, all of standard mathematics, no matter how
deep or abstract, can amazingly enough be derived from a relatively small set
of axioms\index{axiom} or first principles. The development of these axioms is
among the most impressive and important accomplishments of mathematics in the
20th century. Ultimately, these axioms can be broken down into a set of rules
for manipulating symbols that any technically oriented person can follow.

We will not spend much time trying to convey a deep, higher-level
understanding of the meaning of the axioms. This kind of understanding
requires some mathematical sophistication as well as an understanding of the
philosophy underlying the foundations of mathematics and typically develops
over time as you work with mathematics.  Our goal, instead, is to give you the
immediate ability to follow how theorems\index{theorem} are derived from the
axioms and from other theorems.  This will be similar to learning the syntax
of a computer language, which lets you follow the details in a program but
does not necessarily give you the ability to write non-trivial programs on
your own, an ability that comes with practice. For now don't be alarmed by
abstract-sounding names of the axioms; just focus on the rules for
manipulating the symbols, which follow the simple conventions of the
Metamath\index{Metamath} language.

The axioms that underlie all of standard mathematics consist of axioms of logic
and axioms of set theory. The axioms of logic are divided into two
subcategories, propositional calculus\index{propositional calculus} (sometimes
called sentential logic\index{sentential logic}) and predicate calculus
(sometimes called first-order logic\index{first-order logic}\index{quantifier
theory}\index{predicate calculus} or quantifier theory).  Propositional
calculus is a prerequisite for predicate calculus, and predicate calculus is a
prerequisite for set theory.  The version of set theory most commonly used is
Zermelo--Fraenkel set theory\index{Zermelo--Fraenkel set theory}\index{set theory}
with the axiom of choice,
often abbreviated as ZFC\index{ZFC}.

Here in a nutshell is what the axioms are all about in an informal way. The
connection between this description and symbols we will show you won't be
immediately apparent and in principle needn't ever be.  Our description just
tries to summarize what mathematicians think about when they work with the
axioms.

Logic is a set of rules that allow us determine truths given other truths.
Put another way,
logic is more or less the translation of what we would consider common sense
into a rigorous set of axioms.\index{axioms of logic}  Suppose $\varphi$,
$\psi$, and $\chi$ (the Greek letters phi, psi, and chi) represent statements
that are either true or false, and $x$ is a variable\index{variable!in predicate
calculus} ranging over some group of mathematical objects (sets, integers,
real numbers, etc.). In mathematics, a ``statement'' really means a formula,
and $\psi$ could be for example ``$x = 2$.''
Propositional calculus\index{propositional calculus}
allows us to use variables that are either true or false
and make deductions such as
``if $\varphi$ implies $\psi$ and $\psi$ implies $\chi$, then $\varphi$
implies $\chi$.''
Predicate calculus\index{predicate calculus}
extends propositional calculus by also allowing us
to discuss statements about objects (not just true and false values), including
statements about ``all'' or ``at least one'' object.
For example, predicate calculus allows to say,
``if $\varphi$ is true for all $x$, then $\varphi$ is true for some $x$.''
The logic used in \texttt{set.mm} is standard classical logic
(as opposed to other logic systems like intuitionistic logic).

Set theory\index{set theory} has to do with the manipulation of objects and
collections of objects, specifically the abstract, imaginary objects that
mathematics deals with, such as numbers. Everything that is claimed to exist
in mathematics is considered to be a set.  A set called the empty
set\index{empty set} contains nothing.  We represent the empty set by
$\varnothing$.  Many sets can be built up from the empty set.  There is a set
represented by $\{\varnothing\}$ that contains the empty set, another set
represented by $\{\varnothing,\{\varnothing\}\}$ that contains this set as
well as the empty set, another set represented by $\{\{\varnothing\}\}$ that
contains just the set that contains the empty set, and so on ad infinitum. All
mathematical objects, no matter how complex, are defined as being identical to
certain sets: the integer\index{integer} 0 is defined as the empty set, the
integer 1 is defined as $\{\varnothing\}$, the integer 2 is defined as
$\{\varnothing,\{\varnothing\}\}$.  (How these definitions were chosen doesn't
matter now, but the idea behind it is that these sets have the properties we
expect of integers once suitable operations are defined.)  Mathematical
operations, such as addition, are defined in terms of operations on
sets---their union\index{set union}, intersection\index{set intersection}, and
so on---operations you may have used in elementary school when you worked
with groups of apples and oranges.

With a leap of faith, the axioms also postulate the existence of infinite
sets\index{infinite set}, such as the set of all non-negative integers ($0, 1,
2,\ldots$, also called ``natural numbers''\index{natural number}).  This set
can't be represented with the brace notation\index{brace notation} we just
showed you, but requires a more complicated notation called ``class
abstraction.''\index{class abstraction}\index{abstraction class}  For
example, the infinite set $\{ x |
\mbox{``$x$ is a natural number''} \} $ means the ``set of all objects $x$
such that $x$ is a natural number'' i.e.\ the set of natural numbers; here,
``$x$ is a natural number'' is a rather complicated formula when broken down
into the primitive symbols.\label{expandom}\footnote{The statement ``$x$ is a
natural number'' is formally expressed as ``$x \in \omega$,'' where $\in$
(stylized epsilon) means ``is in'' or ``is an element of'' and $\omega$
(omega) means ``the set of natural numbers.''  When ``$x\in\omega$'' is
completely expanded in terms of the primitive symbols of set theory, the
result is  $\lnot$ $($ $\lnot$ $($ $\forall$ $z$ $($ $\lnot$ $\forall$ $w$ $($
$z$ $\in$ $w$ $\rightarrow$ $\lnot$ $w$ $\in$ $x$ $)$ $\rightarrow$ $z$ $\in$
$x$ $)$ $\rightarrow$ $($ $\forall$ $z$ $($ $\lnot$ $($ $\forall$ $w$ $($ $w$
$\in$ $z$ $\rightarrow$ $w$ $\in$ $x$ $)$ $\rightarrow$ $\forall$ $w$ $\lnot$
$w$ $\in$ $z$ $)$ $\rightarrow$ $\lnot$ $\forall$ $w$ $($ $w$ $\in$ $z$
$\rightarrow$ $\lnot$ $\forall$ $v$ $($ $v$ $\in$ $z$ $\rightarrow$ $\lnot$
$v$ $\in$ $w$ $)$ $)$ $)$ $\rightarrow$ $\lnot$ $\forall$ $z$ $\forall$ $w$
$($ $\lnot$ $($ $z$ $\in$ $x$ $\rightarrow$ $\lnot$ $w$ $\in$ $x$ $)$
$\rightarrow$ $($ $\lnot$ $z$ $\in$ $w$ $\rightarrow$ $($ $\lnot$ $z$ $=$ $w$
$\rightarrow$ $w$ $\in$ $z$ $)$ $)$ $)$ $)$ $)$ $\rightarrow$ $\lnot$
$\forall$ $y$ $($ $\lnot$ $($ $\lnot$ $($ $\forall$ $z$ $($ $\lnot$ $\forall$
$w$ $($ $z$ $\in$ $w$ $\rightarrow$ $\lnot$ $w$ $\in$ $y$ $)$ $\rightarrow$
$z$ $\in$ $y$ $)$ $\rightarrow$ $($ $\forall$ $z$ $($ $\lnot$ $($ $\forall$
$w$ $($ $w$ $\in$ $z$ $\rightarrow$ $w$ $\in$ $y$ $)$ $\rightarrow$ $\forall$
$w$ $\lnot$ $w$ $\in$ $z$ $)$ $\rightarrow$ $\lnot$ $\forall$ $w$ $($ $w$
$\in$ $z$ $\rightarrow$ $\lnot$ $\forall$ $v$ $($ $v$ $\in$ $z$ $\rightarrow$
$\lnot$ $v$ $\in$ $w$ $)$ $)$ $)$ $\rightarrow$ $\lnot$ $\forall$ $z$
$\forall$ $w$ $($ $\lnot$ $($ $z$ $\in$ $y$ $\rightarrow$ $\lnot$ $w$ $\in$
$y$ $)$ $\rightarrow$ $($ $\lnot$ $z$ $\in$ $w$ $\rightarrow$ $($ $\lnot$ $z$
$=$ $w$ $\rightarrow$ $w$ $\in$ $z$ $)$ $)$ $)$ $)$ $\rightarrow$ $($
$\forall$ $z$ $\lnot$ $z$ $\in$ $y$ $\rightarrow$ $\lnot$ $\forall$ $w$ $($
$\lnot$ $($ $w$ $\in$ $y$ $\rightarrow$ $\lnot$ $\forall$ $z$ $($ $w$ $\in$
$z$ $\rightarrow$ $\lnot$ $z$ $\in$ $y$ $)$ $)$ $\rightarrow$ $\lnot$ $($
$\lnot$ $\forall$ $z$ $($ $w$ $\in$ $z$ $\rightarrow$ $\lnot$ $z$ $\in$ $y$
$)$ $\rightarrow$ $w$ $\in$ $y$ $)$ $)$ $)$ $)$ $\rightarrow$ $x$ $\in$ $y$
$)$ $)$ $)$. Section~\ref{hierarchy} shows the hierarchy of definitions that
leads up to this expression.}\index{stylized epsilon ($\in$)}\index{omega
($\omega$)}  Actually, the primitive symbols don't even include the brace
notation.  The brace notation is a high-level definition, which you can find in
Section~\ref{hierarchy}.

Interestingly, the arithmetic of integers\index{integer} and
rationals\index{rational number} can be developed without appealing to the
existence of an infinite set, whereas the arithmetic of real
numbers\index{real number} requires it.

Each variable\index{variable!in set theory} in the axioms of set theory
represents an arbitrary set, and the axioms specify the legal kinds of things
you can do with these variables at a very primitive level.

Now, you may think that numbers and arithmetic are a lot more intuitive and
fundamental than sets and therefore should be the foundation of mathematics.
What is really the case is that you've dealt with numbers all your life and
are comfortable with a few rules for manipulating them such as addition and
multiplication.  Those rules only cover a small portion of what can be done
with numbers and only a very tiny fraction of the rest of mathematics.  If you
look at any elementary book on number theory, you will quickly become lost if
these are the only rules that you know.  Even though such books may present a
list of ``axioms''\index{axiom} for arithmetic, the ability to use the axioms
and to understand proofs of theorems\index{theorem} (facts) about numbers
requires an implicit mathematical talent that frustrates many people
from studying abstract mathematics.  The kind of mathematics that most people
know limits them to the practical, everyday usage of blindly manipulating
numbers and formulas, without any understanding of why those rules are correct
nor any ability to go any further.  For example, do you know why multiplying
two negative numbers yields a positive number?  Starting with set theory, you
will also start off blindly manipulating symbols according to the rules we give
you, but with the advantage that these rules will allow you, in principle, to
access {\em all} of mathematics, not just a tiny part of it.

Of course, concrete examples are often helpful in the learning process. For
example, you can verify that $2\cdot 3=3 \cdot 2$ by actually grouping
objects and can easily ``see'' how it generalizes to $x\cdot y = y\cdot x$,
even though you might not be able to rigorously prove it.  Similarly, in set
theory it can be helpful to understand how the axioms of set theory apply to
(and are correct for) small finite collections of objects.  You should be aware
that in set theory intuition can be misleading for infinite collections, and
rigorous proofs become more important.  For example, while $x\cdot y = y\cdot
x$ is correct for finite ordinals (which are the natural numbers), it is not
usually true for infinite ordinals.

\section{The Axioms for All of Mathematics}

In this section\index{axioms for mathematics}, we will show you the axioms
for all of standard mathematics (i.e.\ logic and set theory) as they are
traditionally presented.  The traditional presentation is useful for someone
with the mathematical experience needed to correctly manipulate high-level
abstract concepts.  For someone without this talent, knowing how to actually
make use of these axioms can be difficult.  The purpose of this section is to
allow you to see how the version of the axioms used in the standard
Metamath\index{Metamath} database \texttt{set.mm}\index{set
theory database (\texttt{set.mm})} relates to  the typical version
in textbooks, and also to give you an informal feel for them.

\subsection{Propositional Calculus}

Propositional calculus\index{propositional calculus} concerns itself with
statements that can be interpreted as either true or false.  Some examples of
statements (outside of mathematics) that are either true or false are ``It is
raining today'' and ``The United States has a female president.'' In
mathematics, as we mentioned, statements are really formulas.

In propositional calculus, we don't care what the statements are.  We also
treat a logical combination of statements, such as ``It is raining today and
the United States has a female president,'' no differently from a single
statement.  Statements and their combinations are called well-formed formulas
(wffs)\index{well-formed formula (wff)}.  We define wffs only in terms of
other wffs and don't define what a ``starting'' wff is.  As is common practice
in the literature, we use Greek letters to represent wffs.

Specifically, suppose $\varphi$ and $\psi$ are wffs.  Then the combinations
$\varphi\rightarrow\psi$ (``$\varphi$ implies $\psi$,'' also read ``if
$\varphi$ then $\psi$'')\index{implication ($\rightarrow$)} and $\lnot\varphi$
(``not $\varphi$'')\index{negation ($\lnot$)} are also wffs.

The three axioms of propositional calculus\index{axioms of propositional
calculus} are all wffs of the following form:\footnote{A remarkable result of
C.~A.~Meredith\index{Meredith, C. A.} squeezes these three axioms into the
single axiom $((((\varphi\rightarrow \psi)\rightarrow(\neg \chi\rightarrow\neg
\theta))\rightarrow \chi )\rightarrow \tau)\rightarrow((\tau\rightarrow
\varphi)\rightarrow(\theta\rightarrow \varphi))$ \cite{CAMeredith},
which is believed to be the shortest possible.}
\begin{center}
     $\varphi\rightarrow(\psi\rightarrow \varphi)$\\

     $(\varphi\rightarrow (\psi\rightarrow \chi))\rightarrow
((\varphi\rightarrow  \psi)\rightarrow (\varphi\rightarrow \chi))$\\

     $(\neg \varphi\rightarrow \neg\psi)\rightarrow (\psi\rightarrow
\varphi)$
\end{center}

These three axioms are widely used.
They are attributed to Jan {\L}ukasiewicz
(pronounced woo-kah-SHAY-vitch) and was popularized by Alonzo Church,
who called it system P2. (Thanks to Ted Ulrich for this information.)

There are an infinite number of axioms, one for each possible
wff\index{well-formed formula (wff)} of the above form.  (For this reason,
axioms such as the above are often called ``axiom schemes.''\index{axiom
scheme})  Each Greek letter in the axioms may be substituted with a more
complex wff to result in another axiom.  For example, substituting
$\neg(\varphi\rightarrow\chi)$ for $\varphi$ in the first axiom yields
$\neg(\varphi\rightarrow\chi)\rightarrow(\psi\rightarrow
\neg(\varphi\rightarrow\chi))$, which is still an axiom.

To deduce new true statements (theorems\index{theorem}) from the axioms, a
rule\index{rule} called ``modus ponens''\index{modus ponens} is used.  This
rule states that if the wff $\varphi$ is an axiom or a theorem, and the wff
$\varphi\rightarrow\psi$ is an axiom or a theorem, then the wff $\psi$ is also
a theorem\index{theorem}.

As a non-mathematical example of modus ponens, suppose we have proved (or
taken as an axiom) ``Bob is a man'' and separately have proved (or taken as
an axiom) ``If Bob is a man, then Bob is a human.''  Using the rule of modus
ponens, we can logically deduce, ``Bob is a human.''

From Metamath's\index{Metamath} point of view, the axioms and the rule of
modus ponens just define a mechanical means for deducing new true statements
from existing true statements, and that is the complete content of
propositional calculus as far as Metamath is concerned.  You can read a logic
textbook to gain a better understanding of their meaning, or you can just let
their meaning slowly become apparent to you after you use them for a while.

It is actually rather easy to check to see if a formula is a theorem of
propositional calculus.  Theorems of propositional calculus are also called
``tautologies.''\index{tautology}  The technique to check whether a formula is
a tautology is called the ``truth table method,''\index{truth table} and it
works like this.  A wff $\varphi\rightarrow\psi$ is false whenever $\varphi$ is true
and $\psi$ is false.  Otherwise it is true.  A wff $\lnot\varphi$ is false
whenever $\varphi$ is true and false otherwise. To verify a tautology such as
$\varphi\rightarrow(\psi\rightarrow \varphi)$, you break it down into sub-wffs and
construct a truth table that accounts for all possible combinations of true
and false assigned to the wff metavariables:
\begin{center}\begin{tabular}{|c|c|c|c|}\hline
\mbox{$\varphi$} & \mbox{$\psi$} & \mbox{$\psi\rightarrow\varphi$}
    & \mbox{$\varphi\rightarrow(\psi\rightarrow \varphi)$} \\ \hline \hline
              T   &  T    &      T       &        T    \\ \hline
              T   &  F    &      T       &        T    \\ \hline
              F   &  T    &      F       &        T    \\ \hline
              F   &  F    &      T       &        T    \\ \hline
\end{tabular}\end{center}
If all entries in the last column are true, the formula is a tautology.

Now, the truth table method doesn't tell you how to prove the tautology from
the axioms, but only that a proof exists.  Finding an actual proof (especially
one that is short and elegant) can be challenging.  Methods do exist for
automatically generating proofs in propositional calculus, but the proofs that
result can sometimes be very long.  In the Metamath \texttt{set.mm}\index{set
theory database (\texttt{set.mm})} database, most
or all proofs were created manually.

Section \ref{metadefprop} discusses various definitions
that make propositional calculus easier to use.
For example, we define:

\begin{itemize}
\item $\varphi \vee \psi$
  is true if either $\varphi$ or $\psi$ (or both) are true
  (this is disjunction\index{disjunction ($\vee$)}
  aka logical {\sc or}\index{logical {\sc or} ($\vee$)}).

\item $\varphi \wedge \psi$
  is true if both $\varphi$ and $\psi$ are true
  (this is conjunction\index{conjunction ($\wedge$)}
  aka logical {\sc and}\index{logical {\sc and} ($\wedge$)}).

\item $\varphi \leftrightarrow \psi$
  is true if $\varphi$ and $\psi$ have the same value, that is,
  they are both true or both false
  (this is the biconditional\index{biconditional ($\leftrightarrow$)}).
\end{itemize}

\subsection{Predicate Calculus}

Predicate calculus\index{predicate calculus} introduces the concept of
``individual variables,''\index{variable!in predicate calculus}\index{individual
variable} which
we will usually just call ``variables.''
These variables can represent something other than true or false (wffs),
and will always represent sets when we get to set theory.  There are also
three new symbols $\forall$\index{universal quantifier ($\forall$)},
$=$\index{equality ($=$)}, and $\in$\index{stylized epsilon ($\in$)},
read ``for all,'' ``equals,'' and ``is an element of''
respectively.  We will represent variables with the letters $x$, $y$, $z$, and
$w$, as is common practice in the literature.
For example, $\forall x \varphi$ means ``for all possible values of
$x$, $\varphi$ is true.''

In predicate calculus, we extend the definition of a wff\index{well-formed
formula (wff)}.  If $\varphi$ is a wff and $x$ and $y$ are variables, then
$\forall x \, \varphi$, $x=y$, and $x\in y$ are wffs. Note that these three new
types of wffs can be considered ``starting'' wffs from which we can build
other wffs with $\rightarrow$ and $\neg$ .  The concept of a starting wff was
absent in propositional calculus.  But starting wff or not, all we are really
concerned with is whether our wffs are correctly constructed according to
these mechanical rules.

A quick aside:
To prevent confusion, it might be best at this point to think of the variables
of Metamath\index{Metamath} as ``metavariables,''\index{metavariable} because
they are not quite the same as the variables we are introducing here.  A
(meta)variable in Metamath can be a wff or an individual variable, as well
as many other things; in general, it represents a kind of place holder for an
unspecified sequence of math symbols\index{math symbol}.

Unlike propositional calculus, no decision procedure\index{decision procedure}
analogous to the truth table method exists (nor theoretically can exist) that
will definitely determine whether a formula is a theorem of predicate
calculus.  Much of the work in the field of automated theorem
proving\index{automated theorem proving} has been dedicated to coming up with
clever heuristics for proving theorems of predicate calculus, but they can
never be guaranteed to work always.

Section \ref{metadefpred} discusses various definitions
that make predicate calculus easier to use.
For example, we define
$\exists x \varphi$ to mean
``there exists at least one possible value of $x$ where $\varphi$ is true.''

We now turn to looking at how predicate calculus can be formally
represented.

\subsubsection{Common Axioms}

There is a new rule of inference in predicate calculus:  if $\varphi$ is
an axiom or a theorem, then $\forall x \,\varphi$ is also a
theorem\index{theorem}.  This is called the rule of
``generalization.''\index{rule of generalization}
This is easily represented in Metamath.

In standard texts of logic, there are often two axioms of predicate
calculus\index{axioms of predicate calculus}:
\begin{center}
  $\forall x \,\varphi ( x ) \rightarrow \varphi ( y )$,
      where ``$y$ is properly substituted for $x$.''\\
  $\forall x ( \varphi \rightarrow \psi )\rightarrow ( \varphi \rightarrow
    \forall x\, \psi )$,
    where ``$x$ is not free in $\varphi$.''
\end{center}

Now at first glance, this seems simple:  just two axioms.  However,
conditional clauses are attached to each axiom describing requirements that
may seem puzzling to you.  In addition, the first axiom puts a variable symbol
in parentheses after each wff, seemingly violating our definition of a
wff\index{well-formed formula (wff)}; this is just an informal way of
referring to some arbitrary variable that may occur in the wff.  The
conditional clauses do, of course, have a precise meaning, but as it turns out
the precise meaning is somewhat complicated and awkward to formalize in a
way that a computer can handle easily.  Unlike propositional calculus, a
certain amount of mathematical sophistication and practice is needed to be
able to easily grasp and manipulate these concepts correctly.

Predicate calculus may be presented with or without axioms for
equality\index{axioms of equality}\index{equality ($=$)}. We will require the
axioms of equality as a prerequisite for the version of set theory we will
use.  The axioms for equality, when included, are often represented using these
two axioms:
\begin{center}
$x=x$\\ \ \\
$x=y\rightarrow (\varphi(x,x)\rightarrow\varphi(x,y))$ where ``$\varphi(x,y)$
   arises from $\varphi(x,x)$ by replacing some, but not necessarily all,
   free\index{free variable}
   occurrences of $x$ by $y$,\\ provided that $y$ is free for $x$
   in $\varphi(x,x)$.'' \end{center}
% (Mendelson p. 95)
The first equality axiom is simple, but again,
the condition on the second one is
somewhat awkward to implement on a computer.

\subsubsection{Tarski System S2}

Of course, we are not the first to notice the complications of these
predicate calculus axioms when being rigorous.

Well-known logician Alfred Tarski published in 1965
a system he called system S2\cite[p.~77]{Tarski1965}.
Tarski's system is \textit{exactly equivalent} to the traditional textbook
formalization, but (by clever use of equality axioms) it eliminates the
latter's primitive notions of ``proper substitution'' and ``free variable,''
replacing them with direct substitution and the notion of a variable
not occurring in a formula (which we express with distinct variable
constraints).

In advocating his system, Tarski wrote, ``The relatively complicated
character of [free variables and proper substitution] is a source
of certain inconveniences of both practical and theoretical nature;
this is clearly experienced both in teaching an elementary course of
mathematical logic and in formalizing the syntax of predicate logic for
some theoretical purposes''\cite[p.~61]{Tarski1965}\index{Tarski, Alfred}.

\subsubsection{Developing a Metamath Representation}

The standard textbook axioms of predicate calculus are somewhat
cumbersome to implement on a computer because of the complex notions of
``free variable''\index{free variable} and ``proper
substitution.''\index{proper substitution}\index{substitution!proper}
While it is possible to use the Metamath\index{Metamath} language to
implement these concepts, we have chosen not to implement them
as primitive constructs in the
\texttt{set.mm} set theory database.  Instead, we have eliminated them
within the axioms
by carefully crafting the axioms so as to avoid them,
building on Tarski's system S2.  This makes it
easy for a beginner to follow the steps in a proof without knowing any
advanced concepts other than the simple concept of
replacing\index{substitution!variable}\index{variable substitution}
variables with expressions.

In order to develop the concepts of free variable and proper
substitution from the axioms, we use an additional
Metamath statement type called ``disjoint variable
restriction''\index{disjoint variables} that we have not encountered
before.  In the context of the axioms, the statement \texttt{\$d} $ x\,
y$\index{\texttt{\$d} statement} simply means that $x$ and $y$ must be
distinct\index{distinct variables}, i.e.\ they may not be simultaneously
substituted\index{substitution!variable}\index{variable substitution}
with the same variable.  The statement \texttt{\$d} $ x\, \varphi$ means
variable $x$ must not occur in wff $\varphi$.  For the precise
definition of \texttt{\$d}, see Section~\ref{dollard}.

\subsubsection{Metamath representation}

The Metamath axiom system for predicate calculus
defined in set.mm uses Tarski's system S2.
As noted above, this has a different representation
than the traditional textbook formalization,
but it is \textit{exactly equivalent} to the textbook formalization,
and it is \textit{much} easier to work with.
This is reproduced as system S3 in Section 6 of
Megill's formalization \cite{Megill}\index{Megill, Norman}.

There is one exception, Tarski's axiom of existence,
which we label as axiom ax-6.
In the case of ax-6, Tarski's version is weaker because it includes a
distinct variable proviso. If we wish, we can also weaken our version
in this way and still have a metalogically complete system. Theorem
ax6 shows this by deriving, in the presence of the other axioms, our
ax-6 from Tarski's weaker version ax6v. However, we chose the stronger
version for our system because it is simpler to state and easier to use.

Tarski's system was designed for proving specific theorems rather than
more general theorem schemes. However, theorem schemes are much more
efficient than specific theorems for building a body of mathematical
knowledge, since they can be reused with different instances as
needed. While Tarski does derive some theorem schemes from his axioms,
their proofs require concepts that are ``outside'' of the system, such as
induction on formula length. The verification of such proofs is difficult
to automate in a proof verifier. (Specifically, Tarski treats the formulas
of his system as set-theoretical objects. In order to verify the proofs
of his theorem schemes, a proof verifier would need a significant amount
of set theory built into it.)

The Metamath axiom system for predicate calculus extends
Tarski's system to eliminate this difficulty. The additional
``auxilliary'' axiom
schemes (as we will call them in this section; see below) endow Tarski's
system with a nice property we call
metalogical completeness \cite[Remark 9.6]{Megill}\index{Megill, Norman}.
As a result, we can prove any theorem scheme
expressable in the ``simple metalogic'' of Tarski's system by using
only Metamath's direct substitution rule applied to the axiom system
(and no other metalogical or set-theoretical notions ``outside'' of the
system). Simple metalogic consists of schemes containing wff metavariables
(with no arguments) and/or set (also called ``individual'') metavariables,
accompanied by optional provisos each stating that two specified set
metavariables must be distinct or that a specified set metavariable may
not occur in a specified wff metavariable. Metamath's logic and set theory
axiom and rule schemes are all examples of simple metalogic. The schemes
of traditional predicate calculus with equality are examples which are
not simple metalogic, because they use wff metavariables with arguments
and have ``free for'' and ``not free in'' side conditions.

A rigorous justification for this system, using an older but
exactly equivalent set of axioms, can be
found in \cite{Megill}\index{Megill, Norman}.

This allows us to
take a different approach in the Metamath\index{Metamath} database
\texttt{set.mm}\index{set theory database (\texttt{set.mm})}.  We do not
directly use the primitive notions of ``free variable''\index{free variable}
and ``proper substitution''\index{proper
substitution}\index{substitution!proper} at all as primitive constructs.
Instead, we use a set
of axioms that are almost as simple to manipulate as those of
propositional calculus.  Our axiom system avoids complex primitive
notions by effectively embedding the complexity into the axioms
themselves.  As a result, we will end up with a larger number of axioms,
but they are ideally suited for a computer language such as Metamath.
(Section~\ref{metaaxioms} shows these axioms.)

We will not elaborate further
on the ``free variable'' and ``proper substitution''
concepts here.  You may consult
\cite[ch.\ 3--4]{Hamilton}\index{Hamilton, Alan G.} (as well as
many other books) for a precise explanation
of these concepts.  If you intend to do serious mathematical work, it is wise
to become familiar with the traditional textbook approach; even though the
concepts embedded in their axioms require a higher level of sophistication,
they can be more practical to deal with on an everyday, informal basis.  Even
if you are just developing Metamath proofs, familiarity with the traditional
approach can help you arrive at a proof outline much faster, which you can
then convert to the detail required by Metamath.

We do develop proper substitution rules later on, but in set.mm
they are defined as derived constructs; they are not primitives.

You should also note that our system of predicate calculus is specifically
tailored for set theory; thus there are only two specific predicates $=$ and
$\in$ and no functions\index{function!in predicate calculus}
or constants\index{constant!in predicate calculus} unlike more general systems.
We later add these.

\subsection{Set Theory}

Traditional Zermelo--Fraenkel set theory\index{Zermelo--Fraenkel set
theory}\index{set theory} with the Axiom of Choice
has 10 axioms, which can be expressed in the
language of predicate calculus.  In this section, we will list only the
names and brief English descriptions of these axioms, since we will give
you the precise formulas used by the Metamath\index{Metamath} set theory
database \texttt{set.mm} later on.

In the descriptions of the axioms, we assume that $x$, $y$, $z$, $w$, and $v$
represent sets.  These are the same as the variables\index{variable!in set
theory} in our predicate calculus system above, except that now we informally
think of the variables as ranging over sets.  Note that the terms
``object,''\index{object} ``set,''\index{set} ``element,''\index{element}
``collection,''\index{collection} and ``family''\index{family} are synonymous,
as are ``is an element of,'' ``is a member of,''\index{member} ``is contained
in,'' and ``belongs to.''  The different terms are used for convenience; for
example, ``a collection of sets'' is less confusing than ``a set of sets.''
A set $x$ is said to be a ``subset''\index{subset} of $y$ if every element of
$x$ is also an element of $y$; we also say $x$ is ``included in''
$y$.

The axioms are very general and apply to almost any conceivable mathematical
object, and this level of abstraction can be overwhelming at first.  To gain an
intuitive feel, it can be helpful to draw a picture illustrating the concept;
for example, a circle containing dots could represent a collection of sets,
and a smaller circle drawn inside the circle could represent a subset.
Overlapping circles can illustrate intersection and union.  Circles that
illustrate the concepts of set theory are frequently used in elementary
textbooks and are called Venn diagrams\index{Venn diagram}.\index{axioms of
set theory}

1. Axiom of Extensionality:  Two sets are identical if they contain the same
   elements.\index{Axiom of Extensionality}

2. Axiom of Pairing:  The set $\{ x , y \}$ exists.\index{Axiom of Pairing}

3. Axiom of Power Sets:  The power set of a set (the collection of all of
   its subsets) exists.  For example, the power set of $\{x,y\}$ is
   $\{\varnothing,\{x\},\{y\},\{x,y\}\}$ and it exists.\index{Axiom
of Power Sets}

4. Axiom of the Null Set:  The empty set $\varnothing$ exists.\index{Axiom of
the Null Set}

5. Axiom of Union:  The union of a set (the set containing the elements of
   its members) exists.  For example, the union of $\{\{x,y\},\{z\}\}$ is
 $\{x,y,z\}$ and
   it exists.\index{Axiom of Union}

6. Axiom of Regularity:  Roughly, no set can contain itself, nor can there
   be membership ``loops,'' such as a set being an
   element of one of its members.\index{Axiom of Regularity}

7. Axiom of Infinity:  An infinite set exists.  An example of an infinite
   set is the set of all
   integers.\index{Axiom of Infinity}

8. Axiom of Separation:  The set exists that is obtained by restricting $x$
   with some property.  For example, if the set of all integers exists,
   then the set of all even integers exists.\index{Axiom of Separation}

9. Axiom of Replacement:  The range of a function whose domain is restricted
   to the elements of a set $x$, is also a set.  For example, there
   is a function
   from integers (the function's domain) to their squares (its
   range).  If we
   restrict the domain to even integers, its range will become the set of
   squares of even integers, so this axiom asserts that the set of
    squares of even numbers exists.  Technical note:  In general, the
   ``function'' need not be a set but can be a proper class.
   \index{Axiom of Replacement}

10. Axiom of Choice:  Let $x$ be a set whose members are pairwise
  disjoint\index{disjoint sets} (i.e,
  whose members contain no elements in common).  Then there exists another
  set containing one element from each member of $x$.  For
  example, if $x$ is
  $\{\{y,z\},\{w,v\}\}$, where $y$, $z$, $w$, and $v$ are
  different sets, then a set such as $\{z,w\}$
  exists (but the axiom doesn't tell
  us which one).  (Actually the Axiom
  of Choice is redundant if the set $x$, as in this example, has a finite
  number of elements.)\index{Axiom of Choice}

The Axiom of Choice is usually considered an extension of ZF set theory rather
than a proper part of it.  It is sometimes considered philosophically
controversial because it specifies the existence of a set without specifying
what the set is. Constructive logics, including intuitionistic logic,
do not accept the axiom of choice.
Since there is some lingering controversy, we often prefer proofs that do
not use the axiom of choice (where there is a known alternative), and
in some cases we will use weaker axioms than the full axiom of choice.
That said, the axiom of choice is a powerful and widely-accepted tool,
so we do use it when needed.
ZF set theory that includes the Axiom of Choice is
called Zermelo--Fraenkel set theory with choice (ZFC\index{ZFC set theory}).

When expressed symbolically, the Axiom of Separation and the Axiom of
Replacement contain wff symbols and therefore each represent infinitely many
axioms, one for each possible wff. For this reason, they are often called
axiom schemes\index{axiom scheme}\index{well-formed formula (wff)}.

It turns out that the Axiom of the Null Set, the Axiom of Pairing, and the
Axiom of Separation can be derived from the other axioms and are therefore
unnecessary, although they tend to be included in standard texts for various
reasons (historical, philosophical, and possibly because some authors may not
know this).  In the Metamath\index{Metamath} set theory database, these
redundant axioms are derived from the other ones instead of truly
being considered axioms.
This is in keeping with our general goal of minimizing the number of
axioms we must depend on.

\subsection{Other Axioms}

Above we qualified the phrase ``all of mathematics'' with ``essentially.''
The main important missing piece is the ability to do category theory,
which requires huge sets (inaccessible cardinals) larger than those
postulated by the ZFC axioms. The Tarski--Grothendieck Axiom postulates
the existence of such sets.
Note that this is the same axiom used by Mizar for supporting
category theory.
The Tarski--Grothendieck axiom
can be viewed as a very strong replacement of the Axiom of Infinity,
the Axiom of Choice, and the Axiom of Power Sets.
The \texttt{set.mm} database includes this axiom; see the database
for details about it.
Again, we only use this axiom when we need to.
You are only likely to encounter or use this axiom if you are doing
category theory, since its use is highly specialized,
so we will not list the Tarsky-Grothendieck axiom
in the short list of axioms below.

Can there be even more axioms?
Of course.
G\"{o}del showed that no finite set of axioms or axiom schemes can completely
describe any consistent theory strong enough to include arithmetic.
But practically speaking, the ones above are the accepted foundation that
almost all mathematicians explicitly or implicitly base their work on.

\section{The Axioms in the Metamath Language}\label{metaaxioms}

Here we list the axioms as they appear in
\texttt{set.mm}\index{set theory database (\texttt{set.mm})} so you can
look them up there easily.  Incidentally, the \texttt{show statement
/tex} command\index{\texttt{show statement} command} was used to
typeset them.

%macros from show statement /tex
\newbox\mlinebox
\newbox\mtrialbox
\newbox\startprefix  % Prefix for first line of a formula
\newbox\contprefix  % Prefix for continuation line of a formula
\def\startm{  % Initialize formula line
  \setbox\mlinebox=\hbox{\unhcopy\startprefix}
}
\def\m#1{  % Add a symbol to the formula
  \setbox\mtrialbox=\hbox{\unhcopy\mlinebox $\,#1$}
  \ifdim\wd\mtrialbox>\hsize
    \box\mlinebox
    \setbox\mlinebox=\hbox{\unhcopy\contprefix $\,#1$}
  \else
    \setbox\mlinebox=\hbox{\unhbox\mtrialbox}
  \fi
}
\def\endm{  % Output the last line of a formula
  \box\mlinebox
}

% \SLASH for \ , \TOR for \/ (text OR), \TAND for /\ (text and)
% This embeds a following forced space to force the space.
\newcommand\SLASH{\char`\\~}
\newcommand\TOR{\char`\\/~}
\newcommand\TAND{/\char`\\~}
%
% Macro to output metamath raw text.
% This assumes \startprefix and \contprefix are set.
% NOTE: "\" is tricky to escape, use \SLASH, \TOR, and \TAND inside.
% Any use of "$ { ~ ^" must be escaped; ~ and ^ must be escaped specially.
% We escape { and } for consistency.
% For more about how this macro written, see:
% https://stackoverflow.com/questions/4073674/
% how-to-disable-indentation-in-particular-section-in-latex/4075706
% Use frenchspacing, or "e." will get an extra space after it.
\newlength\mystoreparindent
\newlength\mystorehangindent
\newenvironment{mmraw}{%
\setlength{\mystoreparindent}{\the\parindent}
\setlength{\mystorehangindent}{\the\hangindent}
\setlength{\parindent}{0pt} % TODO - we'll put in the \startprefix instead
\setlength{\hangindent}{\wd\the\contprefix}
\begin{flushleft}
\begin{frenchspacing}
\begin{tt}
{\unhcopy\startprefix}%
}{%
\end{tt}
\end{frenchspacing}
\end{flushleft}
\setlength{\parindent}{\mystoreparindent}
\setlength{\hangindent}{\mystorehangindent}
\vskip 1ex
}

\needspace{5\baselineskip}
\subsection{Propositional Calculus}\label{propcalc}\index{axioms of
propositional calculus}

\needspace{2\baselineskip}
Axiom of Simplification.\label{ax1}

\setbox\startprefix=\hbox{\tt \ \ ax-1\ \$a\ }
\setbox\contprefix=\hbox{\tt \ \ \ \ \ \ \ \ \ \ }
\startm
\m{\vdash}\m{(}\m{\varphi}\m{\rightarrow}\m{(}\m{\psi}\m{\rightarrow}\m{\varphi}\m{)}
\m{)}
\endm

\needspace{3\baselineskip}
\noindent Axiom of Distribution.

\setbox\startprefix=\hbox{\tt \ \ ax-2\ \$a\ }
\setbox\contprefix=\hbox{\tt \ \ \ \ \ \ \ \ \ \ }
\startm
\m{\vdash}\m{(}\m{(}\m{\varphi}\m{\rightarrow}\m{(}\m{\psi}\m{\rightarrow}\m{\chi}
\m{)}\m{)}\m{\rightarrow}\m{(}\m{(}\m{\varphi}\m{\rightarrow}\m{\psi}\m{)}\m{
\rightarrow}\m{(}\m{\varphi}\m{\rightarrow}\m{\chi}\m{)}\m{)}\m{)}
\endm

\needspace{2\baselineskip}
\noindent Axiom of Contraposition.

\setbox\startprefix=\hbox{\tt \ \ ax-3\ \$a\ }
\setbox\contprefix=\hbox{\tt \ \ \ \ \ \ \ \ \ \ }
\startm
\m{\vdash}\m{(}\m{(}\m{\lnot}\m{\varphi}\m{\rightarrow}\m{\lnot}\m{\psi}\m{)}\m{
\rightarrow}\m{(}\m{\psi}\m{\rightarrow}\m{\varphi}\m{)}\m{)}
\endm


\needspace{4\baselineskip}
\noindent Rule of Modus Ponens.\label{axmp}\index{modus ponens}

\setbox\startprefix=\hbox{\tt \ \ min\ \$e\ }
\setbox\contprefix=\hbox{\tt \ \ \ \ \ \ \ \ \ }
\startm
\m{\vdash}\m{\varphi}
\endm

\setbox\startprefix=\hbox{\tt \ \ maj\ \$e\ }
\setbox\contprefix=\hbox{\tt \ \ \ \ \ \ \ \ \ }
\startm
\m{\vdash}\m{(}\m{\varphi}\m{\rightarrow}\m{\psi}\m{)}
\endm

\setbox\startprefix=\hbox{\tt \ \ ax-mp\ \$a\ }
\setbox\contprefix=\hbox{\tt \ \ \ \ \ \ \ \ \ \ \ }
\startm
\m{\vdash}\m{\psi}
\endm


\needspace{7\baselineskip}
\subsection{Axioms of Predicate Calculus with Equality---Tarski's S2}\index{axioms of predicate calculus}

\needspace{3\baselineskip}
\noindent Rule of Generalization.\index{rule of generalization}

\setbox\startprefix=\hbox{\tt \ \ ax-g.1\ \$e\ }
\setbox\contprefix=\hbox{\tt \ \ \ \ \ \ \ \ \ \ \ \ }
\startm
\m{\vdash}\m{\varphi}
\endm

\setbox\startprefix=\hbox{\tt \ \ ax-gen\ \$a\ }
\setbox\contprefix=\hbox{\tt \ \ \ \ \ \ \ \ \ \ \ \ }
\startm
\m{\vdash}\m{\forall}\m{x}\m{\varphi}
\endm

\needspace{2\baselineskip}
\noindent Axiom of Quantified Implication.

\setbox\startprefix=\hbox{\tt \ \ ax-4\ \$a\ }
\setbox\contprefix=\hbox{\tt \ \ \ \ \ \ \ \ \ \ }
\startm
\m{\vdash}\m{(}\m{\forall}\m{x}\m{(}\m{\forall}\m{x}\m{\varphi}\m{\rightarrow}\m{
\psi}\m{)}\m{\rightarrow}\m{(}\m{\forall}\m{x}\m{\varphi}\m{\rightarrow}\m{
\forall}\m{x}\m{\psi}\m{)}\m{)}
\endm

\needspace{3\baselineskip}
\noindent Axiom of Distinctness.

% Aka: Add $d x ph $.
\setbox\startprefix=\hbox{\tt \ \ ax-5\ \$a\ }
\setbox\contprefix=\hbox{\tt \ \ \ \ \ \ \ \ \ \ }
\startm
\m{\vdash}\m{(}\m{\varphi}\m{\rightarrow}\m{\forall}\m{x}\m{\varphi}\m{)}\m{where}\m{ }\m{\$d}\m{ }\m{x}\m{ }\m{\varphi}\m{ }\m{(}\m{x}\m{ }\m{does}\m{ }\m{not}\m{ }\m{occur}\m{ }\m{in}\m{ }\m{\varphi}\m{)}
\endm

\needspace{2\baselineskip}
\noindent Axiom of Existence.

\setbox\startprefix=\hbox{\tt \ \ ax-6\ \$a\ }
\setbox\contprefix=\hbox{\tt \ \ \ \ \ \ \ \ \ \ }
\startm
\m{\vdash}\m{(}\m{\forall}\m{x}\m{(}\m{x}\m{=}\m{y}\m{\rightarrow}\m{\forall}
\m{x}\m{\varphi}\m{)}\m{\rightarrow}\m{\varphi}\m{)}
\endm

\needspace{2\baselineskip}
\noindent Axiom of Equality.

\setbox\startprefix=\hbox{\tt \ \ ax-7\ \$a\ }
\setbox\contprefix=\hbox{\tt \ \ \ \ \ \ \ \ \ \ }
\startm
\m{\vdash}\m{(}\m{x}\m{=}\m{y}\m{\rightarrow}\m{(}\m{x}\m{=}\m{z}\m{
\rightarrow}\m{y}\m{=}\m{z}\m{)}\m{)}
\endm

\needspace{2\baselineskip}
\noindent Axiom of Left Equality for Binary Predicate.

\setbox\startprefix=\hbox{\tt \ \ ax-8\ \$a\ }
\setbox\contprefix=\hbox{\tt \ \ \ \ \ \ \ \ \ \ \ }
\startm
\m{\vdash}\m{(}\m{x}\m{=}\m{y}\m{\rightarrow}\m{(}\m{x}\m{\in}\m{z}\m{
\rightarrow}\m{y}\m{\in}\m{z}\m{)}\m{)}
\endm

\needspace{2\baselineskip}
\noindent Axiom of Right Equality for Binary Predicate.

\setbox\startprefix=\hbox{\tt \ \ ax-9\ \$a\ }
\setbox\contprefix=\hbox{\tt \ \ \ \ \ \ \ \ \ \ \ }
\startm
\m{\vdash}\m{(}\m{x}\m{=}\m{y}\m{\rightarrow}\m{(}\m{z}\m{\in}\m{x}\m{
\rightarrow}\m{z}\m{\in}\m{y}\m{)}\m{)}
\endm


\needspace{4\baselineskip}
\subsection{Axioms of Predicate Calculus with Equality---Auxiliary}\index{axioms of predicate calculus - auxiliary}

\needspace{2\baselineskip}
\noindent Axiom of Quantified Negation.

\setbox\startprefix=\hbox{\tt \ \ ax-10\ \$a\ }
\setbox\contprefix=\hbox{\tt \ \ \ \ \ \ \ \ \ \ }
\startm
\m{\vdash}\m{(}\m{\lnot}\m{\forall}\m{x}\m{\lnot}\m{\forall}\m{x}\m{\varphi}\m{
\rightarrow}\m{\varphi}\m{)}
\endm

\needspace{2\baselineskip}
\noindent Axiom of Quantifier Commutation.

\setbox\startprefix=\hbox{\tt \ \ ax-11\ \$a\ }
\setbox\contprefix=\hbox{\tt \ \ \ \ \ \ \ \ \ \ }
\startm
\m{\vdash}\m{(}\m{\forall}\m{x}\m{\forall}\m{y}\m{\varphi}\m{\rightarrow}\m{
\forall}\m{y}\m{\forall}\m{x}\m{\varphi}\m{)}
\endm

\needspace{3\baselineskip}
\noindent Axiom of Substitution.

\setbox\startprefix=\hbox{\tt \ \ ax-12\ \$a\ }
\setbox\contprefix=\hbox{\tt \ \ \ \ \ \ \ \ \ \ \ }
\startm
\m{\vdash}\m{(}\m{\lnot}\m{\forall}\m{x}\m{\,x}\m{=}\m{y}\m{\rightarrow}\m{(}
\m{x}\m{=}\m{y}\m{\rightarrow}\m{(}\m{\varphi}\m{\rightarrow}\m{\forall}\m{x}\m{(}
\m{x}\m{=}\m{y}\m{\rightarrow}\m{\varphi}\m{)}\m{)}\m{)}\m{)}
\endm

\needspace{3\baselineskip}
\noindent Axiom of Quantified Equality.

\setbox\startprefix=\hbox{\tt \ \ ax-13\ \$a\ }
\setbox\contprefix=\hbox{\tt \ \ \ \ \ \ \ \ \ \ \ }
\startm
\m{\vdash}\m{(}\m{\lnot}\m{\forall}\m{z}\m{\,z}\m{=}\m{x}\m{\rightarrow}\m{(}
\m{\lnot}\m{\forall}\m{z}\m{\,z}\m{=}\m{y}\m{\rightarrow}\m{(}\m{x}\m{=}\m{y}
\m{\rightarrow}\m{\forall}\m{z}\m{\,x}\m{=}\m{y}\m{)}\m{)}\m{)}
\endm

% \noindent Axiom of Quantifier Substitution
%
% \setbox\startprefix=\hbox{\tt \ \ ax-c11n\ \$a\ }
% \setbox\contprefix=\hbox{\tt \ \ \ \ \ \ \ \ \ \ \ }
% \startm
% \m{\vdash}\m{(}\m{\forall}\m{x}\m{\,x}\m{=}\m{y}\m{\rightarrow}\m{(}\m{\forall}
% \m{x}\m{\varphi}\m{\rightarrow}\m{\forall}\m{y}\m{\varphi}\m{)}\m{)}
% \endm
%
% \noindent Axiom of Distinct Variables. (This axiom requires
% that two individual variables
% be distinct\index{\texttt{\$d} statement}\index{distinct
% variables}.)
%
% \setbox\startprefix=\hbox{\tt \ \ \ \ \ \ \ \ \$d\ }
% \setbox\contprefix=\hbox{\tt \ \ \ \ \ \ \ \ \ \ \ }
% \startm
% \m{x}\m{\,}\m{y}
% \endm
%
% \setbox\startprefix=\hbox{\tt \ \ ax-c16\ \$a\ }
% \setbox\contprefix=\hbox{\tt \ \ \ \ \ \ \ \ \ \ \ }
% \startm
% \m{\vdash}\m{(}\m{\forall}\m{x}\m{\,x}\m{=}\m{y}\m{\rightarrow}\m{(}\m{\varphi}\m{
% \rightarrow}\m{\forall}\m{x}\m{\varphi}\m{)}\m{)}
% \endm

% \noindent Axiom of Quantifier Introduction (2).  (This axiom requires
% that the individual variable not occur in the
% wff\index{\texttt{\$d} statement}\index{distinct variables}.)
%
% \setbox\startprefix=\hbox{\tt \ \ \ \ \ \ \ \ \$d\ }
% \setbox\contprefix=\hbox{\tt \ \ \ \ \ \ \ \ \ \ \ }
% \startm
% \m{x}\m{\,}\m{\varphi}
% \endm
% \setbox\startprefix=\hbox{\tt \ \ ax-5\ \$a\ }
% \setbox\contprefix=\hbox{\tt \ \ \ \ \ \ \ \ \ \ \ }
% \startm
% \m{\vdash}\m{(}\m{\varphi}\m{\rightarrow}\m{\forall}\m{x}\m{\varphi}\m{)}
% \endm

\subsection{Set Theory}\label{mmsettheoryaxioms}

In order to make the axioms of set theory\index{axioms of set theory} a little
more compact, there are several definitions from logic that we make use of
implicitly, namely, ``logical {\sc and},''\index{conjunction ($\wedge$)}
\index{logical {\sc and} ($\wedge$)} ``logical equivalence,''\index{logical
equivalence ($\leftrightarrow$)}\index{biconditional ($\leftrightarrow$)} and
``there exists.''\index{existential quantifier ($\exists$)}

\begin{center}\begin{tabular}{rcl}
  $( \varphi \wedge \psi )$ &\mbox{stands for}& $\neg ( \varphi
     \rightarrow \neg \psi )$\\
  $( \varphi \leftrightarrow \psi )$& \mbox{stands
     for}& $( ( \varphi \rightarrow \psi ) \wedge
     ( \psi \rightarrow \varphi ) )$\\
  $\exists x \,\varphi$ &\mbox{stands for}& $\neg \forall x \neg \varphi$
\end{tabular}\end{center}

In addition, the axioms of set theory require that all variables be
dis\-tinct,\index{distinct variables}\footnote{Set theory axioms can be
devised so that {\em no} variables are required to be distinct,
provided we replace \texttt{ax-c16} with an axiom stating that ``at
least two things exist,'' thus
making \texttt{ax-5} the only other axiom requiring the
\texttt{\$d} statement.  These axioms are unconventional and are not
presented here, but they can be found on the \url{http://metamath.org}
web site.  See also the Comment on
p.~\pageref{nodd}.}\index{\texttt{\$d} statement} thus we also assume:
\begin{center}
  \texttt{\$d }$x\,y\,z\,w$
\end{center}

\needspace{2\baselineskip}
\noindent Axiom of Extensionality.\index{Axiom of Extensionality}

\setbox\startprefix=\hbox{\tt \ \ ax-ext\ \$a\ }
\setbox\contprefix=\hbox{\tt \ \ \ \ \ \ \ \ \ \ \ \ }
\startm
\m{\vdash}\m{(}\m{\forall}\m{x}\m{(}\m{x}\m{\in}\m{y}\m{\leftrightarrow}\m{x}
\m{\in}\m{z}\m{)}\m{\rightarrow}\m{y}\m{=}\m{z}\m{)}
\endm

\needspace{3\baselineskip}
\noindent Axiom of Replacement.\index{Axiom of Replacement}

\setbox\startprefix=\hbox{\tt \ \ ax-rep\ \$a\ }
\setbox\contprefix=\hbox{\tt \ \ \ \ \ \ \ \ \ \ \ \ }
\startm
\m{\vdash}\m{(}\m{\forall}\m{w}\m{\exists}\m{y}\m{\forall}\m{z}\m{(}\m{%
\forall}\m{y}\m{\varphi}\m{\rightarrow}\m{z}\m{=}\m{y}\m{)}\m{\rightarrow}\m{%
\exists}\m{y}\m{\forall}\m{z}\m{(}\m{z}\m{\in}\m{y}\m{\leftrightarrow}\m{%
\exists}\m{w}\m{(}\m{w}\m{\in}\m{x}\m{\wedge}\m{\forall}\m{y}\m{\varphi}\m{)}%
\m{)}\m{)}
\endm

\needspace{2\baselineskip}
\noindent Axiom of Union.\index{Axiom of Union}

\setbox\startprefix=\hbox{\tt \ \ ax-un\ \$a\ }
\setbox\contprefix=\hbox{\tt \ \ \ \ \ \ \ \ \ \ \ }
\startm
\m{\vdash}\m{\exists}\m{x}\m{\forall}\m{y}\m{(}\m{\exists}\m{x}\m{(}\m{y}\m{
\in}\m{x}\m{\wedge}\m{x}\m{\in}\m{z}\m{)}\m{\rightarrow}\m{y}\m{\in}\m{x}\m{)}
\endm

\needspace{2\baselineskip}
\noindent Axiom of Power Sets.\index{Axiom of Power Sets}

\setbox\startprefix=\hbox{\tt \ \ ax-pow\ \$a\ }
\setbox\contprefix=\hbox{\tt \ \ \ \ \ \ \ \ \ \ \ \ }
\startm
\m{\vdash}\m{\exists}\m{x}\m{\forall}\m{y}\m{(}\m{\forall}\m{x}\m{(}\m{x}\m{
\in}\m{y}\m{\rightarrow}\m{x}\m{\in}\m{z}\m{)}\m{\rightarrow}\m{y}\m{\in}\m{x}
\m{)}
\endm

\needspace{3\baselineskip}
\noindent Axiom of Regularity.\index{Axiom of Regularity}

\setbox\startprefix=\hbox{\tt \ \ ax-reg\ \$a\ }
\setbox\contprefix=\hbox{\tt \ \ \ \ \ \ \ \ \ \ \ \ }
\startm
\m{\vdash}\m{(}\m{\exists}\m{x}\m{\,x}\m{\in}\m{y}\m{\rightarrow}\m{\exists}
\m{x}\m{(}\m{x}\m{\in}\m{y}\m{\wedge}\m{\forall}\m{z}\m{(}\m{z}\m{\in}\m{x}\m{
\rightarrow}\m{\lnot}\m{z}\m{\in}\m{y}\m{)}\m{)}\m{)}
\endm

\needspace{3\baselineskip}
\noindent Axiom of Infinity.\index{Axiom of Infinity}

\setbox\startprefix=\hbox{\tt \ \ ax-inf\ \$a\ }
\setbox\contprefix=\hbox{\tt \ \ \ \ \ \ \ \ \ \ \ \ \ \ \ }
\startm
\m{\vdash}\m{\exists}\m{x}\m{(}\m{y}\m{\in}\m{x}\m{\wedge}\m{\forall}\m{y}%
\m{(}\m{y}\m{\in}\m{x}\m{\rightarrow}\m{\exists}\m{z}\m{(}\m{y}\m{\in}\m{z}\m{%
\wedge}\m{z}\m{\in}\m{x}\m{)}\m{)}\m{)}
\endm

\needspace{4\baselineskip}
\noindent Axiom of Choice.\index{Axiom of Choice}

\setbox\startprefix=\hbox{\tt \ \ ax-ac\ \$a\ }
\setbox\contprefix=\hbox{\tt \ \ \ \ \ \ \ \ \ \ \ \ \ \ }
\startm
\m{\vdash}\m{\exists}\m{x}\m{\forall}\m{y}\m{\forall}\m{z}\m{(}\m{(}\m{y}\m{%
\in}\m{z}\m{\wedge}\m{z}\m{\in}\m{w}\m{)}\m{\rightarrow}\m{\exists}\m{w}\m{%
\forall}\m{y}\m{(}\m{\exists}\m{w}\m{(}\m{(}\m{y}\m{\in}\m{z}\m{\wedge}\m{z}%
\m{\in}\m{w}\m{)}\m{\wedge}\m{(}\m{y}\m{\in}\m{w}\m{\wedge}\m{w}\m{\in}\m{x}%
\m{)}\m{)}\m{\leftrightarrow}\m{y}\m{=}\m{w}\m{)}\m{)}
\endm

\subsection{That's It}

There you have it, the axioms for (essentially) all of mathematics!
Wonder at them and stare at them in awe.  Put a copy in your wallet, and
you will carry in your pocket the encoding for all theorems ever proved
and that ever will be proved, from the most mundane to the most
profound.

\section{A Hierarchy of Definitions}\label{hierarchy}

The axioms in the previous section in principle embody everything that can be
done within standard mathematics.  However, it is impractical to accomplish
very much by using them directly, for even simple concepts (from a human
perspective) can involve extremely long, incomprehensible formulas.
Mathematics is made practical by introducing definitions\index{definition}.
Definitions usually introduce new symbols, or at least new relationships among
existing symbols, to abbreviate more complex formulas.  An important
requirement for a definition is that there exist a straightforward
(algorithmic) method for eliminating the abbreviation by expanding it into the
more primitive symbol string that it represents.  Some
important definitions included in
the file \texttt{set.mm} are listed in this section for reference, and also to
give you a feel for why something like $\omega$\index{omega ($\omega$)} (the
set of natural numbers\index{natural number} 0, 1, 2,\ldots) becomes very
complicated when completely expanded into primitive symbols.

What is the motivation for definitions, aside from allowing complicated
expressions to be expressed more simply?  In the case of  $\omega$, one goal is
to provide a basis for the theory of natural numbers.\index{natural number}
Before set theory was invented, a set of axioms for arithmetic, called Peano's
postulates\index{Peano's postulates}, was devised and shown to have the
properties one expects for natural numbers.  Now anyone can postulate a
set of axioms, but if the axioms are inconsistent contradictions can be derived
from them.  Once a contradiction is derived, anything can be trivially
proved, including
all the facts of arithmetic and their negations.  To ensure that an
axiom system is at least as reliable as the axioms for set theory, we can
define sets and operations on those sets that satisfy the new axioms. In the
\texttt{set.mm} Metamath database, we prove that the elements of $\omega$ satisfy
Peano's postulates, and it's a long and hard journey to get there directly
from the axioms of set theory.  But the result is confidence in the
foundations of arithmetic.  And there is another advantage:  we now have all
the tools of set theory at our disposal for manipulating objects that obey the
axioms for arithmetic.

What are the criteria we use for definitions?  First, and of utmost importance,
the definition should not be {\em creative}\index{creative
definition}\index{definition!creative}, that
is it should not allow an expression that previously qualified as a wff but
was not provable, to become provable.   Second, the definition should be {\em
eliminable}\index{definition!eliminability}, that is, there should exist an
algorithmic method for converting any expression using the definition into
a logically equivalent expression that previously qualified as a wff.

In almost all cases below, definitions connect two expressions with either
$\leftrightarrow$ or $=$.  Eliminating\footnote{Here we mean the
elimination that a human might do in his or her head.  To eliminate them as
part of a Metamath proof we would invoke one of a number of
theorems that deal with transitivity of equivalence or equality; there are
many such examples in the proofs in \texttt{set.mm}.} such a definition is a
simple matter of substituting the expression on the left-hand side ({\em
definiendum}\index{definiendum} or thing being defined) with the equivalent,
more primitive expression on the right-hand side ({\em
definiens}\index{definiens} or definition).

Often a definition has variables on the right-hand side which do not appear on
the left-hand side; these are called {\em dummy variables}.\index{dummy
variable!in definitions}  In this case, any
allowable substitution (such as a new, distinct
variable) can be used when the definition is eliminated.  Dummy variables may
be used only if they are {\em effectively bound}\index{effectively bound
variable}, meaning that the definition will remain logically equivalent upon
any substitution of a dummy variable with any other {\em qualifying
expression}\index{qualifying expression}, i.e.\ any symbol string (such as
another variable) that
meets the restrictions on the dummy variable imposed by \texttt{\$d} and
\texttt{\$f} statements.  For example, we could define a constant $\perp$
(inverted tee, meaning logical ``false'') as $( \varphi \wedge \lnot \varphi
)$, i.e.\ ``phi and not phi.''  Here $\varphi$ is effectively bound because the
definition remains logically equivalent when we replace $\varphi$ with any
other wff.  (It is actually \texttt{df-fal}
in \texttt{set.mm}, which defines $\perp$.)

There are two cases where eliminating definitions is a little more
complex.  These cases are the definitions \texttt{df-bi} and
\texttt{df-cleq}.  The first stretches the concept of a definition a
little, as in effect it ``defines a definition;'' however, it meets our
requirements for a definition in that it is eliminable and does not
strengthen the language.  Theorem \texttt{bii} shows the substitution
needed to eliminate the $\leftrightarrow$\index{logical equivalence
($\leftrightarrow$)}\index{biconditional ($\leftrightarrow$)} symbol.

Definition \texttt{df-cleq}\index{equality ($=$)} extends the usage of
the equality symbol to include ``classes''\index{class} in set theory.  The
reason it is potentially problematic is that it can lead to statements which
do not follow from logic alone but presuppose the Axiom of
Extensionality\index{Axiom of Extensionality}, so we include this axiom
as a hypothesis for the definition.  We could have made \texttt{df-cleq} directly
eliminable by introducing a new equality symbol, but have chosen not to do so
in keeping with standard textbook practice.  Definitions such as \texttt{df-cleq}
that extend the meaning of existing symbols must be introduced carefully so
that they do not lead to contradictions.  Definition \texttt{df-clel} also
extends the meaning of an existing symbol ($\in$); while it doesn't strengthen
the language like \texttt{df-cleq}, this is not obvious and it must also be
subject to the same scrutiny.

Exercise:  Study how the wff $x\in\omega$, meaning ``$x$ is a natural
number,'' could be expanded in terms of primitive symbols, starting with the
definitions \texttt{df-clel} on p.~\pageref{dfclel} and \texttt{df-om} on
p.~\pageref{dfom} and working your way back.  Don't bother to work out the
details; just make sure that you understand how you could do it in principle.
The answer is shown in the footnote on p.~\pageref{expandom}.  If you
actually do work it out, you won't get exactly the same answer because we used
a few simplifications such as discarding occurrences of $\lnot\lnot$ (double
negation).

In the definitions below, we have placed the {\sc ascii} Metamath source
below each of the formulas to help you become familiar with the
notation in the database.  For simplicity, the necessary \texttt{\$f}
and \texttt{\$d} statements are not shown.  If you are in doubt, use the
\texttt{show statement}\index{\texttt{show statement} command} command
in the Metamath program to see the full statement.
A selection of this notation is summarized in Appendix~\ref{ASCII}.

To understand the motivation for these definitions, you should consult the
references indicated:  Takeuti and Zaring \cite{Takeuti}\index{Takeuti, Gaisi},
Quine \cite{Quine}\index{Quine, Willard Van Orman}, Bell and Machover
\cite{Bell}\index{Bell, J. L.}, and Enderton \cite{Enderton}\index{Enderton,
Herbert B.}.  Our list of definitions is provided more for reference than as a
learning aid.  However, by looking at a few of them you can gain a feel for
how the hierarchy is built up.  The definitions are a representative sample of
the many definitions
in \texttt{set.mm}, but they are complete with respect to the
theorem examples we will present in Section~\ref{sometheorems}.  Also, some are
slightly different from, but logically equivalent to, the ones in \texttt{set.mm}
(some of which have been revised over time to shorten them, for example).

\subsection{Definitions for Propositional Calculus}\label{metadefprop}

The symbols $\varphi$, $\psi$, and $\chi$ represent wffs.

Our first definition introduces the biconditional
connective\footnote{The term ``connective'' is informally used to mean a
symbol that is placed between two variables or adjacent to a variable,
whereas a mathematical ``constant'' usually indicates a symbol such as
the number 0 that may replace a variable or metavariable.  From
Metamath's point of view, there is no distinction between a connective
and a constant; both are constants in the Metamath
language.}\index{connective}\index{constant} (also called logical
equivalence)\index{logical equivalence
($\leftrightarrow$)}\index{biconditional ($\leftrightarrow$)}.  Unlike
most traditional developments, we have chosen not to have a separate
symbol such as ``Df.'' to mean ``is defined as.''  Instead, we will use
the biconditional connective for this purpose, as it lets us use
logic to manipulate definitions directly.  Here we state the properties
of the biconditional connective with a carefully crafted \texttt{\$a}
statement, which effectively uses the biconditional connective to define
itself.  The $\leftrightarrow$ symbol can be eliminated from a formula
using theorem \texttt{bii}, which is derived later.

\vskip 2ex
\noindent Define the biconditional connective.\label{df-bi}

\vskip 0.5ex
\setbox\startprefix=\hbox{\tt \ \ df-bi\ \$a\ }
\setbox\contprefix=\hbox{\tt \ \ \ \ \ \ \ \ \ \ \ }
\startm
\m{\vdash}\m{\lnot}\m{(}\m{(}\m{(}\m{\varphi}\m{\leftrightarrow}\m{\psi}\m{)}%
\m{\rightarrow}\m{\lnot}\m{(}\m{(}\m{\varphi}\m{\rightarrow}\m{\psi}\m{)}\m{%
\rightarrow}\m{\lnot}\m{(}\m{\psi}\m{\rightarrow}\m{\varphi}\m{)}\m{)}\m{)}\m{%
\rightarrow}\m{\lnot}\m{(}\m{\lnot}\m{(}\m{(}\m{\varphi}\m{\rightarrow}\m{%
\psi}\m{)}\m{\rightarrow}\m{\lnot}\m{(}\m{\psi}\m{\rightarrow}\m{\varphi}\m{)}%
\m{)}\m{\rightarrow}\m{(}\m{\varphi}\m{\leftrightarrow}\m{\psi}\m{)}\m{)}\m{)}
\endm
\begin{mmraw}%
|- -. ( ( ( ph <-> ps ) -> -. ( ( ph -> ps ) ->
-. ( ps -> ph ) ) ) -> -. ( -. ( ( ph -> ps ) -> -. (
ps -> ph ) ) -> ( ph <-> ps ) ) ) \$.
\end{mmraw}

\noindent This theorem relates the biconditional connective to primitive
connectives and can be used to eliminate the $\leftrightarrow$ symbol from any
wff.

\vskip 0.5ex
\setbox\startprefix=\hbox{\tt \ \ bii\ \$p\ }
\setbox\contprefix=\hbox{\tt \ \ \ \ \ \ \ \ \ }
\startm
\m{\vdash}\m{(}\m{(}\m{\varphi}\m{\leftrightarrow}\m{\psi}\m{)}\m{\leftrightarrow}
\m{\lnot}\m{(}\m{(}\m{\varphi}\m{\rightarrow}\m{\psi}\m{)}\m{\rightarrow}\m{\lnot}
\m{(}\m{\psi}\m{\rightarrow}\m{\varphi}\m{)}\m{)}\m{)}
\endm
\begin{mmraw}%
|- ( ( ph <-> ps ) <-> -. ( ( ph -> ps ) -> -. ( ps -> ph ) ) ) \$= ... \$.
\end{mmraw}

\noindent Define disjunction ({\sc or}).\index{disjunction ($\vee$)}%
\index{logical or (vee)@logical {\sc or} ($\vee$)}%
\index{df-or@\texttt{df-or}}\label{df-or}

\vskip 0.5ex
\setbox\startprefix=\hbox{\tt \ \ df-or\ \$a\ }
\setbox\contprefix=\hbox{\tt \ \ \ \ \ \ \ \ \ \ \ }
\startm
\m{\vdash}\m{(}\m{(}\m{\varphi}\m{\vee}\m{\psi}\m{)}\m{\leftrightarrow}\m{(}\m{
\lnot}\m{\varphi}\m{\rightarrow}\m{\psi}\m{)}\m{)}
\endm
\begin{mmraw}%
|- ( ( ph \TOR ps ) <-> ( -. ph -> ps ) ) \$.
\end{mmraw}

\noindent Define conjunction ({\sc and}).\index{conjunction ($\wedge$)}%
\index{logical {\sc and} ($\wedge$)}%
\index{df-an@\texttt{df-an}}\label{df-an}

\vskip 0.5ex
\setbox\startprefix=\hbox{\tt \ \ df-an\ \$a\ }
\setbox\contprefix=\hbox{\tt \ \ \ \ \ \ \ \ \ \ \ }
\startm
\m{\vdash}\m{(}\m{(}\m{\varphi}\m{\wedge}\m{\psi}\m{)}\m{\leftrightarrow}\m{\lnot}
\m{(}\m{\varphi}\m{\rightarrow}\m{\lnot}\m{\psi}\m{)}\m{)}
\endm
\begin{mmraw}%
|- ( ( ph \TAND ps ) <-> -. ( ph -> -. ps ) ) \$.
\end{mmraw}

\noindent Define disjunction ({\sc or}) of 3 wffs.%
\index{df-3or@\texttt{df-3or}}\label{df-3or}

\vskip 0.5ex
\setbox\startprefix=\hbox{\tt \ \ df-3or\ \$a\ }
\setbox\contprefix=\hbox{\tt \ \ \ \ \ \ \ \ \ \ \ \ }
\startm
\m{\vdash}\m{(}\m{(}\m{\varphi}\m{\vee}\m{\psi}\m{\vee}\m{\chi}\m{)}\m{
\leftrightarrow}\m{(}\m{(}\m{\varphi}\m{\vee}\m{\psi}\m{)}\m{\vee}\m{\chi}\m{)}
\m{)}
\endm
\begin{mmraw}%
|- ( ( ph \TOR ps \TOR ch ) <-> ( ( ph \TOR ps ) \TOR ch ) ) \$.
\end{mmraw}

\noindent Define conjunction ({\sc and}) of 3 wffs.%
\index{df-3an}\label{df-3an}

\vskip 0.5ex
\setbox\startprefix=\hbox{\tt \ \ df-3an\ \$a\ }
\setbox\contprefix=\hbox{\tt \ \ \ \ \ \ \ \ \ \ \ \ }
\startm
\m{\vdash}\m{(}\m{(}\m{\varphi}\m{\wedge}\m{\psi}\m{\wedge}\m{\chi}\m{)}\m{
\leftrightarrow}\m{(}\m{(}\m{\varphi}\m{\wedge}\m{\psi}\m{)}\m{\wedge}\m{\chi}
\m{)}\m{)}
\endm

\begin{mmraw}%
|- ( ( ph \TAND ps \TAND ch ) <-> ( ( ph \TAND ps ) \TAND ch ) ) \$.
\end{mmraw}

\subsection{Definitions for Predicate Calculus}\label{metadefpred}

The symbols $x$, $y$, and $z$ represent individual variables of predicate
calculus.  In this section, they are not necessarily distinct unless it is
explicitly
mentioned.

\vskip 2ex
\noindent Define existential quantification.
The expression $\exists x \varphi$ means
``there exists an $x$ where $\varphi$ is true.''\index{existential quantifier
($\exists$)}\label{df-ex}

\vskip 0.5ex
\setbox\startprefix=\hbox{\tt \ \ df-ex\ \$a\ }
\setbox\contprefix=\hbox{\tt \ \ \ \ \ \ \ \ \ \ \ }
\startm
\m{\vdash}\m{(}\m{\exists}\m{x}\m{\varphi}\m{\leftrightarrow}\m{\lnot}\m{\forall}
\m{x}\m{\lnot}\m{\varphi}\m{)}
\endm
\begin{mmraw}%
|- ( E. x ph <-> -. A. x -. ph ) \$.
\end{mmraw}

\noindent Define proper substitution.\index{proper
substitution}\index{substitution!proper}\label{df-sb}
In our notation, we use $[ y / x ] \varphi$ to mean ``the wff that
results when $y$ is properly substituted for $x$ in the wff
$\varphi$.''\footnote{
This can also be described
as substituting $x$ with $y$, $y$ properly replaces $x$, or
$x$ is properly replaced by $y$.}
% This is elsb4, though it currently says: ( [ x / y ] z e. y <-> z e. x )
For example,
$[ y / x ] z \in x$ is the same as $z \in y$.
One way to remember this notation is to notice that it looks like division
and recall that $( y / x ) \cdot x $ is $y$ (when $x \neq 0$).
The notation is different from the notation $\varphi ( x | y )$
that is sometimes used, because the latter notation is ambiguous for us:
for example, we don't know whether $\lnot \varphi ( x | y )$ is to be
interpreted as $\lnot ( \varphi ( x | y ) )$ or
$( \lnot \varphi ) ( x | y )$.\footnote{Because of the way
we initially defined wffs, this is the case
with any postfix connective\index{postfix connective} (one occurring after the
symbols being connected) or infix connective\index{infix connective} (one
occurring between the symbols being connected).  Metamath does not have a
built-in notion of operator binding strength that could eliminate the
ambiguity.  The initial parenthesis effectively provides a prefix
connective\index{prefix connective} to eliminate ambiguity.  Some conventions,
such as Polish notation\index{Polish notation} used in the 1930's and 1940's
by Polish logicians, use only prefix connectives and thus allow the total
elimination of parentheses, at the expense of readability.  In Metamath we
could actually redefine all notation to be Polish if we wanted to without
having to change any proofs!}  Other texts often use $\varphi(y)$ to indicate
our $[ y / x ] \varphi$, but this notation is even more ambiguous since there is
no explicit indication of what is being substituted.
Note that this
definition is valid even when
$x$ and $y$ are the same variable.  The first conjunct is a ``trick'' used to
achieve this property, making the definition look somewhat peculiar at
first.

\vskip 0.5ex
\setbox\startprefix=\hbox{\tt \ \ df-sb\ \$a\ }
\setbox\contprefix=\hbox{\tt \ \ \ \ \ \ \ \ \ \ \ }
\startm
\m{\vdash}\m{(}\m{[}\m{y}\m{/}\m{x}\m{]}\m{\varphi}\m{\leftrightarrow}\m{(}%
\m{(}\m{x}\m{=}\m{y}\m{\rightarrow}\m{\varphi}\m{)}\m{\wedge}\m{\exists}\m{x}%
\m{(}\m{x}\m{=}\m{y}\m{\wedge}\m{\varphi}\m{)}\m{)}\m{)}
\endm
\begin{mmraw}%
|- ( [ y / x ] ph <-> ( ( x = y -> ph ) \TAND E. x ( x = y \TAND ph ) ) ) \$.
\end{mmraw}


\noindent Define existential uniqueness\index{existential uniqueness
quantifier ($\exists "!$)} (``there exists exactly one'').  Note that $y$ is a
variable distinct from $x$ and not occurring in $\varphi$.\label{df-eu}

\vskip 0.5ex
\setbox\startprefix=\hbox{\tt \ \ df-eu\ \$a\ }
\setbox\contprefix=\hbox{\tt \ \ \ \ \ \ \ \ \ \ \ }
\startm
\m{\vdash}\m{(}\m{\exists}\m{{!}}\m{x}\m{\varphi}\m{\leftrightarrow}\m{\exists}
\m{y}\m{\forall}\m{x}\m{(}\m{\varphi}\m{\leftrightarrow}\m{x}\m{=}\m{y}\m{)}\m{)}
\endm

\begin{mmraw}%
|- ( E! x ph <-> E. y A. x ( ph <-> x = y ) ) \$.
\end{mmraw}

\subsection{Definitions for Set Theory}\label{setdefinitions}

The symbols $x$, $y$, $z$, and $w$ represent individual variables of
predicate calculus, which in set theory are understood to be sets.
However, using only the constructs shown so far would be very inconvenient.

To make set theory more practical, we introduce the notion of a ``class.''
A class\index{class} is either a set variable (such as $x$) or an
expression of the form $\{ x | \varphi\}$ (called an ``abstraction
class''\index{abstraction class}\index{class abstraction}).  Note that
sets (i.e.\ individual variables) always exist (this is a theorem of
logic, namely $\exists y \, y = x$ for any set $x$), whereas classes may
or may not exist (i.e.\ $\exists y \, y = A$ may or may not be true).
If a class does not exist it is called a ``proper class.''\index{proper
class}\index{class!proper} Definitions \texttt{df-clab},
\texttt{df-cleq}, and \texttt{df-clel} can be used to convert an
expression containing classes into one containing only set variables and
wff metavariables.

The symbols $A$, $B$, $C$, $D$, $ F$, $G$, and $R$ are metavariables that range
over classes.  A class metavariable $A$ may be eliminated from a wff by
replacing it with $\{ x|\varphi\}$ where neither $x$ nor $\varphi$ occur in
the wff.

The theory of classes can be shown to be an eliminable and conservative
extension of set theory. The \textbf{eliminability}
property shows that for every
formula in the extended language we can build a logically equivalent
formula in the basic language; so that even if the extended language
provides more ease to convey and formulate mathematical ideas for set
theory, its expressive power does not in fact strengthen the basic
language's expressive power.
The \textbf{conservation} property shows that for
every proof of a formula of the basic language in the extended system
we can build another proof of the same formula in the basic system;
so that, concerning theorems on sets only, the deductive powers of
the extended system and of the basic system are identical. Together,
these properties mean that the extended language can be treated as a
definitional extension that is \textbf{sound}.

A rigorous justification, which we will not give here, can be found in
Levy \cite[pp.~357-366]{Levy} supplementing his informal introduction to class
theory on pp.~7-17. Two other good treatments of class theory are provided
by Quine \cite[pp.~15-21]{Quine}\index{Quine, Willard Van Orman}
and also \cite[pp.~10-14]{Takeuti}\index{Takeuti, Gaisi}.
Quine's exposition (he calls them virtual classes)
is nicely written and very readable.

In the rest of this
section, individual variables are always assumed to be distinct from
each other unless otherwise indicated.  In addition, dummy variables on the
right-hand side of a definition do not occur in the class and wff
metavariables in the definition.

The definitions we present here are a partial but self-contained
collection selected from several hundred that appear in the current
\texttt{set.mm} database.  They are adequate for a basic development of
elementary set theory.

\vskip 2ex
\noindent Define the abstraction class.\index{abstraction class}\index{class
abstraction}\label{df-clab}  $x$ and $y$
need not be distinct.  Definition 2.1 of Quine, p.~16.  This definition may
seem puzzling since it is shorter than the expression being defined and does not
buy us anything in terms of brevity.  The reason we introduce this definition
is because it fits in neatly with the extension of the $\in$ connective
provided by \texttt{df-clel}.

\vskip 0.5ex
\setbox\startprefix=\hbox{\tt \ \ df-clab\ \$a\ }
\setbox\contprefix=\hbox{\tt \ \ \ \ \ \ \ \ \ \ \ \ \ }
\startm
\m{\vdash}\m{(}\m{x}\m{\in}\m{\{}\m{y}\m{|}\m{\varphi}\m{\}}\m{%
\leftrightarrow}\m{[}\m{x}\m{/}\m{y}\m{]}\m{\varphi}\m{)}
\endm
\begin{mmraw}%
|- ( x e. \{ y | ph \} <-> [ x / y ] ph ) \$.
\end{mmraw}

\noindent Define the equality connective between classes\index{class
equality}\label{df-cleq}.  See Quine or Chapter 4 of Takeuti and Zaring for its
justification and methods for eliminating it.  This is an example of a
somewhat ``dangerous'' definition, because it extends the use of the
existing equality symbol rather than introducing a new symbol, allowing
us to make statements in the original language that may not be true.
For example, it permits us to deduce $y = z \leftrightarrow \forall x (
x \in y \leftrightarrow x \in z )$ which is not a theorem of logic but
rather presupposes the Axiom of Extensionality,\index{Axiom of
Extensionality} which we include as a hypothesis so that we can know
when this axiom is assumed in a proof (with the \texttt{show
trace{\char`\_}back} command).  We could avoid the danger by introducing
another symbol, say $\eqcirc$, in place of $=$; this
would also have the advantage of making elimination of the definition
straightforward and would eliminate the need for Extensionality as a
hypothesis.  We would then also have the advantage of being able to
identify exactly where Extensionality truly comes into play.  One of our
theorems would be $x \eqcirc y \leftrightarrow x = y$
by invoking Extensionality.  However in keeping with standard practice
we retain the ``dangerous'' definition.

\vskip 0.5ex
\setbox\startprefix=\hbox{\tt \ \ df-cleq.1\ \$e\ }
\setbox\contprefix=\hbox{\tt \ \ \ \ \ \ \ \ \ \ \ \ \ \ \ }
\startm
\m{\vdash}\m{(}\m{\forall}\m{x}\m{(}\m{x}\m{\in}\m{y}\m{\leftrightarrow}\m{x}
\m{\in}\m{z}\m{)}\m{\rightarrow}\m{y}\m{=}\m{z}\m{)}
\endm
\setbox\startprefix=\hbox{\tt \ \ df-cleq\ \$a\ }
\setbox\contprefix=\hbox{\tt \ \ \ \ \ \ \ \ \ \ \ \ \ }
\startm
\m{\vdash}\m{(}\m{A}\m{=}\m{B}\m{\leftrightarrow}\m{\forall}\m{x}\m{(}\m{x}\m{
\in}\m{A}\m{\leftrightarrow}\m{x}\m{\in}\m{B}\m{)}\m{)}
\endm
% We need to reset the startprefix and contprefix.
\setbox\startprefix=\hbox{\tt \ \ df-cleq.1\ \$e\ }
\setbox\contprefix=\hbox{\tt \ \ \ \ \ \ \ \ \ \ \ \ \ \ \ }
\begin{mmraw}%
|- ( A. x ( x e. y <-> x e. z ) -> y = z ) \$.
\end{mmraw}
\setbox\startprefix=\hbox{\tt \ \ df-cleq\ \$a\ }
\setbox\contprefix=\hbox{\tt \ \ \ \ \ \ \ \ \ \ \ \ \ }
\begin{mmraw}%
|- ( A = B <-> A. x ( x e. A <-> x e. B ) ) \$.
\end{mmraw}

\noindent Define the membership connective between classes\index{class
membership}.  Theorem 6.3 of Quine, p.~41, which we adopt as a definition.
Note that it extends the use of the existing membership symbol, but unlike
{\tt df-cleq} it does not extend the set of valid wffs of logic when the class
metavariables are replaced with set variables.\label{dfclel}\label{df-clel}

\vskip 0.5ex
\setbox\startprefix=\hbox{\tt \ \ df-clel\ \$a\ }
\setbox\contprefix=\hbox{\tt \ \ \ \ \ \ \ \ \ \ \ \ \ }
\startm
\m{\vdash}\m{(}\m{A}\m{\in}\m{B}\m{\leftrightarrow}\m{\exists}\m{x}\m{(}\m{x}
\m{=}\m{A}\m{\wedge}\m{x}\m{\in}\m{B}\m{)}\m{)}
\endm
\begin{mmraw}%
|- ( A e. B <-> E. x ( x = A \TAND x e. B ) ) \$.?
\end{mmraw}

\noindent Define inequality.

\vskip 0.5ex
\setbox\startprefix=\hbox{\tt \ \ df-ne\ \$a\ }
\setbox\contprefix=\hbox{\tt \ \ \ \ \ \ \ \ \ \ \ }
\startm
\m{\vdash}\m{(}\m{A}\m{\ne}\m{B}\m{\leftrightarrow}\m{\lnot}\m{A}\m{=}\m{B}%
\m{)}
\endm
\begin{mmraw}%
|- ( A =/= B <-> -. A = B ) \$.
\end{mmraw}

\noindent Define restricted universal quantification.\index{universal
quantifier ($\forall$)!restricted}  Enderton, p.~22.

\vskip 0.5ex
\setbox\startprefix=\hbox{\tt \ \ df-ral\ \$a\ }
\setbox\contprefix=\hbox{\tt \ \ \ \ \ \ \ \ \ \ \ \ }
\startm
\m{\vdash}\m{(}\m{\forall}\m{x}\m{\in}\m{A}\m{\varphi}\m{\leftrightarrow}\m{%
\forall}\m{x}\m{(}\m{x}\m{\in}\m{A}\m{\rightarrow}\m{\varphi}\m{)}\m{)}
\endm
\begin{mmraw}%
|- ( A. x e. A ph <-> A. x ( x e. A -> ph ) ) \$.
\end{mmraw}

\noindent Define restricted existential quantification.\index{existential
quantifier ($\exists$)!restricted}  Enderton, p.~22.

\vskip 0.5ex
\setbox\startprefix=\hbox{\tt \ \ df-rex\ \$a\ }
\setbox\contprefix=\hbox{\tt \ \ \ \ \ \ \ \ \ \ \ \ }
\startm
\m{\vdash}\m{(}\m{\exists}\m{x}\m{\in}\m{A}\m{\varphi}\m{\leftrightarrow}\m{%
\exists}\m{x}\m{(}\m{x}\m{\in}\m{A}\m{\wedge}\m{\varphi}\m{)}\m{)}
\endm
\begin{mmraw}%
|- ( E. x e. A ph <-> E. x ( x e. A \TAND ph ) ) \$.
\end{mmraw}

\noindent Define the universal class\index{universal class ($V$)}.  Definition
5.20, p.~21, of Takeuti and Zaring.\label{df-v}

\vskip 0.5ex
\setbox\startprefix=\hbox{\tt \ \ df-v\ \$a\ }
\setbox\contprefix=\hbox{\tt \ \ \ \ \ \ \ \ \ \ }
\startm
\m{\vdash}\m{{\rm V}}\m{=}\m{\{}\m{x}\m{|}\m{x}\m{=}\m{x}\m{\}}
\endm
\begin{mmraw}%
|- {\char`\_}V = \{ x | x = x \} \$.
\end{mmraw}

\noindent Define the subclass\index{subclass}\index{subset} relationship
between two classes (called the subset relation if the classes are sets i.e.\
are not proper).  Definition 5.9 of Takeuti and Zaring, p.~17.\label{df-ss}

\vskip 0.5ex
\setbox\startprefix=\hbox{\tt \ \ df-ss\ \$a\ }
\setbox\contprefix=\hbox{\tt \ \ \ \ \ \ \ \ \ \ \ }
\startm
\m{\vdash}\m{(}\m{A}\m{\subseteq}\m{B}\m{\leftrightarrow}\m{\forall}\m{x}\m{(}
\m{x}\m{\in}\m{A}\m{\rightarrow}\m{x}\m{\in}\m{B}\m{)}\m{)}
\endm
\begin{mmraw}%
|- ( A C\_ B <-> A. x ( x e. A -> x e. B ) ) \$.
\end{mmraw}

\noindent Define the union\index{union} of two classes.  Definition 5.6 of Takeuti and Zaring,
p.~16.\label{df-un}

\vskip 0.5ex
\setbox\startprefix=\hbox{\tt \ \ df-un\ \$a\ }
\setbox\contprefix=\hbox{\tt \ \ \ \ \ \ \ \ \ \ \ }
\startm
\m{\vdash}\m{(}\m{A}\m{\cup}\m{B}\m{)}\m{=}\m{\{}\m{x}\m{|}\m{(}\m{x}\m{\in}
\m{A}\m{\vee}\m{x}\m{\in}\m{B}\m{)}\m{\}}
\endm
\begin{mmraw}%
( A u. B ) = \{ x | ( x e. A \TOR x e. B ) \} \$.
\end{mmraw}

\noindent Define the intersection\index{intersection} of two classes.  Definition 5.6 of
Takeuti and Zaring, p.~16.\label{df-in}

\vskip 0.5ex
\setbox\startprefix=\hbox{\tt \ \ df-in\ \$a\ }
\setbox\contprefix=\hbox{\tt \ \ \ \ \ \ \ \ \ \ \ }
\startm
\m{\vdash}\m{(}\m{A}\m{\cap}\m{B}\m{)}\m{=}\m{\{}\m{x}\m{|}\m{(}\m{x}\m{\in}
\m{A}\m{\wedge}\m{x}\m{\in}\m{B}\m{)}\m{\}}
\endm
% Caret ^ requires special treatment
\begin{mmraw}%
|- ( A i\^{}i B ) = \{ x | ( x e. A \TAND x e. B ) \} \$.
\end{mmraw}

\noindent Define class difference\index{class difference}\index{set difference}.
Definition 5.12 of Takeuti and Zaring, p.~20.  Several notations are used in
the literature; we chose the $\setminus$ convention instead of a minus sign to
reserve the latter for later use in, e.g., arithmetic.\label{df-dif}

\vskip 0.5ex
\setbox\startprefix=\hbox{\tt \ \ df-dif\ \$a\ }
\setbox\contprefix=\hbox{\tt \ \ \ \ \ \ \ \ \ \ \ \ }
\startm
\m{\vdash}\m{(}\m{A}\m{\setminus}\m{B}\m{)}\m{=}\m{\{}\m{x}\m{|}\m{(}\m{x}\m{
\in}\m{A}\m{\wedge}\m{\lnot}\m{x}\m{\in}\m{B}\m{)}\m{\}}
\endm
\begin{mmraw}%
( A \SLASH B ) = \{ x | ( x e. A \TAND -. x e. B ) \} \$.
\end{mmraw}

\noindent Define the empty or null set\index{empty set}\index{null set}.
Compare  Definition 5.14 of Takeuti and Zaring, p.~20.\label{df-nul}

\vskip 0.5ex
\setbox\startprefix=\hbox{\tt \ \ df-nul\ \$a\ }
\setbox\contprefix=\hbox{\tt \ \ \ \ \ \ \ \ \ \ }
\startm
\m{\vdash}\m{\varnothing}\m{=}\m{(}\m{{\rm V}}\m{\setminus}\m{{\rm V}}\m{)}
\endm
\begin{mmraw}%
|- (/) = ( {\char`\_}V \SLASH {\char`\_}V ) \$.
\end{mmraw}

\noindent Define power class\index{power set}\index{power class}.  Definition 5.10 of
Takeuti and Zaring, p.~17, but we also let it apply to proper classes.  (Note
that \verb$~P$ is the symbol for calligraphic P, the tilde
suggesting ``curly;'' see Appendix~\ref{ASCII}.)\label{df-pw}

\vskip 0.5ex
\setbox\startprefix=\hbox{\tt \ \ df-pw\ \$a\ }
\setbox\contprefix=\hbox{\tt \ \ \ \ \ \ \ \ \ \ \ }
\startm
\m{\vdash}\m{{\cal P}}\m{A}\m{=}\m{\{}\m{x}\m{|}\m{x}\m{\subseteq}\m{A}\m{\}}
\endm
% Special incantation required to put ~ into the text
\begin{mmraw}%
|- \char`\~P~A = \{ x | x C\_ A \} \$.
\end{mmraw}

\noindent Define the singleton of a class\index{singleton}.  Definition 7.1 of
Quine, p.~48.  It is well-defined for proper classes, although
it is not very meaningful in this case, where it evaluates to the empty
set.

\vskip 0.5ex
\setbox\startprefix=\hbox{\tt \ \ df-sn\ \$a\ }
\setbox\contprefix=\hbox{\tt \ \ \ \ \ \ \ \ \ \ \ }
\startm
\m{\vdash}\m{\{}\m{A}\m{\}}\m{=}\m{\{}\m{x}\m{|}\m{x}\m{=}\m{A}\m{\}}
\endm
\begin{mmraw}%
|- \{ A \} = \{ x | x = A \} \$.
\end{mmraw}%

\noindent Define an unordered pair of classes\index{unordered pair}\index{pair}.  Definition
7.1 of Quine, p.~48.

\vskip 0.5ex
\setbox\startprefix=\hbox{\tt \ \ df-pr\ \$a\ }
\setbox\contprefix=\hbox{\tt \ \ \ \ \ \ \ \ \ \ \ }
\startm
\m{\vdash}\m{\{}\m{A}\m{,}\m{B}\m{\}}\m{=}\m{(}\m{\{}\m{A}\m{\}}\m{\cup}\m{\{}
\m{B}\m{\}}\m{)}
\endm
\begin{mmraw}%
|- \{ A , B \} = ( \{ A \} u. \{ B \} ) \$.
\end{mmraw}

\noindent Define an unordered triple of classes\index{unordered triple}.  Definition of
Enderton, p.~19.

\vskip 0.5ex
\setbox\startprefix=\hbox{\tt \ \ df-tp\ \$a\ }
\setbox\contprefix=\hbox{\tt \ \ \ \ \ \ \ \ \ \ \ }
\startm
\m{\vdash}\m{\{}\m{A}\m{,}\m{B}\m{,}\m{C}\m{\}}\m{=}\m{(}\m{\{}\m{A}\m{,}\m{B}
\m{\}}\m{\cup}\m{\{}\m{C}\m{\}}\m{)}
\endm
\begin{mmraw}%
|- \{ A , B , C \} = ( \{ A , B \} u. \{ C \} ) \$.
\end{mmraw}%

\noindent Kuratowski's\index{Kuratowski, Kazimierz} ordered pair\index{ordered
pair} definition.  Definition 9.1 of Quine, p.~58. For proper classes it is
not meaningful but is well-defined for convenience.  (Note that \verb$<.$
stands for $\langle$ whereas \verb$<$ stands for $<$, and similarly for
\verb$>.$\,.)\label{df-op}

\vskip 0.5ex
\setbox\startprefix=\hbox{\tt \ \ df-op\ \$a\ }
\setbox\contprefix=\hbox{\tt \ \ \ \ \ \ \ \ \ \ \ }
\startm
\m{\vdash}\m{\langle}\m{A}\m{,}\m{B}\m{\rangle}\m{=}\m{\{}\m{\{}\m{A}\m{\}}
\m{,}\m{\{}\m{A}\m{,}\m{B}\m{\}}\m{\}}
\endm
\begin{mmraw}%
|- <. A , B >. = \{ \{ A \} , \{ A , B \} \} \$.
\end{mmraw}

\noindent Define the union of a class\index{union}.  Definition 5.5, p.~16,
of Takeuti and Zaring.

\vskip 0.5ex
\setbox\startprefix=\hbox{\tt \ \ df-uni\ \$a\ }
\setbox\contprefix=\hbox{\tt \ \ \ \ \ \ \ \ \ \ \ \ }
\startm
\m{\vdash}\m{\bigcup}\m{A}\m{=}\m{\{}\m{x}\m{|}\m{\exists}\m{y}\m{(}\m{x}\m{
\in}\m{y}\m{\wedge}\m{y}\m{\in}\m{A}\m{)}\m{\}}
\endm
\begin{mmraw}%
|- U. A = \{ x | E. y ( x e. y \TAND y e. A ) \} \$.
\end{mmraw}

\noindent Define the intersection\index{intersection} of a class.  Definition 7.35,
p.~44, of Takeuti and Zaring.

\vskip 0.5ex
\setbox\startprefix=\hbox{\tt \ \ df-int\ \$a\ }
\setbox\contprefix=\hbox{\tt \ \ \ \ \ \ \ \ \ \ \ \ }
\startm
\m{\vdash}\m{\bigcap}\m{A}\m{=}\m{\{}\m{x}\m{|}\m{\forall}\m{y}\m{(}\m{y}\m{
\in}\m{A}\m{\rightarrow}\m{x}\m{\in}\m{y}\m{)}\m{\}}
\endm
\begin{mmraw}%
|- |\^{}| A = \{ x | A. y ( y e. A -> x e. y ) \} \$.
\end{mmraw}

\noindent Define a transitive class\index{transitive class}\index{transitive
set}.  This should not be confused with a transitive relation, which is a different
concept.  Definition from p.~71 of Enderton, extended to classes.

\vskip 0.5ex
\setbox\startprefix=\hbox{\tt \ \ df-tr\ \$a\ }
\setbox\contprefix=\hbox{\tt \ \ \ \ \ \ \ \ \ \ \ }
\startm
\m{\vdash}\m{(}\m{\mbox{\rm Tr}}\m{A}\m{\leftrightarrow}\m{\bigcup}\m{A}\m{
\subseteq}\m{A}\m{)}
\endm
\begin{mmraw}%
|- ( Tr A <-> U. A C\_ A ) \$.
\end{mmraw}

\noindent Define a notation for a general binary relation\index{binary
relation}.  Definition 6.18, p.~29, of Takeuti and Zaring, generalized to
arbitrary classes.  This definition is well-defined, although not very
meaningful, when classes $A$ and/or $B$ are proper.\label{dfbr}  The lack of
parentheses (or any other connective) creates no ambiguity since we are defining
an atomic wff.

\vskip 0.5ex
\setbox\startprefix=\hbox{\tt \ \ df-br\ \$a\ }
\setbox\contprefix=\hbox{\tt \ \ \ \ \ \ \ \ \ \ \ }
\startm
\m{\vdash}\m{(}\m{A}\m{\,R}\m{\,B}\m{\leftrightarrow}\m{\langle}\m{A}\m{,}\m{B}
\m{\rangle}\m{\in}\m{R}\m{)}
\endm
\begin{mmraw}%
|- ( A R B <-> <. A , B >. e. R ) \$.
\end{mmraw}

\noindent Define an abstraction class of ordered pairs\index{abstraction
class!of ordered
pairs}.  A special case of Definition 4.16, p.~14, of Takeuti and Zaring.
Note that $ z $ must be distinct from $ x $ and $ y $,
and $ z $ must not occur in $\varphi$, but $ x $ and $ y $ may be
identical and may appear in $\varphi$.

\vskip 0.5ex
\setbox\startprefix=\hbox{\tt \ \ df-opab\ \$a\ }
\setbox\contprefix=\hbox{\tt \ \ \ \ \ \ \ \ \ \ \ \ \ }
\startm
\m{\vdash}\m{\{}\m{\langle}\m{x}\m{,}\m{y}\m{\rangle}\m{|}\m{\varphi}\m{\}}\m{=}
\m{\{}\m{z}\m{|}\m{\exists}\m{x}\m{\exists}\m{y}\m{(}\m{z}\m{=}\m{\langle}\m{x}
\m{,}\m{y}\m{\rangle}\m{\wedge}\m{\varphi}\m{)}\m{\}}
\endm

\begin{mmraw}%
|- \{ <. x , y >. | ph \} = \{ z | E. x E. y ( z =
<. x , y >. /\ ph ) \} \$.
\end{mmraw}

\noindent Define the epsilon relation\index{epsilon relation}.  Similar to Definition
6.22, p.~30, of Takeuti and Zaring.

\vskip 0.5ex
\setbox\startprefix=\hbox{\tt \ \ df-eprel\ \$a\ }
\setbox\contprefix=\hbox{\tt \ \ \ \ \ \ \ \ \ \ \ \ \ \ }
\startm
\m{\vdash}\m{{\rm E}}\m{=}\m{\{}\m{\langle}\m{x}\m{,}\m{y}\m{\rangle}\m{|}\m{x}\m{
\in}\m{y}\m{\}}
\endm
\begin{mmraw}%
|- \_E = \{ <. x , y >. | x e. y \} \$.
\end{mmraw}

\noindent Define a founded relation\index{founded relation}.  $R$ is a founded
relation on $A$ iff\index{iff} (if and only if) each nonempty subset of $A$
has an ``$R$-minimal element.''  Similar to Definition 6.21, p.~30, of
Takeuti and Zaring.

\vskip 0.5ex
\setbox\startprefix=\hbox{\tt \ \ df-fr\ \$a\ }
\setbox\contprefix=\hbox{\tt \ \ \ \ \ \ \ \ \ \ \ }
\startm
\m{\vdash}\m{(}\m{R}\m{\,\mbox{\rm Fr}}\m{\,A}\m{\leftrightarrow}\m{\forall}\m{x}
\m{(}\m{(}\m{x}\m{\subseteq}\m{A}\m{\wedge}\m{\lnot}\m{x}\m{=}\m{\varnothing}
\m{)}\m{\rightarrow}\m{\exists}\m{y}\m{(}\m{y}\m{\in}\m{x}\m{\wedge}\m{(}\m{x}
\m{\cap}\m{\{}\m{z}\m{|}\m{z}\m{\,R}\m{\,y}\m{\}}\m{)}\m{=}\m{\varnothing}\m{)}
\m{)}\m{)}
\endm
\begin{mmraw}%
|- ( R Fr A <-> A. x ( ( x C\_ A \TAND -. x = (/) ) ->
E. y ( y e. x \TAND ( x i\^{}i \{ z | z R y \} ) = (/) ) ) ) \$.
\end{mmraw}

\noindent Define a well-ordering\index{well-ordering}.  $R$ is a well-ordering of $A$ iff
it is founded on $A$ and the elements of $A$ are pairwise $R$-comparable.
Similar to Definition 6.24(2), p.~30, of Takeuti and Zaring.

\vskip 0.5ex
\setbox\startprefix=\hbox{\tt \ \ df-we\ \$a\ }
\setbox\contprefix=\hbox{\tt \ \ \ \ \ \ \ \ \ \ \ }
\startm
\m{\vdash}\m{(}\m{R}\m{\,\mbox{\rm We}}\m{\,A}\m{\leftrightarrow}\m{(}\m{R}\m{\,
\mbox{\rm Fr}}\m{\,A}\m{\wedge}\m{\forall}\m{x}\m{\forall}\m{y}\m{(}\m{(}\m{x}\m{
\in}\m{A}\m{\wedge}\m{y}\m{\in}\m{A}\m{)}\m{\rightarrow}\m{(}\m{x}\m{\,R}\m{\,y}
\m{\vee}\m{x}\m{=}\m{y}\m{\vee}\m{y}\m{\,R}\m{\,x}\m{)}\m{)}\m{)}\m{)}
\endm
\begin{mmraw}%
( R We A <-> ( R Fr A \TAND A. x A. y ( ( x e.
A \TAND y e. A ) -> ( x R y \TOR x = y \TOR y R x ) ) ) ) \$.
\end{mmraw}

\noindent Define the ordinal predicate\index{ordinal predicate}, which is true for a class
that is transitive and is well-ordered by the epsilon relation.  Similar to
definition on p.~468, Bell and Machover.

\vskip 0.5ex
\setbox\startprefix=\hbox{\tt \ \ df-ord\ \$a\ }
\setbox\contprefix=\hbox{\tt \ \ \ \ \ \ \ \ \ \ \ \ }
\startm
\m{\vdash}\m{(}\m{\mbox{\rm Ord}}\m{\,A}\m{\leftrightarrow}\m{(}
\m{\mbox{\rm Tr}}\m{\,A}\m{\wedge}\m{E}\m{\,\mbox{\rm We}}\m{\,A}\m{)}\m{)}
\endm
\begin{mmraw}%
|- ( Ord A <-> ( Tr A \TAND E We A ) ) \$.
\end{mmraw}

\noindent Define the class of all ordinal numbers\index{ordinal number}.  An ordinal number is
a set that satisfies the ordinal predicate.  Definition 7.11 of Takeuti and
Zaring, p.~38.

\vskip 0.5ex
\setbox\startprefix=\hbox{\tt \ \ df-on\ \$a\ }
\setbox\contprefix=\hbox{\tt \ \ \ \ \ \ \ \ \ \ \ }
\startm
\m{\vdash}\m{\,\mbox{\rm On}}\m{=}\m{\{}\m{x}\m{|}\m{\mbox{\rm Ord}}\m{\,x}
\m{\}}
\endm
\begin{mmraw}%
|- On = \{ x | Ord x \} \$.
\end{mmraw}

\noindent Define the limit ordinal predicate\index{limit ordinal}, which is true for a
non-empty ordinal that is not a successor (i.e.\ that is the union of itself).
Compare Bell and Machover, p.~471 and Exercise (1), p.~42 of Takeuti and
Zaring.

\vskip 0.5ex
\setbox\startprefix=\hbox{\tt \ \ df-lim\ \$a\ }
\setbox\contprefix=\hbox{\tt \ \ \ \ \ \ \ \ \ \ \ \ }
\startm
\m{\vdash}\m{(}\m{\mbox{\rm Lim}}\m{\,A}\m{\leftrightarrow}\m{(}\m{\mbox{
\rm Ord}}\m{\,A}\m{\wedge}\m{\lnot}\m{A}\m{=}\m{\varnothing}\m{\wedge}\m{A}
\m{=}\m{\bigcup}\m{A}\m{)}\m{)}
\endm
\begin{mmraw}%
|- ( Lim A <-> ( Ord A \TAND -. A = (/) \TAND A = U. A ) ) \$.
\end{mmraw}

\noindent Define the successor\index{successor} of a class.  Definition 7.22 of Takeuti
and Zaring, p.~41.  Our definition is a generalization to classes, although it
is meaningless when classes are proper.

\vskip 0.5ex
\setbox\startprefix=\hbox{\tt \ \ df-suc\ \$a\ }
\setbox\contprefix=\hbox{\tt \ \ \ \ \ \ \ \ \ \ \ \ }
\startm
\m{\vdash}\m{\,\mbox{\rm suc}}\m{\,A}\m{=}\m{(}\m{A}\m{\cup}\m{\{}\m{A}\m{\}}
\m{)}
\endm
\begin{mmraw}%
|- suc A = ( A u. \{ A \} ) \$.
\end{mmraw}

\noindent Define the class of natural numbers\index{natural number}\index{omega
($\omega$)}.  Compare Bell and Machover, p.~471.\label{dfom}

\vskip 0.5ex
\setbox\startprefix=\hbox{\tt \ \ df-om\ \$a\ }
\setbox\contprefix=\hbox{\tt \ \ \ \ \ \ \ \ \ \ \ }
\startm
\m{\vdash}\m{\omega}\m{=}\m{\{}\m{x}\m{|}\m{(}\m{\mbox{\rm Ord}}\m{\,x}\m{
\wedge}\m{\forall}\m{y}\m{(}\m{\mbox{\rm Lim}}\m{\,y}\m{\rightarrow}\m{x}\m{
\in}\m{y}\m{)}\m{)}\m{\}}
\endm
\begin{mmraw}%
|- om = \{ x | ( Ord x \TAND A. y ( Lim y -> x e. y ) ) \} \$.
\end{mmraw}

\noindent Define the Cartesian product (also called the
cross product)\index{Cartesian product}\index{cross product}
of two classes.  Definition 9.11 of Quine, p.~64.

\vskip 0.5ex
\setbox\startprefix=\hbox{\tt \ \ df-xp\ \$a\ }
\setbox\contprefix=\hbox{\tt \ \ \ \ \ \ \ \ \ \ \ }
\startm
\m{\vdash}\m{(}\m{A}\m{\times}\m{B}\m{)}\m{=}\m{\{}\m{\langle}\m{x}\m{,}\m{y}
\m{\rangle}\m{|}\m{(}\m{x}\m{\in}\m{A}\m{\wedge}\m{y}\m{\in}\m{B}\m{)}\m{\}}
\endm
\begin{mmraw}%
|- ( A X. B ) = \{ <. x , y >. | ( x e. A \TAND y e. B) \} \$.
\end{mmraw}

\noindent Define a relation\index{relation}.  Definition 6.4(1) of Takeuti and
Zaring, p.~23.

\vskip 0.5ex
\setbox\startprefix=\hbox{\tt \ \ df-rel\ \$a\ }
\setbox\contprefix=\hbox{\tt \ \ \ \ \ \ \ \ \ \ \ \ }
\startm
\m{\vdash}\m{(}\m{\mbox{\rm Rel}}\m{\,A}\m{\leftrightarrow}\m{A}\m{\subseteq}
\m{(}\m{{\rm V}}\m{\times}\m{{\rm V}}\m{)}\m{)}
\endm
\begin{mmraw}%
|- ( Rel A <-> A C\_ ( {\char`\_}V X. {\char`\_}V ) ) \$.
\end{mmraw}

\noindent Define the domain\index{domain} of a class.  Definition 6.5(1) of
Takeuti and Zaring, p.~24.

\vskip 0.5ex
\setbox\startprefix=\hbox{\tt \ \ df-dm\ \$a\ }
\setbox\contprefix=\hbox{\tt \ \ \ \ \ \ \ \ \ \ \ }
\startm
\m{\vdash}\m{\,\mbox{\rm dom}}\m{A}\m{=}\m{\{}\m{x}\m{|}\m{\exists}\m{y}\m{
\langle}\m{x}\m{,}\m{y}\m{\rangle}\m{\in}\m{A}\m{\}}
\endm
\begin{mmraw}%
|- dom A = \{ x | E. y <. x , y >. e. A \} \$.
\end{mmraw}

\noindent Define the range\index{range} of a class.  Definition 6.5(2) of
Takeuti and Zaring, p.~24.

\vskip 0.5ex
\setbox\startprefix=\hbox{\tt \ \ df-rn\ \$a\ }
\setbox\contprefix=\hbox{\tt \ \ \ \ \ \ \ \ \ \ \ }
\startm
\m{\vdash}\m{\,\mbox{\rm ran}}\m{A}\m{=}\m{\{}\m{y}\m{|}\m{\exists}\m{x}\m{
\langle}\m{x}\m{,}\m{y}\m{\rangle}\m{\in}\m{A}\m{\}}
\endm
\begin{mmraw}%
|- ran A = \{ y | E. x <. x , y >. e. A \} \$.
\end{mmraw}

\noindent Define the restriction\index{restriction} of a class.  Definition
6.6(1) of Takeuti and Zaring, p.~24.

\vskip 0.5ex
\setbox\startprefix=\hbox{\tt \ \ df-res\ \$a\ }
\setbox\contprefix=\hbox{\tt \ \ \ \ \ \ \ \ \ \ \ \ }
\startm
\m{\vdash}\m{(}\m{A}\m{\restriction}\m{B}\m{)}\m{=}\m{(}\m{A}\m{\cap}\m{(}\m{B}
\m{\times}\m{{\rm V}}\m{)}\m{)}
\endm
\begin{mmraw}%
|- ( A |` B ) = ( A i\^{}i ( B X. {\char`\_}V ) ) \$.
\end{mmraw}

\noindent Define the image\index{image} of a class.  Definition 6.6(2) of
Takeuti and Zaring, p.~24.

\vskip 0.5ex
\setbox\startprefix=\hbox{\tt \ \ df-ima\ \$a\ }
\setbox\contprefix=\hbox{\tt \ \ \ \ \ \ \ \ \ \ \ \ }
\startm
\m{\vdash}\m{(}\m{A}\m{``}\m{B}\m{)}\m{=}\m{\,\mbox{\rm ran}}\m{\,(}\m{A}\m{
\restriction}\m{B}\m{)}
\endm
\begin{mmraw}%
|- ( A " B ) = ran ( A |` B ) \$.
\end{mmraw}

\noindent Define the composition\index{composition} of two classes.  Definition 6.6(3) of
Takeuti and Zaring, p.~24.

\vskip 0.5ex
\setbox\startprefix=\hbox{\tt \ \ df-co\ \$a\ }
\setbox\contprefix=\hbox{\tt \ \ \ \ \ \ \ \ \ \ \ \ }
\startm
\m{\vdash}\m{(}\m{A}\m{\circ}\m{B}\m{)}\m{=}\m{\{}\m{\langle}\m{x}\m{,}\m{y}\m{
\rangle}\m{|}\m{\exists}\m{z}\m{(}\m{\langle}\m{x}\m{,}\m{z}\m{\rangle}\m{\in}
\m{B}\m{\wedge}\m{\langle}\m{z}\m{,}\m{y}\m{\rangle}\m{\in}\m{A}\m{)}\m{\}}
\endm
\begin{mmraw}%
|- ( A o. B ) = \{ <. x , y >. | E. z ( <. x , z
>. e. B \TAND <. z , y >. e. A ) \} \$.
\end{mmraw}

\noindent Define a function\index{function}.  Definition 6.4(4) of Takeuti and
Zaring, p.~24.

\vskip 0.5ex
\setbox\startprefix=\hbox{\tt \ \ df-fun\ \$a\ }
\setbox\contprefix=\hbox{\tt \ \ \ \ \ \ \ \ \ \ \ \ }
\startm
\m{\vdash}\m{(}\m{\mbox{\rm Fun}}\m{\,A}\m{\leftrightarrow}\m{(}
\m{\mbox{\rm Rel}}\m{\,A}\m{\wedge}
\m{\forall}\m{x}\m{\exists}\m{z}\m{\forall}\m{y}\m{(}
\m{\langle}\m{x}\m{,}\m{y}\m{\rangle}\m{\in}\m{A}\m{\rightarrow}\m{y}\m{=}\m{z}
\m{)}\m{)}\m{)}
\endm
\begin{mmraw}%
|- ( Fun A <-> ( Rel A /\ A. x E. z A. y ( <. x
   , y >. e. A -> y = z ) ) ) \$.
\end{mmraw}

\noindent Define a function with domain.  Definition 6.15(1) of Takeuti and
Zaring, p.~27.

\vskip 0.5ex
\setbox\startprefix=\hbox{\tt \ \ df-fn\ \$a\ }
\setbox\contprefix=\hbox{\tt \ \ \ \ \ \ \ \ \ \ \ }
\startm
\m{\vdash}\m{(}\m{A}\m{\,\mbox{\rm Fn}}\m{\,B}\m{\leftrightarrow}\m{(}
\m{\mbox{\rm Fun}}\m{\,A}\m{\wedge}\m{\mbox{\rm dom}}\m{\,A}\m{=}\m{B}\m{)}
\m{)}
\endm
\begin{mmraw}%
|- ( A Fn B <-> ( Fun A \TAND dom A = B ) ) \$.
\end{mmraw}

\noindent Define a function with domain and co-domain.  Definition 6.15(3)
of Takeuti and Zaring, p.~27.

\vskip 0.5ex
\setbox\startprefix=\hbox{\tt \ \ df-f\ \$a\ }
\setbox\contprefix=\hbox{\tt \ \ \ \ \ \ \ \ \ \ }
\startm
\m{\vdash}\m{(}\m{F}\m{:}\m{A}\m{\longrightarrow}\m{B}\m{
\leftrightarrow}\m{(}\m{F}\m{\,\mbox{\rm Fn}}\m{\,A}\m{\wedge}\m{
\mbox{\rm ran}}\m{\,F}\m{\subseteq}\m{B}\m{)}\m{)}
\endm
\begin{mmraw}%
|- ( F : A --> B <-> ( F Fn A \TAND ran F C\_ B ) ) \$.
\end{mmraw}

\noindent Define a one-to-one function\index{one-to-one function}.  Compare
Definition 6.15(5) of Takeuti and Zaring, p.~27.

\vskip 0.5ex
\setbox\startprefix=\hbox{\tt \ \ df-f1\ \$a\ }
\setbox\contprefix=\hbox{\tt \ \ \ \ \ \ \ \ \ \ \ }
\startm
\m{\vdash}\m{(}\m{F}\m{:}\m{A}\m{
\raisebox{.5ex}{${\textstyle{\:}_{\mbox{\footnotesize\rm
1\tt -\rm 1}}}\atop{\textstyle{
\longrightarrow}\atop{\textstyle{}^{\mbox{\footnotesize\rm {\ }}}}}$}
}\m{B}
\m{\leftrightarrow}\m{(}\m{F}\m{:}\m{A}\m{\longrightarrow}\m{B}
\m{\wedge}\m{\forall}\m{y}\m{\exists}\m{z}\m{\forall}\m{x}\m{(}\m{\langle}\m{x}
\m{,}\m{y}\m{\rangle}\m{\in}\m{F}\m{\rightarrow}\m{x}\m{=}\m{z}\m{)}\m{)}\m{)}
\endm
\begin{mmraw}%
|- ( F : A -1-1-> B <-> ( F : A --> B \TAND
   A. y E. z A. x ( <. x , y >. e. F -> x = z ) ) ) \$.
\end{mmraw}

\noindent Define an onto function\index{onto function}.  Definition 6.15(4) of Takeuti and
Zaring, p.~27.

\vskip 0.5ex
\setbox\startprefix=\hbox{\tt \ \ df-fo\ \$a\ }
\setbox\contprefix=\hbox{\tt \ \ \ \ \ \ \ \ \ \ \ }
\startm
\m{\vdash}\m{(}\m{F}\m{:}\m{A}\m{
\raisebox{.5ex}{${\textstyle{\:}_{\mbox{\footnotesize\rm
{\ }}}}\atop{\textstyle{
\longrightarrow}\atop{\textstyle{}^{\mbox{\footnotesize\rm onto}}}}$}
}\m{B}
\m{\leftrightarrow}\m{(}\m{F}\m{\,\mbox{\rm Fn}}\m{\,A}\m{\wedge}
\m{\mbox{\rm ran}}\m{\,F}\m{=}\m{B}\m{)}\m{)}
\endm
\begin{mmraw}%
|- ( F : A -onto-> B <-> ( F Fn A /\ ran F = B ) ) \$.
\end{mmraw}

\noindent Define a one-to-one, onto function.  Compare Definition 6.15(6) of
Takeuti and Zaring, p.~27.

\vskip 0.5ex
\setbox\startprefix=\hbox{\tt \ \ df-f1o\ \$a\ }
\setbox\contprefix=\hbox{\tt \ \ \ \ \ \ \ \ \ \ \ \ }
\startm
\m{\vdash}\m{(}\m{F}\m{:}\m{A}
\m{
\raisebox{.5ex}{${\textstyle{\:}_{\mbox{\footnotesize\rm
1\tt -\rm 1}}}\atop{\textstyle{
\longrightarrow}\atop{\textstyle{}^{\mbox{\footnotesize\rm onto}}}}$}
}
\m{B}
\m{\leftrightarrow}\m{(}\m{F}\m{:}\m{A}
\m{
\raisebox{.5ex}{${\textstyle{\:}_{\mbox{\footnotesize\rm
1\tt -\rm 1}}}\atop{\textstyle{
\longrightarrow}\atop{\textstyle{}^{\mbox{\footnotesize\rm {\ }}}}}$}
}
\m{B}\m{\wedge}\m{F}\m{:}\m{A}
\m{
\raisebox{.5ex}{${\textstyle{\:}_{\mbox{\footnotesize\rm
{\ }}}}\atop{\textstyle{
\longrightarrow}\atop{\textstyle{}^{\mbox{\footnotesize\rm onto}}}}$}
}
\m{B}\m{)}\m{)}
\endm
\begin{mmraw}%
|- ( F : A -1-1-onto-> B <-> ( F : A -1-1-> B? \TAND F : A -onto-> B ) ) \$.?
\end{mmraw}

\noindent Define the value of a function\index{function value}.  This
definition applies to any class and evaluates to the empty set when it is not
meaningful. Note that $ F`A$ means the same thing as the more familiar $ F(A)$
notation for a function's value at $A$.  The $ F`A$ notation is common in
formal set theory.\label{df-fv}

\vskip 0.5ex
\setbox\startprefix=\hbox{\tt \ \ df-fv\ \$a\ }
\setbox\contprefix=\hbox{\tt \ \ \ \ \ \ \ \ \ \ \ }
\startm
\m{\vdash}\m{(}\m{F}\m{`}\m{A}\m{)}\m{=}\m{\bigcup}\m{\{}\m{x}\m{|}\m{(}\m{F}%
\m{``}\m{\{}\m{A}\m{\}}\m{)}\m{=}\m{\{}\m{x}\m{\}}\m{\}}
\endm
\begin{mmraw}%
|- ( F ` A ) = U. \{ x | ( F " \{ A \} ) = \{ x \} \} \$.
\end{mmraw}

\noindent Define the result of an operation.\index{operation}  Here, $F$ is
     an operation on two
     values (such as $+$ for real numbers).   This is defined for proper
     classes $A$ and $B$ even though not meaningful in that case.  However,
     the definition can be meaningful when $F$ is a proper class.\label{dfopr}

\vskip 0.5ex
\setbox\startprefix=\hbox{\tt \ \ df-opr\ \$a\ }
\setbox\contprefix=\hbox{\tt \ \ \ \ \ \ \ \ \ \ \ \ }
\startm
\m{\vdash}\m{(}\m{A}\m{\,F}\m{\,B}\m{)}\m{=}\m{(}\m{F}\m{`}\m{\langle}\m{A}%
\m{,}\m{B}\m{\rangle}\m{)}
\endm
\begin{mmraw}%
|- ( A F B ) = ( F ` <. A , B >. ) \$.
\end{mmraw}

\section{Tricks of the Trade}\label{tricks}

In the \texttt{set.mm}\index{set theory database (\texttt{set.mm})} database our goal
was usually to conform to modern notation.  However in some cases the
relationship to standard textbook language may be obscured by several
unconventional devices we used to simplify the development and to take
advantage of the Metamath language.  In this section we will describe some
common conventions used in \texttt{set.mm}.

\begin{itemize}
\item
The turnstile\index{turnstile ({$\,\vdash$})} symbol, $\vdash$, meaning ``it
is provable that,'' is the first token of all assertions and hypotheses that
aren't syntax constructions.  This is a standard convention in logic.  (We
mentioned this earlier, but this symbol is bothersome to some people without a
logic background.  It has no deeper meaning but just provides us with a way to
distinguish syntax constructions from ordinary mathematical statements.)

\item
A hypothesis of the form

\vskip 1ex
\setbox\startprefix=\hbox{\tt \ \ \ \ \ \ \ \ \ \$e\ }
\setbox\contprefix=\hbox{\tt \ \ \ \ \ \ \ \ \ \ }
\startm
\m{\vdash}\m{(}\m{\varphi}\m{\rightarrow}\m{\forall}\m{x}\m{\varphi}\m{)}
\endm
\vskip 1ex

should be read ``assume variable $x$ is (effectively) not free in wff
$\varphi$.''\index{effectively not free}
Literally, this says ``assume it is provable that $\varphi \rightarrow \forall
x\, \varphi$.''  This device lets us avoid the complexities associated with
the standard treatment of free and bound variables.
%% Uncomment this when uncommenting section {formalspec} below
The footnote on p.~\pageref{effectivelybound} discusses this further.

\item
A statement of one of the forms

\vskip 1ex
\setbox\startprefix=\hbox{\tt \ \ \ \ \ \ \ \ \ \$a\ }
\setbox\contprefix=\hbox{\tt \ \ \ \ \ \ \ \ \ \ }
\startm
\m{\vdash}\m{(}\m{\lnot}\m{\forall}\m{x}\m{\,x}\m{=}\m{y}\m{\rightarrow}
\m{\ldots}\m{)}
\endm
\setbox\startprefix=\hbox{\tt \ \ \ \ \ \ \ \ \ \$p\ }
\setbox\contprefix=\hbox{\tt \ \ \ \ \ \ \ \ \ \ }
\startm
\m{\vdash}\m{(}\m{\lnot}\m{\forall}\m{x}\m{\,x}\m{=}\m{y}\m{\rightarrow}
\m{\ldots}\m{)}
\endm
\vskip 1ex

should be read ``if $x$ and $y$ are distinct variables, then...''  This
antecedent provides us with a technical device to avoid the need for the
\texttt{\$d} statement early in our development of predicate calculus,
permitting symbol manipulations to be as conceptually simple as those in
propositional calculus.  However, the \texttt{\$d} statement eventually
becomes a requirement, and after that this device is rarely used.

\item
The statement

\vskip 1ex
\setbox\startprefix=\hbox{\tt \ \ \ \ \ \ \ \ \ \$d\ }
\setbox\contprefix=\hbox{\tt \ \ \ \ \ \ \ \ \ \ }
\startm
\m{x}\m{\,y}
\endm
\vskip 1ex

should be read ``assume $x$ and $y$ are distinct variables.''

\item
The statement

\vskip 1ex
\setbox\startprefix=\hbox{\tt \ \ \ \ \ \ \ \ \ \$d\ }
\setbox\contprefix=\hbox{\tt \ \ \ \ \ \ \ \ \ \ }
\startm
\m{x}\m{\,\varphi}
\endm
\vskip 1ex

should be read ``assume $x$ does not occur in $\varphi$.''

\item
The statement

\vskip 1ex
\setbox\startprefix=\hbox{\tt \ \ \ \ \ \ \ \ \ \$d\ }
\setbox\contprefix=\hbox{\tt \ \ \ \ \ \ \ \ \ \ }
\startm
\m{x}\m{\,A}
\endm
\vskip 1ex

should be read ``assume variable $x$ does not occur in class $A$.''

\item
The restriction and hypothesis group

\vskip 1ex
\setbox\startprefix=\hbox{\tt \ \ \ \ \ \ \ \ \ \$d\ }
\setbox\contprefix=\hbox{\tt \ \ \ \ \ \ \ \ \ \ }
\startm
\m{x}\m{\,A}
\endm
\setbox\startprefix=\hbox{\tt \ \ \ \ \ \ \ \ \ \$d\ }
\setbox\contprefix=\hbox{\tt \ \ \ \ \ \ \ \ \ \ }
\startm
\m{x}\m{\,\psi}
\endm
\setbox\startprefix=\hbox{\tt \ \ \ \ \ \ \ \ \ \$e\ }
\setbox\contprefix=\hbox{\tt \ \ \ \ \ \ \ \ \ \ }
\startm
\m{\vdash}\m{(}\m{x}\m{=}\m{A}\m{\rightarrow}\m{(}\m{\varphi}\m{\leftrightarrow}
\m{\psi}\m{)}\m{)}
\endm
\vskip 1ex

is frequently used in place of explicit substitution, meaning ``assume
$\psi$ results from the proper substitution of $A$ for $x$ in
$\varphi$.''  Sometimes ``\texttt{\$e} $\vdash ( \psi \rightarrow
\forall x \, \psi )$'' is used instead of ``\texttt{\$d} $x\, \psi $,''
which requires only that $x$ be effectively not free in $\varphi$ but
not necessarily absent from it.  The use of implicit
substitution\index{substitution!implicit} is partly a matter of personal
style, although it may make proofs somewhat shorter than would be the
case with explicit substitution.

\item
The hypothesis


\vskip 1ex
\setbox\startprefix=\hbox{\tt \ \ \ \ \ \ \ \ \ \$e\ }
\setbox\contprefix=\hbox{\tt \ \ \ \ \ \ \ \ \ \ }
\startm
\m{\vdash}\m{A}\m{\in}\m{{\rm V}}
\endm
\vskip 1ex

should be read ``assume class $A$ is a set (i.e.\ exists).''
This is a convenient convention used by Quine.

\item
The restriction and hypothesis

\vskip 1ex
\setbox\startprefix=\hbox{\tt \ \ \ \ \ \ \ \ \ \$d\ }
\setbox\contprefix=\hbox{\tt \ \ \ \ \ \ \ \ \ \ }
\startm
\m{x}\m{\,y}
\endm
\setbox\startprefix=\hbox{\tt \ \ \ \ \ \ \ \ \ \$e\ }
\setbox\contprefix=\hbox{\tt \ \ \ \ \ \ \ \ \ \ }
\startm
\m{\vdash}\m{(}\m{y}\m{\in}\m{A}\m{\rightarrow}\m{\forall}\m{x}\m{\,y}
\m{\in}\m{A}\m{)}
\endm
\vskip 1ex

should be read ``assume variable $x$ is
(effectively) not free in class $A$.''

\end{itemize}

\section{A Theorem Sampler}\label{sometheorems}

In this section we list some of the more important theorems that are proved in
the \texttt{set.mm} database, and they illustrate the kinds of things that can be
done with Metamath.  While all of these facts are well-known results,
Metamath offers the advantage of easily allowing you to trace their
derivation back to axioms.  Our intent here is not to try to explain the
details or motivation; for this we refer you to the textbooks that are
mentioned in the descriptions.  (The \texttt{set.mm} file has bibliographic
references for the text references.)  Their proofs often embody important
concepts you may wish to explore with the Metamath program (see
Section~\ref{exploring}).  All the symbols that are used here are defined in
Section~\ref{hierarchy}.  For brevity we haven't included the \texttt{\$d}
restrictions or \texttt{\$f} hypotheses for these theorems; when you are
uncertain consult the \texttt{set.mm} database.

We start with \texttt{syl} (principle of the syllogism).
In \textit{Principia Mathematica}
Whitehead and Russell call this ``the principle of the
syllogism... because... the syllogism in Barbara is derived from them''
\cite[quote after Theorem *2.06 p.~101]{PM}.
Some authors call this law a ``hypothetical syllogism.''
As of 2019 \texttt{syl} is the most commonly referenced proven
assertion in the \texttt{set.mm} database.\footnote{
The Metamath program command \texttt{show usage}
shows the number of references.
On 2019-04-29 (commit 71cbbbdb387e) \texttt{syl} was directly referenced
10,819 times. The second most commonly referenced proven assertion
was \texttt{eqid}, which was directly referenced 7,738 times.
}

\vskip 2ex
\noindent Theorem syl (principle of the syllogism)\index{Syllogism}%
\index{\texttt{syl}}\label{syl}.

\vskip 0.5ex
\setbox\startprefix=\hbox{\tt \ \ syl.1\ \$e\ }
\setbox\contprefix=\hbox{\tt \ \ \ \ \ \ \ \ \ \ \ }
\startm
\m{\vdash}\m{(}\m{\varphi}\m{ \rightarrow }\m{\psi}\m{)}
\endm
\setbox\startprefix=\hbox{\tt \ \ syl.2\ \$e\ }
\setbox\contprefix=\hbox{\tt \ \ \ \ \ \ \ \ \ \ \ }
\startm
\m{\vdash}\m{(}\m{\psi}\m{ \rightarrow }\m{\chi}\m{)}
\endm
\setbox\startprefix=\hbox{\tt \ \ syl\ \$p\ }
\setbox\contprefix=\hbox{\tt \ \ \ \ \ \ \ \ \ }
\startm
\m{\vdash}\m{(}\m{\varphi}\m{ \rightarrow }\m{\chi}\m{)}
\endm
\vskip 2ex

The following theorem is not very deep but provides us with a notational device
that is frequently used.  It allows us to use the expression ``$A \in V$'' as
a compact way of saying that class $A$ exists, i.e.\ is a set.

\vskip 2ex
\noindent Two ways to say ``$A$ is a set'':  $A$ is a member of the universe
$V$ if and only if $A$ exists (i.e.\ there exists a set equal to $A$).
Theorem 6.9 of Quine, p. 43.

\vskip 0.5ex
\setbox\startprefix=\hbox{\tt \ \ isset\ \$p\ }
\setbox\contprefix=\hbox{\tt \ \ \ \ \ \ \ \ \ \ \ }
\startm
\m{\vdash}\m{(}\m{A}\m{\in}\m{{\rm V}}\m{\leftrightarrow}\m{\exists}\m{x}\m{\,x}\m{=}
\m{A}\m{)}
\endm
\vskip 1ex

Next we prove the axioms of standard ZF set theory that were missing from our
axiom system.  From our point of view they are theorems since they
can be derived from the other axioms.

\vskip 2ex
\noindent Axiom of Separation\index{Axiom of Separation}
(Aussonderung)\index{Aussonderung} proved from the other axioms of ZF set
theory.  Compare Exercise 4 of Takeuti and Zaring, p.~22.

\vskip 0.5ex
\setbox\startprefix=\hbox{\tt \ \ inex1.1\ \$e\ }
\setbox\contprefix=\hbox{\tt \ \ \ \ \ \ \ \ \ \ \ \ \ \ \ }
\startm
\m{\vdash}\m{A}\m{\in}\m{{\rm V}}
\endm
\setbox\startprefix=\hbox{\tt \ \ inex\ \$p\ }
\setbox\contprefix=\hbox{\tt \ \ \ \ \ \ \ \ \ \ \ \ \ }
\startm
\m{\vdash}\m{(}\m{A}\m{\cap}\m{B}\m{)}\m{\in}\m{{\rm V}}
\endm
\vskip 1ex

\noindent Axiom of the Null Set\index{Axiom of the Null Set} proved from the
other axioms of ZF set theory. Corollary 5.16 of Takeuti and Zaring, p.~20.

\vskip 0.5ex
\setbox\startprefix=\hbox{\tt \ \ 0ex\ \$p\ }
\setbox\contprefix=\hbox{\tt \ \ \ \ \ \ \ \ \ \ \ \ }
\startm
\m{\vdash}\m{\varnothing}\m{\in}\m{{\rm V}}
\endm
\vskip 1ex

\noindent The Axiom of Pairing\index{Axiom of Pairing} proved from the other
axioms of ZF set theory.  Theorem 7.13 of Quine, p.~51.
\vskip 0.5ex
\setbox\startprefix=\hbox{\tt \ \ prex\ \$p\ }
\setbox\contprefix=\hbox{\tt \ \ \ \ \ \ \ \ \ \ \ \ \ \ }
\startm
\m{\vdash}\m{\{}\m{A}\m{,}\m{B}\m{\}}\m{\in}\m{{\rm V}}
\endm
\vskip 2ex

Next we will list some famous or important theorems that are proved in
the \texttt{set.mm} database.  None of them except \texttt{omex}
require the Axiom of Infinity, as you can verify with the \texttt{show
trace{\char`\_}back} Metamath command.

\vskip 2ex
\noindent The resolution of Russell's paradox\index{Russell's paradox}.  There
exists no set containing the set of all sets which are not members of
themselves.  Proposition 4.14 of Takeuti and Zaring, p.~14.

\vskip 0.5ex
\setbox\startprefix=\hbox{\tt \ \ ru\ \$p\ }
\setbox\contprefix=\hbox{\tt \ \ \ \ \ \ \ \ }
\startm
\m{\vdash}\m{\lnot}\m{\exists}\m{x}\m{\,x}\m{=}\m{\{}\m{y}\m{|}\m{\lnot}\m{y}
\m{\in}\m{y}\m{\}}
\endm
\vskip 1ex

\noindent Cantor's theorem\index{Cantor's theorem}.  No set can be mapped onto
its power set.  Compare Theorem 6B(b) of Enderton, p.~132.

\vskip 0.5ex
\setbox\startprefix=\hbox{\tt \ \ canth.1\ \$e\ }
\setbox\contprefix=\hbox{\tt \ \ \ \ \ \ \ \ \ \ \ \ \ }
\startm
\m{\vdash}\m{A}\m{\in}\m{{\rm V}}
\endm
\setbox\startprefix=\hbox{\tt \ \ canth\ \$p\ }
\setbox\contprefix=\hbox{\tt \ \ \ \ \ \ \ \ \ \ \ }
\startm
\m{\vdash}\m{\lnot}\m{F}\m{:}\m{A}\m{\raisebox{.5ex}{${\textstyle{\:}_{
\mbox{\footnotesize\rm {\ }}}}\atop{\textstyle{\longrightarrow}\atop{
\textstyle{}^{\mbox{\footnotesize\rm onto}}}}$}}\m{{\cal P}}\m{A}
\endm
\vskip 1ex

\noindent The Burali-Forti paradox\index{Burali-Forti paradox}.  No set
contains all ordinal numbers. Enderton, p.~194.  (Burali-Forti was one person,
not two.)

\vskip 0.5ex
\setbox\startprefix=\hbox{\tt \ \ onprc\ \$p\ }
\setbox\contprefix=\hbox{\tt \ \ \ \ \ \ \ \ \ \ \ \ }
\startm
\m{\vdash}\m{\lnot}\m{\mbox{\rm On}}\m{\in}\m{{\rm V}}
\endm
\vskip 1ex

\noindent Peano's postulates\index{Peano's postulates} for arithmetic.
Proposition 7.30 of Takeuti and Zaring, pp.~42--43.  The objects being
described are the members of $\omega$ i.e.\ the natural numbers 0, 1,
2,\ldots.  The successor\index{successor} operation suc means ``plus
one.''  \texttt{peano1} says that 0 (which is defined as the empty set)
is a natural number.  \texttt{peano2} says that if $A$ is a natural
number, so is $A+1$.  \texttt{peano3} says that 0 is not the successor
of any natural number.  \texttt{peano4} says that two natural numbers
are equal if and only if their successors are equal.  \texttt{peano5} is
essentially the same as mathematical induction.

\vskip 1ex
\setbox\startprefix=\hbox{\tt \ \ peano1\ \$p\ }
\setbox\contprefix=\hbox{\tt \ \ \ \ \ \ \ \ \ \ \ \ }
\startm
\m{\vdash}\m{\varnothing}\m{\in}\m{\omega}
\endm
\vskip 1.5ex

\setbox\startprefix=\hbox{\tt \ \ peano2\ \$p\ }
\setbox\contprefix=\hbox{\tt \ \ \ \ \ \ \ \ \ \ \ \ }
\startm
\m{\vdash}\m{(}\m{A}\m{\in}\m{\omega}\m{\rightarrow}\m{{\rm suc}}\m{A}\m{\in}%
\m{\omega}\m{)}
\endm
\vskip 1.5ex

\setbox\startprefix=\hbox{\tt \ \ peano3\ \$p\ }
\setbox\contprefix=\hbox{\tt \ \ \ \ \ \ \ \ \ \ \ \ }
\startm
\m{\vdash}\m{(}\m{A}\m{\in}\m{\omega}\m{\rightarrow}\m{\lnot}\m{{\rm suc}}%
\m{A}\m{=}\m{\varnothing}\m{)}
\endm
\vskip 1.5ex

\setbox\startprefix=\hbox{\tt \ \ peano4\ \$p\ }
\setbox\contprefix=\hbox{\tt \ \ \ \ \ \ \ \ \ \ \ \ }
\startm
\m{\vdash}\m{(}\m{(}\m{A}\m{\in}\m{\omega}\m{\wedge}\m{B}\m{\in}\m{\omega}%
\m{)}\m{\rightarrow}\m{(}\m{{\rm suc}}\m{A}\m{=}\m{{\rm suc}}\m{B}\m{%
\leftrightarrow}\m{A}\m{=}\m{B}\m{)}\m{)}
\endm
\vskip 1.5ex

\setbox\startprefix=\hbox{\tt \ \ peano5\ \$p\ }
\setbox\contprefix=\hbox{\tt \ \ \ \ \ \ \ \ \ \ \ \ }
\startm
\m{\vdash}\m{(}\m{(}\m{\varnothing}\m{\in}\m{A}\m{\wedge}\m{\forall}\m{x}\m{%
\in}\m{\omega}\m{(}\m{x}\m{\in}\m{A}\m{\rightarrow}\m{{\rm suc}}\m{x}\m{\in}%
\m{A}\m{)}\m{)}\m{\rightarrow}\m{\omega}\m{\subseteq}\m{A}\m{)}
\endm
\vskip 1.5ex

\noindent Finite Induction (mathematical induction).\index{finite
induction}\index{mathematical induction} The first hypothesis is the
basis and the second is the induction hypothesis.  Theorem Schema 22 of
Suppes, p.~136.

\vskip 0.5ex
\setbox\startprefix=\hbox{\tt \ \ findes.1\ \$e\ }
\setbox\contprefix=\hbox{\tt \ \ \ \ \ \ \ \ \ \ \ \ \ \ }
\startm
\m{\vdash}\m{[}\m{\varnothing}\m{/}\m{x}\m{]}\m{\varphi}
\endm
\setbox\startprefix=\hbox{\tt \ \ findes.2\ \$e\ }
\setbox\contprefix=\hbox{\tt \ \ \ \ \ \ \ \ \ \ \ \ \ \ }
\startm
\m{\vdash}\m{(}\m{x}\m{\in}\m{\omega}\m{\rightarrow}\m{(}\m{\varphi}\m{%
\rightarrow}\m{[}\m{{\rm suc}}\m{x}\m{/}\m{x}\m{]}\m{\varphi}\m{)}\m{)}
\endm
\setbox\startprefix=\hbox{\tt \ \ findes\ \$p\ }
\setbox\contprefix=\hbox{\tt \ \ \ \ \ \ \ \ \ \ \ \ }
\startm
\m{\vdash}\m{(}\m{x}\m{\in}\m{\omega}\m{\rightarrow}\m{\varphi}\m{)}
\endm
\vskip 1ex

\noindent Transfinite Induction with explicit substitution.  The first
hypothesis is the basis, the second is the induction hypothesis for
successors, and the third is the induction hypothesis for limit
ordinals.  Theorem Schema 4 of Suppes, p. 197.

\vskip 0.5ex
\setbox\startprefix=\hbox{\tt \ \ tfindes.1\ \$e\ }
\setbox\contprefix=\hbox{\tt \ \ \ \ \ \ \ \ \ \ \ \ \ \ \ }
\startm
\m{\vdash}\m{[}\m{\varnothing}\m{/}\m{x}\m{]}\m{\varphi}
\endm
\setbox\startprefix=\hbox{\tt \ \ tfindes.2\ \$e\ }
\setbox\contprefix=\hbox{\tt \ \ \ \ \ \ \ \ \ \ \ \ \ \ \ }
\startm
\m{\vdash}\m{(}\m{x}\m{\in}\m{{\rm On}}\m{\rightarrow}\m{(}\m{\varphi}\m{%
\rightarrow}\m{[}\m{{\rm suc}}\m{x}\m{/}\m{x}\m{]}\m{\varphi}\m{)}\m{)}
\endm
\setbox\startprefix=\hbox{\tt \ \ tfindes.3\ \$e\ }
\setbox\contprefix=\hbox{\tt \ \ \ \ \ \ \ \ \ \ \ \ \ \ \ }
\startm
\m{\vdash}\m{(}\m{{\rm Lim}}\m{y}\m{\rightarrow}\m{(}\m{\forall}\m{x}\m{\in}%
\m{y}\m{\varphi}\m{\rightarrow}\m{[}\m{y}\m{/}\m{x}\m{]}\m{\varphi}\m{)}\m{)}
\endm
\setbox\startprefix=\hbox{\tt \ \ tfindes\ \$p\ }
\setbox\contprefix=\hbox{\tt \ \ \ \ \ \ \ \ \ \ \ \ \ }
\startm
\m{\vdash}\m{(}\m{x}\m{\in}\m{{\rm On}}\m{\rightarrow}\m{\varphi}\m{)}
\endm
\vskip 1ex

\noindent Principle of Transfinite Recursion.\index{transfinite
recursion} Theorem 7.41 of Takeuti and Zaring, p.~47.  Transfinite
recursion is the key theorem that allows arithmetic of ordinals to be
rigorously defined, and has many other important uses as well.
Hypotheses \texttt{tfr.1} and \texttt{tfr.2} specify a certain (proper)
class $ F$.  The complicated definition of $ F$ is not important in
itself; what is important is that there be such an $ F$ with the
required properties, and we show this by displaying $ F$ explicitly.
\texttt{tfr1} states that $ F$ is a function whose domain is the set of
ordinal numbers.  \texttt{tfr2} states that any value of $ F$ is
completely determined by its previous values and the values of an
auxiliary function, $G$.  \texttt{tfr3} states that $ F$ is unique,
i.e.\ it is the only function that satisfies \texttt{tfr1} and
\texttt{tfr2}.  Note that $ f$ is an individual variable like $x$ and
$y$; it is just a mnemonic to remind us that $A$ is a collection of
functions.

\vskip 0.5ex
\setbox\startprefix=\hbox{\tt \ \ tfr.1\ \$e\ }
\setbox\contprefix=\hbox{\tt \ \ \ \ \ \ \ \ \ \ \ }
\startm
\m{\vdash}\m{A}\m{=}\m{\{}\m{f}\m{|}\m{\exists}\m{x}\m{\in}\m{{\rm On}}\m{(}%
\m{f}\m{{\rm Fn}}\m{x}\m{\wedge}\m{\forall}\m{y}\m{\in}\m{x}\m{(}\m{f}\m{`}%
\m{y}\m{)}\m{=}\m{(}\m{G}\m{`}\m{(}\m{f}\m{\restriction}\m{y}\m{)}\m{)}\m{)}%
\m{\}}
\endm
\setbox\startprefix=\hbox{\tt \ \ tfr.2\ \$e\ }
\setbox\contprefix=\hbox{\tt \ \ \ \ \ \ \ \ \ \ \ }
\startm
\m{\vdash}\m{F}\m{=}\m{\bigcup}\m{A}
\endm
\setbox\startprefix=\hbox{\tt \ \ tfr1\ \$p\ }
\setbox\contprefix=\hbox{\tt \ \ \ \ \ \ \ \ \ \ }
\startm
\m{\vdash}\m{F}\m{{\rm Fn}}\m{{\rm On}}
\endm
\setbox\startprefix=\hbox{\tt \ \ tfr2\ \$p\ }
\setbox\contprefix=\hbox{\tt \ \ \ \ \ \ \ \ \ \ }
\startm
\m{\vdash}\m{(}\m{z}\m{\in}\m{{\rm On}}\m{\rightarrow}\m{(}\m{F}\m{`}\m{z}%
\m{)}\m{=}\m{(}\m{G}\m{`}\m{(}\m{F}\m{\restriction}\m{z}\m{)}\m{)}\m{)}
\endm
\setbox\startprefix=\hbox{\tt \ \ tfr3\ \$p\ }
\setbox\contprefix=\hbox{\tt \ \ \ \ \ \ \ \ \ \ }
\startm
\m{\vdash}\m{(}\m{(}\m{B}\m{{\rm Fn}}\m{{\rm On}}\m{\wedge}\m{\forall}\m{x}\m{%
\in}\m{{\rm On}}\m{(}\m{B}\m{`}\m{x}\m{)}\m{=}\m{(}\m{G}\m{`}\m{(}\m{B}\m{%
\restriction}\m{x}\m{)}\m{)}\m{)}\m{\rightarrow}\m{B}\m{=}\m{F}\m{)}
\endm
\vskip 1ex

\noindent The existence of omega (the class of natural numbers).\index{natural
number}\index{omega ($\omega$)}\index{Axiom of Infinity}  Axiom 7 of Takeuti
and Zaring, p.~43.  (This is the only theorem in this section requiring the
Axiom of Infinity.)

\vskip 0.5ex
\setbox\startprefix=\hbox{\tt \
\ omex\ \$p\ }
\setbox\contprefix=\hbox{\tt \ \ \ \ \ \ \ \ \ \ }
\startm
\m{\vdash}\m{\omega}\m{\in}\m{{\rm V}}
\endm
%\vskip 2ex


\section{Axioms for Real and Complex Numbers}\label{real}
\index{real and complex numbers!axioms for}

This section presents the axioms for real and complex numbers, along
with some commentary about them.  Analysis
textbooks implicitly or explicitly use these axioms or their equivalents
as their starting point.  In the database \texttt{set.mm}, we define real
and complex numbers as (rather complicated) specific sets and derive these
axioms as {\em theorems} from the axioms of ZF set theory, using a method
called Dedekind cuts.  We omit the details of this construction, which you can
follow if you wish using the \texttt{set.mm} database in conjunction with the
textbooks referenced therein.

Once we prove those theorems, we then restate these proven theorems as axioms.
This lets us easily identify which axioms are needed for a particular complex number proof, without the obfuscation of the set theory used to derive them.
As a result,
the construction is actually unimportant other
than to show that sets exist that satisfy the axioms, and thus that the axioms
are consistent if ZF set theory is consistent.  When working with real numbers
you can think of them as being the actual sets resulting
from the construction (for definiteness), or you can
think of them as otherwise unspecified sets that happen to satisfy the axioms.
The derivation is not easy, but the fact that it works is quite remarkable
and lends support to the idea that ZFC set theory is all we need to
provide a foundation for essentially all of mathematics.

\needspace{3\baselineskip}
\subsection{The Axioms for Real and Complex Numbers Themselves}\label{realactual}

For the axioms we are given (or postulate) 8 classes:  $\mathbb{C}$ (the
set of complex numbers), $\mathbb{R}$ (the set of real numbers, a subset
of $\mathbb{C}$), $0$ (zero), $1$ (one), $i$ (square root of
$-1$), $+$ (plus), $\cdot$ (times), and
$<_{\mathbb{R}}$ (less than for just the real numbers).
Subtraction and division are defined terms and are not part of the
axioms; for their definitions see \texttt{set.mm}.

Note that the notation $(A+B)$ (and similarly $(A\cdot B)$) specifies a class
called an {\em operation},\index{operation} and is the function value of the
class $+$ at ordered pair $\langle A,B \rangle$.  An operation is defined by
statement \texttt{df-opr} on p.~\pageref{dfopr}.
The notation $A <_{\mathbb{R}} B$ specifies a
wff called a {\em binary relation}\index{binary relation} and means $\langle A,B \rangle \in \,<_{\mathbb{R}}$, as defined by statement \texttt{df-br} on p.~\pageref{dfbr}.

Our set of 8 given classes is assumed to satisfy the following 22 axioms
(in the axioms listed below, $<$ really means $<_{\mathbb{R}}$).

\vskip 2ex

\noindent 1. The real numbers are a subset of the complex numbers.

%\vskip 0.5ex
\setbox\startprefix=\hbox{\tt \ \ ax-resscn\ \$p\ }
\setbox\contprefix=\hbox{\tt \ \ \ \ \ \ \ \ \ \ \ \ \ \ }
\startm
\m{\vdash}\m{\mathbb{R}}\m{\subseteq}\m{\mathbb{C}}
\endm
%\vskip 1ex

\noindent 2. One is a complex number.

%\vskip 0.5ex
\setbox\startprefix=\hbox{\tt \ \ ax-1cn\ \$p\ }
\setbox\contprefix=\hbox{\tt \ \ \ \ \ \ \ \ \ \ \ }
\startm
\m{\vdash}\m{1}\m{\in}\m{\mathbb{C}}
\endm
%\vskip 1ex

\noindent 3. The imaginary number $i$ is a complex number.

%\vskip 0.5ex
\setbox\startprefix=\hbox{\tt \ \ ax-icn\ \$p\ }
\setbox\contprefix=\hbox{\tt \ \ \ \ \ \ \ \ \ \ \ }
\startm
\m{\vdash}\m{i}\m{\in}\m{\mathbb{C}}
\endm
%\vskip 1ex

\noindent 4. Complex numbers are closed under addition.

%\vskip 0.5ex
\setbox\startprefix=\hbox{\tt \ \ ax-addcl\ \$p\ }
\setbox\contprefix=\hbox{\tt \ \ \ \ \ \ \ \ \ \ \ \ \ }
\startm
\m{\vdash}\m{(}\m{(}\m{A}\m{\in}\m{\mathbb{C}}\m{\wedge}\m{B}\m{\in}\m{\mathbb{C}}%
\m{)}\m{\rightarrow}\m{(}\m{A}\m{+}\m{B}\m{)}\m{\in}\m{\mathbb{C}}\m{)}
\endm
%\vskip 1ex

\noindent 5. Real numbers are closed under addition.

%\vskip 0.5ex
\setbox\startprefix=\hbox{\tt \ \ ax-addrcl\ \$p\ }
\setbox\contprefix=\hbox{\tt \ \ \ \ \ \ \ \ \ \ \ \ \ \ }
\startm
\m{\vdash}\m{(}\m{(}\m{A}\m{\in}\m{\mathbb{R}}\m{\wedge}\m{B}\m{\in}\m{\mathbb{R}}%
\m{)}\m{\rightarrow}\m{(}\m{A}\m{+}\m{B}\m{)}\m{\in}\m{\mathbb{R}}\m{)}
\endm
%\vskip 1ex

\noindent 6. Complex numbers are closed under multiplication.

%\vskip 0.5ex
\setbox\startprefix=\hbox{\tt \ \ ax-mulcl\ \$p\ }
\setbox\contprefix=\hbox{\tt \ \ \ \ \ \ \ \ \ \ \ \ \ }
\startm
\m{\vdash}\m{(}\m{(}\m{A}\m{\in}\m{\mathbb{C}}\m{\wedge}\m{B}\m{\in}\m{\mathbb{C}}%
\m{)}\m{\rightarrow}\m{(}\m{A}\m{\cdot}\m{B}\m{)}\m{\in}\m{\mathbb{C}}\m{)}
\endm
%\vskip 1ex

\noindent 7. Real numbers are closed under multiplication.

%\vskip 0.5ex
\setbox\startprefix=\hbox{\tt \ \ ax-mulrcl\ \$p\ }
\setbox\contprefix=\hbox{\tt \ \ \ \ \ \ \ \ \ \ \ \ \ \ }
\startm
\m{\vdash}\m{(}\m{(}\m{A}\m{\in}\m{\mathbb{R}}\m{\wedge}\m{B}\m{\in}\m{\mathbb{R}}%
\m{)}\m{\rightarrow}\m{(}\m{A}\m{\cdot}\m{B}\m{)}\m{\in}\m{\mathbb{R}}\m{)}
\endm
%\vskip 1ex

\noindent 8. Multiplication of complex numbers is commutative.

%\vskip 0.5ex
\setbox\startprefix=\hbox{\tt \ \ ax-mulcom\ \$p\ }
\setbox\contprefix=\hbox{\tt \ \ \ \ \ \ \ \ \ \ \ \ \ \ }
\startm
\m{\vdash}\m{(}\m{(}\m{A}\m{\in}\m{\mathbb{C}}\m{\wedge}\m{B}\m{\in}\m{\mathbb{C}}%
\m{)}\m{\rightarrow}\m{(}\m{A}\m{\cdot}\m{B}\m{)}\m{=}\m{(}\m{B}\m{\cdot}\m{A}%
\m{)}\m{)}
\endm
%\vskip 1ex

\noindent 9. Addition of complex numbers is associative.

%\vskip 0.5ex
\setbox\startprefix=\hbox{\tt \ \ ax-addass\ \$p\ }
\setbox\contprefix=\hbox{\tt \ \ \ \ \ \ \ \ \ \ \ \ \ \ }
\startm
\m{\vdash}\m{(}\m{(}\m{A}\m{\in}\m{\mathbb{C}}\m{\wedge}\m{B}\m{\in}\m{\mathbb{C}}%
\m{\wedge}\m{C}\m{\in}\m{\mathbb{C}}\m{)}\m{\rightarrow}\m{(}\m{(}\m{A}\m{+}%
\m{B}\m{)}\m{+}\m{C}\m{)}\m{=}\m{(}\m{A}\m{+}\m{(}\m{B}\m{+}\m{C}\m{)}\m{)}%
\m{)}
\endm
%\vskip 1ex

\noindent 10. Multiplication of complex numbers is associative.

%\vskip 0.5ex
\setbox\startprefix=\hbox{\tt \ \ ax-mulass\ \$p\ }
\setbox\contprefix=\hbox{\tt \ \ \ \ \ \ \ \ \ \ \ \ \ \ }
\startm
\m{\vdash}\m{(}\m{(}\m{A}\m{\in}\m{\mathbb{C}}\m{\wedge}\m{B}\m{\in}\m{\mathbb{C}}%
\m{\wedge}\m{C}\m{\in}\m{\mathbb{C}}\m{)}\m{\rightarrow}\m{(}\m{(}\m{A}\m{\cdot}%
\m{B}\m{)}\m{\cdot}\m{C}\m{)}\m{=}\m{(}\m{A}\m{\cdot}\m{(}\m{B}\m{\cdot}\m{C}%
\m{)}\m{)}\m{)}
\endm
%\vskip 1ex

\noindent 11. Multiplication distributes over addition for complex numbers.

%\vskip 0.5ex
\setbox\startprefix=\hbox{\tt \ \ ax-distr\ \$p\ }
\setbox\contprefix=\hbox{\tt \ \ \ \ \ \ \ \ \ \ \ \ \ }
\startm
\m{\vdash}\m{(}\m{(}\m{A}\m{\in}\m{\mathbb{C}}\m{\wedge}\m{B}\m{\in}\m{\mathbb{C}}%
\m{\wedge}\m{C}\m{\in}\m{\mathbb{C}}\m{)}\m{\rightarrow}\m{(}\m{A}\m{\cdot}\m{(}%
\m{B}\m{+}\m{C}\m{)}\m{)}\m{=}\m{(}\m{(}\m{A}\m{\cdot}\m{B}\m{)}\m{+}\m{(}%
\m{A}\m{\cdot}\m{C}\m{)}\m{)}\m{)}
\endm
%\vskip 1ex

\noindent 12. The square of $i$ equals $-1$ (expressed as $i$-squared plus 1 is
0).

%\vskip 0.5ex
\setbox\startprefix=\hbox{\tt \ \ ax-i2m1\ \$p\ }
\setbox\contprefix=\hbox{\tt \ \ \ \ \ \ \ \ \ \ \ \ }
\startm
\m{\vdash}\m{(}\m{(}\m{i}\m{\cdot}\m{i}\m{)}\m{+}\m{1}\m{)}\m{=}\m{0}
\endm
%\vskip 1ex

\noindent 13. One and zero are distinct.

%\vskip 0.5ex
\setbox\startprefix=\hbox{\tt \ \ ax-1ne0\ \$p\ }
\setbox\contprefix=\hbox{\tt \ \ \ \ \ \ \ \ \ \ \ \ }
\startm
\m{\vdash}\m{1}\m{\ne}\m{0}
\endm
%\vskip 1ex

\noindent 14. One is an identity element for real multiplication.

%\vskip 0.5ex
\setbox\startprefix=\hbox{\tt \ \ ax-1rid\ \$p\ }
\setbox\contprefix=\hbox{\tt \ \ \ \ \ \ \ \ \ \ \ }
\startm
\m{\vdash}\m{(}\m{A}\m{\in}\m{\mathbb{R}}\m{\rightarrow}\m{(}\m{A}\m{\cdot}\m{1}%
\m{)}\m{=}\m{A}\m{)}
\endm
%\vskip 1ex

\noindent 15. Every real number has a negative.

%\vskip 0.5ex
\setbox\startprefix=\hbox{\tt \ \ ax-rnegex\ \$p\ }
\setbox\contprefix=\hbox{\tt \ \ \ \ \ \ \ \ \ \ \ \ \ \ }
\startm
\m{\vdash}\m{(}\m{A}\m{\in}\m{\mathbb{R}}\m{\rightarrow}\m{\exists}\m{x}\m{\in}%
\m{\mathbb{R}}\m{(}\m{A}\m{+}\m{x}\m{)}\m{=}\m{0}\m{)}
\endm
%\vskip 1ex

\noindent 16. Every nonzero real number has a reciprocal.

%\vskip 0.5ex
\setbox\startprefix=\hbox{\tt \ \ ax-rrecex\ \$p\ }
\setbox\contprefix=\hbox{\tt \ \ \ \ \ \ \ \ \ \ \ \ \ \ }
\startm
\m{\vdash}\m{(}\m{A}\m{\in}\m{\mathbb{R}}\m{\rightarrow}\m{(}\m{A}\m{\ne}\m{0}%
\m{\rightarrow}\m{\exists}\m{x}\m{\in}\m{\mathbb{R}}\m{(}\m{A}\m{\cdot}%
\m{x}\m{)}\m{=}\m{1}\m{)}\m{)}
\endm
%\vskip 1ex

\noindent 17. A complex number can be expressed in terms of two reals.

%\vskip 0.5ex
\setbox\startprefix=\hbox{\tt \ \ ax-cnre\ \$p\ }
\setbox\contprefix=\hbox{\tt \ \ \ \ \ \ \ \ \ \ \ \ }
\startm
\m{\vdash}\m{(}\m{A}\m{\in}\m{\mathbb{C}}\m{\rightarrow}\m{\exists}\m{x}\m{\in}%
\m{\mathbb{R}}\m{\exists}\m{y}\m{\in}\m{\mathbb{R}}\m{A}\m{=}\m{(}\m{x}\m{+}\m{(}%
\m{y}\m{\cdot}\m{i}\m{)}\m{)}\m{)}
\endm
%\vskip 1ex

\noindent 18. Ordering on reals satisfies strict trichotomy.

%\vskip 0.5ex
\setbox\startprefix=\hbox{\tt \ \ ax-pre-lttri\ \$p\ }
\setbox\contprefix=\hbox{\tt \ \ \ \ \ \ \ \ \ \ \ \ \ }
\startm
\m{\vdash}\m{(}\m{(}\m{A}\m{\in}\m{\mathbb{R}}\m{\wedge}\m{B}\m{\in}\m{\mathbb{R}}%
\m{)}\m{\rightarrow}\m{(}\m{A}\m{<}\m{B}\m{\leftrightarrow}\m{\lnot}\m{(}\m{A}%
\m{=}\m{B}\m{\vee}\m{B}\m{<}\m{A}\m{)}\m{)}\m{)}
\endm
%\vskip 1ex

\noindent 19. Ordering on reals is transitive.

%\vskip 0.5ex
\setbox\startprefix=\hbox{\tt \ \ ax-pre-lttrn\ \$p\ }
\setbox\contprefix=\hbox{\tt \ \ \ \ \ \ \ \ \ \ \ \ \ }
\startm
\m{\vdash}\m{(}\m{(}\m{A}\m{\in}\m{\mathbb{R}}\m{\wedge}\m{B}\m{\in}\m{\mathbb{R}}%
\m{\wedge}\m{C}\m{\in}\m{\mathbb{R}}\m{)}\m{\rightarrow}\m{(}\m{(}\m{A}\m{<}%
\m{B}\m{\wedge}\m{B}\m{<}\m{C}\m{)}\m{\rightarrow}\m{A}\m{<}\m{C}\m{)}\m{)}
\endm
%\vskip 1ex

\noindent 20. Ordering on reals is preserved after addition to both sides.

%\vskip 0.5ex
\setbox\startprefix=\hbox{\tt \ \ ax-pre-ltadd\ \$p\ }
\setbox\contprefix=\hbox{\tt \ \ \ \ \ \ \ \ \ \ \ \ \ }
\startm
\m{\vdash}\m{(}\m{(}\m{A}\m{\in}\m{\mathbb{R}}\m{\wedge}\m{B}\m{\in}\m{\mathbb{R}}%
\m{\wedge}\m{C}\m{\in}\m{\mathbb{R}}\m{)}\m{\rightarrow}\m{(}\m{A}\m{<}\m{B}\m{%
\rightarrow}\m{(}\m{C}\m{+}\m{A}\m{)}\m{<}\m{(}\m{C}\m{+}\m{B}\m{)}\m{)}\m{)}
\endm
%\vskip 1ex

\noindent 21. The product of two positive reals is positive.

%\vskip 0.5ex
\setbox\startprefix=\hbox{\tt \ \ ax-pre-mulgt0\ \$p\ }
\setbox\contprefix=\hbox{\tt \ \ \ \ \ \ \ \ \ \ \ \ \ \ }
\startm
\m{\vdash}\m{(}\m{(}\m{A}\m{\in}\m{\mathbb{R}}\m{\wedge}\m{B}\m{\in}\m{\mathbb{R}}%
\m{)}\m{\rightarrow}\m{(}\m{(}\m{0}\m{<}\m{A}\m{\wedge}\m{0}%
\m{<}\m{B}\m{)}\m{\rightarrow}\m{0}\m{<}\m{(}\m{A}\m{\cdot}\m{B}\m{)}%
\m{)}\m{)}
\endm
%\vskip 1ex

\noindent 22. A non-empty, bounded-above set of reals has a supremum.

%\vskip 0.5ex
\setbox\startprefix=\hbox{\tt \ \ ax-pre-sup\ \$p\ }
\setbox\contprefix=\hbox{\tt \ \ \ \ \ \ \ \ \ \ \ }
\startm
\m{\vdash}\m{(}\m{(}\m{A}\m{\subseteq}\m{\mathbb{R}}\m{\wedge}\m{A}\m{\ne}\m{%
\varnothing}\m{\wedge}\m{\exists}\m{x}\m{\in}\m{\mathbb{R}}\m{\forall}\m{y}\m{%
\in}\m{A}\m{\,y}\m{<}\m{x}\m{)}\m{\rightarrow}\m{\exists}\m{x}\m{\in}\m{%
\mathbb{R}}\m{(}\m{\forall}\m{y}\m{\in}\m{A}\m{\lnot}\m{x}\m{<}\m{y}\m{\wedge}\m{%
\forall}\m{y}\m{\in}\m{\mathbb{R}}\m{(}\m{y}\m{<}\m{x}\m{\rightarrow}\m{\exists}%
\m{z}\m{\in}\m{A}\m{\,y}\m{<}\m{z}\m{)}\m{)}\m{)}
\endm

% NOTE: The \m{...} expressions above could be represented as
% $ \vdash ( ( A \subseteq \mathbb{R} \wedge A \ne \varnothing \wedge \exists x \in \mathbb{R} \forall y \in A \,y < x ) \rightarrow \exists x \in \mathbb{R} ( \forall y \in A \lnot x < y \wedge \forall y \in \mathbb{R} ( y < x \rightarrow \exists z \in A \,y < z ) ) ) $

\vskip 2ex

This completes the set of axioms for real and complex numbers.  You may
wish to look at how subtraction, division, and decimal numbers
are defined in \texttt{set.mm}, and for fun look at the proof of $2+
2 = 4$ (theorem \texttt{2p2e4} in \texttt{set.mm})
as discussed in section \ref{2p2e4}.

In \texttt{set.mm} we define the non-negative integers $\mathbb{N}$, the integers
$\mathbb{Z}$, and the rationals $\mathbb{Q}$ as subsets of $\mathbb{R}$.  This leads
to the nice inclusion $\mathbb{N} \subseteq \mathbb{Z} \subseteq \mathbb{Q} \subseteq
\mathbb{R} \subseteq \mathbb{C}$, giving us a uniform framework in which, for
example, a property such as commutativity of complex number addition
automatically applies to integers.  The natural numbers $\mathbb{N}$
are different from the set $\omega$ we defined earlier, but both satisfy
Peano's postulates.

\subsection{Complex Number Axioms in Analysis Texts}

Most analysis texts construct complex numbers as ordered pairs of reals,
leading to construction-dependent properties that satisfy these axioms
but are not stated in their pure form.  (This is also done in
\texttt{set.mm} but our axioms are extracted from that construction.)
Other texts will simply state that $\mathbb{R}$ is a ``complete ordered
subfield of $\mathbb{C}$,'' leading to redundant axioms when this phrase
is completely expanded out.  In fact I have not seen a text with the
axioms in the explicit form above.
None of these axioms is unique individually, but this carefully worked out
collection of axioms is the result of years of work
by the Metamath community.

\subsection{Eliminating Unnecessary Complex Number Axioms}

We once had more axioms for real and complex numbers, but over years of time
we (the Metamath community)
have found ways to eliminate them (by proving them from other axioms)
or weaken them (by making weaker claims without reducing what
can be proved).
In particular, here are statements that used to be complex number
axioms but have since been formally proven (with Metamath) to be redundant:

\begin{itemize}
\item
  $\mathbb{C} \in V$.
  At one time this was listed as a ``complex number axiom.''
  However, this is not properly speaking a complex number axiom,
  and in any case its proof uses axioms of set theory.
  Proven redundant by Mario Carneiro\index{Carneiro, Mario} on
  17-Nov-2014 (see \texttt{axcnex}).
\item
  $((A \in \mathbb{C} \land B \in \mathbb{C}$) $\rightarrow$
  $(A + B) = (B + A))$.
  Proved redundant by Eric Schmidt\index{Schmidt, Eric} on 19-Jun-2012,
  and formalized by Scott Fenton\index{Fenton, Scott} on 3-Jan-2013
  (see \texttt{addcom}).
\item
  $(A \in \mathbb{C} \rightarrow (A + 0) = A)$.
  Proved redundant by Eric Schmidt on 19-Jun-2012,
  and formalized by Scott Fenton on 3-Jan-2013
  (see \texttt{addid1}).
\item
  $(A \in \mathbb{C} \rightarrow \exists x \in \mathbb{C} (A + x) = 0)$.
  Proved redundant by Eric Schmidt and formalized on 21-May-2007
  (see \texttt{cnegex}).
\item
  $((A \in \mathbb{C} \land A \ne 0) \rightarrow \exists x \in \mathbb{C} (A \cdot x) = 1)$.
  Proved redundant by Eric Schmidt and formalized on 22-May-2007
  (see \texttt{recex}).
\item
  $0 \in \mathbb{R}$.
  Proved redundant by Eric Schmidt on 19-Feb-2005 and formalized 21-May-2007
  (see \texttt{0re}).
\end{itemize}

We could eliminate 0 as an axiomatic object by defining it as
$( ( i \cdot i ) + 1 )$
and replacing it with this expression throughout the axioms. If this
is done, axiom ax-i2m1 becomes redundant. However, the remaining axioms
would become longer and less intuitive.

Eric Schmidt's paper analyzing this axiom system \cite{Schmidt}
presented a proof that these remaining axioms,
with the possible exception of ax-mulcom, are independent of the others.
It is currently an open question if ax-mulcom is independent of the others.

\section{Two Plus Two Equals Four}\label{2p2e4}

Here is a proof that $2 + 2 = 4$, as proven in the theorem \texttt{2p2e4}
in the database \texttt{set.mm}.
This is a useful demonstration of what a Metamath proof can look like.
This proof may have more steps than you're used to, but each step is rigorously
proven all the way back to the axioms of logic and set theory.
This display was originally generated by the Metamath program
as an {\sc HTML} file.

In the table showing the proof ``Step'' is the sequential step number,
while its associated ``Expression'' is an expression that we have proved.
``Ref'' is the name of a theorem or axiom that justifies that expression,
and ``Hyp'' refers to previous steps (if any) that the theorem or axiom
needs so that we can use it.  Expressions are indented further than
the expressions that depend on them to show their interdependencies.

\begin{table}[!htbp]
\caption{Two plus two equals four}
\begin{tabular}{lllll}
\textbf{Step} & \textbf{Hyp} & \textbf{Ref} & \textbf{Expression} & \\
1  &       & df-2    & $ \; \; \vdash 2 = 1 + 1$  & \\
2  & 1     & oveq2i  & $ \; \vdash (2 + 2) = (2 + (1 + 1))$ & \\
3  &       & df-4    & $ \; \; \vdash 4 = (3 + 1)$ & \\
4  &       & df-3    & $ \; \; \; \vdash 3 = (2 + 1)$ & \\
5  & 4     & oveq1i  & $ \; \; \vdash (3 + 1) = ((2 + 1) + 1)$ & \\
6  &       & 2cn     & $ \; \; \; \vdash 2 \in \mathbb{C}$ & \\
7  &       & ax-1cn  & $ \; \; \; \vdash 1 \in \mathbb{C}$ & \\
8  & 6,7,7 & addassi & $ \; \; \vdash ((2 + 1) + 1) = (2 + (1 + 1))$ & \\
9  & 3,5,8 & 3eqtri  & $ \; \vdash 4 = (2 + (1 + 1))$ & \\
10 & 2,9   & eqtr4i  & $ \vdash (2 + 2) = 4$ & \\
\end{tabular}
\end{table}

Step 1 says that we can assert that $2 = 1 + 1$ because it is
justified by \texttt{df-2}.
What is \texttt{df-2}?
It is simply the definition of $2$, which in our system is defined as being
equal to $1 + 1$.  This shows how we can use definitions in proofs.

Look at Step 2 of the proof. In the Ref column, we see that it references
a previously proved theorem, \texttt{oveq2i}.
It turns out that
theorem \texttt{oveq2i} requires a
hypothesis, and in the Hyp column of Step 2 we indicate that Step 1 will
satisfy (match) this hypothesis.
If we looked at \texttt{oveq2i}
we would find that it proves that given some hypothesis
$A = B$, we can prove that $( C F A ) = ( C F B )$.
If we use \texttt{oveq2i} and apply step 1's result as the hypothesis,
that will mean that $A = 2$ and $B = ( 1 + 1 )$ within this use of
\texttt{oveq2i}.
We are free to select any value of $C$ and $F$ (subject to syntax constraints),
so we are free to select $C = 2$ and $F = +$,
producing our desired result,
$ (2 + 2) = (2 + (1 + 1))$.

Step 2 is an example of substitution.
In the end, every step in every proof uses only this one substitution rule.
All the rules of logic, and all the axioms, are expressed so that
they can be used via this one substitution rule.
So once you master substitution, you can master every Metamath proof,
no exceptions.

Each step is clear and can be immediately checked.
In the {\sc HTML} display you can even click on each reference to see why it is
justified, making it easy to see why the proof works.

\section{Deduction}\label{deduction}

Strictly speaking,
a deduction (also called an inference) is a kind of statement that needs
some hypotheses to be true in order for its conclusion to be true.
A theorem, on the other hand, has no hypotheses.
Informally we often call both of them theorems, but in this section we
will stick to the strict definitions.

It sometimes happens that we have proved a deduction of the form
$\varphi \Rightarrow \psi$\index{$\Rightarrow$}
(given hypothesis $\varphi$ we can prove $\psi$)
and we want to then prove a theorem of the form
$\varphi \rightarrow \psi$.

Converting a deduction (which uses a hypothesis) into a theorem
(which does not) is not as simple as you might think.
The deduction says, ``if we can prove $\varphi$ then we can prove $\psi$,''
which is in some sense weaker than saying
``$\varphi$ implies $\psi$.''
There is no axiom of logic that permits us to directly obtain the theorem
given the deduction.\footnote{
The conversion of a deduction to a theorem does not even hold in general
for quantum propositional calculus,
which is a weak subset of classical propositional calculus.
It has been shown that adding the Standard Deduction Theorem (discussed below)
to quantum propositional calculus turns it into classical
propositional calculus!
}

This is in contrast to going the other way.
If we have the theorem ($\varphi \rightarrow \psi$),
it is easy to recover the deduction
($\varphi \Rightarrow \psi$)
using modus ponens\index{modus ponens}
(\texttt{ax-mp}; see section \ref{axmp}).

In the following subsections we first discuss the standard deduction theorem
(the traditional but awkward way to convert deductions into theorems) and
the weak deduction theorem (a limited version of the standard deduction
theorem that is easier to use and was once widely used in
\texttt{set.mm}\index{set theory database (\texttt{set.mm})}).
In section \ref{deductionstyle} we discuss
deduction style, the newer approach we now recommend in most cases.
Deduction style uses ``deduction form,'' a form that
prefixes each hypothesis (other than definitions) and the
conclusion with a universal antecedent (``$\varphi \rightarrow$'').
Deduction style is widely used in \texttt{set.mm},
so it is useful to understand it and \textit{why} it is widely used.
Section \ref{naturaldeduction}
briefly discusses our approach for using natural deduction
within \texttt{set.mm},
as that approach is deeply related to deduction style.
We conclude with a summary of the strengths of
our approach, which we believe are compelling.

\subsection{The Standard Deduction Theorem}\label{standarddeductiontheorem}

It is possible to make use of information
contained in the deduction or its proof to assist us with the proof of
the related theorem.
In traditional logic books, there is a metatheorem called the
Deduction Theorem\index{Deduction Theorem}\index{Standard Deduction Theorem},
discovered independently by Herbrand and Tarski around 1930.
The Deduction Theorem, which we often call the Standard Deduction Theorem,
provides an algorithm for constructing a proof of a theorem from the
proof of its corresponding deduction. See, for example,
\cite[p.~56]{Margaris}\index{Margaris, Angelo}.
To construct a proof for a theorem, the
algorithm looks at each step in the proof of the original deduction and
rewrites the step with several steps wherein the hypothesis is eliminated
and becomes an antecedent.

In ordinary mathematics, no one actually carries out the algorithm,
because (in its most basic form) it involves an exponential explosion of
the number of proof steps as more hypotheses are eliminated. Instead,
the Standard Deduction Theorem is invoked simply to claim that it can
be done in principle, without actually doing it.
What's more, the algorithm is not as simple as it might first appear
when applying it rigorously.
There is a subtle restriction on the Standard Deduction Theorem
that must be taken into account involving the axiom of generalization
when working with predicate calculus (see the literature for more detail).

One of the goals of Metamath is to let you plainly see, with as few
underlying concepts as possible, how mathematics can be derived directly
from the axioms, and not indirectly according to some hidden rules
buried inside a program or understood only by logicians. If we added
the Standard Deduction Theorem to the language and proof verifier,
that would greatly complicate both and largely defeat Metamath's goal
of simplicity. In principle, we could show direct proofs by expanding
out the proof steps generated by the algorithm of the Standard Deduction
Theorem, but that is not feasible in practice because the number of proof
steps quickly becomes huge, even astronomical.
Since the algorithm of the Standard Deduction Theorem is driven by the proof,
we would have to go through that proof
all over again---starting from axioms---in order to obtain the theorem form.
In terms of proof length, there would be no savings over just
proving the theorem directly instead of first proving the deduction form.

\subsection{Weak Deduction Theorem}\label{weakdeductiontheorem}

We have developed
a more efficient method for proving a theorem from a deduction
that can be used instead of the Standard Deduction Theorem
in many (but not all) cases.
We call this more efficient method the
Weak Deduction Theorem\index{Weak Deduction Theorem}.\footnote{
There is also an unrelated ``Weak Deduction Theorem''
in the field of relevance logic, so to avoid confusion we could call
ours the ``Weak Deduction Theorem for Classical Logic.''}
Unlike the Standard Deduction Theorem, the Weak Deduction Theorem produces the
theorem directly from a special substitution instance of the deduction,
using a small, fixed number of steps roughly proportional to the length
of the final theorem.

If you come to a proof referencing the Weak Deduction Theorem
\texttt{dedth} (or one of its variants \texttt{dedthxx}),
here is how to follow the proof without getting into the details:
just click on the theorem referenced in the step
just before the reference to \texttt{dedth} and ignore everything else.
Theorem \texttt{dedth} simply turns a hypothesis into an antecedent
(i.e. the hypothesis followed by $\rightarrow$
is placed in front of the assertion, and the hypothesis
itself is eliminated) given certain conditions.

The Weak Deduction Theorem
eliminates a hypothesis $\varphi$, making it become an antecedent.
It does this by proving an expression
$ \varphi \rightarrow \psi $ given two hypotheses:
(1)
$ ( A = {\rm if} ( \varphi , A , B ) \rightarrow ( \varphi \leftrightarrow \chi ) ) $
and
(2) $\chi$.
Note that it requires that a proof exists for $\varphi$ when the class variable
$A$ is replaced with a specific class $B$. The hypothesis $\chi$
should be assigned to the inference.
You can see the details of the proof of the Weak Deduction Theorem
in theorem \texttt{dedth}.

The Weak Deduction Theorem
is probably easier to understand by studying proofs that make use of it.
For example, let's look at the proof of \texttt{renegcl}, which proves that
$ \vdash ( A \in \mathbb{R} \rightarrow - A \in \mathbb{R} )$:

\needspace{4\baselineskip}
\begin{longtabu} {l l l X}
\textbf{Step} & \textbf{Hyp} & \textbf{Ref} & \textbf{Expression} \\
  1 &  & negeq &
  $\vdash$ $($ $A$ $=$ ${\rm if}$ $($ $A$ $\in$ $\mathbb{R}$ $,$ $A$ $,$ $1$ $)$ $\rightarrow$
  $\textrm{-}$ $A$ $=$ $\textrm{-}$ ${\rm if}$ $($ $A$ $\in$ $\mathbb{R}$
  $,$ $A$ $,$ $1$ $)$ $)$ \\
 2 & 1 & eleq1d &
    $\vdash$ $($ $A$ $=$ ${\rm if}$ $($ $A$ $\in$ $\mathbb{R}$ $,$ $A$ $,$ $1$ $)$ $\rightarrow$ $($
    $\textrm{-}$ $A$ $\in$ $\mathbb{R}$ $\leftrightarrow$
    $\textrm{-}$ ${\rm if}$ $($ $A$ $\in$ $\mathbb{R}$ $,$ $A$ $,$ $1$ $)$ $\in$
    $\mathbb{R}$ $)$ $)$ \\
 3 &  & 1re & $\vdash 1 \in \mathbb{R}$ \\
 4 & 3 & elimel &
   $\vdash {\rm if} ( A \in \mathbb{R} , A , 1 ) \in \mathbb{R}$ \\
 5 & 4 & renegcli &
   $\vdash \textrm{-} {\rm if} ( A \in \mathbb{R} , A , 1 ) \in \mathbb{R}$ \\
 6 & 2,5 & dedth &
   $\vdash ( A \in \mathbb{R} \rightarrow \textrm{-} A \in \mathbb{R}$ ) \\
\end{longtabu}

The somewhat strange-looking steps in \texttt{renegcl} before step 5 are
technical stuff that makes this magic work, and they can be ignored
for a quick overview of the proof. To continue following the ``important''
part of the proof of \texttt{renegcl},
you can look at the reference to \texttt{renegcli} at step 5.

That said, let's briefly look at how
\texttt{renegcl} uses the
Weak Deduction Theorem (\texttt{dedth}) to do its job,
in case you want to do something similar or want understand it more deeply.
Let's work backwards in the proof of \texttt{renegcl}.
Step 6 applies \texttt{dedth} to produce our goal result
$ \vdash ( A \in \mathbb{R} \rightarrow\, - A \in \mathbb{R} )$.
This requires on the one hand the (substituted) deduction
\texttt{renegcli} in step 5.
By itself \texttt{renegcli} proves the deduction
$ \vdash A \in \mathbb{R} \Rightarrow\, \vdash - A \in \mathbb{R}$;
this is the deduction form we are trying to turn into theorem form,
and thus
\texttt{renegcli} has a separate hypothesis that must be fulfilled.
To fulfill the hypothesis of the invocation of
\texttt{renegcli} in step 5, it is eventually
reduced to the already proven theorem $1 \in \mathbb{R}$ in step 3.
Step 4 connects steps 3 and 5; step 4 invokes
\texttt{elimel}, a special case of \texttt{elimhyp} that eliminates
a membership hypothesis for the weak deduction theorem.
On the other hand, the equivalence of the conclusion of
\texttt{renegcl}
$( - A \in \mathbb{R} )$ and the substituted conclusion of
\texttt{renegcli} must be proven, which is done in steps 2 and 1.

The weak deduction theorem has limitations.
In particular, we must be able to prove a special case of the deduction's
hypothesis as a stand-alone theorem.
For example, we used $1 \in \mathbb{R}$ in step 3 of \texttt{renegcl}.

We used to use the weak deduction theorem
extensively within \texttt{set.mm}.
However, we now recommend applying ``deduction style''
instead in most cases, as deduction style is
often an easier and clearer approach.
Therefore, we will now describe deduction style.

\subsection{Deduction Style}\label{deductionstyle}

We now prefer to write assertions in ``deduction form''
instead of writing a proof that would require use of the standard or
weak deduction theorem.
We call this appraoch
``deduction style.''\index{deduction style}

It will be easier to explain this by first defining some terms:

\begin{itemize}
\item \textbf{closed form}\index{closed form}\index{forms!closed}:
A kind of assertion (theorem) with no hypotheses.
Typically its label has no special suffix.
An example is \texttt{unss}, which states:
$\vdash ( ( A \subseteq C \wedge B \subseteq C ) \leftrightarrow ( A \cup B )
\subseteq C )\label{eq:unss}$
\item \textbf{deduction form}\index{deduction form}\index{forms!deduction}:
A kind of assertion with one or more hypotheses
where the conclusion is an implication with
a wff variable as the antecedent (usually $\varphi$), and every hypothesis
(\$e statement)
is either (1) an implication with the same antecedent as the conclusion or
(2) a definition.
A definition
can be for a class variable (this is a class variable followed by ``='')
or a wff variable (this is a wff variable followed by $\leftrightarrow$);
class variable definitions are more common.
In practice, a proof
in deduction form will also contain many steps that are implications
where the antecedent is either that wff variable (normally $\varphi$)
or is
a conjunction (...$\land$...) including that wff variable ($\varphi$).
If an assertion is in deduction form, and other forms are also available,
then we suffix its label with ``d.''
An example is \texttt{unssd}, which states\footnote{
For brevity we show here (and in other places)
a $\&$\index{$\&$} between hypotheses\index{hypotheses}
and a $\Rightarrow$\index{$\Rightarrow$}\index{conclusion}
between the hypotheses and the conclusion.
This notation is technically not part of the Metamath language, but is
instead a convenient abbreviation to show both the hypotheses and conclusion.}:
$\vdash ( \varphi \rightarrow A \subseteq C )\quad\&\quad \vdash ( \varphi
    \rightarrow B \subseteq C )\quad\Rightarrow\quad \vdash ( \varphi
    \rightarrow ( A \cup B ) \subseteq C )\label{eq:unssd}$
\item \textbf{inference form}\index{inference form}\index{forms!inference}:
A kind of assertion with one or more hypotheses that is not in deduction form
(e.g., there is no common antecedent).
If an assertion is in inference form, and other forms are also available,
then we suffix its label with ``i.''
An example is \texttt{unssi}, which states:
$\vdash A \subseteq C\quad\&\quad \vdash B \subseteq C\quad\Rightarrow\quad
    \vdash ( A \cup B ) \subseteq C\label{eq:unssi}$
\end{itemize}

When using deduction style we express an assertion in deduction form.
This form prefixes each hypothesis (other than definitions) and the
conclusion with a universal antecedent (``$\varphi \rightarrow$'').
The antecedent (e.g., $\varphi$)
mimics the context handled in the deduction theorem, eliminating
the need to directly use the deduction theorem.

Once you have an assertion in deduction form, you can easily convert it
to inference form or closed form:

\begin{itemize}
\item To
prove some assertion Ti in inference form, given assertion Td in deduction
form, there is a simple mechanical process you can use. First take each
Ti hypothesis and insert a \texttt{T.} $\rightarrow$ prefix (``true implies'')
using \texttt{a1i}. You
can then use the existing assertion Td to prove the resulting conclusion
with a \texttt{T.} $\rightarrow$ prefix.
Finally, you can remove that prefix using \texttt{mptru},
resulting in the conclusion you wanted to prove.
\item To
prove some assertion T in closed form, given assertion Td in deduction
form, there is another simple mechanical process you can use. First,
select an expression that is the conjunction (...$\land$...) of all of the
consequents of every hypothesis of Td. Next, prove that this expression
implies each of the separate hypotheses of Td in turn by eliminating
conjuncts (there are a variety of proven assertions to do this, including
\texttt{simpl},
\texttt{simpr},
\texttt{3simpa},
\texttt{3simpb},
\texttt{3simpc},
\texttt{simp1},
\texttt{simp2},
and
\texttt{simp3}).
If the
expression has nested conjunctions, inner conjuncts can be broken out by
chaining the above theorems with \texttt{syl}
(see section \ref{syl}).\footnote{
There are actually many theorems
(labeled simp* such as \texttt{simp333}) that break out inner conjuncts in one
step, but rather than learning them you can just use the chaining we
just described to prove them, and then let the Metamath program command
\texttt{minimize{\char`\_}with}\index{\texttt{minimize{\char`\_}with} command}
figure out the right ones needed to collapse them.}
As your final step, you can then apply the already-proven assertion Td
(which is in deduction form), proving assertion T in closed form.
\end{itemize}

We can also easily convert any assertion T in closed form to its related
assertion Ti in inference form by applying
modus ponens\index{modus ponens} (see section \ref{axmp}).

The deduction form antecedent can also be used to represent the context
necessary to support natural deduction systems, so we will now
discuss natural deduction.

\subsection{Natural Deduction}\label{naturaldeduction}

Natural deduction\index{natural deduction}
(ND) systems, as such, were originally introduced in
1934 by two logicians working independently: Ja\'skowski and Gentzen. ND
systems are supposed to reconstruct, in a formally proper way, traditional
ways of mathematical reasoning (such as conditional proof, indirect proof,
and proof by cases). As reconstructions they were naturally influenced
by previous work, and many specific ND systems and notations have been
developed since their original work.

There are many ND variants, but
Indrzejczak \cite[p.~31-32]{Indrzejczak}\index{Indrzejczak, Andrzej}
suggests that any natural deductive system must satisfy at
least these three criteria:

\begin{itemize}
\item ``There are some means for entering assumptions into a proof and
also for eliminating them. Usually it requires some bookkeeping devices
for indicating the scope of an assumption, and showing that a part of
a proof depending on eliminated assumption is discharged.
\item There are no (or, at least, a very limited set of) axioms, because
their role is taken over by the set of primitive rules for introduction
and elimination of logical constants which means that elementary
inferences instead of formulae are taken as primitive.
\item (A genuine) ND system admits a lot of freedom in proof construction
and possibility of applying several proof search strategies, like
conditional proof, proof by cases, proof by reductio ad absurdum etc.''
\end{itemize}

The Metamath Proof Explorer (MPE) as defined in \texttt{set.mm}
is fundamentally a Hilbert-style system.
That is, MPE is based on a larger number of axioms (compared
to natural deduction systems), a very small set of rules of inference
(modus ponens), and the context is not changed by the rules of inference
in the middle of a proof. That said, MPE proofs can be developed using
the natural deduction (ND) approach as originally developed by Ja\'skowski
and Gentzen.

The most common and recommended approach for applying ND in MPE is to use
deduction form\index{deduction form}%
\index{forms!deduction}
and apply the MPE proven assertions that are equivalent to ND rules.
For example, MPE's \texttt{jca} is equivalent to ND rule $\land$-I
(and-insertion).
We maintain a list of equivalences that you may consult.
This approach for applying an ND approach within MPE relies on Metamath's
wff metavariables in an essential way, and is described in more detail
in the presentation ``Natural Deductions in the Metamath Proof Language''
by Mario Carneiro \cite{CarneiroND}\index{Carneiro, Mario}.

In this style many steps are an implication, whose antecedent mimics
the context ($\Gamma$) of most ND systems. To add an assumption, simply add
it to the implication antecedent (typically using
\texttt{simpr}),
and use that
new antecedent for all later claims in the same scope. If you wish to
use an assertion in an ND hypothesis scope that is outside the current
ND hypothesis scope, modify the assertion so that the ND hypothesis
assumption is added to its antecedent (typically using \texttt{adantr}). Most
proof steps will be proved using rules that have hypotheses and results
of the form $\varphi \rightarrow$ ...

An example may make this clearer.
Let's show theorem 5.5 of
\cite[p.~18]{Clemente}\index{Clemente Laboreo, Daniel}
along with a line by line translation using the usual
translation of natural deduction (ND) in the Metamath Proof Explorer
(MPE) notation (this is proof \texttt{ex-natded5.5}).
The proof's original goal was to prove
$\lnot \psi$ given two hypotheses,
$( \psi \rightarrow \chi )$ and $ \lnot \chi$.
We will translate these statements into MPE deduction form
by prefixing them all with $\varphi \rightarrow$.
As a result, in MPE the goal is stated as
$( \varphi \rightarrow \lnot \psi )$, and the two hypotheses are stated as
$( \varphi \rightarrow ( \psi \rightarrow \chi ) )$ and
$( \varphi \rightarrow \lnot \chi )$.

The following table shows the proof in Fitch natural deduction style
and its MPE equivalent.
The \textit{\#} column shows the original numbering,
\textit{MPE\#} shows the number in the equivalent MPE proof
(which we will show later),
\textit{ND Expression} shows the original proof claim in ND notation,
and \textit{MPE Translation} shows its translation into MPE
as discussed in this section.
The final columns show the rationale in ND and MPE respectively.

\needspace{4\baselineskip}
{\setlength{\extrarowsep}{4pt} % Keep rows from being too close together
\begin{longtabu}   { @{} c c X X X X }
\textbf{\#} & \textbf{MPE\#} & \textbf{ND Ex\-pres\-sion} &
\textbf{MPE Trans\-lation} & \textbf{ND Ration\-ale} &
\textbf{MPE Ra\-tio\-nale} \\
\endhead

1 & 2;3 &
$( \psi \rightarrow \chi )$ &
$( \varphi \rightarrow ( \psi \rightarrow \chi ) )$ &
Given &
\$e; \texttt{adantr} to put in ND hypothesis \\

2 & 5 &
$ \lnot \chi$ &
$( \varphi \rightarrow \lnot \chi )$ &
Given &
\$e; \texttt{adantr} to put in ND hypothesis \\

3 & 1 &
... $\vert$ $\psi$ &
$( \varphi \rightarrow \psi )$ &
ND hypothesis assumption &
\texttt{simpr} \\

4 & 4 &
... $\chi$ &
$( ( \varphi \land \psi ) \rightarrow \chi )$ &
$\rightarrow$\,E 1,3 &
\texttt{mpd} 1,3 \\

5 & 6 &
... $\lnot \chi$ &
$( ( \varphi \land \psi ) \rightarrow \lnot \chi )$ &
IT 2 &
\texttt{adantr} 5 \\

6 & 7 &
$\lnot \psi$ &
$( \varphi \rightarrow \lnot \psi )$ &
$\land$\,I 3,4,5 &
\texttt{pm2.65da} 4,6 \\

\end{longtabu}
}


The original used Latin letters; we have replaced them with Greek letters
to follow Metamath naming conventions and so that it is easier to follow
the Metamath translation. The Metamath line-for-line translation of
this natural deduction approach precedes every line with an antecedent
including $\varphi$ and uses the Metamath equivalents of the natural deduction
rules. To add an assumption, the antecedent is modified to include it
(typically by using \texttt{adantr};
\texttt{simpr} is useful when you want to
depend directly on the new assumption, as is shown here).

In Metamath we can represent the two given statements as these hypotheses:

\needspace{2\baselineskip}
\begin{itemize}
\item ex-natded5.5.1 $\vdash ( \varphi \rightarrow ( \psi \rightarrow \chi ) )$
\item ex-natded5.5.2 $\vdash ( \varphi \rightarrow \lnot \chi )$
\end{itemize}

\needspace{4\baselineskip}
Here is the proof in Metamath as a line-by-line translation:

\begin{longtabu}   { l l l X }
\textbf{Step} & \textbf{Hyp} & \textbf{Ref} & \textbf{Ex\-pres\-sion} \\
\endhead
1 & & simpr & $\vdash ( ( \varphi \land \psi ) \rightarrow \psi )$ \\
2 & & ex-natded5.5.1 &
  $\vdash ( \varphi \rightarrow ( \psi \rightarrow \chi ) )$ \\
3 & 2 & adantr &
 $\vdash ( ( \varphi \land \psi ) \rightarrow ( \psi \rightarrow \chi ) )$ \\
4 & 1, 3 & mpd &
 $\vdash ( ( \varphi \land \psi ) \rightarrow \chi ) $ \\
5 & & ex-natded5.5.2 &
 $\vdash ( \varphi \rightarrow \lnot \chi )$ \\
6 & 5 & adantr &
 $\vdash ( ( \varphi \land \psi ) \rightarrow \lnot \chi )$ \\
7 & 4, 6 & pm2.65da &
 $\vdash ( \varphi \rightarrow \lnot \psi )$ \\
\end{longtabu}

Only using specific natural deduction rules directly can lead to very
long proofs, for exactly the same reason that only using axioms directly
in Hilbert-style proofs can lead to very long proofs.
If the goal is short and clear proofs,
then it is better to reuse already-proven assertions
in deduction form than to start from scratch each time
and using only basic natural deduction rules.

\subsection{Strengths of Our Approach}

As far as we know there is nothing else in the literature like either the
weak deduction theorem or Mario Carneiro\index{Carneiro, Mario}'s
natural deduction method.
In order to
transform a hypothesis into an antecedent, the literature's standard
``Deduction Theorem''\index{Deduction Theorem}\index{Standard Deduction Theorem}
requires metalogic outside of the notions provided
by the axiom system. We instead generally prefer to use Mario Carneiro's
natural deduction method, then use the weak deduction theorem in cases
where that is difficult to apply, and only then use the full standard
deduction theorem as a last resort.

The weak deduction theorem\index{Weak Deduction Theorem}
does not require any additional metalogic
but converts an inference directly into a closed form theorem, with
a rigorous proof that uses only the axiom system. Unlike the standard
Deduction Theorem, there is no implicit external justification that we
have to trust in order to use it.

Mario Carneiro's natural deduction\index{natural deduction}
method also does not require any new metalogical
notions. It avoids the Deduction Theorem's metalogic by prefixing the
hypotheses and conclusion of every would-be inference with a universal
antecedent (``$\varphi \rightarrow$'') from the very start.

We think it is impressive and satisfying that we can do so much in a
practical sense without stepping outside of our Hilbert-style axiom system.
Of course our axiomatization, which is in the form of schemes,
contains a metalogic of its own that we exploit. But this metalogic
is relatively simple, and for our Deduction Theorem alternatives,
we primarily use just the direct substitution of expressions for
metavariables.

\begin{sloppy}
\section{Exploring the Set The\-o\-ry Data\-base}\label{exploring}
\end{sloppy}
% NOTE: All examples performed in this section are
% recorded wtih "set width 61" % on set.mm as of 2019-05-28
% commit c1e7849557661260f77cfdf0f97ac4354fbb4f4d.

At this point you may wish to study the \texttt{set.mm}\index{set theory
database (\texttt{set.mm})} file in more detail.  Pay particular
attention to the assumptions needed to define wffs\index{well-formed
formula (wff)} (which are not included above), the variable types
(\texttt{\$f}\index{\texttt{\$f} statement} statements), and the
definitions that are introduced.  Start with some simple theorems in
propositional calculus, making sure you understand in detail each step
of a proof.  Once you get past the first few proofs and become familiar
with the Metamath language, any part of the \texttt{set.mm} database
will be as easy to follow, step by step, as any other part---you won't
have to undergo a ``quantum leap'' in mathematical sophistication to be
able to follow a deep proof in set theory.

Next, you may want to explore how concepts such as natural numbers are
defined and described.  This is probably best done in conjunction with
standard set theory textbooks, which can help give you a higher-level
understanding.  The \texttt{set.mm} database provides references that will get
you started.  From there, you will be on your way towards a very deep,
rigorous understanding of abstract mathematics.

The Metamath\index{Metamath} program can help you peruse a Metamath data\-base,
wheth\-er you are trying to figure out how a certain step follows in a proof or
just have a general curiosity.  We will go through some examples of the
commands, using the \texttt{set.mm}\index{set theory database (\texttt{set.mm})}
database provided with the Metamath software.  These should help get you
started.  See Chapter~\ref{commands} for a more detailed description of
the commands.  Note that we have included the full spelling of all commands to
prevent ambiguity with future commands.  In practice you may type just the
characters needed to specify each command keyword\index{command keyword}
unambiguously, often just one or two characters per keyword, and you don't
need to type them in upper case.

First run the Metamath program as described earlier.  You should see the
\verb/MM>/ prompt.  Read in the \texttt{set.mm} file:\index{\texttt{read}
command}

\begin{verbatim}
MM> read set.mm
Reading source file "set.mm"... 34554442 bytes
34554442 bytes were read into the source buffer.
The source has 155711 statements; 2254 are $a and 32250 are $p.
No errors were found.  However, proofs were not checked.
Type VERIFY PROOF * if you want to check them.
\end{verbatim}

As with most examples in this book, what you will see
will be slightly different because we are continuously
improving our databases (including \texttt{set.mm}).

Let's check the database integrity.  This may take a minute or two to run if
your computer is slow.

\begin{verbatim}
MM> verify proof *
0 10%  20%  30%  40%  50%  60%  70%  80%  90% 100%
..................................................
All proofs in the database were verified in 2.84 s.
\end{verbatim}

No errors were reported, so every proof is correct.

You need to know the names (labels) of theorems before you can look at them.
Often just examining the database file(s) with a text editor is the best
approach.  In \texttt{set.mm} there are many detailed comments, especially near
the beginning, that can help guide you. The \texttt{search} command in the
Metamath program is also handy.  The \texttt{comments} qualifier will list the
statements whose associated comment (the one immediately before it) contain a
string you give it.  For example, if you are studying Enderton's {\em Elements
of Set Theory} \cite{Enderton}\index{Enderton, Herbert B.} you may want to see
the references to it in the database.  The search string \texttt{enderton} is not
case sensitive.  (This will not show you all the database theorems that are in
Enderton's book because there is usually only one citation for a given
theorem, which may appear in several textbooks.)\index{\texttt{search}
command}

\begin{verbatim}
MM> search * "enderton" / comments
12067 unineq $p "... Exercise 20 of [Enderton] p. 32 and ..."
12459 undif2 $p "...Corollary 6K of [Enderton] p. 144. (C..."
12953 df-tp $a "...s. Definition of [Enderton] p. 19. (Co..."
13689 unissb $p ".... Exercise 5 of [Enderton] p. 26 and ..."
\end{verbatim}
\begin{center}
(etc.)
\end{center}

Or you may want to see what theorems have something to do with
conjunction (logical {\sc and}).  The quotes around the search
string are optional when there's no ambiguity.\index{\texttt{search}
command}

\begin{verbatim}
MM> search * conjunction / comments
120 a1d $p "...be replaced with a conjunction ( ~ df-an )..."
662 df-bi $a "...viated form after conjunction is introdu..."
1319 wa $a "...ff definition to include conjunction ('and')."
1321 df-an $a "Define conjunction (logical 'and'). Defini..."
1420 imnan $p "...tion in terms of conjunction. (Contribu..."
\end{verbatim}
\begin{center}
(etc.)
\end{center}


Now we will start to look at some details.  Let's look at the first
axiom of propositional calculus
(we could use \texttt{sh st} to abbreviate
\texttt{show statement}).\index{\texttt{show statement} command}

\begin{verbatim}
MM> show statement ax-1/full
Statement 19 is located on line 881 of the file "set.mm".
"Axiom _Simp_.  Axiom A1 of [Margaris] p. 49.  One of the 3
axioms of propositional calculus.  The 3 axioms are also
        ...
19 ax-1 $a |- ( ph -> ( ps -> ph ) ) $.
Its mandatory hypotheses in RPN order are:
  wph $f wff ph $.
  wps $f wff ps $.
The statement and its hypotheses require the variables:  ph
      ps
The variables it contains are:  ph ps


Statement 49 is located on line 11182 of the file "set.mm".
Its statement number for HTML pages is 6.
"Axiom _Simp_.  Axiom A1 of [Margaris] p. 49.  One of the 3
axioms of propositional calculus.  The 3 axioms are also
given as Definition 2.1 of [Hamilton] p. 28.
...
49 ax-1 $a |- ( ph -> ( ps -> ph ) ) $.
Its mandatory hypotheses in RPN order are:
  wph $f wff ph $.
  wps $f wff ps $.
The statement and its hypotheses require the variables:
  ph ps
The variables it contains are:  ph ps
\end{verbatim}

Compare this to \texttt{ax-1} on p.~\pageref{ax1}.  You can see that the
symbol \texttt{ph} is the {\sc ascii} notation for $\varphi$, etc.  To
see the mathematical symbols for any expression you may typeset it in
\LaTeX\ (type \texttt{help tex} for instructions)\index{latex@{\LaTeX}}
or, easier, just use a text editor to look at the comments where symbols
are first introduced in \texttt{set.mm}.  The hypotheses \texttt{wph}
and \texttt{wps} required by \texttt{ax-1} mean that variables
\texttt{ph} and \texttt{ps} must be wffs.

Next we'll pick a simple theorem of propositional calculus, the Principle of
Identity, which is proved directly from the axioms.  We'll look at the
statement then its proof.\index{\texttt{show statement}
command}

\begin{verbatim}
MM> show statement id1/full
Statement 116 is located on line 11371 of the file "set.mm".
Its statement number for HTML pages is 22.
"Principle of identity.  Theorem *2.08 of [WhiteheadRussell]
p. 101.  This version is proved directly from the axioms for
demonstration purposes.
...
116 id1 $p |- ( ph -> ph ) $= ... $.
Its mandatory hypotheses in RPN order are:
  wph $f wff ph $.
Its optional hypotheses are:  wps wch wth wta wet
      wze wsi wrh wmu wla wka
The statement and its hypotheses require the variables:  ph
These additional variables are allowed in its proof:
      ps ch th ta et ze si rh mu la ka
The variables it contains are:  ph
\end{verbatim}

The optional variables\index{optional variable} \texttt{ps}, \texttt{ch}, etc.\ are
available for use in a proof of this statement if we wish, and were we to do
so we would make use of optional hypotheses \texttt{wps}, \texttt{wch}, etc.  (See
Section~\ref{dollaref} for the meaning of ``optional
hypothesis.''\index{optional hypothesis}) The reason these show up in the
statement display is that statement \texttt{id1} happens to be in their scope
(see Section~\ref{scoping} for the definition of ``scope''\index{scope}), but
in fact in propositional calculus we will never make use of optional
hypotheses or variables.  This becomes important after quantifiers are
introduced, where ``dummy'' variables\index{dummy variable} are often needed
in the middle of a proof.

Let's look at the proof of statement \texttt{id1}.  We'll use the
\texttt{show proof} command, which by default suppresses the
``non-essential'' steps that construct the wffs.\index{\texttt{show proof}
command}
We will display the proof in ``lemmon' format (a non-indented format
with explicit previous step number references) and renumber the
displayed steps:

\begin{verbatim}
MM> show proof id1 /lemmon/renumber
1 ax-1           $a |- ( ph -> ( ph -> ph ) )
2 ax-1           $a |- ( ph -> ( ( ph -> ph ) -> ph ) )
3 ax-2           $a |- ( ( ph -> ( ( ph -> ph ) -> ph ) ) ->
                     ( ( ph -> ( ph -> ph ) ) -> ( ph -> ph )
                                                          ) )
4 2,3 ax-mp      $a |- ( ( ph -> ( ph -> ph ) ) -> ( ph -> ph
                                                          ) )
5 1,4 ax-mp      $a |- ( ph -> ph )
\end{verbatim}

If you have read Section~\ref{trialrun}, you'll know how to interpret this
proof.  Step~2, for example, is an application of axiom \texttt{ax-1}.  This
proof is identical to the one in Hamilton's {\em Logic for Mathematicians}
\cite[p.~32]{Hamilton}\index{Hamilton, Alan G.}.

You may want to look at what
substitutions\index{substitution!variable}\index{variable substitution} are
made into \texttt{ax-1} to arrive at step~2.  The command to do this needs to
know the ``real'' step number, so we'll display the proof again without
the \texttt{renumber} qualifier.\index{\texttt{show proof}
command}

\begin{verbatim}
MM> show proof id1 /lemmon
 9 ax-1          $a |- ( ph -> ( ph -> ph ) )
20 ax-1          $a |- ( ph -> ( ( ph -> ph ) -> ph ) )
24 ax-2          $a |- ( ( ph -> ( ( ph -> ph ) -> ph ) ) ->
                     ( ( ph -> ( ph -> ph ) ) -> ( ph -> ph )
                                                          ) )
25 20,24 ax-mp   $a |- ( ( ph -> ( ph -> ph ) ) -> ( ph -> ph
                                                          ) )
26 9,25 ax-mp    $a |- ( ph -> ph )
\end{verbatim}

The ``real'' step number is 20.  Let's look at its details.

\begin{verbatim}
MM> show proof id1 /detailed_step 20
Proof step 20:  min=ax-1 $a |- ( ph -> ( ( ph -> ph ) -> ph )
  )
This step assigns source "ax-1" ($a) to target "min" ($e).
The source assertion requires the hypotheses "wph" ($f, step
18) and "wps" ($f, step 19).  The parent assertion of the
target hypothesis is "ax-mp" ($a, step 25).
The source assertion before substitution was:
    ax-1 $a |- ( ph -> ( ps -> ph ) )
The following substitutions were made to the source
assertion:
    Variable  Substituted with
     ph        ph
     ps        ( ph -> ph )
The target hypothesis before substitution was:
    min $e |- ph
The following substitution was made to the target hypothesis:
    Variable  Substituted with
     ph        ( ph -> ( ( ph -> ph ) -> ph ) )
\end{verbatim}

This shows the substitutions\index{substitution!variable}\index{variable
substitution} made to the variables in \texttt{ax-1}.  References are made to
steps 18 and 19 which are not shown in our proof display.  To see these steps,
you can display the proof with the \texttt{all} qualifier.

Let's look at a slightly more advanced proof of propositional calculus.  Note
that \verb+/\+ is the symbol for $\wedge$ (logical {\sc and}, also
called conjunction).\index{conjunction ($\wedge$)}
\index{logical {\sc and} ($\wedge$)}

\begin{verbatim}
MM> show statement prth/full
Statement 1791 is located on line 15503 of the file "set.mm".
Its statement number for HTML pages is 559.
"Conjoin antecedents and consequents of two premises.  This
is the closed theorem form of ~ anim12d .  Theorem *3.47 of
[WhiteheadRussell] p. 113.  It was proved by Leibniz,
and it evidently pleased him enough to call it
_praeclarum theorema_ (splendid theorem).
...
1791 prth $p |- ( ( ( ph -> ps ) /\ ( ch -> th ) ) -> ( ( ph
      /\ ch ) -> ( ps /\ th ) ) ) $= ... $.
Its mandatory hypotheses in RPN order are:
  wph $f wff ph $.
  wps $f wff ps $.
  wch $f wff ch $.
  wth $f wff th $.
Its optional hypotheses are:  wta wet wze wsi wrh wmu wla wka
The statement and its hypotheses require the variables:  ph
      ps ch th
These additional variables are allowed in its proof:  ta et
      ze si rh mu la ka
The variables it contains are:  ph ps ch th


MM> show proof prth /lemmon/renumber
1 simpl          $p |- ( ( ( ph -> ps ) /\ ( ch -> th ) ) ->
                                               ( ph -> ps ) )
2 simpr          $p |- ( ( ( ph -> ps ) /\ ( ch -> th ) ) ->
                                               ( ch -> th ) )
3 1,2 anim12d    $p |- ( ( ( ph -> ps ) /\ ( ch -> th ) ) ->
                           ( ( ph /\ ch ) -> ( ps /\ th ) ) )
\end{verbatim}

There are references to a lot of unfamiliar statements.  To see what they are,
you may type the following:

\begin{verbatim}
MM> show proof prth /statement_summary
Summary of statements used in the proof of "prth":

Statement simpl is located on line 14748 of the file
"set.mm".
"Elimination of a conjunct.  Theorem *3.26 (Simp) of
[WhiteheadRussell] p. 112. ..."
  simpl $p |- ( ( ph /\ ps ) -> ph ) $= ... $.

Statement simpr is located on line 14777 of the file
"set.mm".
"Elimination of a conjunct.  Theorem *3.27 (Simp) of
[WhiteheadRussell] ..."
  simpr $p |- ( ( ph /\ ps ) -> ps ) $= ... $.

Statement anim12d is located on line 15445 of the file
"set.mm".
"Conjoin antecedents and consequents in a deduction.
..."
  anim12d.1 $e |- ( ph -> ( ps -> ch ) ) $.
  anim12d.2 $e |- ( ph -> ( th -> ta ) ) $.
  anim12d $p |- ( ph -> ( ( ps /\ th ) -> ( ch /\ ta ) ) )
      $= ... $.
\end{verbatim}
\begin{center}
(etc.)
\end{center}

Of course you can look at each of these statements and their proofs, and
so on, back to the axioms of propositional calculus if you wish.

The \texttt{search} command is useful for finding statements when you
know all or part of their contents.  The following example finds all
statements containing \verb@ph -> ps@ followed by \verb@ch -> th@.  The
\verb@$*@ is a wildcard that matches anything; the \texttt{\$} before the
\verb$*$ prevents conflicts with math symbol token names.  The \verb@*@ after
\texttt{SEARCH} is also a wildcard that in this case means ``match any label.''
\index{\texttt{search} command}

% I'm omitting this one, since readers are unlikely to see it:
% 1096 bisymOLD $p |- ( ( ( ph -> ps ) -> ( ch -> th ) ) -> ( (
%   ( ps -> ph ) -> ( th -> ch ) ) -> ( ( ph <-> ps ) -> ( ch
%    <-> th ) ) ) )
\begin{verbatim}
MM> search * "ph -> ps $* ch -> th"
1791 prth $p |- ( ( ( ph -> ps ) /\ ( ch -> th ) ) -> ( ( ph
    /\ ch ) -> ( ps /\ th ) ) )
2455 pm3.48 $p |- ( ( ( ph -> ps ) /\ ( ch -> th ) ) -> ( (
    ph \/ ch ) -> ( ps \/ th ) ) )
117859 pm11.71 $p |- ( ( E. x ph /\ E. y ch ) -> ( ( A. x (
    ph -> ps ) /\ A. y ( ch -> th ) ) <-> A. x A. y ( ( ph /\
    ch ) -> ( ps /\ th ) ) ) )
\end{verbatim}

Three statements, \texttt{prth}, \texttt{pm3.48},
 and \texttt{pm11.71}, were found to match.

To see what axioms\index{axiom} and definitions\index{definition}
\texttt{prth} ultimately depends on for its proof, you can have the
program backtrack through the hierarchy\index{hierarchy} of theorems and
definitions.\index{\texttt{show trace{\char`\_}back} command}

\begin{verbatim}
MM> show trace_back prth /essential/axioms
Statement "prth" assumes the following axioms ($a
statements):
  ax-1 ax-2 ax-3 ax-mp df-bi df-an
\end{verbatim}

Note that the 3 axioms of propositional calculus and the modus ponens rule are
needed (as expected); in addition, there are a couple of definitions that are used
along the way.  Note that Metamath makes no distinction\index{axiom vs.\
definition} between axioms\index{axiom} and definitions\index{definition}.  In
\texttt{set.mm} they have been distinguished artificially by prefixing their
labels\index{labels in \texttt{set.mm}} with \texttt{ax-} and \texttt{df-}
respectively.  For example, \texttt{df-an} defines conjunction (logical {\sc
and}), which is represented by the symbol \verb+/\+.
Section~\ref{definitions} discusses the philosophy of definitions, and the
Metamath language takes a particularly simple, conservative approach by using
the \texttt{\$a}\index{\texttt{\$a} statement} statement for both axioms and
definitions.

You can also have the program compute how many steps a proof
has\index{proof length} if we were to follow it all the way back to
\texttt{\$a} statements.

\begin{verbatim}
MM> show trace_back prth /essential/count_steps
The statement's actual proof has 3 steps.  Backtracking, a
total of 79 different subtheorems are used.  The statement
and subtheorems have a total of 274 actual steps.  If
subtheorems used only once were eliminated, there would be a
total of 38 subtheorems, and the statement and subtheorems
would have a total of 185 steps.  The proof would have 28349
steps if fully expanded back to axiom references.  The
maximum path length is 38.  A longest path is:  prth <-
anim12d <- syl2and <- sylan2d <- ancomsd <- ancom <- pm3.22
<- pm3.21 <- pm3.2 <- ex <- sylbir <- biimpri <- bicomi <-
bicom1 <- bi2 <- dfbi1 <- impbii <- bi3 <- simprim <- impi <-
con1i <- nsyl2 <- mt3d <- con1d <- notnot1 <- con2i <- nsyl3
<- mt2d <- con2d <- notnot2 <- pm2.18d <- pm2.18 <- pm2.21 <-
pm2.21d <- a1d <- syl <- mpd <- a2i <- a2i.1 .
\end{verbatim}

This tells us that we would have to inspect 274 steps if we want to
verify the proof completely starting from the axioms.  A few more
statistics are also shown.  There are one or more paths back to axioms
that are the longest; this command ferrets out one of them and shows it
to you.  There may be a sense in which the longest path length is
related to how ``deep'' the theorem is.

We might also be curious about what proofs depend on the theorem
\texttt{prth}.  If it is never used later on, we could eliminate it as
redundant if it has no intrinsic interest by itself.\index{\texttt{show
usage} command}

% I decided to show the OLD values here.
\begin{verbatim}
MM> show usage prth
Statement "prth" is directly referenced in the proofs of 18
statements:
  mo3 moOLD 2mo 2moOLD euind reuind reuss2 reusv3i opelopabt
  wemaplem2 rexanre rlimcn2 o1of2 o1rlimmul 2sqlem6 spanuni
  heicant pm11.71
\end{verbatim}

Thus \texttt{prth} is directly used by 18 proofs.
We can use the \texttt{/recursive} qualifier to include indirect use:

\begin{verbatim}
MM> show usage prth /recursive
Statement "prth" directly or indirectly affects the proofs of
24214 statements:
  mo3 mo mo3OLD eu2 moOLD eu2OLD eu3OLD mo4f mo4 eu4 mopick
...
\end{verbatim}

\subsection{A Note on the ``Compact'' Proof Format}

The Metamath program will display proofs in a ``compact''\index{compact proof}
format whenever the proof is stored in compressed format in the database.  It
may be be slightly confusing unless you know how to interpret it.
For example,
if you display the complete proof of theorem \texttt{id1} it will start
off as follows:

\begin{verbatim}
MM> show proof id1 /lemmon/all
 1 wph           $f wff ph
 2 wph           $f wff ph
 3 wph           $f wff ph
 4 2,3 wi    @4: $a wff ( ph -> ph )
 5 1,4 wi    @5: $a wff ( ph -> ( ph -> ph ) )
 6 @4            $a wff ( ph -> ph )
\end{verbatim}

\begin{center}
{etc.}
\end{center}

Step 4 has a ``local label,''\index{local label} \texttt{@4}, assigned to it.
Later on, at step 6, the label \texttt{@1} is referenced instead of
displaying the explicit proof for that step.  This technique takes advantage
of the fact that steps in a proof often repeat, especially during the
construction of wffs.  The compact format reduces the number of steps in the
proof display and may be preferred by some people.

If you want to see the normal format with the ``true'' step numbers, you can
use the following workaround:\index{\texttt{save proof} command}

\begin{verbatim}
MM> save proof id1 /normal
The proof of "id1" has been reformatted and saved internally.
Remember to use WRITE SOURCE to save it permanently.
MM> show proof id1 /lemmon/all
 1 wph           $f wff ph
 2 wph           $f wff ph
 3 wph           $f wff ph
 4 2,3 wi        $a wff ( ph -> ph )
 5 1,4 wi        $a wff ( ph -> ( ph -> ph ) )
 6 wph           $f wff ph
 7 wph           $f wff ph
 8 6,7 wi        $a wff ( ph -> ph )
\end{verbatim}

\begin{center}
{etc.}
\end{center}

Note that the original 6 steps are now 8 steps.  However, the format is
now the same as that described in Chapter~\ref{using}.

\chapter{The Metamath Language}
\label{languagespec}

\begin{quote}
  {\em Thus mathematics may be defined as the subject in which we never know
what we are talking about, nor whether what we are saying is true.}
    \flushright\sc  Bertrand Russell\footnote{\cite[p.~84]{Russell2}.}\\
\end{quote}\index{Russell, Bertrand}

Probably the most striking feature of the Metamath language is its almost
complete absence of hard-wired syntax. Metamath\index{Metamath} does not
understand any mathematics or logic other than that needed to construct finite
sequences of symbols according to a small set of simple, built-in rules.  The
only rule it uses in a proof is the substitution of an expression (symbol
sequence) for a variable, subject to a simple constraint to prevent
bound-variable clashes.  The primitive notions built into Metamath involve the
simple manipulation of finite objects (symbols) that we as humans can easily
visualize and that computers can easily deal with.  They seem to be just
about the simplest notions possible that are required to do standard
mathematics.

This chapter serves as a reference manual for the Metamath\index{Metamath}
language. It covers the tedious technical details of the language, some of
which you may wish to skim in a first reading.  On the other hand, you should
pay close attention to the defined terms in {\bf boldface}; they have precise
meanings that are important to keep in mind for later understanding.  It may
be best to first become familiar with the examples in Chapter~\ref{using} to
gain some motivation for the language.

%% Uncomment this when uncommenting section {formalspec} below
If you have some knowledge of set theory, you may wish to study this
chapter in conjunction with the formal set-theoretical description of the
Metamath language in Appendix~\ref{formalspec}.

We will use the name ``Metamath''\index{Metamath} to mean either the Metamath
computer language or the Metamath software associated with the computer
language.  We will not distinguish these two when the context is clear.

The next section contains the complete specification of the Metamath
language.
It serves as an
authoritative reference and presents the syntax in enough detail to
write a parser\index{parsing Metamath} and proof verifier.  The
specification is terse and it is probably hard to learn the language
directly from it, but we include it here for those impatient people who
prefer to see everything up front before looking at verbose expository
material.  Later sections explain this material and provide examples.
We will repeat the definitions in those sections, and you may skip the
next section at first reading and proceed to Section~\ref{tut1}
(p.~\pageref{tut1}).

\section{Specification of the Metamath Language}\label{spec}
\index{Metamath!specification}

\begin{quote}
  {\em Sometimes one has to say difficult things, but one ought to say
them as simply as one knows how.}
    \flushright\sc  G. H. Hardy\footnote{As quoted in
    \cite{deMillo}, p.~273.}\\
\end{quote}\index{Hardy, G. H.}

\subsection{Preliminaries}\label{spec1}

% Space is technically a printable character, so we'll word things
% carefully so it's unambiguous.
A Metamath {\bf database}\index{database} is built up from a top-level source
file together with any source files that are brought in through file inclusion
commands (see below).  The only characters that are allowed to appear in a
Metamath source file are the 94 non-whitespace printable {\sc
ascii}\index{ascii@{\sc ascii}} characters, which are digits, upper and lower
case letters, and the following 32 special
characters\index{special characters}:\label{spec1chars}

\begin{verbatim}
! " # $ % & ' ( ) * + , - . / :
; < = > ? @ [ \ ] ^ _ ` { | } ~
\end{verbatim}
plus the following characters which are the ``white space'' characters:
space (a printable character),
tab, carriage return, line feed, and form feed.\label{whitespace}
We will use \texttt{typewriter}
font to display the printable characters.

A Metamath database consists of a sequence of three kinds of {\bf
tokens}\index{token} separated by {\bf white space}\index{white space}
(which is any sequence of one or more white space characters).  The set
of {\bf keyword}\index{keyword} tokens is \texttt{\$\char`\{},
\texttt{\$\char`\}}, \texttt{\$c}, \texttt{\$v}, \texttt{\$f},
\texttt{\$e}, \texttt{\$d}, \texttt{\$a}, \texttt{\$p}, \texttt{\$.},
\texttt{\$=}, \texttt{\$(}, \texttt{\$)}, \texttt{\$[}, and
\texttt{\$]}.  The last four are called {\bf auxiliary}\index{auxiliary
keyword} or preprocessing keywords.  A {\bf label}\index{label} token
consists of any combination of letters, digits, and the characters
hyphen, underscore, and period.  A {\bf math symbol}\index{math symbol}
token may consist of any combination of the 93 printable standard {\sc
ascii} characters other than space or \texttt{\$}~. All tokens are
case-sensitive.

\subsection{Preprocessing}

The token \texttt{\$(} begins a {\bf comment} and
\texttt{\$)} ends a comment.\index{\texttt{\$(}
and \texttt{\$)} auxiliary keywords}\index{comment}
Comments may contain any of
the 94 non-whitespace printable characters and white space,
except they may not contain the
2-character sequences \texttt{\$(} or \texttt{\$)} (comments do not nest).
Comments are ignored (treated
like white space) for the purpose of parsing, e.g.,
\texttt{\$( \$[ \$)} is a comment.
See p.~\pageref{mathcomments} for comment typesetting conventions; these
conventions may be ignored for the purpose of parsing.

A {\bf file inclusion command} consists of \texttt{\$[} followed by a file name
followed by \texttt{\$]}.\index{\texttt{\$[} and \texttt{\$]} auxiliary
keywords}\index{included file}\index{file inclusion}
It is only allowed in the outermost scope (i.e., not between
\texttt{\$\char`\{} and \texttt{\$\char`\}})
and must not be inside a statement (e.g., it may not occur
between the label of a \texttt{\$a} statement and its \texttt{\$.}).
The file name may not
contain a \texttt{\$} or white space.  The file must exist.
The case-sensitivity
of its name follows the conventions of the operating system.  The contents of
the file replace the inclusion command.
Included files may include other files.
Only the first reference to a given file is included; any later
references to the same file (whether in the top-level file or in included
files) cause the inclusion command to be ignored (treated like white space).
A verifier may assume that file names with different strings
refer to different files for the purpose of ignoring later references.
A file self-reference is ignored, as is any reference to the top-level file
(to avoid loops).
Included files may not include a \texttt{\$(} without a matching \texttt{\$)},
may not include a \texttt{\$[} without a matching \texttt{\$]}, and may
not include incomplete statements (e.g., a \texttt{\$a} without a matching
\texttt{\$.}).
It is currently unspecified if path references are relative to the process'
current directory or the file's containing directory, so databases should
avoid using pathname separators (e.g., ``/'') in file names.

Like all tokens, the \texttt{\$(}, \texttt{\$)}, \texttt{\$[}, and \texttt{\$]} keywords
must be surrounded by white space.

\subsection{Basic Syntax}

After preprocessing, a database will consist of a sequence of {\bf
statements}.
These are the scoping statements \texttt{\$\char`\{} and
\texttt{\$\char`\}}, along with the \texttt{\$c}, \texttt{\$v},
\texttt{\$f}, \texttt{\$e}, \texttt{\$d}, \texttt{\$a}, and \texttt{\$p}
statements.

A {\bf scoping statement}\index{scoping statement} consists only of its
keyword, \texttt{\$\char`\{} or \texttt{\$\char`\}}.
A \texttt{\$\char`\{} begins a {\bf
block}\index{block} and a matching \texttt{\$\char`\}} ends the block.
Every \texttt{\$\char`\{}
must have a matching \texttt{\$\char`\}}.
Defining it recursively, we say a block
contains a sequence of zero or more tokens other
than \texttt{\$\char`\{} and \texttt{\$\char`\}} and
possibly other blocks.  There is an {\bf outermost
block}\index{block!outermost} not bracketed by \texttt{\$\char`\{} \texttt{\$\char`\}}; the end
of the outermost block is the end of the database.

% LaTeX bug? can't do \bf\tt

A {\bf \$v} or {\bf \$c statement}\index{\texttt{\$v} statement}\index{\texttt{\$c}
statement} consists of the keyword token \texttt{\$v} or \texttt{\$c} respectively,
followed by one or more math symbols,
% The word "token" is used to distinguish "$." from the sentence-ending period.
followed by the \texttt{\$.}\ token.
These
statements {\bf declare}\index{declaration} the math symbols to be {\bf
variables}\index{variable!Metamath} or {\bf constants}\index{constant}
respectively. The same math symbol may not occur twice in a given \texttt{\$v} or
\texttt{\$c} statement.

%c%A math symbol becomes an {\bf active}\index{active math symbol}
%c%when declared and stays active until the end of the block in which it is
%c%declared.  A math symbol may not be declared a second time while it is active,
%c%but it may be declared again after it becomes inactive.

A math symbol becomes {\bf active}\index{active math symbol} when declared
and stays active until the end of the block in which it is declared.  A
variable may not be declared a second time while it is active, but it
may be declared again (as a variable, but not as a constant) after it
becomes inactive.  A constant must be declared in the outermost block and may
not be declared a second time.\index{redeclaration of symbols}

A {\bf \$f statement}\index{\texttt{\$f} statement} consists of a label,
followed by \texttt{\$f}, followed by its typecode (an active constant),
followed by an
active variable, followed by the \texttt{\$.}\ token.  A {\bf \$e
statement}\index{\texttt{\$e} statement} consists of a label, followed
by \texttt{\$e}, followed by its typecode (an active constant),
followed by zero or more
active math symbols, followed by the \texttt{\$.}\ token.  A {\bf
hypothesis}\index{hypothesis} is a \texttt{\$f} or \texttt{\$e}
statement.
The type declared by a \texttt{\$f} statement for a given label
is global even if the variable is not
(e.g., a database may not have \texttt{wff P} in one local scope
and \texttt{class P} in another).

A {\bf simple \$d statement}\index{\texttt{\$d} statement!simple}
consists of \texttt{\$d}, followed by two different active variables,
followed by the \texttt{\$.}\ token.  A {\bf compound \$d
statement}\index{\texttt{\$d} statement!compound} consists of
\texttt{\$d}, followed by three or more variables (all different),
followed by the \texttt{\$.}\ token.  The order of the variables in a
\texttt{\$d} statement is unimportant.  A compound \texttt{\$d}
statement is equivalent to a set of simple \texttt{\$d} statements, one
for each possible pair of variables occurring in the compound
\texttt{\$d} statement.  Henceforth in this specification we shall
assume all \texttt{\$d} statements are simple.  A \texttt{\$d} statement
is also called a {\bf disjoint} (or {\bf distinct}) {\bf variable
restriction}.\index{disjoint-variable restriction}

A {\bf \$a statement}\index{\texttt{\$a} statement} consists of a label,
followed by \texttt{\$a}, followed by its typecode (an active constant),
followed by
zero or more active math symbols, followed by the \texttt{\$.}\ token.  A {\bf
\$p statement}\index{\texttt{\$p} statement} consists of a label,
followed by \texttt{\$p}, followed by its typecode (an active constant),
followed by
zero or more active math symbols, followed by \texttt{\$=}, followed by
a sequence of labels, followed by the \texttt{\$.}\ token.  An {\bf
assertion}\index{assertion} is a \texttt{\$a} or \texttt{\$p} statement.

A \texttt{\$f}, \texttt{\$e}, or \texttt{\$d} statement is {\bf active}\index{active
statement} from the place it occurs until the end of the block it occurs in.
A \texttt{\$a} or \texttt{\$p} statement is {\bf active} from the place it occurs
through the end of the database.
There may not be two active \texttt{\$f} statements containing the same
variable.  Each variable in a \texttt{\$e}, \texttt{\$a}, or
\texttt{\$p} statement must exist in an active \texttt{\$f}
statement.\footnote{This requirement can greatly simplify the
unification algorithm (substitution calculation) required by proof
verification.}

%The label that begins each \texttt{\$f}, \texttt{\$e}, \texttt{\$a}, and
%\texttt{\$p} statement must be unique.
Each label token must be unique, and
no label token may match any math symbol
token.\label{namespace}\footnote{This
restriction was added on June 24, 2006.
It is not theoretically necessary but is imposed to make it easier to
write certain parsers.}

The set of {\bf mandatory variables}\index{mandatory variable} associated with
an assertion is the set of (zero or more) variables in the assertion and in any
active \texttt{\$e} statements.  The (possibly empty) set of {\bf mandatory
hypotheses}\index{mandatory hypothesis} is the set of all active \texttt{\$f}
statements containing mandatory variables, together with all active \texttt{\$e}
statements.
The set of {\bf mandatory {\bf \$d} statements}\index{mandatory
disjoint-variable restriction} associated with an assertion are those active
\texttt{\$d} statements whose variables are both among the assertion's
mandatory variables.

\subsection{Proof Verification}\label{spec4}

The sequence of labels between the \texttt{\$=} and \texttt{\$.}\ tokens
in a \texttt{\$p} statement is a {\bf proof}.\index{proof!Metamath} Each
label in a proof must be the label of an active statement other than the
\texttt{\$p} statement itself; thus a label must refer either to an
active hypothesis of the \texttt{\$p} statement or to an earlier
assertion.

An {\bf expression}\index{expression} is a sequence of math symbols. A {\bf
substitution map}\index{substitution map} associates a set of variables with a
set of expressions.  It is acceptable for a variable to be mapped to an
expression containing it.  A {\bf
substitution}\index{substitution!variable}\index{variable substitution} is the
simultaneous replacement of all variables in one or more expressions with the
expressions that the variables map to.

A proof is scanned in order of its label sequence.  If the label refers to an
active hypothesis, the expression in the hypothesis is pushed onto a
stack.\index{stack}\index{RPN stack}  If the label refers to an assertion, a
(unique) substitution must exist that, when made to the mandatory hypotheses
of the referenced assertion, causes them to match the topmost (i.e.\ most
recent) entries of the stack, in order of occurrence of the mandatory
hypotheses, with the topmost stack entry matching the last mandatory
hypothesis of the referenced assertion.  As many stack entries as there are
mandatory hypotheses are then popped from the stack.  The same substitution is
made to the referenced assertion, and the result is pushed onto the stack.
After the last label in the proof is processed, the stack must have a single
entry that matches the expression in the \texttt{\$p} statement containing the
proof.

%c%{\footnotesize\begin{quotation}\index{redeclaration of symbols}
%c%{{\em Comment.}\label{spec4comment} Whenever a math symbol token occurs in a
%c%{\texttt{\$c} or \texttt{\$v} statement, it is considered to designate a distinct new
%c%{symbol, even if the same token was previously declared (and is now inactive).
%c%{Thus a math token declared as a constant in two different blocks is considered
%c%{to designate two distinct constants (even though they have the same name).
%c%{The two constants will not match in a proof that references both blocks.
%c%{However, a proof referencing both blocks is acceptable as long as it doesn't
%c%{require that the constants match.  Similarly, a token declared to be a
%c%{constant for a referenced assertion will not match the same token declared to
%c%{be a variable for the \texttt{\$p} statement containing the proof.  In the case
%c%{of a token declared to be a variable for a referenced assertion, this is not
%c%{an issue since the variable can be substituted with whatever expression is
%c%{needed to achieve the required match.
%c%{\end{quotation}}
%c2%A proof may reference an assertion that contains or whose hypotheses contain a
%c2%constant that is not active for the \texttt{\$p} statement containing the proof.
%c2%However, the final result of the proof may not contain that constant. A proof
%c2%may also reference an assertion that contains or whose hypotheses contain a
%c2%variable that is not active for the \texttt{\$p} statement containing the proof.
%c2%That variable, of course, will be substituted with whatever expression is
%c2%needed to achieve the required match.

A proof may contain a \texttt{?}\ in place of a label to indicate an unknown step
(Section~\ref{unknown}).  A proof verifier may ignore any proof containing
\texttt{?}\ but should warn the user that the proof is incomplete.

A {\bf compressed proof}\index{compressed proof}\index{proof!compressed} is an
alternate proof notation described in Appen\-dix~\ref{compressed}; also see
references to ``compressed proof'' in the Index.  Compressed proofs are a
Metamath language extension which a complete proof verifier should be able to
parse and verify.

\subsubsection{Verifying Disjoint Variable Restrictions}

Each substitution made in a proof must be checked to verify that any
disjoint variable restrictions are satisfied, as follows.

If two variables replaced by a substitution exist in a mandatory \texttt{\$d}
statement\index{\texttt{\$d} statement} of the assertion referenced, the two
expressions resulting from the substitution must satisfy the following
conditions.  First, the two expressions must have no variables in common.
Second, each possible pair of variables, one from each expression, must exist
in an active \texttt{\$d} statement of the \texttt{\$p} statement containing the
proof.

\vskip 1ex

This ends the specification of the Metamath language;
see Appendix \ref{BNF} for its syntax in
Extended Backus--Naur Form (EBNF)\index{Extended Backus--Naur Form}\index{EBNF}.

\section{The Basic Keywords}\label{tut1}

Our expository material begins here.

Like most computer languages, Metamath\index{Metamath} takes its input from
one or more {\bf source files}\index{source file} which contain characters
expressed in the standard {\sc ascii} (American Standard Code for Information
Interchange)\index{ascii@{\sc ascii}} code for computers.  A source file
consists of a series of {\bf tokens}\index{token}, which are strings of
non-whitespace
printable characters (from the set of 94 shown on p.~\pageref{spec1chars})
separated by {\bf white space}\index{white space} (spaces, tabs, carriage
returns, line feeds, and form feeds). Any string consisting only of these
characters is treated the same as a single space.  The non-whitespace printable
characters\index{printable character} that Metamath recognizes are the 94
characters on standard {\sc ascii} keyboards.

Metamath has the ability to join several files together to form its
input (Section~\ref{include}).  We call the aggregate contents of all
the files after they have been joined together a {\bf
database}\index{database} to distinguish it from an individual source
file.  The tokens in a database consist of {\bf
keywords}\index{keyword}, which are built into the language, together
with two kinds of user-defined tokens called {\bf labels}\index{label}
and {\bf math symbols}\index{math symbol}.  (Often we will simply say
{\bf symbol}\index{symbol} instead of math symbol for brevity).  The set
of {\bf basic keywords}\index{basic keyword} is
\texttt{\$c}\index{\texttt{\$c} statement},
\texttt{\$v}\index{\texttt{\$v} statement},
\texttt{\$e}\index{\texttt{\$e} statement},
\texttt{\$f}\index{\texttt{\$f} statement},
\texttt{\$d}\index{\texttt{\$d} statement},
\texttt{\$a}\index{\texttt{\$a} statement},
\texttt{\$p}\index{\texttt{\$p} statement},
\texttt{\$=}\index{\texttt{\$=} keyword},
\texttt{\$.}\index{\texttt{\$.}\ keyword},
\texttt{\$\char`\{}\index{\texttt{\$\char`\{} and \texttt{\$\char`\}}
keywords}, and \texttt{\$\char`\}}.  This is the complete set of
syntactical elements of what we call the {\bf basic
language}\index{basic language} of Metamath, and with them you can
express all of the mathematics that were intended by the design of
Metamath.  You should make it a point to become very familiar with them.
Table~\ref{basickeywords} lists the basic keywords along with a brief
description of their functions.  For now, this description will give you
only a vague notion of what the keywords are for; later we will describe
the keywords in detail.


\begin{table}[htp] \caption{Summary of the basic Metamath
keywords} \label{basickeywords}
\begin{center}
\begin{tabular}{|p{4pc}|l|}
\hline
\em \centering Keyword&\em Description\\
\hline
\hline
\centering
   \texttt{\$c}&Constant symbol declaration\\
\hline
\centering
   \texttt{\$v}&Variable symbol declaration\\
\hline
\centering
   \texttt{\$d}&Disjoint variable restriction\\
\hline
\centering
   \texttt{\$f}&Variable-type (``floating'') hypothesis\\
\hline
\centering
   \texttt{\$e}&Logical (``essential'') hypothesis\\
\hline
\centering
   \texttt{\$a}&Axiomatic assertion\\
\hline
\centering
   \texttt{\$p}&Provable assertion\\
\hline
\centering
   \texttt{\$=}&Start of proof in \texttt{\$p} statement\\
\hline
\centering
   \texttt{\$.}&End of the above statement types\\
\hline
\centering
   \texttt{\$\char`\{}&Start of block\\
\hline
\centering
   \texttt{\$\char`\}}&End of block\\
\hline
\end{tabular}
\end{center}
\end{table}

%For LaTeX bug(?) where it puts tables on blank page instead of btwn text
%May have to adjust if text changes
%\newpage

There are some additional keywords, called {\bf auxiliary
keywords}\index{auxiliary keyword} that help make Metamath\index{Metamath}
more practical. These are part of the {\bf extended language}\index{extended
language}. They provide you with a means to put comments into a Metamath
source file\index{source file} and reference other source files.  We will
introduce these in later sections. Table~\ref{otherkeywords} summarizes them
so that you can recognize them now if you want to peruse some source
files while learning the basic keywords.


\begin{table}[htp] \caption{Auxiliary Metamath
keywords} \label{otherkeywords}
\begin{center}
\begin{tabular}{|p{4pc}|l|}
\hline
\em \centering Keyword&\em Description\\
\hline
\hline
\centering
   \texttt{\$(}&Start of comment\\
\hline
\centering
   \texttt{\$)}&End of comment\\
\hline
\centering
   \texttt{\$[}&Start of included source file name\\
\hline
\centering
   \texttt{\$]}&End of included source file name\\
\hline
\end{tabular}
\end{center}
\end{table}
\index{\texttt{\$(} and \texttt{\$)} auxiliary keywords}
\index{\texttt{\$[} and \texttt{\$]} auxiliary keywords}


Unlike those in some computer languages, the keywords\index{keyword} are short
two-character sequences rather than English-like words.  While this may make
them slightly more difficult to remember at first, their brevity allows
them to blend in with the mathematics being described, not
distract from it, like punctuation marks.


\subsection{User-Defined Tokens}\label{dollardollar}\index{token}

As you may have noticed, all keywords\index{keyword} begin with the \texttt{\$}
character.  This mundane monetary symbol is not ordinarily used in higher
mathematics (outside of grant proposals), so we have appropriated it to
distinguish the Metamath\index{Metamath} keywords from ordinary mathematical
symbols. The \texttt{\$} character is thus considered special and may not be
used as a character in a user-defined token.  All tokens and keywords are
case-sensitive; for example, \texttt{n} is considered to be a different character
from \texttt{N}.  Case-sensitivity makes the available {\sc ascii} character set
as rich as possible.

\subsubsection{Math Symbol Tokens}\index{token}

Math symbols\index{math symbol} are tokens used to represent the symbols
that appear in ordinary mathematical formulas.  They may consist of any
combination of the 93 non-whitespace printable {\sc ascii} characters other than
\texttt{\$}~. Some examples are \texttt{x}, \texttt{+}, \texttt{(},
\texttt{|-}, \verb$!%@?&$, and \texttt{bounded}.  For readability, it is
best to try to make these look as similar to actual mathematical symbols
as possible, within the constraints of the {\sc ascii} character set, in
order to make the resulting mathematical expressions more readable.

In the Metamath\index{Metamath} language, you express ordinary
mathematical formulas and statements as sequences of math symbols such
as \texttt{2 + 2 = 4} (five symbols, all constants).\footnote{To
eliminate ambiguity with other expressions, this is expressed in the set
theory database \texttt{set.mm} as \texttt{|- ( 2 + 2
 ) = 4 }, whose \LaTeX\ equivalent is $\vdash
(2+2)=4$.  The \,$\vdash$ means ``is a theorem'' and the
parentheses allow explicit associative grouping.}\index{turnstile
({$\,\vdash$})} They may even be English
sentences, as in \texttt{E is closed and bounded} (five symbols)---here
\texttt{E} would be a variable and the other four symbols constants.  In
principle, a Metamath database could be constructed to work with almost
any unambiguous English-language mathematical statement, but as a
practical matter the definitions needed to provide for all possible
syntax variations would be cumbersome and distracting and possibly have
subtle pitfalls accidentally built in.  We generally recommend that you
express mathematical statements with compact standard mathematical
symbols whenever possible and put their English-language descriptions in
comments.  Axioms\index{axiom} and definitions\index{definition}
(\texttt{\$a}\index{\texttt{\$a} statement} statements) are the only
places where Metamath will not detect an error, and doing this will help
reduce the number of definitions needed.

You are free to use any tokens\index{token} you like for math
symbols\index{math symbol}.  Appendix~\ref{ASCII} recommends token names to
use for symbols in set theory, and we suggest you adopt these in order to be
able to include the \texttt{set.mm} set theory database in your database.  For
printouts, you can convert the tokens in a database
to standard mathematical symbols with the \LaTeX\ typesetting program.  The
Metamath command \texttt{open tex} {\em filename}\index{\texttt{open tex} command}
produces output that can be read by \LaTeX.\index{latex@{\LaTeX}}
The correspondence
between tokens and the actual symbols is made by \texttt{latexdef}
statements inside a special database comment tagged
with \texttt{\$t}.\index{\texttt{\$t} comment}\index{typesetting comment}
  You can edit
this comment to change the definitions or add new ones.
Appendix~\ref{ASCII} describes how to do this in more detail.

% White space\index{white space} is normally used to separate math
% symbol\index{math symbol} tokens, but they may be juxtaposed without white
% space in \texttt{\$d}\index{\texttt{\$d} statement}, \texttt{\$e}\index{\texttt{\$e}
% statement}, \texttt{\$f}\index{\texttt{\$f} statement}, \texttt{\$a}\index{\texttt{\$a}
% statement}, and \texttt{\$p}\index{\texttt{\$p} statement} statements when no
% ambiguity will result.  Specifically, Metamath parses the math symbol sequence
% in one of these statements in the following manner:  when the math symbol
% sequence has been broken up into tokens\index{token} up to a given character,
% the next token is the longest string of characters that could constitute a
% math symbol that is active\index{active
% math symbol} at that point.  (See Section~\ref{scoping} for the
% definition of an active math symbol.)  For example, if \texttt{-}, \texttt{>}, and
% \texttt{->} are the only active math symbols, the juxtaposition \texttt{>-} will be
% interpreted as the two symbols \texttt{>} and \texttt{-}, whereas \texttt{->} will
% always be interpreted as that single symbol.\footnote{For better readability we
% recommend a white space between each token.  This also makes searching for a
% symbol easier to do with an editor.  Omission of optional white space is useful
% for reducing typing when assigning an expression to a temporary
% variable\index{temporary variable} with the \texttt{let variable} Metamath
% program command.}\index{\texttt{let variable} command}
%
% Keywords\index{keyword} may be placed next to math symbols without white
% space\index{white space} between them.\footnote{Again, we do not recommend
% this for readability.}
%
% The math symbols\index{math symbol} in \texttt{\$c}\index{\texttt{\$c} statement}
% and \texttt{\$v}\index{\texttt{\$v} statement} statements must always be separated
% by white space\index{white
% space}, for the obvious reason that these statements define the names
% of the symbols.
%
% Math symbols referred to in comments (see Section~\ref{comments}) must also be
% separated by white space.  This allows you to make comments about symbols that
% are not yet active\index{active
% math symbol}.  (The ``math mode'' feature of comments is also a quick and
% easy way to obtain word processing text with embedded mathematical symbols,
% independently of the main purpose of Metamath; the way to do this is described
% in Section~\ref{comments})

\subsubsection{Label Tokens}\index{token}\index{label}

Label tokens are used to identify Metamath\index{Metamath} statements for
later reference. Label tokens may contain only letters, digits, and the three
characters period, hyphen, and underscore:
\begin{verbatim}
. - _
\end{verbatim}

A label is {\bf declared}\index{label declaration} by placing it immediately
before the keyword of the statement it identifies.  For example, the label
\texttt{axiom.1} might be declared as follows:
\begin{verbatim}
axiom.1 $a |- x = x $.
\end{verbatim}

Each \texttt{\$e}\index{\texttt{\$e} statement},
\texttt{\$f}\index{\texttt{\$f} statement},
\texttt{\$a}\index{\texttt{\$a} statement}, and
\texttt{\$p}\index{\texttt{\$p} statement} statement in a database must
have a label declared for it.  No other statement types may have label
declarations.  Every label must be unique.

A label (and the statement it identifies) is {\bf referenced}\index{label
reference} by including the label between the \texttt{\$=}\index{\texttt{\$=}
keyword} and \texttt{\$.}\index{\texttt{\$.}\ keyword}\ keywords in a \texttt{\$p}
statement.  The sequence of labels\index{label sequence} between these two
keywords is called a {\bf proof}\index{proof}.  An example of a statement with
a proof that we will encounter later (Section~\ref{proof}) is
\begin{verbatim}
wnew $p wff ( s -> ( r -> p ) )
     $= ws wr wp w2 w2 $.
\end{verbatim}

You don't have to know what this means just yet, but you should know that the
label \texttt{wnew} is declared by this \texttt{\$p} statement and that the labels
\texttt{ws}, \texttt{wr}, \texttt{wp}, and \texttt{w2} are assumed to have been declared
earlier in the database and are referenced here.

\subsection{Constants and Variables}
\index{constant}
\index{variable}

An {\bf expression}\index{expression} is any sequence of math
symbols, possibly empty.

The basic Metamath\index{Metamath} language\index{basic language} has two
kinds of math symbols\index{math symbol}:  {\bf constants}\index{constant} and
{\bf variables}\index{variable}.  In a Metamath proof, a constant may not be
substituted with any expression.  A variable can be
substituted\index{substitution!variable}\index{variable substitution} with any
expression.  This sequence may include other variables and may even include
the variable being substituted.  This substitution takes place when proofs are
verified, and it will be described in Section~\ref{proof}.  The \texttt{\$f}
statement (described later in Section~\ref{dollaref}) is used to specify the
{\bf type} of a variable (i.e.\ what kind of
variable it is)\index{variable type}\index{type} and
give it a meaning typically
associated with a ``metavariable''\index{metavariable}\footnote{A metavariable
is a variable that ranges over the syntactical elements of the object language
being discussed; for example, one metavariable might represent a variable of
the object language and another metavariable might represent a formula in the
object language.} in ordinary mathematics; for example, a variable may be
specified to be a wff or well-formed formula (in logic), a set (in set
theory), or a non-negative integer (in number theory).

%\subsection{The \texttt{\$c} and \texttt{\$v} Declaration Statements}
\subsection{The \texttt{\$c} and \texttt{\$v} Declaration Statements}
\index{\texttt{\$c} statement}
\index{constant declaration}
\index{\texttt{\$v} statement}
\index{variable declaration}

Constants are introduced or {\bf declared}\index{constant declaration}
with \texttt{\$c}\index{\texttt{\$c} statement} statements, and
variables are declared\index{variable declaration} with
\texttt{\$v}\index{\texttt{\$v} statement} statements.  A {\bf simple}
declaration\index{simple declaration} statement introduces a single
constant or variable.  Its syntax is one of the following:
\begin{center}
  \texttt{\$c} {\em math-symbol} \texttt{\$.}\\
  \texttt{\$v} {\em math-symbol} \texttt{\$.}
\end{center}
The notation {\em math-symbol} means any math symbol token\index{token}.

Some examples of simple declaration statements are:
\begin{center}
  \texttt{\$c + \$.}\\
  \texttt{\$c -> \$.}\\
  \texttt{\$c ( \$.}\\
  \texttt{\$v x \$.}\\
  \texttt{\$v y2 \$.}
\end{center}

The characters in a math symbol\index{math symbol} being declared are
irrelevant to Meta\-math; for example, we could declare a right parenthesis to
be a variable,
\begin{center}
  \texttt{\$v ) \$.}\\
\end{center}
although this would be unconventional.

A {\bf compound} declaration\index{compound declaration} statement is a
shorthand for declaring several symbols at once.  Its syntax is one of the
following:
\begin{center}
  \texttt{\$c} {\em math-symbol}\ \,$\cdots$\ {\em math-symbol} \texttt{\$.}\\
  \texttt{\$v} {\em math-symbol}\ \,$\cdots$\ {\em math-symbol} \texttt{\$.}
\end{center}\index{\texttt{\$c} statement}
Here, the ellipsis (\ldots) means any number of {\em math-symbol}\,s.

An example of a compound declaration statement is:
\begin{center}
  \texttt{\$v x y mu \$.}\\
\end{center}
This is equivalent to the three simple declaration statements
\begin{center}
  \texttt{\$v x \$.}\\
  \texttt{\$v y \$.}\\
  \texttt{\$v mu \$.}\\
\end{center}
\index{\texttt{\$v} statement}

There are certain rules on where in the database math symbols may be declared,
what sections of the database are aware of them (i.e.\ where they are
``active''), and when they may be declared more than once.  These will be
discussed in Section~\ref{scoping} and specifically on
p.~\pageref{redeclaration}.

\subsection{The \texttt{\$d} Statement}\label{dollard}
\index{\texttt{\$d} statement}

The \texttt{\$d} statement is called a {\bf disjoint-variable restriction}.  The
syntax of the {\bf simple} version of this statement is
\begin{center}
  \texttt{\$d} {\em variable variable} \texttt{\$.}
\end{center}
where each {\em variable} is a previously declared variable and the two {\em
variable}\,s are different.  (More specifically, each  {\em variable} must be
an {\bf active} variable\index{active math symbol}, which means there must be
a previous \texttt{\$v} statement whose {\bf scope}\index{scope} includes the
\texttt{\$d} statement.  These terms will be defined when we discuss scoping
statements in Section~\ref{scoping}.)

In ordinary mathematics, formulas may arise that are true if the variables in
them are distinct\index{distinct variables}, but become false when those
variables are made identical. For example, the formula in logic $\exists x\,x
\neq y$, which means ``for a given $y$, there exists an $x$ that is not equal
to $y$,'' is true in most mathematical theories (namely all non-trivial
theories\index{non-trivial theory}, i.e.\ those that describe more than one
individual, such as arithmetic).  However, if we substitute $y$ with $x$, we
obtain $\exists x\,x \neq x$, which is always false, as it means ``there
exists something that is not equal to itself.''\footnote{If you are a
logician, you will recognize this as the improper substitution\index{proper
substitution}\index{substitution!proper} of a free variable\index{free
variable} with a bound variable\index{bound variable}.  Metamath makes no
inherent distinction between free and bound variables; instead, you let
Metamath know what substitutions are permissible by using \texttt{\$d} statements
in the right way in your axiom system.}\index{free vs.\ bound variable}  The
\texttt{\$d} statement allows you to specify a restriction that forbids the
substitution of one variable with another.  In
this case, we would use the statement
\begin{center}
  \texttt{\$d x y \$.}
\end{center}\index{\texttt{\$d} statement}
to specify this restriction.

The order in which the variables appear in a \texttt{\$d} statement is not
important.  We could also use
\begin{center}
  \texttt{\$d y x \$.}
\end{center}

The \texttt{\$d} statement is actually more general than this, as the
``disjoint''\index{disjoint variables} in its name suggests.  The full meaning
is that if any substitution is made to its two variables (during the
course of a proof that references a \texttt{\$a} or \texttt{\$p} statement
associated with the \texttt{\$d}), the two expressions that result from the
substitution must have no variables in common.  In addition, each possible
pair of variables, one from each expression, must be in a \texttt{\$d} statement
associated with the statement being proved.  (This requirement forces the
statement being proved to ``inherit'' the original disjoint variable
restriction.)

For example, suppose \texttt{u} is a variable.  If the restriction
\begin{center}
  \texttt{\$d A B \$.}
\end{center}
has been specified for a theorem referenced in a
proof, we may not substitute \texttt{A} with \mbox{\tt a + u} and
\texttt{B} with \mbox{\tt b + u} because these two symbol sequences have the
variable \texttt{u} in common.  Furthermore, if \texttt{a} and \texttt{b} are
variables, we may not substitute \texttt{A} with \texttt{a} and \texttt{B} with \texttt{b}
unless we have also specified \texttt{\$d a b} for the theorem being proved; in
other words, the \texttt{\$d} property associated with a pair of variables must
be effectively preserved after substitution.

The \texttt{\$d}\index{\texttt{\$d} statement} statement does {\em not} mean ``the
two variables may not be substituted with the same thing,'' as you might think
at first.  For example, substituting each of \texttt{A} and \texttt{B} in the above
example with identical symbol sequences consisting only of constants does not
cause a disjoint variable conflict, because two symbol sequences have no
variables in common (since they have no variables, period).  Similarly, a
conflict will not occur by substituting the two variables in a \texttt{\$d}
statement with the empty symbol sequence\index{empty substitution}.

The \texttt{\$d} statement does not have a direct counterpart in
ordinary mathematics, partly because the variables\index{variable} of
Metamath are not really the same as the variables\index{variable!in
ordinary mathematics} of ordinary mathematics but rather are
metavariables\index{metavariable} ranging over them (as well as over
other kinds of symbols and groups of symbols).  Depending on the
situation, we may informally interpret the \texttt{\$d} statement in
different ways.  Suppose, for example, that \texttt{x} and \texttt{y}
are variables ranging over numbers (more precisely, that \texttt{x} and
\texttt{y} are metavariables ranging over variables that range over
numbers), and that \texttt{ph} ($\varphi$) and \texttt{ps} ($\psi$) are
variables (more precisely, metavariables) ranging over formulas.  We can
make the following interpretations that correspond to the informal
language of ordinary mathematics:
\begin{quote}
\begin{tabbing}
\texttt{\$d x y \$.} means ``assume $x$ and $y$ are
distinct variables.''\\
\texttt{\$d x ph \$.} means ``assume $x$ does not
occur in $\varphi$.''\\
\texttt{\$d ph ps \$.} \=means ``assume $\varphi$ and
$\psi$ have no variables\\ \>in common.''
\end{tabbing}
\end{quote}\index{\texttt{\$d} statement}

\subsubsection{Compound \texttt{\$d} Statements}

The {\bf compound} version of the \texttt{\$d} statement is a shorthand for
specifying several variables whose substitutions must be pairwise disjoint.
Its syntax is:
\begin{center}
  \texttt{\$d} {\em variable}\ \,$\cdots$\ {\em variable} \texttt{\$.}
\end{center}\index{\texttt{\$d} statement}
Here, {\em variable} represents the token of a previously declared
variable (specifically, an active variable) and all {\em variable}\,s are
different.  The compound \texttt{\$d}
statement is internally broken up by Metamath into one simple \texttt{\$d}
statement for each possible pair of variables in the original \texttt{\$d}
statement.  For example,
\begin{center}
  \texttt{\$d w x y z \$.}
\end{center}
is equivalent to
\begin{center}
  \texttt{\$d w x \$.}\\
  \texttt{\$d w y \$.}\\
  \texttt{\$d w z \$.}\\
  \texttt{\$d x y \$.}\\
  \texttt{\$d x z \$.}\\
  \texttt{\$d y z \$.}
\end{center}

Two or more simple \texttt{\$d} statements specifying the same variable pair are
internally combined into a single \texttt{\$d} statement.  Thus the set of three
statements
\begin{center}
  \texttt{\$d x y \$.}
  \texttt{\$d x y \$.}
  \texttt{\$d y x \$.}
\end{center}
is equivalent to
\begin{center}
  \texttt{\$d x y \$.}
\end{center}

Similarly, compound \texttt{\$d} statements, after being internally broken up,
internally have their common variable pairs combined.  For example the
set of statements
\begin{center}
  \texttt{\$d x y A \$.}
  \texttt{\$d x y B \$.}
\end{center}
is equivalent to
\begin{center}
  \texttt{\$d x y \$.}
  \texttt{\$d x A \$.}
  \texttt{\$d y A \$.}
  \texttt{\$d x y \$.}
  \texttt{\$d x B \$.}
  \texttt{\$d y B \$.}
\end{center}
which is equivalent to
\begin{center}
  \texttt{\$d x y \$.}
  \texttt{\$d x A \$.}
  \texttt{\$d y A \$.}
  \texttt{\$d x B \$.}
  \texttt{\$d y B \$.}
\end{center}

Metamath\index{Metamath} automatically verifies that all \texttt{\$d}
restrictions are met whenever it verifies proofs.  \texttt{\$d} statements are
never referenced directly in proofs (this is why they do not have
labels\index{label}), but Metamath is always aware of which ones must be
satisfied (i.e.\ are active) and will notify you with an error message if any
violation occurs.

To illustrate how Metamath detects a missing \texttt{\$d}
statement, we will look at the following example from the
\texttt{set.mm} database.

\begin{verbatim}
$d x z $.  $d y z $.
$( Theorem to add distinct quantifier to atomic formula. $)
ax17eq $p |- ( x = y -> A. z x = y ) $=...
\end{verbatim}

This statement has the obvious requirement that $z$ must be
distinct\index{distinct variables} from $x$ in theorem \texttt{ax17eq} that
states $x=y \rightarrow \forall z \, x=y$ (well, obvious if you're a logician,
for otherwise we could conclude  $x=y \rightarrow \forall x \, x=y$, which is
false when the free variables $x$ and $y$ are equal).

Let's look at what happens if we edit the database to comment out this
requirement.

\begin{verbatim}
$( $d x z $. $) $d y z $.
$( Theorem to add distinct quantifier to atomic formula. $)
ax17eq $p |- ( x = y -> A. z x = y ) $=...
\end{verbatim}

When it tries to verify the proof, Metamath will tell you that \texttt{x} and
\texttt{z} must be disjoint, because one of its steps references an axiom or
theorem that has this requirement.

\begin{verbatim}
MM> verify proof ax17eq
ax17eq ?Error at statement 1918, label "ax17eq", type "$p":
      vz wal wi vx vy vz ax-13 vx vy weq vz vx ax-c16 vx vy
                                               ^^^^^
There is a disjoint variable ($d) violation at proof step 29.
Assertion "ax-c16" requires that variables "x" and "y" be
disjoint.  But "x" was substituted with "z" and "y" was
substituted with "x".  The assertion being proved, "ax17eq",
does not require that variables "z" and "x" be disjoint.
\end{verbatim}

We can see the substitutions into \texttt{ax-c16} with the following command.

\begin{verbatim}
MM> show proof ax17eq / detailed_step 29
Proof step 29:  pm2.61dd.2=ax-c16 $a |- ( A. z z = x -> ( x =
  y -> A. z x = y ) )
This step assigns source "ax-c16" ($a) to target "pm2.61dd.2"
($e).  The source assertion requires the hypotheses "wph"
($f, step 26), "vx" ($f, step 27), and "vy" ($f, step 28).
The parent assertion of the target hypothesis is "pm2.61dd"
($p, step 36).
The source assertion before substitution was:
    ax-c16 $a |- ( A. x x = y -> ( ph -> A. x ph ) )
The following substitutions were made to the source
assertion:
    Variable  Substituted with
     x         z
     y         x
     ph        x = y
The target hypothesis before substitution was:
    pm2.61dd.2 $e |- ( ph -> ch )
The following substitutions were made to the target
hypothesis:
    Variable  Substituted with
     ph        A. z z = x
     ch        ( x = y -> A. z x = y )
\end{verbatim}

The disjoint variable restrictions of \texttt{ax-c16} can be seen from the
\texttt{show state\-ment} command.  The line that begins ``\texttt{Its mandatory
dis\-joint var\-i\-able pairs are:}\ldots'' lists any \texttt{\$d} variable
pairs in brackets.

\begin{verbatim}
MM> show statement ax-c16/full
Statement 3033 is located on line 9338 of the file "set.mm".
"Axiom of Distinct Variables. ..."
  ax-c16 $a |- ( A. x x = y -> ( ph -> A. x ph ) ) $.
Its mandatory hypotheses in RPN order are:
  wph $f wff ph $.
  vx $f setvar x $.
  vy $f setvar y $.
Its mandatory disjoint variable pairs are:  <x,y>
The statement and its hypotheses require the variables:  x y
      ph
The variables it contains are:  x y ph
\end{verbatim}

Since Metamath will always detect when \texttt{\$d}\index{\texttt{\$d} statement}
statements are needed for a proof, you don't have to worry too much about
forgetting to put one in; it can always be added if you see the error message
above.  If you put in unnecessary \texttt{\$d} statements, the worst that could
happen is that your theorem might not be as general as it could be, and this
may limit its use later on.

On the other hand, when you introduce axioms (\texttt{\$a}\index{\texttt{\$a}
statement} statements), you must be very careful to properly specify the
necessary associated \texttt{\$d} statements since Metamath has no way of knowing
whether your axioms are correct.  For example, Metamath would have no idea
that \texttt{ax-c16}, which we are telling it is an axiom of logic, would lead to
contradictions if we omitted its associated \texttt{\$d} statement.

% This was previously a comment in footnote-sized type, but it can be
% hard to read this much text in a small size.
% As a result, it's been changed to normally-sized text.
\label{nodd}
You may wonder if it is possible to develop standard
mathematics in the Metamath language without the \texttt{\$d}\index{\texttt{\$d}
statement} statement, since it seems like a nuisance that complicates proof
verification. The \texttt{\$d} statement is not needed in certain subsets of
mathematics such as propositional calculus.  However, dummy
variables\index{dummy variable!eliminating} and their associated \texttt{\$d}
statements are impossible to avoid in proofs in standard first-order logic as
well as in the variant used in \texttt{set.mm}.  In fact, there is no upper bound to
the number of dummy variables that might be needed in a proof of a theorem of
first-order logic containing 3 or more variables, as shown by H.\
Andr\'{e}ka\index{Andr{\'{e}}ka, H.} \cite{Nemeti}.  A first-order system that
avoids them entirely is given in \cite{Megill}\index{Megill, Norman}; the
trick there is simply to embed harmlessly the necessary dummy variables into a
theorem being proved so that they aren't ``dummy'' anymore, then interpret the
resulting longer theorem so as to ignore the embedded dummy variables.  If
this interests you, the system in \texttt{set.mm} obtained from \texttt{ax-1}
through \texttt{ax-c14} in \texttt{set.mm}, and deleting \texttt{ax-c16} and \texttt{ax-5},
requires no \texttt{\$d} statements but is logically complete in the sense
described in \cite{Megill}.  This means it can prove any theorem of
first-order logic as long as we add to the theorem an antecedent that embeds
dummy and any other variables that must be distinct.  In a similar fashion,
axioms for set theory can be devised that
do not require distinct variable
provisos\index{Set theory without distinct variable provisos},
as explained at
\url{http://us.metamath.org/mpeuni/mmzfcnd.html}.
Together, these in principle allow all of
mathematics to be developed under Metamath without a \texttt{\$d} statement,
although the length of the resulting theorems will grow as more and
more dummy variables become required in their proofs.

\subsection{The \texttt{\$f}
and \texttt{\$e} Statements}\label{dollaref}
\index{\texttt{\$e} statement}
\index{\texttt{\$f} statement}
\index{floating hypothesis}
\index{essential hypothesis}
\index{variable-type hypothesis}
\index{logical hypothesis}
\index{hypothesis}

Metamath has two kinds of hypo\-theses, the \texttt{\$f}\index{\texttt{\$f}
statement} or {\bf variable-type} hypothesis and the \texttt{\$e} or {\bf logical}
hypo\-the\-sis.\index{\texttt{\$d} statement}\footnote{Strictly speaking, the
\texttt{\$d} statement is also a hypothesis, but it is never directly referenced
in a proof, so we call it a restriction rather than a hypothesis to lessen
confusion.  The checking for violations of \texttt{\$d} restrictions is automatic
and built into Metamath's proof-checking algorithm.} The letters \texttt{f} and
\texttt{e} stand for ``floating''\index{floating hypothesis} (roughly meaning
used only if relevant) and ``essential''\index{essential hypothesis} (meaning
always used) respectively, for reasons that will become apparent
when we discuss frames in
Section~\ref{frames} and scoping in Section~\ref{scoping}. The syntax of these
are as follows:
\begin{center}
  {\em label} \texttt{\$f} {\em typecode} {\em variable} \texttt{\$.}\\
  {\em label} \texttt{\$e} {\em typecode}
      {\em math-symbol}\ \,$\cdots$\ {\em math-symbol} \texttt{\$.}\\
\end{center}
\index{\texttt{\$e} statement}
\index{\texttt{\$f} statement}
A hypothesis must have a {\em label}\index{label}.  The expression in a
\texttt{\$e} hypothesis consists of a typecode (an active constant math symbol)
followed by a sequence
of zero or more math symbols. Each math symbol (including {\em constant}
and {\em variable}) must be a previously declared constant or variable.  (In
addition, each math symbol must be active, which will be covered when we
discuss scoping statements in Section~\ref{scoping}.)  You use a \texttt{\$f}
hypothesis to specify the
nature or {\bf type}\index{variable type}\index{type} of a variable (such as ``let $x$ be an
integer'') and use a \texttt{\$e} hypothesis to express a logical truth (such as
``assume $x$ is prime'') that must be established in order for an assertion
requiring it to also be true.

A variable must have its type specified in a \texttt{\$f} statement before
it may be used in a \texttt{\$e}, \texttt{\$a}, or \texttt{\$p}
statement.  There may be only one (active) \texttt{\$f} statement for a
given variable.  (``Active'' is defined in Section~\ref{scoping}.)

In ordinary mathematics, theorems\index{theorem} are often expressed in the
form ``Assume $P$; then $Q$,'' where $Q$ is a statement that you can derive
if you start with statement $P$.\index{free variable}\footnote{A stronger
version of a theorem like this would be the {\em single} formula $P\rightarrow
Q$ ($P$ implies $Q$) from which the weaker version above follows by the rule
of modus ponens in logic.  We are not discussing this stronger form here.  In
the weaker form, we are saying only that if we can {\em prove} $P$, then we can
{\em prove} $Q$.  In a logician's language, if $x$ is the only free variable
in $P$ and $Q$, the stronger form is equivalent to $\forall x ( P \rightarrow
Q)$ (for all $x$, $P$ implies $Q$), whereas the weaker form is equivalent to
$\forall x P \rightarrow \forall x Q$. The stronger form implies the weaker,
but not vice-versa.  To be precise, the weaker form of the theorem is more
properly called an ``inference'' rather than a theorem.}\index{inference}
In the
Metamath\index{Metamath} language, you would express mathematical statement
$P$ as a hypothesis (a \texttt{\$e} Metamath language statement in this case) and
statement $Q$ as a provable assertion (a \texttt{\$p}\index{\texttt{\$p} statement}
statement).

Some examples of hypotheses you might encounter in logic and set theory are
\begin{center}
  \texttt{stmt1 \$f wff P \$.}\\
  \texttt{stmt2 \$f setvar x \$.}\\
  \texttt{stmt3 \$e |- ( P -> Q ) \$.}
\end{center}
\index{\texttt{\$e} statement}
\index{\texttt{\$f} statement}
Informally, these would be read, ``Let $P$ be a well-formed-formula,'' ``Let
$x$ be an (individual) variable,'' and ``Assume we have proved $P \rightarrow
Q$.''  The turnstile symbol \,$\vdash$\index{turnstile ({$\,\vdash$})} is
commonly used in logic texts to mean ``a proof exists for.''

To summarize:
\begin{itemize}
\item A \texttt{\$f} hypothesis tells Metamath the type or kind of its variable.
It is analogous to a variable declaration in a computer language that
tells the compiler that a variable is an integer or a floating-point
number.
\item The \texttt{\$e} hypothesis corresponds to what you would usually call a
``hypothesis'' in ordinary mathematics.
\end{itemize}

Before an assertion\index{assertion} (\texttt{\$a} or \texttt{\$p} statement) can be
referenced in a proof, all of its associated \texttt{\$f} and \texttt{\$e} hypotheses
(i.e.\ those \texttt{\$e} hypotheses that are active) must be satisfied (i.e.
established by the proof).  The meaning of ``associated'' (which we will call
{\bf mandatory} in Section~\ref{frames}) will become clear when we discuss
scoping later.

Note that after any \texttt{\$f}, \texttt{\$e},
\texttt{\$a}, or \texttt{\$p} token there is a required
\textit{typecode}\index{typecode}.
The typecode is a constant used to enforce types of expressions.
This will become clearer once we learn more about
assertions (\texttt{\$a} and \texttt{\$p} statements).
An example may also clarify their purpose.
In the
\texttt{set.mm}\index{set theory database (\texttt{set.mm})}%
\index{Metamath Proof Explorer}
database,
the following typecodes are used:

\begin{itemize}
\item \texttt{wff} :
  Well-formed formula (wff) symbol
  (read: ``the following symbol sequence is a wff'').
% The *textual* typecode for turnstile is "|-", but when read it's a little
% confusing, so I intentionally display the mathematical symbol here instead
% (I think it's clearer in this context).
\item \texttt{$\vdash$} :
  Turnstile (read: ``the following symbol sequence is provable'' or
  ``a proof exists for'').
\item \texttt{setvar} :
  Individual set variable type (read: ``the following is an
  individual set variable'').
  Note that this is \textit{not} the type of an arbitrary set expression,
  instead, it is used to ensure that there is only a single symbol used
  after quantifiers like for-all ($\forall$) and there-exists ($\exists$).
\item \texttt{class} :
  An expression that is a syntactically valid class expression.
  All valid set expressions are also valid class expression, so expressions
  of sets normally have the \texttt{class} typecode.
  Use the \texttt{class} typecode,
  \textit{not} the \texttt{setvar} typecode,
  for the type of set expressions unless you are specifically identifying
  a single set variable.
\end{itemize}

\subsection{Assertions (\texttt{\$a} and \texttt{\$p} Statements)}
\index{\texttt{\$a} statement}
\index{\texttt{\$p} statement}\index{assertion}\index{axiomatic assertion}
\index{provable assertion}

There are two types of assertions, \texttt{\$a}\index{\texttt{\$a} statement}
statements ({\bf axiomatic assertions}) and \texttt{\$p} statements ({\bf
provable assertions}).  Their syntax is as follows:
\begin{center}
  {\em label} \texttt{\$a} {\em typecode} {\em math-symbol} \ldots
         {\em math-symbol} \texttt{\$.}\\
  {\em label} \texttt{\$p} {\em typecode} {\em math-symbol} \ldots
        {\em math-symbol} \texttt{\$=} {\em proof} \texttt{\$.}
\end{center}
\index{\texttt{\$a} statement}
\index{\texttt{\$p} statement}
\index{\texttt{\$=} keyword}
An assertion always requires a {\em label}\index{label}. The expression in an
assertion consists of a typecode (an active constant)
followed by a sequence of zero
or more math symbols.  Each math symbol, including any {\em constant}, must be a
previously declared constant or variable.  (In addition, each math symbol
must be active, which will be covered when we discuss scoping statements in
Section~\ref{scoping}.)

A \texttt{\$a} statement is usually a definition of syntax (for example, if $P$
and $Q$ are wffs then so is $(P\to Q)$), an axiom\index{axiom} of ordinary
mathematics (for example, $x=x$), or a definition\index{definition} of
ordinary mathematics (for example, $x\ne y$ means $\lnot x=y$). A \texttt{\$p}
statement is a claim that a certain combination of math symbols follows from
previous assertions and is accompanied by a proof that demonstrates it.

Assertions can also be referenced in (later) proofs in order to derive new
assertions from them. The label of an assertion is used to refer to it in a
proof. Section~\ref{proof} will describe the proof in detail.

Assertions also provide the primary means for communicating the mathematical
results in the database to people.  Proofs (when conveniently displayed)
communicate to people how the results were arrived at.

\subsubsection{The \texttt{\$a} Statement}
\index{\texttt{\$a} statement}

Axiomatic assertions (\texttt{\$a} statements) represent the starting points from
which other assertions (\texttt{\$p}\index{\texttt{\$p} statement} statements) are
derived.  Their most obvious use is for specifying ordinary mathematical
axioms\index{axiom}, but they are also used for two other purposes.

First, Metamath\index{Metamath} needs to know the syntax of symbol
sequences that constitute valid mathematical statements.  A Metamath
proof must be broken down into much more detail than ordinary
mathematical proofs that you may be used to thinking of (even the
``complete'' proofs of formal logic\index{formal logic}).  This is one
of the things that makes Metamath a general-purpose language,
independent of any system of logic or even syntax.  If you want to use a
substitution instance of an assertion as a step in a proof, you must
first prove that the substitution is syntactically correct (or if you
prefer, you must ``construct'' it), showing for example that the
expression you are substituting for a wff metavariable is a valid wff.
The \texttt{\$a}\index{\texttt{\$a} statement} statement is used to
specify those combinations of symbols that are considered syntactically
valid, such as the legal forms of wffs.

Second, \texttt{\$a} statements are used to specify what are ordinarily thought of
as definitions, i.e.\ new combinations of symbols that abbreviate other
combinations of symbols.  Metamath makes no distinction\index{axiom vs.\
definition} between axioms\index{axiom} and definitions\index{definition}.
Indeed, it has been argued that such distinction should not be made even in
ordinary mathematics; see Section~\ref{definitions}, which discusses the
philosophy of definitions.  Section~\ref{hierarchy} discusses some
technical requirements for definitions.  In \texttt{set.mm} we adopt the
convention of prefixing axiom labels with \texttt{ax-} and definition labels with
\texttt{df-}\index{label}.

The results that can be derived with the Metamath language are only as good as
the \texttt{\$a}\index{\texttt{\$a} statement} statements used as their starting
point.  We cannot stress this too strongly.  For example, Metamath will
not prevent you from specifying $x\neq x$ as an axiom of logic.  It is
essential that you scrutinize all \texttt{\$a} statements with great care.
Because they are a source of potential pitfalls, it is best not to add new
ones (usually new definitions) casually; rather you should carefully evaluate
each one's necessity and advantages.

Once you have in place all of the basic axioms\index{axiom} and
rules\index{rule} of a mathematical theory, the only \texttt{\$a} statements that
you will be adding will be what are ordinarily called definitions.  In
principle, definitions should be in some sense eliminable from the language of
a theory according to some convention (usually involving logical equivalence
or equality).  The most common convention is that any formula that was
syntactically valid but not provable before the definition was introduced will
not become provable after the definition is introduced.  In an ideal world,
definitions should not be present at all if one is to have absolute confidence
in a mathematical result.  However, they are necessary to make
mathematics practical, for otherwise the resulting formulas would be
extremely long and incomprehensible.  Since the nature of definitions (in the
most general sense) does not permit them to automatically be verified as
``proper,''\index{proper definition}\index{definition!proper} the judgment of
the mathematician is required to ensure it.  (In \texttt{set.mm} effort was made
to make almost all definitions directly eliminable and thus minimize the need
for such judgment.)

If you are not a mathematician, it may be best not to add or change any
\texttt{\$a}\index{\texttt{\$a} statement} statements but instead use
the mathematical language already provided in standard databases.  This
way Metamath will not allow you to make a mistake (i.e.\ prove a false
result).


\subsection{Frames}\label{frames}

We now introduce the concept of a collection of related Metamath statements
called a frame.  Every assertion (\texttt{\$a} or \texttt{\$p} statement) in the database has
an associated frame.

A {\bf frame}\index{frame} is a sequence of \texttt{\$d}, \texttt{\$f},
and \texttt{\$e} statements (zero or more of each) followed by one
\texttt{\$a} or \texttt{\$p} statement, subject to certain conditions we
will describe.  For simplicity we will assume that all math symbol
tokens used are declared at the beginning of the database with
\texttt{\$c} and \texttt{\$v} statements (which are not properly part of
a frame).  Also for simplicity we will assume there are only simple
\texttt{\$d} statements (those with only two variables) and imagine any
compound \texttt{\$d} statements (those with more than two variables) as
broken up into simple ones.

A frame groups together those hypotheses (and \texttt{\$d} statements) relevant
to an assertion (\texttt{\$a} or \texttt{\$p} statement).  The statements in a frame
may or may not be physically adjacent in a database; we will cover
this in our discussion of scoping statements
in Section~\ref{scoping}.

A frame has the following properties:
\begin{enumerate}
 \item The set of variables contained in its \texttt{\$f} statements must
be identical to the set of variables contained in its \texttt{\$e},
\texttt{\$a}, and/or \texttt{\$p} statements.  In other words, each
variable in a \texttt{\$e}, \texttt{\$a}, or \texttt{\$p} statement must
have an associated ``variable type'' defined for it in a \texttt{\$f}
statement.
  \item No two \texttt{\$f} statements may contain the same variable.
  \item Any \texttt{\$f} statement
must occur before a \texttt{\$e} statement in which its variable occurs.
\end{enumerate}

The first property determines the set of variables occurring in a frame.
These are the {\bf mandatory
variables}\index{mandatory variable} of the frame.  The second property
tells us there must be only one type specified for a variable.
The last property is not a theoretical requirement but it
makes parsing of the database easier.

For our examples, we assume our database has the following declarations:

\begin{verbatim}
$v P Q R $.
$c -> ( ) |- wff $.
\end{verbatim}

The following sequence of statements, describing the modus ponens inference
rule, is an example of a frame:

\begin{verbatim}
wp  $f wff P $.
wq  $f wff Q $.
maj $e |- ( P -> Q ) $.
min $e |- P $.
mp  $a |- Q $.
\end{verbatim}

The following sequence of statements is not a frame because \texttt{R} does not
occur in the \texttt{\$e}'s or the \texttt{\$a}:

\begin{verbatim}
wp  $f wff P $.
wq  $f wff Q $.
wr  $f wff R $.
maj $e |- ( P -> Q ) $.
min $e |- P $.
mp  $a |- Q $.
\end{verbatim}

The following sequence of statements is not a frame because \texttt{Q} does not
occur in a \texttt{\$f}:

\begin{verbatim}
wp  $f wff P $.
maj $e |- ( P -> Q ) $.
min $e |- P $.
mp  $a |- Q $.
\end{verbatim}

The following sequence of statements is not a frame because the \texttt{\$a} statement is
not the last one:

\begin{verbatim}
wp  $f wff P $.
wq  $f wff Q $.
maj $e |- ( P -> Q ) $.
mp  $a |- Q $.
min $e |- P $.
\end{verbatim}

Associated with a frame is a sequence of {\bf mandatory
hypotheses}\index{mandatory hypothesis}.  This is simply the set of all
\texttt{\$f} and \texttt{\$e} statements in the frame, in the order they
appear.  A frame can be referenced in a later proof using the label of
the \texttt{\$a} or \texttt{\$p} assertion statement, and the proof
makes an assignment to each mandatory hypothesis in the order in which
it appears.  This means the order of the hypotheses, once chosen, must
not be changed so as not to affect later proofs referencing the frame's
assertion statement.  (The Metamath proof verifier will, of course, flag
an error if a proof becomes incorrect by doing this.)  Since proofs make
use of ``Reverse Polish notation,'' described in Section~\ref{proof}, we
call this order the {\bf RPN order}\index{RPN order} of the hypotheses.

Note that \texttt{\$d} statements are not part of the set of mandatory
hypotheses, and their order doesn't matter (as long as they satisfy the
fourth property for a frame described above).  The \texttt{\$d}
statements specify restrictions on variables that must be satisfied (and
are checked by the proof verifier) when expressions are substituted for
them in a proof, and the \texttt{\$d} statements themselves are never
referenced directly in a proof.

A frame with a \texttt{\$p} (provable) statement requires a proof as part of the
\texttt{\$p} statement.  Sometimes in a proof we want to make use of temporary or
dummy variables\index{dummy variable} that do not occur in the \texttt{\$p}
statement or its mandatory hypotheses.  To accommodate this we define an {\bf
extended frame}\index{extended frame} as a frame together with zero or more
\texttt{\$d} and \texttt{\$f} statements that reference variables not among the
mandatory variables of the frame.  Any new variables referenced are called the
{\bf optional variables}\index{optional variable} of the extended frame. If a
\texttt{\$f} statement references an optional variable it is called an {\bf
optional hypothesis}\index{optional hypothesis}, and if one or both of the
variables in a \texttt{\$d} statement are optional variables it is called an {\bf
optional disjoint-variable restriction}\index{optional disjoint-variable
restriction}.  Properties 2 and 3 for a frame also apply to an extended
frame.

The concept of optional variables is not meaningful for frames with \texttt{\$a}
statements, since those statements have no proofs that might make use of them.
There is no restriction on including optional hypotheses in the extended frame
for a \texttt{\$a} statement, but they serve no purpose.

The following set of statements is an example of an extended frame, which
contains an optional variable \texttt{R} and an optional hypothesis \texttt{wr}.  In
this example, we suppose the rule of modus ponens is not an axiom but is
derived as a theorem from earlier statements (we omit its presumed proof).
Variable \texttt{R} may be used in its proof if desired (although this would
probably have no advantage in propositional calculus).  Note that the sequence
of mandatory hypotheses in RPN order is still \texttt{wp}, \texttt{wq}, \texttt{maj},
\texttt{min} (i.e.\ \texttt{wr} is omitted), and this sequence is still assumed
whenever the assertion \texttt{mp} is referenced in a subsequent proof.

\begin{verbatim}
wp  $f wff P $.
wq  $f wff Q $.
wr  $f wff R $.
maj $e |- ( P -> Q ) $.
min $e |- P $.
mp  $p |- Q $= ... $.
\end{verbatim}

Every frame is an extended frame, but not every extended frame is a frame, as
this example shows.  The underlying frame for an extended frame is
obtained by simply removing all statements containing optional variables.
Any proof referencing an assertion will ignore any extensions to its
frame, which means we may add or delete optional hypotheses at will without
affecting subsequent proofs.

The conceptually simplest way of organizing a Metamath database is as a
sequence of extended frames.  The scoping statements
\texttt{\$\char`\{}\index{\texttt{\$\char`\{} and \texttt{\$\char`\}}
keywords} and \texttt{\$\char`\}} can be used to delimit the start and
end of an extended frame, leading to the following possible structure for a
database.  \label{framelist}

\vskip 2ex
\setbox\startprefix=\hbox{\tt \ \ \ \ \ \ \ \ }
\setbox\contprefix=\hbox{}
\startm
\m{\mbox{(\texttt{\$v} {\em and} \texttt{\$c}\,{\em statements})}}
\endm
\startm
\m{\mbox{\texttt{\$\char`\{}}}
\endm
\startm
\m{\mbox{\texttt{\ \ } {\em extended frame}}}
\endm
\startm
\m{\mbox{\texttt{\$\char`\}}}}
\endm
\startm
\m{\mbox{\texttt{\$\char`\{}}}
\endm
\startm
\m{\mbox{\texttt{\ \ } {\em extended frame}}}
\endm
\startm
\m{\mbox{\texttt{\$\char`\}}}}
\endm
\startm
\m{\mbox{\texttt{\ \ \ \ \ \ \ \ \ }}\vdots}
\endm
\vskip 2ex

In practice, this structure is inconvenient because we have to repeat
any \texttt{\$f}, \texttt{\$e}, and \texttt{\$d} statements over and
over again rather than stating them once for use by several assertions.
The scoping statements, which we will discuss next, allow this to be
done.  In principle, any Metamath database can be converted to the above
format, and the above format is the most convenient to use when studying
a Metamath database as a formal system%
%% Uncomment this when uncommenting section {formalspec} below
   (Appendix \ref{formalspec})%
.
In fact, Metamath internally converts the database to the above format.
The command \texttt{show statement} in the Metamath program will show
you the contents of the frame for any \texttt{\$a} or \texttt{\$p}
statement, as well as its extension in the case of a \texttt{\$p}
statement.

%c%(provided that all ``local'' variables and constants with limited scope have
%c%unique names),

During our discussion of scoping statements, it may be helpful to
think in terms of the equivalent sequence of frames that will result when
the database is parsed.  Scoping (other than the limited
use above to delimit frames) is not a theoretical requirement for
Metamath but makes it more convenient.


\subsection{Scoping Statements (\texttt{\$\{} and \texttt{\$\}})}\label{scoping}
\index{\texttt{\$\char`\{} and \texttt{\$\char`\}} keywords}\index{scoping statement}

%c%Some Metamath statements may be needed only temporarily to
%c%serve a specific purpose, and after we're done with them we would like to
%c%disregard or ignore them.  For example, when we're finished using a variable,
%c%we might want to
%c%we might want to free up the token\index{token} used to name it so that the
%c%token can be used for other purposes later on, such as a different kind of
%c%variable or even a constant.  In the terminology of computer programming, we
%c%might want to let some symbol declarations be ``local'' rather than ``global.''
%c%\index{local symbol}\index{global symbol}

The {\bf scoping} statements, \texttt{\$\char`\{} ({\bf start of block}) and \texttt{\$\char`\}}
({\bf end of block})\index{block}, provide a means for controlling the portion
of a database over which certain statement types are recognized.  The
syntax of a scoping statement is very simple; it just consists of the
statement's keyword:
\begin{center}
\texttt{\$\char`\{}\\
\texttt{\$\char`\}}
\end{center}
\index{\texttt{\$\char`\{} and \texttt{\$\char`\}} keywords}

For example, consider the following database where we have stripped out
all tokens except the scoping statement keywords.  For the purpose of the
discussion, we have added subscripts to the scoping statements; these subscripts
do not appear in the actual database.
\[
 \mbox{\tt \ \$\char`\{}_1
 \mbox{\tt \ \$\char`\{}_2
 \mbox{\tt \ \$\char`\}}_2
 \mbox{\tt \ \$\char`\{}_3
 \mbox{\tt \ \$\char`\{}_4
 \mbox{\tt \ \$\char`\}}_4
 \mbox{\tt \ \$\char`\}}_3
 \mbox{\tt \ \$\char`\}}_1
\]
Each \texttt{\$\char`\{} statement in this example is said to be {\bf
matched} with the \texttt{\$\char`\}} statement that has the same
subscript.  Each pair of matched scoping statements defines a region of
the database called a {\bf block}.\index{block} Blocks can be {\bf
nested}\index{nested block} inside other blocks; in the example, the
block defined by $\mbox{\tt \$\char`\{}_4$ and $\mbox{\tt \$\char`\}}_4$
is nested inside the block defined by $\mbox{\tt \$\char`\{}_3$ and
$\mbox{\tt \$\char`\}}_3$ as well as inside the block defined by
$\mbox{\tt \$\char`\{}_1$ and $\mbox{\tt \$\char`\}}_1$.  In general, a
block may be empty, it may contain only non-scoping
statements,\footnote{Those statements other than \texttt{\$\char`\{} and
\texttt{\$\char`\}}.}\index{non-scoping statement} or it may contain any
mixture of other blocks and non-scoping statements.  (This is called a
``recursive'' definition\index{recursive definition} of a block.)

Associated with each block is a number called its {\bf nesting
level}\index{nesting level} that indicates how deeply the block is nested.
The nesting levels of the blocks in our example are as follows:
\[
  \underbrace{
    \mbox{\tt \ }
    \underbrace{
     \mbox{\tt \$\char`\{\ }
     \underbrace{
       \mbox{\tt \$\char`\{\ }
       \mbox{\tt \$\char`\}}
     }_{2}
     \mbox{\tt \ }
     \underbrace{
       \mbox{\tt \$\char`\{\ }
       \underbrace{
         \mbox{\tt \$\char`\{\ }
         \mbox{\tt \$\char`\}}
       }_{3}
       \mbox{\tt \ \$\char`\}}
     }_{2}
     \mbox{\tt \ \$\char`\}}
   }_{1}
   \mbox{\tt \ }
 }_{0}
\]
\index{\texttt{\$\char`\{} and \texttt{\$\char`\}} keywords}
The entire database is considered to be one big block (the {\bf outermost}
block) with a nesting level of 0.  The outermost block is {\em not} bracketed
by scoping statements.\footnote{The language was designed this way so that
several source files can be joined together more easily.}\index{outermost
block}

All non-scoping Metamath statements become recognized or {\bf
active}\index{active statement} at the place where they appear.\footnote{To
keep things slightly simpler, we do not bother to define the concept of
``active'' for the scoping statements.}  Certain of these statement types
become inactive at the end of the block in which they appear; these statement
types are:
\begin{center}
  \texttt{\$c}, \texttt{\$v}, \texttt{\$d}, \texttt{\$e}, and \texttt{\$f}.
%  \texttt{\$v}, \texttt{\$f}, \texttt{\$e}, and \texttt{\$d}.
\end{center}
\index{\texttt{\$c} statement}
\index{\texttt{\$d} statement}
\index{\texttt{\$e} statement}
\index{\texttt{\$f} statement}
\index{\texttt{\$v} statement}
The other statement types remain active forever (i.e.\ through the end of the
database); they are:
\begin{center}
  \texttt{\$a} and \texttt{\$p}.
%  \texttt{\$c}, \texttt{\$a}, and \texttt{\$p}.
\end{center}
\index{\texttt{\$a} statement}
\index{\texttt{\$p} statement}
Any statement (of these 7 types) located in the outermost
block\index{outermost block} will remain active through the end of the
database and thus are effectively ``global'' statements.\index{global
statement}

All \texttt{\$c} statements must be placed in the outermost block.  Since they are
therefore always global, they could be considered as belonging to both of the
above categories.

The {\bf scope}\index{scope} of a statement is the set of statements that
recognize it as active.

%c%The concept of ``active'' is also defined for math symbols\index{math
%c%symbol}.  Math symbols (constants\index{constant} and
%c%variables\index{variable}) become {\bf active}\index{active
%c%math symbol} in the \texttt{\$c}\index{\texttt{\$c}
%c%statement} and \texttt{\$v}\index{\texttt{\$v} statement} statements that
%c%declare them.  They become inactive when their declaration statements become
%c%inactive.

The concept of ``active'' is also defined for math symbols\index{math
symbol}.  Math symbols (constants\index{constant} and
variables\index{variable}) become {\bf active}\index{active math symbol}
in the \texttt{\$c}\index{\texttt{\$c} statement} and
\texttt{\$v}\index{\texttt{\$v} statement} statements that declare them.
A variable becomes inactive when its declaration statement becomes
inactive.  Because all \texttt{\$c} statements must be in the outermost
block, a constant will never become inactive after it is declared.

\subsubsection{Redeclaration of Math Symbols}
\index{redeclaration of symbols}\label{redeclaration}

%c%A math symbol may not be declared a second time while it is active, but it may
%c%be declared again after it becomes inactive.

A variable may not be declared a second time while it is active, but it may be
declared again after it becomes inactive.  This provides a convenient way to
introduce ``local'' variables,\index{local variable} i.e.\ temporary variables
for use in the frame of an assertion or in a proof without keeping them around
forever.  A previously declared variable may not be redeclared as a constant.

A constant may not be redeclared.  And, as mentioned above, constants must be
declared in the outermost block.

The reason variables may have limited scope but not constants is that an
assertion (\texttt{\$a} or \texttt{\$p} statement) remains available for use in
proofs through the end of the database.  Variables in an assertion's frame may
be substituted with whatever is needed in a proof step that references the
assertion, whereas constants remain fixed and may not be substituted with
anything.  The particular token used for a variable in an assertion's frame is
irrelevant when the assertion is referenced in a proof, and it doesn't matter
if that token is not available outside of the referenced assertion's frame.
Constants, however, must be globally fixed.

There is no theoretical
benefit for the feature allowing variables to be active for limited scopes
rather than global. It is just a convenience that allows them, for example, to
be locally grouped together with their corresponding \texttt{\$f} variable-type
declarations.

%c%If you declare a math symbol more than once, internally Metamath considers it a
%c%new distinct symbol, even though it has the same name.  If you are unaware of
%c%this, you may find that what you think are correct proofs are incorrectly
%c%rejected as invalid, because Metamath may tell you that a constant you
%c%previously declared does not match a newly declared math symbol with the same
%c%name.  For details on this subtle point, see the Comment on
%c%p.~\pageref{spec4comment}.  This is done purposely to allow temporary
%c%constants to be introduced while developing a subtheory, then allow their math
%c%symbol tokens to be reused later on; in general they will not refer to the
%c%same thing.  In practice, you would not ordinarily reuse the names of
%c%constants because it would tend to be confusing to the reader.  The reuse of
%c%names of variables, on the other hand, is something that is often useful to do
%c%(for example it is done frequently in \texttt{set.mm}).  Since variables in an
%c%assertion referenced in a proof can be substituted as needed to achieve a
%c%symbol match, this is not an issue.

% (This section covers a somewhat advanced topic you may want to skip
% at first reading.)
%
% Under certain circumstances, math symbol\index{math symbol}
% tokens\index{token} may be redeclared (i.e.\ the token
% may appear in more than
% one \texttt{\$c}\index{\texttt{\$c} statement} or \texttt{\$v}\index{\texttt{\$v}
% statement} statement).  You might want to do this say, to make temporary use
% of a variable name without having to worry about its affect elsewhere,
% somewhat analogous to declaring a local variable in a standard computer
% language.  Understanding what goes on when math symbol tokens are redeclared
% is a little tricky to understand at first, since it requires that we
% distinguish the token itself from the math symbol that it names.  It will help
% if we first take a peek at the internal workings of the
% Metamath\index{Metamath} program.
%
% Metamath reserves a memory location for each occurrence of a
% token\index{token} in a declaration statement (\texttt{\$c}\index{\texttt{\$c}
% statement} or \texttt{\$v}\index{\texttt{\$v} statement}).  If a given token appears
% in more than one declaration statement, it will refer to more than one memory
% locations.  A math symbol\index{math symbol} may be thought of as being one of
% these memory locations rather than as the token itself.  Only one of the
% memory locations associated with a given token may be active at any one time.
% The math symbol (memory location) that gets looked up when the token appears
% in a non-declaration statement is the one that happens to be active at that
% time.
%
% We now look at the rules for the redeclaration\index{redeclaration of symbols}
% of math symbol tokens.
% \begin{itemize}
% \item A math symbol token may not be declared twice in the
% same block.\footnote{While there is no theoretical reason for disallowing
% this, it was decided in the design of Metamath that allowing it would offer no
% advantage and might cause confusion.}
% \item An inactive math symbol may always be
% redeclared.
% \item  An active math symbol may be redeclared in a different (i.e.\
% inner) block\index{block} from the one it became active in.
% \end{itemize}
%
% When a math symbol token is redeclared, it conceptually refers to a different
% math symbol, just as it would be if it were called a different name.  In
% addition, the original math symbol that it referred to, if it was active,
% temporarily becomes inactive.  At the end of the block in which the
% redeclaration occurred, the new math symbol\index{math symbol} becomes
% inactive and the original symbol becomes active again.  This concept is
% illustrated in the following example, where the symbol \texttt{e} is
% ordinarily a constant (say Euler's constant, 2.71828...) but
% temporarily we want to use it as a ``local'' variable, say as a coefficient
% in the equation $a x^4 + b x^3 + c x^2 + d x + e$:
% \[
%   \mbox{\tt \$\char`\{\ \$c e \$.}
%   \underbrace{
%     \ \ldots\ %
%     \mbox{\tt \$\char`\{}\ \ldots\ %
%   }_{\mbox{\rm region A}}
%   \mbox{\tt \$v e \$.}
%   \underbrace{
%     \mbox{\ \ \ \ldots\ \ \ }
%   }_{\mbox{\rm region B}}
%   \mbox{\tt \$\char`\}}
%   \underbrace{
%     \mbox{\ \ \ \ldots\ \ \ }
%   }_{\mbox{\rm region C}}
%   \mbox{\tt \$\char`\}}
% \]
% \index{\texttt{\$\char`\{} and \texttt{\$\char`\}} keywords}
% In region A, the token \texttt{e} refers to a constant.  It is redeclared as a
% variable in region B, and any reference to it in this region will refer to this
% variable.  In region C, the redeclaration becomes inactive, and the original
% declaration becomes active again.  In region C, the token \texttt{x} refers to the
% original constant.
%
% As a practical matter, overuse of math symbol\index{math symbol}
% redeclarations\index{redeclaration of symbols} can be confusing (even though
% it is well-defined) and is best avoided when possible.  Here are some good
% general guidelines you can follow.  Usually, you should declare all
% constants\index{constant} in the outermost block\index{outermost block},
% especially if they are general-purpose (such as the token \verb$A.$, meaning
% $\forall$ or ``for all'').  This will make them ``globally'' active (although
% as in the example above local redeclarations will temporarily make them
% inactive.)  Most or all variables\index{variable}, on the other hand, could be
% declared in inner blocks, so that the token for them can be used later for a
% different type of variable or a constant.  (The names of the variables you
% choose are not used when you refer to an assertion\index{assertion} in a
% proof, whereas constants must match exactly.  A locally declared constant will
% not match a globally declared constant in a proof, even if they use the same
% token, because Metamath internally considers them to be different math
% symbols.)  To avoid confusion, you should generally avoid redeclaring active
% variables.  If you must redeclare them, do so at the beginning of a block.
% The temporary declaration of constants in inner blocks might be occasionally
% appropriate when you make use of a temporary definition to prove lemmas
% leading to a main result that does not make direct use of the definition.
% This way, you will not clutter up your database with a large number of
% seldom-used global constant symbols.  You might want to note that while
% inactive constants may not appear directly in an assertion (a \texttt{\$a}\index{\texttt{\$a}
% statement} or \texttt{\$p}\index{\texttt{\$p} statement}
% statement), they may be indirectly used in the proof of a \texttt{\$p} statement
% so long as they do not appear in the final math symbol sequence constructed by
% the proof.  In the end, you will have to use your best judgment, taking into
% account standard mathematical usage of the symbols as well as consideration
% for the reader of your work.
%
% \subsubsection{Reuse of Labels}\index{reuse of labels}\index{label}
%
% The \texttt{\$e}\index{\texttt{\$e} statement}, \texttt{\$f}\index{\texttt{\$f}
% statement}, \texttt{\$a}\index{\texttt{\$a} statement}, and
% \texttt{\$p}\index{\texttt{\$p}
% statement} statement types require labels, which allow them to be
% referenced later inside proofs.  A label is considered {\bf
% active}\index{active label} when the statement it is associated with is
% active.  The token\index{token} for a label may be reused
% (redeclared)\index{redeclaration of labels} provided that it is not being used
% for a currently active label.  (Unlike the tokens for math symbols, active
% label tokens may not be redeclared in an inner scope.)  Note that the labels
% of \texttt{\$a} and \texttt{\$p} statements can never be reused after these
% statements appear, because these statements remain active through the end of
% the database.
%
% You might find the reuse of labels a convenient way to have standard names for
% temporary hypotheses, such as \texttt{h1}, \texttt{h2}, etc.  This way you don't have
% to invent unique names for each of them, and in some cases it may be less
% confusing to the reader (although in other cases it might be more confusing, if
% the hypothesis is located far away from the assertion that uses
% it).\footnote{The current implementation requires that all labels, even
% inactive ones, be unique.}

\subsubsection{Frames Revisited}\index{frames and scoping statements}

Now that we have covered scoping, we will look at how an arbitrary
Metamath database can be converted to the simple sequence of extended
frames described on p.~\pageref{framelist}.  This is also how Metamath
stores the database internally when it reads in the database
source.\label{frameconvert} The method is simple.  First, we collect all
constant and variable (\texttt{\$c} and \texttt{\$v}) declarations in
the database, ignoring duplicate declarations of the same variable in
different scopes.  We then put our collected \texttt{\$c} and
\texttt{\$v} declarations at the beginning of the database, so that
their scope is the entire database.  Next, for each assertion in the
database, we determine its frame and extended frame.  The extended frame
is simply the \texttt{\$f}, \texttt{\$e}, and \texttt{\$d} statements
that are active.  The frame is the extended frame with all optional
hypotheses removed.

An equivalent way of saying this is that the extended frame of an assertion
is the collection of all \texttt{\$f}, \texttt{\$e}, and \texttt{\$d} statements
whose scope includes the assertion.
The \texttt{\$f} and \texttt{\$e} statements
occur in the order they appear
(order is irrelevant for \texttt{\$d} statements).

%c%, renaming any
%c%redeclared variables as needed so that all of them have unique names.  (The
%c%exact renaming convention is unimportant.  You might imagine renaming
%c%different declarations of math symbol \texttt{a} as \texttt{a\$1}, \texttt{a\$2}, etc.\
%c%which would prevent any conflicts since \texttt{\$} is not a legal character in a
%c%math symbol token.)

\section{The Anatomy of a Proof} \label{proof}
\index{proof!Metamath, description of}

Each provable assertion (\texttt{\$p}\index{\texttt{\$p} statement} statement) in a
database must include a {\bf proof}\index{proof}.  The proof is located
between the \texttt{\$=}\index{\texttt{\$=} keyword} and \texttt{\$.}\ keywords in the
\texttt{\$p} statement.

In the basic Metamath language\index{basic language}, a proof is a
sequence of statement labels.  This label sequence\index{label sequence}
serves as a set of instructions that the Metamath program uses to
construct a series of math symbol sequences.  The construction must
ultimately result in the math symbol sequence contained between the
\texttt{\$p}\index{\texttt{\$p} statement} and
\texttt{\$=}\index{\texttt{\$=} keyword} keywords of the \texttt{\$p}
statement.  Otherwise, the Metamath program will consider the proof
incorrect, and it will notify you with an appropriate error message when
you ask it to verify the proof.\footnote{To make the loading faster, the
Metamath program does not automatically verify proofs when you
\texttt{read} in a database unless you use the \texttt{/verify}
qualifier.  After a database has been read in, you may use the
\texttt{verify proof *} command to verify proofs.}\index{\texttt{verify
proof} command} Each label in a proof is said to {\bf
reference}\index{label reference} its corresponding statement.

Associated with any assertion\index{assertion} (\texttt{\$p} or
\texttt{\$a}\index{\texttt{\$a} statement} statement) is a set of
hypotheses (\texttt{\$f}\index{\texttt{\$f} statement} or
\texttt{\$e}\index{\texttt{\$e} statement} statements) that are active
with respect to that assertion.  Some are mandatory and the others are
optional.  You should review these concepts if necessary.

Each label\index{label} in a proof must be either the label of a
previous assertion (\texttt{\$a}\index{\texttt{\$a} statement} or
\texttt{\$p}\index{\texttt{\$p} statement} statement) or the label of an
active hypothesis (\texttt{\$e} or \texttt{\$f}\index{\texttt{\$f}
statement} statement) of the \texttt{\$p} statement containing the
proof.  Hypothesis labels may reference both the
mandatory\index{mandatory hypothesis} and the optional hypotheses of the
\texttt{\$p} statement.

The label sequence in a proof specifies a construction in {\bf reverse Polish
notation}\index{reverse Polish notation (RPN)} (RPN).  You may be familiar
with RPN if you have used older
Hewlett--Packard or similar hand-held calculators.
In the calculator analogy, a hypothesis label\index{hypothesis label} is like
a number and an assertion label\index{assertion label} is like an operation
(more precisely, an $n$-ary operation when the
assertion has $n$ \texttt{\$e}-hypotheses).
On an RPN calculator, an operation takes one or more previous numbers in an
input sequence, performs a calculation on them, and replaces those numbers and
itself with the result of the calculation.  For example, the input sequence
$2,3,+$ on an RPN calculator results in $5$, and the input sequence
$2,3,5,{\times},+$ results in $2,15,+$ which results in $17$.

Understanding how RPN is processed involves the concept of a {\bf
stack}\index{stack}\index{RPN stack}, which can be thought of as a set of
temporary memory locations that hold intermediate results.  When Metamath
encounters a hypothesis label it places or {\bf pushes}\index{push} the math
symbol sequence of the hypothesis onto the stack.  When Metamath encounters an
assertion label, it associates the most recent stack entries with the {\em
mandatory} hypotheses\index{mandatory hypothesis} of the assertion, in the
order where the most recent stack entry is associated with the last mandatory
hypothesis of the assertion.  It then determines what
substitutions\index{substitution!variable}\index{variable substitution} have
to be made into the variables of the assertion's mandatory hypotheses to make
them identical to the associated stack entries.  It then makes those same
substitutions into the assertion itself.  Finally, Metamath removes or {\bf
pops}\index{pop} the matched hypotheses from the stack and pushes the
substituted assertion onto the stack.

For the purpose of matching the mandatory hypothesis to the most recent stack
entries, whether a hypothesis is a \texttt{\$e} or \texttt{\$f} statement is
irrelevant.  The only important thing is that a set of
substitutions\footnote{In the Metamath spec (Section~\ref{spec}), we use the
singular term ``substitution'' to refer to the set of substitutions we talk
about here.} exist that allow a match (and if they don't, the proof verifier
will let you know with an error message).  The Metamath language is specified
in such a way that if a set of substitutions exists, it will be unique.
Specifically, the requirement that each variable have a type specified for it
with a \texttt{\$f} statement ensures the uniqueness.

We will illustrate this with an example.
Consider the following Metamath source file:
\begin{verbatim}
$c ( ) -> wff $.
$v p q r s $.
wp $f wff p $.
wq $f wff q $.
wr $f wff r $.
ws $f wff s $.
w2 $a wff ( p -> q ) $.
wnew $p wff ( s -> ( r -> p ) ) $= ws wr wp w2 w2 $.
\end{verbatim}
This Metamath source example shows the definition and ``proof'' (i.e.,
construction) of a well-formed formula (wff)\index{well-formed formula (wff)}
in propositional calculus.  (You may wish to type this example into a file to
experiment with the Metamath program.)  The first two statements declare
(introduce the names of) four constants and four variables.  The next four
statements specify the variable types, namely that
each variable is assumed to be a wff.  Statement \texttt{w2} defines (postulates)
a way to produce a new wff, \texttt{( p -> q )}, from two given wffs \texttt{p} and
\texttt{q}. The mandatory hypotheses of \texttt{w2} are \texttt{wp} and \texttt{wq}.
Statement \texttt{wnew} claims that \texttt{( s -> ( r -> p ) )} is a wff given
three wffs \texttt{s}, \texttt{r}, and \texttt{p}.  More precisely, \texttt{wnew} claims
that the sequence of ten symbols \texttt{wff ( s -> ( r -> p ) )} is provable from
previous assertions and the hypotheses of \texttt{wnew}.  Metamath does not know
or care what a wff is, and as far as it is concerned
the typecode \texttt{wff} is just an
arbitrary constant symbol in a math symbol sequence.  The mandatory hypotheses
of \texttt{wnew} are \texttt{wp}, \texttt{wr}, and \texttt{ws}; \texttt{wq} is an optional
hypothesis.  In our particular proof, the optional hypothesis is not
referenced, but in general, any combination of active (i.e.\ optional and
mandatory) hypotheses could be referenced.  The proof of statement \texttt{wnew}
is the sequence of five labels starting with \texttt{ws} (step~1) and ending with
\texttt{w2} (step~5).

When Metamath verifies the proof, it scans the proof from left to right.  We
will examine what happens at each step of the proof.  The stack starts off
empty.  At step 1, Metamath looks up label \texttt{ws} and determines that it is a
hypothesis, so it pushes the symbol sequence of statement \texttt{ws} onto the
stack:

\begin{center}\begin{tabular}{|l|l|}\hline
{Stack location} & {Contents} \\ \hline \hline
1 & \texttt{wff s} \\ \hline
\end{tabular}\end{center}

Metamath sees that the labels \texttt{wr} and \texttt{wp} in steps~2 and 3 are also
hypotheses, so it pushes them onto the stack.  After step~3, the stack looks
like
this:

\begin{center}\begin{tabular}{|l|l|}\hline
{Stack location} & {Contents} \\ \hline \hline
3 & \texttt{wff p} \\ \hline
2 & \texttt{wff r} \\ \hline
1 & \texttt{wff s} \\ \hline
\end{tabular}\end{center}

At step 4, Metamath sees that label \texttt{w2} is an assertion, so it must do
some processing.  First, it associates the mandatory hypotheses of \texttt{w2},
which are \texttt{wp} and \texttt{wq}, with stack locations~2 and 3, {\em in that
order}. Metamath determines that the only possible way
to make hypothesis \texttt{wp} match (become identical to) stack location~2 and
\texttt{wq} match stack location 3 is to substitute variable \texttt{p} with \texttt{r}
and \texttt{q} with \texttt{p}.  Metamath makes these substitutions into \texttt{w2} and
obtains the symbol sequence \texttt{wff ( r -> p )}.  It removes the hypotheses
from stack locations~2 and 3, then places the result into stack location~2:

\begin{center}\begin{tabular}{|l|l|}\hline
{Stack location} & {Contents} \\ \hline \hline
2 & \texttt{wff ( r -> p )} \\ \hline
1 & \texttt{wff s} \\ \hline
\end{tabular}\end{center}

At step 5, Metamath sees that label \texttt{w2} is an assertion, so it must again
do some processing.  First, it matches the mandatory hypotheses of \texttt{w2},
which are \texttt{wp} and \texttt{wq}, to stack locations 1 and 2.
Metamath determines that the only possible way to make the
hypotheses match is to substitute variable \texttt{p} with \texttt{s} and \texttt{q} with
\texttt{( r -> p )}.  Metamath makes these substitutions into \texttt{w2} and obtains
the symbol
sequence \texttt{wff ( s -> ( r -> p ) )}.  It removes stack
locations 1 and 2, then places the result into stack location~1:

\begin{center}\begin{tabular}{|l|l|}\hline
{Stack location} & {Contents} \\ \hline \hline
1 & \texttt{wff ( s -> ( r -> p ) )} \\ \hline
\end{tabular}\end{center}

After Metamath finishes processing the proof, it checks to see that the
stack contains exactly one element and that this element is
the same as the math symbol sequence in the
\texttt{\$p}\index{\texttt{\$p} statement} statement.  This is the case for our
proof of \texttt{wnew},
so we have proved \texttt{wnew} successfully.  If the result
differs, Metamath will notify you with an error message.  An error message
will also result if the stack contains more than one entry at the end of the
proof, or if the stack did not contain enough entries at any point in the
proof to match all of the mandatory hypotheses\index{mandatory hypothesis} of
an assertion.  Finally, Metamath will notify you with an error message if no
substitution is possible that will make a referenced assertion's hypothesis
match the
stack entries.  You may want to experiment with the different kinds of errors
that Metamath will detect by making some small changes in the proof of our
example.

Metamath's proof notation was designed primarily to express proofs in a
relatively compact manner, not for readability by humans.  Metamath can display
proofs in a number of different ways with the \texttt{show proof}\index{\texttt{show
proof} command} command.  The
\texttt{/lemmon} qualifier displays it in a format that is easier to read when the
proofs are short, and you saw examples of its use in Chapter~\ref{using}.  For
longer proofs, it is useful to see the tree structure of the proof.  A tree
structure is displayed when the \texttt{/lemmon} qualifier is omitted.  You will
probably find this display more convenient as you get used to it. The tree
display of the proof in our example looks like
this:\label{treeproof}\index{tree-style proof}\index{proof!tree-style}
\begin{verbatim}
1     wp=ws    $f wff s
2        wp=wr    $f wff r
3        wq=wp    $f wff p
4     wq=w2    $a wff ( r -> p )
5  wnew=w2  $a wff ( s -> ( r -> p ) )
\end{verbatim}
The number to the left of each line is the step number.  Following it is a
{\bf hypothesis association}\index{hypothesis association}, consisting of two
labels\index{label} separated by \texttt{=}.  To the left of the \texttt{=} (except
in the last step) is the label of a hypothesis of an assertion referenced
later in the proof; here, steps 1 and 4 are the hypothesis associations for
the assertion \texttt{w2} that is referenced in step 5.  A hypothesis association
is indented one level more than the assertion that uses it, so it is easy to
find the corresponding assertion by moving directly down until the indentation
level decreases to one less than where you started from.  To the right of each
\texttt{=} is the proof step label for that proof step.  The statement keyword of
the proof step label is listed next, followed by the content of the top of the
stack (the most recent stack entry) as it exists after that proof step is
processed.  With a little practice, you should have no trouble reading proofs
displayed in this format.

Metamath proofs include the syntax construction of a formula.
In standard mathematics, this kind of
construction is not considered a proper part of the proof at all, and it
certainly becomes rather boring after a while.
Therefore,
by default the \texttt{show proof}\index{\texttt{show proof}
command} command does not show the syntax construction.
Historically \texttt{show proof} command
\textit{did} show the syntax construction, and you needed to add the
\texttt{/essential} option to hide, them, but today
\texttt{/essential} is the default and you need to use
\texttt{/all} to see the syntax constructions.

When verifying a proof, Metamath will check that no mandatory
\texttt{\$d}\index{\texttt{\$d} statement}\index{mandatory \texttt{\$d}
statement} statement of an assertion referenced in a proof is violated
when substitutions\index{substitution!variable}\index{variable
substitution} are made to the variables in the assertion.  For details
see Section~\ref{spec4} or \ref{dollard}.

\subsection{The Concept of Unification} \label{unify}

During the course of verifying a proof, when Metamath\index{Metamath}
encounters an assertion label\index{assertion label}, it associates the
mandatory hypotheses\index{mandatory hypothesis} of the assertion with the top
entries of the RPN stack\index{stack}\index{RPN stack}.  Metamath then
determines what substitutions\index{substitution!variable}\index{variable
substitution} it must make to the variables in the assertion's mandatory
hypotheses in order for these hypotheses to become identical to their
corresponding stack entries.  This process is called {\bf
unification}\index{unification}.  (We also informally use the term
``unification'' to refer to a set of substitutions that results from the
process, as in ``two unifications are possible.'')  After the substitutions
are made, the hypotheses are said to be {\bf unified}.

If no such substitutions are possible, Metamath will consider the proof
incorrect and notify you with an error message.
% (deleted 3/10/07, per suggestion of Mel O'Cat:)
% The syntax of the
% Metamath language ensures that if a set of substitutions exists, it
% will be unique.

The general algorithm for unification described in the literature is
somewhat complex.
However, in the case of Metamath it is intentionally trivial.
Mandatory hypotheses must be
pushed on the proof stack in the order in which they appear.
In addition, each variable must have its type specified
with a \texttt{\$f} hypothesis before it is used
and that each \texttt{\$f} hypothesis
have the restricted syntax of a typecode (a constant) followed by a variable.
The typecode in the \texttt{\$f} hypothesis must match the first symbol of
the corresponding RPN stack entry (which will also be a constant), so
the only possible match for the variable in the \texttt{\$f} hypothesis is
the sequence of symbols in the stack entry after the initial constant.

In the Proof Assistant\index{Proof Assistant}, a more general unification
algorithm is used.  While a proof is being developed, sometimes not enough
information is available to determine a unique unification.  In this case
Metamath will ask you to pick the correct one.\index{ambiguous
unification}\index{unification!ambiguous}

\section{Extensions to the Metamath Language}\index{extended
language}

\subsection{Comments in the Metamath Language}\label{comments}
\index{markup notation}
\index{comments!markup notation}

The commenting feature allows you to annotate the contents of
a database.  Just as with most
computer languages, comments are ignored for the purpose of interpreting the
contents of the database. Comments effectively act as
additional white space\index{white
space} between tokens
when a database is parsed.

A comment may be placed at the beginning, end, or
between any two tokens\index{token} in a source file.

Comments have the following syntax:
\begin{center}
 \texttt{\$(} {\em text} \texttt{\$)}
\end{center}
Here,\index{\texttt{\$(} and \texttt{\$)} auxiliary
keywords}\index{comment} {\em text} is a string, possibly empty, of any
characters in Metamath's character set (p.~\pageref{spec1chars}), except
that the character strings \texttt{\$(} and \texttt{\$)} may not appear
in {\em text}.  Thus nested comments are not
permitted:\footnote{Computer languages have differing standards for
nested comments, and rather than picking one it was felt simplest not to
allow them at all, at least in the current version (0.177) of
Metamath\index{Metamath!limitations of version 0.177}.} Metamath will
complain if you give it
\begin{center}
 \texttt{\$( This is a \$( nested \$) comment.\ \$)}
\end{center}
To compensate for this non-nesting behavior, I often change all \texttt{\$}'s
to \texttt{@}'s in sections of Metamath code I wish to comment out.

The Metamath program supports a number of markup mechanisms and conventions
to generate good-looking results in \LaTeX\ and {\sc html},
as discussed below.
These markup features have to do only with how the comments are typeset,
and have no effect on how Metamath verifies the proofs in the database.
The improper
use of them may result in incorrectly typeset output, but no Metamath
error messages will result during the \texttt{read} and \texttt{verify
proof} commands.  (However, the \texttt{write
theorem\texttt{\char`\_}list} command
will check for markup errors as a side-effect of its
{\sc html} generation.)
Section~\ref{texout} has instructions for creating \LaTeX\ output, and
section~\ref{htmlout} has instructions for creating
{\sc html}\index{HTML} output.

\subsubsection{Headings}\label{commentheadings}

If the \texttt{\$(} is immediately followed by a new line
starting with a heading marker, it is a header.
This can start with:

\begin{itemize}
 \item[] \texttt{\#\#\#\#} - major part header
 \item[] \texttt{\#*\#*} - section header
 \item[] \texttt{=-=-} - subsection header
 \item[] \texttt{-.-.} - subsubsection header
\end{itemize}

The line following the marker line
will be used for the table of contents entry, after trimming spaces.
The next line should be another (closing) matching marker line.
Any text after that
but before the closing \texttt{\$}, such as an extended description of the
section, will be included on the \texttt{mmtheoremsNNN.html} page.

For more information, run
\texttt{help write theorem\char`\_list}.

\subsubsection{Math mode}
\label{mathcomments}
\index{\texttt{`} inside comments}
\index{\texttt{\char`\~} inside comments}
\index{math mode}

Inside of comments, a string of tokens\index{token} enclosed in
grave accents\index{grave accent (\texttt{`})} (\texttt{`}) will be converted
to standard mathematical symbols during
{\sc HTML}\index{HTML} or \LaTeX\ output
typesetting,\index{latex@{\LaTeX}} according to the information in the
special \texttt{\$t}\index{\texttt{\$t} comment}\index{typesetting
comment} comment in the database
(see section~\ref{tcomment} for information about the typesetting
comment, and Appendix~\ref{ASCII} to see examples of its results).

The first grave accent\index{grave accent (\texttt{`})} \texttt{`}
causes the output processor to enter {\bf math mode}\index{math mode}
and the second one exits it.
In this
mode, the characters following the \texttt{`} are interpreted as a
sequence of math symbol tokens separated by white space\index{white
space}.  The tokens are looked up in the \texttt{\$t}
comment\index{\texttt{\$t} comment}\index{typesetting comment} and if
found, they will be replaced by the standard mathematical symbols that
they correspond to before being placed in the typeset output file.  If
not found, the symbol will be output as is and a warning will be issued.
The tokens do not have to be active in the database, although a warning
will be issued if they are not declared with \texttt{\$c} or
\texttt{\$v} statements.

Two consecutive
grave accents \texttt{``} are treated as a single actual grave accent
(both inside and outside of math mode) and will not cause the output
processor to enter or exit math mode.

Here is an example of its use\index{Pierce's axiom}:
\begin{center}
\texttt{\$( Pierce's axiom, ` ( ( ph -> ps ) -> ph ) -> ph ` ,\\
         is not very intuitive. \$)}
\end{center}
becomes
\begin{center}
   \texttt{\$(} Pierce's axiom, $((\varphi \rightarrow \psi)\rightarrow
\varphi)\rightarrow \varphi$, is not very intuitive. \texttt{\$)}
\end{center}

Note that the math symbol tokens\index{token} must be surrounded by white
space\index{white space}.
%, since there is no context that allows ambiguity to be
%resolved, as is the case with math symbol sequences in some of the Metamath
%statements.
White space should also surround the \texttt{`}
delimiters.

The math mode feature also gives you a quick and easy way to generate
text containing mathematical symbols, independently of the intended
purpose of Metamath.\index{Metamath!using as a math editor} To do this,
simply create your text with grave accents surrounding your formulas,
after making sure that your math symbols are mapped to \LaTeX\ symbols
as described in Appendix~\ref{ASCII}.  It is easier if you start with a
database with predefined symbols such as \texttt{set.mm}.  Use your
grave-quoted math string to replace an existing comment, then typeset
the statement corresponding to that comment following the instructions
from the \texttt{help tex} command in the Metamath program.  You will
then probably want to edit the resulting file with a text editor to fine
tune it to your exact needs.

\subsubsection{Label Mode}\index{label mode}

Outside of math mode, a tilde\index{tilde (\texttt{\char`\~})} \verb/~/
indicates to Metamath's\index{Metamath} output processor that the
token\index{token} that follows (i.e.\ the characters up to the next
white space\index{white space}) represents a statement label or URL.
This formatting mode is called {\bf label mode}\index{label mode}.
If a literal tilde
is desired (outside of math mode) instead of label mode,
use two tildes in a row to represent it.

When generating a \LaTeX\ output file,
the following token will be formatted in \texttt{typewriter}
font, and the tilde removed, to make it stand out from the rest of the text.
This formatting will be applied to all characters after the
tilde up to the first white space\index{white space}.
Whether
or not the token is an actual statement label is not checked, and the
token does not have to have the correct syntax for a label; no error
messages will be produced.  The only effect of the label mode on the
output is that typewriter font will be used for the tokens that are
placed in the \LaTeX\ output file.

When generating {\sc html},
the tokens after the tilde {\em must} be a URL (either http: or https:)
or a valid label.
Error messages will be issued during that output if they aren't.
A hyperlink will be generated to that URL or label.

\subsubsection{Link to bibliographical reference}\index{citation}%
\index{link to bibliographical reference}

Bibliographical references are handled specially when generating
{\sc html} if formatted specially.
Text in the form \texttt{[}{\em author}\texttt{]}
is considered a link to a bibliographical reference.
See \texttt{help html} and \texttt{help write
bibliography} in the Metamath program for more
information.
% \index{\texttt{\char`\[}\ldots\texttt{]} inside comments}
See also Sections~\ref{tcomment} and \ref{wrbib}.

The \texttt{[}{\em author}\texttt{]} notation will also create an entry in
the bibliography cross-reference file generated by \texttt{write
bibliography} (Section~\ref{wrbib}) for {\sc HTML}.
For this to work properly, the
surrounding comment must be formatted as follows:
\begin{quote}
    {\em keyword} {\em label} {\em noise-word}
     \texttt{[}{\em author}\texttt{] p.} {\em number}
\end{quote}
for example
\begin{verbatim}
     Theorem 5.2 of [Monk] p. 223
\end{verbatim}
The {\em keyword} is not case sensitive and must be one of the following:
\begin{verbatim}
     theorem lemma definition compare proposition corollary
     axiom rule remark exercise problem notation example
     property figure postulate equation scheme chapter
\end{verbatim}
The optional {\em label} may consist of more than one
(non-{\em keyword} and non-{\em noise-word}) word.
The optional {\em noise-word} is one of:
\begin{verbatim}
     of in from on
\end{verbatim}
and is  ignored when the cross-reference file is created.  The
\texttt{write
biblio\-graphy} command will perform error checking to verify the
above format.\index{error checking}

\subsubsection{Parentheticals}\label{parentheticals}

The end of a comment may include one or more parenthicals, that is,
statements enclosed in parentheses.
The Metamath program looks for certain parentheticals and can issue
warnings based on them.
They are:

\begin{itemize}
 \item[] \texttt{(Contributed by }
   \textit{NAME}\texttt{,} \textit{DATE}\texttt{.)} -
   document the original contributor's name and the date it was created.
 \item[] \texttt{(Revised by }
   \textit{NAME}\texttt{,} \textit{DATE}\texttt{.)} -
   document the contributor's name and creation date
   that resulted in significant revision
   (not just an automated minimization or shortening).
 \item[] \texttt{(Proof shortened by }
   \textit{NAME}\texttt{,} \textit{DATE}\texttt{.)} -
   document the contributor's name and date that developed a significant
   shortening of the proof (not just an automated minimization).
 \item[] \texttt{(Proof modification is discouraged.)} -
   Note that this proof should normally not be modified.
 \item[] \texttt{(New usage is discouraged.)} -
   Note that this assertion should normally not be used.
\end{itemize}

The \textit{DATE} must be in form YYYY-MMM-DD, where MMM is the
English abbreviation of that month.

\subsubsection{Other markup}\label{othermarkup}
\index{markup notation}

There are other markup notations for generating good-looking results
beyond math mode and label mode:

\begin{itemize}
 \item[]
         \texttt{\char`\_} (underscore)\index{\texttt{\char`\_} inside comments} -
             Italicize text starting from
              {\em space}\texttt{\char`\_}{\em non-space} (i.e.\ \texttt{\char`\_}
              with a space before it and a non-space character after it) until
             the next
             {\em non-space}\texttt{\char`\_}{\em space}.  Normal
             punctuation (e.g.\ a trailing
             comma or period) is ignored when determining {\em space}.
 \item[]
         \texttt{\char`\_} (underscore) - {\em
         non-space}\texttt{\char`\_}{\em non-space-string}, where
          {\em non-space-string} is a string of non-space characters,
         will make {\em non-space-string} become a subscript.
 \item[]
         \texttt{<HTML>}...\texttt{</HTML>} - do not convert
         ``\texttt{<}'' and ``\texttt{>}''
         in the enclosed text when generating {\sc HTML},
         otherwise process markup normally. This allows direct insertion
         of {\sc html} commands.
 \item[]
       ``\texttt{\&}ref\texttt{;}'' - insert an {\sc HTML}
         character reference.
         This is how to insert arbitrary Unicode characters
         (such as accented characters).  Currently only directly supported
         when generating {\sc HTML}.
\end{itemize}

It is recommended that spaces surround any \texttt{\char`\~} and
\texttt{`} tokens in the comment and that a space follow the {\em label}
after a \texttt{\char`\~} token.  This will make global substitutions
to change labels and symbol names much easier and also eliminate any
future chance of ambiguity.  Spaces around these tokens are automatically
removed in the final output to conform with normal rules of punctuation;
for example, a space between a trailing \texttt{`} and a left parenthesis
will be removed.

A good way to become familiar with the markup notation is to look at
the extensive examples in the \texttt{set.mm} database.

\subsection{The Typesetting Comment (\texttt{\$t})}\label{tcomment}

The typesetting comment \texttt{\$t} in the input database file
provides the information necessary to produce good-looking results.
It provides \LaTeX\ and {\sc html}
definitions for math symbols,
as well supporting as some
customization of the generated web page.
If you add a new token to a database, you should also
update the \texttt{\$t} comment information if you want to eventually
create output in \LaTeX\ or {\sc HTML}.
See the
\texttt{set.mm}\index{set theory database (\texttt{set.mm})} database
file for an extensive example of a \texttt{\$t} comment illustrating
many of the features described below.

Programs that do not need to generate good-looking presentation results,
such as programs that only verify Metamath databases,
can completely ignore typesetting comments
and just treat them as normal comments.
Even the Metamath program only consults the
\texttt{\$t} comment information when it needs to generate typeset output
in \LaTeX\ or {\sc HTML}
(e.g., when you open a \LaTeX\ output file with the \texttt{open tex} command).

We will first discuss the syntax of typesetting comments, and then
briefly discuss how this can be used within the Metamath program.

\subsubsection{Typesetting Comment Syntax Overview}

The typesetting comment is identified by the token
\texttt{\$t}\index{\texttt{\$t} comment}\index{typesetting comment} in
the comment, and the typesetting comment ends at the matching
\texttt{\$)}:
\[
  \mbox{\tt \$(\ }
  \mbox{\tt \$t\ }
  \underbrace{
    \mbox{\tt \ \ \ \ \ \ \ \ \ \ \ }
    \cdots
    \mbox{\tt \ \ \ \ \ \ \ \ \ \ \ }
  }_{\mbox{Typesetting definitions go here}}
  \mbox{\tt \ \$)}
\]

There must be one or more white space characters, and only white space
characters, between the \texttt{\$(} that starts the comment
and the \texttt{\$t} symbol,
and the \texttt{\$t} must be followed by one
or more white space characters
(see section \ref{whitespace} for the definition of white space characters).
The typesetting comment continues until the comment end token \texttt{\$)}
(which must be preceded by one or more white space characters).

In version 0.177\index{Metamath!limitations of version 0.177} of the
Metamath program, there may be only one \texttt{\$t} comment in a
database.  This restriction may be lifted in the future to allow
many \texttt{\$t} comments in a database.

Between the \texttt{\$t} symbol (and its following white space) and the
comment end token \texttt{\$)} (and its preceding white space)
is a sequence of one or more typesetting definitions, where
each definition has the form
\textit{definition-type arg arg ... ;}.
Each of the zero or more \textit{arg} values
can be either a typesetting data or a keyword
(what keywords are allowed, and where, depends on the specific
\textit{definition-type}).
The \textit{definition-type}, and each argument \textit{arg},
are separated by one or more white space characters.
Every definition ends in an unquoted semicolon;
white space is not required before the terminating semicolon of a definition.
Each definition should start on a new line.\footnote{This
restriction of the current version of Metamath
(0.177)\index{Metamath!limitations of version 0.177} may be removed
in a future version, but you should do it anyway for readability.}

For example, this typesetting definition:
\begin{center}
 \verb$latexdef "C_" as "\subseteq";$
\end{center}
defines the token \verb$C_$ as the \LaTeX\ symbol $\subseteq$ (which means
``subset'').

Typesetting data is a sequence of one or more quoted strings
(if there is more than one, they are connected by \texttt{\char`\+}).
Often a single quoted string is used to provide data for a definition, using
either double (\texttt{\char`\"}) or single (\texttt{'}) quotation marks.
However,
{\em a quoted string (enclosed in quotation marks) may not include
line breaks.}
A quoted string
may include a quotation mark that matches the enclosing quotes by repeating
the quotation mark twice.  Here are some examples:

\begin{tabu}   { l l }
\textbf{Example} & \textbf{Meaning} \\
\texttt{\char`\"a\char`\"\char`\"b\char`\"} & \texttt{a\char`\"b} \\
\texttt{'c''d'} & \texttt{c'd} \\
\texttt{\char`\"e''f\char`\"} & \texttt{e''f} \\
\texttt{'g\char`\"\char`\"h'} & \texttt{g\char`\"\char`\"h} \\
\end{tabu}

Finally, a long quoted string
may be broken up into multiple quoted strings (considered, as a whole,
a single quoted string) and joined with \texttt{\char`\+}.
You can even use multiple lines as long as a
'+' is at the end of every line except the last one.
The \texttt{\char`\+} should be preceded and followed by at least one
white space character.
Thus, for example,
\begin{center}
 \texttt{\char`\"ab\char`\"\ \char`\+\ \char`\"cd\char`\"
    \ \char`\+\ \\ 'ef'}
\end{center}
is the same as
\begin{center}
 \texttt{\char`\"abcdef\char`\"}
\end{center}

{\sc c}-style comments \texttt{/*}\ldots\texttt{*/} are also supported.

In practice, whenever you add a new math token you will often want to add
typesetting definitions using
\texttt{latexdef}, \texttt{htmldef}, and
\texttt{althtmldef}, as described below.
That way, they will all be up to date.
Of course, whether or not you want to use all three definitions will
depend on how the database is intended to be used.

Below we discuss the different possible \textit{definition-kind} options.
We will show data surrounded by double quotes (in practice they can also use
single quotes and/or be a sequence joined by \texttt{+}s).
We will use specific names for the \textit{data} to make clear what
the data is used for, such as
{\em math-token} (for a Metamath math token,
{\em latex-string} (for string to be placed in a \LaTeX\ stream),
{\em {\sc html}-code} (for {\sc html} code),
and {\em filename} (for a filename).

\subsubsection{Typesetting Comment - \LaTeX}

The syntax for a \LaTeX\ definition is:
\begin{center}
 \texttt{latexdef "}{\em math-token}\texttt{" as "}{\em latex-string}\texttt{";}
\end{center}
\index{latex definitions@\LaTeX\ definitions}%
\index{\texttt{latexdef} statement}

The {\em token-string} and {\em latex-string} are the data
(character strings) for
the token and the \LaTeX\ definition of the token, respectively,

These \LaTeX\ definitions are used by the Metamath program
when it is asked to product \LaTeX output using
the \texttt{write tex} command.

\subsubsection{Typesetting Comment - {\sc html}}

The key kinds of {\sc HTML} definitions have the following syntax:

\vskip 1ex
    \texttt{htmldef "}{\em math-token}\texttt{" as "}{\em
    {\sc html}-code}\texttt{";}\index{\texttt{htmldef} statement}
                    \ \ \ \ \ \ldots

    \texttt{althtmldef "}{\em math-token}\texttt{" as "}{\em
{\sc html}-code}\texttt{";}\index{\texttt{althtmldef} statement}

                    \ \ \ \ \ \ldots

Note that in {\sc HTML} there are two possible definitions for math tokens.
This feature is useful when
an alternate representation of symbols is desired, for example one that
uses Unicode entities and another uses {\sc gif} images.

There are many other typesetting definitions that can control {\sc HTML}.
These include:

\vskip 1ex

    \texttt{htmldef "}{\em math-token}\texttt{" as "}{\em {\sc
    html}-code}\texttt{";}

    \texttt{htmltitle "}{\em {\sc html}-code}\texttt{";}%
\index{\texttt{htmltitle} statement}

    \texttt{htmlhome "}{\em {\sc html}-code}\texttt{";}%
\index{\texttt{htmlhome} statement}

    \texttt{htmlvarcolor "}{\em {\sc html}-code}\texttt{";}%
\index{\texttt{htmlvarcolor} statement}

    \texttt{htmlbibliography "}{\em filename}\texttt{";}%
\index{\texttt{htmlbibliography} statement}

\vskip 1ex

\noindent The \texttt{htmltitle} is the {\sc html} code for a common
title, such as ``Metamath Proof Explorer.''  The \texttt{htmlhome} is
code for a link back to the home page.  The \texttt{htmlvarcolor} is
code for a color key that appears at the bottom of each proof.  The file
specified by {\em filename} is an {\sc html} file that is assumed to
have a \texttt{<A NAME=}\ldots\texttt{>} tag for each bibiographic
reference in the database comments.  For example, if
\texttt{[Monk]}\index{\texttt{\char`\[}\ldots\texttt{]} inside comments}
occurs in the comment for a theorem, then \texttt{<A NAME='Monk'>} must
be present in the file; if not, a warning message is given.

Associated with
\texttt{althtmldef}
are the statements
\vskip 1ex

    \texttt{htmldir "}{\em
      directoryname}\texttt{";}\index{\texttt{htmldir} statement}

    \texttt{althtmldir "}{\em
     directoryname}\texttt{";}\index{\texttt{althtmldir} statement}

\vskip 1ex
\noindent giving the directories of the {\sc gif} and Unicode versions
respectively; their purpose is to provide cross-linking between the
two versions in the generated web pages.

When two different types of pages need to be produced from a single
database, such as the Hilbert Space Explorer that extends the Metamath
Proof Explorer, ``extended'' variables may be declared in the
\texttt{\$t} comment:
\vskip 1ex

    \texttt{exthtmltitle "}{\em {\sc html}-code}\texttt{";}%
\index{\texttt{exthtmltitle} statement}

    \texttt{exthtmlhome "}{\em {\sc html}-code}\texttt{";}%
\index{\texttt{exthtmlhome} statement}

    \texttt{exthtmlbibliography "}{\em filename}\texttt{";}%
\index{\texttt{exthtmlbibliography} statement}

\vskip 1ex
\noindent When these are declared, you also must declare
\vskip 1ex

    \texttt{exthtmllabel "}{\em label}\texttt{";}%
\index{\texttt{exthtmllabel} statement}

\vskip 1ex \noindent that identifies the database statement where the
``extended'' section of the database starts (in our example, where the
Hilbert Space Explorer starts).  During the generation of web pages for
that starting statement and the statements after it, the {\sc html} code
assigned to \texttt{exthtmltitle} and \texttt{exthtmlhome} is used
instead of that assigned to \texttt{htmltitle} and \texttt{htmlhome},
respectively.

\begin{sloppy}
\subsection{Additional Information Com\-ment (\texttt{\$j})} \label{jcomment}
\end{sloppy}

The additional information comment, aka the
\texttt{\$j}\index{\texttt{\$j} comment}\index{additional information comment}
comment,
provides a way to add additional structured information that can
be optionally parsed by systems.

The additional information comment is parsed the same way as the
typesetting comment (\texttt{\$t}) (see section \ref{tcomment}).
That is,
the additional information comment begins with the token
\texttt{\$j} within a comment,
and continues until the comment close \texttt{\$)}.
Within an additional information comment is a sequence of one or more
commands of the form \texttt{command arg arg ... ;}
where each of the zero or more \texttt{arg} values
can be either a quoted string or a keyword.
Note that every command ends in an unquoted semicolon.
If a verifier is parsing an additional information comment, but
doesn't recognize a particular command, it must skip the command
by finding the end of the command (an unquoted semicolon).

A database may have 0 or more additional information comments.
Note, however, that a verifier may ignore these comments entirely or only
process certain commands in an additional information comment.
The \texttt{mmj2} verifier supports many commands in additional information
comments.
We encourage systems that process additional information comments
to coordinate so that they will use the same command for the same effect.

Examples of additional information comments with various commands
(from the \texttt{set.mm} database) are:

\begin{itemize}
   \item Define the syntax and logical typecodes,
     and declare that our grammar is
     unambiguous (verifiable using the KLR parser, with compositing depth 5).
\begin{verbatim}
  $( $j
    syntax 'wff';
    syntax '|-' as 'wff';
    unambiguous 'klr 5';
  $)
\end{verbatim}

   \item Register $\lnot$ and $\rightarrow$ as primitive expressions
           (lacking definitions).
\begin{verbatim}
  $( $j primitive 'wn' 'wi'; $)
\end{verbatim}

   \item There is a special justification for \texttt{df-bi}.
\begin{verbatim}
  $( $j justification 'bijust' for 'df-bi'; $)
\end{verbatim}

   \item Register $\leftrightarrow$ as an equality for its type (wff).
\begin{verbatim}
  $( $j
    equality 'wb' from 'biid' 'bicomi' 'bitri';
    definition 'dfbi1' for 'wb';
  $)
\end{verbatim}

   \item Theorem \texttt{notbii} is the congruence law for negation.
\begin{verbatim}
  $( $j congruence 'notbii'; $)
\end{verbatim}

   \item Add \texttt{setvar} as a typecode.
\begin{verbatim}
  $( $j syntax 'setvar'; $)
\end{verbatim}

   \item Register $=$ as an equality for its type (\texttt{class}).
\begin{verbatim}
  $( $j equality 'wceq' from 'eqid' 'eqcomi' 'eqtri'; $)
\end{verbatim}

\end{itemize}


\subsection{Including Other Files in a Metamath Source File} \label{include}
\index{\texttt{\$[} and \texttt{\$]} auxiliary keywords}

The keywords \texttt{\$[} and \texttt{\$]} specify a file to be
included\index{included file}\index{file inclusion} at that point in a
Metamath\index{Metamath} source file\index{source file}.  The syntax for
including a file is as follows:
\begin{center}
\texttt{\$[} {\em file-name} \texttt{\$]}
\end{center}

The {\em file-name} should be a single token\index{token} with the same syntax
as a math symbol (i.e., all 93 non-whitespace
printable characters other than \texttt{\$} are
allowed, subject to the file-naming limitations of your operating system).
Comments may appear between the \texttt{\$[} and \texttt{\$]} keywords.  Included
files may include other files, which may in turn include other files, and so
on.

For example, suppose you want to use the set theory database as the starting
point for your own theory.  The first line in your file could be
\begin{center}
\texttt{\$[ set.mm \$]}
\end{center} All of the information (axioms, theorems,
etc.) in \texttt{set.mm} and any files that {\em it} includes will become
available for you to reference in your file. This can help make your work more
modular. A drawback to including files is that if you change the name of a
symbol or the label of a statement, you must also remember to update any
references in any file that includes it.


The naming conventions for included files are the same as those of your
operating system.\footnote{On the Macintosh, prior to Mac OS X,
 a colon is used to separate disk
and folder names from your file name.  For example, {\em volume}\texttt{:}{\em
file-name} refers to the root directory, {\em volume}\texttt{:}{\em
folder-name}\texttt{:}{\em file-name} refers to a folder in root, and {\em
volume}\texttt{:}{\em folder-name}\texttt{:}\ldots\texttt{:}{\em file-name} refers to a
deeper folder.  A simple {\em file-name} refers to a file in the folder from
which you launch the Metamath application.  Under Mac OS X and later,
the Metamath program is run under the Terminal application, which
conforms to Unix naming conventions.}\index{Macintosh file
names}\index{file names!Macintosh}\label{includef} For compatibility among
operating systems, you should keep the file names as simple as possible.  A
good convention to use is {\em file}\texttt{.mm} where {\em file} is eight
characters or less, in lower case.

There is no limit to the nesting depth of included files.  One thing that you
should be aware of is that if two included files themselves include a common
third file, only the {\em first} reference to this common file will be read
in.  This allows you to include two or more files that build on a common
starting file without having to worry about label and symbol conflicts that
would occur if the common file were read in more than once.  (In fact, if a
file includes itself, the self-reference will be ignored, although of course
it would not make any sense to do that.)  This feature also means, however,
that if you try to include a common file in several inner blocks, the result
might not be what you expect, since only the first reference will be replaced
with the included file (unlike the include statement in most other computer
languages).  Thus you would normally include common files only in the
outermost block\index{outermost block}.

\subsection{Compressed Proof Format}\label{compressed1}\index{compressed
proof}\index{proof!compressed}

The proof notation presented in Section~\ref{proof} is called a
{\bf normal proof}\index{normal proof}\index{proof!normal} and in principle is
sufficient to express any proof.  However, proofs often contain steps and
subproofs that are identical.  This is particularly true in typical
Metamath\index{Metamath} applications, because Metamath requires that the math
symbol sequence (usually containing a formula) at each step be separately
constructed, that is, built up piece by piece. As a result, a lot of
repetition often results.  The {\bf compressed proof} format allows Metamath
to take advantage of this redundancy to shorten proofs.

The specification for the compressed proof format is given in
Appen\-dix~\ref{compressed}.

Normally you need not concern yourself with the details of the compressed
proof format, since the Metamath program will allow you to convert from
the normal format to the compressed format with ease, and will also
automatically convert from the compressed format when proofs are displayed.
The overall structure of the compressed format is as follows:
\begin{center}
  \texttt{\$= ( } {\em label-list} \texttt{) } {\em compressed-proof\ }\ \texttt{\$.}
\end{center}
\index{\texttt{\$=} keyword}
The first \texttt{(} serves as a flag to Metamath that a compressed proof
follows.  The {\em label-list} includes all statements referred to by the
proof except the mandatory hypotheses\index{mandatory hypothesis}.  The {\em
compressed-proof} is a compact encoding of the proof, using upper-case
letters, and can be thought of as a large integer in base 26.  White
space\index{white space} inside a {\em compressed-proof} is
optional and is ignored.

It is important to note that the order of the mandatory hypotheses of
the statement being proved must not be changed if the compressed proof
format is used, otherwise the proof will become incorrect.  The reason
for this is that the mandatory hypotheses are not mentioned explicitly
in the compressed proof in order to make the compression more efficient.
If you wish to change the order of mandatory hypotheses, you must first
convert the proof back to normal format using the \texttt{save proof
{\em statement} /normal}\index{\texttt{save proof} command} command.
Later, you can go back to compressed format with \texttt{save proof {\em
statement} /compressed}.

During error checking with the \texttt{verify proof} command, an error
found in a compressed proof may point to a character in {\em
compressed-proof}, which may not be very meaningful to you.  In this
case, try to \texttt{save proof /normal} first, then do the
\texttt{verify proof} again.  In general, it is best to make sure a
proof is correct before saving it in compressed format, because severe
errors are less likely to be recoverable than in normal format.

\subsection{Specifying Unknown Proofs or Subproofs}\label{unknown}

In a proof under development, any step or subproof that is not yet known
may be represented with a single \texttt{?}.  For the purposes of
parsing the proof, the \texttt{?}\ \index{\texttt{]}@\texttt{?}\ inside
proofs} will push a single entry onto the RPN stack just as if it were a
hypothesis.  While developing a proof with the Proof
Assistant\index{Proof Assistant}, a partially developed proof may be
saved with the \texttt{save new{\char`\_}proof}\index{\texttt{save
new{\char`\_}proof} command} command, and \texttt{?}'s will be placed at
the appropriate places.

All \texttt{\$p}\index{\texttt{\$p} statement} statements must have
proofs, even if they are entirely unknown.  Before creating a proof with
the Proof Assistant, you should specify a completely unknown proof as
follows:
\begin{center}
  {\em label} \texttt{\$p} {\em statement} \texttt{\$= ?\ \$.}
\end{center}
\index{\texttt{\$=} keyword}
\index{\texttt{]}@\texttt{?}\ inside proofs}

The \texttt{verify proof}\index{\texttt{verify proof} command} command
will check the known portions of a partial proof for errors, but will
warn you that the statement has not been proved.

Note that partially developed proofs may be saved in compressed format
if desired.  In this case, you will see one or more \texttt{?}'s in the
{\em compressed-proof} part.\index{compressed
proof}\index{proof!compressed}

\section{Axioms vs.\ Definitions}\label{definitions}

The \textit{basic}
Metamath\index{Metamath} language and program
make no distinction\index{axiom vs.\
definition} between axioms\index{axiom} and
definitions.\index{definition} The \texttt{\$a}\index{\texttt{\$a}
statement} statement is used for both.  At first, this may seem
puzzling.  In the minds of many mathematicians, the distinction is
clear, even obvious, and hardly worth discussing.  A definition is
considered to be merely an abbreviation that can be replaced by the
expression for which it stands; although unless one actually does this,
to be precise then one should say that a theorem\index{theorem} is a
consequence of the axioms {\em and} the definitions that are used in the
formulation of the theorem \cite[p.~20]{Behnke}.\index{Behnke, H.}

\subsection{What is a Definition?}

What is a definition?  In its simplest form, a definition introduces a new
symbol and provides an unambiguous rule to transform an expression containing
the new symbol to one without it.  The concept of a ``proper
definition''\index{proper definition}\index{definition!proper} (as opposed to
a creative definition)\index{creative definition}\index{definition!creative}
that is usually agreed upon is (1) the definition should not strengthen the
language and (2) any symbols introduced by the definition should be eliminable
from the language \cite{Nemesszeghy}\index{Nemesszeghy, E. Z.}.  In other
words, they are mere typographical conveniences that do not belong to the
system and are theoretically superfluous.  This may seem obvious, but in fact
the nature of definitions can be subtle, sometimes requiring difficult
metatheorems to establish that they are not creative.

A more conservative stance was taken by logician S.
Le\'{s}niewski.\index{Le\'{s}niewski, S.}
\begin{quote}
Le\'{s}niewski
regards definitions as theses of the system.  In this respect they do
not differ either from the axioms or from theorems, i.e.\ from the
theses added to the system on the basis of the rule of substitution or
the rule of detachment [modus ponens].  Once definitions have been
accepted as theses of the system, it becomes necessary to consider them
as true propositions in the same sense in which axioms are true
\cite{Lejewski}.
\end{quote}\index{Lejewski, Czeslaw}

Let us look at some simple examples of definitions in propositional
calculus.  Consider the definition of logical {\sc or}
(disjunction):\index{disjunction ($\vee$)} ``$P\vee Q$ denotes $\neg P
\rightarrow Q$ (not $P$ implies $Q$).''  It is very easy to recognize a
statement making use of this definition, because it introduces the new
symbol $\vee$ that did not previously exist in the language.  It is easy
to see that no new theorems of the original language will result from
this definition.

Next, consider a definition that eliminates parentheses:  ``$P
\rightarrow Q\rightarrow R$ denotes $P\rightarrow (Q \rightarrow R)$.''
This is more subtle, because no new symbols are introduced.  The reason
this definition is considered proper is that no new symbol sequences
that are valid wffs (well-formed formulas)\index{well-formed formula
(wff)} in the original language will result from the definition, since
``$P \rightarrow Q\rightarrow R$'' is not a wff in the original
language.  Here, we implicitly make use of the fact that there is a
decision procedure that allows us to determine whether or not a symbol
sequence is a wff, and this fact allows us to use symbol sequences that
are not wffs to represent other things (such as wffs) by means of the
definition.  However, to justify the definition as not being creative we
need to prove that ``$P \rightarrow Q\rightarrow R$'' is in fact not a
wff in the original language, and this is more difficult than in the
case where we simply introduce a new symbol.

%Now let's take this reasoning to an extreme.  Propositional calculus is a
%decidable theory,\footnote{This means that a mechanical algorithm exists to
%determine whether or not a wff is a theorem.} so in principle we could make use
%of symbol sequences that are not theorems to represent other things (say, to
%encode actual theorems in a more compact way).  For example, let us extend the
%language by defining a wff ``$P$'' in the extended language as the theorem
%``$P\rightarrow P$''\footnote{This is one of the first theorems proved in the
%Metamath database \texttt{set.mm}.}\index{set
%theory database (\texttt{set.mm})} in the original language whenever ``$P$'' is
%not a theorem in the original language.  In the extended language, any wff
%``$Q$'' thus represents a theorem; to find out what theorem (in the original
%language) ``$Q$'' represents, we determine whether ``$Q$'' is a theorem in the
%original language (before the definition was introduced).  If so, we're done; if
%not, we replace ``$Q$'' by ``$Q\rightarrow Q$'' to eliminate the definition.
%This definition is therefore eliminable, and it does not ``strengthen'' the
%language because any wff that is not a theorem is not in the set of statements
%provable in the original language and thus is available for use by definitions.
%
%Of course, a definition such as this would render practically useless the
%communication of theorems of propositional calculus; but
%this is just a human shortcoming, since we can't always easily discern what is
%and is not a theorem by inspection.  In fact, the extended theory with this
%definition has no more and no less information than the original theory; it just
%expresses certain theorems of the form ``$P\rightarrow P$''
%in a more compact way.
%
%The point here is that what constitutes a proper definition is a matter of
%judgment about whether a symbol sequence can easily be recognized by a human
%as invalid in some sense (for example, not a wff); if so, the symbol sequence
%can be appropriated for use by a definition in order to make the extended
%language more compact.  Metamath\index{Metamath} lacks the ability to make this
%judgment, since as far as Metamath is concerned the definition of a wff, for
%example, is arbitrary.  You define for Metamath how wffs\index{well-formed
%formula (wff)} are constructed according to your own preferred style.  The
%concept of a wff may not even exist in a given formal system\index{formal
%system}.  Metamath treats all definitions as if they were new axioms, and it
%is up to the human mathematician to judge whether the definition is ``proper''
%'\index{proper definition}\index{definition!proper} in some agreed-upon way.

What constitutes a definition\index{definition} versus\index{axiom vs.\
definition} an axiom\index{axiom} is sometimes arbitrary in mathematical
literature.  For example, the connectives $\vee$ ({\sc or}), $\wedge$
({\sc and}), and $\leftrightarrow$ (equivalent to) in propositional
calculus are usually considered defined symbols that can be used as
abbreviations for expressions containing the ``primitive'' connectives
$\rightarrow$ and $\neg$.  This is the way we treat them in the standard
logic and set theory database \texttt{set.mm}\index{set theory database
(\texttt{set.mm})}.  However, the first three connectives can also be
considered ``primitive,'' and axiom systems have been devised that treat
all of them as such.  For example,
\cite[p.~35]{Goodstein}\index{Goodstein, R. L.} presents one with 15
axioms, some of which in fact coincide with what we have chosen to call
definitions in \texttt{set.mm}.  In certain subsets of classical
propositional calculus, such as the intuitionist
fragment\index{intuitionism}, it can be shown that one cannot make do
with just $\rightarrow$ and $\neg$ but must treat additional connectives
as primitive in order for the system to make sense.\footnote{Two nice
systems that make the transition from intuitionistic and other weak
fragments to classical logic just by adding axioms are given in
\cite{Robinsont}\index{Robinson, T. Thacher}.}

\subsection{The Approach to Definitions in \texttt{set.mm}}

In set theory, recursive definitions define a newly introduced symbol in
terms of itself.
The justification of recursive definitions, using
several ``recursion theorems,'' is usually one of the first
sophisticated proofs a student encounters when learning set theory, and
there is a significant amount of implicit metalogic behind a recursive
definition even though the definition itself is typically simple to
state.

Metamath itself has no built-in technical limitation that prevents
multiple-part recursive definitions in the traditional textbook style.
However, because the recursive definition requires advanced metalogic
to justify, eliminating a recursive definition is very difficult and
often not even shown in textbooks.

\subsubsection{Direct definitions instead of recursive definitions}

It is, however, possible to substitute one kind of complexity
for another.  We can eliminate the need for metalogical justification by
defining the operation directly with an explicit (but complicated)
expression, then deriving the recursive definition directly as a
theorem, using a recursion theorem ``in reverse.''
The elimination
of a direct definition is a matter of simple mechanical substitution.
We do this in
\texttt{set.mm}, as follows.

In \texttt{set.mm} our goal was to introduce almost all definitions in
the form of two expressions connected by either $\leftrightarrow$ or
$=$, where the thing being defined does not appear on the right hand
side.  Quine calls this form ``a genuine or direct definition'' \cite[p.
174]{Quine}\index{Quine, Willard Van Orman}, which makes the definitions
very easy to eliminate and the metalogic\index{metalogic} needed to
justify them as simple as possible.
Put another way, we had a goal of being able to
eliminate all definitions with direct mechanical substitution and to
verify easily the soundness of the definitions.

\subsubsection{Example of direct definitions}

We achieved this goal in almost all cases in \texttt{set.mm}.
Sometimes this makes the definitions more complex and less
intuitive.
For example, the traditional way to define addition of
natural numbers is to define an operation called {\em
successor}\index{successor} (which means ``plus one'' and is denoted by
``${\rm suc}$''), then define addition recursively\index{recursive
definition} with the two definitions $n + 0 = n$ and $m + {\rm suc}\,n =
{\rm suc} (m + n)$.  Although this definition seems simple and obvious,
the method to eliminate the definition is not obvious:  in the second
part of the definition, addition is defined in terms of itself.  By
eliminating the definition, we don't mean repeatedly applying it to
specific $m$ and $n$ but rather showing the explicit, closed-form
set-theoretical expression that $m + n$ represents, that will work for
any $m$ and $n$ and that does not have a $+$ sign on its right-hand
side.  For a recursive definition like this not to be circular
(creative), there are some hidden, underlying assumptions we must make,
for example that the natural numbers have a certain kind of order.

In \texttt{set.mm} we chose to start with the direct (though complex and
nonintuitive) definition then derive from it the standard recursive
definition.
For example, the closed-form definition used in \texttt{set.mm}
for the addition operation on ordinals\index{ordinal
addition}\index{addition!of ordinals} (of which natural numbers are a
subset) is

\setbox\startprefix=\hbox{\tt \ \ df-oadd\ \$a\ }
\setbox\contprefix=\hbox{\tt \ \ \ \ \ \ \ \ \ \ \ \ \ }
\startm
\m{\vdash}\m{+_o}\m{=}\m{(}\m{x}\m{\in}\m{{\rm On}}\m{,}\m{y}\m{\in}\m{{\rm
On}}\m{\mapsto}\m{(}\m{{\rm rec}}\m{(}\m{(}\m{z}\m{\in}\m{{\rm
V}}\m{\mapsto}\m{{\rm suc}}\m{z}\m{)}\m{,}\m{x}\m{)}\m{`}\m{y}\m{)}\m{)}
\endm
\noindent which depends on ${\rm rec}$.

\subsubsection{Recursion operators}

The above definition of \texttt{df-oadd} depends on the definition of
${\rm rec}$, a ``recursion operator''\index{recursion operator} with
the definition \texttt{df-rdg}:

\setbox\startprefix=\hbox{\tt \ \ df-rdg\ \$a\ }
\setbox\contprefix=\hbox{\tt \ \ \ \ \ \ \ \ \ \ \ \ }
\startm
\m{\vdash}\m{{\rm
rec}}\m{(}\m{F}\m{,}\m{I}\m{)}\m{=}\m{\mathrm{recs}}\m{(}\m{(}\m{g}\m{\in}\m{{\rm
V}}\m{\mapsto}\m{{\rm if}}\m{(}\m{g}\m{=}\m{\varnothing}\m{,}\m{I}\m{,}\m{{\rm
if}}\m{(}\m{{\rm Lim}}\m{{\rm dom}}\m{g}\m{,}\m{\bigcup}\m{{\rm
ran}}\m{g}\m{,}\m{(}\m{F}\m{`}\m{(}\m{g}\m{`}\m{\bigcup}\m{{\rm
dom}}\m{g}\m{)}\m{)}\m{)}\m{)}\m{)}\m{)}
\endm

\noindent which can be further broken down with definitions shown in
Section~\ref{setdefinitions}.

This definition of ${\rm rec}$
defines a recursive definition generator on ${\rm On}$ (the class of ordinal
numbers) with characteristic function $F$ and initial value $I$.
This operation allows us to define, with
compact direct definitions, functions that are usually defined in
textbooks with recursive definitions.
The price paid with our approach
is the complexity of our ${\rm rec}$ operation
(especially when {\tt df-recs} that it is built on is also eliminated).
But once we get past this hurdle, definitions that would otherwise be
recursive become relatively simple, as in for example {\tt oav}, from
which we prove the recursive textbook definition as theorems {\tt oa0}, {\tt
oasuc}, and {\tt oalim} (with the help of theorems {\tt rdg0}, {\tt rdgsuc},
and {\tt rdglim2a}).  We can also restrict the ${\rm rec}$ operation to
define otherwise recursive functions on the natural numbers $\omega$; see {\tt
fr0g} and {\tt frsuc}.  Our ${\rm rec}$ operation apparently does not appear
in published literature, although closely related is Definition 25.2 of
[Quine] p. 177, which he uses to ``turn...a recursion into a genuine or
direct definition" (p. 174).  Note that the ${\rm if}$ operations (see
{\tt df-if}) select cases based on whether the domain of $g$ is zero, a
successor, or a limit ordinal.

An important use of this definition ${\rm rec}$ is in the recursive sequence
generator {\tt df-seq} on the natural numbers (as a subset of the
complex infinite sequences such as the factorial function {\tt df-fac} and
integer powers {\tt df-exp}).

The definition of ${\rm rec}$ depends on ${\rm recs}$.
New direct usage of the more powerful (and more primitive) ${\rm recs}$
construct is discouraged, but it is available when needed.
This
defines a function $\mathrm{recs} ( F )$ on ${\rm On}$, the class of ordinal
numbers, by transfinite recursion given a rule $F$ which sets the next
value given all values so far.
Unlike {\tt df-rdg} which restricts the
update rule to use only the previous value, this version allows the
update rule to use all previous values, which is why it is described
as ``strong,'' although it is actually more primitive.  See {\tt
recsfnon} and {\tt recsval} for the primary contract of this definition.
It is defined as:

\setbox\startprefix=\hbox{\tt \ \ df-recs\ \$a\ }
\setbox\contprefix=\hbox{\tt \ \ \ \ \ \ \ \ \ \ \ \ \ }
\startm
\m{\vdash}\m{\mathrm{recs}}\m{(}\m{F}\m{)}\m{=}\m{\bigcup}\m{\{}\m{f}\m{|}\m{\exists}\m{x}\m{\in}\m{{\rm
On}}\m{(}\m{f}\m{{\rm
Fn}}\m{x}\m{\wedge}\m{\forall}\m{y}\m{\in}\m{x}\m{(}\m{f}\m{`}\m{y}\m{)}\m{=}\m{(}\m{F}\m{`}\m{(}\m{f}\m{\restriction}\m{y}\m{)}\m{)}\m{)}\m{\}}
\endm

\subsubsection{Closing comments on direct definitions}

From these direct definitions the simpler, more
intuitive recursive definition is derived as a set of theorems.\index{natural
number}\index{addition}\index{recursive definition}\index{ordinal addition}
The end result is the same, but we completely eliminate the rather complex
metalogic that justifies the recursive definition.

Recursive definitions are often considered more efficient and intuitive than
direct ones once the metalogic has been learned or possibly just accepted as
correct.  However, it was felt that direct definition in \texttt{set.mm}
maximizes rigor by minimizing metalogic.  It can be eliminated effortlessly,
something that is difficult to do with a recursive definition.

Again, Metamath itself has no built-in technical limitation that prevents
multiple-part recursive definitions in the traditional textbook style.
Instead, our goal is to eliminate all definitions with
direct mechanical substitution and to verify easily the soundness of
definitions.

\subsection{Adding Constraints on Definitions}

The basic Metamath language and the Metamath program do
not have any built-in constraints on definitions, since they are just
\$a statements.

However, nothing prevents a verification system from verifying
additional rules to impose further limitations on definitions.
For example, the \texttt{mmj2}\index{mmj2} program
supports various kinds of
additional information comments (see section \ref{jcomment}).
One of their uses is to optionally verify additional constraints,
including constraints to verify that definitions meet certain
requirements.
These additional checks are required by the
continuous integration (CI)\index{continuous integration (CI)}
checks of the
\texttt{set.mm}\index{set theory database (\texttt{set.mm})}%
\index{Metamath Proof Explorer}
database.
This approach enables us to optionally impose additional requirements
on definitions if we wish, without necessarily imposing those rules on
all databases or requiring all verification systems to implement them.
In addition, this allows us to impose specialized constraints tailored
to one database while not requiring other systems to implement
those specialized constraints.

We impose two constraints on the
\texttt{set.mm}\index{set theory database (\texttt{set.mm})}%
\index{Metamath Proof Explorer} database
via the \texttt{mmj2}\index{mmj2} program that are worth discussing here:
a parse check and a definition soundness check.

% On February 11, 2019 8:32:32 PM EST, saueran@oregonstate.edu wrote:
% The following addition to the end of set.mm is accepted by the mmj2
% parser and definition checker and the metamath verifier(at least it was
% when I checked, you should check it too), and creates a contradiction by
% proving the theorem |- ph.
% ${
% wleftp $a wff ( ( ph ) $.
% wbothp $a wff ( ph ) $.
% df-leftp $a |- ( ( ( ph ) <-> -. ph ) $.
% df-bothp $a |- ( ( ph ) <-> ph ) $.
% anything $p |- ph $=
%   ( wbothp wn wi wleftp df-leftp biimpi df-bothp mpbir mpbi simplim ax-mp)
%   ABZAMACZDZCZMOEZOCQAEZNDZRNAFGSHIOFJMNKLAHJ $.
% $}
%
% This particular problem is countered by enabling, within mmj2,
% SetParser,mmj.verify.LRParser

First,
we enable a parse check in \texttt{mmj2} (through its
\texttt{SetParser} command) that requires that all new definitions
must \textit{not} create an ambiguous parse for a KLR(5) parser.
This prevents some errors such as definitions with imbalanced parentheses.

Second, we run a definition soundness check specific to
\texttt{set.mm} or databases similar to it.
(through the \texttt{definitionCheck} macro).
Some \texttt{\$a} statements (including all ax-* statemnets)
are excluded from these checks, as they will
always fail this simple check,
but they are appropriate for most definitions.
This check imposes a set of additional rules:

\begin{enumerate}

\item New definitions must be introduced using $=$ or $\leftrightarrow$.

\item No \texttt{\$a} statement introduced before this one is allowed to use the
symbol being defined in this definition, and the definition is not
allowed to use itself (except once, in the definiendum).

\item Every variable in the definiens should not be distinct

\item Every dummy variable in the definiendum
are required to be distinct from each other and from variables in
the definiendum.
To determine this, the system will look for a "justification" theorem
in the database, and if it is not there, attempt to briefly prove
$( \varphi \rightarrow \forall x \varphi )$  for each dummy variable x.

\item Every dummy variable should be a set variable,
unless there is a justification theorem available.

\item Every dummy variable must be bound
(if the system cannot determine this a justification theorem must be
provided).

\end{enumerate}

\subsection{Summary of Approach to Definitions}

In short, when being rigorous it turns out that
definitions can be subtle, sometimes requiring difficult
metatheorems to establish that they are not creative.

Instead of building such complications into the Metamath language itself,
the basic Metmath language and program simply treat traditional
axioms and definitions as the same kind of \texttt{\$a} statement.
We have then built various tools to enable people to
verify additional conditions as their creators believe is appropriate
for those specific databases, without complicating the Metamath foundations.

\chapter{The Metamath Program}\label{commands}

This chapter provides a reference manual for the
Metamath program.\index{Metamath!commands}

Current instructions for obtaining and installing the Metamath program
can be found at the \url{http://metamath.org} web site.
For Windows, there is a pre-compiled version called
\texttt{metamath.exe}.  For Unix, Linux, and Mac OS X
(which we will refer to collectively as ``Unix''), the Metamath program
can be compiled from its source code with the command
\begin{verbatim}
gcc *.c -o metamath
\end{verbatim}
using the \texttt{gcc} {\sc c} compiler available on those systems.

In the command syntax descriptions below, fields enclosed in square
brackets [\ ] are optional.  File names may be optionally enclosed in
single or double quotes.  This is useful if the file name contains
spaces or
slashes (\texttt{/}), such as in Unix path names, \index{Unix file
names}\index{file names!Unix} that might be confused with Metamath
command qualifiers.\index{Unix file names}\index{file names!Unix}

\section{Invoking Metamath}

Unix, Linux, and Mac OS X
have a command-line interface called the {\em
bash shell}.  (In Mac OS X, select the Terminal application from
Applications/Utilities.)  To invoke Metamath from the bash shell prompt,
assuming that the Metamath program is in the current directory, type
\begin{verbatim}
bash$ ./metamath
\end{verbatim}

To invoke Metamath from a Windows DOS or Command Prompt, assuming that
the Metamath program is in the current directory (or in a directory
included in the Path system environment variable), type
\begin{verbatim}
C:\metamath>metamath
\end{verbatim}

To use command-line arguments at invocation, the command-line arguments
should be a list of Metamath commands, surrounded by quotes if they
contain spaces.  In Windows, the surrounding quotes must be double (not
single) quotes.  For example, to read the database file \texttt{set.mm},
verify all proofs, and exit the program, type (under Unix)
\begin{verbatim}
bash$ ./metamath 'read set.mm' 'verify proof *' exit
\end{verbatim}
Note that in Unix, any directory path with \texttt{/}'s must be
surrounded by quotes so Metamath will not interpret the \texttt{/} as a
command qualifier.  So if \texttt{set.mm} is in the \texttt{/tmp}
directory, use for the above example
\begin{verbatim}
bash$ ./metamath 'read "/tmp/set.mm"' 'verify proof *' exit
\end{verbatim}

For convenience, if the command-line has one argument and no spaces in
the argument, the command is implicitly assumed to be \texttt{read}.  In
this one special case, \texttt{/}'s are not interpreted as command
qualifiers, so you don't need quotes around a Unix file name.  Thus
\begin{verbatim}
bash$ ./metamath /tmp/set.mm
\end{verbatim}
and
\begin{verbatim}
bash$ ./metamath "read '/tmp/set.mm'"
\end{verbatim}
are equivalent.


\section{Controlling Metamath}

The Metamath program was first developed on a {\sc vax/vms} system, and
some aspects of its command line behavior reflect this heritage.
Hopefully you will find it reasonably user-friendly once you get used to
it.

Each command line is a sequence of English-like words separated by
spaces, as in \texttt{show settings}.  Command words are not case
sensitive, and only as many letters are needed as are necessary to
eliminate ambiguity; for example, \texttt{sh se} would work for the
command \texttt{show settings}.  In some cases arguments such as file
names, statement labels, or symbol names are required; these are
case-sensitive (although file names may not be on some operating
systems).

A command line is entered by typing it in then pressing the {\em return}
({\em enter}) key.  To find out what commands are available, type
\texttt{?} at the \texttt{MM>} prompt.  To find out the choices at any
point in a command, press {\em return} and you will be prompted for
them.  The default choice (the one selected if you just press {\em
return}) is shown in brackets (\texttt{<>}).

You may also type \texttt{?} in place of a command word to force
Metamath to tell you what the choices are.  The \texttt{?} method won't
work, though, if a non-keyword argument such as a file name is expected
at that point, because the program will think that \texttt{?} is the
value of the argument.

Some commands have one or more optional qualifiers which modify the
behavior of the command.  Qualifiers are preceded by a slash
(\texttt{/}), such as in \texttt{read set.mm / verify}.  Spaces are
optional around the \texttt{/}.  If you need to use a space or
slash in a command
argument, as in a Unix file name, put single or double quotes around the
command argument.

The \texttt{open log} command will save everything you see on the
screen and is useful to help you recover should something go wrong in a
proof, or if you want to document a bug.

If a command responds with more than a screenful, you will be
prompted to \texttt{<return> to continue, Q to quit, or S to scroll to
end}.  \texttt{Q} or \texttt{q} (not case-sensitive) will complete the
command internally but will suppress further output until the next
\texttt{MM>} prompt.  \texttt{s} will suppress further pausing until the
next \texttt{MM>} prompt.  After the first screen, you are also
presented with the choice of \texttt{b} to go back a screenful.  Note
that \texttt{b} may also be entered at the \texttt{MM>} prompt
immediately after a command to scroll back through the output of that
command.

A command line enclosed in quotes is executed by your operating system.
See Section~\ref{oscmd}.

{\em Warning:} Pressing {\sc ctrl-c} will abort the Metamath program
unconditionally.  This means any unsaved work will be lost.


\subsection{\texttt{exit} Command}\index{\texttt{exit} command}

Syntax:  \texttt{exit} [\texttt{/force}]

This command exits from Metamath.  If there have been changes to the
source with the \texttt{save proof} or \texttt{save new{\char`\_}proof}
commands, you will be given an opportunity to \texttt{write source} to
permanently save the changes.

In Proof Assistant\index{Proof Assistant} mode, the \texttt{exit} command will
return to the \verb/MM>/ prompt. If there were changes to the proof, you will
be given an opportunity to \texttt{save new{\char`\_}proof}.

The \texttt{quit} command is a synonym for \texttt{exit}.

Optional qualifier:
    \texttt{/force} - Do not prompt if changes were not saved.  This qualifier is
        useful in \texttt{submit} command files (Section~\ref{sbmt})
        to ensure predictable behavior.





\subsection{\texttt{open log} Command}\index{\texttt{open log} command}
Syntax:  \texttt{open log} {\em file-name}

This command will open a log file that will store everything you see on
the screen.  It is useful to help recovery from a mistake in a long Proof
Assistant session, or to document bugs.\index{Metamath!bugs}

The log file can be closed with \texttt{close log}.  It will automatically be
closed upon exiting Metamath.



\subsection{\texttt{close log} Command}\index{\texttt{close log} command}
Syntax:  \texttt{close log}

The \texttt{close log} command closes a log file if one is open.  See
also \texttt{open log}.




\subsection{\texttt{submit} Command}\index{\texttt{submit} command}\label{sbmt}
Syntax:  \texttt{submit} {\em filename}

This command causes further command lines to be taken from the specified
file.  Note that any line beginning with an exclamation point (\texttt{!}) is
treated as a comment (i.e.\ ignored).  Also note that the scrolling
of the screen output is continuous, so you may want to open a log file
(see \texttt{open log}) to record the results that fly by on the screen.
After the lines in the file are exhausted, Metamath returns to its
normal user interface mode.

The \texttt{submit} command can be called recursively (i.e. from inside
of a \texttt{submit} command file).


Optional command qualifier:

    \texttt{/silent} - suppresses the screen output but still
        records the output in a log file if one is open.


\subsection{\texttt{erase} Command}\index{\texttt{erase} command}
Syntax:  \texttt{erase}

This command will reset Metamath to its starting state, deleting any
data\-base that was \texttt{read} in.
 If there have been changes to the
source with the \texttt{save proof} or \texttt{save new{\char`\_}proof}
commands, you will be given an opportunity to \texttt{write source} to
permanently save the changes.



\subsection{\texttt{set echo} Command}\index{\texttt{set echo} command}
Syntax:  \texttt{set echo on} or \texttt{set echo off}

The \texttt{set echo on} command will cause command lines to be echoed with any
abbreviations expanded.  While learning the Metamath commands, this
feature will show you the exact command that your abbreviated input
corresponds to.



\subsection{\texttt{set scroll} Command}\index{\texttt{set scroll} command}
Syntax:  \texttt{set scroll prompted} or \texttt{set scroll continuous}

The Metamath command line interface starts off in the \texttt{prompted} mode,
which means that you will be prompted to continue or quit after each
full screen in a long listing.  In \texttt{continuous} mode, long listings will be
scrolled without pausing.

% LaTeX bug? (1) \texttt{\_} puts out different character than
% \texttt{{\char`\_}}
%  = \verb$_$  (2) \texttt{{\char`\_}} puts out garbage in \subsection
%  argument
\subsection{\texttt{set width} Command}\index{\texttt{set
width} command}
Syntax:  \texttt{set width} {\em number}

Metamath assumes the width of your screen is 79 characters (chosen
because the Command Prompt in Windows XP has a wrapping bug at column
80).  If your screen is wider or narrower, this command allows you to
change this default screen width.  A larger width is advantageous for
logging proofs to an output file to be printed on a wide printer.  A
smaller width may be necessary on some terminals; in this case, the
wrapping of the information messages may sometimes seem somewhat
unnatural, however.  In \LaTeX\index{latex@{\LaTeX}!characters per line},
there is normally a maximum of 61 characters per line with typewriter
font.  (The examples in this book were produced with 61 characters per
line.)

\subsection{\texttt{set height} Command}\index{\texttt{set
height} command}
Syntax:  \texttt{set height} {\em number}

Metamath assumes your screen height is 24 lines of characters.  If your
screen is taller or shorter, this command lets you to change the number
of lines at which the display pauses and prompts you to continue.

\subsection{\texttt{beep} Command}\index{\texttt{beep} command}

Syntax:  \texttt{beep}

This command will produce a beep.  By typing it ahead after a
long-running command has started, it will alert you that the command is
finished.  For convenience, \texttt{b} is an abbreviation for
\texttt{beep}.

Note:  If \texttt{b} is typed at the \texttt{MM>} prompt immediately
after the end of a multiple-page display paged with ``\texttt{Press
<return> for more}...'' prompts, then the \texttt{b} will back up to the
previous page rather than perform the \texttt{beep} command.
In that case you must type the unabbreviated \texttt{beep} form
of the command.

\subsection{\texttt{more} Command}\index{\texttt{more} command}

Syntax:  \texttt{more} {\em filename}

This command will display the contents of an {\sc ascii} file on your
screen.  (This command is provided for convenience but is not very
powerful.  See Section~\ref{oscmd} to invoke your operating system's
command to do this, such as the \texttt{more} command in Unix.)

\subsection{Operating System Commands}\index{operating system
command}\label{oscmd}

A line enclosed in single or double quotes will be executed by your
computer's operating system if it has a command line interface.  For
example, on a {\sc vax/vms} system,
\verb/MM> 'dir'/
will print disk directory contents.  Note that this feature will not
work on the Macintosh prior to Mac OS X, which does not have a command
line interface.

For your convenience, the trailing quote is optional.

\subsection{Size Limitations in Metamath}

In general, there are no fixed, predefined limits\index{Metamath!memory
limits} on how many labels, tokens\index{token}, statements, etc.\ that
you may have in a database file.  The Metamath program uses 32-bit
variables (64-bit on 64-bit CPUs) as indices for almost all internal
arrays, which are allocated dynamically as needed.



\section{Reading and Writing Files}

The following commands create new files:  the \texttt{open} commands;
the \texttt{write} commands; the \texttt{/html},
\texttt{/alt{\char`\_}html}, \texttt{/brief{\char`\_}html},
\texttt{/brief{\char`\_}alt{\char`\_}html} qualifiers of \texttt{show
statement}, and \texttt{midi}.  The following commands append to files
previously opened:  the \texttt{/tex} qualifier of \texttt{show proof}
and \texttt{show new{\char`\_}proof}; the \texttt{/tex} and
\texttt{/simple{\char`\_}tex} qualifiers of \texttt{show statement}; the
\texttt{close} commands; and all screen dialog between \texttt{open log}
and \texttt{close log}.

The commands that create new files will not overwrite an existing {\em
filename} but will rename the existing one to {\em
filename}\texttt{{\char`\~}1}.  An existing {\em
filename}\texttt{{\char`\~}1} is renamed {\em
filename}\texttt{{\char`\~}2}, etc.\ up to {\em
filename}\texttt{{\char`\~}9}.  An existing {\em
filename}\texttt{{\char`\~}9} is deleted.  This makes recovery from
mistakes easier but also will clutter up your directory, so occasionally
you may want to clean up (delete) these \texttt{{\char`\~}}$n$ files.


\subsection{\texttt{read} Command}\index{\texttt{read} command}
Syntax:  \texttt{read} {\em file-name} [\texttt{/verify}]

This command will read in a Metamath language source file and any included
files.  Normally it will be the first thing you do when entering Metamath.
Statement syntax is checked, but proof syntax is not checked.
Note that the file name may be enclosed in single or double quotes;
this is useful if the file name contains slashes, as might be the case
under Unix.

If you are getting an ``\texttt{?Expected VERIFY}'' error
when trying to read a Unix file name with slashes, you probably haven't
quoted it.\index{Unix file names}\index{file names!Unix}

If you are prompted for the file name (by pressing {\em return}
 after \texttt{read})
you should {\em not} put quotes around it, even if it is a Unix file name
with slashes.

Optional command qualifier:

    \texttt{/verify} - Verify all proofs as the database is read in.  This
         qualifier will slow down reading in the file.  See \texttt{verify
         proof} for more information on file error-checking.

See also \texttt{erase}.



\subsection{\texttt{write source} Command}\index{\texttt{write source} command}
Syntax:  \texttt{write source} {\em filename}
[\texttt{/rewrap}]
[\texttt{/split}]
% TeX doesn't handle this long line with tt text very well,
% so force a line break here.
[\texttt{/keep\_includes}] {\\}
[\texttt{/no\_versioning}]

This command will write the contents of a Metamath\index{database}
database into a file.\index{source file}

Optional command qualifiers:

\texttt{/rewrap} -
Reformats statements and comments according to the
convention used in the set.mm database.
It unwraps the
lines in the comment before each \texttt{\$a} and \texttt{\$p} statement,
then it
rewraps the line.  You should compare the output to the original
to make sure that the desired effect results; if not, go back to
the original source.  The wrapped line length honors the
\texttt{set width}
parameter currently in effect.  Note:  Text
enclosed in \texttt{<HTML>}...\texttt{</HTML>} tags is not modified by the
\texttt{/rewrap} qualifier.
Proofs themselves are not reformatted;
use \texttt{save proof * / compressed} to do that.
An isolated tilde (\~{}) is always kept on the same line as the following
symbol, so you can find all comment references to a symbol by
searching for \~{} followed by a space and that symbol
(this is useful for finding cross-references).
Incidentally, \texttt{save proof} also honors the \texttt{set width}
parameter currently in effect.

\texttt{/split} - Files included in the source using the expression
\$[ \textit{inclfile} \$] will be
written into separate files instead of being included in a single output
file.  The name of each separately written included file will be the
\textit{inclfile} argument of its inclusion command.

\texttt{/keep\_includes} - If a source file has includes but is written as a
single file by omitting \texttt{/split}, by default the included files will
be deleted (actually just renamed with a \char`\~1 suffix unless
\texttt{/no\_versioning} is specified) to prevent the possibly confusing
source duplication in both the output file and the included file.
The \texttt{/keep\_includes} qualifier will prevent this deletion.

\texttt{/no\_versioning} - Backup files suffixed with \char`\~1 are not created.


\section{Showing Status and Statements}



\subsection{\texttt{show settings} Command}\index{\texttt{show settings} command}
Syntax:  \texttt{show settings}

This command shows the state of various parameters.

\subsection{\texttt{show memory} Command}\index{\texttt{show memory} command}
Syntax:  \texttt{show memory}

This command shows the available memory left.  It is not meaningful
on most modern operating systems,
which have virtual memory.\index{Metamath!memory usage}


\subsection{\texttt{show labels} Command}\index{\texttt{show labels} command}
Syntax:  \texttt{show labels} {\em label-match} [\texttt{/all}]
   [\texttt{/linear}]

This command shows the labels of \texttt{\$a} and \texttt{\$p}
statements that match {\em label-match}.  A \verb$*$ in {label-match}
matches zero or more characters.  For example, \verb$*abc*def$ will match all
labels containing \verb$abc$ and ending with \verb$def$.

Optional command qualifiers:

   \texttt{/all} - Include matches for \texttt{\$e} and \texttt{\$f}
   statement labels.

   \texttt{/linear} - Display only one label per line.  This can be useful for
       building scripts in conjunction with the utilities under the
       \texttt{tools}\index{\texttt{tools} command} command.



\subsection{\texttt{show statement} Command}\index{\texttt{show statement} command}
Syntax:  \texttt{show statement} {\em label-match} [{\em qualifiers} (see below)]

This command provides information about a statement.  Only statements
that have labels (\texttt{\$f}\index{\texttt{\$f} statement},
\texttt{\$e}\index{\texttt{\$e} statement},
\texttt{\$a}\index{\texttt{\$a} statement}, and
\texttt{\$p}\index{\texttt{\$p} statement}) may be specified.
If {\em label-match}
contains wildcard (\verb$*$) characters, all matching statements will be
displayed in the order they occur in the database.

Optional qualifiers (only one qualifier at a time is allowed):

    \texttt{/comment} - This qualifier includes the comment that immediately
       precedes the statement.

    \texttt{/full} - Show complete information about each statement,
       and show all
       statements matching {\em label} (including \texttt{\$e}
       and \texttt{\$f} statements).

    \texttt{/tex} - This qualifier will write the statement information to the
       \LaTeX\ file previously opened with \texttt{open tex}.  See
       Section~\ref{texout}.

    \texttt{/simple{\char`\_}tex} - The same as \texttt{/tex}, except that
       \LaTeX\ macros are not used for formatting equations, allowing easier
       manual edits of the output for slide presentations, etc.

    \texttt{/html}\index{html generation@{\sc html} generation},
       \texttt{/alt{\char`\_}html}, \texttt{/brief{\char`\_}html},
       \texttt{/brief{\char`\_}alt{\char`\_}html} -
       These qualifiers invoke a special mode of
       \texttt{show statement} that
       creates a web page for the statement.  They may not be used with
       any other qualifier.  See Section~\ref{htmlout} or
       \texttt{help html} in the program.


\subsection{\texttt{search} Command}\index{\texttt{search} command}
Syntax:  search {\em label-match}
\texttt{"}{\em symbol-match}{\tt}" [\texttt{/all}] [\texttt{/comments}]
[\texttt{/join}]

This command searches all \texttt{\$a} and \texttt{\$p} statements
matching {\em label-match} for occurrences of {\em symbol-match}.  A
\verb@*@ in {\em label-match} matches any label character.  A \verb@$*@
in {\em symbol-match} matches any sequence of symbols.  The symbols in
{\em symbol-match} must be separated by white space.  The quotes
surrounding {\em symbol-match} may be single or double quotes.  For
example, \texttt{search b}\verb@* "-> $* ch"@ will list all statements
whose labels begin with \texttt{b} and contain the symbols \verb@->@ and
\texttt{ch} surrounding any symbol sequence (including no symbol
sequence).  The wildcards \texttt{?} and \texttt{\$?} are also available
to match individual characters in labels and symbols respectively; see
\texttt{help search} in the Metamath program for details on their usage.

Optional command qualifiers:

    \texttt{/all} - Also search \texttt{\$e} and \texttt{\$f} statements.

    \texttt{/comments} - Search the comment that immediately precedes each
        label-matched statement for {\em symbol-match}.  In this case
        {\em symbol-match} is an arbitrary, non-case-sensitive character
        string.  Quotes around {\em symbol-match} are optional if there
        is no ambiguity.

    \texttt{/join} - In the case of a \texttt{\$a} or \texttt{\$p} statement,
	prepend its \texttt{\$e}
	hypotheses for searching. The
	\texttt{/join} qualifier has no effect in \texttt{/comments} mode.

\section{Displaying and Verifying Proofs}


\subsection{\texttt{show proof} Command}\index{\texttt{show proof} command}
Syntax:  \texttt{show proof} {\em label-match} [{\em qualifiers} (see below)]

This command displays the proof of the specified
\texttt{\$p}\index{\texttt{\$p} statement} statement in various formats.
The {\em label-match} may contain wildcard (\verb@$*@) characters to match
multiple statements.  Without any qualifiers, only the logical steps
will be shown (i.e.\ syntax construction steps will be omitted), in an
indented format.

Most of the time, you will use
    \texttt{show proof} {\em label}
to see just the proof steps corresponding to logical inferences.

Optional command qualifiers:

    \texttt{/essential} - The proof tree is trimmed of all
        \texttt{\$f}\index{\texttt{\$f} statement} hypotheses before
        being displayed.  (This is the default, and it is redundant to
        specify it.)

    \texttt{/all} - the proof tree is not trimmed of all \texttt{\$f} hypotheses before
        being displayed.  \texttt{/essential} and \texttt{/all} are mutually exclusive.

    \texttt{/from{\char`\_}step} {\em step} - The display starts at the specified
        step.  If
        this qualifier is omitted, the display starts at the first step.

    \texttt{/to{\char`\_}step} {\em step} - The display ends at the specified
        step.  If this
        qualifier is omitted, the display ends at the last step.

    \texttt{/tree{\char`\_}depth} {\em number} - Only
         steps at less than the specified proof
        tree depth are displayed.  Sometimes useful for obtaining an overview of
        the proof.

    \texttt{/reverse} - The steps are displayed in reverse order.

    \texttt{/renumber} - When used with \texttt{/essential}, the steps are renumbered
        to correspond only to the essential steps.

    \texttt{/tex} - The proof is converted to \LaTeX\ and\index{latex@{\LaTeX}}
        stored in the file opened
        with \texttt{open tex}.  See Section~\ref{texout} or
        \texttt{help tex} in the program.

    \texttt{/lemmon} - The proof is displayed in a non-indented format known
        as Lemmon style, with explicit previous step number references.
        If this qualifier is omitted, steps are indented in a tree format.

    \texttt{/start{\char`\_}column} {\em number} - Overrides the default column
        (16)
        at which the formula display starts in a Lemmon-style display.  May be
        used only in conjunction with \texttt{/lemmon}.

    \texttt{/normal} - The proof is displayed in normal format suitable for
        inclusion in a Metamath source file.  May not be used with any other
        qualifier.

    \texttt{/compressed} - The proof is displayed in compressed format
        suitable for inclusion in a Metamath source file.  May not be used with
        any other qualifier.

    \texttt{/statement{\char`\_}summary} - Summarizes all statements (like a
        brief \texttt{show statement})
        used by the proof.  It may not be used with any other qualifier
        except \texttt{/essential}.

    \texttt{/detailed{\char`\_}step} {\em step} - Shows the details of what is
        happening at
        a specific proof step.  May not be used with any other qualifier.
        The {\em step} is the step number shown when displaying a
        proof without the \texttt{/renumber} qualifier.


\subsection{\texttt{show usage} Command}\index{\texttt{show usage} command}
Syntax:  \texttt{show usage} {\em label-match} [\texttt{/recursive}]

This command lists the statements whose proofs make direct reference to
the statement specified.

Optional command qualifier:

    \texttt{/recursive} - Also include statements whose proofs ultimately
        depend on the statement specified.



\subsection{\texttt{show trace\_back} Command}\index{\texttt{show
       trace{\char`\_}back} command}
Syntax:  \texttt{show trace{\char`\_}back} {\em label-match} [\texttt{/essential}] [\texttt{/axioms}]
    [\texttt{/tree}] [\texttt{/depth} {\em number}]

This command lists all statements that the proof of the \texttt{\$p}
statement(s) specified by {\em label-match} depends on.

Optional command qualifiers:

    \texttt{/essential} - Restrict the trace-back to \texttt{\$e}
        \index{\texttt{\$e} statement} hypotheses of proof trees.

    \texttt{/axioms} - List only the axioms that the proof ultimately depends on.

    \texttt{/tree} - Display the trace-back in an indented tree format.

    \texttt{/depth} {\em number} - Restrict the \texttt{/tree} trace-back to the
        specified indentation depth.

    \texttt{/count{\char`\_}steps} - Count the number of steps the proof has
       all the way back to axioms.  If \texttt{/essential} is specified,
       expansions of variable-type hypotheses (syntax constructions) are not counted.

\subsection{\texttt{verify proof} Command}\index{\texttt{verify proof} command}
Syntax:  \texttt{verify proof} {\em label-match} [\texttt{/syntax{\char`\_}only}]

This command verifies the proofs of the specified statements.  {\em
label-match} may contain wild card characters (\texttt{*}) to verify
more than one proof; for example \verb/*abc*def/ will match all labels
containing \texttt{abc} and ending with \texttt{def}.
The command \texttt{verify proof *} will verify all proofs in the database.

Optional command qualifier:

    \texttt{/syntax{\char`\_}only} - This qualifier will perform a check of syntax
        and RPN
        stack violations only.  It will not verify that the proof is
        correct.  This qualifier is useful for quickly determining which
        proofs are incomplete (i.e.\ are under development and have \texttt{?}'s
        in them).

{\em Note:} \texttt{read}, followed by \texttt{verify proof *}, will ensure
 the database is free
from errors in the Metamath language but will not check the markup notation
in comments.
You can also check the markup notation by running \texttt{verify markup *}
as discussed in Section~\ref{verifymarkup}; see also the discussion
on generating {\sc HTML} in Section~\ref{htmlout}.

\subsection{\texttt{verify markup} Command}\index{\texttt{verify markup} command}\label{verifymarkup}
Syntax:  \texttt{verify markup} {\em label-match}
[\texttt{/date{\char`\_}skip}]
[\texttt{/top{\char`\_}date{\char`\_}skip}] {\\}
[\texttt{/file{\char`\_}skip}]
[\texttt{/verbose}]

This command checks comment markup and other informal conventions we have
adopted.  It error-checks the latexdef, htmldef, and althtmldef statements
in the \texttt{\$t} statement of a Metamath source file.\index{error checking}
It error-checks any \texttt{`}...\texttt{`},
\texttt{\char`\~}~\textit{label},
and bibliographic markups in statement descriptions.
It checks that
\texttt{\$p} and \texttt{\$a} statements
have the same content when their labels start with
``ax'' and ``ax-'' respectively but are otherwise identical, for example
ax4 and ax-4.
It also verifies the date consistency of ``(Contributed by...),''
``(Revised by...),'' and ``(Proof shortened by...)'' tags in the comment
above each \texttt{\$a} and \texttt{\$p} statement.

Optional command qualifiers:

    \texttt{/date{\char`\_}skip} - This qualifier will
        skip date consistency checking,
        which is usually not required for databases other than
	\texttt{set.mm}.

    \texttt{/top{\char`\_}date{\char`\_}skip} - This qualifier will check date consistency except
        that the version date at the top of the database file will not
        be checked.  Only one of
        \texttt{/date{\char`\_}skip} and
        \texttt{/top{\char`\_}date{\char`\_}skip} may be
        specified.

    \texttt{/file{\char`\_}skip} - This qualifier will skip checks that require
        external files to be present, such as checking GIF existence and
        bibliographic links to mmset.html or equivalent.  It is useful
        for doing a quick check from a directory without these files.

    \texttt{/verbose} - Provides more information.  Currently it provides a list
        of axXXX vs. ax-XXX matches.

\subsection{\texttt{save proof} Command}\index{\texttt{save proof} command}
Syntax:  \texttt{save proof} {\em label-match} [\texttt{/normal}]
   [\texttt{/compressed}]

The \texttt{save proof} command will reformat a proof in one of two formats and
replace the existing proof in the source buffer\index{source
buffer}.  It is useful for
converting between proof formats.  Note that a proof will not be
permanently saved until a \texttt{write source} command is issued.

Optional command qualifiers:

    \texttt{/normal} - The proof is saved in the normal format (i.e., as a
        sequence
        of labels, which is the defined format of the basic Metamath
        language).\index{basic language}  This is the default format that
        is used if a qualifier
        is omitted.

    \texttt{/compressed} - The proof is saved in the compressed format which
        reduces storage requirements for a database.
        See Appendix~\ref{compressed}.




\section{Creating Proofs}\label{pfcommands}\index{Proof Assistant}

Before using the Proof Assistant, you must add a \texttt{\$p} to your
source file (using a text editor) containing the statement you want to
prove.  Its proof should consist of a single \texttt{?}, meaning
``unknown step.''  Example:
\begin{verbatim}
equid $p x = x $= ? $.
\end{verbatim}

To enter the Proof assistant, type \texttt{prove} {\em label}, e.g.
\texttt{prove equid}.  Metamath will respond with the \texttt{MM-PA>}
prompt.

Proofs are created working backwards from the statement being proved,
primarily using a series of \texttt{assign} commands.  A proof is
complete when all steps are assigned to statements and all steps are
unified and completely known.  During the creation of a proof, Metamath
will allow only operations that are legal based on what is known up to
that point.  For example, it will not allow an \texttt{assign} of a
statement that cannot be unified with the unknown proof step being
assigned.

{\em Important:}
The Proof Assistant is
{\em not} a tool to help you discover proofs.  It is just a tool to help
you add them to the database.  For a tutorial read
Section~\ref{frstprf}.
To practice using the Proof Assistant, you may
want to \texttt{prove} an existing theorem, then delete all steps with
\texttt{delete all}, then re-create it with the Proof Assistant while
looking at its proof display (before deletion).
You might want to figure out your first few proofs completely
and write them down by hand, before using the Proof Assistant, though
not everyone finds that effective.

{\em Important:}
The \texttt{undo} command if very helpful when entering a proof, because
it allows you to undo a previously-entered step.
In addition, we suggest that you
keep track of your work with a log file (\texttt{open
log}) and save it frequently (\texttt{save new{\char`\_}proof},
\texttt{write source}).
You can use \texttt{delete} to reverse an \texttt{assign}.
You can also do \texttt{delete floating{\char`\_}hypotheses}, then
\texttt{initialize all}, then \texttt{unify all /interactive} to
reinitialize bad unifications made accidentally or by bad
\texttt{assign}s.  You cannot reverse a \texttt{delete} except by
a relevant \texttt{undo} or using
\texttt{exit /force} then reentering the Proof Assistant to recover from
the last \texttt{save new{\char`\_}proof}.

The following commands are available in the Proof Assistant (at the
\texttt{MM-PA>} prompt) to help you create your proof.  See the
individual commands for more detail.

\begin{itemize}
\item[]
    \texttt{show new{\char`\_}proof} [\texttt{/all},...] - Displays the
        proof in progress.  You will use this command a lot; see \texttt{help
        show new{\char`\_}proof} to become familiar with its qualifiers.  The
        qualifiers \texttt{/unknown} and \texttt{/not{\char`\_}unified} are
        useful for seeing the work remaining to be done.  The combination
        \texttt{/all/unknown} is useful for identifying dummy variables that must be
        assigned, or attempts to use illegal syntax, when \texttt{improve all}
        is unable to complete the syntax constructions.  Unknown variables are
        shown as \texttt{\$1}, \texttt{\$2},...
\item[]
    \texttt{assign} {\em step} {\em label} - Assigns an unknown {\em step}
        number with the statement
        specified by {\em label}.
\item[]
    \texttt{let variable} {\em variable}
        \texttt{= "}{\em symbol sequence}\texttt{"}
          - Forces a symbol
        sequence to replace an unknown variable (such as \texttt{\$1}) in a proof.
        It is useful
        for helping difficult unifications, and it is necessary when you have
        dummy variables that eventually must be assigned a name.
\item[]
    \texttt{let step} {\em step} \texttt{= "}{\em symbol sequence}\texttt{"} -
          Forces a symbol sequence
        to replace the contents of a proof step, provided it can be
        unified with the existing step contents.  (I rarely use this.)
\item[]
    \texttt{unify step} {\em step} (or \texttt{unify all}) - Unifies
        the source and target of
        a step.  If you specify a specific step, you will be prompted
        to select among the unifications that are possible.  If you
        specify \texttt{all}, all steps with unique unifications, but only
        those steps, will be
        unified.  \texttt{unify all /interactive} goes through all non-unified
        steps.
\item[]
    \texttt{initialize} {\em step} (or \texttt{all}) - De-unifies the target and source of
        a step (or all steps), as well as the hypotheses of the source,
        and makes all variables in the source unknown.  Useful to recover from
        an \texttt{assign} or \texttt{let} mistake that
        resulted in incorrect unifications.
\item[]
    \texttt{delete} {\em step} (or \texttt{all} or \texttt{floating{\char`\_}hypotheses}) -
        Deletes the specified
        step(s).  \texttt{delete floating{\char`\_}hypotheses}, then \texttt{initialize all}, then
        \texttt{unify all /interactive} is useful for recovering from mistakes
        where incorrect unifications assigned wrong math symbol strings to
        variables.
\item[]
    \texttt{improve} {\em step} (or \texttt{all}) -
      Automatically creates a proof for steps (with no unknown variables)
      whose proof requires no statements with \texttt{\$e} hypotheses.  Useful
      for filling in proofs of \texttt{\$f} hypotheses.  The \texttt{/depth}
      qualifier will also try statements whose \texttt{\$e} hypotheses contain
      no new variables.  {\em Warning:} Save your work (with \texttt{save
      new{\char`\_}proof} then \texttt{write source}) before using
      \texttt{/depth = 2} or greater, since the search time grows
      exponentially and may never terminate in a reasonable time, and you
      cannot interrupt the search.  I have found that it is rare for
      \texttt{/depth = 3} or greater to be useful.
 \item[]
    \texttt{save new{\char`\_}proof} - Saves the proof in progress in the program's
        internal database buffer.  To save it permanently into the database file,
        use \texttt{write source} after
        \texttt{save new{\char`\_}proof}.  To revert to the last
        \texttt{save new{\char`\_}proof},
        \texttt{exit /force} from the Proof Assistant then re-enter the Proof
        Assistant.
 \item[]
    \texttt{match step} {\em step} (or \texttt{match all}) - Shows what
        statements are
        possibilities for the \texttt{assign} statement. (This command
        is not very
        useful in its present form and hopefully will be improved
        eventually.  In the meantime, use the \texttt{search} statement for
        candidates matching specific math token combinations.)
 \item[]
 \texttt{minimize{\char`\_}with}\index{\texttt{minimize{\char`\_}with} command}
% 3/10/07 Note: line-breaking the above results in duplicate index entries
     - After a proof is complete, this command will attempt
        to match other database theorems to the proof to see if the proof
        size can be reduced as a result.  See \texttt{help
        minimize{\char`\_}with} in the
        Metamath program for its usage.
 \item[]
 \texttt{undo}\index{\texttt{undo} command}
    - Undo the effect of a proof-changing command (all but the
      \texttt{show} and \texttt{save} commands above).
 \item[]
 \texttt{redo}\index{\texttt{redo} command}
    - Reverse the previous \texttt{undo}.
\end{itemize}

The following commands set parameters that may be relevant to your proof.
Consult the individual \texttt{help set}... commands.
\begin{itemize}
   \item[] \texttt{set unification{\char`\_}timeout}
 \item[]
    \texttt{set search{\char`\_}limit}
  \item[]
    \texttt{set empty{\char`\_}substitution} - note that default is \texttt{off}
\end{itemize}

Type \texttt{exit} to exit the \texttt{MM-PA>}
 prompt and get back to the \texttt{MM>} prompt.
Another \texttt{exit} will then get you out of Metamath.



\subsection{\texttt{prove} Command}\index{\texttt{prove} command}
Syntax:  \texttt{prove} {\em label}

This command will enter the Proof Assistant\index{Proof Assistant}, which will
allow you to create or edit the proof of the specified statement.
The command-line prompt will change from \texttt{MM>} to \texttt{MM-PA>}.

Note:  In the present version (0.177) of
Metamath\index{Metamath!limitations of version 0.177}, the Proof
Assistant does not verify that \texttt{\$d}\index{\texttt{\$d}
statement} restrictions are met as a proof is being built.  After you
have completed a proof, you should type \texttt{save new{\char`\_}proof}
followed by \texttt{verify proof} {\em label} (where {\em label} is the
statement you are proving with the \texttt{prove} command) to verify the
\texttt{\$d} restrictions.

See also: \texttt{exit}

\subsection{\texttt{set unification\_timeout} Command}\index{\texttt{set
unification{\char`\_}timeout} command}
Syntax:  \texttt{set unification{\char`\_}timeout} {\em number}

(This command is available outside the Proof Assistant but affects the
Proof Assistant\index{Proof Assistant} only.)

Sometimes the Proof Assistant will inform you that a unification
time-out occurred.  This may happen when you try to \texttt{unify}
formulas with many temporary variables\index{temporary variable}
(\texttt{\$1}, \texttt{\$2}, etc.), since the time to compute all possible
unifications may grow exponentially with the number of variables.  If
you want Metamath to try harder (and you're willing to wait longer) you
may increase this parameter.  \texttt{show settings} will show you the
current value.

\subsection{\texttt{set empty\_substitution} Command}\index{\texttt{set
empty{\char`\_}substitution} command}
% These long names can't break well in narrow mode, and even "sloppy"
% is not enough. Work around this by not demanding justification.
\begin{flushleft}
Syntax:  \texttt{set empty{\char`\_}substitution on} or \texttt{set
empty{\char`\_}substitution off}
\end{flushleft}

(This command is available outside the Proof Assistant but affects the
Proof Assistant\index{Proof Assistant} only.)

The Metamath language allows variables to be
substituted\index{substitution!variable}\index{variable substitution}
with empty symbol sequences\index{empty substitution}.  However, in many
formal systems\index{formal system} this will never happen in a valid
proof.  Allowing for this possibility increases the likelihood of
ambiguous unifications\index{ambiguous
unification}\index{unification!ambiguous} during proof creation.
The default is that
empty substitutions are not allowed; for formal systems requiring them,
you must \texttt{set empty{\char`\_}substitution on}.
(An example where this must be \texttt{on}
would be a system that implements a Deduction Rule and in
which deductions from empty assumption lists would be permissible.  The
MIU-system\index{MIU-system} described in Appendix~\ref{MIU} is another
example.)
Note that empty substitutions are
always permissible in proof verification (VERIFY PROOF...) outside the
Proof Assistant.  (See the MIU system in the Metamath book for an example
of a system needing empty substitutions; another example would be a
system that implements a Deduction Rule and in which deductions from
empty assumption lists would be permissible.)

It is better to leave this \texttt{off} when working with \texttt{set.mm}.
Note that this command does not affect the way proofs are verified with
the \texttt{verify proof} command.  Outside of the Proof Assistant,
substitution of empty sequences for math symbols is always allowed.

\subsection{\texttt{set search\_limit} Command}\index{\texttt{set
search{\char`\_}limit} command} Syntax:  \texttt{set search{\char`\_}limit} {\em
number}

(This command is available outside the Proof Assistant but affects the
Proof Assistant\index{Proof Assistant} only.)

This command sets a parameter that determines when the \texttt{improve} command
in Proof Assistant mode gives up.  If you want \texttt{improve} to search harder,
you may increase it.  The \texttt{show settings} command tells you its current
value.


\subsection{\texttt{show new\_proof} Command}\index{\texttt{show
new{\char`\_}proof} command}
Syntax:  \texttt{show new{\char`\_}proof} [{\em
qualifiers} (see below)]

This command (available only in Proof Assistant mode) displays the proof
in progress.  It is identical to the \texttt{show proof} command, except that
there is no statement argument (since it is the statement being proved) and
the following qualifiers are not available:

    \texttt{/statement{\char`\_}summary}

    \texttt{/detailed{\char`\_}step}

Also, the following additional qualifiers are available:

    \texttt{/unknown} - Shows only steps that have no statement assigned.

    \texttt{/not{\char`\_}unified} - Shows only steps that have not been unified.

Note that \texttt{/essential}, \texttt{/depth}, \texttt{/unknown}, and
\texttt{/not{\char`\_}unified} may be
used in any combination; each of them effectively filters out additional
steps from the proof display.

See also:  \texttt{show proof}






\subsection{\texttt{assign} Command}\index{\texttt{assign} command}
Syntax:   \texttt{assign} {\em step} {\em label} [\texttt{/no{\char`\_}unify}]

   and:   \texttt{assign first} {\em label}

   and:   \texttt{assign last} {\em label}


This command, available in the Proof Assistant only, assigns an unknown
step (one with \texttt{?} in the \texttt{show new{\char`\_}proof}
listing) with the statement specified by {\em label}.  The assignment
will not be allowed if the statement cannot be unified with the step.

If \texttt{last} is specified instead of {\em step} number, the last
step that is shown by \texttt{show new{\char`\_}proof /unknown} will be
used.  This can be useful for building a proof with a command file (see
\texttt{help submit}).  It also makes building proofs faster when you know
the assignment for the last step.

If \texttt{first} is specified instead of {\em step} number, the first
step that is shown by \texttt{show new{\char`\_}proof /unknown} will be
used.

If {\em step} is zero or negative, the -{\em step}th from last unknown
step, as shown by \texttt{show new{\char`\_}proof /unknown}, will be
used.  \texttt{assign -1} {\em label} will assign the penultimate
unknown step, \texttt{assign -2} {\em label} the antepenultimate, and
\texttt{assign 0} {\em label} is the same as \texttt{assign last} {\em
label}.

Optional command qualifier:

    \texttt{/no{\char`\_}unify} - do not prompt user to select a unification if there is
        more than one possibility.  This is useful for noninteractive
        command files.  Later, the user can \texttt{unify all /interactive}.
        (The assignment will still be automatically unified if there is only
        one possibility and will be refused if unification is not possible.)



\subsection{\texttt{match} Command}\index{\texttt{match} command}
Syntax:  \texttt{match step} {\em step} [\texttt{/max{\char`\_}essential{\char`\_}hyp}
{\em number}]

    and:  \texttt{match all} [\texttt{/essential}]
          [\texttt{/max{\char`\_}essential{\char`\_}hyp} {\em number}]

This command, available in the Proof Assistant only, shows what
statements can be unified with the specified step(s).  {\em Note:} In
its current form, this command is not very useful because of the large
number of matches it reports.
It may be enhanced in the future.  In the meantime, the \texttt{search}
command can often provide finer control over locating theorems of interest.

Optional command qualifiers:

    \texttt{/max{\char`\_}essential{\char`\_}hyp} {\em number} - filters out
        of the list any statements
        with more than the specified number of
        \texttt{\$e}\index{\texttt{\$e} statement} hypotheses.

    \texttt{/essential{\char`\_}only} - in the \texttt{match all} statement, only
        the steps that
        would be listed in the \texttt{show new{\char`\_}proof /essential} display are
        matched.



\subsection{\texttt{let} Command}\index{\texttt{let} command}
Syntax: \texttt{let variable} {\em variable} = \verb/"/{\em symbol-sequence}\verb/"/

  and: \texttt{let step} {\em step} = \verb/"/{\em symbol-sequence}\verb/"/

These commands, available in the Proof Assistant\index{Proof Assistant}
only, assign a temporary variable\index{temporary variable} or unknown
step with a specific symbol sequence.  They are useful in the middle of
creating a proof, when you know what should be in the proof step but the
unification algorithm doesn't yet have enough information to completely
specify the temporary variables.  A ``temporary variable'' is one that
has the form \texttt{\$}{\em nn} in the proof display, such as
\texttt{\$1}, \texttt{\$2}, etc.  The {\em symbol-sequence} may contain
other unknown variables if desired.  Examples:

    \verb/let variable $32 = "A = B"/

    \verb/let variable $32 = "A = $35"/

    \verb/let step 10 = '|- x = x'/

    \verb/let step -2 = "|- ( $7 = ph )"/

Any symbol sequence will be accepted for the \texttt{let variable}
command.  Only those symbol sequences that can be unified with the step
will be accepted for \texttt{let step}.

The \texttt{let} commands ``zap'' the proof with information that can
only be verified when the proof is built up further.  If you make an
error, the command sequence \texttt{delete
floating{\char`\_}hypotheses}, \texttt{initialize all}, and
\texttt{unify all /interactive} will undo a bad \texttt{let} assignment.

If {\em step} is zero or negative, the -{\em step}th from last unknown
step, as shown by \texttt{show new{\char`\_}proof /unknown}, will be
used.  The command \texttt{let step 0} = \verb/"/{\em
symbol-sequence}\verb/"/ will use the last unknown step, \texttt{let
step -1} = \verb/"/{\em symbol-sequence}\verb/"/ the penultimate, etc.
If {\em step} is positive, \texttt{let step} may be used to assign known
(in the sense of having previously been assigned a label with
\texttt{assign}) as well as unknown steps.

Either single or double quotes can surround the {\em symbol-sequence} as
long as they are different from any quotes inside a {\em
symbol-sequence}.  If {\em symbol-sequence} contains both kinds of
quotes, see the instructions at the end of \texttt{help let} in the
Metamath program.


\subsection{\texttt{unify} Command}\index{\texttt{unify} command}
Syntax:  \texttt{unify step} {\em step}

      and:   \texttt{unify all} [\texttt{/interactive}]

These commands, available in the Proof Assistant only, unify the source
and target of the specified step(s). If you specify a specific step, you
will be prompted to select among the unifications that are possible.  If
you specify \texttt{all}, only those steps with unique unifications will be
unified.

Optional command qualifier for \texttt{unify all}:

    \texttt{/interactive} - You will be prompted to select among the
        unifications
        that are possible for any steps that do not have unique
        unifications.  (Otherwise \texttt{unify all} will bypass these.)

See also \texttt{set unification{\char`\_}timeout}.  The default is
100000, but increasing it to 1000000 can help difficult cases.  Manually
assigning some or all of the unknown variables with the \texttt{let
variable} command also helps difficult cases.



\subsection{\texttt{initialize} Command}\index{\texttt{initialize} command}
Syntax:  \texttt{initialize step} {\em step}

    and: \texttt{initialize all}

These commands, available in the Proof Assistant\index{Proof Assistant}
only, ``de-unify'' the target and source of a step (or all steps), as
well as the hypotheses of the source, and makes all variables in the
source and the source's hypotheses unknown.  This command is useful to
help recover from incorrect unifications that resulted from an incorrect
\texttt{assign}, \texttt{let}, or unification choice.  Part or all of
the command sequence \texttt{delete floating{\char`\_}hypotheses},
\texttt{initialize all}, and \texttt{unify all /interactive} will recover
from incorrect unifications.

See also:  \texttt{unify} and \texttt{delete}



\subsection{\texttt{delete} Command}\index{\texttt{delete} command}
Syntax:  \texttt{delete step} {\em step}

   and:      \texttt{delete all} -- {\em Warning: dangerous!}

   and:      \texttt{delete floating{\char`\_}hypotheses}

These commands are available in the Proof Assistant only.  The
\texttt{delete step} command deletes the proof tree section that
branches off of the specified step and makes the step become unknown.
\texttt{delete all} is equivalent to \texttt{delete step} {\em step}
where {\em step} is the last step in the proof (i.e.\ the beginning of
the proof tree).

In most cases the \texttt{undo} command is the best way to undo
a previous step.
An alternative is to salvage your last \texttt{save
new{\char`\_}proof} by exiting and reentering the Proof Assistant.
For this to work, keep a log file open to record your work
and to do \texttt{save new{\char`\_}proof} frequently, especially before
\texttt{delete}.

\texttt{delete floating{\char`\_}hypotheses} will delete all sections of
the proof that branch off of \texttt{\$f}\index{\texttt{\$f} statement}
statements.  It is sometimes useful to do this before an
\texttt{initialize} command to recover from an error.  Note that once a
proof step with a \texttt{\$f} hypothesis as the target is completely
known, the \texttt{improve} command can usually fill in the proof for
that step.  Unlike the deletion of logical steps, \texttt{delete
floating{\char`\_}hypotheses} is a relatively safe command that is
usually easy to recover from.



\subsection{\texttt{improve} Command}\index{\texttt{improve} command}
\label{improve}
Syntax:  \texttt{improve} {\em step} [\texttt{/depth} {\em number}]
                                               [\texttt{/no{\char`\_}distinct}]

   and:   \texttt{improve first} [\texttt{/depth} {\em number}]
                                              [\texttt{/no{\char`\_}distinct}]

   and:   \texttt{improve last} [\texttt{/depth} {\em number}]
                                              [\texttt{/no{\char`\_}distinct}]

   and:   \texttt{improve all} [\texttt{/depth} {\em number}]
                                              [\texttt{/no{\char`\_}distinct}]

These commands, available in the Proof Assistant\index{Proof Assistant}
only, try to find proofs automatically for unknown steps whose symbol
sequences are completely known.  They are primarily useful for filling in
proofs of \texttt{\$f}\index{\texttt{\$f} statement} hypotheses.  The
search will be restricted to statements having no
\texttt{\$e}\index{\texttt{\$e} statement} hypotheses.

\begin{sloppypar} % narrow
Note:  If memory is limited, \texttt{improve all} on a large proof may
overflow memory.  If you use \texttt{set unification{\char`\_}timeout 1}
before \texttt{improve all}, there will usually be sufficient
improvement to easily recover and completely \texttt{improve} the proof
later on a larger computer.  Warning:  Once memory has overflowed, there
is no recovery.  If in doubt, save the intermediate proof (\texttt{save
new{\char`\_}proof} then \texttt{write source}) before \texttt{improve
all}.
\end{sloppypar}

If \texttt{last} is specified instead of {\em step} number, the last
step that is shown by \texttt{show new{\char`\_}proof /unknown} will be
used.

If \texttt{first} is specified instead of {\em step} number, the first
step that is shown by \texttt{show new{\char`\_}proof /unknown} will be
used.

If {\em step} is zero or negative, the -{\em step}th from last unknown
step, as shown by \texttt{show new{\char`\_}proof /unknown}, will be
used.  \texttt{improve -1} will use the penultimate
unknown step, \texttt{improve -2} {\em label} the antepenultimate, and
\texttt{improve 0} is the same as \texttt{improve last}.

Optional command qualifier:

    \texttt{/depth} {\em number} - This qualifier will cause the search
        to include
        statements with \texttt{\$e} hypotheses (but no new variables in
        the \texttt{\$e}
        hypotheses), provided that the backtracking has not exceeded the
        specified depth. {\em Warning:}  Try \texttt{/depth 1},
        then \texttt{2}, then \texttt{3}, etc.
        in sequence because of possible exponential blowups.  Save your
        work before trying \texttt{/depth} greater than \texttt{1}!

    \texttt{/no{\char`\_}distinct} - Skip trial statements that have
        \texttt{\$d}\index{\texttt{\$d} statement} requirements.
        This qualifier will prevent assignments that might violate \texttt{\$d}
        requirements but it also could miss possible legal assignments.

See also: \texttt{set search{\char`\_}limit}

\subsection{\texttt{save new\_proof} Command}\index{\texttt{save
new{\char`\_}proof} command}
Syntax:  \texttt{save new{\char`\_}proof} {\em label} [\texttt{/normal}]
   [\texttt{/compressed}]

The \texttt{save new{\char`\_}proof} command is available in the Proof
Assistant only.  It saves the proof in progress in the source
buffer\index{source buffer}.  \texttt{save new{\char`\_}proof} may be
used to save a completed proof, or it may be used to save a proof in
progress in order to work on it later.  If an incomplete proof is saved,
any user assignments with \texttt{let step} or \texttt{let variable}
will be lost, as will any ambiguous unifications\index{ambiguous
unification}\index{unification!ambiguous} that were resolved manually.
To help make recovery easier, it can be helpful to \texttt{improve all}
before \texttt{save new{\char`\_}proof} so that the incomplete proof
will have as much information as possible.

Note that the proof will not be permanently saved until a \texttt{write
source} command is issued.

Optional command qualifiers:

    \texttt{/normal} - The proof is saved in the normal format (i.e., as a
        sequence of labels, which is the defined format of the basic Metamath
        language).\index{basic language}  This is the default format that
        is used if a qualifier is omitted.

    \texttt{/compressed} - The proof is saved in the compressed format, which
        reduces storage requirements for a database.
        (See Appendix~\ref{compressed}.)


\section{Creating \LaTeX\ Output}\label{texout}\index{latex@{\LaTeX}}

You can generate \LaTeX\ output given the
information in a database.
The database must already include the necessary typesetting information
(see section \ref{tcomment} for how to provide this information).

The \texttt{show statement} and \texttt{show proof} commands each have a
special \texttt{/tex} command qualifier that produces \LaTeX\ output.
(The \texttt{show statement} command also has the
\texttt{/simple{\char`\_}tex} qualifier for output that is easier to
edit by hand.)  Before you can use them, you must open a \LaTeX\ file to
which to send their output.  A typical complete session will use this
sequence of Metamath commands:

\begin{verbatim}
read set.mm
open tex example.tex
show statement a1i /tex
show proof a1i /all/lemmon/renumber/tex
show statement uneq2 /tex
show proof uneq2 /all/lemmon/renumber/tex
close tex
\end{verbatim}

See Section~\ref{mathcomments} for information on comment markup and
Appendix~\ref{ASCII} for information on how math symbol translation is
specified.

To format and print the \LaTeX\ source, you will need the \LaTeX\
program, which is standard on most Linux installations and available for
Windows.  On Linux, in order to create a {\sc pdf} file, you will
typically type at the shell prompt
\begin{verbatim}
$ pdflatex example.tex
\end{verbatim}

\subsection{\texttt{open tex} Command}\index{\texttt{open tex} command}
Syntax:  \texttt{open tex} {\em file-name} [\texttt{/no{\char`\_}header}]

This command opens a file for writing \LaTeX\
source\index{latex@{\LaTeX}} and writes a \LaTeX\ header to the file.
\LaTeX\ source can be written with the \texttt{show proof}, \texttt{show
new{\char`\_}proof}, and \texttt{show statement} commands using the
\texttt{/tex} qualifier.

The mapping to \LaTeX\ symbols is defined in a special comment
containing a \texttt{\$t} token, described in Appendix~\ref{ASCII}.

There is an optional command qualifier:

    \texttt{/no{\char`\_}header} - This qualifier prevents a standard
        \LaTeX\ header and trailer
        from being included with the output \LaTeX\ code.


\subsection{\texttt{close tex} Command}\index{\texttt{close tex} command}
Syntax:  \texttt{close tex}

This command writes a trailer to any \LaTeX\ file\index{latex@{\LaTeX}}
that was opened with \texttt{open tex} (unless
\texttt{/no{\char`\_}header} was used with \texttt{open tex}) and closes
the \LaTeX\ file.


\section{Creating {\sc HTML} Output}\label{htmlout}

You can generate {\sc html} web pages given the
information in a database.
The database must already include the necessary typesetting information
(see section \ref{tcomment} for how to provide this information).
The ability to produce {\sc html} web pages was added in Metamath version
0.07.30.

To create an {\sc html} output file(s) for \texttt{\$a} or \texttt{\$p}
statement(s), use
\begin{quote}
    \texttt{show statement} {\em label-match} \texttt{/html}
\end{quote}
The output file will be named {\em label-match}\texttt{.html}
for each match.  When {\em
label-match} has wildcard (\texttt{*}) characters, all statements with
matching labels will have {\sc html} files produced for them.  Also,
when {\em label-match} has a wildcard (\texttt{*}) character, two additional
files, \texttt{mmdefinitions.html} and \texttt{mmascii.html} will be
produced.  To produce {\em only} these two additional files, you can use
\texttt{?*}, which will not match any statement label, in place of {\em
label-match}.

There are three other qualifiers for \texttt{show statement} that also
generate {\sc HTML} code.  These are \texttt{/alt{\char`\_}html},
\texttt{/brief{\char`\_}html}, and
\texttt{/brief{\char`\_}alt{\char`\_}html}, and are described in the
next section.

The command
\begin{quote}
    \texttt{show statement} {\em label-match} \texttt{/alt{\char`\_}html}
\end{quote}
does the same as \texttt{show statement} {\em label-match} \texttt{/html},
except that the {\sc html} code for the symbols is taken from
\texttt{althtmldef} statements instead of \texttt{htmldef} statements in
the \texttt{\$t} comment.

The command
\begin{verbatim}
show statement * /brief_html
\end{verbatim}
invokes a special mode that just produces definition and theorem lists
accompanied by their symbol strings, in a format suitable for copying and
pasting into another web page (such as the tutorial pages on the
Metamath web site).

Finally, the command
\begin{verbatim}
show statement * /brief_alt_html
\end{verbatim}
does the same as \texttt{show statement * / brief{\char`\_}html}
for the alternate {\sc html}
symbol representation.

A statement's comment can include a special notation that provides a
certain amount of control over the {\sc HTML} version of the comment.  See
Section~\ref{mathcomments} (p.~\pageref{mathcomments}) for the comment
markup features.

The \texttt{write theorem{\char`\_}list} and \texttt{write bibliography}
commands, which are described below, provide as a side effect complete
error checking for all of the features described in this
section.\index{error checking}

\subsection{\texttt{write theorem\_list}
Command}\index{\texttt{write theorem{\char`\_}list} command}
Syntax:  \texttt{write theorem{\char`\_}list}
[\texttt{/theorems{\char`\_}per{\char`\_}page} {\em number}]

This command writes a list of all of the \texttt{\$a} and \texttt{\$p}
statements in the database into a web page file
 called \texttt{mmtheorems.html}.
When additional files are needed, they are called
\texttt{mmtheorems2.html}, \texttt{mmtheorems3.html}, etc.

Optional command qualifier:

    \texttt{/theorems{\char`\_}per{\char`\_}page} {\em number} -
 This qualifier specifies the number of statements to
        write per web page.  The default is 100.

{\em Note:} In version 0.177\index{Metamath!limitations of version
0.177} of Metamath, the ``Nearby theorems'' links on the individual
web pages presuppose 100 theorems per page when linking to the theorem
list pages.  Therefore the \texttt{/theorems{\char`\_}per{\char`\_}page}
qualifier, if it specifies a number other than 100, will cause the
individual web pages to be out of sync and should not be used to
generate the main theorem list for the web site.  This may be
fixed in a future version.


\subsection{\texttt{write bibliography}\label{wrbib}
Command}\index{\texttt{write bibliography} command}
Syntax:  \texttt{write bibliography} {\em filename}

This command reads an existing {\sc html} bibliographic cross-reference
file, normally called \texttt{mmbiblio.html}, and updates it per the
bibliographic links in the database comments.  The file is updated
between the {\sc html} comment lines \texttt{<!--
{\char`\#}START{\char`\#} -->} and \texttt{<!-- {\char`\#}END{\char`\#}
-->}.  The original input file is renamed to {\em
filename}\texttt{{\char`\~}1}.

A bibliographic reference is indicated with the reference name
in brackets, such as  \texttt{Theorem 3.1 of
[Monk] p.\ 22}.
See Section~\ref{htmlout} (p.~\pageref{htmlout}) for
syntax details.


\subsection{\texttt{write recent\_additions}
Command}\index{\texttt{write recent{\char`\_}additions} command}
Syntax:  \texttt{write recent{\char`\_}additions} {\em filename}
[\texttt{/limit} {\em number}]

This command reads an existing ``Recent Additions'' {\sc html} file,
normally called \texttt{mmrecent.html}, and updates it with the
descriptions of the most recently added theorems to the database.
 The file is updated between
the {\sc html} comment lines \texttt{<!-- {\char`\#}START{\char`\#} -->}
and \texttt{<!-- {\char`\#}END{\char`\#} -->}.  The original input file
is renamed to {\em filename}\texttt{{\char`\~}1}.

Optional command qualifier:

    \texttt{/limit} {\em number} -
 This qualifier specifies the number of most recent theorems to
   write to the output file.  The default is 100.


\section{Text File Utilities}

\subsection{\texttt{tools} Command}\index{\texttt{tools} command}
Syntax:  \texttt{tools}

This command invokes an easy-to-use, general purpose utility for
manipulating the contents of {\sc ascii} text files.  Upon typing
\texttt{tools}, the command-line prompt will change to \texttt{TOOLS>}
until you type \texttt{exit}.  The \texttt{tools} commands can be used
to perform simple, global edits on an input/output file,
such as making a character string substitution on each line, adding a
string to each line, and so on.  A typical use of this utility is
to build a \texttt{submit} input file to perform a common operation on a
list of statements obtained from \texttt{show label} or \texttt{show
usage}.

The actions of most of the \texttt{tools} commands can also be
performed with equivalent (and more powerful) Unix shell commands, and
some users may find those more efficient.  But for Windows users or
users not comfortable with Unix, \texttt{tools} provides an
easy-to-learn alternative that is adequate for most of the
script-building tasks needed to use the Metamath program effectively.

\subsection{\texttt{help} Command (in \texttt{tools})}
Syntax:  \texttt{help}

The \texttt{help} command lists the commands available in the
\texttt{tools} utility, along with a brief description.  Each command,
in turn, has its own help, such as \texttt{help add}.  As with
Metamath's \texttt{MM>} prompt, a complete command can be entered at
once, or just the command word can be typed, causing the program to
prompt for each argument.

\vskip 1ex
\noindent Line-by-line editing commands:

  \texttt{add} - Add a specified string to each line in a file.

  \texttt{clean} - Trim spaces and tabs on each line in a file; convert
         characters.

  \texttt{delete} - Delete a section of each line in a file.

  \texttt{insert} - Insert a string at a specified column in each line of
        a file.

  \texttt{substitute} - Make a simple substitution on each line of the file.

  \texttt{tag} - Like \texttt{add}, but restricted to a range of lines.

  \texttt{swap} - Swap the two halves of each line in a file.

\vskip 1ex
\noindent Other file-processing commands:

  \texttt{break} - Break up (tokenize) a file into a list of tokens (one per
        line).

  \texttt{build} - Build a file with multiple tokens per line from a list.

  \texttt{count} - Count the occurrences in a file of a specified string.

  \texttt{number} - Create a list of numbers.

  \texttt{parallel} - Put two files in parallel.

  \texttt{reverse} - Reverse the order of the lines in a file.

  \texttt{right} - Right-justify lines in a file (useful before sorting
         numbers).

%  \texttt{tag} - Tag edit updates in a program for revision control.

  \texttt{sort} - Sort the lines in a file with key starting at
         specified string.

  \texttt{match} - Extract lines containing (or not) a specified string.

  \texttt{unduplicate} - Eliminate duplicate occurrences of lines in a file.

  \texttt{duplicate} - Extract first occurrence of any line occurring
         more than

   \ \ \    once in a file, discarding lines occurring exactly once.

  \texttt{unique} - Extract lines occurring exactly once in a file.

  \texttt{type} (10 lines) - Display the first few lines in a file.
                  Similar to Unix \texttt{head}.

  \texttt{copy} - Similar to Unix \texttt{cat} but safe (same input
         and output file allowed).

  \texttt{submit} - Run a script containing \texttt{tools} commands.

\vskip 1ex

\noindent Note:
  \texttt{unduplicate}, \texttt{duplicate}, and \texttt{unique} also
 sort the lines as a side effect.


\subsection{Using \texttt{tools} to Build Metamath \texttt{submit}
Scripts}

The \texttt{break} command is typically used to break up a series of
statement labels, such as the output of Metamath's \texttt{show usage},
into one label per line.  The other \texttt{tools} commands can then be
used to add strings before and after each statement label to specify
commands to be performed on the statement.  The \texttt{parallel}
command is useful when a statement label must be mentioned more than
once on a line.

Very often a \texttt{submit} script for Metamath will require multiple
command lines for each statement being processed.  For example, you may
want to enter the Proof Assistant, \texttt{minimize{\char`\_}with} your
latest theorem, \texttt{save} the new proof, and \texttt{exit} the Proof
Assistant.  To accomplish this, you can build a file with these four
commands for each statement on a single line, separating each command
with a designated character such as \texttt{@}.  Then at the end you can
\texttt{substitute} each \texttt{@} with \texttt{{\char`\\}n} to break
up the lines into individual command lines (see \texttt{help
substitute}).


\subsection{Example of a \texttt{tools} Session}

To give you a quick feel for the \texttt{tools} utility, we show a
simple session where we create a file \texttt{n.txt} with 3 lines, add
strings before and after each line, and display the lines on the screen.
You can experiment with the various commands to gain experience with the
\texttt{tools} utility.

\begin{verbatim}
MM> tools
Entering the Text Tools utilities.
Type HELP for help, EXIT to exit.
TOOLS> number
Output file <n.tmp>? n.txt
First number <1>?
Last number <10>? 3
Increment <1>?
TOOLS> add
Input/output file? n.txt
String to add to beginning of each line <>? This is line
String to add to end of each line <>? .
The file n.txt has 3 lines; 3 were changed.
First change is on line 1:
This is line 1.
TOOLS> type n.txt
This is line 1.
This is line 2.
This is line 3.
TOOLS> exit
Exiting the Text Tools.
Type EXIT again to exit Metamath.
MM>
\end{verbatim}



\appendix
\chapter{Sample Representations}
\label{ASCII}

This Appendix provides a sample of {\sc ASCII} representations,
their corresponding traditional mathematical symbols,
and a discussion of their meanings
in the \texttt{set.mm} database.
These are provided in order of appearance.
This is only a partial list, and new definitions are routinely added.
A complete list is available at \url{http://metamath.org}.

These {\sc ASCII} representations, along
with information on how to display them,
are defined in the \texttt{set.mm} database file inside
a special comment called a \texttt{\$t} {\em
comment}\index{\texttt{\$t} comment} or {\em typesetting
comment.}\index{typesetting comment}
A typesetting comment
is indicated by the appearance of the
two-character string \texttt{\$t} at the beginning of the comment.
For more information,
see Section~\ref{tcomment}, p.~\pageref{tcomment}.

In the following table the ``{\sc ASCII}'' column shows the {\sc ASCII}
representation,
``Symbol'' shows the mathematical symbolic display
that corresponds to that {\sc ASCII} representation, ``Labels'' shows
the key label(s) that define the representation, and
``Description'' provides a description about the symbol.
As usual, ``iff'' is short for ``if and only if.''\index{iff}
In most cases the ``{\sc ASCII}'' column only shows
the key token, but it will sometimes show a sequence of tokens
if that is necessary for clarity.

{\setlength{\extrarowsep}{4pt} % Keep rows from being too close together
\begin{longtabu}   { @{} c c l X }
\textbf{ASCII} & \textbf{Symbol} & \textbf{Labels} & \textbf{Description} \\
\endhead
\texttt{|-} & $\vdash$ & &
  ``It is provable that...'' \\
\texttt{ph} & $\varphi$ & \texttt{wph} &
  The wff (boolean) variable phi,
  conventionally the first wff variable. \\
\texttt{ps} & $\psi$ & \texttt{wps} &
  The wff (boolean) variable psi,
  conventionally the second wff variable. \\
\texttt{ch} & $\chi$ & \texttt{wch} &
  The wff (boolean) variable chi,
  conventionally the third wff variable. \\
\texttt{-.} & $\lnot$ & \texttt{wn} &
  Logical not. E.g., if $\varphi$ is true, then $\lnot \varphi$ is false. \\
\texttt{->} & $\rightarrow$ & \texttt{wi} &
  Implies, also known as material implication.
  In classical logic the expression $\varphi \rightarrow \psi$ is true
  if either $\varphi$ is false or $\psi$ is true (or both), that is,
  $\varphi \rightarrow \psi$ has the same meaning as
  $\lnot \varphi \lor \psi$ (as proven in theorem \texttt{imor}). \\
\texttt{<->} & $\leftrightarrow$ &
  \hyperref[df-bi]{\texttt{df-bi}} &
  Biconditional (aka is-equals for boolean values).
  $\varphi \leftrightarrow \psi$ is true iff
  $\varphi$ and $\psi$ have the same value. \\
\texttt{\char`\\/} & $\lor$ &
  \makecell[tl]{{\hyperref[df-or]{\texttt{df-or}}}, \\
	         \hyperref[df-3or]{\texttt{df-3or}}} &
  Disjunction (logical ``or''). $\varphi \lor \psi$ is true iff
  $\varphi$, $\psi$, or both are true. \\
\texttt{/\char`\\} & $\land$ &
  \makecell[tl]{{\hyperref[df-an]{\texttt{df-an}}}, \\
                 \hyperref[df-3an]{\texttt{df-3an}}} &
  Conjunction (logical ``and''). $\varphi \land \psi$ is true iff
  both $\varphi$ and $\psi$ are true. \\
\texttt{A.} & $\forall$ &
  \texttt{wal} &
  For all; the wff $\forall x \varphi$ is true iff
  $\varphi$ is true for all values of $x$. \\
\texttt{E.} & $\exists$ &
  \hyperref[df-ex]{\texttt{df-ex}} &
  There exists; the wff
  $\exists x \varphi$ is true iff
  there is at least one $x$ where $\varphi$ is true. \\
\texttt{[ y / x ]} & $[ y / x ]$ &
  \hyperref[df-sb]{\texttt{df-sb}} &
  The wff $[ y / x ] \varphi$ produces
  the result when $y$ is properly substituted for $x$ in $\varphi$
  ($y$ replaces $x$).
  % This is elsb4
  % ( [ x / y ] z e. y <-> z e. x )
  For example,
  $[ x / y ] z \in y$ is the same as $z \in x$. \\
\texttt{E!} & $\exists !$ &
  \hyperref[df-eu]{\texttt{df-eu}} &
  There exists exactly one;
  $\exists ! x \varphi$ is true iff
  there is at least one $x$ where $\varphi$ is true. \\
\texttt{\{ y | phi \}}  & $ \{ y | \varphi \}$ &
  \hyperref[df-clab]{\texttt{df-clab}} &
  The class of all sets where $\varphi$ is true. \\
\texttt{=} & $ = $ &
  \hyperref[df-cleq]{\texttt{df-cleq}} &
  Class equality; $A = B$ iff $A$ equals $B$. \\
\texttt{e.} & $ \in $ &
  \hyperref[df-clel]{\texttt{df-clel}} &
  Class membership; $A \in B$ if $A$ is a member of $B$. \\
\texttt{{\char`\_}V} & {\rm V} &
  \hyperref[df-v]{\texttt{df-v}} &
  Class of all sets (not itself a set). \\
\texttt{C\_} & $ \subseteq $ &
  \hyperref[df-ss]{\texttt{df-ss}} &
  Subclass (subset); $A \subseteq B$ is true iff
  $A$ is a subclass of $B$. \\
\texttt{u.} & $ \cup $ &
  \hyperref[df-un]{\texttt{df-un}} &
  $A \cup B$ is the union of classes $A$ and $B$. \\
\texttt{i^i} & $ \cap $ &
  \hyperref[df-in]{\texttt{df-in}} &
  $A \cap B$ is the intersection of classes $A$ and $B$. \\
\texttt{\char`\\} & $ \setminus $ &
  \hyperref[df-dif]{\texttt{df-dif}} &
  $A \setminus B$ (set difference)
  is the class of all sets in $A$ except for those in $B$. \\
\texttt{(/)} & $ \varnothing $ &
  \hyperref[df-nul]{\texttt{df-nul}} &
  $ \varnothing $ is the empty set (aka null set). \\
\texttt{\char`\~P} & $ \cal P $ &
  \hyperref[df-pw]{\texttt{df-pw}} &
  Power class. \\
\texttt{<.\ A , B >.} & $\langle A , B \rangle$ &
  \hyperref[df-op]{\texttt{df-op}} &
  The ordered pair $\langle A , B \rangle$. \\
\texttt{( F ` A )} & $ ( F ` A ) $ &
  \hyperref[df-fv]{\texttt{df-fv}} &
  The value of function $F$ when applied to $A$. \\
\texttt{\_i} & $ i $ &
  \texttt{df-i} &
  The square root of negative one. \\
\texttt{x.} & $ \cdot $ &
  \texttt{df-mul} &
  Complex number multiplication; $2~\cdot~3~=~6$. \\
\texttt{CC} & $ \mathbb{C} $ &
  \texttt{df-c} &
  The set of complex numbers. \\
\texttt{RR} & $ \mathbb{R} $ &
  \texttt{df-r} &
  The set of real numbers. \\
\end{longtabu}
} % end of extrarowsep

\chapter{Compressed Proofs}
\label{compressed}\index{compressed proof}\index{proof!compressed}

The proofs in the \texttt{set.mm} set theory database are stored in compressed
format for efficiency.  Normally you needn't concern yourself with the
compressed format, since you can display it with the usual proof display tools
in the Metamath program (\texttt{show proof}\ldots) or convert it to the normal
RPN proof format described in Section~\ref{proof} (with \texttt{save proof}
{\em label} \texttt{/normal}).  However for sake of completeness we describe the
format here and show how it maps to the normal RPN proof format.

A compressed proof, located between \texttt{\$=} and \texttt{\$.}\ keywords, consists
of a left parenthesis, a sequence of statement labels, a right parenthesis,
and a sequence of upper-case letters \texttt{A} through \texttt{Z} (with optional
white space between them).  White space must surround the parentheses
and the labels.  The left parenthesis tells Metamath that a
compressed proof follows.  (A normal RPN proof consists of just a sequence of
labels, and a parenthesis is not a legal character in a label.)

The sequence of upper-case letters corresponds to a sequence of integers
with the following mapping.  Each integer corresponds to a proof step as
described later.
\begin{center}
  \texttt{A} = 1 \\
  \texttt{B} = 2 \\
   \ldots \\
  \texttt{T} = 20 \\
  \texttt{UA} = 21 \\
  \texttt{UB} = 22 \\
   \ldots \\
  \texttt{UT} = 40 \\
  \texttt{VA} = 41 \\
  \texttt{VB} = 42 \\
   \ldots \\
  \texttt{YT} = 120 \\
  \texttt{UUA} = 121 \\
   \ldots \\
  \texttt{YYT} = 620 \\
  \texttt{UUUA} = 621 \\
   etc.
\end{center}

In other words, \texttt{A} through \texttt{T} represent the
least-significant digit in base 20, and \texttt{U} through \texttt{Y}
represent zero or more most-significant digits in base 5, where the
digits start counting at 1 instead of the usual 0. With this scheme, we
don't need white space between these ``numbers.''

(In the design of the compressed proof format, only upper-case letters,
as opposed to say all non-whitespace printable {\sc ascii} characters other than
%\texttt{\$}, was chosen to make the compressed proof a little less
%displeasing to the eye, at the expense of a typical 20\% compression
\texttt{\$}, were chosen so as not to collide with most text editor
searches, at the expense of a typical 20\% compression
loss.  The base 5/base 20 grouping, as opposed to say base 6/base 19,
was chosen by experimentally determining the grouping that resulted in
best typical compression.)

The letter \texttt{Z} identifies (tags) a proof step that is identical to one
that occurs later on in the proof; it helps shorten the proof by not requiring
that identical proof steps be proved over and over again (which happens often
when building wff's).  The \texttt{Z} is placed immediately after the
least-significant digit (letters \texttt{A} through \texttt{T}) that ends the integer
corresponding to the step to later be referenced.

The integers that the upper-case letters correspond to are mapped to labels as
follows.  If the statement being proved has $m$ mandatory hypotheses, integers
1 through $m$ correspond to the labels of these hypotheses in the order shown
by the \texttt{show statement ... / full} command, i.e., the RPN order\index{RPN
order} of the mandatory
hypotheses.  Integers $m+1$ through $m+n$ correspond to the labels enclosed in
the parentheses of the compressed proof, in the order that they appear, where
$n$ is the number of those labels.  Integers $m+n+1$ on up don't directly
correspond to statement labels but point to proof steps identified with the
letter \texttt{Z}, so that these proof steps can be referenced later in the
proof.  Integer $m+n+1$ corresponds to the first step tagged with a \texttt{Z},
$m+n+2$ to the second step tagged with a \texttt{Z}, etc.  When the compressed
proof is converted to a normal proof, the entire subproof of a step tagged
with \texttt{Z} replaces the reference to that step.

For efficiency, Metamath works with compressed proofs directly, without
converting them internally to normal proofs.  In addition to the usual
error-checking, an error message is given if (1) a label in the label list in
parentheses does not refer to a previous \texttt{\$p} or \texttt{\$a} statement or a
non-mandatory hypothesis of the statement being proved and (2) a proof step
tagged with \texttt{Z} is referenced before the step tagged with the \texttt{Z}.

Just as in a normal proof under development (Section~\ref{unknown}), any step
or subproof that is not yet known may be represented with a single \texttt{?}.
White space does not have to appear between the \texttt{?}\ and the upper-case
letters (or other \texttt{?}'s) representing the remainder of the proof.

% April 1, 2004 Appendix C has been added back in with corrections.
%
% May 20, 2003 Appendix C was removed for now because there was a problem found
% by Bob Solovay
%
% Also, removed earlier \ref{formalspec} 's (3 cases above)
%
% Bob Solovay wrote on 30 Nov 2002:
%%%%%%%%%%%%% (start of email comment )
%      3. My next set of comments concern appendix C. I read this before I
% read Chapter 4. So I first noted that the system as presented in the
% Appendix lacked a certain formal property that I thought desirable. I
% then came up with a revised formal system that had this property. Upon
% reading Chapter 4, I noticed that the revised system was closer to the
% treatment in Chapter 4 than the system in Appendix C.
%
%         First a very minor correction:
%
%         On page 142 line 2: The condition that V(e) != V(f) should only be
% required of e, f in T such that e != f.
%
%         Here is a natural property [transitivity] that one would like
% the formal system to have:
%
%         Let Gamma be a set of statements. Suppose that the statement Phi
% is provable from Gamma and that the statement Psi is provable from Gamma
% \cup {Phi}. Then Psi is provable from Gamma.
%
%         I shall present an example to show that this property does not
% hold for the formal systems of Appendix C:
%
%         I write the example in metamath style:
%
% $c A B C D E $.
% $v x y
%
% ${
% tx $f A x $.
% ty $f B y $.
% ax1 $a C x y $.
% $}
%
% ${
% tx $f A x $.
% ty $f B y $.
% ax2-h1 $e C x y $.
% ax2 $a D y $.
% $}
%
% ${
% ty $f B y $.
% ax3-h1 $e D y $.
% ax3 $a E y $.
% $}
%
% $(These three axioms are Gamma $)
%
% ${
% tx $f A x $.
% ty $f B y $.
% Phi $p D y $=
% tx ty tx ty ax1 ax2 $.
% $}
%
% ${
% ty $f B y $.
% Psi $p E y $=
% ty ty Phi ax3 $.
% $}
%
%
% I omit the formal proofs of the following claims. [I will be glad to
% supply them upon request.]
%
% 1) Psi is not provable from Gamma;
%
% 2) Psi is provable from Gamma + Phi.
%
% Here "provable" refers to the formalism of Appendix C.
%
% The trouble of course is that Psi is lacking the variable declaration
%
% $f Ax $.
%
% In the Metamath system there is no trouble proving Psi. I attach a
% metamath file that shows this and which has been checked by the
% metamath program.
%
% I next want to indicate how I think the treatment in Appendix C should
% be revised so as to conform more closely to the metamath system of the
% main text. The revised system *does* have the transitivity property.
%
% We want to give revised definitions of "statement" and
% "provable". [cf. sections C.2.4. and C.2.5] Our new definitions will
% use the definitions given in Appendix C. So we take the following
% tack. We refer to the original notions as o-statement and o-provable. And
% we refer to the notions we are defining as n-statement and n-provable.
%
%         A n-statement is an o-statement in which the only variables
% that appear in the T component are mandatory.
%
%         To any o-statement we can associate its reduct which is a
% n-statement by dropping all the elements of T or D which contain
% non-mandatory variables.
%
%         An n-statement gamma is n-provable if there is an o-statement
% gamma' which has gamma as its reduct andf such that gamma' is
% o-provable.
%
%         It seems to me [though I am not completely sure on this point]
% that n-provability corresponds to metamath provability as discussed
% say in Chapter 4.
%
%         Attached to this letter is the metamath proof of Phi and Psi
% from Gamma discussed above.
%
%         I am still brooding over the question of whether metamath
% correctly formalizes set-theory. No doubt I will have some questions
% re this after my thoughts become clearer.
%%%%%%%%%%%%%%%% (end of email comment)

%%%%%%%%%%%%%%%% (start of 2nd email comment from Bob Solovay 1-Apr-04)
%
%         I hope that Appendix C is the one that gives a "formal" treatment
% of Metamath. At any rate, thats the appendix I want to comment on.
%
%         I'm going to suggest two changes in the definition.
%
%         First change (in the definition of statement): Require that the
% sets D, T, and E be finite.
%
%         Probably things are fine as you give them. But in the applications
% to the main metamath system they will always be finite, and its useful in
% thinking about things [at least for me] to stick to the finite case.
%
%         Second change:
%
%         First let me give an approximate description. Remove the dummy
% variables from the statement. Instead, include them in the proof.
%
%         More formally: Require that T consists of type declarations only
% for mandatory variables. Require that all the pairs in D consist of
% mandatory variables.
%
%         At the start of a proof we are allowed to declare a finite number
% of dummy variables [provided that none of them appear in any of the
% statements in E \cup {A}. We have to supply type declarations for all the
% dummy variables. We are allowed to add new $d statements referring to
% either the mandatory or dummy variables. But we require that no new $d
% statement references only mandatory variables.
%
%         I find this way of doing things more conceptual than the treatment
% in Appendix C. But the change [which I will use implicitly in later
% letters about doing Peano] is mainly aesthetic. I definitely claim that my
% results on doing Peano all apply to Metamath as it is presented in your
% book.
%
%         --Bob
%
%%%%%%%%%%%%%%%% (end of 2nd email comment)

%%
%% When uncommenting the below, also uncomment references above to {formalspec}
%%
\chapter{Metamath's Formal System}\label{formalspec}\index{Metamath!as a formal
system}

\section{Introduction}

\begin{quote}
  {\em Perfection is when there is no longer anything more to take away.}
    \flushright\sc Antoine de
     Saint-Exupery\footnote{\cite[p.~3-25]{Campbell}.}\\
\end{quote}\index{de Saint-Exupery, Antoine}

This appendix describes the theory behind the Metamath language in an abstract
way intended for mathematicians.  Specifically, we construct two
set-theo\-ret\-i\-cal objects:  a ``formal system'' (roughly, a set of syntax
rules, axioms, and logical rules) and its ``universe'' (roughly, the set of
theorems derivable in the formal system).  The Metamath computer language
provides us with a way to describe specific formal systems and, with the aid of
a proof provided by the user, to verify that given theorems
belong to their universes.

To understand this appendix, you need a basic knowledge of informal set theory.
It should be sufficient to understand, for example, Ch.\ 1 of Munkres' {\em
Topology} \cite{Munkres}\index{Munkres, James R.} or the
introductory set theory chapter
in many textbooks that introduce abstract mathematics. (Note that there are
minor notational differences among authors; e.g.\ Munkres uses $\subset$ instead
of our $\subseteq$ for ``subset.''  We use ``included in'' to mean ``a subset
of,'' and ``belongs to'' or ``is contained in'' to mean ``is an element of.'')
What we call a ``formal'' description here, unlike earlier, is actually an
informal description in the ordinary language of mathematicians.  However we
provide sufficient detail so that a mathematician could easily formalize it,
even in the language of Metamath itself if desired.  To understand the logic
examples at the end of this appendix, familiarity with an introductory book on
mathematical logic would be helpful.

\section{The Formal Description}

\subsection[Preliminaries]{Preliminaries\protect\footnotemark}%
\footnotetext{This section is taken mostly verbatim
from Tarski \cite[p.~63]{Tarski1965}\index{Tarski, Alfred}.}

By $\omega$ we denote the set of all natural numbers (non-negative integers).
Each natural number $n$ is identified with the set of all smaller numbers: $n =
\{ m | m < n \}$.  The formula $m < n$ is thus equivalent to the condition: $m
\in n$ and $m,n \in \omega$. In particular, 0 is the number zero and at the
same time the empty set $\varnothing$, $1=\{0\}$, $2=\{0,1\}$, etc. ${}^B A$
denotes the set of all functions on $B$ to $A$ (i.e.\ with domain $B$ and range
included in $A$).  The members of ${}^\omega A$ are what are called {\em simple
infinite sequences},\index{simple infinite sequence}
with all {\em terms}\index{term} in $A$.  In case $n \in \omega$, the
members of ${}^n A$ are referred to as {\em finite $n$-termed
sequences},\index{finite $n$-termed
sequence} again
with terms in $A$.  The consecutive terms (function values) of a finite or
infinite sequence $f$ are denoted by $f_0, f_1, \ldots ,f_n,\ldots$.  Every
finite sequence $f \in \bigcup _{n \in \omega} {}^n A$ uniquely determines the
number $n$ such that $f \in {}^n A$; $n$ is called the {\em
length}\index{length of a sequence ({$"|\ "|$})} of $f$ and
is denoted by $|f|$.  $\langle a \rangle$ is the sequence $f$ with $|f|=1$ and
$f_0=a$; $\langle a,b \rangle$ is the sequence $f$ with $|f|=2$, $f_0=a$,
$f_1=b$; etc.  Given two finite sequences $f$ and $g$, we denote by $f\frown g$
their {\em concatenation},\index{concatenation} i.e., the
finite sequence $h$ determined by the
conditions:
\begin{eqnarray*}
& |h| = |f|+|g|;&  \\
& h_n = f_n & \mbox{\ for\ } n < |f|;  \\
& h_{|f|+n} = g_n & \mbox{\ for\ } n < |g|.
\end{eqnarray*}

\subsection{Constants, Variables, and Expressions}

A formal system has a set of {\em symbols}\index{symbol!in
a formal system} denoted
by $\mbox{\em SM}$.  A
precise set-theo\-ret\-i\-cal definition of this set is unimportant; a symbol
could be considered a primitive or atomic element if we wish.  We assume this
set is divided into two disjoint subsets:  a set $\mbox{\em CN}$ of {\em
constants}\index{constant!in a formal system} and a set $\mbox{\em VR}$ of
{\em variables}.\index{variable!in a formal system}  $\mbox{\em CN}$ and
$\mbox{\em VR}$ are each assumed to consist of countably many symbols which
may be arranged in finite or simple infinite sequences $c_0, c_1, \ldots$ and
$v_0, v_1, \ldots$ respectively, without repeating terms.  We will represent
arbitrary symbols by metavariables $\alpha$, $\beta$, etc.

{\footnotesize\begin{quotation}
{\em Comment.} The variables $v_0, v_1, \ldots$ of our formal system
correspond to what are usually considered ``metavariables'' in
descriptions of specific formal systems in the literature.  Typically,
when describing a specific formal system a book will postulate a set of
primitive objects called variables, then proceed to describe their
properties using metavariables that range over them, never mentioning
again the actual variables themselves.  Our formal system does not
mention these primitive variable objects at all but deals directly with
metavariables, as its primitive objects, from the start.  This is a
subtle but key distinction you should keep in mind, and it makes our
definition of ``formal system'' somewhat different from that typically
found in the literature.  (So, the $\alpha$, $\beta$, etc.\ above are
actually ``metametavariables'' when used to represent $v_0, v_1,
\ldots$.)
\end{quotation}}

Finite sequences all terms of which are symbols are called {\em
expressions}.\index{expression!in a formal system}  $\mbox{\em EX}$ is
the set of all expressions; thus
\begin{displaymath}
\mbox{\em EX} = \bigcup _{n \in \omega} {}^n \mbox{\em SM}.
\end{displaymath}

A {\em constant-prefixed expression}\index{constant-prefixed expression}
is an expression of non-zero length
whose first term is a constant.  We denote the set of all constant-prefixed
expressions by $\mbox{\em EX}_C = \{ e \in \mbox{\em EX} | ( |e| > 0 \wedge
e_0 \in \mbox{\em CN} ) \}$.

A {\em constant-variable pair}\index{constant-variable pair}
is an expression of length 2 whose first term
is a constant and whose second term is a variable.  We denote the set of all
constant-variable pairs by $\mbox{\em EX}_2 = \{ e \in \mbox{\em EX}_C | ( |e|
= 2 \wedge e_1 \in \mbox{\em VR} ) \}$.


{\footnotesize\begin{quotation}
{\em Relationship to Metamath.} In general, the set $\mbox{\em SM}$
corresponds to the set of declared math symbols in a Metamath database, the
set $\mbox{\em CN}$ to those declared with \texttt{\$c} statements, and the set
$\mbox{\em VR}$ to those declared with \texttt{\$v} statements.  Of course a
Metamath database can only have a finite number of math symbols, whereas
formal systems in general can have an infinite number, although the number of
Metamath math symbols available is in principle unlimited.

The set $\mbox{\em EX}_C$ corresponds to the set of permissible expressions
for \texttt{\$e}, \texttt{\$a}, and \texttt{\$p} statements.  The set $\mbox{\em EX}_2$
corresponds to the set of permissible expressions for \texttt{\$f} statements.
\end{quotation}}

We denote by ${\cal V}(e)$ the set of all variables in an expression $e \in
\mbox{\em EX}$, i.e.\ the set of all $\alpha \in \mbox{\em VR}$ such that
$\alpha = e_n$ for some $n < |e|$.  We also denote (with abuse of notation) by
${\cal V}(E)$ the set of all variables in a collection of expressions $E
\subseteq \mbox{\em EX}$, i.e.\ $\bigcup _{e \in E} {\cal V}(e)$.


\subsection{Substitution}

Given a function $F$ from $\mbox{\em VR}$ to
$\mbox{\em EX}$, we
denote by $\sigma_{F}$ or just $\sigma$ the function from $\mbox{\em EX}$ to
$\mbox{\em EX}$ defined recursively for nonempty sequences by
\begin{eqnarray*}
& \sigma(<\alpha>) = F(\alpha) & \mbox{for\ } \alpha \in \mbox{\em VR}; \\
& \sigma(<\alpha>) = <\alpha> & \mbox{for\ } \alpha \not\in \mbox{\em VR}; \\
& \sigma(g \frown h) = \sigma(g) \frown
    \sigma(h) & \mbox{for\ } g,h \in \mbox{\em EX}.
\end{eqnarray*}
We also define $\sigma(\varnothing)=\varnothing$.  We call $\sigma$ a {\em
simultaneous substitution}\index{substitution!variable}\index{variable
substitution} (or just {\em substitution}) with {\em substitution
map}\index{substitution map} $F$.

We also denote (with abuse of notation) by $\sigma(E)$ a substitution on a
collection of expressions $E \subseteq \mbox{\em EX}$, i.e.\ the set $\{
\sigma(e) | e \in E \}$.  The collection $\sigma(E)$ may of course contain
fewer expressions than $E$ because duplicate expressions could result from the
substitution.

\subsection{Statements}

We denote by $\mbox{\em DV}$ the set of all
unordered pairs $\{\alpha, \beta \} \subseteq \mbox{\em VR}$ such that $\alpha
\neq \beta$.  $\mbox{\em DV}$ stands for ``distinct variables.''

A {\em pre-statement}\index{pre-statement!in a formal system} is a
quadruple $\langle D,T,H,A \rangle$ such that
$D\subseteq \mbox{\em DV}$, $T\subseteq \mbox{\em EX}_2$, $H\subseteq
\mbox{\em EX}_C$ and $H$ is finite,
$A\in \mbox{\em EX}_C$, ${\cal V}(H\cup\{A\}) \subseteq
{\cal V}(T)$, and $\forall e,f\in T {\ } {\cal V}(e) \neq {\cal V}(f)$ (or
equivalently, $e_1 \ne f_1$) whenever $e \neq f$. The terms of the quadruple are called {\em
distinct-variable restrictions},\index{disjoint-variable restriction!in a
formal system} {\em variable-type hypotheses},\index{variable-type
hypothesis!in a formal system} {\em logical hypotheses},\index{logical
hypothesis!in a formal system} and the {\em assertion}\index{assertion!in a
formal system} respectively.  We denote by $T_M$ ({\em mandatory variable-type
hypotheses}\index{mandatory variable-type hypothesis!in a formal system}) the
subset of $T$ such that ${\cal V}(T_M) ={\cal V}(H \cup \{A\})$.  We denote by
$D_M=\{\{\alpha,\beta\}\in D|\{\alpha,\beta\}\subseteq {\cal V}(T_M)\}$ the
{\em mandatory distinct-variable restrictions}\index{mandatory
disjoint-variable restriction!in a formal system} of the pre-statement.
The set
of {\em mandatory hypotheses}\index{mandatory hypothesis!in a formal system}
is $T_M\cup H$.  We call the quadruple $\langle D_M,T_M,H,A \rangle$
the {\em reduct}\index{reduct!in a formal system} of
the pre-statement $\langle D,T,H,A \rangle$.

A {\em statement} is the reduct of some pre-statement\index{statement!in a
formal system}.  A statement is therefore a special kind of pre-statement;
in particular, a statement is the reduct of itself.

{\footnotesize\begin{quotation}
{\em Comment.}  $T$ is a set of expressions, each of length 2, that associate
a set of constants (``variable types'') with a set of variables.  The
condition ${\cal V}(H\cup\{A\}) \subseteq {\cal V}(T) $
means that each variable occurring in a statement's logical
hypotheses or assertion must have an associated variable-type hypothesis or
``type declaration,'' in  analogy to a computer programming language, where a
variable must be declared to be say, a string or an integer.  The requirement
that $\forall e,f\in T \, e_1 \ne f_1$ for $e\neq f$
means that each variable must be
associated with a unique constant designating its variable type; e.g., a
variable might be a ``wff'' or a ``set'' but not both.

Distinct-variable restrictions are used to specify what variable substitutions
are permissible to make for the statement to remain valid.  For example, in
the theorem scheme of set theory $\lnot\forall x\,x=y$ we may not substitute
the same variable for both $x$ and $y$.  On the other hand, the theorem scheme
$x=y\to y=x$ does not require that $x$ and $y$ be distinct, so we do not
require a distinct-variable restriction, although having one
would cause no harm other than making the scheme less general.

A mandatory variable-type hypothesis is one whose variable exists in a logical
hypothesis or the assertion.  A provable pre-statement
(defined below) may require
non-mandatory variable-type hypotheses that effectively introduce ``dummy''
variables for use in its proof.  Any number of dummy variables might
be required by a specific proof; indeed, it has been shown by H.\
Andr\'{e}ka\index{Andr{\'{e}}ka, H.} \cite{Nemeti} that there is no finite
upper bound to the number of dummy variables needed to prove an arbitrary
theorem in first-order logic (with equality) having a fixed number $n>2$ of
individual variables.  (See also the Comment on p.~\pageref{nodd}.)
For this reason we do not set a finite size bound on the collections $D$ and
$T$, although in an actual application (Metamath database) these will of
course be finite, increased to whatever size is necessary as more
proofs are added.
\end{quotation}}

{\footnotesize\begin{quotation}
{\em Relationship to Metamath.} A pre-statement of a formal system
corresponds to an extended frame in a Metamath database
(Section~\ref{frames}).  The collections $D$, $T$, and $H$ correspond
respectively to the \texttt{\$d}, \texttt{\$f}, and \texttt{\$e}
statement collections in an extended frame.  The expression $A$
corresponds to the \texttt{\$a} (or \texttt{\$p}) statement in an
extended frame.

A statement of a formal system corresponds to a frame in a Metamath
database.
\end{quotation}}

\subsection{Formal Systems}

A {\em formal system}\index{formal system} is a
triple $\langle \mbox{\em CN},\mbox{\em
VR},\Gamma\rangle$ where $\Gamma$ is a set of statements.  The members of
$\Gamma$ are called {\em axiomatic statements}.\index{axiomatic
statement!in a formal system}  Sometimes we will refer to a
formal system by just $\Gamma$ when $\mbox{\em CN}$ and $\mbox{\em VR}$ are
understood.

Given a formal system $\Gamma$, the {\em closure}\index{closure}\footnote{This
definition of closure incorporates a simplification due to
Josh Purinton.\index{Purinton, Josh}.} of a
pre-statement
$\langle D,T,H,A \rangle$ is the smallest set $C$ of expressions
such that:
%\begin{enumerate}
%  \item $T\cup H\subseteq C$; and
%  \item If for some axiomatic statement
%    $\langle D_M',T_M',H',A' \rangle \in \Gamma_A$, for
%    some $E \subseteq C$, some $F \subseteq C-T$ (where ``-'' denotes
%    set difference), and some substitution
%    $\sigma$ we have
%    \begin{enumerate}
%       \item $\sigma(T_M') = E$ (where, as above, the $M$ denotes the
%           mandatory variable-type hypotheses of $T^A$);
%       \item $\sigma(H') = F$;
%       \item for all $\{\alpha,\beta\}\in D^A$ and $\subseteq
%         {\cal V}(T_M')$, for all $\gamma\in {\cal V}(\sigma(\langle \alpha
%         \rangle))$, and for all $\delta\in  {\cal V}(\sigma(\langle \beta
%         \rangle))$, we have $\{\gamma, \delta\} \in D$;
%   \end{enumerate}
%   then $\sigma(A') \in C$.
%\end{enumerate}
\begin{list}{}{\itemsep 0.0pt}
  \item[1.] $T\cup H\subseteq C$; and
  \item[2.] If for some axiomatic statement
    $\langle D_M',T_M',H',A' \rangle \in
       \Gamma$ and for some substitution
    $\sigma$ we have
    \begin{enumerate}
       \item[a.] $\sigma(T_M' \cup H') \subseteq C$; and
       \item[b.] for all $\{\alpha,\beta\}\in D_M'$, for all $\gamma\in
         {\cal V}(\sigma(\langle \alpha
         \rangle))$, and for all $\delta\in  {\cal V}(\sigma(\langle \beta
         \rangle))$, we have $\{\gamma, \delta\} \in D$;
   \end{enumerate}
   then $\sigma(A') \in C$.
\end{list}
A pre-statement $\langle D,T,H,A
\rangle$ is {\em provable}\index{provable statement!in a formal
system} if $A\in C$ i.e.\ if its assertion belongs to its
closure.  A statement is {\em provable} if it is
the reduct of a provable pre-statement.
The {\em universe}\index{universe of a formal system}
of a formal system is
the collection of all of its provable statements.  Note that the
set of axiomatic statements $\Gamma$ in a formal system is a subset of its
universe.

{\footnotesize\begin{quotation}
{\em Comment.} The first condition in the definition of closure simply says
that the hypotheses of the pre-statement are in its closure.

Condition 2(a) says that a substitution exists that makes the
mandatory hypotheses of an axiomatic statement exactly match some members of
the closure.  This is what we explicitly demonstrate in a Metamath language
proof.

%Conditions 2(a) and 2(b) say that a substitution exists that makes the
%(mandatory) hypotheses of an axiomatic statement exactly match some members of
%the closure.  This is what we explicitly demonstrate with a Metamath language
%proof.
%
%The set of expressions $F$ in condition 2(b) excludes the variable-type
%hypotheses; this is done because non-mandatory variable-type hypotheses are
%effectively ``dropped'' as irrelevant whereas logical hypotheses must be
%retained to achieve a consistent logical system.

Condition 2(b) describes how distinct-variable restrictions in the axiomatic
statement must be met.  It means that after a substitution for two variables
that must be distinct, the resulting two expressions must either contain no
variables, or if they do, they may not have variables in common, and each pair
of any variables they do have, with one variable from each expression, must be
specified as distinct in the original statement.
\end{quotation}}

{\footnotesize\begin{quotation}
{\em Relationship to Metamath.} Axiomatic statements
 and provable statements in a formal
system correspond to the frames for \texttt{\$a} and \texttt{\$p} statements
respectively in a Metamath database.  The set of axiomatic statements is a
subset of the set of provable statements in a formal system, although in a
Metamath database a \texttt{\$a} statement is distinguished by not having a
proof.  A Metamath language proof for a \texttt{\$p} statement tells the computer
how to explicitly construct a series of members of the closure ultimately
leading to a demonstration that the assertion
being proved is in the closure.  The actual closure typically contains
an infinite number of expressions.  A formal system itself does not have
an explicit object called a ``proof'' but rather the existence of a proof
is implied indirectly by membership of an assertion in a provable
statement's closure.  We do this to make the formal system easier
to describe in the language of set theory.

We also note that once established as provable, a statement may be considered
to acquire the same status as an axiomatic statement, because if the set of
axiomatic statements is extended with a provable statement, the universe of
the formal system remains unchanged (provided that $\mbox{\em VR}$ is
infinite).
In practice, this means we can build a hierarchy of provable statements to
more efficiently establish additional provable statements.  This is
what we do in Metamath when we allow proofs to reference previous
\texttt{\$p} statements as well as previous \texttt{\$a} statements.
\end{quotation}}

\section{Examples of Formal Systems}

{\footnotesize\begin{quotation}
{\em Relationship to Metamath.} The examples in this section, except Example~2,
are for the most part exact equivalents of the development in the set
theory database \texttt{set.mm}.  You may want to compare Examples~1, 3, and 5
to Section~\ref{metaaxioms}, Example 4 to Sections~\ref{metadefprop} and
\ref{metadefpred}, and Example 6 to
Section~\ref{setdefinitions}.\label{exampleref}
\end{quotation}}

\subsection{Example~1---Propositional Calculus}\index{propositional calculus}

Classical propositional calculus can be described by the following formal
system.  We assume the set of variables is infinite.  Rather than denoting the
constants and variables by $c_0, c_1, \ldots$ and $v_0, v_1, \ldots$, for
readability we will instead use more conventional symbols, with the
understanding of course that they denote distinct primitive objects.
Also for readability we may omit commas between successive terms of a
sequence; thus $\langle \mbox{wff\ } \varphi\rangle$ denotes
$\langle \mbox{wff}, \varphi\rangle$.

Let
\begin{itemize}
  \item[] $\mbox{\em CN}=\{\mbox{wff}, \vdash, \to, \lnot, (,)\}$
  \item[] $\mbox{\em VR}=\{\varphi,\psi,\chi,\ldots\}$
  \item[] $T = \{\langle \mbox{wff\ } \varphi\rangle,
             \langle \mbox{wff\ } \psi\rangle,
             \langle \mbox{wff\ } \chi\rangle,\ldots\}$, i.e.\ those
             expressions of length 2 whose first member is $\mbox{\rm wff}$
             and whose second member belongs to $\mbox{\em VR}$.\footnote{For
convenience we let $T$ be an infinite set; the definition of a statement
permits this in principle.  Since a Metamath source file has a finite size, in
practice we must of course use appropriate finite subsets of this $T$,
specifically ones containing at least the mandatory variable-type
hypotheses.  Similarly, in the source file we introduce new variables as
required, with the understanding that a potentially infinite number of
them are available.}
\noindent Then $\Gamma$ consists of the axiomatic statements that
are the reducts of the following pre-statements:
    \begin{itemize}
      \item[] $\langle\varnothing,T,\varnothing,
               \langle \mbox{wff\ }(\varphi\to\psi)\rangle\rangle$
      \item[] $\langle\varnothing,T,\varnothing,
               \langle \mbox{wff\ }\lnot\varphi\rangle\rangle$
      \item[] $\langle\varnothing,T,\varnothing,
               \langle \vdash(\varphi\to(\psi\to\varphi))
               \rangle\rangle$
      \item[] $\langle\varnothing,T,
               \varnothing,
               \langle \vdash((\varphi\to(\psi\to\chi))\to
               ((\varphi\to\psi)\to(\varphi\to\chi)))
               \rangle\rangle$
      \item[] $\langle\varnothing,T,
               \varnothing,
               \langle \vdash((\lnot\varphi\to\lnot\psi)\to
               (\psi\to\varphi))\rangle\rangle$
      \item[] $\langle\varnothing,T,
               \{\langle\vdash(\varphi\to\psi)\rangle,
                 \langle\vdash\varphi\rangle\},
               \langle\vdash\psi\rangle\rangle$
    \end{itemize}
\end{itemize}

(For example, the reduct of $\langle\varnothing,T,\varnothing,
               \langle \mbox{wff\ }(\varphi\to\psi)\rangle\rangle$
is
\begin{itemize}
\item[] $\langle\varnothing,
\{\langle \mbox{wff\ } \varphi\rangle,
             \langle \mbox{wff\ } \psi\rangle\},
             \varnothing,
               \langle \mbox{wff\ }(\varphi\to\psi)\rangle\rangle$,
\end{itemize}
which is the first axiomatic statement.)

We call the members of $\mbox{\em VR}$ {\em wff variables} or (in the context
of first-order logic which we will describe shortly) {\em wff metavariables}.
Note that the symbols $\phi$, $\psi$, etc.\ denote actual specific members of
$\mbox{\em VR}$; they are not metavariables of our expository language (which
we denote with $\alpha$, $\beta$, etc.) but are instead (meta)constant symbols
(members of $\mbox{\em SM}$) from the point of view of our expository
language.  The equivalent system of propositional calculus described in
\cite{Tarski1965} also uses the symbols $\phi$, $\psi$, etc.\ to denote wff
metavariables, but in \cite{Tarski1965} unlike here those are metavariables of
the expository language and not primitive symbols of the formal system.

The first two statements define wffs: if $\varphi$ and $\psi$ are wffs, so is
$(\varphi \to \psi)$; if $\varphi$ is a wff, so is $\lnot\varphi$. The next
three are the axioms of propositional calculus: if $\varphi$ and $\psi$ are
wffs, then $\vdash (\varphi \to (\psi \to \varphi))$ is an (axiomatic)
theorem; etc. The
last is the rule of modus ponens: if $\varphi$ and $\psi$ are wffs, and
$\vdash (\varphi\to\psi)$ and $\vdash \varphi$ are theorems, then $\vdash
\psi$ is a theorem.

The correspondence to ordinary propositional calculus is as follows.  We
consider only provable statements of the form $\langle\varnothing,
T,\varnothing,A\rangle$ with $T$ defined as above.  The first term of the
assertion $A$ of any such statement is either ``wff'' or ``$\vdash$''.  A
statement for which the first term is ``wff'' is a {\em wff} of propositional
calculus, and one where the first term is ``$\vdash$'' is a {\em
theorem (scheme)} of propositional calculus.

The universe of this formal system also contains many other provable
statements.  Those with distinct-variable restrictions are irrelevant because
propositional calculus has no constraints on substitutions.  Those that have
logical hypotheses we call {\em inferences}\index{inference} when
the logical hypotheses are of the form
$\langle\vdash\rangle\frown w$ where $w$ is a wff (with the leading constant
term ``wff'' removed).  Inferences (other than the modus ponens rule) are not a
proper part of propositional calculus but are convenient to use when building a
hierarchy of provable statements.  A provable statement with a nonsense
hypothesis such as $\langle \to,\vdash,\lnot\rangle$, and this same expression
as its assertion, we consider irrelevant; no use can be made of it in
proving theorems, since there is no way to eliminate the nonsense hypothesis.

{\footnotesize\begin{quotation}
{\em Comment.} Our use of parentheses in the definition of a wff illustrates
how axiomatic statements should be carefully stated in a way that
ties in unambiguously with the substitutions allowed by the formal system.
There are many ways we could have defined wffs---for example, Polish
prefix notation would have allowed us to omit parentheses entirely, at
the expense of readability---but we must define them in a way that is
unambiguous.  For example, if we had omitted parentheses from the
definition of $(\varphi\to \psi)$, the wff $\lnot\varphi\to \psi$ could
be interpreted as either $\lnot(\varphi\to\psi)$ or $(\lnot\varphi\to\psi)$
and would have allowed us to prove nonsense.  Note that there is no
concept of operator binding precedence built into our formal system.
\end{quotation}}

\begin{sloppy}
\subsection{Example~2---Predicate Calculus with Equality}\index{predicate
calculus}
\end{sloppy}

Here we extend Example~1 to include predicate calculus with equality,
illustrating the use of distinct-variable restrictions.  This system is the
same as Tarski's system $\mathfrak{S}_2$ in \cite{Tarski1965} (except that the
axioms of propositional calculus are different but equivalent, and a redundant
axiom is omitted).  We extend $\mbox{\em CN}$ with the constants
$\{\mbox{var},\forall,=\}$.  We extend $\mbox{\em VR}$ with an infinite set of
{\em individual metavariables}\index{individual
metavariable} $\{x,y,z,\ldots\}$ and denote this subset
$\mbox{\em Vr}$.

We also join to $\mbox{\em CN}$ a possibly infinite set $\mbox{\em Pr}$ of {\em
predicates} $\{R,S,\ldots\}$.  We associate with $\mbox{\em Pr}$ a function
$\mbox{rnk}$ from $\mbox{\em Pr}$ to $\omega$, and for $\alpha\in \mbox{\em
Pr}$ we call $\mbox{rnk}(\alpha)$ the {\em rank} of the predicate $\alpha$,
which is simply the number of ``arguments'' that the predicate has.  (Most
applications of predicate calculus will have a finite number of predicates;
for example, set theory has the single two-argument or binary predicate $\in$,
which is usually written with its arguments surrounding the predicate symbol
rather than with the prefix notation we will use for the general case.)  As a
device to facilitate our discussion, we will let $\mbox{\em Vs}$ be any fixed
one-to-one function from $\omega$ to $\mbox{\em Vr}$; thus $\mbox{\em Vs}$ is
any simple infinite sequence of individual metavariables with no repeating
terms.

In this example we will not include the function symbols that are often part of
formalizations of predicate calculus.  Using metalogical arguments that are
beyond the scope of our discussion, it can be shown that our formalization is
equivalent when functions are introduced via appropriate definitions.

We extend the set $T$ defined in Example~1 with the expressions
$\{\langle \mbox{var\ } x\rangle,$ $ \langle \mbox{var\ } y\rangle, \langle
\mbox{var\ } z\rangle,\ldots\}$.  We extend the $\Gamma$ above
with the axiomatic statements that are the reducts of the following
pre-statements:
\begin{list}{}{\itemsep 0.0pt}
      \item[] $\langle\varnothing,T,\varnothing,
               \langle \mbox{wff\ }\forall x\,\varphi\rangle\rangle$
      \item[] $\langle\varnothing,T,\varnothing,
               \langle \mbox{wff\ }x=y\rangle\rangle$
      \item[] $\langle\varnothing,T,
               \{\langle\vdash\varphi\rangle\},
               \langle\vdash\forall x\,\varphi\rangle\rangle$
      \item[] $\langle\varnothing,T,\varnothing,
               \langle \vdash((\forall x(\varphi\to\psi)
                  \to(\forall x\,\varphi\to\forall x\,\psi))
               \rangle\rangle$
      \item[] $\langle\{\{x,\varphi\}\},T,\varnothing,
               \langle \vdash(\varphi\to\forall x\,\varphi)
               \rangle\rangle$
      \item[] $\langle\{\{x,y\}\},T,\varnothing,
               \langle \vdash\lnot\forall x\lnot x=y
               \rangle\rangle$
      \item[] $\langle\varnothing,T,\varnothing,
               \langle \vdash(x=z
                  \to(x=y\to z=y))
               \rangle\rangle$
      \item[] $\langle\varnothing,T,\varnothing,
               \langle \vdash(y=z
                  \to(x=y\to x=z))
               \rangle\rangle$
\end{list}
These are the axioms not involving predicate symbols. The first two statements
extend the definition of a wff.  The third is the rule of generalization.  The
fifth states, in effect, ``For a wff $\varphi$ and variable $x$,
$\vdash(\varphi\to\forall x\,\varphi)$, provided that $x$ does not occur in
$\varphi$.''  The sixth states ``For variables $x$ and $y$,
$\vdash\lnot\forall x\lnot x = y$, provided that $x$ and $y$ are distinct.''
(This proviso is not necessary but was included by Tarski to
weaken the axiom and still show that the system is logically complete.)

Finally, for each predicate symbol $\alpha\in \mbox{\em Pr}$, we add to
$\Gamma$ an axiomatic statement, extending the definition of wff,
that is the reduct of the following pre-statement:
\begin{displaymath}
    \langle\varnothing,T,\varnothing,
            \langle \mbox{wff},\alpha\rangle\
            \frown \mbox{\em Vs}\restriction\mbox{rnk}(\alpha)\rangle
\end{displaymath}
and for each $\alpha\in \mbox{\em Pr}$ and each $n < \mbox{rnk}(\alpha)$
we add to $\Gamma$ an equality axiom that is the reduct of the
following pre-statement:
\begin{eqnarray*}
    \lefteqn{\langle\varnothing,T,\varnothing,
            \langle
      \vdash,(,\mbox{\em Vs}_n,=,\mbox{\em Vs}_{\mbox{rnk}(\alpha)},\to,
            (,\alpha\rangle\frown \mbox{\em Vs}\restriction\mbox{rnk}(\alpha)} \\
  & & \frown
            \langle\to,\alpha\rangle\frown \mbox{\em Vs}\restriction n\frown
            \langle \mbox{\em Vs}_{\mbox{rnk}(\alpha)}\rangle \\
 & & \frown
            \mbox{\em Vs}\restriction(\mbox{rnk}(\alpha)\setminus(n+1))\frown
            \langle),)\rangle\rangle
\end{eqnarray*}
where $\restriction$ denotes function domain restriction and $\setminus$
denotes set difference.  Recall that a subscript on $\mbox{\em Vs}$
denotes one of its terms.  (In the above two axiom sets commas are placed
between successive terms of sequences to prevent ambiguity, and if you examine
them with care you will be able to distinguish those parentheses that denote
constant symbols from those of our expository language that delimit function
arguments.  Although it might have been better to use boldface for our
primitive symbols, unfortunately boldface was not available for all characters
on the \LaTeX\ system used to typeset this text.)  These seemingly forbidding
axioms can be understood by analogy to concatenation of substrings in a
computer language.  They are actually relatively simple for each specific case
and will become clearer by looking at the special case of a binary predicate
$\alpha = R$ where $\mbox{rnk}(R)=2$.  Letting $\mbox{\em Vs}$ be the sequence
$\langle x,y,z,\ldots\rangle$, the axioms we would add to $\Gamma$ for this
case would be the wff extension and two equality axioms that are the
reducts of the pre-statements:
\begin{list}{}{\itemsep 0.0pt}
      \item[] $\langle\varnothing,T,\varnothing,
               \langle \mbox{wff\ }R x y\rangle\rangle$
      \item[] $\langle\varnothing,T,\varnothing,
               \langle \vdash(x=z
                  \to(R x y \to R z y))
               \rangle\rangle$
      \item[] $\langle\varnothing,T,\varnothing,
               \langle \vdash(y=z
                  \to(R x y \to R x z))
               \rangle\rangle$
\end{list}
Study these carefully to see how the general axioms above evaluate to
them.  In practice, typically only a few special cases such as this would be
needed, and in any case the Metamath language will only permit us to describe
a finite number of predicates, as opposed to the infinite number permitted by
the formal system.  (If an infinite number should be needed for some reason,
we could not define the formal system directly in the Metamath language but
could instead define it metalogically under set theory as we
do in this appendix, and only the underlying set theory, with its single
binary predicate, would be defined directly in the Metamath language.)


{\footnotesize\begin{quotation}
{\em Comment.}  As we noted earlier, the specific variables denoted by the
symbols $x,y,z,\ldots\in \mbox{\em Vr}\subseteq \mbox{\em VR}\subseteq
\mbox{\em SM}$ in Example~2 are not the actual variables of ordinary predicate
calculus but should be thought of as metavariables ranging over them.  For
example, a distinct-variable restriction would be meaningless for actual
variables of ordinary predicate calculus since two different actual variables
are by definition distinct.  And when we talk about an arbitrary
representative $\alpha\in \mbox{\em Vr}$, $\alpha$ is a metavariable (in our
expository language) that ranges over metavariables (which are primitives of
our formal system) each of which ranges over the actual individual variables
of predicate calculus (which are never mentioned in our formal system).

The constant called ``var'' above is called \texttt{setvar} in the
\texttt{set.mm} database file, but it means the same thing.  I felt
that ``var'' is a more meaningful name in the context of predicate
calculus, whose use is not limited to set theory.  For consistency we
stick with the name ``var'' throughout this Appendix, even after set
theory is introduced.
\end{quotation}}

\subsection{Free Variables and Proper Substitution}\index{free variable}
\index{proper substitution}\index{substitution!proper}

Typical representations of mathematical axioms use concepts such
as ``free variable,'' ``bound variable,'' and ``proper substitution''
as primitive notions.
A free variable is a variable that
is not a parameter of any container expression.
A bound variable is the opposite of a free variable; it is a
a variable that has been bound in a container expression.
For example, in the expression $\forall x \varphi$ (for all $x$, $\varphi$
is true), the variable $x$
is bound within the for-all ($\forall$) expression.
It is possible to change one variable to another, and that process is called
``proper substitution.''
In most books, proper substitution has a somewhat complicated recursive
definition with multiple cases based on the occurrences of free and
bound variables.
You may consult
\cite[ch.\ 3--4]{Hamilton}\index{Hamilton, Alan G.} (as well as
many other texts) for more formal details about these terms.

Using these concepts as \texttt{primitives} creates complications
for computer implementations.

In the system of Example~2, there are no primitive notions of free variable
and proper substitution.  Tarski \cite{Tarski1965} shows that this system is
logically equivalent to the more typical textbook systems that do have these
primitive notions, if we introduce these notions with appropriate definitions
and metalogic.  We could also define axioms for such systems directly,
although the recursive definitions of free variable and proper substitution
would be messy and awkward to work with.  Instead, we mention two devices that
can be used in practice to mimic these notions.  (1) Instead of introducing
special notation to express (as a logical hypothesis) ``where $x$ is not free
in $\varphi$'' we can use the logical hypothesis $\vdash(\varphi\to\forall
x\,\varphi)$.\label{effectivelybound}\index{effectively
not free}\footnote{This is a slightly weaker requirement than ``where $x$ is
not free in $\varphi$.''  If we let $\varphi$ be $x=x$, we have the theorem
$(x=x\to\forall x\,x=x)$ which satisfies the hypothesis, even though $x$ is
free in $x=x$ .  In a case like this we say that $x$ is {\em effectively not
free}\index{effectively not free} in $x=x$, since $x=x$ is logically
equivalent to $\forall x\,x=x$ in which $x$ is bound.} (2) It can be shown
that the wff $((x=y\to\varphi)\wedge\exists x(x=y\wedge\varphi))$ (with the
usual definitions of $\wedge$ and $\exists$; see Example~4 below) is logically
equivalent to ``the wff that results from proper substitution of $y$ for $x$
in $\varphi$.''  This works whether or not $x$ and $y$ are distinct.

\subsection{Metalogical Completeness}\index{metalogical completeness}

In the system of Example~2, the
following are provable pre-statements (and their reducts are
provable statements):
\begin{eqnarray*}
      & \langle\{\{x,y\}\},T,\varnothing,
               \langle \vdash\lnot\forall x\lnot x=y
               \rangle\rangle & \\
     &  \langle\varnothing,T,\varnothing,
               \langle \vdash\lnot\forall x\lnot x=x
               \rangle\rangle &
\end{eqnarray*}
whereas the following pre-statement is not to my knowledge provable (but
in any case we will pretend it's not for sake of illustration):
\begin{eqnarray*}
     &  \langle\varnothing,T,\varnothing,
               \langle \vdash\lnot\forall x\lnot x=y
               \rangle\rangle &
\end{eqnarray*}
In other words, we can prove ``$\lnot\forall x\lnot x=y$ where $x$ and $y$ are
distinct'' and separately prove ``$\lnot\forall x\lnot x=x$'', but we can't
prove the combined general case ``$\lnot\forall x\lnot x=y$'' that has no
proviso.  Now this does not compromise logical completeness, because the
variables are really metavariables and the two provable cases together cover
all possible cases.  The third case can be considered a metatheorem whose
direct proof, using the system of Example~2, lies outside the capability of the
formal system.

Also, in the system of Example~2 the following pre-statement is not to my
knowledge provable (again, a conjecture that we will pretend to be the case):
\begin{eqnarray*}
     & \langle\varnothing,T,\varnothing,
               \langle \vdash(\forall x\, \varphi\to\varphi)
               \rangle\rangle &
\end{eqnarray*}
Instead, we can only prove specific cases of $\varphi$ involving individual
metavariables, and by induction on formula length, prove as a metatheorem
outside of our formal system the general statement above.  The details of this
proof are found in \cite{Kalish}.

There does, however, exist a system of predicate calculus in which all such
``simple metatheorems'' as those above can be proved directly, and we present
it in Example~3. A {\em simple metatheorem}\index{simple metatheorem}
is any statement of the formal
system of Example~2 where all distinct variable restrictions consist of either
two individual metavariables or an individual metavariable and a wff
metavariable, and which is provable by combining cases outside the system as
above.  A system is {\em metalogically complete}\index{metalogical
completeness} if all of its simple
metatheorems are (directly) provable statements. The precise definition of
``simple metatheorem'' and the proof of the ``metalogical completeness'' of
Example~3 is found in Remark 9.6 and Theorem 9.7 of \cite{Megill}.\index{Megill,
Norman}

\begin{sloppy}
\subsection{Example~3---Metalogically Complete Predicate
Calculus with
Equality}
\end{sloppy}

For simplicity we will assume there is one binary predicate $R$;
this system suffices for set theory, where the $R$ is of course the $\in$
predicate.  We label the axioms as they appear in \cite{Megill}.  This
system is logically equivalent to that of Example~2 (when the latter is
restricted to this single binary predicate) but is also metalogically
complete.\index{metalogical completeness}

Let
\begin{itemize}
  \item[] $\mbox{\em CN}=\{\mbox{wff}, \mbox{var}, \vdash, \to, \lnot, (,),\forall,=,R\}$.
  \item[] $\mbox{\em VR}=\{\varphi,\psi,\chi,\ldots\}\cup\{x,y,z,\ldots\}$.
  \item[] $T = \{\langle \mbox{wff\ } \varphi\rangle,
             \langle \mbox{wff\ } \psi\rangle,
             \langle \mbox{wff\ } \chi\rangle,\ldots\}\cup
       \{\langle \mbox{var\ } x\rangle, \langle \mbox{var\ } y\rangle, \langle
       \mbox{var\ }z\rangle,\ldots\}$.

\noindent Then
  $\Gamma$ consists of the reducts of the following pre-statements:
    \begin{itemize}
      \item[] $\langle\varnothing,T,\varnothing,
               \langle \mbox{wff\ }(\varphi\to\psi)\rangle\rangle$
      \item[] $\langle\varnothing,T,\varnothing,
               \langle \mbox{wff\ }\lnot\varphi\rangle\rangle$
      \item[] $\langle\varnothing,T,\varnothing,
               \langle \mbox{wff\ }\forall x\,\varphi\rangle\rangle$
      \item[] $\langle\varnothing,T,\varnothing,
               \langle \mbox{wff\ }x=y\rangle\rangle$
      \item[] $\langle\varnothing,T,\varnothing,
               \langle \mbox{wff\ }Rxy\rangle\rangle$
      \item[(C1$'$)] $\langle\varnothing,T,\varnothing,
               \langle \vdash(\varphi\to(\psi\to\varphi))
               \rangle\rangle$
      \item[(C2$'$)] $\langle\varnothing,T,
               \varnothing,
               \langle \vdash((\varphi\to(\psi\to\chi))\to
               ((\varphi\to\psi)\to(\varphi\to\chi)))
               \rangle\rangle$
      \item[(C3$'$)] $\langle\varnothing,T,
               \varnothing,
               \langle \vdash((\lnot\varphi\to\lnot\psi)\to
               (\psi\to\varphi))\rangle\rangle$
      \item[(C4$'$)] $\langle\varnothing,T,
               \varnothing,
               \langle \vdash(\forall x(\forall x\,\varphi\to\psi)\to
                 (\forall x\,\varphi\to\forall x\,\psi))\rangle\rangle$
      \item[(C5$'$)] $\langle\varnothing,T,
               \varnothing,
               \langle \vdash(\forall x\,\varphi\to\varphi)\rangle\rangle$
      \item[(C6$'$)] $\langle\varnothing,T,
               \varnothing,
               \langle \vdash(\forall x\forall y\,\varphi\to
                 \forall y\forall x\,\varphi)\rangle\rangle$
      \item[(C7$'$)] $\langle\varnothing,T,
               \varnothing,
               \langle \vdash(\lnot\varphi\to\forall x\lnot\forall x\,\varphi
                 )\rangle\rangle$
      \item[(C8$'$)] $\langle\varnothing,T,
               \varnothing,
               \langle \vdash(x=y\to(x=z\to y=z))\rangle\rangle$
      \item[(C9$'$)] $\langle\varnothing,T,
               \varnothing,
               \langle \vdash(\lnot\forall x\, x=y\to(\lnot\forall x\, x=z\to
                 (y=z\to\forall x\, y=z)))\rangle\rangle$
      \item[(C10$'$)] $\langle\varnothing,T,
               \varnothing,
               \langle \vdash(\forall x(x=y\to\forall x\,\varphi)\to
                 \varphi))\rangle\rangle$
      \item[(C11$'$)] $\langle\varnothing,T,
               \varnothing,
               \langle \vdash(\forall x\, x=y\to(\forall x\,\varphi
               \to\forall y\,\varphi))\rangle\rangle$
      \item[(C12$'$)] $\langle\varnothing,T,
               \varnothing,
               \langle \vdash(x=y\to(Rxz\to Ryz))\rangle\rangle$
      \item[(C13$'$)] $\langle\varnothing,T,
               \varnothing,
               \langle \vdash(x=y\to(Rzx\to Rzy))\rangle\rangle$
      \item[(C15$'$)] $\langle\varnothing,T,
               \varnothing,
               \langle \vdash(\lnot\forall x\, x=y\to(x=y\to(\varphi
                 \to\forall x(x=y\to\varphi))))\rangle\rangle$
      \item[(C16$'$)] $\langle\{\{x,y\}\},T,
               \varnothing,
               \langle \vdash(\forall x\, x=y\to(\varphi\to\forall x\,\varphi)
                 )\rangle\rangle$
      \item[(C5)] $\langle\{\{x,\varphi\}\},T,\varnothing,
               \langle \vdash(\varphi\to\forall x\,\varphi)
               \rangle\rangle$
      \item[(MP)] $\langle\varnothing,T,
               \{\langle\vdash(\varphi\to\psi)\rangle,
                 \langle\vdash\varphi\rangle\},
               \langle\vdash\psi\rangle\rangle$
      \item[(Gen)] $\langle\varnothing,T,
               \{\langle\vdash\varphi\rangle\},
               \langle\vdash\forall x\,\varphi\rangle\rangle$
    \end{itemize}
\end{itemize}

While it is known that these axioms are ``metalogically complete,'' it is
not known whether they are independent (i.e.\ none is
redundant) in the metalogical sense; specifically, whether any axiom (possibly
with additional non-mandatory distinct-variable restrictions, for use with any
dummy variables in its proof) is provable from the others.  Note that
metalogical independence is a weaker requirement than independence in the
usual logical sense.  Not all of the above axioms are logically independent:
for example, C9$'$ can be proved as a metatheorem from the others, outside the
formal system, by combining the possible cases of distinct variables.

\subsection{Example~4---Adding Definitions}\index{definition}
There are several ways to add definitions to a formal system.  Probably the
most proper way is to consider definitions not as part of the formal system at
all but rather as abbreviations that are part of the expository metalogic
outside the formal system.  For convenience, though, we may use the formal
system itself to incorporate definitions, adding them as axiomatic extensions
to the system.  This could be done by adding a constant representing the
concept ``is defined as'' along with axioms for it. But there is a nicer way,
at least in this writer's opinion, that introduces definitions as direct
extensions to the language rather than as extralogical primitive notions.  We
introduce additional logical connectives and provide axioms for them.  For
systems of logic such as Examples 1 through 3, the additional axioms must be
conservative in the sense that no wff of the original system that was not a
theorem (when the initial term ``wff'' is replaced by ``$\vdash$'' of course)
becomes a theorem of the extended system.  In this example we extend Example~3
(or 2) with standard abbreviations of logic.

We extend $\mbox{\em CN}$ of Example~3 with new constants $\{\leftrightarrow,
\wedge,\vee,\exists\}$, corresponding to logical equivalence,\index{logical
equivalence ($\leftrightarrow$)}\index{biconditional ($\leftrightarrow$)}
conjunction,\index{conjunction ($\wedge$)} disjunction,\index{disjunction
($\vee$)} and the existential quantifier.\index{existential quantifier
($\exists$)}  We extend $\Gamma$ with the axiomatic statements that are
the reducts of the following pre-statements:
\begin{list}{}{\itemsep 0.0pt}
      \item[] $\langle\varnothing,T,\varnothing,
               \langle \mbox{wff\ }(\varphi\leftrightarrow\psi)\rangle\rangle$
      \item[] $\langle\varnothing,T,\varnothing,
               \langle \mbox{wff\ }(\varphi\vee\psi)\rangle\rangle$
      \item[] $\langle\varnothing,T,\varnothing,
               \langle \mbox{wff\ }(\varphi\wedge\psi)\rangle\rangle$
      \item[] $\langle\varnothing,T,\varnothing,
               \langle \mbox{wff\ }\exists x\, \varphi\rangle\rangle$
  \item[] $\langle\varnothing,T,\varnothing,
     \langle\vdash ( ( \varphi \leftrightarrow \psi ) \to
     ( \varphi \to \psi ) )\rangle\rangle$
  \item[] $\langle\varnothing,T,\varnothing,
     \langle\vdash ((\varphi\leftrightarrow\psi)\to
    (\psi\to\varphi))\rangle\rangle$
  \item[] $\langle\varnothing,T,\varnothing,
     \langle\vdash ((\varphi\to\psi)\to(
     (\psi\to\varphi)\to(\varphi
     \leftrightarrow\psi)))\rangle\rangle$
  \item[] $\langle\varnothing,T,\varnothing,
     \langle\vdash (( \varphi \wedge \psi ) \leftrightarrow\neg ( \varphi
     \to \neg \psi )) \rangle\rangle$
  \item[] $\langle\varnothing,T,\varnothing,
     \langle\vdash (( \varphi \vee \psi ) \leftrightarrow (\neg \varphi
     \to \psi )) \rangle\rangle$
  \item[] $\langle\varnothing,T,\varnothing,
     \langle\vdash (\exists x \,\varphi\leftrightarrow
     \lnot \forall x \lnot \varphi)\rangle\rangle$
\end{list}
The first three logical axioms (statements containing ``$\vdash$'') introduce
and effectively define logical equivalence, ``$\leftrightarrow$''.  The last
three use ``$\leftrightarrow$'' to effectively mean ``is defined as.''

\subsection{Example~5---ZFC Set Theory}\index{ZFC set theory}

Here we add to the system of Example~4 the axioms of Zermelo--Fraenkel set
theory with Choice.  For convenience we make use of the
definitions in Example~4.

In the $\mbox{\em CN}$ of Example~4 (which extends Example~3), we replace the symbol $R$
with the symbol $\in$.
More explicitly, we remove from $\Gamma$ of Example~4 the three
axiomatic statements containing $R$ and replace them with the
reducts of the following:
\begin{list}{}{\itemsep 0.0pt}
      \item[] $\langle\varnothing,T,\varnothing,
               \langle \mbox{wff\ }x\in y\rangle\rangle$
      \item[] $\langle\varnothing,T,
               \varnothing,
               \langle \vdash(x=y\to(x\in z\to y\in z))\rangle\rangle$
      \item[] $\langle\varnothing,T,
               \varnothing,
               \langle \vdash(x=y\to(z\in x\to z\in y))\rangle\rangle$
\end{list}
Letting $D=\{\{\alpha,\beta\}\in \mbox{\em DV}\,|\alpha,\beta\in \mbox{\em
Vr}\}$ (in other words all individual variables must be distinct), we extend
$\Gamma$ with the ZFC axioms, called
\index{Axiom of Extensionality}
\index{Axiom of Replacement}
\index{Axiom of Union}
\index{Axiom of Power Sets}
\index{Axiom of Regularity}
\index{Axiom of Infinity}
\index{Axiom of Choice}
Extensionality, Replacement, Union, Power
Set, Regularity, Infinity, and Choice, that are the reducts of:
\begin{list}{}{\itemsep 0.0pt}
      \item[Ext] $\langle D,T,
               \varnothing,
               \langle\vdash (\forall x(x\in y\leftrightarrow x \in z)\to y
               =z) \rangle\rangle$
      \item[Rep] $\langle D,T,
               \varnothing,
               \langle\vdash\exists x ( \exists y \forall z (\varphi \to z = y
                        ) \to
                        \forall z ( z \in x \leftrightarrow \exists x ( x \in
                        y \wedge \forall y\,\varphi ) ) )\rangle\rangle$
      \item[Un] $\langle D,T,
               \varnothing,
               \langle\vdash \exists x \forall y ( \exists x ( y \in x \wedge
               x \in z ) \to y \in x ) \rangle\rangle$
      \item[Pow] $\langle D,T,
               \varnothing,
               \langle\vdash \exists x \forall y ( \forall x ( x \in y \to x
               \in z ) \to y \in x ) \rangle\rangle$
      \item[Reg] $\langle D,T,
               \varnothing,
               \langle\vdash (  x \in y \to
                 \exists x ( x \in y \wedge \forall z ( z \in x \to \lnot z
                \in y ) ) ) \rangle\rangle$
      \item[Inf] $\langle D,T,
               \varnothing,
               \langle\vdash \exists x(y\in x\wedge\forall y(y\in
               x\to
               \exists z(y \in z\wedge z\in x))) \rangle\rangle$
      \item[AC] $\langle D,T,
               \varnothing,
               \langle\vdash \exists x \forall y \forall z ( ( y \in z
               \wedge z \in w ) \to \exists w \forall y ( \exists w
              ( ( y \in z \wedge z \in w ) \wedge ( y \in w \wedge w \in x
              ) ) \leftrightarrow y = w ) ) \rangle\rangle$
\end{list}

\subsection{Example~6---Class Notation in Set Theory}\label{class}

A powerful device that makes set theory easier (and that we have
been using all along in our informal expository language) is {\em class
abstraction notation}.\index{class abstraction}\index{abstraction class}  The
definitions we introduce are rigorously justified
as conservative by Takeuti and Zaring \cite{Takeuti}\index{Takeuti, Gaisi} or
Quine \cite{Quine}\index{Quine, Willard Van Orman}.  The key idea is to
introduce the notation $\{x|\mbox{---}\}$ which means ``the class of all $x$
such that ---'' for abstraction classes and introduce (meta)variables that
range over them.  An abstraction class may or may not be a set, depending on
whether it exists (as a set).  A class that does not exist is
called a {\em proper class}.\index{proper class}\index{class!proper}

To illustrate the use of abstraction classes we will provide some examples
of definitions that make use of them:  the empty set, class union, and
unordered pair.  Many other such definitions can be found in the
Metamath set theory database,
\texttt{set.mm}.\index{set theory database (\texttt{set.mm})}

% We intentionally break up the sequence of math symbols here
% because otherwise the overlong line goes beyond the page in narrow mode.
We extend $\mbox{\em CN}$ of Example~5 with new symbols $\{$
$\mbox{class},$ $\{,$ $|,$ $\},$ $\varnothing,$ $\cup,$ $,$ $\}$
where the inner braces and last comma are
constant symbols. (As before,
our dual use of some mathematical symbols for both our expository
language and as primitives of the formal system should be clear from context.)

We extend $\mbox{\em VR}$ of Example~5 with a set of {\em class
variables}\index{class variable}
$\{A,B,C,\ldots\}$. We extend the $T$ of Example~5 with $\{\langle
\mbox{class\ } A\rangle, \langle \mbox{class\ }B\rangle, \langle \mbox{class\ }
C\rangle,\ldots\}$.

To
introduce our definitions,
we add to $\Gamma$ of Example~5 the axiomatic statements
that are the reducts of the following pre-statements:
\begin{list}{}{\itemsep 0.0pt}
      \item[] $\langle\varnothing,T,\varnothing,
               \langle \mbox{class\ }x\rangle\rangle$
      \item[] $\langle\varnothing,T,\varnothing,
               \langle \mbox{class\ }\{x|\varphi\}\rangle\rangle$
      \item[] $\langle\varnothing,T,\varnothing,
               \langle \mbox{wff\ }A=B\rangle\rangle$
      \item[] $\langle\varnothing,T,\varnothing,
               \langle \mbox{wff\ }A\in B\rangle\rangle$
      \item[Ab] $\langle\varnothing,T,\varnothing,
               \langle \vdash ( y \in \{ x |\varphi\} \leftrightarrow
                  ( ( x = y \to\varphi) \wedge \exists x ( x = y
                  \wedge\varphi) ))
               \rangle\rangle$
      \item[Eq] $\langle\{\{x,A\},\{x,B\}\},T,\varnothing,
               \langle \vdash ( A = B \leftrightarrow
               \forall x ( x \in A \leftrightarrow x \in B ) )
               \rangle\rangle$
      \item[El] $\langle\{\{x,A\},\{x,B\}\},T,\varnothing,
               \langle \vdash ( A \in B \leftrightarrow \exists x
               ( x = A \wedge x \in B ) )
               \rangle\rangle$
\end{list}
Here we say that an individual variable is a class; $\{x|\varphi\}$ is a
class; and we extend the definition of a wff to include class equality and
membership.  Axiom Ab defines membership of a variable in a class abstraction;
the right-hand side can be read as ``the wff that results from proper
substitution of $y$ for $x$ in $\varphi$.''\footnote{Note that this definition
makes unnecessary the introduction of a separate notation similar to
$\varphi(x|y)$ for proper substitution, although we may choose to do so to be
conventional.  Incidentally, $\varphi(x|y)$ as it stands would be ambiguous in
the formal systems of our examples, since we wouldn't know whether
$\lnot\varphi(x|y)$ meant $\lnot(\varphi(x|y))$ or $(\lnot\varphi)(x|y)$.
Instead, we would have to use an unambiguous variant such as $(\varphi\,
x|y)$.}  Axioms Eq and El extend the meaning of the existing equality and
membership connectives.  This is potentially dangerous and requires careful
justification.  For example, from Eq we can derive the Axiom of Extensionality
with predicate logic alone; thus in principle we should include the Axiom of
Extensionality as a logical hypothesis.  However we do not bother to do this
since we have already presupposed that axiom earlier. The distinct variable
restrictions should be read ``where $x$ does not occur in $A$ or $B$.''  We
typically do this when the right-hand side of a definition involves an
individual variable not in the expression being defined; it is done so that
the right-hand side remains independent of the particular ``dummy'' variable
we use.

We continue to add to $\Gamma$ the following definitions
(i.e. the reducts of the following pre-statements) for empty
set,\index{empty set} class union,\index{union} and unordered
pair.\index{unordered pair}  They should be self-explanatory.  Analogous to our
use of ``$\leftrightarrow$'' to define new wffs in Example~4, we use ``$=$''
to define new abstraction terms, and both may be read informally as ``is
defined as'' in this context.
\begin{list}{}{\itemsep 0.0pt}
      \item[] $\langle\varnothing,T,\varnothing,
               \langle \mbox{class\ }\varnothing\rangle\rangle$
      \item[] $\langle\varnothing,T,\varnothing,
               \langle \vdash \varnothing = \{ x | \lnot x = x \}
               \rangle\rangle$
      \item[] $\langle\varnothing,T,\varnothing,
               \langle \mbox{class\ }(A\cup B)\rangle\rangle$
      \item[] $\langle\{\{x,A\},\{x,B\}\},T,\varnothing,
               \langle \vdash ( A \cup B ) = \{ x | ( x \in A \vee x \in B ) \}
               \rangle\rangle$
      \item[] $\langle\varnothing,T,\varnothing,
               \langle \mbox{class\ }\{A,B\}\rangle\rangle$
      \item[] $\langle\{\{x,A\},\{x,B\}\},T,\varnothing,
               \langle \vdash \{ A , B \} = \{ x | ( x = A \vee x = B ) \}
               \rangle\rangle$
\end{list}

\section{Metamath as a Formal System}\label{theorymm}

This section presupposes a familiarity with the Metamath computer language.

Our theory describes formal systems and their universes.  The Metamath
language provides a way of representing these set-theoretical objects to
a computer.  A Metamath database, being a finite set of {\sc ascii}
characters, can usually describe only a subset of a formal system and
its universe, which are typically infinite.  However the database can
contain as large a finite subset of the formal system and its universe
as we wish.  (Of course a Metamath set theory database can, in
principle, indirectly describe an entire infinite formal system by
formalizing the expository language in this Appendix.)

For purpose of our discussion, we assume the Metamath database
is in the simple form described on p.~\pageref{framelist},
consisting of all constant and variable declarations at the beginning,
followed by a sequence of extended frames each
delimited by \texttt{\$\char`\{} and \texttt{\$\char`\}}.  Any Metamath database can
be converted to this form, as described on p.~\pageref{frameconvert}.

The math symbol tokens of a Metamath source file, which are declared
with \texttt{\$c} and \texttt{\$v} statements, are names we assign to
representatives of $\mbox{\em CN}$ and $\mbox{\em VR}$.  For
definiteness we could assume that the first math symbol declared as a
variable corresponds to $v_0$, the second to $v_1$, etc., although the
exact correspondence we choose is not important.

In the Metamath language, each \texttt{\$d}, \texttt{\$f}, and
 \texttt{\$e} source
statement in an extended frame (Section~\ref{frames})
corresponds respectively to a member of the
collections $D$, $T$, and $H$ in a formal system statement $\langle
D_M,T_M,H,A\rangle$.  The math symbol strings following these Metamath keywords
correspond to a variable pair (in the case of \texttt{\$d}) or an expression (for
the other two keywords). The math symbol string following a \texttt{\$a} source
statement corresponds to expression $A$ in an axiomatic statement of the
formal system; the one following a \texttt{\$p} source statement corresponds to
$A$ in a provable statement that is not axiomatic.  In other words, each
extended frame in a Metamath database corresponds to
a pre-statement of the formal system, and a frame corresponds to
a statement of the formal system.  (Don't confuse the two meanings of
``statement'' here.  A statement of the formal system corresponds to the
several statements in a Metamath database that may constitute a
frame.)

In order for the computer to verify that a formal system statement is
provable, each \texttt{\$p} source statement is accompanied by a proof.
However, the proof does not correspond to anything in the formal system
but is simply a way of communicating to the computer the information
needed for its verification.  The proof tells the computer {\em how to
construct} specific members of closure of the formal system
pre-statement corresponding to the extended frame of the \texttt{\$p}
statement.  The final result of the construction is the member of the
closure that matches the \texttt{\$p} statement.  The abstract formal
system, on the other hand, is concerned only with the {\em existence} of
members of the closure.

As mentioned on p.~\pageref{exampleref}, Examples 1 and 3--6 in the
previous Section parallel the development of logic and set theory in the
Metamath database
\texttt{set.mm}.\index{set theory database (\texttt{set.mm})} You may
find it instructive to compare them.


\chapter{The MIU System}
\label{MIU}
\index{formal system}
\index{MIU-system}

The following is a listing of the file \texttt{miu.mm}.  It is self-explanatory.

%%%%%%%%%%%%%%%%%%%%%%%%%%%%%%%%%%%%%%%%%%%%%%%%%%%%%%%%%%%%

\begin{verbatim}
$( The MIU-system:  A simple formal system $)

$( Note:  This formal system is unusual in that it allows
empty wffs.  To work with a proof, you must type
SET EMPTY_SUBSTITUTION ON before using the PROVE command.
By default, this is OFF in order to reduce the number of
ambiguous unification possibilities that have to be selected
during the construction of a proof.  $)

$(
Hofstadter's MIU-system is a simple example of a formal
system that illustrates some concepts of Metamath.  See
Douglas R. Hofstadter, _Goedel, Escher, Bach:  An Eternal
Golden Braid_ (Vintage Books, New York, 1979), pp. 33ff. for
a description of the MIU-system.

The system has 3 constant symbols, M, I, and U.  The sole
axiom of the system is MI. There are 4 rules:
     Rule I:  If you possess a string whose last letter is I,
     you can add on a U at the end.
     Rule II:  Suppose you have Mx.  Then you may add Mxx to
     your collection.
     Rule III:  If III occurs in one of the strings in your
     collection, you may make a new string with U in place
     of III.
     Rule IV:  If UU occurs inside one of your strings, you
     can drop it.
Unfortunately, Rules III and IV do not have unique results:
strings could have more than one occurrence of III or UU.
This requires that we introduce the concept of an "MIU
well-formed formula" or wff, which allows us to construct
unique symbol sequences to which Rules III and IV can be
applied.
$)

$( First, we declare the constant symbols of the language.
Note that we need two symbols to distinguish the assertion
that a sequence is a wff from the assertion that it is a
theorem; we have arbitrarily chosen "wff" and "|-". $)
      $c M I U |- wff $. $( Declare constants $)

$( Next, we declare some variables. $)
     $v x y $.

$( Throughout our theory, we shall assume that these
variables represent wffs. $)
 wx   $f wff x $.
 wy   $f wff y $.

$( Define MIU-wffs.  We allow the empty sequence to be a
wff. $)

$( The empty sequence is a wff. $)
 we   $a wff $.
$( "M" after any wff is a wff. $)
 wM   $a wff x M $.
$( "I" after any wff is a wff. $)
 wI   $a wff x I $.
$( "U" after any wff is a wff. $)
 wU   $a wff x U $.

$( Assert the axiom. $)
 ax   $a |- M I $.

$( Assert the rules. $)
 ${
   Ia   $e |- x I $.
$( Given any theorem ending with "I", it remains a theorem
if "U" is added after it.  (We distinguish the label I_
from the math symbol I to conform to the 24-Jun-2006
Metamath spec.) $)
   I_    $a |- x I U $.
 $}
 ${
IIa  $e |- M x $.
$( Given any theorem starting with "M", it remains a theorem
if the part after the "M" is added again after it. $)
   II   $a |- M x x $.
 $}
 ${
   IIIa $e |- x I I I y $.
$( Given any theorem with "III" in the middle, it remains a
theorem if the "III" is replaced with "U". $)
   III  $a |- x U y $.
 $}
 ${
   IVa  $e |- x U U y $.
$( Given any theorem with "UU" in the middle, it remains a
theorem if the "UU" is deleted. $)
   IV   $a |- x y $.
  $}

$( Now we prove the theorem MUIIU.  You may be interested in
comparing this proof with that of Hofstadter (pp. 35 - 36).
$)
 theorem1  $p |- M U I I U $=
      we wM wU wI we wI wU we wU wI wU we wM we wI wU we wM
      wI wI wI we wI wI we wI ax II II I_ III II IV $.
\end{verbatim}\index{well-formed formula (wff)}

The \texttt{show proof /lemmon/renumber} command
yields the following display.  It is very similar
to the one in \cite[pp.~35--36]{Hofstadter}.\index{Hofstadter, Douglas R.}

\begin{verbatim}
1 ax             $a |- M I
2 1 II           $a |- M I I
3 2 II           $a |- M I I I I
4 3 I_           $a |- M I I I I U
5 4 III          $a |- M U I U
6 5 II           $a |- M U I U U I U
7 6 IV           $a |- M U I I U
\end{verbatim}

We note that Hofstadter's ``MU-puzzle,'' which asks whether
MU is a theorem of the MIU-system, cannot be answered using
the system above because the MU-puzzle is a question {\em
about} the system.  To prove the answer to the MU-puzzle,
a much more elaborate system is needed, namely one that
models the MIU-system within set theory.  (Incidentally, the
answer to the MU-puzzle is no.)

\chapter{Metamath Language EBNF}%
\label{BNF}%
\index{Metamath Language EBNF}

The following is a formal description of the basic Metamath language syntax
(with compressed proofs and support for unknown proof steps).
It is defined using the
Extended Backus--Naur Form (EBNF)\index{Extended Backus--Naur Form}\index{EBNF}
notation from W3C\index{W3C}
\textit{Extensible Markup Language (XML) 1.0 (Fifth Edition)}
(W3C Recommendation 26 November 2008) at
\url{https://www.w3.org/TR/xml/#sec-notation}.

The \texttt{database}
rule is processed until the end of the file (\texttt{EOF}).
The rules eventually require reading whitespace-separated tokens.
A token has an upper-case definition (see below)
or is a string constant in a non-token (such as \texttt{'\$a'}).
We intend for this to be correct, but if there is a conflict the
rules of section \ref{spec} govern. That section also discusses
non-syntax restrictions not shown here
(e.g., that each new label token
defined in a \texttt{hypothesis-stmt} or \texttt{assert-stmt}
must be unique).

\begin{verbatim}
database ::= outermost-scope-stmt*

outermost-scope-stmt ::=
  include-stmt | constant-stmt | stmt

/* File inclusion command; process file as a database.
   Databases should NOT have a comment in the filename. */
include-stmt ::= '$[' filename '$]'

/* Constant symbols declaration. */
constant-stmt ::= '$c' constant+ '$.'

/* A normal statement can occur in any scope. */
stmt ::= block | variable-stmt | disjoint-stmt |
  hypothesis-stmt | assert-stmt

/* A block. You can have 0 statements in a block. */
block ::= '${' stmt* '$}'

/* Variable symbols declaration. */
variable-stmt ::= '$v' variable+ '$.'

/* Disjoint variables. Simple disjoint statements have
   2 variables, i.e., "variable*" is empty for them. */
disjoint-stmt ::= '$d' variable variable variable* '$.'

hypothesis-stmt ::= floating-stmt | essential-stmt

/* Floating (variable-type) hypothesis. */
floating-stmt ::= LABEL '$f' typecode variable '$.'

/* Essential (logical) hypothesis. */
essential-stmt ::= LABEL '$e' typecode MATH-SYMBOL* '$.'

assert-stmt ::= axiom-stmt | provable-stmt

/* Axiomatic assertion. */
axiom-stmt ::= LABEL '$a' typecode MATH-SYMBOL* '$.'

/* Provable assertion. */
provable-stmt ::= LABEL '$p' typecode MATH-SYMBOL*
  '$=' proof '$.'

/* A proof. Proofs may be interspersed by comments.
   If '?' is in a proof it's an "incomplete" proof. */
proof ::= uncompressed-proof | compressed-proof
uncompressed-proof ::= (LABEL | '?')+
compressed-proof ::= '(' LABEL* ')' COMPRESSED-PROOF-BLOCK+

typecode ::= constant

filename ::= MATH-SYMBOL /* No whitespace or '$' */
constant ::= MATH-SYMBOL
variable ::= MATH-SYMBOL
\end{verbatim}

\needspace{2\baselineskip}
A \texttt{frame} is a sequence of 0 or more
\texttt{disjoint-{\allowbreak}stmt} and
\texttt{hypotheses-{\allowbreak}stmt} statements
(possibly interleaved with other non-\texttt{assert-stmt} statements)
followed by one \texttt{assert-stmt}.

\needspace{3\baselineskip}
Here are the rules for lexical processing (tokenization) beyond
the constant tokens shown above.
By convention these tokenization rules have upper-case names.
Every token is read for the longest possible length.
Whitespace-separated tokens are read sequentially;
note that the separating whitespace and \texttt{\$(} ... \texttt{\$)}
comments are skipped.

If a token definition uses another token definition, the whole thing
is considered a single token.
A pattern that is only part of a full token has a name beginning
with an underscore (``\_'').
An implementation could tokenize many tokens as a
\texttt{PRINTABLE-SEQUENCE}
and then check if it meets the more specific rule shown here.

Comments do not nest, and both \texttt{\$(} and \texttt{\$)}
have to be surrounded
by at least one whitespace character (\texttt{\_WHITECHAR}).
Technically comments end without consuming the trailing
\texttt{\_WHITECHAR}, but the trailing
\texttt{\_WHITECHAR} gets ignored anyway so we ignore that detail here.
Metamath language processors
are not required to support \texttt{\$)} followed
immediately by a bare end-of-file, because the closing
comment symbol is supposed to be followed by a
\texttt{\_WHITECHAR} such as a newline.

\begin{verbatim}
PRINTABLE-SEQUENCE ::= _PRINTABLE-CHARACTER+

MATH-SYMBOL ::= (_PRINTABLE-CHARACTER - '$')+

/* ASCII non-whitespace printable characters */
_PRINTABLE-CHARACTER ::= [#x21-#x7e]

LABEL ::= ( _LETTER-OR-DIGIT | '.' | '-' | '_' )+

_LETTER-OR-DIGIT ::= [A-Za-z0-9]

COMPRESSED-PROOF-BLOCK ::= ([A-Z] | '?')+

/* Define whitespace between tokens. The -> SKIP
   means that when whitespace is seen, it is
   skipped and we simply read again. */
WHITESPACE ::= (_WHITECHAR+ | _COMMENT) -> SKIP

/* Comments. $( ... $) and do not nest. */
_COMMENT ::= '$(' (_WHITECHAR+ (PRINTABLE-SEQUENCE - '$)'))*
  _WHITECHAR+ '$)' _WHITECHAR

/* Whitespace: (' ' | '\t' | '\r' | '\n' | '\f') */
_WHITECHAR ::= [#x20#x09#x0d#x0a#x0c]
\end{verbatim}
% This EBNF was developed as a collaboration between
% David A. Wheeler\index{Wheeler, David A.},
% Mario Carneiro\index{Carneiro, Mario}, and
% Benoit Jubin\index{Jubin, Benoit}, inspired by a request
% (and a lot of initial work) by Benoit Jubin.
%
% \chapter{Disclaimer and Trademarks}
%
% Information in this document is subject to change without notice and does not
% represent a commitment on the part of Norman Megill.
% \vspace{2ex}
%
% \noindent Norman D. Megill makes no warranties, either express or implied,
% regarding the Metamath computer software package.
%
% \vspace{2ex}
%
% \noindent Any trademarks mentioned in this book are the property of
% their respective owners.  The name ``Metamath'' is a trademark of
% Norman Megill.
%
\cleardoublepage
\phantomsection  % fixes the link anchor
\addcontentsline{toc}{chapter}{\bibname}

\bibliography{metamath}
%\input{metamath.bbl}

\raggedright
\cleardoublepage
\phantomsection % fixes the link anchor
\addcontentsline{toc}{chapter}{\indexname}
%\printindex   ??
\input{metamath.ind}

\end{document}



\raggedright
\cleardoublepage
\phantomsection % fixes the link anchor
\addcontentsline{toc}{chapter}{\indexname}
%\printindex   ??
% metamath.tex - Version of 2-Jun-2019
% If you change the date above, also change the "Printed date" below.
% SPDX-License-Identifier: CC0-1.0
%
%                              PUBLIC DOMAIN
%
% This file (specifically, the version of this file with the above date)
% has been released into the Public Domain per the
% Creative Commons CC0 1.0 Universal (CC0 1.0) Public Domain Dedication
% https://creativecommons.org/publicdomain/zero/1.0/
%
% The public domain release applies worldwide.  In case this is not
% legally possible, the right is granted to use the work for any purpose,
% without any conditions, unless such conditions are required by law.
%
% Several short, attributed quotations from copyrighted works
% appear in this file under the ``fair use'' provision of Section 107 of
% the United States Copyright Act (Title 17 of the {\em United States
% Code}).  The public-domain status of this file is not applicable to
% those quotations.
%
% Norman Megill - email: nm(at)alum(dot)mit(dot)edu
%
% David A. Wheeler also donates his improvements to this file to the
% public domain per the CC0.  He works at the Institute for Defense Analyses
% (IDA), but IDA has agreed that this Metamath work is outside its "lane"
% and is not a work by IDA.  This was specifically confirmed by
% Margaret E. Myers (Division Director of the Information Technology
% and Systems Division) on 2019-05-24 and by Ben Lindorf (General Counsel)
% on 2019-05-22.

% This file, 'metamath.tex', is self-contained with everything needed to
% generate the the PDF file 'metamath.pdf' (the _Metamath_ book) on
% standard LaTeX 2e installations.  The auxiliary files are embedded with
% "filecontents" commands.  To generate metamath.pdf file, run these
% commands under Linux or Cygwin in the directory that contains
% 'metamath.tex':
%
%   rm -f realref.sty metamath.bib
%   touch metamath.ind
%   pdflatex metamath
%   pdflatex metamath
%   bibtex metamath
%   makeindex metamath
%   pdflatex metamath
%   pdflatex metamath
%
% The warnings that occur in the initial runs of pdflatex can be ignored.
% For the final run,
%
%   egrep -i 'error|warn' metamath.log
%
% should show exactly these 5 warnings:
%
%   LaTeX Warning: File `realref.sty' already exists on the system.
%   LaTeX Warning: File `metamath.bib' already exists on the system.
%   LaTeX Font Warning: Font shape `OMS/cmtt/m/n' undefined
%   LaTeX Font Warning: Font shape `OMS/cmtt/bx/n' undefined
%   LaTeX Font Warning: Some font shapes were not available, defaults
%       substituted.
%
% Search for "Uncomment" below if you want to suppress hyperlink boxes
% in the PDF output file
%
% TYPOGRAPHICAL NOTES:
% * It is customary to use an en dash (--) to "connect" names of different
%   people (and to denote ranges), and use a hyphen (-) for a
%   single compound name. Examples of connected multiple people are
%   Zermelo--Fraenkel, Schr\"{o}der--Bernstein, Tarski--Grothendieck,
%   Hewlett--Packard, and Backus--Naur.  Examples of a single person with
%   a compound name include Levi-Civita, Mittag-Leffler, and Burali-Forti.
% * Use non-breaking spaces after page abbreviations, e.g.,
%   p.~\pageref{note2002}.
%
% --------------------------- Start of realref.sty -----------------------------
\begin{filecontents}{realref.sty}
% Save the following as realref.sty.
% You can then use it with \usepackage{realref}
%
% This has \pageref jumping to the page on which the ref appears,
% \ref jumping to the point of the anchor, and \sectionref
% jumping to the start of section.
%
% Author:  Anthony Williams
%          Software Engineer
%          Nortel Networks Optical Components Ltd
% Date:    9 Nov 2001 (posted to comp.text.tex)
%
% The following declaration was made by Anthony Williams on
% 24 Jul 2006 (private email to Norman Megill):
%
%   ``I hereby donate the code for realref.sty posted on the
%   comp.text.tex newsgroup on 9th November 2001, accessible from
%   http://groups.google.com/group/comp.text.tex/msg/5a0e1cc13ea7fbb2
%   to the public domain.''
%
\ProvidesPackage{realref}
\RequirePackage[plainpages=false,pdfpagelabels=true]{hyperref}
\def\realref@anchorname{}
\AtBeginDocument{%
% ensure every label is a possible hyperlink target
\let\realref@oldrefstepcounter\refstepcounter%
\DeclareRobustCommand{\refstepcounter}[1]{\realref@oldrefstepcounter{#1}
\edef\realref@anchorname{\string #1.\@currentlabel}%
}%
\let\realref@oldlabel\label%
\DeclareRobustCommand{\label}[1]{\realref@oldlabel{#1}\hypertarget{#1}{}%
\@bsphack\protected@write\@auxout{}{%
    \string\expandafter\gdef\protect\csname
    page@num.#1\string\endcsname{\thepage}%
    \string\expandafter\gdef\protect\csname
    ref@num.#1\string\endcsname{\@currentlabel}%
    \string\expandafter\gdef\protect\csname
    sectionref@name.#1\string\endcsname{\realref@anchorname}%
}\@esphack}%
\DeclareRobustCommand\pageref[1]{{\edef\a{\csname
            page@num.#1\endcsname}\expandafter\hyperlink{page.\a}{\a}}}%
\DeclareRobustCommand\ref[1]{{\edef\a{\csname
            ref@num.#1\endcsname}\hyperlink{#1}{\a}}}%
\DeclareRobustCommand\sectionref[1]{{\edef\a{\csname
            ref@num.#1\endcsname}\edef\b{\csname
            sectionref@name.#1\endcsname}\hyperlink{\b}{\a}}}%
}
\end{filecontents}
% ---------------------------- End of realref.sty ------------------------------

% --------------------------- Start of metamath.bib -----------------------------
\begin{filecontents}{metamath.bib}
@book{Albers, editor = "Donald J. Albers and G. L. Alexanderson",
  title = "Mathematical People",
  publisher = "Contemporary Books, Inc.",
  address = "Chicago",
  note = "[QA28.M37]",
  year = 1985 }
@book{Anderson, author = "Alan Ross Anderson and Nuel D. Belnap",
  title = "Entailment",
  publisher = "Princeton University Press",
  address = "Princeton",
  volume = 1,
  note = "[QA9.A634 1975 v.1]",
  year = 1975}
@book{Barrow, author = "John D. Barrow",
  title = "Theories of Everything:  The Quest for Ultimate Explanation",
  publisher = "Oxford University Press",
  address = "Oxford",
  note = "[Q175.B225]",
  year = 1991 }
@book{Behnke,
  editor = "H. Behnke and F. Backmann and K. Fladt and W. S{\"{u}}ss",
  title = "Fundamentals of Mathematics",
  volume = "I",
  publisher = "The MIT Press",
  address = "Cambridge, Massachusetts",
  note = "[QA37.2.B413]",
  year = 1974 }
@book{Bell, author = "J. L. Bell and M. Machover",
  title = "A Course in Mathematical Logic",
  publisher = "North-Holland",
  address = "Amsterdam",
  note = "[QA9.B3953]",
  year = 1977 }
@inproceedings{Blass, author = "Andrea Blass",
  title = "The Interaction Between Category Theory and Set Theory",
  pages = "5--29",
  booktitle = "Mathematical Applications of Category Theory (Proceedings
     of the Special Session on Mathematical Applications
     Category Theory, 89th Annual Meeting of the American Mathematical
     Society, held in Denver, Colorado January 5--9, 1983)",
  editor = "John Walter Gray",
  year = 1983,
  note = "[QA169.A47 1983]",
  publisher = "American Mathematical Society",
  address = "Providence, Rhode Island"}
@proceedings{Bledsoe, editor = "W. W. Bledsoe and D. W. Loveland",
  title = "Automated Theorem Proving:  After 25 Years (Proceedings
     of the Special Session on Automatic Theorem Proving,
     89th Annual Meeting of the American Mathematical
     Society, held in Denver, Colorado January 5--9, 1983)",
  year = 1983,
  note = "[QA76.9.A96.S64 1983]",
  publisher = "American Mathematical Society",
  address = "Providence, Rhode Island" }
@book{Boolos, author = "George S. Boolos and Richard C. Jeffrey",
  title = "Computability and Log\-ic",
  publisher = "Cambridge University Press",
  edition = "third",
  address = "Cambridge",
  note = "[QA9.59.B66 1989]",
  year = 1989 }
@book{Campbell, author = "John Campbell",
  title = "Programmer's Progress",
  publisher = "White Star Software",
  address = "Box 51623, Palo Alto, CA 94303",
  year = 1991 }
@article{DBLP:journals/corr/Carneiro14,
  author    = {Mario Carneiro},
  title     = {Conversion of {HOL} Light proofs into Metamath},
  journal   = {CoRR},
  volume    = {abs/1412.8091},
  year      = {2014},
  url       = {http://arxiv.org/abs/1412.8091},
  archivePrefix = {arXiv},
  eprint    = {1412.8091},
  timestamp = {Mon, 13 Aug 2018 16:47:05 +0200},
  biburl    = {https://dblp.org/rec/bib/journals/corr/Carneiro14},
  bibsource = {dblp computer science bibliography, https://dblp.org}
}
@article{CarneiroND,
  author    = {Mario Carneiro},
  title     = {Natural Deductions in the Metamath Proof Language},
  url       = {http://us.metamath.org/ocat/natded.pdf},
  year      = 2014
}
@inproceedings{Chou, author = "Shang-Ching Chou",
  title = "Proving Elementary Geometry Theorems Using {W}u's Algorithm",
  pages = "243--286",
  booktitle = "Automated Theorem Proving:  After 25 Years (Proceedings
     of the Special Session on Automatic Theorem Proving,
     89th Annual Meeting of the American Mathematical
     Society, held in Denver, Colorado January 5--9, 1983)",
  editor = "W. W. Bledsoe and D. W. Loveland",
  year = 1983,
  note = "[QA76.9.A96.S64 1983]",
  publisher = "American Mathematical Society",
  address = "Providence, Rhode Island" }
@book{Clemente, author = "Daniel Clemente Laboreo",
  title = "Introduction to natural deduction",
  year = 2014,
  url = "http://www.danielclemente.com/logica/dn.en.pdf" }
@incollection{Courant, author = "Richard Courant and Herbert Robbins",
  title = "Topology",
  pages = "573--590",
  booktitle = "The World of Mathematics, Volume One",
  editor = "James R. Newman",
  publisher = "Simon and Schuster",
  address = "New York",
  note = "[QA3.W67 1988]",
  year = 1956 }
@book{Curry, author = "Haskell B. Curry",
  title = "Foundations of Mathematical Logic",
  publisher = "Dover Publications, Inc.",
  address = "New York",
  note = "[QA9.C976 1977]",
  year = 1977 }
@book{Davis, author = "Philip J. Davis and Reuben Hersh",
  title = "The Mathematical Experience",
  publisher = "Birkh{\"{a}}user Boston",
  address = "Boston",
  note = "[QA8.4.D37 1982]",
  year = 1981 }
@incollection{deMillo,
  author = "Richard de Millo and Richard Lipton and Alan Perlis",
  title = "Social Processes and Proofs of Theorems and Programs",
  pages = "267--285",
  booktitle = "New Directions in the Philosophy of Mathematics",
  editor = "Thomas Tymoczko",
  publisher = "Birkh{\"{a}}user Boston, Inc.",
  address = "Boston",
  note = "[QA8.6.N48 1986]",
  year = 1986 }
@book{Edwards, author = "Robert E. Edwards",
  title = "A Formal Background to Mathematics",
  publisher = "Springer-Verlag",
  address = "New York",
  note = "[QA37.2.E38 v.1a]",
  year = 1979 }
@book{Enderton, author = "Herbert B. Enderton",
  title = "Elements of Set Theory",
  publisher = "Academic Press, Inc.",
  address = "San Diego",
  note = "[QA248.E5]",
  year = 1977 }
@book{Goodstein, author = "R. L. Goodstein",
  title = "Development of Mathematical Logic",
  publisher = "Springer-Verlag New York Inc.",
  address = "New York",
  note = "[QA9.G6554]",
  year = 1971 }
@book{Guillen, author = "Michael Guillen",
  title = "Bridges to Infinity",
  publisher = "Jeremy P. Tarcher, Inc.",
  address = "Los Angeles",
  note = "[QA93.G8]",
  year = 1983 }
@book{Hamilton, author = "Alan G. Hamilton",
  title = "Logic for Mathematicians",
  edition = "revised",
  publisher = "Cambridge University Press",
  address = "Cambridge",
  note = "[QA9.H298]",
  year = 1988 }
@unpublished{Harrison, author = "John Robert Harrison",
  title = "Metatheory and Reflection in Theorem Proving:
    A Survey and Critique",
  note = "Technical Report
    CRC-053.
    SRI Cambridge,
    Millers Yard, Cambridge, UK,
    1995.
    Available on the Web as
{\verb+http:+}\-{\verb+//www.cl.cam.ac.uk/users/jrh/papers/reflect.html+}"}
@TECHREPORT{Harrison-thesis,
        author          = "John Robert Harrison",
        title           = "Theorem Proving with the Real Numbers",
        institution   = "University of Cambridge Computer
                         Lab\-o\-ra\-to\-ry",
        address         = "New Museums Site, Pembroke Street, Cambridge,
                           CB2 3QG, UK",
        year            = 1996,
        number          = 408,
        type            = "Technical Report",
        note            = "Author's PhD thesis,
   available on the Web at
{\verb+http:+}\-{\verb+//www.cl.cam.ac.uk+}\-{\verb+/users+}\-{\verb+/jrh+}%
\-{\verb+/papers+}\-{\verb+/thesis.html+}"}
@book{Herrlich, author = "Horst Herrlich and George E. Strecker",
  title = "Category Theory:  An Introduction",
  publisher = "Allyn and Bacon Inc.",
  address = "Boston",
  note = "[QA169.H567]",
  year = 1973 }
@article{Hindley, author = "J. Roger Hindley and David Meredith",
  title = "Principal Type-Schemes and Condensed Detachment",
  journal = "The Journal of Symbolic Logic",
  volume = 55,
  year = 1990,
  note = "[QA.J87]",
  pages = "90--105" }
@book{Hofstadter, author = "Douglas R. Hofstadter",
  title = "G{\"{o}}del, Escher, Bach",
  publisher = "Basic Books, Inc.",
  address = "New York",
  note = "[QA9.H63 1980]",
  year = 1979 }
@article{Indrzejczak, author= "Andrzej Indrzejczak",
  title = "Natural Deduction, Hybrid Systems and Modal Logic",
  journal = "Trends in Logic",
  volume = 30,
  publisher = "Springer",
  year = 2010 }
@article{Kalish, author = "D. Kalish and R. Montague",
  title = "On {T}arski's Formalization of Predicate Logic with Identity",
  journal = "Archiv f{\"{u}}r Mathematische Logik und Grundlagenfor\-schung",
  volume = 7,
  year = 1965,
  note = "[QA.A673]",
  pages = "81--101" }
@article{Kalman, author = "J. A. Kalman",
  title = "Condensed Detachment as a Rule of Inference",
  journal = "Studia Logica",
  volume = 42,
  number = 4,
  year = 1983,
  note = "[B18.P6.S933]",
  pages = "443-451" }
@book{Kline, author = "Morris Kline",
  title = "Mathematical Thought from Ancient to Modern Times",
  publisher = "Oxford University Press",
  address = "New York",
  note = "[QA21.K516 1990 v.3]",
  year = 1972 }
@book{Klinel, author = "Morris Kline",
  title = "Mathematics, The Loss of Certainty",
  publisher = "Oxford University Press",
  address = "New York",
  note = "[QA21.K525]",
  year = 1980 }
@book{Kramer, author = "Edna E. Kramer",
  title = "The Nature and Growth of Modern Mathematics",
  publisher = "Princeton University Press",
  address = "Princeton, New Jersey",
  note = "[QA93.K89 1981]",
  year = 1981 }
@article{Knill, author = "Oliver Knill",
  title = "Some Fundamental Theorems in Mathematics",
  year = "2018",
  url = "https://arxiv.org/abs/1807.08416" }
@book{Landau, author = "Edmund Landau",
  title = "Foundations of Analysis",
  publisher = "Chelsea Publishing Company",
  address = "New York",
  edition = "second",
  note = "[QA241.L2541 1960]",
  year = 1960 }
@article{Leblanc, author = "Hugues Leblanc",
  title = "On {M}eyer and {L}ambert's Quantificational Calculus {FQ}",
  journal = "The Journal of Symbolic Logic",
  volume = 33,
  year = 1968,
  note = "[QA.J87]",
  pages = "275--280" }
@article{Lejewski, author = "Czeslaw Lejewski",
  title = "On Implicational Definitions",
  journal = "Studia Logica",
  volume = 8,
  year = 1958,
  note = "[B18.P6.S933]",
  pages = "189--208" }
@book{Levy, author = "Azriel Levy",
  title = "Basic Set Theory",
  publisher = "Dover Publications",
  address = "Mineola, NY",
  year = "2002"
}
@book{Margaris, author = "Angelo Margaris",
  title = "First Order Mathematical Logic",
  publisher = "Blaisdell Publishing Company",
  address = "Waltham, Massachusetts",
  note = "[QA9.M327]",
  year = 1967}
@book{Manin, author = "Yu I. Manin",
  title = "A Course in Mathematical Logic",
  publisher = "Springer-Verlag",
  address = "New York",
  note = "[QA9.M29613]",
  year = "1977" }
@article{Mathias, author = "Adrian R. D. Mathias",
  title = "A Term of Length 4,523,659,424,929",
  journal = "Synthese",
  volume = 133,
  year = 2002,
  note = "[Q.S993]",
  pages = "75--86" }
@article{Megill, author = "Norman D. Megill",
  title = "A Finitely Axiomatized Formalization of Predicate Calculus
     with Equality",
  journal = "Notre Dame Journal of Formal Logic",
  volume = 36,
  year = 1995,
  note = "[QA.N914]",
  pages = "435--453" }
@unpublished{Megillc, author = "Norman D. Megill",
  title = "A Shorter Equivalent of the Axiom of Choice",
  month = "June",
  note = "Unpublished",
  year = 1991 }
@article{MegillBunder, author = "Norman D. Megill and Martin W.
    Bunder",
  title = "Weaker {D}-Complete Logics",
  journal = "Journal of the IGPL",
  volume = 4,
  year = 1996,
  pages = "215--225",
  note = "Available on the Web at
{\verb+http:+}\-{\verb+//www.mpi-sb.mpg.de+}\-{\verb+/igpl+}%
\-{\verb+/Journal+}\-{\verb+/V4-2+}\-{\verb+/#Megill+}"}
}
@book{Mendelson, author = "Elliott Mendelson",
  title = "Introduction to Mathematical Logic",
  edition = "second",
  publisher = "D. Van Nostrand Company, Inc.",
  address = "New York",
  note = "[QA9.M537 1979]",
  year = 1979 }
@article{Meredith, author = "David Meredith",
  title = "In Memoriam {C}arew {A}rthur {M}eredith (1904-1976)",
  journal = "Notre Dame Journal of Formal Logic",
  volume = 18,
  year = 1977,
  note = "[QA.N914]",
  pages = "513--516" }
@article{CAMeredith, author = "C. A. Meredith",
  title = "Single Axioms for the Systems ({C},{N}), ({C},{O}) and ({A},{N})
      of the Two-Valued Propositional Calculus",
  journal = "The Journal of Computing Systems",
  volume = 3,
  year = 1953,
  pages = "155--164" }
@article{Monk, author = "J. Donald Monk",
  title = "Provability With Finitely Many Variables",
  journal = "The Journal of Symbolic Logic",
  volume = 27,
  year = 1971,
  note = "[QA.J87]",
  pages = "353--358" }
@article{Monks, author = "J. Donald Monk",
  title = "Substitutionless Predicate Logic With Identity",
  journal = "Archiv f{\"{u}}r Mathematische Logik und Grundlagenfor\-schung",
  volume = 7,
  year = 1965,
  pages = "103--121" }
  %% Took out this from above to prevent LaTeX underfull warning:
  % note = "[QA.A673]",
@book{Moore, author = "A. W. Moore",
  title = "The Infinite",
  publisher = "Routledge",
  address = "New York",
  note = "[BD411.M59]",
  year = 1989}
@book{Munkres, author = "James R. Munkres",
  title = "Topology: A First Course",
  publisher = "Prentice-Hall, Inc.",
  address = "Englewood Cliffs, New Jersey",
  note = "[QA611.M82]",
  year = 1975}
@article{Nemesszeghy, author = "E. Z. Nemesszeghy and E. A. Nemesszeghy",
  title = "On Strongly Creative Definitions:  A Reply to {V}. {F}. {R}ickey",
  journal = "Logique et Analyse (N.\ S.)",
  year = 1977,
  volume = 20,
  note = "[BC.L832]",
  pages = "111--115" }
@unpublished{Nemeti, author = "N{\'{e}}meti, I.",
  title = "Algebraizations of Quantifier Logics, an Overview",
  note = "Version 11.4, preprint, Mathematical Institute, Budapest,
    1994.  A shortened version without proofs appeared in
    ``Algebraizations of quantifier logics, an introductory overview,''
   {\em Studia Logica}, 50:485--569, 1991 [B18.P6.S933]"}
@article{Pavicic, author = "M. Pavi{\v{c}}i{\'{c}}",
  title = "A New Axiomatization of Unified Quantum Logic",
  journal = "International Journal of Theoretical Physics",
  year = 1992,
  volume = 31,
  note = "[QC.I626]",
  pages = "1753 --1766" }
@book{Penrose, author = "Roger Penrose",
  title = "The Emperor's New Mind",
  publisher = "Oxford University Press",
  address = "New York",
  note = "[Q335.P415]",
  year = 1989 }
@book{PetersonI, author = "Ivars Peterson",
  title = "The Mathematical Tourist",
  publisher = "W. H. Freeman and Company",
  address = "New York",
  note = "[QA93.P475]",
  year = 1988 }
@article{Peterson, author = "Jeremy George Peterson",
  title = "An automatic theorem prover for substitution and detachment systems",
  journal = "Notre Dame Journal of Formal Logic",
  volume = 19,
  year = 1978,
  note = "[QA.N914]",
  pages = "119--122" }
@book{Quine, author = "Willard Van Orman Quine",
  title = "Set Theory and Its Logic",
  edition = "revised",
  publisher = "The Belknap Press of Harvard University Press",
  address = "Cambridge, Massachusetts",
  note = "[QA248.Q7 1969]",
  year = 1969 }
@article{Robinson, author = "J. A. Robinson",
  title = "A Machine-Oriented Logic Based on the Resolution Principle",
  journal = "Journal of the Association for Computing Machinery",
  year = 1965,
  volume = 12,
  pages = "23--41" }
@article{RobinsonT, author = "T. Thacher Robinson",
  title = "Independence of Two Nice Sets of Axioms for the Propositional
    Calculus",
  journal = "The Journal of Symbolic Logic",
  volume = 33,
  year = 1968,
  note = "[QA.J87]",
  pages = "265--270" }
@book{Rucker, author = "Rudy Rucker",
  title = "Infinity and the Mind:  The Science and Philosophy of the
    Infinite",
  publisher = "Bantam Books, Inc.",
  address = "New York",
  note = "[QA9.R79 1982]",
  year = 1982 }
@book{Russell, author = "Bertrand Russell",
  title = "Mysticism and Logic, and Other Essays",
  publisher = "Barnes \& Noble Books",
  address = "Totowa, New Jersey",
  note = "[B1649.R963.M9 1981]",
  year = 1981 }
@article{Russell2, author = "Bertrand Russell",
  title = "Recent Work on the Principles of Mathematics",
  journal = "International Monthly",
  volume = 4,
  year = 1901,
  pages = "84"}
@article{Schmidt, author = "Eric Schmidt",
  title = "Reductions in Norman Megill's axiom system for complex numbers",
  url = "http://us.metamath.org/downloads/schmidt-cnaxioms.pdf",
  year = "2012" }
@book{Shoenfield, author = "Joseph R. Shoenfield",
  title = "Mathematical Logic",
  publisher = "Addison-Wesley Publishing Company, Inc.",
  address = "Reading, Massachusetts",
  year = 1967,
  note = "[QA9.S52]" }
@book{Smullyan, author = "Raymond M. Smullyan",
  title = "Theory of Formal Systems",
  publisher = "Princeton University Press",
  address = "Princeton, New Jersey",
  year = 1961,
  note = "[QA248.5.S55]" }
@book{Solow, author = "Daniel Solow",
  title = "How to Read and Do Proofs:  An Introduction to Mathematical
    Thought Process",
  publisher = "John Wiley \& Sons",
  address = "New York",
  year = 1982,
  note = "[QA9.S577]" }
@book{Stark, author = "Harold M. Stark",
  title = "An Introduction to Number Theory",
  publisher = "Markham Publishing Company",
  address = "Chicago",
  note = "[QA241.S72 1978]",
  year = 1970 }
@article{Swart, author = "E. R. Swart",
  title = "The Philosophical Implications of the Four-Color Problem",
  journal = "American Mathematical Monthly",
  year = 1980,
  volume = 87,
  month = "November",
  note = "[QA.A5125]",
  pages = "697--707" }
@book{Szpiro, author = "George G. Szpiro",
  title = "Poincar{\'{e}}'s Prize: The Hundred-Year Quest to Solve One
    of Math's Greatest Puzzles",
  publisher = "Penguin Books Ltd",
  address = "London",
  note = "[QA43.S985 2007]",
  year = 2007}
@book{Takeuti, author = "Gaisi Takeuti and Wilson M. Zaring",
  title = "Introduction to Axiomatic Set Theory",
  edition = "second",
  publisher = "Springer-Verlag New York Inc.",
  address = "New York",
  note = "[QA248.T136 1982]",
  year = 1982}
@inproceedings{Tarski, author = "Alfred Tarski",
  title = "What is Elementary Geometry",
  pages = "16--29",
  booktitle = "The Axiomatic Method, with Special Reference to Geometry and
     Physics (Proceedings of an International Symposium held at the University
     of California, Berkeley, December 26, 1957 --- January 4, 1958)",
  editor = "Leon Henkin and Patrick Suppes and Alfred Tarski",
  year = 1959,
  publisher = "North-Holland Publishing Company",
  address = "Amsterdam"}
@article{Tarski1965, author = "Alfred Tarski",
  title = "A Simplified Formalization of Predicate Logic with Identity",
  journal = "Archiv f{\"{u}}r Mathematische Logik und Grundlagenforschung",
  volume = 7,
  year = 1965,
  note = "[QA.A673]",
  pages = "61--79" }
@book{Tymoczko,
  title = "New Directions in the Philosophy of Mathematics",
  editor = "Thomas Tymoczko",
  publisher = "Birkh{\"{a}}user Boston, Inc.",
  address = "Boston",
  note = "[QA8.6.N48 1986]",
  year = 1986 }
@incollection{Wang,
  author = "Hao Wang",
  title = "Theory and Practice in Mathematics",
  pages = "129--152",
  booktitle = "New Directions in the Philosophy of Mathematics",
  editor = "Thomas Tymoczko",
  publisher = "Birkh{\"{a}}user Boston, Inc.",
  address = "Boston",
  note = "[QA8.6.N48 1986]",
  year = 1986 }
@manual{Webster,
  title = "Webster's New Collegiate Dictionary",
  organization = "G. \& C. Merriam Co.",
  address = "Springfield, Massachusetts",
  note = "[PE1628.W4M4 1977]",
  year = 1977 }
@manual{Whitehead, author = "Alfred North Whitehead",
  title = "An Introduction to Mathematics",
  year = 1911 }
@book{PM, author = "Alfred North Whitehead and Bertrand Russell",
  title = "Principia Mathematica",
  edition = "second",
  publisher = "Cambridge University Press",
  address = "Cambridge",
  year = "1927",
  note = "(3 vols.) [QA9.W592 1927]" }
@article{DBLP:journals/corr/Whalen16,
  author    = {Daniel Whalen},
  title     = {Holophrasm: a neural Automated Theorem Prover for higher-order logic},
  journal   = {CoRR},
  volume    = {abs/1608.02644},
  year      = {2016},
  url       = {http://arxiv.org/abs/1608.02644},
  archivePrefix = {arXiv},
  eprint    = {1608.02644},
  timestamp = {Mon, 13 Aug 2018 16:46:19 +0200},
  biburl    = {https://dblp.org/rec/bib/journals/corr/Whalen16},
  bibsource = {dblp computer science bibliography, https://dblp.org} }
@article{Wiedijk-revisited,
  author = {Freek Wiedijk},
  title = {The QED Manifesto Revisited},
  year = {2007},
  url = {http://mizar.org/trybulec65/8.pdf} }
@book{Wolfram,
  author = "Stephen Wolfram",
  title = "Mathematica:  A System for Doing Mathematics by Computer",
  edition = "second",
  publisher = "Addison-Wesley Publishing Co.",
  address = "Redwood City, California",
  note = "[QA76.95.W65 1991]",
  year = 1991 }
@book{Wos, author = "Larry Wos and Ross Overbeek and Ewing Lusk and Jim Boyle",
  title = "Automated Reasoning:  Introduction and Applications",
  edition = "second",
  publisher = "McGraw-Hill, Inc.",
  address = "New York",
  note = "[QA76.9.A96.A93 1992]",
  year = 1992 }

%
%
%[1] Church, Alonzo, Introduction to Mathematical Logic,
% Volume 1, Princeton University Press, Princeton, N. J., 1956.
%
%[2] Cohen, Paul J., Set Theory and the Continuum Hypothesis,
% W. A. Benjamin, Inc., Reading, Mass., 1966.
%
%[3] Hamilton, Alan G., Logic for Mathematicians, Cambridge
% University Press,
% Cambridge, 1988.

%[6] Kleene, Stephen Cole, Introduction to Metamathematics, D.  Van
% Nostrand Company, Inc., Princeton (1952).

%[13] Tarski, Alfred, "A simplified formalization of predicate
% logic with identity," Archiv fur Mathematische Logik und
% Grundlagenforschung, vol. 7 (1965), pp. 61-79.

%[14] Tarski, Alfred and Steven Givant, A Formalization of Set
% Theory Without Variables, American Mathematical Society Colloquium
% Publications, vol. 41, American Mathematical Society,
% Providence, R. I., 1987.

%[15] Zeman, J. J., Modal Logic, Oxford University Press, Oxford, 1973.
\end{filecontents}
% --------------------------- End of metamath.bib -----------------------------


%Book: Metamath
%Author:  Norman Megill Email:  nm at alum.mit.edu
%Author:  David A. Wheeler Email:  dwheeler at dwheeler.com

% A book template example
% http://www.stsci.edu/ftp/software/tex/bookstuff/book.template

\documentclass[leqno]{book} % LaTeX 2e. 10pt. Use [leqno,12pt] for 12pt
% hyperref 2002/05/27 v6.72r  (couldn't get pagebackref to work)
\usepackage[plainpages=false,pdfpagelabels=true]{hyperref}

\usepackage{needspace}     % Enable control over page breaks
\usepackage{breqn}         % automatic equation breaking
\usepackage{microtype}     % microtypography, reduces hyphenation

% Packages for flexible tables.  We need to be able to
% wrap text within a cell (with automatically-determined widths) AND
% split a table automatically across multiple pages.
% * "tabularx" wraps text in cells but only 1 page
% * "longtable" goes across pages but by itself is incompatible with tabularx
% * "ltxtable" combines longtable and tabularx, but table contents
%    must be in a separate file.
% * "ltablex" combines tabularx and longtable - must install specially
% * "booktabs" is recommended as a way to improve the look of tables,
%   but doesn't add these capabilities.
% * "tabu" much more capable and seems to be recommended. So use that.

\usepackage{makecell}      % Enable forced line splits within a table cell
% v4.13 needed for tabu: https://tex.stackexchange.com/questions/600724/dimension-too-large-after-recent-longtable-update
\usepackage{longtable}[=v4.13] % Enable multi-page tables  
\usepackage{tabu}          % Multi-page tables with wrapped text in a cell

% You can find more Tex packages using commands like:
% tlmgr search --file tabu.sty
% find /usr/share/texmf-dist/ -name '*tab*'
%
%%%%%%%%%%%%%%%%%%%%%%%%%%%%%%%%%%%%%%%%%%%%%%%%%%%%%%%%%%%%%%%%%%%%%%%%%%%%
% Uncomment the next 3 lines to suppress boxes and colors on the hyperlinks
%%%%%%%%%%%%%%%%%%%%%%%%%%%%%%%%%%%%%%%%%%%%%%%%%%%%%%%%%%%%%%%%%%%%%%%%%%%%
%\hypersetup{
%colorlinks,citecolor=black,filecolor=black,linkcolor=black,urlcolor=black
%}
%
\usepackage{realref}

% Restarting page numbers: try?
%   \printglossary
%   \cleardoublepage
%   \pagenumbering{arabic}
%   \setcounter{page}{1}    ???needed
%   \include{chap1}

% not used:
% \def\R2Lurl#1#2{\mbox{\href{#1}\texttt{#2}}}

\usepackage{amssymb}

% Version 1 of book: margins: t=.4, b=.2, ll=.4, rr=.55
% \usepackage{anysize}
% % \papersize{<height>}{<width>}
% % \marginsize{<left>}{<right>}{<top>}{<bottom>}
% \papersize{9in}{6in}
% % l/r 0.6124-0.6170 works t/b 0.2418-0.3411 = 192pp. 0.2926-03118=exact
% \marginsize{0.7147in}{0.5147in}{0.4012in}{0.2012in}

\usepackage{anysize}
% \papersize{<height>}{<width>}
% \marginsize{<left>}{<right>}{<top>}{<bottom>}
\papersize{9in}{6in}
% l/r 0.85in&0.6431-0.6539 works t/b ?-?
%\marginsize{0.85in}{0.6485in}{0.55in}{0.35in}
\marginsize{0.8in}{0.65in}{0.5in}{0.3in}

% \usepackage[papersize={3.6in,4.8in},hmargin=0.1in,vmargin={0.1in,0.1in}]{geometry}  % page geometry
\usepackage{special-settings}

\raggedbottom
\makeindex

\begin{document}
% Discourage page widows and orphans:
\clubpenalty=300
\widowpenalty=300

%%%%%%% load in AMS fonts %%%%%%% % LaTeX 2.09 - obsolete in LaTeX 2e
%\input{amssym.def}
%\input{amssym.tex}
%\input{c:/texmf/tex/plain/amsfonts/amssym.def}
%\input{c:/texmf/tex/plain/amsfonts/amssym.tex}

\bibliographystyle{plain}
\pagenumbering{roman}
\pagestyle{headings}

\thispagestyle{empty}

\hfill
\vfill

\begin{center}
{\LARGE\bf Metamath} \\
\vspace{1ex}
{\large A Computer Language for Mathematical Proofs} \\
\vspace{7ex}
{\large Norman Megill} \\
\vspace{7ex}
with extensive revisions by \\
\vspace{1ex}
{\large David A. Wheeler} \\
\vspace{7ex}
% Printed date. If changing the date below, also fix the date at the beginning.
2019-06-02
\end{center}

\vfill
\hfill

\newpage
\thispagestyle{empty}

\hfill
\vfill

\begin{center}
$\sim$\ {\sc Public Domain}\ $\sim$

\vspace{2ex}
This book (including its later revisions)
has been released into the Public Domain by Norman Megill per the
Creative Commons CC0 1.0 Universal (CC0 1.0) Public Domain Dedication.
David A. Wheeler has done the same.
This public domain release applies worldwide.  In case this is not
legally possible, the right is granted to use the work for any purpose,
without any conditions, unless such conditions are required by law.
See \url{https://creativecommons.org/publicdomain/zero/1.0/}.

\vspace{3ex}
Several short, attributed quotations from copyrighted works
appear in this book under the ``fair use'' provision of Section 107 of
the United States Copyright Act (Title 17 of the {\em United States
Code}).  The public-domain status of this book is not applicable to
those quotations.

\vspace{3ex}
Any trademarks used in this book are the property of their owners.

% QA76.9.L63.M??

% \vspace{1ex}
%
% \vspace{1ex}
% {\small Permission is granted to make and distribute verbatim copies of this
% book
% provided the copyright notice and this
% permission notice are preserved on all copies.}
%
% \vspace{1ex}
% {\small Permission is granted to copy and distribute modified versions of this
% book under the conditions for verbatim copying, provided that the
% entire
% resulting derived work is distributed under the terms of a permission
% notice
% identical to this one.}
%
% \vspace{1ex}
% {\small Permission is granted to copy and distribute translations of this
% book into another language, under the above conditions for modified
% versions,
% except that this permission notice may be stated in a translation
% approved by the
% author.}
%
% \vspace{1ex}
% %{\small   For a copy of the \LaTeX\ source files for this book, contact
% %the author.} \\
% \ \\
% \ \\

\vspace{7ex}
% ISBN: 1-4116-3724-0 \\
% ISBN: 978-1-4116-3724-5 \\
ISBN: 978-0-359-70223-7 \\
{\ } \\
Lulu Press \\
Morrisville, North Carolina\\
USA


\hfill
\vfill

Norman Megill\\ 93 Bridge St., Lexington, MA 02421 \\
E-mail address: \texttt{nm{\char`\@}alum.mit.edu} \\
\vspace{7ex}
David A. Wheeler \\
E-mail address: \texttt{dwheeler{\char`\@}dwheeler.com} \\
% See notes added at end of Preface for revision history. \\
% For current information on the Metamath software see \\
\vspace{7ex}
\url{http://metamath.org}
\end{center}

\hfill
\vfill

{\parindent0pt%
\footnotesize{%
Cover: Aleph null ($\aleph_0$) is the symbol for the
first infinite cardinal number, discovered by Georg Cantor in 1873.
We use a red aleph null (with dark outline and gold glow) as the Metamath logo.
Credit: Norman Megill (1994) and Giovanni Mascellani (2019),
public domain.%
\index{aleph null}%
\index{Metamath!logo}\index{Cantor, Georg}\index{Mascellani, Giovanni}}}

% \newpage
% \thispagestyle{empty}
%
% \hfill
% \vfill
%
% \begin{center}
% {\it To my son Robin Dwight Megill}
% \end{center}
%
% \vfill
% \hfill
%
% \newpage

\tableofcontents
%\listoftables

\chapter*{Preface}
\markboth{PREFACE}{PREFACE}
\addcontentsline{toc}{section}{Preface}


% (For current information, see the notes added at the
% end of this preface on p.~\pageref{note2002}.)

\subsubsection{Overview}

Metamath\index{Metamath} is a computer language and an associated computer
program for archiving, verifying, and studying mathematical proofs at a very
detailed level.  The Metamath language incorporates no mathematics per se but
treats all mathematical statements as mere sequences of symbols.  You provide
Metamath with certain special sequences (axioms) that tell it what rules
of inference are allowed.  Metamath is not limited to any specific field of
mathematics.  The Metamath language is simple and robust, with an
almost total absence of hard-wired syntax, and
we\footnote{Unless otherwise noted, the words
``I,'' ``me,'' and ``my'' refer to Norman Megill\index{Megill, Norman}, while
``we,'' ``us,'' and ``our'' refer to Norman Megill and
David A. Wheeler\index{Wheeler, David A.}.}
believe that it
provides about the simplest possible framework that allows essentially all of
mathematics to be expressed with absolute rigor.

% index test
%\newcommand{\nn}[1]{#1n}
%\index{aaa@bbb}
%\index{abc!def}
%\index{abd|see{qqq}}
%\index{abe|nn}
%\index{abf|emph}
%\index{abg|(}
%\index{abg|)}

Using the Metamath language, you can build formal or mathematical
systems\index{formal system}\footnote{A formal or mathematical system consists
of a collection of symbols (such as $2$, $4$, $+$ and $=$), syntax rules that
describe how symbols may be combined to form a legal expression (called a
well-formed formula or {\em wff}, pronounced ``whiff''), some starting wffs
called axioms, and inference rules that describe how theorems may be derived
(proved) from the axioms.  A theorem is a mathematical fact such as $2+2=4$.
Strictly speaking, even an obvious fact such as this must be proved from
axioms to be formally acceptable to a mathematician.}\index{theorem}
\index{axiom}\index{rule}\index{well-formed formula (wff)} that involve
inferences from axioms.  Although a database is provided
that includes a recommended set of axioms for standard mathematics, if you
wish you can supply your own symbols, syntax, axioms, rules, and definitions.

The name ``Metamath'' was chosen to suggest that the language provides a
means for {\em describing} mathematics rather than {\em being} the
mathematics itself.  Actually in some sense any mathematical language is
metamathematical.  Symbols written on paper, or stored in a computer,
are not mathematics itself but rather a way of expressing mathematics.
For example ``7'' and ``VII'' are symbols for denoting the number seven
in Arabic and Roman numerals; neither {\em is} the number seven.

If you are able to understand and write computer programs, you should be able
to follow abstract mathematics with the aid of Metamath.  Used in conjunction
with standard textbooks, Metamath can guide you step-by-step towards an
understanding of abstract mathematics from a very rigorous viewpoint, even if
you have no formal abstract mathematics background.  By using a single,
consistent notation to express proofs, once you grasp its basic concepts
Metamath provides you with the ability to immediately follow and dissect
proofs even in totally unfamiliar areas.

Of course, just being able follow a proof will not necessarily give you an
intuitive familiarity with mathematics.  Memorizing the rules of chess does not
give you the ability to appreciate the game of a master, and knowing how the
notes on a musical score map to piano keys does not give you the ability to
hear in your head how it would sound.  But each of these can be a first step.

Metamath allows you to explore proofs in the sense that you can see the
theorem referenced at any step expanded in as much detail as you want, right
down to the underlying axioms of logic and set theory (in the case of the set
theory database provided).  While Metamath will not replace the higher-level
understanding that can only be acquired through exercises and hard work, being
able to see how gaps in a proof are filled in can give you increased
confidence that can speed up the learning process and save you time when you
get stuck.

The Metamath language breaks down a mathematical proof into its tiniest
possible parts.  These can be pieced together, like interlocking
pieces in a puzzle, only in a way that produces correct and absolutely rigorous
mathematics.

The nature of Metamath\index{Metamath} enforces very precise mathematical
thinking, similar to that involved in writing a computer program.  A crucial
difference, though, is that once a proof is verified (by the Metamath program)
to be correct, it is definitely correct; it can never have a hidden
``bug.''\index{computer program bugs}  After getting used to the kind of rigor
and accuracy provided by Metamath, you might even be tempted to
adopt the attitude that a proof should never be considered correct until it
has been verified by a computer, just as you would not completely trust a
manual calculation until you have verified it on a
calculator.

My goal
for Metamath was a system for describing and verifying
mathematics that is completely universal yet conceptually as simple as
possible.  In approaching mathematics from an axiomatic, formal viewpoint, I
wanted Metamath to be able to handle almost any mathematical system, not
necessarily with ease, but at least in principle and hopefully in practice. I
wanted it to verify proofs with absolute rigor, and for this reason Metamath
is what might be thought of as a ``compile-only'' language rather than an
algorithmic or Turing-machine language (Pascal, C, Prolog, Mathematica,
etc.).  In other words, a database written in the Metamath
language doesn't ``do'' anything; it merely exhibits mathematical knowledge
and permits this knowledge to be verified as being correct.  A program in an
algorithmic language can potentially have hidden bugs\index{computer program
bugs} as well as possibly being hard to understand.  But each token in a
Metamath database must be consistent with the database's earlier
contents according to simple, fixed rules.
If a database is verified
to be correct,\footnote{This includes
verification that a sequential list of proof steps results in the specified
theorem.} then the mathematical content is correct if the
verifier is correct and the axioms are correct.
The verification program could be incorrect, but the verification algorithm
is relatively simple (making it unlikely to be implemented incorrectly
by the Metamath program),
and there are over a dozen Metamath database verifiers
written by different people in different programming languages
(so these different verifiers can act as multiple reviewers of a database).
The most-used Metamath database, the Metamath Proof Explorer
(aka \texttt{set.mm}\index{set theory database (\texttt{set.mm})}%
\index{Metamath Proof Explorer}),
is currently verified by four different Metamath verifiers written by
four different people in four different languages, including the
original Metamath program described in this book.
The only ``bugs'' that can exist are in the statement of the axioms,
for example if the axioms are inconsistent (a famous problem shown to be
unsolvable by G\"{o}del's incompleteness theorem\index{G\"{o}del's
incompleteness theorem}).
However, real mathematical systems have very few axioms, and these can
be carefully studied.
All of this provides extraordinarily high confidence that the verified database
is in fact correct.

The Metamath program
doesn't prove theorems automatically but is designed to verify proofs
that you supply to it.
The underlying Metamath language is completely general and has no built-in,
preconceived notions about your formal system\index{formal system}, its logic
or its syntax.
For constructing proofs, the Metamath program has a Proof Assistant\index{Proof
Assistant} which helps you fill in some of a proof step's details, shows you
what choices you have at any step, and verifies the proof as you build it; but
you are still expected to provide the proof.

There are many other programs that can process or generate information
in the Metamath language, and more continue to be written.
This is in part because the Metamath language itself is very simple
and intentionally easy to automatically process.
Some programs, such as \texttt{mmj2}\index{mmj2}, include a proof assistant
that can automate some steps beyond what the Metamath program can do.
Mario Carneiro has developed an algorithm for converting proofs from
the OpenTheory interchange format, which can be translated to and from
any of the HOL family of proof languages (HOL4, HOL Light, ProofPower,
and Isabelle), into the
Metamath language \cite{DBLP:journals/corr/Carneiro14}\index{Carneiro, Mario}.
Daniel Whalen has developed Holophrasm, which can automatically
prove many Metamath proofs using
machine learning\index{machine learning}\index{artificial intelligence}
approaches
(including multiple neural networks\index{neural networks})\cite{DBLP:journals/corr/Whalen16}\index{Whalen, Daniel}.
However,
a discussion of these other programs is beyond the scope of this book.

Like most computer languages, the Metamath\index{Metamath} language uses the
standard ({\sc ascii}) characters on a computer keyboard, so it cannot
directly represent many of the special symbols that mathematicians use.  A
useful feature of the Metamath program is its ability to convert its notation
into the \LaTeX\ typesetting language.\index{latex@{\LaTeX}}  This feature
lets you convert the {\sc ascii} tokens you've defined into standard
mathematical symbols, so you end up with symbols and formulas you are familiar
with instead of somewhat cryptic {\sc ascii} representations of them.
The Metamath program can also generate HTML\index{HTML}, making it easy
to view results on the web and to see related information by using
hypertext links.

Metamath is probably conceptually different from anything you've seen
before and some aspects may take some getting used to.  This book will
help you decide whether Metamath suits your specific needs.

\subsubsection{Setting Your Expectations}
It is important for you to understand what Metamath\index{Metamath} is and is
not.  As mentioned, the Metamath program
is {\em not} an automated theorem prover but
rather a proof verifier.  Developing a database can be tedious, hard work,
especially if you want to make the proofs as short as possible, but it becomes
easier as you build up a collection of useful theorems.  The purpose of
Metamath is simply to document existing mathematics in an absolutely rigorous,
computer-verifiable way, not to aid directly in the creation of new
mathematics.  It also is not a magic solution for learning abstract
mathematics, although it may be helpful to be able to actually see the implied
rigor behind what you are learning from textbooks, as well as providing hints
to work out proofs that you are stumped on.

As of this writing, a sizable set theory database has been developed to
provide a foundation for many fields of mathematics, but much more work would
be required to develop useful databases for specific fields.

Metamath\index{Metamath} ``knows no math;'' it just provides a framework in
which to express mathematics.  Its language is very small.  You can define two
kinds of symbols, constants\index{constant} and variables\index{variable}.
The only thing Metamath knows how to do is to substitute strings of symbols
for the variables\index{substitution!variable}\index{variable substitution} in
an expression based on instructions you provide it in a proof, subject to
certain constraints you specify for the variables.  Even the decimal
representation of a number is merely a string of certain constants (digits)
which together, in a specific context, correspond to whatever mathematical
object you choose to define for it; unlike other computer languages, there is
no actual number stored inside the computer.  In a proof, you in effect
instruct Metamath what symbol substitutions to make in previous axioms or
theorems and join a sequence of them together to result in the desired
theorem.  This kind of symbol manipulation captures the essence of mathematics
at a preaxiomatic level.

\subsubsection{Metamath and Mathematical Literature}

In advanced mathematical literature, proofs are usually presented in the form
of short outlines that often only an expert can follow.  This is partly out of
a desire for brevity, but it would also be unwise (even if it were practical)
to present proofs in complete formal detail, since the overall picture would
be lost.\index{formal proof}

A solution I envision\label{envision} that would allow mathematics to remain
acceptable to the expert, yet increase its accessibility to non-specialists,
consists of a combination of the traditional short, informal proof in print
accompanied by a complete formal proof stored in a computer database.  In an
analogy with a computer program, the informal proof is like a set of comments
that describe the overall reasoning and content of the proof, whereas the
computer database is like the actual program and provides a means for anyone,
even a non-expert, to follow the proof in as much detail as desired, exploring
it back through layers of theorems (like subroutines that call other
subroutines) all the way back to the axioms of the theory.  In addition, the
computer database would have the advantage of providing absolute assurance
that the proof is correct, since each step can be verified automatically.

There are several other approaches besides Metamath to a project such
as this.  Section~\ref{proofverifiers} discusses some of these.

To us, a noble goal would be a database with hundreds of thousands of
theorems and their computer-verifiable proofs, encompassing a significant
fraction of known mathematics and available for instant access.
These would be fully verified by multiple independently-implemented verifiers,
to provide extremely high confidence that the proofs are completely correct.
The database would allow people to investigate whatever details they were
interested in, so that they could confirm whatever portions they wished.
Whether or not Metamath is an appropriate choice remains to be seen, but in
principle we believe it is sufficient.

\subsubsection{Formalism}

Over the past fifty years, a group of French mathematicians working
collectively under the pseudonym of Bourbaki\index{Bourbaki, Nicolas} have
co-authored a series of monographs that attempt to rigorously and
consistently formalize large bodies of mathematics from foundations.  On the
one hand, certainly such an effort has its merits; on the other hand, the
Bourbaki project has been criticized for its ``scholasticism'' and
``hyperaxiomatics'' that hide the intuitive steps that lead to the results
\cite[p.~191]{Barrow}\index{Barrow, John D.}.

Metamath unabashedly carries this philosophy to its extreme and no doubt is
subject to the same kind of criticism.  Nonetheless I think that in
conjunction with conventional approaches to mathematics Metamath can serve a
useful purpose.  The Bourbaki approach is essentially pedagogic, requiring the
reader to become intimately familiar with each detail in a very large
hierarchy before he or she can proceed to the next step.  The difference with
Metamath is that the ``reader'' (user) knows that all details are contained in
its computer database, available as needed; it does not demand that the user
know everything but conveniently makes available those portions that are of
interest.  As the body of all mathematical knowledge grows larger and larger,
no one individual can have a thorough grasp of its entirety.  Metamath
can finalize and put to rest any questions about the validity of any part of it
and can make any part of it accessible, in principle, to a non-specialist.

\subsubsection{A Personal Note}
Why did I develop Metamath\index{Metamath}?  I enjoy abstract mathematics, but
I sometimes get lost in a barrage of definitions and start to lose confidence
that my proofs are correct.  Or I reach a point where I lose sight of how
anything I'm doing relates to the axioms that a theory is based on and am
sometimes suspicious that there may be some overlooked implicit axiom
accidentally introduced along the way (as happened historically with Euclidean
geometry\index{Euclidean geometry}, whose omission of Pasch's
axiom\index{Pasch's axiom} went unnoticed for 2000 years
\cite[p.~160]{Davis}!). I'm also somewhat lazy and wish to avoid the effort
involved in re-verifying the gaps in informal proofs ``left to the reader;'' I
prefer to figure them out just once and not have to go through the same
frustration a year from now when I've forgotten what I did.  Metamath provides
better recovery of my efforts than scraps of paper that I can't
decipher anymore.  But mostly I find very appealing the idea of rigorously
archiving mathematical knowledge in a computer database, providing precision,
certainty, and elimination of human error.

\subsubsection{Note on Bibliography and Index}

The Bibliography usually includes the Library of Congress classification
for a work to make it easier for you to find it in on a university
library shelf.  The Index has author references to pages where their works
are cited, even though the authors' names may not appear on those pages.

\subsubsection{Acknowledgments}

Acknowledgments are first due to my wife, Deborah (who passed away on
September 4, 1998), for critiquing the manu\-script but most of all for
her patience and support.  I also wish to thank Joe Wright, Richard
Becker, Clarke Evans, Buddha Buck, and Jeremy Henty for helpful
comments.  Any errors, omissions, and other shortcomings are of course
my responsibility.

\subsubsection{Note Added June 22, 2005}\label{note2002}

The original, unpublished version of this book was written in 1997 and
distributed via the web.  The present edition has been updated to
reflect the current Metamath program and databases, as well as more
current {\sc url}s for Internet sites.  Thanks to Josh
Purinton\index{Purinton, Josh}, One Hand
Clapping, Mel L.\ O'Cat, and Roy F. Longton for pointing out
typographical and other errors.  I have also benefitted from numerous
discussions with Raph Levien\index{Levien, Raph}, who has extended
Metamath's philosophy of rigor to result in his {\em
Ghilbert}\index{Ghilbert} proof language (\url{http://ghilbert.org}).

Robert (Bob) Solovay\index{Solovay, Robert} communicated a new result of
A.~R.~D.~Mathias on the system of Bourbaki, and the text has been
updated accordingly (p.~\pageref{bourbaki}).

Bob also pointed out a clarification of the literature regarding
category theory and inaccessible cardinals\index{category
theory}\index{cardinal, inaccessible} (p.~\pageref{categoryth}),
and a misleading statement was removed from the text.  Specifically,
contrary to a statement in previous editions, it is possible to express
``There is a proper class of inaccessible cardinals'' in the language of
ZFC.  This can be done as follows:  ``For every set $x$ there is an
inaccessible cardinal $\kappa$ such that $\kappa$ is not in $x$.''
Bob writes:\footnote{Private communication, Nov.~30, 2002.}
\begin{quotation}
     This axiom is how Grothendieck presents category theory.  To each
inaccessible cardinal $\kappa$ one associates a Grothendieck universe
\index{Grothendieck, Alexander} $U(\kappa)$.  $U(\kappa)$ consists of
those sets which lie in a transitive set of cardinality less than
$\kappa$.  Instead of the ``category of all groups,'' one works relative
to a universe [considering the category of groups of cardinality less
than $\kappa$].  Now the category whose objects are all categories
``relative to the universe $U(\kappa)$'' will be a category not
relative to this universe but to the next universe.

     All of the things category theorists like to do can be done in this
framework.  The only controversial point is whether the Grothen\-dieck
axiom is too strong for the needs of category theorists.  Mac Lane
\index{Mac Lane, Saunders} argues that ``one universe is enough'' and
Feferman\index{Feferman, Solomon} has argued that one can get by with
ordinary ZFC.  I don't find Feferman's arguments persuasive.  Mac Lane
may be right, but when I think about category theory I do it \`{a} la
Grothendieck.

        By the way Mizar\index{Mizar} adds the axiom ``there is a proper
class of inaccessibles'' precisely so as to do category theory.
\end{quotation}

The most current information on the Metamath program and databases can
always be found at \url{http://metamath.org}.


\subsubsection{Note Added June 24, 2006}\label{note2006}

The Metamath spec was restricted slightly to make parsers easier to
write.  See the footnote on p.~\pageref{namespace}.

%\subsubsection{Note Added July 24, 2006}\label{note2006b}
\subsubsection{Note Added March 10, 2007}\label{note2006b}

I am grateful to Anthony Williams\index{Williams, Anthony} for writing
the \LaTeX\ package called {\tt realref.sty} and contributing it to the
public domain.  This package allows the internal hyperlinks in a {\sc
pdf} file to anchor to specific page numbers instead of just section
titles, making the navigation of the {\sc pdf} file for this book much
more pleasant and ``logical.''

A typographical error found by Martin Kiselkov was corrected.
A confusing remark about unification was deleted per suggestion of
Mel O'Cat.

\subsubsection{Note Added May 27, 2009}\label{note2009}

Several typos found by Kim Sparre were corrected.  A note was added that
the Poincar\'{e} conjecture has been proved (p.~\pageref{poincare}).

\subsubsection{Note Added Nov. 17, 2014}\label{note2014}

The statement of the Schr\"{o}der--Bernstein theorem was corrected in
Section~\ref{trust}.  Thanks to Bob Solovay for pointing out the error.

\subsubsection{Note Added May 25, 2016}\label{note2016}

Thanks to Jerry James for correcting 16 typos.

\subsubsection{Note Added February 25, 2019}\label{note201902}

David A. Wheeler\index{Wheeler, David A.}
made a large number of improvements and updates,
in coordination with Norman Megill.
The predicate calculus axioms were renumbered, and the text makes
it clear that they are based on Tarski's system S2;
the one slight deviation in axiom ax-6 is explained and justified.
The real and complex number axioms were modified to be consistent with
\texttt{set.mm}\index{set theory database (\texttt{set.mm})}%
\index{Metamath Proof Explorer}.
Long-awaited specification changes ``1--8'' were made,
which clarified previously ambiguous points.
Some errors in the text involving \texttt{\$f} and
\texttt{\$d} statements were corrected (the spec was correct, but
the in-book explanations unintentionally contradicted the spec).
We now have a system for automatically generating narrow PDFs,
so that those with smartphones can have easy access to the current
version of this document.
A new section on deduction was added;
it discusses the standard deduction theorem,
the weak deduction theorem,
deduction style, and natural deduction.
Many minor corrections (too numerous to list here) were also made.

\subsubsection{Note Added March 7, 2019}\label{note201903}

This added a description of the Matamath language syntax in
Extended Backus--Naur Form (EBNF)\index{Extended Backus--Naur Form}\index{EBNF}
in Appendix \ref{BNF}, added a brief explanation about typecodes,
inserted more examples in the deduction section,
and added a variety of smaller improvements.

\subsubsection{Note Added April 7, 2019}\label{note201904}

This version clarified the proper substitution notation, improved the
discussion on the weak deduction theorem and natural deduction,
documented the \texttt{undo} command, updated the information on
\texttt{write source}, changed the typecode
from \texttt{set} to \texttt{setvar} to be consistent with the current
version of \texttt{set.mm}, added more documentation about comment markup
(e.g., documented how to create headings), and clarified the
differences between various assertion forms (in particular deduction form).

\subsubsection{Note Added June 2, 2019}\label{note201906}

This version fixes a large number of small issues reported by
Beno\^{i}t Jubin\index{Jubin, Beno\^{i}t}, such as editorial issues
and the need to document \texttt{verify markup} (thank you!).
This version also includes specific examples
of forms (deduction form, inference form, and closed form).
We call this version the ``second edition'';
the previous edition formally published in 2007 had a slightly different title
(\textit{Metamath: A Computer Language for Pure Mathematics}).

\chapter{Introduction}
\pagenumbering{arabic}

\begin{quotation}
  {\em {\em I.M.:}  No, no.  There's nothing subjective about it!  Everybody
knows what a proof is.  Just read some books, take courses from a competent
mathematician, and you'll catch on.

{\em Student:}  Are you sure?

{\em I.M.:}  Well---it is possible that you won't, if you don't have any
aptitude for it.  That can happen, too.

{\em Student:}  Then {\em you} decide what a proof is, and if I don't learn
to decide in the same way, you decide I don't have any aptitude.

{\em I.M.:}  If not me, then who?}
    \flushright\sc  ``The Ideal Mathematician''
    \index{Davis, Phillip J.}
    \footnote{\cite{Davis}, p.~40.}\\
\end{quotation}

Brilliant mathematicians have discovered almost
unimaginably profound results that rank among the crowning intellectual
achievements of mankind.  However, there is a sense in which modern abstract
mathematics is behind the times, stuck in an era before computers existed.
While no one disputes the remarkable results that have been achieved,
communicating these results in a precise way to the uninitiated is virtually
impossible.  To describe these results, a terse informal language is used which
despite its elegance is very difficult to learn.  This informal language is not
imprecise, far from it, but rather it often has omitted detail
and symbols with hidden context that are
implicitly understood by an expert but few others.  Extremely complex technical
meanings are associated with innocent-sounding English words such as
``compact'' and ``measurable'' that barely hint at what is actually being
said.  Anyone who does not keep the precise technical meaning constantly in
mind is bound to fail, and acquiring the ability to do this can be achieved
only through much practice and hard work.  Only the few who complete the
painful learning experience can join the small in-group of pure
mathematicians.  The informal language effectively cuts off the true nature of
their knowledge from most everyone else.

Metamath\index{Metamath} makes abstract mathematics more concrete.  It allows
a computer to keep track of the complexity associated with each word or symbol
with absolute rigor.  You can explore this complexity at your leisure, to
whatever degree you desire.  Whether or not you believe that concepts such as
infinity actually ``exist'' outside of the mind, Metamath lets you get to the
foundation for what's really being said.

Metamath also enables completely rigorous and thorough proof verification.
Its language is simple enough so that you
don't have to rely on the authority of experts but can verify the results
yourself, step by step.  If you want to attempt to derive your own results,
Metamath will not let you make a mistake in reasoning.
Even professional mathematicians make mistakes; Metamath makes it possible
to thoroughly verify that proofs are correct.

Metamath\index{Metamath} is a computer language and an associated computer
program for archiving, verifying, and studying mathematical proofs at a very
detailed level.
The Metamath language
describes formal\index{formal system} mathematical
systems and expresses proofs of theorems in those systems.  Such a language
is called a metalanguage\index{metalanguage} by mathematicians.
The Metamath program is a computer program that verifies
proofs expressed in the Metamath language.
The Metamath program does not have the built-in
ability to make logical inferences; it just makes a series of symbol
substitutions according to instructions given to it in a proof
and verifies that the result matches the expected theorem.  It makes logical
inferences based only on rules of logic that are contained in a set of
axioms\index{axiom}, or first principles, that you provide to it as the
starting point for proofs.

The complete specification of the Metamath language is only four pages long
(Section~\ref{spec}, p.~\pageref{spec}).  Its simplicity may at first make you
wonder how it can do much of anything at all.  But in fact the kinds of
symbol manipulations it performs are the ones that are implicitly done in all
mathematical systems at the lowest level.  You can learn it relatively quickly
and have complete confidence in any mathematical proof that it verifies.  On
the other hand, it is powerful and general enough so that virtually any
mathematical theory, from the most basic to the deeply abstract, can be
described with it.

Although in principle Metamath can be used with any
kind of mathematics, it is best suited for abstract or ``pure'' mathematics
that is mostly concerned with theorems and their proofs, as opposed to the
kind of mathematics that deals with the practical manipulation of numbers.
Examples of branches of pure mathematics are logic\index{logic},\footnote{Logic
is the study of statements that are universally true regardless of the objects
being described by the statements.  An example is the statement, ``if $P$
implies $Q$, then either $P$ is false or $Q$ is true.''} set theory\index{set
theory},\footnote{Set theory is the study of general-purpose mathematical objects called
``sets,'' and from it essentially all of mathematics can be derived.  For
example, numbers can be defined as specific sets, and their properties
can be explored using the tools of set theory.} number theory\index{number
theory},\footnote{Number theory deals with the properties of positive and
negative integers (whole numbers).} group theory\index{group
theory},\footnote{Group theory studies the properties of mathematical objects
called groups that obey a simple set of axioms and have properties of symmetry
that make them useful in many other fields.} abstract algebra\index{abstract
algebra},\footnote{Abstract algebra includes group theory and also studies
groups with additional properties that qualify them as ``rings'' and
``fields.''  The set of real numbers is a familiar example of a field.},
analysis\index{analysis} \index{real and complex numbers}\footnote{Analysis is
the study of real and complex numbers.} and
topology\index{topology}.\footnote{One area studied by topology are properties
that remain unchanged when geometrical objects undergo stretching
deformations; for example a doughnut and a coffee cup each have one hole (the
cup's hole is in its handle) and are thus considered topologically
equivalent.  In general, though, topology is the study of abstract
mathematical objects that obey a certain (surprisingly simple) set of axioms.
See, for example, Munkres \cite{Munkres}\index{Munkres, James R.}.} Even in
physics, Metamath could be applied to certain branches that make use of
abstract mathematics, such as quantum logic\index{quantum logic} (used to study
aspects of quantum mechanics\index{quantum mechanics}).

On the other hand, Metamath\index{Metamath} is less suited to applications
that deal primarily with intensive numeric computations.  Metamath does not
have any built-in representation of numbers\index{Metamath!representation of
numbers}; instead, a specific string of symbols (digits) must be syntactically
constructed as part of any proof in which an ordinary number is used.  For
this reason, numbers in Metamath are best limited to specific constants that
arise during the course of a theorem or its proof.  Numbers are only a tiny
part of the world of abstract mathematics.  The exclusion of built-in numbers
was a conscious decision to help achieve Metamath's simplicity, and there are
other software tools if you have different mathematical needs.
If you wish to quickly solve algebraic problems, the computer algebra
programs\index{computer algebra system} {\sc
macsyma}\index{macsyma@{\sc macsyma}}, Mathematica\index{Mathematica}, and
Maple\index{Maple} are specifically suited to handling numbers and
algebra efficiently.
If you wish to simply calculate numeric or matrix expressions easily,
tools such as Octave\index{Octave} may be a better choice.

After learning Metamath's basic statement types, any
tech\-ni\-cal\-ly ori\-ent\-ed person, mathematician or not, can
immediately trace
any theorem proved in the language as far back as he or she wants, all the way
to the axioms on which the theorem is based.  This ability suggests a
non-traditional way of learning about pure mathematics.  Used in conjunction
with traditional methods, Metamath could make pure mathematics accessible to
people who are not sufficiently skilled to figure out the implicit detail in
ordinary textbook proofs.  Once you learn the axioms of a theory, you can have
complete confidence that everything you need to understand a proof you are
studying is all there, at your beck and call, allowing you to focus in on any
proof step you don't understand in as much depth as you need, without worrying
about getting stuck on a step you can't figure out.\footnote{On the other
hand, writing proofs in the Metamath language is challenging, requiring
a degree of rigor far in excess of that normally taught to students.  In a
classroom setting, I doubt that writing Metamath proofs would ever replace
traditional homework exercises involving informal proofs, because the time
needed to work out the details would not allow a course to
cover much material.  For students who have trouble grasping the implied rigor
in traditional material, writing a few simple proofs in the Metamath language
might help clarify fuzzy thought processes.  Although somewhat difficult at
first, it eventually becomes fun to do, like solving a puzzle, because of the
instant feedback provided by the computer.}

Metamath\index{Metamath} is probably unlike anything you have
encountered before.  In this first chapter we will look at the philosophy and
use of computers in mathematics in order to better understand the motivation
behind Metamath.  The material in this chapter is not required in order to use
Metamath.  You may skip it if you are impatient, but I hope you will find it
educational and enjoyable.  If you want to start experimenting with the
Metamath program right away, proceed directly to Chapter~\ref{using}
(p.~\pageref{using}).  To
learn the Metamath language, skim Chapter~\ref{using} then proceed to
Chapter~\ref{languagespec} (p.~\pageref{languagespec}).

\section{Mathematics as a Computer Language}

\begin{quote}
  {\em The study of mathematics is apt to commence in
dis\-ap\-point\-ment.\ldots \\
We are told that by its aid the stars are weighted
and the billions of molecules in a drop of water are counted.  Yet, like the
ghost of Hamlet's father, this great science eludes the efforts of our mental
weapons to grasp it.}
  \flushright\sc  Alfred North Whitehead\footnote{\cite{Whitehead}, ch.\ 1.}\\
\end{quote}\index{Whitehead, Alfred North}

\subsection{Is Mathematics ``User-Friendly''?}

Suppose you have no formal training in abstract mathematics.  But popular
books you've read offer tempting glimpses of this world filled with profound
ideas that have stirred the human spirit.  You are not satisfied with the
informal, watered-down descriptions you've read but feel it is important to
grasp the underlying mathematics itself to understand its true meaning. It's
not practical to go back to school to learn it, though; you don't want to
dedicate years of your life to it.  There are many important things in life,
and you have to set priorities for what's important to you.  What would happen
if you tried to pursue it on your own, in your spare time?

After all, you were able to learn a computer programming language such as
Pascal on your own without too much difficulty, even though you had no formal
training in computers.  You don't claim to be an expert in software design,
but you can write a passable program when necessary to suit your needs.  Even
more important, you know that you can look at anyone else's Pascal program, no
matter how complex, and with enough patience figure out exactly how it works,
even though you are not a specialist.  Pascal allows you do anything that a
computer can do, at least in principle.  Thus you know you have the ability,
in principle, to follow anything that a computer program can do:  you just
have to break it down into small enough pieces.

Here's an imaginary scenario of what might happen if you na\-ive\-ly a\-dopted
this same view of abstract mathematics and tried to pick it up on your own, in
a period of time comparable to, say, learning a computer programming
language.

\subsubsection{A Non-Mathematician's Quest for Truth}

\begin{quote}
  {\em \ldots my daughters have been studying (chemistry) for several
se\-mes\-ters, think they have learned differential and integral calculus in
school, and yet even today don't know why $x\cdot y=y\cdot x$ is true.}
  \flushright\sc  Edmund Landau\footnote{\cite{Landau}, p.~vi.}\\
\end{quote}\index{Landau, Edmund}

\begin{quote}
  {\em Minus times minus is plus,\\
The reason for this we need not discuss.}
  \flushright\sc W.\ H.\ Auden\footnote{As quoted in \cite{Guillen}, p.~64.}\\
\end{quote}\index{Auden, W.\ H.}\index{Guillen, Michael}

We'll suppose you are a technically oriented professional, perhaps an engineer, a
computer programmer, or a physicist, but probably not a mathematician.  You
consider yourself reasonably intelligent.  You did well in school, learning a
variety of methods and techniques in practical mathematics such as calculus and
differential equations.  But rarely did your courses get into anything
resembling modern abstract mathematics, and proofs were something that appeared
only occasionally in your textbooks, a kind of necessary evil that was
supposed to convince you of a certain key result.  Most of your
homework consisted of exercises that gave you practice in the techniques, and
you were hardly ever asked to come up with a proof of your own.

You find yourself curious about advanced, abstract mathematics.  You are
driven by an inner conviction that it is important to understand and
appreciate some of the most profound knowledge discovered by mankind.  But it
seems very hard to learn, something that only certain gifted longhairs can
access and understand.  You are frustrated that it seems forever cut off from
you.

Eventually your curiosity drives you to do something about it.
You set for yourself a goal of ``really'' understanding mathematics:  not just
how to manipulate equations in algebra or calculus according to cookbook
rules, but rather to gain a deep understanding of where those rules come from.
In fact, you're not thinking about this kind of ordinary mathematics at all,
but about a much more abstract, ethereal realm of pure mathematics, where
famous results such as G\"{o}del's incompleteness theorem\index{G\"{o}del's
incompleteness theorem} and Cantor's different kinds of infinities
reside.

You have probably read a number of popular books, with titles like {\em
Infinity and the Mind} \cite{Rucker}\index{Rucker, Rudy}, on topics such as
these.  You found them inspiring but at the same time somewhat
unsatisfactory.  They gave you a general idea of what these results are about,
but if someone asked you to prove them, you wouldn't have the faintest idea of
where to begin.   Sure, you could give the same overall outline that you
learned from the popular books; and in a general sort of way, you do have an
understanding.  But deep down inside, you know that there is a rigor that is
missing, that probably there are many subtle steps and pitfalls along the way,
and ultimately it seems you have to place your trust in the experts in the
field.  You don't like this; you want to be able to verify these results for
yourself.

So where do you go next?  As a first step, you decide to look up some of the
original papers on the theorems you are curious about, or better, obtain some
standard textbooks in the field.  You look up a theorem you want to
understand.  Sure enough, it's there, but it's expressed with strange
terms and odd symbols that mean absolutely nothing to you.  It might as well be written in
a foreign language you've never seen before, whose symbols are totally alien.
You look at the proof, and you haven't the foggiest notion what each step
means, much less how one step follows from another.  Well, obviously you have
a lot to learn if you want to understand this stuff.

You feel that you could probably understand it by
going back to college for another three to six years and getting a math
degree.  But that does not fit in with your career and the other things in
your life and would serve no practical purpose.  You decide to seek a quicker
path.  You figure you'll just trace your way back to the beginning, step by
step, as you would do with a computer program, until you understand it.  But
you quickly find that this is not possible, since you can't even understand
enough to know what you have to trace back to.

Maybe a different approach is in order---maybe you should start at the
beginning and work your way up.  First, you read the introduction to the book
to find out what the prerequisites are.  In a similar fashion, you trace your
way back through two or three more books, finally arriving at one that seems
to start at a beginning:  it lists the axioms of arithmetic.  ``Aha!'' you
naively think, ``This must be the starting point, the source of all mathematical
knowledge.'' Or at least the starting point for mathematics dealing with
numbers; you have to start somewhere and have no idea what the starting point
for other mathematics would be.  But the word ``axioms'' looks promising.  So
you eagerly read along and work through some elementary exercises at the
beginning of the book.  You feel vaguely bothered:  these
don't seem like axioms at all, at least not in the sense that you want to
think of axioms.  Axioms imply a starting point from which everything else can
be built up, according to precise rules specified in the axiom system.  Even
though you can understand the first few proofs in an informal way,
and are able to do some of the
exercises, it's hard to pin down precisely what the
rules are.   Sure, each step seems to follow logically from the others, but
exactly what does that mean?  Is the ``logic'' just a matter of common sense,
something vague that we all understand but can never quite state precisely?

You've spent a number of years, off and on, programming computers, and you
know that in the case of computer languages there is no question of what the
rules are---they are precise and crystal clear.  If you follow them, your
program will work, and if you don't, it won't.  No matter how complex a
program, it can always be broken down into simpler and simpler pieces, until
you can ultimately identify the bits that are moved around to perform a
specific function.  Some programs might require a lot of perseverance to
accomplish this, but if you focus on a specific portion of it, you don't even
necessarily have to know how the rest of it works. Shouldn't there be an
analogy in mathematics?

You decide to apply the ultimate test:  you ask yourself how a computer could
verify or ensure that the steps in these proofs follow from one another.
Certainly mathematics must be at least as precisely defined as a computer
language, if not more so; after all, computer science itself is based on it.
If you can get a computer to verify these proofs, then you should also be
able, in principle, to understand them yourself in a very crystal clear,
precise way.

You're in for a surprise:  you can conceive of no way to convert the
proofs, which are in English, to a form that the computer can understand.
The proofs are filled with phrases such as ``assume there exists a unique
$x$\ldots'' and ``given any $y$, let $z$ be the number such that\ldots''  This
isn't the kind of logic you are used to in computer programming, where
everything, even arithmetic, reduces to Boolean ones and zeroes if you care to
break it down sufficiently.  Even though you think you understand the proofs,
there seems to be some kind of higher reasoning involved rather than precise
rules that define how you manipulate the symbols in the axioms.  Whatever it
is, it just isn't obvious how you would express it to a computer, and the more
you think about it, the more puzzled and confused you get, to the point where
you even wonder whether {\em you} really understand it.  There's a lot more to
these axioms of arithmetic than meets the eye.

Nobody ever talked about this in school in your applied math and engineering
courses.  You just learned the rules they gave you, not quite understanding
how or why they worked, sometimes vaguely suspicious or uncertain of them, and
through homework problems and osmosis learned how to present solutions that
satisfied the instructor and earned you an ``A.''  Rarely did you actually
``prove'' anything in a rigorous way, and the math majors who did do stuff
like that seemed to be in a different world.

Of course, there are computer algebra programs that can do mathematics, and
rather impressively.  They can instantly solve the integrals that you
struggled with in freshman calculus, and do much, much more.  But when you
look at these programs, what you see is a big collection of algorithms and
techniques that evolved and were added to over time, along with some basic
software that manipulates symbols.  Each algorithm that is built in is the
result of someone's theorem whose proof is omitted; you just have to trust the
person who proved it and the person who programmed it in and hope there are no
bugs.\index{computer program bugs}  Somehow this doesn't seem to be the
essence of mathematics.  Although computer algebra systems can generate
theorems with amazing speed, they can't actually prove a single one of them.

After some puzzlement, you revisit some popular books on what mathematics is
all about.  Somewhere you read that all of mathematics is actually derived
from something called ``set theory.''  This is a little confusing, because
nowhere in the book that presented the axioms of arithmetic was there any
mention of set theory, or if there was, it seemed to be just a tool that helps
you describe things better---the set of even numbers, that sort of thing.  If
set theory is the basis for all mathematics, then why are additional axioms
needed for arithmetic?

Something is wrong but you're not sure what.  One of your friends is a pure
mathematician.  He knows he is unable to communicate to you what he does for a
living and seems to have little interest in trying.  You do know that for him,
proofs are what mathematics is all about. You ask him what a proof is, and he
essentially tells you that, while of course it's based on logic, really it's
something you learn by doing it over and over until you pick it up.  He refers
you to a book, {\em How to Read and Do Proofs} \cite{Solow}.\index{Solow,
Daniel}  Although this book helps you understand traditional informal proofs,
there is still something missing you can't seem to pin down yet.

You ask your friend how you would go about having a computer verify a proof.
At first he seems puzzled by the question; why would you want to do that?
Then he says it's not something that would make any sense to do, but he's
heard that you'd have to break the proof down into thousands or even millions
of individual steps to do such a thing, because the reasoning involved is at
such a high level of abstraction.  He says that maybe it's something you could
do up to a point, but the computer would be completely impractical once you
get into any meaningful mathematics.  There, the only way you can verify a
proof is by hand, and you can only acquire the ability to do this by
specializing in the field for a couple of years in grad school.  Anyway, he
thinks it all has to do with set theory, although he has never taken a formal
course in set theory but just learned what he needed as he went along.

You are intrigued and amazed.  Apparently a mathematician can grasp as a
single concept something that would take a computer a thousand or a million
steps to verify, and have complete confidence in it.  Each one of these
thousand or million steps must be absolutely correct, or else the whole proof
is meaningless.  If you added a million numbers by hand, would you trust the
result?  How do you really know that all these steps are correct, that there
isn't some subtle pitfall in one of these million steps, like a bug in a
computer program?\index{computer program bugs}  After all, you've read that
famous mathematicians have occasionally made mistakes, and you certainly know
you've made your share on your math homework problems in school.

You recall the analogy with a computer program.  Sure, you can understand what
a large computer program such as a word processor does, as a single high-level
concept or a small set of such concepts, but your ability to understand it in
no way ensures that the program is correct and doesn't have hidden bugs.  Even
if you wrote the program yourself you can't really know this; most large
programs that you've written have had bugs that crop up at some later date, no
matter how careful you tried to be while writing them.

OK, so now it seems the reason you can't figure out how to make a
computer verify proofs is because each step really corresponds to a
million small steps.  Well, you say, a computer can do a million
calculations in a second, so maybe it's still practical to do.  Now the
puzzle becomes how to figure out what the million steps are that each
English-language step corresponds to.  Your mathematician friend hasn't
a clue, but suggests that maybe you would find the answer by studying
set theory.  Actually, your friend thinks you're a little off the wall
for even wondering such a thing.  For him, this is not what mathematics
is all about.

The subject of set theory keeps popping up, so you decide it's
time to look it up.

You decide to start off on a careful footing, so you start reading a couple of
very elementary books on set theory.  A lot of it seems pretty obvious, like
intersections, subsets, and Venn diagrams.  You thumb through one of the
books; nowhere is anything about axioms mentioned. The other book relegates to
an appendix a brief discussion that mentions a set of axioms called
``Zermelo--Fraenkel set theory''\index{Zermelo--Fraenkel set theory} and states
them in English.  You look at them and have no idea what they really mean or
what you can do with them.  The comments in this appendix say that the purpose
of mentioning them is to expose you to the idea, but imply that they are not
necessary for basic understanding and that they are really the subject matter
of advanced treatments where fine points such as a certain paradox (Russell's
paradox\index{Russell's paradox}\footnote{Russell's paradox assumes that there
exists a set $S$ that is a collection of all sets that don't contain
themselves.  Now, either $S$ contains itself or it doesn't.  If it contains
itself, it contradicts its definition.  But if it doesn't contain itself, it
also contradicts its definition.  Russell's paradox is resolved in ZF set
theory by denying that such a set $S$ exists.}) are resolved.  Wait a
minute---shouldn't the axioms be a starting point, not an ending point?  If
there are paradoxes that arise without the axioms, how do you know you won't
stumble across one accidentally when using the informal approach?

And nowhere do these books describe how ``all of mathematics can be
derived from set theory'' which by now you've heard a few times.

You find a more advanced book on set theory.  This one actually lists the
axioms of ZF set theory in plain English on page one.  {\em Now} you think
your quest has ended and you've finally found the source of all mathematical
knowledge; you just have to understand what it means.  Here, in one place, is
the basis for all of mathematics!  You stare at the axioms in awe, puzzle over
them, memorize them, hoping that if you just meditate on them long enough they
will become clear.  Of course, you haven't the slightest idea how the rest of
mathematics is ``derived'' from them; in particular, if these are the axioms
of mathematics, then why do arithmetic, group theory, and so on need their own
axioms?

You start reading this advanced book carefully, pondering the meaning of every
word, because by now you're really determined to get to the bottom of this.
The first thing the book does is explain how the axioms came about, which was
to resolve Russell's paradox.\index{Russell's paradox}  In fact that seems to
be the main purpose of their existence; that they supposedly can be used to
derive all of mathematics seems irrelevant and is not even mentioned.  Well,
you go on.  You hope the book will explain to you clearly, step by step, how
to derive things from the axioms.  After all, this is the starting point of
mathematics, like a book that explains the basics of a computer programming
language.  But something is missing.  You find you can't even understand the
first proof or do the first exercise.  Symbols such as $\exists$ and $\forall$
permeate the page without any mention of where they came from or how to
manipulate them; the author assumes you are totally familiar with them and
doesn't even tell you what they mean.  By now you know that $\exists$ means
``there exists'' and $\forall$ means ``for all,'' but shouldn't the rules for
manipulating these symbols be part of the axioms?  You still have no idea
how you could even describe the axioms to a computer.

Certainly there is something much different here from the technical
literature you're used to reading.  A computer language manual almost
always explains very clearly what all the symbols mean, precisely what
they do, and the rules used for combining them, and you work your way up
from there.

After glancing at four or five other such books, you come to the realization
that there is another whole field of study that you need just to get to the
point at which you can understand the axioms of set theory.  The field is
called ``logic.''  In fact, some of the books did recommend it as a
prerequisite, but it just didn't sink in.  You assumed logic was, well, just
logic, something that a person with common sense intuitively understood.  Why
waste your time reading boring treatises on symbolic logic, the manipulation
of 1's and 0's that computers do, when you already know that?  But this is a
different kind of logic, quite alien to you.  The subject of {\sc nand} and
{\sc nor} gates is not even touched upon or in any case has to do with only a
very small part of this field.

So your quest continues.  Skimming through the first couple of introductory
books, you get a general idea of what logic is about and what quantifiers
(``for all,'' ``there exists'') mean, but you find their examples somewhat
trivial and mildly annoying (``all dogs are animals,'' ``some animals are
dogs,'' and such).  But all you want to know is what the rules are for
manipulating the symbols so you can apply them to set theory.  Some formulas
describing the relationships among quantifiers ($\exists$ and $\forall$) are
listed in tables, along with some verbal reasoning to justify them.
Presumably, if you want to find out if a formula is correct, you go through
this same kind of mental reasoning process, possibly using images of dogs and
animals. Intuitively, the formulas seem to make sense.  But when you ask
yourself, ``What are the rules I need to get a computer to figure out whether
this formula is correct?'', you still don't know.  Certainly you don't ask the
computer to imagine dogs and animals.

You look at some more advanced logic books.  Many of them have an introductory
chapter summarizing set theory, which turns out to be a prerequisite.  You
need logic to understand set theory, but it seems you also need set theory to
understand logic!  These books jump right into proving rather advanced
theorems about logic, without offering the faintest clue about where the logic
came from that allows them to prove these theorems.

Luckily, you come across an elementary book of logic that, halfway through,
after the usual truth tables and metaphors, presents in a clear, precise way
what you've been looking for all along: the axioms!  They're divided into
propositional calculus (also called sentential logic) and predicate calculus
(also called first-order logic),\index{first-order logic} with rules so simple
and crystal clear that now you can finally program a computer to understand
them.  Indeed, they're no harder than learning how to play a game of chess.
As far as what you seem to need is concerned, the whole book could have been
written in five pages!

{\em Now} you think you've found the ultimate source of mathematical
truth.  So---the axioms of mathematics consist of these axioms of logic,
together with the axioms of ZF set theory. (By now you've also been able to
figure out how to translate the ZF axioms from English into the
actual symbols of logic which you can now manipulate according to
precise, easy-to-understand rules.)

Of course, you still don't understand how ``all of mathematics can be
derived from set theory,'' but maybe this will reveal itself in due
course.

You eagerly set out to program the axioms and rules into a computer and start
to look at the theorems you will have to prove as the logic is developed.  All
sorts of important theorems start popping up:  the deduction
theorem,\index{deduction theorem} the substitution theorem,\index{substitution
theorem} the completeness theorem of propositional calculus,\index{first-order
logic!completeness} the completeness theorem of predicate calculus.  Uh-oh,
there seems to be trouble.  They all get harder and harder, and not one of
them can be derived with the axioms and rules of logic you've just been
handed.  Instead, they all require ``metalogic'' for their proofs, a kind of
mixture of logic and set theory that allows you to prove things {\em about}
the axioms and theorems of logic rather than {\em with} them.

You plow ahead anyway.  A month later, you've spent much of your
free time getting the computer to verify proofs in propositional calculus.
You've programmed in the axioms, but you've also had to program in the
deduction theorem, the substitution theorem, and the completeness theorem of
propositional calculus, which by now you've resigned yourself to treating as
rather complex additional axioms, since they can't be proved from the axioms
you were given.  You can now get the computer to verify and even generate
complete, rigorous, formal proofs\index{formal proof}.  Never mind that they
may have 100,000 steps---at least now you can have complete, absolute
confidence in them.  Unfortunately, the only theorems you have proved are
pretty trivial and you can easily verify them in a few minutes with truth
tables, if not by inspection.

It looks like your mathematician friend was right.  Getting the computer to do
serious mathematics with this kind of rigor seems almost hopeless.  Even
worse, it seems that the further along you get, the more ``axioms'' you have
to add, as each new theorem seems to involve additional ``metamathematical''
reasoning that hasn't been formalized, and none of it can be derived from the
axioms of logic.  Not only do the proofs keep growing exponentially as you get
further along, but the program to verify them keeps getting bigger and bigger
as you program in more ``metatheorems.''\index{metatheorem}\footnote{A
metatheorem is usually a statement that is too general to be directly provable
in a theory.  For example, ``if $n_1$, $n_2$, and $n_3$ are integers, then
$n_1+n_2+n_3$ is an integer'' is a theorem of number theory.  But ``for any
integer $k > 1$, if $n_1, \ldots, n_k$ are integers, then $n_1+\ldots +n_k$ is
an integer'' is a metatheorem, in other words a family of theorems, one for
every $k$.  The reason it is not a theorem is that the general sum $n_1+\ldots
+n_k$ (as a function of $k$) is not an operation that can be defined directly
in number theory.} The bugs\index{computer program bugs} that have cropped up
so far have already made you start to lose faith in the rigor you seem to have
achieved, and you know it's just going to get worse as your program gets larger.

\subsection{Mathematics and the Non-Specialist}

\begin{quote}
  {\em A real proof is not checkable by a machine, or even by any mathematician
not privy to the gestalt, the mode of thought of the particular field of
mathematics in which the proof is located.}
  \flushright\sc  Davis and Hersh\index{Davis, Phillip J.}
  \footnote{\cite{Davis}, p.~354.}\\
\end{quote}

The bulk of abstract or theoretical mathematics is ordinarily outside
the reach of anyone but a few specialists in each field who have completed
the necessary difficult internship in order to enter its coterie.  The
typical intelligent layperson has no reasonable hope of understanding much of
it, nor even the specialist mathematician of understanding other fields.  It
is like a foreign language that has no dictionary to look up the translation;
the only way you can learn it is by living in the country for a few years.  It
is argued that the effort involved in learning a specialty is a necessary
process for acquiring a deep understanding.  Of course, this is almost certainly
true if one is to make significant contributions to a field; in particular,
``doing'' proofs is probably the most important part of a mathematician's
training.  But is it also necessary to deny outsiders access to it?  Is it
necessary that abstract mathematics be so hard for a layperson to grasp?

A computer normally is of no help whatsoever.  Most published proofs are
actually just series of hints written in an informal style that requires
considerable knowledge of the field to understand.  These are the ``real
proofs'' referred to by Davis and Hersh.\index{informal proof}  There is an
implicit understanding that, in principle, such a proof could be converted to
a complete formal proof\index{formal proof}.  However, it is said that no one
would ever attempt such a conversion, even if they could, because that would
presumably require millions of steps (Section~\ref{dream}).  Unfortunately the
informal style automatically excludes the understanding of the proof
by anyone who hasn't gone through the necessary apprenticeship. The
best that the intelligent layperson can do is to read popular books about deep
and famous results; while this can be helpful, it can also be misleading, and
the lack of detail usually leaves the reader with no ability whatsoever to
explore any aspect of the field being described.

The statements of theorems often use sophisticated notation that makes them
inaccessible to the non-specialist.  For a non-specialist who wants to achieve
a deeper understanding of a proof, the process of tracing definitions and
lemmas back through their hierarchy\index{hierarchy} quickly becomes confusing
and discouraging.  Textbooks are usually written to train mathematicians or to
communicate to people who are already mathematicians, and large gaps in proofs
are often left as exercises to the reader who is left at an impasse if he or
she becomes stuck.

I believe that eventually computers will enable non-specialists and even
intelligent laypersons to follow almost any mathematical proof in any field.
Metamath is an attempt in that direction.  If all of mathematics were as
easily accessible as a computer programming language, I could envision
computer programmers and hobbyists who otherwise lack mathematical
sophistication exploring and being amazed by the world of theorems and proofs
in obscure specialties, perhaps even coming up with results of their own.  A
tremendous advantage would be that anyone could experiment with conjectures in
any field---the computer would offer instant feedback as to whether
an inference step was correct.

Mathematicians sometimes have to put up with the annoyance of
cranks\index{cranks} who lack a fundamental understanding of mathematics but
insist that their ``proofs'' of, say, Fermat's Last Theorem\index{Fermat's
Last Theorem} be taken seriously.  I think part of the problem is that these
people are misled by informal mathematical language, treating it as if they
were reading ordinary expository English and failing to appreciate the
implicit underlying rigor.  Such cranks are rare in the field of computers,
because computer languages are much more explicit, and ultimately the proof is
in whether a computer program works or not.  With easily accessible
computer-based abstract mathematics, a mathematician could say to a crank,
``don't bother me until you've demonstrated your claim on the computer!''

% 22-May-04 nm
% Attempt to move De Millo quote so it doesn't separate from attribution
% CHANGE THIS NUMBER (AND ELIMINATE IF POSSIBLE) WHEN ABOVE TEXT CHANGES
\vspace{-0.5em}

\subsection{An Impossible Dream?}\label{dream}

\begin{quote}
  {\em Even quite basic theorems would demand almost unbelievably vast
  books to display their proofs.}
    \flushright\sc  Robert E. Edwards\footnote{\cite{Edwards}, p.~68.}\\
\end{quote}\index{Edwards, Robert E.}

\begin{quote}
  {\em Oh, of course no one ever really {\em does} it.  It would take
  forever!  You just show that you could do it, that's sufficient.}
    \flushright\sc  ``The Ideal Mathematician''
    \index{Davis, Phillip J.}\footnote{\cite{Davis},
p.~40.}\\
\end{quote}

\begin{quote}
  {\em There is a theorem in the primitive notation of set theory that
  corresponds to the arithmetic theorem `$1000+2000=3000$'.  The formula
  would be forbiddingly long\ldots even if [one] knows the definitions
  and is asked to simplify the long formula according to them, chances are
  he will make errors and arrive at some incorrect result.}
    \flushright\sc  Hao Wang\footnote{\cite{Wang}, p.~140.}\\
\end{quote}\index{Wang, Hao}

% 22-May-04 nm
% Attempt to move De Millo quote so it doesn't separate from attribution
% CHANGE THIS NUMBER (AND ELIMINATE IF POSSIBLE) WHEN ABOVE TEXT CHANGES
\vspace{-0.5em}

\begin{quote}
  {\em The {\em Principia Mathematica} was the crowning achievement of the
  formalists.  It was also the deathblow of the formalist view.\ldots
  {[Rus\-sell]} failed, in three enormous volumes, to get beyond the elementary
  facts of arithmetic.  He showed what can be done in principle and what
  cannot be done in practice.  If the mathematical process were really
  one of strict, logical progression, we would still be counting our
  fingers.\ldots
  One theoretician estimates\ldots that a demonstration of one of
  Ramanujan's conjectures assuming set theory and elementary analysis would
  take about two thousand pages; the length of a deduction from first principles
  is nearly in\-con\-ceiv\-a\-ble\ldots The probabilists argue that\ldots any
  very long proof can at best be viewed as only probably correct\ldots}
  \flushright\sc Richard de Millo et. al.\footnote{\cite{deMillo}, pp.~269,
  271.}\\
\end{quote}\index{de Millo, Richard}

A number of writers have conveyed the impression that the kind of absolute
rigor provided by Metamath\index{Metamath} is an impossible dream, suggesting
that a complete, formal verification\index{formal proof} of a typical theorem
would take millions of steps in untold volumes of books.  Even if it could be
done, the thinking sometimes goes, all meaning would be lost in such a
monstrous, tedious verification.\index{informal proof}\index{proof length}

These writers assume, however, that in order to achieve the kind of complete
formal verification they desire one must break down a proof into individual
primitive steps that make direct reference to the axioms.  This is
not necessary.  There is no reason not to make use of previously proved
theorems rather than proving them over and over.

Just as important, definitions\index{definition} can be introduced along
the way, allowing very complex formulas to be represented with few
symbols.  Not doing this can lead to absurdly long formulas.  For
example, the mere statement of
G\"{o}del's incompleteness theorem\index{G\"{o}del's
incompleteness theorem}, which can be expressed with a small number of
defined symbols, would require about 20,000 primitive symbols to express
it.\index{Boolos, George S.}\footnote{George S.\ Boolos, lecture at
Massachusetts Institute of Technology, spring 1990.} An extreme example
is Bourbaki's\label{bourbaki} language for set theory, which requires
4,523,659,424,929 symbols plus 1,179,618,517,981 disambiguatory links
(lines connecting symbol pairs, usually drawn below or above the
formula) to express the number
``one'' \cite{Mathias}.\index{Mathias, Adrian R. D.}\index{Bourbaki,
Nicolas}
% http://www.dpmms.cam.ac.uk/~ardm/

A hierarchy\index{hierarchy} of theorems and definitions permits an
exponential growth in the formula sizes and primitive proof steps to be
described with only a linear growth in the number of symbols used.  Of course,
this is how ordinary informal mathematics is normally done anyway, but with
Metamath\index{Metamath} it can be done with absolute rigor and precision.

\subsection{Beauty}


\begin{quote}
  {\em No one shall be able to drive us from the paradise that Cantor has
created for us.}
   \flushright\sc  David Hilbert\footnote{As quoted in \cite{Moore}, p.~131.}\\
\end{quote}\index{Hilbert, David}

\needspace{3\baselineskip}
\begin{quote}
  {\em Mathematics possesses not only truth, but some supreme beauty ---a
  beauty cold and austere, like that of a sculpture.}
    \flushright\sc  Bertrand
    Russell\footnote{\cite{Russell}.}\\
\end{quote}\index{Russell, Bertrand}

\begin{quote}
  {\em Euclid alone has looked on Beauty bare.}
  \flushright\sc Edna Millay\footnote{As quoted in \cite{Davis}, p.~150.}\\
\end{quote}\index{Millay, Edna}

For most people, abstract mathematics is distant, strange, and
incomprehensible.  Many popular books have tried to convey some of the sense
of beauty in famous theorems.  But even an intelligent layperson is left with
only a general idea of what a theorem is about and is hardly given the tools
needed to make use of it.  Traditionally, it is only after years of arduous
study that one can grasp the concepts needed for deep understanding.
Metamath\index{Metamath} allows you to approach the proof of the theorem from
a quite different perspective, peeling apart the formulas and definitions
layer by layer until an entirely different kind of understanding is achieved.
Every step of the proof is there, pieced together with absolute precision and
instantly available for inspection through a microscope with a magnification
as powerful as you desire.

A proof in itself can be considered an object of beauty.  Constructing an
elegant proof is an art.  Once a famous theorem has been proved, often
considerable effort is made to find simpler and more easily understood
proofs.  Creating and communicating elegant proofs is a major concern of
mathematicians.  Metamath is one way of providing a common language for
archiving and preserving this information.

The length of a proof can, to a certain extent, be considered an
objective measure of its ``beauty,'' since shorter proofs are usually
considered more elegant.  In the set theory database
\texttt{set.mm}\index{set theory database (\texttt{set.mm})}%
\index{Metamath Proof Explorer}
provided with Metamath, one goal was to make all proofs as short as possible.

\needspace{4\baselineskip}
\subsection{Simplicity}

\begin{quote}
  {\em God made man simple; man's complex problems are of his own
  devising.}
    \flushright\sc Eccles. 7:29\footnote{Jerusalem Bible.}\\
\end{quote}\index{Bible}

\needspace{3\baselineskip}
\begin{quote}
  {\em God made integers, all else is the work of man.}
    \flushright\sc Leopold Kronecker\footnote{{\em Jahresbericht
	der Deutschen Mathematiker-Vereinigung }, vol. 2, p. 19.}\\
\end{quote}\index{Kronecker, Leopold}

\needspace{3\baselineskip}
\begin{quote}
  {\em For what is clear and easily comprehended attracts; the
  complicated repels.}
    \flushright\sc David Hilbert\footnote{As quoted in \cite{deMillo},
p.~273.}\\
\end{quote}\index{Hilbert, David}

The Metamath\index{Metamath} language is simple and Spartan.  Metamath treats
all mathematical expressions as simple sequences of symbols, devoid of meaning.
The higher-level or ``metamathematical'' notions underlying Metamath are about
as simple as they could possibly be.  Each individual step in a proof involves
a single basic concept, the substitution of an expression for a variable, so
that in principle almost anyone, whether mathematician or not, can
completely understand how it was arrived at.

In one of its most basic applications, Metamath\index{Metamath} can be used to
develop the foundations of mathematics\index{foundations of mathematics} from
the very beginning.  This is done in the set theory database that is provided
with the Metamath package and is the subject matter
of Chapter~\ref{fol}. Any language (a metalanguage\index{metalanguage})
used to describe mathematics (an object language\index{object language}) must
have a mathematical content of its own, but it is desirable to keep this
content down to a bare minimum, namely that needed to make use of the
inference rules specified by the axioms.  With any metalanguage there is a
``chicken and egg'' problem somewhat like circular reasoning:  you must assume
the validity of the mathematics of the metalanguage in order to prove the
validity of the mathematics of the object language.  The mathematical content
of Metamath itself is quite limited.  Like the rules of a game of chess, the
essential concepts are simple enough so that virtually anyone should be able to
understand them (although that in itself will not let you play like
a master).  The symbols that Metamath manipulates do not in themselves
have any intrinsic meaning.  Your interpretation of the axioms that you supply
to Metamath is what gives them meaning.  Metamath is an attempt to strip down
mathematical thought to its bare essence and show you exactly how the symbols
are manipulated.

Philosophers and logicians, with various motivations, have often thought it
important to study ``weak'' fragments of logic\index{weak logic}
\cite{Anderson}\index{Anderson, Alan Ross} \cite{MegillBunder}\index{Megill,
Norman}\index{Bunder, Martin}, other unconventional systems of logic (such as
``modal'' logic\index{modal logic} \cite[ch.\ 27]{Boolos}\index{Boolos, George
S.}), and quantum logic\index{quantum logic} in physics
\cite{Pavicic}\index{Pavi{\v{c}}i{\'{c}}, M.}.  Metamath\index{Metamath}
provides a framework in which such systems can be expressed, with an absolute
precision that makes all underlying metamathematical assumptions rigorous and
crystal clear.

Some schools of philosophical thought, for example
intuitionism\index{intuitionism} and constructivism\index{constructivism},
demand that the notions underlying any mathematical system be as simple and
concrete as possible.  Metamath should meet the requirements of these
philosophies.  Metamath must be taught the symbols, axioms\index{axiom}, and
rules\index{rule} for a specific theory, from the skeptical (such as
intuitionism\index{intuitionism}\footnote{Intuitionism does not accept the law
of excluded middle (``either something is true or it is not true'').  See
\cite[p.~xi]{Tymoczko}\index{Tymoczko, Thomas} for discussion and references
on this topic.  Consider the theorem, ``There exist irrational numbers $a$ and
$b$ such that $a^b$ is rational.''  An intuitionist would reject the following
proof:  If $\sqrt{2}^{\sqrt{2}}$ is rational, we are done.  Otherwise, let
$a=\sqrt{2}^{\sqrt{2}}$ and $b=\sqrt{2}$. Then $a^b=2$, which is rational.})
to the bold (such as the axiom of choice in set theory\footnote{The axiom of
choice\index{Axiom of Choice} asserts that given any collection of pairwise
disjoint nonempty sets, there exists a set that has exactly one element in
common with each set of the collection.  It is used to prove many important
theorems in standard mathematics.  Some philosophers object to it because it
asserts the existence of a set without specifying what the set contains
\cite[p.~154]{Enderton}\index{Enderton, Herbert B.}.  In one foundation for
mathematics due to Quine\index{Quine, Willard Van Orman}, that has not been
otherwise shown to be inconsistent, the axiom of choice turns out to be false
\cite[p.~23]{Curry}\index{Curry, Haskell B.}.  The \texttt{show
trace{\char`\_}back} command of the Metamath program allows you to find out
whether the axiom of choice, or any other axiom, was assumed by a
proof.}\index{\texttt{show trace{\char`\_}back} command}).

The simplicity of the Metamath language lets the algorithm (computer program)
that verifies the validity of a Metamath proof be straightforward and
robust.  You can have confidence that the theorems it verifies really can be
derived from your axioms.

\subsection{Rigor}

\begin{quote}
  {\em Rigor became a goal with the Greeks\ldots But the efforts to
  pursue rigor to the utmost have led to an impasse in which there is
  no longer any agreement on what it really means.  Mathematics remains
  alive and vital, but only on a pragmatic basis.}
    \flushright\sc  Morris Kline\footnote{\cite{Kline}, p.~1209.}\\
\end{quote}\index{Kline, Morris}

Kline refers to a much deeper kind of rigor than that which we will discuss in
this section.  G\"{o}del's incompleteness theorem\index{G\"{o}del's
incompleteness theorem} showed that it is impossible to achieve absolute rigor
in standard mathematics because we can never prove that mathematics is
consistent (free from contradictions).\index{consistent theory}  If
mathematics is consistent, we will never know it, but must rely on faith.  If
mathematics is inconsistent, the best we can hope for is that some clever
future mathematician will discover the inconsistency.  In this case, the
axioms would probably be revised slightly to eliminate the inconsistency, as
was done in the case of Russell's paradox,\index{Russell's paradox} but the
bulk of mathematics would probably not be affected by such a discovery.
Russell's paradox, for example, did not affect most of the remarkable results
achieved by 19th-century and earlier mathematicians.  It mainly invalidated
some of Gottlob Frege's\index{Frege, Gottlob} work on the foundations of
mathematics in the late 1800's; in fact Frege's work inspired Russell's
discovery.  Despite the paradox, Frege's work contains important concepts that
have significantly influenced modern logic.  Kline's {\em Mathematics, The
Loss of Certainty} \cite{Klinel}\index{Kline, Morris} has an interesting
discussion of this topic.

What {\em can} be achieved with absolute certainty\index{certainty} is the
knowledge that if we assume the axioms are consistent and true, then the
results derived from them are true.  Part of the beauty of mathematics is that
it is the one area of human endeavor where absolute certainty can be achieved
in this sense.  A mathematical truth will remain such for eternity.  However,
our actual knowledge of whether a particular statement is a mathematical truth
is only as certain as the correctness of the proof that establishes it.  If
the proof of a statement is questionable or vague, we can't have absolute
confidence in the truth that the statement claims.

Let us look at some traditional ways of expressing proofs.

Except in the field of formal logic\index{formal logic}, almost all
traditional proofs in mathematics are really not proofs at all, but rather
proof outlines or hints as to how to go about constructing the proof.  Many
gaps\index{gaps in proofs} are left for the reader to fill in. There are
several reasons for this.  First, it is usually assumed in mathematical
literature that the person reading the proof is a mathematician familiar with
the specialty being described, and that the missing steps are obvious to such
a reader or at least that the reader is capable of filling them in.  This
attitude is fine for professional mathematicians in the specialty, but
unfortunately it often has the drawback of cutting off the rest of the world,
including mathematicians in other specialties, from understanding the proof.
We discussed one possible resolution to this on p.~\pageref{envision}.
Second, it is often assumed that a complete formal proof\index{formal proof}
would require countless millions of symbols (Section~\ref{dream}). This might
be true if the proof were to be expressed directly in terms of the axioms of
logic and set theory,\index{set theory} but it is usually not true if we allow
ourselves a hierarchy\index{hierarchy} of definitions and theorems to build
upon, using a notation that allows us to introduce new symbols, definitions,
and theorems in a precisely specified way.

Even in formal logic,\index{formal logic} formal proofs\index{formal proof}
that are considered complete still contain hidden or implicit information.
For example, a ``proof'' is usually defined as a sequence of
wffs,\index{well-formed formula (wff)}\footnote{A {\em wff} or well-formed
formula is a mathematical expression (string of symbols) constructed according
to some precise rules.  A formal mathematical system\index{formal system}
contains (1) the rules for constructing syntactically correct
wffs,\index{syntax rules} (2) a list of starting wffs called
axioms,\index{axiom} and (3) one or more rules prescribing how to derive new
wffs, called theorems\index{theorem}, from the axioms or previously derived
theorems.  An example of such a system is contained in
Metamath's\index{Metamath} set theory database, which defines a formal
system\index{formal system} from which all of standard mathematics can be
derived.  Section~\ref{startf} steps you through a complete example of a formal
system, and you may want to skim it now if you are unfamiliar with the
concept.} each of which is an axiom or follows from a rule applied to previous
wffs in the sequence.  The implicit part of the proof is the algorithm by
which a sequence of symbols is verified to be a valid wff, given the
definition of a wff.  The algorithm in this case is rather simple, but for a
computer to verify the proof,\index{automated proof verification} it must have
the algorithm built into its verification program.\footnote{It is possible, of
course, to specify wff construction syntax outside of the program itself
with a suitable input language (the Metamath language being an example), but
some proof-verification or theorem-proving programs lack the ability to extend
wff syntax in such a fashion.} If one deals exclusively with axioms and
elementary wffs, it is straightforward to implement such an algorithm.  But as
more and more definitions are added to the theory in order to make the
expression of wffs more compact, the algorithm becomes more and more
complicated.  A computer program that implements the algorithm becomes larger
and harder to understand as each definition is introduced, and thus more prone
to bugs.\index{computer program bugs}  The larger the program, the
more suspicious the mathematician may be about
the validity of its algorithms.  This is especially true because
computer programs are inherently hard to follow to begin with, and few people
enjoy verifying them manually in detail.

Metamath\index{Metamath} takes a different approach.  Metamath's ``knowledge''
is limited to the ability to substitute variables for expressions, subject to
some simple constraints.  Once the basic algorithm of Metamath is assumed to
be debugged, and perhaps independently confirmed, it
can be trusted once and for all.  The information that Metamath needs to
``understand'' mathematics is contained entirely in the body of knowledge
presented to Metamath.  Any errors in reasoning can only be errors in the
axioms or definitions contained in this body of knowledge.  As a
``constructive'' language\index{constructive language} Metamath has no
conditional branches or loops like the ones that make computer programs hard
to decipher; instead, the language can only build new sequences of symbols
from earlier sequences  of symbols.

The simplicity of the rules that underlie Metamath not only makes Metamath
easy to learn but also gives Metamath a great deal of flexibility. For
example, Metamath is not limited to describing standard first-order
logic\index{first-order logic}; higher-order logics\index{higher-order logic}
and fragments of logic\index{weak logic} can be described just as easily.
Metamath gives you the freedom to define whatever wff notation you prefer; it
has no built-in conception of the syntax of a wff.\index{well-formed formula
(wff)}  With suitable axioms and definitions, Metamath can even describe and
prove things about itself.\index{Metamath!self-description}  (John
Harrison\index{Harrison, John} discusses the ``reflection''
principle\index{reflection principle} involved in self-descriptive systems in
\cite{Harrison}.)

The flexibility of Metamath requires that its proofs specify a lot of detail,
much more than in an ordinary ``formal'' proof.\index{formal proof}  For
example, in an ordinary formal proof, a single step consists of displaying the
wff that constitutes that step.  In order for a computer program to verify
that the step is acceptable, it first must verify that the symbol sequence
being displayed is an acceptable wff.\index{automated proof verification} Most
proof verifiers have at least basic wff syntax built into their programs.
Metamath has no hard-wired knowledge of what constitutes a wff built into it;
instead every wff must be explicitly constructed based on rules defining wffs
that are present in a database.  Thus a single step in an ordinary formal
proof may be correspond to many steps in a Metamath proof. Despite the larger
number of steps, though, this does not mean that a Metamath proof must be
significantly larger than an ordinary formal proof. The reason is that since
we have constructed the wff from scratch, we know what the wff is, so there is
no reason to display it.  We only need to refer to a sequence of statements
that construct it.  In a sense, the display of the wff in an ordinary formal
proof is an implicit proof of its own validity as a wff; Metamath just makes
the proof explicit. (Section~\ref{proof} describes Metamath's proof notation.)

\section{Computers and Mathematicians}

\begin{quote}
  {\em The computer is important, but not to mathematics.}
    \flushright\sc  Paul Halmos\footnote{As quoted in \cite{Albers}, p.~121.}\\
\end{quote}\index{Halmos, Paul}

Pure mathematicians have traditionally been indifferent to computers, even to
the point of disdain.\index{computers and pure mathematics}  Computer science
itself is sometimes considered to fall in the mundane realm of ``applied''
mathematics, perhaps essential for the real world but intellectually unexciting
to those who seek the deepest truths in mathematics.  Perhaps a reason for this
attitude towards computers is that there is little or no computer software that
meets their needs, and there may be a general feeling that such software could
not even exist.  On the one hand, there are the practical computer algebra
systems, which can perform amazing symbolic manipulations in algebra and
calculus,\index{computer algebra system} yet can't prove the simplest
existence theorem, if the idea of a proof is present at all.  On the other
hand, there are specialized automated theorem provers that technically speaking
may generate correct proofs.\index{automated theorem proving}  But sometimes
their specialized input notation may be cryptic and their output perceived to
be long, inelegant, incomprehensible proofs.    The output
may be viewed with suspicion, since the program that generates it tends to be
very large, and its size increases the potential for bugs\index{computer
program bugs}.  Such a proof may be considered trustworthy only if
independently verified and ``understood'' by a human, but no one wants to
waste their time on such a boring, unrewarding chore.



\needspace{4\baselineskip}
\subsection{Trusting the Computer}

\begin{quote}
  {\em \ldots I continue to find the quasi-empirical interpretation of
  computer proofs to be the more plausible.\ldots Since not
  everything that claims to be a computer proof can be
  accepted as valid, what are the mathematical criteria for acceptable
  computer proofs?}
    \flushright\sc  Thomas Tymoczko\footnote{\cite{Tymoczko}, p.~245.}\\
\end{quote}\index{Tymoczko, Thomas}

In some cases, computers have been essential tools for proving famous
theorems.  But if a proof is so long and obscure that it can be verified in a
practical way only with a computer, it is vaguely felt to be suspicious.  For
example, proving the famous four-color theorem\index{four-color
theorem}\index{proof length} (``a map needs no more than four colors to
prevent any two adjacent countries from having the same color'') can presently
only be done with the aid of a very complex computer program which originally
required 1200 hours of computer time. There has been considerable debate about
whether such a proof can be trusted and whether such a proof is ``real''
mathematics \cite{Swart}\index{Swart, E. R.}.\index{trusting computers}

However, under normal circumstances even a skeptical mathematician would have a
great deal of confidence in the result of multiplying two numbers on a pocket
calculator, even though the precise details of what goes on are hidden from its
user.  Even the verification on a supercomputer that a huge number is prime is
trusted, especially if there is independent verification; no one bothers to
debate the philosophical significance of its ``proof,'' even though the actual
proof would be so large that it would be completely impractical to ever write
it down on paper.  It seems that if the algorithm used by the computer is
simple enough to be readily understood, then the computer can be trusted.

Metamath\index{Metamath} adopts this philosophy.  The simplicity of its
language makes it easy to learn, and because of its simplicity one can have
essentially absolute confidence that a proof is correct. All axioms, rules, and
definitions are available for inspection at any time because they are defined
by the user; there are no hidden or built-in rules that may be prone to subtle
bugs\index{computer program bugs}.  The basic algorithm at the heart of
Metamath is simple and fixed, and it can be assumed to be bug-free and robust
with a degree of confidence approaching certainty.
Independently written implementations of the Metamath verifier
can reduce any residual doubt on the part of a skeptic even further;
there are now over a dozen such implementations, written by many people.

\subsection{Trusting the Mathematician}\label{trust}

\begin{quote}
  {\em There is no Algebraist nor Mathematician so expert in his science, as
  to place entire confidence in any truth immediately upon his discovery of it,
  or regard it as any thing, but a mere probability.  Every time he runs over
  his proofs, his confidence encreases; but still more by the approbation of
  his friends; and is rais'd to its utmost perfection by the universal assent
  and applauses of the learned world.}
  \flushright\sc David Hume\footnote{{\em A Treatise of Human Nature}, as
  quoted in \cite{deMillo}, p.~267.}\\
\end{quote}\index{Hume, David}

\begin{quote}
  {\em Stanislaw Ulam estimates that mathematicians publish 200,000 theorems
  every year.  A number of these are subsequently contradicted or otherwise
  disallowed, others are thrown into doubt, and most are ignored.}
  \flushright\sc Richard de Millo et. al.\footnote{\cite{deMillo}, p.~269.}\\
\end{quote}\index{Ulam, Stanislaw}

Whether or not the computer can be trusted, humans  of course will occasionally
err. Only the most memorable proofs get independently verified, and of these
only a handful of truly great ones achieve the status of being ``known''
mathematical truths that are used without giving a second thought to their
correctness.

There are many famous examples of incorrect theorems and proofs in
mathematical literature.\index{errors in proofs}

\begin{itemize}
\item There have been thousands of purported proofs of Fermat's Last
Theorem\index{Fermat's Last Theorem} (``no integer solutions exist to $x^n +
y^n = z^n$ for $n > 2$''), by amateurs, cranks, and well-regarded
mathematicians \cite[p.~5]{Stark}\index{Stark, Harold M}.  Fermat wrote a note
in his copy of Bachet's {\em Diophantus} that he found ``a truly marvelous
proof of this theorem but this margin is too narrow to contain it''
\cite[p.~507]{Kramer}.  A recent, much publicized proof by Yoichi
Miyaoka\index{Miyaoka, Yoichi} was shown to be incorrect ({\em Science News},
April 9, 1988, p.~230).  The theorem was finally proved by Andrew
Wiles\index{Wiles, Andrew} ({\em Science News}, July 3, 1993, p.~5), but it
initially had some gaps and took over a year after its announcement to be
checked thoroughly by experts.  On Oct. 25, 1994, Wiles announced that the last
gap found in his proof had been filled in.
  \item In 1882, M. Pasch discovered that an axiom was omitted from Euclid's
formulation of geometry\index{Euclidean geometry}; without it, the proofs of
important theorems of Euclid are not valid.  Pasch's axiom\index{Pasch's
axiom} states that a line that intersects one side of a triangle must also
intersect another side, provided that it does not touch any of the triangle's
vertices.  The omission of Pasch's axiom went unnoticed for 2000
years \cite[p.~160]{Davis}, in spite of (one presumes) the thousands of
students, instructors, and mathematicians who studied Euclid.
  \item The first published proof of the famous Schr\"{o}der--Bernstein
theorem\index{Schr\"{o}der--Bernstein theorem} in set theory was incorrect
\cite[p.~148]{Enderton}\index{Enderton, Herbert B.}.  This theorem states
that if there exists a 1-to-1 function\footnote{A {\em set}\index{set} is any
collection of objects. A {\em function}\index{function} or {\em
mapping}\index{mapping} is a rule that assigns to each element of one set
(called the function's {\em domain}\index{domain}) an element from another
set.} from set $A$ into set $B$ and vice-versa, then sets $A$ and $B$ have
a 1-to-1 correspondence.  Although it sounds simple and obvious,
the standard proof is quite long and complex.
  \item In the early 1900's, Hilbert\index{Hilbert, David} published a
purported proof of the continuum hypothesis\index{continuum hypothesis}, which
was eventually established as unprovable by Cohen\index{Cohen, Paul} in 1963
\cite[p.~166]{Enderton}.  The continuum hypothesis states that no
infinity\index{infinity} (``transfinite cardinal number'')\index{cardinal,
transfinite} exists whose size (or ``cardinality''\index{cardinality}) is
between the size of the set of integers and the size of the set of real
numbers.  This hypothesis originated with German mathematician Georg
Cantor\index{Cantor, Georg} in the late 1800's, and his inability to prove it
is said to have contributed to mental illness that afflicted him in his later
years.
  \item An incorrect proof of the four-color theorem\index{four-color theorem}
was published by Kempe\index{Kempe, A. B.} in 1879
\cite[p.~582]{Courant}\index{Courant, Richard}; it stood for 11 years before
its flaw was discovered.  This theorem states that any map can be colored
using only four colors, so that no two adjacent countries have the same
color.  In 1976 the theorem was finally proved by the famous computer-assisted
proof of Haken, Appel, and Koch \cite{Swart}\index{Appel, K.}\index{Haken,
W.}\index{Koch, K.}.  Or at least it seems that way.  Mathematician
H.~S.~M.~Coxeter\index{Coxeter, H. S. M.} has doubts \cite[p.~58]{Davis}:  ``I
have a feeling that that is an untidy kind of use of the computers, and the more
you correspond with Haken and Appel, the more shaky you seem to be.''
  \item Many false ``proofs'' of the Poincar\'{e}
conjecture\index{Poincar\'{e} conjecture} have been proposed over the years.
This conjecture states that any object that mathematically behaves like a
three-dimensional sphere is a three-dimensional sphere topologically,
regardless of how it is distorted.  In March 1986, mathematicians Colin
Rourke\index{Rourke, Colin} and Eduardo R\^{e}go\index{R\^{e}go, Eduardo}
caused  a stir in the mathematical community by announcing that they had found
a proof; in November of that year the proof was found to be false \cite[p.
218]{PetersonI}.  It was finally proved in 2003 by Grigory Perelman
\label{poincare}\index{Szpiro, George}\index{Perelman, Grigory}\cite{Szpiro}.
 \end{itemize}

Many counterexamples to ``theorems'' in recent mathematical
literature related to Clifford algebras\index{Clifford algebras}
 have been found by Pertti
Lounesto (who passed away in 2002).\index{Lounesto, Pertti}
See the web page \url{http://mathforum.org/library/view/4933.html}.
% http://users.tkk.fi/~ppuska/mirror/Lounesto/counterexamples.htm

One of the purposes of Metamath\index{Metamath} is to allow proofs to be
expressed with absolute precision.  Developing a proof in the Metamath
language can be challenging, because Metamath will not permit even the
tiniest mistake.\index{errors in proofs}  But once the proof is created, its
correctness can be trusted immediately, without having to depend on the
process of peer review for confirmation.

\section{The Use of Computers in Mathematics}

\subsection{Computer Algebra Systems}

For the most part, you will find that Metamath\index{Metamath} is not a
practical tool for manipulating numbers.  (Even proving that $2 + 2 = 4$, if
you start with set theory, can be quite complex!)  Several commercial
mathematics packages are quite good at arithmetic, algebra, and calculus, and
as practical tools they are invaluable.\index{computer algebra system} But
they have no notion of proof, and cannot understand statements starting with
``there exists such and such...''.

Software packages such as Mathematica \cite{Wolfram}\index{Mathematica} do not
concern themselves with proofs but instead work directly with known results.
These packages primarily emphasize heuristic rules such as the substitution of
equals for equals to achieve simpler expressions or expressions in a different
form.  Starting with a rich collection of built-in rules and algorithms, users
can add to the collection by means of a powerful programming language.
However, results such as, say, the existence of a certain abstract object
without displaying the actual object cannot be expressed (directly) in their
languages.  The idea of a proof from a small set of axioms is absent.  Instead
this software simply assumes that each fact or rule you add to the built-in
collection of algorithms is valid.  One way to view the software is as a large
collection of axioms from which the software, with certain goals, attempts to
derive new theorems, for example equating a complex expression with a simpler
equivalent. But the terms ``theorem''\index{theorem} and
``proof,''\index{proof} for example, are not even mentioned in the index of
the user's manual for Mathematica.\index{Mathematica and proofs}  What is also
unsatisfactory from a philosophical point of view is that there is no way to
ensure the validity of the results other than by trusting the writer of each
application module or tediously checking each module by hand, similar to
checking a computer program for bugs.\index{computer program
bugs}\footnote{Two examples illustrate why the knowledge database of computer
algebra systems should sometimes be regarded with a certain caution.  If you
ask Mathematica (version 3.0) to \texttt{Solve[x\^{ }n + y\^{ }n == z\^{ }n , n]}
it will respond with \texttt{\{\{n-\char`\>-2\}, \{n-\char`\>-1\},
\{n-\char`\>1\}, \{n-\char`\>2\}\}}. In other words, Mathematica seems to
``know'' that Fermat's Last Theorem\index{Fermat's Last Theorem} is true!  (At
the time this version of Mathematica was released this fact was unknown.)  If
you ask Maple\index{Maple} to \texttt{solve(x\^{ }2 = 2\^{ }x)} then
\texttt{simplify(\{"\})}, it returns the solution set \texttt{\{2, 4\}}, apparently
unaware that $-0.7666647$\ldots is also a solution.} While of course extremely
valuable in applied mathematics, computer algebra systems tend to be of little
interest to the theoretical mathematician except as aids for exploring certain
specific problems.

Because of possible bugs, trusting the output of a computer algebra system for
use as theorems in a proof-verifier would defeat the latter's goal of rigor.
On the other hand, a fact such that a certain relatively large number is
prime, while easy for a computer algebra system to derive, might have a long,
tedious proof that could overwhelm a proof-verifier. One approach for linking
computer algebra systems to a proof-verifier while retaining the advantages of
both is to add a hypothesis to each such theorem indicating its source.  For
example, a constant {\sc maple} could indicate the theorem came from the Maple
package, and would mean ``assuming Maple is consistent, then\ldots''  This and
many other topics concerning the formalization of mathematics are discussed in
John Harrison's\index{Harrison, John} very interesting
PhD thesis~\cite{Harrison-thesis}.

\subsection{Automated Theorem Provers}\label{theoremprovers}

A mathematical theory is ``decidable''\index{decidable theory} if a mechanical
method or algorithm exists that is guaranteed to determine whether or not a
particular formula is a theorem.  Among the few theories that are decidable is
elementary geometry,\index{Euclidean geometry} as was shown by a classic
result of logician Alfred Tarski\index{Tarski, Alfred} in 1948
\cite{Tarski}.\footnote{Tarski's result actually applies to a subset of the
geometry discussed in elementary textbooks.  This subset includes most of what
would be considered elementary geometry but it is not powerful enough to
express, among other things, the notions of the circumference and area of a
circle.  Extending the theory in a way that includes notions such as these
makes the theory undecidable, as was also shown by Tarski.  Tarski's algorithm
is far too inefficient to implement practically on a computer.  A practical
algorithm for proving a smaller subset of geometry theorems (those not
involving concepts of ``order'' or ``continuity'') was discovered by Chinese
mathematician Wu Wen-ts\"{u}n in 1977 \cite{Chou}\index{Chou,
Shang-Ching}.}\index{Wen-ts{\"{u}}n, Wu}  But most theories, including
elementary arithmetic, are undecidable.  This fact contributes to keeping
mathematics alive and well, since many mathematicians believe
that they will never be
replaced by computers (if they believe Roger Penrose's argument that a
computer can never replace the brain \cite{Penrose}\index{Penrose, Roger}).
In fact,  elementary geometry is often considered a ``dead'' field
for the simple reason that it is decidable.

On the other hand, the undecidability of a theory does not mean that one cannot
use a computer to search for proofs, providing one is willing to give up if a
proof is not found after a reasonable amount of time.  The field of automated
theorem proving\index{automated theorem proving} specializes in pursuing such
computer searches.  Among the more successful results to date are those based
on an algorithm known as Robinson's resolution principle
\cite{Robinson}\index{Robinson's resolution principle}.

Automated theorem provers can be excellent tools for those willing to learn
how to use them.  But they are not widely used in mainstream pure
mathematics, even though they could probably be useful in many areas.  There
are several reasons for this.  Probably most important, the main goal in pure
mathematics is to arrive at results that are considered to be deep or
important; proving them is essential but secondary.  Usually, an automated
theorem prover cannot assist in this main goal, and by the time the main goal
is achieved, the mathematician may have already figured out the proof as a
by-product.  There is also a notational problem.  Mathematicians are used to
using very compact syntax where one or two symbols (heavily dependent on
context) can represent very complex concepts; this is part of the
hierarchy\index{hierarchy} they have built up to tackle difficult problems.  A
theorem prover on the other hand might require that a theorem be expressed in
``first-order logic,''\index{first-order logic} which is the logic on which
most of mathematics is ultimately based but which is not ordinarily used
directly because expressions can become very long.  Some automated theorem
provers are experimental programs, limited in their use to very specialized
areas, and the goal of many is simply research into the nature of automated
theorem proving itself.  Finally, much research remains to be done to enable
them to prove very deep theorems.  One significant result was a
computer proof by Larry Wos\index{Wos, Larry} and colleagues that every Robbins
algebra\index{Robbins algebra} is a Boolean  algebra\index{Boolean algebra}
({\em New York Times}, Dec. 10, 1996).\footnote{In 1933, E.~V.\
Huntington\index{Huntington, E. V.}
presented the following axiom system for
Boolean algebra with a unary operation $n$ and a binary operation $+$:
\begin{center}
    $x + y = y + x$ \\
    $(x + y) + z = x + (y + z)$ \\
    $n(n(x) + y) + n(n(x) + n(y)) = x$
\end{center}
Herbert Robbins\index{Robbins, Herbert}, a student of Huntington, conjectured
that the last equation can be replaced with a simpler one:
\begin{center}
    $n(n(x + y) + n(x + n(y))) = x$
\end{center}
Robbins and Huntington could not find a proof.  The problem was
later studied unsuccessfully by Tarski\index{Tarski, Alfred} and his
students, and it remained an unsolved problem until a
computer found the proof in 1996.  For more information on
the Robbins algebra problem see \cite{Wos}.}

How does Metamath\index{Metamath} relate to automated theorem provers?  A
theorem prover is primarily concerned with one theorem at a time (perhaps
tapping into a small database of known theorems) whereas Metamath is more like
a theorem archiving system, storing both the theorem and its proof in a
database for access and verification.  Metamath is one answer to ``what do you
do with the output of a theorem prover?''  and could be viewed as the
next step in the process.  Automated theorem provers could be useful tools for
helping develop its database.
Note that very long, automatically
generated proofs can make your database fat and ugly and cause Metamath's proof
verification to take a long time to run.  Unless you have a particularly good
program that generates very concise proofs, it might be best to consider the
use of automatically generated proofs as a quick-and-dirty approach, to be
manually rewritten at some later date.

The program {\sc otter}\index{otter@{\sc otter}}\footnote{\url{http://www.cs.unm.edu/\~mccune/otter/}.}, later succeeded by
prover9\index{prover9}\footnote{\url{https://www.cs.unm.edu/~mccune/mace4/}.},
have been historically influential.
The E prover\index{E prover}\footnote{\url{https://github.com/eprover/eprover}.}
is a maintained automated theorem prover
for full first-order logic with equality.
There are many other automated theorem provers as well.

If you want to combine automated theorem provers with Metamath
consider investigating
the book {\em Automated Reasoning:  Introduction and Applications}
\cite{Wos}\index{Wos, Larry}.  This book discusses
how to use {\sc otter} in a way that can
not only able to generate
relatively efficient proofs, it can even be instructed to search for
shorter proofs.  The effective use of {\sc otter} (and similar tools)
does require a certain
amount of experience, skill, and patience.  The axiom system used in the
\texttt{set.mm}\index{set theory database (\texttt{set.mm})} set theory
database can be expressed to {\sc otter} using a method described in
\cite{Megill}.\index{Megill, Norman}\footnote{To use those axioms with
{\sc otter}, they must be restated in a way that eliminates the need for
``dummy variables.''\index{dummy variable!eliminating} See the Comment
on p.~\pageref{nodd}.} When successful, this method tends to generate
short and clever proofs, but my experiments with it indicate that the
method will find proofs within a reasonable time only for relatively
easy theorems.  It is still fun to experiment with.

Reference \cite{Bledsoe}\index{Bledsoe, W. W.} surveys a number of approaches
people have explored in the field of automated theorem proving\index{automated
theorem proving}.

\subsection{Interactive Theorem Provers}\label{interactivetheoremprovers}

Finding proofs completely automatically is difficult, so there
are some interactive theorem provers that allow a human to guide the
computer to find a proof.
Examples include
HOL Light\index{HOL light}%
\footnote{\url{https://www.cl.cam.ac.uk/~jrh13/hol-light/}.},
Isabelle\index{Isabelle}%
\footnote{\url{http://www.cl.cam.ac.uk/Research/HVG/Isabelle}.},
{\sc hol}\index{hol@{\sc hol}}%
\footnote{\url{https://hol-theorem-prover.org/}.},
and
Coq\index{Coq}\footnote{\url{https://coq.inria.fr/}.}.

A major difference between most of these tools and Metamath is that the
``proofs'' are actually programs that guide the program to find a proof,
and not the proof itself.
For example, an Isabelle/HOL proof might apply a step
\texttt{apply (blast dest: rearrange reduction)}. The \texttt{blast}
instruction applies
an automatic tableux prover and returns if it found a sequence of proof
steps that work... but the sequence is not considered part of the proof.

A good overview of
higher-level proof verification languages (such as {\sc lcf}\index{lcf@{\sc
lcf}} and {\sc hol}\index{hol@{\sc hol}})
is given in \cite{Harrison}.  All of these languages are fundamentally
different from Metamath in that much of the mathematical foundational
knowledge is embedded in the underlying proof-verification program, rather
than placed directly in the database that is being verified.
These can have a steep learning curve for those without a mathematical
background.  For example, one usually must have a fair understanding of
mathematical logic in order to follow their proofs.

\subsection{Proof Verifiers}\label{proofverifiers}

A proof verifier is a program that doesn't generate proofs but instead
verifies proofs that you give it.  Many proof verifiers have limited built-in
automated proof capabilities, such as figuring out simple logical inferences
(while still being guided by a person who provides the overall proof).  Metamath
has no built-in automated proof capability other than the limited
capability of its Proof Assistant.

Proof-verification languages are not used as frequently as they might be.
Pure mathematicians are more concerned with producing new results, and such
detail and rigor would interfere with that goal.  The use of computers in pure
mathematics is primarily focused on automated theorem provers (not verifiers),
again with the ultimate goal of aiding the creation of new mathematics.
Automated theorem provers are usually concerned with attacking one theorem at
time rather than making a large, organized database easily available to the
user.  Metamath is one way to help close this gap.

By itself Metamath is a mostly a proof verifier.
This does not mean that other approaches can't be used; the difference
is that in Metamath, the results of various provers must be recorded
step-by-step so that they can be verified.

Another proof-verification language is Mizar,\index{Mizar} which can display
its proofs in the informal language that mathematicians are accustomed to.
Information on the Mizar language is available at \url{http://mizar.org}.

For the working mathematician, Mizar is an excellent tool for rigorously
documenting proofs. Mizar typesets its proofs in the informal English used by
mathematicians (and, while fine for them, are just as inscrutable by
laypersons!). A price paid for Mizar is a relatively steep learning curve of a
couple of weeks.  Several mathematicians are actively formalizing different
areas of mathematics using Mizar and publishing the proofs in a dedicated
journal. Unfortunately the task of formalizing mathematics is still looked
down upon to a certain extent since it doesn't involve the creation of ``new''
mathematics.

The closest system to Metamath is
the {\em Ghilbert}\index{Ghilbert} proof language (\url{http://ghilbert.org})
system developed by
Raph Levien\index{Levien, Raph}.
Ghilbert is a formal proof checker heavily inspired by Metamath.
Ghilbert statements are s-expressions (a la Lisp), which is easy
for computers to parse but many people find them hard to read.
There are a number of differences in their specific constructs, but
there is at least one tool to translate some Metamath materials into Ghilbert.
As of 2019 the Ghilbert community is smaller and less active than the
Metamath community.
That said, the Metamath and Ghilbert communities overlap, and fruitful
conversations between them have occurred many times over the years.

\subsection{Creating a Database of Formalized Mathematics}\label{mathdatabase}

Besides Metamath, there are several other ongoing projects with the goal of
formalizing mathematics into computer-verifiable databases.
Understanding some history will help.

The {\sc qed}\index{qed project@{\sc qed} project}%
\footnote{\url{http://www-unix.mcs.anl.gov/qed}.}
project arose in 1993 and its goals were outlined in the
{\sc qed} manifesto.
The {\sc qed} manifesto was
a proposal for a computer-based database of all mathematical knowledge,
strictly formalized and with all proofs having been checked automatically.
The project had a conference in 1994 and another in 1995;
there was also a ``twenty years of the {\sc qed} manifesto'' workshop
in 2014.
Its ideals are regularly reraised.

In a 2007 paper, Freek Wiedijk identified two reasons
for the failure of the {\sc qed} project as originally envisioned:%
\cite{Wiedijk-revisited}\index{Wiedijk, Freek}

\begin{itemize}
\item Very few people are working on formalization of mathematics. There is no compelling application for fully mechanized mathematics.
\item Formalized mathematics does not yet resemble traditional mathematics. This is partly due to the complexity of mathematical notation, and partly to the limitations of existing theorem provers and proof assistants.
\end{itemize}

But this did not end the dream of
formalizing mathematics into computer-verifiable databases.
The problems that led to the {\sc qed} manifesto are still with us,
even though the challenges were harder than originally considered.
What has happened instead is that various independent projects have
worked towards formalizing mathematics into computer-verifiable databases,
each simultaneously competing and cooperating with each other.

A concrete way to see this is
Freek Wiedijk's ``Formalizing 100 Theorems'' list%
\footnote{\url{http://www.cs.ru.nl/\%7Efreek/100/}.}
which shows the progress different systems have made on a challenge list
of 100 mathematical theorems.%
\footnote{ This is not the only list of ``interesting'' theorems.
Another interesting list was posted by Oliver Knill's list
\cite{Knill}\index{Knill, Oliver}.}
The top systems as of February 2019
(in order of the number of challenges completed) are
HOL Light, Isabelle, Metamath, Coq, and Mizar.

The Metamath 100%
\footnote{\url{http://us.metamath.org/mm\_100.html}}
page (maintained by David A. Wheeler\index{Wheeler, David A.})
shows the progress of Metamath (specifically its \texttt{set.mm} database)
against this challenge list maintained by Freek Wiedijk.
The Metamath \texttt{set.mm} database
has made a lot of progress over the years,
in part because working to prove those challenge theorems required
defining various terms and proving their properties as a prerequisite.
Here are just a few of the many statements that have been
formally proven with Metamath:

% The entries of this cause the narrow display to break poorly,
% since the short amount of text means LaTeX doesn't get a lot to work with
% and the itemize format gives it even *less* margin than usual.
% No one will mind if we make just this list flushleft, since this list
% will be internally consistent.
\begin{flushleft}
\begin{itemize}
\item 1. The Irrationality of the Square Root of 2
  (\texttt{sqr2irr}, by Norman Megill, 2001-08-20)
\item 2. The Fundamental Theorem of Algebra
  (\texttt{fta}, by Mario Carneiro, 2014-09-15)
\item 22. The Non-Denumerability of the Continuum
  (\texttt{ruc}, by Norman Megill, 2004-08-13)
\item 54. The Konigsberg Bridge Problem
  (\texttt{konigsberg}, by Mario Carneiro, 2015-04-16)
\item 83. The Friendship Theorem
  (\texttt{friendship}, by Alexander W. van der Vekens, 2018-10-09)
\end{itemize}
\end{flushleft}

We thank all of those who have developed at least one of the Metamath 100
proofs, and we particularly thank
Mario Carneiro\index{Carneiro, Mario}
who has contributed the most Metamath 100 proofs as of 2019.
The Metamath 100 page shows the list of all people who have contributed a
proof, and links to graphs and charts showing progress over time.
We encourage others to work on proving theorems not yet proven in Metamath,
since doing so improves the work as a whole.

Each of the math formalization systems (including Metamath)
has different strengths and weaknesses, depending on what you value.
Key aspects that differentiate Metamath from the other top systems are:

\begin{itemize}
\item Metamath is not tied to any particular set of axioms.
\item Metamath can show every step of every proof, no exceptions.
  Most other provers only assert that a proof can be found, and do not
  show every step. This also makes verification fast, because
  the system does not need to rediscover proof details.
\item The Metamath verifier has been re-implemented in many different
  programming languages, so verification can be done by multiple
  implementations.  In particular, the
  \texttt{set.mm}\index{set theory database (\texttt{set.mm})}%
  \index{Metamath Proof Explorer} database is verified by
  four different verifiers
  written in four different languages by four different authors.
  This greatly reduces the risk of accepting an invalid
  proof due to an error in the verifier.
\item Proofs stay proven.  In some systems, changes to the system's
  syntax or how a tactic works causes proofs to fail in later versions,
  causing older work to become essentially lost.
  Metamath's language is
  extremely small and fixed, so once a proof is added to a database,
  the database can be rechecked with later versions of the Metamath program
  and with other verifiers of Metamath databases.
  If an axiom or key definition needs to be changed, it is easy to
  manipulate the database as a whole to handle the change
  without touching the underlying verifier.
  Since re-verification of an entire database takes seconds, there
  is never a reason to delay complete verification.
  This aspect is especially compelling if your
  goal is to have a long-term database of proofs.
\item Licensing is generous.  The main Metamath databases are released to
  the public domain, and the main Metamath program is open source software
  under a standard, widely-used license.
\item Substitutions are easy to understand, even by those who are not
  professional mathematicians.
\end{itemize}

Of course, other systems may have advantages over Metamath
that are more compelling, depending on what you value.
In any case, we hope this helps you understand Metamath
within a wider context.

\subsection{In Summary}\label{computers-summary}

To summarize our discussions of computers and mathematics, computer algebra
systems can be viewed as theorem generators focusing on a narrow realm of
mathematics (numbers and their properties), automated theorem provers as proof
generators for specific theorems in a much broader realm covered by a built-in
formal system such as first-order logic, interactive theorem
provers require human guidance, proof verifiers verify proofs but
historically they have been
restricted to first-order logic.
Metamath, in contrast,
is a proof verifier and documenter whose realm is essentially unlimited.

\section{Mathematics and Metamath}

\subsection{Standard Mathematics}

There are a number of ways that Metamath\index{Metamath} can be used with
standard mathematics.  The most satisfying way philosophically is to start at
the very beginning, and develop the desired mathematics from the axioms of
logic and set theory.\index{set theory}  This is the approach taken in the
\texttt{set.mm}\index{set theory database (\texttt{set.mm})}%
\index{Metamath Proof Explorer}
database (also known as the Metamath Proof Explorer).
Among other things, this database builds up to the
axioms of real and complex numbers\index{analysis}\index{real and complex
numbers} (see Section~\ref{real}), and a standard development of analysis, for
example, could start at that point, using it as a basis.   Besides this
philosophical advantage, there are practical advantages to having all of the
tools of set theory available in the supporting infrastructure.

On the other hand, you may wish to start with the standard axioms of a
mathematical theory without going through the set theoretical proofs of those
axioms.  You will need mathematical logic to make inferences, but if you wish
you can simply introduce theorems\index{theorem} of logic as
``axioms''\index{axiom} wherever you need them, with the implicit assumption
that in principle they can be proved, if they are obvious to you.  If you
choose this approach, you will probably want to review the notation used in
\texttt{set.mm}\index{set theory database (\texttt{set.mm})} so that your own
notation will be consistent with it.

\subsection{Other Formal Systems}
\index{formal system}

Unlike some programs, Metamath\index{Metamath} is not limited to any specific
area of mathematics, nor committed to any particular mathematical philosophy
such as classical logic versus intuitionism, nor limited, say, to expressions
in first-order logic.  Although the database \texttt{set.mm}
describes standard logic and set theory, Meta\-math
is actually a general-purpose language for describing a wide variety of formal
systems.\index{formal system}  Non-standard systems such as modal
logic,\index{modal logic} intuitionist logic\index{intuitionism}, higher-order
logic\index{higher-order logic}, quantum logic\index{quantum logic}, and
category theory\index{category theory} can all be described with the Metamath
language.  You define the symbols you prefer and tell Metamath the axioms and
rules you want to start from, and Metamath will verify any inferences you make
from those axioms and rules.  A simple example of a non-standard formal system
is Hofstadter's\index{Hofstadter, Douglas R.} MIU system,\index{MIU-system}
whose Metamath description is presented in Appendix~\ref{MIU}.

This is not hypothetical.
The largest Metamath database is
\texttt{set.mm}\index{set theory database (\texttt{set.mm}}%
\index{Metamath Proof Explorer}), aka the Metamath Proof Explorer,
which uses the most common axioms for mathematical foundations
(specifically classical logic combined with Zermelo--Fraenkel
set theory\index{Zermelo--Fraenkel set theory} with the Axiom of Choice).
But other Metamath databases are available:

\begin{itemize}
\item The database
  \texttt{iset.mm}\index{intuitionistic logic database (\texttt{iset.mm})},
  aka the
  Intuitionistic Logic Explorer\index{Intuitionistic Logic Explorer},
  uses intuitionistic logic (a constructivist point of view)
  instead of classical logic.
\item The database
  \texttt{nf.mm}\index{New Foundations database (\texttt{nf.mm})},
  aka the
  New Foundations Explorer\index{New Foundations Explorer},
  constructs mathematics from scratch,
  starting from Quine's New Foundations (NF) set theory axioms.
\item The database
  \texttt{hol.mm}\index{Higher-order Logic database (\texttt{hol.mm})},
  aka the
  Higher-Order Logic (HOL) Explorer\index{Higher-Order Logic (HOL) Explorer},
  starts with HOL (also called simple type theory) and derives
  equivalents to ZFC axioms, connecting the two approaches.
\end{itemize}

Since the days of David Hilbert,\index{Hilbert, David} mathematicians have
been concerned with the fact that the metalanguage\index{metalanguage} used to
describe mathematics may be stronger than the mathematics being described.
Metamath\index{Metamath}'s underlying finitary\index{finitary proof},
constructive nature provides a good philosophical basis for studying even the
weakest logics.\index{weak logic}

The usual treatment of many non-standard formal systems\index{formal
system} uses model theory\index{model theory} or proof theory\index{proof
theory} to describe these systems; these theories, in turn, are based on
standard set theory.  In other words, a non-standard formal system is defined
as a set with certain properties, and standard set theory is used to derive
additional properties of this set.  The standard set theory database provided
with Metamath can be used for this purpose, and when used this way
the development of a special
axiom system for the non-standard formal system becomes unnecessary.  The
model- or proof-theoretic approach often allows you to prove much deeper
results with less effort.

Metamath supports both approaches.  You can define the non-standard
formal system directly, or define the non-standard formal system as
a set with certain properties, whichever you find most helpful.

%\section{Additional Remarks}

\subsection{Metamath and Its Philosophy}

Closely related to Metamath\index{Metamath} is a philosophy or way of looking
at mathematics. This philosophy is related to the formalist
philosophy\index{formalism} of Hilbert\index{Hilbert, David} and his followers
\cite[pp.~1203--1208]{Kline}\index{Kline, Morris}
\cite[p.~6]{Behnke}\index{Behnke, H.}. In this philosophy, mathematics is
viewed as nothing more than a set of rules that manipulate symbols, together
with the consequences of those rules.  While the mathematics being described
may be complex, the rules used to describe it (the
``metamathematics''\index{metamathematics}) should be as simple as possible.
In particular, proofs should be restricted to dealing with concrete objects
(the symbols we write on paper rather than the abstract concepts they
represent) in a constructive manner; these are called ``finitary''
proofs\index{finitary proof} \cite[pp.~2--3]{Shoenfield}\index{Shoenfield,
Joseph R.}.

Whether or not you find Metamath interesting or useful will in part depend on
the appeal you find in its philosophy, and this appeal will probably depend on
your particular goals with respect to mathematics.  For example, if you are a
pure mathematician at the forefront of discovering new mathematical knowledge,
you will probably find that the rigid formality of Metamath stifles your
creativity.  On the other hand, we would argue that once this knowledge is
discovered, there are advantages to documenting it in a standard format that
will make it accessible to others.  Sixty years from now, your field may be
dormant, and as Davis and Hersh put it, your ``writings would become less
translatable than those of the Maya'' \cite[p.~37]{Davis}\index{Davis, Phillip
J.}.


\subsection{A History of the Approach Behind Metamath}

Probably the one work that has had the most motivating influence on
Metamath\index{Metamath} is Whitehead and Russell's monumental {\em Principia
Mathematica} \cite{PM}\index{Whitehead, Alfred North}\index{Russell,
Bertrand}\index{principia mathematica@{\em Principia Mathematica}}, whose aim
was to deduce all of mathematics from a small number of primitive ideas, in a
very explicit way that in principle anyone could understand and follow.  While
this work was tremendously influential in its time, from a modern perspective
it suffers from several drawbacks.  Both its notation and its underlying
axioms are now considered dated and are no longer used.  From our point of
view, its development is not really as accessible as we would like to see; for
practical reasons, proofs become more and more sketchy as its mathematics
progresses, and working them out in fine detail requires a degree of
mathematical skill and patience that many people don't have.  There are
numerous small errors, which is understandable given the tedious, technical
nature of the proofs and the lack of a computer to verify the details.
However, even today {\em Principia Mathematica} stands out as the work closest
in spirit to Metamath.  It remains a mind-boggling work, and one can't help
but be amazed at seeing ``$1+1=2$'' finally appear on page 83 of Volume II
(Theorem *110.643).

The origin of the proof notation used by Metamath dates back to the 1950's,
when the logician C.~A.~Meredith expressed his proofs in a compact notation
called ``condensed detachment''\index{condensed detachment}
\cite{Hindley}\index{Hindley, J. Roger} \cite{Kalman}\index{Kalman, J. A.}
\cite{Meredith}\index{Meredith, C. A.} \cite{Peterson}\index{Peterson, Jeremy
George}.  This notation allows proofs to be communicated unambiguously by
merely referencing the axiom\index{axiom}, rule\index{rule}, or
theorem\index{theorem} used at each step, without explicitly indicating the
substitutions\index{substitution!variable}\index{variable substitution} that
have to be made to the variables in that axiom, rule, or theorem.  Ordinarily,
condensed detachment is more or less limited to propositional
calculus\index{propositional calculus}.  The concept has been extended to
first-order logic\index{first-order logic} in \cite{Megill}\index{Megill,
Norman}, making it is easy to write a small computer program to verify proofs
of simple first-order logic theorems.\index{condensed detachment!and
first-order logic}

A key concept behind the notation of condensed detachment is called
``unification,''\index{unification} which is an algorithm for determining what
substitutions\index{substitution!variable}\index{variable substitution} to
variables have to be made to make two expressions match each other.
Unification was first precisely defined by the logician J.~A.~Robinson, who
used it in the development of a powerful
theorem-proving technique called the ``resolution principle''
\cite{Robinson}\index{Robinson's resolution principle}. Metamath does not make
use of the resolution principle, which is intended for systems of first-order
logic.\index{first-order logic}  Metamath's use is not restricted to
first-order logic, and as we have mentioned it does not automatically discover
proofs.  However, unification is a key idea behind Metamath's proof
notation, and Metamath makes use of a very simple version of it
(Section~\ref{unify}).

\subsection{Metamath and First-Order Logic}

First-order logic\index{first-order logic} is the supporting structure
for standard mathematics.  On top of it is set theory, which contains
the axioms from which virtually all of mathematics can be derived---a
remarkable fact.\index{category
theory}\index{cardinal, inaccessible}\label{categoryth}\footnote{An exception seems
to be category theory.  There are several schools of thought on whether
category theory is derivable from set theory.  At a minimum, it appears
that an additional axiom is needed that asserts the existence of an
``inaccessible cardinal'' (a type of infinity so large that standard set
theory can't prove or deny that it exists).
%
%%%% (I took this out that was in previous editions:)
% But it is also argued that not just one but a ``proper class'' of them
% is needed, and the existence of proper classes is impossible in standard
% set theory.  (A proper class is a collection of sets so huge that no set
% can contain it as an element.  Proper classes can lead to
% inconsistencies such as ``Russell's paradox.''  The axioms of standard
% set theory are devised so as to deny the existence of proper classes.)
%
For more information, see
\cite[pp.~328--331]{Herrlich}\index{Herrlich, Horst} and
\cite{Blass}\index{Blass, Andrea}.}

One of the things that makes Metamath\index{Metamath} more practical for
first-order theories is a set of axioms for first-order logic designed
specifically with Metamath's approach in mind.  These are included in
the database \texttt{set.mm}\index{set theory database (\texttt{set.mm})}.
See Chapter~\ref{fol} for a detailed
description; the axioms are shown in Section~\ref{metaaxioms}.  While
logically equivalent to standard axiom systems, our axiom system breaks
up the standard axioms into smaller pieces such that from them, you can
directly derive what in other systems can only be derived as higher-level
``metatheorems.''\index{metatheorem}  In other words, it is more powerful than
the standard axioms from a metalogical point of view.  A rigorous
justification for this system and its ``metalogical
completeness''\index{metalogical completeness} is found in
\cite{Megill}\index{Megill, Norman}.  The system is closely related to a
system developed by Monk\index{Monk, J. Donald} and Tarski\index{Tarski,
Alfred} in 1965 \cite{Monks}.

For example, the formula $\exists x \, x = y $ (given $y$, there exists some
$x$ equal to it) is a theorem of logic,\footnote{Specifically, it is a theorem
of those systems of logic that assume non-empty domains.  It is not a theorem
of more general systems that include the empty domain\index{empty domain}, in
which nothing exists, period!  Such systems are called ``free
logics.''\index{free logic} For a discussion of these systems, see
\cite{Leblanc}\index{Leblanc, Hugues}.  Since our use for logic is as a basis
for set theory, which has a non-empty domain, it is more convenient (and more
traditional) to use a less general system.  An interesting curiosity is that,
using a free logic as a basis for Zermelo--Fraenkel set
theory\index{Zermelo--Fraenkel set theory} (with the redundant Axiom of the
Null Set omitted),\index{Axiom of the Null Set} we cannot even prove the
existence of a single set without assuming the axiom of infinity!\index{Axiom
of Infinity}} whether or not $x$ and $y$ are distinct variables\index{distinct
variables}.  In many systems of logic, we would have to prove two theorems to
arrive at this result.  First we would prove ``$\exists x \, x = x $,'' then
we would separately prove ``$\exists x \, x = y $, where $x$ and $y$ are
distinct variables.''  We would then combine these two special cases ``outside
of the system'' (i.e.\ in our heads) to be able to claim, ``$\exists x \, x =
y $, regardless of whether $x$ and $y$ are distinct.''  In other words, the
combination of the two special cases is a
metatheorem.  In the system of logic
used in Metamath's set theory\index{set theory database (\texttt{set.mm})}
database, the axioms of logic are broken down into small pieces that allow
them to be reassembled in such a way that theorems such as these can be proved
directly.

Breaking down the axioms in this way makes them look peculiar and not very
intuitive at first, but rest assured that they are correct and complete.  Their
correctness is ensured because they are theorem schemes of standard first-order
logic (which you can easily verify if you are a logician).  Their completeness
follows from the fact that we explicitly derive the standard axioms of
first-order logic as theorems.  Deriving the standard axioms is somewhat
tricky, but once we're there, we have at our disposal a system that is less
awkward to work with in formal proofs\index{formal proof}.  In technical terms
that logicians understand, we eliminate the cumbersome concepts of ``free
variable,''\index{free variable} ``bound variable,''\index{bound variable} and
``proper substitution''\index{proper substitution}\index{substitution!proper}
as primitive notions.  These concepts are present in our system but are
defined in terms of concepts expressed by the axioms and can be eliminated in
principle.  In standard systems, these concepts are really like additional,
implicit axioms\index{implicit axiom} that are somewhat complex and cannot be
eliminated.

The traditional approach to logic, wherein free variables and proper
substitution is defined, is also possible to do directly in the Metamath
language.  However, the notation tends to become awkward, and there are
disadvantages:  for example, extending the definition of a wff with a
definition is awkward, because the free variable and proper substitution
concepts have to have their definitions also extended.  Our choice of
axioms for \texttt{set.mm} is to a certain extent a matter of style, in
an attempt to achieve overall simplicity, but you should also be aware
that the traditional approach is possible as well if you should choose
it.

\chapter{Using the Metamath Program}
\label{using}

\section{Installation}

The way that you install Metamath\index{Metamath!installation} on your
computer system will vary for different computers.  Current
instructions are provided with the Metamath program download at
\url{http://metamath.org}.  In general, the installation is simple.
There is one file containing the Metamath program itself.  This file is
usually called \texttt{metamath} or \texttt{metamath.}{\em xxx} where
{\em xxx} is the convention (such as \texttt{exe}) for an executable
program on your operating system.  There are several additional files
containing samples of the Metamath language, all ending with
\texttt{.mm}.  The file \texttt{set.mm}\index{set theory database
(\texttt{set.mm})} contains logic and set theory and can be used as a
starting point for other areas of mathematics.

You will also need a text editor\index{text editor} capable of editing plain
{\sc ascii}\footnote{American Standard Code for Information Interchange.} text
in order to prepare your input files.\index{ascii@{\sc ascii}}  Most computers
have this capability built in.  Note that plain text is not necessarily the
default for some word processing programs\index{word processor}, especially if
they can handle different fonts; for example, with Microsoft Word\index{Word
(Microsoft)}, you must save the file in the format ``Text Only With Line
Breaks'' to get a plain text\index{plain text} file.\footnote{It is
recommended that all lines in a Metamath source file be 79 characters or less
in length for compatibility among different computer terminals.  When creating
a source file on an editor such as Word, select a monospaced
font\index{monospaced font} such as Courier\index{Courier font} or
Monaco\index{Monaco font} to make this easier to achieve.  Better yet,
just use a plain text editor such as Notepad.}

On some computer systems, Metamath does not have the capability to print
its output directly; instead, you send its output to a file (using the
\texttt{open} commands described later).  The way you print this output
file depends on your computer.\index{printers} Some computers have a
print command, whereas with others, you may have to read the file into
an editor and print it from there.

If you want to print your Metamath source files with typeset formulas
containing standard mathematical symbols, you will need the \LaTeX\
typesetting program\index{latex@{\LaTeX}}, which is widely and freely
available for most operating systems.  It runs natively on Unix and
Linux, and can be installed on Windows as part of the free Cygwin
package (\url{http://cygwin.com}).

You can also produce {\sc html}\footnote{HyperText Markup Language.}
web pages.  The {\tt help html} command in the Metamath program will
assist you with this feature.

\section{Your First Formal System}\label{start}
\subsection{From Nothing to Zero}\label{startf}

To give you a feel for what the Metamath\index{Metamath} language looks like,
we will take a look at a very simple example from formal number
theory\index{number theory}.  This example is taken from
Mendelson\index{Mendelson, Elliot} \cite[p. 123]{Mendelson}.\footnote{To keep
the example simple, we have changed the formalism slightly, and what we call
axioms\index{axiom} are strictly speaking theorems\index{theorem} in
\cite{Mendelson}.}  We will look at a small subset of this theory, namely that
part needed for the first number theory theorem proved in \cite{Mendelson}.

First we will look at a standard formal proof\index{formal proof} for the
example we have picked, then we will look at the Metamath version.  If you
have never been exposed to formal proofs, the notation may seem to be such
overkill to express such simple notions that you may wonder if you are missing
something.  You aren't.  The concepts involved are in fact very simple, and a
detailed breakdown in this fashion is necessary to express the proof in a way
that can be verified mechanically.  And as you will see, Metamath breaks the
proof down into even finer pieces so that the mechanical verification process
can be about as simple as possible.

Before we can introduce the axioms\index{axiom} of the theory, we must define
the syntax rules for forming legal expressions\index{syntax rules}
(combinations of symbols) with which those axioms can be used. The number 0 is
a {\bf term}\index{term}; and if $ t$ and $r$ are terms, so is $(t+r)$. Here,
$ t$ and $r$ are ``metavariables''\index{metavariable} ranging over terms; they
themselves do not appear as symbols in an actual term.  Some examples of
actual terms are $(0 + 0)$ and $((0+0)+0)$.  (Note that our theory describes
only the number zero and sums of zeroes.  Of course, not much can be done with
such a trivial theory, but remember that we have picked a very small subset of
complete number theory for our example.  The important thing for you to focus
on is our definitions that describe how symbols are combined to form valid
expressions, and not on the content or meaning of those expressions.) If $ t$
and $r$ are terms, an expression of the form $ t=r$ is a {\bf wff}
(well-formed formula)\index{well-formed formula (wff)}; and if $P$ and $Q$ are
wffs, so is $(P\rightarrow Q)$ (which means ``$P$ implies
$Q$''\index{implication ($\rightarrow$)} or ``if $P$ then $Q$'').
Here $P$ and $Q$ are metavariables ranging over wffs.  Examples of actual
wffs are $0=0$, $(0+0)=0$, $(0=0 \rightarrow (0+0)=0)$, and $(0=0\rightarrow
(0=0\rightarrow 0=(0+0)))$.  (Our notation makes use of more parentheses than
are customary, but the elimination of ambiguity this way simplifies our
example by avoiding the need to define operator precedence\index{operator
precedence}.)

The {\bf axioms}\index{axiom} of our theory are all wffs of the following
form, where $ t$, $r$, and $s$ are any terms:

%Latex p. 92
\renewcommand{\theequation}{A\arabic{equation}}

\begin{equation}
(t=r\rightarrow (t=s\rightarrow r=s))
\end{equation}
\begin{equation}
(t+0)=t
\end{equation}

Note that there are an infinite number of axioms since there are an infinite
number of possible terms.  A1 and A2 are properly called ``axiom
schemes,''\index{axiom scheme} but we will refer to them as ``axioms'' for
brevity.

An axiom is a {\bf theorem}; and if $P$ and $(P\rightarrow Q)$ are theorems
(where $P$ and $Q$ are wffs), then $Q$ is also a theorem.\index{theorem}  The
second part of this definition is called the modus ponens (MP) rule of
inference\index{inference rule}\index{modus ponens}.  It allows us to obtain
new theorems from old ones.

The {\bf proof}\index{proof} of a theorem is a sequence of one or more
theorems, each of which is either an axiom or the result of modus ponens
applied to two previous theorems in the sequence, and the last of which is the
theorem being proved.

The theorem we will prove for our example is very simple:  $ t=t$.  The proof of
our theorem follows.  Study it carefully until you feel sure you
understand it.\label{zeroproof}

% Use tabu so that lines will wrap automatically as needed.
\begin{tabu} { l X X }
1. & $(t+0)=t$ & (by axiom A2) \\
2. & $(t+0)=t$ & (by axiom A2) \\
3. & $((t+0)=t \rightarrow ((t+0)=t\rightarrow t=t))$ & (by axiom A1) \\
4. & $((t+0)=t\rightarrow t=t)$ & (by MP applied to steps 2 and 3) \\
5. & $t=t$ & (by MP applied to steps 1 and 4) \\
\end{tabu}

(You may wonder why step 1 is repeated twice.  This is not necessary in the
formal language we have defined, but in Metamath's ``reverse Polish
notation''\index{reverse Polish notation (RPN)} for proofs, a previous step
can be referred to only once.  The repetition of step~1 here will enable you
to see more clearly the correspondence of this proof with the
Metamath\index{Metamath} version on p.~\pageref{demoproof}.)

Our theorem is more properly called a ``theorem scheme,''\index{theorem
scheme} for it represents an infinite number of theorems, one for each
possible term $ t$.  Two examples of actual theorems would be $0=0$ and
$(0+0)=(0+0)$.  Rarely do we prove actual theorems, since by proving schemes
we can prove an infinite number of theorems in one fell swoop.  Similarly, our
proof should really be called a ``proof scheme.''\index{proof scheme}  To
obtain an actual proof, pick an actual term to use in place of $ t$, and
substitute it for $ t$ throughout the proof.

Let's discuss what we have done here.  The axioms\index{axiom} of our theory,
A1 and A2, are trivial and obvious.  Everyone knows that adding zero to
something doesn't change it, and also that if two things are equal to a third,
then they are equal to each other. In fact, stating the trivial and obvious is
a goal to strive for in any axiomatic system.  From trivial and obvious truths
that everyone agrees upon, we can prove results that are not so obvious yet
have absolute faith in them.  If we trust the axioms and the rules, we must,
by definition, trust the consequences of those axioms and rules, if logic is
to mean anything at all.

Our rule of inference\index{rule}, modus ponens\index{modus ponens}, is also
pretty obvious once you understand what it means.  If we prove a fact $P$, and
we also prove that $P$ implies $Q$, then $Q$ necessarily follows as a new
fact.  The rule provides us with a means for obtaining new facts (i.e.\
theorems\index{theorem}) from old ones.

The theorem that we have proved, $ t=t$, is so fundamental that you may wonder
why it isn't one of the axioms\index{axiom}.  In some axiom systems of
arithmetic, it {\em is} an axiom.  The choice of axioms in a theory is to some
extent arbitrary and even an art form, constrained only by the requirement
that any two equivalent axiom systems be able to derive each other as
theorems.  We could imagine that the inventor of our axiom system originally
included $ t=t$ as an axiom, then discovered that it could be derived as a
theorem from the other axioms.  Because of this, it was not necessary to
keep it as an axiom.  By eliminating it, the final set of axioms became
that much simpler.

Unless you have worked with formal proofs\index{formal proof} before, it
probably wasn't apparent to you that $ t=t$ could be derived from our two
axioms until you saw the proof. While you certainly believe that $ t=t$ is
true, you might not be able to convince an imaginary skeptic who believes only
in our two axioms until you produce the proof.  Formal proofs such as this are
hard to come up with when you first start working with them, but after you get
used to them they can become interesting and fun.  Once you understand the
idea behind formal proofs you will have grasped the fundamental principle that
underlies all of mathematics.  As the mathematics becomes more sophisticated,
its proofs become more challenging, but ultimately they all can be broken down
into individual steps as simple as the ones in our proof above.

Mendelson's\index{Mendelson, Elliot} book, from which our example was taken,
contains a number of detailed formal proofs such as these, and you may be
interested in looking it up.  The book is intended for mathematicians,
however, and most of it is rather advanced.  Popular literature describing
formal proofs\index{formal proof} include \cite[p.~296]{Rucker}\index{Rucker,
Rudy} and \cite[pp.~204--230]{Hofstadter}\index{Hofstadter, Douglas R.}.

\subsection{Converting It to Metamath}\label{convert}

Formal proofs\index{formal proof} such as the one in our example break down
logical reasoning into small, precise steps that leave little doubt that the
results follow from the axioms\index{axiom}.  You might think that this would
be the finest breakdown we can achieve in mathematics.  However, there is more
to the proof than meets the eye. Although our axioms were rather simple, a lot
of verbiage was needed before we could even state them:  we needed to define
``term,'' ``wff,'' and so on.  In addition, there are a number of implied
rules that we haven't even mentioned. For example, how do we know that step 3
of our proof follows from axiom A1? There is some hidden reasoning involved in
determining this.  Axiom A1 has two occurrences of the letter $ t$.  One of
the implied rules states that whatever we substitute for $ t$ must be a legal
term\index{term}.\footnote{Some authors make this implied rule explicit by
stating, ``only expressions of the above form are terms,'' after defining
``term.''}  The expression $ t+0$ is pretty obviously a legal term whenever $
t$ is, but suppose we wanted to substitute a huge term with thousands of
symbols?  Certainly a lot of work would be involved in determining that it
really is a term, but in ordinary formal proofs all of this work would be
considered a single ``step.''

To express our axiom system in the Metamath\index{Metamath} language, we must
describe this auxiliary information in addition to the axioms themselves.
Metamath does not know what a ``term'' or a ``wff''\index{well-formed formula
(wff)} is.  In Metamath, the specification of the ways in which we can combine
symbols to obtain terms and wffs are like little axioms in themselves.  These
auxiliary axioms are expressed in the same notation as the ``real''
axioms\index{axiom}, and Metamath does not distinguish between the two.  The
distinction is made by you, i.e.\ by the way in which you interpret the
notation you have chosen to express these two kinds of axioms.

The Metamath language breaks down mathematical proofs into tiny pieces, much
more so than in ordinary formal proofs\index{formal proof}.  If a single
step\index{proof step} involves the
substitution\index{substitution!variable}\index{variable substitution} of a
complex term for one of its variables, Metamath must see this single step
broken down into many small steps.  This fine-grained breakdown is what gives
Metamath generality and flexibility as it lets it not be limited to any
particular mathematical notation.

Metamath's proof notation is not, in itself, intended to be read by humans but
rather is in a compact format intended for a machine.  The Metamath program
will convert this notation to a form you can understand, using the \texttt{show
proof}\index{\texttt{show proof} command} command.  You can tell the program what
level of detail of the proof you want to look at.  You may want to look at
just the logical inference steps that correspond
to ordinary formal proof steps,
or you may want to see the fine-grained steps that prove that an expression is
a term.

Here, without further ado, is our example converted to the
Metamath\index{Metamath} language:\index{metavariable}\label{demo0}

\begin{verbatim}
$( Declare the constant symbols we will use $)
    $c 0 + = -> ( ) term wff |- $.
$( Declare the metavariables we will use $)
    $v t r s P Q $.
$( Specify properties of the metavariables $)
    tt $f term t $.
    tr $f term r $.
    ts $f term s $.
    wp $f wff P $.
    wq $f wff Q $.
$( Define "term" and "wff" $)
    tze $a term 0 $.
    tpl $a term ( t + r ) $.
    weq $a wff t = r $.
    wim $a wff ( P -> Q ) $.
$( State the axioms $)
    a1 $a |- ( t = r -> ( t = s -> r = s ) ) $.
    a2 $a |- ( t + 0 ) = t $.
$( Define the modus ponens inference rule $)
    ${
       min $e |- P $.
       maj $e |- ( P -> Q ) $.
       mp  $a |- Q $.
    $}
$( Prove a theorem $)
    th1 $p |- t = t $=
  $( Here is its proof: $)
       tt tze tpl tt weq tt tt weq tt a2 tt tze tpl
       tt weq tt tze tpl tt weq tt tt weq wim tt a2
       tt tze tpl tt tt a1 mp mp
     $.
\end{verbatim}\index{metavariable}

A ``database''\index{database} is a set of one or more {\sc ascii} source
files.  Here's a brief description of this Metamath\index{Metamath} database
(which consists of this single source file), so that you can understand in
general terms what is going on.  To understand the source file in detail, you
should read Chapter~\ref{languagespec}.

The database is a sequence of ``tokens,''\index{token} which are normally
separated by spaces or line breaks.  The only tokens that are built into
the Metamath language are those beginning with \texttt{\$}.  These tokens
are called ``keywords.''\index{keyword}  All other tokens are
user-defined, and their names are arbitrary.

As you might have guessed, the Metamath token \texttt{\$(}\index{\texttt{\$(} and
\texttt{\$)} auxiliary keywords} starts a comment and \texttt{\$)} ends a comment.

The Metamath tokens \texttt{\$c}\index{\texttt{\$c} statement},
\texttt{\$v}\index{\texttt{\$v} statement},
\texttt{\$e}\index{\texttt{\$e} statement},
\texttt{\$f}\index{\texttt{\$f} statement},
\texttt{\$a}\index{\texttt{\$a} statement}, and
\texttt{\$p}\index{\texttt{\$p} statement} specify ``statements'' that
end with \texttt{\$.}\,.\index{\texttt{\$.}\ keyword}

The Metamath tokens \texttt{\$c} and \texttt{\$v} each declare\index{constant
declaration}\index{variable declaration} a list of user-defined tokens, called
``math symbols,''\index{math symbol} that the database will reference later
on.  All of the math symbols they define you have seen earlier except the
turnstile symbol \texttt{|-} ($\vdash$)\index{turnstile ({$\,\vdash$})}, which is
commonly used by logicians to mean ``a proof exists for.''  For us
the turnstile is just a
convenient symbol that distinguishes expressions that are axioms\index{axiom}
or theorems\index{theorem} from expressions that are terms or wffs.

The \texttt{\$c} statement declares ``constants''\index{constant} and
the \texttt{\$v} statement declares
``variables''\index{variable}\index{constant declaration}\index{variable
declaration} (or more precisely, metavariables\index{metavariable}).  A
variable may be substituted\index{substitution!variable}\index{variable
substitution} with sequences of math symbols whereas a constant may not
be substituted with anything.

It may seem redundant to require both \texttt{\$c}\index{\texttt{\$c} statement} and
\texttt{\$v}\index{\texttt{\$v} statement} statements (since any math
symbol\index{math symbol} not specified with a \texttt{\$c} statement could be
presumed to be a variable), but this provides for better error checking and
also allows math symbols to be redeclared\index{redeclaration of symbols}
(Section~\ref{scoping}).

The token \texttt{\$f}\index{\texttt{\$f} statement} specifies a
statement called a ``variable-type hypothesis'' (also called a
``floating hypothesis'') and \texttt{\$e}\index{\texttt{\$e} statement}
specifies a ``logical hypothesis'' (also called an ``essential
hypothesis'').\index{hypothesis}\index{variable-type
hypothesis}\index{logical hypothesis}\index{floating
hypothesis}\index{essential hypothesis} The token
\texttt{\$a}\index{\texttt{\$a} statement} specifies an ``axiomatic
assertion,''\index{axiomatic assertion} and
\texttt{\$p}\index{\texttt{\$p} statement} specifies a ``provable
assertion.''\index{provable assertion} To the left of each occurrence of
these four tokens is a ``label''\index{label} that identifies the
hypothesis or assertion for later reference.  For example, the label of
the first axiomatic assertion is \texttt{tze}.  A \texttt{\$f} statement
must contain exactly two math symbols, a constant followed by a
variable.  The \texttt{\$e}, \texttt{\$a}, and \texttt{\$p} statements
each start with a constant followed by, in general, an arbitrary
sequence of math symbols.

Associated with each assertion\index{assertion} is a set of hypotheses
that must be satisfied in order for the assertion to be used in a proof.
These are called the ``mandatory hypotheses''\index{mandatory
hypothesis} of the assertion.  Among those hypotheses whose ``scope''
(described below) includes the assertion, \texttt{\$e} hypotheses are
always mandatory and \texttt{\$f}\index{\texttt{\$f} statement}
hypotheses are mandatory when they share their variable with the
assertion or its \texttt{\$e} hypotheses.  The exact rules for
determining which hypotheses are mandatory are described in detail in
Sections~\ref{frames} and \ref{scoping}.  For example, the mandatory
hypotheses of assertion \texttt{tpl} are \texttt{tt} and \texttt{tr},
whereas assertion \texttt{tze} has no mandatory hypotheses because it
contains no variables and has no \texttt{\$e}\index{\texttt{\$e}
statement} hypothesis.  Metamath's \texttt{show statement}
command\index{\texttt{show statement} command}, described in the next
section, will show you a statement's mandatory hypotheses.

Sometimes we need to make a hypothesis relevant to only certain
assertions.  The set of statements to which a hypothesis is relevant is
called its ``scope.''  The Metamath brackets,
\texttt{\$\char`\{}\index{\texttt{\$\char`\{} and \texttt{\$\char`\}}
keywords} and \texttt{\$\char`\}}, define a ``block''\index{block} that
delimits the scope of any hypothesis contained between them.  The
assertion \texttt{mp} has mandatory hypotheses \texttt{wp}, \texttt{wq},
\texttt{min}, and \texttt{maj}.  The only mandatory hypothesis of
\texttt{th1}, on the other hand, is \texttt{tt}, since \texttt{th1}
occurs outside of the block containing \texttt{min} and \texttt{maj}.

Note that \texttt{\$\char`\{} and \texttt{\$\char`\}} do not affect the
scope of assertions (\texttt{\$a} and \texttt{\$p}).  Assertions are always
available to be referenced by any later proof in the source file.

Each provable assertion (\texttt{\$p}\index{\texttt{\$p} statement}
statement) has two parts.  The first part is the
assertion\index{assertion} itself, which is a sequence of math
symbol\index{math symbol} tokens placed between the \texttt{\$p} token
and a \texttt{\$=}\index{\texttt{\$=} keyword} token.  The second part
is a ``proof,'' which is a list of label tokens placed between the
\texttt{\$=} token and the \texttt{\$.}\index{\texttt{\$.}\ keyword}\
token that ends the statement.\footnote{If you've looked at the
\texttt{set.mm} database, you may have noticed another notation used for
proofs.  The other notation is called ``compressed.''\index{compressed
proof}\index{proof!compressed} It reduces the amount of space needed to
store a proof in the database and is described in
Appendix~\ref{compressed}.  In the example above, we use
``normal''\index{normal proof}\index{proof!normal} notation.} The proof
acts as a series of instructions to the Metamath program, telling it how
to build up the sequence of math symbols contained in the assertion part of
the \texttt{\$p} statement, making use of the hypotheses of the
\texttt{\$p} statement and previous assertions.  The construction takes
place according to precise rules.  If the list of labels in the proof
causes these rules to be violated, or if the final sequence that results
does not match the assertion, the Metamath program will notify you with
an error message.

If you are familiar with reverse Polish notation (RPN), which is sometimes used
on pocket calculators, here in a nutshell is how a proof works.  Each
hypothesis label\index{hypothesis label} in the proof is pushed\index{push}
onto the RPN stack\index{stack}\index{RPN stack} as it is encountered. Each
assertion label\index{assertion label} pops\index{pop} off the stack as many
entries as the referenced assertion has mandatory hypotheses.  Variable
substitutions\index{substitution!variable}\index{variable substitution} are
computed which, when made to the referenced assertion's mandatory hypotheses,
cause these hypotheses to match the stack entries. These same substitutions
are then made to the variables in the referenced assertion itself, which is
then pushed onto the stack.  At the end of the proof, there should be one
stack entry, namely the assertion being proved.  This process is explained in
detail in Section~\ref{proof}.

Metamath's proof notation is not very readable for humans, but it allows the
proof to be stored compactly in a file.  The Metamath\index{Metamath} program
has proof display features that let you see what's going on in a more
readable way, as you will see in the next section.

The rules used in verifying a proof are not based on any built-in syntax of the
symbol sequence in an assertion\index{assertion} nor on any built-in meanings
attached to specific symbol names.  They are based strictly on symbol
matching:  constants\index{constant} must match themselves, and
variables\index{variable} may be replaced with anything that allows a match to
occur.  For example, instead of \texttt{term}, \texttt{0}, and \verb$|-$ we could
have just as well used \texttt{yellow}, \texttt{zero}, and \texttt{provable}, as long
as we did so consistently throughout the database.  Also, we could have used
\texttt{is provable} (two tokens) instead of \verb$|-$ (one token) throughout the
database.  In each of these cases, the proof would be exactly the same.  The
independence of proofs and notation means that you have a lot of flexibility to
change the notation you use without having to change any proofs.

\section{A Trial Run}\label{trialrun}

Now you are ready to try out the Metamath\index{Metamath} program.

On all computer systems, Metamath has a standard ``command line
interface'' (CLI)\index{command line interface (CLI)} that allows you to
interact with it.  You supply commands to the CLI by typing them on the
keyboard and pressing your keyboard's {\em return} key after each line
you enter.  The CLI is designed to be easy to use and has built-in help
features.

The first thing you should do is to use a text editor to create a file
called \texttt{demo0.mm} and type into it the Metamath source shown on
p.~\pageref{demo0}.  Actually, this file is included with your Metamath
software package, so check that first.  If you type it in, make sure
that you save it in the form of ``plain {\sc ascii} text with line
breaks.''  Most word processors will have this feature.

Next you must run the Metamath program.  Depending on your computer
system and how Metamath is installed, this could range from clicking the
mouse on the Metamath icon to typing \texttt{run metamath} to typing
simply \texttt{metamath}.  (Metamath's {\tt help invoke} command describes
alternate ways of invoking the Metamath program.)

When you first enter Metamath\index{Metamath}, it will be at the CLI, waiting
for your input. You will see something like the following on your screen:
\begin{verbatim}
Metamath - Version 0.177 27-Apr-2019
Type HELP for help, EXIT to exit.
MM>
\end{verbatim}
The \texttt{MM>} prompt means that Metamath is waiting for a command.
Command keywords\index{command keyword} are not case sensitive;
we will use lower-case commands in our examples.
The version number and its release date will probably be different on your
system from the one we show above.

The first thing that you need to do is to read in your
database:\index{\texttt{read} command}\footnote{If a directory path is
needed on Unix,\index{Unix file names}\index{file names!Unix} you should
enclose the path/file name in quotes to prevent Metamath from thinking
that the \texttt{/} in the path name is a command qualifier, e.g.,
\texttt{read \char`\"db/set.mm\char`\"}.  Quotes are optional when there
is no ambiguity.}
\begin{verbatim}
MM> read demo0.mm
\end{verbatim}
Remember to press the {\em return} key after entering this command.  If
you omit the file name, Metamath will prompt you for one.   The syntax for
specifying a Macintosh file name path is given in a footnote on
p.~\pageref{includef}.\index{Macintosh file names}\index{file
names!Macintosh}

If there are any syntax errors in the database, Metamath will let you know
when it reads in the file.  The one thing that Metamath does not check when
reading in a database is that all proofs are correct, because this would
slow it down too much.  It is a good idea to periodically verify the proofs in
a database you are making changes to.  To do this, use the following command
(and do it for your \texttt{demo0.mm} file now).  Note that the \texttt{*} is a
``wild card'' meaning all proofs in the file.\index{\texttt{verify proof} command}
\begin{verbatim}
MM> verify proof *
\end{verbatim}
Metamath will report any proofs that are incorrect.

It is often useful to save the information that the Metamath program displays
on the screen. You can save everything that happens on the screen by opening a
log file. You may want to do this before you read in a database so that you
can examine any errors later on.  To open a log file, type
\begin{verbatim}
MM> open log abc.log
\end{verbatim}
This will open a file called \texttt{abc.log}, and everything that appears on the
screen from this point on will be stored in this file.  The name of the log file
is arbitrary. To close the log file, type
\begin{verbatim}
MM> close log
\end{verbatim}

Several commands let you examine what's inside your database.
Section~\ref{exploring} has an overview of some useful ones.  The
\texttt{show labels} command lets you see what statement
labels\index{label} exist.  A \texttt{*} matches any combination of
characters, and \texttt{t*} refers to all labels starting with the
letter \texttt{t}.\index{\texttt{show labels} command} The \texttt{/all}
is a ``command qualifier''\index{command qualifier} that tells Metamath
to include labels of hypotheses.  (To see the syntax explained, type
\texttt{help show labels}.)  Type
\begin{verbatim}
MM> show labels t* /all
\end{verbatim}
Metamath will respond with
\begin{verbatim}
The statement number, label, and type are shown.
3 tt $f       4 tr $f       5 ts $f       8 tze $a
9 tpl $a      19 th1 $p
\end{verbatim}

You can use the \texttt{show statement} command to get information about a
particular statement.\index{\texttt{show statement} command}
For example, you can get information about the statement with label \texttt{mp}
by typing
\begin{verbatim}
MM> show statement mp /full
\end{verbatim}
Metamath will respond with
\begin{verbatim}
Statement 17 is located on line 43 of the file
"demo0.mm".
"Define the modus ponens inference rule"
17 mp $a |- Q $.
Its mandatory hypotheses in RPN order are:
  wp $f wff P $.
  wq $f wff Q $.
  min $e |- P $.
  maj $e |- ( P -> Q ) $.
The statement and its hypotheses require the
      variables:  Q P
The variables it contains are:  Q P
\end{verbatim}
The mandatory hypotheses\index{mandatory hypothesis} and their
order\index{RPN order} are
useful to know when you are trying to understand or debug a proof.

Now you are ready to look at what's really inside our proof.  First, here is
how to look at every step in the proof---not just the ones corresponding to an
ordinary formal proof\index{formal proof}, but also the ones that build up the
formulas that appear in each ordinary formal proof step.\index{\texttt{show
proof} command}
\begin{verbatim}
MM> show proof th1 /lemmon /all
\end{verbatim}

This will display the proof on the screen in the following format:
\begin{verbatim}
 1 tt            $f term t
 2 tze           $a term 0
 3 1,2 tpl       $a term ( t + 0 )
 4 tt            $f term t
 5 3,4 weq       $a wff ( t + 0 ) = t
 6 tt            $f term t
 7 tt            $f term t
 8 6,7 weq       $a wff t = t
 9 tt            $f term t
10 9 a2          $a |- ( t + 0 ) = t
11 tt            $f term t
12 tze           $a term 0
13 11,12 tpl     $a term ( t + 0 )
14 tt            $f term t
15 13,14 weq     $a wff ( t + 0 ) = t
16 tt            $f term t
17 tze           $a term 0
18 16,17 tpl     $a term ( t + 0 )
19 tt            $f term t
20 18,19 weq     $a wff ( t + 0 ) = t
21 tt            $f term t
22 tt            $f term t
23 21,22 weq     $a wff t = t
24 20,23 wim     $a wff ( ( t + 0 ) = t -> t = t )
25 tt            $f term t
26 25 a2         $a |- ( t + 0 ) = t
27 tt            $f term t
28 tze           $a term 0
29 27,28 tpl     $a term ( t + 0 )
30 tt            $f term t
31 tt            $f term t
32 29,30,31 a1   $a |- ( ( t + 0 ) = t -> ( ( t + 0 )
                                     = t -> t = t ) )
33 15,24,26,32 mp  $a |- ( ( t + 0 ) = t -> t = t )
34 5,8,10,33 mp  $a |- t = t
\end{verbatim}

The \texttt{/lemmon} command qualifier specifies what is known as a Lemmon-style
display\index{Lemmon-style proof}\index{proof!Lemmon-style}.  Omitting the
\texttt{/lemmon} qualifier results in a tree-style proof (see
p.~\pageref{treeproof} for an example) that is somewhat less explicit but
easier to follow once you get used to it.\index{tree-style
proof}\index{proof!tree-style}

The first number on each line is the step
number of the proof.  Any numbers that follow are step numbers assigned to the
hypotheses of the statement referenced by that step.  Next is the label of
the statement referenced by the step.  The statement type of the statement
referenced comes next, followed by the math symbol\index{math symbol} string
constructed by the proof up to that step.

The last step, 34, contains the statement that is being proved.

Looking at a small piece of the proof, notice that steps 3 and 4 have
established that
\texttt{( t + 0 )} and \texttt{t} are \texttt{term}\,s, and step 5 makes use of steps 3 and
4 to establish that \texttt{( t + 0 ) = t} is a \texttt{wff}.  Let Metamath
itself tell us in detail what is happening in step 5.  Note that the
``target hypothesis'' refers to where step 5 is eventually used, i.e., in step
34.
\begin{verbatim}
MM> show proof th1 /detailed_step 5
Proof step 5:  wp=weq $a wff ( t + 0 ) = t
This step assigns source "weq" ($a) to target "wp"
($f).  The source assertion requires the hypotheses
"tt" ($f, step 3) and "tr" ($f, step 4).  The parent
assertion of the target hypothesis is "mp" ($a,
step 34).
The source assertion before substitution was:
    weq $a wff t = r
The following substitutions were made to the source
assertion:
    Variable  Substituted with
     t         ( t + 0 )
     r         t
The target hypothesis before substitution was:
    wp $f wff P
The following substitution was made to the target
hypothesis:
    Variable  Substituted with
     P         ( t + 0 ) = t
\end{verbatim}

The full proof just shown is useful to understand what is going on in detail.
However, most of the time you will just be interested in
the ``essential'' or logical steps of a proof, i.e.\ those steps
that correspond to an
ordinary formal proof\index{formal proof}.  If you type
\begin{verbatim}
MM> show proof th1 /lemmon /renumber
\end{verbatim}
you will see\label{demoproof}
\begin{verbatim}
1 a2             $a |- ( t + 0 ) = t
2 a2             $a |- ( t + 0 ) = t
3 a1             $a |- ( ( t + 0 ) = t -> ( ( t + 0 )
                                     = t -> t = t ) )
4 2,3 mp         $a |- ( ( t + 0 ) = t -> t = t )
5 1,4 mp         $a |- t = t
\end{verbatim}
Compare this to the formal proof on p.~\pageref{zeroproof} and
notice the resemblance.
By default Metamath
does not show \texttt{\$f}\index{\texttt{\$f}
statement} hypotheses and everything branching off of them in the proof tree
when the proof is displayed; this makes the proof look more like an ordinary
mathematical proof, which does not normally incorporate the explicit
construction of expressions.
This is called the ``essential'' view
(at one time you had to add the
\texttt{/essential} qualifier in the \texttt{show proof}
command to get this view, but this is now the default).
You can could use the \texttt{/all} qualifier in the \texttt{show
proof} command to also show the explicit construction of expressions.
The \texttt{/renumber} qualifier means to renumber
the steps to correspond only to what is displayed.\index{\texttt{show proof}
command}

To exit Metamath, type\index{\texttt{exit} command}
\begin{verbatim}
MM> exit
\end{verbatim}

\subsection{Some Hints for Using the Command Line Interface}

We will conclude this quick introduction to Metamath\index{Metamath} with some
helpful hints on how to navigate your way through the commands.
\index{command line interface (CLI)}

When you type commands into Metamath's CLI, you only have to type as many
characters of a command keyword\index{command keyword} as are needed to make
it unambiguous.  If you type too few characters, Metamath will tell you what
the choices are.  In the case of the \texttt{read} command, only the \texttt{r} is
needed to specify it unambiguously, so you could have typed\index{\texttt{read}
command}
\begin{verbatim}
MM> r demo0.mm
\end{verbatim}
instead of
\begin{verbatim}
MM> read demo0.mm
\end{verbatim}
In our description, we always show the full command words.  When using the
Metamath CLI commands in a command file (to be read with the \texttt{submit}
command)\index{\texttt{submit} command}, it is good practice to use
the unabbreviated command to ensure your instructions will not become ambiguous
if more commands are added to the Metamath program in the future.

The command keywords\index{command
keyword} are not case sensitive; you may type either \texttt{read} or
\texttt{ReAd}.  File names may or may not be case sensitive, depending on your
computer's operating system.  Metamath label\index{label} and math
symbol\index{math symbol} tokens\index{token} are case-sensitive.

The \texttt{help} command\index{\texttt{help} command} will provide you
with a list of topics you can get help on.  You can then type
\texttt{help} {\em topic} to get help on that topic.

If you are uncertain of a command's spelling, just type as many characters
as you remember of the command.  If you have not typed enough characters to
specify it unambiguously, Metamath will tell you what choices you have.

\begin{verbatim}
MM> show s
         ^
?Ambiguous keyword - please specify SETTINGS,
STATEMENT, or SOURCE.
\end{verbatim}

If you don't know what argument to use as part of a command, type a
\texttt{?}\index{\texttt{]}@\texttt{?}\ in command lines}\ at the
argument position.  Metamath will tell you what it expected there.

\begin{verbatim}
MM> show ?
         ^
?Expected SETTINGS, LABELS, STATEMENT, SOURCE, PROOF,
MEMORY, TRACE_BACK, or USAGE.
\end{verbatim}

Finally, you may type just the first word or words of a command followed
by {\em return}.  Metamath will prompt you for the remaining part of the
command, showing you the choices at each step.  For example, instead of
typing \texttt{show statement th1 /full} you could interact in the
following manner:
\begin{verbatim}
MM> show
SETTINGS, LABELS, STATEMENT, SOURCE, PROOF,
MEMORY, TRACE_BACK, or USAGE <SETTINGS>? st
What is the statement label <th1>?
/ or nothing <nothing>? /
TEX, COMMENT_ONLY, or FULL <TEX>? f
/ or nothing <nothing>?
19 th1 $p |- t = t $= ... $.
\end{verbatim}
After each \texttt{?}\ in this mode, you must give Metamath the
information it requests.  Sometimes Metamath gives you a list of choices
with the default choice indicated by brackets \texttt{< > }. Pressing
{\em return} after the \texttt{?}\ will select the default choice.
Answering anything else will override the default.  Note that the
\texttt{/} in command qualifiers is considered a separate
token\index{token} by the parser, and this is why it is asked for
separately.

\section{Your First Proof}\label{frstprf}

Proofs are developed with the aid of the Proof Assistant\index{Proof
Assistant}.  We will now show you how the proof of theorem \texttt{th1}
was built.  So that you can repeat these steps, we will first have the
Proof Assistant erase the proof in Metamath's source buffer\index{source
buffer}, then reconstruct it.  (The source buffer is the place in memory
where Metamath stores the information in the database when it is
\texttt{read}\index{\texttt{read} command} in.  New or modified proofs
are kept in the source buffer until a \texttt{write source}
command\index{\texttt{write source} command} is issued.)  In practice, you
would place a \texttt{?}\index{\texttt{]}@\texttt{?}\ inside proofs}\
between \texttt{\$=}\index{\texttt{\$=} keyword} and
\texttt{\$.}\index{\texttt{\$.}\ keyword}\ in the database to indicate
to Metamath\index{Metamath} that the proof is unknown, and that would be
your starting point.  Whenever the \texttt{verify proof} command encounters
a proof with a \texttt{?}\ in place of a proof step, the statement is
identified as not proved.

When I first started creating Metamath proofs, I would write down
on a piece of paper the complete
formal proof\index{formal proof} as it would appear
in a \texttt{show proof} command\index{\texttt{show proof} command}; see
the display of \texttt{show proof th1 /lemmon /re\-num\-ber} above as an
example.  After you get used to using the Proof Assistant\index{Proof
Assistant} you may get to a point where you can ``see'' the proof in your mind
and let the Proof Assistant guide you in filling in the details, at least for
simpler proofs, but until you gain that experience you may find it very useful
to write down all the details in advance.
Otherwise you may waste a lot of time as you let it take you down a wrong path.
However, others do not find this approach as helpful.
For example, Thomas Brendan Leahy\index{Leahy, Thomas Brendan}
finds that it is more helpful to him to interactively
work backward from a machine-readable statement.
David A. Wheeler\index{Wheeler, David A.}
writes down a general approach, but develops the proof
interactively by switching between
working forwards (from hypotheses and facts likely to be useful) and
backwards (from the goal) until the forwards and backwards approaches meet.
In the end, use whatever approach works for you.

A proof is developed with the Proof Assistant by working backwards, starting
with the theorem\index{theorem} to be proved, and assigning each unknown step
with a theorem or hypothesis until no more unknown steps remain.  The Proof
Assistant will not let you make an assignment unless it can be ``unified''
with the unknown step.  This means that a
substitution\index{substitution!variable}\index{variable substitution} of
variables exists that will make the assignment match the unknown step.  On the
other hand, in the middle of a proof, when working backwards, often more than
one unification\index{unification} (set of substitutions) is possible, since
there is not enough information available at that point to uniquely establish
it.  In this case you can tell Metamath which unification to choose, or you
can continue to assign unknown steps until enough information is available to
make the unification unique.

We will assume you have entered Metamath and read in the database as described
above.  The following dialog shows how the proof was developed.  For more
details on what some of the commands do, refer to Section~\ref{pfcommands}.
\index{\texttt{prove} command}

\begin{verbatim}
MM> prove th1
Entering the Proof Assistant.  Type HELP for help, EXIT
to exit.  You will be working on the proof of statement th1:
  $p |- t = t
Note:  The proof you are starting with is already complete.
MM-PA>
\end{verbatim}

The \verb/MM-PA>/ prompt means we are inside the Proof
Assistant.\index{Proof Assistant} Most of the regular Metamath commands
(\texttt{show statement}, etc.) are still available if you need them.

\begin{verbatim}
MM-PA> delete all
The entire proof was deleted.
\end{verbatim}

We have deleted the whole proof so we can start from scratch.

\begin{verbatim}
MM-PA> show new_proof/lemmon/all
1 ?              $? |- t = t
\end{verbatim}

The \texttt{show new{\char`\_}proof} command\index{\texttt{show
new{\char`\_}proof} command} is like \texttt{show proof} except that we
don't specify a statement; instead, the proof we're working on is
displayed.

\begin{verbatim}
MM-PA> assign 1 mp
To undo the assignment, DELETE STEP 5 and INITIALIZE, UNIFY
if needed.
3   min=?  $? |- $2
4   maj=?  $? |- ( $2 -> t = t )
\end{verbatim}

The \texttt{assign} command\index{\texttt{assign} command} above means
``assign step 1 with the statement whose label is \texttt{mp}.''  Note
that step renumbering will constantly occur as you assign steps in the
middle of a proof; in general all steps from the step you assign until
the end of the proof will get moved up.  In this case, what used to be
step 1 is now step 5, because the (partial) proof now has five steps:
the four hypotheses of the \texttt{mp} statement and the \texttt{mp}
statement itself.  Let's look at all the steps in our partial proof:

\begin{verbatim}
MM-PA> show new_proof/lemmon/all
1 ?              $? wff $2
2 ?              $? wff t = t
3 ?              $? |- $2
4 ?              $? |- ( $2 -> t = t )
5 1,2,3,4 mp     $a |- t = t
\end{verbatim}

The symbol \texttt{\$2} is a temporary variable\index{temporary
variable} that represents a symbol sequence not yet known.  In the final
proof, all temporary variables will be eliminated.  The general format
for a temporary variable is \texttt{\$} followed by an integer.  Note
that \texttt{\$} is not a legal character in a math symbol (see
Section~\ref{dollardollar}, p.~\pageref{dollardollar}), so there will
never be a naming conflict between real symbols and temporary variables.

Unknown steps 1 and 2 are constructions of the two wffs used by the
modus ponens rule.  As you will see at the end of this section, the
Proof Assistant\index{Proof Assistant} can usually figure these steps
out by itself, and we will not have to worry about them.  Therefore from
here on we will display only the ``essential'' hypotheses, i.e.\ those
steps that correspond to traditional formal proofs\index{formal proof}.

\begin{verbatim}
MM-PA> show new_proof/lemmon
3 ?              $? |- $2
4 ?              $? |- ( $2 -> t = t )
5 3,4 mp         $a |- t = t
\end{verbatim}

Unknown steps 3 and 4 are the ones we must focus on.  They correspond to the
minor and major premises of the modus ponens rule.  We will assign them as
follows.  Notice that because of the step renumbering that takes place
after an assignment, it is advantageous to assign unknown steps in reverse
order, because earlier steps will not get renumbered.

\begin{verbatim}
MM-PA> assign 4 mp
To undo the assignment, DELETE STEP 8 and INITIALIZE, UNIFY
if needed.
3   min=?  $? |- $2
6     min=?  $? |- $4
7     maj=?  $? |- ( $4 -> ( $2 -> t = t ) )
\end{verbatim}

We are now going to describe an obscure feature that you will probably
never use but should be aware of.  The Metamath language allows empty
symbol sequences to be substituted for variables, but in most formal
systems this feature is never used.  One of the few examples where is it
used is the MIU-system\index{MIU-system} described in
Appendix~\ref{MIU}.  But such systems are rare, and by default this
feature is turned off in the Proof Assistant.  (It is always allowed for
{\tt verify proof}.)  Let us turn it on and see what
happens.\index{\texttt{set empty{\char`\_}substitution} command}

\begin{verbatim}
MM-PA> set empty_substitution on
Substitutions with empty symbol sequences is now allowed.
\end{verbatim}

With this feature enabled, more unifications will be
ambiguous\index{ambiguous unification}\index{unification!ambiguous} in
the middle of a proof, because
substitution\index{substitution!variable}\index{variable substitution}
of variables with empty symbol sequences will become an additional
possibility.  Let's see what happens when we make our next assignment.

\begin{verbatim}
MM-PA> assign 3 a2
There are 2 possible unifications.  Please select the correct
    one or Q if you want to UNIFY later.
Unify:  |- $6
 with:  |- ( $9 + 0 ) = $9
Unification #1 of 2 (weight = 7):
  Replace "$6" with "( + 0 ) ="
  Replace "$9" with ""
  Accept (A), reject (R), or quit (Q) <A>? r
\end{verbatim}

The first choice presented is the wrong one.  If we had selected it,
temporary variable \texttt{\$6} would have been assigned a truncated
wff, and temporary variable \texttt{\$9} would have been assigned an
empty sequence (which is not allowed in our system).  With this choice,
eventually we would reach a point where we would get stuck because
we would end up with steps impossible to prove.  (You may want to
try it.)  We typed \texttt{r} to reject the choice.

\begin{verbatim}
Unification #2 of 2 (weight = 21):
  Replace "$6" with "( $9 + 0 ) = $9"
  Accept (A), reject (R), or quit (Q) <A>? q
To undo the assignment, DELETE STEP 4 and INITIALIZE, UNIFY
if needed.
 7     min=?  $? |- $8
 8     maj=?  $? |- ( $8 -> ( $6 -> t = t ) )
\end{verbatim}

The second choice is correct, and normally we would type \texttt{a}
to accept it.  But instead we typed \texttt{q} to show what will happen:
it will leave the step with an unknown unification, which can be
seen as follows:

\begin{verbatim}
MM-PA> show new_proof/not_unified
 4   min    $a |- $6
        =a2  = |- ( $9 + 0 ) = $9
\end{verbatim}

Later we can unify this with the \texttt{unify}
\texttt{all/interactive} command.

The important point to remember is that occasionally you will be
presented with several unification choices while entering a proof, when
the program determines that there is not enough information yet to make
an unambiguous choice automatically (and this can happen even with
\texttt{set empty{\char`\_}substitution} turned off).  Usually it is
obvious by inspection which choice is correct, since incorrect ones will
tend to be meaningless fragments of wffs.  In addition, the correct
choice will usually be the first one presented, unlike our example
above.

Enough of this digression.  Let us go back to the default setting.

\begin{verbatim}
MM-PA> set empty_substitution off
The ability to substitute empty expressions for variables
has been turned off.  Note that this may make the Proof
Assistant too restrictive in some cases.
\end{verbatim}

If we delete the proof, start over, and get to the point where
we digressed above, there will no longer be an ambiguous unification.

\begin{verbatim}
MM-PA> assign 3 a2
To undo the assignment, DELETE STEP 4 and INITIALIZE, UNIFY
if needed.
 7     min=?  $? |- $4
 8     maj=?  $? |- ( $4 -> ( ( $5 + 0 ) = $5 -> t = t ) )
\end{verbatim}

Let us look at our proof so far, and continue.

\begin{verbatim}
MM-PA> show new_proof/lemmon
 4 a2            $a |- ( $5 + 0 ) = $5
 7 ?             $? |- $4
 8 ?             $? |- ( $4 -> ( ( $5 + 0 ) = $5 -> t = t ) )
 9 7,8 mp        $a |- ( ( $5 + 0 ) = $5 -> t = t )
10 4,9 mp        $a |- t = t
MM-PA> assign 8 a1
To undo the assignment, DELETE STEP 11 and INITIALIZE, UNIFY
if needed.
 7     min=?  $? |- ( t + 0 ) = t
MM-PA> assign 7 a2
To undo the assignment, DELETE STEP 8 and INITIALIZE, UNIFY
if needed.
MM-PA> show new_proof/lemmon
 4 a2            $a |- ( t + 0 ) = t
 8 a2            $a |- ( t + 0 ) = t
12 a1            $a |- ( ( t + 0 ) = t -> ( ( t + 0 ) = t ->
                                                    t = t ) )
13 8,12 mp       $a |- ( ( t + 0 ) = t -> t = t )
14 4,13 mp       $a |- t = t
\end{verbatim}

Now all temporary variables and unknown steps have been eliminated from the
``essential'' part of the proof.  When this is achieved, the Proof
Assistant\index{Proof Assistant} can usually figure out the rest of the proof
automatically.  (Note that the \texttt{improve} command can occasionally be
useful for filling in essential steps as well, but it only tries to make use
of statements that introduce no new variables in their hypotheses, which is
not the case for \texttt{mp}. Also it will not try to improve steps containing
temporary variables.)  Let's look at the complete proof, then run
the \texttt{improve} command, then look at it again.

\begin{verbatim}
MM-PA> show new_proof/lemmon/all
 1 ?             $? wff ( t + 0 ) = t
 2 ?             $? wff t = t
 3 ?             $? term t
 4 3 a2          $a |- ( t + 0 ) = t
 5 ?             $? wff ( t + 0 ) = t
 6 ?             $? wff ( ( t + 0 ) = t -> t = t )
 7 ?             $? term t
 8 7 a2          $a |- ( t + 0 ) = t
 9 ?             $? term ( t + 0 )
10 ?             $? term t
11 ?             $? term t
12 9,10,11 a1    $a |- ( ( t + 0 ) = t -> ( ( t + 0 ) = t ->
                                                    t = t ) )
13 5,6,8,12 mp   $a |- ( ( t + 0 ) = t -> t = t )
14 1,2,4,13 mp   $a |- t = t
\end{verbatim}

\begin{verbatim}
MM-PA> improve all
A proof of length 1 was found for step 11.
A proof of length 1 was found for step 10.
A proof of length 3 was found for step 9.
A proof of length 1 was found for step 7.
A proof of length 9 was found for step 6.
A proof of length 5 was found for step 5.
A proof of length 1 was found for step 3.
A proof of length 3 was found for step 2.
A proof of length 5 was found for step 1.
Steps 1 and above have been renumbered.
CONGRATULATIONS!  The proof is complete.  Use SAVE
NEW_PROOF to save it.  Note:  The Proof Assistant does
not detect $d violations.  After saving the proof, you
should verify it with VERIFY PROOF.
\end{verbatim}

The \texttt{save new{\char`\_}proof} command\index{\texttt{save
new{\char`\_}proof} command} will save the proof in the database.  Here
we will just display it in a form that can be clipped out of a log file
and inserted manually into the database source file with a text
editor.\index{normal proof}\index{proof!normal}

\begin{verbatim}
MM-PA> show new_proof/normal
---------Clip out the proof below this line:
      tt tze tpl tt weq tt tt weq tt a2 tt tze tpl tt weq
      tt tze tpl tt weq tt tt weq wim tt a2 tt tze tpl tt
      tt a1 mp mp $.
---------The proof of 'th1' to clip out ends above this line.
\end{verbatim}

There is another proof format called ``compressed''\index{compressed
proof}\index{proof!compressed} that you will see in databases.  It is
not important to understand how it is encoded but only to recognize it
when you see it.  Its only purpose is to reduce storage requirements for
large proofs.  A compressed proof can always be converted to a normal
one and vice-versa, and the Metamath \texttt{show proof}
commands\index{\texttt{show proof} command} work equally well with
compressed proofs.  The compressed proof format is described in
Appendix~\ref{compressed}.

\begin{verbatim}
MM-PA> show new_proof/compressed
---------Clip out the proof below this line:
      ( tze tpl weq a2 wim a1 mp ) ABCZADZAADZAEZJJKFLIA
      AGHH $.
---------The proof of 'th1' to clip out ends above this line.
\end{verbatim}

Now we will exit the Proof Assistant.  Since we made changes to the proof,
it will warn us that we have not saved it.  In this case, we don't care.

\begin{verbatim}
MM-PA> exit
Warning:  You have not saved changes to the proof.
Do you want to EXIT anyway (Y, N) <N>? y
Exiting the Proof Assistant.
Type EXIT again to exit Metamath.
\end{verbatim}

The Proof Assistant\index{Proof Assistant} has several other commands
that can help you while creating proofs.  See Section~\ref{pfcommands}
for a list of them.

A command that is often useful is \texttt{minimize{\char`\_}with
*/brief}, which tries to shorten the proof.  It can make the process
more efficient by letting you write a somewhat ``sloppy'' proof then
clean up some of the fine details of optimization for you (although it
can't perform miracles such as restructuring the overall proof).

\section{A Note About Editing a Data\-base File}

Once your source file contains proofs, there are some restrictions on
how you can edit it so that the proofs remain valid.  Pay particular
attention to these rules, since otherwise you can lose a lot of work.
It is a good idea to periodically verify all proofs with \texttt{verify
proof *} to ensure their integrity.

If your file contains only normal (as opposed to compressed) proofs, the
main rule is that you may not change the order of the mandatory
hypotheses\index{mandatory hypothesis} of any statement referenced in a
later proof.  For example, if you swap the order of the major and minor
premise in the modus ponens rule, all proofs making use of that rule
will become incorrect.  The \texttt{show statement}
command\index{\texttt{show statement} command} will show you the
mandatory hypotheses of a statement and their order.

If a statement has a compressed proof, you also must not change the
order of {\em its} mandatory hypotheses.  The compressed proof format
makes use of this information as part of the compression technique.
Note that swapping the names of two variables in a theorem will change
the order of its mandatory hypotheses.

The safest way to edit a statement, say \texttt{mytheorem}, is to
duplicate it then rename the original to \texttt{mytheoremOLD}
throughout the database.  Once the edited version is re-proved, all
statements referencing \texttt{mytheoremOLD} can be updated in the Proof
Assistant using \texttt{minimize{\char`\_}with
mytheorem
/allow{\char`\_}growth}.\index{\texttt{minimize{\char`\_}with} command}
% 3/10/07 Note: line-breaking the above results in duplicate index entries

\chapter{Abstract Mathematics Revealed}\label{fol}

\section{Logic and Set Theory}\label{logicandsettheory}

\begin{quote}
  {\em Set theory can be viewed as a form of exact theology.}
  \flushright\sc  Rudy Rucker\footnote{\cite{Barrow}, p.~31.}\\
\end{quote}\index{Rucker, Rudy}

Despite its seeming complexity, all of standard mathematics, no matter how
deep or abstract, can amazingly enough be derived from a relatively small set
of axioms\index{axiom} or first principles. The development of these axioms is
among the most impressive and important accomplishments of mathematics in the
20th century. Ultimately, these axioms can be broken down into a set of rules
for manipulating symbols that any technically oriented person can follow.

We will not spend much time trying to convey a deep, higher-level
understanding of the meaning of the axioms. This kind of understanding
requires some mathematical sophistication as well as an understanding of the
philosophy underlying the foundations of mathematics and typically develops
over time as you work with mathematics.  Our goal, instead, is to give you the
immediate ability to follow how theorems\index{theorem} are derived from the
axioms and from other theorems.  This will be similar to learning the syntax
of a computer language, which lets you follow the details in a program but
does not necessarily give you the ability to write non-trivial programs on
your own, an ability that comes with practice. For now don't be alarmed by
abstract-sounding names of the axioms; just focus on the rules for
manipulating the symbols, which follow the simple conventions of the
Metamath\index{Metamath} language.

The axioms that underlie all of standard mathematics consist of axioms of logic
and axioms of set theory. The axioms of logic are divided into two
subcategories, propositional calculus\index{propositional calculus} (sometimes
called sentential logic\index{sentential logic}) and predicate calculus
(sometimes called first-order logic\index{first-order logic}\index{quantifier
theory}\index{predicate calculus} or quantifier theory).  Propositional
calculus is a prerequisite for predicate calculus, and predicate calculus is a
prerequisite for set theory.  The version of set theory most commonly used is
Zermelo--Fraenkel set theory\index{Zermelo--Fraenkel set theory}\index{set theory}
with the axiom of choice,
often abbreviated as ZFC\index{ZFC}.

Here in a nutshell is what the axioms are all about in an informal way. The
connection between this description and symbols we will show you won't be
immediately apparent and in principle needn't ever be.  Our description just
tries to summarize what mathematicians think about when they work with the
axioms.

Logic is a set of rules that allow us determine truths given other truths.
Put another way,
logic is more or less the translation of what we would consider common sense
into a rigorous set of axioms.\index{axioms of logic}  Suppose $\varphi$,
$\psi$, and $\chi$ (the Greek letters phi, psi, and chi) represent statements
that are either true or false, and $x$ is a variable\index{variable!in predicate
calculus} ranging over some group of mathematical objects (sets, integers,
real numbers, etc.). In mathematics, a ``statement'' really means a formula,
and $\psi$ could be for example ``$x = 2$.''
Propositional calculus\index{propositional calculus}
allows us to use variables that are either true or false
and make deductions such as
``if $\varphi$ implies $\psi$ and $\psi$ implies $\chi$, then $\varphi$
implies $\chi$.''
Predicate calculus\index{predicate calculus}
extends propositional calculus by also allowing us
to discuss statements about objects (not just true and false values), including
statements about ``all'' or ``at least one'' object.
For example, predicate calculus allows to say,
``if $\varphi$ is true for all $x$, then $\varphi$ is true for some $x$.''
The logic used in \texttt{set.mm} is standard classical logic
(as opposed to other logic systems like intuitionistic logic).

Set theory\index{set theory} has to do with the manipulation of objects and
collections of objects, specifically the abstract, imaginary objects that
mathematics deals with, such as numbers. Everything that is claimed to exist
in mathematics is considered to be a set.  A set called the empty
set\index{empty set} contains nothing.  We represent the empty set by
$\varnothing$.  Many sets can be built up from the empty set.  There is a set
represented by $\{\varnothing\}$ that contains the empty set, another set
represented by $\{\varnothing,\{\varnothing\}\}$ that contains this set as
well as the empty set, another set represented by $\{\{\varnothing\}\}$ that
contains just the set that contains the empty set, and so on ad infinitum. All
mathematical objects, no matter how complex, are defined as being identical to
certain sets: the integer\index{integer} 0 is defined as the empty set, the
integer 1 is defined as $\{\varnothing\}$, the integer 2 is defined as
$\{\varnothing,\{\varnothing\}\}$.  (How these definitions were chosen doesn't
matter now, but the idea behind it is that these sets have the properties we
expect of integers once suitable operations are defined.)  Mathematical
operations, such as addition, are defined in terms of operations on
sets---their union\index{set union}, intersection\index{set intersection}, and
so on---operations you may have used in elementary school when you worked
with groups of apples and oranges.

With a leap of faith, the axioms also postulate the existence of infinite
sets\index{infinite set}, such as the set of all non-negative integers ($0, 1,
2,\ldots$, also called ``natural numbers''\index{natural number}).  This set
can't be represented with the brace notation\index{brace notation} we just
showed you, but requires a more complicated notation called ``class
abstraction.''\index{class abstraction}\index{abstraction class}  For
example, the infinite set $\{ x |
\mbox{``$x$ is a natural number''} \} $ means the ``set of all objects $x$
such that $x$ is a natural number'' i.e.\ the set of natural numbers; here,
``$x$ is a natural number'' is a rather complicated formula when broken down
into the primitive symbols.\label{expandom}\footnote{The statement ``$x$ is a
natural number'' is formally expressed as ``$x \in \omega$,'' where $\in$
(stylized epsilon) means ``is in'' or ``is an element of'' and $\omega$
(omega) means ``the set of natural numbers.''  When ``$x\in\omega$'' is
completely expanded in terms of the primitive symbols of set theory, the
result is  $\lnot$ $($ $\lnot$ $($ $\forall$ $z$ $($ $\lnot$ $\forall$ $w$ $($
$z$ $\in$ $w$ $\rightarrow$ $\lnot$ $w$ $\in$ $x$ $)$ $\rightarrow$ $z$ $\in$
$x$ $)$ $\rightarrow$ $($ $\forall$ $z$ $($ $\lnot$ $($ $\forall$ $w$ $($ $w$
$\in$ $z$ $\rightarrow$ $w$ $\in$ $x$ $)$ $\rightarrow$ $\forall$ $w$ $\lnot$
$w$ $\in$ $z$ $)$ $\rightarrow$ $\lnot$ $\forall$ $w$ $($ $w$ $\in$ $z$
$\rightarrow$ $\lnot$ $\forall$ $v$ $($ $v$ $\in$ $z$ $\rightarrow$ $\lnot$
$v$ $\in$ $w$ $)$ $)$ $)$ $\rightarrow$ $\lnot$ $\forall$ $z$ $\forall$ $w$
$($ $\lnot$ $($ $z$ $\in$ $x$ $\rightarrow$ $\lnot$ $w$ $\in$ $x$ $)$
$\rightarrow$ $($ $\lnot$ $z$ $\in$ $w$ $\rightarrow$ $($ $\lnot$ $z$ $=$ $w$
$\rightarrow$ $w$ $\in$ $z$ $)$ $)$ $)$ $)$ $)$ $\rightarrow$ $\lnot$
$\forall$ $y$ $($ $\lnot$ $($ $\lnot$ $($ $\forall$ $z$ $($ $\lnot$ $\forall$
$w$ $($ $z$ $\in$ $w$ $\rightarrow$ $\lnot$ $w$ $\in$ $y$ $)$ $\rightarrow$
$z$ $\in$ $y$ $)$ $\rightarrow$ $($ $\forall$ $z$ $($ $\lnot$ $($ $\forall$
$w$ $($ $w$ $\in$ $z$ $\rightarrow$ $w$ $\in$ $y$ $)$ $\rightarrow$ $\forall$
$w$ $\lnot$ $w$ $\in$ $z$ $)$ $\rightarrow$ $\lnot$ $\forall$ $w$ $($ $w$
$\in$ $z$ $\rightarrow$ $\lnot$ $\forall$ $v$ $($ $v$ $\in$ $z$ $\rightarrow$
$\lnot$ $v$ $\in$ $w$ $)$ $)$ $)$ $\rightarrow$ $\lnot$ $\forall$ $z$
$\forall$ $w$ $($ $\lnot$ $($ $z$ $\in$ $y$ $\rightarrow$ $\lnot$ $w$ $\in$
$y$ $)$ $\rightarrow$ $($ $\lnot$ $z$ $\in$ $w$ $\rightarrow$ $($ $\lnot$ $z$
$=$ $w$ $\rightarrow$ $w$ $\in$ $z$ $)$ $)$ $)$ $)$ $\rightarrow$ $($
$\forall$ $z$ $\lnot$ $z$ $\in$ $y$ $\rightarrow$ $\lnot$ $\forall$ $w$ $($
$\lnot$ $($ $w$ $\in$ $y$ $\rightarrow$ $\lnot$ $\forall$ $z$ $($ $w$ $\in$
$z$ $\rightarrow$ $\lnot$ $z$ $\in$ $y$ $)$ $)$ $\rightarrow$ $\lnot$ $($
$\lnot$ $\forall$ $z$ $($ $w$ $\in$ $z$ $\rightarrow$ $\lnot$ $z$ $\in$ $y$
$)$ $\rightarrow$ $w$ $\in$ $y$ $)$ $)$ $)$ $)$ $\rightarrow$ $x$ $\in$ $y$
$)$ $)$ $)$. Section~\ref{hierarchy} shows the hierarchy of definitions that
leads up to this expression.}\index{stylized epsilon ($\in$)}\index{omega
($\omega$)}  Actually, the primitive symbols don't even include the brace
notation.  The brace notation is a high-level definition, which you can find in
Section~\ref{hierarchy}.

Interestingly, the arithmetic of integers\index{integer} and
rationals\index{rational number} can be developed without appealing to the
existence of an infinite set, whereas the arithmetic of real
numbers\index{real number} requires it.

Each variable\index{variable!in set theory} in the axioms of set theory
represents an arbitrary set, and the axioms specify the legal kinds of things
you can do with these variables at a very primitive level.

Now, you may think that numbers and arithmetic are a lot more intuitive and
fundamental than sets and therefore should be the foundation of mathematics.
What is really the case is that you've dealt with numbers all your life and
are comfortable with a few rules for manipulating them such as addition and
multiplication.  Those rules only cover a small portion of what can be done
with numbers and only a very tiny fraction of the rest of mathematics.  If you
look at any elementary book on number theory, you will quickly become lost if
these are the only rules that you know.  Even though such books may present a
list of ``axioms''\index{axiom} for arithmetic, the ability to use the axioms
and to understand proofs of theorems\index{theorem} (facts) about numbers
requires an implicit mathematical talent that frustrates many people
from studying abstract mathematics.  The kind of mathematics that most people
know limits them to the practical, everyday usage of blindly manipulating
numbers and formulas, without any understanding of why those rules are correct
nor any ability to go any further.  For example, do you know why multiplying
two negative numbers yields a positive number?  Starting with set theory, you
will also start off blindly manipulating symbols according to the rules we give
you, but with the advantage that these rules will allow you, in principle, to
access {\em all} of mathematics, not just a tiny part of it.

Of course, concrete examples are often helpful in the learning process. For
example, you can verify that $2\cdot 3=3 \cdot 2$ by actually grouping
objects and can easily ``see'' how it generalizes to $x\cdot y = y\cdot x$,
even though you might not be able to rigorously prove it.  Similarly, in set
theory it can be helpful to understand how the axioms of set theory apply to
(and are correct for) small finite collections of objects.  You should be aware
that in set theory intuition can be misleading for infinite collections, and
rigorous proofs become more important.  For example, while $x\cdot y = y\cdot
x$ is correct for finite ordinals (which are the natural numbers), it is not
usually true for infinite ordinals.

\section{The Axioms for All of Mathematics}

In this section\index{axioms for mathematics}, we will show you the axioms
for all of standard mathematics (i.e.\ logic and set theory) as they are
traditionally presented.  The traditional presentation is useful for someone
with the mathematical experience needed to correctly manipulate high-level
abstract concepts.  For someone without this talent, knowing how to actually
make use of these axioms can be difficult.  The purpose of this section is to
allow you to see how the version of the axioms used in the standard
Metamath\index{Metamath} database \texttt{set.mm}\index{set
theory database (\texttt{set.mm})} relates to  the typical version
in textbooks, and also to give you an informal feel for them.

\subsection{Propositional Calculus}

Propositional calculus\index{propositional calculus} concerns itself with
statements that can be interpreted as either true or false.  Some examples of
statements (outside of mathematics) that are either true or false are ``It is
raining today'' and ``The United States has a female president.'' In
mathematics, as we mentioned, statements are really formulas.

In propositional calculus, we don't care what the statements are.  We also
treat a logical combination of statements, such as ``It is raining today and
the United States has a female president,'' no differently from a single
statement.  Statements and their combinations are called well-formed formulas
(wffs)\index{well-formed formula (wff)}.  We define wffs only in terms of
other wffs and don't define what a ``starting'' wff is.  As is common practice
in the literature, we use Greek letters to represent wffs.

Specifically, suppose $\varphi$ and $\psi$ are wffs.  Then the combinations
$\varphi\rightarrow\psi$ (``$\varphi$ implies $\psi$,'' also read ``if
$\varphi$ then $\psi$'')\index{implication ($\rightarrow$)} and $\lnot\varphi$
(``not $\varphi$'')\index{negation ($\lnot$)} are also wffs.

The three axioms of propositional calculus\index{axioms of propositional
calculus} are all wffs of the following form:\footnote{A remarkable result of
C.~A.~Meredith\index{Meredith, C. A.} squeezes these three axioms into the
single axiom $((((\varphi\rightarrow \psi)\rightarrow(\neg \chi\rightarrow\neg
\theta))\rightarrow \chi )\rightarrow \tau)\rightarrow((\tau\rightarrow
\varphi)\rightarrow(\theta\rightarrow \varphi))$ \cite{CAMeredith},
which is believed to be the shortest possible.}
\begin{center}
     $\varphi\rightarrow(\psi\rightarrow \varphi)$\\

     $(\varphi\rightarrow (\psi\rightarrow \chi))\rightarrow
((\varphi\rightarrow  \psi)\rightarrow (\varphi\rightarrow \chi))$\\

     $(\neg \varphi\rightarrow \neg\psi)\rightarrow (\psi\rightarrow
\varphi)$
\end{center}

These three axioms are widely used.
They are attributed to Jan {\L}ukasiewicz
(pronounced woo-kah-SHAY-vitch) and was popularized by Alonzo Church,
who called it system P2. (Thanks to Ted Ulrich for this information.)

There are an infinite number of axioms, one for each possible
wff\index{well-formed formula (wff)} of the above form.  (For this reason,
axioms such as the above are often called ``axiom schemes.''\index{axiom
scheme})  Each Greek letter in the axioms may be substituted with a more
complex wff to result in another axiom.  For example, substituting
$\neg(\varphi\rightarrow\chi)$ for $\varphi$ in the first axiom yields
$\neg(\varphi\rightarrow\chi)\rightarrow(\psi\rightarrow
\neg(\varphi\rightarrow\chi))$, which is still an axiom.

To deduce new true statements (theorems\index{theorem}) from the axioms, a
rule\index{rule} called ``modus ponens''\index{modus ponens} is used.  This
rule states that if the wff $\varphi$ is an axiom or a theorem, and the wff
$\varphi\rightarrow\psi$ is an axiom or a theorem, then the wff $\psi$ is also
a theorem\index{theorem}.

As a non-mathematical example of modus ponens, suppose we have proved (or
taken as an axiom) ``Bob is a man'' and separately have proved (or taken as
an axiom) ``If Bob is a man, then Bob is a human.''  Using the rule of modus
ponens, we can logically deduce, ``Bob is a human.''

From Metamath's\index{Metamath} point of view, the axioms and the rule of
modus ponens just define a mechanical means for deducing new true statements
from existing true statements, and that is the complete content of
propositional calculus as far as Metamath is concerned.  You can read a logic
textbook to gain a better understanding of their meaning, or you can just let
their meaning slowly become apparent to you after you use them for a while.

It is actually rather easy to check to see if a formula is a theorem of
propositional calculus.  Theorems of propositional calculus are also called
``tautologies.''\index{tautology}  The technique to check whether a formula is
a tautology is called the ``truth table method,''\index{truth table} and it
works like this.  A wff $\varphi\rightarrow\psi$ is false whenever $\varphi$ is true
and $\psi$ is false.  Otherwise it is true.  A wff $\lnot\varphi$ is false
whenever $\varphi$ is true and false otherwise. To verify a tautology such as
$\varphi\rightarrow(\psi\rightarrow \varphi)$, you break it down into sub-wffs and
construct a truth table that accounts for all possible combinations of true
and false assigned to the wff metavariables:
\begin{center}\begin{tabular}{|c|c|c|c|}\hline
\mbox{$\varphi$} & \mbox{$\psi$} & \mbox{$\psi\rightarrow\varphi$}
    & \mbox{$\varphi\rightarrow(\psi\rightarrow \varphi)$} \\ \hline \hline
              T   &  T    &      T       &        T    \\ \hline
              T   &  F    &      T       &        T    \\ \hline
              F   &  T    &      F       &        T    \\ \hline
              F   &  F    &      T       &        T    \\ \hline
\end{tabular}\end{center}
If all entries in the last column are true, the formula is a tautology.

Now, the truth table method doesn't tell you how to prove the tautology from
the axioms, but only that a proof exists.  Finding an actual proof (especially
one that is short and elegant) can be challenging.  Methods do exist for
automatically generating proofs in propositional calculus, but the proofs that
result can sometimes be very long.  In the Metamath \texttt{set.mm}\index{set
theory database (\texttt{set.mm})} database, most
or all proofs were created manually.

Section \ref{metadefprop} discusses various definitions
that make propositional calculus easier to use.
For example, we define:

\begin{itemize}
\item $\varphi \vee \psi$
  is true if either $\varphi$ or $\psi$ (or both) are true
  (this is disjunction\index{disjunction ($\vee$)}
  aka logical {\sc or}\index{logical {\sc or} ($\vee$)}).

\item $\varphi \wedge \psi$
  is true if both $\varphi$ and $\psi$ are true
  (this is conjunction\index{conjunction ($\wedge$)}
  aka logical {\sc and}\index{logical {\sc and} ($\wedge$)}).

\item $\varphi \leftrightarrow \psi$
  is true if $\varphi$ and $\psi$ have the same value, that is,
  they are both true or both false
  (this is the biconditional\index{biconditional ($\leftrightarrow$)}).
\end{itemize}

\subsection{Predicate Calculus}

Predicate calculus\index{predicate calculus} introduces the concept of
``individual variables,''\index{variable!in predicate calculus}\index{individual
variable} which
we will usually just call ``variables.''
These variables can represent something other than true or false (wffs),
and will always represent sets when we get to set theory.  There are also
three new symbols $\forall$\index{universal quantifier ($\forall$)},
$=$\index{equality ($=$)}, and $\in$\index{stylized epsilon ($\in$)},
read ``for all,'' ``equals,'' and ``is an element of''
respectively.  We will represent variables with the letters $x$, $y$, $z$, and
$w$, as is common practice in the literature.
For example, $\forall x \varphi$ means ``for all possible values of
$x$, $\varphi$ is true.''

In predicate calculus, we extend the definition of a wff\index{well-formed
formula (wff)}.  If $\varphi$ is a wff and $x$ and $y$ are variables, then
$\forall x \, \varphi$, $x=y$, and $x\in y$ are wffs. Note that these three new
types of wffs can be considered ``starting'' wffs from which we can build
other wffs with $\rightarrow$ and $\neg$ .  The concept of a starting wff was
absent in propositional calculus.  But starting wff or not, all we are really
concerned with is whether our wffs are correctly constructed according to
these mechanical rules.

A quick aside:
To prevent confusion, it might be best at this point to think of the variables
of Metamath\index{Metamath} as ``metavariables,''\index{metavariable} because
they are not quite the same as the variables we are introducing here.  A
(meta)variable in Metamath can be a wff or an individual variable, as well
as many other things; in general, it represents a kind of place holder for an
unspecified sequence of math symbols\index{math symbol}.

Unlike propositional calculus, no decision procedure\index{decision procedure}
analogous to the truth table method exists (nor theoretically can exist) that
will definitely determine whether a formula is a theorem of predicate
calculus.  Much of the work in the field of automated theorem
proving\index{automated theorem proving} has been dedicated to coming up with
clever heuristics for proving theorems of predicate calculus, but they can
never be guaranteed to work always.

Section \ref{metadefpred} discusses various definitions
that make predicate calculus easier to use.
For example, we define
$\exists x \varphi$ to mean
``there exists at least one possible value of $x$ where $\varphi$ is true.''

We now turn to looking at how predicate calculus can be formally
represented.

\subsubsection{Common Axioms}

There is a new rule of inference in predicate calculus:  if $\varphi$ is
an axiom or a theorem, then $\forall x \,\varphi$ is also a
theorem\index{theorem}.  This is called the rule of
``generalization.''\index{rule of generalization}
This is easily represented in Metamath.

In standard texts of logic, there are often two axioms of predicate
calculus\index{axioms of predicate calculus}:
\begin{center}
  $\forall x \,\varphi ( x ) \rightarrow \varphi ( y )$,
      where ``$y$ is properly substituted for $x$.''\\
  $\forall x ( \varphi \rightarrow \psi )\rightarrow ( \varphi \rightarrow
    \forall x\, \psi )$,
    where ``$x$ is not free in $\varphi$.''
\end{center}

Now at first glance, this seems simple:  just two axioms.  However,
conditional clauses are attached to each axiom describing requirements that
may seem puzzling to you.  In addition, the first axiom puts a variable symbol
in parentheses after each wff, seemingly violating our definition of a
wff\index{well-formed formula (wff)}; this is just an informal way of
referring to some arbitrary variable that may occur in the wff.  The
conditional clauses do, of course, have a precise meaning, but as it turns out
the precise meaning is somewhat complicated and awkward to formalize in a
way that a computer can handle easily.  Unlike propositional calculus, a
certain amount of mathematical sophistication and practice is needed to be
able to easily grasp and manipulate these concepts correctly.

Predicate calculus may be presented with or without axioms for
equality\index{axioms of equality}\index{equality ($=$)}. We will require the
axioms of equality as a prerequisite for the version of set theory we will
use.  The axioms for equality, when included, are often represented using these
two axioms:
\begin{center}
$x=x$\\ \ \\
$x=y\rightarrow (\varphi(x,x)\rightarrow\varphi(x,y))$ where ``$\varphi(x,y)$
   arises from $\varphi(x,x)$ by replacing some, but not necessarily all,
   free\index{free variable}
   occurrences of $x$ by $y$,\\ provided that $y$ is free for $x$
   in $\varphi(x,x)$.'' \end{center}
% (Mendelson p. 95)
The first equality axiom is simple, but again,
the condition on the second one is
somewhat awkward to implement on a computer.

\subsubsection{Tarski System S2}

Of course, we are not the first to notice the complications of these
predicate calculus axioms when being rigorous.

Well-known logician Alfred Tarski published in 1965
a system he called system S2\cite[p.~77]{Tarski1965}.
Tarski's system is \textit{exactly equivalent} to the traditional textbook
formalization, but (by clever use of equality axioms) it eliminates the
latter's primitive notions of ``proper substitution'' and ``free variable,''
replacing them with direct substitution and the notion of a variable
not occurring in a formula (which we express with distinct variable
constraints).

In advocating his system, Tarski wrote, ``The relatively complicated
character of [free variables and proper substitution] is a source
of certain inconveniences of both practical and theoretical nature;
this is clearly experienced both in teaching an elementary course of
mathematical logic and in formalizing the syntax of predicate logic for
some theoretical purposes''\cite[p.~61]{Tarski1965}\index{Tarski, Alfred}.

\subsubsection{Developing a Metamath Representation}

The standard textbook axioms of predicate calculus are somewhat
cumbersome to implement on a computer because of the complex notions of
``free variable''\index{free variable} and ``proper
substitution.''\index{proper substitution}\index{substitution!proper}
While it is possible to use the Metamath\index{Metamath} language to
implement these concepts, we have chosen not to implement them
as primitive constructs in the
\texttt{set.mm} set theory database.  Instead, we have eliminated them
within the axioms
by carefully crafting the axioms so as to avoid them,
building on Tarski's system S2.  This makes it
easy for a beginner to follow the steps in a proof without knowing any
advanced concepts other than the simple concept of
replacing\index{substitution!variable}\index{variable substitution}
variables with expressions.

In order to develop the concepts of free variable and proper
substitution from the axioms, we use an additional
Metamath statement type called ``disjoint variable
restriction''\index{disjoint variables} that we have not encountered
before.  In the context of the axioms, the statement \texttt{\$d} $ x\,
y$\index{\texttt{\$d} statement} simply means that $x$ and $y$ must be
distinct\index{distinct variables}, i.e.\ they may not be simultaneously
substituted\index{substitution!variable}\index{variable substitution}
with the same variable.  The statement \texttt{\$d} $ x\, \varphi$ means
variable $x$ must not occur in wff $\varphi$.  For the precise
definition of \texttt{\$d}, see Section~\ref{dollard}.

\subsubsection{Metamath representation}

The Metamath axiom system for predicate calculus
defined in set.mm uses Tarski's system S2.
As noted above, this has a different representation
than the traditional textbook formalization,
but it is \textit{exactly equivalent} to the textbook formalization,
and it is \textit{much} easier to work with.
This is reproduced as system S3 in Section 6 of
Megill's formalization \cite{Megill}\index{Megill, Norman}.

There is one exception, Tarski's axiom of existence,
which we label as axiom ax-6.
In the case of ax-6, Tarski's version is weaker because it includes a
distinct variable proviso. If we wish, we can also weaken our version
in this way and still have a metalogically complete system. Theorem
ax6 shows this by deriving, in the presence of the other axioms, our
ax-6 from Tarski's weaker version ax6v. However, we chose the stronger
version for our system because it is simpler to state and easier to use.

Tarski's system was designed for proving specific theorems rather than
more general theorem schemes. However, theorem schemes are much more
efficient than specific theorems for building a body of mathematical
knowledge, since they can be reused with different instances as
needed. While Tarski does derive some theorem schemes from his axioms,
their proofs require concepts that are ``outside'' of the system, such as
induction on formula length. The verification of such proofs is difficult
to automate in a proof verifier. (Specifically, Tarski treats the formulas
of his system as set-theoretical objects. In order to verify the proofs
of his theorem schemes, a proof verifier would need a significant amount
of set theory built into it.)

The Metamath axiom system for predicate calculus extends
Tarski's system to eliminate this difficulty. The additional
``auxilliary'' axiom
schemes (as we will call them in this section; see below) endow Tarski's
system with a nice property we call
metalogical completeness \cite[Remark 9.6]{Megill}\index{Megill, Norman}.
As a result, we can prove any theorem scheme
expressable in the ``simple metalogic'' of Tarski's system by using
only Metamath's direct substitution rule applied to the axiom system
(and no other metalogical or set-theoretical notions ``outside'' of the
system). Simple metalogic consists of schemes containing wff metavariables
(with no arguments) and/or set (also called ``individual'') metavariables,
accompanied by optional provisos each stating that two specified set
metavariables must be distinct or that a specified set metavariable may
not occur in a specified wff metavariable. Metamath's logic and set theory
axiom and rule schemes are all examples of simple metalogic. The schemes
of traditional predicate calculus with equality are examples which are
not simple metalogic, because they use wff metavariables with arguments
and have ``free for'' and ``not free in'' side conditions.

A rigorous justification for this system, using an older but
exactly equivalent set of axioms, can be
found in \cite{Megill}\index{Megill, Norman}.

This allows us to
take a different approach in the Metamath\index{Metamath} database
\texttt{set.mm}\index{set theory database (\texttt{set.mm})}.  We do not
directly use the primitive notions of ``free variable''\index{free variable}
and ``proper substitution''\index{proper
substitution}\index{substitution!proper} at all as primitive constructs.
Instead, we use a set
of axioms that are almost as simple to manipulate as those of
propositional calculus.  Our axiom system avoids complex primitive
notions by effectively embedding the complexity into the axioms
themselves.  As a result, we will end up with a larger number of axioms,
but they are ideally suited for a computer language such as Metamath.
(Section~\ref{metaaxioms} shows these axioms.)

We will not elaborate further
on the ``free variable'' and ``proper substitution''
concepts here.  You may consult
\cite[ch.\ 3--4]{Hamilton}\index{Hamilton, Alan G.} (as well as
many other books) for a precise explanation
of these concepts.  If you intend to do serious mathematical work, it is wise
to become familiar with the traditional textbook approach; even though the
concepts embedded in their axioms require a higher level of sophistication,
they can be more practical to deal with on an everyday, informal basis.  Even
if you are just developing Metamath proofs, familiarity with the traditional
approach can help you arrive at a proof outline much faster, which you can
then convert to the detail required by Metamath.

We do develop proper substitution rules later on, but in set.mm
they are defined as derived constructs; they are not primitives.

You should also note that our system of predicate calculus is specifically
tailored for set theory; thus there are only two specific predicates $=$ and
$\in$ and no functions\index{function!in predicate calculus}
or constants\index{constant!in predicate calculus} unlike more general systems.
We later add these.

\subsection{Set Theory}

Traditional Zermelo--Fraenkel set theory\index{Zermelo--Fraenkel set
theory}\index{set theory} with the Axiom of Choice
has 10 axioms, which can be expressed in the
language of predicate calculus.  In this section, we will list only the
names and brief English descriptions of these axioms, since we will give
you the precise formulas used by the Metamath\index{Metamath} set theory
database \texttt{set.mm} later on.

In the descriptions of the axioms, we assume that $x$, $y$, $z$, $w$, and $v$
represent sets.  These are the same as the variables\index{variable!in set
theory} in our predicate calculus system above, except that now we informally
think of the variables as ranging over sets.  Note that the terms
``object,''\index{object} ``set,''\index{set} ``element,''\index{element}
``collection,''\index{collection} and ``family''\index{family} are synonymous,
as are ``is an element of,'' ``is a member of,''\index{member} ``is contained
in,'' and ``belongs to.''  The different terms are used for convenience; for
example, ``a collection of sets'' is less confusing than ``a set of sets.''
A set $x$ is said to be a ``subset''\index{subset} of $y$ if every element of
$x$ is also an element of $y$; we also say $x$ is ``included in''
$y$.

The axioms are very general and apply to almost any conceivable mathematical
object, and this level of abstraction can be overwhelming at first.  To gain an
intuitive feel, it can be helpful to draw a picture illustrating the concept;
for example, a circle containing dots could represent a collection of sets,
and a smaller circle drawn inside the circle could represent a subset.
Overlapping circles can illustrate intersection and union.  Circles that
illustrate the concepts of set theory are frequently used in elementary
textbooks and are called Venn diagrams\index{Venn diagram}.\index{axioms of
set theory}

1. Axiom of Extensionality:  Two sets are identical if they contain the same
   elements.\index{Axiom of Extensionality}

2. Axiom of Pairing:  The set $\{ x , y \}$ exists.\index{Axiom of Pairing}

3. Axiom of Power Sets:  The power set of a set (the collection of all of
   its subsets) exists.  For example, the power set of $\{x,y\}$ is
   $\{\varnothing,\{x\},\{y\},\{x,y\}\}$ and it exists.\index{Axiom
of Power Sets}

4. Axiom of the Null Set:  The empty set $\varnothing$ exists.\index{Axiom of
the Null Set}

5. Axiom of Union:  The union of a set (the set containing the elements of
   its members) exists.  For example, the union of $\{\{x,y\},\{z\}\}$ is
 $\{x,y,z\}$ and
   it exists.\index{Axiom of Union}

6. Axiom of Regularity:  Roughly, no set can contain itself, nor can there
   be membership ``loops,'' such as a set being an
   element of one of its members.\index{Axiom of Regularity}

7. Axiom of Infinity:  An infinite set exists.  An example of an infinite
   set is the set of all
   integers.\index{Axiom of Infinity}

8. Axiom of Separation:  The set exists that is obtained by restricting $x$
   with some property.  For example, if the set of all integers exists,
   then the set of all even integers exists.\index{Axiom of Separation}

9. Axiom of Replacement:  The range of a function whose domain is restricted
   to the elements of a set $x$, is also a set.  For example, there
   is a function
   from integers (the function's domain) to their squares (its
   range).  If we
   restrict the domain to even integers, its range will become the set of
   squares of even integers, so this axiom asserts that the set of
    squares of even numbers exists.  Technical note:  In general, the
   ``function'' need not be a set but can be a proper class.
   \index{Axiom of Replacement}

10. Axiom of Choice:  Let $x$ be a set whose members are pairwise
  disjoint\index{disjoint sets} (i.e,
  whose members contain no elements in common).  Then there exists another
  set containing one element from each member of $x$.  For
  example, if $x$ is
  $\{\{y,z\},\{w,v\}\}$, where $y$, $z$, $w$, and $v$ are
  different sets, then a set such as $\{z,w\}$
  exists (but the axiom doesn't tell
  us which one).  (Actually the Axiom
  of Choice is redundant if the set $x$, as in this example, has a finite
  number of elements.)\index{Axiom of Choice}

The Axiom of Choice is usually considered an extension of ZF set theory rather
than a proper part of it.  It is sometimes considered philosophically
controversial because it specifies the existence of a set without specifying
what the set is. Constructive logics, including intuitionistic logic,
do not accept the axiom of choice.
Since there is some lingering controversy, we often prefer proofs that do
not use the axiom of choice (where there is a known alternative), and
in some cases we will use weaker axioms than the full axiom of choice.
That said, the axiom of choice is a powerful and widely-accepted tool,
so we do use it when needed.
ZF set theory that includes the Axiom of Choice is
called Zermelo--Fraenkel set theory with choice (ZFC\index{ZFC set theory}).

When expressed symbolically, the Axiom of Separation and the Axiom of
Replacement contain wff symbols and therefore each represent infinitely many
axioms, one for each possible wff. For this reason, they are often called
axiom schemes\index{axiom scheme}\index{well-formed formula (wff)}.

It turns out that the Axiom of the Null Set, the Axiom of Pairing, and the
Axiom of Separation can be derived from the other axioms and are therefore
unnecessary, although they tend to be included in standard texts for various
reasons (historical, philosophical, and possibly because some authors may not
know this).  In the Metamath\index{Metamath} set theory database, these
redundant axioms are derived from the other ones instead of truly
being considered axioms.
This is in keeping with our general goal of minimizing the number of
axioms we must depend on.

\subsection{Other Axioms}

Above we qualified the phrase ``all of mathematics'' with ``essentially.''
The main important missing piece is the ability to do category theory,
which requires huge sets (inaccessible cardinals) larger than those
postulated by the ZFC axioms. The Tarski--Grothendieck Axiom postulates
the existence of such sets.
Note that this is the same axiom used by Mizar for supporting
category theory.
The Tarski--Grothendieck axiom
can be viewed as a very strong replacement of the Axiom of Infinity,
the Axiom of Choice, and the Axiom of Power Sets.
The \texttt{set.mm} database includes this axiom; see the database
for details about it.
Again, we only use this axiom when we need to.
You are only likely to encounter or use this axiom if you are doing
category theory, since its use is highly specialized,
so we will not list the Tarsky-Grothendieck axiom
in the short list of axioms below.

Can there be even more axioms?
Of course.
G\"{o}del showed that no finite set of axioms or axiom schemes can completely
describe any consistent theory strong enough to include arithmetic.
But practically speaking, the ones above are the accepted foundation that
almost all mathematicians explicitly or implicitly base their work on.

\section{The Axioms in the Metamath Language}\label{metaaxioms}

Here we list the axioms as they appear in
\texttt{set.mm}\index{set theory database (\texttt{set.mm})} so you can
look them up there easily.  Incidentally, the \texttt{show statement
/tex} command\index{\texttt{show statement} command} was used to
typeset them.

%macros from show statement /tex
\newbox\mlinebox
\newbox\mtrialbox
\newbox\startprefix  % Prefix for first line of a formula
\newbox\contprefix  % Prefix for continuation line of a formula
\def\startm{  % Initialize formula line
  \setbox\mlinebox=\hbox{\unhcopy\startprefix}
}
\def\m#1{  % Add a symbol to the formula
  \setbox\mtrialbox=\hbox{\unhcopy\mlinebox $\,#1$}
  \ifdim\wd\mtrialbox>\hsize
    \box\mlinebox
    \setbox\mlinebox=\hbox{\unhcopy\contprefix $\,#1$}
  \else
    \setbox\mlinebox=\hbox{\unhbox\mtrialbox}
  \fi
}
\def\endm{  % Output the last line of a formula
  \box\mlinebox
}

% \SLASH for \ , \TOR for \/ (text OR), \TAND for /\ (text and)
% This embeds a following forced space to force the space.
\newcommand\SLASH{\char`\\~}
\newcommand\TOR{\char`\\/~}
\newcommand\TAND{/\char`\\~}
%
% Macro to output metamath raw text.
% This assumes \startprefix and \contprefix are set.
% NOTE: "\" is tricky to escape, use \SLASH, \TOR, and \TAND inside.
% Any use of "$ { ~ ^" must be escaped; ~ and ^ must be escaped specially.
% We escape { and } for consistency.
% For more about how this macro written, see:
% https://stackoverflow.com/questions/4073674/
% how-to-disable-indentation-in-particular-section-in-latex/4075706
% Use frenchspacing, or "e." will get an extra space after it.
\newlength\mystoreparindent
\newlength\mystorehangindent
\newenvironment{mmraw}{%
\setlength{\mystoreparindent}{\the\parindent}
\setlength{\mystorehangindent}{\the\hangindent}
\setlength{\parindent}{0pt} % TODO - we'll put in the \startprefix instead
\setlength{\hangindent}{\wd\the\contprefix}
\begin{flushleft}
\begin{frenchspacing}
\begin{tt}
{\unhcopy\startprefix}%
}{%
\end{tt}
\end{frenchspacing}
\end{flushleft}
\setlength{\parindent}{\mystoreparindent}
\setlength{\hangindent}{\mystorehangindent}
\vskip 1ex
}

\needspace{5\baselineskip}
\subsection{Propositional Calculus}\label{propcalc}\index{axioms of
propositional calculus}

\needspace{2\baselineskip}
Axiom of Simplification.\label{ax1}

\setbox\startprefix=\hbox{\tt \ \ ax-1\ \$a\ }
\setbox\contprefix=\hbox{\tt \ \ \ \ \ \ \ \ \ \ }
\startm
\m{\vdash}\m{(}\m{\varphi}\m{\rightarrow}\m{(}\m{\psi}\m{\rightarrow}\m{\varphi}\m{)}
\m{)}
\endm

\needspace{3\baselineskip}
\noindent Axiom of Distribution.

\setbox\startprefix=\hbox{\tt \ \ ax-2\ \$a\ }
\setbox\contprefix=\hbox{\tt \ \ \ \ \ \ \ \ \ \ }
\startm
\m{\vdash}\m{(}\m{(}\m{\varphi}\m{\rightarrow}\m{(}\m{\psi}\m{\rightarrow}\m{\chi}
\m{)}\m{)}\m{\rightarrow}\m{(}\m{(}\m{\varphi}\m{\rightarrow}\m{\psi}\m{)}\m{
\rightarrow}\m{(}\m{\varphi}\m{\rightarrow}\m{\chi}\m{)}\m{)}\m{)}
\endm

\needspace{2\baselineskip}
\noindent Axiom of Contraposition.

\setbox\startprefix=\hbox{\tt \ \ ax-3\ \$a\ }
\setbox\contprefix=\hbox{\tt \ \ \ \ \ \ \ \ \ \ }
\startm
\m{\vdash}\m{(}\m{(}\m{\lnot}\m{\varphi}\m{\rightarrow}\m{\lnot}\m{\psi}\m{)}\m{
\rightarrow}\m{(}\m{\psi}\m{\rightarrow}\m{\varphi}\m{)}\m{)}
\endm


\needspace{4\baselineskip}
\noindent Rule of Modus Ponens.\label{axmp}\index{modus ponens}

\setbox\startprefix=\hbox{\tt \ \ min\ \$e\ }
\setbox\contprefix=\hbox{\tt \ \ \ \ \ \ \ \ \ }
\startm
\m{\vdash}\m{\varphi}
\endm

\setbox\startprefix=\hbox{\tt \ \ maj\ \$e\ }
\setbox\contprefix=\hbox{\tt \ \ \ \ \ \ \ \ \ }
\startm
\m{\vdash}\m{(}\m{\varphi}\m{\rightarrow}\m{\psi}\m{)}
\endm

\setbox\startprefix=\hbox{\tt \ \ ax-mp\ \$a\ }
\setbox\contprefix=\hbox{\tt \ \ \ \ \ \ \ \ \ \ \ }
\startm
\m{\vdash}\m{\psi}
\endm


\needspace{7\baselineskip}
\subsection{Axioms of Predicate Calculus with Equality---Tarski's S2}\index{axioms of predicate calculus}

\needspace{3\baselineskip}
\noindent Rule of Generalization.\index{rule of generalization}

\setbox\startprefix=\hbox{\tt \ \ ax-g.1\ \$e\ }
\setbox\contprefix=\hbox{\tt \ \ \ \ \ \ \ \ \ \ \ \ }
\startm
\m{\vdash}\m{\varphi}
\endm

\setbox\startprefix=\hbox{\tt \ \ ax-gen\ \$a\ }
\setbox\contprefix=\hbox{\tt \ \ \ \ \ \ \ \ \ \ \ \ }
\startm
\m{\vdash}\m{\forall}\m{x}\m{\varphi}
\endm

\needspace{2\baselineskip}
\noindent Axiom of Quantified Implication.

\setbox\startprefix=\hbox{\tt \ \ ax-4\ \$a\ }
\setbox\contprefix=\hbox{\tt \ \ \ \ \ \ \ \ \ \ }
\startm
\m{\vdash}\m{(}\m{\forall}\m{x}\m{(}\m{\forall}\m{x}\m{\varphi}\m{\rightarrow}\m{
\psi}\m{)}\m{\rightarrow}\m{(}\m{\forall}\m{x}\m{\varphi}\m{\rightarrow}\m{
\forall}\m{x}\m{\psi}\m{)}\m{)}
\endm

\needspace{3\baselineskip}
\noindent Axiom of Distinctness.

% Aka: Add $d x ph $.
\setbox\startprefix=\hbox{\tt \ \ ax-5\ \$a\ }
\setbox\contprefix=\hbox{\tt \ \ \ \ \ \ \ \ \ \ }
\startm
\m{\vdash}\m{(}\m{\varphi}\m{\rightarrow}\m{\forall}\m{x}\m{\varphi}\m{)}\m{where}\m{ }\m{\$d}\m{ }\m{x}\m{ }\m{\varphi}\m{ }\m{(}\m{x}\m{ }\m{does}\m{ }\m{not}\m{ }\m{occur}\m{ }\m{in}\m{ }\m{\varphi}\m{)}
\endm

\needspace{2\baselineskip}
\noindent Axiom of Existence.

\setbox\startprefix=\hbox{\tt \ \ ax-6\ \$a\ }
\setbox\contprefix=\hbox{\tt \ \ \ \ \ \ \ \ \ \ }
\startm
\m{\vdash}\m{(}\m{\forall}\m{x}\m{(}\m{x}\m{=}\m{y}\m{\rightarrow}\m{\forall}
\m{x}\m{\varphi}\m{)}\m{\rightarrow}\m{\varphi}\m{)}
\endm

\needspace{2\baselineskip}
\noindent Axiom of Equality.

\setbox\startprefix=\hbox{\tt \ \ ax-7\ \$a\ }
\setbox\contprefix=\hbox{\tt \ \ \ \ \ \ \ \ \ \ }
\startm
\m{\vdash}\m{(}\m{x}\m{=}\m{y}\m{\rightarrow}\m{(}\m{x}\m{=}\m{z}\m{
\rightarrow}\m{y}\m{=}\m{z}\m{)}\m{)}
\endm

\needspace{2\baselineskip}
\noindent Axiom of Left Equality for Binary Predicate.

\setbox\startprefix=\hbox{\tt \ \ ax-8\ \$a\ }
\setbox\contprefix=\hbox{\tt \ \ \ \ \ \ \ \ \ \ \ }
\startm
\m{\vdash}\m{(}\m{x}\m{=}\m{y}\m{\rightarrow}\m{(}\m{x}\m{\in}\m{z}\m{
\rightarrow}\m{y}\m{\in}\m{z}\m{)}\m{)}
\endm

\needspace{2\baselineskip}
\noindent Axiom of Right Equality for Binary Predicate.

\setbox\startprefix=\hbox{\tt \ \ ax-9\ \$a\ }
\setbox\contprefix=\hbox{\tt \ \ \ \ \ \ \ \ \ \ \ }
\startm
\m{\vdash}\m{(}\m{x}\m{=}\m{y}\m{\rightarrow}\m{(}\m{z}\m{\in}\m{x}\m{
\rightarrow}\m{z}\m{\in}\m{y}\m{)}\m{)}
\endm


\needspace{4\baselineskip}
\subsection{Axioms of Predicate Calculus with Equality---Auxiliary}\index{axioms of predicate calculus - auxiliary}

\needspace{2\baselineskip}
\noindent Axiom of Quantified Negation.

\setbox\startprefix=\hbox{\tt \ \ ax-10\ \$a\ }
\setbox\contprefix=\hbox{\tt \ \ \ \ \ \ \ \ \ \ }
\startm
\m{\vdash}\m{(}\m{\lnot}\m{\forall}\m{x}\m{\lnot}\m{\forall}\m{x}\m{\varphi}\m{
\rightarrow}\m{\varphi}\m{)}
\endm

\needspace{2\baselineskip}
\noindent Axiom of Quantifier Commutation.

\setbox\startprefix=\hbox{\tt \ \ ax-11\ \$a\ }
\setbox\contprefix=\hbox{\tt \ \ \ \ \ \ \ \ \ \ }
\startm
\m{\vdash}\m{(}\m{\forall}\m{x}\m{\forall}\m{y}\m{\varphi}\m{\rightarrow}\m{
\forall}\m{y}\m{\forall}\m{x}\m{\varphi}\m{)}
\endm

\needspace{3\baselineskip}
\noindent Axiom of Substitution.

\setbox\startprefix=\hbox{\tt \ \ ax-12\ \$a\ }
\setbox\contprefix=\hbox{\tt \ \ \ \ \ \ \ \ \ \ \ }
\startm
\m{\vdash}\m{(}\m{\lnot}\m{\forall}\m{x}\m{\,x}\m{=}\m{y}\m{\rightarrow}\m{(}
\m{x}\m{=}\m{y}\m{\rightarrow}\m{(}\m{\varphi}\m{\rightarrow}\m{\forall}\m{x}\m{(}
\m{x}\m{=}\m{y}\m{\rightarrow}\m{\varphi}\m{)}\m{)}\m{)}\m{)}
\endm

\needspace{3\baselineskip}
\noindent Axiom of Quantified Equality.

\setbox\startprefix=\hbox{\tt \ \ ax-13\ \$a\ }
\setbox\contprefix=\hbox{\tt \ \ \ \ \ \ \ \ \ \ \ }
\startm
\m{\vdash}\m{(}\m{\lnot}\m{\forall}\m{z}\m{\,z}\m{=}\m{x}\m{\rightarrow}\m{(}
\m{\lnot}\m{\forall}\m{z}\m{\,z}\m{=}\m{y}\m{\rightarrow}\m{(}\m{x}\m{=}\m{y}
\m{\rightarrow}\m{\forall}\m{z}\m{\,x}\m{=}\m{y}\m{)}\m{)}\m{)}
\endm

% \noindent Axiom of Quantifier Substitution
%
% \setbox\startprefix=\hbox{\tt \ \ ax-c11n\ \$a\ }
% \setbox\contprefix=\hbox{\tt \ \ \ \ \ \ \ \ \ \ \ }
% \startm
% \m{\vdash}\m{(}\m{\forall}\m{x}\m{\,x}\m{=}\m{y}\m{\rightarrow}\m{(}\m{\forall}
% \m{x}\m{\varphi}\m{\rightarrow}\m{\forall}\m{y}\m{\varphi}\m{)}\m{)}
% \endm
%
% \noindent Axiom of Distinct Variables. (This axiom requires
% that two individual variables
% be distinct\index{\texttt{\$d} statement}\index{distinct
% variables}.)
%
% \setbox\startprefix=\hbox{\tt \ \ \ \ \ \ \ \ \$d\ }
% \setbox\contprefix=\hbox{\tt \ \ \ \ \ \ \ \ \ \ \ }
% \startm
% \m{x}\m{\,}\m{y}
% \endm
%
% \setbox\startprefix=\hbox{\tt \ \ ax-c16\ \$a\ }
% \setbox\contprefix=\hbox{\tt \ \ \ \ \ \ \ \ \ \ \ }
% \startm
% \m{\vdash}\m{(}\m{\forall}\m{x}\m{\,x}\m{=}\m{y}\m{\rightarrow}\m{(}\m{\varphi}\m{
% \rightarrow}\m{\forall}\m{x}\m{\varphi}\m{)}\m{)}
% \endm

% \noindent Axiom of Quantifier Introduction (2).  (This axiom requires
% that the individual variable not occur in the
% wff\index{\texttt{\$d} statement}\index{distinct variables}.)
%
% \setbox\startprefix=\hbox{\tt \ \ \ \ \ \ \ \ \$d\ }
% \setbox\contprefix=\hbox{\tt \ \ \ \ \ \ \ \ \ \ \ }
% \startm
% \m{x}\m{\,}\m{\varphi}
% \endm
% \setbox\startprefix=\hbox{\tt \ \ ax-5\ \$a\ }
% \setbox\contprefix=\hbox{\tt \ \ \ \ \ \ \ \ \ \ \ }
% \startm
% \m{\vdash}\m{(}\m{\varphi}\m{\rightarrow}\m{\forall}\m{x}\m{\varphi}\m{)}
% \endm

\subsection{Set Theory}\label{mmsettheoryaxioms}

In order to make the axioms of set theory\index{axioms of set theory} a little
more compact, there are several definitions from logic that we make use of
implicitly, namely, ``logical {\sc and},''\index{conjunction ($\wedge$)}
\index{logical {\sc and} ($\wedge$)} ``logical equivalence,''\index{logical
equivalence ($\leftrightarrow$)}\index{biconditional ($\leftrightarrow$)} and
``there exists.''\index{existential quantifier ($\exists$)}

\begin{center}\begin{tabular}{rcl}
  $( \varphi \wedge \psi )$ &\mbox{stands for}& $\neg ( \varphi
     \rightarrow \neg \psi )$\\
  $( \varphi \leftrightarrow \psi )$& \mbox{stands
     for}& $( ( \varphi \rightarrow \psi ) \wedge
     ( \psi \rightarrow \varphi ) )$\\
  $\exists x \,\varphi$ &\mbox{stands for}& $\neg \forall x \neg \varphi$
\end{tabular}\end{center}

In addition, the axioms of set theory require that all variables be
dis\-tinct,\index{distinct variables}\footnote{Set theory axioms can be
devised so that {\em no} variables are required to be distinct,
provided we replace \texttt{ax-c16} with an axiom stating that ``at
least two things exist,'' thus
making \texttt{ax-5} the only other axiom requiring the
\texttt{\$d} statement.  These axioms are unconventional and are not
presented here, but they can be found on the \url{http://metamath.org}
web site.  See also the Comment on
p.~\pageref{nodd}.}\index{\texttt{\$d} statement} thus we also assume:
\begin{center}
  \texttt{\$d }$x\,y\,z\,w$
\end{center}

\needspace{2\baselineskip}
\noindent Axiom of Extensionality.\index{Axiom of Extensionality}

\setbox\startprefix=\hbox{\tt \ \ ax-ext\ \$a\ }
\setbox\contprefix=\hbox{\tt \ \ \ \ \ \ \ \ \ \ \ \ }
\startm
\m{\vdash}\m{(}\m{\forall}\m{x}\m{(}\m{x}\m{\in}\m{y}\m{\leftrightarrow}\m{x}
\m{\in}\m{z}\m{)}\m{\rightarrow}\m{y}\m{=}\m{z}\m{)}
\endm

\needspace{3\baselineskip}
\noindent Axiom of Replacement.\index{Axiom of Replacement}

\setbox\startprefix=\hbox{\tt \ \ ax-rep\ \$a\ }
\setbox\contprefix=\hbox{\tt \ \ \ \ \ \ \ \ \ \ \ \ }
\startm
\m{\vdash}\m{(}\m{\forall}\m{w}\m{\exists}\m{y}\m{\forall}\m{z}\m{(}\m{%
\forall}\m{y}\m{\varphi}\m{\rightarrow}\m{z}\m{=}\m{y}\m{)}\m{\rightarrow}\m{%
\exists}\m{y}\m{\forall}\m{z}\m{(}\m{z}\m{\in}\m{y}\m{\leftrightarrow}\m{%
\exists}\m{w}\m{(}\m{w}\m{\in}\m{x}\m{\wedge}\m{\forall}\m{y}\m{\varphi}\m{)}%
\m{)}\m{)}
\endm

\needspace{2\baselineskip}
\noindent Axiom of Union.\index{Axiom of Union}

\setbox\startprefix=\hbox{\tt \ \ ax-un\ \$a\ }
\setbox\contprefix=\hbox{\tt \ \ \ \ \ \ \ \ \ \ \ }
\startm
\m{\vdash}\m{\exists}\m{x}\m{\forall}\m{y}\m{(}\m{\exists}\m{x}\m{(}\m{y}\m{
\in}\m{x}\m{\wedge}\m{x}\m{\in}\m{z}\m{)}\m{\rightarrow}\m{y}\m{\in}\m{x}\m{)}
\endm

\needspace{2\baselineskip}
\noindent Axiom of Power Sets.\index{Axiom of Power Sets}

\setbox\startprefix=\hbox{\tt \ \ ax-pow\ \$a\ }
\setbox\contprefix=\hbox{\tt \ \ \ \ \ \ \ \ \ \ \ \ }
\startm
\m{\vdash}\m{\exists}\m{x}\m{\forall}\m{y}\m{(}\m{\forall}\m{x}\m{(}\m{x}\m{
\in}\m{y}\m{\rightarrow}\m{x}\m{\in}\m{z}\m{)}\m{\rightarrow}\m{y}\m{\in}\m{x}
\m{)}
\endm

\needspace{3\baselineskip}
\noindent Axiom of Regularity.\index{Axiom of Regularity}

\setbox\startprefix=\hbox{\tt \ \ ax-reg\ \$a\ }
\setbox\contprefix=\hbox{\tt \ \ \ \ \ \ \ \ \ \ \ \ }
\startm
\m{\vdash}\m{(}\m{\exists}\m{x}\m{\,x}\m{\in}\m{y}\m{\rightarrow}\m{\exists}
\m{x}\m{(}\m{x}\m{\in}\m{y}\m{\wedge}\m{\forall}\m{z}\m{(}\m{z}\m{\in}\m{x}\m{
\rightarrow}\m{\lnot}\m{z}\m{\in}\m{y}\m{)}\m{)}\m{)}
\endm

\needspace{3\baselineskip}
\noindent Axiom of Infinity.\index{Axiom of Infinity}

\setbox\startprefix=\hbox{\tt \ \ ax-inf\ \$a\ }
\setbox\contprefix=\hbox{\tt \ \ \ \ \ \ \ \ \ \ \ \ \ \ \ }
\startm
\m{\vdash}\m{\exists}\m{x}\m{(}\m{y}\m{\in}\m{x}\m{\wedge}\m{\forall}\m{y}%
\m{(}\m{y}\m{\in}\m{x}\m{\rightarrow}\m{\exists}\m{z}\m{(}\m{y}\m{\in}\m{z}\m{%
\wedge}\m{z}\m{\in}\m{x}\m{)}\m{)}\m{)}
\endm

\needspace{4\baselineskip}
\noindent Axiom of Choice.\index{Axiom of Choice}

\setbox\startprefix=\hbox{\tt \ \ ax-ac\ \$a\ }
\setbox\contprefix=\hbox{\tt \ \ \ \ \ \ \ \ \ \ \ \ \ \ }
\startm
\m{\vdash}\m{\exists}\m{x}\m{\forall}\m{y}\m{\forall}\m{z}\m{(}\m{(}\m{y}\m{%
\in}\m{z}\m{\wedge}\m{z}\m{\in}\m{w}\m{)}\m{\rightarrow}\m{\exists}\m{w}\m{%
\forall}\m{y}\m{(}\m{\exists}\m{w}\m{(}\m{(}\m{y}\m{\in}\m{z}\m{\wedge}\m{z}%
\m{\in}\m{w}\m{)}\m{\wedge}\m{(}\m{y}\m{\in}\m{w}\m{\wedge}\m{w}\m{\in}\m{x}%
\m{)}\m{)}\m{\leftrightarrow}\m{y}\m{=}\m{w}\m{)}\m{)}
\endm

\subsection{That's It}

There you have it, the axioms for (essentially) all of mathematics!
Wonder at them and stare at them in awe.  Put a copy in your wallet, and
you will carry in your pocket the encoding for all theorems ever proved
and that ever will be proved, from the most mundane to the most
profound.

\section{A Hierarchy of Definitions}\label{hierarchy}

The axioms in the previous section in principle embody everything that can be
done within standard mathematics.  However, it is impractical to accomplish
very much by using them directly, for even simple concepts (from a human
perspective) can involve extremely long, incomprehensible formulas.
Mathematics is made practical by introducing definitions\index{definition}.
Definitions usually introduce new symbols, or at least new relationships among
existing symbols, to abbreviate more complex formulas.  An important
requirement for a definition is that there exist a straightforward
(algorithmic) method for eliminating the abbreviation by expanding it into the
more primitive symbol string that it represents.  Some
important definitions included in
the file \texttt{set.mm} are listed in this section for reference, and also to
give you a feel for why something like $\omega$\index{omega ($\omega$)} (the
set of natural numbers\index{natural number} 0, 1, 2,\ldots) becomes very
complicated when completely expanded into primitive symbols.

What is the motivation for definitions, aside from allowing complicated
expressions to be expressed more simply?  In the case of  $\omega$, one goal is
to provide a basis for the theory of natural numbers.\index{natural number}
Before set theory was invented, a set of axioms for arithmetic, called Peano's
postulates\index{Peano's postulates}, was devised and shown to have the
properties one expects for natural numbers.  Now anyone can postulate a
set of axioms, but if the axioms are inconsistent contradictions can be derived
from them.  Once a contradiction is derived, anything can be trivially
proved, including
all the facts of arithmetic and their negations.  To ensure that an
axiom system is at least as reliable as the axioms for set theory, we can
define sets and operations on those sets that satisfy the new axioms. In the
\texttt{set.mm} Metamath database, we prove that the elements of $\omega$ satisfy
Peano's postulates, and it's a long and hard journey to get there directly
from the axioms of set theory.  But the result is confidence in the
foundations of arithmetic.  And there is another advantage:  we now have all
the tools of set theory at our disposal for manipulating objects that obey the
axioms for arithmetic.

What are the criteria we use for definitions?  First, and of utmost importance,
the definition should not be {\em creative}\index{creative
definition}\index{definition!creative}, that
is it should not allow an expression that previously qualified as a wff but
was not provable, to become provable.   Second, the definition should be {\em
eliminable}\index{definition!eliminability}, that is, there should exist an
algorithmic method for converting any expression using the definition into
a logically equivalent expression that previously qualified as a wff.

In almost all cases below, definitions connect two expressions with either
$\leftrightarrow$ or $=$.  Eliminating\footnote{Here we mean the
elimination that a human might do in his or her head.  To eliminate them as
part of a Metamath proof we would invoke one of a number of
theorems that deal with transitivity of equivalence or equality; there are
many such examples in the proofs in \texttt{set.mm}.} such a definition is a
simple matter of substituting the expression on the left-hand side ({\em
definiendum}\index{definiendum} or thing being defined) with the equivalent,
more primitive expression on the right-hand side ({\em
definiens}\index{definiens} or definition).

Often a definition has variables on the right-hand side which do not appear on
the left-hand side; these are called {\em dummy variables}.\index{dummy
variable!in definitions}  In this case, any
allowable substitution (such as a new, distinct
variable) can be used when the definition is eliminated.  Dummy variables may
be used only if they are {\em effectively bound}\index{effectively bound
variable}, meaning that the definition will remain logically equivalent upon
any substitution of a dummy variable with any other {\em qualifying
expression}\index{qualifying expression}, i.e.\ any symbol string (such as
another variable) that
meets the restrictions on the dummy variable imposed by \texttt{\$d} and
\texttt{\$f} statements.  For example, we could define a constant $\perp$
(inverted tee, meaning logical ``false'') as $( \varphi \wedge \lnot \varphi
)$, i.e.\ ``phi and not phi.''  Here $\varphi$ is effectively bound because the
definition remains logically equivalent when we replace $\varphi$ with any
other wff.  (It is actually \texttt{df-fal}
in \texttt{set.mm}, which defines $\perp$.)

There are two cases where eliminating definitions is a little more
complex.  These cases are the definitions \texttt{df-bi} and
\texttt{df-cleq}.  The first stretches the concept of a definition a
little, as in effect it ``defines a definition;'' however, it meets our
requirements for a definition in that it is eliminable and does not
strengthen the language.  Theorem \texttt{bii} shows the substitution
needed to eliminate the $\leftrightarrow$\index{logical equivalence
($\leftrightarrow$)}\index{biconditional ($\leftrightarrow$)} symbol.

Definition \texttt{df-cleq}\index{equality ($=$)} extends the usage of
the equality symbol to include ``classes''\index{class} in set theory.  The
reason it is potentially problematic is that it can lead to statements which
do not follow from logic alone but presuppose the Axiom of
Extensionality\index{Axiom of Extensionality}, so we include this axiom
as a hypothesis for the definition.  We could have made \texttt{df-cleq} directly
eliminable by introducing a new equality symbol, but have chosen not to do so
in keeping with standard textbook practice.  Definitions such as \texttt{df-cleq}
that extend the meaning of existing symbols must be introduced carefully so
that they do not lead to contradictions.  Definition \texttt{df-clel} also
extends the meaning of an existing symbol ($\in$); while it doesn't strengthen
the language like \texttt{df-cleq}, this is not obvious and it must also be
subject to the same scrutiny.

Exercise:  Study how the wff $x\in\omega$, meaning ``$x$ is a natural
number,'' could be expanded in terms of primitive symbols, starting with the
definitions \texttt{df-clel} on p.~\pageref{dfclel} and \texttt{df-om} on
p.~\pageref{dfom} and working your way back.  Don't bother to work out the
details; just make sure that you understand how you could do it in principle.
The answer is shown in the footnote on p.~\pageref{expandom}.  If you
actually do work it out, you won't get exactly the same answer because we used
a few simplifications such as discarding occurrences of $\lnot\lnot$ (double
negation).

In the definitions below, we have placed the {\sc ascii} Metamath source
below each of the formulas to help you become familiar with the
notation in the database.  For simplicity, the necessary \texttt{\$f}
and \texttt{\$d} statements are not shown.  If you are in doubt, use the
\texttt{show statement}\index{\texttt{show statement} command} command
in the Metamath program to see the full statement.
A selection of this notation is summarized in Appendix~\ref{ASCII}.

To understand the motivation for these definitions, you should consult the
references indicated:  Takeuti and Zaring \cite{Takeuti}\index{Takeuti, Gaisi},
Quine \cite{Quine}\index{Quine, Willard Van Orman}, Bell and Machover
\cite{Bell}\index{Bell, J. L.}, and Enderton \cite{Enderton}\index{Enderton,
Herbert B.}.  Our list of definitions is provided more for reference than as a
learning aid.  However, by looking at a few of them you can gain a feel for
how the hierarchy is built up.  The definitions are a representative sample of
the many definitions
in \texttt{set.mm}, but they are complete with respect to the
theorem examples we will present in Section~\ref{sometheorems}.  Also, some are
slightly different from, but logically equivalent to, the ones in \texttt{set.mm}
(some of which have been revised over time to shorten them, for example).

\subsection{Definitions for Propositional Calculus}\label{metadefprop}

The symbols $\varphi$, $\psi$, and $\chi$ represent wffs.

Our first definition introduces the biconditional
connective\footnote{The term ``connective'' is informally used to mean a
symbol that is placed between two variables or adjacent to a variable,
whereas a mathematical ``constant'' usually indicates a symbol such as
the number 0 that may replace a variable or metavariable.  From
Metamath's point of view, there is no distinction between a connective
and a constant; both are constants in the Metamath
language.}\index{connective}\index{constant} (also called logical
equivalence)\index{logical equivalence
($\leftrightarrow$)}\index{biconditional ($\leftrightarrow$)}.  Unlike
most traditional developments, we have chosen not to have a separate
symbol such as ``Df.'' to mean ``is defined as.''  Instead, we will use
the biconditional connective for this purpose, as it lets us use
logic to manipulate definitions directly.  Here we state the properties
of the biconditional connective with a carefully crafted \texttt{\$a}
statement, which effectively uses the biconditional connective to define
itself.  The $\leftrightarrow$ symbol can be eliminated from a formula
using theorem \texttt{bii}, which is derived later.

\vskip 2ex
\noindent Define the biconditional connective.\label{df-bi}

\vskip 0.5ex
\setbox\startprefix=\hbox{\tt \ \ df-bi\ \$a\ }
\setbox\contprefix=\hbox{\tt \ \ \ \ \ \ \ \ \ \ \ }
\startm
\m{\vdash}\m{\lnot}\m{(}\m{(}\m{(}\m{\varphi}\m{\leftrightarrow}\m{\psi}\m{)}%
\m{\rightarrow}\m{\lnot}\m{(}\m{(}\m{\varphi}\m{\rightarrow}\m{\psi}\m{)}\m{%
\rightarrow}\m{\lnot}\m{(}\m{\psi}\m{\rightarrow}\m{\varphi}\m{)}\m{)}\m{)}\m{%
\rightarrow}\m{\lnot}\m{(}\m{\lnot}\m{(}\m{(}\m{\varphi}\m{\rightarrow}\m{%
\psi}\m{)}\m{\rightarrow}\m{\lnot}\m{(}\m{\psi}\m{\rightarrow}\m{\varphi}\m{)}%
\m{)}\m{\rightarrow}\m{(}\m{\varphi}\m{\leftrightarrow}\m{\psi}\m{)}\m{)}\m{)}
\endm
\begin{mmraw}%
|- -. ( ( ( ph <-> ps ) -> -. ( ( ph -> ps ) ->
-. ( ps -> ph ) ) ) -> -. ( -. ( ( ph -> ps ) -> -. (
ps -> ph ) ) -> ( ph <-> ps ) ) ) \$.
\end{mmraw}

\noindent This theorem relates the biconditional connective to primitive
connectives and can be used to eliminate the $\leftrightarrow$ symbol from any
wff.

\vskip 0.5ex
\setbox\startprefix=\hbox{\tt \ \ bii\ \$p\ }
\setbox\contprefix=\hbox{\tt \ \ \ \ \ \ \ \ \ }
\startm
\m{\vdash}\m{(}\m{(}\m{\varphi}\m{\leftrightarrow}\m{\psi}\m{)}\m{\leftrightarrow}
\m{\lnot}\m{(}\m{(}\m{\varphi}\m{\rightarrow}\m{\psi}\m{)}\m{\rightarrow}\m{\lnot}
\m{(}\m{\psi}\m{\rightarrow}\m{\varphi}\m{)}\m{)}\m{)}
\endm
\begin{mmraw}%
|- ( ( ph <-> ps ) <-> -. ( ( ph -> ps ) -> -. ( ps -> ph ) ) ) \$= ... \$.
\end{mmraw}

\noindent Define disjunction ({\sc or}).\index{disjunction ($\vee$)}%
\index{logical or (vee)@logical {\sc or} ($\vee$)}%
\index{df-or@\texttt{df-or}}\label{df-or}

\vskip 0.5ex
\setbox\startprefix=\hbox{\tt \ \ df-or\ \$a\ }
\setbox\contprefix=\hbox{\tt \ \ \ \ \ \ \ \ \ \ \ }
\startm
\m{\vdash}\m{(}\m{(}\m{\varphi}\m{\vee}\m{\psi}\m{)}\m{\leftrightarrow}\m{(}\m{
\lnot}\m{\varphi}\m{\rightarrow}\m{\psi}\m{)}\m{)}
\endm
\begin{mmraw}%
|- ( ( ph \TOR ps ) <-> ( -. ph -> ps ) ) \$.
\end{mmraw}

\noindent Define conjunction ({\sc and}).\index{conjunction ($\wedge$)}%
\index{logical {\sc and} ($\wedge$)}%
\index{df-an@\texttt{df-an}}\label{df-an}

\vskip 0.5ex
\setbox\startprefix=\hbox{\tt \ \ df-an\ \$a\ }
\setbox\contprefix=\hbox{\tt \ \ \ \ \ \ \ \ \ \ \ }
\startm
\m{\vdash}\m{(}\m{(}\m{\varphi}\m{\wedge}\m{\psi}\m{)}\m{\leftrightarrow}\m{\lnot}
\m{(}\m{\varphi}\m{\rightarrow}\m{\lnot}\m{\psi}\m{)}\m{)}
\endm
\begin{mmraw}%
|- ( ( ph \TAND ps ) <-> -. ( ph -> -. ps ) ) \$.
\end{mmraw}

\noindent Define disjunction ({\sc or}) of 3 wffs.%
\index{df-3or@\texttt{df-3or}}\label{df-3or}

\vskip 0.5ex
\setbox\startprefix=\hbox{\tt \ \ df-3or\ \$a\ }
\setbox\contprefix=\hbox{\tt \ \ \ \ \ \ \ \ \ \ \ \ }
\startm
\m{\vdash}\m{(}\m{(}\m{\varphi}\m{\vee}\m{\psi}\m{\vee}\m{\chi}\m{)}\m{
\leftrightarrow}\m{(}\m{(}\m{\varphi}\m{\vee}\m{\psi}\m{)}\m{\vee}\m{\chi}\m{)}
\m{)}
\endm
\begin{mmraw}%
|- ( ( ph \TOR ps \TOR ch ) <-> ( ( ph \TOR ps ) \TOR ch ) ) \$.
\end{mmraw}

\noindent Define conjunction ({\sc and}) of 3 wffs.%
\index{df-3an}\label{df-3an}

\vskip 0.5ex
\setbox\startprefix=\hbox{\tt \ \ df-3an\ \$a\ }
\setbox\contprefix=\hbox{\tt \ \ \ \ \ \ \ \ \ \ \ \ }
\startm
\m{\vdash}\m{(}\m{(}\m{\varphi}\m{\wedge}\m{\psi}\m{\wedge}\m{\chi}\m{)}\m{
\leftrightarrow}\m{(}\m{(}\m{\varphi}\m{\wedge}\m{\psi}\m{)}\m{\wedge}\m{\chi}
\m{)}\m{)}
\endm

\begin{mmraw}%
|- ( ( ph \TAND ps \TAND ch ) <-> ( ( ph \TAND ps ) \TAND ch ) ) \$.
\end{mmraw}

\subsection{Definitions for Predicate Calculus}\label{metadefpred}

The symbols $x$, $y$, and $z$ represent individual variables of predicate
calculus.  In this section, they are not necessarily distinct unless it is
explicitly
mentioned.

\vskip 2ex
\noindent Define existential quantification.
The expression $\exists x \varphi$ means
``there exists an $x$ where $\varphi$ is true.''\index{existential quantifier
($\exists$)}\label{df-ex}

\vskip 0.5ex
\setbox\startprefix=\hbox{\tt \ \ df-ex\ \$a\ }
\setbox\contprefix=\hbox{\tt \ \ \ \ \ \ \ \ \ \ \ }
\startm
\m{\vdash}\m{(}\m{\exists}\m{x}\m{\varphi}\m{\leftrightarrow}\m{\lnot}\m{\forall}
\m{x}\m{\lnot}\m{\varphi}\m{)}
\endm
\begin{mmraw}%
|- ( E. x ph <-> -. A. x -. ph ) \$.
\end{mmraw}

\noindent Define proper substitution.\index{proper
substitution}\index{substitution!proper}\label{df-sb}
In our notation, we use $[ y / x ] \varphi$ to mean ``the wff that
results when $y$ is properly substituted for $x$ in the wff
$\varphi$.''\footnote{
This can also be described
as substituting $x$ with $y$, $y$ properly replaces $x$, or
$x$ is properly replaced by $y$.}
% This is elsb4, though it currently says: ( [ x / y ] z e. y <-> z e. x )
For example,
$[ y / x ] z \in x$ is the same as $z \in y$.
One way to remember this notation is to notice that it looks like division
and recall that $( y / x ) \cdot x $ is $y$ (when $x \neq 0$).
The notation is different from the notation $\varphi ( x | y )$
that is sometimes used, because the latter notation is ambiguous for us:
for example, we don't know whether $\lnot \varphi ( x | y )$ is to be
interpreted as $\lnot ( \varphi ( x | y ) )$ or
$( \lnot \varphi ) ( x | y )$.\footnote{Because of the way
we initially defined wffs, this is the case
with any postfix connective\index{postfix connective} (one occurring after the
symbols being connected) or infix connective\index{infix connective} (one
occurring between the symbols being connected).  Metamath does not have a
built-in notion of operator binding strength that could eliminate the
ambiguity.  The initial parenthesis effectively provides a prefix
connective\index{prefix connective} to eliminate ambiguity.  Some conventions,
such as Polish notation\index{Polish notation} used in the 1930's and 1940's
by Polish logicians, use only prefix connectives and thus allow the total
elimination of parentheses, at the expense of readability.  In Metamath we
could actually redefine all notation to be Polish if we wanted to without
having to change any proofs!}  Other texts often use $\varphi(y)$ to indicate
our $[ y / x ] \varphi$, but this notation is even more ambiguous since there is
no explicit indication of what is being substituted.
Note that this
definition is valid even when
$x$ and $y$ are the same variable.  The first conjunct is a ``trick'' used to
achieve this property, making the definition look somewhat peculiar at
first.

\vskip 0.5ex
\setbox\startprefix=\hbox{\tt \ \ df-sb\ \$a\ }
\setbox\contprefix=\hbox{\tt \ \ \ \ \ \ \ \ \ \ \ }
\startm
\m{\vdash}\m{(}\m{[}\m{y}\m{/}\m{x}\m{]}\m{\varphi}\m{\leftrightarrow}\m{(}%
\m{(}\m{x}\m{=}\m{y}\m{\rightarrow}\m{\varphi}\m{)}\m{\wedge}\m{\exists}\m{x}%
\m{(}\m{x}\m{=}\m{y}\m{\wedge}\m{\varphi}\m{)}\m{)}\m{)}
\endm
\begin{mmraw}%
|- ( [ y / x ] ph <-> ( ( x = y -> ph ) \TAND E. x ( x = y \TAND ph ) ) ) \$.
\end{mmraw}


\noindent Define existential uniqueness\index{existential uniqueness
quantifier ($\exists "!$)} (``there exists exactly one'').  Note that $y$ is a
variable distinct from $x$ and not occurring in $\varphi$.\label{df-eu}

\vskip 0.5ex
\setbox\startprefix=\hbox{\tt \ \ df-eu\ \$a\ }
\setbox\contprefix=\hbox{\tt \ \ \ \ \ \ \ \ \ \ \ }
\startm
\m{\vdash}\m{(}\m{\exists}\m{{!}}\m{x}\m{\varphi}\m{\leftrightarrow}\m{\exists}
\m{y}\m{\forall}\m{x}\m{(}\m{\varphi}\m{\leftrightarrow}\m{x}\m{=}\m{y}\m{)}\m{)}
\endm

\begin{mmraw}%
|- ( E! x ph <-> E. y A. x ( ph <-> x = y ) ) \$.
\end{mmraw}

\subsection{Definitions for Set Theory}\label{setdefinitions}

The symbols $x$, $y$, $z$, and $w$ represent individual variables of
predicate calculus, which in set theory are understood to be sets.
However, using only the constructs shown so far would be very inconvenient.

To make set theory more practical, we introduce the notion of a ``class.''
A class\index{class} is either a set variable (such as $x$) or an
expression of the form $\{ x | \varphi\}$ (called an ``abstraction
class''\index{abstraction class}\index{class abstraction}).  Note that
sets (i.e.\ individual variables) always exist (this is a theorem of
logic, namely $\exists y \, y = x$ for any set $x$), whereas classes may
or may not exist (i.e.\ $\exists y \, y = A$ may or may not be true).
If a class does not exist it is called a ``proper class.''\index{proper
class}\index{class!proper} Definitions \texttt{df-clab},
\texttt{df-cleq}, and \texttt{df-clel} can be used to convert an
expression containing classes into one containing only set variables and
wff metavariables.

The symbols $A$, $B$, $C$, $D$, $ F$, $G$, and $R$ are metavariables that range
over classes.  A class metavariable $A$ may be eliminated from a wff by
replacing it with $\{ x|\varphi\}$ where neither $x$ nor $\varphi$ occur in
the wff.

The theory of classes can be shown to be an eliminable and conservative
extension of set theory. The \textbf{eliminability}
property shows that for every
formula in the extended language we can build a logically equivalent
formula in the basic language; so that even if the extended language
provides more ease to convey and formulate mathematical ideas for set
theory, its expressive power does not in fact strengthen the basic
language's expressive power.
The \textbf{conservation} property shows that for
every proof of a formula of the basic language in the extended system
we can build another proof of the same formula in the basic system;
so that, concerning theorems on sets only, the deductive powers of
the extended system and of the basic system are identical. Together,
these properties mean that the extended language can be treated as a
definitional extension that is \textbf{sound}.

A rigorous justification, which we will not give here, can be found in
Levy \cite[pp.~357-366]{Levy} supplementing his informal introduction to class
theory on pp.~7-17. Two other good treatments of class theory are provided
by Quine \cite[pp.~15-21]{Quine}\index{Quine, Willard Van Orman}
and also \cite[pp.~10-14]{Takeuti}\index{Takeuti, Gaisi}.
Quine's exposition (he calls them virtual classes)
is nicely written and very readable.

In the rest of this
section, individual variables are always assumed to be distinct from
each other unless otherwise indicated.  In addition, dummy variables on the
right-hand side of a definition do not occur in the class and wff
metavariables in the definition.

The definitions we present here are a partial but self-contained
collection selected from several hundred that appear in the current
\texttt{set.mm} database.  They are adequate for a basic development of
elementary set theory.

\vskip 2ex
\noindent Define the abstraction class.\index{abstraction class}\index{class
abstraction}\label{df-clab}  $x$ and $y$
need not be distinct.  Definition 2.1 of Quine, p.~16.  This definition may
seem puzzling since it is shorter than the expression being defined and does not
buy us anything in terms of brevity.  The reason we introduce this definition
is because it fits in neatly with the extension of the $\in$ connective
provided by \texttt{df-clel}.

\vskip 0.5ex
\setbox\startprefix=\hbox{\tt \ \ df-clab\ \$a\ }
\setbox\contprefix=\hbox{\tt \ \ \ \ \ \ \ \ \ \ \ \ \ }
\startm
\m{\vdash}\m{(}\m{x}\m{\in}\m{\{}\m{y}\m{|}\m{\varphi}\m{\}}\m{%
\leftrightarrow}\m{[}\m{x}\m{/}\m{y}\m{]}\m{\varphi}\m{)}
\endm
\begin{mmraw}%
|- ( x e. \{ y | ph \} <-> [ x / y ] ph ) \$.
\end{mmraw}

\noindent Define the equality connective between classes\index{class
equality}\label{df-cleq}.  See Quine or Chapter 4 of Takeuti and Zaring for its
justification and methods for eliminating it.  This is an example of a
somewhat ``dangerous'' definition, because it extends the use of the
existing equality symbol rather than introducing a new symbol, allowing
us to make statements in the original language that may not be true.
For example, it permits us to deduce $y = z \leftrightarrow \forall x (
x \in y \leftrightarrow x \in z )$ which is not a theorem of logic but
rather presupposes the Axiom of Extensionality,\index{Axiom of
Extensionality} which we include as a hypothesis so that we can know
when this axiom is assumed in a proof (with the \texttt{show
trace{\char`\_}back} command).  We could avoid the danger by introducing
another symbol, say $\eqcirc$, in place of $=$; this
would also have the advantage of making elimination of the definition
straightforward and would eliminate the need for Extensionality as a
hypothesis.  We would then also have the advantage of being able to
identify exactly where Extensionality truly comes into play.  One of our
theorems would be $x \eqcirc y \leftrightarrow x = y$
by invoking Extensionality.  However in keeping with standard practice
we retain the ``dangerous'' definition.

\vskip 0.5ex
\setbox\startprefix=\hbox{\tt \ \ df-cleq.1\ \$e\ }
\setbox\contprefix=\hbox{\tt \ \ \ \ \ \ \ \ \ \ \ \ \ \ \ }
\startm
\m{\vdash}\m{(}\m{\forall}\m{x}\m{(}\m{x}\m{\in}\m{y}\m{\leftrightarrow}\m{x}
\m{\in}\m{z}\m{)}\m{\rightarrow}\m{y}\m{=}\m{z}\m{)}
\endm
\setbox\startprefix=\hbox{\tt \ \ df-cleq\ \$a\ }
\setbox\contprefix=\hbox{\tt \ \ \ \ \ \ \ \ \ \ \ \ \ }
\startm
\m{\vdash}\m{(}\m{A}\m{=}\m{B}\m{\leftrightarrow}\m{\forall}\m{x}\m{(}\m{x}\m{
\in}\m{A}\m{\leftrightarrow}\m{x}\m{\in}\m{B}\m{)}\m{)}
\endm
% We need to reset the startprefix and contprefix.
\setbox\startprefix=\hbox{\tt \ \ df-cleq.1\ \$e\ }
\setbox\contprefix=\hbox{\tt \ \ \ \ \ \ \ \ \ \ \ \ \ \ \ }
\begin{mmraw}%
|- ( A. x ( x e. y <-> x e. z ) -> y = z ) \$.
\end{mmraw}
\setbox\startprefix=\hbox{\tt \ \ df-cleq\ \$a\ }
\setbox\contprefix=\hbox{\tt \ \ \ \ \ \ \ \ \ \ \ \ \ }
\begin{mmraw}%
|- ( A = B <-> A. x ( x e. A <-> x e. B ) ) \$.
\end{mmraw}

\noindent Define the membership connective between classes\index{class
membership}.  Theorem 6.3 of Quine, p.~41, which we adopt as a definition.
Note that it extends the use of the existing membership symbol, but unlike
{\tt df-cleq} it does not extend the set of valid wffs of logic when the class
metavariables are replaced with set variables.\label{dfclel}\label{df-clel}

\vskip 0.5ex
\setbox\startprefix=\hbox{\tt \ \ df-clel\ \$a\ }
\setbox\contprefix=\hbox{\tt \ \ \ \ \ \ \ \ \ \ \ \ \ }
\startm
\m{\vdash}\m{(}\m{A}\m{\in}\m{B}\m{\leftrightarrow}\m{\exists}\m{x}\m{(}\m{x}
\m{=}\m{A}\m{\wedge}\m{x}\m{\in}\m{B}\m{)}\m{)}
\endm
\begin{mmraw}%
|- ( A e. B <-> E. x ( x = A \TAND x e. B ) ) \$.?
\end{mmraw}

\noindent Define inequality.

\vskip 0.5ex
\setbox\startprefix=\hbox{\tt \ \ df-ne\ \$a\ }
\setbox\contprefix=\hbox{\tt \ \ \ \ \ \ \ \ \ \ \ }
\startm
\m{\vdash}\m{(}\m{A}\m{\ne}\m{B}\m{\leftrightarrow}\m{\lnot}\m{A}\m{=}\m{B}%
\m{)}
\endm
\begin{mmraw}%
|- ( A =/= B <-> -. A = B ) \$.
\end{mmraw}

\noindent Define restricted universal quantification.\index{universal
quantifier ($\forall$)!restricted}  Enderton, p.~22.

\vskip 0.5ex
\setbox\startprefix=\hbox{\tt \ \ df-ral\ \$a\ }
\setbox\contprefix=\hbox{\tt \ \ \ \ \ \ \ \ \ \ \ \ }
\startm
\m{\vdash}\m{(}\m{\forall}\m{x}\m{\in}\m{A}\m{\varphi}\m{\leftrightarrow}\m{%
\forall}\m{x}\m{(}\m{x}\m{\in}\m{A}\m{\rightarrow}\m{\varphi}\m{)}\m{)}
\endm
\begin{mmraw}%
|- ( A. x e. A ph <-> A. x ( x e. A -> ph ) ) \$.
\end{mmraw}

\noindent Define restricted existential quantification.\index{existential
quantifier ($\exists$)!restricted}  Enderton, p.~22.

\vskip 0.5ex
\setbox\startprefix=\hbox{\tt \ \ df-rex\ \$a\ }
\setbox\contprefix=\hbox{\tt \ \ \ \ \ \ \ \ \ \ \ \ }
\startm
\m{\vdash}\m{(}\m{\exists}\m{x}\m{\in}\m{A}\m{\varphi}\m{\leftrightarrow}\m{%
\exists}\m{x}\m{(}\m{x}\m{\in}\m{A}\m{\wedge}\m{\varphi}\m{)}\m{)}
\endm
\begin{mmraw}%
|- ( E. x e. A ph <-> E. x ( x e. A \TAND ph ) ) \$.
\end{mmraw}

\noindent Define the universal class\index{universal class ($V$)}.  Definition
5.20, p.~21, of Takeuti and Zaring.\label{df-v}

\vskip 0.5ex
\setbox\startprefix=\hbox{\tt \ \ df-v\ \$a\ }
\setbox\contprefix=\hbox{\tt \ \ \ \ \ \ \ \ \ \ }
\startm
\m{\vdash}\m{{\rm V}}\m{=}\m{\{}\m{x}\m{|}\m{x}\m{=}\m{x}\m{\}}
\endm
\begin{mmraw}%
|- {\char`\_}V = \{ x | x = x \} \$.
\end{mmraw}

\noindent Define the subclass\index{subclass}\index{subset} relationship
between two classes (called the subset relation if the classes are sets i.e.\
are not proper).  Definition 5.9 of Takeuti and Zaring, p.~17.\label{df-ss}

\vskip 0.5ex
\setbox\startprefix=\hbox{\tt \ \ df-ss\ \$a\ }
\setbox\contprefix=\hbox{\tt \ \ \ \ \ \ \ \ \ \ \ }
\startm
\m{\vdash}\m{(}\m{A}\m{\subseteq}\m{B}\m{\leftrightarrow}\m{\forall}\m{x}\m{(}
\m{x}\m{\in}\m{A}\m{\rightarrow}\m{x}\m{\in}\m{B}\m{)}\m{)}
\endm
\begin{mmraw}%
|- ( A C\_ B <-> A. x ( x e. A -> x e. B ) ) \$.
\end{mmraw}

\noindent Define the union\index{union} of two classes.  Definition 5.6 of Takeuti and Zaring,
p.~16.\label{df-un}

\vskip 0.5ex
\setbox\startprefix=\hbox{\tt \ \ df-un\ \$a\ }
\setbox\contprefix=\hbox{\tt \ \ \ \ \ \ \ \ \ \ \ }
\startm
\m{\vdash}\m{(}\m{A}\m{\cup}\m{B}\m{)}\m{=}\m{\{}\m{x}\m{|}\m{(}\m{x}\m{\in}
\m{A}\m{\vee}\m{x}\m{\in}\m{B}\m{)}\m{\}}
\endm
\begin{mmraw}%
( A u. B ) = \{ x | ( x e. A \TOR x e. B ) \} \$.
\end{mmraw}

\noindent Define the intersection\index{intersection} of two classes.  Definition 5.6 of
Takeuti and Zaring, p.~16.\label{df-in}

\vskip 0.5ex
\setbox\startprefix=\hbox{\tt \ \ df-in\ \$a\ }
\setbox\contprefix=\hbox{\tt \ \ \ \ \ \ \ \ \ \ \ }
\startm
\m{\vdash}\m{(}\m{A}\m{\cap}\m{B}\m{)}\m{=}\m{\{}\m{x}\m{|}\m{(}\m{x}\m{\in}
\m{A}\m{\wedge}\m{x}\m{\in}\m{B}\m{)}\m{\}}
\endm
% Caret ^ requires special treatment
\begin{mmraw}%
|- ( A i\^{}i B ) = \{ x | ( x e. A \TAND x e. B ) \} \$.
\end{mmraw}

\noindent Define class difference\index{class difference}\index{set difference}.
Definition 5.12 of Takeuti and Zaring, p.~20.  Several notations are used in
the literature; we chose the $\setminus$ convention instead of a minus sign to
reserve the latter for later use in, e.g., arithmetic.\label{df-dif}

\vskip 0.5ex
\setbox\startprefix=\hbox{\tt \ \ df-dif\ \$a\ }
\setbox\contprefix=\hbox{\tt \ \ \ \ \ \ \ \ \ \ \ \ }
\startm
\m{\vdash}\m{(}\m{A}\m{\setminus}\m{B}\m{)}\m{=}\m{\{}\m{x}\m{|}\m{(}\m{x}\m{
\in}\m{A}\m{\wedge}\m{\lnot}\m{x}\m{\in}\m{B}\m{)}\m{\}}
\endm
\begin{mmraw}%
( A \SLASH B ) = \{ x | ( x e. A \TAND -. x e. B ) \} \$.
\end{mmraw}

\noindent Define the empty or null set\index{empty set}\index{null set}.
Compare  Definition 5.14 of Takeuti and Zaring, p.~20.\label{df-nul}

\vskip 0.5ex
\setbox\startprefix=\hbox{\tt \ \ df-nul\ \$a\ }
\setbox\contprefix=\hbox{\tt \ \ \ \ \ \ \ \ \ \ }
\startm
\m{\vdash}\m{\varnothing}\m{=}\m{(}\m{{\rm V}}\m{\setminus}\m{{\rm V}}\m{)}
\endm
\begin{mmraw}%
|- (/) = ( {\char`\_}V \SLASH {\char`\_}V ) \$.
\end{mmraw}

\noindent Define power class\index{power set}\index{power class}.  Definition 5.10 of
Takeuti and Zaring, p.~17, but we also let it apply to proper classes.  (Note
that \verb$~P$ is the symbol for calligraphic P, the tilde
suggesting ``curly;'' see Appendix~\ref{ASCII}.)\label{df-pw}

\vskip 0.5ex
\setbox\startprefix=\hbox{\tt \ \ df-pw\ \$a\ }
\setbox\contprefix=\hbox{\tt \ \ \ \ \ \ \ \ \ \ \ }
\startm
\m{\vdash}\m{{\cal P}}\m{A}\m{=}\m{\{}\m{x}\m{|}\m{x}\m{\subseteq}\m{A}\m{\}}
\endm
% Special incantation required to put ~ into the text
\begin{mmraw}%
|- \char`\~P~A = \{ x | x C\_ A \} \$.
\end{mmraw}

\noindent Define the singleton of a class\index{singleton}.  Definition 7.1 of
Quine, p.~48.  It is well-defined for proper classes, although
it is not very meaningful in this case, where it evaluates to the empty
set.

\vskip 0.5ex
\setbox\startprefix=\hbox{\tt \ \ df-sn\ \$a\ }
\setbox\contprefix=\hbox{\tt \ \ \ \ \ \ \ \ \ \ \ }
\startm
\m{\vdash}\m{\{}\m{A}\m{\}}\m{=}\m{\{}\m{x}\m{|}\m{x}\m{=}\m{A}\m{\}}
\endm
\begin{mmraw}%
|- \{ A \} = \{ x | x = A \} \$.
\end{mmraw}%

\noindent Define an unordered pair of classes\index{unordered pair}\index{pair}.  Definition
7.1 of Quine, p.~48.

\vskip 0.5ex
\setbox\startprefix=\hbox{\tt \ \ df-pr\ \$a\ }
\setbox\contprefix=\hbox{\tt \ \ \ \ \ \ \ \ \ \ \ }
\startm
\m{\vdash}\m{\{}\m{A}\m{,}\m{B}\m{\}}\m{=}\m{(}\m{\{}\m{A}\m{\}}\m{\cup}\m{\{}
\m{B}\m{\}}\m{)}
\endm
\begin{mmraw}%
|- \{ A , B \} = ( \{ A \} u. \{ B \} ) \$.
\end{mmraw}

\noindent Define an unordered triple of classes\index{unordered triple}.  Definition of
Enderton, p.~19.

\vskip 0.5ex
\setbox\startprefix=\hbox{\tt \ \ df-tp\ \$a\ }
\setbox\contprefix=\hbox{\tt \ \ \ \ \ \ \ \ \ \ \ }
\startm
\m{\vdash}\m{\{}\m{A}\m{,}\m{B}\m{,}\m{C}\m{\}}\m{=}\m{(}\m{\{}\m{A}\m{,}\m{B}
\m{\}}\m{\cup}\m{\{}\m{C}\m{\}}\m{)}
\endm
\begin{mmraw}%
|- \{ A , B , C \} = ( \{ A , B \} u. \{ C \} ) \$.
\end{mmraw}%

\noindent Kuratowski's\index{Kuratowski, Kazimierz} ordered pair\index{ordered
pair} definition.  Definition 9.1 of Quine, p.~58. For proper classes it is
not meaningful but is well-defined for convenience.  (Note that \verb$<.$
stands for $\langle$ whereas \verb$<$ stands for $<$, and similarly for
\verb$>.$\,.)\label{df-op}

\vskip 0.5ex
\setbox\startprefix=\hbox{\tt \ \ df-op\ \$a\ }
\setbox\contprefix=\hbox{\tt \ \ \ \ \ \ \ \ \ \ \ }
\startm
\m{\vdash}\m{\langle}\m{A}\m{,}\m{B}\m{\rangle}\m{=}\m{\{}\m{\{}\m{A}\m{\}}
\m{,}\m{\{}\m{A}\m{,}\m{B}\m{\}}\m{\}}
\endm
\begin{mmraw}%
|- <. A , B >. = \{ \{ A \} , \{ A , B \} \} \$.
\end{mmraw}

\noindent Define the union of a class\index{union}.  Definition 5.5, p.~16,
of Takeuti and Zaring.

\vskip 0.5ex
\setbox\startprefix=\hbox{\tt \ \ df-uni\ \$a\ }
\setbox\contprefix=\hbox{\tt \ \ \ \ \ \ \ \ \ \ \ \ }
\startm
\m{\vdash}\m{\bigcup}\m{A}\m{=}\m{\{}\m{x}\m{|}\m{\exists}\m{y}\m{(}\m{x}\m{
\in}\m{y}\m{\wedge}\m{y}\m{\in}\m{A}\m{)}\m{\}}
\endm
\begin{mmraw}%
|- U. A = \{ x | E. y ( x e. y \TAND y e. A ) \} \$.
\end{mmraw}

\noindent Define the intersection\index{intersection} of a class.  Definition 7.35,
p.~44, of Takeuti and Zaring.

\vskip 0.5ex
\setbox\startprefix=\hbox{\tt \ \ df-int\ \$a\ }
\setbox\contprefix=\hbox{\tt \ \ \ \ \ \ \ \ \ \ \ \ }
\startm
\m{\vdash}\m{\bigcap}\m{A}\m{=}\m{\{}\m{x}\m{|}\m{\forall}\m{y}\m{(}\m{y}\m{
\in}\m{A}\m{\rightarrow}\m{x}\m{\in}\m{y}\m{)}\m{\}}
\endm
\begin{mmraw}%
|- |\^{}| A = \{ x | A. y ( y e. A -> x e. y ) \} \$.
\end{mmraw}

\noindent Define a transitive class\index{transitive class}\index{transitive
set}.  This should not be confused with a transitive relation, which is a different
concept.  Definition from p.~71 of Enderton, extended to classes.

\vskip 0.5ex
\setbox\startprefix=\hbox{\tt \ \ df-tr\ \$a\ }
\setbox\contprefix=\hbox{\tt \ \ \ \ \ \ \ \ \ \ \ }
\startm
\m{\vdash}\m{(}\m{\mbox{\rm Tr}}\m{A}\m{\leftrightarrow}\m{\bigcup}\m{A}\m{
\subseteq}\m{A}\m{)}
\endm
\begin{mmraw}%
|- ( Tr A <-> U. A C\_ A ) \$.
\end{mmraw}

\noindent Define a notation for a general binary relation\index{binary
relation}.  Definition 6.18, p.~29, of Takeuti and Zaring, generalized to
arbitrary classes.  This definition is well-defined, although not very
meaningful, when classes $A$ and/or $B$ are proper.\label{dfbr}  The lack of
parentheses (or any other connective) creates no ambiguity since we are defining
an atomic wff.

\vskip 0.5ex
\setbox\startprefix=\hbox{\tt \ \ df-br\ \$a\ }
\setbox\contprefix=\hbox{\tt \ \ \ \ \ \ \ \ \ \ \ }
\startm
\m{\vdash}\m{(}\m{A}\m{\,R}\m{\,B}\m{\leftrightarrow}\m{\langle}\m{A}\m{,}\m{B}
\m{\rangle}\m{\in}\m{R}\m{)}
\endm
\begin{mmraw}%
|- ( A R B <-> <. A , B >. e. R ) \$.
\end{mmraw}

\noindent Define an abstraction class of ordered pairs\index{abstraction
class!of ordered
pairs}.  A special case of Definition 4.16, p.~14, of Takeuti and Zaring.
Note that $ z $ must be distinct from $ x $ and $ y $,
and $ z $ must not occur in $\varphi$, but $ x $ and $ y $ may be
identical and may appear in $\varphi$.

\vskip 0.5ex
\setbox\startprefix=\hbox{\tt \ \ df-opab\ \$a\ }
\setbox\contprefix=\hbox{\tt \ \ \ \ \ \ \ \ \ \ \ \ \ }
\startm
\m{\vdash}\m{\{}\m{\langle}\m{x}\m{,}\m{y}\m{\rangle}\m{|}\m{\varphi}\m{\}}\m{=}
\m{\{}\m{z}\m{|}\m{\exists}\m{x}\m{\exists}\m{y}\m{(}\m{z}\m{=}\m{\langle}\m{x}
\m{,}\m{y}\m{\rangle}\m{\wedge}\m{\varphi}\m{)}\m{\}}
\endm

\begin{mmraw}%
|- \{ <. x , y >. | ph \} = \{ z | E. x E. y ( z =
<. x , y >. /\ ph ) \} \$.
\end{mmraw}

\noindent Define the epsilon relation\index{epsilon relation}.  Similar to Definition
6.22, p.~30, of Takeuti and Zaring.

\vskip 0.5ex
\setbox\startprefix=\hbox{\tt \ \ df-eprel\ \$a\ }
\setbox\contprefix=\hbox{\tt \ \ \ \ \ \ \ \ \ \ \ \ \ \ }
\startm
\m{\vdash}\m{{\rm E}}\m{=}\m{\{}\m{\langle}\m{x}\m{,}\m{y}\m{\rangle}\m{|}\m{x}\m{
\in}\m{y}\m{\}}
\endm
\begin{mmraw}%
|- \_E = \{ <. x , y >. | x e. y \} \$.
\end{mmraw}

\noindent Define a founded relation\index{founded relation}.  $R$ is a founded
relation on $A$ iff\index{iff} (if and only if) each nonempty subset of $A$
has an ``$R$-minimal element.''  Similar to Definition 6.21, p.~30, of
Takeuti and Zaring.

\vskip 0.5ex
\setbox\startprefix=\hbox{\tt \ \ df-fr\ \$a\ }
\setbox\contprefix=\hbox{\tt \ \ \ \ \ \ \ \ \ \ \ }
\startm
\m{\vdash}\m{(}\m{R}\m{\,\mbox{\rm Fr}}\m{\,A}\m{\leftrightarrow}\m{\forall}\m{x}
\m{(}\m{(}\m{x}\m{\subseteq}\m{A}\m{\wedge}\m{\lnot}\m{x}\m{=}\m{\varnothing}
\m{)}\m{\rightarrow}\m{\exists}\m{y}\m{(}\m{y}\m{\in}\m{x}\m{\wedge}\m{(}\m{x}
\m{\cap}\m{\{}\m{z}\m{|}\m{z}\m{\,R}\m{\,y}\m{\}}\m{)}\m{=}\m{\varnothing}\m{)}
\m{)}\m{)}
\endm
\begin{mmraw}%
|- ( R Fr A <-> A. x ( ( x C\_ A \TAND -. x = (/) ) ->
E. y ( y e. x \TAND ( x i\^{}i \{ z | z R y \} ) = (/) ) ) ) \$.
\end{mmraw}

\noindent Define a well-ordering\index{well-ordering}.  $R$ is a well-ordering of $A$ iff
it is founded on $A$ and the elements of $A$ are pairwise $R$-comparable.
Similar to Definition 6.24(2), p.~30, of Takeuti and Zaring.

\vskip 0.5ex
\setbox\startprefix=\hbox{\tt \ \ df-we\ \$a\ }
\setbox\contprefix=\hbox{\tt \ \ \ \ \ \ \ \ \ \ \ }
\startm
\m{\vdash}\m{(}\m{R}\m{\,\mbox{\rm We}}\m{\,A}\m{\leftrightarrow}\m{(}\m{R}\m{\,
\mbox{\rm Fr}}\m{\,A}\m{\wedge}\m{\forall}\m{x}\m{\forall}\m{y}\m{(}\m{(}\m{x}\m{
\in}\m{A}\m{\wedge}\m{y}\m{\in}\m{A}\m{)}\m{\rightarrow}\m{(}\m{x}\m{\,R}\m{\,y}
\m{\vee}\m{x}\m{=}\m{y}\m{\vee}\m{y}\m{\,R}\m{\,x}\m{)}\m{)}\m{)}\m{)}
\endm
\begin{mmraw}%
( R We A <-> ( R Fr A \TAND A. x A. y ( ( x e.
A \TAND y e. A ) -> ( x R y \TOR x = y \TOR y R x ) ) ) ) \$.
\end{mmraw}

\noindent Define the ordinal predicate\index{ordinal predicate}, which is true for a class
that is transitive and is well-ordered by the epsilon relation.  Similar to
definition on p.~468, Bell and Machover.

\vskip 0.5ex
\setbox\startprefix=\hbox{\tt \ \ df-ord\ \$a\ }
\setbox\contprefix=\hbox{\tt \ \ \ \ \ \ \ \ \ \ \ \ }
\startm
\m{\vdash}\m{(}\m{\mbox{\rm Ord}}\m{\,A}\m{\leftrightarrow}\m{(}
\m{\mbox{\rm Tr}}\m{\,A}\m{\wedge}\m{E}\m{\,\mbox{\rm We}}\m{\,A}\m{)}\m{)}
\endm
\begin{mmraw}%
|- ( Ord A <-> ( Tr A \TAND E We A ) ) \$.
\end{mmraw}

\noindent Define the class of all ordinal numbers\index{ordinal number}.  An ordinal number is
a set that satisfies the ordinal predicate.  Definition 7.11 of Takeuti and
Zaring, p.~38.

\vskip 0.5ex
\setbox\startprefix=\hbox{\tt \ \ df-on\ \$a\ }
\setbox\contprefix=\hbox{\tt \ \ \ \ \ \ \ \ \ \ \ }
\startm
\m{\vdash}\m{\,\mbox{\rm On}}\m{=}\m{\{}\m{x}\m{|}\m{\mbox{\rm Ord}}\m{\,x}
\m{\}}
\endm
\begin{mmraw}%
|- On = \{ x | Ord x \} \$.
\end{mmraw}

\noindent Define the limit ordinal predicate\index{limit ordinal}, which is true for a
non-empty ordinal that is not a successor (i.e.\ that is the union of itself).
Compare Bell and Machover, p.~471 and Exercise (1), p.~42 of Takeuti and
Zaring.

\vskip 0.5ex
\setbox\startprefix=\hbox{\tt \ \ df-lim\ \$a\ }
\setbox\contprefix=\hbox{\tt \ \ \ \ \ \ \ \ \ \ \ \ }
\startm
\m{\vdash}\m{(}\m{\mbox{\rm Lim}}\m{\,A}\m{\leftrightarrow}\m{(}\m{\mbox{
\rm Ord}}\m{\,A}\m{\wedge}\m{\lnot}\m{A}\m{=}\m{\varnothing}\m{\wedge}\m{A}
\m{=}\m{\bigcup}\m{A}\m{)}\m{)}
\endm
\begin{mmraw}%
|- ( Lim A <-> ( Ord A \TAND -. A = (/) \TAND A = U. A ) ) \$.
\end{mmraw}

\noindent Define the successor\index{successor} of a class.  Definition 7.22 of Takeuti
and Zaring, p.~41.  Our definition is a generalization to classes, although it
is meaningless when classes are proper.

\vskip 0.5ex
\setbox\startprefix=\hbox{\tt \ \ df-suc\ \$a\ }
\setbox\contprefix=\hbox{\tt \ \ \ \ \ \ \ \ \ \ \ \ }
\startm
\m{\vdash}\m{\,\mbox{\rm suc}}\m{\,A}\m{=}\m{(}\m{A}\m{\cup}\m{\{}\m{A}\m{\}}
\m{)}
\endm
\begin{mmraw}%
|- suc A = ( A u. \{ A \} ) \$.
\end{mmraw}

\noindent Define the class of natural numbers\index{natural number}\index{omega
($\omega$)}.  Compare Bell and Machover, p.~471.\label{dfom}

\vskip 0.5ex
\setbox\startprefix=\hbox{\tt \ \ df-om\ \$a\ }
\setbox\contprefix=\hbox{\tt \ \ \ \ \ \ \ \ \ \ \ }
\startm
\m{\vdash}\m{\omega}\m{=}\m{\{}\m{x}\m{|}\m{(}\m{\mbox{\rm Ord}}\m{\,x}\m{
\wedge}\m{\forall}\m{y}\m{(}\m{\mbox{\rm Lim}}\m{\,y}\m{\rightarrow}\m{x}\m{
\in}\m{y}\m{)}\m{)}\m{\}}
\endm
\begin{mmraw}%
|- om = \{ x | ( Ord x \TAND A. y ( Lim y -> x e. y ) ) \} \$.
\end{mmraw}

\noindent Define the Cartesian product (also called the
cross product)\index{Cartesian product}\index{cross product}
of two classes.  Definition 9.11 of Quine, p.~64.

\vskip 0.5ex
\setbox\startprefix=\hbox{\tt \ \ df-xp\ \$a\ }
\setbox\contprefix=\hbox{\tt \ \ \ \ \ \ \ \ \ \ \ }
\startm
\m{\vdash}\m{(}\m{A}\m{\times}\m{B}\m{)}\m{=}\m{\{}\m{\langle}\m{x}\m{,}\m{y}
\m{\rangle}\m{|}\m{(}\m{x}\m{\in}\m{A}\m{\wedge}\m{y}\m{\in}\m{B}\m{)}\m{\}}
\endm
\begin{mmraw}%
|- ( A X. B ) = \{ <. x , y >. | ( x e. A \TAND y e. B) \} \$.
\end{mmraw}

\noindent Define a relation\index{relation}.  Definition 6.4(1) of Takeuti and
Zaring, p.~23.

\vskip 0.5ex
\setbox\startprefix=\hbox{\tt \ \ df-rel\ \$a\ }
\setbox\contprefix=\hbox{\tt \ \ \ \ \ \ \ \ \ \ \ \ }
\startm
\m{\vdash}\m{(}\m{\mbox{\rm Rel}}\m{\,A}\m{\leftrightarrow}\m{A}\m{\subseteq}
\m{(}\m{{\rm V}}\m{\times}\m{{\rm V}}\m{)}\m{)}
\endm
\begin{mmraw}%
|- ( Rel A <-> A C\_ ( {\char`\_}V X. {\char`\_}V ) ) \$.
\end{mmraw}

\noindent Define the domain\index{domain} of a class.  Definition 6.5(1) of
Takeuti and Zaring, p.~24.

\vskip 0.5ex
\setbox\startprefix=\hbox{\tt \ \ df-dm\ \$a\ }
\setbox\contprefix=\hbox{\tt \ \ \ \ \ \ \ \ \ \ \ }
\startm
\m{\vdash}\m{\,\mbox{\rm dom}}\m{A}\m{=}\m{\{}\m{x}\m{|}\m{\exists}\m{y}\m{
\langle}\m{x}\m{,}\m{y}\m{\rangle}\m{\in}\m{A}\m{\}}
\endm
\begin{mmraw}%
|- dom A = \{ x | E. y <. x , y >. e. A \} \$.
\end{mmraw}

\noindent Define the range\index{range} of a class.  Definition 6.5(2) of
Takeuti and Zaring, p.~24.

\vskip 0.5ex
\setbox\startprefix=\hbox{\tt \ \ df-rn\ \$a\ }
\setbox\contprefix=\hbox{\tt \ \ \ \ \ \ \ \ \ \ \ }
\startm
\m{\vdash}\m{\,\mbox{\rm ran}}\m{A}\m{=}\m{\{}\m{y}\m{|}\m{\exists}\m{x}\m{
\langle}\m{x}\m{,}\m{y}\m{\rangle}\m{\in}\m{A}\m{\}}
\endm
\begin{mmraw}%
|- ran A = \{ y | E. x <. x , y >. e. A \} \$.
\end{mmraw}

\noindent Define the restriction\index{restriction} of a class.  Definition
6.6(1) of Takeuti and Zaring, p.~24.

\vskip 0.5ex
\setbox\startprefix=\hbox{\tt \ \ df-res\ \$a\ }
\setbox\contprefix=\hbox{\tt \ \ \ \ \ \ \ \ \ \ \ \ }
\startm
\m{\vdash}\m{(}\m{A}\m{\restriction}\m{B}\m{)}\m{=}\m{(}\m{A}\m{\cap}\m{(}\m{B}
\m{\times}\m{{\rm V}}\m{)}\m{)}
\endm
\begin{mmraw}%
|- ( A |` B ) = ( A i\^{}i ( B X. {\char`\_}V ) ) \$.
\end{mmraw}

\noindent Define the image\index{image} of a class.  Definition 6.6(2) of
Takeuti and Zaring, p.~24.

\vskip 0.5ex
\setbox\startprefix=\hbox{\tt \ \ df-ima\ \$a\ }
\setbox\contprefix=\hbox{\tt \ \ \ \ \ \ \ \ \ \ \ \ }
\startm
\m{\vdash}\m{(}\m{A}\m{``}\m{B}\m{)}\m{=}\m{\,\mbox{\rm ran}}\m{\,(}\m{A}\m{
\restriction}\m{B}\m{)}
\endm
\begin{mmraw}%
|- ( A " B ) = ran ( A |` B ) \$.
\end{mmraw}

\noindent Define the composition\index{composition} of two classes.  Definition 6.6(3) of
Takeuti and Zaring, p.~24.

\vskip 0.5ex
\setbox\startprefix=\hbox{\tt \ \ df-co\ \$a\ }
\setbox\contprefix=\hbox{\tt \ \ \ \ \ \ \ \ \ \ \ \ }
\startm
\m{\vdash}\m{(}\m{A}\m{\circ}\m{B}\m{)}\m{=}\m{\{}\m{\langle}\m{x}\m{,}\m{y}\m{
\rangle}\m{|}\m{\exists}\m{z}\m{(}\m{\langle}\m{x}\m{,}\m{z}\m{\rangle}\m{\in}
\m{B}\m{\wedge}\m{\langle}\m{z}\m{,}\m{y}\m{\rangle}\m{\in}\m{A}\m{)}\m{\}}
\endm
\begin{mmraw}%
|- ( A o. B ) = \{ <. x , y >. | E. z ( <. x , z
>. e. B \TAND <. z , y >. e. A ) \} \$.
\end{mmraw}

\noindent Define a function\index{function}.  Definition 6.4(4) of Takeuti and
Zaring, p.~24.

\vskip 0.5ex
\setbox\startprefix=\hbox{\tt \ \ df-fun\ \$a\ }
\setbox\contprefix=\hbox{\tt \ \ \ \ \ \ \ \ \ \ \ \ }
\startm
\m{\vdash}\m{(}\m{\mbox{\rm Fun}}\m{\,A}\m{\leftrightarrow}\m{(}
\m{\mbox{\rm Rel}}\m{\,A}\m{\wedge}
\m{\forall}\m{x}\m{\exists}\m{z}\m{\forall}\m{y}\m{(}
\m{\langle}\m{x}\m{,}\m{y}\m{\rangle}\m{\in}\m{A}\m{\rightarrow}\m{y}\m{=}\m{z}
\m{)}\m{)}\m{)}
\endm
\begin{mmraw}%
|- ( Fun A <-> ( Rel A /\ A. x E. z A. y ( <. x
   , y >. e. A -> y = z ) ) ) \$.
\end{mmraw}

\noindent Define a function with domain.  Definition 6.15(1) of Takeuti and
Zaring, p.~27.

\vskip 0.5ex
\setbox\startprefix=\hbox{\tt \ \ df-fn\ \$a\ }
\setbox\contprefix=\hbox{\tt \ \ \ \ \ \ \ \ \ \ \ }
\startm
\m{\vdash}\m{(}\m{A}\m{\,\mbox{\rm Fn}}\m{\,B}\m{\leftrightarrow}\m{(}
\m{\mbox{\rm Fun}}\m{\,A}\m{\wedge}\m{\mbox{\rm dom}}\m{\,A}\m{=}\m{B}\m{)}
\m{)}
\endm
\begin{mmraw}%
|- ( A Fn B <-> ( Fun A \TAND dom A = B ) ) \$.
\end{mmraw}

\noindent Define a function with domain and co-domain.  Definition 6.15(3)
of Takeuti and Zaring, p.~27.

\vskip 0.5ex
\setbox\startprefix=\hbox{\tt \ \ df-f\ \$a\ }
\setbox\contprefix=\hbox{\tt \ \ \ \ \ \ \ \ \ \ }
\startm
\m{\vdash}\m{(}\m{F}\m{:}\m{A}\m{\longrightarrow}\m{B}\m{
\leftrightarrow}\m{(}\m{F}\m{\,\mbox{\rm Fn}}\m{\,A}\m{\wedge}\m{
\mbox{\rm ran}}\m{\,F}\m{\subseteq}\m{B}\m{)}\m{)}
\endm
\begin{mmraw}%
|- ( F : A --> B <-> ( F Fn A \TAND ran F C\_ B ) ) \$.
\end{mmraw}

\noindent Define a one-to-one function\index{one-to-one function}.  Compare
Definition 6.15(5) of Takeuti and Zaring, p.~27.

\vskip 0.5ex
\setbox\startprefix=\hbox{\tt \ \ df-f1\ \$a\ }
\setbox\contprefix=\hbox{\tt \ \ \ \ \ \ \ \ \ \ \ }
\startm
\m{\vdash}\m{(}\m{F}\m{:}\m{A}\m{
\raisebox{.5ex}{${\textstyle{\:}_{\mbox{\footnotesize\rm
1\tt -\rm 1}}}\atop{\textstyle{
\longrightarrow}\atop{\textstyle{}^{\mbox{\footnotesize\rm {\ }}}}}$}
}\m{B}
\m{\leftrightarrow}\m{(}\m{F}\m{:}\m{A}\m{\longrightarrow}\m{B}
\m{\wedge}\m{\forall}\m{y}\m{\exists}\m{z}\m{\forall}\m{x}\m{(}\m{\langle}\m{x}
\m{,}\m{y}\m{\rangle}\m{\in}\m{F}\m{\rightarrow}\m{x}\m{=}\m{z}\m{)}\m{)}\m{)}
\endm
\begin{mmraw}%
|- ( F : A -1-1-> B <-> ( F : A --> B \TAND
   A. y E. z A. x ( <. x , y >. e. F -> x = z ) ) ) \$.
\end{mmraw}

\noindent Define an onto function\index{onto function}.  Definition 6.15(4) of Takeuti and
Zaring, p.~27.

\vskip 0.5ex
\setbox\startprefix=\hbox{\tt \ \ df-fo\ \$a\ }
\setbox\contprefix=\hbox{\tt \ \ \ \ \ \ \ \ \ \ \ }
\startm
\m{\vdash}\m{(}\m{F}\m{:}\m{A}\m{
\raisebox{.5ex}{${\textstyle{\:}_{\mbox{\footnotesize\rm
{\ }}}}\atop{\textstyle{
\longrightarrow}\atop{\textstyle{}^{\mbox{\footnotesize\rm onto}}}}$}
}\m{B}
\m{\leftrightarrow}\m{(}\m{F}\m{\,\mbox{\rm Fn}}\m{\,A}\m{\wedge}
\m{\mbox{\rm ran}}\m{\,F}\m{=}\m{B}\m{)}\m{)}
\endm
\begin{mmraw}%
|- ( F : A -onto-> B <-> ( F Fn A /\ ran F = B ) ) \$.
\end{mmraw}

\noindent Define a one-to-one, onto function.  Compare Definition 6.15(6) of
Takeuti and Zaring, p.~27.

\vskip 0.5ex
\setbox\startprefix=\hbox{\tt \ \ df-f1o\ \$a\ }
\setbox\contprefix=\hbox{\tt \ \ \ \ \ \ \ \ \ \ \ \ }
\startm
\m{\vdash}\m{(}\m{F}\m{:}\m{A}
\m{
\raisebox{.5ex}{${\textstyle{\:}_{\mbox{\footnotesize\rm
1\tt -\rm 1}}}\atop{\textstyle{
\longrightarrow}\atop{\textstyle{}^{\mbox{\footnotesize\rm onto}}}}$}
}
\m{B}
\m{\leftrightarrow}\m{(}\m{F}\m{:}\m{A}
\m{
\raisebox{.5ex}{${\textstyle{\:}_{\mbox{\footnotesize\rm
1\tt -\rm 1}}}\atop{\textstyle{
\longrightarrow}\atop{\textstyle{}^{\mbox{\footnotesize\rm {\ }}}}}$}
}
\m{B}\m{\wedge}\m{F}\m{:}\m{A}
\m{
\raisebox{.5ex}{${\textstyle{\:}_{\mbox{\footnotesize\rm
{\ }}}}\atop{\textstyle{
\longrightarrow}\atop{\textstyle{}^{\mbox{\footnotesize\rm onto}}}}$}
}
\m{B}\m{)}\m{)}
\endm
\begin{mmraw}%
|- ( F : A -1-1-onto-> B <-> ( F : A -1-1-> B? \TAND F : A -onto-> B ) ) \$.?
\end{mmraw}

\noindent Define the value of a function\index{function value}.  This
definition applies to any class and evaluates to the empty set when it is not
meaningful. Note that $ F`A$ means the same thing as the more familiar $ F(A)$
notation for a function's value at $A$.  The $ F`A$ notation is common in
formal set theory.\label{df-fv}

\vskip 0.5ex
\setbox\startprefix=\hbox{\tt \ \ df-fv\ \$a\ }
\setbox\contprefix=\hbox{\tt \ \ \ \ \ \ \ \ \ \ \ }
\startm
\m{\vdash}\m{(}\m{F}\m{`}\m{A}\m{)}\m{=}\m{\bigcup}\m{\{}\m{x}\m{|}\m{(}\m{F}%
\m{``}\m{\{}\m{A}\m{\}}\m{)}\m{=}\m{\{}\m{x}\m{\}}\m{\}}
\endm
\begin{mmraw}%
|- ( F ` A ) = U. \{ x | ( F " \{ A \} ) = \{ x \} \} \$.
\end{mmraw}

\noindent Define the result of an operation.\index{operation}  Here, $F$ is
     an operation on two
     values (such as $+$ for real numbers).   This is defined for proper
     classes $A$ and $B$ even though not meaningful in that case.  However,
     the definition can be meaningful when $F$ is a proper class.\label{dfopr}

\vskip 0.5ex
\setbox\startprefix=\hbox{\tt \ \ df-opr\ \$a\ }
\setbox\contprefix=\hbox{\tt \ \ \ \ \ \ \ \ \ \ \ \ }
\startm
\m{\vdash}\m{(}\m{A}\m{\,F}\m{\,B}\m{)}\m{=}\m{(}\m{F}\m{`}\m{\langle}\m{A}%
\m{,}\m{B}\m{\rangle}\m{)}
\endm
\begin{mmraw}%
|- ( A F B ) = ( F ` <. A , B >. ) \$.
\end{mmraw}

\section{Tricks of the Trade}\label{tricks}

In the \texttt{set.mm}\index{set theory database (\texttt{set.mm})} database our goal
was usually to conform to modern notation.  However in some cases the
relationship to standard textbook language may be obscured by several
unconventional devices we used to simplify the development and to take
advantage of the Metamath language.  In this section we will describe some
common conventions used in \texttt{set.mm}.

\begin{itemize}
\item
The turnstile\index{turnstile ({$\,\vdash$})} symbol, $\vdash$, meaning ``it
is provable that,'' is the first token of all assertions and hypotheses that
aren't syntax constructions.  This is a standard convention in logic.  (We
mentioned this earlier, but this symbol is bothersome to some people without a
logic background.  It has no deeper meaning but just provides us with a way to
distinguish syntax constructions from ordinary mathematical statements.)

\item
A hypothesis of the form

\vskip 1ex
\setbox\startprefix=\hbox{\tt \ \ \ \ \ \ \ \ \ \$e\ }
\setbox\contprefix=\hbox{\tt \ \ \ \ \ \ \ \ \ \ }
\startm
\m{\vdash}\m{(}\m{\varphi}\m{\rightarrow}\m{\forall}\m{x}\m{\varphi}\m{)}
\endm
\vskip 1ex

should be read ``assume variable $x$ is (effectively) not free in wff
$\varphi$.''\index{effectively not free}
Literally, this says ``assume it is provable that $\varphi \rightarrow \forall
x\, \varphi$.''  This device lets us avoid the complexities associated with
the standard treatment of free and bound variables.
%% Uncomment this when uncommenting section {formalspec} below
The footnote on p.~\pageref{effectivelybound} discusses this further.

\item
A statement of one of the forms

\vskip 1ex
\setbox\startprefix=\hbox{\tt \ \ \ \ \ \ \ \ \ \$a\ }
\setbox\contprefix=\hbox{\tt \ \ \ \ \ \ \ \ \ \ }
\startm
\m{\vdash}\m{(}\m{\lnot}\m{\forall}\m{x}\m{\,x}\m{=}\m{y}\m{\rightarrow}
\m{\ldots}\m{)}
\endm
\setbox\startprefix=\hbox{\tt \ \ \ \ \ \ \ \ \ \$p\ }
\setbox\contprefix=\hbox{\tt \ \ \ \ \ \ \ \ \ \ }
\startm
\m{\vdash}\m{(}\m{\lnot}\m{\forall}\m{x}\m{\,x}\m{=}\m{y}\m{\rightarrow}
\m{\ldots}\m{)}
\endm
\vskip 1ex

should be read ``if $x$ and $y$ are distinct variables, then...''  This
antecedent provides us with a technical device to avoid the need for the
\texttt{\$d} statement early in our development of predicate calculus,
permitting symbol manipulations to be as conceptually simple as those in
propositional calculus.  However, the \texttt{\$d} statement eventually
becomes a requirement, and after that this device is rarely used.

\item
The statement

\vskip 1ex
\setbox\startprefix=\hbox{\tt \ \ \ \ \ \ \ \ \ \$d\ }
\setbox\contprefix=\hbox{\tt \ \ \ \ \ \ \ \ \ \ }
\startm
\m{x}\m{\,y}
\endm
\vskip 1ex

should be read ``assume $x$ and $y$ are distinct variables.''

\item
The statement

\vskip 1ex
\setbox\startprefix=\hbox{\tt \ \ \ \ \ \ \ \ \ \$d\ }
\setbox\contprefix=\hbox{\tt \ \ \ \ \ \ \ \ \ \ }
\startm
\m{x}\m{\,\varphi}
\endm
\vskip 1ex

should be read ``assume $x$ does not occur in $\varphi$.''

\item
The statement

\vskip 1ex
\setbox\startprefix=\hbox{\tt \ \ \ \ \ \ \ \ \ \$d\ }
\setbox\contprefix=\hbox{\tt \ \ \ \ \ \ \ \ \ \ }
\startm
\m{x}\m{\,A}
\endm
\vskip 1ex

should be read ``assume variable $x$ does not occur in class $A$.''

\item
The restriction and hypothesis group

\vskip 1ex
\setbox\startprefix=\hbox{\tt \ \ \ \ \ \ \ \ \ \$d\ }
\setbox\contprefix=\hbox{\tt \ \ \ \ \ \ \ \ \ \ }
\startm
\m{x}\m{\,A}
\endm
\setbox\startprefix=\hbox{\tt \ \ \ \ \ \ \ \ \ \$d\ }
\setbox\contprefix=\hbox{\tt \ \ \ \ \ \ \ \ \ \ }
\startm
\m{x}\m{\,\psi}
\endm
\setbox\startprefix=\hbox{\tt \ \ \ \ \ \ \ \ \ \$e\ }
\setbox\contprefix=\hbox{\tt \ \ \ \ \ \ \ \ \ \ }
\startm
\m{\vdash}\m{(}\m{x}\m{=}\m{A}\m{\rightarrow}\m{(}\m{\varphi}\m{\leftrightarrow}
\m{\psi}\m{)}\m{)}
\endm
\vskip 1ex

is frequently used in place of explicit substitution, meaning ``assume
$\psi$ results from the proper substitution of $A$ for $x$ in
$\varphi$.''  Sometimes ``\texttt{\$e} $\vdash ( \psi \rightarrow
\forall x \, \psi )$'' is used instead of ``\texttt{\$d} $x\, \psi $,''
which requires only that $x$ be effectively not free in $\varphi$ but
not necessarily absent from it.  The use of implicit
substitution\index{substitution!implicit} is partly a matter of personal
style, although it may make proofs somewhat shorter than would be the
case with explicit substitution.

\item
The hypothesis


\vskip 1ex
\setbox\startprefix=\hbox{\tt \ \ \ \ \ \ \ \ \ \$e\ }
\setbox\contprefix=\hbox{\tt \ \ \ \ \ \ \ \ \ \ }
\startm
\m{\vdash}\m{A}\m{\in}\m{{\rm V}}
\endm
\vskip 1ex

should be read ``assume class $A$ is a set (i.e.\ exists).''
This is a convenient convention used by Quine.

\item
The restriction and hypothesis

\vskip 1ex
\setbox\startprefix=\hbox{\tt \ \ \ \ \ \ \ \ \ \$d\ }
\setbox\contprefix=\hbox{\tt \ \ \ \ \ \ \ \ \ \ }
\startm
\m{x}\m{\,y}
\endm
\setbox\startprefix=\hbox{\tt \ \ \ \ \ \ \ \ \ \$e\ }
\setbox\contprefix=\hbox{\tt \ \ \ \ \ \ \ \ \ \ }
\startm
\m{\vdash}\m{(}\m{y}\m{\in}\m{A}\m{\rightarrow}\m{\forall}\m{x}\m{\,y}
\m{\in}\m{A}\m{)}
\endm
\vskip 1ex

should be read ``assume variable $x$ is
(effectively) not free in class $A$.''

\end{itemize}

\section{A Theorem Sampler}\label{sometheorems}

In this section we list some of the more important theorems that are proved in
the \texttt{set.mm} database, and they illustrate the kinds of things that can be
done with Metamath.  While all of these facts are well-known results,
Metamath offers the advantage of easily allowing you to trace their
derivation back to axioms.  Our intent here is not to try to explain the
details or motivation; for this we refer you to the textbooks that are
mentioned in the descriptions.  (The \texttt{set.mm} file has bibliographic
references for the text references.)  Their proofs often embody important
concepts you may wish to explore with the Metamath program (see
Section~\ref{exploring}).  All the symbols that are used here are defined in
Section~\ref{hierarchy}.  For brevity we haven't included the \texttt{\$d}
restrictions or \texttt{\$f} hypotheses for these theorems; when you are
uncertain consult the \texttt{set.mm} database.

We start with \texttt{syl} (principle of the syllogism).
In \textit{Principia Mathematica}
Whitehead and Russell call this ``the principle of the
syllogism... because... the syllogism in Barbara is derived from them''
\cite[quote after Theorem *2.06 p.~101]{PM}.
Some authors call this law a ``hypothetical syllogism.''
As of 2019 \texttt{syl} is the most commonly referenced proven
assertion in the \texttt{set.mm} database.\footnote{
The Metamath program command \texttt{show usage}
shows the number of references.
On 2019-04-29 (commit 71cbbbdb387e) \texttt{syl} was directly referenced
10,819 times. The second most commonly referenced proven assertion
was \texttt{eqid}, which was directly referenced 7,738 times.
}

\vskip 2ex
\noindent Theorem syl (principle of the syllogism)\index{Syllogism}%
\index{\texttt{syl}}\label{syl}.

\vskip 0.5ex
\setbox\startprefix=\hbox{\tt \ \ syl.1\ \$e\ }
\setbox\contprefix=\hbox{\tt \ \ \ \ \ \ \ \ \ \ \ }
\startm
\m{\vdash}\m{(}\m{\varphi}\m{ \rightarrow }\m{\psi}\m{)}
\endm
\setbox\startprefix=\hbox{\tt \ \ syl.2\ \$e\ }
\setbox\contprefix=\hbox{\tt \ \ \ \ \ \ \ \ \ \ \ }
\startm
\m{\vdash}\m{(}\m{\psi}\m{ \rightarrow }\m{\chi}\m{)}
\endm
\setbox\startprefix=\hbox{\tt \ \ syl\ \$p\ }
\setbox\contprefix=\hbox{\tt \ \ \ \ \ \ \ \ \ }
\startm
\m{\vdash}\m{(}\m{\varphi}\m{ \rightarrow }\m{\chi}\m{)}
\endm
\vskip 2ex

The following theorem is not very deep but provides us with a notational device
that is frequently used.  It allows us to use the expression ``$A \in V$'' as
a compact way of saying that class $A$ exists, i.e.\ is a set.

\vskip 2ex
\noindent Two ways to say ``$A$ is a set'':  $A$ is a member of the universe
$V$ if and only if $A$ exists (i.e.\ there exists a set equal to $A$).
Theorem 6.9 of Quine, p. 43.

\vskip 0.5ex
\setbox\startprefix=\hbox{\tt \ \ isset\ \$p\ }
\setbox\contprefix=\hbox{\tt \ \ \ \ \ \ \ \ \ \ \ }
\startm
\m{\vdash}\m{(}\m{A}\m{\in}\m{{\rm V}}\m{\leftrightarrow}\m{\exists}\m{x}\m{\,x}\m{=}
\m{A}\m{)}
\endm
\vskip 1ex

Next we prove the axioms of standard ZF set theory that were missing from our
axiom system.  From our point of view they are theorems since they
can be derived from the other axioms.

\vskip 2ex
\noindent Axiom of Separation\index{Axiom of Separation}
(Aussonderung)\index{Aussonderung} proved from the other axioms of ZF set
theory.  Compare Exercise 4 of Takeuti and Zaring, p.~22.

\vskip 0.5ex
\setbox\startprefix=\hbox{\tt \ \ inex1.1\ \$e\ }
\setbox\contprefix=\hbox{\tt \ \ \ \ \ \ \ \ \ \ \ \ \ \ \ }
\startm
\m{\vdash}\m{A}\m{\in}\m{{\rm V}}
\endm
\setbox\startprefix=\hbox{\tt \ \ inex\ \$p\ }
\setbox\contprefix=\hbox{\tt \ \ \ \ \ \ \ \ \ \ \ \ \ }
\startm
\m{\vdash}\m{(}\m{A}\m{\cap}\m{B}\m{)}\m{\in}\m{{\rm V}}
\endm
\vskip 1ex

\noindent Axiom of the Null Set\index{Axiom of the Null Set} proved from the
other axioms of ZF set theory. Corollary 5.16 of Takeuti and Zaring, p.~20.

\vskip 0.5ex
\setbox\startprefix=\hbox{\tt \ \ 0ex\ \$p\ }
\setbox\contprefix=\hbox{\tt \ \ \ \ \ \ \ \ \ \ \ \ }
\startm
\m{\vdash}\m{\varnothing}\m{\in}\m{{\rm V}}
\endm
\vskip 1ex

\noindent The Axiom of Pairing\index{Axiom of Pairing} proved from the other
axioms of ZF set theory.  Theorem 7.13 of Quine, p.~51.
\vskip 0.5ex
\setbox\startprefix=\hbox{\tt \ \ prex\ \$p\ }
\setbox\contprefix=\hbox{\tt \ \ \ \ \ \ \ \ \ \ \ \ \ \ }
\startm
\m{\vdash}\m{\{}\m{A}\m{,}\m{B}\m{\}}\m{\in}\m{{\rm V}}
\endm
\vskip 2ex

Next we will list some famous or important theorems that are proved in
the \texttt{set.mm} database.  None of them except \texttt{omex}
require the Axiom of Infinity, as you can verify with the \texttt{show
trace{\char`\_}back} Metamath command.

\vskip 2ex
\noindent The resolution of Russell's paradox\index{Russell's paradox}.  There
exists no set containing the set of all sets which are not members of
themselves.  Proposition 4.14 of Takeuti and Zaring, p.~14.

\vskip 0.5ex
\setbox\startprefix=\hbox{\tt \ \ ru\ \$p\ }
\setbox\contprefix=\hbox{\tt \ \ \ \ \ \ \ \ }
\startm
\m{\vdash}\m{\lnot}\m{\exists}\m{x}\m{\,x}\m{=}\m{\{}\m{y}\m{|}\m{\lnot}\m{y}
\m{\in}\m{y}\m{\}}
\endm
\vskip 1ex

\noindent Cantor's theorem\index{Cantor's theorem}.  No set can be mapped onto
its power set.  Compare Theorem 6B(b) of Enderton, p.~132.

\vskip 0.5ex
\setbox\startprefix=\hbox{\tt \ \ canth.1\ \$e\ }
\setbox\contprefix=\hbox{\tt \ \ \ \ \ \ \ \ \ \ \ \ \ }
\startm
\m{\vdash}\m{A}\m{\in}\m{{\rm V}}
\endm
\setbox\startprefix=\hbox{\tt \ \ canth\ \$p\ }
\setbox\contprefix=\hbox{\tt \ \ \ \ \ \ \ \ \ \ \ }
\startm
\m{\vdash}\m{\lnot}\m{F}\m{:}\m{A}\m{\raisebox{.5ex}{${\textstyle{\:}_{
\mbox{\footnotesize\rm {\ }}}}\atop{\textstyle{\longrightarrow}\atop{
\textstyle{}^{\mbox{\footnotesize\rm onto}}}}$}}\m{{\cal P}}\m{A}
\endm
\vskip 1ex

\noindent The Burali-Forti paradox\index{Burali-Forti paradox}.  No set
contains all ordinal numbers. Enderton, p.~194.  (Burali-Forti was one person,
not two.)

\vskip 0.5ex
\setbox\startprefix=\hbox{\tt \ \ onprc\ \$p\ }
\setbox\contprefix=\hbox{\tt \ \ \ \ \ \ \ \ \ \ \ \ }
\startm
\m{\vdash}\m{\lnot}\m{\mbox{\rm On}}\m{\in}\m{{\rm V}}
\endm
\vskip 1ex

\noindent Peano's postulates\index{Peano's postulates} for arithmetic.
Proposition 7.30 of Takeuti and Zaring, pp.~42--43.  The objects being
described are the members of $\omega$ i.e.\ the natural numbers 0, 1,
2,\ldots.  The successor\index{successor} operation suc means ``plus
one.''  \texttt{peano1} says that 0 (which is defined as the empty set)
is a natural number.  \texttt{peano2} says that if $A$ is a natural
number, so is $A+1$.  \texttt{peano3} says that 0 is not the successor
of any natural number.  \texttt{peano4} says that two natural numbers
are equal if and only if their successors are equal.  \texttt{peano5} is
essentially the same as mathematical induction.

\vskip 1ex
\setbox\startprefix=\hbox{\tt \ \ peano1\ \$p\ }
\setbox\contprefix=\hbox{\tt \ \ \ \ \ \ \ \ \ \ \ \ }
\startm
\m{\vdash}\m{\varnothing}\m{\in}\m{\omega}
\endm
\vskip 1.5ex

\setbox\startprefix=\hbox{\tt \ \ peano2\ \$p\ }
\setbox\contprefix=\hbox{\tt \ \ \ \ \ \ \ \ \ \ \ \ }
\startm
\m{\vdash}\m{(}\m{A}\m{\in}\m{\omega}\m{\rightarrow}\m{{\rm suc}}\m{A}\m{\in}%
\m{\omega}\m{)}
\endm
\vskip 1.5ex

\setbox\startprefix=\hbox{\tt \ \ peano3\ \$p\ }
\setbox\contprefix=\hbox{\tt \ \ \ \ \ \ \ \ \ \ \ \ }
\startm
\m{\vdash}\m{(}\m{A}\m{\in}\m{\omega}\m{\rightarrow}\m{\lnot}\m{{\rm suc}}%
\m{A}\m{=}\m{\varnothing}\m{)}
\endm
\vskip 1.5ex

\setbox\startprefix=\hbox{\tt \ \ peano4\ \$p\ }
\setbox\contprefix=\hbox{\tt \ \ \ \ \ \ \ \ \ \ \ \ }
\startm
\m{\vdash}\m{(}\m{(}\m{A}\m{\in}\m{\omega}\m{\wedge}\m{B}\m{\in}\m{\omega}%
\m{)}\m{\rightarrow}\m{(}\m{{\rm suc}}\m{A}\m{=}\m{{\rm suc}}\m{B}\m{%
\leftrightarrow}\m{A}\m{=}\m{B}\m{)}\m{)}
\endm
\vskip 1.5ex

\setbox\startprefix=\hbox{\tt \ \ peano5\ \$p\ }
\setbox\contprefix=\hbox{\tt \ \ \ \ \ \ \ \ \ \ \ \ }
\startm
\m{\vdash}\m{(}\m{(}\m{\varnothing}\m{\in}\m{A}\m{\wedge}\m{\forall}\m{x}\m{%
\in}\m{\omega}\m{(}\m{x}\m{\in}\m{A}\m{\rightarrow}\m{{\rm suc}}\m{x}\m{\in}%
\m{A}\m{)}\m{)}\m{\rightarrow}\m{\omega}\m{\subseteq}\m{A}\m{)}
\endm
\vskip 1.5ex

\noindent Finite Induction (mathematical induction).\index{finite
induction}\index{mathematical induction} The first hypothesis is the
basis and the second is the induction hypothesis.  Theorem Schema 22 of
Suppes, p.~136.

\vskip 0.5ex
\setbox\startprefix=\hbox{\tt \ \ findes.1\ \$e\ }
\setbox\contprefix=\hbox{\tt \ \ \ \ \ \ \ \ \ \ \ \ \ \ }
\startm
\m{\vdash}\m{[}\m{\varnothing}\m{/}\m{x}\m{]}\m{\varphi}
\endm
\setbox\startprefix=\hbox{\tt \ \ findes.2\ \$e\ }
\setbox\contprefix=\hbox{\tt \ \ \ \ \ \ \ \ \ \ \ \ \ \ }
\startm
\m{\vdash}\m{(}\m{x}\m{\in}\m{\omega}\m{\rightarrow}\m{(}\m{\varphi}\m{%
\rightarrow}\m{[}\m{{\rm suc}}\m{x}\m{/}\m{x}\m{]}\m{\varphi}\m{)}\m{)}
\endm
\setbox\startprefix=\hbox{\tt \ \ findes\ \$p\ }
\setbox\contprefix=\hbox{\tt \ \ \ \ \ \ \ \ \ \ \ \ }
\startm
\m{\vdash}\m{(}\m{x}\m{\in}\m{\omega}\m{\rightarrow}\m{\varphi}\m{)}
\endm
\vskip 1ex

\noindent Transfinite Induction with explicit substitution.  The first
hypothesis is the basis, the second is the induction hypothesis for
successors, and the third is the induction hypothesis for limit
ordinals.  Theorem Schema 4 of Suppes, p. 197.

\vskip 0.5ex
\setbox\startprefix=\hbox{\tt \ \ tfindes.1\ \$e\ }
\setbox\contprefix=\hbox{\tt \ \ \ \ \ \ \ \ \ \ \ \ \ \ \ }
\startm
\m{\vdash}\m{[}\m{\varnothing}\m{/}\m{x}\m{]}\m{\varphi}
\endm
\setbox\startprefix=\hbox{\tt \ \ tfindes.2\ \$e\ }
\setbox\contprefix=\hbox{\tt \ \ \ \ \ \ \ \ \ \ \ \ \ \ \ }
\startm
\m{\vdash}\m{(}\m{x}\m{\in}\m{{\rm On}}\m{\rightarrow}\m{(}\m{\varphi}\m{%
\rightarrow}\m{[}\m{{\rm suc}}\m{x}\m{/}\m{x}\m{]}\m{\varphi}\m{)}\m{)}
\endm
\setbox\startprefix=\hbox{\tt \ \ tfindes.3\ \$e\ }
\setbox\contprefix=\hbox{\tt \ \ \ \ \ \ \ \ \ \ \ \ \ \ \ }
\startm
\m{\vdash}\m{(}\m{{\rm Lim}}\m{y}\m{\rightarrow}\m{(}\m{\forall}\m{x}\m{\in}%
\m{y}\m{\varphi}\m{\rightarrow}\m{[}\m{y}\m{/}\m{x}\m{]}\m{\varphi}\m{)}\m{)}
\endm
\setbox\startprefix=\hbox{\tt \ \ tfindes\ \$p\ }
\setbox\contprefix=\hbox{\tt \ \ \ \ \ \ \ \ \ \ \ \ \ }
\startm
\m{\vdash}\m{(}\m{x}\m{\in}\m{{\rm On}}\m{\rightarrow}\m{\varphi}\m{)}
\endm
\vskip 1ex

\noindent Principle of Transfinite Recursion.\index{transfinite
recursion} Theorem 7.41 of Takeuti and Zaring, p.~47.  Transfinite
recursion is the key theorem that allows arithmetic of ordinals to be
rigorously defined, and has many other important uses as well.
Hypotheses \texttt{tfr.1} and \texttt{tfr.2} specify a certain (proper)
class $ F$.  The complicated definition of $ F$ is not important in
itself; what is important is that there be such an $ F$ with the
required properties, and we show this by displaying $ F$ explicitly.
\texttt{tfr1} states that $ F$ is a function whose domain is the set of
ordinal numbers.  \texttt{tfr2} states that any value of $ F$ is
completely determined by its previous values and the values of an
auxiliary function, $G$.  \texttt{tfr3} states that $ F$ is unique,
i.e.\ it is the only function that satisfies \texttt{tfr1} and
\texttt{tfr2}.  Note that $ f$ is an individual variable like $x$ and
$y$; it is just a mnemonic to remind us that $A$ is a collection of
functions.

\vskip 0.5ex
\setbox\startprefix=\hbox{\tt \ \ tfr.1\ \$e\ }
\setbox\contprefix=\hbox{\tt \ \ \ \ \ \ \ \ \ \ \ }
\startm
\m{\vdash}\m{A}\m{=}\m{\{}\m{f}\m{|}\m{\exists}\m{x}\m{\in}\m{{\rm On}}\m{(}%
\m{f}\m{{\rm Fn}}\m{x}\m{\wedge}\m{\forall}\m{y}\m{\in}\m{x}\m{(}\m{f}\m{`}%
\m{y}\m{)}\m{=}\m{(}\m{G}\m{`}\m{(}\m{f}\m{\restriction}\m{y}\m{)}\m{)}\m{)}%
\m{\}}
\endm
\setbox\startprefix=\hbox{\tt \ \ tfr.2\ \$e\ }
\setbox\contprefix=\hbox{\tt \ \ \ \ \ \ \ \ \ \ \ }
\startm
\m{\vdash}\m{F}\m{=}\m{\bigcup}\m{A}
\endm
\setbox\startprefix=\hbox{\tt \ \ tfr1\ \$p\ }
\setbox\contprefix=\hbox{\tt \ \ \ \ \ \ \ \ \ \ }
\startm
\m{\vdash}\m{F}\m{{\rm Fn}}\m{{\rm On}}
\endm
\setbox\startprefix=\hbox{\tt \ \ tfr2\ \$p\ }
\setbox\contprefix=\hbox{\tt \ \ \ \ \ \ \ \ \ \ }
\startm
\m{\vdash}\m{(}\m{z}\m{\in}\m{{\rm On}}\m{\rightarrow}\m{(}\m{F}\m{`}\m{z}%
\m{)}\m{=}\m{(}\m{G}\m{`}\m{(}\m{F}\m{\restriction}\m{z}\m{)}\m{)}\m{)}
\endm
\setbox\startprefix=\hbox{\tt \ \ tfr3\ \$p\ }
\setbox\contprefix=\hbox{\tt \ \ \ \ \ \ \ \ \ \ }
\startm
\m{\vdash}\m{(}\m{(}\m{B}\m{{\rm Fn}}\m{{\rm On}}\m{\wedge}\m{\forall}\m{x}\m{%
\in}\m{{\rm On}}\m{(}\m{B}\m{`}\m{x}\m{)}\m{=}\m{(}\m{G}\m{`}\m{(}\m{B}\m{%
\restriction}\m{x}\m{)}\m{)}\m{)}\m{\rightarrow}\m{B}\m{=}\m{F}\m{)}
\endm
\vskip 1ex

\noindent The existence of omega (the class of natural numbers).\index{natural
number}\index{omega ($\omega$)}\index{Axiom of Infinity}  Axiom 7 of Takeuti
and Zaring, p.~43.  (This is the only theorem in this section requiring the
Axiom of Infinity.)

\vskip 0.5ex
\setbox\startprefix=\hbox{\tt \
\ omex\ \$p\ }
\setbox\contprefix=\hbox{\tt \ \ \ \ \ \ \ \ \ \ }
\startm
\m{\vdash}\m{\omega}\m{\in}\m{{\rm V}}
\endm
%\vskip 2ex


\section{Axioms for Real and Complex Numbers}\label{real}
\index{real and complex numbers!axioms for}

This section presents the axioms for real and complex numbers, along
with some commentary about them.  Analysis
textbooks implicitly or explicitly use these axioms or their equivalents
as their starting point.  In the database \texttt{set.mm}, we define real
and complex numbers as (rather complicated) specific sets and derive these
axioms as {\em theorems} from the axioms of ZF set theory, using a method
called Dedekind cuts.  We omit the details of this construction, which you can
follow if you wish using the \texttt{set.mm} database in conjunction with the
textbooks referenced therein.

Once we prove those theorems, we then restate these proven theorems as axioms.
This lets us easily identify which axioms are needed for a particular complex number proof, without the obfuscation of the set theory used to derive them.
As a result,
the construction is actually unimportant other
than to show that sets exist that satisfy the axioms, and thus that the axioms
are consistent if ZF set theory is consistent.  When working with real numbers
you can think of them as being the actual sets resulting
from the construction (for definiteness), or you can
think of them as otherwise unspecified sets that happen to satisfy the axioms.
The derivation is not easy, but the fact that it works is quite remarkable
and lends support to the idea that ZFC set theory is all we need to
provide a foundation for essentially all of mathematics.

\needspace{3\baselineskip}
\subsection{The Axioms for Real and Complex Numbers Themselves}\label{realactual}

For the axioms we are given (or postulate) 8 classes:  $\mathbb{C}$ (the
set of complex numbers), $\mathbb{R}$ (the set of real numbers, a subset
of $\mathbb{C}$), $0$ (zero), $1$ (one), $i$ (square root of
$-1$), $+$ (plus), $\cdot$ (times), and
$<_{\mathbb{R}}$ (less than for just the real numbers).
Subtraction and division are defined terms and are not part of the
axioms; for their definitions see \texttt{set.mm}.

Note that the notation $(A+B)$ (and similarly $(A\cdot B)$) specifies a class
called an {\em operation},\index{operation} and is the function value of the
class $+$ at ordered pair $\langle A,B \rangle$.  An operation is defined by
statement \texttt{df-opr} on p.~\pageref{dfopr}.
The notation $A <_{\mathbb{R}} B$ specifies a
wff called a {\em binary relation}\index{binary relation} and means $\langle A,B \rangle \in \,<_{\mathbb{R}}$, as defined by statement \texttt{df-br} on p.~\pageref{dfbr}.

Our set of 8 given classes is assumed to satisfy the following 22 axioms
(in the axioms listed below, $<$ really means $<_{\mathbb{R}}$).

\vskip 2ex

\noindent 1. The real numbers are a subset of the complex numbers.

%\vskip 0.5ex
\setbox\startprefix=\hbox{\tt \ \ ax-resscn\ \$p\ }
\setbox\contprefix=\hbox{\tt \ \ \ \ \ \ \ \ \ \ \ \ \ \ }
\startm
\m{\vdash}\m{\mathbb{R}}\m{\subseteq}\m{\mathbb{C}}
\endm
%\vskip 1ex

\noindent 2. One is a complex number.

%\vskip 0.5ex
\setbox\startprefix=\hbox{\tt \ \ ax-1cn\ \$p\ }
\setbox\contprefix=\hbox{\tt \ \ \ \ \ \ \ \ \ \ \ }
\startm
\m{\vdash}\m{1}\m{\in}\m{\mathbb{C}}
\endm
%\vskip 1ex

\noindent 3. The imaginary number $i$ is a complex number.

%\vskip 0.5ex
\setbox\startprefix=\hbox{\tt \ \ ax-icn\ \$p\ }
\setbox\contprefix=\hbox{\tt \ \ \ \ \ \ \ \ \ \ \ }
\startm
\m{\vdash}\m{i}\m{\in}\m{\mathbb{C}}
\endm
%\vskip 1ex

\noindent 4. Complex numbers are closed under addition.

%\vskip 0.5ex
\setbox\startprefix=\hbox{\tt \ \ ax-addcl\ \$p\ }
\setbox\contprefix=\hbox{\tt \ \ \ \ \ \ \ \ \ \ \ \ \ }
\startm
\m{\vdash}\m{(}\m{(}\m{A}\m{\in}\m{\mathbb{C}}\m{\wedge}\m{B}\m{\in}\m{\mathbb{C}}%
\m{)}\m{\rightarrow}\m{(}\m{A}\m{+}\m{B}\m{)}\m{\in}\m{\mathbb{C}}\m{)}
\endm
%\vskip 1ex

\noindent 5. Real numbers are closed under addition.

%\vskip 0.5ex
\setbox\startprefix=\hbox{\tt \ \ ax-addrcl\ \$p\ }
\setbox\contprefix=\hbox{\tt \ \ \ \ \ \ \ \ \ \ \ \ \ \ }
\startm
\m{\vdash}\m{(}\m{(}\m{A}\m{\in}\m{\mathbb{R}}\m{\wedge}\m{B}\m{\in}\m{\mathbb{R}}%
\m{)}\m{\rightarrow}\m{(}\m{A}\m{+}\m{B}\m{)}\m{\in}\m{\mathbb{R}}\m{)}
\endm
%\vskip 1ex

\noindent 6. Complex numbers are closed under multiplication.

%\vskip 0.5ex
\setbox\startprefix=\hbox{\tt \ \ ax-mulcl\ \$p\ }
\setbox\contprefix=\hbox{\tt \ \ \ \ \ \ \ \ \ \ \ \ \ }
\startm
\m{\vdash}\m{(}\m{(}\m{A}\m{\in}\m{\mathbb{C}}\m{\wedge}\m{B}\m{\in}\m{\mathbb{C}}%
\m{)}\m{\rightarrow}\m{(}\m{A}\m{\cdot}\m{B}\m{)}\m{\in}\m{\mathbb{C}}\m{)}
\endm
%\vskip 1ex

\noindent 7. Real numbers are closed under multiplication.

%\vskip 0.5ex
\setbox\startprefix=\hbox{\tt \ \ ax-mulrcl\ \$p\ }
\setbox\contprefix=\hbox{\tt \ \ \ \ \ \ \ \ \ \ \ \ \ \ }
\startm
\m{\vdash}\m{(}\m{(}\m{A}\m{\in}\m{\mathbb{R}}\m{\wedge}\m{B}\m{\in}\m{\mathbb{R}}%
\m{)}\m{\rightarrow}\m{(}\m{A}\m{\cdot}\m{B}\m{)}\m{\in}\m{\mathbb{R}}\m{)}
\endm
%\vskip 1ex

\noindent 8. Multiplication of complex numbers is commutative.

%\vskip 0.5ex
\setbox\startprefix=\hbox{\tt \ \ ax-mulcom\ \$p\ }
\setbox\contprefix=\hbox{\tt \ \ \ \ \ \ \ \ \ \ \ \ \ \ }
\startm
\m{\vdash}\m{(}\m{(}\m{A}\m{\in}\m{\mathbb{C}}\m{\wedge}\m{B}\m{\in}\m{\mathbb{C}}%
\m{)}\m{\rightarrow}\m{(}\m{A}\m{\cdot}\m{B}\m{)}\m{=}\m{(}\m{B}\m{\cdot}\m{A}%
\m{)}\m{)}
\endm
%\vskip 1ex

\noindent 9. Addition of complex numbers is associative.

%\vskip 0.5ex
\setbox\startprefix=\hbox{\tt \ \ ax-addass\ \$p\ }
\setbox\contprefix=\hbox{\tt \ \ \ \ \ \ \ \ \ \ \ \ \ \ }
\startm
\m{\vdash}\m{(}\m{(}\m{A}\m{\in}\m{\mathbb{C}}\m{\wedge}\m{B}\m{\in}\m{\mathbb{C}}%
\m{\wedge}\m{C}\m{\in}\m{\mathbb{C}}\m{)}\m{\rightarrow}\m{(}\m{(}\m{A}\m{+}%
\m{B}\m{)}\m{+}\m{C}\m{)}\m{=}\m{(}\m{A}\m{+}\m{(}\m{B}\m{+}\m{C}\m{)}\m{)}%
\m{)}
\endm
%\vskip 1ex

\noindent 10. Multiplication of complex numbers is associative.

%\vskip 0.5ex
\setbox\startprefix=\hbox{\tt \ \ ax-mulass\ \$p\ }
\setbox\contprefix=\hbox{\tt \ \ \ \ \ \ \ \ \ \ \ \ \ \ }
\startm
\m{\vdash}\m{(}\m{(}\m{A}\m{\in}\m{\mathbb{C}}\m{\wedge}\m{B}\m{\in}\m{\mathbb{C}}%
\m{\wedge}\m{C}\m{\in}\m{\mathbb{C}}\m{)}\m{\rightarrow}\m{(}\m{(}\m{A}\m{\cdot}%
\m{B}\m{)}\m{\cdot}\m{C}\m{)}\m{=}\m{(}\m{A}\m{\cdot}\m{(}\m{B}\m{\cdot}\m{C}%
\m{)}\m{)}\m{)}
\endm
%\vskip 1ex

\noindent 11. Multiplication distributes over addition for complex numbers.

%\vskip 0.5ex
\setbox\startprefix=\hbox{\tt \ \ ax-distr\ \$p\ }
\setbox\contprefix=\hbox{\tt \ \ \ \ \ \ \ \ \ \ \ \ \ }
\startm
\m{\vdash}\m{(}\m{(}\m{A}\m{\in}\m{\mathbb{C}}\m{\wedge}\m{B}\m{\in}\m{\mathbb{C}}%
\m{\wedge}\m{C}\m{\in}\m{\mathbb{C}}\m{)}\m{\rightarrow}\m{(}\m{A}\m{\cdot}\m{(}%
\m{B}\m{+}\m{C}\m{)}\m{)}\m{=}\m{(}\m{(}\m{A}\m{\cdot}\m{B}\m{)}\m{+}\m{(}%
\m{A}\m{\cdot}\m{C}\m{)}\m{)}\m{)}
\endm
%\vskip 1ex

\noindent 12. The square of $i$ equals $-1$ (expressed as $i$-squared plus 1 is
0).

%\vskip 0.5ex
\setbox\startprefix=\hbox{\tt \ \ ax-i2m1\ \$p\ }
\setbox\contprefix=\hbox{\tt \ \ \ \ \ \ \ \ \ \ \ \ }
\startm
\m{\vdash}\m{(}\m{(}\m{i}\m{\cdot}\m{i}\m{)}\m{+}\m{1}\m{)}\m{=}\m{0}
\endm
%\vskip 1ex

\noindent 13. One and zero are distinct.

%\vskip 0.5ex
\setbox\startprefix=\hbox{\tt \ \ ax-1ne0\ \$p\ }
\setbox\contprefix=\hbox{\tt \ \ \ \ \ \ \ \ \ \ \ \ }
\startm
\m{\vdash}\m{1}\m{\ne}\m{0}
\endm
%\vskip 1ex

\noindent 14. One is an identity element for real multiplication.

%\vskip 0.5ex
\setbox\startprefix=\hbox{\tt \ \ ax-1rid\ \$p\ }
\setbox\contprefix=\hbox{\tt \ \ \ \ \ \ \ \ \ \ \ }
\startm
\m{\vdash}\m{(}\m{A}\m{\in}\m{\mathbb{R}}\m{\rightarrow}\m{(}\m{A}\m{\cdot}\m{1}%
\m{)}\m{=}\m{A}\m{)}
\endm
%\vskip 1ex

\noindent 15. Every real number has a negative.

%\vskip 0.5ex
\setbox\startprefix=\hbox{\tt \ \ ax-rnegex\ \$p\ }
\setbox\contprefix=\hbox{\tt \ \ \ \ \ \ \ \ \ \ \ \ \ \ }
\startm
\m{\vdash}\m{(}\m{A}\m{\in}\m{\mathbb{R}}\m{\rightarrow}\m{\exists}\m{x}\m{\in}%
\m{\mathbb{R}}\m{(}\m{A}\m{+}\m{x}\m{)}\m{=}\m{0}\m{)}
\endm
%\vskip 1ex

\noindent 16. Every nonzero real number has a reciprocal.

%\vskip 0.5ex
\setbox\startprefix=\hbox{\tt \ \ ax-rrecex\ \$p\ }
\setbox\contprefix=\hbox{\tt \ \ \ \ \ \ \ \ \ \ \ \ \ \ }
\startm
\m{\vdash}\m{(}\m{A}\m{\in}\m{\mathbb{R}}\m{\rightarrow}\m{(}\m{A}\m{\ne}\m{0}%
\m{\rightarrow}\m{\exists}\m{x}\m{\in}\m{\mathbb{R}}\m{(}\m{A}\m{\cdot}%
\m{x}\m{)}\m{=}\m{1}\m{)}\m{)}
\endm
%\vskip 1ex

\noindent 17. A complex number can be expressed in terms of two reals.

%\vskip 0.5ex
\setbox\startprefix=\hbox{\tt \ \ ax-cnre\ \$p\ }
\setbox\contprefix=\hbox{\tt \ \ \ \ \ \ \ \ \ \ \ \ }
\startm
\m{\vdash}\m{(}\m{A}\m{\in}\m{\mathbb{C}}\m{\rightarrow}\m{\exists}\m{x}\m{\in}%
\m{\mathbb{R}}\m{\exists}\m{y}\m{\in}\m{\mathbb{R}}\m{A}\m{=}\m{(}\m{x}\m{+}\m{(}%
\m{y}\m{\cdot}\m{i}\m{)}\m{)}\m{)}
\endm
%\vskip 1ex

\noindent 18. Ordering on reals satisfies strict trichotomy.

%\vskip 0.5ex
\setbox\startprefix=\hbox{\tt \ \ ax-pre-lttri\ \$p\ }
\setbox\contprefix=\hbox{\tt \ \ \ \ \ \ \ \ \ \ \ \ \ }
\startm
\m{\vdash}\m{(}\m{(}\m{A}\m{\in}\m{\mathbb{R}}\m{\wedge}\m{B}\m{\in}\m{\mathbb{R}}%
\m{)}\m{\rightarrow}\m{(}\m{A}\m{<}\m{B}\m{\leftrightarrow}\m{\lnot}\m{(}\m{A}%
\m{=}\m{B}\m{\vee}\m{B}\m{<}\m{A}\m{)}\m{)}\m{)}
\endm
%\vskip 1ex

\noindent 19. Ordering on reals is transitive.

%\vskip 0.5ex
\setbox\startprefix=\hbox{\tt \ \ ax-pre-lttrn\ \$p\ }
\setbox\contprefix=\hbox{\tt \ \ \ \ \ \ \ \ \ \ \ \ \ }
\startm
\m{\vdash}\m{(}\m{(}\m{A}\m{\in}\m{\mathbb{R}}\m{\wedge}\m{B}\m{\in}\m{\mathbb{R}}%
\m{\wedge}\m{C}\m{\in}\m{\mathbb{R}}\m{)}\m{\rightarrow}\m{(}\m{(}\m{A}\m{<}%
\m{B}\m{\wedge}\m{B}\m{<}\m{C}\m{)}\m{\rightarrow}\m{A}\m{<}\m{C}\m{)}\m{)}
\endm
%\vskip 1ex

\noindent 20. Ordering on reals is preserved after addition to both sides.

%\vskip 0.5ex
\setbox\startprefix=\hbox{\tt \ \ ax-pre-ltadd\ \$p\ }
\setbox\contprefix=\hbox{\tt \ \ \ \ \ \ \ \ \ \ \ \ \ }
\startm
\m{\vdash}\m{(}\m{(}\m{A}\m{\in}\m{\mathbb{R}}\m{\wedge}\m{B}\m{\in}\m{\mathbb{R}}%
\m{\wedge}\m{C}\m{\in}\m{\mathbb{R}}\m{)}\m{\rightarrow}\m{(}\m{A}\m{<}\m{B}\m{%
\rightarrow}\m{(}\m{C}\m{+}\m{A}\m{)}\m{<}\m{(}\m{C}\m{+}\m{B}\m{)}\m{)}\m{)}
\endm
%\vskip 1ex

\noindent 21. The product of two positive reals is positive.

%\vskip 0.5ex
\setbox\startprefix=\hbox{\tt \ \ ax-pre-mulgt0\ \$p\ }
\setbox\contprefix=\hbox{\tt \ \ \ \ \ \ \ \ \ \ \ \ \ \ }
\startm
\m{\vdash}\m{(}\m{(}\m{A}\m{\in}\m{\mathbb{R}}\m{\wedge}\m{B}\m{\in}\m{\mathbb{R}}%
\m{)}\m{\rightarrow}\m{(}\m{(}\m{0}\m{<}\m{A}\m{\wedge}\m{0}%
\m{<}\m{B}\m{)}\m{\rightarrow}\m{0}\m{<}\m{(}\m{A}\m{\cdot}\m{B}\m{)}%
\m{)}\m{)}
\endm
%\vskip 1ex

\noindent 22. A non-empty, bounded-above set of reals has a supremum.

%\vskip 0.5ex
\setbox\startprefix=\hbox{\tt \ \ ax-pre-sup\ \$p\ }
\setbox\contprefix=\hbox{\tt \ \ \ \ \ \ \ \ \ \ \ }
\startm
\m{\vdash}\m{(}\m{(}\m{A}\m{\subseteq}\m{\mathbb{R}}\m{\wedge}\m{A}\m{\ne}\m{%
\varnothing}\m{\wedge}\m{\exists}\m{x}\m{\in}\m{\mathbb{R}}\m{\forall}\m{y}\m{%
\in}\m{A}\m{\,y}\m{<}\m{x}\m{)}\m{\rightarrow}\m{\exists}\m{x}\m{\in}\m{%
\mathbb{R}}\m{(}\m{\forall}\m{y}\m{\in}\m{A}\m{\lnot}\m{x}\m{<}\m{y}\m{\wedge}\m{%
\forall}\m{y}\m{\in}\m{\mathbb{R}}\m{(}\m{y}\m{<}\m{x}\m{\rightarrow}\m{\exists}%
\m{z}\m{\in}\m{A}\m{\,y}\m{<}\m{z}\m{)}\m{)}\m{)}
\endm

% NOTE: The \m{...} expressions above could be represented as
% $ \vdash ( ( A \subseteq \mathbb{R} \wedge A \ne \varnothing \wedge \exists x \in \mathbb{R} \forall y \in A \,y < x ) \rightarrow \exists x \in \mathbb{R} ( \forall y \in A \lnot x < y \wedge \forall y \in \mathbb{R} ( y < x \rightarrow \exists z \in A \,y < z ) ) ) $

\vskip 2ex

This completes the set of axioms for real and complex numbers.  You may
wish to look at how subtraction, division, and decimal numbers
are defined in \texttt{set.mm}, and for fun look at the proof of $2+
2 = 4$ (theorem \texttt{2p2e4} in \texttt{set.mm})
as discussed in section \ref{2p2e4}.

In \texttt{set.mm} we define the non-negative integers $\mathbb{N}$, the integers
$\mathbb{Z}$, and the rationals $\mathbb{Q}$ as subsets of $\mathbb{R}$.  This leads
to the nice inclusion $\mathbb{N} \subseteq \mathbb{Z} \subseteq \mathbb{Q} \subseteq
\mathbb{R} \subseteq \mathbb{C}$, giving us a uniform framework in which, for
example, a property such as commutativity of complex number addition
automatically applies to integers.  The natural numbers $\mathbb{N}$
are different from the set $\omega$ we defined earlier, but both satisfy
Peano's postulates.

\subsection{Complex Number Axioms in Analysis Texts}

Most analysis texts construct complex numbers as ordered pairs of reals,
leading to construction-dependent properties that satisfy these axioms
but are not stated in their pure form.  (This is also done in
\texttt{set.mm} but our axioms are extracted from that construction.)
Other texts will simply state that $\mathbb{R}$ is a ``complete ordered
subfield of $\mathbb{C}$,'' leading to redundant axioms when this phrase
is completely expanded out.  In fact I have not seen a text with the
axioms in the explicit form above.
None of these axioms is unique individually, but this carefully worked out
collection of axioms is the result of years of work
by the Metamath community.

\subsection{Eliminating Unnecessary Complex Number Axioms}

We once had more axioms for real and complex numbers, but over years of time
we (the Metamath community)
have found ways to eliminate them (by proving them from other axioms)
or weaken them (by making weaker claims without reducing what
can be proved).
In particular, here are statements that used to be complex number
axioms but have since been formally proven (with Metamath) to be redundant:

\begin{itemize}
\item
  $\mathbb{C} \in V$.
  At one time this was listed as a ``complex number axiom.''
  However, this is not properly speaking a complex number axiom,
  and in any case its proof uses axioms of set theory.
  Proven redundant by Mario Carneiro\index{Carneiro, Mario} on
  17-Nov-2014 (see \texttt{axcnex}).
\item
  $((A \in \mathbb{C} \land B \in \mathbb{C}$) $\rightarrow$
  $(A + B) = (B + A))$.
  Proved redundant by Eric Schmidt\index{Schmidt, Eric} on 19-Jun-2012,
  and formalized by Scott Fenton\index{Fenton, Scott} on 3-Jan-2013
  (see \texttt{addcom}).
\item
  $(A \in \mathbb{C} \rightarrow (A + 0) = A)$.
  Proved redundant by Eric Schmidt on 19-Jun-2012,
  and formalized by Scott Fenton on 3-Jan-2013
  (see \texttt{addid1}).
\item
  $(A \in \mathbb{C} \rightarrow \exists x \in \mathbb{C} (A + x) = 0)$.
  Proved redundant by Eric Schmidt and formalized on 21-May-2007
  (see \texttt{cnegex}).
\item
  $((A \in \mathbb{C} \land A \ne 0) \rightarrow \exists x \in \mathbb{C} (A \cdot x) = 1)$.
  Proved redundant by Eric Schmidt and formalized on 22-May-2007
  (see \texttt{recex}).
\item
  $0 \in \mathbb{R}$.
  Proved redundant by Eric Schmidt on 19-Feb-2005 and formalized 21-May-2007
  (see \texttt{0re}).
\end{itemize}

We could eliminate 0 as an axiomatic object by defining it as
$( ( i \cdot i ) + 1 )$
and replacing it with this expression throughout the axioms. If this
is done, axiom ax-i2m1 becomes redundant. However, the remaining axioms
would become longer and less intuitive.

Eric Schmidt's paper analyzing this axiom system \cite{Schmidt}
presented a proof that these remaining axioms,
with the possible exception of ax-mulcom, are independent of the others.
It is currently an open question if ax-mulcom is independent of the others.

\section{Two Plus Two Equals Four}\label{2p2e4}

Here is a proof that $2 + 2 = 4$, as proven in the theorem \texttt{2p2e4}
in the database \texttt{set.mm}.
This is a useful demonstration of what a Metamath proof can look like.
This proof may have more steps than you're used to, but each step is rigorously
proven all the way back to the axioms of logic and set theory.
This display was originally generated by the Metamath program
as an {\sc HTML} file.

In the table showing the proof ``Step'' is the sequential step number,
while its associated ``Expression'' is an expression that we have proved.
``Ref'' is the name of a theorem or axiom that justifies that expression,
and ``Hyp'' refers to previous steps (if any) that the theorem or axiom
needs so that we can use it.  Expressions are indented further than
the expressions that depend on them to show their interdependencies.

\begin{table}[!htbp]
\caption{Two plus two equals four}
\begin{tabular}{lllll}
\textbf{Step} & \textbf{Hyp} & \textbf{Ref} & \textbf{Expression} & \\
1  &       & df-2    & $ \; \; \vdash 2 = 1 + 1$  & \\
2  & 1     & oveq2i  & $ \; \vdash (2 + 2) = (2 + (1 + 1))$ & \\
3  &       & df-4    & $ \; \; \vdash 4 = (3 + 1)$ & \\
4  &       & df-3    & $ \; \; \; \vdash 3 = (2 + 1)$ & \\
5  & 4     & oveq1i  & $ \; \; \vdash (3 + 1) = ((2 + 1) + 1)$ & \\
6  &       & 2cn     & $ \; \; \; \vdash 2 \in \mathbb{C}$ & \\
7  &       & ax-1cn  & $ \; \; \; \vdash 1 \in \mathbb{C}$ & \\
8  & 6,7,7 & addassi & $ \; \; \vdash ((2 + 1) + 1) = (2 + (1 + 1))$ & \\
9  & 3,5,8 & 3eqtri  & $ \; \vdash 4 = (2 + (1 + 1))$ & \\
10 & 2,9   & eqtr4i  & $ \vdash (2 + 2) = 4$ & \\
\end{tabular}
\end{table}

Step 1 says that we can assert that $2 = 1 + 1$ because it is
justified by \texttt{df-2}.
What is \texttt{df-2}?
It is simply the definition of $2$, which in our system is defined as being
equal to $1 + 1$.  This shows how we can use definitions in proofs.

Look at Step 2 of the proof. In the Ref column, we see that it references
a previously proved theorem, \texttt{oveq2i}.
It turns out that
theorem \texttt{oveq2i} requires a
hypothesis, and in the Hyp column of Step 2 we indicate that Step 1 will
satisfy (match) this hypothesis.
If we looked at \texttt{oveq2i}
we would find that it proves that given some hypothesis
$A = B$, we can prove that $( C F A ) = ( C F B )$.
If we use \texttt{oveq2i} and apply step 1's result as the hypothesis,
that will mean that $A = 2$ and $B = ( 1 + 1 )$ within this use of
\texttt{oveq2i}.
We are free to select any value of $C$ and $F$ (subject to syntax constraints),
so we are free to select $C = 2$ and $F = +$,
producing our desired result,
$ (2 + 2) = (2 + (1 + 1))$.

Step 2 is an example of substitution.
In the end, every step in every proof uses only this one substitution rule.
All the rules of logic, and all the axioms, are expressed so that
they can be used via this one substitution rule.
So once you master substitution, you can master every Metamath proof,
no exceptions.

Each step is clear and can be immediately checked.
In the {\sc HTML} display you can even click on each reference to see why it is
justified, making it easy to see why the proof works.

\section{Deduction}\label{deduction}

Strictly speaking,
a deduction (also called an inference) is a kind of statement that needs
some hypotheses to be true in order for its conclusion to be true.
A theorem, on the other hand, has no hypotheses.
Informally we often call both of them theorems, but in this section we
will stick to the strict definitions.

It sometimes happens that we have proved a deduction of the form
$\varphi \Rightarrow \psi$\index{$\Rightarrow$}
(given hypothesis $\varphi$ we can prove $\psi$)
and we want to then prove a theorem of the form
$\varphi \rightarrow \psi$.

Converting a deduction (which uses a hypothesis) into a theorem
(which does not) is not as simple as you might think.
The deduction says, ``if we can prove $\varphi$ then we can prove $\psi$,''
which is in some sense weaker than saying
``$\varphi$ implies $\psi$.''
There is no axiom of logic that permits us to directly obtain the theorem
given the deduction.\footnote{
The conversion of a deduction to a theorem does not even hold in general
for quantum propositional calculus,
which is a weak subset of classical propositional calculus.
It has been shown that adding the Standard Deduction Theorem (discussed below)
to quantum propositional calculus turns it into classical
propositional calculus!
}

This is in contrast to going the other way.
If we have the theorem ($\varphi \rightarrow \psi$),
it is easy to recover the deduction
($\varphi \Rightarrow \psi$)
using modus ponens\index{modus ponens}
(\texttt{ax-mp}; see section \ref{axmp}).

In the following subsections we first discuss the standard deduction theorem
(the traditional but awkward way to convert deductions into theorems) and
the weak deduction theorem (a limited version of the standard deduction
theorem that is easier to use and was once widely used in
\texttt{set.mm}\index{set theory database (\texttt{set.mm})}).
In section \ref{deductionstyle} we discuss
deduction style, the newer approach we now recommend in most cases.
Deduction style uses ``deduction form,'' a form that
prefixes each hypothesis (other than definitions) and the
conclusion with a universal antecedent (``$\varphi \rightarrow$'').
Deduction style is widely used in \texttt{set.mm},
so it is useful to understand it and \textit{why} it is widely used.
Section \ref{naturaldeduction}
briefly discusses our approach for using natural deduction
within \texttt{set.mm},
as that approach is deeply related to deduction style.
We conclude with a summary of the strengths of
our approach, which we believe are compelling.

\subsection{The Standard Deduction Theorem}\label{standarddeductiontheorem}

It is possible to make use of information
contained in the deduction or its proof to assist us with the proof of
the related theorem.
In traditional logic books, there is a metatheorem called the
Deduction Theorem\index{Deduction Theorem}\index{Standard Deduction Theorem},
discovered independently by Herbrand and Tarski around 1930.
The Deduction Theorem, which we often call the Standard Deduction Theorem,
provides an algorithm for constructing a proof of a theorem from the
proof of its corresponding deduction. See, for example,
\cite[p.~56]{Margaris}\index{Margaris, Angelo}.
To construct a proof for a theorem, the
algorithm looks at each step in the proof of the original deduction and
rewrites the step with several steps wherein the hypothesis is eliminated
and becomes an antecedent.

In ordinary mathematics, no one actually carries out the algorithm,
because (in its most basic form) it involves an exponential explosion of
the number of proof steps as more hypotheses are eliminated. Instead,
the Standard Deduction Theorem is invoked simply to claim that it can
be done in principle, without actually doing it.
What's more, the algorithm is not as simple as it might first appear
when applying it rigorously.
There is a subtle restriction on the Standard Deduction Theorem
that must be taken into account involving the axiom of generalization
when working with predicate calculus (see the literature for more detail).

One of the goals of Metamath is to let you plainly see, with as few
underlying concepts as possible, how mathematics can be derived directly
from the axioms, and not indirectly according to some hidden rules
buried inside a program or understood only by logicians. If we added
the Standard Deduction Theorem to the language and proof verifier,
that would greatly complicate both and largely defeat Metamath's goal
of simplicity. In principle, we could show direct proofs by expanding
out the proof steps generated by the algorithm of the Standard Deduction
Theorem, but that is not feasible in practice because the number of proof
steps quickly becomes huge, even astronomical.
Since the algorithm of the Standard Deduction Theorem is driven by the proof,
we would have to go through that proof
all over again---starting from axioms---in order to obtain the theorem form.
In terms of proof length, there would be no savings over just
proving the theorem directly instead of first proving the deduction form.

\subsection{Weak Deduction Theorem}\label{weakdeductiontheorem}

We have developed
a more efficient method for proving a theorem from a deduction
that can be used instead of the Standard Deduction Theorem
in many (but not all) cases.
We call this more efficient method the
Weak Deduction Theorem\index{Weak Deduction Theorem}.\footnote{
There is also an unrelated ``Weak Deduction Theorem''
in the field of relevance logic, so to avoid confusion we could call
ours the ``Weak Deduction Theorem for Classical Logic.''}
Unlike the Standard Deduction Theorem, the Weak Deduction Theorem produces the
theorem directly from a special substitution instance of the deduction,
using a small, fixed number of steps roughly proportional to the length
of the final theorem.

If you come to a proof referencing the Weak Deduction Theorem
\texttt{dedth} (or one of its variants \texttt{dedthxx}),
here is how to follow the proof without getting into the details:
just click on the theorem referenced in the step
just before the reference to \texttt{dedth} and ignore everything else.
Theorem \texttt{dedth} simply turns a hypothesis into an antecedent
(i.e. the hypothesis followed by $\rightarrow$
is placed in front of the assertion, and the hypothesis
itself is eliminated) given certain conditions.

The Weak Deduction Theorem
eliminates a hypothesis $\varphi$, making it become an antecedent.
It does this by proving an expression
$ \varphi \rightarrow \psi $ given two hypotheses:
(1)
$ ( A = {\rm if} ( \varphi , A , B ) \rightarrow ( \varphi \leftrightarrow \chi ) ) $
and
(2) $\chi$.
Note that it requires that a proof exists for $\varphi$ when the class variable
$A$ is replaced with a specific class $B$. The hypothesis $\chi$
should be assigned to the inference.
You can see the details of the proof of the Weak Deduction Theorem
in theorem \texttt{dedth}.

The Weak Deduction Theorem
is probably easier to understand by studying proofs that make use of it.
For example, let's look at the proof of \texttt{renegcl}, which proves that
$ \vdash ( A \in \mathbb{R} \rightarrow - A \in \mathbb{R} )$:

\needspace{4\baselineskip}
\begin{longtabu} {l l l X}
\textbf{Step} & \textbf{Hyp} & \textbf{Ref} & \textbf{Expression} \\
  1 &  & negeq &
  $\vdash$ $($ $A$ $=$ ${\rm if}$ $($ $A$ $\in$ $\mathbb{R}$ $,$ $A$ $,$ $1$ $)$ $\rightarrow$
  $\textrm{-}$ $A$ $=$ $\textrm{-}$ ${\rm if}$ $($ $A$ $\in$ $\mathbb{R}$
  $,$ $A$ $,$ $1$ $)$ $)$ \\
 2 & 1 & eleq1d &
    $\vdash$ $($ $A$ $=$ ${\rm if}$ $($ $A$ $\in$ $\mathbb{R}$ $,$ $A$ $,$ $1$ $)$ $\rightarrow$ $($
    $\textrm{-}$ $A$ $\in$ $\mathbb{R}$ $\leftrightarrow$
    $\textrm{-}$ ${\rm if}$ $($ $A$ $\in$ $\mathbb{R}$ $,$ $A$ $,$ $1$ $)$ $\in$
    $\mathbb{R}$ $)$ $)$ \\
 3 &  & 1re & $\vdash 1 \in \mathbb{R}$ \\
 4 & 3 & elimel &
   $\vdash {\rm if} ( A \in \mathbb{R} , A , 1 ) \in \mathbb{R}$ \\
 5 & 4 & renegcli &
   $\vdash \textrm{-} {\rm if} ( A \in \mathbb{R} , A , 1 ) \in \mathbb{R}$ \\
 6 & 2,5 & dedth &
   $\vdash ( A \in \mathbb{R} \rightarrow \textrm{-} A \in \mathbb{R}$ ) \\
\end{longtabu}

The somewhat strange-looking steps in \texttt{renegcl} before step 5 are
technical stuff that makes this magic work, and they can be ignored
for a quick overview of the proof. To continue following the ``important''
part of the proof of \texttt{renegcl},
you can look at the reference to \texttt{renegcli} at step 5.

That said, let's briefly look at how
\texttt{renegcl} uses the
Weak Deduction Theorem (\texttt{dedth}) to do its job,
in case you want to do something similar or want understand it more deeply.
Let's work backwards in the proof of \texttt{renegcl}.
Step 6 applies \texttt{dedth} to produce our goal result
$ \vdash ( A \in \mathbb{R} \rightarrow\, - A \in \mathbb{R} )$.
This requires on the one hand the (substituted) deduction
\texttt{renegcli} in step 5.
By itself \texttt{renegcli} proves the deduction
$ \vdash A \in \mathbb{R} \Rightarrow\, \vdash - A \in \mathbb{R}$;
this is the deduction form we are trying to turn into theorem form,
and thus
\texttt{renegcli} has a separate hypothesis that must be fulfilled.
To fulfill the hypothesis of the invocation of
\texttt{renegcli} in step 5, it is eventually
reduced to the already proven theorem $1 \in \mathbb{R}$ in step 3.
Step 4 connects steps 3 and 5; step 4 invokes
\texttt{elimel}, a special case of \texttt{elimhyp} that eliminates
a membership hypothesis for the weak deduction theorem.
On the other hand, the equivalence of the conclusion of
\texttt{renegcl}
$( - A \in \mathbb{R} )$ and the substituted conclusion of
\texttt{renegcli} must be proven, which is done in steps 2 and 1.

The weak deduction theorem has limitations.
In particular, we must be able to prove a special case of the deduction's
hypothesis as a stand-alone theorem.
For example, we used $1 \in \mathbb{R}$ in step 3 of \texttt{renegcl}.

We used to use the weak deduction theorem
extensively within \texttt{set.mm}.
However, we now recommend applying ``deduction style''
instead in most cases, as deduction style is
often an easier and clearer approach.
Therefore, we will now describe deduction style.

\subsection{Deduction Style}\label{deductionstyle}

We now prefer to write assertions in ``deduction form''
instead of writing a proof that would require use of the standard or
weak deduction theorem.
We call this appraoch
``deduction style.''\index{deduction style}

It will be easier to explain this by first defining some terms:

\begin{itemize}
\item \textbf{closed form}\index{closed form}\index{forms!closed}:
A kind of assertion (theorem) with no hypotheses.
Typically its label has no special suffix.
An example is \texttt{unss}, which states:
$\vdash ( ( A \subseteq C \wedge B \subseteq C ) \leftrightarrow ( A \cup B )
\subseteq C )\label{eq:unss}$
\item \textbf{deduction form}\index{deduction form}\index{forms!deduction}:
A kind of assertion with one or more hypotheses
where the conclusion is an implication with
a wff variable as the antecedent (usually $\varphi$), and every hypothesis
(\$e statement)
is either (1) an implication with the same antecedent as the conclusion or
(2) a definition.
A definition
can be for a class variable (this is a class variable followed by ``='')
or a wff variable (this is a wff variable followed by $\leftrightarrow$);
class variable definitions are more common.
In practice, a proof
in deduction form will also contain many steps that are implications
where the antecedent is either that wff variable (normally $\varphi$)
or is
a conjunction (...$\land$...) including that wff variable ($\varphi$).
If an assertion is in deduction form, and other forms are also available,
then we suffix its label with ``d.''
An example is \texttt{unssd}, which states\footnote{
For brevity we show here (and in other places)
a $\&$\index{$\&$} between hypotheses\index{hypotheses}
and a $\Rightarrow$\index{$\Rightarrow$}\index{conclusion}
between the hypotheses and the conclusion.
This notation is technically not part of the Metamath language, but is
instead a convenient abbreviation to show both the hypotheses and conclusion.}:
$\vdash ( \varphi \rightarrow A \subseteq C )\quad\&\quad \vdash ( \varphi
    \rightarrow B \subseteq C )\quad\Rightarrow\quad \vdash ( \varphi
    \rightarrow ( A \cup B ) \subseteq C )\label{eq:unssd}$
\item \textbf{inference form}\index{inference form}\index{forms!inference}:
A kind of assertion with one or more hypotheses that is not in deduction form
(e.g., there is no common antecedent).
If an assertion is in inference form, and other forms are also available,
then we suffix its label with ``i.''
An example is \texttt{unssi}, which states:
$\vdash A \subseteq C\quad\&\quad \vdash B \subseteq C\quad\Rightarrow\quad
    \vdash ( A \cup B ) \subseteq C\label{eq:unssi}$
\end{itemize}

When using deduction style we express an assertion in deduction form.
This form prefixes each hypothesis (other than definitions) and the
conclusion with a universal antecedent (``$\varphi \rightarrow$'').
The antecedent (e.g., $\varphi$)
mimics the context handled in the deduction theorem, eliminating
the need to directly use the deduction theorem.

Once you have an assertion in deduction form, you can easily convert it
to inference form or closed form:

\begin{itemize}
\item To
prove some assertion Ti in inference form, given assertion Td in deduction
form, there is a simple mechanical process you can use. First take each
Ti hypothesis and insert a \texttt{T.} $\rightarrow$ prefix (``true implies'')
using \texttt{a1i}. You
can then use the existing assertion Td to prove the resulting conclusion
with a \texttt{T.} $\rightarrow$ prefix.
Finally, you can remove that prefix using \texttt{mptru},
resulting in the conclusion you wanted to prove.
\item To
prove some assertion T in closed form, given assertion Td in deduction
form, there is another simple mechanical process you can use. First,
select an expression that is the conjunction (...$\land$...) of all of the
consequents of every hypothesis of Td. Next, prove that this expression
implies each of the separate hypotheses of Td in turn by eliminating
conjuncts (there are a variety of proven assertions to do this, including
\texttt{simpl},
\texttt{simpr},
\texttt{3simpa},
\texttt{3simpb},
\texttt{3simpc},
\texttt{simp1},
\texttt{simp2},
and
\texttt{simp3}).
If the
expression has nested conjunctions, inner conjuncts can be broken out by
chaining the above theorems with \texttt{syl}
(see section \ref{syl}).\footnote{
There are actually many theorems
(labeled simp* such as \texttt{simp333}) that break out inner conjuncts in one
step, but rather than learning them you can just use the chaining we
just described to prove them, and then let the Metamath program command
\texttt{minimize{\char`\_}with}\index{\texttt{minimize{\char`\_}with} command}
figure out the right ones needed to collapse them.}
As your final step, you can then apply the already-proven assertion Td
(which is in deduction form), proving assertion T in closed form.
\end{itemize}

We can also easily convert any assertion T in closed form to its related
assertion Ti in inference form by applying
modus ponens\index{modus ponens} (see section \ref{axmp}).

The deduction form antecedent can also be used to represent the context
necessary to support natural deduction systems, so we will now
discuss natural deduction.

\subsection{Natural Deduction}\label{naturaldeduction}

Natural deduction\index{natural deduction}
(ND) systems, as such, were originally introduced in
1934 by two logicians working independently: Ja\'skowski and Gentzen. ND
systems are supposed to reconstruct, in a formally proper way, traditional
ways of mathematical reasoning (such as conditional proof, indirect proof,
and proof by cases). As reconstructions they were naturally influenced
by previous work, and many specific ND systems and notations have been
developed since their original work.

There are many ND variants, but
Indrzejczak \cite[p.~31-32]{Indrzejczak}\index{Indrzejczak, Andrzej}
suggests that any natural deductive system must satisfy at
least these three criteria:

\begin{itemize}
\item ``There are some means for entering assumptions into a proof and
also for eliminating them. Usually it requires some bookkeeping devices
for indicating the scope of an assumption, and showing that a part of
a proof depending on eliminated assumption is discharged.
\item There are no (or, at least, a very limited set of) axioms, because
their role is taken over by the set of primitive rules for introduction
and elimination of logical constants which means that elementary
inferences instead of formulae are taken as primitive.
\item (A genuine) ND system admits a lot of freedom in proof construction
and possibility of applying several proof search strategies, like
conditional proof, proof by cases, proof by reductio ad absurdum etc.''
\end{itemize}

The Metamath Proof Explorer (MPE) as defined in \texttt{set.mm}
is fundamentally a Hilbert-style system.
That is, MPE is based on a larger number of axioms (compared
to natural deduction systems), a very small set of rules of inference
(modus ponens), and the context is not changed by the rules of inference
in the middle of a proof. That said, MPE proofs can be developed using
the natural deduction (ND) approach as originally developed by Ja\'skowski
and Gentzen.

The most common and recommended approach for applying ND in MPE is to use
deduction form\index{deduction form}%
\index{forms!deduction}
and apply the MPE proven assertions that are equivalent to ND rules.
For example, MPE's \texttt{jca} is equivalent to ND rule $\land$-I
(and-insertion).
We maintain a list of equivalences that you may consult.
This approach for applying an ND approach within MPE relies on Metamath's
wff metavariables in an essential way, and is described in more detail
in the presentation ``Natural Deductions in the Metamath Proof Language''
by Mario Carneiro \cite{CarneiroND}\index{Carneiro, Mario}.

In this style many steps are an implication, whose antecedent mimics
the context ($\Gamma$) of most ND systems. To add an assumption, simply add
it to the implication antecedent (typically using
\texttt{simpr}),
and use that
new antecedent for all later claims in the same scope. If you wish to
use an assertion in an ND hypothesis scope that is outside the current
ND hypothesis scope, modify the assertion so that the ND hypothesis
assumption is added to its antecedent (typically using \texttt{adantr}). Most
proof steps will be proved using rules that have hypotheses and results
of the form $\varphi \rightarrow$ ...

An example may make this clearer.
Let's show theorem 5.5 of
\cite[p.~18]{Clemente}\index{Clemente Laboreo, Daniel}
along with a line by line translation using the usual
translation of natural deduction (ND) in the Metamath Proof Explorer
(MPE) notation (this is proof \texttt{ex-natded5.5}).
The proof's original goal was to prove
$\lnot \psi$ given two hypotheses,
$( \psi \rightarrow \chi )$ and $ \lnot \chi$.
We will translate these statements into MPE deduction form
by prefixing them all with $\varphi \rightarrow$.
As a result, in MPE the goal is stated as
$( \varphi \rightarrow \lnot \psi )$, and the two hypotheses are stated as
$( \varphi \rightarrow ( \psi \rightarrow \chi ) )$ and
$( \varphi \rightarrow \lnot \chi )$.

The following table shows the proof in Fitch natural deduction style
and its MPE equivalent.
The \textit{\#} column shows the original numbering,
\textit{MPE\#} shows the number in the equivalent MPE proof
(which we will show later),
\textit{ND Expression} shows the original proof claim in ND notation,
and \textit{MPE Translation} shows its translation into MPE
as discussed in this section.
The final columns show the rationale in ND and MPE respectively.

\needspace{4\baselineskip}
{\setlength{\extrarowsep}{4pt} % Keep rows from being too close together
\begin{longtabu}   { @{} c c X X X X }
\textbf{\#} & \textbf{MPE\#} & \textbf{ND Ex\-pres\-sion} &
\textbf{MPE Trans\-lation} & \textbf{ND Ration\-ale} &
\textbf{MPE Ra\-tio\-nale} \\
\endhead

1 & 2;3 &
$( \psi \rightarrow \chi )$ &
$( \varphi \rightarrow ( \psi \rightarrow \chi ) )$ &
Given &
\$e; \texttt{adantr} to put in ND hypothesis \\

2 & 5 &
$ \lnot \chi$ &
$( \varphi \rightarrow \lnot \chi )$ &
Given &
\$e; \texttt{adantr} to put in ND hypothesis \\

3 & 1 &
... $\vert$ $\psi$ &
$( \varphi \rightarrow \psi )$ &
ND hypothesis assumption &
\texttt{simpr} \\

4 & 4 &
... $\chi$ &
$( ( \varphi \land \psi ) \rightarrow \chi )$ &
$\rightarrow$\,E 1,3 &
\texttt{mpd} 1,3 \\

5 & 6 &
... $\lnot \chi$ &
$( ( \varphi \land \psi ) \rightarrow \lnot \chi )$ &
IT 2 &
\texttt{adantr} 5 \\

6 & 7 &
$\lnot \psi$ &
$( \varphi \rightarrow \lnot \psi )$ &
$\land$\,I 3,4,5 &
\texttt{pm2.65da} 4,6 \\

\end{longtabu}
}


The original used Latin letters; we have replaced them with Greek letters
to follow Metamath naming conventions and so that it is easier to follow
the Metamath translation. The Metamath line-for-line translation of
this natural deduction approach precedes every line with an antecedent
including $\varphi$ and uses the Metamath equivalents of the natural deduction
rules. To add an assumption, the antecedent is modified to include it
(typically by using \texttt{adantr};
\texttt{simpr} is useful when you want to
depend directly on the new assumption, as is shown here).

In Metamath we can represent the two given statements as these hypotheses:

\needspace{2\baselineskip}
\begin{itemize}
\item ex-natded5.5.1 $\vdash ( \varphi \rightarrow ( \psi \rightarrow \chi ) )$
\item ex-natded5.5.2 $\vdash ( \varphi \rightarrow \lnot \chi )$
\end{itemize}

\needspace{4\baselineskip}
Here is the proof in Metamath as a line-by-line translation:

\begin{longtabu}   { l l l X }
\textbf{Step} & \textbf{Hyp} & \textbf{Ref} & \textbf{Ex\-pres\-sion} \\
\endhead
1 & & simpr & $\vdash ( ( \varphi \land \psi ) \rightarrow \psi )$ \\
2 & & ex-natded5.5.1 &
  $\vdash ( \varphi \rightarrow ( \psi \rightarrow \chi ) )$ \\
3 & 2 & adantr &
 $\vdash ( ( \varphi \land \psi ) \rightarrow ( \psi \rightarrow \chi ) )$ \\
4 & 1, 3 & mpd &
 $\vdash ( ( \varphi \land \psi ) \rightarrow \chi ) $ \\
5 & & ex-natded5.5.2 &
 $\vdash ( \varphi \rightarrow \lnot \chi )$ \\
6 & 5 & adantr &
 $\vdash ( ( \varphi \land \psi ) \rightarrow \lnot \chi )$ \\
7 & 4, 6 & pm2.65da &
 $\vdash ( \varphi \rightarrow \lnot \psi )$ \\
\end{longtabu}

Only using specific natural deduction rules directly can lead to very
long proofs, for exactly the same reason that only using axioms directly
in Hilbert-style proofs can lead to very long proofs.
If the goal is short and clear proofs,
then it is better to reuse already-proven assertions
in deduction form than to start from scratch each time
and using only basic natural deduction rules.

\subsection{Strengths of Our Approach}

As far as we know there is nothing else in the literature like either the
weak deduction theorem or Mario Carneiro\index{Carneiro, Mario}'s
natural deduction method.
In order to
transform a hypothesis into an antecedent, the literature's standard
``Deduction Theorem''\index{Deduction Theorem}\index{Standard Deduction Theorem}
requires metalogic outside of the notions provided
by the axiom system. We instead generally prefer to use Mario Carneiro's
natural deduction method, then use the weak deduction theorem in cases
where that is difficult to apply, and only then use the full standard
deduction theorem as a last resort.

The weak deduction theorem\index{Weak Deduction Theorem}
does not require any additional metalogic
but converts an inference directly into a closed form theorem, with
a rigorous proof that uses only the axiom system. Unlike the standard
Deduction Theorem, there is no implicit external justification that we
have to trust in order to use it.

Mario Carneiro's natural deduction\index{natural deduction}
method also does not require any new metalogical
notions. It avoids the Deduction Theorem's metalogic by prefixing the
hypotheses and conclusion of every would-be inference with a universal
antecedent (``$\varphi \rightarrow$'') from the very start.

We think it is impressive and satisfying that we can do so much in a
practical sense without stepping outside of our Hilbert-style axiom system.
Of course our axiomatization, which is in the form of schemes,
contains a metalogic of its own that we exploit. But this metalogic
is relatively simple, and for our Deduction Theorem alternatives,
we primarily use just the direct substitution of expressions for
metavariables.

\begin{sloppy}
\section{Exploring the Set The\-o\-ry Data\-base}\label{exploring}
\end{sloppy}
% NOTE: All examples performed in this section are
% recorded wtih "set width 61" % on set.mm as of 2019-05-28
% commit c1e7849557661260f77cfdf0f97ac4354fbb4f4d.

At this point you may wish to study the \texttt{set.mm}\index{set theory
database (\texttt{set.mm})} file in more detail.  Pay particular
attention to the assumptions needed to define wffs\index{well-formed
formula (wff)} (which are not included above), the variable types
(\texttt{\$f}\index{\texttt{\$f} statement} statements), and the
definitions that are introduced.  Start with some simple theorems in
propositional calculus, making sure you understand in detail each step
of a proof.  Once you get past the first few proofs and become familiar
with the Metamath language, any part of the \texttt{set.mm} database
will be as easy to follow, step by step, as any other part---you won't
have to undergo a ``quantum leap'' in mathematical sophistication to be
able to follow a deep proof in set theory.

Next, you may want to explore how concepts such as natural numbers are
defined and described.  This is probably best done in conjunction with
standard set theory textbooks, which can help give you a higher-level
understanding.  The \texttt{set.mm} database provides references that will get
you started.  From there, you will be on your way towards a very deep,
rigorous understanding of abstract mathematics.

The Metamath\index{Metamath} program can help you peruse a Metamath data\-base,
wheth\-er you are trying to figure out how a certain step follows in a proof or
just have a general curiosity.  We will go through some examples of the
commands, using the \texttt{set.mm}\index{set theory database (\texttt{set.mm})}
database provided with the Metamath software.  These should help get you
started.  See Chapter~\ref{commands} for a more detailed description of
the commands.  Note that we have included the full spelling of all commands to
prevent ambiguity with future commands.  In practice you may type just the
characters needed to specify each command keyword\index{command keyword}
unambiguously, often just one or two characters per keyword, and you don't
need to type them in upper case.

First run the Metamath program as described earlier.  You should see the
\verb/MM>/ prompt.  Read in the \texttt{set.mm} file:\index{\texttt{read}
command}

\begin{verbatim}
MM> read set.mm
Reading source file "set.mm"... 34554442 bytes
34554442 bytes were read into the source buffer.
The source has 155711 statements; 2254 are $a and 32250 are $p.
No errors were found.  However, proofs were not checked.
Type VERIFY PROOF * if you want to check them.
\end{verbatim}

As with most examples in this book, what you will see
will be slightly different because we are continuously
improving our databases (including \texttt{set.mm}).

Let's check the database integrity.  This may take a minute or two to run if
your computer is slow.

\begin{verbatim}
MM> verify proof *
0 10%  20%  30%  40%  50%  60%  70%  80%  90% 100%
..................................................
All proofs in the database were verified in 2.84 s.
\end{verbatim}

No errors were reported, so every proof is correct.

You need to know the names (labels) of theorems before you can look at them.
Often just examining the database file(s) with a text editor is the best
approach.  In \texttt{set.mm} there are many detailed comments, especially near
the beginning, that can help guide you. The \texttt{search} command in the
Metamath program is also handy.  The \texttt{comments} qualifier will list the
statements whose associated comment (the one immediately before it) contain a
string you give it.  For example, if you are studying Enderton's {\em Elements
of Set Theory} \cite{Enderton}\index{Enderton, Herbert B.} you may want to see
the references to it in the database.  The search string \texttt{enderton} is not
case sensitive.  (This will not show you all the database theorems that are in
Enderton's book because there is usually only one citation for a given
theorem, which may appear in several textbooks.)\index{\texttt{search}
command}

\begin{verbatim}
MM> search * "enderton" / comments
12067 unineq $p "... Exercise 20 of [Enderton] p. 32 and ..."
12459 undif2 $p "...Corollary 6K of [Enderton] p. 144. (C..."
12953 df-tp $a "...s. Definition of [Enderton] p. 19. (Co..."
13689 unissb $p ".... Exercise 5 of [Enderton] p. 26 and ..."
\end{verbatim}
\begin{center}
(etc.)
\end{center}

Or you may want to see what theorems have something to do with
conjunction (logical {\sc and}).  The quotes around the search
string are optional when there's no ambiguity.\index{\texttt{search}
command}

\begin{verbatim}
MM> search * conjunction / comments
120 a1d $p "...be replaced with a conjunction ( ~ df-an )..."
662 df-bi $a "...viated form after conjunction is introdu..."
1319 wa $a "...ff definition to include conjunction ('and')."
1321 df-an $a "Define conjunction (logical 'and'). Defini..."
1420 imnan $p "...tion in terms of conjunction. (Contribu..."
\end{verbatim}
\begin{center}
(etc.)
\end{center}


Now we will start to look at some details.  Let's look at the first
axiom of propositional calculus
(we could use \texttt{sh st} to abbreviate
\texttt{show statement}).\index{\texttt{show statement} command}

\begin{verbatim}
MM> show statement ax-1/full
Statement 19 is located on line 881 of the file "set.mm".
"Axiom _Simp_.  Axiom A1 of [Margaris] p. 49.  One of the 3
axioms of propositional calculus.  The 3 axioms are also
        ...
19 ax-1 $a |- ( ph -> ( ps -> ph ) ) $.
Its mandatory hypotheses in RPN order are:
  wph $f wff ph $.
  wps $f wff ps $.
The statement and its hypotheses require the variables:  ph
      ps
The variables it contains are:  ph ps


Statement 49 is located on line 11182 of the file "set.mm".
Its statement number for HTML pages is 6.
"Axiom _Simp_.  Axiom A1 of [Margaris] p. 49.  One of the 3
axioms of propositional calculus.  The 3 axioms are also
given as Definition 2.1 of [Hamilton] p. 28.
...
49 ax-1 $a |- ( ph -> ( ps -> ph ) ) $.
Its mandatory hypotheses in RPN order are:
  wph $f wff ph $.
  wps $f wff ps $.
The statement and its hypotheses require the variables:
  ph ps
The variables it contains are:  ph ps
\end{verbatim}

Compare this to \texttt{ax-1} on p.~\pageref{ax1}.  You can see that the
symbol \texttt{ph} is the {\sc ascii} notation for $\varphi$, etc.  To
see the mathematical symbols for any expression you may typeset it in
\LaTeX\ (type \texttt{help tex} for instructions)\index{latex@{\LaTeX}}
or, easier, just use a text editor to look at the comments where symbols
are first introduced in \texttt{set.mm}.  The hypotheses \texttt{wph}
and \texttt{wps} required by \texttt{ax-1} mean that variables
\texttt{ph} and \texttt{ps} must be wffs.

Next we'll pick a simple theorem of propositional calculus, the Principle of
Identity, which is proved directly from the axioms.  We'll look at the
statement then its proof.\index{\texttt{show statement}
command}

\begin{verbatim}
MM> show statement id1/full
Statement 116 is located on line 11371 of the file "set.mm".
Its statement number for HTML pages is 22.
"Principle of identity.  Theorem *2.08 of [WhiteheadRussell]
p. 101.  This version is proved directly from the axioms for
demonstration purposes.
...
116 id1 $p |- ( ph -> ph ) $= ... $.
Its mandatory hypotheses in RPN order are:
  wph $f wff ph $.
Its optional hypotheses are:  wps wch wth wta wet
      wze wsi wrh wmu wla wka
The statement and its hypotheses require the variables:  ph
These additional variables are allowed in its proof:
      ps ch th ta et ze si rh mu la ka
The variables it contains are:  ph
\end{verbatim}

The optional variables\index{optional variable} \texttt{ps}, \texttt{ch}, etc.\ are
available for use in a proof of this statement if we wish, and were we to do
so we would make use of optional hypotheses \texttt{wps}, \texttt{wch}, etc.  (See
Section~\ref{dollaref} for the meaning of ``optional
hypothesis.''\index{optional hypothesis}) The reason these show up in the
statement display is that statement \texttt{id1} happens to be in their scope
(see Section~\ref{scoping} for the definition of ``scope''\index{scope}), but
in fact in propositional calculus we will never make use of optional
hypotheses or variables.  This becomes important after quantifiers are
introduced, where ``dummy'' variables\index{dummy variable} are often needed
in the middle of a proof.

Let's look at the proof of statement \texttt{id1}.  We'll use the
\texttt{show proof} command, which by default suppresses the
``non-essential'' steps that construct the wffs.\index{\texttt{show proof}
command}
We will display the proof in ``lemmon' format (a non-indented format
with explicit previous step number references) and renumber the
displayed steps:

\begin{verbatim}
MM> show proof id1 /lemmon/renumber
1 ax-1           $a |- ( ph -> ( ph -> ph ) )
2 ax-1           $a |- ( ph -> ( ( ph -> ph ) -> ph ) )
3 ax-2           $a |- ( ( ph -> ( ( ph -> ph ) -> ph ) ) ->
                     ( ( ph -> ( ph -> ph ) ) -> ( ph -> ph )
                                                          ) )
4 2,3 ax-mp      $a |- ( ( ph -> ( ph -> ph ) ) -> ( ph -> ph
                                                          ) )
5 1,4 ax-mp      $a |- ( ph -> ph )
\end{verbatim}

If you have read Section~\ref{trialrun}, you'll know how to interpret this
proof.  Step~2, for example, is an application of axiom \texttt{ax-1}.  This
proof is identical to the one in Hamilton's {\em Logic for Mathematicians}
\cite[p.~32]{Hamilton}\index{Hamilton, Alan G.}.

You may want to look at what
substitutions\index{substitution!variable}\index{variable substitution} are
made into \texttt{ax-1} to arrive at step~2.  The command to do this needs to
know the ``real'' step number, so we'll display the proof again without
the \texttt{renumber} qualifier.\index{\texttt{show proof}
command}

\begin{verbatim}
MM> show proof id1 /lemmon
 9 ax-1          $a |- ( ph -> ( ph -> ph ) )
20 ax-1          $a |- ( ph -> ( ( ph -> ph ) -> ph ) )
24 ax-2          $a |- ( ( ph -> ( ( ph -> ph ) -> ph ) ) ->
                     ( ( ph -> ( ph -> ph ) ) -> ( ph -> ph )
                                                          ) )
25 20,24 ax-mp   $a |- ( ( ph -> ( ph -> ph ) ) -> ( ph -> ph
                                                          ) )
26 9,25 ax-mp    $a |- ( ph -> ph )
\end{verbatim}

The ``real'' step number is 20.  Let's look at its details.

\begin{verbatim}
MM> show proof id1 /detailed_step 20
Proof step 20:  min=ax-1 $a |- ( ph -> ( ( ph -> ph ) -> ph )
  )
This step assigns source "ax-1" ($a) to target "min" ($e).
The source assertion requires the hypotheses "wph" ($f, step
18) and "wps" ($f, step 19).  The parent assertion of the
target hypothesis is "ax-mp" ($a, step 25).
The source assertion before substitution was:
    ax-1 $a |- ( ph -> ( ps -> ph ) )
The following substitutions were made to the source
assertion:
    Variable  Substituted with
     ph        ph
     ps        ( ph -> ph )
The target hypothesis before substitution was:
    min $e |- ph
The following substitution was made to the target hypothesis:
    Variable  Substituted with
     ph        ( ph -> ( ( ph -> ph ) -> ph ) )
\end{verbatim}

This shows the substitutions\index{substitution!variable}\index{variable
substitution} made to the variables in \texttt{ax-1}.  References are made to
steps 18 and 19 which are not shown in our proof display.  To see these steps,
you can display the proof with the \texttt{all} qualifier.

Let's look at a slightly more advanced proof of propositional calculus.  Note
that \verb+/\+ is the symbol for $\wedge$ (logical {\sc and}, also
called conjunction).\index{conjunction ($\wedge$)}
\index{logical {\sc and} ($\wedge$)}

\begin{verbatim}
MM> show statement prth/full
Statement 1791 is located on line 15503 of the file "set.mm".
Its statement number for HTML pages is 559.
"Conjoin antecedents and consequents of two premises.  This
is the closed theorem form of ~ anim12d .  Theorem *3.47 of
[WhiteheadRussell] p. 113.  It was proved by Leibniz,
and it evidently pleased him enough to call it
_praeclarum theorema_ (splendid theorem).
...
1791 prth $p |- ( ( ( ph -> ps ) /\ ( ch -> th ) ) -> ( ( ph
      /\ ch ) -> ( ps /\ th ) ) ) $= ... $.
Its mandatory hypotheses in RPN order are:
  wph $f wff ph $.
  wps $f wff ps $.
  wch $f wff ch $.
  wth $f wff th $.
Its optional hypotheses are:  wta wet wze wsi wrh wmu wla wka
The statement and its hypotheses require the variables:  ph
      ps ch th
These additional variables are allowed in its proof:  ta et
      ze si rh mu la ka
The variables it contains are:  ph ps ch th


MM> show proof prth /lemmon/renumber
1 simpl          $p |- ( ( ( ph -> ps ) /\ ( ch -> th ) ) ->
                                               ( ph -> ps ) )
2 simpr          $p |- ( ( ( ph -> ps ) /\ ( ch -> th ) ) ->
                                               ( ch -> th ) )
3 1,2 anim12d    $p |- ( ( ( ph -> ps ) /\ ( ch -> th ) ) ->
                           ( ( ph /\ ch ) -> ( ps /\ th ) ) )
\end{verbatim}

There are references to a lot of unfamiliar statements.  To see what they are,
you may type the following:

\begin{verbatim}
MM> show proof prth /statement_summary
Summary of statements used in the proof of "prth":

Statement simpl is located on line 14748 of the file
"set.mm".
"Elimination of a conjunct.  Theorem *3.26 (Simp) of
[WhiteheadRussell] p. 112. ..."
  simpl $p |- ( ( ph /\ ps ) -> ph ) $= ... $.

Statement simpr is located on line 14777 of the file
"set.mm".
"Elimination of a conjunct.  Theorem *3.27 (Simp) of
[WhiteheadRussell] ..."
  simpr $p |- ( ( ph /\ ps ) -> ps ) $= ... $.

Statement anim12d is located on line 15445 of the file
"set.mm".
"Conjoin antecedents and consequents in a deduction.
..."
  anim12d.1 $e |- ( ph -> ( ps -> ch ) ) $.
  anim12d.2 $e |- ( ph -> ( th -> ta ) ) $.
  anim12d $p |- ( ph -> ( ( ps /\ th ) -> ( ch /\ ta ) ) )
      $= ... $.
\end{verbatim}
\begin{center}
(etc.)
\end{center}

Of course you can look at each of these statements and their proofs, and
so on, back to the axioms of propositional calculus if you wish.

The \texttt{search} command is useful for finding statements when you
know all or part of their contents.  The following example finds all
statements containing \verb@ph -> ps@ followed by \verb@ch -> th@.  The
\verb@$*@ is a wildcard that matches anything; the \texttt{\$} before the
\verb$*$ prevents conflicts with math symbol token names.  The \verb@*@ after
\texttt{SEARCH} is also a wildcard that in this case means ``match any label.''
\index{\texttt{search} command}

% I'm omitting this one, since readers are unlikely to see it:
% 1096 bisymOLD $p |- ( ( ( ph -> ps ) -> ( ch -> th ) ) -> ( (
%   ( ps -> ph ) -> ( th -> ch ) ) -> ( ( ph <-> ps ) -> ( ch
%    <-> th ) ) ) )
\begin{verbatim}
MM> search * "ph -> ps $* ch -> th"
1791 prth $p |- ( ( ( ph -> ps ) /\ ( ch -> th ) ) -> ( ( ph
    /\ ch ) -> ( ps /\ th ) ) )
2455 pm3.48 $p |- ( ( ( ph -> ps ) /\ ( ch -> th ) ) -> ( (
    ph \/ ch ) -> ( ps \/ th ) ) )
117859 pm11.71 $p |- ( ( E. x ph /\ E. y ch ) -> ( ( A. x (
    ph -> ps ) /\ A. y ( ch -> th ) ) <-> A. x A. y ( ( ph /\
    ch ) -> ( ps /\ th ) ) ) )
\end{verbatim}

Three statements, \texttt{prth}, \texttt{pm3.48},
 and \texttt{pm11.71}, were found to match.

To see what axioms\index{axiom} and definitions\index{definition}
\texttt{prth} ultimately depends on for its proof, you can have the
program backtrack through the hierarchy\index{hierarchy} of theorems and
definitions.\index{\texttt{show trace{\char`\_}back} command}

\begin{verbatim}
MM> show trace_back prth /essential/axioms
Statement "prth" assumes the following axioms ($a
statements):
  ax-1 ax-2 ax-3 ax-mp df-bi df-an
\end{verbatim}

Note that the 3 axioms of propositional calculus and the modus ponens rule are
needed (as expected); in addition, there are a couple of definitions that are used
along the way.  Note that Metamath makes no distinction\index{axiom vs.\
definition} between axioms\index{axiom} and definitions\index{definition}.  In
\texttt{set.mm} they have been distinguished artificially by prefixing their
labels\index{labels in \texttt{set.mm}} with \texttt{ax-} and \texttt{df-}
respectively.  For example, \texttt{df-an} defines conjunction (logical {\sc
and}), which is represented by the symbol \verb+/\+.
Section~\ref{definitions} discusses the philosophy of definitions, and the
Metamath language takes a particularly simple, conservative approach by using
the \texttt{\$a}\index{\texttt{\$a} statement} statement for both axioms and
definitions.

You can also have the program compute how many steps a proof
has\index{proof length} if we were to follow it all the way back to
\texttt{\$a} statements.

\begin{verbatim}
MM> show trace_back prth /essential/count_steps
The statement's actual proof has 3 steps.  Backtracking, a
total of 79 different subtheorems are used.  The statement
and subtheorems have a total of 274 actual steps.  If
subtheorems used only once were eliminated, there would be a
total of 38 subtheorems, and the statement and subtheorems
would have a total of 185 steps.  The proof would have 28349
steps if fully expanded back to axiom references.  The
maximum path length is 38.  A longest path is:  prth <-
anim12d <- syl2and <- sylan2d <- ancomsd <- ancom <- pm3.22
<- pm3.21 <- pm3.2 <- ex <- sylbir <- biimpri <- bicomi <-
bicom1 <- bi2 <- dfbi1 <- impbii <- bi3 <- simprim <- impi <-
con1i <- nsyl2 <- mt3d <- con1d <- notnot1 <- con2i <- nsyl3
<- mt2d <- con2d <- notnot2 <- pm2.18d <- pm2.18 <- pm2.21 <-
pm2.21d <- a1d <- syl <- mpd <- a2i <- a2i.1 .
\end{verbatim}

This tells us that we would have to inspect 274 steps if we want to
verify the proof completely starting from the axioms.  A few more
statistics are also shown.  There are one or more paths back to axioms
that are the longest; this command ferrets out one of them and shows it
to you.  There may be a sense in which the longest path length is
related to how ``deep'' the theorem is.

We might also be curious about what proofs depend on the theorem
\texttt{prth}.  If it is never used later on, we could eliminate it as
redundant if it has no intrinsic interest by itself.\index{\texttt{show
usage} command}

% I decided to show the OLD values here.
\begin{verbatim}
MM> show usage prth
Statement "prth" is directly referenced in the proofs of 18
statements:
  mo3 moOLD 2mo 2moOLD euind reuind reuss2 reusv3i opelopabt
  wemaplem2 rexanre rlimcn2 o1of2 o1rlimmul 2sqlem6 spanuni
  heicant pm11.71
\end{verbatim}

Thus \texttt{prth} is directly used by 18 proofs.
We can use the \texttt{/recursive} qualifier to include indirect use:

\begin{verbatim}
MM> show usage prth /recursive
Statement "prth" directly or indirectly affects the proofs of
24214 statements:
  mo3 mo mo3OLD eu2 moOLD eu2OLD eu3OLD mo4f mo4 eu4 mopick
...
\end{verbatim}

\subsection{A Note on the ``Compact'' Proof Format}

The Metamath program will display proofs in a ``compact''\index{compact proof}
format whenever the proof is stored in compressed format in the database.  It
may be be slightly confusing unless you know how to interpret it.
For example,
if you display the complete proof of theorem \texttt{id1} it will start
off as follows:

\begin{verbatim}
MM> show proof id1 /lemmon/all
 1 wph           $f wff ph
 2 wph           $f wff ph
 3 wph           $f wff ph
 4 2,3 wi    @4: $a wff ( ph -> ph )
 5 1,4 wi    @5: $a wff ( ph -> ( ph -> ph ) )
 6 @4            $a wff ( ph -> ph )
\end{verbatim}

\begin{center}
{etc.}
\end{center}

Step 4 has a ``local label,''\index{local label} \texttt{@4}, assigned to it.
Later on, at step 6, the label \texttt{@1} is referenced instead of
displaying the explicit proof for that step.  This technique takes advantage
of the fact that steps in a proof often repeat, especially during the
construction of wffs.  The compact format reduces the number of steps in the
proof display and may be preferred by some people.

If you want to see the normal format with the ``true'' step numbers, you can
use the following workaround:\index{\texttt{save proof} command}

\begin{verbatim}
MM> save proof id1 /normal
The proof of "id1" has been reformatted and saved internally.
Remember to use WRITE SOURCE to save it permanently.
MM> show proof id1 /lemmon/all
 1 wph           $f wff ph
 2 wph           $f wff ph
 3 wph           $f wff ph
 4 2,3 wi        $a wff ( ph -> ph )
 5 1,4 wi        $a wff ( ph -> ( ph -> ph ) )
 6 wph           $f wff ph
 7 wph           $f wff ph
 8 6,7 wi        $a wff ( ph -> ph )
\end{verbatim}

\begin{center}
{etc.}
\end{center}

Note that the original 6 steps are now 8 steps.  However, the format is
now the same as that described in Chapter~\ref{using}.

\chapter{The Metamath Language}
\label{languagespec}

\begin{quote}
  {\em Thus mathematics may be defined as the subject in which we never know
what we are talking about, nor whether what we are saying is true.}
    \flushright\sc  Bertrand Russell\footnote{\cite[p.~84]{Russell2}.}\\
\end{quote}\index{Russell, Bertrand}

Probably the most striking feature of the Metamath language is its almost
complete absence of hard-wired syntax. Metamath\index{Metamath} does not
understand any mathematics or logic other than that needed to construct finite
sequences of symbols according to a small set of simple, built-in rules.  The
only rule it uses in a proof is the substitution of an expression (symbol
sequence) for a variable, subject to a simple constraint to prevent
bound-variable clashes.  The primitive notions built into Metamath involve the
simple manipulation of finite objects (symbols) that we as humans can easily
visualize and that computers can easily deal with.  They seem to be just
about the simplest notions possible that are required to do standard
mathematics.

This chapter serves as a reference manual for the Metamath\index{Metamath}
language. It covers the tedious technical details of the language, some of
which you may wish to skim in a first reading.  On the other hand, you should
pay close attention to the defined terms in {\bf boldface}; they have precise
meanings that are important to keep in mind for later understanding.  It may
be best to first become familiar with the examples in Chapter~\ref{using} to
gain some motivation for the language.

%% Uncomment this when uncommenting section {formalspec} below
If you have some knowledge of set theory, you may wish to study this
chapter in conjunction with the formal set-theoretical description of the
Metamath language in Appendix~\ref{formalspec}.

We will use the name ``Metamath''\index{Metamath} to mean either the Metamath
computer language or the Metamath software associated with the computer
language.  We will not distinguish these two when the context is clear.

The next section contains the complete specification of the Metamath
language.
It serves as an
authoritative reference and presents the syntax in enough detail to
write a parser\index{parsing Metamath} and proof verifier.  The
specification is terse and it is probably hard to learn the language
directly from it, but we include it here for those impatient people who
prefer to see everything up front before looking at verbose expository
material.  Later sections explain this material and provide examples.
We will repeat the definitions in those sections, and you may skip the
next section at first reading and proceed to Section~\ref{tut1}
(p.~\pageref{tut1}).

\section{Specification of the Metamath Language}\label{spec}
\index{Metamath!specification}

\begin{quote}
  {\em Sometimes one has to say difficult things, but one ought to say
them as simply as one knows how.}
    \flushright\sc  G. H. Hardy\footnote{As quoted in
    \cite{deMillo}, p.~273.}\\
\end{quote}\index{Hardy, G. H.}

\subsection{Preliminaries}\label{spec1}

% Space is technically a printable character, so we'll word things
% carefully so it's unambiguous.
A Metamath {\bf database}\index{database} is built up from a top-level source
file together with any source files that are brought in through file inclusion
commands (see below).  The only characters that are allowed to appear in a
Metamath source file are the 94 non-whitespace printable {\sc
ascii}\index{ascii@{\sc ascii}} characters, which are digits, upper and lower
case letters, and the following 32 special
characters\index{special characters}:\label{spec1chars}

\begin{verbatim}
! " # $ % & ' ( ) * + , - . / :
; < = > ? @ [ \ ] ^ _ ` { | } ~
\end{verbatim}
plus the following characters which are the ``white space'' characters:
space (a printable character),
tab, carriage return, line feed, and form feed.\label{whitespace}
We will use \texttt{typewriter}
font to display the printable characters.

A Metamath database consists of a sequence of three kinds of {\bf
tokens}\index{token} separated by {\bf white space}\index{white space}
(which is any sequence of one or more white space characters).  The set
of {\bf keyword}\index{keyword} tokens is \texttt{\$\char`\{},
\texttt{\$\char`\}}, \texttt{\$c}, \texttt{\$v}, \texttt{\$f},
\texttt{\$e}, \texttt{\$d}, \texttt{\$a}, \texttt{\$p}, \texttt{\$.},
\texttt{\$=}, \texttt{\$(}, \texttt{\$)}, \texttt{\$[}, and
\texttt{\$]}.  The last four are called {\bf auxiliary}\index{auxiliary
keyword} or preprocessing keywords.  A {\bf label}\index{label} token
consists of any combination of letters, digits, and the characters
hyphen, underscore, and period.  A {\bf math symbol}\index{math symbol}
token may consist of any combination of the 93 printable standard {\sc
ascii} characters other than space or \texttt{\$}~. All tokens are
case-sensitive.

\subsection{Preprocessing}

The token \texttt{\$(} begins a {\bf comment} and
\texttt{\$)} ends a comment.\index{\texttt{\$(}
and \texttt{\$)} auxiliary keywords}\index{comment}
Comments may contain any of
the 94 non-whitespace printable characters and white space,
except they may not contain the
2-character sequences \texttt{\$(} or \texttt{\$)} (comments do not nest).
Comments are ignored (treated
like white space) for the purpose of parsing, e.g.,
\texttt{\$( \$[ \$)} is a comment.
See p.~\pageref{mathcomments} for comment typesetting conventions; these
conventions may be ignored for the purpose of parsing.

A {\bf file inclusion command} consists of \texttt{\$[} followed by a file name
followed by \texttt{\$]}.\index{\texttt{\$[} and \texttt{\$]} auxiliary
keywords}\index{included file}\index{file inclusion}
It is only allowed in the outermost scope (i.e., not between
\texttt{\$\char`\{} and \texttt{\$\char`\}})
and must not be inside a statement (e.g., it may not occur
between the label of a \texttt{\$a} statement and its \texttt{\$.}).
The file name may not
contain a \texttt{\$} or white space.  The file must exist.
The case-sensitivity
of its name follows the conventions of the operating system.  The contents of
the file replace the inclusion command.
Included files may include other files.
Only the first reference to a given file is included; any later
references to the same file (whether in the top-level file or in included
files) cause the inclusion command to be ignored (treated like white space).
A verifier may assume that file names with different strings
refer to different files for the purpose of ignoring later references.
A file self-reference is ignored, as is any reference to the top-level file
(to avoid loops).
Included files may not include a \texttt{\$(} without a matching \texttt{\$)},
may not include a \texttt{\$[} without a matching \texttt{\$]}, and may
not include incomplete statements (e.g., a \texttt{\$a} without a matching
\texttt{\$.}).
It is currently unspecified if path references are relative to the process'
current directory or the file's containing directory, so databases should
avoid using pathname separators (e.g., ``/'') in file names.

Like all tokens, the \texttt{\$(}, \texttt{\$)}, \texttt{\$[}, and \texttt{\$]} keywords
must be surrounded by white space.

\subsection{Basic Syntax}

After preprocessing, a database will consist of a sequence of {\bf
statements}.
These are the scoping statements \texttt{\$\char`\{} and
\texttt{\$\char`\}}, along with the \texttt{\$c}, \texttt{\$v},
\texttt{\$f}, \texttt{\$e}, \texttt{\$d}, \texttt{\$a}, and \texttt{\$p}
statements.

A {\bf scoping statement}\index{scoping statement} consists only of its
keyword, \texttt{\$\char`\{} or \texttt{\$\char`\}}.
A \texttt{\$\char`\{} begins a {\bf
block}\index{block} and a matching \texttt{\$\char`\}} ends the block.
Every \texttt{\$\char`\{}
must have a matching \texttt{\$\char`\}}.
Defining it recursively, we say a block
contains a sequence of zero or more tokens other
than \texttt{\$\char`\{} and \texttt{\$\char`\}} and
possibly other blocks.  There is an {\bf outermost
block}\index{block!outermost} not bracketed by \texttt{\$\char`\{} \texttt{\$\char`\}}; the end
of the outermost block is the end of the database.

% LaTeX bug? can't do \bf\tt

A {\bf \$v} or {\bf \$c statement}\index{\texttt{\$v} statement}\index{\texttt{\$c}
statement} consists of the keyword token \texttt{\$v} or \texttt{\$c} respectively,
followed by one or more math symbols,
% The word "token" is used to distinguish "$." from the sentence-ending period.
followed by the \texttt{\$.}\ token.
These
statements {\bf declare}\index{declaration} the math symbols to be {\bf
variables}\index{variable!Metamath} or {\bf constants}\index{constant}
respectively. The same math symbol may not occur twice in a given \texttt{\$v} or
\texttt{\$c} statement.

%c%A math symbol becomes an {\bf active}\index{active math symbol}
%c%when declared and stays active until the end of the block in which it is
%c%declared.  A math symbol may not be declared a second time while it is active,
%c%but it may be declared again after it becomes inactive.

A math symbol becomes {\bf active}\index{active math symbol} when declared
and stays active until the end of the block in which it is declared.  A
variable may not be declared a second time while it is active, but it
may be declared again (as a variable, but not as a constant) after it
becomes inactive.  A constant must be declared in the outermost block and may
not be declared a second time.\index{redeclaration of symbols}

A {\bf \$f statement}\index{\texttt{\$f} statement} consists of a label,
followed by \texttt{\$f}, followed by its typecode (an active constant),
followed by an
active variable, followed by the \texttt{\$.}\ token.  A {\bf \$e
statement}\index{\texttt{\$e} statement} consists of a label, followed
by \texttt{\$e}, followed by its typecode (an active constant),
followed by zero or more
active math symbols, followed by the \texttt{\$.}\ token.  A {\bf
hypothesis}\index{hypothesis} is a \texttt{\$f} or \texttt{\$e}
statement.
The type declared by a \texttt{\$f} statement for a given label
is global even if the variable is not
(e.g., a database may not have \texttt{wff P} in one local scope
and \texttt{class P} in another).

A {\bf simple \$d statement}\index{\texttt{\$d} statement!simple}
consists of \texttt{\$d}, followed by two different active variables,
followed by the \texttt{\$.}\ token.  A {\bf compound \$d
statement}\index{\texttt{\$d} statement!compound} consists of
\texttt{\$d}, followed by three or more variables (all different),
followed by the \texttt{\$.}\ token.  The order of the variables in a
\texttt{\$d} statement is unimportant.  A compound \texttt{\$d}
statement is equivalent to a set of simple \texttt{\$d} statements, one
for each possible pair of variables occurring in the compound
\texttt{\$d} statement.  Henceforth in this specification we shall
assume all \texttt{\$d} statements are simple.  A \texttt{\$d} statement
is also called a {\bf disjoint} (or {\bf distinct}) {\bf variable
restriction}.\index{disjoint-variable restriction}

A {\bf \$a statement}\index{\texttt{\$a} statement} consists of a label,
followed by \texttt{\$a}, followed by its typecode (an active constant),
followed by
zero or more active math symbols, followed by the \texttt{\$.}\ token.  A {\bf
\$p statement}\index{\texttt{\$p} statement} consists of a label,
followed by \texttt{\$p}, followed by its typecode (an active constant),
followed by
zero or more active math symbols, followed by \texttt{\$=}, followed by
a sequence of labels, followed by the \texttt{\$.}\ token.  An {\bf
assertion}\index{assertion} is a \texttt{\$a} or \texttt{\$p} statement.

A \texttt{\$f}, \texttt{\$e}, or \texttt{\$d} statement is {\bf active}\index{active
statement} from the place it occurs until the end of the block it occurs in.
A \texttt{\$a} or \texttt{\$p} statement is {\bf active} from the place it occurs
through the end of the database.
There may not be two active \texttt{\$f} statements containing the same
variable.  Each variable in a \texttt{\$e}, \texttt{\$a}, or
\texttt{\$p} statement must exist in an active \texttt{\$f}
statement.\footnote{This requirement can greatly simplify the
unification algorithm (substitution calculation) required by proof
verification.}

%The label that begins each \texttt{\$f}, \texttt{\$e}, \texttt{\$a}, and
%\texttt{\$p} statement must be unique.
Each label token must be unique, and
no label token may match any math symbol
token.\label{namespace}\footnote{This
restriction was added on June 24, 2006.
It is not theoretically necessary but is imposed to make it easier to
write certain parsers.}

The set of {\bf mandatory variables}\index{mandatory variable} associated with
an assertion is the set of (zero or more) variables in the assertion and in any
active \texttt{\$e} statements.  The (possibly empty) set of {\bf mandatory
hypotheses}\index{mandatory hypothesis} is the set of all active \texttt{\$f}
statements containing mandatory variables, together with all active \texttt{\$e}
statements.
The set of {\bf mandatory {\bf \$d} statements}\index{mandatory
disjoint-variable restriction} associated with an assertion are those active
\texttt{\$d} statements whose variables are both among the assertion's
mandatory variables.

\subsection{Proof Verification}\label{spec4}

The sequence of labels between the \texttt{\$=} and \texttt{\$.}\ tokens
in a \texttt{\$p} statement is a {\bf proof}.\index{proof!Metamath} Each
label in a proof must be the label of an active statement other than the
\texttt{\$p} statement itself; thus a label must refer either to an
active hypothesis of the \texttt{\$p} statement or to an earlier
assertion.

An {\bf expression}\index{expression} is a sequence of math symbols. A {\bf
substitution map}\index{substitution map} associates a set of variables with a
set of expressions.  It is acceptable for a variable to be mapped to an
expression containing it.  A {\bf
substitution}\index{substitution!variable}\index{variable substitution} is the
simultaneous replacement of all variables in one or more expressions with the
expressions that the variables map to.

A proof is scanned in order of its label sequence.  If the label refers to an
active hypothesis, the expression in the hypothesis is pushed onto a
stack.\index{stack}\index{RPN stack}  If the label refers to an assertion, a
(unique) substitution must exist that, when made to the mandatory hypotheses
of the referenced assertion, causes them to match the topmost (i.e.\ most
recent) entries of the stack, in order of occurrence of the mandatory
hypotheses, with the topmost stack entry matching the last mandatory
hypothesis of the referenced assertion.  As many stack entries as there are
mandatory hypotheses are then popped from the stack.  The same substitution is
made to the referenced assertion, and the result is pushed onto the stack.
After the last label in the proof is processed, the stack must have a single
entry that matches the expression in the \texttt{\$p} statement containing the
proof.

%c%{\footnotesize\begin{quotation}\index{redeclaration of symbols}
%c%{{\em Comment.}\label{spec4comment} Whenever a math symbol token occurs in a
%c%{\texttt{\$c} or \texttt{\$v} statement, it is considered to designate a distinct new
%c%{symbol, even if the same token was previously declared (and is now inactive).
%c%{Thus a math token declared as a constant in two different blocks is considered
%c%{to designate two distinct constants (even though they have the same name).
%c%{The two constants will not match in a proof that references both blocks.
%c%{However, a proof referencing both blocks is acceptable as long as it doesn't
%c%{require that the constants match.  Similarly, a token declared to be a
%c%{constant for a referenced assertion will not match the same token declared to
%c%{be a variable for the \texttt{\$p} statement containing the proof.  In the case
%c%{of a token declared to be a variable for a referenced assertion, this is not
%c%{an issue since the variable can be substituted with whatever expression is
%c%{needed to achieve the required match.
%c%{\end{quotation}}
%c2%A proof may reference an assertion that contains or whose hypotheses contain a
%c2%constant that is not active for the \texttt{\$p} statement containing the proof.
%c2%However, the final result of the proof may not contain that constant. A proof
%c2%may also reference an assertion that contains or whose hypotheses contain a
%c2%variable that is not active for the \texttt{\$p} statement containing the proof.
%c2%That variable, of course, will be substituted with whatever expression is
%c2%needed to achieve the required match.

A proof may contain a \texttt{?}\ in place of a label to indicate an unknown step
(Section~\ref{unknown}).  A proof verifier may ignore any proof containing
\texttt{?}\ but should warn the user that the proof is incomplete.

A {\bf compressed proof}\index{compressed proof}\index{proof!compressed} is an
alternate proof notation described in Appen\-dix~\ref{compressed}; also see
references to ``compressed proof'' in the Index.  Compressed proofs are a
Metamath language extension which a complete proof verifier should be able to
parse and verify.

\subsubsection{Verifying Disjoint Variable Restrictions}

Each substitution made in a proof must be checked to verify that any
disjoint variable restrictions are satisfied, as follows.

If two variables replaced by a substitution exist in a mandatory \texttt{\$d}
statement\index{\texttt{\$d} statement} of the assertion referenced, the two
expressions resulting from the substitution must satisfy the following
conditions.  First, the two expressions must have no variables in common.
Second, each possible pair of variables, one from each expression, must exist
in an active \texttt{\$d} statement of the \texttt{\$p} statement containing the
proof.

\vskip 1ex

This ends the specification of the Metamath language;
see Appendix \ref{BNF} for its syntax in
Extended Backus--Naur Form (EBNF)\index{Extended Backus--Naur Form}\index{EBNF}.

\section{The Basic Keywords}\label{tut1}

Our expository material begins here.

Like most computer languages, Metamath\index{Metamath} takes its input from
one or more {\bf source files}\index{source file} which contain characters
expressed in the standard {\sc ascii} (American Standard Code for Information
Interchange)\index{ascii@{\sc ascii}} code for computers.  A source file
consists of a series of {\bf tokens}\index{token}, which are strings of
non-whitespace
printable characters (from the set of 94 shown on p.~\pageref{spec1chars})
separated by {\bf white space}\index{white space} (spaces, tabs, carriage
returns, line feeds, and form feeds). Any string consisting only of these
characters is treated the same as a single space.  The non-whitespace printable
characters\index{printable character} that Metamath recognizes are the 94
characters on standard {\sc ascii} keyboards.

Metamath has the ability to join several files together to form its
input (Section~\ref{include}).  We call the aggregate contents of all
the files after they have been joined together a {\bf
database}\index{database} to distinguish it from an individual source
file.  The tokens in a database consist of {\bf
keywords}\index{keyword}, which are built into the language, together
with two kinds of user-defined tokens called {\bf labels}\index{label}
and {\bf math symbols}\index{math symbol}.  (Often we will simply say
{\bf symbol}\index{symbol} instead of math symbol for brevity).  The set
of {\bf basic keywords}\index{basic keyword} is
\texttt{\$c}\index{\texttt{\$c} statement},
\texttt{\$v}\index{\texttt{\$v} statement},
\texttt{\$e}\index{\texttt{\$e} statement},
\texttt{\$f}\index{\texttt{\$f} statement},
\texttt{\$d}\index{\texttt{\$d} statement},
\texttt{\$a}\index{\texttt{\$a} statement},
\texttt{\$p}\index{\texttt{\$p} statement},
\texttt{\$=}\index{\texttt{\$=} keyword},
\texttt{\$.}\index{\texttt{\$.}\ keyword},
\texttt{\$\char`\{}\index{\texttt{\$\char`\{} and \texttt{\$\char`\}}
keywords}, and \texttt{\$\char`\}}.  This is the complete set of
syntactical elements of what we call the {\bf basic
language}\index{basic language} of Metamath, and with them you can
express all of the mathematics that were intended by the design of
Metamath.  You should make it a point to become very familiar with them.
Table~\ref{basickeywords} lists the basic keywords along with a brief
description of their functions.  For now, this description will give you
only a vague notion of what the keywords are for; later we will describe
the keywords in detail.


\begin{table}[htp] \caption{Summary of the basic Metamath
keywords} \label{basickeywords}
\begin{center}
\begin{tabular}{|p{4pc}|l|}
\hline
\em \centering Keyword&\em Description\\
\hline
\hline
\centering
   \texttt{\$c}&Constant symbol declaration\\
\hline
\centering
   \texttt{\$v}&Variable symbol declaration\\
\hline
\centering
   \texttt{\$d}&Disjoint variable restriction\\
\hline
\centering
   \texttt{\$f}&Variable-type (``floating'') hypothesis\\
\hline
\centering
   \texttt{\$e}&Logical (``essential'') hypothesis\\
\hline
\centering
   \texttt{\$a}&Axiomatic assertion\\
\hline
\centering
   \texttt{\$p}&Provable assertion\\
\hline
\centering
   \texttt{\$=}&Start of proof in \texttt{\$p} statement\\
\hline
\centering
   \texttt{\$.}&End of the above statement types\\
\hline
\centering
   \texttt{\$\char`\{}&Start of block\\
\hline
\centering
   \texttt{\$\char`\}}&End of block\\
\hline
\end{tabular}
\end{center}
\end{table}

%For LaTeX bug(?) where it puts tables on blank page instead of btwn text
%May have to adjust if text changes
%\newpage

There are some additional keywords, called {\bf auxiliary
keywords}\index{auxiliary keyword} that help make Metamath\index{Metamath}
more practical. These are part of the {\bf extended language}\index{extended
language}. They provide you with a means to put comments into a Metamath
source file\index{source file} and reference other source files.  We will
introduce these in later sections. Table~\ref{otherkeywords} summarizes them
so that you can recognize them now if you want to peruse some source
files while learning the basic keywords.


\begin{table}[htp] \caption{Auxiliary Metamath
keywords} \label{otherkeywords}
\begin{center}
\begin{tabular}{|p{4pc}|l|}
\hline
\em \centering Keyword&\em Description\\
\hline
\hline
\centering
   \texttt{\$(}&Start of comment\\
\hline
\centering
   \texttt{\$)}&End of comment\\
\hline
\centering
   \texttt{\$[}&Start of included source file name\\
\hline
\centering
   \texttt{\$]}&End of included source file name\\
\hline
\end{tabular}
\end{center}
\end{table}
\index{\texttt{\$(} and \texttt{\$)} auxiliary keywords}
\index{\texttt{\$[} and \texttt{\$]} auxiliary keywords}


Unlike those in some computer languages, the keywords\index{keyword} are short
two-character sequences rather than English-like words.  While this may make
them slightly more difficult to remember at first, their brevity allows
them to blend in with the mathematics being described, not
distract from it, like punctuation marks.


\subsection{User-Defined Tokens}\label{dollardollar}\index{token}

As you may have noticed, all keywords\index{keyword} begin with the \texttt{\$}
character.  This mundane monetary symbol is not ordinarily used in higher
mathematics (outside of grant proposals), so we have appropriated it to
distinguish the Metamath\index{Metamath} keywords from ordinary mathematical
symbols. The \texttt{\$} character is thus considered special and may not be
used as a character in a user-defined token.  All tokens and keywords are
case-sensitive; for example, \texttt{n} is considered to be a different character
from \texttt{N}.  Case-sensitivity makes the available {\sc ascii} character set
as rich as possible.

\subsubsection{Math Symbol Tokens}\index{token}

Math symbols\index{math symbol} are tokens used to represent the symbols
that appear in ordinary mathematical formulas.  They may consist of any
combination of the 93 non-whitespace printable {\sc ascii} characters other than
\texttt{\$}~. Some examples are \texttt{x}, \texttt{+}, \texttt{(},
\texttt{|-}, \verb$!%@?&$, and \texttt{bounded}.  For readability, it is
best to try to make these look as similar to actual mathematical symbols
as possible, within the constraints of the {\sc ascii} character set, in
order to make the resulting mathematical expressions more readable.

In the Metamath\index{Metamath} language, you express ordinary
mathematical formulas and statements as sequences of math symbols such
as \texttt{2 + 2 = 4} (five symbols, all constants).\footnote{To
eliminate ambiguity with other expressions, this is expressed in the set
theory database \texttt{set.mm} as \texttt{|- ( 2 + 2
 ) = 4 }, whose \LaTeX\ equivalent is $\vdash
(2+2)=4$.  The \,$\vdash$ means ``is a theorem'' and the
parentheses allow explicit associative grouping.}\index{turnstile
({$\,\vdash$})} They may even be English
sentences, as in \texttt{E is closed and bounded} (five symbols)---here
\texttt{E} would be a variable and the other four symbols constants.  In
principle, a Metamath database could be constructed to work with almost
any unambiguous English-language mathematical statement, but as a
practical matter the definitions needed to provide for all possible
syntax variations would be cumbersome and distracting and possibly have
subtle pitfalls accidentally built in.  We generally recommend that you
express mathematical statements with compact standard mathematical
symbols whenever possible and put their English-language descriptions in
comments.  Axioms\index{axiom} and definitions\index{definition}
(\texttt{\$a}\index{\texttt{\$a} statement} statements) are the only
places where Metamath will not detect an error, and doing this will help
reduce the number of definitions needed.

You are free to use any tokens\index{token} you like for math
symbols\index{math symbol}.  Appendix~\ref{ASCII} recommends token names to
use for symbols in set theory, and we suggest you adopt these in order to be
able to include the \texttt{set.mm} set theory database in your database.  For
printouts, you can convert the tokens in a database
to standard mathematical symbols with the \LaTeX\ typesetting program.  The
Metamath command \texttt{open tex} {\em filename}\index{\texttt{open tex} command}
produces output that can be read by \LaTeX.\index{latex@{\LaTeX}}
The correspondence
between tokens and the actual symbols is made by \texttt{latexdef}
statements inside a special database comment tagged
with \texttt{\$t}.\index{\texttt{\$t} comment}\index{typesetting comment}
  You can edit
this comment to change the definitions or add new ones.
Appendix~\ref{ASCII} describes how to do this in more detail.

% White space\index{white space} is normally used to separate math
% symbol\index{math symbol} tokens, but they may be juxtaposed without white
% space in \texttt{\$d}\index{\texttt{\$d} statement}, \texttt{\$e}\index{\texttt{\$e}
% statement}, \texttt{\$f}\index{\texttt{\$f} statement}, \texttt{\$a}\index{\texttt{\$a}
% statement}, and \texttt{\$p}\index{\texttt{\$p} statement} statements when no
% ambiguity will result.  Specifically, Metamath parses the math symbol sequence
% in one of these statements in the following manner:  when the math symbol
% sequence has been broken up into tokens\index{token} up to a given character,
% the next token is the longest string of characters that could constitute a
% math symbol that is active\index{active
% math symbol} at that point.  (See Section~\ref{scoping} for the
% definition of an active math symbol.)  For example, if \texttt{-}, \texttt{>}, and
% \texttt{->} are the only active math symbols, the juxtaposition \texttt{>-} will be
% interpreted as the two symbols \texttt{>} and \texttt{-}, whereas \texttt{->} will
% always be interpreted as that single symbol.\footnote{For better readability we
% recommend a white space between each token.  This also makes searching for a
% symbol easier to do with an editor.  Omission of optional white space is useful
% for reducing typing when assigning an expression to a temporary
% variable\index{temporary variable} with the \texttt{let variable} Metamath
% program command.}\index{\texttt{let variable} command}
%
% Keywords\index{keyword} may be placed next to math symbols without white
% space\index{white space} between them.\footnote{Again, we do not recommend
% this for readability.}
%
% The math symbols\index{math symbol} in \texttt{\$c}\index{\texttt{\$c} statement}
% and \texttt{\$v}\index{\texttt{\$v} statement} statements must always be separated
% by white space\index{white
% space}, for the obvious reason that these statements define the names
% of the symbols.
%
% Math symbols referred to in comments (see Section~\ref{comments}) must also be
% separated by white space.  This allows you to make comments about symbols that
% are not yet active\index{active
% math symbol}.  (The ``math mode'' feature of comments is also a quick and
% easy way to obtain word processing text with embedded mathematical symbols,
% independently of the main purpose of Metamath; the way to do this is described
% in Section~\ref{comments})

\subsubsection{Label Tokens}\index{token}\index{label}

Label tokens are used to identify Metamath\index{Metamath} statements for
later reference. Label tokens may contain only letters, digits, and the three
characters period, hyphen, and underscore:
\begin{verbatim}
. - _
\end{verbatim}

A label is {\bf declared}\index{label declaration} by placing it immediately
before the keyword of the statement it identifies.  For example, the label
\texttt{axiom.1} might be declared as follows:
\begin{verbatim}
axiom.1 $a |- x = x $.
\end{verbatim}

Each \texttt{\$e}\index{\texttt{\$e} statement},
\texttt{\$f}\index{\texttt{\$f} statement},
\texttt{\$a}\index{\texttt{\$a} statement}, and
\texttt{\$p}\index{\texttt{\$p} statement} statement in a database must
have a label declared for it.  No other statement types may have label
declarations.  Every label must be unique.

A label (and the statement it identifies) is {\bf referenced}\index{label
reference} by including the label between the \texttt{\$=}\index{\texttt{\$=}
keyword} and \texttt{\$.}\index{\texttt{\$.}\ keyword}\ keywords in a \texttt{\$p}
statement.  The sequence of labels\index{label sequence} between these two
keywords is called a {\bf proof}\index{proof}.  An example of a statement with
a proof that we will encounter later (Section~\ref{proof}) is
\begin{verbatim}
wnew $p wff ( s -> ( r -> p ) )
     $= ws wr wp w2 w2 $.
\end{verbatim}

You don't have to know what this means just yet, but you should know that the
label \texttt{wnew} is declared by this \texttt{\$p} statement and that the labels
\texttt{ws}, \texttt{wr}, \texttt{wp}, and \texttt{w2} are assumed to have been declared
earlier in the database and are referenced here.

\subsection{Constants and Variables}
\index{constant}
\index{variable}

An {\bf expression}\index{expression} is any sequence of math
symbols, possibly empty.

The basic Metamath\index{Metamath} language\index{basic language} has two
kinds of math symbols\index{math symbol}:  {\bf constants}\index{constant} and
{\bf variables}\index{variable}.  In a Metamath proof, a constant may not be
substituted with any expression.  A variable can be
substituted\index{substitution!variable}\index{variable substitution} with any
expression.  This sequence may include other variables and may even include
the variable being substituted.  This substitution takes place when proofs are
verified, and it will be described in Section~\ref{proof}.  The \texttt{\$f}
statement (described later in Section~\ref{dollaref}) is used to specify the
{\bf type} of a variable (i.e.\ what kind of
variable it is)\index{variable type}\index{type} and
give it a meaning typically
associated with a ``metavariable''\index{metavariable}\footnote{A metavariable
is a variable that ranges over the syntactical elements of the object language
being discussed; for example, one metavariable might represent a variable of
the object language and another metavariable might represent a formula in the
object language.} in ordinary mathematics; for example, a variable may be
specified to be a wff or well-formed formula (in logic), a set (in set
theory), or a non-negative integer (in number theory).

%\subsection{The \texttt{\$c} and \texttt{\$v} Declaration Statements}
\subsection{The \texttt{\$c} and \texttt{\$v} Declaration Statements}
\index{\texttt{\$c} statement}
\index{constant declaration}
\index{\texttt{\$v} statement}
\index{variable declaration}

Constants are introduced or {\bf declared}\index{constant declaration}
with \texttt{\$c}\index{\texttt{\$c} statement} statements, and
variables are declared\index{variable declaration} with
\texttt{\$v}\index{\texttt{\$v} statement} statements.  A {\bf simple}
declaration\index{simple declaration} statement introduces a single
constant or variable.  Its syntax is one of the following:
\begin{center}
  \texttt{\$c} {\em math-symbol} \texttt{\$.}\\
  \texttt{\$v} {\em math-symbol} \texttt{\$.}
\end{center}
The notation {\em math-symbol} means any math symbol token\index{token}.

Some examples of simple declaration statements are:
\begin{center}
  \texttt{\$c + \$.}\\
  \texttt{\$c -> \$.}\\
  \texttt{\$c ( \$.}\\
  \texttt{\$v x \$.}\\
  \texttt{\$v y2 \$.}
\end{center}

The characters in a math symbol\index{math symbol} being declared are
irrelevant to Meta\-math; for example, we could declare a right parenthesis to
be a variable,
\begin{center}
  \texttt{\$v ) \$.}\\
\end{center}
although this would be unconventional.

A {\bf compound} declaration\index{compound declaration} statement is a
shorthand for declaring several symbols at once.  Its syntax is one of the
following:
\begin{center}
  \texttt{\$c} {\em math-symbol}\ \,$\cdots$\ {\em math-symbol} \texttt{\$.}\\
  \texttt{\$v} {\em math-symbol}\ \,$\cdots$\ {\em math-symbol} \texttt{\$.}
\end{center}\index{\texttt{\$c} statement}
Here, the ellipsis (\ldots) means any number of {\em math-symbol}\,s.

An example of a compound declaration statement is:
\begin{center}
  \texttt{\$v x y mu \$.}\\
\end{center}
This is equivalent to the three simple declaration statements
\begin{center}
  \texttt{\$v x \$.}\\
  \texttt{\$v y \$.}\\
  \texttt{\$v mu \$.}\\
\end{center}
\index{\texttt{\$v} statement}

There are certain rules on where in the database math symbols may be declared,
what sections of the database are aware of them (i.e.\ where they are
``active''), and when they may be declared more than once.  These will be
discussed in Section~\ref{scoping} and specifically on
p.~\pageref{redeclaration}.

\subsection{The \texttt{\$d} Statement}\label{dollard}
\index{\texttt{\$d} statement}

The \texttt{\$d} statement is called a {\bf disjoint-variable restriction}.  The
syntax of the {\bf simple} version of this statement is
\begin{center}
  \texttt{\$d} {\em variable variable} \texttt{\$.}
\end{center}
where each {\em variable} is a previously declared variable and the two {\em
variable}\,s are different.  (More specifically, each  {\em variable} must be
an {\bf active} variable\index{active math symbol}, which means there must be
a previous \texttt{\$v} statement whose {\bf scope}\index{scope} includes the
\texttt{\$d} statement.  These terms will be defined when we discuss scoping
statements in Section~\ref{scoping}.)

In ordinary mathematics, formulas may arise that are true if the variables in
them are distinct\index{distinct variables}, but become false when those
variables are made identical. For example, the formula in logic $\exists x\,x
\neq y$, which means ``for a given $y$, there exists an $x$ that is not equal
to $y$,'' is true in most mathematical theories (namely all non-trivial
theories\index{non-trivial theory}, i.e.\ those that describe more than one
individual, such as arithmetic).  However, if we substitute $y$ with $x$, we
obtain $\exists x\,x \neq x$, which is always false, as it means ``there
exists something that is not equal to itself.''\footnote{If you are a
logician, you will recognize this as the improper substitution\index{proper
substitution}\index{substitution!proper} of a free variable\index{free
variable} with a bound variable\index{bound variable}.  Metamath makes no
inherent distinction between free and bound variables; instead, you let
Metamath know what substitutions are permissible by using \texttt{\$d} statements
in the right way in your axiom system.}\index{free vs.\ bound variable}  The
\texttt{\$d} statement allows you to specify a restriction that forbids the
substitution of one variable with another.  In
this case, we would use the statement
\begin{center}
  \texttt{\$d x y \$.}
\end{center}\index{\texttt{\$d} statement}
to specify this restriction.

The order in which the variables appear in a \texttt{\$d} statement is not
important.  We could also use
\begin{center}
  \texttt{\$d y x \$.}
\end{center}

The \texttt{\$d} statement is actually more general than this, as the
``disjoint''\index{disjoint variables} in its name suggests.  The full meaning
is that if any substitution is made to its two variables (during the
course of a proof that references a \texttt{\$a} or \texttt{\$p} statement
associated with the \texttt{\$d}), the two expressions that result from the
substitution must have no variables in common.  In addition, each possible
pair of variables, one from each expression, must be in a \texttt{\$d} statement
associated with the statement being proved.  (This requirement forces the
statement being proved to ``inherit'' the original disjoint variable
restriction.)

For example, suppose \texttt{u} is a variable.  If the restriction
\begin{center}
  \texttt{\$d A B \$.}
\end{center}
has been specified for a theorem referenced in a
proof, we may not substitute \texttt{A} with \mbox{\tt a + u} and
\texttt{B} with \mbox{\tt b + u} because these two symbol sequences have the
variable \texttt{u} in common.  Furthermore, if \texttt{a} and \texttt{b} are
variables, we may not substitute \texttt{A} with \texttt{a} and \texttt{B} with \texttt{b}
unless we have also specified \texttt{\$d a b} for the theorem being proved; in
other words, the \texttt{\$d} property associated with a pair of variables must
be effectively preserved after substitution.

The \texttt{\$d}\index{\texttt{\$d} statement} statement does {\em not} mean ``the
two variables may not be substituted with the same thing,'' as you might think
at first.  For example, substituting each of \texttt{A} and \texttt{B} in the above
example with identical symbol sequences consisting only of constants does not
cause a disjoint variable conflict, because two symbol sequences have no
variables in common (since they have no variables, period).  Similarly, a
conflict will not occur by substituting the two variables in a \texttt{\$d}
statement with the empty symbol sequence\index{empty substitution}.

The \texttt{\$d} statement does not have a direct counterpart in
ordinary mathematics, partly because the variables\index{variable} of
Metamath are not really the same as the variables\index{variable!in
ordinary mathematics} of ordinary mathematics but rather are
metavariables\index{metavariable} ranging over them (as well as over
other kinds of symbols and groups of symbols).  Depending on the
situation, we may informally interpret the \texttt{\$d} statement in
different ways.  Suppose, for example, that \texttt{x} and \texttt{y}
are variables ranging over numbers (more precisely, that \texttt{x} and
\texttt{y} are metavariables ranging over variables that range over
numbers), and that \texttt{ph} ($\varphi$) and \texttt{ps} ($\psi$) are
variables (more precisely, metavariables) ranging over formulas.  We can
make the following interpretations that correspond to the informal
language of ordinary mathematics:
\begin{quote}
\begin{tabbing}
\texttt{\$d x y \$.} means ``assume $x$ and $y$ are
distinct variables.''\\
\texttt{\$d x ph \$.} means ``assume $x$ does not
occur in $\varphi$.''\\
\texttt{\$d ph ps \$.} \=means ``assume $\varphi$ and
$\psi$ have no variables\\ \>in common.''
\end{tabbing}
\end{quote}\index{\texttt{\$d} statement}

\subsubsection{Compound \texttt{\$d} Statements}

The {\bf compound} version of the \texttt{\$d} statement is a shorthand for
specifying several variables whose substitutions must be pairwise disjoint.
Its syntax is:
\begin{center}
  \texttt{\$d} {\em variable}\ \,$\cdots$\ {\em variable} \texttt{\$.}
\end{center}\index{\texttt{\$d} statement}
Here, {\em variable} represents the token of a previously declared
variable (specifically, an active variable) and all {\em variable}\,s are
different.  The compound \texttt{\$d}
statement is internally broken up by Metamath into one simple \texttt{\$d}
statement for each possible pair of variables in the original \texttt{\$d}
statement.  For example,
\begin{center}
  \texttt{\$d w x y z \$.}
\end{center}
is equivalent to
\begin{center}
  \texttt{\$d w x \$.}\\
  \texttt{\$d w y \$.}\\
  \texttt{\$d w z \$.}\\
  \texttt{\$d x y \$.}\\
  \texttt{\$d x z \$.}\\
  \texttt{\$d y z \$.}
\end{center}

Two or more simple \texttt{\$d} statements specifying the same variable pair are
internally combined into a single \texttt{\$d} statement.  Thus the set of three
statements
\begin{center}
  \texttt{\$d x y \$.}
  \texttt{\$d x y \$.}
  \texttt{\$d y x \$.}
\end{center}
is equivalent to
\begin{center}
  \texttt{\$d x y \$.}
\end{center}

Similarly, compound \texttt{\$d} statements, after being internally broken up,
internally have their common variable pairs combined.  For example the
set of statements
\begin{center}
  \texttt{\$d x y A \$.}
  \texttt{\$d x y B \$.}
\end{center}
is equivalent to
\begin{center}
  \texttt{\$d x y \$.}
  \texttt{\$d x A \$.}
  \texttt{\$d y A \$.}
  \texttt{\$d x y \$.}
  \texttt{\$d x B \$.}
  \texttt{\$d y B \$.}
\end{center}
which is equivalent to
\begin{center}
  \texttt{\$d x y \$.}
  \texttt{\$d x A \$.}
  \texttt{\$d y A \$.}
  \texttt{\$d x B \$.}
  \texttt{\$d y B \$.}
\end{center}

Metamath\index{Metamath} automatically verifies that all \texttt{\$d}
restrictions are met whenever it verifies proofs.  \texttt{\$d} statements are
never referenced directly in proofs (this is why they do not have
labels\index{label}), but Metamath is always aware of which ones must be
satisfied (i.e.\ are active) and will notify you with an error message if any
violation occurs.

To illustrate how Metamath detects a missing \texttt{\$d}
statement, we will look at the following example from the
\texttt{set.mm} database.

\begin{verbatim}
$d x z $.  $d y z $.
$( Theorem to add distinct quantifier to atomic formula. $)
ax17eq $p |- ( x = y -> A. z x = y ) $=...
\end{verbatim}

This statement has the obvious requirement that $z$ must be
distinct\index{distinct variables} from $x$ in theorem \texttt{ax17eq} that
states $x=y \rightarrow \forall z \, x=y$ (well, obvious if you're a logician,
for otherwise we could conclude  $x=y \rightarrow \forall x \, x=y$, which is
false when the free variables $x$ and $y$ are equal).

Let's look at what happens if we edit the database to comment out this
requirement.

\begin{verbatim}
$( $d x z $. $) $d y z $.
$( Theorem to add distinct quantifier to atomic formula. $)
ax17eq $p |- ( x = y -> A. z x = y ) $=...
\end{verbatim}

When it tries to verify the proof, Metamath will tell you that \texttt{x} and
\texttt{z} must be disjoint, because one of its steps references an axiom or
theorem that has this requirement.

\begin{verbatim}
MM> verify proof ax17eq
ax17eq ?Error at statement 1918, label "ax17eq", type "$p":
      vz wal wi vx vy vz ax-13 vx vy weq vz vx ax-c16 vx vy
                                               ^^^^^
There is a disjoint variable ($d) violation at proof step 29.
Assertion "ax-c16" requires that variables "x" and "y" be
disjoint.  But "x" was substituted with "z" and "y" was
substituted with "x".  The assertion being proved, "ax17eq",
does not require that variables "z" and "x" be disjoint.
\end{verbatim}

We can see the substitutions into \texttt{ax-c16} with the following command.

\begin{verbatim}
MM> show proof ax17eq / detailed_step 29
Proof step 29:  pm2.61dd.2=ax-c16 $a |- ( A. z z = x -> ( x =
  y -> A. z x = y ) )
This step assigns source "ax-c16" ($a) to target "pm2.61dd.2"
($e).  The source assertion requires the hypotheses "wph"
($f, step 26), "vx" ($f, step 27), and "vy" ($f, step 28).
The parent assertion of the target hypothesis is "pm2.61dd"
($p, step 36).
The source assertion before substitution was:
    ax-c16 $a |- ( A. x x = y -> ( ph -> A. x ph ) )
The following substitutions were made to the source
assertion:
    Variable  Substituted with
     x         z
     y         x
     ph        x = y
The target hypothesis before substitution was:
    pm2.61dd.2 $e |- ( ph -> ch )
The following substitutions were made to the target
hypothesis:
    Variable  Substituted with
     ph        A. z z = x
     ch        ( x = y -> A. z x = y )
\end{verbatim}

The disjoint variable restrictions of \texttt{ax-c16} can be seen from the
\texttt{show state\-ment} command.  The line that begins ``\texttt{Its mandatory
dis\-joint var\-i\-able pairs are:}\ldots'' lists any \texttt{\$d} variable
pairs in brackets.

\begin{verbatim}
MM> show statement ax-c16/full
Statement 3033 is located on line 9338 of the file "set.mm".
"Axiom of Distinct Variables. ..."
  ax-c16 $a |- ( A. x x = y -> ( ph -> A. x ph ) ) $.
Its mandatory hypotheses in RPN order are:
  wph $f wff ph $.
  vx $f setvar x $.
  vy $f setvar y $.
Its mandatory disjoint variable pairs are:  <x,y>
The statement and its hypotheses require the variables:  x y
      ph
The variables it contains are:  x y ph
\end{verbatim}

Since Metamath will always detect when \texttt{\$d}\index{\texttt{\$d} statement}
statements are needed for a proof, you don't have to worry too much about
forgetting to put one in; it can always be added if you see the error message
above.  If you put in unnecessary \texttt{\$d} statements, the worst that could
happen is that your theorem might not be as general as it could be, and this
may limit its use later on.

On the other hand, when you introduce axioms (\texttt{\$a}\index{\texttt{\$a}
statement} statements), you must be very careful to properly specify the
necessary associated \texttt{\$d} statements since Metamath has no way of knowing
whether your axioms are correct.  For example, Metamath would have no idea
that \texttt{ax-c16}, which we are telling it is an axiom of logic, would lead to
contradictions if we omitted its associated \texttt{\$d} statement.

% This was previously a comment in footnote-sized type, but it can be
% hard to read this much text in a small size.
% As a result, it's been changed to normally-sized text.
\label{nodd}
You may wonder if it is possible to develop standard
mathematics in the Metamath language without the \texttt{\$d}\index{\texttt{\$d}
statement} statement, since it seems like a nuisance that complicates proof
verification. The \texttt{\$d} statement is not needed in certain subsets of
mathematics such as propositional calculus.  However, dummy
variables\index{dummy variable!eliminating} and their associated \texttt{\$d}
statements are impossible to avoid in proofs in standard first-order logic as
well as in the variant used in \texttt{set.mm}.  In fact, there is no upper bound to
the number of dummy variables that might be needed in a proof of a theorem of
first-order logic containing 3 or more variables, as shown by H.\
Andr\'{e}ka\index{Andr{\'{e}}ka, H.} \cite{Nemeti}.  A first-order system that
avoids them entirely is given in \cite{Megill}\index{Megill, Norman}; the
trick there is simply to embed harmlessly the necessary dummy variables into a
theorem being proved so that they aren't ``dummy'' anymore, then interpret the
resulting longer theorem so as to ignore the embedded dummy variables.  If
this interests you, the system in \texttt{set.mm} obtained from \texttt{ax-1}
through \texttt{ax-c14} in \texttt{set.mm}, and deleting \texttt{ax-c16} and \texttt{ax-5},
requires no \texttt{\$d} statements but is logically complete in the sense
described in \cite{Megill}.  This means it can prove any theorem of
first-order logic as long as we add to the theorem an antecedent that embeds
dummy and any other variables that must be distinct.  In a similar fashion,
axioms for set theory can be devised that
do not require distinct variable
provisos\index{Set theory without distinct variable provisos},
as explained at
\url{http://us.metamath.org/mpeuni/mmzfcnd.html}.
Together, these in principle allow all of
mathematics to be developed under Metamath without a \texttt{\$d} statement,
although the length of the resulting theorems will grow as more and
more dummy variables become required in their proofs.

\subsection{The \texttt{\$f}
and \texttt{\$e} Statements}\label{dollaref}
\index{\texttt{\$e} statement}
\index{\texttt{\$f} statement}
\index{floating hypothesis}
\index{essential hypothesis}
\index{variable-type hypothesis}
\index{logical hypothesis}
\index{hypothesis}

Metamath has two kinds of hypo\-theses, the \texttt{\$f}\index{\texttt{\$f}
statement} or {\bf variable-type} hypothesis and the \texttt{\$e} or {\bf logical}
hypo\-the\-sis.\index{\texttt{\$d} statement}\footnote{Strictly speaking, the
\texttt{\$d} statement is also a hypothesis, but it is never directly referenced
in a proof, so we call it a restriction rather than a hypothesis to lessen
confusion.  The checking for violations of \texttt{\$d} restrictions is automatic
and built into Metamath's proof-checking algorithm.} The letters \texttt{f} and
\texttt{e} stand for ``floating''\index{floating hypothesis} (roughly meaning
used only if relevant) and ``essential''\index{essential hypothesis} (meaning
always used) respectively, for reasons that will become apparent
when we discuss frames in
Section~\ref{frames} and scoping in Section~\ref{scoping}. The syntax of these
are as follows:
\begin{center}
  {\em label} \texttt{\$f} {\em typecode} {\em variable} \texttt{\$.}\\
  {\em label} \texttt{\$e} {\em typecode}
      {\em math-symbol}\ \,$\cdots$\ {\em math-symbol} \texttt{\$.}\\
\end{center}
\index{\texttt{\$e} statement}
\index{\texttt{\$f} statement}
A hypothesis must have a {\em label}\index{label}.  The expression in a
\texttt{\$e} hypothesis consists of a typecode (an active constant math symbol)
followed by a sequence
of zero or more math symbols. Each math symbol (including {\em constant}
and {\em variable}) must be a previously declared constant or variable.  (In
addition, each math symbol must be active, which will be covered when we
discuss scoping statements in Section~\ref{scoping}.)  You use a \texttt{\$f}
hypothesis to specify the
nature or {\bf type}\index{variable type}\index{type} of a variable (such as ``let $x$ be an
integer'') and use a \texttt{\$e} hypothesis to express a logical truth (such as
``assume $x$ is prime'') that must be established in order for an assertion
requiring it to also be true.

A variable must have its type specified in a \texttt{\$f} statement before
it may be used in a \texttt{\$e}, \texttt{\$a}, or \texttt{\$p}
statement.  There may be only one (active) \texttt{\$f} statement for a
given variable.  (``Active'' is defined in Section~\ref{scoping}.)

In ordinary mathematics, theorems\index{theorem} are often expressed in the
form ``Assume $P$; then $Q$,'' where $Q$ is a statement that you can derive
if you start with statement $P$.\index{free variable}\footnote{A stronger
version of a theorem like this would be the {\em single} formula $P\rightarrow
Q$ ($P$ implies $Q$) from which the weaker version above follows by the rule
of modus ponens in logic.  We are not discussing this stronger form here.  In
the weaker form, we are saying only that if we can {\em prove} $P$, then we can
{\em prove} $Q$.  In a logician's language, if $x$ is the only free variable
in $P$ and $Q$, the stronger form is equivalent to $\forall x ( P \rightarrow
Q)$ (for all $x$, $P$ implies $Q$), whereas the weaker form is equivalent to
$\forall x P \rightarrow \forall x Q$. The stronger form implies the weaker,
but not vice-versa.  To be precise, the weaker form of the theorem is more
properly called an ``inference'' rather than a theorem.}\index{inference}
In the
Metamath\index{Metamath} language, you would express mathematical statement
$P$ as a hypothesis (a \texttt{\$e} Metamath language statement in this case) and
statement $Q$ as a provable assertion (a \texttt{\$p}\index{\texttt{\$p} statement}
statement).

Some examples of hypotheses you might encounter in logic and set theory are
\begin{center}
  \texttt{stmt1 \$f wff P \$.}\\
  \texttt{stmt2 \$f setvar x \$.}\\
  \texttt{stmt3 \$e |- ( P -> Q ) \$.}
\end{center}
\index{\texttt{\$e} statement}
\index{\texttt{\$f} statement}
Informally, these would be read, ``Let $P$ be a well-formed-formula,'' ``Let
$x$ be an (individual) variable,'' and ``Assume we have proved $P \rightarrow
Q$.''  The turnstile symbol \,$\vdash$\index{turnstile ({$\,\vdash$})} is
commonly used in logic texts to mean ``a proof exists for.''

To summarize:
\begin{itemize}
\item A \texttt{\$f} hypothesis tells Metamath the type or kind of its variable.
It is analogous to a variable declaration in a computer language that
tells the compiler that a variable is an integer or a floating-point
number.
\item The \texttt{\$e} hypothesis corresponds to what you would usually call a
``hypothesis'' in ordinary mathematics.
\end{itemize}

Before an assertion\index{assertion} (\texttt{\$a} or \texttt{\$p} statement) can be
referenced in a proof, all of its associated \texttt{\$f} and \texttt{\$e} hypotheses
(i.e.\ those \texttt{\$e} hypotheses that are active) must be satisfied (i.e.
established by the proof).  The meaning of ``associated'' (which we will call
{\bf mandatory} in Section~\ref{frames}) will become clear when we discuss
scoping later.

Note that after any \texttt{\$f}, \texttt{\$e},
\texttt{\$a}, or \texttt{\$p} token there is a required
\textit{typecode}\index{typecode}.
The typecode is a constant used to enforce types of expressions.
This will become clearer once we learn more about
assertions (\texttt{\$a} and \texttt{\$p} statements).
An example may also clarify their purpose.
In the
\texttt{set.mm}\index{set theory database (\texttt{set.mm})}%
\index{Metamath Proof Explorer}
database,
the following typecodes are used:

\begin{itemize}
\item \texttt{wff} :
  Well-formed formula (wff) symbol
  (read: ``the following symbol sequence is a wff'').
% The *textual* typecode for turnstile is "|-", but when read it's a little
% confusing, so I intentionally display the mathematical symbol here instead
% (I think it's clearer in this context).
\item \texttt{$\vdash$} :
  Turnstile (read: ``the following symbol sequence is provable'' or
  ``a proof exists for'').
\item \texttt{setvar} :
  Individual set variable type (read: ``the following is an
  individual set variable'').
  Note that this is \textit{not} the type of an arbitrary set expression,
  instead, it is used to ensure that there is only a single symbol used
  after quantifiers like for-all ($\forall$) and there-exists ($\exists$).
\item \texttt{class} :
  An expression that is a syntactically valid class expression.
  All valid set expressions are also valid class expression, so expressions
  of sets normally have the \texttt{class} typecode.
  Use the \texttt{class} typecode,
  \textit{not} the \texttt{setvar} typecode,
  for the type of set expressions unless you are specifically identifying
  a single set variable.
\end{itemize}

\subsection{Assertions (\texttt{\$a} and \texttt{\$p} Statements)}
\index{\texttt{\$a} statement}
\index{\texttt{\$p} statement}\index{assertion}\index{axiomatic assertion}
\index{provable assertion}

There are two types of assertions, \texttt{\$a}\index{\texttt{\$a} statement}
statements ({\bf axiomatic assertions}) and \texttt{\$p} statements ({\bf
provable assertions}).  Their syntax is as follows:
\begin{center}
  {\em label} \texttt{\$a} {\em typecode} {\em math-symbol} \ldots
         {\em math-symbol} \texttt{\$.}\\
  {\em label} \texttt{\$p} {\em typecode} {\em math-symbol} \ldots
        {\em math-symbol} \texttt{\$=} {\em proof} \texttt{\$.}
\end{center}
\index{\texttt{\$a} statement}
\index{\texttt{\$p} statement}
\index{\texttt{\$=} keyword}
An assertion always requires a {\em label}\index{label}. The expression in an
assertion consists of a typecode (an active constant)
followed by a sequence of zero
or more math symbols.  Each math symbol, including any {\em constant}, must be a
previously declared constant or variable.  (In addition, each math symbol
must be active, which will be covered when we discuss scoping statements in
Section~\ref{scoping}.)

A \texttt{\$a} statement is usually a definition of syntax (for example, if $P$
and $Q$ are wffs then so is $(P\to Q)$), an axiom\index{axiom} of ordinary
mathematics (for example, $x=x$), or a definition\index{definition} of
ordinary mathematics (for example, $x\ne y$ means $\lnot x=y$). A \texttt{\$p}
statement is a claim that a certain combination of math symbols follows from
previous assertions and is accompanied by a proof that demonstrates it.

Assertions can also be referenced in (later) proofs in order to derive new
assertions from them. The label of an assertion is used to refer to it in a
proof. Section~\ref{proof} will describe the proof in detail.

Assertions also provide the primary means for communicating the mathematical
results in the database to people.  Proofs (when conveniently displayed)
communicate to people how the results were arrived at.

\subsubsection{The \texttt{\$a} Statement}
\index{\texttt{\$a} statement}

Axiomatic assertions (\texttt{\$a} statements) represent the starting points from
which other assertions (\texttt{\$p}\index{\texttt{\$p} statement} statements) are
derived.  Their most obvious use is for specifying ordinary mathematical
axioms\index{axiom}, but they are also used for two other purposes.

First, Metamath\index{Metamath} needs to know the syntax of symbol
sequences that constitute valid mathematical statements.  A Metamath
proof must be broken down into much more detail than ordinary
mathematical proofs that you may be used to thinking of (even the
``complete'' proofs of formal logic\index{formal logic}).  This is one
of the things that makes Metamath a general-purpose language,
independent of any system of logic or even syntax.  If you want to use a
substitution instance of an assertion as a step in a proof, you must
first prove that the substitution is syntactically correct (or if you
prefer, you must ``construct'' it), showing for example that the
expression you are substituting for a wff metavariable is a valid wff.
The \texttt{\$a}\index{\texttt{\$a} statement} statement is used to
specify those combinations of symbols that are considered syntactically
valid, such as the legal forms of wffs.

Second, \texttt{\$a} statements are used to specify what are ordinarily thought of
as definitions, i.e.\ new combinations of symbols that abbreviate other
combinations of symbols.  Metamath makes no distinction\index{axiom vs.\
definition} between axioms\index{axiom} and definitions\index{definition}.
Indeed, it has been argued that such distinction should not be made even in
ordinary mathematics; see Section~\ref{definitions}, which discusses the
philosophy of definitions.  Section~\ref{hierarchy} discusses some
technical requirements for definitions.  In \texttt{set.mm} we adopt the
convention of prefixing axiom labels with \texttt{ax-} and definition labels with
\texttt{df-}\index{label}.

The results that can be derived with the Metamath language are only as good as
the \texttt{\$a}\index{\texttt{\$a} statement} statements used as their starting
point.  We cannot stress this too strongly.  For example, Metamath will
not prevent you from specifying $x\neq x$ as an axiom of logic.  It is
essential that you scrutinize all \texttt{\$a} statements with great care.
Because they are a source of potential pitfalls, it is best not to add new
ones (usually new definitions) casually; rather you should carefully evaluate
each one's necessity and advantages.

Once you have in place all of the basic axioms\index{axiom} and
rules\index{rule} of a mathematical theory, the only \texttt{\$a} statements that
you will be adding will be what are ordinarily called definitions.  In
principle, definitions should be in some sense eliminable from the language of
a theory according to some convention (usually involving logical equivalence
or equality).  The most common convention is that any formula that was
syntactically valid but not provable before the definition was introduced will
not become provable after the definition is introduced.  In an ideal world,
definitions should not be present at all if one is to have absolute confidence
in a mathematical result.  However, they are necessary to make
mathematics practical, for otherwise the resulting formulas would be
extremely long and incomprehensible.  Since the nature of definitions (in the
most general sense) does not permit them to automatically be verified as
``proper,''\index{proper definition}\index{definition!proper} the judgment of
the mathematician is required to ensure it.  (In \texttt{set.mm} effort was made
to make almost all definitions directly eliminable and thus minimize the need
for such judgment.)

If you are not a mathematician, it may be best not to add or change any
\texttt{\$a}\index{\texttt{\$a} statement} statements but instead use
the mathematical language already provided in standard databases.  This
way Metamath will not allow you to make a mistake (i.e.\ prove a false
result).


\subsection{Frames}\label{frames}

We now introduce the concept of a collection of related Metamath statements
called a frame.  Every assertion (\texttt{\$a} or \texttt{\$p} statement) in the database has
an associated frame.

A {\bf frame}\index{frame} is a sequence of \texttt{\$d}, \texttt{\$f},
and \texttt{\$e} statements (zero or more of each) followed by one
\texttt{\$a} or \texttt{\$p} statement, subject to certain conditions we
will describe.  For simplicity we will assume that all math symbol
tokens used are declared at the beginning of the database with
\texttt{\$c} and \texttt{\$v} statements (which are not properly part of
a frame).  Also for simplicity we will assume there are only simple
\texttt{\$d} statements (those with only two variables) and imagine any
compound \texttt{\$d} statements (those with more than two variables) as
broken up into simple ones.

A frame groups together those hypotheses (and \texttt{\$d} statements) relevant
to an assertion (\texttt{\$a} or \texttt{\$p} statement).  The statements in a frame
may or may not be physically adjacent in a database; we will cover
this in our discussion of scoping statements
in Section~\ref{scoping}.

A frame has the following properties:
\begin{enumerate}
 \item The set of variables contained in its \texttt{\$f} statements must
be identical to the set of variables contained in its \texttt{\$e},
\texttt{\$a}, and/or \texttt{\$p} statements.  In other words, each
variable in a \texttt{\$e}, \texttt{\$a}, or \texttt{\$p} statement must
have an associated ``variable type'' defined for it in a \texttt{\$f}
statement.
  \item No two \texttt{\$f} statements may contain the same variable.
  \item Any \texttt{\$f} statement
must occur before a \texttt{\$e} statement in which its variable occurs.
\end{enumerate}

The first property determines the set of variables occurring in a frame.
These are the {\bf mandatory
variables}\index{mandatory variable} of the frame.  The second property
tells us there must be only one type specified for a variable.
The last property is not a theoretical requirement but it
makes parsing of the database easier.

For our examples, we assume our database has the following declarations:

\begin{verbatim}
$v P Q R $.
$c -> ( ) |- wff $.
\end{verbatim}

The following sequence of statements, describing the modus ponens inference
rule, is an example of a frame:

\begin{verbatim}
wp  $f wff P $.
wq  $f wff Q $.
maj $e |- ( P -> Q ) $.
min $e |- P $.
mp  $a |- Q $.
\end{verbatim}

The following sequence of statements is not a frame because \texttt{R} does not
occur in the \texttt{\$e}'s or the \texttt{\$a}:

\begin{verbatim}
wp  $f wff P $.
wq  $f wff Q $.
wr  $f wff R $.
maj $e |- ( P -> Q ) $.
min $e |- P $.
mp  $a |- Q $.
\end{verbatim}

The following sequence of statements is not a frame because \texttt{Q} does not
occur in a \texttt{\$f}:

\begin{verbatim}
wp  $f wff P $.
maj $e |- ( P -> Q ) $.
min $e |- P $.
mp  $a |- Q $.
\end{verbatim}

The following sequence of statements is not a frame because the \texttt{\$a} statement is
not the last one:

\begin{verbatim}
wp  $f wff P $.
wq  $f wff Q $.
maj $e |- ( P -> Q ) $.
mp  $a |- Q $.
min $e |- P $.
\end{verbatim}

Associated with a frame is a sequence of {\bf mandatory
hypotheses}\index{mandatory hypothesis}.  This is simply the set of all
\texttt{\$f} and \texttt{\$e} statements in the frame, in the order they
appear.  A frame can be referenced in a later proof using the label of
the \texttt{\$a} or \texttt{\$p} assertion statement, and the proof
makes an assignment to each mandatory hypothesis in the order in which
it appears.  This means the order of the hypotheses, once chosen, must
not be changed so as not to affect later proofs referencing the frame's
assertion statement.  (The Metamath proof verifier will, of course, flag
an error if a proof becomes incorrect by doing this.)  Since proofs make
use of ``Reverse Polish notation,'' described in Section~\ref{proof}, we
call this order the {\bf RPN order}\index{RPN order} of the hypotheses.

Note that \texttt{\$d} statements are not part of the set of mandatory
hypotheses, and their order doesn't matter (as long as they satisfy the
fourth property for a frame described above).  The \texttt{\$d}
statements specify restrictions on variables that must be satisfied (and
are checked by the proof verifier) when expressions are substituted for
them in a proof, and the \texttt{\$d} statements themselves are never
referenced directly in a proof.

A frame with a \texttt{\$p} (provable) statement requires a proof as part of the
\texttt{\$p} statement.  Sometimes in a proof we want to make use of temporary or
dummy variables\index{dummy variable} that do not occur in the \texttt{\$p}
statement or its mandatory hypotheses.  To accommodate this we define an {\bf
extended frame}\index{extended frame} as a frame together with zero or more
\texttt{\$d} and \texttt{\$f} statements that reference variables not among the
mandatory variables of the frame.  Any new variables referenced are called the
{\bf optional variables}\index{optional variable} of the extended frame. If a
\texttt{\$f} statement references an optional variable it is called an {\bf
optional hypothesis}\index{optional hypothesis}, and if one or both of the
variables in a \texttt{\$d} statement are optional variables it is called an {\bf
optional disjoint-variable restriction}\index{optional disjoint-variable
restriction}.  Properties 2 and 3 for a frame also apply to an extended
frame.

The concept of optional variables is not meaningful for frames with \texttt{\$a}
statements, since those statements have no proofs that might make use of them.
There is no restriction on including optional hypotheses in the extended frame
for a \texttt{\$a} statement, but they serve no purpose.

The following set of statements is an example of an extended frame, which
contains an optional variable \texttt{R} and an optional hypothesis \texttt{wr}.  In
this example, we suppose the rule of modus ponens is not an axiom but is
derived as a theorem from earlier statements (we omit its presumed proof).
Variable \texttt{R} may be used in its proof if desired (although this would
probably have no advantage in propositional calculus).  Note that the sequence
of mandatory hypotheses in RPN order is still \texttt{wp}, \texttt{wq}, \texttt{maj},
\texttt{min} (i.e.\ \texttt{wr} is omitted), and this sequence is still assumed
whenever the assertion \texttt{mp} is referenced in a subsequent proof.

\begin{verbatim}
wp  $f wff P $.
wq  $f wff Q $.
wr  $f wff R $.
maj $e |- ( P -> Q ) $.
min $e |- P $.
mp  $p |- Q $= ... $.
\end{verbatim}

Every frame is an extended frame, but not every extended frame is a frame, as
this example shows.  The underlying frame for an extended frame is
obtained by simply removing all statements containing optional variables.
Any proof referencing an assertion will ignore any extensions to its
frame, which means we may add or delete optional hypotheses at will without
affecting subsequent proofs.

The conceptually simplest way of organizing a Metamath database is as a
sequence of extended frames.  The scoping statements
\texttt{\$\char`\{}\index{\texttt{\$\char`\{} and \texttt{\$\char`\}}
keywords} and \texttt{\$\char`\}} can be used to delimit the start and
end of an extended frame, leading to the following possible structure for a
database.  \label{framelist}

\vskip 2ex
\setbox\startprefix=\hbox{\tt \ \ \ \ \ \ \ \ }
\setbox\contprefix=\hbox{}
\startm
\m{\mbox{(\texttt{\$v} {\em and} \texttt{\$c}\,{\em statements})}}
\endm
\startm
\m{\mbox{\texttt{\$\char`\{}}}
\endm
\startm
\m{\mbox{\texttt{\ \ } {\em extended frame}}}
\endm
\startm
\m{\mbox{\texttt{\$\char`\}}}}
\endm
\startm
\m{\mbox{\texttt{\$\char`\{}}}
\endm
\startm
\m{\mbox{\texttt{\ \ } {\em extended frame}}}
\endm
\startm
\m{\mbox{\texttt{\$\char`\}}}}
\endm
\startm
\m{\mbox{\texttt{\ \ \ \ \ \ \ \ \ }}\vdots}
\endm
\vskip 2ex

In practice, this structure is inconvenient because we have to repeat
any \texttt{\$f}, \texttt{\$e}, and \texttt{\$d} statements over and
over again rather than stating them once for use by several assertions.
The scoping statements, which we will discuss next, allow this to be
done.  In principle, any Metamath database can be converted to the above
format, and the above format is the most convenient to use when studying
a Metamath database as a formal system%
%% Uncomment this when uncommenting section {formalspec} below
   (Appendix \ref{formalspec})%
.
In fact, Metamath internally converts the database to the above format.
The command \texttt{show statement} in the Metamath program will show
you the contents of the frame for any \texttt{\$a} or \texttt{\$p}
statement, as well as its extension in the case of a \texttt{\$p}
statement.

%c%(provided that all ``local'' variables and constants with limited scope have
%c%unique names),

During our discussion of scoping statements, it may be helpful to
think in terms of the equivalent sequence of frames that will result when
the database is parsed.  Scoping (other than the limited
use above to delimit frames) is not a theoretical requirement for
Metamath but makes it more convenient.


\subsection{Scoping Statements (\texttt{\$\{} and \texttt{\$\}})}\label{scoping}
\index{\texttt{\$\char`\{} and \texttt{\$\char`\}} keywords}\index{scoping statement}

%c%Some Metamath statements may be needed only temporarily to
%c%serve a specific purpose, and after we're done with them we would like to
%c%disregard or ignore them.  For example, when we're finished using a variable,
%c%we might want to
%c%we might want to free up the token\index{token} used to name it so that the
%c%token can be used for other purposes later on, such as a different kind of
%c%variable or even a constant.  In the terminology of computer programming, we
%c%might want to let some symbol declarations be ``local'' rather than ``global.''
%c%\index{local symbol}\index{global symbol}

The {\bf scoping} statements, \texttt{\$\char`\{} ({\bf start of block}) and \texttt{\$\char`\}}
({\bf end of block})\index{block}, provide a means for controlling the portion
of a database over which certain statement types are recognized.  The
syntax of a scoping statement is very simple; it just consists of the
statement's keyword:
\begin{center}
\texttt{\$\char`\{}\\
\texttt{\$\char`\}}
\end{center}
\index{\texttt{\$\char`\{} and \texttt{\$\char`\}} keywords}

For example, consider the following database where we have stripped out
all tokens except the scoping statement keywords.  For the purpose of the
discussion, we have added subscripts to the scoping statements; these subscripts
do not appear in the actual database.
\[
 \mbox{\tt \ \$\char`\{}_1
 \mbox{\tt \ \$\char`\{}_2
 \mbox{\tt \ \$\char`\}}_2
 \mbox{\tt \ \$\char`\{}_3
 \mbox{\tt \ \$\char`\{}_4
 \mbox{\tt \ \$\char`\}}_4
 \mbox{\tt \ \$\char`\}}_3
 \mbox{\tt \ \$\char`\}}_1
\]
Each \texttt{\$\char`\{} statement in this example is said to be {\bf
matched} with the \texttt{\$\char`\}} statement that has the same
subscript.  Each pair of matched scoping statements defines a region of
the database called a {\bf block}.\index{block} Blocks can be {\bf
nested}\index{nested block} inside other blocks; in the example, the
block defined by $\mbox{\tt \$\char`\{}_4$ and $\mbox{\tt \$\char`\}}_4$
is nested inside the block defined by $\mbox{\tt \$\char`\{}_3$ and
$\mbox{\tt \$\char`\}}_3$ as well as inside the block defined by
$\mbox{\tt \$\char`\{}_1$ and $\mbox{\tt \$\char`\}}_1$.  In general, a
block may be empty, it may contain only non-scoping
statements,\footnote{Those statements other than \texttt{\$\char`\{} and
\texttt{\$\char`\}}.}\index{non-scoping statement} or it may contain any
mixture of other blocks and non-scoping statements.  (This is called a
``recursive'' definition\index{recursive definition} of a block.)

Associated with each block is a number called its {\bf nesting
level}\index{nesting level} that indicates how deeply the block is nested.
The nesting levels of the blocks in our example are as follows:
\[
  \underbrace{
    \mbox{\tt \ }
    \underbrace{
     \mbox{\tt \$\char`\{\ }
     \underbrace{
       \mbox{\tt \$\char`\{\ }
       \mbox{\tt \$\char`\}}
     }_{2}
     \mbox{\tt \ }
     \underbrace{
       \mbox{\tt \$\char`\{\ }
       \underbrace{
         \mbox{\tt \$\char`\{\ }
         \mbox{\tt \$\char`\}}
       }_{3}
       \mbox{\tt \ \$\char`\}}
     }_{2}
     \mbox{\tt \ \$\char`\}}
   }_{1}
   \mbox{\tt \ }
 }_{0}
\]
\index{\texttt{\$\char`\{} and \texttt{\$\char`\}} keywords}
The entire database is considered to be one big block (the {\bf outermost}
block) with a nesting level of 0.  The outermost block is {\em not} bracketed
by scoping statements.\footnote{The language was designed this way so that
several source files can be joined together more easily.}\index{outermost
block}

All non-scoping Metamath statements become recognized or {\bf
active}\index{active statement} at the place where they appear.\footnote{To
keep things slightly simpler, we do not bother to define the concept of
``active'' for the scoping statements.}  Certain of these statement types
become inactive at the end of the block in which they appear; these statement
types are:
\begin{center}
  \texttt{\$c}, \texttt{\$v}, \texttt{\$d}, \texttt{\$e}, and \texttt{\$f}.
%  \texttt{\$v}, \texttt{\$f}, \texttt{\$e}, and \texttt{\$d}.
\end{center}
\index{\texttt{\$c} statement}
\index{\texttt{\$d} statement}
\index{\texttt{\$e} statement}
\index{\texttt{\$f} statement}
\index{\texttt{\$v} statement}
The other statement types remain active forever (i.e.\ through the end of the
database); they are:
\begin{center}
  \texttt{\$a} and \texttt{\$p}.
%  \texttt{\$c}, \texttt{\$a}, and \texttt{\$p}.
\end{center}
\index{\texttt{\$a} statement}
\index{\texttt{\$p} statement}
Any statement (of these 7 types) located in the outermost
block\index{outermost block} will remain active through the end of the
database and thus are effectively ``global'' statements.\index{global
statement}

All \texttt{\$c} statements must be placed in the outermost block.  Since they are
therefore always global, they could be considered as belonging to both of the
above categories.

The {\bf scope}\index{scope} of a statement is the set of statements that
recognize it as active.

%c%The concept of ``active'' is also defined for math symbols\index{math
%c%symbol}.  Math symbols (constants\index{constant} and
%c%variables\index{variable}) become {\bf active}\index{active
%c%math symbol} in the \texttt{\$c}\index{\texttt{\$c}
%c%statement} and \texttt{\$v}\index{\texttt{\$v} statement} statements that
%c%declare them.  They become inactive when their declaration statements become
%c%inactive.

The concept of ``active'' is also defined for math symbols\index{math
symbol}.  Math symbols (constants\index{constant} and
variables\index{variable}) become {\bf active}\index{active math symbol}
in the \texttt{\$c}\index{\texttt{\$c} statement} and
\texttt{\$v}\index{\texttt{\$v} statement} statements that declare them.
A variable becomes inactive when its declaration statement becomes
inactive.  Because all \texttt{\$c} statements must be in the outermost
block, a constant will never become inactive after it is declared.

\subsubsection{Redeclaration of Math Symbols}
\index{redeclaration of symbols}\label{redeclaration}

%c%A math symbol may not be declared a second time while it is active, but it may
%c%be declared again after it becomes inactive.

A variable may not be declared a second time while it is active, but it may be
declared again after it becomes inactive.  This provides a convenient way to
introduce ``local'' variables,\index{local variable} i.e.\ temporary variables
for use in the frame of an assertion or in a proof without keeping them around
forever.  A previously declared variable may not be redeclared as a constant.

A constant may not be redeclared.  And, as mentioned above, constants must be
declared in the outermost block.

The reason variables may have limited scope but not constants is that an
assertion (\texttt{\$a} or \texttt{\$p} statement) remains available for use in
proofs through the end of the database.  Variables in an assertion's frame may
be substituted with whatever is needed in a proof step that references the
assertion, whereas constants remain fixed and may not be substituted with
anything.  The particular token used for a variable in an assertion's frame is
irrelevant when the assertion is referenced in a proof, and it doesn't matter
if that token is not available outside of the referenced assertion's frame.
Constants, however, must be globally fixed.

There is no theoretical
benefit for the feature allowing variables to be active for limited scopes
rather than global. It is just a convenience that allows them, for example, to
be locally grouped together with their corresponding \texttt{\$f} variable-type
declarations.

%c%If you declare a math symbol more than once, internally Metamath considers it a
%c%new distinct symbol, even though it has the same name.  If you are unaware of
%c%this, you may find that what you think are correct proofs are incorrectly
%c%rejected as invalid, because Metamath may tell you that a constant you
%c%previously declared does not match a newly declared math symbol with the same
%c%name.  For details on this subtle point, see the Comment on
%c%p.~\pageref{spec4comment}.  This is done purposely to allow temporary
%c%constants to be introduced while developing a subtheory, then allow their math
%c%symbol tokens to be reused later on; in general they will not refer to the
%c%same thing.  In practice, you would not ordinarily reuse the names of
%c%constants because it would tend to be confusing to the reader.  The reuse of
%c%names of variables, on the other hand, is something that is often useful to do
%c%(for example it is done frequently in \texttt{set.mm}).  Since variables in an
%c%assertion referenced in a proof can be substituted as needed to achieve a
%c%symbol match, this is not an issue.

% (This section covers a somewhat advanced topic you may want to skip
% at first reading.)
%
% Under certain circumstances, math symbol\index{math symbol}
% tokens\index{token} may be redeclared (i.e.\ the token
% may appear in more than
% one \texttt{\$c}\index{\texttt{\$c} statement} or \texttt{\$v}\index{\texttt{\$v}
% statement} statement).  You might want to do this say, to make temporary use
% of a variable name without having to worry about its affect elsewhere,
% somewhat analogous to declaring a local variable in a standard computer
% language.  Understanding what goes on when math symbol tokens are redeclared
% is a little tricky to understand at first, since it requires that we
% distinguish the token itself from the math symbol that it names.  It will help
% if we first take a peek at the internal workings of the
% Metamath\index{Metamath} program.
%
% Metamath reserves a memory location for each occurrence of a
% token\index{token} in a declaration statement (\texttt{\$c}\index{\texttt{\$c}
% statement} or \texttt{\$v}\index{\texttt{\$v} statement}).  If a given token appears
% in more than one declaration statement, it will refer to more than one memory
% locations.  A math symbol\index{math symbol} may be thought of as being one of
% these memory locations rather than as the token itself.  Only one of the
% memory locations associated with a given token may be active at any one time.
% The math symbol (memory location) that gets looked up when the token appears
% in a non-declaration statement is the one that happens to be active at that
% time.
%
% We now look at the rules for the redeclaration\index{redeclaration of symbols}
% of math symbol tokens.
% \begin{itemize}
% \item A math symbol token may not be declared twice in the
% same block.\footnote{While there is no theoretical reason for disallowing
% this, it was decided in the design of Metamath that allowing it would offer no
% advantage and might cause confusion.}
% \item An inactive math symbol may always be
% redeclared.
% \item  An active math symbol may be redeclared in a different (i.e.\
% inner) block\index{block} from the one it became active in.
% \end{itemize}
%
% When a math symbol token is redeclared, it conceptually refers to a different
% math symbol, just as it would be if it were called a different name.  In
% addition, the original math symbol that it referred to, if it was active,
% temporarily becomes inactive.  At the end of the block in which the
% redeclaration occurred, the new math symbol\index{math symbol} becomes
% inactive and the original symbol becomes active again.  This concept is
% illustrated in the following example, where the symbol \texttt{e} is
% ordinarily a constant (say Euler's constant, 2.71828...) but
% temporarily we want to use it as a ``local'' variable, say as a coefficient
% in the equation $a x^4 + b x^3 + c x^2 + d x + e$:
% \[
%   \mbox{\tt \$\char`\{\ \$c e \$.}
%   \underbrace{
%     \ \ldots\ %
%     \mbox{\tt \$\char`\{}\ \ldots\ %
%   }_{\mbox{\rm region A}}
%   \mbox{\tt \$v e \$.}
%   \underbrace{
%     \mbox{\ \ \ \ldots\ \ \ }
%   }_{\mbox{\rm region B}}
%   \mbox{\tt \$\char`\}}
%   \underbrace{
%     \mbox{\ \ \ \ldots\ \ \ }
%   }_{\mbox{\rm region C}}
%   \mbox{\tt \$\char`\}}
% \]
% \index{\texttt{\$\char`\{} and \texttt{\$\char`\}} keywords}
% In region A, the token \texttt{e} refers to a constant.  It is redeclared as a
% variable in region B, and any reference to it in this region will refer to this
% variable.  In region C, the redeclaration becomes inactive, and the original
% declaration becomes active again.  In region C, the token \texttt{x} refers to the
% original constant.
%
% As a practical matter, overuse of math symbol\index{math symbol}
% redeclarations\index{redeclaration of symbols} can be confusing (even though
% it is well-defined) and is best avoided when possible.  Here are some good
% general guidelines you can follow.  Usually, you should declare all
% constants\index{constant} in the outermost block\index{outermost block},
% especially if they are general-purpose (such as the token \verb$A.$, meaning
% $\forall$ or ``for all'').  This will make them ``globally'' active (although
% as in the example above local redeclarations will temporarily make them
% inactive.)  Most or all variables\index{variable}, on the other hand, could be
% declared in inner blocks, so that the token for them can be used later for a
% different type of variable or a constant.  (The names of the variables you
% choose are not used when you refer to an assertion\index{assertion} in a
% proof, whereas constants must match exactly.  A locally declared constant will
% not match a globally declared constant in a proof, even if they use the same
% token, because Metamath internally considers them to be different math
% symbols.)  To avoid confusion, you should generally avoid redeclaring active
% variables.  If you must redeclare them, do so at the beginning of a block.
% The temporary declaration of constants in inner blocks might be occasionally
% appropriate when you make use of a temporary definition to prove lemmas
% leading to a main result that does not make direct use of the definition.
% This way, you will not clutter up your database with a large number of
% seldom-used global constant symbols.  You might want to note that while
% inactive constants may not appear directly in an assertion (a \texttt{\$a}\index{\texttt{\$a}
% statement} or \texttt{\$p}\index{\texttt{\$p} statement}
% statement), they may be indirectly used in the proof of a \texttt{\$p} statement
% so long as they do not appear in the final math symbol sequence constructed by
% the proof.  In the end, you will have to use your best judgment, taking into
% account standard mathematical usage of the symbols as well as consideration
% for the reader of your work.
%
% \subsubsection{Reuse of Labels}\index{reuse of labels}\index{label}
%
% The \texttt{\$e}\index{\texttt{\$e} statement}, \texttt{\$f}\index{\texttt{\$f}
% statement}, \texttt{\$a}\index{\texttt{\$a} statement}, and
% \texttt{\$p}\index{\texttt{\$p}
% statement} statement types require labels, which allow them to be
% referenced later inside proofs.  A label is considered {\bf
% active}\index{active label} when the statement it is associated with is
% active.  The token\index{token} for a label may be reused
% (redeclared)\index{redeclaration of labels} provided that it is not being used
% for a currently active label.  (Unlike the tokens for math symbols, active
% label tokens may not be redeclared in an inner scope.)  Note that the labels
% of \texttt{\$a} and \texttt{\$p} statements can never be reused after these
% statements appear, because these statements remain active through the end of
% the database.
%
% You might find the reuse of labels a convenient way to have standard names for
% temporary hypotheses, such as \texttt{h1}, \texttt{h2}, etc.  This way you don't have
% to invent unique names for each of them, and in some cases it may be less
% confusing to the reader (although in other cases it might be more confusing, if
% the hypothesis is located far away from the assertion that uses
% it).\footnote{The current implementation requires that all labels, even
% inactive ones, be unique.}

\subsubsection{Frames Revisited}\index{frames and scoping statements}

Now that we have covered scoping, we will look at how an arbitrary
Metamath database can be converted to the simple sequence of extended
frames described on p.~\pageref{framelist}.  This is also how Metamath
stores the database internally when it reads in the database
source.\label{frameconvert} The method is simple.  First, we collect all
constant and variable (\texttt{\$c} and \texttt{\$v}) declarations in
the database, ignoring duplicate declarations of the same variable in
different scopes.  We then put our collected \texttt{\$c} and
\texttt{\$v} declarations at the beginning of the database, so that
their scope is the entire database.  Next, for each assertion in the
database, we determine its frame and extended frame.  The extended frame
is simply the \texttt{\$f}, \texttt{\$e}, and \texttt{\$d} statements
that are active.  The frame is the extended frame with all optional
hypotheses removed.

An equivalent way of saying this is that the extended frame of an assertion
is the collection of all \texttt{\$f}, \texttt{\$e}, and \texttt{\$d} statements
whose scope includes the assertion.
The \texttt{\$f} and \texttt{\$e} statements
occur in the order they appear
(order is irrelevant for \texttt{\$d} statements).

%c%, renaming any
%c%redeclared variables as needed so that all of them have unique names.  (The
%c%exact renaming convention is unimportant.  You might imagine renaming
%c%different declarations of math symbol \texttt{a} as \texttt{a\$1}, \texttt{a\$2}, etc.\
%c%which would prevent any conflicts since \texttt{\$} is not a legal character in a
%c%math symbol token.)

\section{The Anatomy of a Proof} \label{proof}
\index{proof!Metamath, description of}

Each provable assertion (\texttt{\$p}\index{\texttt{\$p} statement} statement) in a
database must include a {\bf proof}\index{proof}.  The proof is located
between the \texttt{\$=}\index{\texttt{\$=} keyword} and \texttt{\$.}\ keywords in the
\texttt{\$p} statement.

In the basic Metamath language\index{basic language}, a proof is a
sequence of statement labels.  This label sequence\index{label sequence}
serves as a set of instructions that the Metamath program uses to
construct a series of math symbol sequences.  The construction must
ultimately result in the math symbol sequence contained between the
\texttt{\$p}\index{\texttt{\$p} statement} and
\texttt{\$=}\index{\texttt{\$=} keyword} keywords of the \texttt{\$p}
statement.  Otherwise, the Metamath program will consider the proof
incorrect, and it will notify you with an appropriate error message when
you ask it to verify the proof.\footnote{To make the loading faster, the
Metamath program does not automatically verify proofs when you
\texttt{read} in a database unless you use the \texttt{/verify}
qualifier.  After a database has been read in, you may use the
\texttt{verify proof *} command to verify proofs.}\index{\texttt{verify
proof} command} Each label in a proof is said to {\bf
reference}\index{label reference} its corresponding statement.

Associated with any assertion\index{assertion} (\texttt{\$p} or
\texttt{\$a}\index{\texttt{\$a} statement} statement) is a set of
hypotheses (\texttt{\$f}\index{\texttt{\$f} statement} or
\texttt{\$e}\index{\texttt{\$e} statement} statements) that are active
with respect to that assertion.  Some are mandatory and the others are
optional.  You should review these concepts if necessary.

Each label\index{label} in a proof must be either the label of a
previous assertion (\texttt{\$a}\index{\texttt{\$a} statement} or
\texttt{\$p}\index{\texttt{\$p} statement} statement) or the label of an
active hypothesis (\texttt{\$e} or \texttt{\$f}\index{\texttt{\$f}
statement} statement) of the \texttt{\$p} statement containing the
proof.  Hypothesis labels may reference both the
mandatory\index{mandatory hypothesis} and the optional hypotheses of the
\texttt{\$p} statement.

The label sequence in a proof specifies a construction in {\bf reverse Polish
notation}\index{reverse Polish notation (RPN)} (RPN).  You may be familiar
with RPN if you have used older
Hewlett--Packard or similar hand-held calculators.
In the calculator analogy, a hypothesis label\index{hypothesis label} is like
a number and an assertion label\index{assertion label} is like an operation
(more precisely, an $n$-ary operation when the
assertion has $n$ \texttt{\$e}-hypotheses).
On an RPN calculator, an operation takes one or more previous numbers in an
input sequence, performs a calculation on them, and replaces those numbers and
itself with the result of the calculation.  For example, the input sequence
$2,3,+$ on an RPN calculator results in $5$, and the input sequence
$2,3,5,{\times},+$ results in $2,15,+$ which results in $17$.

Understanding how RPN is processed involves the concept of a {\bf
stack}\index{stack}\index{RPN stack}, which can be thought of as a set of
temporary memory locations that hold intermediate results.  When Metamath
encounters a hypothesis label it places or {\bf pushes}\index{push} the math
symbol sequence of the hypothesis onto the stack.  When Metamath encounters an
assertion label, it associates the most recent stack entries with the {\em
mandatory} hypotheses\index{mandatory hypothesis} of the assertion, in the
order where the most recent stack entry is associated with the last mandatory
hypothesis of the assertion.  It then determines what
substitutions\index{substitution!variable}\index{variable substitution} have
to be made into the variables of the assertion's mandatory hypotheses to make
them identical to the associated stack entries.  It then makes those same
substitutions into the assertion itself.  Finally, Metamath removes or {\bf
pops}\index{pop} the matched hypotheses from the stack and pushes the
substituted assertion onto the stack.

For the purpose of matching the mandatory hypothesis to the most recent stack
entries, whether a hypothesis is a \texttt{\$e} or \texttt{\$f} statement is
irrelevant.  The only important thing is that a set of
substitutions\footnote{In the Metamath spec (Section~\ref{spec}), we use the
singular term ``substitution'' to refer to the set of substitutions we talk
about here.} exist that allow a match (and if they don't, the proof verifier
will let you know with an error message).  The Metamath language is specified
in such a way that if a set of substitutions exists, it will be unique.
Specifically, the requirement that each variable have a type specified for it
with a \texttt{\$f} statement ensures the uniqueness.

We will illustrate this with an example.
Consider the following Metamath source file:
\begin{verbatim}
$c ( ) -> wff $.
$v p q r s $.
wp $f wff p $.
wq $f wff q $.
wr $f wff r $.
ws $f wff s $.
w2 $a wff ( p -> q ) $.
wnew $p wff ( s -> ( r -> p ) ) $= ws wr wp w2 w2 $.
\end{verbatim}
This Metamath source example shows the definition and ``proof'' (i.e.,
construction) of a well-formed formula (wff)\index{well-formed formula (wff)}
in propositional calculus.  (You may wish to type this example into a file to
experiment with the Metamath program.)  The first two statements declare
(introduce the names of) four constants and four variables.  The next four
statements specify the variable types, namely that
each variable is assumed to be a wff.  Statement \texttt{w2} defines (postulates)
a way to produce a new wff, \texttt{( p -> q )}, from two given wffs \texttt{p} and
\texttt{q}. The mandatory hypotheses of \texttt{w2} are \texttt{wp} and \texttt{wq}.
Statement \texttt{wnew} claims that \texttt{( s -> ( r -> p ) )} is a wff given
three wffs \texttt{s}, \texttt{r}, and \texttt{p}.  More precisely, \texttt{wnew} claims
that the sequence of ten symbols \texttt{wff ( s -> ( r -> p ) )} is provable from
previous assertions and the hypotheses of \texttt{wnew}.  Metamath does not know
or care what a wff is, and as far as it is concerned
the typecode \texttt{wff} is just an
arbitrary constant symbol in a math symbol sequence.  The mandatory hypotheses
of \texttt{wnew} are \texttt{wp}, \texttt{wr}, and \texttt{ws}; \texttt{wq} is an optional
hypothesis.  In our particular proof, the optional hypothesis is not
referenced, but in general, any combination of active (i.e.\ optional and
mandatory) hypotheses could be referenced.  The proof of statement \texttt{wnew}
is the sequence of five labels starting with \texttt{ws} (step~1) and ending with
\texttt{w2} (step~5).

When Metamath verifies the proof, it scans the proof from left to right.  We
will examine what happens at each step of the proof.  The stack starts off
empty.  At step 1, Metamath looks up label \texttt{ws} and determines that it is a
hypothesis, so it pushes the symbol sequence of statement \texttt{ws} onto the
stack:

\begin{center}\begin{tabular}{|l|l|}\hline
{Stack location} & {Contents} \\ \hline \hline
1 & \texttt{wff s} \\ \hline
\end{tabular}\end{center}

Metamath sees that the labels \texttt{wr} and \texttt{wp} in steps~2 and 3 are also
hypotheses, so it pushes them onto the stack.  After step~3, the stack looks
like
this:

\begin{center}\begin{tabular}{|l|l|}\hline
{Stack location} & {Contents} \\ \hline \hline
3 & \texttt{wff p} \\ \hline
2 & \texttt{wff r} \\ \hline
1 & \texttt{wff s} \\ \hline
\end{tabular}\end{center}

At step 4, Metamath sees that label \texttt{w2} is an assertion, so it must do
some processing.  First, it associates the mandatory hypotheses of \texttt{w2},
which are \texttt{wp} and \texttt{wq}, with stack locations~2 and 3, {\em in that
order}. Metamath determines that the only possible way
to make hypothesis \texttt{wp} match (become identical to) stack location~2 and
\texttt{wq} match stack location 3 is to substitute variable \texttt{p} with \texttt{r}
and \texttt{q} with \texttt{p}.  Metamath makes these substitutions into \texttt{w2} and
obtains the symbol sequence \texttt{wff ( r -> p )}.  It removes the hypotheses
from stack locations~2 and 3, then places the result into stack location~2:

\begin{center}\begin{tabular}{|l|l|}\hline
{Stack location} & {Contents} \\ \hline \hline
2 & \texttt{wff ( r -> p )} \\ \hline
1 & \texttt{wff s} \\ \hline
\end{tabular}\end{center}

At step 5, Metamath sees that label \texttt{w2} is an assertion, so it must again
do some processing.  First, it matches the mandatory hypotheses of \texttt{w2},
which are \texttt{wp} and \texttt{wq}, to stack locations 1 and 2.
Metamath determines that the only possible way to make the
hypotheses match is to substitute variable \texttt{p} with \texttt{s} and \texttt{q} with
\texttt{( r -> p )}.  Metamath makes these substitutions into \texttt{w2} and obtains
the symbol
sequence \texttt{wff ( s -> ( r -> p ) )}.  It removes stack
locations 1 and 2, then places the result into stack location~1:

\begin{center}\begin{tabular}{|l|l|}\hline
{Stack location} & {Contents} \\ \hline \hline
1 & \texttt{wff ( s -> ( r -> p ) )} \\ \hline
\end{tabular}\end{center}

After Metamath finishes processing the proof, it checks to see that the
stack contains exactly one element and that this element is
the same as the math symbol sequence in the
\texttt{\$p}\index{\texttt{\$p} statement} statement.  This is the case for our
proof of \texttt{wnew},
so we have proved \texttt{wnew} successfully.  If the result
differs, Metamath will notify you with an error message.  An error message
will also result if the stack contains more than one entry at the end of the
proof, or if the stack did not contain enough entries at any point in the
proof to match all of the mandatory hypotheses\index{mandatory hypothesis} of
an assertion.  Finally, Metamath will notify you with an error message if no
substitution is possible that will make a referenced assertion's hypothesis
match the
stack entries.  You may want to experiment with the different kinds of errors
that Metamath will detect by making some small changes in the proof of our
example.

Metamath's proof notation was designed primarily to express proofs in a
relatively compact manner, not for readability by humans.  Metamath can display
proofs in a number of different ways with the \texttt{show proof}\index{\texttt{show
proof} command} command.  The
\texttt{/lemmon} qualifier displays it in a format that is easier to read when the
proofs are short, and you saw examples of its use in Chapter~\ref{using}.  For
longer proofs, it is useful to see the tree structure of the proof.  A tree
structure is displayed when the \texttt{/lemmon} qualifier is omitted.  You will
probably find this display more convenient as you get used to it. The tree
display of the proof in our example looks like
this:\label{treeproof}\index{tree-style proof}\index{proof!tree-style}
\begin{verbatim}
1     wp=ws    $f wff s
2        wp=wr    $f wff r
3        wq=wp    $f wff p
4     wq=w2    $a wff ( r -> p )
5  wnew=w2  $a wff ( s -> ( r -> p ) )
\end{verbatim}
The number to the left of each line is the step number.  Following it is a
{\bf hypothesis association}\index{hypothesis association}, consisting of two
labels\index{label} separated by \texttt{=}.  To the left of the \texttt{=} (except
in the last step) is the label of a hypothesis of an assertion referenced
later in the proof; here, steps 1 and 4 are the hypothesis associations for
the assertion \texttt{w2} that is referenced in step 5.  A hypothesis association
is indented one level more than the assertion that uses it, so it is easy to
find the corresponding assertion by moving directly down until the indentation
level decreases to one less than where you started from.  To the right of each
\texttt{=} is the proof step label for that proof step.  The statement keyword of
the proof step label is listed next, followed by the content of the top of the
stack (the most recent stack entry) as it exists after that proof step is
processed.  With a little practice, you should have no trouble reading proofs
displayed in this format.

Metamath proofs include the syntax construction of a formula.
In standard mathematics, this kind of
construction is not considered a proper part of the proof at all, and it
certainly becomes rather boring after a while.
Therefore,
by default the \texttt{show proof}\index{\texttt{show proof}
command} command does not show the syntax construction.
Historically \texttt{show proof} command
\textit{did} show the syntax construction, and you needed to add the
\texttt{/essential} option to hide, them, but today
\texttt{/essential} is the default and you need to use
\texttt{/all} to see the syntax constructions.

When verifying a proof, Metamath will check that no mandatory
\texttt{\$d}\index{\texttt{\$d} statement}\index{mandatory \texttt{\$d}
statement} statement of an assertion referenced in a proof is violated
when substitutions\index{substitution!variable}\index{variable
substitution} are made to the variables in the assertion.  For details
see Section~\ref{spec4} or \ref{dollard}.

\subsection{The Concept of Unification} \label{unify}

During the course of verifying a proof, when Metamath\index{Metamath}
encounters an assertion label\index{assertion label}, it associates the
mandatory hypotheses\index{mandatory hypothesis} of the assertion with the top
entries of the RPN stack\index{stack}\index{RPN stack}.  Metamath then
determines what substitutions\index{substitution!variable}\index{variable
substitution} it must make to the variables in the assertion's mandatory
hypotheses in order for these hypotheses to become identical to their
corresponding stack entries.  This process is called {\bf
unification}\index{unification}.  (We also informally use the term
``unification'' to refer to a set of substitutions that results from the
process, as in ``two unifications are possible.'')  After the substitutions
are made, the hypotheses are said to be {\bf unified}.

If no such substitutions are possible, Metamath will consider the proof
incorrect and notify you with an error message.
% (deleted 3/10/07, per suggestion of Mel O'Cat:)
% The syntax of the
% Metamath language ensures that if a set of substitutions exists, it
% will be unique.

The general algorithm for unification described in the literature is
somewhat complex.
However, in the case of Metamath it is intentionally trivial.
Mandatory hypotheses must be
pushed on the proof stack in the order in which they appear.
In addition, each variable must have its type specified
with a \texttt{\$f} hypothesis before it is used
and that each \texttt{\$f} hypothesis
have the restricted syntax of a typecode (a constant) followed by a variable.
The typecode in the \texttt{\$f} hypothesis must match the first symbol of
the corresponding RPN stack entry (which will also be a constant), so
the only possible match for the variable in the \texttt{\$f} hypothesis is
the sequence of symbols in the stack entry after the initial constant.

In the Proof Assistant\index{Proof Assistant}, a more general unification
algorithm is used.  While a proof is being developed, sometimes not enough
information is available to determine a unique unification.  In this case
Metamath will ask you to pick the correct one.\index{ambiguous
unification}\index{unification!ambiguous}

\section{Extensions to the Metamath Language}\index{extended
language}

\subsection{Comments in the Metamath Language}\label{comments}
\index{markup notation}
\index{comments!markup notation}

The commenting feature allows you to annotate the contents of
a database.  Just as with most
computer languages, comments are ignored for the purpose of interpreting the
contents of the database. Comments effectively act as
additional white space\index{white
space} between tokens
when a database is parsed.

A comment may be placed at the beginning, end, or
between any two tokens\index{token} in a source file.

Comments have the following syntax:
\begin{center}
 \texttt{\$(} {\em text} \texttt{\$)}
\end{center}
Here,\index{\texttt{\$(} and \texttt{\$)} auxiliary
keywords}\index{comment} {\em text} is a string, possibly empty, of any
characters in Metamath's character set (p.~\pageref{spec1chars}), except
that the character strings \texttt{\$(} and \texttt{\$)} may not appear
in {\em text}.  Thus nested comments are not
permitted:\footnote{Computer languages have differing standards for
nested comments, and rather than picking one it was felt simplest not to
allow them at all, at least in the current version (0.177) of
Metamath\index{Metamath!limitations of version 0.177}.} Metamath will
complain if you give it
\begin{center}
 \texttt{\$( This is a \$( nested \$) comment.\ \$)}
\end{center}
To compensate for this non-nesting behavior, I often change all \texttt{\$}'s
to \texttt{@}'s in sections of Metamath code I wish to comment out.

The Metamath program supports a number of markup mechanisms and conventions
to generate good-looking results in \LaTeX\ and {\sc html},
as discussed below.
These markup features have to do only with how the comments are typeset,
and have no effect on how Metamath verifies the proofs in the database.
The improper
use of them may result in incorrectly typeset output, but no Metamath
error messages will result during the \texttt{read} and \texttt{verify
proof} commands.  (However, the \texttt{write
theorem\texttt{\char`\_}list} command
will check for markup errors as a side-effect of its
{\sc html} generation.)
Section~\ref{texout} has instructions for creating \LaTeX\ output, and
section~\ref{htmlout} has instructions for creating
{\sc html}\index{HTML} output.

\subsubsection{Headings}\label{commentheadings}

If the \texttt{\$(} is immediately followed by a new line
starting with a heading marker, it is a header.
This can start with:

\begin{itemize}
 \item[] \texttt{\#\#\#\#} - major part header
 \item[] \texttt{\#*\#*} - section header
 \item[] \texttt{=-=-} - subsection header
 \item[] \texttt{-.-.} - subsubsection header
\end{itemize}

The line following the marker line
will be used for the table of contents entry, after trimming spaces.
The next line should be another (closing) matching marker line.
Any text after that
but before the closing \texttt{\$}, such as an extended description of the
section, will be included on the \texttt{mmtheoremsNNN.html} page.

For more information, run
\texttt{help write theorem\char`\_list}.

\subsubsection{Math mode}
\label{mathcomments}
\index{\texttt{`} inside comments}
\index{\texttt{\char`\~} inside comments}
\index{math mode}

Inside of comments, a string of tokens\index{token} enclosed in
grave accents\index{grave accent (\texttt{`})} (\texttt{`}) will be converted
to standard mathematical symbols during
{\sc HTML}\index{HTML} or \LaTeX\ output
typesetting,\index{latex@{\LaTeX}} according to the information in the
special \texttt{\$t}\index{\texttt{\$t} comment}\index{typesetting
comment} comment in the database
(see section~\ref{tcomment} for information about the typesetting
comment, and Appendix~\ref{ASCII} to see examples of its results).

The first grave accent\index{grave accent (\texttt{`})} \texttt{`}
causes the output processor to enter {\bf math mode}\index{math mode}
and the second one exits it.
In this
mode, the characters following the \texttt{`} are interpreted as a
sequence of math symbol tokens separated by white space\index{white
space}.  The tokens are looked up in the \texttt{\$t}
comment\index{\texttt{\$t} comment}\index{typesetting comment} and if
found, they will be replaced by the standard mathematical symbols that
they correspond to before being placed in the typeset output file.  If
not found, the symbol will be output as is and a warning will be issued.
The tokens do not have to be active in the database, although a warning
will be issued if they are not declared with \texttt{\$c} or
\texttt{\$v} statements.

Two consecutive
grave accents \texttt{``} are treated as a single actual grave accent
(both inside and outside of math mode) and will not cause the output
processor to enter or exit math mode.

Here is an example of its use\index{Pierce's axiom}:
\begin{center}
\texttt{\$( Pierce's axiom, ` ( ( ph -> ps ) -> ph ) -> ph ` ,\\
         is not very intuitive. \$)}
\end{center}
becomes
\begin{center}
   \texttt{\$(} Pierce's axiom, $((\varphi \rightarrow \psi)\rightarrow
\varphi)\rightarrow \varphi$, is not very intuitive. \texttt{\$)}
\end{center}

Note that the math symbol tokens\index{token} must be surrounded by white
space\index{white space}.
%, since there is no context that allows ambiguity to be
%resolved, as is the case with math symbol sequences in some of the Metamath
%statements.
White space should also surround the \texttt{`}
delimiters.

The math mode feature also gives you a quick and easy way to generate
text containing mathematical symbols, independently of the intended
purpose of Metamath.\index{Metamath!using as a math editor} To do this,
simply create your text with grave accents surrounding your formulas,
after making sure that your math symbols are mapped to \LaTeX\ symbols
as described in Appendix~\ref{ASCII}.  It is easier if you start with a
database with predefined symbols such as \texttt{set.mm}.  Use your
grave-quoted math string to replace an existing comment, then typeset
the statement corresponding to that comment following the instructions
from the \texttt{help tex} command in the Metamath program.  You will
then probably want to edit the resulting file with a text editor to fine
tune it to your exact needs.

\subsubsection{Label Mode}\index{label mode}

Outside of math mode, a tilde\index{tilde (\texttt{\char`\~})} \verb/~/
indicates to Metamath's\index{Metamath} output processor that the
token\index{token} that follows (i.e.\ the characters up to the next
white space\index{white space}) represents a statement label or URL.
This formatting mode is called {\bf label mode}\index{label mode}.
If a literal tilde
is desired (outside of math mode) instead of label mode,
use two tildes in a row to represent it.

When generating a \LaTeX\ output file,
the following token will be formatted in \texttt{typewriter}
font, and the tilde removed, to make it stand out from the rest of the text.
This formatting will be applied to all characters after the
tilde up to the first white space\index{white space}.
Whether
or not the token is an actual statement label is not checked, and the
token does not have to have the correct syntax for a label; no error
messages will be produced.  The only effect of the label mode on the
output is that typewriter font will be used for the tokens that are
placed in the \LaTeX\ output file.

When generating {\sc html},
the tokens after the tilde {\em must} be a URL (either http: or https:)
or a valid label.
Error messages will be issued during that output if they aren't.
A hyperlink will be generated to that URL or label.

\subsubsection{Link to bibliographical reference}\index{citation}%
\index{link to bibliographical reference}

Bibliographical references are handled specially when generating
{\sc html} if formatted specially.
Text in the form \texttt{[}{\em author}\texttt{]}
is considered a link to a bibliographical reference.
See \texttt{help html} and \texttt{help write
bibliography} in the Metamath program for more
information.
% \index{\texttt{\char`\[}\ldots\texttt{]} inside comments}
See also Sections~\ref{tcomment} and \ref{wrbib}.

The \texttt{[}{\em author}\texttt{]} notation will also create an entry in
the bibliography cross-reference file generated by \texttt{write
bibliography} (Section~\ref{wrbib}) for {\sc HTML}.
For this to work properly, the
surrounding comment must be formatted as follows:
\begin{quote}
    {\em keyword} {\em label} {\em noise-word}
     \texttt{[}{\em author}\texttt{] p.} {\em number}
\end{quote}
for example
\begin{verbatim}
     Theorem 5.2 of [Monk] p. 223
\end{verbatim}
The {\em keyword} is not case sensitive and must be one of the following:
\begin{verbatim}
     theorem lemma definition compare proposition corollary
     axiom rule remark exercise problem notation example
     property figure postulate equation scheme chapter
\end{verbatim}
The optional {\em label} may consist of more than one
(non-{\em keyword} and non-{\em noise-word}) word.
The optional {\em noise-word} is one of:
\begin{verbatim}
     of in from on
\end{verbatim}
and is  ignored when the cross-reference file is created.  The
\texttt{write
biblio\-graphy} command will perform error checking to verify the
above format.\index{error checking}

\subsubsection{Parentheticals}\label{parentheticals}

The end of a comment may include one or more parenthicals, that is,
statements enclosed in parentheses.
The Metamath program looks for certain parentheticals and can issue
warnings based on them.
They are:

\begin{itemize}
 \item[] \texttt{(Contributed by }
   \textit{NAME}\texttt{,} \textit{DATE}\texttt{.)} -
   document the original contributor's name and the date it was created.
 \item[] \texttt{(Revised by }
   \textit{NAME}\texttt{,} \textit{DATE}\texttt{.)} -
   document the contributor's name and creation date
   that resulted in significant revision
   (not just an automated minimization or shortening).
 \item[] \texttt{(Proof shortened by }
   \textit{NAME}\texttt{,} \textit{DATE}\texttt{.)} -
   document the contributor's name and date that developed a significant
   shortening of the proof (not just an automated minimization).
 \item[] \texttt{(Proof modification is discouraged.)} -
   Note that this proof should normally not be modified.
 \item[] \texttt{(New usage is discouraged.)} -
   Note that this assertion should normally not be used.
\end{itemize}

The \textit{DATE} must be in form YYYY-MMM-DD, where MMM is the
English abbreviation of that month.

\subsubsection{Other markup}\label{othermarkup}
\index{markup notation}

There are other markup notations for generating good-looking results
beyond math mode and label mode:

\begin{itemize}
 \item[]
         \texttt{\char`\_} (underscore)\index{\texttt{\char`\_} inside comments} -
             Italicize text starting from
              {\em space}\texttt{\char`\_}{\em non-space} (i.e.\ \texttt{\char`\_}
              with a space before it and a non-space character after it) until
             the next
             {\em non-space}\texttt{\char`\_}{\em space}.  Normal
             punctuation (e.g.\ a trailing
             comma or period) is ignored when determining {\em space}.
 \item[]
         \texttt{\char`\_} (underscore) - {\em
         non-space}\texttt{\char`\_}{\em non-space-string}, where
          {\em non-space-string} is a string of non-space characters,
         will make {\em non-space-string} become a subscript.
 \item[]
         \texttt{<HTML>}...\texttt{</HTML>} - do not convert
         ``\texttt{<}'' and ``\texttt{>}''
         in the enclosed text when generating {\sc HTML},
         otherwise process markup normally. This allows direct insertion
         of {\sc html} commands.
 \item[]
       ``\texttt{\&}ref\texttt{;}'' - insert an {\sc HTML}
         character reference.
         This is how to insert arbitrary Unicode characters
         (such as accented characters).  Currently only directly supported
         when generating {\sc HTML}.
\end{itemize}

It is recommended that spaces surround any \texttt{\char`\~} and
\texttt{`} tokens in the comment and that a space follow the {\em label}
after a \texttt{\char`\~} token.  This will make global substitutions
to change labels and symbol names much easier and also eliminate any
future chance of ambiguity.  Spaces around these tokens are automatically
removed in the final output to conform with normal rules of punctuation;
for example, a space between a trailing \texttt{`} and a left parenthesis
will be removed.

A good way to become familiar with the markup notation is to look at
the extensive examples in the \texttt{set.mm} database.

\subsection{The Typesetting Comment (\texttt{\$t})}\label{tcomment}

The typesetting comment \texttt{\$t} in the input database file
provides the information necessary to produce good-looking results.
It provides \LaTeX\ and {\sc html}
definitions for math symbols,
as well supporting as some
customization of the generated web page.
If you add a new token to a database, you should also
update the \texttt{\$t} comment information if you want to eventually
create output in \LaTeX\ or {\sc HTML}.
See the
\texttt{set.mm}\index{set theory database (\texttt{set.mm})} database
file for an extensive example of a \texttt{\$t} comment illustrating
many of the features described below.

Programs that do not need to generate good-looking presentation results,
such as programs that only verify Metamath databases,
can completely ignore typesetting comments
and just treat them as normal comments.
Even the Metamath program only consults the
\texttt{\$t} comment information when it needs to generate typeset output
in \LaTeX\ or {\sc HTML}
(e.g., when you open a \LaTeX\ output file with the \texttt{open tex} command).

We will first discuss the syntax of typesetting comments, and then
briefly discuss how this can be used within the Metamath program.

\subsubsection{Typesetting Comment Syntax Overview}

The typesetting comment is identified by the token
\texttt{\$t}\index{\texttt{\$t} comment}\index{typesetting comment} in
the comment, and the typesetting comment ends at the matching
\texttt{\$)}:
\[
  \mbox{\tt \$(\ }
  \mbox{\tt \$t\ }
  \underbrace{
    \mbox{\tt \ \ \ \ \ \ \ \ \ \ \ }
    \cdots
    \mbox{\tt \ \ \ \ \ \ \ \ \ \ \ }
  }_{\mbox{Typesetting definitions go here}}
  \mbox{\tt \ \$)}
\]

There must be one or more white space characters, and only white space
characters, between the \texttt{\$(} that starts the comment
and the \texttt{\$t} symbol,
and the \texttt{\$t} must be followed by one
or more white space characters
(see section \ref{whitespace} for the definition of white space characters).
The typesetting comment continues until the comment end token \texttt{\$)}
(which must be preceded by one or more white space characters).

In version 0.177\index{Metamath!limitations of version 0.177} of the
Metamath program, there may be only one \texttt{\$t} comment in a
database.  This restriction may be lifted in the future to allow
many \texttt{\$t} comments in a database.

Between the \texttt{\$t} symbol (and its following white space) and the
comment end token \texttt{\$)} (and its preceding white space)
is a sequence of one or more typesetting definitions, where
each definition has the form
\textit{definition-type arg arg ... ;}.
Each of the zero or more \textit{arg} values
can be either a typesetting data or a keyword
(what keywords are allowed, and where, depends on the specific
\textit{definition-type}).
The \textit{definition-type}, and each argument \textit{arg},
are separated by one or more white space characters.
Every definition ends in an unquoted semicolon;
white space is not required before the terminating semicolon of a definition.
Each definition should start on a new line.\footnote{This
restriction of the current version of Metamath
(0.177)\index{Metamath!limitations of version 0.177} may be removed
in a future version, but you should do it anyway for readability.}

For example, this typesetting definition:
\begin{center}
 \verb$latexdef "C_" as "\subseteq";$
\end{center}
defines the token \verb$C_$ as the \LaTeX\ symbol $\subseteq$ (which means
``subset'').

Typesetting data is a sequence of one or more quoted strings
(if there is more than one, they are connected by \texttt{\char`\+}).
Often a single quoted string is used to provide data for a definition, using
either double (\texttt{\char`\"}) or single (\texttt{'}) quotation marks.
However,
{\em a quoted string (enclosed in quotation marks) may not include
line breaks.}
A quoted string
may include a quotation mark that matches the enclosing quotes by repeating
the quotation mark twice.  Here are some examples:

\begin{tabu}   { l l }
\textbf{Example} & \textbf{Meaning} \\
\texttt{\char`\"a\char`\"\char`\"b\char`\"} & \texttt{a\char`\"b} \\
\texttt{'c''d'} & \texttt{c'd} \\
\texttt{\char`\"e''f\char`\"} & \texttt{e''f} \\
\texttt{'g\char`\"\char`\"h'} & \texttt{g\char`\"\char`\"h} \\
\end{tabu}

Finally, a long quoted string
may be broken up into multiple quoted strings (considered, as a whole,
a single quoted string) and joined with \texttt{\char`\+}.
You can even use multiple lines as long as a
'+' is at the end of every line except the last one.
The \texttt{\char`\+} should be preceded and followed by at least one
white space character.
Thus, for example,
\begin{center}
 \texttt{\char`\"ab\char`\"\ \char`\+\ \char`\"cd\char`\"
    \ \char`\+\ \\ 'ef'}
\end{center}
is the same as
\begin{center}
 \texttt{\char`\"abcdef\char`\"}
\end{center}

{\sc c}-style comments \texttt{/*}\ldots\texttt{*/} are also supported.

In practice, whenever you add a new math token you will often want to add
typesetting definitions using
\texttt{latexdef}, \texttt{htmldef}, and
\texttt{althtmldef}, as described below.
That way, they will all be up to date.
Of course, whether or not you want to use all three definitions will
depend on how the database is intended to be used.

Below we discuss the different possible \textit{definition-kind} options.
We will show data surrounded by double quotes (in practice they can also use
single quotes and/or be a sequence joined by \texttt{+}s).
We will use specific names for the \textit{data} to make clear what
the data is used for, such as
{\em math-token} (for a Metamath math token,
{\em latex-string} (for string to be placed in a \LaTeX\ stream),
{\em {\sc html}-code} (for {\sc html} code),
and {\em filename} (for a filename).

\subsubsection{Typesetting Comment - \LaTeX}

The syntax for a \LaTeX\ definition is:
\begin{center}
 \texttt{latexdef "}{\em math-token}\texttt{" as "}{\em latex-string}\texttt{";}
\end{center}
\index{latex definitions@\LaTeX\ definitions}%
\index{\texttt{latexdef} statement}

The {\em token-string} and {\em latex-string} are the data
(character strings) for
the token and the \LaTeX\ definition of the token, respectively,

These \LaTeX\ definitions are used by the Metamath program
when it is asked to product \LaTeX output using
the \texttt{write tex} command.

\subsubsection{Typesetting Comment - {\sc html}}

The key kinds of {\sc HTML} definitions have the following syntax:

\vskip 1ex
    \texttt{htmldef "}{\em math-token}\texttt{" as "}{\em
    {\sc html}-code}\texttt{";}\index{\texttt{htmldef} statement}
                    \ \ \ \ \ \ldots

    \texttt{althtmldef "}{\em math-token}\texttt{" as "}{\em
{\sc html}-code}\texttt{";}\index{\texttt{althtmldef} statement}

                    \ \ \ \ \ \ldots

Note that in {\sc HTML} there are two possible definitions for math tokens.
This feature is useful when
an alternate representation of symbols is desired, for example one that
uses Unicode entities and another uses {\sc gif} images.

There are many other typesetting definitions that can control {\sc HTML}.
These include:

\vskip 1ex

    \texttt{htmldef "}{\em math-token}\texttt{" as "}{\em {\sc
    html}-code}\texttt{";}

    \texttt{htmltitle "}{\em {\sc html}-code}\texttt{";}%
\index{\texttt{htmltitle} statement}

    \texttt{htmlhome "}{\em {\sc html}-code}\texttt{";}%
\index{\texttt{htmlhome} statement}

    \texttt{htmlvarcolor "}{\em {\sc html}-code}\texttt{";}%
\index{\texttt{htmlvarcolor} statement}

    \texttt{htmlbibliography "}{\em filename}\texttt{";}%
\index{\texttt{htmlbibliography} statement}

\vskip 1ex

\noindent The \texttt{htmltitle} is the {\sc html} code for a common
title, such as ``Metamath Proof Explorer.''  The \texttt{htmlhome} is
code for a link back to the home page.  The \texttt{htmlvarcolor} is
code for a color key that appears at the bottom of each proof.  The file
specified by {\em filename} is an {\sc html} file that is assumed to
have a \texttt{<A NAME=}\ldots\texttt{>} tag for each bibiographic
reference in the database comments.  For example, if
\texttt{[Monk]}\index{\texttt{\char`\[}\ldots\texttt{]} inside comments}
occurs in the comment for a theorem, then \texttt{<A NAME='Monk'>} must
be present in the file; if not, a warning message is given.

Associated with
\texttt{althtmldef}
are the statements
\vskip 1ex

    \texttt{htmldir "}{\em
      directoryname}\texttt{";}\index{\texttt{htmldir} statement}

    \texttt{althtmldir "}{\em
     directoryname}\texttt{";}\index{\texttt{althtmldir} statement}

\vskip 1ex
\noindent giving the directories of the {\sc gif} and Unicode versions
respectively; their purpose is to provide cross-linking between the
two versions in the generated web pages.

When two different types of pages need to be produced from a single
database, such as the Hilbert Space Explorer that extends the Metamath
Proof Explorer, ``extended'' variables may be declared in the
\texttt{\$t} comment:
\vskip 1ex

    \texttt{exthtmltitle "}{\em {\sc html}-code}\texttt{";}%
\index{\texttt{exthtmltitle} statement}

    \texttt{exthtmlhome "}{\em {\sc html}-code}\texttt{";}%
\index{\texttt{exthtmlhome} statement}

    \texttt{exthtmlbibliography "}{\em filename}\texttt{";}%
\index{\texttt{exthtmlbibliography} statement}

\vskip 1ex
\noindent When these are declared, you also must declare
\vskip 1ex

    \texttt{exthtmllabel "}{\em label}\texttt{";}%
\index{\texttt{exthtmllabel} statement}

\vskip 1ex \noindent that identifies the database statement where the
``extended'' section of the database starts (in our example, where the
Hilbert Space Explorer starts).  During the generation of web pages for
that starting statement and the statements after it, the {\sc html} code
assigned to \texttt{exthtmltitle} and \texttt{exthtmlhome} is used
instead of that assigned to \texttt{htmltitle} and \texttt{htmlhome},
respectively.

\begin{sloppy}
\subsection{Additional Information Com\-ment (\texttt{\$j})} \label{jcomment}
\end{sloppy}

The additional information comment, aka the
\texttt{\$j}\index{\texttt{\$j} comment}\index{additional information comment}
comment,
provides a way to add additional structured information that can
be optionally parsed by systems.

The additional information comment is parsed the same way as the
typesetting comment (\texttt{\$t}) (see section \ref{tcomment}).
That is,
the additional information comment begins with the token
\texttt{\$j} within a comment,
and continues until the comment close \texttt{\$)}.
Within an additional information comment is a sequence of one or more
commands of the form \texttt{command arg arg ... ;}
where each of the zero or more \texttt{arg} values
can be either a quoted string or a keyword.
Note that every command ends in an unquoted semicolon.
If a verifier is parsing an additional information comment, but
doesn't recognize a particular command, it must skip the command
by finding the end of the command (an unquoted semicolon).

A database may have 0 or more additional information comments.
Note, however, that a verifier may ignore these comments entirely or only
process certain commands in an additional information comment.
The \texttt{mmj2} verifier supports many commands in additional information
comments.
We encourage systems that process additional information comments
to coordinate so that they will use the same command for the same effect.

Examples of additional information comments with various commands
(from the \texttt{set.mm} database) are:

\begin{itemize}
   \item Define the syntax and logical typecodes,
     and declare that our grammar is
     unambiguous (verifiable using the KLR parser, with compositing depth 5).
\begin{verbatim}
  $( $j
    syntax 'wff';
    syntax '|-' as 'wff';
    unambiguous 'klr 5';
  $)
\end{verbatim}

   \item Register $\lnot$ and $\rightarrow$ as primitive expressions
           (lacking definitions).
\begin{verbatim}
  $( $j primitive 'wn' 'wi'; $)
\end{verbatim}

   \item There is a special justification for \texttt{df-bi}.
\begin{verbatim}
  $( $j justification 'bijust' for 'df-bi'; $)
\end{verbatim}

   \item Register $\leftrightarrow$ as an equality for its type (wff).
\begin{verbatim}
  $( $j
    equality 'wb' from 'biid' 'bicomi' 'bitri';
    definition 'dfbi1' for 'wb';
  $)
\end{verbatim}

   \item Theorem \texttt{notbii} is the congruence law for negation.
\begin{verbatim}
  $( $j congruence 'notbii'; $)
\end{verbatim}

   \item Add \texttt{setvar} as a typecode.
\begin{verbatim}
  $( $j syntax 'setvar'; $)
\end{verbatim}

   \item Register $=$ as an equality for its type (\texttt{class}).
\begin{verbatim}
  $( $j equality 'wceq' from 'eqid' 'eqcomi' 'eqtri'; $)
\end{verbatim}

\end{itemize}


\subsection{Including Other Files in a Metamath Source File} \label{include}
\index{\texttt{\$[} and \texttt{\$]} auxiliary keywords}

The keywords \texttt{\$[} and \texttt{\$]} specify a file to be
included\index{included file}\index{file inclusion} at that point in a
Metamath\index{Metamath} source file\index{source file}.  The syntax for
including a file is as follows:
\begin{center}
\texttt{\$[} {\em file-name} \texttt{\$]}
\end{center}

The {\em file-name} should be a single token\index{token} with the same syntax
as a math symbol (i.e., all 93 non-whitespace
printable characters other than \texttt{\$} are
allowed, subject to the file-naming limitations of your operating system).
Comments may appear between the \texttt{\$[} and \texttt{\$]} keywords.  Included
files may include other files, which may in turn include other files, and so
on.

For example, suppose you want to use the set theory database as the starting
point for your own theory.  The first line in your file could be
\begin{center}
\texttt{\$[ set.mm \$]}
\end{center} All of the information (axioms, theorems,
etc.) in \texttt{set.mm} and any files that {\em it} includes will become
available for you to reference in your file. This can help make your work more
modular. A drawback to including files is that if you change the name of a
symbol or the label of a statement, you must also remember to update any
references in any file that includes it.


The naming conventions for included files are the same as those of your
operating system.\footnote{On the Macintosh, prior to Mac OS X,
 a colon is used to separate disk
and folder names from your file name.  For example, {\em volume}\texttt{:}{\em
file-name} refers to the root directory, {\em volume}\texttt{:}{\em
folder-name}\texttt{:}{\em file-name} refers to a folder in root, and {\em
volume}\texttt{:}{\em folder-name}\texttt{:}\ldots\texttt{:}{\em file-name} refers to a
deeper folder.  A simple {\em file-name} refers to a file in the folder from
which you launch the Metamath application.  Under Mac OS X and later,
the Metamath program is run under the Terminal application, which
conforms to Unix naming conventions.}\index{Macintosh file
names}\index{file names!Macintosh}\label{includef} For compatibility among
operating systems, you should keep the file names as simple as possible.  A
good convention to use is {\em file}\texttt{.mm} where {\em file} is eight
characters or less, in lower case.

There is no limit to the nesting depth of included files.  One thing that you
should be aware of is that if two included files themselves include a common
third file, only the {\em first} reference to this common file will be read
in.  This allows you to include two or more files that build on a common
starting file without having to worry about label and symbol conflicts that
would occur if the common file were read in more than once.  (In fact, if a
file includes itself, the self-reference will be ignored, although of course
it would not make any sense to do that.)  This feature also means, however,
that if you try to include a common file in several inner blocks, the result
might not be what you expect, since only the first reference will be replaced
with the included file (unlike the include statement in most other computer
languages).  Thus you would normally include common files only in the
outermost block\index{outermost block}.

\subsection{Compressed Proof Format}\label{compressed1}\index{compressed
proof}\index{proof!compressed}

The proof notation presented in Section~\ref{proof} is called a
{\bf normal proof}\index{normal proof}\index{proof!normal} and in principle is
sufficient to express any proof.  However, proofs often contain steps and
subproofs that are identical.  This is particularly true in typical
Metamath\index{Metamath} applications, because Metamath requires that the math
symbol sequence (usually containing a formula) at each step be separately
constructed, that is, built up piece by piece. As a result, a lot of
repetition often results.  The {\bf compressed proof} format allows Metamath
to take advantage of this redundancy to shorten proofs.

The specification for the compressed proof format is given in
Appen\-dix~\ref{compressed}.

Normally you need not concern yourself with the details of the compressed
proof format, since the Metamath program will allow you to convert from
the normal format to the compressed format with ease, and will also
automatically convert from the compressed format when proofs are displayed.
The overall structure of the compressed format is as follows:
\begin{center}
  \texttt{\$= ( } {\em label-list} \texttt{) } {\em compressed-proof\ }\ \texttt{\$.}
\end{center}
\index{\texttt{\$=} keyword}
The first \texttt{(} serves as a flag to Metamath that a compressed proof
follows.  The {\em label-list} includes all statements referred to by the
proof except the mandatory hypotheses\index{mandatory hypothesis}.  The {\em
compressed-proof} is a compact encoding of the proof, using upper-case
letters, and can be thought of as a large integer in base 26.  White
space\index{white space} inside a {\em compressed-proof} is
optional and is ignored.

It is important to note that the order of the mandatory hypotheses of
the statement being proved must not be changed if the compressed proof
format is used, otherwise the proof will become incorrect.  The reason
for this is that the mandatory hypotheses are not mentioned explicitly
in the compressed proof in order to make the compression more efficient.
If you wish to change the order of mandatory hypotheses, you must first
convert the proof back to normal format using the \texttt{save proof
{\em statement} /normal}\index{\texttt{save proof} command} command.
Later, you can go back to compressed format with \texttt{save proof {\em
statement} /compressed}.

During error checking with the \texttt{verify proof} command, an error
found in a compressed proof may point to a character in {\em
compressed-proof}, which may not be very meaningful to you.  In this
case, try to \texttt{save proof /normal} first, then do the
\texttt{verify proof} again.  In general, it is best to make sure a
proof is correct before saving it in compressed format, because severe
errors are less likely to be recoverable than in normal format.

\subsection{Specifying Unknown Proofs or Subproofs}\label{unknown}

In a proof under development, any step or subproof that is not yet known
may be represented with a single \texttt{?}.  For the purposes of
parsing the proof, the \texttt{?}\ \index{\texttt{]}@\texttt{?}\ inside
proofs} will push a single entry onto the RPN stack just as if it were a
hypothesis.  While developing a proof with the Proof
Assistant\index{Proof Assistant}, a partially developed proof may be
saved with the \texttt{save new{\char`\_}proof}\index{\texttt{save
new{\char`\_}proof} command} command, and \texttt{?}'s will be placed at
the appropriate places.

All \texttt{\$p}\index{\texttt{\$p} statement} statements must have
proofs, even if they are entirely unknown.  Before creating a proof with
the Proof Assistant, you should specify a completely unknown proof as
follows:
\begin{center}
  {\em label} \texttt{\$p} {\em statement} \texttt{\$= ?\ \$.}
\end{center}
\index{\texttt{\$=} keyword}
\index{\texttt{]}@\texttt{?}\ inside proofs}

The \texttt{verify proof}\index{\texttt{verify proof} command} command
will check the known portions of a partial proof for errors, but will
warn you that the statement has not been proved.

Note that partially developed proofs may be saved in compressed format
if desired.  In this case, you will see one or more \texttt{?}'s in the
{\em compressed-proof} part.\index{compressed
proof}\index{proof!compressed}

\section{Axioms vs.\ Definitions}\label{definitions}

The \textit{basic}
Metamath\index{Metamath} language and program
make no distinction\index{axiom vs.\
definition} between axioms\index{axiom} and
definitions.\index{definition} The \texttt{\$a}\index{\texttt{\$a}
statement} statement is used for both.  At first, this may seem
puzzling.  In the minds of many mathematicians, the distinction is
clear, even obvious, and hardly worth discussing.  A definition is
considered to be merely an abbreviation that can be replaced by the
expression for which it stands; although unless one actually does this,
to be precise then one should say that a theorem\index{theorem} is a
consequence of the axioms {\em and} the definitions that are used in the
formulation of the theorem \cite[p.~20]{Behnke}.\index{Behnke, H.}

\subsection{What is a Definition?}

What is a definition?  In its simplest form, a definition introduces a new
symbol and provides an unambiguous rule to transform an expression containing
the new symbol to one without it.  The concept of a ``proper
definition''\index{proper definition}\index{definition!proper} (as opposed to
a creative definition)\index{creative definition}\index{definition!creative}
that is usually agreed upon is (1) the definition should not strengthen the
language and (2) any symbols introduced by the definition should be eliminable
from the language \cite{Nemesszeghy}\index{Nemesszeghy, E. Z.}.  In other
words, they are mere typographical conveniences that do not belong to the
system and are theoretically superfluous.  This may seem obvious, but in fact
the nature of definitions can be subtle, sometimes requiring difficult
metatheorems to establish that they are not creative.

A more conservative stance was taken by logician S.
Le\'{s}niewski.\index{Le\'{s}niewski, S.}
\begin{quote}
Le\'{s}niewski
regards definitions as theses of the system.  In this respect they do
not differ either from the axioms or from theorems, i.e.\ from the
theses added to the system on the basis of the rule of substitution or
the rule of detachment [modus ponens].  Once definitions have been
accepted as theses of the system, it becomes necessary to consider them
as true propositions in the same sense in which axioms are true
\cite{Lejewski}.
\end{quote}\index{Lejewski, Czeslaw}

Let us look at some simple examples of definitions in propositional
calculus.  Consider the definition of logical {\sc or}
(disjunction):\index{disjunction ($\vee$)} ``$P\vee Q$ denotes $\neg P
\rightarrow Q$ (not $P$ implies $Q$).''  It is very easy to recognize a
statement making use of this definition, because it introduces the new
symbol $\vee$ that did not previously exist in the language.  It is easy
to see that no new theorems of the original language will result from
this definition.

Next, consider a definition that eliminates parentheses:  ``$P
\rightarrow Q\rightarrow R$ denotes $P\rightarrow (Q \rightarrow R)$.''
This is more subtle, because no new symbols are introduced.  The reason
this definition is considered proper is that no new symbol sequences
that are valid wffs (well-formed formulas)\index{well-formed formula
(wff)} in the original language will result from the definition, since
``$P \rightarrow Q\rightarrow R$'' is not a wff in the original
language.  Here, we implicitly make use of the fact that there is a
decision procedure that allows us to determine whether or not a symbol
sequence is a wff, and this fact allows us to use symbol sequences that
are not wffs to represent other things (such as wffs) by means of the
definition.  However, to justify the definition as not being creative we
need to prove that ``$P \rightarrow Q\rightarrow R$'' is in fact not a
wff in the original language, and this is more difficult than in the
case where we simply introduce a new symbol.

%Now let's take this reasoning to an extreme.  Propositional calculus is a
%decidable theory,\footnote{This means that a mechanical algorithm exists to
%determine whether or not a wff is a theorem.} so in principle we could make use
%of symbol sequences that are not theorems to represent other things (say, to
%encode actual theorems in a more compact way).  For example, let us extend the
%language by defining a wff ``$P$'' in the extended language as the theorem
%``$P\rightarrow P$''\footnote{This is one of the first theorems proved in the
%Metamath database \texttt{set.mm}.}\index{set
%theory database (\texttt{set.mm})} in the original language whenever ``$P$'' is
%not a theorem in the original language.  In the extended language, any wff
%``$Q$'' thus represents a theorem; to find out what theorem (in the original
%language) ``$Q$'' represents, we determine whether ``$Q$'' is a theorem in the
%original language (before the definition was introduced).  If so, we're done; if
%not, we replace ``$Q$'' by ``$Q\rightarrow Q$'' to eliminate the definition.
%This definition is therefore eliminable, and it does not ``strengthen'' the
%language because any wff that is not a theorem is not in the set of statements
%provable in the original language and thus is available for use by definitions.
%
%Of course, a definition such as this would render practically useless the
%communication of theorems of propositional calculus; but
%this is just a human shortcoming, since we can't always easily discern what is
%and is not a theorem by inspection.  In fact, the extended theory with this
%definition has no more and no less information than the original theory; it just
%expresses certain theorems of the form ``$P\rightarrow P$''
%in a more compact way.
%
%The point here is that what constitutes a proper definition is a matter of
%judgment about whether a symbol sequence can easily be recognized by a human
%as invalid in some sense (for example, not a wff); if so, the symbol sequence
%can be appropriated for use by a definition in order to make the extended
%language more compact.  Metamath\index{Metamath} lacks the ability to make this
%judgment, since as far as Metamath is concerned the definition of a wff, for
%example, is arbitrary.  You define for Metamath how wffs\index{well-formed
%formula (wff)} are constructed according to your own preferred style.  The
%concept of a wff may not even exist in a given formal system\index{formal
%system}.  Metamath treats all definitions as if they were new axioms, and it
%is up to the human mathematician to judge whether the definition is ``proper''
%'\index{proper definition}\index{definition!proper} in some agreed-upon way.

What constitutes a definition\index{definition} versus\index{axiom vs.\
definition} an axiom\index{axiom} is sometimes arbitrary in mathematical
literature.  For example, the connectives $\vee$ ({\sc or}), $\wedge$
({\sc and}), and $\leftrightarrow$ (equivalent to) in propositional
calculus are usually considered defined symbols that can be used as
abbreviations for expressions containing the ``primitive'' connectives
$\rightarrow$ and $\neg$.  This is the way we treat them in the standard
logic and set theory database \texttt{set.mm}\index{set theory database
(\texttt{set.mm})}.  However, the first three connectives can also be
considered ``primitive,'' and axiom systems have been devised that treat
all of them as such.  For example,
\cite[p.~35]{Goodstein}\index{Goodstein, R. L.} presents one with 15
axioms, some of which in fact coincide with what we have chosen to call
definitions in \texttt{set.mm}.  In certain subsets of classical
propositional calculus, such as the intuitionist
fragment\index{intuitionism}, it can be shown that one cannot make do
with just $\rightarrow$ and $\neg$ but must treat additional connectives
as primitive in order for the system to make sense.\footnote{Two nice
systems that make the transition from intuitionistic and other weak
fragments to classical logic just by adding axioms are given in
\cite{Robinsont}\index{Robinson, T. Thacher}.}

\subsection{The Approach to Definitions in \texttt{set.mm}}

In set theory, recursive definitions define a newly introduced symbol in
terms of itself.
The justification of recursive definitions, using
several ``recursion theorems,'' is usually one of the first
sophisticated proofs a student encounters when learning set theory, and
there is a significant amount of implicit metalogic behind a recursive
definition even though the definition itself is typically simple to
state.

Metamath itself has no built-in technical limitation that prevents
multiple-part recursive definitions in the traditional textbook style.
However, because the recursive definition requires advanced metalogic
to justify, eliminating a recursive definition is very difficult and
often not even shown in textbooks.

\subsubsection{Direct definitions instead of recursive definitions}

It is, however, possible to substitute one kind of complexity
for another.  We can eliminate the need for metalogical justification by
defining the operation directly with an explicit (but complicated)
expression, then deriving the recursive definition directly as a
theorem, using a recursion theorem ``in reverse.''
The elimination
of a direct definition is a matter of simple mechanical substitution.
We do this in
\texttt{set.mm}, as follows.

In \texttt{set.mm} our goal was to introduce almost all definitions in
the form of two expressions connected by either $\leftrightarrow$ or
$=$, where the thing being defined does not appear on the right hand
side.  Quine calls this form ``a genuine or direct definition'' \cite[p.
174]{Quine}\index{Quine, Willard Van Orman}, which makes the definitions
very easy to eliminate and the metalogic\index{metalogic} needed to
justify them as simple as possible.
Put another way, we had a goal of being able to
eliminate all definitions with direct mechanical substitution and to
verify easily the soundness of the definitions.

\subsubsection{Example of direct definitions}

We achieved this goal in almost all cases in \texttt{set.mm}.
Sometimes this makes the definitions more complex and less
intuitive.
For example, the traditional way to define addition of
natural numbers is to define an operation called {\em
successor}\index{successor} (which means ``plus one'' and is denoted by
``${\rm suc}$''), then define addition recursively\index{recursive
definition} with the two definitions $n + 0 = n$ and $m + {\rm suc}\,n =
{\rm suc} (m + n)$.  Although this definition seems simple and obvious,
the method to eliminate the definition is not obvious:  in the second
part of the definition, addition is defined in terms of itself.  By
eliminating the definition, we don't mean repeatedly applying it to
specific $m$ and $n$ but rather showing the explicit, closed-form
set-theoretical expression that $m + n$ represents, that will work for
any $m$ and $n$ and that does not have a $+$ sign on its right-hand
side.  For a recursive definition like this not to be circular
(creative), there are some hidden, underlying assumptions we must make,
for example that the natural numbers have a certain kind of order.

In \texttt{set.mm} we chose to start with the direct (though complex and
nonintuitive) definition then derive from it the standard recursive
definition.
For example, the closed-form definition used in \texttt{set.mm}
for the addition operation on ordinals\index{ordinal
addition}\index{addition!of ordinals} (of which natural numbers are a
subset) is

\setbox\startprefix=\hbox{\tt \ \ df-oadd\ \$a\ }
\setbox\contprefix=\hbox{\tt \ \ \ \ \ \ \ \ \ \ \ \ \ }
\startm
\m{\vdash}\m{+_o}\m{=}\m{(}\m{x}\m{\in}\m{{\rm On}}\m{,}\m{y}\m{\in}\m{{\rm
On}}\m{\mapsto}\m{(}\m{{\rm rec}}\m{(}\m{(}\m{z}\m{\in}\m{{\rm
V}}\m{\mapsto}\m{{\rm suc}}\m{z}\m{)}\m{,}\m{x}\m{)}\m{`}\m{y}\m{)}\m{)}
\endm
\noindent which depends on ${\rm rec}$.

\subsubsection{Recursion operators}

The above definition of \texttt{df-oadd} depends on the definition of
${\rm rec}$, a ``recursion operator''\index{recursion operator} with
the definition \texttt{df-rdg}:

\setbox\startprefix=\hbox{\tt \ \ df-rdg\ \$a\ }
\setbox\contprefix=\hbox{\tt \ \ \ \ \ \ \ \ \ \ \ \ }
\startm
\m{\vdash}\m{{\rm
rec}}\m{(}\m{F}\m{,}\m{I}\m{)}\m{=}\m{\mathrm{recs}}\m{(}\m{(}\m{g}\m{\in}\m{{\rm
V}}\m{\mapsto}\m{{\rm if}}\m{(}\m{g}\m{=}\m{\varnothing}\m{,}\m{I}\m{,}\m{{\rm
if}}\m{(}\m{{\rm Lim}}\m{{\rm dom}}\m{g}\m{,}\m{\bigcup}\m{{\rm
ran}}\m{g}\m{,}\m{(}\m{F}\m{`}\m{(}\m{g}\m{`}\m{\bigcup}\m{{\rm
dom}}\m{g}\m{)}\m{)}\m{)}\m{)}\m{)}\m{)}
\endm

\noindent which can be further broken down with definitions shown in
Section~\ref{setdefinitions}.

This definition of ${\rm rec}$
defines a recursive definition generator on ${\rm On}$ (the class of ordinal
numbers) with characteristic function $F$ and initial value $I$.
This operation allows us to define, with
compact direct definitions, functions that are usually defined in
textbooks with recursive definitions.
The price paid with our approach
is the complexity of our ${\rm rec}$ operation
(especially when {\tt df-recs} that it is built on is also eliminated).
But once we get past this hurdle, definitions that would otherwise be
recursive become relatively simple, as in for example {\tt oav}, from
which we prove the recursive textbook definition as theorems {\tt oa0}, {\tt
oasuc}, and {\tt oalim} (with the help of theorems {\tt rdg0}, {\tt rdgsuc},
and {\tt rdglim2a}).  We can also restrict the ${\rm rec}$ operation to
define otherwise recursive functions on the natural numbers $\omega$; see {\tt
fr0g} and {\tt frsuc}.  Our ${\rm rec}$ operation apparently does not appear
in published literature, although closely related is Definition 25.2 of
[Quine] p. 177, which he uses to ``turn...a recursion into a genuine or
direct definition" (p. 174).  Note that the ${\rm if}$ operations (see
{\tt df-if}) select cases based on whether the domain of $g$ is zero, a
successor, or a limit ordinal.

An important use of this definition ${\rm rec}$ is in the recursive sequence
generator {\tt df-seq} on the natural numbers (as a subset of the
complex infinite sequences such as the factorial function {\tt df-fac} and
integer powers {\tt df-exp}).

The definition of ${\rm rec}$ depends on ${\rm recs}$.
New direct usage of the more powerful (and more primitive) ${\rm recs}$
construct is discouraged, but it is available when needed.
This
defines a function $\mathrm{recs} ( F )$ on ${\rm On}$, the class of ordinal
numbers, by transfinite recursion given a rule $F$ which sets the next
value given all values so far.
Unlike {\tt df-rdg} which restricts the
update rule to use only the previous value, this version allows the
update rule to use all previous values, which is why it is described
as ``strong,'' although it is actually more primitive.  See {\tt
recsfnon} and {\tt recsval} for the primary contract of this definition.
It is defined as:

\setbox\startprefix=\hbox{\tt \ \ df-recs\ \$a\ }
\setbox\contprefix=\hbox{\tt \ \ \ \ \ \ \ \ \ \ \ \ \ }
\startm
\m{\vdash}\m{\mathrm{recs}}\m{(}\m{F}\m{)}\m{=}\m{\bigcup}\m{\{}\m{f}\m{|}\m{\exists}\m{x}\m{\in}\m{{\rm
On}}\m{(}\m{f}\m{{\rm
Fn}}\m{x}\m{\wedge}\m{\forall}\m{y}\m{\in}\m{x}\m{(}\m{f}\m{`}\m{y}\m{)}\m{=}\m{(}\m{F}\m{`}\m{(}\m{f}\m{\restriction}\m{y}\m{)}\m{)}\m{)}\m{\}}
\endm

\subsubsection{Closing comments on direct definitions}

From these direct definitions the simpler, more
intuitive recursive definition is derived as a set of theorems.\index{natural
number}\index{addition}\index{recursive definition}\index{ordinal addition}
The end result is the same, but we completely eliminate the rather complex
metalogic that justifies the recursive definition.

Recursive definitions are often considered more efficient and intuitive than
direct ones once the metalogic has been learned or possibly just accepted as
correct.  However, it was felt that direct definition in \texttt{set.mm}
maximizes rigor by minimizing metalogic.  It can be eliminated effortlessly,
something that is difficult to do with a recursive definition.

Again, Metamath itself has no built-in technical limitation that prevents
multiple-part recursive definitions in the traditional textbook style.
Instead, our goal is to eliminate all definitions with
direct mechanical substitution and to verify easily the soundness of
definitions.

\subsection{Adding Constraints on Definitions}

The basic Metamath language and the Metamath program do
not have any built-in constraints on definitions, since they are just
\$a statements.

However, nothing prevents a verification system from verifying
additional rules to impose further limitations on definitions.
For example, the \texttt{mmj2}\index{mmj2} program
supports various kinds of
additional information comments (see section \ref{jcomment}).
One of their uses is to optionally verify additional constraints,
including constraints to verify that definitions meet certain
requirements.
These additional checks are required by the
continuous integration (CI)\index{continuous integration (CI)}
checks of the
\texttt{set.mm}\index{set theory database (\texttt{set.mm})}%
\index{Metamath Proof Explorer}
database.
This approach enables us to optionally impose additional requirements
on definitions if we wish, without necessarily imposing those rules on
all databases or requiring all verification systems to implement them.
In addition, this allows us to impose specialized constraints tailored
to one database while not requiring other systems to implement
those specialized constraints.

We impose two constraints on the
\texttt{set.mm}\index{set theory database (\texttt{set.mm})}%
\index{Metamath Proof Explorer} database
via the \texttt{mmj2}\index{mmj2} program that are worth discussing here:
a parse check and a definition soundness check.

% On February 11, 2019 8:32:32 PM EST, saueran@oregonstate.edu wrote:
% The following addition to the end of set.mm is accepted by the mmj2
% parser and definition checker and the metamath verifier(at least it was
% when I checked, you should check it too), and creates a contradiction by
% proving the theorem |- ph.
% ${
% wleftp $a wff ( ( ph ) $.
% wbothp $a wff ( ph ) $.
% df-leftp $a |- ( ( ( ph ) <-> -. ph ) $.
% df-bothp $a |- ( ( ph ) <-> ph ) $.
% anything $p |- ph $=
%   ( wbothp wn wi wleftp df-leftp biimpi df-bothp mpbir mpbi simplim ax-mp)
%   ABZAMACZDZCZMOEZOCQAEZNDZRNAFGSHIOFJMNKLAHJ $.
% $}
%
% This particular problem is countered by enabling, within mmj2,
% SetParser,mmj.verify.LRParser

First,
we enable a parse check in \texttt{mmj2} (through its
\texttt{SetParser} command) that requires that all new definitions
must \textit{not} create an ambiguous parse for a KLR(5) parser.
This prevents some errors such as definitions with imbalanced parentheses.

Second, we run a definition soundness check specific to
\texttt{set.mm} or databases similar to it.
(through the \texttt{definitionCheck} macro).
Some \texttt{\$a} statements (including all ax-* statemnets)
are excluded from these checks, as they will
always fail this simple check,
but they are appropriate for most definitions.
This check imposes a set of additional rules:

\begin{enumerate}

\item New definitions must be introduced using $=$ or $\leftrightarrow$.

\item No \texttt{\$a} statement introduced before this one is allowed to use the
symbol being defined in this definition, and the definition is not
allowed to use itself (except once, in the definiendum).

\item Every variable in the definiens should not be distinct

\item Every dummy variable in the definiendum
are required to be distinct from each other and from variables in
the definiendum.
To determine this, the system will look for a "justification" theorem
in the database, and if it is not there, attempt to briefly prove
$( \varphi \rightarrow \forall x \varphi )$  for each dummy variable x.

\item Every dummy variable should be a set variable,
unless there is a justification theorem available.

\item Every dummy variable must be bound
(if the system cannot determine this a justification theorem must be
provided).

\end{enumerate}

\subsection{Summary of Approach to Definitions}

In short, when being rigorous it turns out that
definitions can be subtle, sometimes requiring difficult
metatheorems to establish that they are not creative.

Instead of building such complications into the Metamath language itself,
the basic Metmath language and program simply treat traditional
axioms and definitions as the same kind of \texttt{\$a} statement.
We have then built various tools to enable people to
verify additional conditions as their creators believe is appropriate
for those specific databases, without complicating the Metamath foundations.

\chapter{The Metamath Program}\label{commands}

This chapter provides a reference manual for the
Metamath program.\index{Metamath!commands}

Current instructions for obtaining and installing the Metamath program
can be found at the \url{http://metamath.org} web site.
For Windows, there is a pre-compiled version called
\texttt{metamath.exe}.  For Unix, Linux, and Mac OS X
(which we will refer to collectively as ``Unix''), the Metamath program
can be compiled from its source code with the command
\begin{verbatim}
gcc *.c -o metamath
\end{verbatim}
using the \texttt{gcc} {\sc c} compiler available on those systems.

In the command syntax descriptions below, fields enclosed in square
brackets [\ ] are optional.  File names may be optionally enclosed in
single or double quotes.  This is useful if the file name contains
spaces or
slashes (\texttt{/}), such as in Unix path names, \index{Unix file
names}\index{file names!Unix} that might be confused with Metamath
command qualifiers.\index{Unix file names}\index{file names!Unix}

\section{Invoking Metamath}

Unix, Linux, and Mac OS X
have a command-line interface called the {\em
bash shell}.  (In Mac OS X, select the Terminal application from
Applications/Utilities.)  To invoke Metamath from the bash shell prompt,
assuming that the Metamath program is in the current directory, type
\begin{verbatim}
bash$ ./metamath
\end{verbatim}

To invoke Metamath from a Windows DOS or Command Prompt, assuming that
the Metamath program is in the current directory (or in a directory
included in the Path system environment variable), type
\begin{verbatim}
C:\metamath>metamath
\end{verbatim}

To use command-line arguments at invocation, the command-line arguments
should be a list of Metamath commands, surrounded by quotes if they
contain spaces.  In Windows, the surrounding quotes must be double (not
single) quotes.  For example, to read the database file \texttt{set.mm},
verify all proofs, and exit the program, type (under Unix)
\begin{verbatim}
bash$ ./metamath 'read set.mm' 'verify proof *' exit
\end{verbatim}
Note that in Unix, any directory path with \texttt{/}'s must be
surrounded by quotes so Metamath will not interpret the \texttt{/} as a
command qualifier.  So if \texttt{set.mm} is in the \texttt{/tmp}
directory, use for the above example
\begin{verbatim}
bash$ ./metamath 'read "/tmp/set.mm"' 'verify proof *' exit
\end{verbatim}

For convenience, if the command-line has one argument and no spaces in
the argument, the command is implicitly assumed to be \texttt{read}.  In
this one special case, \texttt{/}'s are not interpreted as command
qualifiers, so you don't need quotes around a Unix file name.  Thus
\begin{verbatim}
bash$ ./metamath /tmp/set.mm
\end{verbatim}
and
\begin{verbatim}
bash$ ./metamath "read '/tmp/set.mm'"
\end{verbatim}
are equivalent.


\section{Controlling Metamath}

The Metamath program was first developed on a {\sc vax/vms} system, and
some aspects of its command line behavior reflect this heritage.
Hopefully you will find it reasonably user-friendly once you get used to
it.

Each command line is a sequence of English-like words separated by
spaces, as in \texttt{show settings}.  Command words are not case
sensitive, and only as many letters are needed as are necessary to
eliminate ambiguity; for example, \texttt{sh se} would work for the
command \texttt{show settings}.  In some cases arguments such as file
names, statement labels, or symbol names are required; these are
case-sensitive (although file names may not be on some operating
systems).

A command line is entered by typing it in then pressing the {\em return}
({\em enter}) key.  To find out what commands are available, type
\texttt{?} at the \texttt{MM>} prompt.  To find out the choices at any
point in a command, press {\em return} and you will be prompted for
them.  The default choice (the one selected if you just press {\em
return}) is shown in brackets (\texttt{<>}).

You may also type \texttt{?} in place of a command word to force
Metamath to tell you what the choices are.  The \texttt{?} method won't
work, though, if a non-keyword argument such as a file name is expected
at that point, because the program will think that \texttt{?} is the
value of the argument.

Some commands have one or more optional qualifiers which modify the
behavior of the command.  Qualifiers are preceded by a slash
(\texttt{/}), such as in \texttt{read set.mm / verify}.  Spaces are
optional around the \texttt{/}.  If you need to use a space or
slash in a command
argument, as in a Unix file name, put single or double quotes around the
command argument.

The \texttt{open log} command will save everything you see on the
screen and is useful to help you recover should something go wrong in a
proof, or if you want to document a bug.

If a command responds with more than a screenful, you will be
prompted to \texttt{<return> to continue, Q to quit, or S to scroll to
end}.  \texttt{Q} or \texttt{q} (not case-sensitive) will complete the
command internally but will suppress further output until the next
\texttt{MM>} prompt.  \texttt{s} will suppress further pausing until the
next \texttt{MM>} prompt.  After the first screen, you are also
presented with the choice of \texttt{b} to go back a screenful.  Note
that \texttt{b} may also be entered at the \texttt{MM>} prompt
immediately after a command to scroll back through the output of that
command.

A command line enclosed in quotes is executed by your operating system.
See Section~\ref{oscmd}.

{\em Warning:} Pressing {\sc ctrl-c} will abort the Metamath program
unconditionally.  This means any unsaved work will be lost.


\subsection{\texttt{exit} Command}\index{\texttt{exit} command}

Syntax:  \texttt{exit} [\texttt{/force}]

This command exits from Metamath.  If there have been changes to the
source with the \texttt{save proof} or \texttt{save new{\char`\_}proof}
commands, you will be given an opportunity to \texttt{write source} to
permanently save the changes.

In Proof Assistant\index{Proof Assistant} mode, the \texttt{exit} command will
return to the \verb/MM>/ prompt. If there were changes to the proof, you will
be given an opportunity to \texttt{save new{\char`\_}proof}.

The \texttt{quit} command is a synonym for \texttt{exit}.

Optional qualifier:
    \texttt{/force} - Do not prompt if changes were not saved.  This qualifier is
        useful in \texttt{submit} command files (Section~\ref{sbmt})
        to ensure predictable behavior.





\subsection{\texttt{open log} Command}\index{\texttt{open log} command}
Syntax:  \texttt{open log} {\em file-name}

This command will open a log file that will store everything you see on
the screen.  It is useful to help recovery from a mistake in a long Proof
Assistant session, or to document bugs.\index{Metamath!bugs}

The log file can be closed with \texttt{close log}.  It will automatically be
closed upon exiting Metamath.



\subsection{\texttt{close log} Command}\index{\texttt{close log} command}
Syntax:  \texttt{close log}

The \texttt{close log} command closes a log file if one is open.  See
also \texttt{open log}.




\subsection{\texttt{submit} Command}\index{\texttt{submit} command}\label{sbmt}
Syntax:  \texttt{submit} {\em filename}

This command causes further command lines to be taken from the specified
file.  Note that any line beginning with an exclamation point (\texttt{!}) is
treated as a comment (i.e.\ ignored).  Also note that the scrolling
of the screen output is continuous, so you may want to open a log file
(see \texttt{open log}) to record the results that fly by on the screen.
After the lines in the file are exhausted, Metamath returns to its
normal user interface mode.

The \texttt{submit} command can be called recursively (i.e. from inside
of a \texttt{submit} command file).


Optional command qualifier:

    \texttt{/silent} - suppresses the screen output but still
        records the output in a log file if one is open.


\subsection{\texttt{erase} Command}\index{\texttt{erase} command}
Syntax:  \texttt{erase}

This command will reset Metamath to its starting state, deleting any
data\-base that was \texttt{read} in.
 If there have been changes to the
source with the \texttt{save proof} or \texttt{save new{\char`\_}proof}
commands, you will be given an opportunity to \texttt{write source} to
permanently save the changes.



\subsection{\texttt{set echo} Command}\index{\texttt{set echo} command}
Syntax:  \texttt{set echo on} or \texttt{set echo off}

The \texttt{set echo on} command will cause command lines to be echoed with any
abbreviations expanded.  While learning the Metamath commands, this
feature will show you the exact command that your abbreviated input
corresponds to.



\subsection{\texttt{set scroll} Command}\index{\texttt{set scroll} command}
Syntax:  \texttt{set scroll prompted} or \texttt{set scroll continuous}

The Metamath command line interface starts off in the \texttt{prompted} mode,
which means that you will be prompted to continue or quit after each
full screen in a long listing.  In \texttt{continuous} mode, long listings will be
scrolled without pausing.

% LaTeX bug? (1) \texttt{\_} puts out different character than
% \texttt{{\char`\_}}
%  = \verb$_$  (2) \texttt{{\char`\_}} puts out garbage in \subsection
%  argument
\subsection{\texttt{set width} Command}\index{\texttt{set
width} command}
Syntax:  \texttt{set width} {\em number}

Metamath assumes the width of your screen is 79 characters (chosen
because the Command Prompt in Windows XP has a wrapping bug at column
80).  If your screen is wider or narrower, this command allows you to
change this default screen width.  A larger width is advantageous for
logging proofs to an output file to be printed on a wide printer.  A
smaller width may be necessary on some terminals; in this case, the
wrapping of the information messages may sometimes seem somewhat
unnatural, however.  In \LaTeX\index{latex@{\LaTeX}!characters per line},
there is normally a maximum of 61 characters per line with typewriter
font.  (The examples in this book were produced with 61 characters per
line.)

\subsection{\texttt{set height} Command}\index{\texttt{set
height} command}
Syntax:  \texttt{set height} {\em number}

Metamath assumes your screen height is 24 lines of characters.  If your
screen is taller or shorter, this command lets you to change the number
of lines at which the display pauses and prompts you to continue.

\subsection{\texttt{beep} Command}\index{\texttt{beep} command}

Syntax:  \texttt{beep}

This command will produce a beep.  By typing it ahead after a
long-running command has started, it will alert you that the command is
finished.  For convenience, \texttt{b} is an abbreviation for
\texttt{beep}.

Note:  If \texttt{b} is typed at the \texttt{MM>} prompt immediately
after the end of a multiple-page display paged with ``\texttt{Press
<return> for more}...'' prompts, then the \texttt{b} will back up to the
previous page rather than perform the \texttt{beep} command.
In that case you must type the unabbreviated \texttt{beep} form
of the command.

\subsection{\texttt{more} Command}\index{\texttt{more} command}

Syntax:  \texttt{more} {\em filename}

This command will display the contents of an {\sc ascii} file on your
screen.  (This command is provided for convenience but is not very
powerful.  See Section~\ref{oscmd} to invoke your operating system's
command to do this, such as the \texttt{more} command in Unix.)

\subsection{Operating System Commands}\index{operating system
command}\label{oscmd}

A line enclosed in single or double quotes will be executed by your
computer's operating system if it has a command line interface.  For
example, on a {\sc vax/vms} system,
\verb/MM> 'dir'/
will print disk directory contents.  Note that this feature will not
work on the Macintosh prior to Mac OS X, which does not have a command
line interface.

For your convenience, the trailing quote is optional.

\subsection{Size Limitations in Metamath}

In general, there are no fixed, predefined limits\index{Metamath!memory
limits} on how many labels, tokens\index{token}, statements, etc.\ that
you may have in a database file.  The Metamath program uses 32-bit
variables (64-bit on 64-bit CPUs) as indices for almost all internal
arrays, which are allocated dynamically as needed.



\section{Reading and Writing Files}

The following commands create new files:  the \texttt{open} commands;
the \texttt{write} commands; the \texttt{/html},
\texttt{/alt{\char`\_}html}, \texttt{/brief{\char`\_}html},
\texttt{/brief{\char`\_}alt{\char`\_}html} qualifiers of \texttt{show
statement}, and \texttt{midi}.  The following commands append to files
previously opened:  the \texttt{/tex} qualifier of \texttt{show proof}
and \texttt{show new{\char`\_}proof}; the \texttt{/tex} and
\texttt{/simple{\char`\_}tex} qualifiers of \texttt{show statement}; the
\texttt{close} commands; and all screen dialog between \texttt{open log}
and \texttt{close log}.

The commands that create new files will not overwrite an existing {\em
filename} but will rename the existing one to {\em
filename}\texttt{{\char`\~}1}.  An existing {\em
filename}\texttt{{\char`\~}1} is renamed {\em
filename}\texttt{{\char`\~}2}, etc.\ up to {\em
filename}\texttt{{\char`\~}9}.  An existing {\em
filename}\texttt{{\char`\~}9} is deleted.  This makes recovery from
mistakes easier but also will clutter up your directory, so occasionally
you may want to clean up (delete) these \texttt{{\char`\~}}$n$ files.


\subsection{\texttt{read} Command}\index{\texttt{read} command}
Syntax:  \texttt{read} {\em file-name} [\texttt{/verify}]

This command will read in a Metamath language source file and any included
files.  Normally it will be the first thing you do when entering Metamath.
Statement syntax is checked, but proof syntax is not checked.
Note that the file name may be enclosed in single or double quotes;
this is useful if the file name contains slashes, as might be the case
under Unix.

If you are getting an ``\texttt{?Expected VERIFY}'' error
when trying to read a Unix file name with slashes, you probably haven't
quoted it.\index{Unix file names}\index{file names!Unix}

If you are prompted for the file name (by pressing {\em return}
 after \texttt{read})
you should {\em not} put quotes around it, even if it is a Unix file name
with slashes.

Optional command qualifier:

    \texttt{/verify} - Verify all proofs as the database is read in.  This
         qualifier will slow down reading in the file.  See \texttt{verify
         proof} for more information on file error-checking.

See also \texttt{erase}.



\subsection{\texttt{write source} Command}\index{\texttt{write source} command}
Syntax:  \texttt{write source} {\em filename}
[\texttt{/rewrap}]
[\texttt{/split}]
% TeX doesn't handle this long line with tt text very well,
% so force a line break here.
[\texttt{/keep\_includes}] {\\}
[\texttt{/no\_versioning}]

This command will write the contents of a Metamath\index{database}
database into a file.\index{source file}

Optional command qualifiers:

\texttt{/rewrap} -
Reformats statements and comments according to the
convention used in the set.mm database.
It unwraps the
lines in the comment before each \texttt{\$a} and \texttt{\$p} statement,
then it
rewraps the line.  You should compare the output to the original
to make sure that the desired effect results; if not, go back to
the original source.  The wrapped line length honors the
\texttt{set width}
parameter currently in effect.  Note:  Text
enclosed in \texttt{<HTML>}...\texttt{</HTML>} tags is not modified by the
\texttt{/rewrap} qualifier.
Proofs themselves are not reformatted;
use \texttt{save proof * / compressed} to do that.
An isolated tilde (\~{}) is always kept on the same line as the following
symbol, so you can find all comment references to a symbol by
searching for \~{} followed by a space and that symbol
(this is useful for finding cross-references).
Incidentally, \texttt{save proof} also honors the \texttt{set width}
parameter currently in effect.

\texttt{/split} - Files included in the source using the expression
\$[ \textit{inclfile} \$] will be
written into separate files instead of being included in a single output
file.  The name of each separately written included file will be the
\textit{inclfile} argument of its inclusion command.

\texttt{/keep\_includes} - If a source file has includes but is written as a
single file by omitting \texttt{/split}, by default the included files will
be deleted (actually just renamed with a \char`\~1 suffix unless
\texttt{/no\_versioning} is specified) to prevent the possibly confusing
source duplication in both the output file and the included file.
The \texttt{/keep\_includes} qualifier will prevent this deletion.

\texttt{/no\_versioning} - Backup files suffixed with \char`\~1 are not created.


\section{Showing Status and Statements}



\subsection{\texttt{show settings} Command}\index{\texttt{show settings} command}
Syntax:  \texttt{show settings}

This command shows the state of various parameters.

\subsection{\texttt{show memory} Command}\index{\texttt{show memory} command}
Syntax:  \texttt{show memory}

This command shows the available memory left.  It is not meaningful
on most modern operating systems,
which have virtual memory.\index{Metamath!memory usage}


\subsection{\texttt{show labels} Command}\index{\texttt{show labels} command}
Syntax:  \texttt{show labels} {\em label-match} [\texttt{/all}]
   [\texttt{/linear}]

This command shows the labels of \texttt{\$a} and \texttt{\$p}
statements that match {\em label-match}.  A \verb$*$ in {label-match}
matches zero or more characters.  For example, \verb$*abc*def$ will match all
labels containing \verb$abc$ and ending with \verb$def$.

Optional command qualifiers:

   \texttt{/all} - Include matches for \texttt{\$e} and \texttt{\$f}
   statement labels.

   \texttt{/linear} - Display only one label per line.  This can be useful for
       building scripts in conjunction with the utilities under the
       \texttt{tools}\index{\texttt{tools} command} command.



\subsection{\texttt{show statement} Command}\index{\texttt{show statement} command}
Syntax:  \texttt{show statement} {\em label-match} [{\em qualifiers} (see below)]

This command provides information about a statement.  Only statements
that have labels (\texttt{\$f}\index{\texttt{\$f} statement},
\texttt{\$e}\index{\texttt{\$e} statement},
\texttt{\$a}\index{\texttt{\$a} statement}, and
\texttt{\$p}\index{\texttt{\$p} statement}) may be specified.
If {\em label-match}
contains wildcard (\verb$*$) characters, all matching statements will be
displayed in the order they occur in the database.

Optional qualifiers (only one qualifier at a time is allowed):

    \texttt{/comment} - This qualifier includes the comment that immediately
       precedes the statement.

    \texttt{/full} - Show complete information about each statement,
       and show all
       statements matching {\em label} (including \texttt{\$e}
       and \texttt{\$f} statements).

    \texttt{/tex} - This qualifier will write the statement information to the
       \LaTeX\ file previously opened with \texttt{open tex}.  See
       Section~\ref{texout}.

    \texttt{/simple{\char`\_}tex} - The same as \texttt{/tex}, except that
       \LaTeX\ macros are not used for formatting equations, allowing easier
       manual edits of the output for slide presentations, etc.

    \texttt{/html}\index{html generation@{\sc html} generation},
       \texttt{/alt{\char`\_}html}, \texttt{/brief{\char`\_}html},
       \texttt{/brief{\char`\_}alt{\char`\_}html} -
       These qualifiers invoke a special mode of
       \texttt{show statement} that
       creates a web page for the statement.  They may not be used with
       any other qualifier.  See Section~\ref{htmlout} or
       \texttt{help html} in the program.


\subsection{\texttt{search} Command}\index{\texttt{search} command}
Syntax:  search {\em label-match}
\texttt{"}{\em symbol-match}{\tt}" [\texttt{/all}] [\texttt{/comments}]
[\texttt{/join}]

This command searches all \texttt{\$a} and \texttt{\$p} statements
matching {\em label-match} for occurrences of {\em symbol-match}.  A
\verb@*@ in {\em label-match} matches any label character.  A \verb@$*@
in {\em symbol-match} matches any sequence of symbols.  The symbols in
{\em symbol-match} must be separated by white space.  The quotes
surrounding {\em symbol-match} may be single or double quotes.  For
example, \texttt{search b}\verb@* "-> $* ch"@ will list all statements
whose labels begin with \texttt{b} and contain the symbols \verb@->@ and
\texttt{ch} surrounding any symbol sequence (including no symbol
sequence).  The wildcards \texttt{?} and \texttt{\$?} are also available
to match individual characters in labels and symbols respectively; see
\texttt{help search} in the Metamath program for details on their usage.

Optional command qualifiers:

    \texttt{/all} - Also search \texttt{\$e} and \texttt{\$f} statements.

    \texttt{/comments} - Search the comment that immediately precedes each
        label-matched statement for {\em symbol-match}.  In this case
        {\em symbol-match} is an arbitrary, non-case-sensitive character
        string.  Quotes around {\em symbol-match} are optional if there
        is no ambiguity.

    \texttt{/join} - In the case of a \texttt{\$a} or \texttt{\$p} statement,
	prepend its \texttt{\$e}
	hypotheses for searching. The
	\texttt{/join} qualifier has no effect in \texttt{/comments} mode.

\section{Displaying and Verifying Proofs}


\subsection{\texttt{show proof} Command}\index{\texttt{show proof} command}
Syntax:  \texttt{show proof} {\em label-match} [{\em qualifiers} (see below)]

This command displays the proof of the specified
\texttt{\$p}\index{\texttt{\$p} statement} statement in various formats.
The {\em label-match} may contain wildcard (\verb@$*@) characters to match
multiple statements.  Without any qualifiers, only the logical steps
will be shown (i.e.\ syntax construction steps will be omitted), in an
indented format.

Most of the time, you will use
    \texttt{show proof} {\em label}
to see just the proof steps corresponding to logical inferences.

Optional command qualifiers:

    \texttt{/essential} - The proof tree is trimmed of all
        \texttt{\$f}\index{\texttt{\$f} statement} hypotheses before
        being displayed.  (This is the default, and it is redundant to
        specify it.)

    \texttt{/all} - the proof tree is not trimmed of all \texttt{\$f} hypotheses before
        being displayed.  \texttt{/essential} and \texttt{/all} are mutually exclusive.

    \texttt{/from{\char`\_}step} {\em step} - The display starts at the specified
        step.  If
        this qualifier is omitted, the display starts at the first step.

    \texttt{/to{\char`\_}step} {\em step} - The display ends at the specified
        step.  If this
        qualifier is omitted, the display ends at the last step.

    \texttt{/tree{\char`\_}depth} {\em number} - Only
         steps at less than the specified proof
        tree depth are displayed.  Sometimes useful for obtaining an overview of
        the proof.

    \texttt{/reverse} - The steps are displayed in reverse order.

    \texttt{/renumber} - When used with \texttt{/essential}, the steps are renumbered
        to correspond only to the essential steps.

    \texttt{/tex} - The proof is converted to \LaTeX\ and\index{latex@{\LaTeX}}
        stored in the file opened
        with \texttt{open tex}.  See Section~\ref{texout} or
        \texttt{help tex} in the program.

    \texttt{/lemmon} - The proof is displayed in a non-indented format known
        as Lemmon style, with explicit previous step number references.
        If this qualifier is omitted, steps are indented in a tree format.

    \texttt{/start{\char`\_}column} {\em number} - Overrides the default column
        (16)
        at which the formula display starts in a Lemmon-style display.  May be
        used only in conjunction with \texttt{/lemmon}.

    \texttt{/normal} - The proof is displayed in normal format suitable for
        inclusion in a Metamath source file.  May not be used with any other
        qualifier.

    \texttt{/compressed} - The proof is displayed in compressed format
        suitable for inclusion in a Metamath source file.  May not be used with
        any other qualifier.

    \texttt{/statement{\char`\_}summary} - Summarizes all statements (like a
        brief \texttt{show statement})
        used by the proof.  It may not be used with any other qualifier
        except \texttt{/essential}.

    \texttt{/detailed{\char`\_}step} {\em step} - Shows the details of what is
        happening at
        a specific proof step.  May not be used with any other qualifier.
        The {\em step} is the step number shown when displaying a
        proof without the \texttt{/renumber} qualifier.


\subsection{\texttt{show usage} Command}\index{\texttt{show usage} command}
Syntax:  \texttt{show usage} {\em label-match} [\texttt{/recursive}]

This command lists the statements whose proofs make direct reference to
the statement specified.

Optional command qualifier:

    \texttt{/recursive} - Also include statements whose proofs ultimately
        depend on the statement specified.



\subsection{\texttt{show trace\_back} Command}\index{\texttt{show
       trace{\char`\_}back} command}
Syntax:  \texttt{show trace{\char`\_}back} {\em label-match} [\texttt{/essential}] [\texttt{/axioms}]
    [\texttt{/tree}] [\texttt{/depth} {\em number}]

This command lists all statements that the proof of the \texttt{\$p}
statement(s) specified by {\em label-match} depends on.

Optional command qualifiers:

    \texttt{/essential} - Restrict the trace-back to \texttt{\$e}
        \index{\texttt{\$e} statement} hypotheses of proof trees.

    \texttt{/axioms} - List only the axioms that the proof ultimately depends on.

    \texttt{/tree} - Display the trace-back in an indented tree format.

    \texttt{/depth} {\em number} - Restrict the \texttt{/tree} trace-back to the
        specified indentation depth.

    \texttt{/count{\char`\_}steps} - Count the number of steps the proof has
       all the way back to axioms.  If \texttt{/essential} is specified,
       expansions of variable-type hypotheses (syntax constructions) are not counted.

\subsection{\texttt{verify proof} Command}\index{\texttt{verify proof} command}
Syntax:  \texttt{verify proof} {\em label-match} [\texttt{/syntax{\char`\_}only}]

This command verifies the proofs of the specified statements.  {\em
label-match} may contain wild card characters (\texttt{*}) to verify
more than one proof; for example \verb/*abc*def/ will match all labels
containing \texttt{abc} and ending with \texttt{def}.
The command \texttt{verify proof *} will verify all proofs in the database.

Optional command qualifier:

    \texttt{/syntax{\char`\_}only} - This qualifier will perform a check of syntax
        and RPN
        stack violations only.  It will not verify that the proof is
        correct.  This qualifier is useful for quickly determining which
        proofs are incomplete (i.e.\ are under development and have \texttt{?}'s
        in them).

{\em Note:} \texttt{read}, followed by \texttt{verify proof *}, will ensure
 the database is free
from errors in the Metamath language but will not check the markup notation
in comments.
You can also check the markup notation by running \texttt{verify markup *}
as discussed in Section~\ref{verifymarkup}; see also the discussion
on generating {\sc HTML} in Section~\ref{htmlout}.

\subsection{\texttt{verify markup} Command}\index{\texttt{verify markup} command}\label{verifymarkup}
Syntax:  \texttt{verify markup} {\em label-match}
[\texttt{/date{\char`\_}skip}]
[\texttt{/top{\char`\_}date{\char`\_}skip}] {\\}
[\texttt{/file{\char`\_}skip}]
[\texttt{/verbose}]

This command checks comment markup and other informal conventions we have
adopted.  It error-checks the latexdef, htmldef, and althtmldef statements
in the \texttt{\$t} statement of a Metamath source file.\index{error checking}
It error-checks any \texttt{`}...\texttt{`},
\texttt{\char`\~}~\textit{label},
and bibliographic markups in statement descriptions.
It checks that
\texttt{\$p} and \texttt{\$a} statements
have the same content when their labels start with
``ax'' and ``ax-'' respectively but are otherwise identical, for example
ax4 and ax-4.
It also verifies the date consistency of ``(Contributed by...),''
``(Revised by...),'' and ``(Proof shortened by...)'' tags in the comment
above each \texttt{\$a} and \texttt{\$p} statement.

Optional command qualifiers:

    \texttt{/date{\char`\_}skip} - This qualifier will
        skip date consistency checking,
        which is usually not required for databases other than
	\texttt{set.mm}.

    \texttt{/top{\char`\_}date{\char`\_}skip} - This qualifier will check date consistency except
        that the version date at the top of the database file will not
        be checked.  Only one of
        \texttt{/date{\char`\_}skip} and
        \texttt{/top{\char`\_}date{\char`\_}skip} may be
        specified.

    \texttt{/file{\char`\_}skip} - This qualifier will skip checks that require
        external files to be present, such as checking GIF existence and
        bibliographic links to mmset.html or equivalent.  It is useful
        for doing a quick check from a directory without these files.

    \texttt{/verbose} - Provides more information.  Currently it provides a list
        of axXXX vs. ax-XXX matches.

\subsection{\texttt{save proof} Command}\index{\texttt{save proof} command}
Syntax:  \texttt{save proof} {\em label-match} [\texttt{/normal}]
   [\texttt{/compressed}]

The \texttt{save proof} command will reformat a proof in one of two formats and
replace the existing proof in the source buffer\index{source
buffer}.  It is useful for
converting between proof formats.  Note that a proof will not be
permanently saved until a \texttt{write source} command is issued.

Optional command qualifiers:

    \texttt{/normal} - The proof is saved in the normal format (i.e., as a
        sequence
        of labels, which is the defined format of the basic Metamath
        language).\index{basic language}  This is the default format that
        is used if a qualifier
        is omitted.

    \texttt{/compressed} - The proof is saved in the compressed format which
        reduces storage requirements for a database.
        See Appendix~\ref{compressed}.




\section{Creating Proofs}\label{pfcommands}\index{Proof Assistant}

Before using the Proof Assistant, you must add a \texttt{\$p} to your
source file (using a text editor) containing the statement you want to
prove.  Its proof should consist of a single \texttt{?}, meaning
``unknown step.''  Example:
\begin{verbatim}
equid $p x = x $= ? $.
\end{verbatim}

To enter the Proof assistant, type \texttt{prove} {\em label}, e.g.
\texttt{prove equid}.  Metamath will respond with the \texttt{MM-PA>}
prompt.

Proofs are created working backwards from the statement being proved,
primarily using a series of \texttt{assign} commands.  A proof is
complete when all steps are assigned to statements and all steps are
unified and completely known.  During the creation of a proof, Metamath
will allow only operations that are legal based on what is known up to
that point.  For example, it will not allow an \texttt{assign} of a
statement that cannot be unified with the unknown proof step being
assigned.

{\em Important:}
The Proof Assistant is
{\em not} a tool to help you discover proofs.  It is just a tool to help
you add them to the database.  For a tutorial read
Section~\ref{frstprf}.
To practice using the Proof Assistant, you may
want to \texttt{prove} an existing theorem, then delete all steps with
\texttt{delete all}, then re-create it with the Proof Assistant while
looking at its proof display (before deletion).
You might want to figure out your first few proofs completely
and write them down by hand, before using the Proof Assistant, though
not everyone finds that effective.

{\em Important:}
The \texttt{undo} command if very helpful when entering a proof, because
it allows you to undo a previously-entered step.
In addition, we suggest that you
keep track of your work with a log file (\texttt{open
log}) and save it frequently (\texttt{save new{\char`\_}proof},
\texttt{write source}).
You can use \texttt{delete} to reverse an \texttt{assign}.
You can also do \texttt{delete floating{\char`\_}hypotheses}, then
\texttt{initialize all}, then \texttt{unify all /interactive} to
reinitialize bad unifications made accidentally or by bad
\texttt{assign}s.  You cannot reverse a \texttt{delete} except by
a relevant \texttt{undo} or using
\texttt{exit /force} then reentering the Proof Assistant to recover from
the last \texttt{save new{\char`\_}proof}.

The following commands are available in the Proof Assistant (at the
\texttt{MM-PA>} prompt) to help you create your proof.  See the
individual commands for more detail.

\begin{itemize}
\item[]
    \texttt{show new{\char`\_}proof} [\texttt{/all},...] - Displays the
        proof in progress.  You will use this command a lot; see \texttt{help
        show new{\char`\_}proof} to become familiar with its qualifiers.  The
        qualifiers \texttt{/unknown} and \texttt{/not{\char`\_}unified} are
        useful for seeing the work remaining to be done.  The combination
        \texttt{/all/unknown} is useful for identifying dummy variables that must be
        assigned, or attempts to use illegal syntax, when \texttt{improve all}
        is unable to complete the syntax constructions.  Unknown variables are
        shown as \texttt{\$1}, \texttt{\$2},...
\item[]
    \texttt{assign} {\em step} {\em label} - Assigns an unknown {\em step}
        number with the statement
        specified by {\em label}.
\item[]
    \texttt{let variable} {\em variable}
        \texttt{= "}{\em symbol sequence}\texttt{"}
          - Forces a symbol
        sequence to replace an unknown variable (such as \texttt{\$1}) in a proof.
        It is useful
        for helping difficult unifications, and it is necessary when you have
        dummy variables that eventually must be assigned a name.
\item[]
    \texttt{let step} {\em step} \texttt{= "}{\em symbol sequence}\texttt{"} -
          Forces a symbol sequence
        to replace the contents of a proof step, provided it can be
        unified with the existing step contents.  (I rarely use this.)
\item[]
    \texttt{unify step} {\em step} (or \texttt{unify all}) - Unifies
        the source and target of
        a step.  If you specify a specific step, you will be prompted
        to select among the unifications that are possible.  If you
        specify \texttt{all}, all steps with unique unifications, but only
        those steps, will be
        unified.  \texttt{unify all /interactive} goes through all non-unified
        steps.
\item[]
    \texttt{initialize} {\em step} (or \texttt{all}) - De-unifies the target and source of
        a step (or all steps), as well as the hypotheses of the source,
        and makes all variables in the source unknown.  Useful to recover from
        an \texttt{assign} or \texttt{let} mistake that
        resulted in incorrect unifications.
\item[]
    \texttt{delete} {\em step} (or \texttt{all} or \texttt{floating{\char`\_}hypotheses}) -
        Deletes the specified
        step(s).  \texttt{delete floating{\char`\_}hypotheses}, then \texttt{initialize all}, then
        \texttt{unify all /interactive} is useful for recovering from mistakes
        where incorrect unifications assigned wrong math symbol strings to
        variables.
\item[]
    \texttt{improve} {\em step} (or \texttt{all}) -
      Automatically creates a proof for steps (with no unknown variables)
      whose proof requires no statements with \texttt{\$e} hypotheses.  Useful
      for filling in proofs of \texttt{\$f} hypotheses.  The \texttt{/depth}
      qualifier will also try statements whose \texttt{\$e} hypotheses contain
      no new variables.  {\em Warning:} Save your work (with \texttt{save
      new{\char`\_}proof} then \texttt{write source}) before using
      \texttt{/depth = 2} or greater, since the search time grows
      exponentially and may never terminate in a reasonable time, and you
      cannot interrupt the search.  I have found that it is rare for
      \texttt{/depth = 3} or greater to be useful.
 \item[]
    \texttt{save new{\char`\_}proof} - Saves the proof in progress in the program's
        internal database buffer.  To save it permanently into the database file,
        use \texttt{write source} after
        \texttt{save new{\char`\_}proof}.  To revert to the last
        \texttt{save new{\char`\_}proof},
        \texttt{exit /force} from the Proof Assistant then re-enter the Proof
        Assistant.
 \item[]
    \texttt{match step} {\em step} (or \texttt{match all}) - Shows what
        statements are
        possibilities for the \texttt{assign} statement. (This command
        is not very
        useful in its present form and hopefully will be improved
        eventually.  In the meantime, use the \texttt{search} statement for
        candidates matching specific math token combinations.)
 \item[]
 \texttt{minimize{\char`\_}with}\index{\texttt{minimize{\char`\_}with} command}
% 3/10/07 Note: line-breaking the above results in duplicate index entries
     - After a proof is complete, this command will attempt
        to match other database theorems to the proof to see if the proof
        size can be reduced as a result.  See \texttt{help
        minimize{\char`\_}with} in the
        Metamath program for its usage.
 \item[]
 \texttt{undo}\index{\texttt{undo} command}
    - Undo the effect of a proof-changing command (all but the
      \texttt{show} and \texttt{save} commands above).
 \item[]
 \texttt{redo}\index{\texttt{redo} command}
    - Reverse the previous \texttt{undo}.
\end{itemize}

The following commands set parameters that may be relevant to your proof.
Consult the individual \texttt{help set}... commands.
\begin{itemize}
   \item[] \texttt{set unification{\char`\_}timeout}
 \item[]
    \texttt{set search{\char`\_}limit}
  \item[]
    \texttt{set empty{\char`\_}substitution} - note that default is \texttt{off}
\end{itemize}

Type \texttt{exit} to exit the \texttt{MM-PA>}
 prompt and get back to the \texttt{MM>} prompt.
Another \texttt{exit} will then get you out of Metamath.



\subsection{\texttt{prove} Command}\index{\texttt{prove} command}
Syntax:  \texttt{prove} {\em label}

This command will enter the Proof Assistant\index{Proof Assistant}, which will
allow you to create or edit the proof of the specified statement.
The command-line prompt will change from \texttt{MM>} to \texttt{MM-PA>}.

Note:  In the present version (0.177) of
Metamath\index{Metamath!limitations of version 0.177}, the Proof
Assistant does not verify that \texttt{\$d}\index{\texttt{\$d}
statement} restrictions are met as a proof is being built.  After you
have completed a proof, you should type \texttt{save new{\char`\_}proof}
followed by \texttt{verify proof} {\em label} (where {\em label} is the
statement you are proving with the \texttt{prove} command) to verify the
\texttt{\$d} restrictions.

See also: \texttt{exit}

\subsection{\texttt{set unification\_timeout} Command}\index{\texttt{set
unification{\char`\_}timeout} command}
Syntax:  \texttt{set unification{\char`\_}timeout} {\em number}

(This command is available outside the Proof Assistant but affects the
Proof Assistant\index{Proof Assistant} only.)

Sometimes the Proof Assistant will inform you that a unification
time-out occurred.  This may happen when you try to \texttt{unify}
formulas with many temporary variables\index{temporary variable}
(\texttt{\$1}, \texttt{\$2}, etc.), since the time to compute all possible
unifications may grow exponentially with the number of variables.  If
you want Metamath to try harder (and you're willing to wait longer) you
may increase this parameter.  \texttt{show settings} will show you the
current value.

\subsection{\texttt{set empty\_substitution} Command}\index{\texttt{set
empty{\char`\_}substitution} command}
% These long names can't break well in narrow mode, and even "sloppy"
% is not enough. Work around this by not demanding justification.
\begin{flushleft}
Syntax:  \texttt{set empty{\char`\_}substitution on} or \texttt{set
empty{\char`\_}substitution off}
\end{flushleft}

(This command is available outside the Proof Assistant but affects the
Proof Assistant\index{Proof Assistant} only.)

The Metamath language allows variables to be
substituted\index{substitution!variable}\index{variable substitution}
with empty symbol sequences\index{empty substitution}.  However, in many
formal systems\index{formal system} this will never happen in a valid
proof.  Allowing for this possibility increases the likelihood of
ambiguous unifications\index{ambiguous
unification}\index{unification!ambiguous} during proof creation.
The default is that
empty substitutions are not allowed; for formal systems requiring them,
you must \texttt{set empty{\char`\_}substitution on}.
(An example where this must be \texttt{on}
would be a system that implements a Deduction Rule and in
which deductions from empty assumption lists would be permissible.  The
MIU-system\index{MIU-system} described in Appendix~\ref{MIU} is another
example.)
Note that empty substitutions are
always permissible in proof verification (VERIFY PROOF...) outside the
Proof Assistant.  (See the MIU system in the Metamath book for an example
of a system needing empty substitutions; another example would be a
system that implements a Deduction Rule and in which deductions from
empty assumption lists would be permissible.)

It is better to leave this \texttt{off} when working with \texttt{set.mm}.
Note that this command does not affect the way proofs are verified with
the \texttt{verify proof} command.  Outside of the Proof Assistant,
substitution of empty sequences for math symbols is always allowed.

\subsection{\texttt{set search\_limit} Command}\index{\texttt{set
search{\char`\_}limit} command} Syntax:  \texttt{set search{\char`\_}limit} {\em
number}

(This command is available outside the Proof Assistant but affects the
Proof Assistant\index{Proof Assistant} only.)

This command sets a parameter that determines when the \texttt{improve} command
in Proof Assistant mode gives up.  If you want \texttt{improve} to search harder,
you may increase it.  The \texttt{show settings} command tells you its current
value.


\subsection{\texttt{show new\_proof} Command}\index{\texttt{show
new{\char`\_}proof} command}
Syntax:  \texttt{show new{\char`\_}proof} [{\em
qualifiers} (see below)]

This command (available only in Proof Assistant mode) displays the proof
in progress.  It is identical to the \texttt{show proof} command, except that
there is no statement argument (since it is the statement being proved) and
the following qualifiers are not available:

    \texttt{/statement{\char`\_}summary}

    \texttt{/detailed{\char`\_}step}

Also, the following additional qualifiers are available:

    \texttt{/unknown} - Shows only steps that have no statement assigned.

    \texttt{/not{\char`\_}unified} - Shows only steps that have not been unified.

Note that \texttt{/essential}, \texttt{/depth}, \texttt{/unknown}, and
\texttt{/not{\char`\_}unified} may be
used in any combination; each of them effectively filters out additional
steps from the proof display.

See also:  \texttt{show proof}






\subsection{\texttt{assign} Command}\index{\texttt{assign} command}
Syntax:   \texttt{assign} {\em step} {\em label} [\texttt{/no{\char`\_}unify}]

   and:   \texttt{assign first} {\em label}

   and:   \texttt{assign last} {\em label}


This command, available in the Proof Assistant only, assigns an unknown
step (one with \texttt{?} in the \texttt{show new{\char`\_}proof}
listing) with the statement specified by {\em label}.  The assignment
will not be allowed if the statement cannot be unified with the step.

If \texttt{last} is specified instead of {\em step} number, the last
step that is shown by \texttt{show new{\char`\_}proof /unknown} will be
used.  This can be useful for building a proof with a command file (see
\texttt{help submit}).  It also makes building proofs faster when you know
the assignment for the last step.

If \texttt{first} is specified instead of {\em step} number, the first
step that is shown by \texttt{show new{\char`\_}proof /unknown} will be
used.

If {\em step} is zero or negative, the -{\em step}th from last unknown
step, as shown by \texttt{show new{\char`\_}proof /unknown}, will be
used.  \texttt{assign -1} {\em label} will assign the penultimate
unknown step, \texttt{assign -2} {\em label} the antepenultimate, and
\texttt{assign 0} {\em label} is the same as \texttt{assign last} {\em
label}.

Optional command qualifier:

    \texttt{/no{\char`\_}unify} - do not prompt user to select a unification if there is
        more than one possibility.  This is useful for noninteractive
        command files.  Later, the user can \texttt{unify all /interactive}.
        (The assignment will still be automatically unified if there is only
        one possibility and will be refused if unification is not possible.)



\subsection{\texttt{match} Command}\index{\texttt{match} command}
Syntax:  \texttt{match step} {\em step} [\texttt{/max{\char`\_}essential{\char`\_}hyp}
{\em number}]

    and:  \texttt{match all} [\texttt{/essential}]
          [\texttt{/max{\char`\_}essential{\char`\_}hyp} {\em number}]

This command, available in the Proof Assistant only, shows what
statements can be unified with the specified step(s).  {\em Note:} In
its current form, this command is not very useful because of the large
number of matches it reports.
It may be enhanced in the future.  In the meantime, the \texttt{search}
command can often provide finer control over locating theorems of interest.

Optional command qualifiers:

    \texttt{/max{\char`\_}essential{\char`\_}hyp} {\em number} - filters out
        of the list any statements
        with more than the specified number of
        \texttt{\$e}\index{\texttt{\$e} statement} hypotheses.

    \texttt{/essential{\char`\_}only} - in the \texttt{match all} statement, only
        the steps that
        would be listed in the \texttt{show new{\char`\_}proof /essential} display are
        matched.



\subsection{\texttt{let} Command}\index{\texttt{let} command}
Syntax: \texttt{let variable} {\em variable} = \verb/"/{\em symbol-sequence}\verb/"/

  and: \texttt{let step} {\em step} = \verb/"/{\em symbol-sequence}\verb/"/

These commands, available in the Proof Assistant\index{Proof Assistant}
only, assign a temporary variable\index{temporary variable} or unknown
step with a specific symbol sequence.  They are useful in the middle of
creating a proof, when you know what should be in the proof step but the
unification algorithm doesn't yet have enough information to completely
specify the temporary variables.  A ``temporary variable'' is one that
has the form \texttt{\$}{\em nn} in the proof display, such as
\texttt{\$1}, \texttt{\$2}, etc.  The {\em symbol-sequence} may contain
other unknown variables if desired.  Examples:

    \verb/let variable $32 = "A = B"/

    \verb/let variable $32 = "A = $35"/

    \verb/let step 10 = '|- x = x'/

    \verb/let step -2 = "|- ( $7 = ph )"/

Any symbol sequence will be accepted for the \texttt{let variable}
command.  Only those symbol sequences that can be unified with the step
will be accepted for \texttt{let step}.

The \texttt{let} commands ``zap'' the proof with information that can
only be verified when the proof is built up further.  If you make an
error, the command sequence \texttt{delete
floating{\char`\_}hypotheses}, \texttt{initialize all}, and
\texttt{unify all /interactive} will undo a bad \texttt{let} assignment.

If {\em step} is zero or negative, the -{\em step}th from last unknown
step, as shown by \texttt{show new{\char`\_}proof /unknown}, will be
used.  The command \texttt{let step 0} = \verb/"/{\em
symbol-sequence}\verb/"/ will use the last unknown step, \texttt{let
step -1} = \verb/"/{\em symbol-sequence}\verb/"/ the penultimate, etc.
If {\em step} is positive, \texttt{let step} may be used to assign known
(in the sense of having previously been assigned a label with
\texttt{assign}) as well as unknown steps.

Either single or double quotes can surround the {\em symbol-sequence} as
long as they are different from any quotes inside a {\em
symbol-sequence}.  If {\em symbol-sequence} contains both kinds of
quotes, see the instructions at the end of \texttt{help let} in the
Metamath program.


\subsection{\texttt{unify} Command}\index{\texttt{unify} command}
Syntax:  \texttt{unify step} {\em step}

      and:   \texttt{unify all} [\texttt{/interactive}]

These commands, available in the Proof Assistant only, unify the source
and target of the specified step(s). If you specify a specific step, you
will be prompted to select among the unifications that are possible.  If
you specify \texttt{all}, only those steps with unique unifications will be
unified.

Optional command qualifier for \texttt{unify all}:

    \texttt{/interactive} - You will be prompted to select among the
        unifications
        that are possible for any steps that do not have unique
        unifications.  (Otherwise \texttt{unify all} will bypass these.)

See also \texttt{set unification{\char`\_}timeout}.  The default is
100000, but increasing it to 1000000 can help difficult cases.  Manually
assigning some or all of the unknown variables with the \texttt{let
variable} command also helps difficult cases.



\subsection{\texttt{initialize} Command}\index{\texttt{initialize} command}
Syntax:  \texttt{initialize step} {\em step}

    and: \texttt{initialize all}

These commands, available in the Proof Assistant\index{Proof Assistant}
only, ``de-unify'' the target and source of a step (or all steps), as
well as the hypotheses of the source, and makes all variables in the
source and the source's hypotheses unknown.  This command is useful to
help recover from incorrect unifications that resulted from an incorrect
\texttt{assign}, \texttt{let}, or unification choice.  Part or all of
the command sequence \texttt{delete floating{\char`\_}hypotheses},
\texttt{initialize all}, and \texttt{unify all /interactive} will recover
from incorrect unifications.

See also:  \texttt{unify} and \texttt{delete}



\subsection{\texttt{delete} Command}\index{\texttt{delete} command}
Syntax:  \texttt{delete step} {\em step}

   and:      \texttt{delete all} -- {\em Warning: dangerous!}

   and:      \texttt{delete floating{\char`\_}hypotheses}

These commands are available in the Proof Assistant only.  The
\texttt{delete step} command deletes the proof tree section that
branches off of the specified step and makes the step become unknown.
\texttt{delete all} is equivalent to \texttt{delete step} {\em step}
where {\em step} is the last step in the proof (i.e.\ the beginning of
the proof tree).

In most cases the \texttt{undo} command is the best way to undo
a previous step.
An alternative is to salvage your last \texttt{save
new{\char`\_}proof} by exiting and reentering the Proof Assistant.
For this to work, keep a log file open to record your work
and to do \texttt{save new{\char`\_}proof} frequently, especially before
\texttt{delete}.

\texttt{delete floating{\char`\_}hypotheses} will delete all sections of
the proof that branch off of \texttt{\$f}\index{\texttt{\$f} statement}
statements.  It is sometimes useful to do this before an
\texttt{initialize} command to recover from an error.  Note that once a
proof step with a \texttt{\$f} hypothesis as the target is completely
known, the \texttt{improve} command can usually fill in the proof for
that step.  Unlike the deletion of logical steps, \texttt{delete
floating{\char`\_}hypotheses} is a relatively safe command that is
usually easy to recover from.



\subsection{\texttt{improve} Command}\index{\texttt{improve} command}
\label{improve}
Syntax:  \texttt{improve} {\em step} [\texttt{/depth} {\em number}]
                                               [\texttt{/no{\char`\_}distinct}]

   and:   \texttt{improve first} [\texttt{/depth} {\em number}]
                                              [\texttt{/no{\char`\_}distinct}]

   and:   \texttt{improve last} [\texttt{/depth} {\em number}]
                                              [\texttt{/no{\char`\_}distinct}]

   and:   \texttt{improve all} [\texttt{/depth} {\em number}]
                                              [\texttt{/no{\char`\_}distinct}]

These commands, available in the Proof Assistant\index{Proof Assistant}
only, try to find proofs automatically for unknown steps whose symbol
sequences are completely known.  They are primarily useful for filling in
proofs of \texttt{\$f}\index{\texttt{\$f} statement} hypotheses.  The
search will be restricted to statements having no
\texttt{\$e}\index{\texttt{\$e} statement} hypotheses.

\begin{sloppypar} % narrow
Note:  If memory is limited, \texttt{improve all} on a large proof may
overflow memory.  If you use \texttt{set unification{\char`\_}timeout 1}
before \texttt{improve all}, there will usually be sufficient
improvement to easily recover and completely \texttt{improve} the proof
later on a larger computer.  Warning:  Once memory has overflowed, there
is no recovery.  If in doubt, save the intermediate proof (\texttt{save
new{\char`\_}proof} then \texttt{write source}) before \texttt{improve
all}.
\end{sloppypar}

If \texttt{last} is specified instead of {\em step} number, the last
step that is shown by \texttt{show new{\char`\_}proof /unknown} will be
used.

If \texttt{first} is specified instead of {\em step} number, the first
step that is shown by \texttt{show new{\char`\_}proof /unknown} will be
used.

If {\em step} is zero or negative, the -{\em step}th from last unknown
step, as shown by \texttt{show new{\char`\_}proof /unknown}, will be
used.  \texttt{improve -1} will use the penultimate
unknown step, \texttt{improve -2} {\em label} the antepenultimate, and
\texttt{improve 0} is the same as \texttt{improve last}.

Optional command qualifier:

    \texttt{/depth} {\em number} - This qualifier will cause the search
        to include
        statements with \texttt{\$e} hypotheses (but no new variables in
        the \texttt{\$e}
        hypotheses), provided that the backtracking has not exceeded the
        specified depth. {\em Warning:}  Try \texttt{/depth 1},
        then \texttt{2}, then \texttt{3}, etc.
        in sequence because of possible exponential blowups.  Save your
        work before trying \texttt{/depth} greater than \texttt{1}!

    \texttt{/no{\char`\_}distinct} - Skip trial statements that have
        \texttt{\$d}\index{\texttt{\$d} statement} requirements.
        This qualifier will prevent assignments that might violate \texttt{\$d}
        requirements but it also could miss possible legal assignments.

See also: \texttt{set search{\char`\_}limit}

\subsection{\texttt{save new\_proof} Command}\index{\texttt{save
new{\char`\_}proof} command}
Syntax:  \texttt{save new{\char`\_}proof} {\em label} [\texttt{/normal}]
   [\texttt{/compressed}]

The \texttt{save new{\char`\_}proof} command is available in the Proof
Assistant only.  It saves the proof in progress in the source
buffer\index{source buffer}.  \texttt{save new{\char`\_}proof} may be
used to save a completed proof, or it may be used to save a proof in
progress in order to work on it later.  If an incomplete proof is saved,
any user assignments with \texttt{let step} or \texttt{let variable}
will be lost, as will any ambiguous unifications\index{ambiguous
unification}\index{unification!ambiguous} that were resolved manually.
To help make recovery easier, it can be helpful to \texttt{improve all}
before \texttt{save new{\char`\_}proof} so that the incomplete proof
will have as much information as possible.

Note that the proof will not be permanently saved until a \texttt{write
source} command is issued.

Optional command qualifiers:

    \texttt{/normal} - The proof is saved in the normal format (i.e., as a
        sequence of labels, which is the defined format of the basic Metamath
        language).\index{basic language}  This is the default format that
        is used if a qualifier is omitted.

    \texttt{/compressed} - The proof is saved in the compressed format, which
        reduces storage requirements for a database.
        (See Appendix~\ref{compressed}.)


\section{Creating \LaTeX\ Output}\label{texout}\index{latex@{\LaTeX}}

You can generate \LaTeX\ output given the
information in a database.
The database must already include the necessary typesetting information
(see section \ref{tcomment} for how to provide this information).

The \texttt{show statement} and \texttt{show proof} commands each have a
special \texttt{/tex} command qualifier that produces \LaTeX\ output.
(The \texttt{show statement} command also has the
\texttt{/simple{\char`\_}tex} qualifier for output that is easier to
edit by hand.)  Before you can use them, you must open a \LaTeX\ file to
which to send their output.  A typical complete session will use this
sequence of Metamath commands:

\begin{verbatim}
read set.mm
open tex example.tex
show statement a1i /tex
show proof a1i /all/lemmon/renumber/tex
show statement uneq2 /tex
show proof uneq2 /all/lemmon/renumber/tex
close tex
\end{verbatim}

See Section~\ref{mathcomments} for information on comment markup and
Appendix~\ref{ASCII} for information on how math symbol translation is
specified.

To format and print the \LaTeX\ source, you will need the \LaTeX\
program, which is standard on most Linux installations and available for
Windows.  On Linux, in order to create a {\sc pdf} file, you will
typically type at the shell prompt
\begin{verbatim}
$ pdflatex example.tex
\end{verbatim}

\subsection{\texttt{open tex} Command}\index{\texttt{open tex} command}
Syntax:  \texttt{open tex} {\em file-name} [\texttt{/no{\char`\_}header}]

This command opens a file for writing \LaTeX\
source\index{latex@{\LaTeX}} and writes a \LaTeX\ header to the file.
\LaTeX\ source can be written with the \texttt{show proof}, \texttt{show
new{\char`\_}proof}, and \texttt{show statement} commands using the
\texttt{/tex} qualifier.

The mapping to \LaTeX\ symbols is defined in a special comment
containing a \texttt{\$t} token, described in Appendix~\ref{ASCII}.

There is an optional command qualifier:

    \texttt{/no{\char`\_}header} - This qualifier prevents a standard
        \LaTeX\ header and trailer
        from being included with the output \LaTeX\ code.


\subsection{\texttt{close tex} Command}\index{\texttt{close tex} command}
Syntax:  \texttt{close tex}

This command writes a trailer to any \LaTeX\ file\index{latex@{\LaTeX}}
that was opened with \texttt{open tex} (unless
\texttt{/no{\char`\_}header} was used with \texttt{open tex}) and closes
the \LaTeX\ file.


\section{Creating {\sc HTML} Output}\label{htmlout}

You can generate {\sc html} web pages given the
information in a database.
The database must already include the necessary typesetting information
(see section \ref{tcomment} for how to provide this information).
The ability to produce {\sc html} web pages was added in Metamath version
0.07.30.

To create an {\sc html} output file(s) for \texttt{\$a} or \texttt{\$p}
statement(s), use
\begin{quote}
    \texttt{show statement} {\em label-match} \texttt{/html}
\end{quote}
The output file will be named {\em label-match}\texttt{.html}
for each match.  When {\em
label-match} has wildcard (\texttt{*}) characters, all statements with
matching labels will have {\sc html} files produced for them.  Also,
when {\em label-match} has a wildcard (\texttt{*}) character, two additional
files, \texttt{mmdefinitions.html} and \texttt{mmascii.html} will be
produced.  To produce {\em only} these two additional files, you can use
\texttt{?*}, which will not match any statement label, in place of {\em
label-match}.

There are three other qualifiers for \texttt{show statement} that also
generate {\sc HTML} code.  These are \texttt{/alt{\char`\_}html},
\texttt{/brief{\char`\_}html}, and
\texttt{/brief{\char`\_}alt{\char`\_}html}, and are described in the
next section.

The command
\begin{quote}
    \texttt{show statement} {\em label-match} \texttt{/alt{\char`\_}html}
\end{quote}
does the same as \texttt{show statement} {\em label-match} \texttt{/html},
except that the {\sc html} code for the symbols is taken from
\texttt{althtmldef} statements instead of \texttt{htmldef} statements in
the \texttt{\$t} comment.

The command
\begin{verbatim}
show statement * /brief_html
\end{verbatim}
invokes a special mode that just produces definition and theorem lists
accompanied by their symbol strings, in a format suitable for copying and
pasting into another web page (such as the tutorial pages on the
Metamath web site).

Finally, the command
\begin{verbatim}
show statement * /brief_alt_html
\end{verbatim}
does the same as \texttt{show statement * / brief{\char`\_}html}
for the alternate {\sc html}
symbol representation.

A statement's comment can include a special notation that provides a
certain amount of control over the {\sc HTML} version of the comment.  See
Section~\ref{mathcomments} (p.~\pageref{mathcomments}) for the comment
markup features.

The \texttt{write theorem{\char`\_}list} and \texttt{write bibliography}
commands, which are described below, provide as a side effect complete
error checking for all of the features described in this
section.\index{error checking}

\subsection{\texttt{write theorem\_list}
Command}\index{\texttt{write theorem{\char`\_}list} command}
Syntax:  \texttt{write theorem{\char`\_}list}
[\texttt{/theorems{\char`\_}per{\char`\_}page} {\em number}]

This command writes a list of all of the \texttt{\$a} and \texttt{\$p}
statements in the database into a web page file
 called \texttt{mmtheorems.html}.
When additional files are needed, they are called
\texttt{mmtheorems2.html}, \texttt{mmtheorems3.html}, etc.

Optional command qualifier:

    \texttt{/theorems{\char`\_}per{\char`\_}page} {\em number} -
 This qualifier specifies the number of statements to
        write per web page.  The default is 100.

{\em Note:} In version 0.177\index{Metamath!limitations of version
0.177} of Metamath, the ``Nearby theorems'' links on the individual
web pages presuppose 100 theorems per page when linking to the theorem
list pages.  Therefore the \texttt{/theorems{\char`\_}per{\char`\_}page}
qualifier, if it specifies a number other than 100, will cause the
individual web pages to be out of sync and should not be used to
generate the main theorem list for the web site.  This may be
fixed in a future version.


\subsection{\texttt{write bibliography}\label{wrbib}
Command}\index{\texttt{write bibliography} command}
Syntax:  \texttt{write bibliography} {\em filename}

This command reads an existing {\sc html} bibliographic cross-reference
file, normally called \texttt{mmbiblio.html}, and updates it per the
bibliographic links in the database comments.  The file is updated
between the {\sc html} comment lines \texttt{<!--
{\char`\#}START{\char`\#} -->} and \texttt{<!-- {\char`\#}END{\char`\#}
-->}.  The original input file is renamed to {\em
filename}\texttt{{\char`\~}1}.

A bibliographic reference is indicated with the reference name
in brackets, such as  \texttt{Theorem 3.1 of
[Monk] p.\ 22}.
See Section~\ref{htmlout} (p.~\pageref{htmlout}) for
syntax details.


\subsection{\texttt{write recent\_additions}
Command}\index{\texttt{write recent{\char`\_}additions} command}
Syntax:  \texttt{write recent{\char`\_}additions} {\em filename}
[\texttt{/limit} {\em number}]

This command reads an existing ``Recent Additions'' {\sc html} file,
normally called \texttt{mmrecent.html}, and updates it with the
descriptions of the most recently added theorems to the database.
 The file is updated between
the {\sc html} comment lines \texttt{<!-- {\char`\#}START{\char`\#} -->}
and \texttt{<!-- {\char`\#}END{\char`\#} -->}.  The original input file
is renamed to {\em filename}\texttt{{\char`\~}1}.

Optional command qualifier:

    \texttt{/limit} {\em number} -
 This qualifier specifies the number of most recent theorems to
   write to the output file.  The default is 100.


\section{Text File Utilities}

\subsection{\texttt{tools} Command}\index{\texttt{tools} command}
Syntax:  \texttt{tools}

This command invokes an easy-to-use, general purpose utility for
manipulating the contents of {\sc ascii} text files.  Upon typing
\texttt{tools}, the command-line prompt will change to \texttt{TOOLS>}
until you type \texttt{exit}.  The \texttt{tools} commands can be used
to perform simple, global edits on an input/output file,
such as making a character string substitution on each line, adding a
string to each line, and so on.  A typical use of this utility is
to build a \texttt{submit} input file to perform a common operation on a
list of statements obtained from \texttt{show label} or \texttt{show
usage}.

The actions of most of the \texttt{tools} commands can also be
performed with equivalent (and more powerful) Unix shell commands, and
some users may find those more efficient.  But for Windows users or
users not comfortable with Unix, \texttt{tools} provides an
easy-to-learn alternative that is adequate for most of the
script-building tasks needed to use the Metamath program effectively.

\subsection{\texttt{help} Command (in \texttt{tools})}
Syntax:  \texttt{help}

The \texttt{help} command lists the commands available in the
\texttt{tools} utility, along with a brief description.  Each command,
in turn, has its own help, such as \texttt{help add}.  As with
Metamath's \texttt{MM>} prompt, a complete command can be entered at
once, or just the command word can be typed, causing the program to
prompt for each argument.

\vskip 1ex
\noindent Line-by-line editing commands:

  \texttt{add} - Add a specified string to each line in a file.

  \texttt{clean} - Trim spaces and tabs on each line in a file; convert
         characters.

  \texttt{delete} - Delete a section of each line in a file.

  \texttt{insert} - Insert a string at a specified column in each line of
        a file.

  \texttt{substitute} - Make a simple substitution on each line of the file.

  \texttt{tag} - Like \texttt{add}, but restricted to a range of lines.

  \texttt{swap} - Swap the two halves of each line in a file.

\vskip 1ex
\noindent Other file-processing commands:

  \texttt{break} - Break up (tokenize) a file into a list of tokens (one per
        line).

  \texttt{build} - Build a file with multiple tokens per line from a list.

  \texttt{count} - Count the occurrences in a file of a specified string.

  \texttt{number} - Create a list of numbers.

  \texttt{parallel} - Put two files in parallel.

  \texttt{reverse} - Reverse the order of the lines in a file.

  \texttt{right} - Right-justify lines in a file (useful before sorting
         numbers).

%  \texttt{tag} - Tag edit updates in a program for revision control.

  \texttt{sort} - Sort the lines in a file with key starting at
         specified string.

  \texttt{match} - Extract lines containing (or not) a specified string.

  \texttt{unduplicate} - Eliminate duplicate occurrences of lines in a file.

  \texttt{duplicate} - Extract first occurrence of any line occurring
         more than

   \ \ \    once in a file, discarding lines occurring exactly once.

  \texttt{unique} - Extract lines occurring exactly once in a file.

  \texttt{type} (10 lines) - Display the first few lines in a file.
                  Similar to Unix \texttt{head}.

  \texttt{copy} - Similar to Unix \texttt{cat} but safe (same input
         and output file allowed).

  \texttt{submit} - Run a script containing \texttt{tools} commands.

\vskip 1ex

\noindent Note:
  \texttt{unduplicate}, \texttt{duplicate}, and \texttt{unique} also
 sort the lines as a side effect.


\subsection{Using \texttt{tools} to Build Metamath \texttt{submit}
Scripts}

The \texttt{break} command is typically used to break up a series of
statement labels, such as the output of Metamath's \texttt{show usage},
into one label per line.  The other \texttt{tools} commands can then be
used to add strings before and after each statement label to specify
commands to be performed on the statement.  The \texttt{parallel}
command is useful when a statement label must be mentioned more than
once on a line.

Very often a \texttt{submit} script for Metamath will require multiple
command lines for each statement being processed.  For example, you may
want to enter the Proof Assistant, \texttt{minimize{\char`\_}with} your
latest theorem, \texttt{save} the new proof, and \texttt{exit} the Proof
Assistant.  To accomplish this, you can build a file with these four
commands for each statement on a single line, separating each command
with a designated character such as \texttt{@}.  Then at the end you can
\texttt{substitute} each \texttt{@} with \texttt{{\char`\\}n} to break
up the lines into individual command lines (see \texttt{help
substitute}).


\subsection{Example of a \texttt{tools} Session}

To give you a quick feel for the \texttt{tools} utility, we show a
simple session where we create a file \texttt{n.txt} with 3 lines, add
strings before and after each line, and display the lines on the screen.
You can experiment with the various commands to gain experience with the
\texttt{tools} utility.

\begin{verbatim}
MM> tools
Entering the Text Tools utilities.
Type HELP for help, EXIT to exit.
TOOLS> number
Output file <n.tmp>? n.txt
First number <1>?
Last number <10>? 3
Increment <1>?
TOOLS> add
Input/output file? n.txt
String to add to beginning of each line <>? This is line
String to add to end of each line <>? .
The file n.txt has 3 lines; 3 were changed.
First change is on line 1:
This is line 1.
TOOLS> type n.txt
This is line 1.
This is line 2.
This is line 3.
TOOLS> exit
Exiting the Text Tools.
Type EXIT again to exit Metamath.
MM>
\end{verbatim}



\appendix
\chapter{Sample Representations}
\label{ASCII}

This Appendix provides a sample of {\sc ASCII} representations,
their corresponding traditional mathematical symbols,
and a discussion of their meanings
in the \texttt{set.mm} database.
These are provided in order of appearance.
This is only a partial list, and new definitions are routinely added.
A complete list is available at \url{http://metamath.org}.

These {\sc ASCII} representations, along
with information on how to display them,
are defined in the \texttt{set.mm} database file inside
a special comment called a \texttt{\$t} {\em
comment}\index{\texttt{\$t} comment} or {\em typesetting
comment.}\index{typesetting comment}
A typesetting comment
is indicated by the appearance of the
two-character string \texttt{\$t} at the beginning of the comment.
For more information,
see Section~\ref{tcomment}, p.~\pageref{tcomment}.

In the following table the ``{\sc ASCII}'' column shows the {\sc ASCII}
representation,
``Symbol'' shows the mathematical symbolic display
that corresponds to that {\sc ASCII} representation, ``Labels'' shows
the key label(s) that define the representation, and
``Description'' provides a description about the symbol.
As usual, ``iff'' is short for ``if and only if.''\index{iff}
In most cases the ``{\sc ASCII}'' column only shows
the key token, but it will sometimes show a sequence of tokens
if that is necessary for clarity.

{\setlength{\extrarowsep}{4pt} % Keep rows from being too close together
\begin{longtabu}   { @{} c c l X }
\textbf{ASCII} & \textbf{Symbol} & \textbf{Labels} & \textbf{Description} \\
\endhead
\texttt{|-} & $\vdash$ & &
  ``It is provable that...'' \\
\texttt{ph} & $\varphi$ & \texttt{wph} &
  The wff (boolean) variable phi,
  conventionally the first wff variable. \\
\texttt{ps} & $\psi$ & \texttt{wps} &
  The wff (boolean) variable psi,
  conventionally the second wff variable. \\
\texttt{ch} & $\chi$ & \texttt{wch} &
  The wff (boolean) variable chi,
  conventionally the third wff variable. \\
\texttt{-.} & $\lnot$ & \texttt{wn} &
  Logical not. E.g., if $\varphi$ is true, then $\lnot \varphi$ is false. \\
\texttt{->} & $\rightarrow$ & \texttt{wi} &
  Implies, also known as material implication.
  In classical logic the expression $\varphi \rightarrow \psi$ is true
  if either $\varphi$ is false or $\psi$ is true (or both), that is,
  $\varphi \rightarrow \psi$ has the same meaning as
  $\lnot \varphi \lor \psi$ (as proven in theorem \texttt{imor}). \\
\texttt{<->} & $\leftrightarrow$ &
  \hyperref[df-bi]{\texttt{df-bi}} &
  Biconditional (aka is-equals for boolean values).
  $\varphi \leftrightarrow \psi$ is true iff
  $\varphi$ and $\psi$ have the same value. \\
\texttt{\char`\\/} & $\lor$ &
  \makecell[tl]{{\hyperref[df-or]{\texttt{df-or}}}, \\
	         \hyperref[df-3or]{\texttt{df-3or}}} &
  Disjunction (logical ``or''). $\varphi \lor \psi$ is true iff
  $\varphi$, $\psi$, or both are true. \\
\texttt{/\char`\\} & $\land$ &
  \makecell[tl]{{\hyperref[df-an]{\texttt{df-an}}}, \\
                 \hyperref[df-3an]{\texttt{df-3an}}} &
  Conjunction (logical ``and''). $\varphi \land \psi$ is true iff
  both $\varphi$ and $\psi$ are true. \\
\texttt{A.} & $\forall$ &
  \texttt{wal} &
  For all; the wff $\forall x \varphi$ is true iff
  $\varphi$ is true for all values of $x$. \\
\texttt{E.} & $\exists$ &
  \hyperref[df-ex]{\texttt{df-ex}} &
  There exists; the wff
  $\exists x \varphi$ is true iff
  there is at least one $x$ where $\varphi$ is true. \\
\texttt{[ y / x ]} & $[ y / x ]$ &
  \hyperref[df-sb]{\texttt{df-sb}} &
  The wff $[ y / x ] \varphi$ produces
  the result when $y$ is properly substituted for $x$ in $\varphi$
  ($y$ replaces $x$).
  % This is elsb4
  % ( [ x / y ] z e. y <-> z e. x )
  For example,
  $[ x / y ] z \in y$ is the same as $z \in x$. \\
\texttt{E!} & $\exists !$ &
  \hyperref[df-eu]{\texttt{df-eu}} &
  There exists exactly one;
  $\exists ! x \varphi$ is true iff
  there is at least one $x$ where $\varphi$ is true. \\
\texttt{\{ y | phi \}}  & $ \{ y | \varphi \}$ &
  \hyperref[df-clab]{\texttt{df-clab}} &
  The class of all sets where $\varphi$ is true. \\
\texttt{=} & $ = $ &
  \hyperref[df-cleq]{\texttt{df-cleq}} &
  Class equality; $A = B$ iff $A$ equals $B$. \\
\texttt{e.} & $ \in $ &
  \hyperref[df-clel]{\texttt{df-clel}} &
  Class membership; $A \in B$ if $A$ is a member of $B$. \\
\texttt{{\char`\_}V} & {\rm V} &
  \hyperref[df-v]{\texttt{df-v}} &
  Class of all sets (not itself a set). \\
\texttt{C\_} & $ \subseteq $ &
  \hyperref[df-ss]{\texttt{df-ss}} &
  Subclass (subset); $A \subseteq B$ is true iff
  $A$ is a subclass of $B$. \\
\texttt{u.} & $ \cup $ &
  \hyperref[df-un]{\texttt{df-un}} &
  $A \cup B$ is the union of classes $A$ and $B$. \\
\texttt{i^i} & $ \cap $ &
  \hyperref[df-in]{\texttt{df-in}} &
  $A \cap B$ is the intersection of classes $A$ and $B$. \\
\texttt{\char`\\} & $ \setminus $ &
  \hyperref[df-dif]{\texttt{df-dif}} &
  $A \setminus B$ (set difference)
  is the class of all sets in $A$ except for those in $B$. \\
\texttt{(/)} & $ \varnothing $ &
  \hyperref[df-nul]{\texttt{df-nul}} &
  $ \varnothing $ is the empty set (aka null set). \\
\texttt{\char`\~P} & $ \cal P $ &
  \hyperref[df-pw]{\texttt{df-pw}} &
  Power class. \\
\texttt{<.\ A , B >.} & $\langle A , B \rangle$ &
  \hyperref[df-op]{\texttt{df-op}} &
  The ordered pair $\langle A , B \rangle$. \\
\texttt{( F ` A )} & $ ( F ` A ) $ &
  \hyperref[df-fv]{\texttt{df-fv}} &
  The value of function $F$ when applied to $A$. \\
\texttt{\_i} & $ i $ &
  \texttt{df-i} &
  The square root of negative one. \\
\texttt{x.} & $ \cdot $ &
  \texttt{df-mul} &
  Complex number multiplication; $2~\cdot~3~=~6$. \\
\texttt{CC} & $ \mathbb{C} $ &
  \texttt{df-c} &
  The set of complex numbers. \\
\texttt{RR} & $ \mathbb{R} $ &
  \texttt{df-r} &
  The set of real numbers. \\
\end{longtabu}
} % end of extrarowsep

\chapter{Compressed Proofs}
\label{compressed}\index{compressed proof}\index{proof!compressed}

The proofs in the \texttt{set.mm} set theory database are stored in compressed
format for efficiency.  Normally you needn't concern yourself with the
compressed format, since you can display it with the usual proof display tools
in the Metamath program (\texttt{show proof}\ldots) or convert it to the normal
RPN proof format described in Section~\ref{proof} (with \texttt{save proof}
{\em label} \texttt{/normal}).  However for sake of completeness we describe the
format here and show how it maps to the normal RPN proof format.

A compressed proof, located between \texttt{\$=} and \texttt{\$.}\ keywords, consists
of a left parenthesis, a sequence of statement labels, a right parenthesis,
and a sequence of upper-case letters \texttt{A} through \texttt{Z} (with optional
white space between them).  White space must surround the parentheses
and the labels.  The left parenthesis tells Metamath that a
compressed proof follows.  (A normal RPN proof consists of just a sequence of
labels, and a parenthesis is not a legal character in a label.)

The sequence of upper-case letters corresponds to a sequence of integers
with the following mapping.  Each integer corresponds to a proof step as
described later.
\begin{center}
  \texttt{A} = 1 \\
  \texttt{B} = 2 \\
   \ldots \\
  \texttt{T} = 20 \\
  \texttt{UA} = 21 \\
  \texttt{UB} = 22 \\
   \ldots \\
  \texttt{UT} = 40 \\
  \texttt{VA} = 41 \\
  \texttt{VB} = 42 \\
   \ldots \\
  \texttt{YT} = 120 \\
  \texttt{UUA} = 121 \\
   \ldots \\
  \texttt{YYT} = 620 \\
  \texttt{UUUA} = 621 \\
   etc.
\end{center}

In other words, \texttt{A} through \texttt{T} represent the
least-significant digit in base 20, and \texttt{U} through \texttt{Y}
represent zero or more most-significant digits in base 5, where the
digits start counting at 1 instead of the usual 0. With this scheme, we
don't need white space between these ``numbers.''

(In the design of the compressed proof format, only upper-case letters,
as opposed to say all non-whitespace printable {\sc ascii} characters other than
%\texttt{\$}, was chosen to make the compressed proof a little less
%displeasing to the eye, at the expense of a typical 20\% compression
\texttt{\$}, were chosen so as not to collide with most text editor
searches, at the expense of a typical 20\% compression
loss.  The base 5/base 20 grouping, as opposed to say base 6/base 19,
was chosen by experimentally determining the grouping that resulted in
best typical compression.)

The letter \texttt{Z} identifies (tags) a proof step that is identical to one
that occurs later on in the proof; it helps shorten the proof by not requiring
that identical proof steps be proved over and over again (which happens often
when building wff's).  The \texttt{Z} is placed immediately after the
least-significant digit (letters \texttt{A} through \texttt{T}) that ends the integer
corresponding to the step to later be referenced.

The integers that the upper-case letters correspond to are mapped to labels as
follows.  If the statement being proved has $m$ mandatory hypotheses, integers
1 through $m$ correspond to the labels of these hypotheses in the order shown
by the \texttt{show statement ... / full} command, i.e., the RPN order\index{RPN
order} of the mandatory
hypotheses.  Integers $m+1$ through $m+n$ correspond to the labels enclosed in
the parentheses of the compressed proof, in the order that they appear, where
$n$ is the number of those labels.  Integers $m+n+1$ on up don't directly
correspond to statement labels but point to proof steps identified with the
letter \texttt{Z}, so that these proof steps can be referenced later in the
proof.  Integer $m+n+1$ corresponds to the first step tagged with a \texttt{Z},
$m+n+2$ to the second step tagged with a \texttt{Z}, etc.  When the compressed
proof is converted to a normal proof, the entire subproof of a step tagged
with \texttt{Z} replaces the reference to that step.

For efficiency, Metamath works with compressed proofs directly, without
converting them internally to normal proofs.  In addition to the usual
error-checking, an error message is given if (1) a label in the label list in
parentheses does not refer to a previous \texttt{\$p} or \texttt{\$a} statement or a
non-mandatory hypothesis of the statement being proved and (2) a proof step
tagged with \texttt{Z} is referenced before the step tagged with the \texttt{Z}.

Just as in a normal proof under development (Section~\ref{unknown}), any step
or subproof that is not yet known may be represented with a single \texttt{?}.
White space does not have to appear between the \texttt{?}\ and the upper-case
letters (or other \texttt{?}'s) representing the remainder of the proof.

% April 1, 2004 Appendix C has been added back in with corrections.
%
% May 20, 2003 Appendix C was removed for now because there was a problem found
% by Bob Solovay
%
% Also, removed earlier \ref{formalspec} 's (3 cases above)
%
% Bob Solovay wrote on 30 Nov 2002:
%%%%%%%%%%%%% (start of email comment )
%      3. My next set of comments concern appendix C. I read this before I
% read Chapter 4. So I first noted that the system as presented in the
% Appendix lacked a certain formal property that I thought desirable. I
% then came up with a revised formal system that had this property. Upon
% reading Chapter 4, I noticed that the revised system was closer to the
% treatment in Chapter 4 than the system in Appendix C.
%
%         First a very minor correction:
%
%         On page 142 line 2: The condition that V(e) != V(f) should only be
% required of e, f in T such that e != f.
%
%         Here is a natural property [transitivity] that one would like
% the formal system to have:
%
%         Let Gamma be a set of statements. Suppose that the statement Phi
% is provable from Gamma and that the statement Psi is provable from Gamma
% \cup {Phi}. Then Psi is provable from Gamma.
%
%         I shall present an example to show that this property does not
% hold for the formal systems of Appendix C:
%
%         I write the example in metamath style:
%
% $c A B C D E $.
% $v x y
%
% ${
% tx $f A x $.
% ty $f B y $.
% ax1 $a C x y $.
% $}
%
% ${
% tx $f A x $.
% ty $f B y $.
% ax2-h1 $e C x y $.
% ax2 $a D y $.
% $}
%
% ${
% ty $f B y $.
% ax3-h1 $e D y $.
% ax3 $a E y $.
% $}
%
% $(These three axioms are Gamma $)
%
% ${
% tx $f A x $.
% ty $f B y $.
% Phi $p D y $=
% tx ty tx ty ax1 ax2 $.
% $}
%
% ${
% ty $f B y $.
% Psi $p E y $=
% ty ty Phi ax3 $.
% $}
%
%
% I omit the formal proofs of the following claims. [I will be glad to
% supply them upon request.]
%
% 1) Psi is not provable from Gamma;
%
% 2) Psi is provable from Gamma + Phi.
%
% Here "provable" refers to the formalism of Appendix C.
%
% The trouble of course is that Psi is lacking the variable declaration
%
% $f Ax $.
%
% In the Metamath system there is no trouble proving Psi. I attach a
% metamath file that shows this and which has been checked by the
% metamath program.
%
% I next want to indicate how I think the treatment in Appendix C should
% be revised so as to conform more closely to the metamath system of the
% main text. The revised system *does* have the transitivity property.
%
% We want to give revised definitions of "statement" and
% "provable". [cf. sections C.2.4. and C.2.5] Our new definitions will
% use the definitions given in Appendix C. So we take the following
% tack. We refer to the original notions as o-statement and o-provable. And
% we refer to the notions we are defining as n-statement and n-provable.
%
%         A n-statement is an o-statement in which the only variables
% that appear in the T component are mandatory.
%
%         To any o-statement we can associate its reduct which is a
% n-statement by dropping all the elements of T or D which contain
% non-mandatory variables.
%
%         An n-statement gamma is n-provable if there is an o-statement
% gamma' which has gamma as its reduct andf such that gamma' is
% o-provable.
%
%         It seems to me [though I am not completely sure on this point]
% that n-provability corresponds to metamath provability as discussed
% say in Chapter 4.
%
%         Attached to this letter is the metamath proof of Phi and Psi
% from Gamma discussed above.
%
%         I am still brooding over the question of whether metamath
% correctly formalizes set-theory. No doubt I will have some questions
% re this after my thoughts become clearer.
%%%%%%%%%%%%%%%% (end of email comment)

%%%%%%%%%%%%%%%% (start of 2nd email comment from Bob Solovay 1-Apr-04)
%
%         I hope that Appendix C is the one that gives a "formal" treatment
% of Metamath. At any rate, thats the appendix I want to comment on.
%
%         I'm going to suggest two changes in the definition.
%
%         First change (in the definition of statement): Require that the
% sets D, T, and E be finite.
%
%         Probably things are fine as you give them. But in the applications
% to the main metamath system they will always be finite, and its useful in
% thinking about things [at least for me] to stick to the finite case.
%
%         Second change:
%
%         First let me give an approximate description. Remove the dummy
% variables from the statement. Instead, include them in the proof.
%
%         More formally: Require that T consists of type declarations only
% for mandatory variables. Require that all the pairs in D consist of
% mandatory variables.
%
%         At the start of a proof we are allowed to declare a finite number
% of dummy variables [provided that none of them appear in any of the
% statements in E \cup {A}. We have to supply type declarations for all the
% dummy variables. We are allowed to add new $d statements referring to
% either the mandatory or dummy variables. But we require that no new $d
% statement references only mandatory variables.
%
%         I find this way of doing things more conceptual than the treatment
% in Appendix C. But the change [which I will use implicitly in later
% letters about doing Peano] is mainly aesthetic. I definitely claim that my
% results on doing Peano all apply to Metamath as it is presented in your
% book.
%
%         --Bob
%
%%%%%%%%%%%%%%%% (end of 2nd email comment)

%%
%% When uncommenting the below, also uncomment references above to {formalspec}
%%
\chapter{Metamath's Formal System}\label{formalspec}\index{Metamath!as a formal
system}

\section{Introduction}

\begin{quote}
  {\em Perfection is when there is no longer anything more to take away.}
    \flushright\sc Antoine de
     Saint-Exupery\footnote{\cite[p.~3-25]{Campbell}.}\\
\end{quote}\index{de Saint-Exupery, Antoine}

This appendix describes the theory behind the Metamath language in an abstract
way intended for mathematicians.  Specifically, we construct two
set-theo\-ret\-i\-cal objects:  a ``formal system'' (roughly, a set of syntax
rules, axioms, and logical rules) and its ``universe'' (roughly, the set of
theorems derivable in the formal system).  The Metamath computer language
provides us with a way to describe specific formal systems and, with the aid of
a proof provided by the user, to verify that given theorems
belong to their universes.

To understand this appendix, you need a basic knowledge of informal set theory.
It should be sufficient to understand, for example, Ch.\ 1 of Munkres' {\em
Topology} \cite{Munkres}\index{Munkres, James R.} or the
introductory set theory chapter
in many textbooks that introduce abstract mathematics. (Note that there are
minor notational differences among authors; e.g.\ Munkres uses $\subset$ instead
of our $\subseteq$ for ``subset.''  We use ``included in'' to mean ``a subset
of,'' and ``belongs to'' or ``is contained in'' to mean ``is an element of.'')
What we call a ``formal'' description here, unlike earlier, is actually an
informal description in the ordinary language of mathematicians.  However we
provide sufficient detail so that a mathematician could easily formalize it,
even in the language of Metamath itself if desired.  To understand the logic
examples at the end of this appendix, familiarity with an introductory book on
mathematical logic would be helpful.

\section{The Formal Description}

\subsection[Preliminaries]{Preliminaries\protect\footnotemark}%
\footnotetext{This section is taken mostly verbatim
from Tarski \cite[p.~63]{Tarski1965}\index{Tarski, Alfred}.}

By $\omega$ we denote the set of all natural numbers (non-negative integers).
Each natural number $n$ is identified with the set of all smaller numbers: $n =
\{ m | m < n \}$.  The formula $m < n$ is thus equivalent to the condition: $m
\in n$ and $m,n \in \omega$. In particular, 0 is the number zero and at the
same time the empty set $\varnothing$, $1=\{0\}$, $2=\{0,1\}$, etc. ${}^B A$
denotes the set of all functions on $B$ to $A$ (i.e.\ with domain $B$ and range
included in $A$).  The members of ${}^\omega A$ are what are called {\em simple
infinite sequences},\index{simple infinite sequence}
with all {\em terms}\index{term} in $A$.  In case $n \in \omega$, the
members of ${}^n A$ are referred to as {\em finite $n$-termed
sequences},\index{finite $n$-termed
sequence} again
with terms in $A$.  The consecutive terms (function values) of a finite or
infinite sequence $f$ are denoted by $f_0, f_1, \ldots ,f_n,\ldots$.  Every
finite sequence $f \in \bigcup _{n \in \omega} {}^n A$ uniquely determines the
number $n$ such that $f \in {}^n A$; $n$ is called the {\em
length}\index{length of a sequence ({$"|\ "|$})} of $f$ and
is denoted by $|f|$.  $\langle a \rangle$ is the sequence $f$ with $|f|=1$ and
$f_0=a$; $\langle a,b \rangle$ is the sequence $f$ with $|f|=2$, $f_0=a$,
$f_1=b$; etc.  Given two finite sequences $f$ and $g$, we denote by $f\frown g$
their {\em concatenation},\index{concatenation} i.e., the
finite sequence $h$ determined by the
conditions:
\begin{eqnarray*}
& |h| = |f|+|g|;&  \\
& h_n = f_n & \mbox{\ for\ } n < |f|;  \\
& h_{|f|+n} = g_n & \mbox{\ for\ } n < |g|.
\end{eqnarray*}

\subsection{Constants, Variables, and Expressions}

A formal system has a set of {\em symbols}\index{symbol!in
a formal system} denoted
by $\mbox{\em SM}$.  A
precise set-theo\-ret\-i\-cal definition of this set is unimportant; a symbol
could be considered a primitive or atomic element if we wish.  We assume this
set is divided into two disjoint subsets:  a set $\mbox{\em CN}$ of {\em
constants}\index{constant!in a formal system} and a set $\mbox{\em VR}$ of
{\em variables}.\index{variable!in a formal system}  $\mbox{\em CN}$ and
$\mbox{\em VR}$ are each assumed to consist of countably many symbols which
may be arranged in finite or simple infinite sequences $c_0, c_1, \ldots$ and
$v_0, v_1, \ldots$ respectively, without repeating terms.  We will represent
arbitrary symbols by metavariables $\alpha$, $\beta$, etc.

{\footnotesize\begin{quotation}
{\em Comment.} The variables $v_0, v_1, \ldots$ of our formal system
correspond to what are usually considered ``metavariables'' in
descriptions of specific formal systems in the literature.  Typically,
when describing a specific formal system a book will postulate a set of
primitive objects called variables, then proceed to describe their
properties using metavariables that range over them, never mentioning
again the actual variables themselves.  Our formal system does not
mention these primitive variable objects at all but deals directly with
metavariables, as its primitive objects, from the start.  This is a
subtle but key distinction you should keep in mind, and it makes our
definition of ``formal system'' somewhat different from that typically
found in the literature.  (So, the $\alpha$, $\beta$, etc.\ above are
actually ``metametavariables'' when used to represent $v_0, v_1,
\ldots$.)
\end{quotation}}

Finite sequences all terms of which are symbols are called {\em
expressions}.\index{expression!in a formal system}  $\mbox{\em EX}$ is
the set of all expressions; thus
\begin{displaymath}
\mbox{\em EX} = \bigcup _{n \in \omega} {}^n \mbox{\em SM}.
\end{displaymath}

A {\em constant-prefixed expression}\index{constant-prefixed expression}
is an expression of non-zero length
whose first term is a constant.  We denote the set of all constant-prefixed
expressions by $\mbox{\em EX}_C = \{ e \in \mbox{\em EX} | ( |e| > 0 \wedge
e_0 \in \mbox{\em CN} ) \}$.

A {\em constant-variable pair}\index{constant-variable pair}
is an expression of length 2 whose first term
is a constant and whose second term is a variable.  We denote the set of all
constant-variable pairs by $\mbox{\em EX}_2 = \{ e \in \mbox{\em EX}_C | ( |e|
= 2 \wedge e_1 \in \mbox{\em VR} ) \}$.


{\footnotesize\begin{quotation}
{\em Relationship to Metamath.} In general, the set $\mbox{\em SM}$
corresponds to the set of declared math symbols in a Metamath database, the
set $\mbox{\em CN}$ to those declared with \texttt{\$c} statements, and the set
$\mbox{\em VR}$ to those declared with \texttt{\$v} statements.  Of course a
Metamath database can only have a finite number of math symbols, whereas
formal systems in general can have an infinite number, although the number of
Metamath math symbols available is in principle unlimited.

The set $\mbox{\em EX}_C$ corresponds to the set of permissible expressions
for \texttt{\$e}, \texttt{\$a}, and \texttt{\$p} statements.  The set $\mbox{\em EX}_2$
corresponds to the set of permissible expressions for \texttt{\$f} statements.
\end{quotation}}

We denote by ${\cal V}(e)$ the set of all variables in an expression $e \in
\mbox{\em EX}$, i.e.\ the set of all $\alpha \in \mbox{\em VR}$ such that
$\alpha = e_n$ for some $n < |e|$.  We also denote (with abuse of notation) by
${\cal V}(E)$ the set of all variables in a collection of expressions $E
\subseteq \mbox{\em EX}$, i.e.\ $\bigcup _{e \in E} {\cal V}(e)$.


\subsection{Substitution}

Given a function $F$ from $\mbox{\em VR}$ to
$\mbox{\em EX}$, we
denote by $\sigma_{F}$ or just $\sigma$ the function from $\mbox{\em EX}$ to
$\mbox{\em EX}$ defined recursively for nonempty sequences by
\begin{eqnarray*}
& \sigma(<\alpha>) = F(\alpha) & \mbox{for\ } \alpha \in \mbox{\em VR}; \\
& \sigma(<\alpha>) = <\alpha> & \mbox{for\ } \alpha \not\in \mbox{\em VR}; \\
& \sigma(g \frown h) = \sigma(g) \frown
    \sigma(h) & \mbox{for\ } g,h \in \mbox{\em EX}.
\end{eqnarray*}
We also define $\sigma(\varnothing)=\varnothing$.  We call $\sigma$ a {\em
simultaneous substitution}\index{substitution!variable}\index{variable
substitution} (or just {\em substitution}) with {\em substitution
map}\index{substitution map} $F$.

We also denote (with abuse of notation) by $\sigma(E)$ a substitution on a
collection of expressions $E \subseteq \mbox{\em EX}$, i.e.\ the set $\{
\sigma(e) | e \in E \}$.  The collection $\sigma(E)$ may of course contain
fewer expressions than $E$ because duplicate expressions could result from the
substitution.

\subsection{Statements}

We denote by $\mbox{\em DV}$ the set of all
unordered pairs $\{\alpha, \beta \} \subseteq \mbox{\em VR}$ such that $\alpha
\neq \beta$.  $\mbox{\em DV}$ stands for ``distinct variables.''

A {\em pre-statement}\index{pre-statement!in a formal system} is a
quadruple $\langle D,T,H,A \rangle$ such that
$D\subseteq \mbox{\em DV}$, $T\subseteq \mbox{\em EX}_2$, $H\subseteq
\mbox{\em EX}_C$ and $H$ is finite,
$A\in \mbox{\em EX}_C$, ${\cal V}(H\cup\{A\}) \subseteq
{\cal V}(T)$, and $\forall e,f\in T {\ } {\cal V}(e) \neq {\cal V}(f)$ (or
equivalently, $e_1 \ne f_1$) whenever $e \neq f$. The terms of the quadruple are called {\em
distinct-variable restrictions},\index{disjoint-variable restriction!in a
formal system} {\em variable-type hypotheses},\index{variable-type
hypothesis!in a formal system} {\em logical hypotheses},\index{logical
hypothesis!in a formal system} and the {\em assertion}\index{assertion!in a
formal system} respectively.  We denote by $T_M$ ({\em mandatory variable-type
hypotheses}\index{mandatory variable-type hypothesis!in a formal system}) the
subset of $T$ such that ${\cal V}(T_M) ={\cal V}(H \cup \{A\})$.  We denote by
$D_M=\{\{\alpha,\beta\}\in D|\{\alpha,\beta\}\subseteq {\cal V}(T_M)\}$ the
{\em mandatory distinct-variable restrictions}\index{mandatory
disjoint-variable restriction!in a formal system} of the pre-statement.
The set
of {\em mandatory hypotheses}\index{mandatory hypothesis!in a formal system}
is $T_M\cup H$.  We call the quadruple $\langle D_M,T_M,H,A \rangle$
the {\em reduct}\index{reduct!in a formal system} of
the pre-statement $\langle D,T,H,A \rangle$.

A {\em statement} is the reduct of some pre-statement\index{statement!in a
formal system}.  A statement is therefore a special kind of pre-statement;
in particular, a statement is the reduct of itself.

{\footnotesize\begin{quotation}
{\em Comment.}  $T$ is a set of expressions, each of length 2, that associate
a set of constants (``variable types'') with a set of variables.  The
condition ${\cal V}(H\cup\{A\}) \subseteq {\cal V}(T) $
means that each variable occurring in a statement's logical
hypotheses or assertion must have an associated variable-type hypothesis or
``type declaration,'' in  analogy to a computer programming language, where a
variable must be declared to be say, a string or an integer.  The requirement
that $\forall e,f\in T \, e_1 \ne f_1$ for $e\neq f$
means that each variable must be
associated with a unique constant designating its variable type; e.g., a
variable might be a ``wff'' or a ``set'' but not both.

Distinct-variable restrictions are used to specify what variable substitutions
are permissible to make for the statement to remain valid.  For example, in
the theorem scheme of set theory $\lnot\forall x\,x=y$ we may not substitute
the same variable for both $x$ and $y$.  On the other hand, the theorem scheme
$x=y\to y=x$ does not require that $x$ and $y$ be distinct, so we do not
require a distinct-variable restriction, although having one
would cause no harm other than making the scheme less general.

A mandatory variable-type hypothesis is one whose variable exists in a logical
hypothesis or the assertion.  A provable pre-statement
(defined below) may require
non-mandatory variable-type hypotheses that effectively introduce ``dummy''
variables for use in its proof.  Any number of dummy variables might
be required by a specific proof; indeed, it has been shown by H.\
Andr\'{e}ka\index{Andr{\'{e}}ka, H.} \cite{Nemeti} that there is no finite
upper bound to the number of dummy variables needed to prove an arbitrary
theorem in first-order logic (with equality) having a fixed number $n>2$ of
individual variables.  (See also the Comment on p.~\pageref{nodd}.)
For this reason we do not set a finite size bound on the collections $D$ and
$T$, although in an actual application (Metamath database) these will of
course be finite, increased to whatever size is necessary as more
proofs are added.
\end{quotation}}

{\footnotesize\begin{quotation}
{\em Relationship to Metamath.} A pre-statement of a formal system
corresponds to an extended frame in a Metamath database
(Section~\ref{frames}).  The collections $D$, $T$, and $H$ correspond
respectively to the \texttt{\$d}, \texttt{\$f}, and \texttt{\$e}
statement collections in an extended frame.  The expression $A$
corresponds to the \texttt{\$a} (or \texttt{\$p}) statement in an
extended frame.

A statement of a formal system corresponds to a frame in a Metamath
database.
\end{quotation}}

\subsection{Formal Systems}

A {\em formal system}\index{formal system} is a
triple $\langle \mbox{\em CN},\mbox{\em
VR},\Gamma\rangle$ where $\Gamma$ is a set of statements.  The members of
$\Gamma$ are called {\em axiomatic statements}.\index{axiomatic
statement!in a formal system}  Sometimes we will refer to a
formal system by just $\Gamma$ when $\mbox{\em CN}$ and $\mbox{\em VR}$ are
understood.

Given a formal system $\Gamma$, the {\em closure}\index{closure}\footnote{This
definition of closure incorporates a simplification due to
Josh Purinton.\index{Purinton, Josh}.} of a
pre-statement
$\langle D,T,H,A \rangle$ is the smallest set $C$ of expressions
such that:
%\begin{enumerate}
%  \item $T\cup H\subseteq C$; and
%  \item If for some axiomatic statement
%    $\langle D_M',T_M',H',A' \rangle \in \Gamma_A$, for
%    some $E \subseteq C$, some $F \subseteq C-T$ (where ``-'' denotes
%    set difference), and some substitution
%    $\sigma$ we have
%    \begin{enumerate}
%       \item $\sigma(T_M') = E$ (where, as above, the $M$ denotes the
%           mandatory variable-type hypotheses of $T^A$);
%       \item $\sigma(H') = F$;
%       \item for all $\{\alpha,\beta\}\in D^A$ and $\subseteq
%         {\cal V}(T_M')$, for all $\gamma\in {\cal V}(\sigma(\langle \alpha
%         \rangle))$, and for all $\delta\in  {\cal V}(\sigma(\langle \beta
%         \rangle))$, we have $\{\gamma, \delta\} \in D$;
%   \end{enumerate}
%   then $\sigma(A') \in C$.
%\end{enumerate}
\begin{list}{}{\itemsep 0.0pt}
  \item[1.] $T\cup H\subseteq C$; and
  \item[2.] If for some axiomatic statement
    $\langle D_M',T_M',H',A' \rangle \in
       \Gamma$ and for some substitution
    $\sigma$ we have
    \begin{enumerate}
       \item[a.] $\sigma(T_M' \cup H') \subseteq C$; and
       \item[b.] for all $\{\alpha,\beta\}\in D_M'$, for all $\gamma\in
         {\cal V}(\sigma(\langle \alpha
         \rangle))$, and for all $\delta\in  {\cal V}(\sigma(\langle \beta
         \rangle))$, we have $\{\gamma, \delta\} \in D$;
   \end{enumerate}
   then $\sigma(A') \in C$.
\end{list}
A pre-statement $\langle D,T,H,A
\rangle$ is {\em provable}\index{provable statement!in a formal
system} if $A\in C$ i.e.\ if its assertion belongs to its
closure.  A statement is {\em provable} if it is
the reduct of a provable pre-statement.
The {\em universe}\index{universe of a formal system}
of a formal system is
the collection of all of its provable statements.  Note that the
set of axiomatic statements $\Gamma$ in a formal system is a subset of its
universe.

{\footnotesize\begin{quotation}
{\em Comment.} The first condition in the definition of closure simply says
that the hypotheses of the pre-statement are in its closure.

Condition 2(a) says that a substitution exists that makes the
mandatory hypotheses of an axiomatic statement exactly match some members of
the closure.  This is what we explicitly demonstrate in a Metamath language
proof.

%Conditions 2(a) and 2(b) say that a substitution exists that makes the
%(mandatory) hypotheses of an axiomatic statement exactly match some members of
%the closure.  This is what we explicitly demonstrate with a Metamath language
%proof.
%
%The set of expressions $F$ in condition 2(b) excludes the variable-type
%hypotheses; this is done because non-mandatory variable-type hypotheses are
%effectively ``dropped'' as irrelevant whereas logical hypotheses must be
%retained to achieve a consistent logical system.

Condition 2(b) describes how distinct-variable restrictions in the axiomatic
statement must be met.  It means that after a substitution for two variables
that must be distinct, the resulting two expressions must either contain no
variables, or if they do, they may not have variables in common, and each pair
of any variables they do have, with one variable from each expression, must be
specified as distinct in the original statement.
\end{quotation}}

{\footnotesize\begin{quotation}
{\em Relationship to Metamath.} Axiomatic statements
 and provable statements in a formal
system correspond to the frames for \texttt{\$a} and \texttt{\$p} statements
respectively in a Metamath database.  The set of axiomatic statements is a
subset of the set of provable statements in a formal system, although in a
Metamath database a \texttt{\$a} statement is distinguished by not having a
proof.  A Metamath language proof for a \texttt{\$p} statement tells the computer
how to explicitly construct a series of members of the closure ultimately
leading to a demonstration that the assertion
being proved is in the closure.  The actual closure typically contains
an infinite number of expressions.  A formal system itself does not have
an explicit object called a ``proof'' but rather the existence of a proof
is implied indirectly by membership of an assertion in a provable
statement's closure.  We do this to make the formal system easier
to describe in the language of set theory.

We also note that once established as provable, a statement may be considered
to acquire the same status as an axiomatic statement, because if the set of
axiomatic statements is extended with a provable statement, the universe of
the formal system remains unchanged (provided that $\mbox{\em VR}$ is
infinite).
In practice, this means we can build a hierarchy of provable statements to
more efficiently establish additional provable statements.  This is
what we do in Metamath when we allow proofs to reference previous
\texttt{\$p} statements as well as previous \texttt{\$a} statements.
\end{quotation}}

\section{Examples of Formal Systems}

{\footnotesize\begin{quotation}
{\em Relationship to Metamath.} The examples in this section, except Example~2,
are for the most part exact equivalents of the development in the set
theory database \texttt{set.mm}.  You may want to compare Examples~1, 3, and 5
to Section~\ref{metaaxioms}, Example 4 to Sections~\ref{metadefprop} and
\ref{metadefpred}, and Example 6 to
Section~\ref{setdefinitions}.\label{exampleref}
\end{quotation}}

\subsection{Example~1---Propositional Calculus}\index{propositional calculus}

Classical propositional calculus can be described by the following formal
system.  We assume the set of variables is infinite.  Rather than denoting the
constants and variables by $c_0, c_1, \ldots$ and $v_0, v_1, \ldots$, for
readability we will instead use more conventional symbols, with the
understanding of course that they denote distinct primitive objects.
Also for readability we may omit commas between successive terms of a
sequence; thus $\langle \mbox{wff\ } \varphi\rangle$ denotes
$\langle \mbox{wff}, \varphi\rangle$.

Let
\begin{itemize}
  \item[] $\mbox{\em CN}=\{\mbox{wff}, \vdash, \to, \lnot, (,)\}$
  \item[] $\mbox{\em VR}=\{\varphi,\psi,\chi,\ldots\}$
  \item[] $T = \{\langle \mbox{wff\ } \varphi\rangle,
             \langle \mbox{wff\ } \psi\rangle,
             \langle \mbox{wff\ } \chi\rangle,\ldots\}$, i.e.\ those
             expressions of length 2 whose first member is $\mbox{\rm wff}$
             and whose second member belongs to $\mbox{\em VR}$.\footnote{For
convenience we let $T$ be an infinite set; the definition of a statement
permits this in principle.  Since a Metamath source file has a finite size, in
practice we must of course use appropriate finite subsets of this $T$,
specifically ones containing at least the mandatory variable-type
hypotheses.  Similarly, in the source file we introduce new variables as
required, with the understanding that a potentially infinite number of
them are available.}
\noindent Then $\Gamma$ consists of the axiomatic statements that
are the reducts of the following pre-statements:
    \begin{itemize}
      \item[] $\langle\varnothing,T,\varnothing,
               \langle \mbox{wff\ }(\varphi\to\psi)\rangle\rangle$
      \item[] $\langle\varnothing,T,\varnothing,
               \langle \mbox{wff\ }\lnot\varphi\rangle\rangle$
      \item[] $\langle\varnothing,T,\varnothing,
               \langle \vdash(\varphi\to(\psi\to\varphi))
               \rangle\rangle$
      \item[] $\langle\varnothing,T,
               \varnothing,
               \langle \vdash((\varphi\to(\psi\to\chi))\to
               ((\varphi\to\psi)\to(\varphi\to\chi)))
               \rangle\rangle$
      \item[] $\langle\varnothing,T,
               \varnothing,
               \langle \vdash((\lnot\varphi\to\lnot\psi)\to
               (\psi\to\varphi))\rangle\rangle$
      \item[] $\langle\varnothing,T,
               \{\langle\vdash(\varphi\to\psi)\rangle,
                 \langle\vdash\varphi\rangle\},
               \langle\vdash\psi\rangle\rangle$
    \end{itemize}
\end{itemize}

(For example, the reduct of $\langle\varnothing,T,\varnothing,
               \langle \mbox{wff\ }(\varphi\to\psi)\rangle\rangle$
is
\begin{itemize}
\item[] $\langle\varnothing,
\{\langle \mbox{wff\ } \varphi\rangle,
             \langle \mbox{wff\ } \psi\rangle\},
             \varnothing,
               \langle \mbox{wff\ }(\varphi\to\psi)\rangle\rangle$,
\end{itemize}
which is the first axiomatic statement.)

We call the members of $\mbox{\em VR}$ {\em wff variables} or (in the context
of first-order logic which we will describe shortly) {\em wff metavariables}.
Note that the symbols $\phi$, $\psi$, etc.\ denote actual specific members of
$\mbox{\em VR}$; they are not metavariables of our expository language (which
we denote with $\alpha$, $\beta$, etc.) but are instead (meta)constant symbols
(members of $\mbox{\em SM}$) from the point of view of our expository
language.  The equivalent system of propositional calculus described in
\cite{Tarski1965} also uses the symbols $\phi$, $\psi$, etc.\ to denote wff
metavariables, but in \cite{Tarski1965} unlike here those are metavariables of
the expository language and not primitive symbols of the formal system.

The first two statements define wffs: if $\varphi$ and $\psi$ are wffs, so is
$(\varphi \to \psi)$; if $\varphi$ is a wff, so is $\lnot\varphi$. The next
three are the axioms of propositional calculus: if $\varphi$ and $\psi$ are
wffs, then $\vdash (\varphi \to (\psi \to \varphi))$ is an (axiomatic)
theorem; etc. The
last is the rule of modus ponens: if $\varphi$ and $\psi$ are wffs, and
$\vdash (\varphi\to\psi)$ and $\vdash \varphi$ are theorems, then $\vdash
\psi$ is a theorem.

The correspondence to ordinary propositional calculus is as follows.  We
consider only provable statements of the form $\langle\varnothing,
T,\varnothing,A\rangle$ with $T$ defined as above.  The first term of the
assertion $A$ of any such statement is either ``wff'' or ``$\vdash$''.  A
statement for which the first term is ``wff'' is a {\em wff} of propositional
calculus, and one where the first term is ``$\vdash$'' is a {\em
theorem (scheme)} of propositional calculus.

The universe of this formal system also contains many other provable
statements.  Those with distinct-variable restrictions are irrelevant because
propositional calculus has no constraints on substitutions.  Those that have
logical hypotheses we call {\em inferences}\index{inference} when
the logical hypotheses are of the form
$\langle\vdash\rangle\frown w$ where $w$ is a wff (with the leading constant
term ``wff'' removed).  Inferences (other than the modus ponens rule) are not a
proper part of propositional calculus but are convenient to use when building a
hierarchy of provable statements.  A provable statement with a nonsense
hypothesis such as $\langle \to,\vdash,\lnot\rangle$, and this same expression
as its assertion, we consider irrelevant; no use can be made of it in
proving theorems, since there is no way to eliminate the nonsense hypothesis.

{\footnotesize\begin{quotation}
{\em Comment.} Our use of parentheses in the definition of a wff illustrates
how axiomatic statements should be carefully stated in a way that
ties in unambiguously with the substitutions allowed by the formal system.
There are many ways we could have defined wffs---for example, Polish
prefix notation would have allowed us to omit parentheses entirely, at
the expense of readability---but we must define them in a way that is
unambiguous.  For example, if we had omitted parentheses from the
definition of $(\varphi\to \psi)$, the wff $\lnot\varphi\to \psi$ could
be interpreted as either $\lnot(\varphi\to\psi)$ or $(\lnot\varphi\to\psi)$
and would have allowed us to prove nonsense.  Note that there is no
concept of operator binding precedence built into our formal system.
\end{quotation}}

\begin{sloppy}
\subsection{Example~2---Predicate Calculus with Equality}\index{predicate
calculus}
\end{sloppy}

Here we extend Example~1 to include predicate calculus with equality,
illustrating the use of distinct-variable restrictions.  This system is the
same as Tarski's system $\mathfrak{S}_2$ in \cite{Tarski1965} (except that the
axioms of propositional calculus are different but equivalent, and a redundant
axiom is omitted).  We extend $\mbox{\em CN}$ with the constants
$\{\mbox{var},\forall,=\}$.  We extend $\mbox{\em VR}$ with an infinite set of
{\em individual metavariables}\index{individual
metavariable} $\{x,y,z,\ldots\}$ and denote this subset
$\mbox{\em Vr}$.

We also join to $\mbox{\em CN}$ a possibly infinite set $\mbox{\em Pr}$ of {\em
predicates} $\{R,S,\ldots\}$.  We associate with $\mbox{\em Pr}$ a function
$\mbox{rnk}$ from $\mbox{\em Pr}$ to $\omega$, and for $\alpha\in \mbox{\em
Pr}$ we call $\mbox{rnk}(\alpha)$ the {\em rank} of the predicate $\alpha$,
which is simply the number of ``arguments'' that the predicate has.  (Most
applications of predicate calculus will have a finite number of predicates;
for example, set theory has the single two-argument or binary predicate $\in$,
which is usually written with its arguments surrounding the predicate symbol
rather than with the prefix notation we will use for the general case.)  As a
device to facilitate our discussion, we will let $\mbox{\em Vs}$ be any fixed
one-to-one function from $\omega$ to $\mbox{\em Vr}$; thus $\mbox{\em Vs}$ is
any simple infinite sequence of individual metavariables with no repeating
terms.

In this example we will not include the function symbols that are often part of
formalizations of predicate calculus.  Using metalogical arguments that are
beyond the scope of our discussion, it can be shown that our formalization is
equivalent when functions are introduced via appropriate definitions.

We extend the set $T$ defined in Example~1 with the expressions
$\{\langle \mbox{var\ } x\rangle,$ $ \langle \mbox{var\ } y\rangle, \langle
\mbox{var\ } z\rangle,\ldots\}$.  We extend the $\Gamma$ above
with the axiomatic statements that are the reducts of the following
pre-statements:
\begin{list}{}{\itemsep 0.0pt}
      \item[] $\langle\varnothing,T,\varnothing,
               \langle \mbox{wff\ }\forall x\,\varphi\rangle\rangle$
      \item[] $\langle\varnothing,T,\varnothing,
               \langle \mbox{wff\ }x=y\rangle\rangle$
      \item[] $\langle\varnothing,T,
               \{\langle\vdash\varphi\rangle\},
               \langle\vdash\forall x\,\varphi\rangle\rangle$
      \item[] $\langle\varnothing,T,\varnothing,
               \langle \vdash((\forall x(\varphi\to\psi)
                  \to(\forall x\,\varphi\to\forall x\,\psi))
               \rangle\rangle$
      \item[] $\langle\{\{x,\varphi\}\},T,\varnothing,
               \langle \vdash(\varphi\to\forall x\,\varphi)
               \rangle\rangle$
      \item[] $\langle\{\{x,y\}\},T,\varnothing,
               \langle \vdash\lnot\forall x\lnot x=y
               \rangle\rangle$
      \item[] $\langle\varnothing,T,\varnothing,
               \langle \vdash(x=z
                  \to(x=y\to z=y))
               \rangle\rangle$
      \item[] $\langle\varnothing,T,\varnothing,
               \langle \vdash(y=z
                  \to(x=y\to x=z))
               \rangle\rangle$
\end{list}
These are the axioms not involving predicate symbols. The first two statements
extend the definition of a wff.  The third is the rule of generalization.  The
fifth states, in effect, ``For a wff $\varphi$ and variable $x$,
$\vdash(\varphi\to\forall x\,\varphi)$, provided that $x$ does not occur in
$\varphi$.''  The sixth states ``For variables $x$ and $y$,
$\vdash\lnot\forall x\lnot x = y$, provided that $x$ and $y$ are distinct.''
(This proviso is not necessary but was included by Tarski to
weaken the axiom and still show that the system is logically complete.)

Finally, for each predicate symbol $\alpha\in \mbox{\em Pr}$, we add to
$\Gamma$ an axiomatic statement, extending the definition of wff,
that is the reduct of the following pre-statement:
\begin{displaymath}
    \langle\varnothing,T,\varnothing,
            \langle \mbox{wff},\alpha\rangle\
            \frown \mbox{\em Vs}\restriction\mbox{rnk}(\alpha)\rangle
\end{displaymath}
and for each $\alpha\in \mbox{\em Pr}$ and each $n < \mbox{rnk}(\alpha)$
we add to $\Gamma$ an equality axiom that is the reduct of the
following pre-statement:
\begin{eqnarray*}
    \lefteqn{\langle\varnothing,T,\varnothing,
            \langle
      \vdash,(,\mbox{\em Vs}_n,=,\mbox{\em Vs}_{\mbox{rnk}(\alpha)},\to,
            (,\alpha\rangle\frown \mbox{\em Vs}\restriction\mbox{rnk}(\alpha)} \\
  & & \frown
            \langle\to,\alpha\rangle\frown \mbox{\em Vs}\restriction n\frown
            \langle \mbox{\em Vs}_{\mbox{rnk}(\alpha)}\rangle \\
 & & \frown
            \mbox{\em Vs}\restriction(\mbox{rnk}(\alpha)\setminus(n+1))\frown
            \langle),)\rangle\rangle
\end{eqnarray*}
where $\restriction$ denotes function domain restriction and $\setminus$
denotes set difference.  Recall that a subscript on $\mbox{\em Vs}$
denotes one of its terms.  (In the above two axiom sets commas are placed
between successive terms of sequences to prevent ambiguity, and if you examine
them with care you will be able to distinguish those parentheses that denote
constant symbols from those of our expository language that delimit function
arguments.  Although it might have been better to use boldface for our
primitive symbols, unfortunately boldface was not available for all characters
on the \LaTeX\ system used to typeset this text.)  These seemingly forbidding
axioms can be understood by analogy to concatenation of substrings in a
computer language.  They are actually relatively simple for each specific case
and will become clearer by looking at the special case of a binary predicate
$\alpha = R$ where $\mbox{rnk}(R)=2$.  Letting $\mbox{\em Vs}$ be the sequence
$\langle x,y,z,\ldots\rangle$, the axioms we would add to $\Gamma$ for this
case would be the wff extension and two equality axioms that are the
reducts of the pre-statements:
\begin{list}{}{\itemsep 0.0pt}
      \item[] $\langle\varnothing,T,\varnothing,
               \langle \mbox{wff\ }R x y\rangle\rangle$
      \item[] $\langle\varnothing,T,\varnothing,
               \langle \vdash(x=z
                  \to(R x y \to R z y))
               \rangle\rangle$
      \item[] $\langle\varnothing,T,\varnothing,
               \langle \vdash(y=z
                  \to(R x y \to R x z))
               \rangle\rangle$
\end{list}
Study these carefully to see how the general axioms above evaluate to
them.  In practice, typically only a few special cases such as this would be
needed, and in any case the Metamath language will only permit us to describe
a finite number of predicates, as opposed to the infinite number permitted by
the formal system.  (If an infinite number should be needed for some reason,
we could not define the formal system directly in the Metamath language but
could instead define it metalogically under set theory as we
do in this appendix, and only the underlying set theory, with its single
binary predicate, would be defined directly in the Metamath language.)


{\footnotesize\begin{quotation}
{\em Comment.}  As we noted earlier, the specific variables denoted by the
symbols $x,y,z,\ldots\in \mbox{\em Vr}\subseteq \mbox{\em VR}\subseteq
\mbox{\em SM}$ in Example~2 are not the actual variables of ordinary predicate
calculus but should be thought of as metavariables ranging over them.  For
example, a distinct-variable restriction would be meaningless for actual
variables of ordinary predicate calculus since two different actual variables
are by definition distinct.  And when we talk about an arbitrary
representative $\alpha\in \mbox{\em Vr}$, $\alpha$ is a metavariable (in our
expository language) that ranges over metavariables (which are primitives of
our formal system) each of which ranges over the actual individual variables
of predicate calculus (which are never mentioned in our formal system).

The constant called ``var'' above is called \texttt{setvar} in the
\texttt{set.mm} database file, but it means the same thing.  I felt
that ``var'' is a more meaningful name in the context of predicate
calculus, whose use is not limited to set theory.  For consistency we
stick with the name ``var'' throughout this Appendix, even after set
theory is introduced.
\end{quotation}}

\subsection{Free Variables and Proper Substitution}\index{free variable}
\index{proper substitution}\index{substitution!proper}

Typical representations of mathematical axioms use concepts such
as ``free variable,'' ``bound variable,'' and ``proper substitution''
as primitive notions.
A free variable is a variable that
is not a parameter of any container expression.
A bound variable is the opposite of a free variable; it is a
a variable that has been bound in a container expression.
For example, in the expression $\forall x \varphi$ (for all $x$, $\varphi$
is true), the variable $x$
is bound within the for-all ($\forall$) expression.
It is possible to change one variable to another, and that process is called
``proper substitution.''
In most books, proper substitution has a somewhat complicated recursive
definition with multiple cases based on the occurrences of free and
bound variables.
You may consult
\cite[ch.\ 3--4]{Hamilton}\index{Hamilton, Alan G.} (as well as
many other texts) for more formal details about these terms.

Using these concepts as \texttt{primitives} creates complications
for computer implementations.

In the system of Example~2, there are no primitive notions of free variable
and proper substitution.  Tarski \cite{Tarski1965} shows that this system is
logically equivalent to the more typical textbook systems that do have these
primitive notions, if we introduce these notions with appropriate definitions
and metalogic.  We could also define axioms for such systems directly,
although the recursive definitions of free variable and proper substitution
would be messy and awkward to work with.  Instead, we mention two devices that
can be used in practice to mimic these notions.  (1) Instead of introducing
special notation to express (as a logical hypothesis) ``where $x$ is not free
in $\varphi$'' we can use the logical hypothesis $\vdash(\varphi\to\forall
x\,\varphi)$.\label{effectivelybound}\index{effectively
not free}\footnote{This is a slightly weaker requirement than ``where $x$ is
not free in $\varphi$.''  If we let $\varphi$ be $x=x$, we have the theorem
$(x=x\to\forall x\,x=x)$ which satisfies the hypothesis, even though $x$ is
free in $x=x$ .  In a case like this we say that $x$ is {\em effectively not
free}\index{effectively not free} in $x=x$, since $x=x$ is logically
equivalent to $\forall x\,x=x$ in which $x$ is bound.} (2) It can be shown
that the wff $((x=y\to\varphi)\wedge\exists x(x=y\wedge\varphi))$ (with the
usual definitions of $\wedge$ and $\exists$; see Example~4 below) is logically
equivalent to ``the wff that results from proper substitution of $y$ for $x$
in $\varphi$.''  This works whether or not $x$ and $y$ are distinct.

\subsection{Metalogical Completeness}\index{metalogical completeness}

In the system of Example~2, the
following are provable pre-statements (and their reducts are
provable statements):
\begin{eqnarray*}
      & \langle\{\{x,y\}\},T,\varnothing,
               \langle \vdash\lnot\forall x\lnot x=y
               \rangle\rangle & \\
     &  \langle\varnothing,T,\varnothing,
               \langle \vdash\lnot\forall x\lnot x=x
               \rangle\rangle &
\end{eqnarray*}
whereas the following pre-statement is not to my knowledge provable (but
in any case we will pretend it's not for sake of illustration):
\begin{eqnarray*}
     &  \langle\varnothing,T,\varnothing,
               \langle \vdash\lnot\forall x\lnot x=y
               \rangle\rangle &
\end{eqnarray*}
In other words, we can prove ``$\lnot\forall x\lnot x=y$ where $x$ and $y$ are
distinct'' and separately prove ``$\lnot\forall x\lnot x=x$'', but we can't
prove the combined general case ``$\lnot\forall x\lnot x=y$'' that has no
proviso.  Now this does not compromise logical completeness, because the
variables are really metavariables and the two provable cases together cover
all possible cases.  The third case can be considered a metatheorem whose
direct proof, using the system of Example~2, lies outside the capability of the
formal system.

Also, in the system of Example~2 the following pre-statement is not to my
knowledge provable (again, a conjecture that we will pretend to be the case):
\begin{eqnarray*}
     & \langle\varnothing,T,\varnothing,
               \langle \vdash(\forall x\, \varphi\to\varphi)
               \rangle\rangle &
\end{eqnarray*}
Instead, we can only prove specific cases of $\varphi$ involving individual
metavariables, and by induction on formula length, prove as a metatheorem
outside of our formal system the general statement above.  The details of this
proof are found in \cite{Kalish}.

There does, however, exist a system of predicate calculus in which all such
``simple metatheorems'' as those above can be proved directly, and we present
it in Example~3. A {\em simple metatheorem}\index{simple metatheorem}
is any statement of the formal
system of Example~2 where all distinct variable restrictions consist of either
two individual metavariables or an individual metavariable and a wff
metavariable, and which is provable by combining cases outside the system as
above.  A system is {\em metalogically complete}\index{metalogical
completeness} if all of its simple
metatheorems are (directly) provable statements. The precise definition of
``simple metatheorem'' and the proof of the ``metalogical completeness'' of
Example~3 is found in Remark 9.6 and Theorem 9.7 of \cite{Megill}.\index{Megill,
Norman}

\begin{sloppy}
\subsection{Example~3---Metalogically Complete Predicate
Calculus with
Equality}
\end{sloppy}

For simplicity we will assume there is one binary predicate $R$;
this system suffices for set theory, where the $R$ is of course the $\in$
predicate.  We label the axioms as they appear in \cite{Megill}.  This
system is logically equivalent to that of Example~2 (when the latter is
restricted to this single binary predicate) but is also metalogically
complete.\index{metalogical completeness}

Let
\begin{itemize}
  \item[] $\mbox{\em CN}=\{\mbox{wff}, \mbox{var}, \vdash, \to, \lnot, (,),\forall,=,R\}$.
  \item[] $\mbox{\em VR}=\{\varphi,\psi,\chi,\ldots\}\cup\{x,y,z,\ldots\}$.
  \item[] $T = \{\langle \mbox{wff\ } \varphi\rangle,
             \langle \mbox{wff\ } \psi\rangle,
             \langle \mbox{wff\ } \chi\rangle,\ldots\}\cup
       \{\langle \mbox{var\ } x\rangle, \langle \mbox{var\ } y\rangle, \langle
       \mbox{var\ }z\rangle,\ldots\}$.

\noindent Then
  $\Gamma$ consists of the reducts of the following pre-statements:
    \begin{itemize}
      \item[] $\langle\varnothing,T,\varnothing,
               \langle \mbox{wff\ }(\varphi\to\psi)\rangle\rangle$
      \item[] $\langle\varnothing,T,\varnothing,
               \langle \mbox{wff\ }\lnot\varphi\rangle\rangle$
      \item[] $\langle\varnothing,T,\varnothing,
               \langle \mbox{wff\ }\forall x\,\varphi\rangle\rangle$
      \item[] $\langle\varnothing,T,\varnothing,
               \langle \mbox{wff\ }x=y\rangle\rangle$
      \item[] $\langle\varnothing,T,\varnothing,
               \langle \mbox{wff\ }Rxy\rangle\rangle$
      \item[(C1$'$)] $\langle\varnothing,T,\varnothing,
               \langle \vdash(\varphi\to(\psi\to\varphi))
               \rangle\rangle$
      \item[(C2$'$)] $\langle\varnothing,T,
               \varnothing,
               \langle \vdash((\varphi\to(\psi\to\chi))\to
               ((\varphi\to\psi)\to(\varphi\to\chi)))
               \rangle\rangle$
      \item[(C3$'$)] $\langle\varnothing,T,
               \varnothing,
               \langle \vdash((\lnot\varphi\to\lnot\psi)\to
               (\psi\to\varphi))\rangle\rangle$
      \item[(C4$'$)] $\langle\varnothing,T,
               \varnothing,
               \langle \vdash(\forall x(\forall x\,\varphi\to\psi)\to
                 (\forall x\,\varphi\to\forall x\,\psi))\rangle\rangle$
      \item[(C5$'$)] $\langle\varnothing,T,
               \varnothing,
               \langle \vdash(\forall x\,\varphi\to\varphi)\rangle\rangle$
      \item[(C6$'$)] $\langle\varnothing,T,
               \varnothing,
               \langle \vdash(\forall x\forall y\,\varphi\to
                 \forall y\forall x\,\varphi)\rangle\rangle$
      \item[(C7$'$)] $\langle\varnothing,T,
               \varnothing,
               \langle \vdash(\lnot\varphi\to\forall x\lnot\forall x\,\varphi
                 )\rangle\rangle$
      \item[(C8$'$)] $\langle\varnothing,T,
               \varnothing,
               \langle \vdash(x=y\to(x=z\to y=z))\rangle\rangle$
      \item[(C9$'$)] $\langle\varnothing,T,
               \varnothing,
               \langle \vdash(\lnot\forall x\, x=y\to(\lnot\forall x\, x=z\to
                 (y=z\to\forall x\, y=z)))\rangle\rangle$
      \item[(C10$'$)] $\langle\varnothing,T,
               \varnothing,
               \langle \vdash(\forall x(x=y\to\forall x\,\varphi)\to
                 \varphi))\rangle\rangle$
      \item[(C11$'$)] $\langle\varnothing,T,
               \varnothing,
               \langle \vdash(\forall x\, x=y\to(\forall x\,\varphi
               \to\forall y\,\varphi))\rangle\rangle$
      \item[(C12$'$)] $\langle\varnothing,T,
               \varnothing,
               \langle \vdash(x=y\to(Rxz\to Ryz))\rangle\rangle$
      \item[(C13$'$)] $\langle\varnothing,T,
               \varnothing,
               \langle \vdash(x=y\to(Rzx\to Rzy))\rangle\rangle$
      \item[(C15$'$)] $\langle\varnothing,T,
               \varnothing,
               \langle \vdash(\lnot\forall x\, x=y\to(x=y\to(\varphi
                 \to\forall x(x=y\to\varphi))))\rangle\rangle$
      \item[(C16$'$)] $\langle\{\{x,y\}\},T,
               \varnothing,
               \langle \vdash(\forall x\, x=y\to(\varphi\to\forall x\,\varphi)
                 )\rangle\rangle$
      \item[(C5)] $\langle\{\{x,\varphi\}\},T,\varnothing,
               \langle \vdash(\varphi\to\forall x\,\varphi)
               \rangle\rangle$
      \item[(MP)] $\langle\varnothing,T,
               \{\langle\vdash(\varphi\to\psi)\rangle,
                 \langle\vdash\varphi\rangle\},
               \langle\vdash\psi\rangle\rangle$
      \item[(Gen)] $\langle\varnothing,T,
               \{\langle\vdash\varphi\rangle\},
               \langle\vdash\forall x\,\varphi\rangle\rangle$
    \end{itemize}
\end{itemize}

While it is known that these axioms are ``metalogically complete,'' it is
not known whether they are independent (i.e.\ none is
redundant) in the metalogical sense; specifically, whether any axiom (possibly
with additional non-mandatory distinct-variable restrictions, for use with any
dummy variables in its proof) is provable from the others.  Note that
metalogical independence is a weaker requirement than independence in the
usual logical sense.  Not all of the above axioms are logically independent:
for example, C9$'$ can be proved as a metatheorem from the others, outside the
formal system, by combining the possible cases of distinct variables.

\subsection{Example~4---Adding Definitions}\index{definition}
There are several ways to add definitions to a formal system.  Probably the
most proper way is to consider definitions not as part of the formal system at
all but rather as abbreviations that are part of the expository metalogic
outside the formal system.  For convenience, though, we may use the formal
system itself to incorporate definitions, adding them as axiomatic extensions
to the system.  This could be done by adding a constant representing the
concept ``is defined as'' along with axioms for it. But there is a nicer way,
at least in this writer's opinion, that introduces definitions as direct
extensions to the language rather than as extralogical primitive notions.  We
introduce additional logical connectives and provide axioms for them.  For
systems of logic such as Examples 1 through 3, the additional axioms must be
conservative in the sense that no wff of the original system that was not a
theorem (when the initial term ``wff'' is replaced by ``$\vdash$'' of course)
becomes a theorem of the extended system.  In this example we extend Example~3
(or 2) with standard abbreviations of logic.

We extend $\mbox{\em CN}$ of Example~3 with new constants $\{\leftrightarrow,
\wedge,\vee,\exists\}$, corresponding to logical equivalence,\index{logical
equivalence ($\leftrightarrow$)}\index{biconditional ($\leftrightarrow$)}
conjunction,\index{conjunction ($\wedge$)} disjunction,\index{disjunction
($\vee$)} and the existential quantifier.\index{existential quantifier
($\exists$)}  We extend $\Gamma$ with the axiomatic statements that are
the reducts of the following pre-statements:
\begin{list}{}{\itemsep 0.0pt}
      \item[] $\langle\varnothing,T,\varnothing,
               \langle \mbox{wff\ }(\varphi\leftrightarrow\psi)\rangle\rangle$
      \item[] $\langle\varnothing,T,\varnothing,
               \langle \mbox{wff\ }(\varphi\vee\psi)\rangle\rangle$
      \item[] $\langle\varnothing,T,\varnothing,
               \langle \mbox{wff\ }(\varphi\wedge\psi)\rangle\rangle$
      \item[] $\langle\varnothing,T,\varnothing,
               \langle \mbox{wff\ }\exists x\, \varphi\rangle\rangle$
  \item[] $\langle\varnothing,T,\varnothing,
     \langle\vdash ( ( \varphi \leftrightarrow \psi ) \to
     ( \varphi \to \psi ) )\rangle\rangle$
  \item[] $\langle\varnothing,T,\varnothing,
     \langle\vdash ((\varphi\leftrightarrow\psi)\to
    (\psi\to\varphi))\rangle\rangle$
  \item[] $\langle\varnothing,T,\varnothing,
     \langle\vdash ((\varphi\to\psi)\to(
     (\psi\to\varphi)\to(\varphi
     \leftrightarrow\psi)))\rangle\rangle$
  \item[] $\langle\varnothing,T,\varnothing,
     \langle\vdash (( \varphi \wedge \psi ) \leftrightarrow\neg ( \varphi
     \to \neg \psi )) \rangle\rangle$
  \item[] $\langle\varnothing,T,\varnothing,
     \langle\vdash (( \varphi \vee \psi ) \leftrightarrow (\neg \varphi
     \to \psi )) \rangle\rangle$
  \item[] $\langle\varnothing,T,\varnothing,
     \langle\vdash (\exists x \,\varphi\leftrightarrow
     \lnot \forall x \lnot \varphi)\rangle\rangle$
\end{list}
The first three logical axioms (statements containing ``$\vdash$'') introduce
and effectively define logical equivalence, ``$\leftrightarrow$''.  The last
three use ``$\leftrightarrow$'' to effectively mean ``is defined as.''

\subsection{Example~5---ZFC Set Theory}\index{ZFC set theory}

Here we add to the system of Example~4 the axioms of Zermelo--Fraenkel set
theory with Choice.  For convenience we make use of the
definitions in Example~4.

In the $\mbox{\em CN}$ of Example~4 (which extends Example~3), we replace the symbol $R$
with the symbol $\in$.
More explicitly, we remove from $\Gamma$ of Example~4 the three
axiomatic statements containing $R$ and replace them with the
reducts of the following:
\begin{list}{}{\itemsep 0.0pt}
      \item[] $\langle\varnothing,T,\varnothing,
               \langle \mbox{wff\ }x\in y\rangle\rangle$
      \item[] $\langle\varnothing,T,
               \varnothing,
               \langle \vdash(x=y\to(x\in z\to y\in z))\rangle\rangle$
      \item[] $\langle\varnothing,T,
               \varnothing,
               \langle \vdash(x=y\to(z\in x\to z\in y))\rangle\rangle$
\end{list}
Letting $D=\{\{\alpha,\beta\}\in \mbox{\em DV}\,|\alpha,\beta\in \mbox{\em
Vr}\}$ (in other words all individual variables must be distinct), we extend
$\Gamma$ with the ZFC axioms, called
\index{Axiom of Extensionality}
\index{Axiom of Replacement}
\index{Axiom of Union}
\index{Axiom of Power Sets}
\index{Axiom of Regularity}
\index{Axiom of Infinity}
\index{Axiom of Choice}
Extensionality, Replacement, Union, Power
Set, Regularity, Infinity, and Choice, that are the reducts of:
\begin{list}{}{\itemsep 0.0pt}
      \item[Ext] $\langle D,T,
               \varnothing,
               \langle\vdash (\forall x(x\in y\leftrightarrow x \in z)\to y
               =z) \rangle\rangle$
      \item[Rep] $\langle D,T,
               \varnothing,
               \langle\vdash\exists x ( \exists y \forall z (\varphi \to z = y
                        ) \to
                        \forall z ( z \in x \leftrightarrow \exists x ( x \in
                        y \wedge \forall y\,\varphi ) ) )\rangle\rangle$
      \item[Un] $\langle D,T,
               \varnothing,
               \langle\vdash \exists x \forall y ( \exists x ( y \in x \wedge
               x \in z ) \to y \in x ) \rangle\rangle$
      \item[Pow] $\langle D,T,
               \varnothing,
               \langle\vdash \exists x \forall y ( \forall x ( x \in y \to x
               \in z ) \to y \in x ) \rangle\rangle$
      \item[Reg] $\langle D,T,
               \varnothing,
               \langle\vdash (  x \in y \to
                 \exists x ( x \in y \wedge \forall z ( z \in x \to \lnot z
                \in y ) ) ) \rangle\rangle$
      \item[Inf] $\langle D,T,
               \varnothing,
               \langle\vdash \exists x(y\in x\wedge\forall y(y\in
               x\to
               \exists z(y \in z\wedge z\in x))) \rangle\rangle$
      \item[AC] $\langle D,T,
               \varnothing,
               \langle\vdash \exists x \forall y \forall z ( ( y \in z
               \wedge z \in w ) \to \exists w \forall y ( \exists w
              ( ( y \in z \wedge z \in w ) \wedge ( y \in w \wedge w \in x
              ) ) \leftrightarrow y = w ) ) \rangle\rangle$
\end{list}

\subsection{Example~6---Class Notation in Set Theory}\label{class}

A powerful device that makes set theory easier (and that we have
been using all along in our informal expository language) is {\em class
abstraction notation}.\index{class abstraction}\index{abstraction class}  The
definitions we introduce are rigorously justified
as conservative by Takeuti and Zaring \cite{Takeuti}\index{Takeuti, Gaisi} or
Quine \cite{Quine}\index{Quine, Willard Van Orman}.  The key idea is to
introduce the notation $\{x|\mbox{---}\}$ which means ``the class of all $x$
such that ---'' for abstraction classes and introduce (meta)variables that
range over them.  An abstraction class may or may not be a set, depending on
whether it exists (as a set).  A class that does not exist is
called a {\em proper class}.\index{proper class}\index{class!proper}

To illustrate the use of abstraction classes we will provide some examples
of definitions that make use of them:  the empty set, class union, and
unordered pair.  Many other such definitions can be found in the
Metamath set theory database,
\texttt{set.mm}.\index{set theory database (\texttt{set.mm})}

% We intentionally break up the sequence of math symbols here
% because otherwise the overlong line goes beyond the page in narrow mode.
We extend $\mbox{\em CN}$ of Example~5 with new symbols $\{$
$\mbox{class},$ $\{,$ $|,$ $\},$ $\varnothing,$ $\cup,$ $,$ $\}$
where the inner braces and last comma are
constant symbols. (As before,
our dual use of some mathematical symbols for both our expository
language and as primitives of the formal system should be clear from context.)

We extend $\mbox{\em VR}$ of Example~5 with a set of {\em class
variables}\index{class variable}
$\{A,B,C,\ldots\}$. We extend the $T$ of Example~5 with $\{\langle
\mbox{class\ } A\rangle, \langle \mbox{class\ }B\rangle, \langle \mbox{class\ }
C\rangle,\ldots\}$.

To
introduce our definitions,
we add to $\Gamma$ of Example~5 the axiomatic statements
that are the reducts of the following pre-statements:
\begin{list}{}{\itemsep 0.0pt}
      \item[] $\langle\varnothing,T,\varnothing,
               \langle \mbox{class\ }x\rangle\rangle$
      \item[] $\langle\varnothing,T,\varnothing,
               \langle \mbox{class\ }\{x|\varphi\}\rangle\rangle$
      \item[] $\langle\varnothing,T,\varnothing,
               \langle \mbox{wff\ }A=B\rangle\rangle$
      \item[] $\langle\varnothing,T,\varnothing,
               \langle \mbox{wff\ }A\in B\rangle\rangle$
      \item[Ab] $\langle\varnothing,T,\varnothing,
               \langle \vdash ( y \in \{ x |\varphi\} \leftrightarrow
                  ( ( x = y \to\varphi) \wedge \exists x ( x = y
                  \wedge\varphi) ))
               \rangle\rangle$
      \item[Eq] $\langle\{\{x,A\},\{x,B\}\},T,\varnothing,
               \langle \vdash ( A = B \leftrightarrow
               \forall x ( x \in A \leftrightarrow x \in B ) )
               \rangle\rangle$
      \item[El] $\langle\{\{x,A\},\{x,B\}\},T,\varnothing,
               \langle \vdash ( A \in B \leftrightarrow \exists x
               ( x = A \wedge x \in B ) )
               \rangle\rangle$
\end{list}
Here we say that an individual variable is a class; $\{x|\varphi\}$ is a
class; and we extend the definition of a wff to include class equality and
membership.  Axiom Ab defines membership of a variable in a class abstraction;
the right-hand side can be read as ``the wff that results from proper
substitution of $y$ for $x$ in $\varphi$.''\footnote{Note that this definition
makes unnecessary the introduction of a separate notation similar to
$\varphi(x|y)$ for proper substitution, although we may choose to do so to be
conventional.  Incidentally, $\varphi(x|y)$ as it stands would be ambiguous in
the formal systems of our examples, since we wouldn't know whether
$\lnot\varphi(x|y)$ meant $\lnot(\varphi(x|y))$ or $(\lnot\varphi)(x|y)$.
Instead, we would have to use an unambiguous variant such as $(\varphi\,
x|y)$.}  Axioms Eq and El extend the meaning of the existing equality and
membership connectives.  This is potentially dangerous and requires careful
justification.  For example, from Eq we can derive the Axiom of Extensionality
with predicate logic alone; thus in principle we should include the Axiom of
Extensionality as a logical hypothesis.  However we do not bother to do this
since we have already presupposed that axiom earlier. The distinct variable
restrictions should be read ``where $x$ does not occur in $A$ or $B$.''  We
typically do this when the right-hand side of a definition involves an
individual variable not in the expression being defined; it is done so that
the right-hand side remains independent of the particular ``dummy'' variable
we use.

We continue to add to $\Gamma$ the following definitions
(i.e. the reducts of the following pre-statements) for empty
set,\index{empty set} class union,\index{union} and unordered
pair.\index{unordered pair}  They should be self-explanatory.  Analogous to our
use of ``$\leftrightarrow$'' to define new wffs in Example~4, we use ``$=$''
to define new abstraction terms, and both may be read informally as ``is
defined as'' in this context.
\begin{list}{}{\itemsep 0.0pt}
      \item[] $\langle\varnothing,T,\varnothing,
               \langle \mbox{class\ }\varnothing\rangle\rangle$
      \item[] $\langle\varnothing,T,\varnothing,
               \langle \vdash \varnothing = \{ x | \lnot x = x \}
               \rangle\rangle$
      \item[] $\langle\varnothing,T,\varnothing,
               \langle \mbox{class\ }(A\cup B)\rangle\rangle$
      \item[] $\langle\{\{x,A\},\{x,B\}\},T,\varnothing,
               \langle \vdash ( A \cup B ) = \{ x | ( x \in A \vee x \in B ) \}
               \rangle\rangle$
      \item[] $\langle\varnothing,T,\varnothing,
               \langle \mbox{class\ }\{A,B\}\rangle\rangle$
      \item[] $\langle\{\{x,A\},\{x,B\}\},T,\varnothing,
               \langle \vdash \{ A , B \} = \{ x | ( x = A \vee x = B ) \}
               \rangle\rangle$
\end{list}

\section{Metamath as a Formal System}\label{theorymm}

This section presupposes a familiarity with the Metamath computer language.

Our theory describes formal systems and their universes.  The Metamath
language provides a way of representing these set-theoretical objects to
a computer.  A Metamath database, being a finite set of {\sc ascii}
characters, can usually describe only a subset of a formal system and
its universe, which are typically infinite.  However the database can
contain as large a finite subset of the formal system and its universe
as we wish.  (Of course a Metamath set theory database can, in
principle, indirectly describe an entire infinite formal system by
formalizing the expository language in this Appendix.)

For purpose of our discussion, we assume the Metamath database
is in the simple form described on p.~\pageref{framelist},
consisting of all constant and variable declarations at the beginning,
followed by a sequence of extended frames each
delimited by \texttt{\$\char`\{} and \texttt{\$\char`\}}.  Any Metamath database can
be converted to this form, as described on p.~\pageref{frameconvert}.

The math symbol tokens of a Metamath source file, which are declared
with \texttt{\$c} and \texttt{\$v} statements, are names we assign to
representatives of $\mbox{\em CN}$ and $\mbox{\em VR}$.  For
definiteness we could assume that the first math symbol declared as a
variable corresponds to $v_0$, the second to $v_1$, etc., although the
exact correspondence we choose is not important.

In the Metamath language, each \texttt{\$d}, \texttt{\$f}, and
 \texttt{\$e} source
statement in an extended frame (Section~\ref{frames})
corresponds respectively to a member of the
collections $D$, $T$, and $H$ in a formal system statement $\langle
D_M,T_M,H,A\rangle$.  The math symbol strings following these Metamath keywords
correspond to a variable pair (in the case of \texttt{\$d}) or an expression (for
the other two keywords). The math symbol string following a \texttt{\$a} source
statement corresponds to expression $A$ in an axiomatic statement of the
formal system; the one following a \texttt{\$p} source statement corresponds to
$A$ in a provable statement that is not axiomatic.  In other words, each
extended frame in a Metamath database corresponds to
a pre-statement of the formal system, and a frame corresponds to
a statement of the formal system.  (Don't confuse the two meanings of
``statement'' here.  A statement of the formal system corresponds to the
several statements in a Metamath database that may constitute a
frame.)

In order for the computer to verify that a formal system statement is
provable, each \texttt{\$p} source statement is accompanied by a proof.
However, the proof does not correspond to anything in the formal system
but is simply a way of communicating to the computer the information
needed for its verification.  The proof tells the computer {\em how to
construct} specific members of closure of the formal system
pre-statement corresponding to the extended frame of the \texttt{\$p}
statement.  The final result of the construction is the member of the
closure that matches the \texttt{\$p} statement.  The abstract formal
system, on the other hand, is concerned only with the {\em existence} of
members of the closure.

As mentioned on p.~\pageref{exampleref}, Examples 1 and 3--6 in the
previous Section parallel the development of logic and set theory in the
Metamath database
\texttt{set.mm}.\index{set theory database (\texttt{set.mm})} You may
find it instructive to compare them.


\chapter{The MIU System}
\label{MIU}
\index{formal system}
\index{MIU-system}

The following is a listing of the file \texttt{miu.mm}.  It is self-explanatory.

%%%%%%%%%%%%%%%%%%%%%%%%%%%%%%%%%%%%%%%%%%%%%%%%%%%%%%%%%%%%

\begin{verbatim}
$( The MIU-system:  A simple formal system $)

$( Note:  This formal system is unusual in that it allows
empty wffs.  To work with a proof, you must type
SET EMPTY_SUBSTITUTION ON before using the PROVE command.
By default, this is OFF in order to reduce the number of
ambiguous unification possibilities that have to be selected
during the construction of a proof.  $)

$(
Hofstadter's MIU-system is a simple example of a formal
system that illustrates some concepts of Metamath.  See
Douglas R. Hofstadter, _Goedel, Escher, Bach:  An Eternal
Golden Braid_ (Vintage Books, New York, 1979), pp. 33ff. for
a description of the MIU-system.

The system has 3 constant symbols, M, I, and U.  The sole
axiom of the system is MI. There are 4 rules:
     Rule I:  If you possess a string whose last letter is I,
     you can add on a U at the end.
     Rule II:  Suppose you have Mx.  Then you may add Mxx to
     your collection.
     Rule III:  If III occurs in one of the strings in your
     collection, you may make a new string with U in place
     of III.
     Rule IV:  If UU occurs inside one of your strings, you
     can drop it.
Unfortunately, Rules III and IV do not have unique results:
strings could have more than one occurrence of III or UU.
This requires that we introduce the concept of an "MIU
well-formed formula" or wff, which allows us to construct
unique symbol sequences to which Rules III and IV can be
applied.
$)

$( First, we declare the constant symbols of the language.
Note that we need two symbols to distinguish the assertion
that a sequence is a wff from the assertion that it is a
theorem; we have arbitrarily chosen "wff" and "|-". $)
      $c M I U |- wff $. $( Declare constants $)

$( Next, we declare some variables. $)
     $v x y $.

$( Throughout our theory, we shall assume that these
variables represent wffs. $)
 wx   $f wff x $.
 wy   $f wff y $.

$( Define MIU-wffs.  We allow the empty sequence to be a
wff. $)

$( The empty sequence is a wff. $)
 we   $a wff $.
$( "M" after any wff is a wff. $)
 wM   $a wff x M $.
$( "I" after any wff is a wff. $)
 wI   $a wff x I $.
$( "U" after any wff is a wff. $)
 wU   $a wff x U $.

$( Assert the axiom. $)
 ax   $a |- M I $.

$( Assert the rules. $)
 ${
   Ia   $e |- x I $.
$( Given any theorem ending with "I", it remains a theorem
if "U" is added after it.  (We distinguish the label I_
from the math symbol I to conform to the 24-Jun-2006
Metamath spec.) $)
   I_    $a |- x I U $.
 $}
 ${
IIa  $e |- M x $.
$( Given any theorem starting with "M", it remains a theorem
if the part after the "M" is added again after it. $)
   II   $a |- M x x $.
 $}
 ${
   IIIa $e |- x I I I y $.
$( Given any theorem with "III" in the middle, it remains a
theorem if the "III" is replaced with "U". $)
   III  $a |- x U y $.
 $}
 ${
   IVa  $e |- x U U y $.
$( Given any theorem with "UU" in the middle, it remains a
theorem if the "UU" is deleted. $)
   IV   $a |- x y $.
  $}

$( Now we prove the theorem MUIIU.  You may be interested in
comparing this proof with that of Hofstadter (pp. 35 - 36).
$)
 theorem1  $p |- M U I I U $=
      we wM wU wI we wI wU we wU wI wU we wM we wI wU we wM
      wI wI wI we wI wI we wI ax II II I_ III II IV $.
\end{verbatim}\index{well-formed formula (wff)}

The \texttt{show proof /lemmon/renumber} command
yields the following display.  It is very similar
to the one in \cite[pp.~35--36]{Hofstadter}.\index{Hofstadter, Douglas R.}

\begin{verbatim}
1 ax             $a |- M I
2 1 II           $a |- M I I
3 2 II           $a |- M I I I I
4 3 I_           $a |- M I I I I U
5 4 III          $a |- M U I U
6 5 II           $a |- M U I U U I U
7 6 IV           $a |- M U I I U
\end{verbatim}

We note that Hofstadter's ``MU-puzzle,'' which asks whether
MU is a theorem of the MIU-system, cannot be answered using
the system above because the MU-puzzle is a question {\em
about} the system.  To prove the answer to the MU-puzzle,
a much more elaborate system is needed, namely one that
models the MIU-system within set theory.  (Incidentally, the
answer to the MU-puzzle is no.)

\chapter{Metamath Language EBNF}%
\label{BNF}%
\index{Metamath Language EBNF}

The following is a formal description of the basic Metamath language syntax
(with compressed proofs and support for unknown proof steps).
It is defined using the
Extended Backus--Naur Form (EBNF)\index{Extended Backus--Naur Form}\index{EBNF}
notation from W3C\index{W3C}
\textit{Extensible Markup Language (XML) 1.0 (Fifth Edition)}
(W3C Recommendation 26 November 2008) at
\url{https://www.w3.org/TR/xml/#sec-notation}.

The \texttt{database}
rule is processed until the end of the file (\texttt{EOF}).
The rules eventually require reading whitespace-separated tokens.
A token has an upper-case definition (see below)
or is a string constant in a non-token (such as \texttt{'\$a'}).
We intend for this to be correct, but if there is a conflict the
rules of section \ref{spec} govern. That section also discusses
non-syntax restrictions not shown here
(e.g., that each new label token
defined in a \texttt{hypothesis-stmt} or \texttt{assert-stmt}
must be unique).

\begin{verbatim}
database ::= outermost-scope-stmt*

outermost-scope-stmt ::=
  include-stmt | constant-stmt | stmt

/* File inclusion command; process file as a database.
   Databases should NOT have a comment in the filename. */
include-stmt ::= '$[' filename '$]'

/* Constant symbols declaration. */
constant-stmt ::= '$c' constant+ '$.'

/* A normal statement can occur in any scope. */
stmt ::= block | variable-stmt | disjoint-stmt |
  hypothesis-stmt | assert-stmt

/* A block. You can have 0 statements in a block. */
block ::= '${' stmt* '$}'

/* Variable symbols declaration. */
variable-stmt ::= '$v' variable+ '$.'

/* Disjoint variables. Simple disjoint statements have
   2 variables, i.e., "variable*" is empty for them. */
disjoint-stmt ::= '$d' variable variable variable* '$.'

hypothesis-stmt ::= floating-stmt | essential-stmt

/* Floating (variable-type) hypothesis. */
floating-stmt ::= LABEL '$f' typecode variable '$.'

/* Essential (logical) hypothesis. */
essential-stmt ::= LABEL '$e' typecode MATH-SYMBOL* '$.'

assert-stmt ::= axiom-stmt | provable-stmt

/* Axiomatic assertion. */
axiom-stmt ::= LABEL '$a' typecode MATH-SYMBOL* '$.'

/* Provable assertion. */
provable-stmt ::= LABEL '$p' typecode MATH-SYMBOL*
  '$=' proof '$.'

/* A proof. Proofs may be interspersed by comments.
   If '?' is in a proof it's an "incomplete" proof. */
proof ::= uncompressed-proof | compressed-proof
uncompressed-proof ::= (LABEL | '?')+
compressed-proof ::= '(' LABEL* ')' COMPRESSED-PROOF-BLOCK+

typecode ::= constant

filename ::= MATH-SYMBOL /* No whitespace or '$' */
constant ::= MATH-SYMBOL
variable ::= MATH-SYMBOL
\end{verbatim}

\needspace{2\baselineskip}
A \texttt{frame} is a sequence of 0 or more
\texttt{disjoint-{\allowbreak}stmt} and
\texttt{hypotheses-{\allowbreak}stmt} statements
(possibly interleaved with other non-\texttt{assert-stmt} statements)
followed by one \texttt{assert-stmt}.

\needspace{3\baselineskip}
Here are the rules for lexical processing (tokenization) beyond
the constant tokens shown above.
By convention these tokenization rules have upper-case names.
Every token is read for the longest possible length.
Whitespace-separated tokens are read sequentially;
note that the separating whitespace and \texttt{\$(} ... \texttt{\$)}
comments are skipped.

If a token definition uses another token definition, the whole thing
is considered a single token.
A pattern that is only part of a full token has a name beginning
with an underscore (``\_'').
An implementation could tokenize many tokens as a
\texttt{PRINTABLE-SEQUENCE}
and then check if it meets the more specific rule shown here.

Comments do not nest, and both \texttt{\$(} and \texttt{\$)}
have to be surrounded
by at least one whitespace character (\texttt{\_WHITECHAR}).
Technically comments end without consuming the trailing
\texttt{\_WHITECHAR}, but the trailing
\texttt{\_WHITECHAR} gets ignored anyway so we ignore that detail here.
Metamath language processors
are not required to support \texttt{\$)} followed
immediately by a bare end-of-file, because the closing
comment symbol is supposed to be followed by a
\texttt{\_WHITECHAR} such as a newline.

\begin{verbatim}
PRINTABLE-SEQUENCE ::= _PRINTABLE-CHARACTER+

MATH-SYMBOL ::= (_PRINTABLE-CHARACTER - '$')+

/* ASCII non-whitespace printable characters */
_PRINTABLE-CHARACTER ::= [#x21-#x7e]

LABEL ::= ( _LETTER-OR-DIGIT | '.' | '-' | '_' )+

_LETTER-OR-DIGIT ::= [A-Za-z0-9]

COMPRESSED-PROOF-BLOCK ::= ([A-Z] | '?')+

/* Define whitespace between tokens. The -> SKIP
   means that when whitespace is seen, it is
   skipped and we simply read again. */
WHITESPACE ::= (_WHITECHAR+ | _COMMENT) -> SKIP

/* Comments. $( ... $) and do not nest. */
_COMMENT ::= '$(' (_WHITECHAR+ (PRINTABLE-SEQUENCE - '$)'))*
  _WHITECHAR+ '$)' _WHITECHAR

/* Whitespace: (' ' | '\t' | '\r' | '\n' | '\f') */
_WHITECHAR ::= [#x20#x09#x0d#x0a#x0c]
\end{verbatim}
% This EBNF was developed as a collaboration between
% David A. Wheeler\index{Wheeler, David A.},
% Mario Carneiro\index{Carneiro, Mario}, and
% Benoit Jubin\index{Jubin, Benoit}, inspired by a request
% (and a lot of initial work) by Benoit Jubin.
%
% \chapter{Disclaimer and Trademarks}
%
% Information in this document is subject to change without notice and does not
% represent a commitment on the part of Norman Megill.
% \vspace{2ex}
%
% \noindent Norman D. Megill makes no warranties, either express or implied,
% regarding the Metamath computer software package.
%
% \vspace{2ex}
%
% \noindent Any trademarks mentioned in this book are the property of
% their respective owners.  The name ``Metamath'' is a trademark of
% Norman Megill.
%
\cleardoublepage
\phantomsection  % fixes the link anchor
\addcontentsline{toc}{chapter}{\bibname}

\bibliography{metamath}
%\input{metamath.bbl}

\raggedright
\cleardoublepage
\phantomsection % fixes the link anchor
\addcontentsline{toc}{chapter}{\indexname}
%\printindex   ??
\input{metamath.ind}

\end{document}



\end{document}



\raggedright
\cleardoublepage
\phantomsection % fixes the link anchor
\addcontentsline{toc}{chapter}{\indexname}
%\printindex   ??
% metamath.tex - Version of 2-Jun-2019
% If you change the date above, also change the "Printed date" below.
% SPDX-License-Identifier: CC0-1.0
%
%                              PUBLIC DOMAIN
%
% This file (specifically, the version of this file with the above date)
% has been released into the Public Domain per the
% Creative Commons CC0 1.0 Universal (CC0 1.0) Public Domain Dedication
% https://creativecommons.org/publicdomain/zero/1.0/
%
% The public domain release applies worldwide.  In case this is not
% legally possible, the right is granted to use the work for any purpose,
% without any conditions, unless such conditions are required by law.
%
% Several short, attributed quotations from copyrighted works
% appear in this file under the ``fair use'' provision of Section 107 of
% the United States Copyright Act (Title 17 of the {\em United States
% Code}).  The public-domain status of this file is not applicable to
% those quotations.
%
% Norman Megill - email: nm(at)alum(dot)mit(dot)edu
%
% David A. Wheeler also donates his improvements to this file to the
% public domain per the CC0.  He works at the Institute for Defense Analyses
% (IDA), but IDA has agreed that this Metamath work is outside its "lane"
% and is not a work by IDA.  This was specifically confirmed by
% Margaret E. Myers (Division Director of the Information Technology
% and Systems Division) on 2019-05-24 and by Ben Lindorf (General Counsel)
% on 2019-05-22.

% This file, 'metamath.tex', is self-contained with everything needed to
% generate the the PDF file 'metamath.pdf' (the _Metamath_ book) on
% standard LaTeX 2e installations.  The auxiliary files are embedded with
% "filecontents" commands.  To generate metamath.pdf file, run these
% commands under Linux or Cygwin in the directory that contains
% 'metamath.tex':
%
%   rm -f realref.sty metamath.bib
%   touch metamath.ind
%   pdflatex metamath
%   pdflatex metamath
%   bibtex metamath
%   makeindex metamath
%   pdflatex metamath
%   pdflatex metamath
%
% The warnings that occur in the initial runs of pdflatex can be ignored.
% For the final run,
%
%   egrep -i 'error|warn' metamath.log
%
% should show exactly these 5 warnings:
%
%   LaTeX Warning: File `realref.sty' already exists on the system.
%   LaTeX Warning: File `metamath.bib' already exists on the system.
%   LaTeX Font Warning: Font shape `OMS/cmtt/m/n' undefined
%   LaTeX Font Warning: Font shape `OMS/cmtt/bx/n' undefined
%   LaTeX Font Warning: Some font shapes were not available, defaults
%       substituted.
%
% Search for "Uncomment" below if you want to suppress hyperlink boxes
% in the PDF output file
%
% TYPOGRAPHICAL NOTES:
% * It is customary to use an en dash (--) to "connect" names of different
%   people (and to denote ranges), and use a hyphen (-) for a
%   single compound name. Examples of connected multiple people are
%   Zermelo--Fraenkel, Schr\"{o}der--Bernstein, Tarski--Grothendieck,
%   Hewlett--Packard, and Backus--Naur.  Examples of a single person with
%   a compound name include Levi-Civita, Mittag-Leffler, and Burali-Forti.
% * Use non-breaking spaces after page abbreviations, e.g.,
%   p.~\pageref{note2002}.
%
% --------------------------- Start of realref.sty -----------------------------
\begin{filecontents}{realref.sty}
% Save the following as realref.sty.
% You can then use it with \usepackage{realref}
%
% This has \pageref jumping to the page on which the ref appears,
% \ref jumping to the point of the anchor, and \sectionref
% jumping to the start of section.
%
% Author:  Anthony Williams
%          Software Engineer
%          Nortel Networks Optical Components Ltd
% Date:    9 Nov 2001 (posted to comp.text.tex)
%
% The following declaration was made by Anthony Williams on
% 24 Jul 2006 (private email to Norman Megill):
%
%   ``I hereby donate the code for realref.sty posted on the
%   comp.text.tex newsgroup on 9th November 2001, accessible from
%   http://groups.google.com/group/comp.text.tex/msg/5a0e1cc13ea7fbb2
%   to the public domain.''
%
\ProvidesPackage{realref}
\RequirePackage[plainpages=false,pdfpagelabels=true]{hyperref}
\def\realref@anchorname{}
\AtBeginDocument{%
% ensure every label is a possible hyperlink target
\let\realref@oldrefstepcounter\refstepcounter%
\DeclareRobustCommand{\refstepcounter}[1]{\realref@oldrefstepcounter{#1}
\edef\realref@anchorname{\string #1.\@currentlabel}%
}%
\let\realref@oldlabel\label%
\DeclareRobustCommand{\label}[1]{\realref@oldlabel{#1}\hypertarget{#1}{}%
\@bsphack\protected@write\@auxout{}{%
    \string\expandafter\gdef\protect\csname
    page@num.#1\string\endcsname{\thepage}%
    \string\expandafter\gdef\protect\csname
    ref@num.#1\string\endcsname{\@currentlabel}%
    \string\expandafter\gdef\protect\csname
    sectionref@name.#1\string\endcsname{\realref@anchorname}%
}\@esphack}%
\DeclareRobustCommand\pageref[1]{{\edef\a{\csname
            page@num.#1\endcsname}\expandafter\hyperlink{page.\a}{\a}}}%
\DeclareRobustCommand\ref[1]{{\edef\a{\csname
            ref@num.#1\endcsname}\hyperlink{#1}{\a}}}%
\DeclareRobustCommand\sectionref[1]{{\edef\a{\csname
            ref@num.#1\endcsname}\edef\b{\csname
            sectionref@name.#1\endcsname}\hyperlink{\b}{\a}}}%
}
\end{filecontents}
% ---------------------------- End of realref.sty ------------------------------

% --------------------------- Start of metamath.bib -----------------------------
\begin{filecontents}{metamath.bib}
@book{Albers, editor = "Donald J. Albers and G. L. Alexanderson",
  title = "Mathematical People",
  publisher = "Contemporary Books, Inc.",
  address = "Chicago",
  note = "[QA28.M37]",
  year = 1985 }
@book{Anderson, author = "Alan Ross Anderson and Nuel D. Belnap",
  title = "Entailment",
  publisher = "Princeton University Press",
  address = "Princeton",
  volume = 1,
  note = "[QA9.A634 1975 v.1]",
  year = 1975}
@book{Barrow, author = "John D. Barrow",
  title = "Theories of Everything:  The Quest for Ultimate Explanation",
  publisher = "Oxford University Press",
  address = "Oxford",
  note = "[Q175.B225]",
  year = 1991 }
@book{Behnke,
  editor = "H. Behnke and F. Backmann and K. Fladt and W. S{\"{u}}ss",
  title = "Fundamentals of Mathematics",
  volume = "I",
  publisher = "The MIT Press",
  address = "Cambridge, Massachusetts",
  note = "[QA37.2.B413]",
  year = 1974 }
@book{Bell, author = "J. L. Bell and M. Machover",
  title = "A Course in Mathematical Logic",
  publisher = "North-Holland",
  address = "Amsterdam",
  note = "[QA9.B3953]",
  year = 1977 }
@inproceedings{Blass, author = "Andrea Blass",
  title = "The Interaction Between Category Theory and Set Theory",
  pages = "5--29",
  booktitle = "Mathematical Applications of Category Theory (Proceedings
     of the Special Session on Mathematical Applications
     Category Theory, 89th Annual Meeting of the American Mathematical
     Society, held in Denver, Colorado January 5--9, 1983)",
  editor = "John Walter Gray",
  year = 1983,
  note = "[QA169.A47 1983]",
  publisher = "American Mathematical Society",
  address = "Providence, Rhode Island"}
@proceedings{Bledsoe, editor = "W. W. Bledsoe and D. W. Loveland",
  title = "Automated Theorem Proving:  After 25 Years (Proceedings
     of the Special Session on Automatic Theorem Proving,
     89th Annual Meeting of the American Mathematical
     Society, held in Denver, Colorado January 5--9, 1983)",
  year = 1983,
  note = "[QA76.9.A96.S64 1983]",
  publisher = "American Mathematical Society",
  address = "Providence, Rhode Island" }
@book{Boolos, author = "George S. Boolos and Richard C. Jeffrey",
  title = "Computability and Log\-ic",
  publisher = "Cambridge University Press",
  edition = "third",
  address = "Cambridge",
  note = "[QA9.59.B66 1989]",
  year = 1989 }
@book{Campbell, author = "John Campbell",
  title = "Programmer's Progress",
  publisher = "White Star Software",
  address = "Box 51623, Palo Alto, CA 94303",
  year = 1991 }
@article{DBLP:journals/corr/Carneiro14,
  author    = {Mario Carneiro},
  title     = {Conversion of {HOL} Light proofs into Metamath},
  journal   = {CoRR},
  volume    = {abs/1412.8091},
  year      = {2014},
  url       = {http://arxiv.org/abs/1412.8091},
  archivePrefix = {arXiv},
  eprint    = {1412.8091},
  timestamp = {Mon, 13 Aug 2018 16:47:05 +0200},
  biburl    = {https://dblp.org/rec/bib/journals/corr/Carneiro14},
  bibsource = {dblp computer science bibliography, https://dblp.org}
}
@article{CarneiroND,
  author    = {Mario Carneiro},
  title     = {Natural Deductions in the Metamath Proof Language},
  url       = {http://us.metamath.org/ocat/natded.pdf},
  year      = 2014
}
@inproceedings{Chou, author = "Shang-Ching Chou",
  title = "Proving Elementary Geometry Theorems Using {W}u's Algorithm",
  pages = "243--286",
  booktitle = "Automated Theorem Proving:  After 25 Years (Proceedings
     of the Special Session on Automatic Theorem Proving,
     89th Annual Meeting of the American Mathematical
     Society, held in Denver, Colorado January 5--9, 1983)",
  editor = "W. W. Bledsoe and D. W. Loveland",
  year = 1983,
  note = "[QA76.9.A96.S64 1983]",
  publisher = "American Mathematical Society",
  address = "Providence, Rhode Island" }
@book{Clemente, author = "Daniel Clemente Laboreo",
  title = "Introduction to natural deduction",
  year = 2014,
  url = "http://www.danielclemente.com/logica/dn.en.pdf" }
@incollection{Courant, author = "Richard Courant and Herbert Robbins",
  title = "Topology",
  pages = "573--590",
  booktitle = "The World of Mathematics, Volume One",
  editor = "James R. Newman",
  publisher = "Simon and Schuster",
  address = "New York",
  note = "[QA3.W67 1988]",
  year = 1956 }
@book{Curry, author = "Haskell B. Curry",
  title = "Foundations of Mathematical Logic",
  publisher = "Dover Publications, Inc.",
  address = "New York",
  note = "[QA9.C976 1977]",
  year = 1977 }
@book{Davis, author = "Philip J. Davis and Reuben Hersh",
  title = "The Mathematical Experience",
  publisher = "Birkh{\"{a}}user Boston",
  address = "Boston",
  note = "[QA8.4.D37 1982]",
  year = 1981 }
@incollection{deMillo,
  author = "Richard de Millo and Richard Lipton and Alan Perlis",
  title = "Social Processes and Proofs of Theorems and Programs",
  pages = "267--285",
  booktitle = "New Directions in the Philosophy of Mathematics",
  editor = "Thomas Tymoczko",
  publisher = "Birkh{\"{a}}user Boston, Inc.",
  address = "Boston",
  note = "[QA8.6.N48 1986]",
  year = 1986 }
@book{Edwards, author = "Robert E. Edwards",
  title = "A Formal Background to Mathematics",
  publisher = "Springer-Verlag",
  address = "New York",
  note = "[QA37.2.E38 v.1a]",
  year = 1979 }
@book{Enderton, author = "Herbert B. Enderton",
  title = "Elements of Set Theory",
  publisher = "Academic Press, Inc.",
  address = "San Diego",
  note = "[QA248.E5]",
  year = 1977 }
@book{Goodstein, author = "R. L. Goodstein",
  title = "Development of Mathematical Logic",
  publisher = "Springer-Verlag New York Inc.",
  address = "New York",
  note = "[QA9.G6554]",
  year = 1971 }
@book{Guillen, author = "Michael Guillen",
  title = "Bridges to Infinity",
  publisher = "Jeremy P. Tarcher, Inc.",
  address = "Los Angeles",
  note = "[QA93.G8]",
  year = 1983 }
@book{Hamilton, author = "Alan G. Hamilton",
  title = "Logic for Mathematicians",
  edition = "revised",
  publisher = "Cambridge University Press",
  address = "Cambridge",
  note = "[QA9.H298]",
  year = 1988 }
@unpublished{Harrison, author = "John Robert Harrison",
  title = "Metatheory and Reflection in Theorem Proving:
    A Survey and Critique",
  note = "Technical Report
    CRC-053.
    SRI Cambridge,
    Millers Yard, Cambridge, UK,
    1995.
    Available on the Web as
{\verb+http:+}\-{\verb+//www.cl.cam.ac.uk/users/jrh/papers/reflect.html+}"}
@TECHREPORT{Harrison-thesis,
        author          = "John Robert Harrison",
        title           = "Theorem Proving with the Real Numbers",
        institution   = "University of Cambridge Computer
                         Lab\-o\-ra\-to\-ry",
        address         = "New Museums Site, Pembroke Street, Cambridge,
                           CB2 3QG, UK",
        year            = 1996,
        number          = 408,
        type            = "Technical Report",
        note            = "Author's PhD thesis,
   available on the Web at
{\verb+http:+}\-{\verb+//www.cl.cam.ac.uk+}\-{\verb+/users+}\-{\verb+/jrh+}%
\-{\verb+/papers+}\-{\verb+/thesis.html+}"}
@book{Herrlich, author = "Horst Herrlich and George E. Strecker",
  title = "Category Theory:  An Introduction",
  publisher = "Allyn and Bacon Inc.",
  address = "Boston",
  note = "[QA169.H567]",
  year = 1973 }
@article{Hindley, author = "J. Roger Hindley and David Meredith",
  title = "Principal Type-Schemes and Condensed Detachment",
  journal = "The Journal of Symbolic Logic",
  volume = 55,
  year = 1990,
  note = "[QA.J87]",
  pages = "90--105" }
@book{Hofstadter, author = "Douglas R. Hofstadter",
  title = "G{\"{o}}del, Escher, Bach",
  publisher = "Basic Books, Inc.",
  address = "New York",
  note = "[QA9.H63 1980]",
  year = 1979 }
@article{Indrzejczak, author= "Andrzej Indrzejczak",
  title = "Natural Deduction, Hybrid Systems and Modal Logic",
  journal = "Trends in Logic",
  volume = 30,
  publisher = "Springer",
  year = 2010 }
@article{Kalish, author = "D. Kalish and R. Montague",
  title = "On {T}arski's Formalization of Predicate Logic with Identity",
  journal = "Archiv f{\"{u}}r Mathematische Logik und Grundlagenfor\-schung",
  volume = 7,
  year = 1965,
  note = "[QA.A673]",
  pages = "81--101" }
@article{Kalman, author = "J. A. Kalman",
  title = "Condensed Detachment as a Rule of Inference",
  journal = "Studia Logica",
  volume = 42,
  number = 4,
  year = 1983,
  note = "[B18.P6.S933]",
  pages = "443-451" }
@book{Kline, author = "Morris Kline",
  title = "Mathematical Thought from Ancient to Modern Times",
  publisher = "Oxford University Press",
  address = "New York",
  note = "[QA21.K516 1990 v.3]",
  year = 1972 }
@book{Klinel, author = "Morris Kline",
  title = "Mathematics, The Loss of Certainty",
  publisher = "Oxford University Press",
  address = "New York",
  note = "[QA21.K525]",
  year = 1980 }
@book{Kramer, author = "Edna E. Kramer",
  title = "The Nature and Growth of Modern Mathematics",
  publisher = "Princeton University Press",
  address = "Princeton, New Jersey",
  note = "[QA93.K89 1981]",
  year = 1981 }
@article{Knill, author = "Oliver Knill",
  title = "Some Fundamental Theorems in Mathematics",
  year = "2018",
  url = "https://arxiv.org/abs/1807.08416" }
@book{Landau, author = "Edmund Landau",
  title = "Foundations of Analysis",
  publisher = "Chelsea Publishing Company",
  address = "New York",
  edition = "second",
  note = "[QA241.L2541 1960]",
  year = 1960 }
@article{Leblanc, author = "Hugues Leblanc",
  title = "On {M}eyer and {L}ambert's Quantificational Calculus {FQ}",
  journal = "The Journal of Symbolic Logic",
  volume = 33,
  year = 1968,
  note = "[QA.J87]",
  pages = "275--280" }
@article{Lejewski, author = "Czeslaw Lejewski",
  title = "On Implicational Definitions",
  journal = "Studia Logica",
  volume = 8,
  year = 1958,
  note = "[B18.P6.S933]",
  pages = "189--208" }
@book{Levy, author = "Azriel Levy",
  title = "Basic Set Theory",
  publisher = "Dover Publications",
  address = "Mineola, NY",
  year = "2002"
}
@book{Margaris, author = "Angelo Margaris",
  title = "First Order Mathematical Logic",
  publisher = "Blaisdell Publishing Company",
  address = "Waltham, Massachusetts",
  note = "[QA9.M327]",
  year = 1967}
@book{Manin, author = "Yu I. Manin",
  title = "A Course in Mathematical Logic",
  publisher = "Springer-Verlag",
  address = "New York",
  note = "[QA9.M29613]",
  year = "1977" }
@article{Mathias, author = "Adrian R. D. Mathias",
  title = "A Term of Length 4,523,659,424,929",
  journal = "Synthese",
  volume = 133,
  year = 2002,
  note = "[Q.S993]",
  pages = "75--86" }
@article{Megill, author = "Norman D. Megill",
  title = "A Finitely Axiomatized Formalization of Predicate Calculus
     with Equality",
  journal = "Notre Dame Journal of Formal Logic",
  volume = 36,
  year = 1995,
  note = "[QA.N914]",
  pages = "435--453" }
@unpublished{Megillc, author = "Norman D. Megill",
  title = "A Shorter Equivalent of the Axiom of Choice",
  month = "June",
  note = "Unpublished",
  year = 1991 }
@article{MegillBunder, author = "Norman D. Megill and Martin W.
    Bunder",
  title = "Weaker {D}-Complete Logics",
  journal = "Journal of the IGPL",
  volume = 4,
  year = 1996,
  pages = "215--225",
  note = "Available on the Web at
{\verb+http:+}\-{\verb+//www.mpi-sb.mpg.de+}\-{\verb+/igpl+}%
\-{\verb+/Journal+}\-{\verb+/V4-2+}\-{\verb+/#Megill+}"}
}
@book{Mendelson, author = "Elliott Mendelson",
  title = "Introduction to Mathematical Logic",
  edition = "second",
  publisher = "D. Van Nostrand Company, Inc.",
  address = "New York",
  note = "[QA9.M537 1979]",
  year = 1979 }
@article{Meredith, author = "David Meredith",
  title = "In Memoriam {C}arew {A}rthur {M}eredith (1904-1976)",
  journal = "Notre Dame Journal of Formal Logic",
  volume = 18,
  year = 1977,
  note = "[QA.N914]",
  pages = "513--516" }
@article{CAMeredith, author = "C. A. Meredith",
  title = "Single Axioms for the Systems ({C},{N}), ({C},{O}) and ({A},{N})
      of the Two-Valued Propositional Calculus",
  journal = "The Journal of Computing Systems",
  volume = 3,
  year = 1953,
  pages = "155--164" }
@article{Monk, author = "J. Donald Monk",
  title = "Provability With Finitely Many Variables",
  journal = "The Journal of Symbolic Logic",
  volume = 27,
  year = 1971,
  note = "[QA.J87]",
  pages = "353--358" }
@article{Monks, author = "J. Donald Monk",
  title = "Substitutionless Predicate Logic With Identity",
  journal = "Archiv f{\"{u}}r Mathematische Logik und Grundlagenfor\-schung",
  volume = 7,
  year = 1965,
  pages = "103--121" }
  %% Took out this from above to prevent LaTeX underfull warning:
  % note = "[QA.A673]",
@book{Moore, author = "A. W. Moore",
  title = "The Infinite",
  publisher = "Routledge",
  address = "New York",
  note = "[BD411.M59]",
  year = 1989}
@book{Munkres, author = "James R. Munkres",
  title = "Topology: A First Course",
  publisher = "Prentice-Hall, Inc.",
  address = "Englewood Cliffs, New Jersey",
  note = "[QA611.M82]",
  year = 1975}
@article{Nemesszeghy, author = "E. Z. Nemesszeghy and E. A. Nemesszeghy",
  title = "On Strongly Creative Definitions:  A Reply to {V}. {F}. {R}ickey",
  journal = "Logique et Analyse (N.\ S.)",
  year = 1977,
  volume = 20,
  note = "[BC.L832]",
  pages = "111--115" }
@unpublished{Nemeti, author = "N{\'{e}}meti, I.",
  title = "Algebraizations of Quantifier Logics, an Overview",
  note = "Version 11.4, preprint, Mathematical Institute, Budapest,
    1994.  A shortened version without proofs appeared in
    ``Algebraizations of quantifier logics, an introductory overview,''
   {\em Studia Logica}, 50:485--569, 1991 [B18.P6.S933]"}
@article{Pavicic, author = "M. Pavi{\v{c}}i{\'{c}}",
  title = "A New Axiomatization of Unified Quantum Logic",
  journal = "International Journal of Theoretical Physics",
  year = 1992,
  volume = 31,
  note = "[QC.I626]",
  pages = "1753 --1766" }
@book{Penrose, author = "Roger Penrose",
  title = "The Emperor's New Mind",
  publisher = "Oxford University Press",
  address = "New York",
  note = "[Q335.P415]",
  year = 1989 }
@book{PetersonI, author = "Ivars Peterson",
  title = "The Mathematical Tourist",
  publisher = "W. H. Freeman and Company",
  address = "New York",
  note = "[QA93.P475]",
  year = 1988 }
@article{Peterson, author = "Jeremy George Peterson",
  title = "An automatic theorem prover for substitution and detachment systems",
  journal = "Notre Dame Journal of Formal Logic",
  volume = 19,
  year = 1978,
  note = "[QA.N914]",
  pages = "119--122" }
@book{Quine, author = "Willard Van Orman Quine",
  title = "Set Theory and Its Logic",
  edition = "revised",
  publisher = "The Belknap Press of Harvard University Press",
  address = "Cambridge, Massachusetts",
  note = "[QA248.Q7 1969]",
  year = 1969 }
@article{Robinson, author = "J. A. Robinson",
  title = "A Machine-Oriented Logic Based on the Resolution Principle",
  journal = "Journal of the Association for Computing Machinery",
  year = 1965,
  volume = 12,
  pages = "23--41" }
@article{RobinsonT, author = "T. Thacher Robinson",
  title = "Independence of Two Nice Sets of Axioms for the Propositional
    Calculus",
  journal = "The Journal of Symbolic Logic",
  volume = 33,
  year = 1968,
  note = "[QA.J87]",
  pages = "265--270" }
@book{Rucker, author = "Rudy Rucker",
  title = "Infinity and the Mind:  The Science and Philosophy of the
    Infinite",
  publisher = "Bantam Books, Inc.",
  address = "New York",
  note = "[QA9.R79 1982]",
  year = 1982 }
@book{Russell, author = "Bertrand Russell",
  title = "Mysticism and Logic, and Other Essays",
  publisher = "Barnes \& Noble Books",
  address = "Totowa, New Jersey",
  note = "[B1649.R963.M9 1981]",
  year = 1981 }
@article{Russell2, author = "Bertrand Russell",
  title = "Recent Work on the Principles of Mathematics",
  journal = "International Monthly",
  volume = 4,
  year = 1901,
  pages = "84"}
@article{Schmidt, author = "Eric Schmidt",
  title = "Reductions in Norman Megill's axiom system for complex numbers",
  url = "http://us.metamath.org/downloads/schmidt-cnaxioms.pdf",
  year = "2012" }
@book{Shoenfield, author = "Joseph R. Shoenfield",
  title = "Mathematical Logic",
  publisher = "Addison-Wesley Publishing Company, Inc.",
  address = "Reading, Massachusetts",
  year = 1967,
  note = "[QA9.S52]" }
@book{Smullyan, author = "Raymond M. Smullyan",
  title = "Theory of Formal Systems",
  publisher = "Princeton University Press",
  address = "Princeton, New Jersey",
  year = 1961,
  note = "[QA248.5.S55]" }
@book{Solow, author = "Daniel Solow",
  title = "How to Read and Do Proofs:  An Introduction to Mathematical
    Thought Process",
  publisher = "John Wiley \& Sons",
  address = "New York",
  year = 1982,
  note = "[QA9.S577]" }
@book{Stark, author = "Harold M. Stark",
  title = "An Introduction to Number Theory",
  publisher = "Markham Publishing Company",
  address = "Chicago",
  note = "[QA241.S72 1978]",
  year = 1970 }
@article{Swart, author = "E. R. Swart",
  title = "The Philosophical Implications of the Four-Color Problem",
  journal = "American Mathematical Monthly",
  year = 1980,
  volume = 87,
  month = "November",
  note = "[QA.A5125]",
  pages = "697--707" }
@book{Szpiro, author = "George G. Szpiro",
  title = "Poincar{\'{e}}'s Prize: The Hundred-Year Quest to Solve One
    of Math's Greatest Puzzles",
  publisher = "Penguin Books Ltd",
  address = "London",
  note = "[QA43.S985 2007]",
  year = 2007}
@book{Takeuti, author = "Gaisi Takeuti and Wilson M. Zaring",
  title = "Introduction to Axiomatic Set Theory",
  edition = "second",
  publisher = "Springer-Verlag New York Inc.",
  address = "New York",
  note = "[QA248.T136 1982]",
  year = 1982}
@inproceedings{Tarski, author = "Alfred Tarski",
  title = "What is Elementary Geometry",
  pages = "16--29",
  booktitle = "The Axiomatic Method, with Special Reference to Geometry and
     Physics (Proceedings of an International Symposium held at the University
     of California, Berkeley, December 26, 1957 --- January 4, 1958)",
  editor = "Leon Henkin and Patrick Suppes and Alfred Tarski",
  year = 1959,
  publisher = "North-Holland Publishing Company",
  address = "Amsterdam"}
@article{Tarski1965, author = "Alfred Tarski",
  title = "A Simplified Formalization of Predicate Logic with Identity",
  journal = "Archiv f{\"{u}}r Mathematische Logik und Grundlagenforschung",
  volume = 7,
  year = 1965,
  note = "[QA.A673]",
  pages = "61--79" }
@book{Tymoczko,
  title = "New Directions in the Philosophy of Mathematics",
  editor = "Thomas Tymoczko",
  publisher = "Birkh{\"{a}}user Boston, Inc.",
  address = "Boston",
  note = "[QA8.6.N48 1986]",
  year = 1986 }
@incollection{Wang,
  author = "Hao Wang",
  title = "Theory and Practice in Mathematics",
  pages = "129--152",
  booktitle = "New Directions in the Philosophy of Mathematics",
  editor = "Thomas Tymoczko",
  publisher = "Birkh{\"{a}}user Boston, Inc.",
  address = "Boston",
  note = "[QA8.6.N48 1986]",
  year = 1986 }
@manual{Webster,
  title = "Webster's New Collegiate Dictionary",
  organization = "G. \& C. Merriam Co.",
  address = "Springfield, Massachusetts",
  note = "[PE1628.W4M4 1977]",
  year = 1977 }
@manual{Whitehead, author = "Alfred North Whitehead",
  title = "An Introduction to Mathematics",
  year = 1911 }
@book{PM, author = "Alfred North Whitehead and Bertrand Russell",
  title = "Principia Mathematica",
  edition = "second",
  publisher = "Cambridge University Press",
  address = "Cambridge",
  year = "1927",
  note = "(3 vols.) [QA9.W592 1927]" }
@article{DBLP:journals/corr/Whalen16,
  author    = {Daniel Whalen},
  title     = {Holophrasm: a neural Automated Theorem Prover for higher-order logic},
  journal   = {CoRR},
  volume    = {abs/1608.02644},
  year      = {2016},
  url       = {http://arxiv.org/abs/1608.02644},
  archivePrefix = {arXiv},
  eprint    = {1608.02644},
  timestamp = {Mon, 13 Aug 2018 16:46:19 +0200},
  biburl    = {https://dblp.org/rec/bib/journals/corr/Whalen16},
  bibsource = {dblp computer science bibliography, https://dblp.org} }
@article{Wiedijk-revisited,
  author = {Freek Wiedijk},
  title = {The QED Manifesto Revisited},
  year = {2007},
  url = {http://mizar.org/trybulec65/8.pdf} }
@book{Wolfram,
  author = "Stephen Wolfram",
  title = "Mathematica:  A System for Doing Mathematics by Computer",
  edition = "second",
  publisher = "Addison-Wesley Publishing Co.",
  address = "Redwood City, California",
  note = "[QA76.95.W65 1991]",
  year = 1991 }
@book{Wos, author = "Larry Wos and Ross Overbeek and Ewing Lusk and Jim Boyle",
  title = "Automated Reasoning:  Introduction and Applications",
  edition = "second",
  publisher = "McGraw-Hill, Inc.",
  address = "New York",
  note = "[QA76.9.A96.A93 1992]",
  year = 1992 }

%
%
%[1] Church, Alonzo, Introduction to Mathematical Logic,
% Volume 1, Princeton University Press, Princeton, N. J., 1956.
%
%[2] Cohen, Paul J., Set Theory and the Continuum Hypothesis,
% W. A. Benjamin, Inc., Reading, Mass., 1966.
%
%[3] Hamilton, Alan G., Logic for Mathematicians, Cambridge
% University Press,
% Cambridge, 1988.

%[6] Kleene, Stephen Cole, Introduction to Metamathematics, D.  Van
% Nostrand Company, Inc., Princeton (1952).

%[13] Tarski, Alfred, "A simplified formalization of predicate
% logic with identity," Archiv fur Mathematische Logik und
% Grundlagenforschung, vol. 7 (1965), pp. 61-79.

%[14] Tarski, Alfred and Steven Givant, A Formalization of Set
% Theory Without Variables, American Mathematical Society Colloquium
% Publications, vol. 41, American Mathematical Society,
% Providence, R. I., 1987.

%[15] Zeman, J. J., Modal Logic, Oxford University Press, Oxford, 1973.
\end{filecontents}
% --------------------------- End of metamath.bib -----------------------------


%Book: Metamath
%Author:  Norman Megill Email:  nm at alum.mit.edu
%Author:  David A. Wheeler Email:  dwheeler at dwheeler.com

% A book template example
% http://www.stsci.edu/ftp/software/tex/bookstuff/book.template

\documentclass[leqno]{book} % LaTeX 2e. 10pt. Use [leqno,12pt] for 12pt
% hyperref 2002/05/27 v6.72r  (couldn't get pagebackref to work)
\usepackage[plainpages=false,pdfpagelabels=true]{hyperref}

\usepackage{needspace}     % Enable control over page breaks
\usepackage{breqn}         % automatic equation breaking
\usepackage{microtype}     % microtypography, reduces hyphenation

% Packages for flexible tables.  We need to be able to
% wrap text within a cell (with automatically-determined widths) AND
% split a table automatically across multiple pages.
% * "tabularx" wraps text in cells but only 1 page
% * "longtable" goes across pages but by itself is incompatible with tabularx
% * "ltxtable" combines longtable and tabularx, but table contents
%    must be in a separate file.
% * "ltablex" combines tabularx and longtable - must install specially
% * "booktabs" is recommended as a way to improve the look of tables,
%   but doesn't add these capabilities.
% * "tabu" much more capable and seems to be recommended. So use that.

\usepackage{makecell}      % Enable forced line splits within a table cell
% v4.13 needed for tabu: https://tex.stackexchange.com/questions/600724/dimension-too-large-after-recent-longtable-update
\usepackage{longtable}[=v4.13] % Enable multi-page tables  
\usepackage{tabu}          % Multi-page tables with wrapped text in a cell

% You can find more Tex packages using commands like:
% tlmgr search --file tabu.sty
% find /usr/share/texmf-dist/ -name '*tab*'
%
%%%%%%%%%%%%%%%%%%%%%%%%%%%%%%%%%%%%%%%%%%%%%%%%%%%%%%%%%%%%%%%%%%%%%%%%%%%%
% Uncomment the next 3 lines to suppress boxes and colors on the hyperlinks
%%%%%%%%%%%%%%%%%%%%%%%%%%%%%%%%%%%%%%%%%%%%%%%%%%%%%%%%%%%%%%%%%%%%%%%%%%%%
%\hypersetup{
%colorlinks,citecolor=black,filecolor=black,linkcolor=black,urlcolor=black
%}
%
\usepackage{realref}

% Restarting page numbers: try?
%   \printglossary
%   \cleardoublepage
%   \pagenumbering{arabic}
%   \setcounter{page}{1}    ???needed
%   \include{chap1}

% not used:
% \def\R2Lurl#1#2{\mbox{\href{#1}\texttt{#2}}}

\usepackage{amssymb}

% Version 1 of book: margins: t=.4, b=.2, ll=.4, rr=.55
% \usepackage{anysize}
% % \papersize{<height>}{<width>}
% % \marginsize{<left>}{<right>}{<top>}{<bottom>}
% \papersize{9in}{6in}
% % l/r 0.6124-0.6170 works t/b 0.2418-0.3411 = 192pp. 0.2926-03118=exact
% \marginsize{0.7147in}{0.5147in}{0.4012in}{0.2012in}

\usepackage{anysize}
% \papersize{<height>}{<width>}
% \marginsize{<left>}{<right>}{<top>}{<bottom>}
\papersize{9in}{6in}
% l/r 0.85in&0.6431-0.6539 works t/b ?-?
%\marginsize{0.85in}{0.6485in}{0.55in}{0.35in}
\marginsize{0.8in}{0.65in}{0.5in}{0.3in}

% \usepackage[papersize={3.6in,4.8in},hmargin=0.1in,vmargin={0.1in,0.1in}]{geometry}  % page geometry
\usepackage{special-settings}

\raggedbottom
\makeindex

\begin{document}
% Discourage page widows and orphans:
\clubpenalty=300
\widowpenalty=300

%%%%%%% load in AMS fonts %%%%%%% % LaTeX 2.09 - obsolete in LaTeX 2e
%\input{amssym.def}
%\input{amssym.tex}
%\input{c:/texmf/tex/plain/amsfonts/amssym.def}
%\input{c:/texmf/tex/plain/amsfonts/amssym.tex}

\bibliographystyle{plain}
\pagenumbering{roman}
\pagestyle{headings}

\thispagestyle{empty}

\hfill
\vfill

\begin{center}
{\LARGE\bf Metamath} \\
\vspace{1ex}
{\large A Computer Language for Mathematical Proofs} \\
\vspace{7ex}
{\large Norman Megill} \\
\vspace{7ex}
with extensive revisions by \\
\vspace{1ex}
{\large David A. Wheeler} \\
\vspace{7ex}
% Printed date. If changing the date below, also fix the date at the beginning.
2019-06-02
\end{center}

\vfill
\hfill

\newpage
\thispagestyle{empty}

\hfill
\vfill

\begin{center}
$\sim$\ {\sc Public Domain}\ $\sim$

\vspace{2ex}
This book (including its later revisions)
has been released into the Public Domain by Norman Megill per the
Creative Commons CC0 1.0 Universal (CC0 1.0) Public Domain Dedication.
David A. Wheeler has done the same.
This public domain release applies worldwide.  In case this is not
legally possible, the right is granted to use the work for any purpose,
without any conditions, unless such conditions are required by law.
See \url{https://creativecommons.org/publicdomain/zero/1.0/}.

\vspace{3ex}
Several short, attributed quotations from copyrighted works
appear in this book under the ``fair use'' provision of Section 107 of
the United States Copyright Act (Title 17 of the {\em United States
Code}).  The public-domain status of this book is not applicable to
those quotations.

\vspace{3ex}
Any trademarks used in this book are the property of their owners.

% QA76.9.L63.M??

% \vspace{1ex}
%
% \vspace{1ex}
% {\small Permission is granted to make and distribute verbatim copies of this
% book
% provided the copyright notice and this
% permission notice are preserved on all copies.}
%
% \vspace{1ex}
% {\small Permission is granted to copy and distribute modified versions of this
% book under the conditions for verbatim copying, provided that the
% entire
% resulting derived work is distributed under the terms of a permission
% notice
% identical to this one.}
%
% \vspace{1ex}
% {\small Permission is granted to copy and distribute translations of this
% book into another language, under the above conditions for modified
% versions,
% except that this permission notice may be stated in a translation
% approved by the
% author.}
%
% \vspace{1ex}
% %{\small   For a copy of the \LaTeX\ source files for this book, contact
% %the author.} \\
% \ \\
% \ \\

\vspace{7ex}
% ISBN: 1-4116-3724-0 \\
% ISBN: 978-1-4116-3724-5 \\
ISBN: 978-0-359-70223-7 \\
{\ } \\
Lulu Press \\
Morrisville, North Carolina\\
USA


\hfill
\vfill

Norman Megill\\ 93 Bridge St., Lexington, MA 02421 \\
E-mail address: \texttt{nm{\char`\@}alum.mit.edu} \\
\vspace{7ex}
David A. Wheeler \\
E-mail address: \texttt{dwheeler{\char`\@}dwheeler.com} \\
% See notes added at end of Preface for revision history. \\
% For current information on the Metamath software see \\
\vspace{7ex}
\url{http://metamath.org}
\end{center}

\hfill
\vfill

{\parindent0pt%
\footnotesize{%
Cover: Aleph null ($\aleph_0$) is the symbol for the
first infinite cardinal number, discovered by Georg Cantor in 1873.
We use a red aleph null (with dark outline and gold glow) as the Metamath logo.
Credit: Norman Megill (1994) and Giovanni Mascellani (2019),
public domain.%
\index{aleph null}%
\index{Metamath!logo}\index{Cantor, Georg}\index{Mascellani, Giovanni}}}

% \newpage
% \thispagestyle{empty}
%
% \hfill
% \vfill
%
% \begin{center}
% {\it To my son Robin Dwight Megill}
% \end{center}
%
% \vfill
% \hfill
%
% \newpage

\tableofcontents
%\listoftables

\chapter*{Preface}
\markboth{PREFACE}{PREFACE}
\addcontentsline{toc}{section}{Preface}


% (For current information, see the notes added at the
% end of this preface on p.~\pageref{note2002}.)

\subsubsection{Overview}

Metamath\index{Metamath} is a computer language and an associated computer
program for archiving, verifying, and studying mathematical proofs at a very
detailed level.  The Metamath language incorporates no mathematics per se but
treats all mathematical statements as mere sequences of symbols.  You provide
Metamath with certain special sequences (axioms) that tell it what rules
of inference are allowed.  Metamath is not limited to any specific field of
mathematics.  The Metamath language is simple and robust, with an
almost total absence of hard-wired syntax, and
we\footnote{Unless otherwise noted, the words
``I,'' ``me,'' and ``my'' refer to Norman Megill\index{Megill, Norman}, while
``we,'' ``us,'' and ``our'' refer to Norman Megill and
David A. Wheeler\index{Wheeler, David A.}.}
believe that it
provides about the simplest possible framework that allows essentially all of
mathematics to be expressed with absolute rigor.

% index test
%\newcommand{\nn}[1]{#1n}
%\index{aaa@bbb}
%\index{abc!def}
%\index{abd|see{qqq}}
%\index{abe|nn}
%\index{abf|emph}
%\index{abg|(}
%\index{abg|)}

Using the Metamath language, you can build formal or mathematical
systems\index{formal system}\footnote{A formal or mathematical system consists
of a collection of symbols (such as $2$, $4$, $+$ and $=$), syntax rules that
describe how symbols may be combined to form a legal expression (called a
well-formed formula or {\em wff}, pronounced ``whiff''), some starting wffs
called axioms, and inference rules that describe how theorems may be derived
(proved) from the axioms.  A theorem is a mathematical fact such as $2+2=4$.
Strictly speaking, even an obvious fact such as this must be proved from
axioms to be formally acceptable to a mathematician.}\index{theorem}
\index{axiom}\index{rule}\index{well-formed formula (wff)} that involve
inferences from axioms.  Although a database is provided
that includes a recommended set of axioms for standard mathematics, if you
wish you can supply your own symbols, syntax, axioms, rules, and definitions.

The name ``Metamath'' was chosen to suggest that the language provides a
means for {\em describing} mathematics rather than {\em being} the
mathematics itself.  Actually in some sense any mathematical language is
metamathematical.  Symbols written on paper, or stored in a computer,
are not mathematics itself but rather a way of expressing mathematics.
For example ``7'' and ``VII'' are symbols for denoting the number seven
in Arabic and Roman numerals; neither {\em is} the number seven.

If you are able to understand and write computer programs, you should be able
to follow abstract mathematics with the aid of Metamath.  Used in conjunction
with standard textbooks, Metamath can guide you step-by-step towards an
understanding of abstract mathematics from a very rigorous viewpoint, even if
you have no formal abstract mathematics background.  By using a single,
consistent notation to express proofs, once you grasp its basic concepts
Metamath provides you with the ability to immediately follow and dissect
proofs even in totally unfamiliar areas.

Of course, just being able follow a proof will not necessarily give you an
intuitive familiarity with mathematics.  Memorizing the rules of chess does not
give you the ability to appreciate the game of a master, and knowing how the
notes on a musical score map to piano keys does not give you the ability to
hear in your head how it would sound.  But each of these can be a first step.

Metamath allows you to explore proofs in the sense that you can see the
theorem referenced at any step expanded in as much detail as you want, right
down to the underlying axioms of logic and set theory (in the case of the set
theory database provided).  While Metamath will not replace the higher-level
understanding that can only be acquired through exercises and hard work, being
able to see how gaps in a proof are filled in can give you increased
confidence that can speed up the learning process and save you time when you
get stuck.

The Metamath language breaks down a mathematical proof into its tiniest
possible parts.  These can be pieced together, like interlocking
pieces in a puzzle, only in a way that produces correct and absolutely rigorous
mathematics.

The nature of Metamath\index{Metamath} enforces very precise mathematical
thinking, similar to that involved in writing a computer program.  A crucial
difference, though, is that once a proof is verified (by the Metamath program)
to be correct, it is definitely correct; it can never have a hidden
``bug.''\index{computer program bugs}  After getting used to the kind of rigor
and accuracy provided by Metamath, you might even be tempted to
adopt the attitude that a proof should never be considered correct until it
has been verified by a computer, just as you would not completely trust a
manual calculation until you have verified it on a
calculator.

My goal
for Metamath was a system for describing and verifying
mathematics that is completely universal yet conceptually as simple as
possible.  In approaching mathematics from an axiomatic, formal viewpoint, I
wanted Metamath to be able to handle almost any mathematical system, not
necessarily with ease, but at least in principle and hopefully in practice. I
wanted it to verify proofs with absolute rigor, and for this reason Metamath
is what might be thought of as a ``compile-only'' language rather than an
algorithmic or Turing-machine language (Pascal, C, Prolog, Mathematica,
etc.).  In other words, a database written in the Metamath
language doesn't ``do'' anything; it merely exhibits mathematical knowledge
and permits this knowledge to be verified as being correct.  A program in an
algorithmic language can potentially have hidden bugs\index{computer program
bugs} as well as possibly being hard to understand.  But each token in a
Metamath database must be consistent with the database's earlier
contents according to simple, fixed rules.
If a database is verified
to be correct,\footnote{This includes
verification that a sequential list of proof steps results in the specified
theorem.} then the mathematical content is correct if the
verifier is correct and the axioms are correct.
The verification program could be incorrect, but the verification algorithm
is relatively simple (making it unlikely to be implemented incorrectly
by the Metamath program),
and there are over a dozen Metamath database verifiers
written by different people in different programming languages
(so these different verifiers can act as multiple reviewers of a database).
The most-used Metamath database, the Metamath Proof Explorer
(aka \texttt{set.mm}\index{set theory database (\texttt{set.mm})}%
\index{Metamath Proof Explorer}),
is currently verified by four different Metamath verifiers written by
four different people in four different languages, including the
original Metamath program described in this book.
The only ``bugs'' that can exist are in the statement of the axioms,
for example if the axioms are inconsistent (a famous problem shown to be
unsolvable by G\"{o}del's incompleteness theorem\index{G\"{o}del's
incompleteness theorem}).
However, real mathematical systems have very few axioms, and these can
be carefully studied.
All of this provides extraordinarily high confidence that the verified database
is in fact correct.

The Metamath program
doesn't prove theorems automatically but is designed to verify proofs
that you supply to it.
The underlying Metamath language is completely general and has no built-in,
preconceived notions about your formal system\index{formal system}, its logic
or its syntax.
For constructing proofs, the Metamath program has a Proof Assistant\index{Proof
Assistant} which helps you fill in some of a proof step's details, shows you
what choices you have at any step, and verifies the proof as you build it; but
you are still expected to provide the proof.

There are many other programs that can process or generate information
in the Metamath language, and more continue to be written.
This is in part because the Metamath language itself is very simple
and intentionally easy to automatically process.
Some programs, such as \texttt{mmj2}\index{mmj2}, include a proof assistant
that can automate some steps beyond what the Metamath program can do.
Mario Carneiro has developed an algorithm for converting proofs from
the OpenTheory interchange format, which can be translated to and from
any of the HOL family of proof languages (HOL4, HOL Light, ProofPower,
and Isabelle), into the
Metamath language \cite{DBLP:journals/corr/Carneiro14}\index{Carneiro, Mario}.
Daniel Whalen has developed Holophrasm, which can automatically
prove many Metamath proofs using
machine learning\index{machine learning}\index{artificial intelligence}
approaches
(including multiple neural networks\index{neural networks})\cite{DBLP:journals/corr/Whalen16}\index{Whalen, Daniel}.
However,
a discussion of these other programs is beyond the scope of this book.

Like most computer languages, the Metamath\index{Metamath} language uses the
standard ({\sc ascii}) characters on a computer keyboard, so it cannot
directly represent many of the special symbols that mathematicians use.  A
useful feature of the Metamath program is its ability to convert its notation
into the \LaTeX\ typesetting language.\index{latex@{\LaTeX}}  This feature
lets you convert the {\sc ascii} tokens you've defined into standard
mathematical symbols, so you end up with symbols and formulas you are familiar
with instead of somewhat cryptic {\sc ascii} representations of them.
The Metamath program can also generate HTML\index{HTML}, making it easy
to view results on the web and to see related information by using
hypertext links.

Metamath is probably conceptually different from anything you've seen
before and some aspects may take some getting used to.  This book will
help you decide whether Metamath suits your specific needs.

\subsubsection{Setting Your Expectations}
It is important for you to understand what Metamath\index{Metamath} is and is
not.  As mentioned, the Metamath program
is {\em not} an automated theorem prover but
rather a proof verifier.  Developing a database can be tedious, hard work,
especially if you want to make the proofs as short as possible, but it becomes
easier as you build up a collection of useful theorems.  The purpose of
Metamath is simply to document existing mathematics in an absolutely rigorous,
computer-verifiable way, not to aid directly in the creation of new
mathematics.  It also is not a magic solution for learning abstract
mathematics, although it may be helpful to be able to actually see the implied
rigor behind what you are learning from textbooks, as well as providing hints
to work out proofs that you are stumped on.

As of this writing, a sizable set theory database has been developed to
provide a foundation for many fields of mathematics, but much more work would
be required to develop useful databases for specific fields.

Metamath\index{Metamath} ``knows no math;'' it just provides a framework in
which to express mathematics.  Its language is very small.  You can define two
kinds of symbols, constants\index{constant} and variables\index{variable}.
The only thing Metamath knows how to do is to substitute strings of symbols
for the variables\index{substitution!variable}\index{variable substitution} in
an expression based on instructions you provide it in a proof, subject to
certain constraints you specify for the variables.  Even the decimal
representation of a number is merely a string of certain constants (digits)
which together, in a specific context, correspond to whatever mathematical
object you choose to define for it; unlike other computer languages, there is
no actual number stored inside the computer.  In a proof, you in effect
instruct Metamath what symbol substitutions to make in previous axioms or
theorems and join a sequence of them together to result in the desired
theorem.  This kind of symbol manipulation captures the essence of mathematics
at a preaxiomatic level.

\subsubsection{Metamath and Mathematical Literature}

In advanced mathematical literature, proofs are usually presented in the form
of short outlines that often only an expert can follow.  This is partly out of
a desire for brevity, but it would also be unwise (even if it were practical)
to present proofs in complete formal detail, since the overall picture would
be lost.\index{formal proof}

A solution I envision\label{envision} that would allow mathematics to remain
acceptable to the expert, yet increase its accessibility to non-specialists,
consists of a combination of the traditional short, informal proof in print
accompanied by a complete formal proof stored in a computer database.  In an
analogy with a computer program, the informal proof is like a set of comments
that describe the overall reasoning and content of the proof, whereas the
computer database is like the actual program and provides a means for anyone,
even a non-expert, to follow the proof in as much detail as desired, exploring
it back through layers of theorems (like subroutines that call other
subroutines) all the way back to the axioms of the theory.  In addition, the
computer database would have the advantage of providing absolute assurance
that the proof is correct, since each step can be verified automatically.

There are several other approaches besides Metamath to a project such
as this.  Section~\ref{proofverifiers} discusses some of these.

To us, a noble goal would be a database with hundreds of thousands of
theorems and their computer-verifiable proofs, encompassing a significant
fraction of known mathematics and available for instant access.
These would be fully verified by multiple independently-implemented verifiers,
to provide extremely high confidence that the proofs are completely correct.
The database would allow people to investigate whatever details they were
interested in, so that they could confirm whatever portions they wished.
Whether or not Metamath is an appropriate choice remains to be seen, but in
principle we believe it is sufficient.

\subsubsection{Formalism}

Over the past fifty years, a group of French mathematicians working
collectively under the pseudonym of Bourbaki\index{Bourbaki, Nicolas} have
co-authored a series of monographs that attempt to rigorously and
consistently formalize large bodies of mathematics from foundations.  On the
one hand, certainly such an effort has its merits; on the other hand, the
Bourbaki project has been criticized for its ``scholasticism'' and
``hyperaxiomatics'' that hide the intuitive steps that lead to the results
\cite[p.~191]{Barrow}\index{Barrow, John D.}.

Metamath unabashedly carries this philosophy to its extreme and no doubt is
subject to the same kind of criticism.  Nonetheless I think that in
conjunction with conventional approaches to mathematics Metamath can serve a
useful purpose.  The Bourbaki approach is essentially pedagogic, requiring the
reader to become intimately familiar with each detail in a very large
hierarchy before he or she can proceed to the next step.  The difference with
Metamath is that the ``reader'' (user) knows that all details are contained in
its computer database, available as needed; it does not demand that the user
know everything but conveniently makes available those portions that are of
interest.  As the body of all mathematical knowledge grows larger and larger,
no one individual can have a thorough grasp of its entirety.  Metamath
can finalize and put to rest any questions about the validity of any part of it
and can make any part of it accessible, in principle, to a non-specialist.

\subsubsection{A Personal Note}
Why did I develop Metamath\index{Metamath}?  I enjoy abstract mathematics, but
I sometimes get lost in a barrage of definitions and start to lose confidence
that my proofs are correct.  Or I reach a point where I lose sight of how
anything I'm doing relates to the axioms that a theory is based on and am
sometimes suspicious that there may be some overlooked implicit axiom
accidentally introduced along the way (as happened historically with Euclidean
geometry\index{Euclidean geometry}, whose omission of Pasch's
axiom\index{Pasch's axiom} went unnoticed for 2000 years
\cite[p.~160]{Davis}!). I'm also somewhat lazy and wish to avoid the effort
involved in re-verifying the gaps in informal proofs ``left to the reader;'' I
prefer to figure them out just once and not have to go through the same
frustration a year from now when I've forgotten what I did.  Metamath provides
better recovery of my efforts than scraps of paper that I can't
decipher anymore.  But mostly I find very appealing the idea of rigorously
archiving mathematical knowledge in a computer database, providing precision,
certainty, and elimination of human error.

\subsubsection{Note on Bibliography and Index}

The Bibliography usually includes the Library of Congress classification
for a work to make it easier for you to find it in on a university
library shelf.  The Index has author references to pages where their works
are cited, even though the authors' names may not appear on those pages.

\subsubsection{Acknowledgments}

Acknowledgments are first due to my wife, Deborah (who passed away on
September 4, 1998), for critiquing the manu\-script but most of all for
her patience and support.  I also wish to thank Joe Wright, Richard
Becker, Clarke Evans, Buddha Buck, and Jeremy Henty for helpful
comments.  Any errors, omissions, and other shortcomings are of course
my responsibility.

\subsubsection{Note Added June 22, 2005}\label{note2002}

The original, unpublished version of this book was written in 1997 and
distributed via the web.  The present edition has been updated to
reflect the current Metamath program and databases, as well as more
current {\sc url}s for Internet sites.  Thanks to Josh
Purinton\index{Purinton, Josh}, One Hand
Clapping, Mel L.\ O'Cat, and Roy F. Longton for pointing out
typographical and other errors.  I have also benefitted from numerous
discussions with Raph Levien\index{Levien, Raph}, who has extended
Metamath's philosophy of rigor to result in his {\em
Ghilbert}\index{Ghilbert} proof language (\url{http://ghilbert.org}).

Robert (Bob) Solovay\index{Solovay, Robert} communicated a new result of
A.~R.~D.~Mathias on the system of Bourbaki, and the text has been
updated accordingly (p.~\pageref{bourbaki}).

Bob also pointed out a clarification of the literature regarding
category theory and inaccessible cardinals\index{category
theory}\index{cardinal, inaccessible} (p.~\pageref{categoryth}),
and a misleading statement was removed from the text.  Specifically,
contrary to a statement in previous editions, it is possible to express
``There is a proper class of inaccessible cardinals'' in the language of
ZFC.  This can be done as follows:  ``For every set $x$ there is an
inaccessible cardinal $\kappa$ such that $\kappa$ is not in $x$.''
Bob writes:\footnote{Private communication, Nov.~30, 2002.}
\begin{quotation}
     This axiom is how Grothendieck presents category theory.  To each
inaccessible cardinal $\kappa$ one associates a Grothendieck universe
\index{Grothendieck, Alexander} $U(\kappa)$.  $U(\kappa)$ consists of
those sets which lie in a transitive set of cardinality less than
$\kappa$.  Instead of the ``category of all groups,'' one works relative
to a universe [considering the category of groups of cardinality less
than $\kappa$].  Now the category whose objects are all categories
``relative to the universe $U(\kappa)$'' will be a category not
relative to this universe but to the next universe.

     All of the things category theorists like to do can be done in this
framework.  The only controversial point is whether the Grothen\-dieck
axiom is too strong for the needs of category theorists.  Mac Lane
\index{Mac Lane, Saunders} argues that ``one universe is enough'' and
Feferman\index{Feferman, Solomon} has argued that one can get by with
ordinary ZFC.  I don't find Feferman's arguments persuasive.  Mac Lane
may be right, but when I think about category theory I do it \`{a} la
Grothendieck.

        By the way Mizar\index{Mizar} adds the axiom ``there is a proper
class of inaccessibles'' precisely so as to do category theory.
\end{quotation}

The most current information on the Metamath program and databases can
always be found at \url{http://metamath.org}.


\subsubsection{Note Added June 24, 2006}\label{note2006}

The Metamath spec was restricted slightly to make parsers easier to
write.  See the footnote on p.~\pageref{namespace}.

%\subsubsection{Note Added July 24, 2006}\label{note2006b}
\subsubsection{Note Added March 10, 2007}\label{note2006b}

I am grateful to Anthony Williams\index{Williams, Anthony} for writing
the \LaTeX\ package called {\tt realref.sty} and contributing it to the
public domain.  This package allows the internal hyperlinks in a {\sc
pdf} file to anchor to specific page numbers instead of just section
titles, making the navigation of the {\sc pdf} file for this book much
more pleasant and ``logical.''

A typographical error found by Martin Kiselkov was corrected.
A confusing remark about unification was deleted per suggestion of
Mel O'Cat.

\subsubsection{Note Added May 27, 2009}\label{note2009}

Several typos found by Kim Sparre were corrected.  A note was added that
the Poincar\'{e} conjecture has been proved (p.~\pageref{poincare}).

\subsubsection{Note Added Nov. 17, 2014}\label{note2014}

The statement of the Schr\"{o}der--Bernstein theorem was corrected in
Section~\ref{trust}.  Thanks to Bob Solovay for pointing out the error.

\subsubsection{Note Added May 25, 2016}\label{note2016}

Thanks to Jerry James for correcting 16 typos.

\subsubsection{Note Added February 25, 2019}\label{note201902}

David A. Wheeler\index{Wheeler, David A.}
made a large number of improvements and updates,
in coordination with Norman Megill.
The predicate calculus axioms were renumbered, and the text makes
it clear that they are based on Tarski's system S2;
the one slight deviation in axiom ax-6 is explained and justified.
The real and complex number axioms were modified to be consistent with
\texttt{set.mm}\index{set theory database (\texttt{set.mm})}%
\index{Metamath Proof Explorer}.
Long-awaited specification changes ``1--8'' were made,
which clarified previously ambiguous points.
Some errors in the text involving \texttt{\$f} and
\texttt{\$d} statements were corrected (the spec was correct, but
the in-book explanations unintentionally contradicted the spec).
We now have a system for automatically generating narrow PDFs,
so that those with smartphones can have easy access to the current
version of this document.
A new section on deduction was added;
it discusses the standard deduction theorem,
the weak deduction theorem,
deduction style, and natural deduction.
Many minor corrections (too numerous to list here) were also made.

\subsubsection{Note Added March 7, 2019}\label{note201903}

This added a description of the Matamath language syntax in
Extended Backus--Naur Form (EBNF)\index{Extended Backus--Naur Form}\index{EBNF}
in Appendix \ref{BNF}, added a brief explanation about typecodes,
inserted more examples in the deduction section,
and added a variety of smaller improvements.

\subsubsection{Note Added April 7, 2019}\label{note201904}

This version clarified the proper substitution notation, improved the
discussion on the weak deduction theorem and natural deduction,
documented the \texttt{undo} command, updated the information on
\texttt{write source}, changed the typecode
from \texttt{set} to \texttt{setvar} to be consistent with the current
version of \texttt{set.mm}, added more documentation about comment markup
(e.g., documented how to create headings), and clarified the
differences between various assertion forms (in particular deduction form).

\subsubsection{Note Added June 2, 2019}\label{note201906}

This version fixes a large number of small issues reported by
Beno\^{i}t Jubin\index{Jubin, Beno\^{i}t}, such as editorial issues
and the need to document \texttt{verify markup} (thank you!).
This version also includes specific examples
of forms (deduction form, inference form, and closed form).
We call this version the ``second edition'';
the previous edition formally published in 2007 had a slightly different title
(\textit{Metamath: A Computer Language for Pure Mathematics}).

\chapter{Introduction}
\pagenumbering{arabic}

\begin{quotation}
  {\em {\em I.M.:}  No, no.  There's nothing subjective about it!  Everybody
knows what a proof is.  Just read some books, take courses from a competent
mathematician, and you'll catch on.

{\em Student:}  Are you sure?

{\em I.M.:}  Well---it is possible that you won't, if you don't have any
aptitude for it.  That can happen, too.

{\em Student:}  Then {\em you} decide what a proof is, and if I don't learn
to decide in the same way, you decide I don't have any aptitude.

{\em I.M.:}  If not me, then who?}
    \flushright\sc  ``The Ideal Mathematician''
    \index{Davis, Phillip J.}
    \footnote{\cite{Davis}, p.~40.}\\
\end{quotation}

Brilliant mathematicians have discovered almost
unimaginably profound results that rank among the crowning intellectual
achievements of mankind.  However, there is a sense in which modern abstract
mathematics is behind the times, stuck in an era before computers existed.
While no one disputes the remarkable results that have been achieved,
communicating these results in a precise way to the uninitiated is virtually
impossible.  To describe these results, a terse informal language is used which
despite its elegance is very difficult to learn.  This informal language is not
imprecise, far from it, but rather it often has omitted detail
and symbols with hidden context that are
implicitly understood by an expert but few others.  Extremely complex technical
meanings are associated with innocent-sounding English words such as
``compact'' and ``measurable'' that barely hint at what is actually being
said.  Anyone who does not keep the precise technical meaning constantly in
mind is bound to fail, and acquiring the ability to do this can be achieved
only through much practice and hard work.  Only the few who complete the
painful learning experience can join the small in-group of pure
mathematicians.  The informal language effectively cuts off the true nature of
their knowledge from most everyone else.

Metamath\index{Metamath} makes abstract mathematics more concrete.  It allows
a computer to keep track of the complexity associated with each word or symbol
with absolute rigor.  You can explore this complexity at your leisure, to
whatever degree you desire.  Whether or not you believe that concepts such as
infinity actually ``exist'' outside of the mind, Metamath lets you get to the
foundation for what's really being said.

Metamath also enables completely rigorous and thorough proof verification.
Its language is simple enough so that you
don't have to rely on the authority of experts but can verify the results
yourself, step by step.  If you want to attempt to derive your own results,
Metamath will not let you make a mistake in reasoning.
Even professional mathematicians make mistakes; Metamath makes it possible
to thoroughly verify that proofs are correct.

Metamath\index{Metamath} is a computer language and an associated computer
program for archiving, verifying, and studying mathematical proofs at a very
detailed level.
The Metamath language
describes formal\index{formal system} mathematical
systems and expresses proofs of theorems in those systems.  Such a language
is called a metalanguage\index{metalanguage} by mathematicians.
The Metamath program is a computer program that verifies
proofs expressed in the Metamath language.
The Metamath program does not have the built-in
ability to make logical inferences; it just makes a series of symbol
substitutions according to instructions given to it in a proof
and verifies that the result matches the expected theorem.  It makes logical
inferences based only on rules of logic that are contained in a set of
axioms\index{axiom}, or first principles, that you provide to it as the
starting point for proofs.

The complete specification of the Metamath language is only four pages long
(Section~\ref{spec}, p.~\pageref{spec}).  Its simplicity may at first make you
wonder how it can do much of anything at all.  But in fact the kinds of
symbol manipulations it performs are the ones that are implicitly done in all
mathematical systems at the lowest level.  You can learn it relatively quickly
and have complete confidence in any mathematical proof that it verifies.  On
the other hand, it is powerful and general enough so that virtually any
mathematical theory, from the most basic to the deeply abstract, can be
described with it.

Although in principle Metamath can be used with any
kind of mathematics, it is best suited for abstract or ``pure'' mathematics
that is mostly concerned with theorems and their proofs, as opposed to the
kind of mathematics that deals with the practical manipulation of numbers.
Examples of branches of pure mathematics are logic\index{logic},\footnote{Logic
is the study of statements that are universally true regardless of the objects
being described by the statements.  An example is the statement, ``if $P$
implies $Q$, then either $P$ is false or $Q$ is true.''} set theory\index{set
theory},\footnote{Set theory is the study of general-purpose mathematical objects called
``sets,'' and from it essentially all of mathematics can be derived.  For
example, numbers can be defined as specific sets, and their properties
can be explored using the tools of set theory.} number theory\index{number
theory},\footnote{Number theory deals with the properties of positive and
negative integers (whole numbers).} group theory\index{group
theory},\footnote{Group theory studies the properties of mathematical objects
called groups that obey a simple set of axioms and have properties of symmetry
that make them useful in many other fields.} abstract algebra\index{abstract
algebra},\footnote{Abstract algebra includes group theory and also studies
groups with additional properties that qualify them as ``rings'' and
``fields.''  The set of real numbers is a familiar example of a field.},
analysis\index{analysis} \index{real and complex numbers}\footnote{Analysis is
the study of real and complex numbers.} and
topology\index{topology}.\footnote{One area studied by topology are properties
that remain unchanged when geometrical objects undergo stretching
deformations; for example a doughnut and a coffee cup each have one hole (the
cup's hole is in its handle) and are thus considered topologically
equivalent.  In general, though, topology is the study of abstract
mathematical objects that obey a certain (surprisingly simple) set of axioms.
See, for example, Munkres \cite{Munkres}\index{Munkres, James R.}.} Even in
physics, Metamath could be applied to certain branches that make use of
abstract mathematics, such as quantum logic\index{quantum logic} (used to study
aspects of quantum mechanics\index{quantum mechanics}).

On the other hand, Metamath\index{Metamath} is less suited to applications
that deal primarily with intensive numeric computations.  Metamath does not
have any built-in representation of numbers\index{Metamath!representation of
numbers}; instead, a specific string of symbols (digits) must be syntactically
constructed as part of any proof in which an ordinary number is used.  For
this reason, numbers in Metamath are best limited to specific constants that
arise during the course of a theorem or its proof.  Numbers are only a tiny
part of the world of abstract mathematics.  The exclusion of built-in numbers
was a conscious decision to help achieve Metamath's simplicity, and there are
other software tools if you have different mathematical needs.
If you wish to quickly solve algebraic problems, the computer algebra
programs\index{computer algebra system} {\sc
macsyma}\index{macsyma@{\sc macsyma}}, Mathematica\index{Mathematica}, and
Maple\index{Maple} are specifically suited to handling numbers and
algebra efficiently.
If you wish to simply calculate numeric or matrix expressions easily,
tools such as Octave\index{Octave} may be a better choice.

After learning Metamath's basic statement types, any
tech\-ni\-cal\-ly ori\-ent\-ed person, mathematician or not, can
immediately trace
any theorem proved in the language as far back as he or she wants, all the way
to the axioms on which the theorem is based.  This ability suggests a
non-traditional way of learning about pure mathematics.  Used in conjunction
with traditional methods, Metamath could make pure mathematics accessible to
people who are not sufficiently skilled to figure out the implicit detail in
ordinary textbook proofs.  Once you learn the axioms of a theory, you can have
complete confidence that everything you need to understand a proof you are
studying is all there, at your beck and call, allowing you to focus in on any
proof step you don't understand in as much depth as you need, without worrying
about getting stuck on a step you can't figure out.\footnote{On the other
hand, writing proofs in the Metamath language is challenging, requiring
a degree of rigor far in excess of that normally taught to students.  In a
classroom setting, I doubt that writing Metamath proofs would ever replace
traditional homework exercises involving informal proofs, because the time
needed to work out the details would not allow a course to
cover much material.  For students who have trouble grasping the implied rigor
in traditional material, writing a few simple proofs in the Metamath language
might help clarify fuzzy thought processes.  Although somewhat difficult at
first, it eventually becomes fun to do, like solving a puzzle, because of the
instant feedback provided by the computer.}

Metamath\index{Metamath} is probably unlike anything you have
encountered before.  In this first chapter we will look at the philosophy and
use of computers in mathematics in order to better understand the motivation
behind Metamath.  The material in this chapter is not required in order to use
Metamath.  You may skip it if you are impatient, but I hope you will find it
educational and enjoyable.  If you want to start experimenting with the
Metamath program right away, proceed directly to Chapter~\ref{using}
(p.~\pageref{using}).  To
learn the Metamath language, skim Chapter~\ref{using} then proceed to
Chapter~\ref{languagespec} (p.~\pageref{languagespec}).

\section{Mathematics as a Computer Language}

\begin{quote}
  {\em The study of mathematics is apt to commence in
dis\-ap\-point\-ment.\ldots \\
We are told that by its aid the stars are weighted
and the billions of molecules in a drop of water are counted.  Yet, like the
ghost of Hamlet's father, this great science eludes the efforts of our mental
weapons to grasp it.}
  \flushright\sc  Alfred North Whitehead\footnote{\cite{Whitehead}, ch.\ 1.}\\
\end{quote}\index{Whitehead, Alfred North}

\subsection{Is Mathematics ``User-Friendly''?}

Suppose you have no formal training in abstract mathematics.  But popular
books you've read offer tempting glimpses of this world filled with profound
ideas that have stirred the human spirit.  You are not satisfied with the
informal, watered-down descriptions you've read but feel it is important to
grasp the underlying mathematics itself to understand its true meaning. It's
not practical to go back to school to learn it, though; you don't want to
dedicate years of your life to it.  There are many important things in life,
and you have to set priorities for what's important to you.  What would happen
if you tried to pursue it on your own, in your spare time?

After all, you were able to learn a computer programming language such as
Pascal on your own without too much difficulty, even though you had no formal
training in computers.  You don't claim to be an expert in software design,
but you can write a passable program when necessary to suit your needs.  Even
more important, you know that you can look at anyone else's Pascal program, no
matter how complex, and with enough patience figure out exactly how it works,
even though you are not a specialist.  Pascal allows you do anything that a
computer can do, at least in principle.  Thus you know you have the ability,
in principle, to follow anything that a computer program can do:  you just
have to break it down into small enough pieces.

Here's an imaginary scenario of what might happen if you na\-ive\-ly a\-dopted
this same view of abstract mathematics and tried to pick it up on your own, in
a period of time comparable to, say, learning a computer programming
language.

\subsubsection{A Non-Mathematician's Quest for Truth}

\begin{quote}
  {\em \ldots my daughters have been studying (chemistry) for several
se\-mes\-ters, think they have learned differential and integral calculus in
school, and yet even today don't know why $x\cdot y=y\cdot x$ is true.}
  \flushright\sc  Edmund Landau\footnote{\cite{Landau}, p.~vi.}\\
\end{quote}\index{Landau, Edmund}

\begin{quote}
  {\em Minus times minus is plus,\\
The reason for this we need not discuss.}
  \flushright\sc W.\ H.\ Auden\footnote{As quoted in \cite{Guillen}, p.~64.}\\
\end{quote}\index{Auden, W.\ H.}\index{Guillen, Michael}

We'll suppose you are a technically oriented professional, perhaps an engineer, a
computer programmer, or a physicist, but probably not a mathematician.  You
consider yourself reasonably intelligent.  You did well in school, learning a
variety of methods and techniques in practical mathematics such as calculus and
differential equations.  But rarely did your courses get into anything
resembling modern abstract mathematics, and proofs were something that appeared
only occasionally in your textbooks, a kind of necessary evil that was
supposed to convince you of a certain key result.  Most of your
homework consisted of exercises that gave you practice in the techniques, and
you were hardly ever asked to come up with a proof of your own.

You find yourself curious about advanced, abstract mathematics.  You are
driven by an inner conviction that it is important to understand and
appreciate some of the most profound knowledge discovered by mankind.  But it
seems very hard to learn, something that only certain gifted longhairs can
access and understand.  You are frustrated that it seems forever cut off from
you.

Eventually your curiosity drives you to do something about it.
You set for yourself a goal of ``really'' understanding mathematics:  not just
how to manipulate equations in algebra or calculus according to cookbook
rules, but rather to gain a deep understanding of where those rules come from.
In fact, you're not thinking about this kind of ordinary mathematics at all,
but about a much more abstract, ethereal realm of pure mathematics, where
famous results such as G\"{o}del's incompleteness theorem\index{G\"{o}del's
incompleteness theorem} and Cantor's different kinds of infinities
reside.

You have probably read a number of popular books, with titles like {\em
Infinity and the Mind} \cite{Rucker}\index{Rucker, Rudy}, on topics such as
these.  You found them inspiring but at the same time somewhat
unsatisfactory.  They gave you a general idea of what these results are about,
but if someone asked you to prove them, you wouldn't have the faintest idea of
where to begin.   Sure, you could give the same overall outline that you
learned from the popular books; and in a general sort of way, you do have an
understanding.  But deep down inside, you know that there is a rigor that is
missing, that probably there are many subtle steps and pitfalls along the way,
and ultimately it seems you have to place your trust in the experts in the
field.  You don't like this; you want to be able to verify these results for
yourself.

So where do you go next?  As a first step, you decide to look up some of the
original papers on the theorems you are curious about, or better, obtain some
standard textbooks in the field.  You look up a theorem you want to
understand.  Sure enough, it's there, but it's expressed with strange
terms and odd symbols that mean absolutely nothing to you.  It might as well be written in
a foreign language you've never seen before, whose symbols are totally alien.
You look at the proof, and you haven't the foggiest notion what each step
means, much less how one step follows from another.  Well, obviously you have
a lot to learn if you want to understand this stuff.

You feel that you could probably understand it by
going back to college for another three to six years and getting a math
degree.  But that does not fit in with your career and the other things in
your life and would serve no practical purpose.  You decide to seek a quicker
path.  You figure you'll just trace your way back to the beginning, step by
step, as you would do with a computer program, until you understand it.  But
you quickly find that this is not possible, since you can't even understand
enough to know what you have to trace back to.

Maybe a different approach is in order---maybe you should start at the
beginning and work your way up.  First, you read the introduction to the book
to find out what the prerequisites are.  In a similar fashion, you trace your
way back through two or three more books, finally arriving at one that seems
to start at a beginning:  it lists the axioms of arithmetic.  ``Aha!'' you
naively think, ``This must be the starting point, the source of all mathematical
knowledge.'' Or at least the starting point for mathematics dealing with
numbers; you have to start somewhere and have no idea what the starting point
for other mathematics would be.  But the word ``axioms'' looks promising.  So
you eagerly read along and work through some elementary exercises at the
beginning of the book.  You feel vaguely bothered:  these
don't seem like axioms at all, at least not in the sense that you want to
think of axioms.  Axioms imply a starting point from which everything else can
be built up, according to precise rules specified in the axiom system.  Even
though you can understand the first few proofs in an informal way,
and are able to do some of the
exercises, it's hard to pin down precisely what the
rules are.   Sure, each step seems to follow logically from the others, but
exactly what does that mean?  Is the ``logic'' just a matter of common sense,
something vague that we all understand but can never quite state precisely?

You've spent a number of years, off and on, programming computers, and you
know that in the case of computer languages there is no question of what the
rules are---they are precise and crystal clear.  If you follow them, your
program will work, and if you don't, it won't.  No matter how complex a
program, it can always be broken down into simpler and simpler pieces, until
you can ultimately identify the bits that are moved around to perform a
specific function.  Some programs might require a lot of perseverance to
accomplish this, but if you focus on a specific portion of it, you don't even
necessarily have to know how the rest of it works. Shouldn't there be an
analogy in mathematics?

You decide to apply the ultimate test:  you ask yourself how a computer could
verify or ensure that the steps in these proofs follow from one another.
Certainly mathematics must be at least as precisely defined as a computer
language, if not more so; after all, computer science itself is based on it.
If you can get a computer to verify these proofs, then you should also be
able, in principle, to understand them yourself in a very crystal clear,
precise way.

You're in for a surprise:  you can conceive of no way to convert the
proofs, which are in English, to a form that the computer can understand.
The proofs are filled with phrases such as ``assume there exists a unique
$x$\ldots'' and ``given any $y$, let $z$ be the number such that\ldots''  This
isn't the kind of logic you are used to in computer programming, where
everything, even arithmetic, reduces to Boolean ones and zeroes if you care to
break it down sufficiently.  Even though you think you understand the proofs,
there seems to be some kind of higher reasoning involved rather than precise
rules that define how you manipulate the symbols in the axioms.  Whatever it
is, it just isn't obvious how you would express it to a computer, and the more
you think about it, the more puzzled and confused you get, to the point where
you even wonder whether {\em you} really understand it.  There's a lot more to
these axioms of arithmetic than meets the eye.

Nobody ever talked about this in school in your applied math and engineering
courses.  You just learned the rules they gave you, not quite understanding
how or why they worked, sometimes vaguely suspicious or uncertain of them, and
through homework problems and osmosis learned how to present solutions that
satisfied the instructor and earned you an ``A.''  Rarely did you actually
``prove'' anything in a rigorous way, and the math majors who did do stuff
like that seemed to be in a different world.

Of course, there are computer algebra programs that can do mathematics, and
rather impressively.  They can instantly solve the integrals that you
struggled with in freshman calculus, and do much, much more.  But when you
look at these programs, what you see is a big collection of algorithms and
techniques that evolved and were added to over time, along with some basic
software that manipulates symbols.  Each algorithm that is built in is the
result of someone's theorem whose proof is omitted; you just have to trust the
person who proved it and the person who programmed it in and hope there are no
bugs.\index{computer program bugs}  Somehow this doesn't seem to be the
essence of mathematics.  Although computer algebra systems can generate
theorems with amazing speed, they can't actually prove a single one of them.

After some puzzlement, you revisit some popular books on what mathematics is
all about.  Somewhere you read that all of mathematics is actually derived
from something called ``set theory.''  This is a little confusing, because
nowhere in the book that presented the axioms of arithmetic was there any
mention of set theory, or if there was, it seemed to be just a tool that helps
you describe things better---the set of even numbers, that sort of thing.  If
set theory is the basis for all mathematics, then why are additional axioms
needed for arithmetic?

Something is wrong but you're not sure what.  One of your friends is a pure
mathematician.  He knows he is unable to communicate to you what he does for a
living and seems to have little interest in trying.  You do know that for him,
proofs are what mathematics is all about. You ask him what a proof is, and he
essentially tells you that, while of course it's based on logic, really it's
something you learn by doing it over and over until you pick it up.  He refers
you to a book, {\em How to Read and Do Proofs} \cite{Solow}.\index{Solow,
Daniel}  Although this book helps you understand traditional informal proofs,
there is still something missing you can't seem to pin down yet.

You ask your friend how you would go about having a computer verify a proof.
At first he seems puzzled by the question; why would you want to do that?
Then he says it's not something that would make any sense to do, but he's
heard that you'd have to break the proof down into thousands or even millions
of individual steps to do such a thing, because the reasoning involved is at
such a high level of abstraction.  He says that maybe it's something you could
do up to a point, but the computer would be completely impractical once you
get into any meaningful mathematics.  There, the only way you can verify a
proof is by hand, and you can only acquire the ability to do this by
specializing in the field for a couple of years in grad school.  Anyway, he
thinks it all has to do with set theory, although he has never taken a formal
course in set theory but just learned what he needed as he went along.

You are intrigued and amazed.  Apparently a mathematician can grasp as a
single concept something that would take a computer a thousand or a million
steps to verify, and have complete confidence in it.  Each one of these
thousand or million steps must be absolutely correct, or else the whole proof
is meaningless.  If you added a million numbers by hand, would you trust the
result?  How do you really know that all these steps are correct, that there
isn't some subtle pitfall in one of these million steps, like a bug in a
computer program?\index{computer program bugs}  After all, you've read that
famous mathematicians have occasionally made mistakes, and you certainly know
you've made your share on your math homework problems in school.

You recall the analogy with a computer program.  Sure, you can understand what
a large computer program such as a word processor does, as a single high-level
concept or a small set of such concepts, but your ability to understand it in
no way ensures that the program is correct and doesn't have hidden bugs.  Even
if you wrote the program yourself you can't really know this; most large
programs that you've written have had bugs that crop up at some later date, no
matter how careful you tried to be while writing them.

OK, so now it seems the reason you can't figure out how to make a
computer verify proofs is because each step really corresponds to a
million small steps.  Well, you say, a computer can do a million
calculations in a second, so maybe it's still practical to do.  Now the
puzzle becomes how to figure out what the million steps are that each
English-language step corresponds to.  Your mathematician friend hasn't
a clue, but suggests that maybe you would find the answer by studying
set theory.  Actually, your friend thinks you're a little off the wall
for even wondering such a thing.  For him, this is not what mathematics
is all about.

The subject of set theory keeps popping up, so you decide it's
time to look it up.

You decide to start off on a careful footing, so you start reading a couple of
very elementary books on set theory.  A lot of it seems pretty obvious, like
intersections, subsets, and Venn diagrams.  You thumb through one of the
books; nowhere is anything about axioms mentioned. The other book relegates to
an appendix a brief discussion that mentions a set of axioms called
``Zermelo--Fraenkel set theory''\index{Zermelo--Fraenkel set theory} and states
them in English.  You look at them and have no idea what they really mean or
what you can do with them.  The comments in this appendix say that the purpose
of mentioning them is to expose you to the idea, but imply that they are not
necessary for basic understanding and that they are really the subject matter
of advanced treatments where fine points such as a certain paradox (Russell's
paradox\index{Russell's paradox}\footnote{Russell's paradox assumes that there
exists a set $S$ that is a collection of all sets that don't contain
themselves.  Now, either $S$ contains itself or it doesn't.  If it contains
itself, it contradicts its definition.  But if it doesn't contain itself, it
also contradicts its definition.  Russell's paradox is resolved in ZF set
theory by denying that such a set $S$ exists.}) are resolved.  Wait a
minute---shouldn't the axioms be a starting point, not an ending point?  If
there are paradoxes that arise without the axioms, how do you know you won't
stumble across one accidentally when using the informal approach?

And nowhere do these books describe how ``all of mathematics can be
derived from set theory'' which by now you've heard a few times.

You find a more advanced book on set theory.  This one actually lists the
axioms of ZF set theory in plain English on page one.  {\em Now} you think
your quest has ended and you've finally found the source of all mathematical
knowledge; you just have to understand what it means.  Here, in one place, is
the basis for all of mathematics!  You stare at the axioms in awe, puzzle over
them, memorize them, hoping that if you just meditate on them long enough they
will become clear.  Of course, you haven't the slightest idea how the rest of
mathematics is ``derived'' from them; in particular, if these are the axioms
of mathematics, then why do arithmetic, group theory, and so on need their own
axioms?

You start reading this advanced book carefully, pondering the meaning of every
word, because by now you're really determined to get to the bottom of this.
The first thing the book does is explain how the axioms came about, which was
to resolve Russell's paradox.\index{Russell's paradox}  In fact that seems to
be the main purpose of their existence; that they supposedly can be used to
derive all of mathematics seems irrelevant and is not even mentioned.  Well,
you go on.  You hope the book will explain to you clearly, step by step, how
to derive things from the axioms.  After all, this is the starting point of
mathematics, like a book that explains the basics of a computer programming
language.  But something is missing.  You find you can't even understand the
first proof or do the first exercise.  Symbols such as $\exists$ and $\forall$
permeate the page without any mention of where they came from or how to
manipulate them; the author assumes you are totally familiar with them and
doesn't even tell you what they mean.  By now you know that $\exists$ means
``there exists'' and $\forall$ means ``for all,'' but shouldn't the rules for
manipulating these symbols be part of the axioms?  You still have no idea
how you could even describe the axioms to a computer.

Certainly there is something much different here from the technical
literature you're used to reading.  A computer language manual almost
always explains very clearly what all the symbols mean, precisely what
they do, and the rules used for combining them, and you work your way up
from there.

After glancing at four or five other such books, you come to the realization
that there is another whole field of study that you need just to get to the
point at which you can understand the axioms of set theory.  The field is
called ``logic.''  In fact, some of the books did recommend it as a
prerequisite, but it just didn't sink in.  You assumed logic was, well, just
logic, something that a person with common sense intuitively understood.  Why
waste your time reading boring treatises on symbolic logic, the manipulation
of 1's and 0's that computers do, when you already know that?  But this is a
different kind of logic, quite alien to you.  The subject of {\sc nand} and
{\sc nor} gates is not even touched upon or in any case has to do with only a
very small part of this field.

So your quest continues.  Skimming through the first couple of introductory
books, you get a general idea of what logic is about and what quantifiers
(``for all,'' ``there exists'') mean, but you find their examples somewhat
trivial and mildly annoying (``all dogs are animals,'' ``some animals are
dogs,'' and such).  But all you want to know is what the rules are for
manipulating the symbols so you can apply them to set theory.  Some formulas
describing the relationships among quantifiers ($\exists$ and $\forall$) are
listed in tables, along with some verbal reasoning to justify them.
Presumably, if you want to find out if a formula is correct, you go through
this same kind of mental reasoning process, possibly using images of dogs and
animals. Intuitively, the formulas seem to make sense.  But when you ask
yourself, ``What are the rules I need to get a computer to figure out whether
this formula is correct?'', you still don't know.  Certainly you don't ask the
computer to imagine dogs and animals.

You look at some more advanced logic books.  Many of them have an introductory
chapter summarizing set theory, which turns out to be a prerequisite.  You
need logic to understand set theory, but it seems you also need set theory to
understand logic!  These books jump right into proving rather advanced
theorems about logic, without offering the faintest clue about where the logic
came from that allows them to prove these theorems.

Luckily, you come across an elementary book of logic that, halfway through,
after the usual truth tables and metaphors, presents in a clear, precise way
what you've been looking for all along: the axioms!  They're divided into
propositional calculus (also called sentential logic) and predicate calculus
(also called first-order logic),\index{first-order logic} with rules so simple
and crystal clear that now you can finally program a computer to understand
them.  Indeed, they're no harder than learning how to play a game of chess.
As far as what you seem to need is concerned, the whole book could have been
written in five pages!

{\em Now} you think you've found the ultimate source of mathematical
truth.  So---the axioms of mathematics consist of these axioms of logic,
together with the axioms of ZF set theory. (By now you've also been able to
figure out how to translate the ZF axioms from English into the
actual symbols of logic which you can now manipulate according to
precise, easy-to-understand rules.)

Of course, you still don't understand how ``all of mathematics can be
derived from set theory,'' but maybe this will reveal itself in due
course.

You eagerly set out to program the axioms and rules into a computer and start
to look at the theorems you will have to prove as the logic is developed.  All
sorts of important theorems start popping up:  the deduction
theorem,\index{deduction theorem} the substitution theorem,\index{substitution
theorem} the completeness theorem of propositional calculus,\index{first-order
logic!completeness} the completeness theorem of predicate calculus.  Uh-oh,
there seems to be trouble.  They all get harder and harder, and not one of
them can be derived with the axioms and rules of logic you've just been
handed.  Instead, they all require ``metalogic'' for their proofs, a kind of
mixture of logic and set theory that allows you to prove things {\em about}
the axioms and theorems of logic rather than {\em with} them.

You plow ahead anyway.  A month later, you've spent much of your
free time getting the computer to verify proofs in propositional calculus.
You've programmed in the axioms, but you've also had to program in the
deduction theorem, the substitution theorem, and the completeness theorem of
propositional calculus, which by now you've resigned yourself to treating as
rather complex additional axioms, since they can't be proved from the axioms
you were given.  You can now get the computer to verify and even generate
complete, rigorous, formal proofs\index{formal proof}.  Never mind that they
may have 100,000 steps---at least now you can have complete, absolute
confidence in them.  Unfortunately, the only theorems you have proved are
pretty trivial and you can easily verify them in a few minutes with truth
tables, if not by inspection.

It looks like your mathematician friend was right.  Getting the computer to do
serious mathematics with this kind of rigor seems almost hopeless.  Even
worse, it seems that the further along you get, the more ``axioms'' you have
to add, as each new theorem seems to involve additional ``metamathematical''
reasoning that hasn't been formalized, and none of it can be derived from the
axioms of logic.  Not only do the proofs keep growing exponentially as you get
further along, but the program to verify them keeps getting bigger and bigger
as you program in more ``metatheorems.''\index{metatheorem}\footnote{A
metatheorem is usually a statement that is too general to be directly provable
in a theory.  For example, ``if $n_1$, $n_2$, and $n_3$ are integers, then
$n_1+n_2+n_3$ is an integer'' is a theorem of number theory.  But ``for any
integer $k > 1$, if $n_1, \ldots, n_k$ are integers, then $n_1+\ldots +n_k$ is
an integer'' is a metatheorem, in other words a family of theorems, one for
every $k$.  The reason it is not a theorem is that the general sum $n_1+\ldots
+n_k$ (as a function of $k$) is not an operation that can be defined directly
in number theory.} The bugs\index{computer program bugs} that have cropped up
so far have already made you start to lose faith in the rigor you seem to have
achieved, and you know it's just going to get worse as your program gets larger.

\subsection{Mathematics and the Non-Specialist}

\begin{quote}
  {\em A real proof is not checkable by a machine, or even by any mathematician
not privy to the gestalt, the mode of thought of the particular field of
mathematics in which the proof is located.}
  \flushright\sc  Davis and Hersh\index{Davis, Phillip J.}
  \footnote{\cite{Davis}, p.~354.}\\
\end{quote}

The bulk of abstract or theoretical mathematics is ordinarily outside
the reach of anyone but a few specialists in each field who have completed
the necessary difficult internship in order to enter its coterie.  The
typical intelligent layperson has no reasonable hope of understanding much of
it, nor even the specialist mathematician of understanding other fields.  It
is like a foreign language that has no dictionary to look up the translation;
the only way you can learn it is by living in the country for a few years.  It
is argued that the effort involved in learning a specialty is a necessary
process for acquiring a deep understanding.  Of course, this is almost certainly
true if one is to make significant contributions to a field; in particular,
``doing'' proofs is probably the most important part of a mathematician's
training.  But is it also necessary to deny outsiders access to it?  Is it
necessary that abstract mathematics be so hard for a layperson to grasp?

A computer normally is of no help whatsoever.  Most published proofs are
actually just series of hints written in an informal style that requires
considerable knowledge of the field to understand.  These are the ``real
proofs'' referred to by Davis and Hersh.\index{informal proof}  There is an
implicit understanding that, in principle, such a proof could be converted to
a complete formal proof\index{formal proof}.  However, it is said that no one
would ever attempt such a conversion, even if they could, because that would
presumably require millions of steps (Section~\ref{dream}).  Unfortunately the
informal style automatically excludes the understanding of the proof
by anyone who hasn't gone through the necessary apprenticeship. The
best that the intelligent layperson can do is to read popular books about deep
and famous results; while this can be helpful, it can also be misleading, and
the lack of detail usually leaves the reader with no ability whatsoever to
explore any aspect of the field being described.

The statements of theorems often use sophisticated notation that makes them
inaccessible to the non-specialist.  For a non-specialist who wants to achieve
a deeper understanding of a proof, the process of tracing definitions and
lemmas back through their hierarchy\index{hierarchy} quickly becomes confusing
and discouraging.  Textbooks are usually written to train mathematicians or to
communicate to people who are already mathematicians, and large gaps in proofs
are often left as exercises to the reader who is left at an impasse if he or
she becomes stuck.

I believe that eventually computers will enable non-specialists and even
intelligent laypersons to follow almost any mathematical proof in any field.
Metamath is an attempt in that direction.  If all of mathematics were as
easily accessible as a computer programming language, I could envision
computer programmers and hobbyists who otherwise lack mathematical
sophistication exploring and being amazed by the world of theorems and proofs
in obscure specialties, perhaps even coming up with results of their own.  A
tremendous advantage would be that anyone could experiment with conjectures in
any field---the computer would offer instant feedback as to whether
an inference step was correct.

Mathematicians sometimes have to put up with the annoyance of
cranks\index{cranks} who lack a fundamental understanding of mathematics but
insist that their ``proofs'' of, say, Fermat's Last Theorem\index{Fermat's
Last Theorem} be taken seriously.  I think part of the problem is that these
people are misled by informal mathematical language, treating it as if they
were reading ordinary expository English and failing to appreciate the
implicit underlying rigor.  Such cranks are rare in the field of computers,
because computer languages are much more explicit, and ultimately the proof is
in whether a computer program works or not.  With easily accessible
computer-based abstract mathematics, a mathematician could say to a crank,
``don't bother me until you've demonstrated your claim on the computer!''

% 22-May-04 nm
% Attempt to move De Millo quote so it doesn't separate from attribution
% CHANGE THIS NUMBER (AND ELIMINATE IF POSSIBLE) WHEN ABOVE TEXT CHANGES
\vspace{-0.5em}

\subsection{An Impossible Dream?}\label{dream}

\begin{quote}
  {\em Even quite basic theorems would demand almost unbelievably vast
  books to display their proofs.}
    \flushright\sc  Robert E. Edwards\footnote{\cite{Edwards}, p.~68.}\\
\end{quote}\index{Edwards, Robert E.}

\begin{quote}
  {\em Oh, of course no one ever really {\em does} it.  It would take
  forever!  You just show that you could do it, that's sufficient.}
    \flushright\sc  ``The Ideal Mathematician''
    \index{Davis, Phillip J.}\footnote{\cite{Davis},
p.~40.}\\
\end{quote}

\begin{quote}
  {\em There is a theorem in the primitive notation of set theory that
  corresponds to the arithmetic theorem `$1000+2000=3000$'.  The formula
  would be forbiddingly long\ldots even if [one] knows the definitions
  and is asked to simplify the long formula according to them, chances are
  he will make errors and arrive at some incorrect result.}
    \flushright\sc  Hao Wang\footnote{\cite{Wang}, p.~140.}\\
\end{quote}\index{Wang, Hao}

% 22-May-04 nm
% Attempt to move De Millo quote so it doesn't separate from attribution
% CHANGE THIS NUMBER (AND ELIMINATE IF POSSIBLE) WHEN ABOVE TEXT CHANGES
\vspace{-0.5em}

\begin{quote}
  {\em The {\em Principia Mathematica} was the crowning achievement of the
  formalists.  It was also the deathblow of the formalist view.\ldots
  {[Rus\-sell]} failed, in three enormous volumes, to get beyond the elementary
  facts of arithmetic.  He showed what can be done in principle and what
  cannot be done in practice.  If the mathematical process were really
  one of strict, logical progression, we would still be counting our
  fingers.\ldots
  One theoretician estimates\ldots that a demonstration of one of
  Ramanujan's conjectures assuming set theory and elementary analysis would
  take about two thousand pages; the length of a deduction from first principles
  is nearly in\-con\-ceiv\-a\-ble\ldots The probabilists argue that\ldots any
  very long proof can at best be viewed as only probably correct\ldots}
  \flushright\sc Richard de Millo et. al.\footnote{\cite{deMillo}, pp.~269,
  271.}\\
\end{quote}\index{de Millo, Richard}

A number of writers have conveyed the impression that the kind of absolute
rigor provided by Metamath\index{Metamath} is an impossible dream, suggesting
that a complete, formal verification\index{formal proof} of a typical theorem
would take millions of steps in untold volumes of books.  Even if it could be
done, the thinking sometimes goes, all meaning would be lost in such a
monstrous, tedious verification.\index{informal proof}\index{proof length}

These writers assume, however, that in order to achieve the kind of complete
formal verification they desire one must break down a proof into individual
primitive steps that make direct reference to the axioms.  This is
not necessary.  There is no reason not to make use of previously proved
theorems rather than proving them over and over.

Just as important, definitions\index{definition} can be introduced along
the way, allowing very complex formulas to be represented with few
symbols.  Not doing this can lead to absurdly long formulas.  For
example, the mere statement of
G\"{o}del's incompleteness theorem\index{G\"{o}del's
incompleteness theorem}, which can be expressed with a small number of
defined symbols, would require about 20,000 primitive symbols to express
it.\index{Boolos, George S.}\footnote{George S.\ Boolos, lecture at
Massachusetts Institute of Technology, spring 1990.} An extreme example
is Bourbaki's\label{bourbaki} language for set theory, which requires
4,523,659,424,929 symbols plus 1,179,618,517,981 disambiguatory links
(lines connecting symbol pairs, usually drawn below or above the
formula) to express the number
``one'' \cite{Mathias}.\index{Mathias, Adrian R. D.}\index{Bourbaki,
Nicolas}
% http://www.dpmms.cam.ac.uk/~ardm/

A hierarchy\index{hierarchy} of theorems and definitions permits an
exponential growth in the formula sizes and primitive proof steps to be
described with only a linear growth in the number of symbols used.  Of course,
this is how ordinary informal mathematics is normally done anyway, but with
Metamath\index{Metamath} it can be done with absolute rigor and precision.

\subsection{Beauty}


\begin{quote}
  {\em No one shall be able to drive us from the paradise that Cantor has
created for us.}
   \flushright\sc  David Hilbert\footnote{As quoted in \cite{Moore}, p.~131.}\\
\end{quote}\index{Hilbert, David}

\needspace{3\baselineskip}
\begin{quote}
  {\em Mathematics possesses not only truth, but some supreme beauty ---a
  beauty cold and austere, like that of a sculpture.}
    \flushright\sc  Bertrand
    Russell\footnote{\cite{Russell}.}\\
\end{quote}\index{Russell, Bertrand}

\begin{quote}
  {\em Euclid alone has looked on Beauty bare.}
  \flushright\sc Edna Millay\footnote{As quoted in \cite{Davis}, p.~150.}\\
\end{quote}\index{Millay, Edna}

For most people, abstract mathematics is distant, strange, and
incomprehensible.  Many popular books have tried to convey some of the sense
of beauty in famous theorems.  But even an intelligent layperson is left with
only a general idea of what a theorem is about and is hardly given the tools
needed to make use of it.  Traditionally, it is only after years of arduous
study that one can grasp the concepts needed for deep understanding.
Metamath\index{Metamath} allows you to approach the proof of the theorem from
a quite different perspective, peeling apart the formulas and definitions
layer by layer until an entirely different kind of understanding is achieved.
Every step of the proof is there, pieced together with absolute precision and
instantly available for inspection through a microscope with a magnification
as powerful as you desire.

A proof in itself can be considered an object of beauty.  Constructing an
elegant proof is an art.  Once a famous theorem has been proved, often
considerable effort is made to find simpler and more easily understood
proofs.  Creating and communicating elegant proofs is a major concern of
mathematicians.  Metamath is one way of providing a common language for
archiving and preserving this information.

The length of a proof can, to a certain extent, be considered an
objective measure of its ``beauty,'' since shorter proofs are usually
considered more elegant.  In the set theory database
\texttt{set.mm}\index{set theory database (\texttt{set.mm})}%
\index{Metamath Proof Explorer}
provided with Metamath, one goal was to make all proofs as short as possible.

\needspace{4\baselineskip}
\subsection{Simplicity}

\begin{quote}
  {\em God made man simple; man's complex problems are of his own
  devising.}
    \flushright\sc Eccles. 7:29\footnote{Jerusalem Bible.}\\
\end{quote}\index{Bible}

\needspace{3\baselineskip}
\begin{quote}
  {\em God made integers, all else is the work of man.}
    \flushright\sc Leopold Kronecker\footnote{{\em Jahresbericht
	der Deutschen Mathematiker-Vereinigung }, vol. 2, p. 19.}\\
\end{quote}\index{Kronecker, Leopold}

\needspace{3\baselineskip}
\begin{quote}
  {\em For what is clear and easily comprehended attracts; the
  complicated repels.}
    \flushright\sc David Hilbert\footnote{As quoted in \cite{deMillo},
p.~273.}\\
\end{quote}\index{Hilbert, David}

The Metamath\index{Metamath} language is simple and Spartan.  Metamath treats
all mathematical expressions as simple sequences of symbols, devoid of meaning.
The higher-level or ``metamathematical'' notions underlying Metamath are about
as simple as they could possibly be.  Each individual step in a proof involves
a single basic concept, the substitution of an expression for a variable, so
that in principle almost anyone, whether mathematician or not, can
completely understand how it was arrived at.

In one of its most basic applications, Metamath\index{Metamath} can be used to
develop the foundations of mathematics\index{foundations of mathematics} from
the very beginning.  This is done in the set theory database that is provided
with the Metamath package and is the subject matter
of Chapter~\ref{fol}. Any language (a metalanguage\index{metalanguage})
used to describe mathematics (an object language\index{object language}) must
have a mathematical content of its own, but it is desirable to keep this
content down to a bare minimum, namely that needed to make use of the
inference rules specified by the axioms.  With any metalanguage there is a
``chicken and egg'' problem somewhat like circular reasoning:  you must assume
the validity of the mathematics of the metalanguage in order to prove the
validity of the mathematics of the object language.  The mathematical content
of Metamath itself is quite limited.  Like the rules of a game of chess, the
essential concepts are simple enough so that virtually anyone should be able to
understand them (although that in itself will not let you play like
a master).  The symbols that Metamath manipulates do not in themselves
have any intrinsic meaning.  Your interpretation of the axioms that you supply
to Metamath is what gives them meaning.  Metamath is an attempt to strip down
mathematical thought to its bare essence and show you exactly how the symbols
are manipulated.

Philosophers and logicians, with various motivations, have often thought it
important to study ``weak'' fragments of logic\index{weak logic}
\cite{Anderson}\index{Anderson, Alan Ross} \cite{MegillBunder}\index{Megill,
Norman}\index{Bunder, Martin}, other unconventional systems of logic (such as
``modal'' logic\index{modal logic} \cite[ch.\ 27]{Boolos}\index{Boolos, George
S.}), and quantum logic\index{quantum logic} in physics
\cite{Pavicic}\index{Pavi{\v{c}}i{\'{c}}, M.}.  Metamath\index{Metamath}
provides a framework in which such systems can be expressed, with an absolute
precision that makes all underlying metamathematical assumptions rigorous and
crystal clear.

Some schools of philosophical thought, for example
intuitionism\index{intuitionism} and constructivism\index{constructivism},
demand that the notions underlying any mathematical system be as simple and
concrete as possible.  Metamath should meet the requirements of these
philosophies.  Metamath must be taught the symbols, axioms\index{axiom}, and
rules\index{rule} for a specific theory, from the skeptical (such as
intuitionism\index{intuitionism}\footnote{Intuitionism does not accept the law
of excluded middle (``either something is true or it is not true'').  See
\cite[p.~xi]{Tymoczko}\index{Tymoczko, Thomas} for discussion and references
on this topic.  Consider the theorem, ``There exist irrational numbers $a$ and
$b$ such that $a^b$ is rational.''  An intuitionist would reject the following
proof:  If $\sqrt{2}^{\sqrt{2}}$ is rational, we are done.  Otherwise, let
$a=\sqrt{2}^{\sqrt{2}}$ and $b=\sqrt{2}$. Then $a^b=2$, which is rational.})
to the bold (such as the axiom of choice in set theory\footnote{The axiom of
choice\index{Axiom of Choice} asserts that given any collection of pairwise
disjoint nonempty sets, there exists a set that has exactly one element in
common with each set of the collection.  It is used to prove many important
theorems in standard mathematics.  Some philosophers object to it because it
asserts the existence of a set without specifying what the set contains
\cite[p.~154]{Enderton}\index{Enderton, Herbert B.}.  In one foundation for
mathematics due to Quine\index{Quine, Willard Van Orman}, that has not been
otherwise shown to be inconsistent, the axiom of choice turns out to be false
\cite[p.~23]{Curry}\index{Curry, Haskell B.}.  The \texttt{show
trace{\char`\_}back} command of the Metamath program allows you to find out
whether the axiom of choice, or any other axiom, was assumed by a
proof.}\index{\texttt{show trace{\char`\_}back} command}).

The simplicity of the Metamath language lets the algorithm (computer program)
that verifies the validity of a Metamath proof be straightforward and
robust.  You can have confidence that the theorems it verifies really can be
derived from your axioms.

\subsection{Rigor}

\begin{quote}
  {\em Rigor became a goal with the Greeks\ldots But the efforts to
  pursue rigor to the utmost have led to an impasse in which there is
  no longer any agreement on what it really means.  Mathematics remains
  alive and vital, but only on a pragmatic basis.}
    \flushright\sc  Morris Kline\footnote{\cite{Kline}, p.~1209.}\\
\end{quote}\index{Kline, Morris}

Kline refers to a much deeper kind of rigor than that which we will discuss in
this section.  G\"{o}del's incompleteness theorem\index{G\"{o}del's
incompleteness theorem} showed that it is impossible to achieve absolute rigor
in standard mathematics because we can never prove that mathematics is
consistent (free from contradictions).\index{consistent theory}  If
mathematics is consistent, we will never know it, but must rely on faith.  If
mathematics is inconsistent, the best we can hope for is that some clever
future mathematician will discover the inconsistency.  In this case, the
axioms would probably be revised slightly to eliminate the inconsistency, as
was done in the case of Russell's paradox,\index{Russell's paradox} but the
bulk of mathematics would probably not be affected by such a discovery.
Russell's paradox, for example, did not affect most of the remarkable results
achieved by 19th-century and earlier mathematicians.  It mainly invalidated
some of Gottlob Frege's\index{Frege, Gottlob} work on the foundations of
mathematics in the late 1800's; in fact Frege's work inspired Russell's
discovery.  Despite the paradox, Frege's work contains important concepts that
have significantly influenced modern logic.  Kline's {\em Mathematics, The
Loss of Certainty} \cite{Klinel}\index{Kline, Morris} has an interesting
discussion of this topic.

What {\em can} be achieved with absolute certainty\index{certainty} is the
knowledge that if we assume the axioms are consistent and true, then the
results derived from them are true.  Part of the beauty of mathematics is that
it is the one area of human endeavor where absolute certainty can be achieved
in this sense.  A mathematical truth will remain such for eternity.  However,
our actual knowledge of whether a particular statement is a mathematical truth
is only as certain as the correctness of the proof that establishes it.  If
the proof of a statement is questionable or vague, we can't have absolute
confidence in the truth that the statement claims.

Let us look at some traditional ways of expressing proofs.

Except in the field of formal logic\index{formal logic}, almost all
traditional proofs in mathematics are really not proofs at all, but rather
proof outlines or hints as to how to go about constructing the proof.  Many
gaps\index{gaps in proofs} are left for the reader to fill in. There are
several reasons for this.  First, it is usually assumed in mathematical
literature that the person reading the proof is a mathematician familiar with
the specialty being described, and that the missing steps are obvious to such
a reader or at least that the reader is capable of filling them in.  This
attitude is fine for professional mathematicians in the specialty, but
unfortunately it often has the drawback of cutting off the rest of the world,
including mathematicians in other specialties, from understanding the proof.
We discussed one possible resolution to this on p.~\pageref{envision}.
Second, it is often assumed that a complete formal proof\index{formal proof}
would require countless millions of symbols (Section~\ref{dream}). This might
be true if the proof were to be expressed directly in terms of the axioms of
logic and set theory,\index{set theory} but it is usually not true if we allow
ourselves a hierarchy\index{hierarchy} of definitions and theorems to build
upon, using a notation that allows us to introduce new symbols, definitions,
and theorems in a precisely specified way.

Even in formal logic,\index{formal logic} formal proofs\index{formal proof}
that are considered complete still contain hidden or implicit information.
For example, a ``proof'' is usually defined as a sequence of
wffs,\index{well-formed formula (wff)}\footnote{A {\em wff} or well-formed
formula is a mathematical expression (string of symbols) constructed according
to some precise rules.  A formal mathematical system\index{formal system}
contains (1) the rules for constructing syntactically correct
wffs,\index{syntax rules} (2) a list of starting wffs called
axioms,\index{axiom} and (3) one or more rules prescribing how to derive new
wffs, called theorems\index{theorem}, from the axioms or previously derived
theorems.  An example of such a system is contained in
Metamath's\index{Metamath} set theory database, which defines a formal
system\index{formal system} from which all of standard mathematics can be
derived.  Section~\ref{startf} steps you through a complete example of a formal
system, and you may want to skim it now if you are unfamiliar with the
concept.} each of which is an axiom or follows from a rule applied to previous
wffs in the sequence.  The implicit part of the proof is the algorithm by
which a sequence of symbols is verified to be a valid wff, given the
definition of a wff.  The algorithm in this case is rather simple, but for a
computer to verify the proof,\index{automated proof verification} it must have
the algorithm built into its verification program.\footnote{It is possible, of
course, to specify wff construction syntax outside of the program itself
with a suitable input language (the Metamath language being an example), but
some proof-verification or theorem-proving programs lack the ability to extend
wff syntax in such a fashion.} If one deals exclusively with axioms and
elementary wffs, it is straightforward to implement such an algorithm.  But as
more and more definitions are added to the theory in order to make the
expression of wffs more compact, the algorithm becomes more and more
complicated.  A computer program that implements the algorithm becomes larger
and harder to understand as each definition is introduced, and thus more prone
to bugs.\index{computer program bugs}  The larger the program, the
more suspicious the mathematician may be about
the validity of its algorithms.  This is especially true because
computer programs are inherently hard to follow to begin with, and few people
enjoy verifying them manually in detail.

Metamath\index{Metamath} takes a different approach.  Metamath's ``knowledge''
is limited to the ability to substitute variables for expressions, subject to
some simple constraints.  Once the basic algorithm of Metamath is assumed to
be debugged, and perhaps independently confirmed, it
can be trusted once and for all.  The information that Metamath needs to
``understand'' mathematics is contained entirely in the body of knowledge
presented to Metamath.  Any errors in reasoning can only be errors in the
axioms or definitions contained in this body of knowledge.  As a
``constructive'' language\index{constructive language} Metamath has no
conditional branches or loops like the ones that make computer programs hard
to decipher; instead, the language can only build new sequences of symbols
from earlier sequences  of symbols.

The simplicity of the rules that underlie Metamath not only makes Metamath
easy to learn but also gives Metamath a great deal of flexibility. For
example, Metamath is not limited to describing standard first-order
logic\index{first-order logic}; higher-order logics\index{higher-order logic}
and fragments of logic\index{weak logic} can be described just as easily.
Metamath gives you the freedom to define whatever wff notation you prefer; it
has no built-in conception of the syntax of a wff.\index{well-formed formula
(wff)}  With suitable axioms and definitions, Metamath can even describe and
prove things about itself.\index{Metamath!self-description}  (John
Harrison\index{Harrison, John} discusses the ``reflection''
principle\index{reflection principle} involved in self-descriptive systems in
\cite{Harrison}.)

The flexibility of Metamath requires that its proofs specify a lot of detail,
much more than in an ordinary ``formal'' proof.\index{formal proof}  For
example, in an ordinary formal proof, a single step consists of displaying the
wff that constitutes that step.  In order for a computer program to verify
that the step is acceptable, it first must verify that the symbol sequence
being displayed is an acceptable wff.\index{automated proof verification} Most
proof verifiers have at least basic wff syntax built into their programs.
Metamath has no hard-wired knowledge of what constitutes a wff built into it;
instead every wff must be explicitly constructed based on rules defining wffs
that are present in a database.  Thus a single step in an ordinary formal
proof may be correspond to many steps in a Metamath proof. Despite the larger
number of steps, though, this does not mean that a Metamath proof must be
significantly larger than an ordinary formal proof. The reason is that since
we have constructed the wff from scratch, we know what the wff is, so there is
no reason to display it.  We only need to refer to a sequence of statements
that construct it.  In a sense, the display of the wff in an ordinary formal
proof is an implicit proof of its own validity as a wff; Metamath just makes
the proof explicit. (Section~\ref{proof} describes Metamath's proof notation.)

\section{Computers and Mathematicians}

\begin{quote}
  {\em The computer is important, but not to mathematics.}
    \flushright\sc  Paul Halmos\footnote{As quoted in \cite{Albers}, p.~121.}\\
\end{quote}\index{Halmos, Paul}

Pure mathematicians have traditionally been indifferent to computers, even to
the point of disdain.\index{computers and pure mathematics}  Computer science
itself is sometimes considered to fall in the mundane realm of ``applied''
mathematics, perhaps essential for the real world but intellectually unexciting
to those who seek the deepest truths in mathematics.  Perhaps a reason for this
attitude towards computers is that there is little or no computer software that
meets their needs, and there may be a general feeling that such software could
not even exist.  On the one hand, there are the practical computer algebra
systems, which can perform amazing symbolic manipulations in algebra and
calculus,\index{computer algebra system} yet can't prove the simplest
existence theorem, if the idea of a proof is present at all.  On the other
hand, there are specialized automated theorem provers that technically speaking
may generate correct proofs.\index{automated theorem proving}  But sometimes
their specialized input notation may be cryptic and their output perceived to
be long, inelegant, incomprehensible proofs.    The output
may be viewed with suspicion, since the program that generates it tends to be
very large, and its size increases the potential for bugs\index{computer
program bugs}.  Such a proof may be considered trustworthy only if
independently verified and ``understood'' by a human, but no one wants to
waste their time on such a boring, unrewarding chore.



\needspace{4\baselineskip}
\subsection{Trusting the Computer}

\begin{quote}
  {\em \ldots I continue to find the quasi-empirical interpretation of
  computer proofs to be the more plausible.\ldots Since not
  everything that claims to be a computer proof can be
  accepted as valid, what are the mathematical criteria for acceptable
  computer proofs?}
    \flushright\sc  Thomas Tymoczko\footnote{\cite{Tymoczko}, p.~245.}\\
\end{quote}\index{Tymoczko, Thomas}

In some cases, computers have been essential tools for proving famous
theorems.  But if a proof is so long and obscure that it can be verified in a
practical way only with a computer, it is vaguely felt to be suspicious.  For
example, proving the famous four-color theorem\index{four-color
theorem}\index{proof length} (``a map needs no more than four colors to
prevent any two adjacent countries from having the same color'') can presently
only be done with the aid of a very complex computer program which originally
required 1200 hours of computer time. There has been considerable debate about
whether such a proof can be trusted and whether such a proof is ``real''
mathematics \cite{Swart}\index{Swart, E. R.}.\index{trusting computers}

However, under normal circumstances even a skeptical mathematician would have a
great deal of confidence in the result of multiplying two numbers on a pocket
calculator, even though the precise details of what goes on are hidden from its
user.  Even the verification on a supercomputer that a huge number is prime is
trusted, especially if there is independent verification; no one bothers to
debate the philosophical significance of its ``proof,'' even though the actual
proof would be so large that it would be completely impractical to ever write
it down on paper.  It seems that if the algorithm used by the computer is
simple enough to be readily understood, then the computer can be trusted.

Metamath\index{Metamath} adopts this philosophy.  The simplicity of its
language makes it easy to learn, and because of its simplicity one can have
essentially absolute confidence that a proof is correct. All axioms, rules, and
definitions are available for inspection at any time because they are defined
by the user; there are no hidden or built-in rules that may be prone to subtle
bugs\index{computer program bugs}.  The basic algorithm at the heart of
Metamath is simple and fixed, and it can be assumed to be bug-free and robust
with a degree of confidence approaching certainty.
Independently written implementations of the Metamath verifier
can reduce any residual doubt on the part of a skeptic even further;
there are now over a dozen such implementations, written by many people.

\subsection{Trusting the Mathematician}\label{trust}

\begin{quote}
  {\em There is no Algebraist nor Mathematician so expert in his science, as
  to place entire confidence in any truth immediately upon his discovery of it,
  or regard it as any thing, but a mere probability.  Every time he runs over
  his proofs, his confidence encreases; but still more by the approbation of
  his friends; and is rais'd to its utmost perfection by the universal assent
  and applauses of the learned world.}
  \flushright\sc David Hume\footnote{{\em A Treatise of Human Nature}, as
  quoted in \cite{deMillo}, p.~267.}\\
\end{quote}\index{Hume, David}

\begin{quote}
  {\em Stanislaw Ulam estimates that mathematicians publish 200,000 theorems
  every year.  A number of these are subsequently contradicted or otherwise
  disallowed, others are thrown into doubt, and most are ignored.}
  \flushright\sc Richard de Millo et. al.\footnote{\cite{deMillo}, p.~269.}\\
\end{quote}\index{Ulam, Stanislaw}

Whether or not the computer can be trusted, humans  of course will occasionally
err. Only the most memorable proofs get independently verified, and of these
only a handful of truly great ones achieve the status of being ``known''
mathematical truths that are used without giving a second thought to their
correctness.

There are many famous examples of incorrect theorems and proofs in
mathematical literature.\index{errors in proofs}

\begin{itemize}
\item There have been thousands of purported proofs of Fermat's Last
Theorem\index{Fermat's Last Theorem} (``no integer solutions exist to $x^n +
y^n = z^n$ for $n > 2$''), by amateurs, cranks, and well-regarded
mathematicians \cite[p.~5]{Stark}\index{Stark, Harold M}.  Fermat wrote a note
in his copy of Bachet's {\em Diophantus} that he found ``a truly marvelous
proof of this theorem but this margin is too narrow to contain it''
\cite[p.~507]{Kramer}.  A recent, much publicized proof by Yoichi
Miyaoka\index{Miyaoka, Yoichi} was shown to be incorrect ({\em Science News},
April 9, 1988, p.~230).  The theorem was finally proved by Andrew
Wiles\index{Wiles, Andrew} ({\em Science News}, July 3, 1993, p.~5), but it
initially had some gaps and took over a year after its announcement to be
checked thoroughly by experts.  On Oct. 25, 1994, Wiles announced that the last
gap found in his proof had been filled in.
  \item In 1882, M. Pasch discovered that an axiom was omitted from Euclid's
formulation of geometry\index{Euclidean geometry}; without it, the proofs of
important theorems of Euclid are not valid.  Pasch's axiom\index{Pasch's
axiom} states that a line that intersects one side of a triangle must also
intersect another side, provided that it does not touch any of the triangle's
vertices.  The omission of Pasch's axiom went unnoticed for 2000
years \cite[p.~160]{Davis}, in spite of (one presumes) the thousands of
students, instructors, and mathematicians who studied Euclid.
  \item The first published proof of the famous Schr\"{o}der--Bernstein
theorem\index{Schr\"{o}der--Bernstein theorem} in set theory was incorrect
\cite[p.~148]{Enderton}\index{Enderton, Herbert B.}.  This theorem states
that if there exists a 1-to-1 function\footnote{A {\em set}\index{set} is any
collection of objects. A {\em function}\index{function} or {\em
mapping}\index{mapping} is a rule that assigns to each element of one set
(called the function's {\em domain}\index{domain}) an element from another
set.} from set $A$ into set $B$ and vice-versa, then sets $A$ and $B$ have
a 1-to-1 correspondence.  Although it sounds simple and obvious,
the standard proof is quite long and complex.
  \item In the early 1900's, Hilbert\index{Hilbert, David} published a
purported proof of the continuum hypothesis\index{continuum hypothesis}, which
was eventually established as unprovable by Cohen\index{Cohen, Paul} in 1963
\cite[p.~166]{Enderton}.  The continuum hypothesis states that no
infinity\index{infinity} (``transfinite cardinal number'')\index{cardinal,
transfinite} exists whose size (or ``cardinality''\index{cardinality}) is
between the size of the set of integers and the size of the set of real
numbers.  This hypothesis originated with German mathematician Georg
Cantor\index{Cantor, Georg} in the late 1800's, and his inability to prove it
is said to have contributed to mental illness that afflicted him in his later
years.
  \item An incorrect proof of the four-color theorem\index{four-color theorem}
was published by Kempe\index{Kempe, A. B.} in 1879
\cite[p.~582]{Courant}\index{Courant, Richard}; it stood for 11 years before
its flaw was discovered.  This theorem states that any map can be colored
using only four colors, so that no two adjacent countries have the same
color.  In 1976 the theorem was finally proved by the famous computer-assisted
proof of Haken, Appel, and Koch \cite{Swart}\index{Appel, K.}\index{Haken,
W.}\index{Koch, K.}.  Or at least it seems that way.  Mathematician
H.~S.~M.~Coxeter\index{Coxeter, H. S. M.} has doubts \cite[p.~58]{Davis}:  ``I
have a feeling that that is an untidy kind of use of the computers, and the more
you correspond with Haken and Appel, the more shaky you seem to be.''
  \item Many false ``proofs'' of the Poincar\'{e}
conjecture\index{Poincar\'{e} conjecture} have been proposed over the years.
This conjecture states that any object that mathematically behaves like a
three-dimensional sphere is a three-dimensional sphere topologically,
regardless of how it is distorted.  In March 1986, mathematicians Colin
Rourke\index{Rourke, Colin} and Eduardo R\^{e}go\index{R\^{e}go, Eduardo}
caused  a stir in the mathematical community by announcing that they had found
a proof; in November of that year the proof was found to be false \cite[p.
218]{PetersonI}.  It was finally proved in 2003 by Grigory Perelman
\label{poincare}\index{Szpiro, George}\index{Perelman, Grigory}\cite{Szpiro}.
 \end{itemize}

Many counterexamples to ``theorems'' in recent mathematical
literature related to Clifford algebras\index{Clifford algebras}
 have been found by Pertti
Lounesto (who passed away in 2002).\index{Lounesto, Pertti}
See the web page \url{http://mathforum.org/library/view/4933.html}.
% http://users.tkk.fi/~ppuska/mirror/Lounesto/counterexamples.htm

One of the purposes of Metamath\index{Metamath} is to allow proofs to be
expressed with absolute precision.  Developing a proof in the Metamath
language can be challenging, because Metamath will not permit even the
tiniest mistake.\index{errors in proofs}  But once the proof is created, its
correctness can be trusted immediately, without having to depend on the
process of peer review for confirmation.

\section{The Use of Computers in Mathematics}

\subsection{Computer Algebra Systems}

For the most part, you will find that Metamath\index{Metamath} is not a
practical tool for manipulating numbers.  (Even proving that $2 + 2 = 4$, if
you start with set theory, can be quite complex!)  Several commercial
mathematics packages are quite good at arithmetic, algebra, and calculus, and
as practical tools they are invaluable.\index{computer algebra system} But
they have no notion of proof, and cannot understand statements starting with
``there exists such and such...''.

Software packages such as Mathematica \cite{Wolfram}\index{Mathematica} do not
concern themselves with proofs but instead work directly with known results.
These packages primarily emphasize heuristic rules such as the substitution of
equals for equals to achieve simpler expressions or expressions in a different
form.  Starting with a rich collection of built-in rules and algorithms, users
can add to the collection by means of a powerful programming language.
However, results such as, say, the existence of a certain abstract object
without displaying the actual object cannot be expressed (directly) in their
languages.  The idea of a proof from a small set of axioms is absent.  Instead
this software simply assumes that each fact or rule you add to the built-in
collection of algorithms is valid.  One way to view the software is as a large
collection of axioms from which the software, with certain goals, attempts to
derive new theorems, for example equating a complex expression with a simpler
equivalent. But the terms ``theorem''\index{theorem} and
``proof,''\index{proof} for example, are not even mentioned in the index of
the user's manual for Mathematica.\index{Mathematica and proofs}  What is also
unsatisfactory from a philosophical point of view is that there is no way to
ensure the validity of the results other than by trusting the writer of each
application module or tediously checking each module by hand, similar to
checking a computer program for bugs.\index{computer program
bugs}\footnote{Two examples illustrate why the knowledge database of computer
algebra systems should sometimes be regarded with a certain caution.  If you
ask Mathematica (version 3.0) to \texttt{Solve[x\^{ }n + y\^{ }n == z\^{ }n , n]}
it will respond with \texttt{\{\{n-\char`\>-2\}, \{n-\char`\>-1\},
\{n-\char`\>1\}, \{n-\char`\>2\}\}}. In other words, Mathematica seems to
``know'' that Fermat's Last Theorem\index{Fermat's Last Theorem} is true!  (At
the time this version of Mathematica was released this fact was unknown.)  If
you ask Maple\index{Maple} to \texttt{solve(x\^{ }2 = 2\^{ }x)} then
\texttt{simplify(\{"\})}, it returns the solution set \texttt{\{2, 4\}}, apparently
unaware that $-0.7666647$\ldots is also a solution.} While of course extremely
valuable in applied mathematics, computer algebra systems tend to be of little
interest to the theoretical mathematician except as aids for exploring certain
specific problems.

Because of possible bugs, trusting the output of a computer algebra system for
use as theorems in a proof-verifier would defeat the latter's goal of rigor.
On the other hand, a fact such that a certain relatively large number is
prime, while easy for a computer algebra system to derive, might have a long,
tedious proof that could overwhelm a proof-verifier. One approach for linking
computer algebra systems to a proof-verifier while retaining the advantages of
both is to add a hypothesis to each such theorem indicating its source.  For
example, a constant {\sc maple} could indicate the theorem came from the Maple
package, and would mean ``assuming Maple is consistent, then\ldots''  This and
many other topics concerning the formalization of mathematics are discussed in
John Harrison's\index{Harrison, John} very interesting
PhD thesis~\cite{Harrison-thesis}.

\subsection{Automated Theorem Provers}\label{theoremprovers}

A mathematical theory is ``decidable''\index{decidable theory} if a mechanical
method or algorithm exists that is guaranteed to determine whether or not a
particular formula is a theorem.  Among the few theories that are decidable is
elementary geometry,\index{Euclidean geometry} as was shown by a classic
result of logician Alfred Tarski\index{Tarski, Alfred} in 1948
\cite{Tarski}.\footnote{Tarski's result actually applies to a subset of the
geometry discussed in elementary textbooks.  This subset includes most of what
would be considered elementary geometry but it is not powerful enough to
express, among other things, the notions of the circumference and area of a
circle.  Extending the theory in a way that includes notions such as these
makes the theory undecidable, as was also shown by Tarski.  Tarski's algorithm
is far too inefficient to implement practically on a computer.  A practical
algorithm for proving a smaller subset of geometry theorems (those not
involving concepts of ``order'' or ``continuity'') was discovered by Chinese
mathematician Wu Wen-ts\"{u}n in 1977 \cite{Chou}\index{Chou,
Shang-Ching}.}\index{Wen-ts{\"{u}}n, Wu}  But most theories, including
elementary arithmetic, are undecidable.  This fact contributes to keeping
mathematics alive and well, since many mathematicians believe
that they will never be
replaced by computers (if they believe Roger Penrose's argument that a
computer can never replace the brain \cite{Penrose}\index{Penrose, Roger}).
In fact,  elementary geometry is often considered a ``dead'' field
for the simple reason that it is decidable.

On the other hand, the undecidability of a theory does not mean that one cannot
use a computer to search for proofs, providing one is willing to give up if a
proof is not found after a reasonable amount of time.  The field of automated
theorem proving\index{automated theorem proving} specializes in pursuing such
computer searches.  Among the more successful results to date are those based
on an algorithm known as Robinson's resolution principle
\cite{Robinson}\index{Robinson's resolution principle}.

Automated theorem provers can be excellent tools for those willing to learn
how to use them.  But they are not widely used in mainstream pure
mathematics, even though they could probably be useful in many areas.  There
are several reasons for this.  Probably most important, the main goal in pure
mathematics is to arrive at results that are considered to be deep or
important; proving them is essential but secondary.  Usually, an automated
theorem prover cannot assist in this main goal, and by the time the main goal
is achieved, the mathematician may have already figured out the proof as a
by-product.  There is also a notational problem.  Mathematicians are used to
using very compact syntax where one or two symbols (heavily dependent on
context) can represent very complex concepts; this is part of the
hierarchy\index{hierarchy} they have built up to tackle difficult problems.  A
theorem prover on the other hand might require that a theorem be expressed in
``first-order logic,''\index{first-order logic} which is the logic on which
most of mathematics is ultimately based but which is not ordinarily used
directly because expressions can become very long.  Some automated theorem
provers are experimental programs, limited in their use to very specialized
areas, and the goal of many is simply research into the nature of automated
theorem proving itself.  Finally, much research remains to be done to enable
them to prove very deep theorems.  One significant result was a
computer proof by Larry Wos\index{Wos, Larry} and colleagues that every Robbins
algebra\index{Robbins algebra} is a Boolean  algebra\index{Boolean algebra}
({\em New York Times}, Dec. 10, 1996).\footnote{In 1933, E.~V.\
Huntington\index{Huntington, E. V.}
presented the following axiom system for
Boolean algebra with a unary operation $n$ and a binary operation $+$:
\begin{center}
    $x + y = y + x$ \\
    $(x + y) + z = x + (y + z)$ \\
    $n(n(x) + y) + n(n(x) + n(y)) = x$
\end{center}
Herbert Robbins\index{Robbins, Herbert}, a student of Huntington, conjectured
that the last equation can be replaced with a simpler one:
\begin{center}
    $n(n(x + y) + n(x + n(y))) = x$
\end{center}
Robbins and Huntington could not find a proof.  The problem was
later studied unsuccessfully by Tarski\index{Tarski, Alfred} and his
students, and it remained an unsolved problem until a
computer found the proof in 1996.  For more information on
the Robbins algebra problem see \cite{Wos}.}

How does Metamath\index{Metamath} relate to automated theorem provers?  A
theorem prover is primarily concerned with one theorem at a time (perhaps
tapping into a small database of known theorems) whereas Metamath is more like
a theorem archiving system, storing both the theorem and its proof in a
database for access and verification.  Metamath is one answer to ``what do you
do with the output of a theorem prover?''  and could be viewed as the
next step in the process.  Automated theorem provers could be useful tools for
helping develop its database.
Note that very long, automatically
generated proofs can make your database fat and ugly and cause Metamath's proof
verification to take a long time to run.  Unless you have a particularly good
program that generates very concise proofs, it might be best to consider the
use of automatically generated proofs as a quick-and-dirty approach, to be
manually rewritten at some later date.

The program {\sc otter}\index{otter@{\sc otter}}\footnote{\url{http://www.cs.unm.edu/\~mccune/otter/}.}, later succeeded by
prover9\index{prover9}\footnote{\url{https://www.cs.unm.edu/~mccune/mace4/}.},
have been historically influential.
The E prover\index{E prover}\footnote{\url{https://github.com/eprover/eprover}.}
is a maintained automated theorem prover
for full first-order logic with equality.
There are many other automated theorem provers as well.

If you want to combine automated theorem provers with Metamath
consider investigating
the book {\em Automated Reasoning:  Introduction and Applications}
\cite{Wos}\index{Wos, Larry}.  This book discusses
how to use {\sc otter} in a way that can
not only able to generate
relatively efficient proofs, it can even be instructed to search for
shorter proofs.  The effective use of {\sc otter} (and similar tools)
does require a certain
amount of experience, skill, and patience.  The axiom system used in the
\texttt{set.mm}\index{set theory database (\texttt{set.mm})} set theory
database can be expressed to {\sc otter} using a method described in
\cite{Megill}.\index{Megill, Norman}\footnote{To use those axioms with
{\sc otter}, they must be restated in a way that eliminates the need for
``dummy variables.''\index{dummy variable!eliminating} See the Comment
on p.~\pageref{nodd}.} When successful, this method tends to generate
short and clever proofs, but my experiments with it indicate that the
method will find proofs within a reasonable time only for relatively
easy theorems.  It is still fun to experiment with.

Reference \cite{Bledsoe}\index{Bledsoe, W. W.} surveys a number of approaches
people have explored in the field of automated theorem proving\index{automated
theorem proving}.

\subsection{Interactive Theorem Provers}\label{interactivetheoremprovers}

Finding proofs completely automatically is difficult, so there
are some interactive theorem provers that allow a human to guide the
computer to find a proof.
Examples include
HOL Light\index{HOL light}%
\footnote{\url{https://www.cl.cam.ac.uk/~jrh13/hol-light/}.},
Isabelle\index{Isabelle}%
\footnote{\url{http://www.cl.cam.ac.uk/Research/HVG/Isabelle}.},
{\sc hol}\index{hol@{\sc hol}}%
\footnote{\url{https://hol-theorem-prover.org/}.},
and
Coq\index{Coq}\footnote{\url{https://coq.inria.fr/}.}.

A major difference between most of these tools and Metamath is that the
``proofs'' are actually programs that guide the program to find a proof,
and not the proof itself.
For example, an Isabelle/HOL proof might apply a step
\texttt{apply (blast dest: rearrange reduction)}. The \texttt{blast}
instruction applies
an automatic tableux prover and returns if it found a sequence of proof
steps that work... but the sequence is not considered part of the proof.

A good overview of
higher-level proof verification languages (such as {\sc lcf}\index{lcf@{\sc
lcf}} and {\sc hol}\index{hol@{\sc hol}})
is given in \cite{Harrison}.  All of these languages are fundamentally
different from Metamath in that much of the mathematical foundational
knowledge is embedded in the underlying proof-verification program, rather
than placed directly in the database that is being verified.
These can have a steep learning curve for those without a mathematical
background.  For example, one usually must have a fair understanding of
mathematical logic in order to follow their proofs.

\subsection{Proof Verifiers}\label{proofverifiers}

A proof verifier is a program that doesn't generate proofs but instead
verifies proofs that you give it.  Many proof verifiers have limited built-in
automated proof capabilities, such as figuring out simple logical inferences
(while still being guided by a person who provides the overall proof).  Metamath
has no built-in automated proof capability other than the limited
capability of its Proof Assistant.

Proof-verification languages are not used as frequently as they might be.
Pure mathematicians are more concerned with producing new results, and such
detail and rigor would interfere with that goal.  The use of computers in pure
mathematics is primarily focused on automated theorem provers (not verifiers),
again with the ultimate goal of aiding the creation of new mathematics.
Automated theorem provers are usually concerned with attacking one theorem at
time rather than making a large, organized database easily available to the
user.  Metamath is one way to help close this gap.

By itself Metamath is a mostly a proof verifier.
This does not mean that other approaches can't be used; the difference
is that in Metamath, the results of various provers must be recorded
step-by-step so that they can be verified.

Another proof-verification language is Mizar,\index{Mizar} which can display
its proofs in the informal language that mathematicians are accustomed to.
Information on the Mizar language is available at \url{http://mizar.org}.

For the working mathematician, Mizar is an excellent tool for rigorously
documenting proofs. Mizar typesets its proofs in the informal English used by
mathematicians (and, while fine for them, are just as inscrutable by
laypersons!). A price paid for Mizar is a relatively steep learning curve of a
couple of weeks.  Several mathematicians are actively formalizing different
areas of mathematics using Mizar and publishing the proofs in a dedicated
journal. Unfortunately the task of formalizing mathematics is still looked
down upon to a certain extent since it doesn't involve the creation of ``new''
mathematics.

The closest system to Metamath is
the {\em Ghilbert}\index{Ghilbert} proof language (\url{http://ghilbert.org})
system developed by
Raph Levien\index{Levien, Raph}.
Ghilbert is a formal proof checker heavily inspired by Metamath.
Ghilbert statements are s-expressions (a la Lisp), which is easy
for computers to parse but many people find them hard to read.
There are a number of differences in their specific constructs, but
there is at least one tool to translate some Metamath materials into Ghilbert.
As of 2019 the Ghilbert community is smaller and less active than the
Metamath community.
That said, the Metamath and Ghilbert communities overlap, and fruitful
conversations between them have occurred many times over the years.

\subsection{Creating a Database of Formalized Mathematics}\label{mathdatabase}

Besides Metamath, there are several other ongoing projects with the goal of
formalizing mathematics into computer-verifiable databases.
Understanding some history will help.

The {\sc qed}\index{qed project@{\sc qed} project}%
\footnote{\url{http://www-unix.mcs.anl.gov/qed}.}
project arose in 1993 and its goals were outlined in the
{\sc qed} manifesto.
The {\sc qed} manifesto was
a proposal for a computer-based database of all mathematical knowledge,
strictly formalized and with all proofs having been checked automatically.
The project had a conference in 1994 and another in 1995;
there was also a ``twenty years of the {\sc qed} manifesto'' workshop
in 2014.
Its ideals are regularly reraised.

In a 2007 paper, Freek Wiedijk identified two reasons
for the failure of the {\sc qed} project as originally envisioned:%
\cite{Wiedijk-revisited}\index{Wiedijk, Freek}

\begin{itemize}
\item Very few people are working on formalization of mathematics. There is no compelling application for fully mechanized mathematics.
\item Formalized mathematics does not yet resemble traditional mathematics. This is partly due to the complexity of mathematical notation, and partly to the limitations of existing theorem provers and proof assistants.
\end{itemize}

But this did not end the dream of
formalizing mathematics into computer-verifiable databases.
The problems that led to the {\sc qed} manifesto are still with us,
even though the challenges were harder than originally considered.
What has happened instead is that various independent projects have
worked towards formalizing mathematics into computer-verifiable databases,
each simultaneously competing and cooperating with each other.

A concrete way to see this is
Freek Wiedijk's ``Formalizing 100 Theorems'' list%
\footnote{\url{http://www.cs.ru.nl/\%7Efreek/100/}.}
which shows the progress different systems have made on a challenge list
of 100 mathematical theorems.%
\footnote{ This is not the only list of ``interesting'' theorems.
Another interesting list was posted by Oliver Knill's list
\cite{Knill}\index{Knill, Oliver}.}
The top systems as of February 2019
(in order of the number of challenges completed) are
HOL Light, Isabelle, Metamath, Coq, and Mizar.

The Metamath 100%
\footnote{\url{http://us.metamath.org/mm\_100.html}}
page (maintained by David A. Wheeler\index{Wheeler, David A.})
shows the progress of Metamath (specifically its \texttt{set.mm} database)
against this challenge list maintained by Freek Wiedijk.
The Metamath \texttt{set.mm} database
has made a lot of progress over the years,
in part because working to prove those challenge theorems required
defining various terms and proving their properties as a prerequisite.
Here are just a few of the many statements that have been
formally proven with Metamath:

% The entries of this cause the narrow display to break poorly,
% since the short amount of text means LaTeX doesn't get a lot to work with
% and the itemize format gives it even *less* margin than usual.
% No one will mind if we make just this list flushleft, since this list
% will be internally consistent.
\begin{flushleft}
\begin{itemize}
\item 1. The Irrationality of the Square Root of 2
  (\texttt{sqr2irr}, by Norman Megill, 2001-08-20)
\item 2. The Fundamental Theorem of Algebra
  (\texttt{fta}, by Mario Carneiro, 2014-09-15)
\item 22. The Non-Denumerability of the Continuum
  (\texttt{ruc}, by Norman Megill, 2004-08-13)
\item 54. The Konigsberg Bridge Problem
  (\texttt{konigsberg}, by Mario Carneiro, 2015-04-16)
\item 83. The Friendship Theorem
  (\texttt{friendship}, by Alexander W. van der Vekens, 2018-10-09)
\end{itemize}
\end{flushleft}

We thank all of those who have developed at least one of the Metamath 100
proofs, and we particularly thank
Mario Carneiro\index{Carneiro, Mario}
who has contributed the most Metamath 100 proofs as of 2019.
The Metamath 100 page shows the list of all people who have contributed a
proof, and links to graphs and charts showing progress over time.
We encourage others to work on proving theorems not yet proven in Metamath,
since doing so improves the work as a whole.

Each of the math formalization systems (including Metamath)
has different strengths and weaknesses, depending on what you value.
Key aspects that differentiate Metamath from the other top systems are:

\begin{itemize}
\item Metamath is not tied to any particular set of axioms.
\item Metamath can show every step of every proof, no exceptions.
  Most other provers only assert that a proof can be found, and do not
  show every step. This also makes verification fast, because
  the system does not need to rediscover proof details.
\item The Metamath verifier has been re-implemented in many different
  programming languages, so verification can be done by multiple
  implementations.  In particular, the
  \texttt{set.mm}\index{set theory database (\texttt{set.mm})}%
  \index{Metamath Proof Explorer} database is verified by
  four different verifiers
  written in four different languages by four different authors.
  This greatly reduces the risk of accepting an invalid
  proof due to an error in the verifier.
\item Proofs stay proven.  In some systems, changes to the system's
  syntax or how a tactic works causes proofs to fail in later versions,
  causing older work to become essentially lost.
  Metamath's language is
  extremely small and fixed, so once a proof is added to a database,
  the database can be rechecked with later versions of the Metamath program
  and with other verifiers of Metamath databases.
  If an axiom or key definition needs to be changed, it is easy to
  manipulate the database as a whole to handle the change
  without touching the underlying verifier.
  Since re-verification of an entire database takes seconds, there
  is never a reason to delay complete verification.
  This aspect is especially compelling if your
  goal is to have a long-term database of proofs.
\item Licensing is generous.  The main Metamath databases are released to
  the public domain, and the main Metamath program is open source software
  under a standard, widely-used license.
\item Substitutions are easy to understand, even by those who are not
  professional mathematicians.
\end{itemize}

Of course, other systems may have advantages over Metamath
that are more compelling, depending on what you value.
In any case, we hope this helps you understand Metamath
within a wider context.

\subsection{In Summary}\label{computers-summary}

To summarize our discussions of computers and mathematics, computer algebra
systems can be viewed as theorem generators focusing on a narrow realm of
mathematics (numbers and their properties), automated theorem provers as proof
generators for specific theorems in a much broader realm covered by a built-in
formal system such as first-order logic, interactive theorem
provers require human guidance, proof verifiers verify proofs but
historically they have been
restricted to first-order logic.
Metamath, in contrast,
is a proof verifier and documenter whose realm is essentially unlimited.

\section{Mathematics and Metamath}

\subsection{Standard Mathematics}

There are a number of ways that Metamath\index{Metamath} can be used with
standard mathematics.  The most satisfying way philosophically is to start at
the very beginning, and develop the desired mathematics from the axioms of
logic and set theory.\index{set theory}  This is the approach taken in the
\texttt{set.mm}\index{set theory database (\texttt{set.mm})}%
\index{Metamath Proof Explorer}
database (also known as the Metamath Proof Explorer).
Among other things, this database builds up to the
axioms of real and complex numbers\index{analysis}\index{real and complex
numbers} (see Section~\ref{real}), and a standard development of analysis, for
example, could start at that point, using it as a basis.   Besides this
philosophical advantage, there are practical advantages to having all of the
tools of set theory available in the supporting infrastructure.

On the other hand, you may wish to start with the standard axioms of a
mathematical theory without going through the set theoretical proofs of those
axioms.  You will need mathematical logic to make inferences, but if you wish
you can simply introduce theorems\index{theorem} of logic as
``axioms''\index{axiom} wherever you need them, with the implicit assumption
that in principle they can be proved, if they are obvious to you.  If you
choose this approach, you will probably want to review the notation used in
\texttt{set.mm}\index{set theory database (\texttt{set.mm})} so that your own
notation will be consistent with it.

\subsection{Other Formal Systems}
\index{formal system}

Unlike some programs, Metamath\index{Metamath} is not limited to any specific
area of mathematics, nor committed to any particular mathematical philosophy
such as classical logic versus intuitionism, nor limited, say, to expressions
in first-order logic.  Although the database \texttt{set.mm}
describes standard logic and set theory, Meta\-math
is actually a general-purpose language for describing a wide variety of formal
systems.\index{formal system}  Non-standard systems such as modal
logic,\index{modal logic} intuitionist logic\index{intuitionism}, higher-order
logic\index{higher-order logic}, quantum logic\index{quantum logic}, and
category theory\index{category theory} can all be described with the Metamath
language.  You define the symbols you prefer and tell Metamath the axioms and
rules you want to start from, and Metamath will verify any inferences you make
from those axioms and rules.  A simple example of a non-standard formal system
is Hofstadter's\index{Hofstadter, Douglas R.} MIU system,\index{MIU-system}
whose Metamath description is presented in Appendix~\ref{MIU}.

This is not hypothetical.
The largest Metamath database is
\texttt{set.mm}\index{set theory database (\texttt{set.mm}}%
\index{Metamath Proof Explorer}), aka the Metamath Proof Explorer,
which uses the most common axioms for mathematical foundations
(specifically classical logic combined with Zermelo--Fraenkel
set theory\index{Zermelo--Fraenkel set theory} with the Axiom of Choice).
But other Metamath databases are available:

\begin{itemize}
\item The database
  \texttt{iset.mm}\index{intuitionistic logic database (\texttt{iset.mm})},
  aka the
  Intuitionistic Logic Explorer\index{Intuitionistic Logic Explorer},
  uses intuitionistic logic (a constructivist point of view)
  instead of classical logic.
\item The database
  \texttt{nf.mm}\index{New Foundations database (\texttt{nf.mm})},
  aka the
  New Foundations Explorer\index{New Foundations Explorer},
  constructs mathematics from scratch,
  starting from Quine's New Foundations (NF) set theory axioms.
\item The database
  \texttt{hol.mm}\index{Higher-order Logic database (\texttt{hol.mm})},
  aka the
  Higher-Order Logic (HOL) Explorer\index{Higher-Order Logic (HOL) Explorer},
  starts with HOL (also called simple type theory) and derives
  equivalents to ZFC axioms, connecting the two approaches.
\end{itemize}

Since the days of David Hilbert,\index{Hilbert, David} mathematicians have
been concerned with the fact that the metalanguage\index{metalanguage} used to
describe mathematics may be stronger than the mathematics being described.
Metamath\index{Metamath}'s underlying finitary\index{finitary proof},
constructive nature provides a good philosophical basis for studying even the
weakest logics.\index{weak logic}

The usual treatment of many non-standard formal systems\index{formal
system} uses model theory\index{model theory} or proof theory\index{proof
theory} to describe these systems; these theories, in turn, are based on
standard set theory.  In other words, a non-standard formal system is defined
as a set with certain properties, and standard set theory is used to derive
additional properties of this set.  The standard set theory database provided
with Metamath can be used for this purpose, and when used this way
the development of a special
axiom system for the non-standard formal system becomes unnecessary.  The
model- or proof-theoretic approach often allows you to prove much deeper
results with less effort.

Metamath supports both approaches.  You can define the non-standard
formal system directly, or define the non-standard formal system as
a set with certain properties, whichever you find most helpful.

%\section{Additional Remarks}

\subsection{Metamath and Its Philosophy}

Closely related to Metamath\index{Metamath} is a philosophy or way of looking
at mathematics. This philosophy is related to the formalist
philosophy\index{formalism} of Hilbert\index{Hilbert, David} and his followers
\cite[pp.~1203--1208]{Kline}\index{Kline, Morris}
\cite[p.~6]{Behnke}\index{Behnke, H.}. In this philosophy, mathematics is
viewed as nothing more than a set of rules that manipulate symbols, together
with the consequences of those rules.  While the mathematics being described
may be complex, the rules used to describe it (the
``metamathematics''\index{metamathematics}) should be as simple as possible.
In particular, proofs should be restricted to dealing with concrete objects
(the symbols we write on paper rather than the abstract concepts they
represent) in a constructive manner; these are called ``finitary''
proofs\index{finitary proof} \cite[pp.~2--3]{Shoenfield}\index{Shoenfield,
Joseph R.}.

Whether or not you find Metamath interesting or useful will in part depend on
the appeal you find in its philosophy, and this appeal will probably depend on
your particular goals with respect to mathematics.  For example, if you are a
pure mathematician at the forefront of discovering new mathematical knowledge,
you will probably find that the rigid formality of Metamath stifles your
creativity.  On the other hand, we would argue that once this knowledge is
discovered, there are advantages to documenting it in a standard format that
will make it accessible to others.  Sixty years from now, your field may be
dormant, and as Davis and Hersh put it, your ``writings would become less
translatable than those of the Maya'' \cite[p.~37]{Davis}\index{Davis, Phillip
J.}.


\subsection{A History of the Approach Behind Metamath}

Probably the one work that has had the most motivating influence on
Metamath\index{Metamath} is Whitehead and Russell's monumental {\em Principia
Mathematica} \cite{PM}\index{Whitehead, Alfred North}\index{Russell,
Bertrand}\index{principia mathematica@{\em Principia Mathematica}}, whose aim
was to deduce all of mathematics from a small number of primitive ideas, in a
very explicit way that in principle anyone could understand and follow.  While
this work was tremendously influential in its time, from a modern perspective
it suffers from several drawbacks.  Both its notation and its underlying
axioms are now considered dated and are no longer used.  From our point of
view, its development is not really as accessible as we would like to see; for
practical reasons, proofs become more and more sketchy as its mathematics
progresses, and working them out in fine detail requires a degree of
mathematical skill and patience that many people don't have.  There are
numerous small errors, which is understandable given the tedious, technical
nature of the proofs and the lack of a computer to verify the details.
However, even today {\em Principia Mathematica} stands out as the work closest
in spirit to Metamath.  It remains a mind-boggling work, and one can't help
but be amazed at seeing ``$1+1=2$'' finally appear on page 83 of Volume II
(Theorem *110.643).

The origin of the proof notation used by Metamath dates back to the 1950's,
when the logician C.~A.~Meredith expressed his proofs in a compact notation
called ``condensed detachment''\index{condensed detachment}
\cite{Hindley}\index{Hindley, J. Roger} \cite{Kalman}\index{Kalman, J. A.}
\cite{Meredith}\index{Meredith, C. A.} \cite{Peterson}\index{Peterson, Jeremy
George}.  This notation allows proofs to be communicated unambiguously by
merely referencing the axiom\index{axiom}, rule\index{rule}, or
theorem\index{theorem} used at each step, without explicitly indicating the
substitutions\index{substitution!variable}\index{variable substitution} that
have to be made to the variables in that axiom, rule, or theorem.  Ordinarily,
condensed detachment is more or less limited to propositional
calculus\index{propositional calculus}.  The concept has been extended to
first-order logic\index{first-order logic} in \cite{Megill}\index{Megill,
Norman}, making it is easy to write a small computer program to verify proofs
of simple first-order logic theorems.\index{condensed detachment!and
first-order logic}

A key concept behind the notation of condensed detachment is called
``unification,''\index{unification} which is an algorithm for determining what
substitutions\index{substitution!variable}\index{variable substitution} to
variables have to be made to make two expressions match each other.
Unification was first precisely defined by the logician J.~A.~Robinson, who
used it in the development of a powerful
theorem-proving technique called the ``resolution principle''
\cite{Robinson}\index{Robinson's resolution principle}. Metamath does not make
use of the resolution principle, which is intended for systems of first-order
logic.\index{first-order logic}  Metamath's use is not restricted to
first-order logic, and as we have mentioned it does not automatically discover
proofs.  However, unification is a key idea behind Metamath's proof
notation, and Metamath makes use of a very simple version of it
(Section~\ref{unify}).

\subsection{Metamath and First-Order Logic}

First-order logic\index{first-order logic} is the supporting structure
for standard mathematics.  On top of it is set theory, which contains
the axioms from which virtually all of mathematics can be derived---a
remarkable fact.\index{category
theory}\index{cardinal, inaccessible}\label{categoryth}\footnote{An exception seems
to be category theory.  There are several schools of thought on whether
category theory is derivable from set theory.  At a minimum, it appears
that an additional axiom is needed that asserts the existence of an
``inaccessible cardinal'' (a type of infinity so large that standard set
theory can't prove or deny that it exists).
%
%%%% (I took this out that was in previous editions:)
% But it is also argued that not just one but a ``proper class'' of them
% is needed, and the existence of proper classes is impossible in standard
% set theory.  (A proper class is a collection of sets so huge that no set
% can contain it as an element.  Proper classes can lead to
% inconsistencies such as ``Russell's paradox.''  The axioms of standard
% set theory are devised so as to deny the existence of proper classes.)
%
For more information, see
\cite[pp.~328--331]{Herrlich}\index{Herrlich, Horst} and
\cite{Blass}\index{Blass, Andrea}.}

One of the things that makes Metamath\index{Metamath} more practical for
first-order theories is a set of axioms for first-order logic designed
specifically with Metamath's approach in mind.  These are included in
the database \texttt{set.mm}\index{set theory database (\texttt{set.mm})}.
See Chapter~\ref{fol} for a detailed
description; the axioms are shown in Section~\ref{metaaxioms}.  While
logically equivalent to standard axiom systems, our axiom system breaks
up the standard axioms into smaller pieces such that from them, you can
directly derive what in other systems can only be derived as higher-level
``metatheorems.''\index{metatheorem}  In other words, it is more powerful than
the standard axioms from a metalogical point of view.  A rigorous
justification for this system and its ``metalogical
completeness''\index{metalogical completeness} is found in
\cite{Megill}\index{Megill, Norman}.  The system is closely related to a
system developed by Monk\index{Monk, J. Donald} and Tarski\index{Tarski,
Alfred} in 1965 \cite{Monks}.

For example, the formula $\exists x \, x = y $ (given $y$, there exists some
$x$ equal to it) is a theorem of logic,\footnote{Specifically, it is a theorem
of those systems of logic that assume non-empty domains.  It is not a theorem
of more general systems that include the empty domain\index{empty domain}, in
which nothing exists, period!  Such systems are called ``free
logics.''\index{free logic} For a discussion of these systems, see
\cite{Leblanc}\index{Leblanc, Hugues}.  Since our use for logic is as a basis
for set theory, which has a non-empty domain, it is more convenient (and more
traditional) to use a less general system.  An interesting curiosity is that,
using a free logic as a basis for Zermelo--Fraenkel set
theory\index{Zermelo--Fraenkel set theory} (with the redundant Axiom of the
Null Set omitted),\index{Axiom of the Null Set} we cannot even prove the
existence of a single set without assuming the axiom of infinity!\index{Axiom
of Infinity}} whether or not $x$ and $y$ are distinct variables\index{distinct
variables}.  In many systems of logic, we would have to prove two theorems to
arrive at this result.  First we would prove ``$\exists x \, x = x $,'' then
we would separately prove ``$\exists x \, x = y $, where $x$ and $y$ are
distinct variables.''  We would then combine these two special cases ``outside
of the system'' (i.e.\ in our heads) to be able to claim, ``$\exists x \, x =
y $, regardless of whether $x$ and $y$ are distinct.''  In other words, the
combination of the two special cases is a
metatheorem.  In the system of logic
used in Metamath's set theory\index{set theory database (\texttt{set.mm})}
database, the axioms of logic are broken down into small pieces that allow
them to be reassembled in such a way that theorems such as these can be proved
directly.

Breaking down the axioms in this way makes them look peculiar and not very
intuitive at first, but rest assured that they are correct and complete.  Their
correctness is ensured because they are theorem schemes of standard first-order
logic (which you can easily verify if you are a logician).  Their completeness
follows from the fact that we explicitly derive the standard axioms of
first-order logic as theorems.  Deriving the standard axioms is somewhat
tricky, but once we're there, we have at our disposal a system that is less
awkward to work with in formal proofs\index{formal proof}.  In technical terms
that logicians understand, we eliminate the cumbersome concepts of ``free
variable,''\index{free variable} ``bound variable,''\index{bound variable} and
``proper substitution''\index{proper substitution}\index{substitution!proper}
as primitive notions.  These concepts are present in our system but are
defined in terms of concepts expressed by the axioms and can be eliminated in
principle.  In standard systems, these concepts are really like additional,
implicit axioms\index{implicit axiom} that are somewhat complex and cannot be
eliminated.

The traditional approach to logic, wherein free variables and proper
substitution is defined, is also possible to do directly in the Metamath
language.  However, the notation tends to become awkward, and there are
disadvantages:  for example, extending the definition of a wff with a
definition is awkward, because the free variable and proper substitution
concepts have to have their definitions also extended.  Our choice of
axioms for \texttt{set.mm} is to a certain extent a matter of style, in
an attempt to achieve overall simplicity, but you should also be aware
that the traditional approach is possible as well if you should choose
it.

\chapter{Using the Metamath Program}
\label{using}

\section{Installation}

The way that you install Metamath\index{Metamath!installation} on your
computer system will vary for different computers.  Current
instructions are provided with the Metamath program download at
\url{http://metamath.org}.  In general, the installation is simple.
There is one file containing the Metamath program itself.  This file is
usually called \texttt{metamath} or \texttt{metamath.}{\em xxx} where
{\em xxx} is the convention (such as \texttt{exe}) for an executable
program on your operating system.  There are several additional files
containing samples of the Metamath language, all ending with
\texttt{.mm}.  The file \texttt{set.mm}\index{set theory database
(\texttt{set.mm})} contains logic and set theory and can be used as a
starting point for other areas of mathematics.

You will also need a text editor\index{text editor} capable of editing plain
{\sc ascii}\footnote{American Standard Code for Information Interchange.} text
in order to prepare your input files.\index{ascii@{\sc ascii}}  Most computers
have this capability built in.  Note that plain text is not necessarily the
default for some word processing programs\index{word processor}, especially if
they can handle different fonts; for example, with Microsoft Word\index{Word
(Microsoft)}, you must save the file in the format ``Text Only With Line
Breaks'' to get a plain text\index{plain text} file.\footnote{It is
recommended that all lines in a Metamath source file be 79 characters or less
in length for compatibility among different computer terminals.  When creating
a source file on an editor such as Word, select a monospaced
font\index{monospaced font} such as Courier\index{Courier font} or
Monaco\index{Monaco font} to make this easier to achieve.  Better yet,
just use a plain text editor such as Notepad.}

On some computer systems, Metamath does not have the capability to print
its output directly; instead, you send its output to a file (using the
\texttt{open} commands described later).  The way you print this output
file depends on your computer.\index{printers} Some computers have a
print command, whereas with others, you may have to read the file into
an editor and print it from there.

If you want to print your Metamath source files with typeset formulas
containing standard mathematical symbols, you will need the \LaTeX\
typesetting program\index{latex@{\LaTeX}}, which is widely and freely
available for most operating systems.  It runs natively on Unix and
Linux, and can be installed on Windows as part of the free Cygwin
package (\url{http://cygwin.com}).

You can also produce {\sc html}\footnote{HyperText Markup Language.}
web pages.  The {\tt help html} command in the Metamath program will
assist you with this feature.

\section{Your First Formal System}\label{start}
\subsection{From Nothing to Zero}\label{startf}

To give you a feel for what the Metamath\index{Metamath} language looks like,
we will take a look at a very simple example from formal number
theory\index{number theory}.  This example is taken from
Mendelson\index{Mendelson, Elliot} \cite[p. 123]{Mendelson}.\footnote{To keep
the example simple, we have changed the formalism slightly, and what we call
axioms\index{axiom} are strictly speaking theorems\index{theorem} in
\cite{Mendelson}.}  We will look at a small subset of this theory, namely that
part needed for the first number theory theorem proved in \cite{Mendelson}.

First we will look at a standard formal proof\index{formal proof} for the
example we have picked, then we will look at the Metamath version.  If you
have never been exposed to formal proofs, the notation may seem to be such
overkill to express such simple notions that you may wonder if you are missing
something.  You aren't.  The concepts involved are in fact very simple, and a
detailed breakdown in this fashion is necessary to express the proof in a way
that can be verified mechanically.  And as you will see, Metamath breaks the
proof down into even finer pieces so that the mechanical verification process
can be about as simple as possible.

Before we can introduce the axioms\index{axiom} of the theory, we must define
the syntax rules for forming legal expressions\index{syntax rules}
(combinations of symbols) with which those axioms can be used. The number 0 is
a {\bf term}\index{term}; and if $ t$ and $r$ are terms, so is $(t+r)$. Here,
$ t$ and $r$ are ``metavariables''\index{metavariable} ranging over terms; they
themselves do not appear as symbols in an actual term.  Some examples of
actual terms are $(0 + 0)$ and $((0+0)+0)$.  (Note that our theory describes
only the number zero and sums of zeroes.  Of course, not much can be done with
such a trivial theory, but remember that we have picked a very small subset of
complete number theory for our example.  The important thing for you to focus
on is our definitions that describe how symbols are combined to form valid
expressions, and not on the content or meaning of those expressions.) If $ t$
and $r$ are terms, an expression of the form $ t=r$ is a {\bf wff}
(well-formed formula)\index{well-formed formula (wff)}; and if $P$ and $Q$ are
wffs, so is $(P\rightarrow Q)$ (which means ``$P$ implies
$Q$''\index{implication ($\rightarrow$)} or ``if $P$ then $Q$'').
Here $P$ and $Q$ are metavariables ranging over wffs.  Examples of actual
wffs are $0=0$, $(0+0)=0$, $(0=0 \rightarrow (0+0)=0)$, and $(0=0\rightarrow
(0=0\rightarrow 0=(0+0)))$.  (Our notation makes use of more parentheses than
are customary, but the elimination of ambiguity this way simplifies our
example by avoiding the need to define operator precedence\index{operator
precedence}.)

The {\bf axioms}\index{axiom} of our theory are all wffs of the following
form, where $ t$, $r$, and $s$ are any terms:

%Latex p. 92
\renewcommand{\theequation}{A\arabic{equation}}

\begin{equation}
(t=r\rightarrow (t=s\rightarrow r=s))
\end{equation}
\begin{equation}
(t+0)=t
\end{equation}

Note that there are an infinite number of axioms since there are an infinite
number of possible terms.  A1 and A2 are properly called ``axiom
schemes,''\index{axiom scheme} but we will refer to them as ``axioms'' for
brevity.

An axiom is a {\bf theorem}; and if $P$ and $(P\rightarrow Q)$ are theorems
(where $P$ and $Q$ are wffs), then $Q$ is also a theorem.\index{theorem}  The
second part of this definition is called the modus ponens (MP) rule of
inference\index{inference rule}\index{modus ponens}.  It allows us to obtain
new theorems from old ones.

The {\bf proof}\index{proof} of a theorem is a sequence of one or more
theorems, each of which is either an axiom or the result of modus ponens
applied to two previous theorems in the sequence, and the last of which is the
theorem being proved.

The theorem we will prove for our example is very simple:  $ t=t$.  The proof of
our theorem follows.  Study it carefully until you feel sure you
understand it.\label{zeroproof}

% Use tabu so that lines will wrap automatically as needed.
\begin{tabu} { l X X }
1. & $(t+0)=t$ & (by axiom A2) \\
2. & $(t+0)=t$ & (by axiom A2) \\
3. & $((t+0)=t \rightarrow ((t+0)=t\rightarrow t=t))$ & (by axiom A1) \\
4. & $((t+0)=t\rightarrow t=t)$ & (by MP applied to steps 2 and 3) \\
5. & $t=t$ & (by MP applied to steps 1 and 4) \\
\end{tabu}

(You may wonder why step 1 is repeated twice.  This is not necessary in the
formal language we have defined, but in Metamath's ``reverse Polish
notation''\index{reverse Polish notation (RPN)} for proofs, a previous step
can be referred to only once.  The repetition of step~1 here will enable you
to see more clearly the correspondence of this proof with the
Metamath\index{Metamath} version on p.~\pageref{demoproof}.)

Our theorem is more properly called a ``theorem scheme,''\index{theorem
scheme} for it represents an infinite number of theorems, one for each
possible term $ t$.  Two examples of actual theorems would be $0=0$ and
$(0+0)=(0+0)$.  Rarely do we prove actual theorems, since by proving schemes
we can prove an infinite number of theorems in one fell swoop.  Similarly, our
proof should really be called a ``proof scheme.''\index{proof scheme}  To
obtain an actual proof, pick an actual term to use in place of $ t$, and
substitute it for $ t$ throughout the proof.

Let's discuss what we have done here.  The axioms\index{axiom} of our theory,
A1 and A2, are trivial and obvious.  Everyone knows that adding zero to
something doesn't change it, and also that if two things are equal to a third,
then they are equal to each other. In fact, stating the trivial and obvious is
a goal to strive for in any axiomatic system.  From trivial and obvious truths
that everyone agrees upon, we can prove results that are not so obvious yet
have absolute faith in them.  If we trust the axioms and the rules, we must,
by definition, trust the consequences of those axioms and rules, if logic is
to mean anything at all.

Our rule of inference\index{rule}, modus ponens\index{modus ponens}, is also
pretty obvious once you understand what it means.  If we prove a fact $P$, and
we also prove that $P$ implies $Q$, then $Q$ necessarily follows as a new
fact.  The rule provides us with a means for obtaining new facts (i.e.\
theorems\index{theorem}) from old ones.

The theorem that we have proved, $ t=t$, is so fundamental that you may wonder
why it isn't one of the axioms\index{axiom}.  In some axiom systems of
arithmetic, it {\em is} an axiom.  The choice of axioms in a theory is to some
extent arbitrary and even an art form, constrained only by the requirement
that any two equivalent axiom systems be able to derive each other as
theorems.  We could imagine that the inventor of our axiom system originally
included $ t=t$ as an axiom, then discovered that it could be derived as a
theorem from the other axioms.  Because of this, it was not necessary to
keep it as an axiom.  By eliminating it, the final set of axioms became
that much simpler.

Unless you have worked with formal proofs\index{formal proof} before, it
probably wasn't apparent to you that $ t=t$ could be derived from our two
axioms until you saw the proof. While you certainly believe that $ t=t$ is
true, you might not be able to convince an imaginary skeptic who believes only
in our two axioms until you produce the proof.  Formal proofs such as this are
hard to come up with when you first start working with them, but after you get
used to them they can become interesting and fun.  Once you understand the
idea behind formal proofs you will have grasped the fundamental principle that
underlies all of mathematics.  As the mathematics becomes more sophisticated,
its proofs become more challenging, but ultimately they all can be broken down
into individual steps as simple as the ones in our proof above.

Mendelson's\index{Mendelson, Elliot} book, from which our example was taken,
contains a number of detailed formal proofs such as these, and you may be
interested in looking it up.  The book is intended for mathematicians,
however, and most of it is rather advanced.  Popular literature describing
formal proofs\index{formal proof} include \cite[p.~296]{Rucker}\index{Rucker,
Rudy} and \cite[pp.~204--230]{Hofstadter}\index{Hofstadter, Douglas R.}.

\subsection{Converting It to Metamath}\label{convert}

Formal proofs\index{formal proof} such as the one in our example break down
logical reasoning into small, precise steps that leave little doubt that the
results follow from the axioms\index{axiom}.  You might think that this would
be the finest breakdown we can achieve in mathematics.  However, there is more
to the proof than meets the eye. Although our axioms were rather simple, a lot
of verbiage was needed before we could even state them:  we needed to define
``term,'' ``wff,'' and so on.  In addition, there are a number of implied
rules that we haven't even mentioned. For example, how do we know that step 3
of our proof follows from axiom A1? There is some hidden reasoning involved in
determining this.  Axiom A1 has two occurrences of the letter $ t$.  One of
the implied rules states that whatever we substitute for $ t$ must be a legal
term\index{term}.\footnote{Some authors make this implied rule explicit by
stating, ``only expressions of the above form are terms,'' after defining
``term.''}  The expression $ t+0$ is pretty obviously a legal term whenever $
t$ is, but suppose we wanted to substitute a huge term with thousands of
symbols?  Certainly a lot of work would be involved in determining that it
really is a term, but in ordinary formal proofs all of this work would be
considered a single ``step.''

To express our axiom system in the Metamath\index{Metamath} language, we must
describe this auxiliary information in addition to the axioms themselves.
Metamath does not know what a ``term'' or a ``wff''\index{well-formed formula
(wff)} is.  In Metamath, the specification of the ways in which we can combine
symbols to obtain terms and wffs are like little axioms in themselves.  These
auxiliary axioms are expressed in the same notation as the ``real''
axioms\index{axiom}, and Metamath does not distinguish between the two.  The
distinction is made by you, i.e.\ by the way in which you interpret the
notation you have chosen to express these two kinds of axioms.

The Metamath language breaks down mathematical proofs into tiny pieces, much
more so than in ordinary formal proofs\index{formal proof}.  If a single
step\index{proof step} involves the
substitution\index{substitution!variable}\index{variable substitution} of a
complex term for one of its variables, Metamath must see this single step
broken down into many small steps.  This fine-grained breakdown is what gives
Metamath generality and flexibility as it lets it not be limited to any
particular mathematical notation.

Metamath's proof notation is not, in itself, intended to be read by humans but
rather is in a compact format intended for a machine.  The Metamath program
will convert this notation to a form you can understand, using the \texttt{show
proof}\index{\texttt{show proof} command} command.  You can tell the program what
level of detail of the proof you want to look at.  You may want to look at
just the logical inference steps that correspond
to ordinary formal proof steps,
or you may want to see the fine-grained steps that prove that an expression is
a term.

Here, without further ado, is our example converted to the
Metamath\index{Metamath} language:\index{metavariable}\label{demo0}

\begin{verbatim}
$( Declare the constant symbols we will use $)
    $c 0 + = -> ( ) term wff |- $.
$( Declare the metavariables we will use $)
    $v t r s P Q $.
$( Specify properties of the metavariables $)
    tt $f term t $.
    tr $f term r $.
    ts $f term s $.
    wp $f wff P $.
    wq $f wff Q $.
$( Define "term" and "wff" $)
    tze $a term 0 $.
    tpl $a term ( t + r ) $.
    weq $a wff t = r $.
    wim $a wff ( P -> Q ) $.
$( State the axioms $)
    a1 $a |- ( t = r -> ( t = s -> r = s ) ) $.
    a2 $a |- ( t + 0 ) = t $.
$( Define the modus ponens inference rule $)
    ${
       min $e |- P $.
       maj $e |- ( P -> Q ) $.
       mp  $a |- Q $.
    $}
$( Prove a theorem $)
    th1 $p |- t = t $=
  $( Here is its proof: $)
       tt tze tpl tt weq tt tt weq tt a2 tt tze tpl
       tt weq tt tze tpl tt weq tt tt weq wim tt a2
       tt tze tpl tt tt a1 mp mp
     $.
\end{verbatim}\index{metavariable}

A ``database''\index{database} is a set of one or more {\sc ascii} source
files.  Here's a brief description of this Metamath\index{Metamath} database
(which consists of this single source file), so that you can understand in
general terms what is going on.  To understand the source file in detail, you
should read Chapter~\ref{languagespec}.

The database is a sequence of ``tokens,''\index{token} which are normally
separated by spaces or line breaks.  The only tokens that are built into
the Metamath language are those beginning with \texttt{\$}.  These tokens
are called ``keywords.''\index{keyword}  All other tokens are
user-defined, and their names are arbitrary.

As you might have guessed, the Metamath token \texttt{\$(}\index{\texttt{\$(} and
\texttt{\$)} auxiliary keywords} starts a comment and \texttt{\$)} ends a comment.

The Metamath tokens \texttt{\$c}\index{\texttt{\$c} statement},
\texttt{\$v}\index{\texttt{\$v} statement},
\texttt{\$e}\index{\texttt{\$e} statement},
\texttt{\$f}\index{\texttt{\$f} statement},
\texttt{\$a}\index{\texttt{\$a} statement}, and
\texttt{\$p}\index{\texttt{\$p} statement} specify ``statements'' that
end with \texttt{\$.}\,.\index{\texttt{\$.}\ keyword}

The Metamath tokens \texttt{\$c} and \texttt{\$v} each declare\index{constant
declaration}\index{variable declaration} a list of user-defined tokens, called
``math symbols,''\index{math symbol} that the database will reference later
on.  All of the math symbols they define you have seen earlier except the
turnstile symbol \texttt{|-} ($\vdash$)\index{turnstile ({$\,\vdash$})}, which is
commonly used by logicians to mean ``a proof exists for.''  For us
the turnstile is just a
convenient symbol that distinguishes expressions that are axioms\index{axiom}
or theorems\index{theorem} from expressions that are terms or wffs.

The \texttt{\$c} statement declares ``constants''\index{constant} and
the \texttt{\$v} statement declares
``variables''\index{variable}\index{constant declaration}\index{variable
declaration} (or more precisely, metavariables\index{metavariable}).  A
variable may be substituted\index{substitution!variable}\index{variable
substitution} with sequences of math symbols whereas a constant may not
be substituted with anything.

It may seem redundant to require both \texttt{\$c}\index{\texttt{\$c} statement} and
\texttt{\$v}\index{\texttt{\$v} statement} statements (since any math
symbol\index{math symbol} not specified with a \texttt{\$c} statement could be
presumed to be a variable), but this provides for better error checking and
also allows math symbols to be redeclared\index{redeclaration of symbols}
(Section~\ref{scoping}).

The token \texttt{\$f}\index{\texttt{\$f} statement} specifies a
statement called a ``variable-type hypothesis'' (also called a
``floating hypothesis'') and \texttt{\$e}\index{\texttt{\$e} statement}
specifies a ``logical hypothesis'' (also called an ``essential
hypothesis'').\index{hypothesis}\index{variable-type
hypothesis}\index{logical hypothesis}\index{floating
hypothesis}\index{essential hypothesis} The token
\texttt{\$a}\index{\texttt{\$a} statement} specifies an ``axiomatic
assertion,''\index{axiomatic assertion} and
\texttt{\$p}\index{\texttt{\$p} statement} specifies a ``provable
assertion.''\index{provable assertion} To the left of each occurrence of
these four tokens is a ``label''\index{label} that identifies the
hypothesis or assertion for later reference.  For example, the label of
the first axiomatic assertion is \texttt{tze}.  A \texttt{\$f} statement
must contain exactly two math symbols, a constant followed by a
variable.  The \texttt{\$e}, \texttt{\$a}, and \texttt{\$p} statements
each start with a constant followed by, in general, an arbitrary
sequence of math symbols.

Associated with each assertion\index{assertion} is a set of hypotheses
that must be satisfied in order for the assertion to be used in a proof.
These are called the ``mandatory hypotheses''\index{mandatory
hypothesis} of the assertion.  Among those hypotheses whose ``scope''
(described below) includes the assertion, \texttt{\$e} hypotheses are
always mandatory and \texttt{\$f}\index{\texttt{\$f} statement}
hypotheses are mandatory when they share their variable with the
assertion or its \texttt{\$e} hypotheses.  The exact rules for
determining which hypotheses are mandatory are described in detail in
Sections~\ref{frames} and \ref{scoping}.  For example, the mandatory
hypotheses of assertion \texttt{tpl} are \texttt{tt} and \texttt{tr},
whereas assertion \texttt{tze} has no mandatory hypotheses because it
contains no variables and has no \texttt{\$e}\index{\texttt{\$e}
statement} hypothesis.  Metamath's \texttt{show statement}
command\index{\texttt{show statement} command}, described in the next
section, will show you a statement's mandatory hypotheses.

Sometimes we need to make a hypothesis relevant to only certain
assertions.  The set of statements to which a hypothesis is relevant is
called its ``scope.''  The Metamath brackets,
\texttt{\$\char`\{}\index{\texttt{\$\char`\{} and \texttt{\$\char`\}}
keywords} and \texttt{\$\char`\}}, define a ``block''\index{block} that
delimits the scope of any hypothesis contained between them.  The
assertion \texttt{mp} has mandatory hypotheses \texttt{wp}, \texttt{wq},
\texttt{min}, and \texttt{maj}.  The only mandatory hypothesis of
\texttt{th1}, on the other hand, is \texttt{tt}, since \texttt{th1}
occurs outside of the block containing \texttt{min} and \texttt{maj}.

Note that \texttt{\$\char`\{} and \texttt{\$\char`\}} do not affect the
scope of assertions (\texttt{\$a} and \texttt{\$p}).  Assertions are always
available to be referenced by any later proof in the source file.

Each provable assertion (\texttt{\$p}\index{\texttt{\$p} statement}
statement) has two parts.  The first part is the
assertion\index{assertion} itself, which is a sequence of math
symbol\index{math symbol} tokens placed between the \texttt{\$p} token
and a \texttt{\$=}\index{\texttt{\$=} keyword} token.  The second part
is a ``proof,'' which is a list of label tokens placed between the
\texttt{\$=} token and the \texttt{\$.}\index{\texttt{\$.}\ keyword}\
token that ends the statement.\footnote{If you've looked at the
\texttt{set.mm} database, you may have noticed another notation used for
proofs.  The other notation is called ``compressed.''\index{compressed
proof}\index{proof!compressed} It reduces the amount of space needed to
store a proof in the database and is described in
Appendix~\ref{compressed}.  In the example above, we use
``normal''\index{normal proof}\index{proof!normal} notation.} The proof
acts as a series of instructions to the Metamath program, telling it how
to build up the sequence of math symbols contained in the assertion part of
the \texttt{\$p} statement, making use of the hypotheses of the
\texttt{\$p} statement and previous assertions.  The construction takes
place according to precise rules.  If the list of labels in the proof
causes these rules to be violated, or if the final sequence that results
does not match the assertion, the Metamath program will notify you with
an error message.

If you are familiar with reverse Polish notation (RPN), which is sometimes used
on pocket calculators, here in a nutshell is how a proof works.  Each
hypothesis label\index{hypothesis label} in the proof is pushed\index{push}
onto the RPN stack\index{stack}\index{RPN stack} as it is encountered. Each
assertion label\index{assertion label} pops\index{pop} off the stack as many
entries as the referenced assertion has mandatory hypotheses.  Variable
substitutions\index{substitution!variable}\index{variable substitution} are
computed which, when made to the referenced assertion's mandatory hypotheses,
cause these hypotheses to match the stack entries. These same substitutions
are then made to the variables in the referenced assertion itself, which is
then pushed onto the stack.  At the end of the proof, there should be one
stack entry, namely the assertion being proved.  This process is explained in
detail in Section~\ref{proof}.

Metamath's proof notation is not very readable for humans, but it allows the
proof to be stored compactly in a file.  The Metamath\index{Metamath} program
has proof display features that let you see what's going on in a more
readable way, as you will see in the next section.

The rules used in verifying a proof are not based on any built-in syntax of the
symbol sequence in an assertion\index{assertion} nor on any built-in meanings
attached to specific symbol names.  They are based strictly on symbol
matching:  constants\index{constant} must match themselves, and
variables\index{variable} may be replaced with anything that allows a match to
occur.  For example, instead of \texttt{term}, \texttt{0}, and \verb$|-$ we could
have just as well used \texttt{yellow}, \texttt{zero}, and \texttt{provable}, as long
as we did so consistently throughout the database.  Also, we could have used
\texttt{is provable} (two tokens) instead of \verb$|-$ (one token) throughout the
database.  In each of these cases, the proof would be exactly the same.  The
independence of proofs and notation means that you have a lot of flexibility to
change the notation you use without having to change any proofs.

\section{A Trial Run}\label{trialrun}

Now you are ready to try out the Metamath\index{Metamath} program.

On all computer systems, Metamath has a standard ``command line
interface'' (CLI)\index{command line interface (CLI)} that allows you to
interact with it.  You supply commands to the CLI by typing them on the
keyboard and pressing your keyboard's {\em return} key after each line
you enter.  The CLI is designed to be easy to use and has built-in help
features.

The first thing you should do is to use a text editor to create a file
called \texttt{demo0.mm} and type into it the Metamath source shown on
p.~\pageref{demo0}.  Actually, this file is included with your Metamath
software package, so check that first.  If you type it in, make sure
that you save it in the form of ``plain {\sc ascii} text with line
breaks.''  Most word processors will have this feature.

Next you must run the Metamath program.  Depending on your computer
system and how Metamath is installed, this could range from clicking the
mouse on the Metamath icon to typing \texttt{run metamath} to typing
simply \texttt{metamath}.  (Metamath's {\tt help invoke} command describes
alternate ways of invoking the Metamath program.)

When you first enter Metamath\index{Metamath}, it will be at the CLI, waiting
for your input. You will see something like the following on your screen:
\begin{verbatim}
Metamath - Version 0.177 27-Apr-2019
Type HELP for help, EXIT to exit.
MM>
\end{verbatim}
The \texttt{MM>} prompt means that Metamath is waiting for a command.
Command keywords\index{command keyword} are not case sensitive;
we will use lower-case commands in our examples.
The version number and its release date will probably be different on your
system from the one we show above.

The first thing that you need to do is to read in your
database:\index{\texttt{read} command}\footnote{If a directory path is
needed on Unix,\index{Unix file names}\index{file names!Unix} you should
enclose the path/file name in quotes to prevent Metamath from thinking
that the \texttt{/} in the path name is a command qualifier, e.g.,
\texttt{read \char`\"db/set.mm\char`\"}.  Quotes are optional when there
is no ambiguity.}
\begin{verbatim}
MM> read demo0.mm
\end{verbatim}
Remember to press the {\em return} key after entering this command.  If
you omit the file name, Metamath will prompt you for one.   The syntax for
specifying a Macintosh file name path is given in a footnote on
p.~\pageref{includef}.\index{Macintosh file names}\index{file
names!Macintosh}

If there are any syntax errors in the database, Metamath will let you know
when it reads in the file.  The one thing that Metamath does not check when
reading in a database is that all proofs are correct, because this would
slow it down too much.  It is a good idea to periodically verify the proofs in
a database you are making changes to.  To do this, use the following command
(and do it for your \texttt{demo0.mm} file now).  Note that the \texttt{*} is a
``wild card'' meaning all proofs in the file.\index{\texttt{verify proof} command}
\begin{verbatim}
MM> verify proof *
\end{verbatim}
Metamath will report any proofs that are incorrect.

It is often useful to save the information that the Metamath program displays
on the screen. You can save everything that happens on the screen by opening a
log file. You may want to do this before you read in a database so that you
can examine any errors later on.  To open a log file, type
\begin{verbatim}
MM> open log abc.log
\end{verbatim}
This will open a file called \texttt{abc.log}, and everything that appears on the
screen from this point on will be stored in this file.  The name of the log file
is arbitrary. To close the log file, type
\begin{verbatim}
MM> close log
\end{verbatim}

Several commands let you examine what's inside your database.
Section~\ref{exploring} has an overview of some useful ones.  The
\texttt{show labels} command lets you see what statement
labels\index{label} exist.  A \texttt{*} matches any combination of
characters, and \texttt{t*} refers to all labels starting with the
letter \texttt{t}.\index{\texttt{show labels} command} The \texttt{/all}
is a ``command qualifier''\index{command qualifier} that tells Metamath
to include labels of hypotheses.  (To see the syntax explained, type
\texttt{help show labels}.)  Type
\begin{verbatim}
MM> show labels t* /all
\end{verbatim}
Metamath will respond with
\begin{verbatim}
The statement number, label, and type are shown.
3 tt $f       4 tr $f       5 ts $f       8 tze $a
9 tpl $a      19 th1 $p
\end{verbatim}

You can use the \texttt{show statement} command to get information about a
particular statement.\index{\texttt{show statement} command}
For example, you can get information about the statement with label \texttt{mp}
by typing
\begin{verbatim}
MM> show statement mp /full
\end{verbatim}
Metamath will respond with
\begin{verbatim}
Statement 17 is located on line 43 of the file
"demo0.mm".
"Define the modus ponens inference rule"
17 mp $a |- Q $.
Its mandatory hypotheses in RPN order are:
  wp $f wff P $.
  wq $f wff Q $.
  min $e |- P $.
  maj $e |- ( P -> Q ) $.
The statement and its hypotheses require the
      variables:  Q P
The variables it contains are:  Q P
\end{verbatim}
The mandatory hypotheses\index{mandatory hypothesis} and their
order\index{RPN order} are
useful to know when you are trying to understand or debug a proof.

Now you are ready to look at what's really inside our proof.  First, here is
how to look at every step in the proof---not just the ones corresponding to an
ordinary formal proof\index{formal proof}, but also the ones that build up the
formulas that appear in each ordinary formal proof step.\index{\texttt{show
proof} command}
\begin{verbatim}
MM> show proof th1 /lemmon /all
\end{verbatim}

This will display the proof on the screen in the following format:
\begin{verbatim}
 1 tt            $f term t
 2 tze           $a term 0
 3 1,2 tpl       $a term ( t + 0 )
 4 tt            $f term t
 5 3,4 weq       $a wff ( t + 0 ) = t
 6 tt            $f term t
 7 tt            $f term t
 8 6,7 weq       $a wff t = t
 9 tt            $f term t
10 9 a2          $a |- ( t + 0 ) = t
11 tt            $f term t
12 tze           $a term 0
13 11,12 tpl     $a term ( t + 0 )
14 tt            $f term t
15 13,14 weq     $a wff ( t + 0 ) = t
16 tt            $f term t
17 tze           $a term 0
18 16,17 tpl     $a term ( t + 0 )
19 tt            $f term t
20 18,19 weq     $a wff ( t + 0 ) = t
21 tt            $f term t
22 tt            $f term t
23 21,22 weq     $a wff t = t
24 20,23 wim     $a wff ( ( t + 0 ) = t -> t = t )
25 tt            $f term t
26 25 a2         $a |- ( t + 0 ) = t
27 tt            $f term t
28 tze           $a term 0
29 27,28 tpl     $a term ( t + 0 )
30 tt            $f term t
31 tt            $f term t
32 29,30,31 a1   $a |- ( ( t + 0 ) = t -> ( ( t + 0 )
                                     = t -> t = t ) )
33 15,24,26,32 mp  $a |- ( ( t + 0 ) = t -> t = t )
34 5,8,10,33 mp  $a |- t = t
\end{verbatim}

The \texttt{/lemmon} command qualifier specifies what is known as a Lemmon-style
display\index{Lemmon-style proof}\index{proof!Lemmon-style}.  Omitting the
\texttt{/lemmon} qualifier results in a tree-style proof (see
p.~\pageref{treeproof} for an example) that is somewhat less explicit but
easier to follow once you get used to it.\index{tree-style
proof}\index{proof!tree-style}

The first number on each line is the step
number of the proof.  Any numbers that follow are step numbers assigned to the
hypotheses of the statement referenced by that step.  Next is the label of
the statement referenced by the step.  The statement type of the statement
referenced comes next, followed by the math symbol\index{math symbol} string
constructed by the proof up to that step.

The last step, 34, contains the statement that is being proved.

Looking at a small piece of the proof, notice that steps 3 and 4 have
established that
\texttt{( t + 0 )} and \texttt{t} are \texttt{term}\,s, and step 5 makes use of steps 3 and
4 to establish that \texttt{( t + 0 ) = t} is a \texttt{wff}.  Let Metamath
itself tell us in detail what is happening in step 5.  Note that the
``target hypothesis'' refers to where step 5 is eventually used, i.e., in step
34.
\begin{verbatim}
MM> show proof th1 /detailed_step 5
Proof step 5:  wp=weq $a wff ( t + 0 ) = t
This step assigns source "weq" ($a) to target "wp"
($f).  The source assertion requires the hypotheses
"tt" ($f, step 3) and "tr" ($f, step 4).  The parent
assertion of the target hypothesis is "mp" ($a,
step 34).
The source assertion before substitution was:
    weq $a wff t = r
The following substitutions were made to the source
assertion:
    Variable  Substituted with
     t         ( t + 0 )
     r         t
The target hypothesis before substitution was:
    wp $f wff P
The following substitution was made to the target
hypothesis:
    Variable  Substituted with
     P         ( t + 0 ) = t
\end{verbatim}

The full proof just shown is useful to understand what is going on in detail.
However, most of the time you will just be interested in
the ``essential'' or logical steps of a proof, i.e.\ those steps
that correspond to an
ordinary formal proof\index{formal proof}.  If you type
\begin{verbatim}
MM> show proof th1 /lemmon /renumber
\end{verbatim}
you will see\label{demoproof}
\begin{verbatim}
1 a2             $a |- ( t + 0 ) = t
2 a2             $a |- ( t + 0 ) = t
3 a1             $a |- ( ( t + 0 ) = t -> ( ( t + 0 )
                                     = t -> t = t ) )
4 2,3 mp         $a |- ( ( t + 0 ) = t -> t = t )
5 1,4 mp         $a |- t = t
\end{verbatim}
Compare this to the formal proof on p.~\pageref{zeroproof} and
notice the resemblance.
By default Metamath
does not show \texttt{\$f}\index{\texttt{\$f}
statement} hypotheses and everything branching off of them in the proof tree
when the proof is displayed; this makes the proof look more like an ordinary
mathematical proof, which does not normally incorporate the explicit
construction of expressions.
This is called the ``essential'' view
(at one time you had to add the
\texttt{/essential} qualifier in the \texttt{show proof}
command to get this view, but this is now the default).
You can could use the \texttt{/all} qualifier in the \texttt{show
proof} command to also show the explicit construction of expressions.
The \texttt{/renumber} qualifier means to renumber
the steps to correspond only to what is displayed.\index{\texttt{show proof}
command}

To exit Metamath, type\index{\texttt{exit} command}
\begin{verbatim}
MM> exit
\end{verbatim}

\subsection{Some Hints for Using the Command Line Interface}

We will conclude this quick introduction to Metamath\index{Metamath} with some
helpful hints on how to navigate your way through the commands.
\index{command line interface (CLI)}

When you type commands into Metamath's CLI, you only have to type as many
characters of a command keyword\index{command keyword} as are needed to make
it unambiguous.  If you type too few characters, Metamath will tell you what
the choices are.  In the case of the \texttt{read} command, only the \texttt{r} is
needed to specify it unambiguously, so you could have typed\index{\texttt{read}
command}
\begin{verbatim}
MM> r demo0.mm
\end{verbatim}
instead of
\begin{verbatim}
MM> read demo0.mm
\end{verbatim}
In our description, we always show the full command words.  When using the
Metamath CLI commands in a command file (to be read with the \texttt{submit}
command)\index{\texttt{submit} command}, it is good practice to use
the unabbreviated command to ensure your instructions will not become ambiguous
if more commands are added to the Metamath program in the future.

The command keywords\index{command
keyword} are not case sensitive; you may type either \texttt{read} or
\texttt{ReAd}.  File names may or may not be case sensitive, depending on your
computer's operating system.  Metamath label\index{label} and math
symbol\index{math symbol} tokens\index{token} are case-sensitive.

The \texttt{help} command\index{\texttt{help} command} will provide you
with a list of topics you can get help on.  You can then type
\texttt{help} {\em topic} to get help on that topic.

If you are uncertain of a command's spelling, just type as many characters
as you remember of the command.  If you have not typed enough characters to
specify it unambiguously, Metamath will tell you what choices you have.

\begin{verbatim}
MM> show s
         ^
?Ambiguous keyword - please specify SETTINGS,
STATEMENT, or SOURCE.
\end{verbatim}

If you don't know what argument to use as part of a command, type a
\texttt{?}\index{\texttt{]}@\texttt{?}\ in command lines}\ at the
argument position.  Metamath will tell you what it expected there.

\begin{verbatim}
MM> show ?
         ^
?Expected SETTINGS, LABELS, STATEMENT, SOURCE, PROOF,
MEMORY, TRACE_BACK, or USAGE.
\end{verbatim}

Finally, you may type just the first word or words of a command followed
by {\em return}.  Metamath will prompt you for the remaining part of the
command, showing you the choices at each step.  For example, instead of
typing \texttt{show statement th1 /full} you could interact in the
following manner:
\begin{verbatim}
MM> show
SETTINGS, LABELS, STATEMENT, SOURCE, PROOF,
MEMORY, TRACE_BACK, or USAGE <SETTINGS>? st
What is the statement label <th1>?
/ or nothing <nothing>? /
TEX, COMMENT_ONLY, or FULL <TEX>? f
/ or nothing <nothing>?
19 th1 $p |- t = t $= ... $.
\end{verbatim}
After each \texttt{?}\ in this mode, you must give Metamath the
information it requests.  Sometimes Metamath gives you a list of choices
with the default choice indicated by brackets \texttt{< > }. Pressing
{\em return} after the \texttt{?}\ will select the default choice.
Answering anything else will override the default.  Note that the
\texttt{/} in command qualifiers is considered a separate
token\index{token} by the parser, and this is why it is asked for
separately.

\section{Your First Proof}\label{frstprf}

Proofs are developed with the aid of the Proof Assistant\index{Proof
Assistant}.  We will now show you how the proof of theorem \texttt{th1}
was built.  So that you can repeat these steps, we will first have the
Proof Assistant erase the proof in Metamath's source buffer\index{source
buffer}, then reconstruct it.  (The source buffer is the place in memory
where Metamath stores the information in the database when it is
\texttt{read}\index{\texttt{read} command} in.  New or modified proofs
are kept in the source buffer until a \texttt{write source}
command\index{\texttt{write source} command} is issued.)  In practice, you
would place a \texttt{?}\index{\texttt{]}@\texttt{?}\ inside proofs}\
between \texttt{\$=}\index{\texttt{\$=} keyword} and
\texttt{\$.}\index{\texttt{\$.}\ keyword}\ in the database to indicate
to Metamath\index{Metamath} that the proof is unknown, and that would be
your starting point.  Whenever the \texttt{verify proof} command encounters
a proof with a \texttt{?}\ in place of a proof step, the statement is
identified as not proved.

When I first started creating Metamath proofs, I would write down
on a piece of paper the complete
formal proof\index{formal proof} as it would appear
in a \texttt{show proof} command\index{\texttt{show proof} command}; see
the display of \texttt{show proof th1 /lemmon /re\-num\-ber} above as an
example.  After you get used to using the Proof Assistant\index{Proof
Assistant} you may get to a point where you can ``see'' the proof in your mind
and let the Proof Assistant guide you in filling in the details, at least for
simpler proofs, but until you gain that experience you may find it very useful
to write down all the details in advance.
Otherwise you may waste a lot of time as you let it take you down a wrong path.
However, others do not find this approach as helpful.
For example, Thomas Brendan Leahy\index{Leahy, Thomas Brendan}
finds that it is more helpful to him to interactively
work backward from a machine-readable statement.
David A. Wheeler\index{Wheeler, David A.}
writes down a general approach, but develops the proof
interactively by switching between
working forwards (from hypotheses and facts likely to be useful) and
backwards (from the goal) until the forwards and backwards approaches meet.
In the end, use whatever approach works for you.

A proof is developed with the Proof Assistant by working backwards, starting
with the theorem\index{theorem} to be proved, and assigning each unknown step
with a theorem or hypothesis until no more unknown steps remain.  The Proof
Assistant will not let you make an assignment unless it can be ``unified''
with the unknown step.  This means that a
substitution\index{substitution!variable}\index{variable substitution} of
variables exists that will make the assignment match the unknown step.  On the
other hand, in the middle of a proof, when working backwards, often more than
one unification\index{unification} (set of substitutions) is possible, since
there is not enough information available at that point to uniquely establish
it.  In this case you can tell Metamath which unification to choose, or you
can continue to assign unknown steps until enough information is available to
make the unification unique.

We will assume you have entered Metamath and read in the database as described
above.  The following dialog shows how the proof was developed.  For more
details on what some of the commands do, refer to Section~\ref{pfcommands}.
\index{\texttt{prove} command}

\begin{verbatim}
MM> prove th1
Entering the Proof Assistant.  Type HELP for help, EXIT
to exit.  You will be working on the proof of statement th1:
  $p |- t = t
Note:  The proof you are starting with is already complete.
MM-PA>
\end{verbatim}

The \verb/MM-PA>/ prompt means we are inside the Proof
Assistant.\index{Proof Assistant} Most of the regular Metamath commands
(\texttt{show statement}, etc.) are still available if you need them.

\begin{verbatim}
MM-PA> delete all
The entire proof was deleted.
\end{verbatim}

We have deleted the whole proof so we can start from scratch.

\begin{verbatim}
MM-PA> show new_proof/lemmon/all
1 ?              $? |- t = t
\end{verbatim}

The \texttt{show new{\char`\_}proof} command\index{\texttt{show
new{\char`\_}proof} command} is like \texttt{show proof} except that we
don't specify a statement; instead, the proof we're working on is
displayed.

\begin{verbatim}
MM-PA> assign 1 mp
To undo the assignment, DELETE STEP 5 and INITIALIZE, UNIFY
if needed.
3   min=?  $? |- $2
4   maj=?  $? |- ( $2 -> t = t )
\end{verbatim}

The \texttt{assign} command\index{\texttt{assign} command} above means
``assign step 1 with the statement whose label is \texttt{mp}.''  Note
that step renumbering will constantly occur as you assign steps in the
middle of a proof; in general all steps from the step you assign until
the end of the proof will get moved up.  In this case, what used to be
step 1 is now step 5, because the (partial) proof now has five steps:
the four hypotheses of the \texttt{mp} statement and the \texttt{mp}
statement itself.  Let's look at all the steps in our partial proof:

\begin{verbatim}
MM-PA> show new_proof/lemmon/all
1 ?              $? wff $2
2 ?              $? wff t = t
3 ?              $? |- $2
4 ?              $? |- ( $2 -> t = t )
5 1,2,3,4 mp     $a |- t = t
\end{verbatim}

The symbol \texttt{\$2} is a temporary variable\index{temporary
variable} that represents a symbol sequence not yet known.  In the final
proof, all temporary variables will be eliminated.  The general format
for a temporary variable is \texttt{\$} followed by an integer.  Note
that \texttt{\$} is not a legal character in a math symbol (see
Section~\ref{dollardollar}, p.~\pageref{dollardollar}), so there will
never be a naming conflict between real symbols and temporary variables.

Unknown steps 1 and 2 are constructions of the two wffs used by the
modus ponens rule.  As you will see at the end of this section, the
Proof Assistant\index{Proof Assistant} can usually figure these steps
out by itself, and we will not have to worry about them.  Therefore from
here on we will display only the ``essential'' hypotheses, i.e.\ those
steps that correspond to traditional formal proofs\index{formal proof}.

\begin{verbatim}
MM-PA> show new_proof/lemmon
3 ?              $? |- $2
4 ?              $? |- ( $2 -> t = t )
5 3,4 mp         $a |- t = t
\end{verbatim}

Unknown steps 3 and 4 are the ones we must focus on.  They correspond to the
minor and major premises of the modus ponens rule.  We will assign them as
follows.  Notice that because of the step renumbering that takes place
after an assignment, it is advantageous to assign unknown steps in reverse
order, because earlier steps will not get renumbered.

\begin{verbatim}
MM-PA> assign 4 mp
To undo the assignment, DELETE STEP 8 and INITIALIZE, UNIFY
if needed.
3   min=?  $? |- $2
6     min=?  $? |- $4
7     maj=?  $? |- ( $4 -> ( $2 -> t = t ) )
\end{verbatim}

We are now going to describe an obscure feature that you will probably
never use but should be aware of.  The Metamath language allows empty
symbol sequences to be substituted for variables, but in most formal
systems this feature is never used.  One of the few examples where is it
used is the MIU-system\index{MIU-system} described in
Appendix~\ref{MIU}.  But such systems are rare, and by default this
feature is turned off in the Proof Assistant.  (It is always allowed for
{\tt verify proof}.)  Let us turn it on and see what
happens.\index{\texttt{set empty{\char`\_}substitution} command}

\begin{verbatim}
MM-PA> set empty_substitution on
Substitutions with empty symbol sequences is now allowed.
\end{verbatim}

With this feature enabled, more unifications will be
ambiguous\index{ambiguous unification}\index{unification!ambiguous} in
the middle of a proof, because
substitution\index{substitution!variable}\index{variable substitution}
of variables with empty symbol sequences will become an additional
possibility.  Let's see what happens when we make our next assignment.

\begin{verbatim}
MM-PA> assign 3 a2
There are 2 possible unifications.  Please select the correct
    one or Q if you want to UNIFY later.
Unify:  |- $6
 with:  |- ( $9 + 0 ) = $9
Unification #1 of 2 (weight = 7):
  Replace "$6" with "( + 0 ) ="
  Replace "$9" with ""
  Accept (A), reject (R), or quit (Q) <A>? r
\end{verbatim}

The first choice presented is the wrong one.  If we had selected it,
temporary variable \texttt{\$6} would have been assigned a truncated
wff, and temporary variable \texttt{\$9} would have been assigned an
empty sequence (which is not allowed in our system).  With this choice,
eventually we would reach a point where we would get stuck because
we would end up with steps impossible to prove.  (You may want to
try it.)  We typed \texttt{r} to reject the choice.

\begin{verbatim}
Unification #2 of 2 (weight = 21):
  Replace "$6" with "( $9 + 0 ) = $9"
  Accept (A), reject (R), or quit (Q) <A>? q
To undo the assignment, DELETE STEP 4 and INITIALIZE, UNIFY
if needed.
 7     min=?  $? |- $8
 8     maj=?  $? |- ( $8 -> ( $6 -> t = t ) )
\end{verbatim}

The second choice is correct, and normally we would type \texttt{a}
to accept it.  But instead we typed \texttt{q} to show what will happen:
it will leave the step with an unknown unification, which can be
seen as follows:

\begin{verbatim}
MM-PA> show new_proof/not_unified
 4   min    $a |- $6
        =a2  = |- ( $9 + 0 ) = $9
\end{verbatim}

Later we can unify this with the \texttt{unify}
\texttt{all/interactive} command.

The important point to remember is that occasionally you will be
presented with several unification choices while entering a proof, when
the program determines that there is not enough information yet to make
an unambiguous choice automatically (and this can happen even with
\texttt{set empty{\char`\_}substitution} turned off).  Usually it is
obvious by inspection which choice is correct, since incorrect ones will
tend to be meaningless fragments of wffs.  In addition, the correct
choice will usually be the first one presented, unlike our example
above.

Enough of this digression.  Let us go back to the default setting.

\begin{verbatim}
MM-PA> set empty_substitution off
The ability to substitute empty expressions for variables
has been turned off.  Note that this may make the Proof
Assistant too restrictive in some cases.
\end{verbatim}

If we delete the proof, start over, and get to the point where
we digressed above, there will no longer be an ambiguous unification.

\begin{verbatim}
MM-PA> assign 3 a2
To undo the assignment, DELETE STEP 4 and INITIALIZE, UNIFY
if needed.
 7     min=?  $? |- $4
 8     maj=?  $? |- ( $4 -> ( ( $5 + 0 ) = $5 -> t = t ) )
\end{verbatim}

Let us look at our proof so far, and continue.

\begin{verbatim}
MM-PA> show new_proof/lemmon
 4 a2            $a |- ( $5 + 0 ) = $5
 7 ?             $? |- $4
 8 ?             $? |- ( $4 -> ( ( $5 + 0 ) = $5 -> t = t ) )
 9 7,8 mp        $a |- ( ( $5 + 0 ) = $5 -> t = t )
10 4,9 mp        $a |- t = t
MM-PA> assign 8 a1
To undo the assignment, DELETE STEP 11 and INITIALIZE, UNIFY
if needed.
 7     min=?  $? |- ( t + 0 ) = t
MM-PA> assign 7 a2
To undo the assignment, DELETE STEP 8 and INITIALIZE, UNIFY
if needed.
MM-PA> show new_proof/lemmon
 4 a2            $a |- ( t + 0 ) = t
 8 a2            $a |- ( t + 0 ) = t
12 a1            $a |- ( ( t + 0 ) = t -> ( ( t + 0 ) = t ->
                                                    t = t ) )
13 8,12 mp       $a |- ( ( t + 0 ) = t -> t = t )
14 4,13 mp       $a |- t = t
\end{verbatim}

Now all temporary variables and unknown steps have been eliminated from the
``essential'' part of the proof.  When this is achieved, the Proof
Assistant\index{Proof Assistant} can usually figure out the rest of the proof
automatically.  (Note that the \texttt{improve} command can occasionally be
useful for filling in essential steps as well, but it only tries to make use
of statements that introduce no new variables in their hypotheses, which is
not the case for \texttt{mp}. Also it will not try to improve steps containing
temporary variables.)  Let's look at the complete proof, then run
the \texttt{improve} command, then look at it again.

\begin{verbatim}
MM-PA> show new_proof/lemmon/all
 1 ?             $? wff ( t + 0 ) = t
 2 ?             $? wff t = t
 3 ?             $? term t
 4 3 a2          $a |- ( t + 0 ) = t
 5 ?             $? wff ( t + 0 ) = t
 6 ?             $? wff ( ( t + 0 ) = t -> t = t )
 7 ?             $? term t
 8 7 a2          $a |- ( t + 0 ) = t
 9 ?             $? term ( t + 0 )
10 ?             $? term t
11 ?             $? term t
12 9,10,11 a1    $a |- ( ( t + 0 ) = t -> ( ( t + 0 ) = t ->
                                                    t = t ) )
13 5,6,8,12 mp   $a |- ( ( t + 0 ) = t -> t = t )
14 1,2,4,13 mp   $a |- t = t
\end{verbatim}

\begin{verbatim}
MM-PA> improve all
A proof of length 1 was found for step 11.
A proof of length 1 was found for step 10.
A proof of length 3 was found for step 9.
A proof of length 1 was found for step 7.
A proof of length 9 was found for step 6.
A proof of length 5 was found for step 5.
A proof of length 1 was found for step 3.
A proof of length 3 was found for step 2.
A proof of length 5 was found for step 1.
Steps 1 and above have been renumbered.
CONGRATULATIONS!  The proof is complete.  Use SAVE
NEW_PROOF to save it.  Note:  The Proof Assistant does
not detect $d violations.  After saving the proof, you
should verify it with VERIFY PROOF.
\end{verbatim}

The \texttt{save new{\char`\_}proof} command\index{\texttt{save
new{\char`\_}proof} command} will save the proof in the database.  Here
we will just display it in a form that can be clipped out of a log file
and inserted manually into the database source file with a text
editor.\index{normal proof}\index{proof!normal}

\begin{verbatim}
MM-PA> show new_proof/normal
---------Clip out the proof below this line:
      tt tze tpl tt weq tt tt weq tt a2 tt tze tpl tt weq
      tt tze tpl tt weq tt tt weq wim tt a2 tt tze tpl tt
      tt a1 mp mp $.
---------The proof of 'th1' to clip out ends above this line.
\end{verbatim}

There is another proof format called ``compressed''\index{compressed
proof}\index{proof!compressed} that you will see in databases.  It is
not important to understand how it is encoded but only to recognize it
when you see it.  Its only purpose is to reduce storage requirements for
large proofs.  A compressed proof can always be converted to a normal
one and vice-versa, and the Metamath \texttt{show proof}
commands\index{\texttt{show proof} command} work equally well with
compressed proofs.  The compressed proof format is described in
Appendix~\ref{compressed}.

\begin{verbatim}
MM-PA> show new_proof/compressed
---------Clip out the proof below this line:
      ( tze tpl weq a2 wim a1 mp ) ABCZADZAADZAEZJJKFLIA
      AGHH $.
---------The proof of 'th1' to clip out ends above this line.
\end{verbatim}

Now we will exit the Proof Assistant.  Since we made changes to the proof,
it will warn us that we have not saved it.  In this case, we don't care.

\begin{verbatim}
MM-PA> exit
Warning:  You have not saved changes to the proof.
Do you want to EXIT anyway (Y, N) <N>? y
Exiting the Proof Assistant.
Type EXIT again to exit Metamath.
\end{verbatim}

The Proof Assistant\index{Proof Assistant} has several other commands
that can help you while creating proofs.  See Section~\ref{pfcommands}
for a list of them.

A command that is often useful is \texttt{minimize{\char`\_}with
*/brief}, which tries to shorten the proof.  It can make the process
more efficient by letting you write a somewhat ``sloppy'' proof then
clean up some of the fine details of optimization for you (although it
can't perform miracles such as restructuring the overall proof).

\section{A Note About Editing a Data\-base File}

Once your source file contains proofs, there are some restrictions on
how you can edit it so that the proofs remain valid.  Pay particular
attention to these rules, since otherwise you can lose a lot of work.
It is a good idea to periodically verify all proofs with \texttt{verify
proof *} to ensure their integrity.

If your file contains only normal (as opposed to compressed) proofs, the
main rule is that you may not change the order of the mandatory
hypotheses\index{mandatory hypothesis} of any statement referenced in a
later proof.  For example, if you swap the order of the major and minor
premise in the modus ponens rule, all proofs making use of that rule
will become incorrect.  The \texttt{show statement}
command\index{\texttt{show statement} command} will show you the
mandatory hypotheses of a statement and their order.

If a statement has a compressed proof, you also must not change the
order of {\em its} mandatory hypotheses.  The compressed proof format
makes use of this information as part of the compression technique.
Note that swapping the names of two variables in a theorem will change
the order of its mandatory hypotheses.

The safest way to edit a statement, say \texttt{mytheorem}, is to
duplicate it then rename the original to \texttt{mytheoremOLD}
throughout the database.  Once the edited version is re-proved, all
statements referencing \texttt{mytheoremOLD} can be updated in the Proof
Assistant using \texttt{minimize{\char`\_}with
mytheorem
/allow{\char`\_}growth}.\index{\texttt{minimize{\char`\_}with} command}
% 3/10/07 Note: line-breaking the above results in duplicate index entries

\chapter{Abstract Mathematics Revealed}\label{fol}

\section{Logic and Set Theory}\label{logicandsettheory}

\begin{quote}
  {\em Set theory can be viewed as a form of exact theology.}
  \flushright\sc  Rudy Rucker\footnote{\cite{Barrow}, p.~31.}\\
\end{quote}\index{Rucker, Rudy}

Despite its seeming complexity, all of standard mathematics, no matter how
deep or abstract, can amazingly enough be derived from a relatively small set
of axioms\index{axiom} or first principles. The development of these axioms is
among the most impressive and important accomplishments of mathematics in the
20th century. Ultimately, these axioms can be broken down into a set of rules
for manipulating symbols that any technically oriented person can follow.

We will not spend much time trying to convey a deep, higher-level
understanding of the meaning of the axioms. This kind of understanding
requires some mathematical sophistication as well as an understanding of the
philosophy underlying the foundations of mathematics and typically develops
over time as you work with mathematics.  Our goal, instead, is to give you the
immediate ability to follow how theorems\index{theorem} are derived from the
axioms and from other theorems.  This will be similar to learning the syntax
of a computer language, which lets you follow the details in a program but
does not necessarily give you the ability to write non-trivial programs on
your own, an ability that comes with practice. For now don't be alarmed by
abstract-sounding names of the axioms; just focus on the rules for
manipulating the symbols, which follow the simple conventions of the
Metamath\index{Metamath} language.

The axioms that underlie all of standard mathematics consist of axioms of logic
and axioms of set theory. The axioms of logic are divided into two
subcategories, propositional calculus\index{propositional calculus} (sometimes
called sentential logic\index{sentential logic}) and predicate calculus
(sometimes called first-order logic\index{first-order logic}\index{quantifier
theory}\index{predicate calculus} or quantifier theory).  Propositional
calculus is a prerequisite for predicate calculus, and predicate calculus is a
prerequisite for set theory.  The version of set theory most commonly used is
Zermelo--Fraenkel set theory\index{Zermelo--Fraenkel set theory}\index{set theory}
with the axiom of choice,
often abbreviated as ZFC\index{ZFC}.

Here in a nutshell is what the axioms are all about in an informal way. The
connection between this description and symbols we will show you won't be
immediately apparent and in principle needn't ever be.  Our description just
tries to summarize what mathematicians think about when they work with the
axioms.

Logic is a set of rules that allow us determine truths given other truths.
Put another way,
logic is more or less the translation of what we would consider common sense
into a rigorous set of axioms.\index{axioms of logic}  Suppose $\varphi$,
$\psi$, and $\chi$ (the Greek letters phi, psi, and chi) represent statements
that are either true or false, and $x$ is a variable\index{variable!in predicate
calculus} ranging over some group of mathematical objects (sets, integers,
real numbers, etc.). In mathematics, a ``statement'' really means a formula,
and $\psi$ could be for example ``$x = 2$.''
Propositional calculus\index{propositional calculus}
allows us to use variables that are either true or false
and make deductions such as
``if $\varphi$ implies $\psi$ and $\psi$ implies $\chi$, then $\varphi$
implies $\chi$.''
Predicate calculus\index{predicate calculus}
extends propositional calculus by also allowing us
to discuss statements about objects (not just true and false values), including
statements about ``all'' or ``at least one'' object.
For example, predicate calculus allows to say,
``if $\varphi$ is true for all $x$, then $\varphi$ is true for some $x$.''
The logic used in \texttt{set.mm} is standard classical logic
(as opposed to other logic systems like intuitionistic logic).

Set theory\index{set theory} has to do with the manipulation of objects and
collections of objects, specifically the abstract, imaginary objects that
mathematics deals with, such as numbers. Everything that is claimed to exist
in mathematics is considered to be a set.  A set called the empty
set\index{empty set} contains nothing.  We represent the empty set by
$\varnothing$.  Many sets can be built up from the empty set.  There is a set
represented by $\{\varnothing\}$ that contains the empty set, another set
represented by $\{\varnothing,\{\varnothing\}\}$ that contains this set as
well as the empty set, another set represented by $\{\{\varnothing\}\}$ that
contains just the set that contains the empty set, and so on ad infinitum. All
mathematical objects, no matter how complex, are defined as being identical to
certain sets: the integer\index{integer} 0 is defined as the empty set, the
integer 1 is defined as $\{\varnothing\}$, the integer 2 is defined as
$\{\varnothing,\{\varnothing\}\}$.  (How these definitions were chosen doesn't
matter now, but the idea behind it is that these sets have the properties we
expect of integers once suitable operations are defined.)  Mathematical
operations, such as addition, are defined in terms of operations on
sets---their union\index{set union}, intersection\index{set intersection}, and
so on---operations you may have used in elementary school when you worked
with groups of apples and oranges.

With a leap of faith, the axioms also postulate the existence of infinite
sets\index{infinite set}, such as the set of all non-negative integers ($0, 1,
2,\ldots$, also called ``natural numbers''\index{natural number}).  This set
can't be represented with the brace notation\index{brace notation} we just
showed you, but requires a more complicated notation called ``class
abstraction.''\index{class abstraction}\index{abstraction class}  For
example, the infinite set $\{ x |
\mbox{``$x$ is a natural number''} \} $ means the ``set of all objects $x$
such that $x$ is a natural number'' i.e.\ the set of natural numbers; here,
``$x$ is a natural number'' is a rather complicated formula when broken down
into the primitive symbols.\label{expandom}\footnote{The statement ``$x$ is a
natural number'' is formally expressed as ``$x \in \omega$,'' where $\in$
(stylized epsilon) means ``is in'' or ``is an element of'' and $\omega$
(omega) means ``the set of natural numbers.''  When ``$x\in\omega$'' is
completely expanded in terms of the primitive symbols of set theory, the
result is  $\lnot$ $($ $\lnot$ $($ $\forall$ $z$ $($ $\lnot$ $\forall$ $w$ $($
$z$ $\in$ $w$ $\rightarrow$ $\lnot$ $w$ $\in$ $x$ $)$ $\rightarrow$ $z$ $\in$
$x$ $)$ $\rightarrow$ $($ $\forall$ $z$ $($ $\lnot$ $($ $\forall$ $w$ $($ $w$
$\in$ $z$ $\rightarrow$ $w$ $\in$ $x$ $)$ $\rightarrow$ $\forall$ $w$ $\lnot$
$w$ $\in$ $z$ $)$ $\rightarrow$ $\lnot$ $\forall$ $w$ $($ $w$ $\in$ $z$
$\rightarrow$ $\lnot$ $\forall$ $v$ $($ $v$ $\in$ $z$ $\rightarrow$ $\lnot$
$v$ $\in$ $w$ $)$ $)$ $)$ $\rightarrow$ $\lnot$ $\forall$ $z$ $\forall$ $w$
$($ $\lnot$ $($ $z$ $\in$ $x$ $\rightarrow$ $\lnot$ $w$ $\in$ $x$ $)$
$\rightarrow$ $($ $\lnot$ $z$ $\in$ $w$ $\rightarrow$ $($ $\lnot$ $z$ $=$ $w$
$\rightarrow$ $w$ $\in$ $z$ $)$ $)$ $)$ $)$ $)$ $\rightarrow$ $\lnot$
$\forall$ $y$ $($ $\lnot$ $($ $\lnot$ $($ $\forall$ $z$ $($ $\lnot$ $\forall$
$w$ $($ $z$ $\in$ $w$ $\rightarrow$ $\lnot$ $w$ $\in$ $y$ $)$ $\rightarrow$
$z$ $\in$ $y$ $)$ $\rightarrow$ $($ $\forall$ $z$ $($ $\lnot$ $($ $\forall$
$w$ $($ $w$ $\in$ $z$ $\rightarrow$ $w$ $\in$ $y$ $)$ $\rightarrow$ $\forall$
$w$ $\lnot$ $w$ $\in$ $z$ $)$ $\rightarrow$ $\lnot$ $\forall$ $w$ $($ $w$
$\in$ $z$ $\rightarrow$ $\lnot$ $\forall$ $v$ $($ $v$ $\in$ $z$ $\rightarrow$
$\lnot$ $v$ $\in$ $w$ $)$ $)$ $)$ $\rightarrow$ $\lnot$ $\forall$ $z$
$\forall$ $w$ $($ $\lnot$ $($ $z$ $\in$ $y$ $\rightarrow$ $\lnot$ $w$ $\in$
$y$ $)$ $\rightarrow$ $($ $\lnot$ $z$ $\in$ $w$ $\rightarrow$ $($ $\lnot$ $z$
$=$ $w$ $\rightarrow$ $w$ $\in$ $z$ $)$ $)$ $)$ $)$ $\rightarrow$ $($
$\forall$ $z$ $\lnot$ $z$ $\in$ $y$ $\rightarrow$ $\lnot$ $\forall$ $w$ $($
$\lnot$ $($ $w$ $\in$ $y$ $\rightarrow$ $\lnot$ $\forall$ $z$ $($ $w$ $\in$
$z$ $\rightarrow$ $\lnot$ $z$ $\in$ $y$ $)$ $)$ $\rightarrow$ $\lnot$ $($
$\lnot$ $\forall$ $z$ $($ $w$ $\in$ $z$ $\rightarrow$ $\lnot$ $z$ $\in$ $y$
$)$ $\rightarrow$ $w$ $\in$ $y$ $)$ $)$ $)$ $)$ $\rightarrow$ $x$ $\in$ $y$
$)$ $)$ $)$. Section~\ref{hierarchy} shows the hierarchy of definitions that
leads up to this expression.}\index{stylized epsilon ($\in$)}\index{omega
($\omega$)}  Actually, the primitive symbols don't even include the brace
notation.  The brace notation is a high-level definition, which you can find in
Section~\ref{hierarchy}.

Interestingly, the arithmetic of integers\index{integer} and
rationals\index{rational number} can be developed without appealing to the
existence of an infinite set, whereas the arithmetic of real
numbers\index{real number} requires it.

Each variable\index{variable!in set theory} in the axioms of set theory
represents an arbitrary set, and the axioms specify the legal kinds of things
you can do with these variables at a very primitive level.

Now, you may think that numbers and arithmetic are a lot more intuitive and
fundamental than sets and therefore should be the foundation of mathematics.
What is really the case is that you've dealt with numbers all your life and
are comfortable with a few rules for manipulating them such as addition and
multiplication.  Those rules only cover a small portion of what can be done
with numbers and only a very tiny fraction of the rest of mathematics.  If you
look at any elementary book on number theory, you will quickly become lost if
these are the only rules that you know.  Even though such books may present a
list of ``axioms''\index{axiom} for arithmetic, the ability to use the axioms
and to understand proofs of theorems\index{theorem} (facts) about numbers
requires an implicit mathematical talent that frustrates many people
from studying abstract mathematics.  The kind of mathematics that most people
know limits them to the practical, everyday usage of blindly manipulating
numbers and formulas, without any understanding of why those rules are correct
nor any ability to go any further.  For example, do you know why multiplying
two negative numbers yields a positive number?  Starting with set theory, you
will also start off blindly manipulating symbols according to the rules we give
you, but with the advantage that these rules will allow you, in principle, to
access {\em all} of mathematics, not just a tiny part of it.

Of course, concrete examples are often helpful in the learning process. For
example, you can verify that $2\cdot 3=3 \cdot 2$ by actually grouping
objects and can easily ``see'' how it generalizes to $x\cdot y = y\cdot x$,
even though you might not be able to rigorously prove it.  Similarly, in set
theory it can be helpful to understand how the axioms of set theory apply to
(and are correct for) small finite collections of objects.  You should be aware
that in set theory intuition can be misleading for infinite collections, and
rigorous proofs become more important.  For example, while $x\cdot y = y\cdot
x$ is correct for finite ordinals (which are the natural numbers), it is not
usually true for infinite ordinals.

\section{The Axioms for All of Mathematics}

In this section\index{axioms for mathematics}, we will show you the axioms
for all of standard mathematics (i.e.\ logic and set theory) as they are
traditionally presented.  The traditional presentation is useful for someone
with the mathematical experience needed to correctly manipulate high-level
abstract concepts.  For someone without this talent, knowing how to actually
make use of these axioms can be difficult.  The purpose of this section is to
allow you to see how the version of the axioms used in the standard
Metamath\index{Metamath} database \texttt{set.mm}\index{set
theory database (\texttt{set.mm})} relates to  the typical version
in textbooks, and also to give you an informal feel for them.

\subsection{Propositional Calculus}

Propositional calculus\index{propositional calculus} concerns itself with
statements that can be interpreted as either true or false.  Some examples of
statements (outside of mathematics) that are either true or false are ``It is
raining today'' and ``The United States has a female president.'' In
mathematics, as we mentioned, statements are really formulas.

In propositional calculus, we don't care what the statements are.  We also
treat a logical combination of statements, such as ``It is raining today and
the United States has a female president,'' no differently from a single
statement.  Statements and their combinations are called well-formed formulas
(wffs)\index{well-formed formula (wff)}.  We define wffs only in terms of
other wffs and don't define what a ``starting'' wff is.  As is common practice
in the literature, we use Greek letters to represent wffs.

Specifically, suppose $\varphi$ and $\psi$ are wffs.  Then the combinations
$\varphi\rightarrow\psi$ (``$\varphi$ implies $\psi$,'' also read ``if
$\varphi$ then $\psi$'')\index{implication ($\rightarrow$)} and $\lnot\varphi$
(``not $\varphi$'')\index{negation ($\lnot$)} are also wffs.

The three axioms of propositional calculus\index{axioms of propositional
calculus} are all wffs of the following form:\footnote{A remarkable result of
C.~A.~Meredith\index{Meredith, C. A.} squeezes these three axioms into the
single axiom $((((\varphi\rightarrow \psi)\rightarrow(\neg \chi\rightarrow\neg
\theta))\rightarrow \chi )\rightarrow \tau)\rightarrow((\tau\rightarrow
\varphi)\rightarrow(\theta\rightarrow \varphi))$ \cite{CAMeredith},
which is believed to be the shortest possible.}
\begin{center}
     $\varphi\rightarrow(\psi\rightarrow \varphi)$\\

     $(\varphi\rightarrow (\psi\rightarrow \chi))\rightarrow
((\varphi\rightarrow  \psi)\rightarrow (\varphi\rightarrow \chi))$\\

     $(\neg \varphi\rightarrow \neg\psi)\rightarrow (\psi\rightarrow
\varphi)$
\end{center}

These three axioms are widely used.
They are attributed to Jan {\L}ukasiewicz
(pronounced woo-kah-SHAY-vitch) and was popularized by Alonzo Church,
who called it system P2. (Thanks to Ted Ulrich for this information.)

There are an infinite number of axioms, one for each possible
wff\index{well-formed formula (wff)} of the above form.  (For this reason,
axioms such as the above are often called ``axiom schemes.''\index{axiom
scheme})  Each Greek letter in the axioms may be substituted with a more
complex wff to result in another axiom.  For example, substituting
$\neg(\varphi\rightarrow\chi)$ for $\varphi$ in the first axiom yields
$\neg(\varphi\rightarrow\chi)\rightarrow(\psi\rightarrow
\neg(\varphi\rightarrow\chi))$, which is still an axiom.

To deduce new true statements (theorems\index{theorem}) from the axioms, a
rule\index{rule} called ``modus ponens''\index{modus ponens} is used.  This
rule states that if the wff $\varphi$ is an axiom or a theorem, and the wff
$\varphi\rightarrow\psi$ is an axiom or a theorem, then the wff $\psi$ is also
a theorem\index{theorem}.

As a non-mathematical example of modus ponens, suppose we have proved (or
taken as an axiom) ``Bob is a man'' and separately have proved (or taken as
an axiom) ``If Bob is a man, then Bob is a human.''  Using the rule of modus
ponens, we can logically deduce, ``Bob is a human.''

From Metamath's\index{Metamath} point of view, the axioms and the rule of
modus ponens just define a mechanical means for deducing new true statements
from existing true statements, and that is the complete content of
propositional calculus as far as Metamath is concerned.  You can read a logic
textbook to gain a better understanding of their meaning, or you can just let
their meaning slowly become apparent to you after you use them for a while.

It is actually rather easy to check to see if a formula is a theorem of
propositional calculus.  Theorems of propositional calculus are also called
``tautologies.''\index{tautology}  The technique to check whether a formula is
a tautology is called the ``truth table method,''\index{truth table} and it
works like this.  A wff $\varphi\rightarrow\psi$ is false whenever $\varphi$ is true
and $\psi$ is false.  Otherwise it is true.  A wff $\lnot\varphi$ is false
whenever $\varphi$ is true and false otherwise. To verify a tautology such as
$\varphi\rightarrow(\psi\rightarrow \varphi)$, you break it down into sub-wffs and
construct a truth table that accounts for all possible combinations of true
and false assigned to the wff metavariables:
\begin{center}\begin{tabular}{|c|c|c|c|}\hline
\mbox{$\varphi$} & \mbox{$\psi$} & \mbox{$\psi\rightarrow\varphi$}
    & \mbox{$\varphi\rightarrow(\psi\rightarrow \varphi)$} \\ \hline \hline
              T   &  T    &      T       &        T    \\ \hline
              T   &  F    &      T       &        T    \\ \hline
              F   &  T    &      F       &        T    \\ \hline
              F   &  F    &      T       &        T    \\ \hline
\end{tabular}\end{center}
If all entries in the last column are true, the formula is a tautology.

Now, the truth table method doesn't tell you how to prove the tautology from
the axioms, but only that a proof exists.  Finding an actual proof (especially
one that is short and elegant) can be challenging.  Methods do exist for
automatically generating proofs in propositional calculus, but the proofs that
result can sometimes be very long.  In the Metamath \texttt{set.mm}\index{set
theory database (\texttt{set.mm})} database, most
or all proofs were created manually.

Section \ref{metadefprop} discusses various definitions
that make propositional calculus easier to use.
For example, we define:

\begin{itemize}
\item $\varphi \vee \psi$
  is true if either $\varphi$ or $\psi$ (or both) are true
  (this is disjunction\index{disjunction ($\vee$)}
  aka logical {\sc or}\index{logical {\sc or} ($\vee$)}).

\item $\varphi \wedge \psi$
  is true if both $\varphi$ and $\psi$ are true
  (this is conjunction\index{conjunction ($\wedge$)}
  aka logical {\sc and}\index{logical {\sc and} ($\wedge$)}).

\item $\varphi \leftrightarrow \psi$
  is true if $\varphi$ and $\psi$ have the same value, that is,
  they are both true or both false
  (this is the biconditional\index{biconditional ($\leftrightarrow$)}).
\end{itemize}

\subsection{Predicate Calculus}

Predicate calculus\index{predicate calculus} introduces the concept of
``individual variables,''\index{variable!in predicate calculus}\index{individual
variable} which
we will usually just call ``variables.''
These variables can represent something other than true or false (wffs),
and will always represent sets when we get to set theory.  There are also
three new symbols $\forall$\index{universal quantifier ($\forall$)},
$=$\index{equality ($=$)}, and $\in$\index{stylized epsilon ($\in$)},
read ``for all,'' ``equals,'' and ``is an element of''
respectively.  We will represent variables with the letters $x$, $y$, $z$, and
$w$, as is common practice in the literature.
For example, $\forall x \varphi$ means ``for all possible values of
$x$, $\varphi$ is true.''

In predicate calculus, we extend the definition of a wff\index{well-formed
formula (wff)}.  If $\varphi$ is a wff and $x$ and $y$ are variables, then
$\forall x \, \varphi$, $x=y$, and $x\in y$ are wffs. Note that these three new
types of wffs can be considered ``starting'' wffs from which we can build
other wffs with $\rightarrow$ and $\neg$ .  The concept of a starting wff was
absent in propositional calculus.  But starting wff or not, all we are really
concerned with is whether our wffs are correctly constructed according to
these mechanical rules.

A quick aside:
To prevent confusion, it might be best at this point to think of the variables
of Metamath\index{Metamath} as ``metavariables,''\index{metavariable} because
they are not quite the same as the variables we are introducing here.  A
(meta)variable in Metamath can be a wff or an individual variable, as well
as many other things; in general, it represents a kind of place holder for an
unspecified sequence of math symbols\index{math symbol}.

Unlike propositional calculus, no decision procedure\index{decision procedure}
analogous to the truth table method exists (nor theoretically can exist) that
will definitely determine whether a formula is a theorem of predicate
calculus.  Much of the work in the field of automated theorem
proving\index{automated theorem proving} has been dedicated to coming up with
clever heuristics for proving theorems of predicate calculus, but they can
never be guaranteed to work always.

Section \ref{metadefpred} discusses various definitions
that make predicate calculus easier to use.
For example, we define
$\exists x \varphi$ to mean
``there exists at least one possible value of $x$ where $\varphi$ is true.''

We now turn to looking at how predicate calculus can be formally
represented.

\subsubsection{Common Axioms}

There is a new rule of inference in predicate calculus:  if $\varphi$ is
an axiom or a theorem, then $\forall x \,\varphi$ is also a
theorem\index{theorem}.  This is called the rule of
``generalization.''\index{rule of generalization}
This is easily represented in Metamath.

In standard texts of logic, there are often two axioms of predicate
calculus\index{axioms of predicate calculus}:
\begin{center}
  $\forall x \,\varphi ( x ) \rightarrow \varphi ( y )$,
      where ``$y$ is properly substituted for $x$.''\\
  $\forall x ( \varphi \rightarrow \psi )\rightarrow ( \varphi \rightarrow
    \forall x\, \psi )$,
    where ``$x$ is not free in $\varphi$.''
\end{center}

Now at first glance, this seems simple:  just two axioms.  However,
conditional clauses are attached to each axiom describing requirements that
may seem puzzling to you.  In addition, the first axiom puts a variable symbol
in parentheses after each wff, seemingly violating our definition of a
wff\index{well-formed formula (wff)}; this is just an informal way of
referring to some arbitrary variable that may occur in the wff.  The
conditional clauses do, of course, have a precise meaning, but as it turns out
the precise meaning is somewhat complicated and awkward to formalize in a
way that a computer can handle easily.  Unlike propositional calculus, a
certain amount of mathematical sophistication and practice is needed to be
able to easily grasp and manipulate these concepts correctly.

Predicate calculus may be presented with or without axioms for
equality\index{axioms of equality}\index{equality ($=$)}. We will require the
axioms of equality as a prerequisite for the version of set theory we will
use.  The axioms for equality, when included, are often represented using these
two axioms:
\begin{center}
$x=x$\\ \ \\
$x=y\rightarrow (\varphi(x,x)\rightarrow\varphi(x,y))$ where ``$\varphi(x,y)$
   arises from $\varphi(x,x)$ by replacing some, but not necessarily all,
   free\index{free variable}
   occurrences of $x$ by $y$,\\ provided that $y$ is free for $x$
   in $\varphi(x,x)$.'' \end{center}
% (Mendelson p. 95)
The first equality axiom is simple, but again,
the condition on the second one is
somewhat awkward to implement on a computer.

\subsubsection{Tarski System S2}

Of course, we are not the first to notice the complications of these
predicate calculus axioms when being rigorous.

Well-known logician Alfred Tarski published in 1965
a system he called system S2\cite[p.~77]{Tarski1965}.
Tarski's system is \textit{exactly equivalent} to the traditional textbook
formalization, but (by clever use of equality axioms) it eliminates the
latter's primitive notions of ``proper substitution'' and ``free variable,''
replacing them with direct substitution and the notion of a variable
not occurring in a formula (which we express with distinct variable
constraints).

In advocating his system, Tarski wrote, ``The relatively complicated
character of [free variables and proper substitution] is a source
of certain inconveniences of both practical and theoretical nature;
this is clearly experienced both in teaching an elementary course of
mathematical logic and in formalizing the syntax of predicate logic for
some theoretical purposes''\cite[p.~61]{Tarski1965}\index{Tarski, Alfred}.

\subsubsection{Developing a Metamath Representation}

The standard textbook axioms of predicate calculus are somewhat
cumbersome to implement on a computer because of the complex notions of
``free variable''\index{free variable} and ``proper
substitution.''\index{proper substitution}\index{substitution!proper}
While it is possible to use the Metamath\index{Metamath} language to
implement these concepts, we have chosen not to implement them
as primitive constructs in the
\texttt{set.mm} set theory database.  Instead, we have eliminated them
within the axioms
by carefully crafting the axioms so as to avoid them,
building on Tarski's system S2.  This makes it
easy for a beginner to follow the steps in a proof without knowing any
advanced concepts other than the simple concept of
replacing\index{substitution!variable}\index{variable substitution}
variables with expressions.

In order to develop the concepts of free variable and proper
substitution from the axioms, we use an additional
Metamath statement type called ``disjoint variable
restriction''\index{disjoint variables} that we have not encountered
before.  In the context of the axioms, the statement \texttt{\$d} $ x\,
y$\index{\texttt{\$d} statement} simply means that $x$ and $y$ must be
distinct\index{distinct variables}, i.e.\ they may not be simultaneously
substituted\index{substitution!variable}\index{variable substitution}
with the same variable.  The statement \texttt{\$d} $ x\, \varphi$ means
variable $x$ must not occur in wff $\varphi$.  For the precise
definition of \texttt{\$d}, see Section~\ref{dollard}.

\subsubsection{Metamath representation}

The Metamath axiom system for predicate calculus
defined in set.mm uses Tarski's system S2.
As noted above, this has a different representation
than the traditional textbook formalization,
but it is \textit{exactly equivalent} to the textbook formalization,
and it is \textit{much} easier to work with.
This is reproduced as system S3 in Section 6 of
Megill's formalization \cite{Megill}\index{Megill, Norman}.

There is one exception, Tarski's axiom of existence,
which we label as axiom ax-6.
In the case of ax-6, Tarski's version is weaker because it includes a
distinct variable proviso. If we wish, we can also weaken our version
in this way and still have a metalogically complete system. Theorem
ax6 shows this by deriving, in the presence of the other axioms, our
ax-6 from Tarski's weaker version ax6v. However, we chose the stronger
version for our system because it is simpler to state and easier to use.

Tarski's system was designed for proving specific theorems rather than
more general theorem schemes. However, theorem schemes are much more
efficient than specific theorems for building a body of mathematical
knowledge, since they can be reused with different instances as
needed. While Tarski does derive some theorem schemes from his axioms,
their proofs require concepts that are ``outside'' of the system, such as
induction on formula length. The verification of such proofs is difficult
to automate in a proof verifier. (Specifically, Tarski treats the formulas
of his system as set-theoretical objects. In order to verify the proofs
of his theorem schemes, a proof verifier would need a significant amount
of set theory built into it.)

The Metamath axiom system for predicate calculus extends
Tarski's system to eliminate this difficulty. The additional
``auxilliary'' axiom
schemes (as we will call them in this section; see below) endow Tarski's
system with a nice property we call
metalogical completeness \cite[Remark 9.6]{Megill}\index{Megill, Norman}.
As a result, we can prove any theorem scheme
expressable in the ``simple metalogic'' of Tarski's system by using
only Metamath's direct substitution rule applied to the axiom system
(and no other metalogical or set-theoretical notions ``outside'' of the
system). Simple metalogic consists of schemes containing wff metavariables
(with no arguments) and/or set (also called ``individual'') metavariables,
accompanied by optional provisos each stating that two specified set
metavariables must be distinct or that a specified set metavariable may
not occur in a specified wff metavariable. Metamath's logic and set theory
axiom and rule schemes are all examples of simple metalogic. The schemes
of traditional predicate calculus with equality are examples which are
not simple metalogic, because they use wff metavariables with arguments
and have ``free for'' and ``not free in'' side conditions.

A rigorous justification for this system, using an older but
exactly equivalent set of axioms, can be
found in \cite{Megill}\index{Megill, Norman}.

This allows us to
take a different approach in the Metamath\index{Metamath} database
\texttt{set.mm}\index{set theory database (\texttt{set.mm})}.  We do not
directly use the primitive notions of ``free variable''\index{free variable}
and ``proper substitution''\index{proper
substitution}\index{substitution!proper} at all as primitive constructs.
Instead, we use a set
of axioms that are almost as simple to manipulate as those of
propositional calculus.  Our axiom system avoids complex primitive
notions by effectively embedding the complexity into the axioms
themselves.  As a result, we will end up with a larger number of axioms,
but they are ideally suited for a computer language such as Metamath.
(Section~\ref{metaaxioms} shows these axioms.)

We will not elaborate further
on the ``free variable'' and ``proper substitution''
concepts here.  You may consult
\cite[ch.\ 3--4]{Hamilton}\index{Hamilton, Alan G.} (as well as
many other books) for a precise explanation
of these concepts.  If you intend to do serious mathematical work, it is wise
to become familiar with the traditional textbook approach; even though the
concepts embedded in their axioms require a higher level of sophistication,
they can be more practical to deal with on an everyday, informal basis.  Even
if you are just developing Metamath proofs, familiarity with the traditional
approach can help you arrive at a proof outline much faster, which you can
then convert to the detail required by Metamath.

We do develop proper substitution rules later on, but in set.mm
they are defined as derived constructs; they are not primitives.

You should also note that our system of predicate calculus is specifically
tailored for set theory; thus there are only two specific predicates $=$ and
$\in$ and no functions\index{function!in predicate calculus}
or constants\index{constant!in predicate calculus} unlike more general systems.
We later add these.

\subsection{Set Theory}

Traditional Zermelo--Fraenkel set theory\index{Zermelo--Fraenkel set
theory}\index{set theory} with the Axiom of Choice
has 10 axioms, which can be expressed in the
language of predicate calculus.  In this section, we will list only the
names and brief English descriptions of these axioms, since we will give
you the precise formulas used by the Metamath\index{Metamath} set theory
database \texttt{set.mm} later on.

In the descriptions of the axioms, we assume that $x$, $y$, $z$, $w$, and $v$
represent sets.  These are the same as the variables\index{variable!in set
theory} in our predicate calculus system above, except that now we informally
think of the variables as ranging over sets.  Note that the terms
``object,''\index{object} ``set,''\index{set} ``element,''\index{element}
``collection,''\index{collection} and ``family''\index{family} are synonymous,
as are ``is an element of,'' ``is a member of,''\index{member} ``is contained
in,'' and ``belongs to.''  The different terms are used for convenience; for
example, ``a collection of sets'' is less confusing than ``a set of sets.''
A set $x$ is said to be a ``subset''\index{subset} of $y$ if every element of
$x$ is also an element of $y$; we also say $x$ is ``included in''
$y$.

The axioms are very general and apply to almost any conceivable mathematical
object, and this level of abstraction can be overwhelming at first.  To gain an
intuitive feel, it can be helpful to draw a picture illustrating the concept;
for example, a circle containing dots could represent a collection of sets,
and a smaller circle drawn inside the circle could represent a subset.
Overlapping circles can illustrate intersection and union.  Circles that
illustrate the concepts of set theory are frequently used in elementary
textbooks and are called Venn diagrams\index{Venn diagram}.\index{axioms of
set theory}

1. Axiom of Extensionality:  Two sets are identical if they contain the same
   elements.\index{Axiom of Extensionality}

2. Axiom of Pairing:  The set $\{ x , y \}$ exists.\index{Axiom of Pairing}

3. Axiom of Power Sets:  The power set of a set (the collection of all of
   its subsets) exists.  For example, the power set of $\{x,y\}$ is
   $\{\varnothing,\{x\},\{y\},\{x,y\}\}$ and it exists.\index{Axiom
of Power Sets}

4. Axiom of the Null Set:  The empty set $\varnothing$ exists.\index{Axiom of
the Null Set}

5. Axiom of Union:  The union of a set (the set containing the elements of
   its members) exists.  For example, the union of $\{\{x,y\},\{z\}\}$ is
 $\{x,y,z\}$ and
   it exists.\index{Axiom of Union}

6. Axiom of Regularity:  Roughly, no set can contain itself, nor can there
   be membership ``loops,'' such as a set being an
   element of one of its members.\index{Axiom of Regularity}

7. Axiom of Infinity:  An infinite set exists.  An example of an infinite
   set is the set of all
   integers.\index{Axiom of Infinity}

8. Axiom of Separation:  The set exists that is obtained by restricting $x$
   with some property.  For example, if the set of all integers exists,
   then the set of all even integers exists.\index{Axiom of Separation}

9. Axiom of Replacement:  The range of a function whose domain is restricted
   to the elements of a set $x$, is also a set.  For example, there
   is a function
   from integers (the function's domain) to their squares (its
   range).  If we
   restrict the domain to even integers, its range will become the set of
   squares of even integers, so this axiom asserts that the set of
    squares of even numbers exists.  Technical note:  In general, the
   ``function'' need not be a set but can be a proper class.
   \index{Axiom of Replacement}

10. Axiom of Choice:  Let $x$ be a set whose members are pairwise
  disjoint\index{disjoint sets} (i.e,
  whose members contain no elements in common).  Then there exists another
  set containing one element from each member of $x$.  For
  example, if $x$ is
  $\{\{y,z\},\{w,v\}\}$, where $y$, $z$, $w$, and $v$ are
  different sets, then a set such as $\{z,w\}$
  exists (but the axiom doesn't tell
  us which one).  (Actually the Axiom
  of Choice is redundant if the set $x$, as in this example, has a finite
  number of elements.)\index{Axiom of Choice}

The Axiom of Choice is usually considered an extension of ZF set theory rather
than a proper part of it.  It is sometimes considered philosophically
controversial because it specifies the existence of a set without specifying
what the set is. Constructive logics, including intuitionistic logic,
do not accept the axiom of choice.
Since there is some lingering controversy, we often prefer proofs that do
not use the axiom of choice (where there is a known alternative), and
in some cases we will use weaker axioms than the full axiom of choice.
That said, the axiom of choice is a powerful and widely-accepted tool,
so we do use it when needed.
ZF set theory that includes the Axiom of Choice is
called Zermelo--Fraenkel set theory with choice (ZFC\index{ZFC set theory}).

When expressed symbolically, the Axiom of Separation and the Axiom of
Replacement contain wff symbols and therefore each represent infinitely many
axioms, one for each possible wff. For this reason, they are often called
axiom schemes\index{axiom scheme}\index{well-formed formula (wff)}.

It turns out that the Axiom of the Null Set, the Axiom of Pairing, and the
Axiom of Separation can be derived from the other axioms and are therefore
unnecessary, although they tend to be included in standard texts for various
reasons (historical, philosophical, and possibly because some authors may not
know this).  In the Metamath\index{Metamath} set theory database, these
redundant axioms are derived from the other ones instead of truly
being considered axioms.
This is in keeping with our general goal of minimizing the number of
axioms we must depend on.

\subsection{Other Axioms}

Above we qualified the phrase ``all of mathematics'' with ``essentially.''
The main important missing piece is the ability to do category theory,
which requires huge sets (inaccessible cardinals) larger than those
postulated by the ZFC axioms. The Tarski--Grothendieck Axiom postulates
the existence of such sets.
Note that this is the same axiom used by Mizar for supporting
category theory.
The Tarski--Grothendieck axiom
can be viewed as a very strong replacement of the Axiom of Infinity,
the Axiom of Choice, and the Axiom of Power Sets.
The \texttt{set.mm} database includes this axiom; see the database
for details about it.
Again, we only use this axiom when we need to.
You are only likely to encounter or use this axiom if you are doing
category theory, since its use is highly specialized,
so we will not list the Tarsky-Grothendieck axiom
in the short list of axioms below.

Can there be even more axioms?
Of course.
G\"{o}del showed that no finite set of axioms or axiom schemes can completely
describe any consistent theory strong enough to include arithmetic.
But practically speaking, the ones above are the accepted foundation that
almost all mathematicians explicitly or implicitly base their work on.

\section{The Axioms in the Metamath Language}\label{metaaxioms}

Here we list the axioms as they appear in
\texttt{set.mm}\index{set theory database (\texttt{set.mm})} so you can
look them up there easily.  Incidentally, the \texttt{show statement
/tex} command\index{\texttt{show statement} command} was used to
typeset them.

%macros from show statement /tex
\newbox\mlinebox
\newbox\mtrialbox
\newbox\startprefix  % Prefix for first line of a formula
\newbox\contprefix  % Prefix for continuation line of a formula
\def\startm{  % Initialize formula line
  \setbox\mlinebox=\hbox{\unhcopy\startprefix}
}
\def\m#1{  % Add a symbol to the formula
  \setbox\mtrialbox=\hbox{\unhcopy\mlinebox $\,#1$}
  \ifdim\wd\mtrialbox>\hsize
    \box\mlinebox
    \setbox\mlinebox=\hbox{\unhcopy\contprefix $\,#1$}
  \else
    \setbox\mlinebox=\hbox{\unhbox\mtrialbox}
  \fi
}
\def\endm{  % Output the last line of a formula
  \box\mlinebox
}

% \SLASH for \ , \TOR for \/ (text OR), \TAND for /\ (text and)
% This embeds a following forced space to force the space.
\newcommand\SLASH{\char`\\~}
\newcommand\TOR{\char`\\/~}
\newcommand\TAND{/\char`\\~}
%
% Macro to output metamath raw text.
% This assumes \startprefix and \contprefix are set.
% NOTE: "\" is tricky to escape, use \SLASH, \TOR, and \TAND inside.
% Any use of "$ { ~ ^" must be escaped; ~ and ^ must be escaped specially.
% We escape { and } for consistency.
% For more about how this macro written, see:
% https://stackoverflow.com/questions/4073674/
% how-to-disable-indentation-in-particular-section-in-latex/4075706
% Use frenchspacing, or "e." will get an extra space after it.
\newlength\mystoreparindent
\newlength\mystorehangindent
\newenvironment{mmraw}{%
\setlength{\mystoreparindent}{\the\parindent}
\setlength{\mystorehangindent}{\the\hangindent}
\setlength{\parindent}{0pt} % TODO - we'll put in the \startprefix instead
\setlength{\hangindent}{\wd\the\contprefix}
\begin{flushleft}
\begin{frenchspacing}
\begin{tt}
{\unhcopy\startprefix}%
}{%
\end{tt}
\end{frenchspacing}
\end{flushleft}
\setlength{\parindent}{\mystoreparindent}
\setlength{\hangindent}{\mystorehangindent}
\vskip 1ex
}

\needspace{5\baselineskip}
\subsection{Propositional Calculus}\label{propcalc}\index{axioms of
propositional calculus}

\needspace{2\baselineskip}
Axiom of Simplification.\label{ax1}

\setbox\startprefix=\hbox{\tt \ \ ax-1\ \$a\ }
\setbox\contprefix=\hbox{\tt \ \ \ \ \ \ \ \ \ \ }
\startm
\m{\vdash}\m{(}\m{\varphi}\m{\rightarrow}\m{(}\m{\psi}\m{\rightarrow}\m{\varphi}\m{)}
\m{)}
\endm

\needspace{3\baselineskip}
\noindent Axiom of Distribution.

\setbox\startprefix=\hbox{\tt \ \ ax-2\ \$a\ }
\setbox\contprefix=\hbox{\tt \ \ \ \ \ \ \ \ \ \ }
\startm
\m{\vdash}\m{(}\m{(}\m{\varphi}\m{\rightarrow}\m{(}\m{\psi}\m{\rightarrow}\m{\chi}
\m{)}\m{)}\m{\rightarrow}\m{(}\m{(}\m{\varphi}\m{\rightarrow}\m{\psi}\m{)}\m{
\rightarrow}\m{(}\m{\varphi}\m{\rightarrow}\m{\chi}\m{)}\m{)}\m{)}
\endm

\needspace{2\baselineskip}
\noindent Axiom of Contraposition.

\setbox\startprefix=\hbox{\tt \ \ ax-3\ \$a\ }
\setbox\contprefix=\hbox{\tt \ \ \ \ \ \ \ \ \ \ }
\startm
\m{\vdash}\m{(}\m{(}\m{\lnot}\m{\varphi}\m{\rightarrow}\m{\lnot}\m{\psi}\m{)}\m{
\rightarrow}\m{(}\m{\psi}\m{\rightarrow}\m{\varphi}\m{)}\m{)}
\endm


\needspace{4\baselineskip}
\noindent Rule of Modus Ponens.\label{axmp}\index{modus ponens}

\setbox\startprefix=\hbox{\tt \ \ min\ \$e\ }
\setbox\contprefix=\hbox{\tt \ \ \ \ \ \ \ \ \ }
\startm
\m{\vdash}\m{\varphi}
\endm

\setbox\startprefix=\hbox{\tt \ \ maj\ \$e\ }
\setbox\contprefix=\hbox{\tt \ \ \ \ \ \ \ \ \ }
\startm
\m{\vdash}\m{(}\m{\varphi}\m{\rightarrow}\m{\psi}\m{)}
\endm

\setbox\startprefix=\hbox{\tt \ \ ax-mp\ \$a\ }
\setbox\contprefix=\hbox{\tt \ \ \ \ \ \ \ \ \ \ \ }
\startm
\m{\vdash}\m{\psi}
\endm


\needspace{7\baselineskip}
\subsection{Axioms of Predicate Calculus with Equality---Tarski's S2}\index{axioms of predicate calculus}

\needspace{3\baselineskip}
\noindent Rule of Generalization.\index{rule of generalization}

\setbox\startprefix=\hbox{\tt \ \ ax-g.1\ \$e\ }
\setbox\contprefix=\hbox{\tt \ \ \ \ \ \ \ \ \ \ \ \ }
\startm
\m{\vdash}\m{\varphi}
\endm

\setbox\startprefix=\hbox{\tt \ \ ax-gen\ \$a\ }
\setbox\contprefix=\hbox{\tt \ \ \ \ \ \ \ \ \ \ \ \ }
\startm
\m{\vdash}\m{\forall}\m{x}\m{\varphi}
\endm

\needspace{2\baselineskip}
\noindent Axiom of Quantified Implication.

\setbox\startprefix=\hbox{\tt \ \ ax-4\ \$a\ }
\setbox\contprefix=\hbox{\tt \ \ \ \ \ \ \ \ \ \ }
\startm
\m{\vdash}\m{(}\m{\forall}\m{x}\m{(}\m{\forall}\m{x}\m{\varphi}\m{\rightarrow}\m{
\psi}\m{)}\m{\rightarrow}\m{(}\m{\forall}\m{x}\m{\varphi}\m{\rightarrow}\m{
\forall}\m{x}\m{\psi}\m{)}\m{)}
\endm

\needspace{3\baselineskip}
\noindent Axiom of Distinctness.

% Aka: Add $d x ph $.
\setbox\startprefix=\hbox{\tt \ \ ax-5\ \$a\ }
\setbox\contprefix=\hbox{\tt \ \ \ \ \ \ \ \ \ \ }
\startm
\m{\vdash}\m{(}\m{\varphi}\m{\rightarrow}\m{\forall}\m{x}\m{\varphi}\m{)}\m{where}\m{ }\m{\$d}\m{ }\m{x}\m{ }\m{\varphi}\m{ }\m{(}\m{x}\m{ }\m{does}\m{ }\m{not}\m{ }\m{occur}\m{ }\m{in}\m{ }\m{\varphi}\m{)}
\endm

\needspace{2\baselineskip}
\noindent Axiom of Existence.

\setbox\startprefix=\hbox{\tt \ \ ax-6\ \$a\ }
\setbox\contprefix=\hbox{\tt \ \ \ \ \ \ \ \ \ \ }
\startm
\m{\vdash}\m{(}\m{\forall}\m{x}\m{(}\m{x}\m{=}\m{y}\m{\rightarrow}\m{\forall}
\m{x}\m{\varphi}\m{)}\m{\rightarrow}\m{\varphi}\m{)}
\endm

\needspace{2\baselineskip}
\noindent Axiom of Equality.

\setbox\startprefix=\hbox{\tt \ \ ax-7\ \$a\ }
\setbox\contprefix=\hbox{\tt \ \ \ \ \ \ \ \ \ \ }
\startm
\m{\vdash}\m{(}\m{x}\m{=}\m{y}\m{\rightarrow}\m{(}\m{x}\m{=}\m{z}\m{
\rightarrow}\m{y}\m{=}\m{z}\m{)}\m{)}
\endm

\needspace{2\baselineskip}
\noindent Axiom of Left Equality for Binary Predicate.

\setbox\startprefix=\hbox{\tt \ \ ax-8\ \$a\ }
\setbox\contprefix=\hbox{\tt \ \ \ \ \ \ \ \ \ \ \ }
\startm
\m{\vdash}\m{(}\m{x}\m{=}\m{y}\m{\rightarrow}\m{(}\m{x}\m{\in}\m{z}\m{
\rightarrow}\m{y}\m{\in}\m{z}\m{)}\m{)}
\endm

\needspace{2\baselineskip}
\noindent Axiom of Right Equality for Binary Predicate.

\setbox\startprefix=\hbox{\tt \ \ ax-9\ \$a\ }
\setbox\contprefix=\hbox{\tt \ \ \ \ \ \ \ \ \ \ \ }
\startm
\m{\vdash}\m{(}\m{x}\m{=}\m{y}\m{\rightarrow}\m{(}\m{z}\m{\in}\m{x}\m{
\rightarrow}\m{z}\m{\in}\m{y}\m{)}\m{)}
\endm


\needspace{4\baselineskip}
\subsection{Axioms of Predicate Calculus with Equality---Auxiliary}\index{axioms of predicate calculus - auxiliary}

\needspace{2\baselineskip}
\noindent Axiom of Quantified Negation.

\setbox\startprefix=\hbox{\tt \ \ ax-10\ \$a\ }
\setbox\contprefix=\hbox{\tt \ \ \ \ \ \ \ \ \ \ }
\startm
\m{\vdash}\m{(}\m{\lnot}\m{\forall}\m{x}\m{\lnot}\m{\forall}\m{x}\m{\varphi}\m{
\rightarrow}\m{\varphi}\m{)}
\endm

\needspace{2\baselineskip}
\noindent Axiom of Quantifier Commutation.

\setbox\startprefix=\hbox{\tt \ \ ax-11\ \$a\ }
\setbox\contprefix=\hbox{\tt \ \ \ \ \ \ \ \ \ \ }
\startm
\m{\vdash}\m{(}\m{\forall}\m{x}\m{\forall}\m{y}\m{\varphi}\m{\rightarrow}\m{
\forall}\m{y}\m{\forall}\m{x}\m{\varphi}\m{)}
\endm

\needspace{3\baselineskip}
\noindent Axiom of Substitution.

\setbox\startprefix=\hbox{\tt \ \ ax-12\ \$a\ }
\setbox\contprefix=\hbox{\tt \ \ \ \ \ \ \ \ \ \ \ }
\startm
\m{\vdash}\m{(}\m{\lnot}\m{\forall}\m{x}\m{\,x}\m{=}\m{y}\m{\rightarrow}\m{(}
\m{x}\m{=}\m{y}\m{\rightarrow}\m{(}\m{\varphi}\m{\rightarrow}\m{\forall}\m{x}\m{(}
\m{x}\m{=}\m{y}\m{\rightarrow}\m{\varphi}\m{)}\m{)}\m{)}\m{)}
\endm

\needspace{3\baselineskip}
\noindent Axiom of Quantified Equality.

\setbox\startprefix=\hbox{\tt \ \ ax-13\ \$a\ }
\setbox\contprefix=\hbox{\tt \ \ \ \ \ \ \ \ \ \ \ }
\startm
\m{\vdash}\m{(}\m{\lnot}\m{\forall}\m{z}\m{\,z}\m{=}\m{x}\m{\rightarrow}\m{(}
\m{\lnot}\m{\forall}\m{z}\m{\,z}\m{=}\m{y}\m{\rightarrow}\m{(}\m{x}\m{=}\m{y}
\m{\rightarrow}\m{\forall}\m{z}\m{\,x}\m{=}\m{y}\m{)}\m{)}\m{)}
\endm

% \noindent Axiom of Quantifier Substitution
%
% \setbox\startprefix=\hbox{\tt \ \ ax-c11n\ \$a\ }
% \setbox\contprefix=\hbox{\tt \ \ \ \ \ \ \ \ \ \ \ }
% \startm
% \m{\vdash}\m{(}\m{\forall}\m{x}\m{\,x}\m{=}\m{y}\m{\rightarrow}\m{(}\m{\forall}
% \m{x}\m{\varphi}\m{\rightarrow}\m{\forall}\m{y}\m{\varphi}\m{)}\m{)}
% \endm
%
% \noindent Axiom of Distinct Variables. (This axiom requires
% that two individual variables
% be distinct\index{\texttt{\$d} statement}\index{distinct
% variables}.)
%
% \setbox\startprefix=\hbox{\tt \ \ \ \ \ \ \ \ \$d\ }
% \setbox\contprefix=\hbox{\tt \ \ \ \ \ \ \ \ \ \ \ }
% \startm
% \m{x}\m{\,}\m{y}
% \endm
%
% \setbox\startprefix=\hbox{\tt \ \ ax-c16\ \$a\ }
% \setbox\contprefix=\hbox{\tt \ \ \ \ \ \ \ \ \ \ \ }
% \startm
% \m{\vdash}\m{(}\m{\forall}\m{x}\m{\,x}\m{=}\m{y}\m{\rightarrow}\m{(}\m{\varphi}\m{
% \rightarrow}\m{\forall}\m{x}\m{\varphi}\m{)}\m{)}
% \endm

% \noindent Axiom of Quantifier Introduction (2).  (This axiom requires
% that the individual variable not occur in the
% wff\index{\texttt{\$d} statement}\index{distinct variables}.)
%
% \setbox\startprefix=\hbox{\tt \ \ \ \ \ \ \ \ \$d\ }
% \setbox\contprefix=\hbox{\tt \ \ \ \ \ \ \ \ \ \ \ }
% \startm
% \m{x}\m{\,}\m{\varphi}
% \endm
% \setbox\startprefix=\hbox{\tt \ \ ax-5\ \$a\ }
% \setbox\contprefix=\hbox{\tt \ \ \ \ \ \ \ \ \ \ \ }
% \startm
% \m{\vdash}\m{(}\m{\varphi}\m{\rightarrow}\m{\forall}\m{x}\m{\varphi}\m{)}
% \endm

\subsection{Set Theory}\label{mmsettheoryaxioms}

In order to make the axioms of set theory\index{axioms of set theory} a little
more compact, there are several definitions from logic that we make use of
implicitly, namely, ``logical {\sc and},''\index{conjunction ($\wedge$)}
\index{logical {\sc and} ($\wedge$)} ``logical equivalence,''\index{logical
equivalence ($\leftrightarrow$)}\index{biconditional ($\leftrightarrow$)} and
``there exists.''\index{existential quantifier ($\exists$)}

\begin{center}\begin{tabular}{rcl}
  $( \varphi \wedge \psi )$ &\mbox{stands for}& $\neg ( \varphi
     \rightarrow \neg \psi )$\\
  $( \varphi \leftrightarrow \psi )$& \mbox{stands
     for}& $( ( \varphi \rightarrow \psi ) \wedge
     ( \psi \rightarrow \varphi ) )$\\
  $\exists x \,\varphi$ &\mbox{stands for}& $\neg \forall x \neg \varphi$
\end{tabular}\end{center}

In addition, the axioms of set theory require that all variables be
dis\-tinct,\index{distinct variables}\footnote{Set theory axioms can be
devised so that {\em no} variables are required to be distinct,
provided we replace \texttt{ax-c16} with an axiom stating that ``at
least two things exist,'' thus
making \texttt{ax-5} the only other axiom requiring the
\texttt{\$d} statement.  These axioms are unconventional and are not
presented here, but they can be found on the \url{http://metamath.org}
web site.  See also the Comment on
p.~\pageref{nodd}.}\index{\texttt{\$d} statement} thus we also assume:
\begin{center}
  \texttt{\$d }$x\,y\,z\,w$
\end{center}

\needspace{2\baselineskip}
\noindent Axiom of Extensionality.\index{Axiom of Extensionality}

\setbox\startprefix=\hbox{\tt \ \ ax-ext\ \$a\ }
\setbox\contprefix=\hbox{\tt \ \ \ \ \ \ \ \ \ \ \ \ }
\startm
\m{\vdash}\m{(}\m{\forall}\m{x}\m{(}\m{x}\m{\in}\m{y}\m{\leftrightarrow}\m{x}
\m{\in}\m{z}\m{)}\m{\rightarrow}\m{y}\m{=}\m{z}\m{)}
\endm

\needspace{3\baselineskip}
\noindent Axiom of Replacement.\index{Axiom of Replacement}

\setbox\startprefix=\hbox{\tt \ \ ax-rep\ \$a\ }
\setbox\contprefix=\hbox{\tt \ \ \ \ \ \ \ \ \ \ \ \ }
\startm
\m{\vdash}\m{(}\m{\forall}\m{w}\m{\exists}\m{y}\m{\forall}\m{z}\m{(}\m{%
\forall}\m{y}\m{\varphi}\m{\rightarrow}\m{z}\m{=}\m{y}\m{)}\m{\rightarrow}\m{%
\exists}\m{y}\m{\forall}\m{z}\m{(}\m{z}\m{\in}\m{y}\m{\leftrightarrow}\m{%
\exists}\m{w}\m{(}\m{w}\m{\in}\m{x}\m{\wedge}\m{\forall}\m{y}\m{\varphi}\m{)}%
\m{)}\m{)}
\endm

\needspace{2\baselineskip}
\noindent Axiom of Union.\index{Axiom of Union}

\setbox\startprefix=\hbox{\tt \ \ ax-un\ \$a\ }
\setbox\contprefix=\hbox{\tt \ \ \ \ \ \ \ \ \ \ \ }
\startm
\m{\vdash}\m{\exists}\m{x}\m{\forall}\m{y}\m{(}\m{\exists}\m{x}\m{(}\m{y}\m{
\in}\m{x}\m{\wedge}\m{x}\m{\in}\m{z}\m{)}\m{\rightarrow}\m{y}\m{\in}\m{x}\m{)}
\endm

\needspace{2\baselineskip}
\noindent Axiom of Power Sets.\index{Axiom of Power Sets}

\setbox\startprefix=\hbox{\tt \ \ ax-pow\ \$a\ }
\setbox\contprefix=\hbox{\tt \ \ \ \ \ \ \ \ \ \ \ \ }
\startm
\m{\vdash}\m{\exists}\m{x}\m{\forall}\m{y}\m{(}\m{\forall}\m{x}\m{(}\m{x}\m{
\in}\m{y}\m{\rightarrow}\m{x}\m{\in}\m{z}\m{)}\m{\rightarrow}\m{y}\m{\in}\m{x}
\m{)}
\endm

\needspace{3\baselineskip}
\noindent Axiom of Regularity.\index{Axiom of Regularity}

\setbox\startprefix=\hbox{\tt \ \ ax-reg\ \$a\ }
\setbox\contprefix=\hbox{\tt \ \ \ \ \ \ \ \ \ \ \ \ }
\startm
\m{\vdash}\m{(}\m{\exists}\m{x}\m{\,x}\m{\in}\m{y}\m{\rightarrow}\m{\exists}
\m{x}\m{(}\m{x}\m{\in}\m{y}\m{\wedge}\m{\forall}\m{z}\m{(}\m{z}\m{\in}\m{x}\m{
\rightarrow}\m{\lnot}\m{z}\m{\in}\m{y}\m{)}\m{)}\m{)}
\endm

\needspace{3\baselineskip}
\noindent Axiom of Infinity.\index{Axiom of Infinity}

\setbox\startprefix=\hbox{\tt \ \ ax-inf\ \$a\ }
\setbox\contprefix=\hbox{\tt \ \ \ \ \ \ \ \ \ \ \ \ \ \ \ }
\startm
\m{\vdash}\m{\exists}\m{x}\m{(}\m{y}\m{\in}\m{x}\m{\wedge}\m{\forall}\m{y}%
\m{(}\m{y}\m{\in}\m{x}\m{\rightarrow}\m{\exists}\m{z}\m{(}\m{y}\m{\in}\m{z}\m{%
\wedge}\m{z}\m{\in}\m{x}\m{)}\m{)}\m{)}
\endm

\needspace{4\baselineskip}
\noindent Axiom of Choice.\index{Axiom of Choice}

\setbox\startprefix=\hbox{\tt \ \ ax-ac\ \$a\ }
\setbox\contprefix=\hbox{\tt \ \ \ \ \ \ \ \ \ \ \ \ \ \ }
\startm
\m{\vdash}\m{\exists}\m{x}\m{\forall}\m{y}\m{\forall}\m{z}\m{(}\m{(}\m{y}\m{%
\in}\m{z}\m{\wedge}\m{z}\m{\in}\m{w}\m{)}\m{\rightarrow}\m{\exists}\m{w}\m{%
\forall}\m{y}\m{(}\m{\exists}\m{w}\m{(}\m{(}\m{y}\m{\in}\m{z}\m{\wedge}\m{z}%
\m{\in}\m{w}\m{)}\m{\wedge}\m{(}\m{y}\m{\in}\m{w}\m{\wedge}\m{w}\m{\in}\m{x}%
\m{)}\m{)}\m{\leftrightarrow}\m{y}\m{=}\m{w}\m{)}\m{)}
\endm

\subsection{That's It}

There you have it, the axioms for (essentially) all of mathematics!
Wonder at them and stare at them in awe.  Put a copy in your wallet, and
you will carry in your pocket the encoding for all theorems ever proved
and that ever will be proved, from the most mundane to the most
profound.

\section{A Hierarchy of Definitions}\label{hierarchy}

The axioms in the previous section in principle embody everything that can be
done within standard mathematics.  However, it is impractical to accomplish
very much by using them directly, for even simple concepts (from a human
perspective) can involve extremely long, incomprehensible formulas.
Mathematics is made practical by introducing definitions\index{definition}.
Definitions usually introduce new symbols, or at least new relationships among
existing symbols, to abbreviate more complex formulas.  An important
requirement for a definition is that there exist a straightforward
(algorithmic) method for eliminating the abbreviation by expanding it into the
more primitive symbol string that it represents.  Some
important definitions included in
the file \texttt{set.mm} are listed in this section for reference, and also to
give you a feel for why something like $\omega$\index{omega ($\omega$)} (the
set of natural numbers\index{natural number} 0, 1, 2,\ldots) becomes very
complicated when completely expanded into primitive symbols.

What is the motivation for definitions, aside from allowing complicated
expressions to be expressed more simply?  In the case of  $\omega$, one goal is
to provide a basis for the theory of natural numbers.\index{natural number}
Before set theory was invented, a set of axioms for arithmetic, called Peano's
postulates\index{Peano's postulates}, was devised and shown to have the
properties one expects for natural numbers.  Now anyone can postulate a
set of axioms, but if the axioms are inconsistent contradictions can be derived
from them.  Once a contradiction is derived, anything can be trivially
proved, including
all the facts of arithmetic and their negations.  To ensure that an
axiom system is at least as reliable as the axioms for set theory, we can
define sets and operations on those sets that satisfy the new axioms. In the
\texttt{set.mm} Metamath database, we prove that the elements of $\omega$ satisfy
Peano's postulates, and it's a long and hard journey to get there directly
from the axioms of set theory.  But the result is confidence in the
foundations of arithmetic.  And there is another advantage:  we now have all
the tools of set theory at our disposal for manipulating objects that obey the
axioms for arithmetic.

What are the criteria we use for definitions?  First, and of utmost importance,
the definition should not be {\em creative}\index{creative
definition}\index{definition!creative}, that
is it should not allow an expression that previously qualified as a wff but
was not provable, to become provable.   Second, the definition should be {\em
eliminable}\index{definition!eliminability}, that is, there should exist an
algorithmic method for converting any expression using the definition into
a logically equivalent expression that previously qualified as a wff.

In almost all cases below, definitions connect two expressions with either
$\leftrightarrow$ or $=$.  Eliminating\footnote{Here we mean the
elimination that a human might do in his or her head.  To eliminate them as
part of a Metamath proof we would invoke one of a number of
theorems that deal with transitivity of equivalence or equality; there are
many such examples in the proofs in \texttt{set.mm}.} such a definition is a
simple matter of substituting the expression on the left-hand side ({\em
definiendum}\index{definiendum} or thing being defined) with the equivalent,
more primitive expression on the right-hand side ({\em
definiens}\index{definiens} or definition).

Often a definition has variables on the right-hand side which do not appear on
the left-hand side; these are called {\em dummy variables}.\index{dummy
variable!in definitions}  In this case, any
allowable substitution (such as a new, distinct
variable) can be used when the definition is eliminated.  Dummy variables may
be used only if they are {\em effectively bound}\index{effectively bound
variable}, meaning that the definition will remain logically equivalent upon
any substitution of a dummy variable with any other {\em qualifying
expression}\index{qualifying expression}, i.e.\ any symbol string (such as
another variable) that
meets the restrictions on the dummy variable imposed by \texttt{\$d} and
\texttt{\$f} statements.  For example, we could define a constant $\perp$
(inverted tee, meaning logical ``false'') as $( \varphi \wedge \lnot \varphi
)$, i.e.\ ``phi and not phi.''  Here $\varphi$ is effectively bound because the
definition remains logically equivalent when we replace $\varphi$ with any
other wff.  (It is actually \texttt{df-fal}
in \texttt{set.mm}, which defines $\perp$.)

There are two cases where eliminating definitions is a little more
complex.  These cases are the definitions \texttt{df-bi} and
\texttt{df-cleq}.  The first stretches the concept of a definition a
little, as in effect it ``defines a definition;'' however, it meets our
requirements for a definition in that it is eliminable and does not
strengthen the language.  Theorem \texttt{bii} shows the substitution
needed to eliminate the $\leftrightarrow$\index{logical equivalence
($\leftrightarrow$)}\index{biconditional ($\leftrightarrow$)} symbol.

Definition \texttt{df-cleq}\index{equality ($=$)} extends the usage of
the equality symbol to include ``classes''\index{class} in set theory.  The
reason it is potentially problematic is that it can lead to statements which
do not follow from logic alone but presuppose the Axiom of
Extensionality\index{Axiom of Extensionality}, so we include this axiom
as a hypothesis for the definition.  We could have made \texttt{df-cleq} directly
eliminable by introducing a new equality symbol, but have chosen not to do so
in keeping with standard textbook practice.  Definitions such as \texttt{df-cleq}
that extend the meaning of existing symbols must be introduced carefully so
that they do not lead to contradictions.  Definition \texttt{df-clel} also
extends the meaning of an existing symbol ($\in$); while it doesn't strengthen
the language like \texttt{df-cleq}, this is not obvious and it must also be
subject to the same scrutiny.

Exercise:  Study how the wff $x\in\omega$, meaning ``$x$ is a natural
number,'' could be expanded in terms of primitive symbols, starting with the
definitions \texttt{df-clel} on p.~\pageref{dfclel} and \texttt{df-om} on
p.~\pageref{dfom} and working your way back.  Don't bother to work out the
details; just make sure that you understand how you could do it in principle.
The answer is shown in the footnote on p.~\pageref{expandom}.  If you
actually do work it out, you won't get exactly the same answer because we used
a few simplifications such as discarding occurrences of $\lnot\lnot$ (double
negation).

In the definitions below, we have placed the {\sc ascii} Metamath source
below each of the formulas to help you become familiar with the
notation in the database.  For simplicity, the necessary \texttt{\$f}
and \texttt{\$d} statements are not shown.  If you are in doubt, use the
\texttt{show statement}\index{\texttt{show statement} command} command
in the Metamath program to see the full statement.
A selection of this notation is summarized in Appendix~\ref{ASCII}.

To understand the motivation for these definitions, you should consult the
references indicated:  Takeuti and Zaring \cite{Takeuti}\index{Takeuti, Gaisi},
Quine \cite{Quine}\index{Quine, Willard Van Orman}, Bell and Machover
\cite{Bell}\index{Bell, J. L.}, and Enderton \cite{Enderton}\index{Enderton,
Herbert B.}.  Our list of definitions is provided more for reference than as a
learning aid.  However, by looking at a few of them you can gain a feel for
how the hierarchy is built up.  The definitions are a representative sample of
the many definitions
in \texttt{set.mm}, but they are complete with respect to the
theorem examples we will present in Section~\ref{sometheorems}.  Also, some are
slightly different from, but logically equivalent to, the ones in \texttt{set.mm}
(some of which have been revised over time to shorten them, for example).

\subsection{Definitions for Propositional Calculus}\label{metadefprop}

The symbols $\varphi$, $\psi$, and $\chi$ represent wffs.

Our first definition introduces the biconditional
connective\footnote{The term ``connective'' is informally used to mean a
symbol that is placed between two variables or adjacent to a variable,
whereas a mathematical ``constant'' usually indicates a symbol such as
the number 0 that may replace a variable or metavariable.  From
Metamath's point of view, there is no distinction between a connective
and a constant; both are constants in the Metamath
language.}\index{connective}\index{constant} (also called logical
equivalence)\index{logical equivalence
($\leftrightarrow$)}\index{biconditional ($\leftrightarrow$)}.  Unlike
most traditional developments, we have chosen not to have a separate
symbol such as ``Df.'' to mean ``is defined as.''  Instead, we will use
the biconditional connective for this purpose, as it lets us use
logic to manipulate definitions directly.  Here we state the properties
of the biconditional connective with a carefully crafted \texttt{\$a}
statement, which effectively uses the biconditional connective to define
itself.  The $\leftrightarrow$ symbol can be eliminated from a formula
using theorem \texttt{bii}, which is derived later.

\vskip 2ex
\noindent Define the biconditional connective.\label{df-bi}

\vskip 0.5ex
\setbox\startprefix=\hbox{\tt \ \ df-bi\ \$a\ }
\setbox\contprefix=\hbox{\tt \ \ \ \ \ \ \ \ \ \ \ }
\startm
\m{\vdash}\m{\lnot}\m{(}\m{(}\m{(}\m{\varphi}\m{\leftrightarrow}\m{\psi}\m{)}%
\m{\rightarrow}\m{\lnot}\m{(}\m{(}\m{\varphi}\m{\rightarrow}\m{\psi}\m{)}\m{%
\rightarrow}\m{\lnot}\m{(}\m{\psi}\m{\rightarrow}\m{\varphi}\m{)}\m{)}\m{)}\m{%
\rightarrow}\m{\lnot}\m{(}\m{\lnot}\m{(}\m{(}\m{\varphi}\m{\rightarrow}\m{%
\psi}\m{)}\m{\rightarrow}\m{\lnot}\m{(}\m{\psi}\m{\rightarrow}\m{\varphi}\m{)}%
\m{)}\m{\rightarrow}\m{(}\m{\varphi}\m{\leftrightarrow}\m{\psi}\m{)}\m{)}\m{)}
\endm
\begin{mmraw}%
|- -. ( ( ( ph <-> ps ) -> -. ( ( ph -> ps ) ->
-. ( ps -> ph ) ) ) -> -. ( -. ( ( ph -> ps ) -> -. (
ps -> ph ) ) -> ( ph <-> ps ) ) ) \$.
\end{mmraw}

\noindent This theorem relates the biconditional connective to primitive
connectives and can be used to eliminate the $\leftrightarrow$ symbol from any
wff.

\vskip 0.5ex
\setbox\startprefix=\hbox{\tt \ \ bii\ \$p\ }
\setbox\contprefix=\hbox{\tt \ \ \ \ \ \ \ \ \ }
\startm
\m{\vdash}\m{(}\m{(}\m{\varphi}\m{\leftrightarrow}\m{\psi}\m{)}\m{\leftrightarrow}
\m{\lnot}\m{(}\m{(}\m{\varphi}\m{\rightarrow}\m{\psi}\m{)}\m{\rightarrow}\m{\lnot}
\m{(}\m{\psi}\m{\rightarrow}\m{\varphi}\m{)}\m{)}\m{)}
\endm
\begin{mmraw}%
|- ( ( ph <-> ps ) <-> -. ( ( ph -> ps ) -> -. ( ps -> ph ) ) ) \$= ... \$.
\end{mmraw}

\noindent Define disjunction ({\sc or}).\index{disjunction ($\vee$)}%
\index{logical or (vee)@logical {\sc or} ($\vee$)}%
\index{df-or@\texttt{df-or}}\label{df-or}

\vskip 0.5ex
\setbox\startprefix=\hbox{\tt \ \ df-or\ \$a\ }
\setbox\contprefix=\hbox{\tt \ \ \ \ \ \ \ \ \ \ \ }
\startm
\m{\vdash}\m{(}\m{(}\m{\varphi}\m{\vee}\m{\psi}\m{)}\m{\leftrightarrow}\m{(}\m{
\lnot}\m{\varphi}\m{\rightarrow}\m{\psi}\m{)}\m{)}
\endm
\begin{mmraw}%
|- ( ( ph \TOR ps ) <-> ( -. ph -> ps ) ) \$.
\end{mmraw}

\noindent Define conjunction ({\sc and}).\index{conjunction ($\wedge$)}%
\index{logical {\sc and} ($\wedge$)}%
\index{df-an@\texttt{df-an}}\label{df-an}

\vskip 0.5ex
\setbox\startprefix=\hbox{\tt \ \ df-an\ \$a\ }
\setbox\contprefix=\hbox{\tt \ \ \ \ \ \ \ \ \ \ \ }
\startm
\m{\vdash}\m{(}\m{(}\m{\varphi}\m{\wedge}\m{\psi}\m{)}\m{\leftrightarrow}\m{\lnot}
\m{(}\m{\varphi}\m{\rightarrow}\m{\lnot}\m{\psi}\m{)}\m{)}
\endm
\begin{mmraw}%
|- ( ( ph \TAND ps ) <-> -. ( ph -> -. ps ) ) \$.
\end{mmraw}

\noindent Define disjunction ({\sc or}) of 3 wffs.%
\index{df-3or@\texttt{df-3or}}\label{df-3or}

\vskip 0.5ex
\setbox\startprefix=\hbox{\tt \ \ df-3or\ \$a\ }
\setbox\contprefix=\hbox{\tt \ \ \ \ \ \ \ \ \ \ \ \ }
\startm
\m{\vdash}\m{(}\m{(}\m{\varphi}\m{\vee}\m{\psi}\m{\vee}\m{\chi}\m{)}\m{
\leftrightarrow}\m{(}\m{(}\m{\varphi}\m{\vee}\m{\psi}\m{)}\m{\vee}\m{\chi}\m{)}
\m{)}
\endm
\begin{mmraw}%
|- ( ( ph \TOR ps \TOR ch ) <-> ( ( ph \TOR ps ) \TOR ch ) ) \$.
\end{mmraw}

\noindent Define conjunction ({\sc and}) of 3 wffs.%
\index{df-3an}\label{df-3an}

\vskip 0.5ex
\setbox\startprefix=\hbox{\tt \ \ df-3an\ \$a\ }
\setbox\contprefix=\hbox{\tt \ \ \ \ \ \ \ \ \ \ \ \ }
\startm
\m{\vdash}\m{(}\m{(}\m{\varphi}\m{\wedge}\m{\psi}\m{\wedge}\m{\chi}\m{)}\m{
\leftrightarrow}\m{(}\m{(}\m{\varphi}\m{\wedge}\m{\psi}\m{)}\m{\wedge}\m{\chi}
\m{)}\m{)}
\endm

\begin{mmraw}%
|- ( ( ph \TAND ps \TAND ch ) <-> ( ( ph \TAND ps ) \TAND ch ) ) \$.
\end{mmraw}

\subsection{Definitions for Predicate Calculus}\label{metadefpred}

The symbols $x$, $y$, and $z$ represent individual variables of predicate
calculus.  In this section, they are not necessarily distinct unless it is
explicitly
mentioned.

\vskip 2ex
\noindent Define existential quantification.
The expression $\exists x \varphi$ means
``there exists an $x$ where $\varphi$ is true.''\index{existential quantifier
($\exists$)}\label{df-ex}

\vskip 0.5ex
\setbox\startprefix=\hbox{\tt \ \ df-ex\ \$a\ }
\setbox\contprefix=\hbox{\tt \ \ \ \ \ \ \ \ \ \ \ }
\startm
\m{\vdash}\m{(}\m{\exists}\m{x}\m{\varphi}\m{\leftrightarrow}\m{\lnot}\m{\forall}
\m{x}\m{\lnot}\m{\varphi}\m{)}
\endm
\begin{mmraw}%
|- ( E. x ph <-> -. A. x -. ph ) \$.
\end{mmraw}

\noindent Define proper substitution.\index{proper
substitution}\index{substitution!proper}\label{df-sb}
In our notation, we use $[ y / x ] \varphi$ to mean ``the wff that
results when $y$ is properly substituted for $x$ in the wff
$\varphi$.''\footnote{
This can also be described
as substituting $x$ with $y$, $y$ properly replaces $x$, or
$x$ is properly replaced by $y$.}
% This is elsb4, though it currently says: ( [ x / y ] z e. y <-> z e. x )
For example,
$[ y / x ] z \in x$ is the same as $z \in y$.
One way to remember this notation is to notice that it looks like division
and recall that $( y / x ) \cdot x $ is $y$ (when $x \neq 0$).
The notation is different from the notation $\varphi ( x | y )$
that is sometimes used, because the latter notation is ambiguous for us:
for example, we don't know whether $\lnot \varphi ( x | y )$ is to be
interpreted as $\lnot ( \varphi ( x | y ) )$ or
$( \lnot \varphi ) ( x | y )$.\footnote{Because of the way
we initially defined wffs, this is the case
with any postfix connective\index{postfix connective} (one occurring after the
symbols being connected) or infix connective\index{infix connective} (one
occurring between the symbols being connected).  Metamath does not have a
built-in notion of operator binding strength that could eliminate the
ambiguity.  The initial parenthesis effectively provides a prefix
connective\index{prefix connective} to eliminate ambiguity.  Some conventions,
such as Polish notation\index{Polish notation} used in the 1930's and 1940's
by Polish logicians, use only prefix connectives and thus allow the total
elimination of parentheses, at the expense of readability.  In Metamath we
could actually redefine all notation to be Polish if we wanted to without
having to change any proofs!}  Other texts often use $\varphi(y)$ to indicate
our $[ y / x ] \varphi$, but this notation is even more ambiguous since there is
no explicit indication of what is being substituted.
Note that this
definition is valid even when
$x$ and $y$ are the same variable.  The first conjunct is a ``trick'' used to
achieve this property, making the definition look somewhat peculiar at
first.

\vskip 0.5ex
\setbox\startprefix=\hbox{\tt \ \ df-sb\ \$a\ }
\setbox\contprefix=\hbox{\tt \ \ \ \ \ \ \ \ \ \ \ }
\startm
\m{\vdash}\m{(}\m{[}\m{y}\m{/}\m{x}\m{]}\m{\varphi}\m{\leftrightarrow}\m{(}%
\m{(}\m{x}\m{=}\m{y}\m{\rightarrow}\m{\varphi}\m{)}\m{\wedge}\m{\exists}\m{x}%
\m{(}\m{x}\m{=}\m{y}\m{\wedge}\m{\varphi}\m{)}\m{)}\m{)}
\endm
\begin{mmraw}%
|- ( [ y / x ] ph <-> ( ( x = y -> ph ) \TAND E. x ( x = y \TAND ph ) ) ) \$.
\end{mmraw}


\noindent Define existential uniqueness\index{existential uniqueness
quantifier ($\exists "!$)} (``there exists exactly one'').  Note that $y$ is a
variable distinct from $x$ and not occurring in $\varphi$.\label{df-eu}

\vskip 0.5ex
\setbox\startprefix=\hbox{\tt \ \ df-eu\ \$a\ }
\setbox\contprefix=\hbox{\tt \ \ \ \ \ \ \ \ \ \ \ }
\startm
\m{\vdash}\m{(}\m{\exists}\m{{!}}\m{x}\m{\varphi}\m{\leftrightarrow}\m{\exists}
\m{y}\m{\forall}\m{x}\m{(}\m{\varphi}\m{\leftrightarrow}\m{x}\m{=}\m{y}\m{)}\m{)}
\endm

\begin{mmraw}%
|- ( E! x ph <-> E. y A. x ( ph <-> x = y ) ) \$.
\end{mmraw}

\subsection{Definitions for Set Theory}\label{setdefinitions}

The symbols $x$, $y$, $z$, and $w$ represent individual variables of
predicate calculus, which in set theory are understood to be sets.
However, using only the constructs shown so far would be very inconvenient.

To make set theory more practical, we introduce the notion of a ``class.''
A class\index{class} is either a set variable (such as $x$) or an
expression of the form $\{ x | \varphi\}$ (called an ``abstraction
class''\index{abstraction class}\index{class abstraction}).  Note that
sets (i.e.\ individual variables) always exist (this is a theorem of
logic, namely $\exists y \, y = x$ for any set $x$), whereas classes may
or may not exist (i.e.\ $\exists y \, y = A$ may or may not be true).
If a class does not exist it is called a ``proper class.''\index{proper
class}\index{class!proper} Definitions \texttt{df-clab},
\texttt{df-cleq}, and \texttt{df-clel} can be used to convert an
expression containing classes into one containing only set variables and
wff metavariables.

The symbols $A$, $B$, $C$, $D$, $ F$, $G$, and $R$ are metavariables that range
over classes.  A class metavariable $A$ may be eliminated from a wff by
replacing it with $\{ x|\varphi\}$ where neither $x$ nor $\varphi$ occur in
the wff.

The theory of classes can be shown to be an eliminable and conservative
extension of set theory. The \textbf{eliminability}
property shows that for every
formula in the extended language we can build a logically equivalent
formula in the basic language; so that even if the extended language
provides more ease to convey and formulate mathematical ideas for set
theory, its expressive power does not in fact strengthen the basic
language's expressive power.
The \textbf{conservation} property shows that for
every proof of a formula of the basic language in the extended system
we can build another proof of the same formula in the basic system;
so that, concerning theorems on sets only, the deductive powers of
the extended system and of the basic system are identical. Together,
these properties mean that the extended language can be treated as a
definitional extension that is \textbf{sound}.

A rigorous justification, which we will not give here, can be found in
Levy \cite[pp.~357-366]{Levy} supplementing his informal introduction to class
theory on pp.~7-17. Two other good treatments of class theory are provided
by Quine \cite[pp.~15-21]{Quine}\index{Quine, Willard Van Orman}
and also \cite[pp.~10-14]{Takeuti}\index{Takeuti, Gaisi}.
Quine's exposition (he calls them virtual classes)
is nicely written and very readable.

In the rest of this
section, individual variables are always assumed to be distinct from
each other unless otherwise indicated.  In addition, dummy variables on the
right-hand side of a definition do not occur in the class and wff
metavariables in the definition.

The definitions we present here are a partial but self-contained
collection selected from several hundred that appear in the current
\texttt{set.mm} database.  They are adequate for a basic development of
elementary set theory.

\vskip 2ex
\noindent Define the abstraction class.\index{abstraction class}\index{class
abstraction}\label{df-clab}  $x$ and $y$
need not be distinct.  Definition 2.1 of Quine, p.~16.  This definition may
seem puzzling since it is shorter than the expression being defined and does not
buy us anything in terms of brevity.  The reason we introduce this definition
is because it fits in neatly with the extension of the $\in$ connective
provided by \texttt{df-clel}.

\vskip 0.5ex
\setbox\startprefix=\hbox{\tt \ \ df-clab\ \$a\ }
\setbox\contprefix=\hbox{\tt \ \ \ \ \ \ \ \ \ \ \ \ \ }
\startm
\m{\vdash}\m{(}\m{x}\m{\in}\m{\{}\m{y}\m{|}\m{\varphi}\m{\}}\m{%
\leftrightarrow}\m{[}\m{x}\m{/}\m{y}\m{]}\m{\varphi}\m{)}
\endm
\begin{mmraw}%
|- ( x e. \{ y | ph \} <-> [ x / y ] ph ) \$.
\end{mmraw}

\noindent Define the equality connective between classes\index{class
equality}\label{df-cleq}.  See Quine or Chapter 4 of Takeuti and Zaring for its
justification and methods for eliminating it.  This is an example of a
somewhat ``dangerous'' definition, because it extends the use of the
existing equality symbol rather than introducing a new symbol, allowing
us to make statements in the original language that may not be true.
For example, it permits us to deduce $y = z \leftrightarrow \forall x (
x \in y \leftrightarrow x \in z )$ which is not a theorem of logic but
rather presupposes the Axiom of Extensionality,\index{Axiom of
Extensionality} which we include as a hypothesis so that we can know
when this axiom is assumed in a proof (with the \texttt{show
trace{\char`\_}back} command).  We could avoid the danger by introducing
another symbol, say $\eqcirc$, in place of $=$; this
would also have the advantage of making elimination of the definition
straightforward and would eliminate the need for Extensionality as a
hypothesis.  We would then also have the advantage of being able to
identify exactly where Extensionality truly comes into play.  One of our
theorems would be $x \eqcirc y \leftrightarrow x = y$
by invoking Extensionality.  However in keeping with standard practice
we retain the ``dangerous'' definition.

\vskip 0.5ex
\setbox\startprefix=\hbox{\tt \ \ df-cleq.1\ \$e\ }
\setbox\contprefix=\hbox{\tt \ \ \ \ \ \ \ \ \ \ \ \ \ \ \ }
\startm
\m{\vdash}\m{(}\m{\forall}\m{x}\m{(}\m{x}\m{\in}\m{y}\m{\leftrightarrow}\m{x}
\m{\in}\m{z}\m{)}\m{\rightarrow}\m{y}\m{=}\m{z}\m{)}
\endm
\setbox\startprefix=\hbox{\tt \ \ df-cleq\ \$a\ }
\setbox\contprefix=\hbox{\tt \ \ \ \ \ \ \ \ \ \ \ \ \ }
\startm
\m{\vdash}\m{(}\m{A}\m{=}\m{B}\m{\leftrightarrow}\m{\forall}\m{x}\m{(}\m{x}\m{
\in}\m{A}\m{\leftrightarrow}\m{x}\m{\in}\m{B}\m{)}\m{)}
\endm
% We need to reset the startprefix and contprefix.
\setbox\startprefix=\hbox{\tt \ \ df-cleq.1\ \$e\ }
\setbox\contprefix=\hbox{\tt \ \ \ \ \ \ \ \ \ \ \ \ \ \ \ }
\begin{mmraw}%
|- ( A. x ( x e. y <-> x e. z ) -> y = z ) \$.
\end{mmraw}
\setbox\startprefix=\hbox{\tt \ \ df-cleq\ \$a\ }
\setbox\contprefix=\hbox{\tt \ \ \ \ \ \ \ \ \ \ \ \ \ }
\begin{mmraw}%
|- ( A = B <-> A. x ( x e. A <-> x e. B ) ) \$.
\end{mmraw}

\noindent Define the membership connective between classes\index{class
membership}.  Theorem 6.3 of Quine, p.~41, which we adopt as a definition.
Note that it extends the use of the existing membership symbol, but unlike
{\tt df-cleq} it does not extend the set of valid wffs of logic when the class
metavariables are replaced with set variables.\label{dfclel}\label{df-clel}

\vskip 0.5ex
\setbox\startprefix=\hbox{\tt \ \ df-clel\ \$a\ }
\setbox\contprefix=\hbox{\tt \ \ \ \ \ \ \ \ \ \ \ \ \ }
\startm
\m{\vdash}\m{(}\m{A}\m{\in}\m{B}\m{\leftrightarrow}\m{\exists}\m{x}\m{(}\m{x}
\m{=}\m{A}\m{\wedge}\m{x}\m{\in}\m{B}\m{)}\m{)}
\endm
\begin{mmraw}%
|- ( A e. B <-> E. x ( x = A \TAND x e. B ) ) \$.?
\end{mmraw}

\noindent Define inequality.

\vskip 0.5ex
\setbox\startprefix=\hbox{\tt \ \ df-ne\ \$a\ }
\setbox\contprefix=\hbox{\tt \ \ \ \ \ \ \ \ \ \ \ }
\startm
\m{\vdash}\m{(}\m{A}\m{\ne}\m{B}\m{\leftrightarrow}\m{\lnot}\m{A}\m{=}\m{B}%
\m{)}
\endm
\begin{mmraw}%
|- ( A =/= B <-> -. A = B ) \$.
\end{mmraw}

\noindent Define restricted universal quantification.\index{universal
quantifier ($\forall$)!restricted}  Enderton, p.~22.

\vskip 0.5ex
\setbox\startprefix=\hbox{\tt \ \ df-ral\ \$a\ }
\setbox\contprefix=\hbox{\tt \ \ \ \ \ \ \ \ \ \ \ \ }
\startm
\m{\vdash}\m{(}\m{\forall}\m{x}\m{\in}\m{A}\m{\varphi}\m{\leftrightarrow}\m{%
\forall}\m{x}\m{(}\m{x}\m{\in}\m{A}\m{\rightarrow}\m{\varphi}\m{)}\m{)}
\endm
\begin{mmraw}%
|- ( A. x e. A ph <-> A. x ( x e. A -> ph ) ) \$.
\end{mmraw}

\noindent Define restricted existential quantification.\index{existential
quantifier ($\exists$)!restricted}  Enderton, p.~22.

\vskip 0.5ex
\setbox\startprefix=\hbox{\tt \ \ df-rex\ \$a\ }
\setbox\contprefix=\hbox{\tt \ \ \ \ \ \ \ \ \ \ \ \ }
\startm
\m{\vdash}\m{(}\m{\exists}\m{x}\m{\in}\m{A}\m{\varphi}\m{\leftrightarrow}\m{%
\exists}\m{x}\m{(}\m{x}\m{\in}\m{A}\m{\wedge}\m{\varphi}\m{)}\m{)}
\endm
\begin{mmraw}%
|- ( E. x e. A ph <-> E. x ( x e. A \TAND ph ) ) \$.
\end{mmraw}

\noindent Define the universal class\index{universal class ($V$)}.  Definition
5.20, p.~21, of Takeuti and Zaring.\label{df-v}

\vskip 0.5ex
\setbox\startprefix=\hbox{\tt \ \ df-v\ \$a\ }
\setbox\contprefix=\hbox{\tt \ \ \ \ \ \ \ \ \ \ }
\startm
\m{\vdash}\m{{\rm V}}\m{=}\m{\{}\m{x}\m{|}\m{x}\m{=}\m{x}\m{\}}
\endm
\begin{mmraw}%
|- {\char`\_}V = \{ x | x = x \} \$.
\end{mmraw}

\noindent Define the subclass\index{subclass}\index{subset} relationship
between two classes (called the subset relation if the classes are sets i.e.\
are not proper).  Definition 5.9 of Takeuti and Zaring, p.~17.\label{df-ss}

\vskip 0.5ex
\setbox\startprefix=\hbox{\tt \ \ df-ss\ \$a\ }
\setbox\contprefix=\hbox{\tt \ \ \ \ \ \ \ \ \ \ \ }
\startm
\m{\vdash}\m{(}\m{A}\m{\subseteq}\m{B}\m{\leftrightarrow}\m{\forall}\m{x}\m{(}
\m{x}\m{\in}\m{A}\m{\rightarrow}\m{x}\m{\in}\m{B}\m{)}\m{)}
\endm
\begin{mmraw}%
|- ( A C\_ B <-> A. x ( x e. A -> x e. B ) ) \$.
\end{mmraw}

\noindent Define the union\index{union} of two classes.  Definition 5.6 of Takeuti and Zaring,
p.~16.\label{df-un}

\vskip 0.5ex
\setbox\startprefix=\hbox{\tt \ \ df-un\ \$a\ }
\setbox\contprefix=\hbox{\tt \ \ \ \ \ \ \ \ \ \ \ }
\startm
\m{\vdash}\m{(}\m{A}\m{\cup}\m{B}\m{)}\m{=}\m{\{}\m{x}\m{|}\m{(}\m{x}\m{\in}
\m{A}\m{\vee}\m{x}\m{\in}\m{B}\m{)}\m{\}}
\endm
\begin{mmraw}%
( A u. B ) = \{ x | ( x e. A \TOR x e. B ) \} \$.
\end{mmraw}

\noindent Define the intersection\index{intersection} of two classes.  Definition 5.6 of
Takeuti and Zaring, p.~16.\label{df-in}

\vskip 0.5ex
\setbox\startprefix=\hbox{\tt \ \ df-in\ \$a\ }
\setbox\contprefix=\hbox{\tt \ \ \ \ \ \ \ \ \ \ \ }
\startm
\m{\vdash}\m{(}\m{A}\m{\cap}\m{B}\m{)}\m{=}\m{\{}\m{x}\m{|}\m{(}\m{x}\m{\in}
\m{A}\m{\wedge}\m{x}\m{\in}\m{B}\m{)}\m{\}}
\endm
% Caret ^ requires special treatment
\begin{mmraw}%
|- ( A i\^{}i B ) = \{ x | ( x e. A \TAND x e. B ) \} \$.
\end{mmraw}

\noindent Define class difference\index{class difference}\index{set difference}.
Definition 5.12 of Takeuti and Zaring, p.~20.  Several notations are used in
the literature; we chose the $\setminus$ convention instead of a minus sign to
reserve the latter for later use in, e.g., arithmetic.\label{df-dif}

\vskip 0.5ex
\setbox\startprefix=\hbox{\tt \ \ df-dif\ \$a\ }
\setbox\contprefix=\hbox{\tt \ \ \ \ \ \ \ \ \ \ \ \ }
\startm
\m{\vdash}\m{(}\m{A}\m{\setminus}\m{B}\m{)}\m{=}\m{\{}\m{x}\m{|}\m{(}\m{x}\m{
\in}\m{A}\m{\wedge}\m{\lnot}\m{x}\m{\in}\m{B}\m{)}\m{\}}
\endm
\begin{mmraw}%
( A \SLASH B ) = \{ x | ( x e. A \TAND -. x e. B ) \} \$.
\end{mmraw}

\noindent Define the empty or null set\index{empty set}\index{null set}.
Compare  Definition 5.14 of Takeuti and Zaring, p.~20.\label{df-nul}

\vskip 0.5ex
\setbox\startprefix=\hbox{\tt \ \ df-nul\ \$a\ }
\setbox\contprefix=\hbox{\tt \ \ \ \ \ \ \ \ \ \ }
\startm
\m{\vdash}\m{\varnothing}\m{=}\m{(}\m{{\rm V}}\m{\setminus}\m{{\rm V}}\m{)}
\endm
\begin{mmraw}%
|- (/) = ( {\char`\_}V \SLASH {\char`\_}V ) \$.
\end{mmraw}

\noindent Define power class\index{power set}\index{power class}.  Definition 5.10 of
Takeuti and Zaring, p.~17, but we also let it apply to proper classes.  (Note
that \verb$~P$ is the symbol for calligraphic P, the tilde
suggesting ``curly;'' see Appendix~\ref{ASCII}.)\label{df-pw}

\vskip 0.5ex
\setbox\startprefix=\hbox{\tt \ \ df-pw\ \$a\ }
\setbox\contprefix=\hbox{\tt \ \ \ \ \ \ \ \ \ \ \ }
\startm
\m{\vdash}\m{{\cal P}}\m{A}\m{=}\m{\{}\m{x}\m{|}\m{x}\m{\subseteq}\m{A}\m{\}}
\endm
% Special incantation required to put ~ into the text
\begin{mmraw}%
|- \char`\~P~A = \{ x | x C\_ A \} \$.
\end{mmraw}

\noindent Define the singleton of a class\index{singleton}.  Definition 7.1 of
Quine, p.~48.  It is well-defined for proper classes, although
it is not very meaningful in this case, where it evaluates to the empty
set.

\vskip 0.5ex
\setbox\startprefix=\hbox{\tt \ \ df-sn\ \$a\ }
\setbox\contprefix=\hbox{\tt \ \ \ \ \ \ \ \ \ \ \ }
\startm
\m{\vdash}\m{\{}\m{A}\m{\}}\m{=}\m{\{}\m{x}\m{|}\m{x}\m{=}\m{A}\m{\}}
\endm
\begin{mmraw}%
|- \{ A \} = \{ x | x = A \} \$.
\end{mmraw}%

\noindent Define an unordered pair of classes\index{unordered pair}\index{pair}.  Definition
7.1 of Quine, p.~48.

\vskip 0.5ex
\setbox\startprefix=\hbox{\tt \ \ df-pr\ \$a\ }
\setbox\contprefix=\hbox{\tt \ \ \ \ \ \ \ \ \ \ \ }
\startm
\m{\vdash}\m{\{}\m{A}\m{,}\m{B}\m{\}}\m{=}\m{(}\m{\{}\m{A}\m{\}}\m{\cup}\m{\{}
\m{B}\m{\}}\m{)}
\endm
\begin{mmraw}%
|- \{ A , B \} = ( \{ A \} u. \{ B \} ) \$.
\end{mmraw}

\noindent Define an unordered triple of classes\index{unordered triple}.  Definition of
Enderton, p.~19.

\vskip 0.5ex
\setbox\startprefix=\hbox{\tt \ \ df-tp\ \$a\ }
\setbox\contprefix=\hbox{\tt \ \ \ \ \ \ \ \ \ \ \ }
\startm
\m{\vdash}\m{\{}\m{A}\m{,}\m{B}\m{,}\m{C}\m{\}}\m{=}\m{(}\m{\{}\m{A}\m{,}\m{B}
\m{\}}\m{\cup}\m{\{}\m{C}\m{\}}\m{)}
\endm
\begin{mmraw}%
|- \{ A , B , C \} = ( \{ A , B \} u. \{ C \} ) \$.
\end{mmraw}%

\noindent Kuratowski's\index{Kuratowski, Kazimierz} ordered pair\index{ordered
pair} definition.  Definition 9.1 of Quine, p.~58. For proper classes it is
not meaningful but is well-defined for convenience.  (Note that \verb$<.$
stands for $\langle$ whereas \verb$<$ stands for $<$, and similarly for
\verb$>.$\,.)\label{df-op}

\vskip 0.5ex
\setbox\startprefix=\hbox{\tt \ \ df-op\ \$a\ }
\setbox\contprefix=\hbox{\tt \ \ \ \ \ \ \ \ \ \ \ }
\startm
\m{\vdash}\m{\langle}\m{A}\m{,}\m{B}\m{\rangle}\m{=}\m{\{}\m{\{}\m{A}\m{\}}
\m{,}\m{\{}\m{A}\m{,}\m{B}\m{\}}\m{\}}
\endm
\begin{mmraw}%
|- <. A , B >. = \{ \{ A \} , \{ A , B \} \} \$.
\end{mmraw}

\noindent Define the union of a class\index{union}.  Definition 5.5, p.~16,
of Takeuti and Zaring.

\vskip 0.5ex
\setbox\startprefix=\hbox{\tt \ \ df-uni\ \$a\ }
\setbox\contprefix=\hbox{\tt \ \ \ \ \ \ \ \ \ \ \ \ }
\startm
\m{\vdash}\m{\bigcup}\m{A}\m{=}\m{\{}\m{x}\m{|}\m{\exists}\m{y}\m{(}\m{x}\m{
\in}\m{y}\m{\wedge}\m{y}\m{\in}\m{A}\m{)}\m{\}}
\endm
\begin{mmraw}%
|- U. A = \{ x | E. y ( x e. y \TAND y e. A ) \} \$.
\end{mmraw}

\noindent Define the intersection\index{intersection} of a class.  Definition 7.35,
p.~44, of Takeuti and Zaring.

\vskip 0.5ex
\setbox\startprefix=\hbox{\tt \ \ df-int\ \$a\ }
\setbox\contprefix=\hbox{\tt \ \ \ \ \ \ \ \ \ \ \ \ }
\startm
\m{\vdash}\m{\bigcap}\m{A}\m{=}\m{\{}\m{x}\m{|}\m{\forall}\m{y}\m{(}\m{y}\m{
\in}\m{A}\m{\rightarrow}\m{x}\m{\in}\m{y}\m{)}\m{\}}
\endm
\begin{mmraw}%
|- |\^{}| A = \{ x | A. y ( y e. A -> x e. y ) \} \$.
\end{mmraw}

\noindent Define a transitive class\index{transitive class}\index{transitive
set}.  This should not be confused with a transitive relation, which is a different
concept.  Definition from p.~71 of Enderton, extended to classes.

\vskip 0.5ex
\setbox\startprefix=\hbox{\tt \ \ df-tr\ \$a\ }
\setbox\contprefix=\hbox{\tt \ \ \ \ \ \ \ \ \ \ \ }
\startm
\m{\vdash}\m{(}\m{\mbox{\rm Tr}}\m{A}\m{\leftrightarrow}\m{\bigcup}\m{A}\m{
\subseteq}\m{A}\m{)}
\endm
\begin{mmraw}%
|- ( Tr A <-> U. A C\_ A ) \$.
\end{mmraw}

\noindent Define a notation for a general binary relation\index{binary
relation}.  Definition 6.18, p.~29, of Takeuti and Zaring, generalized to
arbitrary classes.  This definition is well-defined, although not very
meaningful, when classes $A$ and/or $B$ are proper.\label{dfbr}  The lack of
parentheses (or any other connective) creates no ambiguity since we are defining
an atomic wff.

\vskip 0.5ex
\setbox\startprefix=\hbox{\tt \ \ df-br\ \$a\ }
\setbox\contprefix=\hbox{\tt \ \ \ \ \ \ \ \ \ \ \ }
\startm
\m{\vdash}\m{(}\m{A}\m{\,R}\m{\,B}\m{\leftrightarrow}\m{\langle}\m{A}\m{,}\m{B}
\m{\rangle}\m{\in}\m{R}\m{)}
\endm
\begin{mmraw}%
|- ( A R B <-> <. A , B >. e. R ) \$.
\end{mmraw}

\noindent Define an abstraction class of ordered pairs\index{abstraction
class!of ordered
pairs}.  A special case of Definition 4.16, p.~14, of Takeuti and Zaring.
Note that $ z $ must be distinct from $ x $ and $ y $,
and $ z $ must not occur in $\varphi$, but $ x $ and $ y $ may be
identical and may appear in $\varphi$.

\vskip 0.5ex
\setbox\startprefix=\hbox{\tt \ \ df-opab\ \$a\ }
\setbox\contprefix=\hbox{\tt \ \ \ \ \ \ \ \ \ \ \ \ \ }
\startm
\m{\vdash}\m{\{}\m{\langle}\m{x}\m{,}\m{y}\m{\rangle}\m{|}\m{\varphi}\m{\}}\m{=}
\m{\{}\m{z}\m{|}\m{\exists}\m{x}\m{\exists}\m{y}\m{(}\m{z}\m{=}\m{\langle}\m{x}
\m{,}\m{y}\m{\rangle}\m{\wedge}\m{\varphi}\m{)}\m{\}}
\endm

\begin{mmraw}%
|- \{ <. x , y >. | ph \} = \{ z | E. x E. y ( z =
<. x , y >. /\ ph ) \} \$.
\end{mmraw}

\noindent Define the epsilon relation\index{epsilon relation}.  Similar to Definition
6.22, p.~30, of Takeuti and Zaring.

\vskip 0.5ex
\setbox\startprefix=\hbox{\tt \ \ df-eprel\ \$a\ }
\setbox\contprefix=\hbox{\tt \ \ \ \ \ \ \ \ \ \ \ \ \ \ }
\startm
\m{\vdash}\m{{\rm E}}\m{=}\m{\{}\m{\langle}\m{x}\m{,}\m{y}\m{\rangle}\m{|}\m{x}\m{
\in}\m{y}\m{\}}
\endm
\begin{mmraw}%
|- \_E = \{ <. x , y >. | x e. y \} \$.
\end{mmraw}

\noindent Define a founded relation\index{founded relation}.  $R$ is a founded
relation on $A$ iff\index{iff} (if and only if) each nonempty subset of $A$
has an ``$R$-minimal element.''  Similar to Definition 6.21, p.~30, of
Takeuti and Zaring.

\vskip 0.5ex
\setbox\startprefix=\hbox{\tt \ \ df-fr\ \$a\ }
\setbox\contprefix=\hbox{\tt \ \ \ \ \ \ \ \ \ \ \ }
\startm
\m{\vdash}\m{(}\m{R}\m{\,\mbox{\rm Fr}}\m{\,A}\m{\leftrightarrow}\m{\forall}\m{x}
\m{(}\m{(}\m{x}\m{\subseteq}\m{A}\m{\wedge}\m{\lnot}\m{x}\m{=}\m{\varnothing}
\m{)}\m{\rightarrow}\m{\exists}\m{y}\m{(}\m{y}\m{\in}\m{x}\m{\wedge}\m{(}\m{x}
\m{\cap}\m{\{}\m{z}\m{|}\m{z}\m{\,R}\m{\,y}\m{\}}\m{)}\m{=}\m{\varnothing}\m{)}
\m{)}\m{)}
\endm
\begin{mmraw}%
|- ( R Fr A <-> A. x ( ( x C\_ A \TAND -. x = (/) ) ->
E. y ( y e. x \TAND ( x i\^{}i \{ z | z R y \} ) = (/) ) ) ) \$.
\end{mmraw}

\noindent Define a well-ordering\index{well-ordering}.  $R$ is a well-ordering of $A$ iff
it is founded on $A$ and the elements of $A$ are pairwise $R$-comparable.
Similar to Definition 6.24(2), p.~30, of Takeuti and Zaring.

\vskip 0.5ex
\setbox\startprefix=\hbox{\tt \ \ df-we\ \$a\ }
\setbox\contprefix=\hbox{\tt \ \ \ \ \ \ \ \ \ \ \ }
\startm
\m{\vdash}\m{(}\m{R}\m{\,\mbox{\rm We}}\m{\,A}\m{\leftrightarrow}\m{(}\m{R}\m{\,
\mbox{\rm Fr}}\m{\,A}\m{\wedge}\m{\forall}\m{x}\m{\forall}\m{y}\m{(}\m{(}\m{x}\m{
\in}\m{A}\m{\wedge}\m{y}\m{\in}\m{A}\m{)}\m{\rightarrow}\m{(}\m{x}\m{\,R}\m{\,y}
\m{\vee}\m{x}\m{=}\m{y}\m{\vee}\m{y}\m{\,R}\m{\,x}\m{)}\m{)}\m{)}\m{)}
\endm
\begin{mmraw}%
( R We A <-> ( R Fr A \TAND A. x A. y ( ( x e.
A \TAND y e. A ) -> ( x R y \TOR x = y \TOR y R x ) ) ) ) \$.
\end{mmraw}

\noindent Define the ordinal predicate\index{ordinal predicate}, which is true for a class
that is transitive and is well-ordered by the epsilon relation.  Similar to
definition on p.~468, Bell and Machover.

\vskip 0.5ex
\setbox\startprefix=\hbox{\tt \ \ df-ord\ \$a\ }
\setbox\contprefix=\hbox{\tt \ \ \ \ \ \ \ \ \ \ \ \ }
\startm
\m{\vdash}\m{(}\m{\mbox{\rm Ord}}\m{\,A}\m{\leftrightarrow}\m{(}
\m{\mbox{\rm Tr}}\m{\,A}\m{\wedge}\m{E}\m{\,\mbox{\rm We}}\m{\,A}\m{)}\m{)}
\endm
\begin{mmraw}%
|- ( Ord A <-> ( Tr A \TAND E We A ) ) \$.
\end{mmraw}

\noindent Define the class of all ordinal numbers\index{ordinal number}.  An ordinal number is
a set that satisfies the ordinal predicate.  Definition 7.11 of Takeuti and
Zaring, p.~38.

\vskip 0.5ex
\setbox\startprefix=\hbox{\tt \ \ df-on\ \$a\ }
\setbox\contprefix=\hbox{\tt \ \ \ \ \ \ \ \ \ \ \ }
\startm
\m{\vdash}\m{\,\mbox{\rm On}}\m{=}\m{\{}\m{x}\m{|}\m{\mbox{\rm Ord}}\m{\,x}
\m{\}}
\endm
\begin{mmraw}%
|- On = \{ x | Ord x \} \$.
\end{mmraw}

\noindent Define the limit ordinal predicate\index{limit ordinal}, which is true for a
non-empty ordinal that is not a successor (i.e.\ that is the union of itself).
Compare Bell and Machover, p.~471 and Exercise (1), p.~42 of Takeuti and
Zaring.

\vskip 0.5ex
\setbox\startprefix=\hbox{\tt \ \ df-lim\ \$a\ }
\setbox\contprefix=\hbox{\tt \ \ \ \ \ \ \ \ \ \ \ \ }
\startm
\m{\vdash}\m{(}\m{\mbox{\rm Lim}}\m{\,A}\m{\leftrightarrow}\m{(}\m{\mbox{
\rm Ord}}\m{\,A}\m{\wedge}\m{\lnot}\m{A}\m{=}\m{\varnothing}\m{\wedge}\m{A}
\m{=}\m{\bigcup}\m{A}\m{)}\m{)}
\endm
\begin{mmraw}%
|- ( Lim A <-> ( Ord A \TAND -. A = (/) \TAND A = U. A ) ) \$.
\end{mmraw}

\noindent Define the successor\index{successor} of a class.  Definition 7.22 of Takeuti
and Zaring, p.~41.  Our definition is a generalization to classes, although it
is meaningless when classes are proper.

\vskip 0.5ex
\setbox\startprefix=\hbox{\tt \ \ df-suc\ \$a\ }
\setbox\contprefix=\hbox{\tt \ \ \ \ \ \ \ \ \ \ \ \ }
\startm
\m{\vdash}\m{\,\mbox{\rm suc}}\m{\,A}\m{=}\m{(}\m{A}\m{\cup}\m{\{}\m{A}\m{\}}
\m{)}
\endm
\begin{mmraw}%
|- suc A = ( A u. \{ A \} ) \$.
\end{mmraw}

\noindent Define the class of natural numbers\index{natural number}\index{omega
($\omega$)}.  Compare Bell and Machover, p.~471.\label{dfom}

\vskip 0.5ex
\setbox\startprefix=\hbox{\tt \ \ df-om\ \$a\ }
\setbox\contprefix=\hbox{\tt \ \ \ \ \ \ \ \ \ \ \ }
\startm
\m{\vdash}\m{\omega}\m{=}\m{\{}\m{x}\m{|}\m{(}\m{\mbox{\rm Ord}}\m{\,x}\m{
\wedge}\m{\forall}\m{y}\m{(}\m{\mbox{\rm Lim}}\m{\,y}\m{\rightarrow}\m{x}\m{
\in}\m{y}\m{)}\m{)}\m{\}}
\endm
\begin{mmraw}%
|- om = \{ x | ( Ord x \TAND A. y ( Lim y -> x e. y ) ) \} \$.
\end{mmraw}

\noindent Define the Cartesian product (also called the
cross product)\index{Cartesian product}\index{cross product}
of two classes.  Definition 9.11 of Quine, p.~64.

\vskip 0.5ex
\setbox\startprefix=\hbox{\tt \ \ df-xp\ \$a\ }
\setbox\contprefix=\hbox{\tt \ \ \ \ \ \ \ \ \ \ \ }
\startm
\m{\vdash}\m{(}\m{A}\m{\times}\m{B}\m{)}\m{=}\m{\{}\m{\langle}\m{x}\m{,}\m{y}
\m{\rangle}\m{|}\m{(}\m{x}\m{\in}\m{A}\m{\wedge}\m{y}\m{\in}\m{B}\m{)}\m{\}}
\endm
\begin{mmraw}%
|- ( A X. B ) = \{ <. x , y >. | ( x e. A \TAND y e. B) \} \$.
\end{mmraw}

\noindent Define a relation\index{relation}.  Definition 6.4(1) of Takeuti and
Zaring, p.~23.

\vskip 0.5ex
\setbox\startprefix=\hbox{\tt \ \ df-rel\ \$a\ }
\setbox\contprefix=\hbox{\tt \ \ \ \ \ \ \ \ \ \ \ \ }
\startm
\m{\vdash}\m{(}\m{\mbox{\rm Rel}}\m{\,A}\m{\leftrightarrow}\m{A}\m{\subseteq}
\m{(}\m{{\rm V}}\m{\times}\m{{\rm V}}\m{)}\m{)}
\endm
\begin{mmraw}%
|- ( Rel A <-> A C\_ ( {\char`\_}V X. {\char`\_}V ) ) \$.
\end{mmraw}

\noindent Define the domain\index{domain} of a class.  Definition 6.5(1) of
Takeuti and Zaring, p.~24.

\vskip 0.5ex
\setbox\startprefix=\hbox{\tt \ \ df-dm\ \$a\ }
\setbox\contprefix=\hbox{\tt \ \ \ \ \ \ \ \ \ \ \ }
\startm
\m{\vdash}\m{\,\mbox{\rm dom}}\m{A}\m{=}\m{\{}\m{x}\m{|}\m{\exists}\m{y}\m{
\langle}\m{x}\m{,}\m{y}\m{\rangle}\m{\in}\m{A}\m{\}}
\endm
\begin{mmraw}%
|- dom A = \{ x | E. y <. x , y >. e. A \} \$.
\end{mmraw}

\noindent Define the range\index{range} of a class.  Definition 6.5(2) of
Takeuti and Zaring, p.~24.

\vskip 0.5ex
\setbox\startprefix=\hbox{\tt \ \ df-rn\ \$a\ }
\setbox\contprefix=\hbox{\tt \ \ \ \ \ \ \ \ \ \ \ }
\startm
\m{\vdash}\m{\,\mbox{\rm ran}}\m{A}\m{=}\m{\{}\m{y}\m{|}\m{\exists}\m{x}\m{
\langle}\m{x}\m{,}\m{y}\m{\rangle}\m{\in}\m{A}\m{\}}
\endm
\begin{mmraw}%
|- ran A = \{ y | E. x <. x , y >. e. A \} \$.
\end{mmraw}

\noindent Define the restriction\index{restriction} of a class.  Definition
6.6(1) of Takeuti and Zaring, p.~24.

\vskip 0.5ex
\setbox\startprefix=\hbox{\tt \ \ df-res\ \$a\ }
\setbox\contprefix=\hbox{\tt \ \ \ \ \ \ \ \ \ \ \ \ }
\startm
\m{\vdash}\m{(}\m{A}\m{\restriction}\m{B}\m{)}\m{=}\m{(}\m{A}\m{\cap}\m{(}\m{B}
\m{\times}\m{{\rm V}}\m{)}\m{)}
\endm
\begin{mmraw}%
|- ( A |` B ) = ( A i\^{}i ( B X. {\char`\_}V ) ) \$.
\end{mmraw}

\noindent Define the image\index{image} of a class.  Definition 6.6(2) of
Takeuti and Zaring, p.~24.

\vskip 0.5ex
\setbox\startprefix=\hbox{\tt \ \ df-ima\ \$a\ }
\setbox\contprefix=\hbox{\tt \ \ \ \ \ \ \ \ \ \ \ \ }
\startm
\m{\vdash}\m{(}\m{A}\m{``}\m{B}\m{)}\m{=}\m{\,\mbox{\rm ran}}\m{\,(}\m{A}\m{
\restriction}\m{B}\m{)}
\endm
\begin{mmraw}%
|- ( A " B ) = ran ( A |` B ) \$.
\end{mmraw}

\noindent Define the composition\index{composition} of two classes.  Definition 6.6(3) of
Takeuti and Zaring, p.~24.

\vskip 0.5ex
\setbox\startprefix=\hbox{\tt \ \ df-co\ \$a\ }
\setbox\contprefix=\hbox{\tt \ \ \ \ \ \ \ \ \ \ \ \ }
\startm
\m{\vdash}\m{(}\m{A}\m{\circ}\m{B}\m{)}\m{=}\m{\{}\m{\langle}\m{x}\m{,}\m{y}\m{
\rangle}\m{|}\m{\exists}\m{z}\m{(}\m{\langle}\m{x}\m{,}\m{z}\m{\rangle}\m{\in}
\m{B}\m{\wedge}\m{\langle}\m{z}\m{,}\m{y}\m{\rangle}\m{\in}\m{A}\m{)}\m{\}}
\endm
\begin{mmraw}%
|- ( A o. B ) = \{ <. x , y >. | E. z ( <. x , z
>. e. B \TAND <. z , y >. e. A ) \} \$.
\end{mmraw}

\noindent Define a function\index{function}.  Definition 6.4(4) of Takeuti and
Zaring, p.~24.

\vskip 0.5ex
\setbox\startprefix=\hbox{\tt \ \ df-fun\ \$a\ }
\setbox\contprefix=\hbox{\tt \ \ \ \ \ \ \ \ \ \ \ \ }
\startm
\m{\vdash}\m{(}\m{\mbox{\rm Fun}}\m{\,A}\m{\leftrightarrow}\m{(}
\m{\mbox{\rm Rel}}\m{\,A}\m{\wedge}
\m{\forall}\m{x}\m{\exists}\m{z}\m{\forall}\m{y}\m{(}
\m{\langle}\m{x}\m{,}\m{y}\m{\rangle}\m{\in}\m{A}\m{\rightarrow}\m{y}\m{=}\m{z}
\m{)}\m{)}\m{)}
\endm
\begin{mmraw}%
|- ( Fun A <-> ( Rel A /\ A. x E. z A. y ( <. x
   , y >. e. A -> y = z ) ) ) \$.
\end{mmraw}

\noindent Define a function with domain.  Definition 6.15(1) of Takeuti and
Zaring, p.~27.

\vskip 0.5ex
\setbox\startprefix=\hbox{\tt \ \ df-fn\ \$a\ }
\setbox\contprefix=\hbox{\tt \ \ \ \ \ \ \ \ \ \ \ }
\startm
\m{\vdash}\m{(}\m{A}\m{\,\mbox{\rm Fn}}\m{\,B}\m{\leftrightarrow}\m{(}
\m{\mbox{\rm Fun}}\m{\,A}\m{\wedge}\m{\mbox{\rm dom}}\m{\,A}\m{=}\m{B}\m{)}
\m{)}
\endm
\begin{mmraw}%
|- ( A Fn B <-> ( Fun A \TAND dom A = B ) ) \$.
\end{mmraw}

\noindent Define a function with domain and co-domain.  Definition 6.15(3)
of Takeuti and Zaring, p.~27.

\vskip 0.5ex
\setbox\startprefix=\hbox{\tt \ \ df-f\ \$a\ }
\setbox\contprefix=\hbox{\tt \ \ \ \ \ \ \ \ \ \ }
\startm
\m{\vdash}\m{(}\m{F}\m{:}\m{A}\m{\longrightarrow}\m{B}\m{
\leftrightarrow}\m{(}\m{F}\m{\,\mbox{\rm Fn}}\m{\,A}\m{\wedge}\m{
\mbox{\rm ran}}\m{\,F}\m{\subseteq}\m{B}\m{)}\m{)}
\endm
\begin{mmraw}%
|- ( F : A --> B <-> ( F Fn A \TAND ran F C\_ B ) ) \$.
\end{mmraw}

\noindent Define a one-to-one function\index{one-to-one function}.  Compare
Definition 6.15(5) of Takeuti and Zaring, p.~27.

\vskip 0.5ex
\setbox\startprefix=\hbox{\tt \ \ df-f1\ \$a\ }
\setbox\contprefix=\hbox{\tt \ \ \ \ \ \ \ \ \ \ \ }
\startm
\m{\vdash}\m{(}\m{F}\m{:}\m{A}\m{
\raisebox{.5ex}{${\textstyle{\:}_{\mbox{\footnotesize\rm
1\tt -\rm 1}}}\atop{\textstyle{
\longrightarrow}\atop{\textstyle{}^{\mbox{\footnotesize\rm {\ }}}}}$}
}\m{B}
\m{\leftrightarrow}\m{(}\m{F}\m{:}\m{A}\m{\longrightarrow}\m{B}
\m{\wedge}\m{\forall}\m{y}\m{\exists}\m{z}\m{\forall}\m{x}\m{(}\m{\langle}\m{x}
\m{,}\m{y}\m{\rangle}\m{\in}\m{F}\m{\rightarrow}\m{x}\m{=}\m{z}\m{)}\m{)}\m{)}
\endm
\begin{mmraw}%
|- ( F : A -1-1-> B <-> ( F : A --> B \TAND
   A. y E. z A. x ( <. x , y >. e. F -> x = z ) ) ) \$.
\end{mmraw}

\noindent Define an onto function\index{onto function}.  Definition 6.15(4) of Takeuti and
Zaring, p.~27.

\vskip 0.5ex
\setbox\startprefix=\hbox{\tt \ \ df-fo\ \$a\ }
\setbox\contprefix=\hbox{\tt \ \ \ \ \ \ \ \ \ \ \ }
\startm
\m{\vdash}\m{(}\m{F}\m{:}\m{A}\m{
\raisebox{.5ex}{${\textstyle{\:}_{\mbox{\footnotesize\rm
{\ }}}}\atop{\textstyle{
\longrightarrow}\atop{\textstyle{}^{\mbox{\footnotesize\rm onto}}}}$}
}\m{B}
\m{\leftrightarrow}\m{(}\m{F}\m{\,\mbox{\rm Fn}}\m{\,A}\m{\wedge}
\m{\mbox{\rm ran}}\m{\,F}\m{=}\m{B}\m{)}\m{)}
\endm
\begin{mmraw}%
|- ( F : A -onto-> B <-> ( F Fn A /\ ran F = B ) ) \$.
\end{mmraw}

\noindent Define a one-to-one, onto function.  Compare Definition 6.15(6) of
Takeuti and Zaring, p.~27.

\vskip 0.5ex
\setbox\startprefix=\hbox{\tt \ \ df-f1o\ \$a\ }
\setbox\contprefix=\hbox{\tt \ \ \ \ \ \ \ \ \ \ \ \ }
\startm
\m{\vdash}\m{(}\m{F}\m{:}\m{A}
\m{
\raisebox{.5ex}{${\textstyle{\:}_{\mbox{\footnotesize\rm
1\tt -\rm 1}}}\atop{\textstyle{
\longrightarrow}\atop{\textstyle{}^{\mbox{\footnotesize\rm onto}}}}$}
}
\m{B}
\m{\leftrightarrow}\m{(}\m{F}\m{:}\m{A}
\m{
\raisebox{.5ex}{${\textstyle{\:}_{\mbox{\footnotesize\rm
1\tt -\rm 1}}}\atop{\textstyle{
\longrightarrow}\atop{\textstyle{}^{\mbox{\footnotesize\rm {\ }}}}}$}
}
\m{B}\m{\wedge}\m{F}\m{:}\m{A}
\m{
\raisebox{.5ex}{${\textstyle{\:}_{\mbox{\footnotesize\rm
{\ }}}}\atop{\textstyle{
\longrightarrow}\atop{\textstyle{}^{\mbox{\footnotesize\rm onto}}}}$}
}
\m{B}\m{)}\m{)}
\endm
\begin{mmraw}%
|- ( F : A -1-1-onto-> B <-> ( F : A -1-1-> B? \TAND F : A -onto-> B ) ) \$.?
\end{mmraw}

\noindent Define the value of a function\index{function value}.  This
definition applies to any class and evaluates to the empty set when it is not
meaningful. Note that $ F`A$ means the same thing as the more familiar $ F(A)$
notation for a function's value at $A$.  The $ F`A$ notation is common in
formal set theory.\label{df-fv}

\vskip 0.5ex
\setbox\startprefix=\hbox{\tt \ \ df-fv\ \$a\ }
\setbox\contprefix=\hbox{\tt \ \ \ \ \ \ \ \ \ \ \ }
\startm
\m{\vdash}\m{(}\m{F}\m{`}\m{A}\m{)}\m{=}\m{\bigcup}\m{\{}\m{x}\m{|}\m{(}\m{F}%
\m{``}\m{\{}\m{A}\m{\}}\m{)}\m{=}\m{\{}\m{x}\m{\}}\m{\}}
\endm
\begin{mmraw}%
|- ( F ` A ) = U. \{ x | ( F " \{ A \} ) = \{ x \} \} \$.
\end{mmraw}

\noindent Define the result of an operation.\index{operation}  Here, $F$ is
     an operation on two
     values (such as $+$ for real numbers).   This is defined for proper
     classes $A$ and $B$ even though not meaningful in that case.  However,
     the definition can be meaningful when $F$ is a proper class.\label{dfopr}

\vskip 0.5ex
\setbox\startprefix=\hbox{\tt \ \ df-opr\ \$a\ }
\setbox\contprefix=\hbox{\tt \ \ \ \ \ \ \ \ \ \ \ \ }
\startm
\m{\vdash}\m{(}\m{A}\m{\,F}\m{\,B}\m{)}\m{=}\m{(}\m{F}\m{`}\m{\langle}\m{A}%
\m{,}\m{B}\m{\rangle}\m{)}
\endm
\begin{mmraw}%
|- ( A F B ) = ( F ` <. A , B >. ) \$.
\end{mmraw}

\section{Tricks of the Trade}\label{tricks}

In the \texttt{set.mm}\index{set theory database (\texttt{set.mm})} database our goal
was usually to conform to modern notation.  However in some cases the
relationship to standard textbook language may be obscured by several
unconventional devices we used to simplify the development and to take
advantage of the Metamath language.  In this section we will describe some
common conventions used in \texttt{set.mm}.

\begin{itemize}
\item
The turnstile\index{turnstile ({$\,\vdash$})} symbol, $\vdash$, meaning ``it
is provable that,'' is the first token of all assertions and hypotheses that
aren't syntax constructions.  This is a standard convention in logic.  (We
mentioned this earlier, but this symbol is bothersome to some people without a
logic background.  It has no deeper meaning but just provides us with a way to
distinguish syntax constructions from ordinary mathematical statements.)

\item
A hypothesis of the form

\vskip 1ex
\setbox\startprefix=\hbox{\tt \ \ \ \ \ \ \ \ \ \$e\ }
\setbox\contprefix=\hbox{\tt \ \ \ \ \ \ \ \ \ \ }
\startm
\m{\vdash}\m{(}\m{\varphi}\m{\rightarrow}\m{\forall}\m{x}\m{\varphi}\m{)}
\endm
\vskip 1ex

should be read ``assume variable $x$ is (effectively) not free in wff
$\varphi$.''\index{effectively not free}
Literally, this says ``assume it is provable that $\varphi \rightarrow \forall
x\, \varphi$.''  This device lets us avoid the complexities associated with
the standard treatment of free and bound variables.
%% Uncomment this when uncommenting section {formalspec} below
The footnote on p.~\pageref{effectivelybound} discusses this further.

\item
A statement of one of the forms

\vskip 1ex
\setbox\startprefix=\hbox{\tt \ \ \ \ \ \ \ \ \ \$a\ }
\setbox\contprefix=\hbox{\tt \ \ \ \ \ \ \ \ \ \ }
\startm
\m{\vdash}\m{(}\m{\lnot}\m{\forall}\m{x}\m{\,x}\m{=}\m{y}\m{\rightarrow}
\m{\ldots}\m{)}
\endm
\setbox\startprefix=\hbox{\tt \ \ \ \ \ \ \ \ \ \$p\ }
\setbox\contprefix=\hbox{\tt \ \ \ \ \ \ \ \ \ \ }
\startm
\m{\vdash}\m{(}\m{\lnot}\m{\forall}\m{x}\m{\,x}\m{=}\m{y}\m{\rightarrow}
\m{\ldots}\m{)}
\endm
\vskip 1ex

should be read ``if $x$ and $y$ are distinct variables, then...''  This
antecedent provides us with a technical device to avoid the need for the
\texttt{\$d} statement early in our development of predicate calculus,
permitting symbol manipulations to be as conceptually simple as those in
propositional calculus.  However, the \texttt{\$d} statement eventually
becomes a requirement, and after that this device is rarely used.

\item
The statement

\vskip 1ex
\setbox\startprefix=\hbox{\tt \ \ \ \ \ \ \ \ \ \$d\ }
\setbox\contprefix=\hbox{\tt \ \ \ \ \ \ \ \ \ \ }
\startm
\m{x}\m{\,y}
\endm
\vskip 1ex

should be read ``assume $x$ and $y$ are distinct variables.''

\item
The statement

\vskip 1ex
\setbox\startprefix=\hbox{\tt \ \ \ \ \ \ \ \ \ \$d\ }
\setbox\contprefix=\hbox{\tt \ \ \ \ \ \ \ \ \ \ }
\startm
\m{x}\m{\,\varphi}
\endm
\vskip 1ex

should be read ``assume $x$ does not occur in $\varphi$.''

\item
The statement

\vskip 1ex
\setbox\startprefix=\hbox{\tt \ \ \ \ \ \ \ \ \ \$d\ }
\setbox\contprefix=\hbox{\tt \ \ \ \ \ \ \ \ \ \ }
\startm
\m{x}\m{\,A}
\endm
\vskip 1ex

should be read ``assume variable $x$ does not occur in class $A$.''

\item
The restriction and hypothesis group

\vskip 1ex
\setbox\startprefix=\hbox{\tt \ \ \ \ \ \ \ \ \ \$d\ }
\setbox\contprefix=\hbox{\tt \ \ \ \ \ \ \ \ \ \ }
\startm
\m{x}\m{\,A}
\endm
\setbox\startprefix=\hbox{\tt \ \ \ \ \ \ \ \ \ \$d\ }
\setbox\contprefix=\hbox{\tt \ \ \ \ \ \ \ \ \ \ }
\startm
\m{x}\m{\,\psi}
\endm
\setbox\startprefix=\hbox{\tt \ \ \ \ \ \ \ \ \ \$e\ }
\setbox\contprefix=\hbox{\tt \ \ \ \ \ \ \ \ \ \ }
\startm
\m{\vdash}\m{(}\m{x}\m{=}\m{A}\m{\rightarrow}\m{(}\m{\varphi}\m{\leftrightarrow}
\m{\psi}\m{)}\m{)}
\endm
\vskip 1ex

is frequently used in place of explicit substitution, meaning ``assume
$\psi$ results from the proper substitution of $A$ for $x$ in
$\varphi$.''  Sometimes ``\texttt{\$e} $\vdash ( \psi \rightarrow
\forall x \, \psi )$'' is used instead of ``\texttt{\$d} $x\, \psi $,''
which requires only that $x$ be effectively not free in $\varphi$ but
not necessarily absent from it.  The use of implicit
substitution\index{substitution!implicit} is partly a matter of personal
style, although it may make proofs somewhat shorter than would be the
case with explicit substitution.

\item
The hypothesis


\vskip 1ex
\setbox\startprefix=\hbox{\tt \ \ \ \ \ \ \ \ \ \$e\ }
\setbox\contprefix=\hbox{\tt \ \ \ \ \ \ \ \ \ \ }
\startm
\m{\vdash}\m{A}\m{\in}\m{{\rm V}}
\endm
\vskip 1ex

should be read ``assume class $A$ is a set (i.e.\ exists).''
This is a convenient convention used by Quine.

\item
The restriction and hypothesis

\vskip 1ex
\setbox\startprefix=\hbox{\tt \ \ \ \ \ \ \ \ \ \$d\ }
\setbox\contprefix=\hbox{\tt \ \ \ \ \ \ \ \ \ \ }
\startm
\m{x}\m{\,y}
\endm
\setbox\startprefix=\hbox{\tt \ \ \ \ \ \ \ \ \ \$e\ }
\setbox\contprefix=\hbox{\tt \ \ \ \ \ \ \ \ \ \ }
\startm
\m{\vdash}\m{(}\m{y}\m{\in}\m{A}\m{\rightarrow}\m{\forall}\m{x}\m{\,y}
\m{\in}\m{A}\m{)}
\endm
\vskip 1ex

should be read ``assume variable $x$ is
(effectively) not free in class $A$.''

\end{itemize}

\section{A Theorem Sampler}\label{sometheorems}

In this section we list some of the more important theorems that are proved in
the \texttt{set.mm} database, and they illustrate the kinds of things that can be
done with Metamath.  While all of these facts are well-known results,
Metamath offers the advantage of easily allowing you to trace their
derivation back to axioms.  Our intent here is not to try to explain the
details or motivation; for this we refer you to the textbooks that are
mentioned in the descriptions.  (The \texttt{set.mm} file has bibliographic
references for the text references.)  Their proofs often embody important
concepts you may wish to explore with the Metamath program (see
Section~\ref{exploring}).  All the symbols that are used here are defined in
Section~\ref{hierarchy}.  For brevity we haven't included the \texttt{\$d}
restrictions or \texttt{\$f} hypotheses for these theorems; when you are
uncertain consult the \texttt{set.mm} database.

We start with \texttt{syl} (principle of the syllogism).
In \textit{Principia Mathematica}
Whitehead and Russell call this ``the principle of the
syllogism... because... the syllogism in Barbara is derived from them''
\cite[quote after Theorem *2.06 p.~101]{PM}.
Some authors call this law a ``hypothetical syllogism.''
As of 2019 \texttt{syl} is the most commonly referenced proven
assertion in the \texttt{set.mm} database.\footnote{
The Metamath program command \texttt{show usage}
shows the number of references.
On 2019-04-29 (commit 71cbbbdb387e) \texttt{syl} was directly referenced
10,819 times. The second most commonly referenced proven assertion
was \texttt{eqid}, which was directly referenced 7,738 times.
}

\vskip 2ex
\noindent Theorem syl (principle of the syllogism)\index{Syllogism}%
\index{\texttt{syl}}\label{syl}.

\vskip 0.5ex
\setbox\startprefix=\hbox{\tt \ \ syl.1\ \$e\ }
\setbox\contprefix=\hbox{\tt \ \ \ \ \ \ \ \ \ \ \ }
\startm
\m{\vdash}\m{(}\m{\varphi}\m{ \rightarrow }\m{\psi}\m{)}
\endm
\setbox\startprefix=\hbox{\tt \ \ syl.2\ \$e\ }
\setbox\contprefix=\hbox{\tt \ \ \ \ \ \ \ \ \ \ \ }
\startm
\m{\vdash}\m{(}\m{\psi}\m{ \rightarrow }\m{\chi}\m{)}
\endm
\setbox\startprefix=\hbox{\tt \ \ syl\ \$p\ }
\setbox\contprefix=\hbox{\tt \ \ \ \ \ \ \ \ \ }
\startm
\m{\vdash}\m{(}\m{\varphi}\m{ \rightarrow }\m{\chi}\m{)}
\endm
\vskip 2ex

The following theorem is not very deep but provides us with a notational device
that is frequently used.  It allows us to use the expression ``$A \in V$'' as
a compact way of saying that class $A$ exists, i.e.\ is a set.

\vskip 2ex
\noindent Two ways to say ``$A$ is a set'':  $A$ is a member of the universe
$V$ if and only if $A$ exists (i.e.\ there exists a set equal to $A$).
Theorem 6.9 of Quine, p. 43.

\vskip 0.5ex
\setbox\startprefix=\hbox{\tt \ \ isset\ \$p\ }
\setbox\contprefix=\hbox{\tt \ \ \ \ \ \ \ \ \ \ \ }
\startm
\m{\vdash}\m{(}\m{A}\m{\in}\m{{\rm V}}\m{\leftrightarrow}\m{\exists}\m{x}\m{\,x}\m{=}
\m{A}\m{)}
\endm
\vskip 1ex

Next we prove the axioms of standard ZF set theory that were missing from our
axiom system.  From our point of view they are theorems since they
can be derived from the other axioms.

\vskip 2ex
\noindent Axiom of Separation\index{Axiom of Separation}
(Aussonderung)\index{Aussonderung} proved from the other axioms of ZF set
theory.  Compare Exercise 4 of Takeuti and Zaring, p.~22.

\vskip 0.5ex
\setbox\startprefix=\hbox{\tt \ \ inex1.1\ \$e\ }
\setbox\contprefix=\hbox{\tt \ \ \ \ \ \ \ \ \ \ \ \ \ \ \ }
\startm
\m{\vdash}\m{A}\m{\in}\m{{\rm V}}
\endm
\setbox\startprefix=\hbox{\tt \ \ inex\ \$p\ }
\setbox\contprefix=\hbox{\tt \ \ \ \ \ \ \ \ \ \ \ \ \ }
\startm
\m{\vdash}\m{(}\m{A}\m{\cap}\m{B}\m{)}\m{\in}\m{{\rm V}}
\endm
\vskip 1ex

\noindent Axiom of the Null Set\index{Axiom of the Null Set} proved from the
other axioms of ZF set theory. Corollary 5.16 of Takeuti and Zaring, p.~20.

\vskip 0.5ex
\setbox\startprefix=\hbox{\tt \ \ 0ex\ \$p\ }
\setbox\contprefix=\hbox{\tt \ \ \ \ \ \ \ \ \ \ \ \ }
\startm
\m{\vdash}\m{\varnothing}\m{\in}\m{{\rm V}}
\endm
\vskip 1ex

\noindent The Axiom of Pairing\index{Axiom of Pairing} proved from the other
axioms of ZF set theory.  Theorem 7.13 of Quine, p.~51.
\vskip 0.5ex
\setbox\startprefix=\hbox{\tt \ \ prex\ \$p\ }
\setbox\contprefix=\hbox{\tt \ \ \ \ \ \ \ \ \ \ \ \ \ \ }
\startm
\m{\vdash}\m{\{}\m{A}\m{,}\m{B}\m{\}}\m{\in}\m{{\rm V}}
\endm
\vskip 2ex

Next we will list some famous or important theorems that are proved in
the \texttt{set.mm} database.  None of them except \texttt{omex}
require the Axiom of Infinity, as you can verify with the \texttt{show
trace{\char`\_}back} Metamath command.

\vskip 2ex
\noindent The resolution of Russell's paradox\index{Russell's paradox}.  There
exists no set containing the set of all sets which are not members of
themselves.  Proposition 4.14 of Takeuti and Zaring, p.~14.

\vskip 0.5ex
\setbox\startprefix=\hbox{\tt \ \ ru\ \$p\ }
\setbox\contprefix=\hbox{\tt \ \ \ \ \ \ \ \ }
\startm
\m{\vdash}\m{\lnot}\m{\exists}\m{x}\m{\,x}\m{=}\m{\{}\m{y}\m{|}\m{\lnot}\m{y}
\m{\in}\m{y}\m{\}}
\endm
\vskip 1ex

\noindent Cantor's theorem\index{Cantor's theorem}.  No set can be mapped onto
its power set.  Compare Theorem 6B(b) of Enderton, p.~132.

\vskip 0.5ex
\setbox\startprefix=\hbox{\tt \ \ canth.1\ \$e\ }
\setbox\contprefix=\hbox{\tt \ \ \ \ \ \ \ \ \ \ \ \ \ }
\startm
\m{\vdash}\m{A}\m{\in}\m{{\rm V}}
\endm
\setbox\startprefix=\hbox{\tt \ \ canth\ \$p\ }
\setbox\contprefix=\hbox{\tt \ \ \ \ \ \ \ \ \ \ \ }
\startm
\m{\vdash}\m{\lnot}\m{F}\m{:}\m{A}\m{\raisebox{.5ex}{${\textstyle{\:}_{
\mbox{\footnotesize\rm {\ }}}}\atop{\textstyle{\longrightarrow}\atop{
\textstyle{}^{\mbox{\footnotesize\rm onto}}}}$}}\m{{\cal P}}\m{A}
\endm
\vskip 1ex

\noindent The Burali-Forti paradox\index{Burali-Forti paradox}.  No set
contains all ordinal numbers. Enderton, p.~194.  (Burali-Forti was one person,
not two.)

\vskip 0.5ex
\setbox\startprefix=\hbox{\tt \ \ onprc\ \$p\ }
\setbox\contprefix=\hbox{\tt \ \ \ \ \ \ \ \ \ \ \ \ }
\startm
\m{\vdash}\m{\lnot}\m{\mbox{\rm On}}\m{\in}\m{{\rm V}}
\endm
\vskip 1ex

\noindent Peano's postulates\index{Peano's postulates} for arithmetic.
Proposition 7.30 of Takeuti and Zaring, pp.~42--43.  The objects being
described are the members of $\omega$ i.e.\ the natural numbers 0, 1,
2,\ldots.  The successor\index{successor} operation suc means ``plus
one.''  \texttt{peano1} says that 0 (which is defined as the empty set)
is a natural number.  \texttt{peano2} says that if $A$ is a natural
number, so is $A+1$.  \texttt{peano3} says that 0 is not the successor
of any natural number.  \texttt{peano4} says that two natural numbers
are equal if and only if their successors are equal.  \texttt{peano5} is
essentially the same as mathematical induction.

\vskip 1ex
\setbox\startprefix=\hbox{\tt \ \ peano1\ \$p\ }
\setbox\contprefix=\hbox{\tt \ \ \ \ \ \ \ \ \ \ \ \ }
\startm
\m{\vdash}\m{\varnothing}\m{\in}\m{\omega}
\endm
\vskip 1.5ex

\setbox\startprefix=\hbox{\tt \ \ peano2\ \$p\ }
\setbox\contprefix=\hbox{\tt \ \ \ \ \ \ \ \ \ \ \ \ }
\startm
\m{\vdash}\m{(}\m{A}\m{\in}\m{\omega}\m{\rightarrow}\m{{\rm suc}}\m{A}\m{\in}%
\m{\omega}\m{)}
\endm
\vskip 1.5ex

\setbox\startprefix=\hbox{\tt \ \ peano3\ \$p\ }
\setbox\contprefix=\hbox{\tt \ \ \ \ \ \ \ \ \ \ \ \ }
\startm
\m{\vdash}\m{(}\m{A}\m{\in}\m{\omega}\m{\rightarrow}\m{\lnot}\m{{\rm suc}}%
\m{A}\m{=}\m{\varnothing}\m{)}
\endm
\vskip 1.5ex

\setbox\startprefix=\hbox{\tt \ \ peano4\ \$p\ }
\setbox\contprefix=\hbox{\tt \ \ \ \ \ \ \ \ \ \ \ \ }
\startm
\m{\vdash}\m{(}\m{(}\m{A}\m{\in}\m{\omega}\m{\wedge}\m{B}\m{\in}\m{\omega}%
\m{)}\m{\rightarrow}\m{(}\m{{\rm suc}}\m{A}\m{=}\m{{\rm suc}}\m{B}\m{%
\leftrightarrow}\m{A}\m{=}\m{B}\m{)}\m{)}
\endm
\vskip 1.5ex

\setbox\startprefix=\hbox{\tt \ \ peano5\ \$p\ }
\setbox\contprefix=\hbox{\tt \ \ \ \ \ \ \ \ \ \ \ \ }
\startm
\m{\vdash}\m{(}\m{(}\m{\varnothing}\m{\in}\m{A}\m{\wedge}\m{\forall}\m{x}\m{%
\in}\m{\omega}\m{(}\m{x}\m{\in}\m{A}\m{\rightarrow}\m{{\rm suc}}\m{x}\m{\in}%
\m{A}\m{)}\m{)}\m{\rightarrow}\m{\omega}\m{\subseteq}\m{A}\m{)}
\endm
\vskip 1.5ex

\noindent Finite Induction (mathematical induction).\index{finite
induction}\index{mathematical induction} The first hypothesis is the
basis and the second is the induction hypothesis.  Theorem Schema 22 of
Suppes, p.~136.

\vskip 0.5ex
\setbox\startprefix=\hbox{\tt \ \ findes.1\ \$e\ }
\setbox\contprefix=\hbox{\tt \ \ \ \ \ \ \ \ \ \ \ \ \ \ }
\startm
\m{\vdash}\m{[}\m{\varnothing}\m{/}\m{x}\m{]}\m{\varphi}
\endm
\setbox\startprefix=\hbox{\tt \ \ findes.2\ \$e\ }
\setbox\contprefix=\hbox{\tt \ \ \ \ \ \ \ \ \ \ \ \ \ \ }
\startm
\m{\vdash}\m{(}\m{x}\m{\in}\m{\omega}\m{\rightarrow}\m{(}\m{\varphi}\m{%
\rightarrow}\m{[}\m{{\rm suc}}\m{x}\m{/}\m{x}\m{]}\m{\varphi}\m{)}\m{)}
\endm
\setbox\startprefix=\hbox{\tt \ \ findes\ \$p\ }
\setbox\contprefix=\hbox{\tt \ \ \ \ \ \ \ \ \ \ \ \ }
\startm
\m{\vdash}\m{(}\m{x}\m{\in}\m{\omega}\m{\rightarrow}\m{\varphi}\m{)}
\endm
\vskip 1ex

\noindent Transfinite Induction with explicit substitution.  The first
hypothesis is the basis, the second is the induction hypothesis for
successors, and the third is the induction hypothesis for limit
ordinals.  Theorem Schema 4 of Suppes, p. 197.

\vskip 0.5ex
\setbox\startprefix=\hbox{\tt \ \ tfindes.1\ \$e\ }
\setbox\contprefix=\hbox{\tt \ \ \ \ \ \ \ \ \ \ \ \ \ \ \ }
\startm
\m{\vdash}\m{[}\m{\varnothing}\m{/}\m{x}\m{]}\m{\varphi}
\endm
\setbox\startprefix=\hbox{\tt \ \ tfindes.2\ \$e\ }
\setbox\contprefix=\hbox{\tt \ \ \ \ \ \ \ \ \ \ \ \ \ \ \ }
\startm
\m{\vdash}\m{(}\m{x}\m{\in}\m{{\rm On}}\m{\rightarrow}\m{(}\m{\varphi}\m{%
\rightarrow}\m{[}\m{{\rm suc}}\m{x}\m{/}\m{x}\m{]}\m{\varphi}\m{)}\m{)}
\endm
\setbox\startprefix=\hbox{\tt \ \ tfindes.3\ \$e\ }
\setbox\contprefix=\hbox{\tt \ \ \ \ \ \ \ \ \ \ \ \ \ \ \ }
\startm
\m{\vdash}\m{(}\m{{\rm Lim}}\m{y}\m{\rightarrow}\m{(}\m{\forall}\m{x}\m{\in}%
\m{y}\m{\varphi}\m{\rightarrow}\m{[}\m{y}\m{/}\m{x}\m{]}\m{\varphi}\m{)}\m{)}
\endm
\setbox\startprefix=\hbox{\tt \ \ tfindes\ \$p\ }
\setbox\contprefix=\hbox{\tt \ \ \ \ \ \ \ \ \ \ \ \ \ }
\startm
\m{\vdash}\m{(}\m{x}\m{\in}\m{{\rm On}}\m{\rightarrow}\m{\varphi}\m{)}
\endm
\vskip 1ex

\noindent Principle of Transfinite Recursion.\index{transfinite
recursion} Theorem 7.41 of Takeuti and Zaring, p.~47.  Transfinite
recursion is the key theorem that allows arithmetic of ordinals to be
rigorously defined, and has many other important uses as well.
Hypotheses \texttt{tfr.1} and \texttt{tfr.2} specify a certain (proper)
class $ F$.  The complicated definition of $ F$ is not important in
itself; what is important is that there be such an $ F$ with the
required properties, and we show this by displaying $ F$ explicitly.
\texttt{tfr1} states that $ F$ is a function whose domain is the set of
ordinal numbers.  \texttt{tfr2} states that any value of $ F$ is
completely determined by its previous values and the values of an
auxiliary function, $G$.  \texttt{tfr3} states that $ F$ is unique,
i.e.\ it is the only function that satisfies \texttt{tfr1} and
\texttt{tfr2}.  Note that $ f$ is an individual variable like $x$ and
$y$; it is just a mnemonic to remind us that $A$ is a collection of
functions.

\vskip 0.5ex
\setbox\startprefix=\hbox{\tt \ \ tfr.1\ \$e\ }
\setbox\contprefix=\hbox{\tt \ \ \ \ \ \ \ \ \ \ \ }
\startm
\m{\vdash}\m{A}\m{=}\m{\{}\m{f}\m{|}\m{\exists}\m{x}\m{\in}\m{{\rm On}}\m{(}%
\m{f}\m{{\rm Fn}}\m{x}\m{\wedge}\m{\forall}\m{y}\m{\in}\m{x}\m{(}\m{f}\m{`}%
\m{y}\m{)}\m{=}\m{(}\m{G}\m{`}\m{(}\m{f}\m{\restriction}\m{y}\m{)}\m{)}\m{)}%
\m{\}}
\endm
\setbox\startprefix=\hbox{\tt \ \ tfr.2\ \$e\ }
\setbox\contprefix=\hbox{\tt \ \ \ \ \ \ \ \ \ \ \ }
\startm
\m{\vdash}\m{F}\m{=}\m{\bigcup}\m{A}
\endm
\setbox\startprefix=\hbox{\tt \ \ tfr1\ \$p\ }
\setbox\contprefix=\hbox{\tt \ \ \ \ \ \ \ \ \ \ }
\startm
\m{\vdash}\m{F}\m{{\rm Fn}}\m{{\rm On}}
\endm
\setbox\startprefix=\hbox{\tt \ \ tfr2\ \$p\ }
\setbox\contprefix=\hbox{\tt \ \ \ \ \ \ \ \ \ \ }
\startm
\m{\vdash}\m{(}\m{z}\m{\in}\m{{\rm On}}\m{\rightarrow}\m{(}\m{F}\m{`}\m{z}%
\m{)}\m{=}\m{(}\m{G}\m{`}\m{(}\m{F}\m{\restriction}\m{z}\m{)}\m{)}\m{)}
\endm
\setbox\startprefix=\hbox{\tt \ \ tfr3\ \$p\ }
\setbox\contprefix=\hbox{\tt \ \ \ \ \ \ \ \ \ \ }
\startm
\m{\vdash}\m{(}\m{(}\m{B}\m{{\rm Fn}}\m{{\rm On}}\m{\wedge}\m{\forall}\m{x}\m{%
\in}\m{{\rm On}}\m{(}\m{B}\m{`}\m{x}\m{)}\m{=}\m{(}\m{G}\m{`}\m{(}\m{B}\m{%
\restriction}\m{x}\m{)}\m{)}\m{)}\m{\rightarrow}\m{B}\m{=}\m{F}\m{)}
\endm
\vskip 1ex

\noindent The existence of omega (the class of natural numbers).\index{natural
number}\index{omega ($\omega$)}\index{Axiom of Infinity}  Axiom 7 of Takeuti
and Zaring, p.~43.  (This is the only theorem in this section requiring the
Axiom of Infinity.)

\vskip 0.5ex
\setbox\startprefix=\hbox{\tt \
\ omex\ \$p\ }
\setbox\contprefix=\hbox{\tt \ \ \ \ \ \ \ \ \ \ }
\startm
\m{\vdash}\m{\omega}\m{\in}\m{{\rm V}}
\endm
%\vskip 2ex


\section{Axioms for Real and Complex Numbers}\label{real}
\index{real and complex numbers!axioms for}

This section presents the axioms for real and complex numbers, along
with some commentary about them.  Analysis
textbooks implicitly or explicitly use these axioms or their equivalents
as their starting point.  In the database \texttt{set.mm}, we define real
and complex numbers as (rather complicated) specific sets and derive these
axioms as {\em theorems} from the axioms of ZF set theory, using a method
called Dedekind cuts.  We omit the details of this construction, which you can
follow if you wish using the \texttt{set.mm} database in conjunction with the
textbooks referenced therein.

Once we prove those theorems, we then restate these proven theorems as axioms.
This lets us easily identify which axioms are needed for a particular complex number proof, without the obfuscation of the set theory used to derive them.
As a result,
the construction is actually unimportant other
than to show that sets exist that satisfy the axioms, and thus that the axioms
are consistent if ZF set theory is consistent.  When working with real numbers
you can think of them as being the actual sets resulting
from the construction (for definiteness), or you can
think of them as otherwise unspecified sets that happen to satisfy the axioms.
The derivation is not easy, but the fact that it works is quite remarkable
and lends support to the idea that ZFC set theory is all we need to
provide a foundation for essentially all of mathematics.

\needspace{3\baselineskip}
\subsection{The Axioms for Real and Complex Numbers Themselves}\label{realactual}

For the axioms we are given (or postulate) 8 classes:  $\mathbb{C}$ (the
set of complex numbers), $\mathbb{R}$ (the set of real numbers, a subset
of $\mathbb{C}$), $0$ (zero), $1$ (one), $i$ (square root of
$-1$), $+$ (plus), $\cdot$ (times), and
$<_{\mathbb{R}}$ (less than for just the real numbers).
Subtraction and division are defined terms and are not part of the
axioms; for their definitions see \texttt{set.mm}.

Note that the notation $(A+B)$ (and similarly $(A\cdot B)$) specifies a class
called an {\em operation},\index{operation} and is the function value of the
class $+$ at ordered pair $\langle A,B \rangle$.  An operation is defined by
statement \texttt{df-opr} on p.~\pageref{dfopr}.
The notation $A <_{\mathbb{R}} B$ specifies a
wff called a {\em binary relation}\index{binary relation} and means $\langle A,B \rangle \in \,<_{\mathbb{R}}$, as defined by statement \texttt{df-br} on p.~\pageref{dfbr}.

Our set of 8 given classes is assumed to satisfy the following 22 axioms
(in the axioms listed below, $<$ really means $<_{\mathbb{R}}$).

\vskip 2ex

\noindent 1. The real numbers are a subset of the complex numbers.

%\vskip 0.5ex
\setbox\startprefix=\hbox{\tt \ \ ax-resscn\ \$p\ }
\setbox\contprefix=\hbox{\tt \ \ \ \ \ \ \ \ \ \ \ \ \ \ }
\startm
\m{\vdash}\m{\mathbb{R}}\m{\subseteq}\m{\mathbb{C}}
\endm
%\vskip 1ex

\noindent 2. One is a complex number.

%\vskip 0.5ex
\setbox\startprefix=\hbox{\tt \ \ ax-1cn\ \$p\ }
\setbox\contprefix=\hbox{\tt \ \ \ \ \ \ \ \ \ \ \ }
\startm
\m{\vdash}\m{1}\m{\in}\m{\mathbb{C}}
\endm
%\vskip 1ex

\noindent 3. The imaginary number $i$ is a complex number.

%\vskip 0.5ex
\setbox\startprefix=\hbox{\tt \ \ ax-icn\ \$p\ }
\setbox\contprefix=\hbox{\tt \ \ \ \ \ \ \ \ \ \ \ }
\startm
\m{\vdash}\m{i}\m{\in}\m{\mathbb{C}}
\endm
%\vskip 1ex

\noindent 4. Complex numbers are closed under addition.

%\vskip 0.5ex
\setbox\startprefix=\hbox{\tt \ \ ax-addcl\ \$p\ }
\setbox\contprefix=\hbox{\tt \ \ \ \ \ \ \ \ \ \ \ \ \ }
\startm
\m{\vdash}\m{(}\m{(}\m{A}\m{\in}\m{\mathbb{C}}\m{\wedge}\m{B}\m{\in}\m{\mathbb{C}}%
\m{)}\m{\rightarrow}\m{(}\m{A}\m{+}\m{B}\m{)}\m{\in}\m{\mathbb{C}}\m{)}
\endm
%\vskip 1ex

\noindent 5. Real numbers are closed under addition.

%\vskip 0.5ex
\setbox\startprefix=\hbox{\tt \ \ ax-addrcl\ \$p\ }
\setbox\contprefix=\hbox{\tt \ \ \ \ \ \ \ \ \ \ \ \ \ \ }
\startm
\m{\vdash}\m{(}\m{(}\m{A}\m{\in}\m{\mathbb{R}}\m{\wedge}\m{B}\m{\in}\m{\mathbb{R}}%
\m{)}\m{\rightarrow}\m{(}\m{A}\m{+}\m{B}\m{)}\m{\in}\m{\mathbb{R}}\m{)}
\endm
%\vskip 1ex

\noindent 6. Complex numbers are closed under multiplication.

%\vskip 0.5ex
\setbox\startprefix=\hbox{\tt \ \ ax-mulcl\ \$p\ }
\setbox\contprefix=\hbox{\tt \ \ \ \ \ \ \ \ \ \ \ \ \ }
\startm
\m{\vdash}\m{(}\m{(}\m{A}\m{\in}\m{\mathbb{C}}\m{\wedge}\m{B}\m{\in}\m{\mathbb{C}}%
\m{)}\m{\rightarrow}\m{(}\m{A}\m{\cdot}\m{B}\m{)}\m{\in}\m{\mathbb{C}}\m{)}
\endm
%\vskip 1ex

\noindent 7. Real numbers are closed under multiplication.

%\vskip 0.5ex
\setbox\startprefix=\hbox{\tt \ \ ax-mulrcl\ \$p\ }
\setbox\contprefix=\hbox{\tt \ \ \ \ \ \ \ \ \ \ \ \ \ \ }
\startm
\m{\vdash}\m{(}\m{(}\m{A}\m{\in}\m{\mathbb{R}}\m{\wedge}\m{B}\m{\in}\m{\mathbb{R}}%
\m{)}\m{\rightarrow}\m{(}\m{A}\m{\cdot}\m{B}\m{)}\m{\in}\m{\mathbb{R}}\m{)}
\endm
%\vskip 1ex

\noindent 8. Multiplication of complex numbers is commutative.

%\vskip 0.5ex
\setbox\startprefix=\hbox{\tt \ \ ax-mulcom\ \$p\ }
\setbox\contprefix=\hbox{\tt \ \ \ \ \ \ \ \ \ \ \ \ \ \ }
\startm
\m{\vdash}\m{(}\m{(}\m{A}\m{\in}\m{\mathbb{C}}\m{\wedge}\m{B}\m{\in}\m{\mathbb{C}}%
\m{)}\m{\rightarrow}\m{(}\m{A}\m{\cdot}\m{B}\m{)}\m{=}\m{(}\m{B}\m{\cdot}\m{A}%
\m{)}\m{)}
\endm
%\vskip 1ex

\noindent 9. Addition of complex numbers is associative.

%\vskip 0.5ex
\setbox\startprefix=\hbox{\tt \ \ ax-addass\ \$p\ }
\setbox\contprefix=\hbox{\tt \ \ \ \ \ \ \ \ \ \ \ \ \ \ }
\startm
\m{\vdash}\m{(}\m{(}\m{A}\m{\in}\m{\mathbb{C}}\m{\wedge}\m{B}\m{\in}\m{\mathbb{C}}%
\m{\wedge}\m{C}\m{\in}\m{\mathbb{C}}\m{)}\m{\rightarrow}\m{(}\m{(}\m{A}\m{+}%
\m{B}\m{)}\m{+}\m{C}\m{)}\m{=}\m{(}\m{A}\m{+}\m{(}\m{B}\m{+}\m{C}\m{)}\m{)}%
\m{)}
\endm
%\vskip 1ex

\noindent 10. Multiplication of complex numbers is associative.

%\vskip 0.5ex
\setbox\startprefix=\hbox{\tt \ \ ax-mulass\ \$p\ }
\setbox\contprefix=\hbox{\tt \ \ \ \ \ \ \ \ \ \ \ \ \ \ }
\startm
\m{\vdash}\m{(}\m{(}\m{A}\m{\in}\m{\mathbb{C}}\m{\wedge}\m{B}\m{\in}\m{\mathbb{C}}%
\m{\wedge}\m{C}\m{\in}\m{\mathbb{C}}\m{)}\m{\rightarrow}\m{(}\m{(}\m{A}\m{\cdot}%
\m{B}\m{)}\m{\cdot}\m{C}\m{)}\m{=}\m{(}\m{A}\m{\cdot}\m{(}\m{B}\m{\cdot}\m{C}%
\m{)}\m{)}\m{)}
\endm
%\vskip 1ex

\noindent 11. Multiplication distributes over addition for complex numbers.

%\vskip 0.5ex
\setbox\startprefix=\hbox{\tt \ \ ax-distr\ \$p\ }
\setbox\contprefix=\hbox{\tt \ \ \ \ \ \ \ \ \ \ \ \ \ }
\startm
\m{\vdash}\m{(}\m{(}\m{A}\m{\in}\m{\mathbb{C}}\m{\wedge}\m{B}\m{\in}\m{\mathbb{C}}%
\m{\wedge}\m{C}\m{\in}\m{\mathbb{C}}\m{)}\m{\rightarrow}\m{(}\m{A}\m{\cdot}\m{(}%
\m{B}\m{+}\m{C}\m{)}\m{)}\m{=}\m{(}\m{(}\m{A}\m{\cdot}\m{B}\m{)}\m{+}\m{(}%
\m{A}\m{\cdot}\m{C}\m{)}\m{)}\m{)}
\endm
%\vskip 1ex

\noindent 12. The square of $i$ equals $-1$ (expressed as $i$-squared plus 1 is
0).

%\vskip 0.5ex
\setbox\startprefix=\hbox{\tt \ \ ax-i2m1\ \$p\ }
\setbox\contprefix=\hbox{\tt \ \ \ \ \ \ \ \ \ \ \ \ }
\startm
\m{\vdash}\m{(}\m{(}\m{i}\m{\cdot}\m{i}\m{)}\m{+}\m{1}\m{)}\m{=}\m{0}
\endm
%\vskip 1ex

\noindent 13. One and zero are distinct.

%\vskip 0.5ex
\setbox\startprefix=\hbox{\tt \ \ ax-1ne0\ \$p\ }
\setbox\contprefix=\hbox{\tt \ \ \ \ \ \ \ \ \ \ \ \ }
\startm
\m{\vdash}\m{1}\m{\ne}\m{0}
\endm
%\vskip 1ex

\noindent 14. One is an identity element for real multiplication.

%\vskip 0.5ex
\setbox\startprefix=\hbox{\tt \ \ ax-1rid\ \$p\ }
\setbox\contprefix=\hbox{\tt \ \ \ \ \ \ \ \ \ \ \ }
\startm
\m{\vdash}\m{(}\m{A}\m{\in}\m{\mathbb{R}}\m{\rightarrow}\m{(}\m{A}\m{\cdot}\m{1}%
\m{)}\m{=}\m{A}\m{)}
\endm
%\vskip 1ex

\noindent 15. Every real number has a negative.

%\vskip 0.5ex
\setbox\startprefix=\hbox{\tt \ \ ax-rnegex\ \$p\ }
\setbox\contprefix=\hbox{\tt \ \ \ \ \ \ \ \ \ \ \ \ \ \ }
\startm
\m{\vdash}\m{(}\m{A}\m{\in}\m{\mathbb{R}}\m{\rightarrow}\m{\exists}\m{x}\m{\in}%
\m{\mathbb{R}}\m{(}\m{A}\m{+}\m{x}\m{)}\m{=}\m{0}\m{)}
\endm
%\vskip 1ex

\noindent 16. Every nonzero real number has a reciprocal.

%\vskip 0.5ex
\setbox\startprefix=\hbox{\tt \ \ ax-rrecex\ \$p\ }
\setbox\contprefix=\hbox{\tt \ \ \ \ \ \ \ \ \ \ \ \ \ \ }
\startm
\m{\vdash}\m{(}\m{A}\m{\in}\m{\mathbb{R}}\m{\rightarrow}\m{(}\m{A}\m{\ne}\m{0}%
\m{\rightarrow}\m{\exists}\m{x}\m{\in}\m{\mathbb{R}}\m{(}\m{A}\m{\cdot}%
\m{x}\m{)}\m{=}\m{1}\m{)}\m{)}
\endm
%\vskip 1ex

\noindent 17. A complex number can be expressed in terms of two reals.

%\vskip 0.5ex
\setbox\startprefix=\hbox{\tt \ \ ax-cnre\ \$p\ }
\setbox\contprefix=\hbox{\tt \ \ \ \ \ \ \ \ \ \ \ \ }
\startm
\m{\vdash}\m{(}\m{A}\m{\in}\m{\mathbb{C}}\m{\rightarrow}\m{\exists}\m{x}\m{\in}%
\m{\mathbb{R}}\m{\exists}\m{y}\m{\in}\m{\mathbb{R}}\m{A}\m{=}\m{(}\m{x}\m{+}\m{(}%
\m{y}\m{\cdot}\m{i}\m{)}\m{)}\m{)}
\endm
%\vskip 1ex

\noindent 18. Ordering on reals satisfies strict trichotomy.

%\vskip 0.5ex
\setbox\startprefix=\hbox{\tt \ \ ax-pre-lttri\ \$p\ }
\setbox\contprefix=\hbox{\tt \ \ \ \ \ \ \ \ \ \ \ \ \ }
\startm
\m{\vdash}\m{(}\m{(}\m{A}\m{\in}\m{\mathbb{R}}\m{\wedge}\m{B}\m{\in}\m{\mathbb{R}}%
\m{)}\m{\rightarrow}\m{(}\m{A}\m{<}\m{B}\m{\leftrightarrow}\m{\lnot}\m{(}\m{A}%
\m{=}\m{B}\m{\vee}\m{B}\m{<}\m{A}\m{)}\m{)}\m{)}
\endm
%\vskip 1ex

\noindent 19. Ordering on reals is transitive.

%\vskip 0.5ex
\setbox\startprefix=\hbox{\tt \ \ ax-pre-lttrn\ \$p\ }
\setbox\contprefix=\hbox{\tt \ \ \ \ \ \ \ \ \ \ \ \ \ }
\startm
\m{\vdash}\m{(}\m{(}\m{A}\m{\in}\m{\mathbb{R}}\m{\wedge}\m{B}\m{\in}\m{\mathbb{R}}%
\m{\wedge}\m{C}\m{\in}\m{\mathbb{R}}\m{)}\m{\rightarrow}\m{(}\m{(}\m{A}\m{<}%
\m{B}\m{\wedge}\m{B}\m{<}\m{C}\m{)}\m{\rightarrow}\m{A}\m{<}\m{C}\m{)}\m{)}
\endm
%\vskip 1ex

\noindent 20. Ordering on reals is preserved after addition to both sides.

%\vskip 0.5ex
\setbox\startprefix=\hbox{\tt \ \ ax-pre-ltadd\ \$p\ }
\setbox\contprefix=\hbox{\tt \ \ \ \ \ \ \ \ \ \ \ \ \ }
\startm
\m{\vdash}\m{(}\m{(}\m{A}\m{\in}\m{\mathbb{R}}\m{\wedge}\m{B}\m{\in}\m{\mathbb{R}}%
\m{\wedge}\m{C}\m{\in}\m{\mathbb{R}}\m{)}\m{\rightarrow}\m{(}\m{A}\m{<}\m{B}\m{%
\rightarrow}\m{(}\m{C}\m{+}\m{A}\m{)}\m{<}\m{(}\m{C}\m{+}\m{B}\m{)}\m{)}\m{)}
\endm
%\vskip 1ex

\noindent 21. The product of two positive reals is positive.

%\vskip 0.5ex
\setbox\startprefix=\hbox{\tt \ \ ax-pre-mulgt0\ \$p\ }
\setbox\contprefix=\hbox{\tt \ \ \ \ \ \ \ \ \ \ \ \ \ \ }
\startm
\m{\vdash}\m{(}\m{(}\m{A}\m{\in}\m{\mathbb{R}}\m{\wedge}\m{B}\m{\in}\m{\mathbb{R}}%
\m{)}\m{\rightarrow}\m{(}\m{(}\m{0}\m{<}\m{A}\m{\wedge}\m{0}%
\m{<}\m{B}\m{)}\m{\rightarrow}\m{0}\m{<}\m{(}\m{A}\m{\cdot}\m{B}\m{)}%
\m{)}\m{)}
\endm
%\vskip 1ex

\noindent 22. A non-empty, bounded-above set of reals has a supremum.

%\vskip 0.5ex
\setbox\startprefix=\hbox{\tt \ \ ax-pre-sup\ \$p\ }
\setbox\contprefix=\hbox{\tt \ \ \ \ \ \ \ \ \ \ \ }
\startm
\m{\vdash}\m{(}\m{(}\m{A}\m{\subseteq}\m{\mathbb{R}}\m{\wedge}\m{A}\m{\ne}\m{%
\varnothing}\m{\wedge}\m{\exists}\m{x}\m{\in}\m{\mathbb{R}}\m{\forall}\m{y}\m{%
\in}\m{A}\m{\,y}\m{<}\m{x}\m{)}\m{\rightarrow}\m{\exists}\m{x}\m{\in}\m{%
\mathbb{R}}\m{(}\m{\forall}\m{y}\m{\in}\m{A}\m{\lnot}\m{x}\m{<}\m{y}\m{\wedge}\m{%
\forall}\m{y}\m{\in}\m{\mathbb{R}}\m{(}\m{y}\m{<}\m{x}\m{\rightarrow}\m{\exists}%
\m{z}\m{\in}\m{A}\m{\,y}\m{<}\m{z}\m{)}\m{)}\m{)}
\endm

% NOTE: The \m{...} expressions above could be represented as
% $ \vdash ( ( A \subseteq \mathbb{R} \wedge A \ne \varnothing \wedge \exists x \in \mathbb{R} \forall y \in A \,y < x ) \rightarrow \exists x \in \mathbb{R} ( \forall y \in A \lnot x < y \wedge \forall y \in \mathbb{R} ( y < x \rightarrow \exists z \in A \,y < z ) ) ) $

\vskip 2ex

This completes the set of axioms for real and complex numbers.  You may
wish to look at how subtraction, division, and decimal numbers
are defined in \texttt{set.mm}, and for fun look at the proof of $2+
2 = 4$ (theorem \texttt{2p2e4} in \texttt{set.mm})
as discussed in section \ref{2p2e4}.

In \texttt{set.mm} we define the non-negative integers $\mathbb{N}$, the integers
$\mathbb{Z}$, and the rationals $\mathbb{Q}$ as subsets of $\mathbb{R}$.  This leads
to the nice inclusion $\mathbb{N} \subseteq \mathbb{Z} \subseteq \mathbb{Q} \subseteq
\mathbb{R} \subseteq \mathbb{C}$, giving us a uniform framework in which, for
example, a property such as commutativity of complex number addition
automatically applies to integers.  The natural numbers $\mathbb{N}$
are different from the set $\omega$ we defined earlier, but both satisfy
Peano's postulates.

\subsection{Complex Number Axioms in Analysis Texts}

Most analysis texts construct complex numbers as ordered pairs of reals,
leading to construction-dependent properties that satisfy these axioms
but are not stated in their pure form.  (This is also done in
\texttt{set.mm} but our axioms are extracted from that construction.)
Other texts will simply state that $\mathbb{R}$ is a ``complete ordered
subfield of $\mathbb{C}$,'' leading to redundant axioms when this phrase
is completely expanded out.  In fact I have not seen a text with the
axioms in the explicit form above.
None of these axioms is unique individually, but this carefully worked out
collection of axioms is the result of years of work
by the Metamath community.

\subsection{Eliminating Unnecessary Complex Number Axioms}

We once had more axioms for real and complex numbers, but over years of time
we (the Metamath community)
have found ways to eliminate them (by proving them from other axioms)
or weaken them (by making weaker claims without reducing what
can be proved).
In particular, here are statements that used to be complex number
axioms but have since been formally proven (with Metamath) to be redundant:

\begin{itemize}
\item
  $\mathbb{C} \in V$.
  At one time this was listed as a ``complex number axiom.''
  However, this is not properly speaking a complex number axiom,
  and in any case its proof uses axioms of set theory.
  Proven redundant by Mario Carneiro\index{Carneiro, Mario} on
  17-Nov-2014 (see \texttt{axcnex}).
\item
  $((A \in \mathbb{C} \land B \in \mathbb{C}$) $\rightarrow$
  $(A + B) = (B + A))$.
  Proved redundant by Eric Schmidt\index{Schmidt, Eric} on 19-Jun-2012,
  and formalized by Scott Fenton\index{Fenton, Scott} on 3-Jan-2013
  (see \texttt{addcom}).
\item
  $(A \in \mathbb{C} \rightarrow (A + 0) = A)$.
  Proved redundant by Eric Schmidt on 19-Jun-2012,
  and formalized by Scott Fenton on 3-Jan-2013
  (see \texttt{addid1}).
\item
  $(A \in \mathbb{C} \rightarrow \exists x \in \mathbb{C} (A + x) = 0)$.
  Proved redundant by Eric Schmidt and formalized on 21-May-2007
  (see \texttt{cnegex}).
\item
  $((A \in \mathbb{C} \land A \ne 0) \rightarrow \exists x \in \mathbb{C} (A \cdot x) = 1)$.
  Proved redundant by Eric Schmidt and formalized on 22-May-2007
  (see \texttt{recex}).
\item
  $0 \in \mathbb{R}$.
  Proved redundant by Eric Schmidt on 19-Feb-2005 and formalized 21-May-2007
  (see \texttt{0re}).
\end{itemize}

We could eliminate 0 as an axiomatic object by defining it as
$( ( i \cdot i ) + 1 )$
and replacing it with this expression throughout the axioms. If this
is done, axiom ax-i2m1 becomes redundant. However, the remaining axioms
would become longer and less intuitive.

Eric Schmidt's paper analyzing this axiom system \cite{Schmidt}
presented a proof that these remaining axioms,
with the possible exception of ax-mulcom, are independent of the others.
It is currently an open question if ax-mulcom is independent of the others.

\section{Two Plus Two Equals Four}\label{2p2e4}

Here is a proof that $2 + 2 = 4$, as proven in the theorem \texttt{2p2e4}
in the database \texttt{set.mm}.
This is a useful demonstration of what a Metamath proof can look like.
This proof may have more steps than you're used to, but each step is rigorously
proven all the way back to the axioms of logic and set theory.
This display was originally generated by the Metamath program
as an {\sc HTML} file.

In the table showing the proof ``Step'' is the sequential step number,
while its associated ``Expression'' is an expression that we have proved.
``Ref'' is the name of a theorem or axiom that justifies that expression,
and ``Hyp'' refers to previous steps (if any) that the theorem or axiom
needs so that we can use it.  Expressions are indented further than
the expressions that depend on them to show their interdependencies.

\begin{table}[!htbp]
\caption{Two plus two equals four}
\begin{tabular}{lllll}
\textbf{Step} & \textbf{Hyp} & \textbf{Ref} & \textbf{Expression} & \\
1  &       & df-2    & $ \; \; \vdash 2 = 1 + 1$  & \\
2  & 1     & oveq2i  & $ \; \vdash (2 + 2) = (2 + (1 + 1))$ & \\
3  &       & df-4    & $ \; \; \vdash 4 = (3 + 1)$ & \\
4  &       & df-3    & $ \; \; \; \vdash 3 = (2 + 1)$ & \\
5  & 4     & oveq1i  & $ \; \; \vdash (3 + 1) = ((2 + 1) + 1)$ & \\
6  &       & 2cn     & $ \; \; \; \vdash 2 \in \mathbb{C}$ & \\
7  &       & ax-1cn  & $ \; \; \; \vdash 1 \in \mathbb{C}$ & \\
8  & 6,7,7 & addassi & $ \; \; \vdash ((2 + 1) + 1) = (2 + (1 + 1))$ & \\
9  & 3,5,8 & 3eqtri  & $ \; \vdash 4 = (2 + (1 + 1))$ & \\
10 & 2,9   & eqtr4i  & $ \vdash (2 + 2) = 4$ & \\
\end{tabular}
\end{table}

Step 1 says that we can assert that $2 = 1 + 1$ because it is
justified by \texttt{df-2}.
What is \texttt{df-2}?
It is simply the definition of $2$, which in our system is defined as being
equal to $1 + 1$.  This shows how we can use definitions in proofs.

Look at Step 2 of the proof. In the Ref column, we see that it references
a previously proved theorem, \texttt{oveq2i}.
It turns out that
theorem \texttt{oveq2i} requires a
hypothesis, and in the Hyp column of Step 2 we indicate that Step 1 will
satisfy (match) this hypothesis.
If we looked at \texttt{oveq2i}
we would find that it proves that given some hypothesis
$A = B$, we can prove that $( C F A ) = ( C F B )$.
If we use \texttt{oveq2i} and apply step 1's result as the hypothesis,
that will mean that $A = 2$ and $B = ( 1 + 1 )$ within this use of
\texttt{oveq2i}.
We are free to select any value of $C$ and $F$ (subject to syntax constraints),
so we are free to select $C = 2$ and $F = +$,
producing our desired result,
$ (2 + 2) = (2 + (1 + 1))$.

Step 2 is an example of substitution.
In the end, every step in every proof uses only this one substitution rule.
All the rules of logic, and all the axioms, are expressed so that
they can be used via this one substitution rule.
So once you master substitution, you can master every Metamath proof,
no exceptions.

Each step is clear and can be immediately checked.
In the {\sc HTML} display you can even click on each reference to see why it is
justified, making it easy to see why the proof works.

\section{Deduction}\label{deduction}

Strictly speaking,
a deduction (also called an inference) is a kind of statement that needs
some hypotheses to be true in order for its conclusion to be true.
A theorem, on the other hand, has no hypotheses.
Informally we often call both of them theorems, but in this section we
will stick to the strict definitions.

It sometimes happens that we have proved a deduction of the form
$\varphi \Rightarrow \psi$\index{$\Rightarrow$}
(given hypothesis $\varphi$ we can prove $\psi$)
and we want to then prove a theorem of the form
$\varphi \rightarrow \psi$.

Converting a deduction (which uses a hypothesis) into a theorem
(which does not) is not as simple as you might think.
The deduction says, ``if we can prove $\varphi$ then we can prove $\psi$,''
which is in some sense weaker than saying
``$\varphi$ implies $\psi$.''
There is no axiom of logic that permits us to directly obtain the theorem
given the deduction.\footnote{
The conversion of a deduction to a theorem does not even hold in general
for quantum propositional calculus,
which is a weak subset of classical propositional calculus.
It has been shown that adding the Standard Deduction Theorem (discussed below)
to quantum propositional calculus turns it into classical
propositional calculus!
}

This is in contrast to going the other way.
If we have the theorem ($\varphi \rightarrow \psi$),
it is easy to recover the deduction
($\varphi \Rightarrow \psi$)
using modus ponens\index{modus ponens}
(\texttt{ax-mp}; see section \ref{axmp}).

In the following subsections we first discuss the standard deduction theorem
(the traditional but awkward way to convert deductions into theorems) and
the weak deduction theorem (a limited version of the standard deduction
theorem that is easier to use and was once widely used in
\texttt{set.mm}\index{set theory database (\texttt{set.mm})}).
In section \ref{deductionstyle} we discuss
deduction style, the newer approach we now recommend in most cases.
Deduction style uses ``deduction form,'' a form that
prefixes each hypothesis (other than definitions) and the
conclusion with a universal antecedent (``$\varphi \rightarrow$'').
Deduction style is widely used in \texttt{set.mm},
so it is useful to understand it and \textit{why} it is widely used.
Section \ref{naturaldeduction}
briefly discusses our approach for using natural deduction
within \texttt{set.mm},
as that approach is deeply related to deduction style.
We conclude with a summary of the strengths of
our approach, which we believe are compelling.

\subsection{The Standard Deduction Theorem}\label{standarddeductiontheorem}

It is possible to make use of information
contained in the deduction or its proof to assist us with the proof of
the related theorem.
In traditional logic books, there is a metatheorem called the
Deduction Theorem\index{Deduction Theorem}\index{Standard Deduction Theorem},
discovered independently by Herbrand and Tarski around 1930.
The Deduction Theorem, which we often call the Standard Deduction Theorem,
provides an algorithm for constructing a proof of a theorem from the
proof of its corresponding deduction. See, for example,
\cite[p.~56]{Margaris}\index{Margaris, Angelo}.
To construct a proof for a theorem, the
algorithm looks at each step in the proof of the original deduction and
rewrites the step with several steps wherein the hypothesis is eliminated
and becomes an antecedent.

In ordinary mathematics, no one actually carries out the algorithm,
because (in its most basic form) it involves an exponential explosion of
the number of proof steps as more hypotheses are eliminated. Instead,
the Standard Deduction Theorem is invoked simply to claim that it can
be done in principle, without actually doing it.
What's more, the algorithm is not as simple as it might first appear
when applying it rigorously.
There is a subtle restriction on the Standard Deduction Theorem
that must be taken into account involving the axiom of generalization
when working with predicate calculus (see the literature for more detail).

One of the goals of Metamath is to let you plainly see, with as few
underlying concepts as possible, how mathematics can be derived directly
from the axioms, and not indirectly according to some hidden rules
buried inside a program or understood only by logicians. If we added
the Standard Deduction Theorem to the language and proof verifier,
that would greatly complicate both and largely defeat Metamath's goal
of simplicity. In principle, we could show direct proofs by expanding
out the proof steps generated by the algorithm of the Standard Deduction
Theorem, but that is not feasible in practice because the number of proof
steps quickly becomes huge, even astronomical.
Since the algorithm of the Standard Deduction Theorem is driven by the proof,
we would have to go through that proof
all over again---starting from axioms---in order to obtain the theorem form.
In terms of proof length, there would be no savings over just
proving the theorem directly instead of first proving the deduction form.

\subsection{Weak Deduction Theorem}\label{weakdeductiontheorem}

We have developed
a more efficient method for proving a theorem from a deduction
that can be used instead of the Standard Deduction Theorem
in many (but not all) cases.
We call this more efficient method the
Weak Deduction Theorem\index{Weak Deduction Theorem}.\footnote{
There is also an unrelated ``Weak Deduction Theorem''
in the field of relevance logic, so to avoid confusion we could call
ours the ``Weak Deduction Theorem for Classical Logic.''}
Unlike the Standard Deduction Theorem, the Weak Deduction Theorem produces the
theorem directly from a special substitution instance of the deduction,
using a small, fixed number of steps roughly proportional to the length
of the final theorem.

If you come to a proof referencing the Weak Deduction Theorem
\texttt{dedth} (or one of its variants \texttt{dedthxx}),
here is how to follow the proof without getting into the details:
just click on the theorem referenced in the step
just before the reference to \texttt{dedth} and ignore everything else.
Theorem \texttt{dedth} simply turns a hypothesis into an antecedent
(i.e. the hypothesis followed by $\rightarrow$
is placed in front of the assertion, and the hypothesis
itself is eliminated) given certain conditions.

The Weak Deduction Theorem
eliminates a hypothesis $\varphi$, making it become an antecedent.
It does this by proving an expression
$ \varphi \rightarrow \psi $ given two hypotheses:
(1)
$ ( A = {\rm if} ( \varphi , A , B ) \rightarrow ( \varphi \leftrightarrow \chi ) ) $
and
(2) $\chi$.
Note that it requires that a proof exists for $\varphi$ when the class variable
$A$ is replaced with a specific class $B$. The hypothesis $\chi$
should be assigned to the inference.
You can see the details of the proof of the Weak Deduction Theorem
in theorem \texttt{dedth}.

The Weak Deduction Theorem
is probably easier to understand by studying proofs that make use of it.
For example, let's look at the proof of \texttt{renegcl}, which proves that
$ \vdash ( A \in \mathbb{R} \rightarrow - A \in \mathbb{R} )$:

\needspace{4\baselineskip}
\begin{longtabu} {l l l X}
\textbf{Step} & \textbf{Hyp} & \textbf{Ref} & \textbf{Expression} \\
  1 &  & negeq &
  $\vdash$ $($ $A$ $=$ ${\rm if}$ $($ $A$ $\in$ $\mathbb{R}$ $,$ $A$ $,$ $1$ $)$ $\rightarrow$
  $\textrm{-}$ $A$ $=$ $\textrm{-}$ ${\rm if}$ $($ $A$ $\in$ $\mathbb{R}$
  $,$ $A$ $,$ $1$ $)$ $)$ \\
 2 & 1 & eleq1d &
    $\vdash$ $($ $A$ $=$ ${\rm if}$ $($ $A$ $\in$ $\mathbb{R}$ $,$ $A$ $,$ $1$ $)$ $\rightarrow$ $($
    $\textrm{-}$ $A$ $\in$ $\mathbb{R}$ $\leftrightarrow$
    $\textrm{-}$ ${\rm if}$ $($ $A$ $\in$ $\mathbb{R}$ $,$ $A$ $,$ $1$ $)$ $\in$
    $\mathbb{R}$ $)$ $)$ \\
 3 &  & 1re & $\vdash 1 \in \mathbb{R}$ \\
 4 & 3 & elimel &
   $\vdash {\rm if} ( A \in \mathbb{R} , A , 1 ) \in \mathbb{R}$ \\
 5 & 4 & renegcli &
   $\vdash \textrm{-} {\rm if} ( A \in \mathbb{R} , A , 1 ) \in \mathbb{R}$ \\
 6 & 2,5 & dedth &
   $\vdash ( A \in \mathbb{R} \rightarrow \textrm{-} A \in \mathbb{R}$ ) \\
\end{longtabu}

The somewhat strange-looking steps in \texttt{renegcl} before step 5 are
technical stuff that makes this magic work, and they can be ignored
for a quick overview of the proof. To continue following the ``important''
part of the proof of \texttt{renegcl},
you can look at the reference to \texttt{renegcli} at step 5.

That said, let's briefly look at how
\texttt{renegcl} uses the
Weak Deduction Theorem (\texttt{dedth}) to do its job,
in case you want to do something similar or want understand it more deeply.
Let's work backwards in the proof of \texttt{renegcl}.
Step 6 applies \texttt{dedth} to produce our goal result
$ \vdash ( A \in \mathbb{R} \rightarrow\, - A \in \mathbb{R} )$.
This requires on the one hand the (substituted) deduction
\texttt{renegcli} in step 5.
By itself \texttt{renegcli} proves the deduction
$ \vdash A \in \mathbb{R} \Rightarrow\, \vdash - A \in \mathbb{R}$;
this is the deduction form we are trying to turn into theorem form,
and thus
\texttt{renegcli} has a separate hypothesis that must be fulfilled.
To fulfill the hypothesis of the invocation of
\texttt{renegcli} in step 5, it is eventually
reduced to the already proven theorem $1 \in \mathbb{R}$ in step 3.
Step 4 connects steps 3 and 5; step 4 invokes
\texttt{elimel}, a special case of \texttt{elimhyp} that eliminates
a membership hypothesis for the weak deduction theorem.
On the other hand, the equivalence of the conclusion of
\texttt{renegcl}
$( - A \in \mathbb{R} )$ and the substituted conclusion of
\texttt{renegcli} must be proven, which is done in steps 2 and 1.

The weak deduction theorem has limitations.
In particular, we must be able to prove a special case of the deduction's
hypothesis as a stand-alone theorem.
For example, we used $1 \in \mathbb{R}$ in step 3 of \texttt{renegcl}.

We used to use the weak deduction theorem
extensively within \texttt{set.mm}.
However, we now recommend applying ``deduction style''
instead in most cases, as deduction style is
often an easier and clearer approach.
Therefore, we will now describe deduction style.

\subsection{Deduction Style}\label{deductionstyle}

We now prefer to write assertions in ``deduction form''
instead of writing a proof that would require use of the standard or
weak deduction theorem.
We call this appraoch
``deduction style.''\index{deduction style}

It will be easier to explain this by first defining some terms:

\begin{itemize}
\item \textbf{closed form}\index{closed form}\index{forms!closed}:
A kind of assertion (theorem) with no hypotheses.
Typically its label has no special suffix.
An example is \texttt{unss}, which states:
$\vdash ( ( A \subseteq C \wedge B \subseteq C ) \leftrightarrow ( A \cup B )
\subseteq C )\label{eq:unss}$
\item \textbf{deduction form}\index{deduction form}\index{forms!deduction}:
A kind of assertion with one or more hypotheses
where the conclusion is an implication with
a wff variable as the antecedent (usually $\varphi$), and every hypothesis
(\$e statement)
is either (1) an implication with the same antecedent as the conclusion or
(2) a definition.
A definition
can be for a class variable (this is a class variable followed by ``='')
or a wff variable (this is a wff variable followed by $\leftrightarrow$);
class variable definitions are more common.
In practice, a proof
in deduction form will also contain many steps that are implications
where the antecedent is either that wff variable (normally $\varphi$)
or is
a conjunction (...$\land$...) including that wff variable ($\varphi$).
If an assertion is in deduction form, and other forms are also available,
then we suffix its label with ``d.''
An example is \texttt{unssd}, which states\footnote{
For brevity we show here (and in other places)
a $\&$\index{$\&$} between hypotheses\index{hypotheses}
and a $\Rightarrow$\index{$\Rightarrow$}\index{conclusion}
between the hypotheses and the conclusion.
This notation is technically not part of the Metamath language, but is
instead a convenient abbreviation to show both the hypotheses and conclusion.}:
$\vdash ( \varphi \rightarrow A \subseteq C )\quad\&\quad \vdash ( \varphi
    \rightarrow B \subseteq C )\quad\Rightarrow\quad \vdash ( \varphi
    \rightarrow ( A \cup B ) \subseteq C )\label{eq:unssd}$
\item \textbf{inference form}\index{inference form}\index{forms!inference}:
A kind of assertion with one or more hypotheses that is not in deduction form
(e.g., there is no common antecedent).
If an assertion is in inference form, and other forms are also available,
then we suffix its label with ``i.''
An example is \texttt{unssi}, which states:
$\vdash A \subseteq C\quad\&\quad \vdash B \subseteq C\quad\Rightarrow\quad
    \vdash ( A \cup B ) \subseteq C\label{eq:unssi}$
\end{itemize}

When using deduction style we express an assertion in deduction form.
This form prefixes each hypothesis (other than definitions) and the
conclusion with a universal antecedent (``$\varphi \rightarrow$'').
The antecedent (e.g., $\varphi$)
mimics the context handled in the deduction theorem, eliminating
the need to directly use the deduction theorem.

Once you have an assertion in deduction form, you can easily convert it
to inference form or closed form:

\begin{itemize}
\item To
prove some assertion Ti in inference form, given assertion Td in deduction
form, there is a simple mechanical process you can use. First take each
Ti hypothesis and insert a \texttt{T.} $\rightarrow$ prefix (``true implies'')
using \texttt{a1i}. You
can then use the existing assertion Td to prove the resulting conclusion
with a \texttt{T.} $\rightarrow$ prefix.
Finally, you can remove that prefix using \texttt{mptru},
resulting in the conclusion you wanted to prove.
\item To
prove some assertion T in closed form, given assertion Td in deduction
form, there is another simple mechanical process you can use. First,
select an expression that is the conjunction (...$\land$...) of all of the
consequents of every hypothesis of Td. Next, prove that this expression
implies each of the separate hypotheses of Td in turn by eliminating
conjuncts (there are a variety of proven assertions to do this, including
\texttt{simpl},
\texttt{simpr},
\texttt{3simpa},
\texttt{3simpb},
\texttt{3simpc},
\texttt{simp1},
\texttt{simp2},
and
\texttt{simp3}).
If the
expression has nested conjunctions, inner conjuncts can be broken out by
chaining the above theorems with \texttt{syl}
(see section \ref{syl}).\footnote{
There are actually many theorems
(labeled simp* such as \texttt{simp333}) that break out inner conjuncts in one
step, but rather than learning them you can just use the chaining we
just described to prove them, and then let the Metamath program command
\texttt{minimize{\char`\_}with}\index{\texttt{minimize{\char`\_}with} command}
figure out the right ones needed to collapse them.}
As your final step, you can then apply the already-proven assertion Td
(which is in deduction form), proving assertion T in closed form.
\end{itemize}

We can also easily convert any assertion T in closed form to its related
assertion Ti in inference form by applying
modus ponens\index{modus ponens} (see section \ref{axmp}).

The deduction form antecedent can also be used to represent the context
necessary to support natural deduction systems, so we will now
discuss natural deduction.

\subsection{Natural Deduction}\label{naturaldeduction}

Natural deduction\index{natural deduction}
(ND) systems, as such, were originally introduced in
1934 by two logicians working independently: Ja\'skowski and Gentzen. ND
systems are supposed to reconstruct, in a formally proper way, traditional
ways of mathematical reasoning (such as conditional proof, indirect proof,
and proof by cases). As reconstructions they were naturally influenced
by previous work, and many specific ND systems and notations have been
developed since their original work.

There are many ND variants, but
Indrzejczak \cite[p.~31-32]{Indrzejczak}\index{Indrzejczak, Andrzej}
suggests that any natural deductive system must satisfy at
least these three criteria:

\begin{itemize}
\item ``There are some means for entering assumptions into a proof and
also for eliminating them. Usually it requires some bookkeeping devices
for indicating the scope of an assumption, and showing that a part of
a proof depending on eliminated assumption is discharged.
\item There are no (or, at least, a very limited set of) axioms, because
their role is taken over by the set of primitive rules for introduction
and elimination of logical constants which means that elementary
inferences instead of formulae are taken as primitive.
\item (A genuine) ND system admits a lot of freedom in proof construction
and possibility of applying several proof search strategies, like
conditional proof, proof by cases, proof by reductio ad absurdum etc.''
\end{itemize}

The Metamath Proof Explorer (MPE) as defined in \texttt{set.mm}
is fundamentally a Hilbert-style system.
That is, MPE is based on a larger number of axioms (compared
to natural deduction systems), a very small set of rules of inference
(modus ponens), and the context is not changed by the rules of inference
in the middle of a proof. That said, MPE proofs can be developed using
the natural deduction (ND) approach as originally developed by Ja\'skowski
and Gentzen.

The most common and recommended approach for applying ND in MPE is to use
deduction form\index{deduction form}%
\index{forms!deduction}
and apply the MPE proven assertions that are equivalent to ND rules.
For example, MPE's \texttt{jca} is equivalent to ND rule $\land$-I
(and-insertion).
We maintain a list of equivalences that you may consult.
This approach for applying an ND approach within MPE relies on Metamath's
wff metavariables in an essential way, and is described in more detail
in the presentation ``Natural Deductions in the Metamath Proof Language''
by Mario Carneiro \cite{CarneiroND}\index{Carneiro, Mario}.

In this style many steps are an implication, whose antecedent mimics
the context ($\Gamma$) of most ND systems. To add an assumption, simply add
it to the implication antecedent (typically using
\texttt{simpr}),
and use that
new antecedent for all later claims in the same scope. If you wish to
use an assertion in an ND hypothesis scope that is outside the current
ND hypothesis scope, modify the assertion so that the ND hypothesis
assumption is added to its antecedent (typically using \texttt{adantr}). Most
proof steps will be proved using rules that have hypotheses and results
of the form $\varphi \rightarrow$ ...

An example may make this clearer.
Let's show theorem 5.5 of
\cite[p.~18]{Clemente}\index{Clemente Laboreo, Daniel}
along with a line by line translation using the usual
translation of natural deduction (ND) in the Metamath Proof Explorer
(MPE) notation (this is proof \texttt{ex-natded5.5}).
The proof's original goal was to prove
$\lnot \psi$ given two hypotheses,
$( \psi \rightarrow \chi )$ and $ \lnot \chi$.
We will translate these statements into MPE deduction form
by prefixing them all with $\varphi \rightarrow$.
As a result, in MPE the goal is stated as
$( \varphi \rightarrow \lnot \psi )$, and the two hypotheses are stated as
$( \varphi \rightarrow ( \psi \rightarrow \chi ) )$ and
$( \varphi \rightarrow \lnot \chi )$.

The following table shows the proof in Fitch natural deduction style
and its MPE equivalent.
The \textit{\#} column shows the original numbering,
\textit{MPE\#} shows the number in the equivalent MPE proof
(which we will show later),
\textit{ND Expression} shows the original proof claim in ND notation,
and \textit{MPE Translation} shows its translation into MPE
as discussed in this section.
The final columns show the rationale in ND and MPE respectively.

\needspace{4\baselineskip}
{\setlength{\extrarowsep}{4pt} % Keep rows from being too close together
\begin{longtabu}   { @{} c c X X X X }
\textbf{\#} & \textbf{MPE\#} & \textbf{ND Ex\-pres\-sion} &
\textbf{MPE Trans\-lation} & \textbf{ND Ration\-ale} &
\textbf{MPE Ra\-tio\-nale} \\
\endhead

1 & 2;3 &
$( \psi \rightarrow \chi )$ &
$( \varphi \rightarrow ( \psi \rightarrow \chi ) )$ &
Given &
\$e; \texttt{adantr} to put in ND hypothesis \\

2 & 5 &
$ \lnot \chi$ &
$( \varphi \rightarrow \lnot \chi )$ &
Given &
\$e; \texttt{adantr} to put in ND hypothesis \\

3 & 1 &
... $\vert$ $\psi$ &
$( \varphi \rightarrow \psi )$ &
ND hypothesis assumption &
\texttt{simpr} \\

4 & 4 &
... $\chi$ &
$( ( \varphi \land \psi ) \rightarrow \chi )$ &
$\rightarrow$\,E 1,3 &
\texttt{mpd} 1,3 \\

5 & 6 &
... $\lnot \chi$ &
$( ( \varphi \land \psi ) \rightarrow \lnot \chi )$ &
IT 2 &
\texttt{adantr} 5 \\

6 & 7 &
$\lnot \psi$ &
$( \varphi \rightarrow \lnot \psi )$ &
$\land$\,I 3,4,5 &
\texttt{pm2.65da} 4,6 \\

\end{longtabu}
}


The original used Latin letters; we have replaced them with Greek letters
to follow Metamath naming conventions and so that it is easier to follow
the Metamath translation. The Metamath line-for-line translation of
this natural deduction approach precedes every line with an antecedent
including $\varphi$ and uses the Metamath equivalents of the natural deduction
rules. To add an assumption, the antecedent is modified to include it
(typically by using \texttt{adantr};
\texttt{simpr} is useful when you want to
depend directly on the new assumption, as is shown here).

In Metamath we can represent the two given statements as these hypotheses:

\needspace{2\baselineskip}
\begin{itemize}
\item ex-natded5.5.1 $\vdash ( \varphi \rightarrow ( \psi \rightarrow \chi ) )$
\item ex-natded5.5.2 $\vdash ( \varphi \rightarrow \lnot \chi )$
\end{itemize}

\needspace{4\baselineskip}
Here is the proof in Metamath as a line-by-line translation:

\begin{longtabu}   { l l l X }
\textbf{Step} & \textbf{Hyp} & \textbf{Ref} & \textbf{Ex\-pres\-sion} \\
\endhead
1 & & simpr & $\vdash ( ( \varphi \land \psi ) \rightarrow \psi )$ \\
2 & & ex-natded5.5.1 &
  $\vdash ( \varphi \rightarrow ( \psi \rightarrow \chi ) )$ \\
3 & 2 & adantr &
 $\vdash ( ( \varphi \land \psi ) \rightarrow ( \psi \rightarrow \chi ) )$ \\
4 & 1, 3 & mpd &
 $\vdash ( ( \varphi \land \psi ) \rightarrow \chi ) $ \\
5 & & ex-natded5.5.2 &
 $\vdash ( \varphi \rightarrow \lnot \chi )$ \\
6 & 5 & adantr &
 $\vdash ( ( \varphi \land \psi ) \rightarrow \lnot \chi )$ \\
7 & 4, 6 & pm2.65da &
 $\vdash ( \varphi \rightarrow \lnot \psi )$ \\
\end{longtabu}

Only using specific natural deduction rules directly can lead to very
long proofs, for exactly the same reason that only using axioms directly
in Hilbert-style proofs can lead to very long proofs.
If the goal is short and clear proofs,
then it is better to reuse already-proven assertions
in deduction form than to start from scratch each time
and using only basic natural deduction rules.

\subsection{Strengths of Our Approach}

As far as we know there is nothing else in the literature like either the
weak deduction theorem or Mario Carneiro\index{Carneiro, Mario}'s
natural deduction method.
In order to
transform a hypothesis into an antecedent, the literature's standard
``Deduction Theorem''\index{Deduction Theorem}\index{Standard Deduction Theorem}
requires metalogic outside of the notions provided
by the axiom system. We instead generally prefer to use Mario Carneiro's
natural deduction method, then use the weak deduction theorem in cases
where that is difficult to apply, and only then use the full standard
deduction theorem as a last resort.

The weak deduction theorem\index{Weak Deduction Theorem}
does not require any additional metalogic
but converts an inference directly into a closed form theorem, with
a rigorous proof that uses only the axiom system. Unlike the standard
Deduction Theorem, there is no implicit external justification that we
have to trust in order to use it.

Mario Carneiro's natural deduction\index{natural deduction}
method also does not require any new metalogical
notions. It avoids the Deduction Theorem's metalogic by prefixing the
hypotheses and conclusion of every would-be inference with a universal
antecedent (``$\varphi \rightarrow$'') from the very start.

We think it is impressive and satisfying that we can do so much in a
practical sense without stepping outside of our Hilbert-style axiom system.
Of course our axiomatization, which is in the form of schemes,
contains a metalogic of its own that we exploit. But this metalogic
is relatively simple, and for our Deduction Theorem alternatives,
we primarily use just the direct substitution of expressions for
metavariables.

\begin{sloppy}
\section{Exploring the Set The\-o\-ry Data\-base}\label{exploring}
\end{sloppy}
% NOTE: All examples performed in this section are
% recorded wtih "set width 61" % on set.mm as of 2019-05-28
% commit c1e7849557661260f77cfdf0f97ac4354fbb4f4d.

At this point you may wish to study the \texttt{set.mm}\index{set theory
database (\texttt{set.mm})} file in more detail.  Pay particular
attention to the assumptions needed to define wffs\index{well-formed
formula (wff)} (which are not included above), the variable types
(\texttt{\$f}\index{\texttt{\$f} statement} statements), and the
definitions that are introduced.  Start with some simple theorems in
propositional calculus, making sure you understand in detail each step
of a proof.  Once you get past the first few proofs and become familiar
with the Metamath language, any part of the \texttt{set.mm} database
will be as easy to follow, step by step, as any other part---you won't
have to undergo a ``quantum leap'' in mathematical sophistication to be
able to follow a deep proof in set theory.

Next, you may want to explore how concepts such as natural numbers are
defined and described.  This is probably best done in conjunction with
standard set theory textbooks, which can help give you a higher-level
understanding.  The \texttt{set.mm} database provides references that will get
you started.  From there, you will be on your way towards a very deep,
rigorous understanding of abstract mathematics.

The Metamath\index{Metamath} program can help you peruse a Metamath data\-base,
wheth\-er you are trying to figure out how a certain step follows in a proof or
just have a general curiosity.  We will go through some examples of the
commands, using the \texttt{set.mm}\index{set theory database (\texttt{set.mm})}
database provided with the Metamath software.  These should help get you
started.  See Chapter~\ref{commands} for a more detailed description of
the commands.  Note that we have included the full spelling of all commands to
prevent ambiguity with future commands.  In practice you may type just the
characters needed to specify each command keyword\index{command keyword}
unambiguously, often just one or two characters per keyword, and you don't
need to type them in upper case.

First run the Metamath program as described earlier.  You should see the
\verb/MM>/ prompt.  Read in the \texttt{set.mm} file:\index{\texttt{read}
command}

\begin{verbatim}
MM> read set.mm
Reading source file "set.mm"... 34554442 bytes
34554442 bytes were read into the source buffer.
The source has 155711 statements; 2254 are $a and 32250 are $p.
No errors were found.  However, proofs were not checked.
Type VERIFY PROOF * if you want to check them.
\end{verbatim}

As with most examples in this book, what you will see
will be slightly different because we are continuously
improving our databases (including \texttt{set.mm}).

Let's check the database integrity.  This may take a minute or two to run if
your computer is slow.

\begin{verbatim}
MM> verify proof *
0 10%  20%  30%  40%  50%  60%  70%  80%  90% 100%
..................................................
All proofs in the database were verified in 2.84 s.
\end{verbatim}

No errors were reported, so every proof is correct.

You need to know the names (labels) of theorems before you can look at them.
Often just examining the database file(s) with a text editor is the best
approach.  In \texttt{set.mm} there are many detailed comments, especially near
the beginning, that can help guide you. The \texttt{search} command in the
Metamath program is also handy.  The \texttt{comments} qualifier will list the
statements whose associated comment (the one immediately before it) contain a
string you give it.  For example, if you are studying Enderton's {\em Elements
of Set Theory} \cite{Enderton}\index{Enderton, Herbert B.} you may want to see
the references to it in the database.  The search string \texttt{enderton} is not
case sensitive.  (This will not show you all the database theorems that are in
Enderton's book because there is usually only one citation for a given
theorem, which may appear in several textbooks.)\index{\texttt{search}
command}

\begin{verbatim}
MM> search * "enderton" / comments
12067 unineq $p "... Exercise 20 of [Enderton] p. 32 and ..."
12459 undif2 $p "...Corollary 6K of [Enderton] p. 144. (C..."
12953 df-tp $a "...s. Definition of [Enderton] p. 19. (Co..."
13689 unissb $p ".... Exercise 5 of [Enderton] p. 26 and ..."
\end{verbatim}
\begin{center}
(etc.)
\end{center}

Or you may want to see what theorems have something to do with
conjunction (logical {\sc and}).  The quotes around the search
string are optional when there's no ambiguity.\index{\texttt{search}
command}

\begin{verbatim}
MM> search * conjunction / comments
120 a1d $p "...be replaced with a conjunction ( ~ df-an )..."
662 df-bi $a "...viated form after conjunction is introdu..."
1319 wa $a "...ff definition to include conjunction ('and')."
1321 df-an $a "Define conjunction (logical 'and'). Defini..."
1420 imnan $p "...tion in terms of conjunction. (Contribu..."
\end{verbatim}
\begin{center}
(etc.)
\end{center}


Now we will start to look at some details.  Let's look at the first
axiom of propositional calculus
(we could use \texttt{sh st} to abbreviate
\texttt{show statement}).\index{\texttt{show statement} command}

\begin{verbatim}
MM> show statement ax-1/full
Statement 19 is located on line 881 of the file "set.mm".
"Axiom _Simp_.  Axiom A1 of [Margaris] p. 49.  One of the 3
axioms of propositional calculus.  The 3 axioms are also
        ...
19 ax-1 $a |- ( ph -> ( ps -> ph ) ) $.
Its mandatory hypotheses in RPN order are:
  wph $f wff ph $.
  wps $f wff ps $.
The statement and its hypotheses require the variables:  ph
      ps
The variables it contains are:  ph ps


Statement 49 is located on line 11182 of the file "set.mm".
Its statement number for HTML pages is 6.
"Axiom _Simp_.  Axiom A1 of [Margaris] p. 49.  One of the 3
axioms of propositional calculus.  The 3 axioms are also
given as Definition 2.1 of [Hamilton] p. 28.
...
49 ax-1 $a |- ( ph -> ( ps -> ph ) ) $.
Its mandatory hypotheses in RPN order are:
  wph $f wff ph $.
  wps $f wff ps $.
The statement and its hypotheses require the variables:
  ph ps
The variables it contains are:  ph ps
\end{verbatim}

Compare this to \texttt{ax-1} on p.~\pageref{ax1}.  You can see that the
symbol \texttt{ph} is the {\sc ascii} notation for $\varphi$, etc.  To
see the mathematical symbols for any expression you may typeset it in
\LaTeX\ (type \texttt{help tex} for instructions)\index{latex@{\LaTeX}}
or, easier, just use a text editor to look at the comments where symbols
are first introduced in \texttt{set.mm}.  The hypotheses \texttt{wph}
and \texttt{wps} required by \texttt{ax-1} mean that variables
\texttt{ph} and \texttt{ps} must be wffs.

Next we'll pick a simple theorem of propositional calculus, the Principle of
Identity, which is proved directly from the axioms.  We'll look at the
statement then its proof.\index{\texttt{show statement}
command}

\begin{verbatim}
MM> show statement id1/full
Statement 116 is located on line 11371 of the file "set.mm".
Its statement number for HTML pages is 22.
"Principle of identity.  Theorem *2.08 of [WhiteheadRussell]
p. 101.  This version is proved directly from the axioms for
demonstration purposes.
...
116 id1 $p |- ( ph -> ph ) $= ... $.
Its mandatory hypotheses in RPN order are:
  wph $f wff ph $.
Its optional hypotheses are:  wps wch wth wta wet
      wze wsi wrh wmu wla wka
The statement and its hypotheses require the variables:  ph
These additional variables are allowed in its proof:
      ps ch th ta et ze si rh mu la ka
The variables it contains are:  ph
\end{verbatim}

The optional variables\index{optional variable} \texttt{ps}, \texttt{ch}, etc.\ are
available for use in a proof of this statement if we wish, and were we to do
so we would make use of optional hypotheses \texttt{wps}, \texttt{wch}, etc.  (See
Section~\ref{dollaref} for the meaning of ``optional
hypothesis.''\index{optional hypothesis}) The reason these show up in the
statement display is that statement \texttt{id1} happens to be in their scope
(see Section~\ref{scoping} for the definition of ``scope''\index{scope}), but
in fact in propositional calculus we will never make use of optional
hypotheses or variables.  This becomes important after quantifiers are
introduced, where ``dummy'' variables\index{dummy variable} are often needed
in the middle of a proof.

Let's look at the proof of statement \texttt{id1}.  We'll use the
\texttt{show proof} command, which by default suppresses the
``non-essential'' steps that construct the wffs.\index{\texttt{show proof}
command}
We will display the proof in ``lemmon' format (a non-indented format
with explicit previous step number references) and renumber the
displayed steps:

\begin{verbatim}
MM> show proof id1 /lemmon/renumber
1 ax-1           $a |- ( ph -> ( ph -> ph ) )
2 ax-1           $a |- ( ph -> ( ( ph -> ph ) -> ph ) )
3 ax-2           $a |- ( ( ph -> ( ( ph -> ph ) -> ph ) ) ->
                     ( ( ph -> ( ph -> ph ) ) -> ( ph -> ph )
                                                          ) )
4 2,3 ax-mp      $a |- ( ( ph -> ( ph -> ph ) ) -> ( ph -> ph
                                                          ) )
5 1,4 ax-mp      $a |- ( ph -> ph )
\end{verbatim}

If you have read Section~\ref{trialrun}, you'll know how to interpret this
proof.  Step~2, for example, is an application of axiom \texttt{ax-1}.  This
proof is identical to the one in Hamilton's {\em Logic for Mathematicians}
\cite[p.~32]{Hamilton}\index{Hamilton, Alan G.}.

You may want to look at what
substitutions\index{substitution!variable}\index{variable substitution} are
made into \texttt{ax-1} to arrive at step~2.  The command to do this needs to
know the ``real'' step number, so we'll display the proof again without
the \texttt{renumber} qualifier.\index{\texttt{show proof}
command}

\begin{verbatim}
MM> show proof id1 /lemmon
 9 ax-1          $a |- ( ph -> ( ph -> ph ) )
20 ax-1          $a |- ( ph -> ( ( ph -> ph ) -> ph ) )
24 ax-2          $a |- ( ( ph -> ( ( ph -> ph ) -> ph ) ) ->
                     ( ( ph -> ( ph -> ph ) ) -> ( ph -> ph )
                                                          ) )
25 20,24 ax-mp   $a |- ( ( ph -> ( ph -> ph ) ) -> ( ph -> ph
                                                          ) )
26 9,25 ax-mp    $a |- ( ph -> ph )
\end{verbatim}

The ``real'' step number is 20.  Let's look at its details.

\begin{verbatim}
MM> show proof id1 /detailed_step 20
Proof step 20:  min=ax-1 $a |- ( ph -> ( ( ph -> ph ) -> ph )
  )
This step assigns source "ax-1" ($a) to target "min" ($e).
The source assertion requires the hypotheses "wph" ($f, step
18) and "wps" ($f, step 19).  The parent assertion of the
target hypothesis is "ax-mp" ($a, step 25).
The source assertion before substitution was:
    ax-1 $a |- ( ph -> ( ps -> ph ) )
The following substitutions were made to the source
assertion:
    Variable  Substituted with
     ph        ph
     ps        ( ph -> ph )
The target hypothesis before substitution was:
    min $e |- ph
The following substitution was made to the target hypothesis:
    Variable  Substituted with
     ph        ( ph -> ( ( ph -> ph ) -> ph ) )
\end{verbatim}

This shows the substitutions\index{substitution!variable}\index{variable
substitution} made to the variables in \texttt{ax-1}.  References are made to
steps 18 and 19 which are not shown in our proof display.  To see these steps,
you can display the proof with the \texttt{all} qualifier.

Let's look at a slightly more advanced proof of propositional calculus.  Note
that \verb+/\+ is the symbol for $\wedge$ (logical {\sc and}, also
called conjunction).\index{conjunction ($\wedge$)}
\index{logical {\sc and} ($\wedge$)}

\begin{verbatim}
MM> show statement prth/full
Statement 1791 is located on line 15503 of the file "set.mm".
Its statement number for HTML pages is 559.
"Conjoin antecedents and consequents of two premises.  This
is the closed theorem form of ~ anim12d .  Theorem *3.47 of
[WhiteheadRussell] p. 113.  It was proved by Leibniz,
and it evidently pleased him enough to call it
_praeclarum theorema_ (splendid theorem).
...
1791 prth $p |- ( ( ( ph -> ps ) /\ ( ch -> th ) ) -> ( ( ph
      /\ ch ) -> ( ps /\ th ) ) ) $= ... $.
Its mandatory hypotheses in RPN order are:
  wph $f wff ph $.
  wps $f wff ps $.
  wch $f wff ch $.
  wth $f wff th $.
Its optional hypotheses are:  wta wet wze wsi wrh wmu wla wka
The statement and its hypotheses require the variables:  ph
      ps ch th
These additional variables are allowed in its proof:  ta et
      ze si rh mu la ka
The variables it contains are:  ph ps ch th


MM> show proof prth /lemmon/renumber
1 simpl          $p |- ( ( ( ph -> ps ) /\ ( ch -> th ) ) ->
                                               ( ph -> ps ) )
2 simpr          $p |- ( ( ( ph -> ps ) /\ ( ch -> th ) ) ->
                                               ( ch -> th ) )
3 1,2 anim12d    $p |- ( ( ( ph -> ps ) /\ ( ch -> th ) ) ->
                           ( ( ph /\ ch ) -> ( ps /\ th ) ) )
\end{verbatim}

There are references to a lot of unfamiliar statements.  To see what they are,
you may type the following:

\begin{verbatim}
MM> show proof prth /statement_summary
Summary of statements used in the proof of "prth":

Statement simpl is located on line 14748 of the file
"set.mm".
"Elimination of a conjunct.  Theorem *3.26 (Simp) of
[WhiteheadRussell] p. 112. ..."
  simpl $p |- ( ( ph /\ ps ) -> ph ) $= ... $.

Statement simpr is located on line 14777 of the file
"set.mm".
"Elimination of a conjunct.  Theorem *3.27 (Simp) of
[WhiteheadRussell] ..."
  simpr $p |- ( ( ph /\ ps ) -> ps ) $= ... $.

Statement anim12d is located on line 15445 of the file
"set.mm".
"Conjoin antecedents and consequents in a deduction.
..."
  anim12d.1 $e |- ( ph -> ( ps -> ch ) ) $.
  anim12d.2 $e |- ( ph -> ( th -> ta ) ) $.
  anim12d $p |- ( ph -> ( ( ps /\ th ) -> ( ch /\ ta ) ) )
      $= ... $.
\end{verbatim}
\begin{center}
(etc.)
\end{center}

Of course you can look at each of these statements and their proofs, and
so on, back to the axioms of propositional calculus if you wish.

The \texttt{search} command is useful for finding statements when you
know all or part of their contents.  The following example finds all
statements containing \verb@ph -> ps@ followed by \verb@ch -> th@.  The
\verb@$*@ is a wildcard that matches anything; the \texttt{\$} before the
\verb$*$ prevents conflicts with math symbol token names.  The \verb@*@ after
\texttt{SEARCH} is also a wildcard that in this case means ``match any label.''
\index{\texttt{search} command}

% I'm omitting this one, since readers are unlikely to see it:
% 1096 bisymOLD $p |- ( ( ( ph -> ps ) -> ( ch -> th ) ) -> ( (
%   ( ps -> ph ) -> ( th -> ch ) ) -> ( ( ph <-> ps ) -> ( ch
%    <-> th ) ) ) )
\begin{verbatim}
MM> search * "ph -> ps $* ch -> th"
1791 prth $p |- ( ( ( ph -> ps ) /\ ( ch -> th ) ) -> ( ( ph
    /\ ch ) -> ( ps /\ th ) ) )
2455 pm3.48 $p |- ( ( ( ph -> ps ) /\ ( ch -> th ) ) -> ( (
    ph \/ ch ) -> ( ps \/ th ) ) )
117859 pm11.71 $p |- ( ( E. x ph /\ E. y ch ) -> ( ( A. x (
    ph -> ps ) /\ A. y ( ch -> th ) ) <-> A. x A. y ( ( ph /\
    ch ) -> ( ps /\ th ) ) ) )
\end{verbatim}

Three statements, \texttt{prth}, \texttt{pm3.48},
 and \texttt{pm11.71}, were found to match.

To see what axioms\index{axiom} and definitions\index{definition}
\texttt{prth} ultimately depends on for its proof, you can have the
program backtrack through the hierarchy\index{hierarchy} of theorems and
definitions.\index{\texttt{show trace{\char`\_}back} command}

\begin{verbatim}
MM> show trace_back prth /essential/axioms
Statement "prth" assumes the following axioms ($a
statements):
  ax-1 ax-2 ax-3 ax-mp df-bi df-an
\end{verbatim}

Note that the 3 axioms of propositional calculus and the modus ponens rule are
needed (as expected); in addition, there are a couple of definitions that are used
along the way.  Note that Metamath makes no distinction\index{axiom vs.\
definition} between axioms\index{axiom} and definitions\index{definition}.  In
\texttt{set.mm} they have been distinguished artificially by prefixing their
labels\index{labels in \texttt{set.mm}} with \texttt{ax-} and \texttt{df-}
respectively.  For example, \texttt{df-an} defines conjunction (logical {\sc
and}), which is represented by the symbol \verb+/\+.
Section~\ref{definitions} discusses the philosophy of definitions, and the
Metamath language takes a particularly simple, conservative approach by using
the \texttt{\$a}\index{\texttt{\$a} statement} statement for both axioms and
definitions.

You can also have the program compute how many steps a proof
has\index{proof length} if we were to follow it all the way back to
\texttt{\$a} statements.

\begin{verbatim}
MM> show trace_back prth /essential/count_steps
The statement's actual proof has 3 steps.  Backtracking, a
total of 79 different subtheorems are used.  The statement
and subtheorems have a total of 274 actual steps.  If
subtheorems used only once were eliminated, there would be a
total of 38 subtheorems, and the statement and subtheorems
would have a total of 185 steps.  The proof would have 28349
steps if fully expanded back to axiom references.  The
maximum path length is 38.  A longest path is:  prth <-
anim12d <- syl2and <- sylan2d <- ancomsd <- ancom <- pm3.22
<- pm3.21 <- pm3.2 <- ex <- sylbir <- biimpri <- bicomi <-
bicom1 <- bi2 <- dfbi1 <- impbii <- bi3 <- simprim <- impi <-
con1i <- nsyl2 <- mt3d <- con1d <- notnot1 <- con2i <- nsyl3
<- mt2d <- con2d <- notnot2 <- pm2.18d <- pm2.18 <- pm2.21 <-
pm2.21d <- a1d <- syl <- mpd <- a2i <- a2i.1 .
\end{verbatim}

This tells us that we would have to inspect 274 steps if we want to
verify the proof completely starting from the axioms.  A few more
statistics are also shown.  There are one or more paths back to axioms
that are the longest; this command ferrets out one of them and shows it
to you.  There may be a sense in which the longest path length is
related to how ``deep'' the theorem is.

We might also be curious about what proofs depend on the theorem
\texttt{prth}.  If it is never used later on, we could eliminate it as
redundant if it has no intrinsic interest by itself.\index{\texttt{show
usage} command}

% I decided to show the OLD values here.
\begin{verbatim}
MM> show usage prth
Statement "prth" is directly referenced in the proofs of 18
statements:
  mo3 moOLD 2mo 2moOLD euind reuind reuss2 reusv3i opelopabt
  wemaplem2 rexanre rlimcn2 o1of2 o1rlimmul 2sqlem6 spanuni
  heicant pm11.71
\end{verbatim}

Thus \texttt{prth} is directly used by 18 proofs.
We can use the \texttt{/recursive} qualifier to include indirect use:

\begin{verbatim}
MM> show usage prth /recursive
Statement "prth" directly or indirectly affects the proofs of
24214 statements:
  mo3 mo mo3OLD eu2 moOLD eu2OLD eu3OLD mo4f mo4 eu4 mopick
...
\end{verbatim}

\subsection{A Note on the ``Compact'' Proof Format}

The Metamath program will display proofs in a ``compact''\index{compact proof}
format whenever the proof is stored in compressed format in the database.  It
may be be slightly confusing unless you know how to interpret it.
For example,
if you display the complete proof of theorem \texttt{id1} it will start
off as follows:

\begin{verbatim}
MM> show proof id1 /lemmon/all
 1 wph           $f wff ph
 2 wph           $f wff ph
 3 wph           $f wff ph
 4 2,3 wi    @4: $a wff ( ph -> ph )
 5 1,4 wi    @5: $a wff ( ph -> ( ph -> ph ) )
 6 @4            $a wff ( ph -> ph )
\end{verbatim}

\begin{center}
{etc.}
\end{center}

Step 4 has a ``local label,''\index{local label} \texttt{@4}, assigned to it.
Later on, at step 6, the label \texttt{@1} is referenced instead of
displaying the explicit proof for that step.  This technique takes advantage
of the fact that steps in a proof often repeat, especially during the
construction of wffs.  The compact format reduces the number of steps in the
proof display and may be preferred by some people.

If you want to see the normal format with the ``true'' step numbers, you can
use the following workaround:\index{\texttt{save proof} command}

\begin{verbatim}
MM> save proof id1 /normal
The proof of "id1" has been reformatted and saved internally.
Remember to use WRITE SOURCE to save it permanently.
MM> show proof id1 /lemmon/all
 1 wph           $f wff ph
 2 wph           $f wff ph
 3 wph           $f wff ph
 4 2,3 wi        $a wff ( ph -> ph )
 5 1,4 wi        $a wff ( ph -> ( ph -> ph ) )
 6 wph           $f wff ph
 7 wph           $f wff ph
 8 6,7 wi        $a wff ( ph -> ph )
\end{verbatim}

\begin{center}
{etc.}
\end{center}

Note that the original 6 steps are now 8 steps.  However, the format is
now the same as that described in Chapter~\ref{using}.

\chapter{The Metamath Language}
\label{languagespec}

\begin{quote}
  {\em Thus mathematics may be defined as the subject in which we never know
what we are talking about, nor whether what we are saying is true.}
    \flushright\sc  Bertrand Russell\footnote{\cite[p.~84]{Russell2}.}\\
\end{quote}\index{Russell, Bertrand}

Probably the most striking feature of the Metamath language is its almost
complete absence of hard-wired syntax. Metamath\index{Metamath} does not
understand any mathematics or logic other than that needed to construct finite
sequences of symbols according to a small set of simple, built-in rules.  The
only rule it uses in a proof is the substitution of an expression (symbol
sequence) for a variable, subject to a simple constraint to prevent
bound-variable clashes.  The primitive notions built into Metamath involve the
simple manipulation of finite objects (symbols) that we as humans can easily
visualize and that computers can easily deal with.  They seem to be just
about the simplest notions possible that are required to do standard
mathematics.

This chapter serves as a reference manual for the Metamath\index{Metamath}
language. It covers the tedious technical details of the language, some of
which you may wish to skim in a first reading.  On the other hand, you should
pay close attention to the defined terms in {\bf boldface}; they have precise
meanings that are important to keep in mind for later understanding.  It may
be best to first become familiar with the examples in Chapter~\ref{using} to
gain some motivation for the language.

%% Uncomment this when uncommenting section {formalspec} below
If you have some knowledge of set theory, you may wish to study this
chapter in conjunction with the formal set-theoretical description of the
Metamath language in Appendix~\ref{formalspec}.

We will use the name ``Metamath''\index{Metamath} to mean either the Metamath
computer language or the Metamath software associated with the computer
language.  We will not distinguish these two when the context is clear.

The next section contains the complete specification of the Metamath
language.
It serves as an
authoritative reference and presents the syntax in enough detail to
write a parser\index{parsing Metamath} and proof verifier.  The
specification is terse and it is probably hard to learn the language
directly from it, but we include it here for those impatient people who
prefer to see everything up front before looking at verbose expository
material.  Later sections explain this material and provide examples.
We will repeat the definitions in those sections, and you may skip the
next section at first reading and proceed to Section~\ref{tut1}
(p.~\pageref{tut1}).

\section{Specification of the Metamath Language}\label{spec}
\index{Metamath!specification}

\begin{quote}
  {\em Sometimes one has to say difficult things, but one ought to say
them as simply as one knows how.}
    \flushright\sc  G. H. Hardy\footnote{As quoted in
    \cite{deMillo}, p.~273.}\\
\end{quote}\index{Hardy, G. H.}

\subsection{Preliminaries}\label{spec1}

% Space is technically a printable character, so we'll word things
% carefully so it's unambiguous.
A Metamath {\bf database}\index{database} is built up from a top-level source
file together with any source files that are brought in through file inclusion
commands (see below).  The only characters that are allowed to appear in a
Metamath source file are the 94 non-whitespace printable {\sc
ascii}\index{ascii@{\sc ascii}} characters, which are digits, upper and lower
case letters, and the following 32 special
characters\index{special characters}:\label{spec1chars}

\begin{verbatim}
! " # $ % & ' ( ) * + , - . / :
; < = > ? @ [ \ ] ^ _ ` { | } ~
\end{verbatim}
plus the following characters which are the ``white space'' characters:
space (a printable character),
tab, carriage return, line feed, and form feed.\label{whitespace}
We will use \texttt{typewriter}
font to display the printable characters.

A Metamath database consists of a sequence of three kinds of {\bf
tokens}\index{token} separated by {\bf white space}\index{white space}
(which is any sequence of one or more white space characters).  The set
of {\bf keyword}\index{keyword} tokens is \texttt{\$\char`\{},
\texttt{\$\char`\}}, \texttt{\$c}, \texttt{\$v}, \texttt{\$f},
\texttt{\$e}, \texttt{\$d}, \texttt{\$a}, \texttt{\$p}, \texttt{\$.},
\texttt{\$=}, \texttt{\$(}, \texttt{\$)}, \texttt{\$[}, and
\texttt{\$]}.  The last four are called {\bf auxiliary}\index{auxiliary
keyword} or preprocessing keywords.  A {\bf label}\index{label} token
consists of any combination of letters, digits, and the characters
hyphen, underscore, and period.  A {\bf math symbol}\index{math symbol}
token may consist of any combination of the 93 printable standard {\sc
ascii} characters other than space or \texttt{\$}~. All tokens are
case-sensitive.

\subsection{Preprocessing}

The token \texttt{\$(} begins a {\bf comment} and
\texttt{\$)} ends a comment.\index{\texttt{\$(}
and \texttt{\$)} auxiliary keywords}\index{comment}
Comments may contain any of
the 94 non-whitespace printable characters and white space,
except they may not contain the
2-character sequences \texttt{\$(} or \texttt{\$)} (comments do not nest).
Comments are ignored (treated
like white space) for the purpose of parsing, e.g.,
\texttt{\$( \$[ \$)} is a comment.
See p.~\pageref{mathcomments} for comment typesetting conventions; these
conventions may be ignored for the purpose of parsing.

A {\bf file inclusion command} consists of \texttt{\$[} followed by a file name
followed by \texttt{\$]}.\index{\texttt{\$[} and \texttt{\$]} auxiliary
keywords}\index{included file}\index{file inclusion}
It is only allowed in the outermost scope (i.e., not between
\texttt{\$\char`\{} and \texttt{\$\char`\}})
and must not be inside a statement (e.g., it may not occur
between the label of a \texttt{\$a} statement and its \texttt{\$.}).
The file name may not
contain a \texttt{\$} or white space.  The file must exist.
The case-sensitivity
of its name follows the conventions of the operating system.  The contents of
the file replace the inclusion command.
Included files may include other files.
Only the first reference to a given file is included; any later
references to the same file (whether in the top-level file or in included
files) cause the inclusion command to be ignored (treated like white space).
A verifier may assume that file names with different strings
refer to different files for the purpose of ignoring later references.
A file self-reference is ignored, as is any reference to the top-level file
(to avoid loops).
Included files may not include a \texttt{\$(} without a matching \texttt{\$)},
may not include a \texttt{\$[} without a matching \texttt{\$]}, and may
not include incomplete statements (e.g., a \texttt{\$a} without a matching
\texttt{\$.}).
It is currently unspecified if path references are relative to the process'
current directory or the file's containing directory, so databases should
avoid using pathname separators (e.g., ``/'') in file names.

Like all tokens, the \texttt{\$(}, \texttt{\$)}, \texttt{\$[}, and \texttt{\$]} keywords
must be surrounded by white space.

\subsection{Basic Syntax}

After preprocessing, a database will consist of a sequence of {\bf
statements}.
These are the scoping statements \texttt{\$\char`\{} and
\texttt{\$\char`\}}, along with the \texttt{\$c}, \texttt{\$v},
\texttt{\$f}, \texttt{\$e}, \texttt{\$d}, \texttt{\$a}, and \texttt{\$p}
statements.

A {\bf scoping statement}\index{scoping statement} consists only of its
keyword, \texttt{\$\char`\{} or \texttt{\$\char`\}}.
A \texttt{\$\char`\{} begins a {\bf
block}\index{block} and a matching \texttt{\$\char`\}} ends the block.
Every \texttt{\$\char`\{}
must have a matching \texttt{\$\char`\}}.
Defining it recursively, we say a block
contains a sequence of zero or more tokens other
than \texttt{\$\char`\{} and \texttt{\$\char`\}} and
possibly other blocks.  There is an {\bf outermost
block}\index{block!outermost} not bracketed by \texttt{\$\char`\{} \texttt{\$\char`\}}; the end
of the outermost block is the end of the database.

% LaTeX bug? can't do \bf\tt

A {\bf \$v} or {\bf \$c statement}\index{\texttt{\$v} statement}\index{\texttt{\$c}
statement} consists of the keyword token \texttt{\$v} or \texttt{\$c} respectively,
followed by one or more math symbols,
% The word "token" is used to distinguish "$." from the sentence-ending period.
followed by the \texttt{\$.}\ token.
These
statements {\bf declare}\index{declaration} the math symbols to be {\bf
variables}\index{variable!Metamath} or {\bf constants}\index{constant}
respectively. The same math symbol may not occur twice in a given \texttt{\$v} or
\texttt{\$c} statement.

%c%A math symbol becomes an {\bf active}\index{active math symbol}
%c%when declared and stays active until the end of the block in which it is
%c%declared.  A math symbol may not be declared a second time while it is active,
%c%but it may be declared again after it becomes inactive.

A math symbol becomes {\bf active}\index{active math symbol} when declared
and stays active until the end of the block in which it is declared.  A
variable may not be declared a second time while it is active, but it
may be declared again (as a variable, but not as a constant) after it
becomes inactive.  A constant must be declared in the outermost block and may
not be declared a second time.\index{redeclaration of symbols}

A {\bf \$f statement}\index{\texttt{\$f} statement} consists of a label,
followed by \texttt{\$f}, followed by its typecode (an active constant),
followed by an
active variable, followed by the \texttt{\$.}\ token.  A {\bf \$e
statement}\index{\texttt{\$e} statement} consists of a label, followed
by \texttt{\$e}, followed by its typecode (an active constant),
followed by zero or more
active math symbols, followed by the \texttt{\$.}\ token.  A {\bf
hypothesis}\index{hypothesis} is a \texttt{\$f} or \texttt{\$e}
statement.
The type declared by a \texttt{\$f} statement for a given label
is global even if the variable is not
(e.g., a database may not have \texttt{wff P} in one local scope
and \texttt{class P} in another).

A {\bf simple \$d statement}\index{\texttt{\$d} statement!simple}
consists of \texttt{\$d}, followed by two different active variables,
followed by the \texttt{\$.}\ token.  A {\bf compound \$d
statement}\index{\texttt{\$d} statement!compound} consists of
\texttt{\$d}, followed by three or more variables (all different),
followed by the \texttt{\$.}\ token.  The order of the variables in a
\texttt{\$d} statement is unimportant.  A compound \texttt{\$d}
statement is equivalent to a set of simple \texttt{\$d} statements, one
for each possible pair of variables occurring in the compound
\texttt{\$d} statement.  Henceforth in this specification we shall
assume all \texttt{\$d} statements are simple.  A \texttt{\$d} statement
is also called a {\bf disjoint} (or {\bf distinct}) {\bf variable
restriction}.\index{disjoint-variable restriction}

A {\bf \$a statement}\index{\texttt{\$a} statement} consists of a label,
followed by \texttt{\$a}, followed by its typecode (an active constant),
followed by
zero or more active math symbols, followed by the \texttt{\$.}\ token.  A {\bf
\$p statement}\index{\texttt{\$p} statement} consists of a label,
followed by \texttt{\$p}, followed by its typecode (an active constant),
followed by
zero or more active math symbols, followed by \texttt{\$=}, followed by
a sequence of labels, followed by the \texttt{\$.}\ token.  An {\bf
assertion}\index{assertion} is a \texttt{\$a} or \texttt{\$p} statement.

A \texttt{\$f}, \texttt{\$e}, or \texttt{\$d} statement is {\bf active}\index{active
statement} from the place it occurs until the end of the block it occurs in.
A \texttt{\$a} or \texttt{\$p} statement is {\bf active} from the place it occurs
through the end of the database.
There may not be two active \texttt{\$f} statements containing the same
variable.  Each variable in a \texttt{\$e}, \texttt{\$a}, or
\texttt{\$p} statement must exist in an active \texttt{\$f}
statement.\footnote{This requirement can greatly simplify the
unification algorithm (substitution calculation) required by proof
verification.}

%The label that begins each \texttt{\$f}, \texttt{\$e}, \texttt{\$a}, and
%\texttt{\$p} statement must be unique.
Each label token must be unique, and
no label token may match any math symbol
token.\label{namespace}\footnote{This
restriction was added on June 24, 2006.
It is not theoretically necessary but is imposed to make it easier to
write certain parsers.}

The set of {\bf mandatory variables}\index{mandatory variable} associated with
an assertion is the set of (zero or more) variables in the assertion and in any
active \texttt{\$e} statements.  The (possibly empty) set of {\bf mandatory
hypotheses}\index{mandatory hypothesis} is the set of all active \texttt{\$f}
statements containing mandatory variables, together with all active \texttt{\$e}
statements.
The set of {\bf mandatory {\bf \$d} statements}\index{mandatory
disjoint-variable restriction} associated with an assertion are those active
\texttt{\$d} statements whose variables are both among the assertion's
mandatory variables.

\subsection{Proof Verification}\label{spec4}

The sequence of labels between the \texttt{\$=} and \texttt{\$.}\ tokens
in a \texttt{\$p} statement is a {\bf proof}.\index{proof!Metamath} Each
label in a proof must be the label of an active statement other than the
\texttt{\$p} statement itself; thus a label must refer either to an
active hypothesis of the \texttt{\$p} statement or to an earlier
assertion.

An {\bf expression}\index{expression} is a sequence of math symbols. A {\bf
substitution map}\index{substitution map} associates a set of variables with a
set of expressions.  It is acceptable for a variable to be mapped to an
expression containing it.  A {\bf
substitution}\index{substitution!variable}\index{variable substitution} is the
simultaneous replacement of all variables in one or more expressions with the
expressions that the variables map to.

A proof is scanned in order of its label sequence.  If the label refers to an
active hypothesis, the expression in the hypothesis is pushed onto a
stack.\index{stack}\index{RPN stack}  If the label refers to an assertion, a
(unique) substitution must exist that, when made to the mandatory hypotheses
of the referenced assertion, causes them to match the topmost (i.e.\ most
recent) entries of the stack, in order of occurrence of the mandatory
hypotheses, with the topmost stack entry matching the last mandatory
hypothesis of the referenced assertion.  As many stack entries as there are
mandatory hypotheses are then popped from the stack.  The same substitution is
made to the referenced assertion, and the result is pushed onto the stack.
After the last label in the proof is processed, the stack must have a single
entry that matches the expression in the \texttt{\$p} statement containing the
proof.

%c%{\footnotesize\begin{quotation}\index{redeclaration of symbols}
%c%{{\em Comment.}\label{spec4comment} Whenever a math symbol token occurs in a
%c%{\texttt{\$c} or \texttt{\$v} statement, it is considered to designate a distinct new
%c%{symbol, even if the same token was previously declared (and is now inactive).
%c%{Thus a math token declared as a constant in two different blocks is considered
%c%{to designate two distinct constants (even though they have the same name).
%c%{The two constants will not match in a proof that references both blocks.
%c%{However, a proof referencing both blocks is acceptable as long as it doesn't
%c%{require that the constants match.  Similarly, a token declared to be a
%c%{constant for a referenced assertion will not match the same token declared to
%c%{be a variable for the \texttt{\$p} statement containing the proof.  In the case
%c%{of a token declared to be a variable for a referenced assertion, this is not
%c%{an issue since the variable can be substituted with whatever expression is
%c%{needed to achieve the required match.
%c%{\end{quotation}}
%c2%A proof may reference an assertion that contains or whose hypotheses contain a
%c2%constant that is not active for the \texttt{\$p} statement containing the proof.
%c2%However, the final result of the proof may not contain that constant. A proof
%c2%may also reference an assertion that contains or whose hypotheses contain a
%c2%variable that is not active for the \texttt{\$p} statement containing the proof.
%c2%That variable, of course, will be substituted with whatever expression is
%c2%needed to achieve the required match.

A proof may contain a \texttt{?}\ in place of a label to indicate an unknown step
(Section~\ref{unknown}).  A proof verifier may ignore any proof containing
\texttt{?}\ but should warn the user that the proof is incomplete.

A {\bf compressed proof}\index{compressed proof}\index{proof!compressed} is an
alternate proof notation described in Appen\-dix~\ref{compressed}; also see
references to ``compressed proof'' in the Index.  Compressed proofs are a
Metamath language extension which a complete proof verifier should be able to
parse and verify.

\subsubsection{Verifying Disjoint Variable Restrictions}

Each substitution made in a proof must be checked to verify that any
disjoint variable restrictions are satisfied, as follows.

If two variables replaced by a substitution exist in a mandatory \texttt{\$d}
statement\index{\texttt{\$d} statement} of the assertion referenced, the two
expressions resulting from the substitution must satisfy the following
conditions.  First, the two expressions must have no variables in common.
Second, each possible pair of variables, one from each expression, must exist
in an active \texttt{\$d} statement of the \texttt{\$p} statement containing the
proof.

\vskip 1ex

This ends the specification of the Metamath language;
see Appendix \ref{BNF} for its syntax in
Extended Backus--Naur Form (EBNF)\index{Extended Backus--Naur Form}\index{EBNF}.

\section{The Basic Keywords}\label{tut1}

Our expository material begins here.

Like most computer languages, Metamath\index{Metamath} takes its input from
one or more {\bf source files}\index{source file} which contain characters
expressed in the standard {\sc ascii} (American Standard Code for Information
Interchange)\index{ascii@{\sc ascii}} code for computers.  A source file
consists of a series of {\bf tokens}\index{token}, which are strings of
non-whitespace
printable characters (from the set of 94 shown on p.~\pageref{spec1chars})
separated by {\bf white space}\index{white space} (spaces, tabs, carriage
returns, line feeds, and form feeds). Any string consisting only of these
characters is treated the same as a single space.  The non-whitespace printable
characters\index{printable character} that Metamath recognizes are the 94
characters on standard {\sc ascii} keyboards.

Metamath has the ability to join several files together to form its
input (Section~\ref{include}).  We call the aggregate contents of all
the files after they have been joined together a {\bf
database}\index{database} to distinguish it from an individual source
file.  The tokens in a database consist of {\bf
keywords}\index{keyword}, which are built into the language, together
with two kinds of user-defined tokens called {\bf labels}\index{label}
and {\bf math symbols}\index{math symbol}.  (Often we will simply say
{\bf symbol}\index{symbol} instead of math symbol for brevity).  The set
of {\bf basic keywords}\index{basic keyword} is
\texttt{\$c}\index{\texttt{\$c} statement},
\texttt{\$v}\index{\texttt{\$v} statement},
\texttt{\$e}\index{\texttt{\$e} statement},
\texttt{\$f}\index{\texttt{\$f} statement},
\texttt{\$d}\index{\texttt{\$d} statement},
\texttt{\$a}\index{\texttt{\$a} statement},
\texttt{\$p}\index{\texttt{\$p} statement},
\texttt{\$=}\index{\texttt{\$=} keyword},
\texttt{\$.}\index{\texttt{\$.}\ keyword},
\texttt{\$\char`\{}\index{\texttt{\$\char`\{} and \texttt{\$\char`\}}
keywords}, and \texttt{\$\char`\}}.  This is the complete set of
syntactical elements of what we call the {\bf basic
language}\index{basic language} of Metamath, and with them you can
express all of the mathematics that were intended by the design of
Metamath.  You should make it a point to become very familiar with them.
Table~\ref{basickeywords} lists the basic keywords along with a brief
description of their functions.  For now, this description will give you
only a vague notion of what the keywords are for; later we will describe
the keywords in detail.


\begin{table}[htp] \caption{Summary of the basic Metamath
keywords} \label{basickeywords}
\begin{center}
\begin{tabular}{|p{4pc}|l|}
\hline
\em \centering Keyword&\em Description\\
\hline
\hline
\centering
   \texttt{\$c}&Constant symbol declaration\\
\hline
\centering
   \texttt{\$v}&Variable symbol declaration\\
\hline
\centering
   \texttt{\$d}&Disjoint variable restriction\\
\hline
\centering
   \texttt{\$f}&Variable-type (``floating'') hypothesis\\
\hline
\centering
   \texttt{\$e}&Logical (``essential'') hypothesis\\
\hline
\centering
   \texttt{\$a}&Axiomatic assertion\\
\hline
\centering
   \texttt{\$p}&Provable assertion\\
\hline
\centering
   \texttt{\$=}&Start of proof in \texttt{\$p} statement\\
\hline
\centering
   \texttt{\$.}&End of the above statement types\\
\hline
\centering
   \texttt{\$\char`\{}&Start of block\\
\hline
\centering
   \texttt{\$\char`\}}&End of block\\
\hline
\end{tabular}
\end{center}
\end{table}

%For LaTeX bug(?) where it puts tables on blank page instead of btwn text
%May have to adjust if text changes
%\newpage

There are some additional keywords, called {\bf auxiliary
keywords}\index{auxiliary keyword} that help make Metamath\index{Metamath}
more practical. These are part of the {\bf extended language}\index{extended
language}. They provide you with a means to put comments into a Metamath
source file\index{source file} and reference other source files.  We will
introduce these in later sections. Table~\ref{otherkeywords} summarizes them
so that you can recognize them now if you want to peruse some source
files while learning the basic keywords.


\begin{table}[htp] \caption{Auxiliary Metamath
keywords} \label{otherkeywords}
\begin{center}
\begin{tabular}{|p{4pc}|l|}
\hline
\em \centering Keyword&\em Description\\
\hline
\hline
\centering
   \texttt{\$(}&Start of comment\\
\hline
\centering
   \texttt{\$)}&End of comment\\
\hline
\centering
   \texttt{\$[}&Start of included source file name\\
\hline
\centering
   \texttt{\$]}&End of included source file name\\
\hline
\end{tabular}
\end{center}
\end{table}
\index{\texttt{\$(} and \texttt{\$)} auxiliary keywords}
\index{\texttt{\$[} and \texttt{\$]} auxiliary keywords}


Unlike those in some computer languages, the keywords\index{keyword} are short
two-character sequences rather than English-like words.  While this may make
them slightly more difficult to remember at first, their brevity allows
them to blend in with the mathematics being described, not
distract from it, like punctuation marks.


\subsection{User-Defined Tokens}\label{dollardollar}\index{token}

As you may have noticed, all keywords\index{keyword} begin with the \texttt{\$}
character.  This mundane monetary symbol is not ordinarily used in higher
mathematics (outside of grant proposals), so we have appropriated it to
distinguish the Metamath\index{Metamath} keywords from ordinary mathematical
symbols. The \texttt{\$} character is thus considered special and may not be
used as a character in a user-defined token.  All tokens and keywords are
case-sensitive; for example, \texttt{n} is considered to be a different character
from \texttt{N}.  Case-sensitivity makes the available {\sc ascii} character set
as rich as possible.

\subsubsection{Math Symbol Tokens}\index{token}

Math symbols\index{math symbol} are tokens used to represent the symbols
that appear in ordinary mathematical formulas.  They may consist of any
combination of the 93 non-whitespace printable {\sc ascii} characters other than
\texttt{\$}~. Some examples are \texttt{x}, \texttt{+}, \texttt{(},
\texttt{|-}, \verb$!%@?&$, and \texttt{bounded}.  For readability, it is
best to try to make these look as similar to actual mathematical symbols
as possible, within the constraints of the {\sc ascii} character set, in
order to make the resulting mathematical expressions more readable.

In the Metamath\index{Metamath} language, you express ordinary
mathematical formulas and statements as sequences of math symbols such
as \texttt{2 + 2 = 4} (five symbols, all constants).\footnote{To
eliminate ambiguity with other expressions, this is expressed in the set
theory database \texttt{set.mm} as \texttt{|- ( 2 + 2
 ) = 4 }, whose \LaTeX\ equivalent is $\vdash
(2+2)=4$.  The \,$\vdash$ means ``is a theorem'' and the
parentheses allow explicit associative grouping.}\index{turnstile
({$\,\vdash$})} They may even be English
sentences, as in \texttt{E is closed and bounded} (five symbols)---here
\texttt{E} would be a variable and the other four symbols constants.  In
principle, a Metamath database could be constructed to work with almost
any unambiguous English-language mathematical statement, but as a
practical matter the definitions needed to provide for all possible
syntax variations would be cumbersome and distracting and possibly have
subtle pitfalls accidentally built in.  We generally recommend that you
express mathematical statements with compact standard mathematical
symbols whenever possible and put their English-language descriptions in
comments.  Axioms\index{axiom} and definitions\index{definition}
(\texttt{\$a}\index{\texttt{\$a} statement} statements) are the only
places where Metamath will not detect an error, and doing this will help
reduce the number of definitions needed.

You are free to use any tokens\index{token} you like for math
symbols\index{math symbol}.  Appendix~\ref{ASCII} recommends token names to
use for symbols in set theory, and we suggest you adopt these in order to be
able to include the \texttt{set.mm} set theory database in your database.  For
printouts, you can convert the tokens in a database
to standard mathematical symbols with the \LaTeX\ typesetting program.  The
Metamath command \texttt{open tex} {\em filename}\index{\texttt{open tex} command}
produces output that can be read by \LaTeX.\index{latex@{\LaTeX}}
The correspondence
between tokens and the actual symbols is made by \texttt{latexdef}
statements inside a special database comment tagged
with \texttt{\$t}.\index{\texttt{\$t} comment}\index{typesetting comment}
  You can edit
this comment to change the definitions or add new ones.
Appendix~\ref{ASCII} describes how to do this in more detail.

% White space\index{white space} is normally used to separate math
% symbol\index{math symbol} tokens, but they may be juxtaposed without white
% space in \texttt{\$d}\index{\texttt{\$d} statement}, \texttt{\$e}\index{\texttt{\$e}
% statement}, \texttt{\$f}\index{\texttt{\$f} statement}, \texttt{\$a}\index{\texttt{\$a}
% statement}, and \texttt{\$p}\index{\texttt{\$p} statement} statements when no
% ambiguity will result.  Specifically, Metamath parses the math symbol sequence
% in one of these statements in the following manner:  when the math symbol
% sequence has been broken up into tokens\index{token} up to a given character,
% the next token is the longest string of characters that could constitute a
% math symbol that is active\index{active
% math symbol} at that point.  (See Section~\ref{scoping} for the
% definition of an active math symbol.)  For example, if \texttt{-}, \texttt{>}, and
% \texttt{->} are the only active math symbols, the juxtaposition \texttt{>-} will be
% interpreted as the two symbols \texttt{>} and \texttt{-}, whereas \texttt{->} will
% always be interpreted as that single symbol.\footnote{For better readability we
% recommend a white space between each token.  This also makes searching for a
% symbol easier to do with an editor.  Omission of optional white space is useful
% for reducing typing when assigning an expression to a temporary
% variable\index{temporary variable} with the \texttt{let variable} Metamath
% program command.}\index{\texttt{let variable} command}
%
% Keywords\index{keyword} may be placed next to math symbols without white
% space\index{white space} between them.\footnote{Again, we do not recommend
% this for readability.}
%
% The math symbols\index{math symbol} in \texttt{\$c}\index{\texttt{\$c} statement}
% and \texttt{\$v}\index{\texttt{\$v} statement} statements must always be separated
% by white space\index{white
% space}, for the obvious reason that these statements define the names
% of the symbols.
%
% Math symbols referred to in comments (see Section~\ref{comments}) must also be
% separated by white space.  This allows you to make comments about symbols that
% are not yet active\index{active
% math symbol}.  (The ``math mode'' feature of comments is also a quick and
% easy way to obtain word processing text with embedded mathematical symbols,
% independently of the main purpose of Metamath; the way to do this is described
% in Section~\ref{comments})

\subsubsection{Label Tokens}\index{token}\index{label}

Label tokens are used to identify Metamath\index{Metamath} statements for
later reference. Label tokens may contain only letters, digits, and the three
characters period, hyphen, and underscore:
\begin{verbatim}
. - _
\end{verbatim}

A label is {\bf declared}\index{label declaration} by placing it immediately
before the keyword of the statement it identifies.  For example, the label
\texttt{axiom.1} might be declared as follows:
\begin{verbatim}
axiom.1 $a |- x = x $.
\end{verbatim}

Each \texttt{\$e}\index{\texttt{\$e} statement},
\texttt{\$f}\index{\texttt{\$f} statement},
\texttt{\$a}\index{\texttt{\$a} statement}, and
\texttt{\$p}\index{\texttt{\$p} statement} statement in a database must
have a label declared for it.  No other statement types may have label
declarations.  Every label must be unique.

A label (and the statement it identifies) is {\bf referenced}\index{label
reference} by including the label between the \texttt{\$=}\index{\texttt{\$=}
keyword} and \texttt{\$.}\index{\texttt{\$.}\ keyword}\ keywords in a \texttt{\$p}
statement.  The sequence of labels\index{label sequence} between these two
keywords is called a {\bf proof}\index{proof}.  An example of a statement with
a proof that we will encounter later (Section~\ref{proof}) is
\begin{verbatim}
wnew $p wff ( s -> ( r -> p ) )
     $= ws wr wp w2 w2 $.
\end{verbatim}

You don't have to know what this means just yet, but you should know that the
label \texttt{wnew} is declared by this \texttt{\$p} statement and that the labels
\texttt{ws}, \texttt{wr}, \texttt{wp}, and \texttt{w2} are assumed to have been declared
earlier in the database and are referenced here.

\subsection{Constants and Variables}
\index{constant}
\index{variable}

An {\bf expression}\index{expression} is any sequence of math
symbols, possibly empty.

The basic Metamath\index{Metamath} language\index{basic language} has two
kinds of math symbols\index{math symbol}:  {\bf constants}\index{constant} and
{\bf variables}\index{variable}.  In a Metamath proof, a constant may not be
substituted with any expression.  A variable can be
substituted\index{substitution!variable}\index{variable substitution} with any
expression.  This sequence may include other variables and may even include
the variable being substituted.  This substitution takes place when proofs are
verified, and it will be described in Section~\ref{proof}.  The \texttt{\$f}
statement (described later in Section~\ref{dollaref}) is used to specify the
{\bf type} of a variable (i.e.\ what kind of
variable it is)\index{variable type}\index{type} and
give it a meaning typically
associated with a ``metavariable''\index{metavariable}\footnote{A metavariable
is a variable that ranges over the syntactical elements of the object language
being discussed; for example, one metavariable might represent a variable of
the object language and another metavariable might represent a formula in the
object language.} in ordinary mathematics; for example, a variable may be
specified to be a wff or well-formed formula (in logic), a set (in set
theory), or a non-negative integer (in number theory).

%\subsection{The \texttt{\$c} and \texttt{\$v} Declaration Statements}
\subsection{The \texttt{\$c} and \texttt{\$v} Declaration Statements}
\index{\texttt{\$c} statement}
\index{constant declaration}
\index{\texttt{\$v} statement}
\index{variable declaration}

Constants are introduced or {\bf declared}\index{constant declaration}
with \texttt{\$c}\index{\texttt{\$c} statement} statements, and
variables are declared\index{variable declaration} with
\texttt{\$v}\index{\texttt{\$v} statement} statements.  A {\bf simple}
declaration\index{simple declaration} statement introduces a single
constant or variable.  Its syntax is one of the following:
\begin{center}
  \texttt{\$c} {\em math-symbol} \texttt{\$.}\\
  \texttt{\$v} {\em math-symbol} \texttt{\$.}
\end{center}
The notation {\em math-symbol} means any math symbol token\index{token}.

Some examples of simple declaration statements are:
\begin{center}
  \texttt{\$c + \$.}\\
  \texttt{\$c -> \$.}\\
  \texttt{\$c ( \$.}\\
  \texttt{\$v x \$.}\\
  \texttt{\$v y2 \$.}
\end{center}

The characters in a math symbol\index{math symbol} being declared are
irrelevant to Meta\-math; for example, we could declare a right parenthesis to
be a variable,
\begin{center}
  \texttt{\$v ) \$.}\\
\end{center}
although this would be unconventional.

A {\bf compound} declaration\index{compound declaration} statement is a
shorthand for declaring several symbols at once.  Its syntax is one of the
following:
\begin{center}
  \texttt{\$c} {\em math-symbol}\ \,$\cdots$\ {\em math-symbol} \texttt{\$.}\\
  \texttt{\$v} {\em math-symbol}\ \,$\cdots$\ {\em math-symbol} \texttt{\$.}
\end{center}\index{\texttt{\$c} statement}
Here, the ellipsis (\ldots) means any number of {\em math-symbol}\,s.

An example of a compound declaration statement is:
\begin{center}
  \texttt{\$v x y mu \$.}\\
\end{center}
This is equivalent to the three simple declaration statements
\begin{center}
  \texttt{\$v x \$.}\\
  \texttt{\$v y \$.}\\
  \texttt{\$v mu \$.}\\
\end{center}
\index{\texttt{\$v} statement}

There are certain rules on where in the database math symbols may be declared,
what sections of the database are aware of them (i.e.\ where they are
``active''), and when they may be declared more than once.  These will be
discussed in Section~\ref{scoping} and specifically on
p.~\pageref{redeclaration}.

\subsection{The \texttt{\$d} Statement}\label{dollard}
\index{\texttt{\$d} statement}

The \texttt{\$d} statement is called a {\bf disjoint-variable restriction}.  The
syntax of the {\bf simple} version of this statement is
\begin{center}
  \texttt{\$d} {\em variable variable} \texttt{\$.}
\end{center}
where each {\em variable} is a previously declared variable and the two {\em
variable}\,s are different.  (More specifically, each  {\em variable} must be
an {\bf active} variable\index{active math symbol}, which means there must be
a previous \texttt{\$v} statement whose {\bf scope}\index{scope} includes the
\texttt{\$d} statement.  These terms will be defined when we discuss scoping
statements in Section~\ref{scoping}.)

In ordinary mathematics, formulas may arise that are true if the variables in
them are distinct\index{distinct variables}, but become false when those
variables are made identical. For example, the formula in logic $\exists x\,x
\neq y$, which means ``for a given $y$, there exists an $x$ that is not equal
to $y$,'' is true in most mathematical theories (namely all non-trivial
theories\index{non-trivial theory}, i.e.\ those that describe more than one
individual, such as arithmetic).  However, if we substitute $y$ with $x$, we
obtain $\exists x\,x \neq x$, which is always false, as it means ``there
exists something that is not equal to itself.''\footnote{If you are a
logician, you will recognize this as the improper substitution\index{proper
substitution}\index{substitution!proper} of a free variable\index{free
variable} with a bound variable\index{bound variable}.  Metamath makes no
inherent distinction between free and bound variables; instead, you let
Metamath know what substitutions are permissible by using \texttt{\$d} statements
in the right way in your axiom system.}\index{free vs.\ bound variable}  The
\texttt{\$d} statement allows you to specify a restriction that forbids the
substitution of one variable with another.  In
this case, we would use the statement
\begin{center}
  \texttt{\$d x y \$.}
\end{center}\index{\texttt{\$d} statement}
to specify this restriction.

The order in which the variables appear in a \texttt{\$d} statement is not
important.  We could also use
\begin{center}
  \texttt{\$d y x \$.}
\end{center}

The \texttt{\$d} statement is actually more general than this, as the
``disjoint''\index{disjoint variables} in its name suggests.  The full meaning
is that if any substitution is made to its two variables (during the
course of a proof that references a \texttt{\$a} or \texttt{\$p} statement
associated with the \texttt{\$d}), the two expressions that result from the
substitution must have no variables in common.  In addition, each possible
pair of variables, one from each expression, must be in a \texttt{\$d} statement
associated with the statement being proved.  (This requirement forces the
statement being proved to ``inherit'' the original disjoint variable
restriction.)

For example, suppose \texttt{u} is a variable.  If the restriction
\begin{center}
  \texttt{\$d A B \$.}
\end{center}
has been specified for a theorem referenced in a
proof, we may not substitute \texttt{A} with \mbox{\tt a + u} and
\texttt{B} with \mbox{\tt b + u} because these two symbol sequences have the
variable \texttt{u} in common.  Furthermore, if \texttt{a} and \texttt{b} are
variables, we may not substitute \texttt{A} with \texttt{a} and \texttt{B} with \texttt{b}
unless we have also specified \texttt{\$d a b} for the theorem being proved; in
other words, the \texttt{\$d} property associated with a pair of variables must
be effectively preserved after substitution.

The \texttt{\$d}\index{\texttt{\$d} statement} statement does {\em not} mean ``the
two variables may not be substituted with the same thing,'' as you might think
at first.  For example, substituting each of \texttt{A} and \texttt{B} in the above
example with identical symbol sequences consisting only of constants does not
cause a disjoint variable conflict, because two symbol sequences have no
variables in common (since they have no variables, period).  Similarly, a
conflict will not occur by substituting the two variables in a \texttt{\$d}
statement with the empty symbol sequence\index{empty substitution}.

The \texttt{\$d} statement does not have a direct counterpart in
ordinary mathematics, partly because the variables\index{variable} of
Metamath are not really the same as the variables\index{variable!in
ordinary mathematics} of ordinary mathematics but rather are
metavariables\index{metavariable} ranging over them (as well as over
other kinds of symbols and groups of symbols).  Depending on the
situation, we may informally interpret the \texttt{\$d} statement in
different ways.  Suppose, for example, that \texttt{x} and \texttt{y}
are variables ranging over numbers (more precisely, that \texttt{x} and
\texttt{y} are metavariables ranging over variables that range over
numbers), and that \texttt{ph} ($\varphi$) and \texttt{ps} ($\psi$) are
variables (more precisely, metavariables) ranging over formulas.  We can
make the following interpretations that correspond to the informal
language of ordinary mathematics:
\begin{quote}
\begin{tabbing}
\texttt{\$d x y \$.} means ``assume $x$ and $y$ are
distinct variables.''\\
\texttt{\$d x ph \$.} means ``assume $x$ does not
occur in $\varphi$.''\\
\texttt{\$d ph ps \$.} \=means ``assume $\varphi$ and
$\psi$ have no variables\\ \>in common.''
\end{tabbing}
\end{quote}\index{\texttt{\$d} statement}

\subsubsection{Compound \texttt{\$d} Statements}

The {\bf compound} version of the \texttt{\$d} statement is a shorthand for
specifying several variables whose substitutions must be pairwise disjoint.
Its syntax is:
\begin{center}
  \texttt{\$d} {\em variable}\ \,$\cdots$\ {\em variable} \texttt{\$.}
\end{center}\index{\texttt{\$d} statement}
Here, {\em variable} represents the token of a previously declared
variable (specifically, an active variable) and all {\em variable}\,s are
different.  The compound \texttt{\$d}
statement is internally broken up by Metamath into one simple \texttt{\$d}
statement for each possible pair of variables in the original \texttt{\$d}
statement.  For example,
\begin{center}
  \texttt{\$d w x y z \$.}
\end{center}
is equivalent to
\begin{center}
  \texttt{\$d w x \$.}\\
  \texttt{\$d w y \$.}\\
  \texttt{\$d w z \$.}\\
  \texttt{\$d x y \$.}\\
  \texttt{\$d x z \$.}\\
  \texttt{\$d y z \$.}
\end{center}

Two or more simple \texttt{\$d} statements specifying the same variable pair are
internally combined into a single \texttt{\$d} statement.  Thus the set of three
statements
\begin{center}
  \texttt{\$d x y \$.}
  \texttt{\$d x y \$.}
  \texttt{\$d y x \$.}
\end{center}
is equivalent to
\begin{center}
  \texttt{\$d x y \$.}
\end{center}

Similarly, compound \texttt{\$d} statements, after being internally broken up,
internally have their common variable pairs combined.  For example the
set of statements
\begin{center}
  \texttt{\$d x y A \$.}
  \texttt{\$d x y B \$.}
\end{center}
is equivalent to
\begin{center}
  \texttt{\$d x y \$.}
  \texttt{\$d x A \$.}
  \texttt{\$d y A \$.}
  \texttt{\$d x y \$.}
  \texttt{\$d x B \$.}
  \texttt{\$d y B \$.}
\end{center}
which is equivalent to
\begin{center}
  \texttt{\$d x y \$.}
  \texttt{\$d x A \$.}
  \texttt{\$d y A \$.}
  \texttt{\$d x B \$.}
  \texttt{\$d y B \$.}
\end{center}

Metamath\index{Metamath} automatically verifies that all \texttt{\$d}
restrictions are met whenever it verifies proofs.  \texttt{\$d} statements are
never referenced directly in proofs (this is why they do not have
labels\index{label}), but Metamath is always aware of which ones must be
satisfied (i.e.\ are active) and will notify you with an error message if any
violation occurs.

To illustrate how Metamath detects a missing \texttt{\$d}
statement, we will look at the following example from the
\texttt{set.mm} database.

\begin{verbatim}
$d x z $.  $d y z $.
$( Theorem to add distinct quantifier to atomic formula. $)
ax17eq $p |- ( x = y -> A. z x = y ) $=...
\end{verbatim}

This statement has the obvious requirement that $z$ must be
distinct\index{distinct variables} from $x$ in theorem \texttt{ax17eq} that
states $x=y \rightarrow \forall z \, x=y$ (well, obvious if you're a logician,
for otherwise we could conclude  $x=y \rightarrow \forall x \, x=y$, which is
false when the free variables $x$ and $y$ are equal).

Let's look at what happens if we edit the database to comment out this
requirement.

\begin{verbatim}
$( $d x z $. $) $d y z $.
$( Theorem to add distinct quantifier to atomic formula. $)
ax17eq $p |- ( x = y -> A. z x = y ) $=...
\end{verbatim}

When it tries to verify the proof, Metamath will tell you that \texttt{x} and
\texttt{z} must be disjoint, because one of its steps references an axiom or
theorem that has this requirement.

\begin{verbatim}
MM> verify proof ax17eq
ax17eq ?Error at statement 1918, label "ax17eq", type "$p":
      vz wal wi vx vy vz ax-13 vx vy weq vz vx ax-c16 vx vy
                                               ^^^^^
There is a disjoint variable ($d) violation at proof step 29.
Assertion "ax-c16" requires that variables "x" and "y" be
disjoint.  But "x" was substituted with "z" and "y" was
substituted with "x".  The assertion being proved, "ax17eq",
does not require that variables "z" and "x" be disjoint.
\end{verbatim}

We can see the substitutions into \texttt{ax-c16} with the following command.

\begin{verbatim}
MM> show proof ax17eq / detailed_step 29
Proof step 29:  pm2.61dd.2=ax-c16 $a |- ( A. z z = x -> ( x =
  y -> A. z x = y ) )
This step assigns source "ax-c16" ($a) to target "pm2.61dd.2"
($e).  The source assertion requires the hypotheses "wph"
($f, step 26), "vx" ($f, step 27), and "vy" ($f, step 28).
The parent assertion of the target hypothesis is "pm2.61dd"
($p, step 36).
The source assertion before substitution was:
    ax-c16 $a |- ( A. x x = y -> ( ph -> A. x ph ) )
The following substitutions were made to the source
assertion:
    Variable  Substituted with
     x         z
     y         x
     ph        x = y
The target hypothesis before substitution was:
    pm2.61dd.2 $e |- ( ph -> ch )
The following substitutions were made to the target
hypothesis:
    Variable  Substituted with
     ph        A. z z = x
     ch        ( x = y -> A. z x = y )
\end{verbatim}

The disjoint variable restrictions of \texttt{ax-c16} can be seen from the
\texttt{show state\-ment} command.  The line that begins ``\texttt{Its mandatory
dis\-joint var\-i\-able pairs are:}\ldots'' lists any \texttt{\$d} variable
pairs in brackets.

\begin{verbatim}
MM> show statement ax-c16/full
Statement 3033 is located on line 9338 of the file "set.mm".
"Axiom of Distinct Variables. ..."
  ax-c16 $a |- ( A. x x = y -> ( ph -> A. x ph ) ) $.
Its mandatory hypotheses in RPN order are:
  wph $f wff ph $.
  vx $f setvar x $.
  vy $f setvar y $.
Its mandatory disjoint variable pairs are:  <x,y>
The statement and its hypotheses require the variables:  x y
      ph
The variables it contains are:  x y ph
\end{verbatim}

Since Metamath will always detect when \texttt{\$d}\index{\texttt{\$d} statement}
statements are needed for a proof, you don't have to worry too much about
forgetting to put one in; it can always be added if you see the error message
above.  If you put in unnecessary \texttt{\$d} statements, the worst that could
happen is that your theorem might not be as general as it could be, and this
may limit its use later on.

On the other hand, when you introduce axioms (\texttt{\$a}\index{\texttt{\$a}
statement} statements), you must be very careful to properly specify the
necessary associated \texttt{\$d} statements since Metamath has no way of knowing
whether your axioms are correct.  For example, Metamath would have no idea
that \texttt{ax-c16}, which we are telling it is an axiom of logic, would lead to
contradictions if we omitted its associated \texttt{\$d} statement.

% This was previously a comment in footnote-sized type, but it can be
% hard to read this much text in a small size.
% As a result, it's been changed to normally-sized text.
\label{nodd}
You may wonder if it is possible to develop standard
mathematics in the Metamath language without the \texttt{\$d}\index{\texttt{\$d}
statement} statement, since it seems like a nuisance that complicates proof
verification. The \texttt{\$d} statement is not needed in certain subsets of
mathematics such as propositional calculus.  However, dummy
variables\index{dummy variable!eliminating} and their associated \texttt{\$d}
statements are impossible to avoid in proofs in standard first-order logic as
well as in the variant used in \texttt{set.mm}.  In fact, there is no upper bound to
the number of dummy variables that might be needed in a proof of a theorem of
first-order logic containing 3 or more variables, as shown by H.\
Andr\'{e}ka\index{Andr{\'{e}}ka, H.} \cite{Nemeti}.  A first-order system that
avoids them entirely is given in \cite{Megill}\index{Megill, Norman}; the
trick there is simply to embed harmlessly the necessary dummy variables into a
theorem being proved so that they aren't ``dummy'' anymore, then interpret the
resulting longer theorem so as to ignore the embedded dummy variables.  If
this interests you, the system in \texttt{set.mm} obtained from \texttt{ax-1}
through \texttt{ax-c14} in \texttt{set.mm}, and deleting \texttt{ax-c16} and \texttt{ax-5},
requires no \texttt{\$d} statements but is logically complete in the sense
described in \cite{Megill}.  This means it can prove any theorem of
first-order logic as long as we add to the theorem an antecedent that embeds
dummy and any other variables that must be distinct.  In a similar fashion,
axioms for set theory can be devised that
do not require distinct variable
provisos\index{Set theory without distinct variable provisos},
as explained at
\url{http://us.metamath.org/mpeuni/mmzfcnd.html}.
Together, these in principle allow all of
mathematics to be developed under Metamath without a \texttt{\$d} statement,
although the length of the resulting theorems will grow as more and
more dummy variables become required in their proofs.

\subsection{The \texttt{\$f}
and \texttt{\$e} Statements}\label{dollaref}
\index{\texttt{\$e} statement}
\index{\texttt{\$f} statement}
\index{floating hypothesis}
\index{essential hypothesis}
\index{variable-type hypothesis}
\index{logical hypothesis}
\index{hypothesis}

Metamath has two kinds of hypo\-theses, the \texttt{\$f}\index{\texttt{\$f}
statement} or {\bf variable-type} hypothesis and the \texttt{\$e} or {\bf logical}
hypo\-the\-sis.\index{\texttt{\$d} statement}\footnote{Strictly speaking, the
\texttt{\$d} statement is also a hypothesis, but it is never directly referenced
in a proof, so we call it a restriction rather than a hypothesis to lessen
confusion.  The checking for violations of \texttt{\$d} restrictions is automatic
and built into Metamath's proof-checking algorithm.} The letters \texttt{f} and
\texttt{e} stand for ``floating''\index{floating hypothesis} (roughly meaning
used only if relevant) and ``essential''\index{essential hypothesis} (meaning
always used) respectively, for reasons that will become apparent
when we discuss frames in
Section~\ref{frames} and scoping in Section~\ref{scoping}. The syntax of these
are as follows:
\begin{center}
  {\em label} \texttt{\$f} {\em typecode} {\em variable} \texttt{\$.}\\
  {\em label} \texttt{\$e} {\em typecode}
      {\em math-symbol}\ \,$\cdots$\ {\em math-symbol} \texttt{\$.}\\
\end{center}
\index{\texttt{\$e} statement}
\index{\texttt{\$f} statement}
A hypothesis must have a {\em label}\index{label}.  The expression in a
\texttt{\$e} hypothesis consists of a typecode (an active constant math symbol)
followed by a sequence
of zero or more math symbols. Each math symbol (including {\em constant}
and {\em variable}) must be a previously declared constant or variable.  (In
addition, each math symbol must be active, which will be covered when we
discuss scoping statements in Section~\ref{scoping}.)  You use a \texttt{\$f}
hypothesis to specify the
nature or {\bf type}\index{variable type}\index{type} of a variable (such as ``let $x$ be an
integer'') and use a \texttt{\$e} hypothesis to express a logical truth (such as
``assume $x$ is prime'') that must be established in order for an assertion
requiring it to also be true.

A variable must have its type specified in a \texttt{\$f} statement before
it may be used in a \texttt{\$e}, \texttt{\$a}, or \texttt{\$p}
statement.  There may be only one (active) \texttt{\$f} statement for a
given variable.  (``Active'' is defined in Section~\ref{scoping}.)

In ordinary mathematics, theorems\index{theorem} are often expressed in the
form ``Assume $P$; then $Q$,'' where $Q$ is a statement that you can derive
if you start with statement $P$.\index{free variable}\footnote{A stronger
version of a theorem like this would be the {\em single} formula $P\rightarrow
Q$ ($P$ implies $Q$) from which the weaker version above follows by the rule
of modus ponens in logic.  We are not discussing this stronger form here.  In
the weaker form, we are saying only that if we can {\em prove} $P$, then we can
{\em prove} $Q$.  In a logician's language, if $x$ is the only free variable
in $P$ and $Q$, the stronger form is equivalent to $\forall x ( P \rightarrow
Q)$ (for all $x$, $P$ implies $Q$), whereas the weaker form is equivalent to
$\forall x P \rightarrow \forall x Q$. The stronger form implies the weaker,
but not vice-versa.  To be precise, the weaker form of the theorem is more
properly called an ``inference'' rather than a theorem.}\index{inference}
In the
Metamath\index{Metamath} language, you would express mathematical statement
$P$ as a hypothesis (a \texttt{\$e} Metamath language statement in this case) and
statement $Q$ as a provable assertion (a \texttt{\$p}\index{\texttt{\$p} statement}
statement).

Some examples of hypotheses you might encounter in logic and set theory are
\begin{center}
  \texttt{stmt1 \$f wff P \$.}\\
  \texttt{stmt2 \$f setvar x \$.}\\
  \texttt{stmt3 \$e |- ( P -> Q ) \$.}
\end{center}
\index{\texttt{\$e} statement}
\index{\texttt{\$f} statement}
Informally, these would be read, ``Let $P$ be a well-formed-formula,'' ``Let
$x$ be an (individual) variable,'' and ``Assume we have proved $P \rightarrow
Q$.''  The turnstile symbol \,$\vdash$\index{turnstile ({$\,\vdash$})} is
commonly used in logic texts to mean ``a proof exists for.''

To summarize:
\begin{itemize}
\item A \texttt{\$f} hypothesis tells Metamath the type or kind of its variable.
It is analogous to a variable declaration in a computer language that
tells the compiler that a variable is an integer or a floating-point
number.
\item The \texttt{\$e} hypothesis corresponds to what you would usually call a
``hypothesis'' in ordinary mathematics.
\end{itemize}

Before an assertion\index{assertion} (\texttt{\$a} or \texttt{\$p} statement) can be
referenced in a proof, all of its associated \texttt{\$f} and \texttt{\$e} hypotheses
(i.e.\ those \texttt{\$e} hypotheses that are active) must be satisfied (i.e.
established by the proof).  The meaning of ``associated'' (which we will call
{\bf mandatory} in Section~\ref{frames}) will become clear when we discuss
scoping later.

Note that after any \texttt{\$f}, \texttt{\$e},
\texttt{\$a}, or \texttt{\$p} token there is a required
\textit{typecode}\index{typecode}.
The typecode is a constant used to enforce types of expressions.
This will become clearer once we learn more about
assertions (\texttt{\$a} and \texttt{\$p} statements).
An example may also clarify their purpose.
In the
\texttt{set.mm}\index{set theory database (\texttt{set.mm})}%
\index{Metamath Proof Explorer}
database,
the following typecodes are used:

\begin{itemize}
\item \texttt{wff} :
  Well-formed formula (wff) symbol
  (read: ``the following symbol sequence is a wff'').
% The *textual* typecode for turnstile is "|-", but when read it's a little
% confusing, so I intentionally display the mathematical symbol here instead
% (I think it's clearer in this context).
\item \texttt{$\vdash$} :
  Turnstile (read: ``the following symbol sequence is provable'' or
  ``a proof exists for'').
\item \texttt{setvar} :
  Individual set variable type (read: ``the following is an
  individual set variable'').
  Note that this is \textit{not} the type of an arbitrary set expression,
  instead, it is used to ensure that there is only a single symbol used
  after quantifiers like for-all ($\forall$) and there-exists ($\exists$).
\item \texttt{class} :
  An expression that is a syntactically valid class expression.
  All valid set expressions are also valid class expression, so expressions
  of sets normally have the \texttt{class} typecode.
  Use the \texttt{class} typecode,
  \textit{not} the \texttt{setvar} typecode,
  for the type of set expressions unless you are specifically identifying
  a single set variable.
\end{itemize}

\subsection{Assertions (\texttt{\$a} and \texttt{\$p} Statements)}
\index{\texttt{\$a} statement}
\index{\texttt{\$p} statement}\index{assertion}\index{axiomatic assertion}
\index{provable assertion}

There are two types of assertions, \texttt{\$a}\index{\texttt{\$a} statement}
statements ({\bf axiomatic assertions}) and \texttt{\$p} statements ({\bf
provable assertions}).  Their syntax is as follows:
\begin{center}
  {\em label} \texttt{\$a} {\em typecode} {\em math-symbol} \ldots
         {\em math-symbol} \texttt{\$.}\\
  {\em label} \texttt{\$p} {\em typecode} {\em math-symbol} \ldots
        {\em math-symbol} \texttt{\$=} {\em proof} \texttt{\$.}
\end{center}
\index{\texttt{\$a} statement}
\index{\texttt{\$p} statement}
\index{\texttt{\$=} keyword}
An assertion always requires a {\em label}\index{label}. The expression in an
assertion consists of a typecode (an active constant)
followed by a sequence of zero
or more math symbols.  Each math symbol, including any {\em constant}, must be a
previously declared constant or variable.  (In addition, each math symbol
must be active, which will be covered when we discuss scoping statements in
Section~\ref{scoping}.)

A \texttt{\$a} statement is usually a definition of syntax (for example, if $P$
and $Q$ are wffs then so is $(P\to Q)$), an axiom\index{axiom} of ordinary
mathematics (for example, $x=x$), or a definition\index{definition} of
ordinary mathematics (for example, $x\ne y$ means $\lnot x=y$). A \texttt{\$p}
statement is a claim that a certain combination of math symbols follows from
previous assertions and is accompanied by a proof that demonstrates it.

Assertions can also be referenced in (later) proofs in order to derive new
assertions from them. The label of an assertion is used to refer to it in a
proof. Section~\ref{proof} will describe the proof in detail.

Assertions also provide the primary means for communicating the mathematical
results in the database to people.  Proofs (when conveniently displayed)
communicate to people how the results were arrived at.

\subsubsection{The \texttt{\$a} Statement}
\index{\texttt{\$a} statement}

Axiomatic assertions (\texttt{\$a} statements) represent the starting points from
which other assertions (\texttt{\$p}\index{\texttt{\$p} statement} statements) are
derived.  Their most obvious use is for specifying ordinary mathematical
axioms\index{axiom}, but they are also used for two other purposes.

First, Metamath\index{Metamath} needs to know the syntax of symbol
sequences that constitute valid mathematical statements.  A Metamath
proof must be broken down into much more detail than ordinary
mathematical proofs that you may be used to thinking of (even the
``complete'' proofs of formal logic\index{formal logic}).  This is one
of the things that makes Metamath a general-purpose language,
independent of any system of logic or even syntax.  If you want to use a
substitution instance of an assertion as a step in a proof, you must
first prove that the substitution is syntactically correct (or if you
prefer, you must ``construct'' it), showing for example that the
expression you are substituting for a wff metavariable is a valid wff.
The \texttt{\$a}\index{\texttt{\$a} statement} statement is used to
specify those combinations of symbols that are considered syntactically
valid, such as the legal forms of wffs.

Second, \texttt{\$a} statements are used to specify what are ordinarily thought of
as definitions, i.e.\ new combinations of symbols that abbreviate other
combinations of symbols.  Metamath makes no distinction\index{axiom vs.\
definition} between axioms\index{axiom} and definitions\index{definition}.
Indeed, it has been argued that such distinction should not be made even in
ordinary mathematics; see Section~\ref{definitions}, which discusses the
philosophy of definitions.  Section~\ref{hierarchy} discusses some
technical requirements for definitions.  In \texttt{set.mm} we adopt the
convention of prefixing axiom labels with \texttt{ax-} and definition labels with
\texttt{df-}\index{label}.

The results that can be derived with the Metamath language are only as good as
the \texttt{\$a}\index{\texttt{\$a} statement} statements used as their starting
point.  We cannot stress this too strongly.  For example, Metamath will
not prevent you from specifying $x\neq x$ as an axiom of logic.  It is
essential that you scrutinize all \texttt{\$a} statements with great care.
Because they are a source of potential pitfalls, it is best not to add new
ones (usually new definitions) casually; rather you should carefully evaluate
each one's necessity and advantages.

Once you have in place all of the basic axioms\index{axiom} and
rules\index{rule} of a mathematical theory, the only \texttt{\$a} statements that
you will be adding will be what are ordinarily called definitions.  In
principle, definitions should be in some sense eliminable from the language of
a theory according to some convention (usually involving logical equivalence
or equality).  The most common convention is that any formula that was
syntactically valid but not provable before the definition was introduced will
not become provable after the definition is introduced.  In an ideal world,
definitions should not be present at all if one is to have absolute confidence
in a mathematical result.  However, they are necessary to make
mathematics practical, for otherwise the resulting formulas would be
extremely long and incomprehensible.  Since the nature of definitions (in the
most general sense) does not permit them to automatically be verified as
``proper,''\index{proper definition}\index{definition!proper} the judgment of
the mathematician is required to ensure it.  (In \texttt{set.mm} effort was made
to make almost all definitions directly eliminable and thus minimize the need
for such judgment.)

If you are not a mathematician, it may be best not to add or change any
\texttt{\$a}\index{\texttt{\$a} statement} statements but instead use
the mathematical language already provided in standard databases.  This
way Metamath will not allow you to make a mistake (i.e.\ prove a false
result).


\subsection{Frames}\label{frames}

We now introduce the concept of a collection of related Metamath statements
called a frame.  Every assertion (\texttt{\$a} or \texttt{\$p} statement) in the database has
an associated frame.

A {\bf frame}\index{frame} is a sequence of \texttt{\$d}, \texttt{\$f},
and \texttt{\$e} statements (zero or more of each) followed by one
\texttt{\$a} or \texttt{\$p} statement, subject to certain conditions we
will describe.  For simplicity we will assume that all math symbol
tokens used are declared at the beginning of the database with
\texttt{\$c} and \texttt{\$v} statements (which are not properly part of
a frame).  Also for simplicity we will assume there are only simple
\texttt{\$d} statements (those with only two variables) and imagine any
compound \texttt{\$d} statements (those with more than two variables) as
broken up into simple ones.

A frame groups together those hypotheses (and \texttt{\$d} statements) relevant
to an assertion (\texttt{\$a} or \texttt{\$p} statement).  The statements in a frame
may or may not be physically adjacent in a database; we will cover
this in our discussion of scoping statements
in Section~\ref{scoping}.

A frame has the following properties:
\begin{enumerate}
 \item The set of variables contained in its \texttt{\$f} statements must
be identical to the set of variables contained in its \texttt{\$e},
\texttt{\$a}, and/or \texttt{\$p} statements.  In other words, each
variable in a \texttt{\$e}, \texttt{\$a}, or \texttt{\$p} statement must
have an associated ``variable type'' defined for it in a \texttt{\$f}
statement.
  \item No two \texttt{\$f} statements may contain the same variable.
  \item Any \texttt{\$f} statement
must occur before a \texttt{\$e} statement in which its variable occurs.
\end{enumerate}

The first property determines the set of variables occurring in a frame.
These are the {\bf mandatory
variables}\index{mandatory variable} of the frame.  The second property
tells us there must be only one type specified for a variable.
The last property is not a theoretical requirement but it
makes parsing of the database easier.

For our examples, we assume our database has the following declarations:

\begin{verbatim}
$v P Q R $.
$c -> ( ) |- wff $.
\end{verbatim}

The following sequence of statements, describing the modus ponens inference
rule, is an example of a frame:

\begin{verbatim}
wp  $f wff P $.
wq  $f wff Q $.
maj $e |- ( P -> Q ) $.
min $e |- P $.
mp  $a |- Q $.
\end{verbatim}

The following sequence of statements is not a frame because \texttt{R} does not
occur in the \texttt{\$e}'s or the \texttt{\$a}:

\begin{verbatim}
wp  $f wff P $.
wq  $f wff Q $.
wr  $f wff R $.
maj $e |- ( P -> Q ) $.
min $e |- P $.
mp  $a |- Q $.
\end{verbatim}

The following sequence of statements is not a frame because \texttt{Q} does not
occur in a \texttt{\$f}:

\begin{verbatim}
wp  $f wff P $.
maj $e |- ( P -> Q ) $.
min $e |- P $.
mp  $a |- Q $.
\end{verbatim}

The following sequence of statements is not a frame because the \texttt{\$a} statement is
not the last one:

\begin{verbatim}
wp  $f wff P $.
wq  $f wff Q $.
maj $e |- ( P -> Q ) $.
mp  $a |- Q $.
min $e |- P $.
\end{verbatim}

Associated with a frame is a sequence of {\bf mandatory
hypotheses}\index{mandatory hypothesis}.  This is simply the set of all
\texttt{\$f} and \texttt{\$e} statements in the frame, in the order they
appear.  A frame can be referenced in a later proof using the label of
the \texttt{\$a} or \texttt{\$p} assertion statement, and the proof
makes an assignment to each mandatory hypothesis in the order in which
it appears.  This means the order of the hypotheses, once chosen, must
not be changed so as not to affect later proofs referencing the frame's
assertion statement.  (The Metamath proof verifier will, of course, flag
an error if a proof becomes incorrect by doing this.)  Since proofs make
use of ``Reverse Polish notation,'' described in Section~\ref{proof}, we
call this order the {\bf RPN order}\index{RPN order} of the hypotheses.

Note that \texttt{\$d} statements are not part of the set of mandatory
hypotheses, and their order doesn't matter (as long as they satisfy the
fourth property for a frame described above).  The \texttt{\$d}
statements specify restrictions on variables that must be satisfied (and
are checked by the proof verifier) when expressions are substituted for
them in a proof, and the \texttt{\$d} statements themselves are never
referenced directly in a proof.

A frame with a \texttt{\$p} (provable) statement requires a proof as part of the
\texttt{\$p} statement.  Sometimes in a proof we want to make use of temporary or
dummy variables\index{dummy variable} that do not occur in the \texttt{\$p}
statement or its mandatory hypotheses.  To accommodate this we define an {\bf
extended frame}\index{extended frame} as a frame together with zero or more
\texttt{\$d} and \texttt{\$f} statements that reference variables not among the
mandatory variables of the frame.  Any new variables referenced are called the
{\bf optional variables}\index{optional variable} of the extended frame. If a
\texttt{\$f} statement references an optional variable it is called an {\bf
optional hypothesis}\index{optional hypothesis}, and if one or both of the
variables in a \texttt{\$d} statement are optional variables it is called an {\bf
optional disjoint-variable restriction}\index{optional disjoint-variable
restriction}.  Properties 2 and 3 for a frame also apply to an extended
frame.

The concept of optional variables is not meaningful for frames with \texttt{\$a}
statements, since those statements have no proofs that might make use of them.
There is no restriction on including optional hypotheses in the extended frame
for a \texttt{\$a} statement, but they serve no purpose.

The following set of statements is an example of an extended frame, which
contains an optional variable \texttt{R} and an optional hypothesis \texttt{wr}.  In
this example, we suppose the rule of modus ponens is not an axiom but is
derived as a theorem from earlier statements (we omit its presumed proof).
Variable \texttt{R} may be used in its proof if desired (although this would
probably have no advantage in propositional calculus).  Note that the sequence
of mandatory hypotheses in RPN order is still \texttt{wp}, \texttt{wq}, \texttt{maj},
\texttt{min} (i.e.\ \texttt{wr} is omitted), and this sequence is still assumed
whenever the assertion \texttt{mp} is referenced in a subsequent proof.

\begin{verbatim}
wp  $f wff P $.
wq  $f wff Q $.
wr  $f wff R $.
maj $e |- ( P -> Q ) $.
min $e |- P $.
mp  $p |- Q $= ... $.
\end{verbatim}

Every frame is an extended frame, but not every extended frame is a frame, as
this example shows.  The underlying frame for an extended frame is
obtained by simply removing all statements containing optional variables.
Any proof referencing an assertion will ignore any extensions to its
frame, which means we may add or delete optional hypotheses at will without
affecting subsequent proofs.

The conceptually simplest way of organizing a Metamath database is as a
sequence of extended frames.  The scoping statements
\texttt{\$\char`\{}\index{\texttt{\$\char`\{} and \texttt{\$\char`\}}
keywords} and \texttt{\$\char`\}} can be used to delimit the start and
end of an extended frame, leading to the following possible structure for a
database.  \label{framelist}

\vskip 2ex
\setbox\startprefix=\hbox{\tt \ \ \ \ \ \ \ \ }
\setbox\contprefix=\hbox{}
\startm
\m{\mbox{(\texttt{\$v} {\em and} \texttt{\$c}\,{\em statements})}}
\endm
\startm
\m{\mbox{\texttt{\$\char`\{}}}
\endm
\startm
\m{\mbox{\texttt{\ \ } {\em extended frame}}}
\endm
\startm
\m{\mbox{\texttt{\$\char`\}}}}
\endm
\startm
\m{\mbox{\texttt{\$\char`\{}}}
\endm
\startm
\m{\mbox{\texttt{\ \ } {\em extended frame}}}
\endm
\startm
\m{\mbox{\texttt{\$\char`\}}}}
\endm
\startm
\m{\mbox{\texttt{\ \ \ \ \ \ \ \ \ }}\vdots}
\endm
\vskip 2ex

In practice, this structure is inconvenient because we have to repeat
any \texttt{\$f}, \texttt{\$e}, and \texttt{\$d} statements over and
over again rather than stating them once for use by several assertions.
The scoping statements, which we will discuss next, allow this to be
done.  In principle, any Metamath database can be converted to the above
format, and the above format is the most convenient to use when studying
a Metamath database as a formal system%
%% Uncomment this when uncommenting section {formalspec} below
   (Appendix \ref{formalspec})%
.
In fact, Metamath internally converts the database to the above format.
The command \texttt{show statement} in the Metamath program will show
you the contents of the frame for any \texttt{\$a} or \texttt{\$p}
statement, as well as its extension in the case of a \texttt{\$p}
statement.

%c%(provided that all ``local'' variables and constants with limited scope have
%c%unique names),

During our discussion of scoping statements, it may be helpful to
think in terms of the equivalent sequence of frames that will result when
the database is parsed.  Scoping (other than the limited
use above to delimit frames) is not a theoretical requirement for
Metamath but makes it more convenient.


\subsection{Scoping Statements (\texttt{\$\{} and \texttt{\$\}})}\label{scoping}
\index{\texttt{\$\char`\{} and \texttt{\$\char`\}} keywords}\index{scoping statement}

%c%Some Metamath statements may be needed only temporarily to
%c%serve a specific purpose, and after we're done with them we would like to
%c%disregard or ignore them.  For example, when we're finished using a variable,
%c%we might want to
%c%we might want to free up the token\index{token} used to name it so that the
%c%token can be used for other purposes later on, such as a different kind of
%c%variable or even a constant.  In the terminology of computer programming, we
%c%might want to let some symbol declarations be ``local'' rather than ``global.''
%c%\index{local symbol}\index{global symbol}

The {\bf scoping} statements, \texttt{\$\char`\{} ({\bf start of block}) and \texttt{\$\char`\}}
({\bf end of block})\index{block}, provide a means for controlling the portion
of a database over which certain statement types are recognized.  The
syntax of a scoping statement is very simple; it just consists of the
statement's keyword:
\begin{center}
\texttt{\$\char`\{}\\
\texttt{\$\char`\}}
\end{center}
\index{\texttt{\$\char`\{} and \texttt{\$\char`\}} keywords}

For example, consider the following database where we have stripped out
all tokens except the scoping statement keywords.  For the purpose of the
discussion, we have added subscripts to the scoping statements; these subscripts
do not appear in the actual database.
\[
 \mbox{\tt \ \$\char`\{}_1
 \mbox{\tt \ \$\char`\{}_2
 \mbox{\tt \ \$\char`\}}_2
 \mbox{\tt \ \$\char`\{}_3
 \mbox{\tt \ \$\char`\{}_4
 \mbox{\tt \ \$\char`\}}_4
 \mbox{\tt \ \$\char`\}}_3
 \mbox{\tt \ \$\char`\}}_1
\]
Each \texttt{\$\char`\{} statement in this example is said to be {\bf
matched} with the \texttt{\$\char`\}} statement that has the same
subscript.  Each pair of matched scoping statements defines a region of
the database called a {\bf block}.\index{block} Blocks can be {\bf
nested}\index{nested block} inside other blocks; in the example, the
block defined by $\mbox{\tt \$\char`\{}_4$ and $\mbox{\tt \$\char`\}}_4$
is nested inside the block defined by $\mbox{\tt \$\char`\{}_3$ and
$\mbox{\tt \$\char`\}}_3$ as well as inside the block defined by
$\mbox{\tt \$\char`\{}_1$ and $\mbox{\tt \$\char`\}}_1$.  In general, a
block may be empty, it may contain only non-scoping
statements,\footnote{Those statements other than \texttt{\$\char`\{} and
\texttt{\$\char`\}}.}\index{non-scoping statement} or it may contain any
mixture of other blocks and non-scoping statements.  (This is called a
``recursive'' definition\index{recursive definition} of a block.)

Associated with each block is a number called its {\bf nesting
level}\index{nesting level} that indicates how deeply the block is nested.
The nesting levels of the blocks in our example are as follows:
\[
  \underbrace{
    \mbox{\tt \ }
    \underbrace{
     \mbox{\tt \$\char`\{\ }
     \underbrace{
       \mbox{\tt \$\char`\{\ }
       \mbox{\tt \$\char`\}}
     }_{2}
     \mbox{\tt \ }
     \underbrace{
       \mbox{\tt \$\char`\{\ }
       \underbrace{
         \mbox{\tt \$\char`\{\ }
         \mbox{\tt \$\char`\}}
       }_{3}
       \mbox{\tt \ \$\char`\}}
     }_{2}
     \mbox{\tt \ \$\char`\}}
   }_{1}
   \mbox{\tt \ }
 }_{0}
\]
\index{\texttt{\$\char`\{} and \texttt{\$\char`\}} keywords}
The entire database is considered to be one big block (the {\bf outermost}
block) with a nesting level of 0.  The outermost block is {\em not} bracketed
by scoping statements.\footnote{The language was designed this way so that
several source files can be joined together more easily.}\index{outermost
block}

All non-scoping Metamath statements become recognized or {\bf
active}\index{active statement} at the place where they appear.\footnote{To
keep things slightly simpler, we do not bother to define the concept of
``active'' for the scoping statements.}  Certain of these statement types
become inactive at the end of the block in which they appear; these statement
types are:
\begin{center}
  \texttt{\$c}, \texttt{\$v}, \texttt{\$d}, \texttt{\$e}, and \texttt{\$f}.
%  \texttt{\$v}, \texttt{\$f}, \texttt{\$e}, and \texttt{\$d}.
\end{center}
\index{\texttt{\$c} statement}
\index{\texttt{\$d} statement}
\index{\texttt{\$e} statement}
\index{\texttt{\$f} statement}
\index{\texttt{\$v} statement}
The other statement types remain active forever (i.e.\ through the end of the
database); they are:
\begin{center}
  \texttt{\$a} and \texttt{\$p}.
%  \texttt{\$c}, \texttt{\$a}, and \texttt{\$p}.
\end{center}
\index{\texttt{\$a} statement}
\index{\texttt{\$p} statement}
Any statement (of these 7 types) located in the outermost
block\index{outermost block} will remain active through the end of the
database and thus are effectively ``global'' statements.\index{global
statement}

All \texttt{\$c} statements must be placed in the outermost block.  Since they are
therefore always global, they could be considered as belonging to both of the
above categories.

The {\bf scope}\index{scope} of a statement is the set of statements that
recognize it as active.

%c%The concept of ``active'' is also defined for math symbols\index{math
%c%symbol}.  Math symbols (constants\index{constant} and
%c%variables\index{variable}) become {\bf active}\index{active
%c%math symbol} in the \texttt{\$c}\index{\texttt{\$c}
%c%statement} and \texttt{\$v}\index{\texttt{\$v} statement} statements that
%c%declare them.  They become inactive when their declaration statements become
%c%inactive.

The concept of ``active'' is also defined for math symbols\index{math
symbol}.  Math symbols (constants\index{constant} and
variables\index{variable}) become {\bf active}\index{active math symbol}
in the \texttt{\$c}\index{\texttt{\$c} statement} and
\texttt{\$v}\index{\texttt{\$v} statement} statements that declare them.
A variable becomes inactive when its declaration statement becomes
inactive.  Because all \texttt{\$c} statements must be in the outermost
block, a constant will never become inactive after it is declared.

\subsubsection{Redeclaration of Math Symbols}
\index{redeclaration of symbols}\label{redeclaration}

%c%A math symbol may not be declared a second time while it is active, but it may
%c%be declared again after it becomes inactive.

A variable may not be declared a second time while it is active, but it may be
declared again after it becomes inactive.  This provides a convenient way to
introduce ``local'' variables,\index{local variable} i.e.\ temporary variables
for use in the frame of an assertion or in a proof without keeping them around
forever.  A previously declared variable may not be redeclared as a constant.

A constant may not be redeclared.  And, as mentioned above, constants must be
declared in the outermost block.

The reason variables may have limited scope but not constants is that an
assertion (\texttt{\$a} or \texttt{\$p} statement) remains available for use in
proofs through the end of the database.  Variables in an assertion's frame may
be substituted with whatever is needed in a proof step that references the
assertion, whereas constants remain fixed and may not be substituted with
anything.  The particular token used for a variable in an assertion's frame is
irrelevant when the assertion is referenced in a proof, and it doesn't matter
if that token is not available outside of the referenced assertion's frame.
Constants, however, must be globally fixed.

There is no theoretical
benefit for the feature allowing variables to be active for limited scopes
rather than global. It is just a convenience that allows them, for example, to
be locally grouped together with their corresponding \texttt{\$f} variable-type
declarations.

%c%If you declare a math symbol more than once, internally Metamath considers it a
%c%new distinct symbol, even though it has the same name.  If you are unaware of
%c%this, you may find that what you think are correct proofs are incorrectly
%c%rejected as invalid, because Metamath may tell you that a constant you
%c%previously declared does not match a newly declared math symbol with the same
%c%name.  For details on this subtle point, see the Comment on
%c%p.~\pageref{spec4comment}.  This is done purposely to allow temporary
%c%constants to be introduced while developing a subtheory, then allow their math
%c%symbol tokens to be reused later on; in general they will not refer to the
%c%same thing.  In practice, you would not ordinarily reuse the names of
%c%constants because it would tend to be confusing to the reader.  The reuse of
%c%names of variables, on the other hand, is something that is often useful to do
%c%(for example it is done frequently in \texttt{set.mm}).  Since variables in an
%c%assertion referenced in a proof can be substituted as needed to achieve a
%c%symbol match, this is not an issue.

% (This section covers a somewhat advanced topic you may want to skip
% at first reading.)
%
% Under certain circumstances, math symbol\index{math symbol}
% tokens\index{token} may be redeclared (i.e.\ the token
% may appear in more than
% one \texttt{\$c}\index{\texttt{\$c} statement} or \texttt{\$v}\index{\texttt{\$v}
% statement} statement).  You might want to do this say, to make temporary use
% of a variable name without having to worry about its affect elsewhere,
% somewhat analogous to declaring a local variable in a standard computer
% language.  Understanding what goes on when math symbol tokens are redeclared
% is a little tricky to understand at first, since it requires that we
% distinguish the token itself from the math symbol that it names.  It will help
% if we first take a peek at the internal workings of the
% Metamath\index{Metamath} program.
%
% Metamath reserves a memory location for each occurrence of a
% token\index{token} in a declaration statement (\texttt{\$c}\index{\texttt{\$c}
% statement} or \texttt{\$v}\index{\texttt{\$v} statement}).  If a given token appears
% in more than one declaration statement, it will refer to more than one memory
% locations.  A math symbol\index{math symbol} may be thought of as being one of
% these memory locations rather than as the token itself.  Only one of the
% memory locations associated with a given token may be active at any one time.
% The math symbol (memory location) that gets looked up when the token appears
% in a non-declaration statement is the one that happens to be active at that
% time.
%
% We now look at the rules for the redeclaration\index{redeclaration of symbols}
% of math symbol tokens.
% \begin{itemize}
% \item A math symbol token may not be declared twice in the
% same block.\footnote{While there is no theoretical reason for disallowing
% this, it was decided in the design of Metamath that allowing it would offer no
% advantage and might cause confusion.}
% \item An inactive math symbol may always be
% redeclared.
% \item  An active math symbol may be redeclared in a different (i.e.\
% inner) block\index{block} from the one it became active in.
% \end{itemize}
%
% When a math symbol token is redeclared, it conceptually refers to a different
% math symbol, just as it would be if it were called a different name.  In
% addition, the original math symbol that it referred to, if it was active,
% temporarily becomes inactive.  At the end of the block in which the
% redeclaration occurred, the new math symbol\index{math symbol} becomes
% inactive and the original symbol becomes active again.  This concept is
% illustrated in the following example, where the symbol \texttt{e} is
% ordinarily a constant (say Euler's constant, 2.71828...) but
% temporarily we want to use it as a ``local'' variable, say as a coefficient
% in the equation $a x^4 + b x^3 + c x^2 + d x + e$:
% \[
%   \mbox{\tt \$\char`\{\ \$c e \$.}
%   \underbrace{
%     \ \ldots\ %
%     \mbox{\tt \$\char`\{}\ \ldots\ %
%   }_{\mbox{\rm region A}}
%   \mbox{\tt \$v e \$.}
%   \underbrace{
%     \mbox{\ \ \ \ldots\ \ \ }
%   }_{\mbox{\rm region B}}
%   \mbox{\tt \$\char`\}}
%   \underbrace{
%     \mbox{\ \ \ \ldots\ \ \ }
%   }_{\mbox{\rm region C}}
%   \mbox{\tt \$\char`\}}
% \]
% \index{\texttt{\$\char`\{} and \texttt{\$\char`\}} keywords}
% In region A, the token \texttt{e} refers to a constant.  It is redeclared as a
% variable in region B, and any reference to it in this region will refer to this
% variable.  In region C, the redeclaration becomes inactive, and the original
% declaration becomes active again.  In region C, the token \texttt{x} refers to the
% original constant.
%
% As a practical matter, overuse of math symbol\index{math symbol}
% redeclarations\index{redeclaration of symbols} can be confusing (even though
% it is well-defined) and is best avoided when possible.  Here are some good
% general guidelines you can follow.  Usually, you should declare all
% constants\index{constant} in the outermost block\index{outermost block},
% especially if they are general-purpose (such as the token \verb$A.$, meaning
% $\forall$ or ``for all'').  This will make them ``globally'' active (although
% as in the example above local redeclarations will temporarily make them
% inactive.)  Most or all variables\index{variable}, on the other hand, could be
% declared in inner blocks, so that the token for them can be used later for a
% different type of variable or a constant.  (The names of the variables you
% choose are not used when you refer to an assertion\index{assertion} in a
% proof, whereas constants must match exactly.  A locally declared constant will
% not match a globally declared constant in a proof, even if they use the same
% token, because Metamath internally considers them to be different math
% symbols.)  To avoid confusion, you should generally avoid redeclaring active
% variables.  If you must redeclare them, do so at the beginning of a block.
% The temporary declaration of constants in inner blocks might be occasionally
% appropriate when you make use of a temporary definition to prove lemmas
% leading to a main result that does not make direct use of the definition.
% This way, you will not clutter up your database with a large number of
% seldom-used global constant symbols.  You might want to note that while
% inactive constants may not appear directly in an assertion (a \texttt{\$a}\index{\texttt{\$a}
% statement} or \texttt{\$p}\index{\texttt{\$p} statement}
% statement), they may be indirectly used in the proof of a \texttt{\$p} statement
% so long as they do not appear in the final math symbol sequence constructed by
% the proof.  In the end, you will have to use your best judgment, taking into
% account standard mathematical usage of the symbols as well as consideration
% for the reader of your work.
%
% \subsubsection{Reuse of Labels}\index{reuse of labels}\index{label}
%
% The \texttt{\$e}\index{\texttt{\$e} statement}, \texttt{\$f}\index{\texttt{\$f}
% statement}, \texttt{\$a}\index{\texttt{\$a} statement}, and
% \texttt{\$p}\index{\texttt{\$p}
% statement} statement types require labels, which allow them to be
% referenced later inside proofs.  A label is considered {\bf
% active}\index{active label} when the statement it is associated with is
% active.  The token\index{token} for a label may be reused
% (redeclared)\index{redeclaration of labels} provided that it is not being used
% for a currently active label.  (Unlike the tokens for math symbols, active
% label tokens may not be redeclared in an inner scope.)  Note that the labels
% of \texttt{\$a} and \texttt{\$p} statements can never be reused after these
% statements appear, because these statements remain active through the end of
% the database.
%
% You might find the reuse of labels a convenient way to have standard names for
% temporary hypotheses, such as \texttt{h1}, \texttt{h2}, etc.  This way you don't have
% to invent unique names for each of them, and in some cases it may be less
% confusing to the reader (although in other cases it might be more confusing, if
% the hypothesis is located far away from the assertion that uses
% it).\footnote{The current implementation requires that all labels, even
% inactive ones, be unique.}

\subsubsection{Frames Revisited}\index{frames and scoping statements}

Now that we have covered scoping, we will look at how an arbitrary
Metamath database can be converted to the simple sequence of extended
frames described on p.~\pageref{framelist}.  This is also how Metamath
stores the database internally when it reads in the database
source.\label{frameconvert} The method is simple.  First, we collect all
constant and variable (\texttt{\$c} and \texttt{\$v}) declarations in
the database, ignoring duplicate declarations of the same variable in
different scopes.  We then put our collected \texttt{\$c} and
\texttt{\$v} declarations at the beginning of the database, so that
their scope is the entire database.  Next, for each assertion in the
database, we determine its frame and extended frame.  The extended frame
is simply the \texttt{\$f}, \texttt{\$e}, and \texttt{\$d} statements
that are active.  The frame is the extended frame with all optional
hypotheses removed.

An equivalent way of saying this is that the extended frame of an assertion
is the collection of all \texttt{\$f}, \texttt{\$e}, and \texttt{\$d} statements
whose scope includes the assertion.
The \texttt{\$f} and \texttt{\$e} statements
occur in the order they appear
(order is irrelevant for \texttt{\$d} statements).

%c%, renaming any
%c%redeclared variables as needed so that all of them have unique names.  (The
%c%exact renaming convention is unimportant.  You might imagine renaming
%c%different declarations of math symbol \texttt{a} as \texttt{a\$1}, \texttt{a\$2}, etc.\
%c%which would prevent any conflicts since \texttt{\$} is not a legal character in a
%c%math symbol token.)

\section{The Anatomy of a Proof} \label{proof}
\index{proof!Metamath, description of}

Each provable assertion (\texttt{\$p}\index{\texttt{\$p} statement} statement) in a
database must include a {\bf proof}\index{proof}.  The proof is located
between the \texttt{\$=}\index{\texttt{\$=} keyword} and \texttt{\$.}\ keywords in the
\texttt{\$p} statement.

In the basic Metamath language\index{basic language}, a proof is a
sequence of statement labels.  This label sequence\index{label sequence}
serves as a set of instructions that the Metamath program uses to
construct a series of math symbol sequences.  The construction must
ultimately result in the math symbol sequence contained between the
\texttt{\$p}\index{\texttt{\$p} statement} and
\texttt{\$=}\index{\texttt{\$=} keyword} keywords of the \texttt{\$p}
statement.  Otherwise, the Metamath program will consider the proof
incorrect, and it will notify you with an appropriate error message when
you ask it to verify the proof.\footnote{To make the loading faster, the
Metamath program does not automatically verify proofs when you
\texttt{read} in a database unless you use the \texttt{/verify}
qualifier.  After a database has been read in, you may use the
\texttt{verify proof *} command to verify proofs.}\index{\texttt{verify
proof} command} Each label in a proof is said to {\bf
reference}\index{label reference} its corresponding statement.

Associated with any assertion\index{assertion} (\texttt{\$p} or
\texttt{\$a}\index{\texttt{\$a} statement} statement) is a set of
hypotheses (\texttt{\$f}\index{\texttt{\$f} statement} or
\texttt{\$e}\index{\texttt{\$e} statement} statements) that are active
with respect to that assertion.  Some are mandatory and the others are
optional.  You should review these concepts if necessary.

Each label\index{label} in a proof must be either the label of a
previous assertion (\texttt{\$a}\index{\texttt{\$a} statement} or
\texttt{\$p}\index{\texttt{\$p} statement} statement) or the label of an
active hypothesis (\texttt{\$e} or \texttt{\$f}\index{\texttt{\$f}
statement} statement) of the \texttt{\$p} statement containing the
proof.  Hypothesis labels may reference both the
mandatory\index{mandatory hypothesis} and the optional hypotheses of the
\texttt{\$p} statement.

The label sequence in a proof specifies a construction in {\bf reverse Polish
notation}\index{reverse Polish notation (RPN)} (RPN).  You may be familiar
with RPN if you have used older
Hewlett--Packard or similar hand-held calculators.
In the calculator analogy, a hypothesis label\index{hypothesis label} is like
a number and an assertion label\index{assertion label} is like an operation
(more precisely, an $n$-ary operation when the
assertion has $n$ \texttt{\$e}-hypotheses).
On an RPN calculator, an operation takes one or more previous numbers in an
input sequence, performs a calculation on them, and replaces those numbers and
itself with the result of the calculation.  For example, the input sequence
$2,3,+$ on an RPN calculator results in $5$, and the input sequence
$2,3,5,{\times},+$ results in $2,15,+$ which results in $17$.

Understanding how RPN is processed involves the concept of a {\bf
stack}\index{stack}\index{RPN stack}, which can be thought of as a set of
temporary memory locations that hold intermediate results.  When Metamath
encounters a hypothesis label it places or {\bf pushes}\index{push} the math
symbol sequence of the hypothesis onto the stack.  When Metamath encounters an
assertion label, it associates the most recent stack entries with the {\em
mandatory} hypotheses\index{mandatory hypothesis} of the assertion, in the
order where the most recent stack entry is associated with the last mandatory
hypothesis of the assertion.  It then determines what
substitutions\index{substitution!variable}\index{variable substitution} have
to be made into the variables of the assertion's mandatory hypotheses to make
them identical to the associated stack entries.  It then makes those same
substitutions into the assertion itself.  Finally, Metamath removes or {\bf
pops}\index{pop} the matched hypotheses from the stack and pushes the
substituted assertion onto the stack.

For the purpose of matching the mandatory hypothesis to the most recent stack
entries, whether a hypothesis is a \texttt{\$e} or \texttt{\$f} statement is
irrelevant.  The only important thing is that a set of
substitutions\footnote{In the Metamath spec (Section~\ref{spec}), we use the
singular term ``substitution'' to refer to the set of substitutions we talk
about here.} exist that allow a match (and if they don't, the proof verifier
will let you know with an error message).  The Metamath language is specified
in such a way that if a set of substitutions exists, it will be unique.
Specifically, the requirement that each variable have a type specified for it
with a \texttt{\$f} statement ensures the uniqueness.

We will illustrate this with an example.
Consider the following Metamath source file:
\begin{verbatim}
$c ( ) -> wff $.
$v p q r s $.
wp $f wff p $.
wq $f wff q $.
wr $f wff r $.
ws $f wff s $.
w2 $a wff ( p -> q ) $.
wnew $p wff ( s -> ( r -> p ) ) $= ws wr wp w2 w2 $.
\end{verbatim}
This Metamath source example shows the definition and ``proof'' (i.e.,
construction) of a well-formed formula (wff)\index{well-formed formula (wff)}
in propositional calculus.  (You may wish to type this example into a file to
experiment with the Metamath program.)  The first two statements declare
(introduce the names of) four constants and four variables.  The next four
statements specify the variable types, namely that
each variable is assumed to be a wff.  Statement \texttt{w2} defines (postulates)
a way to produce a new wff, \texttt{( p -> q )}, from two given wffs \texttt{p} and
\texttt{q}. The mandatory hypotheses of \texttt{w2} are \texttt{wp} and \texttt{wq}.
Statement \texttt{wnew} claims that \texttt{( s -> ( r -> p ) )} is a wff given
three wffs \texttt{s}, \texttt{r}, and \texttt{p}.  More precisely, \texttt{wnew} claims
that the sequence of ten symbols \texttt{wff ( s -> ( r -> p ) )} is provable from
previous assertions and the hypotheses of \texttt{wnew}.  Metamath does not know
or care what a wff is, and as far as it is concerned
the typecode \texttt{wff} is just an
arbitrary constant symbol in a math symbol sequence.  The mandatory hypotheses
of \texttt{wnew} are \texttt{wp}, \texttt{wr}, and \texttt{ws}; \texttt{wq} is an optional
hypothesis.  In our particular proof, the optional hypothesis is not
referenced, but in general, any combination of active (i.e.\ optional and
mandatory) hypotheses could be referenced.  The proof of statement \texttt{wnew}
is the sequence of five labels starting with \texttt{ws} (step~1) and ending with
\texttt{w2} (step~5).

When Metamath verifies the proof, it scans the proof from left to right.  We
will examine what happens at each step of the proof.  The stack starts off
empty.  At step 1, Metamath looks up label \texttt{ws} and determines that it is a
hypothesis, so it pushes the symbol sequence of statement \texttt{ws} onto the
stack:

\begin{center}\begin{tabular}{|l|l|}\hline
{Stack location} & {Contents} \\ \hline \hline
1 & \texttt{wff s} \\ \hline
\end{tabular}\end{center}

Metamath sees that the labels \texttt{wr} and \texttt{wp} in steps~2 and 3 are also
hypotheses, so it pushes them onto the stack.  After step~3, the stack looks
like
this:

\begin{center}\begin{tabular}{|l|l|}\hline
{Stack location} & {Contents} \\ \hline \hline
3 & \texttt{wff p} \\ \hline
2 & \texttt{wff r} \\ \hline
1 & \texttt{wff s} \\ \hline
\end{tabular}\end{center}

At step 4, Metamath sees that label \texttt{w2} is an assertion, so it must do
some processing.  First, it associates the mandatory hypotheses of \texttt{w2},
which are \texttt{wp} and \texttt{wq}, with stack locations~2 and 3, {\em in that
order}. Metamath determines that the only possible way
to make hypothesis \texttt{wp} match (become identical to) stack location~2 and
\texttt{wq} match stack location 3 is to substitute variable \texttt{p} with \texttt{r}
and \texttt{q} with \texttt{p}.  Metamath makes these substitutions into \texttt{w2} and
obtains the symbol sequence \texttt{wff ( r -> p )}.  It removes the hypotheses
from stack locations~2 and 3, then places the result into stack location~2:

\begin{center}\begin{tabular}{|l|l|}\hline
{Stack location} & {Contents} \\ \hline \hline
2 & \texttt{wff ( r -> p )} \\ \hline
1 & \texttt{wff s} \\ \hline
\end{tabular}\end{center}

At step 5, Metamath sees that label \texttt{w2} is an assertion, so it must again
do some processing.  First, it matches the mandatory hypotheses of \texttt{w2},
which are \texttt{wp} and \texttt{wq}, to stack locations 1 and 2.
Metamath determines that the only possible way to make the
hypotheses match is to substitute variable \texttt{p} with \texttt{s} and \texttt{q} with
\texttt{( r -> p )}.  Metamath makes these substitutions into \texttt{w2} and obtains
the symbol
sequence \texttt{wff ( s -> ( r -> p ) )}.  It removes stack
locations 1 and 2, then places the result into stack location~1:

\begin{center}\begin{tabular}{|l|l|}\hline
{Stack location} & {Contents} \\ \hline \hline
1 & \texttt{wff ( s -> ( r -> p ) )} \\ \hline
\end{tabular}\end{center}

After Metamath finishes processing the proof, it checks to see that the
stack contains exactly one element and that this element is
the same as the math symbol sequence in the
\texttt{\$p}\index{\texttt{\$p} statement} statement.  This is the case for our
proof of \texttt{wnew},
so we have proved \texttt{wnew} successfully.  If the result
differs, Metamath will notify you with an error message.  An error message
will also result if the stack contains more than one entry at the end of the
proof, or if the stack did not contain enough entries at any point in the
proof to match all of the mandatory hypotheses\index{mandatory hypothesis} of
an assertion.  Finally, Metamath will notify you with an error message if no
substitution is possible that will make a referenced assertion's hypothesis
match the
stack entries.  You may want to experiment with the different kinds of errors
that Metamath will detect by making some small changes in the proof of our
example.

Metamath's proof notation was designed primarily to express proofs in a
relatively compact manner, not for readability by humans.  Metamath can display
proofs in a number of different ways with the \texttt{show proof}\index{\texttt{show
proof} command} command.  The
\texttt{/lemmon} qualifier displays it in a format that is easier to read when the
proofs are short, and you saw examples of its use in Chapter~\ref{using}.  For
longer proofs, it is useful to see the tree structure of the proof.  A tree
structure is displayed when the \texttt{/lemmon} qualifier is omitted.  You will
probably find this display more convenient as you get used to it. The tree
display of the proof in our example looks like
this:\label{treeproof}\index{tree-style proof}\index{proof!tree-style}
\begin{verbatim}
1     wp=ws    $f wff s
2        wp=wr    $f wff r
3        wq=wp    $f wff p
4     wq=w2    $a wff ( r -> p )
5  wnew=w2  $a wff ( s -> ( r -> p ) )
\end{verbatim}
The number to the left of each line is the step number.  Following it is a
{\bf hypothesis association}\index{hypothesis association}, consisting of two
labels\index{label} separated by \texttt{=}.  To the left of the \texttt{=} (except
in the last step) is the label of a hypothesis of an assertion referenced
later in the proof; here, steps 1 and 4 are the hypothesis associations for
the assertion \texttt{w2} that is referenced in step 5.  A hypothesis association
is indented one level more than the assertion that uses it, so it is easy to
find the corresponding assertion by moving directly down until the indentation
level decreases to one less than where you started from.  To the right of each
\texttt{=} is the proof step label for that proof step.  The statement keyword of
the proof step label is listed next, followed by the content of the top of the
stack (the most recent stack entry) as it exists after that proof step is
processed.  With a little practice, you should have no trouble reading proofs
displayed in this format.

Metamath proofs include the syntax construction of a formula.
In standard mathematics, this kind of
construction is not considered a proper part of the proof at all, and it
certainly becomes rather boring after a while.
Therefore,
by default the \texttt{show proof}\index{\texttt{show proof}
command} command does not show the syntax construction.
Historically \texttt{show proof} command
\textit{did} show the syntax construction, and you needed to add the
\texttt{/essential} option to hide, them, but today
\texttt{/essential} is the default and you need to use
\texttt{/all} to see the syntax constructions.

When verifying a proof, Metamath will check that no mandatory
\texttt{\$d}\index{\texttt{\$d} statement}\index{mandatory \texttt{\$d}
statement} statement of an assertion referenced in a proof is violated
when substitutions\index{substitution!variable}\index{variable
substitution} are made to the variables in the assertion.  For details
see Section~\ref{spec4} or \ref{dollard}.

\subsection{The Concept of Unification} \label{unify}

During the course of verifying a proof, when Metamath\index{Metamath}
encounters an assertion label\index{assertion label}, it associates the
mandatory hypotheses\index{mandatory hypothesis} of the assertion with the top
entries of the RPN stack\index{stack}\index{RPN stack}.  Metamath then
determines what substitutions\index{substitution!variable}\index{variable
substitution} it must make to the variables in the assertion's mandatory
hypotheses in order for these hypotheses to become identical to their
corresponding stack entries.  This process is called {\bf
unification}\index{unification}.  (We also informally use the term
``unification'' to refer to a set of substitutions that results from the
process, as in ``two unifications are possible.'')  After the substitutions
are made, the hypotheses are said to be {\bf unified}.

If no such substitutions are possible, Metamath will consider the proof
incorrect and notify you with an error message.
% (deleted 3/10/07, per suggestion of Mel O'Cat:)
% The syntax of the
% Metamath language ensures that if a set of substitutions exists, it
% will be unique.

The general algorithm for unification described in the literature is
somewhat complex.
However, in the case of Metamath it is intentionally trivial.
Mandatory hypotheses must be
pushed on the proof stack in the order in which they appear.
In addition, each variable must have its type specified
with a \texttt{\$f} hypothesis before it is used
and that each \texttt{\$f} hypothesis
have the restricted syntax of a typecode (a constant) followed by a variable.
The typecode in the \texttt{\$f} hypothesis must match the first symbol of
the corresponding RPN stack entry (which will also be a constant), so
the only possible match for the variable in the \texttt{\$f} hypothesis is
the sequence of symbols in the stack entry after the initial constant.

In the Proof Assistant\index{Proof Assistant}, a more general unification
algorithm is used.  While a proof is being developed, sometimes not enough
information is available to determine a unique unification.  In this case
Metamath will ask you to pick the correct one.\index{ambiguous
unification}\index{unification!ambiguous}

\section{Extensions to the Metamath Language}\index{extended
language}

\subsection{Comments in the Metamath Language}\label{comments}
\index{markup notation}
\index{comments!markup notation}

The commenting feature allows you to annotate the contents of
a database.  Just as with most
computer languages, comments are ignored for the purpose of interpreting the
contents of the database. Comments effectively act as
additional white space\index{white
space} between tokens
when a database is parsed.

A comment may be placed at the beginning, end, or
between any two tokens\index{token} in a source file.

Comments have the following syntax:
\begin{center}
 \texttt{\$(} {\em text} \texttt{\$)}
\end{center}
Here,\index{\texttt{\$(} and \texttt{\$)} auxiliary
keywords}\index{comment} {\em text} is a string, possibly empty, of any
characters in Metamath's character set (p.~\pageref{spec1chars}), except
that the character strings \texttt{\$(} and \texttt{\$)} may not appear
in {\em text}.  Thus nested comments are not
permitted:\footnote{Computer languages have differing standards for
nested comments, and rather than picking one it was felt simplest not to
allow them at all, at least in the current version (0.177) of
Metamath\index{Metamath!limitations of version 0.177}.} Metamath will
complain if you give it
\begin{center}
 \texttt{\$( This is a \$( nested \$) comment.\ \$)}
\end{center}
To compensate for this non-nesting behavior, I often change all \texttt{\$}'s
to \texttt{@}'s in sections of Metamath code I wish to comment out.

The Metamath program supports a number of markup mechanisms and conventions
to generate good-looking results in \LaTeX\ and {\sc html},
as discussed below.
These markup features have to do only with how the comments are typeset,
and have no effect on how Metamath verifies the proofs in the database.
The improper
use of them may result in incorrectly typeset output, but no Metamath
error messages will result during the \texttt{read} and \texttt{verify
proof} commands.  (However, the \texttt{write
theorem\texttt{\char`\_}list} command
will check for markup errors as a side-effect of its
{\sc html} generation.)
Section~\ref{texout} has instructions for creating \LaTeX\ output, and
section~\ref{htmlout} has instructions for creating
{\sc html}\index{HTML} output.

\subsubsection{Headings}\label{commentheadings}

If the \texttt{\$(} is immediately followed by a new line
starting with a heading marker, it is a header.
This can start with:

\begin{itemize}
 \item[] \texttt{\#\#\#\#} - major part header
 \item[] \texttt{\#*\#*} - section header
 \item[] \texttt{=-=-} - subsection header
 \item[] \texttt{-.-.} - subsubsection header
\end{itemize}

The line following the marker line
will be used for the table of contents entry, after trimming spaces.
The next line should be another (closing) matching marker line.
Any text after that
but before the closing \texttt{\$}, such as an extended description of the
section, will be included on the \texttt{mmtheoremsNNN.html} page.

For more information, run
\texttt{help write theorem\char`\_list}.

\subsubsection{Math mode}
\label{mathcomments}
\index{\texttt{`} inside comments}
\index{\texttt{\char`\~} inside comments}
\index{math mode}

Inside of comments, a string of tokens\index{token} enclosed in
grave accents\index{grave accent (\texttt{`})} (\texttt{`}) will be converted
to standard mathematical symbols during
{\sc HTML}\index{HTML} or \LaTeX\ output
typesetting,\index{latex@{\LaTeX}} according to the information in the
special \texttt{\$t}\index{\texttt{\$t} comment}\index{typesetting
comment} comment in the database
(see section~\ref{tcomment} for information about the typesetting
comment, and Appendix~\ref{ASCII} to see examples of its results).

The first grave accent\index{grave accent (\texttt{`})} \texttt{`}
causes the output processor to enter {\bf math mode}\index{math mode}
and the second one exits it.
In this
mode, the characters following the \texttt{`} are interpreted as a
sequence of math symbol tokens separated by white space\index{white
space}.  The tokens are looked up in the \texttt{\$t}
comment\index{\texttt{\$t} comment}\index{typesetting comment} and if
found, they will be replaced by the standard mathematical symbols that
they correspond to before being placed in the typeset output file.  If
not found, the symbol will be output as is and a warning will be issued.
The tokens do not have to be active in the database, although a warning
will be issued if they are not declared with \texttt{\$c} or
\texttt{\$v} statements.

Two consecutive
grave accents \texttt{``} are treated as a single actual grave accent
(both inside and outside of math mode) and will not cause the output
processor to enter or exit math mode.

Here is an example of its use\index{Pierce's axiom}:
\begin{center}
\texttt{\$( Pierce's axiom, ` ( ( ph -> ps ) -> ph ) -> ph ` ,\\
         is not very intuitive. \$)}
\end{center}
becomes
\begin{center}
   \texttt{\$(} Pierce's axiom, $((\varphi \rightarrow \psi)\rightarrow
\varphi)\rightarrow \varphi$, is not very intuitive. \texttt{\$)}
\end{center}

Note that the math symbol tokens\index{token} must be surrounded by white
space\index{white space}.
%, since there is no context that allows ambiguity to be
%resolved, as is the case with math symbol sequences in some of the Metamath
%statements.
White space should also surround the \texttt{`}
delimiters.

The math mode feature also gives you a quick and easy way to generate
text containing mathematical symbols, independently of the intended
purpose of Metamath.\index{Metamath!using as a math editor} To do this,
simply create your text with grave accents surrounding your formulas,
after making sure that your math symbols are mapped to \LaTeX\ symbols
as described in Appendix~\ref{ASCII}.  It is easier if you start with a
database with predefined symbols such as \texttt{set.mm}.  Use your
grave-quoted math string to replace an existing comment, then typeset
the statement corresponding to that comment following the instructions
from the \texttt{help tex} command in the Metamath program.  You will
then probably want to edit the resulting file with a text editor to fine
tune it to your exact needs.

\subsubsection{Label Mode}\index{label mode}

Outside of math mode, a tilde\index{tilde (\texttt{\char`\~})} \verb/~/
indicates to Metamath's\index{Metamath} output processor that the
token\index{token} that follows (i.e.\ the characters up to the next
white space\index{white space}) represents a statement label or URL.
This formatting mode is called {\bf label mode}\index{label mode}.
If a literal tilde
is desired (outside of math mode) instead of label mode,
use two tildes in a row to represent it.

When generating a \LaTeX\ output file,
the following token will be formatted in \texttt{typewriter}
font, and the tilde removed, to make it stand out from the rest of the text.
This formatting will be applied to all characters after the
tilde up to the first white space\index{white space}.
Whether
or not the token is an actual statement label is not checked, and the
token does not have to have the correct syntax for a label; no error
messages will be produced.  The only effect of the label mode on the
output is that typewriter font will be used for the tokens that are
placed in the \LaTeX\ output file.

When generating {\sc html},
the tokens after the tilde {\em must} be a URL (either http: or https:)
or a valid label.
Error messages will be issued during that output if they aren't.
A hyperlink will be generated to that URL or label.

\subsubsection{Link to bibliographical reference}\index{citation}%
\index{link to bibliographical reference}

Bibliographical references are handled specially when generating
{\sc html} if formatted specially.
Text in the form \texttt{[}{\em author}\texttt{]}
is considered a link to a bibliographical reference.
See \texttt{help html} and \texttt{help write
bibliography} in the Metamath program for more
information.
% \index{\texttt{\char`\[}\ldots\texttt{]} inside comments}
See also Sections~\ref{tcomment} and \ref{wrbib}.

The \texttt{[}{\em author}\texttt{]} notation will also create an entry in
the bibliography cross-reference file generated by \texttt{write
bibliography} (Section~\ref{wrbib}) for {\sc HTML}.
For this to work properly, the
surrounding comment must be formatted as follows:
\begin{quote}
    {\em keyword} {\em label} {\em noise-word}
     \texttt{[}{\em author}\texttt{] p.} {\em number}
\end{quote}
for example
\begin{verbatim}
     Theorem 5.2 of [Monk] p. 223
\end{verbatim}
The {\em keyword} is not case sensitive and must be one of the following:
\begin{verbatim}
     theorem lemma definition compare proposition corollary
     axiom rule remark exercise problem notation example
     property figure postulate equation scheme chapter
\end{verbatim}
The optional {\em label} may consist of more than one
(non-{\em keyword} and non-{\em noise-word}) word.
The optional {\em noise-word} is one of:
\begin{verbatim}
     of in from on
\end{verbatim}
and is  ignored when the cross-reference file is created.  The
\texttt{write
biblio\-graphy} command will perform error checking to verify the
above format.\index{error checking}

\subsubsection{Parentheticals}\label{parentheticals}

The end of a comment may include one or more parenthicals, that is,
statements enclosed in parentheses.
The Metamath program looks for certain parentheticals and can issue
warnings based on them.
They are:

\begin{itemize}
 \item[] \texttt{(Contributed by }
   \textit{NAME}\texttt{,} \textit{DATE}\texttt{.)} -
   document the original contributor's name and the date it was created.
 \item[] \texttt{(Revised by }
   \textit{NAME}\texttt{,} \textit{DATE}\texttt{.)} -
   document the contributor's name and creation date
   that resulted in significant revision
   (not just an automated minimization or shortening).
 \item[] \texttt{(Proof shortened by }
   \textit{NAME}\texttt{,} \textit{DATE}\texttt{.)} -
   document the contributor's name and date that developed a significant
   shortening of the proof (not just an automated minimization).
 \item[] \texttt{(Proof modification is discouraged.)} -
   Note that this proof should normally not be modified.
 \item[] \texttt{(New usage is discouraged.)} -
   Note that this assertion should normally not be used.
\end{itemize}

The \textit{DATE} must be in form YYYY-MMM-DD, where MMM is the
English abbreviation of that month.

\subsubsection{Other markup}\label{othermarkup}
\index{markup notation}

There are other markup notations for generating good-looking results
beyond math mode and label mode:

\begin{itemize}
 \item[]
         \texttt{\char`\_} (underscore)\index{\texttt{\char`\_} inside comments} -
             Italicize text starting from
              {\em space}\texttt{\char`\_}{\em non-space} (i.e.\ \texttt{\char`\_}
              with a space before it and a non-space character after it) until
             the next
             {\em non-space}\texttt{\char`\_}{\em space}.  Normal
             punctuation (e.g.\ a trailing
             comma or period) is ignored when determining {\em space}.
 \item[]
         \texttt{\char`\_} (underscore) - {\em
         non-space}\texttt{\char`\_}{\em non-space-string}, where
          {\em non-space-string} is a string of non-space characters,
         will make {\em non-space-string} become a subscript.
 \item[]
         \texttt{<HTML>}...\texttt{</HTML>} - do not convert
         ``\texttt{<}'' and ``\texttt{>}''
         in the enclosed text when generating {\sc HTML},
         otherwise process markup normally. This allows direct insertion
         of {\sc html} commands.
 \item[]
       ``\texttt{\&}ref\texttt{;}'' - insert an {\sc HTML}
         character reference.
         This is how to insert arbitrary Unicode characters
         (such as accented characters).  Currently only directly supported
         when generating {\sc HTML}.
\end{itemize}

It is recommended that spaces surround any \texttt{\char`\~} and
\texttt{`} tokens in the comment and that a space follow the {\em label}
after a \texttt{\char`\~} token.  This will make global substitutions
to change labels and symbol names much easier and also eliminate any
future chance of ambiguity.  Spaces around these tokens are automatically
removed in the final output to conform with normal rules of punctuation;
for example, a space between a trailing \texttt{`} and a left parenthesis
will be removed.

A good way to become familiar with the markup notation is to look at
the extensive examples in the \texttt{set.mm} database.

\subsection{The Typesetting Comment (\texttt{\$t})}\label{tcomment}

The typesetting comment \texttt{\$t} in the input database file
provides the information necessary to produce good-looking results.
It provides \LaTeX\ and {\sc html}
definitions for math symbols,
as well supporting as some
customization of the generated web page.
If you add a new token to a database, you should also
update the \texttt{\$t} comment information if you want to eventually
create output in \LaTeX\ or {\sc HTML}.
See the
\texttt{set.mm}\index{set theory database (\texttt{set.mm})} database
file for an extensive example of a \texttt{\$t} comment illustrating
many of the features described below.

Programs that do not need to generate good-looking presentation results,
such as programs that only verify Metamath databases,
can completely ignore typesetting comments
and just treat them as normal comments.
Even the Metamath program only consults the
\texttt{\$t} comment information when it needs to generate typeset output
in \LaTeX\ or {\sc HTML}
(e.g., when you open a \LaTeX\ output file with the \texttt{open tex} command).

We will first discuss the syntax of typesetting comments, and then
briefly discuss how this can be used within the Metamath program.

\subsubsection{Typesetting Comment Syntax Overview}

The typesetting comment is identified by the token
\texttt{\$t}\index{\texttt{\$t} comment}\index{typesetting comment} in
the comment, and the typesetting comment ends at the matching
\texttt{\$)}:
\[
  \mbox{\tt \$(\ }
  \mbox{\tt \$t\ }
  \underbrace{
    \mbox{\tt \ \ \ \ \ \ \ \ \ \ \ }
    \cdots
    \mbox{\tt \ \ \ \ \ \ \ \ \ \ \ }
  }_{\mbox{Typesetting definitions go here}}
  \mbox{\tt \ \$)}
\]

There must be one or more white space characters, and only white space
characters, between the \texttt{\$(} that starts the comment
and the \texttt{\$t} symbol,
and the \texttt{\$t} must be followed by one
or more white space characters
(see section \ref{whitespace} for the definition of white space characters).
The typesetting comment continues until the comment end token \texttt{\$)}
(which must be preceded by one or more white space characters).

In version 0.177\index{Metamath!limitations of version 0.177} of the
Metamath program, there may be only one \texttt{\$t} comment in a
database.  This restriction may be lifted in the future to allow
many \texttt{\$t} comments in a database.

Between the \texttt{\$t} symbol (and its following white space) and the
comment end token \texttt{\$)} (and its preceding white space)
is a sequence of one or more typesetting definitions, where
each definition has the form
\textit{definition-type arg arg ... ;}.
Each of the zero or more \textit{arg} values
can be either a typesetting data or a keyword
(what keywords are allowed, and where, depends on the specific
\textit{definition-type}).
The \textit{definition-type}, and each argument \textit{arg},
are separated by one or more white space characters.
Every definition ends in an unquoted semicolon;
white space is not required before the terminating semicolon of a definition.
Each definition should start on a new line.\footnote{This
restriction of the current version of Metamath
(0.177)\index{Metamath!limitations of version 0.177} may be removed
in a future version, but you should do it anyway for readability.}

For example, this typesetting definition:
\begin{center}
 \verb$latexdef "C_" as "\subseteq";$
\end{center}
defines the token \verb$C_$ as the \LaTeX\ symbol $\subseteq$ (which means
``subset'').

Typesetting data is a sequence of one or more quoted strings
(if there is more than one, they are connected by \texttt{\char`\+}).
Often a single quoted string is used to provide data for a definition, using
either double (\texttt{\char`\"}) or single (\texttt{'}) quotation marks.
However,
{\em a quoted string (enclosed in quotation marks) may not include
line breaks.}
A quoted string
may include a quotation mark that matches the enclosing quotes by repeating
the quotation mark twice.  Here are some examples:

\begin{tabu}   { l l }
\textbf{Example} & \textbf{Meaning} \\
\texttt{\char`\"a\char`\"\char`\"b\char`\"} & \texttt{a\char`\"b} \\
\texttt{'c''d'} & \texttt{c'd} \\
\texttt{\char`\"e''f\char`\"} & \texttt{e''f} \\
\texttt{'g\char`\"\char`\"h'} & \texttt{g\char`\"\char`\"h} \\
\end{tabu}

Finally, a long quoted string
may be broken up into multiple quoted strings (considered, as a whole,
a single quoted string) and joined with \texttt{\char`\+}.
You can even use multiple lines as long as a
'+' is at the end of every line except the last one.
The \texttt{\char`\+} should be preceded and followed by at least one
white space character.
Thus, for example,
\begin{center}
 \texttt{\char`\"ab\char`\"\ \char`\+\ \char`\"cd\char`\"
    \ \char`\+\ \\ 'ef'}
\end{center}
is the same as
\begin{center}
 \texttt{\char`\"abcdef\char`\"}
\end{center}

{\sc c}-style comments \texttt{/*}\ldots\texttt{*/} are also supported.

In practice, whenever you add a new math token you will often want to add
typesetting definitions using
\texttt{latexdef}, \texttt{htmldef}, and
\texttt{althtmldef}, as described below.
That way, they will all be up to date.
Of course, whether or not you want to use all three definitions will
depend on how the database is intended to be used.

Below we discuss the different possible \textit{definition-kind} options.
We will show data surrounded by double quotes (in practice they can also use
single quotes and/or be a sequence joined by \texttt{+}s).
We will use specific names for the \textit{data} to make clear what
the data is used for, such as
{\em math-token} (for a Metamath math token,
{\em latex-string} (for string to be placed in a \LaTeX\ stream),
{\em {\sc html}-code} (for {\sc html} code),
and {\em filename} (for a filename).

\subsubsection{Typesetting Comment - \LaTeX}

The syntax for a \LaTeX\ definition is:
\begin{center}
 \texttt{latexdef "}{\em math-token}\texttt{" as "}{\em latex-string}\texttt{";}
\end{center}
\index{latex definitions@\LaTeX\ definitions}%
\index{\texttt{latexdef} statement}

The {\em token-string} and {\em latex-string} are the data
(character strings) for
the token and the \LaTeX\ definition of the token, respectively,

These \LaTeX\ definitions are used by the Metamath program
when it is asked to product \LaTeX output using
the \texttt{write tex} command.

\subsubsection{Typesetting Comment - {\sc html}}

The key kinds of {\sc HTML} definitions have the following syntax:

\vskip 1ex
    \texttt{htmldef "}{\em math-token}\texttt{" as "}{\em
    {\sc html}-code}\texttt{";}\index{\texttt{htmldef} statement}
                    \ \ \ \ \ \ldots

    \texttt{althtmldef "}{\em math-token}\texttt{" as "}{\em
{\sc html}-code}\texttt{";}\index{\texttt{althtmldef} statement}

                    \ \ \ \ \ \ldots

Note that in {\sc HTML} there are two possible definitions for math tokens.
This feature is useful when
an alternate representation of symbols is desired, for example one that
uses Unicode entities and another uses {\sc gif} images.

There are many other typesetting definitions that can control {\sc HTML}.
These include:

\vskip 1ex

    \texttt{htmldef "}{\em math-token}\texttt{" as "}{\em {\sc
    html}-code}\texttt{";}

    \texttt{htmltitle "}{\em {\sc html}-code}\texttt{";}%
\index{\texttt{htmltitle} statement}

    \texttt{htmlhome "}{\em {\sc html}-code}\texttt{";}%
\index{\texttt{htmlhome} statement}

    \texttt{htmlvarcolor "}{\em {\sc html}-code}\texttt{";}%
\index{\texttt{htmlvarcolor} statement}

    \texttt{htmlbibliography "}{\em filename}\texttt{";}%
\index{\texttt{htmlbibliography} statement}

\vskip 1ex

\noindent The \texttt{htmltitle} is the {\sc html} code for a common
title, such as ``Metamath Proof Explorer.''  The \texttt{htmlhome} is
code for a link back to the home page.  The \texttt{htmlvarcolor} is
code for a color key that appears at the bottom of each proof.  The file
specified by {\em filename} is an {\sc html} file that is assumed to
have a \texttt{<A NAME=}\ldots\texttt{>} tag for each bibiographic
reference in the database comments.  For example, if
\texttt{[Monk]}\index{\texttt{\char`\[}\ldots\texttt{]} inside comments}
occurs in the comment for a theorem, then \texttt{<A NAME='Monk'>} must
be present in the file; if not, a warning message is given.

Associated with
\texttt{althtmldef}
are the statements
\vskip 1ex

    \texttt{htmldir "}{\em
      directoryname}\texttt{";}\index{\texttt{htmldir} statement}

    \texttt{althtmldir "}{\em
     directoryname}\texttt{";}\index{\texttt{althtmldir} statement}

\vskip 1ex
\noindent giving the directories of the {\sc gif} and Unicode versions
respectively; their purpose is to provide cross-linking between the
two versions in the generated web pages.

When two different types of pages need to be produced from a single
database, such as the Hilbert Space Explorer that extends the Metamath
Proof Explorer, ``extended'' variables may be declared in the
\texttt{\$t} comment:
\vskip 1ex

    \texttt{exthtmltitle "}{\em {\sc html}-code}\texttt{";}%
\index{\texttt{exthtmltitle} statement}

    \texttt{exthtmlhome "}{\em {\sc html}-code}\texttt{";}%
\index{\texttt{exthtmlhome} statement}

    \texttt{exthtmlbibliography "}{\em filename}\texttt{";}%
\index{\texttt{exthtmlbibliography} statement}

\vskip 1ex
\noindent When these are declared, you also must declare
\vskip 1ex

    \texttt{exthtmllabel "}{\em label}\texttt{";}%
\index{\texttt{exthtmllabel} statement}

\vskip 1ex \noindent that identifies the database statement where the
``extended'' section of the database starts (in our example, where the
Hilbert Space Explorer starts).  During the generation of web pages for
that starting statement and the statements after it, the {\sc html} code
assigned to \texttt{exthtmltitle} and \texttt{exthtmlhome} is used
instead of that assigned to \texttt{htmltitle} and \texttt{htmlhome},
respectively.

\begin{sloppy}
\subsection{Additional Information Com\-ment (\texttt{\$j})} \label{jcomment}
\end{sloppy}

The additional information comment, aka the
\texttt{\$j}\index{\texttt{\$j} comment}\index{additional information comment}
comment,
provides a way to add additional structured information that can
be optionally parsed by systems.

The additional information comment is parsed the same way as the
typesetting comment (\texttt{\$t}) (see section \ref{tcomment}).
That is,
the additional information comment begins with the token
\texttt{\$j} within a comment,
and continues until the comment close \texttt{\$)}.
Within an additional information comment is a sequence of one or more
commands of the form \texttt{command arg arg ... ;}
where each of the zero or more \texttt{arg} values
can be either a quoted string or a keyword.
Note that every command ends in an unquoted semicolon.
If a verifier is parsing an additional information comment, but
doesn't recognize a particular command, it must skip the command
by finding the end of the command (an unquoted semicolon).

A database may have 0 or more additional information comments.
Note, however, that a verifier may ignore these comments entirely or only
process certain commands in an additional information comment.
The \texttt{mmj2} verifier supports many commands in additional information
comments.
We encourage systems that process additional information comments
to coordinate so that they will use the same command for the same effect.

Examples of additional information comments with various commands
(from the \texttt{set.mm} database) are:

\begin{itemize}
   \item Define the syntax and logical typecodes,
     and declare that our grammar is
     unambiguous (verifiable using the KLR parser, with compositing depth 5).
\begin{verbatim}
  $( $j
    syntax 'wff';
    syntax '|-' as 'wff';
    unambiguous 'klr 5';
  $)
\end{verbatim}

   \item Register $\lnot$ and $\rightarrow$ as primitive expressions
           (lacking definitions).
\begin{verbatim}
  $( $j primitive 'wn' 'wi'; $)
\end{verbatim}

   \item There is a special justification for \texttt{df-bi}.
\begin{verbatim}
  $( $j justification 'bijust' for 'df-bi'; $)
\end{verbatim}

   \item Register $\leftrightarrow$ as an equality for its type (wff).
\begin{verbatim}
  $( $j
    equality 'wb' from 'biid' 'bicomi' 'bitri';
    definition 'dfbi1' for 'wb';
  $)
\end{verbatim}

   \item Theorem \texttt{notbii} is the congruence law for negation.
\begin{verbatim}
  $( $j congruence 'notbii'; $)
\end{verbatim}

   \item Add \texttt{setvar} as a typecode.
\begin{verbatim}
  $( $j syntax 'setvar'; $)
\end{verbatim}

   \item Register $=$ as an equality for its type (\texttt{class}).
\begin{verbatim}
  $( $j equality 'wceq' from 'eqid' 'eqcomi' 'eqtri'; $)
\end{verbatim}

\end{itemize}


\subsection{Including Other Files in a Metamath Source File} \label{include}
\index{\texttt{\$[} and \texttt{\$]} auxiliary keywords}

The keywords \texttt{\$[} and \texttt{\$]} specify a file to be
included\index{included file}\index{file inclusion} at that point in a
Metamath\index{Metamath} source file\index{source file}.  The syntax for
including a file is as follows:
\begin{center}
\texttt{\$[} {\em file-name} \texttt{\$]}
\end{center}

The {\em file-name} should be a single token\index{token} with the same syntax
as a math symbol (i.e., all 93 non-whitespace
printable characters other than \texttt{\$} are
allowed, subject to the file-naming limitations of your operating system).
Comments may appear between the \texttt{\$[} and \texttt{\$]} keywords.  Included
files may include other files, which may in turn include other files, and so
on.

For example, suppose you want to use the set theory database as the starting
point for your own theory.  The first line in your file could be
\begin{center}
\texttt{\$[ set.mm \$]}
\end{center} All of the information (axioms, theorems,
etc.) in \texttt{set.mm} and any files that {\em it} includes will become
available for you to reference in your file. This can help make your work more
modular. A drawback to including files is that if you change the name of a
symbol or the label of a statement, you must also remember to update any
references in any file that includes it.


The naming conventions for included files are the same as those of your
operating system.\footnote{On the Macintosh, prior to Mac OS X,
 a colon is used to separate disk
and folder names from your file name.  For example, {\em volume}\texttt{:}{\em
file-name} refers to the root directory, {\em volume}\texttt{:}{\em
folder-name}\texttt{:}{\em file-name} refers to a folder in root, and {\em
volume}\texttt{:}{\em folder-name}\texttt{:}\ldots\texttt{:}{\em file-name} refers to a
deeper folder.  A simple {\em file-name} refers to a file in the folder from
which you launch the Metamath application.  Under Mac OS X and later,
the Metamath program is run under the Terminal application, which
conforms to Unix naming conventions.}\index{Macintosh file
names}\index{file names!Macintosh}\label{includef} For compatibility among
operating systems, you should keep the file names as simple as possible.  A
good convention to use is {\em file}\texttt{.mm} where {\em file} is eight
characters or less, in lower case.

There is no limit to the nesting depth of included files.  One thing that you
should be aware of is that if two included files themselves include a common
third file, only the {\em first} reference to this common file will be read
in.  This allows you to include two or more files that build on a common
starting file without having to worry about label and symbol conflicts that
would occur if the common file were read in more than once.  (In fact, if a
file includes itself, the self-reference will be ignored, although of course
it would not make any sense to do that.)  This feature also means, however,
that if you try to include a common file in several inner blocks, the result
might not be what you expect, since only the first reference will be replaced
with the included file (unlike the include statement in most other computer
languages).  Thus you would normally include common files only in the
outermost block\index{outermost block}.

\subsection{Compressed Proof Format}\label{compressed1}\index{compressed
proof}\index{proof!compressed}

The proof notation presented in Section~\ref{proof} is called a
{\bf normal proof}\index{normal proof}\index{proof!normal} and in principle is
sufficient to express any proof.  However, proofs often contain steps and
subproofs that are identical.  This is particularly true in typical
Metamath\index{Metamath} applications, because Metamath requires that the math
symbol sequence (usually containing a formula) at each step be separately
constructed, that is, built up piece by piece. As a result, a lot of
repetition often results.  The {\bf compressed proof} format allows Metamath
to take advantage of this redundancy to shorten proofs.

The specification for the compressed proof format is given in
Appen\-dix~\ref{compressed}.

Normally you need not concern yourself with the details of the compressed
proof format, since the Metamath program will allow you to convert from
the normal format to the compressed format with ease, and will also
automatically convert from the compressed format when proofs are displayed.
The overall structure of the compressed format is as follows:
\begin{center}
  \texttt{\$= ( } {\em label-list} \texttt{) } {\em compressed-proof\ }\ \texttt{\$.}
\end{center}
\index{\texttt{\$=} keyword}
The first \texttt{(} serves as a flag to Metamath that a compressed proof
follows.  The {\em label-list} includes all statements referred to by the
proof except the mandatory hypotheses\index{mandatory hypothesis}.  The {\em
compressed-proof} is a compact encoding of the proof, using upper-case
letters, and can be thought of as a large integer in base 26.  White
space\index{white space} inside a {\em compressed-proof} is
optional and is ignored.

It is important to note that the order of the mandatory hypotheses of
the statement being proved must not be changed if the compressed proof
format is used, otherwise the proof will become incorrect.  The reason
for this is that the mandatory hypotheses are not mentioned explicitly
in the compressed proof in order to make the compression more efficient.
If you wish to change the order of mandatory hypotheses, you must first
convert the proof back to normal format using the \texttt{save proof
{\em statement} /normal}\index{\texttt{save proof} command} command.
Later, you can go back to compressed format with \texttt{save proof {\em
statement} /compressed}.

During error checking with the \texttt{verify proof} command, an error
found in a compressed proof may point to a character in {\em
compressed-proof}, which may not be very meaningful to you.  In this
case, try to \texttt{save proof /normal} first, then do the
\texttt{verify proof} again.  In general, it is best to make sure a
proof is correct before saving it in compressed format, because severe
errors are less likely to be recoverable than in normal format.

\subsection{Specifying Unknown Proofs or Subproofs}\label{unknown}

In a proof under development, any step or subproof that is not yet known
may be represented with a single \texttt{?}.  For the purposes of
parsing the proof, the \texttt{?}\ \index{\texttt{]}@\texttt{?}\ inside
proofs} will push a single entry onto the RPN stack just as if it were a
hypothesis.  While developing a proof with the Proof
Assistant\index{Proof Assistant}, a partially developed proof may be
saved with the \texttt{save new{\char`\_}proof}\index{\texttt{save
new{\char`\_}proof} command} command, and \texttt{?}'s will be placed at
the appropriate places.

All \texttt{\$p}\index{\texttt{\$p} statement} statements must have
proofs, even if they are entirely unknown.  Before creating a proof with
the Proof Assistant, you should specify a completely unknown proof as
follows:
\begin{center}
  {\em label} \texttt{\$p} {\em statement} \texttt{\$= ?\ \$.}
\end{center}
\index{\texttt{\$=} keyword}
\index{\texttt{]}@\texttt{?}\ inside proofs}

The \texttt{verify proof}\index{\texttt{verify proof} command} command
will check the known portions of a partial proof for errors, but will
warn you that the statement has not been proved.

Note that partially developed proofs may be saved in compressed format
if desired.  In this case, you will see one or more \texttt{?}'s in the
{\em compressed-proof} part.\index{compressed
proof}\index{proof!compressed}

\section{Axioms vs.\ Definitions}\label{definitions}

The \textit{basic}
Metamath\index{Metamath} language and program
make no distinction\index{axiom vs.\
definition} between axioms\index{axiom} and
definitions.\index{definition} The \texttt{\$a}\index{\texttt{\$a}
statement} statement is used for both.  At first, this may seem
puzzling.  In the minds of many mathematicians, the distinction is
clear, even obvious, and hardly worth discussing.  A definition is
considered to be merely an abbreviation that can be replaced by the
expression for which it stands; although unless one actually does this,
to be precise then one should say that a theorem\index{theorem} is a
consequence of the axioms {\em and} the definitions that are used in the
formulation of the theorem \cite[p.~20]{Behnke}.\index{Behnke, H.}

\subsection{What is a Definition?}

What is a definition?  In its simplest form, a definition introduces a new
symbol and provides an unambiguous rule to transform an expression containing
the new symbol to one without it.  The concept of a ``proper
definition''\index{proper definition}\index{definition!proper} (as opposed to
a creative definition)\index{creative definition}\index{definition!creative}
that is usually agreed upon is (1) the definition should not strengthen the
language and (2) any symbols introduced by the definition should be eliminable
from the language \cite{Nemesszeghy}\index{Nemesszeghy, E. Z.}.  In other
words, they are mere typographical conveniences that do not belong to the
system and are theoretically superfluous.  This may seem obvious, but in fact
the nature of definitions can be subtle, sometimes requiring difficult
metatheorems to establish that they are not creative.

A more conservative stance was taken by logician S.
Le\'{s}niewski.\index{Le\'{s}niewski, S.}
\begin{quote}
Le\'{s}niewski
regards definitions as theses of the system.  In this respect they do
not differ either from the axioms or from theorems, i.e.\ from the
theses added to the system on the basis of the rule of substitution or
the rule of detachment [modus ponens].  Once definitions have been
accepted as theses of the system, it becomes necessary to consider them
as true propositions in the same sense in which axioms are true
\cite{Lejewski}.
\end{quote}\index{Lejewski, Czeslaw}

Let us look at some simple examples of definitions in propositional
calculus.  Consider the definition of logical {\sc or}
(disjunction):\index{disjunction ($\vee$)} ``$P\vee Q$ denotes $\neg P
\rightarrow Q$ (not $P$ implies $Q$).''  It is very easy to recognize a
statement making use of this definition, because it introduces the new
symbol $\vee$ that did not previously exist in the language.  It is easy
to see that no new theorems of the original language will result from
this definition.

Next, consider a definition that eliminates parentheses:  ``$P
\rightarrow Q\rightarrow R$ denotes $P\rightarrow (Q \rightarrow R)$.''
This is more subtle, because no new symbols are introduced.  The reason
this definition is considered proper is that no new symbol sequences
that are valid wffs (well-formed formulas)\index{well-formed formula
(wff)} in the original language will result from the definition, since
``$P \rightarrow Q\rightarrow R$'' is not a wff in the original
language.  Here, we implicitly make use of the fact that there is a
decision procedure that allows us to determine whether or not a symbol
sequence is a wff, and this fact allows us to use symbol sequences that
are not wffs to represent other things (such as wffs) by means of the
definition.  However, to justify the definition as not being creative we
need to prove that ``$P \rightarrow Q\rightarrow R$'' is in fact not a
wff in the original language, and this is more difficult than in the
case where we simply introduce a new symbol.

%Now let's take this reasoning to an extreme.  Propositional calculus is a
%decidable theory,\footnote{This means that a mechanical algorithm exists to
%determine whether or not a wff is a theorem.} so in principle we could make use
%of symbol sequences that are not theorems to represent other things (say, to
%encode actual theorems in a more compact way).  For example, let us extend the
%language by defining a wff ``$P$'' in the extended language as the theorem
%``$P\rightarrow P$''\footnote{This is one of the first theorems proved in the
%Metamath database \texttt{set.mm}.}\index{set
%theory database (\texttt{set.mm})} in the original language whenever ``$P$'' is
%not a theorem in the original language.  In the extended language, any wff
%``$Q$'' thus represents a theorem; to find out what theorem (in the original
%language) ``$Q$'' represents, we determine whether ``$Q$'' is a theorem in the
%original language (before the definition was introduced).  If so, we're done; if
%not, we replace ``$Q$'' by ``$Q\rightarrow Q$'' to eliminate the definition.
%This definition is therefore eliminable, and it does not ``strengthen'' the
%language because any wff that is not a theorem is not in the set of statements
%provable in the original language and thus is available for use by definitions.
%
%Of course, a definition such as this would render practically useless the
%communication of theorems of propositional calculus; but
%this is just a human shortcoming, since we can't always easily discern what is
%and is not a theorem by inspection.  In fact, the extended theory with this
%definition has no more and no less information than the original theory; it just
%expresses certain theorems of the form ``$P\rightarrow P$''
%in a more compact way.
%
%The point here is that what constitutes a proper definition is a matter of
%judgment about whether a symbol sequence can easily be recognized by a human
%as invalid in some sense (for example, not a wff); if so, the symbol sequence
%can be appropriated for use by a definition in order to make the extended
%language more compact.  Metamath\index{Metamath} lacks the ability to make this
%judgment, since as far as Metamath is concerned the definition of a wff, for
%example, is arbitrary.  You define for Metamath how wffs\index{well-formed
%formula (wff)} are constructed according to your own preferred style.  The
%concept of a wff may not even exist in a given formal system\index{formal
%system}.  Metamath treats all definitions as if they were new axioms, and it
%is up to the human mathematician to judge whether the definition is ``proper''
%'\index{proper definition}\index{definition!proper} in some agreed-upon way.

What constitutes a definition\index{definition} versus\index{axiom vs.\
definition} an axiom\index{axiom} is sometimes arbitrary in mathematical
literature.  For example, the connectives $\vee$ ({\sc or}), $\wedge$
({\sc and}), and $\leftrightarrow$ (equivalent to) in propositional
calculus are usually considered defined symbols that can be used as
abbreviations for expressions containing the ``primitive'' connectives
$\rightarrow$ and $\neg$.  This is the way we treat them in the standard
logic and set theory database \texttt{set.mm}\index{set theory database
(\texttt{set.mm})}.  However, the first three connectives can also be
considered ``primitive,'' and axiom systems have been devised that treat
all of them as such.  For example,
\cite[p.~35]{Goodstein}\index{Goodstein, R. L.} presents one with 15
axioms, some of which in fact coincide with what we have chosen to call
definitions in \texttt{set.mm}.  In certain subsets of classical
propositional calculus, such as the intuitionist
fragment\index{intuitionism}, it can be shown that one cannot make do
with just $\rightarrow$ and $\neg$ but must treat additional connectives
as primitive in order for the system to make sense.\footnote{Two nice
systems that make the transition from intuitionistic and other weak
fragments to classical logic just by adding axioms are given in
\cite{Robinsont}\index{Robinson, T. Thacher}.}

\subsection{The Approach to Definitions in \texttt{set.mm}}

In set theory, recursive definitions define a newly introduced symbol in
terms of itself.
The justification of recursive definitions, using
several ``recursion theorems,'' is usually one of the first
sophisticated proofs a student encounters when learning set theory, and
there is a significant amount of implicit metalogic behind a recursive
definition even though the definition itself is typically simple to
state.

Metamath itself has no built-in technical limitation that prevents
multiple-part recursive definitions in the traditional textbook style.
However, because the recursive definition requires advanced metalogic
to justify, eliminating a recursive definition is very difficult and
often not even shown in textbooks.

\subsubsection{Direct definitions instead of recursive definitions}

It is, however, possible to substitute one kind of complexity
for another.  We can eliminate the need for metalogical justification by
defining the operation directly with an explicit (but complicated)
expression, then deriving the recursive definition directly as a
theorem, using a recursion theorem ``in reverse.''
The elimination
of a direct definition is a matter of simple mechanical substitution.
We do this in
\texttt{set.mm}, as follows.

In \texttt{set.mm} our goal was to introduce almost all definitions in
the form of two expressions connected by either $\leftrightarrow$ or
$=$, where the thing being defined does not appear on the right hand
side.  Quine calls this form ``a genuine or direct definition'' \cite[p.
174]{Quine}\index{Quine, Willard Van Orman}, which makes the definitions
very easy to eliminate and the metalogic\index{metalogic} needed to
justify them as simple as possible.
Put another way, we had a goal of being able to
eliminate all definitions with direct mechanical substitution and to
verify easily the soundness of the definitions.

\subsubsection{Example of direct definitions}

We achieved this goal in almost all cases in \texttt{set.mm}.
Sometimes this makes the definitions more complex and less
intuitive.
For example, the traditional way to define addition of
natural numbers is to define an operation called {\em
successor}\index{successor} (which means ``plus one'' and is denoted by
``${\rm suc}$''), then define addition recursively\index{recursive
definition} with the two definitions $n + 0 = n$ and $m + {\rm suc}\,n =
{\rm suc} (m + n)$.  Although this definition seems simple and obvious,
the method to eliminate the definition is not obvious:  in the second
part of the definition, addition is defined in terms of itself.  By
eliminating the definition, we don't mean repeatedly applying it to
specific $m$ and $n$ but rather showing the explicit, closed-form
set-theoretical expression that $m + n$ represents, that will work for
any $m$ and $n$ and that does not have a $+$ sign on its right-hand
side.  For a recursive definition like this not to be circular
(creative), there are some hidden, underlying assumptions we must make,
for example that the natural numbers have a certain kind of order.

In \texttt{set.mm} we chose to start with the direct (though complex and
nonintuitive) definition then derive from it the standard recursive
definition.
For example, the closed-form definition used in \texttt{set.mm}
for the addition operation on ordinals\index{ordinal
addition}\index{addition!of ordinals} (of which natural numbers are a
subset) is

\setbox\startprefix=\hbox{\tt \ \ df-oadd\ \$a\ }
\setbox\contprefix=\hbox{\tt \ \ \ \ \ \ \ \ \ \ \ \ \ }
\startm
\m{\vdash}\m{+_o}\m{=}\m{(}\m{x}\m{\in}\m{{\rm On}}\m{,}\m{y}\m{\in}\m{{\rm
On}}\m{\mapsto}\m{(}\m{{\rm rec}}\m{(}\m{(}\m{z}\m{\in}\m{{\rm
V}}\m{\mapsto}\m{{\rm suc}}\m{z}\m{)}\m{,}\m{x}\m{)}\m{`}\m{y}\m{)}\m{)}
\endm
\noindent which depends on ${\rm rec}$.

\subsubsection{Recursion operators}

The above definition of \texttt{df-oadd} depends on the definition of
${\rm rec}$, a ``recursion operator''\index{recursion operator} with
the definition \texttt{df-rdg}:

\setbox\startprefix=\hbox{\tt \ \ df-rdg\ \$a\ }
\setbox\contprefix=\hbox{\tt \ \ \ \ \ \ \ \ \ \ \ \ }
\startm
\m{\vdash}\m{{\rm
rec}}\m{(}\m{F}\m{,}\m{I}\m{)}\m{=}\m{\mathrm{recs}}\m{(}\m{(}\m{g}\m{\in}\m{{\rm
V}}\m{\mapsto}\m{{\rm if}}\m{(}\m{g}\m{=}\m{\varnothing}\m{,}\m{I}\m{,}\m{{\rm
if}}\m{(}\m{{\rm Lim}}\m{{\rm dom}}\m{g}\m{,}\m{\bigcup}\m{{\rm
ran}}\m{g}\m{,}\m{(}\m{F}\m{`}\m{(}\m{g}\m{`}\m{\bigcup}\m{{\rm
dom}}\m{g}\m{)}\m{)}\m{)}\m{)}\m{)}\m{)}
\endm

\noindent which can be further broken down with definitions shown in
Section~\ref{setdefinitions}.

This definition of ${\rm rec}$
defines a recursive definition generator on ${\rm On}$ (the class of ordinal
numbers) with characteristic function $F$ and initial value $I$.
This operation allows us to define, with
compact direct definitions, functions that are usually defined in
textbooks with recursive definitions.
The price paid with our approach
is the complexity of our ${\rm rec}$ operation
(especially when {\tt df-recs} that it is built on is also eliminated).
But once we get past this hurdle, definitions that would otherwise be
recursive become relatively simple, as in for example {\tt oav}, from
which we prove the recursive textbook definition as theorems {\tt oa0}, {\tt
oasuc}, and {\tt oalim} (with the help of theorems {\tt rdg0}, {\tt rdgsuc},
and {\tt rdglim2a}).  We can also restrict the ${\rm rec}$ operation to
define otherwise recursive functions on the natural numbers $\omega$; see {\tt
fr0g} and {\tt frsuc}.  Our ${\rm rec}$ operation apparently does not appear
in published literature, although closely related is Definition 25.2 of
[Quine] p. 177, which he uses to ``turn...a recursion into a genuine or
direct definition" (p. 174).  Note that the ${\rm if}$ operations (see
{\tt df-if}) select cases based on whether the domain of $g$ is zero, a
successor, or a limit ordinal.

An important use of this definition ${\rm rec}$ is in the recursive sequence
generator {\tt df-seq} on the natural numbers (as a subset of the
complex infinite sequences such as the factorial function {\tt df-fac} and
integer powers {\tt df-exp}).

The definition of ${\rm rec}$ depends on ${\rm recs}$.
New direct usage of the more powerful (and more primitive) ${\rm recs}$
construct is discouraged, but it is available when needed.
This
defines a function $\mathrm{recs} ( F )$ on ${\rm On}$, the class of ordinal
numbers, by transfinite recursion given a rule $F$ which sets the next
value given all values so far.
Unlike {\tt df-rdg} which restricts the
update rule to use only the previous value, this version allows the
update rule to use all previous values, which is why it is described
as ``strong,'' although it is actually more primitive.  See {\tt
recsfnon} and {\tt recsval} for the primary contract of this definition.
It is defined as:

\setbox\startprefix=\hbox{\tt \ \ df-recs\ \$a\ }
\setbox\contprefix=\hbox{\tt \ \ \ \ \ \ \ \ \ \ \ \ \ }
\startm
\m{\vdash}\m{\mathrm{recs}}\m{(}\m{F}\m{)}\m{=}\m{\bigcup}\m{\{}\m{f}\m{|}\m{\exists}\m{x}\m{\in}\m{{\rm
On}}\m{(}\m{f}\m{{\rm
Fn}}\m{x}\m{\wedge}\m{\forall}\m{y}\m{\in}\m{x}\m{(}\m{f}\m{`}\m{y}\m{)}\m{=}\m{(}\m{F}\m{`}\m{(}\m{f}\m{\restriction}\m{y}\m{)}\m{)}\m{)}\m{\}}
\endm

\subsubsection{Closing comments on direct definitions}

From these direct definitions the simpler, more
intuitive recursive definition is derived as a set of theorems.\index{natural
number}\index{addition}\index{recursive definition}\index{ordinal addition}
The end result is the same, but we completely eliminate the rather complex
metalogic that justifies the recursive definition.

Recursive definitions are often considered more efficient and intuitive than
direct ones once the metalogic has been learned or possibly just accepted as
correct.  However, it was felt that direct definition in \texttt{set.mm}
maximizes rigor by minimizing metalogic.  It can be eliminated effortlessly,
something that is difficult to do with a recursive definition.

Again, Metamath itself has no built-in technical limitation that prevents
multiple-part recursive definitions in the traditional textbook style.
Instead, our goal is to eliminate all definitions with
direct mechanical substitution and to verify easily the soundness of
definitions.

\subsection{Adding Constraints on Definitions}

The basic Metamath language and the Metamath program do
not have any built-in constraints on definitions, since they are just
\$a statements.

However, nothing prevents a verification system from verifying
additional rules to impose further limitations on definitions.
For example, the \texttt{mmj2}\index{mmj2} program
supports various kinds of
additional information comments (see section \ref{jcomment}).
One of their uses is to optionally verify additional constraints,
including constraints to verify that definitions meet certain
requirements.
These additional checks are required by the
continuous integration (CI)\index{continuous integration (CI)}
checks of the
\texttt{set.mm}\index{set theory database (\texttt{set.mm})}%
\index{Metamath Proof Explorer}
database.
This approach enables us to optionally impose additional requirements
on definitions if we wish, without necessarily imposing those rules on
all databases or requiring all verification systems to implement them.
In addition, this allows us to impose specialized constraints tailored
to one database while not requiring other systems to implement
those specialized constraints.

We impose two constraints on the
\texttt{set.mm}\index{set theory database (\texttt{set.mm})}%
\index{Metamath Proof Explorer} database
via the \texttt{mmj2}\index{mmj2} program that are worth discussing here:
a parse check and a definition soundness check.

% On February 11, 2019 8:32:32 PM EST, saueran@oregonstate.edu wrote:
% The following addition to the end of set.mm is accepted by the mmj2
% parser and definition checker and the metamath verifier(at least it was
% when I checked, you should check it too), and creates a contradiction by
% proving the theorem |- ph.
% ${
% wleftp $a wff ( ( ph ) $.
% wbothp $a wff ( ph ) $.
% df-leftp $a |- ( ( ( ph ) <-> -. ph ) $.
% df-bothp $a |- ( ( ph ) <-> ph ) $.
% anything $p |- ph $=
%   ( wbothp wn wi wleftp df-leftp biimpi df-bothp mpbir mpbi simplim ax-mp)
%   ABZAMACZDZCZMOEZOCQAEZNDZRNAFGSHIOFJMNKLAHJ $.
% $}
%
% This particular problem is countered by enabling, within mmj2,
% SetParser,mmj.verify.LRParser

First,
we enable a parse check in \texttt{mmj2} (through its
\texttt{SetParser} command) that requires that all new definitions
must \textit{not} create an ambiguous parse for a KLR(5) parser.
This prevents some errors such as definitions with imbalanced parentheses.

Second, we run a definition soundness check specific to
\texttt{set.mm} or databases similar to it.
(through the \texttt{definitionCheck} macro).
Some \texttt{\$a} statements (including all ax-* statemnets)
are excluded from these checks, as they will
always fail this simple check,
but they are appropriate for most definitions.
This check imposes a set of additional rules:

\begin{enumerate}

\item New definitions must be introduced using $=$ or $\leftrightarrow$.

\item No \texttt{\$a} statement introduced before this one is allowed to use the
symbol being defined in this definition, and the definition is not
allowed to use itself (except once, in the definiendum).

\item Every variable in the definiens should not be distinct

\item Every dummy variable in the definiendum
are required to be distinct from each other and from variables in
the definiendum.
To determine this, the system will look for a "justification" theorem
in the database, and if it is not there, attempt to briefly prove
$( \varphi \rightarrow \forall x \varphi )$  for each dummy variable x.

\item Every dummy variable should be a set variable,
unless there is a justification theorem available.

\item Every dummy variable must be bound
(if the system cannot determine this a justification theorem must be
provided).

\end{enumerate}

\subsection{Summary of Approach to Definitions}

In short, when being rigorous it turns out that
definitions can be subtle, sometimes requiring difficult
metatheorems to establish that they are not creative.

Instead of building such complications into the Metamath language itself,
the basic Metmath language and program simply treat traditional
axioms and definitions as the same kind of \texttt{\$a} statement.
We have then built various tools to enable people to
verify additional conditions as their creators believe is appropriate
for those specific databases, without complicating the Metamath foundations.

\chapter{The Metamath Program}\label{commands}

This chapter provides a reference manual for the
Metamath program.\index{Metamath!commands}

Current instructions for obtaining and installing the Metamath program
can be found at the \url{http://metamath.org} web site.
For Windows, there is a pre-compiled version called
\texttt{metamath.exe}.  For Unix, Linux, and Mac OS X
(which we will refer to collectively as ``Unix''), the Metamath program
can be compiled from its source code with the command
\begin{verbatim}
gcc *.c -o metamath
\end{verbatim}
using the \texttt{gcc} {\sc c} compiler available on those systems.

In the command syntax descriptions below, fields enclosed in square
brackets [\ ] are optional.  File names may be optionally enclosed in
single or double quotes.  This is useful if the file name contains
spaces or
slashes (\texttt{/}), such as in Unix path names, \index{Unix file
names}\index{file names!Unix} that might be confused with Metamath
command qualifiers.\index{Unix file names}\index{file names!Unix}

\section{Invoking Metamath}

Unix, Linux, and Mac OS X
have a command-line interface called the {\em
bash shell}.  (In Mac OS X, select the Terminal application from
Applications/Utilities.)  To invoke Metamath from the bash shell prompt,
assuming that the Metamath program is in the current directory, type
\begin{verbatim}
bash$ ./metamath
\end{verbatim}

To invoke Metamath from a Windows DOS or Command Prompt, assuming that
the Metamath program is in the current directory (or in a directory
included in the Path system environment variable), type
\begin{verbatim}
C:\metamath>metamath
\end{verbatim}

To use command-line arguments at invocation, the command-line arguments
should be a list of Metamath commands, surrounded by quotes if they
contain spaces.  In Windows, the surrounding quotes must be double (not
single) quotes.  For example, to read the database file \texttt{set.mm},
verify all proofs, and exit the program, type (under Unix)
\begin{verbatim}
bash$ ./metamath 'read set.mm' 'verify proof *' exit
\end{verbatim}
Note that in Unix, any directory path with \texttt{/}'s must be
surrounded by quotes so Metamath will not interpret the \texttt{/} as a
command qualifier.  So if \texttt{set.mm} is in the \texttt{/tmp}
directory, use for the above example
\begin{verbatim}
bash$ ./metamath 'read "/tmp/set.mm"' 'verify proof *' exit
\end{verbatim}

For convenience, if the command-line has one argument and no spaces in
the argument, the command is implicitly assumed to be \texttt{read}.  In
this one special case, \texttt{/}'s are not interpreted as command
qualifiers, so you don't need quotes around a Unix file name.  Thus
\begin{verbatim}
bash$ ./metamath /tmp/set.mm
\end{verbatim}
and
\begin{verbatim}
bash$ ./metamath "read '/tmp/set.mm'"
\end{verbatim}
are equivalent.


\section{Controlling Metamath}

The Metamath program was first developed on a {\sc vax/vms} system, and
some aspects of its command line behavior reflect this heritage.
Hopefully you will find it reasonably user-friendly once you get used to
it.

Each command line is a sequence of English-like words separated by
spaces, as in \texttt{show settings}.  Command words are not case
sensitive, and only as many letters are needed as are necessary to
eliminate ambiguity; for example, \texttt{sh se} would work for the
command \texttt{show settings}.  In some cases arguments such as file
names, statement labels, or symbol names are required; these are
case-sensitive (although file names may not be on some operating
systems).

A command line is entered by typing it in then pressing the {\em return}
({\em enter}) key.  To find out what commands are available, type
\texttt{?} at the \texttt{MM>} prompt.  To find out the choices at any
point in a command, press {\em return} and you will be prompted for
them.  The default choice (the one selected if you just press {\em
return}) is shown in brackets (\texttt{<>}).

You may also type \texttt{?} in place of a command word to force
Metamath to tell you what the choices are.  The \texttt{?} method won't
work, though, if a non-keyword argument such as a file name is expected
at that point, because the program will think that \texttt{?} is the
value of the argument.

Some commands have one or more optional qualifiers which modify the
behavior of the command.  Qualifiers are preceded by a slash
(\texttt{/}), such as in \texttt{read set.mm / verify}.  Spaces are
optional around the \texttt{/}.  If you need to use a space or
slash in a command
argument, as in a Unix file name, put single or double quotes around the
command argument.

The \texttt{open log} command will save everything you see on the
screen and is useful to help you recover should something go wrong in a
proof, or if you want to document a bug.

If a command responds with more than a screenful, you will be
prompted to \texttt{<return> to continue, Q to quit, or S to scroll to
end}.  \texttt{Q} or \texttt{q} (not case-sensitive) will complete the
command internally but will suppress further output until the next
\texttt{MM>} prompt.  \texttt{s} will suppress further pausing until the
next \texttt{MM>} prompt.  After the first screen, you are also
presented with the choice of \texttt{b} to go back a screenful.  Note
that \texttt{b} may also be entered at the \texttt{MM>} prompt
immediately after a command to scroll back through the output of that
command.

A command line enclosed in quotes is executed by your operating system.
See Section~\ref{oscmd}.

{\em Warning:} Pressing {\sc ctrl-c} will abort the Metamath program
unconditionally.  This means any unsaved work will be lost.


\subsection{\texttt{exit} Command}\index{\texttt{exit} command}

Syntax:  \texttt{exit} [\texttt{/force}]

This command exits from Metamath.  If there have been changes to the
source with the \texttt{save proof} or \texttt{save new{\char`\_}proof}
commands, you will be given an opportunity to \texttt{write source} to
permanently save the changes.

In Proof Assistant\index{Proof Assistant} mode, the \texttt{exit} command will
return to the \verb/MM>/ prompt. If there were changes to the proof, you will
be given an opportunity to \texttt{save new{\char`\_}proof}.

The \texttt{quit} command is a synonym for \texttt{exit}.

Optional qualifier:
    \texttt{/force} - Do not prompt if changes were not saved.  This qualifier is
        useful in \texttt{submit} command files (Section~\ref{sbmt})
        to ensure predictable behavior.





\subsection{\texttt{open log} Command}\index{\texttt{open log} command}
Syntax:  \texttt{open log} {\em file-name}

This command will open a log file that will store everything you see on
the screen.  It is useful to help recovery from a mistake in a long Proof
Assistant session, or to document bugs.\index{Metamath!bugs}

The log file can be closed with \texttt{close log}.  It will automatically be
closed upon exiting Metamath.



\subsection{\texttt{close log} Command}\index{\texttt{close log} command}
Syntax:  \texttt{close log}

The \texttt{close log} command closes a log file if one is open.  See
also \texttt{open log}.




\subsection{\texttt{submit} Command}\index{\texttt{submit} command}\label{sbmt}
Syntax:  \texttt{submit} {\em filename}

This command causes further command lines to be taken from the specified
file.  Note that any line beginning with an exclamation point (\texttt{!}) is
treated as a comment (i.e.\ ignored).  Also note that the scrolling
of the screen output is continuous, so you may want to open a log file
(see \texttt{open log}) to record the results that fly by on the screen.
After the lines in the file are exhausted, Metamath returns to its
normal user interface mode.

The \texttt{submit} command can be called recursively (i.e. from inside
of a \texttt{submit} command file).


Optional command qualifier:

    \texttt{/silent} - suppresses the screen output but still
        records the output in a log file if one is open.


\subsection{\texttt{erase} Command}\index{\texttt{erase} command}
Syntax:  \texttt{erase}

This command will reset Metamath to its starting state, deleting any
data\-base that was \texttt{read} in.
 If there have been changes to the
source with the \texttt{save proof} or \texttt{save new{\char`\_}proof}
commands, you will be given an opportunity to \texttt{write source} to
permanently save the changes.



\subsection{\texttt{set echo} Command}\index{\texttt{set echo} command}
Syntax:  \texttt{set echo on} or \texttt{set echo off}

The \texttt{set echo on} command will cause command lines to be echoed with any
abbreviations expanded.  While learning the Metamath commands, this
feature will show you the exact command that your abbreviated input
corresponds to.



\subsection{\texttt{set scroll} Command}\index{\texttt{set scroll} command}
Syntax:  \texttt{set scroll prompted} or \texttt{set scroll continuous}

The Metamath command line interface starts off in the \texttt{prompted} mode,
which means that you will be prompted to continue or quit after each
full screen in a long listing.  In \texttt{continuous} mode, long listings will be
scrolled without pausing.

% LaTeX bug? (1) \texttt{\_} puts out different character than
% \texttt{{\char`\_}}
%  = \verb$_$  (2) \texttt{{\char`\_}} puts out garbage in \subsection
%  argument
\subsection{\texttt{set width} Command}\index{\texttt{set
width} command}
Syntax:  \texttt{set width} {\em number}

Metamath assumes the width of your screen is 79 characters (chosen
because the Command Prompt in Windows XP has a wrapping bug at column
80).  If your screen is wider or narrower, this command allows you to
change this default screen width.  A larger width is advantageous for
logging proofs to an output file to be printed on a wide printer.  A
smaller width may be necessary on some terminals; in this case, the
wrapping of the information messages may sometimes seem somewhat
unnatural, however.  In \LaTeX\index{latex@{\LaTeX}!characters per line},
there is normally a maximum of 61 characters per line with typewriter
font.  (The examples in this book were produced with 61 characters per
line.)

\subsection{\texttt{set height} Command}\index{\texttt{set
height} command}
Syntax:  \texttt{set height} {\em number}

Metamath assumes your screen height is 24 lines of characters.  If your
screen is taller or shorter, this command lets you to change the number
of lines at which the display pauses and prompts you to continue.

\subsection{\texttt{beep} Command}\index{\texttt{beep} command}

Syntax:  \texttt{beep}

This command will produce a beep.  By typing it ahead after a
long-running command has started, it will alert you that the command is
finished.  For convenience, \texttt{b} is an abbreviation for
\texttt{beep}.

Note:  If \texttt{b} is typed at the \texttt{MM>} prompt immediately
after the end of a multiple-page display paged with ``\texttt{Press
<return> for more}...'' prompts, then the \texttt{b} will back up to the
previous page rather than perform the \texttt{beep} command.
In that case you must type the unabbreviated \texttt{beep} form
of the command.

\subsection{\texttt{more} Command}\index{\texttt{more} command}

Syntax:  \texttt{more} {\em filename}

This command will display the contents of an {\sc ascii} file on your
screen.  (This command is provided for convenience but is not very
powerful.  See Section~\ref{oscmd} to invoke your operating system's
command to do this, such as the \texttt{more} command in Unix.)

\subsection{Operating System Commands}\index{operating system
command}\label{oscmd}

A line enclosed in single or double quotes will be executed by your
computer's operating system if it has a command line interface.  For
example, on a {\sc vax/vms} system,
\verb/MM> 'dir'/
will print disk directory contents.  Note that this feature will not
work on the Macintosh prior to Mac OS X, which does not have a command
line interface.

For your convenience, the trailing quote is optional.

\subsection{Size Limitations in Metamath}

In general, there are no fixed, predefined limits\index{Metamath!memory
limits} on how many labels, tokens\index{token}, statements, etc.\ that
you may have in a database file.  The Metamath program uses 32-bit
variables (64-bit on 64-bit CPUs) as indices for almost all internal
arrays, which are allocated dynamically as needed.



\section{Reading and Writing Files}

The following commands create new files:  the \texttt{open} commands;
the \texttt{write} commands; the \texttt{/html},
\texttt{/alt{\char`\_}html}, \texttt{/brief{\char`\_}html},
\texttt{/brief{\char`\_}alt{\char`\_}html} qualifiers of \texttt{show
statement}, and \texttt{midi}.  The following commands append to files
previously opened:  the \texttt{/tex} qualifier of \texttt{show proof}
and \texttt{show new{\char`\_}proof}; the \texttt{/tex} and
\texttt{/simple{\char`\_}tex} qualifiers of \texttt{show statement}; the
\texttt{close} commands; and all screen dialog between \texttt{open log}
and \texttt{close log}.

The commands that create new files will not overwrite an existing {\em
filename} but will rename the existing one to {\em
filename}\texttt{{\char`\~}1}.  An existing {\em
filename}\texttt{{\char`\~}1} is renamed {\em
filename}\texttt{{\char`\~}2}, etc.\ up to {\em
filename}\texttt{{\char`\~}9}.  An existing {\em
filename}\texttt{{\char`\~}9} is deleted.  This makes recovery from
mistakes easier but also will clutter up your directory, so occasionally
you may want to clean up (delete) these \texttt{{\char`\~}}$n$ files.


\subsection{\texttt{read} Command}\index{\texttt{read} command}
Syntax:  \texttt{read} {\em file-name} [\texttt{/verify}]

This command will read in a Metamath language source file and any included
files.  Normally it will be the first thing you do when entering Metamath.
Statement syntax is checked, but proof syntax is not checked.
Note that the file name may be enclosed in single or double quotes;
this is useful if the file name contains slashes, as might be the case
under Unix.

If you are getting an ``\texttt{?Expected VERIFY}'' error
when trying to read a Unix file name with slashes, you probably haven't
quoted it.\index{Unix file names}\index{file names!Unix}

If you are prompted for the file name (by pressing {\em return}
 after \texttt{read})
you should {\em not} put quotes around it, even if it is a Unix file name
with slashes.

Optional command qualifier:

    \texttt{/verify} - Verify all proofs as the database is read in.  This
         qualifier will slow down reading in the file.  See \texttt{verify
         proof} for more information on file error-checking.

See also \texttt{erase}.



\subsection{\texttt{write source} Command}\index{\texttt{write source} command}
Syntax:  \texttt{write source} {\em filename}
[\texttt{/rewrap}]
[\texttt{/split}]
% TeX doesn't handle this long line with tt text very well,
% so force a line break here.
[\texttt{/keep\_includes}] {\\}
[\texttt{/no\_versioning}]

This command will write the contents of a Metamath\index{database}
database into a file.\index{source file}

Optional command qualifiers:

\texttt{/rewrap} -
Reformats statements and comments according to the
convention used in the set.mm database.
It unwraps the
lines in the comment before each \texttt{\$a} and \texttt{\$p} statement,
then it
rewraps the line.  You should compare the output to the original
to make sure that the desired effect results; if not, go back to
the original source.  The wrapped line length honors the
\texttt{set width}
parameter currently in effect.  Note:  Text
enclosed in \texttt{<HTML>}...\texttt{</HTML>} tags is not modified by the
\texttt{/rewrap} qualifier.
Proofs themselves are not reformatted;
use \texttt{save proof * / compressed} to do that.
An isolated tilde (\~{}) is always kept on the same line as the following
symbol, so you can find all comment references to a symbol by
searching for \~{} followed by a space and that symbol
(this is useful for finding cross-references).
Incidentally, \texttt{save proof} also honors the \texttt{set width}
parameter currently in effect.

\texttt{/split} - Files included in the source using the expression
\$[ \textit{inclfile} \$] will be
written into separate files instead of being included in a single output
file.  The name of each separately written included file will be the
\textit{inclfile} argument of its inclusion command.

\texttt{/keep\_includes} - If a source file has includes but is written as a
single file by omitting \texttt{/split}, by default the included files will
be deleted (actually just renamed with a \char`\~1 suffix unless
\texttt{/no\_versioning} is specified) to prevent the possibly confusing
source duplication in both the output file and the included file.
The \texttt{/keep\_includes} qualifier will prevent this deletion.

\texttt{/no\_versioning} - Backup files suffixed with \char`\~1 are not created.


\section{Showing Status and Statements}



\subsection{\texttt{show settings} Command}\index{\texttt{show settings} command}
Syntax:  \texttt{show settings}

This command shows the state of various parameters.

\subsection{\texttt{show memory} Command}\index{\texttt{show memory} command}
Syntax:  \texttt{show memory}

This command shows the available memory left.  It is not meaningful
on most modern operating systems,
which have virtual memory.\index{Metamath!memory usage}


\subsection{\texttt{show labels} Command}\index{\texttt{show labels} command}
Syntax:  \texttt{show labels} {\em label-match} [\texttt{/all}]
   [\texttt{/linear}]

This command shows the labels of \texttt{\$a} and \texttt{\$p}
statements that match {\em label-match}.  A \verb$*$ in {label-match}
matches zero or more characters.  For example, \verb$*abc*def$ will match all
labels containing \verb$abc$ and ending with \verb$def$.

Optional command qualifiers:

   \texttt{/all} - Include matches for \texttt{\$e} and \texttt{\$f}
   statement labels.

   \texttt{/linear} - Display only one label per line.  This can be useful for
       building scripts in conjunction with the utilities under the
       \texttt{tools}\index{\texttt{tools} command} command.



\subsection{\texttt{show statement} Command}\index{\texttt{show statement} command}
Syntax:  \texttt{show statement} {\em label-match} [{\em qualifiers} (see below)]

This command provides information about a statement.  Only statements
that have labels (\texttt{\$f}\index{\texttt{\$f} statement},
\texttt{\$e}\index{\texttt{\$e} statement},
\texttt{\$a}\index{\texttt{\$a} statement}, and
\texttt{\$p}\index{\texttt{\$p} statement}) may be specified.
If {\em label-match}
contains wildcard (\verb$*$) characters, all matching statements will be
displayed in the order they occur in the database.

Optional qualifiers (only one qualifier at a time is allowed):

    \texttt{/comment} - This qualifier includes the comment that immediately
       precedes the statement.

    \texttt{/full} - Show complete information about each statement,
       and show all
       statements matching {\em label} (including \texttt{\$e}
       and \texttt{\$f} statements).

    \texttt{/tex} - This qualifier will write the statement information to the
       \LaTeX\ file previously opened with \texttt{open tex}.  See
       Section~\ref{texout}.

    \texttt{/simple{\char`\_}tex} - The same as \texttt{/tex}, except that
       \LaTeX\ macros are not used for formatting equations, allowing easier
       manual edits of the output for slide presentations, etc.

    \texttt{/html}\index{html generation@{\sc html} generation},
       \texttt{/alt{\char`\_}html}, \texttt{/brief{\char`\_}html},
       \texttt{/brief{\char`\_}alt{\char`\_}html} -
       These qualifiers invoke a special mode of
       \texttt{show statement} that
       creates a web page for the statement.  They may not be used with
       any other qualifier.  See Section~\ref{htmlout} or
       \texttt{help html} in the program.


\subsection{\texttt{search} Command}\index{\texttt{search} command}
Syntax:  search {\em label-match}
\texttt{"}{\em symbol-match}{\tt}" [\texttt{/all}] [\texttt{/comments}]
[\texttt{/join}]

This command searches all \texttt{\$a} and \texttt{\$p} statements
matching {\em label-match} for occurrences of {\em symbol-match}.  A
\verb@*@ in {\em label-match} matches any label character.  A \verb@$*@
in {\em symbol-match} matches any sequence of symbols.  The symbols in
{\em symbol-match} must be separated by white space.  The quotes
surrounding {\em symbol-match} may be single or double quotes.  For
example, \texttt{search b}\verb@* "-> $* ch"@ will list all statements
whose labels begin with \texttt{b} and contain the symbols \verb@->@ and
\texttt{ch} surrounding any symbol sequence (including no symbol
sequence).  The wildcards \texttt{?} and \texttt{\$?} are also available
to match individual characters in labels and symbols respectively; see
\texttt{help search} in the Metamath program for details on their usage.

Optional command qualifiers:

    \texttt{/all} - Also search \texttt{\$e} and \texttt{\$f} statements.

    \texttt{/comments} - Search the comment that immediately precedes each
        label-matched statement for {\em symbol-match}.  In this case
        {\em symbol-match} is an arbitrary, non-case-sensitive character
        string.  Quotes around {\em symbol-match} are optional if there
        is no ambiguity.

    \texttt{/join} - In the case of a \texttt{\$a} or \texttt{\$p} statement,
	prepend its \texttt{\$e}
	hypotheses for searching. The
	\texttt{/join} qualifier has no effect in \texttt{/comments} mode.

\section{Displaying and Verifying Proofs}


\subsection{\texttt{show proof} Command}\index{\texttt{show proof} command}
Syntax:  \texttt{show proof} {\em label-match} [{\em qualifiers} (see below)]

This command displays the proof of the specified
\texttt{\$p}\index{\texttt{\$p} statement} statement in various formats.
The {\em label-match} may contain wildcard (\verb@$*@) characters to match
multiple statements.  Without any qualifiers, only the logical steps
will be shown (i.e.\ syntax construction steps will be omitted), in an
indented format.

Most of the time, you will use
    \texttt{show proof} {\em label}
to see just the proof steps corresponding to logical inferences.

Optional command qualifiers:

    \texttt{/essential} - The proof tree is trimmed of all
        \texttt{\$f}\index{\texttt{\$f} statement} hypotheses before
        being displayed.  (This is the default, and it is redundant to
        specify it.)

    \texttt{/all} - the proof tree is not trimmed of all \texttt{\$f} hypotheses before
        being displayed.  \texttt{/essential} and \texttt{/all} are mutually exclusive.

    \texttt{/from{\char`\_}step} {\em step} - The display starts at the specified
        step.  If
        this qualifier is omitted, the display starts at the first step.

    \texttt{/to{\char`\_}step} {\em step} - The display ends at the specified
        step.  If this
        qualifier is omitted, the display ends at the last step.

    \texttt{/tree{\char`\_}depth} {\em number} - Only
         steps at less than the specified proof
        tree depth are displayed.  Sometimes useful for obtaining an overview of
        the proof.

    \texttt{/reverse} - The steps are displayed in reverse order.

    \texttt{/renumber} - When used with \texttt{/essential}, the steps are renumbered
        to correspond only to the essential steps.

    \texttt{/tex} - The proof is converted to \LaTeX\ and\index{latex@{\LaTeX}}
        stored in the file opened
        with \texttt{open tex}.  See Section~\ref{texout} or
        \texttt{help tex} in the program.

    \texttt{/lemmon} - The proof is displayed in a non-indented format known
        as Lemmon style, with explicit previous step number references.
        If this qualifier is omitted, steps are indented in a tree format.

    \texttt{/start{\char`\_}column} {\em number} - Overrides the default column
        (16)
        at which the formula display starts in a Lemmon-style display.  May be
        used only in conjunction with \texttt{/lemmon}.

    \texttt{/normal} - The proof is displayed in normal format suitable for
        inclusion in a Metamath source file.  May not be used with any other
        qualifier.

    \texttt{/compressed} - The proof is displayed in compressed format
        suitable for inclusion in a Metamath source file.  May not be used with
        any other qualifier.

    \texttt{/statement{\char`\_}summary} - Summarizes all statements (like a
        brief \texttt{show statement})
        used by the proof.  It may not be used with any other qualifier
        except \texttt{/essential}.

    \texttt{/detailed{\char`\_}step} {\em step} - Shows the details of what is
        happening at
        a specific proof step.  May not be used with any other qualifier.
        The {\em step} is the step number shown when displaying a
        proof without the \texttt{/renumber} qualifier.


\subsection{\texttt{show usage} Command}\index{\texttt{show usage} command}
Syntax:  \texttt{show usage} {\em label-match} [\texttt{/recursive}]

This command lists the statements whose proofs make direct reference to
the statement specified.

Optional command qualifier:

    \texttt{/recursive} - Also include statements whose proofs ultimately
        depend on the statement specified.



\subsection{\texttt{show trace\_back} Command}\index{\texttt{show
       trace{\char`\_}back} command}
Syntax:  \texttt{show trace{\char`\_}back} {\em label-match} [\texttt{/essential}] [\texttt{/axioms}]
    [\texttt{/tree}] [\texttt{/depth} {\em number}]

This command lists all statements that the proof of the \texttt{\$p}
statement(s) specified by {\em label-match} depends on.

Optional command qualifiers:

    \texttt{/essential} - Restrict the trace-back to \texttt{\$e}
        \index{\texttt{\$e} statement} hypotheses of proof trees.

    \texttt{/axioms} - List only the axioms that the proof ultimately depends on.

    \texttt{/tree} - Display the trace-back in an indented tree format.

    \texttt{/depth} {\em number} - Restrict the \texttt{/tree} trace-back to the
        specified indentation depth.

    \texttt{/count{\char`\_}steps} - Count the number of steps the proof has
       all the way back to axioms.  If \texttt{/essential} is specified,
       expansions of variable-type hypotheses (syntax constructions) are not counted.

\subsection{\texttt{verify proof} Command}\index{\texttt{verify proof} command}
Syntax:  \texttt{verify proof} {\em label-match} [\texttt{/syntax{\char`\_}only}]

This command verifies the proofs of the specified statements.  {\em
label-match} may contain wild card characters (\texttt{*}) to verify
more than one proof; for example \verb/*abc*def/ will match all labels
containing \texttt{abc} and ending with \texttt{def}.
The command \texttt{verify proof *} will verify all proofs in the database.

Optional command qualifier:

    \texttt{/syntax{\char`\_}only} - This qualifier will perform a check of syntax
        and RPN
        stack violations only.  It will not verify that the proof is
        correct.  This qualifier is useful for quickly determining which
        proofs are incomplete (i.e.\ are under development and have \texttt{?}'s
        in them).

{\em Note:} \texttt{read}, followed by \texttt{verify proof *}, will ensure
 the database is free
from errors in the Metamath language but will not check the markup notation
in comments.
You can also check the markup notation by running \texttt{verify markup *}
as discussed in Section~\ref{verifymarkup}; see also the discussion
on generating {\sc HTML} in Section~\ref{htmlout}.

\subsection{\texttt{verify markup} Command}\index{\texttt{verify markup} command}\label{verifymarkup}
Syntax:  \texttt{verify markup} {\em label-match}
[\texttt{/date{\char`\_}skip}]
[\texttt{/top{\char`\_}date{\char`\_}skip}] {\\}
[\texttt{/file{\char`\_}skip}]
[\texttt{/verbose}]

This command checks comment markup and other informal conventions we have
adopted.  It error-checks the latexdef, htmldef, and althtmldef statements
in the \texttt{\$t} statement of a Metamath source file.\index{error checking}
It error-checks any \texttt{`}...\texttt{`},
\texttt{\char`\~}~\textit{label},
and bibliographic markups in statement descriptions.
It checks that
\texttt{\$p} and \texttt{\$a} statements
have the same content when their labels start with
``ax'' and ``ax-'' respectively but are otherwise identical, for example
ax4 and ax-4.
It also verifies the date consistency of ``(Contributed by...),''
``(Revised by...),'' and ``(Proof shortened by...)'' tags in the comment
above each \texttt{\$a} and \texttt{\$p} statement.

Optional command qualifiers:

    \texttt{/date{\char`\_}skip} - This qualifier will
        skip date consistency checking,
        which is usually not required for databases other than
	\texttt{set.mm}.

    \texttt{/top{\char`\_}date{\char`\_}skip} - This qualifier will check date consistency except
        that the version date at the top of the database file will not
        be checked.  Only one of
        \texttt{/date{\char`\_}skip} and
        \texttt{/top{\char`\_}date{\char`\_}skip} may be
        specified.

    \texttt{/file{\char`\_}skip} - This qualifier will skip checks that require
        external files to be present, such as checking GIF existence and
        bibliographic links to mmset.html or equivalent.  It is useful
        for doing a quick check from a directory without these files.

    \texttt{/verbose} - Provides more information.  Currently it provides a list
        of axXXX vs. ax-XXX matches.

\subsection{\texttt{save proof} Command}\index{\texttt{save proof} command}
Syntax:  \texttt{save proof} {\em label-match} [\texttt{/normal}]
   [\texttt{/compressed}]

The \texttt{save proof} command will reformat a proof in one of two formats and
replace the existing proof in the source buffer\index{source
buffer}.  It is useful for
converting between proof formats.  Note that a proof will not be
permanently saved until a \texttt{write source} command is issued.

Optional command qualifiers:

    \texttt{/normal} - The proof is saved in the normal format (i.e., as a
        sequence
        of labels, which is the defined format of the basic Metamath
        language).\index{basic language}  This is the default format that
        is used if a qualifier
        is omitted.

    \texttt{/compressed} - The proof is saved in the compressed format which
        reduces storage requirements for a database.
        See Appendix~\ref{compressed}.




\section{Creating Proofs}\label{pfcommands}\index{Proof Assistant}

Before using the Proof Assistant, you must add a \texttt{\$p} to your
source file (using a text editor) containing the statement you want to
prove.  Its proof should consist of a single \texttt{?}, meaning
``unknown step.''  Example:
\begin{verbatim}
equid $p x = x $= ? $.
\end{verbatim}

To enter the Proof assistant, type \texttt{prove} {\em label}, e.g.
\texttt{prove equid}.  Metamath will respond with the \texttt{MM-PA>}
prompt.

Proofs are created working backwards from the statement being proved,
primarily using a series of \texttt{assign} commands.  A proof is
complete when all steps are assigned to statements and all steps are
unified and completely known.  During the creation of a proof, Metamath
will allow only operations that are legal based on what is known up to
that point.  For example, it will not allow an \texttt{assign} of a
statement that cannot be unified with the unknown proof step being
assigned.

{\em Important:}
The Proof Assistant is
{\em not} a tool to help you discover proofs.  It is just a tool to help
you add them to the database.  For a tutorial read
Section~\ref{frstprf}.
To practice using the Proof Assistant, you may
want to \texttt{prove} an existing theorem, then delete all steps with
\texttt{delete all}, then re-create it with the Proof Assistant while
looking at its proof display (before deletion).
You might want to figure out your first few proofs completely
and write them down by hand, before using the Proof Assistant, though
not everyone finds that effective.

{\em Important:}
The \texttt{undo} command if very helpful when entering a proof, because
it allows you to undo a previously-entered step.
In addition, we suggest that you
keep track of your work with a log file (\texttt{open
log}) and save it frequently (\texttt{save new{\char`\_}proof},
\texttt{write source}).
You can use \texttt{delete} to reverse an \texttt{assign}.
You can also do \texttt{delete floating{\char`\_}hypotheses}, then
\texttt{initialize all}, then \texttt{unify all /interactive} to
reinitialize bad unifications made accidentally or by bad
\texttt{assign}s.  You cannot reverse a \texttt{delete} except by
a relevant \texttt{undo} or using
\texttt{exit /force} then reentering the Proof Assistant to recover from
the last \texttt{save new{\char`\_}proof}.

The following commands are available in the Proof Assistant (at the
\texttt{MM-PA>} prompt) to help you create your proof.  See the
individual commands for more detail.

\begin{itemize}
\item[]
    \texttt{show new{\char`\_}proof} [\texttt{/all},...] - Displays the
        proof in progress.  You will use this command a lot; see \texttt{help
        show new{\char`\_}proof} to become familiar with its qualifiers.  The
        qualifiers \texttt{/unknown} and \texttt{/not{\char`\_}unified} are
        useful for seeing the work remaining to be done.  The combination
        \texttt{/all/unknown} is useful for identifying dummy variables that must be
        assigned, or attempts to use illegal syntax, when \texttt{improve all}
        is unable to complete the syntax constructions.  Unknown variables are
        shown as \texttt{\$1}, \texttt{\$2},...
\item[]
    \texttt{assign} {\em step} {\em label} - Assigns an unknown {\em step}
        number with the statement
        specified by {\em label}.
\item[]
    \texttt{let variable} {\em variable}
        \texttt{= "}{\em symbol sequence}\texttt{"}
          - Forces a symbol
        sequence to replace an unknown variable (such as \texttt{\$1}) in a proof.
        It is useful
        for helping difficult unifications, and it is necessary when you have
        dummy variables that eventually must be assigned a name.
\item[]
    \texttt{let step} {\em step} \texttt{= "}{\em symbol sequence}\texttt{"} -
          Forces a symbol sequence
        to replace the contents of a proof step, provided it can be
        unified with the existing step contents.  (I rarely use this.)
\item[]
    \texttt{unify step} {\em step} (or \texttt{unify all}) - Unifies
        the source and target of
        a step.  If you specify a specific step, you will be prompted
        to select among the unifications that are possible.  If you
        specify \texttt{all}, all steps with unique unifications, but only
        those steps, will be
        unified.  \texttt{unify all /interactive} goes through all non-unified
        steps.
\item[]
    \texttt{initialize} {\em step} (or \texttt{all}) - De-unifies the target and source of
        a step (or all steps), as well as the hypotheses of the source,
        and makes all variables in the source unknown.  Useful to recover from
        an \texttt{assign} or \texttt{let} mistake that
        resulted in incorrect unifications.
\item[]
    \texttt{delete} {\em step} (or \texttt{all} or \texttt{floating{\char`\_}hypotheses}) -
        Deletes the specified
        step(s).  \texttt{delete floating{\char`\_}hypotheses}, then \texttt{initialize all}, then
        \texttt{unify all /interactive} is useful for recovering from mistakes
        where incorrect unifications assigned wrong math symbol strings to
        variables.
\item[]
    \texttt{improve} {\em step} (or \texttt{all}) -
      Automatically creates a proof for steps (with no unknown variables)
      whose proof requires no statements with \texttt{\$e} hypotheses.  Useful
      for filling in proofs of \texttt{\$f} hypotheses.  The \texttt{/depth}
      qualifier will also try statements whose \texttt{\$e} hypotheses contain
      no new variables.  {\em Warning:} Save your work (with \texttt{save
      new{\char`\_}proof} then \texttt{write source}) before using
      \texttt{/depth = 2} or greater, since the search time grows
      exponentially and may never terminate in a reasonable time, and you
      cannot interrupt the search.  I have found that it is rare for
      \texttt{/depth = 3} or greater to be useful.
 \item[]
    \texttt{save new{\char`\_}proof} - Saves the proof in progress in the program's
        internal database buffer.  To save it permanently into the database file,
        use \texttt{write source} after
        \texttt{save new{\char`\_}proof}.  To revert to the last
        \texttt{save new{\char`\_}proof},
        \texttt{exit /force} from the Proof Assistant then re-enter the Proof
        Assistant.
 \item[]
    \texttt{match step} {\em step} (or \texttt{match all}) - Shows what
        statements are
        possibilities for the \texttt{assign} statement. (This command
        is not very
        useful in its present form and hopefully will be improved
        eventually.  In the meantime, use the \texttt{search} statement for
        candidates matching specific math token combinations.)
 \item[]
 \texttt{minimize{\char`\_}with}\index{\texttt{minimize{\char`\_}with} command}
% 3/10/07 Note: line-breaking the above results in duplicate index entries
     - After a proof is complete, this command will attempt
        to match other database theorems to the proof to see if the proof
        size can be reduced as a result.  See \texttt{help
        minimize{\char`\_}with} in the
        Metamath program for its usage.
 \item[]
 \texttt{undo}\index{\texttt{undo} command}
    - Undo the effect of a proof-changing command (all but the
      \texttt{show} and \texttt{save} commands above).
 \item[]
 \texttt{redo}\index{\texttt{redo} command}
    - Reverse the previous \texttt{undo}.
\end{itemize}

The following commands set parameters that may be relevant to your proof.
Consult the individual \texttt{help set}... commands.
\begin{itemize}
   \item[] \texttt{set unification{\char`\_}timeout}
 \item[]
    \texttt{set search{\char`\_}limit}
  \item[]
    \texttt{set empty{\char`\_}substitution} - note that default is \texttt{off}
\end{itemize}

Type \texttt{exit} to exit the \texttt{MM-PA>}
 prompt and get back to the \texttt{MM>} prompt.
Another \texttt{exit} will then get you out of Metamath.



\subsection{\texttt{prove} Command}\index{\texttt{prove} command}
Syntax:  \texttt{prove} {\em label}

This command will enter the Proof Assistant\index{Proof Assistant}, which will
allow you to create or edit the proof of the specified statement.
The command-line prompt will change from \texttt{MM>} to \texttt{MM-PA>}.

Note:  In the present version (0.177) of
Metamath\index{Metamath!limitations of version 0.177}, the Proof
Assistant does not verify that \texttt{\$d}\index{\texttt{\$d}
statement} restrictions are met as a proof is being built.  After you
have completed a proof, you should type \texttt{save new{\char`\_}proof}
followed by \texttt{verify proof} {\em label} (where {\em label} is the
statement you are proving with the \texttt{prove} command) to verify the
\texttt{\$d} restrictions.

See also: \texttt{exit}

\subsection{\texttt{set unification\_timeout} Command}\index{\texttt{set
unification{\char`\_}timeout} command}
Syntax:  \texttt{set unification{\char`\_}timeout} {\em number}

(This command is available outside the Proof Assistant but affects the
Proof Assistant\index{Proof Assistant} only.)

Sometimes the Proof Assistant will inform you that a unification
time-out occurred.  This may happen when you try to \texttt{unify}
formulas with many temporary variables\index{temporary variable}
(\texttt{\$1}, \texttt{\$2}, etc.), since the time to compute all possible
unifications may grow exponentially with the number of variables.  If
you want Metamath to try harder (and you're willing to wait longer) you
may increase this parameter.  \texttt{show settings} will show you the
current value.

\subsection{\texttt{set empty\_substitution} Command}\index{\texttt{set
empty{\char`\_}substitution} command}
% These long names can't break well in narrow mode, and even "sloppy"
% is not enough. Work around this by not demanding justification.
\begin{flushleft}
Syntax:  \texttt{set empty{\char`\_}substitution on} or \texttt{set
empty{\char`\_}substitution off}
\end{flushleft}

(This command is available outside the Proof Assistant but affects the
Proof Assistant\index{Proof Assistant} only.)

The Metamath language allows variables to be
substituted\index{substitution!variable}\index{variable substitution}
with empty symbol sequences\index{empty substitution}.  However, in many
formal systems\index{formal system} this will never happen in a valid
proof.  Allowing for this possibility increases the likelihood of
ambiguous unifications\index{ambiguous
unification}\index{unification!ambiguous} during proof creation.
The default is that
empty substitutions are not allowed; for formal systems requiring them,
you must \texttt{set empty{\char`\_}substitution on}.
(An example where this must be \texttt{on}
would be a system that implements a Deduction Rule and in
which deductions from empty assumption lists would be permissible.  The
MIU-system\index{MIU-system} described in Appendix~\ref{MIU} is another
example.)
Note that empty substitutions are
always permissible in proof verification (VERIFY PROOF...) outside the
Proof Assistant.  (See the MIU system in the Metamath book for an example
of a system needing empty substitutions; another example would be a
system that implements a Deduction Rule and in which deductions from
empty assumption lists would be permissible.)

It is better to leave this \texttt{off} when working with \texttt{set.mm}.
Note that this command does not affect the way proofs are verified with
the \texttt{verify proof} command.  Outside of the Proof Assistant,
substitution of empty sequences for math symbols is always allowed.

\subsection{\texttt{set search\_limit} Command}\index{\texttt{set
search{\char`\_}limit} command} Syntax:  \texttt{set search{\char`\_}limit} {\em
number}

(This command is available outside the Proof Assistant but affects the
Proof Assistant\index{Proof Assistant} only.)

This command sets a parameter that determines when the \texttt{improve} command
in Proof Assistant mode gives up.  If you want \texttt{improve} to search harder,
you may increase it.  The \texttt{show settings} command tells you its current
value.


\subsection{\texttt{show new\_proof} Command}\index{\texttt{show
new{\char`\_}proof} command}
Syntax:  \texttt{show new{\char`\_}proof} [{\em
qualifiers} (see below)]

This command (available only in Proof Assistant mode) displays the proof
in progress.  It is identical to the \texttt{show proof} command, except that
there is no statement argument (since it is the statement being proved) and
the following qualifiers are not available:

    \texttt{/statement{\char`\_}summary}

    \texttt{/detailed{\char`\_}step}

Also, the following additional qualifiers are available:

    \texttt{/unknown} - Shows only steps that have no statement assigned.

    \texttt{/not{\char`\_}unified} - Shows only steps that have not been unified.

Note that \texttt{/essential}, \texttt{/depth}, \texttt{/unknown}, and
\texttt{/not{\char`\_}unified} may be
used in any combination; each of them effectively filters out additional
steps from the proof display.

See also:  \texttt{show proof}






\subsection{\texttt{assign} Command}\index{\texttt{assign} command}
Syntax:   \texttt{assign} {\em step} {\em label} [\texttt{/no{\char`\_}unify}]

   and:   \texttt{assign first} {\em label}

   and:   \texttt{assign last} {\em label}


This command, available in the Proof Assistant only, assigns an unknown
step (one with \texttt{?} in the \texttt{show new{\char`\_}proof}
listing) with the statement specified by {\em label}.  The assignment
will not be allowed if the statement cannot be unified with the step.

If \texttt{last} is specified instead of {\em step} number, the last
step that is shown by \texttt{show new{\char`\_}proof /unknown} will be
used.  This can be useful for building a proof with a command file (see
\texttt{help submit}).  It also makes building proofs faster when you know
the assignment for the last step.

If \texttt{first} is specified instead of {\em step} number, the first
step that is shown by \texttt{show new{\char`\_}proof /unknown} will be
used.

If {\em step} is zero or negative, the -{\em step}th from last unknown
step, as shown by \texttt{show new{\char`\_}proof /unknown}, will be
used.  \texttt{assign -1} {\em label} will assign the penultimate
unknown step, \texttt{assign -2} {\em label} the antepenultimate, and
\texttt{assign 0} {\em label} is the same as \texttt{assign last} {\em
label}.

Optional command qualifier:

    \texttt{/no{\char`\_}unify} - do not prompt user to select a unification if there is
        more than one possibility.  This is useful for noninteractive
        command files.  Later, the user can \texttt{unify all /interactive}.
        (The assignment will still be automatically unified if there is only
        one possibility and will be refused if unification is not possible.)



\subsection{\texttt{match} Command}\index{\texttt{match} command}
Syntax:  \texttt{match step} {\em step} [\texttt{/max{\char`\_}essential{\char`\_}hyp}
{\em number}]

    and:  \texttt{match all} [\texttt{/essential}]
          [\texttt{/max{\char`\_}essential{\char`\_}hyp} {\em number}]

This command, available in the Proof Assistant only, shows what
statements can be unified with the specified step(s).  {\em Note:} In
its current form, this command is not very useful because of the large
number of matches it reports.
It may be enhanced in the future.  In the meantime, the \texttt{search}
command can often provide finer control over locating theorems of interest.

Optional command qualifiers:

    \texttt{/max{\char`\_}essential{\char`\_}hyp} {\em number} - filters out
        of the list any statements
        with more than the specified number of
        \texttt{\$e}\index{\texttt{\$e} statement} hypotheses.

    \texttt{/essential{\char`\_}only} - in the \texttt{match all} statement, only
        the steps that
        would be listed in the \texttt{show new{\char`\_}proof /essential} display are
        matched.



\subsection{\texttt{let} Command}\index{\texttt{let} command}
Syntax: \texttt{let variable} {\em variable} = \verb/"/{\em symbol-sequence}\verb/"/

  and: \texttt{let step} {\em step} = \verb/"/{\em symbol-sequence}\verb/"/

These commands, available in the Proof Assistant\index{Proof Assistant}
only, assign a temporary variable\index{temporary variable} or unknown
step with a specific symbol sequence.  They are useful in the middle of
creating a proof, when you know what should be in the proof step but the
unification algorithm doesn't yet have enough information to completely
specify the temporary variables.  A ``temporary variable'' is one that
has the form \texttt{\$}{\em nn} in the proof display, such as
\texttt{\$1}, \texttt{\$2}, etc.  The {\em symbol-sequence} may contain
other unknown variables if desired.  Examples:

    \verb/let variable $32 = "A = B"/

    \verb/let variable $32 = "A = $35"/

    \verb/let step 10 = '|- x = x'/

    \verb/let step -2 = "|- ( $7 = ph )"/

Any symbol sequence will be accepted for the \texttt{let variable}
command.  Only those symbol sequences that can be unified with the step
will be accepted for \texttt{let step}.

The \texttt{let} commands ``zap'' the proof with information that can
only be verified when the proof is built up further.  If you make an
error, the command sequence \texttt{delete
floating{\char`\_}hypotheses}, \texttt{initialize all}, and
\texttt{unify all /interactive} will undo a bad \texttt{let} assignment.

If {\em step} is zero or negative, the -{\em step}th from last unknown
step, as shown by \texttt{show new{\char`\_}proof /unknown}, will be
used.  The command \texttt{let step 0} = \verb/"/{\em
symbol-sequence}\verb/"/ will use the last unknown step, \texttt{let
step -1} = \verb/"/{\em symbol-sequence}\verb/"/ the penultimate, etc.
If {\em step} is positive, \texttt{let step} may be used to assign known
(in the sense of having previously been assigned a label with
\texttt{assign}) as well as unknown steps.

Either single or double quotes can surround the {\em symbol-sequence} as
long as they are different from any quotes inside a {\em
symbol-sequence}.  If {\em symbol-sequence} contains both kinds of
quotes, see the instructions at the end of \texttt{help let} in the
Metamath program.


\subsection{\texttt{unify} Command}\index{\texttt{unify} command}
Syntax:  \texttt{unify step} {\em step}

      and:   \texttt{unify all} [\texttt{/interactive}]

These commands, available in the Proof Assistant only, unify the source
and target of the specified step(s). If you specify a specific step, you
will be prompted to select among the unifications that are possible.  If
you specify \texttt{all}, only those steps with unique unifications will be
unified.

Optional command qualifier for \texttt{unify all}:

    \texttt{/interactive} - You will be prompted to select among the
        unifications
        that are possible for any steps that do not have unique
        unifications.  (Otherwise \texttt{unify all} will bypass these.)

See also \texttt{set unification{\char`\_}timeout}.  The default is
100000, but increasing it to 1000000 can help difficult cases.  Manually
assigning some or all of the unknown variables with the \texttt{let
variable} command also helps difficult cases.



\subsection{\texttt{initialize} Command}\index{\texttt{initialize} command}
Syntax:  \texttt{initialize step} {\em step}

    and: \texttt{initialize all}

These commands, available in the Proof Assistant\index{Proof Assistant}
only, ``de-unify'' the target and source of a step (or all steps), as
well as the hypotheses of the source, and makes all variables in the
source and the source's hypotheses unknown.  This command is useful to
help recover from incorrect unifications that resulted from an incorrect
\texttt{assign}, \texttt{let}, or unification choice.  Part or all of
the command sequence \texttt{delete floating{\char`\_}hypotheses},
\texttt{initialize all}, and \texttt{unify all /interactive} will recover
from incorrect unifications.

See also:  \texttt{unify} and \texttt{delete}



\subsection{\texttt{delete} Command}\index{\texttt{delete} command}
Syntax:  \texttt{delete step} {\em step}

   and:      \texttt{delete all} -- {\em Warning: dangerous!}

   and:      \texttt{delete floating{\char`\_}hypotheses}

These commands are available in the Proof Assistant only.  The
\texttt{delete step} command deletes the proof tree section that
branches off of the specified step and makes the step become unknown.
\texttt{delete all} is equivalent to \texttt{delete step} {\em step}
where {\em step} is the last step in the proof (i.e.\ the beginning of
the proof tree).

In most cases the \texttt{undo} command is the best way to undo
a previous step.
An alternative is to salvage your last \texttt{save
new{\char`\_}proof} by exiting and reentering the Proof Assistant.
For this to work, keep a log file open to record your work
and to do \texttt{save new{\char`\_}proof} frequently, especially before
\texttt{delete}.

\texttt{delete floating{\char`\_}hypotheses} will delete all sections of
the proof that branch off of \texttt{\$f}\index{\texttt{\$f} statement}
statements.  It is sometimes useful to do this before an
\texttt{initialize} command to recover from an error.  Note that once a
proof step with a \texttt{\$f} hypothesis as the target is completely
known, the \texttt{improve} command can usually fill in the proof for
that step.  Unlike the deletion of logical steps, \texttt{delete
floating{\char`\_}hypotheses} is a relatively safe command that is
usually easy to recover from.



\subsection{\texttt{improve} Command}\index{\texttt{improve} command}
\label{improve}
Syntax:  \texttt{improve} {\em step} [\texttt{/depth} {\em number}]
                                               [\texttt{/no{\char`\_}distinct}]

   and:   \texttt{improve first} [\texttt{/depth} {\em number}]
                                              [\texttt{/no{\char`\_}distinct}]

   and:   \texttt{improve last} [\texttt{/depth} {\em number}]
                                              [\texttt{/no{\char`\_}distinct}]

   and:   \texttt{improve all} [\texttt{/depth} {\em number}]
                                              [\texttt{/no{\char`\_}distinct}]

These commands, available in the Proof Assistant\index{Proof Assistant}
only, try to find proofs automatically for unknown steps whose symbol
sequences are completely known.  They are primarily useful for filling in
proofs of \texttt{\$f}\index{\texttt{\$f} statement} hypotheses.  The
search will be restricted to statements having no
\texttt{\$e}\index{\texttt{\$e} statement} hypotheses.

\begin{sloppypar} % narrow
Note:  If memory is limited, \texttt{improve all} on a large proof may
overflow memory.  If you use \texttt{set unification{\char`\_}timeout 1}
before \texttt{improve all}, there will usually be sufficient
improvement to easily recover and completely \texttt{improve} the proof
later on a larger computer.  Warning:  Once memory has overflowed, there
is no recovery.  If in doubt, save the intermediate proof (\texttt{save
new{\char`\_}proof} then \texttt{write source}) before \texttt{improve
all}.
\end{sloppypar}

If \texttt{last} is specified instead of {\em step} number, the last
step that is shown by \texttt{show new{\char`\_}proof /unknown} will be
used.

If \texttt{first} is specified instead of {\em step} number, the first
step that is shown by \texttt{show new{\char`\_}proof /unknown} will be
used.

If {\em step} is zero or negative, the -{\em step}th from last unknown
step, as shown by \texttt{show new{\char`\_}proof /unknown}, will be
used.  \texttt{improve -1} will use the penultimate
unknown step, \texttt{improve -2} {\em label} the antepenultimate, and
\texttt{improve 0} is the same as \texttt{improve last}.

Optional command qualifier:

    \texttt{/depth} {\em number} - This qualifier will cause the search
        to include
        statements with \texttt{\$e} hypotheses (but no new variables in
        the \texttt{\$e}
        hypotheses), provided that the backtracking has not exceeded the
        specified depth. {\em Warning:}  Try \texttt{/depth 1},
        then \texttt{2}, then \texttt{3}, etc.
        in sequence because of possible exponential blowups.  Save your
        work before trying \texttt{/depth} greater than \texttt{1}!

    \texttt{/no{\char`\_}distinct} - Skip trial statements that have
        \texttt{\$d}\index{\texttt{\$d} statement} requirements.
        This qualifier will prevent assignments that might violate \texttt{\$d}
        requirements but it also could miss possible legal assignments.

See also: \texttt{set search{\char`\_}limit}

\subsection{\texttt{save new\_proof} Command}\index{\texttt{save
new{\char`\_}proof} command}
Syntax:  \texttt{save new{\char`\_}proof} {\em label} [\texttt{/normal}]
   [\texttt{/compressed}]

The \texttt{save new{\char`\_}proof} command is available in the Proof
Assistant only.  It saves the proof in progress in the source
buffer\index{source buffer}.  \texttt{save new{\char`\_}proof} may be
used to save a completed proof, or it may be used to save a proof in
progress in order to work on it later.  If an incomplete proof is saved,
any user assignments with \texttt{let step} or \texttt{let variable}
will be lost, as will any ambiguous unifications\index{ambiguous
unification}\index{unification!ambiguous} that were resolved manually.
To help make recovery easier, it can be helpful to \texttt{improve all}
before \texttt{save new{\char`\_}proof} so that the incomplete proof
will have as much information as possible.

Note that the proof will not be permanently saved until a \texttt{write
source} command is issued.

Optional command qualifiers:

    \texttt{/normal} - The proof is saved in the normal format (i.e., as a
        sequence of labels, which is the defined format of the basic Metamath
        language).\index{basic language}  This is the default format that
        is used if a qualifier is omitted.

    \texttt{/compressed} - The proof is saved in the compressed format, which
        reduces storage requirements for a database.
        (See Appendix~\ref{compressed}.)


\section{Creating \LaTeX\ Output}\label{texout}\index{latex@{\LaTeX}}

You can generate \LaTeX\ output given the
information in a database.
The database must already include the necessary typesetting information
(see section \ref{tcomment} for how to provide this information).

The \texttt{show statement} and \texttt{show proof} commands each have a
special \texttt{/tex} command qualifier that produces \LaTeX\ output.
(The \texttt{show statement} command also has the
\texttt{/simple{\char`\_}tex} qualifier for output that is easier to
edit by hand.)  Before you can use them, you must open a \LaTeX\ file to
which to send their output.  A typical complete session will use this
sequence of Metamath commands:

\begin{verbatim}
read set.mm
open tex example.tex
show statement a1i /tex
show proof a1i /all/lemmon/renumber/tex
show statement uneq2 /tex
show proof uneq2 /all/lemmon/renumber/tex
close tex
\end{verbatim}

See Section~\ref{mathcomments} for information on comment markup and
Appendix~\ref{ASCII} for information on how math symbol translation is
specified.

To format and print the \LaTeX\ source, you will need the \LaTeX\
program, which is standard on most Linux installations and available for
Windows.  On Linux, in order to create a {\sc pdf} file, you will
typically type at the shell prompt
\begin{verbatim}
$ pdflatex example.tex
\end{verbatim}

\subsection{\texttt{open tex} Command}\index{\texttt{open tex} command}
Syntax:  \texttt{open tex} {\em file-name} [\texttt{/no{\char`\_}header}]

This command opens a file for writing \LaTeX\
source\index{latex@{\LaTeX}} and writes a \LaTeX\ header to the file.
\LaTeX\ source can be written with the \texttt{show proof}, \texttt{show
new{\char`\_}proof}, and \texttt{show statement} commands using the
\texttt{/tex} qualifier.

The mapping to \LaTeX\ symbols is defined in a special comment
containing a \texttt{\$t} token, described in Appendix~\ref{ASCII}.

There is an optional command qualifier:

    \texttt{/no{\char`\_}header} - This qualifier prevents a standard
        \LaTeX\ header and trailer
        from being included with the output \LaTeX\ code.


\subsection{\texttt{close tex} Command}\index{\texttt{close tex} command}
Syntax:  \texttt{close tex}

This command writes a trailer to any \LaTeX\ file\index{latex@{\LaTeX}}
that was opened with \texttt{open tex} (unless
\texttt{/no{\char`\_}header} was used with \texttt{open tex}) and closes
the \LaTeX\ file.


\section{Creating {\sc HTML} Output}\label{htmlout}

You can generate {\sc html} web pages given the
information in a database.
The database must already include the necessary typesetting information
(see section \ref{tcomment} for how to provide this information).
The ability to produce {\sc html} web pages was added in Metamath version
0.07.30.

To create an {\sc html} output file(s) for \texttt{\$a} or \texttt{\$p}
statement(s), use
\begin{quote}
    \texttt{show statement} {\em label-match} \texttt{/html}
\end{quote}
The output file will be named {\em label-match}\texttt{.html}
for each match.  When {\em
label-match} has wildcard (\texttt{*}) characters, all statements with
matching labels will have {\sc html} files produced for them.  Also,
when {\em label-match} has a wildcard (\texttt{*}) character, two additional
files, \texttt{mmdefinitions.html} and \texttt{mmascii.html} will be
produced.  To produce {\em only} these two additional files, you can use
\texttt{?*}, which will not match any statement label, in place of {\em
label-match}.

There are three other qualifiers for \texttt{show statement} that also
generate {\sc HTML} code.  These are \texttt{/alt{\char`\_}html},
\texttt{/brief{\char`\_}html}, and
\texttt{/brief{\char`\_}alt{\char`\_}html}, and are described in the
next section.

The command
\begin{quote}
    \texttt{show statement} {\em label-match} \texttt{/alt{\char`\_}html}
\end{quote}
does the same as \texttt{show statement} {\em label-match} \texttt{/html},
except that the {\sc html} code for the symbols is taken from
\texttt{althtmldef} statements instead of \texttt{htmldef} statements in
the \texttt{\$t} comment.

The command
\begin{verbatim}
show statement * /brief_html
\end{verbatim}
invokes a special mode that just produces definition and theorem lists
accompanied by their symbol strings, in a format suitable for copying and
pasting into another web page (such as the tutorial pages on the
Metamath web site).

Finally, the command
\begin{verbatim}
show statement * /brief_alt_html
\end{verbatim}
does the same as \texttt{show statement * / brief{\char`\_}html}
for the alternate {\sc html}
symbol representation.

A statement's comment can include a special notation that provides a
certain amount of control over the {\sc HTML} version of the comment.  See
Section~\ref{mathcomments} (p.~\pageref{mathcomments}) for the comment
markup features.

The \texttt{write theorem{\char`\_}list} and \texttt{write bibliography}
commands, which are described below, provide as a side effect complete
error checking for all of the features described in this
section.\index{error checking}

\subsection{\texttt{write theorem\_list}
Command}\index{\texttt{write theorem{\char`\_}list} command}
Syntax:  \texttt{write theorem{\char`\_}list}
[\texttt{/theorems{\char`\_}per{\char`\_}page} {\em number}]

This command writes a list of all of the \texttt{\$a} and \texttt{\$p}
statements in the database into a web page file
 called \texttt{mmtheorems.html}.
When additional files are needed, they are called
\texttt{mmtheorems2.html}, \texttt{mmtheorems3.html}, etc.

Optional command qualifier:

    \texttt{/theorems{\char`\_}per{\char`\_}page} {\em number} -
 This qualifier specifies the number of statements to
        write per web page.  The default is 100.

{\em Note:} In version 0.177\index{Metamath!limitations of version
0.177} of Metamath, the ``Nearby theorems'' links on the individual
web pages presuppose 100 theorems per page when linking to the theorem
list pages.  Therefore the \texttt{/theorems{\char`\_}per{\char`\_}page}
qualifier, if it specifies a number other than 100, will cause the
individual web pages to be out of sync and should not be used to
generate the main theorem list for the web site.  This may be
fixed in a future version.


\subsection{\texttt{write bibliography}\label{wrbib}
Command}\index{\texttt{write bibliography} command}
Syntax:  \texttt{write bibliography} {\em filename}

This command reads an existing {\sc html} bibliographic cross-reference
file, normally called \texttt{mmbiblio.html}, and updates it per the
bibliographic links in the database comments.  The file is updated
between the {\sc html} comment lines \texttt{<!--
{\char`\#}START{\char`\#} -->} and \texttt{<!-- {\char`\#}END{\char`\#}
-->}.  The original input file is renamed to {\em
filename}\texttt{{\char`\~}1}.

A bibliographic reference is indicated with the reference name
in brackets, such as  \texttt{Theorem 3.1 of
[Monk] p.\ 22}.
See Section~\ref{htmlout} (p.~\pageref{htmlout}) for
syntax details.


\subsection{\texttt{write recent\_additions}
Command}\index{\texttt{write recent{\char`\_}additions} command}
Syntax:  \texttt{write recent{\char`\_}additions} {\em filename}
[\texttt{/limit} {\em number}]

This command reads an existing ``Recent Additions'' {\sc html} file,
normally called \texttt{mmrecent.html}, and updates it with the
descriptions of the most recently added theorems to the database.
 The file is updated between
the {\sc html} comment lines \texttt{<!-- {\char`\#}START{\char`\#} -->}
and \texttt{<!-- {\char`\#}END{\char`\#} -->}.  The original input file
is renamed to {\em filename}\texttt{{\char`\~}1}.

Optional command qualifier:

    \texttt{/limit} {\em number} -
 This qualifier specifies the number of most recent theorems to
   write to the output file.  The default is 100.


\section{Text File Utilities}

\subsection{\texttt{tools} Command}\index{\texttt{tools} command}
Syntax:  \texttt{tools}

This command invokes an easy-to-use, general purpose utility for
manipulating the contents of {\sc ascii} text files.  Upon typing
\texttt{tools}, the command-line prompt will change to \texttt{TOOLS>}
until you type \texttt{exit}.  The \texttt{tools} commands can be used
to perform simple, global edits on an input/output file,
such as making a character string substitution on each line, adding a
string to each line, and so on.  A typical use of this utility is
to build a \texttt{submit} input file to perform a common operation on a
list of statements obtained from \texttt{show label} or \texttt{show
usage}.

The actions of most of the \texttt{tools} commands can also be
performed with equivalent (and more powerful) Unix shell commands, and
some users may find those more efficient.  But for Windows users or
users not comfortable with Unix, \texttt{tools} provides an
easy-to-learn alternative that is adequate for most of the
script-building tasks needed to use the Metamath program effectively.

\subsection{\texttt{help} Command (in \texttt{tools})}
Syntax:  \texttt{help}

The \texttt{help} command lists the commands available in the
\texttt{tools} utility, along with a brief description.  Each command,
in turn, has its own help, such as \texttt{help add}.  As with
Metamath's \texttt{MM>} prompt, a complete command can be entered at
once, or just the command word can be typed, causing the program to
prompt for each argument.

\vskip 1ex
\noindent Line-by-line editing commands:

  \texttt{add} - Add a specified string to each line in a file.

  \texttt{clean} - Trim spaces and tabs on each line in a file; convert
         characters.

  \texttt{delete} - Delete a section of each line in a file.

  \texttt{insert} - Insert a string at a specified column in each line of
        a file.

  \texttt{substitute} - Make a simple substitution on each line of the file.

  \texttt{tag} - Like \texttt{add}, but restricted to a range of lines.

  \texttt{swap} - Swap the two halves of each line in a file.

\vskip 1ex
\noindent Other file-processing commands:

  \texttt{break} - Break up (tokenize) a file into a list of tokens (one per
        line).

  \texttt{build} - Build a file with multiple tokens per line from a list.

  \texttt{count} - Count the occurrences in a file of a specified string.

  \texttt{number} - Create a list of numbers.

  \texttt{parallel} - Put two files in parallel.

  \texttt{reverse} - Reverse the order of the lines in a file.

  \texttt{right} - Right-justify lines in a file (useful before sorting
         numbers).

%  \texttt{tag} - Tag edit updates in a program for revision control.

  \texttt{sort} - Sort the lines in a file with key starting at
         specified string.

  \texttt{match} - Extract lines containing (or not) a specified string.

  \texttt{unduplicate} - Eliminate duplicate occurrences of lines in a file.

  \texttt{duplicate} - Extract first occurrence of any line occurring
         more than

   \ \ \    once in a file, discarding lines occurring exactly once.

  \texttt{unique} - Extract lines occurring exactly once in a file.

  \texttt{type} (10 lines) - Display the first few lines in a file.
                  Similar to Unix \texttt{head}.

  \texttt{copy} - Similar to Unix \texttt{cat} but safe (same input
         and output file allowed).

  \texttt{submit} - Run a script containing \texttt{tools} commands.

\vskip 1ex

\noindent Note:
  \texttt{unduplicate}, \texttt{duplicate}, and \texttt{unique} also
 sort the lines as a side effect.


\subsection{Using \texttt{tools} to Build Metamath \texttt{submit}
Scripts}

The \texttt{break} command is typically used to break up a series of
statement labels, such as the output of Metamath's \texttt{show usage},
into one label per line.  The other \texttt{tools} commands can then be
used to add strings before and after each statement label to specify
commands to be performed on the statement.  The \texttt{parallel}
command is useful when a statement label must be mentioned more than
once on a line.

Very often a \texttt{submit} script for Metamath will require multiple
command lines for each statement being processed.  For example, you may
want to enter the Proof Assistant, \texttt{minimize{\char`\_}with} your
latest theorem, \texttt{save} the new proof, and \texttt{exit} the Proof
Assistant.  To accomplish this, you can build a file with these four
commands for each statement on a single line, separating each command
with a designated character such as \texttt{@}.  Then at the end you can
\texttt{substitute} each \texttt{@} with \texttt{{\char`\\}n} to break
up the lines into individual command lines (see \texttt{help
substitute}).


\subsection{Example of a \texttt{tools} Session}

To give you a quick feel for the \texttt{tools} utility, we show a
simple session where we create a file \texttt{n.txt} with 3 lines, add
strings before and after each line, and display the lines on the screen.
You can experiment with the various commands to gain experience with the
\texttt{tools} utility.

\begin{verbatim}
MM> tools
Entering the Text Tools utilities.
Type HELP for help, EXIT to exit.
TOOLS> number
Output file <n.tmp>? n.txt
First number <1>?
Last number <10>? 3
Increment <1>?
TOOLS> add
Input/output file? n.txt
String to add to beginning of each line <>? This is line
String to add to end of each line <>? .
The file n.txt has 3 lines; 3 were changed.
First change is on line 1:
This is line 1.
TOOLS> type n.txt
This is line 1.
This is line 2.
This is line 3.
TOOLS> exit
Exiting the Text Tools.
Type EXIT again to exit Metamath.
MM>
\end{verbatim}



\appendix
\chapter{Sample Representations}
\label{ASCII}

This Appendix provides a sample of {\sc ASCII} representations,
their corresponding traditional mathematical symbols,
and a discussion of their meanings
in the \texttt{set.mm} database.
These are provided in order of appearance.
This is only a partial list, and new definitions are routinely added.
A complete list is available at \url{http://metamath.org}.

These {\sc ASCII} representations, along
with information on how to display them,
are defined in the \texttt{set.mm} database file inside
a special comment called a \texttt{\$t} {\em
comment}\index{\texttt{\$t} comment} or {\em typesetting
comment.}\index{typesetting comment}
A typesetting comment
is indicated by the appearance of the
two-character string \texttt{\$t} at the beginning of the comment.
For more information,
see Section~\ref{tcomment}, p.~\pageref{tcomment}.

In the following table the ``{\sc ASCII}'' column shows the {\sc ASCII}
representation,
``Symbol'' shows the mathematical symbolic display
that corresponds to that {\sc ASCII} representation, ``Labels'' shows
the key label(s) that define the representation, and
``Description'' provides a description about the symbol.
As usual, ``iff'' is short for ``if and only if.''\index{iff}
In most cases the ``{\sc ASCII}'' column only shows
the key token, but it will sometimes show a sequence of tokens
if that is necessary for clarity.

{\setlength{\extrarowsep}{4pt} % Keep rows from being too close together
\begin{longtabu}   { @{} c c l X }
\textbf{ASCII} & \textbf{Symbol} & \textbf{Labels} & \textbf{Description} \\
\endhead
\texttt{|-} & $\vdash$ & &
  ``It is provable that...'' \\
\texttt{ph} & $\varphi$ & \texttt{wph} &
  The wff (boolean) variable phi,
  conventionally the first wff variable. \\
\texttt{ps} & $\psi$ & \texttt{wps} &
  The wff (boolean) variable psi,
  conventionally the second wff variable. \\
\texttt{ch} & $\chi$ & \texttt{wch} &
  The wff (boolean) variable chi,
  conventionally the third wff variable. \\
\texttt{-.} & $\lnot$ & \texttt{wn} &
  Logical not. E.g., if $\varphi$ is true, then $\lnot \varphi$ is false. \\
\texttt{->} & $\rightarrow$ & \texttt{wi} &
  Implies, also known as material implication.
  In classical logic the expression $\varphi \rightarrow \psi$ is true
  if either $\varphi$ is false or $\psi$ is true (or both), that is,
  $\varphi \rightarrow \psi$ has the same meaning as
  $\lnot \varphi \lor \psi$ (as proven in theorem \texttt{imor}). \\
\texttt{<->} & $\leftrightarrow$ &
  \hyperref[df-bi]{\texttt{df-bi}} &
  Biconditional (aka is-equals for boolean values).
  $\varphi \leftrightarrow \psi$ is true iff
  $\varphi$ and $\psi$ have the same value. \\
\texttt{\char`\\/} & $\lor$ &
  \makecell[tl]{{\hyperref[df-or]{\texttt{df-or}}}, \\
	         \hyperref[df-3or]{\texttt{df-3or}}} &
  Disjunction (logical ``or''). $\varphi \lor \psi$ is true iff
  $\varphi$, $\psi$, or both are true. \\
\texttt{/\char`\\} & $\land$ &
  \makecell[tl]{{\hyperref[df-an]{\texttt{df-an}}}, \\
                 \hyperref[df-3an]{\texttt{df-3an}}} &
  Conjunction (logical ``and''). $\varphi \land \psi$ is true iff
  both $\varphi$ and $\psi$ are true. \\
\texttt{A.} & $\forall$ &
  \texttt{wal} &
  For all; the wff $\forall x \varphi$ is true iff
  $\varphi$ is true for all values of $x$. \\
\texttt{E.} & $\exists$ &
  \hyperref[df-ex]{\texttt{df-ex}} &
  There exists; the wff
  $\exists x \varphi$ is true iff
  there is at least one $x$ where $\varphi$ is true. \\
\texttt{[ y / x ]} & $[ y / x ]$ &
  \hyperref[df-sb]{\texttt{df-sb}} &
  The wff $[ y / x ] \varphi$ produces
  the result when $y$ is properly substituted for $x$ in $\varphi$
  ($y$ replaces $x$).
  % This is elsb4
  % ( [ x / y ] z e. y <-> z e. x )
  For example,
  $[ x / y ] z \in y$ is the same as $z \in x$. \\
\texttt{E!} & $\exists !$ &
  \hyperref[df-eu]{\texttt{df-eu}} &
  There exists exactly one;
  $\exists ! x \varphi$ is true iff
  there is at least one $x$ where $\varphi$ is true. \\
\texttt{\{ y | phi \}}  & $ \{ y | \varphi \}$ &
  \hyperref[df-clab]{\texttt{df-clab}} &
  The class of all sets where $\varphi$ is true. \\
\texttt{=} & $ = $ &
  \hyperref[df-cleq]{\texttt{df-cleq}} &
  Class equality; $A = B$ iff $A$ equals $B$. \\
\texttt{e.} & $ \in $ &
  \hyperref[df-clel]{\texttt{df-clel}} &
  Class membership; $A \in B$ if $A$ is a member of $B$. \\
\texttt{{\char`\_}V} & {\rm V} &
  \hyperref[df-v]{\texttt{df-v}} &
  Class of all sets (not itself a set). \\
\texttt{C\_} & $ \subseteq $ &
  \hyperref[df-ss]{\texttt{df-ss}} &
  Subclass (subset); $A \subseteq B$ is true iff
  $A$ is a subclass of $B$. \\
\texttt{u.} & $ \cup $ &
  \hyperref[df-un]{\texttt{df-un}} &
  $A \cup B$ is the union of classes $A$ and $B$. \\
\texttt{i^i} & $ \cap $ &
  \hyperref[df-in]{\texttt{df-in}} &
  $A \cap B$ is the intersection of classes $A$ and $B$. \\
\texttt{\char`\\} & $ \setminus $ &
  \hyperref[df-dif]{\texttt{df-dif}} &
  $A \setminus B$ (set difference)
  is the class of all sets in $A$ except for those in $B$. \\
\texttt{(/)} & $ \varnothing $ &
  \hyperref[df-nul]{\texttt{df-nul}} &
  $ \varnothing $ is the empty set (aka null set). \\
\texttt{\char`\~P} & $ \cal P $ &
  \hyperref[df-pw]{\texttt{df-pw}} &
  Power class. \\
\texttt{<.\ A , B >.} & $\langle A , B \rangle$ &
  \hyperref[df-op]{\texttt{df-op}} &
  The ordered pair $\langle A , B \rangle$. \\
\texttt{( F ` A )} & $ ( F ` A ) $ &
  \hyperref[df-fv]{\texttt{df-fv}} &
  The value of function $F$ when applied to $A$. \\
\texttt{\_i} & $ i $ &
  \texttt{df-i} &
  The square root of negative one. \\
\texttt{x.} & $ \cdot $ &
  \texttt{df-mul} &
  Complex number multiplication; $2~\cdot~3~=~6$. \\
\texttt{CC} & $ \mathbb{C} $ &
  \texttt{df-c} &
  The set of complex numbers. \\
\texttt{RR} & $ \mathbb{R} $ &
  \texttt{df-r} &
  The set of real numbers. \\
\end{longtabu}
} % end of extrarowsep

\chapter{Compressed Proofs}
\label{compressed}\index{compressed proof}\index{proof!compressed}

The proofs in the \texttt{set.mm} set theory database are stored in compressed
format for efficiency.  Normally you needn't concern yourself with the
compressed format, since you can display it with the usual proof display tools
in the Metamath program (\texttt{show proof}\ldots) or convert it to the normal
RPN proof format described in Section~\ref{proof} (with \texttt{save proof}
{\em label} \texttt{/normal}).  However for sake of completeness we describe the
format here and show how it maps to the normal RPN proof format.

A compressed proof, located between \texttt{\$=} and \texttt{\$.}\ keywords, consists
of a left parenthesis, a sequence of statement labels, a right parenthesis,
and a sequence of upper-case letters \texttt{A} through \texttt{Z} (with optional
white space between them).  White space must surround the parentheses
and the labels.  The left parenthesis tells Metamath that a
compressed proof follows.  (A normal RPN proof consists of just a sequence of
labels, and a parenthesis is not a legal character in a label.)

The sequence of upper-case letters corresponds to a sequence of integers
with the following mapping.  Each integer corresponds to a proof step as
described later.
\begin{center}
  \texttt{A} = 1 \\
  \texttt{B} = 2 \\
   \ldots \\
  \texttt{T} = 20 \\
  \texttt{UA} = 21 \\
  \texttt{UB} = 22 \\
   \ldots \\
  \texttt{UT} = 40 \\
  \texttt{VA} = 41 \\
  \texttt{VB} = 42 \\
   \ldots \\
  \texttt{YT} = 120 \\
  \texttt{UUA} = 121 \\
   \ldots \\
  \texttt{YYT} = 620 \\
  \texttt{UUUA} = 621 \\
   etc.
\end{center}

In other words, \texttt{A} through \texttt{T} represent the
least-significant digit in base 20, and \texttt{U} through \texttt{Y}
represent zero or more most-significant digits in base 5, where the
digits start counting at 1 instead of the usual 0. With this scheme, we
don't need white space between these ``numbers.''

(In the design of the compressed proof format, only upper-case letters,
as opposed to say all non-whitespace printable {\sc ascii} characters other than
%\texttt{\$}, was chosen to make the compressed proof a little less
%displeasing to the eye, at the expense of a typical 20\% compression
\texttt{\$}, were chosen so as not to collide with most text editor
searches, at the expense of a typical 20\% compression
loss.  The base 5/base 20 grouping, as opposed to say base 6/base 19,
was chosen by experimentally determining the grouping that resulted in
best typical compression.)

The letter \texttt{Z} identifies (tags) a proof step that is identical to one
that occurs later on in the proof; it helps shorten the proof by not requiring
that identical proof steps be proved over and over again (which happens often
when building wff's).  The \texttt{Z} is placed immediately after the
least-significant digit (letters \texttt{A} through \texttt{T}) that ends the integer
corresponding to the step to later be referenced.

The integers that the upper-case letters correspond to are mapped to labels as
follows.  If the statement being proved has $m$ mandatory hypotheses, integers
1 through $m$ correspond to the labels of these hypotheses in the order shown
by the \texttt{show statement ... / full} command, i.e., the RPN order\index{RPN
order} of the mandatory
hypotheses.  Integers $m+1$ through $m+n$ correspond to the labels enclosed in
the parentheses of the compressed proof, in the order that they appear, where
$n$ is the number of those labels.  Integers $m+n+1$ on up don't directly
correspond to statement labels but point to proof steps identified with the
letter \texttt{Z}, so that these proof steps can be referenced later in the
proof.  Integer $m+n+1$ corresponds to the first step tagged with a \texttt{Z},
$m+n+2$ to the second step tagged with a \texttt{Z}, etc.  When the compressed
proof is converted to a normal proof, the entire subproof of a step tagged
with \texttt{Z} replaces the reference to that step.

For efficiency, Metamath works with compressed proofs directly, without
converting them internally to normal proofs.  In addition to the usual
error-checking, an error message is given if (1) a label in the label list in
parentheses does not refer to a previous \texttt{\$p} or \texttt{\$a} statement or a
non-mandatory hypothesis of the statement being proved and (2) a proof step
tagged with \texttt{Z} is referenced before the step tagged with the \texttt{Z}.

Just as in a normal proof under development (Section~\ref{unknown}), any step
or subproof that is not yet known may be represented with a single \texttt{?}.
White space does not have to appear between the \texttt{?}\ and the upper-case
letters (or other \texttt{?}'s) representing the remainder of the proof.

% April 1, 2004 Appendix C has been added back in with corrections.
%
% May 20, 2003 Appendix C was removed for now because there was a problem found
% by Bob Solovay
%
% Also, removed earlier \ref{formalspec} 's (3 cases above)
%
% Bob Solovay wrote on 30 Nov 2002:
%%%%%%%%%%%%% (start of email comment )
%      3. My next set of comments concern appendix C. I read this before I
% read Chapter 4. So I first noted that the system as presented in the
% Appendix lacked a certain formal property that I thought desirable. I
% then came up with a revised formal system that had this property. Upon
% reading Chapter 4, I noticed that the revised system was closer to the
% treatment in Chapter 4 than the system in Appendix C.
%
%         First a very minor correction:
%
%         On page 142 line 2: The condition that V(e) != V(f) should only be
% required of e, f in T such that e != f.
%
%         Here is a natural property [transitivity] that one would like
% the formal system to have:
%
%         Let Gamma be a set of statements. Suppose that the statement Phi
% is provable from Gamma and that the statement Psi is provable from Gamma
% \cup {Phi}. Then Psi is provable from Gamma.
%
%         I shall present an example to show that this property does not
% hold for the formal systems of Appendix C:
%
%         I write the example in metamath style:
%
% $c A B C D E $.
% $v x y
%
% ${
% tx $f A x $.
% ty $f B y $.
% ax1 $a C x y $.
% $}
%
% ${
% tx $f A x $.
% ty $f B y $.
% ax2-h1 $e C x y $.
% ax2 $a D y $.
% $}
%
% ${
% ty $f B y $.
% ax3-h1 $e D y $.
% ax3 $a E y $.
% $}
%
% $(These three axioms are Gamma $)
%
% ${
% tx $f A x $.
% ty $f B y $.
% Phi $p D y $=
% tx ty tx ty ax1 ax2 $.
% $}
%
% ${
% ty $f B y $.
% Psi $p E y $=
% ty ty Phi ax3 $.
% $}
%
%
% I omit the formal proofs of the following claims. [I will be glad to
% supply them upon request.]
%
% 1) Psi is not provable from Gamma;
%
% 2) Psi is provable from Gamma + Phi.
%
% Here "provable" refers to the formalism of Appendix C.
%
% The trouble of course is that Psi is lacking the variable declaration
%
% $f Ax $.
%
% In the Metamath system there is no trouble proving Psi. I attach a
% metamath file that shows this and which has been checked by the
% metamath program.
%
% I next want to indicate how I think the treatment in Appendix C should
% be revised so as to conform more closely to the metamath system of the
% main text. The revised system *does* have the transitivity property.
%
% We want to give revised definitions of "statement" and
% "provable". [cf. sections C.2.4. and C.2.5] Our new definitions will
% use the definitions given in Appendix C. So we take the following
% tack. We refer to the original notions as o-statement and o-provable. And
% we refer to the notions we are defining as n-statement and n-provable.
%
%         A n-statement is an o-statement in which the only variables
% that appear in the T component are mandatory.
%
%         To any o-statement we can associate its reduct which is a
% n-statement by dropping all the elements of T or D which contain
% non-mandatory variables.
%
%         An n-statement gamma is n-provable if there is an o-statement
% gamma' which has gamma as its reduct andf such that gamma' is
% o-provable.
%
%         It seems to me [though I am not completely sure on this point]
% that n-provability corresponds to metamath provability as discussed
% say in Chapter 4.
%
%         Attached to this letter is the metamath proof of Phi and Psi
% from Gamma discussed above.
%
%         I am still brooding over the question of whether metamath
% correctly formalizes set-theory. No doubt I will have some questions
% re this after my thoughts become clearer.
%%%%%%%%%%%%%%%% (end of email comment)

%%%%%%%%%%%%%%%% (start of 2nd email comment from Bob Solovay 1-Apr-04)
%
%         I hope that Appendix C is the one that gives a "formal" treatment
% of Metamath. At any rate, thats the appendix I want to comment on.
%
%         I'm going to suggest two changes in the definition.
%
%         First change (in the definition of statement): Require that the
% sets D, T, and E be finite.
%
%         Probably things are fine as you give them. But in the applications
% to the main metamath system they will always be finite, and its useful in
% thinking about things [at least for me] to stick to the finite case.
%
%         Second change:
%
%         First let me give an approximate description. Remove the dummy
% variables from the statement. Instead, include them in the proof.
%
%         More formally: Require that T consists of type declarations only
% for mandatory variables. Require that all the pairs in D consist of
% mandatory variables.
%
%         At the start of a proof we are allowed to declare a finite number
% of dummy variables [provided that none of them appear in any of the
% statements in E \cup {A}. We have to supply type declarations for all the
% dummy variables. We are allowed to add new $d statements referring to
% either the mandatory or dummy variables. But we require that no new $d
% statement references only mandatory variables.
%
%         I find this way of doing things more conceptual than the treatment
% in Appendix C. But the change [which I will use implicitly in later
% letters about doing Peano] is mainly aesthetic. I definitely claim that my
% results on doing Peano all apply to Metamath as it is presented in your
% book.
%
%         --Bob
%
%%%%%%%%%%%%%%%% (end of 2nd email comment)

%%
%% When uncommenting the below, also uncomment references above to {formalspec}
%%
\chapter{Metamath's Formal System}\label{formalspec}\index{Metamath!as a formal
system}

\section{Introduction}

\begin{quote}
  {\em Perfection is when there is no longer anything more to take away.}
    \flushright\sc Antoine de
     Saint-Exupery\footnote{\cite[p.~3-25]{Campbell}.}\\
\end{quote}\index{de Saint-Exupery, Antoine}

This appendix describes the theory behind the Metamath language in an abstract
way intended for mathematicians.  Specifically, we construct two
set-theo\-ret\-i\-cal objects:  a ``formal system'' (roughly, a set of syntax
rules, axioms, and logical rules) and its ``universe'' (roughly, the set of
theorems derivable in the formal system).  The Metamath computer language
provides us with a way to describe specific formal systems and, with the aid of
a proof provided by the user, to verify that given theorems
belong to their universes.

To understand this appendix, you need a basic knowledge of informal set theory.
It should be sufficient to understand, for example, Ch.\ 1 of Munkres' {\em
Topology} \cite{Munkres}\index{Munkres, James R.} or the
introductory set theory chapter
in many textbooks that introduce abstract mathematics. (Note that there are
minor notational differences among authors; e.g.\ Munkres uses $\subset$ instead
of our $\subseteq$ for ``subset.''  We use ``included in'' to mean ``a subset
of,'' and ``belongs to'' or ``is contained in'' to mean ``is an element of.'')
What we call a ``formal'' description here, unlike earlier, is actually an
informal description in the ordinary language of mathematicians.  However we
provide sufficient detail so that a mathematician could easily formalize it,
even in the language of Metamath itself if desired.  To understand the logic
examples at the end of this appendix, familiarity with an introductory book on
mathematical logic would be helpful.

\section{The Formal Description}

\subsection[Preliminaries]{Preliminaries\protect\footnotemark}%
\footnotetext{This section is taken mostly verbatim
from Tarski \cite[p.~63]{Tarski1965}\index{Tarski, Alfred}.}

By $\omega$ we denote the set of all natural numbers (non-negative integers).
Each natural number $n$ is identified with the set of all smaller numbers: $n =
\{ m | m < n \}$.  The formula $m < n$ is thus equivalent to the condition: $m
\in n$ and $m,n \in \omega$. In particular, 0 is the number zero and at the
same time the empty set $\varnothing$, $1=\{0\}$, $2=\{0,1\}$, etc. ${}^B A$
denotes the set of all functions on $B$ to $A$ (i.e.\ with domain $B$ and range
included in $A$).  The members of ${}^\omega A$ are what are called {\em simple
infinite sequences},\index{simple infinite sequence}
with all {\em terms}\index{term} in $A$.  In case $n \in \omega$, the
members of ${}^n A$ are referred to as {\em finite $n$-termed
sequences},\index{finite $n$-termed
sequence} again
with terms in $A$.  The consecutive terms (function values) of a finite or
infinite sequence $f$ are denoted by $f_0, f_1, \ldots ,f_n,\ldots$.  Every
finite sequence $f \in \bigcup _{n \in \omega} {}^n A$ uniquely determines the
number $n$ such that $f \in {}^n A$; $n$ is called the {\em
length}\index{length of a sequence ({$"|\ "|$})} of $f$ and
is denoted by $|f|$.  $\langle a \rangle$ is the sequence $f$ with $|f|=1$ and
$f_0=a$; $\langle a,b \rangle$ is the sequence $f$ with $|f|=2$, $f_0=a$,
$f_1=b$; etc.  Given two finite sequences $f$ and $g$, we denote by $f\frown g$
their {\em concatenation},\index{concatenation} i.e., the
finite sequence $h$ determined by the
conditions:
\begin{eqnarray*}
& |h| = |f|+|g|;&  \\
& h_n = f_n & \mbox{\ for\ } n < |f|;  \\
& h_{|f|+n} = g_n & \mbox{\ for\ } n < |g|.
\end{eqnarray*}

\subsection{Constants, Variables, and Expressions}

A formal system has a set of {\em symbols}\index{symbol!in
a formal system} denoted
by $\mbox{\em SM}$.  A
precise set-theo\-ret\-i\-cal definition of this set is unimportant; a symbol
could be considered a primitive or atomic element if we wish.  We assume this
set is divided into two disjoint subsets:  a set $\mbox{\em CN}$ of {\em
constants}\index{constant!in a formal system} and a set $\mbox{\em VR}$ of
{\em variables}.\index{variable!in a formal system}  $\mbox{\em CN}$ and
$\mbox{\em VR}$ are each assumed to consist of countably many symbols which
may be arranged in finite or simple infinite sequences $c_0, c_1, \ldots$ and
$v_0, v_1, \ldots$ respectively, without repeating terms.  We will represent
arbitrary symbols by metavariables $\alpha$, $\beta$, etc.

{\footnotesize\begin{quotation}
{\em Comment.} The variables $v_0, v_1, \ldots$ of our formal system
correspond to what are usually considered ``metavariables'' in
descriptions of specific formal systems in the literature.  Typically,
when describing a specific formal system a book will postulate a set of
primitive objects called variables, then proceed to describe their
properties using metavariables that range over them, never mentioning
again the actual variables themselves.  Our formal system does not
mention these primitive variable objects at all but deals directly with
metavariables, as its primitive objects, from the start.  This is a
subtle but key distinction you should keep in mind, and it makes our
definition of ``formal system'' somewhat different from that typically
found in the literature.  (So, the $\alpha$, $\beta$, etc.\ above are
actually ``metametavariables'' when used to represent $v_0, v_1,
\ldots$.)
\end{quotation}}

Finite sequences all terms of which are symbols are called {\em
expressions}.\index{expression!in a formal system}  $\mbox{\em EX}$ is
the set of all expressions; thus
\begin{displaymath}
\mbox{\em EX} = \bigcup _{n \in \omega} {}^n \mbox{\em SM}.
\end{displaymath}

A {\em constant-prefixed expression}\index{constant-prefixed expression}
is an expression of non-zero length
whose first term is a constant.  We denote the set of all constant-prefixed
expressions by $\mbox{\em EX}_C = \{ e \in \mbox{\em EX} | ( |e| > 0 \wedge
e_0 \in \mbox{\em CN} ) \}$.

A {\em constant-variable pair}\index{constant-variable pair}
is an expression of length 2 whose first term
is a constant and whose second term is a variable.  We denote the set of all
constant-variable pairs by $\mbox{\em EX}_2 = \{ e \in \mbox{\em EX}_C | ( |e|
= 2 \wedge e_1 \in \mbox{\em VR} ) \}$.


{\footnotesize\begin{quotation}
{\em Relationship to Metamath.} In general, the set $\mbox{\em SM}$
corresponds to the set of declared math symbols in a Metamath database, the
set $\mbox{\em CN}$ to those declared with \texttt{\$c} statements, and the set
$\mbox{\em VR}$ to those declared with \texttt{\$v} statements.  Of course a
Metamath database can only have a finite number of math symbols, whereas
formal systems in general can have an infinite number, although the number of
Metamath math symbols available is in principle unlimited.

The set $\mbox{\em EX}_C$ corresponds to the set of permissible expressions
for \texttt{\$e}, \texttt{\$a}, and \texttt{\$p} statements.  The set $\mbox{\em EX}_2$
corresponds to the set of permissible expressions for \texttt{\$f} statements.
\end{quotation}}

We denote by ${\cal V}(e)$ the set of all variables in an expression $e \in
\mbox{\em EX}$, i.e.\ the set of all $\alpha \in \mbox{\em VR}$ such that
$\alpha = e_n$ for some $n < |e|$.  We also denote (with abuse of notation) by
${\cal V}(E)$ the set of all variables in a collection of expressions $E
\subseteq \mbox{\em EX}$, i.e.\ $\bigcup _{e \in E} {\cal V}(e)$.


\subsection{Substitution}

Given a function $F$ from $\mbox{\em VR}$ to
$\mbox{\em EX}$, we
denote by $\sigma_{F}$ or just $\sigma$ the function from $\mbox{\em EX}$ to
$\mbox{\em EX}$ defined recursively for nonempty sequences by
\begin{eqnarray*}
& \sigma(<\alpha>) = F(\alpha) & \mbox{for\ } \alpha \in \mbox{\em VR}; \\
& \sigma(<\alpha>) = <\alpha> & \mbox{for\ } \alpha \not\in \mbox{\em VR}; \\
& \sigma(g \frown h) = \sigma(g) \frown
    \sigma(h) & \mbox{for\ } g,h \in \mbox{\em EX}.
\end{eqnarray*}
We also define $\sigma(\varnothing)=\varnothing$.  We call $\sigma$ a {\em
simultaneous substitution}\index{substitution!variable}\index{variable
substitution} (or just {\em substitution}) with {\em substitution
map}\index{substitution map} $F$.

We also denote (with abuse of notation) by $\sigma(E)$ a substitution on a
collection of expressions $E \subseteq \mbox{\em EX}$, i.e.\ the set $\{
\sigma(e) | e \in E \}$.  The collection $\sigma(E)$ may of course contain
fewer expressions than $E$ because duplicate expressions could result from the
substitution.

\subsection{Statements}

We denote by $\mbox{\em DV}$ the set of all
unordered pairs $\{\alpha, \beta \} \subseteq \mbox{\em VR}$ such that $\alpha
\neq \beta$.  $\mbox{\em DV}$ stands for ``distinct variables.''

A {\em pre-statement}\index{pre-statement!in a formal system} is a
quadruple $\langle D,T,H,A \rangle$ such that
$D\subseteq \mbox{\em DV}$, $T\subseteq \mbox{\em EX}_2$, $H\subseteq
\mbox{\em EX}_C$ and $H$ is finite,
$A\in \mbox{\em EX}_C$, ${\cal V}(H\cup\{A\}) \subseteq
{\cal V}(T)$, and $\forall e,f\in T {\ } {\cal V}(e) \neq {\cal V}(f)$ (or
equivalently, $e_1 \ne f_1$) whenever $e \neq f$. The terms of the quadruple are called {\em
distinct-variable restrictions},\index{disjoint-variable restriction!in a
formal system} {\em variable-type hypotheses},\index{variable-type
hypothesis!in a formal system} {\em logical hypotheses},\index{logical
hypothesis!in a formal system} and the {\em assertion}\index{assertion!in a
formal system} respectively.  We denote by $T_M$ ({\em mandatory variable-type
hypotheses}\index{mandatory variable-type hypothesis!in a formal system}) the
subset of $T$ such that ${\cal V}(T_M) ={\cal V}(H \cup \{A\})$.  We denote by
$D_M=\{\{\alpha,\beta\}\in D|\{\alpha,\beta\}\subseteq {\cal V}(T_M)\}$ the
{\em mandatory distinct-variable restrictions}\index{mandatory
disjoint-variable restriction!in a formal system} of the pre-statement.
The set
of {\em mandatory hypotheses}\index{mandatory hypothesis!in a formal system}
is $T_M\cup H$.  We call the quadruple $\langle D_M,T_M,H,A \rangle$
the {\em reduct}\index{reduct!in a formal system} of
the pre-statement $\langle D,T,H,A \rangle$.

A {\em statement} is the reduct of some pre-statement\index{statement!in a
formal system}.  A statement is therefore a special kind of pre-statement;
in particular, a statement is the reduct of itself.

{\footnotesize\begin{quotation}
{\em Comment.}  $T$ is a set of expressions, each of length 2, that associate
a set of constants (``variable types'') with a set of variables.  The
condition ${\cal V}(H\cup\{A\}) \subseteq {\cal V}(T) $
means that each variable occurring in a statement's logical
hypotheses or assertion must have an associated variable-type hypothesis or
``type declaration,'' in  analogy to a computer programming language, where a
variable must be declared to be say, a string or an integer.  The requirement
that $\forall e,f\in T \, e_1 \ne f_1$ for $e\neq f$
means that each variable must be
associated with a unique constant designating its variable type; e.g., a
variable might be a ``wff'' or a ``set'' but not both.

Distinct-variable restrictions are used to specify what variable substitutions
are permissible to make for the statement to remain valid.  For example, in
the theorem scheme of set theory $\lnot\forall x\,x=y$ we may not substitute
the same variable for both $x$ and $y$.  On the other hand, the theorem scheme
$x=y\to y=x$ does not require that $x$ and $y$ be distinct, so we do not
require a distinct-variable restriction, although having one
would cause no harm other than making the scheme less general.

A mandatory variable-type hypothesis is one whose variable exists in a logical
hypothesis or the assertion.  A provable pre-statement
(defined below) may require
non-mandatory variable-type hypotheses that effectively introduce ``dummy''
variables for use in its proof.  Any number of dummy variables might
be required by a specific proof; indeed, it has been shown by H.\
Andr\'{e}ka\index{Andr{\'{e}}ka, H.} \cite{Nemeti} that there is no finite
upper bound to the number of dummy variables needed to prove an arbitrary
theorem in first-order logic (with equality) having a fixed number $n>2$ of
individual variables.  (See also the Comment on p.~\pageref{nodd}.)
For this reason we do not set a finite size bound on the collections $D$ and
$T$, although in an actual application (Metamath database) these will of
course be finite, increased to whatever size is necessary as more
proofs are added.
\end{quotation}}

{\footnotesize\begin{quotation}
{\em Relationship to Metamath.} A pre-statement of a formal system
corresponds to an extended frame in a Metamath database
(Section~\ref{frames}).  The collections $D$, $T$, and $H$ correspond
respectively to the \texttt{\$d}, \texttt{\$f}, and \texttt{\$e}
statement collections in an extended frame.  The expression $A$
corresponds to the \texttt{\$a} (or \texttt{\$p}) statement in an
extended frame.

A statement of a formal system corresponds to a frame in a Metamath
database.
\end{quotation}}

\subsection{Formal Systems}

A {\em formal system}\index{formal system} is a
triple $\langle \mbox{\em CN},\mbox{\em
VR},\Gamma\rangle$ where $\Gamma$ is a set of statements.  The members of
$\Gamma$ are called {\em axiomatic statements}.\index{axiomatic
statement!in a formal system}  Sometimes we will refer to a
formal system by just $\Gamma$ when $\mbox{\em CN}$ and $\mbox{\em VR}$ are
understood.

Given a formal system $\Gamma$, the {\em closure}\index{closure}\footnote{This
definition of closure incorporates a simplification due to
Josh Purinton.\index{Purinton, Josh}.} of a
pre-statement
$\langle D,T,H,A \rangle$ is the smallest set $C$ of expressions
such that:
%\begin{enumerate}
%  \item $T\cup H\subseteq C$; and
%  \item If for some axiomatic statement
%    $\langle D_M',T_M',H',A' \rangle \in \Gamma_A$, for
%    some $E \subseteq C$, some $F \subseteq C-T$ (where ``-'' denotes
%    set difference), and some substitution
%    $\sigma$ we have
%    \begin{enumerate}
%       \item $\sigma(T_M') = E$ (where, as above, the $M$ denotes the
%           mandatory variable-type hypotheses of $T^A$);
%       \item $\sigma(H') = F$;
%       \item for all $\{\alpha,\beta\}\in D^A$ and $\subseteq
%         {\cal V}(T_M')$, for all $\gamma\in {\cal V}(\sigma(\langle \alpha
%         \rangle))$, and for all $\delta\in  {\cal V}(\sigma(\langle \beta
%         \rangle))$, we have $\{\gamma, \delta\} \in D$;
%   \end{enumerate}
%   then $\sigma(A') \in C$.
%\end{enumerate}
\begin{list}{}{\itemsep 0.0pt}
  \item[1.] $T\cup H\subseteq C$; and
  \item[2.] If for some axiomatic statement
    $\langle D_M',T_M',H',A' \rangle \in
       \Gamma$ and for some substitution
    $\sigma$ we have
    \begin{enumerate}
       \item[a.] $\sigma(T_M' \cup H') \subseteq C$; and
       \item[b.] for all $\{\alpha,\beta\}\in D_M'$, for all $\gamma\in
         {\cal V}(\sigma(\langle \alpha
         \rangle))$, and for all $\delta\in  {\cal V}(\sigma(\langle \beta
         \rangle))$, we have $\{\gamma, \delta\} \in D$;
   \end{enumerate}
   then $\sigma(A') \in C$.
\end{list}
A pre-statement $\langle D,T,H,A
\rangle$ is {\em provable}\index{provable statement!in a formal
system} if $A\in C$ i.e.\ if its assertion belongs to its
closure.  A statement is {\em provable} if it is
the reduct of a provable pre-statement.
The {\em universe}\index{universe of a formal system}
of a formal system is
the collection of all of its provable statements.  Note that the
set of axiomatic statements $\Gamma$ in a formal system is a subset of its
universe.

{\footnotesize\begin{quotation}
{\em Comment.} The first condition in the definition of closure simply says
that the hypotheses of the pre-statement are in its closure.

Condition 2(a) says that a substitution exists that makes the
mandatory hypotheses of an axiomatic statement exactly match some members of
the closure.  This is what we explicitly demonstrate in a Metamath language
proof.

%Conditions 2(a) and 2(b) say that a substitution exists that makes the
%(mandatory) hypotheses of an axiomatic statement exactly match some members of
%the closure.  This is what we explicitly demonstrate with a Metamath language
%proof.
%
%The set of expressions $F$ in condition 2(b) excludes the variable-type
%hypotheses; this is done because non-mandatory variable-type hypotheses are
%effectively ``dropped'' as irrelevant whereas logical hypotheses must be
%retained to achieve a consistent logical system.

Condition 2(b) describes how distinct-variable restrictions in the axiomatic
statement must be met.  It means that after a substitution for two variables
that must be distinct, the resulting two expressions must either contain no
variables, or if they do, they may not have variables in common, and each pair
of any variables they do have, with one variable from each expression, must be
specified as distinct in the original statement.
\end{quotation}}

{\footnotesize\begin{quotation}
{\em Relationship to Metamath.} Axiomatic statements
 and provable statements in a formal
system correspond to the frames for \texttt{\$a} and \texttt{\$p} statements
respectively in a Metamath database.  The set of axiomatic statements is a
subset of the set of provable statements in a formal system, although in a
Metamath database a \texttt{\$a} statement is distinguished by not having a
proof.  A Metamath language proof for a \texttt{\$p} statement tells the computer
how to explicitly construct a series of members of the closure ultimately
leading to a demonstration that the assertion
being proved is in the closure.  The actual closure typically contains
an infinite number of expressions.  A formal system itself does not have
an explicit object called a ``proof'' but rather the existence of a proof
is implied indirectly by membership of an assertion in a provable
statement's closure.  We do this to make the formal system easier
to describe in the language of set theory.

We also note that once established as provable, a statement may be considered
to acquire the same status as an axiomatic statement, because if the set of
axiomatic statements is extended with a provable statement, the universe of
the formal system remains unchanged (provided that $\mbox{\em VR}$ is
infinite).
In practice, this means we can build a hierarchy of provable statements to
more efficiently establish additional provable statements.  This is
what we do in Metamath when we allow proofs to reference previous
\texttt{\$p} statements as well as previous \texttt{\$a} statements.
\end{quotation}}

\section{Examples of Formal Systems}

{\footnotesize\begin{quotation}
{\em Relationship to Metamath.} The examples in this section, except Example~2,
are for the most part exact equivalents of the development in the set
theory database \texttt{set.mm}.  You may want to compare Examples~1, 3, and 5
to Section~\ref{metaaxioms}, Example 4 to Sections~\ref{metadefprop} and
\ref{metadefpred}, and Example 6 to
Section~\ref{setdefinitions}.\label{exampleref}
\end{quotation}}

\subsection{Example~1---Propositional Calculus}\index{propositional calculus}

Classical propositional calculus can be described by the following formal
system.  We assume the set of variables is infinite.  Rather than denoting the
constants and variables by $c_0, c_1, \ldots$ and $v_0, v_1, \ldots$, for
readability we will instead use more conventional symbols, with the
understanding of course that they denote distinct primitive objects.
Also for readability we may omit commas between successive terms of a
sequence; thus $\langle \mbox{wff\ } \varphi\rangle$ denotes
$\langle \mbox{wff}, \varphi\rangle$.

Let
\begin{itemize}
  \item[] $\mbox{\em CN}=\{\mbox{wff}, \vdash, \to, \lnot, (,)\}$
  \item[] $\mbox{\em VR}=\{\varphi,\psi,\chi,\ldots\}$
  \item[] $T = \{\langle \mbox{wff\ } \varphi\rangle,
             \langle \mbox{wff\ } \psi\rangle,
             \langle \mbox{wff\ } \chi\rangle,\ldots\}$, i.e.\ those
             expressions of length 2 whose first member is $\mbox{\rm wff}$
             and whose second member belongs to $\mbox{\em VR}$.\footnote{For
convenience we let $T$ be an infinite set; the definition of a statement
permits this in principle.  Since a Metamath source file has a finite size, in
practice we must of course use appropriate finite subsets of this $T$,
specifically ones containing at least the mandatory variable-type
hypotheses.  Similarly, in the source file we introduce new variables as
required, with the understanding that a potentially infinite number of
them are available.}
\noindent Then $\Gamma$ consists of the axiomatic statements that
are the reducts of the following pre-statements:
    \begin{itemize}
      \item[] $\langle\varnothing,T,\varnothing,
               \langle \mbox{wff\ }(\varphi\to\psi)\rangle\rangle$
      \item[] $\langle\varnothing,T,\varnothing,
               \langle \mbox{wff\ }\lnot\varphi\rangle\rangle$
      \item[] $\langle\varnothing,T,\varnothing,
               \langle \vdash(\varphi\to(\psi\to\varphi))
               \rangle\rangle$
      \item[] $\langle\varnothing,T,
               \varnothing,
               \langle \vdash((\varphi\to(\psi\to\chi))\to
               ((\varphi\to\psi)\to(\varphi\to\chi)))
               \rangle\rangle$
      \item[] $\langle\varnothing,T,
               \varnothing,
               \langle \vdash((\lnot\varphi\to\lnot\psi)\to
               (\psi\to\varphi))\rangle\rangle$
      \item[] $\langle\varnothing,T,
               \{\langle\vdash(\varphi\to\psi)\rangle,
                 \langle\vdash\varphi\rangle\},
               \langle\vdash\psi\rangle\rangle$
    \end{itemize}
\end{itemize}

(For example, the reduct of $\langle\varnothing,T,\varnothing,
               \langle \mbox{wff\ }(\varphi\to\psi)\rangle\rangle$
is
\begin{itemize}
\item[] $\langle\varnothing,
\{\langle \mbox{wff\ } \varphi\rangle,
             \langle \mbox{wff\ } \psi\rangle\},
             \varnothing,
               \langle \mbox{wff\ }(\varphi\to\psi)\rangle\rangle$,
\end{itemize}
which is the first axiomatic statement.)

We call the members of $\mbox{\em VR}$ {\em wff variables} or (in the context
of first-order logic which we will describe shortly) {\em wff metavariables}.
Note that the symbols $\phi$, $\psi$, etc.\ denote actual specific members of
$\mbox{\em VR}$; they are not metavariables of our expository language (which
we denote with $\alpha$, $\beta$, etc.) but are instead (meta)constant symbols
(members of $\mbox{\em SM}$) from the point of view of our expository
language.  The equivalent system of propositional calculus described in
\cite{Tarski1965} also uses the symbols $\phi$, $\psi$, etc.\ to denote wff
metavariables, but in \cite{Tarski1965} unlike here those are metavariables of
the expository language and not primitive symbols of the formal system.

The first two statements define wffs: if $\varphi$ and $\psi$ are wffs, so is
$(\varphi \to \psi)$; if $\varphi$ is a wff, so is $\lnot\varphi$. The next
three are the axioms of propositional calculus: if $\varphi$ and $\psi$ are
wffs, then $\vdash (\varphi \to (\psi \to \varphi))$ is an (axiomatic)
theorem; etc. The
last is the rule of modus ponens: if $\varphi$ and $\psi$ are wffs, and
$\vdash (\varphi\to\psi)$ and $\vdash \varphi$ are theorems, then $\vdash
\psi$ is a theorem.

The correspondence to ordinary propositional calculus is as follows.  We
consider only provable statements of the form $\langle\varnothing,
T,\varnothing,A\rangle$ with $T$ defined as above.  The first term of the
assertion $A$ of any such statement is either ``wff'' or ``$\vdash$''.  A
statement for which the first term is ``wff'' is a {\em wff} of propositional
calculus, and one where the first term is ``$\vdash$'' is a {\em
theorem (scheme)} of propositional calculus.

The universe of this formal system also contains many other provable
statements.  Those with distinct-variable restrictions are irrelevant because
propositional calculus has no constraints on substitutions.  Those that have
logical hypotheses we call {\em inferences}\index{inference} when
the logical hypotheses are of the form
$\langle\vdash\rangle\frown w$ where $w$ is a wff (with the leading constant
term ``wff'' removed).  Inferences (other than the modus ponens rule) are not a
proper part of propositional calculus but are convenient to use when building a
hierarchy of provable statements.  A provable statement with a nonsense
hypothesis such as $\langle \to,\vdash,\lnot\rangle$, and this same expression
as its assertion, we consider irrelevant; no use can be made of it in
proving theorems, since there is no way to eliminate the nonsense hypothesis.

{\footnotesize\begin{quotation}
{\em Comment.} Our use of parentheses in the definition of a wff illustrates
how axiomatic statements should be carefully stated in a way that
ties in unambiguously with the substitutions allowed by the formal system.
There are many ways we could have defined wffs---for example, Polish
prefix notation would have allowed us to omit parentheses entirely, at
the expense of readability---but we must define them in a way that is
unambiguous.  For example, if we had omitted parentheses from the
definition of $(\varphi\to \psi)$, the wff $\lnot\varphi\to \psi$ could
be interpreted as either $\lnot(\varphi\to\psi)$ or $(\lnot\varphi\to\psi)$
and would have allowed us to prove nonsense.  Note that there is no
concept of operator binding precedence built into our formal system.
\end{quotation}}

\begin{sloppy}
\subsection{Example~2---Predicate Calculus with Equality}\index{predicate
calculus}
\end{sloppy}

Here we extend Example~1 to include predicate calculus with equality,
illustrating the use of distinct-variable restrictions.  This system is the
same as Tarski's system $\mathfrak{S}_2$ in \cite{Tarski1965} (except that the
axioms of propositional calculus are different but equivalent, and a redundant
axiom is omitted).  We extend $\mbox{\em CN}$ with the constants
$\{\mbox{var},\forall,=\}$.  We extend $\mbox{\em VR}$ with an infinite set of
{\em individual metavariables}\index{individual
metavariable} $\{x,y,z,\ldots\}$ and denote this subset
$\mbox{\em Vr}$.

We also join to $\mbox{\em CN}$ a possibly infinite set $\mbox{\em Pr}$ of {\em
predicates} $\{R,S,\ldots\}$.  We associate with $\mbox{\em Pr}$ a function
$\mbox{rnk}$ from $\mbox{\em Pr}$ to $\omega$, and for $\alpha\in \mbox{\em
Pr}$ we call $\mbox{rnk}(\alpha)$ the {\em rank} of the predicate $\alpha$,
which is simply the number of ``arguments'' that the predicate has.  (Most
applications of predicate calculus will have a finite number of predicates;
for example, set theory has the single two-argument or binary predicate $\in$,
which is usually written with its arguments surrounding the predicate symbol
rather than with the prefix notation we will use for the general case.)  As a
device to facilitate our discussion, we will let $\mbox{\em Vs}$ be any fixed
one-to-one function from $\omega$ to $\mbox{\em Vr}$; thus $\mbox{\em Vs}$ is
any simple infinite sequence of individual metavariables with no repeating
terms.

In this example we will not include the function symbols that are often part of
formalizations of predicate calculus.  Using metalogical arguments that are
beyond the scope of our discussion, it can be shown that our formalization is
equivalent when functions are introduced via appropriate definitions.

We extend the set $T$ defined in Example~1 with the expressions
$\{\langle \mbox{var\ } x\rangle,$ $ \langle \mbox{var\ } y\rangle, \langle
\mbox{var\ } z\rangle,\ldots\}$.  We extend the $\Gamma$ above
with the axiomatic statements that are the reducts of the following
pre-statements:
\begin{list}{}{\itemsep 0.0pt}
      \item[] $\langle\varnothing,T,\varnothing,
               \langle \mbox{wff\ }\forall x\,\varphi\rangle\rangle$
      \item[] $\langle\varnothing,T,\varnothing,
               \langle \mbox{wff\ }x=y\rangle\rangle$
      \item[] $\langle\varnothing,T,
               \{\langle\vdash\varphi\rangle\},
               \langle\vdash\forall x\,\varphi\rangle\rangle$
      \item[] $\langle\varnothing,T,\varnothing,
               \langle \vdash((\forall x(\varphi\to\psi)
                  \to(\forall x\,\varphi\to\forall x\,\psi))
               \rangle\rangle$
      \item[] $\langle\{\{x,\varphi\}\},T,\varnothing,
               \langle \vdash(\varphi\to\forall x\,\varphi)
               \rangle\rangle$
      \item[] $\langle\{\{x,y\}\},T,\varnothing,
               \langle \vdash\lnot\forall x\lnot x=y
               \rangle\rangle$
      \item[] $\langle\varnothing,T,\varnothing,
               \langle \vdash(x=z
                  \to(x=y\to z=y))
               \rangle\rangle$
      \item[] $\langle\varnothing,T,\varnothing,
               \langle \vdash(y=z
                  \to(x=y\to x=z))
               \rangle\rangle$
\end{list}
These are the axioms not involving predicate symbols. The first two statements
extend the definition of a wff.  The third is the rule of generalization.  The
fifth states, in effect, ``For a wff $\varphi$ and variable $x$,
$\vdash(\varphi\to\forall x\,\varphi)$, provided that $x$ does not occur in
$\varphi$.''  The sixth states ``For variables $x$ and $y$,
$\vdash\lnot\forall x\lnot x = y$, provided that $x$ and $y$ are distinct.''
(This proviso is not necessary but was included by Tarski to
weaken the axiom and still show that the system is logically complete.)

Finally, for each predicate symbol $\alpha\in \mbox{\em Pr}$, we add to
$\Gamma$ an axiomatic statement, extending the definition of wff,
that is the reduct of the following pre-statement:
\begin{displaymath}
    \langle\varnothing,T,\varnothing,
            \langle \mbox{wff},\alpha\rangle\
            \frown \mbox{\em Vs}\restriction\mbox{rnk}(\alpha)\rangle
\end{displaymath}
and for each $\alpha\in \mbox{\em Pr}$ and each $n < \mbox{rnk}(\alpha)$
we add to $\Gamma$ an equality axiom that is the reduct of the
following pre-statement:
\begin{eqnarray*}
    \lefteqn{\langle\varnothing,T,\varnothing,
            \langle
      \vdash,(,\mbox{\em Vs}_n,=,\mbox{\em Vs}_{\mbox{rnk}(\alpha)},\to,
            (,\alpha\rangle\frown \mbox{\em Vs}\restriction\mbox{rnk}(\alpha)} \\
  & & \frown
            \langle\to,\alpha\rangle\frown \mbox{\em Vs}\restriction n\frown
            \langle \mbox{\em Vs}_{\mbox{rnk}(\alpha)}\rangle \\
 & & \frown
            \mbox{\em Vs}\restriction(\mbox{rnk}(\alpha)\setminus(n+1))\frown
            \langle),)\rangle\rangle
\end{eqnarray*}
where $\restriction$ denotes function domain restriction and $\setminus$
denotes set difference.  Recall that a subscript on $\mbox{\em Vs}$
denotes one of its terms.  (In the above two axiom sets commas are placed
between successive terms of sequences to prevent ambiguity, and if you examine
them with care you will be able to distinguish those parentheses that denote
constant symbols from those of our expository language that delimit function
arguments.  Although it might have been better to use boldface for our
primitive symbols, unfortunately boldface was not available for all characters
on the \LaTeX\ system used to typeset this text.)  These seemingly forbidding
axioms can be understood by analogy to concatenation of substrings in a
computer language.  They are actually relatively simple for each specific case
and will become clearer by looking at the special case of a binary predicate
$\alpha = R$ where $\mbox{rnk}(R)=2$.  Letting $\mbox{\em Vs}$ be the sequence
$\langle x,y,z,\ldots\rangle$, the axioms we would add to $\Gamma$ for this
case would be the wff extension and two equality axioms that are the
reducts of the pre-statements:
\begin{list}{}{\itemsep 0.0pt}
      \item[] $\langle\varnothing,T,\varnothing,
               \langle \mbox{wff\ }R x y\rangle\rangle$
      \item[] $\langle\varnothing,T,\varnothing,
               \langle \vdash(x=z
                  \to(R x y \to R z y))
               \rangle\rangle$
      \item[] $\langle\varnothing,T,\varnothing,
               \langle \vdash(y=z
                  \to(R x y \to R x z))
               \rangle\rangle$
\end{list}
Study these carefully to see how the general axioms above evaluate to
them.  In practice, typically only a few special cases such as this would be
needed, and in any case the Metamath language will only permit us to describe
a finite number of predicates, as opposed to the infinite number permitted by
the formal system.  (If an infinite number should be needed for some reason,
we could not define the formal system directly in the Metamath language but
could instead define it metalogically under set theory as we
do in this appendix, and only the underlying set theory, with its single
binary predicate, would be defined directly in the Metamath language.)


{\footnotesize\begin{quotation}
{\em Comment.}  As we noted earlier, the specific variables denoted by the
symbols $x,y,z,\ldots\in \mbox{\em Vr}\subseteq \mbox{\em VR}\subseteq
\mbox{\em SM}$ in Example~2 are not the actual variables of ordinary predicate
calculus but should be thought of as metavariables ranging over them.  For
example, a distinct-variable restriction would be meaningless for actual
variables of ordinary predicate calculus since two different actual variables
are by definition distinct.  And when we talk about an arbitrary
representative $\alpha\in \mbox{\em Vr}$, $\alpha$ is a metavariable (in our
expository language) that ranges over metavariables (which are primitives of
our formal system) each of which ranges over the actual individual variables
of predicate calculus (which are never mentioned in our formal system).

The constant called ``var'' above is called \texttt{setvar} in the
\texttt{set.mm} database file, but it means the same thing.  I felt
that ``var'' is a more meaningful name in the context of predicate
calculus, whose use is not limited to set theory.  For consistency we
stick with the name ``var'' throughout this Appendix, even after set
theory is introduced.
\end{quotation}}

\subsection{Free Variables and Proper Substitution}\index{free variable}
\index{proper substitution}\index{substitution!proper}

Typical representations of mathematical axioms use concepts such
as ``free variable,'' ``bound variable,'' and ``proper substitution''
as primitive notions.
A free variable is a variable that
is not a parameter of any container expression.
A bound variable is the opposite of a free variable; it is a
a variable that has been bound in a container expression.
For example, in the expression $\forall x \varphi$ (for all $x$, $\varphi$
is true), the variable $x$
is bound within the for-all ($\forall$) expression.
It is possible to change one variable to another, and that process is called
``proper substitution.''
In most books, proper substitution has a somewhat complicated recursive
definition with multiple cases based on the occurrences of free and
bound variables.
You may consult
\cite[ch.\ 3--4]{Hamilton}\index{Hamilton, Alan G.} (as well as
many other texts) for more formal details about these terms.

Using these concepts as \texttt{primitives} creates complications
for computer implementations.

In the system of Example~2, there are no primitive notions of free variable
and proper substitution.  Tarski \cite{Tarski1965} shows that this system is
logically equivalent to the more typical textbook systems that do have these
primitive notions, if we introduce these notions with appropriate definitions
and metalogic.  We could also define axioms for such systems directly,
although the recursive definitions of free variable and proper substitution
would be messy and awkward to work with.  Instead, we mention two devices that
can be used in practice to mimic these notions.  (1) Instead of introducing
special notation to express (as a logical hypothesis) ``where $x$ is not free
in $\varphi$'' we can use the logical hypothesis $\vdash(\varphi\to\forall
x\,\varphi)$.\label{effectivelybound}\index{effectively
not free}\footnote{This is a slightly weaker requirement than ``where $x$ is
not free in $\varphi$.''  If we let $\varphi$ be $x=x$, we have the theorem
$(x=x\to\forall x\,x=x)$ which satisfies the hypothesis, even though $x$ is
free in $x=x$ .  In a case like this we say that $x$ is {\em effectively not
free}\index{effectively not free} in $x=x$, since $x=x$ is logically
equivalent to $\forall x\,x=x$ in which $x$ is bound.} (2) It can be shown
that the wff $((x=y\to\varphi)\wedge\exists x(x=y\wedge\varphi))$ (with the
usual definitions of $\wedge$ and $\exists$; see Example~4 below) is logically
equivalent to ``the wff that results from proper substitution of $y$ for $x$
in $\varphi$.''  This works whether or not $x$ and $y$ are distinct.

\subsection{Metalogical Completeness}\index{metalogical completeness}

In the system of Example~2, the
following are provable pre-statements (and their reducts are
provable statements):
\begin{eqnarray*}
      & \langle\{\{x,y\}\},T,\varnothing,
               \langle \vdash\lnot\forall x\lnot x=y
               \rangle\rangle & \\
     &  \langle\varnothing,T,\varnothing,
               \langle \vdash\lnot\forall x\lnot x=x
               \rangle\rangle &
\end{eqnarray*}
whereas the following pre-statement is not to my knowledge provable (but
in any case we will pretend it's not for sake of illustration):
\begin{eqnarray*}
     &  \langle\varnothing,T,\varnothing,
               \langle \vdash\lnot\forall x\lnot x=y
               \rangle\rangle &
\end{eqnarray*}
In other words, we can prove ``$\lnot\forall x\lnot x=y$ where $x$ and $y$ are
distinct'' and separately prove ``$\lnot\forall x\lnot x=x$'', but we can't
prove the combined general case ``$\lnot\forall x\lnot x=y$'' that has no
proviso.  Now this does not compromise logical completeness, because the
variables are really metavariables and the two provable cases together cover
all possible cases.  The third case can be considered a metatheorem whose
direct proof, using the system of Example~2, lies outside the capability of the
formal system.

Also, in the system of Example~2 the following pre-statement is not to my
knowledge provable (again, a conjecture that we will pretend to be the case):
\begin{eqnarray*}
     & \langle\varnothing,T,\varnothing,
               \langle \vdash(\forall x\, \varphi\to\varphi)
               \rangle\rangle &
\end{eqnarray*}
Instead, we can only prove specific cases of $\varphi$ involving individual
metavariables, and by induction on formula length, prove as a metatheorem
outside of our formal system the general statement above.  The details of this
proof are found in \cite{Kalish}.

There does, however, exist a system of predicate calculus in which all such
``simple metatheorems'' as those above can be proved directly, and we present
it in Example~3. A {\em simple metatheorem}\index{simple metatheorem}
is any statement of the formal
system of Example~2 where all distinct variable restrictions consist of either
two individual metavariables or an individual metavariable and a wff
metavariable, and which is provable by combining cases outside the system as
above.  A system is {\em metalogically complete}\index{metalogical
completeness} if all of its simple
metatheorems are (directly) provable statements. The precise definition of
``simple metatheorem'' and the proof of the ``metalogical completeness'' of
Example~3 is found in Remark 9.6 and Theorem 9.7 of \cite{Megill}.\index{Megill,
Norman}

\begin{sloppy}
\subsection{Example~3---Metalogically Complete Predicate
Calculus with
Equality}
\end{sloppy}

For simplicity we will assume there is one binary predicate $R$;
this system suffices for set theory, where the $R$ is of course the $\in$
predicate.  We label the axioms as they appear in \cite{Megill}.  This
system is logically equivalent to that of Example~2 (when the latter is
restricted to this single binary predicate) but is also metalogically
complete.\index{metalogical completeness}

Let
\begin{itemize}
  \item[] $\mbox{\em CN}=\{\mbox{wff}, \mbox{var}, \vdash, \to, \lnot, (,),\forall,=,R\}$.
  \item[] $\mbox{\em VR}=\{\varphi,\psi,\chi,\ldots\}\cup\{x,y,z,\ldots\}$.
  \item[] $T = \{\langle \mbox{wff\ } \varphi\rangle,
             \langle \mbox{wff\ } \psi\rangle,
             \langle \mbox{wff\ } \chi\rangle,\ldots\}\cup
       \{\langle \mbox{var\ } x\rangle, \langle \mbox{var\ } y\rangle, \langle
       \mbox{var\ }z\rangle,\ldots\}$.

\noindent Then
  $\Gamma$ consists of the reducts of the following pre-statements:
    \begin{itemize}
      \item[] $\langle\varnothing,T,\varnothing,
               \langle \mbox{wff\ }(\varphi\to\psi)\rangle\rangle$
      \item[] $\langle\varnothing,T,\varnothing,
               \langle \mbox{wff\ }\lnot\varphi\rangle\rangle$
      \item[] $\langle\varnothing,T,\varnothing,
               \langle \mbox{wff\ }\forall x\,\varphi\rangle\rangle$
      \item[] $\langle\varnothing,T,\varnothing,
               \langle \mbox{wff\ }x=y\rangle\rangle$
      \item[] $\langle\varnothing,T,\varnothing,
               \langle \mbox{wff\ }Rxy\rangle\rangle$
      \item[(C1$'$)] $\langle\varnothing,T,\varnothing,
               \langle \vdash(\varphi\to(\psi\to\varphi))
               \rangle\rangle$
      \item[(C2$'$)] $\langle\varnothing,T,
               \varnothing,
               \langle \vdash((\varphi\to(\psi\to\chi))\to
               ((\varphi\to\psi)\to(\varphi\to\chi)))
               \rangle\rangle$
      \item[(C3$'$)] $\langle\varnothing,T,
               \varnothing,
               \langle \vdash((\lnot\varphi\to\lnot\psi)\to
               (\psi\to\varphi))\rangle\rangle$
      \item[(C4$'$)] $\langle\varnothing,T,
               \varnothing,
               \langle \vdash(\forall x(\forall x\,\varphi\to\psi)\to
                 (\forall x\,\varphi\to\forall x\,\psi))\rangle\rangle$
      \item[(C5$'$)] $\langle\varnothing,T,
               \varnothing,
               \langle \vdash(\forall x\,\varphi\to\varphi)\rangle\rangle$
      \item[(C6$'$)] $\langle\varnothing,T,
               \varnothing,
               \langle \vdash(\forall x\forall y\,\varphi\to
                 \forall y\forall x\,\varphi)\rangle\rangle$
      \item[(C7$'$)] $\langle\varnothing,T,
               \varnothing,
               \langle \vdash(\lnot\varphi\to\forall x\lnot\forall x\,\varphi
                 )\rangle\rangle$
      \item[(C8$'$)] $\langle\varnothing,T,
               \varnothing,
               \langle \vdash(x=y\to(x=z\to y=z))\rangle\rangle$
      \item[(C9$'$)] $\langle\varnothing,T,
               \varnothing,
               \langle \vdash(\lnot\forall x\, x=y\to(\lnot\forall x\, x=z\to
                 (y=z\to\forall x\, y=z)))\rangle\rangle$
      \item[(C10$'$)] $\langle\varnothing,T,
               \varnothing,
               \langle \vdash(\forall x(x=y\to\forall x\,\varphi)\to
                 \varphi))\rangle\rangle$
      \item[(C11$'$)] $\langle\varnothing,T,
               \varnothing,
               \langle \vdash(\forall x\, x=y\to(\forall x\,\varphi
               \to\forall y\,\varphi))\rangle\rangle$
      \item[(C12$'$)] $\langle\varnothing,T,
               \varnothing,
               \langle \vdash(x=y\to(Rxz\to Ryz))\rangle\rangle$
      \item[(C13$'$)] $\langle\varnothing,T,
               \varnothing,
               \langle \vdash(x=y\to(Rzx\to Rzy))\rangle\rangle$
      \item[(C15$'$)] $\langle\varnothing,T,
               \varnothing,
               \langle \vdash(\lnot\forall x\, x=y\to(x=y\to(\varphi
                 \to\forall x(x=y\to\varphi))))\rangle\rangle$
      \item[(C16$'$)] $\langle\{\{x,y\}\},T,
               \varnothing,
               \langle \vdash(\forall x\, x=y\to(\varphi\to\forall x\,\varphi)
                 )\rangle\rangle$
      \item[(C5)] $\langle\{\{x,\varphi\}\},T,\varnothing,
               \langle \vdash(\varphi\to\forall x\,\varphi)
               \rangle\rangle$
      \item[(MP)] $\langle\varnothing,T,
               \{\langle\vdash(\varphi\to\psi)\rangle,
                 \langle\vdash\varphi\rangle\},
               \langle\vdash\psi\rangle\rangle$
      \item[(Gen)] $\langle\varnothing,T,
               \{\langle\vdash\varphi\rangle\},
               \langle\vdash\forall x\,\varphi\rangle\rangle$
    \end{itemize}
\end{itemize}

While it is known that these axioms are ``metalogically complete,'' it is
not known whether they are independent (i.e.\ none is
redundant) in the metalogical sense; specifically, whether any axiom (possibly
with additional non-mandatory distinct-variable restrictions, for use with any
dummy variables in its proof) is provable from the others.  Note that
metalogical independence is a weaker requirement than independence in the
usual logical sense.  Not all of the above axioms are logically independent:
for example, C9$'$ can be proved as a metatheorem from the others, outside the
formal system, by combining the possible cases of distinct variables.

\subsection{Example~4---Adding Definitions}\index{definition}
There are several ways to add definitions to a formal system.  Probably the
most proper way is to consider definitions not as part of the formal system at
all but rather as abbreviations that are part of the expository metalogic
outside the formal system.  For convenience, though, we may use the formal
system itself to incorporate definitions, adding them as axiomatic extensions
to the system.  This could be done by adding a constant representing the
concept ``is defined as'' along with axioms for it. But there is a nicer way,
at least in this writer's opinion, that introduces definitions as direct
extensions to the language rather than as extralogical primitive notions.  We
introduce additional logical connectives and provide axioms for them.  For
systems of logic such as Examples 1 through 3, the additional axioms must be
conservative in the sense that no wff of the original system that was not a
theorem (when the initial term ``wff'' is replaced by ``$\vdash$'' of course)
becomes a theorem of the extended system.  In this example we extend Example~3
(or 2) with standard abbreviations of logic.

We extend $\mbox{\em CN}$ of Example~3 with new constants $\{\leftrightarrow,
\wedge,\vee,\exists\}$, corresponding to logical equivalence,\index{logical
equivalence ($\leftrightarrow$)}\index{biconditional ($\leftrightarrow$)}
conjunction,\index{conjunction ($\wedge$)} disjunction,\index{disjunction
($\vee$)} and the existential quantifier.\index{existential quantifier
($\exists$)}  We extend $\Gamma$ with the axiomatic statements that are
the reducts of the following pre-statements:
\begin{list}{}{\itemsep 0.0pt}
      \item[] $\langle\varnothing,T,\varnothing,
               \langle \mbox{wff\ }(\varphi\leftrightarrow\psi)\rangle\rangle$
      \item[] $\langle\varnothing,T,\varnothing,
               \langle \mbox{wff\ }(\varphi\vee\psi)\rangle\rangle$
      \item[] $\langle\varnothing,T,\varnothing,
               \langle \mbox{wff\ }(\varphi\wedge\psi)\rangle\rangle$
      \item[] $\langle\varnothing,T,\varnothing,
               \langle \mbox{wff\ }\exists x\, \varphi\rangle\rangle$
  \item[] $\langle\varnothing,T,\varnothing,
     \langle\vdash ( ( \varphi \leftrightarrow \psi ) \to
     ( \varphi \to \psi ) )\rangle\rangle$
  \item[] $\langle\varnothing,T,\varnothing,
     \langle\vdash ((\varphi\leftrightarrow\psi)\to
    (\psi\to\varphi))\rangle\rangle$
  \item[] $\langle\varnothing,T,\varnothing,
     \langle\vdash ((\varphi\to\psi)\to(
     (\psi\to\varphi)\to(\varphi
     \leftrightarrow\psi)))\rangle\rangle$
  \item[] $\langle\varnothing,T,\varnothing,
     \langle\vdash (( \varphi \wedge \psi ) \leftrightarrow\neg ( \varphi
     \to \neg \psi )) \rangle\rangle$
  \item[] $\langle\varnothing,T,\varnothing,
     \langle\vdash (( \varphi \vee \psi ) \leftrightarrow (\neg \varphi
     \to \psi )) \rangle\rangle$
  \item[] $\langle\varnothing,T,\varnothing,
     \langle\vdash (\exists x \,\varphi\leftrightarrow
     \lnot \forall x \lnot \varphi)\rangle\rangle$
\end{list}
The first three logical axioms (statements containing ``$\vdash$'') introduce
and effectively define logical equivalence, ``$\leftrightarrow$''.  The last
three use ``$\leftrightarrow$'' to effectively mean ``is defined as.''

\subsection{Example~5---ZFC Set Theory}\index{ZFC set theory}

Here we add to the system of Example~4 the axioms of Zermelo--Fraenkel set
theory with Choice.  For convenience we make use of the
definitions in Example~4.

In the $\mbox{\em CN}$ of Example~4 (which extends Example~3), we replace the symbol $R$
with the symbol $\in$.
More explicitly, we remove from $\Gamma$ of Example~4 the three
axiomatic statements containing $R$ and replace them with the
reducts of the following:
\begin{list}{}{\itemsep 0.0pt}
      \item[] $\langle\varnothing,T,\varnothing,
               \langle \mbox{wff\ }x\in y\rangle\rangle$
      \item[] $\langle\varnothing,T,
               \varnothing,
               \langle \vdash(x=y\to(x\in z\to y\in z))\rangle\rangle$
      \item[] $\langle\varnothing,T,
               \varnothing,
               \langle \vdash(x=y\to(z\in x\to z\in y))\rangle\rangle$
\end{list}
Letting $D=\{\{\alpha,\beta\}\in \mbox{\em DV}\,|\alpha,\beta\in \mbox{\em
Vr}\}$ (in other words all individual variables must be distinct), we extend
$\Gamma$ with the ZFC axioms, called
\index{Axiom of Extensionality}
\index{Axiom of Replacement}
\index{Axiom of Union}
\index{Axiom of Power Sets}
\index{Axiom of Regularity}
\index{Axiom of Infinity}
\index{Axiom of Choice}
Extensionality, Replacement, Union, Power
Set, Regularity, Infinity, and Choice, that are the reducts of:
\begin{list}{}{\itemsep 0.0pt}
      \item[Ext] $\langle D,T,
               \varnothing,
               \langle\vdash (\forall x(x\in y\leftrightarrow x \in z)\to y
               =z) \rangle\rangle$
      \item[Rep] $\langle D,T,
               \varnothing,
               \langle\vdash\exists x ( \exists y \forall z (\varphi \to z = y
                        ) \to
                        \forall z ( z \in x \leftrightarrow \exists x ( x \in
                        y \wedge \forall y\,\varphi ) ) )\rangle\rangle$
      \item[Un] $\langle D,T,
               \varnothing,
               \langle\vdash \exists x \forall y ( \exists x ( y \in x \wedge
               x \in z ) \to y \in x ) \rangle\rangle$
      \item[Pow] $\langle D,T,
               \varnothing,
               \langle\vdash \exists x \forall y ( \forall x ( x \in y \to x
               \in z ) \to y \in x ) \rangle\rangle$
      \item[Reg] $\langle D,T,
               \varnothing,
               \langle\vdash (  x \in y \to
                 \exists x ( x \in y \wedge \forall z ( z \in x \to \lnot z
                \in y ) ) ) \rangle\rangle$
      \item[Inf] $\langle D,T,
               \varnothing,
               \langle\vdash \exists x(y\in x\wedge\forall y(y\in
               x\to
               \exists z(y \in z\wedge z\in x))) \rangle\rangle$
      \item[AC] $\langle D,T,
               \varnothing,
               \langle\vdash \exists x \forall y \forall z ( ( y \in z
               \wedge z \in w ) \to \exists w \forall y ( \exists w
              ( ( y \in z \wedge z \in w ) \wedge ( y \in w \wedge w \in x
              ) ) \leftrightarrow y = w ) ) \rangle\rangle$
\end{list}

\subsection{Example~6---Class Notation in Set Theory}\label{class}

A powerful device that makes set theory easier (and that we have
been using all along in our informal expository language) is {\em class
abstraction notation}.\index{class abstraction}\index{abstraction class}  The
definitions we introduce are rigorously justified
as conservative by Takeuti and Zaring \cite{Takeuti}\index{Takeuti, Gaisi} or
Quine \cite{Quine}\index{Quine, Willard Van Orman}.  The key idea is to
introduce the notation $\{x|\mbox{---}\}$ which means ``the class of all $x$
such that ---'' for abstraction classes and introduce (meta)variables that
range over them.  An abstraction class may or may not be a set, depending on
whether it exists (as a set).  A class that does not exist is
called a {\em proper class}.\index{proper class}\index{class!proper}

To illustrate the use of abstraction classes we will provide some examples
of definitions that make use of them:  the empty set, class union, and
unordered pair.  Many other such definitions can be found in the
Metamath set theory database,
\texttt{set.mm}.\index{set theory database (\texttt{set.mm})}

% We intentionally break up the sequence of math symbols here
% because otherwise the overlong line goes beyond the page in narrow mode.
We extend $\mbox{\em CN}$ of Example~5 with new symbols $\{$
$\mbox{class},$ $\{,$ $|,$ $\},$ $\varnothing,$ $\cup,$ $,$ $\}$
where the inner braces and last comma are
constant symbols. (As before,
our dual use of some mathematical symbols for both our expository
language and as primitives of the formal system should be clear from context.)

We extend $\mbox{\em VR}$ of Example~5 with a set of {\em class
variables}\index{class variable}
$\{A,B,C,\ldots\}$. We extend the $T$ of Example~5 with $\{\langle
\mbox{class\ } A\rangle, \langle \mbox{class\ }B\rangle, \langle \mbox{class\ }
C\rangle,\ldots\}$.

To
introduce our definitions,
we add to $\Gamma$ of Example~5 the axiomatic statements
that are the reducts of the following pre-statements:
\begin{list}{}{\itemsep 0.0pt}
      \item[] $\langle\varnothing,T,\varnothing,
               \langle \mbox{class\ }x\rangle\rangle$
      \item[] $\langle\varnothing,T,\varnothing,
               \langle \mbox{class\ }\{x|\varphi\}\rangle\rangle$
      \item[] $\langle\varnothing,T,\varnothing,
               \langle \mbox{wff\ }A=B\rangle\rangle$
      \item[] $\langle\varnothing,T,\varnothing,
               \langle \mbox{wff\ }A\in B\rangle\rangle$
      \item[Ab] $\langle\varnothing,T,\varnothing,
               \langle \vdash ( y \in \{ x |\varphi\} \leftrightarrow
                  ( ( x = y \to\varphi) \wedge \exists x ( x = y
                  \wedge\varphi) ))
               \rangle\rangle$
      \item[Eq] $\langle\{\{x,A\},\{x,B\}\},T,\varnothing,
               \langle \vdash ( A = B \leftrightarrow
               \forall x ( x \in A \leftrightarrow x \in B ) )
               \rangle\rangle$
      \item[El] $\langle\{\{x,A\},\{x,B\}\},T,\varnothing,
               \langle \vdash ( A \in B \leftrightarrow \exists x
               ( x = A \wedge x \in B ) )
               \rangle\rangle$
\end{list}
Here we say that an individual variable is a class; $\{x|\varphi\}$ is a
class; and we extend the definition of a wff to include class equality and
membership.  Axiom Ab defines membership of a variable in a class abstraction;
the right-hand side can be read as ``the wff that results from proper
substitution of $y$ for $x$ in $\varphi$.''\footnote{Note that this definition
makes unnecessary the introduction of a separate notation similar to
$\varphi(x|y)$ for proper substitution, although we may choose to do so to be
conventional.  Incidentally, $\varphi(x|y)$ as it stands would be ambiguous in
the formal systems of our examples, since we wouldn't know whether
$\lnot\varphi(x|y)$ meant $\lnot(\varphi(x|y))$ or $(\lnot\varphi)(x|y)$.
Instead, we would have to use an unambiguous variant such as $(\varphi\,
x|y)$.}  Axioms Eq and El extend the meaning of the existing equality and
membership connectives.  This is potentially dangerous and requires careful
justification.  For example, from Eq we can derive the Axiom of Extensionality
with predicate logic alone; thus in principle we should include the Axiom of
Extensionality as a logical hypothesis.  However we do not bother to do this
since we have already presupposed that axiom earlier. The distinct variable
restrictions should be read ``where $x$ does not occur in $A$ or $B$.''  We
typically do this when the right-hand side of a definition involves an
individual variable not in the expression being defined; it is done so that
the right-hand side remains independent of the particular ``dummy'' variable
we use.

We continue to add to $\Gamma$ the following definitions
(i.e. the reducts of the following pre-statements) for empty
set,\index{empty set} class union,\index{union} and unordered
pair.\index{unordered pair}  They should be self-explanatory.  Analogous to our
use of ``$\leftrightarrow$'' to define new wffs in Example~4, we use ``$=$''
to define new abstraction terms, and both may be read informally as ``is
defined as'' in this context.
\begin{list}{}{\itemsep 0.0pt}
      \item[] $\langle\varnothing,T,\varnothing,
               \langle \mbox{class\ }\varnothing\rangle\rangle$
      \item[] $\langle\varnothing,T,\varnothing,
               \langle \vdash \varnothing = \{ x | \lnot x = x \}
               \rangle\rangle$
      \item[] $\langle\varnothing,T,\varnothing,
               \langle \mbox{class\ }(A\cup B)\rangle\rangle$
      \item[] $\langle\{\{x,A\},\{x,B\}\},T,\varnothing,
               \langle \vdash ( A \cup B ) = \{ x | ( x \in A \vee x \in B ) \}
               \rangle\rangle$
      \item[] $\langle\varnothing,T,\varnothing,
               \langle \mbox{class\ }\{A,B\}\rangle\rangle$
      \item[] $\langle\{\{x,A\},\{x,B\}\},T,\varnothing,
               \langle \vdash \{ A , B \} = \{ x | ( x = A \vee x = B ) \}
               \rangle\rangle$
\end{list}

\section{Metamath as a Formal System}\label{theorymm}

This section presupposes a familiarity with the Metamath computer language.

Our theory describes formal systems and their universes.  The Metamath
language provides a way of representing these set-theoretical objects to
a computer.  A Metamath database, being a finite set of {\sc ascii}
characters, can usually describe only a subset of a formal system and
its universe, which are typically infinite.  However the database can
contain as large a finite subset of the formal system and its universe
as we wish.  (Of course a Metamath set theory database can, in
principle, indirectly describe an entire infinite formal system by
formalizing the expository language in this Appendix.)

For purpose of our discussion, we assume the Metamath database
is in the simple form described on p.~\pageref{framelist},
consisting of all constant and variable declarations at the beginning,
followed by a sequence of extended frames each
delimited by \texttt{\$\char`\{} and \texttt{\$\char`\}}.  Any Metamath database can
be converted to this form, as described on p.~\pageref{frameconvert}.

The math symbol tokens of a Metamath source file, which are declared
with \texttt{\$c} and \texttt{\$v} statements, are names we assign to
representatives of $\mbox{\em CN}$ and $\mbox{\em VR}$.  For
definiteness we could assume that the first math symbol declared as a
variable corresponds to $v_0$, the second to $v_1$, etc., although the
exact correspondence we choose is not important.

In the Metamath language, each \texttt{\$d}, \texttt{\$f}, and
 \texttt{\$e} source
statement in an extended frame (Section~\ref{frames})
corresponds respectively to a member of the
collections $D$, $T$, and $H$ in a formal system statement $\langle
D_M,T_M,H,A\rangle$.  The math symbol strings following these Metamath keywords
correspond to a variable pair (in the case of \texttt{\$d}) or an expression (for
the other two keywords). The math symbol string following a \texttt{\$a} source
statement corresponds to expression $A$ in an axiomatic statement of the
formal system; the one following a \texttt{\$p} source statement corresponds to
$A$ in a provable statement that is not axiomatic.  In other words, each
extended frame in a Metamath database corresponds to
a pre-statement of the formal system, and a frame corresponds to
a statement of the formal system.  (Don't confuse the two meanings of
``statement'' here.  A statement of the formal system corresponds to the
several statements in a Metamath database that may constitute a
frame.)

In order for the computer to verify that a formal system statement is
provable, each \texttt{\$p} source statement is accompanied by a proof.
However, the proof does not correspond to anything in the formal system
but is simply a way of communicating to the computer the information
needed for its verification.  The proof tells the computer {\em how to
construct} specific members of closure of the formal system
pre-statement corresponding to the extended frame of the \texttt{\$p}
statement.  The final result of the construction is the member of the
closure that matches the \texttt{\$p} statement.  The abstract formal
system, on the other hand, is concerned only with the {\em existence} of
members of the closure.

As mentioned on p.~\pageref{exampleref}, Examples 1 and 3--6 in the
previous Section parallel the development of logic and set theory in the
Metamath database
\texttt{set.mm}.\index{set theory database (\texttt{set.mm})} You may
find it instructive to compare them.


\chapter{The MIU System}
\label{MIU}
\index{formal system}
\index{MIU-system}

The following is a listing of the file \texttt{miu.mm}.  It is self-explanatory.

%%%%%%%%%%%%%%%%%%%%%%%%%%%%%%%%%%%%%%%%%%%%%%%%%%%%%%%%%%%%

\begin{verbatim}
$( The MIU-system:  A simple formal system $)

$( Note:  This formal system is unusual in that it allows
empty wffs.  To work with a proof, you must type
SET EMPTY_SUBSTITUTION ON before using the PROVE command.
By default, this is OFF in order to reduce the number of
ambiguous unification possibilities that have to be selected
during the construction of a proof.  $)

$(
Hofstadter's MIU-system is a simple example of a formal
system that illustrates some concepts of Metamath.  See
Douglas R. Hofstadter, _Goedel, Escher, Bach:  An Eternal
Golden Braid_ (Vintage Books, New York, 1979), pp. 33ff. for
a description of the MIU-system.

The system has 3 constant symbols, M, I, and U.  The sole
axiom of the system is MI. There are 4 rules:
     Rule I:  If you possess a string whose last letter is I,
     you can add on a U at the end.
     Rule II:  Suppose you have Mx.  Then you may add Mxx to
     your collection.
     Rule III:  If III occurs in one of the strings in your
     collection, you may make a new string with U in place
     of III.
     Rule IV:  If UU occurs inside one of your strings, you
     can drop it.
Unfortunately, Rules III and IV do not have unique results:
strings could have more than one occurrence of III or UU.
This requires that we introduce the concept of an "MIU
well-formed formula" or wff, which allows us to construct
unique symbol sequences to which Rules III and IV can be
applied.
$)

$( First, we declare the constant symbols of the language.
Note that we need two symbols to distinguish the assertion
that a sequence is a wff from the assertion that it is a
theorem; we have arbitrarily chosen "wff" and "|-". $)
      $c M I U |- wff $. $( Declare constants $)

$( Next, we declare some variables. $)
     $v x y $.

$( Throughout our theory, we shall assume that these
variables represent wffs. $)
 wx   $f wff x $.
 wy   $f wff y $.

$( Define MIU-wffs.  We allow the empty sequence to be a
wff. $)

$( The empty sequence is a wff. $)
 we   $a wff $.
$( "M" after any wff is a wff. $)
 wM   $a wff x M $.
$( "I" after any wff is a wff. $)
 wI   $a wff x I $.
$( "U" after any wff is a wff. $)
 wU   $a wff x U $.

$( Assert the axiom. $)
 ax   $a |- M I $.

$( Assert the rules. $)
 ${
   Ia   $e |- x I $.
$( Given any theorem ending with "I", it remains a theorem
if "U" is added after it.  (We distinguish the label I_
from the math symbol I to conform to the 24-Jun-2006
Metamath spec.) $)
   I_    $a |- x I U $.
 $}
 ${
IIa  $e |- M x $.
$( Given any theorem starting with "M", it remains a theorem
if the part after the "M" is added again after it. $)
   II   $a |- M x x $.
 $}
 ${
   IIIa $e |- x I I I y $.
$( Given any theorem with "III" in the middle, it remains a
theorem if the "III" is replaced with "U". $)
   III  $a |- x U y $.
 $}
 ${
   IVa  $e |- x U U y $.
$( Given any theorem with "UU" in the middle, it remains a
theorem if the "UU" is deleted. $)
   IV   $a |- x y $.
  $}

$( Now we prove the theorem MUIIU.  You may be interested in
comparing this proof with that of Hofstadter (pp. 35 - 36).
$)
 theorem1  $p |- M U I I U $=
      we wM wU wI we wI wU we wU wI wU we wM we wI wU we wM
      wI wI wI we wI wI we wI ax II II I_ III II IV $.
\end{verbatim}\index{well-formed formula (wff)}

The \texttt{show proof /lemmon/renumber} command
yields the following display.  It is very similar
to the one in \cite[pp.~35--36]{Hofstadter}.\index{Hofstadter, Douglas R.}

\begin{verbatim}
1 ax             $a |- M I
2 1 II           $a |- M I I
3 2 II           $a |- M I I I I
4 3 I_           $a |- M I I I I U
5 4 III          $a |- M U I U
6 5 II           $a |- M U I U U I U
7 6 IV           $a |- M U I I U
\end{verbatim}

We note that Hofstadter's ``MU-puzzle,'' which asks whether
MU is a theorem of the MIU-system, cannot be answered using
the system above because the MU-puzzle is a question {\em
about} the system.  To prove the answer to the MU-puzzle,
a much more elaborate system is needed, namely one that
models the MIU-system within set theory.  (Incidentally, the
answer to the MU-puzzle is no.)

\chapter{Metamath Language EBNF}%
\label{BNF}%
\index{Metamath Language EBNF}

The following is a formal description of the basic Metamath language syntax
(with compressed proofs and support for unknown proof steps).
It is defined using the
Extended Backus--Naur Form (EBNF)\index{Extended Backus--Naur Form}\index{EBNF}
notation from W3C\index{W3C}
\textit{Extensible Markup Language (XML) 1.0 (Fifth Edition)}
(W3C Recommendation 26 November 2008) at
\url{https://www.w3.org/TR/xml/#sec-notation}.

The \texttt{database}
rule is processed until the end of the file (\texttt{EOF}).
The rules eventually require reading whitespace-separated tokens.
A token has an upper-case definition (see below)
or is a string constant in a non-token (such as \texttt{'\$a'}).
We intend for this to be correct, but if there is a conflict the
rules of section \ref{spec} govern. That section also discusses
non-syntax restrictions not shown here
(e.g., that each new label token
defined in a \texttt{hypothesis-stmt} or \texttt{assert-stmt}
must be unique).

\begin{verbatim}
database ::= outermost-scope-stmt*

outermost-scope-stmt ::=
  include-stmt | constant-stmt | stmt

/* File inclusion command; process file as a database.
   Databases should NOT have a comment in the filename. */
include-stmt ::= '$[' filename '$]'

/* Constant symbols declaration. */
constant-stmt ::= '$c' constant+ '$.'

/* A normal statement can occur in any scope. */
stmt ::= block | variable-stmt | disjoint-stmt |
  hypothesis-stmt | assert-stmt

/* A block. You can have 0 statements in a block. */
block ::= '${' stmt* '$}'

/* Variable symbols declaration. */
variable-stmt ::= '$v' variable+ '$.'

/* Disjoint variables. Simple disjoint statements have
   2 variables, i.e., "variable*" is empty for them. */
disjoint-stmt ::= '$d' variable variable variable* '$.'

hypothesis-stmt ::= floating-stmt | essential-stmt

/* Floating (variable-type) hypothesis. */
floating-stmt ::= LABEL '$f' typecode variable '$.'

/* Essential (logical) hypothesis. */
essential-stmt ::= LABEL '$e' typecode MATH-SYMBOL* '$.'

assert-stmt ::= axiom-stmt | provable-stmt

/* Axiomatic assertion. */
axiom-stmt ::= LABEL '$a' typecode MATH-SYMBOL* '$.'

/* Provable assertion. */
provable-stmt ::= LABEL '$p' typecode MATH-SYMBOL*
  '$=' proof '$.'

/* A proof. Proofs may be interspersed by comments.
   If '?' is in a proof it's an "incomplete" proof. */
proof ::= uncompressed-proof | compressed-proof
uncompressed-proof ::= (LABEL | '?')+
compressed-proof ::= '(' LABEL* ')' COMPRESSED-PROOF-BLOCK+

typecode ::= constant

filename ::= MATH-SYMBOL /* No whitespace or '$' */
constant ::= MATH-SYMBOL
variable ::= MATH-SYMBOL
\end{verbatim}

\needspace{2\baselineskip}
A \texttt{frame} is a sequence of 0 or more
\texttt{disjoint-{\allowbreak}stmt} and
\texttt{hypotheses-{\allowbreak}stmt} statements
(possibly interleaved with other non-\texttt{assert-stmt} statements)
followed by one \texttt{assert-stmt}.

\needspace{3\baselineskip}
Here are the rules for lexical processing (tokenization) beyond
the constant tokens shown above.
By convention these tokenization rules have upper-case names.
Every token is read for the longest possible length.
Whitespace-separated tokens are read sequentially;
note that the separating whitespace and \texttt{\$(} ... \texttt{\$)}
comments are skipped.

If a token definition uses another token definition, the whole thing
is considered a single token.
A pattern that is only part of a full token has a name beginning
with an underscore (``\_'').
An implementation could tokenize many tokens as a
\texttt{PRINTABLE-SEQUENCE}
and then check if it meets the more specific rule shown here.

Comments do not nest, and both \texttt{\$(} and \texttt{\$)}
have to be surrounded
by at least one whitespace character (\texttt{\_WHITECHAR}).
Technically comments end without consuming the trailing
\texttt{\_WHITECHAR}, but the trailing
\texttt{\_WHITECHAR} gets ignored anyway so we ignore that detail here.
Metamath language processors
are not required to support \texttt{\$)} followed
immediately by a bare end-of-file, because the closing
comment symbol is supposed to be followed by a
\texttt{\_WHITECHAR} such as a newline.

\begin{verbatim}
PRINTABLE-SEQUENCE ::= _PRINTABLE-CHARACTER+

MATH-SYMBOL ::= (_PRINTABLE-CHARACTER - '$')+

/* ASCII non-whitespace printable characters */
_PRINTABLE-CHARACTER ::= [#x21-#x7e]

LABEL ::= ( _LETTER-OR-DIGIT | '.' | '-' | '_' )+

_LETTER-OR-DIGIT ::= [A-Za-z0-9]

COMPRESSED-PROOF-BLOCK ::= ([A-Z] | '?')+

/* Define whitespace between tokens. The -> SKIP
   means that when whitespace is seen, it is
   skipped and we simply read again. */
WHITESPACE ::= (_WHITECHAR+ | _COMMENT) -> SKIP

/* Comments. $( ... $) and do not nest. */
_COMMENT ::= '$(' (_WHITECHAR+ (PRINTABLE-SEQUENCE - '$)'))*
  _WHITECHAR+ '$)' _WHITECHAR

/* Whitespace: (' ' | '\t' | '\r' | '\n' | '\f') */
_WHITECHAR ::= [#x20#x09#x0d#x0a#x0c]
\end{verbatim}
% This EBNF was developed as a collaboration between
% David A. Wheeler\index{Wheeler, David A.},
% Mario Carneiro\index{Carneiro, Mario}, and
% Benoit Jubin\index{Jubin, Benoit}, inspired by a request
% (and a lot of initial work) by Benoit Jubin.
%
% \chapter{Disclaimer and Trademarks}
%
% Information in this document is subject to change without notice and does not
% represent a commitment on the part of Norman Megill.
% \vspace{2ex}
%
% \noindent Norman D. Megill makes no warranties, either express or implied,
% regarding the Metamath computer software package.
%
% \vspace{2ex}
%
% \noindent Any trademarks mentioned in this book are the property of
% their respective owners.  The name ``Metamath'' is a trademark of
% Norman Megill.
%
\cleardoublepage
\phantomsection  % fixes the link anchor
\addcontentsline{toc}{chapter}{\bibname}

\bibliography{metamath}
%% metamath.tex - Version of 2-Jun-2019
% If you change the date above, also change the "Printed date" below.
% SPDX-License-Identifier: CC0-1.0
%
%                              PUBLIC DOMAIN
%
% This file (specifically, the version of this file with the above date)
% has been released into the Public Domain per the
% Creative Commons CC0 1.0 Universal (CC0 1.0) Public Domain Dedication
% https://creativecommons.org/publicdomain/zero/1.0/
%
% The public domain release applies worldwide.  In case this is not
% legally possible, the right is granted to use the work for any purpose,
% without any conditions, unless such conditions are required by law.
%
% Several short, attributed quotations from copyrighted works
% appear in this file under the ``fair use'' provision of Section 107 of
% the United States Copyright Act (Title 17 of the {\em United States
% Code}).  The public-domain status of this file is not applicable to
% those quotations.
%
% Norman Megill - email: nm(at)alum(dot)mit(dot)edu
%
% David A. Wheeler also donates his improvements to this file to the
% public domain per the CC0.  He works at the Institute for Defense Analyses
% (IDA), but IDA has agreed that this Metamath work is outside its "lane"
% and is not a work by IDA.  This was specifically confirmed by
% Margaret E. Myers (Division Director of the Information Technology
% and Systems Division) on 2019-05-24 and by Ben Lindorf (General Counsel)
% on 2019-05-22.

% This file, 'metamath.tex', is self-contained with everything needed to
% generate the the PDF file 'metamath.pdf' (the _Metamath_ book) on
% standard LaTeX 2e installations.  The auxiliary files are embedded with
% "filecontents" commands.  To generate metamath.pdf file, run these
% commands under Linux or Cygwin in the directory that contains
% 'metamath.tex':
%
%   rm -f realref.sty metamath.bib
%   touch metamath.ind
%   pdflatex metamath
%   pdflatex metamath
%   bibtex metamath
%   makeindex metamath
%   pdflatex metamath
%   pdflatex metamath
%
% The warnings that occur in the initial runs of pdflatex can be ignored.
% For the final run,
%
%   egrep -i 'error|warn' metamath.log
%
% should show exactly these 5 warnings:
%
%   LaTeX Warning: File `realref.sty' already exists on the system.
%   LaTeX Warning: File `metamath.bib' already exists on the system.
%   LaTeX Font Warning: Font shape `OMS/cmtt/m/n' undefined
%   LaTeX Font Warning: Font shape `OMS/cmtt/bx/n' undefined
%   LaTeX Font Warning: Some font shapes were not available, defaults
%       substituted.
%
% Search for "Uncomment" below if you want to suppress hyperlink boxes
% in the PDF output file
%
% TYPOGRAPHICAL NOTES:
% * It is customary to use an en dash (--) to "connect" names of different
%   people (and to denote ranges), and use a hyphen (-) for a
%   single compound name. Examples of connected multiple people are
%   Zermelo--Fraenkel, Schr\"{o}der--Bernstein, Tarski--Grothendieck,
%   Hewlett--Packard, and Backus--Naur.  Examples of a single person with
%   a compound name include Levi-Civita, Mittag-Leffler, and Burali-Forti.
% * Use non-breaking spaces after page abbreviations, e.g.,
%   p.~\pageref{note2002}.
%
% --------------------------- Start of realref.sty -----------------------------
\begin{filecontents}{realref.sty}
% Save the following as realref.sty.
% You can then use it with \usepackage{realref}
%
% This has \pageref jumping to the page on which the ref appears,
% \ref jumping to the point of the anchor, and \sectionref
% jumping to the start of section.
%
% Author:  Anthony Williams
%          Software Engineer
%          Nortel Networks Optical Components Ltd
% Date:    9 Nov 2001 (posted to comp.text.tex)
%
% The following declaration was made by Anthony Williams on
% 24 Jul 2006 (private email to Norman Megill):
%
%   ``I hereby donate the code for realref.sty posted on the
%   comp.text.tex newsgroup on 9th November 2001, accessible from
%   http://groups.google.com/group/comp.text.tex/msg/5a0e1cc13ea7fbb2
%   to the public domain.''
%
\ProvidesPackage{realref}
\RequirePackage[plainpages=false,pdfpagelabels=true]{hyperref}
\def\realref@anchorname{}
\AtBeginDocument{%
% ensure every label is a possible hyperlink target
\let\realref@oldrefstepcounter\refstepcounter%
\DeclareRobustCommand{\refstepcounter}[1]{\realref@oldrefstepcounter{#1}
\edef\realref@anchorname{\string #1.\@currentlabel}%
}%
\let\realref@oldlabel\label%
\DeclareRobustCommand{\label}[1]{\realref@oldlabel{#1}\hypertarget{#1}{}%
\@bsphack\protected@write\@auxout{}{%
    \string\expandafter\gdef\protect\csname
    page@num.#1\string\endcsname{\thepage}%
    \string\expandafter\gdef\protect\csname
    ref@num.#1\string\endcsname{\@currentlabel}%
    \string\expandafter\gdef\protect\csname
    sectionref@name.#1\string\endcsname{\realref@anchorname}%
}\@esphack}%
\DeclareRobustCommand\pageref[1]{{\edef\a{\csname
            page@num.#1\endcsname}\expandafter\hyperlink{page.\a}{\a}}}%
\DeclareRobustCommand\ref[1]{{\edef\a{\csname
            ref@num.#1\endcsname}\hyperlink{#1}{\a}}}%
\DeclareRobustCommand\sectionref[1]{{\edef\a{\csname
            ref@num.#1\endcsname}\edef\b{\csname
            sectionref@name.#1\endcsname}\hyperlink{\b}{\a}}}%
}
\end{filecontents}
% ---------------------------- End of realref.sty ------------------------------

% --------------------------- Start of metamath.bib -----------------------------
\begin{filecontents}{metamath.bib}
@book{Albers, editor = "Donald J. Albers and G. L. Alexanderson",
  title = "Mathematical People",
  publisher = "Contemporary Books, Inc.",
  address = "Chicago",
  note = "[QA28.M37]",
  year = 1985 }
@book{Anderson, author = "Alan Ross Anderson and Nuel D. Belnap",
  title = "Entailment",
  publisher = "Princeton University Press",
  address = "Princeton",
  volume = 1,
  note = "[QA9.A634 1975 v.1]",
  year = 1975}
@book{Barrow, author = "John D. Barrow",
  title = "Theories of Everything:  The Quest for Ultimate Explanation",
  publisher = "Oxford University Press",
  address = "Oxford",
  note = "[Q175.B225]",
  year = 1991 }
@book{Behnke,
  editor = "H. Behnke and F. Backmann and K. Fladt and W. S{\"{u}}ss",
  title = "Fundamentals of Mathematics",
  volume = "I",
  publisher = "The MIT Press",
  address = "Cambridge, Massachusetts",
  note = "[QA37.2.B413]",
  year = 1974 }
@book{Bell, author = "J. L. Bell and M. Machover",
  title = "A Course in Mathematical Logic",
  publisher = "North-Holland",
  address = "Amsterdam",
  note = "[QA9.B3953]",
  year = 1977 }
@inproceedings{Blass, author = "Andrea Blass",
  title = "The Interaction Between Category Theory and Set Theory",
  pages = "5--29",
  booktitle = "Mathematical Applications of Category Theory (Proceedings
     of the Special Session on Mathematical Applications
     Category Theory, 89th Annual Meeting of the American Mathematical
     Society, held in Denver, Colorado January 5--9, 1983)",
  editor = "John Walter Gray",
  year = 1983,
  note = "[QA169.A47 1983]",
  publisher = "American Mathematical Society",
  address = "Providence, Rhode Island"}
@proceedings{Bledsoe, editor = "W. W. Bledsoe and D. W. Loveland",
  title = "Automated Theorem Proving:  After 25 Years (Proceedings
     of the Special Session on Automatic Theorem Proving,
     89th Annual Meeting of the American Mathematical
     Society, held in Denver, Colorado January 5--9, 1983)",
  year = 1983,
  note = "[QA76.9.A96.S64 1983]",
  publisher = "American Mathematical Society",
  address = "Providence, Rhode Island" }
@book{Boolos, author = "George S. Boolos and Richard C. Jeffrey",
  title = "Computability and Log\-ic",
  publisher = "Cambridge University Press",
  edition = "third",
  address = "Cambridge",
  note = "[QA9.59.B66 1989]",
  year = 1989 }
@book{Campbell, author = "John Campbell",
  title = "Programmer's Progress",
  publisher = "White Star Software",
  address = "Box 51623, Palo Alto, CA 94303",
  year = 1991 }
@article{DBLP:journals/corr/Carneiro14,
  author    = {Mario Carneiro},
  title     = {Conversion of {HOL} Light proofs into Metamath},
  journal   = {CoRR},
  volume    = {abs/1412.8091},
  year      = {2014},
  url       = {http://arxiv.org/abs/1412.8091},
  archivePrefix = {arXiv},
  eprint    = {1412.8091},
  timestamp = {Mon, 13 Aug 2018 16:47:05 +0200},
  biburl    = {https://dblp.org/rec/bib/journals/corr/Carneiro14},
  bibsource = {dblp computer science bibliography, https://dblp.org}
}
@article{CarneiroND,
  author    = {Mario Carneiro},
  title     = {Natural Deductions in the Metamath Proof Language},
  url       = {http://us.metamath.org/ocat/natded.pdf},
  year      = 2014
}
@inproceedings{Chou, author = "Shang-Ching Chou",
  title = "Proving Elementary Geometry Theorems Using {W}u's Algorithm",
  pages = "243--286",
  booktitle = "Automated Theorem Proving:  After 25 Years (Proceedings
     of the Special Session on Automatic Theorem Proving,
     89th Annual Meeting of the American Mathematical
     Society, held in Denver, Colorado January 5--9, 1983)",
  editor = "W. W. Bledsoe and D. W. Loveland",
  year = 1983,
  note = "[QA76.9.A96.S64 1983]",
  publisher = "American Mathematical Society",
  address = "Providence, Rhode Island" }
@book{Clemente, author = "Daniel Clemente Laboreo",
  title = "Introduction to natural deduction",
  year = 2014,
  url = "http://www.danielclemente.com/logica/dn.en.pdf" }
@incollection{Courant, author = "Richard Courant and Herbert Robbins",
  title = "Topology",
  pages = "573--590",
  booktitle = "The World of Mathematics, Volume One",
  editor = "James R. Newman",
  publisher = "Simon and Schuster",
  address = "New York",
  note = "[QA3.W67 1988]",
  year = 1956 }
@book{Curry, author = "Haskell B. Curry",
  title = "Foundations of Mathematical Logic",
  publisher = "Dover Publications, Inc.",
  address = "New York",
  note = "[QA9.C976 1977]",
  year = 1977 }
@book{Davis, author = "Philip J. Davis and Reuben Hersh",
  title = "The Mathematical Experience",
  publisher = "Birkh{\"{a}}user Boston",
  address = "Boston",
  note = "[QA8.4.D37 1982]",
  year = 1981 }
@incollection{deMillo,
  author = "Richard de Millo and Richard Lipton and Alan Perlis",
  title = "Social Processes and Proofs of Theorems and Programs",
  pages = "267--285",
  booktitle = "New Directions in the Philosophy of Mathematics",
  editor = "Thomas Tymoczko",
  publisher = "Birkh{\"{a}}user Boston, Inc.",
  address = "Boston",
  note = "[QA8.6.N48 1986]",
  year = 1986 }
@book{Edwards, author = "Robert E. Edwards",
  title = "A Formal Background to Mathematics",
  publisher = "Springer-Verlag",
  address = "New York",
  note = "[QA37.2.E38 v.1a]",
  year = 1979 }
@book{Enderton, author = "Herbert B. Enderton",
  title = "Elements of Set Theory",
  publisher = "Academic Press, Inc.",
  address = "San Diego",
  note = "[QA248.E5]",
  year = 1977 }
@book{Goodstein, author = "R. L. Goodstein",
  title = "Development of Mathematical Logic",
  publisher = "Springer-Verlag New York Inc.",
  address = "New York",
  note = "[QA9.G6554]",
  year = 1971 }
@book{Guillen, author = "Michael Guillen",
  title = "Bridges to Infinity",
  publisher = "Jeremy P. Tarcher, Inc.",
  address = "Los Angeles",
  note = "[QA93.G8]",
  year = 1983 }
@book{Hamilton, author = "Alan G. Hamilton",
  title = "Logic for Mathematicians",
  edition = "revised",
  publisher = "Cambridge University Press",
  address = "Cambridge",
  note = "[QA9.H298]",
  year = 1988 }
@unpublished{Harrison, author = "John Robert Harrison",
  title = "Metatheory and Reflection in Theorem Proving:
    A Survey and Critique",
  note = "Technical Report
    CRC-053.
    SRI Cambridge,
    Millers Yard, Cambridge, UK,
    1995.
    Available on the Web as
{\verb+http:+}\-{\verb+//www.cl.cam.ac.uk/users/jrh/papers/reflect.html+}"}
@TECHREPORT{Harrison-thesis,
        author          = "John Robert Harrison",
        title           = "Theorem Proving with the Real Numbers",
        institution   = "University of Cambridge Computer
                         Lab\-o\-ra\-to\-ry",
        address         = "New Museums Site, Pembroke Street, Cambridge,
                           CB2 3QG, UK",
        year            = 1996,
        number          = 408,
        type            = "Technical Report",
        note            = "Author's PhD thesis,
   available on the Web at
{\verb+http:+}\-{\verb+//www.cl.cam.ac.uk+}\-{\verb+/users+}\-{\verb+/jrh+}%
\-{\verb+/papers+}\-{\verb+/thesis.html+}"}
@book{Herrlich, author = "Horst Herrlich and George E. Strecker",
  title = "Category Theory:  An Introduction",
  publisher = "Allyn and Bacon Inc.",
  address = "Boston",
  note = "[QA169.H567]",
  year = 1973 }
@article{Hindley, author = "J. Roger Hindley and David Meredith",
  title = "Principal Type-Schemes and Condensed Detachment",
  journal = "The Journal of Symbolic Logic",
  volume = 55,
  year = 1990,
  note = "[QA.J87]",
  pages = "90--105" }
@book{Hofstadter, author = "Douglas R. Hofstadter",
  title = "G{\"{o}}del, Escher, Bach",
  publisher = "Basic Books, Inc.",
  address = "New York",
  note = "[QA9.H63 1980]",
  year = 1979 }
@article{Indrzejczak, author= "Andrzej Indrzejczak",
  title = "Natural Deduction, Hybrid Systems and Modal Logic",
  journal = "Trends in Logic",
  volume = 30,
  publisher = "Springer",
  year = 2010 }
@article{Kalish, author = "D. Kalish and R. Montague",
  title = "On {T}arski's Formalization of Predicate Logic with Identity",
  journal = "Archiv f{\"{u}}r Mathematische Logik und Grundlagenfor\-schung",
  volume = 7,
  year = 1965,
  note = "[QA.A673]",
  pages = "81--101" }
@article{Kalman, author = "J. A. Kalman",
  title = "Condensed Detachment as a Rule of Inference",
  journal = "Studia Logica",
  volume = 42,
  number = 4,
  year = 1983,
  note = "[B18.P6.S933]",
  pages = "443-451" }
@book{Kline, author = "Morris Kline",
  title = "Mathematical Thought from Ancient to Modern Times",
  publisher = "Oxford University Press",
  address = "New York",
  note = "[QA21.K516 1990 v.3]",
  year = 1972 }
@book{Klinel, author = "Morris Kline",
  title = "Mathematics, The Loss of Certainty",
  publisher = "Oxford University Press",
  address = "New York",
  note = "[QA21.K525]",
  year = 1980 }
@book{Kramer, author = "Edna E. Kramer",
  title = "The Nature and Growth of Modern Mathematics",
  publisher = "Princeton University Press",
  address = "Princeton, New Jersey",
  note = "[QA93.K89 1981]",
  year = 1981 }
@article{Knill, author = "Oliver Knill",
  title = "Some Fundamental Theorems in Mathematics",
  year = "2018",
  url = "https://arxiv.org/abs/1807.08416" }
@book{Landau, author = "Edmund Landau",
  title = "Foundations of Analysis",
  publisher = "Chelsea Publishing Company",
  address = "New York",
  edition = "second",
  note = "[QA241.L2541 1960]",
  year = 1960 }
@article{Leblanc, author = "Hugues Leblanc",
  title = "On {M}eyer and {L}ambert's Quantificational Calculus {FQ}",
  journal = "The Journal of Symbolic Logic",
  volume = 33,
  year = 1968,
  note = "[QA.J87]",
  pages = "275--280" }
@article{Lejewski, author = "Czeslaw Lejewski",
  title = "On Implicational Definitions",
  journal = "Studia Logica",
  volume = 8,
  year = 1958,
  note = "[B18.P6.S933]",
  pages = "189--208" }
@book{Levy, author = "Azriel Levy",
  title = "Basic Set Theory",
  publisher = "Dover Publications",
  address = "Mineola, NY",
  year = "2002"
}
@book{Margaris, author = "Angelo Margaris",
  title = "First Order Mathematical Logic",
  publisher = "Blaisdell Publishing Company",
  address = "Waltham, Massachusetts",
  note = "[QA9.M327]",
  year = 1967}
@book{Manin, author = "Yu I. Manin",
  title = "A Course in Mathematical Logic",
  publisher = "Springer-Verlag",
  address = "New York",
  note = "[QA9.M29613]",
  year = "1977" }
@article{Mathias, author = "Adrian R. D. Mathias",
  title = "A Term of Length 4,523,659,424,929",
  journal = "Synthese",
  volume = 133,
  year = 2002,
  note = "[Q.S993]",
  pages = "75--86" }
@article{Megill, author = "Norman D. Megill",
  title = "A Finitely Axiomatized Formalization of Predicate Calculus
     with Equality",
  journal = "Notre Dame Journal of Formal Logic",
  volume = 36,
  year = 1995,
  note = "[QA.N914]",
  pages = "435--453" }
@unpublished{Megillc, author = "Norman D. Megill",
  title = "A Shorter Equivalent of the Axiom of Choice",
  month = "June",
  note = "Unpublished",
  year = 1991 }
@article{MegillBunder, author = "Norman D. Megill and Martin W.
    Bunder",
  title = "Weaker {D}-Complete Logics",
  journal = "Journal of the IGPL",
  volume = 4,
  year = 1996,
  pages = "215--225",
  note = "Available on the Web at
{\verb+http:+}\-{\verb+//www.mpi-sb.mpg.de+}\-{\verb+/igpl+}%
\-{\verb+/Journal+}\-{\verb+/V4-2+}\-{\verb+/#Megill+}"}
}
@book{Mendelson, author = "Elliott Mendelson",
  title = "Introduction to Mathematical Logic",
  edition = "second",
  publisher = "D. Van Nostrand Company, Inc.",
  address = "New York",
  note = "[QA9.M537 1979]",
  year = 1979 }
@article{Meredith, author = "David Meredith",
  title = "In Memoriam {C}arew {A}rthur {M}eredith (1904-1976)",
  journal = "Notre Dame Journal of Formal Logic",
  volume = 18,
  year = 1977,
  note = "[QA.N914]",
  pages = "513--516" }
@article{CAMeredith, author = "C. A. Meredith",
  title = "Single Axioms for the Systems ({C},{N}), ({C},{O}) and ({A},{N})
      of the Two-Valued Propositional Calculus",
  journal = "The Journal of Computing Systems",
  volume = 3,
  year = 1953,
  pages = "155--164" }
@article{Monk, author = "J. Donald Monk",
  title = "Provability With Finitely Many Variables",
  journal = "The Journal of Symbolic Logic",
  volume = 27,
  year = 1971,
  note = "[QA.J87]",
  pages = "353--358" }
@article{Monks, author = "J. Donald Monk",
  title = "Substitutionless Predicate Logic With Identity",
  journal = "Archiv f{\"{u}}r Mathematische Logik und Grundlagenfor\-schung",
  volume = 7,
  year = 1965,
  pages = "103--121" }
  %% Took out this from above to prevent LaTeX underfull warning:
  % note = "[QA.A673]",
@book{Moore, author = "A. W. Moore",
  title = "The Infinite",
  publisher = "Routledge",
  address = "New York",
  note = "[BD411.M59]",
  year = 1989}
@book{Munkres, author = "James R. Munkres",
  title = "Topology: A First Course",
  publisher = "Prentice-Hall, Inc.",
  address = "Englewood Cliffs, New Jersey",
  note = "[QA611.M82]",
  year = 1975}
@article{Nemesszeghy, author = "E. Z. Nemesszeghy and E. A. Nemesszeghy",
  title = "On Strongly Creative Definitions:  A Reply to {V}. {F}. {R}ickey",
  journal = "Logique et Analyse (N.\ S.)",
  year = 1977,
  volume = 20,
  note = "[BC.L832]",
  pages = "111--115" }
@unpublished{Nemeti, author = "N{\'{e}}meti, I.",
  title = "Algebraizations of Quantifier Logics, an Overview",
  note = "Version 11.4, preprint, Mathematical Institute, Budapest,
    1994.  A shortened version without proofs appeared in
    ``Algebraizations of quantifier logics, an introductory overview,''
   {\em Studia Logica}, 50:485--569, 1991 [B18.P6.S933]"}
@article{Pavicic, author = "M. Pavi{\v{c}}i{\'{c}}",
  title = "A New Axiomatization of Unified Quantum Logic",
  journal = "International Journal of Theoretical Physics",
  year = 1992,
  volume = 31,
  note = "[QC.I626]",
  pages = "1753 --1766" }
@book{Penrose, author = "Roger Penrose",
  title = "The Emperor's New Mind",
  publisher = "Oxford University Press",
  address = "New York",
  note = "[Q335.P415]",
  year = 1989 }
@book{PetersonI, author = "Ivars Peterson",
  title = "The Mathematical Tourist",
  publisher = "W. H. Freeman and Company",
  address = "New York",
  note = "[QA93.P475]",
  year = 1988 }
@article{Peterson, author = "Jeremy George Peterson",
  title = "An automatic theorem prover for substitution and detachment systems",
  journal = "Notre Dame Journal of Formal Logic",
  volume = 19,
  year = 1978,
  note = "[QA.N914]",
  pages = "119--122" }
@book{Quine, author = "Willard Van Orman Quine",
  title = "Set Theory and Its Logic",
  edition = "revised",
  publisher = "The Belknap Press of Harvard University Press",
  address = "Cambridge, Massachusetts",
  note = "[QA248.Q7 1969]",
  year = 1969 }
@article{Robinson, author = "J. A. Robinson",
  title = "A Machine-Oriented Logic Based on the Resolution Principle",
  journal = "Journal of the Association for Computing Machinery",
  year = 1965,
  volume = 12,
  pages = "23--41" }
@article{RobinsonT, author = "T. Thacher Robinson",
  title = "Independence of Two Nice Sets of Axioms for the Propositional
    Calculus",
  journal = "The Journal of Symbolic Logic",
  volume = 33,
  year = 1968,
  note = "[QA.J87]",
  pages = "265--270" }
@book{Rucker, author = "Rudy Rucker",
  title = "Infinity and the Mind:  The Science and Philosophy of the
    Infinite",
  publisher = "Bantam Books, Inc.",
  address = "New York",
  note = "[QA9.R79 1982]",
  year = 1982 }
@book{Russell, author = "Bertrand Russell",
  title = "Mysticism and Logic, and Other Essays",
  publisher = "Barnes \& Noble Books",
  address = "Totowa, New Jersey",
  note = "[B1649.R963.M9 1981]",
  year = 1981 }
@article{Russell2, author = "Bertrand Russell",
  title = "Recent Work on the Principles of Mathematics",
  journal = "International Monthly",
  volume = 4,
  year = 1901,
  pages = "84"}
@article{Schmidt, author = "Eric Schmidt",
  title = "Reductions in Norman Megill's axiom system for complex numbers",
  url = "http://us.metamath.org/downloads/schmidt-cnaxioms.pdf",
  year = "2012" }
@book{Shoenfield, author = "Joseph R. Shoenfield",
  title = "Mathematical Logic",
  publisher = "Addison-Wesley Publishing Company, Inc.",
  address = "Reading, Massachusetts",
  year = 1967,
  note = "[QA9.S52]" }
@book{Smullyan, author = "Raymond M. Smullyan",
  title = "Theory of Formal Systems",
  publisher = "Princeton University Press",
  address = "Princeton, New Jersey",
  year = 1961,
  note = "[QA248.5.S55]" }
@book{Solow, author = "Daniel Solow",
  title = "How to Read and Do Proofs:  An Introduction to Mathematical
    Thought Process",
  publisher = "John Wiley \& Sons",
  address = "New York",
  year = 1982,
  note = "[QA9.S577]" }
@book{Stark, author = "Harold M. Stark",
  title = "An Introduction to Number Theory",
  publisher = "Markham Publishing Company",
  address = "Chicago",
  note = "[QA241.S72 1978]",
  year = 1970 }
@article{Swart, author = "E. R. Swart",
  title = "The Philosophical Implications of the Four-Color Problem",
  journal = "American Mathematical Monthly",
  year = 1980,
  volume = 87,
  month = "November",
  note = "[QA.A5125]",
  pages = "697--707" }
@book{Szpiro, author = "George G. Szpiro",
  title = "Poincar{\'{e}}'s Prize: The Hundred-Year Quest to Solve One
    of Math's Greatest Puzzles",
  publisher = "Penguin Books Ltd",
  address = "London",
  note = "[QA43.S985 2007]",
  year = 2007}
@book{Takeuti, author = "Gaisi Takeuti and Wilson M. Zaring",
  title = "Introduction to Axiomatic Set Theory",
  edition = "second",
  publisher = "Springer-Verlag New York Inc.",
  address = "New York",
  note = "[QA248.T136 1982]",
  year = 1982}
@inproceedings{Tarski, author = "Alfred Tarski",
  title = "What is Elementary Geometry",
  pages = "16--29",
  booktitle = "The Axiomatic Method, with Special Reference to Geometry and
     Physics (Proceedings of an International Symposium held at the University
     of California, Berkeley, December 26, 1957 --- January 4, 1958)",
  editor = "Leon Henkin and Patrick Suppes and Alfred Tarski",
  year = 1959,
  publisher = "North-Holland Publishing Company",
  address = "Amsterdam"}
@article{Tarski1965, author = "Alfred Tarski",
  title = "A Simplified Formalization of Predicate Logic with Identity",
  journal = "Archiv f{\"{u}}r Mathematische Logik und Grundlagenforschung",
  volume = 7,
  year = 1965,
  note = "[QA.A673]",
  pages = "61--79" }
@book{Tymoczko,
  title = "New Directions in the Philosophy of Mathematics",
  editor = "Thomas Tymoczko",
  publisher = "Birkh{\"{a}}user Boston, Inc.",
  address = "Boston",
  note = "[QA8.6.N48 1986]",
  year = 1986 }
@incollection{Wang,
  author = "Hao Wang",
  title = "Theory and Practice in Mathematics",
  pages = "129--152",
  booktitle = "New Directions in the Philosophy of Mathematics",
  editor = "Thomas Tymoczko",
  publisher = "Birkh{\"{a}}user Boston, Inc.",
  address = "Boston",
  note = "[QA8.6.N48 1986]",
  year = 1986 }
@manual{Webster,
  title = "Webster's New Collegiate Dictionary",
  organization = "G. \& C. Merriam Co.",
  address = "Springfield, Massachusetts",
  note = "[PE1628.W4M4 1977]",
  year = 1977 }
@manual{Whitehead, author = "Alfred North Whitehead",
  title = "An Introduction to Mathematics",
  year = 1911 }
@book{PM, author = "Alfred North Whitehead and Bertrand Russell",
  title = "Principia Mathematica",
  edition = "second",
  publisher = "Cambridge University Press",
  address = "Cambridge",
  year = "1927",
  note = "(3 vols.) [QA9.W592 1927]" }
@article{DBLP:journals/corr/Whalen16,
  author    = {Daniel Whalen},
  title     = {Holophrasm: a neural Automated Theorem Prover for higher-order logic},
  journal   = {CoRR},
  volume    = {abs/1608.02644},
  year      = {2016},
  url       = {http://arxiv.org/abs/1608.02644},
  archivePrefix = {arXiv},
  eprint    = {1608.02644},
  timestamp = {Mon, 13 Aug 2018 16:46:19 +0200},
  biburl    = {https://dblp.org/rec/bib/journals/corr/Whalen16},
  bibsource = {dblp computer science bibliography, https://dblp.org} }
@article{Wiedijk-revisited,
  author = {Freek Wiedijk},
  title = {The QED Manifesto Revisited},
  year = {2007},
  url = {http://mizar.org/trybulec65/8.pdf} }
@book{Wolfram,
  author = "Stephen Wolfram",
  title = "Mathematica:  A System for Doing Mathematics by Computer",
  edition = "second",
  publisher = "Addison-Wesley Publishing Co.",
  address = "Redwood City, California",
  note = "[QA76.95.W65 1991]",
  year = 1991 }
@book{Wos, author = "Larry Wos and Ross Overbeek and Ewing Lusk and Jim Boyle",
  title = "Automated Reasoning:  Introduction and Applications",
  edition = "second",
  publisher = "McGraw-Hill, Inc.",
  address = "New York",
  note = "[QA76.9.A96.A93 1992]",
  year = 1992 }

%
%
%[1] Church, Alonzo, Introduction to Mathematical Logic,
% Volume 1, Princeton University Press, Princeton, N. J., 1956.
%
%[2] Cohen, Paul J., Set Theory and the Continuum Hypothesis,
% W. A. Benjamin, Inc., Reading, Mass., 1966.
%
%[3] Hamilton, Alan G., Logic for Mathematicians, Cambridge
% University Press,
% Cambridge, 1988.

%[6] Kleene, Stephen Cole, Introduction to Metamathematics, D.  Van
% Nostrand Company, Inc., Princeton (1952).

%[13] Tarski, Alfred, "A simplified formalization of predicate
% logic with identity," Archiv fur Mathematische Logik und
% Grundlagenforschung, vol. 7 (1965), pp. 61-79.

%[14] Tarski, Alfred and Steven Givant, A Formalization of Set
% Theory Without Variables, American Mathematical Society Colloquium
% Publications, vol. 41, American Mathematical Society,
% Providence, R. I., 1987.

%[15] Zeman, J. J., Modal Logic, Oxford University Press, Oxford, 1973.
\end{filecontents}
% --------------------------- End of metamath.bib -----------------------------


%Book: Metamath
%Author:  Norman Megill Email:  nm at alum.mit.edu
%Author:  David A. Wheeler Email:  dwheeler at dwheeler.com

% A book template example
% http://www.stsci.edu/ftp/software/tex/bookstuff/book.template

\documentclass[leqno]{book} % LaTeX 2e. 10pt. Use [leqno,12pt] for 12pt
% hyperref 2002/05/27 v6.72r  (couldn't get pagebackref to work)
\usepackage[plainpages=false,pdfpagelabels=true]{hyperref}

\usepackage{needspace}     % Enable control over page breaks
\usepackage{breqn}         % automatic equation breaking
\usepackage{microtype}     % microtypography, reduces hyphenation

% Packages for flexible tables.  We need to be able to
% wrap text within a cell (with automatically-determined widths) AND
% split a table automatically across multiple pages.
% * "tabularx" wraps text in cells but only 1 page
% * "longtable" goes across pages but by itself is incompatible with tabularx
% * "ltxtable" combines longtable and tabularx, but table contents
%    must be in a separate file.
% * "ltablex" combines tabularx and longtable - must install specially
% * "booktabs" is recommended as a way to improve the look of tables,
%   but doesn't add these capabilities.
% * "tabu" much more capable and seems to be recommended. So use that.

\usepackage{makecell}      % Enable forced line splits within a table cell
% v4.13 needed for tabu: https://tex.stackexchange.com/questions/600724/dimension-too-large-after-recent-longtable-update
\usepackage{longtable}[=v4.13] % Enable multi-page tables  
\usepackage{tabu}          % Multi-page tables with wrapped text in a cell

% You can find more Tex packages using commands like:
% tlmgr search --file tabu.sty
% find /usr/share/texmf-dist/ -name '*tab*'
%
%%%%%%%%%%%%%%%%%%%%%%%%%%%%%%%%%%%%%%%%%%%%%%%%%%%%%%%%%%%%%%%%%%%%%%%%%%%%
% Uncomment the next 3 lines to suppress boxes and colors on the hyperlinks
%%%%%%%%%%%%%%%%%%%%%%%%%%%%%%%%%%%%%%%%%%%%%%%%%%%%%%%%%%%%%%%%%%%%%%%%%%%%
%\hypersetup{
%colorlinks,citecolor=black,filecolor=black,linkcolor=black,urlcolor=black
%}
%
\usepackage{realref}

% Restarting page numbers: try?
%   \printglossary
%   \cleardoublepage
%   \pagenumbering{arabic}
%   \setcounter{page}{1}    ???needed
%   \include{chap1}

% not used:
% \def\R2Lurl#1#2{\mbox{\href{#1}\texttt{#2}}}

\usepackage{amssymb}

% Version 1 of book: margins: t=.4, b=.2, ll=.4, rr=.55
% \usepackage{anysize}
% % \papersize{<height>}{<width>}
% % \marginsize{<left>}{<right>}{<top>}{<bottom>}
% \papersize{9in}{6in}
% % l/r 0.6124-0.6170 works t/b 0.2418-0.3411 = 192pp. 0.2926-03118=exact
% \marginsize{0.7147in}{0.5147in}{0.4012in}{0.2012in}

\usepackage{anysize}
% \papersize{<height>}{<width>}
% \marginsize{<left>}{<right>}{<top>}{<bottom>}
\papersize{9in}{6in}
% l/r 0.85in&0.6431-0.6539 works t/b ?-?
%\marginsize{0.85in}{0.6485in}{0.55in}{0.35in}
\marginsize{0.8in}{0.65in}{0.5in}{0.3in}

% \usepackage[papersize={3.6in,4.8in},hmargin=0.1in,vmargin={0.1in,0.1in}]{geometry}  % page geometry
\usepackage{special-settings}

\raggedbottom
\makeindex

\begin{document}
% Discourage page widows and orphans:
\clubpenalty=300
\widowpenalty=300

%%%%%%% load in AMS fonts %%%%%%% % LaTeX 2.09 - obsolete in LaTeX 2e
%\input{amssym.def}
%\input{amssym.tex}
%\input{c:/texmf/tex/plain/amsfonts/amssym.def}
%\input{c:/texmf/tex/plain/amsfonts/amssym.tex}

\bibliographystyle{plain}
\pagenumbering{roman}
\pagestyle{headings}

\thispagestyle{empty}

\hfill
\vfill

\begin{center}
{\LARGE\bf Metamath} \\
\vspace{1ex}
{\large A Computer Language for Mathematical Proofs} \\
\vspace{7ex}
{\large Norman Megill} \\
\vspace{7ex}
with extensive revisions by \\
\vspace{1ex}
{\large David A. Wheeler} \\
\vspace{7ex}
% Printed date. If changing the date below, also fix the date at the beginning.
2019-06-02
\end{center}

\vfill
\hfill

\newpage
\thispagestyle{empty}

\hfill
\vfill

\begin{center}
$\sim$\ {\sc Public Domain}\ $\sim$

\vspace{2ex}
This book (including its later revisions)
has been released into the Public Domain by Norman Megill per the
Creative Commons CC0 1.0 Universal (CC0 1.0) Public Domain Dedication.
David A. Wheeler has done the same.
This public domain release applies worldwide.  In case this is not
legally possible, the right is granted to use the work for any purpose,
without any conditions, unless such conditions are required by law.
See \url{https://creativecommons.org/publicdomain/zero/1.0/}.

\vspace{3ex}
Several short, attributed quotations from copyrighted works
appear in this book under the ``fair use'' provision of Section 107 of
the United States Copyright Act (Title 17 of the {\em United States
Code}).  The public-domain status of this book is not applicable to
those quotations.

\vspace{3ex}
Any trademarks used in this book are the property of their owners.

% QA76.9.L63.M??

% \vspace{1ex}
%
% \vspace{1ex}
% {\small Permission is granted to make and distribute verbatim copies of this
% book
% provided the copyright notice and this
% permission notice are preserved on all copies.}
%
% \vspace{1ex}
% {\small Permission is granted to copy and distribute modified versions of this
% book under the conditions for verbatim copying, provided that the
% entire
% resulting derived work is distributed under the terms of a permission
% notice
% identical to this one.}
%
% \vspace{1ex}
% {\small Permission is granted to copy and distribute translations of this
% book into another language, under the above conditions for modified
% versions,
% except that this permission notice may be stated in a translation
% approved by the
% author.}
%
% \vspace{1ex}
% %{\small   For a copy of the \LaTeX\ source files for this book, contact
% %the author.} \\
% \ \\
% \ \\

\vspace{7ex}
% ISBN: 1-4116-3724-0 \\
% ISBN: 978-1-4116-3724-5 \\
ISBN: 978-0-359-70223-7 \\
{\ } \\
Lulu Press \\
Morrisville, North Carolina\\
USA


\hfill
\vfill

Norman Megill\\ 93 Bridge St., Lexington, MA 02421 \\
E-mail address: \texttt{nm{\char`\@}alum.mit.edu} \\
\vspace{7ex}
David A. Wheeler \\
E-mail address: \texttt{dwheeler{\char`\@}dwheeler.com} \\
% See notes added at end of Preface for revision history. \\
% For current information on the Metamath software see \\
\vspace{7ex}
\url{http://metamath.org}
\end{center}

\hfill
\vfill

{\parindent0pt%
\footnotesize{%
Cover: Aleph null ($\aleph_0$) is the symbol for the
first infinite cardinal number, discovered by Georg Cantor in 1873.
We use a red aleph null (with dark outline and gold glow) as the Metamath logo.
Credit: Norman Megill (1994) and Giovanni Mascellani (2019),
public domain.%
\index{aleph null}%
\index{Metamath!logo}\index{Cantor, Georg}\index{Mascellani, Giovanni}}}

% \newpage
% \thispagestyle{empty}
%
% \hfill
% \vfill
%
% \begin{center}
% {\it To my son Robin Dwight Megill}
% \end{center}
%
% \vfill
% \hfill
%
% \newpage

\tableofcontents
%\listoftables

\chapter*{Preface}
\markboth{PREFACE}{PREFACE}
\addcontentsline{toc}{section}{Preface}


% (For current information, see the notes added at the
% end of this preface on p.~\pageref{note2002}.)

\subsubsection{Overview}

Metamath\index{Metamath} is a computer language and an associated computer
program for archiving, verifying, and studying mathematical proofs at a very
detailed level.  The Metamath language incorporates no mathematics per se but
treats all mathematical statements as mere sequences of symbols.  You provide
Metamath with certain special sequences (axioms) that tell it what rules
of inference are allowed.  Metamath is not limited to any specific field of
mathematics.  The Metamath language is simple and robust, with an
almost total absence of hard-wired syntax, and
we\footnote{Unless otherwise noted, the words
``I,'' ``me,'' and ``my'' refer to Norman Megill\index{Megill, Norman}, while
``we,'' ``us,'' and ``our'' refer to Norman Megill and
David A. Wheeler\index{Wheeler, David A.}.}
believe that it
provides about the simplest possible framework that allows essentially all of
mathematics to be expressed with absolute rigor.

% index test
%\newcommand{\nn}[1]{#1n}
%\index{aaa@bbb}
%\index{abc!def}
%\index{abd|see{qqq}}
%\index{abe|nn}
%\index{abf|emph}
%\index{abg|(}
%\index{abg|)}

Using the Metamath language, you can build formal or mathematical
systems\index{formal system}\footnote{A formal or mathematical system consists
of a collection of symbols (such as $2$, $4$, $+$ and $=$), syntax rules that
describe how symbols may be combined to form a legal expression (called a
well-formed formula or {\em wff}, pronounced ``whiff''), some starting wffs
called axioms, and inference rules that describe how theorems may be derived
(proved) from the axioms.  A theorem is a mathematical fact such as $2+2=4$.
Strictly speaking, even an obvious fact such as this must be proved from
axioms to be formally acceptable to a mathematician.}\index{theorem}
\index{axiom}\index{rule}\index{well-formed formula (wff)} that involve
inferences from axioms.  Although a database is provided
that includes a recommended set of axioms for standard mathematics, if you
wish you can supply your own symbols, syntax, axioms, rules, and definitions.

The name ``Metamath'' was chosen to suggest that the language provides a
means for {\em describing} mathematics rather than {\em being} the
mathematics itself.  Actually in some sense any mathematical language is
metamathematical.  Symbols written on paper, or stored in a computer,
are not mathematics itself but rather a way of expressing mathematics.
For example ``7'' and ``VII'' are symbols for denoting the number seven
in Arabic and Roman numerals; neither {\em is} the number seven.

If you are able to understand and write computer programs, you should be able
to follow abstract mathematics with the aid of Metamath.  Used in conjunction
with standard textbooks, Metamath can guide you step-by-step towards an
understanding of abstract mathematics from a very rigorous viewpoint, even if
you have no formal abstract mathematics background.  By using a single,
consistent notation to express proofs, once you grasp its basic concepts
Metamath provides you with the ability to immediately follow and dissect
proofs even in totally unfamiliar areas.

Of course, just being able follow a proof will not necessarily give you an
intuitive familiarity with mathematics.  Memorizing the rules of chess does not
give you the ability to appreciate the game of a master, and knowing how the
notes on a musical score map to piano keys does not give you the ability to
hear in your head how it would sound.  But each of these can be a first step.

Metamath allows you to explore proofs in the sense that you can see the
theorem referenced at any step expanded in as much detail as you want, right
down to the underlying axioms of logic and set theory (in the case of the set
theory database provided).  While Metamath will not replace the higher-level
understanding that can only be acquired through exercises and hard work, being
able to see how gaps in a proof are filled in can give you increased
confidence that can speed up the learning process and save you time when you
get stuck.

The Metamath language breaks down a mathematical proof into its tiniest
possible parts.  These can be pieced together, like interlocking
pieces in a puzzle, only in a way that produces correct and absolutely rigorous
mathematics.

The nature of Metamath\index{Metamath} enforces very precise mathematical
thinking, similar to that involved in writing a computer program.  A crucial
difference, though, is that once a proof is verified (by the Metamath program)
to be correct, it is definitely correct; it can never have a hidden
``bug.''\index{computer program bugs}  After getting used to the kind of rigor
and accuracy provided by Metamath, you might even be tempted to
adopt the attitude that a proof should never be considered correct until it
has been verified by a computer, just as you would not completely trust a
manual calculation until you have verified it on a
calculator.

My goal
for Metamath was a system for describing and verifying
mathematics that is completely universal yet conceptually as simple as
possible.  In approaching mathematics from an axiomatic, formal viewpoint, I
wanted Metamath to be able to handle almost any mathematical system, not
necessarily with ease, but at least in principle and hopefully in practice. I
wanted it to verify proofs with absolute rigor, and for this reason Metamath
is what might be thought of as a ``compile-only'' language rather than an
algorithmic or Turing-machine language (Pascal, C, Prolog, Mathematica,
etc.).  In other words, a database written in the Metamath
language doesn't ``do'' anything; it merely exhibits mathematical knowledge
and permits this knowledge to be verified as being correct.  A program in an
algorithmic language can potentially have hidden bugs\index{computer program
bugs} as well as possibly being hard to understand.  But each token in a
Metamath database must be consistent with the database's earlier
contents according to simple, fixed rules.
If a database is verified
to be correct,\footnote{This includes
verification that a sequential list of proof steps results in the specified
theorem.} then the mathematical content is correct if the
verifier is correct and the axioms are correct.
The verification program could be incorrect, but the verification algorithm
is relatively simple (making it unlikely to be implemented incorrectly
by the Metamath program),
and there are over a dozen Metamath database verifiers
written by different people in different programming languages
(so these different verifiers can act as multiple reviewers of a database).
The most-used Metamath database, the Metamath Proof Explorer
(aka \texttt{set.mm}\index{set theory database (\texttt{set.mm})}%
\index{Metamath Proof Explorer}),
is currently verified by four different Metamath verifiers written by
four different people in four different languages, including the
original Metamath program described in this book.
The only ``bugs'' that can exist are in the statement of the axioms,
for example if the axioms are inconsistent (a famous problem shown to be
unsolvable by G\"{o}del's incompleteness theorem\index{G\"{o}del's
incompleteness theorem}).
However, real mathematical systems have very few axioms, and these can
be carefully studied.
All of this provides extraordinarily high confidence that the verified database
is in fact correct.

The Metamath program
doesn't prove theorems automatically but is designed to verify proofs
that you supply to it.
The underlying Metamath language is completely general and has no built-in,
preconceived notions about your formal system\index{formal system}, its logic
or its syntax.
For constructing proofs, the Metamath program has a Proof Assistant\index{Proof
Assistant} which helps you fill in some of a proof step's details, shows you
what choices you have at any step, and verifies the proof as you build it; but
you are still expected to provide the proof.

There are many other programs that can process or generate information
in the Metamath language, and more continue to be written.
This is in part because the Metamath language itself is very simple
and intentionally easy to automatically process.
Some programs, such as \texttt{mmj2}\index{mmj2}, include a proof assistant
that can automate some steps beyond what the Metamath program can do.
Mario Carneiro has developed an algorithm for converting proofs from
the OpenTheory interchange format, which can be translated to and from
any of the HOL family of proof languages (HOL4, HOL Light, ProofPower,
and Isabelle), into the
Metamath language \cite{DBLP:journals/corr/Carneiro14}\index{Carneiro, Mario}.
Daniel Whalen has developed Holophrasm, which can automatically
prove many Metamath proofs using
machine learning\index{machine learning}\index{artificial intelligence}
approaches
(including multiple neural networks\index{neural networks})\cite{DBLP:journals/corr/Whalen16}\index{Whalen, Daniel}.
However,
a discussion of these other programs is beyond the scope of this book.

Like most computer languages, the Metamath\index{Metamath} language uses the
standard ({\sc ascii}) characters on a computer keyboard, so it cannot
directly represent many of the special symbols that mathematicians use.  A
useful feature of the Metamath program is its ability to convert its notation
into the \LaTeX\ typesetting language.\index{latex@{\LaTeX}}  This feature
lets you convert the {\sc ascii} tokens you've defined into standard
mathematical symbols, so you end up with symbols and formulas you are familiar
with instead of somewhat cryptic {\sc ascii} representations of them.
The Metamath program can also generate HTML\index{HTML}, making it easy
to view results on the web and to see related information by using
hypertext links.

Metamath is probably conceptually different from anything you've seen
before and some aspects may take some getting used to.  This book will
help you decide whether Metamath suits your specific needs.

\subsubsection{Setting Your Expectations}
It is important for you to understand what Metamath\index{Metamath} is and is
not.  As mentioned, the Metamath program
is {\em not} an automated theorem prover but
rather a proof verifier.  Developing a database can be tedious, hard work,
especially if you want to make the proofs as short as possible, but it becomes
easier as you build up a collection of useful theorems.  The purpose of
Metamath is simply to document existing mathematics in an absolutely rigorous,
computer-verifiable way, not to aid directly in the creation of new
mathematics.  It also is not a magic solution for learning abstract
mathematics, although it may be helpful to be able to actually see the implied
rigor behind what you are learning from textbooks, as well as providing hints
to work out proofs that you are stumped on.

As of this writing, a sizable set theory database has been developed to
provide a foundation for many fields of mathematics, but much more work would
be required to develop useful databases for specific fields.

Metamath\index{Metamath} ``knows no math;'' it just provides a framework in
which to express mathematics.  Its language is very small.  You can define two
kinds of symbols, constants\index{constant} and variables\index{variable}.
The only thing Metamath knows how to do is to substitute strings of symbols
for the variables\index{substitution!variable}\index{variable substitution} in
an expression based on instructions you provide it in a proof, subject to
certain constraints you specify for the variables.  Even the decimal
representation of a number is merely a string of certain constants (digits)
which together, in a specific context, correspond to whatever mathematical
object you choose to define for it; unlike other computer languages, there is
no actual number stored inside the computer.  In a proof, you in effect
instruct Metamath what symbol substitutions to make in previous axioms or
theorems and join a sequence of them together to result in the desired
theorem.  This kind of symbol manipulation captures the essence of mathematics
at a preaxiomatic level.

\subsubsection{Metamath and Mathematical Literature}

In advanced mathematical literature, proofs are usually presented in the form
of short outlines that often only an expert can follow.  This is partly out of
a desire for brevity, but it would also be unwise (even if it were practical)
to present proofs in complete formal detail, since the overall picture would
be lost.\index{formal proof}

A solution I envision\label{envision} that would allow mathematics to remain
acceptable to the expert, yet increase its accessibility to non-specialists,
consists of a combination of the traditional short, informal proof in print
accompanied by a complete formal proof stored in a computer database.  In an
analogy with a computer program, the informal proof is like a set of comments
that describe the overall reasoning and content of the proof, whereas the
computer database is like the actual program and provides a means for anyone,
even a non-expert, to follow the proof in as much detail as desired, exploring
it back through layers of theorems (like subroutines that call other
subroutines) all the way back to the axioms of the theory.  In addition, the
computer database would have the advantage of providing absolute assurance
that the proof is correct, since each step can be verified automatically.

There are several other approaches besides Metamath to a project such
as this.  Section~\ref{proofverifiers} discusses some of these.

To us, a noble goal would be a database with hundreds of thousands of
theorems and their computer-verifiable proofs, encompassing a significant
fraction of known mathematics and available for instant access.
These would be fully verified by multiple independently-implemented verifiers,
to provide extremely high confidence that the proofs are completely correct.
The database would allow people to investigate whatever details they were
interested in, so that they could confirm whatever portions they wished.
Whether or not Metamath is an appropriate choice remains to be seen, but in
principle we believe it is sufficient.

\subsubsection{Formalism}

Over the past fifty years, a group of French mathematicians working
collectively under the pseudonym of Bourbaki\index{Bourbaki, Nicolas} have
co-authored a series of monographs that attempt to rigorously and
consistently formalize large bodies of mathematics from foundations.  On the
one hand, certainly such an effort has its merits; on the other hand, the
Bourbaki project has been criticized for its ``scholasticism'' and
``hyperaxiomatics'' that hide the intuitive steps that lead to the results
\cite[p.~191]{Barrow}\index{Barrow, John D.}.

Metamath unabashedly carries this philosophy to its extreme and no doubt is
subject to the same kind of criticism.  Nonetheless I think that in
conjunction with conventional approaches to mathematics Metamath can serve a
useful purpose.  The Bourbaki approach is essentially pedagogic, requiring the
reader to become intimately familiar with each detail in a very large
hierarchy before he or she can proceed to the next step.  The difference with
Metamath is that the ``reader'' (user) knows that all details are contained in
its computer database, available as needed; it does not demand that the user
know everything but conveniently makes available those portions that are of
interest.  As the body of all mathematical knowledge grows larger and larger,
no one individual can have a thorough grasp of its entirety.  Metamath
can finalize and put to rest any questions about the validity of any part of it
and can make any part of it accessible, in principle, to a non-specialist.

\subsubsection{A Personal Note}
Why did I develop Metamath\index{Metamath}?  I enjoy abstract mathematics, but
I sometimes get lost in a barrage of definitions and start to lose confidence
that my proofs are correct.  Or I reach a point where I lose sight of how
anything I'm doing relates to the axioms that a theory is based on and am
sometimes suspicious that there may be some overlooked implicit axiom
accidentally introduced along the way (as happened historically with Euclidean
geometry\index{Euclidean geometry}, whose omission of Pasch's
axiom\index{Pasch's axiom} went unnoticed for 2000 years
\cite[p.~160]{Davis}!). I'm also somewhat lazy and wish to avoid the effort
involved in re-verifying the gaps in informal proofs ``left to the reader;'' I
prefer to figure them out just once and not have to go through the same
frustration a year from now when I've forgotten what I did.  Metamath provides
better recovery of my efforts than scraps of paper that I can't
decipher anymore.  But mostly I find very appealing the idea of rigorously
archiving mathematical knowledge in a computer database, providing precision,
certainty, and elimination of human error.

\subsubsection{Note on Bibliography and Index}

The Bibliography usually includes the Library of Congress classification
for a work to make it easier for you to find it in on a university
library shelf.  The Index has author references to pages where their works
are cited, even though the authors' names may not appear on those pages.

\subsubsection{Acknowledgments}

Acknowledgments are first due to my wife, Deborah (who passed away on
September 4, 1998), for critiquing the manu\-script but most of all for
her patience and support.  I also wish to thank Joe Wright, Richard
Becker, Clarke Evans, Buddha Buck, and Jeremy Henty for helpful
comments.  Any errors, omissions, and other shortcomings are of course
my responsibility.

\subsubsection{Note Added June 22, 2005}\label{note2002}

The original, unpublished version of this book was written in 1997 and
distributed via the web.  The present edition has been updated to
reflect the current Metamath program and databases, as well as more
current {\sc url}s for Internet sites.  Thanks to Josh
Purinton\index{Purinton, Josh}, One Hand
Clapping, Mel L.\ O'Cat, and Roy F. Longton for pointing out
typographical and other errors.  I have also benefitted from numerous
discussions with Raph Levien\index{Levien, Raph}, who has extended
Metamath's philosophy of rigor to result in his {\em
Ghilbert}\index{Ghilbert} proof language (\url{http://ghilbert.org}).

Robert (Bob) Solovay\index{Solovay, Robert} communicated a new result of
A.~R.~D.~Mathias on the system of Bourbaki, and the text has been
updated accordingly (p.~\pageref{bourbaki}).

Bob also pointed out a clarification of the literature regarding
category theory and inaccessible cardinals\index{category
theory}\index{cardinal, inaccessible} (p.~\pageref{categoryth}),
and a misleading statement was removed from the text.  Specifically,
contrary to a statement in previous editions, it is possible to express
``There is a proper class of inaccessible cardinals'' in the language of
ZFC.  This can be done as follows:  ``For every set $x$ there is an
inaccessible cardinal $\kappa$ such that $\kappa$ is not in $x$.''
Bob writes:\footnote{Private communication, Nov.~30, 2002.}
\begin{quotation}
     This axiom is how Grothendieck presents category theory.  To each
inaccessible cardinal $\kappa$ one associates a Grothendieck universe
\index{Grothendieck, Alexander} $U(\kappa)$.  $U(\kappa)$ consists of
those sets which lie in a transitive set of cardinality less than
$\kappa$.  Instead of the ``category of all groups,'' one works relative
to a universe [considering the category of groups of cardinality less
than $\kappa$].  Now the category whose objects are all categories
``relative to the universe $U(\kappa)$'' will be a category not
relative to this universe but to the next universe.

     All of the things category theorists like to do can be done in this
framework.  The only controversial point is whether the Grothen\-dieck
axiom is too strong for the needs of category theorists.  Mac Lane
\index{Mac Lane, Saunders} argues that ``one universe is enough'' and
Feferman\index{Feferman, Solomon} has argued that one can get by with
ordinary ZFC.  I don't find Feferman's arguments persuasive.  Mac Lane
may be right, but when I think about category theory I do it \`{a} la
Grothendieck.

        By the way Mizar\index{Mizar} adds the axiom ``there is a proper
class of inaccessibles'' precisely so as to do category theory.
\end{quotation}

The most current information on the Metamath program and databases can
always be found at \url{http://metamath.org}.


\subsubsection{Note Added June 24, 2006}\label{note2006}

The Metamath spec was restricted slightly to make parsers easier to
write.  See the footnote on p.~\pageref{namespace}.

%\subsubsection{Note Added July 24, 2006}\label{note2006b}
\subsubsection{Note Added March 10, 2007}\label{note2006b}

I am grateful to Anthony Williams\index{Williams, Anthony} for writing
the \LaTeX\ package called {\tt realref.sty} and contributing it to the
public domain.  This package allows the internal hyperlinks in a {\sc
pdf} file to anchor to specific page numbers instead of just section
titles, making the navigation of the {\sc pdf} file for this book much
more pleasant and ``logical.''

A typographical error found by Martin Kiselkov was corrected.
A confusing remark about unification was deleted per suggestion of
Mel O'Cat.

\subsubsection{Note Added May 27, 2009}\label{note2009}

Several typos found by Kim Sparre were corrected.  A note was added that
the Poincar\'{e} conjecture has been proved (p.~\pageref{poincare}).

\subsubsection{Note Added Nov. 17, 2014}\label{note2014}

The statement of the Schr\"{o}der--Bernstein theorem was corrected in
Section~\ref{trust}.  Thanks to Bob Solovay for pointing out the error.

\subsubsection{Note Added May 25, 2016}\label{note2016}

Thanks to Jerry James for correcting 16 typos.

\subsubsection{Note Added February 25, 2019}\label{note201902}

David A. Wheeler\index{Wheeler, David A.}
made a large number of improvements and updates,
in coordination with Norman Megill.
The predicate calculus axioms were renumbered, and the text makes
it clear that they are based on Tarski's system S2;
the one slight deviation in axiom ax-6 is explained and justified.
The real and complex number axioms were modified to be consistent with
\texttt{set.mm}\index{set theory database (\texttt{set.mm})}%
\index{Metamath Proof Explorer}.
Long-awaited specification changes ``1--8'' were made,
which clarified previously ambiguous points.
Some errors in the text involving \texttt{\$f} and
\texttt{\$d} statements were corrected (the spec was correct, but
the in-book explanations unintentionally contradicted the spec).
We now have a system for automatically generating narrow PDFs,
so that those with smartphones can have easy access to the current
version of this document.
A new section on deduction was added;
it discusses the standard deduction theorem,
the weak deduction theorem,
deduction style, and natural deduction.
Many minor corrections (too numerous to list here) were also made.

\subsubsection{Note Added March 7, 2019}\label{note201903}

This added a description of the Matamath language syntax in
Extended Backus--Naur Form (EBNF)\index{Extended Backus--Naur Form}\index{EBNF}
in Appendix \ref{BNF}, added a brief explanation about typecodes,
inserted more examples in the deduction section,
and added a variety of smaller improvements.

\subsubsection{Note Added April 7, 2019}\label{note201904}

This version clarified the proper substitution notation, improved the
discussion on the weak deduction theorem and natural deduction,
documented the \texttt{undo} command, updated the information on
\texttt{write source}, changed the typecode
from \texttt{set} to \texttt{setvar} to be consistent with the current
version of \texttt{set.mm}, added more documentation about comment markup
(e.g., documented how to create headings), and clarified the
differences between various assertion forms (in particular deduction form).

\subsubsection{Note Added June 2, 2019}\label{note201906}

This version fixes a large number of small issues reported by
Beno\^{i}t Jubin\index{Jubin, Beno\^{i}t}, such as editorial issues
and the need to document \texttt{verify markup} (thank you!).
This version also includes specific examples
of forms (deduction form, inference form, and closed form).
We call this version the ``second edition'';
the previous edition formally published in 2007 had a slightly different title
(\textit{Metamath: A Computer Language for Pure Mathematics}).

\chapter{Introduction}
\pagenumbering{arabic}

\begin{quotation}
  {\em {\em I.M.:}  No, no.  There's nothing subjective about it!  Everybody
knows what a proof is.  Just read some books, take courses from a competent
mathematician, and you'll catch on.

{\em Student:}  Are you sure?

{\em I.M.:}  Well---it is possible that you won't, if you don't have any
aptitude for it.  That can happen, too.

{\em Student:}  Then {\em you} decide what a proof is, and if I don't learn
to decide in the same way, you decide I don't have any aptitude.

{\em I.M.:}  If not me, then who?}
    \flushright\sc  ``The Ideal Mathematician''
    \index{Davis, Phillip J.}
    \footnote{\cite{Davis}, p.~40.}\\
\end{quotation}

Brilliant mathematicians have discovered almost
unimaginably profound results that rank among the crowning intellectual
achievements of mankind.  However, there is a sense in which modern abstract
mathematics is behind the times, stuck in an era before computers existed.
While no one disputes the remarkable results that have been achieved,
communicating these results in a precise way to the uninitiated is virtually
impossible.  To describe these results, a terse informal language is used which
despite its elegance is very difficult to learn.  This informal language is not
imprecise, far from it, but rather it often has omitted detail
and symbols with hidden context that are
implicitly understood by an expert but few others.  Extremely complex technical
meanings are associated with innocent-sounding English words such as
``compact'' and ``measurable'' that barely hint at what is actually being
said.  Anyone who does not keep the precise technical meaning constantly in
mind is bound to fail, and acquiring the ability to do this can be achieved
only through much practice and hard work.  Only the few who complete the
painful learning experience can join the small in-group of pure
mathematicians.  The informal language effectively cuts off the true nature of
their knowledge from most everyone else.

Metamath\index{Metamath} makes abstract mathematics more concrete.  It allows
a computer to keep track of the complexity associated with each word or symbol
with absolute rigor.  You can explore this complexity at your leisure, to
whatever degree you desire.  Whether or not you believe that concepts such as
infinity actually ``exist'' outside of the mind, Metamath lets you get to the
foundation for what's really being said.

Metamath also enables completely rigorous and thorough proof verification.
Its language is simple enough so that you
don't have to rely on the authority of experts but can verify the results
yourself, step by step.  If you want to attempt to derive your own results,
Metamath will not let you make a mistake in reasoning.
Even professional mathematicians make mistakes; Metamath makes it possible
to thoroughly verify that proofs are correct.

Metamath\index{Metamath} is a computer language and an associated computer
program for archiving, verifying, and studying mathematical proofs at a very
detailed level.
The Metamath language
describes formal\index{formal system} mathematical
systems and expresses proofs of theorems in those systems.  Such a language
is called a metalanguage\index{metalanguage} by mathematicians.
The Metamath program is a computer program that verifies
proofs expressed in the Metamath language.
The Metamath program does not have the built-in
ability to make logical inferences; it just makes a series of symbol
substitutions according to instructions given to it in a proof
and verifies that the result matches the expected theorem.  It makes logical
inferences based only on rules of logic that are contained in a set of
axioms\index{axiom}, or first principles, that you provide to it as the
starting point for proofs.

The complete specification of the Metamath language is only four pages long
(Section~\ref{spec}, p.~\pageref{spec}).  Its simplicity may at first make you
wonder how it can do much of anything at all.  But in fact the kinds of
symbol manipulations it performs are the ones that are implicitly done in all
mathematical systems at the lowest level.  You can learn it relatively quickly
and have complete confidence in any mathematical proof that it verifies.  On
the other hand, it is powerful and general enough so that virtually any
mathematical theory, from the most basic to the deeply abstract, can be
described with it.

Although in principle Metamath can be used with any
kind of mathematics, it is best suited for abstract or ``pure'' mathematics
that is mostly concerned with theorems and their proofs, as opposed to the
kind of mathematics that deals with the practical manipulation of numbers.
Examples of branches of pure mathematics are logic\index{logic},\footnote{Logic
is the study of statements that are universally true regardless of the objects
being described by the statements.  An example is the statement, ``if $P$
implies $Q$, then either $P$ is false or $Q$ is true.''} set theory\index{set
theory},\footnote{Set theory is the study of general-purpose mathematical objects called
``sets,'' and from it essentially all of mathematics can be derived.  For
example, numbers can be defined as specific sets, and their properties
can be explored using the tools of set theory.} number theory\index{number
theory},\footnote{Number theory deals with the properties of positive and
negative integers (whole numbers).} group theory\index{group
theory},\footnote{Group theory studies the properties of mathematical objects
called groups that obey a simple set of axioms and have properties of symmetry
that make them useful in many other fields.} abstract algebra\index{abstract
algebra},\footnote{Abstract algebra includes group theory and also studies
groups with additional properties that qualify them as ``rings'' and
``fields.''  The set of real numbers is a familiar example of a field.},
analysis\index{analysis} \index{real and complex numbers}\footnote{Analysis is
the study of real and complex numbers.} and
topology\index{topology}.\footnote{One area studied by topology are properties
that remain unchanged when geometrical objects undergo stretching
deformations; for example a doughnut and a coffee cup each have one hole (the
cup's hole is in its handle) and are thus considered topologically
equivalent.  In general, though, topology is the study of abstract
mathematical objects that obey a certain (surprisingly simple) set of axioms.
See, for example, Munkres \cite{Munkres}\index{Munkres, James R.}.} Even in
physics, Metamath could be applied to certain branches that make use of
abstract mathematics, such as quantum logic\index{quantum logic} (used to study
aspects of quantum mechanics\index{quantum mechanics}).

On the other hand, Metamath\index{Metamath} is less suited to applications
that deal primarily with intensive numeric computations.  Metamath does not
have any built-in representation of numbers\index{Metamath!representation of
numbers}; instead, a specific string of symbols (digits) must be syntactically
constructed as part of any proof in which an ordinary number is used.  For
this reason, numbers in Metamath are best limited to specific constants that
arise during the course of a theorem or its proof.  Numbers are only a tiny
part of the world of abstract mathematics.  The exclusion of built-in numbers
was a conscious decision to help achieve Metamath's simplicity, and there are
other software tools if you have different mathematical needs.
If you wish to quickly solve algebraic problems, the computer algebra
programs\index{computer algebra system} {\sc
macsyma}\index{macsyma@{\sc macsyma}}, Mathematica\index{Mathematica}, and
Maple\index{Maple} are specifically suited to handling numbers and
algebra efficiently.
If you wish to simply calculate numeric or matrix expressions easily,
tools such as Octave\index{Octave} may be a better choice.

After learning Metamath's basic statement types, any
tech\-ni\-cal\-ly ori\-ent\-ed person, mathematician or not, can
immediately trace
any theorem proved in the language as far back as he or she wants, all the way
to the axioms on which the theorem is based.  This ability suggests a
non-traditional way of learning about pure mathematics.  Used in conjunction
with traditional methods, Metamath could make pure mathematics accessible to
people who are not sufficiently skilled to figure out the implicit detail in
ordinary textbook proofs.  Once you learn the axioms of a theory, you can have
complete confidence that everything you need to understand a proof you are
studying is all there, at your beck and call, allowing you to focus in on any
proof step you don't understand in as much depth as you need, without worrying
about getting stuck on a step you can't figure out.\footnote{On the other
hand, writing proofs in the Metamath language is challenging, requiring
a degree of rigor far in excess of that normally taught to students.  In a
classroom setting, I doubt that writing Metamath proofs would ever replace
traditional homework exercises involving informal proofs, because the time
needed to work out the details would not allow a course to
cover much material.  For students who have trouble grasping the implied rigor
in traditional material, writing a few simple proofs in the Metamath language
might help clarify fuzzy thought processes.  Although somewhat difficult at
first, it eventually becomes fun to do, like solving a puzzle, because of the
instant feedback provided by the computer.}

Metamath\index{Metamath} is probably unlike anything you have
encountered before.  In this first chapter we will look at the philosophy and
use of computers in mathematics in order to better understand the motivation
behind Metamath.  The material in this chapter is not required in order to use
Metamath.  You may skip it if you are impatient, but I hope you will find it
educational and enjoyable.  If you want to start experimenting with the
Metamath program right away, proceed directly to Chapter~\ref{using}
(p.~\pageref{using}).  To
learn the Metamath language, skim Chapter~\ref{using} then proceed to
Chapter~\ref{languagespec} (p.~\pageref{languagespec}).

\section{Mathematics as a Computer Language}

\begin{quote}
  {\em The study of mathematics is apt to commence in
dis\-ap\-point\-ment.\ldots \\
We are told that by its aid the stars are weighted
and the billions of molecules in a drop of water are counted.  Yet, like the
ghost of Hamlet's father, this great science eludes the efforts of our mental
weapons to grasp it.}
  \flushright\sc  Alfred North Whitehead\footnote{\cite{Whitehead}, ch.\ 1.}\\
\end{quote}\index{Whitehead, Alfred North}

\subsection{Is Mathematics ``User-Friendly''?}

Suppose you have no formal training in abstract mathematics.  But popular
books you've read offer tempting glimpses of this world filled with profound
ideas that have stirred the human spirit.  You are not satisfied with the
informal, watered-down descriptions you've read but feel it is important to
grasp the underlying mathematics itself to understand its true meaning. It's
not practical to go back to school to learn it, though; you don't want to
dedicate years of your life to it.  There are many important things in life,
and you have to set priorities for what's important to you.  What would happen
if you tried to pursue it on your own, in your spare time?

After all, you were able to learn a computer programming language such as
Pascal on your own without too much difficulty, even though you had no formal
training in computers.  You don't claim to be an expert in software design,
but you can write a passable program when necessary to suit your needs.  Even
more important, you know that you can look at anyone else's Pascal program, no
matter how complex, and with enough patience figure out exactly how it works,
even though you are not a specialist.  Pascal allows you do anything that a
computer can do, at least in principle.  Thus you know you have the ability,
in principle, to follow anything that a computer program can do:  you just
have to break it down into small enough pieces.

Here's an imaginary scenario of what might happen if you na\-ive\-ly a\-dopted
this same view of abstract mathematics and tried to pick it up on your own, in
a period of time comparable to, say, learning a computer programming
language.

\subsubsection{A Non-Mathematician's Quest for Truth}

\begin{quote}
  {\em \ldots my daughters have been studying (chemistry) for several
se\-mes\-ters, think they have learned differential and integral calculus in
school, and yet even today don't know why $x\cdot y=y\cdot x$ is true.}
  \flushright\sc  Edmund Landau\footnote{\cite{Landau}, p.~vi.}\\
\end{quote}\index{Landau, Edmund}

\begin{quote}
  {\em Minus times minus is plus,\\
The reason for this we need not discuss.}
  \flushright\sc W.\ H.\ Auden\footnote{As quoted in \cite{Guillen}, p.~64.}\\
\end{quote}\index{Auden, W.\ H.}\index{Guillen, Michael}

We'll suppose you are a technically oriented professional, perhaps an engineer, a
computer programmer, or a physicist, but probably not a mathematician.  You
consider yourself reasonably intelligent.  You did well in school, learning a
variety of methods and techniques in practical mathematics such as calculus and
differential equations.  But rarely did your courses get into anything
resembling modern abstract mathematics, and proofs were something that appeared
only occasionally in your textbooks, a kind of necessary evil that was
supposed to convince you of a certain key result.  Most of your
homework consisted of exercises that gave you practice in the techniques, and
you were hardly ever asked to come up with a proof of your own.

You find yourself curious about advanced, abstract mathematics.  You are
driven by an inner conviction that it is important to understand and
appreciate some of the most profound knowledge discovered by mankind.  But it
seems very hard to learn, something that only certain gifted longhairs can
access and understand.  You are frustrated that it seems forever cut off from
you.

Eventually your curiosity drives you to do something about it.
You set for yourself a goal of ``really'' understanding mathematics:  not just
how to manipulate equations in algebra or calculus according to cookbook
rules, but rather to gain a deep understanding of where those rules come from.
In fact, you're not thinking about this kind of ordinary mathematics at all,
but about a much more abstract, ethereal realm of pure mathematics, where
famous results such as G\"{o}del's incompleteness theorem\index{G\"{o}del's
incompleteness theorem} and Cantor's different kinds of infinities
reside.

You have probably read a number of popular books, with titles like {\em
Infinity and the Mind} \cite{Rucker}\index{Rucker, Rudy}, on topics such as
these.  You found them inspiring but at the same time somewhat
unsatisfactory.  They gave you a general idea of what these results are about,
but if someone asked you to prove them, you wouldn't have the faintest idea of
where to begin.   Sure, you could give the same overall outline that you
learned from the popular books; and in a general sort of way, you do have an
understanding.  But deep down inside, you know that there is a rigor that is
missing, that probably there are many subtle steps and pitfalls along the way,
and ultimately it seems you have to place your trust in the experts in the
field.  You don't like this; you want to be able to verify these results for
yourself.

So where do you go next?  As a first step, you decide to look up some of the
original papers on the theorems you are curious about, or better, obtain some
standard textbooks in the field.  You look up a theorem you want to
understand.  Sure enough, it's there, but it's expressed with strange
terms and odd symbols that mean absolutely nothing to you.  It might as well be written in
a foreign language you've never seen before, whose symbols are totally alien.
You look at the proof, and you haven't the foggiest notion what each step
means, much less how one step follows from another.  Well, obviously you have
a lot to learn if you want to understand this stuff.

You feel that you could probably understand it by
going back to college for another three to six years and getting a math
degree.  But that does not fit in with your career and the other things in
your life and would serve no practical purpose.  You decide to seek a quicker
path.  You figure you'll just trace your way back to the beginning, step by
step, as you would do with a computer program, until you understand it.  But
you quickly find that this is not possible, since you can't even understand
enough to know what you have to trace back to.

Maybe a different approach is in order---maybe you should start at the
beginning and work your way up.  First, you read the introduction to the book
to find out what the prerequisites are.  In a similar fashion, you trace your
way back through two or three more books, finally arriving at one that seems
to start at a beginning:  it lists the axioms of arithmetic.  ``Aha!'' you
naively think, ``This must be the starting point, the source of all mathematical
knowledge.'' Or at least the starting point for mathematics dealing with
numbers; you have to start somewhere and have no idea what the starting point
for other mathematics would be.  But the word ``axioms'' looks promising.  So
you eagerly read along and work through some elementary exercises at the
beginning of the book.  You feel vaguely bothered:  these
don't seem like axioms at all, at least not in the sense that you want to
think of axioms.  Axioms imply a starting point from which everything else can
be built up, according to precise rules specified in the axiom system.  Even
though you can understand the first few proofs in an informal way,
and are able to do some of the
exercises, it's hard to pin down precisely what the
rules are.   Sure, each step seems to follow logically from the others, but
exactly what does that mean?  Is the ``logic'' just a matter of common sense,
something vague that we all understand but can never quite state precisely?

You've spent a number of years, off and on, programming computers, and you
know that in the case of computer languages there is no question of what the
rules are---they are precise and crystal clear.  If you follow them, your
program will work, and if you don't, it won't.  No matter how complex a
program, it can always be broken down into simpler and simpler pieces, until
you can ultimately identify the bits that are moved around to perform a
specific function.  Some programs might require a lot of perseverance to
accomplish this, but if you focus on a specific portion of it, you don't even
necessarily have to know how the rest of it works. Shouldn't there be an
analogy in mathematics?

You decide to apply the ultimate test:  you ask yourself how a computer could
verify or ensure that the steps in these proofs follow from one another.
Certainly mathematics must be at least as precisely defined as a computer
language, if not more so; after all, computer science itself is based on it.
If you can get a computer to verify these proofs, then you should also be
able, in principle, to understand them yourself in a very crystal clear,
precise way.

You're in for a surprise:  you can conceive of no way to convert the
proofs, which are in English, to a form that the computer can understand.
The proofs are filled with phrases such as ``assume there exists a unique
$x$\ldots'' and ``given any $y$, let $z$ be the number such that\ldots''  This
isn't the kind of logic you are used to in computer programming, where
everything, even arithmetic, reduces to Boolean ones and zeroes if you care to
break it down sufficiently.  Even though you think you understand the proofs,
there seems to be some kind of higher reasoning involved rather than precise
rules that define how you manipulate the symbols in the axioms.  Whatever it
is, it just isn't obvious how you would express it to a computer, and the more
you think about it, the more puzzled and confused you get, to the point where
you even wonder whether {\em you} really understand it.  There's a lot more to
these axioms of arithmetic than meets the eye.

Nobody ever talked about this in school in your applied math and engineering
courses.  You just learned the rules they gave you, not quite understanding
how or why they worked, sometimes vaguely suspicious or uncertain of them, and
through homework problems and osmosis learned how to present solutions that
satisfied the instructor and earned you an ``A.''  Rarely did you actually
``prove'' anything in a rigorous way, and the math majors who did do stuff
like that seemed to be in a different world.

Of course, there are computer algebra programs that can do mathematics, and
rather impressively.  They can instantly solve the integrals that you
struggled with in freshman calculus, and do much, much more.  But when you
look at these programs, what you see is a big collection of algorithms and
techniques that evolved and were added to over time, along with some basic
software that manipulates symbols.  Each algorithm that is built in is the
result of someone's theorem whose proof is omitted; you just have to trust the
person who proved it and the person who programmed it in and hope there are no
bugs.\index{computer program bugs}  Somehow this doesn't seem to be the
essence of mathematics.  Although computer algebra systems can generate
theorems with amazing speed, they can't actually prove a single one of them.

After some puzzlement, you revisit some popular books on what mathematics is
all about.  Somewhere you read that all of mathematics is actually derived
from something called ``set theory.''  This is a little confusing, because
nowhere in the book that presented the axioms of arithmetic was there any
mention of set theory, or if there was, it seemed to be just a tool that helps
you describe things better---the set of even numbers, that sort of thing.  If
set theory is the basis for all mathematics, then why are additional axioms
needed for arithmetic?

Something is wrong but you're not sure what.  One of your friends is a pure
mathematician.  He knows he is unable to communicate to you what he does for a
living and seems to have little interest in trying.  You do know that for him,
proofs are what mathematics is all about. You ask him what a proof is, and he
essentially tells you that, while of course it's based on logic, really it's
something you learn by doing it over and over until you pick it up.  He refers
you to a book, {\em How to Read and Do Proofs} \cite{Solow}.\index{Solow,
Daniel}  Although this book helps you understand traditional informal proofs,
there is still something missing you can't seem to pin down yet.

You ask your friend how you would go about having a computer verify a proof.
At first he seems puzzled by the question; why would you want to do that?
Then he says it's not something that would make any sense to do, but he's
heard that you'd have to break the proof down into thousands or even millions
of individual steps to do such a thing, because the reasoning involved is at
such a high level of abstraction.  He says that maybe it's something you could
do up to a point, but the computer would be completely impractical once you
get into any meaningful mathematics.  There, the only way you can verify a
proof is by hand, and you can only acquire the ability to do this by
specializing in the field for a couple of years in grad school.  Anyway, he
thinks it all has to do with set theory, although he has never taken a formal
course in set theory but just learned what he needed as he went along.

You are intrigued and amazed.  Apparently a mathematician can grasp as a
single concept something that would take a computer a thousand or a million
steps to verify, and have complete confidence in it.  Each one of these
thousand or million steps must be absolutely correct, or else the whole proof
is meaningless.  If you added a million numbers by hand, would you trust the
result?  How do you really know that all these steps are correct, that there
isn't some subtle pitfall in one of these million steps, like a bug in a
computer program?\index{computer program bugs}  After all, you've read that
famous mathematicians have occasionally made mistakes, and you certainly know
you've made your share on your math homework problems in school.

You recall the analogy with a computer program.  Sure, you can understand what
a large computer program such as a word processor does, as a single high-level
concept or a small set of such concepts, but your ability to understand it in
no way ensures that the program is correct and doesn't have hidden bugs.  Even
if you wrote the program yourself you can't really know this; most large
programs that you've written have had bugs that crop up at some later date, no
matter how careful you tried to be while writing them.

OK, so now it seems the reason you can't figure out how to make a
computer verify proofs is because each step really corresponds to a
million small steps.  Well, you say, a computer can do a million
calculations in a second, so maybe it's still practical to do.  Now the
puzzle becomes how to figure out what the million steps are that each
English-language step corresponds to.  Your mathematician friend hasn't
a clue, but suggests that maybe you would find the answer by studying
set theory.  Actually, your friend thinks you're a little off the wall
for even wondering such a thing.  For him, this is not what mathematics
is all about.

The subject of set theory keeps popping up, so you decide it's
time to look it up.

You decide to start off on a careful footing, so you start reading a couple of
very elementary books on set theory.  A lot of it seems pretty obvious, like
intersections, subsets, and Venn diagrams.  You thumb through one of the
books; nowhere is anything about axioms mentioned. The other book relegates to
an appendix a brief discussion that mentions a set of axioms called
``Zermelo--Fraenkel set theory''\index{Zermelo--Fraenkel set theory} and states
them in English.  You look at them and have no idea what they really mean or
what you can do with them.  The comments in this appendix say that the purpose
of mentioning them is to expose you to the idea, but imply that they are not
necessary for basic understanding and that they are really the subject matter
of advanced treatments where fine points such as a certain paradox (Russell's
paradox\index{Russell's paradox}\footnote{Russell's paradox assumes that there
exists a set $S$ that is a collection of all sets that don't contain
themselves.  Now, either $S$ contains itself or it doesn't.  If it contains
itself, it contradicts its definition.  But if it doesn't contain itself, it
also contradicts its definition.  Russell's paradox is resolved in ZF set
theory by denying that such a set $S$ exists.}) are resolved.  Wait a
minute---shouldn't the axioms be a starting point, not an ending point?  If
there are paradoxes that arise without the axioms, how do you know you won't
stumble across one accidentally when using the informal approach?

And nowhere do these books describe how ``all of mathematics can be
derived from set theory'' which by now you've heard a few times.

You find a more advanced book on set theory.  This one actually lists the
axioms of ZF set theory in plain English on page one.  {\em Now} you think
your quest has ended and you've finally found the source of all mathematical
knowledge; you just have to understand what it means.  Here, in one place, is
the basis for all of mathematics!  You stare at the axioms in awe, puzzle over
them, memorize them, hoping that if you just meditate on them long enough they
will become clear.  Of course, you haven't the slightest idea how the rest of
mathematics is ``derived'' from them; in particular, if these are the axioms
of mathematics, then why do arithmetic, group theory, and so on need their own
axioms?

You start reading this advanced book carefully, pondering the meaning of every
word, because by now you're really determined to get to the bottom of this.
The first thing the book does is explain how the axioms came about, which was
to resolve Russell's paradox.\index{Russell's paradox}  In fact that seems to
be the main purpose of their existence; that they supposedly can be used to
derive all of mathematics seems irrelevant and is not even mentioned.  Well,
you go on.  You hope the book will explain to you clearly, step by step, how
to derive things from the axioms.  After all, this is the starting point of
mathematics, like a book that explains the basics of a computer programming
language.  But something is missing.  You find you can't even understand the
first proof or do the first exercise.  Symbols such as $\exists$ and $\forall$
permeate the page without any mention of where they came from or how to
manipulate them; the author assumes you are totally familiar with them and
doesn't even tell you what they mean.  By now you know that $\exists$ means
``there exists'' and $\forall$ means ``for all,'' but shouldn't the rules for
manipulating these symbols be part of the axioms?  You still have no idea
how you could even describe the axioms to a computer.

Certainly there is something much different here from the technical
literature you're used to reading.  A computer language manual almost
always explains very clearly what all the symbols mean, precisely what
they do, and the rules used for combining them, and you work your way up
from there.

After glancing at four or five other such books, you come to the realization
that there is another whole field of study that you need just to get to the
point at which you can understand the axioms of set theory.  The field is
called ``logic.''  In fact, some of the books did recommend it as a
prerequisite, but it just didn't sink in.  You assumed logic was, well, just
logic, something that a person with common sense intuitively understood.  Why
waste your time reading boring treatises on symbolic logic, the manipulation
of 1's and 0's that computers do, when you already know that?  But this is a
different kind of logic, quite alien to you.  The subject of {\sc nand} and
{\sc nor} gates is not even touched upon or in any case has to do with only a
very small part of this field.

So your quest continues.  Skimming through the first couple of introductory
books, you get a general idea of what logic is about and what quantifiers
(``for all,'' ``there exists'') mean, but you find their examples somewhat
trivial and mildly annoying (``all dogs are animals,'' ``some animals are
dogs,'' and such).  But all you want to know is what the rules are for
manipulating the symbols so you can apply them to set theory.  Some formulas
describing the relationships among quantifiers ($\exists$ and $\forall$) are
listed in tables, along with some verbal reasoning to justify them.
Presumably, if you want to find out if a formula is correct, you go through
this same kind of mental reasoning process, possibly using images of dogs and
animals. Intuitively, the formulas seem to make sense.  But when you ask
yourself, ``What are the rules I need to get a computer to figure out whether
this formula is correct?'', you still don't know.  Certainly you don't ask the
computer to imagine dogs and animals.

You look at some more advanced logic books.  Many of them have an introductory
chapter summarizing set theory, which turns out to be a prerequisite.  You
need logic to understand set theory, but it seems you also need set theory to
understand logic!  These books jump right into proving rather advanced
theorems about logic, without offering the faintest clue about where the logic
came from that allows them to prove these theorems.

Luckily, you come across an elementary book of logic that, halfway through,
after the usual truth tables and metaphors, presents in a clear, precise way
what you've been looking for all along: the axioms!  They're divided into
propositional calculus (also called sentential logic) and predicate calculus
(also called first-order logic),\index{first-order logic} with rules so simple
and crystal clear that now you can finally program a computer to understand
them.  Indeed, they're no harder than learning how to play a game of chess.
As far as what you seem to need is concerned, the whole book could have been
written in five pages!

{\em Now} you think you've found the ultimate source of mathematical
truth.  So---the axioms of mathematics consist of these axioms of logic,
together with the axioms of ZF set theory. (By now you've also been able to
figure out how to translate the ZF axioms from English into the
actual symbols of logic which you can now manipulate according to
precise, easy-to-understand rules.)

Of course, you still don't understand how ``all of mathematics can be
derived from set theory,'' but maybe this will reveal itself in due
course.

You eagerly set out to program the axioms and rules into a computer and start
to look at the theorems you will have to prove as the logic is developed.  All
sorts of important theorems start popping up:  the deduction
theorem,\index{deduction theorem} the substitution theorem,\index{substitution
theorem} the completeness theorem of propositional calculus,\index{first-order
logic!completeness} the completeness theorem of predicate calculus.  Uh-oh,
there seems to be trouble.  They all get harder and harder, and not one of
them can be derived with the axioms and rules of logic you've just been
handed.  Instead, they all require ``metalogic'' for their proofs, a kind of
mixture of logic and set theory that allows you to prove things {\em about}
the axioms and theorems of logic rather than {\em with} them.

You plow ahead anyway.  A month later, you've spent much of your
free time getting the computer to verify proofs in propositional calculus.
You've programmed in the axioms, but you've also had to program in the
deduction theorem, the substitution theorem, and the completeness theorem of
propositional calculus, which by now you've resigned yourself to treating as
rather complex additional axioms, since they can't be proved from the axioms
you were given.  You can now get the computer to verify and even generate
complete, rigorous, formal proofs\index{formal proof}.  Never mind that they
may have 100,000 steps---at least now you can have complete, absolute
confidence in them.  Unfortunately, the only theorems you have proved are
pretty trivial and you can easily verify them in a few minutes with truth
tables, if not by inspection.

It looks like your mathematician friend was right.  Getting the computer to do
serious mathematics with this kind of rigor seems almost hopeless.  Even
worse, it seems that the further along you get, the more ``axioms'' you have
to add, as each new theorem seems to involve additional ``metamathematical''
reasoning that hasn't been formalized, and none of it can be derived from the
axioms of logic.  Not only do the proofs keep growing exponentially as you get
further along, but the program to verify them keeps getting bigger and bigger
as you program in more ``metatheorems.''\index{metatheorem}\footnote{A
metatheorem is usually a statement that is too general to be directly provable
in a theory.  For example, ``if $n_1$, $n_2$, and $n_3$ are integers, then
$n_1+n_2+n_3$ is an integer'' is a theorem of number theory.  But ``for any
integer $k > 1$, if $n_1, \ldots, n_k$ are integers, then $n_1+\ldots +n_k$ is
an integer'' is a metatheorem, in other words a family of theorems, one for
every $k$.  The reason it is not a theorem is that the general sum $n_1+\ldots
+n_k$ (as a function of $k$) is not an operation that can be defined directly
in number theory.} The bugs\index{computer program bugs} that have cropped up
so far have already made you start to lose faith in the rigor you seem to have
achieved, and you know it's just going to get worse as your program gets larger.

\subsection{Mathematics and the Non-Specialist}

\begin{quote}
  {\em A real proof is not checkable by a machine, or even by any mathematician
not privy to the gestalt, the mode of thought of the particular field of
mathematics in which the proof is located.}
  \flushright\sc  Davis and Hersh\index{Davis, Phillip J.}
  \footnote{\cite{Davis}, p.~354.}\\
\end{quote}

The bulk of abstract or theoretical mathematics is ordinarily outside
the reach of anyone but a few specialists in each field who have completed
the necessary difficult internship in order to enter its coterie.  The
typical intelligent layperson has no reasonable hope of understanding much of
it, nor even the specialist mathematician of understanding other fields.  It
is like a foreign language that has no dictionary to look up the translation;
the only way you can learn it is by living in the country for a few years.  It
is argued that the effort involved in learning a specialty is a necessary
process for acquiring a deep understanding.  Of course, this is almost certainly
true if one is to make significant contributions to a field; in particular,
``doing'' proofs is probably the most important part of a mathematician's
training.  But is it also necessary to deny outsiders access to it?  Is it
necessary that abstract mathematics be so hard for a layperson to grasp?

A computer normally is of no help whatsoever.  Most published proofs are
actually just series of hints written in an informal style that requires
considerable knowledge of the field to understand.  These are the ``real
proofs'' referred to by Davis and Hersh.\index{informal proof}  There is an
implicit understanding that, in principle, such a proof could be converted to
a complete formal proof\index{formal proof}.  However, it is said that no one
would ever attempt such a conversion, even if they could, because that would
presumably require millions of steps (Section~\ref{dream}).  Unfortunately the
informal style automatically excludes the understanding of the proof
by anyone who hasn't gone through the necessary apprenticeship. The
best that the intelligent layperson can do is to read popular books about deep
and famous results; while this can be helpful, it can also be misleading, and
the lack of detail usually leaves the reader with no ability whatsoever to
explore any aspect of the field being described.

The statements of theorems often use sophisticated notation that makes them
inaccessible to the non-specialist.  For a non-specialist who wants to achieve
a deeper understanding of a proof, the process of tracing definitions and
lemmas back through their hierarchy\index{hierarchy} quickly becomes confusing
and discouraging.  Textbooks are usually written to train mathematicians or to
communicate to people who are already mathematicians, and large gaps in proofs
are often left as exercises to the reader who is left at an impasse if he or
she becomes stuck.

I believe that eventually computers will enable non-specialists and even
intelligent laypersons to follow almost any mathematical proof in any field.
Metamath is an attempt in that direction.  If all of mathematics were as
easily accessible as a computer programming language, I could envision
computer programmers and hobbyists who otherwise lack mathematical
sophistication exploring and being amazed by the world of theorems and proofs
in obscure specialties, perhaps even coming up with results of their own.  A
tremendous advantage would be that anyone could experiment with conjectures in
any field---the computer would offer instant feedback as to whether
an inference step was correct.

Mathematicians sometimes have to put up with the annoyance of
cranks\index{cranks} who lack a fundamental understanding of mathematics but
insist that their ``proofs'' of, say, Fermat's Last Theorem\index{Fermat's
Last Theorem} be taken seriously.  I think part of the problem is that these
people are misled by informal mathematical language, treating it as if they
were reading ordinary expository English and failing to appreciate the
implicit underlying rigor.  Such cranks are rare in the field of computers,
because computer languages are much more explicit, and ultimately the proof is
in whether a computer program works or not.  With easily accessible
computer-based abstract mathematics, a mathematician could say to a crank,
``don't bother me until you've demonstrated your claim on the computer!''

% 22-May-04 nm
% Attempt to move De Millo quote so it doesn't separate from attribution
% CHANGE THIS NUMBER (AND ELIMINATE IF POSSIBLE) WHEN ABOVE TEXT CHANGES
\vspace{-0.5em}

\subsection{An Impossible Dream?}\label{dream}

\begin{quote}
  {\em Even quite basic theorems would demand almost unbelievably vast
  books to display their proofs.}
    \flushright\sc  Robert E. Edwards\footnote{\cite{Edwards}, p.~68.}\\
\end{quote}\index{Edwards, Robert E.}

\begin{quote}
  {\em Oh, of course no one ever really {\em does} it.  It would take
  forever!  You just show that you could do it, that's sufficient.}
    \flushright\sc  ``The Ideal Mathematician''
    \index{Davis, Phillip J.}\footnote{\cite{Davis},
p.~40.}\\
\end{quote}

\begin{quote}
  {\em There is a theorem in the primitive notation of set theory that
  corresponds to the arithmetic theorem `$1000+2000=3000$'.  The formula
  would be forbiddingly long\ldots even if [one] knows the definitions
  and is asked to simplify the long formula according to them, chances are
  he will make errors and arrive at some incorrect result.}
    \flushright\sc  Hao Wang\footnote{\cite{Wang}, p.~140.}\\
\end{quote}\index{Wang, Hao}

% 22-May-04 nm
% Attempt to move De Millo quote so it doesn't separate from attribution
% CHANGE THIS NUMBER (AND ELIMINATE IF POSSIBLE) WHEN ABOVE TEXT CHANGES
\vspace{-0.5em}

\begin{quote}
  {\em The {\em Principia Mathematica} was the crowning achievement of the
  formalists.  It was also the deathblow of the formalist view.\ldots
  {[Rus\-sell]} failed, in three enormous volumes, to get beyond the elementary
  facts of arithmetic.  He showed what can be done in principle and what
  cannot be done in practice.  If the mathematical process were really
  one of strict, logical progression, we would still be counting our
  fingers.\ldots
  One theoretician estimates\ldots that a demonstration of one of
  Ramanujan's conjectures assuming set theory and elementary analysis would
  take about two thousand pages; the length of a deduction from first principles
  is nearly in\-con\-ceiv\-a\-ble\ldots The probabilists argue that\ldots any
  very long proof can at best be viewed as only probably correct\ldots}
  \flushright\sc Richard de Millo et. al.\footnote{\cite{deMillo}, pp.~269,
  271.}\\
\end{quote}\index{de Millo, Richard}

A number of writers have conveyed the impression that the kind of absolute
rigor provided by Metamath\index{Metamath} is an impossible dream, suggesting
that a complete, formal verification\index{formal proof} of a typical theorem
would take millions of steps in untold volumes of books.  Even if it could be
done, the thinking sometimes goes, all meaning would be lost in such a
monstrous, tedious verification.\index{informal proof}\index{proof length}

These writers assume, however, that in order to achieve the kind of complete
formal verification they desire one must break down a proof into individual
primitive steps that make direct reference to the axioms.  This is
not necessary.  There is no reason not to make use of previously proved
theorems rather than proving them over and over.

Just as important, definitions\index{definition} can be introduced along
the way, allowing very complex formulas to be represented with few
symbols.  Not doing this can lead to absurdly long formulas.  For
example, the mere statement of
G\"{o}del's incompleteness theorem\index{G\"{o}del's
incompleteness theorem}, which can be expressed with a small number of
defined symbols, would require about 20,000 primitive symbols to express
it.\index{Boolos, George S.}\footnote{George S.\ Boolos, lecture at
Massachusetts Institute of Technology, spring 1990.} An extreme example
is Bourbaki's\label{bourbaki} language for set theory, which requires
4,523,659,424,929 symbols plus 1,179,618,517,981 disambiguatory links
(lines connecting symbol pairs, usually drawn below or above the
formula) to express the number
``one'' \cite{Mathias}.\index{Mathias, Adrian R. D.}\index{Bourbaki,
Nicolas}
% http://www.dpmms.cam.ac.uk/~ardm/

A hierarchy\index{hierarchy} of theorems and definitions permits an
exponential growth in the formula sizes and primitive proof steps to be
described with only a linear growth in the number of symbols used.  Of course,
this is how ordinary informal mathematics is normally done anyway, but with
Metamath\index{Metamath} it can be done with absolute rigor and precision.

\subsection{Beauty}


\begin{quote}
  {\em No one shall be able to drive us from the paradise that Cantor has
created for us.}
   \flushright\sc  David Hilbert\footnote{As quoted in \cite{Moore}, p.~131.}\\
\end{quote}\index{Hilbert, David}

\needspace{3\baselineskip}
\begin{quote}
  {\em Mathematics possesses not only truth, but some supreme beauty ---a
  beauty cold and austere, like that of a sculpture.}
    \flushright\sc  Bertrand
    Russell\footnote{\cite{Russell}.}\\
\end{quote}\index{Russell, Bertrand}

\begin{quote}
  {\em Euclid alone has looked on Beauty bare.}
  \flushright\sc Edna Millay\footnote{As quoted in \cite{Davis}, p.~150.}\\
\end{quote}\index{Millay, Edna}

For most people, abstract mathematics is distant, strange, and
incomprehensible.  Many popular books have tried to convey some of the sense
of beauty in famous theorems.  But even an intelligent layperson is left with
only a general idea of what a theorem is about and is hardly given the tools
needed to make use of it.  Traditionally, it is only after years of arduous
study that one can grasp the concepts needed for deep understanding.
Metamath\index{Metamath} allows you to approach the proof of the theorem from
a quite different perspective, peeling apart the formulas and definitions
layer by layer until an entirely different kind of understanding is achieved.
Every step of the proof is there, pieced together with absolute precision and
instantly available for inspection through a microscope with a magnification
as powerful as you desire.

A proof in itself can be considered an object of beauty.  Constructing an
elegant proof is an art.  Once a famous theorem has been proved, often
considerable effort is made to find simpler and more easily understood
proofs.  Creating and communicating elegant proofs is a major concern of
mathematicians.  Metamath is one way of providing a common language for
archiving and preserving this information.

The length of a proof can, to a certain extent, be considered an
objective measure of its ``beauty,'' since shorter proofs are usually
considered more elegant.  In the set theory database
\texttt{set.mm}\index{set theory database (\texttt{set.mm})}%
\index{Metamath Proof Explorer}
provided with Metamath, one goal was to make all proofs as short as possible.

\needspace{4\baselineskip}
\subsection{Simplicity}

\begin{quote}
  {\em God made man simple; man's complex problems are of his own
  devising.}
    \flushright\sc Eccles. 7:29\footnote{Jerusalem Bible.}\\
\end{quote}\index{Bible}

\needspace{3\baselineskip}
\begin{quote}
  {\em God made integers, all else is the work of man.}
    \flushright\sc Leopold Kronecker\footnote{{\em Jahresbericht
	der Deutschen Mathematiker-Vereinigung }, vol. 2, p. 19.}\\
\end{quote}\index{Kronecker, Leopold}

\needspace{3\baselineskip}
\begin{quote}
  {\em For what is clear and easily comprehended attracts; the
  complicated repels.}
    \flushright\sc David Hilbert\footnote{As quoted in \cite{deMillo},
p.~273.}\\
\end{quote}\index{Hilbert, David}

The Metamath\index{Metamath} language is simple and Spartan.  Metamath treats
all mathematical expressions as simple sequences of symbols, devoid of meaning.
The higher-level or ``metamathematical'' notions underlying Metamath are about
as simple as they could possibly be.  Each individual step in a proof involves
a single basic concept, the substitution of an expression for a variable, so
that in principle almost anyone, whether mathematician or not, can
completely understand how it was arrived at.

In one of its most basic applications, Metamath\index{Metamath} can be used to
develop the foundations of mathematics\index{foundations of mathematics} from
the very beginning.  This is done in the set theory database that is provided
with the Metamath package and is the subject matter
of Chapter~\ref{fol}. Any language (a metalanguage\index{metalanguage})
used to describe mathematics (an object language\index{object language}) must
have a mathematical content of its own, but it is desirable to keep this
content down to a bare minimum, namely that needed to make use of the
inference rules specified by the axioms.  With any metalanguage there is a
``chicken and egg'' problem somewhat like circular reasoning:  you must assume
the validity of the mathematics of the metalanguage in order to prove the
validity of the mathematics of the object language.  The mathematical content
of Metamath itself is quite limited.  Like the rules of a game of chess, the
essential concepts are simple enough so that virtually anyone should be able to
understand them (although that in itself will not let you play like
a master).  The symbols that Metamath manipulates do not in themselves
have any intrinsic meaning.  Your interpretation of the axioms that you supply
to Metamath is what gives them meaning.  Metamath is an attempt to strip down
mathematical thought to its bare essence and show you exactly how the symbols
are manipulated.

Philosophers and logicians, with various motivations, have often thought it
important to study ``weak'' fragments of logic\index{weak logic}
\cite{Anderson}\index{Anderson, Alan Ross} \cite{MegillBunder}\index{Megill,
Norman}\index{Bunder, Martin}, other unconventional systems of logic (such as
``modal'' logic\index{modal logic} \cite[ch.\ 27]{Boolos}\index{Boolos, George
S.}), and quantum logic\index{quantum logic} in physics
\cite{Pavicic}\index{Pavi{\v{c}}i{\'{c}}, M.}.  Metamath\index{Metamath}
provides a framework in which such systems can be expressed, with an absolute
precision that makes all underlying metamathematical assumptions rigorous and
crystal clear.

Some schools of philosophical thought, for example
intuitionism\index{intuitionism} and constructivism\index{constructivism},
demand that the notions underlying any mathematical system be as simple and
concrete as possible.  Metamath should meet the requirements of these
philosophies.  Metamath must be taught the symbols, axioms\index{axiom}, and
rules\index{rule} for a specific theory, from the skeptical (such as
intuitionism\index{intuitionism}\footnote{Intuitionism does not accept the law
of excluded middle (``either something is true or it is not true'').  See
\cite[p.~xi]{Tymoczko}\index{Tymoczko, Thomas} for discussion and references
on this topic.  Consider the theorem, ``There exist irrational numbers $a$ and
$b$ such that $a^b$ is rational.''  An intuitionist would reject the following
proof:  If $\sqrt{2}^{\sqrt{2}}$ is rational, we are done.  Otherwise, let
$a=\sqrt{2}^{\sqrt{2}}$ and $b=\sqrt{2}$. Then $a^b=2$, which is rational.})
to the bold (such as the axiom of choice in set theory\footnote{The axiom of
choice\index{Axiom of Choice} asserts that given any collection of pairwise
disjoint nonempty sets, there exists a set that has exactly one element in
common with each set of the collection.  It is used to prove many important
theorems in standard mathematics.  Some philosophers object to it because it
asserts the existence of a set without specifying what the set contains
\cite[p.~154]{Enderton}\index{Enderton, Herbert B.}.  In one foundation for
mathematics due to Quine\index{Quine, Willard Van Orman}, that has not been
otherwise shown to be inconsistent, the axiom of choice turns out to be false
\cite[p.~23]{Curry}\index{Curry, Haskell B.}.  The \texttt{show
trace{\char`\_}back} command of the Metamath program allows you to find out
whether the axiom of choice, or any other axiom, was assumed by a
proof.}\index{\texttt{show trace{\char`\_}back} command}).

The simplicity of the Metamath language lets the algorithm (computer program)
that verifies the validity of a Metamath proof be straightforward and
robust.  You can have confidence that the theorems it verifies really can be
derived from your axioms.

\subsection{Rigor}

\begin{quote}
  {\em Rigor became a goal with the Greeks\ldots But the efforts to
  pursue rigor to the utmost have led to an impasse in which there is
  no longer any agreement on what it really means.  Mathematics remains
  alive and vital, but only on a pragmatic basis.}
    \flushright\sc  Morris Kline\footnote{\cite{Kline}, p.~1209.}\\
\end{quote}\index{Kline, Morris}

Kline refers to a much deeper kind of rigor than that which we will discuss in
this section.  G\"{o}del's incompleteness theorem\index{G\"{o}del's
incompleteness theorem} showed that it is impossible to achieve absolute rigor
in standard mathematics because we can never prove that mathematics is
consistent (free from contradictions).\index{consistent theory}  If
mathematics is consistent, we will never know it, but must rely on faith.  If
mathematics is inconsistent, the best we can hope for is that some clever
future mathematician will discover the inconsistency.  In this case, the
axioms would probably be revised slightly to eliminate the inconsistency, as
was done in the case of Russell's paradox,\index{Russell's paradox} but the
bulk of mathematics would probably not be affected by such a discovery.
Russell's paradox, for example, did not affect most of the remarkable results
achieved by 19th-century and earlier mathematicians.  It mainly invalidated
some of Gottlob Frege's\index{Frege, Gottlob} work on the foundations of
mathematics in the late 1800's; in fact Frege's work inspired Russell's
discovery.  Despite the paradox, Frege's work contains important concepts that
have significantly influenced modern logic.  Kline's {\em Mathematics, The
Loss of Certainty} \cite{Klinel}\index{Kline, Morris} has an interesting
discussion of this topic.

What {\em can} be achieved with absolute certainty\index{certainty} is the
knowledge that if we assume the axioms are consistent and true, then the
results derived from them are true.  Part of the beauty of mathematics is that
it is the one area of human endeavor where absolute certainty can be achieved
in this sense.  A mathematical truth will remain such for eternity.  However,
our actual knowledge of whether a particular statement is a mathematical truth
is only as certain as the correctness of the proof that establishes it.  If
the proof of a statement is questionable or vague, we can't have absolute
confidence in the truth that the statement claims.

Let us look at some traditional ways of expressing proofs.

Except in the field of formal logic\index{formal logic}, almost all
traditional proofs in mathematics are really not proofs at all, but rather
proof outlines or hints as to how to go about constructing the proof.  Many
gaps\index{gaps in proofs} are left for the reader to fill in. There are
several reasons for this.  First, it is usually assumed in mathematical
literature that the person reading the proof is a mathematician familiar with
the specialty being described, and that the missing steps are obvious to such
a reader or at least that the reader is capable of filling them in.  This
attitude is fine for professional mathematicians in the specialty, but
unfortunately it often has the drawback of cutting off the rest of the world,
including mathematicians in other specialties, from understanding the proof.
We discussed one possible resolution to this on p.~\pageref{envision}.
Second, it is often assumed that a complete formal proof\index{formal proof}
would require countless millions of symbols (Section~\ref{dream}). This might
be true if the proof were to be expressed directly in terms of the axioms of
logic and set theory,\index{set theory} but it is usually not true if we allow
ourselves a hierarchy\index{hierarchy} of definitions and theorems to build
upon, using a notation that allows us to introduce new symbols, definitions,
and theorems in a precisely specified way.

Even in formal logic,\index{formal logic} formal proofs\index{formal proof}
that are considered complete still contain hidden or implicit information.
For example, a ``proof'' is usually defined as a sequence of
wffs,\index{well-formed formula (wff)}\footnote{A {\em wff} or well-formed
formula is a mathematical expression (string of symbols) constructed according
to some precise rules.  A formal mathematical system\index{formal system}
contains (1) the rules for constructing syntactically correct
wffs,\index{syntax rules} (2) a list of starting wffs called
axioms,\index{axiom} and (3) one or more rules prescribing how to derive new
wffs, called theorems\index{theorem}, from the axioms or previously derived
theorems.  An example of such a system is contained in
Metamath's\index{Metamath} set theory database, which defines a formal
system\index{formal system} from which all of standard mathematics can be
derived.  Section~\ref{startf} steps you through a complete example of a formal
system, and you may want to skim it now if you are unfamiliar with the
concept.} each of which is an axiom or follows from a rule applied to previous
wffs in the sequence.  The implicit part of the proof is the algorithm by
which a sequence of symbols is verified to be a valid wff, given the
definition of a wff.  The algorithm in this case is rather simple, but for a
computer to verify the proof,\index{automated proof verification} it must have
the algorithm built into its verification program.\footnote{It is possible, of
course, to specify wff construction syntax outside of the program itself
with a suitable input language (the Metamath language being an example), but
some proof-verification or theorem-proving programs lack the ability to extend
wff syntax in such a fashion.} If one deals exclusively with axioms and
elementary wffs, it is straightforward to implement such an algorithm.  But as
more and more definitions are added to the theory in order to make the
expression of wffs more compact, the algorithm becomes more and more
complicated.  A computer program that implements the algorithm becomes larger
and harder to understand as each definition is introduced, and thus more prone
to bugs.\index{computer program bugs}  The larger the program, the
more suspicious the mathematician may be about
the validity of its algorithms.  This is especially true because
computer programs are inherently hard to follow to begin with, and few people
enjoy verifying them manually in detail.

Metamath\index{Metamath} takes a different approach.  Metamath's ``knowledge''
is limited to the ability to substitute variables for expressions, subject to
some simple constraints.  Once the basic algorithm of Metamath is assumed to
be debugged, and perhaps independently confirmed, it
can be trusted once and for all.  The information that Metamath needs to
``understand'' mathematics is contained entirely in the body of knowledge
presented to Metamath.  Any errors in reasoning can only be errors in the
axioms or definitions contained in this body of knowledge.  As a
``constructive'' language\index{constructive language} Metamath has no
conditional branches or loops like the ones that make computer programs hard
to decipher; instead, the language can only build new sequences of symbols
from earlier sequences  of symbols.

The simplicity of the rules that underlie Metamath not only makes Metamath
easy to learn but also gives Metamath a great deal of flexibility. For
example, Metamath is not limited to describing standard first-order
logic\index{first-order logic}; higher-order logics\index{higher-order logic}
and fragments of logic\index{weak logic} can be described just as easily.
Metamath gives you the freedom to define whatever wff notation you prefer; it
has no built-in conception of the syntax of a wff.\index{well-formed formula
(wff)}  With suitable axioms and definitions, Metamath can even describe and
prove things about itself.\index{Metamath!self-description}  (John
Harrison\index{Harrison, John} discusses the ``reflection''
principle\index{reflection principle} involved in self-descriptive systems in
\cite{Harrison}.)

The flexibility of Metamath requires that its proofs specify a lot of detail,
much more than in an ordinary ``formal'' proof.\index{formal proof}  For
example, in an ordinary formal proof, a single step consists of displaying the
wff that constitutes that step.  In order for a computer program to verify
that the step is acceptable, it first must verify that the symbol sequence
being displayed is an acceptable wff.\index{automated proof verification} Most
proof verifiers have at least basic wff syntax built into their programs.
Metamath has no hard-wired knowledge of what constitutes a wff built into it;
instead every wff must be explicitly constructed based on rules defining wffs
that are present in a database.  Thus a single step in an ordinary formal
proof may be correspond to many steps in a Metamath proof. Despite the larger
number of steps, though, this does not mean that a Metamath proof must be
significantly larger than an ordinary formal proof. The reason is that since
we have constructed the wff from scratch, we know what the wff is, so there is
no reason to display it.  We only need to refer to a sequence of statements
that construct it.  In a sense, the display of the wff in an ordinary formal
proof is an implicit proof of its own validity as a wff; Metamath just makes
the proof explicit. (Section~\ref{proof} describes Metamath's proof notation.)

\section{Computers and Mathematicians}

\begin{quote}
  {\em The computer is important, but not to mathematics.}
    \flushright\sc  Paul Halmos\footnote{As quoted in \cite{Albers}, p.~121.}\\
\end{quote}\index{Halmos, Paul}

Pure mathematicians have traditionally been indifferent to computers, even to
the point of disdain.\index{computers and pure mathematics}  Computer science
itself is sometimes considered to fall in the mundane realm of ``applied''
mathematics, perhaps essential for the real world but intellectually unexciting
to those who seek the deepest truths in mathematics.  Perhaps a reason for this
attitude towards computers is that there is little or no computer software that
meets their needs, and there may be a general feeling that such software could
not even exist.  On the one hand, there are the practical computer algebra
systems, which can perform amazing symbolic manipulations in algebra and
calculus,\index{computer algebra system} yet can't prove the simplest
existence theorem, if the idea of a proof is present at all.  On the other
hand, there are specialized automated theorem provers that technically speaking
may generate correct proofs.\index{automated theorem proving}  But sometimes
their specialized input notation may be cryptic and their output perceived to
be long, inelegant, incomprehensible proofs.    The output
may be viewed with suspicion, since the program that generates it tends to be
very large, and its size increases the potential for bugs\index{computer
program bugs}.  Such a proof may be considered trustworthy only if
independently verified and ``understood'' by a human, but no one wants to
waste their time on such a boring, unrewarding chore.



\needspace{4\baselineskip}
\subsection{Trusting the Computer}

\begin{quote}
  {\em \ldots I continue to find the quasi-empirical interpretation of
  computer proofs to be the more plausible.\ldots Since not
  everything that claims to be a computer proof can be
  accepted as valid, what are the mathematical criteria for acceptable
  computer proofs?}
    \flushright\sc  Thomas Tymoczko\footnote{\cite{Tymoczko}, p.~245.}\\
\end{quote}\index{Tymoczko, Thomas}

In some cases, computers have been essential tools for proving famous
theorems.  But if a proof is so long and obscure that it can be verified in a
practical way only with a computer, it is vaguely felt to be suspicious.  For
example, proving the famous four-color theorem\index{four-color
theorem}\index{proof length} (``a map needs no more than four colors to
prevent any two adjacent countries from having the same color'') can presently
only be done with the aid of a very complex computer program which originally
required 1200 hours of computer time. There has been considerable debate about
whether such a proof can be trusted and whether such a proof is ``real''
mathematics \cite{Swart}\index{Swart, E. R.}.\index{trusting computers}

However, under normal circumstances even a skeptical mathematician would have a
great deal of confidence in the result of multiplying two numbers on a pocket
calculator, even though the precise details of what goes on are hidden from its
user.  Even the verification on a supercomputer that a huge number is prime is
trusted, especially if there is independent verification; no one bothers to
debate the philosophical significance of its ``proof,'' even though the actual
proof would be so large that it would be completely impractical to ever write
it down on paper.  It seems that if the algorithm used by the computer is
simple enough to be readily understood, then the computer can be trusted.

Metamath\index{Metamath} adopts this philosophy.  The simplicity of its
language makes it easy to learn, and because of its simplicity one can have
essentially absolute confidence that a proof is correct. All axioms, rules, and
definitions are available for inspection at any time because they are defined
by the user; there are no hidden or built-in rules that may be prone to subtle
bugs\index{computer program bugs}.  The basic algorithm at the heart of
Metamath is simple and fixed, and it can be assumed to be bug-free and robust
with a degree of confidence approaching certainty.
Independently written implementations of the Metamath verifier
can reduce any residual doubt on the part of a skeptic even further;
there are now over a dozen such implementations, written by many people.

\subsection{Trusting the Mathematician}\label{trust}

\begin{quote}
  {\em There is no Algebraist nor Mathematician so expert in his science, as
  to place entire confidence in any truth immediately upon his discovery of it,
  or regard it as any thing, but a mere probability.  Every time he runs over
  his proofs, his confidence encreases; but still more by the approbation of
  his friends; and is rais'd to its utmost perfection by the universal assent
  and applauses of the learned world.}
  \flushright\sc David Hume\footnote{{\em A Treatise of Human Nature}, as
  quoted in \cite{deMillo}, p.~267.}\\
\end{quote}\index{Hume, David}

\begin{quote}
  {\em Stanislaw Ulam estimates that mathematicians publish 200,000 theorems
  every year.  A number of these are subsequently contradicted or otherwise
  disallowed, others are thrown into doubt, and most are ignored.}
  \flushright\sc Richard de Millo et. al.\footnote{\cite{deMillo}, p.~269.}\\
\end{quote}\index{Ulam, Stanislaw}

Whether or not the computer can be trusted, humans  of course will occasionally
err. Only the most memorable proofs get independently verified, and of these
only a handful of truly great ones achieve the status of being ``known''
mathematical truths that are used without giving a second thought to their
correctness.

There are many famous examples of incorrect theorems and proofs in
mathematical literature.\index{errors in proofs}

\begin{itemize}
\item There have been thousands of purported proofs of Fermat's Last
Theorem\index{Fermat's Last Theorem} (``no integer solutions exist to $x^n +
y^n = z^n$ for $n > 2$''), by amateurs, cranks, and well-regarded
mathematicians \cite[p.~5]{Stark}\index{Stark, Harold M}.  Fermat wrote a note
in his copy of Bachet's {\em Diophantus} that he found ``a truly marvelous
proof of this theorem but this margin is too narrow to contain it''
\cite[p.~507]{Kramer}.  A recent, much publicized proof by Yoichi
Miyaoka\index{Miyaoka, Yoichi} was shown to be incorrect ({\em Science News},
April 9, 1988, p.~230).  The theorem was finally proved by Andrew
Wiles\index{Wiles, Andrew} ({\em Science News}, July 3, 1993, p.~5), but it
initially had some gaps and took over a year after its announcement to be
checked thoroughly by experts.  On Oct. 25, 1994, Wiles announced that the last
gap found in his proof had been filled in.
  \item In 1882, M. Pasch discovered that an axiom was omitted from Euclid's
formulation of geometry\index{Euclidean geometry}; without it, the proofs of
important theorems of Euclid are not valid.  Pasch's axiom\index{Pasch's
axiom} states that a line that intersects one side of a triangle must also
intersect another side, provided that it does not touch any of the triangle's
vertices.  The omission of Pasch's axiom went unnoticed for 2000
years \cite[p.~160]{Davis}, in spite of (one presumes) the thousands of
students, instructors, and mathematicians who studied Euclid.
  \item The first published proof of the famous Schr\"{o}der--Bernstein
theorem\index{Schr\"{o}der--Bernstein theorem} in set theory was incorrect
\cite[p.~148]{Enderton}\index{Enderton, Herbert B.}.  This theorem states
that if there exists a 1-to-1 function\footnote{A {\em set}\index{set} is any
collection of objects. A {\em function}\index{function} or {\em
mapping}\index{mapping} is a rule that assigns to each element of one set
(called the function's {\em domain}\index{domain}) an element from another
set.} from set $A$ into set $B$ and vice-versa, then sets $A$ and $B$ have
a 1-to-1 correspondence.  Although it sounds simple and obvious,
the standard proof is quite long and complex.
  \item In the early 1900's, Hilbert\index{Hilbert, David} published a
purported proof of the continuum hypothesis\index{continuum hypothesis}, which
was eventually established as unprovable by Cohen\index{Cohen, Paul} in 1963
\cite[p.~166]{Enderton}.  The continuum hypothesis states that no
infinity\index{infinity} (``transfinite cardinal number'')\index{cardinal,
transfinite} exists whose size (or ``cardinality''\index{cardinality}) is
between the size of the set of integers and the size of the set of real
numbers.  This hypothesis originated with German mathematician Georg
Cantor\index{Cantor, Georg} in the late 1800's, and his inability to prove it
is said to have contributed to mental illness that afflicted him in his later
years.
  \item An incorrect proof of the four-color theorem\index{four-color theorem}
was published by Kempe\index{Kempe, A. B.} in 1879
\cite[p.~582]{Courant}\index{Courant, Richard}; it stood for 11 years before
its flaw was discovered.  This theorem states that any map can be colored
using only four colors, so that no two adjacent countries have the same
color.  In 1976 the theorem was finally proved by the famous computer-assisted
proof of Haken, Appel, and Koch \cite{Swart}\index{Appel, K.}\index{Haken,
W.}\index{Koch, K.}.  Or at least it seems that way.  Mathematician
H.~S.~M.~Coxeter\index{Coxeter, H. S. M.} has doubts \cite[p.~58]{Davis}:  ``I
have a feeling that that is an untidy kind of use of the computers, and the more
you correspond with Haken and Appel, the more shaky you seem to be.''
  \item Many false ``proofs'' of the Poincar\'{e}
conjecture\index{Poincar\'{e} conjecture} have been proposed over the years.
This conjecture states that any object that mathematically behaves like a
three-dimensional sphere is a three-dimensional sphere topologically,
regardless of how it is distorted.  In March 1986, mathematicians Colin
Rourke\index{Rourke, Colin} and Eduardo R\^{e}go\index{R\^{e}go, Eduardo}
caused  a stir in the mathematical community by announcing that they had found
a proof; in November of that year the proof was found to be false \cite[p.
218]{PetersonI}.  It was finally proved in 2003 by Grigory Perelman
\label{poincare}\index{Szpiro, George}\index{Perelman, Grigory}\cite{Szpiro}.
 \end{itemize}

Many counterexamples to ``theorems'' in recent mathematical
literature related to Clifford algebras\index{Clifford algebras}
 have been found by Pertti
Lounesto (who passed away in 2002).\index{Lounesto, Pertti}
See the web page \url{http://mathforum.org/library/view/4933.html}.
% http://users.tkk.fi/~ppuska/mirror/Lounesto/counterexamples.htm

One of the purposes of Metamath\index{Metamath} is to allow proofs to be
expressed with absolute precision.  Developing a proof in the Metamath
language can be challenging, because Metamath will not permit even the
tiniest mistake.\index{errors in proofs}  But once the proof is created, its
correctness can be trusted immediately, without having to depend on the
process of peer review for confirmation.

\section{The Use of Computers in Mathematics}

\subsection{Computer Algebra Systems}

For the most part, you will find that Metamath\index{Metamath} is not a
practical tool for manipulating numbers.  (Even proving that $2 + 2 = 4$, if
you start with set theory, can be quite complex!)  Several commercial
mathematics packages are quite good at arithmetic, algebra, and calculus, and
as practical tools they are invaluable.\index{computer algebra system} But
they have no notion of proof, and cannot understand statements starting with
``there exists such and such...''.

Software packages such as Mathematica \cite{Wolfram}\index{Mathematica} do not
concern themselves with proofs but instead work directly with known results.
These packages primarily emphasize heuristic rules such as the substitution of
equals for equals to achieve simpler expressions or expressions in a different
form.  Starting with a rich collection of built-in rules and algorithms, users
can add to the collection by means of a powerful programming language.
However, results such as, say, the existence of a certain abstract object
without displaying the actual object cannot be expressed (directly) in their
languages.  The idea of a proof from a small set of axioms is absent.  Instead
this software simply assumes that each fact or rule you add to the built-in
collection of algorithms is valid.  One way to view the software is as a large
collection of axioms from which the software, with certain goals, attempts to
derive new theorems, for example equating a complex expression with a simpler
equivalent. But the terms ``theorem''\index{theorem} and
``proof,''\index{proof} for example, are not even mentioned in the index of
the user's manual for Mathematica.\index{Mathematica and proofs}  What is also
unsatisfactory from a philosophical point of view is that there is no way to
ensure the validity of the results other than by trusting the writer of each
application module or tediously checking each module by hand, similar to
checking a computer program for bugs.\index{computer program
bugs}\footnote{Two examples illustrate why the knowledge database of computer
algebra systems should sometimes be regarded with a certain caution.  If you
ask Mathematica (version 3.0) to \texttt{Solve[x\^{ }n + y\^{ }n == z\^{ }n , n]}
it will respond with \texttt{\{\{n-\char`\>-2\}, \{n-\char`\>-1\},
\{n-\char`\>1\}, \{n-\char`\>2\}\}}. In other words, Mathematica seems to
``know'' that Fermat's Last Theorem\index{Fermat's Last Theorem} is true!  (At
the time this version of Mathematica was released this fact was unknown.)  If
you ask Maple\index{Maple} to \texttt{solve(x\^{ }2 = 2\^{ }x)} then
\texttt{simplify(\{"\})}, it returns the solution set \texttt{\{2, 4\}}, apparently
unaware that $-0.7666647$\ldots is also a solution.} While of course extremely
valuable in applied mathematics, computer algebra systems tend to be of little
interest to the theoretical mathematician except as aids for exploring certain
specific problems.

Because of possible bugs, trusting the output of a computer algebra system for
use as theorems in a proof-verifier would defeat the latter's goal of rigor.
On the other hand, a fact such that a certain relatively large number is
prime, while easy for a computer algebra system to derive, might have a long,
tedious proof that could overwhelm a proof-verifier. One approach for linking
computer algebra systems to a proof-verifier while retaining the advantages of
both is to add a hypothesis to each such theorem indicating its source.  For
example, a constant {\sc maple} could indicate the theorem came from the Maple
package, and would mean ``assuming Maple is consistent, then\ldots''  This and
many other topics concerning the formalization of mathematics are discussed in
John Harrison's\index{Harrison, John} very interesting
PhD thesis~\cite{Harrison-thesis}.

\subsection{Automated Theorem Provers}\label{theoremprovers}

A mathematical theory is ``decidable''\index{decidable theory} if a mechanical
method or algorithm exists that is guaranteed to determine whether or not a
particular formula is a theorem.  Among the few theories that are decidable is
elementary geometry,\index{Euclidean geometry} as was shown by a classic
result of logician Alfred Tarski\index{Tarski, Alfred} in 1948
\cite{Tarski}.\footnote{Tarski's result actually applies to a subset of the
geometry discussed in elementary textbooks.  This subset includes most of what
would be considered elementary geometry but it is not powerful enough to
express, among other things, the notions of the circumference and area of a
circle.  Extending the theory in a way that includes notions such as these
makes the theory undecidable, as was also shown by Tarski.  Tarski's algorithm
is far too inefficient to implement practically on a computer.  A practical
algorithm for proving a smaller subset of geometry theorems (those not
involving concepts of ``order'' or ``continuity'') was discovered by Chinese
mathematician Wu Wen-ts\"{u}n in 1977 \cite{Chou}\index{Chou,
Shang-Ching}.}\index{Wen-ts{\"{u}}n, Wu}  But most theories, including
elementary arithmetic, are undecidable.  This fact contributes to keeping
mathematics alive and well, since many mathematicians believe
that they will never be
replaced by computers (if they believe Roger Penrose's argument that a
computer can never replace the brain \cite{Penrose}\index{Penrose, Roger}).
In fact,  elementary geometry is often considered a ``dead'' field
for the simple reason that it is decidable.

On the other hand, the undecidability of a theory does not mean that one cannot
use a computer to search for proofs, providing one is willing to give up if a
proof is not found after a reasonable amount of time.  The field of automated
theorem proving\index{automated theorem proving} specializes in pursuing such
computer searches.  Among the more successful results to date are those based
on an algorithm known as Robinson's resolution principle
\cite{Robinson}\index{Robinson's resolution principle}.

Automated theorem provers can be excellent tools for those willing to learn
how to use them.  But they are not widely used in mainstream pure
mathematics, even though they could probably be useful in many areas.  There
are several reasons for this.  Probably most important, the main goal in pure
mathematics is to arrive at results that are considered to be deep or
important; proving them is essential but secondary.  Usually, an automated
theorem prover cannot assist in this main goal, and by the time the main goal
is achieved, the mathematician may have already figured out the proof as a
by-product.  There is also a notational problem.  Mathematicians are used to
using very compact syntax where one or two symbols (heavily dependent on
context) can represent very complex concepts; this is part of the
hierarchy\index{hierarchy} they have built up to tackle difficult problems.  A
theorem prover on the other hand might require that a theorem be expressed in
``first-order logic,''\index{first-order logic} which is the logic on which
most of mathematics is ultimately based but which is not ordinarily used
directly because expressions can become very long.  Some automated theorem
provers are experimental programs, limited in their use to very specialized
areas, and the goal of many is simply research into the nature of automated
theorem proving itself.  Finally, much research remains to be done to enable
them to prove very deep theorems.  One significant result was a
computer proof by Larry Wos\index{Wos, Larry} and colleagues that every Robbins
algebra\index{Robbins algebra} is a Boolean  algebra\index{Boolean algebra}
({\em New York Times}, Dec. 10, 1996).\footnote{In 1933, E.~V.\
Huntington\index{Huntington, E. V.}
presented the following axiom system for
Boolean algebra with a unary operation $n$ and a binary operation $+$:
\begin{center}
    $x + y = y + x$ \\
    $(x + y) + z = x + (y + z)$ \\
    $n(n(x) + y) + n(n(x) + n(y)) = x$
\end{center}
Herbert Robbins\index{Robbins, Herbert}, a student of Huntington, conjectured
that the last equation can be replaced with a simpler one:
\begin{center}
    $n(n(x + y) + n(x + n(y))) = x$
\end{center}
Robbins and Huntington could not find a proof.  The problem was
later studied unsuccessfully by Tarski\index{Tarski, Alfred} and his
students, and it remained an unsolved problem until a
computer found the proof in 1996.  For more information on
the Robbins algebra problem see \cite{Wos}.}

How does Metamath\index{Metamath} relate to automated theorem provers?  A
theorem prover is primarily concerned with one theorem at a time (perhaps
tapping into a small database of known theorems) whereas Metamath is more like
a theorem archiving system, storing both the theorem and its proof in a
database for access and verification.  Metamath is one answer to ``what do you
do with the output of a theorem prover?''  and could be viewed as the
next step in the process.  Automated theorem provers could be useful tools for
helping develop its database.
Note that very long, automatically
generated proofs can make your database fat and ugly and cause Metamath's proof
verification to take a long time to run.  Unless you have a particularly good
program that generates very concise proofs, it might be best to consider the
use of automatically generated proofs as a quick-and-dirty approach, to be
manually rewritten at some later date.

The program {\sc otter}\index{otter@{\sc otter}}\footnote{\url{http://www.cs.unm.edu/\~mccune/otter/}.}, later succeeded by
prover9\index{prover9}\footnote{\url{https://www.cs.unm.edu/~mccune/mace4/}.},
have been historically influential.
The E prover\index{E prover}\footnote{\url{https://github.com/eprover/eprover}.}
is a maintained automated theorem prover
for full first-order logic with equality.
There are many other automated theorem provers as well.

If you want to combine automated theorem provers with Metamath
consider investigating
the book {\em Automated Reasoning:  Introduction and Applications}
\cite{Wos}\index{Wos, Larry}.  This book discusses
how to use {\sc otter} in a way that can
not only able to generate
relatively efficient proofs, it can even be instructed to search for
shorter proofs.  The effective use of {\sc otter} (and similar tools)
does require a certain
amount of experience, skill, and patience.  The axiom system used in the
\texttt{set.mm}\index{set theory database (\texttt{set.mm})} set theory
database can be expressed to {\sc otter} using a method described in
\cite{Megill}.\index{Megill, Norman}\footnote{To use those axioms with
{\sc otter}, they must be restated in a way that eliminates the need for
``dummy variables.''\index{dummy variable!eliminating} See the Comment
on p.~\pageref{nodd}.} When successful, this method tends to generate
short and clever proofs, but my experiments with it indicate that the
method will find proofs within a reasonable time only for relatively
easy theorems.  It is still fun to experiment with.

Reference \cite{Bledsoe}\index{Bledsoe, W. W.} surveys a number of approaches
people have explored in the field of automated theorem proving\index{automated
theorem proving}.

\subsection{Interactive Theorem Provers}\label{interactivetheoremprovers}

Finding proofs completely automatically is difficult, so there
are some interactive theorem provers that allow a human to guide the
computer to find a proof.
Examples include
HOL Light\index{HOL light}%
\footnote{\url{https://www.cl.cam.ac.uk/~jrh13/hol-light/}.},
Isabelle\index{Isabelle}%
\footnote{\url{http://www.cl.cam.ac.uk/Research/HVG/Isabelle}.},
{\sc hol}\index{hol@{\sc hol}}%
\footnote{\url{https://hol-theorem-prover.org/}.},
and
Coq\index{Coq}\footnote{\url{https://coq.inria.fr/}.}.

A major difference between most of these tools and Metamath is that the
``proofs'' are actually programs that guide the program to find a proof,
and not the proof itself.
For example, an Isabelle/HOL proof might apply a step
\texttt{apply (blast dest: rearrange reduction)}. The \texttt{blast}
instruction applies
an automatic tableux prover and returns if it found a sequence of proof
steps that work... but the sequence is not considered part of the proof.

A good overview of
higher-level proof verification languages (such as {\sc lcf}\index{lcf@{\sc
lcf}} and {\sc hol}\index{hol@{\sc hol}})
is given in \cite{Harrison}.  All of these languages are fundamentally
different from Metamath in that much of the mathematical foundational
knowledge is embedded in the underlying proof-verification program, rather
than placed directly in the database that is being verified.
These can have a steep learning curve for those without a mathematical
background.  For example, one usually must have a fair understanding of
mathematical logic in order to follow their proofs.

\subsection{Proof Verifiers}\label{proofverifiers}

A proof verifier is a program that doesn't generate proofs but instead
verifies proofs that you give it.  Many proof verifiers have limited built-in
automated proof capabilities, such as figuring out simple logical inferences
(while still being guided by a person who provides the overall proof).  Metamath
has no built-in automated proof capability other than the limited
capability of its Proof Assistant.

Proof-verification languages are not used as frequently as they might be.
Pure mathematicians are more concerned with producing new results, and such
detail and rigor would interfere with that goal.  The use of computers in pure
mathematics is primarily focused on automated theorem provers (not verifiers),
again with the ultimate goal of aiding the creation of new mathematics.
Automated theorem provers are usually concerned with attacking one theorem at
time rather than making a large, organized database easily available to the
user.  Metamath is one way to help close this gap.

By itself Metamath is a mostly a proof verifier.
This does not mean that other approaches can't be used; the difference
is that in Metamath, the results of various provers must be recorded
step-by-step so that they can be verified.

Another proof-verification language is Mizar,\index{Mizar} which can display
its proofs in the informal language that mathematicians are accustomed to.
Information on the Mizar language is available at \url{http://mizar.org}.

For the working mathematician, Mizar is an excellent tool for rigorously
documenting proofs. Mizar typesets its proofs in the informal English used by
mathematicians (and, while fine for them, are just as inscrutable by
laypersons!). A price paid for Mizar is a relatively steep learning curve of a
couple of weeks.  Several mathematicians are actively formalizing different
areas of mathematics using Mizar and publishing the proofs in a dedicated
journal. Unfortunately the task of formalizing mathematics is still looked
down upon to a certain extent since it doesn't involve the creation of ``new''
mathematics.

The closest system to Metamath is
the {\em Ghilbert}\index{Ghilbert} proof language (\url{http://ghilbert.org})
system developed by
Raph Levien\index{Levien, Raph}.
Ghilbert is a formal proof checker heavily inspired by Metamath.
Ghilbert statements are s-expressions (a la Lisp), which is easy
for computers to parse but many people find them hard to read.
There are a number of differences in their specific constructs, but
there is at least one tool to translate some Metamath materials into Ghilbert.
As of 2019 the Ghilbert community is smaller and less active than the
Metamath community.
That said, the Metamath and Ghilbert communities overlap, and fruitful
conversations between them have occurred many times over the years.

\subsection{Creating a Database of Formalized Mathematics}\label{mathdatabase}

Besides Metamath, there are several other ongoing projects with the goal of
formalizing mathematics into computer-verifiable databases.
Understanding some history will help.

The {\sc qed}\index{qed project@{\sc qed} project}%
\footnote{\url{http://www-unix.mcs.anl.gov/qed}.}
project arose in 1993 and its goals were outlined in the
{\sc qed} manifesto.
The {\sc qed} manifesto was
a proposal for a computer-based database of all mathematical knowledge,
strictly formalized and with all proofs having been checked automatically.
The project had a conference in 1994 and another in 1995;
there was also a ``twenty years of the {\sc qed} manifesto'' workshop
in 2014.
Its ideals are regularly reraised.

In a 2007 paper, Freek Wiedijk identified two reasons
for the failure of the {\sc qed} project as originally envisioned:%
\cite{Wiedijk-revisited}\index{Wiedijk, Freek}

\begin{itemize}
\item Very few people are working on formalization of mathematics. There is no compelling application for fully mechanized mathematics.
\item Formalized mathematics does not yet resemble traditional mathematics. This is partly due to the complexity of mathematical notation, and partly to the limitations of existing theorem provers and proof assistants.
\end{itemize}

But this did not end the dream of
formalizing mathematics into computer-verifiable databases.
The problems that led to the {\sc qed} manifesto are still with us,
even though the challenges were harder than originally considered.
What has happened instead is that various independent projects have
worked towards formalizing mathematics into computer-verifiable databases,
each simultaneously competing and cooperating with each other.

A concrete way to see this is
Freek Wiedijk's ``Formalizing 100 Theorems'' list%
\footnote{\url{http://www.cs.ru.nl/\%7Efreek/100/}.}
which shows the progress different systems have made on a challenge list
of 100 mathematical theorems.%
\footnote{ This is not the only list of ``interesting'' theorems.
Another interesting list was posted by Oliver Knill's list
\cite{Knill}\index{Knill, Oliver}.}
The top systems as of February 2019
(in order of the number of challenges completed) are
HOL Light, Isabelle, Metamath, Coq, and Mizar.

The Metamath 100%
\footnote{\url{http://us.metamath.org/mm\_100.html}}
page (maintained by David A. Wheeler\index{Wheeler, David A.})
shows the progress of Metamath (specifically its \texttt{set.mm} database)
against this challenge list maintained by Freek Wiedijk.
The Metamath \texttt{set.mm} database
has made a lot of progress over the years,
in part because working to prove those challenge theorems required
defining various terms and proving their properties as a prerequisite.
Here are just a few of the many statements that have been
formally proven with Metamath:

% The entries of this cause the narrow display to break poorly,
% since the short amount of text means LaTeX doesn't get a lot to work with
% and the itemize format gives it even *less* margin than usual.
% No one will mind if we make just this list flushleft, since this list
% will be internally consistent.
\begin{flushleft}
\begin{itemize}
\item 1. The Irrationality of the Square Root of 2
  (\texttt{sqr2irr}, by Norman Megill, 2001-08-20)
\item 2. The Fundamental Theorem of Algebra
  (\texttt{fta}, by Mario Carneiro, 2014-09-15)
\item 22. The Non-Denumerability of the Continuum
  (\texttt{ruc}, by Norman Megill, 2004-08-13)
\item 54. The Konigsberg Bridge Problem
  (\texttt{konigsberg}, by Mario Carneiro, 2015-04-16)
\item 83. The Friendship Theorem
  (\texttt{friendship}, by Alexander W. van der Vekens, 2018-10-09)
\end{itemize}
\end{flushleft}

We thank all of those who have developed at least one of the Metamath 100
proofs, and we particularly thank
Mario Carneiro\index{Carneiro, Mario}
who has contributed the most Metamath 100 proofs as of 2019.
The Metamath 100 page shows the list of all people who have contributed a
proof, and links to graphs and charts showing progress over time.
We encourage others to work on proving theorems not yet proven in Metamath,
since doing so improves the work as a whole.

Each of the math formalization systems (including Metamath)
has different strengths and weaknesses, depending on what you value.
Key aspects that differentiate Metamath from the other top systems are:

\begin{itemize}
\item Metamath is not tied to any particular set of axioms.
\item Metamath can show every step of every proof, no exceptions.
  Most other provers only assert that a proof can be found, and do not
  show every step. This also makes verification fast, because
  the system does not need to rediscover proof details.
\item The Metamath verifier has been re-implemented in many different
  programming languages, so verification can be done by multiple
  implementations.  In particular, the
  \texttt{set.mm}\index{set theory database (\texttt{set.mm})}%
  \index{Metamath Proof Explorer} database is verified by
  four different verifiers
  written in four different languages by four different authors.
  This greatly reduces the risk of accepting an invalid
  proof due to an error in the verifier.
\item Proofs stay proven.  In some systems, changes to the system's
  syntax or how a tactic works causes proofs to fail in later versions,
  causing older work to become essentially lost.
  Metamath's language is
  extremely small and fixed, so once a proof is added to a database,
  the database can be rechecked with later versions of the Metamath program
  and with other verifiers of Metamath databases.
  If an axiom or key definition needs to be changed, it is easy to
  manipulate the database as a whole to handle the change
  without touching the underlying verifier.
  Since re-verification of an entire database takes seconds, there
  is never a reason to delay complete verification.
  This aspect is especially compelling if your
  goal is to have a long-term database of proofs.
\item Licensing is generous.  The main Metamath databases are released to
  the public domain, and the main Metamath program is open source software
  under a standard, widely-used license.
\item Substitutions are easy to understand, even by those who are not
  professional mathematicians.
\end{itemize}

Of course, other systems may have advantages over Metamath
that are more compelling, depending on what you value.
In any case, we hope this helps you understand Metamath
within a wider context.

\subsection{In Summary}\label{computers-summary}

To summarize our discussions of computers and mathematics, computer algebra
systems can be viewed as theorem generators focusing on a narrow realm of
mathematics (numbers and their properties), automated theorem provers as proof
generators for specific theorems in a much broader realm covered by a built-in
formal system such as first-order logic, interactive theorem
provers require human guidance, proof verifiers verify proofs but
historically they have been
restricted to first-order logic.
Metamath, in contrast,
is a proof verifier and documenter whose realm is essentially unlimited.

\section{Mathematics and Metamath}

\subsection{Standard Mathematics}

There are a number of ways that Metamath\index{Metamath} can be used with
standard mathematics.  The most satisfying way philosophically is to start at
the very beginning, and develop the desired mathematics from the axioms of
logic and set theory.\index{set theory}  This is the approach taken in the
\texttt{set.mm}\index{set theory database (\texttt{set.mm})}%
\index{Metamath Proof Explorer}
database (also known as the Metamath Proof Explorer).
Among other things, this database builds up to the
axioms of real and complex numbers\index{analysis}\index{real and complex
numbers} (see Section~\ref{real}), and a standard development of analysis, for
example, could start at that point, using it as a basis.   Besides this
philosophical advantage, there are practical advantages to having all of the
tools of set theory available in the supporting infrastructure.

On the other hand, you may wish to start with the standard axioms of a
mathematical theory without going through the set theoretical proofs of those
axioms.  You will need mathematical logic to make inferences, but if you wish
you can simply introduce theorems\index{theorem} of logic as
``axioms''\index{axiom} wherever you need them, with the implicit assumption
that in principle they can be proved, if they are obvious to you.  If you
choose this approach, you will probably want to review the notation used in
\texttt{set.mm}\index{set theory database (\texttt{set.mm})} so that your own
notation will be consistent with it.

\subsection{Other Formal Systems}
\index{formal system}

Unlike some programs, Metamath\index{Metamath} is not limited to any specific
area of mathematics, nor committed to any particular mathematical philosophy
such as classical logic versus intuitionism, nor limited, say, to expressions
in first-order logic.  Although the database \texttt{set.mm}
describes standard logic and set theory, Meta\-math
is actually a general-purpose language for describing a wide variety of formal
systems.\index{formal system}  Non-standard systems such as modal
logic,\index{modal logic} intuitionist logic\index{intuitionism}, higher-order
logic\index{higher-order logic}, quantum logic\index{quantum logic}, and
category theory\index{category theory} can all be described with the Metamath
language.  You define the symbols you prefer and tell Metamath the axioms and
rules you want to start from, and Metamath will verify any inferences you make
from those axioms and rules.  A simple example of a non-standard formal system
is Hofstadter's\index{Hofstadter, Douglas R.} MIU system,\index{MIU-system}
whose Metamath description is presented in Appendix~\ref{MIU}.

This is not hypothetical.
The largest Metamath database is
\texttt{set.mm}\index{set theory database (\texttt{set.mm}}%
\index{Metamath Proof Explorer}), aka the Metamath Proof Explorer,
which uses the most common axioms for mathematical foundations
(specifically classical logic combined with Zermelo--Fraenkel
set theory\index{Zermelo--Fraenkel set theory} with the Axiom of Choice).
But other Metamath databases are available:

\begin{itemize}
\item The database
  \texttt{iset.mm}\index{intuitionistic logic database (\texttt{iset.mm})},
  aka the
  Intuitionistic Logic Explorer\index{Intuitionistic Logic Explorer},
  uses intuitionistic logic (a constructivist point of view)
  instead of classical logic.
\item The database
  \texttt{nf.mm}\index{New Foundations database (\texttt{nf.mm})},
  aka the
  New Foundations Explorer\index{New Foundations Explorer},
  constructs mathematics from scratch,
  starting from Quine's New Foundations (NF) set theory axioms.
\item The database
  \texttt{hol.mm}\index{Higher-order Logic database (\texttt{hol.mm})},
  aka the
  Higher-Order Logic (HOL) Explorer\index{Higher-Order Logic (HOL) Explorer},
  starts with HOL (also called simple type theory) and derives
  equivalents to ZFC axioms, connecting the two approaches.
\end{itemize}

Since the days of David Hilbert,\index{Hilbert, David} mathematicians have
been concerned with the fact that the metalanguage\index{metalanguage} used to
describe mathematics may be stronger than the mathematics being described.
Metamath\index{Metamath}'s underlying finitary\index{finitary proof},
constructive nature provides a good philosophical basis for studying even the
weakest logics.\index{weak logic}

The usual treatment of many non-standard formal systems\index{formal
system} uses model theory\index{model theory} or proof theory\index{proof
theory} to describe these systems; these theories, in turn, are based on
standard set theory.  In other words, a non-standard formal system is defined
as a set with certain properties, and standard set theory is used to derive
additional properties of this set.  The standard set theory database provided
with Metamath can be used for this purpose, and when used this way
the development of a special
axiom system for the non-standard formal system becomes unnecessary.  The
model- or proof-theoretic approach often allows you to prove much deeper
results with less effort.

Metamath supports both approaches.  You can define the non-standard
formal system directly, or define the non-standard formal system as
a set with certain properties, whichever you find most helpful.

%\section{Additional Remarks}

\subsection{Metamath and Its Philosophy}

Closely related to Metamath\index{Metamath} is a philosophy or way of looking
at mathematics. This philosophy is related to the formalist
philosophy\index{formalism} of Hilbert\index{Hilbert, David} and his followers
\cite[pp.~1203--1208]{Kline}\index{Kline, Morris}
\cite[p.~6]{Behnke}\index{Behnke, H.}. In this philosophy, mathematics is
viewed as nothing more than a set of rules that manipulate symbols, together
with the consequences of those rules.  While the mathematics being described
may be complex, the rules used to describe it (the
``metamathematics''\index{metamathematics}) should be as simple as possible.
In particular, proofs should be restricted to dealing with concrete objects
(the symbols we write on paper rather than the abstract concepts they
represent) in a constructive manner; these are called ``finitary''
proofs\index{finitary proof} \cite[pp.~2--3]{Shoenfield}\index{Shoenfield,
Joseph R.}.

Whether or not you find Metamath interesting or useful will in part depend on
the appeal you find in its philosophy, and this appeal will probably depend on
your particular goals with respect to mathematics.  For example, if you are a
pure mathematician at the forefront of discovering new mathematical knowledge,
you will probably find that the rigid formality of Metamath stifles your
creativity.  On the other hand, we would argue that once this knowledge is
discovered, there are advantages to documenting it in a standard format that
will make it accessible to others.  Sixty years from now, your field may be
dormant, and as Davis and Hersh put it, your ``writings would become less
translatable than those of the Maya'' \cite[p.~37]{Davis}\index{Davis, Phillip
J.}.


\subsection{A History of the Approach Behind Metamath}

Probably the one work that has had the most motivating influence on
Metamath\index{Metamath} is Whitehead and Russell's monumental {\em Principia
Mathematica} \cite{PM}\index{Whitehead, Alfred North}\index{Russell,
Bertrand}\index{principia mathematica@{\em Principia Mathematica}}, whose aim
was to deduce all of mathematics from a small number of primitive ideas, in a
very explicit way that in principle anyone could understand and follow.  While
this work was tremendously influential in its time, from a modern perspective
it suffers from several drawbacks.  Both its notation and its underlying
axioms are now considered dated and are no longer used.  From our point of
view, its development is not really as accessible as we would like to see; for
practical reasons, proofs become more and more sketchy as its mathematics
progresses, and working them out in fine detail requires a degree of
mathematical skill and patience that many people don't have.  There are
numerous small errors, which is understandable given the tedious, technical
nature of the proofs and the lack of a computer to verify the details.
However, even today {\em Principia Mathematica} stands out as the work closest
in spirit to Metamath.  It remains a mind-boggling work, and one can't help
but be amazed at seeing ``$1+1=2$'' finally appear on page 83 of Volume II
(Theorem *110.643).

The origin of the proof notation used by Metamath dates back to the 1950's,
when the logician C.~A.~Meredith expressed his proofs in a compact notation
called ``condensed detachment''\index{condensed detachment}
\cite{Hindley}\index{Hindley, J. Roger} \cite{Kalman}\index{Kalman, J. A.}
\cite{Meredith}\index{Meredith, C. A.} \cite{Peterson}\index{Peterson, Jeremy
George}.  This notation allows proofs to be communicated unambiguously by
merely referencing the axiom\index{axiom}, rule\index{rule}, or
theorem\index{theorem} used at each step, without explicitly indicating the
substitutions\index{substitution!variable}\index{variable substitution} that
have to be made to the variables in that axiom, rule, or theorem.  Ordinarily,
condensed detachment is more or less limited to propositional
calculus\index{propositional calculus}.  The concept has been extended to
first-order logic\index{first-order logic} in \cite{Megill}\index{Megill,
Norman}, making it is easy to write a small computer program to verify proofs
of simple first-order logic theorems.\index{condensed detachment!and
first-order logic}

A key concept behind the notation of condensed detachment is called
``unification,''\index{unification} which is an algorithm for determining what
substitutions\index{substitution!variable}\index{variable substitution} to
variables have to be made to make two expressions match each other.
Unification was first precisely defined by the logician J.~A.~Robinson, who
used it in the development of a powerful
theorem-proving technique called the ``resolution principle''
\cite{Robinson}\index{Robinson's resolution principle}. Metamath does not make
use of the resolution principle, which is intended for systems of first-order
logic.\index{first-order logic}  Metamath's use is not restricted to
first-order logic, and as we have mentioned it does not automatically discover
proofs.  However, unification is a key idea behind Metamath's proof
notation, and Metamath makes use of a very simple version of it
(Section~\ref{unify}).

\subsection{Metamath and First-Order Logic}

First-order logic\index{first-order logic} is the supporting structure
for standard mathematics.  On top of it is set theory, which contains
the axioms from which virtually all of mathematics can be derived---a
remarkable fact.\index{category
theory}\index{cardinal, inaccessible}\label{categoryth}\footnote{An exception seems
to be category theory.  There are several schools of thought on whether
category theory is derivable from set theory.  At a minimum, it appears
that an additional axiom is needed that asserts the existence of an
``inaccessible cardinal'' (a type of infinity so large that standard set
theory can't prove or deny that it exists).
%
%%%% (I took this out that was in previous editions:)
% But it is also argued that not just one but a ``proper class'' of them
% is needed, and the existence of proper classes is impossible in standard
% set theory.  (A proper class is a collection of sets so huge that no set
% can contain it as an element.  Proper classes can lead to
% inconsistencies such as ``Russell's paradox.''  The axioms of standard
% set theory are devised so as to deny the existence of proper classes.)
%
For more information, see
\cite[pp.~328--331]{Herrlich}\index{Herrlich, Horst} and
\cite{Blass}\index{Blass, Andrea}.}

One of the things that makes Metamath\index{Metamath} more practical for
first-order theories is a set of axioms for first-order logic designed
specifically with Metamath's approach in mind.  These are included in
the database \texttt{set.mm}\index{set theory database (\texttt{set.mm})}.
See Chapter~\ref{fol} for a detailed
description; the axioms are shown in Section~\ref{metaaxioms}.  While
logically equivalent to standard axiom systems, our axiom system breaks
up the standard axioms into smaller pieces such that from them, you can
directly derive what in other systems can only be derived as higher-level
``metatheorems.''\index{metatheorem}  In other words, it is more powerful than
the standard axioms from a metalogical point of view.  A rigorous
justification for this system and its ``metalogical
completeness''\index{metalogical completeness} is found in
\cite{Megill}\index{Megill, Norman}.  The system is closely related to a
system developed by Monk\index{Monk, J. Donald} and Tarski\index{Tarski,
Alfred} in 1965 \cite{Monks}.

For example, the formula $\exists x \, x = y $ (given $y$, there exists some
$x$ equal to it) is a theorem of logic,\footnote{Specifically, it is a theorem
of those systems of logic that assume non-empty domains.  It is not a theorem
of more general systems that include the empty domain\index{empty domain}, in
which nothing exists, period!  Such systems are called ``free
logics.''\index{free logic} For a discussion of these systems, see
\cite{Leblanc}\index{Leblanc, Hugues}.  Since our use for logic is as a basis
for set theory, which has a non-empty domain, it is more convenient (and more
traditional) to use a less general system.  An interesting curiosity is that,
using a free logic as a basis for Zermelo--Fraenkel set
theory\index{Zermelo--Fraenkel set theory} (with the redundant Axiom of the
Null Set omitted),\index{Axiom of the Null Set} we cannot even prove the
existence of a single set without assuming the axiom of infinity!\index{Axiom
of Infinity}} whether or not $x$ and $y$ are distinct variables\index{distinct
variables}.  In many systems of logic, we would have to prove two theorems to
arrive at this result.  First we would prove ``$\exists x \, x = x $,'' then
we would separately prove ``$\exists x \, x = y $, where $x$ and $y$ are
distinct variables.''  We would then combine these two special cases ``outside
of the system'' (i.e.\ in our heads) to be able to claim, ``$\exists x \, x =
y $, regardless of whether $x$ and $y$ are distinct.''  In other words, the
combination of the two special cases is a
metatheorem.  In the system of logic
used in Metamath's set theory\index{set theory database (\texttt{set.mm})}
database, the axioms of logic are broken down into small pieces that allow
them to be reassembled in such a way that theorems such as these can be proved
directly.

Breaking down the axioms in this way makes them look peculiar and not very
intuitive at first, but rest assured that they are correct and complete.  Their
correctness is ensured because they are theorem schemes of standard first-order
logic (which you can easily verify if you are a logician).  Their completeness
follows from the fact that we explicitly derive the standard axioms of
first-order logic as theorems.  Deriving the standard axioms is somewhat
tricky, but once we're there, we have at our disposal a system that is less
awkward to work with in formal proofs\index{formal proof}.  In technical terms
that logicians understand, we eliminate the cumbersome concepts of ``free
variable,''\index{free variable} ``bound variable,''\index{bound variable} and
``proper substitution''\index{proper substitution}\index{substitution!proper}
as primitive notions.  These concepts are present in our system but are
defined in terms of concepts expressed by the axioms and can be eliminated in
principle.  In standard systems, these concepts are really like additional,
implicit axioms\index{implicit axiom} that are somewhat complex and cannot be
eliminated.

The traditional approach to logic, wherein free variables and proper
substitution is defined, is also possible to do directly in the Metamath
language.  However, the notation tends to become awkward, and there are
disadvantages:  for example, extending the definition of a wff with a
definition is awkward, because the free variable and proper substitution
concepts have to have their definitions also extended.  Our choice of
axioms for \texttt{set.mm} is to a certain extent a matter of style, in
an attempt to achieve overall simplicity, but you should also be aware
that the traditional approach is possible as well if you should choose
it.

\chapter{Using the Metamath Program}
\label{using}

\section{Installation}

The way that you install Metamath\index{Metamath!installation} on your
computer system will vary for different computers.  Current
instructions are provided with the Metamath program download at
\url{http://metamath.org}.  In general, the installation is simple.
There is one file containing the Metamath program itself.  This file is
usually called \texttt{metamath} or \texttt{metamath.}{\em xxx} where
{\em xxx} is the convention (such as \texttt{exe}) for an executable
program on your operating system.  There are several additional files
containing samples of the Metamath language, all ending with
\texttt{.mm}.  The file \texttt{set.mm}\index{set theory database
(\texttt{set.mm})} contains logic and set theory and can be used as a
starting point for other areas of mathematics.

You will also need a text editor\index{text editor} capable of editing plain
{\sc ascii}\footnote{American Standard Code for Information Interchange.} text
in order to prepare your input files.\index{ascii@{\sc ascii}}  Most computers
have this capability built in.  Note that plain text is not necessarily the
default for some word processing programs\index{word processor}, especially if
they can handle different fonts; for example, with Microsoft Word\index{Word
(Microsoft)}, you must save the file in the format ``Text Only With Line
Breaks'' to get a plain text\index{plain text} file.\footnote{It is
recommended that all lines in a Metamath source file be 79 characters or less
in length for compatibility among different computer terminals.  When creating
a source file on an editor such as Word, select a monospaced
font\index{monospaced font} such as Courier\index{Courier font} or
Monaco\index{Monaco font} to make this easier to achieve.  Better yet,
just use a plain text editor such as Notepad.}

On some computer systems, Metamath does not have the capability to print
its output directly; instead, you send its output to a file (using the
\texttt{open} commands described later).  The way you print this output
file depends on your computer.\index{printers} Some computers have a
print command, whereas with others, you may have to read the file into
an editor and print it from there.

If you want to print your Metamath source files with typeset formulas
containing standard mathematical symbols, you will need the \LaTeX\
typesetting program\index{latex@{\LaTeX}}, which is widely and freely
available for most operating systems.  It runs natively on Unix and
Linux, and can be installed on Windows as part of the free Cygwin
package (\url{http://cygwin.com}).

You can also produce {\sc html}\footnote{HyperText Markup Language.}
web pages.  The {\tt help html} command in the Metamath program will
assist you with this feature.

\section{Your First Formal System}\label{start}
\subsection{From Nothing to Zero}\label{startf}

To give you a feel for what the Metamath\index{Metamath} language looks like,
we will take a look at a very simple example from formal number
theory\index{number theory}.  This example is taken from
Mendelson\index{Mendelson, Elliot} \cite[p. 123]{Mendelson}.\footnote{To keep
the example simple, we have changed the formalism slightly, and what we call
axioms\index{axiom} are strictly speaking theorems\index{theorem} in
\cite{Mendelson}.}  We will look at a small subset of this theory, namely that
part needed for the first number theory theorem proved in \cite{Mendelson}.

First we will look at a standard formal proof\index{formal proof} for the
example we have picked, then we will look at the Metamath version.  If you
have never been exposed to formal proofs, the notation may seem to be such
overkill to express such simple notions that you may wonder if you are missing
something.  You aren't.  The concepts involved are in fact very simple, and a
detailed breakdown in this fashion is necessary to express the proof in a way
that can be verified mechanically.  And as you will see, Metamath breaks the
proof down into even finer pieces so that the mechanical verification process
can be about as simple as possible.

Before we can introduce the axioms\index{axiom} of the theory, we must define
the syntax rules for forming legal expressions\index{syntax rules}
(combinations of symbols) with which those axioms can be used. The number 0 is
a {\bf term}\index{term}; and if $ t$ and $r$ are terms, so is $(t+r)$. Here,
$ t$ and $r$ are ``metavariables''\index{metavariable} ranging over terms; they
themselves do not appear as symbols in an actual term.  Some examples of
actual terms are $(0 + 0)$ and $((0+0)+0)$.  (Note that our theory describes
only the number zero and sums of zeroes.  Of course, not much can be done with
such a trivial theory, but remember that we have picked a very small subset of
complete number theory for our example.  The important thing for you to focus
on is our definitions that describe how symbols are combined to form valid
expressions, and not on the content or meaning of those expressions.) If $ t$
and $r$ are terms, an expression of the form $ t=r$ is a {\bf wff}
(well-formed formula)\index{well-formed formula (wff)}; and if $P$ and $Q$ are
wffs, so is $(P\rightarrow Q)$ (which means ``$P$ implies
$Q$''\index{implication ($\rightarrow$)} or ``if $P$ then $Q$'').
Here $P$ and $Q$ are metavariables ranging over wffs.  Examples of actual
wffs are $0=0$, $(0+0)=0$, $(0=0 \rightarrow (0+0)=0)$, and $(0=0\rightarrow
(0=0\rightarrow 0=(0+0)))$.  (Our notation makes use of more parentheses than
are customary, but the elimination of ambiguity this way simplifies our
example by avoiding the need to define operator precedence\index{operator
precedence}.)

The {\bf axioms}\index{axiom} of our theory are all wffs of the following
form, where $ t$, $r$, and $s$ are any terms:

%Latex p. 92
\renewcommand{\theequation}{A\arabic{equation}}

\begin{equation}
(t=r\rightarrow (t=s\rightarrow r=s))
\end{equation}
\begin{equation}
(t+0)=t
\end{equation}

Note that there are an infinite number of axioms since there are an infinite
number of possible terms.  A1 and A2 are properly called ``axiom
schemes,''\index{axiom scheme} but we will refer to them as ``axioms'' for
brevity.

An axiom is a {\bf theorem}; and if $P$ and $(P\rightarrow Q)$ are theorems
(where $P$ and $Q$ are wffs), then $Q$ is also a theorem.\index{theorem}  The
second part of this definition is called the modus ponens (MP) rule of
inference\index{inference rule}\index{modus ponens}.  It allows us to obtain
new theorems from old ones.

The {\bf proof}\index{proof} of a theorem is a sequence of one or more
theorems, each of which is either an axiom or the result of modus ponens
applied to two previous theorems in the sequence, and the last of which is the
theorem being proved.

The theorem we will prove for our example is very simple:  $ t=t$.  The proof of
our theorem follows.  Study it carefully until you feel sure you
understand it.\label{zeroproof}

% Use tabu so that lines will wrap automatically as needed.
\begin{tabu} { l X X }
1. & $(t+0)=t$ & (by axiom A2) \\
2. & $(t+0)=t$ & (by axiom A2) \\
3. & $((t+0)=t \rightarrow ((t+0)=t\rightarrow t=t))$ & (by axiom A1) \\
4. & $((t+0)=t\rightarrow t=t)$ & (by MP applied to steps 2 and 3) \\
5. & $t=t$ & (by MP applied to steps 1 and 4) \\
\end{tabu}

(You may wonder why step 1 is repeated twice.  This is not necessary in the
formal language we have defined, but in Metamath's ``reverse Polish
notation''\index{reverse Polish notation (RPN)} for proofs, a previous step
can be referred to only once.  The repetition of step~1 here will enable you
to see more clearly the correspondence of this proof with the
Metamath\index{Metamath} version on p.~\pageref{demoproof}.)

Our theorem is more properly called a ``theorem scheme,''\index{theorem
scheme} for it represents an infinite number of theorems, one for each
possible term $ t$.  Two examples of actual theorems would be $0=0$ and
$(0+0)=(0+0)$.  Rarely do we prove actual theorems, since by proving schemes
we can prove an infinite number of theorems in one fell swoop.  Similarly, our
proof should really be called a ``proof scheme.''\index{proof scheme}  To
obtain an actual proof, pick an actual term to use in place of $ t$, and
substitute it for $ t$ throughout the proof.

Let's discuss what we have done here.  The axioms\index{axiom} of our theory,
A1 and A2, are trivial and obvious.  Everyone knows that adding zero to
something doesn't change it, and also that if two things are equal to a third,
then they are equal to each other. In fact, stating the trivial and obvious is
a goal to strive for in any axiomatic system.  From trivial and obvious truths
that everyone agrees upon, we can prove results that are not so obvious yet
have absolute faith in them.  If we trust the axioms and the rules, we must,
by definition, trust the consequences of those axioms and rules, if logic is
to mean anything at all.

Our rule of inference\index{rule}, modus ponens\index{modus ponens}, is also
pretty obvious once you understand what it means.  If we prove a fact $P$, and
we also prove that $P$ implies $Q$, then $Q$ necessarily follows as a new
fact.  The rule provides us with a means for obtaining new facts (i.e.\
theorems\index{theorem}) from old ones.

The theorem that we have proved, $ t=t$, is so fundamental that you may wonder
why it isn't one of the axioms\index{axiom}.  In some axiom systems of
arithmetic, it {\em is} an axiom.  The choice of axioms in a theory is to some
extent arbitrary and even an art form, constrained only by the requirement
that any two equivalent axiom systems be able to derive each other as
theorems.  We could imagine that the inventor of our axiom system originally
included $ t=t$ as an axiom, then discovered that it could be derived as a
theorem from the other axioms.  Because of this, it was not necessary to
keep it as an axiom.  By eliminating it, the final set of axioms became
that much simpler.

Unless you have worked with formal proofs\index{formal proof} before, it
probably wasn't apparent to you that $ t=t$ could be derived from our two
axioms until you saw the proof. While you certainly believe that $ t=t$ is
true, you might not be able to convince an imaginary skeptic who believes only
in our two axioms until you produce the proof.  Formal proofs such as this are
hard to come up with when you first start working with them, but after you get
used to them they can become interesting and fun.  Once you understand the
idea behind formal proofs you will have grasped the fundamental principle that
underlies all of mathematics.  As the mathematics becomes more sophisticated,
its proofs become more challenging, but ultimately they all can be broken down
into individual steps as simple as the ones in our proof above.

Mendelson's\index{Mendelson, Elliot} book, from which our example was taken,
contains a number of detailed formal proofs such as these, and you may be
interested in looking it up.  The book is intended for mathematicians,
however, and most of it is rather advanced.  Popular literature describing
formal proofs\index{formal proof} include \cite[p.~296]{Rucker}\index{Rucker,
Rudy} and \cite[pp.~204--230]{Hofstadter}\index{Hofstadter, Douglas R.}.

\subsection{Converting It to Metamath}\label{convert}

Formal proofs\index{formal proof} such as the one in our example break down
logical reasoning into small, precise steps that leave little doubt that the
results follow from the axioms\index{axiom}.  You might think that this would
be the finest breakdown we can achieve in mathematics.  However, there is more
to the proof than meets the eye. Although our axioms were rather simple, a lot
of verbiage was needed before we could even state them:  we needed to define
``term,'' ``wff,'' and so on.  In addition, there are a number of implied
rules that we haven't even mentioned. For example, how do we know that step 3
of our proof follows from axiom A1? There is some hidden reasoning involved in
determining this.  Axiom A1 has two occurrences of the letter $ t$.  One of
the implied rules states that whatever we substitute for $ t$ must be a legal
term\index{term}.\footnote{Some authors make this implied rule explicit by
stating, ``only expressions of the above form are terms,'' after defining
``term.''}  The expression $ t+0$ is pretty obviously a legal term whenever $
t$ is, but suppose we wanted to substitute a huge term with thousands of
symbols?  Certainly a lot of work would be involved in determining that it
really is a term, but in ordinary formal proofs all of this work would be
considered a single ``step.''

To express our axiom system in the Metamath\index{Metamath} language, we must
describe this auxiliary information in addition to the axioms themselves.
Metamath does not know what a ``term'' or a ``wff''\index{well-formed formula
(wff)} is.  In Metamath, the specification of the ways in which we can combine
symbols to obtain terms and wffs are like little axioms in themselves.  These
auxiliary axioms are expressed in the same notation as the ``real''
axioms\index{axiom}, and Metamath does not distinguish between the two.  The
distinction is made by you, i.e.\ by the way in which you interpret the
notation you have chosen to express these two kinds of axioms.

The Metamath language breaks down mathematical proofs into tiny pieces, much
more so than in ordinary formal proofs\index{formal proof}.  If a single
step\index{proof step} involves the
substitution\index{substitution!variable}\index{variable substitution} of a
complex term for one of its variables, Metamath must see this single step
broken down into many small steps.  This fine-grained breakdown is what gives
Metamath generality and flexibility as it lets it not be limited to any
particular mathematical notation.

Metamath's proof notation is not, in itself, intended to be read by humans but
rather is in a compact format intended for a machine.  The Metamath program
will convert this notation to a form you can understand, using the \texttt{show
proof}\index{\texttt{show proof} command} command.  You can tell the program what
level of detail of the proof you want to look at.  You may want to look at
just the logical inference steps that correspond
to ordinary formal proof steps,
or you may want to see the fine-grained steps that prove that an expression is
a term.

Here, without further ado, is our example converted to the
Metamath\index{Metamath} language:\index{metavariable}\label{demo0}

\begin{verbatim}
$( Declare the constant symbols we will use $)
    $c 0 + = -> ( ) term wff |- $.
$( Declare the metavariables we will use $)
    $v t r s P Q $.
$( Specify properties of the metavariables $)
    tt $f term t $.
    tr $f term r $.
    ts $f term s $.
    wp $f wff P $.
    wq $f wff Q $.
$( Define "term" and "wff" $)
    tze $a term 0 $.
    tpl $a term ( t + r ) $.
    weq $a wff t = r $.
    wim $a wff ( P -> Q ) $.
$( State the axioms $)
    a1 $a |- ( t = r -> ( t = s -> r = s ) ) $.
    a2 $a |- ( t + 0 ) = t $.
$( Define the modus ponens inference rule $)
    ${
       min $e |- P $.
       maj $e |- ( P -> Q ) $.
       mp  $a |- Q $.
    $}
$( Prove a theorem $)
    th1 $p |- t = t $=
  $( Here is its proof: $)
       tt tze tpl tt weq tt tt weq tt a2 tt tze tpl
       tt weq tt tze tpl tt weq tt tt weq wim tt a2
       tt tze tpl tt tt a1 mp mp
     $.
\end{verbatim}\index{metavariable}

A ``database''\index{database} is a set of one or more {\sc ascii} source
files.  Here's a brief description of this Metamath\index{Metamath} database
(which consists of this single source file), so that you can understand in
general terms what is going on.  To understand the source file in detail, you
should read Chapter~\ref{languagespec}.

The database is a sequence of ``tokens,''\index{token} which are normally
separated by spaces or line breaks.  The only tokens that are built into
the Metamath language are those beginning with \texttt{\$}.  These tokens
are called ``keywords.''\index{keyword}  All other tokens are
user-defined, and their names are arbitrary.

As you might have guessed, the Metamath token \texttt{\$(}\index{\texttt{\$(} and
\texttt{\$)} auxiliary keywords} starts a comment and \texttt{\$)} ends a comment.

The Metamath tokens \texttt{\$c}\index{\texttt{\$c} statement},
\texttt{\$v}\index{\texttt{\$v} statement},
\texttt{\$e}\index{\texttt{\$e} statement},
\texttt{\$f}\index{\texttt{\$f} statement},
\texttt{\$a}\index{\texttt{\$a} statement}, and
\texttt{\$p}\index{\texttt{\$p} statement} specify ``statements'' that
end with \texttt{\$.}\,.\index{\texttt{\$.}\ keyword}

The Metamath tokens \texttt{\$c} and \texttt{\$v} each declare\index{constant
declaration}\index{variable declaration} a list of user-defined tokens, called
``math symbols,''\index{math symbol} that the database will reference later
on.  All of the math symbols they define you have seen earlier except the
turnstile symbol \texttt{|-} ($\vdash$)\index{turnstile ({$\,\vdash$})}, which is
commonly used by logicians to mean ``a proof exists for.''  For us
the turnstile is just a
convenient symbol that distinguishes expressions that are axioms\index{axiom}
or theorems\index{theorem} from expressions that are terms or wffs.

The \texttt{\$c} statement declares ``constants''\index{constant} and
the \texttt{\$v} statement declares
``variables''\index{variable}\index{constant declaration}\index{variable
declaration} (or more precisely, metavariables\index{metavariable}).  A
variable may be substituted\index{substitution!variable}\index{variable
substitution} with sequences of math symbols whereas a constant may not
be substituted with anything.

It may seem redundant to require both \texttt{\$c}\index{\texttt{\$c} statement} and
\texttt{\$v}\index{\texttt{\$v} statement} statements (since any math
symbol\index{math symbol} not specified with a \texttt{\$c} statement could be
presumed to be a variable), but this provides for better error checking and
also allows math symbols to be redeclared\index{redeclaration of symbols}
(Section~\ref{scoping}).

The token \texttt{\$f}\index{\texttt{\$f} statement} specifies a
statement called a ``variable-type hypothesis'' (also called a
``floating hypothesis'') and \texttt{\$e}\index{\texttt{\$e} statement}
specifies a ``logical hypothesis'' (also called an ``essential
hypothesis'').\index{hypothesis}\index{variable-type
hypothesis}\index{logical hypothesis}\index{floating
hypothesis}\index{essential hypothesis} The token
\texttt{\$a}\index{\texttt{\$a} statement} specifies an ``axiomatic
assertion,''\index{axiomatic assertion} and
\texttt{\$p}\index{\texttt{\$p} statement} specifies a ``provable
assertion.''\index{provable assertion} To the left of each occurrence of
these four tokens is a ``label''\index{label} that identifies the
hypothesis or assertion for later reference.  For example, the label of
the first axiomatic assertion is \texttt{tze}.  A \texttt{\$f} statement
must contain exactly two math symbols, a constant followed by a
variable.  The \texttt{\$e}, \texttt{\$a}, and \texttt{\$p} statements
each start with a constant followed by, in general, an arbitrary
sequence of math symbols.

Associated with each assertion\index{assertion} is a set of hypotheses
that must be satisfied in order for the assertion to be used in a proof.
These are called the ``mandatory hypotheses''\index{mandatory
hypothesis} of the assertion.  Among those hypotheses whose ``scope''
(described below) includes the assertion, \texttt{\$e} hypotheses are
always mandatory and \texttt{\$f}\index{\texttt{\$f} statement}
hypotheses are mandatory when they share their variable with the
assertion or its \texttt{\$e} hypotheses.  The exact rules for
determining which hypotheses are mandatory are described in detail in
Sections~\ref{frames} and \ref{scoping}.  For example, the mandatory
hypotheses of assertion \texttt{tpl} are \texttt{tt} and \texttt{tr},
whereas assertion \texttt{tze} has no mandatory hypotheses because it
contains no variables and has no \texttt{\$e}\index{\texttt{\$e}
statement} hypothesis.  Metamath's \texttt{show statement}
command\index{\texttt{show statement} command}, described in the next
section, will show you a statement's mandatory hypotheses.

Sometimes we need to make a hypothesis relevant to only certain
assertions.  The set of statements to which a hypothesis is relevant is
called its ``scope.''  The Metamath brackets,
\texttt{\$\char`\{}\index{\texttt{\$\char`\{} and \texttt{\$\char`\}}
keywords} and \texttt{\$\char`\}}, define a ``block''\index{block} that
delimits the scope of any hypothesis contained between them.  The
assertion \texttt{mp} has mandatory hypotheses \texttt{wp}, \texttt{wq},
\texttt{min}, and \texttt{maj}.  The only mandatory hypothesis of
\texttt{th1}, on the other hand, is \texttt{tt}, since \texttt{th1}
occurs outside of the block containing \texttt{min} and \texttt{maj}.

Note that \texttt{\$\char`\{} and \texttt{\$\char`\}} do not affect the
scope of assertions (\texttt{\$a} and \texttt{\$p}).  Assertions are always
available to be referenced by any later proof in the source file.

Each provable assertion (\texttt{\$p}\index{\texttt{\$p} statement}
statement) has two parts.  The first part is the
assertion\index{assertion} itself, which is a sequence of math
symbol\index{math symbol} tokens placed between the \texttt{\$p} token
and a \texttt{\$=}\index{\texttt{\$=} keyword} token.  The second part
is a ``proof,'' which is a list of label tokens placed between the
\texttt{\$=} token and the \texttt{\$.}\index{\texttt{\$.}\ keyword}\
token that ends the statement.\footnote{If you've looked at the
\texttt{set.mm} database, you may have noticed another notation used for
proofs.  The other notation is called ``compressed.''\index{compressed
proof}\index{proof!compressed} It reduces the amount of space needed to
store a proof in the database and is described in
Appendix~\ref{compressed}.  In the example above, we use
``normal''\index{normal proof}\index{proof!normal} notation.} The proof
acts as a series of instructions to the Metamath program, telling it how
to build up the sequence of math symbols contained in the assertion part of
the \texttt{\$p} statement, making use of the hypotheses of the
\texttt{\$p} statement and previous assertions.  The construction takes
place according to precise rules.  If the list of labels in the proof
causes these rules to be violated, or if the final sequence that results
does not match the assertion, the Metamath program will notify you with
an error message.

If you are familiar with reverse Polish notation (RPN), which is sometimes used
on pocket calculators, here in a nutshell is how a proof works.  Each
hypothesis label\index{hypothesis label} in the proof is pushed\index{push}
onto the RPN stack\index{stack}\index{RPN stack} as it is encountered. Each
assertion label\index{assertion label} pops\index{pop} off the stack as many
entries as the referenced assertion has mandatory hypotheses.  Variable
substitutions\index{substitution!variable}\index{variable substitution} are
computed which, when made to the referenced assertion's mandatory hypotheses,
cause these hypotheses to match the stack entries. These same substitutions
are then made to the variables in the referenced assertion itself, which is
then pushed onto the stack.  At the end of the proof, there should be one
stack entry, namely the assertion being proved.  This process is explained in
detail in Section~\ref{proof}.

Metamath's proof notation is not very readable for humans, but it allows the
proof to be stored compactly in a file.  The Metamath\index{Metamath} program
has proof display features that let you see what's going on in a more
readable way, as you will see in the next section.

The rules used in verifying a proof are not based on any built-in syntax of the
symbol sequence in an assertion\index{assertion} nor on any built-in meanings
attached to specific symbol names.  They are based strictly on symbol
matching:  constants\index{constant} must match themselves, and
variables\index{variable} may be replaced with anything that allows a match to
occur.  For example, instead of \texttt{term}, \texttt{0}, and \verb$|-$ we could
have just as well used \texttt{yellow}, \texttt{zero}, and \texttt{provable}, as long
as we did so consistently throughout the database.  Also, we could have used
\texttt{is provable} (two tokens) instead of \verb$|-$ (one token) throughout the
database.  In each of these cases, the proof would be exactly the same.  The
independence of proofs and notation means that you have a lot of flexibility to
change the notation you use without having to change any proofs.

\section{A Trial Run}\label{trialrun}

Now you are ready to try out the Metamath\index{Metamath} program.

On all computer systems, Metamath has a standard ``command line
interface'' (CLI)\index{command line interface (CLI)} that allows you to
interact with it.  You supply commands to the CLI by typing them on the
keyboard and pressing your keyboard's {\em return} key after each line
you enter.  The CLI is designed to be easy to use and has built-in help
features.

The first thing you should do is to use a text editor to create a file
called \texttt{demo0.mm} and type into it the Metamath source shown on
p.~\pageref{demo0}.  Actually, this file is included with your Metamath
software package, so check that first.  If you type it in, make sure
that you save it in the form of ``plain {\sc ascii} text with line
breaks.''  Most word processors will have this feature.

Next you must run the Metamath program.  Depending on your computer
system and how Metamath is installed, this could range from clicking the
mouse on the Metamath icon to typing \texttt{run metamath} to typing
simply \texttt{metamath}.  (Metamath's {\tt help invoke} command describes
alternate ways of invoking the Metamath program.)

When you first enter Metamath\index{Metamath}, it will be at the CLI, waiting
for your input. You will see something like the following on your screen:
\begin{verbatim}
Metamath - Version 0.177 27-Apr-2019
Type HELP for help, EXIT to exit.
MM>
\end{verbatim}
The \texttt{MM>} prompt means that Metamath is waiting for a command.
Command keywords\index{command keyword} are not case sensitive;
we will use lower-case commands in our examples.
The version number and its release date will probably be different on your
system from the one we show above.

The first thing that you need to do is to read in your
database:\index{\texttt{read} command}\footnote{If a directory path is
needed on Unix,\index{Unix file names}\index{file names!Unix} you should
enclose the path/file name in quotes to prevent Metamath from thinking
that the \texttt{/} in the path name is a command qualifier, e.g.,
\texttt{read \char`\"db/set.mm\char`\"}.  Quotes are optional when there
is no ambiguity.}
\begin{verbatim}
MM> read demo0.mm
\end{verbatim}
Remember to press the {\em return} key after entering this command.  If
you omit the file name, Metamath will prompt you for one.   The syntax for
specifying a Macintosh file name path is given in a footnote on
p.~\pageref{includef}.\index{Macintosh file names}\index{file
names!Macintosh}

If there are any syntax errors in the database, Metamath will let you know
when it reads in the file.  The one thing that Metamath does not check when
reading in a database is that all proofs are correct, because this would
slow it down too much.  It is a good idea to periodically verify the proofs in
a database you are making changes to.  To do this, use the following command
(and do it for your \texttt{demo0.mm} file now).  Note that the \texttt{*} is a
``wild card'' meaning all proofs in the file.\index{\texttt{verify proof} command}
\begin{verbatim}
MM> verify proof *
\end{verbatim}
Metamath will report any proofs that are incorrect.

It is often useful to save the information that the Metamath program displays
on the screen. You can save everything that happens on the screen by opening a
log file. You may want to do this before you read in a database so that you
can examine any errors later on.  To open a log file, type
\begin{verbatim}
MM> open log abc.log
\end{verbatim}
This will open a file called \texttt{abc.log}, and everything that appears on the
screen from this point on will be stored in this file.  The name of the log file
is arbitrary. To close the log file, type
\begin{verbatim}
MM> close log
\end{verbatim}

Several commands let you examine what's inside your database.
Section~\ref{exploring} has an overview of some useful ones.  The
\texttt{show labels} command lets you see what statement
labels\index{label} exist.  A \texttt{*} matches any combination of
characters, and \texttt{t*} refers to all labels starting with the
letter \texttt{t}.\index{\texttt{show labels} command} The \texttt{/all}
is a ``command qualifier''\index{command qualifier} that tells Metamath
to include labels of hypotheses.  (To see the syntax explained, type
\texttt{help show labels}.)  Type
\begin{verbatim}
MM> show labels t* /all
\end{verbatim}
Metamath will respond with
\begin{verbatim}
The statement number, label, and type are shown.
3 tt $f       4 tr $f       5 ts $f       8 tze $a
9 tpl $a      19 th1 $p
\end{verbatim}

You can use the \texttt{show statement} command to get information about a
particular statement.\index{\texttt{show statement} command}
For example, you can get information about the statement with label \texttt{mp}
by typing
\begin{verbatim}
MM> show statement mp /full
\end{verbatim}
Metamath will respond with
\begin{verbatim}
Statement 17 is located on line 43 of the file
"demo0.mm".
"Define the modus ponens inference rule"
17 mp $a |- Q $.
Its mandatory hypotheses in RPN order are:
  wp $f wff P $.
  wq $f wff Q $.
  min $e |- P $.
  maj $e |- ( P -> Q ) $.
The statement and its hypotheses require the
      variables:  Q P
The variables it contains are:  Q P
\end{verbatim}
The mandatory hypotheses\index{mandatory hypothesis} and their
order\index{RPN order} are
useful to know when you are trying to understand or debug a proof.

Now you are ready to look at what's really inside our proof.  First, here is
how to look at every step in the proof---not just the ones corresponding to an
ordinary formal proof\index{formal proof}, but also the ones that build up the
formulas that appear in each ordinary formal proof step.\index{\texttt{show
proof} command}
\begin{verbatim}
MM> show proof th1 /lemmon /all
\end{verbatim}

This will display the proof on the screen in the following format:
\begin{verbatim}
 1 tt            $f term t
 2 tze           $a term 0
 3 1,2 tpl       $a term ( t + 0 )
 4 tt            $f term t
 5 3,4 weq       $a wff ( t + 0 ) = t
 6 tt            $f term t
 7 tt            $f term t
 8 6,7 weq       $a wff t = t
 9 tt            $f term t
10 9 a2          $a |- ( t + 0 ) = t
11 tt            $f term t
12 tze           $a term 0
13 11,12 tpl     $a term ( t + 0 )
14 tt            $f term t
15 13,14 weq     $a wff ( t + 0 ) = t
16 tt            $f term t
17 tze           $a term 0
18 16,17 tpl     $a term ( t + 0 )
19 tt            $f term t
20 18,19 weq     $a wff ( t + 0 ) = t
21 tt            $f term t
22 tt            $f term t
23 21,22 weq     $a wff t = t
24 20,23 wim     $a wff ( ( t + 0 ) = t -> t = t )
25 tt            $f term t
26 25 a2         $a |- ( t + 0 ) = t
27 tt            $f term t
28 tze           $a term 0
29 27,28 tpl     $a term ( t + 0 )
30 tt            $f term t
31 tt            $f term t
32 29,30,31 a1   $a |- ( ( t + 0 ) = t -> ( ( t + 0 )
                                     = t -> t = t ) )
33 15,24,26,32 mp  $a |- ( ( t + 0 ) = t -> t = t )
34 5,8,10,33 mp  $a |- t = t
\end{verbatim}

The \texttt{/lemmon} command qualifier specifies what is known as a Lemmon-style
display\index{Lemmon-style proof}\index{proof!Lemmon-style}.  Omitting the
\texttt{/lemmon} qualifier results in a tree-style proof (see
p.~\pageref{treeproof} for an example) that is somewhat less explicit but
easier to follow once you get used to it.\index{tree-style
proof}\index{proof!tree-style}

The first number on each line is the step
number of the proof.  Any numbers that follow are step numbers assigned to the
hypotheses of the statement referenced by that step.  Next is the label of
the statement referenced by the step.  The statement type of the statement
referenced comes next, followed by the math symbol\index{math symbol} string
constructed by the proof up to that step.

The last step, 34, contains the statement that is being proved.

Looking at a small piece of the proof, notice that steps 3 and 4 have
established that
\texttt{( t + 0 )} and \texttt{t} are \texttt{term}\,s, and step 5 makes use of steps 3 and
4 to establish that \texttt{( t + 0 ) = t} is a \texttt{wff}.  Let Metamath
itself tell us in detail what is happening in step 5.  Note that the
``target hypothesis'' refers to where step 5 is eventually used, i.e., in step
34.
\begin{verbatim}
MM> show proof th1 /detailed_step 5
Proof step 5:  wp=weq $a wff ( t + 0 ) = t
This step assigns source "weq" ($a) to target "wp"
($f).  The source assertion requires the hypotheses
"tt" ($f, step 3) and "tr" ($f, step 4).  The parent
assertion of the target hypothesis is "mp" ($a,
step 34).
The source assertion before substitution was:
    weq $a wff t = r
The following substitutions were made to the source
assertion:
    Variable  Substituted with
     t         ( t + 0 )
     r         t
The target hypothesis before substitution was:
    wp $f wff P
The following substitution was made to the target
hypothesis:
    Variable  Substituted with
     P         ( t + 0 ) = t
\end{verbatim}

The full proof just shown is useful to understand what is going on in detail.
However, most of the time you will just be interested in
the ``essential'' or logical steps of a proof, i.e.\ those steps
that correspond to an
ordinary formal proof\index{formal proof}.  If you type
\begin{verbatim}
MM> show proof th1 /lemmon /renumber
\end{verbatim}
you will see\label{demoproof}
\begin{verbatim}
1 a2             $a |- ( t + 0 ) = t
2 a2             $a |- ( t + 0 ) = t
3 a1             $a |- ( ( t + 0 ) = t -> ( ( t + 0 )
                                     = t -> t = t ) )
4 2,3 mp         $a |- ( ( t + 0 ) = t -> t = t )
5 1,4 mp         $a |- t = t
\end{verbatim}
Compare this to the formal proof on p.~\pageref{zeroproof} and
notice the resemblance.
By default Metamath
does not show \texttt{\$f}\index{\texttt{\$f}
statement} hypotheses and everything branching off of them in the proof tree
when the proof is displayed; this makes the proof look more like an ordinary
mathematical proof, which does not normally incorporate the explicit
construction of expressions.
This is called the ``essential'' view
(at one time you had to add the
\texttt{/essential} qualifier in the \texttt{show proof}
command to get this view, but this is now the default).
You can could use the \texttt{/all} qualifier in the \texttt{show
proof} command to also show the explicit construction of expressions.
The \texttt{/renumber} qualifier means to renumber
the steps to correspond only to what is displayed.\index{\texttt{show proof}
command}

To exit Metamath, type\index{\texttt{exit} command}
\begin{verbatim}
MM> exit
\end{verbatim}

\subsection{Some Hints for Using the Command Line Interface}

We will conclude this quick introduction to Metamath\index{Metamath} with some
helpful hints on how to navigate your way through the commands.
\index{command line interface (CLI)}

When you type commands into Metamath's CLI, you only have to type as many
characters of a command keyword\index{command keyword} as are needed to make
it unambiguous.  If you type too few characters, Metamath will tell you what
the choices are.  In the case of the \texttt{read} command, only the \texttt{r} is
needed to specify it unambiguously, so you could have typed\index{\texttt{read}
command}
\begin{verbatim}
MM> r demo0.mm
\end{verbatim}
instead of
\begin{verbatim}
MM> read demo0.mm
\end{verbatim}
In our description, we always show the full command words.  When using the
Metamath CLI commands in a command file (to be read with the \texttt{submit}
command)\index{\texttt{submit} command}, it is good practice to use
the unabbreviated command to ensure your instructions will not become ambiguous
if more commands are added to the Metamath program in the future.

The command keywords\index{command
keyword} are not case sensitive; you may type either \texttt{read} or
\texttt{ReAd}.  File names may or may not be case sensitive, depending on your
computer's operating system.  Metamath label\index{label} and math
symbol\index{math symbol} tokens\index{token} are case-sensitive.

The \texttt{help} command\index{\texttt{help} command} will provide you
with a list of topics you can get help on.  You can then type
\texttt{help} {\em topic} to get help on that topic.

If you are uncertain of a command's spelling, just type as many characters
as you remember of the command.  If you have not typed enough characters to
specify it unambiguously, Metamath will tell you what choices you have.

\begin{verbatim}
MM> show s
         ^
?Ambiguous keyword - please specify SETTINGS,
STATEMENT, or SOURCE.
\end{verbatim}

If you don't know what argument to use as part of a command, type a
\texttt{?}\index{\texttt{]}@\texttt{?}\ in command lines}\ at the
argument position.  Metamath will tell you what it expected there.

\begin{verbatim}
MM> show ?
         ^
?Expected SETTINGS, LABELS, STATEMENT, SOURCE, PROOF,
MEMORY, TRACE_BACK, or USAGE.
\end{verbatim}

Finally, you may type just the first word or words of a command followed
by {\em return}.  Metamath will prompt you for the remaining part of the
command, showing you the choices at each step.  For example, instead of
typing \texttt{show statement th1 /full} you could interact in the
following manner:
\begin{verbatim}
MM> show
SETTINGS, LABELS, STATEMENT, SOURCE, PROOF,
MEMORY, TRACE_BACK, or USAGE <SETTINGS>? st
What is the statement label <th1>?
/ or nothing <nothing>? /
TEX, COMMENT_ONLY, or FULL <TEX>? f
/ or nothing <nothing>?
19 th1 $p |- t = t $= ... $.
\end{verbatim}
After each \texttt{?}\ in this mode, you must give Metamath the
information it requests.  Sometimes Metamath gives you a list of choices
with the default choice indicated by brackets \texttt{< > }. Pressing
{\em return} after the \texttt{?}\ will select the default choice.
Answering anything else will override the default.  Note that the
\texttt{/} in command qualifiers is considered a separate
token\index{token} by the parser, and this is why it is asked for
separately.

\section{Your First Proof}\label{frstprf}

Proofs are developed with the aid of the Proof Assistant\index{Proof
Assistant}.  We will now show you how the proof of theorem \texttt{th1}
was built.  So that you can repeat these steps, we will first have the
Proof Assistant erase the proof in Metamath's source buffer\index{source
buffer}, then reconstruct it.  (The source buffer is the place in memory
where Metamath stores the information in the database when it is
\texttt{read}\index{\texttt{read} command} in.  New or modified proofs
are kept in the source buffer until a \texttt{write source}
command\index{\texttt{write source} command} is issued.)  In practice, you
would place a \texttt{?}\index{\texttt{]}@\texttt{?}\ inside proofs}\
between \texttt{\$=}\index{\texttt{\$=} keyword} and
\texttt{\$.}\index{\texttt{\$.}\ keyword}\ in the database to indicate
to Metamath\index{Metamath} that the proof is unknown, and that would be
your starting point.  Whenever the \texttt{verify proof} command encounters
a proof with a \texttt{?}\ in place of a proof step, the statement is
identified as not proved.

When I first started creating Metamath proofs, I would write down
on a piece of paper the complete
formal proof\index{formal proof} as it would appear
in a \texttt{show proof} command\index{\texttt{show proof} command}; see
the display of \texttt{show proof th1 /lemmon /re\-num\-ber} above as an
example.  After you get used to using the Proof Assistant\index{Proof
Assistant} you may get to a point where you can ``see'' the proof in your mind
and let the Proof Assistant guide you in filling in the details, at least for
simpler proofs, but until you gain that experience you may find it very useful
to write down all the details in advance.
Otherwise you may waste a lot of time as you let it take you down a wrong path.
However, others do not find this approach as helpful.
For example, Thomas Brendan Leahy\index{Leahy, Thomas Brendan}
finds that it is more helpful to him to interactively
work backward from a machine-readable statement.
David A. Wheeler\index{Wheeler, David A.}
writes down a general approach, but develops the proof
interactively by switching between
working forwards (from hypotheses and facts likely to be useful) and
backwards (from the goal) until the forwards and backwards approaches meet.
In the end, use whatever approach works for you.

A proof is developed with the Proof Assistant by working backwards, starting
with the theorem\index{theorem} to be proved, and assigning each unknown step
with a theorem or hypothesis until no more unknown steps remain.  The Proof
Assistant will not let you make an assignment unless it can be ``unified''
with the unknown step.  This means that a
substitution\index{substitution!variable}\index{variable substitution} of
variables exists that will make the assignment match the unknown step.  On the
other hand, in the middle of a proof, when working backwards, often more than
one unification\index{unification} (set of substitutions) is possible, since
there is not enough information available at that point to uniquely establish
it.  In this case you can tell Metamath which unification to choose, or you
can continue to assign unknown steps until enough information is available to
make the unification unique.

We will assume you have entered Metamath and read in the database as described
above.  The following dialog shows how the proof was developed.  For more
details on what some of the commands do, refer to Section~\ref{pfcommands}.
\index{\texttt{prove} command}

\begin{verbatim}
MM> prove th1
Entering the Proof Assistant.  Type HELP for help, EXIT
to exit.  You will be working on the proof of statement th1:
  $p |- t = t
Note:  The proof you are starting with is already complete.
MM-PA>
\end{verbatim}

The \verb/MM-PA>/ prompt means we are inside the Proof
Assistant.\index{Proof Assistant} Most of the regular Metamath commands
(\texttt{show statement}, etc.) are still available if you need them.

\begin{verbatim}
MM-PA> delete all
The entire proof was deleted.
\end{verbatim}

We have deleted the whole proof so we can start from scratch.

\begin{verbatim}
MM-PA> show new_proof/lemmon/all
1 ?              $? |- t = t
\end{verbatim}

The \texttt{show new{\char`\_}proof} command\index{\texttt{show
new{\char`\_}proof} command} is like \texttt{show proof} except that we
don't specify a statement; instead, the proof we're working on is
displayed.

\begin{verbatim}
MM-PA> assign 1 mp
To undo the assignment, DELETE STEP 5 and INITIALIZE, UNIFY
if needed.
3   min=?  $? |- $2
4   maj=?  $? |- ( $2 -> t = t )
\end{verbatim}

The \texttt{assign} command\index{\texttt{assign} command} above means
``assign step 1 with the statement whose label is \texttt{mp}.''  Note
that step renumbering will constantly occur as you assign steps in the
middle of a proof; in general all steps from the step you assign until
the end of the proof will get moved up.  In this case, what used to be
step 1 is now step 5, because the (partial) proof now has five steps:
the four hypotheses of the \texttt{mp} statement and the \texttt{mp}
statement itself.  Let's look at all the steps in our partial proof:

\begin{verbatim}
MM-PA> show new_proof/lemmon/all
1 ?              $? wff $2
2 ?              $? wff t = t
3 ?              $? |- $2
4 ?              $? |- ( $2 -> t = t )
5 1,2,3,4 mp     $a |- t = t
\end{verbatim}

The symbol \texttt{\$2} is a temporary variable\index{temporary
variable} that represents a symbol sequence not yet known.  In the final
proof, all temporary variables will be eliminated.  The general format
for a temporary variable is \texttt{\$} followed by an integer.  Note
that \texttt{\$} is not a legal character in a math symbol (see
Section~\ref{dollardollar}, p.~\pageref{dollardollar}), so there will
never be a naming conflict between real symbols and temporary variables.

Unknown steps 1 and 2 are constructions of the two wffs used by the
modus ponens rule.  As you will see at the end of this section, the
Proof Assistant\index{Proof Assistant} can usually figure these steps
out by itself, and we will not have to worry about them.  Therefore from
here on we will display only the ``essential'' hypotheses, i.e.\ those
steps that correspond to traditional formal proofs\index{formal proof}.

\begin{verbatim}
MM-PA> show new_proof/lemmon
3 ?              $? |- $2
4 ?              $? |- ( $2 -> t = t )
5 3,4 mp         $a |- t = t
\end{verbatim}

Unknown steps 3 and 4 are the ones we must focus on.  They correspond to the
minor and major premises of the modus ponens rule.  We will assign them as
follows.  Notice that because of the step renumbering that takes place
after an assignment, it is advantageous to assign unknown steps in reverse
order, because earlier steps will not get renumbered.

\begin{verbatim}
MM-PA> assign 4 mp
To undo the assignment, DELETE STEP 8 and INITIALIZE, UNIFY
if needed.
3   min=?  $? |- $2
6     min=?  $? |- $4
7     maj=?  $? |- ( $4 -> ( $2 -> t = t ) )
\end{verbatim}

We are now going to describe an obscure feature that you will probably
never use but should be aware of.  The Metamath language allows empty
symbol sequences to be substituted for variables, but in most formal
systems this feature is never used.  One of the few examples where is it
used is the MIU-system\index{MIU-system} described in
Appendix~\ref{MIU}.  But such systems are rare, and by default this
feature is turned off in the Proof Assistant.  (It is always allowed for
{\tt verify proof}.)  Let us turn it on and see what
happens.\index{\texttt{set empty{\char`\_}substitution} command}

\begin{verbatim}
MM-PA> set empty_substitution on
Substitutions with empty symbol sequences is now allowed.
\end{verbatim}

With this feature enabled, more unifications will be
ambiguous\index{ambiguous unification}\index{unification!ambiguous} in
the middle of a proof, because
substitution\index{substitution!variable}\index{variable substitution}
of variables with empty symbol sequences will become an additional
possibility.  Let's see what happens when we make our next assignment.

\begin{verbatim}
MM-PA> assign 3 a2
There are 2 possible unifications.  Please select the correct
    one or Q if you want to UNIFY later.
Unify:  |- $6
 with:  |- ( $9 + 0 ) = $9
Unification #1 of 2 (weight = 7):
  Replace "$6" with "( + 0 ) ="
  Replace "$9" with ""
  Accept (A), reject (R), or quit (Q) <A>? r
\end{verbatim}

The first choice presented is the wrong one.  If we had selected it,
temporary variable \texttt{\$6} would have been assigned a truncated
wff, and temporary variable \texttt{\$9} would have been assigned an
empty sequence (which is not allowed in our system).  With this choice,
eventually we would reach a point where we would get stuck because
we would end up with steps impossible to prove.  (You may want to
try it.)  We typed \texttt{r} to reject the choice.

\begin{verbatim}
Unification #2 of 2 (weight = 21):
  Replace "$6" with "( $9 + 0 ) = $9"
  Accept (A), reject (R), or quit (Q) <A>? q
To undo the assignment, DELETE STEP 4 and INITIALIZE, UNIFY
if needed.
 7     min=?  $? |- $8
 8     maj=?  $? |- ( $8 -> ( $6 -> t = t ) )
\end{verbatim}

The second choice is correct, and normally we would type \texttt{a}
to accept it.  But instead we typed \texttt{q} to show what will happen:
it will leave the step with an unknown unification, which can be
seen as follows:

\begin{verbatim}
MM-PA> show new_proof/not_unified
 4   min    $a |- $6
        =a2  = |- ( $9 + 0 ) = $9
\end{verbatim}

Later we can unify this with the \texttt{unify}
\texttt{all/interactive} command.

The important point to remember is that occasionally you will be
presented with several unification choices while entering a proof, when
the program determines that there is not enough information yet to make
an unambiguous choice automatically (and this can happen even with
\texttt{set empty{\char`\_}substitution} turned off).  Usually it is
obvious by inspection which choice is correct, since incorrect ones will
tend to be meaningless fragments of wffs.  In addition, the correct
choice will usually be the first one presented, unlike our example
above.

Enough of this digression.  Let us go back to the default setting.

\begin{verbatim}
MM-PA> set empty_substitution off
The ability to substitute empty expressions for variables
has been turned off.  Note that this may make the Proof
Assistant too restrictive in some cases.
\end{verbatim}

If we delete the proof, start over, and get to the point where
we digressed above, there will no longer be an ambiguous unification.

\begin{verbatim}
MM-PA> assign 3 a2
To undo the assignment, DELETE STEP 4 and INITIALIZE, UNIFY
if needed.
 7     min=?  $? |- $4
 8     maj=?  $? |- ( $4 -> ( ( $5 + 0 ) = $5 -> t = t ) )
\end{verbatim}

Let us look at our proof so far, and continue.

\begin{verbatim}
MM-PA> show new_proof/lemmon
 4 a2            $a |- ( $5 + 0 ) = $5
 7 ?             $? |- $4
 8 ?             $? |- ( $4 -> ( ( $5 + 0 ) = $5 -> t = t ) )
 9 7,8 mp        $a |- ( ( $5 + 0 ) = $5 -> t = t )
10 4,9 mp        $a |- t = t
MM-PA> assign 8 a1
To undo the assignment, DELETE STEP 11 and INITIALIZE, UNIFY
if needed.
 7     min=?  $? |- ( t + 0 ) = t
MM-PA> assign 7 a2
To undo the assignment, DELETE STEP 8 and INITIALIZE, UNIFY
if needed.
MM-PA> show new_proof/lemmon
 4 a2            $a |- ( t + 0 ) = t
 8 a2            $a |- ( t + 0 ) = t
12 a1            $a |- ( ( t + 0 ) = t -> ( ( t + 0 ) = t ->
                                                    t = t ) )
13 8,12 mp       $a |- ( ( t + 0 ) = t -> t = t )
14 4,13 mp       $a |- t = t
\end{verbatim}

Now all temporary variables and unknown steps have been eliminated from the
``essential'' part of the proof.  When this is achieved, the Proof
Assistant\index{Proof Assistant} can usually figure out the rest of the proof
automatically.  (Note that the \texttt{improve} command can occasionally be
useful for filling in essential steps as well, but it only tries to make use
of statements that introduce no new variables in their hypotheses, which is
not the case for \texttt{mp}. Also it will not try to improve steps containing
temporary variables.)  Let's look at the complete proof, then run
the \texttt{improve} command, then look at it again.

\begin{verbatim}
MM-PA> show new_proof/lemmon/all
 1 ?             $? wff ( t + 0 ) = t
 2 ?             $? wff t = t
 3 ?             $? term t
 4 3 a2          $a |- ( t + 0 ) = t
 5 ?             $? wff ( t + 0 ) = t
 6 ?             $? wff ( ( t + 0 ) = t -> t = t )
 7 ?             $? term t
 8 7 a2          $a |- ( t + 0 ) = t
 9 ?             $? term ( t + 0 )
10 ?             $? term t
11 ?             $? term t
12 9,10,11 a1    $a |- ( ( t + 0 ) = t -> ( ( t + 0 ) = t ->
                                                    t = t ) )
13 5,6,8,12 mp   $a |- ( ( t + 0 ) = t -> t = t )
14 1,2,4,13 mp   $a |- t = t
\end{verbatim}

\begin{verbatim}
MM-PA> improve all
A proof of length 1 was found for step 11.
A proof of length 1 was found for step 10.
A proof of length 3 was found for step 9.
A proof of length 1 was found for step 7.
A proof of length 9 was found for step 6.
A proof of length 5 was found for step 5.
A proof of length 1 was found for step 3.
A proof of length 3 was found for step 2.
A proof of length 5 was found for step 1.
Steps 1 and above have been renumbered.
CONGRATULATIONS!  The proof is complete.  Use SAVE
NEW_PROOF to save it.  Note:  The Proof Assistant does
not detect $d violations.  After saving the proof, you
should verify it with VERIFY PROOF.
\end{verbatim}

The \texttt{save new{\char`\_}proof} command\index{\texttt{save
new{\char`\_}proof} command} will save the proof in the database.  Here
we will just display it in a form that can be clipped out of a log file
and inserted manually into the database source file with a text
editor.\index{normal proof}\index{proof!normal}

\begin{verbatim}
MM-PA> show new_proof/normal
---------Clip out the proof below this line:
      tt tze tpl tt weq tt tt weq tt a2 tt tze tpl tt weq
      tt tze tpl tt weq tt tt weq wim tt a2 tt tze tpl tt
      tt a1 mp mp $.
---------The proof of 'th1' to clip out ends above this line.
\end{verbatim}

There is another proof format called ``compressed''\index{compressed
proof}\index{proof!compressed} that you will see in databases.  It is
not important to understand how it is encoded but only to recognize it
when you see it.  Its only purpose is to reduce storage requirements for
large proofs.  A compressed proof can always be converted to a normal
one and vice-versa, and the Metamath \texttt{show proof}
commands\index{\texttt{show proof} command} work equally well with
compressed proofs.  The compressed proof format is described in
Appendix~\ref{compressed}.

\begin{verbatim}
MM-PA> show new_proof/compressed
---------Clip out the proof below this line:
      ( tze tpl weq a2 wim a1 mp ) ABCZADZAADZAEZJJKFLIA
      AGHH $.
---------The proof of 'th1' to clip out ends above this line.
\end{verbatim}

Now we will exit the Proof Assistant.  Since we made changes to the proof,
it will warn us that we have not saved it.  In this case, we don't care.

\begin{verbatim}
MM-PA> exit
Warning:  You have not saved changes to the proof.
Do you want to EXIT anyway (Y, N) <N>? y
Exiting the Proof Assistant.
Type EXIT again to exit Metamath.
\end{verbatim}

The Proof Assistant\index{Proof Assistant} has several other commands
that can help you while creating proofs.  See Section~\ref{pfcommands}
for a list of them.

A command that is often useful is \texttt{minimize{\char`\_}with
*/brief}, which tries to shorten the proof.  It can make the process
more efficient by letting you write a somewhat ``sloppy'' proof then
clean up some of the fine details of optimization for you (although it
can't perform miracles such as restructuring the overall proof).

\section{A Note About Editing a Data\-base File}

Once your source file contains proofs, there are some restrictions on
how you can edit it so that the proofs remain valid.  Pay particular
attention to these rules, since otherwise you can lose a lot of work.
It is a good idea to periodically verify all proofs with \texttt{verify
proof *} to ensure their integrity.

If your file contains only normal (as opposed to compressed) proofs, the
main rule is that you may not change the order of the mandatory
hypotheses\index{mandatory hypothesis} of any statement referenced in a
later proof.  For example, if you swap the order of the major and minor
premise in the modus ponens rule, all proofs making use of that rule
will become incorrect.  The \texttt{show statement}
command\index{\texttt{show statement} command} will show you the
mandatory hypotheses of a statement and their order.

If a statement has a compressed proof, you also must not change the
order of {\em its} mandatory hypotheses.  The compressed proof format
makes use of this information as part of the compression technique.
Note that swapping the names of two variables in a theorem will change
the order of its mandatory hypotheses.

The safest way to edit a statement, say \texttt{mytheorem}, is to
duplicate it then rename the original to \texttt{mytheoremOLD}
throughout the database.  Once the edited version is re-proved, all
statements referencing \texttt{mytheoremOLD} can be updated in the Proof
Assistant using \texttt{minimize{\char`\_}with
mytheorem
/allow{\char`\_}growth}.\index{\texttt{minimize{\char`\_}with} command}
% 3/10/07 Note: line-breaking the above results in duplicate index entries

\chapter{Abstract Mathematics Revealed}\label{fol}

\section{Logic and Set Theory}\label{logicandsettheory}

\begin{quote}
  {\em Set theory can be viewed as a form of exact theology.}
  \flushright\sc  Rudy Rucker\footnote{\cite{Barrow}, p.~31.}\\
\end{quote}\index{Rucker, Rudy}

Despite its seeming complexity, all of standard mathematics, no matter how
deep or abstract, can amazingly enough be derived from a relatively small set
of axioms\index{axiom} or first principles. The development of these axioms is
among the most impressive and important accomplishments of mathematics in the
20th century. Ultimately, these axioms can be broken down into a set of rules
for manipulating symbols that any technically oriented person can follow.

We will not spend much time trying to convey a deep, higher-level
understanding of the meaning of the axioms. This kind of understanding
requires some mathematical sophistication as well as an understanding of the
philosophy underlying the foundations of mathematics and typically develops
over time as you work with mathematics.  Our goal, instead, is to give you the
immediate ability to follow how theorems\index{theorem} are derived from the
axioms and from other theorems.  This will be similar to learning the syntax
of a computer language, which lets you follow the details in a program but
does not necessarily give you the ability to write non-trivial programs on
your own, an ability that comes with practice. For now don't be alarmed by
abstract-sounding names of the axioms; just focus on the rules for
manipulating the symbols, which follow the simple conventions of the
Metamath\index{Metamath} language.

The axioms that underlie all of standard mathematics consist of axioms of logic
and axioms of set theory. The axioms of logic are divided into two
subcategories, propositional calculus\index{propositional calculus} (sometimes
called sentential logic\index{sentential logic}) and predicate calculus
(sometimes called first-order logic\index{first-order logic}\index{quantifier
theory}\index{predicate calculus} or quantifier theory).  Propositional
calculus is a prerequisite for predicate calculus, and predicate calculus is a
prerequisite for set theory.  The version of set theory most commonly used is
Zermelo--Fraenkel set theory\index{Zermelo--Fraenkel set theory}\index{set theory}
with the axiom of choice,
often abbreviated as ZFC\index{ZFC}.

Here in a nutshell is what the axioms are all about in an informal way. The
connection between this description and symbols we will show you won't be
immediately apparent and in principle needn't ever be.  Our description just
tries to summarize what mathematicians think about when they work with the
axioms.

Logic is a set of rules that allow us determine truths given other truths.
Put another way,
logic is more or less the translation of what we would consider common sense
into a rigorous set of axioms.\index{axioms of logic}  Suppose $\varphi$,
$\psi$, and $\chi$ (the Greek letters phi, psi, and chi) represent statements
that are either true or false, and $x$ is a variable\index{variable!in predicate
calculus} ranging over some group of mathematical objects (sets, integers,
real numbers, etc.). In mathematics, a ``statement'' really means a formula,
and $\psi$ could be for example ``$x = 2$.''
Propositional calculus\index{propositional calculus}
allows us to use variables that are either true or false
and make deductions such as
``if $\varphi$ implies $\psi$ and $\psi$ implies $\chi$, then $\varphi$
implies $\chi$.''
Predicate calculus\index{predicate calculus}
extends propositional calculus by also allowing us
to discuss statements about objects (not just true and false values), including
statements about ``all'' or ``at least one'' object.
For example, predicate calculus allows to say,
``if $\varphi$ is true for all $x$, then $\varphi$ is true for some $x$.''
The logic used in \texttt{set.mm} is standard classical logic
(as opposed to other logic systems like intuitionistic logic).

Set theory\index{set theory} has to do with the manipulation of objects and
collections of objects, specifically the abstract, imaginary objects that
mathematics deals with, such as numbers. Everything that is claimed to exist
in mathematics is considered to be a set.  A set called the empty
set\index{empty set} contains nothing.  We represent the empty set by
$\varnothing$.  Many sets can be built up from the empty set.  There is a set
represented by $\{\varnothing\}$ that contains the empty set, another set
represented by $\{\varnothing,\{\varnothing\}\}$ that contains this set as
well as the empty set, another set represented by $\{\{\varnothing\}\}$ that
contains just the set that contains the empty set, and so on ad infinitum. All
mathematical objects, no matter how complex, are defined as being identical to
certain sets: the integer\index{integer} 0 is defined as the empty set, the
integer 1 is defined as $\{\varnothing\}$, the integer 2 is defined as
$\{\varnothing,\{\varnothing\}\}$.  (How these definitions were chosen doesn't
matter now, but the idea behind it is that these sets have the properties we
expect of integers once suitable operations are defined.)  Mathematical
operations, such as addition, are defined in terms of operations on
sets---their union\index{set union}, intersection\index{set intersection}, and
so on---operations you may have used in elementary school when you worked
with groups of apples and oranges.

With a leap of faith, the axioms also postulate the existence of infinite
sets\index{infinite set}, such as the set of all non-negative integers ($0, 1,
2,\ldots$, also called ``natural numbers''\index{natural number}).  This set
can't be represented with the brace notation\index{brace notation} we just
showed you, but requires a more complicated notation called ``class
abstraction.''\index{class abstraction}\index{abstraction class}  For
example, the infinite set $\{ x |
\mbox{``$x$ is a natural number''} \} $ means the ``set of all objects $x$
such that $x$ is a natural number'' i.e.\ the set of natural numbers; here,
``$x$ is a natural number'' is a rather complicated formula when broken down
into the primitive symbols.\label{expandom}\footnote{The statement ``$x$ is a
natural number'' is formally expressed as ``$x \in \omega$,'' where $\in$
(stylized epsilon) means ``is in'' or ``is an element of'' and $\omega$
(omega) means ``the set of natural numbers.''  When ``$x\in\omega$'' is
completely expanded in terms of the primitive symbols of set theory, the
result is  $\lnot$ $($ $\lnot$ $($ $\forall$ $z$ $($ $\lnot$ $\forall$ $w$ $($
$z$ $\in$ $w$ $\rightarrow$ $\lnot$ $w$ $\in$ $x$ $)$ $\rightarrow$ $z$ $\in$
$x$ $)$ $\rightarrow$ $($ $\forall$ $z$ $($ $\lnot$ $($ $\forall$ $w$ $($ $w$
$\in$ $z$ $\rightarrow$ $w$ $\in$ $x$ $)$ $\rightarrow$ $\forall$ $w$ $\lnot$
$w$ $\in$ $z$ $)$ $\rightarrow$ $\lnot$ $\forall$ $w$ $($ $w$ $\in$ $z$
$\rightarrow$ $\lnot$ $\forall$ $v$ $($ $v$ $\in$ $z$ $\rightarrow$ $\lnot$
$v$ $\in$ $w$ $)$ $)$ $)$ $\rightarrow$ $\lnot$ $\forall$ $z$ $\forall$ $w$
$($ $\lnot$ $($ $z$ $\in$ $x$ $\rightarrow$ $\lnot$ $w$ $\in$ $x$ $)$
$\rightarrow$ $($ $\lnot$ $z$ $\in$ $w$ $\rightarrow$ $($ $\lnot$ $z$ $=$ $w$
$\rightarrow$ $w$ $\in$ $z$ $)$ $)$ $)$ $)$ $)$ $\rightarrow$ $\lnot$
$\forall$ $y$ $($ $\lnot$ $($ $\lnot$ $($ $\forall$ $z$ $($ $\lnot$ $\forall$
$w$ $($ $z$ $\in$ $w$ $\rightarrow$ $\lnot$ $w$ $\in$ $y$ $)$ $\rightarrow$
$z$ $\in$ $y$ $)$ $\rightarrow$ $($ $\forall$ $z$ $($ $\lnot$ $($ $\forall$
$w$ $($ $w$ $\in$ $z$ $\rightarrow$ $w$ $\in$ $y$ $)$ $\rightarrow$ $\forall$
$w$ $\lnot$ $w$ $\in$ $z$ $)$ $\rightarrow$ $\lnot$ $\forall$ $w$ $($ $w$
$\in$ $z$ $\rightarrow$ $\lnot$ $\forall$ $v$ $($ $v$ $\in$ $z$ $\rightarrow$
$\lnot$ $v$ $\in$ $w$ $)$ $)$ $)$ $\rightarrow$ $\lnot$ $\forall$ $z$
$\forall$ $w$ $($ $\lnot$ $($ $z$ $\in$ $y$ $\rightarrow$ $\lnot$ $w$ $\in$
$y$ $)$ $\rightarrow$ $($ $\lnot$ $z$ $\in$ $w$ $\rightarrow$ $($ $\lnot$ $z$
$=$ $w$ $\rightarrow$ $w$ $\in$ $z$ $)$ $)$ $)$ $)$ $\rightarrow$ $($
$\forall$ $z$ $\lnot$ $z$ $\in$ $y$ $\rightarrow$ $\lnot$ $\forall$ $w$ $($
$\lnot$ $($ $w$ $\in$ $y$ $\rightarrow$ $\lnot$ $\forall$ $z$ $($ $w$ $\in$
$z$ $\rightarrow$ $\lnot$ $z$ $\in$ $y$ $)$ $)$ $\rightarrow$ $\lnot$ $($
$\lnot$ $\forall$ $z$ $($ $w$ $\in$ $z$ $\rightarrow$ $\lnot$ $z$ $\in$ $y$
$)$ $\rightarrow$ $w$ $\in$ $y$ $)$ $)$ $)$ $)$ $\rightarrow$ $x$ $\in$ $y$
$)$ $)$ $)$. Section~\ref{hierarchy} shows the hierarchy of definitions that
leads up to this expression.}\index{stylized epsilon ($\in$)}\index{omega
($\omega$)}  Actually, the primitive symbols don't even include the brace
notation.  The brace notation is a high-level definition, which you can find in
Section~\ref{hierarchy}.

Interestingly, the arithmetic of integers\index{integer} and
rationals\index{rational number} can be developed without appealing to the
existence of an infinite set, whereas the arithmetic of real
numbers\index{real number} requires it.

Each variable\index{variable!in set theory} in the axioms of set theory
represents an arbitrary set, and the axioms specify the legal kinds of things
you can do with these variables at a very primitive level.

Now, you may think that numbers and arithmetic are a lot more intuitive and
fundamental than sets and therefore should be the foundation of mathematics.
What is really the case is that you've dealt with numbers all your life and
are comfortable with a few rules for manipulating them such as addition and
multiplication.  Those rules only cover a small portion of what can be done
with numbers and only a very tiny fraction of the rest of mathematics.  If you
look at any elementary book on number theory, you will quickly become lost if
these are the only rules that you know.  Even though such books may present a
list of ``axioms''\index{axiom} for arithmetic, the ability to use the axioms
and to understand proofs of theorems\index{theorem} (facts) about numbers
requires an implicit mathematical talent that frustrates many people
from studying abstract mathematics.  The kind of mathematics that most people
know limits them to the practical, everyday usage of blindly manipulating
numbers and formulas, without any understanding of why those rules are correct
nor any ability to go any further.  For example, do you know why multiplying
two negative numbers yields a positive number?  Starting with set theory, you
will also start off blindly manipulating symbols according to the rules we give
you, but with the advantage that these rules will allow you, in principle, to
access {\em all} of mathematics, not just a tiny part of it.

Of course, concrete examples are often helpful in the learning process. For
example, you can verify that $2\cdot 3=3 \cdot 2$ by actually grouping
objects and can easily ``see'' how it generalizes to $x\cdot y = y\cdot x$,
even though you might not be able to rigorously prove it.  Similarly, in set
theory it can be helpful to understand how the axioms of set theory apply to
(and are correct for) small finite collections of objects.  You should be aware
that in set theory intuition can be misleading for infinite collections, and
rigorous proofs become more important.  For example, while $x\cdot y = y\cdot
x$ is correct for finite ordinals (which are the natural numbers), it is not
usually true for infinite ordinals.

\section{The Axioms for All of Mathematics}

In this section\index{axioms for mathematics}, we will show you the axioms
for all of standard mathematics (i.e.\ logic and set theory) as they are
traditionally presented.  The traditional presentation is useful for someone
with the mathematical experience needed to correctly manipulate high-level
abstract concepts.  For someone without this talent, knowing how to actually
make use of these axioms can be difficult.  The purpose of this section is to
allow you to see how the version of the axioms used in the standard
Metamath\index{Metamath} database \texttt{set.mm}\index{set
theory database (\texttt{set.mm})} relates to  the typical version
in textbooks, and also to give you an informal feel for them.

\subsection{Propositional Calculus}

Propositional calculus\index{propositional calculus} concerns itself with
statements that can be interpreted as either true or false.  Some examples of
statements (outside of mathematics) that are either true or false are ``It is
raining today'' and ``The United States has a female president.'' In
mathematics, as we mentioned, statements are really formulas.

In propositional calculus, we don't care what the statements are.  We also
treat a logical combination of statements, such as ``It is raining today and
the United States has a female president,'' no differently from a single
statement.  Statements and their combinations are called well-formed formulas
(wffs)\index{well-formed formula (wff)}.  We define wffs only in terms of
other wffs and don't define what a ``starting'' wff is.  As is common practice
in the literature, we use Greek letters to represent wffs.

Specifically, suppose $\varphi$ and $\psi$ are wffs.  Then the combinations
$\varphi\rightarrow\psi$ (``$\varphi$ implies $\psi$,'' also read ``if
$\varphi$ then $\psi$'')\index{implication ($\rightarrow$)} and $\lnot\varphi$
(``not $\varphi$'')\index{negation ($\lnot$)} are also wffs.

The three axioms of propositional calculus\index{axioms of propositional
calculus} are all wffs of the following form:\footnote{A remarkable result of
C.~A.~Meredith\index{Meredith, C. A.} squeezes these three axioms into the
single axiom $((((\varphi\rightarrow \psi)\rightarrow(\neg \chi\rightarrow\neg
\theta))\rightarrow \chi )\rightarrow \tau)\rightarrow((\tau\rightarrow
\varphi)\rightarrow(\theta\rightarrow \varphi))$ \cite{CAMeredith},
which is believed to be the shortest possible.}
\begin{center}
     $\varphi\rightarrow(\psi\rightarrow \varphi)$\\

     $(\varphi\rightarrow (\psi\rightarrow \chi))\rightarrow
((\varphi\rightarrow  \psi)\rightarrow (\varphi\rightarrow \chi))$\\

     $(\neg \varphi\rightarrow \neg\psi)\rightarrow (\psi\rightarrow
\varphi)$
\end{center}

These three axioms are widely used.
They are attributed to Jan {\L}ukasiewicz
(pronounced woo-kah-SHAY-vitch) and was popularized by Alonzo Church,
who called it system P2. (Thanks to Ted Ulrich for this information.)

There are an infinite number of axioms, one for each possible
wff\index{well-formed formula (wff)} of the above form.  (For this reason,
axioms such as the above are often called ``axiom schemes.''\index{axiom
scheme})  Each Greek letter in the axioms may be substituted with a more
complex wff to result in another axiom.  For example, substituting
$\neg(\varphi\rightarrow\chi)$ for $\varphi$ in the first axiom yields
$\neg(\varphi\rightarrow\chi)\rightarrow(\psi\rightarrow
\neg(\varphi\rightarrow\chi))$, which is still an axiom.

To deduce new true statements (theorems\index{theorem}) from the axioms, a
rule\index{rule} called ``modus ponens''\index{modus ponens} is used.  This
rule states that if the wff $\varphi$ is an axiom or a theorem, and the wff
$\varphi\rightarrow\psi$ is an axiom or a theorem, then the wff $\psi$ is also
a theorem\index{theorem}.

As a non-mathematical example of modus ponens, suppose we have proved (or
taken as an axiom) ``Bob is a man'' and separately have proved (or taken as
an axiom) ``If Bob is a man, then Bob is a human.''  Using the rule of modus
ponens, we can logically deduce, ``Bob is a human.''

From Metamath's\index{Metamath} point of view, the axioms and the rule of
modus ponens just define a mechanical means for deducing new true statements
from existing true statements, and that is the complete content of
propositional calculus as far as Metamath is concerned.  You can read a logic
textbook to gain a better understanding of their meaning, or you can just let
their meaning slowly become apparent to you after you use them for a while.

It is actually rather easy to check to see if a formula is a theorem of
propositional calculus.  Theorems of propositional calculus are also called
``tautologies.''\index{tautology}  The technique to check whether a formula is
a tautology is called the ``truth table method,''\index{truth table} and it
works like this.  A wff $\varphi\rightarrow\psi$ is false whenever $\varphi$ is true
and $\psi$ is false.  Otherwise it is true.  A wff $\lnot\varphi$ is false
whenever $\varphi$ is true and false otherwise. To verify a tautology such as
$\varphi\rightarrow(\psi\rightarrow \varphi)$, you break it down into sub-wffs and
construct a truth table that accounts for all possible combinations of true
and false assigned to the wff metavariables:
\begin{center}\begin{tabular}{|c|c|c|c|}\hline
\mbox{$\varphi$} & \mbox{$\psi$} & \mbox{$\psi\rightarrow\varphi$}
    & \mbox{$\varphi\rightarrow(\psi\rightarrow \varphi)$} \\ \hline \hline
              T   &  T    &      T       &        T    \\ \hline
              T   &  F    &      T       &        T    \\ \hline
              F   &  T    &      F       &        T    \\ \hline
              F   &  F    &      T       &        T    \\ \hline
\end{tabular}\end{center}
If all entries in the last column are true, the formula is a tautology.

Now, the truth table method doesn't tell you how to prove the tautology from
the axioms, but only that a proof exists.  Finding an actual proof (especially
one that is short and elegant) can be challenging.  Methods do exist for
automatically generating proofs in propositional calculus, but the proofs that
result can sometimes be very long.  In the Metamath \texttt{set.mm}\index{set
theory database (\texttt{set.mm})} database, most
or all proofs were created manually.

Section \ref{metadefprop} discusses various definitions
that make propositional calculus easier to use.
For example, we define:

\begin{itemize}
\item $\varphi \vee \psi$
  is true if either $\varphi$ or $\psi$ (or both) are true
  (this is disjunction\index{disjunction ($\vee$)}
  aka logical {\sc or}\index{logical {\sc or} ($\vee$)}).

\item $\varphi \wedge \psi$
  is true if both $\varphi$ and $\psi$ are true
  (this is conjunction\index{conjunction ($\wedge$)}
  aka logical {\sc and}\index{logical {\sc and} ($\wedge$)}).

\item $\varphi \leftrightarrow \psi$
  is true if $\varphi$ and $\psi$ have the same value, that is,
  they are both true or both false
  (this is the biconditional\index{biconditional ($\leftrightarrow$)}).
\end{itemize}

\subsection{Predicate Calculus}

Predicate calculus\index{predicate calculus} introduces the concept of
``individual variables,''\index{variable!in predicate calculus}\index{individual
variable} which
we will usually just call ``variables.''
These variables can represent something other than true or false (wffs),
and will always represent sets when we get to set theory.  There are also
three new symbols $\forall$\index{universal quantifier ($\forall$)},
$=$\index{equality ($=$)}, and $\in$\index{stylized epsilon ($\in$)},
read ``for all,'' ``equals,'' and ``is an element of''
respectively.  We will represent variables with the letters $x$, $y$, $z$, and
$w$, as is common practice in the literature.
For example, $\forall x \varphi$ means ``for all possible values of
$x$, $\varphi$ is true.''

In predicate calculus, we extend the definition of a wff\index{well-formed
formula (wff)}.  If $\varphi$ is a wff and $x$ and $y$ are variables, then
$\forall x \, \varphi$, $x=y$, and $x\in y$ are wffs. Note that these three new
types of wffs can be considered ``starting'' wffs from which we can build
other wffs with $\rightarrow$ and $\neg$ .  The concept of a starting wff was
absent in propositional calculus.  But starting wff or not, all we are really
concerned with is whether our wffs are correctly constructed according to
these mechanical rules.

A quick aside:
To prevent confusion, it might be best at this point to think of the variables
of Metamath\index{Metamath} as ``metavariables,''\index{metavariable} because
they are not quite the same as the variables we are introducing here.  A
(meta)variable in Metamath can be a wff or an individual variable, as well
as many other things; in general, it represents a kind of place holder for an
unspecified sequence of math symbols\index{math symbol}.

Unlike propositional calculus, no decision procedure\index{decision procedure}
analogous to the truth table method exists (nor theoretically can exist) that
will definitely determine whether a formula is a theorem of predicate
calculus.  Much of the work in the field of automated theorem
proving\index{automated theorem proving} has been dedicated to coming up with
clever heuristics for proving theorems of predicate calculus, but they can
never be guaranteed to work always.

Section \ref{metadefpred} discusses various definitions
that make predicate calculus easier to use.
For example, we define
$\exists x \varphi$ to mean
``there exists at least one possible value of $x$ where $\varphi$ is true.''

We now turn to looking at how predicate calculus can be formally
represented.

\subsubsection{Common Axioms}

There is a new rule of inference in predicate calculus:  if $\varphi$ is
an axiom or a theorem, then $\forall x \,\varphi$ is also a
theorem\index{theorem}.  This is called the rule of
``generalization.''\index{rule of generalization}
This is easily represented in Metamath.

In standard texts of logic, there are often two axioms of predicate
calculus\index{axioms of predicate calculus}:
\begin{center}
  $\forall x \,\varphi ( x ) \rightarrow \varphi ( y )$,
      where ``$y$ is properly substituted for $x$.''\\
  $\forall x ( \varphi \rightarrow \psi )\rightarrow ( \varphi \rightarrow
    \forall x\, \psi )$,
    where ``$x$ is not free in $\varphi$.''
\end{center}

Now at first glance, this seems simple:  just two axioms.  However,
conditional clauses are attached to each axiom describing requirements that
may seem puzzling to you.  In addition, the first axiom puts a variable symbol
in parentheses after each wff, seemingly violating our definition of a
wff\index{well-formed formula (wff)}; this is just an informal way of
referring to some arbitrary variable that may occur in the wff.  The
conditional clauses do, of course, have a precise meaning, but as it turns out
the precise meaning is somewhat complicated and awkward to formalize in a
way that a computer can handle easily.  Unlike propositional calculus, a
certain amount of mathematical sophistication and practice is needed to be
able to easily grasp and manipulate these concepts correctly.

Predicate calculus may be presented with or without axioms for
equality\index{axioms of equality}\index{equality ($=$)}. We will require the
axioms of equality as a prerequisite for the version of set theory we will
use.  The axioms for equality, when included, are often represented using these
two axioms:
\begin{center}
$x=x$\\ \ \\
$x=y\rightarrow (\varphi(x,x)\rightarrow\varphi(x,y))$ where ``$\varphi(x,y)$
   arises from $\varphi(x,x)$ by replacing some, but not necessarily all,
   free\index{free variable}
   occurrences of $x$ by $y$,\\ provided that $y$ is free for $x$
   in $\varphi(x,x)$.'' \end{center}
% (Mendelson p. 95)
The first equality axiom is simple, but again,
the condition on the second one is
somewhat awkward to implement on a computer.

\subsubsection{Tarski System S2}

Of course, we are not the first to notice the complications of these
predicate calculus axioms when being rigorous.

Well-known logician Alfred Tarski published in 1965
a system he called system S2\cite[p.~77]{Tarski1965}.
Tarski's system is \textit{exactly equivalent} to the traditional textbook
formalization, but (by clever use of equality axioms) it eliminates the
latter's primitive notions of ``proper substitution'' and ``free variable,''
replacing them with direct substitution and the notion of a variable
not occurring in a formula (which we express with distinct variable
constraints).

In advocating his system, Tarski wrote, ``The relatively complicated
character of [free variables and proper substitution] is a source
of certain inconveniences of both practical and theoretical nature;
this is clearly experienced both in teaching an elementary course of
mathematical logic and in formalizing the syntax of predicate logic for
some theoretical purposes''\cite[p.~61]{Tarski1965}\index{Tarski, Alfred}.

\subsubsection{Developing a Metamath Representation}

The standard textbook axioms of predicate calculus are somewhat
cumbersome to implement on a computer because of the complex notions of
``free variable''\index{free variable} and ``proper
substitution.''\index{proper substitution}\index{substitution!proper}
While it is possible to use the Metamath\index{Metamath} language to
implement these concepts, we have chosen not to implement them
as primitive constructs in the
\texttt{set.mm} set theory database.  Instead, we have eliminated them
within the axioms
by carefully crafting the axioms so as to avoid them,
building on Tarski's system S2.  This makes it
easy for a beginner to follow the steps in a proof without knowing any
advanced concepts other than the simple concept of
replacing\index{substitution!variable}\index{variable substitution}
variables with expressions.

In order to develop the concepts of free variable and proper
substitution from the axioms, we use an additional
Metamath statement type called ``disjoint variable
restriction''\index{disjoint variables} that we have not encountered
before.  In the context of the axioms, the statement \texttt{\$d} $ x\,
y$\index{\texttt{\$d} statement} simply means that $x$ and $y$ must be
distinct\index{distinct variables}, i.e.\ they may not be simultaneously
substituted\index{substitution!variable}\index{variable substitution}
with the same variable.  The statement \texttt{\$d} $ x\, \varphi$ means
variable $x$ must not occur in wff $\varphi$.  For the precise
definition of \texttt{\$d}, see Section~\ref{dollard}.

\subsubsection{Metamath representation}

The Metamath axiom system for predicate calculus
defined in set.mm uses Tarski's system S2.
As noted above, this has a different representation
than the traditional textbook formalization,
but it is \textit{exactly equivalent} to the textbook formalization,
and it is \textit{much} easier to work with.
This is reproduced as system S3 in Section 6 of
Megill's formalization \cite{Megill}\index{Megill, Norman}.

There is one exception, Tarski's axiom of existence,
which we label as axiom ax-6.
In the case of ax-6, Tarski's version is weaker because it includes a
distinct variable proviso. If we wish, we can also weaken our version
in this way and still have a metalogically complete system. Theorem
ax6 shows this by deriving, in the presence of the other axioms, our
ax-6 from Tarski's weaker version ax6v. However, we chose the stronger
version for our system because it is simpler to state and easier to use.

Tarski's system was designed for proving specific theorems rather than
more general theorem schemes. However, theorem schemes are much more
efficient than specific theorems for building a body of mathematical
knowledge, since they can be reused with different instances as
needed. While Tarski does derive some theorem schemes from his axioms,
their proofs require concepts that are ``outside'' of the system, such as
induction on formula length. The verification of such proofs is difficult
to automate in a proof verifier. (Specifically, Tarski treats the formulas
of his system as set-theoretical objects. In order to verify the proofs
of his theorem schemes, a proof verifier would need a significant amount
of set theory built into it.)

The Metamath axiom system for predicate calculus extends
Tarski's system to eliminate this difficulty. The additional
``auxilliary'' axiom
schemes (as we will call them in this section; see below) endow Tarski's
system with a nice property we call
metalogical completeness \cite[Remark 9.6]{Megill}\index{Megill, Norman}.
As a result, we can prove any theorem scheme
expressable in the ``simple metalogic'' of Tarski's system by using
only Metamath's direct substitution rule applied to the axiom system
(and no other metalogical or set-theoretical notions ``outside'' of the
system). Simple metalogic consists of schemes containing wff metavariables
(with no arguments) and/or set (also called ``individual'') metavariables,
accompanied by optional provisos each stating that two specified set
metavariables must be distinct or that a specified set metavariable may
not occur in a specified wff metavariable. Metamath's logic and set theory
axiom and rule schemes are all examples of simple metalogic. The schemes
of traditional predicate calculus with equality are examples which are
not simple metalogic, because they use wff metavariables with arguments
and have ``free for'' and ``not free in'' side conditions.

A rigorous justification for this system, using an older but
exactly equivalent set of axioms, can be
found in \cite{Megill}\index{Megill, Norman}.

This allows us to
take a different approach in the Metamath\index{Metamath} database
\texttt{set.mm}\index{set theory database (\texttt{set.mm})}.  We do not
directly use the primitive notions of ``free variable''\index{free variable}
and ``proper substitution''\index{proper
substitution}\index{substitution!proper} at all as primitive constructs.
Instead, we use a set
of axioms that are almost as simple to manipulate as those of
propositional calculus.  Our axiom system avoids complex primitive
notions by effectively embedding the complexity into the axioms
themselves.  As a result, we will end up with a larger number of axioms,
but they are ideally suited for a computer language such as Metamath.
(Section~\ref{metaaxioms} shows these axioms.)

We will not elaborate further
on the ``free variable'' and ``proper substitution''
concepts here.  You may consult
\cite[ch.\ 3--4]{Hamilton}\index{Hamilton, Alan G.} (as well as
many other books) for a precise explanation
of these concepts.  If you intend to do serious mathematical work, it is wise
to become familiar with the traditional textbook approach; even though the
concepts embedded in their axioms require a higher level of sophistication,
they can be more practical to deal with on an everyday, informal basis.  Even
if you are just developing Metamath proofs, familiarity with the traditional
approach can help you arrive at a proof outline much faster, which you can
then convert to the detail required by Metamath.

We do develop proper substitution rules later on, but in set.mm
they are defined as derived constructs; they are not primitives.

You should also note that our system of predicate calculus is specifically
tailored for set theory; thus there are only two specific predicates $=$ and
$\in$ and no functions\index{function!in predicate calculus}
or constants\index{constant!in predicate calculus} unlike more general systems.
We later add these.

\subsection{Set Theory}

Traditional Zermelo--Fraenkel set theory\index{Zermelo--Fraenkel set
theory}\index{set theory} with the Axiom of Choice
has 10 axioms, which can be expressed in the
language of predicate calculus.  In this section, we will list only the
names and brief English descriptions of these axioms, since we will give
you the precise formulas used by the Metamath\index{Metamath} set theory
database \texttt{set.mm} later on.

In the descriptions of the axioms, we assume that $x$, $y$, $z$, $w$, and $v$
represent sets.  These are the same as the variables\index{variable!in set
theory} in our predicate calculus system above, except that now we informally
think of the variables as ranging over sets.  Note that the terms
``object,''\index{object} ``set,''\index{set} ``element,''\index{element}
``collection,''\index{collection} and ``family''\index{family} are synonymous,
as are ``is an element of,'' ``is a member of,''\index{member} ``is contained
in,'' and ``belongs to.''  The different terms are used for convenience; for
example, ``a collection of sets'' is less confusing than ``a set of sets.''
A set $x$ is said to be a ``subset''\index{subset} of $y$ if every element of
$x$ is also an element of $y$; we also say $x$ is ``included in''
$y$.

The axioms are very general and apply to almost any conceivable mathematical
object, and this level of abstraction can be overwhelming at first.  To gain an
intuitive feel, it can be helpful to draw a picture illustrating the concept;
for example, a circle containing dots could represent a collection of sets,
and a smaller circle drawn inside the circle could represent a subset.
Overlapping circles can illustrate intersection and union.  Circles that
illustrate the concepts of set theory are frequently used in elementary
textbooks and are called Venn diagrams\index{Venn diagram}.\index{axioms of
set theory}

1. Axiom of Extensionality:  Two sets are identical if they contain the same
   elements.\index{Axiom of Extensionality}

2. Axiom of Pairing:  The set $\{ x , y \}$ exists.\index{Axiom of Pairing}

3. Axiom of Power Sets:  The power set of a set (the collection of all of
   its subsets) exists.  For example, the power set of $\{x,y\}$ is
   $\{\varnothing,\{x\},\{y\},\{x,y\}\}$ and it exists.\index{Axiom
of Power Sets}

4. Axiom of the Null Set:  The empty set $\varnothing$ exists.\index{Axiom of
the Null Set}

5. Axiom of Union:  The union of a set (the set containing the elements of
   its members) exists.  For example, the union of $\{\{x,y\},\{z\}\}$ is
 $\{x,y,z\}$ and
   it exists.\index{Axiom of Union}

6. Axiom of Regularity:  Roughly, no set can contain itself, nor can there
   be membership ``loops,'' such as a set being an
   element of one of its members.\index{Axiom of Regularity}

7. Axiom of Infinity:  An infinite set exists.  An example of an infinite
   set is the set of all
   integers.\index{Axiom of Infinity}

8. Axiom of Separation:  The set exists that is obtained by restricting $x$
   with some property.  For example, if the set of all integers exists,
   then the set of all even integers exists.\index{Axiom of Separation}

9. Axiom of Replacement:  The range of a function whose domain is restricted
   to the elements of a set $x$, is also a set.  For example, there
   is a function
   from integers (the function's domain) to their squares (its
   range).  If we
   restrict the domain to even integers, its range will become the set of
   squares of even integers, so this axiom asserts that the set of
    squares of even numbers exists.  Technical note:  In general, the
   ``function'' need not be a set but can be a proper class.
   \index{Axiom of Replacement}

10. Axiom of Choice:  Let $x$ be a set whose members are pairwise
  disjoint\index{disjoint sets} (i.e,
  whose members contain no elements in common).  Then there exists another
  set containing one element from each member of $x$.  For
  example, if $x$ is
  $\{\{y,z\},\{w,v\}\}$, where $y$, $z$, $w$, and $v$ are
  different sets, then a set such as $\{z,w\}$
  exists (but the axiom doesn't tell
  us which one).  (Actually the Axiom
  of Choice is redundant if the set $x$, as in this example, has a finite
  number of elements.)\index{Axiom of Choice}

The Axiom of Choice is usually considered an extension of ZF set theory rather
than a proper part of it.  It is sometimes considered philosophically
controversial because it specifies the existence of a set without specifying
what the set is. Constructive logics, including intuitionistic logic,
do not accept the axiom of choice.
Since there is some lingering controversy, we often prefer proofs that do
not use the axiom of choice (where there is a known alternative), and
in some cases we will use weaker axioms than the full axiom of choice.
That said, the axiom of choice is a powerful and widely-accepted tool,
so we do use it when needed.
ZF set theory that includes the Axiom of Choice is
called Zermelo--Fraenkel set theory with choice (ZFC\index{ZFC set theory}).

When expressed symbolically, the Axiom of Separation and the Axiom of
Replacement contain wff symbols and therefore each represent infinitely many
axioms, one for each possible wff. For this reason, they are often called
axiom schemes\index{axiom scheme}\index{well-formed formula (wff)}.

It turns out that the Axiom of the Null Set, the Axiom of Pairing, and the
Axiom of Separation can be derived from the other axioms and are therefore
unnecessary, although they tend to be included in standard texts for various
reasons (historical, philosophical, and possibly because some authors may not
know this).  In the Metamath\index{Metamath} set theory database, these
redundant axioms are derived from the other ones instead of truly
being considered axioms.
This is in keeping with our general goal of minimizing the number of
axioms we must depend on.

\subsection{Other Axioms}

Above we qualified the phrase ``all of mathematics'' with ``essentially.''
The main important missing piece is the ability to do category theory,
which requires huge sets (inaccessible cardinals) larger than those
postulated by the ZFC axioms. The Tarski--Grothendieck Axiom postulates
the existence of such sets.
Note that this is the same axiom used by Mizar for supporting
category theory.
The Tarski--Grothendieck axiom
can be viewed as a very strong replacement of the Axiom of Infinity,
the Axiom of Choice, and the Axiom of Power Sets.
The \texttt{set.mm} database includes this axiom; see the database
for details about it.
Again, we only use this axiom when we need to.
You are only likely to encounter or use this axiom if you are doing
category theory, since its use is highly specialized,
so we will not list the Tarsky-Grothendieck axiom
in the short list of axioms below.

Can there be even more axioms?
Of course.
G\"{o}del showed that no finite set of axioms or axiom schemes can completely
describe any consistent theory strong enough to include arithmetic.
But practically speaking, the ones above are the accepted foundation that
almost all mathematicians explicitly or implicitly base their work on.

\section{The Axioms in the Metamath Language}\label{metaaxioms}

Here we list the axioms as they appear in
\texttt{set.mm}\index{set theory database (\texttt{set.mm})} so you can
look them up there easily.  Incidentally, the \texttt{show statement
/tex} command\index{\texttt{show statement} command} was used to
typeset them.

%macros from show statement /tex
\newbox\mlinebox
\newbox\mtrialbox
\newbox\startprefix  % Prefix for first line of a formula
\newbox\contprefix  % Prefix for continuation line of a formula
\def\startm{  % Initialize formula line
  \setbox\mlinebox=\hbox{\unhcopy\startprefix}
}
\def\m#1{  % Add a symbol to the formula
  \setbox\mtrialbox=\hbox{\unhcopy\mlinebox $\,#1$}
  \ifdim\wd\mtrialbox>\hsize
    \box\mlinebox
    \setbox\mlinebox=\hbox{\unhcopy\contprefix $\,#1$}
  \else
    \setbox\mlinebox=\hbox{\unhbox\mtrialbox}
  \fi
}
\def\endm{  % Output the last line of a formula
  \box\mlinebox
}

% \SLASH for \ , \TOR for \/ (text OR), \TAND for /\ (text and)
% This embeds a following forced space to force the space.
\newcommand\SLASH{\char`\\~}
\newcommand\TOR{\char`\\/~}
\newcommand\TAND{/\char`\\~}
%
% Macro to output metamath raw text.
% This assumes \startprefix and \contprefix are set.
% NOTE: "\" is tricky to escape, use \SLASH, \TOR, and \TAND inside.
% Any use of "$ { ~ ^" must be escaped; ~ and ^ must be escaped specially.
% We escape { and } for consistency.
% For more about how this macro written, see:
% https://stackoverflow.com/questions/4073674/
% how-to-disable-indentation-in-particular-section-in-latex/4075706
% Use frenchspacing, or "e." will get an extra space after it.
\newlength\mystoreparindent
\newlength\mystorehangindent
\newenvironment{mmraw}{%
\setlength{\mystoreparindent}{\the\parindent}
\setlength{\mystorehangindent}{\the\hangindent}
\setlength{\parindent}{0pt} % TODO - we'll put in the \startprefix instead
\setlength{\hangindent}{\wd\the\contprefix}
\begin{flushleft}
\begin{frenchspacing}
\begin{tt}
{\unhcopy\startprefix}%
}{%
\end{tt}
\end{frenchspacing}
\end{flushleft}
\setlength{\parindent}{\mystoreparindent}
\setlength{\hangindent}{\mystorehangindent}
\vskip 1ex
}

\needspace{5\baselineskip}
\subsection{Propositional Calculus}\label{propcalc}\index{axioms of
propositional calculus}

\needspace{2\baselineskip}
Axiom of Simplification.\label{ax1}

\setbox\startprefix=\hbox{\tt \ \ ax-1\ \$a\ }
\setbox\contprefix=\hbox{\tt \ \ \ \ \ \ \ \ \ \ }
\startm
\m{\vdash}\m{(}\m{\varphi}\m{\rightarrow}\m{(}\m{\psi}\m{\rightarrow}\m{\varphi}\m{)}
\m{)}
\endm

\needspace{3\baselineskip}
\noindent Axiom of Distribution.

\setbox\startprefix=\hbox{\tt \ \ ax-2\ \$a\ }
\setbox\contprefix=\hbox{\tt \ \ \ \ \ \ \ \ \ \ }
\startm
\m{\vdash}\m{(}\m{(}\m{\varphi}\m{\rightarrow}\m{(}\m{\psi}\m{\rightarrow}\m{\chi}
\m{)}\m{)}\m{\rightarrow}\m{(}\m{(}\m{\varphi}\m{\rightarrow}\m{\psi}\m{)}\m{
\rightarrow}\m{(}\m{\varphi}\m{\rightarrow}\m{\chi}\m{)}\m{)}\m{)}
\endm

\needspace{2\baselineskip}
\noindent Axiom of Contraposition.

\setbox\startprefix=\hbox{\tt \ \ ax-3\ \$a\ }
\setbox\contprefix=\hbox{\tt \ \ \ \ \ \ \ \ \ \ }
\startm
\m{\vdash}\m{(}\m{(}\m{\lnot}\m{\varphi}\m{\rightarrow}\m{\lnot}\m{\psi}\m{)}\m{
\rightarrow}\m{(}\m{\psi}\m{\rightarrow}\m{\varphi}\m{)}\m{)}
\endm


\needspace{4\baselineskip}
\noindent Rule of Modus Ponens.\label{axmp}\index{modus ponens}

\setbox\startprefix=\hbox{\tt \ \ min\ \$e\ }
\setbox\contprefix=\hbox{\tt \ \ \ \ \ \ \ \ \ }
\startm
\m{\vdash}\m{\varphi}
\endm

\setbox\startprefix=\hbox{\tt \ \ maj\ \$e\ }
\setbox\contprefix=\hbox{\tt \ \ \ \ \ \ \ \ \ }
\startm
\m{\vdash}\m{(}\m{\varphi}\m{\rightarrow}\m{\psi}\m{)}
\endm

\setbox\startprefix=\hbox{\tt \ \ ax-mp\ \$a\ }
\setbox\contprefix=\hbox{\tt \ \ \ \ \ \ \ \ \ \ \ }
\startm
\m{\vdash}\m{\psi}
\endm


\needspace{7\baselineskip}
\subsection{Axioms of Predicate Calculus with Equality---Tarski's S2}\index{axioms of predicate calculus}

\needspace{3\baselineskip}
\noindent Rule of Generalization.\index{rule of generalization}

\setbox\startprefix=\hbox{\tt \ \ ax-g.1\ \$e\ }
\setbox\contprefix=\hbox{\tt \ \ \ \ \ \ \ \ \ \ \ \ }
\startm
\m{\vdash}\m{\varphi}
\endm

\setbox\startprefix=\hbox{\tt \ \ ax-gen\ \$a\ }
\setbox\contprefix=\hbox{\tt \ \ \ \ \ \ \ \ \ \ \ \ }
\startm
\m{\vdash}\m{\forall}\m{x}\m{\varphi}
\endm

\needspace{2\baselineskip}
\noindent Axiom of Quantified Implication.

\setbox\startprefix=\hbox{\tt \ \ ax-4\ \$a\ }
\setbox\contprefix=\hbox{\tt \ \ \ \ \ \ \ \ \ \ }
\startm
\m{\vdash}\m{(}\m{\forall}\m{x}\m{(}\m{\forall}\m{x}\m{\varphi}\m{\rightarrow}\m{
\psi}\m{)}\m{\rightarrow}\m{(}\m{\forall}\m{x}\m{\varphi}\m{\rightarrow}\m{
\forall}\m{x}\m{\psi}\m{)}\m{)}
\endm

\needspace{3\baselineskip}
\noindent Axiom of Distinctness.

% Aka: Add $d x ph $.
\setbox\startprefix=\hbox{\tt \ \ ax-5\ \$a\ }
\setbox\contprefix=\hbox{\tt \ \ \ \ \ \ \ \ \ \ }
\startm
\m{\vdash}\m{(}\m{\varphi}\m{\rightarrow}\m{\forall}\m{x}\m{\varphi}\m{)}\m{where}\m{ }\m{\$d}\m{ }\m{x}\m{ }\m{\varphi}\m{ }\m{(}\m{x}\m{ }\m{does}\m{ }\m{not}\m{ }\m{occur}\m{ }\m{in}\m{ }\m{\varphi}\m{)}
\endm

\needspace{2\baselineskip}
\noindent Axiom of Existence.

\setbox\startprefix=\hbox{\tt \ \ ax-6\ \$a\ }
\setbox\contprefix=\hbox{\tt \ \ \ \ \ \ \ \ \ \ }
\startm
\m{\vdash}\m{(}\m{\forall}\m{x}\m{(}\m{x}\m{=}\m{y}\m{\rightarrow}\m{\forall}
\m{x}\m{\varphi}\m{)}\m{\rightarrow}\m{\varphi}\m{)}
\endm

\needspace{2\baselineskip}
\noindent Axiom of Equality.

\setbox\startprefix=\hbox{\tt \ \ ax-7\ \$a\ }
\setbox\contprefix=\hbox{\tt \ \ \ \ \ \ \ \ \ \ }
\startm
\m{\vdash}\m{(}\m{x}\m{=}\m{y}\m{\rightarrow}\m{(}\m{x}\m{=}\m{z}\m{
\rightarrow}\m{y}\m{=}\m{z}\m{)}\m{)}
\endm

\needspace{2\baselineskip}
\noindent Axiom of Left Equality for Binary Predicate.

\setbox\startprefix=\hbox{\tt \ \ ax-8\ \$a\ }
\setbox\contprefix=\hbox{\tt \ \ \ \ \ \ \ \ \ \ \ }
\startm
\m{\vdash}\m{(}\m{x}\m{=}\m{y}\m{\rightarrow}\m{(}\m{x}\m{\in}\m{z}\m{
\rightarrow}\m{y}\m{\in}\m{z}\m{)}\m{)}
\endm

\needspace{2\baselineskip}
\noindent Axiom of Right Equality for Binary Predicate.

\setbox\startprefix=\hbox{\tt \ \ ax-9\ \$a\ }
\setbox\contprefix=\hbox{\tt \ \ \ \ \ \ \ \ \ \ \ }
\startm
\m{\vdash}\m{(}\m{x}\m{=}\m{y}\m{\rightarrow}\m{(}\m{z}\m{\in}\m{x}\m{
\rightarrow}\m{z}\m{\in}\m{y}\m{)}\m{)}
\endm


\needspace{4\baselineskip}
\subsection{Axioms of Predicate Calculus with Equality---Auxiliary}\index{axioms of predicate calculus - auxiliary}

\needspace{2\baselineskip}
\noindent Axiom of Quantified Negation.

\setbox\startprefix=\hbox{\tt \ \ ax-10\ \$a\ }
\setbox\contprefix=\hbox{\tt \ \ \ \ \ \ \ \ \ \ }
\startm
\m{\vdash}\m{(}\m{\lnot}\m{\forall}\m{x}\m{\lnot}\m{\forall}\m{x}\m{\varphi}\m{
\rightarrow}\m{\varphi}\m{)}
\endm

\needspace{2\baselineskip}
\noindent Axiom of Quantifier Commutation.

\setbox\startprefix=\hbox{\tt \ \ ax-11\ \$a\ }
\setbox\contprefix=\hbox{\tt \ \ \ \ \ \ \ \ \ \ }
\startm
\m{\vdash}\m{(}\m{\forall}\m{x}\m{\forall}\m{y}\m{\varphi}\m{\rightarrow}\m{
\forall}\m{y}\m{\forall}\m{x}\m{\varphi}\m{)}
\endm

\needspace{3\baselineskip}
\noindent Axiom of Substitution.

\setbox\startprefix=\hbox{\tt \ \ ax-12\ \$a\ }
\setbox\contprefix=\hbox{\tt \ \ \ \ \ \ \ \ \ \ \ }
\startm
\m{\vdash}\m{(}\m{\lnot}\m{\forall}\m{x}\m{\,x}\m{=}\m{y}\m{\rightarrow}\m{(}
\m{x}\m{=}\m{y}\m{\rightarrow}\m{(}\m{\varphi}\m{\rightarrow}\m{\forall}\m{x}\m{(}
\m{x}\m{=}\m{y}\m{\rightarrow}\m{\varphi}\m{)}\m{)}\m{)}\m{)}
\endm

\needspace{3\baselineskip}
\noindent Axiom of Quantified Equality.

\setbox\startprefix=\hbox{\tt \ \ ax-13\ \$a\ }
\setbox\contprefix=\hbox{\tt \ \ \ \ \ \ \ \ \ \ \ }
\startm
\m{\vdash}\m{(}\m{\lnot}\m{\forall}\m{z}\m{\,z}\m{=}\m{x}\m{\rightarrow}\m{(}
\m{\lnot}\m{\forall}\m{z}\m{\,z}\m{=}\m{y}\m{\rightarrow}\m{(}\m{x}\m{=}\m{y}
\m{\rightarrow}\m{\forall}\m{z}\m{\,x}\m{=}\m{y}\m{)}\m{)}\m{)}
\endm

% \noindent Axiom of Quantifier Substitution
%
% \setbox\startprefix=\hbox{\tt \ \ ax-c11n\ \$a\ }
% \setbox\contprefix=\hbox{\tt \ \ \ \ \ \ \ \ \ \ \ }
% \startm
% \m{\vdash}\m{(}\m{\forall}\m{x}\m{\,x}\m{=}\m{y}\m{\rightarrow}\m{(}\m{\forall}
% \m{x}\m{\varphi}\m{\rightarrow}\m{\forall}\m{y}\m{\varphi}\m{)}\m{)}
% \endm
%
% \noindent Axiom of Distinct Variables. (This axiom requires
% that two individual variables
% be distinct\index{\texttt{\$d} statement}\index{distinct
% variables}.)
%
% \setbox\startprefix=\hbox{\tt \ \ \ \ \ \ \ \ \$d\ }
% \setbox\contprefix=\hbox{\tt \ \ \ \ \ \ \ \ \ \ \ }
% \startm
% \m{x}\m{\,}\m{y}
% \endm
%
% \setbox\startprefix=\hbox{\tt \ \ ax-c16\ \$a\ }
% \setbox\contprefix=\hbox{\tt \ \ \ \ \ \ \ \ \ \ \ }
% \startm
% \m{\vdash}\m{(}\m{\forall}\m{x}\m{\,x}\m{=}\m{y}\m{\rightarrow}\m{(}\m{\varphi}\m{
% \rightarrow}\m{\forall}\m{x}\m{\varphi}\m{)}\m{)}
% \endm

% \noindent Axiom of Quantifier Introduction (2).  (This axiom requires
% that the individual variable not occur in the
% wff\index{\texttt{\$d} statement}\index{distinct variables}.)
%
% \setbox\startprefix=\hbox{\tt \ \ \ \ \ \ \ \ \$d\ }
% \setbox\contprefix=\hbox{\tt \ \ \ \ \ \ \ \ \ \ \ }
% \startm
% \m{x}\m{\,}\m{\varphi}
% \endm
% \setbox\startprefix=\hbox{\tt \ \ ax-5\ \$a\ }
% \setbox\contprefix=\hbox{\tt \ \ \ \ \ \ \ \ \ \ \ }
% \startm
% \m{\vdash}\m{(}\m{\varphi}\m{\rightarrow}\m{\forall}\m{x}\m{\varphi}\m{)}
% \endm

\subsection{Set Theory}\label{mmsettheoryaxioms}

In order to make the axioms of set theory\index{axioms of set theory} a little
more compact, there are several definitions from logic that we make use of
implicitly, namely, ``logical {\sc and},''\index{conjunction ($\wedge$)}
\index{logical {\sc and} ($\wedge$)} ``logical equivalence,''\index{logical
equivalence ($\leftrightarrow$)}\index{biconditional ($\leftrightarrow$)} and
``there exists.''\index{existential quantifier ($\exists$)}

\begin{center}\begin{tabular}{rcl}
  $( \varphi \wedge \psi )$ &\mbox{stands for}& $\neg ( \varphi
     \rightarrow \neg \psi )$\\
  $( \varphi \leftrightarrow \psi )$& \mbox{stands
     for}& $( ( \varphi \rightarrow \psi ) \wedge
     ( \psi \rightarrow \varphi ) )$\\
  $\exists x \,\varphi$ &\mbox{stands for}& $\neg \forall x \neg \varphi$
\end{tabular}\end{center}

In addition, the axioms of set theory require that all variables be
dis\-tinct,\index{distinct variables}\footnote{Set theory axioms can be
devised so that {\em no} variables are required to be distinct,
provided we replace \texttt{ax-c16} with an axiom stating that ``at
least two things exist,'' thus
making \texttt{ax-5} the only other axiom requiring the
\texttt{\$d} statement.  These axioms are unconventional and are not
presented here, but they can be found on the \url{http://metamath.org}
web site.  See also the Comment on
p.~\pageref{nodd}.}\index{\texttt{\$d} statement} thus we also assume:
\begin{center}
  \texttt{\$d }$x\,y\,z\,w$
\end{center}

\needspace{2\baselineskip}
\noindent Axiom of Extensionality.\index{Axiom of Extensionality}

\setbox\startprefix=\hbox{\tt \ \ ax-ext\ \$a\ }
\setbox\contprefix=\hbox{\tt \ \ \ \ \ \ \ \ \ \ \ \ }
\startm
\m{\vdash}\m{(}\m{\forall}\m{x}\m{(}\m{x}\m{\in}\m{y}\m{\leftrightarrow}\m{x}
\m{\in}\m{z}\m{)}\m{\rightarrow}\m{y}\m{=}\m{z}\m{)}
\endm

\needspace{3\baselineskip}
\noindent Axiom of Replacement.\index{Axiom of Replacement}

\setbox\startprefix=\hbox{\tt \ \ ax-rep\ \$a\ }
\setbox\contprefix=\hbox{\tt \ \ \ \ \ \ \ \ \ \ \ \ }
\startm
\m{\vdash}\m{(}\m{\forall}\m{w}\m{\exists}\m{y}\m{\forall}\m{z}\m{(}\m{%
\forall}\m{y}\m{\varphi}\m{\rightarrow}\m{z}\m{=}\m{y}\m{)}\m{\rightarrow}\m{%
\exists}\m{y}\m{\forall}\m{z}\m{(}\m{z}\m{\in}\m{y}\m{\leftrightarrow}\m{%
\exists}\m{w}\m{(}\m{w}\m{\in}\m{x}\m{\wedge}\m{\forall}\m{y}\m{\varphi}\m{)}%
\m{)}\m{)}
\endm

\needspace{2\baselineskip}
\noindent Axiom of Union.\index{Axiom of Union}

\setbox\startprefix=\hbox{\tt \ \ ax-un\ \$a\ }
\setbox\contprefix=\hbox{\tt \ \ \ \ \ \ \ \ \ \ \ }
\startm
\m{\vdash}\m{\exists}\m{x}\m{\forall}\m{y}\m{(}\m{\exists}\m{x}\m{(}\m{y}\m{
\in}\m{x}\m{\wedge}\m{x}\m{\in}\m{z}\m{)}\m{\rightarrow}\m{y}\m{\in}\m{x}\m{)}
\endm

\needspace{2\baselineskip}
\noindent Axiom of Power Sets.\index{Axiom of Power Sets}

\setbox\startprefix=\hbox{\tt \ \ ax-pow\ \$a\ }
\setbox\contprefix=\hbox{\tt \ \ \ \ \ \ \ \ \ \ \ \ }
\startm
\m{\vdash}\m{\exists}\m{x}\m{\forall}\m{y}\m{(}\m{\forall}\m{x}\m{(}\m{x}\m{
\in}\m{y}\m{\rightarrow}\m{x}\m{\in}\m{z}\m{)}\m{\rightarrow}\m{y}\m{\in}\m{x}
\m{)}
\endm

\needspace{3\baselineskip}
\noindent Axiom of Regularity.\index{Axiom of Regularity}

\setbox\startprefix=\hbox{\tt \ \ ax-reg\ \$a\ }
\setbox\contprefix=\hbox{\tt \ \ \ \ \ \ \ \ \ \ \ \ }
\startm
\m{\vdash}\m{(}\m{\exists}\m{x}\m{\,x}\m{\in}\m{y}\m{\rightarrow}\m{\exists}
\m{x}\m{(}\m{x}\m{\in}\m{y}\m{\wedge}\m{\forall}\m{z}\m{(}\m{z}\m{\in}\m{x}\m{
\rightarrow}\m{\lnot}\m{z}\m{\in}\m{y}\m{)}\m{)}\m{)}
\endm

\needspace{3\baselineskip}
\noindent Axiom of Infinity.\index{Axiom of Infinity}

\setbox\startprefix=\hbox{\tt \ \ ax-inf\ \$a\ }
\setbox\contprefix=\hbox{\tt \ \ \ \ \ \ \ \ \ \ \ \ \ \ \ }
\startm
\m{\vdash}\m{\exists}\m{x}\m{(}\m{y}\m{\in}\m{x}\m{\wedge}\m{\forall}\m{y}%
\m{(}\m{y}\m{\in}\m{x}\m{\rightarrow}\m{\exists}\m{z}\m{(}\m{y}\m{\in}\m{z}\m{%
\wedge}\m{z}\m{\in}\m{x}\m{)}\m{)}\m{)}
\endm

\needspace{4\baselineskip}
\noindent Axiom of Choice.\index{Axiom of Choice}

\setbox\startprefix=\hbox{\tt \ \ ax-ac\ \$a\ }
\setbox\contprefix=\hbox{\tt \ \ \ \ \ \ \ \ \ \ \ \ \ \ }
\startm
\m{\vdash}\m{\exists}\m{x}\m{\forall}\m{y}\m{\forall}\m{z}\m{(}\m{(}\m{y}\m{%
\in}\m{z}\m{\wedge}\m{z}\m{\in}\m{w}\m{)}\m{\rightarrow}\m{\exists}\m{w}\m{%
\forall}\m{y}\m{(}\m{\exists}\m{w}\m{(}\m{(}\m{y}\m{\in}\m{z}\m{\wedge}\m{z}%
\m{\in}\m{w}\m{)}\m{\wedge}\m{(}\m{y}\m{\in}\m{w}\m{\wedge}\m{w}\m{\in}\m{x}%
\m{)}\m{)}\m{\leftrightarrow}\m{y}\m{=}\m{w}\m{)}\m{)}
\endm

\subsection{That's It}

There you have it, the axioms for (essentially) all of mathematics!
Wonder at them and stare at them in awe.  Put a copy in your wallet, and
you will carry in your pocket the encoding for all theorems ever proved
and that ever will be proved, from the most mundane to the most
profound.

\section{A Hierarchy of Definitions}\label{hierarchy}

The axioms in the previous section in principle embody everything that can be
done within standard mathematics.  However, it is impractical to accomplish
very much by using them directly, for even simple concepts (from a human
perspective) can involve extremely long, incomprehensible formulas.
Mathematics is made practical by introducing definitions\index{definition}.
Definitions usually introduce new symbols, or at least new relationships among
existing symbols, to abbreviate more complex formulas.  An important
requirement for a definition is that there exist a straightforward
(algorithmic) method for eliminating the abbreviation by expanding it into the
more primitive symbol string that it represents.  Some
important definitions included in
the file \texttt{set.mm} are listed in this section for reference, and also to
give you a feel for why something like $\omega$\index{omega ($\omega$)} (the
set of natural numbers\index{natural number} 0, 1, 2,\ldots) becomes very
complicated when completely expanded into primitive symbols.

What is the motivation for definitions, aside from allowing complicated
expressions to be expressed more simply?  In the case of  $\omega$, one goal is
to provide a basis for the theory of natural numbers.\index{natural number}
Before set theory was invented, a set of axioms for arithmetic, called Peano's
postulates\index{Peano's postulates}, was devised and shown to have the
properties one expects for natural numbers.  Now anyone can postulate a
set of axioms, but if the axioms are inconsistent contradictions can be derived
from them.  Once a contradiction is derived, anything can be trivially
proved, including
all the facts of arithmetic and their negations.  To ensure that an
axiom system is at least as reliable as the axioms for set theory, we can
define sets and operations on those sets that satisfy the new axioms. In the
\texttt{set.mm} Metamath database, we prove that the elements of $\omega$ satisfy
Peano's postulates, and it's a long and hard journey to get there directly
from the axioms of set theory.  But the result is confidence in the
foundations of arithmetic.  And there is another advantage:  we now have all
the tools of set theory at our disposal for manipulating objects that obey the
axioms for arithmetic.

What are the criteria we use for definitions?  First, and of utmost importance,
the definition should not be {\em creative}\index{creative
definition}\index{definition!creative}, that
is it should not allow an expression that previously qualified as a wff but
was not provable, to become provable.   Second, the definition should be {\em
eliminable}\index{definition!eliminability}, that is, there should exist an
algorithmic method for converting any expression using the definition into
a logically equivalent expression that previously qualified as a wff.

In almost all cases below, definitions connect two expressions with either
$\leftrightarrow$ or $=$.  Eliminating\footnote{Here we mean the
elimination that a human might do in his or her head.  To eliminate them as
part of a Metamath proof we would invoke one of a number of
theorems that deal with transitivity of equivalence or equality; there are
many such examples in the proofs in \texttt{set.mm}.} such a definition is a
simple matter of substituting the expression on the left-hand side ({\em
definiendum}\index{definiendum} or thing being defined) with the equivalent,
more primitive expression on the right-hand side ({\em
definiens}\index{definiens} or definition).

Often a definition has variables on the right-hand side which do not appear on
the left-hand side; these are called {\em dummy variables}.\index{dummy
variable!in definitions}  In this case, any
allowable substitution (such as a new, distinct
variable) can be used when the definition is eliminated.  Dummy variables may
be used only if they are {\em effectively bound}\index{effectively bound
variable}, meaning that the definition will remain logically equivalent upon
any substitution of a dummy variable with any other {\em qualifying
expression}\index{qualifying expression}, i.e.\ any symbol string (such as
another variable) that
meets the restrictions on the dummy variable imposed by \texttt{\$d} and
\texttt{\$f} statements.  For example, we could define a constant $\perp$
(inverted tee, meaning logical ``false'') as $( \varphi \wedge \lnot \varphi
)$, i.e.\ ``phi and not phi.''  Here $\varphi$ is effectively bound because the
definition remains logically equivalent when we replace $\varphi$ with any
other wff.  (It is actually \texttt{df-fal}
in \texttt{set.mm}, which defines $\perp$.)

There are two cases where eliminating definitions is a little more
complex.  These cases are the definitions \texttt{df-bi} and
\texttt{df-cleq}.  The first stretches the concept of a definition a
little, as in effect it ``defines a definition;'' however, it meets our
requirements for a definition in that it is eliminable and does not
strengthen the language.  Theorem \texttt{bii} shows the substitution
needed to eliminate the $\leftrightarrow$\index{logical equivalence
($\leftrightarrow$)}\index{biconditional ($\leftrightarrow$)} symbol.

Definition \texttt{df-cleq}\index{equality ($=$)} extends the usage of
the equality symbol to include ``classes''\index{class} in set theory.  The
reason it is potentially problematic is that it can lead to statements which
do not follow from logic alone but presuppose the Axiom of
Extensionality\index{Axiom of Extensionality}, so we include this axiom
as a hypothesis for the definition.  We could have made \texttt{df-cleq} directly
eliminable by introducing a new equality symbol, but have chosen not to do so
in keeping with standard textbook practice.  Definitions such as \texttt{df-cleq}
that extend the meaning of existing symbols must be introduced carefully so
that they do not lead to contradictions.  Definition \texttt{df-clel} also
extends the meaning of an existing symbol ($\in$); while it doesn't strengthen
the language like \texttt{df-cleq}, this is not obvious and it must also be
subject to the same scrutiny.

Exercise:  Study how the wff $x\in\omega$, meaning ``$x$ is a natural
number,'' could be expanded in terms of primitive symbols, starting with the
definitions \texttt{df-clel} on p.~\pageref{dfclel} and \texttt{df-om} on
p.~\pageref{dfom} and working your way back.  Don't bother to work out the
details; just make sure that you understand how you could do it in principle.
The answer is shown in the footnote on p.~\pageref{expandom}.  If you
actually do work it out, you won't get exactly the same answer because we used
a few simplifications such as discarding occurrences of $\lnot\lnot$ (double
negation).

In the definitions below, we have placed the {\sc ascii} Metamath source
below each of the formulas to help you become familiar with the
notation in the database.  For simplicity, the necessary \texttt{\$f}
and \texttt{\$d} statements are not shown.  If you are in doubt, use the
\texttt{show statement}\index{\texttt{show statement} command} command
in the Metamath program to see the full statement.
A selection of this notation is summarized in Appendix~\ref{ASCII}.

To understand the motivation for these definitions, you should consult the
references indicated:  Takeuti and Zaring \cite{Takeuti}\index{Takeuti, Gaisi},
Quine \cite{Quine}\index{Quine, Willard Van Orman}, Bell and Machover
\cite{Bell}\index{Bell, J. L.}, and Enderton \cite{Enderton}\index{Enderton,
Herbert B.}.  Our list of definitions is provided more for reference than as a
learning aid.  However, by looking at a few of them you can gain a feel for
how the hierarchy is built up.  The definitions are a representative sample of
the many definitions
in \texttt{set.mm}, but they are complete with respect to the
theorem examples we will present in Section~\ref{sometheorems}.  Also, some are
slightly different from, but logically equivalent to, the ones in \texttt{set.mm}
(some of which have been revised over time to shorten them, for example).

\subsection{Definitions for Propositional Calculus}\label{metadefprop}

The symbols $\varphi$, $\psi$, and $\chi$ represent wffs.

Our first definition introduces the biconditional
connective\footnote{The term ``connective'' is informally used to mean a
symbol that is placed between two variables or adjacent to a variable,
whereas a mathematical ``constant'' usually indicates a symbol such as
the number 0 that may replace a variable or metavariable.  From
Metamath's point of view, there is no distinction between a connective
and a constant; both are constants in the Metamath
language.}\index{connective}\index{constant} (also called logical
equivalence)\index{logical equivalence
($\leftrightarrow$)}\index{biconditional ($\leftrightarrow$)}.  Unlike
most traditional developments, we have chosen not to have a separate
symbol such as ``Df.'' to mean ``is defined as.''  Instead, we will use
the biconditional connective for this purpose, as it lets us use
logic to manipulate definitions directly.  Here we state the properties
of the biconditional connective with a carefully crafted \texttt{\$a}
statement, which effectively uses the biconditional connective to define
itself.  The $\leftrightarrow$ symbol can be eliminated from a formula
using theorem \texttt{bii}, which is derived later.

\vskip 2ex
\noindent Define the biconditional connective.\label{df-bi}

\vskip 0.5ex
\setbox\startprefix=\hbox{\tt \ \ df-bi\ \$a\ }
\setbox\contprefix=\hbox{\tt \ \ \ \ \ \ \ \ \ \ \ }
\startm
\m{\vdash}\m{\lnot}\m{(}\m{(}\m{(}\m{\varphi}\m{\leftrightarrow}\m{\psi}\m{)}%
\m{\rightarrow}\m{\lnot}\m{(}\m{(}\m{\varphi}\m{\rightarrow}\m{\psi}\m{)}\m{%
\rightarrow}\m{\lnot}\m{(}\m{\psi}\m{\rightarrow}\m{\varphi}\m{)}\m{)}\m{)}\m{%
\rightarrow}\m{\lnot}\m{(}\m{\lnot}\m{(}\m{(}\m{\varphi}\m{\rightarrow}\m{%
\psi}\m{)}\m{\rightarrow}\m{\lnot}\m{(}\m{\psi}\m{\rightarrow}\m{\varphi}\m{)}%
\m{)}\m{\rightarrow}\m{(}\m{\varphi}\m{\leftrightarrow}\m{\psi}\m{)}\m{)}\m{)}
\endm
\begin{mmraw}%
|- -. ( ( ( ph <-> ps ) -> -. ( ( ph -> ps ) ->
-. ( ps -> ph ) ) ) -> -. ( -. ( ( ph -> ps ) -> -. (
ps -> ph ) ) -> ( ph <-> ps ) ) ) \$.
\end{mmraw}

\noindent This theorem relates the biconditional connective to primitive
connectives and can be used to eliminate the $\leftrightarrow$ symbol from any
wff.

\vskip 0.5ex
\setbox\startprefix=\hbox{\tt \ \ bii\ \$p\ }
\setbox\contprefix=\hbox{\tt \ \ \ \ \ \ \ \ \ }
\startm
\m{\vdash}\m{(}\m{(}\m{\varphi}\m{\leftrightarrow}\m{\psi}\m{)}\m{\leftrightarrow}
\m{\lnot}\m{(}\m{(}\m{\varphi}\m{\rightarrow}\m{\psi}\m{)}\m{\rightarrow}\m{\lnot}
\m{(}\m{\psi}\m{\rightarrow}\m{\varphi}\m{)}\m{)}\m{)}
\endm
\begin{mmraw}%
|- ( ( ph <-> ps ) <-> -. ( ( ph -> ps ) -> -. ( ps -> ph ) ) ) \$= ... \$.
\end{mmraw}

\noindent Define disjunction ({\sc or}).\index{disjunction ($\vee$)}%
\index{logical or (vee)@logical {\sc or} ($\vee$)}%
\index{df-or@\texttt{df-or}}\label{df-or}

\vskip 0.5ex
\setbox\startprefix=\hbox{\tt \ \ df-or\ \$a\ }
\setbox\contprefix=\hbox{\tt \ \ \ \ \ \ \ \ \ \ \ }
\startm
\m{\vdash}\m{(}\m{(}\m{\varphi}\m{\vee}\m{\psi}\m{)}\m{\leftrightarrow}\m{(}\m{
\lnot}\m{\varphi}\m{\rightarrow}\m{\psi}\m{)}\m{)}
\endm
\begin{mmraw}%
|- ( ( ph \TOR ps ) <-> ( -. ph -> ps ) ) \$.
\end{mmraw}

\noindent Define conjunction ({\sc and}).\index{conjunction ($\wedge$)}%
\index{logical {\sc and} ($\wedge$)}%
\index{df-an@\texttt{df-an}}\label{df-an}

\vskip 0.5ex
\setbox\startprefix=\hbox{\tt \ \ df-an\ \$a\ }
\setbox\contprefix=\hbox{\tt \ \ \ \ \ \ \ \ \ \ \ }
\startm
\m{\vdash}\m{(}\m{(}\m{\varphi}\m{\wedge}\m{\psi}\m{)}\m{\leftrightarrow}\m{\lnot}
\m{(}\m{\varphi}\m{\rightarrow}\m{\lnot}\m{\psi}\m{)}\m{)}
\endm
\begin{mmraw}%
|- ( ( ph \TAND ps ) <-> -. ( ph -> -. ps ) ) \$.
\end{mmraw}

\noindent Define disjunction ({\sc or}) of 3 wffs.%
\index{df-3or@\texttt{df-3or}}\label{df-3or}

\vskip 0.5ex
\setbox\startprefix=\hbox{\tt \ \ df-3or\ \$a\ }
\setbox\contprefix=\hbox{\tt \ \ \ \ \ \ \ \ \ \ \ \ }
\startm
\m{\vdash}\m{(}\m{(}\m{\varphi}\m{\vee}\m{\psi}\m{\vee}\m{\chi}\m{)}\m{
\leftrightarrow}\m{(}\m{(}\m{\varphi}\m{\vee}\m{\psi}\m{)}\m{\vee}\m{\chi}\m{)}
\m{)}
\endm
\begin{mmraw}%
|- ( ( ph \TOR ps \TOR ch ) <-> ( ( ph \TOR ps ) \TOR ch ) ) \$.
\end{mmraw}

\noindent Define conjunction ({\sc and}) of 3 wffs.%
\index{df-3an}\label{df-3an}

\vskip 0.5ex
\setbox\startprefix=\hbox{\tt \ \ df-3an\ \$a\ }
\setbox\contprefix=\hbox{\tt \ \ \ \ \ \ \ \ \ \ \ \ }
\startm
\m{\vdash}\m{(}\m{(}\m{\varphi}\m{\wedge}\m{\psi}\m{\wedge}\m{\chi}\m{)}\m{
\leftrightarrow}\m{(}\m{(}\m{\varphi}\m{\wedge}\m{\psi}\m{)}\m{\wedge}\m{\chi}
\m{)}\m{)}
\endm

\begin{mmraw}%
|- ( ( ph \TAND ps \TAND ch ) <-> ( ( ph \TAND ps ) \TAND ch ) ) \$.
\end{mmraw}

\subsection{Definitions for Predicate Calculus}\label{metadefpred}

The symbols $x$, $y$, and $z$ represent individual variables of predicate
calculus.  In this section, they are not necessarily distinct unless it is
explicitly
mentioned.

\vskip 2ex
\noindent Define existential quantification.
The expression $\exists x \varphi$ means
``there exists an $x$ where $\varphi$ is true.''\index{existential quantifier
($\exists$)}\label{df-ex}

\vskip 0.5ex
\setbox\startprefix=\hbox{\tt \ \ df-ex\ \$a\ }
\setbox\contprefix=\hbox{\tt \ \ \ \ \ \ \ \ \ \ \ }
\startm
\m{\vdash}\m{(}\m{\exists}\m{x}\m{\varphi}\m{\leftrightarrow}\m{\lnot}\m{\forall}
\m{x}\m{\lnot}\m{\varphi}\m{)}
\endm
\begin{mmraw}%
|- ( E. x ph <-> -. A. x -. ph ) \$.
\end{mmraw}

\noindent Define proper substitution.\index{proper
substitution}\index{substitution!proper}\label{df-sb}
In our notation, we use $[ y / x ] \varphi$ to mean ``the wff that
results when $y$ is properly substituted for $x$ in the wff
$\varphi$.''\footnote{
This can also be described
as substituting $x$ with $y$, $y$ properly replaces $x$, or
$x$ is properly replaced by $y$.}
% This is elsb4, though it currently says: ( [ x / y ] z e. y <-> z e. x )
For example,
$[ y / x ] z \in x$ is the same as $z \in y$.
One way to remember this notation is to notice that it looks like division
and recall that $( y / x ) \cdot x $ is $y$ (when $x \neq 0$).
The notation is different from the notation $\varphi ( x | y )$
that is sometimes used, because the latter notation is ambiguous for us:
for example, we don't know whether $\lnot \varphi ( x | y )$ is to be
interpreted as $\lnot ( \varphi ( x | y ) )$ or
$( \lnot \varphi ) ( x | y )$.\footnote{Because of the way
we initially defined wffs, this is the case
with any postfix connective\index{postfix connective} (one occurring after the
symbols being connected) or infix connective\index{infix connective} (one
occurring between the symbols being connected).  Metamath does not have a
built-in notion of operator binding strength that could eliminate the
ambiguity.  The initial parenthesis effectively provides a prefix
connective\index{prefix connective} to eliminate ambiguity.  Some conventions,
such as Polish notation\index{Polish notation} used in the 1930's and 1940's
by Polish logicians, use only prefix connectives and thus allow the total
elimination of parentheses, at the expense of readability.  In Metamath we
could actually redefine all notation to be Polish if we wanted to without
having to change any proofs!}  Other texts often use $\varphi(y)$ to indicate
our $[ y / x ] \varphi$, but this notation is even more ambiguous since there is
no explicit indication of what is being substituted.
Note that this
definition is valid even when
$x$ and $y$ are the same variable.  The first conjunct is a ``trick'' used to
achieve this property, making the definition look somewhat peculiar at
first.

\vskip 0.5ex
\setbox\startprefix=\hbox{\tt \ \ df-sb\ \$a\ }
\setbox\contprefix=\hbox{\tt \ \ \ \ \ \ \ \ \ \ \ }
\startm
\m{\vdash}\m{(}\m{[}\m{y}\m{/}\m{x}\m{]}\m{\varphi}\m{\leftrightarrow}\m{(}%
\m{(}\m{x}\m{=}\m{y}\m{\rightarrow}\m{\varphi}\m{)}\m{\wedge}\m{\exists}\m{x}%
\m{(}\m{x}\m{=}\m{y}\m{\wedge}\m{\varphi}\m{)}\m{)}\m{)}
\endm
\begin{mmraw}%
|- ( [ y / x ] ph <-> ( ( x = y -> ph ) \TAND E. x ( x = y \TAND ph ) ) ) \$.
\end{mmraw}


\noindent Define existential uniqueness\index{existential uniqueness
quantifier ($\exists "!$)} (``there exists exactly one'').  Note that $y$ is a
variable distinct from $x$ and not occurring in $\varphi$.\label{df-eu}

\vskip 0.5ex
\setbox\startprefix=\hbox{\tt \ \ df-eu\ \$a\ }
\setbox\contprefix=\hbox{\tt \ \ \ \ \ \ \ \ \ \ \ }
\startm
\m{\vdash}\m{(}\m{\exists}\m{{!}}\m{x}\m{\varphi}\m{\leftrightarrow}\m{\exists}
\m{y}\m{\forall}\m{x}\m{(}\m{\varphi}\m{\leftrightarrow}\m{x}\m{=}\m{y}\m{)}\m{)}
\endm

\begin{mmraw}%
|- ( E! x ph <-> E. y A. x ( ph <-> x = y ) ) \$.
\end{mmraw}

\subsection{Definitions for Set Theory}\label{setdefinitions}

The symbols $x$, $y$, $z$, and $w$ represent individual variables of
predicate calculus, which in set theory are understood to be sets.
However, using only the constructs shown so far would be very inconvenient.

To make set theory more practical, we introduce the notion of a ``class.''
A class\index{class} is either a set variable (such as $x$) or an
expression of the form $\{ x | \varphi\}$ (called an ``abstraction
class''\index{abstraction class}\index{class abstraction}).  Note that
sets (i.e.\ individual variables) always exist (this is a theorem of
logic, namely $\exists y \, y = x$ for any set $x$), whereas classes may
or may not exist (i.e.\ $\exists y \, y = A$ may or may not be true).
If a class does not exist it is called a ``proper class.''\index{proper
class}\index{class!proper} Definitions \texttt{df-clab},
\texttt{df-cleq}, and \texttt{df-clel} can be used to convert an
expression containing classes into one containing only set variables and
wff metavariables.

The symbols $A$, $B$, $C$, $D$, $ F$, $G$, and $R$ are metavariables that range
over classes.  A class metavariable $A$ may be eliminated from a wff by
replacing it with $\{ x|\varphi\}$ where neither $x$ nor $\varphi$ occur in
the wff.

The theory of classes can be shown to be an eliminable and conservative
extension of set theory. The \textbf{eliminability}
property shows that for every
formula in the extended language we can build a logically equivalent
formula in the basic language; so that even if the extended language
provides more ease to convey and formulate mathematical ideas for set
theory, its expressive power does not in fact strengthen the basic
language's expressive power.
The \textbf{conservation} property shows that for
every proof of a formula of the basic language in the extended system
we can build another proof of the same formula in the basic system;
so that, concerning theorems on sets only, the deductive powers of
the extended system and of the basic system are identical. Together,
these properties mean that the extended language can be treated as a
definitional extension that is \textbf{sound}.

A rigorous justification, which we will not give here, can be found in
Levy \cite[pp.~357-366]{Levy} supplementing his informal introduction to class
theory on pp.~7-17. Two other good treatments of class theory are provided
by Quine \cite[pp.~15-21]{Quine}\index{Quine, Willard Van Orman}
and also \cite[pp.~10-14]{Takeuti}\index{Takeuti, Gaisi}.
Quine's exposition (he calls them virtual classes)
is nicely written and very readable.

In the rest of this
section, individual variables are always assumed to be distinct from
each other unless otherwise indicated.  In addition, dummy variables on the
right-hand side of a definition do not occur in the class and wff
metavariables in the definition.

The definitions we present here are a partial but self-contained
collection selected from several hundred that appear in the current
\texttt{set.mm} database.  They are adequate for a basic development of
elementary set theory.

\vskip 2ex
\noindent Define the abstraction class.\index{abstraction class}\index{class
abstraction}\label{df-clab}  $x$ and $y$
need not be distinct.  Definition 2.1 of Quine, p.~16.  This definition may
seem puzzling since it is shorter than the expression being defined and does not
buy us anything in terms of brevity.  The reason we introduce this definition
is because it fits in neatly with the extension of the $\in$ connective
provided by \texttt{df-clel}.

\vskip 0.5ex
\setbox\startprefix=\hbox{\tt \ \ df-clab\ \$a\ }
\setbox\contprefix=\hbox{\tt \ \ \ \ \ \ \ \ \ \ \ \ \ }
\startm
\m{\vdash}\m{(}\m{x}\m{\in}\m{\{}\m{y}\m{|}\m{\varphi}\m{\}}\m{%
\leftrightarrow}\m{[}\m{x}\m{/}\m{y}\m{]}\m{\varphi}\m{)}
\endm
\begin{mmraw}%
|- ( x e. \{ y | ph \} <-> [ x / y ] ph ) \$.
\end{mmraw}

\noindent Define the equality connective between classes\index{class
equality}\label{df-cleq}.  See Quine or Chapter 4 of Takeuti and Zaring for its
justification and methods for eliminating it.  This is an example of a
somewhat ``dangerous'' definition, because it extends the use of the
existing equality symbol rather than introducing a new symbol, allowing
us to make statements in the original language that may not be true.
For example, it permits us to deduce $y = z \leftrightarrow \forall x (
x \in y \leftrightarrow x \in z )$ which is not a theorem of logic but
rather presupposes the Axiom of Extensionality,\index{Axiom of
Extensionality} which we include as a hypothesis so that we can know
when this axiom is assumed in a proof (with the \texttt{show
trace{\char`\_}back} command).  We could avoid the danger by introducing
another symbol, say $\eqcirc$, in place of $=$; this
would also have the advantage of making elimination of the definition
straightforward and would eliminate the need for Extensionality as a
hypothesis.  We would then also have the advantage of being able to
identify exactly where Extensionality truly comes into play.  One of our
theorems would be $x \eqcirc y \leftrightarrow x = y$
by invoking Extensionality.  However in keeping with standard practice
we retain the ``dangerous'' definition.

\vskip 0.5ex
\setbox\startprefix=\hbox{\tt \ \ df-cleq.1\ \$e\ }
\setbox\contprefix=\hbox{\tt \ \ \ \ \ \ \ \ \ \ \ \ \ \ \ }
\startm
\m{\vdash}\m{(}\m{\forall}\m{x}\m{(}\m{x}\m{\in}\m{y}\m{\leftrightarrow}\m{x}
\m{\in}\m{z}\m{)}\m{\rightarrow}\m{y}\m{=}\m{z}\m{)}
\endm
\setbox\startprefix=\hbox{\tt \ \ df-cleq\ \$a\ }
\setbox\contprefix=\hbox{\tt \ \ \ \ \ \ \ \ \ \ \ \ \ }
\startm
\m{\vdash}\m{(}\m{A}\m{=}\m{B}\m{\leftrightarrow}\m{\forall}\m{x}\m{(}\m{x}\m{
\in}\m{A}\m{\leftrightarrow}\m{x}\m{\in}\m{B}\m{)}\m{)}
\endm
% We need to reset the startprefix and contprefix.
\setbox\startprefix=\hbox{\tt \ \ df-cleq.1\ \$e\ }
\setbox\contprefix=\hbox{\tt \ \ \ \ \ \ \ \ \ \ \ \ \ \ \ }
\begin{mmraw}%
|- ( A. x ( x e. y <-> x e. z ) -> y = z ) \$.
\end{mmraw}
\setbox\startprefix=\hbox{\tt \ \ df-cleq\ \$a\ }
\setbox\contprefix=\hbox{\tt \ \ \ \ \ \ \ \ \ \ \ \ \ }
\begin{mmraw}%
|- ( A = B <-> A. x ( x e. A <-> x e. B ) ) \$.
\end{mmraw}

\noindent Define the membership connective between classes\index{class
membership}.  Theorem 6.3 of Quine, p.~41, which we adopt as a definition.
Note that it extends the use of the existing membership symbol, but unlike
{\tt df-cleq} it does not extend the set of valid wffs of logic when the class
metavariables are replaced with set variables.\label{dfclel}\label{df-clel}

\vskip 0.5ex
\setbox\startprefix=\hbox{\tt \ \ df-clel\ \$a\ }
\setbox\contprefix=\hbox{\tt \ \ \ \ \ \ \ \ \ \ \ \ \ }
\startm
\m{\vdash}\m{(}\m{A}\m{\in}\m{B}\m{\leftrightarrow}\m{\exists}\m{x}\m{(}\m{x}
\m{=}\m{A}\m{\wedge}\m{x}\m{\in}\m{B}\m{)}\m{)}
\endm
\begin{mmraw}%
|- ( A e. B <-> E. x ( x = A \TAND x e. B ) ) \$.?
\end{mmraw}

\noindent Define inequality.

\vskip 0.5ex
\setbox\startprefix=\hbox{\tt \ \ df-ne\ \$a\ }
\setbox\contprefix=\hbox{\tt \ \ \ \ \ \ \ \ \ \ \ }
\startm
\m{\vdash}\m{(}\m{A}\m{\ne}\m{B}\m{\leftrightarrow}\m{\lnot}\m{A}\m{=}\m{B}%
\m{)}
\endm
\begin{mmraw}%
|- ( A =/= B <-> -. A = B ) \$.
\end{mmraw}

\noindent Define restricted universal quantification.\index{universal
quantifier ($\forall$)!restricted}  Enderton, p.~22.

\vskip 0.5ex
\setbox\startprefix=\hbox{\tt \ \ df-ral\ \$a\ }
\setbox\contprefix=\hbox{\tt \ \ \ \ \ \ \ \ \ \ \ \ }
\startm
\m{\vdash}\m{(}\m{\forall}\m{x}\m{\in}\m{A}\m{\varphi}\m{\leftrightarrow}\m{%
\forall}\m{x}\m{(}\m{x}\m{\in}\m{A}\m{\rightarrow}\m{\varphi}\m{)}\m{)}
\endm
\begin{mmraw}%
|- ( A. x e. A ph <-> A. x ( x e. A -> ph ) ) \$.
\end{mmraw}

\noindent Define restricted existential quantification.\index{existential
quantifier ($\exists$)!restricted}  Enderton, p.~22.

\vskip 0.5ex
\setbox\startprefix=\hbox{\tt \ \ df-rex\ \$a\ }
\setbox\contprefix=\hbox{\tt \ \ \ \ \ \ \ \ \ \ \ \ }
\startm
\m{\vdash}\m{(}\m{\exists}\m{x}\m{\in}\m{A}\m{\varphi}\m{\leftrightarrow}\m{%
\exists}\m{x}\m{(}\m{x}\m{\in}\m{A}\m{\wedge}\m{\varphi}\m{)}\m{)}
\endm
\begin{mmraw}%
|- ( E. x e. A ph <-> E. x ( x e. A \TAND ph ) ) \$.
\end{mmraw}

\noindent Define the universal class\index{universal class ($V$)}.  Definition
5.20, p.~21, of Takeuti and Zaring.\label{df-v}

\vskip 0.5ex
\setbox\startprefix=\hbox{\tt \ \ df-v\ \$a\ }
\setbox\contprefix=\hbox{\tt \ \ \ \ \ \ \ \ \ \ }
\startm
\m{\vdash}\m{{\rm V}}\m{=}\m{\{}\m{x}\m{|}\m{x}\m{=}\m{x}\m{\}}
\endm
\begin{mmraw}%
|- {\char`\_}V = \{ x | x = x \} \$.
\end{mmraw}

\noindent Define the subclass\index{subclass}\index{subset} relationship
between two classes (called the subset relation if the classes are sets i.e.\
are not proper).  Definition 5.9 of Takeuti and Zaring, p.~17.\label{df-ss}

\vskip 0.5ex
\setbox\startprefix=\hbox{\tt \ \ df-ss\ \$a\ }
\setbox\contprefix=\hbox{\tt \ \ \ \ \ \ \ \ \ \ \ }
\startm
\m{\vdash}\m{(}\m{A}\m{\subseteq}\m{B}\m{\leftrightarrow}\m{\forall}\m{x}\m{(}
\m{x}\m{\in}\m{A}\m{\rightarrow}\m{x}\m{\in}\m{B}\m{)}\m{)}
\endm
\begin{mmraw}%
|- ( A C\_ B <-> A. x ( x e. A -> x e. B ) ) \$.
\end{mmraw}

\noindent Define the union\index{union} of two classes.  Definition 5.6 of Takeuti and Zaring,
p.~16.\label{df-un}

\vskip 0.5ex
\setbox\startprefix=\hbox{\tt \ \ df-un\ \$a\ }
\setbox\contprefix=\hbox{\tt \ \ \ \ \ \ \ \ \ \ \ }
\startm
\m{\vdash}\m{(}\m{A}\m{\cup}\m{B}\m{)}\m{=}\m{\{}\m{x}\m{|}\m{(}\m{x}\m{\in}
\m{A}\m{\vee}\m{x}\m{\in}\m{B}\m{)}\m{\}}
\endm
\begin{mmraw}%
( A u. B ) = \{ x | ( x e. A \TOR x e. B ) \} \$.
\end{mmraw}

\noindent Define the intersection\index{intersection} of two classes.  Definition 5.6 of
Takeuti and Zaring, p.~16.\label{df-in}

\vskip 0.5ex
\setbox\startprefix=\hbox{\tt \ \ df-in\ \$a\ }
\setbox\contprefix=\hbox{\tt \ \ \ \ \ \ \ \ \ \ \ }
\startm
\m{\vdash}\m{(}\m{A}\m{\cap}\m{B}\m{)}\m{=}\m{\{}\m{x}\m{|}\m{(}\m{x}\m{\in}
\m{A}\m{\wedge}\m{x}\m{\in}\m{B}\m{)}\m{\}}
\endm
% Caret ^ requires special treatment
\begin{mmraw}%
|- ( A i\^{}i B ) = \{ x | ( x e. A \TAND x e. B ) \} \$.
\end{mmraw}

\noindent Define class difference\index{class difference}\index{set difference}.
Definition 5.12 of Takeuti and Zaring, p.~20.  Several notations are used in
the literature; we chose the $\setminus$ convention instead of a minus sign to
reserve the latter for later use in, e.g., arithmetic.\label{df-dif}

\vskip 0.5ex
\setbox\startprefix=\hbox{\tt \ \ df-dif\ \$a\ }
\setbox\contprefix=\hbox{\tt \ \ \ \ \ \ \ \ \ \ \ \ }
\startm
\m{\vdash}\m{(}\m{A}\m{\setminus}\m{B}\m{)}\m{=}\m{\{}\m{x}\m{|}\m{(}\m{x}\m{
\in}\m{A}\m{\wedge}\m{\lnot}\m{x}\m{\in}\m{B}\m{)}\m{\}}
\endm
\begin{mmraw}%
( A \SLASH B ) = \{ x | ( x e. A \TAND -. x e. B ) \} \$.
\end{mmraw}

\noindent Define the empty or null set\index{empty set}\index{null set}.
Compare  Definition 5.14 of Takeuti and Zaring, p.~20.\label{df-nul}

\vskip 0.5ex
\setbox\startprefix=\hbox{\tt \ \ df-nul\ \$a\ }
\setbox\contprefix=\hbox{\tt \ \ \ \ \ \ \ \ \ \ }
\startm
\m{\vdash}\m{\varnothing}\m{=}\m{(}\m{{\rm V}}\m{\setminus}\m{{\rm V}}\m{)}
\endm
\begin{mmraw}%
|- (/) = ( {\char`\_}V \SLASH {\char`\_}V ) \$.
\end{mmraw}

\noindent Define power class\index{power set}\index{power class}.  Definition 5.10 of
Takeuti and Zaring, p.~17, but we also let it apply to proper classes.  (Note
that \verb$~P$ is the symbol for calligraphic P, the tilde
suggesting ``curly;'' see Appendix~\ref{ASCII}.)\label{df-pw}

\vskip 0.5ex
\setbox\startprefix=\hbox{\tt \ \ df-pw\ \$a\ }
\setbox\contprefix=\hbox{\tt \ \ \ \ \ \ \ \ \ \ \ }
\startm
\m{\vdash}\m{{\cal P}}\m{A}\m{=}\m{\{}\m{x}\m{|}\m{x}\m{\subseteq}\m{A}\m{\}}
\endm
% Special incantation required to put ~ into the text
\begin{mmraw}%
|- \char`\~P~A = \{ x | x C\_ A \} \$.
\end{mmraw}

\noindent Define the singleton of a class\index{singleton}.  Definition 7.1 of
Quine, p.~48.  It is well-defined for proper classes, although
it is not very meaningful in this case, where it evaluates to the empty
set.

\vskip 0.5ex
\setbox\startprefix=\hbox{\tt \ \ df-sn\ \$a\ }
\setbox\contprefix=\hbox{\tt \ \ \ \ \ \ \ \ \ \ \ }
\startm
\m{\vdash}\m{\{}\m{A}\m{\}}\m{=}\m{\{}\m{x}\m{|}\m{x}\m{=}\m{A}\m{\}}
\endm
\begin{mmraw}%
|- \{ A \} = \{ x | x = A \} \$.
\end{mmraw}%

\noindent Define an unordered pair of classes\index{unordered pair}\index{pair}.  Definition
7.1 of Quine, p.~48.

\vskip 0.5ex
\setbox\startprefix=\hbox{\tt \ \ df-pr\ \$a\ }
\setbox\contprefix=\hbox{\tt \ \ \ \ \ \ \ \ \ \ \ }
\startm
\m{\vdash}\m{\{}\m{A}\m{,}\m{B}\m{\}}\m{=}\m{(}\m{\{}\m{A}\m{\}}\m{\cup}\m{\{}
\m{B}\m{\}}\m{)}
\endm
\begin{mmraw}%
|- \{ A , B \} = ( \{ A \} u. \{ B \} ) \$.
\end{mmraw}

\noindent Define an unordered triple of classes\index{unordered triple}.  Definition of
Enderton, p.~19.

\vskip 0.5ex
\setbox\startprefix=\hbox{\tt \ \ df-tp\ \$a\ }
\setbox\contprefix=\hbox{\tt \ \ \ \ \ \ \ \ \ \ \ }
\startm
\m{\vdash}\m{\{}\m{A}\m{,}\m{B}\m{,}\m{C}\m{\}}\m{=}\m{(}\m{\{}\m{A}\m{,}\m{B}
\m{\}}\m{\cup}\m{\{}\m{C}\m{\}}\m{)}
\endm
\begin{mmraw}%
|- \{ A , B , C \} = ( \{ A , B \} u. \{ C \} ) \$.
\end{mmraw}%

\noindent Kuratowski's\index{Kuratowski, Kazimierz} ordered pair\index{ordered
pair} definition.  Definition 9.1 of Quine, p.~58. For proper classes it is
not meaningful but is well-defined for convenience.  (Note that \verb$<.$
stands for $\langle$ whereas \verb$<$ stands for $<$, and similarly for
\verb$>.$\,.)\label{df-op}

\vskip 0.5ex
\setbox\startprefix=\hbox{\tt \ \ df-op\ \$a\ }
\setbox\contprefix=\hbox{\tt \ \ \ \ \ \ \ \ \ \ \ }
\startm
\m{\vdash}\m{\langle}\m{A}\m{,}\m{B}\m{\rangle}\m{=}\m{\{}\m{\{}\m{A}\m{\}}
\m{,}\m{\{}\m{A}\m{,}\m{B}\m{\}}\m{\}}
\endm
\begin{mmraw}%
|- <. A , B >. = \{ \{ A \} , \{ A , B \} \} \$.
\end{mmraw}

\noindent Define the union of a class\index{union}.  Definition 5.5, p.~16,
of Takeuti and Zaring.

\vskip 0.5ex
\setbox\startprefix=\hbox{\tt \ \ df-uni\ \$a\ }
\setbox\contprefix=\hbox{\tt \ \ \ \ \ \ \ \ \ \ \ \ }
\startm
\m{\vdash}\m{\bigcup}\m{A}\m{=}\m{\{}\m{x}\m{|}\m{\exists}\m{y}\m{(}\m{x}\m{
\in}\m{y}\m{\wedge}\m{y}\m{\in}\m{A}\m{)}\m{\}}
\endm
\begin{mmraw}%
|- U. A = \{ x | E. y ( x e. y \TAND y e. A ) \} \$.
\end{mmraw}

\noindent Define the intersection\index{intersection} of a class.  Definition 7.35,
p.~44, of Takeuti and Zaring.

\vskip 0.5ex
\setbox\startprefix=\hbox{\tt \ \ df-int\ \$a\ }
\setbox\contprefix=\hbox{\tt \ \ \ \ \ \ \ \ \ \ \ \ }
\startm
\m{\vdash}\m{\bigcap}\m{A}\m{=}\m{\{}\m{x}\m{|}\m{\forall}\m{y}\m{(}\m{y}\m{
\in}\m{A}\m{\rightarrow}\m{x}\m{\in}\m{y}\m{)}\m{\}}
\endm
\begin{mmraw}%
|- |\^{}| A = \{ x | A. y ( y e. A -> x e. y ) \} \$.
\end{mmraw}

\noindent Define a transitive class\index{transitive class}\index{transitive
set}.  This should not be confused with a transitive relation, which is a different
concept.  Definition from p.~71 of Enderton, extended to classes.

\vskip 0.5ex
\setbox\startprefix=\hbox{\tt \ \ df-tr\ \$a\ }
\setbox\contprefix=\hbox{\tt \ \ \ \ \ \ \ \ \ \ \ }
\startm
\m{\vdash}\m{(}\m{\mbox{\rm Tr}}\m{A}\m{\leftrightarrow}\m{\bigcup}\m{A}\m{
\subseteq}\m{A}\m{)}
\endm
\begin{mmraw}%
|- ( Tr A <-> U. A C\_ A ) \$.
\end{mmraw}

\noindent Define a notation for a general binary relation\index{binary
relation}.  Definition 6.18, p.~29, of Takeuti and Zaring, generalized to
arbitrary classes.  This definition is well-defined, although not very
meaningful, when classes $A$ and/or $B$ are proper.\label{dfbr}  The lack of
parentheses (or any other connective) creates no ambiguity since we are defining
an atomic wff.

\vskip 0.5ex
\setbox\startprefix=\hbox{\tt \ \ df-br\ \$a\ }
\setbox\contprefix=\hbox{\tt \ \ \ \ \ \ \ \ \ \ \ }
\startm
\m{\vdash}\m{(}\m{A}\m{\,R}\m{\,B}\m{\leftrightarrow}\m{\langle}\m{A}\m{,}\m{B}
\m{\rangle}\m{\in}\m{R}\m{)}
\endm
\begin{mmraw}%
|- ( A R B <-> <. A , B >. e. R ) \$.
\end{mmraw}

\noindent Define an abstraction class of ordered pairs\index{abstraction
class!of ordered
pairs}.  A special case of Definition 4.16, p.~14, of Takeuti and Zaring.
Note that $ z $ must be distinct from $ x $ and $ y $,
and $ z $ must not occur in $\varphi$, but $ x $ and $ y $ may be
identical and may appear in $\varphi$.

\vskip 0.5ex
\setbox\startprefix=\hbox{\tt \ \ df-opab\ \$a\ }
\setbox\contprefix=\hbox{\tt \ \ \ \ \ \ \ \ \ \ \ \ \ }
\startm
\m{\vdash}\m{\{}\m{\langle}\m{x}\m{,}\m{y}\m{\rangle}\m{|}\m{\varphi}\m{\}}\m{=}
\m{\{}\m{z}\m{|}\m{\exists}\m{x}\m{\exists}\m{y}\m{(}\m{z}\m{=}\m{\langle}\m{x}
\m{,}\m{y}\m{\rangle}\m{\wedge}\m{\varphi}\m{)}\m{\}}
\endm

\begin{mmraw}%
|- \{ <. x , y >. | ph \} = \{ z | E. x E. y ( z =
<. x , y >. /\ ph ) \} \$.
\end{mmraw}

\noindent Define the epsilon relation\index{epsilon relation}.  Similar to Definition
6.22, p.~30, of Takeuti and Zaring.

\vskip 0.5ex
\setbox\startprefix=\hbox{\tt \ \ df-eprel\ \$a\ }
\setbox\contprefix=\hbox{\tt \ \ \ \ \ \ \ \ \ \ \ \ \ \ }
\startm
\m{\vdash}\m{{\rm E}}\m{=}\m{\{}\m{\langle}\m{x}\m{,}\m{y}\m{\rangle}\m{|}\m{x}\m{
\in}\m{y}\m{\}}
\endm
\begin{mmraw}%
|- \_E = \{ <. x , y >. | x e. y \} \$.
\end{mmraw}

\noindent Define a founded relation\index{founded relation}.  $R$ is a founded
relation on $A$ iff\index{iff} (if and only if) each nonempty subset of $A$
has an ``$R$-minimal element.''  Similar to Definition 6.21, p.~30, of
Takeuti and Zaring.

\vskip 0.5ex
\setbox\startprefix=\hbox{\tt \ \ df-fr\ \$a\ }
\setbox\contprefix=\hbox{\tt \ \ \ \ \ \ \ \ \ \ \ }
\startm
\m{\vdash}\m{(}\m{R}\m{\,\mbox{\rm Fr}}\m{\,A}\m{\leftrightarrow}\m{\forall}\m{x}
\m{(}\m{(}\m{x}\m{\subseteq}\m{A}\m{\wedge}\m{\lnot}\m{x}\m{=}\m{\varnothing}
\m{)}\m{\rightarrow}\m{\exists}\m{y}\m{(}\m{y}\m{\in}\m{x}\m{\wedge}\m{(}\m{x}
\m{\cap}\m{\{}\m{z}\m{|}\m{z}\m{\,R}\m{\,y}\m{\}}\m{)}\m{=}\m{\varnothing}\m{)}
\m{)}\m{)}
\endm
\begin{mmraw}%
|- ( R Fr A <-> A. x ( ( x C\_ A \TAND -. x = (/) ) ->
E. y ( y e. x \TAND ( x i\^{}i \{ z | z R y \} ) = (/) ) ) ) \$.
\end{mmraw}

\noindent Define a well-ordering\index{well-ordering}.  $R$ is a well-ordering of $A$ iff
it is founded on $A$ and the elements of $A$ are pairwise $R$-comparable.
Similar to Definition 6.24(2), p.~30, of Takeuti and Zaring.

\vskip 0.5ex
\setbox\startprefix=\hbox{\tt \ \ df-we\ \$a\ }
\setbox\contprefix=\hbox{\tt \ \ \ \ \ \ \ \ \ \ \ }
\startm
\m{\vdash}\m{(}\m{R}\m{\,\mbox{\rm We}}\m{\,A}\m{\leftrightarrow}\m{(}\m{R}\m{\,
\mbox{\rm Fr}}\m{\,A}\m{\wedge}\m{\forall}\m{x}\m{\forall}\m{y}\m{(}\m{(}\m{x}\m{
\in}\m{A}\m{\wedge}\m{y}\m{\in}\m{A}\m{)}\m{\rightarrow}\m{(}\m{x}\m{\,R}\m{\,y}
\m{\vee}\m{x}\m{=}\m{y}\m{\vee}\m{y}\m{\,R}\m{\,x}\m{)}\m{)}\m{)}\m{)}
\endm
\begin{mmraw}%
( R We A <-> ( R Fr A \TAND A. x A. y ( ( x e.
A \TAND y e. A ) -> ( x R y \TOR x = y \TOR y R x ) ) ) ) \$.
\end{mmraw}

\noindent Define the ordinal predicate\index{ordinal predicate}, which is true for a class
that is transitive and is well-ordered by the epsilon relation.  Similar to
definition on p.~468, Bell and Machover.

\vskip 0.5ex
\setbox\startprefix=\hbox{\tt \ \ df-ord\ \$a\ }
\setbox\contprefix=\hbox{\tt \ \ \ \ \ \ \ \ \ \ \ \ }
\startm
\m{\vdash}\m{(}\m{\mbox{\rm Ord}}\m{\,A}\m{\leftrightarrow}\m{(}
\m{\mbox{\rm Tr}}\m{\,A}\m{\wedge}\m{E}\m{\,\mbox{\rm We}}\m{\,A}\m{)}\m{)}
\endm
\begin{mmraw}%
|- ( Ord A <-> ( Tr A \TAND E We A ) ) \$.
\end{mmraw}

\noindent Define the class of all ordinal numbers\index{ordinal number}.  An ordinal number is
a set that satisfies the ordinal predicate.  Definition 7.11 of Takeuti and
Zaring, p.~38.

\vskip 0.5ex
\setbox\startprefix=\hbox{\tt \ \ df-on\ \$a\ }
\setbox\contprefix=\hbox{\tt \ \ \ \ \ \ \ \ \ \ \ }
\startm
\m{\vdash}\m{\,\mbox{\rm On}}\m{=}\m{\{}\m{x}\m{|}\m{\mbox{\rm Ord}}\m{\,x}
\m{\}}
\endm
\begin{mmraw}%
|- On = \{ x | Ord x \} \$.
\end{mmraw}

\noindent Define the limit ordinal predicate\index{limit ordinal}, which is true for a
non-empty ordinal that is not a successor (i.e.\ that is the union of itself).
Compare Bell and Machover, p.~471 and Exercise (1), p.~42 of Takeuti and
Zaring.

\vskip 0.5ex
\setbox\startprefix=\hbox{\tt \ \ df-lim\ \$a\ }
\setbox\contprefix=\hbox{\tt \ \ \ \ \ \ \ \ \ \ \ \ }
\startm
\m{\vdash}\m{(}\m{\mbox{\rm Lim}}\m{\,A}\m{\leftrightarrow}\m{(}\m{\mbox{
\rm Ord}}\m{\,A}\m{\wedge}\m{\lnot}\m{A}\m{=}\m{\varnothing}\m{\wedge}\m{A}
\m{=}\m{\bigcup}\m{A}\m{)}\m{)}
\endm
\begin{mmraw}%
|- ( Lim A <-> ( Ord A \TAND -. A = (/) \TAND A = U. A ) ) \$.
\end{mmraw}

\noindent Define the successor\index{successor} of a class.  Definition 7.22 of Takeuti
and Zaring, p.~41.  Our definition is a generalization to classes, although it
is meaningless when classes are proper.

\vskip 0.5ex
\setbox\startprefix=\hbox{\tt \ \ df-suc\ \$a\ }
\setbox\contprefix=\hbox{\tt \ \ \ \ \ \ \ \ \ \ \ \ }
\startm
\m{\vdash}\m{\,\mbox{\rm suc}}\m{\,A}\m{=}\m{(}\m{A}\m{\cup}\m{\{}\m{A}\m{\}}
\m{)}
\endm
\begin{mmraw}%
|- suc A = ( A u. \{ A \} ) \$.
\end{mmraw}

\noindent Define the class of natural numbers\index{natural number}\index{omega
($\omega$)}.  Compare Bell and Machover, p.~471.\label{dfom}

\vskip 0.5ex
\setbox\startprefix=\hbox{\tt \ \ df-om\ \$a\ }
\setbox\contprefix=\hbox{\tt \ \ \ \ \ \ \ \ \ \ \ }
\startm
\m{\vdash}\m{\omega}\m{=}\m{\{}\m{x}\m{|}\m{(}\m{\mbox{\rm Ord}}\m{\,x}\m{
\wedge}\m{\forall}\m{y}\m{(}\m{\mbox{\rm Lim}}\m{\,y}\m{\rightarrow}\m{x}\m{
\in}\m{y}\m{)}\m{)}\m{\}}
\endm
\begin{mmraw}%
|- om = \{ x | ( Ord x \TAND A. y ( Lim y -> x e. y ) ) \} \$.
\end{mmraw}

\noindent Define the Cartesian product (also called the
cross product)\index{Cartesian product}\index{cross product}
of two classes.  Definition 9.11 of Quine, p.~64.

\vskip 0.5ex
\setbox\startprefix=\hbox{\tt \ \ df-xp\ \$a\ }
\setbox\contprefix=\hbox{\tt \ \ \ \ \ \ \ \ \ \ \ }
\startm
\m{\vdash}\m{(}\m{A}\m{\times}\m{B}\m{)}\m{=}\m{\{}\m{\langle}\m{x}\m{,}\m{y}
\m{\rangle}\m{|}\m{(}\m{x}\m{\in}\m{A}\m{\wedge}\m{y}\m{\in}\m{B}\m{)}\m{\}}
\endm
\begin{mmraw}%
|- ( A X. B ) = \{ <. x , y >. | ( x e. A \TAND y e. B) \} \$.
\end{mmraw}

\noindent Define a relation\index{relation}.  Definition 6.4(1) of Takeuti and
Zaring, p.~23.

\vskip 0.5ex
\setbox\startprefix=\hbox{\tt \ \ df-rel\ \$a\ }
\setbox\contprefix=\hbox{\tt \ \ \ \ \ \ \ \ \ \ \ \ }
\startm
\m{\vdash}\m{(}\m{\mbox{\rm Rel}}\m{\,A}\m{\leftrightarrow}\m{A}\m{\subseteq}
\m{(}\m{{\rm V}}\m{\times}\m{{\rm V}}\m{)}\m{)}
\endm
\begin{mmraw}%
|- ( Rel A <-> A C\_ ( {\char`\_}V X. {\char`\_}V ) ) \$.
\end{mmraw}

\noindent Define the domain\index{domain} of a class.  Definition 6.5(1) of
Takeuti and Zaring, p.~24.

\vskip 0.5ex
\setbox\startprefix=\hbox{\tt \ \ df-dm\ \$a\ }
\setbox\contprefix=\hbox{\tt \ \ \ \ \ \ \ \ \ \ \ }
\startm
\m{\vdash}\m{\,\mbox{\rm dom}}\m{A}\m{=}\m{\{}\m{x}\m{|}\m{\exists}\m{y}\m{
\langle}\m{x}\m{,}\m{y}\m{\rangle}\m{\in}\m{A}\m{\}}
\endm
\begin{mmraw}%
|- dom A = \{ x | E. y <. x , y >. e. A \} \$.
\end{mmraw}

\noindent Define the range\index{range} of a class.  Definition 6.5(2) of
Takeuti and Zaring, p.~24.

\vskip 0.5ex
\setbox\startprefix=\hbox{\tt \ \ df-rn\ \$a\ }
\setbox\contprefix=\hbox{\tt \ \ \ \ \ \ \ \ \ \ \ }
\startm
\m{\vdash}\m{\,\mbox{\rm ran}}\m{A}\m{=}\m{\{}\m{y}\m{|}\m{\exists}\m{x}\m{
\langle}\m{x}\m{,}\m{y}\m{\rangle}\m{\in}\m{A}\m{\}}
\endm
\begin{mmraw}%
|- ran A = \{ y | E. x <. x , y >. e. A \} \$.
\end{mmraw}

\noindent Define the restriction\index{restriction} of a class.  Definition
6.6(1) of Takeuti and Zaring, p.~24.

\vskip 0.5ex
\setbox\startprefix=\hbox{\tt \ \ df-res\ \$a\ }
\setbox\contprefix=\hbox{\tt \ \ \ \ \ \ \ \ \ \ \ \ }
\startm
\m{\vdash}\m{(}\m{A}\m{\restriction}\m{B}\m{)}\m{=}\m{(}\m{A}\m{\cap}\m{(}\m{B}
\m{\times}\m{{\rm V}}\m{)}\m{)}
\endm
\begin{mmraw}%
|- ( A |` B ) = ( A i\^{}i ( B X. {\char`\_}V ) ) \$.
\end{mmraw}

\noindent Define the image\index{image} of a class.  Definition 6.6(2) of
Takeuti and Zaring, p.~24.

\vskip 0.5ex
\setbox\startprefix=\hbox{\tt \ \ df-ima\ \$a\ }
\setbox\contprefix=\hbox{\tt \ \ \ \ \ \ \ \ \ \ \ \ }
\startm
\m{\vdash}\m{(}\m{A}\m{``}\m{B}\m{)}\m{=}\m{\,\mbox{\rm ran}}\m{\,(}\m{A}\m{
\restriction}\m{B}\m{)}
\endm
\begin{mmraw}%
|- ( A " B ) = ran ( A |` B ) \$.
\end{mmraw}

\noindent Define the composition\index{composition} of two classes.  Definition 6.6(3) of
Takeuti and Zaring, p.~24.

\vskip 0.5ex
\setbox\startprefix=\hbox{\tt \ \ df-co\ \$a\ }
\setbox\contprefix=\hbox{\tt \ \ \ \ \ \ \ \ \ \ \ \ }
\startm
\m{\vdash}\m{(}\m{A}\m{\circ}\m{B}\m{)}\m{=}\m{\{}\m{\langle}\m{x}\m{,}\m{y}\m{
\rangle}\m{|}\m{\exists}\m{z}\m{(}\m{\langle}\m{x}\m{,}\m{z}\m{\rangle}\m{\in}
\m{B}\m{\wedge}\m{\langle}\m{z}\m{,}\m{y}\m{\rangle}\m{\in}\m{A}\m{)}\m{\}}
\endm
\begin{mmraw}%
|- ( A o. B ) = \{ <. x , y >. | E. z ( <. x , z
>. e. B \TAND <. z , y >. e. A ) \} \$.
\end{mmraw}

\noindent Define a function\index{function}.  Definition 6.4(4) of Takeuti and
Zaring, p.~24.

\vskip 0.5ex
\setbox\startprefix=\hbox{\tt \ \ df-fun\ \$a\ }
\setbox\contprefix=\hbox{\tt \ \ \ \ \ \ \ \ \ \ \ \ }
\startm
\m{\vdash}\m{(}\m{\mbox{\rm Fun}}\m{\,A}\m{\leftrightarrow}\m{(}
\m{\mbox{\rm Rel}}\m{\,A}\m{\wedge}
\m{\forall}\m{x}\m{\exists}\m{z}\m{\forall}\m{y}\m{(}
\m{\langle}\m{x}\m{,}\m{y}\m{\rangle}\m{\in}\m{A}\m{\rightarrow}\m{y}\m{=}\m{z}
\m{)}\m{)}\m{)}
\endm
\begin{mmraw}%
|- ( Fun A <-> ( Rel A /\ A. x E. z A. y ( <. x
   , y >. e. A -> y = z ) ) ) \$.
\end{mmraw}

\noindent Define a function with domain.  Definition 6.15(1) of Takeuti and
Zaring, p.~27.

\vskip 0.5ex
\setbox\startprefix=\hbox{\tt \ \ df-fn\ \$a\ }
\setbox\contprefix=\hbox{\tt \ \ \ \ \ \ \ \ \ \ \ }
\startm
\m{\vdash}\m{(}\m{A}\m{\,\mbox{\rm Fn}}\m{\,B}\m{\leftrightarrow}\m{(}
\m{\mbox{\rm Fun}}\m{\,A}\m{\wedge}\m{\mbox{\rm dom}}\m{\,A}\m{=}\m{B}\m{)}
\m{)}
\endm
\begin{mmraw}%
|- ( A Fn B <-> ( Fun A \TAND dom A = B ) ) \$.
\end{mmraw}

\noindent Define a function with domain and co-domain.  Definition 6.15(3)
of Takeuti and Zaring, p.~27.

\vskip 0.5ex
\setbox\startprefix=\hbox{\tt \ \ df-f\ \$a\ }
\setbox\contprefix=\hbox{\tt \ \ \ \ \ \ \ \ \ \ }
\startm
\m{\vdash}\m{(}\m{F}\m{:}\m{A}\m{\longrightarrow}\m{B}\m{
\leftrightarrow}\m{(}\m{F}\m{\,\mbox{\rm Fn}}\m{\,A}\m{\wedge}\m{
\mbox{\rm ran}}\m{\,F}\m{\subseteq}\m{B}\m{)}\m{)}
\endm
\begin{mmraw}%
|- ( F : A --> B <-> ( F Fn A \TAND ran F C\_ B ) ) \$.
\end{mmraw}

\noindent Define a one-to-one function\index{one-to-one function}.  Compare
Definition 6.15(5) of Takeuti and Zaring, p.~27.

\vskip 0.5ex
\setbox\startprefix=\hbox{\tt \ \ df-f1\ \$a\ }
\setbox\contprefix=\hbox{\tt \ \ \ \ \ \ \ \ \ \ \ }
\startm
\m{\vdash}\m{(}\m{F}\m{:}\m{A}\m{
\raisebox{.5ex}{${\textstyle{\:}_{\mbox{\footnotesize\rm
1\tt -\rm 1}}}\atop{\textstyle{
\longrightarrow}\atop{\textstyle{}^{\mbox{\footnotesize\rm {\ }}}}}$}
}\m{B}
\m{\leftrightarrow}\m{(}\m{F}\m{:}\m{A}\m{\longrightarrow}\m{B}
\m{\wedge}\m{\forall}\m{y}\m{\exists}\m{z}\m{\forall}\m{x}\m{(}\m{\langle}\m{x}
\m{,}\m{y}\m{\rangle}\m{\in}\m{F}\m{\rightarrow}\m{x}\m{=}\m{z}\m{)}\m{)}\m{)}
\endm
\begin{mmraw}%
|- ( F : A -1-1-> B <-> ( F : A --> B \TAND
   A. y E. z A. x ( <. x , y >. e. F -> x = z ) ) ) \$.
\end{mmraw}

\noindent Define an onto function\index{onto function}.  Definition 6.15(4) of Takeuti and
Zaring, p.~27.

\vskip 0.5ex
\setbox\startprefix=\hbox{\tt \ \ df-fo\ \$a\ }
\setbox\contprefix=\hbox{\tt \ \ \ \ \ \ \ \ \ \ \ }
\startm
\m{\vdash}\m{(}\m{F}\m{:}\m{A}\m{
\raisebox{.5ex}{${\textstyle{\:}_{\mbox{\footnotesize\rm
{\ }}}}\atop{\textstyle{
\longrightarrow}\atop{\textstyle{}^{\mbox{\footnotesize\rm onto}}}}$}
}\m{B}
\m{\leftrightarrow}\m{(}\m{F}\m{\,\mbox{\rm Fn}}\m{\,A}\m{\wedge}
\m{\mbox{\rm ran}}\m{\,F}\m{=}\m{B}\m{)}\m{)}
\endm
\begin{mmraw}%
|- ( F : A -onto-> B <-> ( F Fn A /\ ran F = B ) ) \$.
\end{mmraw}

\noindent Define a one-to-one, onto function.  Compare Definition 6.15(6) of
Takeuti and Zaring, p.~27.

\vskip 0.5ex
\setbox\startprefix=\hbox{\tt \ \ df-f1o\ \$a\ }
\setbox\contprefix=\hbox{\tt \ \ \ \ \ \ \ \ \ \ \ \ }
\startm
\m{\vdash}\m{(}\m{F}\m{:}\m{A}
\m{
\raisebox{.5ex}{${\textstyle{\:}_{\mbox{\footnotesize\rm
1\tt -\rm 1}}}\atop{\textstyle{
\longrightarrow}\atop{\textstyle{}^{\mbox{\footnotesize\rm onto}}}}$}
}
\m{B}
\m{\leftrightarrow}\m{(}\m{F}\m{:}\m{A}
\m{
\raisebox{.5ex}{${\textstyle{\:}_{\mbox{\footnotesize\rm
1\tt -\rm 1}}}\atop{\textstyle{
\longrightarrow}\atop{\textstyle{}^{\mbox{\footnotesize\rm {\ }}}}}$}
}
\m{B}\m{\wedge}\m{F}\m{:}\m{A}
\m{
\raisebox{.5ex}{${\textstyle{\:}_{\mbox{\footnotesize\rm
{\ }}}}\atop{\textstyle{
\longrightarrow}\atop{\textstyle{}^{\mbox{\footnotesize\rm onto}}}}$}
}
\m{B}\m{)}\m{)}
\endm
\begin{mmraw}%
|- ( F : A -1-1-onto-> B <-> ( F : A -1-1-> B? \TAND F : A -onto-> B ) ) \$.?
\end{mmraw}

\noindent Define the value of a function\index{function value}.  This
definition applies to any class and evaluates to the empty set when it is not
meaningful. Note that $ F`A$ means the same thing as the more familiar $ F(A)$
notation for a function's value at $A$.  The $ F`A$ notation is common in
formal set theory.\label{df-fv}

\vskip 0.5ex
\setbox\startprefix=\hbox{\tt \ \ df-fv\ \$a\ }
\setbox\contprefix=\hbox{\tt \ \ \ \ \ \ \ \ \ \ \ }
\startm
\m{\vdash}\m{(}\m{F}\m{`}\m{A}\m{)}\m{=}\m{\bigcup}\m{\{}\m{x}\m{|}\m{(}\m{F}%
\m{``}\m{\{}\m{A}\m{\}}\m{)}\m{=}\m{\{}\m{x}\m{\}}\m{\}}
\endm
\begin{mmraw}%
|- ( F ` A ) = U. \{ x | ( F " \{ A \} ) = \{ x \} \} \$.
\end{mmraw}

\noindent Define the result of an operation.\index{operation}  Here, $F$ is
     an operation on two
     values (such as $+$ for real numbers).   This is defined for proper
     classes $A$ and $B$ even though not meaningful in that case.  However,
     the definition can be meaningful when $F$ is a proper class.\label{dfopr}

\vskip 0.5ex
\setbox\startprefix=\hbox{\tt \ \ df-opr\ \$a\ }
\setbox\contprefix=\hbox{\tt \ \ \ \ \ \ \ \ \ \ \ \ }
\startm
\m{\vdash}\m{(}\m{A}\m{\,F}\m{\,B}\m{)}\m{=}\m{(}\m{F}\m{`}\m{\langle}\m{A}%
\m{,}\m{B}\m{\rangle}\m{)}
\endm
\begin{mmraw}%
|- ( A F B ) = ( F ` <. A , B >. ) \$.
\end{mmraw}

\section{Tricks of the Trade}\label{tricks}

In the \texttt{set.mm}\index{set theory database (\texttt{set.mm})} database our goal
was usually to conform to modern notation.  However in some cases the
relationship to standard textbook language may be obscured by several
unconventional devices we used to simplify the development and to take
advantage of the Metamath language.  In this section we will describe some
common conventions used in \texttt{set.mm}.

\begin{itemize}
\item
The turnstile\index{turnstile ({$\,\vdash$})} symbol, $\vdash$, meaning ``it
is provable that,'' is the first token of all assertions and hypotheses that
aren't syntax constructions.  This is a standard convention in logic.  (We
mentioned this earlier, but this symbol is bothersome to some people without a
logic background.  It has no deeper meaning but just provides us with a way to
distinguish syntax constructions from ordinary mathematical statements.)

\item
A hypothesis of the form

\vskip 1ex
\setbox\startprefix=\hbox{\tt \ \ \ \ \ \ \ \ \ \$e\ }
\setbox\contprefix=\hbox{\tt \ \ \ \ \ \ \ \ \ \ }
\startm
\m{\vdash}\m{(}\m{\varphi}\m{\rightarrow}\m{\forall}\m{x}\m{\varphi}\m{)}
\endm
\vskip 1ex

should be read ``assume variable $x$ is (effectively) not free in wff
$\varphi$.''\index{effectively not free}
Literally, this says ``assume it is provable that $\varphi \rightarrow \forall
x\, \varphi$.''  This device lets us avoid the complexities associated with
the standard treatment of free and bound variables.
%% Uncomment this when uncommenting section {formalspec} below
The footnote on p.~\pageref{effectivelybound} discusses this further.

\item
A statement of one of the forms

\vskip 1ex
\setbox\startprefix=\hbox{\tt \ \ \ \ \ \ \ \ \ \$a\ }
\setbox\contprefix=\hbox{\tt \ \ \ \ \ \ \ \ \ \ }
\startm
\m{\vdash}\m{(}\m{\lnot}\m{\forall}\m{x}\m{\,x}\m{=}\m{y}\m{\rightarrow}
\m{\ldots}\m{)}
\endm
\setbox\startprefix=\hbox{\tt \ \ \ \ \ \ \ \ \ \$p\ }
\setbox\contprefix=\hbox{\tt \ \ \ \ \ \ \ \ \ \ }
\startm
\m{\vdash}\m{(}\m{\lnot}\m{\forall}\m{x}\m{\,x}\m{=}\m{y}\m{\rightarrow}
\m{\ldots}\m{)}
\endm
\vskip 1ex

should be read ``if $x$ and $y$ are distinct variables, then...''  This
antecedent provides us with a technical device to avoid the need for the
\texttt{\$d} statement early in our development of predicate calculus,
permitting symbol manipulations to be as conceptually simple as those in
propositional calculus.  However, the \texttt{\$d} statement eventually
becomes a requirement, and after that this device is rarely used.

\item
The statement

\vskip 1ex
\setbox\startprefix=\hbox{\tt \ \ \ \ \ \ \ \ \ \$d\ }
\setbox\contprefix=\hbox{\tt \ \ \ \ \ \ \ \ \ \ }
\startm
\m{x}\m{\,y}
\endm
\vskip 1ex

should be read ``assume $x$ and $y$ are distinct variables.''

\item
The statement

\vskip 1ex
\setbox\startprefix=\hbox{\tt \ \ \ \ \ \ \ \ \ \$d\ }
\setbox\contprefix=\hbox{\tt \ \ \ \ \ \ \ \ \ \ }
\startm
\m{x}\m{\,\varphi}
\endm
\vskip 1ex

should be read ``assume $x$ does not occur in $\varphi$.''

\item
The statement

\vskip 1ex
\setbox\startprefix=\hbox{\tt \ \ \ \ \ \ \ \ \ \$d\ }
\setbox\contprefix=\hbox{\tt \ \ \ \ \ \ \ \ \ \ }
\startm
\m{x}\m{\,A}
\endm
\vskip 1ex

should be read ``assume variable $x$ does not occur in class $A$.''

\item
The restriction and hypothesis group

\vskip 1ex
\setbox\startprefix=\hbox{\tt \ \ \ \ \ \ \ \ \ \$d\ }
\setbox\contprefix=\hbox{\tt \ \ \ \ \ \ \ \ \ \ }
\startm
\m{x}\m{\,A}
\endm
\setbox\startprefix=\hbox{\tt \ \ \ \ \ \ \ \ \ \$d\ }
\setbox\contprefix=\hbox{\tt \ \ \ \ \ \ \ \ \ \ }
\startm
\m{x}\m{\,\psi}
\endm
\setbox\startprefix=\hbox{\tt \ \ \ \ \ \ \ \ \ \$e\ }
\setbox\contprefix=\hbox{\tt \ \ \ \ \ \ \ \ \ \ }
\startm
\m{\vdash}\m{(}\m{x}\m{=}\m{A}\m{\rightarrow}\m{(}\m{\varphi}\m{\leftrightarrow}
\m{\psi}\m{)}\m{)}
\endm
\vskip 1ex

is frequently used in place of explicit substitution, meaning ``assume
$\psi$ results from the proper substitution of $A$ for $x$ in
$\varphi$.''  Sometimes ``\texttt{\$e} $\vdash ( \psi \rightarrow
\forall x \, \psi )$'' is used instead of ``\texttt{\$d} $x\, \psi $,''
which requires only that $x$ be effectively not free in $\varphi$ but
not necessarily absent from it.  The use of implicit
substitution\index{substitution!implicit} is partly a matter of personal
style, although it may make proofs somewhat shorter than would be the
case with explicit substitution.

\item
The hypothesis


\vskip 1ex
\setbox\startprefix=\hbox{\tt \ \ \ \ \ \ \ \ \ \$e\ }
\setbox\contprefix=\hbox{\tt \ \ \ \ \ \ \ \ \ \ }
\startm
\m{\vdash}\m{A}\m{\in}\m{{\rm V}}
\endm
\vskip 1ex

should be read ``assume class $A$ is a set (i.e.\ exists).''
This is a convenient convention used by Quine.

\item
The restriction and hypothesis

\vskip 1ex
\setbox\startprefix=\hbox{\tt \ \ \ \ \ \ \ \ \ \$d\ }
\setbox\contprefix=\hbox{\tt \ \ \ \ \ \ \ \ \ \ }
\startm
\m{x}\m{\,y}
\endm
\setbox\startprefix=\hbox{\tt \ \ \ \ \ \ \ \ \ \$e\ }
\setbox\contprefix=\hbox{\tt \ \ \ \ \ \ \ \ \ \ }
\startm
\m{\vdash}\m{(}\m{y}\m{\in}\m{A}\m{\rightarrow}\m{\forall}\m{x}\m{\,y}
\m{\in}\m{A}\m{)}
\endm
\vskip 1ex

should be read ``assume variable $x$ is
(effectively) not free in class $A$.''

\end{itemize}

\section{A Theorem Sampler}\label{sometheorems}

In this section we list some of the more important theorems that are proved in
the \texttt{set.mm} database, and they illustrate the kinds of things that can be
done with Metamath.  While all of these facts are well-known results,
Metamath offers the advantage of easily allowing you to trace their
derivation back to axioms.  Our intent here is not to try to explain the
details or motivation; for this we refer you to the textbooks that are
mentioned in the descriptions.  (The \texttt{set.mm} file has bibliographic
references for the text references.)  Their proofs often embody important
concepts you may wish to explore with the Metamath program (see
Section~\ref{exploring}).  All the symbols that are used here are defined in
Section~\ref{hierarchy}.  For brevity we haven't included the \texttt{\$d}
restrictions or \texttt{\$f} hypotheses for these theorems; when you are
uncertain consult the \texttt{set.mm} database.

We start with \texttt{syl} (principle of the syllogism).
In \textit{Principia Mathematica}
Whitehead and Russell call this ``the principle of the
syllogism... because... the syllogism in Barbara is derived from them''
\cite[quote after Theorem *2.06 p.~101]{PM}.
Some authors call this law a ``hypothetical syllogism.''
As of 2019 \texttt{syl} is the most commonly referenced proven
assertion in the \texttt{set.mm} database.\footnote{
The Metamath program command \texttt{show usage}
shows the number of references.
On 2019-04-29 (commit 71cbbbdb387e) \texttt{syl} was directly referenced
10,819 times. The second most commonly referenced proven assertion
was \texttt{eqid}, which was directly referenced 7,738 times.
}

\vskip 2ex
\noindent Theorem syl (principle of the syllogism)\index{Syllogism}%
\index{\texttt{syl}}\label{syl}.

\vskip 0.5ex
\setbox\startprefix=\hbox{\tt \ \ syl.1\ \$e\ }
\setbox\contprefix=\hbox{\tt \ \ \ \ \ \ \ \ \ \ \ }
\startm
\m{\vdash}\m{(}\m{\varphi}\m{ \rightarrow }\m{\psi}\m{)}
\endm
\setbox\startprefix=\hbox{\tt \ \ syl.2\ \$e\ }
\setbox\contprefix=\hbox{\tt \ \ \ \ \ \ \ \ \ \ \ }
\startm
\m{\vdash}\m{(}\m{\psi}\m{ \rightarrow }\m{\chi}\m{)}
\endm
\setbox\startprefix=\hbox{\tt \ \ syl\ \$p\ }
\setbox\contprefix=\hbox{\tt \ \ \ \ \ \ \ \ \ }
\startm
\m{\vdash}\m{(}\m{\varphi}\m{ \rightarrow }\m{\chi}\m{)}
\endm
\vskip 2ex

The following theorem is not very deep but provides us with a notational device
that is frequently used.  It allows us to use the expression ``$A \in V$'' as
a compact way of saying that class $A$ exists, i.e.\ is a set.

\vskip 2ex
\noindent Two ways to say ``$A$ is a set'':  $A$ is a member of the universe
$V$ if and only if $A$ exists (i.e.\ there exists a set equal to $A$).
Theorem 6.9 of Quine, p. 43.

\vskip 0.5ex
\setbox\startprefix=\hbox{\tt \ \ isset\ \$p\ }
\setbox\contprefix=\hbox{\tt \ \ \ \ \ \ \ \ \ \ \ }
\startm
\m{\vdash}\m{(}\m{A}\m{\in}\m{{\rm V}}\m{\leftrightarrow}\m{\exists}\m{x}\m{\,x}\m{=}
\m{A}\m{)}
\endm
\vskip 1ex

Next we prove the axioms of standard ZF set theory that were missing from our
axiom system.  From our point of view they are theorems since they
can be derived from the other axioms.

\vskip 2ex
\noindent Axiom of Separation\index{Axiom of Separation}
(Aussonderung)\index{Aussonderung} proved from the other axioms of ZF set
theory.  Compare Exercise 4 of Takeuti and Zaring, p.~22.

\vskip 0.5ex
\setbox\startprefix=\hbox{\tt \ \ inex1.1\ \$e\ }
\setbox\contprefix=\hbox{\tt \ \ \ \ \ \ \ \ \ \ \ \ \ \ \ }
\startm
\m{\vdash}\m{A}\m{\in}\m{{\rm V}}
\endm
\setbox\startprefix=\hbox{\tt \ \ inex\ \$p\ }
\setbox\contprefix=\hbox{\tt \ \ \ \ \ \ \ \ \ \ \ \ \ }
\startm
\m{\vdash}\m{(}\m{A}\m{\cap}\m{B}\m{)}\m{\in}\m{{\rm V}}
\endm
\vskip 1ex

\noindent Axiom of the Null Set\index{Axiom of the Null Set} proved from the
other axioms of ZF set theory. Corollary 5.16 of Takeuti and Zaring, p.~20.

\vskip 0.5ex
\setbox\startprefix=\hbox{\tt \ \ 0ex\ \$p\ }
\setbox\contprefix=\hbox{\tt \ \ \ \ \ \ \ \ \ \ \ \ }
\startm
\m{\vdash}\m{\varnothing}\m{\in}\m{{\rm V}}
\endm
\vskip 1ex

\noindent The Axiom of Pairing\index{Axiom of Pairing} proved from the other
axioms of ZF set theory.  Theorem 7.13 of Quine, p.~51.
\vskip 0.5ex
\setbox\startprefix=\hbox{\tt \ \ prex\ \$p\ }
\setbox\contprefix=\hbox{\tt \ \ \ \ \ \ \ \ \ \ \ \ \ \ }
\startm
\m{\vdash}\m{\{}\m{A}\m{,}\m{B}\m{\}}\m{\in}\m{{\rm V}}
\endm
\vskip 2ex

Next we will list some famous or important theorems that are proved in
the \texttt{set.mm} database.  None of them except \texttt{omex}
require the Axiom of Infinity, as you can verify with the \texttt{show
trace{\char`\_}back} Metamath command.

\vskip 2ex
\noindent The resolution of Russell's paradox\index{Russell's paradox}.  There
exists no set containing the set of all sets which are not members of
themselves.  Proposition 4.14 of Takeuti and Zaring, p.~14.

\vskip 0.5ex
\setbox\startprefix=\hbox{\tt \ \ ru\ \$p\ }
\setbox\contprefix=\hbox{\tt \ \ \ \ \ \ \ \ }
\startm
\m{\vdash}\m{\lnot}\m{\exists}\m{x}\m{\,x}\m{=}\m{\{}\m{y}\m{|}\m{\lnot}\m{y}
\m{\in}\m{y}\m{\}}
\endm
\vskip 1ex

\noindent Cantor's theorem\index{Cantor's theorem}.  No set can be mapped onto
its power set.  Compare Theorem 6B(b) of Enderton, p.~132.

\vskip 0.5ex
\setbox\startprefix=\hbox{\tt \ \ canth.1\ \$e\ }
\setbox\contprefix=\hbox{\tt \ \ \ \ \ \ \ \ \ \ \ \ \ }
\startm
\m{\vdash}\m{A}\m{\in}\m{{\rm V}}
\endm
\setbox\startprefix=\hbox{\tt \ \ canth\ \$p\ }
\setbox\contprefix=\hbox{\tt \ \ \ \ \ \ \ \ \ \ \ }
\startm
\m{\vdash}\m{\lnot}\m{F}\m{:}\m{A}\m{\raisebox{.5ex}{${\textstyle{\:}_{
\mbox{\footnotesize\rm {\ }}}}\atop{\textstyle{\longrightarrow}\atop{
\textstyle{}^{\mbox{\footnotesize\rm onto}}}}$}}\m{{\cal P}}\m{A}
\endm
\vskip 1ex

\noindent The Burali-Forti paradox\index{Burali-Forti paradox}.  No set
contains all ordinal numbers. Enderton, p.~194.  (Burali-Forti was one person,
not two.)

\vskip 0.5ex
\setbox\startprefix=\hbox{\tt \ \ onprc\ \$p\ }
\setbox\contprefix=\hbox{\tt \ \ \ \ \ \ \ \ \ \ \ \ }
\startm
\m{\vdash}\m{\lnot}\m{\mbox{\rm On}}\m{\in}\m{{\rm V}}
\endm
\vskip 1ex

\noindent Peano's postulates\index{Peano's postulates} for arithmetic.
Proposition 7.30 of Takeuti and Zaring, pp.~42--43.  The objects being
described are the members of $\omega$ i.e.\ the natural numbers 0, 1,
2,\ldots.  The successor\index{successor} operation suc means ``plus
one.''  \texttt{peano1} says that 0 (which is defined as the empty set)
is a natural number.  \texttt{peano2} says that if $A$ is a natural
number, so is $A+1$.  \texttt{peano3} says that 0 is not the successor
of any natural number.  \texttt{peano4} says that two natural numbers
are equal if and only if their successors are equal.  \texttt{peano5} is
essentially the same as mathematical induction.

\vskip 1ex
\setbox\startprefix=\hbox{\tt \ \ peano1\ \$p\ }
\setbox\contprefix=\hbox{\tt \ \ \ \ \ \ \ \ \ \ \ \ }
\startm
\m{\vdash}\m{\varnothing}\m{\in}\m{\omega}
\endm
\vskip 1.5ex

\setbox\startprefix=\hbox{\tt \ \ peano2\ \$p\ }
\setbox\contprefix=\hbox{\tt \ \ \ \ \ \ \ \ \ \ \ \ }
\startm
\m{\vdash}\m{(}\m{A}\m{\in}\m{\omega}\m{\rightarrow}\m{{\rm suc}}\m{A}\m{\in}%
\m{\omega}\m{)}
\endm
\vskip 1.5ex

\setbox\startprefix=\hbox{\tt \ \ peano3\ \$p\ }
\setbox\contprefix=\hbox{\tt \ \ \ \ \ \ \ \ \ \ \ \ }
\startm
\m{\vdash}\m{(}\m{A}\m{\in}\m{\omega}\m{\rightarrow}\m{\lnot}\m{{\rm suc}}%
\m{A}\m{=}\m{\varnothing}\m{)}
\endm
\vskip 1.5ex

\setbox\startprefix=\hbox{\tt \ \ peano4\ \$p\ }
\setbox\contprefix=\hbox{\tt \ \ \ \ \ \ \ \ \ \ \ \ }
\startm
\m{\vdash}\m{(}\m{(}\m{A}\m{\in}\m{\omega}\m{\wedge}\m{B}\m{\in}\m{\omega}%
\m{)}\m{\rightarrow}\m{(}\m{{\rm suc}}\m{A}\m{=}\m{{\rm suc}}\m{B}\m{%
\leftrightarrow}\m{A}\m{=}\m{B}\m{)}\m{)}
\endm
\vskip 1.5ex

\setbox\startprefix=\hbox{\tt \ \ peano5\ \$p\ }
\setbox\contprefix=\hbox{\tt \ \ \ \ \ \ \ \ \ \ \ \ }
\startm
\m{\vdash}\m{(}\m{(}\m{\varnothing}\m{\in}\m{A}\m{\wedge}\m{\forall}\m{x}\m{%
\in}\m{\omega}\m{(}\m{x}\m{\in}\m{A}\m{\rightarrow}\m{{\rm suc}}\m{x}\m{\in}%
\m{A}\m{)}\m{)}\m{\rightarrow}\m{\omega}\m{\subseteq}\m{A}\m{)}
\endm
\vskip 1.5ex

\noindent Finite Induction (mathematical induction).\index{finite
induction}\index{mathematical induction} The first hypothesis is the
basis and the second is the induction hypothesis.  Theorem Schema 22 of
Suppes, p.~136.

\vskip 0.5ex
\setbox\startprefix=\hbox{\tt \ \ findes.1\ \$e\ }
\setbox\contprefix=\hbox{\tt \ \ \ \ \ \ \ \ \ \ \ \ \ \ }
\startm
\m{\vdash}\m{[}\m{\varnothing}\m{/}\m{x}\m{]}\m{\varphi}
\endm
\setbox\startprefix=\hbox{\tt \ \ findes.2\ \$e\ }
\setbox\contprefix=\hbox{\tt \ \ \ \ \ \ \ \ \ \ \ \ \ \ }
\startm
\m{\vdash}\m{(}\m{x}\m{\in}\m{\omega}\m{\rightarrow}\m{(}\m{\varphi}\m{%
\rightarrow}\m{[}\m{{\rm suc}}\m{x}\m{/}\m{x}\m{]}\m{\varphi}\m{)}\m{)}
\endm
\setbox\startprefix=\hbox{\tt \ \ findes\ \$p\ }
\setbox\contprefix=\hbox{\tt \ \ \ \ \ \ \ \ \ \ \ \ }
\startm
\m{\vdash}\m{(}\m{x}\m{\in}\m{\omega}\m{\rightarrow}\m{\varphi}\m{)}
\endm
\vskip 1ex

\noindent Transfinite Induction with explicit substitution.  The first
hypothesis is the basis, the second is the induction hypothesis for
successors, and the third is the induction hypothesis for limit
ordinals.  Theorem Schema 4 of Suppes, p. 197.

\vskip 0.5ex
\setbox\startprefix=\hbox{\tt \ \ tfindes.1\ \$e\ }
\setbox\contprefix=\hbox{\tt \ \ \ \ \ \ \ \ \ \ \ \ \ \ \ }
\startm
\m{\vdash}\m{[}\m{\varnothing}\m{/}\m{x}\m{]}\m{\varphi}
\endm
\setbox\startprefix=\hbox{\tt \ \ tfindes.2\ \$e\ }
\setbox\contprefix=\hbox{\tt \ \ \ \ \ \ \ \ \ \ \ \ \ \ \ }
\startm
\m{\vdash}\m{(}\m{x}\m{\in}\m{{\rm On}}\m{\rightarrow}\m{(}\m{\varphi}\m{%
\rightarrow}\m{[}\m{{\rm suc}}\m{x}\m{/}\m{x}\m{]}\m{\varphi}\m{)}\m{)}
\endm
\setbox\startprefix=\hbox{\tt \ \ tfindes.3\ \$e\ }
\setbox\contprefix=\hbox{\tt \ \ \ \ \ \ \ \ \ \ \ \ \ \ \ }
\startm
\m{\vdash}\m{(}\m{{\rm Lim}}\m{y}\m{\rightarrow}\m{(}\m{\forall}\m{x}\m{\in}%
\m{y}\m{\varphi}\m{\rightarrow}\m{[}\m{y}\m{/}\m{x}\m{]}\m{\varphi}\m{)}\m{)}
\endm
\setbox\startprefix=\hbox{\tt \ \ tfindes\ \$p\ }
\setbox\contprefix=\hbox{\tt \ \ \ \ \ \ \ \ \ \ \ \ \ }
\startm
\m{\vdash}\m{(}\m{x}\m{\in}\m{{\rm On}}\m{\rightarrow}\m{\varphi}\m{)}
\endm
\vskip 1ex

\noindent Principle of Transfinite Recursion.\index{transfinite
recursion} Theorem 7.41 of Takeuti and Zaring, p.~47.  Transfinite
recursion is the key theorem that allows arithmetic of ordinals to be
rigorously defined, and has many other important uses as well.
Hypotheses \texttt{tfr.1} and \texttt{tfr.2} specify a certain (proper)
class $ F$.  The complicated definition of $ F$ is not important in
itself; what is important is that there be such an $ F$ with the
required properties, and we show this by displaying $ F$ explicitly.
\texttt{tfr1} states that $ F$ is a function whose domain is the set of
ordinal numbers.  \texttt{tfr2} states that any value of $ F$ is
completely determined by its previous values and the values of an
auxiliary function, $G$.  \texttt{tfr3} states that $ F$ is unique,
i.e.\ it is the only function that satisfies \texttt{tfr1} and
\texttt{tfr2}.  Note that $ f$ is an individual variable like $x$ and
$y$; it is just a mnemonic to remind us that $A$ is a collection of
functions.

\vskip 0.5ex
\setbox\startprefix=\hbox{\tt \ \ tfr.1\ \$e\ }
\setbox\contprefix=\hbox{\tt \ \ \ \ \ \ \ \ \ \ \ }
\startm
\m{\vdash}\m{A}\m{=}\m{\{}\m{f}\m{|}\m{\exists}\m{x}\m{\in}\m{{\rm On}}\m{(}%
\m{f}\m{{\rm Fn}}\m{x}\m{\wedge}\m{\forall}\m{y}\m{\in}\m{x}\m{(}\m{f}\m{`}%
\m{y}\m{)}\m{=}\m{(}\m{G}\m{`}\m{(}\m{f}\m{\restriction}\m{y}\m{)}\m{)}\m{)}%
\m{\}}
\endm
\setbox\startprefix=\hbox{\tt \ \ tfr.2\ \$e\ }
\setbox\contprefix=\hbox{\tt \ \ \ \ \ \ \ \ \ \ \ }
\startm
\m{\vdash}\m{F}\m{=}\m{\bigcup}\m{A}
\endm
\setbox\startprefix=\hbox{\tt \ \ tfr1\ \$p\ }
\setbox\contprefix=\hbox{\tt \ \ \ \ \ \ \ \ \ \ }
\startm
\m{\vdash}\m{F}\m{{\rm Fn}}\m{{\rm On}}
\endm
\setbox\startprefix=\hbox{\tt \ \ tfr2\ \$p\ }
\setbox\contprefix=\hbox{\tt \ \ \ \ \ \ \ \ \ \ }
\startm
\m{\vdash}\m{(}\m{z}\m{\in}\m{{\rm On}}\m{\rightarrow}\m{(}\m{F}\m{`}\m{z}%
\m{)}\m{=}\m{(}\m{G}\m{`}\m{(}\m{F}\m{\restriction}\m{z}\m{)}\m{)}\m{)}
\endm
\setbox\startprefix=\hbox{\tt \ \ tfr3\ \$p\ }
\setbox\contprefix=\hbox{\tt \ \ \ \ \ \ \ \ \ \ }
\startm
\m{\vdash}\m{(}\m{(}\m{B}\m{{\rm Fn}}\m{{\rm On}}\m{\wedge}\m{\forall}\m{x}\m{%
\in}\m{{\rm On}}\m{(}\m{B}\m{`}\m{x}\m{)}\m{=}\m{(}\m{G}\m{`}\m{(}\m{B}\m{%
\restriction}\m{x}\m{)}\m{)}\m{)}\m{\rightarrow}\m{B}\m{=}\m{F}\m{)}
\endm
\vskip 1ex

\noindent The existence of omega (the class of natural numbers).\index{natural
number}\index{omega ($\omega$)}\index{Axiom of Infinity}  Axiom 7 of Takeuti
and Zaring, p.~43.  (This is the only theorem in this section requiring the
Axiom of Infinity.)

\vskip 0.5ex
\setbox\startprefix=\hbox{\tt \
\ omex\ \$p\ }
\setbox\contprefix=\hbox{\tt \ \ \ \ \ \ \ \ \ \ }
\startm
\m{\vdash}\m{\omega}\m{\in}\m{{\rm V}}
\endm
%\vskip 2ex


\section{Axioms for Real and Complex Numbers}\label{real}
\index{real and complex numbers!axioms for}

This section presents the axioms for real and complex numbers, along
with some commentary about them.  Analysis
textbooks implicitly or explicitly use these axioms or their equivalents
as their starting point.  In the database \texttt{set.mm}, we define real
and complex numbers as (rather complicated) specific sets and derive these
axioms as {\em theorems} from the axioms of ZF set theory, using a method
called Dedekind cuts.  We omit the details of this construction, which you can
follow if you wish using the \texttt{set.mm} database in conjunction with the
textbooks referenced therein.

Once we prove those theorems, we then restate these proven theorems as axioms.
This lets us easily identify which axioms are needed for a particular complex number proof, without the obfuscation of the set theory used to derive them.
As a result,
the construction is actually unimportant other
than to show that sets exist that satisfy the axioms, and thus that the axioms
are consistent if ZF set theory is consistent.  When working with real numbers
you can think of them as being the actual sets resulting
from the construction (for definiteness), or you can
think of them as otherwise unspecified sets that happen to satisfy the axioms.
The derivation is not easy, but the fact that it works is quite remarkable
and lends support to the idea that ZFC set theory is all we need to
provide a foundation for essentially all of mathematics.

\needspace{3\baselineskip}
\subsection{The Axioms for Real and Complex Numbers Themselves}\label{realactual}

For the axioms we are given (or postulate) 8 classes:  $\mathbb{C}$ (the
set of complex numbers), $\mathbb{R}$ (the set of real numbers, a subset
of $\mathbb{C}$), $0$ (zero), $1$ (one), $i$ (square root of
$-1$), $+$ (plus), $\cdot$ (times), and
$<_{\mathbb{R}}$ (less than for just the real numbers).
Subtraction and division are defined terms and are not part of the
axioms; for their definitions see \texttt{set.mm}.

Note that the notation $(A+B)$ (and similarly $(A\cdot B)$) specifies a class
called an {\em operation},\index{operation} and is the function value of the
class $+$ at ordered pair $\langle A,B \rangle$.  An operation is defined by
statement \texttt{df-opr} on p.~\pageref{dfopr}.
The notation $A <_{\mathbb{R}} B$ specifies a
wff called a {\em binary relation}\index{binary relation} and means $\langle A,B \rangle \in \,<_{\mathbb{R}}$, as defined by statement \texttt{df-br} on p.~\pageref{dfbr}.

Our set of 8 given classes is assumed to satisfy the following 22 axioms
(in the axioms listed below, $<$ really means $<_{\mathbb{R}}$).

\vskip 2ex

\noindent 1. The real numbers are a subset of the complex numbers.

%\vskip 0.5ex
\setbox\startprefix=\hbox{\tt \ \ ax-resscn\ \$p\ }
\setbox\contprefix=\hbox{\tt \ \ \ \ \ \ \ \ \ \ \ \ \ \ }
\startm
\m{\vdash}\m{\mathbb{R}}\m{\subseteq}\m{\mathbb{C}}
\endm
%\vskip 1ex

\noindent 2. One is a complex number.

%\vskip 0.5ex
\setbox\startprefix=\hbox{\tt \ \ ax-1cn\ \$p\ }
\setbox\contprefix=\hbox{\tt \ \ \ \ \ \ \ \ \ \ \ }
\startm
\m{\vdash}\m{1}\m{\in}\m{\mathbb{C}}
\endm
%\vskip 1ex

\noindent 3. The imaginary number $i$ is a complex number.

%\vskip 0.5ex
\setbox\startprefix=\hbox{\tt \ \ ax-icn\ \$p\ }
\setbox\contprefix=\hbox{\tt \ \ \ \ \ \ \ \ \ \ \ }
\startm
\m{\vdash}\m{i}\m{\in}\m{\mathbb{C}}
\endm
%\vskip 1ex

\noindent 4. Complex numbers are closed under addition.

%\vskip 0.5ex
\setbox\startprefix=\hbox{\tt \ \ ax-addcl\ \$p\ }
\setbox\contprefix=\hbox{\tt \ \ \ \ \ \ \ \ \ \ \ \ \ }
\startm
\m{\vdash}\m{(}\m{(}\m{A}\m{\in}\m{\mathbb{C}}\m{\wedge}\m{B}\m{\in}\m{\mathbb{C}}%
\m{)}\m{\rightarrow}\m{(}\m{A}\m{+}\m{B}\m{)}\m{\in}\m{\mathbb{C}}\m{)}
\endm
%\vskip 1ex

\noindent 5. Real numbers are closed under addition.

%\vskip 0.5ex
\setbox\startprefix=\hbox{\tt \ \ ax-addrcl\ \$p\ }
\setbox\contprefix=\hbox{\tt \ \ \ \ \ \ \ \ \ \ \ \ \ \ }
\startm
\m{\vdash}\m{(}\m{(}\m{A}\m{\in}\m{\mathbb{R}}\m{\wedge}\m{B}\m{\in}\m{\mathbb{R}}%
\m{)}\m{\rightarrow}\m{(}\m{A}\m{+}\m{B}\m{)}\m{\in}\m{\mathbb{R}}\m{)}
\endm
%\vskip 1ex

\noindent 6. Complex numbers are closed under multiplication.

%\vskip 0.5ex
\setbox\startprefix=\hbox{\tt \ \ ax-mulcl\ \$p\ }
\setbox\contprefix=\hbox{\tt \ \ \ \ \ \ \ \ \ \ \ \ \ }
\startm
\m{\vdash}\m{(}\m{(}\m{A}\m{\in}\m{\mathbb{C}}\m{\wedge}\m{B}\m{\in}\m{\mathbb{C}}%
\m{)}\m{\rightarrow}\m{(}\m{A}\m{\cdot}\m{B}\m{)}\m{\in}\m{\mathbb{C}}\m{)}
\endm
%\vskip 1ex

\noindent 7. Real numbers are closed under multiplication.

%\vskip 0.5ex
\setbox\startprefix=\hbox{\tt \ \ ax-mulrcl\ \$p\ }
\setbox\contprefix=\hbox{\tt \ \ \ \ \ \ \ \ \ \ \ \ \ \ }
\startm
\m{\vdash}\m{(}\m{(}\m{A}\m{\in}\m{\mathbb{R}}\m{\wedge}\m{B}\m{\in}\m{\mathbb{R}}%
\m{)}\m{\rightarrow}\m{(}\m{A}\m{\cdot}\m{B}\m{)}\m{\in}\m{\mathbb{R}}\m{)}
\endm
%\vskip 1ex

\noindent 8. Multiplication of complex numbers is commutative.

%\vskip 0.5ex
\setbox\startprefix=\hbox{\tt \ \ ax-mulcom\ \$p\ }
\setbox\contprefix=\hbox{\tt \ \ \ \ \ \ \ \ \ \ \ \ \ \ }
\startm
\m{\vdash}\m{(}\m{(}\m{A}\m{\in}\m{\mathbb{C}}\m{\wedge}\m{B}\m{\in}\m{\mathbb{C}}%
\m{)}\m{\rightarrow}\m{(}\m{A}\m{\cdot}\m{B}\m{)}\m{=}\m{(}\m{B}\m{\cdot}\m{A}%
\m{)}\m{)}
\endm
%\vskip 1ex

\noindent 9. Addition of complex numbers is associative.

%\vskip 0.5ex
\setbox\startprefix=\hbox{\tt \ \ ax-addass\ \$p\ }
\setbox\contprefix=\hbox{\tt \ \ \ \ \ \ \ \ \ \ \ \ \ \ }
\startm
\m{\vdash}\m{(}\m{(}\m{A}\m{\in}\m{\mathbb{C}}\m{\wedge}\m{B}\m{\in}\m{\mathbb{C}}%
\m{\wedge}\m{C}\m{\in}\m{\mathbb{C}}\m{)}\m{\rightarrow}\m{(}\m{(}\m{A}\m{+}%
\m{B}\m{)}\m{+}\m{C}\m{)}\m{=}\m{(}\m{A}\m{+}\m{(}\m{B}\m{+}\m{C}\m{)}\m{)}%
\m{)}
\endm
%\vskip 1ex

\noindent 10. Multiplication of complex numbers is associative.

%\vskip 0.5ex
\setbox\startprefix=\hbox{\tt \ \ ax-mulass\ \$p\ }
\setbox\contprefix=\hbox{\tt \ \ \ \ \ \ \ \ \ \ \ \ \ \ }
\startm
\m{\vdash}\m{(}\m{(}\m{A}\m{\in}\m{\mathbb{C}}\m{\wedge}\m{B}\m{\in}\m{\mathbb{C}}%
\m{\wedge}\m{C}\m{\in}\m{\mathbb{C}}\m{)}\m{\rightarrow}\m{(}\m{(}\m{A}\m{\cdot}%
\m{B}\m{)}\m{\cdot}\m{C}\m{)}\m{=}\m{(}\m{A}\m{\cdot}\m{(}\m{B}\m{\cdot}\m{C}%
\m{)}\m{)}\m{)}
\endm
%\vskip 1ex

\noindent 11. Multiplication distributes over addition for complex numbers.

%\vskip 0.5ex
\setbox\startprefix=\hbox{\tt \ \ ax-distr\ \$p\ }
\setbox\contprefix=\hbox{\tt \ \ \ \ \ \ \ \ \ \ \ \ \ }
\startm
\m{\vdash}\m{(}\m{(}\m{A}\m{\in}\m{\mathbb{C}}\m{\wedge}\m{B}\m{\in}\m{\mathbb{C}}%
\m{\wedge}\m{C}\m{\in}\m{\mathbb{C}}\m{)}\m{\rightarrow}\m{(}\m{A}\m{\cdot}\m{(}%
\m{B}\m{+}\m{C}\m{)}\m{)}\m{=}\m{(}\m{(}\m{A}\m{\cdot}\m{B}\m{)}\m{+}\m{(}%
\m{A}\m{\cdot}\m{C}\m{)}\m{)}\m{)}
\endm
%\vskip 1ex

\noindent 12. The square of $i$ equals $-1$ (expressed as $i$-squared plus 1 is
0).

%\vskip 0.5ex
\setbox\startprefix=\hbox{\tt \ \ ax-i2m1\ \$p\ }
\setbox\contprefix=\hbox{\tt \ \ \ \ \ \ \ \ \ \ \ \ }
\startm
\m{\vdash}\m{(}\m{(}\m{i}\m{\cdot}\m{i}\m{)}\m{+}\m{1}\m{)}\m{=}\m{0}
\endm
%\vskip 1ex

\noindent 13. One and zero are distinct.

%\vskip 0.5ex
\setbox\startprefix=\hbox{\tt \ \ ax-1ne0\ \$p\ }
\setbox\contprefix=\hbox{\tt \ \ \ \ \ \ \ \ \ \ \ \ }
\startm
\m{\vdash}\m{1}\m{\ne}\m{0}
\endm
%\vskip 1ex

\noindent 14. One is an identity element for real multiplication.

%\vskip 0.5ex
\setbox\startprefix=\hbox{\tt \ \ ax-1rid\ \$p\ }
\setbox\contprefix=\hbox{\tt \ \ \ \ \ \ \ \ \ \ \ }
\startm
\m{\vdash}\m{(}\m{A}\m{\in}\m{\mathbb{R}}\m{\rightarrow}\m{(}\m{A}\m{\cdot}\m{1}%
\m{)}\m{=}\m{A}\m{)}
\endm
%\vskip 1ex

\noindent 15. Every real number has a negative.

%\vskip 0.5ex
\setbox\startprefix=\hbox{\tt \ \ ax-rnegex\ \$p\ }
\setbox\contprefix=\hbox{\tt \ \ \ \ \ \ \ \ \ \ \ \ \ \ }
\startm
\m{\vdash}\m{(}\m{A}\m{\in}\m{\mathbb{R}}\m{\rightarrow}\m{\exists}\m{x}\m{\in}%
\m{\mathbb{R}}\m{(}\m{A}\m{+}\m{x}\m{)}\m{=}\m{0}\m{)}
\endm
%\vskip 1ex

\noindent 16. Every nonzero real number has a reciprocal.

%\vskip 0.5ex
\setbox\startprefix=\hbox{\tt \ \ ax-rrecex\ \$p\ }
\setbox\contprefix=\hbox{\tt \ \ \ \ \ \ \ \ \ \ \ \ \ \ }
\startm
\m{\vdash}\m{(}\m{A}\m{\in}\m{\mathbb{R}}\m{\rightarrow}\m{(}\m{A}\m{\ne}\m{0}%
\m{\rightarrow}\m{\exists}\m{x}\m{\in}\m{\mathbb{R}}\m{(}\m{A}\m{\cdot}%
\m{x}\m{)}\m{=}\m{1}\m{)}\m{)}
\endm
%\vskip 1ex

\noindent 17. A complex number can be expressed in terms of two reals.

%\vskip 0.5ex
\setbox\startprefix=\hbox{\tt \ \ ax-cnre\ \$p\ }
\setbox\contprefix=\hbox{\tt \ \ \ \ \ \ \ \ \ \ \ \ }
\startm
\m{\vdash}\m{(}\m{A}\m{\in}\m{\mathbb{C}}\m{\rightarrow}\m{\exists}\m{x}\m{\in}%
\m{\mathbb{R}}\m{\exists}\m{y}\m{\in}\m{\mathbb{R}}\m{A}\m{=}\m{(}\m{x}\m{+}\m{(}%
\m{y}\m{\cdot}\m{i}\m{)}\m{)}\m{)}
\endm
%\vskip 1ex

\noindent 18. Ordering on reals satisfies strict trichotomy.

%\vskip 0.5ex
\setbox\startprefix=\hbox{\tt \ \ ax-pre-lttri\ \$p\ }
\setbox\contprefix=\hbox{\tt \ \ \ \ \ \ \ \ \ \ \ \ \ }
\startm
\m{\vdash}\m{(}\m{(}\m{A}\m{\in}\m{\mathbb{R}}\m{\wedge}\m{B}\m{\in}\m{\mathbb{R}}%
\m{)}\m{\rightarrow}\m{(}\m{A}\m{<}\m{B}\m{\leftrightarrow}\m{\lnot}\m{(}\m{A}%
\m{=}\m{B}\m{\vee}\m{B}\m{<}\m{A}\m{)}\m{)}\m{)}
\endm
%\vskip 1ex

\noindent 19. Ordering on reals is transitive.

%\vskip 0.5ex
\setbox\startprefix=\hbox{\tt \ \ ax-pre-lttrn\ \$p\ }
\setbox\contprefix=\hbox{\tt \ \ \ \ \ \ \ \ \ \ \ \ \ }
\startm
\m{\vdash}\m{(}\m{(}\m{A}\m{\in}\m{\mathbb{R}}\m{\wedge}\m{B}\m{\in}\m{\mathbb{R}}%
\m{\wedge}\m{C}\m{\in}\m{\mathbb{R}}\m{)}\m{\rightarrow}\m{(}\m{(}\m{A}\m{<}%
\m{B}\m{\wedge}\m{B}\m{<}\m{C}\m{)}\m{\rightarrow}\m{A}\m{<}\m{C}\m{)}\m{)}
\endm
%\vskip 1ex

\noindent 20. Ordering on reals is preserved after addition to both sides.

%\vskip 0.5ex
\setbox\startprefix=\hbox{\tt \ \ ax-pre-ltadd\ \$p\ }
\setbox\contprefix=\hbox{\tt \ \ \ \ \ \ \ \ \ \ \ \ \ }
\startm
\m{\vdash}\m{(}\m{(}\m{A}\m{\in}\m{\mathbb{R}}\m{\wedge}\m{B}\m{\in}\m{\mathbb{R}}%
\m{\wedge}\m{C}\m{\in}\m{\mathbb{R}}\m{)}\m{\rightarrow}\m{(}\m{A}\m{<}\m{B}\m{%
\rightarrow}\m{(}\m{C}\m{+}\m{A}\m{)}\m{<}\m{(}\m{C}\m{+}\m{B}\m{)}\m{)}\m{)}
\endm
%\vskip 1ex

\noindent 21. The product of two positive reals is positive.

%\vskip 0.5ex
\setbox\startprefix=\hbox{\tt \ \ ax-pre-mulgt0\ \$p\ }
\setbox\contprefix=\hbox{\tt \ \ \ \ \ \ \ \ \ \ \ \ \ \ }
\startm
\m{\vdash}\m{(}\m{(}\m{A}\m{\in}\m{\mathbb{R}}\m{\wedge}\m{B}\m{\in}\m{\mathbb{R}}%
\m{)}\m{\rightarrow}\m{(}\m{(}\m{0}\m{<}\m{A}\m{\wedge}\m{0}%
\m{<}\m{B}\m{)}\m{\rightarrow}\m{0}\m{<}\m{(}\m{A}\m{\cdot}\m{B}\m{)}%
\m{)}\m{)}
\endm
%\vskip 1ex

\noindent 22. A non-empty, bounded-above set of reals has a supremum.

%\vskip 0.5ex
\setbox\startprefix=\hbox{\tt \ \ ax-pre-sup\ \$p\ }
\setbox\contprefix=\hbox{\tt \ \ \ \ \ \ \ \ \ \ \ }
\startm
\m{\vdash}\m{(}\m{(}\m{A}\m{\subseteq}\m{\mathbb{R}}\m{\wedge}\m{A}\m{\ne}\m{%
\varnothing}\m{\wedge}\m{\exists}\m{x}\m{\in}\m{\mathbb{R}}\m{\forall}\m{y}\m{%
\in}\m{A}\m{\,y}\m{<}\m{x}\m{)}\m{\rightarrow}\m{\exists}\m{x}\m{\in}\m{%
\mathbb{R}}\m{(}\m{\forall}\m{y}\m{\in}\m{A}\m{\lnot}\m{x}\m{<}\m{y}\m{\wedge}\m{%
\forall}\m{y}\m{\in}\m{\mathbb{R}}\m{(}\m{y}\m{<}\m{x}\m{\rightarrow}\m{\exists}%
\m{z}\m{\in}\m{A}\m{\,y}\m{<}\m{z}\m{)}\m{)}\m{)}
\endm

% NOTE: The \m{...} expressions above could be represented as
% $ \vdash ( ( A \subseteq \mathbb{R} \wedge A \ne \varnothing \wedge \exists x \in \mathbb{R} \forall y \in A \,y < x ) \rightarrow \exists x \in \mathbb{R} ( \forall y \in A \lnot x < y \wedge \forall y \in \mathbb{R} ( y < x \rightarrow \exists z \in A \,y < z ) ) ) $

\vskip 2ex

This completes the set of axioms for real and complex numbers.  You may
wish to look at how subtraction, division, and decimal numbers
are defined in \texttt{set.mm}, and for fun look at the proof of $2+
2 = 4$ (theorem \texttt{2p2e4} in \texttt{set.mm})
as discussed in section \ref{2p2e4}.

In \texttt{set.mm} we define the non-negative integers $\mathbb{N}$, the integers
$\mathbb{Z}$, and the rationals $\mathbb{Q}$ as subsets of $\mathbb{R}$.  This leads
to the nice inclusion $\mathbb{N} \subseteq \mathbb{Z} \subseteq \mathbb{Q} \subseteq
\mathbb{R} \subseteq \mathbb{C}$, giving us a uniform framework in which, for
example, a property such as commutativity of complex number addition
automatically applies to integers.  The natural numbers $\mathbb{N}$
are different from the set $\omega$ we defined earlier, but both satisfy
Peano's postulates.

\subsection{Complex Number Axioms in Analysis Texts}

Most analysis texts construct complex numbers as ordered pairs of reals,
leading to construction-dependent properties that satisfy these axioms
but are not stated in their pure form.  (This is also done in
\texttt{set.mm} but our axioms are extracted from that construction.)
Other texts will simply state that $\mathbb{R}$ is a ``complete ordered
subfield of $\mathbb{C}$,'' leading to redundant axioms when this phrase
is completely expanded out.  In fact I have not seen a text with the
axioms in the explicit form above.
None of these axioms is unique individually, but this carefully worked out
collection of axioms is the result of years of work
by the Metamath community.

\subsection{Eliminating Unnecessary Complex Number Axioms}

We once had more axioms for real and complex numbers, but over years of time
we (the Metamath community)
have found ways to eliminate them (by proving them from other axioms)
or weaken them (by making weaker claims without reducing what
can be proved).
In particular, here are statements that used to be complex number
axioms but have since been formally proven (with Metamath) to be redundant:

\begin{itemize}
\item
  $\mathbb{C} \in V$.
  At one time this was listed as a ``complex number axiom.''
  However, this is not properly speaking a complex number axiom,
  and in any case its proof uses axioms of set theory.
  Proven redundant by Mario Carneiro\index{Carneiro, Mario} on
  17-Nov-2014 (see \texttt{axcnex}).
\item
  $((A \in \mathbb{C} \land B \in \mathbb{C}$) $\rightarrow$
  $(A + B) = (B + A))$.
  Proved redundant by Eric Schmidt\index{Schmidt, Eric} on 19-Jun-2012,
  and formalized by Scott Fenton\index{Fenton, Scott} on 3-Jan-2013
  (see \texttt{addcom}).
\item
  $(A \in \mathbb{C} \rightarrow (A + 0) = A)$.
  Proved redundant by Eric Schmidt on 19-Jun-2012,
  and formalized by Scott Fenton on 3-Jan-2013
  (see \texttt{addid1}).
\item
  $(A \in \mathbb{C} \rightarrow \exists x \in \mathbb{C} (A + x) = 0)$.
  Proved redundant by Eric Schmidt and formalized on 21-May-2007
  (see \texttt{cnegex}).
\item
  $((A \in \mathbb{C} \land A \ne 0) \rightarrow \exists x \in \mathbb{C} (A \cdot x) = 1)$.
  Proved redundant by Eric Schmidt and formalized on 22-May-2007
  (see \texttt{recex}).
\item
  $0 \in \mathbb{R}$.
  Proved redundant by Eric Schmidt on 19-Feb-2005 and formalized 21-May-2007
  (see \texttt{0re}).
\end{itemize}

We could eliminate 0 as an axiomatic object by defining it as
$( ( i \cdot i ) + 1 )$
and replacing it with this expression throughout the axioms. If this
is done, axiom ax-i2m1 becomes redundant. However, the remaining axioms
would become longer and less intuitive.

Eric Schmidt's paper analyzing this axiom system \cite{Schmidt}
presented a proof that these remaining axioms,
with the possible exception of ax-mulcom, are independent of the others.
It is currently an open question if ax-mulcom is independent of the others.

\section{Two Plus Two Equals Four}\label{2p2e4}

Here is a proof that $2 + 2 = 4$, as proven in the theorem \texttt{2p2e4}
in the database \texttt{set.mm}.
This is a useful demonstration of what a Metamath proof can look like.
This proof may have more steps than you're used to, but each step is rigorously
proven all the way back to the axioms of logic and set theory.
This display was originally generated by the Metamath program
as an {\sc HTML} file.

In the table showing the proof ``Step'' is the sequential step number,
while its associated ``Expression'' is an expression that we have proved.
``Ref'' is the name of a theorem or axiom that justifies that expression,
and ``Hyp'' refers to previous steps (if any) that the theorem or axiom
needs so that we can use it.  Expressions are indented further than
the expressions that depend on them to show their interdependencies.

\begin{table}[!htbp]
\caption{Two plus two equals four}
\begin{tabular}{lllll}
\textbf{Step} & \textbf{Hyp} & \textbf{Ref} & \textbf{Expression} & \\
1  &       & df-2    & $ \; \; \vdash 2 = 1 + 1$  & \\
2  & 1     & oveq2i  & $ \; \vdash (2 + 2) = (2 + (1 + 1))$ & \\
3  &       & df-4    & $ \; \; \vdash 4 = (3 + 1)$ & \\
4  &       & df-3    & $ \; \; \; \vdash 3 = (2 + 1)$ & \\
5  & 4     & oveq1i  & $ \; \; \vdash (3 + 1) = ((2 + 1) + 1)$ & \\
6  &       & 2cn     & $ \; \; \; \vdash 2 \in \mathbb{C}$ & \\
7  &       & ax-1cn  & $ \; \; \; \vdash 1 \in \mathbb{C}$ & \\
8  & 6,7,7 & addassi & $ \; \; \vdash ((2 + 1) + 1) = (2 + (1 + 1))$ & \\
9  & 3,5,8 & 3eqtri  & $ \; \vdash 4 = (2 + (1 + 1))$ & \\
10 & 2,9   & eqtr4i  & $ \vdash (2 + 2) = 4$ & \\
\end{tabular}
\end{table}

Step 1 says that we can assert that $2 = 1 + 1$ because it is
justified by \texttt{df-2}.
What is \texttt{df-2}?
It is simply the definition of $2$, which in our system is defined as being
equal to $1 + 1$.  This shows how we can use definitions in proofs.

Look at Step 2 of the proof. In the Ref column, we see that it references
a previously proved theorem, \texttt{oveq2i}.
It turns out that
theorem \texttt{oveq2i} requires a
hypothesis, and in the Hyp column of Step 2 we indicate that Step 1 will
satisfy (match) this hypothesis.
If we looked at \texttt{oveq2i}
we would find that it proves that given some hypothesis
$A = B$, we can prove that $( C F A ) = ( C F B )$.
If we use \texttt{oveq2i} and apply step 1's result as the hypothesis,
that will mean that $A = 2$ and $B = ( 1 + 1 )$ within this use of
\texttt{oveq2i}.
We are free to select any value of $C$ and $F$ (subject to syntax constraints),
so we are free to select $C = 2$ and $F = +$,
producing our desired result,
$ (2 + 2) = (2 + (1 + 1))$.

Step 2 is an example of substitution.
In the end, every step in every proof uses only this one substitution rule.
All the rules of logic, and all the axioms, are expressed so that
they can be used via this one substitution rule.
So once you master substitution, you can master every Metamath proof,
no exceptions.

Each step is clear and can be immediately checked.
In the {\sc HTML} display you can even click on each reference to see why it is
justified, making it easy to see why the proof works.

\section{Deduction}\label{deduction}

Strictly speaking,
a deduction (also called an inference) is a kind of statement that needs
some hypotheses to be true in order for its conclusion to be true.
A theorem, on the other hand, has no hypotheses.
Informally we often call both of them theorems, but in this section we
will stick to the strict definitions.

It sometimes happens that we have proved a deduction of the form
$\varphi \Rightarrow \psi$\index{$\Rightarrow$}
(given hypothesis $\varphi$ we can prove $\psi$)
and we want to then prove a theorem of the form
$\varphi \rightarrow \psi$.

Converting a deduction (which uses a hypothesis) into a theorem
(which does not) is not as simple as you might think.
The deduction says, ``if we can prove $\varphi$ then we can prove $\psi$,''
which is in some sense weaker than saying
``$\varphi$ implies $\psi$.''
There is no axiom of logic that permits us to directly obtain the theorem
given the deduction.\footnote{
The conversion of a deduction to a theorem does not even hold in general
for quantum propositional calculus,
which is a weak subset of classical propositional calculus.
It has been shown that adding the Standard Deduction Theorem (discussed below)
to quantum propositional calculus turns it into classical
propositional calculus!
}

This is in contrast to going the other way.
If we have the theorem ($\varphi \rightarrow \psi$),
it is easy to recover the deduction
($\varphi \Rightarrow \psi$)
using modus ponens\index{modus ponens}
(\texttt{ax-mp}; see section \ref{axmp}).

In the following subsections we first discuss the standard deduction theorem
(the traditional but awkward way to convert deductions into theorems) and
the weak deduction theorem (a limited version of the standard deduction
theorem that is easier to use and was once widely used in
\texttt{set.mm}\index{set theory database (\texttt{set.mm})}).
In section \ref{deductionstyle} we discuss
deduction style, the newer approach we now recommend in most cases.
Deduction style uses ``deduction form,'' a form that
prefixes each hypothesis (other than definitions) and the
conclusion with a universal antecedent (``$\varphi \rightarrow$'').
Deduction style is widely used in \texttt{set.mm},
so it is useful to understand it and \textit{why} it is widely used.
Section \ref{naturaldeduction}
briefly discusses our approach for using natural deduction
within \texttt{set.mm},
as that approach is deeply related to deduction style.
We conclude with a summary of the strengths of
our approach, which we believe are compelling.

\subsection{The Standard Deduction Theorem}\label{standarddeductiontheorem}

It is possible to make use of information
contained in the deduction or its proof to assist us with the proof of
the related theorem.
In traditional logic books, there is a metatheorem called the
Deduction Theorem\index{Deduction Theorem}\index{Standard Deduction Theorem},
discovered independently by Herbrand and Tarski around 1930.
The Deduction Theorem, which we often call the Standard Deduction Theorem,
provides an algorithm for constructing a proof of a theorem from the
proof of its corresponding deduction. See, for example,
\cite[p.~56]{Margaris}\index{Margaris, Angelo}.
To construct a proof for a theorem, the
algorithm looks at each step in the proof of the original deduction and
rewrites the step with several steps wherein the hypothesis is eliminated
and becomes an antecedent.

In ordinary mathematics, no one actually carries out the algorithm,
because (in its most basic form) it involves an exponential explosion of
the number of proof steps as more hypotheses are eliminated. Instead,
the Standard Deduction Theorem is invoked simply to claim that it can
be done in principle, without actually doing it.
What's more, the algorithm is not as simple as it might first appear
when applying it rigorously.
There is a subtle restriction on the Standard Deduction Theorem
that must be taken into account involving the axiom of generalization
when working with predicate calculus (see the literature for more detail).

One of the goals of Metamath is to let you plainly see, with as few
underlying concepts as possible, how mathematics can be derived directly
from the axioms, and not indirectly according to some hidden rules
buried inside a program or understood only by logicians. If we added
the Standard Deduction Theorem to the language and proof verifier,
that would greatly complicate both and largely defeat Metamath's goal
of simplicity. In principle, we could show direct proofs by expanding
out the proof steps generated by the algorithm of the Standard Deduction
Theorem, but that is not feasible in practice because the number of proof
steps quickly becomes huge, even astronomical.
Since the algorithm of the Standard Deduction Theorem is driven by the proof,
we would have to go through that proof
all over again---starting from axioms---in order to obtain the theorem form.
In terms of proof length, there would be no savings over just
proving the theorem directly instead of first proving the deduction form.

\subsection{Weak Deduction Theorem}\label{weakdeductiontheorem}

We have developed
a more efficient method for proving a theorem from a deduction
that can be used instead of the Standard Deduction Theorem
in many (but not all) cases.
We call this more efficient method the
Weak Deduction Theorem\index{Weak Deduction Theorem}.\footnote{
There is also an unrelated ``Weak Deduction Theorem''
in the field of relevance logic, so to avoid confusion we could call
ours the ``Weak Deduction Theorem for Classical Logic.''}
Unlike the Standard Deduction Theorem, the Weak Deduction Theorem produces the
theorem directly from a special substitution instance of the deduction,
using a small, fixed number of steps roughly proportional to the length
of the final theorem.

If you come to a proof referencing the Weak Deduction Theorem
\texttt{dedth} (or one of its variants \texttt{dedthxx}),
here is how to follow the proof without getting into the details:
just click on the theorem referenced in the step
just before the reference to \texttt{dedth} and ignore everything else.
Theorem \texttt{dedth} simply turns a hypothesis into an antecedent
(i.e. the hypothesis followed by $\rightarrow$
is placed in front of the assertion, and the hypothesis
itself is eliminated) given certain conditions.

The Weak Deduction Theorem
eliminates a hypothesis $\varphi$, making it become an antecedent.
It does this by proving an expression
$ \varphi \rightarrow \psi $ given two hypotheses:
(1)
$ ( A = {\rm if} ( \varphi , A , B ) \rightarrow ( \varphi \leftrightarrow \chi ) ) $
and
(2) $\chi$.
Note that it requires that a proof exists for $\varphi$ when the class variable
$A$ is replaced with a specific class $B$. The hypothesis $\chi$
should be assigned to the inference.
You can see the details of the proof of the Weak Deduction Theorem
in theorem \texttt{dedth}.

The Weak Deduction Theorem
is probably easier to understand by studying proofs that make use of it.
For example, let's look at the proof of \texttt{renegcl}, which proves that
$ \vdash ( A \in \mathbb{R} \rightarrow - A \in \mathbb{R} )$:

\needspace{4\baselineskip}
\begin{longtabu} {l l l X}
\textbf{Step} & \textbf{Hyp} & \textbf{Ref} & \textbf{Expression} \\
  1 &  & negeq &
  $\vdash$ $($ $A$ $=$ ${\rm if}$ $($ $A$ $\in$ $\mathbb{R}$ $,$ $A$ $,$ $1$ $)$ $\rightarrow$
  $\textrm{-}$ $A$ $=$ $\textrm{-}$ ${\rm if}$ $($ $A$ $\in$ $\mathbb{R}$
  $,$ $A$ $,$ $1$ $)$ $)$ \\
 2 & 1 & eleq1d &
    $\vdash$ $($ $A$ $=$ ${\rm if}$ $($ $A$ $\in$ $\mathbb{R}$ $,$ $A$ $,$ $1$ $)$ $\rightarrow$ $($
    $\textrm{-}$ $A$ $\in$ $\mathbb{R}$ $\leftrightarrow$
    $\textrm{-}$ ${\rm if}$ $($ $A$ $\in$ $\mathbb{R}$ $,$ $A$ $,$ $1$ $)$ $\in$
    $\mathbb{R}$ $)$ $)$ \\
 3 &  & 1re & $\vdash 1 \in \mathbb{R}$ \\
 4 & 3 & elimel &
   $\vdash {\rm if} ( A \in \mathbb{R} , A , 1 ) \in \mathbb{R}$ \\
 5 & 4 & renegcli &
   $\vdash \textrm{-} {\rm if} ( A \in \mathbb{R} , A , 1 ) \in \mathbb{R}$ \\
 6 & 2,5 & dedth &
   $\vdash ( A \in \mathbb{R} \rightarrow \textrm{-} A \in \mathbb{R}$ ) \\
\end{longtabu}

The somewhat strange-looking steps in \texttt{renegcl} before step 5 are
technical stuff that makes this magic work, and they can be ignored
for a quick overview of the proof. To continue following the ``important''
part of the proof of \texttt{renegcl},
you can look at the reference to \texttt{renegcli} at step 5.

That said, let's briefly look at how
\texttt{renegcl} uses the
Weak Deduction Theorem (\texttt{dedth}) to do its job,
in case you want to do something similar or want understand it more deeply.
Let's work backwards in the proof of \texttt{renegcl}.
Step 6 applies \texttt{dedth} to produce our goal result
$ \vdash ( A \in \mathbb{R} \rightarrow\, - A \in \mathbb{R} )$.
This requires on the one hand the (substituted) deduction
\texttt{renegcli} in step 5.
By itself \texttt{renegcli} proves the deduction
$ \vdash A \in \mathbb{R} \Rightarrow\, \vdash - A \in \mathbb{R}$;
this is the deduction form we are trying to turn into theorem form,
and thus
\texttt{renegcli} has a separate hypothesis that must be fulfilled.
To fulfill the hypothesis of the invocation of
\texttt{renegcli} in step 5, it is eventually
reduced to the already proven theorem $1 \in \mathbb{R}$ in step 3.
Step 4 connects steps 3 and 5; step 4 invokes
\texttt{elimel}, a special case of \texttt{elimhyp} that eliminates
a membership hypothesis for the weak deduction theorem.
On the other hand, the equivalence of the conclusion of
\texttt{renegcl}
$( - A \in \mathbb{R} )$ and the substituted conclusion of
\texttt{renegcli} must be proven, which is done in steps 2 and 1.

The weak deduction theorem has limitations.
In particular, we must be able to prove a special case of the deduction's
hypothesis as a stand-alone theorem.
For example, we used $1 \in \mathbb{R}$ in step 3 of \texttt{renegcl}.

We used to use the weak deduction theorem
extensively within \texttt{set.mm}.
However, we now recommend applying ``deduction style''
instead in most cases, as deduction style is
often an easier and clearer approach.
Therefore, we will now describe deduction style.

\subsection{Deduction Style}\label{deductionstyle}

We now prefer to write assertions in ``deduction form''
instead of writing a proof that would require use of the standard or
weak deduction theorem.
We call this appraoch
``deduction style.''\index{deduction style}

It will be easier to explain this by first defining some terms:

\begin{itemize}
\item \textbf{closed form}\index{closed form}\index{forms!closed}:
A kind of assertion (theorem) with no hypotheses.
Typically its label has no special suffix.
An example is \texttt{unss}, which states:
$\vdash ( ( A \subseteq C \wedge B \subseteq C ) \leftrightarrow ( A \cup B )
\subseteq C )\label{eq:unss}$
\item \textbf{deduction form}\index{deduction form}\index{forms!deduction}:
A kind of assertion with one or more hypotheses
where the conclusion is an implication with
a wff variable as the antecedent (usually $\varphi$), and every hypothesis
(\$e statement)
is either (1) an implication with the same antecedent as the conclusion or
(2) a definition.
A definition
can be for a class variable (this is a class variable followed by ``='')
or a wff variable (this is a wff variable followed by $\leftrightarrow$);
class variable definitions are more common.
In practice, a proof
in deduction form will also contain many steps that are implications
where the antecedent is either that wff variable (normally $\varphi$)
or is
a conjunction (...$\land$...) including that wff variable ($\varphi$).
If an assertion is in deduction form, and other forms are also available,
then we suffix its label with ``d.''
An example is \texttt{unssd}, which states\footnote{
For brevity we show here (and in other places)
a $\&$\index{$\&$} between hypotheses\index{hypotheses}
and a $\Rightarrow$\index{$\Rightarrow$}\index{conclusion}
between the hypotheses and the conclusion.
This notation is technically not part of the Metamath language, but is
instead a convenient abbreviation to show both the hypotheses and conclusion.}:
$\vdash ( \varphi \rightarrow A \subseteq C )\quad\&\quad \vdash ( \varphi
    \rightarrow B \subseteq C )\quad\Rightarrow\quad \vdash ( \varphi
    \rightarrow ( A \cup B ) \subseteq C )\label{eq:unssd}$
\item \textbf{inference form}\index{inference form}\index{forms!inference}:
A kind of assertion with one or more hypotheses that is not in deduction form
(e.g., there is no common antecedent).
If an assertion is in inference form, and other forms are also available,
then we suffix its label with ``i.''
An example is \texttt{unssi}, which states:
$\vdash A \subseteq C\quad\&\quad \vdash B \subseteq C\quad\Rightarrow\quad
    \vdash ( A \cup B ) \subseteq C\label{eq:unssi}$
\end{itemize}

When using deduction style we express an assertion in deduction form.
This form prefixes each hypothesis (other than definitions) and the
conclusion with a universal antecedent (``$\varphi \rightarrow$'').
The antecedent (e.g., $\varphi$)
mimics the context handled in the deduction theorem, eliminating
the need to directly use the deduction theorem.

Once you have an assertion in deduction form, you can easily convert it
to inference form or closed form:

\begin{itemize}
\item To
prove some assertion Ti in inference form, given assertion Td in deduction
form, there is a simple mechanical process you can use. First take each
Ti hypothesis and insert a \texttt{T.} $\rightarrow$ prefix (``true implies'')
using \texttt{a1i}. You
can then use the existing assertion Td to prove the resulting conclusion
with a \texttt{T.} $\rightarrow$ prefix.
Finally, you can remove that prefix using \texttt{mptru},
resulting in the conclusion you wanted to prove.
\item To
prove some assertion T in closed form, given assertion Td in deduction
form, there is another simple mechanical process you can use. First,
select an expression that is the conjunction (...$\land$...) of all of the
consequents of every hypothesis of Td. Next, prove that this expression
implies each of the separate hypotheses of Td in turn by eliminating
conjuncts (there are a variety of proven assertions to do this, including
\texttt{simpl},
\texttt{simpr},
\texttt{3simpa},
\texttt{3simpb},
\texttt{3simpc},
\texttt{simp1},
\texttt{simp2},
and
\texttt{simp3}).
If the
expression has nested conjunctions, inner conjuncts can be broken out by
chaining the above theorems with \texttt{syl}
(see section \ref{syl}).\footnote{
There are actually many theorems
(labeled simp* such as \texttt{simp333}) that break out inner conjuncts in one
step, but rather than learning them you can just use the chaining we
just described to prove them, and then let the Metamath program command
\texttt{minimize{\char`\_}with}\index{\texttt{minimize{\char`\_}with} command}
figure out the right ones needed to collapse them.}
As your final step, you can then apply the already-proven assertion Td
(which is in deduction form), proving assertion T in closed form.
\end{itemize}

We can also easily convert any assertion T in closed form to its related
assertion Ti in inference form by applying
modus ponens\index{modus ponens} (see section \ref{axmp}).

The deduction form antecedent can also be used to represent the context
necessary to support natural deduction systems, so we will now
discuss natural deduction.

\subsection{Natural Deduction}\label{naturaldeduction}

Natural deduction\index{natural deduction}
(ND) systems, as such, were originally introduced in
1934 by two logicians working independently: Ja\'skowski and Gentzen. ND
systems are supposed to reconstruct, in a formally proper way, traditional
ways of mathematical reasoning (such as conditional proof, indirect proof,
and proof by cases). As reconstructions they were naturally influenced
by previous work, and many specific ND systems and notations have been
developed since their original work.

There are many ND variants, but
Indrzejczak \cite[p.~31-32]{Indrzejczak}\index{Indrzejczak, Andrzej}
suggests that any natural deductive system must satisfy at
least these three criteria:

\begin{itemize}
\item ``There are some means for entering assumptions into a proof and
also for eliminating them. Usually it requires some bookkeeping devices
for indicating the scope of an assumption, and showing that a part of
a proof depending on eliminated assumption is discharged.
\item There are no (or, at least, a very limited set of) axioms, because
their role is taken over by the set of primitive rules for introduction
and elimination of logical constants which means that elementary
inferences instead of formulae are taken as primitive.
\item (A genuine) ND system admits a lot of freedom in proof construction
and possibility of applying several proof search strategies, like
conditional proof, proof by cases, proof by reductio ad absurdum etc.''
\end{itemize}

The Metamath Proof Explorer (MPE) as defined in \texttt{set.mm}
is fundamentally a Hilbert-style system.
That is, MPE is based on a larger number of axioms (compared
to natural deduction systems), a very small set of rules of inference
(modus ponens), and the context is not changed by the rules of inference
in the middle of a proof. That said, MPE proofs can be developed using
the natural deduction (ND) approach as originally developed by Ja\'skowski
and Gentzen.

The most common and recommended approach for applying ND in MPE is to use
deduction form\index{deduction form}%
\index{forms!deduction}
and apply the MPE proven assertions that are equivalent to ND rules.
For example, MPE's \texttt{jca} is equivalent to ND rule $\land$-I
(and-insertion).
We maintain a list of equivalences that you may consult.
This approach for applying an ND approach within MPE relies on Metamath's
wff metavariables in an essential way, and is described in more detail
in the presentation ``Natural Deductions in the Metamath Proof Language''
by Mario Carneiro \cite{CarneiroND}\index{Carneiro, Mario}.

In this style many steps are an implication, whose antecedent mimics
the context ($\Gamma$) of most ND systems. To add an assumption, simply add
it to the implication antecedent (typically using
\texttt{simpr}),
and use that
new antecedent for all later claims in the same scope. If you wish to
use an assertion in an ND hypothesis scope that is outside the current
ND hypothesis scope, modify the assertion so that the ND hypothesis
assumption is added to its antecedent (typically using \texttt{adantr}). Most
proof steps will be proved using rules that have hypotheses and results
of the form $\varphi \rightarrow$ ...

An example may make this clearer.
Let's show theorem 5.5 of
\cite[p.~18]{Clemente}\index{Clemente Laboreo, Daniel}
along with a line by line translation using the usual
translation of natural deduction (ND) in the Metamath Proof Explorer
(MPE) notation (this is proof \texttt{ex-natded5.5}).
The proof's original goal was to prove
$\lnot \psi$ given two hypotheses,
$( \psi \rightarrow \chi )$ and $ \lnot \chi$.
We will translate these statements into MPE deduction form
by prefixing them all with $\varphi \rightarrow$.
As a result, in MPE the goal is stated as
$( \varphi \rightarrow \lnot \psi )$, and the two hypotheses are stated as
$( \varphi \rightarrow ( \psi \rightarrow \chi ) )$ and
$( \varphi \rightarrow \lnot \chi )$.

The following table shows the proof in Fitch natural deduction style
and its MPE equivalent.
The \textit{\#} column shows the original numbering,
\textit{MPE\#} shows the number in the equivalent MPE proof
(which we will show later),
\textit{ND Expression} shows the original proof claim in ND notation,
and \textit{MPE Translation} shows its translation into MPE
as discussed in this section.
The final columns show the rationale in ND and MPE respectively.

\needspace{4\baselineskip}
{\setlength{\extrarowsep}{4pt} % Keep rows from being too close together
\begin{longtabu}   { @{} c c X X X X }
\textbf{\#} & \textbf{MPE\#} & \textbf{ND Ex\-pres\-sion} &
\textbf{MPE Trans\-lation} & \textbf{ND Ration\-ale} &
\textbf{MPE Ra\-tio\-nale} \\
\endhead

1 & 2;3 &
$( \psi \rightarrow \chi )$ &
$( \varphi \rightarrow ( \psi \rightarrow \chi ) )$ &
Given &
\$e; \texttt{adantr} to put in ND hypothesis \\

2 & 5 &
$ \lnot \chi$ &
$( \varphi \rightarrow \lnot \chi )$ &
Given &
\$e; \texttt{adantr} to put in ND hypothesis \\

3 & 1 &
... $\vert$ $\psi$ &
$( \varphi \rightarrow \psi )$ &
ND hypothesis assumption &
\texttt{simpr} \\

4 & 4 &
... $\chi$ &
$( ( \varphi \land \psi ) \rightarrow \chi )$ &
$\rightarrow$\,E 1,3 &
\texttt{mpd} 1,3 \\

5 & 6 &
... $\lnot \chi$ &
$( ( \varphi \land \psi ) \rightarrow \lnot \chi )$ &
IT 2 &
\texttt{adantr} 5 \\

6 & 7 &
$\lnot \psi$ &
$( \varphi \rightarrow \lnot \psi )$ &
$\land$\,I 3,4,5 &
\texttt{pm2.65da} 4,6 \\

\end{longtabu}
}


The original used Latin letters; we have replaced them with Greek letters
to follow Metamath naming conventions and so that it is easier to follow
the Metamath translation. The Metamath line-for-line translation of
this natural deduction approach precedes every line with an antecedent
including $\varphi$ and uses the Metamath equivalents of the natural deduction
rules. To add an assumption, the antecedent is modified to include it
(typically by using \texttt{adantr};
\texttt{simpr} is useful when you want to
depend directly on the new assumption, as is shown here).

In Metamath we can represent the two given statements as these hypotheses:

\needspace{2\baselineskip}
\begin{itemize}
\item ex-natded5.5.1 $\vdash ( \varphi \rightarrow ( \psi \rightarrow \chi ) )$
\item ex-natded5.5.2 $\vdash ( \varphi \rightarrow \lnot \chi )$
\end{itemize}

\needspace{4\baselineskip}
Here is the proof in Metamath as a line-by-line translation:

\begin{longtabu}   { l l l X }
\textbf{Step} & \textbf{Hyp} & \textbf{Ref} & \textbf{Ex\-pres\-sion} \\
\endhead
1 & & simpr & $\vdash ( ( \varphi \land \psi ) \rightarrow \psi )$ \\
2 & & ex-natded5.5.1 &
  $\vdash ( \varphi \rightarrow ( \psi \rightarrow \chi ) )$ \\
3 & 2 & adantr &
 $\vdash ( ( \varphi \land \psi ) \rightarrow ( \psi \rightarrow \chi ) )$ \\
4 & 1, 3 & mpd &
 $\vdash ( ( \varphi \land \psi ) \rightarrow \chi ) $ \\
5 & & ex-natded5.5.2 &
 $\vdash ( \varphi \rightarrow \lnot \chi )$ \\
6 & 5 & adantr &
 $\vdash ( ( \varphi \land \psi ) \rightarrow \lnot \chi )$ \\
7 & 4, 6 & pm2.65da &
 $\vdash ( \varphi \rightarrow \lnot \psi )$ \\
\end{longtabu}

Only using specific natural deduction rules directly can lead to very
long proofs, for exactly the same reason that only using axioms directly
in Hilbert-style proofs can lead to very long proofs.
If the goal is short and clear proofs,
then it is better to reuse already-proven assertions
in deduction form than to start from scratch each time
and using only basic natural deduction rules.

\subsection{Strengths of Our Approach}

As far as we know there is nothing else in the literature like either the
weak deduction theorem or Mario Carneiro\index{Carneiro, Mario}'s
natural deduction method.
In order to
transform a hypothesis into an antecedent, the literature's standard
``Deduction Theorem''\index{Deduction Theorem}\index{Standard Deduction Theorem}
requires metalogic outside of the notions provided
by the axiom system. We instead generally prefer to use Mario Carneiro's
natural deduction method, then use the weak deduction theorem in cases
where that is difficult to apply, and only then use the full standard
deduction theorem as a last resort.

The weak deduction theorem\index{Weak Deduction Theorem}
does not require any additional metalogic
but converts an inference directly into a closed form theorem, with
a rigorous proof that uses only the axiom system. Unlike the standard
Deduction Theorem, there is no implicit external justification that we
have to trust in order to use it.

Mario Carneiro's natural deduction\index{natural deduction}
method also does not require any new metalogical
notions. It avoids the Deduction Theorem's metalogic by prefixing the
hypotheses and conclusion of every would-be inference with a universal
antecedent (``$\varphi \rightarrow$'') from the very start.

We think it is impressive and satisfying that we can do so much in a
practical sense without stepping outside of our Hilbert-style axiom system.
Of course our axiomatization, which is in the form of schemes,
contains a metalogic of its own that we exploit. But this metalogic
is relatively simple, and for our Deduction Theorem alternatives,
we primarily use just the direct substitution of expressions for
metavariables.

\begin{sloppy}
\section{Exploring the Set The\-o\-ry Data\-base}\label{exploring}
\end{sloppy}
% NOTE: All examples performed in this section are
% recorded wtih "set width 61" % on set.mm as of 2019-05-28
% commit c1e7849557661260f77cfdf0f97ac4354fbb4f4d.

At this point you may wish to study the \texttt{set.mm}\index{set theory
database (\texttt{set.mm})} file in more detail.  Pay particular
attention to the assumptions needed to define wffs\index{well-formed
formula (wff)} (which are not included above), the variable types
(\texttt{\$f}\index{\texttt{\$f} statement} statements), and the
definitions that are introduced.  Start with some simple theorems in
propositional calculus, making sure you understand in detail each step
of a proof.  Once you get past the first few proofs and become familiar
with the Metamath language, any part of the \texttt{set.mm} database
will be as easy to follow, step by step, as any other part---you won't
have to undergo a ``quantum leap'' in mathematical sophistication to be
able to follow a deep proof in set theory.

Next, you may want to explore how concepts such as natural numbers are
defined and described.  This is probably best done in conjunction with
standard set theory textbooks, which can help give you a higher-level
understanding.  The \texttt{set.mm} database provides references that will get
you started.  From there, you will be on your way towards a very deep,
rigorous understanding of abstract mathematics.

The Metamath\index{Metamath} program can help you peruse a Metamath data\-base,
wheth\-er you are trying to figure out how a certain step follows in a proof or
just have a general curiosity.  We will go through some examples of the
commands, using the \texttt{set.mm}\index{set theory database (\texttt{set.mm})}
database provided with the Metamath software.  These should help get you
started.  See Chapter~\ref{commands} for a more detailed description of
the commands.  Note that we have included the full spelling of all commands to
prevent ambiguity with future commands.  In practice you may type just the
characters needed to specify each command keyword\index{command keyword}
unambiguously, often just one or two characters per keyword, and you don't
need to type them in upper case.

First run the Metamath program as described earlier.  You should see the
\verb/MM>/ prompt.  Read in the \texttt{set.mm} file:\index{\texttt{read}
command}

\begin{verbatim}
MM> read set.mm
Reading source file "set.mm"... 34554442 bytes
34554442 bytes were read into the source buffer.
The source has 155711 statements; 2254 are $a and 32250 are $p.
No errors were found.  However, proofs were not checked.
Type VERIFY PROOF * if you want to check them.
\end{verbatim}

As with most examples in this book, what you will see
will be slightly different because we are continuously
improving our databases (including \texttt{set.mm}).

Let's check the database integrity.  This may take a minute or two to run if
your computer is slow.

\begin{verbatim}
MM> verify proof *
0 10%  20%  30%  40%  50%  60%  70%  80%  90% 100%
..................................................
All proofs in the database were verified in 2.84 s.
\end{verbatim}

No errors were reported, so every proof is correct.

You need to know the names (labels) of theorems before you can look at them.
Often just examining the database file(s) with a text editor is the best
approach.  In \texttt{set.mm} there are many detailed comments, especially near
the beginning, that can help guide you. The \texttt{search} command in the
Metamath program is also handy.  The \texttt{comments} qualifier will list the
statements whose associated comment (the one immediately before it) contain a
string you give it.  For example, if you are studying Enderton's {\em Elements
of Set Theory} \cite{Enderton}\index{Enderton, Herbert B.} you may want to see
the references to it in the database.  The search string \texttt{enderton} is not
case sensitive.  (This will not show you all the database theorems that are in
Enderton's book because there is usually only one citation for a given
theorem, which may appear in several textbooks.)\index{\texttt{search}
command}

\begin{verbatim}
MM> search * "enderton" / comments
12067 unineq $p "... Exercise 20 of [Enderton] p. 32 and ..."
12459 undif2 $p "...Corollary 6K of [Enderton] p. 144. (C..."
12953 df-tp $a "...s. Definition of [Enderton] p. 19. (Co..."
13689 unissb $p ".... Exercise 5 of [Enderton] p. 26 and ..."
\end{verbatim}
\begin{center}
(etc.)
\end{center}

Or you may want to see what theorems have something to do with
conjunction (logical {\sc and}).  The quotes around the search
string are optional when there's no ambiguity.\index{\texttt{search}
command}

\begin{verbatim}
MM> search * conjunction / comments
120 a1d $p "...be replaced with a conjunction ( ~ df-an )..."
662 df-bi $a "...viated form after conjunction is introdu..."
1319 wa $a "...ff definition to include conjunction ('and')."
1321 df-an $a "Define conjunction (logical 'and'). Defini..."
1420 imnan $p "...tion in terms of conjunction. (Contribu..."
\end{verbatim}
\begin{center}
(etc.)
\end{center}


Now we will start to look at some details.  Let's look at the first
axiom of propositional calculus
(we could use \texttt{sh st} to abbreviate
\texttt{show statement}).\index{\texttt{show statement} command}

\begin{verbatim}
MM> show statement ax-1/full
Statement 19 is located on line 881 of the file "set.mm".
"Axiom _Simp_.  Axiom A1 of [Margaris] p. 49.  One of the 3
axioms of propositional calculus.  The 3 axioms are also
        ...
19 ax-1 $a |- ( ph -> ( ps -> ph ) ) $.
Its mandatory hypotheses in RPN order are:
  wph $f wff ph $.
  wps $f wff ps $.
The statement and its hypotheses require the variables:  ph
      ps
The variables it contains are:  ph ps


Statement 49 is located on line 11182 of the file "set.mm".
Its statement number for HTML pages is 6.
"Axiom _Simp_.  Axiom A1 of [Margaris] p. 49.  One of the 3
axioms of propositional calculus.  The 3 axioms are also
given as Definition 2.1 of [Hamilton] p. 28.
...
49 ax-1 $a |- ( ph -> ( ps -> ph ) ) $.
Its mandatory hypotheses in RPN order are:
  wph $f wff ph $.
  wps $f wff ps $.
The statement and its hypotheses require the variables:
  ph ps
The variables it contains are:  ph ps
\end{verbatim}

Compare this to \texttt{ax-1} on p.~\pageref{ax1}.  You can see that the
symbol \texttt{ph} is the {\sc ascii} notation for $\varphi$, etc.  To
see the mathematical symbols for any expression you may typeset it in
\LaTeX\ (type \texttt{help tex} for instructions)\index{latex@{\LaTeX}}
or, easier, just use a text editor to look at the comments where symbols
are first introduced in \texttt{set.mm}.  The hypotheses \texttt{wph}
and \texttt{wps} required by \texttt{ax-1} mean that variables
\texttt{ph} and \texttt{ps} must be wffs.

Next we'll pick a simple theorem of propositional calculus, the Principle of
Identity, which is proved directly from the axioms.  We'll look at the
statement then its proof.\index{\texttt{show statement}
command}

\begin{verbatim}
MM> show statement id1/full
Statement 116 is located on line 11371 of the file "set.mm".
Its statement number for HTML pages is 22.
"Principle of identity.  Theorem *2.08 of [WhiteheadRussell]
p. 101.  This version is proved directly from the axioms for
demonstration purposes.
...
116 id1 $p |- ( ph -> ph ) $= ... $.
Its mandatory hypotheses in RPN order are:
  wph $f wff ph $.
Its optional hypotheses are:  wps wch wth wta wet
      wze wsi wrh wmu wla wka
The statement and its hypotheses require the variables:  ph
These additional variables are allowed in its proof:
      ps ch th ta et ze si rh mu la ka
The variables it contains are:  ph
\end{verbatim}

The optional variables\index{optional variable} \texttt{ps}, \texttt{ch}, etc.\ are
available for use in a proof of this statement if we wish, and were we to do
so we would make use of optional hypotheses \texttt{wps}, \texttt{wch}, etc.  (See
Section~\ref{dollaref} for the meaning of ``optional
hypothesis.''\index{optional hypothesis}) The reason these show up in the
statement display is that statement \texttt{id1} happens to be in their scope
(see Section~\ref{scoping} for the definition of ``scope''\index{scope}), but
in fact in propositional calculus we will never make use of optional
hypotheses or variables.  This becomes important after quantifiers are
introduced, where ``dummy'' variables\index{dummy variable} are often needed
in the middle of a proof.

Let's look at the proof of statement \texttt{id1}.  We'll use the
\texttt{show proof} command, which by default suppresses the
``non-essential'' steps that construct the wffs.\index{\texttt{show proof}
command}
We will display the proof in ``lemmon' format (a non-indented format
with explicit previous step number references) and renumber the
displayed steps:

\begin{verbatim}
MM> show proof id1 /lemmon/renumber
1 ax-1           $a |- ( ph -> ( ph -> ph ) )
2 ax-1           $a |- ( ph -> ( ( ph -> ph ) -> ph ) )
3 ax-2           $a |- ( ( ph -> ( ( ph -> ph ) -> ph ) ) ->
                     ( ( ph -> ( ph -> ph ) ) -> ( ph -> ph )
                                                          ) )
4 2,3 ax-mp      $a |- ( ( ph -> ( ph -> ph ) ) -> ( ph -> ph
                                                          ) )
5 1,4 ax-mp      $a |- ( ph -> ph )
\end{verbatim}

If you have read Section~\ref{trialrun}, you'll know how to interpret this
proof.  Step~2, for example, is an application of axiom \texttt{ax-1}.  This
proof is identical to the one in Hamilton's {\em Logic for Mathematicians}
\cite[p.~32]{Hamilton}\index{Hamilton, Alan G.}.

You may want to look at what
substitutions\index{substitution!variable}\index{variable substitution} are
made into \texttt{ax-1} to arrive at step~2.  The command to do this needs to
know the ``real'' step number, so we'll display the proof again without
the \texttt{renumber} qualifier.\index{\texttt{show proof}
command}

\begin{verbatim}
MM> show proof id1 /lemmon
 9 ax-1          $a |- ( ph -> ( ph -> ph ) )
20 ax-1          $a |- ( ph -> ( ( ph -> ph ) -> ph ) )
24 ax-2          $a |- ( ( ph -> ( ( ph -> ph ) -> ph ) ) ->
                     ( ( ph -> ( ph -> ph ) ) -> ( ph -> ph )
                                                          ) )
25 20,24 ax-mp   $a |- ( ( ph -> ( ph -> ph ) ) -> ( ph -> ph
                                                          ) )
26 9,25 ax-mp    $a |- ( ph -> ph )
\end{verbatim}

The ``real'' step number is 20.  Let's look at its details.

\begin{verbatim}
MM> show proof id1 /detailed_step 20
Proof step 20:  min=ax-1 $a |- ( ph -> ( ( ph -> ph ) -> ph )
  )
This step assigns source "ax-1" ($a) to target "min" ($e).
The source assertion requires the hypotheses "wph" ($f, step
18) and "wps" ($f, step 19).  The parent assertion of the
target hypothesis is "ax-mp" ($a, step 25).
The source assertion before substitution was:
    ax-1 $a |- ( ph -> ( ps -> ph ) )
The following substitutions were made to the source
assertion:
    Variable  Substituted with
     ph        ph
     ps        ( ph -> ph )
The target hypothesis before substitution was:
    min $e |- ph
The following substitution was made to the target hypothesis:
    Variable  Substituted with
     ph        ( ph -> ( ( ph -> ph ) -> ph ) )
\end{verbatim}

This shows the substitutions\index{substitution!variable}\index{variable
substitution} made to the variables in \texttt{ax-1}.  References are made to
steps 18 and 19 which are not shown in our proof display.  To see these steps,
you can display the proof with the \texttt{all} qualifier.

Let's look at a slightly more advanced proof of propositional calculus.  Note
that \verb+/\+ is the symbol for $\wedge$ (logical {\sc and}, also
called conjunction).\index{conjunction ($\wedge$)}
\index{logical {\sc and} ($\wedge$)}

\begin{verbatim}
MM> show statement prth/full
Statement 1791 is located on line 15503 of the file "set.mm".
Its statement number for HTML pages is 559.
"Conjoin antecedents and consequents of two premises.  This
is the closed theorem form of ~ anim12d .  Theorem *3.47 of
[WhiteheadRussell] p. 113.  It was proved by Leibniz,
and it evidently pleased him enough to call it
_praeclarum theorema_ (splendid theorem).
...
1791 prth $p |- ( ( ( ph -> ps ) /\ ( ch -> th ) ) -> ( ( ph
      /\ ch ) -> ( ps /\ th ) ) ) $= ... $.
Its mandatory hypotheses in RPN order are:
  wph $f wff ph $.
  wps $f wff ps $.
  wch $f wff ch $.
  wth $f wff th $.
Its optional hypotheses are:  wta wet wze wsi wrh wmu wla wka
The statement and its hypotheses require the variables:  ph
      ps ch th
These additional variables are allowed in its proof:  ta et
      ze si rh mu la ka
The variables it contains are:  ph ps ch th


MM> show proof prth /lemmon/renumber
1 simpl          $p |- ( ( ( ph -> ps ) /\ ( ch -> th ) ) ->
                                               ( ph -> ps ) )
2 simpr          $p |- ( ( ( ph -> ps ) /\ ( ch -> th ) ) ->
                                               ( ch -> th ) )
3 1,2 anim12d    $p |- ( ( ( ph -> ps ) /\ ( ch -> th ) ) ->
                           ( ( ph /\ ch ) -> ( ps /\ th ) ) )
\end{verbatim}

There are references to a lot of unfamiliar statements.  To see what they are,
you may type the following:

\begin{verbatim}
MM> show proof prth /statement_summary
Summary of statements used in the proof of "prth":

Statement simpl is located on line 14748 of the file
"set.mm".
"Elimination of a conjunct.  Theorem *3.26 (Simp) of
[WhiteheadRussell] p. 112. ..."
  simpl $p |- ( ( ph /\ ps ) -> ph ) $= ... $.

Statement simpr is located on line 14777 of the file
"set.mm".
"Elimination of a conjunct.  Theorem *3.27 (Simp) of
[WhiteheadRussell] ..."
  simpr $p |- ( ( ph /\ ps ) -> ps ) $= ... $.

Statement anim12d is located on line 15445 of the file
"set.mm".
"Conjoin antecedents and consequents in a deduction.
..."
  anim12d.1 $e |- ( ph -> ( ps -> ch ) ) $.
  anim12d.2 $e |- ( ph -> ( th -> ta ) ) $.
  anim12d $p |- ( ph -> ( ( ps /\ th ) -> ( ch /\ ta ) ) )
      $= ... $.
\end{verbatim}
\begin{center}
(etc.)
\end{center}

Of course you can look at each of these statements and their proofs, and
so on, back to the axioms of propositional calculus if you wish.

The \texttt{search} command is useful for finding statements when you
know all or part of their contents.  The following example finds all
statements containing \verb@ph -> ps@ followed by \verb@ch -> th@.  The
\verb@$*@ is a wildcard that matches anything; the \texttt{\$} before the
\verb$*$ prevents conflicts with math symbol token names.  The \verb@*@ after
\texttt{SEARCH} is also a wildcard that in this case means ``match any label.''
\index{\texttt{search} command}

% I'm omitting this one, since readers are unlikely to see it:
% 1096 bisymOLD $p |- ( ( ( ph -> ps ) -> ( ch -> th ) ) -> ( (
%   ( ps -> ph ) -> ( th -> ch ) ) -> ( ( ph <-> ps ) -> ( ch
%    <-> th ) ) ) )
\begin{verbatim}
MM> search * "ph -> ps $* ch -> th"
1791 prth $p |- ( ( ( ph -> ps ) /\ ( ch -> th ) ) -> ( ( ph
    /\ ch ) -> ( ps /\ th ) ) )
2455 pm3.48 $p |- ( ( ( ph -> ps ) /\ ( ch -> th ) ) -> ( (
    ph \/ ch ) -> ( ps \/ th ) ) )
117859 pm11.71 $p |- ( ( E. x ph /\ E. y ch ) -> ( ( A. x (
    ph -> ps ) /\ A. y ( ch -> th ) ) <-> A. x A. y ( ( ph /\
    ch ) -> ( ps /\ th ) ) ) )
\end{verbatim}

Three statements, \texttt{prth}, \texttt{pm3.48},
 and \texttt{pm11.71}, were found to match.

To see what axioms\index{axiom} and definitions\index{definition}
\texttt{prth} ultimately depends on for its proof, you can have the
program backtrack through the hierarchy\index{hierarchy} of theorems and
definitions.\index{\texttt{show trace{\char`\_}back} command}

\begin{verbatim}
MM> show trace_back prth /essential/axioms
Statement "prth" assumes the following axioms ($a
statements):
  ax-1 ax-2 ax-3 ax-mp df-bi df-an
\end{verbatim}

Note that the 3 axioms of propositional calculus and the modus ponens rule are
needed (as expected); in addition, there are a couple of definitions that are used
along the way.  Note that Metamath makes no distinction\index{axiom vs.\
definition} between axioms\index{axiom} and definitions\index{definition}.  In
\texttt{set.mm} they have been distinguished artificially by prefixing their
labels\index{labels in \texttt{set.mm}} with \texttt{ax-} and \texttt{df-}
respectively.  For example, \texttt{df-an} defines conjunction (logical {\sc
and}), which is represented by the symbol \verb+/\+.
Section~\ref{definitions} discusses the philosophy of definitions, and the
Metamath language takes a particularly simple, conservative approach by using
the \texttt{\$a}\index{\texttt{\$a} statement} statement for both axioms and
definitions.

You can also have the program compute how many steps a proof
has\index{proof length} if we were to follow it all the way back to
\texttt{\$a} statements.

\begin{verbatim}
MM> show trace_back prth /essential/count_steps
The statement's actual proof has 3 steps.  Backtracking, a
total of 79 different subtheorems are used.  The statement
and subtheorems have a total of 274 actual steps.  If
subtheorems used only once were eliminated, there would be a
total of 38 subtheorems, and the statement and subtheorems
would have a total of 185 steps.  The proof would have 28349
steps if fully expanded back to axiom references.  The
maximum path length is 38.  A longest path is:  prth <-
anim12d <- syl2and <- sylan2d <- ancomsd <- ancom <- pm3.22
<- pm3.21 <- pm3.2 <- ex <- sylbir <- biimpri <- bicomi <-
bicom1 <- bi2 <- dfbi1 <- impbii <- bi3 <- simprim <- impi <-
con1i <- nsyl2 <- mt3d <- con1d <- notnot1 <- con2i <- nsyl3
<- mt2d <- con2d <- notnot2 <- pm2.18d <- pm2.18 <- pm2.21 <-
pm2.21d <- a1d <- syl <- mpd <- a2i <- a2i.1 .
\end{verbatim}

This tells us that we would have to inspect 274 steps if we want to
verify the proof completely starting from the axioms.  A few more
statistics are also shown.  There are one or more paths back to axioms
that are the longest; this command ferrets out one of them and shows it
to you.  There may be a sense in which the longest path length is
related to how ``deep'' the theorem is.

We might also be curious about what proofs depend on the theorem
\texttt{prth}.  If it is never used later on, we could eliminate it as
redundant if it has no intrinsic interest by itself.\index{\texttt{show
usage} command}

% I decided to show the OLD values here.
\begin{verbatim}
MM> show usage prth
Statement "prth" is directly referenced in the proofs of 18
statements:
  mo3 moOLD 2mo 2moOLD euind reuind reuss2 reusv3i opelopabt
  wemaplem2 rexanre rlimcn2 o1of2 o1rlimmul 2sqlem6 spanuni
  heicant pm11.71
\end{verbatim}

Thus \texttt{prth} is directly used by 18 proofs.
We can use the \texttt{/recursive} qualifier to include indirect use:

\begin{verbatim}
MM> show usage prth /recursive
Statement "prth" directly or indirectly affects the proofs of
24214 statements:
  mo3 mo mo3OLD eu2 moOLD eu2OLD eu3OLD mo4f mo4 eu4 mopick
...
\end{verbatim}

\subsection{A Note on the ``Compact'' Proof Format}

The Metamath program will display proofs in a ``compact''\index{compact proof}
format whenever the proof is stored in compressed format in the database.  It
may be be slightly confusing unless you know how to interpret it.
For example,
if you display the complete proof of theorem \texttt{id1} it will start
off as follows:

\begin{verbatim}
MM> show proof id1 /lemmon/all
 1 wph           $f wff ph
 2 wph           $f wff ph
 3 wph           $f wff ph
 4 2,3 wi    @4: $a wff ( ph -> ph )
 5 1,4 wi    @5: $a wff ( ph -> ( ph -> ph ) )
 6 @4            $a wff ( ph -> ph )
\end{verbatim}

\begin{center}
{etc.}
\end{center}

Step 4 has a ``local label,''\index{local label} \texttt{@4}, assigned to it.
Later on, at step 6, the label \texttt{@1} is referenced instead of
displaying the explicit proof for that step.  This technique takes advantage
of the fact that steps in a proof often repeat, especially during the
construction of wffs.  The compact format reduces the number of steps in the
proof display and may be preferred by some people.

If you want to see the normal format with the ``true'' step numbers, you can
use the following workaround:\index{\texttt{save proof} command}

\begin{verbatim}
MM> save proof id1 /normal
The proof of "id1" has been reformatted and saved internally.
Remember to use WRITE SOURCE to save it permanently.
MM> show proof id1 /lemmon/all
 1 wph           $f wff ph
 2 wph           $f wff ph
 3 wph           $f wff ph
 4 2,3 wi        $a wff ( ph -> ph )
 5 1,4 wi        $a wff ( ph -> ( ph -> ph ) )
 6 wph           $f wff ph
 7 wph           $f wff ph
 8 6,7 wi        $a wff ( ph -> ph )
\end{verbatim}

\begin{center}
{etc.}
\end{center}

Note that the original 6 steps are now 8 steps.  However, the format is
now the same as that described in Chapter~\ref{using}.

\chapter{The Metamath Language}
\label{languagespec}

\begin{quote}
  {\em Thus mathematics may be defined as the subject in which we never know
what we are talking about, nor whether what we are saying is true.}
    \flushright\sc  Bertrand Russell\footnote{\cite[p.~84]{Russell2}.}\\
\end{quote}\index{Russell, Bertrand}

Probably the most striking feature of the Metamath language is its almost
complete absence of hard-wired syntax. Metamath\index{Metamath} does not
understand any mathematics or logic other than that needed to construct finite
sequences of symbols according to a small set of simple, built-in rules.  The
only rule it uses in a proof is the substitution of an expression (symbol
sequence) for a variable, subject to a simple constraint to prevent
bound-variable clashes.  The primitive notions built into Metamath involve the
simple manipulation of finite objects (symbols) that we as humans can easily
visualize and that computers can easily deal with.  They seem to be just
about the simplest notions possible that are required to do standard
mathematics.

This chapter serves as a reference manual for the Metamath\index{Metamath}
language. It covers the tedious technical details of the language, some of
which you may wish to skim in a first reading.  On the other hand, you should
pay close attention to the defined terms in {\bf boldface}; they have precise
meanings that are important to keep in mind for later understanding.  It may
be best to first become familiar with the examples in Chapter~\ref{using} to
gain some motivation for the language.

%% Uncomment this when uncommenting section {formalspec} below
If you have some knowledge of set theory, you may wish to study this
chapter in conjunction with the formal set-theoretical description of the
Metamath language in Appendix~\ref{formalspec}.

We will use the name ``Metamath''\index{Metamath} to mean either the Metamath
computer language or the Metamath software associated with the computer
language.  We will not distinguish these two when the context is clear.

The next section contains the complete specification of the Metamath
language.
It serves as an
authoritative reference and presents the syntax in enough detail to
write a parser\index{parsing Metamath} and proof verifier.  The
specification is terse and it is probably hard to learn the language
directly from it, but we include it here for those impatient people who
prefer to see everything up front before looking at verbose expository
material.  Later sections explain this material and provide examples.
We will repeat the definitions in those sections, and you may skip the
next section at first reading and proceed to Section~\ref{tut1}
(p.~\pageref{tut1}).

\section{Specification of the Metamath Language}\label{spec}
\index{Metamath!specification}

\begin{quote}
  {\em Sometimes one has to say difficult things, but one ought to say
them as simply as one knows how.}
    \flushright\sc  G. H. Hardy\footnote{As quoted in
    \cite{deMillo}, p.~273.}\\
\end{quote}\index{Hardy, G. H.}

\subsection{Preliminaries}\label{spec1}

% Space is technically a printable character, so we'll word things
% carefully so it's unambiguous.
A Metamath {\bf database}\index{database} is built up from a top-level source
file together with any source files that are brought in through file inclusion
commands (see below).  The only characters that are allowed to appear in a
Metamath source file are the 94 non-whitespace printable {\sc
ascii}\index{ascii@{\sc ascii}} characters, which are digits, upper and lower
case letters, and the following 32 special
characters\index{special characters}:\label{spec1chars}

\begin{verbatim}
! " # $ % & ' ( ) * + , - . / :
; < = > ? @ [ \ ] ^ _ ` { | } ~
\end{verbatim}
plus the following characters which are the ``white space'' characters:
space (a printable character),
tab, carriage return, line feed, and form feed.\label{whitespace}
We will use \texttt{typewriter}
font to display the printable characters.

A Metamath database consists of a sequence of three kinds of {\bf
tokens}\index{token} separated by {\bf white space}\index{white space}
(which is any sequence of one or more white space characters).  The set
of {\bf keyword}\index{keyword} tokens is \texttt{\$\char`\{},
\texttt{\$\char`\}}, \texttt{\$c}, \texttt{\$v}, \texttt{\$f},
\texttt{\$e}, \texttt{\$d}, \texttt{\$a}, \texttt{\$p}, \texttt{\$.},
\texttt{\$=}, \texttt{\$(}, \texttt{\$)}, \texttt{\$[}, and
\texttt{\$]}.  The last four are called {\bf auxiliary}\index{auxiliary
keyword} or preprocessing keywords.  A {\bf label}\index{label} token
consists of any combination of letters, digits, and the characters
hyphen, underscore, and period.  A {\bf math symbol}\index{math symbol}
token may consist of any combination of the 93 printable standard {\sc
ascii} characters other than space or \texttt{\$}~. All tokens are
case-sensitive.

\subsection{Preprocessing}

The token \texttt{\$(} begins a {\bf comment} and
\texttt{\$)} ends a comment.\index{\texttt{\$(}
and \texttt{\$)} auxiliary keywords}\index{comment}
Comments may contain any of
the 94 non-whitespace printable characters and white space,
except they may not contain the
2-character sequences \texttt{\$(} or \texttt{\$)} (comments do not nest).
Comments are ignored (treated
like white space) for the purpose of parsing, e.g.,
\texttt{\$( \$[ \$)} is a comment.
See p.~\pageref{mathcomments} for comment typesetting conventions; these
conventions may be ignored for the purpose of parsing.

A {\bf file inclusion command} consists of \texttt{\$[} followed by a file name
followed by \texttt{\$]}.\index{\texttt{\$[} and \texttt{\$]} auxiliary
keywords}\index{included file}\index{file inclusion}
It is only allowed in the outermost scope (i.e., not between
\texttt{\$\char`\{} and \texttt{\$\char`\}})
and must not be inside a statement (e.g., it may not occur
between the label of a \texttt{\$a} statement and its \texttt{\$.}).
The file name may not
contain a \texttt{\$} or white space.  The file must exist.
The case-sensitivity
of its name follows the conventions of the operating system.  The contents of
the file replace the inclusion command.
Included files may include other files.
Only the first reference to a given file is included; any later
references to the same file (whether in the top-level file or in included
files) cause the inclusion command to be ignored (treated like white space).
A verifier may assume that file names with different strings
refer to different files for the purpose of ignoring later references.
A file self-reference is ignored, as is any reference to the top-level file
(to avoid loops).
Included files may not include a \texttt{\$(} without a matching \texttt{\$)},
may not include a \texttt{\$[} without a matching \texttt{\$]}, and may
not include incomplete statements (e.g., a \texttt{\$a} without a matching
\texttt{\$.}).
It is currently unspecified if path references are relative to the process'
current directory or the file's containing directory, so databases should
avoid using pathname separators (e.g., ``/'') in file names.

Like all tokens, the \texttt{\$(}, \texttt{\$)}, \texttt{\$[}, and \texttt{\$]} keywords
must be surrounded by white space.

\subsection{Basic Syntax}

After preprocessing, a database will consist of a sequence of {\bf
statements}.
These are the scoping statements \texttt{\$\char`\{} and
\texttt{\$\char`\}}, along with the \texttt{\$c}, \texttt{\$v},
\texttt{\$f}, \texttt{\$e}, \texttt{\$d}, \texttt{\$a}, and \texttt{\$p}
statements.

A {\bf scoping statement}\index{scoping statement} consists only of its
keyword, \texttt{\$\char`\{} or \texttt{\$\char`\}}.
A \texttt{\$\char`\{} begins a {\bf
block}\index{block} and a matching \texttt{\$\char`\}} ends the block.
Every \texttt{\$\char`\{}
must have a matching \texttt{\$\char`\}}.
Defining it recursively, we say a block
contains a sequence of zero or more tokens other
than \texttt{\$\char`\{} and \texttt{\$\char`\}} and
possibly other blocks.  There is an {\bf outermost
block}\index{block!outermost} not bracketed by \texttt{\$\char`\{} \texttt{\$\char`\}}; the end
of the outermost block is the end of the database.

% LaTeX bug? can't do \bf\tt

A {\bf \$v} or {\bf \$c statement}\index{\texttt{\$v} statement}\index{\texttt{\$c}
statement} consists of the keyword token \texttt{\$v} or \texttt{\$c} respectively,
followed by one or more math symbols,
% The word "token" is used to distinguish "$." from the sentence-ending period.
followed by the \texttt{\$.}\ token.
These
statements {\bf declare}\index{declaration} the math symbols to be {\bf
variables}\index{variable!Metamath} or {\bf constants}\index{constant}
respectively. The same math symbol may not occur twice in a given \texttt{\$v} or
\texttt{\$c} statement.

%c%A math symbol becomes an {\bf active}\index{active math symbol}
%c%when declared and stays active until the end of the block in which it is
%c%declared.  A math symbol may not be declared a second time while it is active,
%c%but it may be declared again after it becomes inactive.

A math symbol becomes {\bf active}\index{active math symbol} when declared
and stays active until the end of the block in which it is declared.  A
variable may not be declared a second time while it is active, but it
may be declared again (as a variable, but not as a constant) after it
becomes inactive.  A constant must be declared in the outermost block and may
not be declared a second time.\index{redeclaration of symbols}

A {\bf \$f statement}\index{\texttt{\$f} statement} consists of a label,
followed by \texttt{\$f}, followed by its typecode (an active constant),
followed by an
active variable, followed by the \texttt{\$.}\ token.  A {\bf \$e
statement}\index{\texttt{\$e} statement} consists of a label, followed
by \texttt{\$e}, followed by its typecode (an active constant),
followed by zero or more
active math symbols, followed by the \texttt{\$.}\ token.  A {\bf
hypothesis}\index{hypothesis} is a \texttt{\$f} or \texttt{\$e}
statement.
The type declared by a \texttt{\$f} statement for a given label
is global even if the variable is not
(e.g., a database may not have \texttt{wff P} in one local scope
and \texttt{class P} in another).

A {\bf simple \$d statement}\index{\texttt{\$d} statement!simple}
consists of \texttt{\$d}, followed by two different active variables,
followed by the \texttt{\$.}\ token.  A {\bf compound \$d
statement}\index{\texttt{\$d} statement!compound} consists of
\texttt{\$d}, followed by three or more variables (all different),
followed by the \texttt{\$.}\ token.  The order of the variables in a
\texttt{\$d} statement is unimportant.  A compound \texttt{\$d}
statement is equivalent to a set of simple \texttt{\$d} statements, one
for each possible pair of variables occurring in the compound
\texttt{\$d} statement.  Henceforth in this specification we shall
assume all \texttt{\$d} statements are simple.  A \texttt{\$d} statement
is also called a {\bf disjoint} (or {\bf distinct}) {\bf variable
restriction}.\index{disjoint-variable restriction}

A {\bf \$a statement}\index{\texttt{\$a} statement} consists of a label,
followed by \texttt{\$a}, followed by its typecode (an active constant),
followed by
zero or more active math symbols, followed by the \texttt{\$.}\ token.  A {\bf
\$p statement}\index{\texttt{\$p} statement} consists of a label,
followed by \texttt{\$p}, followed by its typecode (an active constant),
followed by
zero or more active math symbols, followed by \texttt{\$=}, followed by
a sequence of labels, followed by the \texttt{\$.}\ token.  An {\bf
assertion}\index{assertion} is a \texttt{\$a} or \texttt{\$p} statement.

A \texttt{\$f}, \texttt{\$e}, or \texttt{\$d} statement is {\bf active}\index{active
statement} from the place it occurs until the end of the block it occurs in.
A \texttt{\$a} or \texttt{\$p} statement is {\bf active} from the place it occurs
through the end of the database.
There may not be two active \texttt{\$f} statements containing the same
variable.  Each variable in a \texttt{\$e}, \texttt{\$a}, or
\texttt{\$p} statement must exist in an active \texttt{\$f}
statement.\footnote{This requirement can greatly simplify the
unification algorithm (substitution calculation) required by proof
verification.}

%The label that begins each \texttt{\$f}, \texttt{\$e}, \texttt{\$a}, and
%\texttt{\$p} statement must be unique.
Each label token must be unique, and
no label token may match any math symbol
token.\label{namespace}\footnote{This
restriction was added on June 24, 2006.
It is not theoretically necessary but is imposed to make it easier to
write certain parsers.}

The set of {\bf mandatory variables}\index{mandatory variable} associated with
an assertion is the set of (zero or more) variables in the assertion and in any
active \texttt{\$e} statements.  The (possibly empty) set of {\bf mandatory
hypotheses}\index{mandatory hypothesis} is the set of all active \texttt{\$f}
statements containing mandatory variables, together with all active \texttt{\$e}
statements.
The set of {\bf mandatory {\bf \$d} statements}\index{mandatory
disjoint-variable restriction} associated with an assertion are those active
\texttt{\$d} statements whose variables are both among the assertion's
mandatory variables.

\subsection{Proof Verification}\label{spec4}

The sequence of labels between the \texttt{\$=} and \texttt{\$.}\ tokens
in a \texttt{\$p} statement is a {\bf proof}.\index{proof!Metamath} Each
label in a proof must be the label of an active statement other than the
\texttt{\$p} statement itself; thus a label must refer either to an
active hypothesis of the \texttt{\$p} statement or to an earlier
assertion.

An {\bf expression}\index{expression} is a sequence of math symbols. A {\bf
substitution map}\index{substitution map} associates a set of variables with a
set of expressions.  It is acceptable for a variable to be mapped to an
expression containing it.  A {\bf
substitution}\index{substitution!variable}\index{variable substitution} is the
simultaneous replacement of all variables in one or more expressions with the
expressions that the variables map to.

A proof is scanned in order of its label sequence.  If the label refers to an
active hypothesis, the expression in the hypothesis is pushed onto a
stack.\index{stack}\index{RPN stack}  If the label refers to an assertion, a
(unique) substitution must exist that, when made to the mandatory hypotheses
of the referenced assertion, causes them to match the topmost (i.e.\ most
recent) entries of the stack, in order of occurrence of the mandatory
hypotheses, with the topmost stack entry matching the last mandatory
hypothesis of the referenced assertion.  As many stack entries as there are
mandatory hypotheses are then popped from the stack.  The same substitution is
made to the referenced assertion, and the result is pushed onto the stack.
After the last label in the proof is processed, the stack must have a single
entry that matches the expression in the \texttt{\$p} statement containing the
proof.

%c%{\footnotesize\begin{quotation}\index{redeclaration of symbols}
%c%{{\em Comment.}\label{spec4comment} Whenever a math symbol token occurs in a
%c%{\texttt{\$c} or \texttt{\$v} statement, it is considered to designate a distinct new
%c%{symbol, even if the same token was previously declared (and is now inactive).
%c%{Thus a math token declared as a constant in two different blocks is considered
%c%{to designate two distinct constants (even though they have the same name).
%c%{The two constants will not match in a proof that references both blocks.
%c%{However, a proof referencing both blocks is acceptable as long as it doesn't
%c%{require that the constants match.  Similarly, a token declared to be a
%c%{constant for a referenced assertion will not match the same token declared to
%c%{be a variable for the \texttt{\$p} statement containing the proof.  In the case
%c%{of a token declared to be a variable for a referenced assertion, this is not
%c%{an issue since the variable can be substituted with whatever expression is
%c%{needed to achieve the required match.
%c%{\end{quotation}}
%c2%A proof may reference an assertion that contains or whose hypotheses contain a
%c2%constant that is not active for the \texttt{\$p} statement containing the proof.
%c2%However, the final result of the proof may not contain that constant. A proof
%c2%may also reference an assertion that contains or whose hypotheses contain a
%c2%variable that is not active for the \texttt{\$p} statement containing the proof.
%c2%That variable, of course, will be substituted with whatever expression is
%c2%needed to achieve the required match.

A proof may contain a \texttt{?}\ in place of a label to indicate an unknown step
(Section~\ref{unknown}).  A proof verifier may ignore any proof containing
\texttt{?}\ but should warn the user that the proof is incomplete.

A {\bf compressed proof}\index{compressed proof}\index{proof!compressed} is an
alternate proof notation described in Appen\-dix~\ref{compressed}; also see
references to ``compressed proof'' in the Index.  Compressed proofs are a
Metamath language extension which a complete proof verifier should be able to
parse and verify.

\subsubsection{Verifying Disjoint Variable Restrictions}

Each substitution made in a proof must be checked to verify that any
disjoint variable restrictions are satisfied, as follows.

If two variables replaced by a substitution exist in a mandatory \texttt{\$d}
statement\index{\texttt{\$d} statement} of the assertion referenced, the two
expressions resulting from the substitution must satisfy the following
conditions.  First, the two expressions must have no variables in common.
Second, each possible pair of variables, one from each expression, must exist
in an active \texttt{\$d} statement of the \texttt{\$p} statement containing the
proof.

\vskip 1ex

This ends the specification of the Metamath language;
see Appendix \ref{BNF} for its syntax in
Extended Backus--Naur Form (EBNF)\index{Extended Backus--Naur Form}\index{EBNF}.

\section{The Basic Keywords}\label{tut1}

Our expository material begins here.

Like most computer languages, Metamath\index{Metamath} takes its input from
one or more {\bf source files}\index{source file} which contain characters
expressed in the standard {\sc ascii} (American Standard Code for Information
Interchange)\index{ascii@{\sc ascii}} code for computers.  A source file
consists of a series of {\bf tokens}\index{token}, which are strings of
non-whitespace
printable characters (from the set of 94 shown on p.~\pageref{spec1chars})
separated by {\bf white space}\index{white space} (spaces, tabs, carriage
returns, line feeds, and form feeds). Any string consisting only of these
characters is treated the same as a single space.  The non-whitespace printable
characters\index{printable character} that Metamath recognizes are the 94
characters on standard {\sc ascii} keyboards.

Metamath has the ability to join several files together to form its
input (Section~\ref{include}).  We call the aggregate contents of all
the files after they have been joined together a {\bf
database}\index{database} to distinguish it from an individual source
file.  The tokens in a database consist of {\bf
keywords}\index{keyword}, which are built into the language, together
with two kinds of user-defined tokens called {\bf labels}\index{label}
and {\bf math symbols}\index{math symbol}.  (Often we will simply say
{\bf symbol}\index{symbol} instead of math symbol for brevity).  The set
of {\bf basic keywords}\index{basic keyword} is
\texttt{\$c}\index{\texttt{\$c} statement},
\texttt{\$v}\index{\texttt{\$v} statement},
\texttt{\$e}\index{\texttt{\$e} statement},
\texttt{\$f}\index{\texttt{\$f} statement},
\texttt{\$d}\index{\texttt{\$d} statement},
\texttt{\$a}\index{\texttt{\$a} statement},
\texttt{\$p}\index{\texttt{\$p} statement},
\texttt{\$=}\index{\texttt{\$=} keyword},
\texttt{\$.}\index{\texttt{\$.}\ keyword},
\texttt{\$\char`\{}\index{\texttt{\$\char`\{} and \texttt{\$\char`\}}
keywords}, and \texttt{\$\char`\}}.  This is the complete set of
syntactical elements of what we call the {\bf basic
language}\index{basic language} of Metamath, and with them you can
express all of the mathematics that were intended by the design of
Metamath.  You should make it a point to become very familiar with them.
Table~\ref{basickeywords} lists the basic keywords along with a brief
description of their functions.  For now, this description will give you
only a vague notion of what the keywords are for; later we will describe
the keywords in detail.


\begin{table}[htp] \caption{Summary of the basic Metamath
keywords} \label{basickeywords}
\begin{center}
\begin{tabular}{|p{4pc}|l|}
\hline
\em \centering Keyword&\em Description\\
\hline
\hline
\centering
   \texttt{\$c}&Constant symbol declaration\\
\hline
\centering
   \texttt{\$v}&Variable symbol declaration\\
\hline
\centering
   \texttt{\$d}&Disjoint variable restriction\\
\hline
\centering
   \texttt{\$f}&Variable-type (``floating'') hypothesis\\
\hline
\centering
   \texttt{\$e}&Logical (``essential'') hypothesis\\
\hline
\centering
   \texttt{\$a}&Axiomatic assertion\\
\hline
\centering
   \texttt{\$p}&Provable assertion\\
\hline
\centering
   \texttt{\$=}&Start of proof in \texttt{\$p} statement\\
\hline
\centering
   \texttt{\$.}&End of the above statement types\\
\hline
\centering
   \texttt{\$\char`\{}&Start of block\\
\hline
\centering
   \texttt{\$\char`\}}&End of block\\
\hline
\end{tabular}
\end{center}
\end{table}

%For LaTeX bug(?) where it puts tables on blank page instead of btwn text
%May have to adjust if text changes
%\newpage

There are some additional keywords, called {\bf auxiliary
keywords}\index{auxiliary keyword} that help make Metamath\index{Metamath}
more practical. These are part of the {\bf extended language}\index{extended
language}. They provide you with a means to put comments into a Metamath
source file\index{source file} and reference other source files.  We will
introduce these in later sections. Table~\ref{otherkeywords} summarizes them
so that you can recognize them now if you want to peruse some source
files while learning the basic keywords.


\begin{table}[htp] \caption{Auxiliary Metamath
keywords} \label{otherkeywords}
\begin{center}
\begin{tabular}{|p{4pc}|l|}
\hline
\em \centering Keyword&\em Description\\
\hline
\hline
\centering
   \texttt{\$(}&Start of comment\\
\hline
\centering
   \texttt{\$)}&End of comment\\
\hline
\centering
   \texttt{\$[}&Start of included source file name\\
\hline
\centering
   \texttt{\$]}&End of included source file name\\
\hline
\end{tabular}
\end{center}
\end{table}
\index{\texttt{\$(} and \texttt{\$)} auxiliary keywords}
\index{\texttt{\$[} and \texttt{\$]} auxiliary keywords}


Unlike those in some computer languages, the keywords\index{keyword} are short
two-character sequences rather than English-like words.  While this may make
them slightly more difficult to remember at first, their brevity allows
them to blend in with the mathematics being described, not
distract from it, like punctuation marks.


\subsection{User-Defined Tokens}\label{dollardollar}\index{token}

As you may have noticed, all keywords\index{keyword} begin with the \texttt{\$}
character.  This mundane monetary symbol is not ordinarily used in higher
mathematics (outside of grant proposals), so we have appropriated it to
distinguish the Metamath\index{Metamath} keywords from ordinary mathematical
symbols. The \texttt{\$} character is thus considered special and may not be
used as a character in a user-defined token.  All tokens and keywords are
case-sensitive; for example, \texttt{n} is considered to be a different character
from \texttt{N}.  Case-sensitivity makes the available {\sc ascii} character set
as rich as possible.

\subsubsection{Math Symbol Tokens}\index{token}

Math symbols\index{math symbol} are tokens used to represent the symbols
that appear in ordinary mathematical formulas.  They may consist of any
combination of the 93 non-whitespace printable {\sc ascii} characters other than
\texttt{\$}~. Some examples are \texttt{x}, \texttt{+}, \texttt{(},
\texttt{|-}, \verb$!%@?&$, and \texttt{bounded}.  For readability, it is
best to try to make these look as similar to actual mathematical symbols
as possible, within the constraints of the {\sc ascii} character set, in
order to make the resulting mathematical expressions more readable.

In the Metamath\index{Metamath} language, you express ordinary
mathematical formulas and statements as sequences of math symbols such
as \texttt{2 + 2 = 4} (five symbols, all constants).\footnote{To
eliminate ambiguity with other expressions, this is expressed in the set
theory database \texttt{set.mm} as \texttt{|- ( 2 + 2
 ) = 4 }, whose \LaTeX\ equivalent is $\vdash
(2+2)=4$.  The \,$\vdash$ means ``is a theorem'' and the
parentheses allow explicit associative grouping.}\index{turnstile
({$\,\vdash$})} They may even be English
sentences, as in \texttt{E is closed and bounded} (five symbols)---here
\texttt{E} would be a variable and the other four symbols constants.  In
principle, a Metamath database could be constructed to work with almost
any unambiguous English-language mathematical statement, but as a
practical matter the definitions needed to provide for all possible
syntax variations would be cumbersome and distracting and possibly have
subtle pitfalls accidentally built in.  We generally recommend that you
express mathematical statements with compact standard mathematical
symbols whenever possible and put their English-language descriptions in
comments.  Axioms\index{axiom} and definitions\index{definition}
(\texttt{\$a}\index{\texttt{\$a} statement} statements) are the only
places where Metamath will not detect an error, and doing this will help
reduce the number of definitions needed.

You are free to use any tokens\index{token} you like for math
symbols\index{math symbol}.  Appendix~\ref{ASCII} recommends token names to
use for symbols in set theory, and we suggest you adopt these in order to be
able to include the \texttt{set.mm} set theory database in your database.  For
printouts, you can convert the tokens in a database
to standard mathematical symbols with the \LaTeX\ typesetting program.  The
Metamath command \texttt{open tex} {\em filename}\index{\texttt{open tex} command}
produces output that can be read by \LaTeX.\index{latex@{\LaTeX}}
The correspondence
between tokens and the actual symbols is made by \texttt{latexdef}
statements inside a special database comment tagged
with \texttt{\$t}.\index{\texttt{\$t} comment}\index{typesetting comment}
  You can edit
this comment to change the definitions or add new ones.
Appendix~\ref{ASCII} describes how to do this in more detail.

% White space\index{white space} is normally used to separate math
% symbol\index{math symbol} tokens, but they may be juxtaposed without white
% space in \texttt{\$d}\index{\texttt{\$d} statement}, \texttt{\$e}\index{\texttt{\$e}
% statement}, \texttt{\$f}\index{\texttt{\$f} statement}, \texttt{\$a}\index{\texttt{\$a}
% statement}, and \texttt{\$p}\index{\texttt{\$p} statement} statements when no
% ambiguity will result.  Specifically, Metamath parses the math symbol sequence
% in one of these statements in the following manner:  when the math symbol
% sequence has been broken up into tokens\index{token} up to a given character,
% the next token is the longest string of characters that could constitute a
% math symbol that is active\index{active
% math symbol} at that point.  (See Section~\ref{scoping} for the
% definition of an active math symbol.)  For example, if \texttt{-}, \texttt{>}, and
% \texttt{->} are the only active math symbols, the juxtaposition \texttt{>-} will be
% interpreted as the two symbols \texttt{>} and \texttt{-}, whereas \texttt{->} will
% always be interpreted as that single symbol.\footnote{For better readability we
% recommend a white space between each token.  This also makes searching for a
% symbol easier to do with an editor.  Omission of optional white space is useful
% for reducing typing when assigning an expression to a temporary
% variable\index{temporary variable} with the \texttt{let variable} Metamath
% program command.}\index{\texttt{let variable} command}
%
% Keywords\index{keyword} may be placed next to math symbols without white
% space\index{white space} between them.\footnote{Again, we do not recommend
% this for readability.}
%
% The math symbols\index{math symbol} in \texttt{\$c}\index{\texttt{\$c} statement}
% and \texttt{\$v}\index{\texttt{\$v} statement} statements must always be separated
% by white space\index{white
% space}, for the obvious reason that these statements define the names
% of the symbols.
%
% Math symbols referred to in comments (see Section~\ref{comments}) must also be
% separated by white space.  This allows you to make comments about symbols that
% are not yet active\index{active
% math symbol}.  (The ``math mode'' feature of comments is also a quick and
% easy way to obtain word processing text with embedded mathematical symbols,
% independently of the main purpose of Metamath; the way to do this is described
% in Section~\ref{comments})

\subsubsection{Label Tokens}\index{token}\index{label}

Label tokens are used to identify Metamath\index{Metamath} statements for
later reference. Label tokens may contain only letters, digits, and the three
characters period, hyphen, and underscore:
\begin{verbatim}
. - _
\end{verbatim}

A label is {\bf declared}\index{label declaration} by placing it immediately
before the keyword of the statement it identifies.  For example, the label
\texttt{axiom.1} might be declared as follows:
\begin{verbatim}
axiom.1 $a |- x = x $.
\end{verbatim}

Each \texttt{\$e}\index{\texttt{\$e} statement},
\texttt{\$f}\index{\texttt{\$f} statement},
\texttt{\$a}\index{\texttt{\$a} statement}, and
\texttt{\$p}\index{\texttt{\$p} statement} statement in a database must
have a label declared for it.  No other statement types may have label
declarations.  Every label must be unique.

A label (and the statement it identifies) is {\bf referenced}\index{label
reference} by including the label between the \texttt{\$=}\index{\texttt{\$=}
keyword} and \texttt{\$.}\index{\texttt{\$.}\ keyword}\ keywords in a \texttt{\$p}
statement.  The sequence of labels\index{label sequence} between these two
keywords is called a {\bf proof}\index{proof}.  An example of a statement with
a proof that we will encounter later (Section~\ref{proof}) is
\begin{verbatim}
wnew $p wff ( s -> ( r -> p ) )
     $= ws wr wp w2 w2 $.
\end{verbatim}

You don't have to know what this means just yet, but you should know that the
label \texttt{wnew} is declared by this \texttt{\$p} statement and that the labels
\texttt{ws}, \texttt{wr}, \texttt{wp}, and \texttt{w2} are assumed to have been declared
earlier in the database and are referenced here.

\subsection{Constants and Variables}
\index{constant}
\index{variable}

An {\bf expression}\index{expression} is any sequence of math
symbols, possibly empty.

The basic Metamath\index{Metamath} language\index{basic language} has two
kinds of math symbols\index{math symbol}:  {\bf constants}\index{constant} and
{\bf variables}\index{variable}.  In a Metamath proof, a constant may not be
substituted with any expression.  A variable can be
substituted\index{substitution!variable}\index{variable substitution} with any
expression.  This sequence may include other variables and may even include
the variable being substituted.  This substitution takes place when proofs are
verified, and it will be described in Section~\ref{proof}.  The \texttt{\$f}
statement (described later in Section~\ref{dollaref}) is used to specify the
{\bf type} of a variable (i.e.\ what kind of
variable it is)\index{variable type}\index{type} and
give it a meaning typically
associated with a ``metavariable''\index{metavariable}\footnote{A metavariable
is a variable that ranges over the syntactical elements of the object language
being discussed; for example, one metavariable might represent a variable of
the object language and another metavariable might represent a formula in the
object language.} in ordinary mathematics; for example, a variable may be
specified to be a wff or well-formed formula (in logic), a set (in set
theory), or a non-negative integer (in number theory).

%\subsection{The \texttt{\$c} and \texttt{\$v} Declaration Statements}
\subsection{The \texttt{\$c} and \texttt{\$v} Declaration Statements}
\index{\texttt{\$c} statement}
\index{constant declaration}
\index{\texttt{\$v} statement}
\index{variable declaration}

Constants are introduced or {\bf declared}\index{constant declaration}
with \texttt{\$c}\index{\texttt{\$c} statement} statements, and
variables are declared\index{variable declaration} with
\texttt{\$v}\index{\texttt{\$v} statement} statements.  A {\bf simple}
declaration\index{simple declaration} statement introduces a single
constant or variable.  Its syntax is one of the following:
\begin{center}
  \texttt{\$c} {\em math-symbol} \texttt{\$.}\\
  \texttt{\$v} {\em math-symbol} \texttt{\$.}
\end{center}
The notation {\em math-symbol} means any math symbol token\index{token}.

Some examples of simple declaration statements are:
\begin{center}
  \texttt{\$c + \$.}\\
  \texttt{\$c -> \$.}\\
  \texttt{\$c ( \$.}\\
  \texttt{\$v x \$.}\\
  \texttt{\$v y2 \$.}
\end{center}

The characters in a math symbol\index{math symbol} being declared are
irrelevant to Meta\-math; for example, we could declare a right parenthesis to
be a variable,
\begin{center}
  \texttt{\$v ) \$.}\\
\end{center}
although this would be unconventional.

A {\bf compound} declaration\index{compound declaration} statement is a
shorthand for declaring several symbols at once.  Its syntax is one of the
following:
\begin{center}
  \texttt{\$c} {\em math-symbol}\ \,$\cdots$\ {\em math-symbol} \texttt{\$.}\\
  \texttt{\$v} {\em math-symbol}\ \,$\cdots$\ {\em math-symbol} \texttt{\$.}
\end{center}\index{\texttt{\$c} statement}
Here, the ellipsis (\ldots) means any number of {\em math-symbol}\,s.

An example of a compound declaration statement is:
\begin{center}
  \texttt{\$v x y mu \$.}\\
\end{center}
This is equivalent to the three simple declaration statements
\begin{center}
  \texttt{\$v x \$.}\\
  \texttt{\$v y \$.}\\
  \texttt{\$v mu \$.}\\
\end{center}
\index{\texttt{\$v} statement}

There are certain rules on where in the database math symbols may be declared,
what sections of the database are aware of them (i.e.\ where they are
``active''), and when they may be declared more than once.  These will be
discussed in Section~\ref{scoping} and specifically on
p.~\pageref{redeclaration}.

\subsection{The \texttt{\$d} Statement}\label{dollard}
\index{\texttt{\$d} statement}

The \texttt{\$d} statement is called a {\bf disjoint-variable restriction}.  The
syntax of the {\bf simple} version of this statement is
\begin{center}
  \texttt{\$d} {\em variable variable} \texttt{\$.}
\end{center}
where each {\em variable} is a previously declared variable and the two {\em
variable}\,s are different.  (More specifically, each  {\em variable} must be
an {\bf active} variable\index{active math symbol}, which means there must be
a previous \texttt{\$v} statement whose {\bf scope}\index{scope} includes the
\texttt{\$d} statement.  These terms will be defined when we discuss scoping
statements in Section~\ref{scoping}.)

In ordinary mathematics, formulas may arise that are true if the variables in
them are distinct\index{distinct variables}, but become false when those
variables are made identical. For example, the formula in logic $\exists x\,x
\neq y$, which means ``for a given $y$, there exists an $x$ that is not equal
to $y$,'' is true in most mathematical theories (namely all non-trivial
theories\index{non-trivial theory}, i.e.\ those that describe more than one
individual, such as arithmetic).  However, if we substitute $y$ with $x$, we
obtain $\exists x\,x \neq x$, which is always false, as it means ``there
exists something that is not equal to itself.''\footnote{If you are a
logician, you will recognize this as the improper substitution\index{proper
substitution}\index{substitution!proper} of a free variable\index{free
variable} with a bound variable\index{bound variable}.  Metamath makes no
inherent distinction between free and bound variables; instead, you let
Metamath know what substitutions are permissible by using \texttt{\$d} statements
in the right way in your axiom system.}\index{free vs.\ bound variable}  The
\texttt{\$d} statement allows you to specify a restriction that forbids the
substitution of one variable with another.  In
this case, we would use the statement
\begin{center}
  \texttt{\$d x y \$.}
\end{center}\index{\texttt{\$d} statement}
to specify this restriction.

The order in which the variables appear in a \texttt{\$d} statement is not
important.  We could also use
\begin{center}
  \texttt{\$d y x \$.}
\end{center}

The \texttt{\$d} statement is actually more general than this, as the
``disjoint''\index{disjoint variables} in its name suggests.  The full meaning
is that if any substitution is made to its two variables (during the
course of a proof that references a \texttt{\$a} or \texttt{\$p} statement
associated with the \texttt{\$d}), the two expressions that result from the
substitution must have no variables in common.  In addition, each possible
pair of variables, one from each expression, must be in a \texttt{\$d} statement
associated with the statement being proved.  (This requirement forces the
statement being proved to ``inherit'' the original disjoint variable
restriction.)

For example, suppose \texttt{u} is a variable.  If the restriction
\begin{center}
  \texttt{\$d A B \$.}
\end{center}
has been specified for a theorem referenced in a
proof, we may not substitute \texttt{A} with \mbox{\tt a + u} and
\texttt{B} with \mbox{\tt b + u} because these two symbol sequences have the
variable \texttt{u} in common.  Furthermore, if \texttt{a} and \texttt{b} are
variables, we may not substitute \texttt{A} with \texttt{a} and \texttt{B} with \texttt{b}
unless we have also specified \texttt{\$d a b} for the theorem being proved; in
other words, the \texttt{\$d} property associated with a pair of variables must
be effectively preserved after substitution.

The \texttt{\$d}\index{\texttt{\$d} statement} statement does {\em not} mean ``the
two variables may not be substituted with the same thing,'' as you might think
at first.  For example, substituting each of \texttt{A} and \texttt{B} in the above
example with identical symbol sequences consisting only of constants does not
cause a disjoint variable conflict, because two symbol sequences have no
variables in common (since they have no variables, period).  Similarly, a
conflict will not occur by substituting the two variables in a \texttt{\$d}
statement with the empty symbol sequence\index{empty substitution}.

The \texttt{\$d} statement does not have a direct counterpart in
ordinary mathematics, partly because the variables\index{variable} of
Metamath are not really the same as the variables\index{variable!in
ordinary mathematics} of ordinary mathematics but rather are
metavariables\index{metavariable} ranging over them (as well as over
other kinds of symbols and groups of symbols).  Depending on the
situation, we may informally interpret the \texttt{\$d} statement in
different ways.  Suppose, for example, that \texttt{x} and \texttt{y}
are variables ranging over numbers (more precisely, that \texttt{x} and
\texttt{y} are metavariables ranging over variables that range over
numbers), and that \texttt{ph} ($\varphi$) and \texttt{ps} ($\psi$) are
variables (more precisely, metavariables) ranging over formulas.  We can
make the following interpretations that correspond to the informal
language of ordinary mathematics:
\begin{quote}
\begin{tabbing}
\texttt{\$d x y \$.} means ``assume $x$ and $y$ are
distinct variables.''\\
\texttt{\$d x ph \$.} means ``assume $x$ does not
occur in $\varphi$.''\\
\texttt{\$d ph ps \$.} \=means ``assume $\varphi$ and
$\psi$ have no variables\\ \>in common.''
\end{tabbing}
\end{quote}\index{\texttt{\$d} statement}

\subsubsection{Compound \texttt{\$d} Statements}

The {\bf compound} version of the \texttt{\$d} statement is a shorthand for
specifying several variables whose substitutions must be pairwise disjoint.
Its syntax is:
\begin{center}
  \texttt{\$d} {\em variable}\ \,$\cdots$\ {\em variable} \texttt{\$.}
\end{center}\index{\texttt{\$d} statement}
Here, {\em variable} represents the token of a previously declared
variable (specifically, an active variable) and all {\em variable}\,s are
different.  The compound \texttt{\$d}
statement is internally broken up by Metamath into one simple \texttt{\$d}
statement for each possible pair of variables in the original \texttt{\$d}
statement.  For example,
\begin{center}
  \texttt{\$d w x y z \$.}
\end{center}
is equivalent to
\begin{center}
  \texttt{\$d w x \$.}\\
  \texttt{\$d w y \$.}\\
  \texttt{\$d w z \$.}\\
  \texttt{\$d x y \$.}\\
  \texttt{\$d x z \$.}\\
  \texttt{\$d y z \$.}
\end{center}

Two or more simple \texttt{\$d} statements specifying the same variable pair are
internally combined into a single \texttt{\$d} statement.  Thus the set of three
statements
\begin{center}
  \texttt{\$d x y \$.}
  \texttt{\$d x y \$.}
  \texttt{\$d y x \$.}
\end{center}
is equivalent to
\begin{center}
  \texttt{\$d x y \$.}
\end{center}

Similarly, compound \texttt{\$d} statements, after being internally broken up,
internally have their common variable pairs combined.  For example the
set of statements
\begin{center}
  \texttt{\$d x y A \$.}
  \texttt{\$d x y B \$.}
\end{center}
is equivalent to
\begin{center}
  \texttt{\$d x y \$.}
  \texttt{\$d x A \$.}
  \texttt{\$d y A \$.}
  \texttt{\$d x y \$.}
  \texttt{\$d x B \$.}
  \texttt{\$d y B \$.}
\end{center}
which is equivalent to
\begin{center}
  \texttt{\$d x y \$.}
  \texttt{\$d x A \$.}
  \texttt{\$d y A \$.}
  \texttt{\$d x B \$.}
  \texttt{\$d y B \$.}
\end{center}

Metamath\index{Metamath} automatically verifies that all \texttt{\$d}
restrictions are met whenever it verifies proofs.  \texttt{\$d} statements are
never referenced directly in proofs (this is why they do not have
labels\index{label}), but Metamath is always aware of which ones must be
satisfied (i.e.\ are active) and will notify you with an error message if any
violation occurs.

To illustrate how Metamath detects a missing \texttt{\$d}
statement, we will look at the following example from the
\texttt{set.mm} database.

\begin{verbatim}
$d x z $.  $d y z $.
$( Theorem to add distinct quantifier to atomic formula. $)
ax17eq $p |- ( x = y -> A. z x = y ) $=...
\end{verbatim}

This statement has the obvious requirement that $z$ must be
distinct\index{distinct variables} from $x$ in theorem \texttt{ax17eq} that
states $x=y \rightarrow \forall z \, x=y$ (well, obvious if you're a logician,
for otherwise we could conclude  $x=y \rightarrow \forall x \, x=y$, which is
false when the free variables $x$ and $y$ are equal).

Let's look at what happens if we edit the database to comment out this
requirement.

\begin{verbatim}
$( $d x z $. $) $d y z $.
$( Theorem to add distinct quantifier to atomic formula. $)
ax17eq $p |- ( x = y -> A. z x = y ) $=...
\end{verbatim}

When it tries to verify the proof, Metamath will tell you that \texttt{x} and
\texttt{z} must be disjoint, because one of its steps references an axiom or
theorem that has this requirement.

\begin{verbatim}
MM> verify proof ax17eq
ax17eq ?Error at statement 1918, label "ax17eq", type "$p":
      vz wal wi vx vy vz ax-13 vx vy weq vz vx ax-c16 vx vy
                                               ^^^^^
There is a disjoint variable ($d) violation at proof step 29.
Assertion "ax-c16" requires that variables "x" and "y" be
disjoint.  But "x" was substituted with "z" and "y" was
substituted with "x".  The assertion being proved, "ax17eq",
does not require that variables "z" and "x" be disjoint.
\end{verbatim}

We can see the substitutions into \texttt{ax-c16} with the following command.

\begin{verbatim}
MM> show proof ax17eq / detailed_step 29
Proof step 29:  pm2.61dd.2=ax-c16 $a |- ( A. z z = x -> ( x =
  y -> A. z x = y ) )
This step assigns source "ax-c16" ($a) to target "pm2.61dd.2"
($e).  The source assertion requires the hypotheses "wph"
($f, step 26), "vx" ($f, step 27), and "vy" ($f, step 28).
The parent assertion of the target hypothesis is "pm2.61dd"
($p, step 36).
The source assertion before substitution was:
    ax-c16 $a |- ( A. x x = y -> ( ph -> A. x ph ) )
The following substitutions were made to the source
assertion:
    Variable  Substituted with
     x         z
     y         x
     ph        x = y
The target hypothesis before substitution was:
    pm2.61dd.2 $e |- ( ph -> ch )
The following substitutions were made to the target
hypothesis:
    Variable  Substituted with
     ph        A. z z = x
     ch        ( x = y -> A. z x = y )
\end{verbatim}

The disjoint variable restrictions of \texttt{ax-c16} can be seen from the
\texttt{show state\-ment} command.  The line that begins ``\texttt{Its mandatory
dis\-joint var\-i\-able pairs are:}\ldots'' lists any \texttt{\$d} variable
pairs in brackets.

\begin{verbatim}
MM> show statement ax-c16/full
Statement 3033 is located on line 9338 of the file "set.mm".
"Axiom of Distinct Variables. ..."
  ax-c16 $a |- ( A. x x = y -> ( ph -> A. x ph ) ) $.
Its mandatory hypotheses in RPN order are:
  wph $f wff ph $.
  vx $f setvar x $.
  vy $f setvar y $.
Its mandatory disjoint variable pairs are:  <x,y>
The statement and its hypotheses require the variables:  x y
      ph
The variables it contains are:  x y ph
\end{verbatim}

Since Metamath will always detect when \texttt{\$d}\index{\texttt{\$d} statement}
statements are needed for a proof, you don't have to worry too much about
forgetting to put one in; it can always be added if you see the error message
above.  If you put in unnecessary \texttt{\$d} statements, the worst that could
happen is that your theorem might not be as general as it could be, and this
may limit its use later on.

On the other hand, when you introduce axioms (\texttt{\$a}\index{\texttt{\$a}
statement} statements), you must be very careful to properly specify the
necessary associated \texttt{\$d} statements since Metamath has no way of knowing
whether your axioms are correct.  For example, Metamath would have no idea
that \texttt{ax-c16}, which we are telling it is an axiom of logic, would lead to
contradictions if we omitted its associated \texttt{\$d} statement.

% This was previously a comment in footnote-sized type, but it can be
% hard to read this much text in a small size.
% As a result, it's been changed to normally-sized text.
\label{nodd}
You may wonder if it is possible to develop standard
mathematics in the Metamath language without the \texttt{\$d}\index{\texttt{\$d}
statement} statement, since it seems like a nuisance that complicates proof
verification. The \texttt{\$d} statement is not needed in certain subsets of
mathematics such as propositional calculus.  However, dummy
variables\index{dummy variable!eliminating} and their associated \texttt{\$d}
statements are impossible to avoid in proofs in standard first-order logic as
well as in the variant used in \texttt{set.mm}.  In fact, there is no upper bound to
the number of dummy variables that might be needed in a proof of a theorem of
first-order logic containing 3 or more variables, as shown by H.\
Andr\'{e}ka\index{Andr{\'{e}}ka, H.} \cite{Nemeti}.  A first-order system that
avoids them entirely is given in \cite{Megill}\index{Megill, Norman}; the
trick there is simply to embed harmlessly the necessary dummy variables into a
theorem being proved so that they aren't ``dummy'' anymore, then interpret the
resulting longer theorem so as to ignore the embedded dummy variables.  If
this interests you, the system in \texttt{set.mm} obtained from \texttt{ax-1}
through \texttt{ax-c14} in \texttt{set.mm}, and deleting \texttt{ax-c16} and \texttt{ax-5},
requires no \texttt{\$d} statements but is logically complete in the sense
described in \cite{Megill}.  This means it can prove any theorem of
first-order logic as long as we add to the theorem an antecedent that embeds
dummy and any other variables that must be distinct.  In a similar fashion,
axioms for set theory can be devised that
do not require distinct variable
provisos\index{Set theory without distinct variable provisos},
as explained at
\url{http://us.metamath.org/mpeuni/mmzfcnd.html}.
Together, these in principle allow all of
mathematics to be developed under Metamath without a \texttt{\$d} statement,
although the length of the resulting theorems will grow as more and
more dummy variables become required in their proofs.

\subsection{The \texttt{\$f}
and \texttt{\$e} Statements}\label{dollaref}
\index{\texttt{\$e} statement}
\index{\texttt{\$f} statement}
\index{floating hypothesis}
\index{essential hypothesis}
\index{variable-type hypothesis}
\index{logical hypothesis}
\index{hypothesis}

Metamath has two kinds of hypo\-theses, the \texttt{\$f}\index{\texttt{\$f}
statement} or {\bf variable-type} hypothesis and the \texttt{\$e} or {\bf logical}
hypo\-the\-sis.\index{\texttt{\$d} statement}\footnote{Strictly speaking, the
\texttt{\$d} statement is also a hypothesis, but it is never directly referenced
in a proof, so we call it a restriction rather than a hypothesis to lessen
confusion.  The checking for violations of \texttt{\$d} restrictions is automatic
and built into Metamath's proof-checking algorithm.} The letters \texttt{f} and
\texttt{e} stand for ``floating''\index{floating hypothesis} (roughly meaning
used only if relevant) and ``essential''\index{essential hypothesis} (meaning
always used) respectively, for reasons that will become apparent
when we discuss frames in
Section~\ref{frames} and scoping in Section~\ref{scoping}. The syntax of these
are as follows:
\begin{center}
  {\em label} \texttt{\$f} {\em typecode} {\em variable} \texttt{\$.}\\
  {\em label} \texttt{\$e} {\em typecode}
      {\em math-symbol}\ \,$\cdots$\ {\em math-symbol} \texttt{\$.}\\
\end{center}
\index{\texttt{\$e} statement}
\index{\texttt{\$f} statement}
A hypothesis must have a {\em label}\index{label}.  The expression in a
\texttt{\$e} hypothesis consists of a typecode (an active constant math symbol)
followed by a sequence
of zero or more math symbols. Each math symbol (including {\em constant}
and {\em variable}) must be a previously declared constant or variable.  (In
addition, each math symbol must be active, which will be covered when we
discuss scoping statements in Section~\ref{scoping}.)  You use a \texttt{\$f}
hypothesis to specify the
nature or {\bf type}\index{variable type}\index{type} of a variable (such as ``let $x$ be an
integer'') and use a \texttt{\$e} hypothesis to express a logical truth (such as
``assume $x$ is prime'') that must be established in order for an assertion
requiring it to also be true.

A variable must have its type specified in a \texttt{\$f} statement before
it may be used in a \texttt{\$e}, \texttt{\$a}, or \texttt{\$p}
statement.  There may be only one (active) \texttt{\$f} statement for a
given variable.  (``Active'' is defined in Section~\ref{scoping}.)

In ordinary mathematics, theorems\index{theorem} are often expressed in the
form ``Assume $P$; then $Q$,'' where $Q$ is a statement that you can derive
if you start with statement $P$.\index{free variable}\footnote{A stronger
version of a theorem like this would be the {\em single} formula $P\rightarrow
Q$ ($P$ implies $Q$) from which the weaker version above follows by the rule
of modus ponens in logic.  We are not discussing this stronger form here.  In
the weaker form, we are saying only that if we can {\em prove} $P$, then we can
{\em prove} $Q$.  In a logician's language, if $x$ is the only free variable
in $P$ and $Q$, the stronger form is equivalent to $\forall x ( P \rightarrow
Q)$ (for all $x$, $P$ implies $Q$), whereas the weaker form is equivalent to
$\forall x P \rightarrow \forall x Q$. The stronger form implies the weaker,
but not vice-versa.  To be precise, the weaker form of the theorem is more
properly called an ``inference'' rather than a theorem.}\index{inference}
In the
Metamath\index{Metamath} language, you would express mathematical statement
$P$ as a hypothesis (a \texttt{\$e} Metamath language statement in this case) and
statement $Q$ as a provable assertion (a \texttt{\$p}\index{\texttt{\$p} statement}
statement).

Some examples of hypotheses you might encounter in logic and set theory are
\begin{center}
  \texttt{stmt1 \$f wff P \$.}\\
  \texttt{stmt2 \$f setvar x \$.}\\
  \texttt{stmt3 \$e |- ( P -> Q ) \$.}
\end{center}
\index{\texttt{\$e} statement}
\index{\texttt{\$f} statement}
Informally, these would be read, ``Let $P$ be a well-formed-formula,'' ``Let
$x$ be an (individual) variable,'' and ``Assume we have proved $P \rightarrow
Q$.''  The turnstile symbol \,$\vdash$\index{turnstile ({$\,\vdash$})} is
commonly used in logic texts to mean ``a proof exists for.''

To summarize:
\begin{itemize}
\item A \texttt{\$f} hypothesis tells Metamath the type or kind of its variable.
It is analogous to a variable declaration in a computer language that
tells the compiler that a variable is an integer or a floating-point
number.
\item The \texttt{\$e} hypothesis corresponds to what you would usually call a
``hypothesis'' in ordinary mathematics.
\end{itemize}

Before an assertion\index{assertion} (\texttt{\$a} or \texttt{\$p} statement) can be
referenced in a proof, all of its associated \texttt{\$f} and \texttt{\$e} hypotheses
(i.e.\ those \texttt{\$e} hypotheses that are active) must be satisfied (i.e.
established by the proof).  The meaning of ``associated'' (which we will call
{\bf mandatory} in Section~\ref{frames}) will become clear when we discuss
scoping later.

Note that after any \texttt{\$f}, \texttt{\$e},
\texttt{\$a}, or \texttt{\$p} token there is a required
\textit{typecode}\index{typecode}.
The typecode is a constant used to enforce types of expressions.
This will become clearer once we learn more about
assertions (\texttt{\$a} and \texttt{\$p} statements).
An example may also clarify their purpose.
In the
\texttt{set.mm}\index{set theory database (\texttt{set.mm})}%
\index{Metamath Proof Explorer}
database,
the following typecodes are used:

\begin{itemize}
\item \texttt{wff} :
  Well-formed formula (wff) symbol
  (read: ``the following symbol sequence is a wff'').
% The *textual* typecode for turnstile is "|-", but when read it's a little
% confusing, so I intentionally display the mathematical symbol here instead
% (I think it's clearer in this context).
\item \texttt{$\vdash$} :
  Turnstile (read: ``the following symbol sequence is provable'' or
  ``a proof exists for'').
\item \texttt{setvar} :
  Individual set variable type (read: ``the following is an
  individual set variable'').
  Note that this is \textit{not} the type of an arbitrary set expression,
  instead, it is used to ensure that there is only a single symbol used
  after quantifiers like for-all ($\forall$) and there-exists ($\exists$).
\item \texttt{class} :
  An expression that is a syntactically valid class expression.
  All valid set expressions are also valid class expression, so expressions
  of sets normally have the \texttt{class} typecode.
  Use the \texttt{class} typecode,
  \textit{not} the \texttt{setvar} typecode,
  for the type of set expressions unless you are specifically identifying
  a single set variable.
\end{itemize}

\subsection{Assertions (\texttt{\$a} and \texttt{\$p} Statements)}
\index{\texttt{\$a} statement}
\index{\texttt{\$p} statement}\index{assertion}\index{axiomatic assertion}
\index{provable assertion}

There are two types of assertions, \texttt{\$a}\index{\texttt{\$a} statement}
statements ({\bf axiomatic assertions}) and \texttt{\$p} statements ({\bf
provable assertions}).  Their syntax is as follows:
\begin{center}
  {\em label} \texttt{\$a} {\em typecode} {\em math-symbol} \ldots
         {\em math-symbol} \texttt{\$.}\\
  {\em label} \texttt{\$p} {\em typecode} {\em math-symbol} \ldots
        {\em math-symbol} \texttt{\$=} {\em proof} \texttt{\$.}
\end{center}
\index{\texttt{\$a} statement}
\index{\texttt{\$p} statement}
\index{\texttt{\$=} keyword}
An assertion always requires a {\em label}\index{label}. The expression in an
assertion consists of a typecode (an active constant)
followed by a sequence of zero
or more math symbols.  Each math symbol, including any {\em constant}, must be a
previously declared constant or variable.  (In addition, each math symbol
must be active, which will be covered when we discuss scoping statements in
Section~\ref{scoping}.)

A \texttt{\$a} statement is usually a definition of syntax (for example, if $P$
and $Q$ are wffs then so is $(P\to Q)$), an axiom\index{axiom} of ordinary
mathematics (for example, $x=x$), or a definition\index{definition} of
ordinary mathematics (for example, $x\ne y$ means $\lnot x=y$). A \texttt{\$p}
statement is a claim that a certain combination of math symbols follows from
previous assertions and is accompanied by a proof that demonstrates it.

Assertions can also be referenced in (later) proofs in order to derive new
assertions from them. The label of an assertion is used to refer to it in a
proof. Section~\ref{proof} will describe the proof in detail.

Assertions also provide the primary means for communicating the mathematical
results in the database to people.  Proofs (when conveniently displayed)
communicate to people how the results were arrived at.

\subsubsection{The \texttt{\$a} Statement}
\index{\texttt{\$a} statement}

Axiomatic assertions (\texttt{\$a} statements) represent the starting points from
which other assertions (\texttt{\$p}\index{\texttt{\$p} statement} statements) are
derived.  Their most obvious use is for specifying ordinary mathematical
axioms\index{axiom}, but they are also used for two other purposes.

First, Metamath\index{Metamath} needs to know the syntax of symbol
sequences that constitute valid mathematical statements.  A Metamath
proof must be broken down into much more detail than ordinary
mathematical proofs that you may be used to thinking of (even the
``complete'' proofs of formal logic\index{formal logic}).  This is one
of the things that makes Metamath a general-purpose language,
independent of any system of logic or even syntax.  If you want to use a
substitution instance of an assertion as a step in a proof, you must
first prove that the substitution is syntactically correct (or if you
prefer, you must ``construct'' it), showing for example that the
expression you are substituting for a wff metavariable is a valid wff.
The \texttt{\$a}\index{\texttt{\$a} statement} statement is used to
specify those combinations of symbols that are considered syntactically
valid, such as the legal forms of wffs.

Second, \texttt{\$a} statements are used to specify what are ordinarily thought of
as definitions, i.e.\ new combinations of symbols that abbreviate other
combinations of symbols.  Metamath makes no distinction\index{axiom vs.\
definition} between axioms\index{axiom} and definitions\index{definition}.
Indeed, it has been argued that such distinction should not be made even in
ordinary mathematics; see Section~\ref{definitions}, which discusses the
philosophy of definitions.  Section~\ref{hierarchy} discusses some
technical requirements for definitions.  In \texttt{set.mm} we adopt the
convention of prefixing axiom labels with \texttt{ax-} and definition labels with
\texttt{df-}\index{label}.

The results that can be derived with the Metamath language are only as good as
the \texttt{\$a}\index{\texttt{\$a} statement} statements used as their starting
point.  We cannot stress this too strongly.  For example, Metamath will
not prevent you from specifying $x\neq x$ as an axiom of logic.  It is
essential that you scrutinize all \texttt{\$a} statements with great care.
Because they are a source of potential pitfalls, it is best not to add new
ones (usually new definitions) casually; rather you should carefully evaluate
each one's necessity and advantages.

Once you have in place all of the basic axioms\index{axiom} and
rules\index{rule} of a mathematical theory, the only \texttt{\$a} statements that
you will be adding will be what are ordinarily called definitions.  In
principle, definitions should be in some sense eliminable from the language of
a theory according to some convention (usually involving logical equivalence
or equality).  The most common convention is that any formula that was
syntactically valid but not provable before the definition was introduced will
not become provable after the definition is introduced.  In an ideal world,
definitions should not be present at all if one is to have absolute confidence
in a mathematical result.  However, they are necessary to make
mathematics practical, for otherwise the resulting formulas would be
extremely long and incomprehensible.  Since the nature of definitions (in the
most general sense) does not permit them to automatically be verified as
``proper,''\index{proper definition}\index{definition!proper} the judgment of
the mathematician is required to ensure it.  (In \texttt{set.mm} effort was made
to make almost all definitions directly eliminable and thus minimize the need
for such judgment.)

If you are not a mathematician, it may be best not to add or change any
\texttt{\$a}\index{\texttt{\$a} statement} statements but instead use
the mathematical language already provided in standard databases.  This
way Metamath will not allow you to make a mistake (i.e.\ prove a false
result).


\subsection{Frames}\label{frames}

We now introduce the concept of a collection of related Metamath statements
called a frame.  Every assertion (\texttt{\$a} or \texttt{\$p} statement) in the database has
an associated frame.

A {\bf frame}\index{frame} is a sequence of \texttt{\$d}, \texttt{\$f},
and \texttt{\$e} statements (zero or more of each) followed by one
\texttt{\$a} or \texttt{\$p} statement, subject to certain conditions we
will describe.  For simplicity we will assume that all math symbol
tokens used are declared at the beginning of the database with
\texttt{\$c} and \texttt{\$v} statements (which are not properly part of
a frame).  Also for simplicity we will assume there are only simple
\texttt{\$d} statements (those with only two variables) and imagine any
compound \texttt{\$d} statements (those with more than two variables) as
broken up into simple ones.

A frame groups together those hypotheses (and \texttt{\$d} statements) relevant
to an assertion (\texttt{\$a} or \texttt{\$p} statement).  The statements in a frame
may or may not be physically adjacent in a database; we will cover
this in our discussion of scoping statements
in Section~\ref{scoping}.

A frame has the following properties:
\begin{enumerate}
 \item The set of variables contained in its \texttt{\$f} statements must
be identical to the set of variables contained in its \texttt{\$e},
\texttt{\$a}, and/or \texttt{\$p} statements.  In other words, each
variable in a \texttt{\$e}, \texttt{\$a}, or \texttt{\$p} statement must
have an associated ``variable type'' defined for it in a \texttt{\$f}
statement.
  \item No two \texttt{\$f} statements may contain the same variable.
  \item Any \texttt{\$f} statement
must occur before a \texttt{\$e} statement in which its variable occurs.
\end{enumerate}

The first property determines the set of variables occurring in a frame.
These are the {\bf mandatory
variables}\index{mandatory variable} of the frame.  The second property
tells us there must be only one type specified for a variable.
The last property is not a theoretical requirement but it
makes parsing of the database easier.

For our examples, we assume our database has the following declarations:

\begin{verbatim}
$v P Q R $.
$c -> ( ) |- wff $.
\end{verbatim}

The following sequence of statements, describing the modus ponens inference
rule, is an example of a frame:

\begin{verbatim}
wp  $f wff P $.
wq  $f wff Q $.
maj $e |- ( P -> Q ) $.
min $e |- P $.
mp  $a |- Q $.
\end{verbatim}

The following sequence of statements is not a frame because \texttt{R} does not
occur in the \texttt{\$e}'s or the \texttt{\$a}:

\begin{verbatim}
wp  $f wff P $.
wq  $f wff Q $.
wr  $f wff R $.
maj $e |- ( P -> Q ) $.
min $e |- P $.
mp  $a |- Q $.
\end{verbatim}

The following sequence of statements is not a frame because \texttt{Q} does not
occur in a \texttt{\$f}:

\begin{verbatim}
wp  $f wff P $.
maj $e |- ( P -> Q ) $.
min $e |- P $.
mp  $a |- Q $.
\end{verbatim}

The following sequence of statements is not a frame because the \texttt{\$a} statement is
not the last one:

\begin{verbatim}
wp  $f wff P $.
wq  $f wff Q $.
maj $e |- ( P -> Q ) $.
mp  $a |- Q $.
min $e |- P $.
\end{verbatim}

Associated with a frame is a sequence of {\bf mandatory
hypotheses}\index{mandatory hypothesis}.  This is simply the set of all
\texttt{\$f} and \texttt{\$e} statements in the frame, in the order they
appear.  A frame can be referenced in a later proof using the label of
the \texttt{\$a} or \texttt{\$p} assertion statement, and the proof
makes an assignment to each mandatory hypothesis in the order in which
it appears.  This means the order of the hypotheses, once chosen, must
not be changed so as not to affect later proofs referencing the frame's
assertion statement.  (The Metamath proof verifier will, of course, flag
an error if a proof becomes incorrect by doing this.)  Since proofs make
use of ``Reverse Polish notation,'' described in Section~\ref{proof}, we
call this order the {\bf RPN order}\index{RPN order} of the hypotheses.

Note that \texttt{\$d} statements are not part of the set of mandatory
hypotheses, and their order doesn't matter (as long as they satisfy the
fourth property for a frame described above).  The \texttt{\$d}
statements specify restrictions on variables that must be satisfied (and
are checked by the proof verifier) when expressions are substituted for
them in a proof, and the \texttt{\$d} statements themselves are never
referenced directly in a proof.

A frame with a \texttt{\$p} (provable) statement requires a proof as part of the
\texttt{\$p} statement.  Sometimes in a proof we want to make use of temporary or
dummy variables\index{dummy variable} that do not occur in the \texttt{\$p}
statement or its mandatory hypotheses.  To accommodate this we define an {\bf
extended frame}\index{extended frame} as a frame together with zero or more
\texttt{\$d} and \texttt{\$f} statements that reference variables not among the
mandatory variables of the frame.  Any new variables referenced are called the
{\bf optional variables}\index{optional variable} of the extended frame. If a
\texttt{\$f} statement references an optional variable it is called an {\bf
optional hypothesis}\index{optional hypothesis}, and if one or both of the
variables in a \texttt{\$d} statement are optional variables it is called an {\bf
optional disjoint-variable restriction}\index{optional disjoint-variable
restriction}.  Properties 2 and 3 for a frame also apply to an extended
frame.

The concept of optional variables is not meaningful for frames with \texttt{\$a}
statements, since those statements have no proofs that might make use of them.
There is no restriction on including optional hypotheses in the extended frame
for a \texttt{\$a} statement, but they serve no purpose.

The following set of statements is an example of an extended frame, which
contains an optional variable \texttt{R} and an optional hypothesis \texttt{wr}.  In
this example, we suppose the rule of modus ponens is not an axiom but is
derived as a theorem from earlier statements (we omit its presumed proof).
Variable \texttt{R} may be used in its proof if desired (although this would
probably have no advantage in propositional calculus).  Note that the sequence
of mandatory hypotheses in RPN order is still \texttt{wp}, \texttt{wq}, \texttt{maj},
\texttt{min} (i.e.\ \texttt{wr} is omitted), and this sequence is still assumed
whenever the assertion \texttt{mp} is referenced in a subsequent proof.

\begin{verbatim}
wp  $f wff P $.
wq  $f wff Q $.
wr  $f wff R $.
maj $e |- ( P -> Q ) $.
min $e |- P $.
mp  $p |- Q $= ... $.
\end{verbatim}

Every frame is an extended frame, but not every extended frame is a frame, as
this example shows.  The underlying frame for an extended frame is
obtained by simply removing all statements containing optional variables.
Any proof referencing an assertion will ignore any extensions to its
frame, which means we may add or delete optional hypotheses at will without
affecting subsequent proofs.

The conceptually simplest way of organizing a Metamath database is as a
sequence of extended frames.  The scoping statements
\texttt{\$\char`\{}\index{\texttt{\$\char`\{} and \texttt{\$\char`\}}
keywords} and \texttt{\$\char`\}} can be used to delimit the start and
end of an extended frame, leading to the following possible structure for a
database.  \label{framelist}

\vskip 2ex
\setbox\startprefix=\hbox{\tt \ \ \ \ \ \ \ \ }
\setbox\contprefix=\hbox{}
\startm
\m{\mbox{(\texttt{\$v} {\em and} \texttt{\$c}\,{\em statements})}}
\endm
\startm
\m{\mbox{\texttt{\$\char`\{}}}
\endm
\startm
\m{\mbox{\texttt{\ \ } {\em extended frame}}}
\endm
\startm
\m{\mbox{\texttt{\$\char`\}}}}
\endm
\startm
\m{\mbox{\texttt{\$\char`\{}}}
\endm
\startm
\m{\mbox{\texttt{\ \ } {\em extended frame}}}
\endm
\startm
\m{\mbox{\texttt{\$\char`\}}}}
\endm
\startm
\m{\mbox{\texttt{\ \ \ \ \ \ \ \ \ }}\vdots}
\endm
\vskip 2ex

In practice, this structure is inconvenient because we have to repeat
any \texttt{\$f}, \texttt{\$e}, and \texttt{\$d} statements over and
over again rather than stating them once for use by several assertions.
The scoping statements, which we will discuss next, allow this to be
done.  In principle, any Metamath database can be converted to the above
format, and the above format is the most convenient to use when studying
a Metamath database as a formal system%
%% Uncomment this when uncommenting section {formalspec} below
   (Appendix \ref{formalspec})%
.
In fact, Metamath internally converts the database to the above format.
The command \texttt{show statement} in the Metamath program will show
you the contents of the frame for any \texttt{\$a} or \texttt{\$p}
statement, as well as its extension in the case of a \texttt{\$p}
statement.

%c%(provided that all ``local'' variables and constants with limited scope have
%c%unique names),

During our discussion of scoping statements, it may be helpful to
think in terms of the equivalent sequence of frames that will result when
the database is parsed.  Scoping (other than the limited
use above to delimit frames) is not a theoretical requirement for
Metamath but makes it more convenient.


\subsection{Scoping Statements (\texttt{\$\{} and \texttt{\$\}})}\label{scoping}
\index{\texttt{\$\char`\{} and \texttt{\$\char`\}} keywords}\index{scoping statement}

%c%Some Metamath statements may be needed only temporarily to
%c%serve a specific purpose, and after we're done with them we would like to
%c%disregard or ignore them.  For example, when we're finished using a variable,
%c%we might want to
%c%we might want to free up the token\index{token} used to name it so that the
%c%token can be used for other purposes later on, such as a different kind of
%c%variable or even a constant.  In the terminology of computer programming, we
%c%might want to let some symbol declarations be ``local'' rather than ``global.''
%c%\index{local symbol}\index{global symbol}

The {\bf scoping} statements, \texttt{\$\char`\{} ({\bf start of block}) and \texttt{\$\char`\}}
({\bf end of block})\index{block}, provide a means for controlling the portion
of a database over which certain statement types are recognized.  The
syntax of a scoping statement is very simple; it just consists of the
statement's keyword:
\begin{center}
\texttt{\$\char`\{}\\
\texttt{\$\char`\}}
\end{center}
\index{\texttt{\$\char`\{} and \texttt{\$\char`\}} keywords}

For example, consider the following database where we have stripped out
all tokens except the scoping statement keywords.  For the purpose of the
discussion, we have added subscripts to the scoping statements; these subscripts
do not appear in the actual database.
\[
 \mbox{\tt \ \$\char`\{}_1
 \mbox{\tt \ \$\char`\{}_2
 \mbox{\tt \ \$\char`\}}_2
 \mbox{\tt \ \$\char`\{}_3
 \mbox{\tt \ \$\char`\{}_4
 \mbox{\tt \ \$\char`\}}_4
 \mbox{\tt \ \$\char`\}}_3
 \mbox{\tt \ \$\char`\}}_1
\]
Each \texttt{\$\char`\{} statement in this example is said to be {\bf
matched} with the \texttt{\$\char`\}} statement that has the same
subscript.  Each pair of matched scoping statements defines a region of
the database called a {\bf block}.\index{block} Blocks can be {\bf
nested}\index{nested block} inside other blocks; in the example, the
block defined by $\mbox{\tt \$\char`\{}_4$ and $\mbox{\tt \$\char`\}}_4$
is nested inside the block defined by $\mbox{\tt \$\char`\{}_3$ and
$\mbox{\tt \$\char`\}}_3$ as well as inside the block defined by
$\mbox{\tt \$\char`\{}_1$ and $\mbox{\tt \$\char`\}}_1$.  In general, a
block may be empty, it may contain only non-scoping
statements,\footnote{Those statements other than \texttt{\$\char`\{} and
\texttt{\$\char`\}}.}\index{non-scoping statement} or it may contain any
mixture of other blocks and non-scoping statements.  (This is called a
``recursive'' definition\index{recursive definition} of a block.)

Associated with each block is a number called its {\bf nesting
level}\index{nesting level} that indicates how deeply the block is nested.
The nesting levels of the blocks in our example are as follows:
\[
  \underbrace{
    \mbox{\tt \ }
    \underbrace{
     \mbox{\tt \$\char`\{\ }
     \underbrace{
       \mbox{\tt \$\char`\{\ }
       \mbox{\tt \$\char`\}}
     }_{2}
     \mbox{\tt \ }
     \underbrace{
       \mbox{\tt \$\char`\{\ }
       \underbrace{
         \mbox{\tt \$\char`\{\ }
         \mbox{\tt \$\char`\}}
       }_{3}
       \mbox{\tt \ \$\char`\}}
     }_{2}
     \mbox{\tt \ \$\char`\}}
   }_{1}
   \mbox{\tt \ }
 }_{0}
\]
\index{\texttt{\$\char`\{} and \texttt{\$\char`\}} keywords}
The entire database is considered to be one big block (the {\bf outermost}
block) with a nesting level of 0.  The outermost block is {\em not} bracketed
by scoping statements.\footnote{The language was designed this way so that
several source files can be joined together more easily.}\index{outermost
block}

All non-scoping Metamath statements become recognized or {\bf
active}\index{active statement} at the place where they appear.\footnote{To
keep things slightly simpler, we do not bother to define the concept of
``active'' for the scoping statements.}  Certain of these statement types
become inactive at the end of the block in which they appear; these statement
types are:
\begin{center}
  \texttt{\$c}, \texttt{\$v}, \texttt{\$d}, \texttt{\$e}, and \texttt{\$f}.
%  \texttt{\$v}, \texttt{\$f}, \texttt{\$e}, and \texttt{\$d}.
\end{center}
\index{\texttt{\$c} statement}
\index{\texttt{\$d} statement}
\index{\texttt{\$e} statement}
\index{\texttt{\$f} statement}
\index{\texttt{\$v} statement}
The other statement types remain active forever (i.e.\ through the end of the
database); they are:
\begin{center}
  \texttt{\$a} and \texttt{\$p}.
%  \texttt{\$c}, \texttt{\$a}, and \texttt{\$p}.
\end{center}
\index{\texttt{\$a} statement}
\index{\texttt{\$p} statement}
Any statement (of these 7 types) located in the outermost
block\index{outermost block} will remain active through the end of the
database and thus are effectively ``global'' statements.\index{global
statement}

All \texttt{\$c} statements must be placed in the outermost block.  Since they are
therefore always global, they could be considered as belonging to both of the
above categories.

The {\bf scope}\index{scope} of a statement is the set of statements that
recognize it as active.

%c%The concept of ``active'' is also defined for math symbols\index{math
%c%symbol}.  Math symbols (constants\index{constant} and
%c%variables\index{variable}) become {\bf active}\index{active
%c%math symbol} in the \texttt{\$c}\index{\texttt{\$c}
%c%statement} and \texttt{\$v}\index{\texttt{\$v} statement} statements that
%c%declare them.  They become inactive when their declaration statements become
%c%inactive.

The concept of ``active'' is also defined for math symbols\index{math
symbol}.  Math symbols (constants\index{constant} and
variables\index{variable}) become {\bf active}\index{active math symbol}
in the \texttt{\$c}\index{\texttt{\$c} statement} and
\texttt{\$v}\index{\texttt{\$v} statement} statements that declare them.
A variable becomes inactive when its declaration statement becomes
inactive.  Because all \texttt{\$c} statements must be in the outermost
block, a constant will never become inactive after it is declared.

\subsubsection{Redeclaration of Math Symbols}
\index{redeclaration of symbols}\label{redeclaration}

%c%A math symbol may not be declared a second time while it is active, but it may
%c%be declared again after it becomes inactive.

A variable may not be declared a second time while it is active, but it may be
declared again after it becomes inactive.  This provides a convenient way to
introduce ``local'' variables,\index{local variable} i.e.\ temporary variables
for use in the frame of an assertion or in a proof without keeping them around
forever.  A previously declared variable may not be redeclared as a constant.

A constant may not be redeclared.  And, as mentioned above, constants must be
declared in the outermost block.

The reason variables may have limited scope but not constants is that an
assertion (\texttt{\$a} or \texttt{\$p} statement) remains available for use in
proofs through the end of the database.  Variables in an assertion's frame may
be substituted with whatever is needed in a proof step that references the
assertion, whereas constants remain fixed and may not be substituted with
anything.  The particular token used for a variable in an assertion's frame is
irrelevant when the assertion is referenced in a proof, and it doesn't matter
if that token is not available outside of the referenced assertion's frame.
Constants, however, must be globally fixed.

There is no theoretical
benefit for the feature allowing variables to be active for limited scopes
rather than global. It is just a convenience that allows them, for example, to
be locally grouped together with their corresponding \texttt{\$f} variable-type
declarations.

%c%If you declare a math symbol more than once, internally Metamath considers it a
%c%new distinct symbol, even though it has the same name.  If you are unaware of
%c%this, you may find that what you think are correct proofs are incorrectly
%c%rejected as invalid, because Metamath may tell you that a constant you
%c%previously declared does not match a newly declared math symbol with the same
%c%name.  For details on this subtle point, see the Comment on
%c%p.~\pageref{spec4comment}.  This is done purposely to allow temporary
%c%constants to be introduced while developing a subtheory, then allow their math
%c%symbol tokens to be reused later on; in general they will not refer to the
%c%same thing.  In practice, you would not ordinarily reuse the names of
%c%constants because it would tend to be confusing to the reader.  The reuse of
%c%names of variables, on the other hand, is something that is often useful to do
%c%(for example it is done frequently in \texttt{set.mm}).  Since variables in an
%c%assertion referenced in a proof can be substituted as needed to achieve a
%c%symbol match, this is not an issue.

% (This section covers a somewhat advanced topic you may want to skip
% at first reading.)
%
% Under certain circumstances, math symbol\index{math symbol}
% tokens\index{token} may be redeclared (i.e.\ the token
% may appear in more than
% one \texttt{\$c}\index{\texttt{\$c} statement} or \texttt{\$v}\index{\texttt{\$v}
% statement} statement).  You might want to do this say, to make temporary use
% of a variable name without having to worry about its affect elsewhere,
% somewhat analogous to declaring a local variable in a standard computer
% language.  Understanding what goes on when math symbol tokens are redeclared
% is a little tricky to understand at first, since it requires that we
% distinguish the token itself from the math symbol that it names.  It will help
% if we first take a peek at the internal workings of the
% Metamath\index{Metamath} program.
%
% Metamath reserves a memory location for each occurrence of a
% token\index{token} in a declaration statement (\texttt{\$c}\index{\texttt{\$c}
% statement} or \texttt{\$v}\index{\texttt{\$v} statement}).  If a given token appears
% in more than one declaration statement, it will refer to more than one memory
% locations.  A math symbol\index{math symbol} may be thought of as being one of
% these memory locations rather than as the token itself.  Only one of the
% memory locations associated with a given token may be active at any one time.
% The math symbol (memory location) that gets looked up when the token appears
% in a non-declaration statement is the one that happens to be active at that
% time.
%
% We now look at the rules for the redeclaration\index{redeclaration of symbols}
% of math symbol tokens.
% \begin{itemize}
% \item A math symbol token may not be declared twice in the
% same block.\footnote{While there is no theoretical reason for disallowing
% this, it was decided in the design of Metamath that allowing it would offer no
% advantage and might cause confusion.}
% \item An inactive math symbol may always be
% redeclared.
% \item  An active math symbol may be redeclared in a different (i.e.\
% inner) block\index{block} from the one it became active in.
% \end{itemize}
%
% When a math symbol token is redeclared, it conceptually refers to a different
% math symbol, just as it would be if it were called a different name.  In
% addition, the original math symbol that it referred to, if it was active,
% temporarily becomes inactive.  At the end of the block in which the
% redeclaration occurred, the new math symbol\index{math symbol} becomes
% inactive and the original symbol becomes active again.  This concept is
% illustrated in the following example, where the symbol \texttt{e} is
% ordinarily a constant (say Euler's constant, 2.71828...) but
% temporarily we want to use it as a ``local'' variable, say as a coefficient
% in the equation $a x^4 + b x^3 + c x^2 + d x + e$:
% \[
%   \mbox{\tt \$\char`\{\ \$c e \$.}
%   \underbrace{
%     \ \ldots\ %
%     \mbox{\tt \$\char`\{}\ \ldots\ %
%   }_{\mbox{\rm region A}}
%   \mbox{\tt \$v e \$.}
%   \underbrace{
%     \mbox{\ \ \ \ldots\ \ \ }
%   }_{\mbox{\rm region B}}
%   \mbox{\tt \$\char`\}}
%   \underbrace{
%     \mbox{\ \ \ \ldots\ \ \ }
%   }_{\mbox{\rm region C}}
%   \mbox{\tt \$\char`\}}
% \]
% \index{\texttt{\$\char`\{} and \texttt{\$\char`\}} keywords}
% In region A, the token \texttt{e} refers to a constant.  It is redeclared as a
% variable in region B, and any reference to it in this region will refer to this
% variable.  In region C, the redeclaration becomes inactive, and the original
% declaration becomes active again.  In region C, the token \texttt{x} refers to the
% original constant.
%
% As a practical matter, overuse of math symbol\index{math symbol}
% redeclarations\index{redeclaration of symbols} can be confusing (even though
% it is well-defined) and is best avoided when possible.  Here are some good
% general guidelines you can follow.  Usually, you should declare all
% constants\index{constant} in the outermost block\index{outermost block},
% especially if they are general-purpose (such as the token \verb$A.$, meaning
% $\forall$ or ``for all'').  This will make them ``globally'' active (although
% as in the example above local redeclarations will temporarily make them
% inactive.)  Most or all variables\index{variable}, on the other hand, could be
% declared in inner blocks, so that the token for them can be used later for a
% different type of variable or a constant.  (The names of the variables you
% choose are not used when you refer to an assertion\index{assertion} in a
% proof, whereas constants must match exactly.  A locally declared constant will
% not match a globally declared constant in a proof, even if they use the same
% token, because Metamath internally considers them to be different math
% symbols.)  To avoid confusion, you should generally avoid redeclaring active
% variables.  If you must redeclare them, do so at the beginning of a block.
% The temporary declaration of constants in inner blocks might be occasionally
% appropriate when you make use of a temporary definition to prove lemmas
% leading to a main result that does not make direct use of the definition.
% This way, you will not clutter up your database with a large number of
% seldom-used global constant symbols.  You might want to note that while
% inactive constants may not appear directly in an assertion (a \texttt{\$a}\index{\texttt{\$a}
% statement} or \texttt{\$p}\index{\texttt{\$p} statement}
% statement), they may be indirectly used in the proof of a \texttt{\$p} statement
% so long as they do not appear in the final math symbol sequence constructed by
% the proof.  In the end, you will have to use your best judgment, taking into
% account standard mathematical usage of the symbols as well as consideration
% for the reader of your work.
%
% \subsubsection{Reuse of Labels}\index{reuse of labels}\index{label}
%
% The \texttt{\$e}\index{\texttt{\$e} statement}, \texttt{\$f}\index{\texttt{\$f}
% statement}, \texttt{\$a}\index{\texttt{\$a} statement}, and
% \texttt{\$p}\index{\texttt{\$p}
% statement} statement types require labels, which allow them to be
% referenced later inside proofs.  A label is considered {\bf
% active}\index{active label} when the statement it is associated with is
% active.  The token\index{token} for a label may be reused
% (redeclared)\index{redeclaration of labels} provided that it is not being used
% for a currently active label.  (Unlike the tokens for math symbols, active
% label tokens may not be redeclared in an inner scope.)  Note that the labels
% of \texttt{\$a} and \texttt{\$p} statements can never be reused after these
% statements appear, because these statements remain active through the end of
% the database.
%
% You might find the reuse of labels a convenient way to have standard names for
% temporary hypotheses, such as \texttt{h1}, \texttt{h2}, etc.  This way you don't have
% to invent unique names for each of them, and in some cases it may be less
% confusing to the reader (although in other cases it might be more confusing, if
% the hypothesis is located far away from the assertion that uses
% it).\footnote{The current implementation requires that all labels, even
% inactive ones, be unique.}

\subsubsection{Frames Revisited}\index{frames and scoping statements}

Now that we have covered scoping, we will look at how an arbitrary
Metamath database can be converted to the simple sequence of extended
frames described on p.~\pageref{framelist}.  This is also how Metamath
stores the database internally when it reads in the database
source.\label{frameconvert} The method is simple.  First, we collect all
constant and variable (\texttt{\$c} and \texttt{\$v}) declarations in
the database, ignoring duplicate declarations of the same variable in
different scopes.  We then put our collected \texttt{\$c} and
\texttt{\$v} declarations at the beginning of the database, so that
their scope is the entire database.  Next, for each assertion in the
database, we determine its frame and extended frame.  The extended frame
is simply the \texttt{\$f}, \texttt{\$e}, and \texttt{\$d} statements
that are active.  The frame is the extended frame with all optional
hypotheses removed.

An equivalent way of saying this is that the extended frame of an assertion
is the collection of all \texttt{\$f}, \texttt{\$e}, and \texttt{\$d} statements
whose scope includes the assertion.
The \texttt{\$f} and \texttt{\$e} statements
occur in the order they appear
(order is irrelevant for \texttt{\$d} statements).

%c%, renaming any
%c%redeclared variables as needed so that all of them have unique names.  (The
%c%exact renaming convention is unimportant.  You might imagine renaming
%c%different declarations of math symbol \texttt{a} as \texttt{a\$1}, \texttt{a\$2}, etc.\
%c%which would prevent any conflicts since \texttt{\$} is not a legal character in a
%c%math symbol token.)

\section{The Anatomy of a Proof} \label{proof}
\index{proof!Metamath, description of}

Each provable assertion (\texttt{\$p}\index{\texttt{\$p} statement} statement) in a
database must include a {\bf proof}\index{proof}.  The proof is located
between the \texttt{\$=}\index{\texttt{\$=} keyword} and \texttt{\$.}\ keywords in the
\texttt{\$p} statement.

In the basic Metamath language\index{basic language}, a proof is a
sequence of statement labels.  This label sequence\index{label sequence}
serves as a set of instructions that the Metamath program uses to
construct a series of math symbol sequences.  The construction must
ultimately result in the math symbol sequence contained between the
\texttt{\$p}\index{\texttt{\$p} statement} and
\texttt{\$=}\index{\texttt{\$=} keyword} keywords of the \texttt{\$p}
statement.  Otherwise, the Metamath program will consider the proof
incorrect, and it will notify you with an appropriate error message when
you ask it to verify the proof.\footnote{To make the loading faster, the
Metamath program does not automatically verify proofs when you
\texttt{read} in a database unless you use the \texttt{/verify}
qualifier.  After a database has been read in, you may use the
\texttt{verify proof *} command to verify proofs.}\index{\texttt{verify
proof} command} Each label in a proof is said to {\bf
reference}\index{label reference} its corresponding statement.

Associated with any assertion\index{assertion} (\texttt{\$p} or
\texttt{\$a}\index{\texttt{\$a} statement} statement) is a set of
hypotheses (\texttt{\$f}\index{\texttt{\$f} statement} or
\texttt{\$e}\index{\texttt{\$e} statement} statements) that are active
with respect to that assertion.  Some are mandatory and the others are
optional.  You should review these concepts if necessary.

Each label\index{label} in a proof must be either the label of a
previous assertion (\texttt{\$a}\index{\texttt{\$a} statement} or
\texttt{\$p}\index{\texttt{\$p} statement} statement) or the label of an
active hypothesis (\texttt{\$e} or \texttt{\$f}\index{\texttt{\$f}
statement} statement) of the \texttt{\$p} statement containing the
proof.  Hypothesis labels may reference both the
mandatory\index{mandatory hypothesis} and the optional hypotheses of the
\texttt{\$p} statement.

The label sequence in a proof specifies a construction in {\bf reverse Polish
notation}\index{reverse Polish notation (RPN)} (RPN).  You may be familiar
with RPN if you have used older
Hewlett--Packard or similar hand-held calculators.
In the calculator analogy, a hypothesis label\index{hypothesis label} is like
a number and an assertion label\index{assertion label} is like an operation
(more precisely, an $n$-ary operation when the
assertion has $n$ \texttt{\$e}-hypotheses).
On an RPN calculator, an operation takes one or more previous numbers in an
input sequence, performs a calculation on them, and replaces those numbers and
itself with the result of the calculation.  For example, the input sequence
$2,3,+$ on an RPN calculator results in $5$, and the input sequence
$2,3,5,{\times},+$ results in $2,15,+$ which results in $17$.

Understanding how RPN is processed involves the concept of a {\bf
stack}\index{stack}\index{RPN stack}, which can be thought of as a set of
temporary memory locations that hold intermediate results.  When Metamath
encounters a hypothesis label it places or {\bf pushes}\index{push} the math
symbol sequence of the hypothesis onto the stack.  When Metamath encounters an
assertion label, it associates the most recent stack entries with the {\em
mandatory} hypotheses\index{mandatory hypothesis} of the assertion, in the
order where the most recent stack entry is associated with the last mandatory
hypothesis of the assertion.  It then determines what
substitutions\index{substitution!variable}\index{variable substitution} have
to be made into the variables of the assertion's mandatory hypotheses to make
them identical to the associated stack entries.  It then makes those same
substitutions into the assertion itself.  Finally, Metamath removes or {\bf
pops}\index{pop} the matched hypotheses from the stack and pushes the
substituted assertion onto the stack.

For the purpose of matching the mandatory hypothesis to the most recent stack
entries, whether a hypothesis is a \texttt{\$e} or \texttt{\$f} statement is
irrelevant.  The only important thing is that a set of
substitutions\footnote{In the Metamath spec (Section~\ref{spec}), we use the
singular term ``substitution'' to refer to the set of substitutions we talk
about here.} exist that allow a match (and if they don't, the proof verifier
will let you know with an error message).  The Metamath language is specified
in such a way that if a set of substitutions exists, it will be unique.
Specifically, the requirement that each variable have a type specified for it
with a \texttt{\$f} statement ensures the uniqueness.

We will illustrate this with an example.
Consider the following Metamath source file:
\begin{verbatim}
$c ( ) -> wff $.
$v p q r s $.
wp $f wff p $.
wq $f wff q $.
wr $f wff r $.
ws $f wff s $.
w2 $a wff ( p -> q ) $.
wnew $p wff ( s -> ( r -> p ) ) $= ws wr wp w2 w2 $.
\end{verbatim}
This Metamath source example shows the definition and ``proof'' (i.e.,
construction) of a well-formed formula (wff)\index{well-formed formula (wff)}
in propositional calculus.  (You may wish to type this example into a file to
experiment with the Metamath program.)  The first two statements declare
(introduce the names of) four constants and four variables.  The next four
statements specify the variable types, namely that
each variable is assumed to be a wff.  Statement \texttt{w2} defines (postulates)
a way to produce a new wff, \texttt{( p -> q )}, from two given wffs \texttt{p} and
\texttt{q}. The mandatory hypotheses of \texttt{w2} are \texttt{wp} and \texttt{wq}.
Statement \texttt{wnew} claims that \texttt{( s -> ( r -> p ) )} is a wff given
three wffs \texttt{s}, \texttt{r}, and \texttt{p}.  More precisely, \texttt{wnew} claims
that the sequence of ten symbols \texttt{wff ( s -> ( r -> p ) )} is provable from
previous assertions and the hypotheses of \texttt{wnew}.  Metamath does not know
or care what a wff is, and as far as it is concerned
the typecode \texttt{wff} is just an
arbitrary constant symbol in a math symbol sequence.  The mandatory hypotheses
of \texttt{wnew} are \texttt{wp}, \texttt{wr}, and \texttt{ws}; \texttt{wq} is an optional
hypothesis.  In our particular proof, the optional hypothesis is not
referenced, but in general, any combination of active (i.e.\ optional and
mandatory) hypotheses could be referenced.  The proof of statement \texttt{wnew}
is the sequence of five labels starting with \texttt{ws} (step~1) and ending with
\texttt{w2} (step~5).

When Metamath verifies the proof, it scans the proof from left to right.  We
will examine what happens at each step of the proof.  The stack starts off
empty.  At step 1, Metamath looks up label \texttt{ws} and determines that it is a
hypothesis, so it pushes the symbol sequence of statement \texttt{ws} onto the
stack:

\begin{center}\begin{tabular}{|l|l|}\hline
{Stack location} & {Contents} \\ \hline \hline
1 & \texttt{wff s} \\ \hline
\end{tabular}\end{center}

Metamath sees that the labels \texttt{wr} and \texttt{wp} in steps~2 and 3 are also
hypotheses, so it pushes them onto the stack.  After step~3, the stack looks
like
this:

\begin{center}\begin{tabular}{|l|l|}\hline
{Stack location} & {Contents} \\ \hline \hline
3 & \texttt{wff p} \\ \hline
2 & \texttt{wff r} \\ \hline
1 & \texttt{wff s} \\ \hline
\end{tabular}\end{center}

At step 4, Metamath sees that label \texttt{w2} is an assertion, so it must do
some processing.  First, it associates the mandatory hypotheses of \texttt{w2},
which are \texttt{wp} and \texttt{wq}, with stack locations~2 and 3, {\em in that
order}. Metamath determines that the only possible way
to make hypothesis \texttt{wp} match (become identical to) stack location~2 and
\texttt{wq} match stack location 3 is to substitute variable \texttt{p} with \texttt{r}
and \texttt{q} with \texttt{p}.  Metamath makes these substitutions into \texttt{w2} and
obtains the symbol sequence \texttt{wff ( r -> p )}.  It removes the hypotheses
from stack locations~2 and 3, then places the result into stack location~2:

\begin{center}\begin{tabular}{|l|l|}\hline
{Stack location} & {Contents} \\ \hline \hline
2 & \texttt{wff ( r -> p )} \\ \hline
1 & \texttt{wff s} \\ \hline
\end{tabular}\end{center}

At step 5, Metamath sees that label \texttt{w2} is an assertion, so it must again
do some processing.  First, it matches the mandatory hypotheses of \texttt{w2},
which are \texttt{wp} and \texttt{wq}, to stack locations 1 and 2.
Metamath determines that the only possible way to make the
hypotheses match is to substitute variable \texttt{p} with \texttt{s} and \texttt{q} with
\texttt{( r -> p )}.  Metamath makes these substitutions into \texttt{w2} and obtains
the symbol
sequence \texttt{wff ( s -> ( r -> p ) )}.  It removes stack
locations 1 and 2, then places the result into stack location~1:

\begin{center}\begin{tabular}{|l|l|}\hline
{Stack location} & {Contents} \\ \hline \hline
1 & \texttt{wff ( s -> ( r -> p ) )} \\ \hline
\end{tabular}\end{center}

After Metamath finishes processing the proof, it checks to see that the
stack contains exactly one element and that this element is
the same as the math symbol sequence in the
\texttt{\$p}\index{\texttt{\$p} statement} statement.  This is the case for our
proof of \texttt{wnew},
so we have proved \texttt{wnew} successfully.  If the result
differs, Metamath will notify you with an error message.  An error message
will also result if the stack contains more than one entry at the end of the
proof, or if the stack did not contain enough entries at any point in the
proof to match all of the mandatory hypotheses\index{mandatory hypothesis} of
an assertion.  Finally, Metamath will notify you with an error message if no
substitution is possible that will make a referenced assertion's hypothesis
match the
stack entries.  You may want to experiment with the different kinds of errors
that Metamath will detect by making some small changes in the proof of our
example.

Metamath's proof notation was designed primarily to express proofs in a
relatively compact manner, not for readability by humans.  Metamath can display
proofs in a number of different ways with the \texttt{show proof}\index{\texttt{show
proof} command} command.  The
\texttt{/lemmon} qualifier displays it in a format that is easier to read when the
proofs are short, and you saw examples of its use in Chapter~\ref{using}.  For
longer proofs, it is useful to see the tree structure of the proof.  A tree
structure is displayed when the \texttt{/lemmon} qualifier is omitted.  You will
probably find this display more convenient as you get used to it. The tree
display of the proof in our example looks like
this:\label{treeproof}\index{tree-style proof}\index{proof!tree-style}
\begin{verbatim}
1     wp=ws    $f wff s
2        wp=wr    $f wff r
3        wq=wp    $f wff p
4     wq=w2    $a wff ( r -> p )
5  wnew=w2  $a wff ( s -> ( r -> p ) )
\end{verbatim}
The number to the left of each line is the step number.  Following it is a
{\bf hypothesis association}\index{hypothesis association}, consisting of two
labels\index{label} separated by \texttt{=}.  To the left of the \texttt{=} (except
in the last step) is the label of a hypothesis of an assertion referenced
later in the proof; here, steps 1 and 4 are the hypothesis associations for
the assertion \texttt{w2} that is referenced in step 5.  A hypothesis association
is indented one level more than the assertion that uses it, so it is easy to
find the corresponding assertion by moving directly down until the indentation
level decreases to one less than where you started from.  To the right of each
\texttt{=} is the proof step label for that proof step.  The statement keyword of
the proof step label is listed next, followed by the content of the top of the
stack (the most recent stack entry) as it exists after that proof step is
processed.  With a little practice, you should have no trouble reading proofs
displayed in this format.

Metamath proofs include the syntax construction of a formula.
In standard mathematics, this kind of
construction is not considered a proper part of the proof at all, and it
certainly becomes rather boring after a while.
Therefore,
by default the \texttt{show proof}\index{\texttt{show proof}
command} command does not show the syntax construction.
Historically \texttt{show proof} command
\textit{did} show the syntax construction, and you needed to add the
\texttt{/essential} option to hide, them, but today
\texttt{/essential} is the default and you need to use
\texttt{/all} to see the syntax constructions.

When verifying a proof, Metamath will check that no mandatory
\texttt{\$d}\index{\texttt{\$d} statement}\index{mandatory \texttt{\$d}
statement} statement of an assertion referenced in a proof is violated
when substitutions\index{substitution!variable}\index{variable
substitution} are made to the variables in the assertion.  For details
see Section~\ref{spec4} or \ref{dollard}.

\subsection{The Concept of Unification} \label{unify}

During the course of verifying a proof, when Metamath\index{Metamath}
encounters an assertion label\index{assertion label}, it associates the
mandatory hypotheses\index{mandatory hypothesis} of the assertion with the top
entries of the RPN stack\index{stack}\index{RPN stack}.  Metamath then
determines what substitutions\index{substitution!variable}\index{variable
substitution} it must make to the variables in the assertion's mandatory
hypotheses in order for these hypotheses to become identical to their
corresponding stack entries.  This process is called {\bf
unification}\index{unification}.  (We also informally use the term
``unification'' to refer to a set of substitutions that results from the
process, as in ``two unifications are possible.'')  After the substitutions
are made, the hypotheses are said to be {\bf unified}.

If no such substitutions are possible, Metamath will consider the proof
incorrect and notify you with an error message.
% (deleted 3/10/07, per suggestion of Mel O'Cat:)
% The syntax of the
% Metamath language ensures that if a set of substitutions exists, it
% will be unique.

The general algorithm for unification described in the literature is
somewhat complex.
However, in the case of Metamath it is intentionally trivial.
Mandatory hypotheses must be
pushed on the proof stack in the order in which they appear.
In addition, each variable must have its type specified
with a \texttt{\$f} hypothesis before it is used
and that each \texttt{\$f} hypothesis
have the restricted syntax of a typecode (a constant) followed by a variable.
The typecode in the \texttt{\$f} hypothesis must match the first symbol of
the corresponding RPN stack entry (which will also be a constant), so
the only possible match for the variable in the \texttt{\$f} hypothesis is
the sequence of symbols in the stack entry after the initial constant.

In the Proof Assistant\index{Proof Assistant}, a more general unification
algorithm is used.  While a proof is being developed, sometimes not enough
information is available to determine a unique unification.  In this case
Metamath will ask you to pick the correct one.\index{ambiguous
unification}\index{unification!ambiguous}

\section{Extensions to the Metamath Language}\index{extended
language}

\subsection{Comments in the Metamath Language}\label{comments}
\index{markup notation}
\index{comments!markup notation}

The commenting feature allows you to annotate the contents of
a database.  Just as with most
computer languages, comments are ignored for the purpose of interpreting the
contents of the database. Comments effectively act as
additional white space\index{white
space} between tokens
when a database is parsed.

A comment may be placed at the beginning, end, or
between any two tokens\index{token} in a source file.

Comments have the following syntax:
\begin{center}
 \texttt{\$(} {\em text} \texttt{\$)}
\end{center}
Here,\index{\texttt{\$(} and \texttt{\$)} auxiliary
keywords}\index{comment} {\em text} is a string, possibly empty, of any
characters in Metamath's character set (p.~\pageref{spec1chars}), except
that the character strings \texttt{\$(} and \texttt{\$)} may not appear
in {\em text}.  Thus nested comments are not
permitted:\footnote{Computer languages have differing standards for
nested comments, and rather than picking one it was felt simplest not to
allow them at all, at least in the current version (0.177) of
Metamath\index{Metamath!limitations of version 0.177}.} Metamath will
complain if you give it
\begin{center}
 \texttt{\$( This is a \$( nested \$) comment.\ \$)}
\end{center}
To compensate for this non-nesting behavior, I often change all \texttt{\$}'s
to \texttt{@}'s in sections of Metamath code I wish to comment out.

The Metamath program supports a number of markup mechanisms and conventions
to generate good-looking results in \LaTeX\ and {\sc html},
as discussed below.
These markup features have to do only with how the comments are typeset,
and have no effect on how Metamath verifies the proofs in the database.
The improper
use of them may result in incorrectly typeset output, but no Metamath
error messages will result during the \texttt{read} and \texttt{verify
proof} commands.  (However, the \texttt{write
theorem\texttt{\char`\_}list} command
will check for markup errors as a side-effect of its
{\sc html} generation.)
Section~\ref{texout} has instructions for creating \LaTeX\ output, and
section~\ref{htmlout} has instructions for creating
{\sc html}\index{HTML} output.

\subsubsection{Headings}\label{commentheadings}

If the \texttt{\$(} is immediately followed by a new line
starting with a heading marker, it is a header.
This can start with:

\begin{itemize}
 \item[] \texttt{\#\#\#\#} - major part header
 \item[] \texttt{\#*\#*} - section header
 \item[] \texttt{=-=-} - subsection header
 \item[] \texttt{-.-.} - subsubsection header
\end{itemize}

The line following the marker line
will be used for the table of contents entry, after trimming spaces.
The next line should be another (closing) matching marker line.
Any text after that
but before the closing \texttt{\$}, such as an extended description of the
section, will be included on the \texttt{mmtheoremsNNN.html} page.

For more information, run
\texttt{help write theorem\char`\_list}.

\subsubsection{Math mode}
\label{mathcomments}
\index{\texttt{`} inside comments}
\index{\texttt{\char`\~} inside comments}
\index{math mode}

Inside of comments, a string of tokens\index{token} enclosed in
grave accents\index{grave accent (\texttt{`})} (\texttt{`}) will be converted
to standard mathematical symbols during
{\sc HTML}\index{HTML} or \LaTeX\ output
typesetting,\index{latex@{\LaTeX}} according to the information in the
special \texttt{\$t}\index{\texttt{\$t} comment}\index{typesetting
comment} comment in the database
(see section~\ref{tcomment} for information about the typesetting
comment, and Appendix~\ref{ASCII} to see examples of its results).

The first grave accent\index{grave accent (\texttt{`})} \texttt{`}
causes the output processor to enter {\bf math mode}\index{math mode}
and the second one exits it.
In this
mode, the characters following the \texttt{`} are interpreted as a
sequence of math symbol tokens separated by white space\index{white
space}.  The tokens are looked up in the \texttt{\$t}
comment\index{\texttt{\$t} comment}\index{typesetting comment} and if
found, they will be replaced by the standard mathematical symbols that
they correspond to before being placed in the typeset output file.  If
not found, the symbol will be output as is and a warning will be issued.
The tokens do not have to be active in the database, although a warning
will be issued if they are not declared with \texttt{\$c} or
\texttt{\$v} statements.

Two consecutive
grave accents \texttt{``} are treated as a single actual grave accent
(both inside and outside of math mode) and will not cause the output
processor to enter or exit math mode.

Here is an example of its use\index{Pierce's axiom}:
\begin{center}
\texttt{\$( Pierce's axiom, ` ( ( ph -> ps ) -> ph ) -> ph ` ,\\
         is not very intuitive. \$)}
\end{center}
becomes
\begin{center}
   \texttt{\$(} Pierce's axiom, $((\varphi \rightarrow \psi)\rightarrow
\varphi)\rightarrow \varphi$, is not very intuitive. \texttt{\$)}
\end{center}

Note that the math symbol tokens\index{token} must be surrounded by white
space\index{white space}.
%, since there is no context that allows ambiguity to be
%resolved, as is the case with math symbol sequences in some of the Metamath
%statements.
White space should also surround the \texttt{`}
delimiters.

The math mode feature also gives you a quick and easy way to generate
text containing mathematical symbols, independently of the intended
purpose of Metamath.\index{Metamath!using as a math editor} To do this,
simply create your text with grave accents surrounding your formulas,
after making sure that your math symbols are mapped to \LaTeX\ symbols
as described in Appendix~\ref{ASCII}.  It is easier if you start with a
database with predefined symbols such as \texttt{set.mm}.  Use your
grave-quoted math string to replace an existing comment, then typeset
the statement corresponding to that comment following the instructions
from the \texttt{help tex} command in the Metamath program.  You will
then probably want to edit the resulting file with a text editor to fine
tune it to your exact needs.

\subsubsection{Label Mode}\index{label mode}

Outside of math mode, a tilde\index{tilde (\texttt{\char`\~})} \verb/~/
indicates to Metamath's\index{Metamath} output processor that the
token\index{token} that follows (i.e.\ the characters up to the next
white space\index{white space}) represents a statement label or URL.
This formatting mode is called {\bf label mode}\index{label mode}.
If a literal tilde
is desired (outside of math mode) instead of label mode,
use two tildes in a row to represent it.

When generating a \LaTeX\ output file,
the following token will be formatted in \texttt{typewriter}
font, and the tilde removed, to make it stand out from the rest of the text.
This formatting will be applied to all characters after the
tilde up to the first white space\index{white space}.
Whether
or not the token is an actual statement label is not checked, and the
token does not have to have the correct syntax for a label; no error
messages will be produced.  The only effect of the label mode on the
output is that typewriter font will be used for the tokens that are
placed in the \LaTeX\ output file.

When generating {\sc html},
the tokens after the tilde {\em must} be a URL (either http: or https:)
or a valid label.
Error messages will be issued during that output if they aren't.
A hyperlink will be generated to that URL or label.

\subsubsection{Link to bibliographical reference}\index{citation}%
\index{link to bibliographical reference}

Bibliographical references are handled specially when generating
{\sc html} if formatted specially.
Text in the form \texttt{[}{\em author}\texttt{]}
is considered a link to a bibliographical reference.
See \texttt{help html} and \texttt{help write
bibliography} in the Metamath program for more
information.
% \index{\texttt{\char`\[}\ldots\texttt{]} inside comments}
See also Sections~\ref{tcomment} and \ref{wrbib}.

The \texttt{[}{\em author}\texttt{]} notation will also create an entry in
the bibliography cross-reference file generated by \texttt{write
bibliography} (Section~\ref{wrbib}) for {\sc HTML}.
For this to work properly, the
surrounding comment must be formatted as follows:
\begin{quote}
    {\em keyword} {\em label} {\em noise-word}
     \texttt{[}{\em author}\texttt{] p.} {\em number}
\end{quote}
for example
\begin{verbatim}
     Theorem 5.2 of [Monk] p. 223
\end{verbatim}
The {\em keyword} is not case sensitive and must be one of the following:
\begin{verbatim}
     theorem lemma definition compare proposition corollary
     axiom rule remark exercise problem notation example
     property figure postulate equation scheme chapter
\end{verbatim}
The optional {\em label} may consist of more than one
(non-{\em keyword} and non-{\em noise-word}) word.
The optional {\em noise-word} is one of:
\begin{verbatim}
     of in from on
\end{verbatim}
and is  ignored when the cross-reference file is created.  The
\texttt{write
biblio\-graphy} command will perform error checking to verify the
above format.\index{error checking}

\subsubsection{Parentheticals}\label{parentheticals}

The end of a comment may include one or more parenthicals, that is,
statements enclosed in parentheses.
The Metamath program looks for certain parentheticals and can issue
warnings based on them.
They are:

\begin{itemize}
 \item[] \texttt{(Contributed by }
   \textit{NAME}\texttt{,} \textit{DATE}\texttt{.)} -
   document the original contributor's name and the date it was created.
 \item[] \texttt{(Revised by }
   \textit{NAME}\texttt{,} \textit{DATE}\texttt{.)} -
   document the contributor's name and creation date
   that resulted in significant revision
   (not just an automated minimization or shortening).
 \item[] \texttt{(Proof shortened by }
   \textit{NAME}\texttt{,} \textit{DATE}\texttt{.)} -
   document the contributor's name and date that developed a significant
   shortening of the proof (not just an automated minimization).
 \item[] \texttt{(Proof modification is discouraged.)} -
   Note that this proof should normally not be modified.
 \item[] \texttt{(New usage is discouraged.)} -
   Note that this assertion should normally not be used.
\end{itemize}

The \textit{DATE} must be in form YYYY-MMM-DD, where MMM is the
English abbreviation of that month.

\subsubsection{Other markup}\label{othermarkup}
\index{markup notation}

There are other markup notations for generating good-looking results
beyond math mode and label mode:

\begin{itemize}
 \item[]
         \texttt{\char`\_} (underscore)\index{\texttt{\char`\_} inside comments} -
             Italicize text starting from
              {\em space}\texttt{\char`\_}{\em non-space} (i.e.\ \texttt{\char`\_}
              with a space before it and a non-space character after it) until
             the next
             {\em non-space}\texttt{\char`\_}{\em space}.  Normal
             punctuation (e.g.\ a trailing
             comma or period) is ignored when determining {\em space}.
 \item[]
         \texttt{\char`\_} (underscore) - {\em
         non-space}\texttt{\char`\_}{\em non-space-string}, where
          {\em non-space-string} is a string of non-space characters,
         will make {\em non-space-string} become a subscript.
 \item[]
         \texttt{<HTML>}...\texttt{</HTML>} - do not convert
         ``\texttt{<}'' and ``\texttt{>}''
         in the enclosed text when generating {\sc HTML},
         otherwise process markup normally. This allows direct insertion
         of {\sc html} commands.
 \item[]
       ``\texttt{\&}ref\texttt{;}'' - insert an {\sc HTML}
         character reference.
         This is how to insert arbitrary Unicode characters
         (such as accented characters).  Currently only directly supported
         when generating {\sc HTML}.
\end{itemize}

It is recommended that spaces surround any \texttt{\char`\~} and
\texttt{`} tokens in the comment and that a space follow the {\em label}
after a \texttt{\char`\~} token.  This will make global substitutions
to change labels and symbol names much easier and also eliminate any
future chance of ambiguity.  Spaces around these tokens are automatically
removed in the final output to conform with normal rules of punctuation;
for example, a space between a trailing \texttt{`} and a left parenthesis
will be removed.

A good way to become familiar with the markup notation is to look at
the extensive examples in the \texttt{set.mm} database.

\subsection{The Typesetting Comment (\texttt{\$t})}\label{tcomment}

The typesetting comment \texttt{\$t} in the input database file
provides the information necessary to produce good-looking results.
It provides \LaTeX\ and {\sc html}
definitions for math symbols,
as well supporting as some
customization of the generated web page.
If you add a new token to a database, you should also
update the \texttt{\$t} comment information if you want to eventually
create output in \LaTeX\ or {\sc HTML}.
See the
\texttt{set.mm}\index{set theory database (\texttt{set.mm})} database
file for an extensive example of a \texttt{\$t} comment illustrating
many of the features described below.

Programs that do not need to generate good-looking presentation results,
such as programs that only verify Metamath databases,
can completely ignore typesetting comments
and just treat them as normal comments.
Even the Metamath program only consults the
\texttt{\$t} comment information when it needs to generate typeset output
in \LaTeX\ or {\sc HTML}
(e.g., when you open a \LaTeX\ output file with the \texttt{open tex} command).

We will first discuss the syntax of typesetting comments, and then
briefly discuss how this can be used within the Metamath program.

\subsubsection{Typesetting Comment Syntax Overview}

The typesetting comment is identified by the token
\texttt{\$t}\index{\texttt{\$t} comment}\index{typesetting comment} in
the comment, and the typesetting comment ends at the matching
\texttt{\$)}:
\[
  \mbox{\tt \$(\ }
  \mbox{\tt \$t\ }
  \underbrace{
    \mbox{\tt \ \ \ \ \ \ \ \ \ \ \ }
    \cdots
    \mbox{\tt \ \ \ \ \ \ \ \ \ \ \ }
  }_{\mbox{Typesetting definitions go here}}
  \mbox{\tt \ \$)}
\]

There must be one or more white space characters, and only white space
characters, between the \texttt{\$(} that starts the comment
and the \texttt{\$t} symbol,
and the \texttt{\$t} must be followed by one
or more white space characters
(see section \ref{whitespace} for the definition of white space characters).
The typesetting comment continues until the comment end token \texttt{\$)}
(which must be preceded by one or more white space characters).

In version 0.177\index{Metamath!limitations of version 0.177} of the
Metamath program, there may be only one \texttt{\$t} comment in a
database.  This restriction may be lifted in the future to allow
many \texttt{\$t} comments in a database.

Between the \texttt{\$t} symbol (and its following white space) and the
comment end token \texttt{\$)} (and its preceding white space)
is a sequence of one or more typesetting definitions, where
each definition has the form
\textit{definition-type arg arg ... ;}.
Each of the zero or more \textit{arg} values
can be either a typesetting data or a keyword
(what keywords are allowed, and where, depends on the specific
\textit{definition-type}).
The \textit{definition-type}, and each argument \textit{arg},
are separated by one or more white space characters.
Every definition ends in an unquoted semicolon;
white space is not required before the terminating semicolon of a definition.
Each definition should start on a new line.\footnote{This
restriction of the current version of Metamath
(0.177)\index{Metamath!limitations of version 0.177} may be removed
in a future version, but you should do it anyway for readability.}

For example, this typesetting definition:
\begin{center}
 \verb$latexdef "C_" as "\subseteq";$
\end{center}
defines the token \verb$C_$ as the \LaTeX\ symbol $\subseteq$ (which means
``subset'').

Typesetting data is a sequence of one or more quoted strings
(if there is more than one, they are connected by \texttt{\char`\+}).
Often a single quoted string is used to provide data for a definition, using
either double (\texttt{\char`\"}) or single (\texttt{'}) quotation marks.
However,
{\em a quoted string (enclosed in quotation marks) may not include
line breaks.}
A quoted string
may include a quotation mark that matches the enclosing quotes by repeating
the quotation mark twice.  Here are some examples:

\begin{tabu}   { l l }
\textbf{Example} & \textbf{Meaning} \\
\texttt{\char`\"a\char`\"\char`\"b\char`\"} & \texttt{a\char`\"b} \\
\texttt{'c''d'} & \texttt{c'd} \\
\texttt{\char`\"e''f\char`\"} & \texttt{e''f} \\
\texttt{'g\char`\"\char`\"h'} & \texttt{g\char`\"\char`\"h} \\
\end{tabu}

Finally, a long quoted string
may be broken up into multiple quoted strings (considered, as a whole,
a single quoted string) and joined with \texttt{\char`\+}.
You can even use multiple lines as long as a
'+' is at the end of every line except the last one.
The \texttt{\char`\+} should be preceded and followed by at least one
white space character.
Thus, for example,
\begin{center}
 \texttt{\char`\"ab\char`\"\ \char`\+\ \char`\"cd\char`\"
    \ \char`\+\ \\ 'ef'}
\end{center}
is the same as
\begin{center}
 \texttt{\char`\"abcdef\char`\"}
\end{center}

{\sc c}-style comments \texttt{/*}\ldots\texttt{*/} are also supported.

In practice, whenever you add a new math token you will often want to add
typesetting definitions using
\texttt{latexdef}, \texttt{htmldef}, and
\texttt{althtmldef}, as described below.
That way, they will all be up to date.
Of course, whether or not you want to use all three definitions will
depend on how the database is intended to be used.

Below we discuss the different possible \textit{definition-kind} options.
We will show data surrounded by double quotes (in practice they can also use
single quotes and/or be a sequence joined by \texttt{+}s).
We will use specific names for the \textit{data} to make clear what
the data is used for, such as
{\em math-token} (for a Metamath math token,
{\em latex-string} (for string to be placed in a \LaTeX\ stream),
{\em {\sc html}-code} (for {\sc html} code),
and {\em filename} (for a filename).

\subsubsection{Typesetting Comment - \LaTeX}

The syntax for a \LaTeX\ definition is:
\begin{center}
 \texttt{latexdef "}{\em math-token}\texttt{" as "}{\em latex-string}\texttt{";}
\end{center}
\index{latex definitions@\LaTeX\ definitions}%
\index{\texttt{latexdef} statement}

The {\em token-string} and {\em latex-string} are the data
(character strings) for
the token and the \LaTeX\ definition of the token, respectively,

These \LaTeX\ definitions are used by the Metamath program
when it is asked to product \LaTeX output using
the \texttt{write tex} command.

\subsubsection{Typesetting Comment - {\sc html}}

The key kinds of {\sc HTML} definitions have the following syntax:

\vskip 1ex
    \texttt{htmldef "}{\em math-token}\texttt{" as "}{\em
    {\sc html}-code}\texttt{";}\index{\texttt{htmldef} statement}
                    \ \ \ \ \ \ldots

    \texttt{althtmldef "}{\em math-token}\texttt{" as "}{\em
{\sc html}-code}\texttt{";}\index{\texttt{althtmldef} statement}

                    \ \ \ \ \ \ldots

Note that in {\sc HTML} there are two possible definitions for math tokens.
This feature is useful when
an alternate representation of symbols is desired, for example one that
uses Unicode entities and another uses {\sc gif} images.

There are many other typesetting definitions that can control {\sc HTML}.
These include:

\vskip 1ex

    \texttt{htmldef "}{\em math-token}\texttt{" as "}{\em {\sc
    html}-code}\texttt{";}

    \texttt{htmltitle "}{\em {\sc html}-code}\texttt{";}%
\index{\texttt{htmltitle} statement}

    \texttt{htmlhome "}{\em {\sc html}-code}\texttt{";}%
\index{\texttt{htmlhome} statement}

    \texttt{htmlvarcolor "}{\em {\sc html}-code}\texttt{";}%
\index{\texttt{htmlvarcolor} statement}

    \texttt{htmlbibliography "}{\em filename}\texttt{";}%
\index{\texttt{htmlbibliography} statement}

\vskip 1ex

\noindent The \texttt{htmltitle} is the {\sc html} code for a common
title, such as ``Metamath Proof Explorer.''  The \texttt{htmlhome} is
code for a link back to the home page.  The \texttt{htmlvarcolor} is
code for a color key that appears at the bottom of each proof.  The file
specified by {\em filename} is an {\sc html} file that is assumed to
have a \texttt{<A NAME=}\ldots\texttt{>} tag for each bibiographic
reference in the database comments.  For example, if
\texttt{[Monk]}\index{\texttt{\char`\[}\ldots\texttt{]} inside comments}
occurs in the comment for a theorem, then \texttt{<A NAME='Monk'>} must
be present in the file; if not, a warning message is given.

Associated with
\texttt{althtmldef}
are the statements
\vskip 1ex

    \texttt{htmldir "}{\em
      directoryname}\texttt{";}\index{\texttt{htmldir} statement}

    \texttt{althtmldir "}{\em
     directoryname}\texttt{";}\index{\texttt{althtmldir} statement}

\vskip 1ex
\noindent giving the directories of the {\sc gif} and Unicode versions
respectively; their purpose is to provide cross-linking between the
two versions in the generated web pages.

When two different types of pages need to be produced from a single
database, such as the Hilbert Space Explorer that extends the Metamath
Proof Explorer, ``extended'' variables may be declared in the
\texttt{\$t} comment:
\vskip 1ex

    \texttt{exthtmltitle "}{\em {\sc html}-code}\texttt{";}%
\index{\texttt{exthtmltitle} statement}

    \texttt{exthtmlhome "}{\em {\sc html}-code}\texttt{";}%
\index{\texttt{exthtmlhome} statement}

    \texttt{exthtmlbibliography "}{\em filename}\texttt{";}%
\index{\texttt{exthtmlbibliography} statement}

\vskip 1ex
\noindent When these are declared, you also must declare
\vskip 1ex

    \texttt{exthtmllabel "}{\em label}\texttt{";}%
\index{\texttt{exthtmllabel} statement}

\vskip 1ex \noindent that identifies the database statement where the
``extended'' section of the database starts (in our example, where the
Hilbert Space Explorer starts).  During the generation of web pages for
that starting statement and the statements after it, the {\sc html} code
assigned to \texttt{exthtmltitle} and \texttt{exthtmlhome} is used
instead of that assigned to \texttt{htmltitle} and \texttt{htmlhome},
respectively.

\begin{sloppy}
\subsection{Additional Information Com\-ment (\texttt{\$j})} \label{jcomment}
\end{sloppy}

The additional information comment, aka the
\texttt{\$j}\index{\texttt{\$j} comment}\index{additional information comment}
comment,
provides a way to add additional structured information that can
be optionally parsed by systems.

The additional information comment is parsed the same way as the
typesetting comment (\texttt{\$t}) (see section \ref{tcomment}).
That is,
the additional information comment begins with the token
\texttt{\$j} within a comment,
and continues until the comment close \texttt{\$)}.
Within an additional information comment is a sequence of one or more
commands of the form \texttt{command arg arg ... ;}
where each of the zero or more \texttt{arg} values
can be either a quoted string or a keyword.
Note that every command ends in an unquoted semicolon.
If a verifier is parsing an additional information comment, but
doesn't recognize a particular command, it must skip the command
by finding the end of the command (an unquoted semicolon).

A database may have 0 or more additional information comments.
Note, however, that a verifier may ignore these comments entirely or only
process certain commands in an additional information comment.
The \texttt{mmj2} verifier supports many commands in additional information
comments.
We encourage systems that process additional information comments
to coordinate so that they will use the same command for the same effect.

Examples of additional information comments with various commands
(from the \texttt{set.mm} database) are:

\begin{itemize}
   \item Define the syntax and logical typecodes,
     and declare that our grammar is
     unambiguous (verifiable using the KLR parser, with compositing depth 5).
\begin{verbatim}
  $( $j
    syntax 'wff';
    syntax '|-' as 'wff';
    unambiguous 'klr 5';
  $)
\end{verbatim}

   \item Register $\lnot$ and $\rightarrow$ as primitive expressions
           (lacking definitions).
\begin{verbatim}
  $( $j primitive 'wn' 'wi'; $)
\end{verbatim}

   \item There is a special justification for \texttt{df-bi}.
\begin{verbatim}
  $( $j justification 'bijust' for 'df-bi'; $)
\end{verbatim}

   \item Register $\leftrightarrow$ as an equality for its type (wff).
\begin{verbatim}
  $( $j
    equality 'wb' from 'biid' 'bicomi' 'bitri';
    definition 'dfbi1' for 'wb';
  $)
\end{verbatim}

   \item Theorem \texttt{notbii} is the congruence law for negation.
\begin{verbatim}
  $( $j congruence 'notbii'; $)
\end{verbatim}

   \item Add \texttt{setvar} as a typecode.
\begin{verbatim}
  $( $j syntax 'setvar'; $)
\end{verbatim}

   \item Register $=$ as an equality for its type (\texttt{class}).
\begin{verbatim}
  $( $j equality 'wceq' from 'eqid' 'eqcomi' 'eqtri'; $)
\end{verbatim}

\end{itemize}


\subsection{Including Other Files in a Metamath Source File} \label{include}
\index{\texttt{\$[} and \texttt{\$]} auxiliary keywords}

The keywords \texttt{\$[} and \texttt{\$]} specify a file to be
included\index{included file}\index{file inclusion} at that point in a
Metamath\index{Metamath} source file\index{source file}.  The syntax for
including a file is as follows:
\begin{center}
\texttt{\$[} {\em file-name} \texttt{\$]}
\end{center}

The {\em file-name} should be a single token\index{token} with the same syntax
as a math symbol (i.e., all 93 non-whitespace
printable characters other than \texttt{\$} are
allowed, subject to the file-naming limitations of your operating system).
Comments may appear between the \texttt{\$[} and \texttt{\$]} keywords.  Included
files may include other files, which may in turn include other files, and so
on.

For example, suppose you want to use the set theory database as the starting
point for your own theory.  The first line in your file could be
\begin{center}
\texttt{\$[ set.mm \$]}
\end{center} All of the information (axioms, theorems,
etc.) in \texttt{set.mm} and any files that {\em it} includes will become
available for you to reference in your file. This can help make your work more
modular. A drawback to including files is that if you change the name of a
symbol or the label of a statement, you must also remember to update any
references in any file that includes it.


The naming conventions for included files are the same as those of your
operating system.\footnote{On the Macintosh, prior to Mac OS X,
 a colon is used to separate disk
and folder names from your file name.  For example, {\em volume}\texttt{:}{\em
file-name} refers to the root directory, {\em volume}\texttt{:}{\em
folder-name}\texttt{:}{\em file-name} refers to a folder in root, and {\em
volume}\texttt{:}{\em folder-name}\texttt{:}\ldots\texttt{:}{\em file-name} refers to a
deeper folder.  A simple {\em file-name} refers to a file in the folder from
which you launch the Metamath application.  Under Mac OS X and later,
the Metamath program is run under the Terminal application, which
conforms to Unix naming conventions.}\index{Macintosh file
names}\index{file names!Macintosh}\label{includef} For compatibility among
operating systems, you should keep the file names as simple as possible.  A
good convention to use is {\em file}\texttt{.mm} where {\em file} is eight
characters or less, in lower case.

There is no limit to the nesting depth of included files.  One thing that you
should be aware of is that if two included files themselves include a common
third file, only the {\em first} reference to this common file will be read
in.  This allows you to include two or more files that build on a common
starting file without having to worry about label and symbol conflicts that
would occur if the common file were read in more than once.  (In fact, if a
file includes itself, the self-reference will be ignored, although of course
it would not make any sense to do that.)  This feature also means, however,
that if you try to include a common file in several inner blocks, the result
might not be what you expect, since only the first reference will be replaced
with the included file (unlike the include statement in most other computer
languages).  Thus you would normally include common files only in the
outermost block\index{outermost block}.

\subsection{Compressed Proof Format}\label{compressed1}\index{compressed
proof}\index{proof!compressed}

The proof notation presented in Section~\ref{proof} is called a
{\bf normal proof}\index{normal proof}\index{proof!normal} and in principle is
sufficient to express any proof.  However, proofs often contain steps and
subproofs that are identical.  This is particularly true in typical
Metamath\index{Metamath} applications, because Metamath requires that the math
symbol sequence (usually containing a formula) at each step be separately
constructed, that is, built up piece by piece. As a result, a lot of
repetition often results.  The {\bf compressed proof} format allows Metamath
to take advantage of this redundancy to shorten proofs.

The specification for the compressed proof format is given in
Appen\-dix~\ref{compressed}.

Normally you need not concern yourself with the details of the compressed
proof format, since the Metamath program will allow you to convert from
the normal format to the compressed format with ease, and will also
automatically convert from the compressed format when proofs are displayed.
The overall structure of the compressed format is as follows:
\begin{center}
  \texttt{\$= ( } {\em label-list} \texttt{) } {\em compressed-proof\ }\ \texttt{\$.}
\end{center}
\index{\texttt{\$=} keyword}
The first \texttt{(} serves as a flag to Metamath that a compressed proof
follows.  The {\em label-list} includes all statements referred to by the
proof except the mandatory hypotheses\index{mandatory hypothesis}.  The {\em
compressed-proof} is a compact encoding of the proof, using upper-case
letters, and can be thought of as a large integer in base 26.  White
space\index{white space} inside a {\em compressed-proof} is
optional and is ignored.

It is important to note that the order of the mandatory hypotheses of
the statement being proved must not be changed if the compressed proof
format is used, otherwise the proof will become incorrect.  The reason
for this is that the mandatory hypotheses are not mentioned explicitly
in the compressed proof in order to make the compression more efficient.
If you wish to change the order of mandatory hypotheses, you must first
convert the proof back to normal format using the \texttt{save proof
{\em statement} /normal}\index{\texttt{save proof} command} command.
Later, you can go back to compressed format with \texttt{save proof {\em
statement} /compressed}.

During error checking with the \texttt{verify proof} command, an error
found in a compressed proof may point to a character in {\em
compressed-proof}, which may not be very meaningful to you.  In this
case, try to \texttt{save proof /normal} first, then do the
\texttt{verify proof} again.  In general, it is best to make sure a
proof is correct before saving it in compressed format, because severe
errors are less likely to be recoverable than in normal format.

\subsection{Specifying Unknown Proofs or Subproofs}\label{unknown}

In a proof under development, any step or subproof that is not yet known
may be represented with a single \texttt{?}.  For the purposes of
parsing the proof, the \texttt{?}\ \index{\texttt{]}@\texttt{?}\ inside
proofs} will push a single entry onto the RPN stack just as if it were a
hypothesis.  While developing a proof with the Proof
Assistant\index{Proof Assistant}, a partially developed proof may be
saved with the \texttt{save new{\char`\_}proof}\index{\texttt{save
new{\char`\_}proof} command} command, and \texttt{?}'s will be placed at
the appropriate places.

All \texttt{\$p}\index{\texttt{\$p} statement} statements must have
proofs, even if they are entirely unknown.  Before creating a proof with
the Proof Assistant, you should specify a completely unknown proof as
follows:
\begin{center}
  {\em label} \texttt{\$p} {\em statement} \texttt{\$= ?\ \$.}
\end{center}
\index{\texttt{\$=} keyword}
\index{\texttt{]}@\texttt{?}\ inside proofs}

The \texttt{verify proof}\index{\texttt{verify proof} command} command
will check the known portions of a partial proof for errors, but will
warn you that the statement has not been proved.

Note that partially developed proofs may be saved in compressed format
if desired.  In this case, you will see one or more \texttt{?}'s in the
{\em compressed-proof} part.\index{compressed
proof}\index{proof!compressed}

\section{Axioms vs.\ Definitions}\label{definitions}

The \textit{basic}
Metamath\index{Metamath} language and program
make no distinction\index{axiom vs.\
definition} between axioms\index{axiom} and
definitions.\index{definition} The \texttt{\$a}\index{\texttt{\$a}
statement} statement is used for both.  At first, this may seem
puzzling.  In the minds of many mathematicians, the distinction is
clear, even obvious, and hardly worth discussing.  A definition is
considered to be merely an abbreviation that can be replaced by the
expression for which it stands; although unless one actually does this,
to be precise then one should say that a theorem\index{theorem} is a
consequence of the axioms {\em and} the definitions that are used in the
formulation of the theorem \cite[p.~20]{Behnke}.\index{Behnke, H.}

\subsection{What is a Definition?}

What is a definition?  In its simplest form, a definition introduces a new
symbol and provides an unambiguous rule to transform an expression containing
the new symbol to one without it.  The concept of a ``proper
definition''\index{proper definition}\index{definition!proper} (as opposed to
a creative definition)\index{creative definition}\index{definition!creative}
that is usually agreed upon is (1) the definition should not strengthen the
language and (2) any symbols introduced by the definition should be eliminable
from the language \cite{Nemesszeghy}\index{Nemesszeghy, E. Z.}.  In other
words, they are mere typographical conveniences that do not belong to the
system and are theoretically superfluous.  This may seem obvious, but in fact
the nature of definitions can be subtle, sometimes requiring difficult
metatheorems to establish that they are not creative.

A more conservative stance was taken by logician S.
Le\'{s}niewski.\index{Le\'{s}niewski, S.}
\begin{quote}
Le\'{s}niewski
regards definitions as theses of the system.  In this respect they do
not differ either from the axioms or from theorems, i.e.\ from the
theses added to the system on the basis of the rule of substitution or
the rule of detachment [modus ponens].  Once definitions have been
accepted as theses of the system, it becomes necessary to consider them
as true propositions in the same sense in which axioms are true
\cite{Lejewski}.
\end{quote}\index{Lejewski, Czeslaw}

Let us look at some simple examples of definitions in propositional
calculus.  Consider the definition of logical {\sc or}
(disjunction):\index{disjunction ($\vee$)} ``$P\vee Q$ denotes $\neg P
\rightarrow Q$ (not $P$ implies $Q$).''  It is very easy to recognize a
statement making use of this definition, because it introduces the new
symbol $\vee$ that did not previously exist in the language.  It is easy
to see that no new theorems of the original language will result from
this definition.

Next, consider a definition that eliminates parentheses:  ``$P
\rightarrow Q\rightarrow R$ denotes $P\rightarrow (Q \rightarrow R)$.''
This is more subtle, because no new symbols are introduced.  The reason
this definition is considered proper is that no new symbol sequences
that are valid wffs (well-formed formulas)\index{well-formed formula
(wff)} in the original language will result from the definition, since
``$P \rightarrow Q\rightarrow R$'' is not a wff in the original
language.  Here, we implicitly make use of the fact that there is a
decision procedure that allows us to determine whether or not a symbol
sequence is a wff, and this fact allows us to use symbol sequences that
are not wffs to represent other things (such as wffs) by means of the
definition.  However, to justify the definition as not being creative we
need to prove that ``$P \rightarrow Q\rightarrow R$'' is in fact not a
wff in the original language, and this is more difficult than in the
case where we simply introduce a new symbol.

%Now let's take this reasoning to an extreme.  Propositional calculus is a
%decidable theory,\footnote{This means that a mechanical algorithm exists to
%determine whether or not a wff is a theorem.} so in principle we could make use
%of symbol sequences that are not theorems to represent other things (say, to
%encode actual theorems in a more compact way).  For example, let us extend the
%language by defining a wff ``$P$'' in the extended language as the theorem
%``$P\rightarrow P$''\footnote{This is one of the first theorems proved in the
%Metamath database \texttt{set.mm}.}\index{set
%theory database (\texttt{set.mm})} in the original language whenever ``$P$'' is
%not a theorem in the original language.  In the extended language, any wff
%``$Q$'' thus represents a theorem; to find out what theorem (in the original
%language) ``$Q$'' represents, we determine whether ``$Q$'' is a theorem in the
%original language (before the definition was introduced).  If so, we're done; if
%not, we replace ``$Q$'' by ``$Q\rightarrow Q$'' to eliminate the definition.
%This definition is therefore eliminable, and it does not ``strengthen'' the
%language because any wff that is not a theorem is not in the set of statements
%provable in the original language and thus is available for use by definitions.
%
%Of course, a definition such as this would render practically useless the
%communication of theorems of propositional calculus; but
%this is just a human shortcoming, since we can't always easily discern what is
%and is not a theorem by inspection.  In fact, the extended theory with this
%definition has no more and no less information than the original theory; it just
%expresses certain theorems of the form ``$P\rightarrow P$''
%in a more compact way.
%
%The point here is that what constitutes a proper definition is a matter of
%judgment about whether a symbol sequence can easily be recognized by a human
%as invalid in some sense (for example, not a wff); if so, the symbol sequence
%can be appropriated for use by a definition in order to make the extended
%language more compact.  Metamath\index{Metamath} lacks the ability to make this
%judgment, since as far as Metamath is concerned the definition of a wff, for
%example, is arbitrary.  You define for Metamath how wffs\index{well-formed
%formula (wff)} are constructed according to your own preferred style.  The
%concept of a wff may not even exist in a given formal system\index{formal
%system}.  Metamath treats all definitions as if they were new axioms, and it
%is up to the human mathematician to judge whether the definition is ``proper''
%'\index{proper definition}\index{definition!proper} in some agreed-upon way.

What constitutes a definition\index{definition} versus\index{axiom vs.\
definition} an axiom\index{axiom} is sometimes arbitrary in mathematical
literature.  For example, the connectives $\vee$ ({\sc or}), $\wedge$
({\sc and}), and $\leftrightarrow$ (equivalent to) in propositional
calculus are usually considered defined symbols that can be used as
abbreviations for expressions containing the ``primitive'' connectives
$\rightarrow$ and $\neg$.  This is the way we treat them in the standard
logic and set theory database \texttt{set.mm}\index{set theory database
(\texttt{set.mm})}.  However, the first three connectives can also be
considered ``primitive,'' and axiom systems have been devised that treat
all of them as such.  For example,
\cite[p.~35]{Goodstein}\index{Goodstein, R. L.} presents one with 15
axioms, some of which in fact coincide with what we have chosen to call
definitions in \texttt{set.mm}.  In certain subsets of classical
propositional calculus, such as the intuitionist
fragment\index{intuitionism}, it can be shown that one cannot make do
with just $\rightarrow$ and $\neg$ but must treat additional connectives
as primitive in order for the system to make sense.\footnote{Two nice
systems that make the transition from intuitionistic and other weak
fragments to classical logic just by adding axioms are given in
\cite{Robinsont}\index{Robinson, T. Thacher}.}

\subsection{The Approach to Definitions in \texttt{set.mm}}

In set theory, recursive definitions define a newly introduced symbol in
terms of itself.
The justification of recursive definitions, using
several ``recursion theorems,'' is usually one of the first
sophisticated proofs a student encounters when learning set theory, and
there is a significant amount of implicit metalogic behind a recursive
definition even though the definition itself is typically simple to
state.

Metamath itself has no built-in technical limitation that prevents
multiple-part recursive definitions in the traditional textbook style.
However, because the recursive definition requires advanced metalogic
to justify, eliminating a recursive definition is very difficult and
often not even shown in textbooks.

\subsubsection{Direct definitions instead of recursive definitions}

It is, however, possible to substitute one kind of complexity
for another.  We can eliminate the need for metalogical justification by
defining the operation directly with an explicit (but complicated)
expression, then deriving the recursive definition directly as a
theorem, using a recursion theorem ``in reverse.''
The elimination
of a direct definition is a matter of simple mechanical substitution.
We do this in
\texttt{set.mm}, as follows.

In \texttt{set.mm} our goal was to introduce almost all definitions in
the form of two expressions connected by either $\leftrightarrow$ or
$=$, where the thing being defined does not appear on the right hand
side.  Quine calls this form ``a genuine or direct definition'' \cite[p.
174]{Quine}\index{Quine, Willard Van Orman}, which makes the definitions
very easy to eliminate and the metalogic\index{metalogic} needed to
justify them as simple as possible.
Put another way, we had a goal of being able to
eliminate all definitions with direct mechanical substitution and to
verify easily the soundness of the definitions.

\subsubsection{Example of direct definitions}

We achieved this goal in almost all cases in \texttt{set.mm}.
Sometimes this makes the definitions more complex and less
intuitive.
For example, the traditional way to define addition of
natural numbers is to define an operation called {\em
successor}\index{successor} (which means ``plus one'' and is denoted by
``${\rm suc}$''), then define addition recursively\index{recursive
definition} with the two definitions $n + 0 = n$ and $m + {\rm suc}\,n =
{\rm suc} (m + n)$.  Although this definition seems simple and obvious,
the method to eliminate the definition is not obvious:  in the second
part of the definition, addition is defined in terms of itself.  By
eliminating the definition, we don't mean repeatedly applying it to
specific $m$ and $n$ but rather showing the explicit, closed-form
set-theoretical expression that $m + n$ represents, that will work for
any $m$ and $n$ and that does not have a $+$ sign on its right-hand
side.  For a recursive definition like this not to be circular
(creative), there are some hidden, underlying assumptions we must make,
for example that the natural numbers have a certain kind of order.

In \texttt{set.mm} we chose to start with the direct (though complex and
nonintuitive) definition then derive from it the standard recursive
definition.
For example, the closed-form definition used in \texttt{set.mm}
for the addition operation on ordinals\index{ordinal
addition}\index{addition!of ordinals} (of which natural numbers are a
subset) is

\setbox\startprefix=\hbox{\tt \ \ df-oadd\ \$a\ }
\setbox\contprefix=\hbox{\tt \ \ \ \ \ \ \ \ \ \ \ \ \ }
\startm
\m{\vdash}\m{+_o}\m{=}\m{(}\m{x}\m{\in}\m{{\rm On}}\m{,}\m{y}\m{\in}\m{{\rm
On}}\m{\mapsto}\m{(}\m{{\rm rec}}\m{(}\m{(}\m{z}\m{\in}\m{{\rm
V}}\m{\mapsto}\m{{\rm suc}}\m{z}\m{)}\m{,}\m{x}\m{)}\m{`}\m{y}\m{)}\m{)}
\endm
\noindent which depends on ${\rm rec}$.

\subsubsection{Recursion operators}

The above definition of \texttt{df-oadd} depends on the definition of
${\rm rec}$, a ``recursion operator''\index{recursion operator} with
the definition \texttt{df-rdg}:

\setbox\startprefix=\hbox{\tt \ \ df-rdg\ \$a\ }
\setbox\contprefix=\hbox{\tt \ \ \ \ \ \ \ \ \ \ \ \ }
\startm
\m{\vdash}\m{{\rm
rec}}\m{(}\m{F}\m{,}\m{I}\m{)}\m{=}\m{\mathrm{recs}}\m{(}\m{(}\m{g}\m{\in}\m{{\rm
V}}\m{\mapsto}\m{{\rm if}}\m{(}\m{g}\m{=}\m{\varnothing}\m{,}\m{I}\m{,}\m{{\rm
if}}\m{(}\m{{\rm Lim}}\m{{\rm dom}}\m{g}\m{,}\m{\bigcup}\m{{\rm
ran}}\m{g}\m{,}\m{(}\m{F}\m{`}\m{(}\m{g}\m{`}\m{\bigcup}\m{{\rm
dom}}\m{g}\m{)}\m{)}\m{)}\m{)}\m{)}\m{)}
\endm

\noindent which can be further broken down with definitions shown in
Section~\ref{setdefinitions}.

This definition of ${\rm rec}$
defines a recursive definition generator on ${\rm On}$ (the class of ordinal
numbers) with characteristic function $F$ and initial value $I$.
This operation allows us to define, with
compact direct definitions, functions that are usually defined in
textbooks with recursive definitions.
The price paid with our approach
is the complexity of our ${\rm rec}$ operation
(especially when {\tt df-recs} that it is built on is also eliminated).
But once we get past this hurdle, definitions that would otherwise be
recursive become relatively simple, as in for example {\tt oav}, from
which we prove the recursive textbook definition as theorems {\tt oa0}, {\tt
oasuc}, and {\tt oalim} (with the help of theorems {\tt rdg0}, {\tt rdgsuc},
and {\tt rdglim2a}).  We can also restrict the ${\rm rec}$ operation to
define otherwise recursive functions on the natural numbers $\omega$; see {\tt
fr0g} and {\tt frsuc}.  Our ${\rm rec}$ operation apparently does not appear
in published literature, although closely related is Definition 25.2 of
[Quine] p. 177, which he uses to ``turn...a recursion into a genuine or
direct definition" (p. 174).  Note that the ${\rm if}$ operations (see
{\tt df-if}) select cases based on whether the domain of $g$ is zero, a
successor, or a limit ordinal.

An important use of this definition ${\rm rec}$ is in the recursive sequence
generator {\tt df-seq} on the natural numbers (as a subset of the
complex infinite sequences such as the factorial function {\tt df-fac} and
integer powers {\tt df-exp}).

The definition of ${\rm rec}$ depends on ${\rm recs}$.
New direct usage of the more powerful (and more primitive) ${\rm recs}$
construct is discouraged, but it is available when needed.
This
defines a function $\mathrm{recs} ( F )$ on ${\rm On}$, the class of ordinal
numbers, by transfinite recursion given a rule $F$ which sets the next
value given all values so far.
Unlike {\tt df-rdg} which restricts the
update rule to use only the previous value, this version allows the
update rule to use all previous values, which is why it is described
as ``strong,'' although it is actually more primitive.  See {\tt
recsfnon} and {\tt recsval} for the primary contract of this definition.
It is defined as:

\setbox\startprefix=\hbox{\tt \ \ df-recs\ \$a\ }
\setbox\contprefix=\hbox{\tt \ \ \ \ \ \ \ \ \ \ \ \ \ }
\startm
\m{\vdash}\m{\mathrm{recs}}\m{(}\m{F}\m{)}\m{=}\m{\bigcup}\m{\{}\m{f}\m{|}\m{\exists}\m{x}\m{\in}\m{{\rm
On}}\m{(}\m{f}\m{{\rm
Fn}}\m{x}\m{\wedge}\m{\forall}\m{y}\m{\in}\m{x}\m{(}\m{f}\m{`}\m{y}\m{)}\m{=}\m{(}\m{F}\m{`}\m{(}\m{f}\m{\restriction}\m{y}\m{)}\m{)}\m{)}\m{\}}
\endm

\subsubsection{Closing comments on direct definitions}

From these direct definitions the simpler, more
intuitive recursive definition is derived as a set of theorems.\index{natural
number}\index{addition}\index{recursive definition}\index{ordinal addition}
The end result is the same, but we completely eliminate the rather complex
metalogic that justifies the recursive definition.

Recursive definitions are often considered more efficient and intuitive than
direct ones once the metalogic has been learned or possibly just accepted as
correct.  However, it was felt that direct definition in \texttt{set.mm}
maximizes rigor by minimizing metalogic.  It can be eliminated effortlessly,
something that is difficult to do with a recursive definition.

Again, Metamath itself has no built-in technical limitation that prevents
multiple-part recursive definitions in the traditional textbook style.
Instead, our goal is to eliminate all definitions with
direct mechanical substitution and to verify easily the soundness of
definitions.

\subsection{Adding Constraints on Definitions}

The basic Metamath language and the Metamath program do
not have any built-in constraints on definitions, since they are just
\$a statements.

However, nothing prevents a verification system from verifying
additional rules to impose further limitations on definitions.
For example, the \texttt{mmj2}\index{mmj2} program
supports various kinds of
additional information comments (see section \ref{jcomment}).
One of their uses is to optionally verify additional constraints,
including constraints to verify that definitions meet certain
requirements.
These additional checks are required by the
continuous integration (CI)\index{continuous integration (CI)}
checks of the
\texttt{set.mm}\index{set theory database (\texttt{set.mm})}%
\index{Metamath Proof Explorer}
database.
This approach enables us to optionally impose additional requirements
on definitions if we wish, without necessarily imposing those rules on
all databases or requiring all verification systems to implement them.
In addition, this allows us to impose specialized constraints tailored
to one database while not requiring other systems to implement
those specialized constraints.

We impose two constraints on the
\texttt{set.mm}\index{set theory database (\texttt{set.mm})}%
\index{Metamath Proof Explorer} database
via the \texttt{mmj2}\index{mmj2} program that are worth discussing here:
a parse check and a definition soundness check.

% On February 11, 2019 8:32:32 PM EST, saueran@oregonstate.edu wrote:
% The following addition to the end of set.mm is accepted by the mmj2
% parser and definition checker and the metamath verifier(at least it was
% when I checked, you should check it too), and creates a contradiction by
% proving the theorem |- ph.
% ${
% wleftp $a wff ( ( ph ) $.
% wbothp $a wff ( ph ) $.
% df-leftp $a |- ( ( ( ph ) <-> -. ph ) $.
% df-bothp $a |- ( ( ph ) <-> ph ) $.
% anything $p |- ph $=
%   ( wbothp wn wi wleftp df-leftp biimpi df-bothp mpbir mpbi simplim ax-mp)
%   ABZAMACZDZCZMOEZOCQAEZNDZRNAFGSHIOFJMNKLAHJ $.
% $}
%
% This particular problem is countered by enabling, within mmj2,
% SetParser,mmj.verify.LRParser

First,
we enable a parse check in \texttt{mmj2} (through its
\texttt{SetParser} command) that requires that all new definitions
must \textit{not} create an ambiguous parse for a KLR(5) parser.
This prevents some errors such as definitions with imbalanced parentheses.

Second, we run a definition soundness check specific to
\texttt{set.mm} or databases similar to it.
(through the \texttt{definitionCheck} macro).
Some \texttt{\$a} statements (including all ax-* statemnets)
are excluded from these checks, as they will
always fail this simple check,
but they are appropriate for most definitions.
This check imposes a set of additional rules:

\begin{enumerate}

\item New definitions must be introduced using $=$ or $\leftrightarrow$.

\item No \texttt{\$a} statement introduced before this one is allowed to use the
symbol being defined in this definition, and the definition is not
allowed to use itself (except once, in the definiendum).

\item Every variable in the definiens should not be distinct

\item Every dummy variable in the definiendum
are required to be distinct from each other and from variables in
the definiendum.
To determine this, the system will look for a "justification" theorem
in the database, and if it is not there, attempt to briefly prove
$( \varphi \rightarrow \forall x \varphi )$  for each dummy variable x.

\item Every dummy variable should be a set variable,
unless there is a justification theorem available.

\item Every dummy variable must be bound
(if the system cannot determine this a justification theorem must be
provided).

\end{enumerate}

\subsection{Summary of Approach to Definitions}

In short, when being rigorous it turns out that
definitions can be subtle, sometimes requiring difficult
metatheorems to establish that they are not creative.

Instead of building such complications into the Metamath language itself,
the basic Metmath language and program simply treat traditional
axioms and definitions as the same kind of \texttt{\$a} statement.
We have then built various tools to enable people to
verify additional conditions as their creators believe is appropriate
for those specific databases, without complicating the Metamath foundations.

\chapter{The Metamath Program}\label{commands}

This chapter provides a reference manual for the
Metamath program.\index{Metamath!commands}

Current instructions for obtaining and installing the Metamath program
can be found at the \url{http://metamath.org} web site.
For Windows, there is a pre-compiled version called
\texttt{metamath.exe}.  For Unix, Linux, and Mac OS X
(which we will refer to collectively as ``Unix''), the Metamath program
can be compiled from its source code with the command
\begin{verbatim}
gcc *.c -o metamath
\end{verbatim}
using the \texttt{gcc} {\sc c} compiler available on those systems.

In the command syntax descriptions below, fields enclosed in square
brackets [\ ] are optional.  File names may be optionally enclosed in
single or double quotes.  This is useful if the file name contains
spaces or
slashes (\texttt{/}), such as in Unix path names, \index{Unix file
names}\index{file names!Unix} that might be confused with Metamath
command qualifiers.\index{Unix file names}\index{file names!Unix}

\section{Invoking Metamath}

Unix, Linux, and Mac OS X
have a command-line interface called the {\em
bash shell}.  (In Mac OS X, select the Terminal application from
Applications/Utilities.)  To invoke Metamath from the bash shell prompt,
assuming that the Metamath program is in the current directory, type
\begin{verbatim}
bash$ ./metamath
\end{verbatim}

To invoke Metamath from a Windows DOS or Command Prompt, assuming that
the Metamath program is in the current directory (or in a directory
included in the Path system environment variable), type
\begin{verbatim}
C:\metamath>metamath
\end{verbatim}

To use command-line arguments at invocation, the command-line arguments
should be a list of Metamath commands, surrounded by quotes if they
contain spaces.  In Windows, the surrounding quotes must be double (not
single) quotes.  For example, to read the database file \texttt{set.mm},
verify all proofs, and exit the program, type (under Unix)
\begin{verbatim}
bash$ ./metamath 'read set.mm' 'verify proof *' exit
\end{verbatim}
Note that in Unix, any directory path with \texttt{/}'s must be
surrounded by quotes so Metamath will not interpret the \texttt{/} as a
command qualifier.  So if \texttt{set.mm} is in the \texttt{/tmp}
directory, use for the above example
\begin{verbatim}
bash$ ./metamath 'read "/tmp/set.mm"' 'verify proof *' exit
\end{verbatim}

For convenience, if the command-line has one argument and no spaces in
the argument, the command is implicitly assumed to be \texttt{read}.  In
this one special case, \texttt{/}'s are not interpreted as command
qualifiers, so you don't need quotes around a Unix file name.  Thus
\begin{verbatim}
bash$ ./metamath /tmp/set.mm
\end{verbatim}
and
\begin{verbatim}
bash$ ./metamath "read '/tmp/set.mm'"
\end{verbatim}
are equivalent.


\section{Controlling Metamath}

The Metamath program was first developed on a {\sc vax/vms} system, and
some aspects of its command line behavior reflect this heritage.
Hopefully you will find it reasonably user-friendly once you get used to
it.

Each command line is a sequence of English-like words separated by
spaces, as in \texttt{show settings}.  Command words are not case
sensitive, and only as many letters are needed as are necessary to
eliminate ambiguity; for example, \texttt{sh se} would work for the
command \texttt{show settings}.  In some cases arguments such as file
names, statement labels, or symbol names are required; these are
case-sensitive (although file names may not be on some operating
systems).

A command line is entered by typing it in then pressing the {\em return}
({\em enter}) key.  To find out what commands are available, type
\texttt{?} at the \texttt{MM>} prompt.  To find out the choices at any
point in a command, press {\em return} and you will be prompted for
them.  The default choice (the one selected if you just press {\em
return}) is shown in brackets (\texttt{<>}).

You may also type \texttt{?} in place of a command word to force
Metamath to tell you what the choices are.  The \texttt{?} method won't
work, though, if a non-keyword argument such as a file name is expected
at that point, because the program will think that \texttt{?} is the
value of the argument.

Some commands have one or more optional qualifiers which modify the
behavior of the command.  Qualifiers are preceded by a slash
(\texttt{/}), such as in \texttt{read set.mm / verify}.  Spaces are
optional around the \texttt{/}.  If you need to use a space or
slash in a command
argument, as in a Unix file name, put single or double quotes around the
command argument.

The \texttt{open log} command will save everything you see on the
screen and is useful to help you recover should something go wrong in a
proof, or if you want to document a bug.

If a command responds with more than a screenful, you will be
prompted to \texttt{<return> to continue, Q to quit, or S to scroll to
end}.  \texttt{Q} or \texttt{q} (not case-sensitive) will complete the
command internally but will suppress further output until the next
\texttt{MM>} prompt.  \texttt{s} will suppress further pausing until the
next \texttt{MM>} prompt.  After the first screen, you are also
presented with the choice of \texttt{b} to go back a screenful.  Note
that \texttt{b} may also be entered at the \texttt{MM>} prompt
immediately after a command to scroll back through the output of that
command.

A command line enclosed in quotes is executed by your operating system.
See Section~\ref{oscmd}.

{\em Warning:} Pressing {\sc ctrl-c} will abort the Metamath program
unconditionally.  This means any unsaved work will be lost.


\subsection{\texttt{exit} Command}\index{\texttt{exit} command}

Syntax:  \texttt{exit} [\texttt{/force}]

This command exits from Metamath.  If there have been changes to the
source with the \texttt{save proof} or \texttt{save new{\char`\_}proof}
commands, you will be given an opportunity to \texttt{write source} to
permanently save the changes.

In Proof Assistant\index{Proof Assistant} mode, the \texttt{exit} command will
return to the \verb/MM>/ prompt. If there were changes to the proof, you will
be given an opportunity to \texttt{save new{\char`\_}proof}.

The \texttt{quit} command is a synonym for \texttt{exit}.

Optional qualifier:
    \texttt{/force} - Do not prompt if changes were not saved.  This qualifier is
        useful in \texttt{submit} command files (Section~\ref{sbmt})
        to ensure predictable behavior.





\subsection{\texttt{open log} Command}\index{\texttt{open log} command}
Syntax:  \texttt{open log} {\em file-name}

This command will open a log file that will store everything you see on
the screen.  It is useful to help recovery from a mistake in a long Proof
Assistant session, or to document bugs.\index{Metamath!bugs}

The log file can be closed with \texttt{close log}.  It will automatically be
closed upon exiting Metamath.



\subsection{\texttt{close log} Command}\index{\texttt{close log} command}
Syntax:  \texttt{close log}

The \texttt{close log} command closes a log file if one is open.  See
also \texttt{open log}.




\subsection{\texttt{submit} Command}\index{\texttt{submit} command}\label{sbmt}
Syntax:  \texttt{submit} {\em filename}

This command causes further command lines to be taken from the specified
file.  Note that any line beginning with an exclamation point (\texttt{!}) is
treated as a comment (i.e.\ ignored).  Also note that the scrolling
of the screen output is continuous, so you may want to open a log file
(see \texttt{open log}) to record the results that fly by on the screen.
After the lines in the file are exhausted, Metamath returns to its
normal user interface mode.

The \texttt{submit} command can be called recursively (i.e. from inside
of a \texttt{submit} command file).


Optional command qualifier:

    \texttt{/silent} - suppresses the screen output but still
        records the output in a log file if one is open.


\subsection{\texttt{erase} Command}\index{\texttt{erase} command}
Syntax:  \texttt{erase}

This command will reset Metamath to its starting state, deleting any
data\-base that was \texttt{read} in.
 If there have been changes to the
source with the \texttt{save proof} or \texttt{save new{\char`\_}proof}
commands, you will be given an opportunity to \texttt{write source} to
permanently save the changes.



\subsection{\texttt{set echo} Command}\index{\texttt{set echo} command}
Syntax:  \texttt{set echo on} or \texttt{set echo off}

The \texttt{set echo on} command will cause command lines to be echoed with any
abbreviations expanded.  While learning the Metamath commands, this
feature will show you the exact command that your abbreviated input
corresponds to.



\subsection{\texttt{set scroll} Command}\index{\texttt{set scroll} command}
Syntax:  \texttt{set scroll prompted} or \texttt{set scroll continuous}

The Metamath command line interface starts off in the \texttt{prompted} mode,
which means that you will be prompted to continue or quit after each
full screen in a long listing.  In \texttt{continuous} mode, long listings will be
scrolled without pausing.

% LaTeX bug? (1) \texttt{\_} puts out different character than
% \texttt{{\char`\_}}
%  = \verb$_$  (2) \texttt{{\char`\_}} puts out garbage in \subsection
%  argument
\subsection{\texttt{set width} Command}\index{\texttt{set
width} command}
Syntax:  \texttt{set width} {\em number}

Metamath assumes the width of your screen is 79 characters (chosen
because the Command Prompt in Windows XP has a wrapping bug at column
80).  If your screen is wider or narrower, this command allows you to
change this default screen width.  A larger width is advantageous for
logging proofs to an output file to be printed on a wide printer.  A
smaller width may be necessary on some terminals; in this case, the
wrapping of the information messages may sometimes seem somewhat
unnatural, however.  In \LaTeX\index{latex@{\LaTeX}!characters per line},
there is normally a maximum of 61 characters per line with typewriter
font.  (The examples in this book were produced with 61 characters per
line.)

\subsection{\texttt{set height} Command}\index{\texttt{set
height} command}
Syntax:  \texttt{set height} {\em number}

Metamath assumes your screen height is 24 lines of characters.  If your
screen is taller or shorter, this command lets you to change the number
of lines at which the display pauses and prompts you to continue.

\subsection{\texttt{beep} Command}\index{\texttt{beep} command}

Syntax:  \texttt{beep}

This command will produce a beep.  By typing it ahead after a
long-running command has started, it will alert you that the command is
finished.  For convenience, \texttt{b} is an abbreviation for
\texttt{beep}.

Note:  If \texttt{b} is typed at the \texttt{MM>} prompt immediately
after the end of a multiple-page display paged with ``\texttt{Press
<return> for more}...'' prompts, then the \texttt{b} will back up to the
previous page rather than perform the \texttt{beep} command.
In that case you must type the unabbreviated \texttt{beep} form
of the command.

\subsection{\texttt{more} Command}\index{\texttt{more} command}

Syntax:  \texttt{more} {\em filename}

This command will display the contents of an {\sc ascii} file on your
screen.  (This command is provided for convenience but is not very
powerful.  See Section~\ref{oscmd} to invoke your operating system's
command to do this, such as the \texttt{more} command in Unix.)

\subsection{Operating System Commands}\index{operating system
command}\label{oscmd}

A line enclosed in single or double quotes will be executed by your
computer's operating system if it has a command line interface.  For
example, on a {\sc vax/vms} system,
\verb/MM> 'dir'/
will print disk directory contents.  Note that this feature will not
work on the Macintosh prior to Mac OS X, which does not have a command
line interface.

For your convenience, the trailing quote is optional.

\subsection{Size Limitations in Metamath}

In general, there are no fixed, predefined limits\index{Metamath!memory
limits} on how many labels, tokens\index{token}, statements, etc.\ that
you may have in a database file.  The Metamath program uses 32-bit
variables (64-bit on 64-bit CPUs) as indices for almost all internal
arrays, which are allocated dynamically as needed.



\section{Reading and Writing Files}

The following commands create new files:  the \texttt{open} commands;
the \texttt{write} commands; the \texttt{/html},
\texttt{/alt{\char`\_}html}, \texttt{/brief{\char`\_}html},
\texttt{/brief{\char`\_}alt{\char`\_}html} qualifiers of \texttt{show
statement}, and \texttt{midi}.  The following commands append to files
previously opened:  the \texttt{/tex} qualifier of \texttt{show proof}
and \texttt{show new{\char`\_}proof}; the \texttt{/tex} and
\texttt{/simple{\char`\_}tex} qualifiers of \texttt{show statement}; the
\texttt{close} commands; and all screen dialog between \texttt{open log}
and \texttt{close log}.

The commands that create new files will not overwrite an existing {\em
filename} but will rename the existing one to {\em
filename}\texttt{{\char`\~}1}.  An existing {\em
filename}\texttt{{\char`\~}1} is renamed {\em
filename}\texttt{{\char`\~}2}, etc.\ up to {\em
filename}\texttt{{\char`\~}9}.  An existing {\em
filename}\texttt{{\char`\~}9} is deleted.  This makes recovery from
mistakes easier but also will clutter up your directory, so occasionally
you may want to clean up (delete) these \texttt{{\char`\~}}$n$ files.


\subsection{\texttt{read} Command}\index{\texttt{read} command}
Syntax:  \texttt{read} {\em file-name} [\texttt{/verify}]

This command will read in a Metamath language source file and any included
files.  Normally it will be the first thing you do when entering Metamath.
Statement syntax is checked, but proof syntax is not checked.
Note that the file name may be enclosed in single or double quotes;
this is useful if the file name contains slashes, as might be the case
under Unix.

If you are getting an ``\texttt{?Expected VERIFY}'' error
when trying to read a Unix file name with slashes, you probably haven't
quoted it.\index{Unix file names}\index{file names!Unix}

If you are prompted for the file name (by pressing {\em return}
 after \texttt{read})
you should {\em not} put quotes around it, even if it is a Unix file name
with slashes.

Optional command qualifier:

    \texttt{/verify} - Verify all proofs as the database is read in.  This
         qualifier will slow down reading in the file.  See \texttt{verify
         proof} for more information on file error-checking.

See also \texttt{erase}.



\subsection{\texttt{write source} Command}\index{\texttt{write source} command}
Syntax:  \texttt{write source} {\em filename}
[\texttt{/rewrap}]
[\texttt{/split}]
% TeX doesn't handle this long line with tt text very well,
% so force a line break here.
[\texttt{/keep\_includes}] {\\}
[\texttt{/no\_versioning}]

This command will write the contents of a Metamath\index{database}
database into a file.\index{source file}

Optional command qualifiers:

\texttt{/rewrap} -
Reformats statements and comments according to the
convention used in the set.mm database.
It unwraps the
lines in the comment before each \texttt{\$a} and \texttt{\$p} statement,
then it
rewraps the line.  You should compare the output to the original
to make sure that the desired effect results; if not, go back to
the original source.  The wrapped line length honors the
\texttt{set width}
parameter currently in effect.  Note:  Text
enclosed in \texttt{<HTML>}...\texttt{</HTML>} tags is not modified by the
\texttt{/rewrap} qualifier.
Proofs themselves are not reformatted;
use \texttt{save proof * / compressed} to do that.
An isolated tilde (\~{}) is always kept on the same line as the following
symbol, so you can find all comment references to a symbol by
searching for \~{} followed by a space and that symbol
(this is useful for finding cross-references).
Incidentally, \texttt{save proof} also honors the \texttt{set width}
parameter currently in effect.

\texttt{/split} - Files included in the source using the expression
\$[ \textit{inclfile} \$] will be
written into separate files instead of being included in a single output
file.  The name of each separately written included file will be the
\textit{inclfile} argument of its inclusion command.

\texttt{/keep\_includes} - If a source file has includes but is written as a
single file by omitting \texttt{/split}, by default the included files will
be deleted (actually just renamed with a \char`\~1 suffix unless
\texttt{/no\_versioning} is specified) to prevent the possibly confusing
source duplication in both the output file and the included file.
The \texttt{/keep\_includes} qualifier will prevent this deletion.

\texttt{/no\_versioning} - Backup files suffixed with \char`\~1 are not created.


\section{Showing Status and Statements}



\subsection{\texttt{show settings} Command}\index{\texttt{show settings} command}
Syntax:  \texttt{show settings}

This command shows the state of various parameters.

\subsection{\texttt{show memory} Command}\index{\texttt{show memory} command}
Syntax:  \texttt{show memory}

This command shows the available memory left.  It is not meaningful
on most modern operating systems,
which have virtual memory.\index{Metamath!memory usage}


\subsection{\texttt{show labels} Command}\index{\texttt{show labels} command}
Syntax:  \texttt{show labels} {\em label-match} [\texttt{/all}]
   [\texttt{/linear}]

This command shows the labels of \texttt{\$a} and \texttt{\$p}
statements that match {\em label-match}.  A \verb$*$ in {label-match}
matches zero or more characters.  For example, \verb$*abc*def$ will match all
labels containing \verb$abc$ and ending with \verb$def$.

Optional command qualifiers:

   \texttt{/all} - Include matches for \texttt{\$e} and \texttt{\$f}
   statement labels.

   \texttt{/linear} - Display only one label per line.  This can be useful for
       building scripts in conjunction with the utilities under the
       \texttt{tools}\index{\texttt{tools} command} command.



\subsection{\texttt{show statement} Command}\index{\texttt{show statement} command}
Syntax:  \texttt{show statement} {\em label-match} [{\em qualifiers} (see below)]

This command provides information about a statement.  Only statements
that have labels (\texttt{\$f}\index{\texttt{\$f} statement},
\texttt{\$e}\index{\texttt{\$e} statement},
\texttt{\$a}\index{\texttt{\$a} statement}, and
\texttt{\$p}\index{\texttt{\$p} statement}) may be specified.
If {\em label-match}
contains wildcard (\verb$*$) characters, all matching statements will be
displayed in the order they occur in the database.

Optional qualifiers (only one qualifier at a time is allowed):

    \texttt{/comment} - This qualifier includes the comment that immediately
       precedes the statement.

    \texttt{/full} - Show complete information about each statement,
       and show all
       statements matching {\em label} (including \texttt{\$e}
       and \texttt{\$f} statements).

    \texttt{/tex} - This qualifier will write the statement information to the
       \LaTeX\ file previously opened with \texttt{open tex}.  See
       Section~\ref{texout}.

    \texttt{/simple{\char`\_}tex} - The same as \texttt{/tex}, except that
       \LaTeX\ macros are not used for formatting equations, allowing easier
       manual edits of the output for slide presentations, etc.

    \texttt{/html}\index{html generation@{\sc html} generation},
       \texttt{/alt{\char`\_}html}, \texttt{/brief{\char`\_}html},
       \texttt{/brief{\char`\_}alt{\char`\_}html} -
       These qualifiers invoke a special mode of
       \texttt{show statement} that
       creates a web page for the statement.  They may not be used with
       any other qualifier.  See Section~\ref{htmlout} or
       \texttt{help html} in the program.


\subsection{\texttt{search} Command}\index{\texttt{search} command}
Syntax:  search {\em label-match}
\texttt{"}{\em symbol-match}{\tt}" [\texttt{/all}] [\texttt{/comments}]
[\texttt{/join}]

This command searches all \texttt{\$a} and \texttt{\$p} statements
matching {\em label-match} for occurrences of {\em symbol-match}.  A
\verb@*@ in {\em label-match} matches any label character.  A \verb@$*@
in {\em symbol-match} matches any sequence of symbols.  The symbols in
{\em symbol-match} must be separated by white space.  The quotes
surrounding {\em symbol-match} may be single or double quotes.  For
example, \texttt{search b}\verb@* "-> $* ch"@ will list all statements
whose labels begin with \texttt{b} and contain the symbols \verb@->@ and
\texttt{ch} surrounding any symbol sequence (including no symbol
sequence).  The wildcards \texttt{?} and \texttt{\$?} are also available
to match individual characters in labels and symbols respectively; see
\texttt{help search} in the Metamath program for details on their usage.

Optional command qualifiers:

    \texttt{/all} - Also search \texttt{\$e} and \texttt{\$f} statements.

    \texttt{/comments} - Search the comment that immediately precedes each
        label-matched statement for {\em symbol-match}.  In this case
        {\em symbol-match} is an arbitrary, non-case-sensitive character
        string.  Quotes around {\em symbol-match} are optional if there
        is no ambiguity.

    \texttt{/join} - In the case of a \texttt{\$a} or \texttt{\$p} statement,
	prepend its \texttt{\$e}
	hypotheses for searching. The
	\texttt{/join} qualifier has no effect in \texttt{/comments} mode.

\section{Displaying and Verifying Proofs}


\subsection{\texttt{show proof} Command}\index{\texttt{show proof} command}
Syntax:  \texttt{show proof} {\em label-match} [{\em qualifiers} (see below)]

This command displays the proof of the specified
\texttt{\$p}\index{\texttt{\$p} statement} statement in various formats.
The {\em label-match} may contain wildcard (\verb@$*@) characters to match
multiple statements.  Without any qualifiers, only the logical steps
will be shown (i.e.\ syntax construction steps will be omitted), in an
indented format.

Most of the time, you will use
    \texttt{show proof} {\em label}
to see just the proof steps corresponding to logical inferences.

Optional command qualifiers:

    \texttt{/essential} - The proof tree is trimmed of all
        \texttt{\$f}\index{\texttt{\$f} statement} hypotheses before
        being displayed.  (This is the default, and it is redundant to
        specify it.)

    \texttt{/all} - the proof tree is not trimmed of all \texttt{\$f} hypotheses before
        being displayed.  \texttt{/essential} and \texttt{/all} are mutually exclusive.

    \texttt{/from{\char`\_}step} {\em step} - The display starts at the specified
        step.  If
        this qualifier is omitted, the display starts at the first step.

    \texttt{/to{\char`\_}step} {\em step} - The display ends at the specified
        step.  If this
        qualifier is omitted, the display ends at the last step.

    \texttt{/tree{\char`\_}depth} {\em number} - Only
         steps at less than the specified proof
        tree depth are displayed.  Sometimes useful for obtaining an overview of
        the proof.

    \texttt{/reverse} - The steps are displayed in reverse order.

    \texttt{/renumber} - When used with \texttt{/essential}, the steps are renumbered
        to correspond only to the essential steps.

    \texttt{/tex} - The proof is converted to \LaTeX\ and\index{latex@{\LaTeX}}
        stored in the file opened
        with \texttt{open tex}.  See Section~\ref{texout} or
        \texttt{help tex} in the program.

    \texttt{/lemmon} - The proof is displayed in a non-indented format known
        as Lemmon style, with explicit previous step number references.
        If this qualifier is omitted, steps are indented in a tree format.

    \texttt{/start{\char`\_}column} {\em number} - Overrides the default column
        (16)
        at which the formula display starts in a Lemmon-style display.  May be
        used only in conjunction with \texttt{/lemmon}.

    \texttt{/normal} - The proof is displayed in normal format suitable for
        inclusion in a Metamath source file.  May not be used with any other
        qualifier.

    \texttt{/compressed} - The proof is displayed in compressed format
        suitable for inclusion in a Metamath source file.  May not be used with
        any other qualifier.

    \texttt{/statement{\char`\_}summary} - Summarizes all statements (like a
        brief \texttt{show statement})
        used by the proof.  It may not be used with any other qualifier
        except \texttt{/essential}.

    \texttt{/detailed{\char`\_}step} {\em step} - Shows the details of what is
        happening at
        a specific proof step.  May not be used with any other qualifier.
        The {\em step} is the step number shown when displaying a
        proof without the \texttt{/renumber} qualifier.


\subsection{\texttt{show usage} Command}\index{\texttt{show usage} command}
Syntax:  \texttt{show usage} {\em label-match} [\texttt{/recursive}]

This command lists the statements whose proofs make direct reference to
the statement specified.

Optional command qualifier:

    \texttt{/recursive} - Also include statements whose proofs ultimately
        depend on the statement specified.



\subsection{\texttt{show trace\_back} Command}\index{\texttt{show
       trace{\char`\_}back} command}
Syntax:  \texttt{show trace{\char`\_}back} {\em label-match} [\texttt{/essential}] [\texttt{/axioms}]
    [\texttt{/tree}] [\texttt{/depth} {\em number}]

This command lists all statements that the proof of the \texttt{\$p}
statement(s) specified by {\em label-match} depends on.

Optional command qualifiers:

    \texttt{/essential} - Restrict the trace-back to \texttt{\$e}
        \index{\texttt{\$e} statement} hypotheses of proof trees.

    \texttt{/axioms} - List only the axioms that the proof ultimately depends on.

    \texttt{/tree} - Display the trace-back in an indented tree format.

    \texttt{/depth} {\em number} - Restrict the \texttt{/tree} trace-back to the
        specified indentation depth.

    \texttt{/count{\char`\_}steps} - Count the number of steps the proof has
       all the way back to axioms.  If \texttt{/essential} is specified,
       expansions of variable-type hypotheses (syntax constructions) are not counted.

\subsection{\texttt{verify proof} Command}\index{\texttt{verify proof} command}
Syntax:  \texttt{verify proof} {\em label-match} [\texttt{/syntax{\char`\_}only}]

This command verifies the proofs of the specified statements.  {\em
label-match} may contain wild card characters (\texttt{*}) to verify
more than one proof; for example \verb/*abc*def/ will match all labels
containing \texttt{abc} and ending with \texttt{def}.
The command \texttt{verify proof *} will verify all proofs in the database.

Optional command qualifier:

    \texttt{/syntax{\char`\_}only} - This qualifier will perform a check of syntax
        and RPN
        stack violations only.  It will not verify that the proof is
        correct.  This qualifier is useful for quickly determining which
        proofs are incomplete (i.e.\ are under development and have \texttt{?}'s
        in them).

{\em Note:} \texttt{read}, followed by \texttt{verify proof *}, will ensure
 the database is free
from errors in the Metamath language but will not check the markup notation
in comments.
You can also check the markup notation by running \texttt{verify markup *}
as discussed in Section~\ref{verifymarkup}; see also the discussion
on generating {\sc HTML} in Section~\ref{htmlout}.

\subsection{\texttt{verify markup} Command}\index{\texttt{verify markup} command}\label{verifymarkup}
Syntax:  \texttt{verify markup} {\em label-match}
[\texttt{/date{\char`\_}skip}]
[\texttt{/top{\char`\_}date{\char`\_}skip}] {\\}
[\texttt{/file{\char`\_}skip}]
[\texttt{/verbose}]

This command checks comment markup and other informal conventions we have
adopted.  It error-checks the latexdef, htmldef, and althtmldef statements
in the \texttt{\$t} statement of a Metamath source file.\index{error checking}
It error-checks any \texttt{`}...\texttt{`},
\texttt{\char`\~}~\textit{label},
and bibliographic markups in statement descriptions.
It checks that
\texttt{\$p} and \texttt{\$a} statements
have the same content when their labels start with
``ax'' and ``ax-'' respectively but are otherwise identical, for example
ax4 and ax-4.
It also verifies the date consistency of ``(Contributed by...),''
``(Revised by...),'' and ``(Proof shortened by...)'' tags in the comment
above each \texttt{\$a} and \texttt{\$p} statement.

Optional command qualifiers:

    \texttt{/date{\char`\_}skip} - This qualifier will
        skip date consistency checking,
        which is usually not required for databases other than
	\texttt{set.mm}.

    \texttt{/top{\char`\_}date{\char`\_}skip} - This qualifier will check date consistency except
        that the version date at the top of the database file will not
        be checked.  Only one of
        \texttt{/date{\char`\_}skip} and
        \texttt{/top{\char`\_}date{\char`\_}skip} may be
        specified.

    \texttt{/file{\char`\_}skip} - This qualifier will skip checks that require
        external files to be present, such as checking GIF existence and
        bibliographic links to mmset.html or equivalent.  It is useful
        for doing a quick check from a directory without these files.

    \texttt{/verbose} - Provides more information.  Currently it provides a list
        of axXXX vs. ax-XXX matches.

\subsection{\texttt{save proof} Command}\index{\texttt{save proof} command}
Syntax:  \texttt{save proof} {\em label-match} [\texttt{/normal}]
   [\texttt{/compressed}]

The \texttt{save proof} command will reformat a proof in one of two formats and
replace the existing proof in the source buffer\index{source
buffer}.  It is useful for
converting between proof formats.  Note that a proof will not be
permanently saved until a \texttt{write source} command is issued.

Optional command qualifiers:

    \texttt{/normal} - The proof is saved in the normal format (i.e., as a
        sequence
        of labels, which is the defined format of the basic Metamath
        language).\index{basic language}  This is the default format that
        is used if a qualifier
        is omitted.

    \texttt{/compressed} - The proof is saved in the compressed format which
        reduces storage requirements for a database.
        See Appendix~\ref{compressed}.




\section{Creating Proofs}\label{pfcommands}\index{Proof Assistant}

Before using the Proof Assistant, you must add a \texttt{\$p} to your
source file (using a text editor) containing the statement you want to
prove.  Its proof should consist of a single \texttt{?}, meaning
``unknown step.''  Example:
\begin{verbatim}
equid $p x = x $= ? $.
\end{verbatim}

To enter the Proof assistant, type \texttt{prove} {\em label}, e.g.
\texttt{prove equid}.  Metamath will respond with the \texttt{MM-PA>}
prompt.

Proofs are created working backwards from the statement being proved,
primarily using a series of \texttt{assign} commands.  A proof is
complete when all steps are assigned to statements and all steps are
unified and completely known.  During the creation of a proof, Metamath
will allow only operations that are legal based on what is known up to
that point.  For example, it will not allow an \texttt{assign} of a
statement that cannot be unified with the unknown proof step being
assigned.

{\em Important:}
The Proof Assistant is
{\em not} a tool to help you discover proofs.  It is just a tool to help
you add them to the database.  For a tutorial read
Section~\ref{frstprf}.
To practice using the Proof Assistant, you may
want to \texttt{prove} an existing theorem, then delete all steps with
\texttt{delete all}, then re-create it with the Proof Assistant while
looking at its proof display (before deletion).
You might want to figure out your first few proofs completely
and write them down by hand, before using the Proof Assistant, though
not everyone finds that effective.

{\em Important:}
The \texttt{undo} command if very helpful when entering a proof, because
it allows you to undo a previously-entered step.
In addition, we suggest that you
keep track of your work with a log file (\texttt{open
log}) and save it frequently (\texttt{save new{\char`\_}proof},
\texttt{write source}).
You can use \texttt{delete} to reverse an \texttt{assign}.
You can also do \texttt{delete floating{\char`\_}hypotheses}, then
\texttt{initialize all}, then \texttt{unify all /interactive} to
reinitialize bad unifications made accidentally or by bad
\texttt{assign}s.  You cannot reverse a \texttt{delete} except by
a relevant \texttt{undo} or using
\texttt{exit /force} then reentering the Proof Assistant to recover from
the last \texttt{save new{\char`\_}proof}.

The following commands are available in the Proof Assistant (at the
\texttt{MM-PA>} prompt) to help you create your proof.  See the
individual commands for more detail.

\begin{itemize}
\item[]
    \texttt{show new{\char`\_}proof} [\texttt{/all},...] - Displays the
        proof in progress.  You will use this command a lot; see \texttt{help
        show new{\char`\_}proof} to become familiar with its qualifiers.  The
        qualifiers \texttt{/unknown} and \texttt{/not{\char`\_}unified} are
        useful for seeing the work remaining to be done.  The combination
        \texttt{/all/unknown} is useful for identifying dummy variables that must be
        assigned, or attempts to use illegal syntax, when \texttt{improve all}
        is unable to complete the syntax constructions.  Unknown variables are
        shown as \texttt{\$1}, \texttt{\$2},...
\item[]
    \texttt{assign} {\em step} {\em label} - Assigns an unknown {\em step}
        number with the statement
        specified by {\em label}.
\item[]
    \texttt{let variable} {\em variable}
        \texttt{= "}{\em symbol sequence}\texttt{"}
          - Forces a symbol
        sequence to replace an unknown variable (such as \texttt{\$1}) in a proof.
        It is useful
        for helping difficult unifications, and it is necessary when you have
        dummy variables that eventually must be assigned a name.
\item[]
    \texttt{let step} {\em step} \texttt{= "}{\em symbol sequence}\texttt{"} -
          Forces a symbol sequence
        to replace the contents of a proof step, provided it can be
        unified with the existing step contents.  (I rarely use this.)
\item[]
    \texttt{unify step} {\em step} (or \texttt{unify all}) - Unifies
        the source and target of
        a step.  If you specify a specific step, you will be prompted
        to select among the unifications that are possible.  If you
        specify \texttt{all}, all steps with unique unifications, but only
        those steps, will be
        unified.  \texttt{unify all /interactive} goes through all non-unified
        steps.
\item[]
    \texttt{initialize} {\em step} (or \texttt{all}) - De-unifies the target and source of
        a step (or all steps), as well as the hypotheses of the source,
        and makes all variables in the source unknown.  Useful to recover from
        an \texttt{assign} or \texttt{let} mistake that
        resulted in incorrect unifications.
\item[]
    \texttt{delete} {\em step} (or \texttt{all} or \texttt{floating{\char`\_}hypotheses}) -
        Deletes the specified
        step(s).  \texttt{delete floating{\char`\_}hypotheses}, then \texttt{initialize all}, then
        \texttt{unify all /interactive} is useful for recovering from mistakes
        where incorrect unifications assigned wrong math symbol strings to
        variables.
\item[]
    \texttt{improve} {\em step} (or \texttt{all}) -
      Automatically creates a proof for steps (with no unknown variables)
      whose proof requires no statements with \texttt{\$e} hypotheses.  Useful
      for filling in proofs of \texttt{\$f} hypotheses.  The \texttt{/depth}
      qualifier will also try statements whose \texttt{\$e} hypotheses contain
      no new variables.  {\em Warning:} Save your work (with \texttt{save
      new{\char`\_}proof} then \texttt{write source}) before using
      \texttt{/depth = 2} or greater, since the search time grows
      exponentially and may never terminate in a reasonable time, and you
      cannot interrupt the search.  I have found that it is rare for
      \texttt{/depth = 3} or greater to be useful.
 \item[]
    \texttt{save new{\char`\_}proof} - Saves the proof in progress in the program's
        internal database buffer.  To save it permanently into the database file,
        use \texttt{write source} after
        \texttt{save new{\char`\_}proof}.  To revert to the last
        \texttt{save new{\char`\_}proof},
        \texttt{exit /force} from the Proof Assistant then re-enter the Proof
        Assistant.
 \item[]
    \texttt{match step} {\em step} (or \texttt{match all}) - Shows what
        statements are
        possibilities for the \texttt{assign} statement. (This command
        is not very
        useful in its present form and hopefully will be improved
        eventually.  In the meantime, use the \texttt{search} statement for
        candidates matching specific math token combinations.)
 \item[]
 \texttt{minimize{\char`\_}with}\index{\texttt{minimize{\char`\_}with} command}
% 3/10/07 Note: line-breaking the above results in duplicate index entries
     - After a proof is complete, this command will attempt
        to match other database theorems to the proof to see if the proof
        size can be reduced as a result.  See \texttt{help
        minimize{\char`\_}with} in the
        Metamath program for its usage.
 \item[]
 \texttt{undo}\index{\texttt{undo} command}
    - Undo the effect of a proof-changing command (all but the
      \texttt{show} and \texttt{save} commands above).
 \item[]
 \texttt{redo}\index{\texttt{redo} command}
    - Reverse the previous \texttt{undo}.
\end{itemize}

The following commands set parameters that may be relevant to your proof.
Consult the individual \texttt{help set}... commands.
\begin{itemize}
   \item[] \texttt{set unification{\char`\_}timeout}
 \item[]
    \texttt{set search{\char`\_}limit}
  \item[]
    \texttt{set empty{\char`\_}substitution} - note that default is \texttt{off}
\end{itemize}

Type \texttt{exit} to exit the \texttt{MM-PA>}
 prompt and get back to the \texttt{MM>} prompt.
Another \texttt{exit} will then get you out of Metamath.



\subsection{\texttt{prove} Command}\index{\texttt{prove} command}
Syntax:  \texttt{prove} {\em label}

This command will enter the Proof Assistant\index{Proof Assistant}, which will
allow you to create or edit the proof of the specified statement.
The command-line prompt will change from \texttt{MM>} to \texttt{MM-PA>}.

Note:  In the present version (0.177) of
Metamath\index{Metamath!limitations of version 0.177}, the Proof
Assistant does not verify that \texttt{\$d}\index{\texttt{\$d}
statement} restrictions are met as a proof is being built.  After you
have completed a proof, you should type \texttt{save new{\char`\_}proof}
followed by \texttt{verify proof} {\em label} (where {\em label} is the
statement you are proving with the \texttt{prove} command) to verify the
\texttt{\$d} restrictions.

See also: \texttt{exit}

\subsection{\texttt{set unification\_timeout} Command}\index{\texttt{set
unification{\char`\_}timeout} command}
Syntax:  \texttt{set unification{\char`\_}timeout} {\em number}

(This command is available outside the Proof Assistant but affects the
Proof Assistant\index{Proof Assistant} only.)

Sometimes the Proof Assistant will inform you that a unification
time-out occurred.  This may happen when you try to \texttt{unify}
formulas with many temporary variables\index{temporary variable}
(\texttt{\$1}, \texttt{\$2}, etc.), since the time to compute all possible
unifications may grow exponentially with the number of variables.  If
you want Metamath to try harder (and you're willing to wait longer) you
may increase this parameter.  \texttt{show settings} will show you the
current value.

\subsection{\texttt{set empty\_substitution} Command}\index{\texttt{set
empty{\char`\_}substitution} command}
% These long names can't break well in narrow mode, and even "sloppy"
% is not enough. Work around this by not demanding justification.
\begin{flushleft}
Syntax:  \texttt{set empty{\char`\_}substitution on} or \texttt{set
empty{\char`\_}substitution off}
\end{flushleft}

(This command is available outside the Proof Assistant but affects the
Proof Assistant\index{Proof Assistant} only.)

The Metamath language allows variables to be
substituted\index{substitution!variable}\index{variable substitution}
with empty symbol sequences\index{empty substitution}.  However, in many
formal systems\index{formal system} this will never happen in a valid
proof.  Allowing for this possibility increases the likelihood of
ambiguous unifications\index{ambiguous
unification}\index{unification!ambiguous} during proof creation.
The default is that
empty substitutions are not allowed; for formal systems requiring them,
you must \texttt{set empty{\char`\_}substitution on}.
(An example where this must be \texttt{on}
would be a system that implements a Deduction Rule and in
which deductions from empty assumption lists would be permissible.  The
MIU-system\index{MIU-system} described in Appendix~\ref{MIU} is another
example.)
Note that empty substitutions are
always permissible in proof verification (VERIFY PROOF...) outside the
Proof Assistant.  (See the MIU system in the Metamath book for an example
of a system needing empty substitutions; another example would be a
system that implements a Deduction Rule and in which deductions from
empty assumption lists would be permissible.)

It is better to leave this \texttt{off} when working with \texttt{set.mm}.
Note that this command does not affect the way proofs are verified with
the \texttt{verify proof} command.  Outside of the Proof Assistant,
substitution of empty sequences for math symbols is always allowed.

\subsection{\texttt{set search\_limit} Command}\index{\texttt{set
search{\char`\_}limit} command} Syntax:  \texttt{set search{\char`\_}limit} {\em
number}

(This command is available outside the Proof Assistant but affects the
Proof Assistant\index{Proof Assistant} only.)

This command sets a parameter that determines when the \texttt{improve} command
in Proof Assistant mode gives up.  If you want \texttt{improve} to search harder,
you may increase it.  The \texttt{show settings} command tells you its current
value.


\subsection{\texttt{show new\_proof} Command}\index{\texttt{show
new{\char`\_}proof} command}
Syntax:  \texttt{show new{\char`\_}proof} [{\em
qualifiers} (see below)]

This command (available only in Proof Assistant mode) displays the proof
in progress.  It is identical to the \texttt{show proof} command, except that
there is no statement argument (since it is the statement being proved) and
the following qualifiers are not available:

    \texttt{/statement{\char`\_}summary}

    \texttt{/detailed{\char`\_}step}

Also, the following additional qualifiers are available:

    \texttt{/unknown} - Shows only steps that have no statement assigned.

    \texttt{/not{\char`\_}unified} - Shows only steps that have not been unified.

Note that \texttt{/essential}, \texttt{/depth}, \texttt{/unknown}, and
\texttt{/not{\char`\_}unified} may be
used in any combination; each of them effectively filters out additional
steps from the proof display.

See also:  \texttt{show proof}






\subsection{\texttt{assign} Command}\index{\texttt{assign} command}
Syntax:   \texttt{assign} {\em step} {\em label} [\texttt{/no{\char`\_}unify}]

   and:   \texttt{assign first} {\em label}

   and:   \texttt{assign last} {\em label}


This command, available in the Proof Assistant only, assigns an unknown
step (one with \texttt{?} in the \texttt{show new{\char`\_}proof}
listing) with the statement specified by {\em label}.  The assignment
will not be allowed if the statement cannot be unified with the step.

If \texttt{last} is specified instead of {\em step} number, the last
step that is shown by \texttt{show new{\char`\_}proof /unknown} will be
used.  This can be useful for building a proof with a command file (see
\texttt{help submit}).  It also makes building proofs faster when you know
the assignment for the last step.

If \texttt{first} is specified instead of {\em step} number, the first
step that is shown by \texttt{show new{\char`\_}proof /unknown} will be
used.

If {\em step} is zero or negative, the -{\em step}th from last unknown
step, as shown by \texttt{show new{\char`\_}proof /unknown}, will be
used.  \texttt{assign -1} {\em label} will assign the penultimate
unknown step, \texttt{assign -2} {\em label} the antepenultimate, and
\texttt{assign 0} {\em label} is the same as \texttt{assign last} {\em
label}.

Optional command qualifier:

    \texttt{/no{\char`\_}unify} - do not prompt user to select a unification if there is
        more than one possibility.  This is useful for noninteractive
        command files.  Later, the user can \texttt{unify all /interactive}.
        (The assignment will still be automatically unified if there is only
        one possibility and will be refused if unification is not possible.)



\subsection{\texttt{match} Command}\index{\texttt{match} command}
Syntax:  \texttt{match step} {\em step} [\texttt{/max{\char`\_}essential{\char`\_}hyp}
{\em number}]

    and:  \texttt{match all} [\texttt{/essential}]
          [\texttt{/max{\char`\_}essential{\char`\_}hyp} {\em number}]

This command, available in the Proof Assistant only, shows what
statements can be unified with the specified step(s).  {\em Note:} In
its current form, this command is not very useful because of the large
number of matches it reports.
It may be enhanced in the future.  In the meantime, the \texttt{search}
command can often provide finer control over locating theorems of interest.

Optional command qualifiers:

    \texttt{/max{\char`\_}essential{\char`\_}hyp} {\em number} - filters out
        of the list any statements
        with more than the specified number of
        \texttt{\$e}\index{\texttt{\$e} statement} hypotheses.

    \texttt{/essential{\char`\_}only} - in the \texttt{match all} statement, only
        the steps that
        would be listed in the \texttt{show new{\char`\_}proof /essential} display are
        matched.



\subsection{\texttt{let} Command}\index{\texttt{let} command}
Syntax: \texttt{let variable} {\em variable} = \verb/"/{\em symbol-sequence}\verb/"/

  and: \texttt{let step} {\em step} = \verb/"/{\em symbol-sequence}\verb/"/

These commands, available in the Proof Assistant\index{Proof Assistant}
only, assign a temporary variable\index{temporary variable} or unknown
step with a specific symbol sequence.  They are useful in the middle of
creating a proof, when you know what should be in the proof step but the
unification algorithm doesn't yet have enough information to completely
specify the temporary variables.  A ``temporary variable'' is one that
has the form \texttt{\$}{\em nn} in the proof display, such as
\texttt{\$1}, \texttt{\$2}, etc.  The {\em symbol-sequence} may contain
other unknown variables if desired.  Examples:

    \verb/let variable $32 = "A = B"/

    \verb/let variable $32 = "A = $35"/

    \verb/let step 10 = '|- x = x'/

    \verb/let step -2 = "|- ( $7 = ph )"/

Any symbol sequence will be accepted for the \texttt{let variable}
command.  Only those symbol sequences that can be unified with the step
will be accepted for \texttt{let step}.

The \texttt{let} commands ``zap'' the proof with information that can
only be verified when the proof is built up further.  If you make an
error, the command sequence \texttt{delete
floating{\char`\_}hypotheses}, \texttt{initialize all}, and
\texttt{unify all /interactive} will undo a bad \texttt{let} assignment.

If {\em step} is zero or negative, the -{\em step}th from last unknown
step, as shown by \texttt{show new{\char`\_}proof /unknown}, will be
used.  The command \texttt{let step 0} = \verb/"/{\em
symbol-sequence}\verb/"/ will use the last unknown step, \texttt{let
step -1} = \verb/"/{\em symbol-sequence}\verb/"/ the penultimate, etc.
If {\em step} is positive, \texttt{let step} may be used to assign known
(in the sense of having previously been assigned a label with
\texttt{assign}) as well as unknown steps.

Either single or double quotes can surround the {\em symbol-sequence} as
long as they are different from any quotes inside a {\em
symbol-sequence}.  If {\em symbol-sequence} contains both kinds of
quotes, see the instructions at the end of \texttt{help let} in the
Metamath program.


\subsection{\texttt{unify} Command}\index{\texttt{unify} command}
Syntax:  \texttt{unify step} {\em step}

      and:   \texttt{unify all} [\texttt{/interactive}]

These commands, available in the Proof Assistant only, unify the source
and target of the specified step(s). If you specify a specific step, you
will be prompted to select among the unifications that are possible.  If
you specify \texttt{all}, only those steps with unique unifications will be
unified.

Optional command qualifier for \texttt{unify all}:

    \texttt{/interactive} - You will be prompted to select among the
        unifications
        that are possible for any steps that do not have unique
        unifications.  (Otherwise \texttt{unify all} will bypass these.)

See also \texttt{set unification{\char`\_}timeout}.  The default is
100000, but increasing it to 1000000 can help difficult cases.  Manually
assigning some or all of the unknown variables with the \texttt{let
variable} command also helps difficult cases.



\subsection{\texttt{initialize} Command}\index{\texttt{initialize} command}
Syntax:  \texttt{initialize step} {\em step}

    and: \texttt{initialize all}

These commands, available in the Proof Assistant\index{Proof Assistant}
only, ``de-unify'' the target and source of a step (or all steps), as
well as the hypotheses of the source, and makes all variables in the
source and the source's hypotheses unknown.  This command is useful to
help recover from incorrect unifications that resulted from an incorrect
\texttt{assign}, \texttt{let}, or unification choice.  Part or all of
the command sequence \texttt{delete floating{\char`\_}hypotheses},
\texttt{initialize all}, and \texttt{unify all /interactive} will recover
from incorrect unifications.

See also:  \texttt{unify} and \texttt{delete}



\subsection{\texttt{delete} Command}\index{\texttt{delete} command}
Syntax:  \texttt{delete step} {\em step}

   and:      \texttt{delete all} -- {\em Warning: dangerous!}

   and:      \texttt{delete floating{\char`\_}hypotheses}

These commands are available in the Proof Assistant only.  The
\texttt{delete step} command deletes the proof tree section that
branches off of the specified step and makes the step become unknown.
\texttt{delete all} is equivalent to \texttt{delete step} {\em step}
where {\em step} is the last step in the proof (i.e.\ the beginning of
the proof tree).

In most cases the \texttt{undo} command is the best way to undo
a previous step.
An alternative is to salvage your last \texttt{save
new{\char`\_}proof} by exiting and reentering the Proof Assistant.
For this to work, keep a log file open to record your work
and to do \texttt{save new{\char`\_}proof} frequently, especially before
\texttt{delete}.

\texttt{delete floating{\char`\_}hypotheses} will delete all sections of
the proof that branch off of \texttt{\$f}\index{\texttt{\$f} statement}
statements.  It is sometimes useful to do this before an
\texttt{initialize} command to recover from an error.  Note that once a
proof step with a \texttt{\$f} hypothesis as the target is completely
known, the \texttt{improve} command can usually fill in the proof for
that step.  Unlike the deletion of logical steps, \texttt{delete
floating{\char`\_}hypotheses} is a relatively safe command that is
usually easy to recover from.



\subsection{\texttt{improve} Command}\index{\texttt{improve} command}
\label{improve}
Syntax:  \texttt{improve} {\em step} [\texttt{/depth} {\em number}]
                                               [\texttt{/no{\char`\_}distinct}]

   and:   \texttt{improve first} [\texttt{/depth} {\em number}]
                                              [\texttt{/no{\char`\_}distinct}]

   and:   \texttt{improve last} [\texttt{/depth} {\em number}]
                                              [\texttt{/no{\char`\_}distinct}]

   and:   \texttt{improve all} [\texttt{/depth} {\em number}]
                                              [\texttt{/no{\char`\_}distinct}]

These commands, available in the Proof Assistant\index{Proof Assistant}
only, try to find proofs automatically for unknown steps whose symbol
sequences are completely known.  They are primarily useful for filling in
proofs of \texttt{\$f}\index{\texttt{\$f} statement} hypotheses.  The
search will be restricted to statements having no
\texttt{\$e}\index{\texttt{\$e} statement} hypotheses.

\begin{sloppypar} % narrow
Note:  If memory is limited, \texttt{improve all} on a large proof may
overflow memory.  If you use \texttt{set unification{\char`\_}timeout 1}
before \texttt{improve all}, there will usually be sufficient
improvement to easily recover and completely \texttt{improve} the proof
later on a larger computer.  Warning:  Once memory has overflowed, there
is no recovery.  If in doubt, save the intermediate proof (\texttt{save
new{\char`\_}proof} then \texttt{write source}) before \texttt{improve
all}.
\end{sloppypar}

If \texttt{last} is specified instead of {\em step} number, the last
step that is shown by \texttt{show new{\char`\_}proof /unknown} will be
used.

If \texttt{first} is specified instead of {\em step} number, the first
step that is shown by \texttt{show new{\char`\_}proof /unknown} will be
used.

If {\em step} is zero or negative, the -{\em step}th from last unknown
step, as shown by \texttt{show new{\char`\_}proof /unknown}, will be
used.  \texttt{improve -1} will use the penultimate
unknown step, \texttt{improve -2} {\em label} the antepenultimate, and
\texttt{improve 0} is the same as \texttt{improve last}.

Optional command qualifier:

    \texttt{/depth} {\em number} - This qualifier will cause the search
        to include
        statements with \texttt{\$e} hypotheses (but no new variables in
        the \texttt{\$e}
        hypotheses), provided that the backtracking has not exceeded the
        specified depth. {\em Warning:}  Try \texttt{/depth 1},
        then \texttt{2}, then \texttt{3}, etc.
        in sequence because of possible exponential blowups.  Save your
        work before trying \texttt{/depth} greater than \texttt{1}!

    \texttt{/no{\char`\_}distinct} - Skip trial statements that have
        \texttt{\$d}\index{\texttt{\$d} statement} requirements.
        This qualifier will prevent assignments that might violate \texttt{\$d}
        requirements but it also could miss possible legal assignments.

See also: \texttt{set search{\char`\_}limit}

\subsection{\texttt{save new\_proof} Command}\index{\texttt{save
new{\char`\_}proof} command}
Syntax:  \texttt{save new{\char`\_}proof} {\em label} [\texttt{/normal}]
   [\texttt{/compressed}]

The \texttt{save new{\char`\_}proof} command is available in the Proof
Assistant only.  It saves the proof in progress in the source
buffer\index{source buffer}.  \texttt{save new{\char`\_}proof} may be
used to save a completed proof, or it may be used to save a proof in
progress in order to work on it later.  If an incomplete proof is saved,
any user assignments with \texttt{let step} or \texttt{let variable}
will be lost, as will any ambiguous unifications\index{ambiguous
unification}\index{unification!ambiguous} that were resolved manually.
To help make recovery easier, it can be helpful to \texttt{improve all}
before \texttt{save new{\char`\_}proof} so that the incomplete proof
will have as much information as possible.

Note that the proof will not be permanently saved until a \texttt{write
source} command is issued.

Optional command qualifiers:

    \texttt{/normal} - The proof is saved in the normal format (i.e., as a
        sequence of labels, which is the defined format of the basic Metamath
        language).\index{basic language}  This is the default format that
        is used if a qualifier is omitted.

    \texttt{/compressed} - The proof is saved in the compressed format, which
        reduces storage requirements for a database.
        (See Appendix~\ref{compressed}.)


\section{Creating \LaTeX\ Output}\label{texout}\index{latex@{\LaTeX}}

You can generate \LaTeX\ output given the
information in a database.
The database must already include the necessary typesetting information
(see section \ref{tcomment} for how to provide this information).

The \texttt{show statement} and \texttt{show proof} commands each have a
special \texttt{/tex} command qualifier that produces \LaTeX\ output.
(The \texttt{show statement} command also has the
\texttt{/simple{\char`\_}tex} qualifier for output that is easier to
edit by hand.)  Before you can use them, you must open a \LaTeX\ file to
which to send their output.  A typical complete session will use this
sequence of Metamath commands:

\begin{verbatim}
read set.mm
open tex example.tex
show statement a1i /tex
show proof a1i /all/lemmon/renumber/tex
show statement uneq2 /tex
show proof uneq2 /all/lemmon/renumber/tex
close tex
\end{verbatim}

See Section~\ref{mathcomments} for information on comment markup and
Appendix~\ref{ASCII} for information on how math symbol translation is
specified.

To format and print the \LaTeX\ source, you will need the \LaTeX\
program, which is standard on most Linux installations and available for
Windows.  On Linux, in order to create a {\sc pdf} file, you will
typically type at the shell prompt
\begin{verbatim}
$ pdflatex example.tex
\end{verbatim}

\subsection{\texttt{open tex} Command}\index{\texttt{open tex} command}
Syntax:  \texttt{open tex} {\em file-name} [\texttt{/no{\char`\_}header}]

This command opens a file for writing \LaTeX\
source\index{latex@{\LaTeX}} and writes a \LaTeX\ header to the file.
\LaTeX\ source can be written with the \texttt{show proof}, \texttt{show
new{\char`\_}proof}, and \texttt{show statement} commands using the
\texttt{/tex} qualifier.

The mapping to \LaTeX\ symbols is defined in a special comment
containing a \texttt{\$t} token, described in Appendix~\ref{ASCII}.

There is an optional command qualifier:

    \texttt{/no{\char`\_}header} - This qualifier prevents a standard
        \LaTeX\ header and trailer
        from being included with the output \LaTeX\ code.


\subsection{\texttt{close tex} Command}\index{\texttt{close tex} command}
Syntax:  \texttt{close tex}

This command writes a trailer to any \LaTeX\ file\index{latex@{\LaTeX}}
that was opened with \texttt{open tex} (unless
\texttt{/no{\char`\_}header} was used with \texttt{open tex}) and closes
the \LaTeX\ file.


\section{Creating {\sc HTML} Output}\label{htmlout}

You can generate {\sc html} web pages given the
information in a database.
The database must already include the necessary typesetting information
(see section \ref{tcomment} for how to provide this information).
The ability to produce {\sc html} web pages was added in Metamath version
0.07.30.

To create an {\sc html} output file(s) for \texttt{\$a} or \texttt{\$p}
statement(s), use
\begin{quote}
    \texttt{show statement} {\em label-match} \texttt{/html}
\end{quote}
The output file will be named {\em label-match}\texttt{.html}
for each match.  When {\em
label-match} has wildcard (\texttt{*}) characters, all statements with
matching labels will have {\sc html} files produced for them.  Also,
when {\em label-match} has a wildcard (\texttt{*}) character, two additional
files, \texttt{mmdefinitions.html} and \texttt{mmascii.html} will be
produced.  To produce {\em only} these two additional files, you can use
\texttt{?*}, which will not match any statement label, in place of {\em
label-match}.

There are three other qualifiers for \texttt{show statement} that also
generate {\sc HTML} code.  These are \texttt{/alt{\char`\_}html},
\texttt{/brief{\char`\_}html}, and
\texttt{/brief{\char`\_}alt{\char`\_}html}, and are described in the
next section.

The command
\begin{quote}
    \texttt{show statement} {\em label-match} \texttt{/alt{\char`\_}html}
\end{quote}
does the same as \texttt{show statement} {\em label-match} \texttt{/html},
except that the {\sc html} code for the symbols is taken from
\texttt{althtmldef} statements instead of \texttt{htmldef} statements in
the \texttt{\$t} comment.

The command
\begin{verbatim}
show statement * /brief_html
\end{verbatim}
invokes a special mode that just produces definition and theorem lists
accompanied by their symbol strings, in a format suitable for copying and
pasting into another web page (such as the tutorial pages on the
Metamath web site).

Finally, the command
\begin{verbatim}
show statement * /brief_alt_html
\end{verbatim}
does the same as \texttt{show statement * / brief{\char`\_}html}
for the alternate {\sc html}
symbol representation.

A statement's comment can include a special notation that provides a
certain amount of control over the {\sc HTML} version of the comment.  See
Section~\ref{mathcomments} (p.~\pageref{mathcomments}) for the comment
markup features.

The \texttt{write theorem{\char`\_}list} and \texttt{write bibliography}
commands, which are described below, provide as a side effect complete
error checking for all of the features described in this
section.\index{error checking}

\subsection{\texttt{write theorem\_list}
Command}\index{\texttt{write theorem{\char`\_}list} command}
Syntax:  \texttt{write theorem{\char`\_}list}
[\texttt{/theorems{\char`\_}per{\char`\_}page} {\em number}]

This command writes a list of all of the \texttt{\$a} and \texttt{\$p}
statements in the database into a web page file
 called \texttt{mmtheorems.html}.
When additional files are needed, they are called
\texttt{mmtheorems2.html}, \texttt{mmtheorems3.html}, etc.

Optional command qualifier:

    \texttt{/theorems{\char`\_}per{\char`\_}page} {\em number} -
 This qualifier specifies the number of statements to
        write per web page.  The default is 100.

{\em Note:} In version 0.177\index{Metamath!limitations of version
0.177} of Metamath, the ``Nearby theorems'' links on the individual
web pages presuppose 100 theorems per page when linking to the theorem
list pages.  Therefore the \texttt{/theorems{\char`\_}per{\char`\_}page}
qualifier, if it specifies a number other than 100, will cause the
individual web pages to be out of sync and should not be used to
generate the main theorem list for the web site.  This may be
fixed in a future version.


\subsection{\texttt{write bibliography}\label{wrbib}
Command}\index{\texttt{write bibliography} command}
Syntax:  \texttt{write bibliography} {\em filename}

This command reads an existing {\sc html} bibliographic cross-reference
file, normally called \texttt{mmbiblio.html}, and updates it per the
bibliographic links in the database comments.  The file is updated
between the {\sc html} comment lines \texttt{<!--
{\char`\#}START{\char`\#} -->} and \texttt{<!-- {\char`\#}END{\char`\#}
-->}.  The original input file is renamed to {\em
filename}\texttt{{\char`\~}1}.

A bibliographic reference is indicated with the reference name
in brackets, such as  \texttt{Theorem 3.1 of
[Monk] p.\ 22}.
See Section~\ref{htmlout} (p.~\pageref{htmlout}) for
syntax details.


\subsection{\texttt{write recent\_additions}
Command}\index{\texttt{write recent{\char`\_}additions} command}
Syntax:  \texttt{write recent{\char`\_}additions} {\em filename}
[\texttt{/limit} {\em number}]

This command reads an existing ``Recent Additions'' {\sc html} file,
normally called \texttt{mmrecent.html}, and updates it with the
descriptions of the most recently added theorems to the database.
 The file is updated between
the {\sc html} comment lines \texttt{<!-- {\char`\#}START{\char`\#} -->}
and \texttt{<!-- {\char`\#}END{\char`\#} -->}.  The original input file
is renamed to {\em filename}\texttt{{\char`\~}1}.

Optional command qualifier:

    \texttt{/limit} {\em number} -
 This qualifier specifies the number of most recent theorems to
   write to the output file.  The default is 100.


\section{Text File Utilities}

\subsection{\texttt{tools} Command}\index{\texttt{tools} command}
Syntax:  \texttt{tools}

This command invokes an easy-to-use, general purpose utility for
manipulating the contents of {\sc ascii} text files.  Upon typing
\texttt{tools}, the command-line prompt will change to \texttt{TOOLS>}
until you type \texttt{exit}.  The \texttt{tools} commands can be used
to perform simple, global edits on an input/output file,
such as making a character string substitution on each line, adding a
string to each line, and so on.  A typical use of this utility is
to build a \texttt{submit} input file to perform a common operation on a
list of statements obtained from \texttt{show label} or \texttt{show
usage}.

The actions of most of the \texttt{tools} commands can also be
performed with equivalent (and more powerful) Unix shell commands, and
some users may find those more efficient.  But for Windows users or
users not comfortable with Unix, \texttt{tools} provides an
easy-to-learn alternative that is adequate for most of the
script-building tasks needed to use the Metamath program effectively.

\subsection{\texttt{help} Command (in \texttt{tools})}
Syntax:  \texttt{help}

The \texttt{help} command lists the commands available in the
\texttt{tools} utility, along with a brief description.  Each command,
in turn, has its own help, such as \texttt{help add}.  As with
Metamath's \texttt{MM>} prompt, a complete command can be entered at
once, or just the command word can be typed, causing the program to
prompt for each argument.

\vskip 1ex
\noindent Line-by-line editing commands:

  \texttt{add} - Add a specified string to each line in a file.

  \texttt{clean} - Trim spaces and tabs on each line in a file; convert
         characters.

  \texttt{delete} - Delete a section of each line in a file.

  \texttt{insert} - Insert a string at a specified column in each line of
        a file.

  \texttt{substitute} - Make a simple substitution on each line of the file.

  \texttt{tag} - Like \texttt{add}, but restricted to a range of lines.

  \texttt{swap} - Swap the two halves of each line in a file.

\vskip 1ex
\noindent Other file-processing commands:

  \texttt{break} - Break up (tokenize) a file into a list of tokens (one per
        line).

  \texttt{build} - Build a file with multiple tokens per line from a list.

  \texttt{count} - Count the occurrences in a file of a specified string.

  \texttt{number} - Create a list of numbers.

  \texttt{parallel} - Put two files in parallel.

  \texttt{reverse} - Reverse the order of the lines in a file.

  \texttt{right} - Right-justify lines in a file (useful before sorting
         numbers).

%  \texttt{tag} - Tag edit updates in a program for revision control.

  \texttt{sort} - Sort the lines in a file with key starting at
         specified string.

  \texttt{match} - Extract lines containing (or not) a specified string.

  \texttt{unduplicate} - Eliminate duplicate occurrences of lines in a file.

  \texttt{duplicate} - Extract first occurrence of any line occurring
         more than

   \ \ \    once in a file, discarding lines occurring exactly once.

  \texttt{unique} - Extract lines occurring exactly once in a file.

  \texttt{type} (10 lines) - Display the first few lines in a file.
                  Similar to Unix \texttt{head}.

  \texttt{copy} - Similar to Unix \texttt{cat} but safe (same input
         and output file allowed).

  \texttt{submit} - Run a script containing \texttt{tools} commands.

\vskip 1ex

\noindent Note:
  \texttt{unduplicate}, \texttt{duplicate}, and \texttt{unique} also
 sort the lines as a side effect.


\subsection{Using \texttt{tools} to Build Metamath \texttt{submit}
Scripts}

The \texttt{break} command is typically used to break up a series of
statement labels, such as the output of Metamath's \texttt{show usage},
into one label per line.  The other \texttt{tools} commands can then be
used to add strings before and after each statement label to specify
commands to be performed on the statement.  The \texttt{parallel}
command is useful when a statement label must be mentioned more than
once on a line.

Very often a \texttt{submit} script for Metamath will require multiple
command lines for each statement being processed.  For example, you may
want to enter the Proof Assistant, \texttt{minimize{\char`\_}with} your
latest theorem, \texttt{save} the new proof, and \texttt{exit} the Proof
Assistant.  To accomplish this, you can build a file with these four
commands for each statement on a single line, separating each command
with a designated character such as \texttt{@}.  Then at the end you can
\texttt{substitute} each \texttt{@} with \texttt{{\char`\\}n} to break
up the lines into individual command lines (see \texttt{help
substitute}).


\subsection{Example of a \texttt{tools} Session}

To give you a quick feel for the \texttt{tools} utility, we show a
simple session where we create a file \texttt{n.txt} with 3 lines, add
strings before and after each line, and display the lines on the screen.
You can experiment with the various commands to gain experience with the
\texttt{tools} utility.

\begin{verbatim}
MM> tools
Entering the Text Tools utilities.
Type HELP for help, EXIT to exit.
TOOLS> number
Output file <n.tmp>? n.txt
First number <1>?
Last number <10>? 3
Increment <1>?
TOOLS> add
Input/output file? n.txt
String to add to beginning of each line <>? This is line
String to add to end of each line <>? .
The file n.txt has 3 lines; 3 were changed.
First change is on line 1:
This is line 1.
TOOLS> type n.txt
This is line 1.
This is line 2.
This is line 3.
TOOLS> exit
Exiting the Text Tools.
Type EXIT again to exit Metamath.
MM>
\end{verbatim}



\appendix
\chapter{Sample Representations}
\label{ASCII}

This Appendix provides a sample of {\sc ASCII} representations,
their corresponding traditional mathematical symbols,
and a discussion of their meanings
in the \texttt{set.mm} database.
These are provided in order of appearance.
This is only a partial list, and new definitions are routinely added.
A complete list is available at \url{http://metamath.org}.

These {\sc ASCII} representations, along
with information on how to display them,
are defined in the \texttt{set.mm} database file inside
a special comment called a \texttt{\$t} {\em
comment}\index{\texttt{\$t} comment} or {\em typesetting
comment.}\index{typesetting comment}
A typesetting comment
is indicated by the appearance of the
two-character string \texttt{\$t} at the beginning of the comment.
For more information,
see Section~\ref{tcomment}, p.~\pageref{tcomment}.

In the following table the ``{\sc ASCII}'' column shows the {\sc ASCII}
representation,
``Symbol'' shows the mathematical symbolic display
that corresponds to that {\sc ASCII} representation, ``Labels'' shows
the key label(s) that define the representation, and
``Description'' provides a description about the symbol.
As usual, ``iff'' is short for ``if and only if.''\index{iff}
In most cases the ``{\sc ASCII}'' column only shows
the key token, but it will sometimes show a sequence of tokens
if that is necessary for clarity.

{\setlength{\extrarowsep}{4pt} % Keep rows from being too close together
\begin{longtabu}   { @{} c c l X }
\textbf{ASCII} & \textbf{Symbol} & \textbf{Labels} & \textbf{Description} \\
\endhead
\texttt{|-} & $\vdash$ & &
  ``It is provable that...'' \\
\texttt{ph} & $\varphi$ & \texttt{wph} &
  The wff (boolean) variable phi,
  conventionally the first wff variable. \\
\texttt{ps} & $\psi$ & \texttt{wps} &
  The wff (boolean) variable psi,
  conventionally the second wff variable. \\
\texttt{ch} & $\chi$ & \texttt{wch} &
  The wff (boolean) variable chi,
  conventionally the third wff variable. \\
\texttt{-.} & $\lnot$ & \texttt{wn} &
  Logical not. E.g., if $\varphi$ is true, then $\lnot \varphi$ is false. \\
\texttt{->} & $\rightarrow$ & \texttt{wi} &
  Implies, also known as material implication.
  In classical logic the expression $\varphi \rightarrow \psi$ is true
  if either $\varphi$ is false or $\psi$ is true (or both), that is,
  $\varphi \rightarrow \psi$ has the same meaning as
  $\lnot \varphi \lor \psi$ (as proven in theorem \texttt{imor}). \\
\texttt{<->} & $\leftrightarrow$ &
  \hyperref[df-bi]{\texttt{df-bi}} &
  Biconditional (aka is-equals for boolean values).
  $\varphi \leftrightarrow \psi$ is true iff
  $\varphi$ and $\psi$ have the same value. \\
\texttt{\char`\\/} & $\lor$ &
  \makecell[tl]{{\hyperref[df-or]{\texttt{df-or}}}, \\
	         \hyperref[df-3or]{\texttt{df-3or}}} &
  Disjunction (logical ``or''). $\varphi \lor \psi$ is true iff
  $\varphi$, $\psi$, or both are true. \\
\texttt{/\char`\\} & $\land$ &
  \makecell[tl]{{\hyperref[df-an]{\texttt{df-an}}}, \\
                 \hyperref[df-3an]{\texttt{df-3an}}} &
  Conjunction (logical ``and''). $\varphi \land \psi$ is true iff
  both $\varphi$ and $\psi$ are true. \\
\texttt{A.} & $\forall$ &
  \texttt{wal} &
  For all; the wff $\forall x \varphi$ is true iff
  $\varphi$ is true for all values of $x$. \\
\texttt{E.} & $\exists$ &
  \hyperref[df-ex]{\texttt{df-ex}} &
  There exists; the wff
  $\exists x \varphi$ is true iff
  there is at least one $x$ where $\varphi$ is true. \\
\texttt{[ y / x ]} & $[ y / x ]$ &
  \hyperref[df-sb]{\texttt{df-sb}} &
  The wff $[ y / x ] \varphi$ produces
  the result when $y$ is properly substituted for $x$ in $\varphi$
  ($y$ replaces $x$).
  % This is elsb4
  % ( [ x / y ] z e. y <-> z e. x )
  For example,
  $[ x / y ] z \in y$ is the same as $z \in x$. \\
\texttt{E!} & $\exists !$ &
  \hyperref[df-eu]{\texttt{df-eu}} &
  There exists exactly one;
  $\exists ! x \varphi$ is true iff
  there is at least one $x$ where $\varphi$ is true. \\
\texttt{\{ y | phi \}}  & $ \{ y | \varphi \}$ &
  \hyperref[df-clab]{\texttt{df-clab}} &
  The class of all sets where $\varphi$ is true. \\
\texttt{=} & $ = $ &
  \hyperref[df-cleq]{\texttt{df-cleq}} &
  Class equality; $A = B$ iff $A$ equals $B$. \\
\texttt{e.} & $ \in $ &
  \hyperref[df-clel]{\texttt{df-clel}} &
  Class membership; $A \in B$ if $A$ is a member of $B$. \\
\texttt{{\char`\_}V} & {\rm V} &
  \hyperref[df-v]{\texttt{df-v}} &
  Class of all sets (not itself a set). \\
\texttt{C\_} & $ \subseteq $ &
  \hyperref[df-ss]{\texttt{df-ss}} &
  Subclass (subset); $A \subseteq B$ is true iff
  $A$ is a subclass of $B$. \\
\texttt{u.} & $ \cup $ &
  \hyperref[df-un]{\texttt{df-un}} &
  $A \cup B$ is the union of classes $A$ and $B$. \\
\texttt{i^i} & $ \cap $ &
  \hyperref[df-in]{\texttt{df-in}} &
  $A \cap B$ is the intersection of classes $A$ and $B$. \\
\texttt{\char`\\} & $ \setminus $ &
  \hyperref[df-dif]{\texttt{df-dif}} &
  $A \setminus B$ (set difference)
  is the class of all sets in $A$ except for those in $B$. \\
\texttt{(/)} & $ \varnothing $ &
  \hyperref[df-nul]{\texttt{df-nul}} &
  $ \varnothing $ is the empty set (aka null set). \\
\texttt{\char`\~P} & $ \cal P $ &
  \hyperref[df-pw]{\texttt{df-pw}} &
  Power class. \\
\texttt{<.\ A , B >.} & $\langle A , B \rangle$ &
  \hyperref[df-op]{\texttt{df-op}} &
  The ordered pair $\langle A , B \rangle$. \\
\texttt{( F ` A )} & $ ( F ` A ) $ &
  \hyperref[df-fv]{\texttt{df-fv}} &
  The value of function $F$ when applied to $A$. \\
\texttt{\_i} & $ i $ &
  \texttt{df-i} &
  The square root of negative one. \\
\texttt{x.} & $ \cdot $ &
  \texttt{df-mul} &
  Complex number multiplication; $2~\cdot~3~=~6$. \\
\texttt{CC} & $ \mathbb{C} $ &
  \texttt{df-c} &
  The set of complex numbers. \\
\texttt{RR} & $ \mathbb{R} $ &
  \texttt{df-r} &
  The set of real numbers. \\
\end{longtabu}
} % end of extrarowsep

\chapter{Compressed Proofs}
\label{compressed}\index{compressed proof}\index{proof!compressed}

The proofs in the \texttt{set.mm} set theory database are stored in compressed
format for efficiency.  Normally you needn't concern yourself with the
compressed format, since you can display it with the usual proof display tools
in the Metamath program (\texttt{show proof}\ldots) or convert it to the normal
RPN proof format described in Section~\ref{proof} (with \texttt{save proof}
{\em label} \texttt{/normal}).  However for sake of completeness we describe the
format here and show how it maps to the normal RPN proof format.

A compressed proof, located between \texttt{\$=} and \texttt{\$.}\ keywords, consists
of a left parenthesis, a sequence of statement labels, a right parenthesis,
and a sequence of upper-case letters \texttt{A} through \texttt{Z} (with optional
white space between them).  White space must surround the parentheses
and the labels.  The left parenthesis tells Metamath that a
compressed proof follows.  (A normal RPN proof consists of just a sequence of
labels, and a parenthesis is not a legal character in a label.)

The sequence of upper-case letters corresponds to a sequence of integers
with the following mapping.  Each integer corresponds to a proof step as
described later.
\begin{center}
  \texttt{A} = 1 \\
  \texttt{B} = 2 \\
   \ldots \\
  \texttt{T} = 20 \\
  \texttt{UA} = 21 \\
  \texttt{UB} = 22 \\
   \ldots \\
  \texttt{UT} = 40 \\
  \texttt{VA} = 41 \\
  \texttt{VB} = 42 \\
   \ldots \\
  \texttt{YT} = 120 \\
  \texttt{UUA} = 121 \\
   \ldots \\
  \texttt{YYT} = 620 \\
  \texttt{UUUA} = 621 \\
   etc.
\end{center}

In other words, \texttt{A} through \texttt{T} represent the
least-significant digit in base 20, and \texttt{U} through \texttt{Y}
represent zero or more most-significant digits in base 5, where the
digits start counting at 1 instead of the usual 0. With this scheme, we
don't need white space between these ``numbers.''

(In the design of the compressed proof format, only upper-case letters,
as opposed to say all non-whitespace printable {\sc ascii} characters other than
%\texttt{\$}, was chosen to make the compressed proof a little less
%displeasing to the eye, at the expense of a typical 20\% compression
\texttt{\$}, were chosen so as not to collide with most text editor
searches, at the expense of a typical 20\% compression
loss.  The base 5/base 20 grouping, as opposed to say base 6/base 19,
was chosen by experimentally determining the grouping that resulted in
best typical compression.)

The letter \texttt{Z} identifies (tags) a proof step that is identical to one
that occurs later on in the proof; it helps shorten the proof by not requiring
that identical proof steps be proved over and over again (which happens often
when building wff's).  The \texttt{Z} is placed immediately after the
least-significant digit (letters \texttt{A} through \texttt{T}) that ends the integer
corresponding to the step to later be referenced.

The integers that the upper-case letters correspond to are mapped to labels as
follows.  If the statement being proved has $m$ mandatory hypotheses, integers
1 through $m$ correspond to the labels of these hypotheses in the order shown
by the \texttt{show statement ... / full} command, i.e., the RPN order\index{RPN
order} of the mandatory
hypotheses.  Integers $m+1$ through $m+n$ correspond to the labels enclosed in
the parentheses of the compressed proof, in the order that they appear, where
$n$ is the number of those labels.  Integers $m+n+1$ on up don't directly
correspond to statement labels but point to proof steps identified with the
letter \texttt{Z}, so that these proof steps can be referenced later in the
proof.  Integer $m+n+1$ corresponds to the first step tagged with a \texttt{Z},
$m+n+2$ to the second step tagged with a \texttt{Z}, etc.  When the compressed
proof is converted to a normal proof, the entire subproof of a step tagged
with \texttt{Z} replaces the reference to that step.

For efficiency, Metamath works with compressed proofs directly, without
converting them internally to normal proofs.  In addition to the usual
error-checking, an error message is given if (1) a label in the label list in
parentheses does not refer to a previous \texttt{\$p} or \texttt{\$a} statement or a
non-mandatory hypothesis of the statement being proved and (2) a proof step
tagged with \texttt{Z} is referenced before the step tagged with the \texttt{Z}.

Just as in a normal proof under development (Section~\ref{unknown}), any step
or subproof that is not yet known may be represented with a single \texttt{?}.
White space does not have to appear between the \texttt{?}\ and the upper-case
letters (or other \texttt{?}'s) representing the remainder of the proof.

% April 1, 2004 Appendix C has been added back in with corrections.
%
% May 20, 2003 Appendix C was removed for now because there was a problem found
% by Bob Solovay
%
% Also, removed earlier \ref{formalspec} 's (3 cases above)
%
% Bob Solovay wrote on 30 Nov 2002:
%%%%%%%%%%%%% (start of email comment )
%      3. My next set of comments concern appendix C. I read this before I
% read Chapter 4. So I first noted that the system as presented in the
% Appendix lacked a certain formal property that I thought desirable. I
% then came up with a revised formal system that had this property. Upon
% reading Chapter 4, I noticed that the revised system was closer to the
% treatment in Chapter 4 than the system in Appendix C.
%
%         First a very minor correction:
%
%         On page 142 line 2: The condition that V(e) != V(f) should only be
% required of e, f in T such that e != f.
%
%         Here is a natural property [transitivity] that one would like
% the formal system to have:
%
%         Let Gamma be a set of statements. Suppose that the statement Phi
% is provable from Gamma and that the statement Psi is provable from Gamma
% \cup {Phi}. Then Psi is provable from Gamma.
%
%         I shall present an example to show that this property does not
% hold for the formal systems of Appendix C:
%
%         I write the example in metamath style:
%
% $c A B C D E $.
% $v x y
%
% ${
% tx $f A x $.
% ty $f B y $.
% ax1 $a C x y $.
% $}
%
% ${
% tx $f A x $.
% ty $f B y $.
% ax2-h1 $e C x y $.
% ax2 $a D y $.
% $}
%
% ${
% ty $f B y $.
% ax3-h1 $e D y $.
% ax3 $a E y $.
% $}
%
% $(These three axioms are Gamma $)
%
% ${
% tx $f A x $.
% ty $f B y $.
% Phi $p D y $=
% tx ty tx ty ax1 ax2 $.
% $}
%
% ${
% ty $f B y $.
% Psi $p E y $=
% ty ty Phi ax3 $.
% $}
%
%
% I omit the formal proofs of the following claims. [I will be glad to
% supply them upon request.]
%
% 1) Psi is not provable from Gamma;
%
% 2) Psi is provable from Gamma + Phi.
%
% Here "provable" refers to the formalism of Appendix C.
%
% The trouble of course is that Psi is lacking the variable declaration
%
% $f Ax $.
%
% In the Metamath system there is no trouble proving Psi. I attach a
% metamath file that shows this and which has been checked by the
% metamath program.
%
% I next want to indicate how I think the treatment in Appendix C should
% be revised so as to conform more closely to the metamath system of the
% main text. The revised system *does* have the transitivity property.
%
% We want to give revised definitions of "statement" and
% "provable". [cf. sections C.2.4. and C.2.5] Our new definitions will
% use the definitions given in Appendix C. So we take the following
% tack. We refer to the original notions as o-statement and o-provable. And
% we refer to the notions we are defining as n-statement and n-provable.
%
%         A n-statement is an o-statement in which the only variables
% that appear in the T component are mandatory.
%
%         To any o-statement we can associate its reduct which is a
% n-statement by dropping all the elements of T or D which contain
% non-mandatory variables.
%
%         An n-statement gamma is n-provable if there is an o-statement
% gamma' which has gamma as its reduct andf such that gamma' is
% o-provable.
%
%         It seems to me [though I am not completely sure on this point]
% that n-provability corresponds to metamath provability as discussed
% say in Chapter 4.
%
%         Attached to this letter is the metamath proof of Phi and Psi
% from Gamma discussed above.
%
%         I am still brooding over the question of whether metamath
% correctly formalizes set-theory. No doubt I will have some questions
% re this after my thoughts become clearer.
%%%%%%%%%%%%%%%% (end of email comment)

%%%%%%%%%%%%%%%% (start of 2nd email comment from Bob Solovay 1-Apr-04)
%
%         I hope that Appendix C is the one that gives a "formal" treatment
% of Metamath. At any rate, thats the appendix I want to comment on.
%
%         I'm going to suggest two changes in the definition.
%
%         First change (in the definition of statement): Require that the
% sets D, T, and E be finite.
%
%         Probably things are fine as you give them. But in the applications
% to the main metamath system they will always be finite, and its useful in
% thinking about things [at least for me] to stick to the finite case.
%
%         Second change:
%
%         First let me give an approximate description. Remove the dummy
% variables from the statement. Instead, include them in the proof.
%
%         More formally: Require that T consists of type declarations only
% for mandatory variables. Require that all the pairs in D consist of
% mandatory variables.
%
%         At the start of a proof we are allowed to declare a finite number
% of dummy variables [provided that none of them appear in any of the
% statements in E \cup {A}. We have to supply type declarations for all the
% dummy variables. We are allowed to add new $d statements referring to
% either the mandatory or dummy variables. But we require that no new $d
% statement references only mandatory variables.
%
%         I find this way of doing things more conceptual than the treatment
% in Appendix C. But the change [which I will use implicitly in later
% letters about doing Peano] is mainly aesthetic. I definitely claim that my
% results on doing Peano all apply to Metamath as it is presented in your
% book.
%
%         --Bob
%
%%%%%%%%%%%%%%%% (end of 2nd email comment)

%%
%% When uncommenting the below, also uncomment references above to {formalspec}
%%
\chapter{Metamath's Formal System}\label{formalspec}\index{Metamath!as a formal
system}

\section{Introduction}

\begin{quote}
  {\em Perfection is when there is no longer anything more to take away.}
    \flushright\sc Antoine de
     Saint-Exupery\footnote{\cite[p.~3-25]{Campbell}.}\\
\end{quote}\index{de Saint-Exupery, Antoine}

This appendix describes the theory behind the Metamath language in an abstract
way intended for mathematicians.  Specifically, we construct two
set-theo\-ret\-i\-cal objects:  a ``formal system'' (roughly, a set of syntax
rules, axioms, and logical rules) and its ``universe'' (roughly, the set of
theorems derivable in the formal system).  The Metamath computer language
provides us with a way to describe specific formal systems and, with the aid of
a proof provided by the user, to verify that given theorems
belong to their universes.

To understand this appendix, you need a basic knowledge of informal set theory.
It should be sufficient to understand, for example, Ch.\ 1 of Munkres' {\em
Topology} \cite{Munkres}\index{Munkres, James R.} or the
introductory set theory chapter
in many textbooks that introduce abstract mathematics. (Note that there are
minor notational differences among authors; e.g.\ Munkres uses $\subset$ instead
of our $\subseteq$ for ``subset.''  We use ``included in'' to mean ``a subset
of,'' and ``belongs to'' or ``is contained in'' to mean ``is an element of.'')
What we call a ``formal'' description here, unlike earlier, is actually an
informal description in the ordinary language of mathematicians.  However we
provide sufficient detail so that a mathematician could easily formalize it,
even in the language of Metamath itself if desired.  To understand the logic
examples at the end of this appendix, familiarity with an introductory book on
mathematical logic would be helpful.

\section{The Formal Description}

\subsection[Preliminaries]{Preliminaries\protect\footnotemark}%
\footnotetext{This section is taken mostly verbatim
from Tarski \cite[p.~63]{Tarski1965}\index{Tarski, Alfred}.}

By $\omega$ we denote the set of all natural numbers (non-negative integers).
Each natural number $n$ is identified with the set of all smaller numbers: $n =
\{ m | m < n \}$.  The formula $m < n$ is thus equivalent to the condition: $m
\in n$ and $m,n \in \omega$. In particular, 0 is the number zero and at the
same time the empty set $\varnothing$, $1=\{0\}$, $2=\{0,1\}$, etc. ${}^B A$
denotes the set of all functions on $B$ to $A$ (i.e.\ with domain $B$ and range
included in $A$).  The members of ${}^\omega A$ are what are called {\em simple
infinite sequences},\index{simple infinite sequence}
with all {\em terms}\index{term} in $A$.  In case $n \in \omega$, the
members of ${}^n A$ are referred to as {\em finite $n$-termed
sequences},\index{finite $n$-termed
sequence} again
with terms in $A$.  The consecutive terms (function values) of a finite or
infinite sequence $f$ are denoted by $f_0, f_1, \ldots ,f_n,\ldots$.  Every
finite sequence $f \in \bigcup _{n \in \omega} {}^n A$ uniquely determines the
number $n$ such that $f \in {}^n A$; $n$ is called the {\em
length}\index{length of a sequence ({$"|\ "|$})} of $f$ and
is denoted by $|f|$.  $\langle a \rangle$ is the sequence $f$ with $|f|=1$ and
$f_0=a$; $\langle a,b \rangle$ is the sequence $f$ with $|f|=2$, $f_0=a$,
$f_1=b$; etc.  Given two finite sequences $f$ and $g$, we denote by $f\frown g$
their {\em concatenation},\index{concatenation} i.e., the
finite sequence $h$ determined by the
conditions:
\begin{eqnarray*}
& |h| = |f|+|g|;&  \\
& h_n = f_n & \mbox{\ for\ } n < |f|;  \\
& h_{|f|+n} = g_n & \mbox{\ for\ } n < |g|.
\end{eqnarray*}

\subsection{Constants, Variables, and Expressions}

A formal system has a set of {\em symbols}\index{symbol!in
a formal system} denoted
by $\mbox{\em SM}$.  A
precise set-theo\-ret\-i\-cal definition of this set is unimportant; a symbol
could be considered a primitive or atomic element if we wish.  We assume this
set is divided into two disjoint subsets:  a set $\mbox{\em CN}$ of {\em
constants}\index{constant!in a formal system} and a set $\mbox{\em VR}$ of
{\em variables}.\index{variable!in a formal system}  $\mbox{\em CN}$ and
$\mbox{\em VR}$ are each assumed to consist of countably many symbols which
may be arranged in finite or simple infinite sequences $c_0, c_1, \ldots$ and
$v_0, v_1, \ldots$ respectively, without repeating terms.  We will represent
arbitrary symbols by metavariables $\alpha$, $\beta$, etc.

{\footnotesize\begin{quotation}
{\em Comment.} The variables $v_0, v_1, \ldots$ of our formal system
correspond to what are usually considered ``metavariables'' in
descriptions of specific formal systems in the literature.  Typically,
when describing a specific formal system a book will postulate a set of
primitive objects called variables, then proceed to describe their
properties using metavariables that range over them, never mentioning
again the actual variables themselves.  Our formal system does not
mention these primitive variable objects at all but deals directly with
metavariables, as its primitive objects, from the start.  This is a
subtle but key distinction you should keep in mind, and it makes our
definition of ``formal system'' somewhat different from that typically
found in the literature.  (So, the $\alpha$, $\beta$, etc.\ above are
actually ``metametavariables'' when used to represent $v_0, v_1,
\ldots$.)
\end{quotation}}

Finite sequences all terms of which are symbols are called {\em
expressions}.\index{expression!in a formal system}  $\mbox{\em EX}$ is
the set of all expressions; thus
\begin{displaymath}
\mbox{\em EX} = \bigcup _{n \in \omega} {}^n \mbox{\em SM}.
\end{displaymath}

A {\em constant-prefixed expression}\index{constant-prefixed expression}
is an expression of non-zero length
whose first term is a constant.  We denote the set of all constant-prefixed
expressions by $\mbox{\em EX}_C = \{ e \in \mbox{\em EX} | ( |e| > 0 \wedge
e_0 \in \mbox{\em CN} ) \}$.

A {\em constant-variable pair}\index{constant-variable pair}
is an expression of length 2 whose first term
is a constant and whose second term is a variable.  We denote the set of all
constant-variable pairs by $\mbox{\em EX}_2 = \{ e \in \mbox{\em EX}_C | ( |e|
= 2 \wedge e_1 \in \mbox{\em VR} ) \}$.


{\footnotesize\begin{quotation}
{\em Relationship to Metamath.} In general, the set $\mbox{\em SM}$
corresponds to the set of declared math symbols in a Metamath database, the
set $\mbox{\em CN}$ to those declared with \texttt{\$c} statements, and the set
$\mbox{\em VR}$ to those declared with \texttt{\$v} statements.  Of course a
Metamath database can only have a finite number of math symbols, whereas
formal systems in general can have an infinite number, although the number of
Metamath math symbols available is in principle unlimited.

The set $\mbox{\em EX}_C$ corresponds to the set of permissible expressions
for \texttt{\$e}, \texttt{\$a}, and \texttt{\$p} statements.  The set $\mbox{\em EX}_2$
corresponds to the set of permissible expressions for \texttt{\$f} statements.
\end{quotation}}

We denote by ${\cal V}(e)$ the set of all variables in an expression $e \in
\mbox{\em EX}$, i.e.\ the set of all $\alpha \in \mbox{\em VR}$ such that
$\alpha = e_n$ for some $n < |e|$.  We also denote (with abuse of notation) by
${\cal V}(E)$ the set of all variables in a collection of expressions $E
\subseteq \mbox{\em EX}$, i.e.\ $\bigcup _{e \in E} {\cal V}(e)$.


\subsection{Substitution}

Given a function $F$ from $\mbox{\em VR}$ to
$\mbox{\em EX}$, we
denote by $\sigma_{F}$ or just $\sigma$ the function from $\mbox{\em EX}$ to
$\mbox{\em EX}$ defined recursively for nonempty sequences by
\begin{eqnarray*}
& \sigma(<\alpha>) = F(\alpha) & \mbox{for\ } \alpha \in \mbox{\em VR}; \\
& \sigma(<\alpha>) = <\alpha> & \mbox{for\ } \alpha \not\in \mbox{\em VR}; \\
& \sigma(g \frown h) = \sigma(g) \frown
    \sigma(h) & \mbox{for\ } g,h \in \mbox{\em EX}.
\end{eqnarray*}
We also define $\sigma(\varnothing)=\varnothing$.  We call $\sigma$ a {\em
simultaneous substitution}\index{substitution!variable}\index{variable
substitution} (or just {\em substitution}) with {\em substitution
map}\index{substitution map} $F$.

We also denote (with abuse of notation) by $\sigma(E)$ a substitution on a
collection of expressions $E \subseteq \mbox{\em EX}$, i.e.\ the set $\{
\sigma(e) | e \in E \}$.  The collection $\sigma(E)$ may of course contain
fewer expressions than $E$ because duplicate expressions could result from the
substitution.

\subsection{Statements}

We denote by $\mbox{\em DV}$ the set of all
unordered pairs $\{\alpha, \beta \} \subseteq \mbox{\em VR}$ such that $\alpha
\neq \beta$.  $\mbox{\em DV}$ stands for ``distinct variables.''

A {\em pre-statement}\index{pre-statement!in a formal system} is a
quadruple $\langle D,T,H,A \rangle$ such that
$D\subseteq \mbox{\em DV}$, $T\subseteq \mbox{\em EX}_2$, $H\subseteq
\mbox{\em EX}_C$ and $H$ is finite,
$A\in \mbox{\em EX}_C$, ${\cal V}(H\cup\{A\}) \subseteq
{\cal V}(T)$, and $\forall e,f\in T {\ } {\cal V}(e) \neq {\cal V}(f)$ (or
equivalently, $e_1 \ne f_1$) whenever $e \neq f$. The terms of the quadruple are called {\em
distinct-variable restrictions},\index{disjoint-variable restriction!in a
formal system} {\em variable-type hypotheses},\index{variable-type
hypothesis!in a formal system} {\em logical hypotheses},\index{logical
hypothesis!in a formal system} and the {\em assertion}\index{assertion!in a
formal system} respectively.  We denote by $T_M$ ({\em mandatory variable-type
hypotheses}\index{mandatory variable-type hypothesis!in a formal system}) the
subset of $T$ such that ${\cal V}(T_M) ={\cal V}(H \cup \{A\})$.  We denote by
$D_M=\{\{\alpha,\beta\}\in D|\{\alpha,\beta\}\subseteq {\cal V}(T_M)\}$ the
{\em mandatory distinct-variable restrictions}\index{mandatory
disjoint-variable restriction!in a formal system} of the pre-statement.
The set
of {\em mandatory hypotheses}\index{mandatory hypothesis!in a formal system}
is $T_M\cup H$.  We call the quadruple $\langle D_M,T_M,H,A \rangle$
the {\em reduct}\index{reduct!in a formal system} of
the pre-statement $\langle D,T,H,A \rangle$.

A {\em statement} is the reduct of some pre-statement\index{statement!in a
formal system}.  A statement is therefore a special kind of pre-statement;
in particular, a statement is the reduct of itself.

{\footnotesize\begin{quotation}
{\em Comment.}  $T$ is a set of expressions, each of length 2, that associate
a set of constants (``variable types'') with a set of variables.  The
condition ${\cal V}(H\cup\{A\}) \subseteq {\cal V}(T) $
means that each variable occurring in a statement's logical
hypotheses or assertion must have an associated variable-type hypothesis or
``type declaration,'' in  analogy to a computer programming language, where a
variable must be declared to be say, a string or an integer.  The requirement
that $\forall e,f\in T \, e_1 \ne f_1$ for $e\neq f$
means that each variable must be
associated with a unique constant designating its variable type; e.g., a
variable might be a ``wff'' or a ``set'' but not both.

Distinct-variable restrictions are used to specify what variable substitutions
are permissible to make for the statement to remain valid.  For example, in
the theorem scheme of set theory $\lnot\forall x\,x=y$ we may not substitute
the same variable for both $x$ and $y$.  On the other hand, the theorem scheme
$x=y\to y=x$ does not require that $x$ and $y$ be distinct, so we do not
require a distinct-variable restriction, although having one
would cause no harm other than making the scheme less general.

A mandatory variable-type hypothesis is one whose variable exists in a logical
hypothesis or the assertion.  A provable pre-statement
(defined below) may require
non-mandatory variable-type hypotheses that effectively introduce ``dummy''
variables for use in its proof.  Any number of dummy variables might
be required by a specific proof; indeed, it has been shown by H.\
Andr\'{e}ka\index{Andr{\'{e}}ka, H.} \cite{Nemeti} that there is no finite
upper bound to the number of dummy variables needed to prove an arbitrary
theorem in first-order logic (with equality) having a fixed number $n>2$ of
individual variables.  (See also the Comment on p.~\pageref{nodd}.)
For this reason we do not set a finite size bound on the collections $D$ and
$T$, although in an actual application (Metamath database) these will of
course be finite, increased to whatever size is necessary as more
proofs are added.
\end{quotation}}

{\footnotesize\begin{quotation}
{\em Relationship to Metamath.} A pre-statement of a formal system
corresponds to an extended frame in a Metamath database
(Section~\ref{frames}).  The collections $D$, $T$, and $H$ correspond
respectively to the \texttt{\$d}, \texttt{\$f}, and \texttt{\$e}
statement collections in an extended frame.  The expression $A$
corresponds to the \texttt{\$a} (or \texttt{\$p}) statement in an
extended frame.

A statement of a formal system corresponds to a frame in a Metamath
database.
\end{quotation}}

\subsection{Formal Systems}

A {\em formal system}\index{formal system} is a
triple $\langle \mbox{\em CN},\mbox{\em
VR},\Gamma\rangle$ where $\Gamma$ is a set of statements.  The members of
$\Gamma$ are called {\em axiomatic statements}.\index{axiomatic
statement!in a formal system}  Sometimes we will refer to a
formal system by just $\Gamma$ when $\mbox{\em CN}$ and $\mbox{\em VR}$ are
understood.

Given a formal system $\Gamma$, the {\em closure}\index{closure}\footnote{This
definition of closure incorporates a simplification due to
Josh Purinton.\index{Purinton, Josh}.} of a
pre-statement
$\langle D,T,H,A \rangle$ is the smallest set $C$ of expressions
such that:
%\begin{enumerate}
%  \item $T\cup H\subseteq C$; and
%  \item If for some axiomatic statement
%    $\langle D_M',T_M',H',A' \rangle \in \Gamma_A$, for
%    some $E \subseteq C$, some $F \subseteq C-T$ (where ``-'' denotes
%    set difference), and some substitution
%    $\sigma$ we have
%    \begin{enumerate}
%       \item $\sigma(T_M') = E$ (where, as above, the $M$ denotes the
%           mandatory variable-type hypotheses of $T^A$);
%       \item $\sigma(H') = F$;
%       \item for all $\{\alpha,\beta\}\in D^A$ and $\subseteq
%         {\cal V}(T_M')$, for all $\gamma\in {\cal V}(\sigma(\langle \alpha
%         \rangle))$, and for all $\delta\in  {\cal V}(\sigma(\langle \beta
%         \rangle))$, we have $\{\gamma, \delta\} \in D$;
%   \end{enumerate}
%   then $\sigma(A') \in C$.
%\end{enumerate}
\begin{list}{}{\itemsep 0.0pt}
  \item[1.] $T\cup H\subseteq C$; and
  \item[2.] If for some axiomatic statement
    $\langle D_M',T_M',H',A' \rangle \in
       \Gamma$ and for some substitution
    $\sigma$ we have
    \begin{enumerate}
       \item[a.] $\sigma(T_M' \cup H') \subseteq C$; and
       \item[b.] for all $\{\alpha,\beta\}\in D_M'$, for all $\gamma\in
         {\cal V}(\sigma(\langle \alpha
         \rangle))$, and for all $\delta\in  {\cal V}(\sigma(\langle \beta
         \rangle))$, we have $\{\gamma, \delta\} \in D$;
   \end{enumerate}
   then $\sigma(A') \in C$.
\end{list}
A pre-statement $\langle D,T,H,A
\rangle$ is {\em provable}\index{provable statement!in a formal
system} if $A\in C$ i.e.\ if its assertion belongs to its
closure.  A statement is {\em provable} if it is
the reduct of a provable pre-statement.
The {\em universe}\index{universe of a formal system}
of a formal system is
the collection of all of its provable statements.  Note that the
set of axiomatic statements $\Gamma$ in a formal system is a subset of its
universe.

{\footnotesize\begin{quotation}
{\em Comment.} The first condition in the definition of closure simply says
that the hypotheses of the pre-statement are in its closure.

Condition 2(a) says that a substitution exists that makes the
mandatory hypotheses of an axiomatic statement exactly match some members of
the closure.  This is what we explicitly demonstrate in a Metamath language
proof.

%Conditions 2(a) and 2(b) say that a substitution exists that makes the
%(mandatory) hypotheses of an axiomatic statement exactly match some members of
%the closure.  This is what we explicitly demonstrate with a Metamath language
%proof.
%
%The set of expressions $F$ in condition 2(b) excludes the variable-type
%hypotheses; this is done because non-mandatory variable-type hypotheses are
%effectively ``dropped'' as irrelevant whereas logical hypotheses must be
%retained to achieve a consistent logical system.

Condition 2(b) describes how distinct-variable restrictions in the axiomatic
statement must be met.  It means that after a substitution for two variables
that must be distinct, the resulting two expressions must either contain no
variables, or if they do, they may not have variables in common, and each pair
of any variables they do have, with one variable from each expression, must be
specified as distinct in the original statement.
\end{quotation}}

{\footnotesize\begin{quotation}
{\em Relationship to Metamath.} Axiomatic statements
 and provable statements in a formal
system correspond to the frames for \texttt{\$a} and \texttt{\$p} statements
respectively in a Metamath database.  The set of axiomatic statements is a
subset of the set of provable statements in a formal system, although in a
Metamath database a \texttt{\$a} statement is distinguished by not having a
proof.  A Metamath language proof for a \texttt{\$p} statement tells the computer
how to explicitly construct a series of members of the closure ultimately
leading to a demonstration that the assertion
being proved is in the closure.  The actual closure typically contains
an infinite number of expressions.  A formal system itself does not have
an explicit object called a ``proof'' but rather the existence of a proof
is implied indirectly by membership of an assertion in a provable
statement's closure.  We do this to make the formal system easier
to describe in the language of set theory.

We also note that once established as provable, a statement may be considered
to acquire the same status as an axiomatic statement, because if the set of
axiomatic statements is extended with a provable statement, the universe of
the formal system remains unchanged (provided that $\mbox{\em VR}$ is
infinite).
In practice, this means we can build a hierarchy of provable statements to
more efficiently establish additional provable statements.  This is
what we do in Metamath when we allow proofs to reference previous
\texttt{\$p} statements as well as previous \texttt{\$a} statements.
\end{quotation}}

\section{Examples of Formal Systems}

{\footnotesize\begin{quotation}
{\em Relationship to Metamath.} The examples in this section, except Example~2,
are for the most part exact equivalents of the development in the set
theory database \texttt{set.mm}.  You may want to compare Examples~1, 3, and 5
to Section~\ref{metaaxioms}, Example 4 to Sections~\ref{metadefprop} and
\ref{metadefpred}, and Example 6 to
Section~\ref{setdefinitions}.\label{exampleref}
\end{quotation}}

\subsection{Example~1---Propositional Calculus}\index{propositional calculus}

Classical propositional calculus can be described by the following formal
system.  We assume the set of variables is infinite.  Rather than denoting the
constants and variables by $c_0, c_1, \ldots$ and $v_0, v_1, \ldots$, for
readability we will instead use more conventional symbols, with the
understanding of course that they denote distinct primitive objects.
Also for readability we may omit commas between successive terms of a
sequence; thus $\langle \mbox{wff\ } \varphi\rangle$ denotes
$\langle \mbox{wff}, \varphi\rangle$.

Let
\begin{itemize}
  \item[] $\mbox{\em CN}=\{\mbox{wff}, \vdash, \to, \lnot, (,)\}$
  \item[] $\mbox{\em VR}=\{\varphi,\psi,\chi,\ldots\}$
  \item[] $T = \{\langle \mbox{wff\ } \varphi\rangle,
             \langle \mbox{wff\ } \psi\rangle,
             \langle \mbox{wff\ } \chi\rangle,\ldots\}$, i.e.\ those
             expressions of length 2 whose first member is $\mbox{\rm wff}$
             and whose second member belongs to $\mbox{\em VR}$.\footnote{For
convenience we let $T$ be an infinite set; the definition of a statement
permits this in principle.  Since a Metamath source file has a finite size, in
practice we must of course use appropriate finite subsets of this $T$,
specifically ones containing at least the mandatory variable-type
hypotheses.  Similarly, in the source file we introduce new variables as
required, with the understanding that a potentially infinite number of
them are available.}
\noindent Then $\Gamma$ consists of the axiomatic statements that
are the reducts of the following pre-statements:
    \begin{itemize}
      \item[] $\langle\varnothing,T,\varnothing,
               \langle \mbox{wff\ }(\varphi\to\psi)\rangle\rangle$
      \item[] $\langle\varnothing,T,\varnothing,
               \langle \mbox{wff\ }\lnot\varphi\rangle\rangle$
      \item[] $\langle\varnothing,T,\varnothing,
               \langle \vdash(\varphi\to(\psi\to\varphi))
               \rangle\rangle$
      \item[] $\langle\varnothing,T,
               \varnothing,
               \langle \vdash((\varphi\to(\psi\to\chi))\to
               ((\varphi\to\psi)\to(\varphi\to\chi)))
               \rangle\rangle$
      \item[] $\langle\varnothing,T,
               \varnothing,
               \langle \vdash((\lnot\varphi\to\lnot\psi)\to
               (\psi\to\varphi))\rangle\rangle$
      \item[] $\langle\varnothing,T,
               \{\langle\vdash(\varphi\to\psi)\rangle,
                 \langle\vdash\varphi\rangle\},
               \langle\vdash\psi\rangle\rangle$
    \end{itemize}
\end{itemize}

(For example, the reduct of $\langle\varnothing,T,\varnothing,
               \langle \mbox{wff\ }(\varphi\to\psi)\rangle\rangle$
is
\begin{itemize}
\item[] $\langle\varnothing,
\{\langle \mbox{wff\ } \varphi\rangle,
             \langle \mbox{wff\ } \psi\rangle\},
             \varnothing,
               \langle \mbox{wff\ }(\varphi\to\psi)\rangle\rangle$,
\end{itemize}
which is the first axiomatic statement.)

We call the members of $\mbox{\em VR}$ {\em wff variables} or (in the context
of first-order logic which we will describe shortly) {\em wff metavariables}.
Note that the symbols $\phi$, $\psi$, etc.\ denote actual specific members of
$\mbox{\em VR}$; they are not metavariables of our expository language (which
we denote with $\alpha$, $\beta$, etc.) but are instead (meta)constant symbols
(members of $\mbox{\em SM}$) from the point of view of our expository
language.  The equivalent system of propositional calculus described in
\cite{Tarski1965} also uses the symbols $\phi$, $\psi$, etc.\ to denote wff
metavariables, but in \cite{Tarski1965} unlike here those are metavariables of
the expository language and not primitive symbols of the formal system.

The first two statements define wffs: if $\varphi$ and $\psi$ are wffs, so is
$(\varphi \to \psi)$; if $\varphi$ is a wff, so is $\lnot\varphi$. The next
three are the axioms of propositional calculus: if $\varphi$ and $\psi$ are
wffs, then $\vdash (\varphi \to (\psi \to \varphi))$ is an (axiomatic)
theorem; etc. The
last is the rule of modus ponens: if $\varphi$ and $\psi$ are wffs, and
$\vdash (\varphi\to\psi)$ and $\vdash \varphi$ are theorems, then $\vdash
\psi$ is a theorem.

The correspondence to ordinary propositional calculus is as follows.  We
consider only provable statements of the form $\langle\varnothing,
T,\varnothing,A\rangle$ with $T$ defined as above.  The first term of the
assertion $A$ of any such statement is either ``wff'' or ``$\vdash$''.  A
statement for which the first term is ``wff'' is a {\em wff} of propositional
calculus, and one where the first term is ``$\vdash$'' is a {\em
theorem (scheme)} of propositional calculus.

The universe of this formal system also contains many other provable
statements.  Those with distinct-variable restrictions are irrelevant because
propositional calculus has no constraints on substitutions.  Those that have
logical hypotheses we call {\em inferences}\index{inference} when
the logical hypotheses are of the form
$\langle\vdash\rangle\frown w$ where $w$ is a wff (with the leading constant
term ``wff'' removed).  Inferences (other than the modus ponens rule) are not a
proper part of propositional calculus but are convenient to use when building a
hierarchy of provable statements.  A provable statement with a nonsense
hypothesis such as $\langle \to,\vdash,\lnot\rangle$, and this same expression
as its assertion, we consider irrelevant; no use can be made of it in
proving theorems, since there is no way to eliminate the nonsense hypothesis.

{\footnotesize\begin{quotation}
{\em Comment.} Our use of parentheses in the definition of a wff illustrates
how axiomatic statements should be carefully stated in a way that
ties in unambiguously with the substitutions allowed by the formal system.
There are many ways we could have defined wffs---for example, Polish
prefix notation would have allowed us to omit parentheses entirely, at
the expense of readability---but we must define them in a way that is
unambiguous.  For example, if we had omitted parentheses from the
definition of $(\varphi\to \psi)$, the wff $\lnot\varphi\to \psi$ could
be interpreted as either $\lnot(\varphi\to\psi)$ or $(\lnot\varphi\to\psi)$
and would have allowed us to prove nonsense.  Note that there is no
concept of operator binding precedence built into our formal system.
\end{quotation}}

\begin{sloppy}
\subsection{Example~2---Predicate Calculus with Equality}\index{predicate
calculus}
\end{sloppy}

Here we extend Example~1 to include predicate calculus with equality,
illustrating the use of distinct-variable restrictions.  This system is the
same as Tarski's system $\mathfrak{S}_2$ in \cite{Tarski1965} (except that the
axioms of propositional calculus are different but equivalent, and a redundant
axiom is omitted).  We extend $\mbox{\em CN}$ with the constants
$\{\mbox{var},\forall,=\}$.  We extend $\mbox{\em VR}$ with an infinite set of
{\em individual metavariables}\index{individual
metavariable} $\{x,y,z,\ldots\}$ and denote this subset
$\mbox{\em Vr}$.

We also join to $\mbox{\em CN}$ a possibly infinite set $\mbox{\em Pr}$ of {\em
predicates} $\{R,S,\ldots\}$.  We associate with $\mbox{\em Pr}$ a function
$\mbox{rnk}$ from $\mbox{\em Pr}$ to $\omega$, and for $\alpha\in \mbox{\em
Pr}$ we call $\mbox{rnk}(\alpha)$ the {\em rank} of the predicate $\alpha$,
which is simply the number of ``arguments'' that the predicate has.  (Most
applications of predicate calculus will have a finite number of predicates;
for example, set theory has the single two-argument or binary predicate $\in$,
which is usually written with its arguments surrounding the predicate symbol
rather than with the prefix notation we will use for the general case.)  As a
device to facilitate our discussion, we will let $\mbox{\em Vs}$ be any fixed
one-to-one function from $\omega$ to $\mbox{\em Vr}$; thus $\mbox{\em Vs}$ is
any simple infinite sequence of individual metavariables with no repeating
terms.

In this example we will not include the function symbols that are often part of
formalizations of predicate calculus.  Using metalogical arguments that are
beyond the scope of our discussion, it can be shown that our formalization is
equivalent when functions are introduced via appropriate definitions.

We extend the set $T$ defined in Example~1 with the expressions
$\{\langle \mbox{var\ } x\rangle,$ $ \langle \mbox{var\ } y\rangle, \langle
\mbox{var\ } z\rangle,\ldots\}$.  We extend the $\Gamma$ above
with the axiomatic statements that are the reducts of the following
pre-statements:
\begin{list}{}{\itemsep 0.0pt}
      \item[] $\langle\varnothing,T,\varnothing,
               \langle \mbox{wff\ }\forall x\,\varphi\rangle\rangle$
      \item[] $\langle\varnothing,T,\varnothing,
               \langle \mbox{wff\ }x=y\rangle\rangle$
      \item[] $\langle\varnothing,T,
               \{\langle\vdash\varphi\rangle\},
               \langle\vdash\forall x\,\varphi\rangle\rangle$
      \item[] $\langle\varnothing,T,\varnothing,
               \langle \vdash((\forall x(\varphi\to\psi)
                  \to(\forall x\,\varphi\to\forall x\,\psi))
               \rangle\rangle$
      \item[] $\langle\{\{x,\varphi\}\},T,\varnothing,
               \langle \vdash(\varphi\to\forall x\,\varphi)
               \rangle\rangle$
      \item[] $\langle\{\{x,y\}\},T,\varnothing,
               \langle \vdash\lnot\forall x\lnot x=y
               \rangle\rangle$
      \item[] $\langle\varnothing,T,\varnothing,
               \langle \vdash(x=z
                  \to(x=y\to z=y))
               \rangle\rangle$
      \item[] $\langle\varnothing,T,\varnothing,
               \langle \vdash(y=z
                  \to(x=y\to x=z))
               \rangle\rangle$
\end{list}
These are the axioms not involving predicate symbols. The first two statements
extend the definition of a wff.  The third is the rule of generalization.  The
fifth states, in effect, ``For a wff $\varphi$ and variable $x$,
$\vdash(\varphi\to\forall x\,\varphi)$, provided that $x$ does not occur in
$\varphi$.''  The sixth states ``For variables $x$ and $y$,
$\vdash\lnot\forall x\lnot x = y$, provided that $x$ and $y$ are distinct.''
(This proviso is not necessary but was included by Tarski to
weaken the axiom and still show that the system is logically complete.)

Finally, for each predicate symbol $\alpha\in \mbox{\em Pr}$, we add to
$\Gamma$ an axiomatic statement, extending the definition of wff,
that is the reduct of the following pre-statement:
\begin{displaymath}
    \langle\varnothing,T,\varnothing,
            \langle \mbox{wff},\alpha\rangle\
            \frown \mbox{\em Vs}\restriction\mbox{rnk}(\alpha)\rangle
\end{displaymath}
and for each $\alpha\in \mbox{\em Pr}$ and each $n < \mbox{rnk}(\alpha)$
we add to $\Gamma$ an equality axiom that is the reduct of the
following pre-statement:
\begin{eqnarray*}
    \lefteqn{\langle\varnothing,T,\varnothing,
            \langle
      \vdash,(,\mbox{\em Vs}_n,=,\mbox{\em Vs}_{\mbox{rnk}(\alpha)},\to,
            (,\alpha\rangle\frown \mbox{\em Vs}\restriction\mbox{rnk}(\alpha)} \\
  & & \frown
            \langle\to,\alpha\rangle\frown \mbox{\em Vs}\restriction n\frown
            \langle \mbox{\em Vs}_{\mbox{rnk}(\alpha)}\rangle \\
 & & \frown
            \mbox{\em Vs}\restriction(\mbox{rnk}(\alpha)\setminus(n+1))\frown
            \langle),)\rangle\rangle
\end{eqnarray*}
where $\restriction$ denotes function domain restriction and $\setminus$
denotes set difference.  Recall that a subscript on $\mbox{\em Vs}$
denotes one of its terms.  (In the above two axiom sets commas are placed
between successive terms of sequences to prevent ambiguity, and if you examine
them with care you will be able to distinguish those parentheses that denote
constant symbols from those of our expository language that delimit function
arguments.  Although it might have been better to use boldface for our
primitive symbols, unfortunately boldface was not available for all characters
on the \LaTeX\ system used to typeset this text.)  These seemingly forbidding
axioms can be understood by analogy to concatenation of substrings in a
computer language.  They are actually relatively simple for each specific case
and will become clearer by looking at the special case of a binary predicate
$\alpha = R$ where $\mbox{rnk}(R)=2$.  Letting $\mbox{\em Vs}$ be the sequence
$\langle x,y,z,\ldots\rangle$, the axioms we would add to $\Gamma$ for this
case would be the wff extension and two equality axioms that are the
reducts of the pre-statements:
\begin{list}{}{\itemsep 0.0pt}
      \item[] $\langle\varnothing,T,\varnothing,
               \langle \mbox{wff\ }R x y\rangle\rangle$
      \item[] $\langle\varnothing,T,\varnothing,
               \langle \vdash(x=z
                  \to(R x y \to R z y))
               \rangle\rangle$
      \item[] $\langle\varnothing,T,\varnothing,
               \langle \vdash(y=z
                  \to(R x y \to R x z))
               \rangle\rangle$
\end{list}
Study these carefully to see how the general axioms above evaluate to
them.  In practice, typically only a few special cases such as this would be
needed, and in any case the Metamath language will only permit us to describe
a finite number of predicates, as opposed to the infinite number permitted by
the formal system.  (If an infinite number should be needed for some reason,
we could not define the formal system directly in the Metamath language but
could instead define it metalogically under set theory as we
do in this appendix, and only the underlying set theory, with its single
binary predicate, would be defined directly in the Metamath language.)


{\footnotesize\begin{quotation}
{\em Comment.}  As we noted earlier, the specific variables denoted by the
symbols $x,y,z,\ldots\in \mbox{\em Vr}\subseteq \mbox{\em VR}\subseteq
\mbox{\em SM}$ in Example~2 are not the actual variables of ordinary predicate
calculus but should be thought of as metavariables ranging over them.  For
example, a distinct-variable restriction would be meaningless for actual
variables of ordinary predicate calculus since two different actual variables
are by definition distinct.  And when we talk about an arbitrary
representative $\alpha\in \mbox{\em Vr}$, $\alpha$ is a metavariable (in our
expository language) that ranges over metavariables (which are primitives of
our formal system) each of which ranges over the actual individual variables
of predicate calculus (which are never mentioned in our formal system).

The constant called ``var'' above is called \texttt{setvar} in the
\texttt{set.mm} database file, but it means the same thing.  I felt
that ``var'' is a more meaningful name in the context of predicate
calculus, whose use is not limited to set theory.  For consistency we
stick with the name ``var'' throughout this Appendix, even after set
theory is introduced.
\end{quotation}}

\subsection{Free Variables and Proper Substitution}\index{free variable}
\index{proper substitution}\index{substitution!proper}

Typical representations of mathematical axioms use concepts such
as ``free variable,'' ``bound variable,'' and ``proper substitution''
as primitive notions.
A free variable is a variable that
is not a parameter of any container expression.
A bound variable is the opposite of a free variable; it is a
a variable that has been bound in a container expression.
For example, in the expression $\forall x \varphi$ (for all $x$, $\varphi$
is true), the variable $x$
is bound within the for-all ($\forall$) expression.
It is possible to change one variable to another, and that process is called
``proper substitution.''
In most books, proper substitution has a somewhat complicated recursive
definition with multiple cases based on the occurrences of free and
bound variables.
You may consult
\cite[ch.\ 3--4]{Hamilton}\index{Hamilton, Alan G.} (as well as
many other texts) for more formal details about these terms.

Using these concepts as \texttt{primitives} creates complications
for computer implementations.

In the system of Example~2, there are no primitive notions of free variable
and proper substitution.  Tarski \cite{Tarski1965} shows that this system is
logically equivalent to the more typical textbook systems that do have these
primitive notions, if we introduce these notions with appropriate definitions
and metalogic.  We could also define axioms for such systems directly,
although the recursive definitions of free variable and proper substitution
would be messy and awkward to work with.  Instead, we mention two devices that
can be used in practice to mimic these notions.  (1) Instead of introducing
special notation to express (as a logical hypothesis) ``where $x$ is not free
in $\varphi$'' we can use the logical hypothesis $\vdash(\varphi\to\forall
x\,\varphi)$.\label{effectivelybound}\index{effectively
not free}\footnote{This is a slightly weaker requirement than ``where $x$ is
not free in $\varphi$.''  If we let $\varphi$ be $x=x$, we have the theorem
$(x=x\to\forall x\,x=x)$ which satisfies the hypothesis, even though $x$ is
free in $x=x$ .  In a case like this we say that $x$ is {\em effectively not
free}\index{effectively not free} in $x=x$, since $x=x$ is logically
equivalent to $\forall x\,x=x$ in which $x$ is bound.} (2) It can be shown
that the wff $((x=y\to\varphi)\wedge\exists x(x=y\wedge\varphi))$ (with the
usual definitions of $\wedge$ and $\exists$; see Example~4 below) is logically
equivalent to ``the wff that results from proper substitution of $y$ for $x$
in $\varphi$.''  This works whether or not $x$ and $y$ are distinct.

\subsection{Metalogical Completeness}\index{metalogical completeness}

In the system of Example~2, the
following are provable pre-statements (and their reducts are
provable statements):
\begin{eqnarray*}
      & \langle\{\{x,y\}\},T,\varnothing,
               \langle \vdash\lnot\forall x\lnot x=y
               \rangle\rangle & \\
     &  \langle\varnothing,T,\varnothing,
               \langle \vdash\lnot\forall x\lnot x=x
               \rangle\rangle &
\end{eqnarray*}
whereas the following pre-statement is not to my knowledge provable (but
in any case we will pretend it's not for sake of illustration):
\begin{eqnarray*}
     &  \langle\varnothing,T,\varnothing,
               \langle \vdash\lnot\forall x\lnot x=y
               \rangle\rangle &
\end{eqnarray*}
In other words, we can prove ``$\lnot\forall x\lnot x=y$ where $x$ and $y$ are
distinct'' and separately prove ``$\lnot\forall x\lnot x=x$'', but we can't
prove the combined general case ``$\lnot\forall x\lnot x=y$'' that has no
proviso.  Now this does not compromise logical completeness, because the
variables are really metavariables and the two provable cases together cover
all possible cases.  The third case can be considered a metatheorem whose
direct proof, using the system of Example~2, lies outside the capability of the
formal system.

Also, in the system of Example~2 the following pre-statement is not to my
knowledge provable (again, a conjecture that we will pretend to be the case):
\begin{eqnarray*}
     & \langle\varnothing,T,\varnothing,
               \langle \vdash(\forall x\, \varphi\to\varphi)
               \rangle\rangle &
\end{eqnarray*}
Instead, we can only prove specific cases of $\varphi$ involving individual
metavariables, and by induction on formula length, prove as a metatheorem
outside of our formal system the general statement above.  The details of this
proof are found in \cite{Kalish}.

There does, however, exist a system of predicate calculus in which all such
``simple metatheorems'' as those above can be proved directly, and we present
it in Example~3. A {\em simple metatheorem}\index{simple metatheorem}
is any statement of the formal
system of Example~2 where all distinct variable restrictions consist of either
two individual metavariables or an individual metavariable and a wff
metavariable, and which is provable by combining cases outside the system as
above.  A system is {\em metalogically complete}\index{metalogical
completeness} if all of its simple
metatheorems are (directly) provable statements. The precise definition of
``simple metatheorem'' and the proof of the ``metalogical completeness'' of
Example~3 is found in Remark 9.6 and Theorem 9.7 of \cite{Megill}.\index{Megill,
Norman}

\begin{sloppy}
\subsection{Example~3---Metalogically Complete Predicate
Calculus with
Equality}
\end{sloppy}

For simplicity we will assume there is one binary predicate $R$;
this system suffices for set theory, where the $R$ is of course the $\in$
predicate.  We label the axioms as they appear in \cite{Megill}.  This
system is logically equivalent to that of Example~2 (when the latter is
restricted to this single binary predicate) but is also metalogically
complete.\index{metalogical completeness}

Let
\begin{itemize}
  \item[] $\mbox{\em CN}=\{\mbox{wff}, \mbox{var}, \vdash, \to, \lnot, (,),\forall,=,R\}$.
  \item[] $\mbox{\em VR}=\{\varphi,\psi,\chi,\ldots\}\cup\{x,y,z,\ldots\}$.
  \item[] $T = \{\langle \mbox{wff\ } \varphi\rangle,
             \langle \mbox{wff\ } \psi\rangle,
             \langle \mbox{wff\ } \chi\rangle,\ldots\}\cup
       \{\langle \mbox{var\ } x\rangle, \langle \mbox{var\ } y\rangle, \langle
       \mbox{var\ }z\rangle,\ldots\}$.

\noindent Then
  $\Gamma$ consists of the reducts of the following pre-statements:
    \begin{itemize}
      \item[] $\langle\varnothing,T,\varnothing,
               \langle \mbox{wff\ }(\varphi\to\psi)\rangle\rangle$
      \item[] $\langle\varnothing,T,\varnothing,
               \langle \mbox{wff\ }\lnot\varphi\rangle\rangle$
      \item[] $\langle\varnothing,T,\varnothing,
               \langle \mbox{wff\ }\forall x\,\varphi\rangle\rangle$
      \item[] $\langle\varnothing,T,\varnothing,
               \langle \mbox{wff\ }x=y\rangle\rangle$
      \item[] $\langle\varnothing,T,\varnothing,
               \langle \mbox{wff\ }Rxy\rangle\rangle$
      \item[(C1$'$)] $\langle\varnothing,T,\varnothing,
               \langle \vdash(\varphi\to(\psi\to\varphi))
               \rangle\rangle$
      \item[(C2$'$)] $\langle\varnothing,T,
               \varnothing,
               \langle \vdash((\varphi\to(\psi\to\chi))\to
               ((\varphi\to\psi)\to(\varphi\to\chi)))
               \rangle\rangle$
      \item[(C3$'$)] $\langle\varnothing,T,
               \varnothing,
               \langle \vdash((\lnot\varphi\to\lnot\psi)\to
               (\psi\to\varphi))\rangle\rangle$
      \item[(C4$'$)] $\langle\varnothing,T,
               \varnothing,
               \langle \vdash(\forall x(\forall x\,\varphi\to\psi)\to
                 (\forall x\,\varphi\to\forall x\,\psi))\rangle\rangle$
      \item[(C5$'$)] $\langle\varnothing,T,
               \varnothing,
               \langle \vdash(\forall x\,\varphi\to\varphi)\rangle\rangle$
      \item[(C6$'$)] $\langle\varnothing,T,
               \varnothing,
               \langle \vdash(\forall x\forall y\,\varphi\to
                 \forall y\forall x\,\varphi)\rangle\rangle$
      \item[(C7$'$)] $\langle\varnothing,T,
               \varnothing,
               \langle \vdash(\lnot\varphi\to\forall x\lnot\forall x\,\varphi
                 )\rangle\rangle$
      \item[(C8$'$)] $\langle\varnothing,T,
               \varnothing,
               \langle \vdash(x=y\to(x=z\to y=z))\rangle\rangle$
      \item[(C9$'$)] $\langle\varnothing,T,
               \varnothing,
               \langle \vdash(\lnot\forall x\, x=y\to(\lnot\forall x\, x=z\to
                 (y=z\to\forall x\, y=z)))\rangle\rangle$
      \item[(C10$'$)] $\langle\varnothing,T,
               \varnothing,
               \langle \vdash(\forall x(x=y\to\forall x\,\varphi)\to
                 \varphi))\rangle\rangle$
      \item[(C11$'$)] $\langle\varnothing,T,
               \varnothing,
               \langle \vdash(\forall x\, x=y\to(\forall x\,\varphi
               \to\forall y\,\varphi))\rangle\rangle$
      \item[(C12$'$)] $\langle\varnothing,T,
               \varnothing,
               \langle \vdash(x=y\to(Rxz\to Ryz))\rangle\rangle$
      \item[(C13$'$)] $\langle\varnothing,T,
               \varnothing,
               \langle \vdash(x=y\to(Rzx\to Rzy))\rangle\rangle$
      \item[(C15$'$)] $\langle\varnothing,T,
               \varnothing,
               \langle \vdash(\lnot\forall x\, x=y\to(x=y\to(\varphi
                 \to\forall x(x=y\to\varphi))))\rangle\rangle$
      \item[(C16$'$)] $\langle\{\{x,y\}\},T,
               \varnothing,
               \langle \vdash(\forall x\, x=y\to(\varphi\to\forall x\,\varphi)
                 )\rangle\rangle$
      \item[(C5)] $\langle\{\{x,\varphi\}\},T,\varnothing,
               \langle \vdash(\varphi\to\forall x\,\varphi)
               \rangle\rangle$
      \item[(MP)] $\langle\varnothing,T,
               \{\langle\vdash(\varphi\to\psi)\rangle,
                 \langle\vdash\varphi\rangle\},
               \langle\vdash\psi\rangle\rangle$
      \item[(Gen)] $\langle\varnothing,T,
               \{\langle\vdash\varphi\rangle\},
               \langle\vdash\forall x\,\varphi\rangle\rangle$
    \end{itemize}
\end{itemize}

While it is known that these axioms are ``metalogically complete,'' it is
not known whether they are independent (i.e.\ none is
redundant) in the metalogical sense; specifically, whether any axiom (possibly
with additional non-mandatory distinct-variable restrictions, for use with any
dummy variables in its proof) is provable from the others.  Note that
metalogical independence is a weaker requirement than independence in the
usual logical sense.  Not all of the above axioms are logically independent:
for example, C9$'$ can be proved as a metatheorem from the others, outside the
formal system, by combining the possible cases of distinct variables.

\subsection{Example~4---Adding Definitions}\index{definition}
There are several ways to add definitions to a formal system.  Probably the
most proper way is to consider definitions not as part of the formal system at
all but rather as abbreviations that are part of the expository metalogic
outside the formal system.  For convenience, though, we may use the formal
system itself to incorporate definitions, adding them as axiomatic extensions
to the system.  This could be done by adding a constant representing the
concept ``is defined as'' along with axioms for it. But there is a nicer way,
at least in this writer's opinion, that introduces definitions as direct
extensions to the language rather than as extralogical primitive notions.  We
introduce additional logical connectives and provide axioms for them.  For
systems of logic such as Examples 1 through 3, the additional axioms must be
conservative in the sense that no wff of the original system that was not a
theorem (when the initial term ``wff'' is replaced by ``$\vdash$'' of course)
becomes a theorem of the extended system.  In this example we extend Example~3
(or 2) with standard abbreviations of logic.

We extend $\mbox{\em CN}$ of Example~3 with new constants $\{\leftrightarrow,
\wedge,\vee,\exists\}$, corresponding to logical equivalence,\index{logical
equivalence ($\leftrightarrow$)}\index{biconditional ($\leftrightarrow$)}
conjunction,\index{conjunction ($\wedge$)} disjunction,\index{disjunction
($\vee$)} and the existential quantifier.\index{existential quantifier
($\exists$)}  We extend $\Gamma$ with the axiomatic statements that are
the reducts of the following pre-statements:
\begin{list}{}{\itemsep 0.0pt}
      \item[] $\langle\varnothing,T,\varnothing,
               \langle \mbox{wff\ }(\varphi\leftrightarrow\psi)\rangle\rangle$
      \item[] $\langle\varnothing,T,\varnothing,
               \langle \mbox{wff\ }(\varphi\vee\psi)\rangle\rangle$
      \item[] $\langle\varnothing,T,\varnothing,
               \langle \mbox{wff\ }(\varphi\wedge\psi)\rangle\rangle$
      \item[] $\langle\varnothing,T,\varnothing,
               \langle \mbox{wff\ }\exists x\, \varphi\rangle\rangle$
  \item[] $\langle\varnothing,T,\varnothing,
     \langle\vdash ( ( \varphi \leftrightarrow \psi ) \to
     ( \varphi \to \psi ) )\rangle\rangle$
  \item[] $\langle\varnothing,T,\varnothing,
     \langle\vdash ((\varphi\leftrightarrow\psi)\to
    (\psi\to\varphi))\rangle\rangle$
  \item[] $\langle\varnothing,T,\varnothing,
     \langle\vdash ((\varphi\to\psi)\to(
     (\psi\to\varphi)\to(\varphi
     \leftrightarrow\psi)))\rangle\rangle$
  \item[] $\langle\varnothing,T,\varnothing,
     \langle\vdash (( \varphi \wedge \psi ) \leftrightarrow\neg ( \varphi
     \to \neg \psi )) \rangle\rangle$
  \item[] $\langle\varnothing,T,\varnothing,
     \langle\vdash (( \varphi \vee \psi ) \leftrightarrow (\neg \varphi
     \to \psi )) \rangle\rangle$
  \item[] $\langle\varnothing,T,\varnothing,
     \langle\vdash (\exists x \,\varphi\leftrightarrow
     \lnot \forall x \lnot \varphi)\rangle\rangle$
\end{list}
The first three logical axioms (statements containing ``$\vdash$'') introduce
and effectively define logical equivalence, ``$\leftrightarrow$''.  The last
three use ``$\leftrightarrow$'' to effectively mean ``is defined as.''

\subsection{Example~5---ZFC Set Theory}\index{ZFC set theory}

Here we add to the system of Example~4 the axioms of Zermelo--Fraenkel set
theory with Choice.  For convenience we make use of the
definitions in Example~4.

In the $\mbox{\em CN}$ of Example~4 (which extends Example~3), we replace the symbol $R$
with the symbol $\in$.
More explicitly, we remove from $\Gamma$ of Example~4 the three
axiomatic statements containing $R$ and replace them with the
reducts of the following:
\begin{list}{}{\itemsep 0.0pt}
      \item[] $\langle\varnothing,T,\varnothing,
               \langle \mbox{wff\ }x\in y\rangle\rangle$
      \item[] $\langle\varnothing,T,
               \varnothing,
               \langle \vdash(x=y\to(x\in z\to y\in z))\rangle\rangle$
      \item[] $\langle\varnothing,T,
               \varnothing,
               \langle \vdash(x=y\to(z\in x\to z\in y))\rangle\rangle$
\end{list}
Letting $D=\{\{\alpha,\beta\}\in \mbox{\em DV}\,|\alpha,\beta\in \mbox{\em
Vr}\}$ (in other words all individual variables must be distinct), we extend
$\Gamma$ with the ZFC axioms, called
\index{Axiom of Extensionality}
\index{Axiom of Replacement}
\index{Axiom of Union}
\index{Axiom of Power Sets}
\index{Axiom of Regularity}
\index{Axiom of Infinity}
\index{Axiom of Choice}
Extensionality, Replacement, Union, Power
Set, Regularity, Infinity, and Choice, that are the reducts of:
\begin{list}{}{\itemsep 0.0pt}
      \item[Ext] $\langle D,T,
               \varnothing,
               \langle\vdash (\forall x(x\in y\leftrightarrow x \in z)\to y
               =z) \rangle\rangle$
      \item[Rep] $\langle D,T,
               \varnothing,
               \langle\vdash\exists x ( \exists y \forall z (\varphi \to z = y
                        ) \to
                        \forall z ( z \in x \leftrightarrow \exists x ( x \in
                        y \wedge \forall y\,\varphi ) ) )\rangle\rangle$
      \item[Un] $\langle D,T,
               \varnothing,
               \langle\vdash \exists x \forall y ( \exists x ( y \in x \wedge
               x \in z ) \to y \in x ) \rangle\rangle$
      \item[Pow] $\langle D,T,
               \varnothing,
               \langle\vdash \exists x \forall y ( \forall x ( x \in y \to x
               \in z ) \to y \in x ) \rangle\rangle$
      \item[Reg] $\langle D,T,
               \varnothing,
               \langle\vdash (  x \in y \to
                 \exists x ( x \in y \wedge \forall z ( z \in x \to \lnot z
                \in y ) ) ) \rangle\rangle$
      \item[Inf] $\langle D,T,
               \varnothing,
               \langle\vdash \exists x(y\in x\wedge\forall y(y\in
               x\to
               \exists z(y \in z\wedge z\in x))) \rangle\rangle$
      \item[AC] $\langle D,T,
               \varnothing,
               \langle\vdash \exists x \forall y \forall z ( ( y \in z
               \wedge z \in w ) \to \exists w \forall y ( \exists w
              ( ( y \in z \wedge z \in w ) \wedge ( y \in w \wedge w \in x
              ) ) \leftrightarrow y = w ) ) \rangle\rangle$
\end{list}

\subsection{Example~6---Class Notation in Set Theory}\label{class}

A powerful device that makes set theory easier (and that we have
been using all along in our informal expository language) is {\em class
abstraction notation}.\index{class abstraction}\index{abstraction class}  The
definitions we introduce are rigorously justified
as conservative by Takeuti and Zaring \cite{Takeuti}\index{Takeuti, Gaisi} or
Quine \cite{Quine}\index{Quine, Willard Van Orman}.  The key idea is to
introduce the notation $\{x|\mbox{---}\}$ which means ``the class of all $x$
such that ---'' for abstraction classes and introduce (meta)variables that
range over them.  An abstraction class may or may not be a set, depending on
whether it exists (as a set).  A class that does not exist is
called a {\em proper class}.\index{proper class}\index{class!proper}

To illustrate the use of abstraction classes we will provide some examples
of definitions that make use of them:  the empty set, class union, and
unordered pair.  Many other such definitions can be found in the
Metamath set theory database,
\texttt{set.mm}.\index{set theory database (\texttt{set.mm})}

% We intentionally break up the sequence of math symbols here
% because otherwise the overlong line goes beyond the page in narrow mode.
We extend $\mbox{\em CN}$ of Example~5 with new symbols $\{$
$\mbox{class},$ $\{,$ $|,$ $\},$ $\varnothing,$ $\cup,$ $,$ $\}$
where the inner braces and last comma are
constant symbols. (As before,
our dual use of some mathematical symbols for both our expository
language and as primitives of the formal system should be clear from context.)

We extend $\mbox{\em VR}$ of Example~5 with a set of {\em class
variables}\index{class variable}
$\{A,B,C,\ldots\}$. We extend the $T$ of Example~5 with $\{\langle
\mbox{class\ } A\rangle, \langle \mbox{class\ }B\rangle, \langle \mbox{class\ }
C\rangle,\ldots\}$.

To
introduce our definitions,
we add to $\Gamma$ of Example~5 the axiomatic statements
that are the reducts of the following pre-statements:
\begin{list}{}{\itemsep 0.0pt}
      \item[] $\langle\varnothing,T,\varnothing,
               \langle \mbox{class\ }x\rangle\rangle$
      \item[] $\langle\varnothing,T,\varnothing,
               \langle \mbox{class\ }\{x|\varphi\}\rangle\rangle$
      \item[] $\langle\varnothing,T,\varnothing,
               \langle \mbox{wff\ }A=B\rangle\rangle$
      \item[] $\langle\varnothing,T,\varnothing,
               \langle \mbox{wff\ }A\in B\rangle\rangle$
      \item[Ab] $\langle\varnothing,T,\varnothing,
               \langle \vdash ( y \in \{ x |\varphi\} \leftrightarrow
                  ( ( x = y \to\varphi) \wedge \exists x ( x = y
                  \wedge\varphi) ))
               \rangle\rangle$
      \item[Eq] $\langle\{\{x,A\},\{x,B\}\},T,\varnothing,
               \langle \vdash ( A = B \leftrightarrow
               \forall x ( x \in A \leftrightarrow x \in B ) )
               \rangle\rangle$
      \item[El] $\langle\{\{x,A\},\{x,B\}\},T,\varnothing,
               \langle \vdash ( A \in B \leftrightarrow \exists x
               ( x = A \wedge x \in B ) )
               \rangle\rangle$
\end{list}
Here we say that an individual variable is a class; $\{x|\varphi\}$ is a
class; and we extend the definition of a wff to include class equality and
membership.  Axiom Ab defines membership of a variable in a class abstraction;
the right-hand side can be read as ``the wff that results from proper
substitution of $y$ for $x$ in $\varphi$.''\footnote{Note that this definition
makes unnecessary the introduction of a separate notation similar to
$\varphi(x|y)$ for proper substitution, although we may choose to do so to be
conventional.  Incidentally, $\varphi(x|y)$ as it stands would be ambiguous in
the formal systems of our examples, since we wouldn't know whether
$\lnot\varphi(x|y)$ meant $\lnot(\varphi(x|y))$ or $(\lnot\varphi)(x|y)$.
Instead, we would have to use an unambiguous variant such as $(\varphi\,
x|y)$.}  Axioms Eq and El extend the meaning of the existing equality and
membership connectives.  This is potentially dangerous and requires careful
justification.  For example, from Eq we can derive the Axiom of Extensionality
with predicate logic alone; thus in principle we should include the Axiom of
Extensionality as a logical hypothesis.  However we do not bother to do this
since we have already presupposed that axiom earlier. The distinct variable
restrictions should be read ``where $x$ does not occur in $A$ or $B$.''  We
typically do this when the right-hand side of a definition involves an
individual variable not in the expression being defined; it is done so that
the right-hand side remains independent of the particular ``dummy'' variable
we use.

We continue to add to $\Gamma$ the following definitions
(i.e. the reducts of the following pre-statements) for empty
set,\index{empty set} class union,\index{union} and unordered
pair.\index{unordered pair}  They should be self-explanatory.  Analogous to our
use of ``$\leftrightarrow$'' to define new wffs in Example~4, we use ``$=$''
to define new abstraction terms, and both may be read informally as ``is
defined as'' in this context.
\begin{list}{}{\itemsep 0.0pt}
      \item[] $\langle\varnothing,T,\varnothing,
               \langle \mbox{class\ }\varnothing\rangle\rangle$
      \item[] $\langle\varnothing,T,\varnothing,
               \langle \vdash \varnothing = \{ x | \lnot x = x \}
               \rangle\rangle$
      \item[] $\langle\varnothing,T,\varnothing,
               \langle \mbox{class\ }(A\cup B)\rangle\rangle$
      \item[] $\langle\{\{x,A\},\{x,B\}\},T,\varnothing,
               \langle \vdash ( A \cup B ) = \{ x | ( x \in A \vee x \in B ) \}
               \rangle\rangle$
      \item[] $\langle\varnothing,T,\varnothing,
               \langle \mbox{class\ }\{A,B\}\rangle\rangle$
      \item[] $\langle\{\{x,A\},\{x,B\}\},T,\varnothing,
               \langle \vdash \{ A , B \} = \{ x | ( x = A \vee x = B ) \}
               \rangle\rangle$
\end{list}

\section{Metamath as a Formal System}\label{theorymm}

This section presupposes a familiarity with the Metamath computer language.

Our theory describes formal systems and their universes.  The Metamath
language provides a way of representing these set-theoretical objects to
a computer.  A Metamath database, being a finite set of {\sc ascii}
characters, can usually describe only a subset of a formal system and
its universe, which are typically infinite.  However the database can
contain as large a finite subset of the formal system and its universe
as we wish.  (Of course a Metamath set theory database can, in
principle, indirectly describe an entire infinite formal system by
formalizing the expository language in this Appendix.)

For purpose of our discussion, we assume the Metamath database
is in the simple form described on p.~\pageref{framelist},
consisting of all constant and variable declarations at the beginning,
followed by a sequence of extended frames each
delimited by \texttt{\$\char`\{} and \texttt{\$\char`\}}.  Any Metamath database can
be converted to this form, as described on p.~\pageref{frameconvert}.

The math symbol tokens of a Metamath source file, which are declared
with \texttt{\$c} and \texttt{\$v} statements, are names we assign to
representatives of $\mbox{\em CN}$ and $\mbox{\em VR}$.  For
definiteness we could assume that the first math symbol declared as a
variable corresponds to $v_0$, the second to $v_1$, etc., although the
exact correspondence we choose is not important.

In the Metamath language, each \texttt{\$d}, \texttt{\$f}, and
 \texttt{\$e} source
statement in an extended frame (Section~\ref{frames})
corresponds respectively to a member of the
collections $D$, $T$, and $H$ in a formal system statement $\langle
D_M,T_M,H,A\rangle$.  The math symbol strings following these Metamath keywords
correspond to a variable pair (in the case of \texttt{\$d}) or an expression (for
the other two keywords). The math symbol string following a \texttt{\$a} source
statement corresponds to expression $A$ in an axiomatic statement of the
formal system; the one following a \texttt{\$p} source statement corresponds to
$A$ in a provable statement that is not axiomatic.  In other words, each
extended frame in a Metamath database corresponds to
a pre-statement of the formal system, and a frame corresponds to
a statement of the formal system.  (Don't confuse the two meanings of
``statement'' here.  A statement of the formal system corresponds to the
several statements in a Metamath database that may constitute a
frame.)

In order for the computer to verify that a formal system statement is
provable, each \texttt{\$p} source statement is accompanied by a proof.
However, the proof does not correspond to anything in the formal system
but is simply a way of communicating to the computer the information
needed for its verification.  The proof tells the computer {\em how to
construct} specific members of closure of the formal system
pre-statement corresponding to the extended frame of the \texttt{\$p}
statement.  The final result of the construction is the member of the
closure that matches the \texttt{\$p} statement.  The abstract formal
system, on the other hand, is concerned only with the {\em existence} of
members of the closure.

As mentioned on p.~\pageref{exampleref}, Examples 1 and 3--6 in the
previous Section parallel the development of logic and set theory in the
Metamath database
\texttt{set.mm}.\index{set theory database (\texttt{set.mm})} You may
find it instructive to compare them.


\chapter{The MIU System}
\label{MIU}
\index{formal system}
\index{MIU-system}

The following is a listing of the file \texttt{miu.mm}.  It is self-explanatory.

%%%%%%%%%%%%%%%%%%%%%%%%%%%%%%%%%%%%%%%%%%%%%%%%%%%%%%%%%%%%

\begin{verbatim}
$( The MIU-system:  A simple formal system $)

$( Note:  This formal system is unusual in that it allows
empty wffs.  To work with a proof, you must type
SET EMPTY_SUBSTITUTION ON before using the PROVE command.
By default, this is OFF in order to reduce the number of
ambiguous unification possibilities that have to be selected
during the construction of a proof.  $)

$(
Hofstadter's MIU-system is a simple example of a formal
system that illustrates some concepts of Metamath.  See
Douglas R. Hofstadter, _Goedel, Escher, Bach:  An Eternal
Golden Braid_ (Vintage Books, New York, 1979), pp. 33ff. for
a description of the MIU-system.

The system has 3 constant symbols, M, I, and U.  The sole
axiom of the system is MI. There are 4 rules:
     Rule I:  If you possess a string whose last letter is I,
     you can add on a U at the end.
     Rule II:  Suppose you have Mx.  Then you may add Mxx to
     your collection.
     Rule III:  If III occurs in one of the strings in your
     collection, you may make a new string with U in place
     of III.
     Rule IV:  If UU occurs inside one of your strings, you
     can drop it.
Unfortunately, Rules III and IV do not have unique results:
strings could have more than one occurrence of III or UU.
This requires that we introduce the concept of an "MIU
well-formed formula" or wff, which allows us to construct
unique symbol sequences to which Rules III and IV can be
applied.
$)

$( First, we declare the constant symbols of the language.
Note that we need two symbols to distinguish the assertion
that a sequence is a wff from the assertion that it is a
theorem; we have arbitrarily chosen "wff" and "|-". $)
      $c M I U |- wff $. $( Declare constants $)

$( Next, we declare some variables. $)
     $v x y $.

$( Throughout our theory, we shall assume that these
variables represent wffs. $)
 wx   $f wff x $.
 wy   $f wff y $.

$( Define MIU-wffs.  We allow the empty sequence to be a
wff. $)

$( The empty sequence is a wff. $)
 we   $a wff $.
$( "M" after any wff is a wff. $)
 wM   $a wff x M $.
$( "I" after any wff is a wff. $)
 wI   $a wff x I $.
$( "U" after any wff is a wff. $)
 wU   $a wff x U $.

$( Assert the axiom. $)
 ax   $a |- M I $.

$( Assert the rules. $)
 ${
   Ia   $e |- x I $.
$( Given any theorem ending with "I", it remains a theorem
if "U" is added after it.  (We distinguish the label I_
from the math symbol I to conform to the 24-Jun-2006
Metamath spec.) $)
   I_    $a |- x I U $.
 $}
 ${
IIa  $e |- M x $.
$( Given any theorem starting with "M", it remains a theorem
if the part after the "M" is added again after it. $)
   II   $a |- M x x $.
 $}
 ${
   IIIa $e |- x I I I y $.
$( Given any theorem with "III" in the middle, it remains a
theorem if the "III" is replaced with "U". $)
   III  $a |- x U y $.
 $}
 ${
   IVa  $e |- x U U y $.
$( Given any theorem with "UU" in the middle, it remains a
theorem if the "UU" is deleted. $)
   IV   $a |- x y $.
  $}

$( Now we prove the theorem MUIIU.  You may be interested in
comparing this proof with that of Hofstadter (pp. 35 - 36).
$)
 theorem1  $p |- M U I I U $=
      we wM wU wI we wI wU we wU wI wU we wM we wI wU we wM
      wI wI wI we wI wI we wI ax II II I_ III II IV $.
\end{verbatim}\index{well-formed formula (wff)}

The \texttt{show proof /lemmon/renumber} command
yields the following display.  It is very similar
to the one in \cite[pp.~35--36]{Hofstadter}.\index{Hofstadter, Douglas R.}

\begin{verbatim}
1 ax             $a |- M I
2 1 II           $a |- M I I
3 2 II           $a |- M I I I I
4 3 I_           $a |- M I I I I U
5 4 III          $a |- M U I U
6 5 II           $a |- M U I U U I U
7 6 IV           $a |- M U I I U
\end{verbatim}

We note that Hofstadter's ``MU-puzzle,'' which asks whether
MU is a theorem of the MIU-system, cannot be answered using
the system above because the MU-puzzle is a question {\em
about} the system.  To prove the answer to the MU-puzzle,
a much more elaborate system is needed, namely one that
models the MIU-system within set theory.  (Incidentally, the
answer to the MU-puzzle is no.)

\chapter{Metamath Language EBNF}%
\label{BNF}%
\index{Metamath Language EBNF}

The following is a formal description of the basic Metamath language syntax
(with compressed proofs and support for unknown proof steps).
It is defined using the
Extended Backus--Naur Form (EBNF)\index{Extended Backus--Naur Form}\index{EBNF}
notation from W3C\index{W3C}
\textit{Extensible Markup Language (XML) 1.0 (Fifth Edition)}
(W3C Recommendation 26 November 2008) at
\url{https://www.w3.org/TR/xml/#sec-notation}.

The \texttt{database}
rule is processed until the end of the file (\texttt{EOF}).
The rules eventually require reading whitespace-separated tokens.
A token has an upper-case definition (see below)
or is a string constant in a non-token (such as \texttt{'\$a'}).
We intend for this to be correct, but if there is a conflict the
rules of section \ref{spec} govern. That section also discusses
non-syntax restrictions not shown here
(e.g., that each new label token
defined in a \texttt{hypothesis-stmt} or \texttt{assert-stmt}
must be unique).

\begin{verbatim}
database ::= outermost-scope-stmt*

outermost-scope-stmt ::=
  include-stmt | constant-stmt | stmt

/* File inclusion command; process file as a database.
   Databases should NOT have a comment in the filename. */
include-stmt ::= '$[' filename '$]'

/* Constant symbols declaration. */
constant-stmt ::= '$c' constant+ '$.'

/* A normal statement can occur in any scope. */
stmt ::= block | variable-stmt | disjoint-stmt |
  hypothesis-stmt | assert-stmt

/* A block. You can have 0 statements in a block. */
block ::= '${' stmt* '$}'

/* Variable symbols declaration. */
variable-stmt ::= '$v' variable+ '$.'

/* Disjoint variables. Simple disjoint statements have
   2 variables, i.e., "variable*" is empty for them. */
disjoint-stmt ::= '$d' variable variable variable* '$.'

hypothesis-stmt ::= floating-stmt | essential-stmt

/* Floating (variable-type) hypothesis. */
floating-stmt ::= LABEL '$f' typecode variable '$.'

/* Essential (logical) hypothesis. */
essential-stmt ::= LABEL '$e' typecode MATH-SYMBOL* '$.'

assert-stmt ::= axiom-stmt | provable-stmt

/* Axiomatic assertion. */
axiom-stmt ::= LABEL '$a' typecode MATH-SYMBOL* '$.'

/* Provable assertion. */
provable-stmt ::= LABEL '$p' typecode MATH-SYMBOL*
  '$=' proof '$.'

/* A proof. Proofs may be interspersed by comments.
   If '?' is in a proof it's an "incomplete" proof. */
proof ::= uncompressed-proof | compressed-proof
uncompressed-proof ::= (LABEL | '?')+
compressed-proof ::= '(' LABEL* ')' COMPRESSED-PROOF-BLOCK+

typecode ::= constant

filename ::= MATH-SYMBOL /* No whitespace or '$' */
constant ::= MATH-SYMBOL
variable ::= MATH-SYMBOL
\end{verbatim}

\needspace{2\baselineskip}
A \texttt{frame} is a sequence of 0 or more
\texttt{disjoint-{\allowbreak}stmt} and
\texttt{hypotheses-{\allowbreak}stmt} statements
(possibly interleaved with other non-\texttt{assert-stmt} statements)
followed by one \texttt{assert-stmt}.

\needspace{3\baselineskip}
Here are the rules for lexical processing (tokenization) beyond
the constant tokens shown above.
By convention these tokenization rules have upper-case names.
Every token is read for the longest possible length.
Whitespace-separated tokens are read sequentially;
note that the separating whitespace and \texttt{\$(} ... \texttt{\$)}
comments are skipped.

If a token definition uses another token definition, the whole thing
is considered a single token.
A pattern that is only part of a full token has a name beginning
with an underscore (``\_'').
An implementation could tokenize many tokens as a
\texttt{PRINTABLE-SEQUENCE}
and then check if it meets the more specific rule shown here.

Comments do not nest, and both \texttt{\$(} and \texttt{\$)}
have to be surrounded
by at least one whitespace character (\texttt{\_WHITECHAR}).
Technically comments end without consuming the trailing
\texttt{\_WHITECHAR}, but the trailing
\texttt{\_WHITECHAR} gets ignored anyway so we ignore that detail here.
Metamath language processors
are not required to support \texttt{\$)} followed
immediately by a bare end-of-file, because the closing
comment symbol is supposed to be followed by a
\texttt{\_WHITECHAR} such as a newline.

\begin{verbatim}
PRINTABLE-SEQUENCE ::= _PRINTABLE-CHARACTER+

MATH-SYMBOL ::= (_PRINTABLE-CHARACTER - '$')+

/* ASCII non-whitespace printable characters */
_PRINTABLE-CHARACTER ::= [#x21-#x7e]

LABEL ::= ( _LETTER-OR-DIGIT | '.' | '-' | '_' )+

_LETTER-OR-DIGIT ::= [A-Za-z0-9]

COMPRESSED-PROOF-BLOCK ::= ([A-Z] | '?')+

/* Define whitespace between tokens. The -> SKIP
   means that when whitespace is seen, it is
   skipped and we simply read again. */
WHITESPACE ::= (_WHITECHAR+ | _COMMENT) -> SKIP

/* Comments. $( ... $) and do not nest. */
_COMMENT ::= '$(' (_WHITECHAR+ (PRINTABLE-SEQUENCE - '$)'))*
  _WHITECHAR+ '$)' _WHITECHAR

/* Whitespace: (' ' | '\t' | '\r' | '\n' | '\f') */
_WHITECHAR ::= [#x20#x09#x0d#x0a#x0c]
\end{verbatim}
% This EBNF was developed as a collaboration between
% David A. Wheeler\index{Wheeler, David A.},
% Mario Carneiro\index{Carneiro, Mario}, and
% Benoit Jubin\index{Jubin, Benoit}, inspired by a request
% (and a lot of initial work) by Benoit Jubin.
%
% \chapter{Disclaimer and Trademarks}
%
% Information in this document is subject to change without notice and does not
% represent a commitment on the part of Norman Megill.
% \vspace{2ex}
%
% \noindent Norman D. Megill makes no warranties, either express or implied,
% regarding the Metamath computer software package.
%
% \vspace{2ex}
%
% \noindent Any trademarks mentioned in this book are the property of
% their respective owners.  The name ``Metamath'' is a trademark of
% Norman Megill.
%
\cleardoublepage
\phantomsection  % fixes the link anchor
\addcontentsline{toc}{chapter}{\bibname}

\bibliography{metamath}
%\input{metamath.bbl}

\raggedright
\cleardoublepage
\phantomsection % fixes the link anchor
\addcontentsline{toc}{chapter}{\indexname}
%\printindex   ??
\input{metamath.ind}

\end{document}



\raggedright
\cleardoublepage
\phantomsection % fixes the link anchor
\addcontentsline{toc}{chapter}{\indexname}
%\printindex   ??
% metamath.tex - Version of 2-Jun-2019
% If you change the date above, also change the "Printed date" below.
% SPDX-License-Identifier: CC0-1.0
%
%                              PUBLIC DOMAIN
%
% This file (specifically, the version of this file with the above date)
% has been released into the Public Domain per the
% Creative Commons CC0 1.0 Universal (CC0 1.0) Public Domain Dedication
% https://creativecommons.org/publicdomain/zero/1.0/
%
% The public domain release applies worldwide.  In case this is not
% legally possible, the right is granted to use the work for any purpose,
% without any conditions, unless such conditions are required by law.
%
% Several short, attributed quotations from copyrighted works
% appear in this file under the ``fair use'' provision of Section 107 of
% the United States Copyright Act (Title 17 of the {\em United States
% Code}).  The public-domain status of this file is not applicable to
% those quotations.
%
% Norman Megill - email: nm(at)alum(dot)mit(dot)edu
%
% David A. Wheeler also donates his improvements to this file to the
% public domain per the CC0.  He works at the Institute for Defense Analyses
% (IDA), but IDA has agreed that this Metamath work is outside its "lane"
% and is not a work by IDA.  This was specifically confirmed by
% Margaret E. Myers (Division Director of the Information Technology
% and Systems Division) on 2019-05-24 and by Ben Lindorf (General Counsel)
% on 2019-05-22.

% This file, 'metamath.tex', is self-contained with everything needed to
% generate the the PDF file 'metamath.pdf' (the _Metamath_ book) on
% standard LaTeX 2e installations.  The auxiliary files are embedded with
% "filecontents" commands.  To generate metamath.pdf file, run these
% commands under Linux or Cygwin in the directory that contains
% 'metamath.tex':
%
%   rm -f realref.sty metamath.bib
%   touch metamath.ind
%   pdflatex metamath
%   pdflatex metamath
%   bibtex metamath
%   makeindex metamath
%   pdflatex metamath
%   pdflatex metamath
%
% The warnings that occur in the initial runs of pdflatex can be ignored.
% For the final run,
%
%   egrep -i 'error|warn' metamath.log
%
% should show exactly these 5 warnings:
%
%   LaTeX Warning: File `realref.sty' already exists on the system.
%   LaTeX Warning: File `metamath.bib' already exists on the system.
%   LaTeX Font Warning: Font shape `OMS/cmtt/m/n' undefined
%   LaTeX Font Warning: Font shape `OMS/cmtt/bx/n' undefined
%   LaTeX Font Warning: Some font shapes were not available, defaults
%       substituted.
%
% Search for "Uncomment" below if you want to suppress hyperlink boxes
% in the PDF output file
%
% TYPOGRAPHICAL NOTES:
% * It is customary to use an en dash (--) to "connect" names of different
%   people (and to denote ranges), and use a hyphen (-) for a
%   single compound name. Examples of connected multiple people are
%   Zermelo--Fraenkel, Schr\"{o}der--Bernstein, Tarski--Grothendieck,
%   Hewlett--Packard, and Backus--Naur.  Examples of a single person with
%   a compound name include Levi-Civita, Mittag-Leffler, and Burali-Forti.
% * Use non-breaking spaces after page abbreviations, e.g.,
%   p.~\pageref{note2002}.
%
% --------------------------- Start of realref.sty -----------------------------
\begin{filecontents}{realref.sty}
% Save the following as realref.sty.
% You can then use it with \usepackage{realref}
%
% This has \pageref jumping to the page on which the ref appears,
% \ref jumping to the point of the anchor, and \sectionref
% jumping to the start of section.
%
% Author:  Anthony Williams
%          Software Engineer
%          Nortel Networks Optical Components Ltd
% Date:    9 Nov 2001 (posted to comp.text.tex)
%
% The following declaration was made by Anthony Williams on
% 24 Jul 2006 (private email to Norman Megill):
%
%   ``I hereby donate the code for realref.sty posted on the
%   comp.text.tex newsgroup on 9th November 2001, accessible from
%   http://groups.google.com/group/comp.text.tex/msg/5a0e1cc13ea7fbb2
%   to the public domain.''
%
\ProvidesPackage{realref}
\RequirePackage[plainpages=false,pdfpagelabels=true]{hyperref}
\def\realref@anchorname{}
\AtBeginDocument{%
% ensure every label is a possible hyperlink target
\let\realref@oldrefstepcounter\refstepcounter%
\DeclareRobustCommand{\refstepcounter}[1]{\realref@oldrefstepcounter{#1}
\edef\realref@anchorname{\string #1.\@currentlabel}%
}%
\let\realref@oldlabel\label%
\DeclareRobustCommand{\label}[1]{\realref@oldlabel{#1}\hypertarget{#1}{}%
\@bsphack\protected@write\@auxout{}{%
    \string\expandafter\gdef\protect\csname
    page@num.#1\string\endcsname{\thepage}%
    \string\expandafter\gdef\protect\csname
    ref@num.#1\string\endcsname{\@currentlabel}%
    \string\expandafter\gdef\protect\csname
    sectionref@name.#1\string\endcsname{\realref@anchorname}%
}\@esphack}%
\DeclareRobustCommand\pageref[1]{{\edef\a{\csname
            page@num.#1\endcsname}\expandafter\hyperlink{page.\a}{\a}}}%
\DeclareRobustCommand\ref[1]{{\edef\a{\csname
            ref@num.#1\endcsname}\hyperlink{#1}{\a}}}%
\DeclareRobustCommand\sectionref[1]{{\edef\a{\csname
            ref@num.#1\endcsname}\edef\b{\csname
            sectionref@name.#1\endcsname}\hyperlink{\b}{\a}}}%
}
\end{filecontents}
% ---------------------------- End of realref.sty ------------------------------

% --------------------------- Start of metamath.bib -----------------------------
\begin{filecontents}{metamath.bib}
@book{Albers, editor = "Donald J. Albers and G. L. Alexanderson",
  title = "Mathematical People",
  publisher = "Contemporary Books, Inc.",
  address = "Chicago",
  note = "[QA28.M37]",
  year = 1985 }
@book{Anderson, author = "Alan Ross Anderson and Nuel D. Belnap",
  title = "Entailment",
  publisher = "Princeton University Press",
  address = "Princeton",
  volume = 1,
  note = "[QA9.A634 1975 v.1]",
  year = 1975}
@book{Barrow, author = "John D. Barrow",
  title = "Theories of Everything:  The Quest for Ultimate Explanation",
  publisher = "Oxford University Press",
  address = "Oxford",
  note = "[Q175.B225]",
  year = 1991 }
@book{Behnke,
  editor = "H. Behnke and F. Backmann and K. Fladt and W. S{\"{u}}ss",
  title = "Fundamentals of Mathematics",
  volume = "I",
  publisher = "The MIT Press",
  address = "Cambridge, Massachusetts",
  note = "[QA37.2.B413]",
  year = 1974 }
@book{Bell, author = "J. L. Bell and M. Machover",
  title = "A Course in Mathematical Logic",
  publisher = "North-Holland",
  address = "Amsterdam",
  note = "[QA9.B3953]",
  year = 1977 }
@inproceedings{Blass, author = "Andrea Blass",
  title = "The Interaction Between Category Theory and Set Theory",
  pages = "5--29",
  booktitle = "Mathematical Applications of Category Theory (Proceedings
     of the Special Session on Mathematical Applications
     Category Theory, 89th Annual Meeting of the American Mathematical
     Society, held in Denver, Colorado January 5--9, 1983)",
  editor = "John Walter Gray",
  year = 1983,
  note = "[QA169.A47 1983]",
  publisher = "American Mathematical Society",
  address = "Providence, Rhode Island"}
@proceedings{Bledsoe, editor = "W. W. Bledsoe and D. W. Loveland",
  title = "Automated Theorem Proving:  After 25 Years (Proceedings
     of the Special Session on Automatic Theorem Proving,
     89th Annual Meeting of the American Mathematical
     Society, held in Denver, Colorado January 5--9, 1983)",
  year = 1983,
  note = "[QA76.9.A96.S64 1983]",
  publisher = "American Mathematical Society",
  address = "Providence, Rhode Island" }
@book{Boolos, author = "George S. Boolos and Richard C. Jeffrey",
  title = "Computability and Log\-ic",
  publisher = "Cambridge University Press",
  edition = "third",
  address = "Cambridge",
  note = "[QA9.59.B66 1989]",
  year = 1989 }
@book{Campbell, author = "John Campbell",
  title = "Programmer's Progress",
  publisher = "White Star Software",
  address = "Box 51623, Palo Alto, CA 94303",
  year = 1991 }
@article{DBLP:journals/corr/Carneiro14,
  author    = {Mario Carneiro},
  title     = {Conversion of {HOL} Light proofs into Metamath},
  journal   = {CoRR},
  volume    = {abs/1412.8091},
  year      = {2014},
  url       = {http://arxiv.org/abs/1412.8091},
  archivePrefix = {arXiv},
  eprint    = {1412.8091},
  timestamp = {Mon, 13 Aug 2018 16:47:05 +0200},
  biburl    = {https://dblp.org/rec/bib/journals/corr/Carneiro14},
  bibsource = {dblp computer science bibliography, https://dblp.org}
}
@article{CarneiroND,
  author    = {Mario Carneiro},
  title     = {Natural Deductions in the Metamath Proof Language},
  url       = {http://us.metamath.org/ocat/natded.pdf},
  year      = 2014
}
@inproceedings{Chou, author = "Shang-Ching Chou",
  title = "Proving Elementary Geometry Theorems Using {W}u's Algorithm",
  pages = "243--286",
  booktitle = "Automated Theorem Proving:  After 25 Years (Proceedings
     of the Special Session on Automatic Theorem Proving,
     89th Annual Meeting of the American Mathematical
     Society, held in Denver, Colorado January 5--9, 1983)",
  editor = "W. W. Bledsoe and D. W. Loveland",
  year = 1983,
  note = "[QA76.9.A96.S64 1983]",
  publisher = "American Mathematical Society",
  address = "Providence, Rhode Island" }
@book{Clemente, author = "Daniel Clemente Laboreo",
  title = "Introduction to natural deduction",
  year = 2014,
  url = "http://www.danielclemente.com/logica/dn.en.pdf" }
@incollection{Courant, author = "Richard Courant and Herbert Robbins",
  title = "Topology",
  pages = "573--590",
  booktitle = "The World of Mathematics, Volume One",
  editor = "James R. Newman",
  publisher = "Simon and Schuster",
  address = "New York",
  note = "[QA3.W67 1988]",
  year = 1956 }
@book{Curry, author = "Haskell B. Curry",
  title = "Foundations of Mathematical Logic",
  publisher = "Dover Publications, Inc.",
  address = "New York",
  note = "[QA9.C976 1977]",
  year = 1977 }
@book{Davis, author = "Philip J. Davis and Reuben Hersh",
  title = "The Mathematical Experience",
  publisher = "Birkh{\"{a}}user Boston",
  address = "Boston",
  note = "[QA8.4.D37 1982]",
  year = 1981 }
@incollection{deMillo,
  author = "Richard de Millo and Richard Lipton and Alan Perlis",
  title = "Social Processes and Proofs of Theorems and Programs",
  pages = "267--285",
  booktitle = "New Directions in the Philosophy of Mathematics",
  editor = "Thomas Tymoczko",
  publisher = "Birkh{\"{a}}user Boston, Inc.",
  address = "Boston",
  note = "[QA8.6.N48 1986]",
  year = 1986 }
@book{Edwards, author = "Robert E. Edwards",
  title = "A Formal Background to Mathematics",
  publisher = "Springer-Verlag",
  address = "New York",
  note = "[QA37.2.E38 v.1a]",
  year = 1979 }
@book{Enderton, author = "Herbert B. Enderton",
  title = "Elements of Set Theory",
  publisher = "Academic Press, Inc.",
  address = "San Diego",
  note = "[QA248.E5]",
  year = 1977 }
@book{Goodstein, author = "R. L. Goodstein",
  title = "Development of Mathematical Logic",
  publisher = "Springer-Verlag New York Inc.",
  address = "New York",
  note = "[QA9.G6554]",
  year = 1971 }
@book{Guillen, author = "Michael Guillen",
  title = "Bridges to Infinity",
  publisher = "Jeremy P. Tarcher, Inc.",
  address = "Los Angeles",
  note = "[QA93.G8]",
  year = 1983 }
@book{Hamilton, author = "Alan G. Hamilton",
  title = "Logic for Mathematicians",
  edition = "revised",
  publisher = "Cambridge University Press",
  address = "Cambridge",
  note = "[QA9.H298]",
  year = 1988 }
@unpublished{Harrison, author = "John Robert Harrison",
  title = "Metatheory and Reflection in Theorem Proving:
    A Survey and Critique",
  note = "Technical Report
    CRC-053.
    SRI Cambridge,
    Millers Yard, Cambridge, UK,
    1995.
    Available on the Web as
{\verb+http:+}\-{\verb+//www.cl.cam.ac.uk/users/jrh/papers/reflect.html+}"}
@TECHREPORT{Harrison-thesis,
        author          = "John Robert Harrison",
        title           = "Theorem Proving with the Real Numbers",
        institution   = "University of Cambridge Computer
                         Lab\-o\-ra\-to\-ry",
        address         = "New Museums Site, Pembroke Street, Cambridge,
                           CB2 3QG, UK",
        year            = 1996,
        number          = 408,
        type            = "Technical Report",
        note            = "Author's PhD thesis,
   available on the Web at
{\verb+http:+}\-{\verb+//www.cl.cam.ac.uk+}\-{\verb+/users+}\-{\verb+/jrh+}%
\-{\verb+/papers+}\-{\verb+/thesis.html+}"}
@book{Herrlich, author = "Horst Herrlich and George E. Strecker",
  title = "Category Theory:  An Introduction",
  publisher = "Allyn and Bacon Inc.",
  address = "Boston",
  note = "[QA169.H567]",
  year = 1973 }
@article{Hindley, author = "J. Roger Hindley and David Meredith",
  title = "Principal Type-Schemes and Condensed Detachment",
  journal = "The Journal of Symbolic Logic",
  volume = 55,
  year = 1990,
  note = "[QA.J87]",
  pages = "90--105" }
@book{Hofstadter, author = "Douglas R. Hofstadter",
  title = "G{\"{o}}del, Escher, Bach",
  publisher = "Basic Books, Inc.",
  address = "New York",
  note = "[QA9.H63 1980]",
  year = 1979 }
@article{Indrzejczak, author= "Andrzej Indrzejczak",
  title = "Natural Deduction, Hybrid Systems and Modal Logic",
  journal = "Trends in Logic",
  volume = 30,
  publisher = "Springer",
  year = 2010 }
@article{Kalish, author = "D. Kalish and R. Montague",
  title = "On {T}arski's Formalization of Predicate Logic with Identity",
  journal = "Archiv f{\"{u}}r Mathematische Logik und Grundlagenfor\-schung",
  volume = 7,
  year = 1965,
  note = "[QA.A673]",
  pages = "81--101" }
@article{Kalman, author = "J. A. Kalman",
  title = "Condensed Detachment as a Rule of Inference",
  journal = "Studia Logica",
  volume = 42,
  number = 4,
  year = 1983,
  note = "[B18.P6.S933]",
  pages = "443-451" }
@book{Kline, author = "Morris Kline",
  title = "Mathematical Thought from Ancient to Modern Times",
  publisher = "Oxford University Press",
  address = "New York",
  note = "[QA21.K516 1990 v.3]",
  year = 1972 }
@book{Klinel, author = "Morris Kline",
  title = "Mathematics, The Loss of Certainty",
  publisher = "Oxford University Press",
  address = "New York",
  note = "[QA21.K525]",
  year = 1980 }
@book{Kramer, author = "Edna E. Kramer",
  title = "The Nature and Growth of Modern Mathematics",
  publisher = "Princeton University Press",
  address = "Princeton, New Jersey",
  note = "[QA93.K89 1981]",
  year = 1981 }
@article{Knill, author = "Oliver Knill",
  title = "Some Fundamental Theorems in Mathematics",
  year = "2018",
  url = "https://arxiv.org/abs/1807.08416" }
@book{Landau, author = "Edmund Landau",
  title = "Foundations of Analysis",
  publisher = "Chelsea Publishing Company",
  address = "New York",
  edition = "second",
  note = "[QA241.L2541 1960]",
  year = 1960 }
@article{Leblanc, author = "Hugues Leblanc",
  title = "On {M}eyer and {L}ambert's Quantificational Calculus {FQ}",
  journal = "The Journal of Symbolic Logic",
  volume = 33,
  year = 1968,
  note = "[QA.J87]",
  pages = "275--280" }
@article{Lejewski, author = "Czeslaw Lejewski",
  title = "On Implicational Definitions",
  journal = "Studia Logica",
  volume = 8,
  year = 1958,
  note = "[B18.P6.S933]",
  pages = "189--208" }
@book{Levy, author = "Azriel Levy",
  title = "Basic Set Theory",
  publisher = "Dover Publications",
  address = "Mineola, NY",
  year = "2002"
}
@book{Margaris, author = "Angelo Margaris",
  title = "First Order Mathematical Logic",
  publisher = "Blaisdell Publishing Company",
  address = "Waltham, Massachusetts",
  note = "[QA9.M327]",
  year = 1967}
@book{Manin, author = "Yu I. Manin",
  title = "A Course in Mathematical Logic",
  publisher = "Springer-Verlag",
  address = "New York",
  note = "[QA9.M29613]",
  year = "1977" }
@article{Mathias, author = "Adrian R. D. Mathias",
  title = "A Term of Length 4,523,659,424,929",
  journal = "Synthese",
  volume = 133,
  year = 2002,
  note = "[Q.S993]",
  pages = "75--86" }
@article{Megill, author = "Norman D. Megill",
  title = "A Finitely Axiomatized Formalization of Predicate Calculus
     with Equality",
  journal = "Notre Dame Journal of Formal Logic",
  volume = 36,
  year = 1995,
  note = "[QA.N914]",
  pages = "435--453" }
@unpublished{Megillc, author = "Norman D. Megill",
  title = "A Shorter Equivalent of the Axiom of Choice",
  month = "June",
  note = "Unpublished",
  year = 1991 }
@article{MegillBunder, author = "Norman D. Megill and Martin W.
    Bunder",
  title = "Weaker {D}-Complete Logics",
  journal = "Journal of the IGPL",
  volume = 4,
  year = 1996,
  pages = "215--225",
  note = "Available on the Web at
{\verb+http:+}\-{\verb+//www.mpi-sb.mpg.de+}\-{\verb+/igpl+}%
\-{\verb+/Journal+}\-{\verb+/V4-2+}\-{\verb+/#Megill+}"}
}
@book{Mendelson, author = "Elliott Mendelson",
  title = "Introduction to Mathematical Logic",
  edition = "second",
  publisher = "D. Van Nostrand Company, Inc.",
  address = "New York",
  note = "[QA9.M537 1979]",
  year = 1979 }
@article{Meredith, author = "David Meredith",
  title = "In Memoriam {C}arew {A}rthur {M}eredith (1904-1976)",
  journal = "Notre Dame Journal of Formal Logic",
  volume = 18,
  year = 1977,
  note = "[QA.N914]",
  pages = "513--516" }
@article{CAMeredith, author = "C. A. Meredith",
  title = "Single Axioms for the Systems ({C},{N}), ({C},{O}) and ({A},{N})
      of the Two-Valued Propositional Calculus",
  journal = "The Journal of Computing Systems",
  volume = 3,
  year = 1953,
  pages = "155--164" }
@article{Monk, author = "J. Donald Monk",
  title = "Provability With Finitely Many Variables",
  journal = "The Journal of Symbolic Logic",
  volume = 27,
  year = 1971,
  note = "[QA.J87]",
  pages = "353--358" }
@article{Monks, author = "J. Donald Monk",
  title = "Substitutionless Predicate Logic With Identity",
  journal = "Archiv f{\"{u}}r Mathematische Logik und Grundlagenfor\-schung",
  volume = 7,
  year = 1965,
  pages = "103--121" }
  %% Took out this from above to prevent LaTeX underfull warning:
  % note = "[QA.A673]",
@book{Moore, author = "A. W. Moore",
  title = "The Infinite",
  publisher = "Routledge",
  address = "New York",
  note = "[BD411.M59]",
  year = 1989}
@book{Munkres, author = "James R. Munkres",
  title = "Topology: A First Course",
  publisher = "Prentice-Hall, Inc.",
  address = "Englewood Cliffs, New Jersey",
  note = "[QA611.M82]",
  year = 1975}
@article{Nemesszeghy, author = "E. Z. Nemesszeghy and E. A. Nemesszeghy",
  title = "On Strongly Creative Definitions:  A Reply to {V}. {F}. {R}ickey",
  journal = "Logique et Analyse (N.\ S.)",
  year = 1977,
  volume = 20,
  note = "[BC.L832]",
  pages = "111--115" }
@unpublished{Nemeti, author = "N{\'{e}}meti, I.",
  title = "Algebraizations of Quantifier Logics, an Overview",
  note = "Version 11.4, preprint, Mathematical Institute, Budapest,
    1994.  A shortened version without proofs appeared in
    ``Algebraizations of quantifier logics, an introductory overview,''
   {\em Studia Logica}, 50:485--569, 1991 [B18.P6.S933]"}
@article{Pavicic, author = "M. Pavi{\v{c}}i{\'{c}}",
  title = "A New Axiomatization of Unified Quantum Logic",
  journal = "International Journal of Theoretical Physics",
  year = 1992,
  volume = 31,
  note = "[QC.I626]",
  pages = "1753 --1766" }
@book{Penrose, author = "Roger Penrose",
  title = "The Emperor's New Mind",
  publisher = "Oxford University Press",
  address = "New York",
  note = "[Q335.P415]",
  year = 1989 }
@book{PetersonI, author = "Ivars Peterson",
  title = "The Mathematical Tourist",
  publisher = "W. H. Freeman and Company",
  address = "New York",
  note = "[QA93.P475]",
  year = 1988 }
@article{Peterson, author = "Jeremy George Peterson",
  title = "An automatic theorem prover for substitution and detachment systems",
  journal = "Notre Dame Journal of Formal Logic",
  volume = 19,
  year = 1978,
  note = "[QA.N914]",
  pages = "119--122" }
@book{Quine, author = "Willard Van Orman Quine",
  title = "Set Theory and Its Logic",
  edition = "revised",
  publisher = "The Belknap Press of Harvard University Press",
  address = "Cambridge, Massachusetts",
  note = "[QA248.Q7 1969]",
  year = 1969 }
@article{Robinson, author = "J. A. Robinson",
  title = "A Machine-Oriented Logic Based on the Resolution Principle",
  journal = "Journal of the Association for Computing Machinery",
  year = 1965,
  volume = 12,
  pages = "23--41" }
@article{RobinsonT, author = "T. Thacher Robinson",
  title = "Independence of Two Nice Sets of Axioms for the Propositional
    Calculus",
  journal = "The Journal of Symbolic Logic",
  volume = 33,
  year = 1968,
  note = "[QA.J87]",
  pages = "265--270" }
@book{Rucker, author = "Rudy Rucker",
  title = "Infinity and the Mind:  The Science and Philosophy of the
    Infinite",
  publisher = "Bantam Books, Inc.",
  address = "New York",
  note = "[QA9.R79 1982]",
  year = 1982 }
@book{Russell, author = "Bertrand Russell",
  title = "Mysticism and Logic, and Other Essays",
  publisher = "Barnes \& Noble Books",
  address = "Totowa, New Jersey",
  note = "[B1649.R963.M9 1981]",
  year = 1981 }
@article{Russell2, author = "Bertrand Russell",
  title = "Recent Work on the Principles of Mathematics",
  journal = "International Monthly",
  volume = 4,
  year = 1901,
  pages = "84"}
@article{Schmidt, author = "Eric Schmidt",
  title = "Reductions in Norman Megill's axiom system for complex numbers",
  url = "http://us.metamath.org/downloads/schmidt-cnaxioms.pdf",
  year = "2012" }
@book{Shoenfield, author = "Joseph R. Shoenfield",
  title = "Mathematical Logic",
  publisher = "Addison-Wesley Publishing Company, Inc.",
  address = "Reading, Massachusetts",
  year = 1967,
  note = "[QA9.S52]" }
@book{Smullyan, author = "Raymond M. Smullyan",
  title = "Theory of Formal Systems",
  publisher = "Princeton University Press",
  address = "Princeton, New Jersey",
  year = 1961,
  note = "[QA248.5.S55]" }
@book{Solow, author = "Daniel Solow",
  title = "How to Read and Do Proofs:  An Introduction to Mathematical
    Thought Process",
  publisher = "John Wiley \& Sons",
  address = "New York",
  year = 1982,
  note = "[QA9.S577]" }
@book{Stark, author = "Harold M. Stark",
  title = "An Introduction to Number Theory",
  publisher = "Markham Publishing Company",
  address = "Chicago",
  note = "[QA241.S72 1978]",
  year = 1970 }
@article{Swart, author = "E. R. Swart",
  title = "The Philosophical Implications of the Four-Color Problem",
  journal = "American Mathematical Monthly",
  year = 1980,
  volume = 87,
  month = "November",
  note = "[QA.A5125]",
  pages = "697--707" }
@book{Szpiro, author = "George G. Szpiro",
  title = "Poincar{\'{e}}'s Prize: The Hundred-Year Quest to Solve One
    of Math's Greatest Puzzles",
  publisher = "Penguin Books Ltd",
  address = "London",
  note = "[QA43.S985 2007]",
  year = 2007}
@book{Takeuti, author = "Gaisi Takeuti and Wilson M. Zaring",
  title = "Introduction to Axiomatic Set Theory",
  edition = "second",
  publisher = "Springer-Verlag New York Inc.",
  address = "New York",
  note = "[QA248.T136 1982]",
  year = 1982}
@inproceedings{Tarski, author = "Alfred Tarski",
  title = "What is Elementary Geometry",
  pages = "16--29",
  booktitle = "The Axiomatic Method, with Special Reference to Geometry and
     Physics (Proceedings of an International Symposium held at the University
     of California, Berkeley, December 26, 1957 --- January 4, 1958)",
  editor = "Leon Henkin and Patrick Suppes and Alfred Tarski",
  year = 1959,
  publisher = "North-Holland Publishing Company",
  address = "Amsterdam"}
@article{Tarski1965, author = "Alfred Tarski",
  title = "A Simplified Formalization of Predicate Logic with Identity",
  journal = "Archiv f{\"{u}}r Mathematische Logik und Grundlagenforschung",
  volume = 7,
  year = 1965,
  note = "[QA.A673]",
  pages = "61--79" }
@book{Tymoczko,
  title = "New Directions in the Philosophy of Mathematics",
  editor = "Thomas Tymoczko",
  publisher = "Birkh{\"{a}}user Boston, Inc.",
  address = "Boston",
  note = "[QA8.6.N48 1986]",
  year = 1986 }
@incollection{Wang,
  author = "Hao Wang",
  title = "Theory and Practice in Mathematics",
  pages = "129--152",
  booktitle = "New Directions in the Philosophy of Mathematics",
  editor = "Thomas Tymoczko",
  publisher = "Birkh{\"{a}}user Boston, Inc.",
  address = "Boston",
  note = "[QA8.6.N48 1986]",
  year = 1986 }
@manual{Webster,
  title = "Webster's New Collegiate Dictionary",
  organization = "G. \& C. Merriam Co.",
  address = "Springfield, Massachusetts",
  note = "[PE1628.W4M4 1977]",
  year = 1977 }
@manual{Whitehead, author = "Alfred North Whitehead",
  title = "An Introduction to Mathematics",
  year = 1911 }
@book{PM, author = "Alfred North Whitehead and Bertrand Russell",
  title = "Principia Mathematica",
  edition = "second",
  publisher = "Cambridge University Press",
  address = "Cambridge",
  year = "1927",
  note = "(3 vols.) [QA9.W592 1927]" }
@article{DBLP:journals/corr/Whalen16,
  author    = {Daniel Whalen},
  title     = {Holophrasm: a neural Automated Theorem Prover for higher-order logic},
  journal   = {CoRR},
  volume    = {abs/1608.02644},
  year      = {2016},
  url       = {http://arxiv.org/abs/1608.02644},
  archivePrefix = {arXiv},
  eprint    = {1608.02644},
  timestamp = {Mon, 13 Aug 2018 16:46:19 +0200},
  biburl    = {https://dblp.org/rec/bib/journals/corr/Whalen16},
  bibsource = {dblp computer science bibliography, https://dblp.org} }
@article{Wiedijk-revisited,
  author = {Freek Wiedijk},
  title = {The QED Manifesto Revisited},
  year = {2007},
  url = {http://mizar.org/trybulec65/8.pdf} }
@book{Wolfram,
  author = "Stephen Wolfram",
  title = "Mathematica:  A System for Doing Mathematics by Computer",
  edition = "second",
  publisher = "Addison-Wesley Publishing Co.",
  address = "Redwood City, California",
  note = "[QA76.95.W65 1991]",
  year = 1991 }
@book{Wos, author = "Larry Wos and Ross Overbeek and Ewing Lusk and Jim Boyle",
  title = "Automated Reasoning:  Introduction and Applications",
  edition = "second",
  publisher = "McGraw-Hill, Inc.",
  address = "New York",
  note = "[QA76.9.A96.A93 1992]",
  year = 1992 }

%
%
%[1] Church, Alonzo, Introduction to Mathematical Logic,
% Volume 1, Princeton University Press, Princeton, N. J., 1956.
%
%[2] Cohen, Paul J., Set Theory and the Continuum Hypothesis,
% W. A. Benjamin, Inc., Reading, Mass., 1966.
%
%[3] Hamilton, Alan G., Logic for Mathematicians, Cambridge
% University Press,
% Cambridge, 1988.

%[6] Kleene, Stephen Cole, Introduction to Metamathematics, D.  Van
% Nostrand Company, Inc., Princeton (1952).

%[13] Tarski, Alfred, "A simplified formalization of predicate
% logic with identity," Archiv fur Mathematische Logik und
% Grundlagenforschung, vol. 7 (1965), pp. 61-79.

%[14] Tarski, Alfred and Steven Givant, A Formalization of Set
% Theory Without Variables, American Mathematical Society Colloquium
% Publications, vol. 41, American Mathematical Society,
% Providence, R. I., 1987.

%[15] Zeman, J. J., Modal Logic, Oxford University Press, Oxford, 1973.
\end{filecontents}
% --------------------------- End of metamath.bib -----------------------------


%Book: Metamath
%Author:  Norman Megill Email:  nm at alum.mit.edu
%Author:  David A. Wheeler Email:  dwheeler at dwheeler.com

% A book template example
% http://www.stsci.edu/ftp/software/tex/bookstuff/book.template

\documentclass[leqno]{book} % LaTeX 2e. 10pt. Use [leqno,12pt] for 12pt
% hyperref 2002/05/27 v6.72r  (couldn't get pagebackref to work)
\usepackage[plainpages=false,pdfpagelabels=true]{hyperref}

\usepackage{needspace}     % Enable control over page breaks
\usepackage{breqn}         % automatic equation breaking
\usepackage{microtype}     % microtypography, reduces hyphenation

% Packages for flexible tables.  We need to be able to
% wrap text within a cell (with automatically-determined widths) AND
% split a table automatically across multiple pages.
% * "tabularx" wraps text in cells but only 1 page
% * "longtable" goes across pages but by itself is incompatible with tabularx
% * "ltxtable" combines longtable and tabularx, but table contents
%    must be in a separate file.
% * "ltablex" combines tabularx and longtable - must install specially
% * "booktabs" is recommended as a way to improve the look of tables,
%   but doesn't add these capabilities.
% * "tabu" much more capable and seems to be recommended. So use that.

\usepackage{makecell}      % Enable forced line splits within a table cell
% v4.13 needed for tabu: https://tex.stackexchange.com/questions/600724/dimension-too-large-after-recent-longtable-update
\usepackage{longtable}[=v4.13] % Enable multi-page tables  
\usepackage{tabu}          % Multi-page tables with wrapped text in a cell

% You can find more Tex packages using commands like:
% tlmgr search --file tabu.sty
% find /usr/share/texmf-dist/ -name '*tab*'
%
%%%%%%%%%%%%%%%%%%%%%%%%%%%%%%%%%%%%%%%%%%%%%%%%%%%%%%%%%%%%%%%%%%%%%%%%%%%%
% Uncomment the next 3 lines to suppress boxes and colors on the hyperlinks
%%%%%%%%%%%%%%%%%%%%%%%%%%%%%%%%%%%%%%%%%%%%%%%%%%%%%%%%%%%%%%%%%%%%%%%%%%%%
%\hypersetup{
%colorlinks,citecolor=black,filecolor=black,linkcolor=black,urlcolor=black
%}
%
\usepackage{realref}

% Restarting page numbers: try?
%   \printglossary
%   \cleardoublepage
%   \pagenumbering{arabic}
%   \setcounter{page}{1}    ???needed
%   \include{chap1}

% not used:
% \def\R2Lurl#1#2{\mbox{\href{#1}\texttt{#2}}}

\usepackage{amssymb}

% Version 1 of book: margins: t=.4, b=.2, ll=.4, rr=.55
% \usepackage{anysize}
% % \papersize{<height>}{<width>}
% % \marginsize{<left>}{<right>}{<top>}{<bottom>}
% \papersize{9in}{6in}
% % l/r 0.6124-0.6170 works t/b 0.2418-0.3411 = 192pp. 0.2926-03118=exact
% \marginsize{0.7147in}{0.5147in}{0.4012in}{0.2012in}

\usepackage{anysize}
% \papersize{<height>}{<width>}
% \marginsize{<left>}{<right>}{<top>}{<bottom>}
\papersize{9in}{6in}
% l/r 0.85in&0.6431-0.6539 works t/b ?-?
%\marginsize{0.85in}{0.6485in}{0.55in}{0.35in}
\marginsize{0.8in}{0.65in}{0.5in}{0.3in}

% \usepackage[papersize={3.6in,4.8in},hmargin=0.1in,vmargin={0.1in,0.1in}]{geometry}  % page geometry
\usepackage{special-settings}

\raggedbottom
\makeindex

\begin{document}
% Discourage page widows and orphans:
\clubpenalty=300
\widowpenalty=300

%%%%%%% load in AMS fonts %%%%%%% % LaTeX 2.09 - obsolete in LaTeX 2e
%\input{amssym.def}
%\input{amssym.tex}
%\input{c:/texmf/tex/plain/amsfonts/amssym.def}
%\input{c:/texmf/tex/plain/amsfonts/amssym.tex}

\bibliographystyle{plain}
\pagenumbering{roman}
\pagestyle{headings}

\thispagestyle{empty}

\hfill
\vfill

\begin{center}
{\LARGE\bf Metamath} \\
\vspace{1ex}
{\large A Computer Language for Mathematical Proofs} \\
\vspace{7ex}
{\large Norman Megill} \\
\vspace{7ex}
with extensive revisions by \\
\vspace{1ex}
{\large David A. Wheeler} \\
\vspace{7ex}
% Printed date. If changing the date below, also fix the date at the beginning.
2019-06-02
\end{center}

\vfill
\hfill

\newpage
\thispagestyle{empty}

\hfill
\vfill

\begin{center}
$\sim$\ {\sc Public Domain}\ $\sim$

\vspace{2ex}
This book (including its later revisions)
has been released into the Public Domain by Norman Megill per the
Creative Commons CC0 1.0 Universal (CC0 1.0) Public Domain Dedication.
David A. Wheeler has done the same.
This public domain release applies worldwide.  In case this is not
legally possible, the right is granted to use the work for any purpose,
without any conditions, unless such conditions are required by law.
See \url{https://creativecommons.org/publicdomain/zero/1.0/}.

\vspace{3ex}
Several short, attributed quotations from copyrighted works
appear in this book under the ``fair use'' provision of Section 107 of
the United States Copyright Act (Title 17 of the {\em United States
Code}).  The public-domain status of this book is not applicable to
those quotations.

\vspace{3ex}
Any trademarks used in this book are the property of their owners.

% QA76.9.L63.M??

% \vspace{1ex}
%
% \vspace{1ex}
% {\small Permission is granted to make and distribute verbatim copies of this
% book
% provided the copyright notice and this
% permission notice are preserved on all copies.}
%
% \vspace{1ex}
% {\small Permission is granted to copy and distribute modified versions of this
% book under the conditions for verbatim copying, provided that the
% entire
% resulting derived work is distributed under the terms of a permission
% notice
% identical to this one.}
%
% \vspace{1ex}
% {\small Permission is granted to copy and distribute translations of this
% book into another language, under the above conditions for modified
% versions,
% except that this permission notice may be stated in a translation
% approved by the
% author.}
%
% \vspace{1ex}
% %{\small   For a copy of the \LaTeX\ source files for this book, contact
% %the author.} \\
% \ \\
% \ \\

\vspace{7ex}
% ISBN: 1-4116-3724-0 \\
% ISBN: 978-1-4116-3724-5 \\
ISBN: 978-0-359-70223-7 \\
{\ } \\
Lulu Press \\
Morrisville, North Carolina\\
USA


\hfill
\vfill

Norman Megill\\ 93 Bridge St., Lexington, MA 02421 \\
E-mail address: \texttt{nm{\char`\@}alum.mit.edu} \\
\vspace{7ex}
David A. Wheeler \\
E-mail address: \texttt{dwheeler{\char`\@}dwheeler.com} \\
% See notes added at end of Preface for revision history. \\
% For current information on the Metamath software see \\
\vspace{7ex}
\url{http://metamath.org}
\end{center}

\hfill
\vfill

{\parindent0pt%
\footnotesize{%
Cover: Aleph null ($\aleph_0$) is the symbol for the
first infinite cardinal number, discovered by Georg Cantor in 1873.
We use a red aleph null (with dark outline and gold glow) as the Metamath logo.
Credit: Norman Megill (1994) and Giovanni Mascellani (2019),
public domain.%
\index{aleph null}%
\index{Metamath!logo}\index{Cantor, Georg}\index{Mascellani, Giovanni}}}

% \newpage
% \thispagestyle{empty}
%
% \hfill
% \vfill
%
% \begin{center}
% {\it To my son Robin Dwight Megill}
% \end{center}
%
% \vfill
% \hfill
%
% \newpage

\tableofcontents
%\listoftables

\chapter*{Preface}
\markboth{PREFACE}{PREFACE}
\addcontentsline{toc}{section}{Preface}


% (For current information, see the notes added at the
% end of this preface on p.~\pageref{note2002}.)

\subsubsection{Overview}

Metamath\index{Metamath} is a computer language and an associated computer
program for archiving, verifying, and studying mathematical proofs at a very
detailed level.  The Metamath language incorporates no mathematics per se but
treats all mathematical statements as mere sequences of symbols.  You provide
Metamath with certain special sequences (axioms) that tell it what rules
of inference are allowed.  Metamath is not limited to any specific field of
mathematics.  The Metamath language is simple and robust, with an
almost total absence of hard-wired syntax, and
we\footnote{Unless otherwise noted, the words
``I,'' ``me,'' and ``my'' refer to Norman Megill\index{Megill, Norman}, while
``we,'' ``us,'' and ``our'' refer to Norman Megill and
David A. Wheeler\index{Wheeler, David A.}.}
believe that it
provides about the simplest possible framework that allows essentially all of
mathematics to be expressed with absolute rigor.

% index test
%\newcommand{\nn}[1]{#1n}
%\index{aaa@bbb}
%\index{abc!def}
%\index{abd|see{qqq}}
%\index{abe|nn}
%\index{abf|emph}
%\index{abg|(}
%\index{abg|)}

Using the Metamath language, you can build formal or mathematical
systems\index{formal system}\footnote{A formal or mathematical system consists
of a collection of symbols (such as $2$, $4$, $+$ and $=$), syntax rules that
describe how symbols may be combined to form a legal expression (called a
well-formed formula or {\em wff}, pronounced ``whiff''), some starting wffs
called axioms, and inference rules that describe how theorems may be derived
(proved) from the axioms.  A theorem is a mathematical fact such as $2+2=4$.
Strictly speaking, even an obvious fact such as this must be proved from
axioms to be formally acceptable to a mathematician.}\index{theorem}
\index{axiom}\index{rule}\index{well-formed formula (wff)} that involve
inferences from axioms.  Although a database is provided
that includes a recommended set of axioms for standard mathematics, if you
wish you can supply your own symbols, syntax, axioms, rules, and definitions.

The name ``Metamath'' was chosen to suggest that the language provides a
means for {\em describing} mathematics rather than {\em being} the
mathematics itself.  Actually in some sense any mathematical language is
metamathematical.  Symbols written on paper, or stored in a computer,
are not mathematics itself but rather a way of expressing mathematics.
For example ``7'' and ``VII'' are symbols for denoting the number seven
in Arabic and Roman numerals; neither {\em is} the number seven.

If you are able to understand and write computer programs, you should be able
to follow abstract mathematics with the aid of Metamath.  Used in conjunction
with standard textbooks, Metamath can guide you step-by-step towards an
understanding of abstract mathematics from a very rigorous viewpoint, even if
you have no formal abstract mathematics background.  By using a single,
consistent notation to express proofs, once you grasp its basic concepts
Metamath provides you with the ability to immediately follow and dissect
proofs even in totally unfamiliar areas.

Of course, just being able follow a proof will not necessarily give you an
intuitive familiarity with mathematics.  Memorizing the rules of chess does not
give you the ability to appreciate the game of a master, and knowing how the
notes on a musical score map to piano keys does not give you the ability to
hear in your head how it would sound.  But each of these can be a first step.

Metamath allows you to explore proofs in the sense that you can see the
theorem referenced at any step expanded in as much detail as you want, right
down to the underlying axioms of logic and set theory (in the case of the set
theory database provided).  While Metamath will not replace the higher-level
understanding that can only be acquired through exercises and hard work, being
able to see how gaps in a proof are filled in can give you increased
confidence that can speed up the learning process and save you time when you
get stuck.

The Metamath language breaks down a mathematical proof into its tiniest
possible parts.  These can be pieced together, like interlocking
pieces in a puzzle, only in a way that produces correct and absolutely rigorous
mathematics.

The nature of Metamath\index{Metamath} enforces very precise mathematical
thinking, similar to that involved in writing a computer program.  A crucial
difference, though, is that once a proof is verified (by the Metamath program)
to be correct, it is definitely correct; it can never have a hidden
``bug.''\index{computer program bugs}  After getting used to the kind of rigor
and accuracy provided by Metamath, you might even be tempted to
adopt the attitude that a proof should never be considered correct until it
has been verified by a computer, just as you would not completely trust a
manual calculation until you have verified it on a
calculator.

My goal
for Metamath was a system for describing and verifying
mathematics that is completely universal yet conceptually as simple as
possible.  In approaching mathematics from an axiomatic, formal viewpoint, I
wanted Metamath to be able to handle almost any mathematical system, not
necessarily with ease, but at least in principle and hopefully in practice. I
wanted it to verify proofs with absolute rigor, and for this reason Metamath
is what might be thought of as a ``compile-only'' language rather than an
algorithmic or Turing-machine language (Pascal, C, Prolog, Mathematica,
etc.).  In other words, a database written in the Metamath
language doesn't ``do'' anything; it merely exhibits mathematical knowledge
and permits this knowledge to be verified as being correct.  A program in an
algorithmic language can potentially have hidden bugs\index{computer program
bugs} as well as possibly being hard to understand.  But each token in a
Metamath database must be consistent with the database's earlier
contents according to simple, fixed rules.
If a database is verified
to be correct,\footnote{This includes
verification that a sequential list of proof steps results in the specified
theorem.} then the mathematical content is correct if the
verifier is correct and the axioms are correct.
The verification program could be incorrect, but the verification algorithm
is relatively simple (making it unlikely to be implemented incorrectly
by the Metamath program),
and there are over a dozen Metamath database verifiers
written by different people in different programming languages
(so these different verifiers can act as multiple reviewers of a database).
The most-used Metamath database, the Metamath Proof Explorer
(aka \texttt{set.mm}\index{set theory database (\texttt{set.mm})}%
\index{Metamath Proof Explorer}),
is currently verified by four different Metamath verifiers written by
four different people in four different languages, including the
original Metamath program described in this book.
The only ``bugs'' that can exist are in the statement of the axioms,
for example if the axioms are inconsistent (a famous problem shown to be
unsolvable by G\"{o}del's incompleteness theorem\index{G\"{o}del's
incompleteness theorem}).
However, real mathematical systems have very few axioms, and these can
be carefully studied.
All of this provides extraordinarily high confidence that the verified database
is in fact correct.

The Metamath program
doesn't prove theorems automatically but is designed to verify proofs
that you supply to it.
The underlying Metamath language is completely general and has no built-in,
preconceived notions about your formal system\index{formal system}, its logic
or its syntax.
For constructing proofs, the Metamath program has a Proof Assistant\index{Proof
Assistant} which helps you fill in some of a proof step's details, shows you
what choices you have at any step, and verifies the proof as you build it; but
you are still expected to provide the proof.

There are many other programs that can process or generate information
in the Metamath language, and more continue to be written.
This is in part because the Metamath language itself is very simple
and intentionally easy to automatically process.
Some programs, such as \texttt{mmj2}\index{mmj2}, include a proof assistant
that can automate some steps beyond what the Metamath program can do.
Mario Carneiro has developed an algorithm for converting proofs from
the OpenTheory interchange format, which can be translated to and from
any of the HOL family of proof languages (HOL4, HOL Light, ProofPower,
and Isabelle), into the
Metamath language \cite{DBLP:journals/corr/Carneiro14}\index{Carneiro, Mario}.
Daniel Whalen has developed Holophrasm, which can automatically
prove many Metamath proofs using
machine learning\index{machine learning}\index{artificial intelligence}
approaches
(including multiple neural networks\index{neural networks})\cite{DBLP:journals/corr/Whalen16}\index{Whalen, Daniel}.
However,
a discussion of these other programs is beyond the scope of this book.

Like most computer languages, the Metamath\index{Metamath} language uses the
standard ({\sc ascii}) characters on a computer keyboard, so it cannot
directly represent many of the special symbols that mathematicians use.  A
useful feature of the Metamath program is its ability to convert its notation
into the \LaTeX\ typesetting language.\index{latex@{\LaTeX}}  This feature
lets you convert the {\sc ascii} tokens you've defined into standard
mathematical symbols, so you end up with symbols and formulas you are familiar
with instead of somewhat cryptic {\sc ascii} representations of them.
The Metamath program can also generate HTML\index{HTML}, making it easy
to view results on the web and to see related information by using
hypertext links.

Metamath is probably conceptually different from anything you've seen
before and some aspects may take some getting used to.  This book will
help you decide whether Metamath suits your specific needs.

\subsubsection{Setting Your Expectations}
It is important for you to understand what Metamath\index{Metamath} is and is
not.  As mentioned, the Metamath program
is {\em not} an automated theorem prover but
rather a proof verifier.  Developing a database can be tedious, hard work,
especially if you want to make the proofs as short as possible, but it becomes
easier as you build up a collection of useful theorems.  The purpose of
Metamath is simply to document existing mathematics in an absolutely rigorous,
computer-verifiable way, not to aid directly in the creation of new
mathematics.  It also is not a magic solution for learning abstract
mathematics, although it may be helpful to be able to actually see the implied
rigor behind what you are learning from textbooks, as well as providing hints
to work out proofs that you are stumped on.

As of this writing, a sizable set theory database has been developed to
provide a foundation for many fields of mathematics, but much more work would
be required to develop useful databases for specific fields.

Metamath\index{Metamath} ``knows no math;'' it just provides a framework in
which to express mathematics.  Its language is very small.  You can define two
kinds of symbols, constants\index{constant} and variables\index{variable}.
The only thing Metamath knows how to do is to substitute strings of symbols
for the variables\index{substitution!variable}\index{variable substitution} in
an expression based on instructions you provide it in a proof, subject to
certain constraints you specify for the variables.  Even the decimal
representation of a number is merely a string of certain constants (digits)
which together, in a specific context, correspond to whatever mathematical
object you choose to define for it; unlike other computer languages, there is
no actual number stored inside the computer.  In a proof, you in effect
instruct Metamath what symbol substitutions to make in previous axioms or
theorems and join a sequence of them together to result in the desired
theorem.  This kind of symbol manipulation captures the essence of mathematics
at a preaxiomatic level.

\subsubsection{Metamath and Mathematical Literature}

In advanced mathematical literature, proofs are usually presented in the form
of short outlines that often only an expert can follow.  This is partly out of
a desire for brevity, but it would also be unwise (even if it were practical)
to present proofs in complete formal detail, since the overall picture would
be lost.\index{formal proof}

A solution I envision\label{envision} that would allow mathematics to remain
acceptable to the expert, yet increase its accessibility to non-specialists,
consists of a combination of the traditional short, informal proof in print
accompanied by a complete formal proof stored in a computer database.  In an
analogy with a computer program, the informal proof is like a set of comments
that describe the overall reasoning and content of the proof, whereas the
computer database is like the actual program and provides a means for anyone,
even a non-expert, to follow the proof in as much detail as desired, exploring
it back through layers of theorems (like subroutines that call other
subroutines) all the way back to the axioms of the theory.  In addition, the
computer database would have the advantage of providing absolute assurance
that the proof is correct, since each step can be verified automatically.

There are several other approaches besides Metamath to a project such
as this.  Section~\ref{proofverifiers} discusses some of these.

To us, a noble goal would be a database with hundreds of thousands of
theorems and their computer-verifiable proofs, encompassing a significant
fraction of known mathematics and available for instant access.
These would be fully verified by multiple independently-implemented verifiers,
to provide extremely high confidence that the proofs are completely correct.
The database would allow people to investigate whatever details they were
interested in, so that they could confirm whatever portions they wished.
Whether or not Metamath is an appropriate choice remains to be seen, but in
principle we believe it is sufficient.

\subsubsection{Formalism}

Over the past fifty years, a group of French mathematicians working
collectively under the pseudonym of Bourbaki\index{Bourbaki, Nicolas} have
co-authored a series of monographs that attempt to rigorously and
consistently formalize large bodies of mathematics from foundations.  On the
one hand, certainly such an effort has its merits; on the other hand, the
Bourbaki project has been criticized for its ``scholasticism'' and
``hyperaxiomatics'' that hide the intuitive steps that lead to the results
\cite[p.~191]{Barrow}\index{Barrow, John D.}.

Metamath unabashedly carries this philosophy to its extreme and no doubt is
subject to the same kind of criticism.  Nonetheless I think that in
conjunction with conventional approaches to mathematics Metamath can serve a
useful purpose.  The Bourbaki approach is essentially pedagogic, requiring the
reader to become intimately familiar with each detail in a very large
hierarchy before he or she can proceed to the next step.  The difference with
Metamath is that the ``reader'' (user) knows that all details are contained in
its computer database, available as needed; it does not demand that the user
know everything but conveniently makes available those portions that are of
interest.  As the body of all mathematical knowledge grows larger and larger,
no one individual can have a thorough grasp of its entirety.  Metamath
can finalize and put to rest any questions about the validity of any part of it
and can make any part of it accessible, in principle, to a non-specialist.

\subsubsection{A Personal Note}
Why did I develop Metamath\index{Metamath}?  I enjoy abstract mathematics, but
I sometimes get lost in a barrage of definitions and start to lose confidence
that my proofs are correct.  Or I reach a point where I lose sight of how
anything I'm doing relates to the axioms that a theory is based on and am
sometimes suspicious that there may be some overlooked implicit axiom
accidentally introduced along the way (as happened historically with Euclidean
geometry\index{Euclidean geometry}, whose omission of Pasch's
axiom\index{Pasch's axiom} went unnoticed for 2000 years
\cite[p.~160]{Davis}!). I'm also somewhat lazy and wish to avoid the effort
involved in re-verifying the gaps in informal proofs ``left to the reader;'' I
prefer to figure them out just once and not have to go through the same
frustration a year from now when I've forgotten what I did.  Metamath provides
better recovery of my efforts than scraps of paper that I can't
decipher anymore.  But mostly I find very appealing the idea of rigorously
archiving mathematical knowledge in a computer database, providing precision,
certainty, and elimination of human error.

\subsubsection{Note on Bibliography and Index}

The Bibliography usually includes the Library of Congress classification
for a work to make it easier for you to find it in on a university
library shelf.  The Index has author references to pages where their works
are cited, even though the authors' names may not appear on those pages.

\subsubsection{Acknowledgments}

Acknowledgments are first due to my wife, Deborah (who passed away on
September 4, 1998), for critiquing the manu\-script but most of all for
her patience and support.  I also wish to thank Joe Wright, Richard
Becker, Clarke Evans, Buddha Buck, and Jeremy Henty for helpful
comments.  Any errors, omissions, and other shortcomings are of course
my responsibility.

\subsubsection{Note Added June 22, 2005}\label{note2002}

The original, unpublished version of this book was written in 1997 and
distributed via the web.  The present edition has been updated to
reflect the current Metamath program and databases, as well as more
current {\sc url}s for Internet sites.  Thanks to Josh
Purinton\index{Purinton, Josh}, One Hand
Clapping, Mel L.\ O'Cat, and Roy F. Longton for pointing out
typographical and other errors.  I have also benefitted from numerous
discussions with Raph Levien\index{Levien, Raph}, who has extended
Metamath's philosophy of rigor to result in his {\em
Ghilbert}\index{Ghilbert} proof language (\url{http://ghilbert.org}).

Robert (Bob) Solovay\index{Solovay, Robert} communicated a new result of
A.~R.~D.~Mathias on the system of Bourbaki, and the text has been
updated accordingly (p.~\pageref{bourbaki}).

Bob also pointed out a clarification of the literature regarding
category theory and inaccessible cardinals\index{category
theory}\index{cardinal, inaccessible} (p.~\pageref{categoryth}),
and a misleading statement was removed from the text.  Specifically,
contrary to a statement in previous editions, it is possible to express
``There is a proper class of inaccessible cardinals'' in the language of
ZFC.  This can be done as follows:  ``For every set $x$ there is an
inaccessible cardinal $\kappa$ such that $\kappa$ is not in $x$.''
Bob writes:\footnote{Private communication, Nov.~30, 2002.}
\begin{quotation}
     This axiom is how Grothendieck presents category theory.  To each
inaccessible cardinal $\kappa$ one associates a Grothendieck universe
\index{Grothendieck, Alexander} $U(\kappa)$.  $U(\kappa)$ consists of
those sets which lie in a transitive set of cardinality less than
$\kappa$.  Instead of the ``category of all groups,'' one works relative
to a universe [considering the category of groups of cardinality less
than $\kappa$].  Now the category whose objects are all categories
``relative to the universe $U(\kappa)$'' will be a category not
relative to this universe but to the next universe.

     All of the things category theorists like to do can be done in this
framework.  The only controversial point is whether the Grothen\-dieck
axiom is too strong for the needs of category theorists.  Mac Lane
\index{Mac Lane, Saunders} argues that ``one universe is enough'' and
Feferman\index{Feferman, Solomon} has argued that one can get by with
ordinary ZFC.  I don't find Feferman's arguments persuasive.  Mac Lane
may be right, but when I think about category theory I do it \`{a} la
Grothendieck.

        By the way Mizar\index{Mizar} adds the axiom ``there is a proper
class of inaccessibles'' precisely so as to do category theory.
\end{quotation}

The most current information on the Metamath program and databases can
always be found at \url{http://metamath.org}.


\subsubsection{Note Added June 24, 2006}\label{note2006}

The Metamath spec was restricted slightly to make parsers easier to
write.  See the footnote on p.~\pageref{namespace}.

%\subsubsection{Note Added July 24, 2006}\label{note2006b}
\subsubsection{Note Added March 10, 2007}\label{note2006b}

I am grateful to Anthony Williams\index{Williams, Anthony} for writing
the \LaTeX\ package called {\tt realref.sty} and contributing it to the
public domain.  This package allows the internal hyperlinks in a {\sc
pdf} file to anchor to specific page numbers instead of just section
titles, making the navigation of the {\sc pdf} file for this book much
more pleasant and ``logical.''

A typographical error found by Martin Kiselkov was corrected.
A confusing remark about unification was deleted per suggestion of
Mel O'Cat.

\subsubsection{Note Added May 27, 2009}\label{note2009}

Several typos found by Kim Sparre were corrected.  A note was added that
the Poincar\'{e} conjecture has been proved (p.~\pageref{poincare}).

\subsubsection{Note Added Nov. 17, 2014}\label{note2014}

The statement of the Schr\"{o}der--Bernstein theorem was corrected in
Section~\ref{trust}.  Thanks to Bob Solovay for pointing out the error.

\subsubsection{Note Added May 25, 2016}\label{note2016}

Thanks to Jerry James for correcting 16 typos.

\subsubsection{Note Added February 25, 2019}\label{note201902}

David A. Wheeler\index{Wheeler, David A.}
made a large number of improvements and updates,
in coordination with Norman Megill.
The predicate calculus axioms were renumbered, and the text makes
it clear that they are based on Tarski's system S2;
the one slight deviation in axiom ax-6 is explained and justified.
The real and complex number axioms were modified to be consistent with
\texttt{set.mm}\index{set theory database (\texttt{set.mm})}%
\index{Metamath Proof Explorer}.
Long-awaited specification changes ``1--8'' were made,
which clarified previously ambiguous points.
Some errors in the text involving \texttt{\$f} and
\texttt{\$d} statements were corrected (the spec was correct, but
the in-book explanations unintentionally contradicted the spec).
We now have a system for automatically generating narrow PDFs,
so that those with smartphones can have easy access to the current
version of this document.
A new section on deduction was added;
it discusses the standard deduction theorem,
the weak deduction theorem,
deduction style, and natural deduction.
Many minor corrections (too numerous to list here) were also made.

\subsubsection{Note Added March 7, 2019}\label{note201903}

This added a description of the Matamath language syntax in
Extended Backus--Naur Form (EBNF)\index{Extended Backus--Naur Form}\index{EBNF}
in Appendix \ref{BNF}, added a brief explanation about typecodes,
inserted more examples in the deduction section,
and added a variety of smaller improvements.

\subsubsection{Note Added April 7, 2019}\label{note201904}

This version clarified the proper substitution notation, improved the
discussion on the weak deduction theorem and natural deduction,
documented the \texttt{undo} command, updated the information on
\texttt{write source}, changed the typecode
from \texttt{set} to \texttt{setvar} to be consistent with the current
version of \texttt{set.mm}, added more documentation about comment markup
(e.g., documented how to create headings), and clarified the
differences between various assertion forms (in particular deduction form).

\subsubsection{Note Added June 2, 2019}\label{note201906}

This version fixes a large number of small issues reported by
Beno\^{i}t Jubin\index{Jubin, Beno\^{i}t}, such as editorial issues
and the need to document \texttt{verify markup} (thank you!).
This version also includes specific examples
of forms (deduction form, inference form, and closed form).
We call this version the ``second edition'';
the previous edition formally published in 2007 had a slightly different title
(\textit{Metamath: A Computer Language for Pure Mathematics}).

\chapter{Introduction}
\pagenumbering{arabic}

\begin{quotation}
  {\em {\em I.M.:}  No, no.  There's nothing subjective about it!  Everybody
knows what a proof is.  Just read some books, take courses from a competent
mathematician, and you'll catch on.

{\em Student:}  Are you sure?

{\em I.M.:}  Well---it is possible that you won't, if you don't have any
aptitude for it.  That can happen, too.

{\em Student:}  Then {\em you} decide what a proof is, and if I don't learn
to decide in the same way, you decide I don't have any aptitude.

{\em I.M.:}  If not me, then who?}
    \flushright\sc  ``The Ideal Mathematician''
    \index{Davis, Phillip J.}
    \footnote{\cite{Davis}, p.~40.}\\
\end{quotation}

Brilliant mathematicians have discovered almost
unimaginably profound results that rank among the crowning intellectual
achievements of mankind.  However, there is a sense in which modern abstract
mathematics is behind the times, stuck in an era before computers existed.
While no one disputes the remarkable results that have been achieved,
communicating these results in a precise way to the uninitiated is virtually
impossible.  To describe these results, a terse informal language is used which
despite its elegance is very difficult to learn.  This informal language is not
imprecise, far from it, but rather it often has omitted detail
and symbols with hidden context that are
implicitly understood by an expert but few others.  Extremely complex technical
meanings are associated with innocent-sounding English words such as
``compact'' and ``measurable'' that barely hint at what is actually being
said.  Anyone who does not keep the precise technical meaning constantly in
mind is bound to fail, and acquiring the ability to do this can be achieved
only through much practice and hard work.  Only the few who complete the
painful learning experience can join the small in-group of pure
mathematicians.  The informal language effectively cuts off the true nature of
their knowledge from most everyone else.

Metamath\index{Metamath} makes abstract mathematics more concrete.  It allows
a computer to keep track of the complexity associated with each word or symbol
with absolute rigor.  You can explore this complexity at your leisure, to
whatever degree you desire.  Whether or not you believe that concepts such as
infinity actually ``exist'' outside of the mind, Metamath lets you get to the
foundation for what's really being said.

Metamath also enables completely rigorous and thorough proof verification.
Its language is simple enough so that you
don't have to rely on the authority of experts but can verify the results
yourself, step by step.  If you want to attempt to derive your own results,
Metamath will not let you make a mistake in reasoning.
Even professional mathematicians make mistakes; Metamath makes it possible
to thoroughly verify that proofs are correct.

Metamath\index{Metamath} is a computer language and an associated computer
program for archiving, verifying, and studying mathematical proofs at a very
detailed level.
The Metamath language
describes formal\index{formal system} mathematical
systems and expresses proofs of theorems in those systems.  Such a language
is called a metalanguage\index{metalanguage} by mathematicians.
The Metamath program is a computer program that verifies
proofs expressed in the Metamath language.
The Metamath program does not have the built-in
ability to make logical inferences; it just makes a series of symbol
substitutions according to instructions given to it in a proof
and verifies that the result matches the expected theorem.  It makes logical
inferences based only on rules of logic that are contained in a set of
axioms\index{axiom}, or first principles, that you provide to it as the
starting point for proofs.

The complete specification of the Metamath language is only four pages long
(Section~\ref{spec}, p.~\pageref{spec}).  Its simplicity may at first make you
wonder how it can do much of anything at all.  But in fact the kinds of
symbol manipulations it performs are the ones that are implicitly done in all
mathematical systems at the lowest level.  You can learn it relatively quickly
and have complete confidence in any mathematical proof that it verifies.  On
the other hand, it is powerful and general enough so that virtually any
mathematical theory, from the most basic to the deeply abstract, can be
described with it.

Although in principle Metamath can be used with any
kind of mathematics, it is best suited for abstract or ``pure'' mathematics
that is mostly concerned with theorems and their proofs, as opposed to the
kind of mathematics that deals with the practical manipulation of numbers.
Examples of branches of pure mathematics are logic\index{logic},\footnote{Logic
is the study of statements that are universally true regardless of the objects
being described by the statements.  An example is the statement, ``if $P$
implies $Q$, then either $P$ is false or $Q$ is true.''} set theory\index{set
theory},\footnote{Set theory is the study of general-purpose mathematical objects called
``sets,'' and from it essentially all of mathematics can be derived.  For
example, numbers can be defined as specific sets, and their properties
can be explored using the tools of set theory.} number theory\index{number
theory},\footnote{Number theory deals with the properties of positive and
negative integers (whole numbers).} group theory\index{group
theory},\footnote{Group theory studies the properties of mathematical objects
called groups that obey a simple set of axioms and have properties of symmetry
that make them useful in many other fields.} abstract algebra\index{abstract
algebra},\footnote{Abstract algebra includes group theory and also studies
groups with additional properties that qualify them as ``rings'' and
``fields.''  The set of real numbers is a familiar example of a field.},
analysis\index{analysis} \index{real and complex numbers}\footnote{Analysis is
the study of real and complex numbers.} and
topology\index{topology}.\footnote{One area studied by topology are properties
that remain unchanged when geometrical objects undergo stretching
deformations; for example a doughnut and a coffee cup each have one hole (the
cup's hole is in its handle) and are thus considered topologically
equivalent.  In general, though, topology is the study of abstract
mathematical objects that obey a certain (surprisingly simple) set of axioms.
See, for example, Munkres \cite{Munkres}\index{Munkres, James R.}.} Even in
physics, Metamath could be applied to certain branches that make use of
abstract mathematics, such as quantum logic\index{quantum logic} (used to study
aspects of quantum mechanics\index{quantum mechanics}).

On the other hand, Metamath\index{Metamath} is less suited to applications
that deal primarily with intensive numeric computations.  Metamath does not
have any built-in representation of numbers\index{Metamath!representation of
numbers}; instead, a specific string of symbols (digits) must be syntactically
constructed as part of any proof in which an ordinary number is used.  For
this reason, numbers in Metamath are best limited to specific constants that
arise during the course of a theorem or its proof.  Numbers are only a tiny
part of the world of abstract mathematics.  The exclusion of built-in numbers
was a conscious decision to help achieve Metamath's simplicity, and there are
other software tools if you have different mathematical needs.
If you wish to quickly solve algebraic problems, the computer algebra
programs\index{computer algebra system} {\sc
macsyma}\index{macsyma@{\sc macsyma}}, Mathematica\index{Mathematica}, and
Maple\index{Maple} are specifically suited to handling numbers and
algebra efficiently.
If you wish to simply calculate numeric or matrix expressions easily,
tools such as Octave\index{Octave} may be a better choice.

After learning Metamath's basic statement types, any
tech\-ni\-cal\-ly ori\-ent\-ed person, mathematician or not, can
immediately trace
any theorem proved in the language as far back as he or she wants, all the way
to the axioms on which the theorem is based.  This ability suggests a
non-traditional way of learning about pure mathematics.  Used in conjunction
with traditional methods, Metamath could make pure mathematics accessible to
people who are not sufficiently skilled to figure out the implicit detail in
ordinary textbook proofs.  Once you learn the axioms of a theory, you can have
complete confidence that everything you need to understand a proof you are
studying is all there, at your beck and call, allowing you to focus in on any
proof step you don't understand in as much depth as you need, without worrying
about getting stuck on a step you can't figure out.\footnote{On the other
hand, writing proofs in the Metamath language is challenging, requiring
a degree of rigor far in excess of that normally taught to students.  In a
classroom setting, I doubt that writing Metamath proofs would ever replace
traditional homework exercises involving informal proofs, because the time
needed to work out the details would not allow a course to
cover much material.  For students who have trouble grasping the implied rigor
in traditional material, writing a few simple proofs in the Metamath language
might help clarify fuzzy thought processes.  Although somewhat difficult at
first, it eventually becomes fun to do, like solving a puzzle, because of the
instant feedback provided by the computer.}

Metamath\index{Metamath} is probably unlike anything you have
encountered before.  In this first chapter we will look at the philosophy and
use of computers in mathematics in order to better understand the motivation
behind Metamath.  The material in this chapter is not required in order to use
Metamath.  You may skip it if you are impatient, but I hope you will find it
educational and enjoyable.  If you want to start experimenting with the
Metamath program right away, proceed directly to Chapter~\ref{using}
(p.~\pageref{using}).  To
learn the Metamath language, skim Chapter~\ref{using} then proceed to
Chapter~\ref{languagespec} (p.~\pageref{languagespec}).

\section{Mathematics as a Computer Language}

\begin{quote}
  {\em The study of mathematics is apt to commence in
dis\-ap\-point\-ment.\ldots \\
We are told that by its aid the stars are weighted
and the billions of molecules in a drop of water are counted.  Yet, like the
ghost of Hamlet's father, this great science eludes the efforts of our mental
weapons to grasp it.}
  \flushright\sc  Alfred North Whitehead\footnote{\cite{Whitehead}, ch.\ 1.}\\
\end{quote}\index{Whitehead, Alfred North}

\subsection{Is Mathematics ``User-Friendly''?}

Suppose you have no formal training in abstract mathematics.  But popular
books you've read offer tempting glimpses of this world filled with profound
ideas that have stirred the human spirit.  You are not satisfied with the
informal, watered-down descriptions you've read but feel it is important to
grasp the underlying mathematics itself to understand its true meaning. It's
not practical to go back to school to learn it, though; you don't want to
dedicate years of your life to it.  There are many important things in life,
and you have to set priorities for what's important to you.  What would happen
if you tried to pursue it on your own, in your spare time?

After all, you were able to learn a computer programming language such as
Pascal on your own without too much difficulty, even though you had no formal
training in computers.  You don't claim to be an expert in software design,
but you can write a passable program when necessary to suit your needs.  Even
more important, you know that you can look at anyone else's Pascal program, no
matter how complex, and with enough patience figure out exactly how it works,
even though you are not a specialist.  Pascal allows you do anything that a
computer can do, at least in principle.  Thus you know you have the ability,
in principle, to follow anything that a computer program can do:  you just
have to break it down into small enough pieces.

Here's an imaginary scenario of what might happen if you na\-ive\-ly a\-dopted
this same view of abstract mathematics and tried to pick it up on your own, in
a period of time comparable to, say, learning a computer programming
language.

\subsubsection{A Non-Mathematician's Quest for Truth}

\begin{quote}
  {\em \ldots my daughters have been studying (chemistry) for several
se\-mes\-ters, think they have learned differential and integral calculus in
school, and yet even today don't know why $x\cdot y=y\cdot x$ is true.}
  \flushright\sc  Edmund Landau\footnote{\cite{Landau}, p.~vi.}\\
\end{quote}\index{Landau, Edmund}

\begin{quote}
  {\em Minus times minus is plus,\\
The reason for this we need not discuss.}
  \flushright\sc W.\ H.\ Auden\footnote{As quoted in \cite{Guillen}, p.~64.}\\
\end{quote}\index{Auden, W.\ H.}\index{Guillen, Michael}

We'll suppose you are a technically oriented professional, perhaps an engineer, a
computer programmer, or a physicist, but probably not a mathematician.  You
consider yourself reasonably intelligent.  You did well in school, learning a
variety of methods and techniques in practical mathematics such as calculus and
differential equations.  But rarely did your courses get into anything
resembling modern abstract mathematics, and proofs were something that appeared
only occasionally in your textbooks, a kind of necessary evil that was
supposed to convince you of a certain key result.  Most of your
homework consisted of exercises that gave you practice in the techniques, and
you were hardly ever asked to come up with a proof of your own.

You find yourself curious about advanced, abstract mathematics.  You are
driven by an inner conviction that it is important to understand and
appreciate some of the most profound knowledge discovered by mankind.  But it
seems very hard to learn, something that only certain gifted longhairs can
access and understand.  You are frustrated that it seems forever cut off from
you.

Eventually your curiosity drives you to do something about it.
You set for yourself a goal of ``really'' understanding mathematics:  not just
how to manipulate equations in algebra or calculus according to cookbook
rules, but rather to gain a deep understanding of where those rules come from.
In fact, you're not thinking about this kind of ordinary mathematics at all,
but about a much more abstract, ethereal realm of pure mathematics, where
famous results such as G\"{o}del's incompleteness theorem\index{G\"{o}del's
incompleteness theorem} and Cantor's different kinds of infinities
reside.

You have probably read a number of popular books, with titles like {\em
Infinity and the Mind} \cite{Rucker}\index{Rucker, Rudy}, on topics such as
these.  You found them inspiring but at the same time somewhat
unsatisfactory.  They gave you a general idea of what these results are about,
but if someone asked you to prove them, you wouldn't have the faintest idea of
where to begin.   Sure, you could give the same overall outline that you
learned from the popular books; and in a general sort of way, you do have an
understanding.  But deep down inside, you know that there is a rigor that is
missing, that probably there are many subtle steps and pitfalls along the way,
and ultimately it seems you have to place your trust in the experts in the
field.  You don't like this; you want to be able to verify these results for
yourself.

So where do you go next?  As a first step, you decide to look up some of the
original papers on the theorems you are curious about, or better, obtain some
standard textbooks in the field.  You look up a theorem you want to
understand.  Sure enough, it's there, but it's expressed with strange
terms and odd symbols that mean absolutely nothing to you.  It might as well be written in
a foreign language you've never seen before, whose symbols are totally alien.
You look at the proof, and you haven't the foggiest notion what each step
means, much less how one step follows from another.  Well, obviously you have
a lot to learn if you want to understand this stuff.

You feel that you could probably understand it by
going back to college for another three to six years and getting a math
degree.  But that does not fit in with your career and the other things in
your life and would serve no practical purpose.  You decide to seek a quicker
path.  You figure you'll just trace your way back to the beginning, step by
step, as you would do with a computer program, until you understand it.  But
you quickly find that this is not possible, since you can't even understand
enough to know what you have to trace back to.

Maybe a different approach is in order---maybe you should start at the
beginning and work your way up.  First, you read the introduction to the book
to find out what the prerequisites are.  In a similar fashion, you trace your
way back through two or three more books, finally arriving at one that seems
to start at a beginning:  it lists the axioms of arithmetic.  ``Aha!'' you
naively think, ``This must be the starting point, the source of all mathematical
knowledge.'' Or at least the starting point for mathematics dealing with
numbers; you have to start somewhere and have no idea what the starting point
for other mathematics would be.  But the word ``axioms'' looks promising.  So
you eagerly read along and work through some elementary exercises at the
beginning of the book.  You feel vaguely bothered:  these
don't seem like axioms at all, at least not in the sense that you want to
think of axioms.  Axioms imply a starting point from which everything else can
be built up, according to precise rules specified in the axiom system.  Even
though you can understand the first few proofs in an informal way,
and are able to do some of the
exercises, it's hard to pin down precisely what the
rules are.   Sure, each step seems to follow logically from the others, but
exactly what does that mean?  Is the ``logic'' just a matter of common sense,
something vague that we all understand but can never quite state precisely?

You've spent a number of years, off and on, programming computers, and you
know that in the case of computer languages there is no question of what the
rules are---they are precise and crystal clear.  If you follow them, your
program will work, and if you don't, it won't.  No matter how complex a
program, it can always be broken down into simpler and simpler pieces, until
you can ultimately identify the bits that are moved around to perform a
specific function.  Some programs might require a lot of perseverance to
accomplish this, but if you focus on a specific portion of it, you don't even
necessarily have to know how the rest of it works. Shouldn't there be an
analogy in mathematics?

You decide to apply the ultimate test:  you ask yourself how a computer could
verify or ensure that the steps in these proofs follow from one another.
Certainly mathematics must be at least as precisely defined as a computer
language, if not more so; after all, computer science itself is based on it.
If you can get a computer to verify these proofs, then you should also be
able, in principle, to understand them yourself in a very crystal clear,
precise way.

You're in for a surprise:  you can conceive of no way to convert the
proofs, which are in English, to a form that the computer can understand.
The proofs are filled with phrases such as ``assume there exists a unique
$x$\ldots'' and ``given any $y$, let $z$ be the number such that\ldots''  This
isn't the kind of logic you are used to in computer programming, where
everything, even arithmetic, reduces to Boolean ones and zeroes if you care to
break it down sufficiently.  Even though you think you understand the proofs,
there seems to be some kind of higher reasoning involved rather than precise
rules that define how you manipulate the symbols in the axioms.  Whatever it
is, it just isn't obvious how you would express it to a computer, and the more
you think about it, the more puzzled and confused you get, to the point where
you even wonder whether {\em you} really understand it.  There's a lot more to
these axioms of arithmetic than meets the eye.

Nobody ever talked about this in school in your applied math and engineering
courses.  You just learned the rules they gave you, not quite understanding
how or why they worked, sometimes vaguely suspicious or uncertain of them, and
through homework problems and osmosis learned how to present solutions that
satisfied the instructor and earned you an ``A.''  Rarely did you actually
``prove'' anything in a rigorous way, and the math majors who did do stuff
like that seemed to be in a different world.

Of course, there are computer algebra programs that can do mathematics, and
rather impressively.  They can instantly solve the integrals that you
struggled with in freshman calculus, and do much, much more.  But when you
look at these programs, what you see is a big collection of algorithms and
techniques that evolved and were added to over time, along with some basic
software that manipulates symbols.  Each algorithm that is built in is the
result of someone's theorem whose proof is omitted; you just have to trust the
person who proved it and the person who programmed it in and hope there are no
bugs.\index{computer program bugs}  Somehow this doesn't seem to be the
essence of mathematics.  Although computer algebra systems can generate
theorems with amazing speed, they can't actually prove a single one of them.

After some puzzlement, you revisit some popular books on what mathematics is
all about.  Somewhere you read that all of mathematics is actually derived
from something called ``set theory.''  This is a little confusing, because
nowhere in the book that presented the axioms of arithmetic was there any
mention of set theory, or if there was, it seemed to be just a tool that helps
you describe things better---the set of even numbers, that sort of thing.  If
set theory is the basis for all mathematics, then why are additional axioms
needed for arithmetic?

Something is wrong but you're not sure what.  One of your friends is a pure
mathematician.  He knows he is unable to communicate to you what he does for a
living and seems to have little interest in trying.  You do know that for him,
proofs are what mathematics is all about. You ask him what a proof is, and he
essentially tells you that, while of course it's based on logic, really it's
something you learn by doing it over and over until you pick it up.  He refers
you to a book, {\em How to Read and Do Proofs} \cite{Solow}.\index{Solow,
Daniel}  Although this book helps you understand traditional informal proofs,
there is still something missing you can't seem to pin down yet.

You ask your friend how you would go about having a computer verify a proof.
At first he seems puzzled by the question; why would you want to do that?
Then he says it's not something that would make any sense to do, but he's
heard that you'd have to break the proof down into thousands or even millions
of individual steps to do such a thing, because the reasoning involved is at
such a high level of abstraction.  He says that maybe it's something you could
do up to a point, but the computer would be completely impractical once you
get into any meaningful mathematics.  There, the only way you can verify a
proof is by hand, and you can only acquire the ability to do this by
specializing in the field for a couple of years in grad school.  Anyway, he
thinks it all has to do with set theory, although he has never taken a formal
course in set theory but just learned what he needed as he went along.

You are intrigued and amazed.  Apparently a mathematician can grasp as a
single concept something that would take a computer a thousand or a million
steps to verify, and have complete confidence in it.  Each one of these
thousand or million steps must be absolutely correct, or else the whole proof
is meaningless.  If you added a million numbers by hand, would you trust the
result?  How do you really know that all these steps are correct, that there
isn't some subtle pitfall in one of these million steps, like a bug in a
computer program?\index{computer program bugs}  After all, you've read that
famous mathematicians have occasionally made mistakes, and you certainly know
you've made your share on your math homework problems in school.

You recall the analogy with a computer program.  Sure, you can understand what
a large computer program such as a word processor does, as a single high-level
concept or a small set of such concepts, but your ability to understand it in
no way ensures that the program is correct and doesn't have hidden bugs.  Even
if you wrote the program yourself you can't really know this; most large
programs that you've written have had bugs that crop up at some later date, no
matter how careful you tried to be while writing them.

OK, so now it seems the reason you can't figure out how to make a
computer verify proofs is because each step really corresponds to a
million small steps.  Well, you say, a computer can do a million
calculations in a second, so maybe it's still practical to do.  Now the
puzzle becomes how to figure out what the million steps are that each
English-language step corresponds to.  Your mathematician friend hasn't
a clue, but suggests that maybe you would find the answer by studying
set theory.  Actually, your friend thinks you're a little off the wall
for even wondering such a thing.  For him, this is not what mathematics
is all about.

The subject of set theory keeps popping up, so you decide it's
time to look it up.

You decide to start off on a careful footing, so you start reading a couple of
very elementary books on set theory.  A lot of it seems pretty obvious, like
intersections, subsets, and Venn diagrams.  You thumb through one of the
books; nowhere is anything about axioms mentioned. The other book relegates to
an appendix a brief discussion that mentions a set of axioms called
``Zermelo--Fraenkel set theory''\index{Zermelo--Fraenkel set theory} and states
them in English.  You look at them and have no idea what they really mean or
what you can do with them.  The comments in this appendix say that the purpose
of mentioning them is to expose you to the idea, but imply that they are not
necessary for basic understanding and that they are really the subject matter
of advanced treatments where fine points such as a certain paradox (Russell's
paradox\index{Russell's paradox}\footnote{Russell's paradox assumes that there
exists a set $S$ that is a collection of all sets that don't contain
themselves.  Now, either $S$ contains itself or it doesn't.  If it contains
itself, it contradicts its definition.  But if it doesn't contain itself, it
also contradicts its definition.  Russell's paradox is resolved in ZF set
theory by denying that such a set $S$ exists.}) are resolved.  Wait a
minute---shouldn't the axioms be a starting point, not an ending point?  If
there are paradoxes that arise without the axioms, how do you know you won't
stumble across one accidentally when using the informal approach?

And nowhere do these books describe how ``all of mathematics can be
derived from set theory'' which by now you've heard a few times.

You find a more advanced book on set theory.  This one actually lists the
axioms of ZF set theory in plain English on page one.  {\em Now} you think
your quest has ended and you've finally found the source of all mathematical
knowledge; you just have to understand what it means.  Here, in one place, is
the basis for all of mathematics!  You stare at the axioms in awe, puzzle over
them, memorize them, hoping that if you just meditate on them long enough they
will become clear.  Of course, you haven't the slightest idea how the rest of
mathematics is ``derived'' from them; in particular, if these are the axioms
of mathematics, then why do arithmetic, group theory, and so on need their own
axioms?

You start reading this advanced book carefully, pondering the meaning of every
word, because by now you're really determined to get to the bottom of this.
The first thing the book does is explain how the axioms came about, which was
to resolve Russell's paradox.\index{Russell's paradox}  In fact that seems to
be the main purpose of their existence; that they supposedly can be used to
derive all of mathematics seems irrelevant and is not even mentioned.  Well,
you go on.  You hope the book will explain to you clearly, step by step, how
to derive things from the axioms.  After all, this is the starting point of
mathematics, like a book that explains the basics of a computer programming
language.  But something is missing.  You find you can't even understand the
first proof or do the first exercise.  Symbols such as $\exists$ and $\forall$
permeate the page without any mention of where they came from or how to
manipulate them; the author assumes you are totally familiar with them and
doesn't even tell you what they mean.  By now you know that $\exists$ means
``there exists'' and $\forall$ means ``for all,'' but shouldn't the rules for
manipulating these symbols be part of the axioms?  You still have no idea
how you could even describe the axioms to a computer.

Certainly there is something much different here from the technical
literature you're used to reading.  A computer language manual almost
always explains very clearly what all the symbols mean, precisely what
they do, and the rules used for combining them, and you work your way up
from there.

After glancing at four or five other such books, you come to the realization
that there is another whole field of study that you need just to get to the
point at which you can understand the axioms of set theory.  The field is
called ``logic.''  In fact, some of the books did recommend it as a
prerequisite, but it just didn't sink in.  You assumed logic was, well, just
logic, something that a person with common sense intuitively understood.  Why
waste your time reading boring treatises on symbolic logic, the manipulation
of 1's and 0's that computers do, when you already know that?  But this is a
different kind of logic, quite alien to you.  The subject of {\sc nand} and
{\sc nor} gates is not even touched upon or in any case has to do with only a
very small part of this field.

So your quest continues.  Skimming through the first couple of introductory
books, you get a general idea of what logic is about and what quantifiers
(``for all,'' ``there exists'') mean, but you find their examples somewhat
trivial and mildly annoying (``all dogs are animals,'' ``some animals are
dogs,'' and such).  But all you want to know is what the rules are for
manipulating the symbols so you can apply them to set theory.  Some formulas
describing the relationships among quantifiers ($\exists$ and $\forall$) are
listed in tables, along with some verbal reasoning to justify them.
Presumably, if you want to find out if a formula is correct, you go through
this same kind of mental reasoning process, possibly using images of dogs and
animals. Intuitively, the formulas seem to make sense.  But when you ask
yourself, ``What are the rules I need to get a computer to figure out whether
this formula is correct?'', you still don't know.  Certainly you don't ask the
computer to imagine dogs and animals.

You look at some more advanced logic books.  Many of them have an introductory
chapter summarizing set theory, which turns out to be a prerequisite.  You
need logic to understand set theory, but it seems you also need set theory to
understand logic!  These books jump right into proving rather advanced
theorems about logic, without offering the faintest clue about where the logic
came from that allows them to prove these theorems.

Luckily, you come across an elementary book of logic that, halfway through,
after the usual truth tables and metaphors, presents in a clear, precise way
what you've been looking for all along: the axioms!  They're divided into
propositional calculus (also called sentential logic) and predicate calculus
(also called first-order logic),\index{first-order logic} with rules so simple
and crystal clear that now you can finally program a computer to understand
them.  Indeed, they're no harder than learning how to play a game of chess.
As far as what you seem to need is concerned, the whole book could have been
written in five pages!

{\em Now} you think you've found the ultimate source of mathematical
truth.  So---the axioms of mathematics consist of these axioms of logic,
together with the axioms of ZF set theory. (By now you've also been able to
figure out how to translate the ZF axioms from English into the
actual symbols of logic which you can now manipulate according to
precise, easy-to-understand rules.)

Of course, you still don't understand how ``all of mathematics can be
derived from set theory,'' but maybe this will reveal itself in due
course.

You eagerly set out to program the axioms and rules into a computer and start
to look at the theorems you will have to prove as the logic is developed.  All
sorts of important theorems start popping up:  the deduction
theorem,\index{deduction theorem} the substitution theorem,\index{substitution
theorem} the completeness theorem of propositional calculus,\index{first-order
logic!completeness} the completeness theorem of predicate calculus.  Uh-oh,
there seems to be trouble.  They all get harder and harder, and not one of
them can be derived with the axioms and rules of logic you've just been
handed.  Instead, they all require ``metalogic'' for their proofs, a kind of
mixture of logic and set theory that allows you to prove things {\em about}
the axioms and theorems of logic rather than {\em with} them.

You plow ahead anyway.  A month later, you've spent much of your
free time getting the computer to verify proofs in propositional calculus.
You've programmed in the axioms, but you've also had to program in the
deduction theorem, the substitution theorem, and the completeness theorem of
propositional calculus, which by now you've resigned yourself to treating as
rather complex additional axioms, since they can't be proved from the axioms
you were given.  You can now get the computer to verify and even generate
complete, rigorous, formal proofs\index{formal proof}.  Never mind that they
may have 100,000 steps---at least now you can have complete, absolute
confidence in them.  Unfortunately, the only theorems you have proved are
pretty trivial and you can easily verify them in a few minutes with truth
tables, if not by inspection.

It looks like your mathematician friend was right.  Getting the computer to do
serious mathematics with this kind of rigor seems almost hopeless.  Even
worse, it seems that the further along you get, the more ``axioms'' you have
to add, as each new theorem seems to involve additional ``metamathematical''
reasoning that hasn't been formalized, and none of it can be derived from the
axioms of logic.  Not only do the proofs keep growing exponentially as you get
further along, but the program to verify them keeps getting bigger and bigger
as you program in more ``metatheorems.''\index{metatheorem}\footnote{A
metatheorem is usually a statement that is too general to be directly provable
in a theory.  For example, ``if $n_1$, $n_2$, and $n_3$ are integers, then
$n_1+n_2+n_3$ is an integer'' is a theorem of number theory.  But ``for any
integer $k > 1$, if $n_1, \ldots, n_k$ are integers, then $n_1+\ldots +n_k$ is
an integer'' is a metatheorem, in other words a family of theorems, one for
every $k$.  The reason it is not a theorem is that the general sum $n_1+\ldots
+n_k$ (as a function of $k$) is not an operation that can be defined directly
in number theory.} The bugs\index{computer program bugs} that have cropped up
so far have already made you start to lose faith in the rigor you seem to have
achieved, and you know it's just going to get worse as your program gets larger.

\subsection{Mathematics and the Non-Specialist}

\begin{quote}
  {\em A real proof is not checkable by a machine, or even by any mathematician
not privy to the gestalt, the mode of thought of the particular field of
mathematics in which the proof is located.}
  \flushright\sc  Davis and Hersh\index{Davis, Phillip J.}
  \footnote{\cite{Davis}, p.~354.}\\
\end{quote}

The bulk of abstract or theoretical mathematics is ordinarily outside
the reach of anyone but a few specialists in each field who have completed
the necessary difficult internship in order to enter its coterie.  The
typical intelligent layperson has no reasonable hope of understanding much of
it, nor even the specialist mathematician of understanding other fields.  It
is like a foreign language that has no dictionary to look up the translation;
the only way you can learn it is by living in the country for a few years.  It
is argued that the effort involved in learning a specialty is a necessary
process for acquiring a deep understanding.  Of course, this is almost certainly
true if one is to make significant contributions to a field; in particular,
``doing'' proofs is probably the most important part of a mathematician's
training.  But is it also necessary to deny outsiders access to it?  Is it
necessary that abstract mathematics be so hard for a layperson to grasp?

A computer normally is of no help whatsoever.  Most published proofs are
actually just series of hints written in an informal style that requires
considerable knowledge of the field to understand.  These are the ``real
proofs'' referred to by Davis and Hersh.\index{informal proof}  There is an
implicit understanding that, in principle, such a proof could be converted to
a complete formal proof\index{formal proof}.  However, it is said that no one
would ever attempt such a conversion, even if they could, because that would
presumably require millions of steps (Section~\ref{dream}).  Unfortunately the
informal style automatically excludes the understanding of the proof
by anyone who hasn't gone through the necessary apprenticeship. The
best that the intelligent layperson can do is to read popular books about deep
and famous results; while this can be helpful, it can also be misleading, and
the lack of detail usually leaves the reader with no ability whatsoever to
explore any aspect of the field being described.

The statements of theorems often use sophisticated notation that makes them
inaccessible to the non-specialist.  For a non-specialist who wants to achieve
a deeper understanding of a proof, the process of tracing definitions and
lemmas back through their hierarchy\index{hierarchy} quickly becomes confusing
and discouraging.  Textbooks are usually written to train mathematicians or to
communicate to people who are already mathematicians, and large gaps in proofs
are often left as exercises to the reader who is left at an impasse if he or
she becomes stuck.

I believe that eventually computers will enable non-specialists and even
intelligent laypersons to follow almost any mathematical proof in any field.
Metamath is an attempt in that direction.  If all of mathematics were as
easily accessible as a computer programming language, I could envision
computer programmers and hobbyists who otherwise lack mathematical
sophistication exploring and being amazed by the world of theorems and proofs
in obscure specialties, perhaps even coming up with results of their own.  A
tremendous advantage would be that anyone could experiment with conjectures in
any field---the computer would offer instant feedback as to whether
an inference step was correct.

Mathematicians sometimes have to put up with the annoyance of
cranks\index{cranks} who lack a fundamental understanding of mathematics but
insist that their ``proofs'' of, say, Fermat's Last Theorem\index{Fermat's
Last Theorem} be taken seriously.  I think part of the problem is that these
people are misled by informal mathematical language, treating it as if they
were reading ordinary expository English and failing to appreciate the
implicit underlying rigor.  Such cranks are rare in the field of computers,
because computer languages are much more explicit, and ultimately the proof is
in whether a computer program works or not.  With easily accessible
computer-based abstract mathematics, a mathematician could say to a crank,
``don't bother me until you've demonstrated your claim on the computer!''

% 22-May-04 nm
% Attempt to move De Millo quote so it doesn't separate from attribution
% CHANGE THIS NUMBER (AND ELIMINATE IF POSSIBLE) WHEN ABOVE TEXT CHANGES
\vspace{-0.5em}

\subsection{An Impossible Dream?}\label{dream}

\begin{quote}
  {\em Even quite basic theorems would demand almost unbelievably vast
  books to display their proofs.}
    \flushright\sc  Robert E. Edwards\footnote{\cite{Edwards}, p.~68.}\\
\end{quote}\index{Edwards, Robert E.}

\begin{quote}
  {\em Oh, of course no one ever really {\em does} it.  It would take
  forever!  You just show that you could do it, that's sufficient.}
    \flushright\sc  ``The Ideal Mathematician''
    \index{Davis, Phillip J.}\footnote{\cite{Davis},
p.~40.}\\
\end{quote}

\begin{quote}
  {\em There is a theorem in the primitive notation of set theory that
  corresponds to the arithmetic theorem `$1000+2000=3000$'.  The formula
  would be forbiddingly long\ldots even if [one] knows the definitions
  and is asked to simplify the long formula according to them, chances are
  he will make errors and arrive at some incorrect result.}
    \flushright\sc  Hao Wang\footnote{\cite{Wang}, p.~140.}\\
\end{quote}\index{Wang, Hao}

% 22-May-04 nm
% Attempt to move De Millo quote so it doesn't separate from attribution
% CHANGE THIS NUMBER (AND ELIMINATE IF POSSIBLE) WHEN ABOVE TEXT CHANGES
\vspace{-0.5em}

\begin{quote}
  {\em The {\em Principia Mathematica} was the crowning achievement of the
  formalists.  It was also the deathblow of the formalist view.\ldots
  {[Rus\-sell]} failed, in three enormous volumes, to get beyond the elementary
  facts of arithmetic.  He showed what can be done in principle and what
  cannot be done in practice.  If the mathematical process were really
  one of strict, logical progression, we would still be counting our
  fingers.\ldots
  One theoretician estimates\ldots that a demonstration of one of
  Ramanujan's conjectures assuming set theory and elementary analysis would
  take about two thousand pages; the length of a deduction from first principles
  is nearly in\-con\-ceiv\-a\-ble\ldots The probabilists argue that\ldots any
  very long proof can at best be viewed as only probably correct\ldots}
  \flushright\sc Richard de Millo et. al.\footnote{\cite{deMillo}, pp.~269,
  271.}\\
\end{quote}\index{de Millo, Richard}

A number of writers have conveyed the impression that the kind of absolute
rigor provided by Metamath\index{Metamath} is an impossible dream, suggesting
that a complete, formal verification\index{formal proof} of a typical theorem
would take millions of steps in untold volumes of books.  Even if it could be
done, the thinking sometimes goes, all meaning would be lost in such a
monstrous, tedious verification.\index{informal proof}\index{proof length}

These writers assume, however, that in order to achieve the kind of complete
formal verification they desire one must break down a proof into individual
primitive steps that make direct reference to the axioms.  This is
not necessary.  There is no reason not to make use of previously proved
theorems rather than proving them over and over.

Just as important, definitions\index{definition} can be introduced along
the way, allowing very complex formulas to be represented with few
symbols.  Not doing this can lead to absurdly long formulas.  For
example, the mere statement of
G\"{o}del's incompleteness theorem\index{G\"{o}del's
incompleteness theorem}, which can be expressed with a small number of
defined symbols, would require about 20,000 primitive symbols to express
it.\index{Boolos, George S.}\footnote{George S.\ Boolos, lecture at
Massachusetts Institute of Technology, spring 1990.} An extreme example
is Bourbaki's\label{bourbaki} language for set theory, which requires
4,523,659,424,929 symbols plus 1,179,618,517,981 disambiguatory links
(lines connecting symbol pairs, usually drawn below or above the
formula) to express the number
``one'' \cite{Mathias}.\index{Mathias, Adrian R. D.}\index{Bourbaki,
Nicolas}
% http://www.dpmms.cam.ac.uk/~ardm/

A hierarchy\index{hierarchy} of theorems and definitions permits an
exponential growth in the formula sizes and primitive proof steps to be
described with only a linear growth in the number of symbols used.  Of course,
this is how ordinary informal mathematics is normally done anyway, but with
Metamath\index{Metamath} it can be done with absolute rigor and precision.

\subsection{Beauty}


\begin{quote}
  {\em No one shall be able to drive us from the paradise that Cantor has
created for us.}
   \flushright\sc  David Hilbert\footnote{As quoted in \cite{Moore}, p.~131.}\\
\end{quote}\index{Hilbert, David}

\needspace{3\baselineskip}
\begin{quote}
  {\em Mathematics possesses not only truth, but some supreme beauty ---a
  beauty cold and austere, like that of a sculpture.}
    \flushright\sc  Bertrand
    Russell\footnote{\cite{Russell}.}\\
\end{quote}\index{Russell, Bertrand}

\begin{quote}
  {\em Euclid alone has looked on Beauty bare.}
  \flushright\sc Edna Millay\footnote{As quoted in \cite{Davis}, p.~150.}\\
\end{quote}\index{Millay, Edna}

For most people, abstract mathematics is distant, strange, and
incomprehensible.  Many popular books have tried to convey some of the sense
of beauty in famous theorems.  But even an intelligent layperson is left with
only a general idea of what a theorem is about and is hardly given the tools
needed to make use of it.  Traditionally, it is only after years of arduous
study that one can grasp the concepts needed for deep understanding.
Metamath\index{Metamath} allows you to approach the proof of the theorem from
a quite different perspective, peeling apart the formulas and definitions
layer by layer until an entirely different kind of understanding is achieved.
Every step of the proof is there, pieced together with absolute precision and
instantly available for inspection through a microscope with a magnification
as powerful as you desire.

A proof in itself can be considered an object of beauty.  Constructing an
elegant proof is an art.  Once a famous theorem has been proved, often
considerable effort is made to find simpler and more easily understood
proofs.  Creating and communicating elegant proofs is a major concern of
mathematicians.  Metamath is one way of providing a common language for
archiving and preserving this information.

The length of a proof can, to a certain extent, be considered an
objective measure of its ``beauty,'' since shorter proofs are usually
considered more elegant.  In the set theory database
\texttt{set.mm}\index{set theory database (\texttt{set.mm})}%
\index{Metamath Proof Explorer}
provided with Metamath, one goal was to make all proofs as short as possible.

\needspace{4\baselineskip}
\subsection{Simplicity}

\begin{quote}
  {\em God made man simple; man's complex problems are of his own
  devising.}
    \flushright\sc Eccles. 7:29\footnote{Jerusalem Bible.}\\
\end{quote}\index{Bible}

\needspace{3\baselineskip}
\begin{quote}
  {\em God made integers, all else is the work of man.}
    \flushright\sc Leopold Kronecker\footnote{{\em Jahresbericht
	der Deutschen Mathematiker-Vereinigung }, vol. 2, p. 19.}\\
\end{quote}\index{Kronecker, Leopold}

\needspace{3\baselineskip}
\begin{quote}
  {\em For what is clear and easily comprehended attracts; the
  complicated repels.}
    \flushright\sc David Hilbert\footnote{As quoted in \cite{deMillo},
p.~273.}\\
\end{quote}\index{Hilbert, David}

The Metamath\index{Metamath} language is simple and Spartan.  Metamath treats
all mathematical expressions as simple sequences of symbols, devoid of meaning.
The higher-level or ``metamathematical'' notions underlying Metamath are about
as simple as they could possibly be.  Each individual step in a proof involves
a single basic concept, the substitution of an expression for a variable, so
that in principle almost anyone, whether mathematician or not, can
completely understand how it was arrived at.

In one of its most basic applications, Metamath\index{Metamath} can be used to
develop the foundations of mathematics\index{foundations of mathematics} from
the very beginning.  This is done in the set theory database that is provided
with the Metamath package and is the subject matter
of Chapter~\ref{fol}. Any language (a metalanguage\index{metalanguage})
used to describe mathematics (an object language\index{object language}) must
have a mathematical content of its own, but it is desirable to keep this
content down to a bare minimum, namely that needed to make use of the
inference rules specified by the axioms.  With any metalanguage there is a
``chicken and egg'' problem somewhat like circular reasoning:  you must assume
the validity of the mathematics of the metalanguage in order to prove the
validity of the mathematics of the object language.  The mathematical content
of Metamath itself is quite limited.  Like the rules of a game of chess, the
essential concepts are simple enough so that virtually anyone should be able to
understand them (although that in itself will not let you play like
a master).  The symbols that Metamath manipulates do not in themselves
have any intrinsic meaning.  Your interpretation of the axioms that you supply
to Metamath is what gives them meaning.  Metamath is an attempt to strip down
mathematical thought to its bare essence and show you exactly how the symbols
are manipulated.

Philosophers and logicians, with various motivations, have often thought it
important to study ``weak'' fragments of logic\index{weak logic}
\cite{Anderson}\index{Anderson, Alan Ross} \cite{MegillBunder}\index{Megill,
Norman}\index{Bunder, Martin}, other unconventional systems of logic (such as
``modal'' logic\index{modal logic} \cite[ch.\ 27]{Boolos}\index{Boolos, George
S.}), and quantum logic\index{quantum logic} in physics
\cite{Pavicic}\index{Pavi{\v{c}}i{\'{c}}, M.}.  Metamath\index{Metamath}
provides a framework in which such systems can be expressed, with an absolute
precision that makes all underlying metamathematical assumptions rigorous and
crystal clear.

Some schools of philosophical thought, for example
intuitionism\index{intuitionism} and constructivism\index{constructivism},
demand that the notions underlying any mathematical system be as simple and
concrete as possible.  Metamath should meet the requirements of these
philosophies.  Metamath must be taught the symbols, axioms\index{axiom}, and
rules\index{rule} for a specific theory, from the skeptical (such as
intuitionism\index{intuitionism}\footnote{Intuitionism does not accept the law
of excluded middle (``either something is true or it is not true'').  See
\cite[p.~xi]{Tymoczko}\index{Tymoczko, Thomas} for discussion and references
on this topic.  Consider the theorem, ``There exist irrational numbers $a$ and
$b$ such that $a^b$ is rational.''  An intuitionist would reject the following
proof:  If $\sqrt{2}^{\sqrt{2}}$ is rational, we are done.  Otherwise, let
$a=\sqrt{2}^{\sqrt{2}}$ and $b=\sqrt{2}$. Then $a^b=2$, which is rational.})
to the bold (such as the axiom of choice in set theory\footnote{The axiom of
choice\index{Axiom of Choice} asserts that given any collection of pairwise
disjoint nonempty sets, there exists a set that has exactly one element in
common with each set of the collection.  It is used to prove many important
theorems in standard mathematics.  Some philosophers object to it because it
asserts the existence of a set without specifying what the set contains
\cite[p.~154]{Enderton}\index{Enderton, Herbert B.}.  In one foundation for
mathematics due to Quine\index{Quine, Willard Van Orman}, that has not been
otherwise shown to be inconsistent, the axiom of choice turns out to be false
\cite[p.~23]{Curry}\index{Curry, Haskell B.}.  The \texttt{show
trace{\char`\_}back} command of the Metamath program allows you to find out
whether the axiom of choice, or any other axiom, was assumed by a
proof.}\index{\texttt{show trace{\char`\_}back} command}).

The simplicity of the Metamath language lets the algorithm (computer program)
that verifies the validity of a Metamath proof be straightforward and
robust.  You can have confidence that the theorems it verifies really can be
derived from your axioms.

\subsection{Rigor}

\begin{quote}
  {\em Rigor became a goal with the Greeks\ldots But the efforts to
  pursue rigor to the utmost have led to an impasse in which there is
  no longer any agreement on what it really means.  Mathematics remains
  alive and vital, but only on a pragmatic basis.}
    \flushright\sc  Morris Kline\footnote{\cite{Kline}, p.~1209.}\\
\end{quote}\index{Kline, Morris}

Kline refers to a much deeper kind of rigor than that which we will discuss in
this section.  G\"{o}del's incompleteness theorem\index{G\"{o}del's
incompleteness theorem} showed that it is impossible to achieve absolute rigor
in standard mathematics because we can never prove that mathematics is
consistent (free from contradictions).\index{consistent theory}  If
mathematics is consistent, we will never know it, but must rely on faith.  If
mathematics is inconsistent, the best we can hope for is that some clever
future mathematician will discover the inconsistency.  In this case, the
axioms would probably be revised slightly to eliminate the inconsistency, as
was done in the case of Russell's paradox,\index{Russell's paradox} but the
bulk of mathematics would probably not be affected by such a discovery.
Russell's paradox, for example, did not affect most of the remarkable results
achieved by 19th-century and earlier mathematicians.  It mainly invalidated
some of Gottlob Frege's\index{Frege, Gottlob} work on the foundations of
mathematics in the late 1800's; in fact Frege's work inspired Russell's
discovery.  Despite the paradox, Frege's work contains important concepts that
have significantly influenced modern logic.  Kline's {\em Mathematics, The
Loss of Certainty} \cite{Klinel}\index{Kline, Morris} has an interesting
discussion of this topic.

What {\em can} be achieved with absolute certainty\index{certainty} is the
knowledge that if we assume the axioms are consistent and true, then the
results derived from them are true.  Part of the beauty of mathematics is that
it is the one area of human endeavor where absolute certainty can be achieved
in this sense.  A mathematical truth will remain such for eternity.  However,
our actual knowledge of whether a particular statement is a mathematical truth
is only as certain as the correctness of the proof that establishes it.  If
the proof of a statement is questionable or vague, we can't have absolute
confidence in the truth that the statement claims.

Let us look at some traditional ways of expressing proofs.

Except in the field of formal logic\index{formal logic}, almost all
traditional proofs in mathematics are really not proofs at all, but rather
proof outlines or hints as to how to go about constructing the proof.  Many
gaps\index{gaps in proofs} are left for the reader to fill in. There are
several reasons for this.  First, it is usually assumed in mathematical
literature that the person reading the proof is a mathematician familiar with
the specialty being described, and that the missing steps are obvious to such
a reader or at least that the reader is capable of filling them in.  This
attitude is fine for professional mathematicians in the specialty, but
unfortunately it often has the drawback of cutting off the rest of the world,
including mathematicians in other specialties, from understanding the proof.
We discussed one possible resolution to this on p.~\pageref{envision}.
Second, it is often assumed that a complete formal proof\index{formal proof}
would require countless millions of symbols (Section~\ref{dream}). This might
be true if the proof were to be expressed directly in terms of the axioms of
logic and set theory,\index{set theory} but it is usually not true if we allow
ourselves a hierarchy\index{hierarchy} of definitions and theorems to build
upon, using a notation that allows us to introduce new symbols, definitions,
and theorems in a precisely specified way.

Even in formal logic,\index{formal logic} formal proofs\index{formal proof}
that are considered complete still contain hidden or implicit information.
For example, a ``proof'' is usually defined as a sequence of
wffs,\index{well-formed formula (wff)}\footnote{A {\em wff} or well-formed
formula is a mathematical expression (string of symbols) constructed according
to some precise rules.  A formal mathematical system\index{formal system}
contains (1) the rules for constructing syntactically correct
wffs,\index{syntax rules} (2) a list of starting wffs called
axioms,\index{axiom} and (3) one or more rules prescribing how to derive new
wffs, called theorems\index{theorem}, from the axioms or previously derived
theorems.  An example of such a system is contained in
Metamath's\index{Metamath} set theory database, which defines a formal
system\index{formal system} from which all of standard mathematics can be
derived.  Section~\ref{startf} steps you through a complete example of a formal
system, and you may want to skim it now if you are unfamiliar with the
concept.} each of which is an axiom or follows from a rule applied to previous
wffs in the sequence.  The implicit part of the proof is the algorithm by
which a sequence of symbols is verified to be a valid wff, given the
definition of a wff.  The algorithm in this case is rather simple, but for a
computer to verify the proof,\index{automated proof verification} it must have
the algorithm built into its verification program.\footnote{It is possible, of
course, to specify wff construction syntax outside of the program itself
with a suitable input language (the Metamath language being an example), but
some proof-verification or theorem-proving programs lack the ability to extend
wff syntax in such a fashion.} If one deals exclusively with axioms and
elementary wffs, it is straightforward to implement such an algorithm.  But as
more and more definitions are added to the theory in order to make the
expression of wffs more compact, the algorithm becomes more and more
complicated.  A computer program that implements the algorithm becomes larger
and harder to understand as each definition is introduced, and thus more prone
to bugs.\index{computer program bugs}  The larger the program, the
more suspicious the mathematician may be about
the validity of its algorithms.  This is especially true because
computer programs are inherently hard to follow to begin with, and few people
enjoy verifying them manually in detail.

Metamath\index{Metamath} takes a different approach.  Metamath's ``knowledge''
is limited to the ability to substitute variables for expressions, subject to
some simple constraints.  Once the basic algorithm of Metamath is assumed to
be debugged, and perhaps independently confirmed, it
can be trusted once and for all.  The information that Metamath needs to
``understand'' mathematics is contained entirely in the body of knowledge
presented to Metamath.  Any errors in reasoning can only be errors in the
axioms or definitions contained in this body of knowledge.  As a
``constructive'' language\index{constructive language} Metamath has no
conditional branches or loops like the ones that make computer programs hard
to decipher; instead, the language can only build new sequences of symbols
from earlier sequences  of symbols.

The simplicity of the rules that underlie Metamath not only makes Metamath
easy to learn but also gives Metamath a great deal of flexibility. For
example, Metamath is not limited to describing standard first-order
logic\index{first-order logic}; higher-order logics\index{higher-order logic}
and fragments of logic\index{weak logic} can be described just as easily.
Metamath gives you the freedom to define whatever wff notation you prefer; it
has no built-in conception of the syntax of a wff.\index{well-formed formula
(wff)}  With suitable axioms and definitions, Metamath can even describe and
prove things about itself.\index{Metamath!self-description}  (John
Harrison\index{Harrison, John} discusses the ``reflection''
principle\index{reflection principle} involved in self-descriptive systems in
\cite{Harrison}.)

The flexibility of Metamath requires that its proofs specify a lot of detail,
much more than in an ordinary ``formal'' proof.\index{formal proof}  For
example, in an ordinary formal proof, a single step consists of displaying the
wff that constitutes that step.  In order for a computer program to verify
that the step is acceptable, it first must verify that the symbol sequence
being displayed is an acceptable wff.\index{automated proof verification} Most
proof verifiers have at least basic wff syntax built into their programs.
Metamath has no hard-wired knowledge of what constitutes a wff built into it;
instead every wff must be explicitly constructed based on rules defining wffs
that are present in a database.  Thus a single step in an ordinary formal
proof may be correspond to many steps in a Metamath proof. Despite the larger
number of steps, though, this does not mean that a Metamath proof must be
significantly larger than an ordinary formal proof. The reason is that since
we have constructed the wff from scratch, we know what the wff is, so there is
no reason to display it.  We only need to refer to a sequence of statements
that construct it.  In a sense, the display of the wff in an ordinary formal
proof is an implicit proof of its own validity as a wff; Metamath just makes
the proof explicit. (Section~\ref{proof} describes Metamath's proof notation.)

\section{Computers and Mathematicians}

\begin{quote}
  {\em The computer is important, but not to mathematics.}
    \flushright\sc  Paul Halmos\footnote{As quoted in \cite{Albers}, p.~121.}\\
\end{quote}\index{Halmos, Paul}

Pure mathematicians have traditionally been indifferent to computers, even to
the point of disdain.\index{computers and pure mathematics}  Computer science
itself is sometimes considered to fall in the mundane realm of ``applied''
mathematics, perhaps essential for the real world but intellectually unexciting
to those who seek the deepest truths in mathematics.  Perhaps a reason for this
attitude towards computers is that there is little or no computer software that
meets their needs, and there may be a general feeling that such software could
not even exist.  On the one hand, there are the practical computer algebra
systems, which can perform amazing symbolic manipulations in algebra and
calculus,\index{computer algebra system} yet can't prove the simplest
existence theorem, if the idea of a proof is present at all.  On the other
hand, there are specialized automated theorem provers that technically speaking
may generate correct proofs.\index{automated theorem proving}  But sometimes
their specialized input notation may be cryptic and their output perceived to
be long, inelegant, incomprehensible proofs.    The output
may be viewed with suspicion, since the program that generates it tends to be
very large, and its size increases the potential for bugs\index{computer
program bugs}.  Such a proof may be considered trustworthy only if
independently verified and ``understood'' by a human, but no one wants to
waste their time on such a boring, unrewarding chore.



\needspace{4\baselineskip}
\subsection{Trusting the Computer}

\begin{quote}
  {\em \ldots I continue to find the quasi-empirical interpretation of
  computer proofs to be the more plausible.\ldots Since not
  everything that claims to be a computer proof can be
  accepted as valid, what are the mathematical criteria for acceptable
  computer proofs?}
    \flushright\sc  Thomas Tymoczko\footnote{\cite{Tymoczko}, p.~245.}\\
\end{quote}\index{Tymoczko, Thomas}

In some cases, computers have been essential tools for proving famous
theorems.  But if a proof is so long and obscure that it can be verified in a
practical way only with a computer, it is vaguely felt to be suspicious.  For
example, proving the famous four-color theorem\index{four-color
theorem}\index{proof length} (``a map needs no more than four colors to
prevent any two adjacent countries from having the same color'') can presently
only be done with the aid of a very complex computer program which originally
required 1200 hours of computer time. There has been considerable debate about
whether such a proof can be trusted and whether such a proof is ``real''
mathematics \cite{Swart}\index{Swart, E. R.}.\index{trusting computers}

However, under normal circumstances even a skeptical mathematician would have a
great deal of confidence in the result of multiplying two numbers on a pocket
calculator, even though the precise details of what goes on are hidden from its
user.  Even the verification on a supercomputer that a huge number is prime is
trusted, especially if there is independent verification; no one bothers to
debate the philosophical significance of its ``proof,'' even though the actual
proof would be so large that it would be completely impractical to ever write
it down on paper.  It seems that if the algorithm used by the computer is
simple enough to be readily understood, then the computer can be trusted.

Metamath\index{Metamath} adopts this philosophy.  The simplicity of its
language makes it easy to learn, and because of its simplicity one can have
essentially absolute confidence that a proof is correct. All axioms, rules, and
definitions are available for inspection at any time because they are defined
by the user; there are no hidden or built-in rules that may be prone to subtle
bugs\index{computer program bugs}.  The basic algorithm at the heart of
Metamath is simple and fixed, and it can be assumed to be bug-free and robust
with a degree of confidence approaching certainty.
Independently written implementations of the Metamath verifier
can reduce any residual doubt on the part of a skeptic even further;
there are now over a dozen such implementations, written by many people.

\subsection{Trusting the Mathematician}\label{trust}

\begin{quote}
  {\em There is no Algebraist nor Mathematician so expert in his science, as
  to place entire confidence in any truth immediately upon his discovery of it,
  or regard it as any thing, but a mere probability.  Every time he runs over
  his proofs, his confidence encreases; but still more by the approbation of
  his friends; and is rais'd to its utmost perfection by the universal assent
  and applauses of the learned world.}
  \flushright\sc David Hume\footnote{{\em A Treatise of Human Nature}, as
  quoted in \cite{deMillo}, p.~267.}\\
\end{quote}\index{Hume, David}

\begin{quote}
  {\em Stanislaw Ulam estimates that mathematicians publish 200,000 theorems
  every year.  A number of these are subsequently contradicted or otherwise
  disallowed, others are thrown into doubt, and most are ignored.}
  \flushright\sc Richard de Millo et. al.\footnote{\cite{deMillo}, p.~269.}\\
\end{quote}\index{Ulam, Stanislaw}

Whether or not the computer can be trusted, humans  of course will occasionally
err. Only the most memorable proofs get independently verified, and of these
only a handful of truly great ones achieve the status of being ``known''
mathematical truths that are used without giving a second thought to their
correctness.

There are many famous examples of incorrect theorems and proofs in
mathematical literature.\index{errors in proofs}

\begin{itemize}
\item There have been thousands of purported proofs of Fermat's Last
Theorem\index{Fermat's Last Theorem} (``no integer solutions exist to $x^n +
y^n = z^n$ for $n > 2$''), by amateurs, cranks, and well-regarded
mathematicians \cite[p.~5]{Stark}\index{Stark, Harold M}.  Fermat wrote a note
in his copy of Bachet's {\em Diophantus} that he found ``a truly marvelous
proof of this theorem but this margin is too narrow to contain it''
\cite[p.~507]{Kramer}.  A recent, much publicized proof by Yoichi
Miyaoka\index{Miyaoka, Yoichi} was shown to be incorrect ({\em Science News},
April 9, 1988, p.~230).  The theorem was finally proved by Andrew
Wiles\index{Wiles, Andrew} ({\em Science News}, July 3, 1993, p.~5), but it
initially had some gaps and took over a year after its announcement to be
checked thoroughly by experts.  On Oct. 25, 1994, Wiles announced that the last
gap found in his proof had been filled in.
  \item In 1882, M. Pasch discovered that an axiom was omitted from Euclid's
formulation of geometry\index{Euclidean geometry}; without it, the proofs of
important theorems of Euclid are not valid.  Pasch's axiom\index{Pasch's
axiom} states that a line that intersects one side of a triangle must also
intersect another side, provided that it does not touch any of the triangle's
vertices.  The omission of Pasch's axiom went unnoticed for 2000
years \cite[p.~160]{Davis}, in spite of (one presumes) the thousands of
students, instructors, and mathematicians who studied Euclid.
  \item The first published proof of the famous Schr\"{o}der--Bernstein
theorem\index{Schr\"{o}der--Bernstein theorem} in set theory was incorrect
\cite[p.~148]{Enderton}\index{Enderton, Herbert B.}.  This theorem states
that if there exists a 1-to-1 function\footnote{A {\em set}\index{set} is any
collection of objects. A {\em function}\index{function} or {\em
mapping}\index{mapping} is a rule that assigns to each element of one set
(called the function's {\em domain}\index{domain}) an element from another
set.} from set $A$ into set $B$ and vice-versa, then sets $A$ and $B$ have
a 1-to-1 correspondence.  Although it sounds simple and obvious,
the standard proof is quite long and complex.
  \item In the early 1900's, Hilbert\index{Hilbert, David} published a
purported proof of the continuum hypothesis\index{continuum hypothesis}, which
was eventually established as unprovable by Cohen\index{Cohen, Paul} in 1963
\cite[p.~166]{Enderton}.  The continuum hypothesis states that no
infinity\index{infinity} (``transfinite cardinal number'')\index{cardinal,
transfinite} exists whose size (or ``cardinality''\index{cardinality}) is
between the size of the set of integers and the size of the set of real
numbers.  This hypothesis originated with German mathematician Georg
Cantor\index{Cantor, Georg} in the late 1800's, and his inability to prove it
is said to have contributed to mental illness that afflicted him in his later
years.
  \item An incorrect proof of the four-color theorem\index{four-color theorem}
was published by Kempe\index{Kempe, A. B.} in 1879
\cite[p.~582]{Courant}\index{Courant, Richard}; it stood for 11 years before
its flaw was discovered.  This theorem states that any map can be colored
using only four colors, so that no two adjacent countries have the same
color.  In 1976 the theorem was finally proved by the famous computer-assisted
proof of Haken, Appel, and Koch \cite{Swart}\index{Appel, K.}\index{Haken,
W.}\index{Koch, K.}.  Or at least it seems that way.  Mathematician
H.~S.~M.~Coxeter\index{Coxeter, H. S. M.} has doubts \cite[p.~58]{Davis}:  ``I
have a feeling that that is an untidy kind of use of the computers, and the more
you correspond with Haken and Appel, the more shaky you seem to be.''
  \item Many false ``proofs'' of the Poincar\'{e}
conjecture\index{Poincar\'{e} conjecture} have been proposed over the years.
This conjecture states that any object that mathematically behaves like a
three-dimensional sphere is a three-dimensional sphere topologically,
regardless of how it is distorted.  In March 1986, mathematicians Colin
Rourke\index{Rourke, Colin} and Eduardo R\^{e}go\index{R\^{e}go, Eduardo}
caused  a stir in the mathematical community by announcing that they had found
a proof; in November of that year the proof was found to be false \cite[p.
218]{PetersonI}.  It was finally proved in 2003 by Grigory Perelman
\label{poincare}\index{Szpiro, George}\index{Perelman, Grigory}\cite{Szpiro}.
 \end{itemize}

Many counterexamples to ``theorems'' in recent mathematical
literature related to Clifford algebras\index{Clifford algebras}
 have been found by Pertti
Lounesto (who passed away in 2002).\index{Lounesto, Pertti}
See the web page \url{http://mathforum.org/library/view/4933.html}.
% http://users.tkk.fi/~ppuska/mirror/Lounesto/counterexamples.htm

One of the purposes of Metamath\index{Metamath} is to allow proofs to be
expressed with absolute precision.  Developing a proof in the Metamath
language can be challenging, because Metamath will not permit even the
tiniest mistake.\index{errors in proofs}  But once the proof is created, its
correctness can be trusted immediately, without having to depend on the
process of peer review for confirmation.

\section{The Use of Computers in Mathematics}

\subsection{Computer Algebra Systems}

For the most part, you will find that Metamath\index{Metamath} is not a
practical tool for manipulating numbers.  (Even proving that $2 + 2 = 4$, if
you start with set theory, can be quite complex!)  Several commercial
mathematics packages are quite good at arithmetic, algebra, and calculus, and
as practical tools they are invaluable.\index{computer algebra system} But
they have no notion of proof, and cannot understand statements starting with
``there exists such and such...''.

Software packages such as Mathematica \cite{Wolfram}\index{Mathematica} do not
concern themselves with proofs but instead work directly with known results.
These packages primarily emphasize heuristic rules such as the substitution of
equals for equals to achieve simpler expressions or expressions in a different
form.  Starting with a rich collection of built-in rules and algorithms, users
can add to the collection by means of a powerful programming language.
However, results such as, say, the existence of a certain abstract object
without displaying the actual object cannot be expressed (directly) in their
languages.  The idea of a proof from a small set of axioms is absent.  Instead
this software simply assumes that each fact or rule you add to the built-in
collection of algorithms is valid.  One way to view the software is as a large
collection of axioms from which the software, with certain goals, attempts to
derive new theorems, for example equating a complex expression with a simpler
equivalent. But the terms ``theorem''\index{theorem} and
``proof,''\index{proof} for example, are not even mentioned in the index of
the user's manual for Mathematica.\index{Mathematica and proofs}  What is also
unsatisfactory from a philosophical point of view is that there is no way to
ensure the validity of the results other than by trusting the writer of each
application module or tediously checking each module by hand, similar to
checking a computer program for bugs.\index{computer program
bugs}\footnote{Two examples illustrate why the knowledge database of computer
algebra systems should sometimes be regarded with a certain caution.  If you
ask Mathematica (version 3.0) to \texttt{Solve[x\^{ }n + y\^{ }n == z\^{ }n , n]}
it will respond with \texttt{\{\{n-\char`\>-2\}, \{n-\char`\>-1\},
\{n-\char`\>1\}, \{n-\char`\>2\}\}}. In other words, Mathematica seems to
``know'' that Fermat's Last Theorem\index{Fermat's Last Theorem} is true!  (At
the time this version of Mathematica was released this fact was unknown.)  If
you ask Maple\index{Maple} to \texttt{solve(x\^{ }2 = 2\^{ }x)} then
\texttt{simplify(\{"\})}, it returns the solution set \texttt{\{2, 4\}}, apparently
unaware that $-0.7666647$\ldots is also a solution.} While of course extremely
valuable in applied mathematics, computer algebra systems tend to be of little
interest to the theoretical mathematician except as aids for exploring certain
specific problems.

Because of possible bugs, trusting the output of a computer algebra system for
use as theorems in a proof-verifier would defeat the latter's goal of rigor.
On the other hand, a fact such that a certain relatively large number is
prime, while easy for a computer algebra system to derive, might have a long,
tedious proof that could overwhelm a proof-verifier. One approach for linking
computer algebra systems to a proof-verifier while retaining the advantages of
both is to add a hypothesis to each such theorem indicating its source.  For
example, a constant {\sc maple} could indicate the theorem came from the Maple
package, and would mean ``assuming Maple is consistent, then\ldots''  This and
many other topics concerning the formalization of mathematics are discussed in
John Harrison's\index{Harrison, John} very interesting
PhD thesis~\cite{Harrison-thesis}.

\subsection{Automated Theorem Provers}\label{theoremprovers}

A mathematical theory is ``decidable''\index{decidable theory} if a mechanical
method or algorithm exists that is guaranteed to determine whether or not a
particular formula is a theorem.  Among the few theories that are decidable is
elementary geometry,\index{Euclidean geometry} as was shown by a classic
result of logician Alfred Tarski\index{Tarski, Alfred} in 1948
\cite{Tarski}.\footnote{Tarski's result actually applies to a subset of the
geometry discussed in elementary textbooks.  This subset includes most of what
would be considered elementary geometry but it is not powerful enough to
express, among other things, the notions of the circumference and area of a
circle.  Extending the theory in a way that includes notions such as these
makes the theory undecidable, as was also shown by Tarski.  Tarski's algorithm
is far too inefficient to implement practically on a computer.  A practical
algorithm for proving a smaller subset of geometry theorems (those not
involving concepts of ``order'' or ``continuity'') was discovered by Chinese
mathematician Wu Wen-ts\"{u}n in 1977 \cite{Chou}\index{Chou,
Shang-Ching}.}\index{Wen-ts{\"{u}}n, Wu}  But most theories, including
elementary arithmetic, are undecidable.  This fact contributes to keeping
mathematics alive and well, since many mathematicians believe
that they will never be
replaced by computers (if they believe Roger Penrose's argument that a
computer can never replace the brain \cite{Penrose}\index{Penrose, Roger}).
In fact,  elementary geometry is often considered a ``dead'' field
for the simple reason that it is decidable.

On the other hand, the undecidability of a theory does not mean that one cannot
use a computer to search for proofs, providing one is willing to give up if a
proof is not found after a reasonable amount of time.  The field of automated
theorem proving\index{automated theorem proving} specializes in pursuing such
computer searches.  Among the more successful results to date are those based
on an algorithm known as Robinson's resolution principle
\cite{Robinson}\index{Robinson's resolution principle}.

Automated theorem provers can be excellent tools for those willing to learn
how to use them.  But they are not widely used in mainstream pure
mathematics, even though they could probably be useful in many areas.  There
are several reasons for this.  Probably most important, the main goal in pure
mathematics is to arrive at results that are considered to be deep or
important; proving them is essential but secondary.  Usually, an automated
theorem prover cannot assist in this main goal, and by the time the main goal
is achieved, the mathematician may have already figured out the proof as a
by-product.  There is also a notational problem.  Mathematicians are used to
using very compact syntax where one or two symbols (heavily dependent on
context) can represent very complex concepts; this is part of the
hierarchy\index{hierarchy} they have built up to tackle difficult problems.  A
theorem prover on the other hand might require that a theorem be expressed in
``first-order logic,''\index{first-order logic} which is the logic on which
most of mathematics is ultimately based but which is not ordinarily used
directly because expressions can become very long.  Some automated theorem
provers are experimental programs, limited in their use to very specialized
areas, and the goal of many is simply research into the nature of automated
theorem proving itself.  Finally, much research remains to be done to enable
them to prove very deep theorems.  One significant result was a
computer proof by Larry Wos\index{Wos, Larry} and colleagues that every Robbins
algebra\index{Robbins algebra} is a Boolean  algebra\index{Boolean algebra}
({\em New York Times}, Dec. 10, 1996).\footnote{In 1933, E.~V.\
Huntington\index{Huntington, E. V.}
presented the following axiom system for
Boolean algebra with a unary operation $n$ and a binary operation $+$:
\begin{center}
    $x + y = y + x$ \\
    $(x + y) + z = x + (y + z)$ \\
    $n(n(x) + y) + n(n(x) + n(y)) = x$
\end{center}
Herbert Robbins\index{Robbins, Herbert}, a student of Huntington, conjectured
that the last equation can be replaced with a simpler one:
\begin{center}
    $n(n(x + y) + n(x + n(y))) = x$
\end{center}
Robbins and Huntington could not find a proof.  The problem was
later studied unsuccessfully by Tarski\index{Tarski, Alfred} and his
students, and it remained an unsolved problem until a
computer found the proof in 1996.  For more information on
the Robbins algebra problem see \cite{Wos}.}

How does Metamath\index{Metamath} relate to automated theorem provers?  A
theorem prover is primarily concerned with one theorem at a time (perhaps
tapping into a small database of known theorems) whereas Metamath is more like
a theorem archiving system, storing both the theorem and its proof in a
database for access and verification.  Metamath is one answer to ``what do you
do with the output of a theorem prover?''  and could be viewed as the
next step in the process.  Automated theorem provers could be useful tools for
helping develop its database.
Note that very long, automatically
generated proofs can make your database fat and ugly and cause Metamath's proof
verification to take a long time to run.  Unless you have a particularly good
program that generates very concise proofs, it might be best to consider the
use of automatically generated proofs as a quick-and-dirty approach, to be
manually rewritten at some later date.

The program {\sc otter}\index{otter@{\sc otter}}\footnote{\url{http://www.cs.unm.edu/\~mccune/otter/}.}, later succeeded by
prover9\index{prover9}\footnote{\url{https://www.cs.unm.edu/~mccune/mace4/}.},
have been historically influential.
The E prover\index{E prover}\footnote{\url{https://github.com/eprover/eprover}.}
is a maintained automated theorem prover
for full first-order logic with equality.
There are many other automated theorem provers as well.

If you want to combine automated theorem provers with Metamath
consider investigating
the book {\em Automated Reasoning:  Introduction and Applications}
\cite{Wos}\index{Wos, Larry}.  This book discusses
how to use {\sc otter} in a way that can
not only able to generate
relatively efficient proofs, it can even be instructed to search for
shorter proofs.  The effective use of {\sc otter} (and similar tools)
does require a certain
amount of experience, skill, and patience.  The axiom system used in the
\texttt{set.mm}\index{set theory database (\texttt{set.mm})} set theory
database can be expressed to {\sc otter} using a method described in
\cite{Megill}.\index{Megill, Norman}\footnote{To use those axioms with
{\sc otter}, they must be restated in a way that eliminates the need for
``dummy variables.''\index{dummy variable!eliminating} See the Comment
on p.~\pageref{nodd}.} When successful, this method tends to generate
short and clever proofs, but my experiments with it indicate that the
method will find proofs within a reasonable time only for relatively
easy theorems.  It is still fun to experiment with.

Reference \cite{Bledsoe}\index{Bledsoe, W. W.} surveys a number of approaches
people have explored in the field of automated theorem proving\index{automated
theorem proving}.

\subsection{Interactive Theorem Provers}\label{interactivetheoremprovers}

Finding proofs completely automatically is difficult, so there
are some interactive theorem provers that allow a human to guide the
computer to find a proof.
Examples include
HOL Light\index{HOL light}%
\footnote{\url{https://www.cl.cam.ac.uk/~jrh13/hol-light/}.},
Isabelle\index{Isabelle}%
\footnote{\url{http://www.cl.cam.ac.uk/Research/HVG/Isabelle}.},
{\sc hol}\index{hol@{\sc hol}}%
\footnote{\url{https://hol-theorem-prover.org/}.},
and
Coq\index{Coq}\footnote{\url{https://coq.inria.fr/}.}.

A major difference between most of these tools and Metamath is that the
``proofs'' are actually programs that guide the program to find a proof,
and not the proof itself.
For example, an Isabelle/HOL proof might apply a step
\texttt{apply (blast dest: rearrange reduction)}. The \texttt{blast}
instruction applies
an automatic tableux prover and returns if it found a sequence of proof
steps that work... but the sequence is not considered part of the proof.

A good overview of
higher-level proof verification languages (such as {\sc lcf}\index{lcf@{\sc
lcf}} and {\sc hol}\index{hol@{\sc hol}})
is given in \cite{Harrison}.  All of these languages are fundamentally
different from Metamath in that much of the mathematical foundational
knowledge is embedded in the underlying proof-verification program, rather
than placed directly in the database that is being verified.
These can have a steep learning curve for those without a mathematical
background.  For example, one usually must have a fair understanding of
mathematical logic in order to follow their proofs.

\subsection{Proof Verifiers}\label{proofverifiers}

A proof verifier is a program that doesn't generate proofs but instead
verifies proofs that you give it.  Many proof verifiers have limited built-in
automated proof capabilities, such as figuring out simple logical inferences
(while still being guided by a person who provides the overall proof).  Metamath
has no built-in automated proof capability other than the limited
capability of its Proof Assistant.

Proof-verification languages are not used as frequently as they might be.
Pure mathematicians are more concerned with producing new results, and such
detail and rigor would interfere with that goal.  The use of computers in pure
mathematics is primarily focused on automated theorem provers (not verifiers),
again with the ultimate goal of aiding the creation of new mathematics.
Automated theorem provers are usually concerned with attacking one theorem at
time rather than making a large, organized database easily available to the
user.  Metamath is one way to help close this gap.

By itself Metamath is a mostly a proof verifier.
This does not mean that other approaches can't be used; the difference
is that in Metamath, the results of various provers must be recorded
step-by-step so that they can be verified.

Another proof-verification language is Mizar,\index{Mizar} which can display
its proofs in the informal language that mathematicians are accustomed to.
Information on the Mizar language is available at \url{http://mizar.org}.

For the working mathematician, Mizar is an excellent tool for rigorously
documenting proofs. Mizar typesets its proofs in the informal English used by
mathematicians (and, while fine for them, are just as inscrutable by
laypersons!). A price paid for Mizar is a relatively steep learning curve of a
couple of weeks.  Several mathematicians are actively formalizing different
areas of mathematics using Mizar and publishing the proofs in a dedicated
journal. Unfortunately the task of formalizing mathematics is still looked
down upon to a certain extent since it doesn't involve the creation of ``new''
mathematics.

The closest system to Metamath is
the {\em Ghilbert}\index{Ghilbert} proof language (\url{http://ghilbert.org})
system developed by
Raph Levien\index{Levien, Raph}.
Ghilbert is a formal proof checker heavily inspired by Metamath.
Ghilbert statements are s-expressions (a la Lisp), which is easy
for computers to parse but many people find them hard to read.
There are a number of differences in their specific constructs, but
there is at least one tool to translate some Metamath materials into Ghilbert.
As of 2019 the Ghilbert community is smaller and less active than the
Metamath community.
That said, the Metamath and Ghilbert communities overlap, and fruitful
conversations between them have occurred many times over the years.

\subsection{Creating a Database of Formalized Mathematics}\label{mathdatabase}

Besides Metamath, there are several other ongoing projects with the goal of
formalizing mathematics into computer-verifiable databases.
Understanding some history will help.

The {\sc qed}\index{qed project@{\sc qed} project}%
\footnote{\url{http://www-unix.mcs.anl.gov/qed}.}
project arose in 1993 and its goals were outlined in the
{\sc qed} manifesto.
The {\sc qed} manifesto was
a proposal for a computer-based database of all mathematical knowledge,
strictly formalized and with all proofs having been checked automatically.
The project had a conference in 1994 and another in 1995;
there was also a ``twenty years of the {\sc qed} manifesto'' workshop
in 2014.
Its ideals are regularly reraised.

In a 2007 paper, Freek Wiedijk identified two reasons
for the failure of the {\sc qed} project as originally envisioned:%
\cite{Wiedijk-revisited}\index{Wiedijk, Freek}

\begin{itemize}
\item Very few people are working on formalization of mathematics. There is no compelling application for fully mechanized mathematics.
\item Formalized mathematics does not yet resemble traditional mathematics. This is partly due to the complexity of mathematical notation, and partly to the limitations of existing theorem provers and proof assistants.
\end{itemize}

But this did not end the dream of
formalizing mathematics into computer-verifiable databases.
The problems that led to the {\sc qed} manifesto are still with us,
even though the challenges were harder than originally considered.
What has happened instead is that various independent projects have
worked towards formalizing mathematics into computer-verifiable databases,
each simultaneously competing and cooperating with each other.

A concrete way to see this is
Freek Wiedijk's ``Formalizing 100 Theorems'' list%
\footnote{\url{http://www.cs.ru.nl/\%7Efreek/100/}.}
which shows the progress different systems have made on a challenge list
of 100 mathematical theorems.%
\footnote{ This is not the only list of ``interesting'' theorems.
Another interesting list was posted by Oliver Knill's list
\cite{Knill}\index{Knill, Oliver}.}
The top systems as of February 2019
(in order of the number of challenges completed) are
HOL Light, Isabelle, Metamath, Coq, and Mizar.

The Metamath 100%
\footnote{\url{http://us.metamath.org/mm\_100.html}}
page (maintained by David A. Wheeler\index{Wheeler, David A.})
shows the progress of Metamath (specifically its \texttt{set.mm} database)
against this challenge list maintained by Freek Wiedijk.
The Metamath \texttt{set.mm} database
has made a lot of progress over the years,
in part because working to prove those challenge theorems required
defining various terms and proving their properties as a prerequisite.
Here are just a few of the many statements that have been
formally proven with Metamath:

% The entries of this cause the narrow display to break poorly,
% since the short amount of text means LaTeX doesn't get a lot to work with
% and the itemize format gives it even *less* margin than usual.
% No one will mind if we make just this list flushleft, since this list
% will be internally consistent.
\begin{flushleft}
\begin{itemize}
\item 1. The Irrationality of the Square Root of 2
  (\texttt{sqr2irr}, by Norman Megill, 2001-08-20)
\item 2. The Fundamental Theorem of Algebra
  (\texttt{fta}, by Mario Carneiro, 2014-09-15)
\item 22. The Non-Denumerability of the Continuum
  (\texttt{ruc}, by Norman Megill, 2004-08-13)
\item 54. The Konigsberg Bridge Problem
  (\texttt{konigsberg}, by Mario Carneiro, 2015-04-16)
\item 83. The Friendship Theorem
  (\texttt{friendship}, by Alexander W. van der Vekens, 2018-10-09)
\end{itemize}
\end{flushleft}

We thank all of those who have developed at least one of the Metamath 100
proofs, and we particularly thank
Mario Carneiro\index{Carneiro, Mario}
who has contributed the most Metamath 100 proofs as of 2019.
The Metamath 100 page shows the list of all people who have contributed a
proof, and links to graphs and charts showing progress over time.
We encourage others to work on proving theorems not yet proven in Metamath,
since doing so improves the work as a whole.

Each of the math formalization systems (including Metamath)
has different strengths and weaknesses, depending on what you value.
Key aspects that differentiate Metamath from the other top systems are:

\begin{itemize}
\item Metamath is not tied to any particular set of axioms.
\item Metamath can show every step of every proof, no exceptions.
  Most other provers only assert that a proof can be found, and do not
  show every step. This also makes verification fast, because
  the system does not need to rediscover proof details.
\item The Metamath verifier has been re-implemented in many different
  programming languages, so verification can be done by multiple
  implementations.  In particular, the
  \texttt{set.mm}\index{set theory database (\texttt{set.mm})}%
  \index{Metamath Proof Explorer} database is verified by
  four different verifiers
  written in four different languages by four different authors.
  This greatly reduces the risk of accepting an invalid
  proof due to an error in the verifier.
\item Proofs stay proven.  In some systems, changes to the system's
  syntax or how a tactic works causes proofs to fail in later versions,
  causing older work to become essentially lost.
  Metamath's language is
  extremely small and fixed, so once a proof is added to a database,
  the database can be rechecked with later versions of the Metamath program
  and with other verifiers of Metamath databases.
  If an axiom or key definition needs to be changed, it is easy to
  manipulate the database as a whole to handle the change
  without touching the underlying verifier.
  Since re-verification of an entire database takes seconds, there
  is never a reason to delay complete verification.
  This aspect is especially compelling if your
  goal is to have a long-term database of proofs.
\item Licensing is generous.  The main Metamath databases are released to
  the public domain, and the main Metamath program is open source software
  under a standard, widely-used license.
\item Substitutions are easy to understand, even by those who are not
  professional mathematicians.
\end{itemize}

Of course, other systems may have advantages over Metamath
that are more compelling, depending on what you value.
In any case, we hope this helps you understand Metamath
within a wider context.

\subsection{In Summary}\label{computers-summary}

To summarize our discussions of computers and mathematics, computer algebra
systems can be viewed as theorem generators focusing on a narrow realm of
mathematics (numbers and their properties), automated theorem provers as proof
generators for specific theorems in a much broader realm covered by a built-in
formal system such as first-order logic, interactive theorem
provers require human guidance, proof verifiers verify proofs but
historically they have been
restricted to first-order logic.
Metamath, in contrast,
is a proof verifier and documenter whose realm is essentially unlimited.

\section{Mathematics and Metamath}

\subsection{Standard Mathematics}

There are a number of ways that Metamath\index{Metamath} can be used with
standard mathematics.  The most satisfying way philosophically is to start at
the very beginning, and develop the desired mathematics from the axioms of
logic and set theory.\index{set theory}  This is the approach taken in the
\texttt{set.mm}\index{set theory database (\texttt{set.mm})}%
\index{Metamath Proof Explorer}
database (also known as the Metamath Proof Explorer).
Among other things, this database builds up to the
axioms of real and complex numbers\index{analysis}\index{real and complex
numbers} (see Section~\ref{real}), and a standard development of analysis, for
example, could start at that point, using it as a basis.   Besides this
philosophical advantage, there are practical advantages to having all of the
tools of set theory available in the supporting infrastructure.

On the other hand, you may wish to start with the standard axioms of a
mathematical theory without going through the set theoretical proofs of those
axioms.  You will need mathematical logic to make inferences, but if you wish
you can simply introduce theorems\index{theorem} of logic as
``axioms''\index{axiom} wherever you need them, with the implicit assumption
that in principle they can be proved, if they are obvious to you.  If you
choose this approach, you will probably want to review the notation used in
\texttt{set.mm}\index{set theory database (\texttt{set.mm})} so that your own
notation will be consistent with it.

\subsection{Other Formal Systems}
\index{formal system}

Unlike some programs, Metamath\index{Metamath} is not limited to any specific
area of mathematics, nor committed to any particular mathematical philosophy
such as classical logic versus intuitionism, nor limited, say, to expressions
in first-order logic.  Although the database \texttt{set.mm}
describes standard logic and set theory, Meta\-math
is actually a general-purpose language for describing a wide variety of formal
systems.\index{formal system}  Non-standard systems such as modal
logic,\index{modal logic} intuitionist logic\index{intuitionism}, higher-order
logic\index{higher-order logic}, quantum logic\index{quantum logic}, and
category theory\index{category theory} can all be described with the Metamath
language.  You define the symbols you prefer and tell Metamath the axioms and
rules you want to start from, and Metamath will verify any inferences you make
from those axioms and rules.  A simple example of a non-standard formal system
is Hofstadter's\index{Hofstadter, Douglas R.} MIU system,\index{MIU-system}
whose Metamath description is presented in Appendix~\ref{MIU}.

This is not hypothetical.
The largest Metamath database is
\texttt{set.mm}\index{set theory database (\texttt{set.mm}}%
\index{Metamath Proof Explorer}), aka the Metamath Proof Explorer,
which uses the most common axioms for mathematical foundations
(specifically classical logic combined with Zermelo--Fraenkel
set theory\index{Zermelo--Fraenkel set theory} with the Axiom of Choice).
But other Metamath databases are available:

\begin{itemize}
\item The database
  \texttt{iset.mm}\index{intuitionistic logic database (\texttt{iset.mm})},
  aka the
  Intuitionistic Logic Explorer\index{Intuitionistic Logic Explorer},
  uses intuitionistic logic (a constructivist point of view)
  instead of classical logic.
\item The database
  \texttt{nf.mm}\index{New Foundations database (\texttt{nf.mm})},
  aka the
  New Foundations Explorer\index{New Foundations Explorer},
  constructs mathematics from scratch,
  starting from Quine's New Foundations (NF) set theory axioms.
\item The database
  \texttt{hol.mm}\index{Higher-order Logic database (\texttt{hol.mm})},
  aka the
  Higher-Order Logic (HOL) Explorer\index{Higher-Order Logic (HOL) Explorer},
  starts with HOL (also called simple type theory) and derives
  equivalents to ZFC axioms, connecting the two approaches.
\end{itemize}

Since the days of David Hilbert,\index{Hilbert, David} mathematicians have
been concerned with the fact that the metalanguage\index{metalanguage} used to
describe mathematics may be stronger than the mathematics being described.
Metamath\index{Metamath}'s underlying finitary\index{finitary proof},
constructive nature provides a good philosophical basis for studying even the
weakest logics.\index{weak logic}

The usual treatment of many non-standard formal systems\index{formal
system} uses model theory\index{model theory} or proof theory\index{proof
theory} to describe these systems; these theories, in turn, are based on
standard set theory.  In other words, a non-standard formal system is defined
as a set with certain properties, and standard set theory is used to derive
additional properties of this set.  The standard set theory database provided
with Metamath can be used for this purpose, and when used this way
the development of a special
axiom system for the non-standard formal system becomes unnecessary.  The
model- or proof-theoretic approach often allows you to prove much deeper
results with less effort.

Metamath supports both approaches.  You can define the non-standard
formal system directly, or define the non-standard formal system as
a set with certain properties, whichever you find most helpful.

%\section{Additional Remarks}

\subsection{Metamath and Its Philosophy}

Closely related to Metamath\index{Metamath} is a philosophy or way of looking
at mathematics. This philosophy is related to the formalist
philosophy\index{formalism} of Hilbert\index{Hilbert, David} and his followers
\cite[pp.~1203--1208]{Kline}\index{Kline, Morris}
\cite[p.~6]{Behnke}\index{Behnke, H.}. In this philosophy, mathematics is
viewed as nothing more than a set of rules that manipulate symbols, together
with the consequences of those rules.  While the mathematics being described
may be complex, the rules used to describe it (the
``metamathematics''\index{metamathematics}) should be as simple as possible.
In particular, proofs should be restricted to dealing with concrete objects
(the symbols we write on paper rather than the abstract concepts they
represent) in a constructive manner; these are called ``finitary''
proofs\index{finitary proof} \cite[pp.~2--3]{Shoenfield}\index{Shoenfield,
Joseph R.}.

Whether or not you find Metamath interesting or useful will in part depend on
the appeal you find in its philosophy, and this appeal will probably depend on
your particular goals with respect to mathematics.  For example, if you are a
pure mathematician at the forefront of discovering new mathematical knowledge,
you will probably find that the rigid formality of Metamath stifles your
creativity.  On the other hand, we would argue that once this knowledge is
discovered, there are advantages to documenting it in a standard format that
will make it accessible to others.  Sixty years from now, your field may be
dormant, and as Davis and Hersh put it, your ``writings would become less
translatable than those of the Maya'' \cite[p.~37]{Davis}\index{Davis, Phillip
J.}.


\subsection{A History of the Approach Behind Metamath}

Probably the one work that has had the most motivating influence on
Metamath\index{Metamath} is Whitehead and Russell's monumental {\em Principia
Mathematica} \cite{PM}\index{Whitehead, Alfred North}\index{Russell,
Bertrand}\index{principia mathematica@{\em Principia Mathematica}}, whose aim
was to deduce all of mathematics from a small number of primitive ideas, in a
very explicit way that in principle anyone could understand and follow.  While
this work was tremendously influential in its time, from a modern perspective
it suffers from several drawbacks.  Both its notation and its underlying
axioms are now considered dated and are no longer used.  From our point of
view, its development is not really as accessible as we would like to see; for
practical reasons, proofs become more and more sketchy as its mathematics
progresses, and working them out in fine detail requires a degree of
mathematical skill and patience that many people don't have.  There are
numerous small errors, which is understandable given the tedious, technical
nature of the proofs and the lack of a computer to verify the details.
However, even today {\em Principia Mathematica} stands out as the work closest
in spirit to Metamath.  It remains a mind-boggling work, and one can't help
but be amazed at seeing ``$1+1=2$'' finally appear on page 83 of Volume II
(Theorem *110.643).

The origin of the proof notation used by Metamath dates back to the 1950's,
when the logician C.~A.~Meredith expressed his proofs in a compact notation
called ``condensed detachment''\index{condensed detachment}
\cite{Hindley}\index{Hindley, J. Roger} \cite{Kalman}\index{Kalman, J. A.}
\cite{Meredith}\index{Meredith, C. A.} \cite{Peterson}\index{Peterson, Jeremy
George}.  This notation allows proofs to be communicated unambiguously by
merely referencing the axiom\index{axiom}, rule\index{rule}, or
theorem\index{theorem} used at each step, without explicitly indicating the
substitutions\index{substitution!variable}\index{variable substitution} that
have to be made to the variables in that axiom, rule, or theorem.  Ordinarily,
condensed detachment is more or less limited to propositional
calculus\index{propositional calculus}.  The concept has been extended to
first-order logic\index{first-order logic} in \cite{Megill}\index{Megill,
Norman}, making it is easy to write a small computer program to verify proofs
of simple first-order logic theorems.\index{condensed detachment!and
first-order logic}

A key concept behind the notation of condensed detachment is called
``unification,''\index{unification} which is an algorithm for determining what
substitutions\index{substitution!variable}\index{variable substitution} to
variables have to be made to make two expressions match each other.
Unification was first precisely defined by the logician J.~A.~Robinson, who
used it in the development of a powerful
theorem-proving technique called the ``resolution principle''
\cite{Robinson}\index{Robinson's resolution principle}. Metamath does not make
use of the resolution principle, which is intended for systems of first-order
logic.\index{first-order logic}  Metamath's use is not restricted to
first-order logic, and as we have mentioned it does not automatically discover
proofs.  However, unification is a key idea behind Metamath's proof
notation, and Metamath makes use of a very simple version of it
(Section~\ref{unify}).

\subsection{Metamath and First-Order Logic}

First-order logic\index{first-order logic} is the supporting structure
for standard mathematics.  On top of it is set theory, which contains
the axioms from which virtually all of mathematics can be derived---a
remarkable fact.\index{category
theory}\index{cardinal, inaccessible}\label{categoryth}\footnote{An exception seems
to be category theory.  There are several schools of thought on whether
category theory is derivable from set theory.  At a minimum, it appears
that an additional axiom is needed that asserts the existence of an
``inaccessible cardinal'' (a type of infinity so large that standard set
theory can't prove or deny that it exists).
%
%%%% (I took this out that was in previous editions:)
% But it is also argued that not just one but a ``proper class'' of them
% is needed, and the existence of proper classes is impossible in standard
% set theory.  (A proper class is a collection of sets so huge that no set
% can contain it as an element.  Proper classes can lead to
% inconsistencies such as ``Russell's paradox.''  The axioms of standard
% set theory are devised so as to deny the existence of proper classes.)
%
For more information, see
\cite[pp.~328--331]{Herrlich}\index{Herrlich, Horst} and
\cite{Blass}\index{Blass, Andrea}.}

One of the things that makes Metamath\index{Metamath} more practical for
first-order theories is a set of axioms for first-order logic designed
specifically with Metamath's approach in mind.  These are included in
the database \texttt{set.mm}\index{set theory database (\texttt{set.mm})}.
See Chapter~\ref{fol} for a detailed
description; the axioms are shown in Section~\ref{metaaxioms}.  While
logically equivalent to standard axiom systems, our axiom system breaks
up the standard axioms into smaller pieces such that from them, you can
directly derive what in other systems can only be derived as higher-level
``metatheorems.''\index{metatheorem}  In other words, it is more powerful than
the standard axioms from a metalogical point of view.  A rigorous
justification for this system and its ``metalogical
completeness''\index{metalogical completeness} is found in
\cite{Megill}\index{Megill, Norman}.  The system is closely related to a
system developed by Monk\index{Monk, J. Donald} and Tarski\index{Tarski,
Alfred} in 1965 \cite{Monks}.

For example, the formula $\exists x \, x = y $ (given $y$, there exists some
$x$ equal to it) is a theorem of logic,\footnote{Specifically, it is a theorem
of those systems of logic that assume non-empty domains.  It is not a theorem
of more general systems that include the empty domain\index{empty domain}, in
which nothing exists, period!  Such systems are called ``free
logics.''\index{free logic} For a discussion of these systems, see
\cite{Leblanc}\index{Leblanc, Hugues}.  Since our use for logic is as a basis
for set theory, which has a non-empty domain, it is more convenient (and more
traditional) to use a less general system.  An interesting curiosity is that,
using a free logic as a basis for Zermelo--Fraenkel set
theory\index{Zermelo--Fraenkel set theory} (with the redundant Axiom of the
Null Set omitted),\index{Axiom of the Null Set} we cannot even prove the
existence of a single set without assuming the axiom of infinity!\index{Axiom
of Infinity}} whether or not $x$ and $y$ are distinct variables\index{distinct
variables}.  In many systems of logic, we would have to prove two theorems to
arrive at this result.  First we would prove ``$\exists x \, x = x $,'' then
we would separately prove ``$\exists x \, x = y $, where $x$ and $y$ are
distinct variables.''  We would then combine these two special cases ``outside
of the system'' (i.e.\ in our heads) to be able to claim, ``$\exists x \, x =
y $, regardless of whether $x$ and $y$ are distinct.''  In other words, the
combination of the two special cases is a
metatheorem.  In the system of logic
used in Metamath's set theory\index{set theory database (\texttt{set.mm})}
database, the axioms of logic are broken down into small pieces that allow
them to be reassembled in such a way that theorems such as these can be proved
directly.

Breaking down the axioms in this way makes them look peculiar and not very
intuitive at first, but rest assured that they are correct and complete.  Their
correctness is ensured because they are theorem schemes of standard first-order
logic (which you can easily verify if you are a logician).  Their completeness
follows from the fact that we explicitly derive the standard axioms of
first-order logic as theorems.  Deriving the standard axioms is somewhat
tricky, but once we're there, we have at our disposal a system that is less
awkward to work with in formal proofs\index{formal proof}.  In technical terms
that logicians understand, we eliminate the cumbersome concepts of ``free
variable,''\index{free variable} ``bound variable,''\index{bound variable} and
``proper substitution''\index{proper substitution}\index{substitution!proper}
as primitive notions.  These concepts are present in our system but are
defined in terms of concepts expressed by the axioms and can be eliminated in
principle.  In standard systems, these concepts are really like additional,
implicit axioms\index{implicit axiom} that are somewhat complex and cannot be
eliminated.

The traditional approach to logic, wherein free variables and proper
substitution is defined, is also possible to do directly in the Metamath
language.  However, the notation tends to become awkward, and there are
disadvantages:  for example, extending the definition of a wff with a
definition is awkward, because the free variable and proper substitution
concepts have to have their definitions also extended.  Our choice of
axioms for \texttt{set.mm} is to a certain extent a matter of style, in
an attempt to achieve overall simplicity, but you should also be aware
that the traditional approach is possible as well if you should choose
it.

\chapter{Using the Metamath Program}
\label{using}

\section{Installation}

The way that you install Metamath\index{Metamath!installation} on your
computer system will vary for different computers.  Current
instructions are provided with the Metamath program download at
\url{http://metamath.org}.  In general, the installation is simple.
There is one file containing the Metamath program itself.  This file is
usually called \texttt{metamath} or \texttt{metamath.}{\em xxx} where
{\em xxx} is the convention (such as \texttt{exe}) for an executable
program on your operating system.  There are several additional files
containing samples of the Metamath language, all ending with
\texttt{.mm}.  The file \texttt{set.mm}\index{set theory database
(\texttt{set.mm})} contains logic and set theory and can be used as a
starting point for other areas of mathematics.

You will also need a text editor\index{text editor} capable of editing plain
{\sc ascii}\footnote{American Standard Code for Information Interchange.} text
in order to prepare your input files.\index{ascii@{\sc ascii}}  Most computers
have this capability built in.  Note that plain text is not necessarily the
default for some word processing programs\index{word processor}, especially if
they can handle different fonts; for example, with Microsoft Word\index{Word
(Microsoft)}, you must save the file in the format ``Text Only With Line
Breaks'' to get a plain text\index{plain text} file.\footnote{It is
recommended that all lines in a Metamath source file be 79 characters or less
in length for compatibility among different computer terminals.  When creating
a source file on an editor such as Word, select a monospaced
font\index{monospaced font} such as Courier\index{Courier font} or
Monaco\index{Monaco font} to make this easier to achieve.  Better yet,
just use a plain text editor such as Notepad.}

On some computer systems, Metamath does not have the capability to print
its output directly; instead, you send its output to a file (using the
\texttt{open} commands described later).  The way you print this output
file depends on your computer.\index{printers} Some computers have a
print command, whereas with others, you may have to read the file into
an editor and print it from there.

If you want to print your Metamath source files with typeset formulas
containing standard mathematical symbols, you will need the \LaTeX\
typesetting program\index{latex@{\LaTeX}}, which is widely and freely
available for most operating systems.  It runs natively on Unix and
Linux, and can be installed on Windows as part of the free Cygwin
package (\url{http://cygwin.com}).

You can also produce {\sc html}\footnote{HyperText Markup Language.}
web pages.  The {\tt help html} command in the Metamath program will
assist you with this feature.

\section{Your First Formal System}\label{start}
\subsection{From Nothing to Zero}\label{startf}

To give you a feel for what the Metamath\index{Metamath} language looks like,
we will take a look at a very simple example from formal number
theory\index{number theory}.  This example is taken from
Mendelson\index{Mendelson, Elliot} \cite[p. 123]{Mendelson}.\footnote{To keep
the example simple, we have changed the formalism slightly, and what we call
axioms\index{axiom} are strictly speaking theorems\index{theorem} in
\cite{Mendelson}.}  We will look at a small subset of this theory, namely that
part needed for the first number theory theorem proved in \cite{Mendelson}.

First we will look at a standard formal proof\index{formal proof} for the
example we have picked, then we will look at the Metamath version.  If you
have never been exposed to formal proofs, the notation may seem to be such
overkill to express such simple notions that you may wonder if you are missing
something.  You aren't.  The concepts involved are in fact very simple, and a
detailed breakdown in this fashion is necessary to express the proof in a way
that can be verified mechanically.  And as you will see, Metamath breaks the
proof down into even finer pieces so that the mechanical verification process
can be about as simple as possible.

Before we can introduce the axioms\index{axiom} of the theory, we must define
the syntax rules for forming legal expressions\index{syntax rules}
(combinations of symbols) with which those axioms can be used. The number 0 is
a {\bf term}\index{term}; and if $ t$ and $r$ are terms, so is $(t+r)$. Here,
$ t$ and $r$ are ``metavariables''\index{metavariable} ranging over terms; they
themselves do not appear as symbols in an actual term.  Some examples of
actual terms are $(0 + 0)$ and $((0+0)+0)$.  (Note that our theory describes
only the number zero and sums of zeroes.  Of course, not much can be done with
such a trivial theory, but remember that we have picked a very small subset of
complete number theory for our example.  The important thing for you to focus
on is our definitions that describe how symbols are combined to form valid
expressions, and not on the content or meaning of those expressions.) If $ t$
and $r$ are terms, an expression of the form $ t=r$ is a {\bf wff}
(well-formed formula)\index{well-formed formula (wff)}; and if $P$ and $Q$ are
wffs, so is $(P\rightarrow Q)$ (which means ``$P$ implies
$Q$''\index{implication ($\rightarrow$)} or ``if $P$ then $Q$'').
Here $P$ and $Q$ are metavariables ranging over wffs.  Examples of actual
wffs are $0=0$, $(0+0)=0$, $(0=0 \rightarrow (0+0)=0)$, and $(0=0\rightarrow
(0=0\rightarrow 0=(0+0)))$.  (Our notation makes use of more parentheses than
are customary, but the elimination of ambiguity this way simplifies our
example by avoiding the need to define operator precedence\index{operator
precedence}.)

The {\bf axioms}\index{axiom} of our theory are all wffs of the following
form, where $ t$, $r$, and $s$ are any terms:

%Latex p. 92
\renewcommand{\theequation}{A\arabic{equation}}

\begin{equation}
(t=r\rightarrow (t=s\rightarrow r=s))
\end{equation}
\begin{equation}
(t+0)=t
\end{equation}

Note that there are an infinite number of axioms since there are an infinite
number of possible terms.  A1 and A2 are properly called ``axiom
schemes,''\index{axiom scheme} but we will refer to them as ``axioms'' for
brevity.

An axiom is a {\bf theorem}; and if $P$ and $(P\rightarrow Q)$ are theorems
(where $P$ and $Q$ are wffs), then $Q$ is also a theorem.\index{theorem}  The
second part of this definition is called the modus ponens (MP) rule of
inference\index{inference rule}\index{modus ponens}.  It allows us to obtain
new theorems from old ones.

The {\bf proof}\index{proof} of a theorem is a sequence of one or more
theorems, each of which is either an axiom or the result of modus ponens
applied to two previous theorems in the sequence, and the last of which is the
theorem being proved.

The theorem we will prove for our example is very simple:  $ t=t$.  The proof of
our theorem follows.  Study it carefully until you feel sure you
understand it.\label{zeroproof}

% Use tabu so that lines will wrap automatically as needed.
\begin{tabu} { l X X }
1. & $(t+0)=t$ & (by axiom A2) \\
2. & $(t+0)=t$ & (by axiom A2) \\
3. & $((t+0)=t \rightarrow ((t+0)=t\rightarrow t=t))$ & (by axiom A1) \\
4. & $((t+0)=t\rightarrow t=t)$ & (by MP applied to steps 2 and 3) \\
5. & $t=t$ & (by MP applied to steps 1 and 4) \\
\end{tabu}

(You may wonder why step 1 is repeated twice.  This is not necessary in the
formal language we have defined, but in Metamath's ``reverse Polish
notation''\index{reverse Polish notation (RPN)} for proofs, a previous step
can be referred to only once.  The repetition of step~1 here will enable you
to see more clearly the correspondence of this proof with the
Metamath\index{Metamath} version on p.~\pageref{demoproof}.)

Our theorem is more properly called a ``theorem scheme,''\index{theorem
scheme} for it represents an infinite number of theorems, one for each
possible term $ t$.  Two examples of actual theorems would be $0=0$ and
$(0+0)=(0+0)$.  Rarely do we prove actual theorems, since by proving schemes
we can prove an infinite number of theorems in one fell swoop.  Similarly, our
proof should really be called a ``proof scheme.''\index{proof scheme}  To
obtain an actual proof, pick an actual term to use in place of $ t$, and
substitute it for $ t$ throughout the proof.

Let's discuss what we have done here.  The axioms\index{axiom} of our theory,
A1 and A2, are trivial and obvious.  Everyone knows that adding zero to
something doesn't change it, and also that if two things are equal to a third,
then they are equal to each other. In fact, stating the trivial and obvious is
a goal to strive for in any axiomatic system.  From trivial and obvious truths
that everyone agrees upon, we can prove results that are not so obvious yet
have absolute faith in them.  If we trust the axioms and the rules, we must,
by definition, trust the consequences of those axioms and rules, if logic is
to mean anything at all.

Our rule of inference\index{rule}, modus ponens\index{modus ponens}, is also
pretty obvious once you understand what it means.  If we prove a fact $P$, and
we also prove that $P$ implies $Q$, then $Q$ necessarily follows as a new
fact.  The rule provides us with a means for obtaining new facts (i.e.\
theorems\index{theorem}) from old ones.

The theorem that we have proved, $ t=t$, is so fundamental that you may wonder
why it isn't one of the axioms\index{axiom}.  In some axiom systems of
arithmetic, it {\em is} an axiom.  The choice of axioms in a theory is to some
extent arbitrary and even an art form, constrained only by the requirement
that any two equivalent axiom systems be able to derive each other as
theorems.  We could imagine that the inventor of our axiom system originally
included $ t=t$ as an axiom, then discovered that it could be derived as a
theorem from the other axioms.  Because of this, it was not necessary to
keep it as an axiom.  By eliminating it, the final set of axioms became
that much simpler.

Unless you have worked with formal proofs\index{formal proof} before, it
probably wasn't apparent to you that $ t=t$ could be derived from our two
axioms until you saw the proof. While you certainly believe that $ t=t$ is
true, you might not be able to convince an imaginary skeptic who believes only
in our two axioms until you produce the proof.  Formal proofs such as this are
hard to come up with when you first start working with them, but after you get
used to them they can become interesting and fun.  Once you understand the
idea behind formal proofs you will have grasped the fundamental principle that
underlies all of mathematics.  As the mathematics becomes more sophisticated,
its proofs become more challenging, but ultimately they all can be broken down
into individual steps as simple as the ones in our proof above.

Mendelson's\index{Mendelson, Elliot} book, from which our example was taken,
contains a number of detailed formal proofs such as these, and you may be
interested in looking it up.  The book is intended for mathematicians,
however, and most of it is rather advanced.  Popular literature describing
formal proofs\index{formal proof} include \cite[p.~296]{Rucker}\index{Rucker,
Rudy} and \cite[pp.~204--230]{Hofstadter}\index{Hofstadter, Douglas R.}.

\subsection{Converting It to Metamath}\label{convert}

Formal proofs\index{formal proof} such as the one in our example break down
logical reasoning into small, precise steps that leave little doubt that the
results follow from the axioms\index{axiom}.  You might think that this would
be the finest breakdown we can achieve in mathematics.  However, there is more
to the proof than meets the eye. Although our axioms were rather simple, a lot
of verbiage was needed before we could even state them:  we needed to define
``term,'' ``wff,'' and so on.  In addition, there are a number of implied
rules that we haven't even mentioned. For example, how do we know that step 3
of our proof follows from axiom A1? There is some hidden reasoning involved in
determining this.  Axiom A1 has two occurrences of the letter $ t$.  One of
the implied rules states that whatever we substitute for $ t$ must be a legal
term\index{term}.\footnote{Some authors make this implied rule explicit by
stating, ``only expressions of the above form are terms,'' after defining
``term.''}  The expression $ t+0$ is pretty obviously a legal term whenever $
t$ is, but suppose we wanted to substitute a huge term with thousands of
symbols?  Certainly a lot of work would be involved in determining that it
really is a term, but in ordinary formal proofs all of this work would be
considered a single ``step.''

To express our axiom system in the Metamath\index{Metamath} language, we must
describe this auxiliary information in addition to the axioms themselves.
Metamath does not know what a ``term'' or a ``wff''\index{well-formed formula
(wff)} is.  In Metamath, the specification of the ways in which we can combine
symbols to obtain terms and wffs are like little axioms in themselves.  These
auxiliary axioms are expressed in the same notation as the ``real''
axioms\index{axiom}, and Metamath does not distinguish between the two.  The
distinction is made by you, i.e.\ by the way in which you interpret the
notation you have chosen to express these two kinds of axioms.

The Metamath language breaks down mathematical proofs into tiny pieces, much
more so than in ordinary formal proofs\index{formal proof}.  If a single
step\index{proof step} involves the
substitution\index{substitution!variable}\index{variable substitution} of a
complex term for one of its variables, Metamath must see this single step
broken down into many small steps.  This fine-grained breakdown is what gives
Metamath generality and flexibility as it lets it not be limited to any
particular mathematical notation.

Metamath's proof notation is not, in itself, intended to be read by humans but
rather is in a compact format intended for a machine.  The Metamath program
will convert this notation to a form you can understand, using the \texttt{show
proof}\index{\texttt{show proof} command} command.  You can tell the program what
level of detail of the proof you want to look at.  You may want to look at
just the logical inference steps that correspond
to ordinary formal proof steps,
or you may want to see the fine-grained steps that prove that an expression is
a term.

Here, without further ado, is our example converted to the
Metamath\index{Metamath} language:\index{metavariable}\label{demo0}

\begin{verbatim}
$( Declare the constant symbols we will use $)
    $c 0 + = -> ( ) term wff |- $.
$( Declare the metavariables we will use $)
    $v t r s P Q $.
$( Specify properties of the metavariables $)
    tt $f term t $.
    tr $f term r $.
    ts $f term s $.
    wp $f wff P $.
    wq $f wff Q $.
$( Define "term" and "wff" $)
    tze $a term 0 $.
    tpl $a term ( t + r ) $.
    weq $a wff t = r $.
    wim $a wff ( P -> Q ) $.
$( State the axioms $)
    a1 $a |- ( t = r -> ( t = s -> r = s ) ) $.
    a2 $a |- ( t + 0 ) = t $.
$( Define the modus ponens inference rule $)
    ${
       min $e |- P $.
       maj $e |- ( P -> Q ) $.
       mp  $a |- Q $.
    $}
$( Prove a theorem $)
    th1 $p |- t = t $=
  $( Here is its proof: $)
       tt tze tpl tt weq tt tt weq tt a2 tt tze tpl
       tt weq tt tze tpl tt weq tt tt weq wim tt a2
       tt tze tpl tt tt a1 mp mp
     $.
\end{verbatim}\index{metavariable}

A ``database''\index{database} is a set of one or more {\sc ascii} source
files.  Here's a brief description of this Metamath\index{Metamath} database
(which consists of this single source file), so that you can understand in
general terms what is going on.  To understand the source file in detail, you
should read Chapter~\ref{languagespec}.

The database is a sequence of ``tokens,''\index{token} which are normally
separated by spaces or line breaks.  The only tokens that are built into
the Metamath language are those beginning with \texttt{\$}.  These tokens
are called ``keywords.''\index{keyword}  All other tokens are
user-defined, and their names are arbitrary.

As you might have guessed, the Metamath token \texttt{\$(}\index{\texttt{\$(} and
\texttt{\$)} auxiliary keywords} starts a comment and \texttt{\$)} ends a comment.

The Metamath tokens \texttt{\$c}\index{\texttt{\$c} statement},
\texttt{\$v}\index{\texttt{\$v} statement},
\texttt{\$e}\index{\texttt{\$e} statement},
\texttt{\$f}\index{\texttt{\$f} statement},
\texttt{\$a}\index{\texttt{\$a} statement}, and
\texttt{\$p}\index{\texttt{\$p} statement} specify ``statements'' that
end with \texttt{\$.}\,.\index{\texttt{\$.}\ keyword}

The Metamath tokens \texttt{\$c} and \texttt{\$v} each declare\index{constant
declaration}\index{variable declaration} a list of user-defined tokens, called
``math symbols,''\index{math symbol} that the database will reference later
on.  All of the math symbols they define you have seen earlier except the
turnstile symbol \texttt{|-} ($\vdash$)\index{turnstile ({$\,\vdash$})}, which is
commonly used by logicians to mean ``a proof exists for.''  For us
the turnstile is just a
convenient symbol that distinguishes expressions that are axioms\index{axiom}
or theorems\index{theorem} from expressions that are terms or wffs.

The \texttt{\$c} statement declares ``constants''\index{constant} and
the \texttt{\$v} statement declares
``variables''\index{variable}\index{constant declaration}\index{variable
declaration} (or more precisely, metavariables\index{metavariable}).  A
variable may be substituted\index{substitution!variable}\index{variable
substitution} with sequences of math symbols whereas a constant may not
be substituted with anything.

It may seem redundant to require both \texttt{\$c}\index{\texttt{\$c} statement} and
\texttt{\$v}\index{\texttt{\$v} statement} statements (since any math
symbol\index{math symbol} not specified with a \texttt{\$c} statement could be
presumed to be a variable), but this provides for better error checking and
also allows math symbols to be redeclared\index{redeclaration of symbols}
(Section~\ref{scoping}).

The token \texttt{\$f}\index{\texttt{\$f} statement} specifies a
statement called a ``variable-type hypothesis'' (also called a
``floating hypothesis'') and \texttt{\$e}\index{\texttt{\$e} statement}
specifies a ``logical hypothesis'' (also called an ``essential
hypothesis'').\index{hypothesis}\index{variable-type
hypothesis}\index{logical hypothesis}\index{floating
hypothesis}\index{essential hypothesis} The token
\texttt{\$a}\index{\texttt{\$a} statement} specifies an ``axiomatic
assertion,''\index{axiomatic assertion} and
\texttt{\$p}\index{\texttt{\$p} statement} specifies a ``provable
assertion.''\index{provable assertion} To the left of each occurrence of
these four tokens is a ``label''\index{label} that identifies the
hypothesis or assertion for later reference.  For example, the label of
the first axiomatic assertion is \texttt{tze}.  A \texttt{\$f} statement
must contain exactly two math symbols, a constant followed by a
variable.  The \texttt{\$e}, \texttt{\$a}, and \texttt{\$p} statements
each start with a constant followed by, in general, an arbitrary
sequence of math symbols.

Associated with each assertion\index{assertion} is a set of hypotheses
that must be satisfied in order for the assertion to be used in a proof.
These are called the ``mandatory hypotheses''\index{mandatory
hypothesis} of the assertion.  Among those hypotheses whose ``scope''
(described below) includes the assertion, \texttt{\$e} hypotheses are
always mandatory and \texttt{\$f}\index{\texttt{\$f} statement}
hypotheses are mandatory when they share their variable with the
assertion or its \texttt{\$e} hypotheses.  The exact rules for
determining which hypotheses are mandatory are described in detail in
Sections~\ref{frames} and \ref{scoping}.  For example, the mandatory
hypotheses of assertion \texttt{tpl} are \texttt{tt} and \texttt{tr},
whereas assertion \texttt{tze} has no mandatory hypotheses because it
contains no variables and has no \texttt{\$e}\index{\texttt{\$e}
statement} hypothesis.  Metamath's \texttt{show statement}
command\index{\texttt{show statement} command}, described in the next
section, will show you a statement's mandatory hypotheses.

Sometimes we need to make a hypothesis relevant to only certain
assertions.  The set of statements to which a hypothesis is relevant is
called its ``scope.''  The Metamath brackets,
\texttt{\$\char`\{}\index{\texttt{\$\char`\{} and \texttt{\$\char`\}}
keywords} and \texttt{\$\char`\}}, define a ``block''\index{block} that
delimits the scope of any hypothesis contained between them.  The
assertion \texttt{mp} has mandatory hypotheses \texttt{wp}, \texttt{wq},
\texttt{min}, and \texttt{maj}.  The only mandatory hypothesis of
\texttt{th1}, on the other hand, is \texttt{tt}, since \texttt{th1}
occurs outside of the block containing \texttt{min} and \texttt{maj}.

Note that \texttt{\$\char`\{} and \texttt{\$\char`\}} do not affect the
scope of assertions (\texttt{\$a} and \texttt{\$p}).  Assertions are always
available to be referenced by any later proof in the source file.

Each provable assertion (\texttt{\$p}\index{\texttt{\$p} statement}
statement) has two parts.  The first part is the
assertion\index{assertion} itself, which is a sequence of math
symbol\index{math symbol} tokens placed between the \texttt{\$p} token
and a \texttt{\$=}\index{\texttt{\$=} keyword} token.  The second part
is a ``proof,'' which is a list of label tokens placed between the
\texttt{\$=} token and the \texttt{\$.}\index{\texttt{\$.}\ keyword}\
token that ends the statement.\footnote{If you've looked at the
\texttt{set.mm} database, you may have noticed another notation used for
proofs.  The other notation is called ``compressed.''\index{compressed
proof}\index{proof!compressed} It reduces the amount of space needed to
store a proof in the database and is described in
Appendix~\ref{compressed}.  In the example above, we use
``normal''\index{normal proof}\index{proof!normal} notation.} The proof
acts as a series of instructions to the Metamath program, telling it how
to build up the sequence of math symbols contained in the assertion part of
the \texttt{\$p} statement, making use of the hypotheses of the
\texttt{\$p} statement and previous assertions.  The construction takes
place according to precise rules.  If the list of labels in the proof
causes these rules to be violated, or if the final sequence that results
does not match the assertion, the Metamath program will notify you with
an error message.

If you are familiar with reverse Polish notation (RPN), which is sometimes used
on pocket calculators, here in a nutshell is how a proof works.  Each
hypothesis label\index{hypothesis label} in the proof is pushed\index{push}
onto the RPN stack\index{stack}\index{RPN stack} as it is encountered. Each
assertion label\index{assertion label} pops\index{pop} off the stack as many
entries as the referenced assertion has mandatory hypotheses.  Variable
substitutions\index{substitution!variable}\index{variable substitution} are
computed which, when made to the referenced assertion's mandatory hypotheses,
cause these hypotheses to match the stack entries. These same substitutions
are then made to the variables in the referenced assertion itself, which is
then pushed onto the stack.  At the end of the proof, there should be one
stack entry, namely the assertion being proved.  This process is explained in
detail in Section~\ref{proof}.

Metamath's proof notation is not very readable for humans, but it allows the
proof to be stored compactly in a file.  The Metamath\index{Metamath} program
has proof display features that let you see what's going on in a more
readable way, as you will see in the next section.

The rules used in verifying a proof are not based on any built-in syntax of the
symbol sequence in an assertion\index{assertion} nor on any built-in meanings
attached to specific symbol names.  They are based strictly on symbol
matching:  constants\index{constant} must match themselves, and
variables\index{variable} may be replaced with anything that allows a match to
occur.  For example, instead of \texttt{term}, \texttt{0}, and \verb$|-$ we could
have just as well used \texttt{yellow}, \texttt{zero}, and \texttt{provable}, as long
as we did so consistently throughout the database.  Also, we could have used
\texttt{is provable} (two tokens) instead of \verb$|-$ (one token) throughout the
database.  In each of these cases, the proof would be exactly the same.  The
independence of proofs and notation means that you have a lot of flexibility to
change the notation you use without having to change any proofs.

\section{A Trial Run}\label{trialrun}

Now you are ready to try out the Metamath\index{Metamath} program.

On all computer systems, Metamath has a standard ``command line
interface'' (CLI)\index{command line interface (CLI)} that allows you to
interact with it.  You supply commands to the CLI by typing them on the
keyboard and pressing your keyboard's {\em return} key after each line
you enter.  The CLI is designed to be easy to use and has built-in help
features.

The first thing you should do is to use a text editor to create a file
called \texttt{demo0.mm} and type into it the Metamath source shown on
p.~\pageref{demo0}.  Actually, this file is included with your Metamath
software package, so check that first.  If you type it in, make sure
that you save it in the form of ``plain {\sc ascii} text with line
breaks.''  Most word processors will have this feature.

Next you must run the Metamath program.  Depending on your computer
system and how Metamath is installed, this could range from clicking the
mouse on the Metamath icon to typing \texttt{run metamath} to typing
simply \texttt{metamath}.  (Metamath's {\tt help invoke} command describes
alternate ways of invoking the Metamath program.)

When you first enter Metamath\index{Metamath}, it will be at the CLI, waiting
for your input. You will see something like the following on your screen:
\begin{verbatim}
Metamath - Version 0.177 27-Apr-2019
Type HELP for help, EXIT to exit.
MM>
\end{verbatim}
The \texttt{MM>} prompt means that Metamath is waiting for a command.
Command keywords\index{command keyword} are not case sensitive;
we will use lower-case commands in our examples.
The version number and its release date will probably be different on your
system from the one we show above.

The first thing that you need to do is to read in your
database:\index{\texttt{read} command}\footnote{If a directory path is
needed on Unix,\index{Unix file names}\index{file names!Unix} you should
enclose the path/file name in quotes to prevent Metamath from thinking
that the \texttt{/} in the path name is a command qualifier, e.g.,
\texttt{read \char`\"db/set.mm\char`\"}.  Quotes are optional when there
is no ambiguity.}
\begin{verbatim}
MM> read demo0.mm
\end{verbatim}
Remember to press the {\em return} key after entering this command.  If
you omit the file name, Metamath will prompt you for one.   The syntax for
specifying a Macintosh file name path is given in a footnote on
p.~\pageref{includef}.\index{Macintosh file names}\index{file
names!Macintosh}

If there are any syntax errors in the database, Metamath will let you know
when it reads in the file.  The one thing that Metamath does not check when
reading in a database is that all proofs are correct, because this would
slow it down too much.  It is a good idea to periodically verify the proofs in
a database you are making changes to.  To do this, use the following command
(and do it for your \texttt{demo0.mm} file now).  Note that the \texttt{*} is a
``wild card'' meaning all proofs in the file.\index{\texttt{verify proof} command}
\begin{verbatim}
MM> verify proof *
\end{verbatim}
Metamath will report any proofs that are incorrect.

It is often useful to save the information that the Metamath program displays
on the screen. You can save everything that happens on the screen by opening a
log file. You may want to do this before you read in a database so that you
can examine any errors later on.  To open a log file, type
\begin{verbatim}
MM> open log abc.log
\end{verbatim}
This will open a file called \texttt{abc.log}, and everything that appears on the
screen from this point on will be stored in this file.  The name of the log file
is arbitrary. To close the log file, type
\begin{verbatim}
MM> close log
\end{verbatim}

Several commands let you examine what's inside your database.
Section~\ref{exploring} has an overview of some useful ones.  The
\texttt{show labels} command lets you see what statement
labels\index{label} exist.  A \texttt{*} matches any combination of
characters, and \texttt{t*} refers to all labels starting with the
letter \texttt{t}.\index{\texttt{show labels} command} The \texttt{/all}
is a ``command qualifier''\index{command qualifier} that tells Metamath
to include labels of hypotheses.  (To see the syntax explained, type
\texttt{help show labels}.)  Type
\begin{verbatim}
MM> show labels t* /all
\end{verbatim}
Metamath will respond with
\begin{verbatim}
The statement number, label, and type are shown.
3 tt $f       4 tr $f       5 ts $f       8 tze $a
9 tpl $a      19 th1 $p
\end{verbatim}

You can use the \texttt{show statement} command to get information about a
particular statement.\index{\texttt{show statement} command}
For example, you can get information about the statement with label \texttt{mp}
by typing
\begin{verbatim}
MM> show statement mp /full
\end{verbatim}
Metamath will respond with
\begin{verbatim}
Statement 17 is located on line 43 of the file
"demo0.mm".
"Define the modus ponens inference rule"
17 mp $a |- Q $.
Its mandatory hypotheses in RPN order are:
  wp $f wff P $.
  wq $f wff Q $.
  min $e |- P $.
  maj $e |- ( P -> Q ) $.
The statement and its hypotheses require the
      variables:  Q P
The variables it contains are:  Q P
\end{verbatim}
The mandatory hypotheses\index{mandatory hypothesis} and their
order\index{RPN order} are
useful to know when you are trying to understand or debug a proof.

Now you are ready to look at what's really inside our proof.  First, here is
how to look at every step in the proof---not just the ones corresponding to an
ordinary formal proof\index{formal proof}, but also the ones that build up the
formulas that appear in each ordinary formal proof step.\index{\texttt{show
proof} command}
\begin{verbatim}
MM> show proof th1 /lemmon /all
\end{verbatim}

This will display the proof on the screen in the following format:
\begin{verbatim}
 1 tt            $f term t
 2 tze           $a term 0
 3 1,2 tpl       $a term ( t + 0 )
 4 tt            $f term t
 5 3,4 weq       $a wff ( t + 0 ) = t
 6 tt            $f term t
 7 tt            $f term t
 8 6,7 weq       $a wff t = t
 9 tt            $f term t
10 9 a2          $a |- ( t + 0 ) = t
11 tt            $f term t
12 tze           $a term 0
13 11,12 tpl     $a term ( t + 0 )
14 tt            $f term t
15 13,14 weq     $a wff ( t + 0 ) = t
16 tt            $f term t
17 tze           $a term 0
18 16,17 tpl     $a term ( t + 0 )
19 tt            $f term t
20 18,19 weq     $a wff ( t + 0 ) = t
21 tt            $f term t
22 tt            $f term t
23 21,22 weq     $a wff t = t
24 20,23 wim     $a wff ( ( t + 0 ) = t -> t = t )
25 tt            $f term t
26 25 a2         $a |- ( t + 0 ) = t
27 tt            $f term t
28 tze           $a term 0
29 27,28 tpl     $a term ( t + 0 )
30 tt            $f term t
31 tt            $f term t
32 29,30,31 a1   $a |- ( ( t + 0 ) = t -> ( ( t + 0 )
                                     = t -> t = t ) )
33 15,24,26,32 mp  $a |- ( ( t + 0 ) = t -> t = t )
34 5,8,10,33 mp  $a |- t = t
\end{verbatim}

The \texttt{/lemmon} command qualifier specifies what is known as a Lemmon-style
display\index{Lemmon-style proof}\index{proof!Lemmon-style}.  Omitting the
\texttt{/lemmon} qualifier results in a tree-style proof (see
p.~\pageref{treeproof} for an example) that is somewhat less explicit but
easier to follow once you get used to it.\index{tree-style
proof}\index{proof!tree-style}

The first number on each line is the step
number of the proof.  Any numbers that follow are step numbers assigned to the
hypotheses of the statement referenced by that step.  Next is the label of
the statement referenced by the step.  The statement type of the statement
referenced comes next, followed by the math symbol\index{math symbol} string
constructed by the proof up to that step.

The last step, 34, contains the statement that is being proved.

Looking at a small piece of the proof, notice that steps 3 and 4 have
established that
\texttt{( t + 0 )} and \texttt{t} are \texttt{term}\,s, and step 5 makes use of steps 3 and
4 to establish that \texttt{( t + 0 ) = t} is a \texttt{wff}.  Let Metamath
itself tell us in detail what is happening in step 5.  Note that the
``target hypothesis'' refers to where step 5 is eventually used, i.e., in step
34.
\begin{verbatim}
MM> show proof th1 /detailed_step 5
Proof step 5:  wp=weq $a wff ( t + 0 ) = t
This step assigns source "weq" ($a) to target "wp"
($f).  The source assertion requires the hypotheses
"tt" ($f, step 3) and "tr" ($f, step 4).  The parent
assertion of the target hypothesis is "mp" ($a,
step 34).
The source assertion before substitution was:
    weq $a wff t = r
The following substitutions were made to the source
assertion:
    Variable  Substituted with
     t         ( t + 0 )
     r         t
The target hypothesis before substitution was:
    wp $f wff P
The following substitution was made to the target
hypothesis:
    Variable  Substituted with
     P         ( t + 0 ) = t
\end{verbatim}

The full proof just shown is useful to understand what is going on in detail.
However, most of the time you will just be interested in
the ``essential'' or logical steps of a proof, i.e.\ those steps
that correspond to an
ordinary formal proof\index{formal proof}.  If you type
\begin{verbatim}
MM> show proof th1 /lemmon /renumber
\end{verbatim}
you will see\label{demoproof}
\begin{verbatim}
1 a2             $a |- ( t + 0 ) = t
2 a2             $a |- ( t + 0 ) = t
3 a1             $a |- ( ( t + 0 ) = t -> ( ( t + 0 )
                                     = t -> t = t ) )
4 2,3 mp         $a |- ( ( t + 0 ) = t -> t = t )
5 1,4 mp         $a |- t = t
\end{verbatim}
Compare this to the formal proof on p.~\pageref{zeroproof} and
notice the resemblance.
By default Metamath
does not show \texttt{\$f}\index{\texttt{\$f}
statement} hypotheses and everything branching off of them in the proof tree
when the proof is displayed; this makes the proof look more like an ordinary
mathematical proof, which does not normally incorporate the explicit
construction of expressions.
This is called the ``essential'' view
(at one time you had to add the
\texttt{/essential} qualifier in the \texttt{show proof}
command to get this view, but this is now the default).
You can could use the \texttt{/all} qualifier in the \texttt{show
proof} command to also show the explicit construction of expressions.
The \texttt{/renumber} qualifier means to renumber
the steps to correspond only to what is displayed.\index{\texttt{show proof}
command}

To exit Metamath, type\index{\texttt{exit} command}
\begin{verbatim}
MM> exit
\end{verbatim}

\subsection{Some Hints for Using the Command Line Interface}

We will conclude this quick introduction to Metamath\index{Metamath} with some
helpful hints on how to navigate your way through the commands.
\index{command line interface (CLI)}

When you type commands into Metamath's CLI, you only have to type as many
characters of a command keyword\index{command keyword} as are needed to make
it unambiguous.  If you type too few characters, Metamath will tell you what
the choices are.  In the case of the \texttt{read} command, only the \texttt{r} is
needed to specify it unambiguously, so you could have typed\index{\texttt{read}
command}
\begin{verbatim}
MM> r demo0.mm
\end{verbatim}
instead of
\begin{verbatim}
MM> read demo0.mm
\end{verbatim}
In our description, we always show the full command words.  When using the
Metamath CLI commands in a command file (to be read with the \texttt{submit}
command)\index{\texttt{submit} command}, it is good practice to use
the unabbreviated command to ensure your instructions will not become ambiguous
if more commands are added to the Metamath program in the future.

The command keywords\index{command
keyword} are not case sensitive; you may type either \texttt{read} or
\texttt{ReAd}.  File names may or may not be case sensitive, depending on your
computer's operating system.  Metamath label\index{label} and math
symbol\index{math symbol} tokens\index{token} are case-sensitive.

The \texttt{help} command\index{\texttt{help} command} will provide you
with a list of topics you can get help on.  You can then type
\texttt{help} {\em topic} to get help on that topic.

If you are uncertain of a command's spelling, just type as many characters
as you remember of the command.  If you have not typed enough characters to
specify it unambiguously, Metamath will tell you what choices you have.

\begin{verbatim}
MM> show s
         ^
?Ambiguous keyword - please specify SETTINGS,
STATEMENT, or SOURCE.
\end{verbatim}

If you don't know what argument to use as part of a command, type a
\texttt{?}\index{\texttt{]}@\texttt{?}\ in command lines}\ at the
argument position.  Metamath will tell you what it expected there.

\begin{verbatim}
MM> show ?
         ^
?Expected SETTINGS, LABELS, STATEMENT, SOURCE, PROOF,
MEMORY, TRACE_BACK, or USAGE.
\end{verbatim}

Finally, you may type just the first word or words of a command followed
by {\em return}.  Metamath will prompt you for the remaining part of the
command, showing you the choices at each step.  For example, instead of
typing \texttt{show statement th1 /full} you could interact in the
following manner:
\begin{verbatim}
MM> show
SETTINGS, LABELS, STATEMENT, SOURCE, PROOF,
MEMORY, TRACE_BACK, or USAGE <SETTINGS>? st
What is the statement label <th1>?
/ or nothing <nothing>? /
TEX, COMMENT_ONLY, or FULL <TEX>? f
/ or nothing <nothing>?
19 th1 $p |- t = t $= ... $.
\end{verbatim}
After each \texttt{?}\ in this mode, you must give Metamath the
information it requests.  Sometimes Metamath gives you a list of choices
with the default choice indicated by brackets \texttt{< > }. Pressing
{\em return} after the \texttt{?}\ will select the default choice.
Answering anything else will override the default.  Note that the
\texttt{/} in command qualifiers is considered a separate
token\index{token} by the parser, and this is why it is asked for
separately.

\section{Your First Proof}\label{frstprf}

Proofs are developed with the aid of the Proof Assistant\index{Proof
Assistant}.  We will now show you how the proof of theorem \texttt{th1}
was built.  So that you can repeat these steps, we will first have the
Proof Assistant erase the proof in Metamath's source buffer\index{source
buffer}, then reconstruct it.  (The source buffer is the place in memory
where Metamath stores the information in the database when it is
\texttt{read}\index{\texttt{read} command} in.  New or modified proofs
are kept in the source buffer until a \texttt{write source}
command\index{\texttt{write source} command} is issued.)  In practice, you
would place a \texttt{?}\index{\texttt{]}@\texttt{?}\ inside proofs}\
between \texttt{\$=}\index{\texttt{\$=} keyword} and
\texttt{\$.}\index{\texttt{\$.}\ keyword}\ in the database to indicate
to Metamath\index{Metamath} that the proof is unknown, and that would be
your starting point.  Whenever the \texttt{verify proof} command encounters
a proof with a \texttt{?}\ in place of a proof step, the statement is
identified as not proved.

When I first started creating Metamath proofs, I would write down
on a piece of paper the complete
formal proof\index{formal proof} as it would appear
in a \texttt{show proof} command\index{\texttt{show proof} command}; see
the display of \texttt{show proof th1 /lemmon /re\-num\-ber} above as an
example.  After you get used to using the Proof Assistant\index{Proof
Assistant} you may get to a point where you can ``see'' the proof in your mind
and let the Proof Assistant guide you in filling in the details, at least for
simpler proofs, but until you gain that experience you may find it very useful
to write down all the details in advance.
Otherwise you may waste a lot of time as you let it take you down a wrong path.
However, others do not find this approach as helpful.
For example, Thomas Brendan Leahy\index{Leahy, Thomas Brendan}
finds that it is more helpful to him to interactively
work backward from a machine-readable statement.
David A. Wheeler\index{Wheeler, David A.}
writes down a general approach, but develops the proof
interactively by switching between
working forwards (from hypotheses and facts likely to be useful) and
backwards (from the goal) until the forwards and backwards approaches meet.
In the end, use whatever approach works for you.

A proof is developed with the Proof Assistant by working backwards, starting
with the theorem\index{theorem} to be proved, and assigning each unknown step
with a theorem or hypothesis until no more unknown steps remain.  The Proof
Assistant will not let you make an assignment unless it can be ``unified''
with the unknown step.  This means that a
substitution\index{substitution!variable}\index{variable substitution} of
variables exists that will make the assignment match the unknown step.  On the
other hand, in the middle of a proof, when working backwards, often more than
one unification\index{unification} (set of substitutions) is possible, since
there is not enough information available at that point to uniquely establish
it.  In this case you can tell Metamath which unification to choose, or you
can continue to assign unknown steps until enough information is available to
make the unification unique.

We will assume you have entered Metamath and read in the database as described
above.  The following dialog shows how the proof was developed.  For more
details on what some of the commands do, refer to Section~\ref{pfcommands}.
\index{\texttt{prove} command}

\begin{verbatim}
MM> prove th1
Entering the Proof Assistant.  Type HELP for help, EXIT
to exit.  You will be working on the proof of statement th1:
  $p |- t = t
Note:  The proof you are starting with is already complete.
MM-PA>
\end{verbatim}

The \verb/MM-PA>/ prompt means we are inside the Proof
Assistant.\index{Proof Assistant} Most of the regular Metamath commands
(\texttt{show statement}, etc.) are still available if you need them.

\begin{verbatim}
MM-PA> delete all
The entire proof was deleted.
\end{verbatim}

We have deleted the whole proof so we can start from scratch.

\begin{verbatim}
MM-PA> show new_proof/lemmon/all
1 ?              $? |- t = t
\end{verbatim}

The \texttt{show new{\char`\_}proof} command\index{\texttt{show
new{\char`\_}proof} command} is like \texttt{show proof} except that we
don't specify a statement; instead, the proof we're working on is
displayed.

\begin{verbatim}
MM-PA> assign 1 mp
To undo the assignment, DELETE STEP 5 and INITIALIZE, UNIFY
if needed.
3   min=?  $? |- $2
4   maj=?  $? |- ( $2 -> t = t )
\end{verbatim}

The \texttt{assign} command\index{\texttt{assign} command} above means
``assign step 1 with the statement whose label is \texttt{mp}.''  Note
that step renumbering will constantly occur as you assign steps in the
middle of a proof; in general all steps from the step you assign until
the end of the proof will get moved up.  In this case, what used to be
step 1 is now step 5, because the (partial) proof now has five steps:
the four hypotheses of the \texttt{mp} statement and the \texttt{mp}
statement itself.  Let's look at all the steps in our partial proof:

\begin{verbatim}
MM-PA> show new_proof/lemmon/all
1 ?              $? wff $2
2 ?              $? wff t = t
3 ?              $? |- $2
4 ?              $? |- ( $2 -> t = t )
5 1,2,3,4 mp     $a |- t = t
\end{verbatim}

The symbol \texttt{\$2} is a temporary variable\index{temporary
variable} that represents a symbol sequence not yet known.  In the final
proof, all temporary variables will be eliminated.  The general format
for a temporary variable is \texttt{\$} followed by an integer.  Note
that \texttt{\$} is not a legal character in a math symbol (see
Section~\ref{dollardollar}, p.~\pageref{dollardollar}), so there will
never be a naming conflict between real symbols and temporary variables.

Unknown steps 1 and 2 are constructions of the two wffs used by the
modus ponens rule.  As you will see at the end of this section, the
Proof Assistant\index{Proof Assistant} can usually figure these steps
out by itself, and we will not have to worry about them.  Therefore from
here on we will display only the ``essential'' hypotheses, i.e.\ those
steps that correspond to traditional formal proofs\index{formal proof}.

\begin{verbatim}
MM-PA> show new_proof/lemmon
3 ?              $? |- $2
4 ?              $? |- ( $2 -> t = t )
5 3,4 mp         $a |- t = t
\end{verbatim}

Unknown steps 3 and 4 are the ones we must focus on.  They correspond to the
minor and major premises of the modus ponens rule.  We will assign them as
follows.  Notice that because of the step renumbering that takes place
after an assignment, it is advantageous to assign unknown steps in reverse
order, because earlier steps will not get renumbered.

\begin{verbatim}
MM-PA> assign 4 mp
To undo the assignment, DELETE STEP 8 and INITIALIZE, UNIFY
if needed.
3   min=?  $? |- $2
6     min=?  $? |- $4
7     maj=?  $? |- ( $4 -> ( $2 -> t = t ) )
\end{verbatim}

We are now going to describe an obscure feature that you will probably
never use but should be aware of.  The Metamath language allows empty
symbol sequences to be substituted for variables, but in most formal
systems this feature is never used.  One of the few examples where is it
used is the MIU-system\index{MIU-system} described in
Appendix~\ref{MIU}.  But such systems are rare, and by default this
feature is turned off in the Proof Assistant.  (It is always allowed for
{\tt verify proof}.)  Let us turn it on and see what
happens.\index{\texttt{set empty{\char`\_}substitution} command}

\begin{verbatim}
MM-PA> set empty_substitution on
Substitutions with empty symbol sequences is now allowed.
\end{verbatim}

With this feature enabled, more unifications will be
ambiguous\index{ambiguous unification}\index{unification!ambiguous} in
the middle of a proof, because
substitution\index{substitution!variable}\index{variable substitution}
of variables with empty symbol sequences will become an additional
possibility.  Let's see what happens when we make our next assignment.

\begin{verbatim}
MM-PA> assign 3 a2
There are 2 possible unifications.  Please select the correct
    one or Q if you want to UNIFY later.
Unify:  |- $6
 with:  |- ( $9 + 0 ) = $9
Unification #1 of 2 (weight = 7):
  Replace "$6" with "( + 0 ) ="
  Replace "$9" with ""
  Accept (A), reject (R), or quit (Q) <A>? r
\end{verbatim}

The first choice presented is the wrong one.  If we had selected it,
temporary variable \texttt{\$6} would have been assigned a truncated
wff, and temporary variable \texttt{\$9} would have been assigned an
empty sequence (which is not allowed in our system).  With this choice,
eventually we would reach a point where we would get stuck because
we would end up with steps impossible to prove.  (You may want to
try it.)  We typed \texttt{r} to reject the choice.

\begin{verbatim}
Unification #2 of 2 (weight = 21):
  Replace "$6" with "( $9 + 0 ) = $9"
  Accept (A), reject (R), or quit (Q) <A>? q
To undo the assignment, DELETE STEP 4 and INITIALIZE, UNIFY
if needed.
 7     min=?  $? |- $8
 8     maj=?  $? |- ( $8 -> ( $6 -> t = t ) )
\end{verbatim}

The second choice is correct, and normally we would type \texttt{a}
to accept it.  But instead we typed \texttt{q} to show what will happen:
it will leave the step with an unknown unification, which can be
seen as follows:

\begin{verbatim}
MM-PA> show new_proof/not_unified
 4   min    $a |- $6
        =a2  = |- ( $9 + 0 ) = $9
\end{verbatim}

Later we can unify this with the \texttt{unify}
\texttt{all/interactive} command.

The important point to remember is that occasionally you will be
presented with several unification choices while entering a proof, when
the program determines that there is not enough information yet to make
an unambiguous choice automatically (and this can happen even with
\texttt{set empty{\char`\_}substitution} turned off).  Usually it is
obvious by inspection which choice is correct, since incorrect ones will
tend to be meaningless fragments of wffs.  In addition, the correct
choice will usually be the first one presented, unlike our example
above.

Enough of this digression.  Let us go back to the default setting.

\begin{verbatim}
MM-PA> set empty_substitution off
The ability to substitute empty expressions for variables
has been turned off.  Note that this may make the Proof
Assistant too restrictive in some cases.
\end{verbatim}

If we delete the proof, start over, and get to the point where
we digressed above, there will no longer be an ambiguous unification.

\begin{verbatim}
MM-PA> assign 3 a2
To undo the assignment, DELETE STEP 4 and INITIALIZE, UNIFY
if needed.
 7     min=?  $? |- $4
 8     maj=?  $? |- ( $4 -> ( ( $5 + 0 ) = $5 -> t = t ) )
\end{verbatim}

Let us look at our proof so far, and continue.

\begin{verbatim}
MM-PA> show new_proof/lemmon
 4 a2            $a |- ( $5 + 0 ) = $5
 7 ?             $? |- $4
 8 ?             $? |- ( $4 -> ( ( $5 + 0 ) = $5 -> t = t ) )
 9 7,8 mp        $a |- ( ( $5 + 0 ) = $5 -> t = t )
10 4,9 mp        $a |- t = t
MM-PA> assign 8 a1
To undo the assignment, DELETE STEP 11 and INITIALIZE, UNIFY
if needed.
 7     min=?  $? |- ( t + 0 ) = t
MM-PA> assign 7 a2
To undo the assignment, DELETE STEP 8 and INITIALIZE, UNIFY
if needed.
MM-PA> show new_proof/lemmon
 4 a2            $a |- ( t + 0 ) = t
 8 a2            $a |- ( t + 0 ) = t
12 a1            $a |- ( ( t + 0 ) = t -> ( ( t + 0 ) = t ->
                                                    t = t ) )
13 8,12 mp       $a |- ( ( t + 0 ) = t -> t = t )
14 4,13 mp       $a |- t = t
\end{verbatim}

Now all temporary variables and unknown steps have been eliminated from the
``essential'' part of the proof.  When this is achieved, the Proof
Assistant\index{Proof Assistant} can usually figure out the rest of the proof
automatically.  (Note that the \texttt{improve} command can occasionally be
useful for filling in essential steps as well, but it only tries to make use
of statements that introduce no new variables in their hypotheses, which is
not the case for \texttt{mp}. Also it will not try to improve steps containing
temporary variables.)  Let's look at the complete proof, then run
the \texttt{improve} command, then look at it again.

\begin{verbatim}
MM-PA> show new_proof/lemmon/all
 1 ?             $? wff ( t + 0 ) = t
 2 ?             $? wff t = t
 3 ?             $? term t
 4 3 a2          $a |- ( t + 0 ) = t
 5 ?             $? wff ( t + 0 ) = t
 6 ?             $? wff ( ( t + 0 ) = t -> t = t )
 7 ?             $? term t
 8 7 a2          $a |- ( t + 0 ) = t
 9 ?             $? term ( t + 0 )
10 ?             $? term t
11 ?             $? term t
12 9,10,11 a1    $a |- ( ( t + 0 ) = t -> ( ( t + 0 ) = t ->
                                                    t = t ) )
13 5,6,8,12 mp   $a |- ( ( t + 0 ) = t -> t = t )
14 1,2,4,13 mp   $a |- t = t
\end{verbatim}

\begin{verbatim}
MM-PA> improve all
A proof of length 1 was found for step 11.
A proof of length 1 was found for step 10.
A proof of length 3 was found for step 9.
A proof of length 1 was found for step 7.
A proof of length 9 was found for step 6.
A proof of length 5 was found for step 5.
A proof of length 1 was found for step 3.
A proof of length 3 was found for step 2.
A proof of length 5 was found for step 1.
Steps 1 and above have been renumbered.
CONGRATULATIONS!  The proof is complete.  Use SAVE
NEW_PROOF to save it.  Note:  The Proof Assistant does
not detect $d violations.  After saving the proof, you
should verify it with VERIFY PROOF.
\end{verbatim}

The \texttt{save new{\char`\_}proof} command\index{\texttt{save
new{\char`\_}proof} command} will save the proof in the database.  Here
we will just display it in a form that can be clipped out of a log file
and inserted manually into the database source file with a text
editor.\index{normal proof}\index{proof!normal}

\begin{verbatim}
MM-PA> show new_proof/normal
---------Clip out the proof below this line:
      tt tze tpl tt weq tt tt weq tt a2 tt tze tpl tt weq
      tt tze tpl tt weq tt tt weq wim tt a2 tt tze tpl tt
      tt a1 mp mp $.
---------The proof of 'th1' to clip out ends above this line.
\end{verbatim}

There is another proof format called ``compressed''\index{compressed
proof}\index{proof!compressed} that you will see in databases.  It is
not important to understand how it is encoded but only to recognize it
when you see it.  Its only purpose is to reduce storage requirements for
large proofs.  A compressed proof can always be converted to a normal
one and vice-versa, and the Metamath \texttt{show proof}
commands\index{\texttt{show proof} command} work equally well with
compressed proofs.  The compressed proof format is described in
Appendix~\ref{compressed}.

\begin{verbatim}
MM-PA> show new_proof/compressed
---------Clip out the proof below this line:
      ( tze tpl weq a2 wim a1 mp ) ABCZADZAADZAEZJJKFLIA
      AGHH $.
---------The proof of 'th1' to clip out ends above this line.
\end{verbatim}

Now we will exit the Proof Assistant.  Since we made changes to the proof,
it will warn us that we have not saved it.  In this case, we don't care.

\begin{verbatim}
MM-PA> exit
Warning:  You have not saved changes to the proof.
Do you want to EXIT anyway (Y, N) <N>? y
Exiting the Proof Assistant.
Type EXIT again to exit Metamath.
\end{verbatim}

The Proof Assistant\index{Proof Assistant} has several other commands
that can help you while creating proofs.  See Section~\ref{pfcommands}
for a list of them.

A command that is often useful is \texttt{minimize{\char`\_}with
*/brief}, which tries to shorten the proof.  It can make the process
more efficient by letting you write a somewhat ``sloppy'' proof then
clean up some of the fine details of optimization for you (although it
can't perform miracles such as restructuring the overall proof).

\section{A Note About Editing a Data\-base File}

Once your source file contains proofs, there are some restrictions on
how you can edit it so that the proofs remain valid.  Pay particular
attention to these rules, since otherwise you can lose a lot of work.
It is a good idea to periodically verify all proofs with \texttt{verify
proof *} to ensure their integrity.

If your file contains only normal (as opposed to compressed) proofs, the
main rule is that you may not change the order of the mandatory
hypotheses\index{mandatory hypothesis} of any statement referenced in a
later proof.  For example, if you swap the order of the major and minor
premise in the modus ponens rule, all proofs making use of that rule
will become incorrect.  The \texttt{show statement}
command\index{\texttt{show statement} command} will show you the
mandatory hypotheses of a statement and their order.

If a statement has a compressed proof, you also must not change the
order of {\em its} mandatory hypotheses.  The compressed proof format
makes use of this information as part of the compression technique.
Note that swapping the names of two variables in a theorem will change
the order of its mandatory hypotheses.

The safest way to edit a statement, say \texttt{mytheorem}, is to
duplicate it then rename the original to \texttt{mytheoremOLD}
throughout the database.  Once the edited version is re-proved, all
statements referencing \texttt{mytheoremOLD} can be updated in the Proof
Assistant using \texttt{minimize{\char`\_}with
mytheorem
/allow{\char`\_}growth}.\index{\texttt{minimize{\char`\_}with} command}
% 3/10/07 Note: line-breaking the above results in duplicate index entries

\chapter{Abstract Mathematics Revealed}\label{fol}

\section{Logic and Set Theory}\label{logicandsettheory}

\begin{quote}
  {\em Set theory can be viewed as a form of exact theology.}
  \flushright\sc  Rudy Rucker\footnote{\cite{Barrow}, p.~31.}\\
\end{quote}\index{Rucker, Rudy}

Despite its seeming complexity, all of standard mathematics, no matter how
deep or abstract, can amazingly enough be derived from a relatively small set
of axioms\index{axiom} or first principles. The development of these axioms is
among the most impressive and important accomplishments of mathematics in the
20th century. Ultimately, these axioms can be broken down into a set of rules
for manipulating symbols that any technically oriented person can follow.

We will not spend much time trying to convey a deep, higher-level
understanding of the meaning of the axioms. This kind of understanding
requires some mathematical sophistication as well as an understanding of the
philosophy underlying the foundations of mathematics and typically develops
over time as you work with mathematics.  Our goal, instead, is to give you the
immediate ability to follow how theorems\index{theorem} are derived from the
axioms and from other theorems.  This will be similar to learning the syntax
of a computer language, which lets you follow the details in a program but
does not necessarily give you the ability to write non-trivial programs on
your own, an ability that comes with practice. For now don't be alarmed by
abstract-sounding names of the axioms; just focus on the rules for
manipulating the symbols, which follow the simple conventions of the
Metamath\index{Metamath} language.

The axioms that underlie all of standard mathematics consist of axioms of logic
and axioms of set theory. The axioms of logic are divided into two
subcategories, propositional calculus\index{propositional calculus} (sometimes
called sentential logic\index{sentential logic}) and predicate calculus
(sometimes called first-order logic\index{first-order logic}\index{quantifier
theory}\index{predicate calculus} or quantifier theory).  Propositional
calculus is a prerequisite for predicate calculus, and predicate calculus is a
prerequisite for set theory.  The version of set theory most commonly used is
Zermelo--Fraenkel set theory\index{Zermelo--Fraenkel set theory}\index{set theory}
with the axiom of choice,
often abbreviated as ZFC\index{ZFC}.

Here in a nutshell is what the axioms are all about in an informal way. The
connection between this description and symbols we will show you won't be
immediately apparent and in principle needn't ever be.  Our description just
tries to summarize what mathematicians think about when they work with the
axioms.

Logic is a set of rules that allow us determine truths given other truths.
Put another way,
logic is more or less the translation of what we would consider common sense
into a rigorous set of axioms.\index{axioms of logic}  Suppose $\varphi$,
$\psi$, and $\chi$ (the Greek letters phi, psi, and chi) represent statements
that are either true or false, and $x$ is a variable\index{variable!in predicate
calculus} ranging over some group of mathematical objects (sets, integers,
real numbers, etc.). In mathematics, a ``statement'' really means a formula,
and $\psi$ could be for example ``$x = 2$.''
Propositional calculus\index{propositional calculus}
allows us to use variables that are either true or false
and make deductions such as
``if $\varphi$ implies $\psi$ and $\psi$ implies $\chi$, then $\varphi$
implies $\chi$.''
Predicate calculus\index{predicate calculus}
extends propositional calculus by also allowing us
to discuss statements about objects (not just true and false values), including
statements about ``all'' or ``at least one'' object.
For example, predicate calculus allows to say,
``if $\varphi$ is true for all $x$, then $\varphi$ is true for some $x$.''
The logic used in \texttt{set.mm} is standard classical logic
(as opposed to other logic systems like intuitionistic logic).

Set theory\index{set theory} has to do with the manipulation of objects and
collections of objects, specifically the abstract, imaginary objects that
mathematics deals with, such as numbers. Everything that is claimed to exist
in mathematics is considered to be a set.  A set called the empty
set\index{empty set} contains nothing.  We represent the empty set by
$\varnothing$.  Many sets can be built up from the empty set.  There is a set
represented by $\{\varnothing\}$ that contains the empty set, another set
represented by $\{\varnothing,\{\varnothing\}\}$ that contains this set as
well as the empty set, another set represented by $\{\{\varnothing\}\}$ that
contains just the set that contains the empty set, and so on ad infinitum. All
mathematical objects, no matter how complex, are defined as being identical to
certain sets: the integer\index{integer} 0 is defined as the empty set, the
integer 1 is defined as $\{\varnothing\}$, the integer 2 is defined as
$\{\varnothing,\{\varnothing\}\}$.  (How these definitions were chosen doesn't
matter now, but the idea behind it is that these sets have the properties we
expect of integers once suitable operations are defined.)  Mathematical
operations, such as addition, are defined in terms of operations on
sets---their union\index{set union}, intersection\index{set intersection}, and
so on---operations you may have used in elementary school when you worked
with groups of apples and oranges.

With a leap of faith, the axioms also postulate the existence of infinite
sets\index{infinite set}, such as the set of all non-negative integers ($0, 1,
2,\ldots$, also called ``natural numbers''\index{natural number}).  This set
can't be represented with the brace notation\index{brace notation} we just
showed you, but requires a more complicated notation called ``class
abstraction.''\index{class abstraction}\index{abstraction class}  For
example, the infinite set $\{ x |
\mbox{``$x$ is a natural number''} \} $ means the ``set of all objects $x$
such that $x$ is a natural number'' i.e.\ the set of natural numbers; here,
``$x$ is a natural number'' is a rather complicated formula when broken down
into the primitive symbols.\label{expandom}\footnote{The statement ``$x$ is a
natural number'' is formally expressed as ``$x \in \omega$,'' where $\in$
(stylized epsilon) means ``is in'' or ``is an element of'' and $\omega$
(omega) means ``the set of natural numbers.''  When ``$x\in\omega$'' is
completely expanded in terms of the primitive symbols of set theory, the
result is  $\lnot$ $($ $\lnot$ $($ $\forall$ $z$ $($ $\lnot$ $\forall$ $w$ $($
$z$ $\in$ $w$ $\rightarrow$ $\lnot$ $w$ $\in$ $x$ $)$ $\rightarrow$ $z$ $\in$
$x$ $)$ $\rightarrow$ $($ $\forall$ $z$ $($ $\lnot$ $($ $\forall$ $w$ $($ $w$
$\in$ $z$ $\rightarrow$ $w$ $\in$ $x$ $)$ $\rightarrow$ $\forall$ $w$ $\lnot$
$w$ $\in$ $z$ $)$ $\rightarrow$ $\lnot$ $\forall$ $w$ $($ $w$ $\in$ $z$
$\rightarrow$ $\lnot$ $\forall$ $v$ $($ $v$ $\in$ $z$ $\rightarrow$ $\lnot$
$v$ $\in$ $w$ $)$ $)$ $)$ $\rightarrow$ $\lnot$ $\forall$ $z$ $\forall$ $w$
$($ $\lnot$ $($ $z$ $\in$ $x$ $\rightarrow$ $\lnot$ $w$ $\in$ $x$ $)$
$\rightarrow$ $($ $\lnot$ $z$ $\in$ $w$ $\rightarrow$ $($ $\lnot$ $z$ $=$ $w$
$\rightarrow$ $w$ $\in$ $z$ $)$ $)$ $)$ $)$ $)$ $\rightarrow$ $\lnot$
$\forall$ $y$ $($ $\lnot$ $($ $\lnot$ $($ $\forall$ $z$ $($ $\lnot$ $\forall$
$w$ $($ $z$ $\in$ $w$ $\rightarrow$ $\lnot$ $w$ $\in$ $y$ $)$ $\rightarrow$
$z$ $\in$ $y$ $)$ $\rightarrow$ $($ $\forall$ $z$ $($ $\lnot$ $($ $\forall$
$w$ $($ $w$ $\in$ $z$ $\rightarrow$ $w$ $\in$ $y$ $)$ $\rightarrow$ $\forall$
$w$ $\lnot$ $w$ $\in$ $z$ $)$ $\rightarrow$ $\lnot$ $\forall$ $w$ $($ $w$
$\in$ $z$ $\rightarrow$ $\lnot$ $\forall$ $v$ $($ $v$ $\in$ $z$ $\rightarrow$
$\lnot$ $v$ $\in$ $w$ $)$ $)$ $)$ $\rightarrow$ $\lnot$ $\forall$ $z$
$\forall$ $w$ $($ $\lnot$ $($ $z$ $\in$ $y$ $\rightarrow$ $\lnot$ $w$ $\in$
$y$ $)$ $\rightarrow$ $($ $\lnot$ $z$ $\in$ $w$ $\rightarrow$ $($ $\lnot$ $z$
$=$ $w$ $\rightarrow$ $w$ $\in$ $z$ $)$ $)$ $)$ $)$ $\rightarrow$ $($
$\forall$ $z$ $\lnot$ $z$ $\in$ $y$ $\rightarrow$ $\lnot$ $\forall$ $w$ $($
$\lnot$ $($ $w$ $\in$ $y$ $\rightarrow$ $\lnot$ $\forall$ $z$ $($ $w$ $\in$
$z$ $\rightarrow$ $\lnot$ $z$ $\in$ $y$ $)$ $)$ $\rightarrow$ $\lnot$ $($
$\lnot$ $\forall$ $z$ $($ $w$ $\in$ $z$ $\rightarrow$ $\lnot$ $z$ $\in$ $y$
$)$ $\rightarrow$ $w$ $\in$ $y$ $)$ $)$ $)$ $)$ $\rightarrow$ $x$ $\in$ $y$
$)$ $)$ $)$. Section~\ref{hierarchy} shows the hierarchy of definitions that
leads up to this expression.}\index{stylized epsilon ($\in$)}\index{omega
($\omega$)}  Actually, the primitive symbols don't even include the brace
notation.  The brace notation is a high-level definition, which you can find in
Section~\ref{hierarchy}.

Interestingly, the arithmetic of integers\index{integer} and
rationals\index{rational number} can be developed without appealing to the
existence of an infinite set, whereas the arithmetic of real
numbers\index{real number} requires it.

Each variable\index{variable!in set theory} in the axioms of set theory
represents an arbitrary set, and the axioms specify the legal kinds of things
you can do with these variables at a very primitive level.

Now, you may think that numbers and arithmetic are a lot more intuitive and
fundamental than sets and therefore should be the foundation of mathematics.
What is really the case is that you've dealt with numbers all your life and
are comfortable with a few rules for manipulating them such as addition and
multiplication.  Those rules only cover a small portion of what can be done
with numbers and only a very tiny fraction of the rest of mathematics.  If you
look at any elementary book on number theory, you will quickly become lost if
these are the only rules that you know.  Even though such books may present a
list of ``axioms''\index{axiom} for arithmetic, the ability to use the axioms
and to understand proofs of theorems\index{theorem} (facts) about numbers
requires an implicit mathematical talent that frustrates many people
from studying abstract mathematics.  The kind of mathematics that most people
know limits them to the practical, everyday usage of blindly manipulating
numbers and formulas, without any understanding of why those rules are correct
nor any ability to go any further.  For example, do you know why multiplying
two negative numbers yields a positive number?  Starting with set theory, you
will also start off blindly manipulating symbols according to the rules we give
you, but with the advantage that these rules will allow you, in principle, to
access {\em all} of mathematics, not just a tiny part of it.

Of course, concrete examples are often helpful in the learning process. For
example, you can verify that $2\cdot 3=3 \cdot 2$ by actually grouping
objects and can easily ``see'' how it generalizes to $x\cdot y = y\cdot x$,
even though you might not be able to rigorously prove it.  Similarly, in set
theory it can be helpful to understand how the axioms of set theory apply to
(and are correct for) small finite collections of objects.  You should be aware
that in set theory intuition can be misleading for infinite collections, and
rigorous proofs become more important.  For example, while $x\cdot y = y\cdot
x$ is correct for finite ordinals (which are the natural numbers), it is not
usually true for infinite ordinals.

\section{The Axioms for All of Mathematics}

In this section\index{axioms for mathematics}, we will show you the axioms
for all of standard mathematics (i.e.\ logic and set theory) as they are
traditionally presented.  The traditional presentation is useful for someone
with the mathematical experience needed to correctly manipulate high-level
abstract concepts.  For someone without this talent, knowing how to actually
make use of these axioms can be difficult.  The purpose of this section is to
allow you to see how the version of the axioms used in the standard
Metamath\index{Metamath} database \texttt{set.mm}\index{set
theory database (\texttt{set.mm})} relates to  the typical version
in textbooks, and also to give you an informal feel for them.

\subsection{Propositional Calculus}

Propositional calculus\index{propositional calculus} concerns itself with
statements that can be interpreted as either true or false.  Some examples of
statements (outside of mathematics) that are either true or false are ``It is
raining today'' and ``The United States has a female president.'' In
mathematics, as we mentioned, statements are really formulas.

In propositional calculus, we don't care what the statements are.  We also
treat a logical combination of statements, such as ``It is raining today and
the United States has a female president,'' no differently from a single
statement.  Statements and their combinations are called well-formed formulas
(wffs)\index{well-formed formula (wff)}.  We define wffs only in terms of
other wffs and don't define what a ``starting'' wff is.  As is common practice
in the literature, we use Greek letters to represent wffs.

Specifically, suppose $\varphi$ and $\psi$ are wffs.  Then the combinations
$\varphi\rightarrow\psi$ (``$\varphi$ implies $\psi$,'' also read ``if
$\varphi$ then $\psi$'')\index{implication ($\rightarrow$)} and $\lnot\varphi$
(``not $\varphi$'')\index{negation ($\lnot$)} are also wffs.

The three axioms of propositional calculus\index{axioms of propositional
calculus} are all wffs of the following form:\footnote{A remarkable result of
C.~A.~Meredith\index{Meredith, C. A.} squeezes these three axioms into the
single axiom $((((\varphi\rightarrow \psi)\rightarrow(\neg \chi\rightarrow\neg
\theta))\rightarrow \chi )\rightarrow \tau)\rightarrow((\tau\rightarrow
\varphi)\rightarrow(\theta\rightarrow \varphi))$ \cite{CAMeredith},
which is believed to be the shortest possible.}
\begin{center}
     $\varphi\rightarrow(\psi\rightarrow \varphi)$\\

     $(\varphi\rightarrow (\psi\rightarrow \chi))\rightarrow
((\varphi\rightarrow  \psi)\rightarrow (\varphi\rightarrow \chi))$\\

     $(\neg \varphi\rightarrow \neg\psi)\rightarrow (\psi\rightarrow
\varphi)$
\end{center}

These three axioms are widely used.
They are attributed to Jan {\L}ukasiewicz
(pronounced woo-kah-SHAY-vitch) and was popularized by Alonzo Church,
who called it system P2. (Thanks to Ted Ulrich for this information.)

There are an infinite number of axioms, one for each possible
wff\index{well-formed formula (wff)} of the above form.  (For this reason,
axioms such as the above are often called ``axiom schemes.''\index{axiom
scheme})  Each Greek letter in the axioms may be substituted with a more
complex wff to result in another axiom.  For example, substituting
$\neg(\varphi\rightarrow\chi)$ for $\varphi$ in the first axiom yields
$\neg(\varphi\rightarrow\chi)\rightarrow(\psi\rightarrow
\neg(\varphi\rightarrow\chi))$, which is still an axiom.

To deduce new true statements (theorems\index{theorem}) from the axioms, a
rule\index{rule} called ``modus ponens''\index{modus ponens} is used.  This
rule states that if the wff $\varphi$ is an axiom or a theorem, and the wff
$\varphi\rightarrow\psi$ is an axiom or a theorem, then the wff $\psi$ is also
a theorem\index{theorem}.

As a non-mathematical example of modus ponens, suppose we have proved (or
taken as an axiom) ``Bob is a man'' and separately have proved (or taken as
an axiom) ``If Bob is a man, then Bob is a human.''  Using the rule of modus
ponens, we can logically deduce, ``Bob is a human.''

From Metamath's\index{Metamath} point of view, the axioms and the rule of
modus ponens just define a mechanical means for deducing new true statements
from existing true statements, and that is the complete content of
propositional calculus as far as Metamath is concerned.  You can read a logic
textbook to gain a better understanding of their meaning, or you can just let
their meaning slowly become apparent to you after you use them for a while.

It is actually rather easy to check to see if a formula is a theorem of
propositional calculus.  Theorems of propositional calculus are also called
``tautologies.''\index{tautology}  The technique to check whether a formula is
a tautology is called the ``truth table method,''\index{truth table} and it
works like this.  A wff $\varphi\rightarrow\psi$ is false whenever $\varphi$ is true
and $\psi$ is false.  Otherwise it is true.  A wff $\lnot\varphi$ is false
whenever $\varphi$ is true and false otherwise. To verify a tautology such as
$\varphi\rightarrow(\psi\rightarrow \varphi)$, you break it down into sub-wffs and
construct a truth table that accounts for all possible combinations of true
and false assigned to the wff metavariables:
\begin{center}\begin{tabular}{|c|c|c|c|}\hline
\mbox{$\varphi$} & \mbox{$\psi$} & \mbox{$\psi\rightarrow\varphi$}
    & \mbox{$\varphi\rightarrow(\psi\rightarrow \varphi)$} \\ \hline \hline
              T   &  T    &      T       &        T    \\ \hline
              T   &  F    &      T       &        T    \\ \hline
              F   &  T    &      F       &        T    \\ \hline
              F   &  F    &      T       &        T    \\ \hline
\end{tabular}\end{center}
If all entries in the last column are true, the formula is a tautology.

Now, the truth table method doesn't tell you how to prove the tautology from
the axioms, but only that a proof exists.  Finding an actual proof (especially
one that is short and elegant) can be challenging.  Methods do exist for
automatically generating proofs in propositional calculus, but the proofs that
result can sometimes be very long.  In the Metamath \texttt{set.mm}\index{set
theory database (\texttt{set.mm})} database, most
or all proofs were created manually.

Section \ref{metadefprop} discusses various definitions
that make propositional calculus easier to use.
For example, we define:

\begin{itemize}
\item $\varphi \vee \psi$
  is true if either $\varphi$ or $\psi$ (or both) are true
  (this is disjunction\index{disjunction ($\vee$)}
  aka logical {\sc or}\index{logical {\sc or} ($\vee$)}).

\item $\varphi \wedge \psi$
  is true if both $\varphi$ and $\psi$ are true
  (this is conjunction\index{conjunction ($\wedge$)}
  aka logical {\sc and}\index{logical {\sc and} ($\wedge$)}).

\item $\varphi \leftrightarrow \psi$
  is true if $\varphi$ and $\psi$ have the same value, that is,
  they are both true or both false
  (this is the biconditional\index{biconditional ($\leftrightarrow$)}).
\end{itemize}

\subsection{Predicate Calculus}

Predicate calculus\index{predicate calculus} introduces the concept of
``individual variables,''\index{variable!in predicate calculus}\index{individual
variable} which
we will usually just call ``variables.''
These variables can represent something other than true or false (wffs),
and will always represent sets when we get to set theory.  There are also
three new symbols $\forall$\index{universal quantifier ($\forall$)},
$=$\index{equality ($=$)}, and $\in$\index{stylized epsilon ($\in$)},
read ``for all,'' ``equals,'' and ``is an element of''
respectively.  We will represent variables with the letters $x$, $y$, $z$, and
$w$, as is common practice in the literature.
For example, $\forall x \varphi$ means ``for all possible values of
$x$, $\varphi$ is true.''

In predicate calculus, we extend the definition of a wff\index{well-formed
formula (wff)}.  If $\varphi$ is a wff and $x$ and $y$ are variables, then
$\forall x \, \varphi$, $x=y$, and $x\in y$ are wffs. Note that these three new
types of wffs can be considered ``starting'' wffs from which we can build
other wffs with $\rightarrow$ and $\neg$ .  The concept of a starting wff was
absent in propositional calculus.  But starting wff or not, all we are really
concerned with is whether our wffs are correctly constructed according to
these mechanical rules.

A quick aside:
To prevent confusion, it might be best at this point to think of the variables
of Metamath\index{Metamath} as ``metavariables,''\index{metavariable} because
they are not quite the same as the variables we are introducing here.  A
(meta)variable in Metamath can be a wff or an individual variable, as well
as many other things; in general, it represents a kind of place holder for an
unspecified sequence of math symbols\index{math symbol}.

Unlike propositional calculus, no decision procedure\index{decision procedure}
analogous to the truth table method exists (nor theoretically can exist) that
will definitely determine whether a formula is a theorem of predicate
calculus.  Much of the work in the field of automated theorem
proving\index{automated theorem proving} has been dedicated to coming up with
clever heuristics for proving theorems of predicate calculus, but they can
never be guaranteed to work always.

Section \ref{metadefpred} discusses various definitions
that make predicate calculus easier to use.
For example, we define
$\exists x \varphi$ to mean
``there exists at least one possible value of $x$ where $\varphi$ is true.''

We now turn to looking at how predicate calculus can be formally
represented.

\subsubsection{Common Axioms}

There is a new rule of inference in predicate calculus:  if $\varphi$ is
an axiom or a theorem, then $\forall x \,\varphi$ is also a
theorem\index{theorem}.  This is called the rule of
``generalization.''\index{rule of generalization}
This is easily represented in Metamath.

In standard texts of logic, there are often two axioms of predicate
calculus\index{axioms of predicate calculus}:
\begin{center}
  $\forall x \,\varphi ( x ) \rightarrow \varphi ( y )$,
      where ``$y$ is properly substituted for $x$.''\\
  $\forall x ( \varphi \rightarrow \psi )\rightarrow ( \varphi \rightarrow
    \forall x\, \psi )$,
    where ``$x$ is not free in $\varphi$.''
\end{center}

Now at first glance, this seems simple:  just two axioms.  However,
conditional clauses are attached to each axiom describing requirements that
may seem puzzling to you.  In addition, the first axiom puts a variable symbol
in parentheses after each wff, seemingly violating our definition of a
wff\index{well-formed formula (wff)}; this is just an informal way of
referring to some arbitrary variable that may occur in the wff.  The
conditional clauses do, of course, have a precise meaning, but as it turns out
the precise meaning is somewhat complicated and awkward to formalize in a
way that a computer can handle easily.  Unlike propositional calculus, a
certain amount of mathematical sophistication and practice is needed to be
able to easily grasp and manipulate these concepts correctly.

Predicate calculus may be presented with or without axioms for
equality\index{axioms of equality}\index{equality ($=$)}. We will require the
axioms of equality as a prerequisite for the version of set theory we will
use.  The axioms for equality, when included, are often represented using these
two axioms:
\begin{center}
$x=x$\\ \ \\
$x=y\rightarrow (\varphi(x,x)\rightarrow\varphi(x,y))$ where ``$\varphi(x,y)$
   arises from $\varphi(x,x)$ by replacing some, but not necessarily all,
   free\index{free variable}
   occurrences of $x$ by $y$,\\ provided that $y$ is free for $x$
   in $\varphi(x,x)$.'' \end{center}
% (Mendelson p. 95)
The first equality axiom is simple, but again,
the condition on the second one is
somewhat awkward to implement on a computer.

\subsubsection{Tarski System S2}

Of course, we are not the first to notice the complications of these
predicate calculus axioms when being rigorous.

Well-known logician Alfred Tarski published in 1965
a system he called system S2\cite[p.~77]{Tarski1965}.
Tarski's system is \textit{exactly equivalent} to the traditional textbook
formalization, but (by clever use of equality axioms) it eliminates the
latter's primitive notions of ``proper substitution'' and ``free variable,''
replacing them with direct substitution and the notion of a variable
not occurring in a formula (which we express with distinct variable
constraints).

In advocating his system, Tarski wrote, ``The relatively complicated
character of [free variables and proper substitution] is a source
of certain inconveniences of both practical and theoretical nature;
this is clearly experienced both in teaching an elementary course of
mathematical logic and in formalizing the syntax of predicate logic for
some theoretical purposes''\cite[p.~61]{Tarski1965}\index{Tarski, Alfred}.

\subsubsection{Developing a Metamath Representation}

The standard textbook axioms of predicate calculus are somewhat
cumbersome to implement on a computer because of the complex notions of
``free variable''\index{free variable} and ``proper
substitution.''\index{proper substitution}\index{substitution!proper}
While it is possible to use the Metamath\index{Metamath} language to
implement these concepts, we have chosen not to implement them
as primitive constructs in the
\texttt{set.mm} set theory database.  Instead, we have eliminated them
within the axioms
by carefully crafting the axioms so as to avoid them,
building on Tarski's system S2.  This makes it
easy for a beginner to follow the steps in a proof without knowing any
advanced concepts other than the simple concept of
replacing\index{substitution!variable}\index{variable substitution}
variables with expressions.

In order to develop the concepts of free variable and proper
substitution from the axioms, we use an additional
Metamath statement type called ``disjoint variable
restriction''\index{disjoint variables} that we have not encountered
before.  In the context of the axioms, the statement \texttt{\$d} $ x\,
y$\index{\texttt{\$d} statement} simply means that $x$ and $y$ must be
distinct\index{distinct variables}, i.e.\ they may not be simultaneously
substituted\index{substitution!variable}\index{variable substitution}
with the same variable.  The statement \texttt{\$d} $ x\, \varphi$ means
variable $x$ must not occur in wff $\varphi$.  For the precise
definition of \texttt{\$d}, see Section~\ref{dollard}.

\subsubsection{Metamath representation}

The Metamath axiom system for predicate calculus
defined in set.mm uses Tarski's system S2.
As noted above, this has a different representation
than the traditional textbook formalization,
but it is \textit{exactly equivalent} to the textbook formalization,
and it is \textit{much} easier to work with.
This is reproduced as system S3 in Section 6 of
Megill's formalization \cite{Megill}\index{Megill, Norman}.

There is one exception, Tarski's axiom of existence,
which we label as axiom ax-6.
In the case of ax-6, Tarski's version is weaker because it includes a
distinct variable proviso. If we wish, we can also weaken our version
in this way and still have a metalogically complete system. Theorem
ax6 shows this by deriving, in the presence of the other axioms, our
ax-6 from Tarski's weaker version ax6v. However, we chose the stronger
version for our system because it is simpler to state and easier to use.

Tarski's system was designed for proving specific theorems rather than
more general theorem schemes. However, theorem schemes are much more
efficient than specific theorems for building a body of mathematical
knowledge, since they can be reused with different instances as
needed. While Tarski does derive some theorem schemes from his axioms,
their proofs require concepts that are ``outside'' of the system, such as
induction on formula length. The verification of such proofs is difficult
to automate in a proof verifier. (Specifically, Tarski treats the formulas
of his system as set-theoretical objects. In order to verify the proofs
of his theorem schemes, a proof verifier would need a significant amount
of set theory built into it.)

The Metamath axiom system for predicate calculus extends
Tarski's system to eliminate this difficulty. The additional
``auxilliary'' axiom
schemes (as we will call them in this section; see below) endow Tarski's
system with a nice property we call
metalogical completeness \cite[Remark 9.6]{Megill}\index{Megill, Norman}.
As a result, we can prove any theorem scheme
expressable in the ``simple metalogic'' of Tarski's system by using
only Metamath's direct substitution rule applied to the axiom system
(and no other metalogical or set-theoretical notions ``outside'' of the
system). Simple metalogic consists of schemes containing wff metavariables
(with no arguments) and/or set (also called ``individual'') metavariables,
accompanied by optional provisos each stating that two specified set
metavariables must be distinct or that a specified set metavariable may
not occur in a specified wff metavariable. Metamath's logic and set theory
axiom and rule schemes are all examples of simple metalogic. The schemes
of traditional predicate calculus with equality are examples which are
not simple metalogic, because they use wff metavariables with arguments
and have ``free for'' and ``not free in'' side conditions.

A rigorous justification for this system, using an older but
exactly equivalent set of axioms, can be
found in \cite{Megill}\index{Megill, Norman}.

This allows us to
take a different approach in the Metamath\index{Metamath} database
\texttt{set.mm}\index{set theory database (\texttt{set.mm})}.  We do not
directly use the primitive notions of ``free variable''\index{free variable}
and ``proper substitution''\index{proper
substitution}\index{substitution!proper} at all as primitive constructs.
Instead, we use a set
of axioms that are almost as simple to manipulate as those of
propositional calculus.  Our axiom system avoids complex primitive
notions by effectively embedding the complexity into the axioms
themselves.  As a result, we will end up with a larger number of axioms,
but they are ideally suited for a computer language such as Metamath.
(Section~\ref{metaaxioms} shows these axioms.)

We will not elaborate further
on the ``free variable'' and ``proper substitution''
concepts here.  You may consult
\cite[ch.\ 3--4]{Hamilton}\index{Hamilton, Alan G.} (as well as
many other books) for a precise explanation
of these concepts.  If you intend to do serious mathematical work, it is wise
to become familiar with the traditional textbook approach; even though the
concepts embedded in their axioms require a higher level of sophistication,
they can be more practical to deal with on an everyday, informal basis.  Even
if you are just developing Metamath proofs, familiarity with the traditional
approach can help you arrive at a proof outline much faster, which you can
then convert to the detail required by Metamath.

We do develop proper substitution rules later on, but in set.mm
they are defined as derived constructs; they are not primitives.

You should also note that our system of predicate calculus is specifically
tailored for set theory; thus there are only two specific predicates $=$ and
$\in$ and no functions\index{function!in predicate calculus}
or constants\index{constant!in predicate calculus} unlike more general systems.
We later add these.

\subsection{Set Theory}

Traditional Zermelo--Fraenkel set theory\index{Zermelo--Fraenkel set
theory}\index{set theory} with the Axiom of Choice
has 10 axioms, which can be expressed in the
language of predicate calculus.  In this section, we will list only the
names and brief English descriptions of these axioms, since we will give
you the precise formulas used by the Metamath\index{Metamath} set theory
database \texttt{set.mm} later on.

In the descriptions of the axioms, we assume that $x$, $y$, $z$, $w$, and $v$
represent sets.  These are the same as the variables\index{variable!in set
theory} in our predicate calculus system above, except that now we informally
think of the variables as ranging over sets.  Note that the terms
``object,''\index{object} ``set,''\index{set} ``element,''\index{element}
``collection,''\index{collection} and ``family''\index{family} are synonymous,
as are ``is an element of,'' ``is a member of,''\index{member} ``is contained
in,'' and ``belongs to.''  The different terms are used for convenience; for
example, ``a collection of sets'' is less confusing than ``a set of sets.''
A set $x$ is said to be a ``subset''\index{subset} of $y$ if every element of
$x$ is also an element of $y$; we also say $x$ is ``included in''
$y$.

The axioms are very general and apply to almost any conceivable mathematical
object, and this level of abstraction can be overwhelming at first.  To gain an
intuitive feel, it can be helpful to draw a picture illustrating the concept;
for example, a circle containing dots could represent a collection of sets,
and a smaller circle drawn inside the circle could represent a subset.
Overlapping circles can illustrate intersection and union.  Circles that
illustrate the concepts of set theory are frequently used in elementary
textbooks and are called Venn diagrams\index{Venn diagram}.\index{axioms of
set theory}

1. Axiom of Extensionality:  Two sets are identical if they contain the same
   elements.\index{Axiom of Extensionality}

2. Axiom of Pairing:  The set $\{ x , y \}$ exists.\index{Axiom of Pairing}

3. Axiom of Power Sets:  The power set of a set (the collection of all of
   its subsets) exists.  For example, the power set of $\{x,y\}$ is
   $\{\varnothing,\{x\},\{y\},\{x,y\}\}$ and it exists.\index{Axiom
of Power Sets}

4. Axiom of the Null Set:  The empty set $\varnothing$ exists.\index{Axiom of
the Null Set}

5. Axiom of Union:  The union of a set (the set containing the elements of
   its members) exists.  For example, the union of $\{\{x,y\},\{z\}\}$ is
 $\{x,y,z\}$ and
   it exists.\index{Axiom of Union}

6. Axiom of Regularity:  Roughly, no set can contain itself, nor can there
   be membership ``loops,'' such as a set being an
   element of one of its members.\index{Axiom of Regularity}

7. Axiom of Infinity:  An infinite set exists.  An example of an infinite
   set is the set of all
   integers.\index{Axiom of Infinity}

8. Axiom of Separation:  The set exists that is obtained by restricting $x$
   with some property.  For example, if the set of all integers exists,
   then the set of all even integers exists.\index{Axiom of Separation}

9. Axiom of Replacement:  The range of a function whose domain is restricted
   to the elements of a set $x$, is also a set.  For example, there
   is a function
   from integers (the function's domain) to their squares (its
   range).  If we
   restrict the domain to even integers, its range will become the set of
   squares of even integers, so this axiom asserts that the set of
    squares of even numbers exists.  Technical note:  In general, the
   ``function'' need not be a set but can be a proper class.
   \index{Axiom of Replacement}

10. Axiom of Choice:  Let $x$ be a set whose members are pairwise
  disjoint\index{disjoint sets} (i.e,
  whose members contain no elements in common).  Then there exists another
  set containing one element from each member of $x$.  For
  example, if $x$ is
  $\{\{y,z\},\{w,v\}\}$, where $y$, $z$, $w$, and $v$ are
  different sets, then a set such as $\{z,w\}$
  exists (but the axiom doesn't tell
  us which one).  (Actually the Axiom
  of Choice is redundant if the set $x$, as in this example, has a finite
  number of elements.)\index{Axiom of Choice}

The Axiom of Choice is usually considered an extension of ZF set theory rather
than a proper part of it.  It is sometimes considered philosophically
controversial because it specifies the existence of a set without specifying
what the set is. Constructive logics, including intuitionistic logic,
do not accept the axiom of choice.
Since there is some lingering controversy, we often prefer proofs that do
not use the axiom of choice (where there is a known alternative), and
in some cases we will use weaker axioms than the full axiom of choice.
That said, the axiom of choice is a powerful and widely-accepted tool,
so we do use it when needed.
ZF set theory that includes the Axiom of Choice is
called Zermelo--Fraenkel set theory with choice (ZFC\index{ZFC set theory}).

When expressed symbolically, the Axiom of Separation and the Axiom of
Replacement contain wff symbols and therefore each represent infinitely many
axioms, one for each possible wff. For this reason, they are often called
axiom schemes\index{axiom scheme}\index{well-formed formula (wff)}.

It turns out that the Axiom of the Null Set, the Axiom of Pairing, and the
Axiom of Separation can be derived from the other axioms and are therefore
unnecessary, although they tend to be included in standard texts for various
reasons (historical, philosophical, and possibly because some authors may not
know this).  In the Metamath\index{Metamath} set theory database, these
redundant axioms are derived from the other ones instead of truly
being considered axioms.
This is in keeping with our general goal of minimizing the number of
axioms we must depend on.

\subsection{Other Axioms}

Above we qualified the phrase ``all of mathematics'' with ``essentially.''
The main important missing piece is the ability to do category theory,
which requires huge sets (inaccessible cardinals) larger than those
postulated by the ZFC axioms. The Tarski--Grothendieck Axiom postulates
the existence of such sets.
Note that this is the same axiom used by Mizar for supporting
category theory.
The Tarski--Grothendieck axiom
can be viewed as a very strong replacement of the Axiom of Infinity,
the Axiom of Choice, and the Axiom of Power Sets.
The \texttt{set.mm} database includes this axiom; see the database
for details about it.
Again, we only use this axiom when we need to.
You are only likely to encounter or use this axiom if you are doing
category theory, since its use is highly specialized,
so we will not list the Tarsky-Grothendieck axiom
in the short list of axioms below.

Can there be even more axioms?
Of course.
G\"{o}del showed that no finite set of axioms or axiom schemes can completely
describe any consistent theory strong enough to include arithmetic.
But practically speaking, the ones above are the accepted foundation that
almost all mathematicians explicitly or implicitly base their work on.

\section{The Axioms in the Metamath Language}\label{metaaxioms}

Here we list the axioms as they appear in
\texttt{set.mm}\index{set theory database (\texttt{set.mm})} so you can
look them up there easily.  Incidentally, the \texttt{show statement
/tex} command\index{\texttt{show statement} command} was used to
typeset them.

%macros from show statement /tex
\newbox\mlinebox
\newbox\mtrialbox
\newbox\startprefix  % Prefix for first line of a formula
\newbox\contprefix  % Prefix for continuation line of a formula
\def\startm{  % Initialize formula line
  \setbox\mlinebox=\hbox{\unhcopy\startprefix}
}
\def\m#1{  % Add a symbol to the formula
  \setbox\mtrialbox=\hbox{\unhcopy\mlinebox $\,#1$}
  \ifdim\wd\mtrialbox>\hsize
    \box\mlinebox
    \setbox\mlinebox=\hbox{\unhcopy\contprefix $\,#1$}
  \else
    \setbox\mlinebox=\hbox{\unhbox\mtrialbox}
  \fi
}
\def\endm{  % Output the last line of a formula
  \box\mlinebox
}

% \SLASH for \ , \TOR for \/ (text OR), \TAND for /\ (text and)
% This embeds a following forced space to force the space.
\newcommand\SLASH{\char`\\~}
\newcommand\TOR{\char`\\/~}
\newcommand\TAND{/\char`\\~}
%
% Macro to output metamath raw text.
% This assumes \startprefix and \contprefix are set.
% NOTE: "\" is tricky to escape, use \SLASH, \TOR, and \TAND inside.
% Any use of "$ { ~ ^" must be escaped; ~ and ^ must be escaped specially.
% We escape { and } for consistency.
% For more about how this macro written, see:
% https://stackoverflow.com/questions/4073674/
% how-to-disable-indentation-in-particular-section-in-latex/4075706
% Use frenchspacing, or "e." will get an extra space after it.
\newlength\mystoreparindent
\newlength\mystorehangindent
\newenvironment{mmraw}{%
\setlength{\mystoreparindent}{\the\parindent}
\setlength{\mystorehangindent}{\the\hangindent}
\setlength{\parindent}{0pt} % TODO - we'll put in the \startprefix instead
\setlength{\hangindent}{\wd\the\contprefix}
\begin{flushleft}
\begin{frenchspacing}
\begin{tt}
{\unhcopy\startprefix}%
}{%
\end{tt}
\end{frenchspacing}
\end{flushleft}
\setlength{\parindent}{\mystoreparindent}
\setlength{\hangindent}{\mystorehangindent}
\vskip 1ex
}

\needspace{5\baselineskip}
\subsection{Propositional Calculus}\label{propcalc}\index{axioms of
propositional calculus}

\needspace{2\baselineskip}
Axiom of Simplification.\label{ax1}

\setbox\startprefix=\hbox{\tt \ \ ax-1\ \$a\ }
\setbox\contprefix=\hbox{\tt \ \ \ \ \ \ \ \ \ \ }
\startm
\m{\vdash}\m{(}\m{\varphi}\m{\rightarrow}\m{(}\m{\psi}\m{\rightarrow}\m{\varphi}\m{)}
\m{)}
\endm

\needspace{3\baselineskip}
\noindent Axiom of Distribution.

\setbox\startprefix=\hbox{\tt \ \ ax-2\ \$a\ }
\setbox\contprefix=\hbox{\tt \ \ \ \ \ \ \ \ \ \ }
\startm
\m{\vdash}\m{(}\m{(}\m{\varphi}\m{\rightarrow}\m{(}\m{\psi}\m{\rightarrow}\m{\chi}
\m{)}\m{)}\m{\rightarrow}\m{(}\m{(}\m{\varphi}\m{\rightarrow}\m{\psi}\m{)}\m{
\rightarrow}\m{(}\m{\varphi}\m{\rightarrow}\m{\chi}\m{)}\m{)}\m{)}
\endm

\needspace{2\baselineskip}
\noindent Axiom of Contraposition.

\setbox\startprefix=\hbox{\tt \ \ ax-3\ \$a\ }
\setbox\contprefix=\hbox{\tt \ \ \ \ \ \ \ \ \ \ }
\startm
\m{\vdash}\m{(}\m{(}\m{\lnot}\m{\varphi}\m{\rightarrow}\m{\lnot}\m{\psi}\m{)}\m{
\rightarrow}\m{(}\m{\psi}\m{\rightarrow}\m{\varphi}\m{)}\m{)}
\endm


\needspace{4\baselineskip}
\noindent Rule of Modus Ponens.\label{axmp}\index{modus ponens}

\setbox\startprefix=\hbox{\tt \ \ min\ \$e\ }
\setbox\contprefix=\hbox{\tt \ \ \ \ \ \ \ \ \ }
\startm
\m{\vdash}\m{\varphi}
\endm

\setbox\startprefix=\hbox{\tt \ \ maj\ \$e\ }
\setbox\contprefix=\hbox{\tt \ \ \ \ \ \ \ \ \ }
\startm
\m{\vdash}\m{(}\m{\varphi}\m{\rightarrow}\m{\psi}\m{)}
\endm

\setbox\startprefix=\hbox{\tt \ \ ax-mp\ \$a\ }
\setbox\contprefix=\hbox{\tt \ \ \ \ \ \ \ \ \ \ \ }
\startm
\m{\vdash}\m{\psi}
\endm


\needspace{7\baselineskip}
\subsection{Axioms of Predicate Calculus with Equality---Tarski's S2}\index{axioms of predicate calculus}

\needspace{3\baselineskip}
\noindent Rule of Generalization.\index{rule of generalization}

\setbox\startprefix=\hbox{\tt \ \ ax-g.1\ \$e\ }
\setbox\contprefix=\hbox{\tt \ \ \ \ \ \ \ \ \ \ \ \ }
\startm
\m{\vdash}\m{\varphi}
\endm

\setbox\startprefix=\hbox{\tt \ \ ax-gen\ \$a\ }
\setbox\contprefix=\hbox{\tt \ \ \ \ \ \ \ \ \ \ \ \ }
\startm
\m{\vdash}\m{\forall}\m{x}\m{\varphi}
\endm

\needspace{2\baselineskip}
\noindent Axiom of Quantified Implication.

\setbox\startprefix=\hbox{\tt \ \ ax-4\ \$a\ }
\setbox\contprefix=\hbox{\tt \ \ \ \ \ \ \ \ \ \ }
\startm
\m{\vdash}\m{(}\m{\forall}\m{x}\m{(}\m{\forall}\m{x}\m{\varphi}\m{\rightarrow}\m{
\psi}\m{)}\m{\rightarrow}\m{(}\m{\forall}\m{x}\m{\varphi}\m{\rightarrow}\m{
\forall}\m{x}\m{\psi}\m{)}\m{)}
\endm

\needspace{3\baselineskip}
\noindent Axiom of Distinctness.

% Aka: Add $d x ph $.
\setbox\startprefix=\hbox{\tt \ \ ax-5\ \$a\ }
\setbox\contprefix=\hbox{\tt \ \ \ \ \ \ \ \ \ \ }
\startm
\m{\vdash}\m{(}\m{\varphi}\m{\rightarrow}\m{\forall}\m{x}\m{\varphi}\m{)}\m{where}\m{ }\m{\$d}\m{ }\m{x}\m{ }\m{\varphi}\m{ }\m{(}\m{x}\m{ }\m{does}\m{ }\m{not}\m{ }\m{occur}\m{ }\m{in}\m{ }\m{\varphi}\m{)}
\endm

\needspace{2\baselineskip}
\noindent Axiom of Existence.

\setbox\startprefix=\hbox{\tt \ \ ax-6\ \$a\ }
\setbox\contprefix=\hbox{\tt \ \ \ \ \ \ \ \ \ \ }
\startm
\m{\vdash}\m{(}\m{\forall}\m{x}\m{(}\m{x}\m{=}\m{y}\m{\rightarrow}\m{\forall}
\m{x}\m{\varphi}\m{)}\m{\rightarrow}\m{\varphi}\m{)}
\endm

\needspace{2\baselineskip}
\noindent Axiom of Equality.

\setbox\startprefix=\hbox{\tt \ \ ax-7\ \$a\ }
\setbox\contprefix=\hbox{\tt \ \ \ \ \ \ \ \ \ \ }
\startm
\m{\vdash}\m{(}\m{x}\m{=}\m{y}\m{\rightarrow}\m{(}\m{x}\m{=}\m{z}\m{
\rightarrow}\m{y}\m{=}\m{z}\m{)}\m{)}
\endm

\needspace{2\baselineskip}
\noindent Axiom of Left Equality for Binary Predicate.

\setbox\startprefix=\hbox{\tt \ \ ax-8\ \$a\ }
\setbox\contprefix=\hbox{\tt \ \ \ \ \ \ \ \ \ \ \ }
\startm
\m{\vdash}\m{(}\m{x}\m{=}\m{y}\m{\rightarrow}\m{(}\m{x}\m{\in}\m{z}\m{
\rightarrow}\m{y}\m{\in}\m{z}\m{)}\m{)}
\endm

\needspace{2\baselineskip}
\noindent Axiom of Right Equality for Binary Predicate.

\setbox\startprefix=\hbox{\tt \ \ ax-9\ \$a\ }
\setbox\contprefix=\hbox{\tt \ \ \ \ \ \ \ \ \ \ \ }
\startm
\m{\vdash}\m{(}\m{x}\m{=}\m{y}\m{\rightarrow}\m{(}\m{z}\m{\in}\m{x}\m{
\rightarrow}\m{z}\m{\in}\m{y}\m{)}\m{)}
\endm


\needspace{4\baselineskip}
\subsection{Axioms of Predicate Calculus with Equality---Auxiliary}\index{axioms of predicate calculus - auxiliary}

\needspace{2\baselineskip}
\noindent Axiom of Quantified Negation.

\setbox\startprefix=\hbox{\tt \ \ ax-10\ \$a\ }
\setbox\contprefix=\hbox{\tt \ \ \ \ \ \ \ \ \ \ }
\startm
\m{\vdash}\m{(}\m{\lnot}\m{\forall}\m{x}\m{\lnot}\m{\forall}\m{x}\m{\varphi}\m{
\rightarrow}\m{\varphi}\m{)}
\endm

\needspace{2\baselineskip}
\noindent Axiom of Quantifier Commutation.

\setbox\startprefix=\hbox{\tt \ \ ax-11\ \$a\ }
\setbox\contprefix=\hbox{\tt \ \ \ \ \ \ \ \ \ \ }
\startm
\m{\vdash}\m{(}\m{\forall}\m{x}\m{\forall}\m{y}\m{\varphi}\m{\rightarrow}\m{
\forall}\m{y}\m{\forall}\m{x}\m{\varphi}\m{)}
\endm

\needspace{3\baselineskip}
\noindent Axiom of Substitution.

\setbox\startprefix=\hbox{\tt \ \ ax-12\ \$a\ }
\setbox\contprefix=\hbox{\tt \ \ \ \ \ \ \ \ \ \ \ }
\startm
\m{\vdash}\m{(}\m{\lnot}\m{\forall}\m{x}\m{\,x}\m{=}\m{y}\m{\rightarrow}\m{(}
\m{x}\m{=}\m{y}\m{\rightarrow}\m{(}\m{\varphi}\m{\rightarrow}\m{\forall}\m{x}\m{(}
\m{x}\m{=}\m{y}\m{\rightarrow}\m{\varphi}\m{)}\m{)}\m{)}\m{)}
\endm

\needspace{3\baselineskip}
\noindent Axiom of Quantified Equality.

\setbox\startprefix=\hbox{\tt \ \ ax-13\ \$a\ }
\setbox\contprefix=\hbox{\tt \ \ \ \ \ \ \ \ \ \ \ }
\startm
\m{\vdash}\m{(}\m{\lnot}\m{\forall}\m{z}\m{\,z}\m{=}\m{x}\m{\rightarrow}\m{(}
\m{\lnot}\m{\forall}\m{z}\m{\,z}\m{=}\m{y}\m{\rightarrow}\m{(}\m{x}\m{=}\m{y}
\m{\rightarrow}\m{\forall}\m{z}\m{\,x}\m{=}\m{y}\m{)}\m{)}\m{)}
\endm

% \noindent Axiom of Quantifier Substitution
%
% \setbox\startprefix=\hbox{\tt \ \ ax-c11n\ \$a\ }
% \setbox\contprefix=\hbox{\tt \ \ \ \ \ \ \ \ \ \ \ }
% \startm
% \m{\vdash}\m{(}\m{\forall}\m{x}\m{\,x}\m{=}\m{y}\m{\rightarrow}\m{(}\m{\forall}
% \m{x}\m{\varphi}\m{\rightarrow}\m{\forall}\m{y}\m{\varphi}\m{)}\m{)}
% \endm
%
% \noindent Axiom of Distinct Variables. (This axiom requires
% that two individual variables
% be distinct\index{\texttt{\$d} statement}\index{distinct
% variables}.)
%
% \setbox\startprefix=\hbox{\tt \ \ \ \ \ \ \ \ \$d\ }
% \setbox\contprefix=\hbox{\tt \ \ \ \ \ \ \ \ \ \ \ }
% \startm
% \m{x}\m{\,}\m{y}
% \endm
%
% \setbox\startprefix=\hbox{\tt \ \ ax-c16\ \$a\ }
% \setbox\contprefix=\hbox{\tt \ \ \ \ \ \ \ \ \ \ \ }
% \startm
% \m{\vdash}\m{(}\m{\forall}\m{x}\m{\,x}\m{=}\m{y}\m{\rightarrow}\m{(}\m{\varphi}\m{
% \rightarrow}\m{\forall}\m{x}\m{\varphi}\m{)}\m{)}
% \endm

% \noindent Axiom of Quantifier Introduction (2).  (This axiom requires
% that the individual variable not occur in the
% wff\index{\texttt{\$d} statement}\index{distinct variables}.)
%
% \setbox\startprefix=\hbox{\tt \ \ \ \ \ \ \ \ \$d\ }
% \setbox\contprefix=\hbox{\tt \ \ \ \ \ \ \ \ \ \ \ }
% \startm
% \m{x}\m{\,}\m{\varphi}
% \endm
% \setbox\startprefix=\hbox{\tt \ \ ax-5\ \$a\ }
% \setbox\contprefix=\hbox{\tt \ \ \ \ \ \ \ \ \ \ \ }
% \startm
% \m{\vdash}\m{(}\m{\varphi}\m{\rightarrow}\m{\forall}\m{x}\m{\varphi}\m{)}
% \endm

\subsection{Set Theory}\label{mmsettheoryaxioms}

In order to make the axioms of set theory\index{axioms of set theory} a little
more compact, there are several definitions from logic that we make use of
implicitly, namely, ``logical {\sc and},''\index{conjunction ($\wedge$)}
\index{logical {\sc and} ($\wedge$)} ``logical equivalence,''\index{logical
equivalence ($\leftrightarrow$)}\index{biconditional ($\leftrightarrow$)} and
``there exists.''\index{existential quantifier ($\exists$)}

\begin{center}\begin{tabular}{rcl}
  $( \varphi \wedge \psi )$ &\mbox{stands for}& $\neg ( \varphi
     \rightarrow \neg \psi )$\\
  $( \varphi \leftrightarrow \psi )$& \mbox{stands
     for}& $( ( \varphi \rightarrow \psi ) \wedge
     ( \psi \rightarrow \varphi ) )$\\
  $\exists x \,\varphi$ &\mbox{stands for}& $\neg \forall x \neg \varphi$
\end{tabular}\end{center}

In addition, the axioms of set theory require that all variables be
dis\-tinct,\index{distinct variables}\footnote{Set theory axioms can be
devised so that {\em no} variables are required to be distinct,
provided we replace \texttt{ax-c16} with an axiom stating that ``at
least two things exist,'' thus
making \texttt{ax-5} the only other axiom requiring the
\texttt{\$d} statement.  These axioms are unconventional and are not
presented here, but they can be found on the \url{http://metamath.org}
web site.  See also the Comment on
p.~\pageref{nodd}.}\index{\texttt{\$d} statement} thus we also assume:
\begin{center}
  \texttt{\$d }$x\,y\,z\,w$
\end{center}

\needspace{2\baselineskip}
\noindent Axiom of Extensionality.\index{Axiom of Extensionality}

\setbox\startprefix=\hbox{\tt \ \ ax-ext\ \$a\ }
\setbox\contprefix=\hbox{\tt \ \ \ \ \ \ \ \ \ \ \ \ }
\startm
\m{\vdash}\m{(}\m{\forall}\m{x}\m{(}\m{x}\m{\in}\m{y}\m{\leftrightarrow}\m{x}
\m{\in}\m{z}\m{)}\m{\rightarrow}\m{y}\m{=}\m{z}\m{)}
\endm

\needspace{3\baselineskip}
\noindent Axiom of Replacement.\index{Axiom of Replacement}

\setbox\startprefix=\hbox{\tt \ \ ax-rep\ \$a\ }
\setbox\contprefix=\hbox{\tt \ \ \ \ \ \ \ \ \ \ \ \ }
\startm
\m{\vdash}\m{(}\m{\forall}\m{w}\m{\exists}\m{y}\m{\forall}\m{z}\m{(}\m{%
\forall}\m{y}\m{\varphi}\m{\rightarrow}\m{z}\m{=}\m{y}\m{)}\m{\rightarrow}\m{%
\exists}\m{y}\m{\forall}\m{z}\m{(}\m{z}\m{\in}\m{y}\m{\leftrightarrow}\m{%
\exists}\m{w}\m{(}\m{w}\m{\in}\m{x}\m{\wedge}\m{\forall}\m{y}\m{\varphi}\m{)}%
\m{)}\m{)}
\endm

\needspace{2\baselineskip}
\noindent Axiom of Union.\index{Axiom of Union}

\setbox\startprefix=\hbox{\tt \ \ ax-un\ \$a\ }
\setbox\contprefix=\hbox{\tt \ \ \ \ \ \ \ \ \ \ \ }
\startm
\m{\vdash}\m{\exists}\m{x}\m{\forall}\m{y}\m{(}\m{\exists}\m{x}\m{(}\m{y}\m{
\in}\m{x}\m{\wedge}\m{x}\m{\in}\m{z}\m{)}\m{\rightarrow}\m{y}\m{\in}\m{x}\m{)}
\endm

\needspace{2\baselineskip}
\noindent Axiom of Power Sets.\index{Axiom of Power Sets}

\setbox\startprefix=\hbox{\tt \ \ ax-pow\ \$a\ }
\setbox\contprefix=\hbox{\tt \ \ \ \ \ \ \ \ \ \ \ \ }
\startm
\m{\vdash}\m{\exists}\m{x}\m{\forall}\m{y}\m{(}\m{\forall}\m{x}\m{(}\m{x}\m{
\in}\m{y}\m{\rightarrow}\m{x}\m{\in}\m{z}\m{)}\m{\rightarrow}\m{y}\m{\in}\m{x}
\m{)}
\endm

\needspace{3\baselineskip}
\noindent Axiom of Regularity.\index{Axiom of Regularity}

\setbox\startprefix=\hbox{\tt \ \ ax-reg\ \$a\ }
\setbox\contprefix=\hbox{\tt \ \ \ \ \ \ \ \ \ \ \ \ }
\startm
\m{\vdash}\m{(}\m{\exists}\m{x}\m{\,x}\m{\in}\m{y}\m{\rightarrow}\m{\exists}
\m{x}\m{(}\m{x}\m{\in}\m{y}\m{\wedge}\m{\forall}\m{z}\m{(}\m{z}\m{\in}\m{x}\m{
\rightarrow}\m{\lnot}\m{z}\m{\in}\m{y}\m{)}\m{)}\m{)}
\endm

\needspace{3\baselineskip}
\noindent Axiom of Infinity.\index{Axiom of Infinity}

\setbox\startprefix=\hbox{\tt \ \ ax-inf\ \$a\ }
\setbox\contprefix=\hbox{\tt \ \ \ \ \ \ \ \ \ \ \ \ \ \ \ }
\startm
\m{\vdash}\m{\exists}\m{x}\m{(}\m{y}\m{\in}\m{x}\m{\wedge}\m{\forall}\m{y}%
\m{(}\m{y}\m{\in}\m{x}\m{\rightarrow}\m{\exists}\m{z}\m{(}\m{y}\m{\in}\m{z}\m{%
\wedge}\m{z}\m{\in}\m{x}\m{)}\m{)}\m{)}
\endm

\needspace{4\baselineskip}
\noindent Axiom of Choice.\index{Axiom of Choice}

\setbox\startprefix=\hbox{\tt \ \ ax-ac\ \$a\ }
\setbox\contprefix=\hbox{\tt \ \ \ \ \ \ \ \ \ \ \ \ \ \ }
\startm
\m{\vdash}\m{\exists}\m{x}\m{\forall}\m{y}\m{\forall}\m{z}\m{(}\m{(}\m{y}\m{%
\in}\m{z}\m{\wedge}\m{z}\m{\in}\m{w}\m{)}\m{\rightarrow}\m{\exists}\m{w}\m{%
\forall}\m{y}\m{(}\m{\exists}\m{w}\m{(}\m{(}\m{y}\m{\in}\m{z}\m{\wedge}\m{z}%
\m{\in}\m{w}\m{)}\m{\wedge}\m{(}\m{y}\m{\in}\m{w}\m{\wedge}\m{w}\m{\in}\m{x}%
\m{)}\m{)}\m{\leftrightarrow}\m{y}\m{=}\m{w}\m{)}\m{)}
\endm

\subsection{That's It}

There you have it, the axioms for (essentially) all of mathematics!
Wonder at them and stare at them in awe.  Put a copy in your wallet, and
you will carry in your pocket the encoding for all theorems ever proved
and that ever will be proved, from the most mundane to the most
profound.

\section{A Hierarchy of Definitions}\label{hierarchy}

The axioms in the previous section in principle embody everything that can be
done within standard mathematics.  However, it is impractical to accomplish
very much by using them directly, for even simple concepts (from a human
perspective) can involve extremely long, incomprehensible formulas.
Mathematics is made practical by introducing definitions\index{definition}.
Definitions usually introduce new symbols, or at least new relationships among
existing symbols, to abbreviate more complex formulas.  An important
requirement for a definition is that there exist a straightforward
(algorithmic) method for eliminating the abbreviation by expanding it into the
more primitive symbol string that it represents.  Some
important definitions included in
the file \texttt{set.mm} are listed in this section for reference, and also to
give you a feel for why something like $\omega$\index{omega ($\omega$)} (the
set of natural numbers\index{natural number} 0, 1, 2,\ldots) becomes very
complicated when completely expanded into primitive symbols.

What is the motivation for definitions, aside from allowing complicated
expressions to be expressed more simply?  In the case of  $\omega$, one goal is
to provide a basis for the theory of natural numbers.\index{natural number}
Before set theory was invented, a set of axioms for arithmetic, called Peano's
postulates\index{Peano's postulates}, was devised and shown to have the
properties one expects for natural numbers.  Now anyone can postulate a
set of axioms, but if the axioms are inconsistent contradictions can be derived
from them.  Once a contradiction is derived, anything can be trivially
proved, including
all the facts of arithmetic and their negations.  To ensure that an
axiom system is at least as reliable as the axioms for set theory, we can
define sets and operations on those sets that satisfy the new axioms. In the
\texttt{set.mm} Metamath database, we prove that the elements of $\omega$ satisfy
Peano's postulates, and it's a long and hard journey to get there directly
from the axioms of set theory.  But the result is confidence in the
foundations of arithmetic.  And there is another advantage:  we now have all
the tools of set theory at our disposal for manipulating objects that obey the
axioms for arithmetic.

What are the criteria we use for definitions?  First, and of utmost importance,
the definition should not be {\em creative}\index{creative
definition}\index{definition!creative}, that
is it should not allow an expression that previously qualified as a wff but
was not provable, to become provable.   Second, the definition should be {\em
eliminable}\index{definition!eliminability}, that is, there should exist an
algorithmic method for converting any expression using the definition into
a logically equivalent expression that previously qualified as a wff.

In almost all cases below, definitions connect two expressions with either
$\leftrightarrow$ or $=$.  Eliminating\footnote{Here we mean the
elimination that a human might do in his or her head.  To eliminate them as
part of a Metamath proof we would invoke one of a number of
theorems that deal with transitivity of equivalence or equality; there are
many such examples in the proofs in \texttt{set.mm}.} such a definition is a
simple matter of substituting the expression on the left-hand side ({\em
definiendum}\index{definiendum} or thing being defined) with the equivalent,
more primitive expression on the right-hand side ({\em
definiens}\index{definiens} or definition).

Often a definition has variables on the right-hand side which do not appear on
the left-hand side; these are called {\em dummy variables}.\index{dummy
variable!in definitions}  In this case, any
allowable substitution (such as a new, distinct
variable) can be used when the definition is eliminated.  Dummy variables may
be used only if they are {\em effectively bound}\index{effectively bound
variable}, meaning that the definition will remain logically equivalent upon
any substitution of a dummy variable with any other {\em qualifying
expression}\index{qualifying expression}, i.e.\ any symbol string (such as
another variable) that
meets the restrictions on the dummy variable imposed by \texttt{\$d} and
\texttt{\$f} statements.  For example, we could define a constant $\perp$
(inverted tee, meaning logical ``false'') as $( \varphi \wedge \lnot \varphi
)$, i.e.\ ``phi and not phi.''  Here $\varphi$ is effectively bound because the
definition remains logically equivalent when we replace $\varphi$ with any
other wff.  (It is actually \texttt{df-fal}
in \texttt{set.mm}, which defines $\perp$.)

There are two cases where eliminating definitions is a little more
complex.  These cases are the definitions \texttt{df-bi} and
\texttt{df-cleq}.  The first stretches the concept of a definition a
little, as in effect it ``defines a definition;'' however, it meets our
requirements for a definition in that it is eliminable and does not
strengthen the language.  Theorem \texttt{bii} shows the substitution
needed to eliminate the $\leftrightarrow$\index{logical equivalence
($\leftrightarrow$)}\index{biconditional ($\leftrightarrow$)} symbol.

Definition \texttt{df-cleq}\index{equality ($=$)} extends the usage of
the equality symbol to include ``classes''\index{class} in set theory.  The
reason it is potentially problematic is that it can lead to statements which
do not follow from logic alone but presuppose the Axiom of
Extensionality\index{Axiom of Extensionality}, so we include this axiom
as a hypothesis for the definition.  We could have made \texttt{df-cleq} directly
eliminable by introducing a new equality symbol, but have chosen not to do so
in keeping with standard textbook practice.  Definitions such as \texttt{df-cleq}
that extend the meaning of existing symbols must be introduced carefully so
that they do not lead to contradictions.  Definition \texttt{df-clel} also
extends the meaning of an existing symbol ($\in$); while it doesn't strengthen
the language like \texttt{df-cleq}, this is not obvious and it must also be
subject to the same scrutiny.

Exercise:  Study how the wff $x\in\omega$, meaning ``$x$ is a natural
number,'' could be expanded in terms of primitive symbols, starting with the
definitions \texttt{df-clel} on p.~\pageref{dfclel} and \texttt{df-om} on
p.~\pageref{dfom} and working your way back.  Don't bother to work out the
details; just make sure that you understand how you could do it in principle.
The answer is shown in the footnote on p.~\pageref{expandom}.  If you
actually do work it out, you won't get exactly the same answer because we used
a few simplifications such as discarding occurrences of $\lnot\lnot$ (double
negation).

In the definitions below, we have placed the {\sc ascii} Metamath source
below each of the formulas to help you become familiar with the
notation in the database.  For simplicity, the necessary \texttt{\$f}
and \texttt{\$d} statements are not shown.  If you are in doubt, use the
\texttt{show statement}\index{\texttt{show statement} command} command
in the Metamath program to see the full statement.
A selection of this notation is summarized in Appendix~\ref{ASCII}.

To understand the motivation for these definitions, you should consult the
references indicated:  Takeuti and Zaring \cite{Takeuti}\index{Takeuti, Gaisi},
Quine \cite{Quine}\index{Quine, Willard Van Orman}, Bell and Machover
\cite{Bell}\index{Bell, J. L.}, and Enderton \cite{Enderton}\index{Enderton,
Herbert B.}.  Our list of definitions is provided more for reference than as a
learning aid.  However, by looking at a few of them you can gain a feel for
how the hierarchy is built up.  The definitions are a representative sample of
the many definitions
in \texttt{set.mm}, but they are complete with respect to the
theorem examples we will present in Section~\ref{sometheorems}.  Also, some are
slightly different from, but logically equivalent to, the ones in \texttt{set.mm}
(some of which have been revised over time to shorten them, for example).

\subsection{Definitions for Propositional Calculus}\label{metadefprop}

The symbols $\varphi$, $\psi$, and $\chi$ represent wffs.

Our first definition introduces the biconditional
connective\footnote{The term ``connective'' is informally used to mean a
symbol that is placed between two variables or adjacent to a variable,
whereas a mathematical ``constant'' usually indicates a symbol such as
the number 0 that may replace a variable or metavariable.  From
Metamath's point of view, there is no distinction between a connective
and a constant; both are constants in the Metamath
language.}\index{connective}\index{constant} (also called logical
equivalence)\index{logical equivalence
($\leftrightarrow$)}\index{biconditional ($\leftrightarrow$)}.  Unlike
most traditional developments, we have chosen not to have a separate
symbol such as ``Df.'' to mean ``is defined as.''  Instead, we will use
the biconditional connective for this purpose, as it lets us use
logic to manipulate definitions directly.  Here we state the properties
of the biconditional connective with a carefully crafted \texttt{\$a}
statement, which effectively uses the biconditional connective to define
itself.  The $\leftrightarrow$ symbol can be eliminated from a formula
using theorem \texttt{bii}, which is derived later.

\vskip 2ex
\noindent Define the biconditional connective.\label{df-bi}

\vskip 0.5ex
\setbox\startprefix=\hbox{\tt \ \ df-bi\ \$a\ }
\setbox\contprefix=\hbox{\tt \ \ \ \ \ \ \ \ \ \ \ }
\startm
\m{\vdash}\m{\lnot}\m{(}\m{(}\m{(}\m{\varphi}\m{\leftrightarrow}\m{\psi}\m{)}%
\m{\rightarrow}\m{\lnot}\m{(}\m{(}\m{\varphi}\m{\rightarrow}\m{\psi}\m{)}\m{%
\rightarrow}\m{\lnot}\m{(}\m{\psi}\m{\rightarrow}\m{\varphi}\m{)}\m{)}\m{)}\m{%
\rightarrow}\m{\lnot}\m{(}\m{\lnot}\m{(}\m{(}\m{\varphi}\m{\rightarrow}\m{%
\psi}\m{)}\m{\rightarrow}\m{\lnot}\m{(}\m{\psi}\m{\rightarrow}\m{\varphi}\m{)}%
\m{)}\m{\rightarrow}\m{(}\m{\varphi}\m{\leftrightarrow}\m{\psi}\m{)}\m{)}\m{)}
\endm
\begin{mmraw}%
|- -. ( ( ( ph <-> ps ) -> -. ( ( ph -> ps ) ->
-. ( ps -> ph ) ) ) -> -. ( -. ( ( ph -> ps ) -> -. (
ps -> ph ) ) -> ( ph <-> ps ) ) ) \$.
\end{mmraw}

\noindent This theorem relates the biconditional connective to primitive
connectives and can be used to eliminate the $\leftrightarrow$ symbol from any
wff.

\vskip 0.5ex
\setbox\startprefix=\hbox{\tt \ \ bii\ \$p\ }
\setbox\contprefix=\hbox{\tt \ \ \ \ \ \ \ \ \ }
\startm
\m{\vdash}\m{(}\m{(}\m{\varphi}\m{\leftrightarrow}\m{\psi}\m{)}\m{\leftrightarrow}
\m{\lnot}\m{(}\m{(}\m{\varphi}\m{\rightarrow}\m{\psi}\m{)}\m{\rightarrow}\m{\lnot}
\m{(}\m{\psi}\m{\rightarrow}\m{\varphi}\m{)}\m{)}\m{)}
\endm
\begin{mmraw}%
|- ( ( ph <-> ps ) <-> -. ( ( ph -> ps ) -> -. ( ps -> ph ) ) ) \$= ... \$.
\end{mmraw}

\noindent Define disjunction ({\sc or}).\index{disjunction ($\vee$)}%
\index{logical or (vee)@logical {\sc or} ($\vee$)}%
\index{df-or@\texttt{df-or}}\label{df-or}

\vskip 0.5ex
\setbox\startprefix=\hbox{\tt \ \ df-or\ \$a\ }
\setbox\contprefix=\hbox{\tt \ \ \ \ \ \ \ \ \ \ \ }
\startm
\m{\vdash}\m{(}\m{(}\m{\varphi}\m{\vee}\m{\psi}\m{)}\m{\leftrightarrow}\m{(}\m{
\lnot}\m{\varphi}\m{\rightarrow}\m{\psi}\m{)}\m{)}
\endm
\begin{mmraw}%
|- ( ( ph \TOR ps ) <-> ( -. ph -> ps ) ) \$.
\end{mmraw}

\noindent Define conjunction ({\sc and}).\index{conjunction ($\wedge$)}%
\index{logical {\sc and} ($\wedge$)}%
\index{df-an@\texttt{df-an}}\label{df-an}

\vskip 0.5ex
\setbox\startprefix=\hbox{\tt \ \ df-an\ \$a\ }
\setbox\contprefix=\hbox{\tt \ \ \ \ \ \ \ \ \ \ \ }
\startm
\m{\vdash}\m{(}\m{(}\m{\varphi}\m{\wedge}\m{\psi}\m{)}\m{\leftrightarrow}\m{\lnot}
\m{(}\m{\varphi}\m{\rightarrow}\m{\lnot}\m{\psi}\m{)}\m{)}
\endm
\begin{mmraw}%
|- ( ( ph \TAND ps ) <-> -. ( ph -> -. ps ) ) \$.
\end{mmraw}

\noindent Define disjunction ({\sc or}) of 3 wffs.%
\index{df-3or@\texttt{df-3or}}\label{df-3or}

\vskip 0.5ex
\setbox\startprefix=\hbox{\tt \ \ df-3or\ \$a\ }
\setbox\contprefix=\hbox{\tt \ \ \ \ \ \ \ \ \ \ \ \ }
\startm
\m{\vdash}\m{(}\m{(}\m{\varphi}\m{\vee}\m{\psi}\m{\vee}\m{\chi}\m{)}\m{
\leftrightarrow}\m{(}\m{(}\m{\varphi}\m{\vee}\m{\psi}\m{)}\m{\vee}\m{\chi}\m{)}
\m{)}
\endm
\begin{mmraw}%
|- ( ( ph \TOR ps \TOR ch ) <-> ( ( ph \TOR ps ) \TOR ch ) ) \$.
\end{mmraw}

\noindent Define conjunction ({\sc and}) of 3 wffs.%
\index{df-3an}\label{df-3an}

\vskip 0.5ex
\setbox\startprefix=\hbox{\tt \ \ df-3an\ \$a\ }
\setbox\contprefix=\hbox{\tt \ \ \ \ \ \ \ \ \ \ \ \ }
\startm
\m{\vdash}\m{(}\m{(}\m{\varphi}\m{\wedge}\m{\psi}\m{\wedge}\m{\chi}\m{)}\m{
\leftrightarrow}\m{(}\m{(}\m{\varphi}\m{\wedge}\m{\psi}\m{)}\m{\wedge}\m{\chi}
\m{)}\m{)}
\endm

\begin{mmraw}%
|- ( ( ph \TAND ps \TAND ch ) <-> ( ( ph \TAND ps ) \TAND ch ) ) \$.
\end{mmraw}

\subsection{Definitions for Predicate Calculus}\label{metadefpred}

The symbols $x$, $y$, and $z$ represent individual variables of predicate
calculus.  In this section, they are not necessarily distinct unless it is
explicitly
mentioned.

\vskip 2ex
\noindent Define existential quantification.
The expression $\exists x \varphi$ means
``there exists an $x$ where $\varphi$ is true.''\index{existential quantifier
($\exists$)}\label{df-ex}

\vskip 0.5ex
\setbox\startprefix=\hbox{\tt \ \ df-ex\ \$a\ }
\setbox\contprefix=\hbox{\tt \ \ \ \ \ \ \ \ \ \ \ }
\startm
\m{\vdash}\m{(}\m{\exists}\m{x}\m{\varphi}\m{\leftrightarrow}\m{\lnot}\m{\forall}
\m{x}\m{\lnot}\m{\varphi}\m{)}
\endm
\begin{mmraw}%
|- ( E. x ph <-> -. A. x -. ph ) \$.
\end{mmraw}

\noindent Define proper substitution.\index{proper
substitution}\index{substitution!proper}\label{df-sb}
In our notation, we use $[ y / x ] \varphi$ to mean ``the wff that
results when $y$ is properly substituted for $x$ in the wff
$\varphi$.''\footnote{
This can also be described
as substituting $x$ with $y$, $y$ properly replaces $x$, or
$x$ is properly replaced by $y$.}
% This is elsb4, though it currently says: ( [ x / y ] z e. y <-> z e. x )
For example,
$[ y / x ] z \in x$ is the same as $z \in y$.
One way to remember this notation is to notice that it looks like division
and recall that $( y / x ) \cdot x $ is $y$ (when $x \neq 0$).
The notation is different from the notation $\varphi ( x | y )$
that is sometimes used, because the latter notation is ambiguous for us:
for example, we don't know whether $\lnot \varphi ( x | y )$ is to be
interpreted as $\lnot ( \varphi ( x | y ) )$ or
$( \lnot \varphi ) ( x | y )$.\footnote{Because of the way
we initially defined wffs, this is the case
with any postfix connective\index{postfix connective} (one occurring after the
symbols being connected) or infix connective\index{infix connective} (one
occurring between the symbols being connected).  Metamath does not have a
built-in notion of operator binding strength that could eliminate the
ambiguity.  The initial parenthesis effectively provides a prefix
connective\index{prefix connective} to eliminate ambiguity.  Some conventions,
such as Polish notation\index{Polish notation} used in the 1930's and 1940's
by Polish logicians, use only prefix connectives and thus allow the total
elimination of parentheses, at the expense of readability.  In Metamath we
could actually redefine all notation to be Polish if we wanted to without
having to change any proofs!}  Other texts often use $\varphi(y)$ to indicate
our $[ y / x ] \varphi$, but this notation is even more ambiguous since there is
no explicit indication of what is being substituted.
Note that this
definition is valid even when
$x$ and $y$ are the same variable.  The first conjunct is a ``trick'' used to
achieve this property, making the definition look somewhat peculiar at
first.

\vskip 0.5ex
\setbox\startprefix=\hbox{\tt \ \ df-sb\ \$a\ }
\setbox\contprefix=\hbox{\tt \ \ \ \ \ \ \ \ \ \ \ }
\startm
\m{\vdash}\m{(}\m{[}\m{y}\m{/}\m{x}\m{]}\m{\varphi}\m{\leftrightarrow}\m{(}%
\m{(}\m{x}\m{=}\m{y}\m{\rightarrow}\m{\varphi}\m{)}\m{\wedge}\m{\exists}\m{x}%
\m{(}\m{x}\m{=}\m{y}\m{\wedge}\m{\varphi}\m{)}\m{)}\m{)}
\endm
\begin{mmraw}%
|- ( [ y / x ] ph <-> ( ( x = y -> ph ) \TAND E. x ( x = y \TAND ph ) ) ) \$.
\end{mmraw}


\noindent Define existential uniqueness\index{existential uniqueness
quantifier ($\exists "!$)} (``there exists exactly one'').  Note that $y$ is a
variable distinct from $x$ and not occurring in $\varphi$.\label{df-eu}

\vskip 0.5ex
\setbox\startprefix=\hbox{\tt \ \ df-eu\ \$a\ }
\setbox\contprefix=\hbox{\tt \ \ \ \ \ \ \ \ \ \ \ }
\startm
\m{\vdash}\m{(}\m{\exists}\m{{!}}\m{x}\m{\varphi}\m{\leftrightarrow}\m{\exists}
\m{y}\m{\forall}\m{x}\m{(}\m{\varphi}\m{\leftrightarrow}\m{x}\m{=}\m{y}\m{)}\m{)}
\endm

\begin{mmraw}%
|- ( E! x ph <-> E. y A. x ( ph <-> x = y ) ) \$.
\end{mmraw}

\subsection{Definitions for Set Theory}\label{setdefinitions}

The symbols $x$, $y$, $z$, and $w$ represent individual variables of
predicate calculus, which in set theory are understood to be sets.
However, using only the constructs shown so far would be very inconvenient.

To make set theory more practical, we introduce the notion of a ``class.''
A class\index{class} is either a set variable (such as $x$) or an
expression of the form $\{ x | \varphi\}$ (called an ``abstraction
class''\index{abstraction class}\index{class abstraction}).  Note that
sets (i.e.\ individual variables) always exist (this is a theorem of
logic, namely $\exists y \, y = x$ for any set $x$), whereas classes may
or may not exist (i.e.\ $\exists y \, y = A$ may or may not be true).
If a class does not exist it is called a ``proper class.''\index{proper
class}\index{class!proper} Definitions \texttt{df-clab},
\texttt{df-cleq}, and \texttt{df-clel} can be used to convert an
expression containing classes into one containing only set variables and
wff metavariables.

The symbols $A$, $B$, $C$, $D$, $ F$, $G$, and $R$ are metavariables that range
over classes.  A class metavariable $A$ may be eliminated from a wff by
replacing it with $\{ x|\varphi\}$ where neither $x$ nor $\varphi$ occur in
the wff.

The theory of classes can be shown to be an eliminable and conservative
extension of set theory. The \textbf{eliminability}
property shows that for every
formula in the extended language we can build a logically equivalent
formula in the basic language; so that even if the extended language
provides more ease to convey and formulate mathematical ideas for set
theory, its expressive power does not in fact strengthen the basic
language's expressive power.
The \textbf{conservation} property shows that for
every proof of a formula of the basic language in the extended system
we can build another proof of the same formula in the basic system;
so that, concerning theorems on sets only, the deductive powers of
the extended system and of the basic system are identical. Together,
these properties mean that the extended language can be treated as a
definitional extension that is \textbf{sound}.

A rigorous justification, which we will not give here, can be found in
Levy \cite[pp.~357-366]{Levy} supplementing his informal introduction to class
theory on pp.~7-17. Two other good treatments of class theory are provided
by Quine \cite[pp.~15-21]{Quine}\index{Quine, Willard Van Orman}
and also \cite[pp.~10-14]{Takeuti}\index{Takeuti, Gaisi}.
Quine's exposition (he calls them virtual classes)
is nicely written and very readable.

In the rest of this
section, individual variables are always assumed to be distinct from
each other unless otherwise indicated.  In addition, dummy variables on the
right-hand side of a definition do not occur in the class and wff
metavariables in the definition.

The definitions we present here are a partial but self-contained
collection selected from several hundred that appear in the current
\texttt{set.mm} database.  They are adequate for a basic development of
elementary set theory.

\vskip 2ex
\noindent Define the abstraction class.\index{abstraction class}\index{class
abstraction}\label{df-clab}  $x$ and $y$
need not be distinct.  Definition 2.1 of Quine, p.~16.  This definition may
seem puzzling since it is shorter than the expression being defined and does not
buy us anything in terms of brevity.  The reason we introduce this definition
is because it fits in neatly with the extension of the $\in$ connective
provided by \texttt{df-clel}.

\vskip 0.5ex
\setbox\startprefix=\hbox{\tt \ \ df-clab\ \$a\ }
\setbox\contprefix=\hbox{\tt \ \ \ \ \ \ \ \ \ \ \ \ \ }
\startm
\m{\vdash}\m{(}\m{x}\m{\in}\m{\{}\m{y}\m{|}\m{\varphi}\m{\}}\m{%
\leftrightarrow}\m{[}\m{x}\m{/}\m{y}\m{]}\m{\varphi}\m{)}
\endm
\begin{mmraw}%
|- ( x e. \{ y | ph \} <-> [ x / y ] ph ) \$.
\end{mmraw}

\noindent Define the equality connective between classes\index{class
equality}\label{df-cleq}.  See Quine or Chapter 4 of Takeuti and Zaring for its
justification and methods for eliminating it.  This is an example of a
somewhat ``dangerous'' definition, because it extends the use of the
existing equality symbol rather than introducing a new symbol, allowing
us to make statements in the original language that may not be true.
For example, it permits us to deduce $y = z \leftrightarrow \forall x (
x \in y \leftrightarrow x \in z )$ which is not a theorem of logic but
rather presupposes the Axiom of Extensionality,\index{Axiom of
Extensionality} which we include as a hypothesis so that we can know
when this axiom is assumed in a proof (with the \texttt{show
trace{\char`\_}back} command).  We could avoid the danger by introducing
another symbol, say $\eqcirc$, in place of $=$; this
would also have the advantage of making elimination of the definition
straightforward and would eliminate the need for Extensionality as a
hypothesis.  We would then also have the advantage of being able to
identify exactly where Extensionality truly comes into play.  One of our
theorems would be $x \eqcirc y \leftrightarrow x = y$
by invoking Extensionality.  However in keeping with standard practice
we retain the ``dangerous'' definition.

\vskip 0.5ex
\setbox\startprefix=\hbox{\tt \ \ df-cleq.1\ \$e\ }
\setbox\contprefix=\hbox{\tt \ \ \ \ \ \ \ \ \ \ \ \ \ \ \ }
\startm
\m{\vdash}\m{(}\m{\forall}\m{x}\m{(}\m{x}\m{\in}\m{y}\m{\leftrightarrow}\m{x}
\m{\in}\m{z}\m{)}\m{\rightarrow}\m{y}\m{=}\m{z}\m{)}
\endm
\setbox\startprefix=\hbox{\tt \ \ df-cleq\ \$a\ }
\setbox\contprefix=\hbox{\tt \ \ \ \ \ \ \ \ \ \ \ \ \ }
\startm
\m{\vdash}\m{(}\m{A}\m{=}\m{B}\m{\leftrightarrow}\m{\forall}\m{x}\m{(}\m{x}\m{
\in}\m{A}\m{\leftrightarrow}\m{x}\m{\in}\m{B}\m{)}\m{)}
\endm
% We need to reset the startprefix and contprefix.
\setbox\startprefix=\hbox{\tt \ \ df-cleq.1\ \$e\ }
\setbox\contprefix=\hbox{\tt \ \ \ \ \ \ \ \ \ \ \ \ \ \ \ }
\begin{mmraw}%
|- ( A. x ( x e. y <-> x e. z ) -> y = z ) \$.
\end{mmraw}
\setbox\startprefix=\hbox{\tt \ \ df-cleq\ \$a\ }
\setbox\contprefix=\hbox{\tt \ \ \ \ \ \ \ \ \ \ \ \ \ }
\begin{mmraw}%
|- ( A = B <-> A. x ( x e. A <-> x e. B ) ) \$.
\end{mmraw}

\noindent Define the membership connective between classes\index{class
membership}.  Theorem 6.3 of Quine, p.~41, which we adopt as a definition.
Note that it extends the use of the existing membership symbol, but unlike
{\tt df-cleq} it does not extend the set of valid wffs of logic when the class
metavariables are replaced with set variables.\label{dfclel}\label{df-clel}

\vskip 0.5ex
\setbox\startprefix=\hbox{\tt \ \ df-clel\ \$a\ }
\setbox\contprefix=\hbox{\tt \ \ \ \ \ \ \ \ \ \ \ \ \ }
\startm
\m{\vdash}\m{(}\m{A}\m{\in}\m{B}\m{\leftrightarrow}\m{\exists}\m{x}\m{(}\m{x}
\m{=}\m{A}\m{\wedge}\m{x}\m{\in}\m{B}\m{)}\m{)}
\endm
\begin{mmraw}%
|- ( A e. B <-> E. x ( x = A \TAND x e. B ) ) \$.?
\end{mmraw}

\noindent Define inequality.

\vskip 0.5ex
\setbox\startprefix=\hbox{\tt \ \ df-ne\ \$a\ }
\setbox\contprefix=\hbox{\tt \ \ \ \ \ \ \ \ \ \ \ }
\startm
\m{\vdash}\m{(}\m{A}\m{\ne}\m{B}\m{\leftrightarrow}\m{\lnot}\m{A}\m{=}\m{B}%
\m{)}
\endm
\begin{mmraw}%
|- ( A =/= B <-> -. A = B ) \$.
\end{mmraw}

\noindent Define restricted universal quantification.\index{universal
quantifier ($\forall$)!restricted}  Enderton, p.~22.

\vskip 0.5ex
\setbox\startprefix=\hbox{\tt \ \ df-ral\ \$a\ }
\setbox\contprefix=\hbox{\tt \ \ \ \ \ \ \ \ \ \ \ \ }
\startm
\m{\vdash}\m{(}\m{\forall}\m{x}\m{\in}\m{A}\m{\varphi}\m{\leftrightarrow}\m{%
\forall}\m{x}\m{(}\m{x}\m{\in}\m{A}\m{\rightarrow}\m{\varphi}\m{)}\m{)}
\endm
\begin{mmraw}%
|- ( A. x e. A ph <-> A. x ( x e. A -> ph ) ) \$.
\end{mmraw}

\noindent Define restricted existential quantification.\index{existential
quantifier ($\exists$)!restricted}  Enderton, p.~22.

\vskip 0.5ex
\setbox\startprefix=\hbox{\tt \ \ df-rex\ \$a\ }
\setbox\contprefix=\hbox{\tt \ \ \ \ \ \ \ \ \ \ \ \ }
\startm
\m{\vdash}\m{(}\m{\exists}\m{x}\m{\in}\m{A}\m{\varphi}\m{\leftrightarrow}\m{%
\exists}\m{x}\m{(}\m{x}\m{\in}\m{A}\m{\wedge}\m{\varphi}\m{)}\m{)}
\endm
\begin{mmraw}%
|- ( E. x e. A ph <-> E. x ( x e. A \TAND ph ) ) \$.
\end{mmraw}

\noindent Define the universal class\index{universal class ($V$)}.  Definition
5.20, p.~21, of Takeuti and Zaring.\label{df-v}

\vskip 0.5ex
\setbox\startprefix=\hbox{\tt \ \ df-v\ \$a\ }
\setbox\contprefix=\hbox{\tt \ \ \ \ \ \ \ \ \ \ }
\startm
\m{\vdash}\m{{\rm V}}\m{=}\m{\{}\m{x}\m{|}\m{x}\m{=}\m{x}\m{\}}
\endm
\begin{mmraw}%
|- {\char`\_}V = \{ x | x = x \} \$.
\end{mmraw}

\noindent Define the subclass\index{subclass}\index{subset} relationship
between two classes (called the subset relation if the classes are sets i.e.\
are not proper).  Definition 5.9 of Takeuti and Zaring, p.~17.\label{df-ss}

\vskip 0.5ex
\setbox\startprefix=\hbox{\tt \ \ df-ss\ \$a\ }
\setbox\contprefix=\hbox{\tt \ \ \ \ \ \ \ \ \ \ \ }
\startm
\m{\vdash}\m{(}\m{A}\m{\subseteq}\m{B}\m{\leftrightarrow}\m{\forall}\m{x}\m{(}
\m{x}\m{\in}\m{A}\m{\rightarrow}\m{x}\m{\in}\m{B}\m{)}\m{)}
\endm
\begin{mmraw}%
|- ( A C\_ B <-> A. x ( x e. A -> x e. B ) ) \$.
\end{mmraw}

\noindent Define the union\index{union} of two classes.  Definition 5.6 of Takeuti and Zaring,
p.~16.\label{df-un}

\vskip 0.5ex
\setbox\startprefix=\hbox{\tt \ \ df-un\ \$a\ }
\setbox\contprefix=\hbox{\tt \ \ \ \ \ \ \ \ \ \ \ }
\startm
\m{\vdash}\m{(}\m{A}\m{\cup}\m{B}\m{)}\m{=}\m{\{}\m{x}\m{|}\m{(}\m{x}\m{\in}
\m{A}\m{\vee}\m{x}\m{\in}\m{B}\m{)}\m{\}}
\endm
\begin{mmraw}%
( A u. B ) = \{ x | ( x e. A \TOR x e. B ) \} \$.
\end{mmraw}

\noindent Define the intersection\index{intersection} of two classes.  Definition 5.6 of
Takeuti and Zaring, p.~16.\label{df-in}

\vskip 0.5ex
\setbox\startprefix=\hbox{\tt \ \ df-in\ \$a\ }
\setbox\contprefix=\hbox{\tt \ \ \ \ \ \ \ \ \ \ \ }
\startm
\m{\vdash}\m{(}\m{A}\m{\cap}\m{B}\m{)}\m{=}\m{\{}\m{x}\m{|}\m{(}\m{x}\m{\in}
\m{A}\m{\wedge}\m{x}\m{\in}\m{B}\m{)}\m{\}}
\endm
% Caret ^ requires special treatment
\begin{mmraw}%
|- ( A i\^{}i B ) = \{ x | ( x e. A \TAND x e. B ) \} \$.
\end{mmraw}

\noindent Define class difference\index{class difference}\index{set difference}.
Definition 5.12 of Takeuti and Zaring, p.~20.  Several notations are used in
the literature; we chose the $\setminus$ convention instead of a minus sign to
reserve the latter for later use in, e.g., arithmetic.\label{df-dif}

\vskip 0.5ex
\setbox\startprefix=\hbox{\tt \ \ df-dif\ \$a\ }
\setbox\contprefix=\hbox{\tt \ \ \ \ \ \ \ \ \ \ \ \ }
\startm
\m{\vdash}\m{(}\m{A}\m{\setminus}\m{B}\m{)}\m{=}\m{\{}\m{x}\m{|}\m{(}\m{x}\m{
\in}\m{A}\m{\wedge}\m{\lnot}\m{x}\m{\in}\m{B}\m{)}\m{\}}
\endm
\begin{mmraw}%
( A \SLASH B ) = \{ x | ( x e. A \TAND -. x e. B ) \} \$.
\end{mmraw}

\noindent Define the empty or null set\index{empty set}\index{null set}.
Compare  Definition 5.14 of Takeuti and Zaring, p.~20.\label{df-nul}

\vskip 0.5ex
\setbox\startprefix=\hbox{\tt \ \ df-nul\ \$a\ }
\setbox\contprefix=\hbox{\tt \ \ \ \ \ \ \ \ \ \ }
\startm
\m{\vdash}\m{\varnothing}\m{=}\m{(}\m{{\rm V}}\m{\setminus}\m{{\rm V}}\m{)}
\endm
\begin{mmraw}%
|- (/) = ( {\char`\_}V \SLASH {\char`\_}V ) \$.
\end{mmraw}

\noindent Define power class\index{power set}\index{power class}.  Definition 5.10 of
Takeuti and Zaring, p.~17, but we also let it apply to proper classes.  (Note
that \verb$~P$ is the symbol for calligraphic P, the tilde
suggesting ``curly;'' see Appendix~\ref{ASCII}.)\label{df-pw}

\vskip 0.5ex
\setbox\startprefix=\hbox{\tt \ \ df-pw\ \$a\ }
\setbox\contprefix=\hbox{\tt \ \ \ \ \ \ \ \ \ \ \ }
\startm
\m{\vdash}\m{{\cal P}}\m{A}\m{=}\m{\{}\m{x}\m{|}\m{x}\m{\subseteq}\m{A}\m{\}}
\endm
% Special incantation required to put ~ into the text
\begin{mmraw}%
|- \char`\~P~A = \{ x | x C\_ A \} \$.
\end{mmraw}

\noindent Define the singleton of a class\index{singleton}.  Definition 7.1 of
Quine, p.~48.  It is well-defined for proper classes, although
it is not very meaningful in this case, where it evaluates to the empty
set.

\vskip 0.5ex
\setbox\startprefix=\hbox{\tt \ \ df-sn\ \$a\ }
\setbox\contprefix=\hbox{\tt \ \ \ \ \ \ \ \ \ \ \ }
\startm
\m{\vdash}\m{\{}\m{A}\m{\}}\m{=}\m{\{}\m{x}\m{|}\m{x}\m{=}\m{A}\m{\}}
\endm
\begin{mmraw}%
|- \{ A \} = \{ x | x = A \} \$.
\end{mmraw}%

\noindent Define an unordered pair of classes\index{unordered pair}\index{pair}.  Definition
7.1 of Quine, p.~48.

\vskip 0.5ex
\setbox\startprefix=\hbox{\tt \ \ df-pr\ \$a\ }
\setbox\contprefix=\hbox{\tt \ \ \ \ \ \ \ \ \ \ \ }
\startm
\m{\vdash}\m{\{}\m{A}\m{,}\m{B}\m{\}}\m{=}\m{(}\m{\{}\m{A}\m{\}}\m{\cup}\m{\{}
\m{B}\m{\}}\m{)}
\endm
\begin{mmraw}%
|- \{ A , B \} = ( \{ A \} u. \{ B \} ) \$.
\end{mmraw}

\noindent Define an unordered triple of classes\index{unordered triple}.  Definition of
Enderton, p.~19.

\vskip 0.5ex
\setbox\startprefix=\hbox{\tt \ \ df-tp\ \$a\ }
\setbox\contprefix=\hbox{\tt \ \ \ \ \ \ \ \ \ \ \ }
\startm
\m{\vdash}\m{\{}\m{A}\m{,}\m{B}\m{,}\m{C}\m{\}}\m{=}\m{(}\m{\{}\m{A}\m{,}\m{B}
\m{\}}\m{\cup}\m{\{}\m{C}\m{\}}\m{)}
\endm
\begin{mmraw}%
|- \{ A , B , C \} = ( \{ A , B \} u. \{ C \} ) \$.
\end{mmraw}%

\noindent Kuratowski's\index{Kuratowski, Kazimierz} ordered pair\index{ordered
pair} definition.  Definition 9.1 of Quine, p.~58. For proper classes it is
not meaningful but is well-defined for convenience.  (Note that \verb$<.$
stands for $\langle$ whereas \verb$<$ stands for $<$, and similarly for
\verb$>.$\,.)\label{df-op}

\vskip 0.5ex
\setbox\startprefix=\hbox{\tt \ \ df-op\ \$a\ }
\setbox\contprefix=\hbox{\tt \ \ \ \ \ \ \ \ \ \ \ }
\startm
\m{\vdash}\m{\langle}\m{A}\m{,}\m{B}\m{\rangle}\m{=}\m{\{}\m{\{}\m{A}\m{\}}
\m{,}\m{\{}\m{A}\m{,}\m{B}\m{\}}\m{\}}
\endm
\begin{mmraw}%
|- <. A , B >. = \{ \{ A \} , \{ A , B \} \} \$.
\end{mmraw}

\noindent Define the union of a class\index{union}.  Definition 5.5, p.~16,
of Takeuti and Zaring.

\vskip 0.5ex
\setbox\startprefix=\hbox{\tt \ \ df-uni\ \$a\ }
\setbox\contprefix=\hbox{\tt \ \ \ \ \ \ \ \ \ \ \ \ }
\startm
\m{\vdash}\m{\bigcup}\m{A}\m{=}\m{\{}\m{x}\m{|}\m{\exists}\m{y}\m{(}\m{x}\m{
\in}\m{y}\m{\wedge}\m{y}\m{\in}\m{A}\m{)}\m{\}}
\endm
\begin{mmraw}%
|- U. A = \{ x | E. y ( x e. y \TAND y e. A ) \} \$.
\end{mmraw}

\noindent Define the intersection\index{intersection} of a class.  Definition 7.35,
p.~44, of Takeuti and Zaring.

\vskip 0.5ex
\setbox\startprefix=\hbox{\tt \ \ df-int\ \$a\ }
\setbox\contprefix=\hbox{\tt \ \ \ \ \ \ \ \ \ \ \ \ }
\startm
\m{\vdash}\m{\bigcap}\m{A}\m{=}\m{\{}\m{x}\m{|}\m{\forall}\m{y}\m{(}\m{y}\m{
\in}\m{A}\m{\rightarrow}\m{x}\m{\in}\m{y}\m{)}\m{\}}
\endm
\begin{mmraw}%
|- |\^{}| A = \{ x | A. y ( y e. A -> x e. y ) \} \$.
\end{mmraw}

\noindent Define a transitive class\index{transitive class}\index{transitive
set}.  This should not be confused with a transitive relation, which is a different
concept.  Definition from p.~71 of Enderton, extended to classes.

\vskip 0.5ex
\setbox\startprefix=\hbox{\tt \ \ df-tr\ \$a\ }
\setbox\contprefix=\hbox{\tt \ \ \ \ \ \ \ \ \ \ \ }
\startm
\m{\vdash}\m{(}\m{\mbox{\rm Tr}}\m{A}\m{\leftrightarrow}\m{\bigcup}\m{A}\m{
\subseteq}\m{A}\m{)}
\endm
\begin{mmraw}%
|- ( Tr A <-> U. A C\_ A ) \$.
\end{mmraw}

\noindent Define a notation for a general binary relation\index{binary
relation}.  Definition 6.18, p.~29, of Takeuti and Zaring, generalized to
arbitrary classes.  This definition is well-defined, although not very
meaningful, when classes $A$ and/or $B$ are proper.\label{dfbr}  The lack of
parentheses (or any other connective) creates no ambiguity since we are defining
an atomic wff.

\vskip 0.5ex
\setbox\startprefix=\hbox{\tt \ \ df-br\ \$a\ }
\setbox\contprefix=\hbox{\tt \ \ \ \ \ \ \ \ \ \ \ }
\startm
\m{\vdash}\m{(}\m{A}\m{\,R}\m{\,B}\m{\leftrightarrow}\m{\langle}\m{A}\m{,}\m{B}
\m{\rangle}\m{\in}\m{R}\m{)}
\endm
\begin{mmraw}%
|- ( A R B <-> <. A , B >. e. R ) \$.
\end{mmraw}

\noindent Define an abstraction class of ordered pairs\index{abstraction
class!of ordered
pairs}.  A special case of Definition 4.16, p.~14, of Takeuti and Zaring.
Note that $ z $ must be distinct from $ x $ and $ y $,
and $ z $ must not occur in $\varphi$, but $ x $ and $ y $ may be
identical and may appear in $\varphi$.

\vskip 0.5ex
\setbox\startprefix=\hbox{\tt \ \ df-opab\ \$a\ }
\setbox\contprefix=\hbox{\tt \ \ \ \ \ \ \ \ \ \ \ \ \ }
\startm
\m{\vdash}\m{\{}\m{\langle}\m{x}\m{,}\m{y}\m{\rangle}\m{|}\m{\varphi}\m{\}}\m{=}
\m{\{}\m{z}\m{|}\m{\exists}\m{x}\m{\exists}\m{y}\m{(}\m{z}\m{=}\m{\langle}\m{x}
\m{,}\m{y}\m{\rangle}\m{\wedge}\m{\varphi}\m{)}\m{\}}
\endm

\begin{mmraw}%
|- \{ <. x , y >. | ph \} = \{ z | E. x E. y ( z =
<. x , y >. /\ ph ) \} \$.
\end{mmraw}

\noindent Define the epsilon relation\index{epsilon relation}.  Similar to Definition
6.22, p.~30, of Takeuti and Zaring.

\vskip 0.5ex
\setbox\startprefix=\hbox{\tt \ \ df-eprel\ \$a\ }
\setbox\contprefix=\hbox{\tt \ \ \ \ \ \ \ \ \ \ \ \ \ \ }
\startm
\m{\vdash}\m{{\rm E}}\m{=}\m{\{}\m{\langle}\m{x}\m{,}\m{y}\m{\rangle}\m{|}\m{x}\m{
\in}\m{y}\m{\}}
\endm
\begin{mmraw}%
|- \_E = \{ <. x , y >. | x e. y \} \$.
\end{mmraw}

\noindent Define a founded relation\index{founded relation}.  $R$ is a founded
relation on $A$ iff\index{iff} (if and only if) each nonempty subset of $A$
has an ``$R$-minimal element.''  Similar to Definition 6.21, p.~30, of
Takeuti and Zaring.

\vskip 0.5ex
\setbox\startprefix=\hbox{\tt \ \ df-fr\ \$a\ }
\setbox\contprefix=\hbox{\tt \ \ \ \ \ \ \ \ \ \ \ }
\startm
\m{\vdash}\m{(}\m{R}\m{\,\mbox{\rm Fr}}\m{\,A}\m{\leftrightarrow}\m{\forall}\m{x}
\m{(}\m{(}\m{x}\m{\subseteq}\m{A}\m{\wedge}\m{\lnot}\m{x}\m{=}\m{\varnothing}
\m{)}\m{\rightarrow}\m{\exists}\m{y}\m{(}\m{y}\m{\in}\m{x}\m{\wedge}\m{(}\m{x}
\m{\cap}\m{\{}\m{z}\m{|}\m{z}\m{\,R}\m{\,y}\m{\}}\m{)}\m{=}\m{\varnothing}\m{)}
\m{)}\m{)}
\endm
\begin{mmraw}%
|- ( R Fr A <-> A. x ( ( x C\_ A \TAND -. x = (/) ) ->
E. y ( y e. x \TAND ( x i\^{}i \{ z | z R y \} ) = (/) ) ) ) \$.
\end{mmraw}

\noindent Define a well-ordering\index{well-ordering}.  $R$ is a well-ordering of $A$ iff
it is founded on $A$ and the elements of $A$ are pairwise $R$-comparable.
Similar to Definition 6.24(2), p.~30, of Takeuti and Zaring.

\vskip 0.5ex
\setbox\startprefix=\hbox{\tt \ \ df-we\ \$a\ }
\setbox\contprefix=\hbox{\tt \ \ \ \ \ \ \ \ \ \ \ }
\startm
\m{\vdash}\m{(}\m{R}\m{\,\mbox{\rm We}}\m{\,A}\m{\leftrightarrow}\m{(}\m{R}\m{\,
\mbox{\rm Fr}}\m{\,A}\m{\wedge}\m{\forall}\m{x}\m{\forall}\m{y}\m{(}\m{(}\m{x}\m{
\in}\m{A}\m{\wedge}\m{y}\m{\in}\m{A}\m{)}\m{\rightarrow}\m{(}\m{x}\m{\,R}\m{\,y}
\m{\vee}\m{x}\m{=}\m{y}\m{\vee}\m{y}\m{\,R}\m{\,x}\m{)}\m{)}\m{)}\m{)}
\endm
\begin{mmraw}%
( R We A <-> ( R Fr A \TAND A. x A. y ( ( x e.
A \TAND y e. A ) -> ( x R y \TOR x = y \TOR y R x ) ) ) ) \$.
\end{mmraw}

\noindent Define the ordinal predicate\index{ordinal predicate}, which is true for a class
that is transitive and is well-ordered by the epsilon relation.  Similar to
definition on p.~468, Bell and Machover.

\vskip 0.5ex
\setbox\startprefix=\hbox{\tt \ \ df-ord\ \$a\ }
\setbox\contprefix=\hbox{\tt \ \ \ \ \ \ \ \ \ \ \ \ }
\startm
\m{\vdash}\m{(}\m{\mbox{\rm Ord}}\m{\,A}\m{\leftrightarrow}\m{(}
\m{\mbox{\rm Tr}}\m{\,A}\m{\wedge}\m{E}\m{\,\mbox{\rm We}}\m{\,A}\m{)}\m{)}
\endm
\begin{mmraw}%
|- ( Ord A <-> ( Tr A \TAND E We A ) ) \$.
\end{mmraw}

\noindent Define the class of all ordinal numbers\index{ordinal number}.  An ordinal number is
a set that satisfies the ordinal predicate.  Definition 7.11 of Takeuti and
Zaring, p.~38.

\vskip 0.5ex
\setbox\startprefix=\hbox{\tt \ \ df-on\ \$a\ }
\setbox\contprefix=\hbox{\tt \ \ \ \ \ \ \ \ \ \ \ }
\startm
\m{\vdash}\m{\,\mbox{\rm On}}\m{=}\m{\{}\m{x}\m{|}\m{\mbox{\rm Ord}}\m{\,x}
\m{\}}
\endm
\begin{mmraw}%
|- On = \{ x | Ord x \} \$.
\end{mmraw}

\noindent Define the limit ordinal predicate\index{limit ordinal}, which is true for a
non-empty ordinal that is not a successor (i.e.\ that is the union of itself).
Compare Bell and Machover, p.~471 and Exercise (1), p.~42 of Takeuti and
Zaring.

\vskip 0.5ex
\setbox\startprefix=\hbox{\tt \ \ df-lim\ \$a\ }
\setbox\contprefix=\hbox{\tt \ \ \ \ \ \ \ \ \ \ \ \ }
\startm
\m{\vdash}\m{(}\m{\mbox{\rm Lim}}\m{\,A}\m{\leftrightarrow}\m{(}\m{\mbox{
\rm Ord}}\m{\,A}\m{\wedge}\m{\lnot}\m{A}\m{=}\m{\varnothing}\m{\wedge}\m{A}
\m{=}\m{\bigcup}\m{A}\m{)}\m{)}
\endm
\begin{mmraw}%
|- ( Lim A <-> ( Ord A \TAND -. A = (/) \TAND A = U. A ) ) \$.
\end{mmraw}

\noindent Define the successor\index{successor} of a class.  Definition 7.22 of Takeuti
and Zaring, p.~41.  Our definition is a generalization to classes, although it
is meaningless when classes are proper.

\vskip 0.5ex
\setbox\startprefix=\hbox{\tt \ \ df-suc\ \$a\ }
\setbox\contprefix=\hbox{\tt \ \ \ \ \ \ \ \ \ \ \ \ }
\startm
\m{\vdash}\m{\,\mbox{\rm suc}}\m{\,A}\m{=}\m{(}\m{A}\m{\cup}\m{\{}\m{A}\m{\}}
\m{)}
\endm
\begin{mmraw}%
|- suc A = ( A u. \{ A \} ) \$.
\end{mmraw}

\noindent Define the class of natural numbers\index{natural number}\index{omega
($\omega$)}.  Compare Bell and Machover, p.~471.\label{dfom}

\vskip 0.5ex
\setbox\startprefix=\hbox{\tt \ \ df-om\ \$a\ }
\setbox\contprefix=\hbox{\tt \ \ \ \ \ \ \ \ \ \ \ }
\startm
\m{\vdash}\m{\omega}\m{=}\m{\{}\m{x}\m{|}\m{(}\m{\mbox{\rm Ord}}\m{\,x}\m{
\wedge}\m{\forall}\m{y}\m{(}\m{\mbox{\rm Lim}}\m{\,y}\m{\rightarrow}\m{x}\m{
\in}\m{y}\m{)}\m{)}\m{\}}
\endm
\begin{mmraw}%
|- om = \{ x | ( Ord x \TAND A. y ( Lim y -> x e. y ) ) \} \$.
\end{mmraw}

\noindent Define the Cartesian product (also called the
cross product)\index{Cartesian product}\index{cross product}
of two classes.  Definition 9.11 of Quine, p.~64.

\vskip 0.5ex
\setbox\startprefix=\hbox{\tt \ \ df-xp\ \$a\ }
\setbox\contprefix=\hbox{\tt \ \ \ \ \ \ \ \ \ \ \ }
\startm
\m{\vdash}\m{(}\m{A}\m{\times}\m{B}\m{)}\m{=}\m{\{}\m{\langle}\m{x}\m{,}\m{y}
\m{\rangle}\m{|}\m{(}\m{x}\m{\in}\m{A}\m{\wedge}\m{y}\m{\in}\m{B}\m{)}\m{\}}
\endm
\begin{mmraw}%
|- ( A X. B ) = \{ <. x , y >. | ( x e. A \TAND y e. B) \} \$.
\end{mmraw}

\noindent Define a relation\index{relation}.  Definition 6.4(1) of Takeuti and
Zaring, p.~23.

\vskip 0.5ex
\setbox\startprefix=\hbox{\tt \ \ df-rel\ \$a\ }
\setbox\contprefix=\hbox{\tt \ \ \ \ \ \ \ \ \ \ \ \ }
\startm
\m{\vdash}\m{(}\m{\mbox{\rm Rel}}\m{\,A}\m{\leftrightarrow}\m{A}\m{\subseteq}
\m{(}\m{{\rm V}}\m{\times}\m{{\rm V}}\m{)}\m{)}
\endm
\begin{mmraw}%
|- ( Rel A <-> A C\_ ( {\char`\_}V X. {\char`\_}V ) ) \$.
\end{mmraw}

\noindent Define the domain\index{domain} of a class.  Definition 6.5(1) of
Takeuti and Zaring, p.~24.

\vskip 0.5ex
\setbox\startprefix=\hbox{\tt \ \ df-dm\ \$a\ }
\setbox\contprefix=\hbox{\tt \ \ \ \ \ \ \ \ \ \ \ }
\startm
\m{\vdash}\m{\,\mbox{\rm dom}}\m{A}\m{=}\m{\{}\m{x}\m{|}\m{\exists}\m{y}\m{
\langle}\m{x}\m{,}\m{y}\m{\rangle}\m{\in}\m{A}\m{\}}
\endm
\begin{mmraw}%
|- dom A = \{ x | E. y <. x , y >. e. A \} \$.
\end{mmraw}

\noindent Define the range\index{range} of a class.  Definition 6.5(2) of
Takeuti and Zaring, p.~24.

\vskip 0.5ex
\setbox\startprefix=\hbox{\tt \ \ df-rn\ \$a\ }
\setbox\contprefix=\hbox{\tt \ \ \ \ \ \ \ \ \ \ \ }
\startm
\m{\vdash}\m{\,\mbox{\rm ran}}\m{A}\m{=}\m{\{}\m{y}\m{|}\m{\exists}\m{x}\m{
\langle}\m{x}\m{,}\m{y}\m{\rangle}\m{\in}\m{A}\m{\}}
\endm
\begin{mmraw}%
|- ran A = \{ y | E. x <. x , y >. e. A \} \$.
\end{mmraw}

\noindent Define the restriction\index{restriction} of a class.  Definition
6.6(1) of Takeuti and Zaring, p.~24.

\vskip 0.5ex
\setbox\startprefix=\hbox{\tt \ \ df-res\ \$a\ }
\setbox\contprefix=\hbox{\tt \ \ \ \ \ \ \ \ \ \ \ \ }
\startm
\m{\vdash}\m{(}\m{A}\m{\restriction}\m{B}\m{)}\m{=}\m{(}\m{A}\m{\cap}\m{(}\m{B}
\m{\times}\m{{\rm V}}\m{)}\m{)}
\endm
\begin{mmraw}%
|- ( A |` B ) = ( A i\^{}i ( B X. {\char`\_}V ) ) \$.
\end{mmraw}

\noindent Define the image\index{image} of a class.  Definition 6.6(2) of
Takeuti and Zaring, p.~24.

\vskip 0.5ex
\setbox\startprefix=\hbox{\tt \ \ df-ima\ \$a\ }
\setbox\contprefix=\hbox{\tt \ \ \ \ \ \ \ \ \ \ \ \ }
\startm
\m{\vdash}\m{(}\m{A}\m{``}\m{B}\m{)}\m{=}\m{\,\mbox{\rm ran}}\m{\,(}\m{A}\m{
\restriction}\m{B}\m{)}
\endm
\begin{mmraw}%
|- ( A " B ) = ran ( A |` B ) \$.
\end{mmraw}

\noindent Define the composition\index{composition} of two classes.  Definition 6.6(3) of
Takeuti and Zaring, p.~24.

\vskip 0.5ex
\setbox\startprefix=\hbox{\tt \ \ df-co\ \$a\ }
\setbox\contprefix=\hbox{\tt \ \ \ \ \ \ \ \ \ \ \ \ }
\startm
\m{\vdash}\m{(}\m{A}\m{\circ}\m{B}\m{)}\m{=}\m{\{}\m{\langle}\m{x}\m{,}\m{y}\m{
\rangle}\m{|}\m{\exists}\m{z}\m{(}\m{\langle}\m{x}\m{,}\m{z}\m{\rangle}\m{\in}
\m{B}\m{\wedge}\m{\langle}\m{z}\m{,}\m{y}\m{\rangle}\m{\in}\m{A}\m{)}\m{\}}
\endm
\begin{mmraw}%
|- ( A o. B ) = \{ <. x , y >. | E. z ( <. x , z
>. e. B \TAND <. z , y >. e. A ) \} \$.
\end{mmraw}

\noindent Define a function\index{function}.  Definition 6.4(4) of Takeuti and
Zaring, p.~24.

\vskip 0.5ex
\setbox\startprefix=\hbox{\tt \ \ df-fun\ \$a\ }
\setbox\contprefix=\hbox{\tt \ \ \ \ \ \ \ \ \ \ \ \ }
\startm
\m{\vdash}\m{(}\m{\mbox{\rm Fun}}\m{\,A}\m{\leftrightarrow}\m{(}
\m{\mbox{\rm Rel}}\m{\,A}\m{\wedge}
\m{\forall}\m{x}\m{\exists}\m{z}\m{\forall}\m{y}\m{(}
\m{\langle}\m{x}\m{,}\m{y}\m{\rangle}\m{\in}\m{A}\m{\rightarrow}\m{y}\m{=}\m{z}
\m{)}\m{)}\m{)}
\endm
\begin{mmraw}%
|- ( Fun A <-> ( Rel A /\ A. x E. z A. y ( <. x
   , y >. e. A -> y = z ) ) ) \$.
\end{mmraw}

\noindent Define a function with domain.  Definition 6.15(1) of Takeuti and
Zaring, p.~27.

\vskip 0.5ex
\setbox\startprefix=\hbox{\tt \ \ df-fn\ \$a\ }
\setbox\contprefix=\hbox{\tt \ \ \ \ \ \ \ \ \ \ \ }
\startm
\m{\vdash}\m{(}\m{A}\m{\,\mbox{\rm Fn}}\m{\,B}\m{\leftrightarrow}\m{(}
\m{\mbox{\rm Fun}}\m{\,A}\m{\wedge}\m{\mbox{\rm dom}}\m{\,A}\m{=}\m{B}\m{)}
\m{)}
\endm
\begin{mmraw}%
|- ( A Fn B <-> ( Fun A \TAND dom A = B ) ) \$.
\end{mmraw}

\noindent Define a function with domain and co-domain.  Definition 6.15(3)
of Takeuti and Zaring, p.~27.

\vskip 0.5ex
\setbox\startprefix=\hbox{\tt \ \ df-f\ \$a\ }
\setbox\contprefix=\hbox{\tt \ \ \ \ \ \ \ \ \ \ }
\startm
\m{\vdash}\m{(}\m{F}\m{:}\m{A}\m{\longrightarrow}\m{B}\m{
\leftrightarrow}\m{(}\m{F}\m{\,\mbox{\rm Fn}}\m{\,A}\m{\wedge}\m{
\mbox{\rm ran}}\m{\,F}\m{\subseteq}\m{B}\m{)}\m{)}
\endm
\begin{mmraw}%
|- ( F : A --> B <-> ( F Fn A \TAND ran F C\_ B ) ) \$.
\end{mmraw}

\noindent Define a one-to-one function\index{one-to-one function}.  Compare
Definition 6.15(5) of Takeuti and Zaring, p.~27.

\vskip 0.5ex
\setbox\startprefix=\hbox{\tt \ \ df-f1\ \$a\ }
\setbox\contprefix=\hbox{\tt \ \ \ \ \ \ \ \ \ \ \ }
\startm
\m{\vdash}\m{(}\m{F}\m{:}\m{A}\m{
\raisebox{.5ex}{${\textstyle{\:}_{\mbox{\footnotesize\rm
1\tt -\rm 1}}}\atop{\textstyle{
\longrightarrow}\atop{\textstyle{}^{\mbox{\footnotesize\rm {\ }}}}}$}
}\m{B}
\m{\leftrightarrow}\m{(}\m{F}\m{:}\m{A}\m{\longrightarrow}\m{B}
\m{\wedge}\m{\forall}\m{y}\m{\exists}\m{z}\m{\forall}\m{x}\m{(}\m{\langle}\m{x}
\m{,}\m{y}\m{\rangle}\m{\in}\m{F}\m{\rightarrow}\m{x}\m{=}\m{z}\m{)}\m{)}\m{)}
\endm
\begin{mmraw}%
|- ( F : A -1-1-> B <-> ( F : A --> B \TAND
   A. y E. z A. x ( <. x , y >. e. F -> x = z ) ) ) \$.
\end{mmraw}

\noindent Define an onto function\index{onto function}.  Definition 6.15(4) of Takeuti and
Zaring, p.~27.

\vskip 0.5ex
\setbox\startprefix=\hbox{\tt \ \ df-fo\ \$a\ }
\setbox\contprefix=\hbox{\tt \ \ \ \ \ \ \ \ \ \ \ }
\startm
\m{\vdash}\m{(}\m{F}\m{:}\m{A}\m{
\raisebox{.5ex}{${\textstyle{\:}_{\mbox{\footnotesize\rm
{\ }}}}\atop{\textstyle{
\longrightarrow}\atop{\textstyle{}^{\mbox{\footnotesize\rm onto}}}}$}
}\m{B}
\m{\leftrightarrow}\m{(}\m{F}\m{\,\mbox{\rm Fn}}\m{\,A}\m{\wedge}
\m{\mbox{\rm ran}}\m{\,F}\m{=}\m{B}\m{)}\m{)}
\endm
\begin{mmraw}%
|- ( F : A -onto-> B <-> ( F Fn A /\ ran F = B ) ) \$.
\end{mmraw}

\noindent Define a one-to-one, onto function.  Compare Definition 6.15(6) of
Takeuti and Zaring, p.~27.

\vskip 0.5ex
\setbox\startprefix=\hbox{\tt \ \ df-f1o\ \$a\ }
\setbox\contprefix=\hbox{\tt \ \ \ \ \ \ \ \ \ \ \ \ }
\startm
\m{\vdash}\m{(}\m{F}\m{:}\m{A}
\m{
\raisebox{.5ex}{${\textstyle{\:}_{\mbox{\footnotesize\rm
1\tt -\rm 1}}}\atop{\textstyle{
\longrightarrow}\atop{\textstyle{}^{\mbox{\footnotesize\rm onto}}}}$}
}
\m{B}
\m{\leftrightarrow}\m{(}\m{F}\m{:}\m{A}
\m{
\raisebox{.5ex}{${\textstyle{\:}_{\mbox{\footnotesize\rm
1\tt -\rm 1}}}\atop{\textstyle{
\longrightarrow}\atop{\textstyle{}^{\mbox{\footnotesize\rm {\ }}}}}$}
}
\m{B}\m{\wedge}\m{F}\m{:}\m{A}
\m{
\raisebox{.5ex}{${\textstyle{\:}_{\mbox{\footnotesize\rm
{\ }}}}\atop{\textstyle{
\longrightarrow}\atop{\textstyle{}^{\mbox{\footnotesize\rm onto}}}}$}
}
\m{B}\m{)}\m{)}
\endm
\begin{mmraw}%
|- ( F : A -1-1-onto-> B <-> ( F : A -1-1-> B? \TAND F : A -onto-> B ) ) \$.?
\end{mmraw}

\noindent Define the value of a function\index{function value}.  This
definition applies to any class and evaluates to the empty set when it is not
meaningful. Note that $ F`A$ means the same thing as the more familiar $ F(A)$
notation for a function's value at $A$.  The $ F`A$ notation is common in
formal set theory.\label{df-fv}

\vskip 0.5ex
\setbox\startprefix=\hbox{\tt \ \ df-fv\ \$a\ }
\setbox\contprefix=\hbox{\tt \ \ \ \ \ \ \ \ \ \ \ }
\startm
\m{\vdash}\m{(}\m{F}\m{`}\m{A}\m{)}\m{=}\m{\bigcup}\m{\{}\m{x}\m{|}\m{(}\m{F}%
\m{``}\m{\{}\m{A}\m{\}}\m{)}\m{=}\m{\{}\m{x}\m{\}}\m{\}}
\endm
\begin{mmraw}%
|- ( F ` A ) = U. \{ x | ( F " \{ A \} ) = \{ x \} \} \$.
\end{mmraw}

\noindent Define the result of an operation.\index{operation}  Here, $F$ is
     an operation on two
     values (such as $+$ for real numbers).   This is defined for proper
     classes $A$ and $B$ even though not meaningful in that case.  However,
     the definition can be meaningful when $F$ is a proper class.\label{dfopr}

\vskip 0.5ex
\setbox\startprefix=\hbox{\tt \ \ df-opr\ \$a\ }
\setbox\contprefix=\hbox{\tt \ \ \ \ \ \ \ \ \ \ \ \ }
\startm
\m{\vdash}\m{(}\m{A}\m{\,F}\m{\,B}\m{)}\m{=}\m{(}\m{F}\m{`}\m{\langle}\m{A}%
\m{,}\m{B}\m{\rangle}\m{)}
\endm
\begin{mmraw}%
|- ( A F B ) = ( F ` <. A , B >. ) \$.
\end{mmraw}

\section{Tricks of the Trade}\label{tricks}

In the \texttt{set.mm}\index{set theory database (\texttt{set.mm})} database our goal
was usually to conform to modern notation.  However in some cases the
relationship to standard textbook language may be obscured by several
unconventional devices we used to simplify the development and to take
advantage of the Metamath language.  In this section we will describe some
common conventions used in \texttt{set.mm}.

\begin{itemize}
\item
The turnstile\index{turnstile ({$\,\vdash$})} symbol, $\vdash$, meaning ``it
is provable that,'' is the first token of all assertions and hypotheses that
aren't syntax constructions.  This is a standard convention in logic.  (We
mentioned this earlier, but this symbol is bothersome to some people without a
logic background.  It has no deeper meaning but just provides us with a way to
distinguish syntax constructions from ordinary mathematical statements.)

\item
A hypothesis of the form

\vskip 1ex
\setbox\startprefix=\hbox{\tt \ \ \ \ \ \ \ \ \ \$e\ }
\setbox\contprefix=\hbox{\tt \ \ \ \ \ \ \ \ \ \ }
\startm
\m{\vdash}\m{(}\m{\varphi}\m{\rightarrow}\m{\forall}\m{x}\m{\varphi}\m{)}
\endm
\vskip 1ex

should be read ``assume variable $x$ is (effectively) not free in wff
$\varphi$.''\index{effectively not free}
Literally, this says ``assume it is provable that $\varphi \rightarrow \forall
x\, \varphi$.''  This device lets us avoid the complexities associated with
the standard treatment of free and bound variables.
%% Uncomment this when uncommenting section {formalspec} below
The footnote on p.~\pageref{effectivelybound} discusses this further.

\item
A statement of one of the forms

\vskip 1ex
\setbox\startprefix=\hbox{\tt \ \ \ \ \ \ \ \ \ \$a\ }
\setbox\contprefix=\hbox{\tt \ \ \ \ \ \ \ \ \ \ }
\startm
\m{\vdash}\m{(}\m{\lnot}\m{\forall}\m{x}\m{\,x}\m{=}\m{y}\m{\rightarrow}
\m{\ldots}\m{)}
\endm
\setbox\startprefix=\hbox{\tt \ \ \ \ \ \ \ \ \ \$p\ }
\setbox\contprefix=\hbox{\tt \ \ \ \ \ \ \ \ \ \ }
\startm
\m{\vdash}\m{(}\m{\lnot}\m{\forall}\m{x}\m{\,x}\m{=}\m{y}\m{\rightarrow}
\m{\ldots}\m{)}
\endm
\vskip 1ex

should be read ``if $x$ and $y$ are distinct variables, then...''  This
antecedent provides us with a technical device to avoid the need for the
\texttt{\$d} statement early in our development of predicate calculus,
permitting symbol manipulations to be as conceptually simple as those in
propositional calculus.  However, the \texttt{\$d} statement eventually
becomes a requirement, and after that this device is rarely used.

\item
The statement

\vskip 1ex
\setbox\startprefix=\hbox{\tt \ \ \ \ \ \ \ \ \ \$d\ }
\setbox\contprefix=\hbox{\tt \ \ \ \ \ \ \ \ \ \ }
\startm
\m{x}\m{\,y}
\endm
\vskip 1ex

should be read ``assume $x$ and $y$ are distinct variables.''

\item
The statement

\vskip 1ex
\setbox\startprefix=\hbox{\tt \ \ \ \ \ \ \ \ \ \$d\ }
\setbox\contprefix=\hbox{\tt \ \ \ \ \ \ \ \ \ \ }
\startm
\m{x}\m{\,\varphi}
\endm
\vskip 1ex

should be read ``assume $x$ does not occur in $\varphi$.''

\item
The statement

\vskip 1ex
\setbox\startprefix=\hbox{\tt \ \ \ \ \ \ \ \ \ \$d\ }
\setbox\contprefix=\hbox{\tt \ \ \ \ \ \ \ \ \ \ }
\startm
\m{x}\m{\,A}
\endm
\vskip 1ex

should be read ``assume variable $x$ does not occur in class $A$.''

\item
The restriction and hypothesis group

\vskip 1ex
\setbox\startprefix=\hbox{\tt \ \ \ \ \ \ \ \ \ \$d\ }
\setbox\contprefix=\hbox{\tt \ \ \ \ \ \ \ \ \ \ }
\startm
\m{x}\m{\,A}
\endm
\setbox\startprefix=\hbox{\tt \ \ \ \ \ \ \ \ \ \$d\ }
\setbox\contprefix=\hbox{\tt \ \ \ \ \ \ \ \ \ \ }
\startm
\m{x}\m{\,\psi}
\endm
\setbox\startprefix=\hbox{\tt \ \ \ \ \ \ \ \ \ \$e\ }
\setbox\contprefix=\hbox{\tt \ \ \ \ \ \ \ \ \ \ }
\startm
\m{\vdash}\m{(}\m{x}\m{=}\m{A}\m{\rightarrow}\m{(}\m{\varphi}\m{\leftrightarrow}
\m{\psi}\m{)}\m{)}
\endm
\vskip 1ex

is frequently used in place of explicit substitution, meaning ``assume
$\psi$ results from the proper substitution of $A$ for $x$ in
$\varphi$.''  Sometimes ``\texttt{\$e} $\vdash ( \psi \rightarrow
\forall x \, \psi )$'' is used instead of ``\texttt{\$d} $x\, \psi $,''
which requires only that $x$ be effectively not free in $\varphi$ but
not necessarily absent from it.  The use of implicit
substitution\index{substitution!implicit} is partly a matter of personal
style, although it may make proofs somewhat shorter than would be the
case with explicit substitution.

\item
The hypothesis


\vskip 1ex
\setbox\startprefix=\hbox{\tt \ \ \ \ \ \ \ \ \ \$e\ }
\setbox\contprefix=\hbox{\tt \ \ \ \ \ \ \ \ \ \ }
\startm
\m{\vdash}\m{A}\m{\in}\m{{\rm V}}
\endm
\vskip 1ex

should be read ``assume class $A$ is a set (i.e.\ exists).''
This is a convenient convention used by Quine.

\item
The restriction and hypothesis

\vskip 1ex
\setbox\startprefix=\hbox{\tt \ \ \ \ \ \ \ \ \ \$d\ }
\setbox\contprefix=\hbox{\tt \ \ \ \ \ \ \ \ \ \ }
\startm
\m{x}\m{\,y}
\endm
\setbox\startprefix=\hbox{\tt \ \ \ \ \ \ \ \ \ \$e\ }
\setbox\contprefix=\hbox{\tt \ \ \ \ \ \ \ \ \ \ }
\startm
\m{\vdash}\m{(}\m{y}\m{\in}\m{A}\m{\rightarrow}\m{\forall}\m{x}\m{\,y}
\m{\in}\m{A}\m{)}
\endm
\vskip 1ex

should be read ``assume variable $x$ is
(effectively) not free in class $A$.''

\end{itemize}

\section{A Theorem Sampler}\label{sometheorems}

In this section we list some of the more important theorems that are proved in
the \texttt{set.mm} database, and they illustrate the kinds of things that can be
done with Metamath.  While all of these facts are well-known results,
Metamath offers the advantage of easily allowing you to trace their
derivation back to axioms.  Our intent here is not to try to explain the
details or motivation; for this we refer you to the textbooks that are
mentioned in the descriptions.  (The \texttt{set.mm} file has bibliographic
references for the text references.)  Their proofs often embody important
concepts you may wish to explore with the Metamath program (see
Section~\ref{exploring}).  All the symbols that are used here are defined in
Section~\ref{hierarchy}.  For brevity we haven't included the \texttt{\$d}
restrictions or \texttt{\$f} hypotheses for these theorems; when you are
uncertain consult the \texttt{set.mm} database.

We start with \texttt{syl} (principle of the syllogism).
In \textit{Principia Mathematica}
Whitehead and Russell call this ``the principle of the
syllogism... because... the syllogism in Barbara is derived from them''
\cite[quote after Theorem *2.06 p.~101]{PM}.
Some authors call this law a ``hypothetical syllogism.''
As of 2019 \texttt{syl} is the most commonly referenced proven
assertion in the \texttt{set.mm} database.\footnote{
The Metamath program command \texttt{show usage}
shows the number of references.
On 2019-04-29 (commit 71cbbbdb387e) \texttt{syl} was directly referenced
10,819 times. The second most commonly referenced proven assertion
was \texttt{eqid}, which was directly referenced 7,738 times.
}

\vskip 2ex
\noindent Theorem syl (principle of the syllogism)\index{Syllogism}%
\index{\texttt{syl}}\label{syl}.

\vskip 0.5ex
\setbox\startprefix=\hbox{\tt \ \ syl.1\ \$e\ }
\setbox\contprefix=\hbox{\tt \ \ \ \ \ \ \ \ \ \ \ }
\startm
\m{\vdash}\m{(}\m{\varphi}\m{ \rightarrow }\m{\psi}\m{)}
\endm
\setbox\startprefix=\hbox{\tt \ \ syl.2\ \$e\ }
\setbox\contprefix=\hbox{\tt \ \ \ \ \ \ \ \ \ \ \ }
\startm
\m{\vdash}\m{(}\m{\psi}\m{ \rightarrow }\m{\chi}\m{)}
\endm
\setbox\startprefix=\hbox{\tt \ \ syl\ \$p\ }
\setbox\contprefix=\hbox{\tt \ \ \ \ \ \ \ \ \ }
\startm
\m{\vdash}\m{(}\m{\varphi}\m{ \rightarrow }\m{\chi}\m{)}
\endm
\vskip 2ex

The following theorem is not very deep but provides us with a notational device
that is frequently used.  It allows us to use the expression ``$A \in V$'' as
a compact way of saying that class $A$ exists, i.e.\ is a set.

\vskip 2ex
\noindent Two ways to say ``$A$ is a set'':  $A$ is a member of the universe
$V$ if and only if $A$ exists (i.e.\ there exists a set equal to $A$).
Theorem 6.9 of Quine, p. 43.

\vskip 0.5ex
\setbox\startprefix=\hbox{\tt \ \ isset\ \$p\ }
\setbox\contprefix=\hbox{\tt \ \ \ \ \ \ \ \ \ \ \ }
\startm
\m{\vdash}\m{(}\m{A}\m{\in}\m{{\rm V}}\m{\leftrightarrow}\m{\exists}\m{x}\m{\,x}\m{=}
\m{A}\m{)}
\endm
\vskip 1ex

Next we prove the axioms of standard ZF set theory that were missing from our
axiom system.  From our point of view they are theorems since they
can be derived from the other axioms.

\vskip 2ex
\noindent Axiom of Separation\index{Axiom of Separation}
(Aussonderung)\index{Aussonderung} proved from the other axioms of ZF set
theory.  Compare Exercise 4 of Takeuti and Zaring, p.~22.

\vskip 0.5ex
\setbox\startprefix=\hbox{\tt \ \ inex1.1\ \$e\ }
\setbox\contprefix=\hbox{\tt \ \ \ \ \ \ \ \ \ \ \ \ \ \ \ }
\startm
\m{\vdash}\m{A}\m{\in}\m{{\rm V}}
\endm
\setbox\startprefix=\hbox{\tt \ \ inex\ \$p\ }
\setbox\contprefix=\hbox{\tt \ \ \ \ \ \ \ \ \ \ \ \ \ }
\startm
\m{\vdash}\m{(}\m{A}\m{\cap}\m{B}\m{)}\m{\in}\m{{\rm V}}
\endm
\vskip 1ex

\noindent Axiom of the Null Set\index{Axiom of the Null Set} proved from the
other axioms of ZF set theory. Corollary 5.16 of Takeuti and Zaring, p.~20.

\vskip 0.5ex
\setbox\startprefix=\hbox{\tt \ \ 0ex\ \$p\ }
\setbox\contprefix=\hbox{\tt \ \ \ \ \ \ \ \ \ \ \ \ }
\startm
\m{\vdash}\m{\varnothing}\m{\in}\m{{\rm V}}
\endm
\vskip 1ex

\noindent The Axiom of Pairing\index{Axiom of Pairing} proved from the other
axioms of ZF set theory.  Theorem 7.13 of Quine, p.~51.
\vskip 0.5ex
\setbox\startprefix=\hbox{\tt \ \ prex\ \$p\ }
\setbox\contprefix=\hbox{\tt \ \ \ \ \ \ \ \ \ \ \ \ \ \ }
\startm
\m{\vdash}\m{\{}\m{A}\m{,}\m{B}\m{\}}\m{\in}\m{{\rm V}}
\endm
\vskip 2ex

Next we will list some famous or important theorems that are proved in
the \texttt{set.mm} database.  None of them except \texttt{omex}
require the Axiom of Infinity, as you can verify with the \texttt{show
trace{\char`\_}back} Metamath command.

\vskip 2ex
\noindent The resolution of Russell's paradox\index{Russell's paradox}.  There
exists no set containing the set of all sets which are not members of
themselves.  Proposition 4.14 of Takeuti and Zaring, p.~14.

\vskip 0.5ex
\setbox\startprefix=\hbox{\tt \ \ ru\ \$p\ }
\setbox\contprefix=\hbox{\tt \ \ \ \ \ \ \ \ }
\startm
\m{\vdash}\m{\lnot}\m{\exists}\m{x}\m{\,x}\m{=}\m{\{}\m{y}\m{|}\m{\lnot}\m{y}
\m{\in}\m{y}\m{\}}
\endm
\vskip 1ex

\noindent Cantor's theorem\index{Cantor's theorem}.  No set can be mapped onto
its power set.  Compare Theorem 6B(b) of Enderton, p.~132.

\vskip 0.5ex
\setbox\startprefix=\hbox{\tt \ \ canth.1\ \$e\ }
\setbox\contprefix=\hbox{\tt \ \ \ \ \ \ \ \ \ \ \ \ \ }
\startm
\m{\vdash}\m{A}\m{\in}\m{{\rm V}}
\endm
\setbox\startprefix=\hbox{\tt \ \ canth\ \$p\ }
\setbox\contprefix=\hbox{\tt \ \ \ \ \ \ \ \ \ \ \ }
\startm
\m{\vdash}\m{\lnot}\m{F}\m{:}\m{A}\m{\raisebox{.5ex}{${\textstyle{\:}_{
\mbox{\footnotesize\rm {\ }}}}\atop{\textstyle{\longrightarrow}\atop{
\textstyle{}^{\mbox{\footnotesize\rm onto}}}}$}}\m{{\cal P}}\m{A}
\endm
\vskip 1ex

\noindent The Burali-Forti paradox\index{Burali-Forti paradox}.  No set
contains all ordinal numbers. Enderton, p.~194.  (Burali-Forti was one person,
not two.)

\vskip 0.5ex
\setbox\startprefix=\hbox{\tt \ \ onprc\ \$p\ }
\setbox\contprefix=\hbox{\tt \ \ \ \ \ \ \ \ \ \ \ \ }
\startm
\m{\vdash}\m{\lnot}\m{\mbox{\rm On}}\m{\in}\m{{\rm V}}
\endm
\vskip 1ex

\noindent Peano's postulates\index{Peano's postulates} for arithmetic.
Proposition 7.30 of Takeuti and Zaring, pp.~42--43.  The objects being
described are the members of $\omega$ i.e.\ the natural numbers 0, 1,
2,\ldots.  The successor\index{successor} operation suc means ``plus
one.''  \texttt{peano1} says that 0 (which is defined as the empty set)
is a natural number.  \texttt{peano2} says that if $A$ is a natural
number, so is $A+1$.  \texttt{peano3} says that 0 is not the successor
of any natural number.  \texttt{peano4} says that two natural numbers
are equal if and only if their successors are equal.  \texttt{peano5} is
essentially the same as mathematical induction.

\vskip 1ex
\setbox\startprefix=\hbox{\tt \ \ peano1\ \$p\ }
\setbox\contprefix=\hbox{\tt \ \ \ \ \ \ \ \ \ \ \ \ }
\startm
\m{\vdash}\m{\varnothing}\m{\in}\m{\omega}
\endm
\vskip 1.5ex

\setbox\startprefix=\hbox{\tt \ \ peano2\ \$p\ }
\setbox\contprefix=\hbox{\tt \ \ \ \ \ \ \ \ \ \ \ \ }
\startm
\m{\vdash}\m{(}\m{A}\m{\in}\m{\omega}\m{\rightarrow}\m{{\rm suc}}\m{A}\m{\in}%
\m{\omega}\m{)}
\endm
\vskip 1.5ex

\setbox\startprefix=\hbox{\tt \ \ peano3\ \$p\ }
\setbox\contprefix=\hbox{\tt \ \ \ \ \ \ \ \ \ \ \ \ }
\startm
\m{\vdash}\m{(}\m{A}\m{\in}\m{\omega}\m{\rightarrow}\m{\lnot}\m{{\rm suc}}%
\m{A}\m{=}\m{\varnothing}\m{)}
\endm
\vskip 1.5ex

\setbox\startprefix=\hbox{\tt \ \ peano4\ \$p\ }
\setbox\contprefix=\hbox{\tt \ \ \ \ \ \ \ \ \ \ \ \ }
\startm
\m{\vdash}\m{(}\m{(}\m{A}\m{\in}\m{\omega}\m{\wedge}\m{B}\m{\in}\m{\omega}%
\m{)}\m{\rightarrow}\m{(}\m{{\rm suc}}\m{A}\m{=}\m{{\rm suc}}\m{B}\m{%
\leftrightarrow}\m{A}\m{=}\m{B}\m{)}\m{)}
\endm
\vskip 1.5ex

\setbox\startprefix=\hbox{\tt \ \ peano5\ \$p\ }
\setbox\contprefix=\hbox{\tt \ \ \ \ \ \ \ \ \ \ \ \ }
\startm
\m{\vdash}\m{(}\m{(}\m{\varnothing}\m{\in}\m{A}\m{\wedge}\m{\forall}\m{x}\m{%
\in}\m{\omega}\m{(}\m{x}\m{\in}\m{A}\m{\rightarrow}\m{{\rm suc}}\m{x}\m{\in}%
\m{A}\m{)}\m{)}\m{\rightarrow}\m{\omega}\m{\subseteq}\m{A}\m{)}
\endm
\vskip 1.5ex

\noindent Finite Induction (mathematical induction).\index{finite
induction}\index{mathematical induction} The first hypothesis is the
basis and the second is the induction hypothesis.  Theorem Schema 22 of
Suppes, p.~136.

\vskip 0.5ex
\setbox\startprefix=\hbox{\tt \ \ findes.1\ \$e\ }
\setbox\contprefix=\hbox{\tt \ \ \ \ \ \ \ \ \ \ \ \ \ \ }
\startm
\m{\vdash}\m{[}\m{\varnothing}\m{/}\m{x}\m{]}\m{\varphi}
\endm
\setbox\startprefix=\hbox{\tt \ \ findes.2\ \$e\ }
\setbox\contprefix=\hbox{\tt \ \ \ \ \ \ \ \ \ \ \ \ \ \ }
\startm
\m{\vdash}\m{(}\m{x}\m{\in}\m{\omega}\m{\rightarrow}\m{(}\m{\varphi}\m{%
\rightarrow}\m{[}\m{{\rm suc}}\m{x}\m{/}\m{x}\m{]}\m{\varphi}\m{)}\m{)}
\endm
\setbox\startprefix=\hbox{\tt \ \ findes\ \$p\ }
\setbox\contprefix=\hbox{\tt \ \ \ \ \ \ \ \ \ \ \ \ }
\startm
\m{\vdash}\m{(}\m{x}\m{\in}\m{\omega}\m{\rightarrow}\m{\varphi}\m{)}
\endm
\vskip 1ex

\noindent Transfinite Induction with explicit substitution.  The first
hypothesis is the basis, the second is the induction hypothesis for
successors, and the third is the induction hypothesis for limit
ordinals.  Theorem Schema 4 of Suppes, p. 197.

\vskip 0.5ex
\setbox\startprefix=\hbox{\tt \ \ tfindes.1\ \$e\ }
\setbox\contprefix=\hbox{\tt \ \ \ \ \ \ \ \ \ \ \ \ \ \ \ }
\startm
\m{\vdash}\m{[}\m{\varnothing}\m{/}\m{x}\m{]}\m{\varphi}
\endm
\setbox\startprefix=\hbox{\tt \ \ tfindes.2\ \$e\ }
\setbox\contprefix=\hbox{\tt \ \ \ \ \ \ \ \ \ \ \ \ \ \ \ }
\startm
\m{\vdash}\m{(}\m{x}\m{\in}\m{{\rm On}}\m{\rightarrow}\m{(}\m{\varphi}\m{%
\rightarrow}\m{[}\m{{\rm suc}}\m{x}\m{/}\m{x}\m{]}\m{\varphi}\m{)}\m{)}
\endm
\setbox\startprefix=\hbox{\tt \ \ tfindes.3\ \$e\ }
\setbox\contprefix=\hbox{\tt \ \ \ \ \ \ \ \ \ \ \ \ \ \ \ }
\startm
\m{\vdash}\m{(}\m{{\rm Lim}}\m{y}\m{\rightarrow}\m{(}\m{\forall}\m{x}\m{\in}%
\m{y}\m{\varphi}\m{\rightarrow}\m{[}\m{y}\m{/}\m{x}\m{]}\m{\varphi}\m{)}\m{)}
\endm
\setbox\startprefix=\hbox{\tt \ \ tfindes\ \$p\ }
\setbox\contprefix=\hbox{\tt \ \ \ \ \ \ \ \ \ \ \ \ \ }
\startm
\m{\vdash}\m{(}\m{x}\m{\in}\m{{\rm On}}\m{\rightarrow}\m{\varphi}\m{)}
\endm
\vskip 1ex

\noindent Principle of Transfinite Recursion.\index{transfinite
recursion} Theorem 7.41 of Takeuti and Zaring, p.~47.  Transfinite
recursion is the key theorem that allows arithmetic of ordinals to be
rigorously defined, and has many other important uses as well.
Hypotheses \texttt{tfr.1} and \texttt{tfr.2} specify a certain (proper)
class $ F$.  The complicated definition of $ F$ is not important in
itself; what is important is that there be such an $ F$ with the
required properties, and we show this by displaying $ F$ explicitly.
\texttt{tfr1} states that $ F$ is a function whose domain is the set of
ordinal numbers.  \texttt{tfr2} states that any value of $ F$ is
completely determined by its previous values and the values of an
auxiliary function, $G$.  \texttt{tfr3} states that $ F$ is unique,
i.e.\ it is the only function that satisfies \texttt{tfr1} and
\texttt{tfr2}.  Note that $ f$ is an individual variable like $x$ and
$y$; it is just a mnemonic to remind us that $A$ is a collection of
functions.

\vskip 0.5ex
\setbox\startprefix=\hbox{\tt \ \ tfr.1\ \$e\ }
\setbox\contprefix=\hbox{\tt \ \ \ \ \ \ \ \ \ \ \ }
\startm
\m{\vdash}\m{A}\m{=}\m{\{}\m{f}\m{|}\m{\exists}\m{x}\m{\in}\m{{\rm On}}\m{(}%
\m{f}\m{{\rm Fn}}\m{x}\m{\wedge}\m{\forall}\m{y}\m{\in}\m{x}\m{(}\m{f}\m{`}%
\m{y}\m{)}\m{=}\m{(}\m{G}\m{`}\m{(}\m{f}\m{\restriction}\m{y}\m{)}\m{)}\m{)}%
\m{\}}
\endm
\setbox\startprefix=\hbox{\tt \ \ tfr.2\ \$e\ }
\setbox\contprefix=\hbox{\tt \ \ \ \ \ \ \ \ \ \ \ }
\startm
\m{\vdash}\m{F}\m{=}\m{\bigcup}\m{A}
\endm
\setbox\startprefix=\hbox{\tt \ \ tfr1\ \$p\ }
\setbox\contprefix=\hbox{\tt \ \ \ \ \ \ \ \ \ \ }
\startm
\m{\vdash}\m{F}\m{{\rm Fn}}\m{{\rm On}}
\endm
\setbox\startprefix=\hbox{\tt \ \ tfr2\ \$p\ }
\setbox\contprefix=\hbox{\tt \ \ \ \ \ \ \ \ \ \ }
\startm
\m{\vdash}\m{(}\m{z}\m{\in}\m{{\rm On}}\m{\rightarrow}\m{(}\m{F}\m{`}\m{z}%
\m{)}\m{=}\m{(}\m{G}\m{`}\m{(}\m{F}\m{\restriction}\m{z}\m{)}\m{)}\m{)}
\endm
\setbox\startprefix=\hbox{\tt \ \ tfr3\ \$p\ }
\setbox\contprefix=\hbox{\tt \ \ \ \ \ \ \ \ \ \ }
\startm
\m{\vdash}\m{(}\m{(}\m{B}\m{{\rm Fn}}\m{{\rm On}}\m{\wedge}\m{\forall}\m{x}\m{%
\in}\m{{\rm On}}\m{(}\m{B}\m{`}\m{x}\m{)}\m{=}\m{(}\m{G}\m{`}\m{(}\m{B}\m{%
\restriction}\m{x}\m{)}\m{)}\m{)}\m{\rightarrow}\m{B}\m{=}\m{F}\m{)}
\endm
\vskip 1ex

\noindent The existence of omega (the class of natural numbers).\index{natural
number}\index{omega ($\omega$)}\index{Axiom of Infinity}  Axiom 7 of Takeuti
and Zaring, p.~43.  (This is the only theorem in this section requiring the
Axiom of Infinity.)

\vskip 0.5ex
\setbox\startprefix=\hbox{\tt \
\ omex\ \$p\ }
\setbox\contprefix=\hbox{\tt \ \ \ \ \ \ \ \ \ \ }
\startm
\m{\vdash}\m{\omega}\m{\in}\m{{\rm V}}
\endm
%\vskip 2ex


\section{Axioms for Real and Complex Numbers}\label{real}
\index{real and complex numbers!axioms for}

This section presents the axioms for real and complex numbers, along
with some commentary about them.  Analysis
textbooks implicitly or explicitly use these axioms or their equivalents
as their starting point.  In the database \texttt{set.mm}, we define real
and complex numbers as (rather complicated) specific sets and derive these
axioms as {\em theorems} from the axioms of ZF set theory, using a method
called Dedekind cuts.  We omit the details of this construction, which you can
follow if you wish using the \texttt{set.mm} database in conjunction with the
textbooks referenced therein.

Once we prove those theorems, we then restate these proven theorems as axioms.
This lets us easily identify which axioms are needed for a particular complex number proof, without the obfuscation of the set theory used to derive them.
As a result,
the construction is actually unimportant other
than to show that sets exist that satisfy the axioms, and thus that the axioms
are consistent if ZF set theory is consistent.  When working with real numbers
you can think of them as being the actual sets resulting
from the construction (for definiteness), or you can
think of them as otherwise unspecified sets that happen to satisfy the axioms.
The derivation is not easy, but the fact that it works is quite remarkable
and lends support to the idea that ZFC set theory is all we need to
provide a foundation for essentially all of mathematics.

\needspace{3\baselineskip}
\subsection{The Axioms for Real and Complex Numbers Themselves}\label{realactual}

For the axioms we are given (or postulate) 8 classes:  $\mathbb{C}$ (the
set of complex numbers), $\mathbb{R}$ (the set of real numbers, a subset
of $\mathbb{C}$), $0$ (zero), $1$ (one), $i$ (square root of
$-1$), $+$ (plus), $\cdot$ (times), and
$<_{\mathbb{R}}$ (less than for just the real numbers).
Subtraction and division are defined terms and are not part of the
axioms; for their definitions see \texttt{set.mm}.

Note that the notation $(A+B)$ (and similarly $(A\cdot B)$) specifies a class
called an {\em operation},\index{operation} and is the function value of the
class $+$ at ordered pair $\langle A,B \rangle$.  An operation is defined by
statement \texttt{df-opr} on p.~\pageref{dfopr}.
The notation $A <_{\mathbb{R}} B$ specifies a
wff called a {\em binary relation}\index{binary relation} and means $\langle A,B \rangle \in \,<_{\mathbb{R}}$, as defined by statement \texttt{df-br} on p.~\pageref{dfbr}.

Our set of 8 given classes is assumed to satisfy the following 22 axioms
(in the axioms listed below, $<$ really means $<_{\mathbb{R}}$).

\vskip 2ex

\noindent 1. The real numbers are a subset of the complex numbers.

%\vskip 0.5ex
\setbox\startprefix=\hbox{\tt \ \ ax-resscn\ \$p\ }
\setbox\contprefix=\hbox{\tt \ \ \ \ \ \ \ \ \ \ \ \ \ \ }
\startm
\m{\vdash}\m{\mathbb{R}}\m{\subseteq}\m{\mathbb{C}}
\endm
%\vskip 1ex

\noindent 2. One is a complex number.

%\vskip 0.5ex
\setbox\startprefix=\hbox{\tt \ \ ax-1cn\ \$p\ }
\setbox\contprefix=\hbox{\tt \ \ \ \ \ \ \ \ \ \ \ }
\startm
\m{\vdash}\m{1}\m{\in}\m{\mathbb{C}}
\endm
%\vskip 1ex

\noindent 3. The imaginary number $i$ is a complex number.

%\vskip 0.5ex
\setbox\startprefix=\hbox{\tt \ \ ax-icn\ \$p\ }
\setbox\contprefix=\hbox{\tt \ \ \ \ \ \ \ \ \ \ \ }
\startm
\m{\vdash}\m{i}\m{\in}\m{\mathbb{C}}
\endm
%\vskip 1ex

\noindent 4. Complex numbers are closed under addition.

%\vskip 0.5ex
\setbox\startprefix=\hbox{\tt \ \ ax-addcl\ \$p\ }
\setbox\contprefix=\hbox{\tt \ \ \ \ \ \ \ \ \ \ \ \ \ }
\startm
\m{\vdash}\m{(}\m{(}\m{A}\m{\in}\m{\mathbb{C}}\m{\wedge}\m{B}\m{\in}\m{\mathbb{C}}%
\m{)}\m{\rightarrow}\m{(}\m{A}\m{+}\m{B}\m{)}\m{\in}\m{\mathbb{C}}\m{)}
\endm
%\vskip 1ex

\noindent 5. Real numbers are closed under addition.

%\vskip 0.5ex
\setbox\startprefix=\hbox{\tt \ \ ax-addrcl\ \$p\ }
\setbox\contprefix=\hbox{\tt \ \ \ \ \ \ \ \ \ \ \ \ \ \ }
\startm
\m{\vdash}\m{(}\m{(}\m{A}\m{\in}\m{\mathbb{R}}\m{\wedge}\m{B}\m{\in}\m{\mathbb{R}}%
\m{)}\m{\rightarrow}\m{(}\m{A}\m{+}\m{B}\m{)}\m{\in}\m{\mathbb{R}}\m{)}
\endm
%\vskip 1ex

\noindent 6. Complex numbers are closed under multiplication.

%\vskip 0.5ex
\setbox\startprefix=\hbox{\tt \ \ ax-mulcl\ \$p\ }
\setbox\contprefix=\hbox{\tt \ \ \ \ \ \ \ \ \ \ \ \ \ }
\startm
\m{\vdash}\m{(}\m{(}\m{A}\m{\in}\m{\mathbb{C}}\m{\wedge}\m{B}\m{\in}\m{\mathbb{C}}%
\m{)}\m{\rightarrow}\m{(}\m{A}\m{\cdot}\m{B}\m{)}\m{\in}\m{\mathbb{C}}\m{)}
\endm
%\vskip 1ex

\noindent 7. Real numbers are closed under multiplication.

%\vskip 0.5ex
\setbox\startprefix=\hbox{\tt \ \ ax-mulrcl\ \$p\ }
\setbox\contprefix=\hbox{\tt \ \ \ \ \ \ \ \ \ \ \ \ \ \ }
\startm
\m{\vdash}\m{(}\m{(}\m{A}\m{\in}\m{\mathbb{R}}\m{\wedge}\m{B}\m{\in}\m{\mathbb{R}}%
\m{)}\m{\rightarrow}\m{(}\m{A}\m{\cdot}\m{B}\m{)}\m{\in}\m{\mathbb{R}}\m{)}
\endm
%\vskip 1ex

\noindent 8. Multiplication of complex numbers is commutative.

%\vskip 0.5ex
\setbox\startprefix=\hbox{\tt \ \ ax-mulcom\ \$p\ }
\setbox\contprefix=\hbox{\tt \ \ \ \ \ \ \ \ \ \ \ \ \ \ }
\startm
\m{\vdash}\m{(}\m{(}\m{A}\m{\in}\m{\mathbb{C}}\m{\wedge}\m{B}\m{\in}\m{\mathbb{C}}%
\m{)}\m{\rightarrow}\m{(}\m{A}\m{\cdot}\m{B}\m{)}\m{=}\m{(}\m{B}\m{\cdot}\m{A}%
\m{)}\m{)}
\endm
%\vskip 1ex

\noindent 9. Addition of complex numbers is associative.

%\vskip 0.5ex
\setbox\startprefix=\hbox{\tt \ \ ax-addass\ \$p\ }
\setbox\contprefix=\hbox{\tt \ \ \ \ \ \ \ \ \ \ \ \ \ \ }
\startm
\m{\vdash}\m{(}\m{(}\m{A}\m{\in}\m{\mathbb{C}}\m{\wedge}\m{B}\m{\in}\m{\mathbb{C}}%
\m{\wedge}\m{C}\m{\in}\m{\mathbb{C}}\m{)}\m{\rightarrow}\m{(}\m{(}\m{A}\m{+}%
\m{B}\m{)}\m{+}\m{C}\m{)}\m{=}\m{(}\m{A}\m{+}\m{(}\m{B}\m{+}\m{C}\m{)}\m{)}%
\m{)}
\endm
%\vskip 1ex

\noindent 10. Multiplication of complex numbers is associative.

%\vskip 0.5ex
\setbox\startprefix=\hbox{\tt \ \ ax-mulass\ \$p\ }
\setbox\contprefix=\hbox{\tt \ \ \ \ \ \ \ \ \ \ \ \ \ \ }
\startm
\m{\vdash}\m{(}\m{(}\m{A}\m{\in}\m{\mathbb{C}}\m{\wedge}\m{B}\m{\in}\m{\mathbb{C}}%
\m{\wedge}\m{C}\m{\in}\m{\mathbb{C}}\m{)}\m{\rightarrow}\m{(}\m{(}\m{A}\m{\cdot}%
\m{B}\m{)}\m{\cdot}\m{C}\m{)}\m{=}\m{(}\m{A}\m{\cdot}\m{(}\m{B}\m{\cdot}\m{C}%
\m{)}\m{)}\m{)}
\endm
%\vskip 1ex

\noindent 11. Multiplication distributes over addition for complex numbers.

%\vskip 0.5ex
\setbox\startprefix=\hbox{\tt \ \ ax-distr\ \$p\ }
\setbox\contprefix=\hbox{\tt \ \ \ \ \ \ \ \ \ \ \ \ \ }
\startm
\m{\vdash}\m{(}\m{(}\m{A}\m{\in}\m{\mathbb{C}}\m{\wedge}\m{B}\m{\in}\m{\mathbb{C}}%
\m{\wedge}\m{C}\m{\in}\m{\mathbb{C}}\m{)}\m{\rightarrow}\m{(}\m{A}\m{\cdot}\m{(}%
\m{B}\m{+}\m{C}\m{)}\m{)}\m{=}\m{(}\m{(}\m{A}\m{\cdot}\m{B}\m{)}\m{+}\m{(}%
\m{A}\m{\cdot}\m{C}\m{)}\m{)}\m{)}
\endm
%\vskip 1ex

\noindent 12. The square of $i$ equals $-1$ (expressed as $i$-squared plus 1 is
0).

%\vskip 0.5ex
\setbox\startprefix=\hbox{\tt \ \ ax-i2m1\ \$p\ }
\setbox\contprefix=\hbox{\tt \ \ \ \ \ \ \ \ \ \ \ \ }
\startm
\m{\vdash}\m{(}\m{(}\m{i}\m{\cdot}\m{i}\m{)}\m{+}\m{1}\m{)}\m{=}\m{0}
\endm
%\vskip 1ex

\noindent 13. One and zero are distinct.

%\vskip 0.5ex
\setbox\startprefix=\hbox{\tt \ \ ax-1ne0\ \$p\ }
\setbox\contprefix=\hbox{\tt \ \ \ \ \ \ \ \ \ \ \ \ }
\startm
\m{\vdash}\m{1}\m{\ne}\m{0}
\endm
%\vskip 1ex

\noindent 14. One is an identity element for real multiplication.

%\vskip 0.5ex
\setbox\startprefix=\hbox{\tt \ \ ax-1rid\ \$p\ }
\setbox\contprefix=\hbox{\tt \ \ \ \ \ \ \ \ \ \ \ }
\startm
\m{\vdash}\m{(}\m{A}\m{\in}\m{\mathbb{R}}\m{\rightarrow}\m{(}\m{A}\m{\cdot}\m{1}%
\m{)}\m{=}\m{A}\m{)}
\endm
%\vskip 1ex

\noindent 15. Every real number has a negative.

%\vskip 0.5ex
\setbox\startprefix=\hbox{\tt \ \ ax-rnegex\ \$p\ }
\setbox\contprefix=\hbox{\tt \ \ \ \ \ \ \ \ \ \ \ \ \ \ }
\startm
\m{\vdash}\m{(}\m{A}\m{\in}\m{\mathbb{R}}\m{\rightarrow}\m{\exists}\m{x}\m{\in}%
\m{\mathbb{R}}\m{(}\m{A}\m{+}\m{x}\m{)}\m{=}\m{0}\m{)}
\endm
%\vskip 1ex

\noindent 16. Every nonzero real number has a reciprocal.

%\vskip 0.5ex
\setbox\startprefix=\hbox{\tt \ \ ax-rrecex\ \$p\ }
\setbox\contprefix=\hbox{\tt \ \ \ \ \ \ \ \ \ \ \ \ \ \ }
\startm
\m{\vdash}\m{(}\m{A}\m{\in}\m{\mathbb{R}}\m{\rightarrow}\m{(}\m{A}\m{\ne}\m{0}%
\m{\rightarrow}\m{\exists}\m{x}\m{\in}\m{\mathbb{R}}\m{(}\m{A}\m{\cdot}%
\m{x}\m{)}\m{=}\m{1}\m{)}\m{)}
\endm
%\vskip 1ex

\noindent 17. A complex number can be expressed in terms of two reals.

%\vskip 0.5ex
\setbox\startprefix=\hbox{\tt \ \ ax-cnre\ \$p\ }
\setbox\contprefix=\hbox{\tt \ \ \ \ \ \ \ \ \ \ \ \ }
\startm
\m{\vdash}\m{(}\m{A}\m{\in}\m{\mathbb{C}}\m{\rightarrow}\m{\exists}\m{x}\m{\in}%
\m{\mathbb{R}}\m{\exists}\m{y}\m{\in}\m{\mathbb{R}}\m{A}\m{=}\m{(}\m{x}\m{+}\m{(}%
\m{y}\m{\cdot}\m{i}\m{)}\m{)}\m{)}
\endm
%\vskip 1ex

\noindent 18. Ordering on reals satisfies strict trichotomy.

%\vskip 0.5ex
\setbox\startprefix=\hbox{\tt \ \ ax-pre-lttri\ \$p\ }
\setbox\contprefix=\hbox{\tt \ \ \ \ \ \ \ \ \ \ \ \ \ }
\startm
\m{\vdash}\m{(}\m{(}\m{A}\m{\in}\m{\mathbb{R}}\m{\wedge}\m{B}\m{\in}\m{\mathbb{R}}%
\m{)}\m{\rightarrow}\m{(}\m{A}\m{<}\m{B}\m{\leftrightarrow}\m{\lnot}\m{(}\m{A}%
\m{=}\m{B}\m{\vee}\m{B}\m{<}\m{A}\m{)}\m{)}\m{)}
\endm
%\vskip 1ex

\noindent 19. Ordering on reals is transitive.

%\vskip 0.5ex
\setbox\startprefix=\hbox{\tt \ \ ax-pre-lttrn\ \$p\ }
\setbox\contprefix=\hbox{\tt \ \ \ \ \ \ \ \ \ \ \ \ \ }
\startm
\m{\vdash}\m{(}\m{(}\m{A}\m{\in}\m{\mathbb{R}}\m{\wedge}\m{B}\m{\in}\m{\mathbb{R}}%
\m{\wedge}\m{C}\m{\in}\m{\mathbb{R}}\m{)}\m{\rightarrow}\m{(}\m{(}\m{A}\m{<}%
\m{B}\m{\wedge}\m{B}\m{<}\m{C}\m{)}\m{\rightarrow}\m{A}\m{<}\m{C}\m{)}\m{)}
\endm
%\vskip 1ex

\noindent 20. Ordering on reals is preserved after addition to both sides.

%\vskip 0.5ex
\setbox\startprefix=\hbox{\tt \ \ ax-pre-ltadd\ \$p\ }
\setbox\contprefix=\hbox{\tt \ \ \ \ \ \ \ \ \ \ \ \ \ }
\startm
\m{\vdash}\m{(}\m{(}\m{A}\m{\in}\m{\mathbb{R}}\m{\wedge}\m{B}\m{\in}\m{\mathbb{R}}%
\m{\wedge}\m{C}\m{\in}\m{\mathbb{R}}\m{)}\m{\rightarrow}\m{(}\m{A}\m{<}\m{B}\m{%
\rightarrow}\m{(}\m{C}\m{+}\m{A}\m{)}\m{<}\m{(}\m{C}\m{+}\m{B}\m{)}\m{)}\m{)}
\endm
%\vskip 1ex

\noindent 21. The product of two positive reals is positive.

%\vskip 0.5ex
\setbox\startprefix=\hbox{\tt \ \ ax-pre-mulgt0\ \$p\ }
\setbox\contprefix=\hbox{\tt \ \ \ \ \ \ \ \ \ \ \ \ \ \ }
\startm
\m{\vdash}\m{(}\m{(}\m{A}\m{\in}\m{\mathbb{R}}\m{\wedge}\m{B}\m{\in}\m{\mathbb{R}}%
\m{)}\m{\rightarrow}\m{(}\m{(}\m{0}\m{<}\m{A}\m{\wedge}\m{0}%
\m{<}\m{B}\m{)}\m{\rightarrow}\m{0}\m{<}\m{(}\m{A}\m{\cdot}\m{B}\m{)}%
\m{)}\m{)}
\endm
%\vskip 1ex

\noindent 22. A non-empty, bounded-above set of reals has a supremum.

%\vskip 0.5ex
\setbox\startprefix=\hbox{\tt \ \ ax-pre-sup\ \$p\ }
\setbox\contprefix=\hbox{\tt \ \ \ \ \ \ \ \ \ \ \ }
\startm
\m{\vdash}\m{(}\m{(}\m{A}\m{\subseteq}\m{\mathbb{R}}\m{\wedge}\m{A}\m{\ne}\m{%
\varnothing}\m{\wedge}\m{\exists}\m{x}\m{\in}\m{\mathbb{R}}\m{\forall}\m{y}\m{%
\in}\m{A}\m{\,y}\m{<}\m{x}\m{)}\m{\rightarrow}\m{\exists}\m{x}\m{\in}\m{%
\mathbb{R}}\m{(}\m{\forall}\m{y}\m{\in}\m{A}\m{\lnot}\m{x}\m{<}\m{y}\m{\wedge}\m{%
\forall}\m{y}\m{\in}\m{\mathbb{R}}\m{(}\m{y}\m{<}\m{x}\m{\rightarrow}\m{\exists}%
\m{z}\m{\in}\m{A}\m{\,y}\m{<}\m{z}\m{)}\m{)}\m{)}
\endm

% NOTE: The \m{...} expressions above could be represented as
% $ \vdash ( ( A \subseteq \mathbb{R} \wedge A \ne \varnothing \wedge \exists x \in \mathbb{R} \forall y \in A \,y < x ) \rightarrow \exists x \in \mathbb{R} ( \forall y \in A \lnot x < y \wedge \forall y \in \mathbb{R} ( y < x \rightarrow \exists z \in A \,y < z ) ) ) $

\vskip 2ex

This completes the set of axioms for real and complex numbers.  You may
wish to look at how subtraction, division, and decimal numbers
are defined in \texttt{set.mm}, and for fun look at the proof of $2+
2 = 4$ (theorem \texttt{2p2e4} in \texttt{set.mm})
as discussed in section \ref{2p2e4}.

In \texttt{set.mm} we define the non-negative integers $\mathbb{N}$, the integers
$\mathbb{Z}$, and the rationals $\mathbb{Q}$ as subsets of $\mathbb{R}$.  This leads
to the nice inclusion $\mathbb{N} \subseteq \mathbb{Z} \subseteq \mathbb{Q} \subseteq
\mathbb{R} \subseteq \mathbb{C}$, giving us a uniform framework in which, for
example, a property such as commutativity of complex number addition
automatically applies to integers.  The natural numbers $\mathbb{N}$
are different from the set $\omega$ we defined earlier, but both satisfy
Peano's postulates.

\subsection{Complex Number Axioms in Analysis Texts}

Most analysis texts construct complex numbers as ordered pairs of reals,
leading to construction-dependent properties that satisfy these axioms
but are not stated in their pure form.  (This is also done in
\texttt{set.mm} but our axioms are extracted from that construction.)
Other texts will simply state that $\mathbb{R}$ is a ``complete ordered
subfield of $\mathbb{C}$,'' leading to redundant axioms when this phrase
is completely expanded out.  In fact I have not seen a text with the
axioms in the explicit form above.
None of these axioms is unique individually, but this carefully worked out
collection of axioms is the result of years of work
by the Metamath community.

\subsection{Eliminating Unnecessary Complex Number Axioms}

We once had more axioms for real and complex numbers, but over years of time
we (the Metamath community)
have found ways to eliminate them (by proving them from other axioms)
or weaken them (by making weaker claims without reducing what
can be proved).
In particular, here are statements that used to be complex number
axioms but have since been formally proven (with Metamath) to be redundant:

\begin{itemize}
\item
  $\mathbb{C} \in V$.
  At one time this was listed as a ``complex number axiom.''
  However, this is not properly speaking a complex number axiom,
  and in any case its proof uses axioms of set theory.
  Proven redundant by Mario Carneiro\index{Carneiro, Mario} on
  17-Nov-2014 (see \texttt{axcnex}).
\item
  $((A \in \mathbb{C} \land B \in \mathbb{C}$) $\rightarrow$
  $(A + B) = (B + A))$.
  Proved redundant by Eric Schmidt\index{Schmidt, Eric} on 19-Jun-2012,
  and formalized by Scott Fenton\index{Fenton, Scott} on 3-Jan-2013
  (see \texttt{addcom}).
\item
  $(A \in \mathbb{C} \rightarrow (A + 0) = A)$.
  Proved redundant by Eric Schmidt on 19-Jun-2012,
  and formalized by Scott Fenton on 3-Jan-2013
  (see \texttt{addid1}).
\item
  $(A \in \mathbb{C} \rightarrow \exists x \in \mathbb{C} (A + x) = 0)$.
  Proved redundant by Eric Schmidt and formalized on 21-May-2007
  (see \texttt{cnegex}).
\item
  $((A \in \mathbb{C} \land A \ne 0) \rightarrow \exists x \in \mathbb{C} (A \cdot x) = 1)$.
  Proved redundant by Eric Schmidt and formalized on 22-May-2007
  (see \texttt{recex}).
\item
  $0 \in \mathbb{R}$.
  Proved redundant by Eric Schmidt on 19-Feb-2005 and formalized 21-May-2007
  (see \texttt{0re}).
\end{itemize}

We could eliminate 0 as an axiomatic object by defining it as
$( ( i \cdot i ) + 1 )$
and replacing it with this expression throughout the axioms. If this
is done, axiom ax-i2m1 becomes redundant. However, the remaining axioms
would become longer and less intuitive.

Eric Schmidt's paper analyzing this axiom system \cite{Schmidt}
presented a proof that these remaining axioms,
with the possible exception of ax-mulcom, are independent of the others.
It is currently an open question if ax-mulcom is independent of the others.

\section{Two Plus Two Equals Four}\label{2p2e4}

Here is a proof that $2 + 2 = 4$, as proven in the theorem \texttt{2p2e4}
in the database \texttt{set.mm}.
This is a useful demonstration of what a Metamath proof can look like.
This proof may have more steps than you're used to, but each step is rigorously
proven all the way back to the axioms of logic and set theory.
This display was originally generated by the Metamath program
as an {\sc HTML} file.

In the table showing the proof ``Step'' is the sequential step number,
while its associated ``Expression'' is an expression that we have proved.
``Ref'' is the name of a theorem or axiom that justifies that expression,
and ``Hyp'' refers to previous steps (if any) that the theorem or axiom
needs so that we can use it.  Expressions are indented further than
the expressions that depend on them to show their interdependencies.

\begin{table}[!htbp]
\caption{Two plus two equals four}
\begin{tabular}{lllll}
\textbf{Step} & \textbf{Hyp} & \textbf{Ref} & \textbf{Expression} & \\
1  &       & df-2    & $ \; \; \vdash 2 = 1 + 1$  & \\
2  & 1     & oveq2i  & $ \; \vdash (2 + 2) = (2 + (1 + 1))$ & \\
3  &       & df-4    & $ \; \; \vdash 4 = (3 + 1)$ & \\
4  &       & df-3    & $ \; \; \; \vdash 3 = (2 + 1)$ & \\
5  & 4     & oveq1i  & $ \; \; \vdash (3 + 1) = ((2 + 1) + 1)$ & \\
6  &       & 2cn     & $ \; \; \; \vdash 2 \in \mathbb{C}$ & \\
7  &       & ax-1cn  & $ \; \; \; \vdash 1 \in \mathbb{C}$ & \\
8  & 6,7,7 & addassi & $ \; \; \vdash ((2 + 1) + 1) = (2 + (1 + 1))$ & \\
9  & 3,5,8 & 3eqtri  & $ \; \vdash 4 = (2 + (1 + 1))$ & \\
10 & 2,9   & eqtr4i  & $ \vdash (2 + 2) = 4$ & \\
\end{tabular}
\end{table}

Step 1 says that we can assert that $2 = 1 + 1$ because it is
justified by \texttt{df-2}.
What is \texttt{df-2}?
It is simply the definition of $2$, which in our system is defined as being
equal to $1 + 1$.  This shows how we can use definitions in proofs.

Look at Step 2 of the proof. In the Ref column, we see that it references
a previously proved theorem, \texttt{oveq2i}.
It turns out that
theorem \texttt{oveq2i} requires a
hypothesis, and in the Hyp column of Step 2 we indicate that Step 1 will
satisfy (match) this hypothesis.
If we looked at \texttt{oveq2i}
we would find that it proves that given some hypothesis
$A = B$, we can prove that $( C F A ) = ( C F B )$.
If we use \texttt{oveq2i} and apply step 1's result as the hypothesis,
that will mean that $A = 2$ and $B = ( 1 + 1 )$ within this use of
\texttt{oveq2i}.
We are free to select any value of $C$ and $F$ (subject to syntax constraints),
so we are free to select $C = 2$ and $F = +$,
producing our desired result,
$ (2 + 2) = (2 + (1 + 1))$.

Step 2 is an example of substitution.
In the end, every step in every proof uses only this one substitution rule.
All the rules of logic, and all the axioms, are expressed so that
they can be used via this one substitution rule.
So once you master substitution, you can master every Metamath proof,
no exceptions.

Each step is clear and can be immediately checked.
In the {\sc HTML} display you can even click on each reference to see why it is
justified, making it easy to see why the proof works.

\section{Deduction}\label{deduction}

Strictly speaking,
a deduction (also called an inference) is a kind of statement that needs
some hypotheses to be true in order for its conclusion to be true.
A theorem, on the other hand, has no hypotheses.
Informally we often call both of them theorems, but in this section we
will stick to the strict definitions.

It sometimes happens that we have proved a deduction of the form
$\varphi \Rightarrow \psi$\index{$\Rightarrow$}
(given hypothesis $\varphi$ we can prove $\psi$)
and we want to then prove a theorem of the form
$\varphi \rightarrow \psi$.

Converting a deduction (which uses a hypothesis) into a theorem
(which does not) is not as simple as you might think.
The deduction says, ``if we can prove $\varphi$ then we can prove $\psi$,''
which is in some sense weaker than saying
``$\varphi$ implies $\psi$.''
There is no axiom of logic that permits us to directly obtain the theorem
given the deduction.\footnote{
The conversion of a deduction to a theorem does not even hold in general
for quantum propositional calculus,
which is a weak subset of classical propositional calculus.
It has been shown that adding the Standard Deduction Theorem (discussed below)
to quantum propositional calculus turns it into classical
propositional calculus!
}

This is in contrast to going the other way.
If we have the theorem ($\varphi \rightarrow \psi$),
it is easy to recover the deduction
($\varphi \Rightarrow \psi$)
using modus ponens\index{modus ponens}
(\texttt{ax-mp}; see section \ref{axmp}).

In the following subsections we first discuss the standard deduction theorem
(the traditional but awkward way to convert deductions into theorems) and
the weak deduction theorem (a limited version of the standard deduction
theorem that is easier to use and was once widely used in
\texttt{set.mm}\index{set theory database (\texttt{set.mm})}).
In section \ref{deductionstyle} we discuss
deduction style, the newer approach we now recommend in most cases.
Deduction style uses ``deduction form,'' a form that
prefixes each hypothesis (other than definitions) and the
conclusion with a universal antecedent (``$\varphi \rightarrow$'').
Deduction style is widely used in \texttt{set.mm},
so it is useful to understand it and \textit{why} it is widely used.
Section \ref{naturaldeduction}
briefly discusses our approach for using natural deduction
within \texttt{set.mm},
as that approach is deeply related to deduction style.
We conclude with a summary of the strengths of
our approach, which we believe are compelling.

\subsection{The Standard Deduction Theorem}\label{standarddeductiontheorem}

It is possible to make use of information
contained in the deduction or its proof to assist us with the proof of
the related theorem.
In traditional logic books, there is a metatheorem called the
Deduction Theorem\index{Deduction Theorem}\index{Standard Deduction Theorem},
discovered independently by Herbrand and Tarski around 1930.
The Deduction Theorem, which we often call the Standard Deduction Theorem,
provides an algorithm for constructing a proof of a theorem from the
proof of its corresponding deduction. See, for example,
\cite[p.~56]{Margaris}\index{Margaris, Angelo}.
To construct a proof for a theorem, the
algorithm looks at each step in the proof of the original deduction and
rewrites the step with several steps wherein the hypothesis is eliminated
and becomes an antecedent.

In ordinary mathematics, no one actually carries out the algorithm,
because (in its most basic form) it involves an exponential explosion of
the number of proof steps as more hypotheses are eliminated. Instead,
the Standard Deduction Theorem is invoked simply to claim that it can
be done in principle, without actually doing it.
What's more, the algorithm is not as simple as it might first appear
when applying it rigorously.
There is a subtle restriction on the Standard Deduction Theorem
that must be taken into account involving the axiom of generalization
when working with predicate calculus (see the literature for more detail).

One of the goals of Metamath is to let you plainly see, with as few
underlying concepts as possible, how mathematics can be derived directly
from the axioms, and not indirectly according to some hidden rules
buried inside a program or understood only by logicians. If we added
the Standard Deduction Theorem to the language and proof verifier,
that would greatly complicate both and largely defeat Metamath's goal
of simplicity. In principle, we could show direct proofs by expanding
out the proof steps generated by the algorithm of the Standard Deduction
Theorem, but that is not feasible in practice because the number of proof
steps quickly becomes huge, even astronomical.
Since the algorithm of the Standard Deduction Theorem is driven by the proof,
we would have to go through that proof
all over again---starting from axioms---in order to obtain the theorem form.
In terms of proof length, there would be no savings over just
proving the theorem directly instead of first proving the deduction form.

\subsection{Weak Deduction Theorem}\label{weakdeductiontheorem}

We have developed
a more efficient method for proving a theorem from a deduction
that can be used instead of the Standard Deduction Theorem
in many (but not all) cases.
We call this more efficient method the
Weak Deduction Theorem\index{Weak Deduction Theorem}.\footnote{
There is also an unrelated ``Weak Deduction Theorem''
in the field of relevance logic, so to avoid confusion we could call
ours the ``Weak Deduction Theorem for Classical Logic.''}
Unlike the Standard Deduction Theorem, the Weak Deduction Theorem produces the
theorem directly from a special substitution instance of the deduction,
using a small, fixed number of steps roughly proportional to the length
of the final theorem.

If you come to a proof referencing the Weak Deduction Theorem
\texttt{dedth} (or one of its variants \texttt{dedthxx}),
here is how to follow the proof without getting into the details:
just click on the theorem referenced in the step
just before the reference to \texttt{dedth} and ignore everything else.
Theorem \texttt{dedth} simply turns a hypothesis into an antecedent
(i.e. the hypothesis followed by $\rightarrow$
is placed in front of the assertion, and the hypothesis
itself is eliminated) given certain conditions.

The Weak Deduction Theorem
eliminates a hypothesis $\varphi$, making it become an antecedent.
It does this by proving an expression
$ \varphi \rightarrow \psi $ given two hypotheses:
(1)
$ ( A = {\rm if} ( \varphi , A , B ) \rightarrow ( \varphi \leftrightarrow \chi ) ) $
and
(2) $\chi$.
Note that it requires that a proof exists for $\varphi$ when the class variable
$A$ is replaced with a specific class $B$. The hypothesis $\chi$
should be assigned to the inference.
You can see the details of the proof of the Weak Deduction Theorem
in theorem \texttt{dedth}.

The Weak Deduction Theorem
is probably easier to understand by studying proofs that make use of it.
For example, let's look at the proof of \texttt{renegcl}, which proves that
$ \vdash ( A \in \mathbb{R} \rightarrow - A \in \mathbb{R} )$:

\needspace{4\baselineskip}
\begin{longtabu} {l l l X}
\textbf{Step} & \textbf{Hyp} & \textbf{Ref} & \textbf{Expression} \\
  1 &  & negeq &
  $\vdash$ $($ $A$ $=$ ${\rm if}$ $($ $A$ $\in$ $\mathbb{R}$ $,$ $A$ $,$ $1$ $)$ $\rightarrow$
  $\textrm{-}$ $A$ $=$ $\textrm{-}$ ${\rm if}$ $($ $A$ $\in$ $\mathbb{R}$
  $,$ $A$ $,$ $1$ $)$ $)$ \\
 2 & 1 & eleq1d &
    $\vdash$ $($ $A$ $=$ ${\rm if}$ $($ $A$ $\in$ $\mathbb{R}$ $,$ $A$ $,$ $1$ $)$ $\rightarrow$ $($
    $\textrm{-}$ $A$ $\in$ $\mathbb{R}$ $\leftrightarrow$
    $\textrm{-}$ ${\rm if}$ $($ $A$ $\in$ $\mathbb{R}$ $,$ $A$ $,$ $1$ $)$ $\in$
    $\mathbb{R}$ $)$ $)$ \\
 3 &  & 1re & $\vdash 1 \in \mathbb{R}$ \\
 4 & 3 & elimel &
   $\vdash {\rm if} ( A \in \mathbb{R} , A , 1 ) \in \mathbb{R}$ \\
 5 & 4 & renegcli &
   $\vdash \textrm{-} {\rm if} ( A \in \mathbb{R} , A , 1 ) \in \mathbb{R}$ \\
 6 & 2,5 & dedth &
   $\vdash ( A \in \mathbb{R} \rightarrow \textrm{-} A \in \mathbb{R}$ ) \\
\end{longtabu}

The somewhat strange-looking steps in \texttt{renegcl} before step 5 are
technical stuff that makes this magic work, and they can be ignored
for a quick overview of the proof. To continue following the ``important''
part of the proof of \texttt{renegcl},
you can look at the reference to \texttt{renegcli} at step 5.

That said, let's briefly look at how
\texttt{renegcl} uses the
Weak Deduction Theorem (\texttt{dedth}) to do its job,
in case you want to do something similar or want understand it more deeply.
Let's work backwards in the proof of \texttt{renegcl}.
Step 6 applies \texttt{dedth} to produce our goal result
$ \vdash ( A \in \mathbb{R} \rightarrow\, - A \in \mathbb{R} )$.
This requires on the one hand the (substituted) deduction
\texttt{renegcli} in step 5.
By itself \texttt{renegcli} proves the deduction
$ \vdash A \in \mathbb{R} \Rightarrow\, \vdash - A \in \mathbb{R}$;
this is the deduction form we are trying to turn into theorem form,
and thus
\texttt{renegcli} has a separate hypothesis that must be fulfilled.
To fulfill the hypothesis of the invocation of
\texttt{renegcli} in step 5, it is eventually
reduced to the already proven theorem $1 \in \mathbb{R}$ in step 3.
Step 4 connects steps 3 and 5; step 4 invokes
\texttt{elimel}, a special case of \texttt{elimhyp} that eliminates
a membership hypothesis for the weak deduction theorem.
On the other hand, the equivalence of the conclusion of
\texttt{renegcl}
$( - A \in \mathbb{R} )$ and the substituted conclusion of
\texttt{renegcli} must be proven, which is done in steps 2 and 1.

The weak deduction theorem has limitations.
In particular, we must be able to prove a special case of the deduction's
hypothesis as a stand-alone theorem.
For example, we used $1 \in \mathbb{R}$ in step 3 of \texttt{renegcl}.

We used to use the weak deduction theorem
extensively within \texttt{set.mm}.
However, we now recommend applying ``deduction style''
instead in most cases, as deduction style is
often an easier and clearer approach.
Therefore, we will now describe deduction style.

\subsection{Deduction Style}\label{deductionstyle}

We now prefer to write assertions in ``deduction form''
instead of writing a proof that would require use of the standard or
weak deduction theorem.
We call this appraoch
``deduction style.''\index{deduction style}

It will be easier to explain this by first defining some terms:

\begin{itemize}
\item \textbf{closed form}\index{closed form}\index{forms!closed}:
A kind of assertion (theorem) with no hypotheses.
Typically its label has no special suffix.
An example is \texttt{unss}, which states:
$\vdash ( ( A \subseteq C \wedge B \subseteq C ) \leftrightarrow ( A \cup B )
\subseteq C )\label{eq:unss}$
\item \textbf{deduction form}\index{deduction form}\index{forms!deduction}:
A kind of assertion with one or more hypotheses
where the conclusion is an implication with
a wff variable as the antecedent (usually $\varphi$), and every hypothesis
(\$e statement)
is either (1) an implication with the same antecedent as the conclusion or
(2) a definition.
A definition
can be for a class variable (this is a class variable followed by ``='')
or a wff variable (this is a wff variable followed by $\leftrightarrow$);
class variable definitions are more common.
In practice, a proof
in deduction form will also contain many steps that are implications
where the antecedent is either that wff variable (normally $\varphi$)
or is
a conjunction (...$\land$...) including that wff variable ($\varphi$).
If an assertion is in deduction form, and other forms are also available,
then we suffix its label with ``d.''
An example is \texttt{unssd}, which states\footnote{
For brevity we show here (and in other places)
a $\&$\index{$\&$} between hypotheses\index{hypotheses}
and a $\Rightarrow$\index{$\Rightarrow$}\index{conclusion}
between the hypotheses and the conclusion.
This notation is technically not part of the Metamath language, but is
instead a convenient abbreviation to show both the hypotheses and conclusion.}:
$\vdash ( \varphi \rightarrow A \subseteq C )\quad\&\quad \vdash ( \varphi
    \rightarrow B \subseteq C )\quad\Rightarrow\quad \vdash ( \varphi
    \rightarrow ( A \cup B ) \subseteq C )\label{eq:unssd}$
\item \textbf{inference form}\index{inference form}\index{forms!inference}:
A kind of assertion with one or more hypotheses that is not in deduction form
(e.g., there is no common antecedent).
If an assertion is in inference form, and other forms are also available,
then we suffix its label with ``i.''
An example is \texttt{unssi}, which states:
$\vdash A \subseteq C\quad\&\quad \vdash B \subseteq C\quad\Rightarrow\quad
    \vdash ( A \cup B ) \subseteq C\label{eq:unssi}$
\end{itemize}

When using deduction style we express an assertion in deduction form.
This form prefixes each hypothesis (other than definitions) and the
conclusion with a universal antecedent (``$\varphi \rightarrow$'').
The antecedent (e.g., $\varphi$)
mimics the context handled in the deduction theorem, eliminating
the need to directly use the deduction theorem.

Once you have an assertion in deduction form, you can easily convert it
to inference form or closed form:

\begin{itemize}
\item To
prove some assertion Ti in inference form, given assertion Td in deduction
form, there is a simple mechanical process you can use. First take each
Ti hypothesis and insert a \texttt{T.} $\rightarrow$ prefix (``true implies'')
using \texttt{a1i}. You
can then use the existing assertion Td to prove the resulting conclusion
with a \texttt{T.} $\rightarrow$ prefix.
Finally, you can remove that prefix using \texttt{mptru},
resulting in the conclusion you wanted to prove.
\item To
prove some assertion T in closed form, given assertion Td in deduction
form, there is another simple mechanical process you can use. First,
select an expression that is the conjunction (...$\land$...) of all of the
consequents of every hypothesis of Td. Next, prove that this expression
implies each of the separate hypotheses of Td in turn by eliminating
conjuncts (there are a variety of proven assertions to do this, including
\texttt{simpl},
\texttt{simpr},
\texttt{3simpa},
\texttt{3simpb},
\texttt{3simpc},
\texttt{simp1},
\texttt{simp2},
and
\texttt{simp3}).
If the
expression has nested conjunctions, inner conjuncts can be broken out by
chaining the above theorems with \texttt{syl}
(see section \ref{syl}).\footnote{
There are actually many theorems
(labeled simp* such as \texttt{simp333}) that break out inner conjuncts in one
step, but rather than learning them you can just use the chaining we
just described to prove them, and then let the Metamath program command
\texttt{minimize{\char`\_}with}\index{\texttt{minimize{\char`\_}with} command}
figure out the right ones needed to collapse them.}
As your final step, you can then apply the already-proven assertion Td
(which is in deduction form), proving assertion T in closed form.
\end{itemize}

We can also easily convert any assertion T in closed form to its related
assertion Ti in inference form by applying
modus ponens\index{modus ponens} (see section \ref{axmp}).

The deduction form antecedent can also be used to represent the context
necessary to support natural deduction systems, so we will now
discuss natural deduction.

\subsection{Natural Deduction}\label{naturaldeduction}

Natural deduction\index{natural deduction}
(ND) systems, as such, were originally introduced in
1934 by two logicians working independently: Ja\'skowski and Gentzen. ND
systems are supposed to reconstruct, in a formally proper way, traditional
ways of mathematical reasoning (such as conditional proof, indirect proof,
and proof by cases). As reconstructions they were naturally influenced
by previous work, and many specific ND systems and notations have been
developed since their original work.

There are many ND variants, but
Indrzejczak \cite[p.~31-32]{Indrzejczak}\index{Indrzejczak, Andrzej}
suggests that any natural deductive system must satisfy at
least these three criteria:

\begin{itemize}
\item ``There are some means for entering assumptions into a proof and
also for eliminating them. Usually it requires some bookkeeping devices
for indicating the scope of an assumption, and showing that a part of
a proof depending on eliminated assumption is discharged.
\item There are no (or, at least, a very limited set of) axioms, because
their role is taken over by the set of primitive rules for introduction
and elimination of logical constants which means that elementary
inferences instead of formulae are taken as primitive.
\item (A genuine) ND system admits a lot of freedom in proof construction
and possibility of applying several proof search strategies, like
conditional proof, proof by cases, proof by reductio ad absurdum etc.''
\end{itemize}

The Metamath Proof Explorer (MPE) as defined in \texttt{set.mm}
is fundamentally a Hilbert-style system.
That is, MPE is based on a larger number of axioms (compared
to natural deduction systems), a very small set of rules of inference
(modus ponens), and the context is not changed by the rules of inference
in the middle of a proof. That said, MPE proofs can be developed using
the natural deduction (ND) approach as originally developed by Ja\'skowski
and Gentzen.

The most common and recommended approach for applying ND in MPE is to use
deduction form\index{deduction form}%
\index{forms!deduction}
and apply the MPE proven assertions that are equivalent to ND rules.
For example, MPE's \texttt{jca} is equivalent to ND rule $\land$-I
(and-insertion).
We maintain a list of equivalences that you may consult.
This approach for applying an ND approach within MPE relies on Metamath's
wff metavariables in an essential way, and is described in more detail
in the presentation ``Natural Deductions in the Metamath Proof Language''
by Mario Carneiro \cite{CarneiroND}\index{Carneiro, Mario}.

In this style many steps are an implication, whose antecedent mimics
the context ($\Gamma$) of most ND systems. To add an assumption, simply add
it to the implication antecedent (typically using
\texttt{simpr}),
and use that
new antecedent for all later claims in the same scope. If you wish to
use an assertion in an ND hypothesis scope that is outside the current
ND hypothesis scope, modify the assertion so that the ND hypothesis
assumption is added to its antecedent (typically using \texttt{adantr}). Most
proof steps will be proved using rules that have hypotheses and results
of the form $\varphi \rightarrow$ ...

An example may make this clearer.
Let's show theorem 5.5 of
\cite[p.~18]{Clemente}\index{Clemente Laboreo, Daniel}
along with a line by line translation using the usual
translation of natural deduction (ND) in the Metamath Proof Explorer
(MPE) notation (this is proof \texttt{ex-natded5.5}).
The proof's original goal was to prove
$\lnot \psi$ given two hypotheses,
$( \psi \rightarrow \chi )$ and $ \lnot \chi$.
We will translate these statements into MPE deduction form
by prefixing them all with $\varphi \rightarrow$.
As a result, in MPE the goal is stated as
$( \varphi \rightarrow \lnot \psi )$, and the two hypotheses are stated as
$( \varphi \rightarrow ( \psi \rightarrow \chi ) )$ and
$( \varphi \rightarrow \lnot \chi )$.

The following table shows the proof in Fitch natural deduction style
and its MPE equivalent.
The \textit{\#} column shows the original numbering,
\textit{MPE\#} shows the number in the equivalent MPE proof
(which we will show later),
\textit{ND Expression} shows the original proof claim in ND notation,
and \textit{MPE Translation} shows its translation into MPE
as discussed in this section.
The final columns show the rationale in ND and MPE respectively.

\needspace{4\baselineskip}
{\setlength{\extrarowsep}{4pt} % Keep rows from being too close together
\begin{longtabu}   { @{} c c X X X X }
\textbf{\#} & \textbf{MPE\#} & \textbf{ND Ex\-pres\-sion} &
\textbf{MPE Trans\-lation} & \textbf{ND Ration\-ale} &
\textbf{MPE Ra\-tio\-nale} \\
\endhead

1 & 2;3 &
$( \psi \rightarrow \chi )$ &
$( \varphi \rightarrow ( \psi \rightarrow \chi ) )$ &
Given &
\$e; \texttt{adantr} to put in ND hypothesis \\

2 & 5 &
$ \lnot \chi$ &
$( \varphi \rightarrow \lnot \chi )$ &
Given &
\$e; \texttt{adantr} to put in ND hypothesis \\

3 & 1 &
... $\vert$ $\psi$ &
$( \varphi \rightarrow \psi )$ &
ND hypothesis assumption &
\texttt{simpr} \\

4 & 4 &
... $\chi$ &
$( ( \varphi \land \psi ) \rightarrow \chi )$ &
$\rightarrow$\,E 1,3 &
\texttt{mpd} 1,3 \\

5 & 6 &
... $\lnot \chi$ &
$( ( \varphi \land \psi ) \rightarrow \lnot \chi )$ &
IT 2 &
\texttt{adantr} 5 \\

6 & 7 &
$\lnot \psi$ &
$( \varphi \rightarrow \lnot \psi )$ &
$\land$\,I 3,4,5 &
\texttt{pm2.65da} 4,6 \\

\end{longtabu}
}


The original used Latin letters; we have replaced them with Greek letters
to follow Metamath naming conventions and so that it is easier to follow
the Metamath translation. The Metamath line-for-line translation of
this natural deduction approach precedes every line with an antecedent
including $\varphi$ and uses the Metamath equivalents of the natural deduction
rules. To add an assumption, the antecedent is modified to include it
(typically by using \texttt{adantr};
\texttt{simpr} is useful when you want to
depend directly on the new assumption, as is shown here).

In Metamath we can represent the two given statements as these hypotheses:

\needspace{2\baselineskip}
\begin{itemize}
\item ex-natded5.5.1 $\vdash ( \varphi \rightarrow ( \psi \rightarrow \chi ) )$
\item ex-natded5.5.2 $\vdash ( \varphi \rightarrow \lnot \chi )$
\end{itemize}

\needspace{4\baselineskip}
Here is the proof in Metamath as a line-by-line translation:

\begin{longtabu}   { l l l X }
\textbf{Step} & \textbf{Hyp} & \textbf{Ref} & \textbf{Ex\-pres\-sion} \\
\endhead
1 & & simpr & $\vdash ( ( \varphi \land \psi ) \rightarrow \psi )$ \\
2 & & ex-natded5.5.1 &
  $\vdash ( \varphi \rightarrow ( \psi \rightarrow \chi ) )$ \\
3 & 2 & adantr &
 $\vdash ( ( \varphi \land \psi ) \rightarrow ( \psi \rightarrow \chi ) )$ \\
4 & 1, 3 & mpd &
 $\vdash ( ( \varphi \land \psi ) \rightarrow \chi ) $ \\
5 & & ex-natded5.5.2 &
 $\vdash ( \varphi \rightarrow \lnot \chi )$ \\
6 & 5 & adantr &
 $\vdash ( ( \varphi \land \psi ) \rightarrow \lnot \chi )$ \\
7 & 4, 6 & pm2.65da &
 $\vdash ( \varphi \rightarrow \lnot \psi )$ \\
\end{longtabu}

Only using specific natural deduction rules directly can lead to very
long proofs, for exactly the same reason that only using axioms directly
in Hilbert-style proofs can lead to very long proofs.
If the goal is short and clear proofs,
then it is better to reuse already-proven assertions
in deduction form than to start from scratch each time
and using only basic natural deduction rules.

\subsection{Strengths of Our Approach}

As far as we know there is nothing else in the literature like either the
weak deduction theorem or Mario Carneiro\index{Carneiro, Mario}'s
natural deduction method.
In order to
transform a hypothesis into an antecedent, the literature's standard
``Deduction Theorem''\index{Deduction Theorem}\index{Standard Deduction Theorem}
requires metalogic outside of the notions provided
by the axiom system. We instead generally prefer to use Mario Carneiro's
natural deduction method, then use the weak deduction theorem in cases
where that is difficult to apply, and only then use the full standard
deduction theorem as a last resort.

The weak deduction theorem\index{Weak Deduction Theorem}
does not require any additional metalogic
but converts an inference directly into a closed form theorem, with
a rigorous proof that uses only the axiom system. Unlike the standard
Deduction Theorem, there is no implicit external justification that we
have to trust in order to use it.

Mario Carneiro's natural deduction\index{natural deduction}
method also does not require any new metalogical
notions. It avoids the Deduction Theorem's metalogic by prefixing the
hypotheses and conclusion of every would-be inference with a universal
antecedent (``$\varphi \rightarrow$'') from the very start.

We think it is impressive and satisfying that we can do so much in a
practical sense without stepping outside of our Hilbert-style axiom system.
Of course our axiomatization, which is in the form of schemes,
contains a metalogic of its own that we exploit. But this metalogic
is relatively simple, and for our Deduction Theorem alternatives,
we primarily use just the direct substitution of expressions for
metavariables.

\begin{sloppy}
\section{Exploring the Set The\-o\-ry Data\-base}\label{exploring}
\end{sloppy}
% NOTE: All examples performed in this section are
% recorded wtih "set width 61" % on set.mm as of 2019-05-28
% commit c1e7849557661260f77cfdf0f97ac4354fbb4f4d.

At this point you may wish to study the \texttt{set.mm}\index{set theory
database (\texttt{set.mm})} file in more detail.  Pay particular
attention to the assumptions needed to define wffs\index{well-formed
formula (wff)} (which are not included above), the variable types
(\texttt{\$f}\index{\texttt{\$f} statement} statements), and the
definitions that are introduced.  Start with some simple theorems in
propositional calculus, making sure you understand in detail each step
of a proof.  Once you get past the first few proofs and become familiar
with the Metamath language, any part of the \texttt{set.mm} database
will be as easy to follow, step by step, as any other part---you won't
have to undergo a ``quantum leap'' in mathematical sophistication to be
able to follow a deep proof in set theory.

Next, you may want to explore how concepts such as natural numbers are
defined and described.  This is probably best done in conjunction with
standard set theory textbooks, which can help give you a higher-level
understanding.  The \texttt{set.mm} database provides references that will get
you started.  From there, you will be on your way towards a very deep,
rigorous understanding of abstract mathematics.

The Metamath\index{Metamath} program can help you peruse a Metamath data\-base,
wheth\-er you are trying to figure out how a certain step follows in a proof or
just have a general curiosity.  We will go through some examples of the
commands, using the \texttt{set.mm}\index{set theory database (\texttt{set.mm})}
database provided with the Metamath software.  These should help get you
started.  See Chapter~\ref{commands} for a more detailed description of
the commands.  Note that we have included the full spelling of all commands to
prevent ambiguity with future commands.  In practice you may type just the
characters needed to specify each command keyword\index{command keyword}
unambiguously, often just one or two characters per keyword, and you don't
need to type them in upper case.

First run the Metamath program as described earlier.  You should see the
\verb/MM>/ prompt.  Read in the \texttt{set.mm} file:\index{\texttt{read}
command}

\begin{verbatim}
MM> read set.mm
Reading source file "set.mm"... 34554442 bytes
34554442 bytes were read into the source buffer.
The source has 155711 statements; 2254 are $a and 32250 are $p.
No errors were found.  However, proofs were not checked.
Type VERIFY PROOF * if you want to check them.
\end{verbatim}

As with most examples in this book, what you will see
will be slightly different because we are continuously
improving our databases (including \texttt{set.mm}).

Let's check the database integrity.  This may take a minute or two to run if
your computer is slow.

\begin{verbatim}
MM> verify proof *
0 10%  20%  30%  40%  50%  60%  70%  80%  90% 100%
..................................................
All proofs in the database were verified in 2.84 s.
\end{verbatim}

No errors were reported, so every proof is correct.

You need to know the names (labels) of theorems before you can look at them.
Often just examining the database file(s) with a text editor is the best
approach.  In \texttt{set.mm} there are many detailed comments, especially near
the beginning, that can help guide you. The \texttt{search} command in the
Metamath program is also handy.  The \texttt{comments} qualifier will list the
statements whose associated comment (the one immediately before it) contain a
string you give it.  For example, if you are studying Enderton's {\em Elements
of Set Theory} \cite{Enderton}\index{Enderton, Herbert B.} you may want to see
the references to it in the database.  The search string \texttt{enderton} is not
case sensitive.  (This will not show you all the database theorems that are in
Enderton's book because there is usually only one citation for a given
theorem, which may appear in several textbooks.)\index{\texttt{search}
command}

\begin{verbatim}
MM> search * "enderton" / comments
12067 unineq $p "... Exercise 20 of [Enderton] p. 32 and ..."
12459 undif2 $p "...Corollary 6K of [Enderton] p. 144. (C..."
12953 df-tp $a "...s. Definition of [Enderton] p. 19. (Co..."
13689 unissb $p ".... Exercise 5 of [Enderton] p. 26 and ..."
\end{verbatim}
\begin{center}
(etc.)
\end{center}

Or you may want to see what theorems have something to do with
conjunction (logical {\sc and}).  The quotes around the search
string are optional when there's no ambiguity.\index{\texttt{search}
command}

\begin{verbatim}
MM> search * conjunction / comments
120 a1d $p "...be replaced with a conjunction ( ~ df-an )..."
662 df-bi $a "...viated form after conjunction is introdu..."
1319 wa $a "...ff definition to include conjunction ('and')."
1321 df-an $a "Define conjunction (logical 'and'). Defini..."
1420 imnan $p "...tion in terms of conjunction. (Contribu..."
\end{verbatim}
\begin{center}
(etc.)
\end{center}


Now we will start to look at some details.  Let's look at the first
axiom of propositional calculus
(we could use \texttt{sh st} to abbreviate
\texttt{show statement}).\index{\texttt{show statement} command}

\begin{verbatim}
MM> show statement ax-1/full
Statement 19 is located on line 881 of the file "set.mm".
"Axiom _Simp_.  Axiom A1 of [Margaris] p. 49.  One of the 3
axioms of propositional calculus.  The 3 axioms are also
        ...
19 ax-1 $a |- ( ph -> ( ps -> ph ) ) $.
Its mandatory hypotheses in RPN order are:
  wph $f wff ph $.
  wps $f wff ps $.
The statement and its hypotheses require the variables:  ph
      ps
The variables it contains are:  ph ps


Statement 49 is located on line 11182 of the file "set.mm".
Its statement number for HTML pages is 6.
"Axiom _Simp_.  Axiom A1 of [Margaris] p. 49.  One of the 3
axioms of propositional calculus.  The 3 axioms are also
given as Definition 2.1 of [Hamilton] p. 28.
...
49 ax-1 $a |- ( ph -> ( ps -> ph ) ) $.
Its mandatory hypotheses in RPN order are:
  wph $f wff ph $.
  wps $f wff ps $.
The statement and its hypotheses require the variables:
  ph ps
The variables it contains are:  ph ps
\end{verbatim}

Compare this to \texttt{ax-1} on p.~\pageref{ax1}.  You can see that the
symbol \texttt{ph} is the {\sc ascii} notation for $\varphi$, etc.  To
see the mathematical symbols for any expression you may typeset it in
\LaTeX\ (type \texttt{help tex} for instructions)\index{latex@{\LaTeX}}
or, easier, just use a text editor to look at the comments where symbols
are first introduced in \texttt{set.mm}.  The hypotheses \texttt{wph}
and \texttt{wps} required by \texttt{ax-1} mean that variables
\texttt{ph} and \texttt{ps} must be wffs.

Next we'll pick a simple theorem of propositional calculus, the Principle of
Identity, which is proved directly from the axioms.  We'll look at the
statement then its proof.\index{\texttt{show statement}
command}

\begin{verbatim}
MM> show statement id1/full
Statement 116 is located on line 11371 of the file "set.mm".
Its statement number for HTML pages is 22.
"Principle of identity.  Theorem *2.08 of [WhiteheadRussell]
p. 101.  This version is proved directly from the axioms for
demonstration purposes.
...
116 id1 $p |- ( ph -> ph ) $= ... $.
Its mandatory hypotheses in RPN order are:
  wph $f wff ph $.
Its optional hypotheses are:  wps wch wth wta wet
      wze wsi wrh wmu wla wka
The statement and its hypotheses require the variables:  ph
These additional variables are allowed in its proof:
      ps ch th ta et ze si rh mu la ka
The variables it contains are:  ph
\end{verbatim}

The optional variables\index{optional variable} \texttt{ps}, \texttt{ch}, etc.\ are
available for use in a proof of this statement if we wish, and were we to do
so we would make use of optional hypotheses \texttt{wps}, \texttt{wch}, etc.  (See
Section~\ref{dollaref} for the meaning of ``optional
hypothesis.''\index{optional hypothesis}) The reason these show up in the
statement display is that statement \texttt{id1} happens to be in their scope
(see Section~\ref{scoping} for the definition of ``scope''\index{scope}), but
in fact in propositional calculus we will never make use of optional
hypotheses or variables.  This becomes important after quantifiers are
introduced, where ``dummy'' variables\index{dummy variable} are often needed
in the middle of a proof.

Let's look at the proof of statement \texttt{id1}.  We'll use the
\texttt{show proof} command, which by default suppresses the
``non-essential'' steps that construct the wffs.\index{\texttt{show proof}
command}
We will display the proof in ``lemmon' format (a non-indented format
with explicit previous step number references) and renumber the
displayed steps:

\begin{verbatim}
MM> show proof id1 /lemmon/renumber
1 ax-1           $a |- ( ph -> ( ph -> ph ) )
2 ax-1           $a |- ( ph -> ( ( ph -> ph ) -> ph ) )
3 ax-2           $a |- ( ( ph -> ( ( ph -> ph ) -> ph ) ) ->
                     ( ( ph -> ( ph -> ph ) ) -> ( ph -> ph )
                                                          ) )
4 2,3 ax-mp      $a |- ( ( ph -> ( ph -> ph ) ) -> ( ph -> ph
                                                          ) )
5 1,4 ax-mp      $a |- ( ph -> ph )
\end{verbatim}

If you have read Section~\ref{trialrun}, you'll know how to interpret this
proof.  Step~2, for example, is an application of axiom \texttt{ax-1}.  This
proof is identical to the one in Hamilton's {\em Logic for Mathematicians}
\cite[p.~32]{Hamilton}\index{Hamilton, Alan G.}.

You may want to look at what
substitutions\index{substitution!variable}\index{variable substitution} are
made into \texttt{ax-1} to arrive at step~2.  The command to do this needs to
know the ``real'' step number, so we'll display the proof again without
the \texttt{renumber} qualifier.\index{\texttt{show proof}
command}

\begin{verbatim}
MM> show proof id1 /lemmon
 9 ax-1          $a |- ( ph -> ( ph -> ph ) )
20 ax-1          $a |- ( ph -> ( ( ph -> ph ) -> ph ) )
24 ax-2          $a |- ( ( ph -> ( ( ph -> ph ) -> ph ) ) ->
                     ( ( ph -> ( ph -> ph ) ) -> ( ph -> ph )
                                                          ) )
25 20,24 ax-mp   $a |- ( ( ph -> ( ph -> ph ) ) -> ( ph -> ph
                                                          ) )
26 9,25 ax-mp    $a |- ( ph -> ph )
\end{verbatim}

The ``real'' step number is 20.  Let's look at its details.

\begin{verbatim}
MM> show proof id1 /detailed_step 20
Proof step 20:  min=ax-1 $a |- ( ph -> ( ( ph -> ph ) -> ph )
  )
This step assigns source "ax-1" ($a) to target "min" ($e).
The source assertion requires the hypotheses "wph" ($f, step
18) and "wps" ($f, step 19).  The parent assertion of the
target hypothesis is "ax-mp" ($a, step 25).
The source assertion before substitution was:
    ax-1 $a |- ( ph -> ( ps -> ph ) )
The following substitutions were made to the source
assertion:
    Variable  Substituted with
     ph        ph
     ps        ( ph -> ph )
The target hypothesis before substitution was:
    min $e |- ph
The following substitution was made to the target hypothesis:
    Variable  Substituted with
     ph        ( ph -> ( ( ph -> ph ) -> ph ) )
\end{verbatim}

This shows the substitutions\index{substitution!variable}\index{variable
substitution} made to the variables in \texttt{ax-1}.  References are made to
steps 18 and 19 which are not shown in our proof display.  To see these steps,
you can display the proof with the \texttt{all} qualifier.

Let's look at a slightly more advanced proof of propositional calculus.  Note
that \verb+/\+ is the symbol for $\wedge$ (logical {\sc and}, also
called conjunction).\index{conjunction ($\wedge$)}
\index{logical {\sc and} ($\wedge$)}

\begin{verbatim}
MM> show statement prth/full
Statement 1791 is located on line 15503 of the file "set.mm".
Its statement number for HTML pages is 559.
"Conjoin antecedents and consequents of two premises.  This
is the closed theorem form of ~ anim12d .  Theorem *3.47 of
[WhiteheadRussell] p. 113.  It was proved by Leibniz,
and it evidently pleased him enough to call it
_praeclarum theorema_ (splendid theorem).
...
1791 prth $p |- ( ( ( ph -> ps ) /\ ( ch -> th ) ) -> ( ( ph
      /\ ch ) -> ( ps /\ th ) ) ) $= ... $.
Its mandatory hypotheses in RPN order are:
  wph $f wff ph $.
  wps $f wff ps $.
  wch $f wff ch $.
  wth $f wff th $.
Its optional hypotheses are:  wta wet wze wsi wrh wmu wla wka
The statement and its hypotheses require the variables:  ph
      ps ch th
These additional variables are allowed in its proof:  ta et
      ze si rh mu la ka
The variables it contains are:  ph ps ch th


MM> show proof prth /lemmon/renumber
1 simpl          $p |- ( ( ( ph -> ps ) /\ ( ch -> th ) ) ->
                                               ( ph -> ps ) )
2 simpr          $p |- ( ( ( ph -> ps ) /\ ( ch -> th ) ) ->
                                               ( ch -> th ) )
3 1,2 anim12d    $p |- ( ( ( ph -> ps ) /\ ( ch -> th ) ) ->
                           ( ( ph /\ ch ) -> ( ps /\ th ) ) )
\end{verbatim}

There are references to a lot of unfamiliar statements.  To see what they are,
you may type the following:

\begin{verbatim}
MM> show proof prth /statement_summary
Summary of statements used in the proof of "prth":

Statement simpl is located on line 14748 of the file
"set.mm".
"Elimination of a conjunct.  Theorem *3.26 (Simp) of
[WhiteheadRussell] p. 112. ..."
  simpl $p |- ( ( ph /\ ps ) -> ph ) $= ... $.

Statement simpr is located on line 14777 of the file
"set.mm".
"Elimination of a conjunct.  Theorem *3.27 (Simp) of
[WhiteheadRussell] ..."
  simpr $p |- ( ( ph /\ ps ) -> ps ) $= ... $.

Statement anim12d is located on line 15445 of the file
"set.mm".
"Conjoin antecedents and consequents in a deduction.
..."
  anim12d.1 $e |- ( ph -> ( ps -> ch ) ) $.
  anim12d.2 $e |- ( ph -> ( th -> ta ) ) $.
  anim12d $p |- ( ph -> ( ( ps /\ th ) -> ( ch /\ ta ) ) )
      $= ... $.
\end{verbatim}
\begin{center}
(etc.)
\end{center}

Of course you can look at each of these statements and their proofs, and
so on, back to the axioms of propositional calculus if you wish.

The \texttt{search} command is useful for finding statements when you
know all or part of their contents.  The following example finds all
statements containing \verb@ph -> ps@ followed by \verb@ch -> th@.  The
\verb@$*@ is a wildcard that matches anything; the \texttt{\$} before the
\verb$*$ prevents conflicts with math symbol token names.  The \verb@*@ after
\texttt{SEARCH} is also a wildcard that in this case means ``match any label.''
\index{\texttt{search} command}

% I'm omitting this one, since readers are unlikely to see it:
% 1096 bisymOLD $p |- ( ( ( ph -> ps ) -> ( ch -> th ) ) -> ( (
%   ( ps -> ph ) -> ( th -> ch ) ) -> ( ( ph <-> ps ) -> ( ch
%    <-> th ) ) ) )
\begin{verbatim}
MM> search * "ph -> ps $* ch -> th"
1791 prth $p |- ( ( ( ph -> ps ) /\ ( ch -> th ) ) -> ( ( ph
    /\ ch ) -> ( ps /\ th ) ) )
2455 pm3.48 $p |- ( ( ( ph -> ps ) /\ ( ch -> th ) ) -> ( (
    ph \/ ch ) -> ( ps \/ th ) ) )
117859 pm11.71 $p |- ( ( E. x ph /\ E. y ch ) -> ( ( A. x (
    ph -> ps ) /\ A. y ( ch -> th ) ) <-> A. x A. y ( ( ph /\
    ch ) -> ( ps /\ th ) ) ) )
\end{verbatim}

Three statements, \texttt{prth}, \texttt{pm3.48},
 and \texttt{pm11.71}, were found to match.

To see what axioms\index{axiom} and definitions\index{definition}
\texttt{prth} ultimately depends on for its proof, you can have the
program backtrack through the hierarchy\index{hierarchy} of theorems and
definitions.\index{\texttt{show trace{\char`\_}back} command}

\begin{verbatim}
MM> show trace_back prth /essential/axioms
Statement "prth" assumes the following axioms ($a
statements):
  ax-1 ax-2 ax-3 ax-mp df-bi df-an
\end{verbatim}

Note that the 3 axioms of propositional calculus and the modus ponens rule are
needed (as expected); in addition, there are a couple of definitions that are used
along the way.  Note that Metamath makes no distinction\index{axiom vs.\
definition} between axioms\index{axiom} and definitions\index{definition}.  In
\texttt{set.mm} they have been distinguished artificially by prefixing their
labels\index{labels in \texttt{set.mm}} with \texttt{ax-} and \texttt{df-}
respectively.  For example, \texttt{df-an} defines conjunction (logical {\sc
and}), which is represented by the symbol \verb+/\+.
Section~\ref{definitions} discusses the philosophy of definitions, and the
Metamath language takes a particularly simple, conservative approach by using
the \texttt{\$a}\index{\texttt{\$a} statement} statement for both axioms and
definitions.

You can also have the program compute how many steps a proof
has\index{proof length} if we were to follow it all the way back to
\texttt{\$a} statements.

\begin{verbatim}
MM> show trace_back prth /essential/count_steps
The statement's actual proof has 3 steps.  Backtracking, a
total of 79 different subtheorems are used.  The statement
and subtheorems have a total of 274 actual steps.  If
subtheorems used only once were eliminated, there would be a
total of 38 subtheorems, and the statement and subtheorems
would have a total of 185 steps.  The proof would have 28349
steps if fully expanded back to axiom references.  The
maximum path length is 38.  A longest path is:  prth <-
anim12d <- syl2and <- sylan2d <- ancomsd <- ancom <- pm3.22
<- pm3.21 <- pm3.2 <- ex <- sylbir <- biimpri <- bicomi <-
bicom1 <- bi2 <- dfbi1 <- impbii <- bi3 <- simprim <- impi <-
con1i <- nsyl2 <- mt3d <- con1d <- notnot1 <- con2i <- nsyl3
<- mt2d <- con2d <- notnot2 <- pm2.18d <- pm2.18 <- pm2.21 <-
pm2.21d <- a1d <- syl <- mpd <- a2i <- a2i.1 .
\end{verbatim}

This tells us that we would have to inspect 274 steps if we want to
verify the proof completely starting from the axioms.  A few more
statistics are also shown.  There are one or more paths back to axioms
that are the longest; this command ferrets out one of them and shows it
to you.  There may be a sense in which the longest path length is
related to how ``deep'' the theorem is.

We might also be curious about what proofs depend on the theorem
\texttt{prth}.  If it is never used later on, we could eliminate it as
redundant if it has no intrinsic interest by itself.\index{\texttt{show
usage} command}

% I decided to show the OLD values here.
\begin{verbatim}
MM> show usage prth
Statement "prth" is directly referenced in the proofs of 18
statements:
  mo3 moOLD 2mo 2moOLD euind reuind reuss2 reusv3i opelopabt
  wemaplem2 rexanre rlimcn2 o1of2 o1rlimmul 2sqlem6 spanuni
  heicant pm11.71
\end{verbatim}

Thus \texttt{prth} is directly used by 18 proofs.
We can use the \texttt{/recursive} qualifier to include indirect use:

\begin{verbatim}
MM> show usage prth /recursive
Statement "prth" directly or indirectly affects the proofs of
24214 statements:
  mo3 mo mo3OLD eu2 moOLD eu2OLD eu3OLD mo4f mo4 eu4 mopick
...
\end{verbatim}

\subsection{A Note on the ``Compact'' Proof Format}

The Metamath program will display proofs in a ``compact''\index{compact proof}
format whenever the proof is stored in compressed format in the database.  It
may be be slightly confusing unless you know how to interpret it.
For example,
if you display the complete proof of theorem \texttt{id1} it will start
off as follows:

\begin{verbatim}
MM> show proof id1 /lemmon/all
 1 wph           $f wff ph
 2 wph           $f wff ph
 3 wph           $f wff ph
 4 2,3 wi    @4: $a wff ( ph -> ph )
 5 1,4 wi    @5: $a wff ( ph -> ( ph -> ph ) )
 6 @4            $a wff ( ph -> ph )
\end{verbatim}

\begin{center}
{etc.}
\end{center}

Step 4 has a ``local label,''\index{local label} \texttt{@4}, assigned to it.
Later on, at step 6, the label \texttt{@1} is referenced instead of
displaying the explicit proof for that step.  This technique takes advantage
of the fact that steps in a proof often repeat, especially during the
construction of wffs.  The compact format reduces the number of steps in the
proof display and may be preferred by some people.

If you want to see the normal format with the ``true'' step numbers, you can
use the following workaround:\index{\texttt{save proof} command}

\begin{verbatim}
MM> save proof id1 /normal
The proof of "id1" has been reformatted and saved internally.
Remember to use WRITE SOURCE to save it permanently.
MM> show proof id1 /lemmon/all
 1 wph           $f wff ph
 2 wph           $f wff ph
 3 wph           $f wff ph
 4 2,3 wi        $a wff ( ph -> ph )
 5 1,4 wi        $a wff ( ph -> ( ph -> ph ) )
 6 wph           $f wff ph
 7 wph           $f wff ph
 8 6,7 wi        $a wff ( ph -> ph )
\end{verbatim}

\begin{center}
{etc.}
\end{center}

Note that the original 6 steps are now 8 steps.  However, the format is
now the same as that described in Chapter~\ref{using}.

\chapter{The Metamath Language}
\label{languagespec}

\begin{quote}
  {\em Thus mathematics may be defined as the subject in which we never know
what we are talking about, nor whether what we are saying is true.}
    \flushright\sc  Bertrand Russell\footnote{\cite[p.~84]{Russell2}.}\\
\end{quote}\index{Russell, Bertrand}

Probably the most striking feature of the Metamath language is its almost
complete absence of hard-wired syntax. Metamath\index{Metamath} does not
understand any mathematics or logic other than that needed to construct finite
sequences of symbols according to a small set of simple, built-in rules.  The
only rule it uses in a proof is the substitution of an expression (symbol
sequence) for a variable, subject to a simple constraint to prevent
bound-variable clashes.  The primitive notions built into Metamath involve the
simple manipulation of finite objects (symbols) that we as humans can easily
visualize and that computers can easily deal with.  They seem to be just
about the simplest notions possible that are required to do standard
mathematics.

This chapter serves as a reference manual for the Metamath\index{Metamath}
language. It covers the tedious technical details of the language, some of
which you may wish to skim in a first reading.  On the other hand, you should
pay close attention to the defined terms in {\bf boldface}; they have precise
meanings that are important to keep in mind for later understanding.  It may
be best to first become familiar with the examples in Chapter~\ref{using} to
gain some motivation for the language.

%% Uncomment this when uncommenting section {formalspec} below
If you have some knowledge of set theory, you may wish to study this
chapter in conjunction with the formal set-theoretical description of the
Metamath language in Appendix~\ref{formalspec}.

We will use the name ``Metamath''\index{Metamath} to mean either the Metamath
computer language or the Metamath software associated with the computer
language.  We will not distinguish these two when the context is clear.

The next section contains the complete specification of the Metamath
language.
It serves as an
authoritative reference and presents the syntax in enough detail to
write a parser\index{parsing Metamath} and proof verifier.  The
specification is terse and it is probably hard to learn the language
directly from it, but we include it here for those impatient people who
prefer to see everything up front before looking at verbose expository
material.  Later sections explain this material and provide examples.
We will repeat the definitions in those sections, and you may skip the
next section at first reading and proceed to Section~\ref{tut1}
(p.~\pageref{tut1}).

\section{Specification of the Metamath Language}\label{spec}
\index{Metamath!specification}

\begin{quote}
  {\em Sometimes one has to say difficult things, but one ought to say
them as simply as one knows how.}
    \flushright\sc  G. H. Hardy\footnote{As quoted in
    \cite{deMillo}, p.~273.}\\
\end{quote}\index{Hardy, G. H.}

\subsection{Preliminaries}\label{spec1}

% Space is technically a printable character, so we'll word things
% carefully so it's unambiguous.
A Metamath {\bf database}\index{database} is built up from a top-level source
file together with any source files that are brought in through file inclusion
commands (see below).  The only characters that are allowed to appear in a
Metamath source file are the 94 non-whitespace printable {\sc
ascii}\index{ascii@{\sc ascii}} characters, which are digits, upper and lower
case letters, and the following 32 special
characters\index{special characters}:\label{spec1chars}

\begin{verbatim}
! " # $ % & ' ( ) * + , - . / :
; < = > ? @ [ \ ] ^ _ ` { | } ~
\end{verbatim}
plus the following characters which are the ``white space'' characters:
space (a printable character),
tab, carriage return, line feed, and form feed.\label{whitespace}
We will use \texttt{typewriter}
font to display the printable characters.

A Metamath database consists of a sequence of three kinds of {\bf
tokens}\index{token} separated by {\bf white space}\index{white space}
(which is any sequence of one or more white space characters).  The set
of {\bf keyword}\index{keyword} tokens is \texttt{\$\char`\{},
\texttt{\$\char`\}}, \texttt{\$c}, \texttt{\$v}, \texttt{\$f},
\texttt{\$e}, \texttt{\$d}, \texttt{\$a}, \texttt{\$p}, \texttt{\$.},
\texttt{\$=}, \texttt{\$(}, \texttt{\$)}, \texttt{\$[}, and
\texttt{\$]}.  The last four are called {\bf auxiliary}\index{auxiliary
keyword} or preprocessing keywords.  A {\bf label}\index{label} token
consists of any combination of letters, digits, and the characters
hyphen, underscore, and period.  A {\bf math symbol}\index{math symbol}
token may consist of any combination of the 93 printable standard {\sc
ascii} characters other than space or \texttt{\$}~. All tokens are
case-sensitive.

\subsection{Preprocessing}

The token \texttt{\$(} begins a {\bf comment} and
\texttt{\$)} ends a comment.\index{\texttt{\$(}
and \texttt{\$)} auxiliary keywords}\index{comment}
Comments may contain any of
the 94 non-whitespace printable characters and white space,
except they may not contain the
2-character sequences \texttt{\$(} or \texttt{\$)} (comments do not nest).
Comments are ignored (treated
like white space) for the purpose of parsing, e.g.,
\texttt{\$( \$[ \$)} is a comment.
See p.~\pageref{mathcomments} for comment typesetting conventions; these
conventions may be ignored for the purpose of parsing.

A {\bf file inclusion command} consists of \texttt{\$[} followed by a file name
followed by \texttt{\$]}.\index{\texttt{\$[} and \texttt{\$]} auxiliary
keywords}\index{included file}\index{file inclusion}
It is only allowed in the outermost scope (i.e., not between
\texttt{\$\char`\{} and \texttt{\$\char`\}})
and must not be inside a statement (e.g., it may not occur
between the label of a \texttt{\$a} statement and its \texttt{\$.}).
The file name may not
contain a \texttt{\$} or white space.  The file must exist.
The case-sensitivity
of its name follows the conventions of the operating system.  The contents of
the file replace the inclusion command.
Included files may include other files.
Only the first reference to a given file is included; any later
references to the same file (whether in the top-level file or in included
files) cause the inclusion command to be ignored (treated like white space).
A verifier may assume that file names with different strings
refer to different files for the purpose of ignoring later references.
A file self-reference is ignored, as is any reference to the top-level file
(to avoid loops).
Included files may not include a \texttt{\$(} without a matching \texttt{\$)},
may not include a \texttt{\$[} without a matching \texttt{\$]}, and may
not include incomplete statements (e.g., a \texttt{\$a} without a matching
\texttt{\$.}).
It is currently unspecified if path references are relative to the process'
current directory or the file's containing directory, so databases should
avoid using pathname separators (e.g., ``/'') in file names.

Like all tokens, the \texttt{\$(}, \texttt{\$)}, \texttt{\$[}, and \texttt{\$]} keywords
must be surrounded by white space.

\subsection{Basic Syntax}

After preprocessing, a database will consist of a sequence of {\bf
statements}.
These are the scoping statements \texttt{\$\char`\{} and
\texttt{\$\char`\}}, along with the \texttt{\$c}, \texttt{\$v},
\texttt{\$f}, \texttt{\$e}, \texttt{\$d}, \texttt{\$a}, and \texttt{\$p}
statements.

A {\bf scoping statement}\index{scoping statement} consists only of its
keyword, \texttt{\$\char`\{} or \texttt{\$\char`\}}.
A \texttt{\$\char`\{} begins a {\bf
block}\index{block} and a matching \texttt{\$\char`\}} ends the block.
Every \texttt{\$\char`\{}
must have a matching \texttt{\$\char`\}}.
Defining it recursively, we say a block
contains a sequence of zero or more tokens other
than \texttt{\$\char`\{} and \texttt{\$\char`\}} and
possibly other blocks.  There is an {\bf outermost
block}\index{block!outermost} not bracketed by \texttt{\$\char`\{} \texttt{\$\char`\}}; the end
of the outermost block is the end of the database.

% LaTeX bug? can't do \bf\tt

A {\bf \$v} or {\bf \$c statement}\index{\texttt{\$v} statement}\index{\texttt{\$c}
statement} consists of the keyword token \texttt{\$v} or \texttt{\$c} respectively,
followed by one or more math symbols,
% The word "token" is used to distinguish "$." from the sentence-ending period.
followed by the \texttt{\$.}\ token.
These
statements {\bf declare}\index{declaration} the math symbols to be {\bf
variables}\index{variable!Metamath} or {\bf constants}\index{constant}
respectively. The same math symbol may not occur twice in a given \texttt{\$v} or
\texttt{\$c} statement.

%c%A math symbol becomes an {\bf active}\index{active math symbol}
%c%when declared and stays active until the end of the block in which it is
%c%declared.  A math symbol may not be declared a second time while it is active,
%c%but it may be declared again after it becomes inactive.

A math symbol becomes {\bf active}\index{active math symbol} when declared
and stays active until the end of the block in which it is declared.  A
variable may not be declared a second time while it is active, but it
may be declared again (as a variable, but not as a constant) after it
becomes inactive.  A constant must be declared in the outermost block and may
not be declared a second time.\index{redeclaration of symbols}

A {\bf \$f statement}\index{\texttt{\$f} statement} consists of a label,
followed by \texttt{\$f}, followed by its typecode (an active constant),
followed by an
active variable, followed by the \texttt{\$.}\ token.  A {\bf \$e
statement}\index{\texttt{\$e} statement} consists of a label, followed
by \texttt{\$e}, followed by its typecode (an active constant),
followed by zero or more
active math symbols, followed by the \texttt{\$.}\ token.  A {\bf
hypothesis}\index{hypothesis} is a \texttt{\$f} or \texttt{\$e}
statement.
The type declared by a \texttt{\$f} statement for a given label
is global even if the variable is not
(e.g., a database may not have \texttt{wff P} in one local scope
and \texttt{class P} in another).

A {\bf simple \$d statement}\index{\texttt{\$d} statement!simple}
consists of \texttt{\$d}, followed by two different active variables,
followed by the \texttt{\$.}\ token.  A {\bf compound \$d
statement}\index{\texttt{\$d} statement!compound} consists of
\texttt{\$d}, followed by three or more variables (all different),
followed by the \texttt{\$.}\ token.  The order of the variables in a
\texttt{\$d} statement is unimportant.  A compound \texttt{\$d}
statement is equivalent to a set of simple \texttt{\$d} statements, one
for each possible pair of variables occurring in the compound
\texttt{\$d} statement.  Henceforth in this specification we shall
assume all \texttt{\$d} statements are simple.  A \texttt{\$d} statement
is also called a {\bf disjoint} (or {\bf distinct}) {\bf variable
restriction}.\index{disjoint-variable restriction}

A {\bf \$a statement}\index{\texttt{\$a} statement} consists of a label,
followed by \texttt{\$a}, followed by its typecode (an active constant),
followed by
zero or more active math symbols, followed by the \texttt{\$.}\ token.  A {\bf
\$p statement}\index{\texttt{\$p} statement} consists of a label,
followed by \texttt{\$p}, followed by its typecode (an active constant),
followed by
zero or more active math symbols, followed by \texttt{\$=}, followed by
a sequence of labels, followed by the \texttt{\$.}\ token.  An {\bf
assertion}\index{assertion} is a \texttt{\$a} or \texttt{\$p} statement.

A \texttt{\$f}, \texttt{\$e}, or \texttt{\$d} statement is {\bf active}\index{active
statement} from the place it occurs until the end of the block it occurs in.
A \texttt{\$a} or \texttt{\$p} statement is {\bf active} from the place it occurs
through the end of the database.
There may not be two active \texttt{\$f} statements containing the same
variable.  Each variable in a \texttt{\$e}, \texttt{\$a}, or
\texttt{\$p} statement must exist in an active \texttt{\$f}
statement.\footnote{This requirement can greatly simplify the
unification algorithm (substitution calculation) required by proof
verification.}

%The label that begins each \texttt{\$f}, \texttt{\$e}, \texttt{\$a}, and
%\texttt{\$p} statement must be unique.
Each label token must be unique, and
no label token may match any math symbol
token.\label{namespace}\footnote{This
restriction was added on June 24, 2006.
It is not theoretically necessary but is imposed to make it easier to
write certain parsers.}

The set of {\bf mandatory variables}\index{mandatory variable} associated with
an assertion is the set of (zero or more) variables in the assertion and in any
active \texttt{\$e} statements.  The (possibly empty) set of {\bf mandatory
hypotheses}\index{mandatory hypothesis} is the set of all active \texttt{\$f}
statements containing mandatory variables, together with all active \texttt{\$e}
statements.
The set of {\bf mandatory {\bf \$d} statements}\index{mandatory
disjoint-variable restriction} associated with an assertion are those active
\texttt{\$d} statements whose variables are both among the assertion's
mandatory variables.

\subsection{Proof Verification}\label{spec4}

The sequence of labels between the \texttt{\$=} and \texttt{\$.}\ tokens
in a \texttt{\$p} statement is a {\bf proof}.\index{proof!Metamath} Each
label in a proof must be the label of an active statement other than the
\texttt{\$p} statement itself; thus a label must refer either to an
active hypothesis of the \texttt{\$p} statement or to an earlier
assertion.

An {\bf expression}\index{expression} is a sequence of math symbols. A {\bf
substitution map}\index{substitution map} associates a set of variables with a
set of expressions.  It is acceptable for a variable to be mapped to an
expression containing it.  A {\bf
substitution}\index{substitution!variable}\index{variable substitution} is the
simultaneous replacement of all variables in one or more expressions with the
expressions that the variables map to.

A proof is scanned in order of its label sequence.  If the label refers to an
active hypothesis, the expression in the hypothesis is pushed onto a
stack.\index{stack}\index{RPN stack}  If the label refers to an assertion, a
(unique) substitution must exist that, when made to the mandatory hypotheses
of the referenced assertion, causes them to match the topmost (i.e.\ most
recent) entries of the stack, in order of occurrence of the mandatory
hypotheses, with the topmost stack entry matching the last mandatory
hypothesis of the referenced assertion.  As many stack entries as there are
mandatory hypotheses are then popped from the stack.  The same substitution is
made to the referenced assertion, and the result is pushed onto the stack.
After the last label in the proof is processed, the stack must have a single
entry that matches the expression in the \texttt{\$p} statement containing the
proof.

%c%{\footnotesize\begin{quotation}\index{redeclaration of symbols}
%c%{{\em Comment.}\label{spec4comment} Whenever a math symbol token occurs in a
%c%{\texttt{\$c} or \texttt{\$v} statement, it is considered to designate a distinct new
%c%{symbol, even if the same token was previously declared (and is now inactive).
%c%{Thus a math token declared as a constant in two different blocks is considered
%c%{to designate two distinct constants (even though they have the same name).
%c%{The two constants will not match in a proof that references both blocks.
%c%{However, a proof referencing both blocks is acceptable as long as it doesn't
%c%{require that the constants match.  Similarly, a token declared to be a
%c%{constant for a referenced assertion will not match the same token declared to
%c%{be a variable for the \texttt{\$p} statement containing the proof.  In the case
%c%{of a token declared to be a variable for a referenced assertion, this is not
%c%{an issue since the variable can be substituted with whatever expression is
%c%{needed to achieve the required match.
%c%{\end{quotation}}
%c2%A proof may reference an assertion that contains or whose hypotheses contain a
%c2%constant that is not active for the \texttt{\$p} statement containing the proof.
%c2%However, the final result of the proof may not contain that constant. A proof
%c2%may also reference an assertion that contains or whose hypotheses contain a
%c2%variable that is not active for the \texttt{\$p} statement containing the proof.
%c2%That variable, of course, will be substituted with whatever expression is
%c2%needed to achieve the required match.

A proof may contain a \texttt{?}\ in place of a label to indicate an unknown step
(Section~\ref{unknown}).  A proof verifier may ignore any proof containing
\texttt{?}\ but should warn the user that the proof is incomplete.

A {\bf compressed proof}\index{compressed proof}\index{proof!compressed} is an
alternate proof notation described in Appen\-dix~\ref{compressed}; also see
references to ``compressed proof'' in the Index.  Compressed proofs are a
Metamath language extension which a complete proof verifier should be able to
parse and verify.

\subsubsection{Verifying Disjoint Variable Restrictions}

Each substitution made in a proof must be checked to verify that any
disjoint variable restrictions are satisfied, as follows.

If two variables replaced by a substitution exist in a mandatory \texttt{\$d}
statement\index{\texttt{\$d} statement} of the assertion referenced, the two
expressions resulting from the substitution must satisfy the following
conditions.  First, the two expressions must have no variables in common.
Second, each possible pair of variables, one from each expression, must exist
in an active \texttt{\$d} statement of the \texttt{\$p} statement containing the
proof.

\vskip 1ex

This ends the specification of the Metamath language;
see Appendix \ref{BNF} for its syntax in
Extended Backus--Naur Form (EBNF)\index{Extended Backus--Naur Form}\index{EBNF}.

\section{The Basic Keywords}\label{tut1}

Our expository material begins here.

Like most computer languages, Metamath\index{Metamath} takes its input from
one or more {\bf source files}\index{source file} which contain characters
expressed in the standard {\sc ascii} (American Standard Code for Information
Interchange)\index{ascii@{\sc ascii}} code for computers.  A source file
consists of a series of {\bf tokens}\index{token}, which are strings of
non-whitespace
printable characters (from the set of 94 shown on p.~\pageref{spec1chars})
separated by {\bf white space}\index{white space} (spaces, tabs, carriage
returns, line feeds, and form feeds). Any string consisting only of these
characters is treated the same as a single space.  The non-whitespace printable
characters\index{printable character} that Metamath recognizes are the 94
characters on standard {\sc ascii} keyboards.

Metamath has the ability to join several files together to form its
input (Section~\ref{include}).  We call the aggregate contents of all
the files after they have been joined together a {\bf
database}\index{database} to distinguish it from an individual source
file.  The tokens in a database consist of {\bf
keywords}\index{keyword}, which are built into the language, together
with two kinds of user-defined tokens called {\bf labels}\index{label}
and {\bf math symbols}\index{math symbol}.  (Often we will simply say
{\bf symbol}\index{symbol} instead of math symbol for brevity).  The set
of {\bf basic keywords}\index{basic keyword} is
\texttt{\$c}\index{\texttt{\$c} statement},
\texttt{\$v}\index{\texttt{\$v} statement},
\texttt{\$e}\index{\texttt{\$e} statement},
\texttt{\$f}\index{\texttt{\$f} statement},
\texttt{\$d}\index{\texttt{\$d} statement},
\texttt{\$a}\index{\texttt{\$a} statement},
\texttt{\$p}\index{\texttt{\$p} statement},
\texttt{\$=}\index{\texttt{\$=} keyword},
\texttt{\$.}\index{\texttt{\$.}\ keyword},
\texttt{\$\char`\{}\index{\texttt{\$\char`\{} and \texttt{\$\char`\}}
keywords}, and \texttt{\$\char`\}}.  This is the complete set of
syntactical elements of what we call the {\bf basic
language}\index{basic language} of Metamath, and with them you can
express all of the mathematics that were intended by the design of
Metamath.  You should make it a point to become very familiar with them.
Table~\ref{basickeywords} lists the basic keywords along with a brief
description of their functions.  For now, this description will give you
only a vague notion of what the keywords are for; later we will describe
the keywords in detail.


\begin{table}[htp] \caption{Summary of the basic Metamath
keywords} \label{basickeywords}
\begin{center}
\begin{tabular}{|p{4pc}|l|}
\hline
\em \centering Keyword&\em Description\\
\hline
\hline
\centering
   \texttt{\$c}&Constant symbol declaration\\
\hline
\centering
   \texttt{\$v}&Variable symbol declaration\\
\hline
\centering
   \texttt{\$d}&Disjoint variable restriction\\
\hline
\centering
   \texttt{\$f}&Variable-type (``floating'') hypothesis\\
\hline
\centering
   \texttt{\$e}&Logical (``essential'') hypothesis\\
\hline
\centering
   \texttt{\$a}&Axiomatic assertion\\
\hline
\centering
   \texttt{\$p}&Provable assertion\\
\hline
\centering
   \texttt{\$=}&Start of proof in \texttt{\$p} statement\\
\hline
\centering
   \texttt{\$.}&End of the above statement types\\
\hline
\centering
   \texttt{\$\char`\{}&Start of block\\
\hline
\centering
   \texttt{\$\char`\}}&End of block\\
\hline
\end{tabular}
\end{center}
\end{table}

%For LaTeX bug(?) where it puts tables on blank page instead of btwn text
%May have to adjust if text changes
%\newpage

There are some additional keywords, called {\bf auxiliary
keywords}\index{auxiliary keyword} that help make Metamath\index{Metamath}
more practical. These are part of the {\bf extended language}\index{extended
language}. They provide you with a means to put comments into a Metamath
source file\index{source file} and reference other source files.  We will
introduce these in later sections. Table~\ref{otherkeywords} summarizes them
so that you can recognize them now if you want to peruse some source
files while learning the basic keywords.


\begin{table}[htp] \caption{Auxiliary Metamath
keywords} \label{otherkeywords}
\begin{center}
\begin{tabular}{|p{4pc}|l|}
\hline
\em \centering Keyword&\em Description\\
\hline
\hline
\centering
   \texttt{\$(}&Start of comment\\
\hline
\centering
   \texttt{\$)}&End of comment\\
\hline
\centering
   \texttt{\$[}&Start of included source file name\\
\hline
\centering
   \texttt{\$]}&End of included source file name\\
\hline
\end{tabular}
\end{center}
\end{table}
\index{\texttt{\$(} and \texttt{\$)} auxiliary keywords}
\index{\texttt{\$[} and \texttt{\$]} auxiliary keywords}


Unlike those in some computer languages, the keywords\index{keyword} are short
two-character sequences rather than English-like words.  While this may make
them slightly more difficult to remember at first, their brevity allows
them to blend in with the mathematics being described, not
distract from it, like punctuation marks.


\subsection{User-Defined Tokens}\label{dollardollar}\index{token}

As you may have noticed, all keywords\index{keyword} begin with the \texttt{\$}
character.  This mundane monetary symbol is not ordinarily used in higher
mathematics (outside of grant proposals), so we have appropriated it to
distinguish the Metamath\index{Metamath} keywords from ordinary mathematical
symbols. The \texttt{\$} character is thus considered special and may not be
used as a character in a user-defined token.  All tokens and keywords are
case-sensitive; for example, \texttt{n} is considered to be a different character
from \texttt{N}.  Case-sensitivity makes the available {\sc ascii} character set
as rich as possible.

\subsubsection{Math Symbol Tokens}\index{token}

Math symbols\index{math symbol} are tokens used to represent the symbols
that appear in ordinary mathematical formulas.  They may consist of any
combination of the 93 non-whitespace printable {\sc ascii} characters other than
\texttt{\$}~. Some examples are \texttt{x}, \texttt{+}, \texttt{(},
\texttt{|-}, \verb$!%@?&$, and \texttt{bounded}.  For readability, it is
best to try to make these look as similar to actual mathematical symbols
as possible, within the constraints of the {\sc ascii} character set, in
order to make the resulting mathematical expressions more readable.

In the Metamath\index{Metamath} language, you express ordinary
mathematical formulas and statements as sequences of math symbols such
as \texttt{2 + 2 = 4} (five symbols, all constants).\footnote{To
eliminate ambiguity with other expressions, this is expressed in the set
theory database \texttt{set.mm} as \texttt{|- ( 2 + 2
 ) = 4 }, whose \LaTeX\ equivalent is $\vdash
(2+2)=4$.  The \,$\vdash$ means ``is a theorem'' and the
parentheses allow explicit associative grouping.}\index{turnstile
({$\,\vdash$})} They may even be English
sentences, as in \texttt{E is closed and bounded} (five symbols)---here
\texttt{E} would be a variable and the other four symbols constants.  In
principle, a Metamath database could be constructed to work with almost
any unambiguous English-language mathematical statement, but as a
practical matter the definitions needed to provide for all possible
syntax variations would be cumbersome and distracting and possibly have
subtle pitfalls accidentally built in.  We generally recommend that you
express mathematical statements with compact standard mathematical
symbols whenever possible and put their English-language descriptions in
comments.  Axioms\index{axiom} and definitions\index{definition}
(\texttt{\$a}\index{\texttt{\$a} statement} statements) are the only
places where Metamath will not detect an error, and doing this will help
reduce the number of definitions needed.

You are free to use any tokens\index{token} you like for math
symbols\index{math symbol}.  Appendix~\ref{ASCII} recommends token names to
use for symbols in set theory, and we suggest you adopt these in order to be
able to include the \texttt{set.mm} set theory database in your database.  For
printouts, you can convert the tokens in a database
to standard mathematical symbols with the \LaTeX\ typesetting program.  The
Metamath command \texttt{open tex} {\em filename}\index{\texttt{open tex} command}
produces output that can be read by \LaTeX.\index{latex@{\LaTeX}}
The correspondence
between tokens and the actual symbols is made by \texttt{latexdef}
statements inside a special database comment tagged
with \texttt{\$t}.\index{\texttt{\$t} comment}\index{typesetting comment}
  You can edit
this comment to change the definitions or add new ones.
Appendix~\ref{ASCII} describes how to do this in more detail.

% White space\index{white space} is normally used to separate math
% symbol\index{math symbol} tokens, but they may be juxtaposed without white
% space in \texttt{\$d}\index{\texttt{\$d} statement}, \texttt{\$e}\index{\texttt{\$e}
% statement}, \texttt{\$f}\index{\texttt{\$f} statement}, \texttt{\$a}\index{\texttt{\$a}
% statement}, and \texttt{\$p}\index{\texttt{\$p} statement} statements when no
% ambiguity will result.  Specifically, Metamath parses the math symbol sequence
% in one of these statements in the following manner:  when the math symbol
% sequence has been broken up into tokens\index{token} up to a given character,
% the next token is the longest string of characters that could constitute a
% math symbol that is active\index{active
% math symbol} at that point.  (See Section~\ref{scoping} for the
% definition of an active math symbol.)  For example, if \texttt{-}, \texttt{>}, and
% \texttt{->} are the only active math symbols, the juxtaposition \texttt{>-} will be
% interpreted as the two symbols \texttt{>} and \texttt{-}, whereas \texttt{->} will
% always be interpreted as that single symbol.\footnote{For better readability we
% recommend a white space between each token.  This also makes searching for a
% symbol easier to do with an editor.  Omission of optional white space is useful
% for reducing typing when assigning an expression to a temporary
% variable\index{temporary variable} with the \texttt{let variable} Metamath
% program command.}\index{\texttt{let variable} command}
%
% Keywords\index{keyword} may be placed next to math symbols without white
% space\index{white space} between them.\footnote{Again, we do not recommend
% this for readability.}
%
% The math symbols\index{math symbol} in \texttt{\$c}\index{\texttt{\$c} statement}
% and \texttt{\$v}\index{\texttt{\$v} statement} statements must always be separated
% by white space\index{white
% space}, for the obvious reason that these statements define the names
% of the symbols.
%
% Math symbols referred to in comments (see Section~\ref{comments}) must also be
% separated by white space.  This allows you to make comments about symbols that
% are not yet active\index{active
% math symbol}.  (The ``math mode'' feature of comments is also a quick and
% easy way to obtain word processing text with embedded mathematical symbols,
% independently of the main purpose of Metamath; the way to do this is described
% in Section~\ref{comments})

\subsubsection{Label Tokens}\index{token}\index{label}

Label tokens are used to identify Metamath\index{Metamath} statements for
later reference. Label tokens may contain only letters, digits, and the three
characters period, hyphen, and underscore:
\begin{verbatim}
. - _
\end{verbatim}

A label is {\bf declared}\index{label declaration} by placing it immediately
before the keyword of the statement it identifies.  For example, the label
\texttt{axiom.1} might be declared as follows:
\begin{verbatim}
axiom.1 $a |- x = x $.
\end{verbatim}

Each \texttt{\$e}\index{\texttt{\$e} statement},
\texttt{\$f}\index{\texttt{\$f} statement},
\texttt{\$a}\index{\texttt{\$a} statement}, and
\texttt{\$p}\index{\texttt{\$p} statement} statement in a database must
have a label declared for it.  No other statement types may have label
declarations.  Every label must be unique.

A label (and the statement it identifies) is {\bf referenced}\index{label
reference} by including the label between the \texttt{\$=}\index{\texttt{\$=}
keyword} and \texttt{\$.}\index{\texttt{\$.}\ keyword}\ keywords in a \texttt{\$p}
statement.  The sequence of labels\index{label sequence} between these two
keywords is called a {\bf proof}\index{proof}.  An example of a statement with
a proof that we will encounter later (Section~\ref{proof}) is
\begin{verbatim}
wnew $p wff ( s -> ( r -> p ) )
     $= ws wr wp w2 w2 $.
\end{verbatim}

You don't have to know what this means just yet, but you should know that the
label \texttt{wnew} is declared by this \texttt{\$p} statement and that the labels
\texttt{ws}, \texttt{wr}, \texttt{wp}, and \texttt{w2} are assumed to have been declared
earlier in the database and are referenced here.

\subsection{Constants and Variables}
\index{constant}
\index{variable}

An {\bf expression}\index{expression} is any sequence of math
symbols, possibly empty.

The basic Metamath\index{Metamath} language\index{basic language} has two
kinds of math symbols\index{math symbol}:  {\bf constants}\index{constant} and
{\bf variables}\index{variable}.  In a Metamath proof, a constant may not be
substituted with any expression.  A variable can be
substituted\index{substitution!variable}\index{variable substitution} with any
expression.  This sequence may include other variables and may even include
the variable being substituted.  This substitution takes place when proofs are
verified, and it will be described in Section~\ref{proof}.  The \texttt{\$f}
statement (described later in Section~\ref{dollaref}) is used to specify the
{\bf type} of a variable (i.e.\ what kind of
variable it is)\index{variable type}\index{type} and
give it a meaning typically
associated with a ``metavariable''\index{metavariable}\footnote{A metavariable
is a variable that ranges over the syntactical elements of the object language
being discussed; for example, one metavariable might represent a variable of
the object language and another metavariable might represent a formula in the
object language.} in ordinary mathematics; for example, a variable may be
specified to be a wff or well-formed formula (in logic), a set (in set
theory), or a non-negative integer (in number theory).

%\subsection{The \texttt{\$c} and \texttt{\$v} Declaration Statements}
\subsection{The \texttt{\$c} and \texttt{\$v} Declaration Statements}
\index{\texttt{\$c} statement}
\index{constant declaration}
\index{\texttt{\$v} statement}
\index{variable declaration}

Constants are introduced or {\bf declared}\index{constant declaration}
with \texttt{\$c}\index{\texttt{\$c} statement} statements, and
variables are declared\index{variable declaration} with
\texttt{\$v}\index{\texttt{\$v} statement} statements.  A {\bf simple}
declaration\index{simple declaration} statement introduces a single
constant or variable.  Its syntax is one of the following:
\begin{center}
  \texttt{\$c} {\em math-symbol} \texttt{\$.}\\
  \texttt{\$v} {\em math-symbol} \texttt{\$.}
\end{center}
The notation {\em math-symbol} means any math symbol token\index{token}.

Some examples of simple declaration statements are:
\begin{center}
  \texttt{\$c + \$.}\\
  \texttt{\$c -> \$.}\\
  \texttt{\$c ( \$.}\\
  \texttt{\$v x \$.}\\
  \texttt{\$v y2 \$.}
\end{center}

The characters in a math symbol\index{math symbol} being declared are
irrelevant to Meta\-math; for example, we could declare a right parenthesis to
be a variable,
\begin{center}
  \texttt{\$v ) \$.}\\
\end{center}
although this would be unconventional.

A {\bf compound} declaration\index{compound declaration} statement is a
shorthand for declaring several symbols at once.  Its syntax is one of the
following:
\begin{center}
  \texttt{\$c} {\em math-symbol}\ \,$\cdots$\ {\em math-symbol} \texttt{\$.}\\
  \texttt{\$v} {\em math-symbol}\ \,$\cdots$\ {\em math-symbol} \texttt{\$.}
\end{center}\index{\texttt{\$c} statement}
Here, the ellipsis (\ldots) means any number of {\em math-symbol}\,s.

An example of a compound declaration statement is:
\begin{center}
  \texttt{\$v x y mu \$.}\\
\end{center}
This is equivalent to the three simple declaration statements
\begin{center}
  \texttt{\$v x \$.}\\
  \texttt{\$v y \$.}\\
  \texttt{\$v mu \$.}\\
\end{center}
\index{\texttt{\$v} statement}

There are certain rules on where in the database math symbols may be declared,
what sections of the database are aware of them (i.e.\ where they are
``active''), and when they may be declared more than once.  These will be
discussed in Section~\ref{scoping} and specifically on
p.~\pageref{redeclaration}.

\subsection{The \texttt{\$d} Statement}\label{dollard}
\index{\texttt{\$d} statement}

The \texttt{\$d} statement is called a {\bf disjoint-variable restriction}.  The
syntax of the {\bf simple} version of this statement is
\begin{center}
  \texttt{\$d} {\em variable variable} \texttt{\$.}
\end{center}
where each {\em variable} is a previously declared variable and the two {\em
variable}\,s are different.  (More specifically, each  {\em variable} must be
an {\bf active} variable\index{active math symbol}, which means there must be
a previous \texttt{\$v} statement whose {\bf scope}\index{scope} includes the
\texttt{\$d} statement.  These terms will be defined when we discuss scoping
statements in Section~\ref{scoping}.)

In ordinary mathematics, formulas may arise that are true if the variables in
them are distinct\index{distinct variables}, but become false when those
variables are made identical. For example, the formula in logic $\exists x\,x
\neq y$, which means ``for a given $y$, there exists an $x$ that is not equal
to $y$,'' is true in most mathematical theories (namely all non-trivial
theories\index{non-trivial theory}, i.e.\ those that describe more than one
individual, such as arithmetic).  However, if we substitute $y$ with $x$, we
obtain $\exists x\,x \neq x$, which is always false, as it means ``there
exists something that is not equal to itself.''\footnote{If you are a
logician, you will recognize this as the improper substitution\index{proper
substitution}\index{substitution!proper} of a free variable\index{free
variable} with a bound variable\index{bound variable}.  Metamath makes no
inherent distinction between free and bound variables; instead, you let
Metamath know what substitutions are permissible by using \texttt{\$d} statements
in the right way in your axiom system.}\index{free vs.\ bound variable}  The
\texttt{\$d} statement allows you to specify a restriction that forbids the
substitution of one variable with another.  In
this case, we would use the statement
\begin{center}
  \texttt{\$d x y \$.}
\end{center}\index{\texttt{\$d} statement}
to specify this restriction.

The order in which the variables appear in a \texttt{\$d} statement is not
important.  We could also use
\begin{center}
  \texttt{\$d y x \$.}
\end{center}

The \texttt{\$d} statement is actually more general than this, as the
``disjoint''\index{disjoint variables} in its name suggests.  The full meaning
is that if any substitution is made to its two variables (during the
course of a proof that references a \texttt{\$a} or \texttt{\$p} statement
associated with the \texttt{\$d}), the two expressions that result from the
substitution must have no variables in common.  In addition, each possible
pair of variables, one from each expression, must be in a \texttt{\$d} statement
associated with the statement being proved.  (This requirement forces the
statement being proved to ``inherit'' the original disjoint variable
restriction.)

For example, suppose \texttt{u} is a variable.  If the restriction
\begin{center}
  \texttt{\$d A B \$.}
\end{center}
has been specified for a theorem referenced in a
proof, we may not substitute \texttt{A} with \mbox{\tt a + u} and
\texttt{B} with \mbox{\tt b + u} because these two symbol sequences have the
variable \texttt{u} in common.  Furthermore, if \texttt{a} and \texttt{b} are
variables, we may not substitute \texttt{A} with \texttt{a} and \texttt{B} with \texttt{b}
unless we have also specified \texttt{\$d a b} for the theorem being proved; in
other words, the \texttt{\$d} property associated with a pair of variables must
be effectively preserved after substitution.

The \texttt{\$d}\index{\texttt{\$d} statement} statement does {\em not} mean ``the
two variables may not be substituted with the same thing,'' as you might think
at first.  For example, substituting each of \texttt{A} and \texttt{B} in the above
example with identical symbol sequences consisting only of constants does not
cause a disjoint variable conflict, because two symbol sequences have no
variables in common (since they have no variables, period).  Similarly, a
conflict will not occur by substituting the two variables in a \texttt{\$d}
statement with the empty symbol sequence\index{empty substitution}.

The \texttt{\$d} statement does not have a direct counterpart in
ordinary mathematics, partly because the variables\index{variable} of
Metamath are not really the same as the variables\index{variable!in
ordinary mathematics} of ordinary mathematics but rather are
metavariables\index{metavariable} ranging over them (as well as over
other kinds of symbols and groups of symbols).  Depending on the
situation, we may informally interpret the \texttt{\$d} statement in
different ways.  Suppose, for example, that \texttt{x} and \texttt{y}
are variables ranging over numbers (more precisely, that \texttt{x} and
\texttt{y} are metavariables ranging over variables that range over
numbers), and that \texttt{ph} ($\varphi$) and \texttt{ps} ($\psi$) are
variables (more precisely, metavariables) ranging over formulas.  We can
make the following interpretations that correspond to the informal
language of ordinary mathematics:
\begin{quote}
\begin{tabbing}
\texttt{\$d x y \$.} means ``assume $x$ and $y$ are
distinct variables.''\\
\texttt{\$d x ph \$.} means ``assume $x$ does not
occur in $\varphi$.''\\
\texttt{\$d ph ps \$.} \=means ``assume $\varphi$ and
$\psi$ have no variables\\ \>in common.''
\end{tabbing}
\end{quote}\index{\texttt{\$d} statement}

\subsubsection{Compound \texttt{\$d} Statements}

The {\bf compound} version of the \texttt{\$d} statement is a shorthand for
specifying several variables whose substitutions must be pairwise disjoint.
Its syntax is:
\begin{center}
  \texttt{\$d} {\em variable}\ \,$\cdots$\ {\em variable} \texttt{\$.}
\end{center}\index{\texttt{\$d} statement}
Here, {\em variable} represents the token of a previously declared
variable (specifically, an active variable) and all {\em variable}\,s are
different.  The compound \texttt{\$d}
statement is internally broken up by Metamath into one simple \texttt{\$d}
statement for each possible pair of variables in the original \texttt{\$d}
statement.  For example,
\begin{center}
  \texttt{\$d w x y z \$.}
\end{center}
is equivalent to
\begin{center}
  \texttt{\$d w x \$.}\\
  \texttt{\$d w y \$.}\\
  \texttt{\$d w z \$.}\\
  \texttt{\$d x y \$.}\\
  \texttt{\$d x z \$.}\\
  \texttt{\$d y z \$.}
\end{center}

Two or more simple \texttt{\$d} statements specifying the same variable pair are
internally combined into a single \texttt{\$d} statement.  Thus the set of three
statements
\begin{center}
  \texttt{\$d x y \$.}
  \texttt{\$d x y \$.}
  \texttt{\$d y x \$.}
\end{center}
is equivalent to
\begin{center}
  \texttt{\$d x y \$.}
\end{center}

Similarly, compound \texttt{\$d} statements, after being internally broken up,
internally have their common variable pairs combined.  For example the
set of statements
\begin{center}
  \texttt{\$d x y A \$.}
  \texttt{\$d x y B \$.}
\end{center}
is equivalent to
\begin{center}
  \texttt{\$d x y \$.}
  \texttt{\$d x A \$.}
  \texttt{\$d y A \$.}
  \texttt{\$d x y \$.}
  \texttt{\$d x B \$.}
  \texttt{\$d y B \$.}
\end{center}
which is equivalent to
\begin{center}
  \texttt{\$d x y \$.}
  \texttt{\$d x A \$.}
  \texttt{\$d y A \$.}
  \texttt{\$d x B \$.}
  \texttt{\$d y B \$.}
\end{center}

Metamath\index{Metamath} automatically verifies that all \texttt{\$d}
restrictions are met whenever it verifies proofs.  \texttt{\$d} statements are
never referenced directly in proofs (this is why they do not have
labels\index{label}), but Metamath is always aware of which ones must be
satisfied (i.e.\ are active) and will notify you with an error message if any
violation occurs.

To illustrate how Metamath detects a missing \texttt{\$d}
statement, we will look at the following example from the
\texttt{set.mm} database.

\begin{verbatim}
$d x z $.  $d y z $.
$( Theorem to add distinct quantifier to atomic formula. $)
ax17eq $p |- ( x = y -> A. z x = y ) $=...
\end{verbatim}

This statement has the obvious requirement that $z$ must be
distinct\index{distinct variables} from $x$ in theorem \texttt{ax17eq} that
states $x=y \rightarrow \forall z \, x=y$ (well, obvious if you're a logician,
for otherwise we could conclude  $x=y \rightarrow \forall x \, x=y$, which is
false when the free variables $x$ and $y$ are equal).

Let's look at what happens if we edit the database to comment out this
requirement.

\begin{verbatim}
$( $d x z $. $) $d y z $.
$( Theorem to add distinct quantifier to atomic formula. $)
ax17eq $p |- ( x = y -> A. z x = y ) $=...
\end{verbatim}

When it tries to verify the proof, Metamath will tell you that \texttt{x} and
\texttt{z} must be disjoint, because one of its steps references an axiom or
theorem that has this requirement.

\begin{verbatim}
MM> verify proof ax17eq
ax17eq ?Error at statement 1918, label "ax17eq", type "$p":
      vz wal wi vx vy vz ax-13 vx vy weq vz vx ax-c16 vx vy
                                               ^^^^^
There is a disjoint variable ($d) violation at proof step 29.
Assertion "ax-c16" requires that variables "x" and "y" be
disjoint.  But "x" was substituted with "z" and "y" was
substituted with "x".  The assertion being proved, "ax17eq",
does not require that variables "z" and "x" be disjoint.
\end{verbatim}

We can see the substitutions into \texttt{ax-c16} with the following command.

\begin{verbatim}
MM> show proof ax17eq / detailed_step 29
Proof step 29:  pm2.61dd.2=ax-c16 $a |- ( A. z z = x -> ( x =
  y -> A. z x = y ) )
This step assigns source "ax-c16" ($a) to target "pm2.61dd.2"
($e).  The source assertion requires the hypotheses "wph"
($f, step 26), "vx" ($f, step 27), and "vy" ($f, step 28).
The parent assertion of the target hypothesis is "pm2.61dd"
($p, step 36).
The source assertion before substitution was:
    ax-c16 $a |- ( A. x x = y -> ( ph -> A. x ph ) )
The following substitutions were made to the source
assertion:
    Variable  Substituted with
     x         z
     y         x
     ph        x = y
The target hypothesis before substitution was:
    pm2.61dd.2 $e |- ( ph -> ch )
The following substitutions were made to the target
hypothesis:
    Variable  Substituted with
     ph        A. z z = x
     ch        ( x = y -> A. z x = y )
\end{verbatim}

The disjoint variable restrictions of \texttt{ax-c16} can be seen from the
\texttt{show state\-ment} command.  The line that begins ``\texttt{Its mandatory
dis\-joint var\-i\-able pairs are:}\ldots'' lists any \texttt{\$d} variable
pairs in brackets.

\begin{verbatim}
MM> show statement ax-c16/full
Statement 3033 is located on line 9338 of the file "set.mm".
"Axiom of Distinct Variables. ..."
  ax-c16 $a |- ( A. x x = y -> ( ph -> A. x ph ) ) $.
Its mandatory hypotheses in RPN order are:
  wph $f wff ph $.
  vx $f setvar x $.
  vy $f setvar y $.
Its mandatory disjoint variable pairs are:  <x,y>
The statement and its hypotheses require the variables:  x y
      ph
The variables it contains are:  x y ph
\end{verbatim}

Since Metamath will always detect when \texttt{\$d}\index{\texttt{\$d} statement}
statements are needed for a proof, you don't have to worry too much about
forgetting to put one in; it can always be added if you see the error message
above.  If you put in unnecessary \texttt{\$d} statements, the worst that could
happen is that your theorem might not be as general as it could be, and this
may limit its use later on.

On the other hand, when you introduce axioms (\texttt{\$a}\index{\texttt{\$a}
statement} statements), you must be very careful to properly specify the
necessary associated \texttt{\$d} statements since Metamath has no way of knowing
whether your axioms are correct.  For example, Metamath would have no idea
that \texttt{ax-c16}, which we are telling it is an axiom of logic, would lead to
contradictions if we omitted its associated \texttt{\$d} statement.

% This was previously a comment in footnote-sized type, but it can be
% hard to read this much text in a small size.
% As a result, it's been changed to normally-sized text.
\label{nodd}
You may wonder if it is possible to develop standard
mathematics in the Metamath language without the \texttt{\$d}\index{\texttt{\$d}
statement} statement, since it seems like a nuisance that complicates proof
verification. The \texttt{\$d} statement is not needed in certain subsets of
mathematics such as propositional calculus.  However, dummy
variables\index{dummy variable!eliminating} and their associated \texttt{\$d}
statements are impossible to avoid in proofs in standard first-order logic as
well as in the variant used in \texttt{set.mm}.  In fact, there is no upper bound to
the number of dummy variables that might be needed in a proof of a theorem of
first-order logic containing 3 or more variables, as shown by H.\
Andr\'{e}ka\index{Andr{\'{e}}ka, H.} \cite{Nemeti}.  A first-order system that
avoids them entirely is given in \cite{Megill}\index{Megill, Norman}; the
trick there is simply to embed harmlessly the necessary dummy variables into a
theorem being proved so that they aren't ``dummy'' anymore, then interpret the
resulting longer theorem so as to ignore the embedded dummy variables.  If
this interests you, the system in \texttt{set.mm} obtained from \texttt{ax-1}
through \texttt{ax-c14} in \texttt{set.mm}, and deleting \texttt{ax-c16} and \texttt{ax-5},
requires no \texttt{\$d} statements but is logically complete in the sense
described in \cite{Megill}.  This means it can prove any theorem of
first-order logic as long as we add to the theorem an antecedent that embeds
dummy and any other variables that must be distinct.  In a similar fashion,
axioms for set theory can be devised that
do not require distinct variable
provisos\index{Set theory without distinct variable provisos},
as explained at
\url{http://us.metamath.org/mpeuni/mmzfcnd.html}.
Together, these in principle allow all of
mathematics to be developed under Metamath without a \texttt{\$d} statement,
although the length of the resulting theorems will grow as more and
more dummy variables become required in their proofs.

\subsection{The \texttt{\$f}
and \texttt{\$e} Statements}\label{dollaref}
\index{\texttt{\$e} statement}
\index{\texttt{\$f} statement}
\index{floating hypothesis}
\index{essential hypothesis}
\index{variable-type hypothesis}
\index{logical hypothesis}
\index{hypothesis}

Metamath has two kinds of hypo\-theses, the \texttt{\$f}\index{\texttt{\$f}
statement} or {\bf variable-type} hypothesis and the \texttt{\$e} or {\bf logical}
hypo\-the\-sis.\index{\texttt{\$d} statement}\footnote{Strictly speaking, the
\texttt{\$d} statement is also a hypothesis, but it is never directly referenced
in a proof, so we call it a restriction rather than a hypothesis to lessen
confusion.  The checking for violations of \texttt{\$d} restrictions is automatic
and built into Metamath's proof-checking algorithm.} The letters \texttt{f} and
\texttt{e} stand for ``floating''\index{floating hypothesis} (roughly meaning
used only if relevant) and ``essential''\index{essential hypothesis} (meaning
always used) respectively, for reasons that will become apparent
when we discuss frames in
Section~\ref{frames} and scoping in Section~\ref{scoping}. The syntax of these
are as follows:
\begin{center}
  {\em label} \texttt{\$f} {\em typecode} {\em variable} \texttt{\$.}\\
  {\em label} \texttt{\$e} {\em typecode}
      {\em math-symbol}\ \,$\cdots$\ {\em math-symbol} \texttt{\$.}\\
\end{center}
\index{\texttt{\$e} statement}
\index{\texttt{\$f} statement}
A hypothesis must have a {\em label}\index{label}.  The expression in a
\texttt{\$e} hypothesis consists of a typecode (an active constant math symbol)
followed by a sequence
of zero or more math symbols. Each math symbol (including {\em constant}
and {\em variable}) must be a previously declared constant or variable.  (In
addition, each math symbol must be active, which will be covered when we
discuss scoping statements in Section~\ref{scoping}.)  You use a \texttt{\$f}
hypothesis to specify the
nature or {\bf type}\index{variable type}\index{type} of a variable (such as ``let $x$ be an
integer'') and use a \texttt{\$e} hypothesis to express a logical truth (such as
``assume $x$ is prime'') that must be established in order for an assertion
requiring it to also be true.

A variable must have its type specified in a \texttt{\$f} statement before
it may be used in a \texttt{\$e}, \texttt{\$a}, or \texttt{\$p}
statement.  There may be only one (active) \texttt{\$f} statement for a
given variable.  (``Active'' is defined in Section~\ref{scoping}.)

In ordinary mathematics, theorems\index{theorem} are often expressed in the
form ``Assume $P$; then $Q$,'' where $Q$ is a statement that you can derive
if you start with statement $P$.\index{free variable}\footnote{A stronger
version of a theorem like this would be the {\em single} formula $P\rightarrow
Q$ ($P$ implies $Q$) from which the weaker version above follows by the rule
of modus ponens in logic.  We are not discussing this stronger form here.  In
the weaker form, we are saying only that if we can {\em prove} $P$, then we can
{\em prove} $Q$.  In a logician's language, if $x$ is the only free variable
in $P$ and $Q$, the stronger form is equivalent to $\forall x ( P \rightarrow
Q)$ (for all $x$, $P$ implies $Q$), whereas the weaker form is equivalent to
$\forall x P \rightarrow \forall x Q$. The stronger form implies the weaker,
but not vice-versa.  To be precise, the weaker form of the theorem is more
properly called an ``inference'' rather than a theorem.}\index{inference}
In the
Metamath\index{Metamath} language, you would express mathematical statement
$P$ as a hypothesis (a \texttt{\$e} Metamath language statement in this case) and
statement $Q$ as a provable assertion (a \texttt{\$p}\index{\texttt{\$p} statement}
statement).

Some examples of hypotheses you might encounter in logic and set theory are
\begin{center}
  \texttt{stmt1 \$f wff P \$.}\\
  \texttt{stmt2 \$f setvar x \$.}\\
  \texttt{stmt3 \$e |- ( P -> Q ) \$.}
\end{center}
\index{\texttt{\$e} statement}
\index{\texttt{\$f} statement}
Informally, these would be read, ``Let $P$ be a well-formed-formula,'' ``Let
$x$ be an (individual) variable,'' and ``Assume we have proved $P \rightarrow
Q$.''  The turnstile symbol \,$\vdash$\index{turnstile ({$\,\vdash$})} is
commonly used in logic texts to mean ``a proof exists for.''

To summarize:
\begin{itemize}
\item A \texttt{\$f} hypothesis tells Metamath the type or kind of its variable.
It is analogous to a variable declaration in a computer language that
tells the compiler that a variable is an integer or a floating-point
number.
\item The \texttt{\$e} hypothesis corresponds to what you would usually call a
``hypothesis'' in ordinary mathematics.
\end{itemize}

Before an assertion\index{assertion} (\texttt{\$a} or \texttt{\$p} statement) can be
referenced in a proof, all of its associated \texttt{\$f} and \texttt{\$e} hypotheses
(i.e.\ those \texttt{\$e} hypotheses that are active) must be satisfied (i.e.
established by the proof).  The meaning of ``associated'' (which we will call
{\bf mandatory} in Section~\ref{frames}) will become clear when we discuss
scoping later.

Note that after any \texttt{\$f}, \texttt{\$e},
\texttt{\$a}, or \texttt{\$p} token there is a required
\textit{typecode}\index{typecode}.
The typecode is a constant used to enforce types of expressions.
This will become clearer once we learn more about
assertions (\texttt{\$a} and \texttt{\$p} statements).
An example may also clarify their purpose.
In the
\texttt{set.mm}\index{set theory database (\texttt{set.mm})}%
\index{Metamath Proof Explorer}
database,
the following typecodes are used:

\begin{itemize}
\item \texttt{wff} :
  Well-formed formula (wff) symbol
  (read: ``the following symbol sequence is a wff'').
% The *textual* typecode for turnstile is "|-", but when read it's a little
% confusing, so I intentionally display the mathematical symbol here instead
% (I think it's clearer in this context).
\item \texttt{$\vdash$} :
  Turnstile (read: ``the following symbol sequence is provable'' or
  ``a proof exists for'').
\item \texttt{setvar} :
  Individual set variable type (read: ``the following is an
  individual set variable'').
  Note that this is \textit{not} the type of an arbitrary set expression,
  instead, it is used to ensure that there is only a single symbol used
  after quantifiers like for-all ($\forall$) and there-exists ($\exists$).
\item \texttt{class} :
  An expression that is a syntactically valid class expression.
  All valid set expressions are also valid class expression, so expressions
  of sets normally have the \texttt{class} typecode.
  Use the \texttt{class} typecode,
  \textit{not} the \texttt{setvar} typecode,
  for the type of set expressions unless you are specifically identifying
  a single set variable.
\end{itemize}

\subsection{Assertions (\texttt{\$a} and \texttt{\$p} Statements)}
\index{\texttt{\$a} statement}
\index{\texttt{\$p} statement}\index{assertion}\index{axiomatic assertion}
\index{provable assertion}

There are two types of assertions, \texttt{\$a}\index{\texttt{\$a} statement}
statements ({\bf axiomatic assertions}) and \texttt{\$p} statements ({\bf
provable assertions}).  Their syntax is as follows:
\begin{center}
  {\em label} \texttt{\$a} {\em typecode} {\em math-symbol} \ldots
         {\em math-symbol} \texttt{\$.}\\
  {\em label} \texttt{\$p} {\em typecode} {\em math-symbol} \ldots
        {\em math-symbol} \texttt{\$=} {\em proof} \texttt{\$.}
\end{center}
\index{\texttt{\$a} statement}
\index{\texttt{\$p} statement}
\index{\texttt{\$=} keyword}
An assertion always requires a {\em label}\index{label}. The expression in an
assertion consists of a typecode (an active constant)
followed by a sequence of zero
or more math symbols.  Each math symbol, including any {\em constant}, must be a
previously declared constant or variable.  (In addition, each math symbol
must be active, which will be covered when we discuss scoping statements in
Section~\ref{scoping}.)

A \texttt{\$a} statement is usually a definition of syntax (for example, if $P$
and $Q$ are wffs then so is $(P\to Q)$), an axiom\index{axiom} of ordinary
mathematics (for example, $x=x$), or a definition\index{definition} of
ordinary mathematics (for example, $x\ne y$ means $\lnot x=y$). A \texttt{\$p}
statement is a claim that a certain combination of math symbols follows from
previous assertions and is accompanied by a proof that demonstrates it.

Assertions can also be referenced in (later) proofs in order to derive new
assertions from them. The label of an assertion is used to refer to it in a
proof. Section~\ref{proof} will describe the proof in detail.

Assertions also provide the primary means for communicating the mathematical
results in the database to people.  Proofs (when conveniently displayed)
communicate to people how the results were arrived at.

\subsubsection{The \texttt{\$a} Statement}
\index{\texttt{\$a} statement}

Axiomatic assertions (\texttt{\$a} statements) represent the starting points from
which other assertions (\texttt{\$p}\index{\texttt{\$p} statement} statements) are
derived.  Their most obvious use is for specifying ordinary mathematical
axioms\index{axiom}, but they are also used for two other purposes.

First, Metamath\index{Metamath} needs to know the syntax of symbol
sequences that constitute valid mathematical statements.  A Metamath
proof must be broken down into much more detail than ordinary
mathematical proofs that you may be used to thinking of (even the
``complete'' proofs of formal logic\index{formal logic}).  This is one
of the things that makes Metamath a general-purpose language,
independent of any system of logic or even syntax.  If you want to use a
substitution instance of an assertion as a step in a proof, you must
first prove that the substitution is syntactically correct (or if you
prefer, you must ``construct'' it), showing for example that the
expression you are substituting for a wff metavariable is a valid wff.
The \texttt{\$a}\index{\texttt{\$a} statement} statement is used to
specify those combinations of symbols that are considered syntactically
valid, such as the legal forms of wffs.

Second, \texttt{\$a} statements are used to specify what are ordinarily thought of
as definitions, i.e.\ new combinations of symbols that abbreviate other
combinations of symbols.  Metamath makes no distinction\index{axiom vs.\
definition} between axioms\index{axiom} and definitions\index{definition}.
Indeed, it has been argued that such distinction should not be made even in
ordinary mathematics; see Section~\ref{definitions}, which discusses the
philosophy of definitions.  Section~\ref{hierarchy} discusses some
technical requirements for definitions.  In \texttt{set.mm} we adopt the
convention of prefixing axiom labels with \texttt{ax-} and definition labels with
\texttt{df-}\index{label}.

The results that can be derived with the Metamath language are only as good as
the \texttt{\$a}\index{\texttt{\$a} statement} statements used as their starting
point.  We cannot stress this too strongly.  For example, Metamath will
not prevent you from specifying $x\neq x$ as an axiom of logic.  It is
essential that you scrutinize all \texttt{\$a} statements with great care.
Because they are a source of potential pitfalls, it is best not to add new
ones (usually new definitions) casually; rather you should carefully evaluate
each one's necessity and advantages.

Once you have in place all of the basic axioms\index{axiom} and
rules\index{rule} of a mathematical theory, the only \texttt{\$a} statements that
you will be adding will be what are ordinarily called definitions.  In
principle, definitions should be in some sense eliminable from the language of
a theory according to some convention (usually involving logical equivalence
or equality).  The most common convention is that any formula that was
syntactically valid but not provable before the definition was introduced will
not become provable after the definition is introduced.  In an ideal world,
definitions should not be present at all if one is to have absolute confidence
in a mathematical result.  However, they are necessary to make
mathematics practical, for otherwise the resulting formulas would be
extremely long and incomprehensible.  Since the nature of definitions (in the
most general sense) does not permit them to automatically be verified as
``proper,''\index{proper definition}\index{definition!proper} the judgment of
the mathematician is required to ensure it.  (In \texttt{set.mm} effort was made
to make almost all definitions directly eliminable and thus minimize the need
for such judgment.)

If you are not a mathematician, it may be best not to add or change any
\texttt{\$a}\index{\texttt{\$a} statement} statements but instead use
the mathematical language already provided in standard databases.  This
way Metamath will not allow you to make a mistake (i.e.\ prove a false
result).


\subsection{Frames}\label{frames}

We now introduce the concept of a collection of related Metamath statements
called a frame.  Every assertion (\texttt{\$a} or \texttt{\$p} statement) in the database has
an associated frame.

A {\bf frame}\index{frame} is a sequence of \texttt{\$d}, \texttt{\$f},
and \texttt{\$e} statements (zero or more of each) followed by one
\texttt{\$a} or \texttt{\$p} statement, subject to certain conditions we
will describe.  For simplicity we will assume that all math symbol
tokens used are declared at the beginning of the database with
\texttt{\$c} and \texttt{\$v} statements (which are not properly part of
a frame).  Also for simplicity we will assume there are only simple
\texttt{\$d} statements (those with only two variables) and imagine any
compound \texttt{\$d} statements (those with more than two variables) as
broken up into simple ones.

A frame groups together those hypotheses (and \texttt{\$d} statements) relevant
to an assertion (\texttt{\$a} or \texttt{\$p} statement).  The statements in a frame
may or may not be physically adjacent in a database; we will cover
this in our discussion of scoping statements
in Section~\ref{scoping}.

A frame has the following properties:
\begin{enumerate}
 \item The set of variables contained in its \texttt{\$f} statements must
be identical to the set of variables contained in its \texttt{\$e},
\texttt{\$a}, and/or \texttt{\$p} statements.  In other words, each
variable in a \texttt{\$e}, \texttt{\$a}, or \texttt{\$p} statement must
have an associated ``variable type'' defined for it in a \texttt{\$f}
statement.
  \item No two \texttt{\$f} statements may contain the same variable.
  \item Any \texttt{\$f} statement
must occur before a \texttt{\$e} statement in which its variable occurs.
\end{enumerate}

The first property determines the set of variables occurring in a frame.
These are the {\bf mandatory
variables}\index{mandatory variable} of the frame.  The second property
tells us there must be only one type specified for a variable.
The last property is not a theoretical requirement but it
makes parsing of the database easier.

For our examples, we assume our database has the following declarations:

\begin{verbatim}
$v P Q R $.
$c -> ( ) |- wff $.
\end{verbatim}

The following sequence of statements, describing the modus ponens inference
rule, is an example of a frame:

\begin{verbatim}
wp  $f wff P $.
wq  $f wff Q $.
maj $e |- ( P -> Q ) $.
min $e |- P $.
mp  $a |- Q $.
\end{verbatim}

The following sequence of statements is not a frame because \texttt{R} does not
occur in the \texttt{\$e}'s or the \texttt{\$a}:

\begin{verbatim}
wp  $f wff P $.
wq  $f wff Q $.
wr  $f wff R $.
maj $e |- ( P -> Q ) $.
min $e |- P $.
mp  $a |- Q $.
\end{verbatim}

The following sequence of statements is not a frame because \texttt{Q} does not
occur in a \texttt{\$f}:

\begin{verbatim}
wp  $f wff P $.
maj $e |- ( P -> Q ) $.
min $e |- P $.
mp  $a |- Q $.
\end{verbatim}

The following sequence of statements is not a frame because the \texttt{\$a} statement is
not the last one:

\begin{verbatim}
wp  $f wff P $.
wq  $f wff Q $.
maj $e |- ( P -> Q ) $.
mp  $a |- Q $.
min $e |- P $.
\end{verbatim}

Associated with a frame is a sequence of {\bf mandatory
hypotheses}\index{mandatory hypothesis}.  This is simply the set of all
\texttt{\$f} and \texttt{\$e} statements in the frame, in the order they
appear.  A frame can be referenced in a later proof using the label of
the \texttt{\$a} or \texttt{\$p} assertion statement, and the proof
makes an assignment to each mandatory hypothesis in the order in which
it appears.  This means the order of the hypotheses, once chosen, must
not be changed so as not to affect later proofs referencing the frame's
assertion statement.  (The Metamath proof verifier will, of course, flag
an error if a proof becomes incorrect by doing this.)  Since proofs make
use of ``Reverse Polish notation,'' described in Section~\ref{proof}, we
call this order the {\bf RPN order}\index{RPN order} of the hypotheses.

Note that \texttt{\$d} statements are not part of the set of mandatory
hypotheses, and their order doesn't matter (as long as they satisfy the
fourth property for a frame described above).  The \texttt{\$d}
statements specify restrictions on variables that must be satisfied (and
are checked by the proof verifier) when expressions are substituted for
them in a proof, and the \texttt{\$d} statements themselves are never
referenced directly in a proof.

A frame with a \texttt{\$p} (provable) statement requires a proof as part of the
\texttt{\$p} statement.  Sometimes in a proof we want to make use of temporary or
dummy variables\index{dummy variable} that do not occur in the \texttt{\$p}
statement or its mandatory hypotheses.  To accommodate this we define an {\bf
extended frame}\index{extended frame} as a frame together with zero or more
\texttt{\$d} and \texttt{\$f} statements that reference variables not among the
mandatory variables of the frame.  Any new variables referenced are called the
{\bf optional variables}\index{optional variable} of the extended frame. If a
\texttt{\$f} statement references an optional variable it is called an {\bf
optional hypothesis}\index{optional hypothesis}, and if one or both of the
variables in a \texttt{\$d} statement are optional variables it is called an {\bf
optional disjoint-variable restriction}\index{optional disjoint-variable
restriction}.  Properties 2 and 3 for a frame also apply to an extended
frame.

The concept of optional variables is not meaningful for frames with \texttt{\$a}
statements, since those statements have no proofs that might make use of them.
There is no restriction on including optional hypotheses in the extended frame
for a \texttt{\$a} statement, but they serve no purpose.

The following set of statements is an example of an extended frame, which
contains an optional variable \texttt{R} and an optional hypothesis \texttt{wr}.  In
this example, we suppose the rule of modus ponens is not an axiom but is
derived as a theorem from earlier statements (we omit its presumed proof).
Variable \texttt{R} may be used in its proof if desired (although this would
probably have no advantage in propositional calculus).  Note that the sequence
of mandatory hypotheses in RPN order is still \texttt{wp}, \texttt{wq}, \texttt{maj},
\texttt{min} (i.e.\ \texttt{wr} is omitted), and this sequence is still assumed
whenever the assertion \texttt{mp} is referenced in a subsequent proof.

\begin{verbatim}
wp  $f wff P $.
wq  $f wff Q $.
wr  $f wff R $.
maj $e |- ( P -> Q ) $.
min $e |- P $.
mp  $p |- Q $= ... $.
\end{verbatim}

Every frame is an extended frame, but not every extended frame is a frame, as
this example shows.  The underlying frame for an extended frame is
obtained by simply removing all statements containing optional variables.
Any proof referencing an assertion will ignore any extensions to its
frame, which means we may add or delete optional hypotheses at will without
affecting subsequent proofs.

The conceptually simplest way of organizing a Metamath database is as a
sequence of extended frames.  The scoping statements
\texttt{\$\char`\{}\index{\texttt{\$\char`\{} and \texttt{\$\char`\}}
keywords} and \texttt{\$\char`\}} can be used to delimit the start and
end of an extended frame, leading to the following possible structure for a
database.  \label{framelist}

\vskip 2ex
\setbox\startprefix=\hbox{\tt \ \ \ \ \ \ \ \ }
\setbox\contprefix=\hbox{}
\startm
\m{\mbox{(\texttt{\$v} {\em and} \texttt{\$c}\,{\em statements})}}
\endm
\startm
\m{\mbox{\texttt{\$\char`\{}}}
\endm
\startm
\m{\mbox{\texttt{\ \ } {\em extended frame}}}
\endm
\startm
\m{\mbox{\texttt{\$\char`\}}}}
\endm
\startm
\m{\mbox{\texttt{\$\char`\{}}}
\endm
\startm
\m{\mbox{\texttt{\ \ } {\em extended frame}}}
\endm
\startm
\m{\mbox{\texttt{\$\char`\}}}}
\endm
\startm
\m{\mbox{\texttt{\ \ \ \ \ \ \ \ \ }}\vdots}
\endm
\vskip 2ex

In practice, this structure is inconvenient because we have to repeat
any \texttt{\$f}, \texttt{\$e}, and \texttt{\$d} statements over and
over again rather than stating them once for use by several assertions.
The scoping statements, which we will discuss next, allow this to be
done.  In principle, any Metamath database can be converted to the above
format, and the above format is the most convenient to use when studying
a Metamath database as a formal system%
%% Uncomment this when uncommenting section {formalspec} below
   (Appendix \ref{formalspec})%
.
In fact, Metamath internally converts the database to the above format.
The command \texttt{show statement} in the Metamath program will show
you the contents of the frame for any \texttt{\$a} or \texttt{\$p}
statement, as well as its extension in the case of a \texttt{\$p}
statement.

%c%(provided that all ``local'' variables and constants with limited scope have
%c%unique names),

During our discussion of scoping statements, it may be helpful to
think in terms of the equivalent sequence of frames that will result when
the database is parsed.  Scoping (other than the limited
use above to delimit frames) is not a theoretical requirement for
Metamath but makes it more convenient.


\subsection{Scoping Statements (\texttt{\$\{} and \texttt{\$\}})}\label{scoping}
\index{\texttt{\$\char`\{} and \texttt{\$\char`\}} keywords}\index{scoping statement}

%c%Some Metamath statements may be needed only temporarily to
%c%serve a specific purpose, and after we're done with them we would like to
%c%disregard or ignore them.  For example, when we're finished using a variable,
%c%we might want to
%c%we might want to free up the token\index{token} used to name it so that the
%c%token can be used for other purposes later on, such as a different kind of
%c%variable or even a constant.  In the terminology of computer programming, we
%c%might want to let some symbol declarations be ``local'' rather than ``global.''
%c%\index{local symbol}\index{global symbol}

The {\bf scoping} statements, \texttt{\$\char`\{} ({\bf start of block}) and \texttt{\$\char`\}}
({\bf end of block})\index{block}, provide a means for controlling the portion
of a database over which certain statement types are recognized.  The
syntax of a scoping statement is very simple; it just consists of the
statement's keyword:
\begin{center}
\texttt{\$\char`\{}\\
\texttt{\$\char`\}}
\end{center}
\index{\texttt{\$\char`\{} and \texttt{\$\char`\}} keywords}

For example, consider the following database where we have stripped out
all tokens except the scoping statement keywords.  For the purpose of the
discussion, we have added subscripts to the scoping statements; these subscripts
do not appear in the actual database.
\[
 \mbox{\tt \ \$\char`\{}_1
 \mbox{\tt \ \$\char`\{}_2
 \mbox{\tt \ \$\char`\}}_2
 \mbox{\tt \ \$\char`\{}_3
 \mbox{\tt \ \$\char`\{}_4
 \mbox{\tt \ \$\char`\}}_4
 \mbox{\tt \ \$\char`\}}_3
 \mbox{\tt \ \$\char`\}}_1
\]
Each \texttt{\$\char`\{} statement in this example is said to be {\bf
matched} with the \texttt{\$\char`\}} statement that has the same
subscript.  Each pair of matched scoping statements defines a region of
the database called a {\bf block}.\index{block} Blocks can be {\bf
nested}\index{nested block} inside other blocks; in the example, the
block defined by $\mbox{\tt \$\char`\{}_4$ and $\mbox{\tt \$\char`\}}_4$
is nested inside the block defined by $\mbox{\tt \$\char`\{}_3$ and
$\mbox{\tt \$\char`\}}_3$ as well as inside the block defined by
$\mbox{\tt \$\char`\{}_1$ and $\mbox{\tt \$\char`\}}_1$.  In general, a
block may be empty, it may contain only non-scoping
statements,\footnote{Those statements other than \texttt{\$\char`\{} and
\texttt{\$\char`\}}.}\index{non-scoping statement} or it may contain any
mixture of other blocks and non-scoping statements.  (This is called a
``recursive'' definition\index{recursive definition} of a block.)

Associated with each block is a number called its {\bf nesting
level}\index{nesting level} that indicates how deeply the block is nested.
The nesting levels of the blocks in our example are as follows:
\[
  \underbrace{
    \mbox{\tt \ }
    \underbrace{
     \mbox{\tt \$\char`\{\ }
     \underbrace{
       \mbox{\tt \$\char`\{\ }
       \mbox{\tt \$\char`\}}
     }_{2}
     \mbox{\tt \ }
     \underbrace{
       \mbox{\tt \$\char`\{\ }
       \underbrace{
         \mbox{\tt \$\char`\{\ }
         \mbox{\tt \$\char`\}}
       }_{3}
       \mbox{\tt \ \$\char`\}}
     }_{2}
     \mbox{\tt \ \$\char`\}}
   }_{1}
   \mbox{\tt \ }
 }_{0}
\]
\index{\texttt{\$\char`\{} and \texttt{\$\char`\}} keywords}
The entire database is considered to be one big block (the {\bf outermost}
block) with a nesting level of 0.  The outermost block is {\em not} bracketed
by scoping statements.\footnote{The language was designed this way so that
several source files can be joined together more easily.}\index{outermost
block}

All non-scoping Metamath statements become recognized or {\bf
active}\index{active statement} at the place where they appear.\footnote{To
keep things slightly simpler, we do not bother to define the concept of
``active'' for the scoping statements.}  Certain of these statement types
become inactive at the end of the block in which they appear; these statement
types are:
\begin{center}
  \texttt{\$c}, \texttt{\$v}, \texttt{\$d}, \texttt{\$e}, and \texttt{\$f}.
%  \texttt{\$v}, \texttt{\$f}, \texttt{\$e}, and \texttt{\$d}.
\end{center}
\index{\texttt{\$c} statement}
\index{\texttt{\$d} statement}
\index{\texttt{\$e} statement}
\index{\texttt{\$f} statement}
\index{\texttt{\$v} statement}
The other statement types remain active forever (i.e.\ through the end of the
database); they are:
\begin{center}
  \texttt{\$a} and \texttt{\$p}.
%  \texttt{\$c}, \texttt{\$a}, and \texttt{\$p}.
\end{center}
\index{\texttt{\$a} statement}
\index{\texttt{\$p} statement}
Any statement (of these 7 types) located in the outermost
block\index{outermost block} will remain active through the end of the
database and thus are effectively ``global'' statements.\index{global
statement}

All \texttt{\$c} statements must be placed in the outermost block.  Since they are
therefore always global, they could be considered as belonging to both of the
above categories.

The {\bf scope}\index{scope} of a statement is the set of statements that
recognize it as active.

%c%The concept of ``active'' is also defined for math symbols\index{math
%c%symbol}.  Math symbols (constants\index{constant} and
%c%variables\index{variable}) become {\bf active}\index{active
%c%math symbol} in the \texttt{\$c}\index{\texttt{\$c}
%c%statement} and \texttt{\$v}\index{\texttt{\$v} statement} statements that
%c%declare them.  They become inactive when their declaration statements become
%c%inactive.

The concept of ``active'' is also defined for math symbols\index{math
symbol}.  Math symbols (constants\index{constant} and
variables\index{variable}) become {\bf active}\index{active math symbol}
in the \texttt{\$c}\index{\texttt{\$c} statement} and
\texttt{\$v}\index{\texttt{\$v} statement} statements that declare them.
A variable becomes inactive when its declaration statement becomes
inactive.  Because all \texttt{\$c} statements must be in the outermost
block, a constant will never become inactive after it is declared.

\subsubsection{Redeclaration of Math Symbols}
\index{redeclaration of symbols}\label{redeclaration}

%c%A math symbol may not be declared a second time while it is active, but it may
%c%be declared again after it becomes inactive.

A variable may not be declared a second time while it is active, but it may be
declared again after it becomes inactive.  This provides a convenient way to
introduce ``local'' variables,\index{local variable} i.e.\ temporary variables
for use in the frame of an assertion or in a proof without keeping them around
forever.  A previously declared variable may not be redeclared as a constant.

A constant may not be redeclared.  And, as mentioned above, constants must be
declared in the outermost block.

The reason variables may have limited scope but not constants is that an
assertion (\texttt{\$a} or \texttt{\$p} statement) remains available for use in
proofs through the end of the database.  Variables in an assertion's frame may
be substituted with whatever is needed in a proof step that references the
assertion, whereas constants remain fixed and may not be substituted with
anything.  The particular token used for a variable in an assertion's frame is
irrelevant when the assertion is referenced in a proof, and it doesn't matter
if that token is not available outside of the referenced assertion's frame.
Constants, however, must be globally fixed.

There is no theoretical
benefit for the feature allowing variables to be active for limited scopes
rather than global. It is just a convenience that allows them, for example, to
be locally grouped together with their corresponding \texttt{\$f} variable-type
declarations.

%c%If you declare a math symbol more than once, internally Metamath considers it a
%c%new distinct symbol, even though it has the same name.  If you are unaware of
%c%this, you may find that what you think are correct proofs are incorrectly
%c%rejected as invalid, because Metamath may tell you that a constant you
%c%previously declared does not match a newly declared math symbol with the same
%c%name.  For details on this subtle point, see the Comment on
%c%p.~\pageref{spec4comment}.  This is done purposely to allow temporary
%c%constants to be introduced while developing a subtheory, then allow their math
%c%symbol tokens to be reused later on; in general they will not refer to the
%c%same thing.  In practice, you would not ordinarily reuse the names of
%c%constants because it would tend to be confusing to the reader.  The reuse of
%c%names of variables, on the other hand, is something that is often useful to do
%c%(for example it is done frequently in \texttt{set.mm}).  Since variables in an
%c%assertion referenced in a proof can be substituted as needed to achieve a
%c%symbol match, this is not an issue.

% (This section covers a somewhat advanced topic you may want to skip
% at first reading.)
%
% Under certain circumstances, math symbol\index{math symbol}
% tokens\index{token} may be redeclared (i.e.\ the token
% may appear in more than
% one \texttt{\$c}\index{\texttt{\$c} statement} or \texttt{\$v}\index{\texttt{\$v}
% statement} statement).  You might want to do this say, to make temporary use
% of a variable name without having to worry about its affect elsewhere,
% somewhat analogous to declaring a local variable in a standard computer
% language.  Understanding what goes on when math symbol tokens are redeclared
% is a little tricky to understand at first, since it requires that we
% distinguish the token itself from the math symbol that it names.  It will help
% if we first take a peek at the internal workings of the
% Metamath\index{Metamath} program.
%
% Metamath reserves a memory location for each occurrence of a
% token\index{token} in a declaration statement (\texttt{\$c}\index{\texttt{\$c}
% statement} or \texttt{\$v}\index{\texttt{\$v} statement}).  If a given token appears
% in more than one declaration statement, it will refer to more than one memory
% locations.  A math symbol\index{math symbol} may be thought of as being one of
% these memory locations rather than as the token itself.  Only one of the
% memory locations associated with a given token may be active at any one time.
% The math symbol (memory location) that gets looked up when the token appears
% in a non-declaration statement is the one that happens to be active at that
% time.
%
% We now look at the rules for the redeclaration\index{redeclaration of symbols}
% of math symbol tokens.
% \begin{itemize}
% \item A math symbol token may not be declared twice in the
% same block.\footnote{While there is no theoretical reason for disallowing
% this, it was decided in the design of Metamath that allowing it would offer no
% advantage and might cause confusion.}
% \item An inactive math symbol may always be
% redeclared.
% \item  An active math symbol may be redeclared in a different (i.e.\
% inner) block\index{block} from the one it became active in.
% \end{itemize}
%
% When a math symbol token is redeclared, it conceptually refers to a different
% math symbol, just as it would be if it were called a different name.  In
% addition, the original math symbol that it referred to, if it was active,
% temporarily becomes inactive.  At the end of the block in which the
% redeclaration occurred, the new math symbol\index{math symbol} becomes
% inactive and the original symbol becomes active again.  This concept is
% illustrated in the following example, where the symbol \texttt{e} is
% ordinarily a constant (say Euler's constant, 2.71828...) but
% temporarily we want to use it as a ``local'' variable, say as a coefficient
% in the equation $a x^4 + b x^3 + c x^2 + d x + e$:
% \[
%   \mbox{\tt \$\char`\{\ \$c e \$.}
%   \underbrace{
%     \ \ldots\ %
%     \mbox{\tt \$\char`\{}\ \ldots\ %
%   }_{\mbox{\rm region A}}
%   \mbox{\tt \$v e \$.}
%   \underbrace{
%     \mbox{\ \ \ \ldots\ \ \ }
%   }_{\mbox{\rm region B}}
%   \mbox{\tt \$\char`\}}
%   \underbrace{
%     \mbox{\ \ \ \ldots\ \ \ }
%   }_{\mbox{\rm region C}}
%   \mbox{\tt \$\char`\}}
% \]
% \index{\texttt{\$\char`\{} and \texttt{\$\char`\}} keywords}
% In region A, the token \texttt{e} refers to a constant.  It is redeclared as a
% variable in region B, and any reference to it in this region will refer to this
% variable.  In region C, the redeclaration becomes inactive, and the original
% declaration becomes active again.  In region C, the token \texttt{x} refers to the
% original constant.
%
% As a practical matter, overuse of math symbol\index{math symbol}
% redeclarations\index{redeclaration of symbols} can be confusing (even though
% it is well-defined) and is best avoided when possible.  Here are some good
% general guidelines you can follow.  Usually, you should declare all
% constants\index{constant} in the outermost block\index{outermost block},
% especially if they are general-purpose (such as the token \verb$A.$, meaning
% $\forall$ or ``for all'').  This will make them ``globally'' active (although
% as in the example above local redeclarations will temporarily make them
% inactive.)  Most or all variables\index{variable}, on the other hand, could be
% declared in inner blocks, so that the token for them can be used later for a
% different type of variable or a constant.  (The names of the variables you
% choose are not used when you refer to an assertion\index{assertion} in a
% proof, whereas constants must match exactly.  A locally declared constant will
% not match a globally declared constant in a proof, even if they use the same
% token, because Metamath internally considers them to be different math
% symbols.)  To avoid confusion, you should generally avoid redeclaring active
% variables.  If you must redeclare them, do so at the beginning of a block.
% The temporary declaration of constants in inner blocks might be occasionally
% appropriate when you make use of a temporary definition to prove lemmas
% leading to a main result that does not make direct use of the definition.
% This way, you will not clutter up your database with a large number of
% seldom-used global constant symbols.  You might want to note that while
% inactive constants may not appear directly in an assertion (a \texttt{\$a}\index{\texttt{\$a}
% statement} or \texttt{\$p}\index{\texttt{\$p} statement}
% statement), they may be indirectly used in the proof of a \texttt{\$p} statement
% so long as they do not appear in the final math symbol sequence constructed by
% the proof.  In the end, you will have to use your best judgment, taking into
% account standard mathematical usage of the symbols as well as consideration
% for the reader of your work.
%
% \subsubsection{Reuse of Labels}\index{reuse of labels}\index{label}
%
% The \texttt{\$e}\index{\texttt{\$e} statement}, \texttt{\$f}\index{\texttt{\$f}
% statement}, \texttt{\$a}\index{\texttt{\$a} statement}, and
% \texttt{\$p}\index{\texttt{\$p}
% statement} statement types require labels, which allow them to be
% referenced later inside proofs.  A label is considered {\bf
% active}\index{active label} when the statement it is associated with is
% active.  The token\index{token} for a label may be reused
% (redeclared)\index{redeclaration of labels} provided that it is not being used
% for a currently active label.  (Unlike the tokens for math symbols, active
% label tokens may not be redeclared in an inner scope.)  Note that the labels
% of \texttt{\$a} and \texttt{\$p} statements can never be reused after these
% statements appear, because these statements remain active through the end of
% the database.
%
% You might find the reuse of labels a convenient way to have standard names for
% temporary hypotheses, such as \texttt{h1}, \texttt{h2}, etc.  This way you don't have
% to invent unique names for each of them, and in some cases it may be less
% confusing to the reader (although in other cases it might be more confusing, if
% the hypothesis is located far away from the assertion that uses
% it).\footnote{The current implementation requires that all labels, even
% inactive ones, be unique.}

\subsubsection{Frames Revisited}\index{frames and scoping statements}

Now that we have covered scoping, we will look at how an arbitrary
Metamath database can be converted to the simple sequence of extended
frames described on p.~\pageref{framelist}.  This is also how Metamath
stores the database internally when it reads in the database
source.\label{frameconvert} The method is simple.  First, we collect all
constant and variable (\texttt{\$c} and \texttt{\$v}) declarations in
the database, ignoring duplicate declarations of the same variable in
different scopes.  We then put our collected \texttt{\$c} and
\texttt{\$v} declarations at the beginning of the database, so that
their scope is the entire database.  Next, for each assertion in the
database, we determine its frame and extended frame.  The extended frame
is simply the \texttt{\$f}, \texttt{\$e}, and \texttt{\$d} statements
that are active.  The frame is the extended frame with all optional
hypotheses removed.

An equivalent way of saying this is that the extended frame of an assertion
is the collection of all \texttt{\$f}, \texttt{\$e}, and \texttt{\$d} statements
whose scope includes the assertion.
The \texttt{\$f} and \texttt{\$e} statements
occur in the order they appear
(order is irrelevant for \texttt{\$d} statements).

%c%, renaming any
%c%redeclared variables as needed so that all of them have unique names.  (The
%c%exact renaming convention is unimportant.  You might imagine renaming
%c%different declarations of math symbol \texttt{a} as \texttt{a\$1}, \texttt{a\$2}, etc.\
%c%which would prevent any conflicts since \texttt{\$} is not a legal character in a
%c%math symbol token.)

\section{The Anatomy of a Proof} \label{proof}
\index{proof!Metamath, description of}

Each provable assertion (\texttt{\$p}\index{\texttt{\$p} statement} statement) in a
database must include a {\bf proof}\index{proof}.  The proof is located
between the \texttt{\$=}\index{\texttt{\$=} keyword} and \texttt{\$.}\ keywords in the
\texttt{\$p} statement.

In the basic Metamath language\index{basic language}, a proof is a
sequence of statement labels.  This label sequence\index{label sequence}
serves as a set of instructions that the Metamath program uses to
construct a series of math symbol sequences.  The construction must
ultimately result in the math symbol sequence contained between the
\texttt{\$p}\index{\texttt{\$p} statement} and
\texttt{\$=}\index{\texttt{\$=} keyword} keywords of the \texttt{\$p}
statement.  Otherwise, the Metamath program will consider the proof
incorrect, and it will notify you with an appropriate error message when
you ask it to verify the proof.\footnote{To make the loading faster, the
Metamath program does not automatically verify proofs when you
\texttt{read} in a database unless you use the \texttt{/verify}
qualifier.  After a database has been read in, you may use the
\texttt{verify proof *} command to verify proofs.}\index{\texttt{verify
proof} command} Each label in a proof is said to {\bf
reference}\index{label reference} its corresponding statement.

Associated with any assertion\index{assertion} (\texttt{\$p} or
\texttt{\$a}\index{\texttt{\$a} statement} statement) is a set of
hypotheses (\texttt{\$f}\index{\texttt{\$f} statement} or
\texttt{\$e}\index{\texttt{\$e} statement} statements) that are active
with respect to that assertion.  Some are mandatory and the others are
optional.  You should review these concepts if necessary.

Each label\index{label} in a proof must be either the label of a
previous assertion (\texttt{\$a}\index{\texttt{\$a} statement} or
\texttt{\$p}\index{\texttt{\$p} statement} statement) or the label of an
active hypothesis (\texttt{\$e} or \texttt{\$f}\index{\texttt{\$f}
statement} statement) of the \texttt{\$p} statement containing the
proof.  Hypothesis labels may reference both the
mandatory\index{mandatory hypothesis} and the optional hypotheses of the
\texttt{\$p} statement.

The label sequence in a proof specifies a construction in {\bf reverse Polish
notation}\index{reverse Polish notation (RPN)} (RPN).  You may be familiar
with RPN if you have used older
Hewlett--Packard or similar hand-held calculators.
In the calculator analogy, a hypothesis label\index{hypothesis label} is like
a number and an assertion label\index{assertion label} is like an operation
(more precisely, an $n$-ary operation when the
assertion has $n$ \texttt{\$e}-hypotheses).
On an RPN calculator, an operation takes one or more previous numbers in an
input sequence, performs a calculation on them, and replaces those numbers and
itself with the result of the calculation.  For example, the input sequence
$2,3,+$ on an RPN calculator results in $5$, and the input sequence
$2,3,5,{\times},+$ results in $2,15,+$ which results in $17$.

Understanding how RPN is processed involves the concept of a {\bf
stack}\index{stack}\index{RPN stack}, which can be thought of as a set of
temporary memory locations that hold intermediate results.  When Metamath
encounters a hypothesis label it places or {\bf pushes}\index{push} the math
symbol sequence of the hypothesis onto the stack.  When Metamath encounters an
assertion label, it associates the most recent stack entries with the {\em
mandatory} hypotheses\index{mandatory hypothesis} of the assertion, in the
order where the most recent stack entry is associated with the last mandatory
hypothesis of the assertion.  It then determines what
substitutions\index{substitution!variable}\index{variable substitution} have
to be made into the variables of the assertion's mandatory hypotheses to make
them identical to the associated stack entries.  It then makes those same
substitutions into the assertion itself.  Finally, Metamath removes or {\bf
pops}\index{pop} the matched hypotheses from the stack and pushes the
substituted assertion onto the stack.

For the purpose of matching the mandatory hypothesis to the most recent stack
entries, whether a hypothesis is a \texttt{\$e} or \texttt{\$f} statement is
irrelevant.  The only important thing is that a set of
substitutions\footnote{In the Metamath spec (Section~\ref{spec}), we use the
singular term ``substitution'' to refer to the set of substitutions we talk
about here.} exist that allow a match (and if they don't, the proof verifier
will let you know with an error message).  The Metamath language is specified
in such a way that if a set of substitutions exists, it will be unique.
Specifically, the requirement that each variable have a type specified for it
with a \texttt{\$f} statement ensures the uniqueness.

We will illustrate this with an example.
Consider the following Metamath source file:
\begin{verbatim}
$c ( ) -> wff $.
$v p q r s $.
wp $f wff p $.
wq $f wff q $.
wr $f wff r $.
ws $f wff s $.
w2 $a wff ( p -> q ) $.
wnew $p wff ( s -> ( r -> p ) ) $= ws wr wp w2 w2 $.
\end{verbatim}
This Metamath source example shows the definition and ``proof'' (i.e.,
construction) of a well-formed formula (wff)\index{well-formed formula (wff)}
in propositional calculus.  (You may wish to type this example into a file to
experiment with the Metamath program.)  The first two statements declare
(introduce the names of) four constants and four variables.  The next four
statements specify the variable types, namely that
each variable is assumed to be a wff.  Statement \texttt{w2} defines (postulates)
a way to produce a new wff, \texttt{( p -> q )}, from two given wffs \texttt{p} and
\texttt{q}. The mandatory hypotheses of \texttt{w2} are \texttt{wp} and \texttt{wq}.
Statement \texttt{wnew} claims that \texttt{( s -> ( r -> p ) )} is a wff given
three wffs \texttt{s}, \texttt{r}, and \texttt{p}.  More precisely, \texttt{wnew} claims
that the sequence of ten symbols \texttt{wff ( s -> ( r -> p ) )} is provable from
previous assertions and the hypotheses of \texttt{wnew}.  Metamath does not know
or care what a wff is, and as far as it is concerned
the typecode \texttt{wff} is just an
arbitrary constant symbol in a math symbol sequence.  The mandatory hypotheses
of \texttt{wnew} are \texttt{wp}, \texttt{wr}, and \texttt{ws}; \texttt{wq} is an optional
hypothesis.  In our particular proof, the optional hypothesis is not
referenced, but in general, any combination of active (i.e.\ optional and
mandatory) hypotheses could be referenced.  The proof of statement \texttt{wnew}
is the sequence of five labels starting with \texttt{ws} (step~1) and ending with
\texttt{w2} (step~5).

When Metamath verifies the proof, it scans the proof from left to right.  We
will examine what happens at each step of the proof.  The stack starts off
empty.  At step 1, Metamath looks up label \texttt{ws} and determines that it is a
hypothesis, so it pushes the symbol sequence of statement \texttt{ws} onto the
stack:

\begin{center}\begin{tabular}{|l|l|}\hline
{Stack location} & {Contents} \\ \hline \hline
1 & \texttt{wff s} \\ \hline
\end{tabular}\end{center}

Metamath sees that the labels \texttt{wr} and \texttt{wp} in steps~2 and 3 are also
hypotheses, so it pushes them onto the stack.  After step~3, the stack looks
like
this:

\begin{center}\begin{tabular}{|l|l|}\hline
{Stack location} & {Contents} \\ \hline \hline
3 & \texttt{wff p} \\ \hline
2 & \texttt{wff r} \\ \hline
1 & \texttt{wff s} \\ \hline
\end{tabular}\end{center}

At step 4, Metamath sees that label \texttt{w2} is an assertion, so it must do
some processing.  First, it associates the mandatory hypotheses of \texttt{w2},
which are \texttt{wp} and \texttt{wq}, with stack locations~2 and 3, {\em in that
order}. Metamath determines that the only possible way
to make hypothesis \texttt{wp} match (become identical to) stack location~2 and
\texttt{wq} match stack location 3 is to substitute variable \texttt{p} with \texttt{r}
and \texttt{q} with \texttt{p}.  Metamath makes these substitutions into \texttt{w2} and
obtains the symbol sequence \texttt{wff ( r -> p )}.  It removes the hypotheses
from stack locations~2 and 3, then places the result into stack location~2:

\begin{center}\begin{tabular}{|l|l|}\hline
{Stack location} & {Contents} \\ \hline \hline
2 & \texttt{wff ( r -> p )} \\ \hline
1 & \texttt{wff s} \\ \hline
\end{tabular}\end{center}

At step 5, Metamath sees that label \texttt{w2} is an assertion, so it must again
do some processing.  First, it matches the mandatory hypotheses of \texttt{w2},
which are \texttt{wp} and \texttt{wq}, to stack locations 1 and 2.
Metamath determines that the only possible way to make the
hypotheses match is to substitute variable \texttt{p} with \texttt{s} and \texttt{q} with
\texttt{( r -> p )}.  Metamath makes these substitutions into \texttt{w2} and obtains
the symbol
sequence \texttt{wff ( s -> ( r -> p ) )}.  It removes stack
locations 1 and 2, then places the result into stack location~1:

\begin{center}\begin{tabular}{|l|l|}\hline
{Stack location} & {Contents} \\ \hline \hline
1 & \texttt{wff ( s -> ( r -> p ) )} \\ \hline
\end{tabular}\end{center}

After Metamath finishes processing the proof, it checks to see that the
stack contains exactly one element and that this element is
the same as the math symbol sequence in the
\texttt{\$p}\index{\texttt{\$p} statement} statement.  This is the case for our
proof of \texttt{wnew},
so we have proved \texttt{wnew} successfully.  If the result
differs, Metamath will notify you with an error message.  An error message
will also result if the stack contains more than one entry at the end of the
proof, or if the stack did not contain enough entries at any point in the
proof to match all of the mandatory hypotheses\index{mandatory hypothesis} of
an assertion.  Finally, Metamath will notify you with an error message if no
substitution is possible that will make a referenced assertion's hypothesis
match the
stack entries.  You may want to experiment with the different kinds of errors
that Metamath will detect by making some small changes in the proof of our
example.

Metamath's proof notation was designed primarily to express proofs in a
relatively compact manner, not for readability by humans.  Metamath can display
proofs in a number of different ways with the \texttt{show proof}\index{\texttt{show
proof} command} command.  The
\texttt{/lemmon} qualifier displays it in a format that is easier to read when the
proofs are short, and you saw examples of its use in Chapter~\ref{using}.  For
longer proofs, it is useful to see the tree structure of the proof.  A tree
structure is displayed when the \texttt{/lemmon} qualifier is omitted.  You will
probably find this display more convenient as you get used to it. The tree
display of the proof in our example looks like
this:\label{treeproof}\index{tree-style proof}\index{proof!tree-style}
\begin{verbatim}
1     wp=ws    $f wff s
2        wp=wr    $f wff r
3        wq=wp    $f wff p
4     wq=w2    $a wff ( r -> p )
5  wnew=w2  $a wff ( s -> ( r -> p ) )
\end{verbatim}
The number to the left of each line is the step number.  Following it is a
{\bf hypothesis association}\index{hypothesis association}, consisting of two
labels\index{label} separated by \texttt{=}.  To the left of the \texttt{=} (except
in the last step) is the label of a hypothesis of an assertion referenced
later in the proof; here, steps 1 and 4 are the hypothesis associations for
the assertion \texttt{w2} that is referenced in step 5.  A hypothesis association
is indented one level more than the assertion that uses it, so it is easy to
find the corresponding assertion by moving directly down until the indentation
level decreases to one less than where you started from.  To the right of each
\texttt{=} is the proof step label for that proof step.  The statement keyword of
the proof step label is listed next, followed by the content of the top of the
stack (the most recent stack entry) as it exists after that proof step is
processed.  With a little practice, you should have no trouble reading proofs
displayed in this format.

Metamath proofs include the syntax construction of a formula.
In standard mathematics, this kind of
construction is not considered a proper part of the proof at all, and it
certainly becomes rather boring after a while.
Therefore,
by default the \texttt{show proof}\index{\texttt{show proof}
command} command does not show the syntax construction.
Historically \texttt{show proof} command
\textit{did} show the syntax construction, and you needed to add the
\texttt{/essential} option to hide, them, but today
\texttt{/essential} is the default and you need to use
\texttt{/all} to see the syntax constructions.

When verifying a proof, Metamath will check that no mandatory
\texttt{\$d}\index{\texttt{\$d} statement}\index{mandatory \texttt{\$d}
statement} statement of an assertion referenced in a proof is violated
when substitutions\index{substitution!variable}\index{variable
substitution} are made to the variables in the assertion.  For details
see Section~\ref{spec4} or \ref{dollard}.

\subsection{The Concept of Unification} \label{unify}

During the course of verifying a proof, when Metamath\index{Metamath}
encounters an assertion label\index{assertion label}, it associates the
mandatory hypotheses\index{mandatory hypothesis} of the assertion with the top
entries of the RPN stack\index{stack}\index{RPN stack}.  Metamath then
determines what substitutions\index{substitution!variable}\index{variable
substitution} it must make to the variables in the assertion's mandatory
hypotheses in order for these hypotheses to become identical to their
corresponding stack entries.  This process is called {\bf
unification}\index{unification}.  (We also informally use the term
``unification'' to refer to a set of substitutions that results from the
process, as in ``two unifications are possible.'')  After the substitutions
are made, the hypotheses are said to be {\bf unified}.

If no such substitutions are possible, Metamath will consider the proof
incorrect and notify you with an error message.
% (deleted 3/10/07, per suggestion of Mel O'Cat:)
% The syntax of the
% Metamath language ensures that if a set of substitutions exists, it
% will be unique.

The general algorithm for unification described in the literature is
somewhat complex.
However, in the case of Metamath it is intentionally trivial.
Mandatory hypotheses must be
pushed on the proof stack in the order in which they appear.
In addition, each variable must have its type specified
with a \texttt{\$f} hypothesis before it is used
and that each \texttt{\$f} hypothesis
have the restricted syntax of a typecode (a constant) followed by a variable.
The typecode in the \texttt{\$f} hypothesis must match the first symbol of
the corresponding RPN stack entry (which will also be a constant), so
the only possible match for the variable in the \texttt{\$f} hypothesis is
the sequence of symbols in the stack entry after the initial constant.

In the Proof Assistant\index{Proof Assistant}, a more general unification
algorithm is used.  While a proof is being developed, sometimes not enough
information is available to determine a unique unification.  In this case
Metamath will ask you to pick the correct one.\index{ambiguous
unification}\index{unification!ambiguous}

\section{Extensions to the Metamath Language}\index{extended
language}

\subsection{Comments in the Metamath Language}\label{comments}
\index{markup notation}
\index{comments!markup notation}

The commenting feature allows you to annotate the contents of
a database.  Just as with most
computer languages, comments are ignored for the purpose of interpreting the
contents of the database. Comments effectively act as
additional white space\index{white
space} between tokens
when a database is parsed.

A comment may be placed at the beginning, end, or
between any two tokens\index{token} in a source file.

Comments have the following syntax:
\begin{center}
 \texttt{\$(} {\em text} \texttt{\$)}
\end{center}
Here,\index{\texttt{\$(} and \texttt{\$)} auxiliary
keywords}\index{comment} {\em text} is a string, possibly empty, of any
characters in Metamath's character set (p.~\pageref{spec1chars}), except
that the character strings \texttt{\$(} and \texttt{\$)} may not appear
in {\em text}.  Thus nested comments are not
permitted:\footnote{Computer languages have differing standards for
nested comments, and rather than picking one it was felt simplest not to
allow them at all, at least in the current version (0.177) of
Metamath\index{Metamath!limitations of version 0.177}.} Metamath will
complain if you give it
\begin{center}
 \texttt{\$( This is a \$( nested \$) comment.\ \$)}
\end{center}
To compensate for this non-nesting behavior, I often change all \texttt{\$}'s
to \texttt{@}'s in sections of Metamath code I wish to comment out.

The Metamath program supports a number of markup mechanisms and conventions
to generate good-looking results in \LaTeX\ and {\sc html},
as discussed below.
These markup features have to do only with how the comments are typeset,
and have no effect on how Metamath verifies the proofs in the database.
The improper
use of them may result in incorrectly typeset output, but no Metamath
error messages will result during the \texttt{read} and \texttt{verify
proof} commands.  (However, the \texttt{write
theorem\texttt{\char`\_}list} command
will check for markup errors as a side-effect of its
{\sc html} generation.)
Section~\ref{texout} has instructions for creating \LaTeX\ output, and
section~\ref{htmlout} has instructions for creating
{\sc html}\index{HTML} output.

\subsubsection{Headings}\label{commentheadings}

If the \texttt{\$(} is immediately followed by a new line
starting with a heading marker, it is a header.
This can start with:

\begin{itemize}
 \item[] \texttt{\#\#\#\#} - major part header
 \item[] \texttt{\#*\#*} - section header
 \item[] \texttt{=-=-} - subsection header
 \item[] \texttt{-.-.} - subsubsection header
\end{itemize}

The line following the marker line
will be used for the table of contents entry, after trimming spaces.
The next line should be another (closing) matching marker line.
Any text after that
but before the closing \texttt{\$}, such as an extended description of the
section, will be included on the \texttt{mmtheoremsNNN.html} page.

For more information, run
\texttt{help write theorem\char`\_list}.

\subsubsection{Math mode}
\label{mathcomments}
\index{\texttt{`} inside comments}
\index{\texttt{\char`\~} inside comments}
\index{math mode}

Inside of comments, a string of tokens\index{token} enclosed in
grave accents\index{grave accent (\texttt{`})} (\texttt{`}) will be converted
to standard mathematical symbols during
{\sc HTML}\index{HTML} or \LaTeX\ output
typesetting,\index{latex@{\LaTeX}} according to the information in the
special \texttt{\$t}\index{\texttt{\$t} comment}\index{typesetting
comment} comment in the database
(see section~\ref{tcomment} for information about the typesetting
comment, and Appendix~\ref{ASCII} to see examples of its results).

The first grave accent\index{grave accent (\texttt{`})} \texttt{`}
causes the output processor to enter {\bf math mode}\index{math mode}
and the second one exits it.
In this
mode, the characters following the \texttt{`} are interpreted as a
sequence of math symbol tokens separated by white space\index{white
space}.  The tokens are looked up in the \texttt{\$t}
comment\index{\texttt{\$t} comment}\index{typesetting comment} and if
found, they will be replaced by the standard mathematical symbols that
they correspond to before being placed in the typeset output file.  If
not found, the symbol will be output as is and a warning will be issued.
The tokens do not have to be active in the database, although a warning
will be issued if they are not declared with \texttt{\$c} or
\texttt{\$v} statements.

Two consecutive
grave accents \texttt{``} are treated as a single actual grave accent
(both inside and outside of math mode) and will not cause the output
processor to enter or exit math mode.

Here is an example of its use\index{Pierce's axiom}:
\begin{center}
\texttt{\$( Pierce's axiom, ` ( ( ph -> ps ) -> ph ) -> ph ` ,\\
         is not very intuitive. \$)}
\end{center}
becomes
\begin{center}
   \texttt{\$(} Pierce's axiom, $((\varphi \rightarrow \psi)\rightarrow
\varphi)\rightarrow \varphi$, is not very intuitive. \texttt{\$)}
\end{center}

Note that the math symbol tokens\index{token} must be surrounded by white
space\index{white space}.
%, since there is no context that allows ambiguity to be
%resolved, as is the case with math symbol sequences in some of the Metamath
%statements.
White space should also surround the \texttt{`}
delimiters.

The math mode feature also gives you a quick and easy way to generate
text containing mathematical symbols, independently of the intended
purpose of Metamath.\index{Metamath!using as a math editor} To do this,
simply create your text with grave accents surrounding your formulas,
after making sure that your math symbols are mapped to \LaTeX\ symbols
as described in Appendix~\ref{ASCII}.  It is easier if you start with a
database with predefined symbols such as \texttt{set.mm}.  Use your
grave-quoted math string to replace an existing comment, then typeset
the statement corresponding to that comment following the instructions
from the \texttt{help tex} command in the Metamath program.  You will
then probably want to edit the resulting file with a text editor to fine
tune it to your exact needs.

\subsubsection{Label Mode}\index{label mode}

Outside of math mode, a tilde\index{tilde (\texttt{\char`\~})} \verb/~/
indicates to Metamath's\index{Metamath} output processor that the
token\index{token} that follows (i.e.\ the characters up to the next
white space\index{white space}) represents a statement label or URL.
This formatting mode is called {\bf label mode}\index{label mode}.
If a literal tilde
is desired (outside of math mode) instead of label mode,
use two tildes in a row to represent it.

When generating a \LaTeX\ output file,
the following token will be formatted in \texttt{typewriter}
font, and the tilde removed, to make it stand out from the rest of the text.
This formatting will be applied to all characters after the
tilde up to the first white space\index{white space}.
Whether
or not the token is an actual statement label is not checked, and the
token does not have to have the correct syntax for a label; no error
messages will be produced.  The only effect of the label mode on the
output is that typewriter font will be used for the tokens that are
placed in the \LaTeX\ output file.

When generating {\sc html},
the tokens after the tilde {\em must} be a URL (either http: or https:)
or a valid label.
Error messages will be issued during that output if they aren't.
A hyperlink will be generated to that URL or label.

\subsubsection{Link to bibliographical reference}\index{citation}%
\index{link to bibliographical reference}

Bibliographical references are handled specially when generating
{\sc html} if formatted specially.
Text in the form \texttt{[}{\em author}\texttt{]}
is considered a link to a bibliographical reference.
See \texttt{help html} and \texttt{help write
bibliography} in the Metamath program for more
information.
% \index{\texttt{\char`\[}\ldots\texttt{]} inside comments}
See also Sections~\ref{tcomment} and \ref{wrbib}.

The \texttt{[}{\em author}\texttt{]} notation will also create an entry in
the bibliography cross-reference file generated by \texttt{write
bibliography} (Section~\ref{wrbib}) for {\sc HTML}.
For this to work properly, the
surrounding comment must be formatted as follows:
\begin{quote}
    {\em keyword} {\em label} {\em noise-word}
     \texttt{[}{\em author}\texttt{] p.} {\em number}
\end{quote}
for example
\begin{verbatim}
     Theorem 5.2 of [Monk] p. 223
\end{verbatim}
The {\em keyword} is not case sensitive and must be one of the following:
\begin{verbatim}
     theorem lemma definition compare proposition corollary
     axiom rule remark exercise problem notation example
     property figure postulate equation scheme chapter
\end{verbatim}
The optional {\em label} may consist of more than one
(non-{\em keyword} and non-{\em noise-word}) word.
The optional {\em noise-word} is one of:
\begin{verbatim}
     of in from on
\end{verbatim}
and is  ignored when the cross-reference file is created.  The
\texttt{write
biblio\-graphy} command will perform error checking to verify the
above format.\index{error checking}

\subsubsection{Parentheticals}\label{parentheticals}

The end of a comment may include one or more parenthicals, that is,
statements enclosed in parentheses.
The Metamath program looks for certain parentheticals and can issue
warnings based on them.
They are:

\begin{itemize}
 \item[] \texttt{(Contributed by }
   \textit{NAME}\texttt{,} \textit{DATE}\texttt{.)} -
   document the original contributor's name and the date it was created.
 \item[] \texttt{(Revised by }
   \textit{NAME}\texttt{,} \textit{DATE}\texttt{.)} -
   document the contributor's name and creation date
   that resulted in significant revision
   (not just an automated minimization or shortening).
 \item[] \texttt{(Proof shortened by }
   \textit{NAME}\texttt{,} \textit{DATE}\texttt{.)} -
   document the contributor's name and date that developed a significant
   shortening of the proof (not just an automated minimization).
 \item[] \texttt{(Proof modification is discouraged.)} -
   Note that this proof should normally not be modified.
 \item[] \texttt{(New usage is discouraged.)} -
   Note that this assertion should normally not be used.
\end{itemize}

The \textit{DATE} must be in form YYYY-MMM-DD, where MMM is the
English abbreviation of that month.

\subsubsection{Other markup}\label{othermarkup}
\index{markup notation}

There are other markup notations for generating good-looking results
beyond math mode and label mode:

\begin{itemize}
 \item[]
         \texttt{\char`\_} (underscore)\index{\texttt{\char`\_} inside comments} -
             Italicize text starting from
              {\em space}\texttt{\char`\_}{\em non-space} (i.e.\ \texttt{\char`\_}
              with a space before it and a non-space character after it) until
             the next
             {\em non-space}\texttt{\char`\_}{\em space}.  Normal
             punctuation (e.g.\ a trailing
             comma or period) is ignored when determining {\em space}.
 \item[]
         \texttt{\char`\_} (underscore) - {\em
         non-space}\texttt{\char`\_}{\em non-space-string}, where
          {\em non-space-string} is a string of non-space characters,
         will make {\em non-space-string} become a subscript.
 \item[]
         \texttt{<HTML>}...\texttt{</HTML>} - do not convert
         ``\texttt{<}'' and ``\texttt{>}''
         in the enclosed text when generating {\sc HTML},
         otherwise process markup normally. This allows direct insertion
         of {\sc html} commands.
 \item[]
       ``\texttt{\&}ref\texttt{;}'' - insert an {\sc HTML}
         character reference.
         This is how to insert arbitrary Unicode characters
         (such as accented characters).  Currently only directly supported
         when generating {\sc HTML}.
\end{itemize}

It is recommended that spaces surround any \texttt{\char`\~} and
\texttt{`} tokens in the comment and that a space follow the {\em label}
after a \texttt{\char`\~} token.  This will make global substitutions
to change labels and symbol names much easier and also eliminate any
future chance of ambiguity.  Spaces around these tokens are automatically
removed in the final output to conform with normal rules of punctuation;
for example, a space between a trailing \texttt{`} and a left parenthesis
will be removed.

A good way to become familiar with the markup notation is to look at
the extensive examples in the \texttt{set.mm} database.

\subsection{The Typesetting Comment (\texttt{\$t})}\label{tcomment}

The typesetting comment \texttt{\$t} in the input database file
provides the information necessary to produce good-looking results.
It provides \LaTeX\ and {\sc html}
definitions for math symbols,
as well supporting as some
customization of the generated web page.
If you add a new token to a database, you should also
update the \texttt{\$t} comment information if you want to eventually
create output in \LaTeX\ or {\sc HTML}.
See the
\texttt{set.mm}\index{set theory database (\texttt{set.mm})} database
file for an extensive example of a \texttt{\$t} comment illustrating
many of the features described below.

Programs that do not need to generate good-looking presentation results,
such as programs that only verify Metamath databases,
can completely ignore typesetting comments
and just treat them as normal comments.
Even the Metamath program only consults the
\texttt{\$t} comment information when it needs to generate typeset output
in \LaTeX\ or {\sc HTML}
(e.g., when you open a \LaTeX\ output file with the \texttt{open tex} command).

We will first discuss the syntax of typesetting comments, and then
briefly discuss how this can be used within the Metamath program.

\subsubsection{Typesetting Comment Syntax Overview}

The typesetting comment is identified by the token
\texttt{\$t}\index{\texttt{\$t} comment}\index{typesetting comment} in
the comment, and the typesetting comment ends at the matching
\texttt{\$)}:
\[
  \mbox{\tt \$(\ }
  \mbox{\tt \$t\ }
  \underbrace{
    \mbox{\tt \ \ \ \ \ \ \ \ \ \ \ }
    \cdots
    \mbox{\tt \ \ \ \ \ \ \ \ \ \ \ }
  }_{\mbox{Typesetting definitions go here}}
  \mbox{\tt \ \$)}
\]

There must be one or more white space characters, and only white space
characters, between the \texttt{\$(} that starts the comment
and the \texttt{\$t} symbol,
and the \texttt{\$t} must be followed by one
or more white space characters
(see section \ref{whitespace} for the definition of white space characters).
The typesetting comment continues until the comment end token \texttt{\$)}
(which must be preceded by one or more white space characters).

In version 0.177\index{Metamath!limitations of version 0.177} of the
Metamath program, there may be only one \texttt{\$t} comment in a
database.  This restriction may be lifted in the future to allow
many \texttt{\$t} comments in a database.

Between the \texttt{\$t} symbol (and its following white space) and the
comment end token \texttt{\$)} (and its preceding white space)
is a sequence of one or more typesetting definitions, where
each definition has the form
\textit{definition-type arg arg ... ;}.
Each of the zero or more \textit{arg} values
can be either a typesetting data or a keyword
(what keywords are allowed, and where, depends on the specific
\textit{definition-type}).
The \textit{definition-type}, and each argument \textit{arg},
are separated by one or more white space characters.
Every definition ends in an unquoted semicolon;
white space is not required before the terminating semicolon of a definition.
Each definition should start on a new line.\footnote{This
restriction of the current version of Metamath
(0.177)\index{Metamath!limitations of version 0.177} may be removed
in a future version, but you should do it anyway for readability.}

For example, this typesetting definition:
\begin{center}
 \verb$latexdef "C_" as "\subseteq";$
\end{center}
defines the token \verb$C_$ as the \LaTeX\ symbol $\subseteq$ (which means
``subset'').

Typesetting data is a sequence of one or more quoted strings
(if there is more than one, they are connected by \texttt{\char`\+}).
Often a single quoted string is used to provide data for a definition, using
either double (\texttt{\char`\"}) or single (\texttt{'}) quotation marks.
However,
{\em a quoted string (enclosed in quotation marks) may not include
line breaks.}
A quoted string
may include a quotation mark that matches the enclosing quotes by repeating
the quotation mark twice.  Here are some examples:

\begin{tabu}   { l l }
\textbf{Example} & \textbf{Meaning} \\
\texttt{\char`\"a\char`\"\char`\"b\char`\"} & \texttt{a\char`\"b} \\
\texttt{'c''d'} & \texttt{c'd} \\
\texttt{\char`\"e''f\char`\"} & \texttt{e''f} \\
\texttt{'g\char`\"\char`\"h'} & \texttt{g\char`\"\char`\"h} \\
\end{tabu}

Finally, a long quoted string
may be broken up into multiple quoted strings (considered, as a whole,
a single quoted string) and joined with \texttt{\char`\+}.
You can even use multiple lines as long as a
'+' is at the end of every line except the last one.
The \texttt{\char`\+} should be preceded and followed by at least one
white space character.
Thus, for example,
\begin{center}
 \texttt{\char`\"ab\char`\"\ \char`\+\ \char`\"cd\char`\"
    \ \char`\+\ \\ 'ef'}
\end{center}
is the same as
\begin{center}
 \texttt{\char`\"abcdef\char`\"}
\end{center}

{\sc c}-style comments \texttt{/*}\ldots\texttt{*/} are also supported.

In practice, whenever you add a new math token you will often want to add
typesetting definitions using
\texttt{latexdef}, \texttt{htmldef}, and
\texttt{althtmldef}, as described below.
That way, they will all be up to date.
Of course, whether or not you want to use all three definitions will
depend on how the database is intended to be used.

Below we discuss the different possible \textit{definition-kind} options.
We will show data surrounded by double quotes (in practice they can also use
single quotes and/or be a sequence joined by \texttt{+}s).
We will use specific names for the \textit{data} to make clear what
the data is used for, such as
{\em math-token} (for a Metamath math token,
{\em latex-string} (for string to be placed in a \LaTeX\ stream),
{\em {\sc html}-code} (for {\sc html} code),
and {\em filename} (for a filename).

\subsubsection{Typesetting Comment - \LaTeX}

The syntax for a \LaTeX\ definition is:
\begin{center}
 \texttt{latexdef "}{\em math-token}\texttt{" as "}{\em latex-string}\texttt{";}
\end{center}
\index{latex definitions@\LaTeX\ definitions}%
\index{\texttt{latexdef} statement}

The {\em token-string} and {\em latex-string} are the data
(character strings) for
the token and the \LaTeX\ definition of the token, respectively,

These \LaTeX\ definitions are used by the Metamath program
when it is asked to product \LaTeX output using
the \texttt{write tex} command.

\subsubsection{Typesetting Comment - {\sc html}}

The key kinds of {\sc HTML} definitions have the following syntax:

\vskip 1ex
    \texttt{htmldef "}{\em math-token}\texttt{" as "}{\em
    {\sc html}-code}\texttt{";}\index{\texttt{htmldef} statement}
                    \ \ \ \ \ \ldots

    \texttt{althtmldef "}{\em math-token}\texttt{" as "}{\em
{\sc html}-code}\texttt{";}\index{\texttt{althtmldef} statement}

                    \ \ \ \ \ \ldots

Note that in {\sc HTML} there are two possible definitions for math tokens.
This feature is useful when
an alternate representation of symbols is desired, for example one that
uses Unicode entities and another uses {\sc gif} images.

There are many other typesetting definitions that can control {\sc HTML}.
These include:

\vskip 1ex

    \texttt{htmldef "}{\em math-token}\texttt{" as "}{\em {\sc
    html}-code}\texttt{";}

    \texttt{htmltitle "}{\em {\sc html}-code}\texttt{";}%
\index{\texttt{htmltitle} statement}

    \texttt{htmlhome "}{\em {\sc html}-code}\texttt{";}%
\index{\texttt{htmlhome} statement}

    \texttt{htmlvarcolor "}{\em {\sc html}-code}\texttt{";}%
\index{\texttt{htmlvarcolor} statement}

    \texttt{htmlbibliography "}{\em filename}\texttt{";}%
\index{\texttt{htmlbibliography} statement}

\vskip 1ex

\noindent The \texttt{htmltitle} is the {\sc html} code for a common
title, such as ``Metamath Proof Explorer.''  The \texttt{htmlhome} is
code for a link back to the home page.  The \texttt{htmlvarcolor} is
code for a color key that appears at the bottom of each proof.  The file
specified by {\em filename} is an {\sc html} file that is assumed to
have a \texttt{<A NAME=}\ldots\texttt{>} tag for each bibiographic
reference in the database comments.  For example, if
\texttt{[Monk]}\index{\texttt{\char`\[}\ldots\texttt{]} inside comments}
occurs in the comment for a theorem, then \texttt{<A NAME='Monk'>} must
be present in the file; if not, a warning message is given.

Associated with
\texttt{althtmldef}
are the statements
\vskip 1ex

    \texttt{htmldir "}{\em
      directoryname}\texttt{";}\index{\texttt{htmldir} statement}

    \texttt{althtmldir "}{\em
     directoryname}\texttt{";}\index{\texttt{althtmldir} statement}

\vskip 1ex
\noindent giving the directories of the {\sc gif} and Unicode versions
respectively; their purpose is to provide cross-linking between the
two versions in the generated web pages.

When two different types of pages need to be produced from a single
database, such as the Hilbert Space Explorer that extends the Metamath
Proof Explorer, ``extended'' variables may be declared in the
\texttt{\$t} comment:
\vskip 1ex

    \texttt{exthtmltitle "}{\em {\sc html}-code}\texttt{";}%
\index{\texttt{exthtmltitle} statement}

    \texttt{exthtmlhome "}{\em {\sc html}-code}\texttt{";}%
\index{\texttt{exthtmlhome} statement}

    \texttt{exthtmlbibliography "}{\em filename}\texttt{";}%
\index{\texttt{exthtmlbibliography} statement}

\vskip 1ex
\noindent When these are declared, you also must declare
\vskip 1ex

    \texttt{exthtmllabel "}{\em label}\texttt{";}%
\index{\texttt{exthtmllabel} statement}

\vskip 1ex \noindent that identifies the database statement where the
``extended'' section of the database starts (in our example, where the
Hilbert Space Explorer starts).  During the generation of web pages for
that starting statement and the statements after it, the {\sc html} code
assigned to \texttt{exthtmltitle} and \texttt{exthtmlhome} is used
instead of that assigned to \texttt{htmltitle} and \texttt{htmlhome},
respectively.

\begin{sloppy}
\subsection{Additional Information Com\-ment (\texttt{\$j})} \label{jcomment}
\end{sloppy}

The additional information comment, aka the
\texttt{\$j}\index{\texttt{\$j} comment}\index{additional information comment}
comment,
provides a way to add additional structured information that can
be optionally parsed by systems.

The additional information comment is parsed the same way as the
typesetting comment (\texttt{\$t}) (see section \ref{tcomment}).
That is,
the additional information comment begins with the token
\texttt{\$j} within a comment,
and continues until the comment close \texttt{\$)}.
Within an additional information comment is a sequence of one or more
commands of the form \texttt{command arg arg ... ;}
where each of the zero or more \texttt{arg} values
can be either a quoted string or a keyword.
Note that every command ends in an unquoted semicolon.
If a verifier is parsing an additional information comment, but
doesn't recognize a particular command, it must skip the command
by finding the end of the command (an unquoted semicolon).

A database may have 0 or more additional information comments.
Note, however, that a verifier may ignore these comments entirely or only
process certain commands in an additional information comment.
The \texttt{mmj2} verifier supports many commands in additional information
comments.
We encourage systems that process additional information comments
to coordinate so that they will use the same command for the same effect.

Examples of additional information comments with various commands
(from the \texttt{set.mm} database) are:

\begin{itemize}
   \item Define the syntax and logical typecodes,
     and declare that our grammar is
     unambiguous (verifiable using the KLR parser, with compositing depth 5).
\begin{verbatim}
  $( $j
    syntax 'wff';
    syntax '|-' as 'wff';
    unambiguous 'klr 5';
  $)
\end{verbatim}

   \item Register $\lnot$ and $\rightarrow$ as primitive expressions
           (lacking definitions).
\begin{verbatim}
  $( $j primitive 'wn' 'wi'; $)
\end{verbatim}

   \item There is a special justification for \texttt{df-bi}.
\begin{verbatim}
  $( $j justification 'bijust' for 'df-bi'; $)
\end{verbatim}

   \item Register $\leftrightarrow$ as an equality for its type (wff).
\begin{verbatim}
  $( $j
    equality 'wb' from 'biid' 'bicomi' 'bitri';
    definition 'dfbi1' for 'wb';
  $)
\end{verbatim}

   \item Theorem \texttt{notbii} is the congruence law for negation.
\begin{verbatim}
  $( $j congruence 'notbii'; $)
\end{verbatim}

   \item Add \texttt{setvar} as a typecode.
\begin{verbatim}
  $( $j syntax 'setvar'; $)
\end{verbatim}

   \item Register $=$ as an equality for its type (\texttt{class}).
\begin{verbatim}
  $( $j equality 'wceq' from 'eqid' 'eqcomi' 'eqtri'; $)
\end{verbatim}

\end{itemize}


\subsection{Including Other Files in a Metamath Source File} \label{include}
\index{\texttt{\$[} and \texttt{\$]} auxiliary keywords}

The keywords \texttt{\$[} and \texttt{\$]} specify a file to be
included\index{included file}\index{file inclusion} at that point in a
Metamath\index{Metamath} source file\index{source file}.  The syntax for
including a file is as follows:
\begin{center}
\texttt{\$[} {\em file-name} \texttt{\$]}
\end{center}

The {\em file-name} should be a single token\index{token} with the same syntax
as a math symbol (i.e., all 93 non-whitespace
printable characters other than \texttt{\$} are
allowed, subject to the file-naming limitations of your operating system).
Comments may appear between the \texttt{\$[} and \texttt{\$]} keywords.  Included
files may include other files, which may in turn include other files, and so
on.

For example, suppose you want to use the set theory database as the starting
point for your own theory.  The first line in your file could be
\begin{center}
\texttt{\$[ set.mm \$]}
\end{center} All of the information (axioms, theorems,
etc.) in \texttt{set.mm} and any files that {\em it} includes will become
available for you to reference in your file. This can help make your work more
modular. A drawback to including files is that if you change the name of a
symbol or the label of a statement, you must also remember to update any
references in any file that includes it.


The naming conventions for included files are the same as those of your
operating system.\footnote{On the Macintosh, prior to Mac OS X,
 a colon is used to separate disk
and folder names from your file name.  For example, {\em volume}\texttt{:}{\em
file-name} refers to the root directory, {\em volume}\texttt{:}{\em
folder-name}\texttt{:}{\em file-name} refers to a folder in root, and {\em
volume}\texttt{:}{\em folder-name}\texttt{:}\ldots\texttt{:}{\em file-name} refers to a
deeper folder.  A simple {\em file-name} refers to a file in the folder from
which you launch the Metamath application.  Under Mac OS X and later,
the Metamath program is run under the Terminal application, which
conforms to Unix naming conventions.}\index{Macintosh file
names}\index{file names!Macintosh}\label{includef} For compatibility among
operating systems, you should keep the file names as simple as possible.  A
good convention to use is {\em file}\texttt{.mm} where {\em file} is eight
characters or less, in lower case.

There is no limit to the nesting depth of included files.  One thing that you
should be aware of is that if two included files themselves include a common
third file, only the {\em first} reference to this common file will be read
in.  This allows you to include two or more files that build on a common
starting file without having to worry about label and symbol conflicts that
would occur if the common file were read in more than once.  (In fact, if a
file includes itself, the self-reference will be ignored, although of course
it would not make any sense to do that.)  This feature also means, however,
that if you try to include a common file in several inner blocks, the result
might not be what you expect, since only the first reference will be replaced
with the included file (unlike the include statement in most other computer
languages).  Thus you would normally include common files only in the
outermost block\index{outermost block}.

\subsection{Compressed Proof Format}\label{compressed1}\index{compressed
proof}\index{proof!compressed}

The proof notation presented in Section~\ref{proof} is called a
{\bf normal proof}\index{normal proof}\index{proof!normal} and in principle is
sufficient to express any proof.  However, proofs often contain steps and
subproofs that are identical.  This is particularly true in typical
Metamath\index{Metamath} applications, because Metamath requires that the math
symbol sequence (usually containing a formula) at each step be separately
constructed, that is, built up piece by piece. As a result, a lot of
repetition often results.  The {\bf compressed proof} format allows Metamath
to take advantage of this redundancy to shorten proofs.

The specification for the compressed proof format is given in
Appen\-dix~\ref{compressed}.

Normally you need not concern yourself with the details of the compressed
proof format, since the Metamath program will allow you to convert from
the normal format to the compressed format with ease, and will also
automatically convert from the compressed format when proofs are displayed.
The overall structure of the compressed format is as follows:
\begin{center}
  \texttt{\$= ( } {\em label-list} \texttt{) } {\em compressed-proof\ }\ \texttt{\$.}
\end{center}
\index{\texttt{\$=} keyword}
The first \texttt{(} serves as a flag to Metamath that a compressed proof
follows.  The {\em label-list} includes all statements referred to by the
proof except the mandatory hypotheses\index{mandatory hypothesis}.  The {\em
compressed-proof} is a compact encoding of the proof, using upper-case
letters, and can be thought of as a large integer in base 26.  White
space\index{white space} inside a {\em compressed-proof} is
optional and is ignored.

It is important to note that the order of the mandatory hypotheses of
the statement being proved must not be changed if the compressed proof
format is used, otherwise the proof will become incorrect.  The reason
for this is that the mandatory hypotheses are not mentioned explicitly
in the compressed proof in order to make the compression more efficient.
If you wish to change the order of mandatory hypotheses, you must first
convert the proof back to normal format using the \texttt{save proof
{\em statement} /normal}\index{\texttt{save proof} command} command.
Later, you can go back to compressed format with \texttt{save proof {\em
statement} /compressed}.

During error checking with the \texttt{verify proof} command, an error
found in a compressed proof may point to a character in {\em
compressed-proof}, which may not be very meaningful to you.  In this
case, try to \texttt{save proof /normal} first, then do the
\texttt{verify proof} again.  In general, it is best to make sure a
proof is correct before saving it in compressed format, because severe
errors are less likely to be recoverable than in normal format.

\subsection{Specifying Unknown Proofs or Subproofs}\label{unknown}

In a proof under development, any step or subproof that is not yet known
may be represented with a single \texttt{?}.  For the purposes of
parsing the proof, the \texttt{?}\ \index{\texttt{]}@\texttt{?}\ inside
proofs} will push a single entry onto the RPN stack just as if it were a
hypothesis.  While developing a proof with the Proof
Assistant\index{Proof Assistant}, a partially developed proof may be
saved with the \texttt{save new{\char`\_}proof}\index{\texttt{save
new{\char`\_}proof} command} command, and \texttt{?}'s will be placed at
the appropriate places.

All \texttt{\$p}\index{\texttt{\$p} statement} statements must have
proofs, even if they are entirely unknown.  Before creating a proof with
the Proof Assistant, you should specify a completely unknown proof as
follows:
\begin{center}
  {\em label} \texttt{\$p} {\em statement} \texttt{\$= ?\ \$.}
\end{center}
\index{\texttt{\$=} keyword}
\index{\texttt{]}@\texttt{?}\ inside proofs}

The \texttt{verify proof}\index{\texttt{verify proof} command} command
will check the known portions of a partial proof for errors, but will
warn you that the statement has not been proved.

Note that partially developed proofs may be saved in compressed format
if desired.  In this case, you will see one or more \texttt{?}'s in the
{\em compressed-proof} part.\index{compressed
proof}\index{proof!compressed}

\section{Axioms vs.\ Definitions}\label{definitions}

The \textit{basic}
Metamath\index{Metamath} language and program
make no distinction\index{axiom vs.\
definition} between axioms\index{axiom} and
definitions.\index{definition} The \texttt{\$a}\index{\texttt{\$a}
statement} statement is used for both.  At first, this may seem
puzzling.  In the minds of many mathematicians, the distinction is
clear, even obvious, and hardly worth discussing.  A definition is
considered to be merely an abbreviation that can be replaced by the
expression for which it stands; although unless one actually does this,
to be precise then one should say that a theorem\index{theorem} is a
consequence of the axioms {\em and} the definitions that are used in the
formulation of the theorem \cite[p.~20]{Behnke}.\index{Behnke, H.}

\subsection{What is a Definition?}

What is a definition?  In its simplest form, a definition introduces a new
symbol and provides an unambiguous rule to transform an expression containing
the new symbol to one without it.  The concept of a ``proper
definition''\index{proper definition}\index{definition!proper} (as opposed to
a creative definition)\index{creative definition}\index{definition!creative}
that is usually agreed upon is (1) the definition should not strengthen the
language and (2) any symbols introduced by the definition should be eliminable
from the language \cite{Nemesszeghy}\index{Nemesszeghy, E. Z.}.  In other
words, they are mere typographical conveniences that do not belong to the
system and are theoretically superfluous.  This may seem obvious, but in fact
the nature of definitions can be subtle, sometimes requiring difficult
metatheorems to establish that they are not creative.

A more conservative stance was taken by logician S.
Le\'{s}niewski.\index{Le\'{s}niewski, S.}
\begin{quote}
Le\'{s}niewski
regards definitions as theses of the system.  In this respect they do
not differ either from the axioms or from theorems, i.e.\ from the
theses added to the system on the basis of the rule of substitution or
the rule of detachment [modus ponens].  Once definitions have been
accepted as theses of the system, it becomes necessary to consider them
as true propositions in the same sense in which axioms are true
\cite{Lejewski}.
\end{quote}\index{Lejewski, Czeslaw}

Let us look at some simple examples of definitions in propositional
calculus.  Consider the definition of logical {\sc or}
(disjunction):\index{disjunction ($\vee$)} ``$P\vee Q$ denotes $\neg P
\rightarrow Q$ (not $P$ implies $Q$).''  It is very easy to recognize a
statement making use of this definition, because it introduces the new
symbol $\vee$ that did not previously exist in the language.  It is easy
to see that no new theorems of the original language will result from
this definition.

Next, consider a definition that eliminates parentheses:  ``$P
\rightarrow Q\rightarrow R$ denotes $P\rightarrow (Q \rightarrow R)$.''
This is more subtle, because no new symbols are introduced.  The reason
this definition is considered proper is that no new symbol sequences
that are valid wffs (well-formed formulas)\index{well-formed formula
(wff)} in the original language will result from the definition, since
``$P \rightarrow Q\rightarrow R$'' is not a wff in the original
language.  Here, we implicitly make use of the fact that there is a
decision procedure that allows us to determine whether or not a symbol
sequence is a wff, and this fact allows us to use symbol sequences that
are not wffs to represent other things (such as wffs) by means of the
definition.  However, to justify the definition as not being creative we
need to prove that ``$P \rightarrow Q\rightarrow R$'' is in fact not a
wff in the original language, and this is more difficult than in the
case where we simply introduce a new symbol.

%Now let's take this reasoning to an extreme.  Propositional calculus is a
%decidable theory,\footnote{This means that a mechanical algorithm exists to
%determine whether or not a wff is a theorem.} so in principle we could make use
%of symbol sequences that are not theorems to represent other things (say, to
%encode actual theorems in a more compact way).  For example, let us extend the
%language by defining a wff ``$P$'' in the extended language as the theorem
%``$P\rightarrow P$''\footnote{This is one of the first theorems proved in the
%Metamath database \texttt{set.mm}.}\index{set
%theory database (\texttt{set.mm})} in the original language whenever ``$P$'' is
%not a theorem in the original language.  In the extended language, any wff
%``$Q$'' thus represents a theorem; to find out what theorem (in the original
%language) ``$Q$'' represents, we determine whether ``$Q$'' is a theorem in the
%original language (before the definition was introduced).  If so, we're done; if
%not, we replace ``$Q$'' by ``$Q\rightarrow Q$'' to eliminate the definition.
%This definition is therefore eliminable, and it does not ``strengthen'' the
%language because any wff that is not a theorem is not in the set of statements
%provable in the original language and thus is available for use by definitions.
%
%Of course, a definition such as this would render practically useless the
%communication of theorems of propositional calculus; but
%this is just a human shortcoming, since we can't always easily discern what is
%and is not a theorem by inspection.  In fact, the extended theory with this
%definition has no more and no less information than the original theory; it just
%expresses certain theorems of the form ``$P\rightarrow P$''
%in a more compact way.
%
%The point here is that what constitutes a proper definition is a matter of
%judgment about whether a symbol sequence can easily be recognized by a human
%as invalid in some sense (for example, not a wff); if so, the symbol sequence
%can be appropriated for use by a definition in order to make the extended
%language more compact.  Metamath\index{Metamath} lacks the ability to make this
%judgment, since as far as Metamath is concerned the definition of a wff, for
%example, is arbitrary.  You define for Metamath how wffs\index{well-formed
%formula (wff)} are constructed according to your own preferred style.  The
%concept of a wff may not even exist in a given formal system\index{formal
%system}.  Metamath treats all definitions as if they were new axioms, and it
%is up to the human mathematician to judge whether the definition is ``proper''
%'\index{proper definition}\index{definition!proper} in some agreed-upon way.

What constitutes a definition\index{definition} versus\index{axiom vs.\
definition} an axiom\index{axiom} is sometimes arbitrary in mathematical
literature.  For example, the connectives $\vee$ ({\sc or}), $\wedge$
({\sc and}), and $\leftrightarrow$ (equivalent to) in propositional
calculus are usually considered defined symbols that can be used as
abbreviations for expressions containing the ``primitive'' connectives
$\rightarrow$ and $\neg$.  This is the way we treat them in the standard
logic and set theory database \texttt{set.mm}\index{set theory database
(\texttt{set.mm})}.  However, the first three connectives can also be
considered ``primitive,'' and axiom systems have been devised that treat
all of them as such.  For example,
\cite[p.~35]{Goodstein}\index{Goodstein, R. L.} presents one with 15
axioms, some of which in fact coincide with what we have chosen to call
definitions in \texttt{set.mm}.  In certain subsets of classical
propositional calculus, such as the intuitionist
fragment\index{intuitionism}, it can be shown that one cannot make do
with just $\rightarrow$ and $\neg$ but must treat additional connectives
as primitive in order for the system to make sense.\footnote{Two nice
systems that make the transition from intuitionistic and other weak
fragments to classical logic just by adding axioms are given in
\cite{Robinsont}\index{Robinson, T. Thacher}.}

\subsection{The Approach to Definitions in \texttt{set.mm}}

In set theory, recursive definitions define a newly introduced symbol in
terms of itself.
The justification of recursive definitions, using
several ``recursion theorems,'' is usually one of the first
sophisticated proofs a student encounters when learning set theory, and
there is a significant amount of implicit metalogic behind a recursive
definition even though the definition itself is typically simple to
state.

Metamath itself has no built-in technical limitation that prevents
multiple-part recursive definitions in the traditional textbook style.
However, because the recursive definition requires advanced metalogic
to justify, eliminating a recursive definition is very difficult and
often not even shown in textbooks.

\subsubsection{Direct definitions instead of recursive definitions}

It is, however, possible to substitute one kind of complexity
for another.  We can eliminate the need for metalogical justification by
defining the operation directly with an explicit (but complicated)
expression, then deriving the recursive definition directly as a
theorem, using a recursion theorem ``in reverse.''
The elimination
of a direct definition is a matter of simple mechanical substitution.
We do this in
\texttt{set.mm}, as follows.

In \texttt{set.mm} our goal was to introduce almost all definitions in
the form of two expressions connected by either $\leftrightarrow$ or
$=$, where the thing being defined does not appear on the right hand
side.  Quine calls this form ``a genuine or direct definition'' \cite[p.
174]{Quine}\index{Quine, Willard Van Orman}, which makes the definitions
very easy to eliminate and the metalogic\index{metalogic} needed to
justify them as simple as possible.
Put another way, we had a goal of being able to
eliminate all definitions with direct mechanical substitution and to
verify easily the soundness of the definitions.

\subsubsection{Example of direct definitions}

We achieved this goal in almost all cases in \texttt{set.mm}.
Sometimes this makes the definitions more complex and less
intuitive.
For example, the traditional way to define addition of
natural numbers is to define an operation called {\em
successor}\index{successor} (which means ``plus one'' and is denoted by
``${\rm suc}$''), then define addition recursively\index{recursive
definition} with the two definitions $n + 0 = n$ and $m + {\rm suc}\,n =
{\rm suc} (m + n)$.  Although this definition seems simple and obvious,
the method to eliminate the definition is not obvious:  in the second
part of the definition, addition is defined in terms of itself.  By
eliminating the definition, we don't mean repeatedly applying it to
specific $m$ and $n$ but rather showing the explicit, closed-form
set-theoretical expression that $m + n$ represents, that will work for
any $m$ and $n$ and that does not have a $+$ sign on its right-hand
side.  For a recursive definition like this not to be circular
(creative), there are some hidden, underlying assumptions we must make,
for example that the natural numbers have a certain kind of order.

In \texttt{set.mm} we chose to start with the direct (though complex and
nonintuitive) definition then derive from it the standard recursive
definition.
For example, the closed-form definition used in \texttt{set.mm}
for the addition operation on ordinals\index{ordinal
addition}\index{addition!of ordinals} (of which natural numbers are a
subset) is

\setbox\startprefix=\hbox{\tt \ \ df-oadd\ \$a\ }
\setbox\contprefix=\hbox{\tt \ \ \ \ \ \ \ \ \ \ \ \ \ }
\startm
\m{\vdash}\m{+_o}\m{=}\m{(}\m{x}\m{\in}\m{{\rm On}}\m{,}\m{y}\m{\in}\m{{\rm
On}}\m{\mapsto}\m{(}\m{{\rm rec}}\m{(}\m{(}\m{z}\m{\in}\m{{\rm
V}}\m{\mapsto}\m{{\rm suc}}\m{z}\m{)}\m{,}\m{x}\m{)}\m{`}\m{y}\m{)}\m{)}
\endm
\noindent which depends on ${\rm rec}$.

\subsubsection{Recursion operators}

The above definition of \texttt{df-oadd} depends on the definition of
${\rm rec}$, a ``recursion operator''\index{recursion operator} with
the definition \texttt{df-rdg}:

\setbox\startprefix=\hbox{\tt \ \ df-rdg\ \$a\ }
\setbox\contprefix=\hbox{\tt \ \ \ \ \ \ \ \ \ \ \ \ }
\startm
\m{\vdash}\m{{\rm
rec}}\m{(}\m{F}\m{,}\m{I}\m{)}\m{=}\m{\mathrm{recs}}\m{(}\m{(}\m{g}\m{\in}\m{{\rm
V}}\m{\mapsto}\m{{\rm if}}\m{(}\m{g}\m{=}\m{\varnothing}\m{,}\m{I}\m{,}\m{{\rm
if}}\m{(}\m{{\rm Lim}}\m{{\rm dom}}\m{g}\m{,}\m{\bigcup}\m{{\rm
ran}}\m{g}\m{,}\m{(}\m{F}\m{`}\m{(}\m{g}\m{`}\m{\bigcup}\m{{\rm
dom}}\m{g}\m{)}\m{)}\m{)}\m{)}\m{)}\m{)}
\endm

\noindent which can be further broken down with definitions shown in
Section~\ref{setdefinitions}.

This definition of ${\rm rec}$
defines a recursive definition generator on ${\rm On}$ (the class of ordinal
numbers) with characteristic function $F$ and initial value $I$.
This operation allows us to define, with
compact direct definitions, functions that are usually defined in
textbooks with recursive definitions.
The price paid with our approach
is the complexity of our ${\rm rec}$ operation
(especially when {\tt df-recs} that it is built on is also eliminated).
But once we get past this hurdle, definitions that would otherwise be
recursive become relatively simple, as in for example {\tt oav}, from
which we prove the recursive textbook definition as theorems {\tt oa0}, {\tt
oasuc}, and {\tt oalim} (with the help of theorems {\tt rdg0}, {\tt rdgsuc},
and {\tt rdglim2a}).  We can also restrict the ${\rm rec}$ operation to
define otherwise recursive functions on the natural numbers $\omega$; see {\tt
fr0g} and {\tt frsuc}.  Our ${\rm rec}$ operation apparently does not appear
in published literature, although closely related is Definition 25.2 of
[Quine] p. 177, which he uses to ``turn...a recursion into a genuine or
direct definition" (p. 174).  Note that the ${\rm if}$ operations (see
{\tt df-if}) select cases based on whether the domain of $g$ is zero, a
successor, or a limit ordinal.

An important use of this definition ${\rm rec}$ is in the recursive sequence
generator {\tt df-seq} on the natural numbers (as a subset of the
complex infinite sequences such as the factorial function {\tt df-fac} and
integer powers {\tt df-exp}).

The definition of ${\rm rec}$ depends on ${\rm recs}$.
New direct usage of the more powerful (and more primitive) ${\rm recs}$
construct is discouraged, but it is available when needed.
This
defines a function $\mathrm{recs} ( F )$ on ${\rm On}$, the class of ordinal
numbers, by transfinite recursion given a rule $F$ which sets the next
value given all values so far.
Unlike {\tt df-rdg} which restricts the
update rule to use only the previous value, this version allows the
update rule to use all previous values, which is why it is described
as ``strong,'' although it is actually more primitive.  See {\tt
recsfnon} and {\tt recsval} for the primary contract of this definition.
It is defined as:

\setbox\startprefix=\hbox{\tt \ \ df-recs\ \$a\ }
\setbox\contprefix=\hbox{\tt \ \ \ \ \ \ \ \ \ \ \ \ \ }
\startm
\m{\vdash}\m{\mathrm{recs}}\m{(}\m{F}\m{)}\m{=}\m{\bigcup}\m{\{}\m{f}\m{|}\m{\exists}\m{x}\m{\in}\m{{\rm
On}}\m{(}\m{f}\m{{\rm
Fn}}\m{x}\m{\wedge}\m{\forall}\m{y}\m{\in}\m{x}\m{(}\m{f}\m{`}\m{y}\m{)}\m{=}\m{(}\m{F}\m{`}\m{(}\m{f}\m{\restriction}\m{y}\m{)}\m{)}\m{)}\m{\}}
\endm

\subsubsection{Closing comments on direct definitions}

From these direct definitions the simpler, more
intuitive recursive definition is derived as a set of theorems.\index{natural
number}\index{addition}\index{recursive definition}\index{ordinal addition}
The end result is the same, but we completely eliminate the rather complex
metalogic that justifies the recursive definition.

Recursive definitions are often considered more efficient and intuitive than
direct ones once the metalogic has been learned or possibly just accepted as
correct.  However, it was felt that direct definition in \texttt{set.mm}
maximizes rigor by minimizing metalogic.  It can be eliminated effortlessly,
something that is difficult to do with a recursive definition.

Again, Metamath itself has no built-in technical limitation that prevents
multiple-part recursive definitions in the traditional textbook style.
Instead, our goal is to eliminate all definitions with
direct mechanical substitution and to verify easily the soundness of
definitions.

\subsection{Adding Constraints on Definitions}

The basic Metamath language and the Metamath program do
not have any built-in constraints on definitions, since they are just
\$a statements.

However, nothing prevents a verification system from verifying
additional rules to impose further limitations on definitions.
For example, the \texttt{mmj2}\index{mmj2} program
supports various kinds of
additional information comments (see section \ref{jcomment}).
One of their uses is to optionally verify additional constraints,
including constraints to verify that definitions meet certain
requirements.
These additional checks are required by the
continuous integration (CI)\index{continuous integration (CI)}
checks of the
\texttt{set.mm}\index{set theory database (\texttt{set.mm})}%
\index{Metamath Proof Explorer}
database.
This approach enables us to optionally impose additional requirements
on definitions if we wish, without necessarily imposing those rules on
all databases or requiring all verification systems to implement them.
In addition, this allows us to impose specialized constraints tailored
to one database while not requiring other systems to implement
those specialized constraints.

We impose two constraints on the
\texttt{set.mm}\index{set theory database (\texttt{set.mm})}%
\index{Metamath Proof Explorer} database
via the \texttt{mmj2}\index{mmj2} program that are worth discussing here:
a parse check and a definition soundness check.

% On February 11, 2019 8:32:32 PM EST, saueran@oregonstate.edu wrote:
% The following addition to the end of set.mm is accepted by the mmj2
% parser and definition checker and the metamath verifier(at least it was
% when I checked, you should check it too), and creates a contradiction by
% proving the theorem |- ph.
% ${
% wleftp $a wff ( ( ph ) $.
% wbothp $a wff ( ph ) $.
% df-leftp $a |- ( ( ( ph ) <-> -. ph ) $.
% df-bothp $a |- ( ( ph ) <-> ph ) $.
% anything $p |- ph $=
%   ( wbothp wn wi wleftp df-leftp biimpi df-bothp mpbir mpbi simplim ax-mp)
%   ABZAMACZDZCZMOEZOCQAEZNDZRNAFGSHIOFJMNKLAHJ $.
% $}
%
% This particular problem is countered by enabling, within mmj2,
% SetParser,mmj.verify.LRParser

First,
we enable a parse check in \texttt{mmj2} (through its
\texttt{SetParser} command) that requires that all new definitions
must \textit{not} create an ambiguous parse for a KLR(5) parser.
This prevents some errors such as definitions with imbalanced parentheses.

Second, we run a definition soundness check specific to
\texttt{set.mm} or databases similar to it.
(through the \texttt{definitionCheck} macro).
Some \texttt{\$a} statements (including all ax-* statemnets)
are excluded from these checks, as they will
always fail this simple check,
but they are appropriate for most definitions.
This check imposes a set of additional rules:

\begin{enumerate}

\item New definitions must be introduced using $=$ or $\leftrightarrow$.

\item No \texttt{\$a} statement introduced before this one is allowed to use the
symbol being defined in this definition, and the definition is not
allowed to use itself (except once, in the definiendum).

\item Every variable in the definiens should not be distinct

\item Every dummy variable in the definiendum
are required to be distinct from each other and from variables in
the definiendum.
To determine this, the system will look for a "justification" theorem
in the database, and if it is not there, attempt to briefly prove
$( \varphi \rightarrow \forall x \varphi )$  for each dummy variable x.

\item Every dummy variable should be a set variable,
unless there is a justification theorem available.

\item Every dummy variable must be bound
(if the system cannot determine this a justification theorem must be
provided).

\end{enumerate}

\subsection{Summary of Approach to Definitions}

In short, when being rigorous it turns out that
definitions can be subtle, sometimes requiring difficult
metatheorems to establish that they are not creative.

Instead of building such complications into the Metamath language itself,
the basic Metmath language and program simply treat traditional
axioms and definitions as the same kind of \texttt{\$a} statement.
We have then built various tools to enable people to
verify additional conditions as their creators believe is appropriate
for those specific databases, without complicating the Metamath foundations.

\chapter{The Metamath Program}\label{commands}

This chapter provides a reference manual for the
Metamath program.\index{Metamath!commands}

Current instructions for obtaining and installing the Metamath program
can be found at the \url{http://metamath.org} web site.
For Windows, there is a pre-compiled version called
\texttt{metamath.exe}.  For Unix, Linux, and Mac OS X
(which we will refer to collectively as ``Unix''), the Metamath program
can be compiled from its source code with the command
\begin{verbatim}
gcc *.c -o metamath
\end{verbatim}
using the \texttt{gcc} {\sc c} compiler available on those systems.

In the command syntax descriptions below, fields enclosed in square
brackets [\ ] are optional.  File names may be optionally enclosed in
single or double quotes.  This is useful if the file name contains
spaces or
slashes (\texttt{/}), such as in Unix path names, \index{Unix file
names}\index{file names!Unix} that might be confused with Metamath
command qualifiers.\index{Unix file names}\index{file names!Unix}

\section{Invoking Metamath}

Unix, Linux, and Mac OS X
have a command-line interface called the {\em
bash shell}.  (In Mac OS X, select the Terminal application from
Applications/Utilities.)  To invoke Metamath from the bash shell prompt,
assuming that the Metamath program is in the current directory, type
\begin{verbatim}
bash$ ./metamath
\end{verbatim}

To invoke Metamath from a Windows DOS or Command Prompt, assuming that
the Metamath program is in the current directory (or in a directory
included in the Path system environment variable), type
\begin{verbatim}
C:\metamath>metamath
\end{verbatim}

To use command-line arguments at invocation, the command-line arguments
should be a list of Metamath commands, surrounded by quotes if they
contain spaces.  In Windows, the surrounding quotes must be double (not
single) quotes.  For example, to read the database file \texttt{set.mm},
verify all proofs, and exit the program, type (under Unix)
\begin{verbatim}
bash$ ./metamath 'read set.mm' 'verify proof *' exit
\end{verbatim}
Note that in Unix, any directory path with \texttt{/}'s must be
surrounded by quotes so Metamath will not interpret the \texttt{/} as a
command qualifier.  So if \texttt{set.mm} is in the \texttt{/tmp}
directory, use for the above example
\begin{verbatim}
bash$ ./metamath 'read "/tmp/set.mm"' 'verify proof *' exit
\end{verbatim}

For convenience, if the command-line has one argument and no spaces in
the argument, the command is implicitly assumed to be \texttt{read}.  In
this one special case, \texttt{/}'s are not interpreted as command
qualifiers, so you don't need quotes around a Unix file name.  Thus
\begin{verbatim}
bash$ ./metamath /tmp/set.mm
\end{verbatim}
and
\begin{verbatim}
bash$ ./metamath "read '/tmp/set.mm'"
\end{verbatim}
are equivalent.


\section{Controlling Metamath}

The Metamath program was first developed on a {\sc vax/vms} system, and
some aspects of its command line behavior reflect this heritage.
Hopefully you will find it reasonably user-friendly once you get used to
it.

Each command line is a sequence of English-like words separated by
spaces, as in \texttt{show settings}.  Command words are not case
sensitive, and only as many letters are needed as are necessary to
eliminate ambiguity; for example, \texttt{sh se} would work for the
command \texttt{show settings}.  In some cases arguments such as file
names, statement labels, or symbol names are required; these are
case-sensitive (although file names may not be on some operating
systems).

A command line is entered by typing it in then pressing the {\em return}
({\em enter}) key.  To find out what commands are available, type
\texttt{?} at the \texttt{MM>} prompt.  To find out the choices at any
point in a command, press {\em return} and you will be prompted for
them.  The default choice (the one selected if you just press {\em
return}) is shown in brackets (\texttt{<>}).

You may also type \texttt{?} in place of a command word to force
Metamath to tell you what the choices are.  The \texttt{?} method won't
work, though, if a non-keyword argument such as a file name is expected
at that point, because the program will think that \texttt{?} is the
value of the argument.

Some commands have one or more optional qualifiers which modify the
behavior of the command.  Qualifiers are preceded by a slash
(\texttt{/}), such as in \texttt{read set.mm / verify}.  Spaces are
optional around the \texttt{/}.  If you need to use a space or
slash in a command
argument, as in a Unix file name, put single or double quotes around the
command argument.

The \texttt{open log} command will save everything you see on the
screen and is useful to help you recover should something go wrong in a
proof, or if you want to document a bug.

If a command responds with more than a screenful, you will be
prompted to \texttt{<return> to continue, Q to quit, or S to scroll to
end}.  \texttt{Q} or \texttt{q} (not case-sensitive) will complete the
command internally but will suppress further output until the next
\texttt{MM>} prompt.  \texttt{s} will suppress further pausing until the
next \texttt{MM>} prompt.  After the first screen, you are also
presented with the choice of \texttt{b} to go back a screenful.  Note
that \texttt{b} may also be entered at the \texttt{MM>} prompt
immediately after a command to scroll back through the output of that
command.

A command line enclosed in quotes is executed by your operating system.
See Section~\ref{oscmd}.

{\em Warning:} Pressing {\sc ctrl-c} will abort the Metamath program
unconditionally.  This means any unsaved work will be lost.


\subsection{\texttt{exit} Command}\index{\texttt{exit} command}

Syntax:  \texttt{exit} [\texttt{/force}]

This command exits from Metamath.  If there have been changes to the
source with the \texttt{save proof} or \texttt{save new{\char`\_}proof}
commands, you will be given an opportunity to \texttt{write source} to
permanently save the changes.

In Proof Assistant\index{Proof Assistant} mode, the \texttt{exit} command will
return to the \verb/MM>/ prompt. If there were changes to the proof, you will
be given an opportunity to \texttt{save new{\char`\_}proof}.

The \texttt{quit} command is a synonym for \texttt{exit}.

Optional qualifier:
    \texttt{/force} - Do not prompt if changes were not saved.  This qualifier is
        useful in \texttt{submit} command files (Section~\ref{sbmt})
        to ensure predictable behavior.





\subsection{\texttt{open log} Command}\index{\texttt{open log} command}
Syntax:  \texttt{open log} {\em file-name}

This command will open a log file that will store everything you see on
the screen.  It is useful to help recovery from a mistake in a long Proof
Assistant session, or to document bugs.\index{Metamath!bugs}

The log file can be closed with \texttt{close log}.  It will automatically be
closed upon exiting Metamath.



\subsection{\texttt{close log} Command}\index{\texttt{close log} command}
Syntax:  \texttt{close log}

The \texttt{close log} command closes a log file if one is open.  See
also \texttt{open log}.




\subsection{\texttt{submit} Command}\index{\texttt{submit} command}\label{sbmt}
Syntax:  \texttt{submit} {\em filename}

This command causes further command lines to be taken from the specified
file.  Note that any line beginning with an exclamation point (\texttt{!}) is
treated as a comment (i.e.\ ignored).  Also note that the scrolling
of the screen output is continuous, so you may want to open a log file
(see \texttt{open log}) to record the results that fly by on the screen.
After the lines in the file are exhausted, Metamath returns to its
normal user interface mode.

The \texttt{submit} command can be called recursively (i.e. from inside
of a \texttt{submit} command file).


Optional command qualifier:

    \texttt{/silent} - suppresses the screen output but still
        records the output in a log file if one is open.


\subsection{\texttt{erase} Command}\index{\texttt{erase} command}
Syntax:  \texttt{erase}

This command will reset Metamath to its starting state, deleting any
data\-base that was \texttt{read} in.
 If there have been changes to the
source with the \texttt{save proof} or \texttt{save new{\char`\_}proof}
commands, you will be given an opportunity to \texttt{write source} to
permanently save the changes.



\subsection{\texttt{set echo} Command}\index{\texttt{set echo} command}
Syntax:  \texttt{set echo on} or \texttt{set echo off}

The \texttt{set echo on} command will cause command lines to be echoed with any
abbreviations expanded.  While learning the Metamath commands, this
feature will show you the exact command that your abbreviated input
corresponds to.



\subsection{\texttt{set scroll} Command}\index{\texttt{set scroll} command}
Syntax:  \texttt{set scroll prompted} or \texttt{set scroll continuous}

The Metamath command line interface starts off in the \texttt{prompted} mode,
which means that you will be prompted to continue or quit after each
full screen in a long listing.  In \texttt{continuous} mode, long listings will be
scrolled without pausing.

% LaTeX bug? (1) \texttt{\_} puts out different character than
% \texttt{{\char`\_}}
%  = \verb$_$  (2) \texttt{{\char`\_}} puts out garbage in \subsection
%  argument
\subsection{\texttt{set width} Command}\index{\texttt{set
width} command}
Syntax:  \texttt{set width} {\em number}

Metamath assumes the width of your screen is 79 characters (chosen
because the Command Prompt in Windows XP has a wrapping bug at column
80).  If your screen is wider or narrower, this command allows you to
change this default screen width.  A larger width is advantageous for
logging proofs to an output file to be printed on a wide printer.  A
smaller width may be necessary on some terminals; in this case, the
wrapping of the information messages may sometimes seem somewhat
unnatural, however.  In \LaTeX\index{latex@{\LaTeX}!characters per line},
there is normally a maximum of 61 characters per line with typewriter
font.  (The examples in this book were produced with 61 characters per
line.)

\subsection{\texttt{set height} Command}\index{\texttt{set
height} command}
Syntax:  \texttt{set height} {\em number}

Metamath assumes your screen height is 24 lines of characters.  If your
screen is taller or shorter, this command lets you to change the number
of lines at which the display pauses and prompts you to continue.

\subsection{\texttt{beep} Command}\index{\texttt{beep} command}

Syntax:  \texttt{beep}

This command will produce a beep.  By typing it ahead after a
long-running command has started, it will alert you that the command is
finished.  For convenience, \texttt{b} is an abbreviation for
\texttt{beep}.

Note:  If \texttt{b} is typed at the \texttt{MM>} prompt immediately
after the end of a multiple-page display paged with ``\texttt{Press
<return> for more}...'' prompts, then the \texttt{b} will back up to the
previous page rather than perform the \texttt{beep} command.
In that case you must type the unabbreviated \texttt{beep} form
of the command.

\subsection{\texttt{more} Command}\index{\texttt{more} command}

Syntax:  \texttt{more} {\em filename}

This command will display the contents of an {\sc ascii} file on your
screen.  (This command is provided for convenience but is not very
powerful.  See Section~\ref{oscmd} to invoke your operating system's
command to do this, such as the \texttt{more} command in Unix.)

\subsection{Operating System Commands}\index{operating system
command}\label{oscmd}

A line enclosed in single or double quotes will be executed by your
computer's operating system if it has a command line interface.  For
example, on a {\sc vax/vms} system,
\verb/MM> 'dir'/
will print disk directory contents.  Note that this feature will not
work on the Macintosh prior to Mac OS X, which does not have a command
line interface.

For your convenience, the trailing quote is optional.

\subsection{Size Limitations in Metamath}

In general, there are no fixed, predefined limits\index{Metamath!memory
limits} on how many labels, tokens\index{token}, statements, etc.\ that
you may have in a database file.  The Metamath program uses 32-bit
variables (64-bit on 64-bit CPUs) as indices for almost all internal
arrays, which are allocated dynamically as needed.



\section{Reading and Writing Files}

The following commands create new files:  the \texttt{open} commands;
the \texttt{write} commands; the \texttt{/html},
\texttt{/alt{\char`\_}html}, \texttt{/brief{\char`\_}html},
\texttt{/brief{\char`\_}alt{\char`\_}html} qualifiers of \texttt{show
statement}, and \texttt{midi}.  The following commands append to files
previously opened:  the \texttt{/tex} qualifier of \texttt{show proof}
and \texttt{show new{\char`\_}proof}; the \texttt{/tex} and
\texttt{/simple{\char`\_}tex} qualifiers of \texttt{show statement}; the
\texttt{close} commands; and all screen dialog between \texttt{open log}
and \texttt{close log}.

The commands that create new files will not overwrite an existing {\em
filename} but will rename the existing one to {\em
filename}\texttt{{\char`\~}1}.  An existing {\em
filename}\texttt{{\char`\~}1} is renamed {\em
filename}\texttt{{\char`\~}2}, etc.\ up to {\em
filename}\texttt{{\char`\~}9}.  An existing {\em
filename}\texttt{{\char`\~}9} is deleted.  This makes recovery from
mistakes easier but also will clutter up your directory, so occasionally
you may want to clean up (delete) these \texttt{{\char`\~}}$n$ files.


\subsection{\texttt{read} Command}\index{\texttt{read} command}
Syntax:  \texttt{read} {\em file-name} [\texttt{/verify}]

This command will read in a Metamath language source file and any included
files.  Normally it will be the first thing you do when entering Metamath.
Statement syntax is checked, but proof syntax is not checked.
Note that the file name may be enclosed in single or double quotes;
this is useful if the file name contains slashes, as might be the case
under Unix.

If you are getting an ``\texttt{?Expected VERIFY}'' error
when trying to read a Unix file name with slashes, you probably haven't
quoted it.\index{Unix file names}\index{file names!Unix}

If you are prompted for the file name (by pressing {\em return}
 after \texttt{read})
you should {\em not} put quotes around it, even if it is a Unix file name
with slashes.

Optional command qualifier:

    \texttt{/verify} - Verify all proofs as the database is read in.  This
         qualifier will slow down reading in the file.  See \texttt{verify
         proof} for more information on file error-checking.

See also \texttt{erase}.



\subsection{\texttt{write source} Command}\index{\texttt{write source} command}
Syntax:  \texttt{write source} {\em filename}
[\texttt{/rewrap}]
[\texttt{/split}]
% TeX doesn't handle this long line with tt text very well,
% so force a line break here.
[\texttt{/keep\_includes}] {\\}
[\texttt{/no\_versioning}]

This command will write the contents of a Metamath\index{database}
database into a file.\index{source file}

Optional command qualifiers:

\texttt{/rewrap} -
Reformats statements and comments according to the
convention used in the set.mm database.
It unwraps the
lines in the comment before each \texttt{\$a} and \texttt{\$p} statement,
then it
rewraps the line.  You should compare the output to the original
to make sure that the desired effect results; if not, go back to
the original source.  The wrapped line length honors the
\texttt{set width}
parameter currently in effect.  Note:  Text
enclosed in \texttt{<HTML>}...\texttt{</HTML>} tags is not modified by the
\texttt{/rewrap} qualifier.
Proofs themselves are not reformatted;
use \texttt{save proof * / compressed} to do that.
An isolated tilde (\~{}) is always kept on the same line as the following
symbol, so you can find all comment references to a symbol by
searching for \~{} followed by a space and that symbol
(this is useful for finding cross-references).
Incidentally, \texttt{save proof} also honors the \texttt{set width}
parameter currently in effect.

\texttt{/split} - Files included in the source using the expression
\$[ \textit{inclfile} \$] will be
written into separate files instead of being included in a single output
file.  The name of each separately written included file will be the
\textit{inclfile} argument of its inclusion command.

\texttt{/keep\_includes} - If a source file has includes but is written as a
single file by omitting \texttt{/split}, by default the included files will
be deleted (actually just renamed with a \char`\~1 suffix unless
\texttt{/no\_versioning} is specified) to prevent the possibly confusing
source duplication in both the output file and the included file.
The \texttt{/keep\_includes} qualifier will prevent this deletion.

\texttt{/no\_versioning} - Backup files suffixed with \char`\~1 are not created.


\section{Showing Status and Statements}



\subsection{\texttt{show settings} Command}\index{\texttt{show settings} command}
Syntax:  \texttt{show settings}

This command shows the state of various parameters.

\subsection{\texttt{show memory} Command}\index{\texttt{show memory} command}
Syntax:  \texttt{show memory}

This command shows the available memory left.  It is not meaningful
on most modern operating systems,
which have virtual memory.\index{Metamath!memory usage}


\subsection{\texttt{show labels} Command}\index{\texttt{show labels} command}
Syntax:  \texttt{show labels} {\em label-match} [\texttt{/all}]
   [\texttt{/linear}]

This command shows the labels of \texttt{\$a} and \texttt{\$p}
statements that match {\em label-match}.  A \verb$*$ in {label-match}
matches zero or more characters.  For example, \verb$*abc*def$ will match all
labels containing \verb$abc$ and ending with \verb$def$.

Optional command qualifiers:

   \texttt{/all} - Include matches for \texttt{\$e} and \texttt{\$f}
   statement labels.

   \texttt{/linear} - Display only one label per line.  This can be useful for
       building scripts in conjunction with the utilities under the
       \texttt{tools}\index{\texttt{tools} command} command.



\subsection{\texttt{show statement} Command}\index{\texttt{show statement} command}
Syntax:  \texttt{show statement} {\em label-match} [{\em qualifiers} (see below)]

This command provides information about a statement.  Only statements
that have labels (\texttt{\$f}\index{\texttt{\$f} statement},
\texttt{\$e}\index{\texttt{\$e} statement},
\texttt{\$a}\index{\texttt{\$a} statement}, and
\texttt{\$p}\index{\texttt{\$p} statement}) may be specified.
If {\em label-match}
contains wildcard (\verb$*$) characters, all matching statements will be
displayed in the order they occur in the database.

Optional qualifiers (only one qualifier at a time is allowed):

    \texttt{/comment} - This qualifier includes the comment that immediately
       precedes the statement.

    \texttt{/full} - Show complete information about each statement,
       and show all
       statements matching {\em label} (including \texttt{\$e}
       and \texttt{\$f} statements).

    \texttt{/tex} - This qualifier will write the statement information to the
       \LaTeX\ file previously opened with \texttt{open tex}.  See
       Section~\ref{texout}.

    \texttt{/simple{\char`\_}tex} - The same as \texttt{/tex}, except that
       \LaTeX\ macros are not used for formatting equations, allowing easier
       manual edits of the output for slide presentations, etc.

    \texttt{/html}\index{html generation@{\sc html} generation},
       \texttt{/alt{\char`\_}html}, \texttt{/brief{\char`\_}html},
       \texttt{/brief{\char`\_}alt{\char`\_}html} -
       These qualifiers invoke a special mode of
       \texttt{show statement} that
       creates a web page for the statement.  They may not be used with
       any other qualifier.  See Section~\ref{htmlout} or
       \texttt{help html} in the program.


\subsection{\texttt{search} Command}\index{\texttt{search} command}
Syntax:  search {\em label-match}
\texttt{"}{\em symbol-match}{\tt}" [\texttt{/all}] [\texttt{/comments}]
[\texttt{/join}]

This command searches all \texttt{\$a} and \texttt{\$p} statements
matching {\em label-match} for occurrences of {\em symbol-match}.  A
\verb@*@ in {\em label-match} matches any label character.  A \verb@$*@
in {\em symbol-match} matches any sequence of symbols.  The symbols in
{\em symbol-match} must be separated by white space.  The quotes
surrounding {\em symbol-match} may be single or double quotes.  For
example, \texttt{search b}\verb@* "-> $* ch"@ will list all statements
whose labels begin with \texttt{b} and contain the symbols \verb@->@ and
\texttt{ch} surrounding any symbol sequence (including no symbol
sequence).  The wildcards \texttt{?} and \texttt{\$?} are also available
to match individual characters in labels and symbols respectively; see
\texttt{help search} in the Metamath program for details on their usage.

Optional command qualifiers:

    \texttt{/all} - Also search \texttt{\$e} and \texttt{\$f} statements.

    \texttt{/comments} - Search the comment that immediately precedes each
        label-matched statement for {\em symbol-match}.  In this case
        {\em symbol-match} is an arbitrary, non-case-sensitive character
        string.  Quotes around {\em symbol-match} are optional if there
        is no ambiguity.

    \texttt{/join} - In the case of a \texttt{\$a} or \texttt{\$p} statement,
	prepend its \texttt{\$e}
	hypotheses for searching. The
	\texttt{/join} qualifier has no effect in \texttt{/comments} mode.

\section{Displaying and Verifying Proofs}


\subsection{\texttt{show proof} Command}\index{\texttt{show proof} command}
Syntax:  \texttt{show proof} {\em label-match} [{\em qualifiers} (see below)]

This command displays the proof of the specified
\texttt{\$p}\index{\texttt{\$p} statement} statement in various formats.
The {\em label-match} may contain wildcard (\verb@$*@) characters to match
multiple statements.  Without any qualifiers, only the logical steps
will be shown (i.e.\ syntax construction steps will be omitted), in an
indented format.

Most of the time, you will use
    \texttt{show proof} {\em label}
to see just the proof steps corresponding to logical inferences.

Optional command qualifiers:

    \texttt{/essential} - The proof tree is trimmed of all
        \texttt{\$f}\index{\texttt{\$f} statement} hypotheses before
        being displayed.  (This is the default, and it is redundant to
        specify it.)

    \texttt{/all} - the proof tree is not trimmed of all \texttt{\$f} hypotheses before
        being displayed.  \texttt{/essential} and \texttt{/all} are mutually exclusive.

    \texttt{/from{\char`\_}step} {\em step} - The display starts at the specified
        step.  If
        this qualifier is omitted, the display starts at the first step.

    \texttt{/to{\char`\_}step} {\em step} - The display ends at the specified
        step.  If this
        qualifier is omitted, the display ends at the last step.

    \texttt{/tree{\char`\_}depth} {\em number} - Only
         steps at less than the specified proof
        tree depth are displayed.  Sometimes useful for obtaining an overview of
        the proof.

    \texttt{/reverse} - The steps are displayed in reverse order.

    \texttt{/renumber} - When used with \texttt{/essential}, the steps are renumbered
        to correspond only to the essential steps.

    \texttt{/tex} - The proof is converted to \LaTeX\ and\index{latex@{\LaTeX}}
        stored in the file opened
        with \texttt{open tex}.  See Section~\ref{texout} or
        \texttt{help tex} in the program.

    \texttt{/lemmon} - The proof is displayed in a non-indented format known
        as Lemmon style, with explicit previous step number references.
        If this qualifier is omitted, steps are indented in a tree format.

    \texttt{/start{\char`\_}column} {\em number} - Overrides the default column
        (16)
        at which the formula display starts in a Lemmon-style display.  May be
        used only in conjunction with \texttt{/lemmon}.

    \texttt{/normal} - The proof is displayed in normal format suitable for
        inclusion in a Metamath source file.  May not be used with any other
        qualifier.

    \texttt{/compressed} - The proof is displayed in compressed format
        suitable for inclusion in a Metamath source file.  May not be used with
        any other qualifier.

    \texttt{/statement{\char`\_}summary} - Summarizes all statements (like a
        brief \texttt{show statement})
        used by the proof.  It may not be used with any other qualifier
        except \texttt{/essential}.

    \texttt{/detailed{\char`\_}step} {\em step} - Shows the details of what is
        happening at
        a specific proof step.  May not be used with any other qualifier.
        The {\em step} is the step number shown when displaying a
        proof without the \texttt{/renumber} qualifier.


\subsection{\texttt{show usage} Command}\index{\texttt{show usage} command}
Syntax:  \texttt{show usage} {\em label-match} [\texttt{/recursive}]

This command lists the statements whose proofs make direct reference to
the statement specified.

Optional command qualifier:

    \texttt{/recursive} - Also include statements whose proofs ultimately
        depend on the statement specified.



\subsection{\texttt{show trace\_back} Command}\index{\texttt{show
       trace{\char`\_}back} command}
Syntax:  \texttt{show trace{\char`\_}back} {\em label-match} [\texttt{/essential}] [\texttt{/axioms}]
    [\texttt{/tree}] [\texttt{/depth} {\em number}]

This command lists all statements that the proof of the \texttt{\$p}
statement(s) specified by {\em label-match} depends on.

Optional command qualifiers:

    \texttt{/essential} - Restrict the trace-back to \texttt{\$e}
        \index{\texttt{\$e} statement} hypotheses of proof trees.

    \texttt{/axioms} - List only the axioms that the proof ultimately depends on.

    \texttt{/tree} - Display the trace-back in an indented tree format.

    \texttt{/depth} {\em number} - Restrict the \texttt{/tree} trace-back to the
        specified indentation depth.

    \texttt{/count{\char`\_}steps} - Count the number of steps the proof has
       all the way back to axioms.  If \texttt{/essential} is specified,
       expansions of variable-type hypotheses (syntax constructions) are not counted.

\subsection{\texttt{verify proof} Command}\index{\texttt{verify proof} command}
Syntax:  \texttt{verify proof} {\em label-match} [\texttt{/syntax{\char`\_}only}]

This command verifies the proofs of the specified statements.  {\em
label-match} may contain wild card characters (\texttt{*}) to verify
more than one proof; for example \verb/*abc*def/ will match all labels
containing \texttt{abc} and ending with \texttt{def}.
The command \texttt{verify proof *} will verify all proofs in the database.

Optional command qualifier:

    \texttt{/syntax{\char`\_}only} - This qualifier will perform a check of syntax
        and RPN
        stack violations only.  It will not verify that the proof is
        correct.  This qualifier is useful for quickly determining which
        proofs are incomplete (i.e.\ are under development and have \texttt{?}'s
        in them).

{\em Note:} \texttt{read}, followed by \texttt{verify proof *}, will ensure
 the database is free
from errors in the Metamath language but will not check the markup notation
in comments.
You can also check the markup notation by running \texttt{verify markup *}
as discussed in Section~\ref{verifymarkup}; see also the discussion
on generating {\sc HTML} in Section~\ref{htmlout}.

\subsection{\texttt{verify markup} Command}\index{\texttt{verify markup} command}\label{verifymarkup}
Syntax:  \texttt{verify markup} {\em label-match}
[\texttt{/date{\char`\_}skip}]
[\texttt{/top{\char`\_}date{\char`\_}skip}] {\\}
[\texttt{/file{\char`\_}skip}]
[\texttt{/verbose}]

This command checks comment markup and other informal conventions we have
adopted.  It error-checks the latexdef, htmldef, and althtmldef statements
in the \texttt{\$t} statement of a Metamath source file.\index{error checking}
It error-checks any \texttt{`}...\texttt{`},
\texttt{\char`\~}~\textit{label},
and bibliographic markups in statement descriptions.
It checks that
\texttt{\$p} and \texttt{\$a} statements
have the same content when their labels start with
``ax'' and ``ax-'' respectively but are otherwise identical, for example
ax4 and ax-4.
It also verifies the date consistency of ``(Contributed by...),''
``(Revised by...),'' and ``(Proof shortened by...)'' tags in the comment
above each \texttt{\$a} and \texttt{\$p} statement.

Optional command qualifiers:

    \texttt{/date{\char`\_}skip} - This qualifier will
        skip date consistency checking,
        which is usually not required for databases other than
	\texttt{set.mm}.

    \texttt{/top{\char`\_}date{\char`\_}skip} - This qualifier will check date consistency except
        that the version date at the top of the database file will not
        be checked.  Only one of
        \texttt{/date{\char`\_}skip} and
        \texttt{/top{\char`\_}date{\char`\_}skip} may be
        specified.

    \texttt{/file{\char`\_}skip} - This qualifier will skip checks that require
        external files to be present, such as checking GIF existence and
        bibliographic links to mmset.html or equivalent.  It is useful
        for doing a quick check from a directory without these files.

    \texttt{/verbose} - Provides more information.  Currently it provides a list
        of axXXX vs. ax-XXX matches.

\subsection{\texttt{save proof} Command}\index{\texttt{save proof} command}
Syntax:  \texttt{save proof} {\em label-match} [\texttt{/normal}]
   [\texttt{/compressed}]

The \texttt{save proof} command will reformat a proof in one of two formats and
replace the existing proof in the source buffer\index{source
buffer}.  It is useful for
converting between proof formats.  Note that a proof will not be
permanently saved until a \texttt{write source} command is issued.

Optional command qualifiers:

    \texttt{/normal} - The proof is saved in the normal format (i.e., as a
        sequence
        of labels, which is the defined format of the basic Metamath
        language).\index{basic language}  This is the default format that
        is used if a qualifier
        is omitted.

    \texttt{/compressed} - The proof is saved in the compressed format which
        reduces storage requirements for a database.
        See Appendix~\ref{compressed}.




\section{Creating Proofs}\label{pfcommands}\index{Proof Assistant}

Before using the Proof Assistant, you must add a \texttt{\$p} to your
source file (using a text editor) containing the statement you want to
prove.  Its proof should consist of a single \texttt{?}, meaning
``unknown step.''  Example:
\begin{verbatim}
equid $p x = x $= ? $.
\end{verbatim}

To enter the Proof assistant, type \texttt{prove} {\em label}, e.g.
\texttt{prove equid}.  Metamath will respond with the \texttt{MM-PA>}
prompt.

Proofs are created working backwards from the statement being proved,
primarily using a series of \texttt{assign} commands.  A proof is
complete when all steps are assigned to statements and all steps are
unified and completely known.  During the creation of a proof, Metamath
will allow only operations that are legal based on what is known up to
that point.  For example, it will not allow an \texttt{assign} of a
statement that cannot be unified with the unknown proof step being
assigned.

{\em Important:}
The Proof Assistant is
{\em not} a tool to help you discover proofs.  It is just a tool to help
you add them to the database.  For a tutorial read
Section~\ref{frstprf}.
To practice using the Proof Assistant, you may
want to \texttt{prove} an existing theorem, then delete all steps with
\texttt{delete all}, then re-create it with the Proof Assistant while
looking at its proof display (before deletion).
You might want to figure out your first few proofs completely
and write them down by hand, before using the Proof Assistant, though
not everyone finds that effective.

{\em Important:}
The \texttt{undo} command if very helpful when entering a proof, because
it allows you to undo a previously-entered step.
In addition, we suggest that you
keep track of your work with a log file (\texttt{open
log}) and save it frequently (\texttt{save new{\char`\_}proof},
\texttt{write source}).
You can use \texttt{delete} to reverse an \texttt{assign}.
You can also do \texttt{delete floating{\char`\_}hypotheses}, then
\texttt{initialize all}, then \texttt{unify all /interactive} to
reinitialize bad unifications made accidentally or by bad
\texttt{assign}s.  You cannot reverse a \texttt{delete} except by
a relevant \texttt{undo} or using
\texttt{exit /force} then reentering the Proof Assistant to recover from
the last \texttt{save new{\char`\_}proof}.

The following commands are available in the Proof Assistant (at the
\texttt{MM-PA>} prompt) to help you create your proof.  See the
individual commands for more detail.

\begin{itemize}
\item[]
    \texttt{show new{\char`\_}proof} [\texttt{/all},...] - Displays the
        proof in progress.  You will use this command a lot; see \texttt{help
        show new{\char`\_}proof} to become familiar with its qualifiers.  The
        qualifiers \texttt{/unknown} and \texttt{/not{\char`\_}unified} are
        useful for seeing the work remaining to be done.  The combination
        \texttt{/all/unknown} is useful for identifying dummy variables that must be
        assigned, or attempts to use illegal syntax, when \texttt{improve all}
        is unable to complete the syntax constructions.  Unknown variables are
        shown as \texttt{\$1}, \texttt{\$2},...
\item[]
    \texttt{assign} {\em step} {\em label} - Assigns an unknown {\em step}
        number with the statement
        specified by {\em label}.
\item[]
    \texttt{let variable} {\em variable}
        \texttt{= "}{\em symbol sequence}\texttt{"}
          - Forces a symbol
        sequence to replace an unknown variable (such as \texttt{\$1}) in a proof.
        It is useful
        for helping difficult unifications, and it is necessary when you have
        dummy variables that eventually must be assigned a name.
\item[]
    \texttt{let step} {\em step} \texttt{= "}{\em symbol sequence}\texttt{"} -
          Forces a symbol sequence
        to replace the contents of a proof step, provided it can be
        unified with the existing step contents.  (I rarely use this.)
\item[]
    \texttt{unify step} {\em step} (or \texttt{unify all}) - Unifies
        the source and target of
        a step.  If you specify a specific step, you will be prompted
        to select among the unifications that are possible.  If you
        specify \texttt{all}, all steps with unique unifications, but only
        those steps, will be
        unified.  \texttt{unify all /interactive} goes through all non-unified
        steps.
\item[]
    \texttt{initialize} {\em step} (or \texttt{all}) - De-unifies the target and source of
        a step (or all steps), as well as the hypotheses of the source,
        and makes all variables in the source unknown.  Useful to recover from
        an \texttt{assign} or \texttt{let} mistake that
        resulted in incorrect unifications.
\item[]
    \texttt{delete} {\em step} (or \texttt{all} or \texttt{floating{\char`\_}hypotheses}) -
        Deletes the specified
        step(s).  \texttt{delete floating{\char`\_}hypotheses}, then \texttt{initialize all}, then
        \texttt{unify all /interactive} is useful for recovering from mistakes
        where incorrect unifications assigned wrong math symbol strings to
        variables.
\item[]
    \texttt{improve} {\em step} (or \texttt{all}) -
      Automatically creates a proof for steps (with no unknown variables)
      whose proof requires no statements with \texttt{\$e} hypotheses.  Useful
      for filling in proofs of \texttt{\$f} hypotheses.  The \texttt{/depth}
      qualifier will also try statements whose \texttt{\$e} hypotheses contain
      no new variables.  {\em Warning:} Save your work (with \texttt{save
      new{\char`\_}proof} then \texttt{write source}) before using
      \texttt{/depth = 2} or greater, since the search time grows
      exponentially and may never terminate in a reasonable time, and you
      cannot interrupt the search.  I have found that it is rare for
      \texttt{/depth = 3} or greater to be useful.
 \item[]
    \texttt{save new{\char`\_}proof} - Saves the proof in progress in the program's
        internal database buffer.  To save it permanently into the database file,
        use \texttt{write source} after
        \texttt{save new{\char`\_}proof}.  To revert to the last
        \texttt{save new{\char`\_}proof},
        \texttt{exit /force} from the Proof Assistant then re-enter the Proof
        Assistant.
 \item[]
    \texttt{match step} {\em step} (or \texttt{match all}) - Shows what
        statements are
        possibilities for the \texttt{assign} statement. (This command
        is not very
        useful in its present form and hopefully will be improved
        eventually.  In the meantime, use the \texttt{search} statement for
        candidates matching specific math token combinations.)
 \item[]
 \texttt{minimize{\char`\_}with}\index{\texttt{minimize{\char`\_}with} command}
% 3/10/07 Note: line-breaking the above results in duplicate index entries
     - After a proof is complete, this command will attempt
        to match other database theorems to the proof to see if the proof
        size can be reduced as a result.  See \texttt{help
        minimize{\char`\_}with} in the
        Metamath program for its usage.
 \item[]
 \texttt{undo}\index{\texttt{undo} command}
    - Undo the effect of a proof-changing command (all but the
      \texttt{show} and \texttt{save} commands above).
 \item[]
 \texttt{redo}\index{\texttt{redo} command}
    - Reverse the previous \texttt{undo}.
\end{itemize}

The following commands set parameters that may be relevant to your proof.
Consult the individual \texttt{help set}... commands.
\begin{itemize}
   \item[] \texttt{set unification{\char`\_}timeout}
 \item[]
    \texttt{set search{\char`\_}limit}
  \item[]
    \texttt{set empty{\char`\_}substitution} - note that default is \texttt{off}
\end{itemize}

Type \texttt{exit} to exit the \texttt{MM-PA>}
 prompt and get back to the \texttt{MM>} prompt.
Another \texttt{exit} will then get you out of Metamath.



\subsection{\texttt{prove} Command}\index{\texttt{prove} command}
Syntax:  \texttt{prove} {\em label}

This command will enter the Proof Assistant\index{Proof Assistant}, which will
allow you to create or edit the proof of the specified statement.
The command-line prompt will change from \texttt{MM>} to \texttt{MM-PA>}.

Note:  In the present version (0.177) of
Metamath\index{Metamath!limitations of version 0.177}, the Proof
Assistant does not verify that \texttt{\$d}\index{\texttt{\$d}
statement} restrictions are met as a proof is being built.  After you
have completed a proof, you should type \texttt{save new{\char`\_}proof}
followed by \texttt{verify proof} {\em label} (where {\em label} is the
statement you are proving with the \texttt{prove} command) to verify the
\texttt{\$d} restrictions.

See also: \texttt{exit}

\subsection{\texttt{set unification\_timeout} Command}\index{\texttt{set
unification{\char`\_}timeout} command}
Syntax:  \texttt{set unification{\char`\_}timeout} {\em number}

(This command is available outside the Proof Assistant but affects the
Proof Assistant\index{Proof Assistant} only.)

Sometimes the Proof Assistant will inform you that a unification
time-out occurred.  This may happen when you try to \texttt{unify}
formulas with many temporary variables\index{temporary variable}
(\texttt{\$1}, \texttt{\$2}, etc.), since the time to compute all possible
unifications may grow exponentially with the number of variables.  If
you want Metamath to try harder (and you're willing to wait longer) you
may increase this parameter.  \texttt{show settings} will show you the
current value.

\subsection{\texttt{set empty\_substitution} Command}\index{\texttt{set
empty{\char`\_}substitution} command}
% These long names can't break well in narrow mode, and even "sloppy"
% is not enough. Work around this by not demanding justification.
\begin{flushleft}
Syntax:  \texttt{set empty{\char`\_}substitution on} or \texttt{set
empty{\char`\_}substitution off}
\end{flushleft}

(This command is available outside the Proof Assistant but affects the
Proof Assistant\index{Proof Assistant} only.)

The Metamath language allows variables to be
substituted\index{substitution!variable}\index{variable substitution}
with empty symbol sequences\index{empty substitution}.  However, in many
formal systems\index{formal system} this will never happen in a valid
proof.  Allowing for this possibility increases the likelihood of
ambiguous unifications\index{ambiguous
unification}\index{unification!ambiguous} during proof creation.
The default is that
empty substitutions are not allowed; for formal systems requiring them,
you must \texttt{set empty{\char`\_}substitution on}.
(An example where this must be \texttt{on}
would be a system that implements a Deduction Rule and in
which deductions from empty assumption lists would be permissible.  The
MIU-system\index{MIU-system} described in Appendix~\ref{MIU} is another
example.)
Note that empty substitutions are
always permissible in proof verification (VERIFY PROOF...) outside the
Proof Assistant.  (See the MIU system in the Metamath book for an example
of a system needing empty substitutions; another example would be a
system that implements a Deduction Rule and in which deductions from
empty assumption lists would be permissible.)

It is better to leave this \texttt{off} when working with \texttt{set.mm}.
Note that this command does not affect the way proofs are verified with
the \texttt{verify proof} command.  Outside of the Proof Assistant,
substitution of empty sequences for math symbols is always allowed.

\subsection{\texttt{set search\_limit} Command}\index{\texttt{set
search{\char`\_}limit} command} Syntax:  \texttt{set search{\char`\_}limit} {\em
number}

(This command is available outside the Proof Assistant but affects the
Proof Assistant\index{Proof Assistant} only.)

This command sets a parameter that determines when the \texttt{improve} command
in Proof Assistant mode gives up.  If you want \texttt{improve} to search harder,
you may increase it.  The \texttt{show settings} command tells you its current
value.


\subsection{\texttt{show new\_proof} Command}\index{\texttt{show
new{\char`\_}proof} command}
Syntax:  \texttt{show new{\char`\_}proof} [{\em
qualifiers} (see below)]

This command (available only in Proof Assistant mode) displays the proof
in progress.  It is identical to the \texttt{show proof} command, except that
there is no statement argument (since it is the statement being proved) and
the following qualifiers are not available:

    \texttt{/statement{\char`\_}summary}

    \texttt{/detailed{\char`\_}step}

Also, the following additional qualifiers are available:

    \texttt{/unknown} - Shows only steps that have no statement assigned.

    \texttt{/not{\char`\_}unified} - Shows only steps that have not been unified.

Note that \texttt{/essential}, \texttt{/depth}, \texttt{/unknown}, and
\texttt{/not{\char`\_}unified} may be
used in any combination; each of them effectively filters out additional
steps from the proof display.

See also:  \texttt{show proof}






\subsection{\texttt{assign} Command}\index{\texttt{assign} command}
Syntax:   \texttt{assign} {\em step} {\em label} [\texttt{/no{\char`\_}unify}]

   and:   \texttt{assign first} {\em label}

   and:   \texttt{assign last} {\em label}


This command, available in the Proof Assistant only, assigns an unknown
step (one with \texttt{?} in the \texttt{show new{\char`\_}proof}
listing) with the statement specified by {\em label}.  The assignment
will not be allowed if the statement cannot be unified with the step.

If \texttt{last} is specified instead of {\em step} number, the last
step that is shown by \texttt{show new{\char`\_}proof /unknown} will be
used.  This can be useful for building a proof with a command file (see
\texttt{help submit}).  It also makes building proofs faster when you know
the assignment for the last step.

If \texttt{first} is specified instead of {\em step} number, the first
step that is shown by \texttt{show new{\char`\_}proof /unknown} will be
used.

If {\em step} is zero or negative, the -{\em step}th from last unknown
step, as shown by \texttt{show new{\char`\_}proof /unknown}, will be
used.  \texttt{assign -1} {\em label} will assign the penultimate
unknown step, \texttt{assign -2} {\em label} the antepenultimate, and
\texttt{assign 0} {\em label} is the same as \texttt{assign last} {\em
label}.

Optional command qualifier:

    \texttt{/no{\char`\_}unify} - do not prompt user to select a unification if there is
        more than one possibility.  This is useful for noninteractive
        command files.  Later, the user can \texttt{unify all /interactive}.
        (The assignment will still be automatically unified if there is only
        one possibility and will be refused if unification is not possible.)



\subsection{\texttt{match} Command}\index{\texttt{match} command}
Syntax:  \texttt{match step} {\em step} [\texttt{/max{\char`\_}essential{\char`\_}hyp}
{\em number}]

    and:  \texttt{match all} [\texttt{/essential}]
          [\texttt{/max{\char`\_}essential{\char`\_}hyp} {\em number}]

This command, available in the Proof Assistant only, shows what
statements can be unified with the specified step(s).  {\em Note:} In
its current form, this command is not very useful because of the large
number of matches it reports.
It may be enhanced in the future.  In the meantime, the \texttt{search}
command can often provide finer control over locating theorems of interest.

Optional command qualifiers:

    \texttt{/max{\char`\_}essential{\char`\_}hyp} {\em number} - filters out
        of the list any statements
        with more than the specified number of
        \texttt{\$e}\index{\texttt{\$e} statement} hypotheses.

    \texttt{/essential{\char`\_}only} - in the \texttt{match all} statement, only
        the steps that
        would be listed in the \texttt{show new{\char`\_}proof /essential} display are
        matched.



\subsection{\texttt{let} Command}\index{\texttt{let} command}
Syntax: \texttt{let variable} {\em variable} = \verb/"/{\em symbol-sequence}\verb/"/

  and: \texttt{let step} {\em step} = \verb/"/{\em symbol-sequence}\verb/"/

These commands, available in the Proof Assistant\index{Proof Assistant}
only, assign a temporary variable\index{temporary variable} or unknown
step with a specific symbol sequence.  They are useful in the middle of
creating a proof, when you know what should be in the proof step but the
unification algorithm doesn't yet have enough information to completely
specify the temporary variables.  A ``temporary variable'' is one that
has the form \texttt{\$}{\em nn} in the proof display, such as
\texttt{\$1}, \texttt{\$2}, etc.  The {\em symbol-sequence} may contain
other unknown variables if desired.  Examples:

    \verb/let variable $32 = "A = B"/

    \verb/let variable $32 = "A = $35"/

    \verb/let step 10 = '|- x = x'/

    \verb/let step -2 = "|- ( $7 = ph )"/

Any symbol sequence will be accepted for the \texttt{let variable}
command.  Only those symbol sequences that can be unified with the step
will be accepted for \texttt{let step}.

The \texttt{let} commands ``zap'' the proof with information that can
only be verified when the proof is built up further.  If you make an
error, the command sequence \texttt{delete
floating{\char`\_}hypotheses}, \texttt{initialize all}, and
\texttt{unify all /interactive} will undo a bad \texttt{let} assignment.

If {\em step} is zero or negative, the -{\em step}th from last unknown
step, as shown by \texttt{show new{\char`\_}proof /unknown}, will be
used.  The command \texttt{let step 0} = \verb/"/{\em
symbol-sequence}\verb/"/ will use the last unknown step, \texttt{let
step -1} = \verb/"/{\em symbol-sequence}\verb/"/ the penultimate, etc.
If {\em step} is positive, \texttt{let step} may be used to assign known
(in the sense of having previously been assigned a label with
\texttt{assign}) as well as unknown steps.

Either single or double quotes can surround the {\em symbol-sequence} as
long as they are different from any quotes inside a {\em
symbol-sequence}.  If {\em symbol-sequence} contains both kinds of
quotes, see the instructions at the end of \texttt{help let} in the
Metamath program.


\subsection{\texttt{unify} Command}\index{\texttt{unify} command}
Syntax:  \texttt{unify step} {\em step}

      and:   \texttt{unify all} [\texttt{/interactive}]

These commands, available in the Proof Assistant only, unify the source
and target of the specified step(s). If you specify a specific step, you
will be prompted to select among the unifications that are possible.  If
you specify \texttt{all}, only those steps with unique unifications will be
unified.

Optional command qualifier for \texttt{unify all}:

    \texttt{/interactive} - You will be prompted to select among the
        unifications
        that are possible for any steps that do not have unique
        unifications.  (Otherwise \texttt{unify all} will bypass these.)

See also \texttt{set unification{\char`\_}timeout}.  The default is
100000, but increasing it to 1000000 can help difficult cases.  Manually
assigning some or all of the unknown variables with the \texttt{let
variable} command also helps difficult cases.



\subsection{\texttt{initialize} Command}\index{\texttt{initialize} command}
Syntax:  \texttt{initialize step} {\em step}

    and: \texttt{initialize all}

These commands, available in the Proof Assistant\index{Proof Assistant}
only, ``de-unify'' the target and source of a step (or all steps), as
well as the hypotheses of the source, and makes all variables in the
source and the source's hypotheses unknown.  This command is useful to
help recover from incorrect unifications that resulted from an incorrect
\texttt{assign}, \texttt{let}, or unification choice.  Part or all of
the command sequence \texttt{delete floating{\char`\_}hypotheses},
\texttt{initialize all}, and \texttt{unify all /interactive} will recover
from incorrect unifications.

See also:  \texttt{unify} and \texttt{delete}



\subsection{\texttt{delete} Command}\index{\texttt{delete} command}
Syntax:  \texttt{delete step} {\em step}

   and:      \texttt{delete all} -- {\em Warning: dangerous!}

   and:      \texttt{delete floating{\char`\_}hypotheses}

These commands are available in the Proof Assistant only.  The
\texttt{delete step} command deletes the proof tree section that
branches off of the specified step and makes the step become unknown.
\texttt{delete all} is equivalent to \texttt{delete step} {\em step}
where {\em step} is the last step in the proof (i.e.\ the beginning of
the proof tree).

In most cases the \texttt{undo} command is the best way to undo
a previous step.
An alternative is to salvage your last \texttt{save
new{\char`\_}proof} by exiting and reentering the Proof Assistant.
For this to work, keep a log file open to record your work
and to do \texttt{save new{\char`\_}proof} frequently, especially before
\texttt{delete}.

\texttt{delete floating{\char`\_}hypotheses} will delete all sections of
the proof that branch off of \texttt{\$f}\index{\texttt{\$f} statement}
statements.  It is sometimes useful to do this before an
\texttt{initialize} command to recover from an error.  Note that once a
proof step with a \texttt{\$f} hypothesis as the target is completely
known, the \texttt{improve} command can usually fill in the proof for
that step.  Unlike the deletion of logical steps, \texttt{delete
floating{\char`\_}hypotheses} is a relatively safe command that is
usually easy to recover from.



\subsection{\texttt{improve} Command}\index{\texttt{improve} command}
\label{improve}
Syntax:  \texttt{improve} {\em step} [\texttt{/depth} {\em number}]
                                               [\texttt{/no{\char`\_}distinct}]

   and:   \texttt{improve first} [\texttt{/depth} {\em number}]
                                              [\texttt{/no{\char`\_}distinct}]

   and:   \texttt{improve last} [\texttt{/depth} {\em number}]
                                              [\texttt{/no{\char`\_}distinct}]

   and:   \texttt{improve all} [\texttt{/depth} {\em number}]
                                              [\texttt{/no{\char`\_}distinct}]

These commands, available in the Proof Assistant\index{Proof Assistant}
only, try to find proofs automatically for unknown steps whose symbol
sequences are completely known.  They are primarily useful for filling in
proofs of \texttt{\$f}\index{\texttt{\$f} statement} hypotheses.  The
search will be restricted to statements having no
\texttt{\$e}\index{\texttt{\$e} statement} hypotheses.

\begin{sloppypar} % narrow
Note:  If memory is limited, \texttt{improve all} on a large proof may
overflow memory.  If you use \texttt{set unification{\char`\_}timeout 1}
before \texttt{improve all}, there will usually be sufficient
improvement to easily recover and completely \texttt{improve} the proof
later on a larger computer.  Warning:  Once memory has overflowed, there
is no recovery.  If in doubt, save the intermediate proof (\texttt{save
new{\char`\_}proof} then \texttt{write source}) before \texttt{improve
all}.
\end{sloppypar}

If \texttt{last} is specified instead of {\em step} number, the last
step that is shown by \texttt{show new{\char`\_}proof /unknown} will be
used.

If \texttt{first} is specified instead of {\em step} number, the first
step that is shown by \texttt{show new{\char`\_}proof /unknown} will be
used.

If {\em step} is zero or negative, the -{\em step}th from last unknown
step, as shown by \texttt{show new{\char`\_}proof /unknown}, will be
used.  \texttt{improve -1} will use the penultimate
unknown step, \texttt{improve -2} {\em label} the antepenultimate, and
\texttt{improve 0} is the same as \texttt{improve last}.

Optional command qualifier:

    \texttt{/depth} {\em number} - This qualifier will cause the search
        to include
        statements with \texttt{\$e} hypotheses (but no new variables in
        the \texttt{\$e}
        hypotheses), provided that the backtracking has not exceeded the
        specified depth. {\em Warning:}  Try \texttt{/depth 1},
        then \texttt{2}, then \texttt{3}, etc.
        in sequence because of possible exponential blowups.  Save your
        work before trying \texttt{/depth} greater than \texttt{1}!

    \texttt{/no{\char`\_}distinct} - Skip trial statements that have
        \texttt{\$d}\index{\texttt{\$d} statement} requirements.
        This qualifier will prevent assignments that might violate \texttt{\$d}
        requirements but it also could miss possible legal assignments.

See also: \texttt{set search{\char`\_}limit}

\subsection{\texttt{save new\_proof} Command}\index{\texttt{save
new{\char`\_}proof} command}
Syntax:  \texttt{save new{\char`\_}proof} {\em label} [\texttt{/normal}]
   [\texttt{/compressed}]

The \texttt{save new{\char`\_}proof} command is available in the Proof
Assistant only.  It saves the proof in progress in the source
buffer\index{source buffer}.  \texttt{save new{\char`\_}proof} may be
used to save a completed proof, or it may be used to save a proof in
progress in order to work on it later.  If an incomplete proof is saved,
any user assignments with \texttt{let step} or \texttt{let variable}
will be lost, as will any ambiguous unifications\index{ambiguous
unification}\index{unification!ambiguous} that were resolved manually.
To help make recovery easier, it can be helpful to \texttt{improve all}
before \texttt{save new{\char`\_}proof} so that the incomplete proof
will have as much information as possible.

Note that the proof will not be permanently saved until a \texttt{write
source} command is issued.

Optional command qualifiers:

    \texttt{/normal} - The proof is saved in the normal format (i.e., as a
        sequence of labels, which is the defined format of the basic Metamath
        language).\index{basic language}  This is the default format that
        is used if a qualifier is omitted.

    \texttt{/compressed} - The proof is saved in the compressed format, which
        reduces storage requirements for a database.
        (See Appendix~\ref{compressed}.)


\section{Creating \LaTeX\ Output}\label{texout}\index{latex@{\LaTeX}}

You can generate \LaTeX\ output given the
information in a database.
The database must already include the necessary typesetting information
(see section \ref{tcomment} for how to provide this information).

The \texttt{show statement} and \texttt{show proof} commands each have a
special \texttt{/tex} command qualifier that produces \LaTeX\ output.
(The \texttt{show statement} command also has the
\texttt{/simple{\char`\_}tex} qualifier for output that is easier to
edit by hand.)  Before you can use them, you must open a \LaTeX\ file to
which to send their output.  A typical complete session will use this
sequence of Metamath commands:

\begin{verbatim}
read set.mm
open tex example.tex
show statement a1i /tex
show proof a1i /all/lemmon/renumber/tex
show statement uneq2 /tex
show proof uneq2 /all/lemmon/renumber/tex
close tex
\end{verbatim}

See Section~\ref{mathcomments} for information on comment markup and
Appendix~\ref{ASCII} for information on how math symbol translation is
specified.

To format and print the \LaTeX\ source, you will need the \LaTeX\
program, which is standard on most Linux installations and available for
Windows.  On Linux, in order to create a {\sc pdf} file, you will
typically type at the shell prompt
\begin{verbatim}
$ pdflatex example.tex
\end{verbatim}

\subsection{\texttt{open tex} Command}\index{\texttt{open tex} command}
Syntax:  \texttt{open tex} {\em file-name} [\texttt{/no{\char`\_}header}]

This command opens a file for writing \LaTeX\
source\index{latex@{\LaTeX}} and writes a \LaTeX\ header to the file.
\LaTeX\ source can be written with the \texttt{show proof}, \texttt{show
new{\char`\_}proof}, and \texttt{show statement} commands using the
\texttt{/tex} qualifier.

The mapping to \LaTeX\ symbols is defined in a special comment
containing a \texttt{\$t} token, described in Appendix~\ref{ASCII}.

There is an optional command qualifier:

    \texttt{/no{\char`\_}header} - This qualifier prevents a standard
        \LaTeX\ header and trailer
        from being included with the output \LaTeX\ code.


\subsection{\texttt{close tex} Command}\index{\texttt{close tex} command}
Syntax:  \texttt{close tex}

This command writes a trailer to any \LaTeX\ file\index{latex@{\LaTeX}}
that was opened with \texttt{open tex} (unless
\texttt{/no{\char`\_}header} was used with \texttt{open tex}) and closes
the \LaTeX\ file.


\section{Creating {\sc HTML} Output}\label{htmlout}

You can generate {\sc html} web pages given the
information in a database.
The database must already include the necessary typesetting information
(see section \ref{tcomment} for how to provide this information).
The ability to produce {\sc html} web pages was added in Metamath version
0.07.30.

To create an {\sc html} output file(s) for \texttt{\$a} or \texttt{\$p}
statement(s), use
\begin{quote}
    \texttt{show statement} {\em label-match} \texttt{/html}
\end{quote}
The output file will be named {\em label-match}\texttt{.html}
for each match.  When {\em
label-match} has wildcard (\texttt{*}) characters, all statements with
matching labels will have {\sc html} files produced for them.  Also,
when {\em label-match} has a wildcard (\texttt{*}) character, two additional
files, \texttt{mmdefinitions.html} and \texttt{mmascii.html} will be
produced.  To produce {\em only} these two additional files, you can use
\texttt{?*}, which will not match any statement label, in place of {\em
label-match}.

There are three other qualifiers for \texttt{show statement} that also
generate {\sc HTML} code.  These are \texttt{/alt{\char`\_}html},
\texttt{/brief{\char`\_}html}, and
\texttt{/brief{\char`\_}alt{\char`\_}html}, and are described in the
next section.

The command
\begin{quote}
    \texttt{show statement} {\em label-match} \texttt{/alt{\char`\_}html}
\end{quote}
does the same as \texttt{show statement} {\em label-match} \texttt{/html},
except that the {\sc html} code for the symbols is taken from
\texttt{althtmldef} statements instead of \texttt{htmldef} statements in
the \texttt{\$t} comment.

The command
\begin{verbatim}
show statement * /brief_html
\end{verbatim}
invokes a special mode that just produces definition and theorem lists
accompanied by their symbol strings, in a format suitable for copying and
pasting into another web page (such as the tutorial pages on the
Metamath web site).

Finally, the command
\begin{verbatim}
show statement * /brief_alt_html
\end{verbatim}
does the same as \texttt{show statement * / brief{\char`\_}html}
for the alternate {\sc html}
symbol representation.

A statement's comment can include a special notation that provides a
certain amount of control over the {\sc HTML} version of the comment.  See
Section~\ref{mathcomments} (p.~\pageref{mathcomments}) for the comment
markup features.

The \texttt{write theorem{\char`\_}list} and \texttt{write bibliography}
commands, which are described below, provide as a side effect complete
error checking for all of the features described in this
section.\index{error checking}

\subsection{\texttt{write theorem\_list}
Command}\index{\texttt{write theorem{\char`\_}list} command}
Syntax:  \texttt{write theorem{\char`\_}list}
[\texttt{/theorems{\char`\_}per{\char`\_}page} {\em number}]

This command writes a list of all of the \texttt{\$a} and \texttt{\$p}
statements in the database into a web page file
 called \texttt{mmtheorems.html}.
When additional files are needed, they are called
\texttt{mmtheorems2.html}, \texttt{mmtheorems3.html}, etc.

Optional command qualifier:

    \texttt{/theorems{\char`\_}per{\char`\_}page} {\em number} -
 This qualifier specifies the number of statements to
        write per web page.  The default is 100.

{\em Note:} In version 0.177\index{Metamath!limitations of version
0.177} of Metamath, the ``Nearby theorems'' links on the individual
web pages presuppose 100 theorems per page when linking to the theorem
list pages.  Therefore the \texttt{/theorems{\char`\_}per{\char`\_}page}
qualifier, if it specifies a number other than 100, will cause the
individual web pages to be out of sync and should not be used to
generate the main theorem list for the web site.  This may be
fixed in a future version.


\subsection{\texttt{write bibliography}\label{wrbib}
Command}\index{\texttt{write bibliography} command}
Syntax:  \texttt{write bibliography} {\em filename}

This command reads an existing {\sc html} bibliographic cross-reference
file, normally called \texttt{mmbiblio.html}, and updates it per the
bibliographic links in the database comments.  The file is updated
between the {\sc html} comment lines \texttt{<!--
{\char`\#}START{\char`\#} -->} and \texttt{<!-- {\char`\#}END{\char`\#}
-->}.  The original input file is renamed to {\em
filename}\texttt{{\char`\~}1}.

A bibliographic reference is indicated with the reference name
in brackets, such as  \texttt{Theorem 3.1 of
[Monk] p.\ 22}.
See Section~\ref{htmlout} (p.~\pageref{htmlout}) for
syntax details.


\subsection{\texttt{write recent\_additions}
Command}\index{\texttt{write recent{\char`\_}additions} command}
Syntax:  \texttt{write recent{\char`\_}additions} {\em filename}
[\texttt{/limit} {\em number}]

This command reads an existing ``Recent Additions'' {\sc html} file,
normally called \texttt{mmrecent.html}, and updates it with the
descriptions of the most recently added theorems to the database.
 The file is updated between
the {\sc html} comment lines \texttt{<!-- {\char`\#}START{\char`\#} -->}
and \texttt{<!-- {\char`\#}END{\char`\#} -->}.  The original input file
is renamed to {\em filename}\texttt{{\char`\~}1}.

Optional command qualifier:

    \texttt{/limit} {\em number} -
 This qualifier specifies the number of most recent theorems to
   write to the output file.  The default is 100.


\section{Text File Utilities}

\subsection{\texttt{tools} Command}\index{\texttt{tools} command}
Syntax:  \texttt{tools}

This command invokes an easy-to-use, general purpose utility for
manipulating the contents of {\sc ascii} text files.  Upon typing
\texttt{tools}, the command-line prompt will change to \texttt{TOOLS>}
until you type \texttt{exit}.  The \texttt{tools} commands can be used
to perform simple, global edits on an input/output file,
such as making a character string substitution on each line, adding a
string to each line, and so on.  A typical use of this utility is
to build a \texttt{submit} input file to perform a common operation on a
list of statements obtained from \texttt{show label} or \texttt{show
usage}.

The actions of most of the \texttt{tools} commands can also be
performed with equivalent (and more powerful) Unix shell commands, and
some users may find those more efficient.  But for Windows users or
users not comfortable with Unix, \texttt{tools} provides an
easy-to-learn alternative that is adequate for most of the
script-building tasks needed to use the Metamath program effectively.

\subsection{\texttt{help} Command (in \texttt{tools})}
Syntax:  \texttt{help}

The \texttt{help} command lists the commands available in the
\texttt{tools} utility, along with a brief description.  Each command,
in turn, has its own help, such as \texttt{help add}.  As with
Metamath's \texttt{MM>} prompt, a complete command can be entered at
once, or just the command word can be typed, causing the program to
prompt for each argument.

\vskip 1ex
\noindent Line-by-line editing commands:

  \texttt{add} - Add a specified string to each line in a file.

  \texttt{clean} - Trim spaces and tabs on each line in a file; convert
         characters.

  \texttt{delete} - Delete a section of each line in a file.

  \texttt{insert} - Insert a string at a specified column in each line of
        a file.

  \texttt{substitute} - Make a simple substitution on each line of the file.

  \texttt{tag} - Like \texttt{add}, but restricted to a range of lines.

  \texttt{swap} - Swap the two halves of each line in a file.

\vskip 1ex
\noindent Other file-processing commands:

  \texttt{break} - Break up (tokenize) a file into a list of tokens (one per
        line).

  \texttt{build} - Build a file with multiple tokens per line from a list.

  \texttt{count} - Count the occurrences in a file of a specified string.

  \texttt{number} - Create a list of numbers.

  \texttt{parallel} - Put two files in parallel.

  \texttt{reverse} - Reverse the order of the lines in a file.

  \texttt{right} - Right-justify lines in a file (useful before sorting
         numbers).

%  \texttt{tag} - Tag edit updates in a program for revision control.

  \texttt{sort} - Sort the lines in a file with key starting at
         specified string.

  \texttt{match} - Extract lines containing (or not) a specified string.

  \texttt{unduplicate} - Eliminate duplicate occurrences of lines in a file.

  \texttt{duplicate} - Extract first occurrence of any line occurring
         more than

   \ \ \    once in a file, discarding lines occurring exactly once.

  \texttt{unique} - Extract lines occurring exactly once in a file.

  \texttt{type} (10 lines) - Display the first few lines in a file.
                  Similar to Unix \texttt{head}.

  \texttt{copy} - Similar to Unix \texttt{cat} but safe (same input
         and output file allowed).

  \texttt{submit} - Run a script containing \texttt{tools} commands.

\vskip 1ex

\noindent Note:
  \texttt{unduplicate}, \texttt{duplicate}, and \texttt{unique} also
 sort the lines as a side effect.


\subsection{Using \texttt{tools} to Build Metamath \texttt{submit}
Scripts}

The \texttt{break} command is typically used to break up a series of
statement labels, such as the output of Metamath's \texttt{show usage},
into one label per line.  The other \texttt{tools} commands can then be
used to add strings before and after each statement label to specify
commands to be performed on the statement.  The \texttt{parallel}
command is useful when a statement label must be mentioned more than
once on a line.

Very often a \texttt{submit} script for Metamath will require multiple
command lines for each statement being processed.  For example, you may
want to enter the Proof Assistant, \texttt{minimize{\char`\_}with} your
latest theorem, \texttt{save} the new proof, and \texttt{exit} the Proof
Assistant.  To accomplish this, you can build a file with these four
commands for each statement on a single line, separating each command
with a designated character such as \texttt{@}.  Then at the end you can
\texttt{substitute} each \texttt{@} with \texttt{{\char`\\}n} to break
up the lines into individual command lines (see \texttt{help
substitute}).


\subsection{Example of a \texttt{tools} Session}

To give you a quick feel for the \texttt{tools} utility, we show a
simple session where we create a file \texttt{n.txt} with 3 lines, add
strings before and after each line, and display the lines on the screen.
You can experiment with the various commands to gain experience with the
\texttt{tools} utility.

\begin{verbatim}
MM> tools
Entering the Text Tools utilities.
Type HELP for help, EXIT to exit.
TOOLS> number
Output file <n.tmp>? n.txt
First number <1>?
Last number <10>? 3
Increment <1>?
TOOLS> add
Input/output file? n.txt
String to add to beginning of each line <>? This is line
String to add to end of each line <>? .
The file n.txt has 3 lines; 3 were changed.
First change is on line 1:
This is line 1.
TOOLS> type n.txt
This is line 1.
This is line 2.
This is line 3.
TOOLS> exit
Exiting the Text Tools.
Type EXIT again to exit Metamath.
MM>
\end{verbatim}



\appendix
\chapter{Sample Representations}
\label{ASCII}

This Appendix provides a sample of {\sc ASCII} representations,
their corresponding traditional mathematical symbols,
and a discussion of their meanings
in the \texttt{set.mm} database.
These are provided in order of appearance.
This is only a partial list, and new definitions are routinely added.
A complete list is available at \url{http://metamath.org}.

These {\sc ASCII} representations, along
with information on how to display them,
are defined in the \texttt{set.mm} database file inside
a special comment called a \texttt{\$t} {\em
comment}\index{\texttt{\$t} comment} or {\em typesetting
comment.}\index{typesetting comment}
A typesetting comment
is indicated by the appearance of the
two-character string \texttt{\$t} at the beginning of the comment.
For more information,
see Section~\ref{tcomment}, p.~\pageref{tcomment}.

In the following table the ``{\sc ASCII}'' column shows the {\sc ASCII}
representation,
``Symbol'' shows the mathematical symbolic display
that corresponds to that {\sc ASCII} representation, ``Labels'' shows
the key label(s) that define the representation, and
``Description'' provides a description about the symbol.
As usual, ``iff'' is short for ``if and only if.''\index{iff}
In most cases the ``{\sc ASCII}'' column only shows
the key token, but it will sometimes show a sequence of tokens
if that is necessary for clarity.

{\setlength{\extrarowsep}{4pt} % Keep rows from being too close together
\begin{longtabu}   { @{} c c l X }
\textbf{ASCII} & \textbf{Symbol} & \textbf{Labels} & \textbf{Description} \\
\endhead
\texttt{|-} & $\vdash$ & &
  ``It is provable that...'' \\
\texttt{ph} & $\varphi$ & \texttt{wph} &
  The wff (boolean) variable phi,
  conventionally the first wff variable. \\
\texttt{ps} & $\psi$ & \texttt{wps} &
  The wff (boolean) variable psi,
  conventionally the second wff variable. \\
\texttt{ch} & $\chi$ & \texttt{wch} &
  The wff (boolean) variable chi,
  conventionally the third wff variable. \\
\texttt{-.} & $\lnot$ & \texttt{wn} &
  Logical not. E.g., if $\varphi$ is true, then $\lnot \varphi$ is false. \\
\texttt{->} & $\rightarrow$ & \texttt{wi} &
  Implies, also known as material implication.
  In classical logic the expression $\varphi \rightarrow \psi$ is true
  if either $\varphi$ is false or $\psi$ is true (or both), that is,
  $\varphi \rightarrow \psi$ has the same meaning as
  $\lnot \varphi \lor \psi$ (as proven in theorem \texttt{imor}). \\
\texttt{<->} & $\leftrightarrow$ &
  \hyperref[df-bi]{\texttt{df-bi}} &
  Biconditional (aka is-equals for boolean values).
  $\varphi \leftrightarrow \psi$ is true iff
  $\varphi$ and $\psi$ have the same value. \\
\texttt{\char`\\/} & $\lor$ &
  \makecell[tl]{{\hyperref[df-or]{\texttt{df-or}}}, \\
	         \hyperref[df-3or]{\texttt{df-3or}}} &
  Disjunction (logical ``or''). $\varphi \lor \psi$ is true iff
  $\varphi$, $\psi$, or both are true. \\
\texttt{/\char`\\} & $\land$ &
  \makecell[tl]{{\hyperref[df-an]{\texttt{df-an}}}, \\
                 \hyperref[df-3an]{\texttt{df-3an}}} &
  Conjunction (logical ``and''). $\varphi \land \psi$ is true iff
  both $\varphi$ and $\psi$ are true. \\
\texttt{A.} & $\forall$ &
  \texttt{wal} &
  For all; the wff $\forall x \varphi$ is true iff
  $\varphi$ is true for all values of $x$. \\
\texttt{E.} & $\exists$ &
  \hyperref[df-ex]{\texttt{df-ex}} &
  There exists; the wff
  $\exists x \varphi$ is true iff
  there is at least one $x$ where $\varphi$ is true. \\
\texttt{[ y / x ]} & $[ y / x ]$ &
  \hyperref[df-sb]{\texttt{df-sb}} &
  The wff $[ y / x ] \varphi$ produces
  the result when $y$ is properly substituted for $x$ in $\varphi$
  ($y$ replaces $x$).
  % This is elsb4
  % ( [ x / y ] z e. y <-> z e. x )
  For example,
  $[ x / y ] z \in y$ is the same as $z \in x$. \\
\texttt{E!} & $\exists !$ &
  \hyperref[df-eu]{\texttt{df-eu}} &
  There exists exactly one;
  $\exists ! x \varphi$ is true iff
  there is at least one $x$ where $\varphi$ is true. \\
\texttt{\{ y | phi \}}  & $ \{ y | \varphi \}$ &
  \hyperref[df-clab]{\texttt{df-clab}} &
  The class of all sets where $\varphi$ is true. \\
\texttt{=} & $ = $ &
  \hyperref[df-cleq]{\texttt{df-cleq}} &
  Class equality; $A = B$ iff $A$ equals $B$. \\
\texttt{e.} & $ \in $ &
  \hyperref[df-clel]{\texttt{df-clel}} &
  Class membership; $A \in B$ if $A$ is a member of $B$. \\
\texttt{{\char`\_}V} & {\rm V} &
  \hyperref[df-v]{\texttt{df-v}} &
  Class of all sets (not itself a set). \\
\texttt{C\_} & $ \subseteq $ &
  \hyperref[df-ss]{\texttt{df-ss}} &
  Subclass (subset); $A \subseteq B$ is true iff
  $A$ is a subclass of $B$. \\
\texttt{u.} & $ \cup $ &
  \hyperref[df-un]{\texttt{df-un}} &
  $A \cup B$ is the union of classes $A$ and $B$. \\
\texttt{i^i} & $ \cap $ &
  \hyperref[df-in]{\texttt{df-in}} &
  $A \cap B$ is the intersection of classes $A$ and $B$. \\
\texttt{\char`\\} & $ \setminus $ &
  \hyperref[df-dif]{\texttt{df-dif}} &
  $A \setminus B$ (set difference)
  is the class of all sets in $A$ except for those in $B$. \\
\texttt{(/)} & $ \varnothing $ &
  \hyperref[df-nul]{\texttt{df-nul}} &
  $ \varnothing $ is the empty set (aka null set). \\
\texttt{\char`\~P} & $ \cal P $ &
  \hyperref[df-pw]{\texttt{df-pw}} &
  Power class. \\
\texttt{<.\ A , B >.} & $\langle A , B \rangle$ &
  \hyperref[df-op]{\texttt{df-op}} &
  The ordered pair $\langle A , B \rangle$. \\
\texttt{( F ` A )} & $ ( F ` A ) $ &
  \hyperref[df-fv]{\texttt{df-fv}} &
  The value of function $F$ when applied to $A$. \\
\texttt{\_i} & $ i $ &
  \texttt{df-i} &
  The square root of negative one. \\
\texttt{x.} & $ \cdot $ &
  \texttt{df-mul} &
  Complex number multiplication; $2~\cdot~3~=~6$. \\
\texttt{CC} & $ \mathbb{C} $ &
  \texttt{df-c} &
  The set of complex numbers. \\
\texttt{RR} & $ \mathbb{R} $ &
  \texttt{df-r} &
  The set of real numbers. \\
\end{longtabu}
} % end of extrarowsep

\chapter{Compressed Proofs}
\label{compressed}\index{compressed proof}\index{proof!compressed}

The proofs in the \texttt{set.mm} set theory database are stored in compressed
format for efficiency.  Normally you needn't concern yourself with the
compressed format, since you can display it with the usual proof display tools
in the Metamath program (\texttt{show proof}\ldots) or convert it to the normal
RPN proof format described in Section~\ref{proof} (with \texttt{save proof}
{\em label} \texttt{/normal}).  However for sake of completeness we describe the
format here and show how it maps to the normal RPN proof format.

A compressed proof, located between \texttt{\$=} and \texttt{\$.}\ keywords, consists
of a left parenthesis, a sequence of statement labels, a right parenthesis,
and a sequence of upper-case letters \texttt{A} through \texttt{Z} (with optional
white space between them).  White space must surround the parentheses
and the labels.  The left parenthesis tells Metamath that a
compressed proof follows.  (A normal RPN proof consists of just a sequence of
labels, and a parenthesis is not a legal character in a label.)

The sequence of upper-case letters corresponds to a sequence of integers
with the following mapping.  Each integer corresponds to a proof step as
described later.
\begin{center}
  \texttt{A} = 1 \\
  \texttt{B} = 2 \\
   \ldots \\
  \texttt{T} = 20 \\
  \texttt{UA} = 21 \\
  \texttt{UB} = 22 \\
   \ldots \\
  \texttt{UT} = 40 \\
  \texttt{VA} = 41 \\
  \texttt{VB} = 42 \\
   \ldots \\
  \texttt{YT} = 120 \\
  \texttt{UUA} = 121 \\
   \ldots \\
  \texttt{YYT} = 620 \\
  \texttt{UUUA} = 621 \\
   etc.
\end{center}

In other words, \texttt{A} through \texttt{T} represent the
least-significant digit in base 20, and \texttt{U} through \texttt{Y}
represent zero or more most-significant digits in base 5, where the
digits start counting at 1 instead of the usual 0. With this scheme, we
don't need white space between these ``numbers.''

(In the design of the compressed proof format, only upper-case letters,
as opposed to say all non-whitespace printable {\sc ascii} characters other than
%\texttt{\$}, was chosen to make the compressed proof a little less
%displeasing to the eye, at the expense of a typical 20\% compression
\texttt{\$}, were chosen so as not to collide with most text editor
searches, at the expense of a typical 20\% compression
loss.  The base 5/base 20 grouping, as opposed to say base 6/base 19,
was chosen by experimentally determining the grouping that resulted in
best typical compression.)

The letter \texttt{Z} identifies (tags) a proof step that is identical to one
that occurs later on in the proof; it helps shorten the proof by not requiring
that identical proof steps be proved over and over again (which happens often
when building wff's).  The \texttt{Z} is placed immediately after the
least-significant digit (letters \texttt{A} through \texttt{T}) that ends the integer
corresponding to the step to later be referenced.

The integers that the upper-case letters correspond to are mapped to labels as
follows.  If the statement being proved has $m$ mandatory hypotheses, integers
1 through $m$ correspond to the labels of these hypotheses in the order shown
by the \texttt{show statement ... / full} command, i.e., the RPN order\index{RPN
order} of the mandatory
hypotheses.  Integers $m+1$ through $m+n$ correspond to the labels enclosed in
the parentheses of the compressed proof, in the order that they appear, where
$n$ is the number of those labels.  Integers $m+n+1$ on up don't directly
correspond to statement labels but point to proof steps identified with the
letter \texttt{Z}, so that these proof steps can be referenced later in the
proof.  Integer $m+n+1$ corresponds to the first step tagged with a \texttt{Z},
$m+n+2$ to the second step tagged with a \texttt{Z}, etc.  When the compressed
proof is converted to a normal proof, the entire subproof of a step tagged
with \texttt{Z} replaces the reference to that step.

For efficiency, Metamath works with compressed proofs directly, without
converting them internally to normal proofs.  In addition to the usual
error-checking, an error message is given if (1) a label in the label list in
parentheses does not refer to a previous \texttt{\$p} or \texttt{\$a} statement or a
non-mandatory hypothesis of the statement being proved and (2) a proof step
tagged with \texttt{Z} is referenced before the step tagged with the \texttt{Z}.

Just as in a normal proof under development (Section~\ref{unknown}), any step
or subproof that is not yet known may be represented with a single \texttt{?}.
White space does not have to appear between the \texttt{?}\ and the upper-case
letters (or other \texttt{?}'s) representing the remainder of the proof.

% April 1, 2004 Appendix C has been added back in with corrections.
%
% May 20, 2003 Appendix C was removed for now because there was a problem found
% by Bob Solovay
%
% Also, removed earlier \ref{formalspec} 's (3 cases above)
%
% Bob Solovay wrote on 30 Nov 2002:
%%%%%%%%%%%%% (start of email comment )
%      3. My next set of comments concern appendix C. I read this before I
% read Chapter 4. So I first noted that the system as presented in the
% Appendix lacked a certain formal property that I thought desirable. I
% then came up with a revised formal system that had this property. Upon
% reading Chapter 4, I noticed that the revised system was closer to the
% treatment in Chapter 4 than the system in Appendix C.
%
%         First a very minor correction:
%
%         On page 142 line 2: The condition that V(e) != V(f) should only be
% required of e, f in T such that e != f.
%
%         Here is a natural property [transitivity] that one would like
% the formal system to have:
%
%         Let Gamma be a set of statements. Suppose that the statement Phi
% is provable from Gamma and that the statement Psi is provable from Gamma
% \cup {Phi}. Then Psi is provable from Gamma.
%
%         I shall present an example to show that this property does not
% hold for the formal systems of Appendix C:
%
%         I write the example in metamath style:
%
% $c A B C D E $.
% $v x y
%
% ${
% tx $f A x $.
% ty $f B y $.
% ax1 $a C x y $.
% $}
%
% ${
% tx $f A x $.
% ty $f B y $.
% ax2-h1 $e C x y $.
% ax2 $a D y $.
% $}
%
% ${
% ty $f B y $.
% ax3-h1 $e D y $.
% ax3 $a E y $.
% $}
%
% $(These three axioms are Gamma $)
%
% ${
% tx $f A x $.
% ty $f B y $.
% Phi $p D y $=
% tx ty tx ty ax1 ax2 $.
% $}
%
% ${
% ty $f B y $.
% Psi $p E y $=
% ty ty Phi ax3 $.
% $}
%
%
% I omit the formal proofs of the following claims. [I will be glad to
% supply them upon request.]
%
% 1) Psi is not provable from Gamma;
%
% 2) Psi is provable from Gamma + Phi.
%
% Here "provable" refers to the formalism of Appendix C.
%
% The trouble of course is that Psi is lacking the variable declaration
%
% $f Ax $.
%
% In the Metamath system there is no trouble proving Psi. I attach a
% metamath file that shows this and which has been checked by the
% metamath program.
%
% I next want to indicate how I think the treatment in Appendix C should
% be revised so as to conform more closely to the metamath system of the
% main text. The revised system *does* have the transitivity property.
%
% We want to give revised definitions of "statement" and
% "provable". [cf. sections C.2.4. and C.2.5] Our new definitions will
% use the definitions given in Appendix C. So we take the following
% tack. We refer to the original notions as o-statement and o-provable. And
% we refer to the notions we are defining as n-statement and n-provable.
%
%         A n-statement is an o-statement in which the only variables
% that appear in the T component are mandatory.
%
%         To any o-statement we can associate its reduct which is a
% n-statement by dropping all the elements of T or D which contain
% non-mandatory variables.
%
%         An n-statement gamma is n-provable if there is an o-statement
% gamma' which has gamma as its reduct andf such that gamma' is
% o-provable.
%
%         It seems to me [though I am not completely sure on this point]
% that n-provability corresponds to metamath provability as discussed
% say in Chapter 4.
%
%         Attached to this letter is the metamath proof of Phi and Psi
% from Gamma discussed above.
%
%         I am still brooding over the question of whether metamath
% correctly formalizes set-theory. No doubt I will have some questions
% re this after my thoughts become clearer.
%%%%%%%%%%%%%%%% (end of email comment)

%%%%%%%%%%%%%%%% (start of 2nd email comment from Bob Solovay 1-Apr-04)
%
%         I hope that Appendix C is the one that gives a "formal" treatment
% of Metamath. At any rate, thats the appendix I want to comment on.
%
%         I'm going to suggest two changes in the definition.
%
%         First change (in the definition of statement): Require that the
% sets D, T, and E be finite.
%
%         Probably things are fine as you give them. But in the applications
% to the main metamath system they will always be finite, and its useful in
% thinking about things [at least for me] to stick to the finite case.
%
%         Second change:
%
%         First let me give an approximate description. Remove the dummy
% variables from the statement. Instead, include them in the proof.
%
%         More formally: Require that T consists of type declarations only
% for mandatory variables. Require that all the pairs in D consist of
% mandatory variables.
%
%         At the start of a proof we are allowed to declare a finite number
% of dummy variables [provided that none of them appear in any of the
% statements in E \cup {A}. We have to supply type declarations for all the
% dummy variables. We are allowed to add new $d statements referring to
% either the mandatory or dummy variables. But we require that no new $d
% statement references only mandatory variables.
%
%         I find this way of doing things more conceptual than the treatment
% in Appendix C. But the change [which I will use implicitly in later
% letters about doing Peano] is mainly aesthetic. I definitely claim that my
% results on doing Peano all apply to Metamath as it is presented in your
% book.
%
%         --Bob
%
%%%%%%%%%%%%%%%% (end of 2nd email comment)

%%
%% When uncommenting the below, also uncomment references above to {formalspec}
%%
\chapter{Metamath's Formal System}\label{formalspec}\index{Metamath!as a formal
system}

\section{Introduction}

\begin{quote}
  {\em Perfection is when there is no longer anything more to take away.}
    \flushright\sc Antoine de
     Saint-Exupery\footnote{\cite[p.~3-25]{Campbell}.}\\
\end{quote}\index{de Saint-Exupery, Antoine}

This appendix describes the theory behind the Metamath language in an abstract
way intended for mathematicians.  Specifically, we construct two
set-theo\-ret\-i\-cal objects:  a ``formal system'' (roughly, a set of syntax
rules, axioms, and logical rules) and its ``universe'' (roughly, the set of
theorems derivable in the formal system).  The Metamath computer language
provides us with a way to describe specific formal systems and, with the aid of
a proof provided by the user, to verify that given theorems
belong to their universes.

To understand this appendix, you need a basic knowledge of informal set theory.
It should be sufficient to understand, for example, Ch.\ 1 of Munkres' {\em
Topology} \cite{Munkres}\index{Munkres, James R.} or the
introductory set theory chapter
in many textbooks that introduce abstract mathematics. (Note that there are
minor notational differences among authors; e.g.\ Munkres uses $\subset$ instead
of our $\subseteq$ for ``subset.''  We use ``included in'' to mean ``a subset
of,'' and ``belongs to'' or ``is contained in'' to mean ``is an element of.'')
What we call a ``formal'' description here, unlike earlier, is actually an
informal description in the ordinary language of mathematicians.  However we
provide sufficient detail so that a mathematician could easily formalize it,
even in the language of Metamath itself if desired.  To understand the logic
examples at the end of this appendix, familiarity with an introductory book on
mathematical logic would be helpful.

\section{The Formal Description}

\subsection[Preliminaries]{Preliminaries\protect\footnotemark}%
\footnotetext{This section is taken mostly verbatim
from Tarski \cite[p.~63]{Tarski1965}\index{Tarski, Alfred}.}

By $\omega$ we denote the set of all natural numbers (non-negative integers).
Each natural number $n$ is identified with the set of all smaller numbers: $n =
\{ m | m < n \}$.  The formula $m < n$ is thus equivalent to the condition: $m
\in n$ and $m,n \in \omega$. In particular, 0 is the number zero and at the
same time the empty set $\varnothing$, $1=\{0\}$, $2=\{0,1\}$, etc. ${}^B A$
denotes the set of all functions on $B$ to $A$ (i.e.\ with domain $B$ and range
included in $A$).  The members of ${}^\omega A$ are what are called {\em simple
infinite sequences},\index{simple infinite sequence}
with all {\em terms}\index{term} in $A$.  In case $n \in \omega$, the
members of ${}^n A$ are referred to as {\em finite $n$-termed
sequences},\index{finite $n$-termed
sequence} again
with terms in $A$.  The consecutive terms (function values) of a finite or
infinite sequence $f$ are denoted by $f_0, f_1, \ldots ,f_n,\ldots$.  Every
finite sequence $f \in \bigcup _{n \in \omega} {}^n A$ uniquely determines the
number $n$ such that $f \in {}^n A$; $n$ is called the {\em
length}\index{length of a sequence ({$"|\ "|$})} of $f$ and
is denoted by $|f|$.  $\langle a \rangle$ is the sequence $f$ with $|f|=1$ and
$f_0=a$; $\langle a,b \rangle$ is the sequence $f$ with $|f|=2$, $f_0=a$,
$f_1=b$; etc.  Given two finite sequences $f$ and $g$, we denote by $f\frown g$
their {\em concatenation},\index{concatenation} i.e., the
finite sequence $h$ determined by the
conditions:
\begin{eqnarray*}
& |h| = |f|+|g|;&  \\
& h_n = f_n & \mbox{\ for\ } n < |f|;  \\
& h_{|f|+n} = g_n & \mbox{\ for\ } n < |g|.
\end{eqnarray*}

\subsection{Constants, Variables, and Expressions}

A formal system has a set of {\em symbols}\index{symbol!in
a formal system} denoted
by $\mbox{\em SM}$.  A
precise set-theo\-ret\-i\-cal definition of this set is unimportant; a symbol
could be considered a primitive or atomic element if we wish.  We assume this
set is divided into two disjoint subsets:  a set $\mbox{\em CN}$ of {\em
constants}\index{constant!in a formal system} and a set $\mbox{\em VR}$ of
{\em variables}.\index{variable!in a formal system}  $\mbox{\em CN}$ and
$\mbox{\em VR}$ are each assumed to consist of countably many symbols which
may be arranged in finite or simple infinite sequences $c_0, c_1, \ldots$ and
$v_0, v_1, \ldots$ respectively, without repeating terms.  We will represent
arbitrary symbols by metavariables $\alpha$, $\beta$, etc.

{\footnotesize\begin{quotation}
{\em Comment.} The variables $v_0, v_1, \ldots$ of our formal system
correspond to what are usually considered ``metavariables'' in
descriptions of specific formal systems in the literature.  Typically,
when describing a specific formal system a book will postulate a set of
primitive objects called variables, then proceed to describe their
properties using metavariables that range over them, never mentioning
again the actual variables themselves.  Our formal system does not
mention these primitive variable objects at all but deals directly with
metavariables, as its primitive objects, from the start.  This is a
subtle but key distinction you should keep in mind, and it makes our
definition of ``formal system'' somewhat different from that typically
found in the literature.  (So, the $\alpha$, $\beta$, etc.\ above are
actually ``metametavariables'' when used to represent $v_0, v_1,
\ldots$.)
\end{quotation}}

Finite sequences all terms of which are symbols are called {\em
expressions}.\index{expression!in a formal system}  $\mbox{\em EX}$ is
the set of all expressions; thus
\begin{displaymath}
\mbox{\em EX} = \bigcup _{n \in \omega} {}^n \mbox{\em SM}.
\end{displaymath}

A {\em constant-prefixed expression}\index{constant-prefixed expression}
is an expression of non-zero length
whose first term is a constant.  We denote the set of all constant-prefixed
expressions by $\mbox{\em EX}_C = \{ e \in \mbox{\em EX} | ( |e| > 0 \wedge
e_0 \in \mbox{\em CN} ) \}$.

A {\em constant-variable pair}\index{constant-variable pair}
is an expression of length 2 whose first term
is a constant and whose second term is a variable.  We denote the set of all
constant-variable pairs by $\mbox{\em EX}_2 = \{ e \in \mbox{\em EX}_C | ( |e|
= 2 \wedge e_1 \in \mbox{\em VR} ) \}$.


{\footnotesize\begin{quotation}
{\em Relationship to Metamath.} In general, the set $\mbox{\em SM}$
corresponds to the set of declared math symbols in a Metamath database, the
set $\mbox{\em CN}$ to those declared with \texttt{\$c} statements, and the set
$\mbox{\em VR}$ to those declared with \texttt{\$v} statements.  Of course a
Metamath database can only have a finite number of math symbols, whereas
formal systems in general can have an infinite number, although the number of
Metamath math symbols available is in principle unlimited.

The set $\mbox{\em EX}_C$ corresponds to the set of permissible expressions
for \texttt{\$e}, \texttt{\$a}, and \texttt{\$p} statements.  The set $\mbox{\em EX}_2$
corresponds to the set of permissible expressions for \texttt{\$f} statements.
\end{quotation}}

We denote by ${\cal V}(e)$ the set of all variables in an expression $e \in
\mbox{\em EX}$, i.e.\ the set of all $\alpha \in \mbox{\em VR}$ such that
$\alpha = e_n$ for some $n < |e|$.  We also denote (with abuse of notation) by
${\cal V}(E)$ the set of all variables in a collection of expressions $E
\subseteq \mbox{\em EX}$, i.e.\ $\bigcup _{e \in E} {\cal V}(e)$.


\subsection{Substitution}

Given a function $F$ from $\mbox{\em VR}$ to
$\mbox{\em EX}$, we
denote by $\sigma_{F}$ or just $\sigma$ the function from $\mbox{\em EX}$ to
$\mbox{\em EX}$ defined recursively for nonempty sequences by
\begin{eqnarray*}
& \sigma(<\alpha>) = F(\alpha) & \mbox{for\ } \alpha \in \mbox{\em VR}; \\
& \sigma(<\alpha>) = <\alpha> & \mbox{for\ } \alpha \not\in \mbox{\em VR}; \\
& \sigma(g \frown h) = \sigma(g) \frown
    \sigma(h) & \mbox{for\ } g,h \in \mbox{\em EX}.
\end{eqnarray*}
We also define $\sigma(\varnothing)=\varnothing$.  We call $\sigma$ a {\em
simultaneous substitution}\index{substitution!variable}\index{variable
substitution} (or just {\em substitution}) with {\em substitution
map}\index{substitution map} $F$.

We also denote (with abuse of notation) by $\sigma(E)$ a substitution on a
collection of expressions $E \subseteq \mbox{\em EX}$, i.e.\ the set $\{
\sigma(e) | e \in E \}$.  The collection $\sigma(E)$ may of course contain
fewer expressions than $E$ because duplicate expressions could result from the
substitution.

\subsection{Statements}

We denote by $\mbox{\em DV}$ the set of all
unordered pairs $\{\alpha, \beta \} \subseteq \mbox{\em VR}$ such that $\alpha
\neq \beta$.  $\mbox{\em DV}$ stands for ``distinct variables.''

A {\em pre-statement}\index{pre-statement!in a formal system} is a
quadruple $\langle D,T,H,A \rangle$ such that
$D\subseteq \mbox{\em DV}$, $T\subseteq \mbox{\em EX}_2$, $H\subseteq
\mbox{\em EX}_C$ and $H$ is finite,
$A\in \mbox{\em EX}_C$, ${\cal V}(H\cup\{A\}) \subseteq
{\cal V}(T)$, and $\forall e,f\in T {\ } {\cal V}(e) \neq {\cal V}(f)$ (or
equivalently, $e_1 \ne f_1$) whenever $e \neq f$. The terms of the quadruple are called {\em
distinct-variable restrictions},\index{disjoint-variable restriction!in a
formal system} {\em variable-type hypotheses},\index{variable-type
hypothesis!in a formal system} {\em logical hypotheses},\index{logical
hypothesis!in a formal system} and the {\em assertion}\index{assertion!in a
formal system} respectively.  We denote by $T_M$ ({\em mandatory variable-type
hypotheses}\index{mandatory variable-type hypothesis!in a formal system}) the
subset of $T$ such that ${\cal V}(T_M) ={\cal V}(H \cup \{A\})$.  We denote by
$D_M=\{\{\alpha,\beta\}\in D|\{\alpha,\beta\}\subseteq {\cal V}(T_M)\}$ the
{\em mandatory distinct-variable restrictions}\index{mandatory
disjoint-variable restriction!in a formal system} of the pre-statement.
The set
of {\em mandatory hypotheses}\index{mandatory hypothesis!in a formal system}
is $T_M\cup H$.  We call the quadruple $\langle D_M,T_M,H,A \rangle$
the {\em reduct}\index{reduct!in a formal system} of
the pre-statement $\langle D,T,H,A \rangle$.

A {\em statement} is the reduct of some pre-statement\index{statement!in a
formal system}.  A statement is therefore a special kind of pre-statement;
in particular, a statement is the reduct of itself.

{\footnotesize\begin{quotation}
{\em Comment.}  $T$ is a set of expressions, each of length 2, that associate
a set of constants (``variable types'') with a set of variables.  The
condition ${\cal V}(H\cup\{A\}) \subseteq {\cal V}(T) $
means that each variable occurring in a statement's logical
hypotheses or assertion must have an associated variable-type hypothesis or
``type declaration,'' in  analogy to a computer programming language, where a
variable must be declared to be say, a string or an integer.  The requirement
that $\forall e,f\in T \, e_1 \ne f_1$ for $e\neq f$
means that each variable must be
associated with a unique constant designating its variable type; e.g., a
variable might be a ``wff'' or a ``set'' but not both.

Distinct-variable restrictions are used to specify what variable substitutions
are permissible to make for the statement to remain valid.  For example, in
the theorem scheme of set theory $\lnot\forall x\,x=y$ we may not substitute
the same variable for both $x$ and $y$.  On the other hand, the theorem scheme
$x=y\to y=x$ does not require that $x$ and $y$ be distinct, so we do not
require a distinct-variable restriction, although having one
would cause no harm other than making the scheme less general.

A mandatory variable-type hypothesis is one whose variable exists in a logical
hypothesis or the assertion.  A provable pre-statement
(defined below) may require
non-mandatory variable-type hypotheses that effectively introduce ``dummy''
variables for use in its proof.  Any number of dummy variables might
be required by a specific proof; indeed, it has been shown by H.\
Andr\'{e}ka\index{Andr{\'{e}}ka, H.} \cite{Nemeti} that there is no finite
upper bound to the number of dummy variables needed to prove an arbitrary
theorem in first-order logic (with equality) having a fixed number $n>2$ of
individual variables.  (See also the Comment on p.~\pageref{nodd}.)
For this reason we do not set a finite size bound on the collections $D$ and
$T$, although in an actual application (Metamath database) these will of
course be finite, increased to whatever size is necessary as more
proofs are added.
\end{quotation}}

{\footnotesize\begin{quotation}
{\em Relationship to Metamath.} A pre-statement of a formal system
corresponds to an extended frame in a Metamath database
(Section~\ref{frames}).  The collections $D$, $T$, and $H$ correspond
respectively to the \texttt{\$d}, \texttt{\$f}, and \texttt{\$e}
statement collections in an extended frame.  The expression $A$
corresponds to the \texttt{\$a} (or \texttt{\$p}) statement in an
extended frame.

A statement of a formal system corresponds to a frame in a Metamath
database.
\end{quotation}}

\subsection{Formal Systems}

A {\em formal system}\index{formal system} is a
triple $\langle \mbox{\em CN},\mbox{\em
VR},\Gamma\rangle$ where $\Gamma$ is a set of statements.  The members of
$\Gamma$ are called {\em axiomatic statements}.\index{axiomatic
statement!in a formal system}  Sometimes we will refer to a
formal system by just $\Gamma$ when $\mbox{\em CN}$ and $\mbox{\em VR}$ are
understood.

Given a formal system $\Gamma$, the {\em closure}\index{closure}\footnote{This
definition of closure incorporates a simplification due to
Josh Purinton.\index{Purinton, Josh}.} of a
pre-statement
$\langle D,T,H,A \rangle$ is the smallest set $C$ of expressions
such that:
%\begin{enumerate}
%  \item $T\cup H\subseteq C$; and
%  \item If for some axiomatic statement
%    $\langle D_M',T_M',H',A' \rangle \in \Gamma_A$, for
%    some $E \subseteq C$, some $F \subseteq C-T$ (where ``-'' denotes
%    set difference), and some substitution
%    $\sigma$ we have
%    \begin{enumerate}
%       \item $\sigma(T_M') = E$ (where, as above, the $M$ denotes the
%           mandatory variable-type hypotheses of $T^A$);
%       \item $\sigma(H') = F$;
%       \item for all $\{\alpha,\beta\}\in D^A$ and $\subseteq
%         {\cal V}(T_M')$, for all $\gamma\in {\cal V}(\sigma(\langle \alpha
%         \rangle))$, and for all $\delta\in  {\cal V}(\sigma(\langle \beta
%         \rangle))$, we have $\{\gamma, \delta\} \in D$;
%   \end{enumerate}
%   then $\sigma(A') \in C$.
%\end{enumerate}
\begin{list}{}{\itemsep 0.0pt}
  \item[1.] $T\cup H\subseteq C$; and
  \item[2.] If for some axiomatic statement
    $\langle D_M',T_M',H',A' \rangle \in
       \Gamma$ and for some substitution
    $\sigma$ we have
    \begin{enumerate}
       \item[a.] $\sigma(T_M' \cup H') \subseteq C$; and
       \item[b.] for all $\{\alpha,\beta\}\in D_M'$, for all $\gamma\in
         {\cal V}(\sigma(\langle \alpha
         \rangle))$, and for all $\delta\in  {\cal V}(\sigma(\langle \beta
         \rangle))$, we have $\{\gamma, \delta\} \in D$;
   \end{enumerate}
   then $\sigma(A') \in C$.
\end{list}
A pre-statement $\langle D,T,H,A
\rangle$ is {\em provable}\index{provable statement!in a formal
system} if $A\in C$ i.e.\ if its assertion belongs to its
closure.  A statement is {\em provable} if it is
the reduct of a provable pre-statement.
The {\em universe}\index{universe of a formal system}
of a formal system is
the collection of all of its provable statements.  Note that the
set of axiomatic statements $\Gamma$ in a formal system is a subset of its
universe.

{\footnotesize\begin{quotation}
{\em Comment.} The first condition in the definition of closure simply says
that the hypotheses of the pre-statement are in its closure.

Condition 2(a) says that a substitution exists that makes the
mandatory hypotheses of an axiomatic statement exactly match some members of
the closure.  This is what we explicitly demonstrate in a Metamath language
proof.

%Conditions 2(a) and 2(b) say that a substitution exists that makes the
%(mandatory) hypotheses of an axiomatic statement exactly match some members of
%the closure.  This is what we explicitly demonstrate with a Metamath language
%proof.
%
%The set of expressions $F$ in condition 2(b) excludes the variable-type
%hypotheses; this is done because non-mandatory variable-type hypotheses are
%effectively ``dropped'' as irrelevant whereas logical hypotheses must be
%retained to achieve a consistent logical system.

Condition 2(b) describes how distinct-variable restrictions in the axiomatic
statement must be met.  It means that after a substitution for two variables
that must be distinct, the resulting two expressions must either contain no
variables, or if they do, they may not have variables in common, and each pair
of any variables they do have, with one variable from each expression, must be
specified as distinct in the original statement.
\end{quotation}}

{\footnotesize\begin{quotation}
{\em Relationship to Metamath.} Axiomatic statements
 and provable statements in a formal
system correspond to the frames for \texttt{\$a} and \texttt{\$p} statements
respectively in a Metamath database.  The set of axiomatic statements is a
subset of the set of provable statements in a formal system, although in a
Metamath database a \texttt{\$a} statement is distinguished by not having a
proof.  A Metamath language proof for a \texttt{\$p} statement tells the computer
how to explicitly construct a series of members of the closure ultimately
leading to a demonstration that the assertion
being proved is in the closure.  The actual closure typically contains
an infinite number of expressions.  A formal system itself does not have
an explicit object called a ``proof'' but rather the existence of a proof
is implied indirectly by membership of an assertion in a provable
statement's closure.  We do this to make the formal system easier
to describe in the language of set theory.

We also note that once established as provable, a statement may be considered
to acquire the same status as an axiomatic statement, because if the set of
axiomatic statements is extended with a provable statement, the universe of
the formal system remains unchanged (provided that $\mbox{\em VR}$ is
infinite).
In practice, this means we can build a hierarchy of provable statements to
more efficiently establish additional provable statements.  This is
what we do in Metamath when we allow proofs to reference previous
\texttt{\$p} statements as well as previous \texttt{\$a} statements.
\end{quotation}}

\section{Examples of Formal Systems}

{\footnotesize\begin{quotation}
{\em Relationship to Metamath.} The examples in this section, except Example~2,
are for the most part exact equivalents of the development in the set
theory database \texttt{set.mm}.  You may want to compare Examples~1, 3, and 5
to Section~\ref{metaaxioms}, Example 4 to Sections~\ref{metadefprop} and
\ref{metadefpred}, and Example 6 to
Section~\ref{setdefinitions}.\label{exampleref}
\end{quotation}}

\subsection{Example~1---Propositional Calculus}\index{propositional calculus}

Classical propositional calculus can be described by the following formal
system.  We assume the set of variables is infinite.  Rather than denoting the
constants and variables by $c_0, c_1, \ldots$ and $v_0, v_1, \ldots$, for
readability we will instead use more conventional symbols, with the
understanding of course that they denote distinct primitive objects.
Also for readability we may omit commas between successive terms of a
sequence; thus $\langle \mbox{wff\ } \varphi\rangle$ denotes
$\langle \mbox{wff}, \varphi\rangle$.

Let
\begin{itemize}
  \item[] $\mbox{\em CN}=\{\mbox{wff}, \vdash, \to, \lnot, (,)\}$
  \item[] $\mbox{\em VR}=\{\varphi,\psi,\chi,\ldots\}$
  \item[] $T = \{\langle \mbox{wff\ } \varphi\rangle,
             \langle \mbox{wff\ } \psi\rangle,
             \langle \mbox{wff\ } \chi\rangle,\ldots\}$, i.e.\ those
             expressions of length 2 whose first member is $\mbox{\rm wff}$
             and whose second member belongs to $\mbox{\em VR}$.\footnote{For
convenience we let $T$ be an infinite set; the definition of a statement
permits this in principle.  Since a Metamath source file has a finite size, in
practice we must of course use appropriate finite subsets of this $T$,
specifically ones containing at least the mandatory variable-type
hypotheses.  Similarly, in the source file we introduce new variables as
required, with the understanding that a potentially infinite number of
them are available.}
\noindent Then $\Gamma$ consists of the axiomatic statements that
are the reducts of the following pre-statements:
    \begin{itemize}
      \item[] $\langle\varnothing,T,\varnothing,
               \langle \mbox{wff\ }(\varphi\to\psi)\rangle\rangle$
      \item[] $\langle\varnothing,T,\varnothing,
               \langle \mbox{wff\ }\lnot\varphi\rangle\rangle$
      \item[] $\langle\varnothing,T,\varnothing,
               \langle \vdash(\varphi\to(\psi\to\varphi))
               \rangle\rangle$
      \item[] $\langle\varnothing,T,
               \varnothing,
               \langle \vdash((\varphi\to(\psi\to\chi))\to
               ((\varphi\to\psi)\to(\varphi\to\chi)))
               \rangle\rangle$
      \item[] $\langle\varnothing,T,
               \varnothing,
               \langle \vdash((\lnot\varphi\to\lnot\psi)\to
               (\psi\to\varphi))\rangle\rangle$
      \item[] $\langle\varnothing,T,
               \{\langle\vdash(\varphi\to\psi)\rangle,
                 \langle\vdash\varphi\rangle\},
               \langle\vdash\psi\rangle\rangle$
    \end{itemize}
\end{itemize}

(For example, the reduct of $\langle\varnothing,T,\varnothing,
               \langle \mbox{wff\ }(\varphi\to\psi)\rangle\rangle$
is
\begin{itemize}
\item[] $\langle\varnothing,
\{\langle \mbox{wff\ } \varphi\rangle,
             \langle \mbox{wff\ } \psi\rangle\},
             \varnothing,
               \langle \mbox{wff\ }(\varphi\to\psi)\rangle\rangle$,
\end{itemize}
which is the first axiomatic statement.)

We call the members of $\mbox{\em VR}$ {\em wff variables} or (in the context
of first-order logic which we will describe shortly) {\em wff metavariables}.
Note that the symbols $\phi$, $\psi$, etc.\ denote actual specific members of
$\mbox{\em VR}$; they are not metavariables of our expository language (which
we denote with $\alpha$, $\beta$, etc.) but are instead (meta)constant symbols
(members of $\mbox{\em SM}$) from the point of view of our expository
language.  The equivalent system of propositional calculus described in
\cite{Tarski1965} also uses the symbols $\phi$, $\psi$, etc.\ to denote wff
metavariables, but in \cite{Tarski1965} unlike here those are metavariables of
the expository language and not primitive symbols of the formal system.

The first two statements define wffs: if $\varphi$ and $\psi$ are wffs, so is
$(\varphi \to \psi)$; if $\varphi$ is a wff, so is $\lnot\varphi$. The next
three are the axioms of propositional calculus: if $\varphi$ and $\psi$ are
wffs, then $\vdash (\varphi \to (\psi \to \varphi))$ is an (axiomatic)
theorem; etc. The
last is the rule of modus ponens: if $\varphi$ and $\psi$ are wffs, and
$\vdash (\varphi\to\psi)$ and $\vdash \varphi$ are theorems, then $\vdash
\psi$ is a theorem.

The correspondence to ordinary propositional calculus is as follows.  We
consider only provable statements of the form $\langle\varnothing,
T,\varnothing,A\rangle$ with $T$ defined as above.  The first term of the
assertion $A$ of any such statement is either ``wff'' or ``$\vdash$''.  A
statement for which the first term is ``wff'' is a {\em wff} of propositional
calculus, and one where the first term is ``$\vdash$'' is a {\em
theorem (scheme)} of propositional calculus.

The universe of this formal system also contains many other provable
statements.  Those with distinct-variable restrictions are irrelevant because
propositional calculus has no constraints on substitutions.  Those that have
logical hypotheses we call {\em inferences}\index{inference} when
the logical hypotheses are of the form
$\langle\vdash\rangle\frown w$ where $w$ is a wff (with the leading constant
term ``wff'' removed).  Inferences (other than the modus ponens rule) are not a
proper part of propositional calculus but are convenient to use when building a
hierarchy of provable statements.  A provable statement with a nonsense
hypothesis such as $\langle \to,\vdash,\lnot\rangle$, and this same expression
as its assertion, we consider irrelevant; no use can be made of it in
proving theorems, since there is no way to eliminate the nonsense hypothesis.

{\footnotesize\begin{quotation}
{\em Comment.} Our use of parentheses in the definition of a wff illustrates
how axiomatic statements should be carefully stated in a way that
ties in unambiguously with the substitutions allowed by the formal system.
There are many ways we could have defined wffs---for example, Polish
prefix notation would have allowed us to omit parentheses entirely, at
the expense of readability---but we must define them in a way that is
unambiguous.  For example, if we had omitted parentheses from the
definition of $(\varphi\to \psi)$, the wff $\lnot\varphi\to \psi$ could
be interpreted as either $\lnot(\varphi\to\psi)$ or $(\lnot\varphi\to\psi)$
and would have allowed us to prove nonsense.  Note that there is no
concept of operator binding precedence built into our formal system.
\end{quotation}}

\begin{sloppy}
\subsection{Example~2---Predicate Calculus with Equality}\index{predicate
calculus}
\end{sloppy}

Here we extend Example~1 to include predicate calculus with equality,
illustrating the use of distinct-variable restrictions.  This system is the
same as Tarski's system $\mathfrak{S}_2$ in \cite{Tarski1965} (except that the
axioms of propositional calculus are different but equivalent, and a redundant
axiom is omitted).  We extend $\mbox{\em CN}$ with the constants
$\{\mbox{var},\forall,=\}$.  We extend $\mbox{\em VR}$ with an infinite set of
{\em individual metavariables}\index{individual
metavariable} $\{x,y,z,\ldots\}$ and denote this subset
$\mbox{\em Vr}$.

We also join to $\mbox{\em CN}$ a possibly infinite set $\mbox{\em Pr}$ of {\em
predicates} $\{R,S,\ldots\}$.  We associate with $\mbox{\em Pr}$ a function
$\mbox{rnk}$ from $\mbox{\em Pr}$ to $\omega$, and for $\alpha\in \mbox{\em
Pr}$ we call $\mbox{rnk}(\alpha)$ the {\em rank} of the predicate $\alpha$,
which is simply the number of ``arguments'' that the predicate has.  (Most
applications of predicate calculus will have a finite number of predicates;
for example, set theory has the single two-argument or binary predicate $\in$,
which is usually written with its arguments surrounding the predicate symbol
rather than with the prefix notation we will use for the general case.)  As a
device to facilitate our discussion, we will let $\mbox{\em Vs}$ be any fixed
one-to-one function from $\omega$ to $\mbox{\em Vr}$; thus $\mbox{\em Vs}$ is
any simple infinite sequence of individual metavariables with no repeating
terms.

In this example we will not include the function symbols that are often part of
formalizations of predicate calculus.  Using metalogical arguments that are
beyond the scope of our discussion, it can be shown that our formalization is
equivalent when functions are introduced via appropriate definitions.

We extend the set $T$ defined in Example~1 with the expressions
$\{\langle \mbox{var\ } x\rangle,$ $ \langle \mbox{var\ } y\rangle, \langle
\mbox{var\ } z\rangle,\ldots\}$.  We extend the $\Gamma$ above
with the axiomatic statements that are the reducts of the following
pre-statements:
\begin{list}{}{\itemsep 0.0pt}
      \item[] $\langle\varnothing,T,\varnothing,
               \langle \mbox{wff\ }\forall x\,\varphi\rangle\rangle$
      \item[] $\langle\varnothing,T,\varnothing,
               \langle \mbox{wff\ }x=y\rangle\rangle$
      \item[] $\langle\varnothing,T,
               \{\langle\vdash\varphi\rangle\},
               \langle\vdash\forall x\,\varphi\rangle\rangle$
      \item[] $\langle\varnothing,T,\varnothing,
               \langle \vdash((\forall x(\varphi\to\psi)
                  \to(\forall x\,\varphi\to\forall x\,\psi))
               \rangle\rangle$
      \item[] $\langle\{\{x,\varphi\}\},T,\varnothing,
               \langle \vdash(\varphi\to\forall x\,\varphi)
               \rangle\rangle$
      \item[] $\langle\{\{x,y\}\},T,\varnothing,
               \langle \vdash\lnot\forall x\lnot x=y
               \rangle\rangle$
      \item[] $\langle\varnothing,T,\varnothing,
               \langle \vdash(x=z
                  \to(x=y\to z=y))
               \rangle\rangle$
      \item[] $\langle\varnothing,T,\varnothing,
               \langle \vdash(y=z
                  \to(x=y\to x=z))
               \rangle\rangle$
\end{list}
These are the axioms not involving predicate symbols. The first two statements
extend the definition of a wff.  The third is the rule of generalization.  The
fifth states, in effect, ``For a wff $\varphi$ and variable $x$,
$\vdash(\varphi\to\forall x\,\varphi)$, provided that $x$ does not occur in
$\varphi$.''  The sixth states ``For variables $x$ and $y$,
$\vdash\lnot\forall x\lnot x = y$, provided that $x$ and $y$ are distinct.''
(This proviso is not necessary but was included by Tarski to
weaken the axiom and still show that the system is logically complete.)

Finally, for each predicate symbol $\alpha\in \mbox{\em Pr}$, we add to
$\Gamma$ an axiomatic statement, extending the definition of wff,
that is the reduct of the following pre-statement:
\begin{displaymath}
    \langle\varnothing,T,\varnothing,
            \langle \mbox{wff},\alpha\rangle\
            \frown \mbox{\em Vs}\restriction\mbox{rnk}(\alpha)\rangle
\end{displaymath}
and for each $\alpha\in \mbox{\em Pr}$ and each $n < \mbox{rnk}(\alpha)$
we add to $\Gamma$ an equality axiom that is the reduct of the
following pre-statement:
\begin{eqnarray*}
    \lefteqn{\langle\varnothing,T,\varnothing,
            \langle
      \vdash,(,\mbox{\em Vs}_n,=,\mbox{\em Vs}_{\mbox{rnk}(\alpha)},\to,
            (,\alpha\rangle\frown \mbox{\em Vs}\restriction\mbox{rnk}(\alpha)} \\
  & & \frown
            \langle\to,\alpha\rangle\frown \mbox{\em Vs}\restriction n\frown
            \langle \mbox{\em Vs}_{\mbox{rnk}(\alpha)}\rangle \\
 & & \frown
            \mbox{\em Vs}\restriction(\mbox{rnk}(\alpha)\setminus(n+1))\frown
            \langle),)\rangle\rangle
\end{eqnarray*}
where $\restriction$ denotes function domain restriction and $\setminus$
denotes set difference.  Recall that a subscript on $\mbox{\em Vs}$
denotes one of its terms.  (In the above two axiom sets commas are placed
between successive terms of sequences to prevent ambiguity, and if you examine
them with care you will be able to distinguish those parentheses that denote
constant symbols from those of our expository language that delimit function
arguments.  Although it might have been better to use boldface for our
primitive symbols, unfortunately boldface was not available for all characters
on the \LaTeX\ system used to typeset this text.)  These seemingly forbidding
axioms can be understood by analogy to concatenation of substrings in a
computer language.  They are actually relatively simple for each specific case
and will become clearer by looking at the special case of a binary predicate
$\alpha = R$ where $\mbox{rnk}(R)=2$.  Letting $\mbox{\em Vs}$ be the sequence
$\langle x,y,z,\ldots\rangle$, the axioms we would add to $\Gamma$ for this
case would be the wff extension and two equality axioms that are the
reducts of the pre-statements:
\begin{list}{}{\itemsep 0.0pt}
      \item[] $\langle\varnothing,T,\varnothing,
               \langle \mbox{wff\ }R x y\rangle\rangle$
      \item[] $\langle\varnothing,T,\varnothing,
               \langle \vdash(x=z
                  \to(R x y \to R z y))
               \rangle\rangle$
      \item[] $\langle\varnothing,T,\varnothing,
               \langle \vdash(y=z
                  \to(R x y \to R x z))
               \rangle\rangle$
\end{list}
Study these carefully to see how the general axioms above evaluate to
them.  In practice, typically only a few special cases such as this would be
needed, and in any case the Metamath language will only permit us to describe
a finite number of predicates, as opposed to the infinite number permitted by
the formal system.  (If an infinite number should be needed for some reason,
we could not define the formal system directly in the Metamath language but
could instead define it metalogically under set theory as we
do in this appendix, and only the underlying set theory, with its single
binary predicate, would be defined directly in the Metamath language.)


{\footnotesize\begin{quotation}
{\em Comment.}  As we noted earlier, the specific variables denoted by the
symbols $x,y,z,\ldots\in \mbox{\em Vr}\subseteq \mbox{\em VR}\subseteq
\mbox{\em SM}$ in Example~2 are not the actual variables of ordinary predicate
calculus but should be thought of as metavariables ranging over them.  For
example, a distinct-variable restriction would be meaningless for actual
variables of ordinary predicate calculus since two different actual variables
are by definition distinct.  And when we talk about an arbitrary
representative $\alpha\in \mbox{\em Vr}$, $\alpha$ is a metavariable (in our
expository language) that ranges over metavariables (which are primitives of
our formal system) each of which ranges over the actual individual variables
of predicate calculus (which are never mentioned in our formal system).

The constant called ``var'' above is called \texttt{setvar} in the
\texttt{set.mm} database file, but it means the same thing.  I felt
that ``var'' is a more meaningful name in the context of predicate
calculus, whose use is not limited to set theory.  For consistency we
stick with the name ``var'' throughout this Appendix, even after set
theory is introduced.
\end{quotation}}

\subsection{Free Variables and Proper Substitution}\index{free variable}
\index{proper substitution}\index{substitution!proper}

Typical representations of mathematical axioms use concepts such
as ``free variable,'' ``bound variable,'' and ``proper substitution''
as primitive notions.
A free variable is a variable that
is not a parameter of any container expression.
A bound variable is the opposite of a free variable; it is a
a variable that has been bound in a container expression.
For example, in the expression $\forall x \varphi$ (for all $x$, $\varphi$
is true), the variable $x$
is bound within the for-all ($\forall$) expression.
It is possible to change one variable to another, and that process is called
``proper substitution.''
In most books, proper substitution has a somewhat complicated recursive
definition with multiple cases based on the occurrences of free and
bound variables.
You may consult
\cite[ch.\ 3--4]{Hamilton}\index{Hamilton, Alan G.} (as well as
many other texts) for more formal details about these terms.

Using these concepts as \texttt{primitives} creates complications
for computer implementations.

In the system of Example~2, there are no primitive notions of free variable
and proper substitution.  Tarski \cite{Tarski1965} shows that this system is
logically equivalent to the more typical textbook systems that do have these
primitive notions, if we introduce these notions with appropriate definitions
and metalogic.  We could also define axioms for such systems directly,
although the recursive definitions of free variable and proper substitution
would be messy and awkward to work with.  Instead, we mention two devices that
can be used in practice to mimic these notions.  (1) Instead of introducing
special notation to express (as a logical hypothesis) ``where $x$ is not free
in $\varphi$'' we can use the logical hypothesis $\vdash(\varphi\to\forall
x\,\varphi)$.\label{effectivelybound}\index{effectively
not free}\footnote{This is a slightly weaker requirement than ``where $x$ is
not free in $\varphi$.''  If we let $\varphi$ be $x=x$, we have the theorem
$(x=x\to\forall x\,x=x)$ which satisfies the hypothesis, even though $x$ is
free in $x=x$ .  In a case like this we say that $x$ is {\em effectively not
free}\index{effectively not free} in $x=x$, since $x=x$ is logically
equivalent to $\forall x\,x=x$ in which $x$ is bound.} (2) It can be shown
that the wff $((x=y\to\varphi)\wedge\exists x(x=y\wedge\varphi))$ (with the
usual definitions of $\wedge$ and $\exists$; see Example~4 below) is logically
equivalent to ``the wff that results from proper substitution of $y$ for $x$
in $\varphi$.''  This works whether or not $x$ and $y$ are distinct.

\subsection{Metalogical Completeness}\index{metalogical completeness}

In the system of Example~2, the
following are provable pre-statements (and their reducts are
provable statements):
\begin{eqnarray*}
      & \langle\{\{x,y\}\},T,\varnothing,
               \langle \vdash\lnot\forall x\lnot x=y
               \rangle\rangle & \\
     &  \langle\varnothing,T,\varnothing,
               \langle \vdash\lnot\forall x\lnot x=x
               \rangle\rangle &
\end{eqnarray*}
whereas the following pre-statement is not to my knowledge provable (but
in any case we will pretend it's not for sake of illustration):
\begin{eqnarray*}
     &  \langle\varnothing,T,\varnothing,
               \langle \vdash\lnot\forall x\lnot x=y
               \rangle\rangle &
\end{eqnarray*}
In other words, we can prove ``$\lnot\forall x\lnot x=y$ where $x$ and $y$ are
distinct'' and separately prove ``$\lnot\forall x\lnot x=x$'', but we can't
prove the combined general case ``$\lnot\forall x\lnot x=y$'' that has no
proviso.  Now this does not compromise logical completeness, because the
variables are really metavariables and the two provable cases together cover
all possible cases.  The third case can be considered a metatheorem whose
direct proof, using the system of Example~2, lies outside the capability of the
formal system.

Also, in the system of Example~2 the following pre-statement is not to my
knowledge provable (again, a conjecture that we will pretend to be the case):
\begin{eqnarray*}
     & \langle\varnothing,T,\varnothing,
               \langle \vdash(\forall x\, \varphi\to\varphi)
               \rangle\rangle &
\end{eqnarray*}
Instead, we can only prove specific cases of $\varphi$ involving individual
metavariables, and by induction on formula length, prove as a metatheorem
outside of our formal system the general statement above.  The details of this
proof are found in \cite{Kalish}.

There does, however, exist a system of predicate calculus in which all such
``simple metatheorems'' as those above can be proved directly, and we present
it in Example~3. A {\em simple metatheorem}\index{simple metatheorem}
is any statement of the formal
system of Example~2 where all distinct variable restrictions consist of either
two individual metavariables or an individual metavariable and a wff
metavariable, and which is provable by combining cases outside the system as
above.  A system is {\em metalogically complete}\index{metalogical
completeness} if all of its simple
metatheorems are (directly) provable statements. The precise definition of
``simple metatheorem'' and the proof of the ``metalogical completeness'' of
Example~3 is found in Remark 9.6 and Theorem 9.7 of \cite{Megill}.\index{Megill,
Norman}

\begin{sloppy}
\subsection{Example~3---Metalogically Complete Predicate
Calculus with
Equality}
\end{sloppy}

For simplicity we will assume there is one binary predicate $R$;
this system suffices for set theory, where the $R$ is of course the $\in$
predicate.  We label the axioms as they appear in \cite{Megill}.  This
system is logically equivalent to that of Example~2 (when the latter is
restricted to this single binary predicate) but is also metalogically
complete.\index{metalogical completeness}

Let
\begin{itemize}
  \item[] $\mbox{\em CN}=\{\mbox{wff}, \mbox{var}, \vdash, \to, \lnot, (,),\forall,=,R\}$.
  \item[] $\mbox{\em VR}=\{\varphi,\psi,\chi,\ldots\}\cup\{x,y,z,\ldots\}$.
  \item[] $T = \{\langle \mbox{wff\ } \varphi\rangle,
             \langle \mbox{wff\ } \psi\rangle,
             \langle \mbox{wff\ } \chi\rangle,\ldots\}\cup
       \{\langle \mbox{var\ } x\rangle, \langle \mbox{var\ } y\rangle, \langle
       \mbox{var\ }z\rangle,\ldots\}$.

\noindent Then
  $\Gamma$ consists of the reducts of the following pre-statements:
    \begin{itemize}
      \item[] $\langle\varnothing,T,\varnothing,
               \langle \mbox{wff\ }(\varphi\to\psi)\rangle\rangle$
      \item[] $\langle\varnothing,T,\varnothing,
               \langle \mbox{wff\ }\lnot\varphi\rangle\rangle$
      \item[] $\langle\varnothing,T,\varnothing,
               \langle \mbox{wff\ }\forall x\,\varphi\rangle\rangle$
      \item[] $\langle\varnothing,T,\varnothing,
               \langle \mbox{wff\ }x=y\rangle\rangle$
      \item[] $\langle\varnothing,T,\varnothing,
               \langle \mbox{wff\ }Rxy\rangle\rangle$
      \item[(C1$'$)] $\langle\varnothing,T,\varnothing,
               \langle \vdash(\varphi\to(\psi\to\varphi))
               \rangle\rangle$
      \item[(C2$'$)] $\langle\varnothing,T,
               \varnothing,
               \langle \vdash((\varphi\to(\psi\to\chi))\to
               ((\varphi\to\psi)\to(\varphi\to\chi)))
               \rangle\rangle$
      \item[(C3$'$)] $\langle\varnothing,T,
               \varnothing,
               \langle \vdash((\lnot\varphi\to\lnot\psi)\to
               (\psi\to\varphi))\rangle\rangle$
      \item[(C4$'$)] $\langle\varnothing,T,
               \varnothing,
               \langle \vdash(\forall x(\forall x\,\varphi\to\psi)\to
                 (\forall x\,\varphi\to\forall x\,\psi))\rangle\rangle$
      \item[(C5$'$)] $\langle\varnothing,T,
               \varnothing,
               \langle \vdash(\forall x\,\varphi\to\varphi)\rangle\rangle$
      \item[(C6$'$)] $\langle\varnothing,T,
               \varnothing,
               \langle \vdash(\forall x\forall y\,\varphi\to
                 \forall y\forall x\,\varphi)\rangle\rangle$
      \item[(C7$'$)] $\langle\varnothing,T,
               \varnothing,
               \langle \vdash(\lnot\varphi\to\forall x\lnot\forall x\,\varphi
                 )\rangle\rangle$
      \item[(C8$'$)] $\langle\varnothing,T,
               \varnothing,
               \langle \vdash(x=y\to(x=z\to y=z))\rangle\rangle$
      \item[(C9$'$)] $\langle\varnothing,T,
               \varnothing,
               \langle \vdash(\lnot\forall x\, x=y\to(\lnot\forall x\, x=z\to
                 (y=z\to\forall x\, y=z)))\rangle\rangle$
      \item[(C10$'$)] $\langle\varnothing,T,
               \varnothing,
               \langle \vdash(\forall x(x=y\to\forall x\,\varphi)\to
                 \varphi))\rangle\rangle$
      \item[(C11$'$)] $\langle\varnothing,T,
               \varnothing,
               \langle \vdash(\forall x\, x=y\to(\forall x\,\varphi
               \to\forall y\,\varphi))\rangle\rangle$
      \item[(C12$'$)] $\langle\varnothing,T,
               \varnothing,
               \langle \vdash(x=y\to(Rxz\to Ryz))\rangle\rangle$
      \item[(C13$'$)] $\langle\varnothing,T,
               \varnothing,
               \langle \vdash(x=y\to(Rzx\to Rzy))\rangle\rangle$
      \item[(C15$'$)] $\langle\varnothing,T,
               \varnothing,
               \langle \vdash(\lnot\forall x\, x=y\to(x=y\to(\varphi
                 \to\forall x(x=y\to\varphi))))\rangle\rangle$
      \item[(C16$'$)] $\langle\{\{x,y\}\},T,
               \varnothing,
               \langle \vdash(\forall x\, x=y\to(\varphi\to\forall x\,\varphi)
                 )\rangle\rangle$
      \item[(C5)] $\langle\{\{x,\varphi\}\},T,\varnothing,
               \langle \vdash(\varphi\to\forall x\,\varphi)
               \rangle\rangle$
      \item[(MP)] $\langle\varnothing,T,
               \{\langle\vdash(\varphi\to\psi)\rangle,
                 \langle\vdash\varphi\rangle\},
               \langle\vdash\psi\rangle\rangle$
      \item[(Gen)] $\langle\varnothing,T,
               \{\langle\vdash\varphi\rangle\},
               \langle\vdash\forall x\,\varphi\rangle\rangle$
    \end{itemize}
\end{itemize}

While it is known that these axioms are ``metalogically complete,'' it is
not known whether they are independent (i.e.\ none is
redundant) in the metalogical sense; specifically, whether any axiom (possibly
with additional non-mandatory distinct-variable restrictions, for use with any
dummy variables in its proof) is provable from the others.  Note that
metalogical independence is a weaker requirement than independence in the
usual logical sense.  Not all of the above axioms are logically independent:
for example, C9$'$ can be proved as a metatheorem from the others, outside the
formal system, by combining the possible cases of distinct variables.

\subsection{Example~4---Adding Definitions}\index{definition}
There are several ways to add definitions to a formal system.  Probably the
most proper way is to consider definitions not as part of the formal system at
all but rather as abbreviations that are part of the expository metalogic
outside the formal system.  For convenience, though, we may use the formal
system itself to incorporate definitions, adding them as axiomatic extensions
to the system.  This could be done by adding a constant representing the
concept ``is defined as'' along with axioms for it. But there is a nicer way,
at least in this writer's opinion, that introduces definitions as direct
extensions to the language rather than as extralogical primitive notions.  We
introduce additional logical connectives and provide axioms for them.  For
systems of logic such as Examples 1 through 3, the additional axioms must be
conservative in the sense that no wff of the original system that was not a
theorem (when the initial term ``wff'' is replaced by ``$\vdash$'' of course)
becomes a theorem of the extended system.  In this example we extend Example~3
(or 2) with standard abbreviations of logic.

We extend $\mbox{\em CN}$ of Example~3 with new constants $\{\leftrightarrow,
\wedge,\vee,\exists\}$, corresponding to logical equivalence,\index{logical
equivalence ($\leftrightarrow$)}\index{biconditional ($\leftrightarrow$)}
conjunction,\index{conjunction ($\wedge$)} disjunction,\index{disjunction
($\vee$)} and the existential quantifier.\index{existential quantifier
($\exists$)}  We extend $\Gamma$ with the axiomatic statements that are
the reducts of the following pre-statements:
\begin{list}{}{\itemsep 0.0pt}
      \item[] $\langle\varnothing,T,\varnothing,
               \langle \mbox{wff\ }(\varphi\leftrightarrow\psi)\rangle\rangle$
      \item[] $\langle\varnothing,T,\varnothing,
               \langle \mbox{wff\ }(\varphi\vee\psi)\rangle\rangle$
      \item[] $\langle\varnothing,T,\varnothing,
               \langle \mbox{wff\ }(\varphi\wedge\psi)\rangle\rangle$
      \item[] $\langle\varnothing,T,\varnothing,
               \langle \mbox{wff\ }\exists x\, \varphi\rangle\rangle$
  \item[] $\langle\varnothing,T,\varnothing,
     \langle\vdash ( ( \varphi \leftrightarrow \psi ) \to
     ( \varphi \to \psi ) )\rangle\rangle$
  \item[] $\langle\varnothing,T,\varnothing,
     \langle\vdash ((\varphi\leftrightarrow\psi)\to
    (\psi\to\varphi))\rangle\rangle$
  \item[] $\langle\varnothing,T,\varnothing,
     \langle\vdash ((\varphi\to\psi)\to(
     (\psi\to\varphi)\to(\varphi
     \leftrightarrow\psi)))\rangle\rangle$
  \item[] $\langle\varnothing,T,\varnothing,
     \langle\vdash (( \varphi \wedge \psi ) \leftrightarrow\neg ( \varphi
     \to \neg \psi )) \rangle\rangle$
  \item[] $\langle\varnothing,T,\varnothing,
     \langle\vdash (( \varphi \vee \psi ) \leftrightarrow (\neg \varphi
     \to \psi )) \rangle\rangle$
  \item[] $\langle\varnothing,T,\varnothing,
     \langle\vdash (\exists x \,\varphi\leftrightarrow
     \lnot \forall x \lnot \varphi)\rangle\rangle$
\end{list}
The first three logical axioms (statements containing ``$\vdash$'') introduce
and effectively define logical equivalence, ``$\leftrightarrow$''.  The last
three use ``$\leftrightarrow$'' to effectively mean ``is defined as.''

\subsection{Example~5---ZFC Set Theory}\index{ZFC set theory}

Here we add to the system of Example~4 the axioms of Zermelo--Fraenkel set
theory with Choice.  For convenience we make use of the
definitions in Example~4.

In the $\mbox{\em CN}$ of Example~4 (which extends Example~3), we replace the symbol $R$
with the symbol $\in$.
More explicitly, we remove from $\Gamma$ of Example~4 the three
axiomatic statements containing $R$ and replace them with the
reducts of the following:
\begin{list}{}{\itemsep 0.0pt}
      \item[] $\langle\varnothing,T,\varnothing,
               \langle \mbox{wff\ }x\in y\rangle\rangle$
      \item[] $\langle\varnothing,T,
               \varnothing,
               \langle \vdash(x=y\to(x\in z\to y\in z))\rangle\rangle$
      \item[] $\langle\varnothing,T,
               \varnothing,
               \langle \vdash(x=y\to(z\in x\to z\in y))\rangle\rangle$
\end{list}
Letting $D=\{\{\alpha,\beta\}\in \mbox{\em DV}\,|\alpha,\beta\in \mbox{\em
Vr}\}$ (in other words all individual variables must be distinct), we extend
$\Gamma$ with the ZFC axioms, called
\index{Axiom of Extensionality}
\index{Axiom of Replacement}
\index{Axiom of Union}
\index{Axiom of Power Sets}
\index{Axiom of Regularity}
\index{Axiom of Infinity}
\index{Axiom of Choice}
Extensionality, Replacement, Union, Power
Set, Regularity, Infinity, and Choice, that are the reducts of:
\begin{list}{}{\itemsep 0.0pt}
      \item[Ext] $\langle D,T,
               \varnothing,
               \langle\vdash (\forall x(x\in y\leftrightarrow x \in z)\to y
               =z) \rangle\rangle$
      \item[Rep] $\langle D,T,
               \varnothing,
               \langle\vdash\exists x ( \exists y \forall z (\varphi \to z = y
                        ) \to
                        \forall z ( z \in x \leftrightarrow \exists x ( x \in
                        y \wedge \forall y\,\varphi ) ) )\rangle\rangle$
      \item[Un] $\langle D,T,
               \varnothing,
               \langle\vdash \exists x \forall y ( \exists x ( y \in x \wedge
               x \in z ) \to y \in x ) \rangle\rangle$
      \item[Pow] $\langle D,T,
               \varnothing,
               \langle\vdash \exists x \forall y ( \forall x ( x \in y \to x
               \in z ) \to y \in x ) \rangle\rangle$
      \item[Reg] $\langle D,T,
               \varnothing,
               \langle\vdash (  x \in y \to
                 \exists x ( x \in y \wedge \forall z ( z \in x \to \lnot z
                \in y ) ) ) \rangle\rangle$
      \item[Inf] $\langle D,T,
               \varnothing,
               \langle\vdash \exists x(y\in x\wedge\forall y(y\in
               x\to
               \exists z(y \in z\wedge z\in x))) \rangle\rangle$
      \item[AC] $\langle D,T,
               \varnothing,
               \langle\vdash \exists x \forall y \forall z ( ( y \in z
               \wedge z \in w ) \to \exists w \forall y ( \exists w
              ( ( y \in z \wedge z \in w ) \wedge ( y \in w \wedge w \in x
              ) ) \leftrightarrow y = w ) ) \rangle\rangle$
\end{list}

\subsection{Example~6---Class Notation in Set Theory}\label{class}

A powerful device that makes set theory easier (and that we have
been using all along in our informal expository language) is {\em class
abstraction notation}.\index{class abstraction}\index{abstraction class}  The
definitions we introduce are rigorously justified
as conservative by Takeuti and Zaring \cite{Takeuti}\index{Takeuti, Gaisi} or
Quine \cite{Quine}\index{Quine, Willard Van Orman}.  The key idea is to
introduce the notation $\{x|\mbox{---}\}$ which means ``the class of all $x$
such that ---'' for abstraction classes and introduce (meta)variables that
range over them.  An abstraction class may or may not be a set, depending on
whether it exists (as a set).  A class that does not exist is
called a {\em proper class}.\index{proper class}\index{class!proper}

To illustrate the use of abstraction classes we will provide some examples
of definitions that make use of them:  the empty set, class union, and
unordered pair.  Many other such definitions can be found in the
Metamath set theory database,
\texttt{set.mm}.\index{set theory database (\texttt{set.mm})}

% We intentionally break up the sequence of math symbols here
% because otherwise the overlong line goes beyond the page in narrow mode.
We extend $\mbox{\em CN}$ of Example~5 with new symbols $\{$
$\mbox{class},$ $\{,$ $|,$ $\},$ $\varnothing,$ $\cup,$ $,$ $\}$
where the inner braces and last comma are
constant symbols. (As before,
our dual use of some mathematical symbols for both our expository
language and as primitives of the formal system should be clear from context.)

We extend $\mbox{\em VR}$ of Example~5 with a set of {\em class
variables}\index{class variable}
$\{A,B,C,\ldots\}$. We extend the $T$ of Example~5 with $\{\langle
\mbox{class\ } A\rangle, \langle \mbox{class\ }B\rangle, \langle \mbox{class\ }
C\rangle,\ldots\}$.

To
introduce our definitions,
we add to $\Gamma$ of Example~5 the axiomatic statements
that are the reducts of the following pre-statements:
\begin{list}{}{\itemsep 0.0pt}
      \item[] $\langle\varnothing,T,\varnothing,
               \langle \mbox{class\ }x\rangle\rangle$
      \item[] $\langle\varnothing,T,\varnothing,
               \langle \mbox{class\ }\{x|\varphi\}\rangle\rangle$
      \item[] $\langle\varnothing,T,\varnothing,
               \langle \mbox{wff\ }A=B\rangle\rangle$
      \item[] $\langle\varnothing,T,\varnothing,
               \langle \mbox{wff\ }A\in B\rangle\rangle$
      \item[Ab] $\langle\varnothing,T,\varnothing,
               \langle \vdash ( y \in \{ x |\varphi\} \leftrightarrow
                  ( ( x = y \to\varphi) \wedge \exists x ( x = y
                  \wedge\varphi) ))
               \rangle\rangle$
      \item[Eq] $\langle\{\{x,A\},\{x,B\}\},T,\varnothing,
               \langle \vdash ( A = B \leftrightarrow
               \forall x ( x \in A \leftrightarrow x \in B ) )
               \rangle\rangle$
      \item[El] $\langle\{\{x,A\},\{x,B\}\},T,\varnothing,
               \langle \vdash ( A \in B \leftrightarrow \exists x
               ( x = A \wedge x \in B ) )
               \rangle\rangle$
\end{list}
Here we say that an individual variable is a class; $\{x|\varphi\}$ is a
class; and we extend the definition of a wff to include class equality and
membership.  Axiom Ab defines membership of a variable in a class abstraction;
the right-hand side can be read as ``the wff that results from proper
substitution of $y$ for $x$ in $\varphi$.''\footnote{Note that this definition
makes unnecessary the introduction of a separate notation similar to
$\varphi(x|y)$ for proper substitution, although we may choose to do so to be
conventional.  Incidentally, $\varphi(x|y)$ as it stands would be ambiguous in
the formal systems of our examples, since we wouldn't know whether
$\lnot\varphi(x|y)$ meant $\lnot(\varphi(x|y))$ or $(\lnot\varphi)(x|y)$.
Instead, we would have to use an unambiguous variant such as $(\varphi\,
x|y)$.}  Axioms Eq and El extend the meaning of the existing equality and
membership connectives.  This is potentially dangerous and requires careful
justification.  For example, from Eq we can derive the Axiom of Extensionality
with predicate logic alone; thus in principle we should include the Axiom of
Extensionality as a logical hypothesis.  However we do not bother to do this
since we have already presupposed that axiom earlier. The distinct variable
restrictions should be read ``where $x$ does not occur in $A$ or $B$.''  We
typically do this when the right-hand side of a definition involves an
individual variable not in the expression being defined; it is done so that
the right-hand side remains independent of the particular ``dummy'' variable
we use.

We continue to add to $\Gamma$ the following definitions
(i.e. the reducts of the following pre-statements) for empty
set,\index{empty set} class union,\index{union} and unordered
pair.\index{unordered pair}  They should be self-explanatory.  Analogous to our
use of ``$\leftrightarrow$'' to define new wffs in Example~4, we use ``$=$''
to define new abstraction terms, and both may be read informally as ``is
defined as'' in this context.
\begin{list}{}{\itemsep 0.0pt}
      \item[] $\langle\varnothing,T,\varnothing,
               \langle \mbox{class\ }\varnothing\rangle\rangle$
      \item[] $\langle\varnothing,T,\varnothing,
               \langle \vdash \varnothing = \{ x | \lnot x = x \}
               \rangle\rangle$
      \item[] $\langle\varnothing,T,\varnothing,
               \langle \mbox{class\ }(A\cup B)\rangle\rangle$
      \item[] $\langle\{\{x,A\},\{x,B\}\},T,\varnothing,
               \langle \vdash ( A \cup B ) = \{ x | ( x \in A \vee x \in B ) \}
               \rangle\rangle$
      \item[] $\langle\varnothing,T,\varnothing,
               \langle \mbox{class\ }\{A,B\}\rangle\rangle$
      \item[] $\langle\{\{x,A\},\{x,B\}\},T,\varnothing,
               \langle \vdash \{ A , B \} = \{ x | ( x = A \vee x = B ) \}
               \rangle\rangle$
\end{list}

\section{Metamath as a Formal System}\label{theorymm}

This section presupposes a familiarity with the Metamath computer language.

Our theory describes formal systems and their universes.  The Metamath
language provides a way of representing these set-theoretical objects to
a computer.  A Metamath database, being a finite set of {\sc ascii}
characters, can usually describe only a subset of a formal system and
its universe, which are typically infinite.  However the database can
contain as large a finite subset of the formal system and its universe
as we wish.  (Of course a Metamath set theory database can, in
principle, indirectly describe an entire infinite formal system by
formalizing the expository language in this Appendix.)

For purpose of our discussion, we assume the Metamath database
is in the simple form described on p.~\pageref{framelist},
consisting of all constant and variable declarations at the beginning,
followed by a sequence of extended frames each
delimited by \texttt{\$\char`\{} and \texttt{\$\char`\}}.  Any Metamath database can
be converted to this form, as described on p.~\pageref{frameconvert}.

The math symbol tokens of a Metamath source file, which are declared
with \texttt{\$c} and \texttt{\$v} statements, are names we assign to
representatives of $\mbox{\em CN}$ and $\mbox{\em VR}$.  For
definiteness we could assume that the first math symbol declared as a
variable corresponds to $v_0$, the second to $v_1$, etc., although the
exact correspondence we choose is not important.

In the Metamath language, each \texttt{\$d}, \texttt{\$f}, and
 \texttt{\$e} source
statement in an extended frame (Section~\ref{frames})
corresponds respectively to a member of the
collections $D$, $T$, and $H$ in a formal system statement $\langle
D_M,T_M,H,A\rangle$.  The math symbol strings following these Metamath keywords
correspond to a variable pair (in the case of \texttt{\$d}) or an expression (for
the other two keywords). The math symbol string following a \texttt{\$a} source
statement corresponds to expression $A$ in an axiomatic statement of the
formal system; the one following a \texttt{\$p} source statement corresponds to
$A$ in a provable statement that is not axiomatic.  In other words, each
extended frame in a Metamath database corresponds to
a pre-statement of the formal system, and a frame corresponds to
a statement of the formal system.  (Don't confuse the two meanings of
``statement'' here.  A statement of the formal system corresponds to the
several statements in a Metamath database that may constitute a
frame.)

In order for the computer to verify that a formal system statement is
provable, each \texttt{\$p} source statement is accompanied by a proof.
However, the proof does not correspond to anything in the formal system
but is simply a way of communicating to the computer the information
needed for its verification.  The proof tells the computer {\em how to
construct} specific members of closure of the formal system
pre-statement corresponding to the extended frame of the \texttt{\$p}
statement.  The final result of the construction is the member of the
closure that matches the \texttt{\$p} statement.  The abstract formal
system, on the other hand, is concerned only with the {\em existence} of
members of the closure.

As mentioned on p.~\pageref{exampleref}, Examples 1 and 3--6 in the
previous Section parallel the development of logic and set theory in the
Metamath database
\texttt{set.mm}.\index{set theory database (\texttt{set.mm})} You may
find it instructive to compare them.


\chapter{The MIU System}
\label{MIU}
\index{formal system}
\index{MIU-system}

The following is a listing of the file \texttt{miu.mm}.  It is self-explanatory.

%%%%%%%%%%%%%%%%%%%%%%%%%%%%%%%%%%%%%%%%%%%%%%%%%%%%%%%%%%%%

\begin{verbatim}
$( The MIU-system:  A simple formal system $)

$( Note:  This formal system is unusual in that it allows
empty wffs.  To work with a proof, you must type
SET EMPTY_SUBSTITUTION ON before using the PROVE command.
By default, this is OFF in order to reduce the number of
ambiguous unification possibilities that have to be selected
during the construction of a proof.  $)

$(
Hofstadter's MIU-system is a simple example of a formal
system that illustrates some concepts of Metamath.  See
Douglas R. Hofstadter, _Goedel, Escher, Bach:  An Eternal
Golden Braid_ (Vintage Books, New York, 1979), pp. 33ff. for
a description of the MIU-system.

The system has 3 constant symbols, M, I, and U.  The sole
axiom of the system is MI. There are 4 rules:
     Rule I:  If you possess a string whose last letter is I,
     you can add on a U at the end.
     Rule II:  Suppose you have Mx.  Then you may add Mxx to
     your collection.
     Rule III:  If III occurs in one of the strings in your
     collection, you may make a new string with U in place
     of III.
     Rule IV:  If UU occurs inside one of your strings, you
     can drop it.
Unfortunately, Rules III and IV do not have unique results:
strings could have more than one occurrence of III or UU.
This requires that we introduce the concept of an "MIU
well-formed formula" or wff, which allows us to construct
unique symbol sequences to which Rules III and IV can be
applied.
$)

$( First, we declare the constant symbols of the language.
Note that we need two symbols to distinguish the assertion
that a sequence is a wff from the assertion that it is a
theorem; we have arbitrarily chosen "wff" and "|-". $)
      $c M I U |- wff $. $( Declare constants $)

$( Next, we declare some variables. $)
     $v x y $.

$( Throughout our theory, we shall assume that these
variables represent wffs. $)
 wx   $f wff x $.
 wy   $f wff y $.

$( Define MIU-wffs.  We allow the empty sequence to be a
wff. $)

$( The empty sequence is a wff. $)
 we   $a wff $.
$( "M" after any wff is a wff. $)
 wM   $a wff x M $.
$( "I" after any wff is a wff. $)
 wI   $a wff x I $.
$( "U" after any wff is a wff. $)
 wU   $a wff x U $.

$( Assert the axiom. $)
 ax   $a |- M I $.

$( Assert the rules. $)
 ${
   Ia   $e |- x I $.
$( Given any theorem ending with "I", it remains a theorem
if "U" is added after it.  (We distinguish the label I_
from the math symbol I to conform to the 24-Jun-2006
Metamath spec.) $)
   I_    $a |- x I U $.
 $}
 ${
IIa  $e |- M x $.
$( Given any theorem starting with "M", it remains a theorem
if the part after the "M" is added again after it. $)
   II   $a |- M x x $.
 $}
 ${
   IIIa $e |- x I I I y $.
$( Given any theorem with "III" in the middle, it remains a
theorem if the "III" is replaced with "U". $)
   III  $a |- x U y $.
 $}
 ${
   IVa  $e |- x U U y $.
$( Given any theorem with "UU" in the middle, it remains a
theorem if the "UU" is deleted. $)
   IV   $a |- x y $.
  $}

$( Now we prove the theorem MUIIU.  You may be interested in
comparing this proof with that of Hofstadter (pp. 35 - 36).
$)
 theorem1  $p |- M U I I U $=
      we wM wU wI we wI wU we wU wI wU we wM we wI wU we wM
      wI wI wI we wI wI we wI ax II II I_ III II IV $.
\end{verbatim}\index{well-formed formula (wff)}

The \texttt{show proof /lemmon/renumber} command
yields the following display.  It is very similar
to the one in \cite[pp.~35--36]{Hofstadter}.\index{Hofstadter, Douglas R.}

\begin{verbatim}
1 ax             $a |- M I
2 1 II           $a |- M I I
3 2 II           $a |- M I I I I
4 3 I_           $a |- M I I I I U
5 4 III          $a |- M U I U
6 5 II           $a |- M U I U U I U
7 6 IV           $a |- M U I I U
\end{verbatim}

We note that Hofstadter's ``MU-puzzle,'' which asks whether
MU is a theorem of the MIU-system, cannot be answered using
the system above because the MU-puzzle is a question {\em
about} the system.  To prove the answer to the MU-puzzle,
a much more elaborate system is needed, namely one that
models the MIU-system within set theory.  (Incidentally, the
answer to the MU-puzzle is no.)

\chapter{Metamath Language EBNF}%
\label{BNF}%
\index{Metamath Language EBNF}

The following is a formal description of the basic Metamath language syntax
(with compressed proofs and support for unknown proof steps).
It is defined using the
Extended Backus--Naur Form (EBNF)\index{Extended Backus--Naur Form}\index{EBNF}
notation from W3C\index{W3C}
\textit{Extensible Markup Language (XML) 1.0 (Fifth Edition)}
(W3C Recommendation 26 November 2008) at
\url{https://www.w3.org/TR/xml/#sec-notation}.

The \texttt{database}
rule is processed until the end of the file (\texttt{EOF}).
The rules eventually require reading whitespace-separated tokens.
A token has an upper-case definition (see below)
or is a string constant in a non-token (such as \texttt{'\$a'}).
We intend for this to be correct, but if there is a conflict the
rules of section \ref{spec} govern. That section also discusses
non-syntax restrictions not shown here
(e.g., that each new label token
defined in a \texttt{hypothesis-stmt} or \texttt{assert-stmt}
must be unique).

\begin{verbatim}
database ::= outermost-scope-stmt*

outermost-scope-stmt ::=
  include-stmt | constant-stmt | stmt

/* File inclusion command; process file as a database.
   Databases should NOT have a comment in the filename. */
include-stmt ::= '$[' filename '$]'

/* Constant symbols declaration. */
constant-stmt ::= '$c' constant+ '$.'

/* A normal statement can occur in any scope. */
stmt ::= block | variable-stmt | disjoint-stmt |
  hypothesis-stmt | assert-stmt

/* A block. You can have 0 statements in a block. */
block ::= '${' stmt* '$}'

/* Variable symbols declaration. */
variable-stmt ::= '$v' variable+ '$.'

/* Disjoint variables. Simple disjoint statements have
   2 variables, i.e., "variable*" is empty for them. */
disjoint-stmt ::= '$d' variable variable variable* '$.'

hypothesis-stmt ::= floating-stmt | essential-stmt

/* Floating (variable-type) hypothesis. */
floating-stmt ::= LABEL '$f' typecode variable '$.'

/* Essential (logical) hypothesis. */
essential-stmt ::= LABEL '$e' typecode MATH-SYMBOL* '$.'

assert-stmt ::= axiom-stmt | provable-stmt

/* Axiomatic assertion. */
axiom-stmt ::= LABEL '$a' typecode MATH-SYMBOL* '$.'

/* Provable assertion. */
provable-stmt ::= LABEL '$p' typecode MATH-SYMBOL*
  '$=' proof '$.'

/* A proof. Proofs may be interspersed by comments.
   If '?' is in a proof it's an "incomplete" proof. */
proof ::= uncompressed-proof | compressed-proof
uncompressed-proof ::= (LABEL | '?')+
compressed-proof ::= '(' LABEL* ')' COMPRESSED-PROOF-BLOCK+

typecode ::= constant

filename ::= MATH-SYMBOL /* No whitespace or '$' */
constant ::= MATH-SYMBOL
variable ::= MATH-SYMBOL
\end{verbatim}

\needspace{2\baselineskip}
A \texttt{frame} is a sequence of 0 or more
\texttt{disjoint-{\allowbreak}stmt} and
\texttt{hypotheses-{\allowbreak}stmt} statements
(possibly interleaved with other non-\texttt{assert-stmt} statements)
followed by one \texttt{assert-stmt}.

\needspace{3\baselineskip}
Here are the rules for lexical processing (tokenization) beyond
the constant tokens shown above.
By convention these tokenization rules have upper-case names.
Every token is read for the longest possible length.
Whitespace-separated tokens are read sequentially;
note that the separating whitespace and \texttt{\$(} ... \texttt{\$)}
comments are skipped.

If a token definition uses another token definition, the whole thing
is considered a single token.
A pattern that is only part of a full token has a name beginning
with an underscore (``\_'').
An implementation could tokenize many tokens as a
\texttt{PRINTABLE-SEQUENCE}
and then check if it meets the more specific rule shown here.

Comments do not nest, and both \texttt{\$(} and \texttt{\$)}
have to be surrounded
by at least one whitespace character (\texttt{\_WHITECHAR}).
Technically comments end without consuming the trailing
\texttt{\_WHITECHAR}, but the trailing
\texttt{\_WHITECHAR} gets ignored anyway so we ignore that detail here.
Metamath language processors
are not required to support \texttt{\$)} followed
immediately by a bare end-of-file, because the closing
comment symbol is supposed to be followed by a
\texttt{\_WHITECHAR} such as a newline.

\begin{verbatim}
PRINTABLE-SEQUENCE ::= _PRINTABLE-CHARACTER+

MATH-SYMBOL ::= (_PRINTABLE-CHARACTER - '$')+

/* ASCII non-whitespace printable characters */
_PRINTABLE-CHARACTER ::= [#x21-#x7e]

LABEL ::= ( _LETTER-OR-DIGIT | '.' | '-' | '_' )+

_LETTER-OR-DIGIT ::= [A-Za-z0-9]

COMPRESSED-PROOF-BLOCK ::= ([A-Z] | '?')+

/* Define whitespace between tokens. The -> SKIP
   means that when whitespace is seen, it is
   skipped and we simply read again. */
WHITESPACE ::= (_WHITECHAR+ | _COMMENT) -> SKIP

/* Comments. $( ... $) and do not nest. */
_COMMENT ::= '$(' (_WHITECHAR+ (PRINTABLE-SEQUENCE - '$)'))*
  _WHITECHAR+ '$)' _WHITECHAR

/* Whitespace: (' ' | '\t' | '\r' | '\n' | '\f') */
_WHITECHAR ::= [#x20#x09#x0d#x0a#x0c]
\end{verbatim}
% This EBNF was developed as a collaboration between
% David A. Wheeler\index{Wheeler, David A.},
% Mario Carneiro\index{Carneiro, Mario}, and
% Benoit Jubin\index{Jubin, Benoit}, inspired by a request
% (and a lot of initial work) by Benoit Jubin.
%
% \chapter{Disclaimer and Trademarks}
%
% Information in this document is subject to change without notice and does not
% represent a commitment on the part of Norman Megill.
% \vspace{2ex}
%
% \noindent Norman D. Megill makes no warranties, either express or implied,
% regarding the Metamath computer software package.
%
% \vspace{2ex}
%
% \noindent Any trademarks mentioned in this book are the property of
% their respective owners.  The name ``Metamath'' is a trademark of
% Norman Megill.
%
\cleardoublepage
\phantomsection  % fixes the link anchor
\addcontentsline{toc}{chapter}{\bibname}

\bibliography{metamath}
%\input{metamath.bbl}

\raggedright
\cleardoublepage
\phantomsection % fixes the link anchor
\addcontentsline{toc}{chapter}{\indexname}
%\printindex   ??
\input{metamath.ind}

\end{document}



\end{document}



\end{document}



\raggedright
\cleardoublepage
\phantomsection % fixes the link anchor
\addcontentsline{toc}{chapter}{\indexname}
%\printindex   ??
% metamath.tex - Version of 2-Jun-2019
% If you change the date above, also change the "Printed date" below.
% SPDX-License-Identifier: CC0-1.0
%
%                              PUBLIC DOMAIN
%
% This file (specifically, the version of this file with the above date)
% has been released into the Public Domain per the
% Creative Commons CC0 1.0 Universal (CC0 1.0) Public Domain Dedication
% https://creativecommons.org/publicdomain/zero/1.0/
%
% The public domain release applies worldwide.  In case this is not
% legally possible, the right is granted to use the work for any purpose,
% without any conditions, unless such conditions are required by law.
%
% Several short, attributed quotations from copyrighted works
% appear in this file under the ``fair use'' provision of Section 107 of
% the United States Copyright Act (Title 17 of the {\em United States
% Code}).  The public-domain status of this file is not applicable to
% those quotations.
%
% Norman Megill - email: nm(at)alum(dot)mit(dot)edu
%
% David A. Wheeler also donates his improvements to this file to the
% public domain per the CC0.  He works at the Institute for Defense Analyses
% (IDA), but IDA has agreed that this Metamath work is outside its "lane"
% and is not a work by IDA.  This was specifically confirmed by
% Margaret E. Myers (Division Director of the Information Technology
% and Systems Division) on 2019-05-24 and by Ben Lindorf (General Counsel)
% on 2019-05-22.

% This file, 'metamath.tex', is self-contained with everything needed to
% generate the the PDF file 'metamath.pdf' (the _Metamath_ book) on
% standard LaTeX 2e installations.  The auxiliary files are embedded with
% "filecontents" commands.  To generate metamath.pdf file, run these
% commands under Linux or Cygwin in the directory that contains
% 'metamath.tex':
%
%   rm -f realref.sty metamath.bib
%   touch metamath.ind
%   pdflatex metamath
%   pdflatex metamath
%   bibtex metamath
%   makeindex metamath
%   pdflatex metamath
%   pdflatex metamath
%
% The warnings that occur in the initial runs of pdflatex can be ignored.
% For the final run,
%
%   egrep -i 'error|warn' metamath.log
%
% should show exactly these 5 warnings:
%
%   LaTeX Warning: File `realref.sty' already exists on the system.
%   LaTeX Warning: File `metamath.bib' already exists on the system.
%   LaTeX Font Warning: Font shape `OMS/cmtt/m/n' undefined
%   LaTeX Font Warning: Font shape `OMS/cmtt/bx/n' undefined
%   LaTeX Font Warning: Some font shapes were not available, defaults
%       substituted.
%
% Search for "Uncomment" below if you want to suppress hyperlink boxes
% in the PDF output file
%
% TYPOGRAPHICAL NOTES:
% * It is customary to use an en dash (--) to "connect" names of different
%   people (and to denote ranges), and use a hyphen (-) for a
%   single compound name. Examples of connected multiple people are
%   Zermelo--Fraenkel, Schr\"{o}der--Bernstein, Tarski--Grothendieck,
%   Hewlett--Packard, and Backus--Naur.  Examples of a single person with
%   a compound name include Levi-Civita, Mittag-Leffler, and Burali-Forti.
% * Use non-breaking spaces after page abbreviations, e.g.,
%   p.~\pageref{note2002}.
%
% --------------------------- Start of realref.sty -----------------------------
\begin{filecontents}{realref.sty}
% Save the following as realref.sty.
% You can then use it with \usepackage{realref}
%
% This has \pageref jumping to the page on which the ref appears,
% \ref jumping to the point of the anchor, and \sectionref
% jumping to the start of section.
%
% Author:  Anthony Williams
%          Software Engineer
%          Nortel Networks Optical Components Ltd
% Date:    9 Nov 2001 (posted to comp.text.tex)
%
% The following declaration was made by Anthony Williams on
% 24 Jul 2006 (private email to Norman Megill):
%
%   ``I hereby donate the code for realref.sty posted on the
%   comp.text.tex newsgroup on 9th November 2001, accessible from
%   http://groups.google.com/group/comp.text.tex/msg/5a0e1cc13ea7fbb2
%   to the public domain.''
%
\ProvidesPackage{realref}
\RequirePackage[plainpages=false,pdfpagelabels=true]{hyperref}
\def\realref@anchorname{}
\AtBeginDocument{%
% ensure every label is a possible hyperlink target
\let\realref@oldrefstepcounter\refstepcounter%
\DeclareRobustCommand{\refstepcounter}[1]{\realref@oldrefstepcounter{#1}
\edef\realref@anchorname{\string #1.\@currentlabel}%
}%
\let\realref@oldlabel\label%
\DeclareRobustCommand{\label}[1]{\realref@oldlabel{#1}\hypertarget{#1}{}%
\@bsphack\protected@write\@auxout{}{%
    \string\expandafter\gdef\protect\csname
    page@num.#1\string\endcsname{\thepage}%
    \string\expandafter\gdef\protect\csname
    ref@num.#1\string\endcsname{\@currentlabel}%
    \string\expandafter\gdef\protect\csname
    sectionref@name.#1\string\endcsname{\realref@anchorname}%
}\@esphack}%
\DeclareRobustCommand\pageref[1]{{\edef\a{\csname
            page@num.#1\endcsname}\expandafter\hyperlink{page.\a}{\a}}}%
\DeclareRobustCommand\ref[1]{{\edef\a{\csname
            ref@num.#1\endcsname}\hyperlink{#1}{\a}}}%
\DeclareRobustCommand\sectionref[1]{{\edef\a{\csname
            ref@num.#1\endcsname}\edef\b{\csname
            sectionref@name.#1\endcsname}\hyperlink{\b}{\a}}}%
}
\end{filecontents}
% ---------------------------- End of realref.sty ------------------------------

% --------------------------- Start of metamath.bib -----------------------------
\begin{filecontents}{metamath.bib}
@book{Albers, editor = "Donald J. Albers and G. L. Alexanderson",
  title = "Mathematical People",
  publisher = "Contemporary Books, Inc.",
  address = "Chicago",
  note = "[QA28.M37]",
  year = 1985 }
@book{Anderson, author = "Alan Ross Anderson and Nuel D. Belnap",
  title = "Entailment",
  publisher = "Princeton University Press",
  address = "Princeton",
  volume = 1,
  note = "[QA9.A634 1975 v.1]",
  year = 1975}
@book{Barrow, author = "John D. Barrow",
  title = "Theories of Everything:  The Quest for Ultimate Explanation",
  publisher = "Oxford University Press",
  address = "Oxford",
  note = "[Q175.B225]",
  year = 1991 }
@book{Behnke,
  editor = "H. Behnke and F. Backmann and K. Fladt and W. S{\"{u}}ss",
  title = "Fundamentals of Mathematics",
  volume = "I",
  publisher = "The MIT Press",
  address = "Cambridge, Massachusetts",
  note = "[QA37.2.B413]",
  year = 1974 }
@book{Bell, author = "J. L. Bell and M. Machover",
  title = "A Course in Mathematical Logic",
  publisher = "North-Holland",
  address = "Amsterdam",
  note = "[QA9.B3953]",
  year = 1977 }
@inproceedings{Blass, author = "Andrea Blass",
  title = "The Interaction Between Category Theory and Set Theory",
  pages = "5--29",
  booktitle = "Mathematical Applications of Category Theory (Proceedings
     of the Special Session on Mathematical Applications
     Category Theory, 89th Annual Meeting of the American Mathematical
     Society, held in Denver, Colorado January 5--9, 1983)",
  editor = "John Walter Gray",
  year = 1983,
  note = "[QA169.A47 1983]",
  publisher = "American Mathematical Society",
  address = "Providence, Rhode Island"}
@proceedings{Bledsoe, editor = "W. W. Bledsoe and D. W. Loveland",
  title = "Automated Theorem Proving:  After 25 Years (Proceedings
     of the Special Session on Automatic Theorem Proving,
     89th Annual Meeting of the American Mathematical
     Society, held in Denver, Colorado January 5--9, 1983)",
  year = 1983,
  note = "[QA76.9.A96.S64 1983]",
  publisher = "American Mathematical Society",
  address = "Providence, Rhode Island" }
@book{Boolos, author = "George S. Boolos and Richard C. Jeffrey",
  title = "Computability and Log\-ic",
  publisher = "Cambridge University Press",
  edition = "third",
  address = "Cambridge",
  note = "[QA9.59.B66 1989]",
  year = 1989 }
@book{Campbell, author = "John Campbell",
  title = "Programmer's Progress",
  publisher = "White Star Software",
  address = "Box 51623, Palo Alto, CA 94303",
  year = 1991 }
@article{DBLP:journals/corr/Carneiro14,
  author    = {Mario Carneiro},
  title     = {Conversion of {HOL} Light proofs into Metamath},
  journal   = {CoRR},
  volume    = {abs/1412.8091},
  year      = {2014},
  url       = {http://arxiv.org/abs/1412.8091},
  archivePrefix = {arXiv},
  eprint    = {1412.8091},
  timestamp = {Mon, 13 Aug 2018 16:47:05 +0200},
  biburl    = {https://dblp.org/rec/bib/journals/corr/Carneiro14},
  bibsource = {dblp computer science bibliography, https://dblp.org}
}
@article{CarneiroND,
  author    = {Mario Carneiro},
  title     = {Natural Deductions in the Metamath Proof Language},
  url       = {http://us.metamath.org/ocat/natded.pdf},
  year      = 2014
}
@inproceedings{Chou, author = "Shang-Ching Chou",
  title = "Proving Elementary Geometry Theorems Using {W}u's Algorithm",
  pages = "243--286",
  booktitle = "Automated Theorem Proving:  After 25 Years (Proceedings
     of the Special Session on Automatic Theorem Proving,
     89th Annual Meeting of the American Mathematical
     Society, held in Denver, Colorado January 5--9, 1983)",
  editor = "W. W. Bledsoe and D. W. Loveland",
  year = 1983,
  note = "[QA76.9.A96.S64 1983]",
  publisher = "American Mathematical Society",
  address = "Providence, Rhode Island" }
@book{Clemente, author = "Daniel Clemente Laboreo",
  title = "Introduction to natural deduction",
  year = 2014,
  url = "http://www.danielclemente.com/logica/dn.en.pdf" }
@incollection{Courant, author = "Richard Courant and Herbert Robbins",
  title = "Topology",
  pages = "573--590",
  booktitle = "The World of Mathematics, Volume One",
  editor = "James R. Newman",
  publisher = "Simon and Schuster",
  address = "New York",
  note = "[QA3.W67 1988]",
  year = 1956 }
@book{Curry, author = "Haskell B. Curry",
  title = "Foundations of Mathematical Logic",
  publisher = "Dover Publications, Inc.",
  address = "New York",
  note = "[QA9.C976 1977]",
  year = 1977 }
@book{Davis, author = "Philip J. Davis and Reuben Hersh",
  title = "The Mathematical Experience",
  publisher = "Birkh{\"{a}}user Boston",
  address = "Boston",
  note = "[QA8.4.D37 1982]",
  year = 1981 }
@incollection{deMillo,
  author = "Richard de Millo and Richard Lipton and Alan Perlis",
  title = "Social Processes and Proofs of Theorems and Programs",
  pages = "267--285",
  booktitle = "New Directions in the Philosophy of Mathematics",
  editor = "Thomas Tymoczko",
  publisher = "Birkh{\"{a}}user Boston, Inc.",
  address = "Boston",
  note = "[QA8.6.N48 1986]",
  year = 1986 }
@book{Edwards, author = "Robert E. Edwards",
  title = "A Formal Background to Mathematics",
  publisher = "Springer-Verlag",
  address = "New York",
  note = "[QA37.2.E38 v.1a]",
  year = 1979 }
@book{Enderton, author = "Herbert B. Enderton",
  title = "Elements of Set Theory",
  publisher = "Academic Press, Inc.",
  address = "San Diego",
  note = "[QA248.E5]",
  year = 1977 }
@book{Goodstein, author = "R. L. Goodstein",
  title = "Development of Mathematical Logic",
  publisher = "Springer-Verlag New York Inc.",
  address = "New York",
  note = "[QA9.G6554]",
  year = 1971 }
@book{Guillen, author = "Michael Guillen",
  title = "Bridges to Infinity",
  publisher = "Jeremy P. Tarcher, Inc.",
  address = "Los Angeles",
  note = "[QA93.G8]",
  year = 1983 }
@book{Hamilton, author = "Alan G. Hamilton",
  title = "Logic for Mathematicians",
  edition = "revised",
  publisher = "Cambridge University Press",
  address = "Cambridge",
  note = "[QA9.H298]",
  year = 1988 }
@unpublished{Harrison, author = "John Robert Harrison",
  title = "Metatheory and Reflection in Theorem Proving:
    A Survey and Critique",
  note = "Technical Report
    CRC-053.
    SRI Cambridge,
    Millers Yard, Cambridge, UK,
    1995.
    Available on the Web as
{\verb+http:+}\-{\verb+//www.cl.cam.ac.uk/users/jrh/papers/reflect.html+}"}
@TECHREPORT{Harrison-thesis,
        author          = "John Robert Harrison",
        title           = "Theorem Proving with the Real Numbers",
        institution   = "University of Cambridge Computer
                         Lab\-o\-ra\-to\-ry",
        address         = "New Museums Site, Pembroke Street, Cambridge,
                           CB2 3QG, UK",
        year            = 1996,
        number          = 408,
        type            = "Technical Report",
        note            = "Author's PhD thesis,
   available on the Web at
{\verb+http:+}\-{\verb+//www.cl.cam.ac.uk+}\-{\verb+/users+}\-{\verb+/jrh+}%
\-{\verb+/papers+}\-{\verb+/thesis.html+}"}
@book{Herrlich, author = "Horst Herrlich and George E. Strecker",
  title = "Category Theory:  An Introduction",
  publisher = "Allyn and Bacon Inc.",
  address = "Boston",
  note = "[QA169.H567]",
  year = 1973 }
@article{Hindley, author = "J. Roger Hindley and David Meredith",
  title = "Principal Type-Schemes and Condensed Detachment",
  journal = "The Journal of Symbolic Logic",
  volume = 55,
  year = 1990,
  note = "[QA.J87]",
  pages = "90--105" }
@book{Hofstadter, author = "Douglas R. Hofstadter",
  title = "G{\"{o}}del, Escher, Bach",
  publisher = "Basic Books, Inc.",
  address = "New York",
  note = "[QA9.H63 1980]",
  year = 1979 }
@article{Indrzejczak, author= "Andrzej Indrzejczak",
  title = "Natural Deduction, Hybrid Systems and Modal Logic",
  journal = "Trends in Logic",
  volume = 30,
  publisher = "Springer",
  year = 2010 }
@article{Kalish, author = "D. Kalish and R. Montague",
  title = "On {T}arski's Formalization of Predicate Logic with Identity",
  journal = "Archiv f{\"{u}}r Mathematische Logik und Grundlagenfor\-schung",
  volume = 7,
  year = 1965,
  note = "[QA.A673]",
  pages = "81--101" }
@article{Kalman, author = "J. A. Kalman",
  title = "Condensed Detachment as a Rule of Inference",
  journal = "Studia Logica",
  volume = 42,
  number = 4,
  year = 1983,
  note = "[B18.P6.S933]",
  pages = "443-451" }
@book{Kline, author = "Morris Kline",
  title = "Mathematical Thought from Ancient to Modern Times",
  publisher = "Oxford University Press",
  address = "New York",
  note = "[QA21.K516 1990 v.3]",
  year = 1972 }
@book{Klinel, author = "Morris Kline",
  title = "Mathematics, The Loss of Certainty",
  publisher = "Oxford University Press",
  address = "New York",
  note = "[QA21.K525]",
  year = 1980 }
@book{Kramer, author = "Edna E. Kramer",
  title = "The Nature and Growth of Modern Mathematics",
  publisher = "Princeton University Press",
  address = "Princeton, New Jersey",
  note = "[QA93.K89 1981]",
  year = 1981 }
@article{Knill, author = "Oliver Knill",
  title = "Some Fundamental Theorems in Mathematics",
  year = "2018",
  url = "https://arxiv.org/abs/1807.08416" }
@book{Landau, author = "Edmund Landau",
  title = "Foundations of Analysis",
  publisher = "Chelsea Publishing Company",
  address = "New York",
  edition = "second",
  note = "[QA241.L2541 1960]",
  year = 1960 }
@article{Leblanc, author = "Hugues Leblanc",
  title = "On {M}eyer and {L}ambert's Quantificational Calculus {FQ}",
  journal = "The Journal of Symbolic Logic",
  volume = 33,
  year = 1968,
  note = "[QA.J87]",
  pages = "275--280" }
@article{Lejewski, author = "Czeslaw Lejewski",
  title = "On Implicational Definitions",
  journal = "Studia Logica",
  volume = 8,
  year = 1958,
  note = "[B18.P6.S933]",
  pages = "189--208" }
@book{Levy, author = "Azriel Levy",
  title = "Basic Set Theory",
  publisher = "Dover Publications",
  address = "Mineola, NY",
  year = "2002"
}
@book{Margaris, author = "Angelo Margaris",
  title = "First Order Mathematical Logic",
  publisher = "Blaisdell Publishing Company",
  address = "Waltham, Massachusetts",
  note = "[QA9.M327]",
  year = 1967}
@book{Manin, author = "Yu I. Manin",
  title = "A Course in Mathematical Logic",
  publisher = "Springer-Verlag",
  address = "New York",
  note = "[QA9.M29613]",
  year = "1977" }
@article{Mathias, author = "Adrian R. D. Mathias",
  title = "A Term of Length 4,523,659,424,929",
  journal = "Synthese",
  volume = 133,
  year = 2002,
  note = "[Q.S993]",
  pages = "75--86" }
@article{Megill, author = "Norman D. Megill",
  title = "A Finitely Axiomatized Formalization of Predicate Calculus
     with Equality",
  journal = "Notre Dame Journal of Formal Logic",
  volume = 36,
  year = 1995,
  note = "[QA.N914]",
  pages = "435--453" }
@unpublished{Megillc, author = "Norman D. Megill",
  title = "A Shorter Equivalent of the Axiom of Choice",
  month = "June",
  note = "Unpublished",
  year = 1991 }
@article{MegillBunder, author = "Norman D. Megill and Martin W.
    Bunder",
  title = "Weaker {D}-Complete Logics",
  journal = "Journal of the IGPL",
  volume = 4,
  year = 1996,
  pages = "215--225",
  note = "Available on the Web at
{\verb+http:+}\-{\verb+//www.mpi-sb.mpg.de+}\-{\verb+/igpl+}%
\-{\verb+/Journal+}\-{\verb+/V4-2+}\-{\verb+/#Megill+}"}
}
@book{Mendelson, author = "Elliott Mendelson",
  title = "Introduction to Mathematical Logic",
  edition = "second",
  publisher = "D. Van Nostrand Company, Inc.",
  address = "New York",
  note = "[QA9.M537 1979]",
  year = 1979 }
@article{Meredith, author = "David Meredith",
  title = "In Memoriam {C}arew {A}rthur {M}eredith (1904-1976)",
  journal = "Notre Dame Journal of Formal Logic",
  volume = 18,
  year = 1977,
  note = "[QA.N914]",
  pages = "513--516" }
@article{CAMeredith, author = "C. A. Meredith",
  title = "Single Axioms for the Systems ({C},{N}), ({C},{O}) and ({A},{N})
      of the Two-Valued Propositional Calculus",
  journal = "The Journal of Computing Systems",
  volume = 3,
  year = 1953,
  pages = "155--164" }
@article{Monk, author = "J. Donald Monk",
  title = "Provability With Finitely Many Variables",
  journal = "The Journal of Symbolic Logic",
  volume = 27,
  year = 1971,
  note = "[QA.J87]",
  pages = "353--358" }
@article{Monks, author = "J. Donald Monk",
  title = "Substitutionless Predicate Logic With Identity",
  journal = "Archiv f{\"{u}}r Mathematische Logik und Grundlagenfor\-schung",
  volume = 7,
  year = 1965,
  pages = "103--121" }
  %% Took out this from above to prevent LaTeX underfull warning:
  % note = "[QA.A673]",
@book{Moore, author = "A. W. Moore",
  title = "The Infinite",
  publisher = "Routledge",
  address = "New York",
  note = "[BD411.M59]",
  year = 1989}
@book{Munkres, author = "James R. Munkres",
  title = "Topology: A First Course",
  publisher = "Prentice-Hall, Inc.",
  address = "Englewood Cliffs, New Jersey",
  note = "[QA611.M82]",
  year = 1975}
@article{Nemesszeghy, author = "E. Z. Nemesszeghy and E. A. Nemesszeghy",
  title = "On Strongly Creative Definitions:  A Reply to {V}. {F}. {R}ickey",
  journal = "Logique et Analyse (N.\ S.)",
  year = 1977,
  volume = 20,
  note = "[BC.L832]",
  pages = "111--115" }
@unpublished{Nemeti, author = "N{\'{e}}meti, I.",
  title = "Algebraizations of Quantifier Logics, an Overview",
  note = "Version 11.4, preprint, Mathematical Institute, Budapest,
    1994.  A shortened version without proofs appeared in
    ``Algebraizations of quantifier logics, an introductory overview,''
   {\em Studia Logica}, 50:485--569, 1991 [B18.P6.S933]"}
@article{Pavicic, author = "M. Pavi{\v{c}}i{\'{c}}",
  title = "A New Axiomatization of Unified Quantum Logic",
  journal = "International Journal of Theoretical Physics",
  year = 1992,
  volume = 31,
  note = "[QC.I626]",
  pages = "1753 --1766" }
@book{Penrose, author = "Roger Penrose",
  title = "The Emperor's New Mind",
  publisher = "Oxford University Press",
  address = "New York",
  note = "[Q335.P415]",
  year = 1989 }
@book{PetersonI, author = "Ivars Peterson",
  title = "The Mathematical Tourist",
  publisher = "W. H. Freeman and Company",
  address = "New York",
  note = "[QA93.P475]",
  year = 1988 }
@article{Peterson, author = "Jeremy George Peterson",
  title = "An automatic theorem prover for substitution and detachment systems",
  journal = "Notre Dame Journal of Formal Logic",
  volume = 19,
  year = 1978,
  note = "[QA.N914]",
  pages = "119--122" }
@book{Quine, author = "Willard Van Orman Quine",
  title = "Set Theory and Its Logic",
  edition = "revised",
  publisher = "The Belknap Press of Harvard University Press",
  address = "Cambridge, Massachusetts",
  note = "[QA248.Q7 1969]",
  year = 1969 }
@article{Robinson, author = "J. A. Robinson",
  title = "A Machine-Oriented Logic Based on the Resolution Principle",
  journal = "Journal of the Association for Computing Machinery",
  year = 1965,
  volume = 12,
  pages = "23--41" }
@article{RobinsonT, author = "T. Thacher Robinson",
  title = "Independence of Two Nice Sets of Axioms for the Propositional
    Calculus",
  journal = "The Journal of Symbolic Logic",
  volume = 33,
  year = 1968,
  note = "[QA.J87]",
  pages = "265--270" }
@book{Rucker, author = "Rudy Rucker",
  title = "Infinity and the Mind:  The Science and Philosophy of the
    Infinite",
  publisher = "Bantam Books, Inc.",
  address = "New York",
  note = "[QA9.R79 1982]",
  year = 1982 }
@book{Russell, author = "Bertrand Russell",
  title = "Mysticism and Logic, and Other Essays",
  publisher = "Barnes \& Noble Books",
  address = "Totowa, New Jersey",
  note = "[B1649.R963.M9 1981]",
  year = 1981 }
@article{Russell2, author = "Bertrand Russell",
  title = "Recent Work on the Principles of Mathematics",
  journal = "International Monthly",
  volume = 4,
  year = 1901,
  pages = "84"}
@article{Schmidt, author = "Eric Schmidt",
  title = "Reductions in Norman Megill's axiom system for complex numbers",
  url = "http://us.metamath.org/downloads/schmidt-cnaxioms.pdf",
  year = "2012" }
@book{Shoenfield, author = "Joseph R. Shoenfield",
  title = "Mathematical Logic",
  publisher = "Addison-Wesley Publishing Company, Inc.",
  address = "Reading, Massachusetts",
  year = 1967,
  note = "[QA9.S52]" }
@book{Smullyan, author = "Raymond M. Smullyan",
  title = "Theory of Formal Systems",
  publisher = "Princeton University Press",
  address = "Princeton, New Jersey",
  year = 1961,
  note = "[QA248.5.S55]" }
@book{Solow, author = "Daniel Solow",
  title = "How to Read and Do Proofs:  An Introduction to Mathematical
    Thought Process",
  publisher = "John Wiley \& Sons",
  address = "New York",
  year = 1982,
  note = "[QA9.S577]" }
@book{Stark, author = "Harold M. Stark",
  title = "An Introduction to Number Theory",
  publisher = "Markham Publishing Company",
  address = "Chicago",
  note = "[QA241.S72 1978]",
  year = 1970 }
@article{Swart, author = "E. R. Swart",
  title = "The Philosophical Implications of the Four-Color Problem",
  journal = "American Mathematical Monthly",
  year = 1980,
  volume = 87,
  month = "November",
  note = "[QA.A5125]",
  pages = "697--707" }
@book{Szpiro, author = "George G. Szpiro",
  title = "Poincar{\'{e}}'s Prize: The Hundred-Year Quest to Solve One
    of Math's Greatest Puzzles",
  publisher = "Penguin Books Ltd",
  address = "London",
  note = "[QA43.S985 2007]",
  year = 2007}
@book{Takeuti, author = "Gaisi Takeuti and Wilson M. Zaring",
  title = "Introduction to Axiomatic Set Theory",
  edition = "second",
  publisher = "Springer-Verlag New York Inc.",
  address = "New York",
  note = "[QA248.T136 1982]",
  year = 1982}
@inproceedings{Tarski, author = "Alfred Tarski",
  title = "What is Elementary Geometry",
  pages = "16--29",
  booktitle = "The Axiomatic Method, with Special Reference to Geometry and
     Physics (Proceedings of an International Symposium held at the University
     of California, Berkeley, December 26, 1957 --- January 4, 1958)",
  editor = "Leon Henkin and Patrick Suppes and Alfred Tarski",
  year = 1959,
  publisher = "North-Holland Publishing Company",
  address = "Amsterdam"}
@article{Tarski1965, author = "Alfred Tarski",
  title = "A Simplified Formalization of Predicate Logic with Identity",
  journal = "Archiv f{\"{u}}r Mathematische Logik und Grundlagenforschung",
  volume = 7,
  year = 1965,
  note = "[QA.A673]",
  pages = "61--79" }
@book{Tymoczko,
  title = "New Directions in the Philosophy of Mathematics",
  editor = "Thomas Tymoczko",
  publisher = "Birkh{\"{a}}user Boston, Inc.",
  address = "Boston",
  note = "[QA8.6.N48 1986]",
  year = 1986 }
@incollection{Wang,
  author = "Hao Wang",
  title = "Theory and Practice in Mathematics",
  pages = "129--152",
  booktitle = "New Directions in the Philosophy of Mathematics",
  editor = "Thomas Tymoczko",
  publisher = "Birkh{\"{a}}user Boston, Inc.",
  address = "Boston",
  note = "[QA8.6.N48 1986]",
  year = 1986 }
@manual{Webster,
  title = "Webster's New Collegiate Dictionary",
  organization = "G. \& C. Merriam Co.",
  address = "Springfield, Massachusetts",
  note = "[PE1628.W4M4 1977]",
  year = 1977 }
@manual{Whitehead, author = "Alfred North Whitehead",
  title = "An Introduction to Mathematics",
  year = 1911 }
@book{PM, author = "Alfred North Whitehead and Bertrand Russell",
  title = "Principia Mathematica",
  edition = "second",
  publisher = "Cambridge University Press",
  address = "Cambridge",
  year = "1927",
  note = "(3 vols.) [QA9.W592 1927]" }
@article{DBLP:journals/corr/Whalen16,
  author    = {Daniel Whalen},
  title     = {Holophrasm: a neural Automated Theorem Prover for higher-order logic},
  journal   = {CoRR},
  volume    = {abs/1608.02644},
  year      = {2016},
  url       = {http://arxiv.org/abs/1608.02644},
  archivePrefix = {arXiv},
  eprint    = {1608.02644},
  timestamp = {Mon, 13 Aug 2018 16:46:19 +0200},
  biburl    = {https://dblp.org/rec/bib/journals/corr/Whalen16},
  bibsource = {dblp computer science bibliography, https://dblp.org} }
@article{Wiedijk-revisited,
  author = {Freek Wiedijk},
  title = {The QED Manifesto Revisited},
  year = {2007},
  url = {http://mizar.org/trybulec65/8.pdf} }
@book{Wolfram,
  author = "Stephen Wolfram",
  title = "Mathematica:  A System for Doing Mathematics by Computer",
  edition = "second",
  publisher = "Addison-Wesley Publishing Co.",
  address = "Redwood City, California",
  note = "[QA76.95.W65 1991]",
  year = 1991 }
@book{Wos, author = "Larry Wos and Ross Overbeek and Ewing Lusk and Jim Boyle",
  title = "Automated Reasoning:  Introduction and Applications",
  edition = "second",
  publisher = "McGraw-Hill, Inc.",
  address = "New York",
  note = "[QA76.9.A96.A93 1992]",
  year = 1992 }

%
%
%[1] Church, Alonzo, Introduction to Mathematical Logic,
% Volume 1, Princeton University Press, Princeton, N. J., 1956.
%
%[2] Cohen, Paul J., Set Theory and the Continuum Hypothesis,
% W. A. Benjamin, Inc., Reading, Mass., 1966.
%
%[3] Hamilton, Alan G., Logic for Mathematicians, Cambridge
% University Press,
% Cambridge, 1988.

%[6] Kleene, Stephen Cole, Introduction to Metamathematics, D.  Van
% Nostrand Company, Inc., Princeton (1952).

%[13] Tarski, Alfred, "A simplified formalization of predicate
% logic with identity," Archiv fur Mathematische Logik und
% Grundlagenforschung, vol. 7 (1965), pp. 61-79.

%[14] Tarski, Alfred and Steven Givant, A Formalization of Set
% Theory Without Variables, American Mathematical Society Colloquium
% Publications, vol. 41, American Mathematical Society,
% Providence, R. I., 1987.

%[15] Zeman, J. J., Modal Logic, Oxford University Press, Oxford, 1973.
\end{filecontents}
% --------------------------- End of metamath.bib -----------------------------


%Book: Metamath
%Author:  Norman Megill Email:  nm at alum.mit.edu
%Author:  David A. Wheeler Email:  dwheeler at dwheeler.com

% A book template example
% http://www.stsci.edu/ftp/software/tex/bookstuff/book.template

\documentclass[leqno]{book} % LaTeX 2e. 10pt. Use [leqno,12pt] for 12pt
% hyperref 2002/05/27 v6.72r  (couldn't get pagebackref to work)
\usepackage[plainpages=false,pdfpagelabels=true]{hyperref}

\usepackage{needspace}     % Enable control over page breaks
\usepackage{breqn}         % automatic equation breaking
\usepackage{microtype}     % microtypography, reduces hyphenation

% Packages for flexible tables.  We need to be able to
% wrap text within a cell (with automatically-determined widths) AND
% split a table automatically across multiple pages.
% * "tabularx" wraps text in cells but only 1 page
% * "longtable" goes across pages but by itself is incompatible with tabularx
% * "ltxtable" combines longtable and tabularx, but table contents
%    must be in a separate file.
% * "ltablex" combines tabularx and longtable - must install specially
% * "booktabs" is recommended as a way to improve the look of tables,
%   but doesn't add these capabilities.
% * "tabu" much more capable and seems to be recommended. So use that.

\usepackage{makecell}      % Enable forced line splits within a table cell
% v4.13 needed for tabu: https://tex.stackexchange.com/questions/600724/dimension-too-large-after-recent-longtable-update
\usepackage{longtable}[=v4.13] % Enable multi-page tables  
\usepackage{tabu}          % Multi-page tables with wrapped text in a cell

% You can find more Tex packages using commands like:
% tlmgr search --file tabu.sty
% find /usr/share/texmf-dist/ -name '*tab*'
%
%%%%%%%%%%%%%%%%%%%%%%%%%%%%%%%%%%%%%%%%%%%%%%%%%%%%%%%%%%%%%%%%%%%%%%%%%%%%
% Uncomment the next 3 lines to suppress boxes and colors on the hyperlinks
%%%%%%%%%%%%%%%%%%%%%%%%%%%%%%%%%%%%%%%%%%%%%%%%%%%%%%%%%%%%%%%%%%%%%%%%%%%%
%\hypersetup{
%colorlinks,citecolor=black,filecolor=black,linkcolor=black,urlcolor=black
%}
%
\usepackage{realref}

% Restarting page numbers: try?
%   \printglossary
%   \cleardoublepage
%   \pagenumbering{arabic}
%   \setcounter{page}{1}    ???needed
%   \include{chap1}

% not used:
% \def\R2Lurl#1#2{\mbox{\href{#1}\texttt{#2}}}

\usepackage{amssymb}

% Version 1 of book: margins: t=.4, b=.2, ll=.4, rr=.55
% \usepackage{anysize}
% % \papersize{<height>}{<width>}
% % \marginsize{<left>}{<right>}{<top>}{<bottom>}
% \papersize{9in}{6in}
% % l/r 0.6124-0.6170 works t/b 0.2418-0.3411 = 192pp. 0.2926-03118=exact
% \marginsize{0.7147in}{0.5147in}{0.4012in}{0.2012in}

\usepackage{anysize}
% \papersize{<height>}{<width>}
% \marginsize{<left>}{<right>}{<top>}{<bottom>}
\papersize{9in}{6in}
% l/r 0.85in&0.6431-0.6539 works t/b ?-?
%\marginsize{0.85in}{0.6485in}{0.55in}{0.35in}
\marginsize{0.8in}{0.65in}{0.5in}{0.3in}

% \usepackage[papersize={3.6in,4.8in},hmargin=0.1in,vmargin={0.1in,0.1in}]{geometry}  % page geometry
\usepackage{special-settings}

\raggedbottom
\makeindex

\begin{document}
% Discourage page widows and orphans:
\clubpenalty=300
\widowpenalty=300

%%%%%%% load in AMS fonts %%%%%%% % LaTeX 2.09 - obsolete in LaTeX 2e
%\input{amssym.def}
%\input{amssym.tex}
%\input{c:/texmf/tex/plain/amsfonts/amssym.def}
%\input{c:/texmf/tex/plain/amsfonts/amssym.tex}

\bibliographystyle{plain}
\pagenumbering{roman}
\pagestyle{headings}

\thispagestyle{empty}

\hfill
\vfill

\begin{center}
{\LARGE\bf Metamath} \\
\vspace{1ex}
{\large A Computer Language for Mathematical Proofs} \\
\vspace{7ex}
{\large Norman Megill} \\
\vspace{7ex}
with extensive revisions by \\
\vspace{1ex}
{\large David A. Wheeler} \\
\vspace{7ex}
% Printed date. If changing the date below, also fix the date at the beginning.
2019-06-02
\end{center}

\vfill
\hfill

\newpage
\thispagestyle{empty}

\hfill
\vfill

\begin{center}
$\sim$\ {\sc Public Domain}\ $\sim$

\vspace{2ex}
This book (including its later revisions)
has been released into the Public Domain by Norman Megill per the
Creative Commons CC0 1.0 Universal (CC0 1.0) Public Domain Dedication.
David A. Wheeler has done the same.
This public domain release applies worldwide.  In case this is not
legally possible, the right is granted to use the work for any purpose,
without any conditions, unless such conditions are required by law.
See \url{https://creativecommons.org/publicdomain/zero/1.0/}.

\vspace{3ex}
Several short, attributed quotations from copyrighted works
appear in this book under the ``fair use'' provision of Section 107 of
the United States Copyright Act (Title 17 of the {\em United States
Code}).  The public-domain status of this book is not applicable to
those quotations.

\vspace{3ex}
Any trademarks used in this book are the property of their owners.

% QA76.9.L63.M??

% \vspace{1ex}
%
% \vspace{1ex}
% {\small Permission is granted to make and distribute verbatim copies of this
% book
% provided the copyright notice and this
% permission notice are preserved on all copies.}
%
% \vspace{1ex}
% {\small Permission is granted to copy and distribute modified versions of this
% book under the conditions for verbatim copying, provided that the
% entire
% resulting derived work is distributed under the terms of a permission
% notice
% identical to this one.}
%
% \vspace{1ex}
% {\small Permission is granted to copy and distribute translations of this
% book into another language, under the above conditions for modified
% versions,
% except that this permission notice may be stated in a translation
% approved by the
% author.}
%
% \vspace{1ex}
% %{\small   For a copy of the \LaTeX\ source files for this book, contact
% %the author.} \\
% \ \\
% \ \\

\vspace{7ex}
% ISBN: 1-4116-3724-0 \\
% ISBN: 978-1-4116-3724-5 \\
ISBN: 978-0-359-70223-7 \\
{\ } \\
Lulu Press \\
Morrisville, North Carolina\\
USA


\hfill
\vfill

Norman Megill\\ 93 Bridge St., Lexington, MA 02421 \\
E-mail address: \texttt{nm{\char`\@}alum.mit.edu} \\
\vspace{7ex}
David A. Wheeler \\
E-mail address: \texttt{dwheeler{\char`\@}dwheeler.com} \\
% See notes added at end of Preface for revision history. \\
% For current information on the Metamath software see \\
\vspace{7ex}
\url{http://metamath.org}
\end{center}

\hfill
\vfill

{\parindent0pt%
\footnotesize{%
Cover: Aleph null ($\aleph_0$) is the symbol for the
first infinite cardinal number, discovered by Georg Cantor in 1873.
We use a red aleph null (with dark outline and gold glow) as the Metamath logo.
Credit: Norman Megill (1994) and Giovanni Mascellani (2019),
public domain.%
\index{aleph null}%
\index{Metamath!logo}\index{Cantor, Georg}\index{Mascellani, Giovanni}}}

% \newpage
% \thispagestyle{empty}
%
% \hfill
% \vfill
%
% \begin{center}
% {\it To my son Robin Dwight Megill}
% \end{center}
%
% \vfill
% \hfill
%
% \newpage

\tableofcontents
%\listoftables

\chapter*{Preface}
\markboth{PREFACE}{PREFACE}
\addcontentsline{toc}{section}{Preface}


% (For current information, see the notes added at the
% end of this preface on p.~\pageref{note2002}.)

\subsubsection{Overview}

Metamath\index{Metamath} is a computer language and an associated computer
program for archiving, verifying, and studying mathematical proofs at a very
detailed level.  The Metamath language incorporates no mathematics per se but
treats all mathematical statements as mere sequences of symbols.  You provide
Metamath with certain special sequences (axioms) that tell it what rules
of inference are allowed.  Metamath is not limited to any specific field of
mathematics.  The Metamath language is simple and robust, with an
almost total absence of hard-wired syntax, and
we\footnote{Unless otherwise noted, the words
``I,'' ``me,'' and ``my'' refer to Norman Megill\index{Megill, Norman}, while
``we,'' ``us,'' and ``our'' refer to Norman Megill and
David A. Wheeler\index{Wheeler, David A.}.}
believe that it
provides about the simplest possible framework that allows essentially all of
mathematics to be expressed with absolute rigor.

% index test
%\newcommand{\nn}[1]{#1n}
%\index{aaa@bbb}
%\index{abc!def}
%\index{abd|see{qqq}}
%\index{abe|nn}
%\index{abf|emph}
%\index{abg|(}
%\index{abg|)}

Using the Metamath language, you can build formal or mathematical
systems\index{formal system}\footnote{A formal or mathematical system consists
of a collection of symbols (such as $2$, $4$, $+$ and $=$), syntax rules that
describe how symbols may be combined to form a legal expression (called a
well-formed formula or {\em wff}, pronounced ``whiff''), some starting wffs
called axioms, and inference rules that describe how theorems may be derived
(proved) from the axioms.  A theorem is a mathematical fact such as $2+2=4$.
Strictly speaking, even an obvious fact such as this must be proved from
axioms to be formally acceptable to a mathematician.}\index{theorem}
\index{axiom}\index{rule}\index{well-formed formula (wff)} that involve
inferences from axioms.  Although a database is provided
that includes a recommended set of axioms for standard mathematics, if you
wish you can supply your own symbols, syntax, axioms, rules, and definitions.

The name ``Metamath'' was chosen to suggest that the language provides a
means for {\em describing} mathematics rather than {\em being} the
mathematics itself.  Actually in some sense any mathematical language is
metamathematical.  Symbols written on paper, or stored in a computer,
are not mathematics itself but rather a way of expressing mathematics.
For example ``7'' and ``VII'' are symbols for denoting the number seven
in Arabic and Roman numerals; neither {\em is} the number seven.

If you are able to understand and write computer programs, you should be able
to follow abstract mathematics with the aid of Metamath.  Used in conjunction
with standard textbooks, Metamath can guide you step-by-step towards an
understanding of abstract mathematics from a very rigorous viewpoint, even if
you have no formal abstract mathematics background.  By using a single,
consistent notation to express proofs, once you grasp its basic concepts
Metamath provides you with the ability to immediately follow and dissect
proofs even in totally unfamiliar areas.

Of course, just being able follow a proof will not necessarily give you an
intuitive familiarity with mathematics.  Memorizing the rules of chess does not
give you the ability to appreciate the game of a master, and knowing how the
notes on a musical score map to piano keys does not give you the ability to
hear in your head how it would sound.  But each of these can be a first step.

Metamath allows you to explore proofs in the sense that you can see the
theorem referenced at any step expanded in as much detail as you want, right
down to the underlying axioms of logic and set theory (in the case of the set
theory database provided).  While Metamath will not replace the higher-level
understanding that can only be acquired through exercises and hard work, being
able to see how gaps in a proof are filled in can give you increased
confidence that can speed up the learning process and save you time when you
get stuck.

The Metamath language breaks down a mathematical proof into its tiniest
possible parts.  These can be pieced together, like interlocking
pieces in a puzzle, only in a way that produces correct and absolutely rigorous
mathematics.

The nature of Metamath\index{Metamath} enforces very precise mathematical
thinking, similar to that involved in writing a computer program.  A crucial
difference, though, is that once a proof is verified (by the Metamath program)
to be correct, it is definitely correct; it can never have a hidden
``bug.''\index{computer program bugs}  After getting used to the kind of rigor
and accuracy provided by Metamath, you might even be tempted to
adopt the attitude that a proof should never be considered correct until it
has been verified by a computer, just as you would not completely trust a
manual calculation until you have verified it on a
calculator.

My goal
for Metamath was a system for describing and verifying
mathematics that is completely universal yet conceptually as simple as
possible.  In approaching mathematics from an axiomatic, formal viewpoint, I
wanted Metamath to be able to handle almost any mathematical system, not
necessarily with ease, but at least in principle and hopefully in practice. I
wanted it to verify proofs with absolute rigor, and for this reason Metamath
is what might be thought of as a ``compile-only'' language rather than an
algorithmic or Turing-machine language (Pascal, C, Prolog, Mathematica,
etc.).  In other words, a database written in the Metamath
language doesn't ``do'' anything; it merely exhibits mathematical knowledge
and permits this knowledge to be verified as being correct.  A program in an
algorithmic language can potentially have hidden bugs\index{computer program
bugs} as well as possibly being hard to understand.  But each token in a
Metamath database must be consistent with the database's earlier
contents according to simple, fixed rules.
If a database is verified
to be correct,\footnote{This includes
verification that a sequential list of proof steps results in the specified
theorem.} then the mathematical content is correct if the
verifier is correct and the axioms are correct.
The verification program could be incorrect, but the verification algorithm
is relatively simple (making it unlikely to be implemented incorrectly
by the Metamath program),
and there are over a dozen Metamath database verifiers
written by different people in different programming languages
(so these different verifiers can act as multiple reviewers of a database).
The most-used Metamath database, the Metamath Proof Explorer
(aka \texttt{set.mm}\index{set theory database (\texttt{set.mm})}%
\index{Metamath Proof Explorer}),
is currently verified by four different Metamath verifiers written by
four different people in four different languages, including the
original Metamath program described in this book.
The only ``bugs'' that can exist are in the statement of the axioms,
for example if the axioms are inconsistent (a famous problem shown to be
unsolvable by G\"{o}del's incompleteness theorem\index{G\"{o}del's
incompleteness theorem}).
However, real mathematical systems have very few axioms, and these can
be carefully studied.
All of this provides extraordinarily high confidence that the verified database
is in fact correct.

The Metamath program
doesn't prove theorems automatically but is designed to verify proofs
that you supply to it.
The underlying Metamath language is completely general and has no built-in,
preconceived notions about your formal system\index{formal system}, its logic
or its syntax.
For constructing proofs, the Metamath program has a Proof Assistant\index{Proof
Assistant} which helps you fill in some of a proof step's details, shows you
what choices you have at any step, and verifies the proof as you build it; but
you are still expected to provide the proof.

There are many other programs that can process or generate information
in the Metamath language, and more continue to be written.
This is in part because the Metamath language itself is very simple
and intentionally easy to automatically process.
Some programs, such as \texttt{mmj2}\index{mmj2}, include a proof assistant
that can automate some steps beyond what the Metamath program can do.
Mario Carneiro has developed an algorithm for converting proofs from
the OpenTheory interchange format, which can be translated to and from
any of the HOL family of proof languages (HOL4, HOL Light, ProofPower,
and Isabelle), into the
Metamath language \cite{DBLP:journals/corr/Carneiro14}\index{Carneiro, Mario}.
Daniel Whalen has developed Holophrasm, which can automatically
prove many Metamath proofs using
machine learning\index{machine learning}\index{artificial intelligence}
approaches
(including multiple neural networks\index{neural networks})\cite{DBLP:journals/corr/Whalen16}\index{Whalen, Daniel}.
However,
a discussion of these other programs is beyond the scope of this book.

Like most computer languages, the Metamath\index{Metamath} language uses the
standard ({\sc ascii}) characters on a computer keyboard, so it cannot
directly represent many of the special symbols that mathematicians use.  A
useful feature of the Metamath program is its ability to convert its notation
into the \LaTeX\ typesetting language.\index{latex@{\LaTeX}}  This feature
lets you convert the {\sc ascii} tokens you've defined into standard
mathematical symbols, so you end up with symbols and formulas you are familiar
with instead of somewhat cryptic {\sc ascii} representations of them.
The Metamath program can also generate HTML\index{HTML}, making it easy
to view results on the web and to see related information by using
hypertext links.

Metamath is probably conceptually different from anything you've seen
before and some aspects may take some getting used to.  This book will
help you decide whether Metamath suits your specific needs.

\subsubsection{Setting Your Expectations}
It is important for you to understand what Metamath\index{Metamath} is and is
not.  As mentioned, the Metamath program
is {\em not} an automated theorem prover but
rather a proof verifier.  Developing a database can be tedious, hard work,
especially if you want to make the proofs as short as possible, but it becomes
easier as you build up a collection of useful theorems.  The purpose of
Metamath is simply to document existing mathematics in an absolutely rigorous,
computer-verifiable way, not to aid directly in the creation of new
mathematics.  It also is not a magic solution for learning abstract
mathematics, although it may be helpful to be able to actually see the implied
rigor behind what you are learning from textbooks, as well as providing hints
to work out proofs that you are stumped on.

As of this writing, a sizable set theory database has been developed to
provide a foundation for many fields of mathematics, but much more work would
be required to develop useful databases for specific fields.

Metamath\index{Metamath} ``knows no math;'' it just provides a framework in
which to express mathematics.  Its language is very small.  You can define two
kinds of symbols, constants\index{constant} and variables\index{variable}.
The only thing Metamath knows how to do is to substitute strings of symbols
for the variables\index{substitution!variable}\index{variable substitution} in
an expression based on instructions you provide it in a proof, subject to
certain constraints you specify for the variables.  Even the decimal
representation of a number is merely a string of certain constants (digits)
which together, in a specific context, correspond to whatever mathematical
object you choose to define for it; unlike other computer languages, there is
no actual number stored inside the computer.  In a proof, you in effect
instruct Metamath what symbol substitutions to make in previous axioms or
theorems and join a sequence of them together to result in the desired
theorem.  This kind of symbol manipulation captures the essence of mathematics
at a preaxiomatic level.

\subsubsection{Metamath and Mathematical Literature}

In advanced mathematical literature, proofs are usually presented in the form
of short outlines that often only an expert can follow.  This is partly out of
a desire for brevity, but it would also be unwise (even if it were practical)
to present proofs in complete formal detail, since the overall picture would
be lost.\index{formal proof}

A solution I envision\label{envision} that would allow mathematics to remain
acceptable to the expert, yet increase its accessibility to non-specialists,
consists of a combination of the traditional short, informal proof in print
accompanied by a complete formal proof stored in a computer database.  In an
analogy with a computer program, the informal proof is like a set of comments
that describe the overall reasoning and content of the proof, whereas the
computer database is like the actual program and provides a means for anyone,
even a non-expert, to follow the proof in as much detail as desired, exploring
it back through layers of theorems (like subroutines that call other
subroutines) all the way back to the axioms of the theory.  In addition, the
computer database would have the advantage of providing absolute assurance
that the proof is correct, since each step can be verified automatically.

There are several other approaches besides Metamath to a project such
as this.  Section~\ref{proofverifiers} discusses some of these.

To us, a noble goal would be a database with hundreds of thousands of
theorems and their computer-verifiable proofs, encompassing a significant
fraction of known mathematics and available for instant access.
These would be fully verified by multiple independently-implemented verifiers,
to provide extremely high confidence that the proofs are completely correct.
The database would allow people to investigate whatever details they were
interested in, so that they could confirm whatever portions they wished.
Whether or not Metamath is an appropriate choice remains to be seen, but in
principle we believe it is sufficient.

\subsubsection{Formalism}

Over the past fifty years, a group of French mathematicians working
collectively under the pseudonym of Bourbaki\index{Bourbaki, Nicolas} have
co-authored a series of monographs that attempt to rigorously and
consistently formalize large bodies of mathematics from foundations.  On the
one hand, certainly such an effort has its merits; on the other hand, the
Bourbaki project has been criticized for its ``scholasticism'' and
``hyperaxiomatics'' that hide the intuitive steps that lead to the results
\cite[p.~191]{Barrow}\index{Barrow, John D.}.

Metamath unabashedly carries this philosophy to its extreme and no doubt is
subject to the same kind of criticism.  Nonetheless I think that in
conjunction with conventional approaches to mathematics Metamath can serve a
useful purpose.  The Bourbaki approach is essentially pedagogic, requiring the
reader to become intimately familiar with each detail in a very large
hierarchy before he or she can proceed to the next step.  The difference with
Metamath is that the ``reader'' (user) knows that all details are contained in
its computer database, available as needed; it does not demand that the user
know everything but conveniently makes available those portions that are of
interest.  As the body of all mathematical knowledge grows larger and larger,
no one individual can have a thorough grasp of its entirety.  Metamath
can finalize and put to rest any questions about the validity of any part of it
and can make any part of it accessible, in principle, to a non-specialist.

\subsubsection{A Personal Note}
Why did I develop Metamath\index{Metamath}?  I enjoy abstract mathematics, but
I sometimes get lost in a barrage of definitions and start to lose confidence
that my proofs are correct.  Or I reach a point where I lose sight of how
anything I'm doing relates to the axioms that a theory is based on and am
sometimes suspicious that there may be some overlooked implicit axiom
accidentally introduced along the way (as happened historically with Euclidean
geometry\index{Euclidean geometry}, whose omission of Pasch's
axiom\index{Pasch's axiom} went unnoticed for 2000 years
\cite[p.~160]{Davis}!). I'm also somewhat lazy and wish to avoid the effort
involved in re-verifying the gaps in informal proofs ``left to the reader;'' I
prefer to figure them out just once and not have to go through the same
frustration a year from now when I've forgotten what I did.  Metamath provides
better recovery of my efforts than scraps of paper that I can't
decipher anymore.  But mostly I find very appealing the idea of rigorously
archiving mathematical knowledge in a computer database, providing precision,
certainty, and elimination of human error.

\subsubsection{Note on Bibliography and Index}

The Bibliography usually includes the Library of Congress classification
for a work to make it easier for you to find it in on a university
library shelf.  The Index has author references to pages where their works
are cited, even though the authors' names may not appear on those pages.

\subsubsection{Acknowledgments}

Acknowledgments are first due to my wife, Deborah (who passed away on
September 4, 1998), for critiquing the manu\-script but most of all for
her patience and support.  I also wish to thank Joe Wright, Richard
Becker, Clarke Evans, Buddha Buck, and Jeremy Henty for helpful
comments.  Any errors, omissions, and other shortcomings are of course
my responsibility.

\subsubsection{Note Added June 22, 2005}\label{note2002}

The original, unpublished version of this book was written in 1997 and
distributed via the web.  The present edition has been updated to
reflect the current Metamath program and databases, as well as more
current {\sc url}s for Internet sites.  Thanks to Josh
Purinton\index{Purinton, Josh}, One Hand
Clapping, Mel L.\ O'Cat, and Roy F. Longton for pointing out
typographical and other errors.  I have also benefitted from numerous
discussions with Raph Levien\index{Levien, Raph}, who has extended
Metamath's philosophy of rigor to result in his {\em
Ghilbert}\index{Ghilbert} proof language (\url{http://ghilbert.org}).

Robert (Bob) Solovay\index{Solovay, Robert} communicated a new result of
A.~R.~D.~Mathias on the system of Bourbaki, and the text has been
updated accordingly (p.~\pageref{bourbaki}).

Bob also pointed out a clarification of the literature regarding
category theory and inaccessible cardinals\index{category
theory}\index{cardinal, inaccessible} (p.~\pageref{categoryth}),
and a misleading statement was removed from the text.  Specifically,
contrary to a statement in previous editions, it is possible to express
``There is a proper class of inaccessible cardinals'' in the language of
ZFC.  This can be done as follows:  ``For every set $x$ there is an
inaccessible cardinal $\kappa$ such that $\kappa$ is not in $x$.''
Bob writes:\footnote{Private communication, Nov.~30, 2002.}
\begin{quotation}
     This axiom is how Grothendieck presents category theory.  To each
inaccessible cardinal $\kappa$ one associates a Grothendieck universe
\index{Grothendieck, Alexander} $U(\kappa)$.  $U(\kappa)$ consists of
those sets which lie in a transitive set of cardinality less than
$\kappa$.  Instead of the ``category of all groups,'' one works relative
to a universe [considering the category of groups of cardinality less
than $\kappa$].  Now the category whose objects are all categories
``relative to the universe $U(\kappa)$'' will be a category not
relative to this universe but to the next universe.

     All of the things category theorists like to do can be done in this
framework.  The only controversial point is whether the Grothen\-dieck
axiom is too strong for the needs of category theorists.  Mac Lane
\index{Mac Lane, Saunders} argues that ``one universe is enough'' and
Feferman\index{Feferman, Solomon} has argued that one can get by with
ordinary ZFC.  I don't find Feferman's arguments persuasive.  Mac Lane
may be right, but when I think about category theory I do it \`{a} la
Grothendieck.

        By the way Mizar\index{Mizar} adds the axiom ``there is a proper
class of inaccessibles'' precisely so as to do category theory.
\end{quotation}

The most current information on the Metamath program and databases can
always be found at \url{http://metamath.org}.


\subsubsection{Note Added June 24, 2006}\label{note2006}

The Metamath spec was restricted slightly to make parsers easier to
write.  See the footnote on p.~\pageref{namespace}.

%\subsubsection{Note Added July 24, 2006}\label{note2006b}
\subsubsection{Note Added March 10, 2007}\label{note2006b}

I am grateful to Anthony Williams\index{Williams, Anthony} for writing
the \LaTeX\ package called {\tt realref.sty} and contributing it to the
public domain.  This package allows the internal hyperlinks in a {\sc
pdf} file to anchor to specific page numbers instead of just section
titles, making the navigation of the {\sc pdf} file for this book much
more pleasant and ``logical.''

A typographical error found by Martin Kiselkov was corrected.
A confusing remark about unification was deleted per suggestion of
Mel O'Cat.

\subsubsection{Note Added May 27, 2009}\label{note2009}

Several typos found by Kim Sparre were corrected.  A note was added that
the Poincar\'{e} conjecture has been proved (p.~\pageref{poincare}).

\subsubsection{Note Added Nov. 17, 2014}\label{note2014}

The statement of the Schr\"{o}der--Bernstein theorem was corrected in
Section~\ref{trust}.  Thanks to Bob Solovay for pointing out the error.

\subsubsection{Note Added May 25, 2016}\label{note2016}

Thanks to Jerry James for correcting 16 typos.

\subsubsection{Note Added February 25, 2019}\label{note201902}

David A. Wheeler\index{Wheeler, David A.}
made a large number of improvements and updates,
in coordination with Norman Megill.
The predicate calculus axioms were renumbered, and the text makes
it clear that they are based on Tarski's system S2;
the one slight deviation in axiom ax-6 is explained and justified.
The real and complex number axioms were modified to be consistent with
\texttt{set.mm}\index{set theory database (\texttt{set.mm})}%
\index{Metamath Proof Explorer}.
Long-awaited specification changes ``1--8'' were made,
which clarified previously ambiguous points.
Some errors in the text involving \texttt{\$f} and
\texttt{\$d} statements were corrected (the spec was correct, but
the in-book explanations unintentionally contradicted the spec).
We now have a system for automatically generating narrow PDFs,
so that those with smartphones can have easy access to the current
version of this document.
A new section on deduction was added;
it discusses the standard deduction theorem,
the weak deduction theorem,
deduction style, and natural deduction.
Many minor corrections (too numerous to list here) were also made.

\subsubsection{Note Added March 7, 2019}\label{note201903}

This added a description of the Matamath language syntax in
Extended Backus--Naur Form (EBNF)\index{Extended Backus--Naur Form}\index{EBNF}
in Appendix \ref{BNF}, added a brief explanation about typecodes,
inserted more examples in the deduction section,
and added a variety of smaller improvements.

\subsubsection{Note Added April 7, 2019}\label{note201904}

This version clarified the proper substitution notation, improved the
discussion on the weak deduction theorem and natural deduction,
documented the \texttt{undo} command, updated the information on
\texttt{write source}, changed the typecode
from \texttt{set} to \texttt{setvar} to be consistent with the current
version of \texttt{set.mm}, added more documentation about comment markup
(e.g., documented how to create headings), and clarified the
differences between various assertion forms (in particular deduction form).

\subsubsection{Note Added June 2, 2019}\label{note201906}

This version fixes a large number of small issues reported by
Beno\^{i}t Jubin\index{Jubin, Beno\^{i}t}, such as editorial issues
and the need to document \texttt{verify markup} (thank you!).
This version also includes specific examples
of forms (deduction form, inference form, and closed form).
We call this version the ``second edition'';
the previous edition formally published in 2007 had a slightly different title
(\textit{Metamath: A Computer Language for Pure Mathematics}).

\chapter{Introduction}
\pagenumbering{arabic}

\begin{quotation}
  {\em {\em I.M.:}  No, no.  There's nothing subjective about it!  Everybody
knows what a proof is.  Just read some books, take courses from a competent
mathematician, and you'll catch on.

{\em Student:}  Are you sure?

{\em I.M.:}  Well---it is possible that you won't, if you don't have any
aptitude for it.  That can happen, too.

{\em Student:}  Then {\em you} decide what a proof is, and if I don't learn
to decide in the same way, you decide I don't have any aptitude.

{\em I.M.:}  If not me, then who?}
    \flushright\sc  ``The Ideal Mathematician''
    \index{Davis, Phillip J.}
    \footnote{\cite{Davis}, p.~40.}\\
\end{quotation}

Brilliant mathematicians have discovered almost
unimaginably profound results that rank among the crowning intellectual
achievements of mankind.  However, there is a sense in which modern abstract
mathematics is behind the times, stuck in an era before computers existed.
While no one disputes the remarkable results that have been achieved,
communicating these results in a precise way to the uninitiated is virtually
impossible.  To describe these results, a terse informal language is used which
despite its elegance is very difficult to learn.  This informal language is not
imprecise, far from it, but rather it often has omitted detail
and symbols with hidden context that are
implicitly understood by an expert but few others.  Extremely complex technical
meanings are associated with innocent-sounding English words such as
``compact'' and ``measurable'' that barely hint at what is actually being
said.  Anyone who does not keep the precise technical meaning constantly in
mind is bound to fail, and acquiring the ability to do this can be achieved
only through much practice and hard work.  Only the few who complete the
painful learning experience can join the small in-group of pure
mathematicians.  The informal language effectively cuts off the true nature of
their knowledge from most everyone else.

Metamath\index{Metamath} makes abstract mathematics more concrete.  It allows
a computer to keep track of the complexity associated with each word or symbol
with absolute rigor.  You can explore this complexity at your leisure, to
whatever degree you desire.  Whether or not you believe that concepts such as
infinity actually ``exist'' outside of the mind, Metamath lets you get to the
foundation for what's really being said.

Metamath also enables completely rigorous and thorough proof verification.
Its language is simple enough so that you
don't have to rely on the authority of experts but can verify the results
yourself, step by step.  If you want to attempt to derive your own results,
Metamath will not let you make a mistake in reasoning.
Even professional mathematicians make mistakes; Metamath makes it possible
to thoroughly verify that proofs are correct.

Metamath\index{Metamath} is a computer language and an associated computer
program for archiving, verifying, and studying mathematical proofs at a very
detailed level.
The Metamath language
describes formal\index{formal system} mathematical
systems and expresses proofs of theorems in those systems.  Such a language
is called a metalanguage\index{metalanguage} by mathematicians.
The Metamath program is a computer program that verifies
proofs expressed in the Metamath language.
The Metamath program does not have the built-in
ability to make logical inferences; it just makes a series of symbol
substitutions according to instructions given to it in a proof
and verifies that the result matches the expected theorem.  It makes logical
inferences based only on rules of logic that are contained in a set of
axioms\index{axiom}, or first principles, that you provide to it as the
starting point for proofs.

The complete specification of the Metamath language is only four pages long
(Section~\ref{spec}, p.~\pageref{spec}).  Its simplicity may at first make you
wonder how it can do much of anything at all.  But in fact the kinds of
symbol manipulations it performs are the ones that are implicitly done in all
mathematical systems at the lowest level.  You can learn it relatively quickly
and have complete confidence in any mathematical proof that it verifies.  On
the other hand, it is powerful and general enough so that virtually any
mathematical theory, from the most basic to the deeply abstract, can be
described with it.

Although in principle Metamath can be used with any
kind of mathematics, it is best suited for abstract or ``pure'' mathematics
that is mostly concerned with theorems and their proofs, as opposed to the
kind of mathematics that deals with the practical manipulation of numbers.
Examples of branches of pure mathematics are logic\index{logic},\footnote{Logic
is the study of statements that are universally true regardless of the objects
being described by the statements.  An example is the statement, ``if $P$
implies $Q$, then either $P$ is false or $Q$ is true.''} set theory\index{set
theory},\footnote{Set theory is the study of general-purpose mathematical objects called
``sets,'' and from it essentially all of mathematics can be derived.  For
example, numbers can be defined as specific sets, and their properties
can be explored using the tools of set theory.} number theory\index{number
theory},\footnote{Number theory deals with the properties of positive and
negative integers (whole numbers).} group theory\index{group
theory},\footnote{Group theory studies the properties of mathematical objects
called groups that obey a simple set of axioms and have properties of symmetry
that make them useful in many other fields.} abstract algebra\index{abstract
algebra},\footnote{Abstract algebra includes group theory and also studies
groups with additional properties that qualify them as ``rings'' and
``fields.''  The set of real numbers is a familiar example of a field.},
analysis\index{analysis} \index{real and complex numbers}\footnote{Analysis is
the study of real and complex numbers.} and
topology\index{topology}.\footnote{One area studied by topology are properties
that remain unchanged when geometrical objects undergo stretching
deformations; for example a doughnut and a coffee cup each have one hole (the
cup's hole is in its handle) and are thus considered topologically
equivalent.  In general, though, topology is the study of abstract
mathematical objects that obey a certain (surprisingly simple) set of axioms.
See, for example, Munkres \cite{Munkres}\index{Munkres, James R.}.} Even in
physics, Metamath could be applied to certain branches that make use of
abstract mathematics, such as quantum logic\index{quantum logic} (used to study
aspects of quantum mechanics\index{quantum mechanics}).

On the other hand, Metamath\index{Metamath} is less suited to applications
that deal primarily with intensive numeric computations.  Metamath does not
have any built-in representation of numbers\index{Metamath!representation of
numbers}; instead, a specific string of symbols (digits) must be syntactically
constructed as part of any proof in which an ordinary number is used.  For
this reason, numbers in Metamath are best limited to specific constants that
arise during the course of a theorem or its proof.  Numbers are only a tiny
part of the world of abstract mathematics.  The exclusion of built-in numbers
was a conscious decision to help achieve Metamath's simplicity, and there are
other software tools if you have different mathematical needs.
If you wish to quickly solve algebraic problems, the computer algebra
programs\index{computer algebra system} {\sc
macsyma}\index{macsyma@{\sc macsyma}}, Mathematica\index{Mathematica}, and
Maple\index{Maple} are specifically suited to handling numbers and
algebra efficiently.
If you wish to simply calculate numeric or matrix expressions easily,
tools such as Octave\index{Octave} may be a better choice.

After learning Metamath's basic statement types, any
tech\-ni\-cal\-ly ori\-ent\-ed person, mathematician or not, can
immediately trace
any theorem proved in the language as far back as he or she wants, all the way
to the axioms on which the theorem is based.  This ability suggests a
non-traditional way of learning about pure mathematics.  Used in conjunction
with traditional methods, Metamath could make pure mathematics accessible to
people who are not sufficiently skilled to figure out the implicit detail in
ordinary textbook proofs.  Once you learn the axioms of a theory, you can have
complete confidence that everything you need to understand a proof you are
studying is all there, at your beck and call, allowing you to focus in on any
proof step you don't understand in as much depth as you need, without worrying
about getting stuck on a step you can't figure out.\footnote{On the other
hand, writing proofs in the Metamath language is challenging, requiring
a degree of rigor far in excess of that normally taught to students.  In a
classroom setting, I doubt that writing Metamath proofs would ever replace
traditional homework exercises involving informal proofs, because the time
needed to work out the details would not allow a course to
cover much material.  For students who have trouble grasping the implied rigor
in traditional material, writing a few simple proofs in the Metamath language
might help clarify fuzzy thought processes.  Although somewhat difficult at
first, it eventually becomes fun to do, like solving a puzzle, because of the
instant feedback provided by the computer.}

Metamath\index{Metamath} is probably unlike anything you have
encountered before.  In this first chapter we will look at the philosophy and
use of computers in mathematics in order to better understand the motivation
behind Metamath.  The material in this chapter is not required in order to use
Metamath.  You may skip it if you are impatient, but I hope you will find it
educational and enjoyable.  If you want to start experimenting with the
Metamath program right away, proceed directly to Chapter~\ref{using}
(p.~\pageref{using}).  To
learn the Metamath language, skim Chapter~\ref{using} then proceed to
Chapter~\ref{languagespec} (p.~\pageref{languagespec}).

\section{Mathematics as a Computer Language}

\begin{quote}
  {\em The study of mathematics is apt to commence in
dis\-ap\-point\-ment.\ldots \\
We are told that by its aid the stars are weighted
and the billions of molecules in a drop of water are counted.  Yet, like the
ghost of Hamlet's father, this great science eludes the efforts of our mental
weapons to grasp it.}
  \flushright\sc  Alfred North Whitehead\footnote{\cite{Whitehead}, ch.\ 1.}\\
\end{quote}\index{Whitehead, Alfred North}

\subsection{Is Mathematics ``User-Friendly''?}

Suppose you have no formal training in abstract mathematics.  But popular
books you've read offer tempting glimpses of this world filled with profound
ideas that have stirred the human spirit.  You are not satisfied with the
informal, watered-down descriptions you've read but feel it is important to
grasp the underlying mathematics itself to understand its true meaning. It's
not practical to go back to school to learn it, though; you don't want to
dedicate years of your life to it.  There are many important things in life,
and you have to set priorities for what's important to you.  What would happen
if you tried to pursue it on your own, in your spare time?

After all, you were able to learn a computer programming language such as
Pascal on your own without too much difficulty, even though you had no formal
training in computers.  You don't claim to be an expert in software design,
but you can write a passable program when necessary to suit your needs.  Even
more important, you know that you can look at anyone else's Pascal program, no
matter how complex, and with enough patience figure out exactly how it works,
even though you are not a specialist.  Pascal allows you do anything that a
computer can do, at least in principle.  Thus you know you have the ability,
in principle, to follow anything that a computer program can do:  you just
have to break it down into small enough pieces.

Here's an imaginary scenario of what might happen if you na\-ive\-ly a\-dopted
this same view of abstract mathematics and tried to pick it up on your own, in
a period of time comparable to, say, learning a computer programming
language.

\subsubsection{A Non-Mathematician's Quest for Truth}

\begin{quote}
  {\em \ldots my daughters have been studying (chemistry) for several
se\-mes\-ters, think they have learned differential and integral calculus in
school, and yet even today don't know why $x\cdot y=y\cdot x$ is true.}
  \flushright\sc  Edmund Landau\footnote{\cite{Landau}, p.~vi.}\\
\end{quote}\index{Landau, Edmund}

\begin{quote}
  {\em Minus times minus is plus,\\
The reason for this we need not discuss.}
  \flushright\sc W.\ H.\ Auden\footnote{As quoted in \cite{Guillen}, p.~64.}\\
\end{quote}\index{Auden, W.\ H.}\index{Guillen, Michael}

We'll suppose you are a technically oriented professional, perhaps an engineer, a
computer programmer, or a physicist, but probably not a mathematician.  You
consider yourself reasonably intelligent.  You did well in school, learning a
variety of methods and techniques in practical mathematics such as calculus and
differential equations.  But rarely did your courses get into anything
resembling modern abstract mathematics, and proofs were something that appeared
only occasionally in your textbooks, a kind of necessary evil that was
supposed to convince you of a certain key result.  Most of your
homework consisted of exercises that gave you practice in the techniques, and
you were hardly ever asked to come up with a proof of your own.

You find yourself curious about advanced, abstract mathematics.  You are
driven by an inner conviction that it is important to understand and
appreciate some of the most profound knowledge discovered by mankind.  But it
seems very hard to learn, something that only certain gifted longhairs can
access and understand.  You are frustrated that it seems forever cut off from
you.

Eventually your curiosity drives you to do something about it.
You set for yourself a goal of ``really'' understanding mathematics:  not just
how to manipulate equations in algebra or calculus according to cookbook
rules, but rather to gain a deep understanding of where those rules come from.
In fact, you're not thinking about this kind of ordinary mathematics at all,
but about a much more abstract, ethereal realm of pure mathematics, where
famous results such as G\"{o}del's incompleteness theorem\index{G\"{o}del's
incompleteness theorem} and Cantor's different kinds of infinities
reside.

You have probably read a number of popular books, with titles like {\em
Infinity and the Mind} \cite{Rucker}\index{Rucker, Rudy}, on topics such as
these.  You found them inspiring but at the same time somewhat
unsatisfactory.  They gave you a general idea of what these results are about,
but if someone asked you to prove them, you wouldn't have the faintest idea of
where to begin.   Sure, you could give the same overall outline that you
learned from the popular books; and in a general sort of way, you do have an
understanding.  But deep down inside, you know that there is a rigor that is
missing, that probably there are many subtle steps and pitfalls along the way,
and ultimately it seems you have to place your trust in the experts in the
field.  You don't like this; you want to be able to verify these results for
yourself.

So where do you go next?  As a first step, you decide to look up some of the
original papers on the theorems you are curious about, or better, obtain some
standard textbooks in the field.  You look up a theorem you want to
understand.  Sure enough, it's there, but it's expressed with strange
terms and odd symbols that mean absolutely nothing to you.  It might as well be written in
a foreign language you've never seen before, whose symbols are totally alien.
You look at the proof, and you haven't the foggiest notion what each step
means, much less how one step follows from another.  Well, obviously you have
a lot to learn if you want to understand this stuff.

You feel that you could probably understand it by
going back to college for another three to six years and getting a math
degree.  But that does not fit in with your career and the other things in
your life and would serve no practical purpose.  You decide to seek a quicker
path.  You figure you'll just trace your way back to the beginning, step by
step, as you would do with a computer program, until you understand it.  But
you quickly find that this is not possible, since you can't even understand
enough to know what you have to trace back to.

Maybe a different approach is in order---maybe you should start at the
beginning and work your way up.  First, you read the introduction to the book
to find out what the prerequisites are.  In a similar fashion, you trace your
way back through two or three more books, finally arriving at one that seems
to start at a beginning:  it lists the axioms of arithmetic.  ``Aha!'' you
naively think, ``This must be the starting point, the source of all mathematical
knowledge.'' Or at least the starting point for mathematics dealing with
numbers; you have to start somewhere and have no idea what the starting point
for other mathematics would be.  But the word ``axioms'' looks promising.  So
you eagerly read along and work through some elementary exercises at the
beginning of the book.  You feel vaguely bothered:  these
don't seem like axioms at all, at least not in the sense that you want to
think of axioms.  Axioms imply a starting point from which everything else can
be built up, according to precise rules specified in the axiom system.  Even
though you can understand the first few proofs in an informal way,
and are able to do some of the
exercises, it's hard to pin down precisely what the
rules are.   Sure, each step seems to follow logically from the others, but
exactly what does that mean?  Is the ``logic'' just a matter of common sense,
something vague that we all understand but can never quite state precisely?

You've spent a number of years, off and on, programming computers, and you
know that in the case of computer languages there is no question of what the
rules are---they are precise and crystal clear.  If you follow them, your
program will work, and if you don't, it won't.  No matter how complex a
program, it can always be broken down into simpler and simpler pieces, until
you can ultimately identify the bits that are moved around to perform a
specific function.  Some programs might require a lot of perseverance to
accomplish this, but if you focus on a specific portion of it, you don't even
necessarily have to know how the rest of it works. Shouldn't there be an
analogy in mathematics?

You decide to apply the ultimate test:  you ask yourself how a computer could
verify or ensure that the steps in these proofs follow from one another.
Certainly mathematics must be at least as precisely defined as a computer
language, if not more so; after all, computer science itself is based on it.
If you can get a computer to verify these proofs, then you should also be
able, in principle, to understand them yourself in a very crystal clear,
precise way.

You're in for a surprise:  you can conceive of no way to convert the
proofs, which are in English, to a form that the computer can understand.
The proofs are filled with phrases such as ``assume there exists a unique
$x$\ldots'' and ``given any $y$, let $z$ be the number such that\ldots''  This
isn't the kind of logic you are used to in computer programming, where
everything, even arithmetic, reduces to Boolean ones and zeroes if you care to
break it down sufficiently.  Even though you think you understand the proofs,
there seems to be some kind of higher reasoning involved rather than precise
rules that define how you manipulate the symbols in the axioms.  Whatever it
is, it just isn't obvious how you would express it to a computer, and the more
you think about it, the more puzzled and confused you get, to the point where
you even wonder whether {\em you} really understand it.  There's a lot more to
these axioms of arithmetic than meets the eye.

Nobody ever talked about this in school in your applied math and engineering
courses.  You just learned the rules they gave you, not quite understanding
how or why they worked, sometimes vaguely suspicious or uncertain of them, and
through homework problems and osmosis learned how to present solutions that
satisfied the instructor and earned you an ``A.''  Rarely did you actually
``prove'' anything in a rigorous way, and the math majors who did do stuff
like that seemed to be in a different world.

Of course, there are computer algebra programs that can do mathematics, and
rather impressively.  They can instantly solve the integrals that you
struggled with in freshman calculus, and do much, much more.  But when you
look at these programs, what you see is a big collection of algorithms and
techniques that evolved and were added to over time, along with some basic
software that manipulates symbols.  Each algorithm that is built in is the
result of someone's theorem whose proof is omitted; you just have to trust the
person who proved it and the person who programmed it in and hope there are no
bugs.\index{computer program bugs}  Somehow this doesn't seem to be the
essence of mathematics.  Although computer algebra systems can generate
theorems with amazing speed, they can't actually prove a single one of them.

After some puzzlement, you revisit some popular books on what mathematics is
all about.  Somewhere you read that all of mathematics is actually derived
from something called ``set theory.''  This is a little confusing, because
nowhere in the book that presented the axioms of arithmetic was there any
mention of set theory, or if there was, it seemed to be just a tool that helps
you describe things better---the set of even numbers, that sort of thing.  If
set theory is the basis for all mathematics, then why are additional axioms
needed for arithmetic?

Something is wrong but you're not sure what.  One of your friends is a pure
mathematician.  He knows he is unable to communicate to you what he does for a
living and seems to have little interest in trying.  You do know that for him,
proofs are what mathematics is all about. You ask him what a proof is, and he
essentially tells you that, while of course it's based on logic, really it's
something you learn by doing it over and over until you pick it up.  He refers
you to a book, {\em How to Read and Do Proofs} \cite{Solow}.\index{Solow,
Daniel}  Although this book helps you understand traditional informal proofs,
there is still something missing you can't seem to pin down yet.

You ask your friend how you would go about having a computer verify a proof.
At first he seems puzzled by the question; why would you want to do that?
Then he says it's not something that would make any sense to do, but he's
heard that you'd have to break the proof down into thousands or even millions
of individual steps to do such a thing, because the reasoning involved is at
such a high level of abstraction.  He says that maybe it's something you could
do up to a point, but the computer would be completely impractical once you
get into any meaningful mathematics.  There, the only way you can verify a
proof is by hand, and you can only acquire the ability to do this by
specializing in the field for a couple of years in grad school.  Anyway, he
thinks it all has to do with set theory, although he has never taken a formal
course in set theory but just learned what he needed as he went along.

You are intrigued and amazed.  Apparently a mathematician can grasp as a
single concept something that would take a computer a thousand or a million
steps to verify, and have complete confidence in it.  Each one of these
thousand or million steps must be absolutely correct, or else the whole proof
is meaningless.  If you added a million numbers by hand, would you trust the
result?  How do you really know that all these steps are correct, that there
isn't some subtle pitfall in one of these million steps, like a bug in a
computer program?\index{computer program bugs}  After all, you've read that
famous mathematicians have occasionally made mistakes, and you certainly know
you've made your share on your math homework problems in school.

You recall the analogy with a computer program.  Sure, you can understand what
a large computer program such as a word processor does, as a single high-level
concept or a small set of such concepts, but your ability to understand it in
no way ensures that the program is correct and doesn't have hidden bugs.  Even
if you wrote the program yourself you can't really know this; most large
programs that you've written have had bugs that crop up at some later date, no
matter how careful you tried to be while writing them.

OK, so now it seems the reason you can't figure out how to make a
computer verify proofs is because each step really corresponds to a
million small steps.  Well, you say, a computer can do a million
calculations in a second, so maybe it's still practical to do.  Now the
puzzle becomes how to figure out what the million steps are that each
English-language step corresponds to.  Your mathematician friend hasn't
a clue, but suggests that maybe you would find the answer by studying
set theory.  Actually, your friend thinks you're a little off the wall
for even wondering such a thing.  For him, this is not what mathematics
is all about.

The subject of set theory keeps popping up, so you decide it's
time to look it up.

You decide to start off on a careful footing, so you start reading a couple of
very elementary books on set theory.  A lot of it seems pretty obvious, like
intersections, subsets, and Venn diagrams.  You thumb through one of the
books; nowhere is anything about axioms mentioned. The other book relegates to
an appendix a brief discussion that mentions a set of axioms called
``Zermelo--Fraenkel set theory''\index{Zermelo--Fraenkel set theory} and states
them in English.  You look at them and have no idea what they really mean or
what you can do with them.  The comments in this appendix say that the purpose
of mentioning them is to expose you to the idea, but imply that they are not
necessary for basic understanding and that they are really the subject matter
of advanced treatments where fine points such as a certain paradox (Russell's
paradox\index{Russell's paradox}\footnote{Russell's paradox assumes that there
exists a set $S$ that is a collection of all sets that don't contain
themselves.  Now, either $S$ contains itself or it doesn't.  If it contains
itself, it contradicts its definition.  But if it doesn't contain itself, it
also contradicts its definition.  Russell's paradox is resolved in ZF set
theory by denying that such a set $S$ exists.}) are resolved.  Wait a
minute---shouldn't the axioms be a starting point, not an ending point?  If
there are paradoxes that arise without the axioms, how do you know you won't
stumble across one accidentally when using the informal approach?

And nowhere do these books describe how ``all of mathematics can be
derived from set theory'' which by now you've heard a few times.

You find a more advanced book on set theory.  This one actually lists the
axioms of ZF set theory in plain English on page one.  {\em Now} you think
your quest has ended and you've finally found the source of all mathematical
knowledge; you just have to understand what it means.  Here, in one place, is
the basis for all of mathematics!  You stare at the axioms in awe, puzzle over
them, memorize them, hoping that if you just meditate on them long enough they
will become clear.  Of course, you haven't the slightest idea how the rest of
mathematics is ``derived'' from them; in particular, if these are the axioms
of mathematics, then why do arithmetic, group theory, and so on need their own
axioms?

You start reading this advanced book carefully, pondering the meaning of every
word, because by now you're really determined to get to the bottom of this.
The first thing the book does is explain how the axioms came about, which was
to resolve Russell's paradox.\index{Russell's paradox}  In fact that seems to
be the main purpose of their existence; that they supposedly can be used to
derive all of mathematics seems irrelevant and is not even mentioned.  Well,
you go on.  You hope the book will explain to you clearly, step by step, how
to derive things from the axioms.  After all, this is the starting point of
mathematics, like a book that explains the basics of a computer programming
language.  But something is missing.  You find you can't even understand the
first proof or do the first exercise.  Symbols such as $\exists$ and $\forall$
permeate the page without any mention of where they came from or how to
manipulate them; the author assumes you are totally familiar with them and
doesn't even tell you what they mean.  By now you know that $\exists$ means
``there exists'' and $\forall$ means ``for all,'' but shouldn't the rules for
manipulating these symbols be part of the axioms?  You still have no idea
how you could even describe the axioms to a computer.

Certainly there is something much different here from the technical
literature you're used to reading.  A computer language manual almost
always explains very clearly what all the symbols mean, precisely what
they do, and the rules used for combining them, and you work your way up
from there.

After glancing at four or five other such books, you come to the realization
that there is another whole field of study that you need just to get to the
point at which you can understand the axioms of set theory.  The field is
called ``logic.''  In fact, some of the books did recommend it as a
prerequisite, but it just didn't sink in.  You assumed logic was, well, just
logic, something that a person with common sense intuitively understood.  Why
waste your time reading boring treatises on symbolic logic, the manipulation
of 1's and 0's that computers do, when you already know that?  But this is a
different kind of logic, quite alien to you.  The subject of {\sc nand} and
{\sc nor} gates is not even touched upon or in any case has to do with only a
very small part of this field.

So your quest continues.  Skimming through the first couple of introductory
books, you get a general idea of what logic is about and what quantifiers
(``for all,'' ``there exists'') mean, but you find their examples somewhat
trivial and mildly annoying (``all dogs are animals,'' ``some animals are
dogs,'' and such).  But all you want to know is what the rules are for
manipulating the symbols so you can apply them to set theory.  Some formulas
describing the relationships among quantifiers ($\exists$ and $\forall$) are
listed in tables, along with some verbal reasoning to justify them.
Presumably, if you want to find out if a formula is correct, you go through
this same kind of mental reasoning process, possibly using images of dogs and
animals. Intuitively, the formulas seem to make sense.  But when you ask
yourself, ``What are the rules I need to get a computer to figure out whether
this formula is correct?'', you still don't know.  Certainly you don't ask the
computer to imagine dogs and animals.

You look at some more advanced logic books.  Many of them have an introductory
chapter summarizing set theory, which turns out to be a prerequisite.  You
need logic to understand set theory, but it seems you also need set theory to
understand logic!  These books jump right into proving rather advanced
theorems about logic, without offering the faintest clue about where the logic
came from that allows them to prove these theorems.

Luckily, you come across an elementary book of logic that, halfway through,
after the usual truth tables and metaphors, presents in a clear, precise way
what you've been looking for all along: the axioms!  They're divided into
propositional calculus (also called sentential logic) and predicate calculus
(also called first-order logic),\index{first-order logic} with rules so simple
and crystal clear that now you can finally program a computer to understand
them.  Indeed, they're no harder than learning how to play a game of chess.
As far as what you seem to need is concerned, the whole book could have been
written in five pages!

{\em Now} you think you've found the ultimate source of mathematical
truth.  So---the axioms of mathematics consist of these axioms of logic,
together with the axioms of ZF set theory. (By now you've also been able to
figure out how to translate the ZF axioms from English into the
actual symbols of logic which you can now manipulate according to
precise, easy-to-understand rules.)

Of course, you still don't understand how ``all of mathematics can be
derived from set theory,'' but maybe this will reveal itself in due
course.

You eagerly set out to program the axioms and rules into a computer and start
to look at the theorems you will have to prove as the logic is developed.  All
sorts of important theorems start popping up:  the deduction
theorem,\index{deduction theorem} the substitution theorem,\index{substitution
theorem} the completeness theorem of propositional calculus,\index{first-order
logic!completeness} the completeness theorem of predicate calculus.  Uh-oh,
there seems to be trouble.  They all get harder and harder, and not one of
them can be derived with the axioms and rules of logic you've just been
handed.  Instead, they all require ``metalogic'' for their proofs, a kind of
mixture of logic and set theory that allows you to prove things {\em about}
the axioms and theorems of logic rather than {\em with} them.

You plow ahead anyway.  A month later, you've spent much of your
free time getting the computer to verify proofs in propositional calculus.
You've programmed in the axioms, but you've also had to program in the
deduction theorem, the substitution theorem, and the completeness theorem of
propositional calculus, which by now you've resigned yourself to treating as
rather complex additional axioms, since they can't be proved from the axioms
you were given.  You can now get the computer to verify and even generate
complete, rigorous, formal proofs\index{formal proof}.  Never mind that they
may have 100,000 steps---at least now you can have complete, absolute
confidence in them.  Unfortunately, the only theorems you have proved are
pretty trivial and you can easily verify them in a few minutes with truth
tables, if not by inspection.

It looks like your mathematician friend was right.  Getting the computer to do
serious mathematics with this kind of rigor seems almost hopeless.  Even
worse, it seems that the further along you get, the more ``axioms'' you have
to add, as each new theorem seems to involve additional ``metamathematical''
reasoning that hasn't been formalized, and none of it can be derived from the
axioms of logic.  Not only do the proofs keep growing exponentially as you get
further along, but the program to verify them keeps getting bigger and bigger
as you program in more ``metatheorems.''\index{metatheorem}\footnote{A
metatheorem is usually a statement that is too general to be directly provable
in a theory.  For example, ``if $n_1$, $n_2$, and $n_3$ are integers, then
$n_1+n_2+n_3$ is an integer'' is a theorem of number theory.  But ``for any
integer $k > 1$, if $n_1, \ldots, n_k$ are integers, then $n_1+\ldots +n_k$ is
an integer'' is a metatheorem, in other words a family of theorems, one for
every $k$.  The reason it is not a theorem is that the general sum $n_1+\ldots
+n_k$ (as a function of $k$) is not an operation that can be defined directly
in number theory.} The bugs\index{computer program bugs} that have cropped up
so far have already made you start to lose faith in the rigor you seem to have
achieved, and you know it's just going to get worse as your program gets larger.

\subsection{Mathematics and the Non-Specialist}

\begin{quote}
  {\em A real proof is not checkable by a machine, or even by any mathematician
not privy to the gestalt, the mode of thought of the particular field of
mathematics in which the proof is located.}
  \flushright\sc  Davis and Hersh\index{Davis, Phillip J.}
  \footnote{\cite{Davis}, p.~354.}\\
\end{quote}

The bulk of abstract or theoretical mathematics is ordinarily outside
the reach of anyone but a few specialists in each field who have completed
the necessary difficult internship in order to enter its coterie.  The
typical intelligent layperson has no reasonable hope of understanding much of
it, nor even the specialist mathematician of understanding other fields.  It
is like a foreign language that has no dictionary to look up the translation;
the only way you can learn it is by living in the country for a few years.  It
is argued that the effort involved in learning a specialty is a necessary
process for acquiring a deep understanding.  Of course, this is almost certainly
true if one is to make significant contributions to a field; in particular,
``doing'' proofs is probably the most important part of a mathematician's
training.  But is it also necessary to deny outsiders access to it?  Is it
necessary that abstract mathematics be so hard for a layperson to grasp?

A computer normally is of no help whatsoever.  Most published proofs are
actually just series of hints written in an informal style that requires
considerable knowledge of the field to understand.  These are the ``real
proofs'' referred to by Davis and Hersh.\index{informal proof}  There is an
implicit understanding that, in principle, such a proof could be converted to
a complete formal proof\index{formal proof}.  However, it is said that no one
would ever attempt such a conversion, even if they could, because that would
presumably require millions of steps (Section~\ref{dream}).  Unfortunately the
informal style automatically excludes the understanding of the proof
by anyone who hasn't gone through the necessary apprenticeship. The
best that the intelligent layperson can do is to read popular books about deep
and famous results; while this can be helpful, it can also be misleading, and
the lack of detail usually leaves the reader with no ability whatsoever to
explore any aspect of the field being described.

The statements of theorems often use sophisticated notation that makes them
inaccessible to the non-specialist.  For a non-specialist who wants to achieve
a deeper understanding of a proof, the process of tracing definitions and
lemmas back through their hierarchy\index{hierarchy} quickly becomes confusing
and discouraging.  Textbooks are usually written to train mathematicians or to
communicate to people who are already mathematicians, and large gaps in proofs
are often left as exercises to the reader who is left at an impasse if he or
she becomes stuck.

I believe that eventually computers will enable non-specialists and even
intelligent laypersons to follow almost any mathematical proof in any field.
Metamath is an attempt in that direction.  If all of mathematics were as
easily accessible as a computer programming language, I could envision
computer programmers and hobbyists who otherwise lack mathematical
sophistication exploring and being amazed by the world of theorems and proofs
in obscure specialties, perhaps even coming up with results of their own.  A
tremendous advantage would be that anyone could experiment with conjectures in
any field---the computer would offer instant feedback as to whether
an inference step was correct.

Mathematicians sometimes have to put up with the annoyance of
cranks\index{cranks} who lack a fundamental understanding of mathematics but
insist that their ``proofs'' of, say, Fermat's Last Theorem\index{Fermat's
Last Theorem} be taken seriously.  I think part of the problem is that these
people are misled by informal mathematical language, treating it as if they
were reading ordinary expository English and failing to appreciate the
implicit underlying rigor.  Such cranks are rare in the field of computers,
because computer languages are much more explicit, and ultimately the proof is
in whether a computer program works or not.  With easily accessible
computer-based abstract mathematics, a mathematician could say to a crank,
``don't bother me until you've demonstrated your claim on the computer!''

% 22-May-04 nm
% Attempt to move De Millo quote so it doesn't separate from attribution
% CHANGE THIS NUMBER (AND ELIMINATE IF POSSIBLE) WHEN ABOVE TEXT CHANGES
\vspace{-0.5em}

\subsection{An Impossible Dream?}\label{dream}

\begin{quote}
  {\em Even quite basic theorems would demand almost unbelievably vast
  books to display their proofs.}
    \flushright\sc  Robert E. Edwards\footnote{\cite{Edwards}, p.~68.}\\
\end{quote}\index{Edwards, Robert E.}

\begin{quote}
  {\em Oh, of course no one ever really {\em does} it.  It would take
  forever!  You just show that you could do it, that's sufficient.}
    \flushright\sc  ``The Ideal Mathematician''
    \index{Davis, Phillip J.}\footnote{\cite{Davis},
p.~40.}\\
\end{quote}

\begin{quote}
  {\em There is a theorem in the primitive notation of set theory that
  corresponds to the arithmetic theorem `$1000+2000=3000$'.  The formula
  would be forbiddingly long\ldots even if [one] knows the definitions
  and is asked to simplify the long formula according to them, chances are
  he will make errors and arrive at some incorrect result.}
    \flushright\sc  Hao Wang\footnote{\cite{Wang}, p.~140.}\\
\end{quote}\index{Wang, Hao}

% 22-May-04 nm
% Attempt to move De Millo quote so it doesn't separate from attribution
% CHANGE THIS NUMBER (AND ELIMINATE IF POSSIBLE) WHEN ABOVE TEXT CHANGES
\vspace{-0.5em}

\begin{quote}
  {\em The {\em Principia Mathematica} was the crowning achievement of the
  formalists.  It was also the deathblow of the formalist view.\ldots
  {[Rus\-sell]} failed, in three enormous volumes, to get beyond the elementary
  facts of arithmetic.  He showed what can be done in principle and what
  cannot be done in practice.  If the mathematical process were really
  one of strict, logical progression, we would still be counting our
  fingers.\ldots
  One theoretician estimates\ldots that a demonstration of one of
  Ramanujan's conjectures assuming set theory and elementary analysis would
  take about two thousand pages; the length of a deduction from first principles
  is nearly in\-con\-ceiv\-a\-ble\ldots The probabilists argue that\ldots any
  very long proof can at best be viewed as only probably correct\ldots}
  \flushright\sc Richard de Millo et. al.\footnote{\cite{deMillo}, pp.~269,
  271.}\\
\end{quote}\index{de Millo, Richard}

A number of writers have conveyed the impression that the kind of absolute
rigor provided by Metamath\index{Metamath} is an impossible dream, suggesting
that a complete, formal verification\index{formal proof} of a typical theorem
would take millions of steps in untold volumes of books.  Even if it could be
done, the thinking sometimes goes, all meaning would be lost in such a
monstrous, tedious verification.\index{informal proof}\index{proof length}

These writers assume, however, that in order to achieve the kind of complete
formal verification they desire one must break down a proof into individual
primitive steps that make direct reference to the axioms.  This is
not necessary.  There is no reason not to make use of previously proved
theorems rather than proving them over and over.

Just as important, definitions\index{definition} can be introduced along
the way, allowing very complex formulas to be represented with few
symbols.  Not doing this can lead to absurdly long formulas.  For
example, the mere statement of
G\"{o}del's incompleteness theorem\index{G\"{o}del's
incompleteness theorem}, which can be expressed with a small number of
defined symbols, would require about 20,000 primitive symbols to express
it.\index{Boolos, George S.}\footnote{George S.\ Boolos, lecture at
Massachusetts Institute of Technology, spring 1990.} An extreme example
is Bourbaki's\label{bourbaki} language for set theory, which requires
4,523,659,424,929 symbols plus 1,179,618,517,981 disambiguatory links
(lines connecting symbol pairs, usually drawn below or above the
formula) to express the number
``one'' \cite{Mathias}.\index{Mathias, Adrian R. D.}\index{Bourbaki,
Nicolas}
% http://www.dpmms.cam.ac.uk/~ardm/

A hierarchy\index{hierarchy} of theorems and definitions permits an
exponential growth in the formula sizes and primitive proof steps to be
described with only a linear growth in the number of symbols used.  Of course,
this is how ordinary informal mathematics is normally done anyway, but with
Metamath\index{Metamath} it can be done with absolute rigor and precision.

\subsection{Beauty}


\begin{quote}
  {\em No one shall be able to drive us from the paradise that Cantor has
created for us.}
   \flushright\sc  David Hilbert\footnote{As quoted in \cite{Moore}, p.~131.}\\
\end{quote}\index{Hilbert, David}

\needspace{3\baselineskip}
\begin{quote}
  {\em Mathematics possesses not only truth, but some supreme beauty ---a
  beauty cold and austere, like that of a sculpture.}
    \flushright\sc  Bertrand
    Russell\footnote{\cite{Russell}.}\\
\end{quote}\index{Russell, Bertrand}

\begin{quote}
  {\em Euclid alone has looked on Beauty bare.}
  \flushright\sc Edna Millay\footnote{As quoted in \cite{Davis}, p.~150.}\\
\end{quote}\index{Millay, Edna}

For most people, abstract mathematics is distant, strange, and
incomprehensible.  Many popular books have tried to convey some of the sense
of beauty in famous theorems.  But even an intelligent layperson is left with
only a general idea of what a theorem is about and is hardly given the tools
needed to make use of it.  Traditionally, it is only after years of arduous
study that one can grasp the concepts needed for deep understanding.
Metamath\index{Metamath} allows you to approach the proof of the theorem from
a quite different perspective, peeling apart the formulas and definitions
layer by layer until an entirely different kind of understanding is achieved.
Every step of the proof is there, pieced together with absolute precision and
instantly available for inspection through a microscope with a magnification
as powerful as you desire.

A proof in itself can be considered an object of beauty.  Constructing an
elegant proof is an art.  Once a famous theorem has been proved, often
considerable effort is made to find simpler and more easily understood
proofs.  Creating and communicating elegant proofs is a major concern of
mathematicians.  Metamath is one way of providing a common language for
archiving and preserving this information.

The length of a proof can, to a certain extent, be considered an
objective measure of its ``beauty,'' since shorter proofs are usually
considered more elegant.  In the set theory database
\texttt{set.mm}\index{set theory database (\texttt{set.mm})}%
\index{Metamath Proof Explorer}
provided with Metamath, one goal was to make all proofs as short as possible.

\needspace{4\baselineskip}
\subsection{Simplicity}

\begin{quote}
  {\em God made man simple; man's complex problems are of his own
  devising.}
    \flushright\sc Eccles. 7:29\footnote{Jerusalem Bible.}\\
\end{quote}\index{Bible}

\needspace{3\baselineskip}
\begin{quote}
  {\em God made integers, all else is the work of man.}
    \flushright\sc Leopold Kronecker\footnote{{\em Jahresbericht
	der Deutschen Mathematiker-Vereinigung }, vol. 2, p. 19.}\\
\end{quote}\index{Kronecker, Leopold}

\needspace{3\baselineskip}
\begin{quote}
  {\em For what is clear and easily comprehended attracts; the
  complicated repels.}
    \flushright\sc David Hilbert\footnote{As quoted in \cite{deMillo},
p.~273.}\\
\end{quote}\index{Hilbert, David}

The Metamath\index{Metamath} language is simple and Spartan.  Metamath treats
all mathematical expressions as simple sequences of symbols, devoid of meaning.
The higher-level or ``metamathematical'' notions underlying Metamath are about
as simple as they could possibly be.  Each individual step in a proof involves
a single basic concept, the substitution of an expression for a variable, so
that in principle almost anyone, whether mathematician or not, can
completely understand how it was arrived at.

In one of its most basic applications, Metamath\index{Metamath} can be used to
develop the foundations of mathematics\index{foundations of mathematics} from
the very beginning.  This is done in the set theory database that is provided
with the Metamath package and is the subject matter
of Chapter~\ref{fol}. Any language (a metalanguage\index{metalanguage})
used to describe mathematics (an object language\index{object language}) must
have a mathematical content of its own, but it is desirable to keep this
content down to a bare minimum, namely that needed to make use of the
inference rules specified by the axioms.  With any metalanguage there is a
``chicken and egg'' problem somewhat like circular reasoning:  you must assume
the validity of the mathematics of the metalanguage in order to prove the
validity of the mathematics of the object language.  The mathematical content
of Metamath itself is quite limited.  Like the rules of a game of chess, the
essential concepts are simple enough so that virtually anyone should be able to
understand them (although that in itself will not let you play like
a master).  The symbols that Metamath manipulates do not in themselves
have any intrinsic meaning.  Your interpretation of the axioms that you supply
to Metamath is what gives them meaning.  Metamath is an attempt to strip down
mathematical thought to its bare essence and show you exactly how the symbols
are manipulated.

Philosophers and logicians, with various motivations, have often thought it
important to study ``weak'' fragments of logic\index{weak logic}
\cite{Anderson}\index{Anderson, Alan Ross} \cite{MegillBunder}\index{Megill,
Norman}\index{Bunder, Martin}, other unconventional systems of logic (such as
``modal'' logic\index{modal logic} \cite[ch.\ 27]{Boolos}\index{Boolos, George
S.}), and quantum logic\index{quantum logic} in physics
\cite{Pavicic}\index{Pavi{\v{c}}i{\'{c}}, M.}.  Metamath\index{Metamath}
provides a framework in which such systems can be expressed, with an absolute
precision that makes all underlying metamathematical assumptions rigorous and
crystal clear.

Some schools of philosophical thought, for example
intuitionism\index{intuitionism} and constructivism\index{constructivism},
demand that the notions underlying any mathematical system be as simple and
concrete as possible.  Metamath should meet the requirements of these
philosophies.  Metamath must be taught the symbols, axioms\index{axiom}, and
rules\index{rule} for a specific theory, from the skeptical (such as
intuitionism\index{intuitionism}\footnote{Intuitionism does not accept the law
of excluded middle (``either something is true or it is not true'').  See
\cite[p.~xi]{Tymoczko}\index{Tymoczko, Thomas} for discussion and references
on this topic.  Consider the theorem, ``There exist irrational numbers $a$ and
$b$ such that $a^b$ is rational.''  An intuitionist would reject the following
proof:  If $\sqrt{2}^{\sqrt{2}}$ is rational, we are done.  Otherwise, let
$a=\sqrt{2}^{\sqrt{2}}$ and $b=\sqrt{2}$. Then $a^b=2$, which is rational.})
to the bold (such as the axiom of choice in set theory\footnote{The axiom of
choice\index{Axiom of Choice} asserts that given any collection of pairwise
disjoint nonempty sets, there exists a set that has exactly one element in
common with each set of the collection.  It is used to prove many important
theorems in standard mathematics.  Some philosophers object to it because it
asserts the existence of a set without specifying what the set contains
\cite[p.~154]{Enderton}\index{Enderton, Herbert B.}.  In one foundation for
mathematics due to Quine\index{Quine, Willard Van Orman}, that has not been
otherwise shown to be inconsistent, the axiom of choice turns out to be false
\cite[p.~23]{Curry}\index{Curry, Haskell B.}.  The \texttt{show
trace{\char`\_}back} command of the Metamath program allows you to find out
whether the axiom of choice, or any other axiom, was assumed by a
proof.}\index{\texttt{show trace{\char`\_}back} command}).

The simplicity of the Metamath language lets the algorithm (computer program)
that verifies the validity of a Metamath proof be straightforward and
robust.  You can have confidence that the theorems it verifies really can be
derived from your axioms.

\subsection{Rigor}

\begin{quote}
  {\em Rigor became a goal with the Greeks\ldots But the efforts to
  pursue rigor to the utmost have led to an impasse in which there is
  no longer any agreement on what it really means.  Mathematics remains
  alive and vital, but only on a pragmatic basis.}
    \flushright\sc  Morris Kline\footnote{\cite{Kline}, p.~1209.}\\
\end{quote}\index{Kline, Morris}

Kline refers to a much deeper kind of rigor than that which we will discuss in
this section.  G\"{o}del's incompleteness theorem\index{G\"{o}del's
incompleteness theorem} showed that it is impossible to achieve absolute rigor
in standard mathematics because we can never prove that mathematics is
consistent (free from contradictions).\index{consistent theory}  If
mathematics is consistent, we will never know it, but must rely on faith.  If
mathematics is inconsistent, the best we can hope for is that some clever
future mathematician will discover the inconsistency.  In this case, the
axioms would probably be revised slightly to eliminate the inconsistency, as
was done in the case of Russell's paradox,\index{Russell's paradox} but the
bulk of mathematics would probably not be affected by such a discovery.
Russell's paradox, for example, did not affect most of the remarkable results
achieved by 19th-century and earlier mathematicians.  It mainly invalidated
some of Gottlob Frege's\index{Frege, Gottlob} work on the foundations of
mathematics in the late 1800's; in fact Frege's work inspired Russell's
discovery.  Despite the paradox, Frege's work contains important concepts that
have significantly influenced modern logic.  Kline's {\em Mathematics, The
Loss of Certainty} \cite{Klinel}\index{Kline, Morris} has an interesting
discussion of this topic.

What {\em can} be achieved with absolute certainty\index{certainty} is the
knowledge that if we assume the axioms are consistent and true, then the
results derived from them are true.  Part of the beauty of mathematics is that
it is the one area of human endeavor where absolute certainty can be achieved
in this sense.  A mathematical truth will remain such for eternity.  However,
our actual knowledge of whether a particular statement is a mathematical truth
is only as certain as the correctness of the proof that establishes it.  If
the proof of a statement is questionable or vague, we can't have absolute
confidence in the truth that the statement claims.

Let us look at some traditional ways of expressing proofs.

Except in the field of formal logic\index{formal logic}, almost all
traditional proofs in mathematics are really not proofs at all, but rather
proof outlines or hints as to how to go about constructing the proof.  Many
gaps\index{gaps in proofs} are left for the reader to fill in. There are
several reasons for this.  First, it is usually assumed in mathematical
literature that the person reading the proof is a mathematician familiar with
the specialty being described, and that the missing steps are obvious to such
a reader or at least that the reader is capable of filling them in.  This
attitude is fine for professional mathematicians in the specialty, but
unfortunately it often has the drawback of cutting off the rest of the world,
including mathematicians in other specialties, from understanding the proof.
We discussed one possible resolution to this on p.~\pageref{envision}.
Second, it is often assumed that a complete formal proof\index{formal proof}
would require countless millions of symbols (Section~\ref{dream}). This might
be true if the proof were to be expressed directly in terms of the axioms of
logic and set theory,\index{set theory} but it is usually not true if we allow
ourselves a hierarchy\index{hierarchy} of definitions and theorems to build
upon, using a notation that allows us to introduce new symbols, definitions,
and theorems in a precisely specified way.

Even in formal logic,\index{formal logic} formal proofs\index{formal proof}
that are considered complete still contain hidden or implicit information.
For example, a ``proof'' is usually defined as a sequence of
wffs,\index{well-formed formula (wff)}\footnote{A {\em wff} or well-formed
formula is a mathematical expression (string of symbols) constructed according
to some precise rules.  A formal mathematical system\index{formal system}
contains (1) the rules for constructing syntactically correct
wffs,\index{syntax rules} (2) a list of starting wffs called
axioms,\index{axiom} and (3) one or more rules prescribing how to derive new
wffs, called theorems\index{theorem}, from the axioms or previously derived
theorems.  An example of such a system is contained in
Metamath's\index{Metamath} set theory database, which defines a formal
system\index{formal system} from which all of standard mathematics can be
derived.  Section~\ref{startf} steps you through a complete example of a formal
system, and you may want to skim it now if you are unfamiliar with the
concept.} each of which is an axiom or follows from a rule applied to previous
wffs in the sequence.  The implicit part of the proof is the algorithm by
which a sequence of symbols is verified to be a valid wff, given the
definition of a wff.  The algorithm in this case is rather simple, but for a
computer to verify the proof,\index{automated proof verification} it must have
the algorithm built into its verification program.\footnote{It is possible, of
course, to specify wff construction syntax outside of the program itself
with a suitable input language (the Metamath language being an example), but
some proof-verification or theorem-proving programs lack the ability to extend
wff syntax in such a fashion.} If one deals exclusively with axioms and
elementary wffs, it is straightforward to implement such an algorithm.  But as
more and more definitions are added to the theory in order to make the
expression of wffs more compact, the algorithm becomes more and more
complicated.  A computer program that implements the algorithm becomes larger
and harder to understand as each definition is introduced, and thus more prone
to bugs.\index{computer program bugs}  The larger the program, the
more suspicious the mathematician may be about
the validity of its algorithms.  This is especially true because
computer programs are inherently hard to follow to begin with, and few people
enjoy verifying them manually in detail.

Metamath\index{Metamath} takes a different approach.  Metamath's ``knowledge''
is limited to the ability to substitute variables for expressions, subject to
some simple constraints.  Once the basic algorithm of Metamath is assumed to
be debugged, and perhaps independently confirmed, it
can be trusted once and for all.  The information that Metamath needs to
``understand'' mathematics is contained entirely in the body of knowledge
presented to Metamath.  Any errors in reasoning can only be errors in the
axioms or definitions contained in this body of knowledge.  As a
``constructive'' language\index{constructive language} Metamath has no
conditional branches or loops like the ones that make computer programs hard
to decipher; instead, the language can only build new sequences of symbols
from earlier sequences  of symbols.

The simplicity of the rules that underlie Metamath not only makes Metamath
easy to learn but also gives Metamath a great deal of flexibility. For
example, Metamath is not limited to describing standard first-order
logic\index{first-order logic}; higher-order logics\index{higher-order logic}
and fragments of logic\index{weak logic} can be described just as easily.
Metamath gives you the freedom to define whatever wff notation you prefer; it
has no built-in conception of the syntax of a wff.\index{well-formed formula
(wff)}  With suitable axioms and definitions, Metamath can even describe and
prove things about itself.\index{Metamath!self-description}  (John
Harrison\index{Harrison, John} discusses the ``reflection''
principle\index{reflection principle} involved in self-descriptive systems in
\cite{Harrison}.)

The flexibility of Metamath requires that its proofs specify a lot of detail,
much more than in an ordinary ``formal'' proof.\index{formal proof}  For
example, in an ordinary formal proof, a single step consists of displaying the
wff that constitutes that step.  In order for a computer program to verify
that the step is acceptable, it first must verify that the symbol sequence
being displayed is an acceptable wff.\index{automated proof verification} Most
proof verifiers have at least basic wff syntax built into their programs.
Metamath has no hard-wired knowledge of what constitutes a wff built into it;
instead every wff must be explicitly constructed based on rules defining wffs
that are present in a database.  Thus a single step in an ordinary formal
proof may be correspond to many steps in a Metamath proof. Despite the larger
number of steps, though, this does not mean that a Metamath proof must be
significantly larger than an ordinary formal proof. The reason is that since
we have constructed the wff from scratch, we know what the wff is, so there is
no reason to display it.  We only need to refer to a sequence of statements
that construct it.  In a sense, the display of the wff in an ordinary formal
proof is an implicit proof of its own validity as a wff; Metamath just makes
the proof explicit. (Section~\ref{proof} describes Metamath's proof notation.)

\section{Computers and Mathematicians}

\begin{quote}
  {\em The computer is important, but not to mathematics.}
    \flushright\sc  Paul Halmos\footnote{As quoted in \cite{Albers}, p.~121.}\\
\end{quote}\index{Halmos, Paul}

Pure mathematicians have traditionally been indifferent to computers, even to
the point of disdain.\index{computers and pure mathematics}  Computer science
itself is sometimes considered to fall in the mundane realm of ``applied''
mathematics, perhaps essential for the real world but intellectually unexciting
to those who seek the deepest truths in mathematics.  Perhaps a reason for this
attitude towards computers is that there is little or no computer software that
meets their needs, and there may be a general feeling that such software could
not even exist.  On the one hand, there are the practical computer algebra
systems, which can perform amazing symbolic manipulations in algebra and
calculus,\index{computer algebra system} yet can't prove the simplest
existence theorem, if the idea of a proof is present at all.  On the other
hand, there are specialized automated theorem provers that technically speaking
may generate correct proofs.\index{automated theorem proving}  But sometimes
their specialized input notation may be cryptic and their output perceived to
be long, inelegant, incomprehensible proofs.    The output
may be viewed with suspicion, since the program that generates it tends to be
very large, and its size increases the potential for bugs\index{computer
program bugs}.  Such a proof may be considered trustworthy only if
independently verified and ``understood'' by a human, but no one wants to
waste their time on such a boring, unrewarding chore.



\needspace{4\baselineskip}
\subsection{Trusting the Computer}

\begin{quote}
  {\em \ldots I continue to find the quasi-empirical interpretation of
  computer proofs to be the more plausible.\ldots Since not
  everything that claims to be a computer proof can be
  accepted as valid, what are the mathematical criteria for acceptable
  computer proofs?}
    \flushright\sc  Thomas Tymoczko\footnote{\cite{Tymoczko}, p.~245.}\\
\end{quote}\index{Tymoczko, Thomas}

In some cases, computers have been essential tools for proving famous
theorems.  But if a proof is so long and obscure that it can be verified in a
practical way only with a computer, it is vaguely felt to be suspicious.  For
example, proving the famous four-color theorem\index{four-color
theorem}\index{proof length} (``a map needs no more than four colors to
prevent any two adjacent countries from having the same color'') can presently
only be done with the aid of a very complex computer program which originally
required 1200 hours of computer time. There has been considerable debate about
whether such a proof can be trusted and whether such a proof is ``real''
mathematics \cite{Swart}\index{Swart, E. R.}.\index{trusting computers}

However, under normal circumstances even a skeptical mathematician would have a
great deal of confidence in the result of multiplying two numbers on a pocket
calculator, even though the precise details of what goes on are hidden from its
user.  Even the verification on a supercomputer that a huge number is prime is
trusted, especially if there is independent verification; no one bothers to
debate the philosophical significance of its ``proof,'' even though the actual
proof would be so large that it would be completely impractical to ever write
it down on paper.  It seems that if the algorithm used by the computer is
simple enough to be readily understood, then the computer can be trusted.

Metamath\index{Metamath} adopts this philosophy.  The simplicity of its
language makes it easy to learn, and because of its simplicity one can have
essentially absolute confidence that a proof is correct. All axioms, rules, and
definitions are available for inspection at any time because they are defined
by the user; there are no hidden or built-in rules that may be prone to subtle
bugs\index{computer program bugs}.  The basic algorithm at the heart of
Metamath is simple and fixed, and it can be assumed to be bug-free and robust
with a degree of confidence approaching certainty.
Independently written implementations of the Metamath verifier
can reduce any residual doubt on the part of a skeptic even further;
there are now over a dozen such implementations, written by many people.

\subsection{Trusting the Mathematician}\label{trust}

\begin{quote}
  {\em There is no Algebraist nor Mathematician so expert in his science, as
  to place entire confidence in any truth immediately upon his discovery of it,
  or regard it as any thing, but a mere probability.  Every time he runs over
  his proofs, his confidence encreases; but still more by the approbation of
  his friends; and is rais'd to its utmost perfection by the universal assent
  and applauses of the learned world.}
  \flushright\sc David Hume\footnote{{\em A Treatise of Human Nature}, as
  quoted in \cite{deMillo}, p.~267.}\\
\end{quote}\index{Hume, David}

\begin{quote}
  {\em Stanislaw Ulam estimates that mathematicians publish 200,000 theorems
  every year.  A number of these are subsequently contradicted or otherwise
  disallowed, others are thrown into doubt, and most are ignored.}
  \flushright\sc Richard de Millo et. al.\footnote{\cite{deMillo}, p.~269.}\\
\end{quote}\index{Ulam, Stanislaw}

Whether or not the computer can be trusted, humans  of course will occasionally
err. Only the most memorable proofs get independently verified, and of these
only a handful of truly great ones achieve the status of being ``known''
mathematical truths that are used without giving a second thought to their
correctness.

There are many famous examples of incorrect theorems and proofs in
mathematical literature.\index{errors in proofs}

\begin{itemize}
\item There have been thousands of purported proofs of Fermat's Last
Theorem\index{Fermat's Last Theorem} (``no integer solutions exist to $x^n +
y^n = z^n$ for $n > 2$''), by amateurs, cranks, and well-regarded
mathematicians \cite[p.~5]{Stark}\index{Stark, Harold M}.  Fermat wrote a note
in his copy of Bachet's {\em Diophantus} that he found ``a truly marvelous
proof of this theorem but this margin is too narrow to contain it''
\cite[p.~507]{Kramer}.  A recent, much publicized proof by Yoichi
Miyaoka\index{Miyaoka, Yoichi} was shown to be incorrect ({\em Science News},
April 9, 1988, p.~230).  The theorem was finally proved by Andrew
Wiles\index{Wiles, Andrew} ({\em Science News}, July 3, 1993, p.~5), but it
initially had some gaps and took over a year after its announcement to be
checked thoroughly by experts.  On Oct. 25, 1994, Wiles announced that the last
gap found in his proof had been filled in.
  \item In 1882, M. Pasch discovered that an axiom was omitted from Euclid's
formulation of geometry\index{Euclidean geometry}; without it, the proofs of
important theorems of Euclid are not valid.  Pasch's axiom\index{Pasch's
axiom} states that a line that intersects one side of a triangle must also
intersect another side, provided that it does not touch any of the triangle's
vertices.  The omission of Pasch's axiom went unnoticed for 2000
years \cite[p.~160]{Davis}, in spite of (one presumes) the thousands of
students, instructors, and mathematicians who studied Euclid.
  \item The first published proof of the famous Schr\"{o}der--Bernstein
theorem\index{Schr\"{o}der--Bernstein theorem} in set theory was incorrect
\cite[p.~148]{Enderton}\index{Enderton, Herbert B.}.  This theorem states
that if there exists a 1-to-1 function\footnote{A {\em set}\index{set} is any
collection of objects. A {\em function}\index{function} or {\em
mapping}\index{mapping} is a rule that assigns to each element of one set
(called the function's {\em domain}\index{domain}) an element from another
set.} from set $A$ into set $B$ and vice-versa, then sets $A$ and $B$ have
a 1-to-1 correspondence.  Although it sounds simple and obvious,
the standard proof is quite long and complex.
  \item In the early 1900's, Hilbert\index{Hilbert, David} published a
purported proof of the continuum hypothesis\index{continuum hypothesis}, which
was eventually established as unprovable by Cohen\index{Cohen, Paul} in 1963
\cite[p.~166]{Enderton}.  The continuum hypothesis states that no
infinity\index{infinity} (``transfinite cardinal number'')\index{cardinal,
transfinite} exists whose size (or ``cardinality''\index{cardinality}) is
between the size of the set of integers and the size of the set of real
numbers.  This hypothesis originated with German mathematician Georg
Cantor\index{Cantor, Georg} in the late 1800's, and his inability to prove it
is said to have contributed to mental illness that afflicted him in his later
years.
  \item An incorrect proof of the four-color theorem\index{four-color theorem}
was published by Kempe\index{Kempe, A. B.} in 1879
\cite[p.~582]{Courant}\index{Courant, Richard}; it stood for 11 years before
its flaw was discovered.  This theorem states that any map can be colored
using only four colors, so that no two adjacent countries have the same
color.  In 1976 the theorem was finally proved by the famous computer-assisted
proof of Haken, Appel, and Koch \cite{Swart}\index{Appel, K.}\index{Haken,
W.}\index{Koch, K.}.  Or at least it seems that way.  Mathematician
H.~S.~M.~Coxeter\index{Coxeter, H. S. M.} has doubts \cite[p.~58]{Davis}:  ``I
have a feeling that that is an untidy kind of use of the computers, and the more
you correspond with Haken and Appel, the more shaky you seem to be.''
  \item Many false ``proofs'' of the Poincar\'{e}
conjecture\index{Poincar\'{e} conjecture} have been proposed over the years.
This conjecture states that any object that mathematically behaves like a
three-dimensional sphere is a three-dimensional sphere topologically,
regardless of how it is distorted.  In March 1986, mathematicians Colin
Rourke\index{Rourke, Colin} and Eduardo R\^{e}go\index{R\^{e}go, Eduardo}
caused  a stir in the mathematical community by announcing that they had found
a proof; in November of that year the proof was found to be false \cite[p.
218]{PetersonI}.  It was finally proved in 2003 by Grigory Perelman
\label{poincare}\index{Szpiro, George}\index{Perelman, Grigory}\cite{Szpiro}.
 \end{itemize}

Many counterexamples to ``theorems'' in recent mathematical
literature related to Clifford algebras\index{Clifford algebras}
 have been found by Pertti
Lounesto (who passed away in 2002).\index{Lounesto, Pertti}
See the web page \url{http://mathforum.org/library/view/4933.html}.
% http://users.tkk.fi/~ppuska/mirror/Lounesto/counterexamples.htm

One of the purposes of Metamath\index{Metamath} is to allow proofs to be
expressed with absolute precision.  Developing a proof in the Metamath
language can be challenging, because Metamath will not permit even the
tiniest mistake.\index{errors in proofs}  But once the proof is created, its
correctness can be trusted immediately, without having to depend on the
process of peer review for confirmation.

\section{The Use of Computers in Mathematics}

\subsection{Computer Algebra Systems}

For the most part, you will find that Metamath\index{Metamath} is not a
practical tool for manipulating numbers.  (Even proving that $2 + 2 = 4$, if
you start with set theory, can be quite complex!)  Several commercial
mathematics packages are quite good at arithmetic, algebra, and calculus, and
as practical tools they are invaluable.\index{computer algebra system} But
they have no notion of proof, and cannot understand statements starting with
``there exists such and such...''.

Software packages such as Mathematica \cite{Wolfram}\index{Mathematica} do not
concern themselves with proofs but instead work directly with known results.
These packages primarily emphasize heuristic rules such as the substitution of
equals for equals to achieve simpler expressions or expressions in a different
form.  Starting with a rich collection of built-in rules and algorithms, users
can add to the collection by means of a powerful programming language.
However, results such as, say, the existence of a certain abstract object
without displaying the actual object cannot be expressed (directly) in their
languages.  The idea of a proof from a small set of axioms is absent.  Instead
this software simply assumes that each fact or rule you add to the built-in
collection of algorithms is valid.  One way to view the software is as a large
collection of axioms from which the software, with certain goals, attempts to
derive new theorems, for example equating a complex expression with a simpler
equivalent. But the terms ``theorem''\index{theorem} and
``proof,''\index{proof} for example, are not even mentioned in the index of
the user's manual for Mathematica.\index{Mathematica and proofs}  What is also
unsatisfactory from a philosophical point of view is that there is no way to
ensure the validity of the results other than by trusting the writer of each
application module or tediously checking each module by hand, similar to
checking a computer program for bugs.\index{computer program
bugs}\footnote{Two examples illustrate why the knowledge database of computer
algebra systems should sometimes be regarded with a certain caution.  If you
ask Mathematica (version 3.0) to \texttt{Solve[x\^{ }n + y\^{ }n == z\^{ }n , n]}
it will respond with \texttt{\{\{n-\char`\>-2\}, \{n-\char`\>-1\},
\{n-\char`\>1\}, \{n-\char`\>2\}\}}. In other words, Mathematica seems to
``know'' that Fermat's Last Theorem\index{Fermat's Last Theorem} is true!  (At
the time this version of Mathematica was released this fact was unknown.)  If
you ask Maple\index{Maple} to \texttt{solve(x\^{ }2 = 2\^{ }x)} then
\texttt{simplify(\{"\})}, it returns the solution set \texttt{\{2, 4\}}, apparently
unaware that $-0.7666647$\ldots is also a solution.} While of course extremely
valuable in applied mathematics, computer algebra systems tend to be of little
interest to the theoretical mathematician except as aids for exploring certain
specific problems.

Because of possible bugs, trusting the output of a computer algebra system for
use as theorems in a proof-verifier would defeat the latter's goal of rigor.
On the other hand, a fact such that a certain relatively large number is
prime, while easy for a computer algebra system to derive, might have a long,
tedious proof that could overwhelm a proof-verifier. One approach for linking
computer algebra systems to a proof-verifier while retaining the advantages of
both is to add a hypothesis to each such theorem indicating its source.  For
example, a constant {\sc maple} could indicate the theorem came from the Maple
package, and would mean ``assuming Maple is consistent, then\ldots''  This and
many other topics concerning the formalization of mathematics are discussed in
John Harrison's\index{Harrison, John} very interesting
PhD thesis~\cite{Harrison-thesis}.

\subsection{Automated Theorem Provers}\label{theoremprovers}

A mathematical theory is ``decidable''\index{decidable theory} if a mechanical
method or algorithm exists that is guaranteed to determine whether or not a
particular formula is a theorem.  Among the few theories that are decidable is
elementary geometry,\index{Euclidean geometry} as was shown by a classic
result of logician Alfred Tarski\index{Tarski, Alfred} in 1948
\cite{Tarski}.\footnote{Tarski's result actually applies to a subset of the
geometry discussed in elementary textbooks.  This subset includes most of what
would be considered elementary geometry but it is not powerful enough to
express, among other things, the notions of the circumference and area of a
circle.  Extending the theory in a way that includes notions such as these
makes the theory undecidable, as was also shown by Tarski.  Tarski's algorithm
is far too inefficient to implement practically on a computer.  A practical
algorithm for proving a smaller subset of geometry theorems (those not
involving concepts of ``order'' or ``continuity'') was discovered by Chinese
mathematician Wu Wen-ts\"{u}n in 1977 \cite{Chou}\index{Chou,
Shang-Ching}.}\index{Wen-ts{\"{u}}n, Wu}  But most theories, including
elementary arithmetic, are undecidable.  This fact contributes to keeping
mathematics alive and well, since many mathematicians believe
that they will never be
replaced by computers (if they believe Roger Penrose's argument that a
computer can never replace the brain \cite{Penrose}\index{Penrose, Roger}).
In fact,  elementary geometry is often considered a ``dead'' field
for the simple reason that it is decidable.

On the other hand, the undecidability of a theory does not mean that one cannot
use a computer to search for proofs, providing one is willing to give up if a
proof is not found after a reasonable amount of time.  The field of automated
theorem proving\index{automated theorem proving} specializes in pursuing such
computer searches.  Among the more successful results to date are those based
on an algorithm known as Robinson's resolution principle
\cite{Robinson}\index{Robinson's resolution principle}.

Automated theorem provers can be excellent tools for those willing to learn
how to use them.  But they are not widely used in mainstream pure
mathematics, even though they could probably be useful in many areas.  There
are several reasons for this.  Probably most important, the main goal in pure
mathematics is to arrive at results that are considered to be deep or
important; proving them is essential but secondary.  Usually, an automated
theorem prover cannot assist in this main goal, and by the time the main goal
is achieved, the mathematician may have already figured out the proof as a
by-product.  There is also a notational problem.  Mathematicians are used to
using very compact syntax where one or two symbols (heavily dependent on
context) can represent very complex concepts; this is part of the
hierarchy\index{hierarchy} they have built up to tackle difficult problems.  A
theorem prover on the other hand might require that a theorem be expressed in
``first-order logic,''\index{first-order logic} which is the logic on which
most of mathematics is ultimately based but which is not ordinarily used
directly because expressions can become very long.  Some automated theorem
provers are experimental programs, limited in their use to very specialized
areas, and the goal of many is simply research into the nature of automated
theorem proving itself.  Finally, much research remains to be done to enable
them to prove very deep theorems.  One significant result was a
computer proof by Larry Wos\index{Wos, Larry} and colleagues that every Robbins
algebra\index{Robbins algebra} is a Boolean  algebra\index{Boolean algebra}
({\em New York Times}, Dec. 10, 1996).\footnote{In 1933, E.~V.\
Huntington\index{Huntington, E. V.}
presented the following axiom system for
Boolean algebra with a unary operation $n$ and a binary operation $+$:
\begin{center}
    $x + y = y + x$ \\
    $(x + y) + z = x + (y + z)$ \\
    $n(n(x) + y) + n(n(x) + n(y)) = x$
\end{center}
Herbert Robbins\index{Robbins, Herbert}, a student of Huntington, conjectured
that the last equation can be replaced with a simpler one:
\begin{center}
    $n(n(x + y) + n(x + n(y))) = x$
\end{center}
Robbins and Huntington could not find a proof.  The problem was
later studied unsuccessfully by Tarski\index{Tarski, Alfred} and his
students, and it remained an unsolved problem until a
computer found the proof in 1996.  For more information on
the Robbins algebra problem see \cite{Wos}.}

How does Metamath\index{Metamath} relate to automated theorem provers?  A
theorem prover is primarily concerned with one theorem at a time (perhaps
tapping into a small database of known theorems) whereas Metamath is more like
a theorem archiving system, storing both the theorem and its proof in a
database for access and verification.  Metamath is one answer to ``what do you
do with the output of a theorem prover?''  and could be viewed as the
next step in the process.  Automated theorem provers could be useful tools for
helping develop its database.
Note that very long, automatically
generated proofs can make your database fat and ugly and cause Metamath's proof
verification to take a long time to run.  Unless you have a particularly good
program that generates very concise proofs, it might be best to consider the
use of automatically generated proofs as a quick-and-dirty approach, to be
manually rewritten at some later date.

The program {\sc otter}\index{otter@{\sc otter}}\footnote{\url{http://www.cs.unm.edu/\~mccune/otter/}.}, later succeeded by
prover9\index{prover9}\footnote{\url{https://www.cs.unm.edu/~mccune/mace4/}.},
have been historically influential.
The E prover\index{E prover}\footnote{\url{https://github.com/eprover/eprover}.}
is a maintained automated theorem prover
for full first-order logic with equality.
There are many other automated theorem provers as well.

If you want to combine automated theorem provers with Metamath
consider investigating
the book {\em Automated Reasoning:  Introduction and Applications}
\cite{Wos}\index{Wos, Larry}.  This book discusses
how to use {\sc otter} in a way that can
not only able to generate
relatively efficient proofs, it can even be instructed to search for
shorter proofs.  The effective use of {\sc otter} (and similar tools)
does require a certain
amount of experience, skill, and patience.  The axiom system used in the
\texttt{set.mm}\index{set theory database (\texttt{set.mm})} set theory
database can be expressed to {\sc otter} using a method described in
\cite{Megill}.\index{Megill, Norman}\footnote{To use those axioms with
{\sc otter}, they must be restated in a way that eliminates the need for
``dummy variables.''\index{dummy variable!eliminating} See the Comment
on p.~\pageref{nodd}.} When successful, this method tends to generate
short and clever proofs, but my experiments with it indicate that the
method will find proofs within a reasonable time only for relatively
easy theorems.  It is still fun to experiment with.

Reference \cite{Bledsoe}\index{Bledsoe, W. W.} surveys a number of approaches
people have explored in the field of automated theorem proving\index{automated
theorem proving}.

\subsection{Interactive Theorem Provers}\label{interactivetheoremprovers}

Finding proofs completely automatically is difficult, so there
are some interactive theorem provers that allow a human to guide the
computer to find a proof.
Examples include
HOL Light\index{HOL light}%
\footnote{\url{https://www.cl.cam.ac.uk/~jrh13/hol-light/}.},
Isabelle\index{Isabelle}%
\footnote{\url{http://www.cl.cam.ac.uk/Research/HVG/Isabelle}.},
{\sc hol}\index{hol@{\sc hol}}%
\footnote{\url{https://hol-theorem-prover.org/}.},
and
Coq\index{Coq}\footnote{\url{https://coq.inria.fr/}.}.

A major difference between most of these tools and Metamath is that the
``proofs'' are actually programs that guide the program to find a proof,
and not the proof itself.
For example, an Isabelle/HOL proof might apply a step
\texttt{apply (blast dest: rearrange reduction)}. The \texttt{blast}
instruction applies
an automatic tableux prover and returns if it found a sequence of proof
steps that work... but the sequence is not considered part of the proof.

A good overview of
higher-level proof verification languages (such as {\sc lcf}\index{lcf@{\sc
lcf}} and {\sc hol}\index{hol@{\sc hol}})
is given in \cite{Harrison}.  All of these languages are fundamentally
different from Metamath in that much of the mathematical foundational
knowledge is embedded in the underlying proof-verification program, rather
than placed directly in the database that is being verified.
These can have a steep learning curve for those without a mathematical
background.  For example, one usually must have a fair understanding of
mathematical logic in order to follow their proofs.

\subsection{Proof Verifiers}\label{proofverifiers}

A proof verifier is a program that doesn't generate proofs but instead
verifies proofs that you give it.  Many proof verifiers have limited built-in
automated proof capabilities, such as figuring out simple logical inferences
(while still being guided by a person who provides the overall proof).  Metamath
has no built-in automated proof capability other than the limited
capability of its Proof Assistant.

Proof-verification languages are not used as frequently as they might be.
Pure mathematicians are more concerned with producing new results, and such
detail and rigor would interfere with that goal.  The use of computers in pure
mathematics is primarily focused on automated theorem provers (not verifiers),
again with the ultimate goal of aiding the creation of new mathematics.
Automated theorem provers are usually concerned with attacking one theorem at
time rather than making a large, organized database easily available to the
user.  Metamath is one way to help close this gap.

By itself Metamath is a mostly a proof verifier.
This does not mean that other approaches can't be used; the difference
is that in Metamath, the results of various provers must be recorded
step-by-step so that they can be verified.

Another proof-verification language is Mizar,\index{Mizar} which can display
its proofs in the informal language that mathematicians are accustomed to.
Information on the Mizar language is available at \url{http://mizar.org}.

For the working mathematician, Mizar is an excellent tool for rigorously
documenting proofs. Mizar typesets its proofs in the informal English used by
mathematicians (and, while fine for them, are just as inscrutable by
laypersons!). A price paid for Mizar is a relatively steep learning curve of a
couple of weeks.  Several mathematicians are actively formalizing different
areas of mathematics using Mizar and publishing the proofs in a dedicated
journal. Unfortunately the task of formalizing mathematics is still looked
down upon to a certain extent since it doesn't involve the creation of ``new''
mathematics.

The closest system to Metamath is
the {\em Ghilbert}\index{Ghilbert} proof language (\url{http://ghilbert.org})
system developed by
Raph Levien\index{Levien, Raph}.
Ghilbert is a formal proof checker heavily inspired by Metamath.
Ghilbert statements are s-expressions (a la Lisp), which is easy
for computers to parse but many people find them hard to read.
There are a number of differences in their specific constructs, but
there is at least one tool to translate some Metamath materials into Ghilbert.
As of 2019 the Ghilbert community is smaller and less active than the
Metamath community.
That said, the Metamath and Ghilbert communities overlap, and fruitful
conversations between them have occurred many times over the years.

\subsection{Creating a Database of Formalized Mathematics}\label{mathdatabase}

Besides Metamath, there are several other ongoing projects with the goal of
formalizing mathematics into computer-verifiable databases.
Understanding some history will help.

The {\sc qed}\index{qed project@{\sc qed} project}%
\footnote{\url{http://www-unix.mcs.anl.gov/qed}.}
project arose in 1993 and its goals were outlined in the
{\sc qed} manifesto.
The {\sc qed} manifesto was
a proposal for a computer-based database of all mathematical knowledge,
strictly formalized and with all proofs having been checked automatically.
The project had a conference in 1994 and another in 1995;
there was also a ``twenty years of the {\sc qed} manifesto'' workshop
in 2014.
Its ideals are regularly reraised.

In a 2007 paper, Freek Wiedijk identified two reasons
for the failure of the {\sc qed} project as originally envisioned:%
\cite{Wiedijk-revisited}\index{Wiedijk, Freek}

\begin{itemize}
\item Very few people are working on formalization of mathematics. There is no compelling application for fully mechanized mathematics.
\item Formalized mathematics does not yet resemble traditional mathematics. This is partly due to the complexity of mathematical notation, and partly to the limitations of existing theorem provers and proof assistants.
\end{itemize}

But this did not end the dream of
formalizing mathematics into computer-verifiable databases.
The problems that led to the {\sc qed} manifesto are still with us,
even though the challenges were harder than originally considered.
What has happened instead is that various independent projects have
worked towards formalizing mathematics into computer-verifiable databases,
each simultaneously competing and cooperating with each other.

A concrete way to see this is
Freek Wiedijk's ``Formalizing 100 Theorems'' list%
\footnote{\url{http://www.cs.ru.nl/\%7Efreek/100/}.}
which shows the progress different systems have made on a challenge list
of 100 mathematical theorems.%
\footnote{ This is not the only list of ``interesting'' theorems.
Another interesting list was posted by Oliver Knill's list
\cite{Knill}\index{Knill, Oliver}.}
The top systems as of February 2019
(in order of the number of challenges completed) are
HOL Light, Isabelle, Metamath, Coq, and Mizar.

The Metamath 100%
\footnote{\url{http://us.metamath.org/mm\_100.html}}
page (maintained by David A. Wheeler\index{Wheeler, David A.})
shows the progress of Metamath (specifically its \texttt{set.mm} database)
against this challenge list maintained by Freek Wiedijk.
The Metamath \texttt{set.mm} database
has made a lot of progress over the years,
in part because working to prove those challenge theorems required
defining various terms and proving their properties as a prerequisite.
Here are just a few of the many statements that have been
formally proven with Metamath:

% The entries of this cause the narrow display to break poorly,
% since the short amount of text means LaTeX doesn't get a lot to work with
% and the itemize format gives it even *less* margin than usual.
% No one will mind if we make just this list flushleft, since this list
% will be internally consistent.
\begin{flushleft}
\begin{itemize}
\item 1. The Irrationality of the Square Root of 2
  (\texttt{sqr2irr}, by Norman Megill, 2001-08-20)
\item 2. The Fundamental Theorem of Algebra
  (\texttt{fta}, by Mario Carneiro, 2014-09-15)
\item 22. The Non-Denumerability of the Continuum
  (\texttt{ruc}, by Norman Megill, 2004-08-13)
\item 54. The Konigsberg Bridge Problem
  (\texttt{konigsberg}, by Mario Carneiro, 2015-04-16)
\item 83. The Friendship Theorem
  (\texttt{friendship}, by Alexander W. van der Vekens, 2018-10-09)
\end{itemize}
\end{flushleft}

We thank all of those who have developed at least one of the Metamath 100
proofs, and we particularly thank
Mario Carneiro\index{Carneiro, Mario}
who has contributed the most Metamath 100 proofs as of 2019.
The Metamath 100 page shows the list of all people who have contributed a
proof, and links to graphs and charts showing progress over time.
We encourage others to work on proving theorems not yet proven in Metamath,
since doing so improves the work as a whole.

Each of the math formalization systems (including Metamath)
has different strengths and weaknesses, depending on what you value.
Key aspects that differentiate Metamath from the other top systems are:

\begin{itemize}
\item Metamath is not tied to any particular set of axioms.
\item Metamath can show every step of every proof, no exceptions.
  Most other provers only assert that a proof can be found, and do not
  show every step. This also makes verification fast, because
  the system does not need to rediscover proof details.
\item The Metamath verifier has been re-implemented in many different
  programming languages, so verification can be done by multiple
  implementations.  In particular, the
  \texttt{set.mm}\index{set theory database (\texttt{set.mm})}%
  \index{Metamath Proof Explorer} database is verified by
  four different verifiers
  written in four different languages by four different authors.
  This greatly reduces the risk of accepting an invalid
  proof due to an error in the verifier.
\item Proofs stay proven.  In some systems, changes to the system's
  syntax or how a tactic works causes proofs to fail in later versions,
  causing older work to become essentially lost.
  Metamath's language is
  extremely small and fixed, so once a proof is added to a database,
  the database can be rechecked with later versions of the Metamath program
  and with other verifiers of Metamath databases.
  If an axiom or key definition needs to be changed, it is easy to
  manipulate the database as a whole to handle the change
  without touching the underlying verifier.
  Since re-verification of an entire database takes seconds, there
  is never a reason to delay complete verification.
  This aspect is especially compelling if your
  goal is to have a long-term database of proofs.
\item Licensing is generous.  The main Metamath databases are released to
  the public domain, and the main Metamath program is open source software
  under a standard, widely-used license.
\item Substitutions are easy to understand, even by those who are not
  professional mathematicians.
\end{itemize}

Of course, other systems may have advantages over Metamath
that are more compelling, depending on what you value.
In any case, we hope this helps you understand Metamath
within a wider context.

\subsection{In Summary}\label{computers-summary}

To summarize our discussions of computers and mathematics, computer algebra
systems can be viewed as theorem generators focusing on a narrow realm of
mathematics (numbers and their properties), automated theorem provers as proof
generators for specific theorems in a much broader realm covered by a built-in
formal system such as first-order logic, interactive theorem
provers require human guidance, proof verifiers verify proofs but
historically they have been
restricted to first-order logic.
Metamath, in contrast,
is a proof verifier and documenter whose realm is essentially unlimited.

\section{Mathematics and Metamath}

\subsection{Standard Mathematics}

There are a number of ways that Metamath\index{Metamath} can be used with
standard mathematics.  The most satisfying way philosophically is to start at
the very beginning, and develop the desired mathematics from the axioms of
logic and set theory.\index{set theory}  This is the approach taken in the
\texttt{set.mm}\index{set theory database (\texttt{set.mm})}%
\index{Metamath Proof Explorer}
database (also known as the Metamath Proof Explorer).
Among other things, this database builds up to the
axioms of real and complex numbers\index{analysis}\index{real and complex
numbers} (see Section~\ref{real}), and a standard development of analysis, for
example, could start at that point, using it as a basis.   Besides this
philosophical advantage, there are practical advantages to having all of the
tools of set theory available in the supporting infrastructure.

On the other hand, you may wish to start with the standard axioms of a
mathematical theory without going through the set theoretical proofs of those
axioms.  You will need mathematical logic to make inferences, but if you wish
you can simply introduce theorems\index{theorem} of logic as
``axioms''\index{axiom} wherever you need them, with the implicit assumption
that in principle they can be proved, if they are obvious to you.  If you
choose this approach, you will probably want to review the notation used in
\texttt{set.mm}\index{set theory database (\texttt{set.mm})} so that your own
notation will be consistent with it.

\subsection{Other Formal Systems}
\index{formal system}

Unlike some programs, Metamath\index{Metamath} is not limited to any specific
area of mathematics, nor committed to any particular mathematical philosophy
such as classical logic versus intuitionism, nor limited, say, to expressions
in first-order logic.  Although the database \texttt{set.mm}
describes standard logic and set theory, Meta\-math
is actually a general-purpose language for describing a wide variety of formal
systems.\index{formal system}  Non-standard systems such as modal
logic,\index{modal logic} intuitionist logic\index{intuitionism}, higher-order
logic\index{higher-order logic}, quantum logic\index{quantum logic}, and
category theory\index{category theory} can all be described with the Metamath
language.  You define the symbols you prefer and tell Metamath the axioms and
rules you want to start from, and Metamath will verify any inferences you make
from those axioms and rules.  A simple example of a non-standard formal system
is Hofstadter's\index{Hofstadter, Douglas R.} MIU system,\index{MIU-system}
whose Metamath description is presented in Appendix~\ref{MIU}.

This is not hypothetical.
The largest Metamath database is
\texttt{set.mm}\index{set theory database (\texttt{set.mm}}%
\index{Metamath Proof Explorer}), aka the Metamath Proof Explorer,
which uses the most common axioms for mathematical foundations
(specifically classical logic combined with Zermelo--Fraenkel
set theory\index{Zermelo--Fraenkel set theory} with the Axiom of Choice).
But other Metamath databases are available:

\begin{itemize}
\item The database
  \texttt{iset.mm}\index{intuitionistic logic database (\texttt{iset.mm})},
  aka the
  Intuitionistic Logic Explorer\index{Intuitionistic Logic Explorer},
  uses intuitionistic logic (a constructivist point of view)
  instead of classical logic.
\item The database
  \texttt{nf.mm}\index{New Foundations database (\texttt{nf.mm})},
  aka the
  New Foundations Explorer\index{New Foundations Explorer},
  constructs mathematics from scratch,
  starting from Quine's New Foundations (NF) set theory axioms.
\item The database
  \texttt{hol.mm}\index{Higher-order Logic database (\texttt{hol.mm})},
  aka the
  Higher-Order Logic (HOL) Explorer\index{Higher-Order Logic (HOL) Explorer},
  starts with HOL (also called simple type theory) and derives
  equivalents to ZFC axioms, connecting the two approaches.
\end{itemize}

Since the days of David Hilbert,\index{Hilbert, David} mathematicians have
been concerned with the fact that the metalanguage\index{metalanguage} used to
describe mathematics may be stronger than the mathematics being described.
Metamath\index{Metamath}'s underlying finitary\index{finitary proof},
constructive nature provides a good philosophical basis for studying even the
weakest logics.\index{weak logic}

The usual treatment of many non-standard formal systems\index{formal
system} uses model theory\index{model theory} or proof theory\index{proof
theory} to describe these systems; these theories, in turn, are based on
standard set theory.  In other words, a non-standard formal system is defined
as a set with certain properties, and standard set theory is used to derive
additional properties of this set.  The standard set theory database provided
with Metamath can be used for this purpose, and when used this way
the development of a special
axiom system for the non-standard formal system becomes unnecessary.  The
model- or proof-theoretic approach often allows you to prove much deeper
results with less effort.

Metamath supports both approaches.  You can define the non-standard
formal system directly, or define the non-standard formal system as
a set with certain properties, whichever you find most helpful.

%\section{Additional Remarks}

\subsection{Metamath and Its Philosophy}

Closely related to Metamath\index{Metamath} is a philosophy or way of looking
at mathematics. This philosophy is related to the formalist
philosophy\index{formalism} of Hilbert\index{Hilbert, David} and his followers
\cite[pp.~1203--1208]{Kline}\index{Kline, Morris}
\cite[p.~6]{Behnke}\index{Behnke, H.}. In this philosophy, mathematics is
viewed as nothing more than a set of rules that manipulate symbols, together
with the consequences of those rules.  While the mathematics being described
may be complex, the rules used to describe it (the
``metamathematics''\index{metamathematics}) should be as simple as possible.
In particular, proofs should be restricted to dealing with concrete objects
(the symbols we write on paper rather than the abstract concepts they
represent) in a constructive manner; these are called ``finitary''
proofs\index{finitary proof} \cite[pp.~2--3]{Shoenfield}\index{Shoenfield,
Joseph R.}.

Whether or not you find Metamath interesting or useful will in part depend on
the appeal you find in its philosophy, and this appeal will probably depend on
your particular goals with respect to mathematics.  For example, if you are a
pure mathematician at the forefront of discovering new mathematical knowledge,
you will probably find that the rigid formality of Metamath stifles your
creativity.  On the other hand, we would argue that once this knowledge is
discovered, there are advantages to documenting it in a standard format that
will make it accessible to others.  Sixty years from now, your field may be
dormant, and as Davis and Hersh put it, your ``writings would become less
translatable than those of the Maya'' \cite[p.~37]{Davis}\index{Davis, Phillip
J.}.


\subsection{A History of the Approach Behind Metamath}

Probably the one work that has had the most motivating influence on
Metamath\index{Metamath} is Whitehead and Russell's monumental {\em Principia
Mathematica} \cite{PM}\index{Whitehead, Alfred North}\index{Russell,
Bertrand}\index{principia mathematica@{\em Principia Mathematica}}, whose aim
was to deduce all of mathematics from a small number of primitive ideas, in a
very explicit way that in principle anyone could understand and follow.  While
this work was tremendously influential in its time, from a modern perspective
it suffers from several drawbacks.  Both its notation and its underlying
axioms are now considered dated and are no longer used.  From our point of
view, its development is not really as accessible as we would like to see; for
practical reasons, proofs become more and more sketchy as its mathematics
progresses, and working them out in fine detail requires a degree of
mathematical skill and patience that many people don't have.  There are
numerous small errors, which is understandable given the tedious, technical
nature of the proofs and the lack of a computer to verify the details.
However, even today {\em Principia Mathematica} stands out as the work closest
in spirit to Metamath.  It remains a mind-boggling work, and one can't help
but be amazed at seeing ``$1+1=2$'' finally appear on page 83 of Volume II
(Theorem *110.643).

The origin of the proof notation used by Metamath dates back to the 1950's,
when the logician C.~A.~Meredith expressed his proofs in a compact notation
called ``condensed detachment''\index{condensed detachment}
\cite{Hindley}\index{Hindley, J. Roger} \cite{Kalman}\index{Kalman, J. A.}
\cite{Meredith}\index{Meredith, C. A.} \cite{Peterson}\index{Peterson, Jeremy
George}.  This notation allows proofs to be communicated unambiguously by
merely referencing the axiom\index{axiom}, rule\index{rule}, or
theorem\index{theorem} used at each step, without explicitly indicating the
substitutions\index{substitution!variable}\index{variable substitution} that
have to be made to the variables in that axiom, rule, or theorem.  Ordinarily,
condensed detachment is more or less limited to propositional
calculus\index{propositional calculus}.  The concept has been extended to
first-order logic\index{first-order logic} in \cite{Megill}\index{Megill,
Norman}, making it is easy to write a small computer program to verify proofs
of simple first-order logic theorems.\index{condensed detachment!and
first-order logic}

A key concept behind the notation of condensed detachment is called
``unification,''\index{unification} which is an algorithm for determining what
substitutions\index{substitution!variable}\index{variable substitution} to
variables have to be made to make two expressions match each other.
Unification was first precisely defined by the logician J.~A.~Robinson, who
used it in the development of a powerful
theorem-proving technique called the ``resolution principle''
\cite{Robinson}\index{Robinson's resolution principle}. Metamath does not make
use of the resolution principle, which is intended for systems of first-order
logic.\index{first-order logic}  Metamath's use is not restricted to
first-order logic, and as we have mentioned it does not automatically discover
proofs.  However, unification is a key idea behind Metamath's proof
notation, and Metamath makes use of a very simple version of it
(Section~\ref{unify}).

\subsection{Metamath and First-Order Logic}

First-order logic\index{first-order logic} is the supporting structure
for standard mathematics.  On top of it is set theory, which contains
the axioms from which virtually all of mathematics can be derived---a
remarkable fact.\index{category
theory}\index{cardinal, inaccessible}\label{categoryth}\footnote{An exception seems
to be category theory.  There are several schools of thought on whether
category theory is derivable from set theory.  At a minimum, it appears
that an additional axiom is needed that asserts the existence of an
``inaccessible cardinal'' (a type of infinity so large that standard set
theory can't prove or deny that it exists).
%
%%%% (I took this out that was in previous editions:)
% But it is also argued that not just one but a ``proper class'' of them
% is needed, and the existence of proper classes is impossible in standard
% set theory.  (A proper class is a collection of sets so huge that no set
% can contain it as an element.  Proper classes can lead to
% inconsistencies such as ``Russell's paradox.''  The axioms of standard
% set theory are devised so as to deny the existence of proper classes.)
%
For more information, see
\cite[pp.~328--331]{Herrlich}\index{Herrlich, Horst} and
\cite{Blass}\index{Blass, Andrea}.}

One of the things that makes Metamath\index{Metamath} more practical for
first-order theories is a set of axioms for first-order logic designed
specifically with Metamath's approach in mind.  These are included in
the database \texttt{set.mm}\index{set theory database (\texttt{set.mm})}.
See Chapter~\ref{fol} for a detailed
description; the axioms are shown in Section~\ref{metaaxioms}.  While
logically equivalent to standard axiom systems, our axiom system breaks
up the standard axioms into smaller pieces such that from them, you can
directly derive what in other systems can only be derived as higher-level
``metatheorems.''\index{metatheorem}  In other words, it is more powerful than
the standard axioms from a metalogical point of view.  A rigorous
justification for this system and its ``metalogical
completeness''\index{metalogical completeness} is found in
\cite{Megill}\index{Megill, Norman}.  The system is closely related to a
system developed by Monk\index{Monk, J. Donald} and Tarski\index{Tarski,
Alfred} in 1965 \cite{Monks}.

For example, the formula $\exists x \, x = y $ (given $y$, there exists some
$x$ equal to it) is a theorem of logic,\footnote{Specifically, it is a theorem
of those systems of logic that assume non-empty domains.  It is not a theorem
of more general systems that include the empty domain\index{empty domain}, in
which nothing exists, period!  Such systems are called ``free
logics.''\index{free logic} For a discussion of these systems, see
\cite{Leblanc}\index{Leblanc, Hugues}.  Since our use for logic is as a basis
for set theory, which has a non-empty domain, it is more convenient (and more
traditional) to use a less general system.  An interesting curiosity is that,
using a free logic as a basis for Zermelo--Fraenkel set
theory\index{Zermelo--Fraenkel set theory} (with the redundant Axiom of the
Null Set omitted),\index{Axiom of the Null Set} we cannot even prove the
existence of a single set without assuming the axiom of infinity!\index{Axiom
of Infinity}} whether or not $x$ and $y$ are distinct variables\index{distinct
variables}.  In many systems of logic, we would have to prove two theorems to
arrive at this result.  First we would prove ``$\exists x \, x = x $,'' then
we would separately prove ``$\exists x \, x = y $, where $x$ and $y$ are
distinct variables.''  We would then combine these two special cases ``outside
of the system'' (i.e.\ in our heads) to be able to claim, ``$\exists x \, x =
y $, regardless of whether $x$ and $y$ are distinct.''  In other words, the
combination of the two special cases is a
metatheorem.  In the system of logic
used in Metamath's set theory\index{set theory database (\texttt{set.mm})}
database, the axioms of logic are broken down into small pieces that allow
them to be reassembled in such a way that theorems such as these can be proved
directly.

Breaking down the axioms in this way makes them look peculiar and not very
intuitive at first, but rest assured that they are correct and complete.  Their
correctness is ensured because they are theorem schemes of standard first-order
logic (which you can easily verify if you are a logician).  Their completeness
follows from the fact that we explicitly derive the standard axioms of
first-order logic as theorems.  Deriving the standard axioms is somewhat
tricky, but once we're there, we have at our disposal a system that is less
awkward to work with in formal proofs\index{formal proof}.  In technical terms
that logicians understand, we eliminate the cumbersome concepts of ``free
variable,''\index{free variable} ``bound variable,''\index{bound variable} and
``proper substitution''\index{proper substitution}\index{substitution!proper}
as primitive notions.  These concepts are present in our system but are
defined in terms of concepts expressed by the axioms and can be eliminated in
principle.  In standard systems, these concepts are really like additional,
implicit axioms\index{implicit axiom} that are somewhat complex and cannot be
eliminated.

The traditional approach to logic, wherein free variables and proper
substitution is defined, is also possible to do directly in the Metamath
language.  However, the notation tends to become awkward, and there are
disadvantages:  for example, extending the definition of a wff with a
definition is awkward, because the free variable and proper substitution
concepts have to have their definitions also extended.  Our choice of
axioms for \texttt{set.mm} is to a certain extent a matter of style, in
an attempt to achieve overall simplicity, but you should also be aware
that the traditional approach is possible as well if you should choose
it.

\chapter{Using the Metamath Program}
\label{using}

\section{Installation}

The way that you install Metamath\index{Metamath!installation} on your
computer system will vary for different computers.  Current
instructions are provided with the Metamath program download at
\url{http://metamath.org}.  In general, the installation is simple.
There is one file containing the Metamath program itself.  This file is
usually called \texttt{metamath} or \texttt{metamath.}{\em xxx} where
{\em xxx} is the convention (such as \texttt{exe}) for an executable
program on your operating system.  There are several additional files
containing samples of the Metamath language, all ending with
\texttt{.mm}.  The file \texttt{set.mm}\index{set theory database
(\texttt{set.mm})} contains logic and set theory and can be used as a
starting point for other areas of mathematics.

You will also need a text editor\index{text editor} capable of editing plain
{\sc ascii}\footnote{American Standard Code for Information Interchange.} text
in order to prepare your input files.\index{ascii@{\sc ascii}}  Most computers
have this capability built in.  Note that plain text is not necessarily the
default for some word processing programs\index{word processor}, especially if
they can handle different fonts; for example, with Microsoft Word\index{Word
(Microsoft)}, you must save the file in the format ``Text Only With Line
Breaks'' to get a plain text\index{plain text} file.\footnote{It is
recommended that all lines in a Metamath source file be 79 characters or less
in length for compatibility among different computer terminals.  When creating
a source file on an editor such as Word, select a monospaced
font\index{monospaced font} such as Courier\index{Courier font} or
Monaco\index{Monaco font} to make this easier to achieve.  Better yet,
just use a plain text editor such as Notepad.}

On some computer systems, Metamath does not have the capability to print
its output directly; instead, you send its output to a file (using the
\texttt{open} commands described later).  The way you print this output
file depends on your computer.\index{printers} Some computers have a
print command, whereas with others, you may have to read the file into
an editor and print it from there.

If you want to print your Metamath source files with typeset formulas
containing standard mathematical symbols, you will need the \LaTeX\
typesetting program\index{latex@{\LaTeX}}, which is widely and freely
available for most operating systems.  It runs natively on Unix and
Linux, and can be installed on Windows as part of the free Cygwin
package (\url{http://cygwin.com}).

You can also produce {\sc html}\footnote{HyperText Markup Language.}
web pages.  The {\tt help html} command in the Metamath program will
assist you with this feature.

\section{Your First Formal System}\label{start}
\subsection{From Nothing to Zero}\label{startf}

To give you a feel for what the Metamath\index{Metamath} language looks like,
we will take a look at a very simple example from formal number
theory\index{number theory}.  This example is taken from
Mendelson\index{Mendelson, Elliot} \cite[p. 123]{Mendelson}.\footnote{To keep
the example simple, we have changed the formalism slightly, and what we call
axioms\index{axiom} are strictly speaking theorems\index{theorem} in
\cite{Mendelson}.}  We will look at a small subset of this theory, namely that
part needed for the first number theory theorem proved in \cite{Mendelson}.

First we will look at a standard formal proof\index{formal proof} for the
example we have picked, then we will look at the Metamath version.  If you
have never been exposed to formal proofs, the notation may seem to be such
overkill to express such simple notions that you may wonder if you are missing
something.  You aren't.  The concepts involved are in fact very simple, and a
detailed breakdown in this fashion is necessary to express the proof in a way
that can be verified mechanically.  And as you will see, Metamath breaks the
proof down into even finer pieces so that the mechanical verification process
can be about as simple as possible.

Before we can introduce the axioms\index{axiom} of the theory, we must define
the syntax rules for forming legal expressions\index{syntax rules}
(combinations of symbols) with which those axioms can be used. The number 0 is
a {\bf term}\index{term}; and if $ t$ and $r$ are terms, so is $(t+r)$. Here,
$ t$ and $r$ are ``metavariables''\index{metavariable} ranging over terms; they
themselves do not appear as symbols in an actual term.  Some examples of
actual terms are $(0 + 0)$ and $((0+0)+0)$.  (Note that our theory describes
only the number zero and sums of zeroes.  Of course, not much can be done with
such a trivial theory, but remember that we have picked a very small subset of
complete number theory for our example.  The important thing for you to focus
on is our definitions that describe how symbols are combined to form valid
expressions, and not on the content or meaning of those expressions.) If $ t$
and $r$ are terms, an expression of the form $ t=r$ is a {\bf wff}
(well-formed formula)\index{well-formed formula (wff)}; and if $P$ and $Q$ are
wffs, so is $(P\rightarrow Q)$ (which means ``$P$ implies
$Q$''\index{implication ($\rightarrow$)} or ``if $P$ then $Q$'').
Here $P$ and $Q$ are metavariables ranging over wffs.  Examples of actual
wffs are $0=0$, $(0+0)=0$, $(0=0 \rightarrow (0+0)=0)$, and $(0=0\rightarrow
(0=0\rightarrow 0=(0+0)))$.  (Our notation makes use of more parentheses than
are customary, but the elimination of ambiguity this way simplifies our
example by avoiding the need to define operator precedence\index{operator
precedence}.)

The {\bf axioms}\index{axiom} of our theory are all wffs of the following
form, where $ t$, $r$, and $s$ are any terms:

%Latex p. 92
\renewcommand{\theequation}{A\arabic{equation}}

\begin{equation}
(t=r\rightarrow (t=s\rightarrow r=s))
\end{equation}
\begin{equation}
(t+0)=t
\end{equation}

Note that there are an infinite number of axioms since there are an infinite
number of possible terms.  A1 and A2 are properly called ``axiom
schemes,''\index{axiom scheme} but we will refer to them as ``axioms'' for
brevity.

An axiom is a {\bf theorem}; and if $P$ and $(P\rightarrow Q)$ are theorems
(where $P$ and $Q$ are wffs), then $Q$ is also a theorem.\index{theorem}  The
second part of this definition is called the modus ponens (MP) rule of
inference\index{inference rule}\index{modus ponens}.  It allows us to obtain
new theorems from old ones.

The {\bf proof}\index{proof} of a theorem is a sequence of one or more
theorems, each of which is either an axiom or the result of modus ponens
applied to two previous theorems in the sequence, and the last of which is the
theorem being proved.

The theorem we will prove for our example is very simple:  $ t=t$.  The proof of
our theorem follows.  Study it carefully until you feel sure you
understand it.\label{zeroproof}

% Use tabu so that lines will wrap automatically as needed.
\begin{tabu} { l X X }
1. & $(t+0)=t$ & (by axiom A2) \\
2. & $(t+0)=t$ & (by axiom A2) \\
3. & $((t+0)=t \rightarrow ((t+0)=t\rightarrow t=t))$ & (by axiom A1) \\
4. & $((t+0)=t\rightarrow t=t)$ & (by MP applied to steps 2 and 3) \\
5. & $t=t$ & (by MP applied to steps 1 and 4) \\
\end{tabu}

(You may wonder why step 1 is repeated twice.  This is not necessary in the
formal language we have defined, but in Metamath's ``reverse Polish
notation''\index{reverse Polish notation (RPN)} for proofs, a previous step
can be referred to only once.  The repetition of step~1 here will enable you
to see more clearly the correspondence of this proof with the
Metamath\index{Metamath} version on p.~\pageref{demoproof}.)

Our theorem is more properly called a ``theorem scheme,''\index{theorem
scheme} for it represents an infinite number of theorems, one for each
possible term $ t$.  Two examples of actual theorems would be $0=0$ and
$(0+0)=(0+0)$.  Rarely do we prove actual theorems, since by proving schemes
we can prove an infinite number of theorems in one fell swoop.  Similarly, our
proof should really be called a ``proof scheme.''\index{proof scheme}  To
obtain an actual proof, pick an actual term to use in place of $ t$, and
substitute it for $ t$ throughout the proof.

Let's discuss what we have done here.  The axioms\index{axiom} of our theory,
A1 and A2, are trivial and obvious.  Everyone knows that adding zero to
something doesn't change it, and also that if two things are equal to a third,
then they are equal to each other. In fact, stating the trivial and obvious is
a goal to strive for in any axiomatic system.  From trivial and obvious truths
that everyone agrees upon, we can prove results that are not so obvious yet
have absolute faith in them.  If we trust the axioms and the rules, we must,
by definition, trust the consequences of those axioms and rules, if logic is
to mean anything at all.

Our rule of inference\index{rule}, modus ponens\index{modus ponens}, is also
pretty obvious once you understand what it means.  If we prove a fact $P$, and
we also prove that $P$ implies $Q$, then $Q$ necessarily follows as a new
fact.  The rule provides us with a means for obtaining new facts (i.e.\
theorems\index{theorem}) from old ones.

The theorem that we have proved, $ t=t$, is so fundamental that you may wonder
why it isn't one of the axioms\index{axiom}.  In some axiom systems of
arithmetic, it {\em is} an axiom.  The choice of axioms in a theory is to some
extent arbitrary and even an art form, constrained only by the requirement
that any two equivalent axiom systems be able to derive each other as
theorems.  We could imagine that the inventor of our axiom system originally
included $ t=t$ as an axiom, then discovered that it could be derived as a
theorem from the other axioms.  Because of this, it was not necessary to
keep it as an axiom.  By eliminating it, the final set of axioms became
that much simpler.

Unless you have worked with formal proofs\index{formal proof} before, it
probably wasn't apparent to you that $ t=t$ could be derived from our two
axioms until you saw the proof. While you certainly believe that $ t=t$ is
true, you might not be able to convince an imaginary skeptic who believes only
in our two axioms until you produce the proof.  Formal proofs such as this are
hard to come up with when you first start working with them, but after you get
used to them they can become interesting and fun.  Once you understand the
idea behind formal proofs you will have grasped the fundamental principle that
underlies all of mathematics.  As the mathematics becomes more sophisticated,
its proofs become more challenging, but ultimately they all can be broken down
into individual steps as simple as the ones in our proof above.

Mendelson's\index{Mendelson, Elliot} book, from which our example was taken,
contains a number of detailed formal proofs such as these, and you may be
interested in looking it up.  The book is intended for mathematicians,
however, and most of it is rather advanced.  Popular literature describing
formal proofs\index{formal proof} include \cite[p.~296]{Rucker}\index{Rucker,
Rudy} and \cite[pp.~204--230]{Hofstadter}\index{Hofstadter, Douglas R.}.

\subsection{Converting It to Metamath}\label{convert}

Formal proofs\index{formal proof} such as the one in our example break down
logical reasoning into small, precise steps that leave little doubt that the
results follow from the axioms\index{axiom}.  You might think that this would
be the finest breakdown we can achieve in mathematics.  However, there is more
to the proof than meets the eye. Although our axioms were rather simple, a lot
of verbiage was needed before we could even state them:  we needed to define
``term,'' ``wff,'' and so on.  In addition, there are a number of implied
rules that we haven't even mentioned. For example, how do we know that step 3
of our proof follows from axiom A1? There is some hidden reasoning involved in
determining this.  Axiom A1 has two occurrences of the letter $ t$.  One of
the implied rules states that whatever we substitute for $ t$ must be a legal
term\index{term}.\footnote{Some authors make this implied rule explicit by
stating, ``only expressions of the above form are terms,'' after defining
``term.''}  The expression $ t+0$ is pretty obviously a legal term whenever $
t$ is, but suppose we wanted to substitute a huge term with thousands of
symbols?  Certainly a lot of work would be involved in determining that it
really is a term, but in ordinary formal proofs all of this work would be
considered a single ``step.''

To express our axiom system in the Metamath\index{Metamath} language, we must
describe this auxiliary information in addition to the axioms themselves.
Metamath does not know what a ``term'' or a ``wff''\index{well-formed formula
(wff)} is.  In Metamath, the specification of the ways in which we can combine
symbols to obtain terms and wffs are like little axioms in themselves.  These
auxiliary axioms are expressed in the same notation as the ``real''
axioms\index{axiom}, and Metamath does not distinguish between the two.  The
distinction is made by you, i.e.\ by the way in which you interpret the
notation you have chosen to express these two kinds of axioms.

The Metamath language breaks down mathematical proofs into tiny pieces, much
more so than in ordinary formal proofs\index{formal proof}.  If a single
step\index{proof step} involves the
substitution\index{substitution!variable}\index{variable substitution} of a
complex term for one of its variables, Metamath must see this single step
broken down into many small steps.  This fine-grained breakdown is what gives
Metamath generality and flexibility as it lets it not be limited to any
particular mathematical notation.

Metamath's proof notation is not, in itself, intended to be read by humans but
rather is in a compact format intended for a machine.  The Metamath program
will convert this notation to a form you can understand, using the \texttt{show
proof}\index{\texttt{show proof} command} command.  You can tell the program what
level of detail of the proof you want to look at.  You may want to look at
just the logical inference steps that correspond
to ordinary formal proof steps,
or you may want to see the fine-grained steps that prove that an expression is
a term.

Here, without further ado, is our example converted to the
Metamath\index{Metamath} language:\index{metavariable}\label{demo0}

\begin{verbatim}
$( Declare the constant symbols we will use $)
    $c 0 + = -> ( ) term wff |- $.
$( Declare the metavariables we will use $)
    $v t r s P Q $.
$( Specify properties of the metavariables $)
    tt $f term t $.
    tr $f term r $.
    ts $f term s $.
    wp $f wff P $.
    wq $f wff Q $.
$( Define "term" and "wff" $)
    tze $a term 0 $.
    tpl $a term ( t + r ) $.
    weq $a wff t = r $.
    wim $a wff ( P -> Q ) $.
$( State the axioms $)
    a1 $a |- ( t = r -> ( t = s -> r = s ) ) $.
    a2 $a |- ( t + 0 ) = t $.
$( Define the modus ponens inference rule $)
    ${
       min $e |- P $.
       maj $e |- ( P -> Q ) $.
       mp  $a |- Q $.
    $}
$( Prove a theorem $)
    th1 $p |- t = t $=
  $( Here is its proof: $)
       tt tze tpl tt weq tt tt weq tt a2 tt tze tpl
       tt weq tt tze tpl tt weq tt tt weq wim tt a2
       tt tze tpl tt tt a1 mp mp
     $.
\end{verbatim}\index{metavariable}

A ``database''\index{database} is a set of one or more {\sc ascii} source
files.  Here's a brief description of this Metamath\index{Metamath} database
(which consists of this single source file), so that you can understand in
general terms what is going on.  To understand the source file in detail, you
should read Chapter~\ref{languagespec}.

The database is a sequence of ``tokens,''\index{token} which are normally
separated by spaces or line breaks.  The only tokens that are built into
the Metamath language are those beginning with \texttt{\$}.  These tokens
are called ``keywords.''\index{keyword}  All other tokens are
user-defined, and their names are arbitrary.

As you might have guessed, the Metamath token \texttt{\$(}\index{\texttt{\$(} and
\texttt{\$)} auxiliary keywords} starts a comment and \texttt{\$)} ends a comment.

The Metamath tokens \texttt{\$c}\index{\texttt{\$c} statement},
\texttt{\$v}\index{\texttt{\$v} statement},
\texttt{\$e}\index{\texttt{\$e} statement},
\texttt{\$f}\index{\texttt{\$f} statement},
\texttt{\$a}\index{\texttt{\$a} statement}, and
\texttt{\$p}\index{\texttt{\$p} statement} specify ``statements'' that
end with \texttt{\$.}\,.\index{\texttt{\$.}\ keyword}

The Metamath tokens \texttt{\$c} and \texttt{\$v} each declare\index{constant
declaration}\index{variable declaration} a list of user-defined tokens, called
``math symbols,''\index{math symbol} that the database will reference later
on.  All of the math symbols they define you have seen earlier except the
turnstile symbol \texttt{|-} ($\vdash$)\index{turnstile ({$\,\vdash$})}, which is
commonly used by logicians to mean ``a proof exists for.''  For us
the turnstile is just a
convenient symbol that distinguishes expressions that are axioms\index{axiom}
or theorems\index{theorem} from expressions that are terms or wffs.

The \texttt{\$c} statement declares ``constants''\index{constant} and
the \texttt{\$v} statement declares
``variables''\index{variable}\index{constant declaration}\index{variable
declaration} (or more precisely, metavariables\index{metavariable}).  A
variable may be substituted\index{substitution!variable}\index{variable
substitution} with sequences of math symbols whereas a constant may not
be substituted with anything.

It may seem redundant to require both \texttt{\$c}\index{\texttt{\$c} statement} and
\texttt{\$v}\index{\texttt{\$v} statement} statements (since any math
symbol\index{math symbol} not specified with a \texttt{\$c} statement could be
presumed to be a variable), but this provides for better error checking and
also allows math symbols to be redeclared\index{redeclaration of symbols}
(Section~\ref{scoping}).

The token \texttt{\$f}\index{\texttt{\$f} statement} specifies a
statement called a ``variable-type hypothesis'' (also called a
``floating hypothesis'') and \texttt{\$e}\index{\texttt{\$e} statement}
specifies a ``logical hypothesis'' (also called an ``essential
hypothesis'').\index{hypothesis}\index{variable-type
hypothesis}\index{logical hypothesis}\index{floating
hypothesis}\index{essential hypothesis} The token
\texttt{\$a}\index{\texttt{\$a} statement} specifies an ``axiomatic
assertion,''\index{axiomatic assertion} and
\texttt{\$p}\index{\texttt{\$p} statement} specifies a ``provable
assertion.''\index{provable assertion} To the left of each occurrence of
these four tokens is a ``label''\index{label} that identifies the
hypothesis or assertion for later reference.  For example, the label of
the first axiomatic assertion is \texttt{tze}.  A \texttt{\$f} statement
must contain exactly two math symbols, a constant followed by a
variable.  The \texttt{\$e}, \texttt{\$a}, and \texttt{\$p} statements
each start with a constant followed by, in general, an arbitrary
sequence of math symbols.

Associated with each assertion\index{assertion} is a set of hypotheses
that must be satisfied in order for the assertion to be used in a proof.
These are called the ``mandatory hypotheses''\index{mandatory
hypothesis} of the assertion.  Among those hypotheses whose ``scope''
(described below) includes the assertion, \texttt{\$e} hypotheses are
always mandatory and \texttt{\$f}\index{\texttt{\$f} statement}
hypotheses are mandatory when they share their variable with the
assertion or its \texttt{\$e} hypotheses.  The exact rules for
determining which hypotheses are mandatory are described in detail in
Sections~\ref{frames} and \ref{scoping}.  For example, the mandatory
hypotheses of assertion \texttt{tpl} are \texttt{tt} and \texttt{tr},
whereas assertion \texttt{tze} has no mandatory hypotheses because it
contains no variables and has no \texttt{\$e}\index{\texttt{\$e}
statement} hypothesis.  Metamath's \texttt{show statement}
command\index{\texttt{show statement} command}, described in the next
section, will show you a statement's mandatory hypotheses.

Sometimes we need to make a hypothesis relevant to only certain
assertions.  The set of statements to which a hypothesis is relevant is
called its ``scope.''  The Metamath brackets,
\texttt{\$\char`\{}\index{\texttt{\$\char`\{} and \texttt{\$\char`\}}
keywords} and \texttt{\$\char`\}}, define a ``block''\index{block} that
delimits the scope of any hypothesis contained between them.  The
assertion \texttt{mp} has mandatory hypotheses \texttt{wp}, \texttt{wq},
\texttt{min}, and \texttt{maj}.  The only mandatory hypothesis of
\texttt{th1}, on the other hand, is \texttt{tt}, since \texttt{th1}
occurs outside of the block containing \texttt{min} and \texttt{maj}.

Note that \texttt{\$\char`\{} and \texttt{\$\char`\}} do not affect the
scope of assertions (\texttt{\$a} and \texttt{\$p}).  Assertions are always
available to be referenced by any later proof in the source file.

Each provable assertion (\texttt{\$p}\index{\texttt{\$p} statement}
statement) has two parts.  The first part is the
assertion\index{assertion} itself, which is a sequence of math
symbol\index{math symbol} tokens placed between the \texttt{\$p} token
and a \texttt{\$=}\index{\texttt{\$=} keyword} token.  The second part
is a ``proof,'' which is a list of label tokens placed between the
\texttt{\$=} token and the \texttt{\$.}\index{\texttt{\$.}\ keyword}\
token that ends the statement.\footnote{If you've looked at the
\texttt{set.mm} database, you may have noticed another notation used for
proofs.  The other notation is called ``compressed.''\index{compressed
proof}\index{proof!compressed} It reduces the amount of space needed to
store a proof in the database and is described in
Appendix~\ref{compressed}.  In the example above, we use
``normal''\index{normal proof}\index{proof!normal} notation.} The proof
acts as a series of instructions to the Metamath program, telling it how
to build up the sequence of math symbols contained in the assertion part of
the \texttt{\$p} statement, making use of the hypotheses of the
\texttt{\$p} statement and previous assertions.  The construction takes
place according to precise rules.  If the list of labels in the proof
causes these rules to be violated, or if the final sequence that results
does not match the assertion, the Metamath program will notify you with
an error message.

If you are familiar with reverse Polish notation (RPN), which is sometimes used
on pocket calculators, here in a nutshell is how a proof works.  Each
hypothesis label\index{hypothesis label} in the proof is pushed\index{push}
onto the RPN stack\index{stack}\index{RPN stack} as it is encountered. Each
assertion label\index{assertion label} pops\index{pop} off the stack as many
entries as the referenced assertion has mandatory hypotheses.  Variable
substitutions\index{substitution!variable}\index{variable substitution} are
computed which, when made to the referenced assertion's mandatory hypotheses,
cause these hypotheses to match the stack entries. These same substitutions
are then made to the variables in the referenced assertion itself, which is
then pushed onto the stack.  At the end of the proof, there should be one
stack entry, namely the assertion being proved.  This process is explained in
detail in Section~\ref{proof}.

Metamath's proof notation is not very readable for humans, but it allows the
proof to be stored compactly in a file.  The Metamath\index{Metamath} program
has proof display features that let you see what's going on in a more
readable way, as you will see in the next section.

The rules used in verifying a proof are not based on any built-in syntax of the
symbol sequence in an assertion\index{assertion} nor on any built-in meanings
attached to specific symbol names.  They are based strictly on symbol
matching:  constants\index{constant} must match themselves, and
variables\index{variable} may be replaced with anything that allows a match to
occur.  For example, instead of \texttt{term}, \texttt{0}, and \verb$|-$ we could
have just as well used \texttt{yellow}, \texttt{zero}, and \texttt{provable}, as long
as we did so consistently throughout the database.  Also, we could have used
\texttt{is provable} (two tokens) instead of \verb$|-$ (one token) throughout the
database.  In each of these cases, the proof would be exactly the same.  The
independence of proofs and notation means that you have a lot of flexibility to
change the notation you use without having to change any proofs.

\section{A Trial Run}\label{trialrun}

Now you are ready to try out the Metamath\index{Metamath} program.

On all computer systems, Metamath has a standard ``command line
interface'' (CLI)\index{command line interface (CLI)} that allows you to
interact with it.  You supply commands to the CLI by typing them on the
keyboard and pressing your keyboard's {\em return} key after each line
you enter.  The CLI is designed to be easy to use and has built-in help
features.

The first thing you should do is to use a text editor to create a file
called \texttt{demo0.mm} and type into it the Metamath source shown on
p.~\pageref{demo0}.  Actually, this file is included with your Metamath
software package, so check that first.  If you type it in, make sure
that you save it in the form of ``plain {\sc ascii} text with line
breaks.''  Most word processors will have this feature.

Next you must run the Metamath program.  Depending on your computer
system and how Metamath is installed, this could range from clicking the
mouse on the Metamath icon to typing \texttt{run metamath} to typing
simply \texttt{metamath}.  (Metamath's {\tt help invoke} command describes
alternate ways of invoking the Metamath program.)

When you first enter Metamath\index{Metamath}, it will be at the CLI, waiting
for your input. You will see something like the following on your screen:
\begin{verbatim}
Metamath - Version 0.177 27-Apr-2019
Type HELP for help, EXIT to exit.
MM>
\end{verbatim}
The \texttt{MM>} prompt means that Metamath is waiting for a command.
Command keywords\index{command keyword} are not case sensitive;
we will use lower-case commands in our examples.
The version number and its release date will probably be different on your
system from the one we show above.

The first thing that you need to do is to read in your
database:\index{\texttt{read} command}\footnote{If a directory path is
needed on Unix,\index{Unix file names}\index{file names!Unix} you should
enclose the path/file name in quotes to prevent Metamath from thinking
that the \texttt{/} in the path name is a command qualifier, e.g.,
\texttt{read \char`\"db/set.mm\char`\"}.  Quotes are optional when there
is no ambiguity.}
\begin{verbatim}
MM> read demo0.mm
\end{verbatim}
Remember to press the {\em return} key after entering this command.  If
you omit the file name, Metamath will prompt you for one.   The syntax for
specifying a Macintosh file name path is given in a footnote on
p.~\pageref{includef}.\index{Macintosh file names}\index{file
names!Macintosh}

If there are any syntax errors in the database, Metamath will let you know
when it reads in the file.  The one thing that Metamath does not check when
reading in a database is that all proofs are correct, because this would
slow it down too much.  It is a good idea to periodically verify the proofs in
a database you are making changes to.  To do this, use the following command
(and do it for your \texttt{demo0.mm} file now).  Note that the \texttt{*} is a
``wild card'' meaning all proofs in the file.\index{\texttt{verify proof} command}
\begin{verbatim}
MM> verify proof *
\end{verbatim}
Metamath will report any proofs that are incorrect.

It is often useful to save the information that the Metamath program displays
on the screen. You can save everything that happens on the screen by opening a
log file. You may want to do this before you read in a database so that you
can examine any errors later on.  To open a log file, type
\begin{verbatim}
MM> open log abc.log
\end{verbatim}
This will open a file called \texttt{abc.log}, and everything that appears on the
screen from this point on will be stored in this file.  The name of the log file
is arbitrary. To close the log file, type
\begin{verbatim}
MM> close log
\end{verbatim}

Several commands let you examine what's inside your database.
Section~\ref{exploring} has an overview of some useful ones.  The
\texttt{show labels} command lets you see what statement
labels\index{label} exist.  A \texttt{*} matches any combination of
characters, and \texttt{t*} refers to all labels starting with the
letter \texttt{t}.\index{\texttt{show labels} command} The \texttt{/all}
is a ``command qualifier''\index{command qualifier} that tells Metamath
to include labels of hypotheses.  (To see the syntax explained, type
\texttt{help show labels}.)  Type
\begin{verbatim}
MM> show labels t* /all
\end{verbatim}
Metamath will respond with
\begin{verbatim}
The statement number, label, and type are shown.
3 tt $f       4 tr $f       5 ts $f       8 tze $a
9 tpl $a      19 th1 $p
\end{verbatim}

You can use the \texttt{show statement} command to get information about a
particular statement.\index{\texttt{show statement} command}
For example, you can get information about the statement with label \texttt{mp}
by typing
\begin{verbatim}
MM> show statement mp /full
\end{verbatim}
Metamath will respond with
\begin{verbatim}
Statement 17 is located on line 43 of the file
"demo0.mm".
"Define the modus ponens inference rule"
17 mp $a |- Q $.
Its mandatory hypotheses in RPN order are:
  wp $f wff P $.
  wq $f wff Q $.
  min $e |- P $.
  maj $e |- ( P -> Q ) $.
The statement and its hypotheses require the
      variables:  Q P
The variables it contains are:  Q P
\end{verbatim}
The mandatory hypotheses\index{mandatory hypothesis} and their
order\index{RPN order} are
useful to know when you are trying to understand or debug a proof.

Now you are ready to look at what's really inside our proof.  First, here is
how to look at every step in the proof---not just the ones corresponding to an
ordinary formal proof\index{formal proof}, but also the ones that build up the
formulas that appear in each ordinary formal proof step.\index{\texttt{show
proof} command}
\begin{verbatim}
MM> show proof th1 /lemmon /all
\end{verbatim}

This will display the proof on the screen in the following format:
\begin{verbatim}
 1 tt            $f term t
 2 tze           $a term 0
 3 1,2 tpl       $a term ( t + 0 )
 4 tt            $f term t
 5 3,4 weq       $a wff ( t + 0 ) = t
 6 tt            $f term t
 7 tt            $f term t
 8 6,7 weq       $a wff t = t
 9 tt            $f term t
10 9 a2          $a |- ( t + 0 ) = t
11 tt            $f term t
12 tze           $a term 0
13 11,12 tpl     $a term ( t + 0 )
14 tt            $f term t
15 13,14 weq     $a wff ( t + 0 ) = t
16 tt            $f term t
17 tze           $a term 0
18 16,17 tpl     $a term ( t + 0 )
19 tt            $f term t
20 18,19 weq     $a wff ( t + 0 ) = t
21 tt            $f term t
22 tt            $f term t
23 21,22 weq     $a wff t = t
24 20,23 wim     $a wff ( ( t + 0 ) = t -> t = t )
25 tt            $f term t
26 25 a2         $a |- ( t + 0 ) = t
27 tt            $f term t
28 tze           $a term 0
29 27,28 tpl     $a term ( t + 0 )
30 tt            $f term t
31 tt            $f term t
32 29,30,31 a1   $a |- ( ( t + 0 ) = t -> ( ( t + 0 )
                                     = t -> t = t ) )
33 15,24,26,32 mp  $a |- ( ( t + 0 ) = t -> t = t )
34 5,8,10,33 mp  $a |- t = t
\end{verbatim}

The \texttt{/lemmon} command qualifier specifies what is known as a Lemmon-style
display\index{Lemmon-style proof}\index{proof!Lemmon-style}.  Omitting the
\texttt{/lemmon} qualifier results in a tree-style proof (see
p.~\pageref{treeproof} for an example) that is somewhat less explicit but
easier to follow once you get used to it.\index{tree-style
proof}\index{proof!tree-style}

The first number on each line is the step
number of the proof.  Any numbers that follow are step numbers assigned to the
hypotheses of the statement referenced by that step.  Next is the label of
the statement referenced by the step.  The statement type of the statement
referenced comes next, followed by the math symbol\index{math symbol} string
constructed by the proof up to that step.

The last step, 34, contains the statement that is being proved.

Looking at a small piece of the proof, notice that steps 3 and 4 have
established that
\texttt{( t + 0 )} and \texttt{t} are \texttt{term}\,s, and step 5 makes use of steps 3 and
4 to establish that \texttt{( t + 0 ) = t} is a \texttt{wff}.  Let Metamath
itself tell us in detail what is happening in step 5.  Note that the
``target hypothesis'' refers to where step 5 is eventually used, i.e., in step
34.
\begin{verbatim}
MM> show proof th1 /detailed_step 5
Proof step 5:  wp=weq $a wff ( t + 0 ) = t
This step assigns source "weq" ($a) to target "wp"
($f).  The source assertion requires the hypotheses
"tt" ($f, step 3) and "tr" ($f, step 4).  The parent
assertion of the target hypothesis is "mp" ($a,
step 34).
The source assertion before substitution was:
    weq $a wff t = r
The following substitutions were made to the source
assertion:
    Variable  Substituted with
     t         ( t + 0 )
     r         t
The target hypothesis before substitution was:
    wp $f wff P
The following substitution was made to the target
hypothesis:
    Variable  Substituted with
     P         ( t + 0 ) = t
\end{verbatim}

The full proof just shown is useful to understand what is going on in detail.
However, most of the time you will just be interested in
the ``essential'' or logical steps of a proof, i.e.\ those steps
that correspond to an
ordinary formal proof\index{formal proof}.  If you type
\begin{verbatim}
MM> show proof th1 /lemmon /renumber
\end{verbatim}
you will see\label{demoproof}
\begin{verbatim}
1 a2             $a |- ( t + 0 ) = t
2 a2             $a |- ( t + 0 ) = t
3 a1             $a |- ( ( t + 0 ) = t -> ( ( t + 0 )
                                     = t -> t = t ) )
4 2,3 mp         $a |- ( ( t + 0 ) = t -> t = t )
5 1,4 mp         $a |- t = t
\end{verbatim}
Compare this to the formal proof on p.~\pageref{zeroproof} and
notice the resemblance.
By default Metamath
does not show \texttt{\$f}\index{\texttt{\$f}
statement} hypotheses and everything branching off of them in the proof tree
when the proof is displayed; this makes the proof look more like an ordinary
mathematical proof, which does not normally incorporate the explicit
construction of expressions.
This is called the ``essential'' view
(at one time you had to add the
\texttt{/essential} qualifier in the \texttt{show proof}
command to get this view, but this is now the default).
You can could use the \texttt{/all} qualifier in the \texttt{show
proof} command to also show the explicit construction of expressions.
The \texttt{/renumber} qualifier means to renumber
the steps to correspond only to what is displayed.\index{\texttt{show proof}
command}

To exit Metamath, type\index{\texttt{exit} command}
\begin{verbatim}
MM> exit
\end{verbatim}

\subsection{Some Hints for Using the Command Line Interface}

We will conclude this quick introduction to Metamath\index{Metamath} with some
helpful hints on how to navigate your way through the commands.
\index{command line interface (CLI)}

When you type commands into Metamath's CLI, you only have to type as many
characters of a command keyword\index{command keyword} as are needed to make
it unambiguous.  If you type too few characters, Metamath will tell you what
the choices are.  In the case of the \texttt{read} command, only the \texttt{r} is
needed to specify it unambiguously, so you could have typed\index{\texttt{read}
command}
\begin{verbatim}
MM> r demo0.mm
\end{verbatim}
instead of
\begin{verbatim}
MM> read demo0.mm
\end{verbatim}
In our description, we always show the full command words.  When using the
Metamath CLI commands in a command file (to be read with the \texttt{submit}
command)\index{\texttt{submit} command}, it is good practice to use
the unabbreviated command to ensure your instructions will not become ambiguous
if more commands are added to the Metamath program in the future.

The command keywords\index{command
keyword} are not case sensitive; you may type either \texttt{read} or
\texttt{ReAd}.  File names may or may not be case sensitive, depending on your
computer's operating system.  Metamath label\index{label} and math
symbol\index{math symbol} tokens\index{token} are case-sensitive.

The \texttt{help} command\index{\texttt{help} command} will provide you
with a list of topics you can get help on.  You can then type
\texttt{help} {\em topic} to get help on that topic.

If you are uncertain of a command's spelling, just type as many characters
as you remember of the command.  If you have not typed enough characters to
specify it unambiguously, Metamath will tell you what choices you have.

\begin{verbatim}
MM> show s
         ^
?Ambiguous keyword - please specify SETTINGS,
STATEMENT, or SOURCE.
\end{verbatim}

If you don't know what argument to use as part of a command, type a
\texttt{?}\index{\texttt{]}@\texttt{?}\ in command lines}\ at the
argument position.  Metamath will tell you what it expected there.

\begin{verbatim}
MM> show ?
         ^
?Expected SETTINGS, LABELS, STATEMENT, SOURCE, PROOF,
MEMORY, TRACE_BACK, or USAGE.
\end{verbatim}

Finally, you may type just the first word or words of a command followed
by {\em return}.  Metamath will prompt you for the remaining part of the
command, showing you the choices at each step.  For example, instead of
typing \texttt{show statement th1 /full} you could interact in the
following manner:
\begin{verbatim}
MM> show
SETTINGS, LABELS, STATEMENT, SOURCE, PROOF,
MEMORY, TRACE_BACK, or USAGE <SETTINGS>? st
What is the statement label <th1>?
/ or nothing <nothing>? /
TEX, COMMENT_ONLY, or FULL <TEX>? f
/ or nothing <nothing>?
19 th1 $p |- t = t $= ... $.
\end{verbatim}
After each \texttt{?}\ in this mode, you must give Metamath the
information it requests.  Sometimes Metamath gives you a list of choices
with the default choice indicated by brackets \texttt{< > }. Pressing
{\em return} after the \texttt{?}\ will select the default choice.
Answering anything else will override the default.  Note that the
\texttt{/} in command qualifiers is considered a separate
token\index{token} by the parser, and this is why it is asked for
separately.

\section{Your First Proof}\label{frstprf}

Proofs are developed with the aid of the Proof Assistant\index{Proof
Assistant}.  We will now show you how the proof of theorem \texttt{th1}
was built.  So that you can repeat these steps, we will first have the
Proof Assistant erase the proof in Metamath's source buffer\index{source
buffer}, then reconstruct it.  (The source buffer is the place in memory
where Metamath stores the information in the database when it is
\texttt{read}\index{\texttt{read} command} in.  New or modified proofs
are kept in the source buffer until a \texttt{write source}
command\index{\texttt{write source} command} is issued.)  In practice, you
would place a \texttt{?}\index{\texttt{]}@\texttt{?}\ inside proofs}\
between \texttt{\$=}\index{\texttt{\$=} keyword} and
\texttt{\$.}\index{\texttt{\$.}\ keyword}\ in the database to indicate
to Metamath\index{Metamath} that the proof is unknown, and that would be
your starting point.  Whenever the \texttt{verify proof} command encounters
a proof with a \texttt{?}\ in place of a proof step, the statement is
identified as not proved.

When I first started creating Metamath proofs, I would write down
on a piece of paper the complete
formal proof\index{formal proof} as it would appear
in a \texttt{show proof} command\index{\texttt{show proof} command}; see
the display of \texttt{show proof th1 /lemmon /re\-num\-ber} above as an
example.  After you get used to using the Proof Assistant\index{Proof
Assistant} you may get to a point where you can ``see'' the proof in your mind
and let the Proof Assistant guide you in filling in the details, at least for
simpler proofs, but until you gain that experience you may find it very useful
to write down all the details in advance.
Otherwise you may waste a lot of time as you let it take you down a wrong path.
However, others do not find this approach as helpful.
For example, Thomas Brendan Leahy\index{Leahy, Thomas Brendan}
finds that it is more helpful to him to interactively
work backward from a machine-readable statement.
David A. Wheeler\index{Wheeler, David A.}
writes down a general approach, but develops the proof
interactively by switching between
working forwards (from hypotheses and facts likely to be useful) and
backwards (from the goal) until the forwards and backwards approaches meet.
In the end, use whatever approach works for you.

A proof is developed with the Proof Assistant by working backwards, starting
with the theorem\index{theorem} to be proved, and assigning each unknown step
with a theorem or hypothesis until no more unknown steps remain.  The Proof
Assistant will not let you make an assignment unless it can be ``unified''
with the unknown step.  This means that a
substitution\index{substitution!variable}\index{variable substitution} of
variables exists that will make the assignment match the unknown step.  On the
other hand, in the middle of a proof, when working backwards, often more than
one unification\index{unification} (set of substitutions) is possible, since
there is not enough information available at that point to uniquely establish
it.  In this case you can tell Metamath which unification to choose, or you
can continue to assign unknown steps until enough information is available to
make the unification unique.

We will assume you have entered Metamath and read in the database as described
above.  The following dialog shows how the proof was developed.  For more
details on what some of the commands do, refer to Section~\ref{pfcommands}.
\index{\texttt{prove} command}

\begin{verbatim}
MM> prove th1
Entering the Proof Assistant.  Type HELP for help, EXIT
to exit.  You will be working on the proof of statement th1:
  $p |- t = t
Note:  The proof you are starting with is already complete.
MM-PA>
\end{verbatim}

The \verb/MM-PA>/ prompt means we are inside the Proof
Assistant.\index{Proof Assistant} Most of the regular Metamath commands
(\texttt{show statement}, etc.) are still available if you need them.

\begin{verbatim}
MM-PA> delete all
The entire proof was deleted.
\end{verbatim}

We have deleted the whole proof so we can start from scratch.

\begin{verbatim}
MM-PA> show new_proof/lemmon/all
1 ?              $? |- t = t
\end{verbatim}

The \texttt{show new{\char`\_}proof} command\index{\texttt{show
new{\char`\_}proof} command} is like \texttt{show proof} except that we
don't specify a statement; instead, the proof we're working on is
displayed.

\begin{verbatim}
MM-PA> assign 1 mp
To undo the assignment, DELETE STEP 5 and INITIALIZE, UNIFY
if needed.
3   min=?  $? |- $2
4   maj=?  $? |- ( $2 -> t = t )
\end{verbatim}

The \texttt{assign} command\index{\texttt{assign} command} above means
``assign step 1 with the statement whose label is \texttt{mp}.''  Note
that step renumbering will constantly occur as you assign steps in the
middle of a proof; in general all steps from the step you assign until
the end of the proof will get moved up.  In this case, what used to be
step 1 is now step 5, because the (partial) proof now has five steps:
the four hypotheses of the \texttt{mp} statement and the \texttt{mp}
statement itself.  Let's look at all the steps in our partial proof:

\begin{verbatim}
MM-PA> show new_proof/lemmon/all
1 ?              $? wff $2
2 ?              $? wff t = t
3 ?              $? |- $2
4 ?              $? |- ( $2 -> t = t )
5 1,2,3,4 mp     $a |- t = t
\end{verbatim}

The symbol \texttt{\$2} is a temporary variable\index{temporary
variable} that represents a symbol sequence not yet known.  In the final
proof, all temporary variables will be eliminated.  The general format
for a temporary variable is \texttt{\$} followed by an integer.  Note
that \texttt{\$} is not a legal character in a math symbol (see
Section~\ref{dollardollar}, p.~\pageref{dollardollar}), so there will
never be a naming conflict between real symbols and temporary variables.

Unknown steps 1 and 2 are constructions of the two wffs used by the
modus ponens rule.  As you will see at the end of this section, the
Proof Assistant\index{Proof Assistant} can usually figure these steps
out by itself, and we will not have to worry about them.  Therefore from
here on we will display only the ``essential'' hypotheses, i.e.\ those
steps that correspond to traditional formal proofs\index{formal proof}.

\begin{verbatim}
MM-PA> show new_proof/lemmon
3 ?              $? |- $2
4 ?              $? |- ( $2 -> t = t )
5 3,4 mp         $a |- t = t
\end{verbatim}

Unknown steps 3 and 4 are the ones we must focus on.  They correspond to the
minor and major premises of the modus ponens rule.  We will assign them as
follows.  Notice that because of the step renumbering that takes place
after an assignment, it is advantageous to assign unknown steps in reverse
order, because earlier steps will not get renumbered.

\begin{verbatim}
MM-PA> assign 4 mp
To undo the assignment, DELETE STEP 8 and INITIALIZE, UNIFY
if needed.
3   min=?  $? |- $2
6     min=?  $? |- $4
7     maj=?  $? |- ( $4 -> ( $2 -> t = t ) )
\end{verbatim}

We are now going to describe an obscure feature that you will probably
never use but should be aware of.  The Metamath language allows empty
symbol sequences to be substituted for variables, but in most formal
systems this feature is never used.  One of the few examples where is it
used is the MIU-system\index{MIU-system} described in
Appendix~\ref{MIU}.  But such systems are rare, and by default this
feature is turned off in the Proof Assistant.  (It is always allowed for
{\tt verify proof}.)  Let us turn it on and see what
happens.\index{\texttt{set empty{\char`\_}substitution} command}

\begin{verbatim}
MM-PA> set empty_substitution on
Substitutions with empty symbol sequences is now allowed.
\end{verbatim}

With this feature enabled, more unifications will be
ambiguous\index{ambiguous unification}\index{unification!ambiguous} in
the middle of a proof, because
substitution\index{substitution!variable}\index{variable substitution}
of variables with empty symbol sequences will become an additional
possibility.  Let's see what happens when we make our next assignment.

\begin{verbatim}
MM-PA> assign 3 a2
There are 2 possible unifications.  Please select the correct
    one or Q if you want to UNIFY later.
Unify:  |- $6
 with:  |- ( $9 + 0 ) = $9
Unification #1 of 2 (weight = 7):
  Replace "$6" with "( + 0 ) ="
  Replace "$9" with ""
  Accept (A), reject (R), or quit (Q) <A>? r
\end{verbatim}

The first choice presented is the wrong one.  If we had selected it,
temporary variable \texttt{\$6} would have been assigned a truncated
wff, and temporary variable \texttt{\$9} would have been assigned an
empty sequence (which is not allowed in our system).  With this choice,
eventually we would reach a point where we would get stuck because
we would end up with steps impossible to prove.  (You may want to
try it.)  We typed \texttt{r} to reject the choice.

\begin{verbatim}
Unification #2 of 2 (weight = 21):
  Replace "$6" with "( $9 + 0 ) = $9"
  Accept (A), reject (R), or quit (Q) <A>? q
To undo the assignment, DELETE STEP 4 and INITIALIZE, UNIFY
if needed.
 7     min=?  $? |- $8
 8     maj=?  $? |- ( $8 -> ( $6 -> t = t ) )
\end{verbatim}

The second choice is correct, and normally we would type \texttt{a}
to accept it.  But instead we typed \texttt{q} to show what will happen:
it will leave the step with an unknown unification, which can be
seen as follows:

\begin{verbatim}
MM-PA> show new_proof/not_unified
 4   min    $a |- $6
        =a2  = |- ( $9 + 0 ) = $9
\end{verbatim}

Later we can unify this with the \texttt{unify}
\texttt{all/interactive} command.

The important point to remember is that occasionally you will be
presented with several unification choices while entering a proof, when
the program determines that there is not enough information yet to make
an unambiguous choice automatically (and this can happen even with
\texttt{set empty{\char`\_}substitution} turned off).  Usually it is
obvious by inspection which choice is correct, since incorrect ones will
tend to be meaningless fragments of wffs.  In addition, the correct
choice will usually be the first one presented, unlike our example
above.

Enough of this digression.  Let us go back to the default setting.

\begin{verbatim}
MM-PA> set empty_substitution off
The ability to substitute empty expressions for variables
has been turned off.  Note that this may make the Proof
Assistant too restrictive in some cases.
\end{verbatim}

If we delete the proof, start over, and get to the point where
we digressed above, there will no longer be an ambiguous unification.

\begin{verbatim}
MM-PA> assign 3 a2
To undo the assignment, DELETE STEP 4 and INITIALIZE, UNIFY
if needed.
 7     min=?  $? |- $4
 8     maj=?  $? |- ( $4 -> ( ( $5 + 0 ) = $5 -> t = t ) )
\end{verbatim}

Let us look at our proof so far, and continue.

\begin{verbatim}
MM-PA> show new_proof/lemmon
 4 a2            $a |- ( $5 + 0 ) = $5
 7 ?             $? |- $4
 8 ?             $? |- ( $4 -> ( ( $5 + 0 ) = $5 -> t = t ) )
 9 7,8 mp        $a |- ( ( $5 + 0 ) = $5 -> t = t )
10 4,9 mp        $a |- t = t
MM-PA> assign 8 a1
To undo the assignment, DELETE STEP 11 and INITIALIZE, UNIFY
if needed.
 7     min=?  $? |- ( t + 0 ) = t
MM-PA> assign 7 a2
To undo the assignment, DELETE STEP 8 and INITIALIZE, UNIFY
if needed.
MM-PA> show new_proof/lemmon
 4 a2            $a |- ( t + 0 ) = t
 8 a2            $a |- ( t + 0 ) = t
12 a1            $a |- ( ( t + 0 ) = t -> ( ( t + 0 ) = t ->
                                                    t = t ) )
13 8,12 mp       $a |- ( ( t + 0 ) = t -> t = t )
14 4,13 mp       $a |- t = t
\end{verbatim}

Now all temporary variables and unknown steps have been eliminated from the
``essential'' part of the proof.  When this is achieved, the Proof
Assistant\index{Proof Assistant} can usually figure out the rest of the proof
automatically.  (Note that the \texttt{improve} command can occasionally be
useful for filling in essential steps as well, but it only tries to make use
of statements that introduce no new variables in their hypotheses, which is
not the case for \texttt{mp}. Also it will not try to improve steps containing
temporary variables.)  Let's look at the complete proof, then run
the \texttt{improve} command, then look at it again.

\begin{verbatim}
MM-PA> show new_proof/lemmon/all
 1 ?             $? wff ( t + 0 ) = t
 2 ?             $? wff t = t
 3 ?             $? term t
 4 3 a2          $a |- ( t + 0 ) = t
 5 ?             $? wff ( t + 0 ) = t
 6 ?             $? wff ( ( t + 0 ) = t -> t = t )
 7 ?             $? term t
 8 7 a2          $a |- ( t + 0 ) = t
 9 ?             $? term ( t + 0 )
10 ?             $? term t
11 ?             $? term t
12 9,10,11 a1    $a |- ( ( t + 0 ) = t -> ( ( t + 0 ) = t ->
                                                    t = t ) )
13 5,6,8,12 mp   $a |- ( ( t + 0 ) = t -> t = t )
14 1,2,4,13 mp   $a |- t = t
\end{verbatim}

\begin{verbatim}
MM-PA> improve all
A proof of length 1 was found for step 11.
A proof of length 1 was found for step 10.
A proof of length 3 was found for step 9.
A proof of length 1 was found for step 7.
A proof of length 9 was found for step 6.
A proof of length 5 was found for step 5.
A proof of length 1 was found for step 3.
A proof of length 3 was found for step 2.
A proof of length 5 was found for step 1.
Steps 1 and above have been renumbered.
CONGRATULATIONS!  The proof is complete.  Use SAVE
NEW_PROOF to save it.  Note:  The Proof Assistant does
not detect $d violations.  After saving the proof, you
should verify it with VERIFY PROOF.
\end{verbatim}

The \texttt{save new{\char`\_}proof} command\index{\texttt{save
new{\char`\_}proof} command} will save the proof in the database.  Here
we will just display it in a form that can be clipped out of a log file
and inserted manually into the database source file with a text
editor.\index{normal proof}\index{proof!normal}

\begin{verbatim}
MM-PA> show new_proof/normal
---------Clip out the proof below this line:
      tt tze tpl tt weq tt tt weq tt a2 tt tze tpl tt weq
      tt tze tpl tt weq tt tt weq wim tt a2 tt tze tpl tt
      tt a1 mp mp $.
---------The proof of 'th1' to clip out ends above this line.
\end{verbatim}

There is another proof format called ``compressed''\index{compressed
proof}\index{proof!compressed} that you will see in databases.  It is
not important to understand how it is encoded but only to recognize it
when you see it.  Its only purpose is to reduce storage requirements for
large proofs.  A compressed proof can always be converted to a normal
one and vice-versa, and the Metamath \texttt{show proof}
commands\index{\texttt{show proof} command} work equally well with
compressed proofs.  The compressed proof format is described in
Appendix~\ref{compressed}.

\begin{verbatim}
MM-PA> show new_proof/compressed
---------Clip out the proof below this line:
      ( tze tpl weq a2 wim a1 mp ) ABCZADZAADZAEZJJKFLIA
      AGHH $.
---------The proof of 'th1' to clip out ends above this line.
\end{verbatim}

Now we will exit the Proof Assistant.  Since we made changes to the proof,
it will warn us that we have not saved it.  In this case, we don't care.

\begin{verbatim}
MM-PA> exit
Warning:  You have not saved changes to the proof.
Do you want to EXIT anyway (Y, N) <N>? y
Exiting the Proof Assistant.
Type EXIT again to exit Metamath.
\end{verbatim}

The Proof Assistant\index{Proof Assistant} has several other commands
that can help you while creating proofs.  See Section~\ref{pfcommands}
for a list of them.

A command that is often useful is \texttt{minimize{\char`\_}with
*/brief}, which tries to shorten the proof.  It can make the process
more efficient by letting you write a somewhat ``sloppy'' proof then
clean up some of the fine details of optimization for you (although it
can't perform miracles such as restructuring the overall proof).

\section{A Note About Editing a Data\-base File}

Once your source file contains proofs, there are some restrictions on
how you can edit it so that the proofs remain valid.  Pay particular
attention to these rules, since otherwise you can lose a lot of work.
It is a good idea to periodically verify all proofs with \texttt{verify
proof *} to ensure their integrity.

If your file contains only normal (as opposed to compressed) proofs, the
main rule is that you may not change the order of the mandatory
hypotheses\index{mandatory hypothesis} of any statement referenced in a
later proof.  For example, if you swap the order of the major and minor
premise in the modus ponens rule, all proofs making use of that rule
will become incorrect.  The \texttt{show statement}
command\index{\texttt{show statement} command} will show you the
mandatory hypotheses of a statement and their order.

If a statement has a compressed proof, you also must not change the
order of {\em its} mandatory hypotheses.  The compressed proof format
makes use of this information as part of the compression technique.
Note that swapping the names of two variables in a theorem will change
the order of its mandatory hypotheses.

The safest way to edit a statement, say \texttt{mytheorem}, is to
duplicate it then rename the original to \texttt{mytheoremOLD}
throughout the database.  Once the edited version is re-proved, all
statements referencing \texttt{mytheoremOLD} can be updated in the Proof
Assistant using \texttt{minimize{\char`\_}with
mytheorem
/allow{\char`\_}growth}.\index{\texttt{minimize{\char`\_}with} command}
% 3/10/07 Note: line-breaking the above results in duplicate index entries

\chapter{Abstract Mathematics Revealed}\label{fol}

\section{Logic and Set Theory}\label{logicandsettheory}

\begin{quote}
  {\em Set theory can be viewed as a form of exact theology.}
  \flushright\sc  Rudy Rucker\footnote{\cite{Barrow}, p.~31.}\\
\end{quote}\index{Rucker, Rudy}

Despite its seeming complexity, all of standard mathematics, no matter how
deep or abstract, can amazingly enough be derived from a relatively small set
of axioms\index{axiom} or first principles. The development of these axioms is
among the most impressive and important accomplishments of mathematics in the
20th century. Ultimately, these axioms can be broken down into a set of rules
for manipulating symbols that any technically oriented person can follow.

We will not spend much time trying to convey a deep, higher-level
understanding of the meaning of the axioms. This kind of understanding
requires some mathematical sophistication as well as an understanding of the
philosophy underlying the foundations of mathematics and typically develops
over time as you work with mathematics.  Our goal, instead, is to give you the
immediate ability to follow how theorems\index{theorem} are derived from the
axioms and from other theorems.  This will be similar to learning the syntax
of a computer language, which lets you follow the details in a program but
does not necessarily give you the ability to write non-trivial programs on
your own, an ability that comes with practice. For now don't be alarmed by
abstract-sounding names of the axioms; just focus on the rules for
manipulating the symbols, which follow the simple conventions of the
Metamath\index{Metamath} language.

The axioms that underlie all of standard mathematics consist of axioms of logic
and axioms of set theory. The axioms of logic are divided into two
subcategories, propositional calculus\index{propositional calculus} (sometimes
called sentential logic\index{sentential logic}) and predicate calculus
(sometimes called first-order logic\index{first-order logic}\index{quantifier
theory}\index{predicate calculus} or quantifier theory).  Propositional
calculus is a prerequisite for predicate calculus, and predicate calculus is a
prerequisite for set theory.  The version of set theory most commonly used is
Zermelo--Fraenkel set theory\index{Zermelo--Fraenkel set theory}\index{set theory}
with the axiom of choice,
often abbreviated as ZFC\index{ZFC}.

Here in a nutshell is what the axioms are all about in an informal way. The
connection between this description and symbols we will show you won't be
immediately apparent and in principle needn't ever be.  Our description just
tries to summarize what mathematicians think about when they work with the
axioms.

Logic is a set of rules that allow us determine truths given other truths.
Put another way,
logic is more or less the translation of what we would consider common sense
into a rigorous set of axioms.\index{axioms of logic}  Suppose $\varphi$,
$\psi$, and $\chi$ (the Greek letters phi, psi, and chi) represent statements
that are either true or false, and $x$ is a variable\index{variable!in predicate
calculus} ranging over some group of mathematical objects (sets, integers,
real numbers, etc.). In mathematics, a ``statement'' really means a formula,
and $\psi$ could be for example ``$x = 2$.''
Propositional calculus\index{propositional calculus}
allows us to use variables that are either true or false
and make deductions such as
``if $\varphi$ implies $\psi$ and $\psi$ implies $\chi$, then $\varphi$
implies $\chi$.''
Predicate calculus\index{predicate calculus}
extends propositional calculus by also allowing us
to discuss statements about objects (not just true and false values), including
statements about ``all'' or ``at least one'' object.
For example, predicate calculus allows to say,
``if $\varphi$ is true for all $x$, then $\varphi$ is true for some $x$.''
The logic used in \texttt{set.mm} is standard classical logic
(as opposed to other logic systems like intuitionistic logic).

Set theory\index{set theory} has to do with the manipulation of objects and
collections of objects, specifically the abstract, imaginary objects that
mathematics deals with, such as numbers. Everything that is claimed to exist
in mathematics is considered to be a set.  A set called the empty
set\index{empty set} contains nothing.  We represent the empty set by
$\varnothing$.  Many sets can be built up from the empty set.  There is a set
represented by $\{\varnothing\}$ that contains the empty set, another set
represented by $\{\varnothing,\{\varnothing\}\}$ that contains this set as
well as the empty set, another set represented by $\{\{\varnothing\}\}$ that
contains just the set that contains the empty set, and so on ad infinitum. All
mathematical objects, no matter how complex, are defined as being identical to
certain sets: the integer\index{integer} 0 is defined as the empty set, the
integer 1 is defined as $\{\varnothing\}$, the integer 2 is defined as
$\{\varnothing,\{\varnothing\}\}$.  (How these definitions were chosen doesn't
matter now, but the idea behind it is that these sets have the properties we
expect of integers once suitable operations are defined.)  Mathematical
operations, such as addition, are defined in terms of operations on
sets---their union\index{set union}, intersection\index{set intersection}, and
so on---operations you may have used in elementary school when you worked
with groups of apples and oranges.

With a leap of faith, the axioms also postulate the existence of infinite
sets\index{infinite set}, such as the set of all non-negative integers ($0, 1,
2,\ldots$, also called ``natural numbers''\index{natural number}).  This set
can't be represented with the brace notation\index{brace notation} we just
showed you, but requires a more complicated notation called ``class
abstraction.''\index{class abstraction}\index{abstraction class}  For
example, the infinite set $\{ x |
\mbox{``$x$ is a natural number''} \} $ means the ``set of all objects $x$
such that $x$ is a natural number'' i.e.\ the set of natural numbers; here,
``$x$ is a natural number'' is a rather complicated formula when broken down
into the primitive symbols.\label{expandom}\footnote{The statement ``$x$ is a
natural number'' is formally expressed as ``$x \in \omega$,'' where $\in$
(stylized epsilon) means ``is in'' or ``is an element of'' and $\omega$
(omega) means ``the set of natural numbers.''  When ``$x\in\omega$'' is
completely expanded in terms of the primitive symbols of set theory, the
result is  $\lnot$ $($ $\lnot$ $($ $\forall$ $z$ $($ $\lnot$ $\forall$ $w$ $($
$z$ $\in$ $w$ $\rightarrow$ $\lnot$ $w$ $\in$ $x$ $)$ $\rightarrow$ $z$ $\in$
$x$ $)$ $\rightarrow$ $($ $\forall$ $z$ $($ $\lnot$ $($ $\forall$ $w$ $($ $w$
$\in$ $z$ $\rightarrow$ $w$ $\in$ $x$ $)$ $\rightarrow$ $\forall$ $w$ $\lnot$
$w$ $\in$ $z$ $)$ $\rightarrow$ $\lnot$ $\forall$ $w$ $($ $w$ $\in$ $z$
$\rightarrow$ $\lnot$ $\forall$ $v$ $($ $v$ $\in$ $z$ $\rightarrow$ $\lnot$
$v$ $\in$ $w$ $)$ $)$ $)$ $\rightarrow$ $\lnot$ $\forall$ $z$ $\forall$ $w$
$($ $\lnot$ $($ $z$ $\in$ $x$ $\rightarrow$ $\lnot$ $w$ $\in$ $x$ $)$
$\rightarrow$ $($ $\lnot$ $z$ $\in$ $w$ $\rightarrow$ $($ $\lnot$ $z$ $=$ $w$
$\rightarrow$ $w$ $\in$ $z$ $)$ $)$ $)$ $)$ $)$ $\rightarrow$ $\lnot$
$\forall$ $y$ $($ $\lnot$ $($ $\lnot$ $($ $\forall$ $z$ $($ $\lnot$ $\forall$
$w$ $($ $z$ $\in$ $w$ $\rightarrow$ $\lnot$ $w$ $\in$ $y$ $)$ $\rightarrow$
$z$ $\in$ $y$ $)$ $\rightarrow$ $($ $\forall$ $z$ $($ $\lnot$ $($ $\forall$
$w$ $($ $w$ $\in$ $z$ $\rightarrow$ $w$ $\in$ $y$ $)$ $\rightarrow$ $\forall$
$w$ $\lnot$ $w$ $\in$ $z$ $)$ $\rightarrow$ $\lnot$ $\forall$ $w$ $($ $w$
$\in$ $z$ $\rightarrow$ $\lnot$ $\forall$ $v$ $($ $v$ $\in$ $z$ $\rightarrow$
$\lnot$ $v$ $\in$ $w$ $)$ $)$ $)$ $\rightarrow$ $\lnot$ $\forall$ $z$
$\forall$ $w$ $($ $\lnot$ $($ $z$ $\in$ $y$ $\rightarrow$ $\lnot$ $w$ $\in$
$y$ $)$ $\rightarrow$ $($ $\lnot$ $z$ $\in$ $w$ $\rightarrow$ $($ $\lnot$ $z$
$=$ $w$ $\rightarrow$ $w$ $\in$ $z$ $)$ $)$ $)$ $)$ $\rightarrow$ $($
$\forall$ $z$ $\lnot$ $z$ $\in$ $y$ $\rightarrow$ $\lnot$ $\forall$ $w$ $($
$\lnot$ $($ $w$ $\in$ $y$ $\rightarrow$ $\lnot$ $\forall$ $z$ $($ $w$ $\in$
$z$ $\rightarrow$ $\lnot$ $z$ $\in$ $y$ $)$ $)$ $\rightarrow$ $\lnot$ $($
$\lnot$ $\forall$ $z$ $($ $w$ $\in$ $z$ $\rightarrow$ $\lnot$ $z$ $\in$ $y$
$)$ $\rightarrow$ $w$ $\in$ $y$ $)$ $)$ $)$ $)$ $\rightarrow$ $x$ $\in$ $y$
$)$ $)$ $)$. Section~\ref{hierarchy} shows the hierarchy of definitions that
leads up to this expression.}\index{stylized epsilon ($\in$)}\index{omega
($\omega$)}  Actually, the primitive symbols don't even include the brace
notation.  The brace notation is a high-level definition, which you can find in
Section~\ref{hierarchy}.

Interestingly, the arithmetic of integers\index{integer} and
rationals\index{rational number} can be developed without appealing to the
existence of an infinite set, whereas the arithmetic of real
numbers\index{real number} requires it.

Each variable\index{variable!in set theory} in the axioms of set theory
represents an arbitrary set, and the axioms specify the legal kinds of things
you can do with these variables at a very primitive level.

Now, you may think that numbers and arithmetic are a lot more intuitive and
fundamental than sets and therefore should be the foundation of mathematics.
What is really the case is that you've dealt with numbers all your life and
are comfortable with a few rules for manipulating them such as addition and
multiplication.  Those rules only cover a small portion of what can be done
with numbers and only a very tiny fraction of the rest of mathematics.  If you
look at any elementary book on number theory, you will quickly become lost if
these are the only rules that you know.  Even though such books may present a
list of ``axioms''\index{axiom} for arithmetic, the ability to use the axioms
and to understand proofs of theorems\index{theorem} (facts) about numbers
requires an implicit mathematical talent that frustrates many people
from studying abstract mathematics.  The kind of mathematics that most people
know limits them to the practical, everyday usage of blindly manipulating
numbers and formulas, without any understanding of why those rules are correct
nor any ability to go any further.  For example, do you know why multiplying
two negative numbers yields a positive number?  Starting with set theory, you
will also start off blindly manipulating symbols according to the rules we give
you, but with the advantage that these rules will allow you, in principle, to
access {\em all} of mathematics, not just a tiny part of it.

Of course, concrete examples are often helpful in the learning process. For
example, you can verify that $2\cdot 3=3 \cdot 2$ by actually grouping
objects and can easily ``see'' how it generalizes to $x\cdot y = y\cdot x$,
even though you might not be able to rigorously prove it.  Similarly, in set
theory it can be helpful to understand how the axioms of set theory apply to
(and are correct for) small finite collections of objects.  You should be aware
that in set theory intuition can be misleading for infinite collections, and
rigorous proofs become more important.  For example, while $x\cdot y = y\cdot
x$ is correct for finite ordinals (which are the natural numbers), it is not
usually true for infinite ordinals.

\section{The Axioms for All of Mathematics}

In this section\index{axioms for mathematics}, we will show you the axioms
for all of standard mathematics (i.e.\ logic and set theory) as they are
traditionally presented.  The traditional presentation is useful for someone
with the mathematical experience needed to correctly manipulate high-level
abstract concepts.  For someone without this talent, knowing how to actually
make use of these axioms can be difficult.  The purpose of this section is to
allow you to see how the version of the axioms used in the standard
Metamath\index{Metamath} database \texttt{set.mm}\index{set
theory database (\texttt{set.mm})} relates to  the typical version
in textbooks, and also to give you an informal feel for them.

\subsection{Propositional Calculus}

Propositional calculus\index{propositional calculus} concerns itself with
statements that can be interpreted as either true or false.  Some examples of
statements (outside of mathematics) that are either true or false are ``It is
raining today'' and ``The United States has a female president.'' In
mathematics, as we mentioned, statements are really formulas.

In propositional calculus, we don't care what the statements are.  We also
treat a logical combination of statements, such as ``It is raining today and
the United States has a female president,'' no differently from a single
statement.  Statements and their combinations are called well-formed formulas
(wffs)\index{well-formed formula (wff)}.  We define wffs only in terms of
other wffs and don't define what a ``starting'' wff is.  As is common practice
in the literature, we use Greek letters to represent wffs.

Specifically, suppose $\varphi$ and $\psi$ are wffs.  Then the combinations
$\varphi\rightarrow\psi$ (``$\varphi$ implies $\psi$,'' also read ``if
$\varphi$ then $\psi$'')\index{implication ($\rightarrow$)} and $\lnot\varphi$
(``not $\varphi$'')\index{negation ($\lnot$)} are also wffs.

The three axioms of propositional calculus\index{axioms of propositional
calculus} are all wffs of the following form:\footnote{A remarkable result of
C.~A.~Meredith\index{Meredith, C. A.} squeezes these three axioms into the
single axiom $((((\varphi\rightarrow \psi)\rightarrow(\neg \chi\rightarrow\neg
\theta))\rightarrow \chi )\rightarrow \tau)\rightarrow((\tau\rightarrow
\varphi)\rightarrow(\theta\rightarrow \varphi))$ \cite{CAMeredith},
which is believed to be the shortest possible.}
\begin{center}
     $\varphi\rightarrow(\psi\rightarrow \varphi)$\\

     $(\varphi\rightarrow (\psi\rightarrow \chi))\rightarrow
((\varphi\rightarrow  \psi)\rightarrow (\varphi\rightarrow \chi))$\\

     $(\neg \varphi\rightarrow \neg\psi)\rightarrow (\psi\rightarrow
\varphi)$
\end{center}

These three axioms are widely used.
They are attributed to Jan {\L}ukasiewicz
(pronounced woo-kah-SHAY-vitch) and was popularized by Alonzo Church,
who called it system P2. (Thanks to Ted Ulrich for this information.)

There are an infinite number of axioms, one for each possible
wff\index{well-formed formula (wff)} of the above form.  (For this reason,
axioms such as the above are often called ``axiom schemes.''\index{axiom
scheme})  Each Greek letter in the axioms may be substituted with a more
complex wff to result in another axiom.  For example, substituting
$\neg(\varphi\rightarrow\chi)$ for $\varphi$ in the first axiom yields
$\neg(\varphi\rightarrow\chi)\rightarrow(\psi\rightarrow
\neg(\varphi\rightarrow\chi))$, which is still an axiom.

To deduce new true statements (theorems\index{theorem}) from the axioms, a
rule\index{rule} called ``modus ponens''\index{modus ponens} is used.  This
rule states that if the wff $\varphi$ is an axiom or a theorem, and the wff
$\varphi\rightarrow\psi$ is an axiom or a theorem, then the wff $\psi$ is also
a theorem\index{theorem}.

As a non-mathematical example of modus ponens, suppose we have proved (or
taken as an axiom) ``Bob is a man'' and separately have proved (or taken as
an axiom) ``If Bob is a man, then Bob is a human.''  Using the rule of modus
ponens, we can logically deduce, ``Bob is a human.''

From Metamath's\index{Metamath} point of view, the axioms and the rule of
modus ponens just define a mechanical means for deducing new true statements
from existing true statements, and that is the complete content of
propositional calculus as far as Metamath is concerned.  You can read a logic
textbook to gain a better understanding of their meaning, or you can just let
their meaning slowly become apparent to you after you use them for a while.

It is actually rather easy to check to see if a formula is a theorem of
propositional calculus.  Theorems of propositional calculus are also called
``tautologies.''\index{tautology}  The technique to check whether a formula is
a tautology is called the ``truth table method,''\index{truth table} and it
works like this.  A wff $\varphi\rightarrow\psi$ is false whenever $\varphi$ is true
and $\psi$ is false.  Otherwise it is true.  A wff $\lnot\varphi$ is false
whenever $\varphi$ is true and false otherwise. To verify a tautology such as
$\varphi\rightarrow(\psi\rightarrow \varphi)$, you break it down into sub-wffs and
construct a truth table that accounts for all possible combinations of true
and false assigned to the wff metavariables:
\begin{center}\begin{tabular}{|c|c|c|c|}\hline
\mbox{$\varphi$} & \mbox{$\psi$} & \mbox{$\psi\rightarrow\varphi$}
    & \mbox{$\varphi\rightarrow(\psi\rightarrow \varphi)$} \\ \hline \hline
              T   &  T    &      T       &        T    \\ \hline
              T   &  F    &      T       &        T    \\ \hline
              F   &  T    &      F       &        T    \\ \hline
              F   &  F    &      T       &        T    \\ \hline
\end{tabular}\end{center}
If all entries in the last column are true, the formula is a tautology.

Now, the truth table method doesn't tell you how to prove the tautology from
the axioms, but only that a proof exists.  Finding an actual proof (especially
one that is short and elegant) can be challenging.  Methods do exist for
automatically generating proofs in propositional calculus, but the proofs that
result can sometimes be very long.  In the Metamath \texttt{set.mm}\index{set
theory database (\texttt{set.mm})} database, most
or all proofs were created manually.

Section \ref{metadefprop} discusses various definitions
that make propositional calculus easier to use.
For example, we define:

\begin{itemize}
\item $\varphi \vee \psi$
  is true if either $\varphi$ or $\psi$ (or both) are true
  (this is disjunction\index{disjunction ($\vee$)}
  aka logical {\sc or}\index{logical {\sc or} ($\vee$)}).

\item $\varphi \wedge \psi$
  is true if both $\varphi$ and $\psi$ are true
  (this is conjunction\index{conjunction ($\wedge$)}
  aka logical {\sc and}\index{logical {\sc and} ($\wedge$)}).

\item $\varphi \leftrightarrow \psi$
  is true if $\varphi$ and $\psi$ have the same value, that is,
  they are both true or both false
  (this is the biconditional\index{biconditional ($\leftrightarrow$)}).
\end{itemize}

\subsection{Predicate Calculus}

Predicate calculus\index{predicate calculus} introduces the concept of
``individual variables,''\index{variable!in predicate calculus}\index{individual
variable} which
we will usually just call ``variables.''
These variables can represent something other than true or false (wffs),
and will always represent sets when we get to set theory.  There are also
three new symbols $\forall$\index{universal quantifier ($\forall$)},
$=$\index{equality ($=$)}, and $\in$\index{stylized epsilon ($\in$)},
read ``for all,'' ``equals,'' and ``is an element of''
respectively.  We will represent variables with the letters $x$, $y$, $z$, and
$w$, as is common practice in the literature.
For example, $\forall x \varphi$ means ``for all possible values of
$x$, $\varphi$ is true.''

In predicate calculus, we extend the definition of a wff\index{well-formed
formula (wff)}.  If $\varphi$ is a wff and $x$ and $y$ are variables, then
$\forall x \, \varphi$, $x=y$, and $x\in y$ are wffs. Note that these three new
types of wffs can be considered ``starting'' wffs from which we can build
other wffs with $\rightarrow$ and $\neg$ .  The concept of a starting wff was
absent in propositional calculus.  But starting wff or not, all we are really
concerned with is whether our wffs are correctly constructed according to
these mechanical rules.

A quick aside:
To prevent confusion, it might be best at this point to think of the variables
of Metamath\index{Metamath} as ``metavariables,''\index{metavariable} because
they are not quite the same as the variables we are introducing here.  A
(meta)variable in Metamath can be a wff or an individual variable, as well
as many other things; in general, it represents a kind of place holder for an
unspecified sequence of math symbols\index{math symbol}.

Unlike propositional calculus, no decision procedure\index{decision procedure}
analogous to the truth table method exists (nor theoretically can exist) that
will definitely determine whether a formula is a theorem of predicate
calculus.  Much of the work in the field of automated theorem
proving\index{automated theorem proving} has been dedicated to coming up with
clever heuristics for proving theorems of predicate calculus, but they can
never be guaranteed to work always.

Section \ref{metadefpred} discusses various definitions
that make predicate calculus easier to use.
For example, we define
$\exists x \varphi$ to mean
``there exists at least one possible value of $x$ where $\varphi$ is true.''

We now turn to looking at how predicate calculus can be formally
represented.

\subsubsection{Common Axioms}

There is a new rule of inference in predicate calculus:  if $\varphi$ is
an axiom or a theorem, then $\forall x \,\varphi$ is also a
theorem\index{theorem}.  This is called the rule of
``generalization.''\index{rule of generalization}
This is easily represented in Metamath.

In standard texts of logic, there are often two axioms of predicate
calculus\index{axioms of predicate calculus}:
\begin{center}
  $\forall x \,\varphi ( x ) \rightarrow \varphi ( y )$,
      where ``$y$ is properly substituted for $x$.''\\
  $\forall x ( \varphi \rightarrow \psi )\rightarrow ( \varphi \rightarrow
    \forall x\, \psi )$,
    where ``$x$ is not free in $\varphi$.''
\end{center}

Now at first glance, this seems simple:  just two axioms.  However,
conditional clauses are attached to each axiom describing requirements that
may seem puzzling to you.  In addition, the first axiom puts a variable symbol
in parentheses after each wff, seemingly violating our definition of a
wff\index{well-formed formula (wff)}; this is just an informal way of
referring to some arbitrary variable that may occur in the wff.  The
conditional clauses do, of course, have a precise meaning, but as it turns out
the precise meaning is somewhat complicated and awkward to formalize in a
way that a computer can handle easily.  Unlike propositional calculus, a
certain amount of mathematical sophistication and practice is needed to be
able to easily grasp and manipulate these concepts correctly.

Predicate calculus may be presented with or without axioms for
equality\index{axioms of equality}\index{equality ($=$)}. We will require the
axioms of equality as a prerequisite for the version of set theory we will
use.  The axioms for equality, when included, are often represented using these
two axioms:
\begin{center}
$x=x$\\ \ \\
$x=y\rightarrow (\varphi(x,x)\rightarrow\varphi(x,y))$ where ``$\varphi(x,y)$
   arises from $\varphi(x,x)$ by replacing some, but not necessarily all,
   free\index{free variable}
   occurrences of $x$ by $y$,\\ provided that $y$ is free for $x$
   in $\varphi(x,x)$.'' \end{center}
% (Mendelson p. 95)
The first equality axiom is simple, but again,
the condition on the second one is
somewhat awkward to implement on a computer.

\subsubsection{Tarski System S2}

Of course, we are not the first to notice the complications of these
predicate calculus axioms when being rigorous.

Well-known logician Alfred Tarski published in 1965
a system he called system S2\cite[p.~77]{Tarski1965}.
Tarski's system is \textit{exactly equivalent} to the traditional textbook
formalization, but (by clever use of equality axioms) it eliminates the
latter's primitive notions of ``proper substitution'' and ``free variable,''
replacing them with direct substitution and the notion of a variable
not occurring in a formula (which we express with distinct variable
constraints).

In advocating his system, Tarski wrote, ``The relatively complicated
character of [free variables and proper substitution] is a source
of certain inconveniences of both practical and theoretical nature;
this is clearly experienced both in teaching an elementary course of
mathematical logic and in formalizing the syntax of predicate logic for
some theoretical purposes''\cite[p.~61]{Tarski1965}\index{Tarski, Alfred}.

\subsubsection{Developing a Metamath Representation}

The standard textbook axioms of predicate calculus are somewhat
cumbersome to implement on a computer because of the complex notions of
``free variable''\index{free variable} and ``proper
substitution.''\index{proper substitution}\index{substitution!proper}
While it is possible to use the Metamath\index{Metamath} language to
implement these concepts, we have chosen not to implement them
as primitive constructs in the
\texttt{set.mm} set theory database.  Instead, we have eliminated them
within the axioms
by carefully crafting the axioms so as to avoid them,
building on Tarski's system S2.  This makes it
easy for a beginner to follow the steps in a proof without knowing any
advanced concepts other than the simple concept of
replacing\index{substitution!variable}\index{variable substitution}
variables with expressions.

In order to develop the concepts of free variable and proper
substitution from the axioms, we use an additional
Metamath statement type called ``disjoint variable
restriction''\index{disjoint variables} that we have not encountered
before.  In the context of the axioms, the statement \texttt{\$d} $ x\,
y$\index{\texttt{\$d} statement} simply means that $x$ and $y$ must be
distinct\index{distinct variables}, i.e.\ they may not be simultaneously
substituted\index{substitution!variable}\index{variable substitution}
with the same variable.  The statement \texttt{\$d} $ x\, \varphi$ means
variable $x$ must not occur in wff $\varphi$.  For the precise
definition of \texttt{\$d}, see Section~\ref{dollard}.

\subsubsection{Metamath representation}

The Metamath axiom system for predicate calculus
defined in set.mm uses Tarski's system S2.
As noted above, this has a different representation
than the traditional textbook formalization,
but it is \textit{exactly equivalent} to the textbook formalization,
and it is \textit{much} easier to work with.
This is reproduced as system S3 in Section 6 of
Megill's formalization \cite{Megill}\index{Megill, Norman}.

There is one exception, Tarski's axiom of existence,
which we label as axiom ax-6.
In the case of ax-6, Tarski's version is weaker because it includes a
distinct variable proviso. If we wish, we can also weaken our version
in this way and still have a metalogically complete system. Theorem
ax6 shows this by deriving, in the presence of the other axioms, our
ax-6 from Tarski's weaker version ax6v. However, we chose the stronger
version for our system because it is simpler to state and easier to use.

Tarski's system was designed for proving specific theorems rather than
more general theorem schemes. However, theorem schemes are much more
efficient than specific theorems for building a body of mathematical
knowledge, since they can be reused with different instances as
needed. While Tarski does derive some theorem schemes from his axioms,
their proofs require concepts that are ``outside'' of the system, such as
induction on formula length. The verification of such proofs is difficult
to automate in a proof verifier. (Specifically, Tarski treats the formulas
of his system as set-theoretical objects. In order to verify the proofs
of his theorem schemes, a proof verifier would need a significant amount
of set theory built into it.)

The Metamath axiom system for predicate calculus extends
Tarski's system to eliminate this difficulty. The additional
``auxilliary'' axiom
schemes (as we will call them in this section; see below) endow Tarski's
system with a nice property we call
metalogical completeness \cite[Remark 9.6]{Megill}\index{Megill, Norman}.
As a result, we can prove any theorem scheme
expressable in the ``simple metalogic'' of Tarski's system by using
only Metamath's direct substitution rule applied to the axiom system
(and no other metalogical or set-theoretical notions ``outside'' of the
system). Simple metalogic consists of schemes containing wff metavariables
(with no arguments) and/or set (also called ``individual'') metavariables,
accompanied by optional provisos each stating that two specified set
metavariables must be distinct or that a specified set metavariable may
not occur in a specified wff metavariable. Metamath's logic and set theory
axiom and rule schemes are all examples of simple metalogic. The schemes
of traditional predicate calculus with equality are examples which are
not simple metalogic, because they use wff metavariables with arguments
and have ``free for'' and ``not free in'' side conditions.

A rigorous justification for this system, using an older but
exactly equivalent set of axioms, can be
found in \cite{Megill}\index{Megill, Norman}.

This allows us to
take a different approach in the Metamath\index{Metamath} database
\texttt{set.mm}\index{set theory database (\texttt{set.mm})}.  We do not
directly use the primitive notions of ``free variable''\index{free variable}
and ``proper substitution''\index{proper
substitution}\index{substitution!proper} at all as primitive constructs.
Instead, we use a set
of axioms that are almost as simple to manipulate as those of
propositional calculus.  Our axiom system avoids complex primitive
notions by effectively embedding the complexity into the axioms
themselves.  As a result, we will end up with a larger number of axioms,
but they are ideally suited for a computer language such as Metamath.
(Section~\ref{metaaxioms} shows these axioms.)

We will not elaborate further
on the ``free variable'' and ``proper substitution''
concepts here.  You may consult
\cite[ch.\ 3--4]{Hamilton}\index{Hamilton, Alan G.} (as well as
many other books) for a precise explanation
of these concepts.  If you intend to do serious mathematical work, it is wise
to become familiar with the traditional textbook approach; even though the
concepts embedded in their axioms require a higher level of sophistication,
they can be more practical to deal with on an everyday, informal basis.  Even
if you are just developing Metamath proofs, familiarity with the traditional
approach can help you arrive at a proof outline much faster, which you can
then convert to the detail required by Metamath.

We do develop proper substitution rules later on, but in set.mm
they are defined as derived constructs; they are not primitives.

You should also note that our system of predicate calculus is specifically
tailored for set theory; thus there are only two specific predicates $=$ and
$\in$ and no functions\index{function!in predicate calculus}
or constants\index{constant!in predicate calculus} unlike more general systems.
We later add these.

\subsection{Set Theory}

Traditional Zermelo--Fraenkel set theory\index{Zermelo--Fraenkel set
theory}\index{set theory} with the Axiom of Choice
has 10 axioms, which can be expressed in the
language of predicate calculus.  In this section, we will list only the
names and brief English descriptions of these axioms, since we will give
you the precise formulas used by the Metamath\index{Metamath} set theory
database \texttt{set.mm} later on.

In the descriptions of the axioms, we assume that $x$, $y$, $z$, $w$, and $v$
represent sets.  These are the same as the variables\index{variable!in set
theory} in our predicate calculus system above, except that now we informally
think of the variables as ranging over sets.  Note that the terms
``object,''\index{object} ``set,''\index{set} ``element,''\index{element}
``collection,''\index{collection} and ``family''\index{family} are synonymous,
as are ``is an element of,'' ``is a member of,''\index{member} ``is contained
in,'' and ``belongs to.''  The different terms are used for convenience; for
example, ``a collection of sets'' is less confusing than ``a set of sets.''
A set $x$ is said to be a ``subset''\index{subset} of $y$ if every element of
$x$ is also an element of $y$; we also say $x$ is ``included in''
$y$.

The axioms are very general and apply to almost any conceivable mathematical
object, and this level of abstraction can be overwhelming at first.  To gain an
intuitive feel, it can be helpful to draw a picture illustrating the concept;
for example, a circle containing dots could represent a collection of sets,
and a smaller circle drawn inside the circle could represent a subset.
Overlapping circles can illustrate intersection and union.  Circles that
illustrate the concepts of set theory are frequently used in elementary
textbooks and are called Venn diagrams\index{Venn diagram}.\index{axioms of
set theory}

1. Axiom of Extensionality:  Two sets are identical if they contain the same
   elements.\index{Axiom of Extensionality}

2. Axiom of Pairing:  The set $\{ x , y \}$ exists.\index{Axiom of Pairing}

3. Axiom of Power Sets:  The power set of a set (the collection of all of
   its subsets) exists.  For example, the power set of $\{x,y\}$ is
   $\{\varnothing,\{x\},\{y\},\{x,y\}\}$ and it exists.\index{Axiom
of Power Sets}

4. Axiom of the Null Set:  The empty set $\varnothing$ exists.\index{Axiom of
the Null Set}

5. Axiom of Union:  The union of a set (the set containing the elements of
   its members) exists.  For example, the union of $\{\{x,y\},\{z\}\}$ is
 $\{x,y,z\}$ and
   it exists.\index{Axiom of Union}

6. Axiom of Regularity:  Roughly, no set can contain itself, nor can there
   be membership ``loops,'' such as a set being an
   element of one of its members.\index{Axiom of Regularity}

7. Axiom of Infinity:  An infinite set exists.  An example of an infinite
   set is the set of all
   integers.\index{Axiom of Infinity}

8. Axiom of Separation:  The set exists that is obtained by restricting $x$
   with some property.  For example, if the set of all integers exists,
   then the set of all even integers exists.\index{Axiom of Separation}

9. Axiom of Replacement:  The range of a function whose domain is restricted
   to the elements of a set $x$, is also a set.  For example, there
   is a function
   from integers (the function's domain) to their squares (its
   range).  If we
   restrict the domain to even integers, its range will become the set of
   squares of even integers, so this axiom asserts that the set of
    squares of even numbers exists.  Technical note:  In general, the
   ``function'' need not be a set but can be a proper class.
   \index{Axiom of Replacement}

10. Axiom of Choice:  Let $x$ be a set whose members are pairwise
  disjoint\index{disjoint sets} (i.e,
  whose members contain no elements in common).  Then there exists another
  set containing one element from each member of $x$.  For
  example, if $x$ is
  $\{\{y,z\},\{w,v\}\}$, where $y$, $z$, $w$, and $v$ are
  different sets, then a set such as $\{z,w\}$
  exists (but the axiom doesn't tell
  us which one).  (Actually the Axiom
  of Choice is redundant if the set $x$, as in this example, has a finite
  number of elements.)\index{Axiom of Choice}

The Axiom of Choice is usually considered an extension of ZF set theory rather
than a proper part of it.  It is sometimes considered philosophically
controversial because it specifies the existence of a set without specifying
what the set is. Constructive logics, including intuitionistic logic,
do not accept the axiom of choice.
Since there is some lingering controversy, we often prefer proofs that do
not use the axiom of choice (where there is a known alternative), and
in some cases we will use weaker axioms than the full axiom of choice.
That said, the axiom of choice is a powerful and widely-accepted tool,
so we do use it when needed.
ZF set theory that includes the Axiom of Choice is
called Zermelo--Fraenkel set theory with choice (ZFC\index{ZFC set theory}).

When expressed symbolically, the Axiom of Separation and the Axiom of
Replacement contain wff symbols and therefore each represent infinitely many
axioms, one for each possible wff. For this reason, they are often called
axiom schemes\index{axiom scheme}\index{well-formed formula (wff)}.

It turns out that the Axiom of the Null Set, the Axiom of Pairing, and the
Axiom of Separation can be derived from the other axioms and are therefore
unnecessary, although they tend to be included in standard texts for various
reasons (historical, philosophical, and possibly because some authors may not
know this).  In the Metamath\index{Metamath} set theory database, these
redundant axioms are derived from the other ones instead of truly
being considered axioms.
This is in keeping with our general goal of minimizing the number of
axioms we must depend on.

\subsection{Other Axioms}

Above we qualified the phrase ``all of mathematics'' with ``essentially.''
The main important missing piece is the ability to do category theory,
which requires huge sets (inaccessible cardinals) larger than those
postulated by the ZFC axioms. The Tarski--Grothendieck Axiom postulates
the existence of such sets.
Note that this is the same axiom used by Mizar for supporting
category theory.
The Tarski--Grothendieck axiom
can be viewed as a very strong replacement of the Axiom of Infinity,
the Axiom of Choice, and the Axiom of Power Sets.
The \texttt{set.mm} database includes this axiom; see the database
for details about it.
Again, we only use this axiom when we need to.
You are only likely to encounter or use this axiom if you are doing
category theory, since its use is highly specialized,
so we will not list the Tarsky-Grothendieck axiom
in the short list of axioms below.

Can there be even more axioms?
Of course.
G\"{o}del showed that no finite set of axioms or axiom schemes can completely
describe any consistent theory strong enough to include arithmetic.
But practically speaking, the ones above are the accepted foundation that
almost all mathematicians explicitly or implicitly base their work on.

\section{The Axioms in the Metamath Language}\label{metaaxioms}

Here we list the axioms as they appear in
\texttt{set.mm}\index{set theory database (\texttt{set.mm})} so you can
look them up there easily.  Incidentally, the \texttt{show statement
/tex} command\index{\texttt{show statement} command} was used to
typeset them.

%macros from show statement /tex
\newbox\mlinebox
\newbox\mtrialbox
\newbox\startprefix  % Prefix for first line of a formula
\newbox\contprefix  % Prefix for continuation line of a formula
\def\startm{  % Initialize formula line
  \setbox\mlinebox=\hbox{\unhcopy\startprefix}
}
\def\m#1{  % Add a symbol to the formula
  \setbox\mtrialbox=\hbox{\unhcopy\mlinebox $\,#1$}
  \ifdim\wd\mtrialbox>\hsize
    \box\mlinebox
    \setbox\mlinebox=\hbox{\unhcopy\contprefix $\,#1$}
  \else
    \setbox\mlinebox=\hbox{\unhbox\mtrialbox}
  \fi
}
\def\endm{  % Output the last line of a formula
  \box\mlinebox
}

% \SLASH for \ , \TOR for \/ (text OR), \TAND for /\ (text and)
% This embeds a following forced space to force the space.
\newcommand\SLASH{\char`\\~}
\newcommand\TOR{\char`\\/~}
\newcommand\TAND{/\char`\\~}
%
% Macro to output metamath raw text.
% This assumes \startprefix and \contprefix are set.
% NOTE: "\" is tricky to escape, use \SLASH, \TOR, and \TAND inside.
% Any use of "$ { ~ ^" must be escaped; ~ and ^ must be escaped specially.
% We escape { and } for consistency.
% For more about how this macro written, see:
% https://stackoverflow.com/questions/4073674/
% how-to-disable-indentation-in-particular-section-in-latex/4075706
% Use frenchspacing, or "e." will get an extra space after it.
\newlength\mystoreparindent
\newlength\mystorehangindent
\newenvironment{mmraw}{%
\setlength{\mystoreparindent}{\the\parindent}
\setlength{\mystorehangindent}{\the\hangindent}
\setlength{\parindent}{0pt} % TODO - we'll put in the \startprefix instead
\setlength{\hangindent}{\wd\the\contprefix}
\begin{flushleft}
\begin{frenchspacing}
\begin{tt}
{\unhcopy\startprefix}%
}{%
\end{tt}
\end{frenchspacing}
\end{flushleft}
\setlength{\parindent}{\mystoreparindent}
\setlength{\hangindent}{\mystorehangindent}
\vskip 1ex
}

\needspace{5\baselineskip}
\subsection{Propositional Calculus}\label{propcalc}\index{axioms of
propositional calculus}

\needspace{2\baselineskip}
Axiom of Simplification.\label{ax1}

\setbox\startprefix=\hbox{\tt \ \ ax-1\ \$a\ }
\setbox\contprefix=\hbox{\tt \ \ \ \ \ \ \ \ \ \ }
\startm
\m{\vdash}\m{(}\m{\varphi}\m{\rightarrow}\m{(}\m{\psi}\m{\rightarrow}\m{\varphi}\m{)}
\m{)}
\endm

\needspace{3\baselineskip}
\noindent Axiom of Distribution.

\setbox\startprefix=\hbox{\tt \ \ ax-2\ \$a\ }
\setbox\contprefix=\hbox{\tt \ \ \ \ \ \ \ \ \ \ }
\startm
\m{\vdash}\m{(}\m{(}\m{\varphi}\m{\rightarrow}\m{(}\m{\psi}\m{\rightarrow}\m{\chi}
\m{)}\m{)}\m{\rightarrow}\m{(}\m{(}\m{\varphi}\m{\rightarrow}\m{\psi}\m{)}\m{
\rightarrow}\m{(}\m{\varphi}\m{\rightarrow}\m{\chi}\m{)}\m{)}\m{)}
\endm

\needspace{2\baselineskip}
\noindent Axiom of Contraposition.

\setbox\startprefix=\hbox{\tt \ \ ax-3\ \$a\ }
\setbox\contprefix=\hbox{\tt \ \ \ \ \ \ \ \ \ \ }
\startm
\m{\vdash}\m{(}\m{(}\m{\lnot}\m{\varphi}\m{\rightarrow}\m{\lnot}\m{\psi}\m{)}\m{
\rightarrow}\m{(}\m{\psi}\m{\rightarrow}\m{\varphi}\m{)}\m{)}
\endm


\needspace{4\baselineskip}
\noindent Rule of Modus Ponens.\label{axmp}\index{modus ponens}

\setbox\startprefix=\hbox{\tt \ \ min\ \$e\ }
\setbox\contprefix=\hbox{\tt \ \ \ \ \ \ \ \ \ }
\startm
\m{\vdash}\m{\varphi}
\endm

\setbox\startprefix=\hbox{\tt \ \ maj\ \$e\ }
\setbox\contprefix=\hbox{\tt \ \ \ \ \ \ \ \ \ }
\startm
\m{\vdash}\m{(}\m{\varphi}\m{\rightarrow}\m{\psi}\m{)}
\endm

\setbox\startprefix=\hbox{\tt \ \ ax-mp\ \$a\ }
\setbox\contprefix=\hbox{\tt \ \ \ \ \ \ \ \ \ \ \ }
\startm
\m{\vdash}\m{\psi}
\endm


\needspace{7\baselineskip}
\subsection{Axioms of Predicate Calculus with Equality---Tarski's S2}\index{axioms of predicate calculus}

\needspace{3\baselineskip}
\noindent Rule of Generalization.\index{rule of generalization}

\setbox\startprefix=\hbox{\tt \ \ ax-g.1\ \$e\ }
\setbox\contprefix=\hbox{\tt \ \ \ \ \ \ \ \ \ \ \ \ }
\startm
\m{\vdash}\m{\varphi}
\endm

\setbox\startprefix=\hbox{\tt \ \ ax-gen\ \$a\ }
\setbox\contprefix=\hbox{\tt \ \ \ \ \ \ \ \ \ \ \ \ }
\startm
\m{\vdash}\m{\forall}\m{x}\m{\varphi}
\endm

\needspace{2\baselineskip}
\noindent Axiom of Quantified Implication.

\setbox\startprefix=\hbox{\tt \ \ ax-4\ \$a\ }
\setbox\contprefix=\hbox{\tt \ \ \ \ \ \ \ \ \ \ }
\startm
\m{\vdash}\m{(}\m{\forall}\m{x}\m{(}\m{\forall}\m{x}\m{\varphi}\m{\rightarrow}\m{
\psi}\m{)}\m{\rightarrow}\m{(}\m{\forall}\m{x}\m{\varphi}\m{\rightarrow}\m{
\forall}\m{x}\m{\psi}\m{)}\m{)}
\endm

\needspace{3\baselineskip}
\noindent Axiom of Distinctness.

% Aka: Add $d x ph $.
\setbox\startprefix=\hbox{\tt \ \ ax-5\ \$a\ }
\setbox\contprefix=\hbox{\tt \ \ \ \ \ \ \ \ \ \ }
\startm
\m{\vdash}\m{(}\m{\varphi}\m{\rightarrow}\m{\forall}\m{x}\m{\varphi}\m{)}\m{where}\m{ }\m{\$d}\m{ }\m{x}\m{ }\m{\varphi}\m{ }\m{(}\m{x}\m{ }\m{does}\m{ }\m{not}\m{ }\m{occur}\m{ }\m{in}\m{ }\m{\varphi}\m{)}
\endm

\needspace{2\baselineskip}
\noindent Axiom of Existence.

\setbox\startprefix=\hbox{\tt \ \ ax-6\ \$a\ }
\setbox\contprefix=\hbox{\tt \ \ \ \ \ \ \ \ \ \ }
\startm
\m{\vdash}\m{(}\m{\forall}\m{x}\m{(}\m{x}\m{=}\m{y}\m{\rightarrow}\m{\forall}
\m{x}\m{\varphi}\m{)}\m{\rightarrow}\m{\varphi}\m{)}
\endm

\needspace{2\baselineskip}
\noindent Axiom of Equality.

\setbox\startprefix=\hbox{\tt \ \ ax-7\ \$a\ }
\setbox\contprefix=\hbox{\tt \ \ \ \ \ \ \ \ \ \ }
\startm
\m{\vdash}\m{(}\m{x}\m{=}\m{y}\m{\rightarrow}\m{(}\m{x}\m{=}\m{z}\m{
\rightarrow}\m{y}\m{=}\m{z}\m{)}\m{)}
\endm

\needspace{2\baselineskip}
\noindent Axiom of Left Equality for Binary Predicate.

\setbox\startprefix=\hbox{\tt \ \ ax-8\ \$a\ }
\setbox\contprefix=\hbox{\tt \ \ \ \ \ \ \ \ \ \ \ }
\startm
\m{\vdash}\m{(}\m{x}\m{=}\m{y}\m{\rightarrow}\m{(}\m{x}\m{\in}\m{z}\m{
\rightarrow}\m{y}\m{\in}\m{z}\m{)}\m{)}
\endm

\needspace{2\baselineskip}
\noindent Axiom of Right Equality for Binary Predicate.

\setbox\startprefix=\hbox{\tt \ \ ax-9\ \$a\ }
\setbox\contprefix=\hbox{\tt \ \ \ \ \ \ \ \ \ \ \ }
\startm
\m{\vdash}\m{(}\m{x}\m{=}\m{y}\m{\rightarrow}\m{(}\m{z}\m{\in}\m{x}\m{
\rightarrow}\m{z}\m{\in}\m{y}\m{)}\m{)}
\endm


\needspace{4\baselineskip}
\subsection{Axioms of Predicate Calculus with Equality---Auxiliary}\index{axioms of predicate calculus - auxiliary}

\needspace{2\baselineskip}
\noindent Axiom of Quantified Negation.

\setbox\startprefix=\hbox{\tt \ \ ax-10\ \$a\ }
\setbox\contprefix=\hbox{\tt \ \ \ \ \ \ \ \ \ \ }
\startm
\m{\vdash}\m{(}\m{\lnot}\m{\forall}\m{x}\m{\lnot}\m{\forall}\m{x}\m{\varphi}\m{
\rightarrow}\m{\varphi}\m{)}
\endm

\needspace{2\baselineskip}
\noindent Axiom of Quantifier Commutation.

\setbox\startprefix=\hbox{\tt \ \ ax-11\ \$a\ }
\setbox\contprefix=\hbox{\tt \ \ \ \ \ \ \ \ \ \ }
\startm
\m{\vdash}\m{(}\m{\forall}\m{x}\m{\forall}\m{y}\m{\varphi}\m{\rightarrow}\m{
\forall}\m{y}\m{\forall}\m{x}\m{\varphi}\m{)}
\endm

\needspace{3\baselineskip}
\noindent Axiom of Substitution.

\setbox\startprefix=\hbox{\tt \ \ ax-12\ \$a\ }
\setbox\contprefix=\hbox{\tt \ \ \ \ \ \ \ \ \ \ \ }
\startm
\m{\vdash}\m{(}\m{\lnot}\m{\forall}\m{x}\m{\,x}\m{=}\m{y}\m{\rightarrow}\m{(}
\m{x}\m{=}\m{y}\m{\rightarrow}\m{(}\m{\varphi}\m{\rightarrow}\m{\forall}\m{x}\m{(}
\m{x}\m{=}\m{y}\m{\rightarrow}\m{\varphi}\m{)}\m{)}\m{)}\m{)}
\endm

\needspace{3\baselineskip}
\noindent Axiom of Quantified Equality.

\setbox\startprefix=\hbox{\tt \ \ ax-13\ \$a\ }
\setbox\contprefix=\hbox{\tt \ \ \ \ \ \ \ \ \ \ \ }
\startm
\m{\vdash}\m{(}\m{\lnot}\m{\forall}\m{z}\m{\,z}\m{=}\m{x}\m{\rightarrow}\m{(}
\m{\lnot}\m{\forall}\m{z}\m{\,z}\m{=}\m{y}\m{\rightarrow}\m{(}\m{x}\m{=}\m{y}
\m{\rightarrow}\m{\forall}\m{z}\m{\,x}\m{=}\m{y}\m{)}\m{)}\m{)}
\endm

% \noindent Axiom of Quantifier Substitution
%
% \setbox\startprefix=\hbox{\tt \ \ ax-c11n\ \$a\ }
% \setbox\contprefix=\hbox{\tt \ \ \ \ \ \ \ \ \ \ \ }
% \startm
% \m{\vdash}\m{(}\m{\forall}\m{x}\m{\,x}\m{=}\m{y}\m{\rightarrow}\m{(}\m{\forall}
% \m{x}\m{\varphi}\m{\rightarrow}\m{\forall}\m{y}\m{\varphi}\m{)}\m{)}
% \endm
%
% \noindent Axiom of Distinct Variables. (This axiom requires
% that two individual variables
% be distinct\index{\texttt{\$d} statement}\index{distinct
% variables}.)
%
% \setbox\startprefix=\hbox{\tt \ \ \ \ \ \ \ \ \$d\ }
% \setbox\contprefix=\hbox{\tt \ \ \ \ \ \ \ \ \ \ \ }
% \startm
% \m{x}\m{\,}\m{y}
% \endm
%
% \setbox\startprefix=\hbox{\tt \ \ ax-c16\ \$a\ }
% \setbox\contprefix=\hbox{\tt \ \ \ \ \ \ \ \ \ \ \ }
% \startm
% \m{\vdash}\m{(}\m{\forall}\m{x}\m{\,x}\m{=}\m{y}\m{\rightarrow}\m{(}\m{\varphi}\m{
% \rightarrow}\m{\forall}\m{x}\m{\varphi}\m{)}\m{)}
% \endm

% \noindent Axiom of Quantifier Introduction (2).  (This axiom requires
% that the individual variable not occur in the
% wff\index{\texttt{\$d} statement}\index{distinct variables}.)
%
% \setbox\startprefix=\hbox{\tt \ \ \ \ \ \ \ \ \$d\ }
% \setbox\contprefix=\hbox{\tt \ \ \ \ \ \ \ \ \ \ \ }
% \startm
% \m{x}\m{\,}\m{\varphi}
% \endm
% \setbox\startprefix=\hbox{\tt \ \ ax-5\ \$a\ }
% \setbox\contprefix=\hbox{\tt \ \ \ \ \ \ \ \ \ \ \ }
% \startm
% \m{\vdash}\m{(}\m{\varphi}\m{\rightarrow}\m{\forall}\m{x}\m{\varphi}\m{)}
% \endm

\subsection{Set Theory}\label{mmsettheoryaxioms}

In order to make the axioms of set theory\index{axioms of set theory} a little
more compact, there are several definitions from logic that we make use of
implicitly, namely, ``logical {\sc and},''\index{conjunction ($\wedge$)}
\index{logical {\sc and} ($\wedge$)} ``logical equivalence,''\index{logical
equivalence ($\leftrightarrow$)}\index{biconditional ($\leftrightarrow$)} and
``there exists.''\index{existential quantifier ($\exists$)}

\begin{center}\begin{tabular}{rcl}
  $( \varphi \wedge \psi )$ &\mbox{stands for}& $\neg ( \varphi
     \rightarrow \neg \psi )$\\
  $( \varphi \leftrightarrow \psi )$& \mbox{stands
     for}& $( ( \varphi \rightarrow \psi ) \wedge
     ( \psi \rightarrow \varphi ) )$\\
  $\exists x \,\varphi$ &\mbox{stands for}& $\neg \forall x \neg \varphi$
\end{tabular}\end{center}

In addition, the axioms of set theory require that all variables be
dis\-tinct,\index{distinct variables}\footnote{Set theory axioms can be
devised so that {\em no} variables are required to be distinct,
provided we replace \texttt{ax-c16} with an axiom stating that ``at
least two things exist,'' thus
making \texttt{ax-5} the only other axiom requiring the
\texttt{\$d} statement.  These axioms are unconventional and are not
presented here, but they can be found on the \url{http://metamath.org}
web site.  See also the Comment on
p.~\pageref{nodd}.}\index{\texttt{\$d} statement} thus we also assume:
\begin{center}
  \texttt{\$d }$x\,y\,z\,w$
\end{center}

\needspace{2\baselineskip}
\noindent Axiom of Extensionality.\index{Axiom of Extensionality}

\setbox\startprefix=\hbox{\tt \ \ ax-ext\ \$a\ }
\setbox\contprefix=\hbox{\tt \ \ \ \ \ \ \ \ \ \ \ \ }
\startm
\m{\vdash}\m{(}\m{\forall}\m{x}\m{(}\m{x}\m{\in}\m{y}\m{\leftrightarrow}\m{x}
\m{\in}\m{z}\m{)}\m{\rightarrow}\m{y}\m{=}\m{z}\m{)}
\endm

\needspace{3\baselineskip}
\noindent Axiom of Replacement.\index{Axiom of Replacement}

\setbox\startprefix=\hbox{\tt \ \ ax-rep\ \$a\ }
\setbox\contprefix=\hbox{\tt \ \ \ \ \ \ \ \ \ \ \ \ }
\startm
\m{\vdash}\m{(}\m{\forall}\m{w}\m{\exists}\m{y}\m{\forall}\m{z}\m{(}\m{%
\forall}\m{y}\m{\varphi}\m{\rightarrow}\m{z}\m{=}\m{y}\m{)}\m{\rightarrow}\m{%
\exists}\m{y}\m{\forall}\m{z}\m{(}\m{z}\m{\in}\m{y}\m{\leftrightarrow}\m{%
\exists}\m{w}\m{(}\m{w}\m{\in}\m{x}\m{\wedge}\m{\forall}\m{y}\m{\varphi}\m{)}%
\m{)}\m{)}
\endm

\needspace{2\baselineskip}
\noindent Axiom of Union.\index{Axiom of Union}

\setbox\startprefix=\hbox{\tt \ \ ax-un\ \$a\ }
\setbox\contprefix=\hbox{\tt \ \ \ \ \ \ \ \ \ \ \ }
\startm
\m{\vdash}\m{\exists}\m{x}\m{\forall}\m{y}\m{(}\m{\exists}\m{x}\m{(}\m{y}\m{
\in}\m{x}\m{\wedge}\m{x}\m{\in}\m{z}\m{)}\m{\rightarrow}\m{y}\m{\in}\m{x}\m{)}
\endm

\needspace{2\baselineskip}
\noindent Axiom of Power Sets.\index{Axiom of Power Sets}

\setbox\startprefix=\hbox{\tt \ \ ax-pow\ \$a\ }
\setbox\contprefix=\hbox{\tt \ \ \ \ \ \ \ \ \ \ \ \ }
\startm
\m{\vdash}\m{\exists}\m{x}\m{\forall}\m{y}\m{(}\m{\forall}\m{x}\m{(}\m{x}\m{
\in}\m{y}\m{\rightarrow}\m{x}\m{\in}\m{z}\m{)}\m{\rightarrow}\m{y}\m{\in}\m{x}
\m{)}
\endm

\needspace{3\baselineskip}
\noindent Axiom of Regularity.\index{Axiom of Regularity}

\setbox\startprefix=\hbox{\tt \ \ ax-reg\ \$a\ }
\setbox\contprefix=\hbox{\tt \ \ \ \ \ \ \ \ \ \ \ \ }
\startm
\m{\vdash}\m{(}\m{\exists}\m{x}\m{\,x}\m{\in}\m{y}\m{\rightarrow}\m{\exists}
\m{x}\m{(}\m{x}\m{\in}\m{y}\m{\wedge}\m{\forall}\m{z}\m{(}\m{z}\m{\in}\m{x}\m{
\rightarrow}\m{\lnot}\m{z}\m{\in}\m{y}\m{)}\m{)}\m{)}
\endm

\needspace{3\baselineskip}
\noindent Axiom of Infinity.\index{Axiom of Infinity}

\setbox\startprefix=\hbox{\tt \ \ ax-inf\ \$a\ }
\setbox\contprefix=\hbox{\tt \ \ \ \ \ \ \ \ \ \ \ \ \ \ \ }
\startm
\m{\vdash}\m{\exists}\m{x}\m{(}\m{y}\m{\in}\m{x}\m{\wedge}\m{\forall}\m{y}%
\m{(}\m{y}\m{\in}\m{x}\m{\rightarrow}\m{\exists}\m{z}\m{(}\m{y}\m{\in}\m{z}\m{%
\wedge}\m{z}\m{\in}\m{x}\m{)}\m{)}\m{)}
\endm

\needspace{4\baselineskip}
\noindent Axiom of Choice.\index{Axiom of Choice}

\setbox\startprefix=\hbox{\tt \ \ ax-ac\ \$a\ }
\setbox\contprefix=\hbox{\tt \ \ \ \ \ \ \ \ \ \ \ \ \ \ }
\startm
\m{\vdash}\m{\exists}\m{x}\m{\forall}\m{y}\m{\forall}\m{z}\m{(}\m{(}\m{y}\m{%
\in}\m{z}\m{\wedge}\m{z}\m{\in}\m{w}\m{)}\m{\rightarrow}\m{\exists}\m{w}\m{%
\forall}\m{y}\m{(}\m{\exists}\m{w}\m{(}\m{(}\m{y}\m{\in}\m{z}\m{\wedge}\m{z}%
\m{\in}\m{w}\m{)}\m{\wedge}\m{(}\m{y}\m{\in}\m{w}\m{\wedge}\m{w}\m{\in}\m{x}%
\m{)}\m{)}\m{\leftrightarrow}\m{y}\m{=}\m{w}\m{)}\m{)}
\endm

\subsection{That's It}

There you have it, the axioms for (essentially) all of mathematics!
Wonder at them and stare at them in awe.  Put a copy in your wallet, and
you will carry in your pocket the encoding for all theorems ever proved
and that ever will be proved, from the most mundane to the most
profound.

\section{A Hierarchy of Definitions}\label{hierarchy}

The axioms in the previous section in principle embody everything that can be
done within standard mathematics.  However, it is impractical to accomplish
very much by using them directly, for even simple concepts (from a human
perspective) can involve extremely long, incomprehensible formulas.
Mathematics is made practical by introducing definitions\index{definition}.
Definitions usually introduce new symbols, or at least new relationships among
existing symbols, to abbreviate more complex formulas.  An important
requirement for a definition is that there exist a straightforward
(algorithmic) method for eliminating the abbreviation by expanding it into the
more primitive symbol string that it represents.  Some
important definitions included in
the file \texttt{set.mm} are listed in this section for reference, and also to
give you a feel for why something like $\omega$\index{omega ($\omega$)} (the
set of natural numbers\index{natural number} 0, 1, 2,\ldots) becomes very
complicated when completely expanded into primitive symbols.

What is the motivation for definitions, aside from allowing complicated
expressions to be expressed more simply?  In the case of  $\omega$, one goal is
to provide a basis for the theory of natural numbers.\index{natural number}
Before set theory was invented, a set of axioms for arithmetic, called Peano's
postulates\index{Peano's postulates}, was devised and shown to have the
properties one expects for natural numbers.  Now anyone can postulate a
set of axioms, but if the axioms are inconsistent contradictions can be derived
from them.  Once a contradiction is derived, anything can be trivially
proved, including
all the facts of arithmetic and their negations.  To ensure that an
axiom system is at least as reliable as the axioms for set theory, we can
define sets and operations on those sets that satisfy the new axioms. In the
\texttt{set.mm} Metamath database, we prove that the elements of $\omega$ satisfy
Peano's postulates, and it's a long and hard journey to get there directly
from the axioms of set theory.  But the result is confidence in the
foundations of arithmetic.  And there is another advantage:  we now have all
the tools of set theory at our disposal for manipulating objects that obey the
axioms for arithmetic.

What are the criteria we use for definitions?  First, and of utmost importance,
the definition should not be {\em creative}\index{creative
definition}\index{definition!creative}, that
is it should not allow an expression that previously qualified as a wff but
was not provable, to become provable.   Second, the definition should be {\em
eliminable}\index{definition!eliminability}, that is, there should exist an
algorithmic method for converting any expression using the definition into
a logically equivalent expression that previously qualified as a wff.

In almost all cases below, definitions connect two expressions with either
$\leftrightarrow$ or $=$.  Eliminating\footnote{Here we mean the
elimination that a human might do in his or her head.  To eliminate them as
part of a Metamath proof we would invoke one of a number of
theorems that deal with transitivity of equivalence or equality; there are
many such examples in the proofs in \texttt{set.mm}.} such a definition is a
simple matter of substituting the expression on the left-hand side ({\em
definiendum}\index{definiendum} or thing being defined) with the equivalent,
more primitive expression on the right-hand side ({\em
definiens}\index{definiens} or definition).

Often a definition has variables on the right-hand side which do not appear on
the left-hand side; these are called {\em dummy variables}.\index{dummy
variable!in definitions}  In this case, any
allowable substitution (such as a new, distinct
variable) can be used when the definition is eliminated.  Dummy variables may
be used only if they are {\em effectively bound}\index{effectively bound
variable}, meaning that the definition will remain logically equivalent upon
any substitution of a dummy variable with any other {\em qualifying
expression}\index{qualifying expression}, i.e.\ any symbol string (such as
another variable) that
meets the restrictions on the dummy variable imposed by \texttt{\$d} and
\texttt{\$f} statements.  For example, we could define a constant $\perp$
(inverted tee, meaning logical ``false'') as $( \varphi \wedge \lnot \varphi
)$, i.e.\ ``phi and not phi.''  Here $\varphi$ is effectively bound because the
definition remains logically equivalent when we replace $\varphi$ with any
other wff.  (It is actually \texttt{df-fal}
in \texttt{set.mm}, which defines $\perp$.)

There are two cases where eliminating definitions is a little more
complex.  These cases are the definitions \texttt{df-bi} and
\texttt{df-cleq}.  The first stretches the concept of a definition a
little, as in effect it ``defines a definition;'' however, it meets our
requirements for a definition in that it is eliminable and does not
strengthen the language.  Theorem \texttt{bii} shows the substitution
needed to eliminate the $\leftrightarrow$\index{logical equivalence
($\leftrightarrow$)}\index{biconditional ($\leftrightarrow$)} symbol.

Definition \texttt{df-cleq}\index{equality ($=$)} extends the usage of
the equality symbol to include ``classes''\index{class} in set theory.  The
reason it is potentially problematic is that it can lead to statements which
do not follow from logic alone but presuppose the Axiom of
Extensionality\index{Axiom of Extensionality}, so we include this axiom
as a hypothesis for the definition.  We could have made \texttt{df-cleq} directly
eliminable by introducing a new equality symbol, but have chosen not to do so
in keeping with standard textbook practice.  Definitions such as \texttt{df-cleq}
that extend the meaning of existing symbols must be introduced carefully so
that they do not lead to contradictions.  Definition \texttt{df-clel} also
extends the meaning of an existing symbol ($\in$); while it doesn't strengthen
the language like \texttt{df-cleq}, this is not obvious and it must also be
subject to the same scrutiny.

Exercise:  Study how the wff $x\in\omega$, meaning ``$x$ is a natural
number,'' could be expanded in terms of primitive symbols, starting with the
definitions \texttt{df-clel} on p.~\pageref{dfclel} and \texttt{df-om} on
p.~\pageref{dfom} and working your way back.  Don't bother to work out the
details; just make sure that you understand how you could do it in principle.
The answer is shown in the footnote on p.~\pageref{expandom}.  If you
actually do work it out, you won't get exactly the same answer because we used
a few simplifications such as discarding occurrences of $\lnot\lnot$ (double
negation).

In the definitions below, we have placed the {\sc ascii} Metamath source
below each of the formulas to help you become familiar with the
notation in the database.  For simplicity, the necessary \texttt{\$f}
and \texttt{\$d} statements are not shown.  If you are in doubt, use the
\texttt{show statement}\index{\texttt{show statement} command} command
in the Metamath program to see the full statement.
A selection of this notation is summarized in Appendix~\ref{ASCII}.

To understand the motivation for these definitions, you should consult the
references indicated:  Takeuti and Zaring \cite{Takeuti}\index{Takeuti, Gaisi},
Quine \cite{Quine}\index{Quine, Willard Van Orman}, Bell and Machover
\cite{Bell}\index{Bell, J. L.}, and Enderton \cite{Enderton}\index{Enderton,
Herbert B.}.  Our list of definitions is provided more for reference than as a
learning aid.  However, by looking at a few of them you can gain a feel for
how the hierarchy is built up.  The definitions are a representative sample of
the many definitions
in \texttt{set.mm}, but they are complete with respect to the
theorem examples we will present in Section~\ref{sometheorems}.  Also, some are
slightly different from, but logically equivalent to, the ones in \texttt{set.mm}
(some of which have been revised over time to shorten them, for example).

\subsection{Definitions for Propositional Calculus}\label{metadefprop}

The symbols $\varphi$, $\psi$, and $\chi$ represent wffs.

Our first definition introduces the biconditional
connective\footnote{The term ``connective'' is informally used to mean a
symbol that is placed between two variables or adjacent to a variable,
whereas a mathematical ``constant'' usually indicates a symbol such as
the number 0 that may replace a variable or metavariable.  From
Metamath's point of view, there is no distinction between a connective
and a constant; both are constants in the Metamath
language.}\index{connective}\index{constant} (also called logical
equivalence)\index{logical equivalence
($\leftrightarrow$)}\index{biconditional ($\leftrightarrow$)}.  Unlike
most traditional developments, we have chosen not to have a separate
symbol such as ``Df.'' to mean ``is defined as.''  Instead, we will use
the biconditional connective for this purpose, as it lets us use
logic to manipulate definitions directly.  Here we state the properties
of the biconditional connective with a carefully crafted \texttt{\$a}
statement, which effectively uses the biconditional connective to define
itself.  The $\leftrightarrow$ symbol can be eliminated from a formula
using theorem \texttt{bii}, which is derived later.

\vskip 2ex
\noindent Define the biconditional connective.\label{df-bi}

\vskip 0.5ex
\setbox\startprefix=\hbox{\tt \ \ df-bi\ \$a\ }
\setbox\contprefix=\hbox{\tt \ \ \ \ \ \ \ \ \ \ \ }
\startm
\m{\vdash}\m{\lnot}\m{(}\m{(}\m{(}\m{\varphi}\m{\leftrightarrow}\m{\psi}\m{)}%
\m{\rightarrow}\m{\lnot}\m{(}\m{(}\m{\varphi}\m{\rightarrow}\m{\psi}\m{)}\m{%
\rightarrow}\m{\lnot}\m{(}\m{\psi}\m{\rightarrow}\m{\varphi}\m{)}\m{)}\m{)}\m{%
\rightarrow}\m{\lnot}\m{(}\m{\lnot}\m{(}\m{(}\m{\varphi}\m{\rightarrow}\m{%
\psi}\m{)}\m{\rightarrow}\m{\lnot}\m{(}\m{\psi}\m{\rightarrow}\m{\varphi}\m{)}%
\m{)}\m{\rightarrow}\m{(}\m{\varphi}\m{\leftrightarrow}\m{\psi}\m{)}\m{)}\m{)}
\endm
\begin{mmraw}%
|- -. ( ( ( ph <-> ps ) -> -. ( ( ph -> ps ) ->
-. ( ps -> ph ) ) ) -> -. ( -. ( ( ph -> ps ) -> -. (
ps -> ph ) ) -> ( ph <-> ps ) ) ) \$.
\end{mmraw}

\noindent This theorem relates the biconditional connective to primitive
connectives and can be used to eliminate the $\leftrightarrow$ symbol from any
wff.

\vskip 0.5ex
\setbox\startprefix=\hbox{\tt \ \ bii\ \$p\ }
\setbox\contprefix=\hbox{\tt \ \ \ \ \ \ \ \ \ }
\startm
\m{\vdash}\m{(}\m{(}\m{\varphi}\m{\leftrightarrow}\m{\psi}\m{)}\m{\leftrightarrow}
\m{\lnot}\m{(}\m{(}\m{\varphi}\m{\rightarrow}\m{\psi}\m{)}\m{\rightarrow}\m{\lnot}
\m{(}\m{\psi}\m{\rightarrow}\m{\varphi}\m{)}\m{)}\m{)}
\endm
\begin{mmraw}%
|- ( ( ph <-> ps ) <-> -. ( ( ph -> ps ) -> -. ( ps -> ph ) ) ) \$= ... \$.
\end{mmraw}

\noindent Define disjunction ({\sc or}).\index{disjunction ($\vee$)}%
\index{logical or (vee)@logical {\sc or} ($\vee$)}%
\index{df-or@\texttt{df-or}}\label{df-or}

\vskip 0.5ex
\setbox\startprefix=\hbox{\tt \ \ df-or\ \$a\ }
\setbox\contprefix=\hbox{\tt \ \ \ \ \ \ \ \ \ \ \ }
\startm
\m{\vdash}\m{(}\m{(}\m{\varphi}\m{\vee}\m{\psi}\m{)}\m{\leftrightarrow}\m{(}\m{
\lnot}\m{\varphi}\m{\rightarrow}\m{\psi}\m{)}\m{)}
\endm
\begin{mmraw}%
|- ( ( ph \TOR ps ) <-> ( -. ph -> ps ) ) \$.
\end{mmraw}

\noindent Define conjunction ({\sc and}).\index{conjunction ($\wedge$)}%
\index{logical {\sc and} ($\wedge$)}%
\index{df-an@\texttt{df-an}}\label{df-an}

\vskip 0.5ex
\setbox\startprefix=\hbox{\tt \ \ df-an\ \$a\ }
\setbox\contprefix=\hbox{\tt \ \ \ \ \ \ \ \ \ \ \ }
\startm
\m{\vdash}\m{(}\m{(}\m{\varphi}\m{\wedge}\m{\psi}\m{)}\m{\leftrightarrow}\m{\lnot}
\m{(}\m{\varphi}\m{\rightarrow}\m{\lnot}\m{\psi}\m{)}\m{)}
\endm
\begin{mmraw}%
|- ( ( ph \TAND ps ) <-> -. ( ph -> -. ps ) ) \$.
\end{mmraw}

\noindent Define disjunction ({\sc or}) of 3 wffs.%
\index{df-3or@\texttt{df-3or}}\label{df-3or}

\vskip 0.5ex
\setbox\startprefix=\hbox{\tt \ \ df-3or\ \$a\ }
\setbox\contprefix=\hbox{\tt \ \ \ \ \ \ \ \ \ \ \ \ }
\startm
\m{\vdash}\m{(}\m{(}\m{\varphi}\m{\vee}\m{\psi}\m{\vee}\m{\chi}\m{)}\m{
\leftrightarrow}\m{(}\m{(}\m{\varphi}\m{\vee}\m{\psi}\m{)}\m{\vee}\m{\chi}\m{)}
\m{)}
\endm
\begin{mmraw}%
|- ( ( ph \TOR ps \TOR ch ) <-> ( ( ph \TOR ps ) \TOR ch ) ) \$.
\end{mmraw}

\noindent Define conjunction ({\sc and}) of 3 wffs.%
\index{df-3an}\label{df-3an}

\vskip 0.5ex
\setbox\startprefix=\hbox{\tt \ \ df-3an\ \$a\ }
\setbox\contprefix=\hbox{\tt \ \ \ \ \ \ \ \ \ \ \ \ }
\startm
\m{\vdash}\m{(}\m{(}\m{\varphi}\m{\wedge}\m{\psi}\m{\wedge}\m{\chi}\m{)}\m{
\leftrightarrow}\m{(}\m{(}\m{\varphi}\m{\wedge}\m{\psi}\m{)}\m{\wedge}\m{\chi}
\m{)}\m{)}
\endm

\begin{mmraw}%
|- ( ( ph \TAND ps \TAND ch ) <-> ( ( ph \TAND ps ) \TAND ch ) ) \$.
\end{mmraw}

\subsection{Definitions for Predicate Calculus}\label{metadefpred}

The symbols $x$, $y$, and $z$ represent individual variables of predicate
calculus.  In this section, they are not necessarily distinct unless it is
explicitly
mentioned.

\vskip 2ex
\noindent Define existential quantification.
The expression $\exists x \varphi$ means
``there exists an $x$ where $\varphi$ is true.''\index{existential quantifier
($\exists$)}\label{df-ex}

\vskip 0.5ex
\setbox\startprefix=\hbox{\tt \ \ df-ex\ \$a\ }
\setbox\contprefix=\hbox{\tt \ \ \ \ \ \ \ \ \ \ \ }
\startm
\m{\vdash}\m{(}\m{\exists}\m{x}\m{\varphi}\m{\leftrightarrow}\m{\lnot}\m{\forall}
\m{x}\m{\lnot}\m{\varphi}\m{)}
\endm
\begin{mmraw}%
|- ( E. x ph <-> -. A. x -. ph ) \$.
\end{mmraw}

\noindent Define proper substitution.\index{proper
substitution}\index{substitution!proper}\label{df-sb}
In our notation, we use $[ y / x ] \varphi$ to mean ``the wff that
results when $y$ is properly substituted for $x$ in the wff
$\varphi$.''\footnote{
This can also be described
as substituting $x$ with $y$, $y$ properly replaces $x$, or
$x$ is properly replaced by $y$.}
% This is elsb4, though it currently says: ( [ x / y ] z e. y <-> z e. x )
For example,
$[ y / x ] z \in x$ is the same as $z \in y$.
One way to remember this notation is to notice that it looks like division
and recall that $( y / x ) \cdot x $ is $y$ (when $x \neq 0$).
The notation is different from the notation $\varphi ( x | y )$
that is sometimes used, because the latter notation is ambiguous for us:
for example, we don't know whether $\lnot \varphi ( x | y )$ is to be
interpreted as $\lnot ( \varphi ( x | y ) )$ or
$( \lnot \varphi ) ( x | y )$.\footnote{Because of the way
we initially defined wffs, this is the case
with any postfix connective\index{postfix connective} (one occurring after the
symbols being connected) or infix connective\index{infix connective} (one
occurring between the symbols being connected).  Metamath does not have a
built-in notion of operator binding strength that could eliminate the
ambiguity.  The initial parenthesis effectively provides a prefix
connective\index{prefix connective} to eliminate ambiguity.  Some conventions,
such as Polish notation\index{Polish notation} used in the 1930's and 1940's
by Polish logicians, use only prefix connectives and thus allow the total
elimination of parentheses, at the expense of readability.  In Metamath we
could actually redefine all notation to be Polish if we wanted to without
having to change any proofs!}  Other texts often use $\varphi(y)$ to indicate
our $[ y / x ] \varphi$, but this notation is even more ambiguous since there is
no explicit indication of what is being substituted.
Note that this
definition is valid even when
$x$ and $y$ are the same variable.  The first conjunct is a ``trick'' used to
achieve this property, making the definition look somewhat peculiar at
first.

\vskip 0.5ex
\setbox\startprefix=\hbox{\tt \ \ df-sb\ \$a\ }
\setbox\contprefix=\hbox{\tt \ \ \ \ \ \ \ \ \ \ \ }
\startm
\m{\vdash}\m{(}\m{[}\m{y}\m{/}\m{x}\m{]}\m{\varphi}\m{\leftrightarrow}\m{(}%
\m{(}\m{x}\m{=}\m{y}\m{\rightarrow}\m{\varphi}\m{)}\m{\wedge}\m{\exists}\m{x}%
\m{(}\m{x}\m{=}\m{y}\m{\wedge}\m{\varphi}\m{)}\m{)}\m{)}
\endm
\begin{mmraw}%
|- ( [ y / x ] ph <-> ( ( x = y -> ph ) \TAND E. x ( x = y \TAND ph ) ) ) \$.
\end{mmraw}


\noindent Define existential uniqueness\index{existential uniqueness
quantifier ($\exists "!$)} (``there exists exactly one'').  Note that $y$ is a
variable distinct from $x$ and not occurring in $\varphi$.\label{df-eu}

\vskip 0.5ex
\setbox\startprefix=\hbox{\tt \ \ df-eu\ \$a\ }
\setbox\contprefix=\hbox{\tt \ \ \ \ \ \ \ \ \ \ \ }
\startm
\m{\vdash}\m{(}\m{\exists}\m{{!}}\m{x}\m{\varphi}\m{\leftrightarrow}\m{\exists}
\m{y}\m{\forall}\m{x}\m{(}\m{\varphi}\m{\leftrightarrow}\m{x}\m{=}\m{y}\m{)}\m{)}
\endm

\begin{mmraw}%
|- ( E! x ph <-> E. y A. x ( ph <-> x = y ) ) \$.
\end{mmraw}

\subsection{Definitions for Set Theory}\label{setdefinitions}

The symbols $x$, $y$, $z$, and $w$ represent individual variables of
predicate calculus, which in set theory are understood to be sets.
However, using only the constructs shown so far would be very inconvenient.

To make set theory more practical, we introduce the notion of a ``class.''
A class\index{class} is either a set variable (such as $x$) or an
expression of the form $\{ x | \varphi\}$ (called an ``abstraction
class''\index{abstraction class}\index{class abstraction}).  Note that
sets (i.e.\ individual variables) always exist (this is a theorem of
logic, namely $\exists y \, y = x$ for any set $x$), whereas classes may
or may not exist (i.e.\ $\exists y \, y = A$ may or may not be true).
If a class does not exist it is called a ``proper class.''\index{proper
class}\index{class!proper} Definitions \texttt{df-clab},
\texttt{df-cleq}, and \texttt{df-clel} can be used to convert an
expression containing classes into one containing only set variables and
wff metavariables.

The symbols $A$, $B$, $C$, $D$, $ F$, $G$, and $R$ are metavariables that range
over classes.  A class metavariable $A$ may be eliminated from a wff by
replacing it with $\{ x|\varphi\}$ where neither $x$ nor $\varphi$ occur in
the wff.

The theory of classes can be shown to be an eliminable and conservative
extension of set theory. The \textbf{eliminability}
property shows that for every
formula in the extended language we can build a logically equivalent
formula in the basic language; so that even if the extended language
provides more ease to convey and formulate mathematical ideas for set
theory, its expressive power does not in fact strengthen the basic
language's expressive power.
The \textbf{conservation} property shows that for
every proof of a formula of the basic language in the extended system
we can build another proof of the same formula in the basic system;
so that, concerning theorems on sets only, the deductive powers of
the extended system and of the basic system are identical. Together,
these properties mean that the extended language can be treated as a
definitional extension that is \textbf{sound}.

A rigorous justification, which we will not give here, can be found in
Levy \cite[pp.~357-366]{Levy} supplementing his informal introduction to class
theory on pp.~7-17. Two other good treatments of class theory are provided
by Quine \cite[pp.~15-21]{Quine}\index{Quine, Willard Van Orman}
and also \cite[pp.~10-14]{Takeuti}\index{Takeuti, Gaisi}.
Quine's exposition (he calls them virtual classes)
is nicely written and very readable.

In the rest of this
section, individual variables are always assumed to be distinct from
each other unless otherwise indicated.  In addition, dummy variables on the
right-hand side of a definition do not occur in the class and wff
metavariables in the definition.

The definitions we present here are a partial but self-contained
collection selected from several hundred that appear in the current
\texttt{set.mm} database.  They are adequate for a basic development of
elementary set theory.

\vskip 2ex
\noindent Define the abstraction class.\index{abstraction class}\index{class
abstraction}\label{df-clab}  $x$ and $y$
need not be distinct.  Definition 2.1 of Quine, p.~16.  This definition may
seem puzzling since it is shorter than the expression being defined and does not
buy us anything in terms of brevity.  The reason we introduce this definition
is because it fits in neatly with the extension of the $\in$ connective
provided by \texttt{df-clel}.

\vskip 0.5ex
\setbox\startprefix=\hbox{\tt \ \ df-clab\ \$a\ }
\setbox\contprefix=\hbox{\tt \ \ \ \ \ \ \ \ \ \ \ \ \ }
\startm
\m{\vdash}\m{(}\m{x}\m{\in}\m{\{}\m{y}\m{|}\m{\varphi}\m{\}}\m{%
\leftrightarrow}\m{[}\m{x}\m{/}\m{y}\m{]}\m{\varphi}\m{)}
\endm
\begin{mmraw}%
|- ( x e. \{ y | ph \} <-> [ x / y ] ph ) \$.
\end{mmraw}

\noindent Define the equality connective between classes\index{class
equality}\label{df-cleq}.  See Quine or Chapter 4 of Takeuti and Zaring for its
justification and methods for eliminating it.  This is an example of a
somewhat ``dangerous'' definition, because it extends the use of the
existing equality symbol rather than introducing a new symbol, allowing
us to make statements in the original language that may not be true.
For example, it permits us to deduce $y = z \leftrightarrow \forall x (
x \in y \leftrightarrow x \in z )$ which is not a theorem of logic but
rather presupposes the Axiom of Extensionality,\index{Axiom of
Extensionality} which we include as a hypothesis so that we can know
when this axiom is assumed in a proof (with the \texttt{show
trace{\char`\_}back} command).  We could avoid the danger by introducing
another symbol, say $\eqcirc$, in place of $=$; this
would also have the advantage of making elimination of the definition
straightforward and would eliminate the need for Extensionality as a
hypothesis.  We would then also have the advantage of being able to
identify exactly where Extensionality truly comes into play.  One of our
theorems would be $x \eqcirc y \leftrightarrow x = y$
by invoking Extensionality.  However in keeping with standard practice
we retain the ``dangerous'' definition.

\vskip 0.5ex
\setbox\startprefix=\hbox{\tt \ \ df-cleq.1\ \$e\ }
\setbox\contprefix=\hbox{\tt \ \ \ \ \ \ \ \ \ \ \ \ \ \ \ }
\startm
\m{\vdash}\m{(}\m{\forall}\m{x}\m{(}\m{x}\m{\in}\m{y}\m{\leftrightarrow}\m{x}
\m{\in}\m{z}\m{)}\m{\rightarrow}\m{y}\m{=}\m{z}\m{)}
\endm
\setbox\startprefix=\hbox{\tt \ \ df-cleq\ \$a\ }
\setbox\contprefix=\hbox{\tt \ \ \ \ \ \ \ \ \ \ \ \ \ }
\startm
\m{\vdash}\m{(}\m{A}\m{=}\m{B}\m{\leftrightarrow}\m{\forall}\m{x}\m{(}\m{x}\m{
\in}\m{A}\m{\leftrightarrow}\m{x}\m{\in}\m{B}\m{)}\m{)}
\endm
% We need to reset the startprefix and contprefix.
\setbox\startprefix=\hbox{\tt \ \ df-cleq.1\ \$e\ }
\setbox\contprefix=\hbox{\tt \ \ \ \ \ \ \ \ \ \ \ \ \ \ \ }
\begin{mmraw}%
|- ( A. x ( x e. y <-> x e. z ) -> y = z ) \$.
\end{mmraw}
\setbox\startprefix=\hbox{\tt \ \ df-cleq\ \$a\ }
\setbox\contprefix=\hbox{\tt \ \ \ \ \ \ \ \ \ \ \ \ \ }
\begin{mmraw}%
|- ( A = B <-> A. x ( x e. A <-> x e. B ) ) \$.
\end{mmraw}

\noindent Define the membership connective between classes\index{class
membership}.  Theorem 6.3 of Quine, p.~41, which we adopt as a definition.
Note that it extends the use of the existing membership symbol, but unlike
{\tt df-cleq} it does not extend the set of valid wffs of logic when the class
metavariables are replaced with set variables.\label{dfclel}\label{df-clel}

\vskip 0.5ex
\setbox\startprefix=\hbox{\tt \ \ df-clel\ \$a\ }
\setbox\contprefix=\hbox{\tt \ \ \ \ \ \ \ \ \ \ \ \ \ }
\startm
\m{\vdash}\m{(}\m{A}\m{\in}\m{B}\m{\leftrightarrow}\m{\exists}\m{x}\m{(}\m{x}
\m{=}\m{A}\m{\wedge}\m{x}\m{\in}\m{B}\m{)}\m{)}
\endm
\begin{mmraw}%
|- ( A e. B <-> E. x ( x = A \TAND x e. B ) ) \$.?
\end{mmraw}

\noindent Define inequality.

\vskip 0.5ex
\setbox\startprefix=\hbox{\tt \ \ df-ne\ \$a\ }
\setbox\contprefix=\hbox{\tt \ \ \ \ \ \ \ \ \ \ \ }
\startm
\m{\vdash}\m{(}\m{A}\m{\ne}\m{B}\m{\leftrightarrow}\m{\lnot}\m{A}\m{=}\m{B}%
\m{)}
\endm
\begin{mmraw}%
|- ( A =/= B <-> -. A = B ) \$.
\end{mmraw}

\noindent Define restricted universal quantification.\index{universal
quantifier ($\forall$)!restricted}  Enderton, p.~22.

\vskip 0.5ex
\setbox\startprefix=\hbox{\tt \ \ df-ral\ \$a\ }
\setbox\contprefix=\hbox{\tt \ \ \ \ \ \ \ \ \ \ \ \ }
\startm
\m{\vdash}\m{(}\m{\forall}\m{x}\m{\in}\m{A}\m{\varphi}\m{\leftrightarrow}\m{%
\forall}\m{x}\m{(}\m{x}\m{\in}\m{A}\m{\rightarrow}\m{\varphi}\m{)}\m{)}
\endm
\begin{mmraw}%
|- ( A. x e. A ph <-> A. x ( x e. A -> ph ) ) \$.
\end{mmraw}

\noindent Define restricted existential quantification.\index{existential
quantifier ($\exists$)!restricted}  Enderton, p.~22.

\vskip 0.5ex
\setbox\startprefix=\hbox{\tt \ \ df-rex\ \$a\ }
\setbox\contprefix=\hbox{\tt \ \ \ \ \ \ \ \ \ \ \ \ }
\startm
\m{\vdash}\m{(}\m{\exists}\m{x}\m{\in}\m{A}\m{\varphi}\m{\leftrightarrow}\m{%
\exists}\m{x}\m{(}\m{x}\m{\in}\m{A}\m{\wedge}\m{\varphi}\m{)}\m{)}
\endm
\begin{mmraw}%
|- ( E. x e. A ph <-> E. x ( x e. A \TAND ph ) ) \$.
\end{mmraw}

\noindent Define the universal class\index{universal class ($V$)}.  Definition
5.20, p.~21, of Takeuti and Zaring.\label{df-v}

\vskip 0.5ex
\setbox\startprefix=\hbox{\tt \ \ df-v\ \$a\ }
\setbox\contprefix=\hbox{\tt \ \ \ \ \ \ \ \ \ \ }
\startm
\m{\vdash}\m{{\rm V}}\m{=}\m{\{}\m{x}\m{|}\m{x}\m{=}\m{x}\m{\}}
\endm
\begin{mmraw}%
|- {\char`\_}V = \{ x | x = x \} \$.
\end{mmraw}

\noindent Define the subclass\index{subclass}\index{subset} relationship
between two classes (called the subset relation if the classes are sets i.e.\
are not proper).  Definition 5.9 of Takeuti and Zaring, p.~17.\label{df-ss}

\vskip 0.5ex
\setbox\startprefix=\hbox{\tt \ \ df-ss\ \$a\ }
\setbox\contprefix=\hbox{\tt \ \ \ \ \ \ \ \ \ \ \ }
\startm
\m{\vdash}\m{(}\m{A}\m{\subseteq}\m{B}\m{\leftrightarrow}\m{\forall}\m{x}\m{(}
\m{x}\m{\in}\m{A}\m{\rightarrow}\m{x}\m{\in}\m{B}\m{)}\m{)}
\endm
\begin{mmraw}%
|- ( A C\_ B <-> A. x ( x e. A -> x e. B ) ) \$.
\end{mmraw}

\noindent Define the union\index{union} of two classes.  Definition 5.6 of Takeuti and Zaring,
p.~16.\label{df-un}

\vskip 0.5ex
\setbox\startprefix=\hbox{\tt \ \ df-un\ \$a\ }
\setbox\contprefix=\hbox{\tt \ \ \ \ \ \ \ \ \ \ \ }
\startm
\m{\vdash}\m{(}\m{A}\m{\cup}\m{B}\m{)}\m{=}\m{\{}\m{x}\m{|}\m{(}\m{x}\m{\in}
\m{A}\m{\vee}\m{x}\m{\in}\m{B}\m{)}\m{\}}
\endm
\begin{mmraw}%
( A u. B ) = \{ x | ( x e. A \TOR x e. B ) \} \$.
\end{mmraw}

\noindent Define the intersection\index{intersection} of two classes.  Definition 5.6 of
Takeuti and Zaring, p.~16.\label{df-in}

\vskip 0.5ex
\setbox\startprefix=\hbox{\tt \ \ df-in\ \$a\ }
\setbox\contprefix=\hbox{\tt \ \ \ \ \ \ \ \ \ \ \ }
\startm
\m{\vdash}\m{(}\m{A}\m{\cap}\m{B}\m{)}\m{=}\m{\{}\m{x}\m{|}\m{(}\m{x}\m{\in}
\m{A}\m{\wedge}\m{x}\m{\in}\m{B}\m{)}\m{\}}
\endm
% Caret ^ requires special treatment
\begin{mmraw}%
|- ( A i\^{}i B ) = \{ x | ( x e. A \TAND x e. B ) \} \$.
\end{mmraw}

\noindent Define class difference\index{class difference}\index{set difference}.
Definition 5.12 of Takeuti and Zaring, p.~20.  Several notations are used in
the literature; we chose the $\setminus$ convention instead of a minus sign to
reserve the latter for later use in, e.g., arithmetic.\label{df-dif}

\vskip 0.5ex
\setbox\startprefix=\hbox{\tt \ \ df-dif\ \$a\ }
\setbox\contprefix=\hbox{\tt \ \ \ \ \ \ \ \ \ \ \ \ }
\startm
\m{\vdash}\m{(}\m{A}\m{\setminus}\m{B}\m{)}\m{=}\m{\{}\m{x}\m{|}\m{(}\m{x}\m{
\in}\m{A}\m{\wedge}\m{\lnot}\m{x}\m{\in}\m{B}\m{)}\m{\}}
\endm
\begin{mmraw}%
( A \SLASH B ) = \{ x | ( x e. A \TAND -. x e. B ) \} \$.
\end{mmraw}

\noindent Define the empty or null set\index{empty set}\index{null set}.
Compare  Definition 5.14 of Takeuti and Zaring, p.~20.\label{df-nul}

\vskip 0.5ex
\setbox\startprefix=\hbox{\tt \ \ df-nul\ \$a\ }
\setbox\contprefix=\hbox{\tt \ \ \ \ \ \ \ \ \ \ }
\startm
\m{\vdash}\m{\varnothing}\m{=}\m{(}\m{{\rm V}}\m{\setminus}\m{{\rm V}}\m{)}
\endm
\begin{mmraw}%
|- (/) = ( {\char`\_}V \SLASH {\char`\_}V ) \$.
\end{mmraw}

\noindent Define power class\index{power set}\index{power class}.  Definition 5.10 of
Takeuti and Zaring, p.~17, but we also let it apply to proper classes.  (Note
that \verb$~P$ is the symbol for calligraphic P, the tilde
suggesting ``curly;'' see Appendix~\ref{ASCII}.)\label{df-pw}

\vskip 0.5ex
\setbox\startprefix=\hbox{\tt \ \ df-pw\ \$a\ }
\setbox\contprefix=\hbox{\tt \ \ \ \ \ \ \ \ \ \ \ }
\startm
\m{\vdash}\m{{\cal P}}\m{A}\m{=}\m{\{}\m{x}\m{|}\m{x}\m{\subseteq}\m{A}\m{\}}
\endm
% Special incantation required to put ~ into the text
\begin{mmraw}%
|- \char`\~P~A = \{ x | x C\_ A \} \$.
\end{mmraw}

\noindent Define the singleton of a class\index{singleton}.  Definition 7.1 of
Quine, p.~48.  It is well-defined for proper classes, although
it is not very meaningful in this case, where it evaluates to the empty
set.

\vskip 0.5ex
\setbox\startprefix=\hbox{\tt \ \ df-sn\ \$a\ }
\setbox\contprefix=\hbox{\tt \ \ \ \ \ \ \ \ \ \ \ }
\startm
\m{\vdash}\m{\{}\m{A}\m{\}}\m{=}\m{\{}\m{x}\m{|}\m{x}\m{=}\m{A}\m{\}}
\endm
\begin{mmraw}%
|- \{ A \} = \{ x | x = A \} \$.
\end{mmraw}%

\noindent Define an unordered pair of classes\index{unordered pair}\index{pair}.  Definition
7.1 of Quine, p.~48.

\vskip 0.5ex
\setbox\startprefix=\hbox{\tt \ \ df-pr\ \$a\ }
\setbox\contprefix=\hbox{\tt \ \ \ \ \ \ \ \ \ \ \ }
\startm
\m{\vdash}\m{\{}\m{A}\m{,}\m{B}\m{\}}\m{=}\m{(}\m{\{}\m{A}\m{\}}\m{\cup}\m{\{}
\m{B}\m{\}}\m{)}
\endm
\begin{mmraw}%
|- \{ A , B \} = ( \{ A \} u. \{ B \} ) \$.
\end{mmraw}

\noindent Define an unordered triple of classes\index{unordered triple}.  Definition of
Enderton, p.~19.

\vskip 0.5ex
\setbox\startprefix=\hbox{\tt \ \ df-tp\ \$a\ }
\setbox\contprefix=\hbox{\tt \ \ \ \ \ \ \ \ \ \ \ }
\startm
\m{\vdash}\m{\{}\m{A}\m{,}\m{B}\m{,}\m{C}\m{\}}\m{=}\m{(}\m{\{}\m{A}\m{,}\m{B}
\m{\}}\m{\cup}\m{\{}\m{C}\m{\}}\m{)}
\endm
\begin{mmraw}%
|- \{ A , B , C \} = ( \{ A , B \} u. \{ C \} ) \$.
\end{mmraw}%

\noindent Kuratowski's\index{Kuratowski, Kazimierz} ordered pair\index{ordered
pair} definition.  Definition 9.1 of Quine, p.~58. For proper classes it is
not meaningful but is well-defined for convenience.  (Note that \verb$<.$
stands for $\langle$ whereas \verb$<$ stands for $<$, and similarly for
\verb$>.$\,.)\label{df-op}

\vskip 0.5ex
\setbox\startprefix=\hbox{\tt \ \ df-op\ \$a\ }
\setbox\contprefix=\hbox{\tt \ \ \ \ \ \ \ \ \ \ \ }
\startm
\m{\vdash}\m{\langle}\m{A}\m{,}\m{B}\m{\rangle}\m{=}\m{\{}\m{\{}\m{A}\m{\}}
\m{,}\m{\{}\m{A}\m{,}\m{B}\m{\}}\m{\}}
\endm
\begin{mmraw}%
|- <. A , B >. = \{ \{ A \} , \{ A , B \} \} \$.
\end{mmraw}

\noindent Define the union of a class\index{union}.  Definition 5.5, p.~16,
of Takeuti and Zaring.

\vskip 0.5ex
\setbox\startprefix=\hbox{\tt \ \ df-uni\ \$a\ }
\setbox\contprefix=\hbox{\tt \ \ \ \ \ \ \ \ \ \ \ \ }
\startm
\m{\vdash}\m{\bigcup}\m{A}\m{=}\m{\{}\m{x}\m{|}\m{\exists}\m{y}\m{(}\m{x}\m{
\in}\m{y}\m{\wedge}\m{y}\m{\in}\m{A}\m{)}\m{\}}
\endm
\begin{mmraw}%
|- U. A = \{ x | E. y ( x e. y \TAND y e. A ) \} \$.
\end{mmraw}

\noindent Define the intersection\index{intersection} of a class.  Definition 7.35,
p.~44, of Takeuti and Zaring.

\vskip 0.5ex
\setbox\startprefix=\hbox{\tt \ \ df-int\ \$a\ }
\setbox\contprefix=\hbox{\tt \ \ \ \ \ \ \ \ \ \ \ \ }
\startm
\m{\vdash}\m{\bigcap}\m{A}\m{=}\m{\{}\m{x}\m{|}\m{\forall}\m{y}\m{(}\m{y}\m{
\in}\m{A}\m{\rightarrow}\m{x}\m{\in}\m{y}\m{)}\m{\}}
\endm
\begin{mmraw}%
|- |\^{}| A = \{ x | A. y ( y e. A -> x e. y ) \} \$.
\end{mmraw}

\noindent Define a transitive class\index{transitive class}\index{transitive
set}.  This should not be confused with a transitive relation, which is a different
concept.  Definition from p.~71 of Enderton, extended to classes.

\vskip 0.5ex
\setbox\startprefix=\hbox{\tt \ \ df-tr\ \$a\ }
\setbox\contprefix=\hbox{\tt \ \ \ \ \ \ \ \ \ \ \ }
\startm
\m{\vdash}\m{(}\m{\mbox{\rm Tr}}\m{A}\m{\leftrightarrow}\m{\bigcup}\m{A}\m{
\subseteq}\m{A}\m{)}
\endm
\begin{mmraw}%
|- ( Tr A <-> U. A C\_ A ) \$.
\end{mmraw}

\noindent Define a notation for a general binary relation\index{binary
relation}.  Definition 6.18, p.~29, of Takeuti and Zaring, generalized to
arbitrary classes.  This definition is well-defined, although not very
meaningful, when classes $A$ and/or $B$ are proper.\label{dfbr}  The lack of
parentheses (or any other connective) creates no ambiguity since we are defining
an atomic wff.

\vskip 0.5ex
\setbox\startprefix=\hbox{\tt \ \ df-br\ \$a\ }
\setbox\contprefix=\hbox{\tt \ \ \ \ \ \ \ \ \ \ \ }
\startm
\m{\vdash}\m{(}\m{A}\m{\,R}\m{\,B}\m{\leftrightarrow}\m{\langle}\m{A}\m{,}\m{B}
\m{\rangle}\m{\in}\m{R}\m{)}
\endm
\begin{mmraw}%
|- ( A R B <-> <. A , B >. e. R ) \$.
\end{mmraw}

\noindent Define an abstraction class of ordered pairs\index{abstraction
class!of ordered
pairs}.  A special case of Definition 4.16, p.~14, of Takeuti and Zaring.
Note that $ z $ must be distinct from $ x $ and $ y $,
and $ z $ must not occur in $\varphi$, but $ x $ and $ y $ may be
identical and may appear in $\varphi$.

\vskip 0.5ex
\setbox\startprefix=\hbox{\tt \ \ df-opab\ \$a\ }
\setbox\contprefix=\hbox{\tt \ \ \ \ \ \ \ \ \ \ \ \ \ }
\startm
\m{\vdash}\m{\{}\m{\langle}\m{x}\m{,}\m{y}\m{\rangle}\m{|}\m{\varphi}\m{\}}\m{=}
\m{\{}\m{z}\m{|}\m{\exists}\m{x}\m{\exists}\m{y}\m{(}\m{z}\m{=}\m{\langle}\m{x}
\m{,}\m{y}\m{\rangle}\m{\wedge}\m{\varphi}\m{)}\m{\}}
\endm

\begin{mmraw}%
|- \{ <. x , y >. | ph \} = \{ z | E. x E. y ( z =
<. x , y >. /\ ph ) \} \$.
\end{mmraw}

\noindent Define the epsilon relation\index{epsilon relation}.  Similar to Definition
6.22, p.~30, of Takeuti and Zaring.

\vskip 0.5ex
\setbox\startprefix=\hbox{\tt \ \ df-eprel\ \$a\ }
\setbox\contprefix=\hbox{\tt \ \ \ \ \ \ \ \ \ \ \ \ \ \ }
\startm
\m{\vdash}\m{{\rm E}}\m{=}\m{\{}\m{\langle}\m{x}\m{,}\m{y}\m{\rangle}\m{|}\m{x}\m{
\in}\m{y}\m{\}}
\endm
\begin{mmraw}%
|- \_E = \{ <. x , y >. | x e. y \} \$.
\end{mmraw}

\noindent Define a founded relation\index{founded relation}.  $R$ is a founded
relation on $A$ iff\index{iff} (if and only if) each nonempty subset of $A$
has an ``$R$-minimal element.''  Similar to Definition 6.21, p.~30, of
Takeuti and Zaring.

\vskip 0.5ex
\setbox\startprefix=\hbox{\tt \ \ df-fr\ \$a\ }
\setbox\contprefix=\hbox{\tt \ \ \ \ \ \ \ \ \ \ \ }
\startm
\m{\vdash}\m{(}\m{R}\m{\,\mbox{\rm Fr}}\m{\,A}\m{\leftrightarrow}\m{\forall}\m{x}
\m{(}\m{(}\m{x}\m{\subseteq}\m{A}\m{\wedge}\m{\lnot}\m{x}\m{=}\m{\varnothing}
\m{)}\m{\rightarrow}\m{\exists}\m{y}\m{(}\m{y}\m{\in}\m{x}\m{\wedge}\m{(}\m{x}
\m{\cap}\m{\{}\m{z}\m{|}\m{z}\m{\,R}\m{\,y}\m{\}}\m{)}\m{=}\m{\varnothing}\m{)}
\m{)}\m{)}
\endm
\begin{mmraw}%
|- ( R Fr A <-> A. x ( ( x C\_ A \TAND -. x = (/) ) ->
E. y ( y e. x \TAND ( x i\^{}i \{ z | z R y \} ) = (/) ) ) ) \$.
\end{mmraw}

\noindent Define a well-ordering\index{well-ordering}.  $R$ is a well-ordering of $A$ iff
it is founded on $A$ and the elements of $A$ are pairwise $R$-comparable.
Similar to Definition 6.24(2), p.~30, of Takeuti and Zaring.

\vskip 0.5ex
\setbox\startprefix=\hbox{\tt \ \ df-we\ \$a\ }
\setbox\contprefix=\hbox{\tt \ \ \ \ \ \ \ \ \ \ \ }
\startm
\m{\vdash}\m{(}\m{R}\m{\,\mbox{\rm We}}\m{\,A}\m{\leftrightarrow}\m{(}\m{R}\m{\,
\mbox{\rm Fr}}\m{\,A}\m{\wedge}\m{\forall}\m{x}\m{\forall}\m{y}\m{(}\m{(}\m{x}\m{
\in}\m{A}\m{\wedge}\m{y}\m{\in}\m{A}\m{)}\m{\rightarrow}\m{(}\m{x}\m{\,R}\m{\,y}
\m{\vee}\m{x}\m{=}\m{y}\m{\vee}\m{y}\m{\,R}\m{\,x}\m{)}\m{)}\m{)}\m{)}
\endm
\begin{mmraw}%
( R We A <-> ( R Fr A \TAND A. x A. y ( ( x e.
A \TAND y e. A ) -> ( x R y \TOR x = y \TOR y R x ) ) ) ) \$.
\end{mmraw}

\noindent Define the ordinal predicate\index{ordinal predicate}, which is true for a class
that is transitive and is well-ordered by the epsilon relation.  Similar to
definition on p.~468, Bell and Machover.

\vskip 0.5ex
\setbox\startprefix=\hbox{\tt \ \ df-ord\ \$a\ }
\setbox\contprefix=\hbox{\tt \ \ \ \ \ \ \ \ \ \ \ \ }
\startm
\m{\vdash}\m{(}\m{\mbox{\rm Ord}}\m{\,A}\m{\leftrightarrow}\m{(}
\m{\mbox{\rm Tr}}\m{\,A}\m{\wedge}\m{E}\m{\,\mbox{\rm We}}\m{\,A}\m{)}\m{)}
\endm
\begin{mmraw}%
|- ( Ord A <-> ( Tr A \TAND E We A ) ) \$.
\end{mmraw}

\noindent Define the class of all ordinal numbers\index{ordinal number}.  An ordinal number is
a set that satisfies the ordinal predicate.  Definition 7.11 of Takeuti and
Zaring, p.~38.

\vskip 0.5ex
\setbox\startprefix=\hbox{\tt \ \ df-on\ \$a\ }
\setbox\contprefix=\hbox{\tt \ \ \ \ \ \ \ \ \ \ \ }
\startm
\m{\vdash}\m{\,\mbox{\rm On}}\m{=}\m{\{}\m{x}\m{|}\m{\mbox{\rm Ord}}\m{\,x}
\m{\}}
\endm
\begin{mmraw}%
|- On = \{ x | Ord x \} \$.
\end{mmraw}

\noindent Define the limit ordinal predicate\index{limit ordinal}, which is true for a
non-empty ordinal that is not a successor (i.e.\ that is the union of itself).
Compare Bell and Machover, p.~471 and Exercise (1), p.~42 of Takeuti and
Zaring.

\vskip 0.5ex
\setbox\startprefix=\hbox{\tt \ \ df-lim\ \$a\ }
\setbox\contprefix=\hbox{\tt \ \ \ \ \ \ \ \ \ \ \ \ }
\startm
\m{\vdash}\m{(}\m{\mbox{\rm Lim}}\m{\,A}\m{\leftrightarrow}\m{(}\m{\mbox{
\rm Ord}}\m{\,A}\m{\wedge}\m{\lnot}\m{A}\m{=}\m{\varnothing}\m{\wedge}\m{A}
\m{=}\m{\bigcup}\m{A}\m{)}\m{)}
\endm
\begin{mmraw}%
|- ( Lim A <-> ( Ord A \TAND -. A = (/) \TAND A = U. A ) ) \$.
\end{mmraw}

\noindent Define the successor\index{successor} of a class.  Definition 7.22 of Takeuti
and Zaring, p.~41.  Our definition is a generalization to classes, although it
is meaningless when classes are proper.

\vskip 0.5ex
\setbox\startprefix=\hbox{\tt \ \ df-suc\ \$a\ }
\setbox\contprefix=\hbox{\tt \ \ \ \ \ \ \ \ \ \ \ \ }
\startm
\m{\vdash}\m{\,\mbox{\rm suc}}\m{\,A}\m{=}\m{(}\m{A}\m{\cup}\m{\{}\m{A}\m{\}}
\m{)}
\endm
\begin{mmraw}%
|- suc A = ( A u. \{ A \} ) \$.
\end{mmraw}

\noindent Define the class of natural numbers\index{natural number}\index{omega
($\omega$)}.  Compare Bell and Machover, p.~471.\label{dfom}

\vskip 0.5ex
\setbox\startprefix=\hbox{\tt \ \ df-om\ \$a\ }
\setbox\contprefix=\hbox{\tt \ \ \ \ \ \ \ \ \ \ \ }
\startm
\m{\vdash}\m{\omega}\m{=}\m{\{}\m{x}\m{|}\m{(}\m{\mbox{\rm Ord}}\m{\,x}\m{
\wedge}\m{\forall}\m{y}\m{(}\m{\mbox{\rm Lim}}\m{\,y}\m{\rightarrow}\m{x}\m{
\in}\m{y}\m{)}\m{)}\m{\}}
\endm
\begin{mmraw}%
|- om = \{ x | ( Ord x \TAND A. y ( Lim y -> x e. y ) ) \} \$.
\end{mmraw}

\noindent Define the Cartesian product (also called the
cross product)\index{Cartesian product}\index{cross product}
of two classes.  Definition 9.11 of Quine, p.~64.

\vskip 0.5ex
\setbox\startprefix=\hbox{\tt \ \ df-xp\ \$a\ }
\setbox\contprefix=\hbox{\tt \ \ \ \ \ \ \ \ \ \ \ }
\startm
\m{\vdash}\m{(}\m{A}\m{\times}\m{B}\m{)}\m{=}\m{\{}\m{\langle}\m{x}\m{,}\m{y}
\m{\rangle}\m{|}\m{(}\m{x}\m{\in}\m{A}\m{\wedge}\m{y}\m{\in}\m{B}\m{)}\m{\}}
\endm
\begin{mmraw}%
|- ( A X. B ) = \{ <. x , y >. | ( x e. A \TAND y e. B) \} \$.
\end{mmraw}

\noindent Define a relation\index{relation}.  Definition 6.4(1) of Takeuti and
Zaring, p.~23.

\vskip 0.5ex
\setbox\startprefix=\hbox{\tt \ \ df-rel\ \$a\ }
\setbox\contprefix=\hbox{\tt \ \ \ \ \ \ \ \ \ \ \ \ }
\startm
\m{\vdash}\m{(}\m{\mbox{\rm Rel}}\m{\,A}\m{\leftrightarrow}\m{A}\m{\subseteq}
\m{(}\m{{\rm V}}\m{\times}\m{{\rm V}}\m{)}\m{)}
\endm
\begin{mmraw}%
|- ( Rel A <-> A C\_ ( {\char`\_}V X. {\char`\_}V ) ) \$.
\end{mmraw}

\noindent Define the domain\index{domain} of a class.  Definition 6.5(1) of
Takeuti and Zaring, p.~24.

\vskip 0.5ex
\setbox\startprefix=\hbox{\tt \ \ df-dm\ \$a\ }
\setbox\contprefix=\hbox{\tt \ \ \ \ \ \ \ \ \ \ \ }
\startm
\m{\vdash}\m{\,\mbox{\rm dom}}\m{A}\m{=}\m{\{}\m{x}\m{|}\m{\exists}\m{y}\m{
\langle}\m{x}\m{,}\m{y}\m{\rangle}\m{\in}\m{A}\m{\}}
\endm
\begin{mmraw}%
|- dom A = \{ x | E. y <. x , y >. e. A \} \$.
\end{mmraw}

\noindent Define the range\index{range} of a class.  Definition 6.5(2) of
Takeuti and Zaring, p.~24.

\vskip 0.5ex
\setbox\startprefix=\hbox{\tt \ \ df-rn\ \$a\ }
\setbox\contprefix=\hbox{\tt \ \ \ \ \ \ \ \ \ \ \ }
\startm
\m{\vdash}\m{\,\mbox{\rm ran}}\m{A}\m{=}\m{\{}\m{y}\m{|}\m{\exists}\m{x}\m{
\langle}\m{x}\m{,}\m{y}\m{\rangle}\m{\in}\m{A}\m{\}}
\endm
\begin{mmraw}%
|- ran A = \{ y | E. x <. x , y >. e. A \} \$.
\end{mmraw}

\noindent Define the restriction\index{restriction} of a class.  Definition
6.6(1) of Takeuti and Zaring, p.~24.

\vskip 0.5ex
\setbox\startprefix=\hbox{\tt \ \ df-res\ \$a\ }
\setbox\contprefix=\hbox{\tt \ \ \ \ \ \ \ \ \ \ \ \ }
\startm
\m{\vdash}\m{(}\m{A}\m{\restriction}\m{B}\m{)}\m{=}\m{(}\m{A}\m{\cap}\m{(}\m{B}
\m{\times}\m{{\rm V}}\m{)}\m{)}
\endm
\begin{mmraw}%
|- ( A |` B ) = ( A i\^{}i ( B X. {\char`\_}V ) ) \$.
\end{mmraw}

\noindent Define the image\index{image} of a class.  Definition 6.6(2) of
Takeuti and Zaring, p.~24.

\vskip 0.5ex
\setbox\startprefix=\hbox{\tt \ \ df-ima\ \$a\ }
\setbox\contprefix=\hbox{\tt \ \ \ \ \ \ \ \ \ \ \ \ }
\startm
\m{\vdash}\m{(}\m{A}\m{``}\m{B}\m{)}\m{=}\m{\,\mbox{\rm ran}}\m{\,(}\m{A}\m{
\restriction}\m{B}\m{)}
\endm
\begin{mmraw}%
|- ( A " B ) = ran ( A |` B ) \$.
\end{mmraw}

\noindent Define the composition\index{composition} of two classes.  Definition 6.6(3) of
Takeuti and Zaring, p.~24.

\vskip 0.5ex
\setbox\startprefix=\hbox{\tt \ \ df-co\ \$a\ }
\setbox\contprefix=\hbox{\tt \ \ \ \ \ \ \ \ \ \ \ \ }
\startm
\m{\vdash}\m{(}\m{A}\m{\circ}\m{B}\m{)}\m{=}\m{\{}\m{\langle}\m{x}\m{,}\m{y}\m{
\rangle}\m{|}\m{\exists}\m{z}\m{(}\m{\langle}\m{x}\m{,}\m{z}\m{\rangle}\m{\in}
\m{B}\m{\wedge}\m{\langle}\m{z}\m{,}\m{y}\m{\rangle}\m{\in}\m{A}\m{)}\m{\}}
\endm
\begin{mmraw}%
|- ( A o. B ) = \{ <. x , y >. | E. z ( <. x , z
>. e. B \TAND <. z , y >. e. A ) \} \$.
\end{mmraw}

\noindent Define a function\index{function}.  Definition 6.4(4) of Takeuti and
Zaring, p.~24.

\vskip 0.5ex
\setbox\startprefix=\hbox{\tt \ \ df-fun\ \$a\ }
\setbox\contprefix=\hbox{\tt \ \ \ \ \ \ \ \ \ \ \ \ }
\startm
\m{\vdash}\m{(}\m{\mbox{\rm Fun}}\m{\,A}\m{\leftrightarrow}\m{(}
\m{\mbox{\rm Rel}}\m{\,A}\m{\wedge}
\m{\forall}\m{x}\m{\exists}\m{z}\m{\forall}\m{y}\m{(}
\m{\langle}\m{x}\m{,}\m{y}\m{\rangle}\m{\in}\m{A}\m{\rightarrow}\m{y}\m{=}\m{z}
\m{)}\m{)}\m{)}
\endm
\begin{mmraw}%
|- ( Fun A <-> ( Rel A /\ A. x E. z A. y ( <. x
   , y >. e. A -> y = z ) ) ) \$.
\end{mmraw}

\noindent Define a function with domain.  Definition 6.15(1) of Takeuti and
Zaring, p.~27.

\vskip 0.5ex
\setbox\startprefix=\hbox{\tt \ \ df-fn\ \$a\ }
\setbox\contprefix=\hbox{\tt \ \ \ \ \ \ \ \ \ \ \ }
\startm
\m{\vdash}\m{(}\m{A}\m{\,\mbox{\rm Fn}}\m{\,B}\m{\leftrightarrow}\m{(}
\m{\mbox{\rm Fun}}\m{\,A}\m{\wedge}\m{\mbox{\rm dom}}\m{\,A}\m{=}\m{B}\m{)}
\m{)}
\endm
\begin{mmraw}%
|- ( A Fn B <-> ( Fun A \TAND dom A = B ) ) \$.
\end{mmraw}

\noindent Define a function with domain and co-domain.  Definition 6.15(3)
of Takeuti and Zaring, p.~27.

\vskip 0.5ex
\setbox\startprefix=\hbox{\tt \ \ df-f\ \$a\ }
\setbox\contprefix=\hbox{\tt \ \ \ \ \ \ \ \ \ \ }
\startm
\m{\vdash}\m{(}\m{F}\m{:}\m{A}\m{\longrightarrow}\m{B}\m{
\leftrightarrow}\m{(}\m{F}\m{\,\mbox{\rm Fn}}\m{\,A}\m{\wedge}\m{
\mbox{\rm ran}}\m{\,F}\m{\subseteq}\m{B}\m{)}\m{)}
\endm
\begin{mmraw}%
|- ( F : A --> B <-> ( F Fn A \TAND ran F C\_ B ) ) \$.
\end{mmraw}

\noindent Define a one-to-one function\index{one-to-one function}.  Compare
Definition 6.15(5) of Takeuti and Zaring, p.~27.

\vskip 0.5ex
\setbox\startprefix=\hbox{\tt \ \ df-f1\ \$a\ }
\setbox\contprefix=\hbox{\tt \ \ \ \ \ \ \ \ \ \ \ }
\startm
\m{\vdash}\m{(}\m{F}\m{:}\m{A}\m{
\raisebox{.5ex}{${\textstyle{\:}_{\mbox{\footnotesize\rm
1\tt -\rm 1}}}\atop{\textstyle{
\longrightarrow}\atop{\textstyle{}^{\mbox{\footnotesize\rm {\ }}}}}$}
}\m{B}
\m{\leftrightarrow}\m{(}\m{F}\m{:}\m{A}\m{\longrightarrow}\m{B}
\m{\wedge}\m{\forall}\m{y}\m{\exists}\m{z}\m{\forall}\m{x}\m{(}\m{\langle}\m{x}
\m{,}\m{y}\m{\rangle}\m{\in}\m{F}\m{\rightarrow}\m{x}\m{=}\m{z}\m{)}\m{)}\m{)}
\endm
\begin{mmraw}%
|- ( F : A -1-1-> B <-> ( F : A --> B \TAND
   A. y E. z A. x ( <. x , y >. e. F -> x = z ) ) ) \$.
\end{mmraw}

\noindent Define an onto function\index{onto function}.  Definition 6.15(4) of Takeuti and
Zaring, p.~27.

\vskip 0.5ex
\setbox\startprefix=\hbox{\tt \ \ df-fo\ \$a\ }
\setbox\contprefix=\hbox{\tt \ \ \ \ \ \ \ \ \ \ \ }
\startm
\m{\vdash}\m{(}\m{F}\m{:}\m{A}\m{
\raisebox{.5ex}{${\textstyle{\:}_{\mbox{\footnotesize\rm
{\ }}}}\atop{\textstyle{
\longrightarrow}\atop{\textstyle{}^{\mbox{\footnotesize\rm onto}}}}$}
}\m{B}
\m{\leftrightarrow}\m{(}\m{F}\m{\,\mbox{\rm Fn}}\m{\,A}\m{\wedge}
\m{\mbox{\rm ran}}\m{\,F}\m{=}\m{B}\m{)}\m{)}
\endm
\begin{mmraw}%
|- ( F : A -onto-> B <-> ( F Fn A /\ ran F = B ) ) \$.
\end{mmraw}

\noindent Define a one-to-one, onto function.  Compare Definition 6.15(6) of
Takeuti and Zaring, p.~27.

\vskip 0.5ex
\setbox\startprefix=\hbox{\tt \ \ df-f1o\ \$a\ }
\setbox\contprefix=\hbox{\tt \ \ \ \ \ \ \ \ \ \ \ \ }
\startm
\m{\vdash}\m{(}\m{F}\m{:}\m{A}
\m{
\raisebox{.5ex}{${\textstyle{\:}_{\mbox{\footnotesize\rm
1\tt -\rm 1}}}\atop{\textstyle{
\longrightarrow}\atop{\textstyle{}^{\mbox{\footnotesize\rm onto}}}}$}
}
\m{B}
\m{\leftrightarrow}\m{(}\m{F}\m{:}\m{A}
\m{
\raisebox{.5ex}{${\textstyle{\:}_{\mbox{\footnotesize\rm
1\tt -\rm 1}}}\atop{\textstyle{
\longrightarrow}\atop{\textstyle{}^{\mbox{\footnotesize\rm {\ }}}}}$}
}
\m{B}\m{\wedge}\m{F}\m{:}\m{A}
\m{
\raisebox{.5ex}{${\textstyle{\:}_{\mbox{\footnotesize\rm
{\ }}}}\atop{\textstyle{
\longrightarrow}\atop{\textstyle{}^{\mbox{\footnotesize\rm onto}}}}$}
}
\m{B}\m{)}\m{)}
\endm
\begin{mmraw}%
|- ( F : A -1-1-onto-> B <-> ( F : A -1-1-> B? \TAND F : A -onto-> B ) ) \$.?
\end{mmraw}

\noindent Define the value of a function\index{function value}.  This
definition applies to any class and evaluates to the empty set when it is not
meaningful. Note that $ F`A$ means the same thing as the more familiar $ F(A)$
notation for a function's value at $A$.  The $ F`A$ notation is common in
formal set theory.\label{df-fv}

\vskip 0.5ex
\setbox\startprefix=\hbox{\tt \ \ df-fv\ \$a\ }
\setbox\contprefix=\hbox{\tt \ \ \ \ \ \ \ \ \ \ \ }
\startm
\m{\vdash}\m{(}\m{F}\m{`}\m{A}\m{)}\m{=}\m{\bigcup}\m{\{}\m{x}\m{|}\m{(}\m{F}%
\m{``}\m{\{}\m{A}\m{\}}\m{)}\m{=}\m{\{}\m{x}\m{\}}\m{\}}
\endm
\begin{mmraw}%
|- ( F ` A ) = U. \{ x | ( F " \{ A \} ) = \{ x \} \} \$.
\end{mmraw}

\noindent Define the result of an operation.\index{operation}  Here, $F$ is
     an operation on two
     values (such as $+$ for real numbers).   This is defined for proper
     classes $A$ and $B$ even though not meaningful in that case.  However,
     the definition can be meaningful when $F$ is a proper class.\label{dfopr}

\vskip 0.5ex
\setbox\startprefix=\hbox{\tt \ \ df-opr\ \$a\ }
\setbox\contprefix=\hbox{\tt \ \ \ \ \ \ \ \ \ \ \ \ }
\startm
\m{\vdash}\m{(}\m{A}\m{\,F}\m{\,B}\m{)}\m{=}\m{(}\m{F}\m{`}\m{\langle}\m{A}%
\m{,}\m{B}\m{\rangle}\m{)}
\endm
\begin{mmraw}%
|- ( A F B ) = ( F ` <. A , B >. ) \$.
\end{mmraw}

\section{Tricks of the Trade}\label{tricks}

In the \texttt{set.mm}\index{set theory database (\texttt{set.mm})} database our goal
was usually to conform to modern notation.  However in some cases the
relationship to standard textbook language may be obscured by several
unconventional devices we used to simplify the development and to take
advantage of the Metamath language.  In this section we will describe some
common conventions used in \texttt{set.mm}.

\begin{itemize}
\item
The turnstile\index{turnstile ({$\,\vdash$})} symbol, $\vdash$, meaning ``it
is provable that,'' is the first token of all assertions and hypotheses that
aren't syntax constructions.  This is a standard convention in logic.  (We
mentioned this earlier, but this symbol is bothersome to some people without a
logic background.  It has no deeper meaning but just provides us with a way to
distinguish syntax constructions from ordinary mathematical statements.)

\item
A hypothesis of the form

\vskip 1ex
\setbox\startprefix=\hbox{\tt \ \ \ \ \ \ \ \ \ \$e\ }
\setbox\contprefix=\hbox{\tt \ \ \ \ \ \ \ \ \ \ }
\startm
\m{\vdash}\m{(}\m{\varphi}\m{\rightarrow}\m{\forall}\m{x}\m{\varphi}\m{)}
\endm
\vskip 1ex

should be read ``assume variable $x$ is (effectively) not free in wff
$\varphi$.''\index{effectively not free}
Literally, this says ``assume it is provable that $\varphi \rightarrow \forall
x\, \varphi$.''  This device lets us avoid the complexities associated with
the standard treatment of free and bound variables.
%% Uncomment this when uncommenting section {formalspec} below
The footnote on p.~\pageref{effectivelybound} discusses this further.

\item
A statement of one of the forms

\vskip 1ex
\setbox\startprefix=\hbox{\tt \ \ \ \ \ \ \ \ \ \$a\ }
\setbox\contprefix=\hbox{\tt \ \ \ \ \ \ \ \ \ \ }
\startm
\m{\vdash}\m{(}\m{\lnot}\m{\forall}\m{x}\m{\,x}\m{=}\m{y}\m{\rightarrow}
\m{\ldots}\m{)}
\endm
\setbox\startprefix=\hbox{\tt \ \ \ \ \ \ \ \ \ \$p\ }
\setbox\contprefix=\hbox{\tt \ \ \ \ \ \ \ \ \ \ }
\startm
\m{\vdash}\m{(}\m{\lnot}\m{\forall}\m{x}\m{\,x}\m{=}\m{y}\m{\rightarrow}
\m{\ldots}\m{)}
\endm
\vskip 1ex

should be read ``if $x$ and $y$ are distinct variables, then...''  This
antecedent provides us with a technical device to avoid the need for the
\texttt{\$d} statement early in our development of predicate calculus,
permitting symbol manipulations to be as conceptually simple as those in
propositional calculus.  However, the \texttt{\$d} statement eventually
becomes a requirement, and after that this device is rarely used.

\item
The statement

\vskip 1ex
\setbox\startprefix=\hbox{\tt \ \ \ \ \ \ \ \ \ \$d\ }
\setbox\contprefix=\hbox{\tt \ \ \ \ \ \ \ \ \ \ }
\startm
\m{x}\m{\,y}
\endm
\vskip 1ex

should be read ``assume $x$ and $y$ are distinct variables.''

\item
The statement

\vskip 1ex
\setbox\startprefix=\hbox{\tt \ \ \ \ \ \ \ \ \ \$d\ }
\setbox\contprefix=\hbox{\tt \ \ \ \ \ \ \ \ \ \ }
\startm
\m{x}\m{\,\varphi}
\endm
\vskip 1ex

should be read ``assume $x$ does not occur in $\varphi$.''

\item
The statement

\vskip 1ex
\setbox\startprefix=\hbox{\tt \ \ \ \ \ \ \ \ \ \$d\ }
\setbox\contprefix=\hbox{\tt \ \ \ \ \ \ \ \ \ \ }
\startm
\m{x}\m{\,A}
\endm
\vskip 1ex

should be read ``assume variable $x$ does not occur in class $A$.''

\item
The restriction and hypothesis group

\vskip 1ex
\setbox\startprefix=\hbox{\tt \ \ \ \ \ \ \ \ \ \$d\ }
\setbox\contprefix=\hbox{\tt \ \ \ \ \ \ \ \ \ \ }
\startm
\m{x}\m{\,A}
\endm
\setbox\startprefix=\hbox{\tt \ \ \ \ \ \ \ \ \ \$d\ }
\setbox\contprefix=\hbox{\tt \ \ \ \ \ \ \ \ \ \ }
\startm
\m{x}\m{\,\psi}
\endm
\setbox\startprefix=\hbox{\tt \ \ \ \ \ \ \ \ \ \$e\ }
\setbox\contprefix=\hbox{\tt \ \ \ \ \ \ \ \ \ \ }
\startm
\m{\vdash}\m{(}\m{x}\m{=}\m{A}\m{\rightarrow}\m{(}\m{\varphi}\m{\leftrightarrow}
\m{\psi}\m{)}\m{)}
\endm
\vskip 1ex

is frequently used in place of explicit substitution, meaning ``assume
$\psi$ results from the proper substitution of $A$ for $x$ in
$\varphi$.''  Sometimes ``\texttt{\$e} $\vdash ( \psi \rightarrow
\forall x \, \psi )$'' is used instead of ``\texttt{\$d} $x\, \psi $,''
which requires only that $x$ be effectively not free in $\varphi$ but
not necessarily absent from it.  The use of implicit
substitution\index{substitution!implicit} is partly a matter of personal
style, although it may make proofs somewhat shorter than would be the
case with explicit substitution.

\item
The hypothesis


\vskip 1ex
\setbox\startprefix=\hbox{\tt \ \ \ \ \ \ \ \ \ \$e\ }
\setbox\contprefix=\hbox{\tt \ \ \ \ \ \ \ \ \ \ }
\startm
\m{\vdash}\m{A}\m{\in}\m{{\rm V}}
\endm
\vskip 1ex

should be read ``assume class $A$ is a set (i.e.\ exists).''
This is a convenient convention used by Quine.

\item
The restriction and hypothesis

\vskip 1ex
\setbox\startprefix=\hbox{\tt \ \ \ \ \ \ \ \ \ \$d\ }
\setbox\contprefix=\hbox{\tt \ \ \ \ \ \ \ \ \ \ }
\startm
\m{x}\m{\,y}
\endm
\setbox\startprefix=\hbox{\tt \ \ \ \ \ \ \ \ \ \$e\ }
\setbox\contprefix=\hbox{\tt \ \ \ \ \ \ \ \ \ \ }
\startm
\m{\vdash}\m{(}\m{y}\m{\in}\m{A}\m{\rightarrow}\m{\forall}\m{x}\m{\,y}
\m{\in}\m{A}\m{)}
\endm
\vskip 1ex

should be read ``assume variable $x$ is
(effectively) not free in class $A$.''

\end{itemize}

\section{A Theorem Sampler}\label{sometheorems}

In this section we list some of the more important theorems that are proved in
the \texttt{set.mm} database, and they illustrate the kinds of things that can be
done with Metamath.  While all of these facts are well-known results,
Metamath offers the advantage of easily allowing you to trace their
derivation back to axioms.  Our intent here is not to try to explain the
details or motivation; for this we refer you to the textbooks that are
mentioned in the descriptions.  (The \texttt{set.mm} file has bibliographic
references for the text references.)  Their proofs often embody important
concepts you may wish to explore with the Metamath program (see
Section~\ref{exploring}).  All the symbols that are used here are defined in
Section~\ref{hierarchy}.  For brevity we haven't included the \texttt{\$d}
restrictions or \texttt{\$f} hypotheses for these theorems; when you are
uncertain consult the \texttt{set.mm} database.

We start with \texttt{syl} (principle of the syllogism).
In \textit{Principia Mathematica}
Whitehead and Russell call this ``the principle of the
syllogism... because... the syllogism in Barbara is derived from them''
\cite[quote after Theorem *2.06 p.~101]{PM}.
Some authors call this law a ``hypothetical syllogism.''
As of 2019 \texttt{syl} is the most commonly referenced proven
assertion in the \texttt{set.mm} database.\footnote{
The Metamath program command \texttt{show usage}
shows the number of references.
On 2019-04-29 (commit 71cbbbdb387e) \texttt{syl} was directly referenced
10,819 times. The second most commonly referenced proven assertion
was \texttt{eqid}, which was directly referenced 7,738 times.
}

\vskip 2ex
\noindent Theorem syl (principle of the syllogism)\index{Syllogism}%
\index{\texttt{syl}}\label{syl}.

\vskip 0.5ex
\setbox\startprefix=\hbox{\tt \ \ syl.1\ \$e\ }
\setbox\contprefix=\hbox{\tt \ \ \ \ \ \ \ \ \ \ \ }
\startm
\m{\vdash}\m{(}\m{\varphi}\m{ \rightarrow }\m{\psi}\m{)}
\endm
\setbox\startprefix=\hbox{\tt \ \ syl.2\ \$e\ }
\setbox\contprefix=\hbox{\tt \ \ \ \ \ \ \ \ \ \ \ }
\startm
\m{\vdash}\m{(}\m{\psi}\m{ \rightarrow }\m{\chi}\m{)}
\endm
\setbox\startprefix=\hbox{\tt \ \ syl\ \$p\ }
\setbox\contprefix=\hbox{\tt \ \ \ \ \ \ \ \ \ }
\startm
\m{\vdash}\m{(}\m{\varphi}\m{ \rightarrow }\m{\chi}\m{)}
\endm
\vskip 2ex

The following theorem is not very deep but provides us with a notational device
that is frequently used.  It allows us to use the expression ``$A \in V$'' as
a compact way of saying that class $A$ exists, i.e.\ is a set.

\vskip 2ex
\noindent Two ways to say ``$A$ is a set'':  $A$ is a member of the universe
$V$ if and only if $A$ exists (i.e.\ there exists a set equal to $A$).
Theorem 6.9 of Quine, p. 43.

\vskip 0.5ex
\setbox\startprefix=\hbox{\tt \ \ isset\ \$p\ }
\setbox\contprefix=\hbox{\tt \ \ \ \ \ \ \ \ \ \ \ }
\startm
\m{\vdash}\m{(}\m{A}\m{\in}\m{{\rm V}}\m{\leftrightarrow}\m{\exists}\m{x}\m{\,x}\m{=}
\m{A}\m{)}
\endm
\vskip 1ex

Next we prove the axioms of standard ZF set theory that were missing from our
axiom system.  From our point of view they are theorems since they
can be derived from the other axioms.

\vskip 2ex
\noindent Axiom of Separation\index{Axiom of Separation}
(Aussonderung)\index{Aussonderung} proved from the other axioms of ZF set
theory.  Compare Exercise 4 of Takeuti and Zaring, p.~22.

\vskip 0.5ex
\setbox\startprefix=\hbox{\tt \ \ inex1.1\ \$e\ }
\setbox\contprefix=\hbox{\tt \ \ \ \ \ \ \ \ \ \ \ \ \ \ \ }
\startm
\m{\vdash}\m{A}\m{\in}\m{{\rm V}}
\endm
\setbox\startprefix=\hbox{\tt \ \ inex\ \$p\ }
\setbox\contprefix=\hbox{\tt \ \ \ \ \ \ \ \ \ \ \ \ \ }
\startm
\m{\vdash}\m{(}\m{A}\m{\cap}\m{B}\m{)}\m{\in}\m{{\rm V}}
\endm
\vskip 1ex

\noindent Axiom of the Null Set\index{Axiom of the Null Set} proved from the
other axioms of ZF set theory. Corollary 5.16 of Takeuti and Zaring, p.~20.

\vskip 0.5ex
\setbox\startprefix=\hbox{\tt \ \ 0ex\ \$p\ }
\setbox\contprefix=\hbox{\tt \ \ \ \ \ \ \ \ \ \ \ \ }
\startm
\m{\vdash}\m{\varnothing}\m{\in}\m{{\rm V}}
\endm
\vskip 1ex

\noindent The Axiom of Pairing\index{Axiom of Pairing} proved from the other
axioms of ZF set theory.  Theorem 7.13 of Quine, p.~51.
\vskip 0.5ex
\setbox\startprefix=\hbox{\tt \ \ prex\ \$p\ }
\setbox\contprefix=\hbox{\tt \ \ \ \ \ \ \ \ \ \ \ \ \ \ }
\startm
\m{\vdash}\m{\{}\m{A}\m{,}\m{B}\m{\}}\m{\in}\m{{\rm V}}
\endm
\vskip 2ex

Next we will list some famous or important theorems that are proved in
the \texttt{set.mm} database.  None of them except \texttt{omex}
require the Axiom of Infinity, as you can verify with the \texttt{show
trace{\char`\_}back} Metamath command.

\vskip 2ex
\noindent The resolution of Russell's paradox\index{Russell's paradox}.  There
exists no set containing the set of all sets which are not members of
themselves.  Proposition 4.14 of Takeuti and Zaring, p.~14.

\vskip 0.5ex
\setbox\startprefix=\hbox{\tt \ \ ru\ \$p\ }
\setbox\contprefix=\hbox{\tt \ \ \ \ \ \ \ \ }
\startm
\m{\vdash}\m{\lnot}\m{\exists}\m{x}\m{\,x}\m{=}\m{\{}\m{y}\m{|}\m{\lnot}\m{y}
\m{\in}\m{y}\m{\}}
\endm
\vskip 1ex

\noindent Cantor's theorem\index{Cantor's theorem}.  No set can be mapped onto
its power set.  Compare Theorem 6B(b) of Enderton, p.~132.

\vskip 0.5ex
\setbox\startprefix=\hbox{\tt \ \ canth.1\ \$e\ }
\setbox\contprefix=\hbox{\tt \ \ \ \ \ \ \ \ \ \ \ \ \ }
\startm
\m{\vdash}\m{A}\m{\in}\m{{\rm V}}
\endm
\setbox\startprefix=\hbox{\tt \ \ canth\ \$p\ }
\setbox\contprefix=\hbox{\tt \ \ \ \ \ \ \ \ \ \ \ }
\startm
\m{\vdash}\m{\lnot}\m{F}\m{:}\m{A}\m{\raisebox{.5ex}{${\textstyle{\:}_{
\mbox{\footnotesize\rm {\ }}}}\atop{\textstyle{\longrightarrow}\atop{
\textstyle{}^{\mbox{\footnotesize\rm onto}}}}$}}\m{{\cal P}}\m{A}
\endm
\vskip 1ex

\noindent The Burali-Forti paradox\index{Burali-Forti paradox}.  No set
contains all ordinal numbers. Enderton, p.~194.  (Burali-Forti was one person,
not two.)

\vskip 0.5ex
\setbox\startprefix=\hbox{\tt \ \ onprc\ \$p\ }
\setbox\contprefix=\hbox{\tt \ \ \ \ \ \ \ \ \ \ \ \ }
\startm
\m{\vdash}\m{\lnot}\m{\mbox{\rm On}}\m{\in}\m{{\rm V}}
\endm
\vskip 1ex

\noindent Peano's postulates\index{Peano's postulates} for arithmetic.
Proposition 7.30 of Takeuti and Zaring, pp.~42--43.  The objects being
described are the members of $\omega$ i.e.\ the natural numbers 0, 1,
2,\ldots.  The successor\index{successor} operation suc means ``plus
one.''  \texttt{peano1} says that 0 (which is defined as the empty set)
is a natural number.  \texttt{peano2} says that if $A$ is a natural
number, so is $A+1$.  \texttt{peano3} says that 0 is not the successor
of any natural number.  \texttt{peano4} says that two natural numbers
are equal if and only if their successors are equal.  \texttt{peano5} is
essentially the same as mathematical induction.

\vskip 1ex
\setbox\startprefix=\hbox{\tt \ \ peano1\ \$p\ }
\setbox\contprefix=\hbox{\tt \ \ \ \ \ \ \ \ \ \ \ \ }
\startm
\m{\vdash}\m{\varnothing}\m{\in}\m{\omega}
\endm
\vskip 1.5ex

\setbox\startprefix=\hbox{\tt \ \ peano2\ \$p\ }
\setbox\contprefix=\hbox{\tt \ \ \ \ \ \ \ \ \ \ \ \ }
\startm
\m{\vdash}\m{(}\m{A}\m{\in}\m{\omega}\m{\rightarrow}\m{{\rm suc}}\m{A}\m{\in}%
\m{\omega}\m{)}
\endm
\vskip 1.5ex

\setbox\startprefix=\hbox{\tt \ \ peano3\ \$p\ }
\setbox\contprefix=\hbox{\tt \ \ \ \ \ \ \ \ \ \ \ \ }
\startm
\m{\vdash}\m{(}\m{A}\m{\in}\m{\omega}\m{\rightarrow}\m{\lnot}\m{{\rm suc}}%
\m{A}\m{=}\m{\varnothing}\m{)}
\endm
\vskip 1.5ex

\setbox\startprefix=\hbox{\tt \ \ peano4\ \$p\ }
\setbox\contprefix=\hbox{\tt \ \ \ \ \ \ \ \ \ \ \ \ }
\startm
\m{\vdash}\m{(}\m{(}\m{A}\m{\in}\m{\omega}\m{\wedge}\m{B}\m{\in}\m{\omega}%
\m{)}\m{\rightarrow}\m{(}\m{{\rm suc}}\m{A}\m{=}\m{{\rm suc}}\m{B}\m{%
\leftrightarrow}\m{A}\m{=}\m{B}\m{)}\m{)}
\endm
\vskip 1.5ex

\setbox\startprefix=\hbox{\tt \ \ peano5\ \$p\ }
\setbox\contprefix=\hbox{\tt \ \ \ \ \ \ \ \ \ \ \ \ }
\startm
\m{\vdash}\m{(}\m{(}\m{\varnothing}\m{\in}\m{A}\m{\wedge}\m{\forall}\m{x}\m{%
\in}\m{\omega}\m{(}\m{x}\m{\in}\m{A}\m{\rightarrow}\m{{\rm suc}}\m{x}\m{\in}%
\m{A}\m{)}\m{)}\m{\rightarrow}\m{\omega}\m{\subseteq}\m{A}\m{)}
\endm
\vskip 1.5ex

\noindent Finite Induction (mathematical induction).\index{finite
induction}\index{mathematical induction} The first hypothesis is the
basis and the second is the induction hypothesis.  Theorem Schema 22 of
Suppes, p.~136.

\vskip 0.5ex
\setbox\startprefix=\hbox{\tt \ \ findes.1\ \$e\ }
\setbox\contprefix=\hbox{\tt \ \ \ \ \ \ \ \ \ \ \ \ \ \ }
\startm
\m{\vdash}\m{[}\m{\varnothing}\m{/}\m{x}\m{]}\m{\varphi}
\endm
\setbox\startprefix=\hbox{\tt \ \ findes.2\ \$e\ }
\setbox\contprefix=\hbox{\tt \ \ \ \ \ \ \ \ \ \ \ \ \ \ }
\startm
\m{\vdash}\m{(}\m{x}\m{\in}\m{\omega}\m{\rightarrow}\m{(}\m{\varphi}\m{%
\rightarrow}\m{[}\m{{\rm suc}}\m{x}\m{/}\m{x}\m{]}\m{\varphi}\m{)}\m{)}
\endm
\setbox\startprefix=\hbox{\tt \ \ findes\ \$p\ }
\setbox\contprefix=\hbox{\tt \ \ \ \ \ \ \ \ \ \ \ \ }
\startm
\m{\vdash}\m{(}\m{x}\m{\in}\m{\omega}\m{\rightarrow}\m{\varphi}\m{)}
\endm
\vskip 1ex

\noindent Transfinite Induction with explicit substitution.  The first
hypothesis is the basis, the second is the induction hypothesis for
successors, and the third is the induction hypothesis for limit
ordinals.  Theorem Schema 4 of Suppes, p. 197.

\vskip 0.5ex
\setbox\startprefix=\hbox{\tt \ \ tfindes.1\ \$e\ }
\setbox\contprefix=\hbox{\tt \ \ \ \ \ \ \ \ \ \ \ \ \ \ \ }
\startm
\m{\vdash}\m{[}\m{\varnothing}\m{/}\m{x}\m{]}\m{\varphi}
\endm
\setbox\startprefix=\hbox{\tt \ \ tfindes.2\ \$e\ }
\setbox\contprefix=\hbox{\tt \ \ \ \ \ \ \ \ \ \ \ \ \ \ \ }
\startm
\m{\vdash}\m{(}\m{x}\m{\in}\m{{\rm On}}\m{\rightarrow}\m{(}\m{\varphi}\m{%
\rightarrow}\m{[}\m{{\rm suc}}\m{x}\m{/}\m{x}\m{]}\m{\varphi}\m{)}\m{)}
\endm
\setbox\startprefix=\hbox{\tt \ \ tfindes.3\ \$e\ }
\setbox\contprefix=\hbox{\tt \ \ \ \ \ \ \ \ \ \ \ \ \ \ \ }
\startm
\m{\vdash}\m{(}\m{{\rm Lim}}\m{y}\m{\rightarrow}\m{(}\m{\forall}\m{x}\m{\in}%
\m{y}\m{\varphi}\m{\rightarrow}\m{[}\m{y}\m{/}\m{x}\m{]}\m{\varphi}\m{)}\m{)}
\endm
\setbox\startprefix=\hbox{\tt \ \ tfindes\ \$p\ }
\setbox\contprefix=\hbox{\tt \ \ \ \ \ \ \ \ \ \ \ \ \ }
\startm
\m{\vdash}\m{(}\m{x}\m{\in}\m{{\rm On}}\m{\rightarrow}\m{\varphi}\m{)}
\endm
\vskip 1ex

\noindent Principle of Transfinite Recursion.\index{transfinite
recursion} Theorem 7.41 of Takeuti and Zaring, p.~47.  Transfinite
recursion is the key theorem that allows arithmetic of ordinals to be
rigorously defined, and has many other important uses as well.
Hypotheses \texttt{tfr.1} and \texttt{tfr.2} specify a certain (proper)
class $ F$.  The complicated definition of $ F$ is not important in
itself; what is important is that there be such an $ F$ with the
required properties, and we show this by displaying $ F$ explicitly.
\texttt{tfr1} states that $ F$ is a function whose domain is the set of
ordinal numbers.  \texttt{tfr2} states that any value of $ F$ is
completely determined by its previous values and the values of an
auxiliary function, $G$.  \texttt{tfr3} states that $ F$ is unique,
i.e.\ it is the only function that satisfies \texttt{tfr1} and
\texttt{tfr2}.  Note that $ f$ is an individual variable like $x$ and
$y$; it is just a mnemonic to remind us that $A$ is a collection of
functions.

\vskip 0.5ex
\setbox\startprefix=\hbox{\tt \ \ tfr.1\ \$e\ }
\setbox\contprefix=\hbox{\tt \ \ \ \ \ \ \ \ \ \ \ }
\startm
\m{\vdash}\m{A}\m{=}\m{\{}\m{f}\m{|}\m{\exists}\m{x}\m{\in}\m{{\rm On}}\m{(}%
\m{f}\m{{\rm Fn}}\m{x}\m{\wedge}\m{\forall}\m{y}\m{\in}\m{x}\m{(}\m{f}\m{`}%
\m{y}\m{)}\m{=}\m{(}\m{G}\m{`}\m{(}\m{f}\m{\restriction}\m{y}\m{)}\m{)}\m{)}%
\m{\}}
\endm
\setbox\startprefix=\hbox{\tt \ \ tfr.2\ \$e\ }
\setbox\contprefix=\hbox{\tt \ \ \ \ \ \ \ \ \ \ \ }
\startm
\m{\vdash}\m{F}\m{=}\m{\bigcup}\m{A}
\endm
\setbox\startprefix=\hbox{\tt \ \ tfr1\ \$p\ }
\setbox\contprefix=\hbox{\tt \ \ \ \ \ \ \ \ \ \ }
\startm
\m{\vdash}\m{F}\m{{\rm Fn}}\m{{\rm On}}
\endm
\setbox\startprefix=\hbox{\tt \ \ tfr2\ \$p\ }
\setbox\contprefix=\hbox{\tt \ \ \ \ \ \ \ \ \ \ }
\startm
\m{\vdash}\m{(}\m{z}\m{\in}\m{{\rm On}}\m{\rightarrow}\m{(}\m{F}\m{`}\m{z}%
\m{)}\m{=}\m{(}\m{G}\m{`}\m{(}\m{F}\m{\restriction}\m{z}\m{)}\m{)}\m{)}
\endm
\setbox\startprefix=\hbox{\tt \ \ tfr3\ \$p\ }
\setbox\contprefix=\hbox{\tt \ \ \ \ \ \ \ \ \ \ }
\startm
\m{\vdash}\m{(}\m{(}\m{B}\m{{\rm Fn}}\m{{\rm On}}\m{\wedge}\m{\forall}\m{x}\m{%
\in}\m{{\rm On}}\m{(}\m{B}\m{`}\m{x}\m{)}\m{=}\m{(}\m{G}\m{`}\m{(}\m{B}\m{%
\restriction}\m{x}\m{)}\m{)}\m{)}\m{\rightarrow}\m{B}\m{=}\m{F}\m{)}
\endm
\vskip 1ex

\noindent The existence of omega (the class of natural numbers).\index{natural
number}\index{omega ($\omega$)}\index{Axiom of Infinity}  Axiom 7 of Takeuti
and Zaring, p.~43.  (This is the only theorem in this section requiring the
Axiom of Infinity.)

\vskip 0.5ex
\setbox\startprefix=\hbox{\tt \
\ omex\ \$p\ }
\setbox\contprefix=\hbox{\tt \ \ \ \ \ \ \ \ \ \ }
\startm
\m{\vdash}\m{\omega}\m{\in}\m{{\rm V}}
\endm
%\vskip 2ex


\section{Axioms for Real and Complex Numbers}\label{real}
\index{real and complex numbers!axioms for}

This section presents the axioms for real and complex numbers, along
with some commentary about them.  Analysis
textbooks implicitly or explicitly use these axioms or their equivalents
as their starting point.  In the database \texttt{set.mm}, we define real
and complex numbers as (rather complicated) specific sets and derive these
axioms as {\em theorems} from the axioms of ZF set theory, using a method
called Dedekind cuts.  We omit the details of this construction, which you can
follow if you wish using the \texttt{set.mm} database in conjunction with the
textbooks referenced therein.

Once we prove those theorems, we then restate these proven theorems as axioms.
This lets us easily identify which axioms are needed for a particular complex number proof, without the obfuscation of the set theory used to derive them.
As a result,
the construction is actually unimportant other
than to show that sets exist that satisfy the axioms, and thus that the axioms
are consistent if ZF set theory is consistent.  When working with real numbers
you can think of them as being the actual sets resulting
from the construction (for definiteness), or you can
think of them as otherwise unspecified sets that happen to satisfy the axioms.
The derivation is not easy, but the fact that it works is quite remarkable
and lends support to the idea that ZFC set theory is all we need to
provide a foundation for essentially all of mathematics.

\needspace{3\baselineskip}
\subsection{The Axioms for Real and Complex Numbers Themselves}\label{realactual}

For the axioms we are given (or postulate) 8 classes:  $\mathbb{C}$ (the
set of complex numbers), $\mathbb{R}$ (the set of real numbers, a subset
of $\mathbb{C}$), $0$ (zero), $1$ (one), $i$ (square root of
$-1$), $+$ (plus), $\cdot$ (times), and
$<_{\mathbb{R}}$ (less than for just the real numbers).
Subtraction and division are defined terms and are not part of the
axioms; for their definitions see \texttt{set.mm}.

Note that the notation $(A+B)$ (and similarly $(A\cdot B)$) specifies a class
called an {\em operation},\index{operation} and is the function value of the
class $+$ at ordered pair $\langle A,B \rangle$.  An operation is defined by
statement \texttt{df-opr} on p.~\pageref{dfopr}.
The notation $A <_{\mathbb{R}} B$ specifies a
wff called a {\em binary relation}\index{binary relation} and means $\langle A,B \rangle \in \,<_{\mathbb{R}}$, as defined by statement \texttt{df-br} on p.~\pageref{dfbr}.

Our set of 8 given classes is assumed to satisfy the following 22 axioms
(in the axioms listed below, $<$ really means $<_{\mathbb{R}}$).

\vskip 2ex

\noindent 1. The real numbers are a subset of the complex numbers.

%\vskip 0.5ex
\setbox\startprefix=\hbox{\tt \ \ ax-resscn\ \$p\ }
\setbox\contprefix=\hbox{\tt \ \ \ \ \ \ \ \ \ \ \ \ \ \ }
\startm
\m{\vdash}\m{\mathbb{R}}\m{\subseteq}\m{\mathbb{C}}
\endm
%\vskip 1ex

\noindent 2. One is a complex number.

%\vskip 0.5ex
\setbox\startprefix=\hbox{\tt \ \ ax-1cn\ \$p\ }
\setbox\contprefix=\hbox{\tt \ \ \ \ \ \ \ \ \ \ \ }
\startm
\m{\vdash}\m{1}\m{\in}\m{\mathbb{C}}
\endm
%\vskip 1ex

\noindent 3. The imaginary number $i$ is a complex number.

%\vskip 0.5ex
\setbox\startprefix=\hbox{\tt \ \ ax-icn\ \$p\ }
\setbox\contprefix=\hbox{\tt \ \ \ \ \ \ \ \ \ \ \ }
\startm
\m{\vdash}\m{i}\m{\in}\m{\mathbb{C}}
\endm
%\vskip 1ex

\noindent 4. Complex numbers are closed under addition.

%\vskip 0.5ex
\setbox\startprefix=\hbox{\tt \ \ ax-addcl\ \$p\ }
\setbox\contprefix=\hbox{\tt \ \ \ \ \ \ \ \ \ \ \ \ \ }
\startm
\m{\vdash}\m{(}\m{(}\m{A}\m{\in}\m{\mathbb{C}}\m{\wedge}\m{B}\m{\in}\m{\mathbb{C}}%
\m{)}\m{\rightarrow}\m{(}\m{A}\m{+}\m{B}\m{)}\m{\in}\m{\mathbb{C}}\m{)}
\endm
%\vskip 1ex

\noindent 5. Real numbers are closed under addition.

%\vskip 0.5ex
\setbox\startprefix=\hbox{\tt \ \ ax-addrcl\ \$p\ }
\setbox\contprefix=\hbox{\tt \ \ \ \ \ \ \ \ \ \ \ \ \ \ }
\startm
\m{\vdash}\m{(}\m{(}\m{A}\m{\in}\m{\mathbb{R}}\m{\wedge}\m{B}\m{\in}\m{\mathbb{R}}%
\m{)}\m{\rightarrow}\m{(}\m{A}\m{+}\m{B}\m{)}\m{\in}\m{\mathbb{R}}\m{)}
\endm
%\vskip 1ex

\noindent 6. Complex numbers are closed under multiplication.

%\vskip 0.5ex
\setbox\startprefix=\hbox{\tt \ \ ax-mulcl\ \$p\ }
\setbox\contprefix=\hbox{\tt \ \ \ \ \ \ \ \ \ \ \ \ \ }
\startm
\m{\vdash}\m{(}\m{(}\m{A}\m{\in}\m{\mathbb{C}}\m{\wedge}\m{B}\m{\in}\m{\mathbb{C}}%
\m{)}\m{\rightarrow}\m{(}\m{A}\m{\cdot}\m{B}\m{)}\m{\in}\m{\mathbb{C}}\m{)}
\endm
%\vskip 1ex

\noindent 7. Real numbers are closed under multiplication.

%\vskip 0.5ex
\setbox\startprefix=\hbox{\tt \ \ ax-mulrcl\ \$p\ }
\setbox\contprefix=\hbox{\tt \ \ \ \ \ \ \ \ \ \ \ \ \ \ }
\startm
\m{\vdash}\m{(}\m{(}\m{A}\m{\in}\m{\mathbb{R}}\m{\wedge}\m{B}\m{\in}\m{\mathbb{R}}%
\m{)}\m{\rightarrow}\m{(}\m{A}\m{\cdot}\m{B}\m{)}\m{\in}\m{\mathbb{R}}\m{)}
\endm
%\vskip 1ex

\noindent 8. Multiplication of complex numbers is commutative.

%\vskip 0.5ex
\setbox\startprefix=\hbox{\tt \ \ ax-mulcom\ \$p\ }
\setbox\contprefix=\hbox{\tt \ \ \ \ \ \ \ \ \ \ \ \ \ \ }
\startm
\m{\vdash}\m{(}\m{(}\m{A}\m{\in}\m{\mathbb{C}}\m{\wedge}\m{B}\m{\in}\m{\mathbb{C}}%
\m{)}\m{\rightarrow}\m{(}\m{A}\m{\cdot}\m{B}\m{)}\m{=}\m{(}\m{B}\m{\cdot}\m{A}%
\m{)}\m{)}
\endm
%\vskip 1ex

\noindent 9. Addition of complex numbers is associative.

%\vskip 0.5ex
\setbox\startprefix=\hbox{\tt \ \ ax-addass\ \$p\ }
\setbox\contprefix=\hbox{\tt \ \ \ \ \ \ \ \ \ \ \ \ \ \ }
\startm
\m{\vdash}\m{(}\m{(}\m{A}\m{\in}\m{\mathbb{C}}\m{\wedge}\m{B}\m{\in}\m{\mathbb{C}}%
\m{\wedge}\m{C}\m{\in}\m{\mathbb{C}}\m{)}\m{\rightarrow}\m{(}\m{(}\m{A}\m{+}%
\m{B}\m{)}\m{+}\m{C}\m{)}\m{=}\m{(}\m{A}\m{+}\m{(}\m{B}\m{+}\m{C}\m{)}\m{)}%
\m{)}
\endm
%\vskip 1ex

\noindent 10. Multiplication of complex numbers is associative.

%\vskip 0.5ex
\setbox\startprefix=\hbox{\tt \ \ ax-mulass\ \$p\ }
\setbox\contprefix=\hbox{\tt \ \ \ \ \ \ \ \ \ \ \ \ \ \ }
\startm
\m{\vdash}\m{(}\m{(}\m{A}\m{\in}\m{\mathbb{C}}\m{\wedge}\m{B}\m{\in}\m{\mathbb{C}}%
\m{\wedge}\m{C}\m{\in}\m{\mathbb{C}}\m{)}\m{\rightarrow}\m{(}\m{(}\m{A}\m{\cdot}%
\m{B}\m{)}\m{\cdot}\m{C}\m{)}\m{=}\m{(}\m{A}\m{\cdot}\m{(}\m{B}\m{\cdot}\m{C}%
\m{)}\m{)}\m{)}
\endm
%\vskip 1ex

\noindent 11. Multiplication distributes over addition for complex numbers.

%\vskip 0.5ex
\setbox\startprefix=\hbox{\tt \ \ ax-distr\ \$p\ }
\setbox\contprefix=\hbox{\tt \ \ \ \ \ \ \ \ \ \ \ \ \ }
\startm
\m{\vdash}\m{(}\m{(}\m{A}\m{\in}\m{\mathbb{C}}\m{\wedge}\m{B}\m{\in}\m{\mathbb{C}}%
\m{\wedge}\m{C}\m{\in}\m{\mathbb{C}}\m{)}\m{\rightarrow}\m{(}\m{A}\m{\cdot}\m{(}%
\m{B}\m{+}\m{C}\m{)}\m{)}\m{=}\m{(}\m{(}\m{A}\m{\cdot}\m{B}\m{)}\m{+}\m{(}%
\m{A}\m{\cdot}\m{C}\m{)}\m{)}\m{)}
\endm
%\vskip 1ex

\noindent 12. The square of $i$ equals $-1$ (expressed as $i$-squared plus 1 is
0).

%\vskip 0.5ex
\setbox\startprefix=\hbox{\tt \ \ ax-i2m1\ \$p\ }
\setbox\contprefix=\hbox{\tt \ \ \ \ \ \ \ \ \ \ \ \ }
\startm
\m{\vdash}\m{(}\m{(}\m{i}\m{\cdot}\m{i}\m{)}\m{+}\m{1}\m{)}\m{=}\m{0}
\endm
%\vskip 1ex

\noindent 13. One and zero are distinct.

%\vskip 0.5ex
\setbox\startprefix=\hbox{\tt \ \ ax-1ne0\ \$p\ }
\setbox\contprefix=\hbox{\tt \ \ \ \ \ \ \ \ \ \ \ \ }
\startm
\m{\vdash}\m{1}\m{\ne}\m{0}
\endm
%\vskip 1ex

\noindent 14. One is an identity element for real multiplication.

%\vskip 0.5ex
\setbox\startprefix=\hbox{\tt \ \ ax-1rid\ \$p\ }
\setbox\contprefix=\hbox{\tt \ \ \ \ \ \ \ \ \ \ \ }
\startm
\m{\vdash}\m{(}\m{A}\m{\in}\m{\mathbb{R}}\m{\rightarrow}\m{(}\m{A}\m{\cdot}\m{1}%
\m{)}\m{=}\m{A}\m{)}
\endm
%\vskip 1ex

\noindent 15. Every real number has a negative.

%\vskip 0.5ex
\setbox\startprefix=\hbox{\tt \ \ ax-rnegex\ \$p\ }
\setbox\contprefix=\hbox{\tt \ \ \ \ \ \ \ \ \ \ \ \ \ \ }
\startm
\m{\vdash}\m{(}\m{A}\m{\in}\m{\mathbb{R}}\m{\rightarrow}\m{\exists}\m{x}\m{\in}%
\m{\mathbb{R}}\m{(}\m{A}\m{+}\m{x}\m{)}\m{=}\m{0}\m{)}
\endm
%\vskip 1ex

\noindent 16. Every nonzero real number has a reciprocal.

%\vskip 0.5ex
\setbox\startprefix=\hbox{\tt \ \ ax-rrecex\ \$p\ }
\setbox\contprefix=\hbox{\tt \ \ \ \ \ \ \ \ \ \ \ \ \ \ }
\startm
\m{\vdash}\m{(}\m{A}\m{\in}\m{\mathbb{R}}\m{\rightarrow}\m{(}\m{A}\m{\ne}\m{0}%
\m{\rightarrow}\m{\exists}\m{x}\m{\in}\m{\mathbb{R}}\m{(}\m{A}\m{\cdot}%
\m{x}\m{)}\m{=}\m{1}\m{)}\m{)}
\endm
%\vskip 1ex

\noindent 17. A complex number can be expressed in terms of two reals.

%\vskip 0.5ex
\setbox\startprefix=\hbox{\tt \ \ ax-cnre\ \$p\ }
\setbox\contprefix=\hbox{\tt \ \ \ \ \ \ \ \ \ \ \ \ }
\startm
\m{\vdash}\m{(}\m{A}\m{\in}\m{\mathbb{C}}\m{\rightarrow}\m{\exists}\m{x}\m{\in}%
\m{\mathbb{R}}\m{\exists}\m{y}\m{\in}\m{\mathbb{R}}\m{A}\m{=}\m{(}\m{x}\m{+}\m{(}%
\m{y}\m{\cdot}\m{i}\m{)}\m{)}\m{)}
\endm
%\vskip 1ex

\noindent 18. Ordering on reals satisfies strict trichotomy.

%\vskip 0.5ex
\setbox\startprefix=\hbox{\tt \ \ ax-pre-lttri\ \$p\ }
\setbox\contprefix=\hbox{\tt \ \ \ \ \ \ \ \ \ \ \ \ \ }
\startm
\m{\vdash}\m{(}\m{(}\m{A}\m{\in}\m{\mathbb{R}}\m{\wedge}\m{B}\m{\in}\m{\mathbb{R}}%
\m{)}\m{\rightarrow}\m{(}\m{A}\m{<}\m{B}\m{\leftrightarrow}\m{\lnot}\m{(}\m{A}%
\m{=}\m{B}\m{\vee}\m{B}\m{<}\m{A}\m{)}\m{)}\m{)}
\endm
%\vskip 1ex

\noindent 19. Ordering on reals is transitive.

%\vskip 0.5ex
\setbox\startprefix=\hbox{\tt \ \ ax-pre-lttrn\ \$p\ }
\setbox\contprefix=\hbox{\tt \ \ \ \ \ \ \ \ \ \ \ \ \ }
\startm
\m{\vdash}\m{(}\m{(}\m{A}\m{\in}\m{\mathbb{R}}\m{\wedge}\m{B}\m{\in}\m{\mathbb{R}}%
\m{\wedge}\m{C}\m{\in}\m{\mathbb{R}}\m{)}\m{\rightarrow}\m{(}\m{(}\m{A}\m{<}%
\m{B}\m{\wedge}\m{B}\m{<}\m{C}\m{)}\m{\rightarrow}\m{A}\m{<}\m{C}\m{)}\m{)}
\endm
%\vskip 1ex

\noindent 20. Ordering on reals is preserved after addition to both sides.

%\vskip 0.5ex
\setbox\startprefix=\hbox{\tt \ \ ax-pre-ltadd\ \$p\ }
\setbox\contprefix=\hbox{\tt \ \ \ \ \ \ \ \ \ \ \ \ \ }
\startm
\m{\vdash}\m{(}\m{(}\m{A}\m{\in}\m{\mathbb{R}}\m{\wedge}\m{B}\m{\in}\m{\mathbb{R}}%
\m{\wedge}\m{C}\m{\in}\m{\mathbb{R}}\m{)}\m{\rightarrow}\m{(}\m{A}\m{<}\m{B}\m{%
\rightarrow}\m{(}\m{C}\m{+}\m{A}\m{)}\m{<}\m{(}\m{C}\m{+}\m{B}\m{)}\m{)}\m{)}
\endm
%\vskip 1ex

\noindent 21. The product of two positive reals is positive.

%\vskip 0.5ex
\setbox\startprefix=\hbox{\tt \ \ ax-pre-mulgt0\ \$p\ }
\setbox\contprefix=\hbox{\tt \ \ \ \ \ \ \ \ \ \ \ \ \ \ }
\startm
\m{\vdash}\m{(}\m{(}\m{A}\m{\in}\m{\mathbb{R}}\m{\wedge}\m{B}\m{\in}\m{\mathbb{R}}%
\m{)}\m{\rightarrow}\m{(}\m{(}\m{0}\m{<}\m{A}\m{\wedge}\m{0}%
\m{<}\m{B}\m{)}\m{\rightarrow}\m{0}\m{<}\m{(}\m{A}\m{\cdot}\m{B}\m{)}%
\m{)}\m{)}
\endm
%\vskip 1ex

\noindent 22. A non-empty, bounded-above set of reals has a supremum.

%\vskip 0.5ex
\setbox\startprefix=\hbox{\tt \ \ ax-pre-sup\ \$p\ }
\setbox\contprefix=\hbox{\tt \ \ \ \ \ \ \ \ \ \ \ }
\startm
\m{\vdash}\m{(}\m{(}\m{A}\m{\subseteq}\m{\mathbb{R}}\m{\wedge}\m{A}\m{\ne}\m{%
\varnothing}\m{\wedge}\m{\exists}\m{x}\m{\in}\m{\mathbb{R}}\m{\forall}\m{y}\m{%
\in}\m{A}\m{\,y}\m{<}\m{x}\m{)}\m{\rightarrow}\m{\exists}\m{x}\m{\in}\m{%
\mathbb{R}}\m{(}\m{\forall}\m{y}\m{\in}\m{A}\m{\lnot}\m{x}\m{<}\m{y}\m{\wedge}\m{%
\forall}\m{y}\m{\in}\m{\mathbb{R}}\m{(}\m{y}\m{<}\m{x}\m{\rightarrow}\m{\exists}%
\m{z}\m{\in}\m{A}\m{\,y}\m{<}\m{z}\m{)}\m{)}\m{)}
\endm

% NOTE: The \m{...} expressions above could be represented as
% $ \vdash ( ( A \subseteq \mathbb{R} \wedge A \ne \varnothing \wedge \exists x \in \mathbb{R} \forall y \in A \,y < x ) \rightarrow \exists x \in \mathbb{R} ( \forall y \in A \lnot x < y \wedge \forall y \in \mathbb{R} ( y < x \rightarrow \exists z \in A \,y < z ) ) ) $

\vskip 2ex

This completes the set of axioms for real and complex numbers.  You may
wish to look at how subtraction, division, and decimal numbers
are defined in \texttt{set.mm}, and for fun look at the proof of $2+
2 = 4$ (theorem \texttt{2p2e4} in \texttt{set.mm})
as discussed in section \ref{2p2e4}.

In \texttt{set.mm} we define the non-negative integers $\mathbb{N}$, the integers
$\mathbb{Z}$, and the rationals $\mathbb{Q}$ as subsets of $\mathbb{R}$.  This leads
to the nice inclusion $\mathbb{N} \subseteq \mathbb{Z} \subseteq \mathbb{Q} \subseteq
\mathbb{R} \subseteq \mathbb{C}$, giving us a uniform framework in which, for
example, a property such as commutativity of complex number addition
automatically applies to integers.  The natural numbers $\mathbb{N}$
are different from the set $\omega$ we defined earlier, but both satisfy
Peano's postulates.

\subsection{Complex Number Axioms in Analysis Texts}

Most analysis texts construct complex numbers as ordered pairs of reals,
leading to construction-dependent properties that satisfy these axioms
but are not stated in their pure form.  (This is also done in
\texttt{set.mm} but our axioms are extracted from that construction.)
Other texts will simply state that $\mathbb{R}$ is a ``complete ordered
subfield of $\mathbb{C}$,'' leading to redundant axioms when this phrase
is completely expanded out.  In fact I have not seen a text with the
axioms in the explicit form above.
None of these axioms is unique individually, but this carefully worked out
collection of axioms is the result of years of work
by the Metamath community.

\subsection{Eliminating Unnecessary Complex Number Axioms}

We once had more axioms for real and complex numbers, but over years of time
we (the Metamath community)
have found ways to eliminate them (by proving them from other axioms)
or weaken them (by making weaker claims without reducing what
can be proved).
In particular, here are statements that used to be complex number
axioms but have since been formally proven (with Metamath) to be redundant:

\begin{itemize}
\item
  $\mathbb{C} \in V$.
  At one time this was listed as a ``complex number axiom.''
  However, this is not properly speaking a complex number axiom,
  and in any case its proof uses axioms of set theory.
  Proven redundant by Mario Carneiro\index{Carneiro, Mario} on
  17-Nov-2014 (see \texttt{axcnex}).
\item
  $((A \in \mathbb{C} \land B \in \mathbb{C}$) $\rightarrow$
  $(A + B) = (B + A))$.
  Proved redundant by Eric Schmidt\index{Schmidt, Eric} on 19-Jun-2012,
  and formalized by Scott Fenton\index{Fenton, Scott} on 3-Jan-2013
  (see \texttt{addcom}).
\item
  $(A \in \mathbb{C} \rightarrow (A + 0) = A)$.
  Proved redundant by Eric Schmidt on 19-Jun-2012,
  and formalized by Scott Fenton on 3-Jan-2013
  (see \texttt{addid1}).
\item
  $(A \in \mathbb{C} \rightarrow \exists x \in \mathbb{C} (A + x) = 0)$.
  Proved redundant by Eric Schmidt and formalized on 21-May-2007
  (see \texttt{cnegex}).
\item
  $((A \in \mathbb{C} \land A \ne 0) \rightarrow \exists x \in \mathbb{C} (A \cdot x) = 1)$.
  Proved redundant by Eric Schmidt and formalized on 22-May-2007
  (see \texttt{recex}).
\item
  $0 \in \mathbb{R}$.
  Proved redundant by Eric Schmidt on 19-Feb-2005 and formalized 21-May-2007
  (see \texttt{0re}).
\end{itemize}

We could eliminate 0 as an axiomatic object by defining it as
$( ( i \cdot i ) + 1 )$
and replacing it with this expression throughout the axioms. If this
is done, axiom ax-i2m1 becomes redundant. However, the remaining axioms
would become longer and less intuitive.

Eric Schmidt's paper analyzing this axiom system \cite{Schmidt}
presented a proof that these remaining axioms,
with the possible exception of ax-mulcom, are independent of the others.
It is currently an open question if ax-mulcom is independent of the others.

\section{Two Plus Two Equals Four}\label{2p2e4}

Here is a proof that $2 + 2 = 4$, as proven in the theorem \texttt{2p2e4}
in the database \texttt{set.mm}.
This is a useful demonstration of what a Metamath proof can look like.
This proof may have more steps than you're used to, but each step is rigorously
proven all the way back to the axioms of logic and set theory.
This display was originally generated by the Metamath program
as an {\sc HTML} file.

In the table showing the proof ``Step'' is the sequential step number,
while its associated ``Expression'' is an expression that we have proved.
``Ref'' is the name of a theorem or axiom that justifies that expression,
and ``Hyp'' refers to previous steps (if any) that the theorem or axiom
needs so that we can use it.  Expressions are indented further than
the expressions that depend on them to show their interdependencies.

\begin{table}[!htbp]
\caption{Two plus two equals four}
\begin{tabular}{lllll}
\textbf{Step} & \textbf{Hyp} & \textbf{Ref} & \textbf{Expression} & \\
1  &       & df-2    & $ \; \; \vdash 2 = 1 + 1$  & \\
2  & 1     & oveq2i  & $ \; \vdash (2 + 2) = (2 + (1 + 1))$ & \\
3  &       & df-4    & $ \; \; \vdash 4 = (3 + 1)$ & \\
4  &       & df-3    & $ \; \; \; \vdash 3 = (2 + 1)$ & \\
5  & 4     & oveq1i  & $ \; \; \vdash (3 + 1) = ((2 + 1) + 1)$ & \\
6  &       & 2cn     & $ \; \; \; \vdash 2 \in \mathbb{C}$ & \\
7  &       & ax-1cn  & $ \; \; \; \vdash 1 \in \mathbb{C}$ & \\
8  & 6,7,7 & addassi & $ \; \; \vdash ((2 + 1) + 1) = (2 + (1 + 1))$ & \\
9  & 3,5,8 & 3eqtri  & $ \; \vdash 4 = (2 + (1 + 1))$ & \\
10 & 2,9   & eqtr4i  & $ \vdash (2 + 2) = 4$ & \\
\end{tabular}
\end{table}

Step 1 says that we can assert that $2 = 1 + 1$ because it is
justified by \texttt{df-2}.
What is \texttt{df-2}?
It is simply the definition of $2$, which in our system is defined as being
equal to $1 + 1$.  This shows how we can use definitions in proofs.

Look at Step 2 of the proof. In the Ref column, we see that it references
a previously proved theorem, \texttt{oveq2i}.
It turns out that
theorem \texttt{oveq2i} requires a
hypothesis, and in the Hyp column of Step 2 we indicate that Step 1 will
satisfy (match) this hypothesis.
If we looked at \texttt{oveq2i}
we would find that it proves that given some hypothesis
$A = B$, we can prove that $( C F A ) = ( C F B )$.
If we use \texttt{oveq2i} and apply step 1's result as the hypothesis,
that will mean that $A = 2$ and $B = ( 1 + 1 )$ within this use of
\texttt{oveq2i}.
We are free to select any value of $C$ and $F$ (subject to syntax constraints),
so we are free to select $C = 2$ and $F = +$,
producing our desired result,
$ (2 + 2) = (2 + (1 + 1))$.

Step 2 is an example of substitution.
In the end, every step in every proof uses only this one substitution rule.
All the rules of logic, and all the axioms, are expressed so that
they can be used via this one substitution rule.
So once you master substitution, you can master every Metamath proof,
no exceptions.

Each step is clear and can be immediately checked.
In the {\sc HTML} display you can even click on each reference to see why it is
justified, making it easy to see why the proof works.

\section{Deduction}\label{deduction}

Strictly speaking,
a deduction (also called an inference) is a kind of statement that needs
some hypotheses to be true in order for its conclusion to be true.
A theorem, on the other hand, has no hypotheses.
Informally we often call both of them theorems, but in this section we
will stick to the strict definitions.

It sometimes happens that we have proved a deduction of the form
$\varphi \Rightarrow \psi$\index{$\Rightarrow$}
(given hypothesis $\varphi$ we can prove $\psi$)
and we want to then prove a theorem of the form
$\varphi \rightarrow \psi$.

Converting a deduction (which uses a hypothesis) into a theorem
(which does not) is not as simple as you might think.
The deduction says, ``if we can prove $\varphi$ then we can prove $\psi$,''
which is in some sense weaker than saying
``$\varphi$ implies $\psi$.''
There is no axiom of logic that permits us to directly obtain the theorem
given the deduction.\footnote{
The conversion of a deduction to a theorem does not even hold in general
for quantum propositional calculus,
which is a weak subset of classical propositional calculus.
It has been shown that adding the Standard Deduction Theorem (discussed below)
to quantum propositional calculus turns it into classical
propositional calculus!
}

This is in contrast to going the other way.
If we have the theorem ($\varphi \rightarrow \psi$),
it is easy to recover the deduction
($\varphi \Rightarrow \psi$)
using modus ponens\index{modus ponens}
(\texttt{ax-mp}; see section \ref{axmp}).

In the following subsections we first discuss the standard deduction theorem
(the traditional but awkward way to convert deductions into theorems) and
the weak deduction theorem (a limited version of the standard deduction
theorem that is easier to use and was once widely used in
\texttt{set.mm}\index{set theory database (\texttt{set.mm})}).
In section \ref{deductionstyle} we discuss
deduction style, the newer approach we now recommend in most cases.
Deduction style uses ``deduction form,'' a form that
prefixes each hypothesis (other than definitions) and the
conclusion with a universal antecedent (``$\varphi \rightarrow$'').
Deduction style is widely used in \texttt{set.mm},
so it is useful to understand it and \textit{why} it is widely used.
Section \ref{naturaldeduction}
briefly discusses our approach for using natural deduction
within \texttt{set.mm},
as that approach is deeply related to deduction style.
We conclude with a summary of the strengths of
our approach, which we believe are compelling.

\subsection{The Standard Deduction Theorem}\label{standarddeductiontheorem}

It is possible to make use of information
contained in the deduction or its proof to assist us with the proof of
the related theorem.
In traditional logic books, there is a metatheorem called the
Deduction Theorem\index{Deduction Theorem}\index{Standard Deduction Theorem},
discovered independently by Herbrand and Tarski around 1930.
The Deduction Theorem, which we often call the Standard Deduction Theorem,
provides an algorithm for constructing a proof of a theorem from the
proof of its corresponding deduction. See, for example,
\cite[p.~56]{Margaris}\index{Margaris, Angelo}.
To construct a proof for a theorem, the
algorithm looks at each step in the proof of the original deduction and
rewrites the step with several steps wherein the hypothesis is eliminated
and becomes an antecedent.

In ordinary mathematics, no one actually carries out the algorithm,
because (in its most basic form) it involves an exponential explosion of
the number of proof steps as more hypotheses are eliminated. Instead,
the Standard Deduction Theorem is invoked simply to claim that it can
be done in principle, without actually doing it.
What's more, the algorithm is not as simple as it might first appear
when applying it rigorously.
There is a subtle restriction on the Standard Deduction Theorem
that must be taken into account involving the axiom of generalization
when working with predicate calculus (see the literature for more detail).

One of the goals of Metamath is to let you plainly see, with as few
underlying concepts as possible, how mathematics can be derived directly
from the axioms, and not indirectly according to some hidden rules
buried inside a program or understood only by logicians. If we added
the Standard Deduction Theorem to the language and proof verifier,
that would greatly complicate both and largely defeat Metamath's goal
of simplicity. In principle, we could show direct proofs by expanding
out the proof steps generated by the algorithm of the Standard Deduction
Theorem, but that is not feasible in practice because the number of proof
steps quickly becomes huge, even astronomical.
Since the algorithm of the Standard Deduction Theorem is driven by the proof,
we would have to go through that proof
all over again---starting from axioms---in order to obtain the theorem form.
In terms of proof length, there would be no savings over just
proving the theorem directly instead of first proving the deduction form.

\subsection{Weak Deduction Theorem}\label{weakdeductiontheorem}

We have developed
a more efficient method for proving a theorem from a deduction
that can be used instead of the Standard Deduction Theorem
in many (but not all) cases.
We call this more efficient method the
Weak Deduction Theorem\index{Weak Deduction Theorem}.\footnote{
There is also an unrelated ``Weak Deduction Theorem''
in the field of relevance logic, so to avoid confusion we could call
ours the ``Weak Deduction Theorem for Classical Logic.''}
Unlike the Standard Deduction Theorem, the Weak Deduction Theorem produces the
theorem directly from a special substitution instance of the deduction,
using a small, fixed number of steps roughly proportional to the length
of the final theorem.

If you come to a proof referencing the Weak Deduction Theorem
\texttt{dedth} (or one of its variants \texttt{dedthxx}),
here is how to follow the proof without getting into the details:
just click on the theorem referenced in the step
just before the reference to \texttt{dedth} and ignore everything else.
Theorem \texttt{dedth} simply turns a hypothesis into an antecedent
(i.e. the hypothesis followed by $\rightarrow$
is placed in front of the assertion, and the hypothesis
itself is eliminated) given certain conditions.

The Weak Deduction Theorem
eliminates a hypothesis $\varphi$, making it become an antecedent.
It does this by proving an expression
$ \varphi \rightarrow \psi $ given two hypotheses:
(1)
$ ( A = {\rm if} ( \varphi , A , B ) \rightarrow ( \varphi \leftrightarrow \chi ) ) $
and
(2) $\chi$.
Note that it requires that a proof exists for $\varphi$ when the class variable
$A$ is replaced with a specific class $B$. The hypothesis $\chi$
should be assigned to the inference.
You can see the details of the proof of the Weak Deduction Theorem
in theorem \texttt{dedth}.

The Weak Deduction Theorem
is probably easier to understand by studying proofs that make use of it.
For example, let's look at the proof of \texttt{renegcl}, which proves that
$ \vdash ( A \in \mathbb{R} \rightarrow - A \in \mathbb{R} )$:

\needspace{4\baselineskip}
\begin{longtabu} {l l l X}
\textbf{Step} & \textbf{Hyp} & \textbf{Ref} & \textbf{Expression} \\
  1 &  & negeq &
  $\vdash$ $($ $A$ $=$ ${\rm if}$ $($ $A$ $\in$ $\mathbb{R}$ $,$ $A$ $,$ $1$ $)$ $\rightarrow$
  $\textrm{-}$ $A$ $=$ $\textrm{-}$ ${\rm if}$ $($ $A$ $\in$ $\mathbb{R}$
  $,$ $A$ $,$ $1$ $)$ $)$ \\
 2 & 1 & eleq1d &
    $\vdash$ $($ $A$ $=$ ${\rm if}$ $($ $A$ $\in$ $\mathbb{R}$ $,$ $A$ $,$ $1$ $)$ $\rightarrow$ $($
    $\textrm{-}$ $A$ $\in$ $\mathbb{R}$ $\leftrightarrow$
    $\textrm{-}$ ${\rm if}$ $($ $A$ $\in$ $\mathbb{R}$ $,$ $A$ $,$ $1$ $)$ $\in$
    $\mathbb{R}$ $)$ $)$ \\
 3 &  & 1re & $\vdash 1 \in \mathbb{R}$ \\
 4 & 3 & elimel &
   $\vdash {\rm if} ( A \in \mathbb{R} , A , 1 ) \in \mathbb{R}$ \\
 5 & 4 & renegcli &
   $\vdash \textrm{-} {\rm if} ( A \in \mathbb{R} , A , 1 ) \in \mathbb{R}$ \\
 6 & 2,5 & dedth &
   $\vdash ( A \in \mathbb{R} \rightarrow \textrm{-} A \in \mathbb{R}$ ) \\
\end{longtabu}

The somewhat strange-looking steps in \texttt{renegcl} before step 5 are
technical stuff that makes this magic work, and they can be ignored
for a quick overview of the proof. To continue following the ``important''
part of the proof of \texttt{renegcl},
you can look at the reference to \texttt{renegcli} at step 5.

That said, let's briefly look at how
\texttt{renegcl} uses the
Weak Deduction Theorem (\texttt{dedth}) to do its job,
in case you want to do something similar or want understand it more deeply.
Let's work backwards in the proof of \texttt{renegcl}.
Step 6 applies \texttt{dedth} to produce our goal result
$ \vdash ( A \in \mathbb{R} \rightarrow\, - A \in \mathbb{R} )$.
This requires on the one hand the (substituted) deduction
\texttt{renegcli} in step 5.
By itself \texttt{renegcli} proves the deduction
$ \vdash A \in \mathbb{R} \Rightarrow\, \vdash - A \in \mathbb{R}$;
this is the deduction form we are trying to turn into theorem form,
and thus
\texttt{renegcli} has a separate hypothesis that must be fulfilled.
To fulfill the hypothesis of the invocation of
\texttt{renegcli} in step 5, it is eventually
reduced to the already proven theorem $1 \in \mathbb{R}$ in step 3.
Step 4 connects steps 3 and 5; step 4 invokes
\texttt{elimel}, a special case of \texttt{elimhyp} that eliminates
a membership hypothesis for the weak deduction theorem.
On the other hand, the equivalence of the conclusion of
\texttt{renegcl}
$( - A \in \mathbb{R} )$ and the substituted conclusion of
\texttt{renegcli} must be proven, which is done in steps 2 and 1.

The weak deduction theorem has limitations.
In particular, we must be able to prove a special case of the deduction's
hypothesis as a stand-alone theorem.
For example, we used $1 \in \mathbb{R}$ in step 3 of \texttt{renegcl}.

We used to use the weak deduction theorem
extensively within \texttt{set.mm}.
However, we now recommend applying ``deduction style''
instead in most cases, as deduction style is
often an easier and clearer approach.
Therefore, we will now describe deduction style.

\subsection{Deduction Style}\label{deductionstyle}

We now prefer to write assertions in ``deduction form''
instead of writing a proof that would require use of the standard or
weak deduction theorem.
We call this appraoch
``deduction style.''\index{deduction style}

It will be easier to explain this by first defining some terms:

\begin{itemize}
\item \textbf{closed form}\index{closed form}\index{forms!closed}:
A kind of assertion (theorem) with no hypotheses.
Typically its label has no special suffix.
An example is \texttt{unss}, which states:
$\vdash ( ( A \subseteq C \wedge B \subseteq C ) \leftrightarrow ( A \cup B )
\subseteq C )\label{eq:unss}$
\item \textbf{deduction form}\index{deduction form}\index{forms!deduction}:
A kind of assertion with one or more hypotheses
where the conclusion is an implication with
a wff variable as the antecedent (usually $\varphi$), and every hypothesis
(\$e statement)
is either (1) an implication with the same antecedent as the conclusion or
(2) a definition.
A definition
can be for a class variable (this is a class variable followed by ``='')
or a wff variable (this is a wff variable followed by $\leftrightarrow$);
class variable definitions are more common.
In practice, a proof
in deduction form will also contain many steps that are implications
where the antecedent is either that wff variable (normally $\varphi$)
or is
a conjunction (...$\land$...) including that wff variable ($\varphi$).
If an assertion is in deduction form, and other forms are also available,
then we suffix its label with ``d.''
An example is \texttt{unssd}, which states\footnote{
For brevity we show here (and in other places)
a $\&$\index{$\&$} between hypotheses\index{hypotheses}
and a $\Rightarrow$\index{$\Rightarrow$}\index{conclusion}
between the hypotheses and the conclusion.
This notation is technically not part of the Metamath language, but is
instead a convenient abbreviation to show both the hypotheses and conclusion.}:
$\vdash ( \varphi \rightarrow A \subseteq C )\quad\&\quad \vdash ( \varphi
    \rightarrow B \subseteq C )\quad\Rightarrow\quad \vdash ( \varphi
    \rightarrow ( A \cup B ) \subseteq C )\label{eq:unssd}$
\item \textbf{inference form}\index{inference form}\index{forms!inference}:
A kind of assertion with one or more hypotheses that is not in deduction form
(e.g., there is no common antecedent).
If an assertion is in inference form, and other forms are also available,
then we suffix its label with ``i.''
An example is \texttt{unssi}, which states:
$\vdash A \subseteq C\quad\&\quad \vdash B \subseteq C\quad\Rightarrow\quad
    \vdash ( A \cup B ) \subseteq C\label{eq:unssi}$
\end{itemize}

When using deduction style we express an assertion in deduction form.
This form prefixes each hypothesis (other than definitions) and the
conclusion with a universal antecedent (``$\varphi \rightarrow$'').
The antecedent (e.g., $\varphi$)
mimics the context handled in the deduction theorem, eliminating
the need to directly use the deduction theorem.

Once you have an assertion in deduction form, you can easily convert it
to inference form or closed form:

\begin{itemize}
\item To
prove some assertion Ti in inference form, given assertion Td in deduction
form, there is a simple mechanical process you can use. First take each
Ti hypothesis and insert a \texttt{T.} $\rightarrow$ prefix (``true implies'')
using \texttt{a1i}. You
can then use the existing assertion Td to prove the resulting conclusion
with a \texttt{T.} $\rightarrow$ prefix.
Finally, you can remove that prefix using \texttt{mptru},
resulting in the conclusion you wanted to prove.
\item To
prove some assertion T in closed form, given assertion Td in deduction
form, there is another simple mechanical process you can use. First,
select an expression that is the conjunction (...$\land$...) of all of the
consequents of every hypothesis of Td. Next, prove that this expression
implies each of the separate hypotheses of Td in turn by eliminating
conjuncts (there are a variety of proven assertions to do this, including
\texttt{simpl},
\texttt{simpr},
\texttt{3simpa},
\texttt{3simpb},
\texttt{3simpc},
\texttt{simp1},
\texttt{simp2},
and
\texttt{simp3}).
If the
expression has nested conjunctions, inner conjuncts can be broken out by
chaining the above theorems with \texttt{syl}
(see section \ref{syl}).\footnote{
There are actually many theorems
(labeled simp* such as \texttt{simp333}) that break out inner conjuncts in one
step, but rather than learning them you can just use the chaining we
just described to prove them, and then let the Metamath program command
\texttt{minimize{\char`\_}with}\index{\texttt{minimize{\char`\_}with} command}
figure out the right ones needed to collapse them.}
As your final step, you can then apply the already-proven assertion Td
(which is in deduction form), proving assertion T in closed form.
\end{itemize}

We can also easily convert any assertion T in closed form to its related
assertion Ti in inference form by applying
modus ponens\index{modus ponens} (see section \ref{axmp}).

The deduction form antecedent can also be used to represent the context
necessary to support natural deduction systems, so we will now
discuss natural deduction.

\subsection{Natural Deduction}\label{naturaldeduction}

Natural deduction\index{natural deduction}
(ND) systems, as such, were originally introduced in
1934 by two logicians working independently: Ja\'skowski and Gentzen. ND
systems are supposed to reconstruct, in a formally proper way, traditional
ways of mathematical reasoning (such as conditional proof, indirect proof,
and proof by cases). As reconstructions they were naturally influenced
by previous work, and many specific ND systems and notations have been
developed since their original work.

There are many ND variants, but
Indrzejczak \cite[p.~31-32]{Indrzejczak}\index{Indrzejczak, Andrzej}
suggests that any natural deductive system must satisfy at
least these three criteria:

\begin{itemize}
\item ``There are some means for entering assumptions into a proof and
also for eliminating them. Usually it requires some bookkeeping devices
for indicating the scope of an assumption, and showing that a part of
a proof depending on eliminated assumption is discharged.
\item There are no (or, at least, a very limited set of) axioms, because
their role is taken over by the set of primitive rules for introduction
and elimination of logical constants which means that elementary
inferences instead of formulae are taken as primitive.
\item (A genuine) ND system admits a lot of freedom in proof construction
and possibility of applying several proof search strategies, like
conditional proof, proof by cases, proof by reductio ad absurdum etc.''
\end{itemize}

The Metamath Proof Explorer (MPE) as defined in \texttt{set.mm}
is fundamentally a Hilbert-style system.
That is, MPE is based on a larger number of axioms (compared
to natural deduction systems), a very small set of rules of inference
(modus ponens), and the context is not changed by the rules of inference
in the middle of a proof. That said, MPE proofs can be developed using
the natural deduction (ND) approach as originally developed by Ja\'skowski
and Gentzen.

The most common and recommended approach for applying ND in MPE is to use
deduction form\index{deduction form}%
\index{forms!deduction}
and apply the MPE proven assertions that are equivalent to ND rules.
For example, MPE's \texttt{jca} is equivalent to ND rule $\land$-I
(and-insertion).
We maintain a list of equivalences that you may consult.
This approach for applying an ND approach within MPE relies on Metamath's
wff metavariables in an essential way, and is described in more detail
in the presentation ``Natural Deductions in the Metamath Proof Language''
by Mario Carneiro \cite{CarneiroND}\index{Carneiro, Mario}.

In this style many steps are an implication, whose antecedent mimics
the context ($\Gamma$) of most ND systems. To add an assumption, simply add
it to the implication antecedent (typically using
\texttt{simpr}),
and use that
new antecedent for all later claims in the same scope. If you wish to
use an assertion in an ND hypothesis scope that is outside the current
ND hypothesis scope, modify the assertion so that the ND hypothesis
assumption is added to its antecedent (typically using \texttt{adantr}). Most
proof steps will be proved using rules that have hypotheses and results
of the form $\varphi \rightarrow$ ...

An example may make this clearer.
Let's show theorem 5.5 of
\cite[p.~18]{Clemente}\index{Clemente Laboreo, Daniel}
along with a line by line translation using the usual
translation of natural deduction (ND) in the Metamath Proof Explorer
(MPE) notation (this is proof \texttt{ex-natded5.5}).
The proof's original goal was to prove
$\lnot \psi$ given two hypotheses,
$( \psi \rightarrow \chi )$ and $ \lnot \chi$.
We will translate these statements into MPE deduction form
by prefixing them all with $\varphi \rightarrow$.
As a result, in MPE the goal is stated as
$( \varphi \rightarrow \lnot \psi )$, and the two hypotheses are stated as
$( \varphi \rightarrow ( \psi \rightarrow \chi ) )$ and
$( \varphi \rightarrow \lnot \chi )$.

The following table shows the proof in Fitch natural deduction style
and its MPE equivalent.
The \textit{\#} column shows the original numbering,
\textit{MPE\#} shows the number in the equivalent MPE proof
(which we will show later),
\textit{ND Expression} shows the original proof claim in ND notation,
and \textit{MPE Translation} shows its translation into MPE
as discussed in this section.
The final columns show the rationale in ND and MPE respectively.

\needspace{4\baselineskip}
{\setlength{\extrarowsep}{4pt} % Keep rows from being too close together
\begin{longtabu}   { @{} c c X X X X }
\textbf{\#} & \textbf{MPE\#} & \textbf{ND Ex\-pres\-sion} &
\textbf{MPE Trans\-lation} & \textbf{ND Ration\-ale} &
\textbf{MPE Ra\-tio\-nale} \\
\endhead

1 & 2;3 &
$( \psi \rightarrow \chi )$ &
$( \varphi \rightarrow ( \psi \rightarrow \chi ) )$ &
Given &
\$e; \texttt{adantr} to put in ND hypothesis \\

2 & 5 &
$ \lnot \chi$ &
$( \varphi \rightarrow \lnot \chi )$ &
Given &
\$e; \texttt{adantr} to put in ND hypothesis \\

3 & 1 &
... $\vert$ $\psi$ &
$( \varphi \rightarrow \psi )$ &
ND hypothesis assumption &
\texttt{simpr} \\

4 & 4 &
... $\chi$ &
$( ( \varphi \land \psi ) \rightarrow \chi )$ &
$\rightarrow$\,E 1,3 &
\texttt{mpd} 1,3 \\

5 & 6 &
... $\lnot \chi$ &
$( ( \varphi \land \psi ) \rightarrow \lnot \chi )$ &
IT 2 &
\texttt{adantr} 5 \\

6 & 7 &
$\lnot \psi$ &
$( \varphi \rightarrow \lnot \psi )$ &
$\land$\,I 3,4,5 &
\texttt{pm2.65da} 4,6 \\

\end{longtabu}
}


The original used Latin letters; we have replaced them with Greek letters
to follow Metamath naming conventions and so that it is easier to follow
the Metamath translation. The Metamath line-for-line translation of
this natural deduction approach precedes every line with an antecedent
including $\varphi$ and uses the Metamath equivalents of the natural deduction
rules. To add an assumption, the antecedent is modified to include it
(typically by using \texttt{adantr};
\texttt{simpr} is useful when you want to
depend directly on the new assumption, as is shown here).

In Metamath we can represent the two given statements as these hypotheses:

\needspace{2\baselineskip}
\begin{itemize}
\item ex-natded5.5.1 $\vdash ( \varphi \rightarrow ( \psi \rightarrow \chi ) )$
\item ex-natded5.5.2 $\vdash ( \varphi \rightarrow \lnot \chi )$
\end{itemize}

\needspace{4\baselineskip}
Here is the proof in Metamath as a line-by-line translation:

\begin{longtabu}   { l l l X }
\textbf{Step} & \textbf{Hyp} & \textbf{Ref} & \textbf{Ex\-pres\-sion} \\
\endhead
1 & & simpr & $\vdash ( ( \varphi \land \psi ) \rightarrow \psi )$ \\
2 & & ex-natded5.5.1 &
  $\vdash ( \varphi \rightarrow ( \psi \rightarrow \chi ) )$ \\
3 & 2 & adantr &
 $\vdash ( ( \varphi \land \psi ) \rightarrow ( \psi \rightarrow \chi ) )$ \\
4 & 1, 3 & mpd &
 $\vdash ( ( \varphi \land \psi ) \rightarrow \chi ) $ \\
5 & & ex-natded5.5.2 &
 $\vdash ( \varphi \rightarrow \lnot \chi )$ \\
6 & 5 & adantr &
 $\vdash ( ( \varphi \land \psi ) \rightarrow \lnot \chi )$ \\
7 & 4, 6 & pm2.65da &
 $\vdash ( \varphi \rightarrow \lnot \psi )$ \\
\end{longtabu}

Only using specific natural deduction rules directly can lead to very
long proofs, for exactly the same reason that only using axioms directly
in Hilbert-style proofs can lead to very long proofs.
If the goal is short and clear proofs,
then it is better to reuse already-proven assertions
in deduction form than to start from scratch each time
and using only basic natural deduction rules.

\subsection{Strengths of Our Approach}

As far as we know there is nothing else in the literature like either the
weak deduction theorem or Mario Carneiro\index{Carneiro, Mario}'s
natural deduction method.
In order to
transform a hypothesis into an antecedent, the literature's standard
``Deduction Theorem''\index{Deduction Theorem}\index{Standard Deduction Theorem}
requires metalogic outside of the notions provided
by the axiom system. We instead generally prefer to use Mario Carneiro's
natural deduction method, then use the weak deduction theorem in cases
where that is difficult to apply, and only then use the full standard
deduction theorem as a last resort.

The weak deduction theorem\index{Weak Deduction Theorem}
does not require any additional metalogic
but converts an inference directly into a closed form theorem, with
a rigorous proof that uses only the axiom system. Unlike the standard
Deduction Theorem, there is no implicit external justification that we
have to trust in order to use it.

Mario Carneiro's natural deduction\index{natural deduction}
method also does not require any new metalogical
notions. It avoids the Deduction Theorem's metalogic by prefixing the
hypotheses and conclusion of every would-be inference with a universal
antecedent (``$\varphi \rightarrow$'') from the very start.

We think it is impressive and satisfying that we can do so much in a
practical sense without stepping outside of our Hilbert-style axiom system.
Of course our axiomatization, which is in the form of schemes,
contains a metalogic of its own that we exploit. But this metalogic
is relatively simple, and for our Deduction Theorem alternatives,
we primarily use just the direct substitution of expressions for
metavariables.

\begin{sloppy}
\section{Exploring the Set The\-o\-ry Data\-base}\label{exploring}
\end{sloppy}
% NOTE: All examples performed in this section are
% recorded wtih "set width 61" % on set.mm as of 2019-05-28
% commit c1e7849557661260f77cfdf0f97ac4354fbb4f4d.

At this point you may wish to study the \texttt{set.mm}\index{set theory
database (\texttt{set.mm})} file in more detail.  Pay particular
attention to the assumptions needed to define wffs\index{well-formed
formula (wff)} (which are not included above), the variable types
(\texttt{\$f}\index{\texttt{\$f} statement} statements), and the
definitions that are introduced.  Start with some simple theorems in
propositional calculus, making sure you understand in detail each step
of a proof.  Once you get past the first few proofs and become familiar
with the Metamath language, any part of the \texttt{set.mm} database
will be as easy to follow, step by step, as any other part---you won't
have to undergo a ``quantum leap'' in mathematical sophistication to be
able to follow a deep proof in set theory.

Next, you may want to explore how concepts such as natural numbers are
defined and described.  This is probably best done in conjunction with
standard set theory textbooks, which can help give you a higher-level
understanding.  The \texttt{set.mm} database provides references that will get
you started.  From there, you will be on your way towards a very deep,
rigorous understanding of abstract mathematics.

The Metamath\index{Metamath} program can help you peruse a Metamath data\-base,
wheth\-er you are trying to figure out how a certain step follows in a proof or
just have a general curiosity.  We will go through some examples of the
commands, using the \texttt{set.mm}\index{set theory database (\texttt{set.mm})}
database provided with the Metamath software.  These should help get you
started.  See Chapter~\ref{commands} for a more detailed description of
the commands.  Note that we have included the full spelling of all commands to
prevent ambiguity with future commands.  In practice you may type just the
characters needed to specify each command keyword\index{command keyword}
unambiguously, often just one or two characters per keyword, and you don't
need to type them in upper case.

First run the Metamath program as described earlier.  You should see the
\verb/MM>/ prompt.  Read in the \texttt{set.mm} file:\index{\texttt{read}
command}

\begin{verbatim}
MM> read set.mm
Reading source file "set.mm"... 34554442 bytes
34554442 bytes were read into the source buffer.
The source has 155711 statements; 2254 are $a and 32250 are $p.
No errors were found.  However, proofs were not checked.
Type VERIFY PROOF * if you want to check them.
\end{verbatim}

As with most examples in this book, what you will see
will be slightly different because we are continuously
improving our databases (including \texttt{set.mm}).

Let's check the database integrity.  This may take a minute or two to run if
your computer is slow.

\begin{verbatim}
MM> verify proof *
0 10%  20%  30%  40%  50%  60%  70%  80%  90% 100%
..................................................
All proofs in the database were verified in 2.84 s.
\end{verbatim}

No errors were reported, so every proof is correct.

You need to know the names (labels) of theorems before you can look at them.
Often just examining the database file(s) with a text editor is the best
approach.  In \texttt{set.mm} there are many detailed comments, especially near
the beginning, that can help guide you. The \texttt{search} command in the
Metamath program is also handy.  The \texttt{comments} qualifier will list the
statements whose associated comment (the one immediately before it) contain a
string you give it.  For example, if you are studying Enderton's {\em Elements
of Set Theory} \cite{Enderton}\index{Enderton, Herbert B.} you may want to see
the references to it in the database.  The search string \texttt{enderton} is not
case sensitive.  (This will not show you all the database theorems that are in
Enderton's book because there is usually only one citation for a given
theorem, which may appear in several textbooks.)\index{\texttt{search}
command}

\begin{verbatim}
MM> search * "enderton" / comments
12067 unineq $p "... Exercise 20 of [Enderton] p. 32 and ..."
12459 undif2 $p "...Corollary 6K of [Enderton] p. 144. (C..."
12953 df-tp $a "...s. Definition of [Enderton] p. 19. (Co..."
13689 unissb $p ".... Exercise 5 of [Enderton] p. 26 and ..."
\end{verbatim}
\begin{center}
(etc.)
\end{center}

Or you may want to see what theorems have something to do with
conjunction (logical {\sc and}).  The quotes around the search
string are optional when there's no ambiguity.\index{\texttt{search}
command}

\begin{verbatim}
MM> search * conjunction / comments
120 a1d $p "...be replaced with a conjunction ( ~ df-an )..."
662 df-bi $a "...viated form after conjunction is introdu..."
1319 wa $a "...ff definition to include conjunction ('and')."
1321 df-an $a "Define conjunction (logical 'and'). Defini..."
1420 imnan $p "...tion in terms of conjunction. (Contribu..."
\end{verbatim}
\begin{center}
(etc.)
\end{center}


Now we will start to look at some details.  Let's look at the first
axiom of propositional calculus
(we could use \texttt{sh st} to abbreviate
\texttt{show statement}).\index{\texttt{show statement} command}

\begin{verbatim}
MM> show statement ax-1/full
Statement 19 is located on line 881 of the file "set.mm".
"Axiom _Simp_.  Axiom A1 of [Margaris] p. 49.  One of the 3
axioms of propositional calculus.  The 3 axioms are also
        ...
19 ax-1 $a |- ( ph -> ( ps -> ph ) ) $.
Its mandatory hypotheses in RPN order are:
  wph $f wff ph $.
  wps $f wff ps $.
The statement and its hypotheses require the variables:  ph
      ps
The variables it contains are:  ph ps


Statement 49 is located on line 11182 of the file "set.mm".
Its statement number for HTML pages is 6.
"Axiom _Simp_.  Axiom A1 of [Margaris] p. 49.  One of the 3
axioms of propositional calculus.  The 3 axioms are also
given as Definition 2.1 of [Hamilton] p. 28.
...
49 ax-1 $a |- ( ph -> ( ps -> ph ) ) $.
Its mandatory hypotheses in RPN order are:
  wph $f wff ph $.
  wps $f wff ps $.
The statement and its hypotheses require the variables:
  ph ps
The variables it contains are:  ph ps
\end{verbatim}

Compare this to \texttt{ax-1} on p.~\pageref{ax1}.  You can see that the
symbol \texttt{ph} is the {\sc ascii} notation for $\varphi$, etc.  To
see the mathematical symbols for any expression you may typeset it in
\LaTeX\ (type \texttt{help tex} for instructions)\index{latex@{\LaTeX}}
or, easier, just use a text editor to look at the comments where symbols
are first introduced in \texttt{set.mm}.  The hypotheses \texttt{wph}
and \texttt{wps} required by \texttt{ax-1} mean that variables
\texttt{ph} and \texttt{ps} must be wffs.

Next we'll pick a simple theorem of propositional calculus, the Principle of
Identity, which is proved directly from the axioms.  We'll look at the
statement then its proof.\index{\texttt{show statement}
command}

\begin{verbatim}
MM> show statement id1/full
Statement 116 is located on line 11371 of the file "set.mm".
Its statement number for HTML pages is 22.
"Principle of identity.  Theorem *2.08 of [WhiteheadRussell]
p. 101.  This version is proved directly from the axioms for
demonstration purposes.
...
116 id1 $p |- ( ph -> ph ) $= ... $.
Its mandatory hypotheses in RPN order are:
  wph $f wff ph $.
Its optional hypotheses are:  wps wch wth wta wet
      wze wsi wrh wmu wla wka
The statement and its hypotheses require the variables:  ph
These additional variables are allowed in its proof:
      ps ch th ta et ze si rh mu la ka
The variables it contains are:  ph
\end{verbatim}

The optional variables\index{optional variable} \texttt{ps}, \texttt{ch}, etc.\ are
available for use in a proof of this statement if we wish, and were we to do
so we would make use of optional hypotheses \texttt{wps}, \texttt{wch}, etc.  (See
Section~\ref{dollaref} for the meaning of ``optional
hypothesis.''\index{optional hypothesis}) The reason these show up in the
statement display is that statement \texttt{id1} happens to be in their scope
(see Section~\ref{scoping} for the definition of ``scope''\index{scope}), but
in fact in propositional calculus we will never make use of optional
hypotheses or variables.  This becomes important after quantifiers are
introduced, where ``dummy'' variables\index{dummy variable} are often needed
in the middle of a proof.

Let's look at the proof of statement \texttt{id1}.  We'll use the
\texttt{show proof} command, which by default suppresses the
``non-essential'' steps that construct the wffs.\index{\texttt{show proof}
command}
We will display the proof in ``lemmon' format (a non-indented format
with explicit previous step number references) and renumber the
displayed steps:

\begin{verbatim}
MM> show proof id1 /lemmon/renumber
1 ax-1           $a |- ( ph -> ( ph -> ph ) )
2 ax-1           $a |- ( ph -> ( ( ph -> ph ) -> ph ) )
3 ax-2           $a |- ( ( ph -> ( ( ph -> ph ) -> ph ) ) ->
                     ( ( ph -> ( ph -> ph ) ) -> ( ph -> ph )
                                                          ) )
4 2,3 ax-mp      $a |- ( ( ph -> ( ph -> ph ) ) -> ( ph -> ph
                                                          ) )
5 1,4 ax-mp      $a |- ( ph -> ph )
\end{verbatim}

If you have read Section~\ref{trialrun}, you'll know how to interpret this
proof.  Step~2, for example, is an application of axiom \texttt{ax-1}.  This
proof is identical to the one in Hamilton's {\em Logic for Mathematicians}
\cite[p.~32]{Hamilton}\index{Hamilton, Alan G.}.

You may want to look at what
substitutions\index{substitution!variable}\index{variable substitution} are
made into \texttt{ax-1} to arrive at step~2.  The command to do this needs to
know the ``real'' step number, so we'll display the proof again without
the \texttt{renumber} qualifier.\index{\texttt{show proof}
command}

\begin{verbatim}
MM> show proof id1 /lemmon
 9 ax-1          $a |- ( ph -> ( ph -> ph ) )
20 ax-1          $a |- ( ph -> ( ( ph -> ph ) -> ph ) )
24 ax-2          $a |- ( ( ph -> ( ( ph -> ph ) -> ph ) ) ->
                     ( ( ph -> ( ph -> ph ) ) -> ( ph -> ph )
                                                          ) )
25 20,24 ax-mp   $a |- ( ( ph -> ( ph -> ph ) ) -> ( ph -> ph
                                                          ) )
26 9,25 ax-mp    $a |- ( ph -> ph )
\end{verbatim}

The ``real'' step number is 20.  Let's look at its details.

\begin{verbatim}
MM> show proof id1 /detailed_step 20
Proof step 20:  min=ax-1 $a |- ( ph -> ( ( ph -> ph ) -> ph )
  )
This step assigns source "ax-1" ($a) to target "min" ($e).
The source assertion requires the hypotheses "wph" ($f, step
18) and "wps" ($f, step 19).  The parent assertion of the
target hypothesis is "ax-mp" ($a, step 25).
The source assertion before substitution was:
    ax-1 $a |- ( ph -> ( ps -> ph ) )
The following substitutions were made to the source
assertion:
    Variable  Substituted with
     ph        ph
     ps        ( ph -> ph )
The target hypothesis before substitution was:
    min $e |- ph
The following substitution was made to the target hypothesis:
    Variable  Substituted with
     ph        ( ph -> ( ( ph -> ph ) -> ph ) )
\end{verbatim}

This shows the substitutions\index{substitution!variable}\index{variable
substitution} made to the variables in \texttt{ax-1}.  References are made to
steps 18 and 19 which are not shown in our proof display.  To see these steps,
you can display the proof with the \texttt{all} qualifier.

Let's look at a slightly more advanced proof of propositional calculus.  Note
that \verb+/\+ is the symbol for $\wedge$ (logical {\sc and}, also
called conjunction).\index{conjunction ($\wedge$)}
\index{logical {\sc and} ($\wedge$)}

\begin{verbatim}
MM> show statement prth/full
Statement 1791 is located on line 15503 of the file "set.mm".
Its statement number for HTML pages is 559.
"Conjoin antecedents and consequents of two premises.  This
is the closed theorem form of ~ anim12d .  Theorem *3.47 of
[WhiteheadRussell] p. 113.  It was proved by Leibniz,
and it evidently pleased him enough to call it
_praeclarum theorema_ (splendid theorem).
...
1791 prth $p |- ( ( ( ph -> ps ) /\ ( ch -> th ) ) -> ( ( ph
      /\ ch ) -> ( ps /\ th ) ) ) $= ... $.
Its mandatory hypotheses in RPN order are:
  wph $f wff ph $.
  wps $f wff ps $.
  wch $f wff ch $.
  wth $f wff th $.
Its optional hypotheses are:  wta wet wze wsi wrh wmu wla wka
The statement and its hypotheses require the variables:  ph
      ps ch th
These additional variables are allowed in its proof:  ta et
      ze si rh mu la ka
The variables it contains are:  ph ps ch th


MM> show proof prth /lemmon/renumber
1 simpl          $p |- ( ( ( ph -> ps ) /\ ( ch -> th ) ) ->
                                               ( ph -> ps ) )
2 simpr          $p |- ( ( ( ph -> ps ) /\ ( ch -> th ) ) ->
                                               ( ch -> th ) )
3 1,2 anim12d    $p |- ( ( ( ph -> ps ) /\ ( ch -> th ) ) ->
                           ( ( ph /\ ch ) -> ( ps /\ th ) ) )
\end{verbatim}

There are references to a lot of unfamiliar statements.  To see what they are,
you may type the following:

\begin{verbatim}
MM> show proof prth /statement_summary
Summary of statements used in the proof of "prth":

Statement simpl is located on line 14748 of the file
"set.mm".
"Elimination of a conjunct.  Theorem *3.26 (Simp) of
[WhiteheadRussell] p. 112. ..."
  simpl $p |- ( ( ph /\ ps ) -> ph ) $= ... $.

Statement simpr is located on line 14777 of the file
"set.mm".
"Elimination of a conjunct.  Theorem *3.27 (Simp) of
[WhiteheadRussell] ..."
  simpr $p |- ( ( ph /\ ps ) -> ps ) $= ... $.

Statement anim12d is located on line 15445 of the file
"set.mm".
"Conjoin antecedents and consequents in a deduction.
..."
  anim12d.1 $e |- ( ph -> ( ps -> ch ) ) $.
  anim12d.2 $e |- ( ph -> ( th -> ta ) ) $.
  anim12d $p |- ( ph -> ( ( ps /\ th ) -> ( ch /\ ta ) ) )
      $= ... $.
\end{verbatim}
\begin{center}
(etc.)
\end{center}

Of course you can look at each of these statements and their proofs, and
so on, back to the axioms of propositional calculus if you wish.

The \texttt{search} command is useful for finding statements when you
know all or part of their contents.  The following example finds all
statements containing \verb@ph -> ps@ followed by \verb@ch -> th@.  The
\verb@$*@ is a wildcard that matches anything; the \texttt{\$} before the
\verb$*$ prevents conflicts with math symbol token names.  The \verb@*@ after
\texttt{SEARCH} is also a wildcard that in this case means ``match any label.''
\index{\texttt{search} command}

% I'm omitting this one, since readers are unlikely to see it:
% 1096 bisymOLD $p |- ( ( ( ph -> ps ) -> ( ch -> th ) ) -> ( (
%   ( ps -> ph ) -> ( th -> ch ) ) -> ( ( ph <-> ps ) -> ( ch
%    <-> th ) ) ) )
\begin{verbatim}
MM> search * "ph -> ps $* ch -> th"
1791 prth $p |- ( ( ( ph -> ps ) /\ ( ch -> th ) ) -> ( ( ph
    /\ ch ) -> ( ps /\ th ) ) )
2455 pm3.48 $p |- ( ( ( ph -> ps ) /\ ( ch -> th ) ) -> ( (
    ph \/ ch ) -> ( ps \/ th ) ) )
117859 pm11.71 $p |- ( ( E. x ph /\ E. y ch ) -> ( ( A. x (
    ph -> ps ) /\ A. y ( ch -> th ) ) <-> A. x A. y ( ( ph /\
    ch ) -> ( ps /\ th ) ) ) )
\end{verbatim}

Three statements, \texttt{prth}, \texttt{pm3.48},
 and \texttt{pm11.71}, were found to match.

To see what axioms\index{axiom} and definitions\index{definition}
\texttt{prth} ultimately depends on for its proof, you can have the
program backtrack through the hierarchy\index{hierarchy} of theorems and
definitions.\index{\texttt{show trace{\char`\_}back} command}

\begin{verbatim}
MM> show trace_back prth /essential/axioms
Statement "prth" assumes the following axioms ($a
statements):
  ax-1 ax-2 ax-3 ax-mp df-bi df-an
\end{verbatim}

Note that the 3 axioms of propositional calculus and the modus ponens rule are
needed (as expected); in addition, there are a couple of definitions that are used
along the way.  Note that Metamath makes no distinction\index{axiom vs.\
definition} between axioms\index{axiom} and definitions\index{definition}.  In
\texttt{set.mm} they have been distinguished artificially by prefixing their
labels\index{labels in \texttt{set.mm}} with \texttt{ax-} and \texttt{df-}
respectively.  For example, \texttt{df-an} defines conjunction (logical {\sc
and}), which is represented by the symbol \verb+/\+.
Section~\ref{definitions} discusses the philosophy of definitions, and the
Metamath language takes a particularly simple, conservative approach by using
the \texttt{\$a}\index{\texttt{\$a} statement} statement for both axioms and
definitions.

You can also have the program compute how many steps a proof
has\index{proof length} if we were to follow it all the way back to
\texttt{\$a} statements.

\begin{verbatim}
MM> show trace_back prth /essential/count_steps
The statement's actual proof has 3 steps.  Backtracking, a
total of 79 different subtheorems are used.  The statement
and subtheorems have a total of 274 actual steps.  If
subtheorems used only once were eliminated, there would be a
total of 38 subtheorems, and the statement and subtheorems
would have a total of 185 steps.  The proof would have 28349
steps if fully expanded back to axiom references.  The
maximum path length is 38.  A longest path is:  prth <-
anim12d <- syl2and <- sylan2d <- ancomsd <- ancom <- pm3.22
<- pm3.21 <- pm3.2 <- ex <- sylbir <- biimpri <- bicomi <-
bicom1 <- bi2 <- dfbi1 <- impbii <- bi3 <- simprim <- impi <-
con1i <- nsyl2 <- mt3d <- con1d <- notnot1 <- con2i <- nsyl3
<- mt2d <- con2d <- notnot2 <- pm2.18d <- pm2.18 <- pm2.21 <-
pm2.21d <- a1d <- syl <- mpd <- a2i <- a2i.1 .
\end{verbatim}

This tells us that we would have to inspect 274 steps if we want to
verify the proof completely starting from the axioms.  A few more
statistics are also shown.  There are one or more paths back to axioms
that are the longest; this command ferrets out one of them and shows it
to you.  There may be a sense in which the longest path length is
related to how ``deep'' the theorem is.

We might also be curious about what proofs depend on the theorem
\texttt{prth}.  If it is never used later on, we could eliminate it as
redundant if it has no intrinsic interest by itself.\index{\texttt{show
usage} command}

% I decided to show the OLD values here.
\begin{verbatim}
MM> show usage prth
Statement "prth" is directly referenced in the proofs of 18
statements:
  mo3 moOLD 2mo 2moOLD euind reuind reuss2 reusv3i opelopabt
  wemaplem2 rexanre rlimcn2 o1of2 o1rlimmul 2sqlem6 spanuni
  heicant pm11.71
\end{verbatim}

Thus \texttt{prth} is directly used by 18 proofs.
We can use the \texttt{/recursive} qualifier to include indirect use:

\begin{verbatim}
MM> show usage prth /recursive
Statement "prth" directly or indirectly affects the proofs of
24214 statements:
  mo3 mo mo3OLD eu2 moOLD eu2OLD eu3OLD mo4f mo4 eu4 mopick
...
\end{verbatim}

\subsection{A Note on the ``Compact'' Proof Format}

The Metamath program will display proofs in a ``compact''\index{compact proof}
format whenever the proof is stored in compressed format in the database.  It
may be be slightly confusing unless you know how to interpret it.
For example,
if you display the complete proof of theorem \texttt{id1} it will start
off as follows:

\begin{verbatim}
MM> show proof id1 /lemmon/all
 1 wph           $f wff ph
 2 wph           $f wff ph
 3 wph           $f wff ph
 4 2,3 wi    @4: $a wff ( ph -> ph )
 5 1,4 wi    @5: $a wff ( ph -> ( ph -> ph ) )
 6 @4            $a wff ( ph -> ph )
\end{verbatim}

\begin{center}
{etc.}
\end{center}

Step 4 has a ``local label,''\index{local label} \texttt{@4}, assigned to it.
Later on, at step 6, the label \texttt{@1} is referenced instead of
displaying the explicit proof for that step.  This technique takes advantage
of the fact that steps in a proof often repeat, especially during the
construction of wffs.  The compact format reduces the number of steps in the
proof display and may be preferred by some people.

If you want to see the normal format with the ``true'' step numbers, you can
use the following workaround:\index{\texttt{save proof} command}

\begin{verbatim}
MM> save proof id1 /normal
The proof of "id1" has been reformatted and saved internally.
Remember to use WRITE SOURCE to save it permanently.
MM> show proof id1 /lemmon/all
 1 wph           $f wff ph
 2 wph           $f wff ph
 3 wph           $f wff ph
 4 2,3 wi        $a wff ( ph -> ph )
 5 1,4 wi        $a wff ( ph -> ( ph -> ph ) )
 6 wph           $f wff ph
 7 wph           $f wff ph
 8 6,7 wi        $a wff ( ph -> ph )
\end{verbatim}

\begin{center}
{etc.}
\end{center}

Note that the original 6 steps are now 8 steps.  However, the format is
now the same as that described in Chapter~\ref{using}.

\chapter{The Metamath Language}
\label{languagespec}

\begin{quote}
  {\em Thus mathematics may be defined as the subject in which we never know
what we are talking about, nor whether what we are saying is true.}
    \flushright\sc  Bertrand Russell\footnote{\cite[p.~84]{Russell2}.}\\
\end{quote}\index{Russell, Bertrand}

Probably the most striking feature of the Metamath language is its almost
complete absence of hard-wired syntax. Metamath\index{Metamath} does not
understand any mathematics or logic other than that needed to construct finite
sequences of symbols according to a small set of simple, built-in rules.  The
only rule it uses in a proof is the substitution of an expression (symbol
sequence) for a variable, subject to a simple constraint to prevent
bound-variable clashes.  The primitive notions built into Metamath involve the
simple manipulation of finite objects (symbols) that we as humans can easily
visualize and that computers can easily deal with.  They seem to be just
about the simplest notions possible that are required to do standard
mathematics.

This chapter serves as a reference manual for the Metamath\index{Metamath}
language. It covers the tedious technical details of the language, some of
which you may wish to skim in a first reading.  On the other hand, you should
pay close attention to the defined terms in {\bf boldface}; they have precise
meanings that are important to keep in mind for later understanding.  It may
be best to first become familiar with the examples in Chapter~\ref{using} to
gain some motivation for the language.

%% Uncomment this when uncommenting section {formalspec} below
If you have some knowledge of set theory, you may wish to study this
chapter in conjunction with the formal set-theoretical description of the
Metamath language in Appendix~\ref{formalspec}.

We will use the name ``Metamath''\index{Metamath} to mean either the Metamath
computer language or the Metamath software associated with the computer
language.  We will not distinguish these two when the context is clear.

The next section contains the complete specification of the Metamath
language.
It serves as an
authoritative reference and presents the syntax in enough detail to
write a parser\index{parsing Metamath} and proof verifier.  The
specification is terse and it is probably hard to learn the language
directly from it, but we include it here for those impatient people who
prefer to see everything up front before looking at verbose expository
material.  Later sections explain this material and provide examples.
We will repeat the definitions in those sections, and you may skip the
next section at first reading and proceed to Section~\ref{tut1}
(p.~\pageref{tut1}).

\section{Specification of the Metamath Language}\label{spec}
\index{Metamath!specification}

\begin{quote}
  {\em Sometimes one has to say difficult things, but one ought to say
them as simply as one knows how.}
    \flushright\sc  G. H. Hardy\footnote{As quoted in
    \cite{deMillo}, p.~273.}\\
\end{quote}\index{Hardy, G. H.}

\subsection{Preliminaries}\label{spec1}

% Space is technically a printable character, so we'll word things
% carefully so it's unambiguous.
A Metamath {\bf database}\index{database} is built up from a top-level source
file together with any source files that are brought in through file inclusion
commands (see below).  The only characters that are allowed to appear in a
Metamath source file are the 94 non-whitespace printable {\sc
ascii}\index{ascii@{\sc ascii}} characters, which are digits, upper and lower
case letters, and the following 32 special
characters\index{special characters}:\label{spec1chars}

\begin{verbatim}
! " # $ % & ' ( ) * + , - . / :
; < = > ? @ [ \ ] ^ _ ` { | } ~
\end{verbatim}
plus the following characters which are the ``white space'' characters:
space (a printable character),
tab, carriage return, line feed, and form feed.\label{whitespace}
We will use \texttt{typewriter}
font to display the printable characters.

A Metamath database consists of a sequence of three kinds of {\bf
tokens}\index{token} separated by {\bf white space}\index{white space}
(which is any sequence of one or more white space characters).  The set
of {\bf keyword}\index{keyword} tokens is \texttt{\$\char`\{},
\texttt{\$\char`\}}, \texttt{\$c}, \texttt{\$v}, \texttt{\$f},
\texttt{\$e}, \texttt{\$d}, \texttt{\$a}, \texttt{\$p}, \texttt{\$.},
\texttt{\$=}, \texttt{\$(}, \texttt{\$)}, \texttt{\$[}, and
\texttt{\$]}.  The last four are called {\bf auxiliary}\index{auxiliary
keyword} or preprocessing keywords.  A {\bf label}\index{label} token
consists of any combination of letters, digits, and the characters
hyphen, underscore, and period.  A {\bf math symbol}\index{math symbol}
token may consist of any combination of the 93 printable standard {\sc
ascii} characters other than space or \texttt{\$}~. All tokens are
case-sensitive.

\subsection{Preprocessing}

The token \texttt{\$(} begins a {\bf comment} and
\texttt{\$)} ends a comment.\index{\texttt{\$(}
and \texttt{\$)} auxiliary keywords}\index{comment}
Comments may contain any of
the 94 non-whitespace printable characters and white space,
except they may not contain the
2-character sequences \texttt{\$(} or \texttt{\$)} (comments do not nest).
Comments are ignored (treated
like white space) for the purpose of parsing, e.g.,
\texttt{\$( \$[ \$)} is a comment.
See p.~\pageref{mathcomments} for comment typesetting conventions; these
conventions may be ignored for the purpose of parsing.

A {\bf file inclusion command} consists of \texttt{\$[} followed by a file name
followed by \texttt{\$]}.\index{\texttt{\$[} and \texttt{\$]} auxiliary
keywords}\index{included file}\index{file inclusion}
It is only allowed in the outermost scope (i.e., not between
\texttt{\$\char`\{} and \texttt{\$\char`\}})
and must not be inside a statement (e.g., it may not occur
between the label of a \texttt{\$a} statement and its \texttt{\$.}).
The file name may not
contain a \texttt{\$} or white space.  The file must exist.
The case-sensitivity
of its name follows the conventions of the operating system.  The contents of
the file replace the inclusion command.
Included files may include other files.
Only the first reference to a given file is included; any later
references to the same file (whether in the top-level file or in included
files) cause the inclusion command to be ignored (treated like white space).
A verifier may assume that file names with different strings
refer to different files for the purpose of ignoring later references.
A file self-reference is ignored, as is any reference to the top-level file
(to avoid loops).
Included files may not include a \texttt{\$(} without a matching \texttt{\$)},
may not include a \texttt{\$[} without a matching \texttt{\$]}, and may
not include incomplete statements (e.g., a \texttt{\$a} without a matching
\texttt{\$.}).
It is currently unspecified if path references are relative to the process'
current directory or the file's containing directory, so databases should
avoid using pathname separators (e.g., ``/'') in file names.

Like all tokens, the \texttt{\$(}, \texttt{\$)}, \texttt{\$[}, and \texttt{\$]} keywords
must be surrounded by white space.

\subsection{Basic Syntax}

After preprocessing, a database will consist of a sequence of {\bf
statements}.
These are the scoping statements \texttt{\$\char`\{} and
\texttt{\$\char`\}}, along with the \texttt{\$c}, \texttt{\$v},
\texttt{\$f}, \texttt{\$e}, \texttt{\$d}, \texttt{\$a}, and \texttt{\$p}
statements.

A {\bf scoping statement}\index{scoping statement} consists only of its
keyword, \texttt{\$\char`\{} or \texttt{\$\char`\}}.
A \texttt{\$\char`\{} begins a {\bf
block}\index{block} and a matching \texttt{\$\char`\}} ends the block.
Every \texttt{\$\char`\{}
must have a matching \texttt{\$\char`\}}.
Defining it recursively, we say a block
contains a sequence of zero or more tokens other
than \texttt{\$\char`\{} and \texttt{\$\char`\}} and
possibly other blocks.  There is an {\bf outermost
block}\index{block!outermost} not bracketed by \texttt{\$\char`\{} \texttt{\$\char`\}}; the end
of the outermost block is the end of the database.

% LaTeX bug? can't do \bf\tt

A {\bf \$v} or {\bf \$c statement}\index{\texttt{\$v} statement}\index{\texttt{\$c}
statement} consists of the keyword token \texttt{\$v} or \texttt{\$c} respectively,
followed by one or more math symbols,
% The word "token" is used to distinguish "$." from the sentence-ending period.
followed by the \texttt{\$.}\ token.
These
statements {\bf declare}\index{declaration} the math symbols to be {\bf
variables}\index{variable!Metamath} or {\bf constants}\index{constant}
respectively. The same math symbol may not occur twice in a given \texttt{\$v} or
\texttt{\$c} statement.

%c%A math symbol becomes an {\bf active}\index{active math symbol}
%c%when declared and stays active until the end of the block in which it is
%c%declared.  A math symbol may not be declared a second time while it is active,
%c%but it may be declared again after it becomes inactive.

A math symbol becomes {\bf active}\index{active math symbol} when declared
and stays active until the end of the block in which it is declared.  A
variable may not be declared a second time while it is active, but it
may be declared again (as a variable, but not as a constant) after it
becomes inactive.  A constant must be declared in the outermost block and may
not be declared a second time.\index{redeclaration of symbols}

A {\bf \$f statement}\index{\texttt{\$f} statement} consists of a label,
followed by \texttt{\$f}, followed by its typecode (an active constant),
followed by an
active variable, followed by the \texttt{\$.}\ token.  A {\bf \$e
statement}\index{\texttt{\$e} statement} consists of a label, followed
by \texttt{\$e}, followed by its typecode (an active constant),
followed by zero or more
active math symbols, followed by the \texttt{\$.}\ token.  A {\bf
hypothesis}\index{hypothesis} is a \texttt{\$f} or \texttt{\$e}
statement.
The type declared by a \texttt{\$f} statement for a given label
is global even if the variable is not
(e.g., a database may not have \texttt{wff P} in one local scope
and \texttt{class P} in another).

A {\bf simple \$d statement}\index{\texttt{\$d} statement!simple}
consists of \texttt{\$d}, followed by two different active variables,
followed by the \texttt{\$.}\ token.  A {\bf compound \$d
statement}\index{\texttt{\$d} statement!compound} consists of
\texttt{\$d}, followed by three or more variables (all different),
followed by the \texttt{\$.}\ token.  The order of the variables in a
\texttt{\$d} statement is unimportant.  A compound \texttt{\$d}
statement is equivalent to a set of simple \texttt{\$d} statements, one
for each possible pair of variables occurring in the compound
\texttt{\$d} statement.  Henceforth in this specification we shall
assume all \texttt{\$d} statements are simple.  A \texttt{\$d} statement
is also called a {\bf disjoint} (or {\bf distinct}) {\bf variable
restriction}.\index{disjoint-variable restriction}

A {\bf \$a statement}\index{\texttt{\$a} statement} consists of a label,
followed by \texttt{\$a}, followed by its typecode (an active constant),
followed by
zero or more active math symbols, followed by the \texttt{\$.}\ token.  A {\bf
\$p statement}\index{\texttt{\$p} statement} consists of a label,
followed by \texttt{\$p}, followed by its typecode (an active constant),
followed by
zero or more active math symbols, followed by \texttt{\$=}, followed by
a sequence of labels, followed by the \texttt{\$.}\ token.  An {\bf
assertion}\index{assertion} is a \texttt{\$a} or \texttt{\$p} statement.

A \texttt{\$f}, \texttt{\$e}, or \texttt{\$d} statement is {\bf active}\index{active
statement} from the place it occurs until the end of the block it occurs in.
A \texttt{\$a} or \texttt{\$p} statement is {\bf active} from the place it occurs
through the end of the database.
There may not be two active \texttt{\$f} statements containing the same
variable.  Each variable in a \texttt{\$e}, \texttt{\$a}, or
\texttt{\$p} statement must exist in an active \texttt{\$f}
statement.\footnote{This requirement can greatly simplify the
unification algorithm (substitution calculation) required by proof
verification.}

%The label that begins each \texttt{\$f}, \texttt{\$e}, \texttt{\$a}, and
%\texttt{\$p} statement must be unique.
Each label token must be unique, and
no label token may match any math symbol
token.\label{namespace}\footnote{This
restriction was added on June 24, 2006.
It is not theoretically necessary but is imposed to make it easier to
write certain parsers.}

The set of {\bf mandatory variables}\index{mandatory variable} associated with
an assertion is the set of (zero or more) variables in the assertion and in any
active \texttt{\$e} statements.  The (possibly empty) set of {\bf mandatory
hypotheses}\index{mandatory hypothesis} is the set of all active \texttt{\$f}
statements containing mandatory variables, together with all active \texttt{\$e}
statements.
The set of {\bf mandatory {\bf \$d} statements}\index{mandatory
disjoint-variable restriction} associated with an assertion are those active
\texttt{\$d} statements whose variables are both among the assertion's
mandatory variables.

\subsection{Proof Verification}\label{spec4}

The sequence of labels between the \texttt{\$=} and \texttt{\$.}\ tokens
in a \texttt{\$p} statement is a {\bf proof}.\index{proof!Metamath} Each
label in a proof must be the label of an active statement other than the
\texttt{\$p} statement itself; thus a label must refer either to an
active hypothesis of the \texttt{\$p} statement or to an earlier
assertion.

An {\bf expression}\index{expression} is a sequence of math symbols. A {\bf
substitution map}\index{substitution map} associates a set of variables with a
set of expressions.  It is acceptable for a variable to be mapped to an
expression containing it.  A {\bf
substitution}\index{substitution!variable}\index{variable substitution} is the
simultaneous replacement of all variables in one or more expressions with the
expressions that the variables map to.

A proof is scanned in order of its label sequence.  If the label refers to an
active hypothesis, the expression in the hypothesis is pushed onto a
stack.\index{stack}\index{RPN stack}  If the label refers to an assertion, a
(unique) substitution must exist that, when made to the mandatory hypotheses
of the referenced assertion, causes them to match the topmost (i.e.\ most
recent) entries of the stack, in order of occurrence of the mandatory
hypotheses, with the topmost stack entry matching the last mandatory
hypothesis of the referenced assertion.  As many stack entries as there are
mandatory hypotheses are then popped from the stack.  The same substitution is
made to the referenced assertion, and the result is pushed onto the stack.
After the last label in the proof is processed, the stack must have a single
entry that matches the expression in the \texttt{\$p} statement containing the
proof.

%c%{\footnotesize\begin{quotation}\index{redeclaration of symbols}
%c%{{\em Comment.}\label{spec4comment} Whenever a math symbol token occurs in a
%c%{\texttt{\$c} or \texttt{\$v} statement, it is considered to designate a distinct new
%c%{symbol, even if the same token was previously declared (and is now inactive).
%c%{Thus a math token declared as a constant in two different blocks is considered
%c%{to designate two distinct constants (even though they have the same name).
%c%{The two constants will not match in a proof that references both blocks.
%c%{However, a proof referencing both blocks is acceptable as long as it doesn't
%c%{require that the constants match.  Similarly, a token declared to be a
%c%{constant for a referenced assertion will not match the same token declared to
%c%{be a variable for the \texttt{\$p} statement containing the proof.  In the case
%c%{of a token declared to be a variable for a referenced assertion, this is not
%c%{an issue since the variable can be substituted with whatever expression is
%c%{needed to achieve the required match.
%c%{\end{quotation}}
%c2%A proof may reference an assertion that contains or whose hypotheses contain a
%c2%constant that is not active for the \texttt{\$p} statement containing the proof.
%c2%However, the final result of the proof may not contain that constant. A proof
%c2%may also reference an assertion that contains or whose hypotheses contain a
%c2%variable that is not active for the \texttt{\$p} statement containing the proof.
%c2%That variable, of course, will be substituted with whatever expression is
%c2%needed to achieve the required match.

A proof may contain a \texttt{?}\ in place of a label to indicate an unknown step
(Section~\ref{unknown}).  A proof verifier may ignore any proof containing
\texttt{?}\ but should warn the user that the proof is incomplete.

A {\bf compressed proof}\index{compressed proof}\index{proof!compressed} is an
alternate proof notation described in Appen\-dix~\ref{compressed}; also see
references to ``compressed proof'' in the Index.  Compressed proofs are a
Metamath language extension which a complete proof verifier should be able to
parse and verify.

\subsubsection{Verifying Disjoint Variable Restrictions}

Each substitution made in a proof must be checked to verify that any
disjoint variable restrictions are satisfied, as follows.

If two variables replaced by a substitution exist in a mandatory \texttt{\$d}
statement\index{\texttt{\$d} statement} of the assertion referenced, the two
expressions resulting from the substitution must satisfy the following
conditions.  First, the two expressions must have no variables in common.
Second, each possible pair of variables, one from each expression, must exist
in an active \texttt{\$d} statement of the \texttt{\$p} statement containing the
proof.

\vskip 1ex

This ends the specification of the Metamath language;
see Appendix \ref{BNF} for its syntax in
Extended Backus--Naur Form (EBNF)\index{Extended Backus--Naur Form}\index{EBNF}.

\section{The Basic Keywords}\label{tut1}

Our expository material begins here.

Like most computer languages, Metamath\index{Metamath} takes its input from
one or more {\bf source files}\index{source file} which contain characters
expressed in the standard {\sc ascii} (American Standard Code for Information
Interchange)\index{ascii@{\sc ascii}} code for computers.  A source file
consists of a series of {\bf tokens}\index{token}, which are strings of
non-whitespace
printable characters (from the set of 94 shown on p.~\pageref{spec1chars})
separated by {\bf white space}\index{white space} (spaces, tabs, carriage
returns, line feeds, and form feeds). Any string consisting only of these
characters is treated the same as a single space.  The non-whitespace printable
characters\index{printable character} that Metamath recognizes are the 94
characters on standard {\sc ascii} keyboards.

Metamath has the ability to join several files together to form its
input (Section~\ref{include}).  We call the aggregate contents of all
the files after they have been joined together a {\bf
database}\index{database} to distinguish it from an individual source
file.  The tokens in a database consist of {\bf
keywords}\index{keyword}, which are built into the language, together
with two kinds of user-defined tokens called {\bf labels}\index{label}
and {\bf math symbols}\index{math symbol}.  (Often we will simply say
{\bf symbol}\index{symbol} instead of math symbol for brevity).  The set
of {\bf basic keywords}\index{basic keyword} is
\texttt{\$c}\index{\texttt{\$c} statement},
\texttt{\$v}\index{\texttt{\$v} statement},
\texttt{\$e}\index{\texttt{\$e} statement},
\texttt{\$f}\index{\texttt{\$f} statement},
\texttt{\$d}\index{\texttt{\$d} statement},
\texttt{\$a}\index{\texttt{\$a} statement},
\texttt{\$p}\index{\texttt{\$p} statement},
\texttt{\$=}\index{\texttt{\$=} keyword},
\texttt{\$.}\index{\texttt{\$.}\ keyword},
\texttt{\$\char`\{}\index{\texttt{\$\char`\{} and \texttt{\$\char`\}}
keywords}, and \texttt{\$\char`\}}.  This is the complete set of
syntactical elements of what we call the {\bf basic
language}\index{basic language} of Metamath, and with them you can
express all of the mathematics that were intended by the design of
Metamath.  You should make it a point to become very familiar with them.
Table~\ref{basickeywords} lists the basic keywords along with a brief
description of their functions.  For now, this description will give you
only a vague notion of what the keywords are for; later we will describe
the keywords in detail.


\begin{table}[htp] \caption{Summary of the basic Metamath
keywords} \label{basickeywords}
\begin{center}
\begin{tabular}{|p{4pc}|l|}
\hline
\em \centering Keyword&\em Description\\
\hline
\hline
\centering
   \texttt{\$c}&Constant symbol declaration\\
\hline
\centering
   \texttt{\$v}&Variable symbol declaration\\
\hline
\centering
   \texttt{\$d}&Disjoint variable restriction\\
\hline
\centering
   \texttt{\$f}&Variable-type (``floating'') hypothesis\\
\hline
\centering
   \texttt{\$e}&Logical (``essential'') hypothesis\\
\hline
\centering
   \texttt{\$a}&Axiomatic assertion\\
\hline
\centering
   \texttt{\$p}&Provable assertion\\
\hline
\centering
   \texttt{\$=}&Start of proof in \texttt{\$p} statement\\
\hline
\centering
   \texttt{\$.}&End of the above statement types\\
\hline
\centering
   \texttt{\$\char`\{}&Start of block\\
\hline
\centering
   \texttt{\$\char`\}}&End of block\\
\hline
\end{tabular}
\end{center}
\end{table}

%For LaTeX bug(?) where it puts tables on blank page instead of btwn text
%May have to adjust if text changes
%\newpage

There are some additional keywords, called {\bf auxiliary
keywords}\index{auxiliary keyword} that help make Metamath\index{Metamath}
more practical. These are part of the {\bf extended language}\index{extended
language}. They provide you with a means to put comments into a Metamath
source file\index{source file} and reference other source files.  We will
introduce these in later sections. Table~\ref{otherkeywords} summarizes them
so that you can recognize them now if you want to peruse some source
files while learning the basic keywords.


\begin{table}[htp] \caption{Auxiliary Metamath
keywords} \label{otherkeywords}
\begin{center}
\begin{tabular}{|p{4pc}|l|}
\hline
\em \centering Keyword&\em Description\\
\hline
\hline
\centering
   \texttt{\$(}&Start of comment\\
\hline
\centering
   \texttt{\$)}&End of comment\\
\hline
\centering
   \texttt{\$[}&Start of included source file name\\
\hline
\centering
   \texttt{\$]}&End of included source file name\\
\hline
\end{tabular}
\end{center}
\end{table}
\index{\texttt{\$(} and \texttt{\$)} auxiliary keywords}
\index{\texttt{\$[} and \texttt{\$]} auxiliary keywords}


Unlike those in some computer languages, the keywords\index{keyword} are short
two-character sequences rather than English-like words.  While this may make
them slightly more difficult to remember at first, their brevity allows
them to blend in with the mathematics being described, not
distract from it, like punctuation marks.


\subsection{User-Defined Tokens}\label{dollardollar}\index{token}

As you may have noticed, all keywords\index{keyword} begin with the \texttt{\$}
character.  This mundane monetary symbol is not ordinarily used in higher
mathematics (outside of grant proposals), so we have appropriated it to
distinguish the Metamath\index{Metamath} keywords from ordinary mathematical
symbols. The \texttt{\$} character is thus considered special and may not be
used as a character in a user-defined token.  All tokens and keywords are
case-sensitive; for example, \texttt{n} is considered to be a different character
from \texttt{N}.  Case-sensitivity makes the available {\sc ascii} character set
as rich as possible.

\subsubsection{Math Symbol Tokens}\index{token}

Math symbols\index{math symbol} are tokens used to represent the symbols
that appear in ordinary mathematical formulas.  They may consist of any
combination of the 93 non-whitespace printable {\sc ascii} characters other than
\texttt{\$}~. Some examples are \texttt{x}, \texttt{+}, \texttt{(},
\texttt{|-}, \verb$!%@?&$, and \texttt{bounded}.  For readability, it is
best to try to make these look as similar to actual mathematical symbols
as possible, within the constraints of the {\sc ascii} character set, in
order to make the resulting mathematical expressions more readable.

In the Metamath\index{Metamath} language, you express ordinary
mathematical formulas and statements as sequences of math symbols such
as \texttt{2 + 2 = 4} (five symbols, all constants).\footnote{To
eliminate ambiguity with other expressions, this is expressed in the set
theory database \texttt{set.mm} as \texttt{|- ( 2 + 2
 ) = 4 }, whose \LaTeX\ equivalent is $\vdash
(2+2)=4$.  The \,$\vdash$ means ``is a theorem'' and the
parentheses allow explicit associative grouping.}\index{turnstile
({$\,\vdash$})} They may even be English
sentences, as in \texttt{E is closed and bounded} (five symbols)---here
\texttt{E} would be a variable and the other four symbols constants.  In
principle, a Metamath database could be constructed to work with almost
any unambiguous English-language mathematical statement, but as a
practical matter the definitions needed to provide for all possible
syntax variations would be cumbersome and distracting and possibly have
subtle pitfalls accidentally built in.  We generally recommend that you
express mathematical statements with compact standard mathematical
symbols whenever possible and put their English-language descriptions in
comments.  Axioms\index{axiom} and definitions\index{definition}
(\texttt{\$a}\index{\texttt{\$a} statement} statements) are the only
places where Metamath will not detect an error, and doing this will help
reduce the number of definitions needed.

You are free to use any tokens\index{token} you like for math
symbols\index{math symbol}.  Appendix~\ref{ASCII} recommends token names to
use for symbols in set theory, and we suggest you adopt these in order to be
able to include the \texttt{set.mm} set theory database in your database.  For
printouts, you can convert the tokens in a database
to standard mathematical symbols with the \LaTeX\ typesetting program.  The
Metamath command \texttt{open tex} {\em filename}\index{\texttt{open tex} command}
produces output that can be read by \LaTeX.\index{latex@{\LaTeX}}
The correspondence
between tokens and the actual symbols is made by \texttt{latexdef}
statements inside a special database comment tagged
with \texttt{\$t}.\index{\texttt{\$t} comment}\index{typesetting comment}
  You can edit
this comment to change the definitions or add new ones.
Appendix~\ref{ASCII} describes how to do this in more detail.

% White space\index{white space} is normally used to separate math
% symbol\index{math symbol} tokens, but they may be juxtaposed without white
% space in \texttt{\$d}\index{\texttt{\$d} statement}, \texttt{\$e}\index{\texttt{\$e}
% statement}, \texttt{\$f}\index{\texttt{\$f} statement}, \texttt{\$a}\index{\texttt{\$a}
% statement}, and \texttt{\$p}\index{\texttt{\$p} statement} statements when no
% ambiguity will result.  Specifically, Metamath parses the math symbol sequence
% in one of these statements in the following manner:  when the math symbol
% sequence has been broken up into tokens\index{token} up to a given character,
% the next token is the longest string of characters that could constitute a
% math symbol that is active\index{active
% math symbol} at that point.  (See Section~\ref{scoping} for the
% definition of an active math symbol.)  For example, if \texttt{-}, \texttt{>}, and
% \texttt{->} are the only active math symbols, the juxtaposition \texttt{>-} will be
% interpreted as the two symbols \texttt{>} and \texttt{-}, whereas \texttt{->} will
% always be interpreted as that single symbol.\footnote{For better readability we
% recommend a white space between each token.  This also makes searching for a
% symbol easier to do with an editor.  Omission of optional white space is useful
% for reducing typing when assigning an expression to a temporary
% variable\index{temporary variable} with the \texttt{let variable} Metamath
% program command.}\index{\texttt{let variable} command}
%
% Keywords\index{keyword} may be placed next to math symbols without white
% space\index{white space} between them.\footnote{Again, we do not recommend
% this for readability.}
%
% The math symbols\index{math symbol} in \texttt{\$c}\index{\texttt{\$c} statement}
% and \texttt{\$v}\index{\texttt{\$v} statement} statements must always be separated
% by white space\index{white
% space}, for the obvious reason that these statements define the names
% of the symbols.
%
% Math symbols referred to in comments (see Section~\ref{comments}) must also be
% separated by white space.  This allows you to make comments about symbols that
% are not yet active\index{active
% math symbol}.  (The ``math mode'' feature of comments is also a quick and
% easy way to obtain word processing text with embedded mathematical symbols,
% independently of the main purpose of Metamath; the way to do this is described
% in Section~\ref{comments})

\subsubsection{Label Tokens}\index{token}\index{label}

Label tokens are used to identify Metamath\index{Metamath} statements for
later reference. Label tokens may contain only letters, digits, and the three
characters period, hyphen, and underscore:
\begin{verbatim}
. - _
\end{verbatim}

A label is {\bf declared}\index{label declaration} by placing it immediately
before the keyword of the statement it identifies.  For example, the label
\texttt{axiom.1} might be declared as follows:
\begin{verbatim}
axiom.1 $a |- x = x $.
\end{verbatim}

Each \texttt{\$e}\index{\texttt{\$e} statement},
\texttt{\$f}\index{\texttt{\$f} statement},
\texttt{\$a}\index{\texttt{\$a} statement}, and
\texttt{\$p}\index{\texttt{\$p} statement} statement in a database must
have a label declared for it.  No other statement types may have label
declarations.  Every label must be unique.

A label (and the statement it identifies) is {\bf referenced}\index{label
reference} by including the label between the \texttt{\$=}\index{\texttt{\$=}
keyword} and \texttt{\$.}\index{\texttt{\$.}\ keyword}\ keywords in a \texttt{\$p}
statement.  The sequence of labels\index{label sequence} between these two
keywords is called a {\bf proof}\index{proof}.  An example of a statement with
a proof that we will encounter later (Section~\ref{proof}) is
\begin{verbatim}
wnew $p wff ( s -> ( r -> p ) )
     $= ws wr wp w2 w2 $.
\end{verbatim}

You don't have to know what this means just yet, but you should know that the
label \texttt{wnew} is declared by this \texttt{\$p} statement and that the labels
\texttt{ws}, \texttt{wr}, \texttt{wp}, and \texttt{w2} are assumed to have been declared
earlier in the database and are referenced here.

\subsection{Constants and Variables}
\index{constant}
\index{variable}

An {\bf expression}\index{expression} is any sequence of math
symbols, possibly empty.

The basic Metamath\index{Metamath} language\index{basic language} has two
kinds of math symbols\index{math symbol}:  {\bf constants}\index{constant} and
{\bf variables}\index{variable}.  In a Metamath proof, a constant may not be
substituted with any expression.  A variable can be
substituted\index{substitution!variable}\index{variable substitution} with any
expression.  This sequence may include other variables and may even include
the variable being substituted.  This substitution takes place when proofs are
verified, and it will be described in Section~\ref{proof}.  The \texttt{\$f}
statement (described later in Section~\ref{dollaref}) is used to specify the
{\bf type} of a variable (i.e.\ what kind of
variable it is)\index{variable type}\index{type} and
give it a meaning typically
associated with a ``metavariable''\index{metavariable}\footnote{A metavariable
is a variable that ranges over the syntactical elements of the object language
being discussed; for example, one metavariable might represent a variable of
the object language and another metavariable might represent a formula in the
object language.} in ordinary mathematics; for example, a variable may be
specified to be a wff or well-formed formula (in logic), a set (in set
theory), or a non-negative integer (in number theory).

%\subsection{The \texttt{\$c} and \texttt{\$v} Declaration Statements}
\subsection{The \texttt{\$c} and \texttt{\$v} Declaration Statements}
\index{\texttt{\$c} statement}
\index{constant declaration}
\index{\texttt{\$v} statement}
\index{variable declaration}

Constants are introduced or {\bf declared}\index{constant declaration}
with \texttt{\$c}\index{\texttt{\$c} statement} statements, and
variables are declared\index{variable declaration} with
\texttt{\$v}\index{\texttt{\$v} statement} statements.  A {\bf simple}
declaration\index{simple declaration} statement introduces a single
constant or variable.  Its syntax is one of the following:
\begin{center}
  \texttt{\$c} {\em math-symbol} \texttt{\$.}\\
  \texttt{\$v} {\em math-symbol} \texttt{\$.}
\end{center}
The notation {\em math-symbol} means any math symbol token\index{token}.

Some examples of simple declaration statements are:
\begin{center}
  \texttt{\$c + \$.}\\
  \texttt{\$c -> \$.}\\
  \texttt{\$c ( \$.}\\
  \texttt{\$v x \$.}\\
  \texttt{\$v y2 \$.}
\end{center}

The characters in a math symbol\index{math symbol} being declared are
irrelevant to Meta\-math; for example, we could declare a right parenthesis to
be a variable,
\begin{center}
  \texttt{\$v ) \$.}\\
\end{center}
although this would be unconventional.

A {\bf compound} declaration\index{compound declaration} statement is a
shorthand for declaring several symbols at once.  Its syntax is one of the
following:
\begin{center}
  \texttt{\$c} {\em math-symbol}\ \,$\cdots$\ {\em math-symbol} \texttt{\$.}\\
  \texttt{\$v} {\em math-symbol}\ \,$\cdots$\ {\em math-symbol} \texttt{\$.}
\end{center}\index{\texttt{\$c} statement}
Here, the ellipsis (\ldots) means any number of {\em math-symbol}\,s.

An example of a compound declaration statement is:
\begin{center}
  \texttt{\$v x y mu \$.}\\
\end{center}
This is equivalent to the three simple declaration statements
\begin{center}
  \texttt{\$v x \$.}\\
  \texttt{\$v y \$.}\\
  \texttt{\$v mu \$.}\\
\end{center}
\index{\texttt{\$v} statement}

There are certain rules on where in the database math symbols may be declared,
what sections of the database are aware of them (i.e.\ where they are
``active''), and when they may be declared more than once.  These will be
discussed in Section~\ref{scoping} and specifically on
p.~\pageref{redeclaration}.

\subsection{The \texttt{\$d} Statement}\label{dollard}
\index{\texttt{\$d} statement}

The \texttt{\$d} statement is called a {\bf disjoint-variable restriction}.  The
syntax of the {\bf simple} version of this statement is
\begin{center}
  \texttt{\$d} {\em variable variable} \texttt{\$.}
\end{center}
where each {\em variable} is a previously declared variable and the two {\em
variable}\,s are different.  (More specifically, each  {\em variable} must be
an {\bf active} variable\index{active math symbol}, which means there must be
a previous \texttt{\$v} statement whose {\bf scope}\index{scope} includes the
\texttt{\$d} statement.  These terms will be defined when we discuss scoping
statements in Section~\ref{scoping}.)

In ordinary mathematics, formulas may arise that are true if the variables in
them are distinct\index{distinct variables}, but become false when those
variables are made identical. For example, the formula in logic $\exists x\,x
\neq y$, which means ``for a given $y$, there exists an $x$ that is not equal
to $y$,'' is true in most mathematical theories (namely all non-trivial
theories\index{non-trivial theory}, i.e.\ those that describe more than one
individual, such as arithmetic).  However, if we substitute $y$ with $x$, we
obtain $\exists x\,x \neq x$, which is always false, as it means ``there
exists something that is not equal to itself.''\footnote{If you are a
logician, you will recognize this as the improper substitution\index{proper
substitution}\index{substitution!proper} of a free variable\index{free
variable} with a bound variable\index{bound variable}.  Metamath makes no
inherent distinction between free and bound variables; instead, you let
Metamath know what substitutions are permissible by using \texttt{\$d} statements
in the right way in your axiom system.}\index{free vs.\ bound variable}  The
\texttt{\$d} statement allows you to specify a restriction that forbids the
substitution of one variable with another.  In
this case, we would use the statement
\begin{center}
  \texttt{\$d x y \$.}
\end{center}\index{\texttt{\$d} statement}
to specify this restriction.

The order in which the variables appear in a \texttt{\$d} statement is not
important.  We could also use
\begin{center}
  \texttt{\$d y x \$.}
\end{center}

The \texttt{\$d} statement is actually more general than this, as the
``disjoint''\index{disjoint variables} in its name suggests.  The full meaning
is that if any substitution is made to its two variables (during the
course of a proof that references a \texttt{\$a} or \texttt{\$p} statement
associated with the \texttt{\$d}), the two expressions that result from the
substitution must have no variables in common.  In addition, each possible
pair of variables, one from each expression, must be in a \texttt{\$d} statement
associated with the statement being proved.  (This requirement forces the
statement being proved to ``inherit'' the original disjoint variable
restriction.)

For example, suppose \texttt{u} is a variable.  If the restriction
\begin{center}
  \texttt{\$d A B \$.}
\end{center}
has been specified for a theorem referenced in a
proof, we may not substitute \texttt{A} with \mbox{\tt a + u} and
\texttt{B} with \mbox{\tt b + u} because these two symbol sequences have the
variable \texttt{u} in common.  Furthermore, if \texttt{a} and \texttt{b} are
variables, we may not substitute \texttt{A} with \texttt{a} and \texttt{B} with \texttt{b}
unless we have also specified \texttt{\$d a b} for the theorem being proved; in
other words, the \texttt{\$d} property associated with a pair of variables must
be effectively preserved after substitution.

The \texttt{\$d}\index{\texttt{\$d} statement} statement does {\em not} mean ``the
two variables may not be substituted with the same thing,'' as you might think
at first.  For example, substituting each of \texttt{A} and \texttt{B} in the above
example with identical symbol sequences consisting only of constants does not
cause a disjoint variable conflict, because two symbol sequences have no
variables in common (since they have no variables, period).  Similarly, a
conflict will not occur by substituting the two variables in a \texttt{\$d}
statement with the empty symbol sequence\index{empty substitution}.

The \texttt{\$d} statement does not have a direct counterpart in
ordinary mathematics, partly because the variables\index{variable} of
Metamath are not really the same as the variables\index{variable!in
ordinary mathematics} of ordinary mathematics but rather are
metavariables\index{metavariable} ranging over them (as well as over
other kinds of symbols and groups of symbols).  Depending on the
situation, we may informally interpret the \texttt{\$d} statement in
different ways.  Suppose, for example, that \texttt{x} and \texttt{y}
are variables ranging over numbers (more precisely, that \texttt{x} and
\texttt{y} are metavariables ranging over variables that range over
numbers), and that \texttt{ph} ($\varphi$) and \texttt{ps} ($\psi$) are
variables (more precisely, metavariables) ranging over formulas.  We can
make the following interpretations that correspond to the informal
language of ordinary mathematics:
\begin{quote}
\begin{tabbing}
\texttt{\$d x y \$.} means ``assume $x$ and $y$ are
distinct variables.''\\
\texttt{\$d x ph \$.} means ``assume $x$ does not
occur in $\varphi$.''\\
\texttt{\$d ph ps \$.} \=means ``assume $\varphi$ and
$\psi$ have no variables\\ \>in common.''
\end{tabbing}
\end{quote}\index{\texttt{\$d} statement}

\subsubsection{Compound \texttt{\$d} Statements}

The {\bf compound} version of the \texttt{\$d} statement is a shorthand for
specifying several variables whose substitutions must be pairwise disjoint.
Its syntax is:
\begin{center}
  \texttt{\$d} {\em variable}\ \,$\cdots$\ {\em variable} \texttt{\$.}
\end{center}\index{\texttt{\$d} statement}
Here, {\em variable} represents the token of a previously declared
variable (specifically, an active variable) and all {\em variable}\,s are
different.  The compound \texttt{\$d}
statement is internally broken up by Metamath into one simple \texttt{\$d}
statement for each possible pair of variables in the original \texttt{\$d}
statement.  For example,
\begin{center}
  \texttt{\$d w x y z \$.}
\end{center}
is equivalent to
\begin{center}
  \texttt{\$d w x \$.}\\
  \texttt{\$d w y \$.}\\
  \texttt{\$d w z \$.}\\
  \texttt{\$d x y \$.}\\
  \texttt{\$d x z \$.}\\
  \texttt{\$d y z \$.}
\end{center}

Two or more simple \texttt{\$d} statements specifying the same variable pair are
internally combined into a single \texttt{\$d} statement.  Thus the set of three
statements
\begin{center}
  \texttt{\$d x y \$.}
  \texttt{\$d x y \$.}
  \texttt{\$d y x \$.}
\end{center}
is equivalent to
\begin{center}
  \texttt{\$d x y \$.}
\end{center}

Similarly, compound \texttt{\$d} statements, after being internally broken up,
internally have their common variable pairs combined.  For example the
set of statements
\begin{center}
  \texttt{\$d x y A \$.}
  \texttt{\$d x y B \$.}
\end{center}
is equivalent to
\begin{center}
  \texttt{\$d x y \$.}
  \texttt{\$d x A \$.}
  \texttt{\$d y A \$.}
  \texttt{\$d x y \$.}
  \texttt{\$d x B \$.}
  \texttt{\$d y B \$.}
\end{center}
which is equivalent to
\begin{center}
  \texttt{\$d x y \$.}
  \texttt{\$d x A \$.}
  \texttt{\$d y A \$.}
  \texttt{\$d x B \$.}
  \texttt{\$d y B \$.}
\end{center}

Metamath\index{Metamath} automatically verifies that all \texttt{\$d}
restrictions are met whenever it verifies proofs.  \texttt{\$d} statements are
never referenced directly in proofs (this is why they do not have
labels\index{label}), but Metamath is always aware of which ones must be
satisfied (i.e.\ are active) and will notify you with an error message if any
violation occurs.

To illustrate how Metamath detects a missing \texttt{\$d}
statement, we will look at the following example from the
\texttt{set.mm} database.

\begin{verbatim}
$d x z $.  $d y z $.
$( Theorem to add distinct quantifier to atomic formula. $)
ax17eq $p |- ( x = y -> A. z x = y ) $=...
\end{verbatim}

This statement has the obvious requirement that $z$ must be
distinct\index{distinct variables} from $x$ in theorem \texttt{ax17eq} that
states $x=y \rightarrow \forall z \, x=y$ (well, obvious if you're a logician,
for otherwise we could conclude  $x=y \rightarrow \forall x \, x=y$, which is
false when the free variables $x$ and $y$ are equal).

Let's look at what happens if we edit the database to comment out this
requirement.

\begin{verbatim}
$( $d x z $. $) $d y z $.
$( Theorem to add distinct quantifier to atomic formula. $)
ax17eq $p |- ( x = y -> A. z x = y ) $=...
\end{verbatim}

When it tries to verify the proof, Metamath will tell you that \texttt{x} and
\texttt{z} must be disjoint, because one of its steps references an axiom or
theorem that has this requirement.

\begin{verbatim}
MM> verify proof ax17eq
ax17eq ?Error at statement 1918, label "ax17eq", type "$p":
      vz wal wi vx vy vz ax-13 vx vy weq vz vx ax-c16 vx vy
                                               ^^^^^
There is a disjoint variable ($d) violation at proof step 29.
Assertion "ax-c16" requires that variables "x" and "y" be
disjoint.  But "x" was substituted with "z" and "y" was
substituted with "x".  The assertion being proved, "ax17eq",
does not require that variables "z" and "x" be disjoint.
\end{verbatim}

We can see the substitutions into \texttt{ax-c16} with the following command.

\begin{verbatim}
MM> show proof ax17eq / detailed_step 29
Proof step 29:  pm2.61dd.2=ax-c16 $a |- ( A. z z = x -> ( x =
  y -> A. z x = y ) )
This step assigns source "ax-c16" ($a) to target "pm2.61dd.2"
($e).  The source assertion requires the hypotheses "wph"
($f, step 26), "vx" ($f, step 27), and "vy" ($f, step 28).
The parent assertion of the target hypothesis is "pm2.61dd"
($p, step 36).
The source assertion before substitution was:
    ax-c16 $a |- ( A. x x = y -> ( ph -> A. x ph ) )
The following substitutions were made to the source
assertion:
    Variable  Substituted with
     x         z
     y         x
     ph        x = y
The target hypothesis before substitution was:
    pm2.61dd.2 $e |- ( ph -> ch )
The following substitutions were made to the target
hypothesis:
    Variable  Substituted with
     ph        A. z z = x
     ch        ( x = y -> A. z x = y )
\end{verbatim}

The disjoint variable restrictions of \texttt{ax-c16} can be seen from the
\texttt{show state\-ment} command.  The line that begins ``\texttt{Its mandatory
dis\-joint var\-i\-able pairs are:}\ldots'' lists any \texttt{\$d} variable
pairs in brackets.

\begin{verbatim}
MM> show statement ax-c16/full
Statement 3033 is located on line 9338 of the file "set.mm".
"Axiom of Distinct Variables. ..."
  ax-c16 $a |- ( A. x x = y -> ( ph -> A. x ph ) ) $.
Its mandatory hypotheses in RPN order are:
  wph $f wff ph $.
  vx $f setvar x $.
  vy $f setvar y $.
Its mandatory disjoint variable pairs are:  <x,y>
The statement and its hypotheses require the variables:  x y
      ph
The variables it contains are:  x y ph
\end{verbatim}

Since Metamath will always detect when \texttt{\$d}\index{\texttt{\$d} statement}
statements are needed for a proof, you don't have to worry too much about
forgetting to put one in; it can always be added if you see the error message
above.  If you put in unnecessary \texttt{\$d} statements, the worst that could
happen is that your theorem might not be as general as it could be, and this
may limit its use later on.

On the other hand, when you introduce axioms (\texttt{\$a}\index{\texttt{\$a}
statement} statements), you must be very careful to properly specify the
necessary associated \texttt{\$d} statements since Metamath has no way of knowing
whether your axioms are correct.  For example, Metamath would have no idea
that \texttt{ax-c16}, which we are telling it is an axiom of logic, would lead to
contradictions if we omitted its associated \texttt{\$d} statement.

% This was previously a comment in footnote-sized type, but it can be
% hard to read this much text in a small size.
% As a result, it's been changed to normally-sized text.
\label{nodd}
You may wonder if it is possible to develop standard
mathematics in the Metamath language without the \texttt{\$d}\index{\texttt{\$d}
statement} statement, since it seems like a nuisance that complicates proof
verification. The \texttt{\$d} statement is not needed in certain subsets of
mathematics such as propositional calculus.  However, dummy
variables\index{dummy variable!eliminating} and their associated \texttt{\$d}
statements are impossible to avoid in proofs in standard first-order logic as
well as in the variant used in \texttt{set.mm}.  In fact, there is no upper bound to
the number of dummy variables that might be needed in a proof of a theorem of
first-order logic containing 3 or more variables, as shown by H.\
Andr\'{e}ka\index{Andr{\'{e}}ka, H.} \cite{Nemeti}.  A first-order system that
avoids them entirely is given in \cite{Megill}\index{Megill, Norman}; the
trick there is simply to embed harmlessly the necessary dummy variables into a
theorem being proved so that they aren't ``dummy'' anymore, then interpret the
resulting longer theorem so as to ignore the embedded dummy variables.  If
this interests you, the system in \texttt{set.mm} obtained from \texttt{ax-1}
through \texttt{ax-c14} in \texttt{set.mm}, and deleting \texttt{ax-c16} and \texttt{ax-5},
requires no \texttt{\$d} statements but is logically complete in the sense
described in \cite{Megill}.  This means it can prove any theorem of
first-order logic as long as we add to the theorem an antecedent that embeds
dummy and any other variables that must be distinct.  In a similar fashion,
axioms for set theory can be devised that
do not require distinct variable
provisos\index{Set theory without distinct variable provisos},
as explained at
\url{http://us.metamath.org/mpeuni/mmzfcnd.html}.
Together, these in principle allow all of
mathematics to be developed under Metamath without a \texttt{\$d} statement,
although the length of the resulting theorems will grow as more and
more dummy variables become required in their proofs.

\subsection{The \texttt{\$f}
and \texttt{\$e} Statements}\label{dollaref}
\index{\texttt{\$e} statement}
\index{\texttt{\$f} statement}
\index{floating hypothesis}
\index{essential hypothesis}
\index{variable-type hypothesis}
\index{logical hypothesis}
\index{hypothesis}

Metamath has two kinds of hypo\-theses, the \texttt{\$f}\index{\texttt{\$f}
statement} or {\bf variable-type} hypothesis and the \texttt{\$e} or {\bf logical}
hypo\-the\-sis.\index{\texttt{\$d} statement}\footnote{Strictly speaking, the
\texttt{\$d} statement is also a hypothesis, but it is never directly referenced
in a proof, so we call it a restriction rather than a hypothesis to lessen
confusion.  The checking for violations of \texttt{\$d} restrictions is automatic
and built into Metamath's proof-checking algorithm.} The letters \texttt{f} and
\texttt{e} stand for ``floating''\index{floating hypothesis} (roughly meaning
used only if relevant) and ``essential''\index{essential hypothesis} (meaning
always used) respectively, for reasons that will become apparent
when we discuss frames in
Section~\ref{frames} and scoping in Section~\ref{scoping}. The syntax of these
are as follows:
\begin{center}
  {\em label} \texttt{\$f} {\em typecode} {\em variable} \texttt{\$.}\\
  {\em label} \texttt{\$e} {\em typecode}
      {\em math-symbol}\ \,$\cdots$\ {\em math-symbol} \texttt{\$.}\\
\end{center}
\index{\texttt{\$e} statement}
\index{\texttt{\$f} statement}
A hypothesis must have a {\em label}\index{label}.  The expression in a
\texttt{\$e} hypothesis consists of a typecode (an active constant math symbol)
followed by a sequence
of zero or more math symbols. Each math symbol (including {\em constant}
and {\em variable}) must be a previously declared constant or variable.  (In
addition, each math symbol must be active, which will be covered when we
discuss scoping statements in Section~\ref{scoping}.)  You use a \texttt{\$f}
hypothesis to specify the
nature or {\bf type}\index{variable type}\index{type} of a variable (such as ``let $x$ be an
integer'') and use a \texttt{\$e} hypothesis to express a logical truth (such as
``assume $x$ is prime'') that must be established in order for an assertion
requiring it to also be true.

A variable must have its type specified in a \texttt{\$f} statement before
it may be used in a \texttt{\$e}, \texttt{\$a}, or \texttt{\$p}
statement.  There may be only one (active) \texttt{\$f} statement for a
given variable.  (``Active'' is defined in Section~\ref{scoping}.)

In ordinary mathematics, theorems\index{theorem} are often expressed in the
form ``Assume $P$; then $Q$,'' where $Q$ is a statement that you can derive
if you start with statement $P$.\index{free variable}\footnote{A stronger
version of a theorem like this would be the {\em single} formula $P\rightarrow
Q$ ($P$ implies $Q$) from which the weaker version above follows by the rule
of modus ponens in logic.  We are not discussing this stronger form here.  In
the weaker form, we are saying only that if we can {\em prove} $P$, then we can
{\em prove} $Q$.  In a logician's language, if $x$ is the only free variable
in $P$ and $Q$, the stronger form is equivalent to $\forall x ( P \rightarrow
Q)$ (for all $x$, $P$ implies $Q$), whereas the weaker form is equivalent to
$\forall x P \rightarrow \forall x Q$. The stronger form implies the weaker,
but not vice-versa.  To be precise, the weaker form of the theorem is more
properly called an ``inference'' rather than a theorem.}\index{inference}
In the
Metamath\index{Metamath} language, you would express mathematical statement
$P$ as a hypothesis (a \texttt{\$e} Metamath language statement in this case) and
statement $Q$ as a provable assertion (a \texttt{\$p}\index{\texttt{\$p} statement}
statement).

Some examples of hypotheses you might encounter in logic and set theory are
\begin{center}
  \texttt{stmt1 \$f wff P \$.}\\
  \texttt{stmt2 \$f setvar x \$.}\\
  \texttt{stmt3 \$e |- ( P -> Q ) \$.}
\end{center}
\index{\texttt{\$e} statement}
\index{\texttt{\$f} statement}
Informally, these would be read, ``Let $P$ be a well-formed-formula,'' ``Let
$x$ be an (individual) variable,'' and ``Assume we have proved $P \rightarrow
Q$.''  The turnstile symbol \,$\vdash$\index{turnstile ({$\,\vdash$})} is
commonly used in logic texts to mean ``a proof exists for.''

To summarize:
\begin{itemize}
\item A \texttt{\$f} hypothesis tells Metamath the type or kind of its variable.
It is analogous to a variable declaration in a computer language that
tells the compiler that a variable is an integer or a floating-point
number.
\item The \texttt{\$e} hypothesis corresponds to what you would usually call a
``hypothesis'' in ordinary mathematics.
\end{itemize}

Before an assertion\index{assertion} (\texttt{\$a} or \texttt{\$p} statement) can be
referenced in a proof, all of its associated \texttt{\$f} and \texttt{\$e} hypotheses
(i.e.\ those \texttt{\$e} hypotheses that are active) must be satisfied (i.e.
established by the proof).  The meaning of ``associated'' (which we will call
{\bf mandatory} in Section~\ref{frames}) will become clear when we discuss
scoping later.

Note that after any \texttt{\$f}, \texttt{\$e},
\texttt{\$a}, or \texttt{\$p} token there is a required
\textit{typecode}\index{typecode}.
The typecode is a constant used to enforce types of expressions.
This will become clearer once we learn more about
assertions (\texttt{\$a} and \texttt{\$p} statements).
An example may also clarify their purpose.
In the
\texttt{set.mm}\index{set theory database (\texttt{set.mm})}%
\index{Metamath Proof Explorer}
database,
the following typecodes are used:

\begin{itemize}
\item \texttt{wff} :
  Well-formed formula (wff) symbol
  (read: ``the following symbol sequence is a wff'').
% The *textual* typecode for turnstile is "|-", but when read it's a little
% confusing, so I intentionally display the mathematical symbol here instead
% (I think it's clearer in this context).
\item \texttt{$\vdash$} :
  Turnstile (read: ``the following symbol sequence is provable'' or
  ``a proof exists for'').
\item \texttt{setvar} :
  Individual set variable type (read: ``the following is an
  individual set variable'').
  Note that this is \textit{not} the type of an arbitrary set expression,
  instead, it is used to ensure that there is only a single symbol used
  after quantifiers like for-all ($\forall$) and there-exists ($\exists$).
\item \texttt{class} :
  An expression that is a syntactically valid class expression.
  All valid set expressions are also valid class expression, so expressions
  of sets normally have the \texttt{class} typecode.
  Use the \texttt{class} typecode,
  \textit{not} the \texttt{setvar} typecode,
  for the type of set expressions unless you are specifically identifying
  a single set variable.
\end{itemize}

\subsection{Assertions (\texttt{\$a} and \texttt{\$p} Statements)}
\index{\texttt{\$a} statement}
\index{\texttt{\$p} statement}\index{assertion}\index{axiomatic assertion}
\index{provable assertion}

There are two types of assertions, \texttt{\$a}\index{\texttt{\$a} statement}
statements ({\bf axiomatic assertions}) and \texttt{\$p} statements ({\bf
provable assertions}).  Their syntax is as follows:
\begin{center}
  {\em label} \texttt{\$a} {\em typecode} {\em math-symbol} \ldots
         {\em math-symbol} \texttt{\$.}\\
  {\em label} \texttt{\$p} {\em typecode} {\em math-symbol} \ldots
        {\em math-symbol} \texttt{\$=} {\em proof} \texttt{\$.}
\end{center}
\index{\texttt{\$a} statement}
\index{\texttt{\$p} statement}
\index{\texttt{\$=} keyword}
An assertion always requires a {\em label}\index{label}. The expression in an
assertion consists of a typecode (an active constant)
followed by a sequence of zero
or more math symbols.  Each math symbol, including any {\em constant}, must be a
previously declared constant or variable.  (In addition, each math symbol
must be active, which will be covered when we discuss scoping statements in
Section~\ref{scoping}.)

A \texttt{\$a} statement is usually a definition of syntax (for example, if $P$
and $Q$ are wffs then so is $(P\to Q)$), an axiom\index{axiom} of ordinary
mathematics (for example, $x=x$), or a definition\index{definition} of
ordinary mathematics (for example, $x\ne y$ means $\lnot x=y$). A \texttt{\$p}
statement is a claim that a certain combination of math symbols follows from
previous assertions and is accompanied by a proof that demonstrates it.

Assertions can also be referenced in (later) proofs in order to derive new
assertions from them. The label of an assertion is used to refer to it in a
proof. Section~\ref{proof} will describe the proof in detail.

Assertions also provide the primary means for communicating the mathematical
results in the database to people.  Proofs (when conveniently displayed)
communicate to people how the results were arrived at.

\subsubsection{The \texttt{\$a} Statement}
\index{\texttt{\$a} statement}

Axiomatic assertions (\texttt{\$a} statements) represent the starting points from
which other assertions (\texttt{\$p}\index{\texttt{\$p} statement} statements) are
derived.  Their most obvious use is for specifying ordinary mathematical
axioms\index{axiom}, but they are also used for two other purposes.

First, Metamath\index{Metamath} needs to know the syntax of symbol
sequences that constitute valid mathematical statements.  A Metamath
proof must be broken down into much more detail than ordinary
mathematical proofs that you may be used to thinking of (even the
``complete'' proofs of formal logic\index{formal logic}).  This is one
of the things that makes Metamath a general-purpose language,
independent of any system of logic or even syntax.  If you want to use a
substitution instance of an assertion as a step in a proof, you must
first prove that the substitution is syntactically correct (or if you
prefer, you must ``construct'' it), showing for example that the
expression you are substituting for a wff metavariable is a valid wff.
The \texttt{\$a}\index{\texttt{\$a} statement} statement is used to
specify those combinations of symbols that are considered syntactically
valid, such as the legal forms of wffs.

Second, \texttt{\$a} statements are used to specify what are ordinarily thought of
as definitions, i.e.\ new combinations of symbols that abbreviate other
combinations of symbols.  Metamath makes no distinction\index{axiom vs.\
definition} between axioms\index{axiom} and definitions\index{definition}.
Indeed, it has been argued that such distinction should not be made even in
ordinary mathematics; see Section~\ref{definitions}, which discusses the
philosophy of definitions.  Section~\ref{hierarchy} discusses some
technical requirements for definitions.  In \texttt{set.mm} we adopt the
convention of prefixing axiom labels with \texttt{ax-} and definition labels with
\texttt{df-}\index{label}.

The results that can be derived with the Metamath language are only as good as
the \texttt{\$a}\index{\texttt{\$a} statement} statements used as their starting
point.  We cannot stress this too strongly.  For example, Metamath will
not prevent you from specifying $x\neq x$ as an axiom of logic.  It is
essential that you scrutinize all \texttt{\$a} statements with great care.
Because they are a source of potential pitfalls, it is best not to add new
ones (usually new definitions) casually; rather you should carefully evaluate
each one's necessity and advantages.

Once you have in place all of the basic axioms\index{axiom} and
rules\index{rule} of a mathematical theory, the only \texttt{\$a} statements that
you will be adding will be what are ordinarily called definitions.  In
principle, definitions should be in some sense eliminable from the language of
a theory according to some convention (usually involving logical equivalence
or equality).  The most common convention is that any formula that was
syntactically valid but not provable before the definition was introduced will
not become provable after the definition is introduced.  In an ideal world,
definitions should not be present at all if one is to have absolute confidence
in a mathematical result.  However, they are necessary to make
mathematics practical, for otherwise the resulting formulas would be
extremely long and incomprehensible.  Since the nature of definitions (in the
most general sense) does not permit them to automatically be verified as
``proper,''\index{proper definition}\index{definition!proper} the judgment of
the mathematician is required to ensure it.  (In \texttt{set.mm} effort was made
to make almost all definitions directly eliminable and thus minimize the need
for such judgment.)

If you are not a mathematician, it may be best not to add or change any
\texttt{\$a}\index{\texttt{\$a} statement} statements but instead use
the mathematical language already provided in standard databases.  This
way Metamath will not allow you to make a mistake (i.e.\ prove a false
result).


\subsection{Frames}\label{frames}

We now introduce the concept of a collection of related Metamath statements
called a frame.  Every assertion (\texttt{\$a} or \texttt{\$p} statement) in the database has
an associated frame.

A {\bf frame}\index{frame} is a sequence of \texttt{\$d}, \texttt{\$f},
and \texttt{\$e} statements (zero or more of each) followed by one
\texttt{\$a} or \texttt{\$p} statement, subject to certain conditions we
will describe.  For simplicity we will assume that all math symbol
tokens used are declared at the beginning of the database with
\texttt{\$c} and \texttt{\$v} statements (which are not properly part of
a frame).  Also for simplicity we will assume there are only simple
\texttt{\$d} statements (those with only two variables) and imagine any
compound \texttt{\$d} statements (those with more than two variables) as
broken up into simple ones.

A frame groups together those hypotheses (and \texttt{\$d} statements) relevant
to an assertion (\texttt{\$a} or \texttt{\$p} statement).  The statements in a frame
may or may not be physically adjacent in a database; we will cover
this in our discussion of scoping statements
in Section~\ref{scoping}.

A frame has the following properties:
\begin{enumerate}
 \item The set of variables contained in its \texttt{\$f} statements must
be identical to the set of variables contained in its \texttt{\$e},
\texttt{\$a}, and/or \texttt{\$p} statements.  In other words, each
variable in a \texttt{\$e}, \texttt{\$a}, or \texttt{\$p} statement must
have an associated ``variable type'' defined for it in a \texttt{\$f}
statement.
  \item No two \texttt{\$f} statements may contain the same variable.
  \item Any \texttt{\$f} statement
must occur before a \texttt{\$e} statement in which its variable occurs.
\end{enumerate}

The first property determines the set of variables occurring in a frame.
These are the {\bf mandatory
variables}\index{mandatory variable} of the frame.  The second property
tells us there must be only one type specified for a variable.
The last property is not a theoretical requirement but it
makes parsing of the database easier.

For our examples, we assume our database has the following declarations:

\begin{verbatim}
$v P Q R $.
$c -> ( ) |- wff $.
\end{verbatim}

The following sequence of statements, describing the modus ponens inference
rule, is an example of a frame:

\begin{verbatim}
wp  $f wff P $.
wq  $f wff Q $.
maj $e |- ( P -> Q ) $.
min $e |- P $.
mp  $a |- Q $.
\end{verbatim}

The following sequence of statements is not a frame because \texttt{R} does not
occur in the \texttt{\$e}'s or the \texttt{\$a}:

\begin{verbatim}
wp  $f wff P $.
wq  $f wff Q $.
wr  $f wff R $.
maj $e |- ( P -> Q ) $.
min $e |- P $.
mp  $a |- Q $.
\end{verbatim}

The following sequence of statements is not a frame because \texttt{Q} does not
occur in a \texttt{\$f}:

\begin{verbatim}
wp  $f wff P $.
maj $e |- ( P -> Q ) $.
min $e |- P $.
mp  $a |- Q $.
\end{verbatim}

The following sequence of statements is not a frame because the \texttt{\$a} statement is
not the last one:

\begin{verbatim}
wp  $f wff P $.
wq  $f wff Q $.
maj $e |- ( P -> Q ) $.
mp  $a |- Q $.
min $e |- P $.
\end{verbatim}

Associated with a frame is a sequence of {\bf mandatory
hypotheses}\index{mandatory hypothesis}.  This is simply the set of all
\texttt{\$f} and \texttt{\$e} statements in the frame, in the order they
appear.  A frame can be referenced in a later proof using the label of
the \texttt{\$a} or \texttt{\$p} assertion statement, and the proof
makes an assignment to each mandatory hypothesis in the order in which
it appears.  This means the order of the hypotheses, once chosen, must
not be changed so as not to affect later proofs referencing the frame's
assertion statement.  (The Metamath proof verifier will, of course, flag
an error if a proof becomes incorrect by doing this.)  Since proofs make
use of ``Reverse Polish notation,'' described in Section~\ref{proof}, we
call this order the {\bf RPN order}\index{RPN order} of the hypotheses.

Note that \texttt{\$d} statements are not part of the set of mandatory
hypotheses, and their order doesn't matter (as long as they satisfy the
fourth property for a frame described above).  The \texttt{\$d}
statements specify restrictions on variables that must be satisfied (and
are checked by the proof verifier) when expressions are substituted for
them in a proof, and the \texttt{\$d} statements themselves are never
referenced directly in a proof.

A frame with a \texttt{\$p} (provable) statement requires a proof as part of the
\texttt{\$p} statement.  Sometimes in a proof we want to make use of temporary or
dummy variables\index{dummy variable} that do not occur in the \texttt{\$p}
statement or its mandatory hypotheses.  To accommodate this we define an {\bf
extended frame}\index{extended frame} as a frame together with zero or more
\texttt{\$d} and \texttt{\$f} statements that reference variables not among the
mandatory variables of the frame.  Any new variables referenced are called the
{\bf optional variables}\index{optional variable} of the extended frame. If a
\texttt{\$f} statement references an optional variable it is called an {\bf
optional hypothesis}\index{optional hypothesis}, and if one or both of the
variables in a \texttt{\$d} statement are optional variables it is called an {\bf
optional disjoint-variable restriction}\index{optional disjoint-variable
restriction}.  Properties 2 and 3 for a frame also apply to an extended
frame.

The concept of optional variables is not meaningful for frames with \texttt{\$a}
statements, since those statements have no proofs that might make use of them.
There is no restriction on including optional hypotheses in the extended frame
for a \texttt{\$a} statement, but they serve no purpose.

The following set of statements is an example of an extended frame, which
contains an optional variable \texttt{R} and an optional hypothesis \texttt{wr}.  In
this example, we suppose the rule of modus ponens is not an axiom but is
derived as a theorem from earlier statements (we omit its presumed proof).
Variable \texttt{R} may be used in its proof if desired (although this would
probably have no advantage in propositional calculus).  Note that the sequence
of mandatory hypotheses in RPN order is still \texttt{wp}, \texttt{wq}, \texttt{maj},
\texttt{min} (i.e.\ \texttt{wr} is omitted), and this sequence is still assumed
whenever the assertion \texttt{mp} is referenced in a subsequent proof.

\begin{verbatim}
wp  $f wff P $.
wq  $f wff Q $.
wr  $f wff R $.
maj $e |- ( P -> Q ) $.
min $e |- P $.
mp  $p |- Q $= ... $.
\end{verbatim}

Every frame is an extended frame, but not every extended frame is a frame, as
this example shows.  The underlying frame for an extended frame is
obtained by simply removing all statements containing optional variables.
Any proof referencing an assertion will ignore any extensions to its
frame, which means we may add or delete optional hypotheses at will without
affecting subsequent proofs.

The conceptually simplest way of organizing a Metamath database is as a
sequence of extended frames.  The scoping statements
\texttt{\$\char`\{}\index{\texttt{\$\char`\{} and \texttt{\$\char`\}}
keywords} and \texttt{\$\char`\}} can be used to delimit the start and
end of an extended frame, leading to the following possible structure for a
database.  \label{framelist}

\vskip 2ex
\setbox\startprefix=\hbox{\tt \ \ \ \ \ \ \ \ }
\setbox\contprefix=\hbox{}
\startm
\m{\mbox{(\texttt{\$v} {\em and} \texttt{\$c}\,{\em statements})}}
\endm
\startm
\m{\mbox{\texttt{\$\char`\{}}}
\endm
\startm
\m{\mbox{\texttt{\ \ } {\em extended frame}}}
\endm
\startm
\m{\mbox{\texttt{\$\char`\}}}}
\endm
\startm
\m{\mbox{\texttt{\$\char`\{}}}
\endm
\startm
\m{\mbox{\texttt{\ \ } {\em extended frame}}}
\endm
\startm
\m{\mbox{\texttt{\$\char`\}}}}
\endm
\startm
\m{\mbox{\texttt{\ \ \ \ \ \ \ \ \ }}\vdots}
\endm
\vskip 2ex

In practice, this structure is inconvenient because we have to repeat
any \texttt{\$f}, \texttt{\$e}, and \texttt{\$d} statements over and
over again rather than stating them once for use by several assertions.
The scoping statements, which we will discuss next, allow this to be
done.  In principle, any Metamath database can be converted to the above
format, and the above format is the most convenient to use when studying
a Metamath database as a formal system%
%% Uncomment this when uncommenting section {formalspec} below
   (Appendix \ref{formalspec})%
.
In fact, Metamath internally converts the database to the above format.
The command \texttt{show statement} in the Metamath program will show
you the contents of the frame for any \texttt{\$a} or \texttt{\$p}
statement, as well as its extension in the case of a \texttt{\$p}
statement.

%c%(provided that all ``local'' variables and constants with limited scope have
%c%unique names),

During our discussion of scoping statements, it may be helpful to
think in terms of the equivalent sequence of frames that will result when
the database is parsed.  Scoping (other than the limited
use above to delimit frames) is not a theoretical requirement for
Metamath but makes it more convenient.


\subsection{Scoping Statements (\texttt{\$\{} and \texttt{\$\}})}\label{scoping}
\index{\texttt{\$\char`\{} and \texttt{\$\char`\}} keywords}\index{scoping statement}

%c%Some Metamath statements may be needed only temporarily to
%c%serve a specific purpose, and after we're done with them we would like to
%c%disregard or ignore them.  For example, when we're finished using a variable,
%c%we might want to
%c%we might want to free up the token\index{token} used to name it so that the
%c%token can be used for other purposes later on, such as a different kind of
%c%variable or even a constant.  In the terminology of computer programming, we
%c%might want to let some symbol declarations be ``local'' rather than ``global.''
%c%\index{local symbol}\index{global symbol}

The {\bf scoping} statements, \texttt{\$\char`\{} ({\bf start of block}) and \texttt{\$\char`\}}
({\bf end of block})\index{block}, provide a means for controlling the portion
of a database over which certain statement types are recognized.  The
syntax of a scoping statement is very simple; it just consists of the
statement's keyword:
\begin{center}
\texttt{\$\char`\{}\\
\texttt{\$\char`\}}
\end{center}
\index{\texttt{\$\char`\{} and \texttt{\$\char`\}} keywords}

For example, consider the following database where we have stripped out
all tokens except the scoping statement keywords.  For the purpose of the
discussion, we have added subscripts to the scoping statements; these subscripts
do not appear in the actual database.
\[
 \mbox{\tt \ \$\char`\{}_1
 \mbox{\tt \ \$\char`\{}_2
 \mbox{\tt \ \$\char`\}}_2
 \mbox{\tt \ \$\char`\{}_3
 \mbox{\tt \ \$\char`\{}_4
 \mbox{\tt \ \$\char`\}}_4
 \mbox{\tt \ \$\char`\}}_3
 \mbox{\tt \ \$\char`\}}_1
\]
Each \texttt{\$\char`\{} statement in this example is said to be {\bf
matched} with the \texttt{\$\char`\}} statement that has the same
subscript.  Each pair of matched scoping statements defines a region of
the database called a {\bf block}.\index{block} Blocks can be {\bf
nested}\index{nested block} inside other blocks; in the example, the
block defined by $\mbox{\tt \$\char`\{}_4$ and $\mbox{\tt \$\char`\}}_4$
is nested inside the block defined by $\mbox{\tt \$\char`\{}_3$ and
$\mbox{\tt \$\char`\}}_3$ as well as inside the block defined by
$\mbox{\tt \$\char`\{}_1$ and $\mbox{\tt \$\char`\}}_1$.  In general, a
block may be empty, it may contain only non-scoping
statements,\footnote{Those statements other than \texttt{\$\char`\{} and
\texttt{\$\char`\}}.}\index{non-scoping statement} or it may contain any
mixture of other blocks and non-scoping statements.  (This is called a
``recursive'' definition\index{recursive definition} of a block.)

Associated with each block is a number called its {\bf nesting
level}\index{nesting level} that indicates how deeply the block is nested.
The nesting levels of the blocks in our example are as follows:
\[
  \underbrace{
    \mbox{\tt \ }
    \underbrace{
     \mbox{\tt \$\char`\{\ }
     \underbrace{
       \mbox{\tt \$\char`\{\ }
       \mbox{\tt \$\char`\}}
     }_{2}
     \mbox{\tt \ }
     \underbrace{
       \mbox{\tt \$\char`\{\ }
       \underbrace{
         \mbox{\tt \$\char`\{\ }
         \mbox{\tt \$\char`\}}
       }_{3}
       \mbox{\tt \ \$\char`\}}
     }_{2}
     \mbox{\tt \ \$\char`\}}
   }_{1}
   \mbox{\tt \ }
 }_{0}
\]
\index{\texttt{\$\char`\{} and \texttt{\$\char`\}} keywords}
The entire database is considered to be one big block (the {\bf outermost}
block) with a nesting level of 0.  The outermost block is {\em not} bracketed
by scoping statements.\footnote{The language was designed this way so that
several source files can be joined together more easily.}\index{outermost
block}

All non-scoping Metamath statements become recognized or {\bf
active}\index{active statement} at the place where they appear.\footnote{To
keep things slightly simpler, we do not bother to define the concept of
``active'' for the scoping statements.}  Certain of these statement types
become inactive at the end of the block in which they appear; these statement
types are:
\begin{center}
  \texttt{\$c}, \texttt{\$v}, \texttt{\$d}, \texttt{\$e}, and \texttt{\$f}.
%  \texttt{\$v}, \texttt{\$f}, \texttt{\$e}, and \texttt{\$d}.
\end{center}
\index{\texttt{\$c} statement}
\index{\texttt{\$d} statement}
\index{\texttt{\$e} statement}
\index{\texttt{\$f} statement}
\index{\texttt{\$v} statement}
The other statement types remain active forever (i.e.\ through the end of the
database); they are:
\begin{center}
  \texttt{\$a} and \texttt{\$p}.
%  \texttt{\$c}, \texttt{\$a}, and \texttt{\$p}.
\end{center}
\index{\texttt{\$a} statement}
\index{\texttt{\$p} statement}
Any statement (of these 7 types) located in the outermost
block\index{outermost block} will remain active through the end of the
database and thus are effectively ``global'' statements.\index{global
statement}

All \texttt{\$c} statements must be placed in the outermost block.  Since they are
therefore always global, they could be considered as belonging to both of the
above categories.

The {\bf scope}\index{scope} of a statement is the set of statements that
recognize it as active.

%c%The concept of ``active'' is also defined for math symbols\index{math
%c%symbol}.  Math symbols (constants\index{constant} and
%c%variables\index{variable}) become {\bf active}\index{active
%c%math symbol} in the \texttt{\$c}\index{\texttt{\$c}
%c%statement} and \texttt{\$v}\index{\texttt{\$v} statement} statements that
%c%declare them.  They become inactive when their declaration statements become
%c%inactive.

The concept of ``active'' is also defined for math symbols\index{math
symbol}.  Math symbols (constants\index{constant} and
variables\index{variable}) become {\bf active}\index{active math symbol}
in the \texttt{\$c}\index{\texttt{\$c} statement} and
\texttt{\$v}\index{\texttt{\$v} statement} statements that declare them.
A variable becomes inactive when its declaration statement becomes
inactive.  Because all \texttt{\$c} statements must be in the outermost
block, a constant will never become inactive after it is declared.

\subsubsection{Redeclaration of Math Symbols}
\index{redeclaration of symbols}\label{redeclaration}

%c%A math symbol may not be declared a second time while it is active, but it may
%c%be declared again after it becomes inactive.

A variable may not be declared a second time while it is active, but it may be
declared again after it becomes inactive.  This provides a convenient way to
introduce ``local'' variables,\index{local variable} i.e.\ temporary variables
for use in the frame of an assertion or in a proof without keeping them around
forever.  A previously declared variable may not be redeclared as a constant.

A constant may not be redeclared.  And, as mentioned above, constants must be
declared in the outermost block.

The reason variables may have limited scope but not constants is that an
assertion (\texttt{\$a} or \texttt{\$p} statement) remains available for use in
proofs through the end of the database.  Variables in an assertion's frame may
be substituted with whatever is needed in a proof step that references the
assertion, whereas constants remain fixed and may not be substituted with
anything.  The particular token used for a variable in an assertion's frame is
irrelevant when the assertion is referenced in a proof, and it doesn't matter
if that token is not available outside of the referenced assertion's frame.
Constants, however, must be globally fixed.

There is no theoretical
benefit for the feature allowing variables to be active for limited scopes
rather than global. It is just a convenience that allows them, for example, to
be locally grouped together with their corresponding \texttt{\$f} variable-type
declarations.

%c%If you declare a math symbol more than once, internally Metamath considers it a
%c%new distinct symbol, even though it has the same name.  If you are unaware of
%c%this, you may find that what you think are correct proofs are incorrectly
%c%rejected as invalid, because Metamath may tell you that a constant you
%c%previously declared does not match a newly declared math symbol with the same
%c%name.  For details on this subtle point, see the Comment on
%c%p.~\pageref{spec4comment}.  This is done purposely to allow temporary
%c%constants to be introduced while developing a subtheory, then allow their math
%c%symbol tokens to be reused later on; in general they will not refer to the
%c%same thing.  In practice, you would not ordinarily reuse the names of
%c%constants because it would tend to be confusing to the reader.  The reuse of
%c%names of variables, on the other hand, is something that is often useful to do
%c%(for example it is done frequently in \texttt{set.mm}).  Since variables in an
%c%assertion referenced in a proof can be substituted as needed to achieve a
%c%symbol match, this is not an issue.

% (This section covers a somewhat advanced topic you may want to skip
% at first reading.)
%
% Under certain circumstances, math symbol\index{math symbol}
% tokens\index{token} may be redeclared (i.e.\ the token
% may appear in more than
% one \texttt{\$c}\index{\texttt{\$c} statement} or \texttt{\$v}\index{\texttt{\$v}
% statement} statement).  You might want to do this say, to make temporary use
% of a variable name without having to worry about its affect elsewhere,
% somewhat analogous to declaring a local variable in a standard computer
% language.  Understanding what goes on when math symbol tokens are redeclared
% is a little tricky to understand at first, since it requires that we
% distinguish the token itself from the math symbol that it names.  It will help
% if we first take a peek at the internal workings of the
% Metamath\index{Metamath} program.
%
% Metamath reserves a memory location for each occurrence of a
% token\index{token} in a declaration statement (\texttt{\$c}\index{\texttt{\$c}
% statement} or \texttt{\$v}\index{\texttt{\$v} statement}).  If a given token appears
% in more than one declaration statement, it will refer to more than one memory
% locations.  A math symbol\index{math symbol} may be thought of as being one of
% these memory locations rather than as the token itself.  Only one of the
% memory locations associated with a given token may be active at any one time.
% The math symbol (memory location) that gets looked up when the token appears
% in a non-declaration statement is the one that happens to be active at that
% time.
%
% We now look at the rules for the redeclaration\index{redeclaration of symbols}
% of math symbol tokens.
% \begin{itemize}
% \item A math symbol token may not be declared twice in the
% same block.\footnote{While there is no theoretical reason for disallowing
% this, it was decided in the design of Metamath that allowing it would offer no
% advantage and might cause confusion.}
% \item An inactive math symbol may always be
% redeclared.
% \item  An active math symbol may be redeclared in a different (i.e.\
% inner) block\index{block} from the one it became active in.
% \end{itemize}
%
% When a math symbol token is redeclared, it conceptually refers to a different
% math symbol, just as it would be if it were called a different name.  In
% addition, the original math symbol that it referred to, if it was active,
% temporarily becomes inactive.  At the end of the block in which the
% redeclaration occurred, the new math symbol\index{math symbol} becomes
% inactive and the original symbol becomes active again.  This concept is
% illustrated in the following example, where the symbol \texttt{e} is
% ordinarily a constant (say Euler's constant, 2.71828...) but
% temporarily we want to use it as a ``local'' variable, say as a coefficient
% in the equation $a x^4 + b x^3 + c x^2 + d x + e$:
% \[
%   \mbox{\tt \$\char`\{\ \$c e \$.}
%   \underbrace{
%     \ \ldots\ %
%     \mbox{\tt \$\char`\{}\ \ldots\ %
%   }_{\mbox{\rm region A}}
%   \mbox{\tt \$v e \$.}
%   \underbrace{
%     \mbox{\ \ \ \ldots\ \ \ }
%   }_{\mbox{\rm region B}}
%   \mbox{\tt \$\char`\}}
%   \underbrace{
%     \mbox{\ \ \ \ldots\ \ \ }
%   }_{\mbox{\rm region C}}
%   \mbox{\tt \$\char`\}}
% \]
% \index{\texttt{\$\char`\{} and \texttt{\$\char`\}} keywords}
% In region A, the token \texttt{e} refers to a constant.  It is redeclared as a
% variable in region B, and any reference to it in this region will refer to this
% variable.  In region C, the redeclaration becomes inactive, and the original
% declaration becomes active again.  In region C, the token \texttt{x} refers to the
% original constant.
%
% As a practical matter, overuse of math symbol\index{math symbol}
% redeclarations\index{redeclaration of symbols} can be confusing (even though
% it is well-defined) and is best avoided when possible.  Here are some good
% general guidelines you can follow.  Usually, you should declare all
% constants\index{constant} in the outermost block\index{outermost block},
% especially if they are general-purpose (such as the token \verb$A.$, meaning
% $\forall$ or ``for all'').  This will make them ``globally'' active (although
% as in the example above local redeclarations will temporarily make them
% inactive.)  Most or all variables\index{variable}, on the other hand, could be
% declared in inner blocks, so that the token for them can be used later for a
% different type of variable or a constant.  (The names of the variables you
% choose are not used when you refer to an assertion\index{assertion} in a
% proof, whereas constants must match exactly.  A locally declared constant will
% not match a globally declared constant in a proof, even if they use the same
% token, because Metamath internally considers them to be different math
% symbols.)  To avoid confusion, you should generally avoid redeclaring active
% variables.  If you must redeclare them, do so at the beginning of a block.
% The temporary declaration of constants in inner blocks might be occasionally
% appropriate when you make use of a temporary definition to prove lemmas
% leading to a main result that does not make direct use of the definition.
% This way, you will not clutter up your database with a large number of
% seldom-used global constant symbols.  You might want to note that while
% inactive constants may not appear directly in an assertion (a \texttt{\$a}\index{\texttt{\$a}
% statement} or \texttt{\$p}\index{\texttt{\$p} statement}
% statement), they may be indirectly used in the proof of a \texttt{\$p} statement
% so long as they do not appear in the final math symbol sequence constructed by
% the proof.  In the end, you will have to use your best judgment, taking into
% account standard mathematical usage of the symbols as well as consideration
% for the reader of your work.
%
% \subsubsection{Reuse of Labels}\index{reuse of labels}\index{label}
%
% The \texttt{\$e}\index{\texttt{\$e} statement}, \texttt{\$f}\index{\texttt{\$f}
% statement}, \texttt{\$a}\index{\texttt{\$a} statement}, and
% \texttt{\$p}\index{\texttt{\$p}
% statement} statement types require labels, which allow them to be
% referenced later inside proofs.  A label is considered {\bf
% active}\index{active label} when the statement it is associated with is
% active.  The token\index{token} for a label may be reused
% (redeclared)\index{redeclaration of labels} provided that it is not being used
% for a currently active label.  (Unlike the tokens for math symbols, active
% label tokens may not be redeclared in an inner scope.)  Note that the labels
% of \texttt{\$a} and \texttt{\$p} statements can never be reused after these
% statements appear, because these statements remain active through the end of
% the database.
%
% You might find the reuse of labels a convenient way to have standard names for
% temporary hypotheses, such as \texttt{h1}, \texttt{h2}, etc.  This way you don't have
% to invent unique names for each of them, and in some cases it may be less
% confusing to the reader (although in other cases it might be more confusing, if
% the hypothesis is located far away from the assertion that uses
% it).\footnote{The current implementation requires that all labels, even
% inactive ones, be unique.}

\subsubsection{Frames Revisited}\index{frames and scoping statements}

Now that we have covered scoping, we will look at how an arbitrary
Metamath database can be converted to the simple sequence of extended
frames described on p.~\pageref{framelist}.  This is also how Metamath
stores the database internally when it reads in the database
source.\label{frameconvert} The method is simple.  First, we collect all
constant and variable (\texttt{\$c} and \texttt{\$v}) declarations in
the database, ignoring duplicate declarations of the same variable in
different scopes.  We then put our collected \texttt{\$c} and
\texttt{\$v} declarations at the beginning of the database, so that
their scope is the entire database.  Next, for each assertion in the
database, we determine its frame and extended frame.  The extended frame
is simply the \texttt{\$f}, \texttt{\$e}, and \texttt{\$d} statements
that are active.  The frame is the extended frame with all optional
hypotheses removed.

An equivalent way of saying this is that the extended frame of an assertion
is the collection of all \texttt{\$f}, \texttt{\$e}, and \texttt{\$d} statements
whose scope includes the assertion.
The \texttt{\$f} and \texttt{\$e} statements
occur in the order they appear
(order is irrelevant for \texttt{\$d} statements).

%c%, renaming any
%c%redeclared variables as needed so that all of them have unique names.  (The
%c%exact renaming convention is unimportant.  You might imagine renaming
%c%different declarations of math symbol \texttt{a} as \texttt{a\$1}, \texttt{a\$2}, etc.\
%c%which would prevent any conflicts since \texttt{\$} is not a legal character in a
%c%math symbol token.)

\section{The Anatomy of a Proof} \label{proof}
\index{proof!Metamath, description of}

Each provable assertion (\texttt{\$p}\index{\texttt{\$p} statement} statement) in a
database must include a {\bf proof}\index{proof}.  The proof is located
between the \texttt{\$=}\index{\texttt{\$=} keyword} and \texttt{\$.}\ keywords in the
\texttt{\$p} statement.

In the basic Metamath language\index{basic language}, a proof is a
sequence of statement labels.  This label sequence\index{label sequence}
serves as a set of instructions that the Metamath program uses to
construct a series of math symbol sequences.  The construction must
ultimately result in the math symbol sequence contained between the
\texttt{\$p}\index{\texttt{\$p} statement} and
\texttt{\$=}\index{\texttt{\$=} keyword} keywords of the \texttt{\$p}
statement.  Otherwise, the Metamath program will consider the proof
incorrect, and it will notify you with an appropriate error message when
you ask it to verify the proof.\footnote{To make the loading faster, the
Metamath program does not automatically verify proofs when you
\texttt{read} in a database unless you use the \texttt{/verify}
qualifier.  After a database has been read in, you may use the
\texttt{verify proof *} command to verify proofs.}\index{\texttt{verify
proof} command} Each label in a proof is said to {\bf
reference}\index{label reference} its corresponding statement.

Associated with any assertion\index{assertion} (\texttt{\$p} or
\texttt{\$a}\index{\texttt{\$a} statement} statement) is a set of
hypotheses (\texttt{\$f}\index{\texttt{\$f} statement} or
\texttt{\$e}\index{\texttt{\$e} statement} statements) that are active
with respect to that assertion.  Some are mandatory and the others are
optional.  You should review these concepts if necessary.

Each label\index{label} in a proof must be either the label of a
previous assertion (\texttt{\$a}\index{\texttt{\$a} statement} or
\texttt{\$p}\index{\texttt{\$p} statement} statement) or the label of an
active hypothesis (\texttt{\$e} or \texttt{\$f}\index{\texttt{\$f}
statement} statement) of the \texttt{\$p} statement containing the
proof.  Hypothesis labels may reference both the
mandatory\index{mandatory hypothesis} and the optional hypotheses of the
\texttt{\$p} statement.

The label sequence in a proof specifies a construction in {\bf reverse Polish
notation}\index{reverse Polish notation (RPN)} (RPN).  You may be familiar
with RPN if you have used older
Hewlett--Packard or similar hand-held calculators.
In the calculator analogy, a hypothesis label\index{hypothesis label} is like
a number and an assertion label\index{assertion label} is like an operation
(more precisely, an $n$-ary operation when the
assertion has $n$ \texttt{\$e}-hypotheses).
On an RPN calculator, an operation takes one or more previous numbers in an
input sequence, performs a calculation on them, and replaces those numbers and
itself with the result of the calculation.  For example, the input sequence
$2,3,+$ on an RPN calculator results in $5$, and the input sequence
$2,3,5,{\times},+$ results in $2,15,+$ which results in $17$.

Understanding how RPN is processed involves the concept of a {\bf
stack}\index{stack}\index{RPN stack}, which can be thought of as a set of
temporary memory locations that hold intermediate results.  When Metamath
encounters a hypothesis label it places or {\bf pushes}\index{push} the math
symbol sequence of the hypothesis onto the stack.  When Metamath encounters an
assertion label, it associates the most recent stack entries with the {\em
mandatory} hypotheses\index{mandatory hypothesis} of the assertion, in the
order where the most recent stack entry is associated with the last mandatory
hypothesis of the assertion.  It then determines what
substitutions\index{substitution!variable}\index{variable substitution} have
to be made into the variables of the assertion's mandatory hypotheses to make
them identical to the associated stack entries.  It then makes those same
substitutions into the assertion itself.  Finally, Metamath removes or {\bf
pops}\index{pop} the matched hypotheses from the stack and pushes the
substituted assertion onto the stack.

For the purpose of matching the mandatory hypothesis to the most recent stack
entries, whether a hypothesis is a \texttt{\$e} or \texttt{\$f} statement is
irrelevant.  The only important thing is that a set of
substitutions\footnote{In the Metamath spec (Section~\ref{spec}), we use the
singular term ``substitution'' to refer to the set of substitutions we talk
about here.} exist that allow a match (and if they don't, the proof verifier
will let you know with an error message).  The Metamath language is specified
in such a way that if a set of substitutions exists, it will be unique.
Specifically, the requirement that each variable have a type specified for it
with a \texttt{\$f} statement ensures the uniqueness.

We will illustrate this with an example.
Consider the following Metamath source file:
\begin{verbatim}
$c ( ) -> wff $.
$v p q r s $.
wp $f wff p $.
wq $f wff q $.
wr $f wff r $.
ws $f wff s $.
w2 $a wff ( p -> q ) $.
wnew $p wff ( s -> ( r -> p ) ) $= ws wr wp w2 w2 $.
\end{verbatim}
This Metamath source example shows the definition and ``proof'' (i.e.,
construction) of a well-formed formula (wff)\index{well-formed formula (wff)}
in propositional calculus.  (You may wish to type this example into a file to
experiment with the Metamath program.)  The first two statements declare
(introduce the names of) four constants and four variables.  The next four
statements specify the variable types, namely that
each variable is assumed to be a wff.  Statement \texttt{w2} defines (postulates)
a way to produce a new wff, \texttt{( p -> q )}, from two given wffs \texttt{p} and
\texttt{q}. The mandatory hypotheses of \texttt{w2} are \texttt{wp} and \texttt{wq}.
Statement \texttt{wnew} claims that \texttt{( s -> ( r -> p ) )} is a wff given
three wffs \texttt{s}, \texttt{r}, and \texttt{p}.  More precisely, \texttt{wnew} claims
that the sequence of ten symbols \texttt{wff ( s -> ( r -> p ) )} is provable from
previous assertions and the hypotheses of \texttt{wnew}.  Metamath does not know
or care what a wff is, and as far as it is concerned
the typecode \texttt{wff} is just an
arbitrary constant symbol in a math symbol sequence.  The mandatory hypotheses
of \texttt{wnew} are \texttt{wp}, \texttt{wr}, and \texttt{ws}; \texttt{wq} is an optional
hypothesis.  In our particular proof, the optional hypothesis is not
referenced, but in general, any combination of active (i.e.\ optional and
mandatory) hypotheses could be referenced.  The proof of statement \texttt{wnew}
is the sequence of five labels starting with \texttt{ws} (step~1) and ending with
\texttt{w2} (step~5).

When Metamath verifies the proof, it scans the proof from left to right.  We
will examine what happens at each step of the proof.  The stack starts off
empty.  At step 1, Metamath looks up label \texttt{ws} and determines that it is a
hypothesis, so it pushes the symbol sequence of statement \texttt{ws} onto the
stack:

\begin{center}\begin{tabular}{|l|l|}\hline
{Stack location} & {Contents} \\ \hline \hline
1 & \texttt{wff s} \\ \hline
\end{tabular}\end{center}

Metamath sees that the labels \texttt{wr} and \texttt{wp} in steps~2 and 3 are also
hypotheses, so it pushes them onto the stack.  After step~3, the stack looks
like
this:

\begin{center}\begin{tabular}{|l|l|}\hline
{Stack location} & {Contents} \\ \hline \hline
3 & \texttt{wff p} \\ \hline
2 & \texttt{wff r} \\ \hline
1 & \texttt{wff s} \\ \hline
\end{tabular}\end{center}

At step 4, Metamath sees that label \texttt{w2} is an assertion, so it must do
some processing.  First, it associates the mandatory hypotheses of \texttt{w2},
which are \texttt{wp} and \texttt{wq}, with stack locations~2 and 3, {\em in that
order}. Metamath determines that the only possible way
to make hypothesis \texttt{wp} match (become identical to) stack location~2 and
\texttt{wq} match stack location 3 is to substitute variable \texttt{p} with \texttt{r}
and \texttt{q} with \texttt{p}.  Metamath makes these substitutions into \texttt{w2} and
obtains the symbol sequence \texttt{wff ( r -> p )}.  It removes the hypotheses
from stack locations~2 and 3, then places the result into stack location~2:

\begin{center}\begin{tabular}{|l|l|}\hline
{Stack location} & {Contents} \\ \hline \hline
2 & \texttt{wff ( r -> p )} \\ \hline
1 & \texttt{wff s} \\ \hline
\end{tabular}\end{center}

At step 5, Metamath sees that label \texttt{w2} is an assertion, so it must again
do some processing.  First, it matches the mandatory hypotheses of \texttt{w2},
which are \texttt{wp} and \texttt{wq}, to stack locations 1 and 2.
Metamath determines that the only possible way to make the
hypotheses match is to substitute variable \texttt{p} with \texttt{s} and \texttt{q} with
\texttt{( r -> p )}.  Metamath makes these substitutions into \texttt{w2} and obtains
the symbol
sequence \texttt{wff ( s -> ( r -> p ) )}.  It removes stack
locations 1 and 2, then places the result into stack location~1:

\begin{center}\begin{tabular}{|l|l|}\hline
{Stack location} & {Contents} \\ \hline \hline
1 & \texttt{wff ( s -> ( r -> p ) )} \\ \hline
\end{tabular}\end{center}

After Metamath finishes processing the proof, it checks to see that the
stack contains exactly one element and that this element is
the same as the math symbol sequence in the
\texttt{\$p}\index{\texttt{\$p} statement} statement.  This is the case for our
proof of \texttt{wnew},
so we have proved \texttt{wnew} successfully.  If the result
differs, Metamath will notify you with an error message.  An error message
will also result if the stack contains more than one entry at the end of the
proof, or if the stack did not contain enough entries at any point in the
proof to match all of the mandatory hypotheses\index{mandatory hypothesis} of
an assertion.  Finally, Metamath will notify you with an error message if no
substitution is possible that will make a referenced assertion's hypothesis
match the
stack entries.  You may want to experiment with the different kinds of errors
that Metamath will detect by making some small changes in the proof of our
example.

Metamath's proof notation was designed primarily to express proofs in a
relatively compact manner, not for readability by humans.  Metamath can display
proofs in a number of different ways with the \texttt{show proof}\index{\texttt{show
proof} command} command.  The
\texttt{/lemmon} qualifier displays it in a format that is easier to read when the
proofs are short, and you saw examples of its use in Chapter~\ref{using}.  For
longer proofs, it is useful to see the tree structure of the proof.  A tree
structure is displayed when the \texttt{/lemmon} qualifier is omitted.  You will
probably find this display more convenient as you get used to it. The tree
display of the proof in our example looks like
this:\label{treeproof}\index{tree-style proof}\index{proof!tree-style}
\begin{verbatim}
1     wp=ws    $f wff s
2        wp=wr    $f wff r
3        wq=wp    $f wff p
4     wq=w2    $a wff ( r -> p )
5  wnew=w2  $a wff ( s -> ( r -> p ) )
\end{verbatim}
The number to the left of each line is the step number.  Following it is a
{\bf hypothesis association}\index{hypothesis association}, consisting of two
labels\index{label} separated by \texttt{=}.  To the left of the \texttt{=} (except
in the last step) is the label of a hypothesis of an assertion referenced
later in the proof; here, steps 1 and 4 are the hypothesis associations for
the assertion \texttt{w2} that is referenced in step 5.  A hypothesis association
is indented one level more than the assertion that uses it, so it is easy to
find the corresponding assertion by moving directly down until the indentation
level decreases to one less than where you started from.  To the right of each
\texttt{=} is the proof step label for that proof step.  The statement keyword of
the proof step label is listed next, followed by the content of the top of the
stack (the most recent stack entry) as it exists after that proof step is
processed.  With a little practice, you should have no trouble reading proofs
displayed in this format.

Metamath proofs include the syntax construction of a formula.
In standard mathematics, this kind of
construction is not considered a proper part of the proof at all, and it
certainly becomes rather boring after a while.
Therefore,
by default the \texttt{show proof}\index{\texttt{show proof}
command} command does not show the syntax construction.
Historically \texttt{show proof} command
\textit{did} show the syntax construction, and you needed to add the
\texttt{/essential} option to hide, them, but today
\texttt{/essential} is the default and you need to use
\texttt{/all} to see the syntax constructions.

When verifying a proof, Metamath will check that no mandatory
\texttt{\$d}\index{\texttt{\$d} statement}\index{mandatory \texttt{\$d}
statement} statement of an assertion referenced in a proof is violated
when substitutions\index{substitution!variable}\index{variable
substitution} are made to the variables in the assertion.  For details
see Section~\ref{spec4} or \ref{dollard}.

\subsection{The Concept of Unification} \label{unify}

During the course of verifying a proof, when Metamath\index{Metamath}
encounters an assertion label\index{assertion label}, it associates the
mandatory hypotheses\index{mandatory hypothesis} of the assertion with the top
entries of the RPN stack\index{stack}\index{RPN stack}.  Metamath then
determines what substitutions\index{substitution!variable}\index{variable
substitution} it must make to the variables in the assertion's mandatory
hypotheses in order for these hypotheses to become identical to their
corresponding stack entries.  This process is called {\bf
unification}\index{unification}.  (We also informally use the term
``unification'' to refer to a set of substitutions that results from the
process, as in ``two unifications are possible.'')  After the substitutions
are made, the hypotheses are said to be {\bf unified}.

If no such substitutions are possible, Metamath will consider the proof
incorrect and notify you with an error message.
% (deleted 3/10/07, per suggestion of Mel O'Cat:)
% The syntax of the
% Metamath language ensures that if a set of substitutions exists, it
% will be unique.

The general algorithm for unification described in the literature is
somewhat complex.
However, in the case of Metamath it is intentionally trivial.
Mandatory hypotheses must be
pushed on the proof stack in the order in which they appear.
In addition, each variable must have its type specified
with a \texttt{\$f} hypothesis before it is used
and that each \texttt{\$f} hypothesis
have the restricted syntax of a typecode (a constant) followed by a variable.
The typecode in the \texttt{\$f} hypothesis must match the first symbol of
the corresponding RPN stack entry (which will also be a constant), so
the only possible match for the variable in the \texttt{\$f} hypothesis is
the sequence of symbols in the stack entry after the initial constant.

In the Proof Assistant\index{Proof Assistant}, a more general unification
algorithm is used.  While a proof is being developed, sometimes not enough
information is available to determine a unique unification.  In this case
Metamath will ask you to pick the correct one.\index{ambiguous
unification}\index{unification!ambiguous}

\section{Extensions to the Metamath Language}\index{extended
language}

\subsection{Comments in the Metamath Language}\label{comments}
\index{markup notation}
\index{comments!markup notation}

The commenting feature allows you to annotate the contents of
a database.  Just as with most
computer languages, comments are ignored for the purpose of interpreting the
contents of the database. Comments effectively act as
additional white space\index{white
space} between tokens
when a database is parsed.

A comment may be placed at the beginning, end, or
between any two tokens\index{token} in a source file.

Comments have the following syntax:
\begin{center}
 \texttt{\$(} {\em text} \texttt{\$)}
\end{center}
Here,\index{\texttt{\$(} and \texttt{\$)} auxiliary
keywords}\index{comment} {\em text} is a string, possibly empty, of any
characters in Metamath's character set (p.~\pageref{spec1chars}), except
that the character strings \texttt{\$(} and \texttt{\$)} may not appear
in {\em text}.  Thus nested comments are not
permitted:\footnote{Computer languages have differing standards for
nested comments, and rather than picking one it was felt simplest not to
allow them at all, at least in the current version (0.177) of
Metamath\index{Metamath!limitations of version 0.177}.} Metamath will
complain if you give it
\begin{center}
 \texttt{\$( This is a \$( nested \$) comment.\ \$)}
\end{center}
To compensate for this non-nesting behavior, I often change all \texttt{\$}'s
to \texttt{@}'s in sections of Metamath code I wish to comment out.

The Metamath program supports a number of markup mechanisms and conventions
to generate good-looking results in \LaTeX\ and {\sc html},
as discussed below.
These markup features have to do only with how the comments are typeset,
and have no effect on how Metamath verifies the proofs in the database.
The improper
use of them may result in incorrectly typeset output, but no Metamath
error messages will result during the \texttt{read} and \texttt{verify
proof} commands.  (However, the \texttt{write
theorem\texttt{\char`\_}list} command
will check for markup errors as a side-effect of its
{\sc html} generation.)
Section~\ref{texout} has instructions for creating \LaTeX\ output, and
section~\ref{htmlout} has instructions for creating
{\sc html}\index{HTML} output.

\subsubsection{Headings}\label{commentheadings}

If the \texttt{\$(} is immediately followed by a new line
starting with a heading marker, it is a header.
This can start with:

\begin{itemize}
 \item[] \texttt{\#\#\#\#} - major part header
 \item[] \texttt{\#*\#*} - section header
 \item[] \texttt{=-=-} - subsection header
 \item[] \texttt{-.-.} - subsubsection header
\end{itemize}

The line following the marker line
will be used for the table of contents entry, after trimming spaces.
The next line should be another (closing) matching marker line.
Any text after that
but before the closing \texttt{\$}, such as an extended description of the
section, will be included on the \texttt{mmtheoremsNNN.html} page.

For more information, run
\texttt{help write theorem\char`\_list}.

\subsubsection{Math mode}
\label{mathcomments}
\index{\texttt{`} inside comments}
\index{\texttt{\char`\~} inside comments}
\index{math mode}

Inside of comments, a string of tokens\index{token} enclosed in
grave accents\index{grave accent (\texttt{`})} (\texttt{`}) will be converted
to standard mathematical symbols during
{\sc HTML}\index{HTML} or \LaTeX\ output
typesetting,\index{latex@{\LaTeX}} according to the information in the
special \texttt{\$t}\index{\texttt{\$t} comment}\index{typesetting
comment} comment in the database
(see section~\ref{tcomment} for information about the typesetting
comment, and Appendix~\ref{ASCII} to see examples of its results).

The first grave accent\index{grave accent (\texttt{`})} \texttt{`}
causes the output processor to enter {\bf math mode}\index{math mode}
and the second one exits it.
In this
mode, the characters following the \texttt{`} are interpreted as a
sequence of math symbol tokens separated by white space\index{white
space}.  The tokens are looked up in the \texttt{\$t}
comment\index{\texttt{\$t} comment}\index{typesetting comment} and if
found, they will be replaced by the standard mathematical symbols that
they correspond to before being placed in the typeset output file.  If
not found, the symbol will be output as is and a warning will be issued.
The tokens do not have to be active in the database, although a warning
will be issued if they are not declared with \texttt{\$c} or
\texttt{\$v} statements.

Two consecutive
grave accents \texttt{``} are treated as a single actual grave accent
(both inside and outside of math mode) and will not cause the output
processor to enter or exit math mode.

Here is an example of its use\index{Pierce's axiom}:
\begin{center}
\texttt{\$( Pierce's axiom, ` ( ( ph -> ps ) -> ph ) -> ph ` ,\\
         is not very intuitive. \$)}
\end{center}
becomes
\begin{center}
   \texttt{\$(} Pierce's axiom, $((\varphi \rightarrow \psi)\rightarrow
\varphi)\rightarrow \varphi$, is not very intuitive. \texttt{\$)}
\end{center}

Note that the math symbol tokens\index{token} must be surrounded by white
space\index{white space}.
%, since there is no context that allows ambiguity to be
%resolved, as is the case with math symbol sequences in some of the Metamath
%statements.
White space should also surround the \texttt{`}
delimiters.

The math mode feature also gives you a quick and easy way to generate
text containing mathematical symbols, independently of the intended
purpose of Metamath.\index{Metamath!using as a math editor} To do this,
simply create your text with grave accents surrounding your formulas,
after making sure that your math symbols are mapped to \LaTeX\ symbols
as described in Appendix~\ref{ASCII}.  It is easier if you start with a
database with predefined symbols such as \texttt{set.mm}.  Use your
grave-quoted math string to replace an existing comment, then typeset
the statement corresponding to that comment following the instructions
from the \texttt{help tex} command in the Metamath program.  You will
then probably want to edit the resulting file with a text editor to fine
tune it to your exact needs.

\subsubsection{Label Mode}\index{label mode}

Outside of math mode, a tilde\index{tilde (\texttt{\char`\~})} \verb/~/
indicates to Metamath's\index{Metamath} output processor that the
token\index{token} that follows (i.e.\ the characters up to the next
white space\index{white space}) represents a statement label or URL.
This formatting mode is called {\bf label mode}\index{label mode}.
If a literal tilde
is desired (outside of math mode) instead of label mode,
use two tildes in a row to represent it.

When generating a \LaTeX\ output file,
the following token will be formatted in \texttt{typewriter}
font, and the tilde removed, to make it stand out from the rest of the text.
This formatting will be applied to all characters after the
tilde up to the first white space\index{white space}.
Whether
or not the token is an actual statement label is not checked, and the
token does not have to have the correct syntax for a label; no error
messages will be produced.  The only effect of the label mode on the
output is that typewriter font will be used for the tokens that are
placed in the \LaTeX\ output file.

When generating {\sc html},
the tokens after the tilde {\em must} be a URL (either http: or https:)
or a valid label.
Error messages will be issued during that output if they aren't.
A hyperlink will be generated to that URL or label.

\subsubsection{Link to bibliographical reference}\index{citation}%
\index{link to bibliographical reference}

Bibliographical references are handled specially when generating
{\sc html} if formatted specially.
Text in the form \texttt{[}{\em author}\texttt{]}
is considered a link to a bibliographical reference.
See \texttt{help html} and \texttt{help write
bibliography} in the Metamath program for more
information.
% \index{\texttt{\char`\[}\ldots\texttt{]} inside comments}
See also Sections~\ref{tcomment} and \ref{wrbib}.

The \texttt{[}{\em author}\texttt{]} notation will also create an entry in
the bibliography cross-reference file generated by \texttt{write
bibliography} (Section~\ref{wrbib}) for {\sc HTML}.
For this to work properly, the
surrounding comment must be formatted as follows:
\begin{quote}
    {\em keyword} {\em label} {\em noise-word}
     \texttt{[}{\em author}\texttt{] p.} {\em number}
\end{quote}
for example
\begin{verbatim}
     Theorem 5.2 of [Monk] p. 223
\end{verbatim}
The {\em keyword} is not case sensitive and must be one of the following:
\begin{verbatim}
     theorem lemma definition compare proposition corollary
     axiom rule remark exercise problem notation example
     property figure postulate equation scheme chapter
\end{verbatim}
The optional {\em label} may consist of more than one
(non-{\em keyword} and non-{\em noise-word}) word.
The optional {\em noise-word} is one of:
\begin{verbatim}
     of in from on
\end{verbatim}
and is  ignored when the cross-reference file is created.  The
\texttt{write
biblio\-graphy} command will perform error checking to verify the
above format.\index{error checking}

\subsubsection{Parentheticals}\label{parentheticals}

The end of a comment may include one or more parenthicals, that is,
statements enclosed in parentheses.
The Metamath program looks for certain parentheticals and can issue
warnings based on them.
They are:

\begin{itemize}
 \item[] \texttt{(Contributed by }
   \textit{NAME}\texttt{,} \textit{DATE}\texttt{.)} -
   document the original contributor's name and the date it was created.
 \item[] \texttt{(Revised by }
   \textit{NAME}\texttt{,} \textit{DATE}\texttt{.)} -
   document the contributor's name and creation date
   that resulted in significant revision
   (not just an automated minimization or shortening).
 \item[] \texttt{(Proof shortened by }
   \textit{NAME}\texttt{,} \textit{DATE}\texttt{.)} -
   document the contributor's name and date that developed a significant
   shortening of the proof (not just an automated minimization).
 \item[] \texttt{(Proof modification is discouraged.)} -
   Note that this proof should normally not be modified.
 \item[] \texttt{(New usage is discouraged.)} -
   Note that this assertion should normally not be used.
\end{itemize}

The \textit{DATE} must be in form YYYY-MMM-DD, where MMM is the
English abbreviation of that month.

\subsubsection{Other markup}\label{othermarkup}
\index{markup notation}

There are other markup notations for generating good-looking results
beyond math mode and label mode:

\begin{itemize}
 \item[]
         \texttt{\char`\_} (underscore)\index{\texttt{\char`\_} inside comments} -
             Italicize text starting from
              {\em space}\texttt{\char`\_}{\em non-space} (i.e.\ \texttt{\char`\_}
              with a space before it and a non-space character after it) until
             the next
             {\em non-space}\texttt{\char`\_}{\em space}.  Normal
             punctuation (e.g.\ a trailing
             comma or period) is ignored when determining {\em space}.
 \item[]
         \texttt{\char`\_} (underscore) - {\em
         non-space}\texttt{\char`\_}{\em non-space-string}, where
          {\em non-space-string} is a string of non-space characters,
         will make {\em non-space-string} become a subscript.
 \item[]
         \texttt{<HTML>}...\texttt{</HTML>} - do not convert
         ``\texttt{<}'' and ``\texttt{>}''
         in the enclosed text when generating {\sc HTML},
         otherwise process markup normally. This allows direct insertion
         of {\sc html} commands.
 \item[]
       ``\texttt{\&}ref\texttt{;}'' - insert an {\sc HTML}
         character reference.
         This is how to insert arbitrary Unicode characters
         (such as accented characters).  Currently only directly supported
         when generating {\sc HTML}.
\end{itemize}

It is recommended that spaces surround any \texttt{\char`\~} and
\texttt{`} tokens in the comment and that a space follow the {\em label}
after a \texttt{\char`\~} token.  This will make global substitutions
to change labels and symbol names much easier and also eliminate any
future chance of ambiguity.  Spaces around these tokens are automatically
removed in the final output to conform with normal rules of punctuation;
for example, a space between a trailing \texttt{`} and a left parenthesis
will be removed.

A good way to become familiar with the markup notation is to look at
the extensive examples in the \texttt{set.mm} database.

\subsection{The Typesetting Comment (\texttt{\$t})}\label{tcomment}

The typesetting comment \texttt{\$t} in the input database file
provides the information necessary to produce good-looking results.
It provides \LaTeX\ and {\sc html}
definitions for math symbols,
as well supporting as some
customization of the generated web page.
If you add a new token to a database, you should also
update the \texttt{\$t} comment information if you want to eventually
create output in \LaTeX\ or {\sc HTML}.
See the
\texttt{set.mm}\index{set theory database (\texttt{set.mm})} database
file for an extensive example of a \texttt{\$t} comment illustrating
many of the features described below.

Programs that do not need to generate good-looking presentation results,
such as programs that only verify Metamath databases,
can completely ignore typesetting comments
and just treat them as normal comments.
Even the Metamath program only consults the
\texttt{\$t} comment information when it needs to generate typeset output
in \LaTeX\ or {\sc HTML}
(e.g., when you open a \LaTeX\ output file with the \texttt{open tex} command).

We will first discuss the syntax of typesetting comments, and then
briefly discuss how this can be used within the Metamath program.

\subsubsection{Typesetting Comment Syntax Overview}

The typesetting comment is identified by the token
\texttt{\$t}\index{\texttt{\$t} comment}\index{typesetting comment} in
the comment, and the typesetting comment ends at the matching
\texttt{\$)}:
\[
  \mbox{\tt \$(\ }
  \mbox{\tt \$t\ }
  \underbrace{
    \mbox{\tt \ \ \ \ \ \ \ \ \ \ \ }
    \cdots
    \mbox{\tt \ \ \ \ \ \ \ \ \ \ \ }
  }_{\mbox{Typesetting definitions go here}}
  \mbox{\tt \ \$)}
\]

There must be one or more white space characters, and only white space
characters, between the \texttt{\$(} that starts the comment
and the \texttt{\$t} symbol,
and the \texttt{\$t} must be followed by one
or more white space characters
(see section \ref{whitespace} for the definition of white space characters).
The typesetting comment continues until the comment end token \texttt{\$)}
(which must be preceded by one or more white space characters).

In version 0.177\index{Metamath!limitations of version 0.177} of the
Metamath program, there may be only one \texttt{\$t} comment in a
database.  This restriction may be lifted in the future to allow
many \texttt{\$t} comments in a database.

Between the \texttt{\$t} symbol (and its following white space) and the
comment end token \texttt{\$)} (and its preceding white space)
is a sequence of one or more typesetting definitions, where
each definition has the form
\textit{definition-type arg arg ... ;}.
Each of the zero or more \textit{arg} values
can be either a typesetting data or a keyword
(what keywords are allowed, and where, depends on the specific
\textit{definition-type}).
The \textit{definition-type}, and each argument \textit{arg},
are separated by one or more white space characters.
Every definition ends in an unquoted semicolon;
white space is not required before the terminating semicolon of a definition.
Each definition should start on a new line.\footnote{This
restriction of the current version of Metamath
(0.177)\index{Metamath!limitations of version 0.177} may be removed
in a future version, but you should do it anyway for readability.}

For example, this typesetting definition:
\begin{center}
 \verb$latexdef "C_" as "\subseteq";$
\end{center}
defines the token \verb$C_$ as the \LaTeX\ symbol $\subseteq$ (which means
``subset'').

Typesetting data is a sequence of one or more quoted strings
(if there is more than one, they are connected by \texttt{\char`\+}).
Often a single quoted string is used to provide data for a definition, using
either double (\texttt{\char`\"}) or single (\texttt{'}) quotation marks.
However,
{\em a quoted string (enclosed in quotation marks) may not include
line breaks.}
A quoted string
may include a quotation mark that matches the enclosing quotes by repeating
the quotation mark twice.  Here are some examples:

\begin{tabu}   { l l }
\textbf{Example} & \textbf{Meaning} \\
\texttt{\char`\"a\char`\"\char`\"b\char`\"} & \texttt{a\char`\"b} \\
\texttt{'c''d'} & \texttt{c'd} \\
\texttt{\char`\"e''f\char`\"} & \texttt{e''f} \\
\texttt{'g\char`\"\char`\"h'} & \texttt{g\char`\"\char`\"h} \\
\end{tabu}

Finally, a long quoted string
may be broken up into multiple quoted strings (considered, as a whole,
a single quoted string) and joined with \texttt{\char`\+}.
You can even use multiple lines as long as a
'+' is at the end of every line except the last one.
The \texttt{\char`\+} should be preceded and followed by at least one
white space character.
Thus, for example,
\begin{center}
 \texttt{\char`\"ab\char`\"\ \char`\+\ \char`\"cd\char`\"
    \ \char`\+\ \\ 'ef'}
\end{center}
is the same as
\begin{center}
 \texttt{\char`\"abcdef\char`\"}
\end{center}

{\sc c}-style comments \texttt{/*}\ldots\texttt{*/} are also supported.

In practice, whenever you add a new math token you will often want to add
typesetting definitions using
\texttt{latexdef}, \texttt{htmldef}, and
\texttt{althtmldef}, as described below.
That way, they will all be up to date.
Of course, whether or not you want to use all three definitions will
depend on how the database is intended to be used.

Below we discuss the different possible \textit{definition-kind} options.
We will show data surrounded by double quotes (in practice they can also use
single quotes and/or be a sequence joined by \texttt{+}s).
We will use specific names for the \textit{data} to make clear what
the data is used for, such as
{\em math-token} (for a Metamath math token,
{\em latex-string} (for string to be placed in a \LaTeX\ stream),
{\em {\sc html}-code} (for {\sc html} code),
and {\em filename} (for a filename).

\subsubsection{Typesetting Comment - \LaTeX}

The syntax for a \LaTeX\ definition is:
\begin{center}
 \texttt{latexdef "}{\em math-token}\texttt{" as "}{\em latex-string}\texttt{";}
\end{center}
\index{latex definitions@\LaTeX\ definitions}%
\index{\texttt{latexdef} statement}

The {\em token-string} and {\em latex-string} are the data
(character strings) for
the token and the \LaTeX\ definition of the token, respectively,

These \LaTeX\ definitions are used by the Metamath program
when it is asked to product \LaTeX output using
the \texttt{write tex} command.

\subsubsection{Typesetting Comment - {\sc html}}

The key kinds of {\sc HTML} definitions have the following syntax:

\vskip 1ex
    \texttt{htmldef "}{\em math-token}\texttt{" as "}{\em
    {\sc html}-code}\texttt{";}\index{\texttt{htmldef} statement}
                    \ \ \ \ \ \ldots

    \texttt{althtmldef "}{\em math-token}\texttt{" as "}{\em
{\sc html}-code}\texttt{";}\index{\texttt{althtmldef} statement}

                    \ \ \ \ \ \ldots

Note that in {\sc HTML} there are two possible definitions for math tokens.
This feature is useful when
an alternate representation of symbols is desired, for example one that
uses Unicode entities and another uses {\sc gif} images.

There are many other typesetting definitions that can control {\sc HTML}.
These include:

\vskip 1ex

    \texttt{htmldef "}{\em math-token}\texttt{" as "}{\em {\sc
    html}-code}\texttt{";}

    \texttt{htmltitle "}{\em {\sc html}-code}\texttt{";}%
\index{\texttt{htmltitle} statement}

    \texttt{htmlhome "}{\em {\sc html}-code}\texttt{";}%
\index{\texttt{htmlhome} statement}

    \texttt{htmlvarcolor "}{\em {\sc html}-code}\texttt{";}%
\index{\texttt{htmlvarcolor} statement}

    \texttt{htmlbibliography "}{\em filename}\texttt{";}%
\index{\texttt{htmlbibliography} statement}

\vskip 1ex

\noindent The \texttt{htmltitle} is the {\sc html} code for a common
title, such as ``Metamath Proof Explorer.''  The \texttt{htmlhome} is
code for a link back to the home page.  The \texttt{htmlvarcolor} is
code for a color key that appears at the bottom of each proof.  The file
specified by {\em filename} is an {\sc html} file that is assumed to
have a \texttt{<A NAME=}\ldots\texttt{>} tag for each bibiographic
reference in the database comments.  For example, if
\texttt{[Monk]}\index{\texttt{\char`\[}\ldots\texttt{]} inside comments}
occurs in the comment for a theorem, then \texttt{<A NAME='Monk'>} must
be present in the file; if not, a warning message is given.

Associated with
\texttt{althtmldef}
are the statements
\vskip 1ex

    \texttt{htmldir "}{\em
      directoryname}\texttt{";}\index{\texttt{htmldir} statement}

    \texttt{althtmldir "}{\em
     directoryname}\texttt{";}\index{\texttt{althtmldir} statement}

\vskip 1ex
\noindent giving the directories of the {\sc gif} and Unicode versions
respectively; their purpose is to provide cross-linking between the
two versions in the generated web pages.

When two different types of pages need to be produced from a single
database, such as the Hilbert Space Explorer that extends the Metamath
Proof Explorer, ``extended'' variables may be declared in the
\texttt{\$t} comment:
\vskip 1ex

    \texttt{exthtmltitle "}{\em {\sc html}-code}\texttt{";}%
\index{\texttt{exthtmltitle} statement}

    \texttt{exthtmlhome "}{\em {\sc html}-code}\texttt{";}%
\index{\texttt{exthtmlhome} statement}

    \texttt{exthtmlbibliography "}{\em filename}\texttt{";}%
\index{\texttt{exthtmlbibliography} statement}

\vskip 1ex
\noindent When these are declared, you also must declare
\vskip 1ex

    \texttt{exthtmllabel "}{\em label}\texttt{";}%
\index{\texttt{exthtmllabel} statement}

\vskip 1ex \noindent that identifies the database statement where the
``extended'' section of the database starts (in our example, where the
Hilbert Space Explorer starts).  During the generation of web pages for
that starting statement and the statements after it, the {\sc html} code
assigned to \texttt{exthtmltitle} and \texttt{exthtmlhome} is used
instead of that assigned to \texttt{htmltitle} and \texttt{htmlhome},
respectively.

\begin{sloppy}
\subsection{Additional Information Com\-ment (\texttt{\$j})} \label{jcomment}
\end{sloppy}

The additional information comment, aka the
\texttt{\$j}\index{\texttt{\$j} comment}\index{additional information comment}
comment,
provides a way to add additional structured information that can
be optionally parsed by systems.

The additional information comment is parsed the same way as the
typesetting comment (\texttt{\$t}) (see section \ref{tcomment}).
That is,
the additional information comment begins with the token
\texttt{\$j} within a comment,
and continues until the comment close \texttt{\$)}.
Within an additional information comment is a sequence of one or more
commands of the form \texttt{command arg arg ... ;}
where each of the zero or more \texttt{arg} values
can be either a quoted string or a keyword.
Note that every command ends in an unquoted semicolon.
If a verifier is parsing an additional information comment, but
doesn't recognize a particular command, it must skip the command
by finding the end of the command (an unquoted semicolon).

A database may have 0 or more additional information comments.
Note, however, that a verifier may ignore these comments entirely or only
process certain commands in an additional information comment.
The \texttt{mmj2} verifier supports many commands in additional information
comments.
We encourage systems that process additional information comments
to coordinate so that they will use the same command for the same effect.

Examples of additional information comments with various commands
(from the \texttt{set.mm} database) are:

\begin{itemize}
   \item Define the syntax and logical typecodes,
     and declare that our grammar is
     unambiguous (verifiable using the KLR parser, with compositing depth 5).
\begin{verbatim}
  $( $j
    syntax 'wff';
    syntax '|-' as 'wff';
    unambiguous 'klr 5';
  $)
\end{verbatim}

   \item Register $\lnot$ and $\rightarrow$ as primitive expressions
           (lacking definitions).
\begin{verbatim}
  $( $j primitive 'wn' 'wi'; $)
\end{verbatim}

   \item There is a special justification for \texttt{df-bi}.
\begin{verbatim}
  $( $j justification 'bijust' for 'df-bi'; $)
\end{verbatim}

   \item Register $\leftrightarrow$ as an equality for its type (wff).
\begin{verbatim}
  $( $j
    equality 'wb' from 'biid' 'bicomi' 'bitri';
    definition 'dfbi1' for 'wb';
  $)
\end{verbatim}

   \item Theorem \texttt{notbii} is the congruence law for negation.
\begin{verbatim}
  $( $j congruence 'notbii'; $)
\end{verbatim}

   \item Add \texttt{setvar} as a typecode.
\begin{verbatim}
  $( $j syntax 'setvar'; $)
\end{verbatim}

   \item Register $=$ as an equality for its type (\texttt{class}).
\begin{verbatim}
  $( $j equality 'wceq' from 'eqid' 'eqcomi' 'eqtri'; $)
\end{verbatim}

\end{itemize}


\subsection{Including Other Files in a Metamath Source File} \label{include}
\index{\texttt{\$[} and \texttt{\$]} auxiliary keywords}

The keywords \texttt{\$[} and \texttt{\$]} specify a file to be
included\index{included file}\index{file inclusion} at that point in a
Metamath\index{Metamath} source file\index{source file}.  The syntax for
including a file is as follows:
\begin{center}
\texttt{\$[} {\em file-name} \texttt{\$]}
\end{center}

The {\em file-name} should be a single token\index{token} with the same syntax
as a math symbol (i.e., all 93 non-whitespace
printable characters other than \texttt{\$} are
allowed, subject to the file-naming limitations of your operating system).
Comments may appear between the \texttt{\$[} and \texttt{\$]} keywords.  Included
files may include other files, which may in turn include other files, and so
on.

For example, suppose you want to use the set theory database as the starting
point for your own theory.  The first line in your file could be
\begin{center}
\texttt{\$[ set.mm \$]}
\end{center} All of the information (axioms, theorems,
etc.) in \texttt{set.mm} and any files that {\em it} includes will become
available for you to reference in your file. This can help make your work more
modular. A drawback to including files is that if you change the name of a
symbol or the label of a statement, you must also remember to update any
references in any file that includes it.


The naming conventions for included files are the same as those of your
operating system.\footnote{On the Macintosh, prior to Mac OS X,
 a colon is used to separate disk
and folder names from your file name.  For example, {\em volume}\texttt{:}{\em
file-name} refers to the root directory, {\em volume}\texttt{:}{\em
folder-name}\texttt{:}{\em file-name} refers to a folder in root, and {\em
volume}\texttt{:}{\em folder-name}\texttt{:}\ldots\texttt{:}{\em file-name} refers to a
deeper folder.  A simple {\em file-name} refers to a file in the folder from
which you launch the Metamath application.  Under Mac OS X and later,
the Metamath program is run under the Terminal application, which
conforms to Unix naming conventions.}\index{Macintosh file
names}\index{file names!Macintosh}\label{includef} For compatibility among
operating systems, you should keep the file names as simple as possible.  A
good convention to use is {\em file}\texttt{.mm} where {\em file} is eight
characters or less, in lower case.

There is no limit to the nesting depth of included files.  One thing that you
should be aware of is that if two included files themselves include a common
third file, only the {\em first} reference to this common file will be read
in.  This allows you to include two or more files that build on a common
starting file without having to worry about label and symbol conflicts that
would occur if the common file were read in more than once.  (In fact, if a
file includes itself, the self-reference will be ignored, although of course
it would not make any sense to do that.)  This feature also means, however,
that if you try to include a common file in several inner blocks, the result
might not be what you expect, since only the first reference will be replaced
with the included file (unlike the include statement in most other computer
languages).  Thus you would normally include common files only in the
outermost block\index{outermost block}.

\subsection{Compressed Proof Format}\label{compressed1}\index{compressed
proof}\index{proof!compressed}

The proof notation presented in Section~\ref{proof} is called a
{\bf normal proof}\index{normal proof}\index{proof!normal} and in principle is
sufficient to express any proof.  However, proofs often contain steps and
subproofs that are identical.  This is particularly true in typical
Metamath\index{Metamath} applications, because Metamath requires that the math
symbol sequence (usually containing a formula) at each step be separately
constructed, that is, built up piece by piece. As a result, a lot of
repetition often results.  The {\bf compressed proof} format allows Metamath
to take advantage of this redundancy to shorten proofs.

The specification for the compressed proof format is given in
Appen\-dix~\ref{compressed}.

Normally you need not concern yourself with the details of the compressed
proof format, since the Metamath program will allow you to convert from
the normal format to the compressed format with ease, and will also
automatically convert from the compressed format when proofs are displayed.
The overall structure of the compressed format is as follows:
\begin{center}
  \texttt{\$= ( } {\em label-list} \texttt{) } {\em compressed-proof\ }\ \texttt{\$.}
\end{center}
\index{\texttt{\$=} keyword}
The first \texttt{(} serves as a flag to Metamath that a compressed proof
follows.  The {\em label-list} includes all statements referred to by the
proof except the mandatory hypotheses\index{mandatory hypothesis}.  The {\em
compressed-proof} is a compact encoding of the proof, using upper-case
letters, and can be thought of as a large integer in base 26.  White
space\index{white space} inside a {\em compressed-proof} is
optional and is ignored.

It is important to note that the order of the mandatory hypotheses of
the statement being proved must not be changed if the compressed proof
format is used, otherwise the proof will become incorrect.  The reason
for this is that the mandatory hypotheses are not mentioned explicitly
in the compressed proof in order to make the compression more efficient.
If you wish to change the order of mandatory hypotheses, you must first
convert the proof back to normal format using the \texttt{save proof
{\em statement} /normal}\index{\texttt{save proof} command} command.
Later, you can go back to compressed format with \texttt{save proof {\em
statement} /compressed}.

During error checking with the \texttt{verify proof} command, an error
found in a compressed proof may point to a character in {\em
compressed-proof}, which may not be very meaningful to you.  In this
case, try to \texttt{save proof /normal} first, then do the
\texttt{verify proof} again.  In general, it is best to make sure a
proof is correct before saving it in compressed format, because severe
errors are less likely to be recoverable than in normal format.

\subsection{Specifying Unknown Proofs or Subproofs}\label{unknown}

In a proof under development, any step or subproof that is not yet known
may be represented with a single \texttt{?}.  For the purposes of
parsing the proof, the \texttt{?}\ \index{\texttt{]}@\texttt{?}\ inside
proofs} will push a single entry onto the RPN stack just as if it were a
hypothesis.  While developing a proof with the Proof
Assistant\index{Proof Assistant}, a partially developed proof may be
saved with the \texttt{save new{\char`\_}proof}\index{\texttt{save
new{\char`\_}proof} command} command, and \texttt{?}'s will be placed at
the appropriate places.

All \texttt{\$p}\index{\texttt{\$p} statement} statements must have
proofs, even if they are entirely unknown.  Before creating a proof with
the Proof Assistant, you should specify a completely unknown proof as
follows:
\begin{center}
  {\em label} \texttt{\$p} {\em statement} \texttt{\$= ?\ \$.}
\end{center}
\index{\texttt{\$=} keyword}
\index{\texttt{]}@\texttt{?}\ inside proofs}

The \texttt{verify proof}\index{\texttt{verify proof} command} command
will check the known portions of a partial proof for errors, but will
warn you that the statement has not been proved.

Note that partially developed proofs may be saved in compressed format
if desired.  In this case, you will see one or more \texttt{?}'s in the
{\em compressed-proof} part.\index{compressed
proof}\index{proof!compressed}

\section{Axioms vs.\ Definitions}\label{definitions}

The \textit{basic}
Metamath\index{Metamath} language and program
make no distinction\index{axiom vs.\
definition} between axioms\index{axiom} and
definitions.\index{definition} The \texttt{\$a}\index{\texttt{\$a}
statement} statement is used for both.  At first, this may seem
puzzling.  In the minds of many mathematicians, the distinction is
clear, even obvious, and hardly worth discussing.  A definition is
considered to be merely an abbreviation that can be replaced by the
expression for which it stands; although unless one actually does this,
to be precise then one should say that a theorem\index{theorem} is a
consequence of the axioms {\em and} the definitions that are used in the
formulation of the theorem \cite[p.~20]{Behnke}.\index{Behnke, H.}

\subsection{What is a Definition?}

What is a definition?  In its simplest form, a definition introduces a new
symbol and provides an unambiguous rule to transform an expression containing
the new symbol to one without it.  The concept of a ``proper
definition''\index{proper definition}\index{definition!proper} (as opposed to
a creative definition)\index{creative definition}\index{definition!creative}
that is usually agreed upon is (1) the definition should not strengthen the
language and (2) any symbols introduced by the definition should be eliminable
from the language \cite{Nemesszeghy}\index{Nemesszeghy, E. Z.}.  In other
words, they are mere typographical conveniences that do not belong to the
system and are theoretically superfluous.  This may seem obvious, but in fact
the nature of definitions can be subtle, sometimes requiring difficult
metatheorems to establish that they are not creative.

A more conservative stance was taken by logician S.
Le\'{s}niewski.\index{Le\'{s}niewski, S.}
\begin{quote}
Le\'{s}niewski
regards definitions as theses of the system.  In this respect they do
not differ either from the axioms or from theorems, i.e.\ from the
theses added to the system on the basis of the rule of substitution or
the rule of detachment [modus ponens].  Once definitions have been
accepted as theses of the system, it becomes necessary to consider them
as true propositions in the same sense in which axioms are true
\cite{Lejewski}.
\end{quote}\index{Lejewski, Czeslaw}

Let us look at some simple examples of definitions in propositional
calculus.  Consider the definition of logical {\sc or}
(disjunction):\index{disjunction ($\vee$)} ``$P\vee Q$ denotes $\neg P
\rightarrow Q$ (not $P$ implies $Q$).''  It is very easy to recognize a
statement making use of this definition, because it introduces the new
symbol $\vee$ that did not previously exist in the language.  It is easy
to see that no new theorems of the original language will result from
this definition.

Next, consider a definition that eliminates parentheses:  ``$P
\rightarrow Q\rightarrow R$ denotes $P\rightarrow (Q \rightarrow R)$.''
This is more subtle, because no new symbols are introduced.  The reason
this definition is considered proper is that no new symbol sequences
that are valid wffs (well-formed formulas)\index{well-formed formula
(wff)} in the original language will result from the definition, since
``$P \rightarrow Q\rightarrow R$'' is not a wff in the original
language.  Here, we implicitly make use of the fact that there is a
decision procedure that allows us to determine whether or not a symbol
sequence is a wff, and this fact allows us to use symbol sequences that
are not wffs to represent other things (such as wffs) by means of the
definition.  However, to justify the definition as not being creative we
need to prove that ``$P \rightarrow Q\rightarrow R$'' is in fact not a
wff in the original language, and this is more difficult than in the
case where we simply introduce a new symbol.

%Now let's take this reasoning to an extreme.  Propositional calculus is a
%decidable theory,\footnote{This means that a mechanical algorithm exists to
%determine whether or not a wff is a theorem.} so in principle we could make use
%of symbol sequences that are not theorems to represent other things (say, to
%encode actual theorems in a more compact way).  For example, let us extend the
%language by defining a wff ``$P$'' in the extended language as the theorem
%``$P\rightarrow P$''\footnote{This is one of the first theorems proved in the
%Metamath database \texttt{set.mm}.}\index{set
%theory database (\texttt{set.mm})} in the original language whenever ``$P$'' is
%not a theorem in the original language.  In the extended language, any wff
%``$Q$'' thus represents a theorem; to find out what theorem (in the original
%language) ``$Q$'' represents, we determine whether ``$Q$'' is a theorem in the
%original language (before the definition was introduced).  If so, we're done; if
%not, we replace ``$Q$'' by ``$Q\rightarrow Q$'' to eliminate the definition.
%This definition is therefore eliminable, and it does not ``strengthen'' the
%language because any wff that is not a theorem is not in the set of statements
%provable in the original language and thus is available for use by definitions.
%
%Of course, a definition such as this would render practically useless the
%communication of theorems of propositional calculus; but
%this is just a human shortcoming, since we can't always easily discern what is
%and is not a theorem by inspection.  In fact, the extended theory with this
%definition has no more and no less information than the original theory; it just
%expresses certain theorems of the form ``$P\rightarrow P$''
%in a more compact way.
%
%The point here is that what constitutes a proper definition is a matter of
%judgment about whether a symbol sequence can easily be recognized by a human
%as invalid in some sense (for example, not a wff); if so, the symbol sequence
%can be appropriated for use by a definition in order to make the extended
%language more compact.  Metamath\index{Metamath} lacks the ability to make this
%judgment, since as far as Metamath is concerned the definition of a wff, for
%example, is arbitrary.  You define for Metamath how wffs\index{well-formed
%formula (wff)} are constructed according to your own preferred style.  The
%concept of a wff may not even exist in a given formal system\index{formal
%system}.  Metamath treats all definitions as if they were new axioms, and it
%is up to the human mathematician to judge whether the definition is ``proper''
%'\index{proper definition}\index{definition!proper} in some agreed-upon way.

What constitutes a definition\index{definition} versus\index{axiom vs.\
definition} an axiom\index{axiom} is sometimes arbitrary in mathematical
literature.  For example, the connectives $\vee$ ({\sc or}), $\wedge$
({\sc and}), and $\leftrightarrow$ (equivalent to) in propositional
calculus are usually considered defined symbols that can be used as
abbreviations for expressions containing the ``primitive'' connectives
$\rightarrow$ and $\neg$.  This is the way we treat them in the standard
logic and set theory database \texttt{set.mm}\index{set theory database
(\texttt{set.mm})}.  However, the first three connectives can also be
considered ``primitive,'' and axiom systems have been devised that treat
all of them as such.  For example,
\cite[p.~35]{Goodstein}\index{Goodstein, R. L.} presents one with 15
axioms, some of which in fact coincide with what we have chosen to call
definitions in \texttt{set.mm}.  In certain subsets of classical
propositional calculus, such as the intuitionist
fragment\index{intuitionism}, it can be shown that one cannot make do
with just $\rightarrow$ and $\neg$ but must treat additional connectives
as primitive in order for the system to make sense.\footnote{Two nice
systems that make the transition from intuitionistic and other weak
fragments to classical logic just by adding axioms are given in
\cite{Robinsont}\index{Robinson, T. Thacher}.}

\subsection{The Approach to Definitions in \texttt{set.mm}}

In set theory, recursive definitions define a newly introduced symbol in
terms of itself.
The justification of recursive definitions, using
several ``recursion theorems,'' is usually one of the first
sophisticated proofs a student encounters when learning set theory, and
there is a significant amount of implicit metalogic behind a recursive
definition even though the definition itself is typically simple to
state.

Metamath itself has no built-in technical limitation that prevents
multiple-part recursive definitions in the traditional textbook style.
However, because the recursive definition requires advanced metalogic
to justify, eliminating a recursive definition is very difficult and
often not even shown in textbooks.

\subsubsection{Direct definitions instead of recursive definitions}

It is, however, possible to substitute one kind of complexity
for another.  We can eliminate the need for metalogical justification by
defining the operation directly with an explicit (but complicated)
expression, then deriving the recursive definition directly as a
theorem, using a recursion theorem ``in reverse.''
The elimination
of a direct definition is a matter of simple mechanical substitution.
We do this in
\texttt{set.mm}, as follows.

In \texttt{set.mm} our goal was to introduce almost all definitions in
the form of two expressions connected by either $\leftrightarrow$ or
$=$, where the thing being defined does not appear on the right hand
side.  Quine calls this form ``a genuine or direct definition'' \cite[p.
174]{Quine}\index{Quine, Willard Van Orman}, which makes the definitions
very easy to eliminate and the metalogic\index{metalogic} needed to
justify them as simple as possible.
Put another way, we had a goal of being able to
eliminate all definitions with direct mechanical substitution and to
verify easily the soundness of the definitions.

\subsubsection{Example of direct definitions}

We achieved this goal in almost all cases in \texttt{set.mm}.
Sometimes this makes the definitions more complex and less
intuitive.
For example, the traditional way to define addition of
natural numbers is to define an operation called {\em
successor}\index{successor} (which means ``plus one'' and is denoted by
``${\rm suc}$''), then define addition recursively\index{recursive
definition} with the two definitions $n + 0 = n$ and $m + {\rm suc}\,n =
{\rm suc} (m + n)$.  Although this definition seems simple and obvious,
the method to eliminate the definition is not obvious:  in the second
part of the definition, addition is defined in terms of itself.  By
eliminating the definition, we don't mean repeatedly applying it to
specific $m$ and $n$ but rather showing the explicit, closed-form
set-theoretical expression that $m + n$ represents, that will work for
any $m$ and $n$ and that does not have a $+$ sign on its right-hand
side.  For a recursive definition like this not to be circular
(creative), there are some hidden, underlying assumptions we must make,
for example that the natural numbers have a certain kind of order.

In \texttt{set.mm} we chose to start with the direct (though complex and
nonintuitive) definition then derive from it the standard recursive
definition.
For example, the closed-form definition used in \texttt{set.mm}
for the addition operation on ordinals\index{ordinal
addition}\index{addition!of ordinals} (of which natural numbers are a
subset) is

\setbox\startprefix=\hbox{\tt \ \ df-oadd\ \$a\ }
\setbox\contprefix=\hbox{\tt \ \ \ \ \ \ \ \ \ \ \ \ \ }
\startm
\m{\vdash}\m{+_o}\m{=}\m{(}\m{x}\m{\in}\m{{\rm On}}\m{,}\m{y}\m{\in}\m{{\rm
On}}\m{\mapsto}\m{(}\m{{\rm rec}}\m{(}\m{(}\m{z}\m{\in}\m{{\rm
V}}\m{\mapsto}\m{{\rm suc}}\m{z}\m{)}\m{,}\m{x}\m{)}\m{`}\m{y}\m{)}\m{)}
\endm
\noindent which depends on ${\rm rec}$.

\subsubsection{Recursion operators}

The above definition of \texttt{df-oadd} depends on the definition of
${\rm rec}$, a ``recursion operator''\index{recursion operator} with
the definition \texttt{df-rdg}:

\setbox\startprefix=\hbox{\tt \ \ df-rdg\ \$a\ }
\setbox\contprefix=\hbox{\tt \ \ \ \ \ \ \ \ \ \ \ \ }
\startm
\m{\vdash}\m{{\rm
rec}}\m{(}\m{F}\m{,}\m{I}\m{)}\m{=}\m{\mathrm{recs}}\m{(}\m{(}\m{g}\m{\in}\m{{\rm
V}}\m{\mapsto}\m{{\rm if}}\m{(}\m{g}\m{=}\m{\varnothing}\m{,}\m{I}\m{,}\m{{\rm
if}}\m{(}\m{{\rm Lim}}\m{{\rm dom}}\m{g}\m{,}\m{\bigcup}\m{{\rm
ran}}\m{g}\m{,}\m{(}\m{F}\m{`}\m{(}\m{g}\m{`}\m{\bigcup}\m{{\rm
dom}}\m{g}\m{)}\m{)}\m{)}\m{)}\m{)}\m{)}
\endm

\noindent which can be further broken down with definitions shown in
Section~\ref{setdefinitions}.

This definition of ${\rm rec}$
defines a recursive definition generator on ${\rm On}$ (the class of ordinal
numbers) with characteristic function $F$ and initial value $I$.
This operation allows us to define, with
compact direct definitions, functions that are usually defined in
textbooks with recursive definitions.
The price paid with our approach
is the complexity of our ${\rm rec}$ operation
(especially when {\tt df-recs} that it is built on is also eliminated).
But once we get past this hurdle, definitions that would otherwise be
recursive become relatively simple, as in for example {\tt oav}, from
which we prove the recursive textbook definition as theorems {\tt oa0}, {\tt
oasuc}, and {\tt oalim} (with the help of theorems {\tt rdg0}, {\tt rdgsuc},
and {\tt rdglim2a}).  We can also restrict the ${\rm rec}$ operation to
define otherwise recursive functions on the natural numbers $\omega$; see {\tt
fr0g} and {\tt frsuc}.  Our ${\rm rec}$ operation apparently does not appear
in published literature, although closely related is Definition 25.2 of
[Quine] p. 177, which he uses to ``turn...a recursion into a genuine or
direct definition" (p. 174).  Note that the ${\rm if}$ operations (see
{\tt df-if}) select cases based on whether the domain of $g$ is zero, a
successor, or a limit ordinal.

An important use of this definition ${\rm rec}$ is in the recursive sequence
generator {\tt df-seq} on the natural numbers (as a subset of the
complex infinite sequences such as the factorial function {\tt df-fac} and
integer powers {\tt df-exp}).

The definition of ${\rm rec}$ depends on ${\rm recs}$.
New direct usage of the more powerful (and more primitive) ${\rm recs}$
construct is discouraged, but it is available when needed.
This
defines a function $\mathrm{recs} ( F )$ on ${\rm On}$, the class of ordinal
numbers, by transfinite recursion given a rule $F$ which sets the next
value given all values so far.
Unlike {\tt df-rdg} which restricts the
update rule to use only the previous value, this version allows the
update rule to use all previous values, which is why it is described
as ``strong,'' although it is actually more primitive.  See {\tt
recsfnon} and {\tt recsval} for the primary contract of this definition.
It is defined as:

\setbox\startprefix=\hbox{\tt \ \ df-recs\ \$a\ }
\setbox\contprefix=\hbox{\tt \ \ \ \ \ \ \ \ \ \ \ \ \ }
\startm
\m{\vdash}\m{\mathrm{recs}}\m{(}\m{F}\m{)}\m{=}\m{\bigcup}\m{\{}\m{f}\m{|}\m{\exists}\m{x}\m{\in}\m{{\rm
On}}\m{(}\m{f}\m{{\rm
Fn}}\m{x}\m{\wedge}\m{\forall}\m{y}\m{\in}\m{x}\m{(}\m{f}\m{`}\m{y}\m{)}\m{=}\m{(}\m{F}\m{`}\m{(}\m{f}\m{\restriction}\m{y}\m{)}\m{)}\m{)}\m{\}}
\endm

\subsubsection{Closing comments on direct definitions}

From these direct definitions the simpler, more
intuitive recursive definition is derived as a set of theorems.\index{natural
number}\index{addition}\index{recursive definition}\index{ordinal addition}
The end result is the same, but we completely eliminate the rather complex
metalogic that justifies the recursive definition.

Recursive definitions are often considered more efficient and intuitive than
direct ones once the metalogic has been learned or possibly just accepted as
correct.  However, it was felt that direct definition in \texttt{set.mm}
maximizes rigor by minimizing metalogic.  It can be eliminated effortlessly,
something that is difficult to do with a recursive definition.

Again, Metamath itself has no built-in technical limitation that prevents
multiple-part recursive definitions in the traditional textbook style.
Instead, our goal is to eliminate all definitions with
direct mechanical substitution and to verify easily the soundness of
definitions.

\subsection{Adding Constraints on Definitions}

The basic Metamath language and the Metamath program do
not have any built-in constraints on definitions, since they are just
\$a statements.

However, nothing prevents a verification system from verifying
additional rules to impose further limitations on definitions.
For example, the \texttt{mmj2}\index{mmj2} program
supports various kinds of
additional information comments (see section \ref{jcomment}).
One of their uses is to optionally verify additional constraints,
including constraints to verify that definitions meet certain
requirements.
These additional checks are required by the
continuous integration (CI)\index{continuous integration (CI)}
checks of the
\texttt{set.mm}\index{set theory database (\texttt{set.mm})}%
\index{Metamath Proof Explorer}
database.
This approach enables us to optionally impose additional requirements
on definitions if we wish, without necessarily imposing those rules on
all databases or requiring all verification systems to implement them.
In addition, this allows us to impose specialized constraints tailored
to one database while not requiring other systems to implement
those specialized constraints.

We impose two constraints on the
\texttt{set.mm}\index{set theory database (\texttt{set.mm})}%
\index{Metamath Proof Explorer} database
via the \texttt{mmj2}\index{mmj2} program that are worth discussing here:
a parse check and a definition soundness check.

% On February 11, 2019 8:32:32 PM EST, saueran@oregonstate.edu wrote:
% The following addition to the end of set.mm is accepted by the mmj2
% parser and definition checker and the metamath verifier(at least it was
% when I checked, you should check it too), and creates a contradiction by
% proving the theorem |- ph.
% ${
% wleftp $a wff ( ( ph ) $.
% wbothp $a wff ( ph ) $.
% df-leftp $a |- ( ( ( ph ) <-> -. ph ) $.
% df-bothp $a |- ( ( ph ) <-> ph ) $.
% anything $p |- ph $=
%   ( wbothp wn wi wleftp df-leftp biimpi df-bothp mpbir mpbi simplim ax-mp)
%   ABZAMACZDZCZMOEZOCQAEZNDZRNAFGSHIOFJMNKLAHJ $.
% $}
%
% This particular problem is countered by enabling, within mmj2,
% SetParser,mmj.verify.LRParser

First,
we enable a parse check in \texttt{mmj2} (through its
\texttt{SetParser} command) that requires that all new definitions
must \textit{not} create an ambiguous parse for a KLR(5) parser.
This prevents some errors such as definitions with imbalanced parentheses.

Second, we run a definition soundness check specific to
\texttt{set.mm} or databases similar to it.
(through the \texttt{definitionCheck} macro).
Some \texttt{\$a} statements (including all ax-* statemnets)
are excluded from these checks, as they will
always fail this simple check,
but they are appropriate for most definitions.
This check imposes a set of additional rules:

\begin{enumerate}

\item New definitions must be introduced using $=$ or $\leftrightarrow$.

\item No \texttt{\$a} statement introduced before this one is allowed to use the
symbol being defined in this definition, and the definition is not
allowed to use itself (except once, in the definiendum).

\item Every variable in the definiens should not be distinct

\item Every dummy variable in the definiendum
are required to be distinct from each other and from variables in
the definiendum.
To determine this, the system will look for a "justification" theorem
in the database, and if it is not there, attempt to briefly prove
$( \varphi \rightarrow \forall x \varphi )$  for each dummy variable x.

\item Every dummy variable should be a set variable,
unless there is a justification theorem available.

\item Every dummy variable must be bound
(if the system cannot determine this a justification theorem must be
provided).

\end{enumerate}

\subsection{Summary of Approach to Definitions}

In short, when being rigorous it turns out that
definitions can be subtle, sometimes requiring difficult
metatheorems to establish that they are not creative.

Instead of building such complications into the Metamath language itself,
the basic Metmath language and program simply treat traditional
axioms and definitions as the same kind of \texttt{\$a} statement.
We have then built various tools to enable people to
verify additional conditions as their creators believe is appropriate
for those specific databases, without complicating the Metamath foundations.

\chapter{The Metamath Program}\label{commands}

This chapter provides a reference manual for the
Metamath program.\index{Metamath!commands}

Current instructions for obtaining and installing the Metamath program
can be found at the \url{http://metamath.org} web site.
For Windows, there is a pre-compiled version called
\texttt{metamath.exe}.  For Unix, Linux, and Mac OS X
(which we will refer to collectively as ``Unix''), the Metamath program
can be compiled from its source code with the command
\begin{verbatim}
gcc *.c -o metamath
\end{verbatim}
using the \texttt{gcc} {\sc c} compiler available on those systems.

In the command syntax descriptions below, fields enclosed in square
brackets [\ ] are optional.  File names may be optionally enclosed in
single or double quotes.  This is useful if the file name contains
spaces or
slashes (\texttt{/}), such as in Unix path names, \index{Unix file
names}\index{file names!Unix} that might be confused with Metamath
command qualifiers.\index{Unix file names}\index{file names!Unix}

\section{Invoking Metamath}

Unix, Linux, and Mac OS X
have a command-line interface called the {\em
bash shell}.  (In Mac OS X, select the Terminal application from
Applications/Utilities.)  To invoke Metamath from the bash shell prompt,
assuming that the Metamath program is in the current directory, type
\begin{verbatim}
bash$ ./metamath
\end{verbatim}

To invoke Metamath from a Windows DOS or Command Prompt, assuming that
the Metamath program is in the current directory (or in a directory
included in the Path system environment variable), type
\begin{verbatim}
C:\metamath>metamath
\end{verbatim}

To use command-line arguments at invocation, the command-line arguments
should be a list of Metamath commands, surrounded by quotes if they
contain spaces.  In Windows, the surrounding quotes must be double (not
single) quotes.  For example, to read the database file \texttt{set.mm},
verify all proofs, and exit the program, type (under Unix)
\begin{verbatim}
bash$ ./metamath 'read set.mm' 'verify proof *' exit
\end{verbatim}
Note that in Unix, any directory path with \texttt{/}'s must be
surrounded by quotes so Metamath will not interpret the \texttt{/} as a
command qualifier.  So if \texttt{set.mm} is in the \texttt{/tmp}
directory, use for the above example
\begin{verbatim}
bash$ ./metamath 'read "/tmp/set.mm"' 'verify proof *' exit
\end{verbatim}

For convenience, if the command-line has one argument and no spaces in
the argument, the command is implicitly assumed to be \texttt{read}.  In
this one special case, \texttt{/}'s are not interpreted as command
qualifiers, so you don't need quotes around a Unix file name.  Thus
\begin{verbatim}
bash$ ./metamath /tmp/set.mm
\end{verbatim}
and
\begin{verbatim}
bash$ ./metamath "read '/tmp/set.mm'"
\end{verbatim}
are equivalent.


\section{Controlling Metamath}

The Metamath program was first developed on a {\sc vax/vms} system, and
some aspects of its command line behavior reflect this heritage.
Hopefully you will find it reasonably user-friendly once you get used to
it.

Each command line is a sequence of English-like words separated by
spaces, as in \texttt{show settings}.  Command words are not case
sensitive, and only as many letters are needed as are necessary to
eliminate ambiguity; for example, \texttt{sh se} would work for the
command \texttt{show settings}.  In some cases arguments such as file
names, statement labels, or symbol names are required; these are
case-sensitive (although file names may not be on some operating
systems).

A command line is entered by typing it in then pressing the {\em return}
({\em enter}) key.  To find out what commands are available, type
\texttt{?} at the \texttt{MM>} prompt.  To find out the choices at any
point in a command, press {\em return} and you will be prompted for
them.  The default choice (the one selected if you just press {\em
return}) is shown in brackets (\texttt{<>}).

You may also type \texttt{?} in place of a command word to force
Metamath to tell you what the choices are.  The \texttt{?} method won't
work, though, if a non-keyword argument such as a file name is expected
at that point, because the program will think that \texttt{?} is the
value of the argument.

Some commands have one or more optional qualifiers which modify the
behavior of the command.  Qualifiers are preceded by a slash
(\texttt{/}), such as in \texttt{read set.mm / verify}.  Spaces are
optional around the \texttt{/}.  If you need to use a space or
slash in a command
argument, as in a Unix file name, put single or double quotes around the
command argument.

The \texttt{open log} command will save everything you see on the
screen and is useful to help you recover should something go wrong in a
proof, or if you want to document a bug.

If a command responds with more than a screenful, you will be
prompted to \texttt{<return> to continue, Q to quit, or S to scroll to
end}.  \texttt{Q} or \texttt{q} (not case-sensitive) will complete the
command internally but will suppress further output until the next
\texttt{MM>} prompt.  \texttt{s} will suppress further pausing until the
next \texttt{MM>} prompt.  After the first screen, you are also
presented with the choice of \texttt{b} to go back a screenful.  Note
that \texttt{b} may also be entered at the \texttt{MM>} prompt
immediately after a command to scroll back through the output of that
command.

A command line enclosed in quotes is executed by your operating system.
See Section~\ref{oscmd}.

{\em Warning:} Pressing {\sc ctrl-c} will abort the Metamath program
unconditionally.  This means any unsaved work will be lost.


\subsection{\texttt{exit} Command}\index{\texttt{exit} command}

Syntax:  \texttt{exit} [\texttt{/force}]

This command exits from Metamath.  If there have been changes to the
source with the \texttt{save proof} or \texttt{save new{\char`\_}proof}
commands, you will be given an opportunity to \texttt{write source} to
permanently save the changes.

In Proof Assistant\index{Proof Assistant} mode, the \texttt{exit} command will
return to the \verb/MM>/ prompt. If there were changes to the proof, you will
be given an opportunity to \texttt{save new{\char`\_}proof}.

The \texttt{quit} command is a synonym for \texttt{exit}.

Optional qualifier:
    \texttt{/force} - Do not prompt if changes were not saved.  This qualifier is
        useful in \texttt{submit} command files (Section~\ref{sbmt})
        to ensure predictable behavior.





\subsection{\texttt{open log} Command}\index{\texttt{open log} command}
Syntax:  \texttt{open log} {\em file-name}

This command will open a log file that will store everything you see on
the screen.  It is useful to help recovery from a mistake in a long Proof
Assistant session, or to document bugs.\index{Metamath!bugs}

The log file can be closed with \texttt{close log}.  It will automatically be
closed upon exiting Metamath.



\subsection{\texttt{close log} Command}\index{\texttt{close log} command}
Syntax:  \texttt{close log}

The \texttt{close log} command closes a log file if one is open.  See
also \texttt{open log}.




\subsection{\texttt{submit} Command}\index{\texttt{submit} command}\label{sbmt}
Syntax:  \texttt{submit} {\em filename}

This command causes further command lines to be taken from the specified
file.  Note that any line beginning with an exclamation point (\texttt{!}) is
treated as a comment (i.e.\ ignored).  Also note that the scrolling
of the screen output is continuous, so you may want to open a log file
(see \texttt{open log}) to record the results that fly by on the screen.
After the lines in the file are exhausted, Metamath returns to its
normal user interface mode.

The \texttt{submit} command can be called recursively (i.e. from inside
of a \texttt{submit} command file).


Optional command qualifier:

    \texttt{/silent} - suppresses the screen output but still
        records the output in a log file if one is open.


\subsection{\texttt{erase} Command}\index{\texttt{erase} command}
Syntax:  \texttt{erase}

This command will reset Metamath to its starting state, deleting any
data\-base that was \texttt{read} in.
 If there have been changes to the
source with the \texttt{save proof} or \texttt{save new{\char`\_}proof}
commands, you will be given an opportunity to \texttt{write source} to
permanently save the changes.



\subsection{\texttt{set echo} Command}\index{\texttt{set echo} command}
Syntax:  \texttt{set echo on} or \texttt{set echo off}

The \texttt{set echo on} command will cause command lines to be echoed with any
abbreviations expanded.  While learning the Metamath commands, this
feature will show you the exact command that your abbreviated input
corresponds to.



\subsection{\texttt{set scroll} Command}\index{\texttt{set scroll} command}
Syntax:  \texttt{set scroll prompted} or \texttt{set scroll continuous}

The Metamath command line interface starts off in the \texttt{prompted} mode,
which means that you will be prompted to continue or quit after each
full screen in a long listing.  In \texttt{continuous} mode, long listings will be
scrolled without pausing.

% LaTeX bug? (1) \texttt{\_} puts out different character than
% \texttt{{\char`\_}}
%  = \verb$_$  (2) \texttt{{\char`\_}} puts out garbage in \subsection
%  argument
\subsection{\texttt{set width} Command}\index{\texttt{set
width} command}
Syntax:  \texttt{set width} {\em number}

Metamath assumes the width of your screen is 79 characters (chosen
because the Command Prompt in Windows XP has a wrapping bug at column
80).  If your screen is wider or narrower, this command allows you to
change this default screen width.  A larger width is advantageous for
logging proofs to an output file to be printed on a wide printer.  A
smaller width may be necessary on some terminals; in this case, the
wrapping of the information messages may sometimes seem somewhat
unnatural, however.  In \LaTeX\index{latex@{\LaTeX}!characters per line},
there is normally a maximum of 61 characters per line with typewriter
font.  (The examples in this book were produced with 61 characters per
line.)

\subsection{\texttt{set height} Command}\index{\texttt{set
height} command}
Syntax:  \texttt{set height} {\em number}

Metamath assumes your screen height is 24 lines of characters.  If your
screen is taller or shorter, this command lets you to change the number
of lines at which the display pauses and prompts you to continue.

\subsection{\texttt{beep} Command}\index{\texttt{beep} command}

Syntax:  \texttt{beep}

This command will produce a beep.  By typing it ahead after a
long-running command has started, it will alert you that the command is
finished.  For convenience, \texttt{b} is an abbreviation for
\texttt{beep}.

Note:  If \texttt{b} is typed at the \texttt{MM>} prompt immediately
after the end of a multiple-page display paged with ``\texttt{Press
<return> for more}...'' prompts, then the \texttt{b} will back up to the
previous page rather than perform the \texttt{beep} command.
In that case you must type the unabbreviated \texttt{beep} form
of the command.

\subsection{\texttt{more} Command}\index{\texttt{more} command}

Syntax:  \texttt{more} {\em filename}

This command will display the contents of an {\sc ascii} file on your
screen.  (This command is provided for convenience but is not very
powerful.  See Section~\ref{oscmd} to invoke your operating system's
command to do this, such as the \texttt{more} command in Unix.)

\subsection{Operating System Commands}\index{operating system
command}\label{oscmd}

A line enclosed in single or double quotes will be executed by your
computer's operating system if it has a command line interface.  For
example, on a {\sc vax/vms} system,
\verb/MM> 'dir'/
will print disk directory contents.  Note that this feature will not
work on the Macintosh prior to Mac OS X, which does not have a command
line interface.

For your convenience, the trailing quote is optional.

\subsection{Size Limitations in Metamath}

In general, there are no fixed, predefined limits\index{Metamath!memory
limits} on how many labels, tokens\index{token}, statements, etc.\ that
you may have in a database file.  The Metamath program uses 32-bit
variables (64-bit on 64-bit CPUs) as indices for almost all internal
arrays, which are allocated dynamically as needed.



\section{Reading and Writing Files}

The following commands create new files:  the \texttt{open} commands;
the \texttt{write} commands; the \texttt{/html},
\texttt{/alt{\char`\_}html}, \texttt{/brief{\char`\_}html},
\texttt{/brief{\char`\_}alt{\char`\_}html} qualifiers of \texttt{show
statement}, and \texttt{midi}.  The following commands append to files
previously opened:  the \texttt{/tex} qualifier of \texttt{show proof}
and \texttt{show new{\char`\_}proof}; the \texttt{/tex} and
\texttt{/simple{\char`\_}tex} qualifiers of \texttt{show statement}; the
\texttt{close} commands; and all screen dialog between \texttt{open log}
and \texttt{close log}.

The commands that create new files will not overwrite an existing {\em
filename} but will rename the existing one to {\em
filename}\texttt{{\char`\~}1}.  An existing {\em
filename}\texttt{{\char`\~}1} is renamed {\em
filename}\texttt{{\char`\~}2}, etc.\ up to {\em
filename}\texttt{{\char`\~}9}.  An existing {\em
filename}\texttt{{\char`\~}9} is deleted.  This makes recovery from
mistakes easier but also will clutter up your directory, so occasionally
you may want to clean up (delete) these \texttt{{\char`\~}}$n$ files.


\subsection{\texttt{read} Command}\index{\texttt{read} command}
Syntax:  \texttt{read} {\em file-name} [\texttt{/verify}]

This command will read in a Metamath language source file and any included
files.  Normally it will be the first thing you do when entering Metamath.
Statement syntax is checked, but proof syntax is not checked.
Note that the file name may be enclosed in single or double quotes;
this is useful if the file name contains slashes, as might be the case
under Unix.

If you are getting an ``\texttt{?Expected VERIFY}'' error
when trying to read a Unix file name with slashes, you probably haven't
quoted it.\index{Unix file names}\index{file names!Unix}

If you are prompted for the file name (by pressing {\em return}
 after \texttt{read})
you should {\em not} put quotes around it, even if it is a Unix file name
with slashes.

Optional command qualifier:

    \texttt{/verify} - Verify all proofs as the database is read in.  This
         qualifier will slow down reading in the file.  See \texttt{verify
         proof} for more information on file error-checking.

See also \texttt{erase}.



\subsection{\texttt{write source} Command}\index{\texttt{write source} command}
Syntax:  \texttt{write source} {\em filename}
[\texttt{/rewrap}]
[\texttt{/split}]
% TeX doesn't handle this long line with tt text very well,
% so force a line break here.
[\texttt{/keep\_includes}] {\\}
[\texttt{/no\_versioning}]

This command will write the contents of a Metamath\index{database}
database into a file.\index{source file}

Optional command qualifiers:

\texttt{/rewrap} -
Reformats statements and comments according to the
convention used in the set.mm database.
It unwraps the
lines in the comment before each \texttt{\$a} and \texttt{\$p} statement,
then it
rewraps the line.  You should compare the output to the original
to make sure that the desired effect results; if not, go back to
the original source.  The wrapped line length honors the
\texttt{set width}
parameter currently in effect.  Note:  Text
enclosed in \texttt{<HTML>}...\texttt{</HTML>} tags is not modified by the
\texttt{/rewrap} qualifier.
Proofs themselves are not reformatted;
use \texttt{save proof * / compressed} to do that.
An isolated tilde (\~{}) is always kept on the same line as the following
symbol, so you can find all comment references to a symbol by
searching for \~{} followed by a space and that symbol
(this is useful for finding cross-references).
Incidentally, \texttt{save proof} also honors the \texttt{set width}
parameter currently in effect.

\texttt{/split} - Files included in the source using the expression
\$[ \textit{inclfile} \$] will be
written into separate files instead of being included in a single output
file.  The name of each separately written included file will be the
\textit{inclfile} argument of its inclusion command.

\texttt{/keep\_includes} - If a source file has includes but is written as a
single file by omitting \texttt{/split}, by default the included files will
be deleted (actually just renamed with a \char`\~1 suffix unless
\texttt{/no\_versioning} is specified) to prevent the possibly confusing
source duplication in both the output file and the included file.
The \texttt{/keep\_includes} qualifier will prevent this deletion.

\texttt{/no\_versioning} - Backup files suffixed with \char`\~1 are not created.


\section{Showing Status and Statements}



\subsection{\texttt{show settings} Command}\index{\texttt{show settings} command}
Syntax:  \texttt{show settings}

This command shows the state of various parameters.

\subsection{\texttt{show memory} Command}\index{\texttt{show memory} command}
Syntax:  \texttt{show memory}

This command shows the available memory left.  It is not meaningful
on most modern operating systems,
which have virtual memory.\index{Metamath!memory usage}


\subsection{\texttt{show labels} Command}\index{\texttt{show labels} command}
Syntax:  \texttt{show labels} {\em label-match} [\texttt{/all}]
   [\texttt{/linear}]

This command shows the labels of \texttt{\$a} and \texttt{\$p}
statements that match {\em label-match}.  A \verb$*$ in {label-match}
matches zero or more characters.  For example, \verb$*abc*def$ will match all
labels containing \verb$abc$ and ending with \verb$def$.

Optional command qualifiers:

   \texttt{/all} - Include matches for \texttt{\$e} and \texttt{\$f}
   statement labels.

   \texttt{/linear} - Display only one label per line.  This can be useful for
       building scripts in conjunction with the utilities under the
       \texttt{tools}\index{\texttt{tools} command} command.



\subsection{\texttt{show statement} Command}\index{\texttt{show statement} command}
Syntax:  \texttt{show statement} {\em label-match} [{\em qualifiers} (see below)]

This command provides information about a statement.  Only statements
that have labels (\texttt{\$f}\index{\texttt{\$f} statement},
\texttt{\$e}\index{\texttt{\$e} statement},
\texttt{\$a}\index{\texttt{\$a} statement}, and
\texttt{\$p}\index{\texttt{\$p} statement}) may be specified.
If {\em label-match}
contains wildcard (\verb$*$) characters, all matching statements will be
displayed in the order they occur in the database.

Optional qualifiers (only one qualifier at a time is allowed):

    \texttt{/comment} - This qualifier includes the comment that immediately
       precedes the statement.

    \texttt{/full} - Show complete information about each statement,
       and show all
       statements matching {\em label} (including \texttt{\$e}
       and \texttt{\$f} statements).

    \texttt{/tex} - This qualifier will write the statement information to the
       \LaTeX\ file previously opened with \texttt{open tex}.  See
       Section~\ref{texout}.

    \texttt{/simple{\char`\_}tex} - The same as \texttt{/tex}, except that
       \LaTeX\ macros are not used for formatting equations, allowing easier
       manual edits of the output for slide presentations, etc.

    \texttt{/html}\index{html generation@{\sc html} generation},
       \texttt{/alt{\char`\_}html}, \texttt{/brief{\char`\_}html},
       \texttt{/brief{\char`\_}alt{\char`\_}html} -
       These qualifiers invoke a special mode of
       \texttt{show statement} that
       creates a web page for the statement.  They may not be used with
       any other qualifier.  See Section~\ref{htmlout} or
       \texttt{help html} in the program.


\subsection{\texttt{search} Command}\index{\texttt{search} command}
Syntax:  search {\em label-match}
\texttt{"}{\em symbol-match}{\tt}" [\texttt{/all}] [\texttt{/comments}]
[\texttt{/join}]

This command searches all \texttt{\$a} and \texttt{\$p} statements
matching {\em label-match} for occurrences of {\em symbol-match}.  A
\verb@*@ in {\em label-match} matches any label character.  A \verb@$*@
in {\em symbol-match} matches any sequence of symbols.  The symbols in
{\em symbol-match} must be separated by white space.  The quotes
surrounding {\em symbol-match} may be single or double quotes.  For
example, \texttt{search b}\verb@* "-> $* ch"@ will list all statements
whose labels begin with \texttt{b} and contain the symbols \verb@->@ and
\texttt{ch} surrounding any symbol sequence (including no symbol
sequence).  The wildcards \texttt{?} and \texttt{\$?} are also available
to match individual characters in labels and symbols respectively; see
\texttt{help search} in the Metamath program for details on their usage.

Optional command qualifiers:

    \texttt{/all} - Also search \texttt{\$e} and \texttt{\$f} statements.

    \texttt{/comments} - Search the comment that immediately precedes each
        label-matched statement for {\em symbol-match}.  In this case
        {\em symbol-match} is an arbitrary, non-case-sensitive character
        string.  Quotes around {\em symbol-match} are optional if there
        is no ambiguity.

    \texttt{/join} - In the case of a \texttt{\$a} or \texttt{\$p} statement,
	prepend its \texttt{\$e}
	hypotheses for searching. The
	\texttt{/join} qualifier has no effect in \texttt{/comments} mode.

\section{Displaying and Verifying Proofs}


\subsection{\texttt{show proof} Command}\index{\texttt{show proof} command}
Syntax:  \texttt{show proof} {\em label-match} [{\em qualifiers} (see below)]

This command displays the proof of the specified
\texttt{\$p}\index{\texttt{\$p} statement} statement in various formats.
The {\em label-match} may contain wildcard (\verb@$*@) characters to match
multiple statements.  Without any qualifiers, only the logical steps
will be shown (i.e.\ syntax construction steps will be omitted), in an
indented format.

Most of the time, you will use
    \texttt{show proof} {\em label}
to see just the proof steps corresponding to logical inferences.

Optional command qualifiers:

    \texttt{/essential} - The proof tree is trimmed of all
        \texttt{\$f}\index{\texttt{\$f} statement} hypotheses before
        being displayed.  (This is the default, and it is redundant to
        specify it.)

    \texttt{/all} - the proof tree is not trimmed of all \texttt{\$f} hypotheses before
        being displayed.  \texttt{/essential} and \texttt{/all} are mutually exclusive.

    \texttt{/from{\char`\_}step} {\em step} - The display starts at the specified
        step.  If
        this qualifier is omitted, the display starts at the first step.

    \texttt{/to{\char`\_}step} {\em step} - The display ends at the specified
        step.  If this
        qualifier is omitted, the display ends at the last step.

    \texttt{/tree{\char`\_}depth} {\em number} - Only
         steps at less than the specified proof
        tree depth are displayed.  Sometimes useful for obtaining an overview of
        the proof.

    \texttt{/reverse} - The steps are displayed in reverse order.

    \texttt{/renumber} - When used with \texttt{/essential}, the steps are renumbered
        to correspond only to the essential steps.

    \texttt{/tex} - The proof is converted to \LaTeX\ and\index{latex@{\LaTeX}}
        stored in the file opened
        with \texttt{open tex}.  See Section~\ref{texout} or
        \texttt{help tex} in the program.

    \texttt{/lemmon} - The proof is displayed in a non-indented format known
        as Lemmon style, with explicit previous step number references.
        If this qualifier is omitted, steps are indented in a tree format.

    \texttt{/start{\char`\_}column} {\em number} - Overrides the default column
        (16)
        at which the formula display starts in a Lemmon-style display.  May be
        used only in conjunction with \texttt{/lemmon}.

    \texttt{/normal} - The proof is displayed in normal format suitable for
        inclusion in a Metamath source file.  May not be used with any other
        qualifier.

    \texttt{/compressed} - The proof is displayed in compressed format
        suitable for inclusion in a Metamath source file.  May not be used with
        any other qualifier.

    \texttt{/statement{\char`\_}summary} - Summarizes all statements (like a
        brief \texttt{show statement})
        used by the proof.  It may not be used with any other qualifier
        except \texttt{/essential}.

    \texttt{/detailed{\char`\_}step} {\em step} - Shows the details of what is
        happening at
        a specific proof step.  May not be used with any other qualifier.
        The {\em step} is the step number shown when displaying a
        proof without the \texttt{/renumber} qualifier.


\subsection{\texttt{show usage} Command}\index{\texttt{show usage} command}
Syntax:  \texttt{show usage} {\em label-match} [\texttt{/recursive}]

This command lists the statements whose proofs make direct reference to
the statement specified.

Optional command qualifier:

    \texttt{/recursive} - Also include statements whose proofs ultimately
        depend on the statement specified.



\subsection{\texttt{show trace\_back} Command}\index{\texttt{show
       trace{\char`\_}back} command}
Syntax:  \texttt{show trace{\char`\_}back} {\em label-match} [\texttt{/essential}] [\texttt{/axioms}]
    [\texttt{/tree}] [\texttt{/depth} {\em number}]

This command lists all statements that the proof of the \texttt{\$p}
statement(s) specified by {\em label-match} depends on.

Optional command qualifiers:

    \texttt{/essential} - Restrict the trace-back to \texttt{\$e}
        \index{\texttt{\$e} statement} hypotheses of proof trees.

    \texttt{/axioms} - List only the axioms that the proof ultimately depends on.

    \texttt{/tree} - Display the trace-back in an indented tree format.

    \texttt{/depth} {\em number} - Restrict the \texttt{/tree} trace-back to the
        specified indentation depth.

    \texttt{/count{\char`\_}steps} - Count the number of steps the proof has
       all the way back to axioms.  If \texttt{/essential} is specified,
       expansions of variable-type hypotheses (syntax constructions) are not counted.

\subsection{\texttt{verify proof} Command}\index{\texttt{verify proof} command}
Syntax:  \texttt{verify proof} {\em label-match} [\texttt{/syntax{\char`\_}only}]

This command verifies the proofs of the specified statements.  {\em
label-match} may contain wild card characters (\texttt{*}) to verify
more than one proof; for example \verb/*abc*def/ will match all labels
containing \texttt{abc} and ending with \texttt{def}.
The command \texttt{verify proof *} will verify all proofs in the database.

Optional command qualifier:

    \texttt{/syntax{\char`\_}only} - This qualifier will perform a check of syntax
        and RPN
        stack violations only.  It will not verify that the proof is
        correct.  This qualifier is useful for quickly determining which
        proofs are incomplete (i.e.\ are under development and have \texttt{?}'s
        in them).

{\em Note:} \texttt{read}, followed by \texttt{verify proof *}, will ensure
 the database is free
from errors in the Metamath language but will not check the markup notation
in comments.
You can also check the markup notation by running \texttt{verify markup *}
as discussed in Section~\ref{verifymarkup}; see also the discussion
on generating {\sc HTML} in Section~\ref{htmlout}.

\subsection{\texttt{verify markup} Command}\index{\texttt{verify markup} command}\label{verifymarkup}
Syntax:  \texttt{verify markup} {\em label-match}
[\texttt{/date{\char`\_}skip}]
[\texttt{/top{\char`\_}date{\char`\_}skip}] {\\}
[\texttt{/file{\char`\_}skip}]
[\texttt{/verbose}]

This command checks comment markup and other informal conventions we have
adopted.  It error-checks the latexdef, htmldef, and althtmldef statements
in the \texttt{\$t} statement of a Metamath source file.\index{error checking}
It error-checks any \texttt{`}...\texttt{`},
\texttt{\char`\~}~\textit{label},
and bibliographic markups in statement descriptions.
It checks that
\texttt{\$p} and \texttt{\$a} statements
have the same content when their labels start with
``ax'' and ``ax-'' respectively but are otherwise identical, for example
ax4 and ax-4.
It also verifies the date consistency of ``(Contributed by...),''
``(Revised by...),'' and ``(Proof shortened by...)'' tags in the comment
above each \texttt{\$a} and \texttt{\$p} statement.

Optional command qualifiers:

    \texttt{/date{\char`\_}skip} - This qualifier will
        skip date consistency checking,
        which is usually not required for databases other than
	\texttt{set.mm}.

    \texttt{/top{\char`\_}date{\char`\_}skip} - This qualifier will check date consistency except
        that the version date at the top of the database file will not
        be checked.  Only one of
        \texttt{/date{\char`\_}skip} and
        \texttt{/top{\char`\_}date{\char`\_}skip} may be
        specified.

    \texttt{/file{\char`\_}skip} - This qualifier will skip checks that require
        external files to be present, such as checking GIF existence and
        bibliographic links to mmset.html or equivalent.  It is useful
        for doing a quick check from a directory without these files.

    \texttt{/verbose} - Provides more information.  Currently it provides a list
        of axXXX vs. ax-XXX matches.

\subsection{\texttt{save proof} Command}\index{\texttt{save proof} command}
Syntax:  \texttt{save proof} {\em label-match} [\texttt{/normal}]
   [\texttt{/compressed}]

The \texttt{save proof} command will reformat a proof in one of two formats and
replace the existing proof in the source buffer\index{source
buffer}.  It is useful for
converting between proof formats.  Note that a proof will not be
permanently saved until a \texttt{write source} command is issued.

Optional command qualifiers:

    \texttt{/normal} - The proof is saved in the normal format (i.e., as a
        sequence
        of labels, which is the defined format of the basic Metamath
        language).\index{basic language}  This is the default format that
        is used if a qualifier
        is omitted.

    \texttt{/compressed} - The proof is saved in the compressed format which
        reduces storage requirements for a database.
        See Appendix~\ref{compressed}.




\section{Creating Proofs}\label{pfcommands}\index{Proof Assistant}

Before using the Proof Assistant, you must add a \texttt{\$p} to your
source file (using a text editor) containing the statement you want to
prove.  Its proof should consist of a single \texttt{?}, meaning
``unknown step.''  Example:
\begin{verbatim}
equid $p x = x $= ? $.
\end{verbatim}

To enter the Proof assistant, type \texttt{prove} {\em label}, e.g.
\texttt{prove equid}.  Metamath will respond with the \texttt{MM-PA>}
prompt.

Proofs are created working backwards from the statement being proved,
primarily using a series of \texttt{assign} commands.  A proof is
complete when all steps are assigned to statements and all steps are
unified and completely known.  During the creation of a proof, Metamath
will allow only operations that are legal based on what is known up to
that point.  For example, it will not allow an \texttt{assign} of a
statement that cannot be unified with the unknown proof step being
assigned.

{\em Important:}
The Proof Assistant is
{\em not} a tool to help you discover proofs.  It is just a tool to help
you add them to the database.  For a tutorial read
Section~\ref{frstprf}.
To practice using the Proof Assistant, you may
want to \texttt{prove} an existing theorem, then delete all steps with
\texttt{delete all}, then re-create it with the Proof Assistant while
looking at its proof display (before deletion).
You might want to figure out your first few proofs completely
and write them down by hand, before using the Proof Assistant, though
not everyone finds that effective.

{\em Important:}
The \texttt{undo} command if very helpful when entering a proof, because
it allows you to undo a previously-entered step.
In addition, we suggest that you
keep track of your work with a log file (\texttt{open
log}) and save it frequently (\texttt{save new{\char`\_}proof},
\texttt{write source}).
You can use \texttt{delete} to reverse an \texttt{assign}.
You can also do \texttt{delete floating{\char`\_}hypotheses}, then
\texttt{initialize all}, then \texttt{unify all /interactive} to
reinitialize bad unifications made accidentally or by bad
\texttt{assign}s.  You cannot reverse a \texttt{delete} except by
a relevant \texttt{undo} or using
\texttt{exit /force} then reentering the Proof Assistant to recover from
the last \texttt{save new{\char`\_}proof}.

The following commands are available in the Proof Assistant (at the
\texttt{MM-PA>} prompt) to help you create your proof.  See the
individual commands for more detail.

\begin{itemize}
\item[]
    \texttt{show new{\char`\_}proof} [\texttt{/all},...] - Displays the
        proof in progress.  You will use this command a lot; see \texttt{help
        show new{\char`\_}proof} to become familiar with its qualifiers.  The
        qualifiers \texttt{/unknown} and \texttt{/not{\char`\_}unified} are
        useful for seeing the work remaining to be done.  The combination
        \texttt{/all/unknown} is useful for identifying dummy variables that must be
        assigned, or attempts to use illegal syntax, when \texttt{improve all}
        is unable to complete the syntax constructions.  Unknown variables are
        shown as \texttt{\$1}, \texttt{\$2},...
\item[]
    \texttt{assign} {\em step} {\em label} - Assigns an unknown {\em step}
        number with the statement
        specified by {\em label}.
\item[]
    \texttt{let variable} {\em variable}
        \texttt{= "}{\em symbol sequence}\texttt{"}
          - Forces a symbol
        sequence to replace an unknown variable (such as \texttt{\$1}) in a proof.
        It is useful
        for helping difficult unifications, and it is necessary when you have
        dummy variables that eventually must be assigned a name.
\item[]
    \texttt{let step} {\em step} \texttt{= "}{\em symbol sequence}\texttt{"} -
          Forces a symbol sequence
        to replace the contents of a proof step, provided it can be
        unified with the existing step contents.  (I rarely use this.)
\item[]
    \texttt{unify step} {\em step} (or \texttt{unify all}) - Unifies
        the source and target of
        a step.  If you specify a specific step, you will be prompted
        to select among the unifications that are possible.  If you
        specify \texttt{all}, all steps with unique unifications, but only
        those steps, will be
        unified.  \texttt{unify all /interactive} goes through all non-unified
        steps.
\item[]
    \texttt{initialize} {\em step} (or \texttt{all}) - De-unifies the target and source of
        a step (or all steps), as well as the hypotheses of the source,
        and makes all variables in the source unknown.  Useful to recover from
        an \texttt{assign} or \texttt{let} mistake that
        resulted in incorrect unifications.
\item[]
    \texttt{delete} {\em step} (or \texttt{all} or \texttt{floating{\char`\_}hypotheses}) -
        Deletes the specified
        step(s).  \texttt{delete floating{\char`\_}hypotheses}, then \texttt{initialize all}, then
        \texttt{unify all /interactive} is useful for recovering from mistakes
        where incorrect unifications assigned wrong math symbol strings to
        variables.
\item[]
    \texttt{improve} {\em step} (or \texttt{all}) -
      Automatically creates a proof for steps (with no unknown variables)
      whose proof requires no statements with \texttt{\$e} hypotheses.  Useful
      for filling in proofs of \texttt{\$f} hypotheses.  The \texttt{/depth}
      qualifier will also try statements whose \texttt{\$e} hypotheses contain
      no new variables.  {\em Warning:} Save your work (with \texttt{save
      new{\char`\_}proof} then \texttt{write source}) before using
      \texttt{/depth = 2} or greater, since the search time grows
      exponentially and may never terminate in a reasonable time, and you
      cannot interrupt the search.  I have found that it is rare for
      \texttt{/depth = 3} or greater to be useful.
 \item[]
    \texttt{save new{\char`\_}proof} - Saves the proof in progress in the program's
        internal database buffer.  To save it permanently into the database file,
        use \texttt{write source} after
        \texttt{save new{\char`\_}proof}.  To revert to the last
        \texttt{save new{\char`\_}proof},
        \texttt{exit /force} from the Proof Assistant then re-enter the Proof
        Assistant.
 \item[]
    \texttt{match step} {\em step} (or \texttt{match all}) - Shows what
        statements are
        possibilities for the \texttt{assign} statement. (This command
        is not very
        useful in its present form and hopefully will be improved
        eventually.  In the meantime, use the \texttt{search} statement for
        candidates matching specific math token combinations.)
 \item[]
 \texttt{minimize{\char`\_}with}\index{\texttt{minimize{\char`\_}with} command}
% 3/10/07 Note: line-breaking the above results in duplicate index entries
     - After a proof is complete, this command will attempt
        to match other database theorems to the proof to see if the proof
        size can be reduced as a result.  See \texttt{help
        minimize{\char`\_}with} in the
        Metamath program for its usage.
 \item[]
 \texttt{undo}\index{\texttt{undo} command}
    - Undo the effect of a proof-changing command (all but the
      \texttt{show} and \texttt{save} commands above).
 \item[]
 \texttt{redo}\index{\texttt{redo} command}
    - Reverse the previous \texttt{undo}.
\end{itemize}

The following commands set parameters that may be relevant to your proof.
Consult the individual \texttt{help set}... commands.
\begin{itemize}
   \item[] \texttt{set unification{\char`\_}timeout}
 \item[]
    \texttt{set search{\char`\_}limit}
  \item[]
    \texttt{set empty{\char`\_}substitution} - note that default is \texttt{off}
\end{itemize}

Type \texttt{exit} to exit the \texttt{MM-PA>}
 prompt and get back to the \texttt{MM>} prompt.
Another \texttt{exit} will then get you out of Metamath.



\subsection{\texttt{prove} Command}\index{\texttt{prove} command}
Syntax:  \texttt{prove} {\em label}

This command will enter the Proof Assistant\index{Proof Assistant}, which will
allow you to create or edit the proof of the specified statement.
The command-line prompt will change from \texttt{MM>} to \texttt{MM-PA>}.

Note:  In the present version (0.177) of
Metamath\index{Metamath!limitations of version 0.177}, the Proof
Assistant does not verify that \texttt{\$d}\index{\texttt{\$d}
statement} restrictions are met as a proof is being built.  After you
have completed a proof, you should type \texttt{save new{\char`\_}proof}
followed by \texttt{verify proof} {\em label} (where {\em label} is the
statement you are proving with the \texttt{prove} command) to verify the
\texttt{\$d} restrictions.

See also: \texttt{exit}

\subsection{\texttt{set unification\_timeout} Command}\index{\texttt{set
unification{\char`\_}timeout} command}
Syntax:  \texttt{set unification{\char`\_}timeout} {\em number}

(This command is available outside the Proof Assistant but affects the
Proof Assistant\index{Proof Assistant} only.)

Sometimes the Proof Assistant will inform you that a unification
time-out occurred.  This may happen when you try to \texttt{unify}
formulas with many temporary variables\index{temporary variable}
(\texttt{\$1}, \texttt{\$2}, etc.), since the time to compute all possible
unifications may grow exponentially with the number of variables.  If
you want Metamath to try harder (and you're willing to wait longer) you
may increase this parameter.  \texttt{show settings} will show you the
current value.

\subsection{\texttt{set empty\_substitution} Command}\index{\texttt{set
empty{\char`\_}substitution} command}
% These long names can't break well in narrow mode, and even "sloppy"
% is not enough. Work around this by not demanding justification.
\begin{flushleft}
Syntax:  \texttt{set empty{\char`\_}substitution on} or \texttt{set
empty{\char`\_}substitution off}
\end{flushleft}

(This command is available outside the Proof Assistant but affects the
Proof Assistant\index{Proof Assistant} only.)

The Metamath language allows variables to be
substituted\index{substitution!variable}\index{variable substitution}
with empty symbol sequences\index{empty substitution}.  However, in many
formal systems\index{formal system} this will never happen in a valid
proof.  Allowing for this possibility increases the likelihood of
ambiguous unifications\index{ambiguous
unification}\index{unification!ambiguous} during proof creation.
The default is that
empty substitutions are not allowed; for formal systems requiring them,
you must \texttt{set empty{\char`\_}substitution on}.
(An example where this must be \texttt{on}
would be a system that implements a Deduction Rule and in
which deductions from empty assumption lists would be permissible.  The
MIU-system\index{MIU-system} described in Appendix~\ref{MIU} is another
example.)
Note that empty substitutions are
always permissible in proof verification (VERIFY PROOF...) outside the
Proof Assistant.  (See the MIU system in the Metamath book for an example
of a system needing empty substitutions; another example would be a
system that implements a Deduction Rule and in which deductions from
empty assumption lists would be permissible.)

It is better to leave this \texttt{off} when working with \texttt{set.mm}.
Note that this command does not affect the way proofs are verified with
the \texttt{verify proof} command.  Outside of the Proof Assistant,
substitution of empty sequences for math symbols is always allowed.

\subsection{\texttt{set search\_limit} Command}\index{\texttt{set
search{\char`\_}limit} command} Syntax:  \texttt{set search{\char`\_}limit} {\em
number}

(This command is available outside the Proof Assistant but affects the
Proof Assistant\index{Proof Assistant} only.)

This command sets a parameter that determines when the \texttt{improve} command
in Proof Assistant mode gives up.  If you want \texttt{improve} to search harder,
you may increase it.  The \texttt{show settings} command tells you its current
value.


\subsection{\texttt{show new\_proof} Command}\index{\texttt{show
new{\char`\_}proof} command}
Syntax:  \texttt{show new{\char`\_}proof} [{\em
qualifiers} (see below)]

This command (available only in Proof Assistant mode) displays the proof
in progress.  It is identical to the \texttt{show proof} command, except that
there is no statement argument (since it is the statement being proved) and
the following qualifiers are not available:

    \texttt{/statement{\char`\_}summary}

    \texttt{/detailed{\char`\_}step}

Also, the following additional qualifiers are available:

    \texttt{/unknown} - Shows only steps that have no statement assigned.

    \texttt{/not{\char`\_}unified} - Shows only steps that have not been unified.

Note that \texttt{/essential}, \texttt{/depth}, \texttt{/unknown}, and
\texttt{/not{\char`\_}unified} may be
used in any combination; each of them effectively filters out additional
steps from the proof display.

See also:  \texttt{show proof}






\subsection{\texttt{assign} Command}\index{\texttt{assign} command}
Syntax:   \texttt{assign} {\em step} {\em label} [\texttt{/no{\char`\_}unify}]

   and:   \texttt{assign first} {\em label}

   and:   \texttt{assign last} {\em label}


This command, available in the Proof Assistant only, assigns an unknown
step (one with \texttt{?} in the \texttt{show new{\char`\_}proof}
listing) with the statement specified by {\em label}.  The assignment
will not be allowed if the statement cannot be unified with the step.

If \texttt{last} is specified instead of {\em step} number, the last
step that is shown by \texttt{show new{\char`\_}proof /unknown} will be
used.  This can be useful for building a proof with a command file (see
\texttt{help submit}).  It also makes building proofs faster when you know
the assignment for the last step.

If \texttt{first} is specified instead of {\em step} number, the first
step that is shown by \texttt{show new{\char`\_}proof /unknown} will be
used.

If {\em step} is zero or negative, the -{\em step}th from last unknown
step, as shown by \texttt{show new{\char`\_}proof /unknown}, will be
used.  \texttt{assign -1} {\em label} will assign the penultimate
unknown step, \texttt{assign -2} {\em label} the antepenultimate, and
\texttt{assign 0} {\em label} is the same as \texttt{assign last} {\em
label}.

Optional command qualifier:

    \texttt{/no{\char`\_}unify} - do not prompt user to select a unification if there is
        more than one possibility.  This is useful for noninteractive
        command files.  Later, the user can \texttt{unify all /interactive}.
        (The assignment will still be automatically unified if there is only
        one possibility and will be refused if unification is not possible.)



\subsection{\texttt{match} Command}\index{\texttt{match} command}
Syntax:  \texttt{match step} {\em step} [\texttt{/max{\char`\_}essential{\char`\_}hyp}
{\em number}]

    and:  \texttt{match all} [\texttt{/essential}]
          [\texttt{/max{\char`\_}essential{\char`\_}hyp} {\em number}]

This command, available in the Proof Assistant only, shows what
statements can be unified with the specified step(s).  {\em Note:} In
its current form, this command is not very useful because of the large
number of matches it reports.
It may be enhanced in the future.  In the meantime, the \texttt{search}
command can often provide finer control over locating theorems of interest.

Optional command qualifiers:

    \texttt{/max{\char`\_}essential{\char`\_}hyp} {\em number} - filters out
        of the list any statements
        with more than the specified number of
        \texttt{\$e}\index{\texttt{\$e} statement} hypotheses.

    \texttt{/essential{\char`\_}only} - in the \texttt{match all} statement, only
        the steps that
        would be listed in the \texttt{show new{\char`\_}proof /essential} display are
        matched.



\subsection{\texttt{let} Command}\index{\texttt{let} command}
Syntax: \texttt{let variable} {\em variable} = \verb/"/{\em symbol-sequence}\verb/"/

  and: \texttt{let step} {\em step} = \verb/"/{\em symbol-sequence}\verb/"/

These commands, available in the Proof Assistant\index{Proof Assistant}
only, assign a temporary variable\index{temporary variable} or unknown
step with a specific symbol sequence.  They are useful in the middle of
creating a proof, when you know what should be in the proof step but the
unification algorithm doesn't yet have enough information to completely
specify the temporary variables.  A ``temporary variable'' is one that
has the form \texttt{\$}{\em nn} in the proof display, such as
\texttt{\$1}, \texttt{\$2}, etc.  The {\em symbol-sequence} may contain
other unknown variables if desired.  Examples:

    \verb/let variable $32 = "A = B"/

    \verb/let variable $32 = "A = $35"/

    \verb/let step 10 = '|- x = x'/

    \verb/let step -2 = "|- ( $7 = ph )"/

Any symbol sequence will be accepted for the \texttt{let variable}
command.  Only those symbol sequences that can be unified with the step
will be accepted for \texttt{let step}.

The \texttt{let} commands ``zap'' the proof with information that can
only be verified when the proof is built up further.  If you make an
error, the command sequence \texttt{delete
floating{\char`\_}hypotheses}, \texttt{initialize all}, and
\texttt{unify all /interactive} will undo a bad \texttt{let} assignment.

If {\em step} is zero or negative, the -{\em step}th from last unknown
step, as shown by \texttt{show new{\char`\_}proof /unknown}, will be
used.  The command \texttt{let step 0} = \verb/"/{\em
symbol-sequence}\verb/"/ will use the last unknown step, \texttt{let
step -1} = \verb/"/{\em symbol-sequence}\verb/"/ the penultimate, etc.
If {\em step} is positive, \texttt{let step} may be used to assign known
(in the sense of having previously been assigned a label with
\texttt{assign}) as well as unknown steps.

Either single or double quotes can surround the {\em symbol-sequence} as
long as they are different from any quotes inside a {\em
symbol-sequence}.  If {\em symbol-sequence} contains both kinds of
quotes, see the instructions at the end of \texttt{help let} in the
Metamath program.


\subsection{\texttt{unify} Command}\index{\texttt{unify} command}
Syntax:  \texttt{unify step} {\em step}

      and:   \texttt{unify all} [\texttt{/interactive}]

These commands, available in the Proof Assistant only, unify the source
and target of the specified step(s). If you specify a specific step, you
will be prompted to select among the unifications that are possible.  If
you specify \texttt{all}, only those steps with unique unifications will be
unified.

Optional command qualifier for \texttt{unify all}:

    \texttt{/interactive} - You will be prompted to select among the
        unifications
        that are possible for any steps that do not have unique
        unifications.  (Otherwise \texttt{unify all} will bypass these.)

See also \texttt{set unification{\char`\_}timeout}.  The default is
100000, but increasing it to 1000000 can help difficult cases.  Manually
assigning some or all of the unknown variables with the \texttt{let
variable} command also helps difficult cases.



\subsection{\texttt{initialize} Command}\index{\texttt{initialize} command}
Syntax:  \texttt{initialize step} {\em step}

    and: \texttt{initialize all}

These commands, available in the Proof Assistant\index{Proof Assistant}
only, ``de-unify'' the target and source of a step (or all steps), as
well as the hypotheses of the source, and makes all variables in the
source and the source's hypotheses unknown.  This command is useful to
help recover from incorrect unifications that resulted from an incorrect
\texttt{assign}, \texttt{let}, or unification choice.  Part or all of
the command sequence \texttt{delete floating{\char`\_}hypotheses},
\texttt{initialize all}, and \texttt{unify all /interactive} will recover
from incorrect unifications.

See also:  \texttt{unify} and \texttt{delete}



\subsection{\texttt{delete} Command}\index{\texttt{delete} command}
Syntax:  \texttt{delete step} {\em step}

   and:      \texttt{delete all} -- {\em Warning: dangerous!}

   and:      \texttt{delete floating{\char`\_}hypotheses}

These commands are available in the Proof Assistant only.  The
\texttt{delete step} command deletes the proof tree section that
branches off of the specified step and makes the step become unknown.
\texttt{delete all} is equivalent to \texttt{delete step} {\em step}
where {\em step} is the last step in the proof (i.e.\ the beginning of
the proof tree).

In most cases the \texttt{undo} command is the best way to undo
a previous step.
An alternative is to salvage your last \texttt{save
new{\char`\_}proof} by exiting and reentering the Proof Assistant.
For this to work, keep a log file open to record your work
and to do \texttt{save new{\char`\_}proof} frequently, especially before
\texttt{delete}.

\texttt{delete floating{\char`\_}hypotheses} will delete all sections of
the proof that branch off of \texttt{\$f}\index{\texttt{\$f} statement}
statements.  It is sometimes useful to do this before an
\texttt{initialize} command to recover from an error.  Note that once a
proof step with a \texttt{\$f} hypothesis as the target is completely
known, the \texttt{improve} command can usually fill in the proof for
that step.  Unlike the deletion of logical steps, \texttt{delete
floating{\char`\_}hypotheses} is a relatively safe command that is
usually easy to recover from.



\subsection{\texttt{improve} Command}\index{\texttt{improve} command}
\label{improve}
Syntax:  \texttt{improve} {\em step} [\texttt{/depth} {\em number}]
                                               [\texttt{/no{\char`\_}distinct}]

   and:   \texttt{improve first} [\texttt{/depth} {\em number}]
                                              [\texttt{/no{\char`\_}distinct}]

   and:   \texttt{improve last} [\texttt{/depth} {\em number}]
                                              [\texttt{/no{\char`\_}distinct}]

   and:   \texttt{improve all} [\texttt{/depth} {\em number}]
                                              [\texttt{/no{\char`\_}distinct}]

These commands, available in the Proof Assistant\index{Proof Assistant}
only, try to find proofs automatically for unknown steps whose symbol
sequences are completely known.  They are primarily useful for filling in
proofs of \texttt{\$f}\index{\texttt{\$f} statement} hypotheses.  The
search will be restricted to statements having no
\texttt{\$e}\index{\texttt{\$e} statement} hypotheses.

\begin{sloppypar} % narrow
Note:  If memory is limited, \texttt{improve all} on a large proof may
overflow memory.  If you use \texttt{set unification{\char`\_}timeout 1}
before \texttt{improve all}, there will usually be sufficient
improvement to easily recover and completely \texttt{improve} the proof
later on a larger computer.  Warning:  Once memory has overflowed, there
is no recovery.  If in doubt, save the intermediate proof (\texttt{save
new{\char`\_}proof} then \texttt{write source}) before \texttt{improve
all}.
\end{sloppypar}

If \texttt{last} is specified instead of {\em step} number, the last
step that is shown by \texttt{show new{\char`\_}proof /unknown} will be
used.

If \texttt{first} is specified instead of {\em step} number, the first
step that is shown by \texttt{show new{\char`\_}proof /unknown} will be
used.

If {\em step} is zero or negative, the -{\em step}th from last unknown
step, as shown by \texttt{show new{\char`\_}proof /unknown}, will be
used.  \texttt{improve -1} will use the penultimate
unknown step, \texttt{improve -2} {\em label} the antepenultimate, and
\texttt{improve 0} is the same as \texttt{improve last}.

Optional command qualifier:

    \texttt{/depth} {\em number} - This qualifier will cause the search
        to include
        statements with \texttt{\$e} hypotheses (but no new variables in
        the \texttt{\$e}
        hypotheses), provided that the backtracking has not exceeded the
        specified depth. {\em Warning:}  Try \texttt{/depth 1},
        then \texttt{2}, then \texttt{3}, etc.
        in sequence because of possible exponential blowups.  Save your
        work before trying \texttt{/depth} greater than \texttt{1}!

    \texttt{/no{\char`\_}distinct} - Skip trial statements that have
        \texttt{\$d}\index{\texttt{\$d} statement} requirements.
        This qualifier will prevent assignments that might violate \texttt{\$d}
        requirements but it also could miss possible legal assignments.

See also: \texttt{set search{\char`\_}limit}

\subsection{\texttt{save new\_proof} Command}\index{\texttt{save
new{\char`\_}proof} command}
Syntax:  \texttt{save new{\char`\_}proof} {\em label} [\texttt{/normal}]
   [\texttt{/compressed}]

The \texttt{save new{\char`\_}proof} command is available in the Proof
Assistant only.  It saves the proof in progress in the source
buffer\index{source buffer}.  \texttt{save new{\char`\_}proof} may be
used to save a completed proof, or it may be used to save a proof in
progress in order to work on it later.  If an incomplete proof is saved,
any user assignments with \texttt{let step} or \texttt{let variable}
will be lost, as will any ambiguous unifications\index{ambiguous
unification}\index{unification!ambiguous} that were resolved manually.
To help make recovery easier, it can be helpful to \texttt{improve all}
before \texttt{save new{\char`\_}proof} so that the incomplete proof
will have as much information as possible.

Note that the proof will not be permanently saved until a \texttt{write
source} command is issued.

Optional command qualifiers:

    \texttt{/normal} - The proof is saved in the normal format (i.e., as a
        sequence of labels, which is the defined format of the basic Metamath
        language).\index{basic language}  This is the default format that
        is used if a qualifier is omitted.

    \texttt{/compressed} - The proof is saved in the compressed format, which
        reduces storage requirements for a database.
        (See Appendix~\ref{compressed}.)


\section{Creating \LaTeX\ Output}\label{texout}\index{latex@{\LaTeX}}

You can generate \LaTeX\ output given the
information in a database.
The database must already include the necessary typesetting information
(see section \ref{tcomment} for how to provide this information).

The \texttt{show statement} and \texttt{show proof} commands each have a
special \texttt{/tex} command qualifier that produces \LaTeX\ output.
(The \texttt{show statement} command also has the
\texttt{/simple{\char`\_}tex} qualifier for output that is easier to
edit by hand.)  Before you can use them, you must open a \LaTeX\ file to
which to send their output.  A typical complete session will use this
sequence of Metamath commands:

\begin{verbatim}
read set.mm
open tex example.tex
show statement a1i /tex
show proof a1i /all/lemmon/renumber/tex
show statement uneq2 /tex
show proof uneq2 /all/lemmon/renumber/tex
close tex
\end{verbatim}

See Section~\ref{mathcomments} for information on comment markup and
Appendix~\ref{ASCII} for information on how math symbol translation is
specified.

To format and print the \LaTeX\ source, you will need the \LaTeX\
program, which is standard on most Linux installations and available for
Windows.  On Linux, in order to create a {\sc pdf} file, you will
typically type at the shell prompt
\begin{verbatim}
$ pdflatex example.tex
\end{verbatim}

\subsection{\texttt{open tex} Command}\index{\texttt{open tex} command}
Syntax:  \texttt{open tex} {\em file-name} [\texttt{/no{\char`\_}header}]

This command opens a file for writing \LaTeX\
source\index{latex@{\LaTeX}} and writes a \LaTeX\ header to the file.
\LaTeX\ source can be written with the \texttt{show proof}, \texttt{show
new{\char`\_}proof}, and \texttt{show statement} commands using the
\texttt{/tex} qualifier.

The mapping to \LaTeX\ symbols is defined in a special comment
containing a \texttt{\$t} token, described in Appendix~\ref{ASCII}.

There is an optional command qualifier:

    \texttt{/no{\char`\_}header} - This qualifier prevents a standard
        \LaTeX\ header and trailer
        from being included with the output \LaTeX\ code.


\subsection{\texttt{close tex} Command}\index{\texttt{close tex} command}
Syntax:  \texttt{close tex}

This command writes a trailer to any \LaTeX\ file\index{latex@{\LaTeX}}
that was opened with \texttt{open tex} (unless
\texttt{/no{\char`\_}header} was used with \texttt{open tex}) and closes
the \LaTeX\ file.


\section{Creating {\sc HTML} Output}\label{htmlout}

You can generate {\sc html} web pages given the
information in a database.
The database must already include the necessary typesetting information
(see section \ref{tcomment} for how to provide this information).
The ability to produce {\sc html} web pages was added in Metamath version
0.07.30.

To create an {\sc html} output file(s) for \texttt{\$a} or \texttt{\$p}
statement(s), use
\begin{quote}
    \texttt{show statement} {\em label-match} \texttt{/html}
\end{quote}
The output file will be named {\em label-match}\texttt{.html}
for each match.  When {\em
label-match} has wildcard (\texttt{*}) characters, all statements with
matching labels will have {\sc html} files produced for them.  Also,
when {\em label-match} has a wildcard (\texttt{*}) character, two additional
files, \texttt{mmdefinitions.html} and \texttt{mmascii.html} will be
produced.  To produce {\em only} these two additional files, you can use
\texttt{?*}, which will not match any statement label, in place of {\em
label-match}.

There are three other qualifiers for \texttt{show statement} that also
generate {\sc HTML} code.  These are \texttt{/alt{\char`\_}html},
\texttt{/brief{\char`\_}html}, and
\texttt{/brief{\char`\_}alt{\char`\_}html}, and are described in the
next section.

The command
\begin{quote}
    \texttt{show statement} {\em label-match} \texttt{/alt{\char`\_}html}
\end{quote}
does the same as \texttt{show statement} {\em label-match} \texttt{/html},
except that the {\sc html} code for the symbols is taken from
\texttt{althtmldef} statements instead of \texttt{htmldef} statements in
the \texttt{\$t} comment.

The command
\begin{verbatim}
show statement * /brief_html
\end{verbatim}
invokes a special mode that just produces definition and theorem lists
accompanied by their symbol strings, in a format suitable for copying and
pasting into another web page (such as the tutorial pages on the
Metamath web site).

Finally, the command
\begin{verbatim}
show statement * /brief_alt_html
\end{verbatim}
does the same as \texttt{show statement * / brief{\char`\_}html}
for the alternate {\sc html}
symbol representation.

A statement's comment can include a special notation that provides a
certain amount of control over the {\sc HTML} version of the comment.  See
Section~\ref{mathcomments} (p.~\pageref{mathcomments}) for the comment
markup features.

The \texttt{write theorem{\char`\_}list} and \texttt{write bibliography}
commands, which are described below, provide as a side effect complete
error checking for all of the features described in this
section.\index{error checking}

\subsection{\texttt{write theorem\_list}
Command}\index{\texttt{write theorem{\char`\_}list} command}
Syntax:  \texttt{write theorem{\char`\_}list}
[\texttt{/theorems{\char`\_}per{\char`\_}page} {\em number}]

This command writes a list of all of the \texttt{\$a} and \texttt{\$p}
statements in the database into a web page file
 called \texttt{mmtheorems.html}.
When additional files are needed, they are called
\texttt{mmtheorems2.html}, \texttt{mmtheorems3.html}, etc.

Optional command qualifier:

    \texttt{/theorems{\char`\_}per{\char`\_}page} {\em number} -
 This qualifier specifies the number of statements to
        write per web page.  The default is 100.

{\em Note:} In version 0.177\index{Metamath!limitations of version
0.177} of Metamath, the ``Nearby theorems'' links on the individual
web pages presuppose 100 theorems per page when linking to the theorem
list pages.  Therefore the \texttt{/theorems{\char`\_}per{\char`\_}page}
qualifier, if it specifies a number other than 100, will cause the
individual web pages to be out of sync and should not be used to
generate the main theorem list for the web site.  This may be
fixed in a future version.


\subsection{\texttt{write bibliography}\label{wrbib}
Command}\index{\texttt{write bibliography} command}
Syntax:  \texttt{write bibliography} {\em filename}

This command reads an existing {\sc html} bibliographic cross-reference
file, normally called \texttt{mmbiblio.html}, and updates it per the
bibliographic links in the database comments.  The file is updated
between the {\sc html} comment lines \texttt{<!--
{\char`\#}START{\char`\#} -->} and \texttt{<!-- {\char`\#}END{\char`\#}
-->}.  The original input file is renamed to {\em
filename}\texttt{{\char`\~}1}.

A bibliographic reference is indicated with the reference name
in brackets, such as  \texttt{Theorem 3.1 of
[Monk] p.\ 22}.
See Section~\ref{htmlout} (p.~\pageref{htmlout}) for
syntax details.


\subsection{\texttt{write recent\_additions}
Command}\index{\texttt{write recent{\char`\_}additions} command}
Syntax:  \texttt{write recent{\char`\_}additions} {\em filename}
[\texttt{/limit} {\em number}]

This command reads an existing ``Recent Additions'' {\sc html} file,
normally called \texttt{mmrecent.html}, and updates it with the
descriptions of the most recently added theorems to the database.
 The file is updated between
the {\sc html} comment lines \texttt{<!-- {\char`\#}START{\char`\#} -->}
and \texttt{<!-- {\char`\#}END{\char`\#} -->}.  The original input file
is renamed to {\em filename}\texttt{{\char`\~}1}.

Optional command qualifier:

    \texttt{/limit} {\em number} -
 This qualifier specifies the number of most recent theorems to
   write to the output file.  The default is 100.


\section{Text File Utilities}

\subsection{\texttt{tools} Command}\index{\texttt{tools} command}
Syntax:  \texttt{tools}

This command invokes an easy-to-use, general purpose utility for
manipulating the contents of {\sc ascii} text files.  Upon typing
\texttt{tools}, the command-line prompt will change to \texttt{TOOLS>}
until you type \texttt{exit}.  The \texttt{tools} commands can be used
to perform simple, global edits on an input/output file,
such as making a character string substitution on each line, adding a
string to each line, and so on.  A typical use of this utility is
to build a \texttt{submit} input file to perform a common operation on a
list of statements obtained from \texttt{show label} or \texttt{show
usage}.

The actions of most of the \texttt{tools} commands can also be
performed with equivalent (and more powerful) Unix shell commands, and
some users may find those more efficient.  But for Windows users or
users not comfortable with Unix, \texttt{tools} provides an
easy-to-learn alternative that is adequate for most of the
script-building tasks needed to use the Metamath program effectively.

\subsection{\texttt{help} Command (in \texttt{tools})}
Syntax:  \texttt{help}

The \texttt{help} command lists the commands available in the
\texttt{tools} utility, along with a brief description.  Each command,
in turn, has its own help, such as \texttt{help add}.  As with
Metamath's \texttt{MM>} prompt, a complete command can be entered at
once, or just the command word can be typed, causing the program to
prompt for each argument.

\vskip 1ex
\noindent Line-by-line editing commands:

  \texttt{add} - Add a specified string to each line in a file.

  \texttt{clean} - Trim spaces and tabs on each line in a file; convert
         characters.

  \texttt{delete} - Delete a section of each line in a file.

  \texttt{insert} - Insert a string at a specified column in each line of
        a file.

  \texttt{substitute} - Make a simple substitution on each line of the file.

  \texttt{tag} - Like \texttt{add}, but restricted to a range of lines.

  \texttt{swap} - Swap the two halves of each line in a file.

\vskip 1ex
\noindent Other file-processing commands:

  \texttt{break} - Break up (tokenize) a file into a list of tokens (one per
        line).

  \texttt{build} - Build a file with multiple tokens per line from a list.

  \texttt{count} - Count the occurrences in a file of a specified string.

  \texttt{number} - Create a list of numbers.

  \texttt{parallel} - Put two files in parallel.

  \texttt{reverse} - Reverse the order of the lines in a file.

  \texttt{right} - Right-justify lines in a file (useful before sorting
         numbers).

%  \texttt{tag} - Tag edit updates in a program for revision control.

  \texttt{sort} - Sort the lines in a file with key starting at
         specified string.

  \texttt{match} - Extract lines containing (or not) a specified string.

  \texttt{unduplicate} - Eliminate duplicate occurrences of lines in a file.

  \texttt{duplicate} - Extract first occurrence of any line occurring
         more than

   \ \ \    once in a file, discarding lines occurring exactly once.

  \texttt{unique} - Extract lines occurring exactly once in a file.

  \texttt{type} (10 lines) - Display the first few lines in a file.
                  Similar to Unix \texttt{head}.

  \texttt{copy} - Similar to Unix \texttt{cat} but safe (same input
         and output file allowed).

  \texttt{submit} - Run a script containing \texttt{tools} commands.

\vskip 1ex

\noindent Note:
  \texttt{unduplicate}, \texttt{duplicate}, and \texttt{unique} also
 sort the lines as a side effect.


\subsection{Using \texttt{tools} to Build Metamath \texttt{submit}
Scripts}

The \texttt{break} command is typically used to break up a series of
statement labels, such as the output of Metamath's \texttt{show usage},
into one label per line.  The other \texttt{tools} commands can then be
used to add strings before and after each statement label to specify
commands to be performed on the statement.  The \texttt{parallel}
command is useful when a statement label must be mentioned more than
once on a line.

Very often a \texttt{submit} script for Metamath will require multiple
command lines for each statement being processed.  For example, you may
want to enter the Proof Assistant, \texttt{minimize{\char`\_}with} your
latest theorem, \texttt{save} the new proof, and \texttt{exit} the Proof
Assistant.  To accomplish this, you can build a file with these four
commands for each statement on a single line, separating each command
with a designated character such as \texttt{@}.  Then at the end you can
\texttt{substitute} each \texttt{@} with \texttt{{\char`\\}n} to break
up the lines into individual command lines (see \texttt{help
substitute}).


\subsection{Example of a \texttt{tools} Session}

To give you a quick feel for the \texttt{tools} utility, we show a
simple session where we create a file \texttt{n.txt} with 3 lines, add
strings before and after each line, and display the lines on the screen.
You can experiment with the various commands to gain experience with the
\texttt{tools} utility.

\begin{verbatim}
MM> tools
Entering the Text Tools utilities.
Type HELP for help, EXIT to exit.
TOOLS> number
Output file <n.tmp>? n.txt
First number <1>?
Last number <10>? 3
Increment <1>?
TOOLS> add
Input/output file? n.txt
String to add to beginning of each line <>? This is line
String to add to end of each line <>? .
The file n.txt has 3 lines; 3 were changed.
First change is on line 1:
This is line 1.
TOOLS> type n.txt
This is line 1.
This is line 2.
This is line 3.
TOOLS> exit
Exiting the Text Tools.
Type EXIT again to exit Metamath.
MM>
\end{verbatim}



\appendix
\chapter{Sample Representations}
\label{ASCII}

This Appendix provides a sample of {\sc ASCII} representations,
their corresponding traditional mathematical symbols,
and a discussion of their meanings
in the \texttt{set.mm} database.
These are provided in order of appearance.
This is only a partial list, and new definitions are routinely added.
A complete list is available at \url{http://metamath.org}.

These {\sc ASCII} representations, along
with information on how to display them,
are defined in the \texttt{set.mm} database file inside
a special comment called a \texttt{\$t} {\em
comment}\index{\texttt{\$t} comment} or {\em typesetting
comment.}\index{typesetting comment}
A typesetting comment
is indicated by the appearance of the
two-character string \texttt{\$t} at the beginning of the comment.
For more information,
see Section~\ref{tcomment}, p.~\pageref{tcomment}.

In the following table the ``{\sc ASCII}'' column shows the {\sc ASCII}
representation,
``Symbol'' shows the mathematical symbolic display
that corresponds to that {\sc ASCII} representation, ``Labels'' shows
the key label(s) that define the representation, and
``Description'' provides a description about the symbol.
As usual, ``iff'' is short for ``if and only if.''\index{iff}
In most cases the ``{\sc ASCII}'' column only shows
the key token, but it will sometimes show a sequence of tokens
if that is necessary for clarity.

{\setlength{\extrarowsep}{4pt} % Keep rows from being too close together
\begin{longtabu}   { @{} c c l X }
\textbf{ASCII} & \textbf{Symbol} & \textbf{Labels} & \textbf{Description} \\
\endhead
\texttt{|-} & $\vdash$ & &
  ``It is provable that...'' \\
\texttt{ph} & $\varphi$ & \texttt{wph} &
  The wff (boolean) variable phi,
  conventionally the first wff variable. \\
\texttt{ps} & $\psi$ & \texttt{wps} &
  The wff (boolean) variable psi,
  conventionally the second wff variable. \\
\texttt{ch} & $\chi$ & \texttt{wch} &
  The wff (boolean) variable chi,
  conventionally the third wff variable. \\
\texttt{-.} & $\lnot$ & \texttt{wn} &
  Logical not. E.g., if $\varphi$ is true, then $\lnot \varphi$ is false. \\
\texttt{->} & $\rightarrow$ & \texttt{wi} &
  Implies, also known as material implication.
  In classical logic the expression $\varphi \rightarrow \psi$ is true
  if either $\varphi$ is false or $\psi$ is true (or both), that is,
  $\varphi \rightarrow \psi$ has the same meaning as
  $\lnot \varphi \lor \psi$ (as proven in theorem \texttt{imor}). \\
\texttt{<->} & $\leftrightarrow$ &
  \hyperref[df-bi]{\texttt{df-bi}} &
  Biconditional (aka is-equals for boolean values).
  $\varphi \leftrightarrow \psi$ is true iff
  $\varphi$ and $\psi$ have the same value. \\
\texttt{\char`\\/} & $\lor$ &
  \makecell[tl]{{\hyperref[df-or]{\texttt{df-or}}}, \\
	         \hyperref[df-3or]{\texttt{df-3or}}} &
  Disjunction (logical ``or''). $\varphi \lor \psi$ is true iff
  $\varphi$, $\psi$, or both are true. \\
\texttt{/\char`\\} & $\land$ &
  \makecell[tl]{{\hyperref[df-an]{\texttt{df-an}}}, \\
                 \hyperref[df-3an]{\texttt{df-3an}}} &
  Conjunction (logical ``and''). $\varphi \land \psi$ is true iff
  both $\varphi$ and $\psi$ are true. \\
\texttt{A.} & $\forall$ &
  \texttt{wal} &
  For all; the wff $\forall x \varphi$ is true iff
  $\varphi$ is true for all values of $x$. \\
\texttt{E.} & $\exists$ &
  \hyperref[df-ex]{\texttt{df-ex}} &
  There exists; the wff
  $\exists x \varphi$ is true iff
  there is at least one $x$ where $\varphi$ is true. \\
\texttt{[ y / x ]} & $[ y / x ]$ &
  \hyperref[df-sb]{\texttt{df-sb}} &
  The wff $[ y / x ] \varphi$ produces
  the result when $y$ is properly substituted for $x$ in $\varphi$
  ($y$ replaces $x$).
  % This is elsb4
  % ( [ x / y ] z e. y <-> z e. x )
  For example,
  $[ x / y ] z \in y$ is the same as $z \in x$. \\
\texttt{E!} & $\exists !$ &
  \hyperref[df-eu]{\texttt{df-eu}} &
  There exists exactly one;
  $\exists ! x \varphi$ is true iff
  there is at least one $x$ where $\varphi$ is true. \\
\texttt{\{ y | phi \}}  & $ \{ y | \varphi \}$ &
  \hyperref[df-clab]{\texttt{df-clab}} &
  The class of all sets where $\varphi$ is true. \\
\texttt{=} & $ = $ &
  \hyperref[df-cleq]{\texttt{df-cleq}} &
  Class equality; $A = B$ iff $A$ equals $B$. \\
\texttt{e.} & $ \in $ &
  \hyperref[df-clel]{\texttt{df-clel}} &
  Class membership; $A \in B$ if $A$ is a member of $B$. \\
\texttt{{\char`\_}V} & {\rm V} &
  \hyperref[df-v]{\texttt{df-v}} &
  Class of all sets (not itself a set). \\
\texttt{C\_} & $ \subseteq $ &
  \hyperref[df-ss]{\texttt{df-ss}} &
  Subclass (subset); $A \subseteq B$ is true iff
  $A$ is a subclass of $B$. \\
\texttt{u.} & $ \cup $ &
  \hyperref[df-un]{\texttt{df-un}} &
  $A \cup B$ is the union of classes $A$ and $B$. \\
\texttt{i^i} & $ \cap $ &
  \hyperref[df-in]{\texttt{df-in}} &
  $A \cap B$ is the intersection of classes $A$ and $B$. \\
\texttt{\char`\\} & $ \setminus $ &
  \hyperref[df-dif]{\texttt{df-dif}} &
  $A \setminus B$ (set difference)
  is the class of all sets in $A$ except for those in $B$. \\
\texttt{(/)} & $ \varnothing $ &
  \hyperref[df-nul]{\texttt{df-nul}} &
  $ \varnothing $ is the empty set (aka null set). \\
\texttt{\char`\~P} & $ \cal P $ &
  \hyperref[df-pw]{\texttt{df-pw}} &
  Power class. \\
\texttt{<.\ A , B >.} & $\langle A , B \rangle$ &
  \hyperref[df-op]{\texttt{df-op}} &
  The ordered pair $\langle A , B \rangle$. \\
\texttt{( F ` A )} & $ ( F ` A ) $ &
  \hyperref[df-fv]{\texttt{df-fv}} &
  The value of function $F$ when applied to $A$. \\
\texttt{\_i} & $ i $ &
  \texttt{df-i} &
  The square root of negative one. \\
\texttt{x.} & $ \cdot $ &
  \texttt{df-mul} &
  Complex number multiplication; $2~\cdot~3~=~6$. \\
\texttt{CC} & $ \mathbb{C} $ &
  \texttt{df-c} &
  The set of complex numbers. \\
\texttt{RR} & $ \mathbb{R} $ &
  \texttt{df-r} &
  The set of real numbers. \\
\end{longtabu}
} % end of extrarowsep

\chapter{Compressed Proofs}
\label{compressed}\index{compressed proof}\index{proof!compressed}

The proofs in the \texttt{set.mm} set theory database are stored in compressed
format for efficiency.  Normally you needn't concern yourself with the
compressed format, since you can display it with the usual proof display tools
in the Metamath program (\texttt{show proof}\ldots) or convert it to the normal
RPN proof format described in Section~\ref{proof} (with \texttt{save proof}
{\em label} \texttt{/normal}).  However for sake of completeness we describe the
format here and show how it maps to the normal RPN proof format.

A compressed proof, located between \texttt{\$=} and \texttt{\$.}\ keywords, consists
of a left parenthesis, a sequence of statement labels, a right parenthesis,
and a sequence of upper-case letters \texttt{A} through \texttt{Z} (with optional
white space between them).  White space must surround the parentheses
and the labels.  The left parenthesis tells Metamath that a
compressed proof follows.  (A normal RPN proof consists of just a sequence of
labels, and a parenthesis is not a legal character in a label.)

The sequence of upper-case letters corresponds to a sequence of integers
with the following mapping.  Each integer corresponds to a proof step as
described later.
\begin{center}
  \texttt{A} = 1 \\
  \texttt{B} = 2 \\
   \ldots \\
  \texttt{T} = 20 \\
  \texttt{UA} = 21 \\
  \texttt{UB} = 22 \\
   \ldots \\
  \texttt{UT} = 40 \\
  \texttt{VA} = 41 \\
  \texttt{VB} = 42 \\
   \ldots \\
  \texttt{YT} = 120 \\
  \texttt{UUA} = 121 \\
   \ldots \\
  \texttt{YYT} = 620 \\
  \texttt{UUUA} = 621 \\
   etc.
\end{center}

In other words, \texttt{A} through \texttt{T} represent the
least-significant digit in base 20, and \texttt{U} through \texttt{Y}
represent zero or more most-significant digits in base 5, where the
digits start counting at 1 instead of the usual 0. With this scheme, we
don't need white space between these ``numbers.''

(In the design of the compressed proof format, only upper-case letters,
as opposed to say all non-whitespace printable {\sc ascii} characters other than
%\texttt{\$}, was chosen to make the compressed proof a little less
%displeasing to the eye, at the expense of a typical 20\% compression
\texttt{\$}, were chosen so as not to collide with most text editor
searches, at the expense of a typical 20\% compression
loss.  The base 5/base 20 grouping, as opposed to say base 6/base 19,
was chosen by experimentally determining the grouping that resulted in
best typical compression.)

The letter \texttt{Z} identifies (tags) a proof step that is identical to one
that occurs later on in the proof; it helps shorten the proof by not requiring
that identical proof steps be proved over and over again (which happens often
when building wff's).  The \texttt{Z} is placed immediately after the
least-significant digit (letters \texttt{A} through \texttt{T}) that ends the integer
corresponding to the step to later be referenced.

The integers that the upper-case letters correspond to are mapped to labels as
follows.  If the statement being proved has $m$ mandatory hypotheses, integers
1 through $m$ correspond to the labels of these hypotheses in the order shown
by the \texttt{show statement ... / full} command, i.e., the RPN order\index{RPN
order} of the mandatory
hypotheses.  Integers $m+1$ through $m+n$ correspond to the labels enclosed in
the parentheses of the compressed proof, in the order that they appear, where
$n$ is the number of those labels.  Integers $m+n+1$ on up don't directly
correspond to statement labels but point to proof steps identified with the
letter \texttt{Z}, so that these proof steps can be referenced later in the
proof.  Integer $m+n+1$ corresponds to the first step tagged with a \texttt{Z},
$m+n+2$ to the second step tagged with a \texttt{Z}, etc.  When the compressed
proof is converted to a normal proof, the entire subproof of a step tagged
with \texttt{Z} replaces the reference to that step.

For efficiency, Metamath works with compressed proofs directly, without
converting them internally to normal proofs.  In addition to the usual
error-checking, an error message is given if (1) a label in the label list in
parentheses does not refer to a previous \texttt{\$p} or \texttt{\$a} statement or a
non-mandatory hypothesis of the statement being proved and (2) a proof step
tagged with \texttt{Z} is referenced before the step tagged with the \texttt{Z}.

Just as in a normal proof under development (Section~\ref{unknown}), any step
or subproof that is not yet known may be represented with a single \texttt{?}.
White space does not have to appear between the \texttt{?}\ and the upper-case
letters (or other \texttt{?}'s) representing the remainder of the proof.

% April 1, 2004 Appendix C has been added back in with corrections.
%
% May 20, 2003 Appendix C was removed for now because there was a problem found
% by Bob Solovay
%
% Also, removed earlier \ref{formalspec} 's (3 cases above)
%
% Bob Solovay wrote on 30 Nov 2002:
%%%%%%%%%%%%% (start of email comment )
%      3. My next set of comments concern appendix C. I read this before I
% read Chapter 4. So I first noted that the system as presented in the
% Appendix lacked a certain formal property that I thought desirable. I
% then came up with a revised formal system that had this property. Upon
% reading Chapter 4, I noticed that the revised system was closer to the
% treatment in Chapter 4 than the system in Appendix C.
%
%         First a very minor correction:
%
%         On page 142 line 2: The condition that V(e) != V(f) should only be
% required of e, f in T such that e != f.
%
%         Here is a natural property [transitivity] that one would like
% the formal system to have:
%
%         Let Gamma be a set of statements. Suppose that the statement Phi
% is provable from Gamma and that the statement Psi is provable from Gamma
% \cup {Phi}. Then Psi is provable from Gamma.
%
%         I shall present an example to show that this property does not
% hold for the formal systems of Appendix C:
%
%         I write the example in metamath style:
%
% $c A B C D E $.
% $v x y
%
% ${
% tx $f A x $.
% ty $f B y $.
% ax1 $a C x y $.
% $}
%
% ${
% tx $f A x $.
% ty $f B y $.
% ax2-h1 $e C x y $.
% ax2 $a D y $.
% $}
%
% ${
% ty $f B y $.
% ax3-h1 $e D y $.
% ax3 $a E y $.
% $}
%
% $(These three axioms are Gamma $)
%
% ${
% tx $f A x $.
% ty $f B y $.
% Phi $p D y $=
% tx ty tx ty ax1 ax2 $.
% $}
%
% ${
% ty $f B y $.
% Psi $p E y $=
% ty ty Phi ax3 $.
% $}
%
%
% I omit the formal proofs of the following claims. [I will be glad to
% supply them upon request.]
%
% 1) Psi is not provable from Gamma;
%
% 2) Psi is provable from Gamma + Phi.
%
% Here "provable" refers to the formalism of Appendix C.
%
% The trouble of course is that Psi is lacking the variable declaration
%
% $f Ax $.
%
% In the Metamath system there is no trouble proving Psi. I attach a
% metamath file that shows this and which has been checked by the
% metamath program.
%
% I next want to indicate how I think the treatment in Appendix C should
% be revised so as to conform more closely to the metamath system of the
% main text. The revised system *does* have the transitivity property.
%
% We want to give revised definitions of "statement" and
% "provable". [cf. sections C.2.4. and C.2.5] Our new definitions will
% use the definitions given in Appendix C. So we take the following
% tack. We refer to the original notions as o-statement and o-provable. And
% we refer to the notions we are defining as n-statement and n-provable.
%
%         A n-statement is an o-statement in which the only variables
% that appear in the T component are mandatory.
%
%         To any o-statement we can associate its reduct which is a
% n-statement by dropping all the elements of T or D which contain
% non-mandatory variables.
%
%         An n-statement gamma is n-provable if there is an o-statement
% gamma' which has gamma as its reduct andf such that gamma' is
% o-provable.
%
%         It seems to me [though I am not completely sure on this point]
% that n-provability corresponds to metamath provability as discussed
% say in Chapter 4.
%
%         Attached to this letter is the metamath proof of Phi and Psi
% from Gamma discussed above.
%
%         I am still brooding over the question of whether metamath
% correctly formalizes set-theory. No doubt I will have some questions
% re this after my thoughts become clearer.
%%%%%%%%%%%%%%%% (end of email comment)

%%%%%%%%%%%%%%%% (start of 2nd email comment from Bob Solovay 1-Apr-04)
%
%         I hope that Appendix C is the one that gives a "formal" treatment
% of Metamath. At any rate, thats the appendix I want to comment on.
%
%         I'm going to suggest two changes in the definition.
%
%         First change (in the definition of statement): Require that the
% sets D, T, and E be finite.
%
%         Probably things are fine as you give them. But in the applications
% to the main metamath system they will always be finite, and its useful in
% thinking about things [at least for me] to stick to the finite case.
%
%         Second change:
%
%         First let me give an approximate description. Remove the dummy
% variables from the statement. Instead, include them in the proof.
%
%         More formally: Require that T consists of type declarations only
% for mandatory variables. Require that all the pairs in D consist of
% mandatory variables.
%
%         At the start of a proof we are allowed to declare a finite number
% of dummy variables [provided that none of them appear in any of the
% statements in E \cup {A}. We have to supply type declarations for all the
% dummy variables. We are allowed to add new $d statements referring to
% either the mandatory or dummy variables. But we require that no new $d
% statement references only mandatory variables.
%
%         I find this way of doing things more conceptual than the treatment
% in Appendix C. But the change [which I will use implicitly in later
% letters about doing Peano] is mainly aesthetic. I definitely claim that my
% results on doing Peano all apply to Metamath as it is presented in your
% book.
%
%         --Bob
%
%%%%%%%%%%%%%%%% (end of 2nd email comment)

%%
%% When uncommenting the below, also uncomment references above to {formalspec}
%%
\chapter{Metamath's Formal System}\label{formalspec}\index{Metamath!as a formal
system}

\section{Introduction}

\begin{quote}
  {\em Perfection is when there is no longer anything more to take away.}
    \flushright\sc Antoine de
     Saint-Exupery\footnote{\cite[p.~3-25]{Campbell}.}\\
\end{quote}\index{de Saint-Exupery, Antoine}

This appendix describes the theory behind the Metamath language in an abstract
way intended for mathematicians.  Specifically, we construct two
set-theo\-ret\-i\-cal objects:  a ``formal system'' (roughly, a set of syntax
rules, axioms, and logical rules) and its ``universe'' (roughly, the set of
theorems derivable in the formal system).  The Metamath computer language
provides us with a way to describe specific formal systems and, with the aid of
a proof provided by the user, to verify that given theorems
belong to their universes.

To understand this appendix, you need a basic knowledge of informal set theory.
It should be sufficient to understand, for example, Ch.\ 1 of Munkres' {\em
Topology} \cite{Munkres}\index{Munkres, James R.} or the
introductory set theory chapter
in many textbooks that introduce abstract mathematics. (Note that there are
minor notational differences among authors; e.g.\ Munkres uses $\subset$ instead
of our $\subseteq$ for ``subset.''  We use ``included in'' to mean ``a subset
of,'' and ``belongs to'' or ``is contained in'' to mean ``is an element of.'')
What we call a ``formal'' description here, unlike earlier, is actually an
informal description in the ordinary language of mathematicians.  However we
provide sufficient detail so that a mathematician could easily formalize it,
even in the language of Metamath itself if desired.  To understand the logic
examples at the end of this appendix, familiarity with an introductory book on
mathematical logic would be helpful.

\section{The Formal Description}

\subsection[Preliminaries]{Preliminaries\protect\footnotemark}%
\footnotetext{This section is taken mostly verbatim
from Tarski \cite[p.~63]{Tarski1965}\index{Tarski, Alfred}.}

By $\omega$ we denote the set of all natural numbers (non-negative integers).
Each natural number $n$ is identified with the set of all smaller numbers: $n =
\{ m | m < n \}$.  The formula $m < n$ is thus equivalent to the condition: $m
\in n$ and $m,n \in \omega$. In particular, 0 is the number zero and at the
same time the empty set $\varnothing$, $1=\{0\}$, $2=\{0,1\}$, etc. ${}^B A$
denotes the set of all functions on $B$ to $A$ (i.e.\ with domain $B$ and range
included in $A$).  The members of ${}^\omega A$ are what are called {\em simple
infinite sequences},\index{simple infinite sequence}
with all {\em terms}\index{term} in $A$.  In case $n \in \omega$, the
members of ${}^n A$ are referred to as {\em finite $n$-termed
sequences},\index{finite $n$-termed
sequence} again
with terms in $A$.  The consecutive terms (function values) of a finite or
infinite sequence $f$ are denoted by $f_0, f_1, \ldots ,f_n,\ldots$.  Every
finite sequence $f \in \bigcup _{n \in \omega} {}^n A$ uniquely determines the
number $n$ such that $f \in {}^n A$; $n$ is called the {\em
length}\index{length of a sequence ({$"|\ "|$})} of $f$ and
is denoted by $|f|$.  $\langle a \rangle$ is the sequence $f$ with $|f|=1$ and
$f_0=a$; $\langle a,b \rangle$ is the sequence $f$ with $|f|=2$, $f_0=a$,
$f_1=b$; etc.  Given two finite sequences $f$ and $g$, we denote by $f\frown g$
their {\em concatenation},\index{concatenation} i.e., the
finite sequence $h$ determined by the
conditions:
\begin{eqnarray*}
& |h| = |f|+|g|;&  \\
& h_n = f_n & \mbox{\ for\ } n < |f|;  \\
& h_{|f|+n} = g_n & \mbox{\ for\ } n < |g|.
\end{eqnarray*}

\subsection{Constants, Variables, and Expressions}

A formal system has a set of {\em symbols}\index{symbol!in
a formal system} denoted
by $\mbox{\em SM}$.  A
precise set-theo\-ret\-i\-cal definition of this set is unimportant; a symbol
could be considered a primitive or atomic element if we wish.  We assume this
set is divided into two disjoint subsets:  a set $\mbox{\em CN}$ of {\em
constants}\index{constant!in a formal system} and a set $\mbox{\em VR}$ of
{\em variables}.\index{variable!in a formal system}  $\mbox{\em CN}$ and
$\mbox{\em VR}$ are each assumed to consist of countably many symbols which
may be arranged in finite or simple infinite sequences $c_0, c_1, \ldots$ and
$v_0, v_1, \ldots$ respectively, without repeating terms.  We will represent
arbitrary symbols by metavariables $\alpha$, $\beta$, etc.

{\footnotesize\begin{quotation}
{\em Comment.} The variables $v_0, v_1, \ldots$ of our formal system
correspond to what are usually considered ``metavariables'' in
descriptions of specific formal systems in the literature.  Typically,
when describing a specific formal system a book will postulate a set of
primitive objects called variables, then proceed to describe their
properties using metavariables that range over them, never mentioning
again the actual variables themselves.  Our formal system does not
mention these primitive variable objects at all but deals directly with
metavariables, as its primitive objects, from the start.  This is a
subtle but key distinction you should keep in mind, and it makes our
definition of ``formal system'' somewhat different from that typically
found in the literature.  (So, the $\alpha$, $\beta$, etc.\ above are
actually ``metametavariables'' when used to represent $v_0, v_1,
\ldots$.)
\end{quotation}}

Finite sequences all terms of which are symbols are called {\em
expressions}.\index{expression!in a formal system}  $\mbox{\em EX}$ is
the set of all expressions; thus
\begin{displaymath}
\mbox{\em EX} = \bigcup _{n \in \omega} {}^n \mbox{\em SM}.
\end{displaymath}

A {\em constant-prefixed expression}\index{constant-prefixed expression}
is an expression of non-zero length
whose first term is a constant.  We denote the set of all constant-prefixed
expressions by $\mbox{\em EX}_C = \{ e \in \mbox{\em EX} | ( |e| > 0 \wedge
e_0 \in \mbox{\em CN} ) \}$.

A {\em constant-variable pair}\index{constant-variable pair}
is an expression of length 2 whose first term
is a constant and whose second term is a variable.  We denote the set of all
constant-variable pairs by $\mbox{\em EX}_2 = \{ e \in \mbox{\em EX}_C | ( |e|
= 2 \wedge e_1 \in \mbox{\em VR} ) \}$.


{\footnotesize\begin{quotation}
{\em Relationship to Metamath.} In general, the set $\mbox{\em SM}$
corresponds to the set of declared math symbols in a Metamath database, the
set $\mbox{\em CN}$ to those declared with \texttt{\$c} statements, and the set
$\mbox{\em VR}$ to those declared with \texttt{\$v} statements.  Of course a
Metamath database can only have a finite number of math symbols, whereas
formal systems in general can have an infinite number, although the number of
Metamath math symbols available is in principle unlimited.

The set $\mbox{\em EX}_C$ corresponds to the set of permissible expressions
for \texttt{\$e}, \texttt{\$a}, and \texttt{\$p} statements.  The set $\mbox{\em EX}_2$
corresponds to the set of permissible expressions for \texttt{\$f} statements.
\end{quotation}}

We denote by ${\cal V}(e)$ the set of all variables in an expression $e \in
\mbox{\em EX}$, i.e.\ the set of all $\alpha \in \mbox{\em VR}$ such that
$\alpha = e_n$ for some $n < |e|$.  We also denote (with abuse of notation) by
${\cal V}(E)$ the set of all variables in a collection of expressions $E
\subseteq \mbox{\em EX}$, i.e.\ $\bigcup _{e \in E} {\cal V}(e)$.


\subsection{Substitution}

Given a function $F$ from $\mbox{\em VR}$ to
$\mbox{\em EX}$, we
denote by $\sigma_{F}$ or just $\sigma$ the function from $\mbox{\em EX}$ to
$\mbox{\em EX}$ defined recursively for nonempty sequences by
\begin{eqnarray*}
& \sigma(<\alpha>) = F(\alpha) & \mbox{for\ } \alpha \in \mbox{\em VR}; \\
& \sigma(<\alpha>) = <\alpha> & \mbox{for\ } \alpha \not\in \mbox{\em VR}; \\
& \sigma(g \frown h) = \sigma(g) \frown
    \sigma(h) & \mbox{for\ } g,h \in \mbox{\em EX}.
\end{eqnarray*}
We also define $\sigma(\varnothing)=\varnothing$.  We call $\sigma$ a {\em
simultaneous substitution}\index{substitution!variable}\index{variable
substitution} (or just {\em substitution}) with {\em substitution
map}\index{substitution map} $F$.

We also denote (with abuse of notation) by $\sigma(E)$ a substitution on a
collection of expressions $E \subseteq \mbox{\em EX}$, i.e.\ the set $\{
\sigma(e) | e \in E \}$.  The collection $\sigma(E)$ may of course contain
fewer expressions than $E$ because duplicate expressions could result from the
substitution.

\subsection{Statements}

We denote by $\mbox{\em DV}$ the set of all
unordered pairs $\{\alpha, \beta \} \subseteq \mbox{\em VR}$ such that $\alpha
\neq \beta$.  $\mbox{\em DV}$ stands for ``distinct variables.''

A {\em pre-statement}\index{pre-statement!in a formal system} is a
quadruple $\langle D,T,H,A \rangle$ such that
$D\subseteq \mbox{\em DV}$, $T\subseteq \mbox{\em EX}_2$, $H\subseteq
\mbox{\em EX}_C$ and $H$ is finite,
$A\in \mbox{\em EX}_C$, ${\cal V}(H\cup\{A\}) \subseteq
{\cal V}(T)$, and $\forall e,f\in T {\ } {\cal V}(e) \neq {\cal V}(f)$ (or
equivalently, $e_1 \ne f_1$) whenever $e \neq f$. The terms of the quadruple are called {\em
distinct-variable restrictions},\index{disjoint-variable restriction!in a
formal system} {\em variable-type hypotheses},\index{variable-type
hypothesis!in a formal system} {\em logical hypotheses},\index{logical
hypothesis!in a formal system} and the {\em assertion}\index{assertion!in a
formal system} respectively.  We denote by $T_M$ ({\em mandatory variable-type
hypotheses}\index{mandatory variable-type hypothesis!in a formal system}) the
subset of $T$ such that ${\cal V}(T_M) ={\cal V}(H \cup \{A\})$.  We denote by
$D_M=\{\{\alpha,\beta\}\in D|\{\alpha,\beta\}\subseteq {\cal V}(T_M)\}$ the
{\em mandatory distinct-variable restrictions}\index{mandatory
disjoint-variable restriction!in a formal system} of the pre-statement.
The set
of {\em mandatory hypotheses}\index{mandatory hypothesis!in a formal system}
is $T_M\cup H$.  We call the quadruple $\langle D_M,T_M,H,A \rangle$
the {\em reduct}\index{reduct!in a formal system} of
the pre-statement $\langle D,T,H,A \rangle$.

A {\em statement} is the reduct of some pre-statement\index{statement!in a
formal system}.  A statement is therefore a special kind of pre-statement;
in particular, a statement is the reduct of itself.

{\footnotesize\begin{quotation}
{\em Comment.}  $T$ is a set of expressions, each of length 2, that associate
a set of constants (``variable types'') with a set of variables.  The
condition ${\cal V}(H\cup\{A\}) \subseteq {\cal V}(T) $
means that each variable occurring in a statement's logical
hypotheses or assertion must have an associated variable-type hypothesis or
``type declaration,'' in  analogy to a computer programming language, where a
variable must be declared to be say, a string or an integer.  The requirement
that $\forall e,f\in T \, e_1 \ne f_1$ for $e\neq f$
means that each variable must be
associated with a unique constant designating its variable type; e.g., a
variable might be a ``wff'' or a ``set'' but not both.

Distinct-variable restrictions are used to specify what variable substitutions
are permissible to make for the statement to remain valid.  For example, in
the theorem scheme of set theory $\lnot\forall x\,x=y$ we may not substitute
the same variable for both $x$ and $y$.  On the other hand, the theorem scheme
$x=y\to y=x$ does not require that $x$ and $y$ be distinct, so we do not
require a distinct-variable restriction, although having one
would cause no harm other than making the scheme less general.

A mandatory variable-type hypothesis is one whose variable exists in a logical
hypothesis or the assertion.  A provable pre-statement
(defined below) may require
non-mandatory variable-type hypotheses that effectively introduce ``dummy''
variables for use in its proof.  Any number of dummy variables might
be required by a specific proof; indeed, it has been shown by H.\
Andr\'{e}ka\index{Andr{\'{e}}ka, H.} \cite{Nemeti} that there is no finite
upper bound to the number of dummy variables needed to prove an arbitrary
theorem in first-order logic (with equality) having a fixed number $n>2$ of
individual variables.  (See also the Comment on p.~\pageref{nodd}.)
For this reason we do not set a finite size bound on the collections $D$ and
$T$, although in an actual application (Metamath database) these will of
course be finite, increased to whatever size is necessary as more
proofs are added.
\end{quotation}}

{\footnotesize\begin{quotation}
{\em Relationship to Metamath.} A pre-statement of a formal system
corresponds to an extended frame in a Metamath database
(Section~\ref{frames}).  The collections $D$, $T$, and $H$ correspond
respectively to the \texttt{\$d}, \texttt{\$f}, and \texttt{\$e}
statement collections in an extended frame.  The expression $A$
corresponds to the \texttt{\$a} (or \texttt{\$p}) statement in an
extended frame.

A statement of a formal system corresponds to a frame in a Metamath
database.
\end{quotation}}

\subsection{Formal Systems}

A {\em formal system}\index{formal system} is a
triple $\langle \mbox{\em CN},\mbox{\em
VR},\Gamma\rangle$ where $\Gamma$ is a set of statements.  The members of
$\Gamma$ are called {\em axiomatic statements}.\index{axiomatic
statement!in a formal system}  Sometimes we will refer to a
formal system by just $\Gamma$ when $\mbox{\em CN}$ and $\mbox{\em VR}$ are
understood.

Given a formal system $\Gamma$, the {\em closure}\index{closure}\footnote{This
definition of closure incorporates a simplification due to
Josh Purinton.\index{Purinton, Josh}.} of a
pre-statement
$\langle D,T,H,A \rangle$ is the smallest set $C$ of expressions
such that:
%\begin{enumerate}
%  \item $T\cup H\subseteq C$; and
%  \item If for some axiomatic statement
%    $\langle D_M',T_M',H',A' \rangle \in \Gamma_A$, for
%    some $E \subseteq C$, some $F \subseteq C-T$ (where ``-'' denotes
%    set difference), and some substitution
%    $\sigma$ we have
%    \begin{enumerate}
%       \item $\sigma(T_M') = E$ (where, as above, the $M$ denotes the
%           mandatory variable-type hypotheses of $T^A$);
%       \item $\sigma(H') = F$;
%       \item for all $\{\alpha,\beta\}\in D^A$ and $\subseteq
%         {\cal V}(T_M')$, for all $\gamma\in {\cal V}(\sigma(\langle \alpha
%         \rangle))$, and for all $\delta\in  {\cal V}(\sigma(\langle \beta
%         \rangle))$, we have $\{\gamma, \delta\} \in D$;
%   \end{enumerate}
%   then $\sigma(A') \in C$.
%\end{enumerate}
\begin{list}{}{\itemsep 0.0pt}
  \item[1.] $T\cup H\subseteq C$; and
  \item[2.] If for some axiomatic statement
    $\langle D_M',T_M',H',A' \rangle \in
       \Gamma$ and for some substitution
    $\sigma$ we have
    \begin{enumerate}
       \item[a.] $\sigma(T_M' \cup H') \subseteq C$; and
       \item[b.] for all $\{\alpha,\beta\}\in D_M'$, for all $\gamma\in
         {\cal V}(\sigma(\langle \alpha
         \rangle))$, and for all $\delta\in  {\cal V}(\sigma(\langle \beta
         \rangle))$, we have $\{\gamma, \delta\} \in D$;
   \end{enumerate}
   then $\sigma(A') \in C$.
\end{list}
A pre-statement $\langle D,T,H,A
\rangle$ is {\em provable}\index{provable statement!in a formal
system} if $A\in C$ i.e.\ if its assertion belongs to its
closure.  A statement is {\em provable} if it is
the reduct of a provable pre-statement.
The {\em universe}\index{universe of a formal system}
of a formal system is
the collection of all of its provable statements.  Note that the
set of axiomatic statements $\Gamma$ in a formal system is a subset of its
universe.

{\footnotesize\begin{quotation}
{\em Comment.} The first condition in the definition of closure simply says
that the hypotheses of the pre-statement are in its closure.

Condition 2(a) says that a substitution exists that makes the
mandatory hypotheses of an axiomatic statement exactly match some members of
the closure.  This is what we explicitly demonstrate in a Metamath language
proof.

%Conditions 2(a) and 2(b) say that a substitution exists that makes the
%(mandatory) hypotheses of an axiomatic statement exactly match some members of
%the closure.  This is what we explicitly demonstrate with a Metamath language
%proof.
%
%The set of expressions $F$ in condition 2(b) excludes the variable-type
%hypotheses; this is done because non-mandatory variable-type hypotheses are
%effectively ``dropped'' as irrelevant whereas logical hypotheses must be
%retained to achieve a consistent logical system.

Condition 2(b) describes how distinct-variable restrictions in the axiomatic
statement must be met.  It means that after a substitution for two variables
that must be distinct, the resulting two expressions must either contain no
variables, or if they do, they may not have variables in common, and each pair
of any variables they do have, with one variable from each expression, must be
specified as distinct in the original statement.
\end{quotation}}

{\footnotesize\begin{quotation}
{\em Relationship to Metamath.} Axiomatic statements
 and provable statements in a formal
system correspond to the frames for \texttt{\$a} and \texttt{\$p} statements
respectively in a Metamath database.  The set of axiomatic statements is a
subset of the set of provable statements in a formal system, although in a
Metamath database a \texttt{\$a} statement is distinguished by not having a
proof.  A Metamath language proof for a \texttt{\$p} statement tells the computer
how to explicitly construct a series of members of the closure ultimately
leading to a demonstration that the assertion
being proved is in the closure.  The actual closure typically contains
an infinite number of expressions.  A formal system itself does not have
an explicit object called a ``proof'' but rather the existence of a proof
is implied indirectly by membership of an assertion in a provable
statement's closure.  We do this to make the formal system easier
to describe in the language of set theory.

We also note that once established as provable, a statement may be considered
to acquire the same status as an axiomatic statement, because if the set of
axiomatic statements is extended with a provable statement, the universe of
the formal system remains unchanged (provided that $\mbox{\em VR}$ is
infinite).
In practice, this means we can build a hierarchy of provable statements to
more efficiently establish additional provable statements.  This is
what we do in Metamath when we allow proofs to reference previous
\texttt{\$p} statements as well as previous \texttt{\$a} statements.
\end{quotation}}

\section{Examples of Formal Systems}

{\footnotesize\begin{quotation}
{\em Relationship to Metamath.} The examples in this section, except Example~2,
are for the most part exact equivalents of the development in the set
theory database \texttt{set.mm}.  You may want to compare Examples~1, 3, and 5
to Section~\ref{metaaxioms}, Example 4 to Sections~\ref{metadefprop} and
\ref{metadefpred}, and Example 6 to
Section~\ref{setdefinitions}.\label{exampleref}
\end{quotation}}

\subsection{Example~1---Propositional Calculus}\index{propositional calculus}

Classical propositional calculus can be described by the following formal
system.  We assume the set of variables is infinite.  Rather than denoting the
constants and variables by $c_0, c_1, \ldots$ and $v_0, v_1, \ldots$, for
readability we will instead use more conventional symbols, with the
understanding of course that they denote distinct primitive objects.
Also for readability we may omit commas between successive terms of a
sequence; thus $\langle \mbox{wff\ } \varphi\rangle$ denotes
$\langle \mbox{wff}, \varphi\rangle$.

Let
\begin{itemize}
  \item[] $\mbox{\em CN}=\{\mbox{wff}, \vdash, \to, \lnot, (,)\}$
  \item[] $\mbox{\em VR}=\{\varphi,\psi,\chi,\ldots\}$
  \item[] $T = \{\langle \mbox{wff\ } \varphi\rangle,
             \langle \mbox{wff\ } \psi\rangle,
             \langle \mbox{wff\ } \chi\rangle,\ldots\}$, i.e.\ those
             expressions of length 2 whose first member is $\mbox{\rm wff}$
             and whose second member belongs to $\mbox{\em VR}$.\footnote{For
convenience we let $T$ be an infinite set; the definition of a statement
permits this in principle.  Since a Metamath source file has a finite size, in
practice we must of course use appropriate finite subsets of this $T$,
specifically ones containing at least the mandatory variable-type
hypotheses.  Similarly, in the source file we introduce new variables as
required, with the understanding that a potentially infinite number of
them are available.}
\noindent Then $\Gamma$ consists of the axiomatic statements that
are the reducts of the following pre-statements:
    \begin{itemize}
      \item[] $\langle\varnothing,T,\varnothing,
               \langle \mbox{wff\ }(\varphi\to\psi)\rangle\rangle$
      \item[] $\langle\varnothing,T,\varnothing,
               \langle \mbox{wff\ }\lnot\varphi\rangle\rangle$
      \item[] $\langle\varnothing,T,\varnothing,
               \langle \vdash(\varphi\to(\psi\to\varphi))
               \rangle\rangle$
      \item[] $\langle\varnothing,T,
               \varnothing,
               \langle \vdash((\varphi\to(\psi\to\chi))\to
               ((\varphi\to\psi)\to(\varphi\to\chi)))
               \rangle\rangle$
      \item[] $\langle\varnothing,T,
               \varnothing,
               \langle \vdash((\lnot\varphi\to\lnot\psi)\to
               (\psi\to\varphi))\rangle\rangle$
      \item[] $\langle\varnothing,T,
               \{\langle\vdash(\varphi\to\psi)\rangle,
                 \langle\vdash\varphi\rangle\},
               \langle\vdash\psi\rangle\rangle$
    \end{itemize}
\end{itemize}

(For example, the reduct of $\langle\varnothing,T,\varnothing,
               \langle \mbox{wff\ }(\varphi\to\psi)\rangle\rangle$
is
\begin{itemize}
\item[] $\langle\varnothing,
\{\langle \mbox{wff\ } \varphi\rangle,
             \langle \mbox{wff\ } \psi\rangle\},
             \varnothing,
               \langle \mbox{wff\ }(\varphi\to\psi)\rangle\rangle$,
\end{itemize}
which is the first axiomatic statement.)

We call the members of $\mbox{\em VR}$ {\em wff variables} or (in the context
of first-order logic which we will describe shortly) {\em wff metavariables}.
Note that the symbols $\phi$, $\psi$, etc.\ denote actual specific members of
$\mbox{\em VR}$; they are not metavariables of our expository language (which
we denote with $\alpha$, $\beta$, etc.) but are instead (meta)constant symbols
(members of $\mbox{\em SM}$) from the point of view of our expository
language.  The equivalent system of propositional calculus described in
\cite{Tarski1965} also uses the symbols $\phi$, $\psi$, etc.\ to denote wff
metavariables, but in \cite{Tarski1965} unlike here those are metavariables of
the expository language and not primitive symbols of the formal system.

The first two statements define wffs: if $\varphi$ and $\psi$ are wffs, so is
$(\varphi \to \psi)$; if $\varphi$ is a wff, so is $\lnot\varphi$. The next
three are the axioms of propositional calculus: if $\varphi$ and $\psi$ are
wffs, then $\vdash (\varphi \to (\psi \to \varphi))$ is an (axiomatic)
theorem; etc. The
last is the rule of modus ponens: if $\varphi$ and $\psi$ are wffs, and
$\vdash (\varphi\to\psi)$ and $\vdash \varphi$ are theorems, then $\vdash
\psi$ is a theorem.

The correspondence to ordinary propositional calculus is as follows.  We
consider only provable statements of the form $\langle\varnothing,
T,\varnothing,A\rangle$ with $T$ defined as above.  The first term of the
assertion $A$ of any such statement is either ``wff'' or ``$\vdash$''.  A
statement for which the first term is ``wff'' is a {\em wff} of propositional
calculus, and one where the first term is ``$\vdash$'' is a {\em
theorem (scheme)} of propositional calculus.

The universe of this formal system also contains many other provable
statements.  Those with distinct-variable restrictions are irrelevant because
propositional calculus has no constraints on substitutions.  Those that have
logical hypotheses we call {\em inferences}\index{inference} when
the logical hypotheses are of the form
$\langle\vdash\rangle\frown w$ where $w$ is a wff (with the leading constant
term ``wff'' removed).  Inferences (other than the modus ponens rule) are not a
proper part of propositional calculus but are convenient to use when building a
hierarchy of provable statements.  A provable statement with a nonsense
hypothesis such as $\langle \to,\vdash,\lnot\rangle$, and this same expression
as its assertion, we consider irrelevant; no use can be made of it in
proving theorems, since there is no way to eliminate the nonsense hypothesis.

{\footnotesize\begin{quotation}
{\em Comment.} Our use of parentheses in the definition of a wff illustrates
how axiomatic statements should be carefully stated in a way that
ties in unambiguously with the substitutions allowed by the formal system.
There are many ways we could have defined wffs---for example, Polish
prefix notation would have allowed us to omit parentheses entirely, at
the expense of readability---but we must define them in a way that is
unambiguous.  For example, if we had omitted parentheses from the
definition of $(\varphi\to \psi)$, the wff $\lnot\varphi\to \psi$ could
be interpreted as either $\lnot(\varphi\to\psi)$ or $(\lnot\varphi\to\psi)$
and would have allowed us to prove nonsense.  Note that there is no
concept of operator binding precedence built into our formal system.
\end{quotation}}

\begin{sloppy}
\subsection{Example~2---Predicate Calculus with Equality}\index{predicate
calculus}
\end{sloppy}

Here we extend Example~1 to include predicate calculus with equality,
illustrating the use of distinct-variable restrictions.  This system is the
same as Tarski's system $\mathfrak{S}_2$ in \cite{Tarski1965} (except that the
axioms of propositional calculus are different but equivalent, and a redundant
axiom is omitted).  We extend $\mbox{\em CN}$ with the constants
$\{\mbox{var},\forall,=\}$.  We extend $\mbox{\em VR}$ with an infinite set of
{\em individual metavariables}\index{individual
metavariable} $\{x,y,z,\ldots\}$ and denote this subset
$\mbox{\em Vr}$.

We also join to $\mbox{\em CN}$ a possibly infinite set $\mbox{\em Pr}$ of {\em
predicates} $\{R,S,\ldots\}$.  We associate with $\mbox{\em Pr}$ a function
$\mbox{rnk}$ from $\mbox{\em Pr}$ to $\omega$, and for $\alpha\in \mbox{\em
Pr}$ we call $\mbox{rnk}(\alpha)$ the {\em rank} of the predicate $\alpha$,
which is simply the number of ``arguments'' that the predicate has.  (Most
applications of predicate calculus will have a finite number of predicates;
for example, set theory has the single two-argument or binary predicate $\in$,
which is usually written with its arguments surrounding the predicate symbol
rather than with the prefix notation we will use for the general case.)  As a
device to facilitate our discussion, we will let $\mbox{\em Vs}$ be any fixed
one-to-one function from $\omega$ to $\mbox{\em Vr}$; thus $\mbox{\em Vs}$ is
any simple infinite sequence of individual metavariables with no repeating
terms.

In this example we will not include the function symbols that are often part of
formalizations of predicate calculus.  Using metalogical arguments that are
beyond the scope of our discussion, it can be shown that our formalization is
equivalent when functions are introduced via appropriate definitions.

We extend the set $T$ defined in Example~1 with the expressions
$\{\langle \mbox{var\ } x\rangle,$ $ \langle \mbox{var\ } y\rangle, \langle
\mbox{var\ } z\rangle,\ldots\}$.  We extend the $\Gamma$ above
with the axiomatic statements that are the reducts of the following
pre-statements:
\begin{list}{}{\itemsep 0.0pt}
      \item[] $\langle\varnothing,T,\varnothing,
               \langle \mbox{wff\ }\forall x\,\varphi\rangle\rangle$
      \item[] $\langle\varnothing,T,\varnothing,
               \langle \mbox{wff\ }x=y\rangle\rangle$
      \item[] $\langle\varnothing,T,
               \{\langle\vdash\varphi\rangle\},
               \langle\vdash\forall x\,\varphi\rangle\rangle$
      \item[] $\langle\varnothing,T,\varnothing,
               \langle \vdash((\forall x(\varphi\to\psi)
                  \to(\forall x\,\varphi\to\forall x\,\psi))
               \rangle\rangle$
      \item[] $\langle\{\{x,\varphi\}\},T,\varnothing,
               \langle \vdash(\varphi\to\forall x\,\varphi)
               \rangle\rangle$
      \item[] $\langle\{\{x,y\}\},T,\varnothing,
               \langle \vdash\lnot\forall x\lnot x=y
               \rangle\rangle$
      \item[] $\langle\varnothing,T,\varnothing,
               \langle \vdash(x=z
                  \to(x=y\to z=y))
               \rangle\rangle$
      \item[] $\langle\varnothing,T,\varnothing,
               \langle \vdash(y=z
                  \to(x=y\to x=z))
               \rangle\rangle$
\end{list}
These are the axioms not involving predicate symbols. The first two statements
extend the definition of a wff.  The third is the rule of generalization.  The
fifth states, in effect, ``For a wff $\varphi$ and variable $x$,
$\vdash(\varphi\to\forall x\,\varphi)$, provided that $x$ does not occur in
$\varphi$.''  The sixth states ``For variables $x$ and $y$,
$\vdash\lnot\forall x\lnot x = y$, provided that $x$ and $y$ are distinct.''
(This proviso is not necessary but was included by Tarski to
weaken the axiom and still show that the system is logically complete.)

Finally, for each predicate symbol $\alpha\in \mbox{\em Pr}$, we add to
$\Gamma$ an axiomatic statement, extending the definition of wff,
that is the reduct of the following pre-statement:
\begin{displaymath}
    \langle\varnothing,T,\varnothing,
            \langle \mbox{wff},\alpha\rangle\
            \frown \mbox{\em Vs}\restriction\mbox{rnk}(\alpha)\rangle
\end{displaymath}
and for each $\alpha\in \mbox{\em Pr}$ and each $n < \mbox{rnk}(\alpha)$
we add to $\Gamma$ an equality axiom that is the reduct of the
following pre-statement:
\begin{eqnarray*}
    \lefteqn{\langle\varnothing,T,\varnothing,
            \langle
      \vdash,(,\mbox{\em Vs}_n,=,\mbox{\em Vs}_{\mbox{rnk}(\alpha)},\to,
            (,\alpha\rangle\frown \mbox{\em Vs}\restriction\mbox{rnk}(\alpha)} \\
  & & \frown
            \langle\to,\alpha\rangle\frown \mbox{\em Vs}\restriction n\frown
            \langle \mbox{\em Vs}_{\mbox{rnk}(\alpha)}\rangle \\
 & & \frown
            \mbox{\em Vs}\restriction(\mbox{rnk}(\alpha)\setminus(n+1))\frown
            \langle),)\rangle\rangle
\end{eqnarray*}
where $\restriction$ denotes function domain restriction and $\setminus$
denotes set difference.  Recall that a subscript on $\mbox{\em Vs}$
denotes one of its terms.  (In the above two axiom sets commas are placed
between successive terms of sequences to prevent ambiguity, and if you examine
them with care you will be able to distinguish those parentheses that denote
constant symbols from those of our expository language that delimit function
arguments.  Although it might have been better to use boldface for our
primitive symbols, unfortunately boldface was not available for all characters
on the \LaTeX\ system used to typeset this text.)  These seemingly forbidding
axioms can be understood by analogy to concatenation of substrings in a
computer language.  They are actually relatively simple for each specific case
and will become clearer by looking at the special case of a binary predicate
$\alpha = R$ where $\mbox{rnk}(R)=2$.  Letting $\mbox{\em Vs}$ be the sequence
$\langle x,y,z,\ldots\rangle$, the axioms we would add to $\Gamma$ for this
case would be the wff extension and two equality axioms that are the
reducts of the pre-statements:
\begin{list}{}{\itemsep 0.0pt}
      \item[] $\langle\varnothing,T,\varnothing,
               \langle \mbox{wff\ }R x y\rangle\rangle$
      \item[] $\langle\varnothing,T,\varnothing,
               \langle \vdash(x=z
                  \to(R x y \to R z y))
               \rangle\rangle$
      \item[] $\langle\varnothing,T,\varnothing,
               \langle \vdash(y=z
                  \to(R x y \to R x z))
               \rangle\rangle$
\end{list}
Study these carefully to see how the general axioms above evaluate to
them.  In practice, typically only a few special cases such as this would be
needed, and in any case the Metamath language will only permit us to describe
a finite number of predicates, as opposed to the infinite number permitted by
the formal system.  (If an infinite number should be needed for some reason,
we could not define the formal system directly in the Metamath language but
could instead define it metalogically under set theory as we
do in this appendix, and only the underlying set theory, with its single
binary predicate, would be defined directly in the Metamath language.)


{\footnotesize\begin{quotation}
{\em Comment.}  As we noted earlier, the specific variables denoted by the
symbols $x,y,z,\ldots\in \mbox{\em Vr}\subseteq \mbox{\em VR}\subseteq
\mbox{\em SM}$ in Example~2 are not the actual variables of ordinary predicate
calculus but should be thought of as metavariables ranging over them.  For
example, a distinct-variable restriction would be meaningless for actual
variables of ordinary predicate calculus since two different actual variables
are by definition distinct.  And when we talk about an arbitrary
representative $\alpha\in \mbox{\em Vr}$, $\alpha$ is a metavariable (in our
expository language) that ranges over metavariables (which are primitives of
our formal system) each of which ranges over the actual individual variables
of predicate calculus (which are never mentioned in our formal system).

The constant called ``var'' above is called \texttt{setvar} in the
\texttt{set.mm} database file, but it means the same thing.  I felt
that ``var'' is a more meaningful name in the context of predicate
calculus, whose use is not limited to set theory.  For consistency we
stick with the name ``var'' throughout this Appendix, even after set
theory is introduced.
\end{quotation}}

\subsection{Free Variables and Proper Substitution}\index{free variable}
\index{proper substitution}\index{substitution!proper}

Typical representations of mathematical axioms use concepts such
as ``free variable,'' ``bound variable,'' and ``proper substitution''
as primitive notions.
A free variable is a variable that
is not a parameter of any container expression.
A bound variable is the opposite of a free variable; it is a
a variable that has been bound in a container expression.
For example, in the expression $\forall x \varphi$ (for all $x$, $\varphi$
is true), the variable $x$
is bound within the for-all ($\forall$) expression.
It is possible to change one variable to another, and that process is called
``proper substitution.''
In most books, proper substitution has a somewhat complicated recursive
definition with multiple cases based on the occurrences of free and
bound variables.
You may consult
\cite[ch.\ 3--4]{Hamilton}\index{Hamilton, Alan G.} (as well as
many other texts) for more formal details about these terms.

Using these concepts as \texttt{primitives} creates complications
for computer implementations.

In the system of Example~2, there are no primitive notions of free variable
and proper substitution.  Tarski \cite{Tarski1965} shows that this system is
logically equivalent to the more typical textbook systems that do have these
primitive notions, if we introduce these notions with appropriate definitions
and metalogic.  We could also define axioms for such systems directly,
although the recursive definitions of free variable and proper substitution
would be messy and awkward to work with.  Instead, we mention two devices that
can be used in practice to mimic these notions.  (1) Instead of introducing
special notation to express (as a logical hypothesis) ``where $x$ is not free
in $\varphi$'' we can use the logical hypothesis $\vdash(\varphi\to\forall
x\,\varphi)$.\label{effectivelybound}\index{effectively
not free}\footnote{This is a slightly weaker requirement than ``where $x$ is
not free in $\varphi$.''  If we let $\varphi$ be $x=x$, we have the theorem
$(x=x\to\forall x\,x=x)$ which satisfies the hypothesis, even though $x$ is
free in $x=x$ .  In a case like this we say that $x$ is {\em effectively not
free}\index{effectively not free} in $x=x$, since $x=x$ is logically
equivalent to $\forall x\,x=x$ in which $x$ is bound.} (2) It can be shown
that the wff $((x=y\to\varphi)\wedge\exists x(x=y\wedge\varphi))$ (with the
usual definitions of $\wedge$ and $\exists$; see Example~4 below) is logically
equivalent to ``the wff that results from proper substitution of $y$ for $x$
in $\varphi$.''  This works whether or not $x$ and $y$ are distinct.

\subsection{Metalogical Completeness}\index{metalogical completeness}

In the system of Example~2, the
following are provable pre-statements (and their reducts are
provable statements):
\begin{eqnarray*}
      & \langle\{\{x,y\}\},T,\varnothing,
               \langle \vdash\lnot\forall x\lnot x=y
               \rangle\rangle & \\
     &  \langle\varnothing,T,\varnothing,
               \langle \vdash\lnot\forall x\lnot x=x
               \rangle\rangle &
\end{eqnarray*}
whereas the following pre-statement is not to my knowledge provable (but
in any case we will pretend it's not for sake of illustration):
\begin{eqnarray*}
     &  \langle\varnothing,T,\varnothing,
               \langle \vdash\lnot\forall x\lnot x=y
               \rangle\rangle &
\end{eqnarray*}
In other words, we can prove ``$\lnot\forall x\lnot x=y$ where $x$ and $y$ are
distinct'' and separately prove ``$\lnot\forall x\lnot x=x$'', but we can't
prove the combined general case ``$\lnot\forall x\lnot x=y$'' that has no
proviso.  Now this does not compromise logical completeness, because the
variables are really metavariables and the two provable cases together cover
all possible cases.  The third case can be considered a metatheorem whose
direct proof, using the system of Example~2, lies outside the capability of the
formal system.

Also, in the system of Example~2 the following pre-statement is not to my
knowledge provable (again, a conjecture that we will pretend to be the case):
\begin{eqnarray*}
     & \langle\varnothing,T,\varnothing,
               \langle \vdash(\forall x\, \varphi\to\varphi)
               \rangle\rangle &
\end{eqnarray*}
Instead, we can only prove specific cases of $\varphi$ involving individual
metavariables, and by induction on formula length, prove as a metatheorem
outside of our formal system the general statement above.  The details of this
proof are found in \cite{Kalish}.

There does, however, exist a system of predicate calculus in which all such
``simple metatheorems'' as those above can be proved directly, and we present
it in Example~3. A {\em simple metatheorem}\index{simple metatheorem}
is any statement of the formal
system of Example~2 where all distinct variable restrictions consist of either
two individual metavariables or an individual metavariable and a wff
metavariable, and which is provable by combining cases outside the system as
above.  A system is {\em metalogically complete}\index{metalogical
completeness} if all of its simple
metatheorems are (directly) provable statements. The precise definition of
``simple metatheorem'' and the proof of the ``metalogical completeness'' of
Example~3 is found in Remark 9.6 and Theorem 9.7 of \cite{Megill}.\index{Megill,
Norman}

\begin{sloppy}
\subsection{Example~3---Metalogically Complete Predicate
Calculus with
Equality}
\end{sloppy}

For simplicity we will assume there is one binary predicate $R$;
this system suffices for set theory, where the $R$ is of course the $\in$
predicate.  We label the axioms as they appear in \cite{Megill}.  This
system is logically equivalent to that of Example~2 (when the latter is
restricted to this single binary predicate) but is also metalogically
complete.\index{metalogical completeness}

Let
\begin{itemize}
  \item[] $\mbox{\em CN}=\{\mbox{wff}, \mbox{var}, \vdash, \to, \lnot, (,),\forall,=,R\}$.
  \item[] $\mbox{\em VR}=\{\varphi,\psi,\chi,\ldots\}\cup\{x,y,z,\ldots\}$.
  \item[] $T = \{\langle \mbox{wff\ } \varphi\rangle,
             \langle \mbox{wff\ } \psi\rangle,
             \langle \mbox{wff\ } \chi\rangle,\ldots\}\cup
       \{\langle \mbox{var\ } x\rangle, \langle \mbox{var\ } y\rangle, \langle
       \mbox{var\ }z\rangle,\ldots\}$.

\noindent Then
  $\Gamma$ consists of the reducts of the following pre-statements:
    \begin{itemize}
      \item[] $\langle\varnothing,T,\varnothing,
               \langle \mbox{wff\ }(\varphi\to\psi)\rangle\rangle$
      \item[] $\langle\varnothing,T,\varnothing,
               \langle \mbox{wff\ }\lnot\varphi\rangle\rangle$
      \item[] $\langle\varnothing,T,\varnothing,
               \langle \mbox{wff\ }\forall x\,\varphi\rangle\rangle$
      \item[] $\langle\varnothing,T,\varnothing,
               \langle \mbox{wff\ }x=y\rangle\rangle$
      \item[] $\langle\varnothing,T,\varnothing,
               \langle \mbox{wff\ }Rxy\rangle\rangle$
      \item[(C1$'$)] $\langle\varnothing,T,\varnothing,
               \langle \vdash(\varphi\to(\psi\to\varphi))
               \rangle\rangle$
      \item[(C2$'$)] $\langle\varnothing,T,
               \varnothing,
               \langle \vdash((\varphi\to(\psi\to\chi))\to
               ((\varphi\to\psi)\to(\varphi\to\chi)))
               \rangle\rangle$
      \item[(C3$'$)] $\langle\varnothing,T,
               \varnothing,
               \langle \vdash((\lnot\varphi\to\lnot\psi)\to
               (\psi\to\varphi))\rangle\rangle$
      \item[(C4$'$)] $\langle\varnothing,T,
               \varnothing,
               \langle \vdash(\forall x(\forall x\,\varphi\to\psi)\to
                 (\forall x\,\varphi\to\forall x\,\psi))\rangle\rangle$
      \item[(C5$'$)] $\langle\varnothing,T,
               \varnothing,
               \langle \vdash(\forall x\,\varphi\to\varphi)\rangle\rangle$
      \item[(C6$'$)] $\langle\varnothing,T,
               \varnothing,
               \langle \vdash(\forall x\forall y\,\varphi\to
                 \forall y\forall x\,\varphi)\rangle\rangle$
      \item[(C7$'$)] $\langle\varnothing,T,
               \varnothing,
               \langle \vdash(\lnot\varphi\to\forall x\lnot\forall x\,\varphi
                 )\rangle\rangle$
      \item[(C8$'$)] $\langle\varnothing,T,
               \varnothing,
               \langle \vdash(x=y\to(x=z\to y=z))\rangle\rangle$
      \item[(C9$'$)] $\langle\varnothing,T,
               \varnothing,
               \langle \vdash(\lnot\forall x\, x=y\to(\lnot\forall x\, x=z\to
                 (y=z\to\forall x\, y=z)))\rangle\rangle$
      \item[(C10$'$)] $\langle\varnothing,T,
               \varnothing,
               \langle \vdash(\forall x(x=y\to\forall x\,\varphi)\to
                 \varphi))\rangle\rangle$
      \item[(C11$'$)] $\langle\varnothing,T,
               \varnothing,
               \langle \vdash(\forall x\, x=y\to(\forall x\,\varphi
               \to\forall y\,\varphi))\rangle\rangle$
      \item[(C12$'$)] $\langle\varnothing,T,
               \varnothing,
               \langle \vdash(x=y\to(Rxz\to Ryz))\rangle\rangle$
      \item[(C13$'$)] $\langle\varnothing,T,
               \varnothing,
               \langle \vdash(x=y\to(Rzx\to Rzy))\rangle\rangle$
      \item[(C15$'$)] $\langle\varnothing,T,
               \varnothing,
               \langle \vdash(\lnot\forall x\, x=y\to(x=y\to(\varphi
                 \to\forall x(x=y\to\varphi))))\rangle\rangle$
      \item[(C16$'$)] $\langle\{\{x,y\}\},T,
               \varnothing,
               \langle \vdash(\forall x\, x=y\to(\varphi\to\forall x\,\varphi)
                 )\rangle\rangle$
      \item[(C5)] $\langle\{\{x,\varphi\}\},T,\varnothing,
               \langle \vdash(\varphi\to\forall x\,\varphi)
               \rangle\rangle$
      \item[(MP)] $\langle\varnothing,T,
               \{\langle\vdash(\varphi\to\psi)\rangle,
                 \langle\vdash\varphi\rangle\},
               \langle\vdash\psi\rangle\rangle$
      \item[(Gen)] $\langle\varnothing,T,
               \{\langle\vdash\varphi\rangle\},
               \langle\vdash\forall x\,\varphi\rangle\rangle$
    \end{itemize}
\end{itemize}

While it is known that these axioms are ``metalogically complete,'' it is
not known whether they are independent (i.e.\ none is
redundant) in the metalogical sense; specifically, whether any axiom (possibly
with additional non-mandatory distinct-variable restrictions, for use with any
dummy variables in its proof) is provable from the others.  Note that
metalogical independence is a weaker requirement than independence in the
usual logical sense.  Not all of the above axioms are logically independent:
for example, C9$'$ can be proved as a metatheorem from the others, outside the
formal system, by combining the possible cases of distinct variables.

\subsection{Example~4---Adding Definitions}\index{definition}
There are several ways to add definitions to a formal system.  Probably the
most proper way is to consider definitions not as part of the formal system at
all but rather as abbreviations that are part of the expository metalogic
outside the formal system.  For convenience, though, we may use the formal
system itself to incorporate definitions, adding them as axiomatic extensions
to the system.  This could be done by adding a constant representing the
concept ``is defined as'' along with axioms for it. But there is a nicer way,
at least in this writer's opinion, that introduces definitions as direct
extensions to the language rather than as extralogical primitive notions.  We
introduce additional logical connectives and provide axioms for them.  For
systems of logic such as Examples 1 through 3, the additional axioms must be
conservative in the sense that no wff of the original system that was not a
theorem (when the initial term ``wff'' is replaced by ``$\vdash$'' of course)
becomes a theorem of the extended system.  In this example we extend Example~3
(or 2) with standard abbreviations of logic.

We extend $\mbox{\em CN}$ of Example~3 with new constants $\{\leftrightarrow,
\wedge,\vee,\exists\}$, corresponding to logical equivalence,\index{logical
equivalence ($\leftrightarrow$)}\index{biconditional ($\leftrightarrow$)}
conjunction,\index{conjunction ($\wedge$)} disjunction,\index{disjunction
($\vee$)} and the existential quantifier.\index{existential quantifier
($\exists$)}  We extend $\Gamma$ with the axiomatic statements that are
the reducts of the following pre-statements:
\begin{list}{}{\itemsep 0.0pt}
      \item[] $\langle\varnothing,T,\varnothing,
               \langle \mbox{wff\ }(\varphi\leftrightarrow\psi)\rangle\rangle$
      \item[] $\langle\varnothing,T,\varnothing,
               \langle \mbox{wff\ }(\varphi\vee\psi)\rangle\rangle$
      \item[] $\langle\varnothing,T,\varnothing,
               \langle \mbox{wff\ }(\varphi\wedge\psi)\rangle\rangle$
      \item[] $\langle\varnothing,T,\varnothing,
               \langle \mbox{wff\ }\exists x\, \varphi\rangle\rangle$
  \item[] $\langle\varnothing,T,\varnothing,
     \langle\vdash ( ( \varphi \leftrightarrow \psi ) \to
     ( \varphi \to \psi ) )\rangle\rangle$
  \item[] $\langle\varnothing,T,\varnothing,
     \langle\vdash ((\varphi\leftrightarrow\psi)\to
    (\psi\to\varphi))\rangle\rangle$
  \item[] $\langle\varnothing,T,\varnothing,
     \langle\vdash ((\varphi\to\psi)\to(
     (\psi\to\varphi)\to(\varphi
     \leftrightarrow\psi)))\rangle\rangle$
  \item[] $\langle\varnothing,T,\varnothing,
     \langle\vdash (( \varphi \wedge \psi ) \leftrightarrow\neg ( \varphi
     \to \neg \psi )) \rangle\rangle$
  \item[] $\langle\varnothing,T,\varnothing,
     \langle\vdash (( \varphi \vee \psi ) \leftrightarrow (\neg \varphi
     \to \psi )) \rangle\rangle$
  \item[] $\langle\varnothing,T,\varnothing,
     \langle\vdash (\exists x \,\varphi\leftrightarrow
     \lnot \forall x \lnot \varphi)\rangle\rangle$
\end{list}
The first three logical axioms (statements containing ``$\vdash$'') introduce
and effectively define logical equivalence, ``$\leftrightarrow$''.  The last
three use ``$\leftrightarrow$'' to effectively mean ``is defined as.''

\subsection{Example~5---ZFC Set Theory}\index{ZFC set theory}

Here we add to the system of Example~4 the axioms of Zermelo--Fraenkel set
theory with Choice.  For convenience we make use of the
definitions in Example~4.

In the $\mbox{\em CN}$ of Example~4 (which extends Example~3), we replace the symbol $R$
with the symbol $\in$.
More explicitly, we remove from $\Gamma$ of Example~4 the three
axiomatic statements containing $R$ and replace them with the
reducts of the following:
\begin{list}{}{\itemsep 0.0pt}
      \item[] $\langle\varnothing,T,\varnothing,
               \langle \mbox{wff\ }x\in y\rangle\rangle$
      \item[] $\langle\varnothing,T,
               \varnothing,
               \langle \vdash(x=y\to(x\in z\to y\in z))\rangle\rangle$
      \item[] $\langle\varnothing,T,
               \varnothing,
               \langle \vdash(x=y\to(z\in x\to z\in y))\rangle\rangle$
\end{list}
Letting $D=\{\{\alpha,\beta\}\in \mbox{\em DV}\,|\alpha,\beta\in \mbox{\em
Vr}\}$ (in other words all individual variables must be distinct), we extend
$\Gamma$ with the ZFC axioms, called
\index{Axiom of Extensionality}
\index{Axiom of Replacement}
\index{Axiom of Union}
\index{Axiom of Power Sets}
\index{Axiom of Regularity}
\index{Axiom of Infinity}
\index{Axiom of Choice}
Extensionality, Replacement, Union, Power
Set, Regularity, Infinity, and Choice, that are the reducts of:
\begin{list}{}{\itemsep 0.0pt}
      \item[Ext] $\langle D,T,
               \varnothing,
               \langle\vdash (\forall x(x\in y\leftrightarrow x \in z)\to y
               =z) \rangle\rangle$
      \item[Rep] $\langle D,T,
               \varnothing,
               \langle\vdash\exists x ( \exists y \forall z (\varphi \to z = y
                        ) \to
                        \forall z ( z \in x \leftrightarrow \exists x ( x \in
                        y \wedge \forall y\,\varphi ) ) )\rangle\rangle$
      \item[Un] $\langle D,T,
               \varnothing,
               \langle\vdash \exists x \forall y ( \exists x ( y \in x \wedge
               x \in z ) \to y \in x ) \rangle\rangle$
      \item[Pow] $\langle D,T,
               \varnothing,
               \langle\vdash \exists x \forall y ( \forall x ( x \in y \to x
               \in z ) \to y \in x ) \rangle\rangle$
      \item[Reg] $\langle D,T,
               \varnothing,
               \langle\vdash (  x \in y \to
                 \exists x ( x \in y \wedge \forall z ( z \in x \to \lnot z
                \in y ) ) ) \rangle\rangle$
      \item[Inf] $\langle D,T,
               \varnothing,
               \langle\vdash \exists x(y\in x\wedge\forall y(y\in
               x\to
               \exists z(y \in z\wedge z\in x))) \rangle\rangle$
      \item[AC] $\langle D,T,
               \varnothing,
               \langle\vdash \exists x \forall y \forall z ( ( y \in z
               \wedge z \in w ) \to \exists w \forall y ( \exists w
              ( ( y \in z \wedge z \in w ) \wedge ( y \in w \wedge w \in x
              ) ) \leftrightarrow y = w ) ) \rangle\rangle$
\end{list}

\subsection{Example~6---Class Notation in Set Theory}\label{class}

A powerful device that makes set theory easier (and that we have
been using all along in our informal expository language) is {\em class
abstraction notation}.\index{class abstraction}\index{abstraction class}  The
definitions we introduce are rigorously justified
as conservative by Takeuti and Zaring \cite{Takeuti}\index{Takeuti, Gaisi} or
Quine \cite{Quine}\index{Quine, Willard Van Orman}.  The key idea is to
introduce the notation $\{x|\mbox{---}\}$ which means ``the class of all $x$
such that ---'' for abstraction classes and introduce (meta)variables that
range over them.  An abstraction class may or may not be a set, depending on
whether it exists (as a set).  A class that does not exist is
called a {\em proper class}.\index{proper class}\index{class!proper}

To illustrate the use of abstraction classes we will provide some examples
of definitions that make use of them:  the empty set, class union, and
unordered pair.  Many other such definitions can be found in the
Metamath set theory database,
\texttt{set.mm}.\index{set theory database (\texttt{set.mm})}

% We intentionally break up the sequence of math symbols here
% because otherwise the overlong line goes beyond the page in narrow mode.
We extend $\mbox{\em CN}$ of Example~5 with new symbols $\{$
$\mbox{class},$ $\{,$ $|,$ $\},$ $\varnothing,$ $\cup,$ $,$ $\}$
where the inner braces and last comma are
constant symbols. (As before,
our dual use of some mathematical symbols for both our expository
language and as primitives of the formal system should be clear from context.)

We extend $\mbox{\em VR}$ of Example~5 with a set of {\em class
variables}\index{class variable}
$\{A,B,C,\ldots\}$. We extend the $T$ of Example~5 with $\{\langle
\mbox{class\ } A\rangle, \langle \mbox{class\ }B\rangle, \langle \mbox{class\ }
C\rangle,\ldots\}$.

To
introduce our definitions,
we add to $\Gamma$ of Example~5 the axiomatic statements
that are the reducts of the following pre-statements:
\begin{list}{}{\itemsep 0.0pt}
      \item[] $\langle\varnothing,T,\varnothing,
               \langle \mbox{class\ }x\rangle\rangle$
      \item[] $\langle\varnothing,T,\varnothing,
               \langle \mbox{class\ }\{x|\varphi\}\rangle\rangle$
      \item[] $\langle\varnothing,T,\varnothing,
               \langle \mbox{wff\ }A=B\rangle\rangle$
      \item[] $\langle\varnothing,T,\varnothing,
               \langle \mbox{wff\ }A\in B\rangle\rangle$
      \item[Ab] $\langle\varnothing,T,\varnothing,
               \langle \vdash ( y \in \{ x |\varphi\} \leftrightarrow
                  ( ( x = y \to\varphi) \wedge \exists x ( x = y
                  \wedge\varphi) ))
               \rangle\rangle$
      \item[Eq] $\langle\{\{x,A\},\{x,B\}\},T,\varnothing,
               \langle \vdash ( A = B \leftrightarrow
               \forall x ( x \in A \leftrightarrow x \in B ) )
               \rangle\rangle$
      \item[El] $\langle\{\{x,A\},\{x,B\}\},T,\varnothing,
               \langle \vdash ( A \in B \leftrightarrow \exists x
               ( x = A \wedge x \in B ) )
               \rangle\rangle$
\end{list}
Here we say that an individual variable is a class; $\{x|\varphi\}$ is a
class; and we extend the definition of a wff to include class equality and
membership.  Axiom Ab defines membership of a variable in a class abstraction;
the right-hand side can be read as ``the wff that results from proper
substitution of $y$ for $x$ in $\varphi$.''\footnote{Note that this definition
makes unnecessary the introduction of a separate notation similar to
$\varphi(x|y)$ for proper substitution, although we may choose to do so to be
conventional.  Incidentally, $\varphi(x|y)$ as it stands would be ambiguous in
the formal systems of our examples, since we wouldn't know whether
$\lnot\varphi(x|y)$ meant $\lnot(\varphi(x|y))$ or $(\lnot\varphi)(x|y)$.
Instead, we would have to use an unambiguous variant such as $(\varphi\,
x|y)$.}  Axioms Eq and El extend the meaning of the existing equality and
membership connectives.  This is potentially dangerous and requires careful
justification.  For example, from Eq we can derive the Axiom of Extensionality
with predicate logic alone; thus in principle we should include the Axiom of
Extensionality as a logical hypothesis.  However we do not bother to do this
since we have already presupposed that axiom earlier. The distinct variable
restrictions should be read ``where $x$ does not occur in $A$ or $B$.''  We
typically do this when the right-hand side of a definition involves an
individual variable not in the expression being defined; it is done so that
the right-hand side remains independent of the particular ``dummy'' variable
we use.

We continue to add to $\Gamma$ the following definitions
(i.e. the reducts of the following pre-statements) for empty
set,\index{empty set} class union,\index{union} and unordered
pair.\index{unordered pair}  They should be self-explanatory.  Analogous to our
use of ``$\leftrightarrow$'' to define new wffs in Example~4, we use ``$=$''
to define new abstraction terms, and both may be read informally as ``is
defined as'' in this context.
\begin{list}{}{\itemsep 0.0pt}
      \item[] $\langle\varnothing,T,\varnothing,
               \langle \mbox{class\ }\varnothing\rangle\rangle$
      \item[] $\langle\varnothing,T,\varnothing,
               \langle \vdash \varnothing = \{ x | \lnot x = x \}
               \rangle\rangle$
      \item[] $\langle\varnothing,T,\varnothing,
               \langle \mbox{class\ }(A\cup B)\rangle\rangle$
      \item[] $\langle\{\{x,A\},\{x,B\}\},T,\varnothing,
               \langle \vdash ( A \cup B ) = \{ x | ( x \in A \vee x \in B ) \}
               \rangle\rangle$
      \item[] $\langle\varnothing,T,\varnothing,
               \langle \mbox{class\ }\{A,B\}\rangle\rangle$
      \item[] $\langle\{\{x,A\},\{x,B\}\},T,\varnothing,
               \langle \vdash \{ A , B \} = \{ x | ( x = A \vee x = B ) \}
               \rangle\rangle$
\end{list}

\section{Metamath as a Formal System}\label{theorymm}

This section presupposes a familiarity with the Metamath computer language.

Our theory describes formal systems and their universes.  The Metamath
language provides a way of representing these set-theoretical objects to
a computer.  A Metamath database, being a finite set of {\sc ascii}
characters, can usually describe only a subset of a formal system and
its universe, which are typically infinite.  However the database can
contain as large a finite subset of the formal system and its universe
as we wish.  (Of course a Metamath set theory database can, in
principle, indirectly describe an entire infinite formal system by
formalizing the expository language in this Appendix.)

For purpose of our discussion, we assume the Metamath database
is in the simple form described on p.~\pageref{framelist},
consisting of all constant and variable declarations at the beginning,
followed by a sequence of extended frames each
delimited by \texttt{\$\char`\{} and \texttt{\$\char`\}}.  Any Metamath database can
be converted to this form, as described on p.~\pageref{frameconvert}.

The math symbol tokens of a Metamath source file, which are declared
with \texttt{\$c} and \texttt{\$v} statements, are names we assign to
representatives of $\mbox{\em CN}$ and $\mbox{\em VR}$.  For
definiteness we could assume that the first math symbol declared as a
variable corresponds to $v_0$, the second to $v_1$, etc., although the
exact correspondence we choose is not important.

In the Metamath language, each \texttt{\$d}, \texttt{\$f}, and
 \texttt{\$e} source
statement in an extended frame (Section~\ref{frames})
corresponds respectively to a member of the
collections $D$, $T$, and $H$ in a formal system statement $\langle
D_M,T_M,H,A\rangle$.  The math symbol strings following these Metamath keywords
correspond to a variable pair (in the case of \texttt{\$d}) or an expression (for
the other two keywords). The math symbol string following a \texttt{\$a} source
statement corresponds to expression $A$ in an axiomatic statement of the
formal system; the one following a \texttt{\$p} source statement corresponds to
$A$ in a provable statement that is not axiomatic.  In other words, each
extended frame in a Metamath database corresponds to
a pre-statement of the formal system, and a frame corresponds to
a statement of the formal system.  (Don't confuse the two meanings of
``statement'' here.  A statement of the formal system corresponds to the
several statements in a Metamath database that may constitute a
frame.)

In order for the computer to verify that a formal system statement is
provable, each \texttt{\$p} source statement is accompanied by a proof.
However, the proof does not correspond to anything in the formal system
but is simply a way of communicating to the computer the information
needed for its verification.  The proof tells the computer {\em how to
construct} specific members of closure of the formal system
pre-statement corresponding to the extended frame of the \texttt{\$p}
statement.  The final result of the construction is the member of the
closure that matches the \texttt{\$p} statement.  The abstract formal
system, on the other hand, is concerned only with the {\em existence} of
members of the closure.

As mentioned on p.~\pageref{exampleref}, Examples 1 and 3--6 in the
previous Section parallel the development of logic and set theory in the
Metamath database
\texttt{set.mm}.\index{set theory database (\texttt{set.mm})} You may
find it instructive to compare them.


\chapter{The MIU System}
\label{MIU}
\index{formal system}
\index{MIU-system}

The following is a listing of the file \texttt{miu.mm}.  It is self-explanatory.

%%%%%%%%%%%%%%%%%%%%%%%%%%%%%%%%%%%%%%%%%%%%%%%%%%%%%%%%%%%%

\begin{verbatim}
$( The MIU-system:  A simple formal system $)

$( Note:  This formal system is unusual in that it allows
empty wffs.  To work with a proof, you must type
SET EMPTY_SUBSTITUTION ON before using the PROVE command.
By default, this is OFF in order to reduce the number of
ambiguous unification possibilities that have to be selected
during the construction of a proof.  $)

$(
Hofstadter's MIU-system is a simple example of a formal
system that illustrates some concepts of Metamath.  See
Douglas R. Hofstadter, _Goedel, Escher, Bach:  An Eternal
Golden Braid_ (Vintage Books, New York, 1979), pp. 33ff. for
a description of the MIU-system.

The system has 3 constant symbols, M, I, and U.  The sole
axiom of the system is MI. There are 4 rules:
     Rule I:  If you possess a string whose last letter is I,
     you can add on a U at the end.
     Rule II:  Suppose you have Mx.  Then you may add Mxx to
     your collection.
     Rule III:  If III occurs in one of the strings in your
     collection, you may make a new string with U in place
     of III.
     Rule IV:  If UU occurs inside one of your strings, you
     can drop it.
Unfortunately, Rules III and IV do not have unique results:
strings could have more than one occurrence of III or UU.
This requires that we introduce the concept of an "MIU
well-formed formula" or wff, which allows us to construct
unique symbol sequences to which Rules III and IV can be
applied.
$)

$( First, we declare the constant symbols of the language.
Note that we need two symbols to distinguish the assertion
that a sequence is a wff from the assertion that it is a
theorem; we have arbitrarily chosen "wff" and "|-". $)
      $c M I U |- wff $. $( Declare constants $)

$( Next, we declare some variables. $)
     $v x y $.

$( Throughout our theory, we shall assume that these
variables represent wffs. $)
 wx   $f wff x $.
 wy   $f wff y $.

$( Define MIU-wffs.  We allow the empty sequence to be a
wff. $)

$( The empty sequence is a wff. $)
 we   $a wff $.
$( "M" after any wff is a wff. $)
 wM   $a wff x M $.
$( "I" after any wff is a wff. $)
 wI   $a wff x I $.
$( "U" after any wff is a wff. $)
 wU   $a wff x U $.

$( Assert the axiom. $)
 ax   $a |- M I $.

$( Assert the rules. $)
 ${
   Ia   $e |- x I $.
$( Given any theorem ending with "I", it remains a theorem
if "U" is added after it.  (We distinguish the label I_
from the math symbol I to conform to the 24-Jun-2006
Metamath spec.) $)
   I_    $a |- x I U $.
 $}
 ${
IIa  $e |- M x $.
$( Given any theorem starting with "M", it remains a theorem
if the part after the "M" is added again after it. $)
   II   $a |- M x x $.
 $}
 ${
   IIIa $e |- x I I I y $.
$( Given any theorem with "III" in the middle, it remains a
theorem if the "III" is replaced with "U". $)
   III  $a |- x U y $.
 $}
 ${
   IVa  $e |- x U U y $.
$( Given any theorem with "UU" in the middle, it remains a
theorem if the "UU" is deleted. $)
   IV   $a |- x y $.
  $}

$( Now we prove the theorem MUIIU.  You may be interested in
comparing this proof with that of Hofstadter (pp. 35 - 36).
$)
 theorem1  $p |- M U I I U $=
      we wM wU wI we wI wU we wU wI wU we wM we wI wU we wM
      wI wI wI we wI wI we wI ax II II I_ III II IV $.
\end{verbatim}\index{well-formed formula (wff)}

The \texttt{show proof /lemmon/renumber} command
yields the following display.  It is very similar
to the one in \cite[pp.~35--36]{Hofstadter}.\index{Hofstadter, Douglas R.}

\begin{verbatim}
1 ax             $a |- M I
2 1 II           $a |- M I I
3 2 II           $a |- M I I I I
4 3 I_           $a |- M I I I I U
5 4 III          $a |- M U I U
6 5 II           $a |- M U I U U I U
7 6 IV           $a |- M U I I U
\end{verbatim}

We note that Hofstadter's ``MU-puzzle,'' which asks whether
MU is a theorem of the MIU-system, cannot be answered using
the system above because the MU-puzzle is a question {\em
about} the system.  To prove the answer to the MU-puzzle,
a much more elaborate system is needed, namely one that
models the MIU-system within set theory.  (Incidentally, the
answer to the MU-puzzle is no.)

\chapter{Metamath Language EBNF}%
\label{BNF}%
\index{Metamath Language EBNF}

The following is a formal description of the basic Metamath language syntax
(with compressed proofs and support for unknown proof steps).
It is defined using the
Extended Backus--Naur Form (EBNF)\index{Extended Backus--Naur Form}\index{EBNF}
notation from W3C\index{W3C}
\textit{Extensible Markup Language (XML) 1.0 (Fifth Edition)}
(W3C Recommendation 26 November 2008) at
\url{https://www.w3.org/TR/xml/#sec-notation}.

The \texttt{database}
rule is processed until the end of the file (\texttt{EOF}).
The rules eventually require reading whitespace-separated tokens.
A token has an upper-case definition (see below)
or is a string constant in a non-token (such as \texttt{'\$a'}).
We intend for this to be correct, but if there is a conflict the
rules of section \ref{spec} govern. That section also discusses
non-syntax restrictions not shown here
(e.g., that each new label token
defined in a \texttt{hypothesis-stmt} or \texttt{assert-stmt}
must be unique).

\begin{verbatim}
database ::= outermost-scope-stmt*

outermost-scope-stmt ::=
  include-stmt | constant-stmt | stmt

/* File inclusion command; process file as a database.
   Databases should NOT have a comment in the filename. */
include-stmt ::= '$[' filename '$]'

/* Constant symbols declaration. */
constant-stmt ::= '$c' constant+ '$.'

/* A normal statement can occur in any scope. */
stmt ::= block | variable-stmt | disjoint-stmt |
  hypothesis-stmt | assert-stmt

/* A block. You can have 0 statements in a block. */
block ::= '${' stmt* '$}'

/* Variable symbols declaration. */
variable-stmt ::= '$v' variable+ '$.'

/* Disjoint variables. Simple disjoint statements have
   2 variables, i.e., "variable*" is empty for them. */
disjoint-stmt ::= '$d' variable variable variable* '$.'

hypothesis-stmt ::= floating-stmt | essential-stmt

/* Floating (variable-type) hypothesis. */
floating-stmt ::= LABEL '$f' typecode variable '$.'

/* Essential (logical) hypothesis. */
essential-stmt ::= LABEL '$e' typecode MATH-SYMBOL* '$.'

assert-stmt ::= axiom-stmt | provable-stmt

/* Axiomatic assertion. */
axiom-stmt ::= LABEL '$a' typecode MATH-SYMBOL* '$.'

/* Provable assertion. */
provable-stmt ::= LABEL '$p' typecode MATH-SYMBOL*
  '$=' proof '$.'

/* A proof. Proofs may be interspersed by comments.
   If '?' is in a proof it's an "incomplete" proof. */
proof ::= uncompressed-proof | compressed-proof
uncompressed-proof ::= (LABEL | '?')+
compressed-proof ::= '(' LABEL* ')' COMPRESSED-PROOF-BLOCK+

typecode ::= constant

filename ::= MATH-SYMBOL /* No whitespace or '$' */
constant ::= MATH-SYMBOL
variable ::= MATH-SYMBOL
\end{verbatim}

\needspace{2\baselineskip}
A \texttt{frame} is a sequence of 0 or more
\texttt{disjoint-{\allowbreak}stmt} and
\texttt{hypotheses-{\allowbreak}stmt} statements
(possibly interleaved with other non-\texttt{assert-stmt} statements)
followed by one \texttt{assert-stmt}.

\needspace{3\baselineskip}
Here are the rules for lexical processing (tokenization) beyond
the constant tokens shown above.
By convention these tokenization rules have upper-case names.
Every token is read for the longest possible length.
Whitespace-separated tokens are read sequentially;
note that the separating whitespace and \texttt{\$(} ... \texttt{\$)}
comments are skipped.

If a token definition uses another token definition, the whole thing
is considered a single token.
A pattern that is only part of a full token has a name beginning
with an underscore (``\_'').
An implementation could tokenize many tokens as a
\texttt{PRINTABLE-SEQUENCE}
and then check if it meets the more specific rule shown here.

Comments do not nest, and both \texttt{\$(} and \texttt{\$)}
have to be surrounded
by at least one whitespace character (\texttt{\_WHITECHAR}).
Technically comments end without consuming the trailing
\texttt{\_WHITECHAR}, but the trailing
\texttt{\_WHITECHAR} gets ignored anyway so we ignore that detail here.
Metamath language processors
are not required to support \texttt{\$)} followed
immediately by a bare end-of-file, because the closing
comment symbol is supposed to be followed by a
\texttt{\_WHITECHAR} such as a newline.

\begin{verbatim}
PRINTABLE-SEQUENCE ::= _PRINTABLE-CHARACTER+

MATH-SYMBOL ::= (_PRINTABLE-CHARACTER - '$')+

/* ASCII non-whitespace printable characters */
_PRINTABLE-CHARACTER ::= [#x21-#x7e]

LABEL ::= ( _LETTER-OR-DIGIT | '.' | '-' | '_' )+

_LETTER-OR-DIGIT ::= [A-Za-z0-9]

COMPRESSED-PROOF-BLOCK ::= ([A-Z] | '?')+

/* Define whitespace between tokens. The -> SKIP
   means that when whitespace is seen, it is
   skipped and we simply read again. */
WHITESPACE ::= (_WHITECHAR+ | _COMMENT) -> SKIP

/* Comments. $( ... $) and do not nest. */
_COMMENT ::= '$(' (_WHITECHAR+ (PRINTABLE-SEQUENCE - '$)'))*
  _WHITECHAR+ '$)' _WHITECHAR

/* Whitespace: (' ' | '\t' | '\r' | '\n' | '\f') */
_WHITECHAR ::= [#x20#x09#x0d#x0a#x0c]
\end{verbatim}
% This EBNF was developed as a collaboration between
% David A. Wheeler\index{Wheeler, David A.},
% Mario Carneiro\index{Carneiro, Mario}, and
% Benoit Jubin\index{Jubin, Benoit}, inspired by a request
% (and a lot of initial work) by Benoit Jubin.
%
% \chapter{Disclaimer and Trademarks}
%
% Information in this document is subject to change without notice and does not
% represent a commitment on the part of Norman Megill.
% \vspace{2ex}
%
% \noindent Norman D. Megill makes no warranties, either express or implied,
% regarding the Metamath computer software package.
%
% \vspace{2ex}
%
% \noindent Any trademarks mentioned in this book are the property of
% their respective owners.  The name ``Metamath'' is a trademark of
% Norman Megill.
%
\cleardoublepage
\phantomsection  % fixes the link anchor
\addcontentsline{toc}{chapter}{\bibname}

\bibliography{metamath}
%% metamath.tex - Version of 2-Jun-2019
% If you change the date above, also change the "Printed date" below.
% SPDX-License-Identifier: CC0-1.0
%
%                              PUBLIC DOMAIN
%
% This file (specifically, the version of this file with the above date)
% has been released into the Public Domain per the
% Creative Commons CC0 1.0 Universal (CC0 1.0) Public Domain Dedication
% https://creativecommons.org/publicdomain/zero/1.0/
%
% The public domain release applies worldwide.  In case this is not
% legally possible, the right is granted to use the work for any purpose,
% without any conditions, unless such conditions are required by law.
%
% Several short, attributed quotations from copyrighted works
% appear in this file under the ``fair use'' provision of Section 107 of
% the United States Copyright Act (Title 17 of the {\em United States
% Code}).  The public-domain status of this file is not applicable to
% those quotations.
%
% Norman Megill - email: nm(at)alum(dot)mit(dot)edu
%
% David A. Wheeler also donates his improvements to this file to the
% public domain per the CC0.  He works at the Institute for Defense Analyses
% (IDA), but IDA has agreed that this Metamath work is outside its "lane"
% and is not a work by IDA.  This was specifically confirmed by
% Margaret E. Myers (Division Director of the Information Technology
% and Systems Division) on 2019-05-24 and by Ben Lindorf (General Counsel)
% on 2019-05-22.

% This file, 'metamath.tex', is self-contained with everything needed to
% generate the the PDF file 'metamath.pdf' (the _Metamath_ book) on
% standard LaTeX 2e installations.  The auxiliary files are embedded with
% "filecontents" commands.  To generate metamath.pdf file, run these
% commands under Linux or Cygwin in the directory that contains
% 'metamath.tex':
%
%   rm -f realref.sty metamath.bib
%   touch metamath.ind
%   pdflatex metamath
%   pdflatex metamath
%   bibtex metamath
%   makeindex metamath
%   pdflatex metamath
%   pdflatex metamath
%
% The warnings that occur in the initial runs of pdflatex can be ignored.
% For the final run,
%
%   egrep -i 'error|warn' metamath.log
%
% should show exactly these 5 warnings:
%
%   LaTeX Warning: File `realref.sty' already exists on the system.
%   LaTeX Warning: File `metamath.bib' already exists on the system.
%   LaTeX Font Warning: Font shape `OMS/cmtt/m/n' undefined
%   LaTeX Font Warning: Font shape `OMS/cmtt/bx/n' undefined
%   LaTeX Font Warning: Some font shapes were not available, defaults
%       substituted.
%
% Search for "Uncomment" below if you want to suppress hyperlink boxes
% in the PDF output file
%
% TYPOGRAPHICAL NOTES:
% * It is customary to use an en dash (--) to "connect" names of different
%   people (and to denote ranges), and use a hyphen (-) for a
%   single compound name. Examples of connected multiple people are
%   Zermelo--Fraenkel, Schr\"{o}der--Bernstein, Tarski--Grothendieck,
%   Hewlett--Packard, and Backus--Naur.  Examples of a single person with
%   a compound name include Levi-Civita, Mittag-Leffler, and Burali-Forti.
% * Use non-breaking spaces after page abbreviations, e.g.,
%   p.~\pageref{note2002}.
%
% --------------------------- Start of realref.sty -----------------------------
\begin{filecontents}{realref.sty}
% Save the following as realref.sty.
% You can then use it with \usepackage{realref}
%
% This has \pageref jumping to the page on which the ref appears,
% \ref jumping to the point of the anchor, and \sectionref
% jumping to the start of section.
%
% Author:  Anthony Williams
%          Software Engineer
%          Nortel Networks Optical Components Ltd
% Date:    9 Nov 2001 (posted to comp.text.tex)
%
% The following declaration was made by Anthony Williams on
% 24 Jul 2006 (private email to Norman Megill):
%
%   ``I hereby donate the code for realref.sty posted on the
%   comp.text.tex newsgroup on 9th November 2001, accessible from
%   http://groups.google.com/group/comp.text.tex/msg/5a0e1cc13ea7fbb2
%   to the public domain.''
%
\ProvidesPackage{realref}
\RequirePackage[plainpages=false,pdfpagelabels=true]{hyperref}
\def\realref@anchorname{}
\AtBeginDocument{%
% ensure every label is a possible hyperlink target
\let\realref@oldrefstepcounter\refstepcounter%
\DeclareRobustCommand{\refstepcounter}[1]{\realref@oldrefstepcounter{#1}
\edef\realref@anchorname{\string #1.\@currentlabel}%
}%
\let\realref@oldlabel\label%
\DeclareRobustCommand{\label}[1]{\realref@oldlabel{#1}\hypertarget{#1}{}%
\@bsphack\protected@write\@auxout{}{%
    \string\expandafter\gdef\protect\csname
    page@num.#1\string\endcsname{\thepage}%
    \string\expandafter\gdef\protect\csname
    ref@num.#1\string\endcsname{\@currentlabel}%
    \string\expandafter\gdef\protect\csname
    sectionref@name.#1\string\endcsname{\realref@anchorname}%
}\@esphack}%
\DeclareRobustCommand\pageref[1]{{\edef\a{\csname
            page@num.#1\endcsname}\expandafter\hyperlink{page.\a}{\a}}}%
\DeclareRobustCommand\ref[1]{{\edef\a{\csname
            ref@num.#1\endcsname}\hyperlink{#1}{\a}}}%
\DeclareRobustCommand\sectionref[1]{{\edef\a{\csname
            ref@num.#1\endcsname}\edef\b{\csname
            sectionref@name.#1\endcsname}\hyperlink{\b}{\a}}}%
}
\end{filecontents}
% ---------------------------- End of realref.sty ------------------------------

% --------------------------- Start of metamath.bib -----------------------------
\begin{filecontents}{metamath.bib}
@book{Albers, editor = "Donald J. Albers and G. L. Alexanderson",
  title = "Mathematical People",
  publisher = "Contemporary Books, Inc.",
  address = "Chicago",
  note = "[QA28.M37]",
  year = 1985 }
@book{Anderson, author = "Alan Ross Anderson and Nuel D. Belnap",
  title = "Entailment",
  publisher = "Princeton University Press",
  address = "Princeton",
  volume = 1,
  note = "[QA9.A634 1975 v.1]",
  year = 1975}
@book{Barrow, author = "John D. Barrow",
  title = "Theories of Everything:  The Quest for Ultimate Explanation",
  publisher = "Oxford University Press",
  address = "Oxford",
  note = "[Q175.B225]",
  year = 1991 }
@book{Behnke,
  editor = "H. Behnke and F. Backmann and K. Fladt and W. S{\"{u}}ss",
  title = "Fundamentals of Mathematics",
  volume = "I",
  publisher = "The MIT Press",
  address = "Cambridge, Massachusetts",
  note = "[QA37.2.B413]",
  year = 1974 }
@book{Bell, author = "J. L. Bell and M. Machover",
  title = "A Course in Mathematical Logic",
  publisher = "North-Holland",
  address = "Amsterdam",
  note = "[QA9.B3953]",
  year = 1977 }
@inproceedings{Blass, author = "Andrea Blass",
  title = "The Interaction Between Category Theory and Set Theory",
  pages = "5--29",
  booktitle = "Mathematical Applications of Category Theory (Proceedings
     of the Special Session on Mathematical Applications
     Category Theory, 89th Annual Meeting of the American Mathematical
     Society, held in Denver, Colorado January 5--9, 1983)",
  editor = "John Walter Gray",
  year = 1983,
  note = "[QA169.A47 1983]",
  publisher = "American Mathematical Society",
  address = "Providence, Rhode Island"}
@proceedings{Bledsoe, editor = "W. W. Bledsoe and D. W. Loveland",
  title = "Automated Theorem Proving:  After 25 Years (Proceedings
     of the Special Session on Automatic Theorem Proving,
     89th Annual Meeting of the American Mathematical
     Society, held in Denver, Colorado January 5--9, 1983)",
  year = 1983,
  note = "[QA76.9.A96.S64 1983]",
  publisher = "American Mathematical Society",
  address = "Providence, Rhode Island" }
@book{Boolos, author = "George S. Boolos and Richard C. Jeffrey",
  title = "Computability and Log\-ic",
  publisher = "Cambridge University Press",
  edition = "third",
  address = "Cambridge",
  note = "[QA9.59.B66 1989]",
  year = 1989 }
@book{Campbell, author = "John Campbell",
  title = "Programmer's Progress",
  publisher = "White Star Software",
  address = "Box 51623, Palo Alto, CA 94303",
  year = 1991 }
@article{DBLP:journals/corr/Carneiro14,
  author    = {Mario Carneiro},
  title     = {Conversion of {HOL} Light proofs into Metamath},
  journal   = {CoRR},
  volume    = {abs/1412.8091},
  year      = {2014},
  url       = {http://arxiv.org/abs/1412.8091},
  archivePrefix = {arXiv},
  eprint    = {1412.8091},
  timestamp = {Mon, 13 Aug 2018 16:47:05 +0200},
  biburl    = {https://dblp.org/rec/bib/journals/corr/Carneiro14},
  bibsource = {dblp computer science bibliography, https://dblp.org}
}
@article{CarneiroND,
  author    = {Mario Carneiro},
  title     = {Natural Deductions in the Metamath Proof Language},
  url       = {http://us.metamath.org/ocat/natded.pdf},
  year      = 2014
}
@inproceedings{Chou, author = "Shang-Ching Chou",
  title = "Proving Elementary Geometry Theorems Using {W}u's Algorithm",
  pages = "243--286",
  booktitle = "Automated Theorem Proving:  After 25 Years (Proceedings
     of the Special Session on Automatic Theorem Proving,
     89th Annual Meeting of the American Mathematical
     Society, held in Denver, Colorado January 5--9, 1983)",
  editor = "W. W. Bledsoe and D. W. Loveland",
  year = 1983,
  note = "[QA76.9.A96.S64 1983]",
  publisher = "American Mathematical Society",
  address = "Providence, Rhode Island" }
@book{Clemente, author = "Daniel Clemente Laboreo",
  title = "Introduction to natural deduction",
  year = 2014,
  url = "http://www.danielclemente.com/logica/dn.en.pdf" }
@incollection{Courant, author = "Richard Courant and Herbert Robbins",
  title = "Topology",
  pages = "573--590",
  booktitle = "The World of Mathematics, Volume One",
  editor = "James R. Newman",
  publisher = "Simon and Schuster",
  address = "New York",
  note = "[QA3.W67 1988]",
  year = 1956 }
@book{Curry, author = "Haskell B. Curry",
  title = "Foundations of Mathematical Logic",
  publisher = "Dover Publications, Inc.",
  address = "New York",
  note = "[QA9.C976 1977]",
  year = 1977 }
@book{Davis, author = "Philip J. Davis and Reuben Hersh",
  title = "The Mathematical Experience",
  publisher = "Birkh{\"{a}}user Boston",
  address = "Boston",
  note = "[QA8.4.D37 1982]",
  year = 1981 }
@incollection{deMillo,
  author = "Richard de Millo and Richard Lipton and Alan Perlis",
  title = "Social Processes and Proofs of Theorems and Programs",
  pages = "267--285",
  booktitle = "New Directions in the Philosophy of Mathematics",
  editor = "Thomas Tymoczko",
  publisher = "Birkh{\"{a}}user Boston, Inc.",
  address = "Boston",
  note = "[QA8.6.N48 1986]",
  year = 1986 }
@book{Edwards, author = "Robert E. Edwards",
  title = "A Formal Background to Mathematics",
  publisher = "Springer-Verlag",
  address = "New York",
  note = "[QA37.2.E38 v.1a]",
  year = 1979 }
@book{Enderton, author = "Herbert B. Enderton",
  title = "Elements of Set Theory",
  publisher = "Academic Press, Inc.",
  address = "San Diego",
  note = "[QA248.E5]",
  year = 1977 }
@book{Goodstein, author = "R. L. Goodstein",
  title = "Development of Mathematical Logic",
  publisher = "Springer-Verlag New York Inc.",
  address = "New York",
  note = "[QA9.G6554]",
  year = 1971 }
@book{Guillen, author = "Michael Guillen",
  title = "Bridges to Infinity",
  publisher = "Jeremy P. Tarcher, Inc.",
  address = "Los Angeles",
  note = "[QA93.G8]",
  year = 1983 }
@book{Hamilton, author = "Alan G. Hamilton",
  title = "Logic for Mathematicians",
  edition = "revised",
  publisher = "Cambridge University Press",
  address = "Cambridge",
  note = "[QA9.H298]",
  year = 1988 }
@unpublished{Harrison, author = "John Robert Harrison",
  title = "Metatheory and Reflection in Theorem Proving:
    A Survey and Critique",
  note = "Technical Report
    CRC-053.
    SRI Cambridge,
    Millers Yard, Cambridge, UK,
    1995.
    Available on the Web as
{\verb+http:+}\-{\verb+//www.cl.cam.ac.uk/users/jrh/papers/reflect.html+}"}
@TECHREPORT{Harrison-thesis,
        author          = "John Robert Harrison",
        title           = "Theorem Proving with the Real Numbers",
        institution   = "University of Cambridge Computer
                         Lab\-o\-ra\-to\-ry",
        address         = "New Museums Site, Pembroke Street, Cambridge,
                           CB2 3QG, UK",
        year            = 1996,
        number          = 408,
        type            = "Technical Report",
        note            = "Author's PhD thesis,
   available on the Web at
{\verb+http:+}\-{\verb+//www.cl.cam.ac.uk+}\-{\verb+/users+}\-{\verb+/jrh+}%
\-{\verb+/papers+}\-{\verb+/thesis.html+}"}
@book{Herrlich, author = "Horst Herrlich and George E. Strecker",
  title = "Category Theory:  An Introduction",
  publisher = "Allyn and Bacon Inc.",
  address = "Boston",
  note = "[QA169.H567]",
  year = 1973 }
@article{Hindley, author = "J. Roger Hindley and David Meredith",
  title = "Principal Type-Schemes and Condensed Detachment",
  journal = "The Journal of Symbolic Logic",
  volume = 55,
  year = 1990,
  note = "[QA.J87]",
  pages = "90--105" }
@book{Hofstadter, author = "Douglas R. Hofstadter",
  title = "G{\"{o}}del, Escher, Bach",
  publisher = "Basic Books, Inc.",
  address = "New York",
  note = "[QA9.H63 1980]",
  year = 1979 }
@article{Indrzejczak, author= "Andrzej Indrzejczak",
  title = "Natural Deduction, Hybrid Systems and Modal Logic",
  journal = "Trends in Logic",
  volume = 30,
  publisher = "Springer",
  year = 2010 }
@article{Kalish, author = "D. Kalish and R. Montague",
  title = "On {T}arski's Formalization of Predicate Logic with Identity",
  journal = "Archiv f{\"{u}}r Mathematische Logik und Grundlagenfor\-schung",
  volume = 7,
  year = 1965,
  note = "[QA.A673]",
  pages = "81--101" }
@article{Kalman, author = "J. A. Kalman",
  title = "Condensed Detachment as a Rule of Inference",
  journal = "Studia Logica",
  volume = 42,
  number = 4,
  year = 1983,
  note = "[B18.P6.S933]",
  pages = "443-451" }
@book{Kline, author = "Morris Kline",
  title = "Mathematical Thought from Ancient to Modern Times",
  publisher = "Oxford University Press",
  address = "New York",
  note = "[QA21.K516 1990 v.3]",
  year = 1972 }
@book{Klinel, author = "Morris Kline",
  title = "Mathematics, The Loss of Certainty",
  publisher = "Oxford University Press",
  address = "New York",
  note = "[QA21.K525]",
  year = 1980 }
@book{Kramer, author = "Edna E. Kramer",
  title = "The Nature and Growth of Modern Mathematics",
  publisher = "Princeton University Press",
  address = "Princeton, New Jersey",
  note = "[QA93.K89 1981]",
  year = 1981 }
@article{Knill, author = "Oliver Knill",
  title = "Some Fundamental Theorems in Mathematics",
  year = "2018",
  url = "https://arxiv.org/abs/1807.08416" }
@book{Landau, author = "Edmund Landau",
  title = "Foundations of Analysis",
  publisher = "Chelsea Publishing Company",
  address = "New York",
  edition = "second",
  note = "[QA241.L2541 1960]",
  year = 1960 }
@article{Leblanc, author = "Hugues Leblanc",
  title = "On {M}eyer and {L}ambert's Quantificational Calculus {FQ}",
  journal = "The Journal of Symbolic Logic",
  volume = 33,
  year = 1968,
  note = "[QA.J87]",
  pages = "275--280" }
@article{Lejewski, author = "Czeslaw Lejewski",
  title = "On Implicational Definitions",
  journal = "Studia Logica",
  volume = 8,
  year = 1958,
  note = "[B18.P6.S933]",
  pages = "189--208" }
@book{Levy, author = "Azriel Levy",
  title = "Basic Set Theory",
  publisher = "Dover Publications",
  address = "Mineola, NY",
  year = "2002"
}
@book{Margaris, author = "Angelo Margaris",
  title = "First Order Mathematical Logic",
  publisher = "Blaisdell Publishing Company",
  address = "Waltham, Massachusetts",
  note = "[QA9.M327]",
  year = 1967}
@book{Manin, author = "Yu I. Manin",
  title = "A Course in Mathematical Logic",
  publisher = "Springer-Verlag",
  address = "New York",
  note = "[QA9.M29613]",
  year = "1977" }
@article{Mathias, author = "Adrian R. D. Mathias",
  title = "A Term of Length 4,523,659,424,929",
  journal = "Synthese",
  volume = 133,
  year = 2002,
  note = "[Q.S993]",
  pages = "75--86" }
@article{Megill, author = "Norman D. Megill",
  title = "A Finitely Axiomatized Formalization of Predicate Calculus
     with Equality",
  journal = "Notre Dame Journal of Formal Logic",
  volume = 36,
  year = 1995,
  note = "[QA.N914]",
  pages = "435--453" }
@unpublished{Megillc, author = "Norman D. Megill",
  title = "A Shorter Equivalent of the Axiom of Choice",
  month = "June",
  note = "Unpublished",
  year = 1991 }
@article{MegillBunder, author = "Norman D. Megill and Martin W.
    Bunder",
  title = "Weaker {D}-Complete Logics",
  journal = "Journal of the IGPL",
  volume = 4,
  year = 1996,
  pages = "215--225",
  note = "Available on the Web at
{\verb+http:+}\-{\verb+//www.mpi-sb.mpg.de+}\-{\verb+/igpl+}%
\-{\verb+/Journal+}\-{\verb+/V4-2+}\-{\verb+/#Megill+}"}
}
@book{Mendelson, author = "Elliott Mendelson",
  title = "Introduction to Mathematical Logic",
  edition = "second",
  publisher = "D. Van Nostrand Company, Inc.",
  address = "New York",
  note = "[QA9.M537 1979]",
  year = 1979 }
@article{Meredith, author = "David Meredith",
  title = "In Memoriam {C}arew {A}rthur {M}eredith (1904-1976)",
  journal = "Notre Dame Journal of Formal Logic",
  volume = 18,
  year = 1977,
  note = "[QA.N914]",
  pages = "513--516" }
@article{CAMeredith, author = "C. A. Meredith",
  title = "Single Axioms for the Systems ({C},{N}), ({C},{O}) and ({A},{N})
      of the Two-Valued Propositional Calculus",
  journal = "The Journal of Computing Systems",
  volume = 3,
  year = 1953,
  pages = "155--164" }
@article{Monk, author = "J. Donald Monk",
  title = "Provability With Finitely Many Variables",
  journal = "The Journal of Symbolic Logic",
  volume = 27,
  year = 1971,
  note = "[QA.J87]",
  pages = "353--358" }
@article{Monks, author = "J. Donald Monk",
  title = "Substitutionless Predicate Logic With Identity",
  journal = "Archiv f{\"{u}}r Mathematische Logik und Grundlagenfor\-schung",
  volume = 7,
  year = 1965,
  pages = "103--121" }
  %% Took out this from above to prevent LaTeX underfull warning:
  % note = "[QA.A673]",
@book{Moore, author = "A. W. Moore",
  title = "The Infinite",
  publisher = "Routledge",
  address = "New York",
  note = "[BD411.M59]",
  year = 1989}
@book{Munkres, author = "James R. Munkres",
  title = "Topology: A First Course",
  publisher = "Prentice-Hall, Inc.",
  address = "Englewood Cliffs, New Jersey",
  note = "[QA611.M82]",
  year = 1975}
@article{Nemesszeghy, author = "E. Z. Nemesszeghy and E. A. Nemesszeghy",
  title = "On Strongly Creative Definitions:  A Reply to {V}. {F}. {R}ickey",
  journal = "Logique et Analyse (N.\ S.)",
  year = 1977,
  volume = 20,
  note = "[BC.L832]",
  pages = "111--115" }
@unpublished{Nemeti, author = "N{\'{e}}meti, I.",
  title = "Algebraizations of Quantifier Logics, an Overview",
  note = "Version 11.4, preprint, Mathematical Institute, Budapest,
    1994.  A shortened version without proofs appeared in
    ``Algebraizations of quantifier logics, an introductory overview,''
   {\em Studia Logica}, 50:485--569, 1991 [B18.P6.S933]"}
@article{Pavicic, author = "M. Pavi{\v{c}}i{\'{c}}",
  title = "A New Axiomatization of Unified Quantum Logic",
  journal = "International Journal of Theoretical Physics",
  year = 1992,
  volume = 31,
  note = "[QC.I626]",
  pages = "1753 --1766" }
@book{Penrose, author = "Roger Penrose",
  title = "The Emperor's New Mind",
  publisher = "Oxford University Press",
  address = "New York",
  note = "[Q335.P415]",
  year = 1989 }
@book{PetersonI, author = "Ivars Peterson",
  title = "The Mathematical Tourist",
  publisher = "W. H. Freeman and Company",
  address = "New York",
  note = "[QA93.P475]",
  year = 1988 }
@article{Peterson, author = "Jeremy George Peterson",
  title = "An automatic theorem prover for substitution and detachment systems",
  journal = "Notre Dame Journal of Formal Logic",
  volume = 19,
  year = 1978,
  note = "[QA.N914]",
  pages = "119--122" }
@book{Quine, author = "Willard Van Orman Quine",
  title = "Set Theory and Its Logic",
  edition = "revised",
  publisher = "The Belknap Press of Harvard University Press",
  address = "Cambridge, Massachusetts",
  note = "[QA248.Q7 1969]",
  year = 1969 }
@article{Robinson, author = "J. A. Robinson",
  title = "A Machine-Oriented Logic Based on the Resolution Principle",
  journal = "Journal of the Association for Computing Machinery",
  year = 1965,
  volume = 12,
  pages = "23--41" }
@article{RobinsonT, author = "T. Thacher Robinson",
  title = "Independence of Two Nice Sets of Axioms for the Propositional
    Calculus",
  journal = "The Journal of Symbolic Logic",
  volume = 33,
  year = 1968,
  note = "[QA.J87]",
  pages = "265--270" }
@book{Rucker, author = "Rudy Rucker",
  title = "Infinity and the Mind:  The Science and Philosophy of the
    Infinite",
  publisher = "Bantam Books, Inc.",
  address = "New York",
  note = "[QA9.R79 1982]",
  year = 1982 }
@book{Russell, author = "Bertrand Russell",
  title = "Mysticism and Logic, and Other Essays",
  publisher = "Barnes \& Noble Books",
  address = "Totowa, New Jersey",
  note = "[B1649.R963.M9 1981]",
  year = 1981 }
@article{Russell2, author = "Bertrand Russell",
  title = "Recent Work on the Principles of Mathematics",
  journal = "International Monthly",
  volume = 4,
  year = 1901,
  pages = "84"}
@article{Schmidt, author = "Eric Schmidt",
  title = "Reductions in Norman Megill's axiom system for complex numbers",
  url = "http://us.metamath.org/downloads/schmidt-cnaxioms.pdf",
  year = "2012" }
@book{Shoenfield, author = "Joseph R. Shoenfield",
  title = "Mathematical Logic",
  publisher = "Addison-Wesley Publishing Company, Inc.",
  address = "Reading, Massachusetts",
  year = 1967,
  note = "[QA9.S52]" }
@book{Smullyan, author = "Raymond M. Smullyan",
  title = "Theory of Formal Systems",
  publisher = "Princeton University Press",
  address = "Princeton, New Jersey",
  year = 1961,
  note = "[QA248.5.S55]" }
@book{Solow, author = "Daniel Solow",
  title = "How to Read and Do Proofs:  An Introduction to Mathematical
    Thought Process",
  publisher = "John Wiley \& Sons",
  address = "New York",
  year = 1982,
  note = "[QA9.S577]" }
@book{Stark, author = "Harold M. Stark",
  title = "An Introduction to Number Theory",
  publisher = "Markham Publishing Company",
  address = "Chicago",
  note = "[QA241.S72 1978]",
  year = 1970 }
@article{Swart, author = "E. R. Swart",
  title = "The Philosophical Implications of the Four-Color Problem",
  journal = "American Mathematical Monthly",
  year = 1980,
  volume = 87,
  month = "November",
  note = "[QA.A5125]",
  pages = "697--707" }
@book{Szpiro, author = "George G. Szpiro",
  title = "Poincar{\'{e}}'s Prize: The Hundred-Year Quest to Solve One
    of Math's Greatest Puzzles",
  publisher = "Penguin Books Ltd",
  address = "London",
  note = "[QA43.S985 2007]",
  year = 2007}
@book{Takeuti, author = "Gaisi Takeuti and Wilson M. Zaring",
  title = "Introduction to Axiomatic Set Theory",
  edition = "second",
  publisher = "Springer-Verlag New York Inc.",
  address = "New York",
  note = "[QA248.T136 1982]",
  year = 1982}
@inproceedings{Tarski, author = "Alfred Tarski",
  title = "What is Elementary Geometry",
  pages = "16--29",
  booktitle = "The Axiomatic Method, with Special Reference to Geometry and
     Physics (Proceedings of an International Symposium held at the University
     of California, Berkeley, December 26, 1957 --- January 4, 1958)",
  editor = "Leon Henkin and Patrick Suppes and Alfred Tarski",
  year = 1959,
  publisher = "North-Holland Publishing Company",
  address = "Amsterdam"}
@article{Tarski1965, author = "Alfred Tarski",
  title = "A Simplified Formalization of Predicate Logic with Identity",
  journal = "Archiv f{\"{u}}r Mathematische Logik und Grundlagenforschung",
  volume = 7,
  year = 1965,
  note = "[QA.A673]",
  pages = "61--79" }
@book{Tymoczko,
  title = "New Directions in the Philosophy of Mathematics",
  editor = "Thomas Tymoczko",
  publisher = "Birkh{\"{a}}user Boston, Inc.",
  address = "Boston",
  note = "[QA8.6.N48 1986]",
  year = 1986 }
@incollection{Wang,
  author = "Hao Wang",
  title = "Theory and Practice in Mathematics",
  pages = "129--152",
  booktitle = "New Directions in the Philosophy of Mathematics",
  editor = "Thomas Tymoczko",
  publisher = "Birkh{\"{a}}user Boston, Inc.",
  address = "Boston",
  note = "[QA8.6.N48 1986]",
  year = 1986 }
@manual{Webster,
  title = "Webster's New Collegiate Dictionary",
  organization = "G. \& C. Merriam Co.",
  address = "Springfield, Massachusetts",
  note = "[PE1628.W4M4 1977]",
  year = 1977 }
@manual{Whitehead, author = "Alfred North Whitehead",
  title = "An Introduction to Mathematics",
  year = 1911 }
@book{PM, author = "Alfred North Whitehead and Bertrand Russell",
  title = "Principia Mathematica",
  edition = "second",
  publisher = "Cambridge University Press",
  address = "Cambridge",
  year = "1927",
  note = "(3 vols.) [QA9.W592 1927]" }
@article{DBLP:journals/corr/Whalen16,
  author    = {Daniel Whalen},
  title     = {Holophrasm: a neural Automated Theorem Prover for higher-order logic},
  journal   = {CoRR},
  volume    = {abs/1608.02644},
  year      = {2016},
  url       = {http://arxiv.org/abs/1608.02644},
  archivePrefix = {arXiv},
  eprint    = {1608.02644},
  timestamp = {Mon, 13 Aug 2018 16:46:19 +0200},
  biburl    = {https://dblp.org/rec/bib/journals/corr/Whalen16},
  bibsource = {dblp computer science bibliography, https://dblp.org} }
@article{Wiedijk-revisited,
  author = {Freek Wiedijk},
  title = {The QED Manifesto Revisited},
  year = {2007},
  url = {http://mizar.org/trybulec65/8.pdf} }
@book{Wolfram,
  author = "Stephen Wolfram",
  title = "Mathematica:  A System for Doing Mathematics by Computer",
  edition = "second",
  publisher = "Addison-Wesley Publishing Co.",
  address = "Redwood City, California",
  note = "[QA76.95.W65 1991]",
  year = 1991 }
@book{Wos, author = "Larry Wos and Ross Overbeek and Ewing Lusk and Jim Boyle",
  title = "Automated Reasoning:  Introduction and Applications",
  edition = "second",
  publisher = "McGraw-Hill, Inc.",
  address = "New York",
  note = "[QA76.9.A96.A93 1992]",
  year = 1992 }

%
%
%[1] Church, Alonzo, Introduction to Mathematical Logic,
% Volume 1, Princeton University Press, Princeton, N. J., 1956.
%
%[2] Cohen, Paul J., Set Theory and the Continuum Hypothesis,
% W. A. Benjamin, Inc., Reading, Mass., 1966.
%
%[3] Hamilton, Alan G., Logic for Mathematicians, Cambridge
% University Press,
% Cambridge, 1988.

%[6] Kleene, Stephen Cole, Introduction to Metamathematics, D.  Van
% Nostrand Company, Inc., Princeton (1952).

%[13] Tarski, Alfred, "A simplified formalization of predicate
% logic with identity," Archiv fur Mathematische Logik und
% Grundlagenforschung, vol. 7 (1965), pp. 61-79.

%[14] Tarski, Alfred and Steven Givant, A Formalization of Set
% Theory Without Variables, American Mathematical Society Colloquium
% Publications, vol. 41, American Mathematical Society,
% Providence, R. I., 1987.

%[15] Zeman, J. J., Modal Logic, Oxford University Press, Oxford, 1973.
\end{filecontents}
% --------------------------- End of metamath.bib -----------------------------


%Book: Metamath
%Author:  Norman Megill Email:  nm at alum.mit.edu
%Author:  David A. Wheeler Email:  dwheeler at dwheeler.com

% A book template example
% http://www.stsci.edu/ftp/software/tex/bookstuff/book.template

\documentclass[leqno]{book} % LaTeX 2e. 10pt. Use [leqno,12pt] for 12pt
% hyperref 2002/05/27 v6.72r  (couldn't get pagebackref to work)
\usepackage[plainpages=false,pdfpagelabels=true]{hyperref}

\usepackage{needspace}     % Enable control over page breaks
\usepackage{breqn}         % automatic equation breaking
\usepackage{microtype}     % microtypography, reduces hyphenation

% Packages for flexible tables.  We need to be able to
% wrap text within a cell (with automatically-determined widths) AND
% split a table automatically across multiple pages.
% * "tabularx" wraps text in cells but only 1 page
% * "longtable" goes across pages but by itself is incompatible with tabularx
% * "ltxtable" combines longtable and tabularx, but table contents
%    must be in a separate file.
% * "ltablex" combines tabularx and longtable - must install specially
% * "booktabs" is recommended as a way to improve the look of tables,
%   but doesn't add these capabilities.
% * "tabu" much more capable and seems to be recommended. So use that.

\usepackage{makecell}      % Enable forced line splits within a table cell
% v4.13 needed for tabu: https://tex.stackexchange.com/questions/600724/dimension-too-large-after-recent-longtable-update
\usepackage{longtable}[=v4.13] % Enable multi-page tables  
\usepackage{tabu}          % Multi-page tables with wrapped text in a cell

% You can find more Tex packages using commands like:
% tlmgr search --file tabu.sty
% find /usr/share/texmf-dist/ -name '*tab*'
%
%%%%%%%%%%%%%%%%%%%%%%%%%%%%%%%%%%%%%%%%%%%%%%%%%%%%%%%%%%%%%%%%%%%%%%%%%%%%
% Uncomment the next 3 lines to suppress boxes and colors on the hyperlinks
%%%%%%%%%%%%%%%%%%%%%%%%%%%%%%%%%%%%%%%%%%%%%%%%%%%%%%%%%%%%%%%%%%%%%%%%%%%%
%\hypersetup{
%colorlinks,citecolor=black,filecolor=black,linkcolor=black,urlcolor=black
%}
%
\usepackage{realref}

% Restarting page numbers: try?
%   \printglossary
%   \cleardoublepage
%   \pagenumbering{arabic}
%   \setcounter{page}{1}    ???needed
%   \include{chap1}

% not used:
% \def\R2Lurl#1#2{\mbox{\href{#1}\texttt{#2}}}

\usepackage{amssymb}

% Version 1 of book: margins: t=.4, b=.2, ll=.4, rr=.55
% \usepackage{anysize}
% % \papersize{<height>}{<width>}
% % \marginsize{<left>}{<right>}{<top>}{<bottom>}
% \papersize{9in}{6in}
% % l/r 0.6124-0.6170 works t/b 0.2418-0.3411 = 192pp. 0.2926-03118=exact
% \marginsize{0.7147in}{0.5147in}{0.4012in}{0.2012in}

\usepackage{anysize}
% \papersize{<height>}{<width>}
% \marginsize{<left>}{<right>}{<top>}{<bottom>}
\papersize{9in}{6in}
% l/r 0.85in&0.6431-0.6539 works t/b ?-?
%\marginsize{0.85in}{0.6485in}{0.55in}{0.35in}
\marginsize{0.8in}{0.65in}{0.5in}{0.3in}

% \usepackage[papersize={3.6in,4.8in},hmargin=0.1in,vmargin={0.1in,0.1in}]{geometry}  % page geometry
\usepackage{special-settings}

\raggedbottom
\makeindex

\begin{document}
% Discourage page widows and orphans:
\clubpenalty=300
\widowpenalty=300

%%%%%%% load in AMS fonts %%%%%%% % LaTeX 2.09 - obsolete in LaTeX 2e
%\input{amssym.def}
%\input{amssym.tex}
%\input{c:/texmf/tex/plain/amsfonts/amssym.def}
%\input{c:/texmf/tex/plain/amsfonts/amssym.tex}

\bibliographystyle{plain}
\pagenumbering{roman}
\pagestyle{headings}

\thispagestyle{empty}

\hfill
\vfill

\begin{center}
{\LARGE\bf Metamath} \\
\vspace{1ex}
{\large A Computer Language for Mathematical Proofs} \\
\vspace{7ex}
{\large Norman Megill} \\
\vspace{7ex}
with extensive revisions by \\
\vspace{1ex}
{\large David A. Wheeler} \\
\vspace{7ex}
% Printed date. If changing the date below, also fix the date at the beginning.
2019-06-02
\end{center}

\vfill
\hfill

\newpage
\thispagestyle{empty}

\hfill
\vfill

\begin{center}
$\sim$\ {\sc Public Domain}\ $\sim$

\vspace{2ex}
This book (including its later revisions)
has been released into the Public Domain by Norman Megill per the
Creative Commons CC0 1.0 Universal (CC0 1.0) Public Domain Dedication.
David A. Wheeler has done the same.
This public domain release applies worldwide.  In case this is not
legally possible, the right is granted to use the work for any purpose,
without any conditions, unless such conditions are required by law.
See \url{https://creativecommons.org/publicdomain/zero/1.0/}.

\vspace{3ex}
Several short, attributed quotations from copyrighted works
appear in this book under the ``fair use'' provision of Section 107 of
the United States Copyright Act (Title 17 of the {\em United States
Code}).  The public-domain status of this book is not applicable to
those quotations.

\vspace{3ex}
Any trademarks used in this book are the property of their owners.

% QA76.9.L63.M??

% \vspace{1ex}
%
% \vspace{1ex}
% {\small Permission is granted to make and distribute verbatim copies of this
% book
% provided the copyright notice and this
% permission notice are preserved on all copies.}
%
% \vspace{1ex}
% {\small Permission is granted to copy and distribute modified versions of this
% book under the conditions for verbatim copying, provided that the
% entire
% resulting derived work is distributed under the terms of a permission
% notice
% identical to this one.}
%
% \vspace{1ex}
% {\small Permission is granted to copy and distribute translations of this
% book into another language, under the above conditions for modified
% versions,
% except that this permission notice may be stated in a translation
% approved by the
% author.}
%
% \vspace{1ex}
% %{\small   For a copy of the \LaTeX\ source files for this book, contact
% %the author.} \\
% \ \\
% \ \\

\vspace{7ex}
% ISBN: 1-4116-3724-0 \\
% ISBN: 978-1-4116-3724-5 \\
ISBN: 978-0-359-70223-7 \\
{\ } \\
Lulu Press \\
Morrisville, North Carolina\\
USA


\hfill
\vfill

Norman Megill\\ 93 Bridge St., Lexington, MA 02421 \\
E-mail address: \texttt{nm{\char`\@}alum.mit.edu} \\
\vspace{7ex}
David A. Wheeler \\
E-mail address: \texttt{dwheeler{\char`\@}dwheeler.com} \\
% See notes added at end of Preface for revision history. \\
% For current information on the Metamath software see \\
\vspace{7ex}
\url{http://metamath.org}
\end{center}

\hfill
\vfill

{\parindent0pt%
\footnotesize{%
Cover: Aleph null ($\aleph_0$) is the symbol for the
first infinite cardinal number, discovered by Georg Cantor in 1873.
We use a red aleph null (with dark outline and gold glow) as the Metamath logo.
Credit: Norman Megill (1994) and Giovanni Mascellani (2019),
public domain.%
\index{aleph null}%
\index{Metamath!logo}\index{Cantor, Georg}\index{Mascellani, Giovanni}}}

% \newpage
% \thispagestyle{empty}
%
% \hfill
% \vfill
%
% \begin{center}
% {\it To my son Robin Dwight Megill}
% \end{center}
%
% \vfill
% \hfill
%
% \newpage

\tableofcontents
%\listoftables

\chapter*{Preface}
\markboth{PREFACE}{PREFACE}
\addcontentsline{toc}{section}{Preface}


% (For current information, see the notes added at the
% end of this preface on p.~\pageref{note2002}.)

\subsubsection{Overview}

Metamath\index{Metamath} is a computer language and an associated computer
program for archiving, verifying, and studying mathematical proofs at a very
detailed level.  The Metamath language incorporates no mathematics per se but
treats all mathematical statements as mere sequences of symbols.  You provide
Metamath with certain special sequences (axioms) that tell it what rules
of inference are allowed.  Metamath is not limited to any specific field of
mathematics.  The Metamath language is simple and robust, with an
almost total absence of hard-wired syntax, and
we\footnote{Unless otherwise noted, the words
``I,'' ``me,'' and ``my'' refer to Norman Megill\index{Megill, Norman}, while
``we,'' ``us,'' and ``our'' refer to Norman Megill and
David A. Wheeler\index{Wheeler, David A.}.}
believe that it
provides about the simplest possible framework that allows essentially all of
mathematics to be expressed with absolute rigor.

% index test
%\newcommand{\nn}[1]{#1n}
%\index{aaa@bbb}
%\index{abc!def}
%\index{abd|see{qqq}}
%\index{abe|nn}
%\index{abf|emph}
%\index{abg|(}
%\index{abg|)}

Using the Metamath language, you can build formal or mathematical
systems\index{formal system}\footnote{A formal or mathematical system consists
of a collection of symbols (such as $2$, $4$, $+$ and $=$), syntax rules that
describe how symbols may be combined to form a legal expression (called a
well-formed formula or {\em wff}, pronounced ``whiff''), some starting wffs
called axioms, and inference rules that describe how theorems may be derived
(proved) from the axioms.  A theorem is a mathematical fact such as $2+2=4$.
Strictly speaking, even an obvious fact such as this must be proved from
axioms to be formally acceptable to a mathematician.}\index{theorem}
\index{axiom}\index{rule}\index{well-formed formula (wff)} that involve
inferences from axioms.  Although a database is provided
that includes a recommended set of axioms for standard mathematics, if you
wish you can supply your own symbols, syntax, axioms, rules, and definitions.

The name ``Metamath'' was chosen to suggest that the language provides a
means for {\em describing} mathematics rather than {\em being} the
mathematics itself.  Actually in some sense any mathematical language is
metamathematical.  Symbols written on paper, or stored in a computer,
are not mathematics itself but rather a way of expressing mathematics.
For example ``7'' and ``VII'' are symbols for denoting the number seven
in Arabic and Roman numerals; neither {\em is} the number seven.

If you are able to understand and write computer programs, you should be able
to follow abstract mathematics with the aid of Metamath.  Used in conjunction
with standard textbooks, Metamath can guide you step-by-step towards an
understanding of abstract mathematics from a very rigorous viewpoint, even if
you have no formal abstract mathematics background.  By using a single,
consistent notation to express proofs, once you grasp its basic concepts
Metamath provides you with the ability to immediately follow and dissect
proofs even in totally unfamiliar areas.

Of course, just being able follow a proof will not necessarily give you an
intuitive familiarity with mathematics.  Memorizing the rules of chess does not
give you the ability to appreciate the game of a master, and knowing how the
notes on a musical score map to piano keys does not give you the ability to
hear in your head how it would sound.  But each of these can be a first step.

Metamath allows you to explore proofs in the sense that you can see the
theorem referenced at any step expanded in as much detail as you want, right
down to the underlying axioms of logic and set theory (in the case of the set
theory database provided).  While Metamath will not replace the higher-level
understanding that can only be acquired through exercises and hard work, being
able to see how gaps in a proof are filled in can give you increased
confidence that can speed up the learning process and save you time when you
get stuck.

The Metamath language breaks down a mathematical proof into its tiniest
possible parts.  These can be pieced together, like interlocking
pieces in a puzzle, only in a way that produces correct and absolutely rigorous
mathematics.

The nature of Metamath\index{Metamath} enforces very precise mathematical
thinking, similar to that involved in writing a computer program.  A crucial
difference, though, is that once a proof is verified (by the Metamath program)
to be correct, it is definitely correct; it can never have a hidden
``bug.''\index{computer program bugs}  After getting used to the kind of rigor
and accuracy provided by Metamath, you might even be tempted to
adopt the attitude that a proof should never be considered correct until it
has been verified by a computer, just as you would not completely trust a
manual calculation until you have verified it on a
calculator.

My goal
for Metamath was a system for describing and verifying
mathematics that is completely universal yet conceptually as simple as
possible.  In approaching mathematics from an axiomatic, formal viewpoint, I
wanted Metamath to be able to handle almost any mathematical system, not
necessarily with ease, but at least in principle and hopefully in practice. I
wanted it to verify proofs with absolute rigor, and for this reason Metamath
is what might be thought of as a ``compile-only'' language rather than an
algorithmic or Turing-machine language (Pascal, C, Prolog, Mathematica,
etc.).  In other words, a database written in the Metamath
language doesn't ``do'' anything; it merely exhibits mathematical knowledge
and permits this knowledge to be verified as being correct.  A program in an
algorithmic language can potentially have hidden bugs\index{computer program
bugs} as well as possibly being hard to understand.  But each token in a
Metamath database must be consistent with the database's earlier
contents according to simple, fixed rules.
If a database is verified
to be correct,\footnote{This includes
verification that a sequential list of proof steps results in the specified
theorem.} then the mathematical content is correct if the
verifier is correct and the axioms are correct.
The verification program could be incorrect, but the verification algorithm
is relatively simple (making it unlikely to be implemented incorrectly
by the Metamath program),
and there are over a dozen Metamath database verifiers
written by different people in different programming languages
(so these different verifiers can act as multiple reviewers of a database).
The most-used Metamath database, the Metamath Proof Explorer
(aka \texttt{set.mm}\index{set theory database (\texttt{set.mm})}%
\index{Metamath Proof Explorer}),
is currently verified by four different Metamath verifiers written by
four different people in four different languages, including the
original Metamath program described in this book.
The only ``bugs'' that can exist are in the statement of the axioms,
for example if the axioms are inconsistent (a famous problem shown to be
unsolvable by G\"{o}del's incompleteness theorem\index{G\"{o}del's
incompleteness theorem}).
However, real mathematical systems have very few axioms, and these can
be carefully studied.
All of this provides extraordinarily high confidence that the verified database
is in fact correct.

The Metamath program
doesn't prove theorems automatically but is designed to verify proofs
that you supply to it.
The underlying Metamath language is completely general and has no built-in,
preconceived notions about your formal system\index{formal system}, its logic
or its syntax.
For constructing proofs, the Metamath program has a Proof Assistant\index{Proof
Assistant} which helps you fill in some of a proof step's details, shows you
what choices you have at any step, and verifies the proof as you build it; but
you are still expected to provide the proof.

There are many other programs that can process or generate information
in the Metamath language, and more continue to be written.
This is in part because the Metamath language itself is very simple
and intentionally easy to automatically process.
Some programs, such as \texttt{mmj2}\index{mmj2}, include a proof assistant
that can automate some steps beyond what the Metamath program can do.
Mario Carneiro has developed an algorithm for converting proofs from
the OpenTheory interchange format, which can be translated to and from
any of the HOL family of proof languages (HOL4, HOL Light, ProofPower,
and Isabelle), into the
Metamath language \cite{DBLP:journals/corr/Carneiro14}\index{Carneiro, Mario}.
Daniel Whalen has developed Holophrasm, which can automatically
prove many Metamath proofs using
machine learning\index{machine learning}\index{artificial intelligence}
approaches
(including multiple neural networks\index{neural networks})\cite{DBLP:journals/corr/Whalen16}\index{Whalen, Daniel}.
However,
a discussion of these other programs is beyond the scope of this book.

Like most computer languages, the Metamath\index{Metamath} language uses the
standard ({\sc ascii}) characters on a computer keyboard, so it cannot
directly represent many of the special symbols that mathematicians use.  A
useful feature of the Metamath program is its ability to convert its notation
into the \LaTeX\ typesetting language.\index{latex@{\LaTeX}}  This feature
lets you convert the {\sc ascii} tokens you've defined into standard
mathematical symbols, so you end up with symbols and formulas you are familiar
with instead of somewhat cryptic {\sc ascii} representations of them.
The Metamath program can also generate HTML\index{HTML}, making it easy
to view results on the web and to see related information by using
hypertext links.

Metamath is probably conceptually different from anything you've seen
before and some aspects may take some getting used to.  This book will
help you decide whether Metamath suits your specific needs.

\subsubsection{Setting Your Expectations}
It is important for you to understand what Metamath\index{Metamath} is and is
not.  As mentioned, the Metamath program
is {\em not} an automated theorem prover but
rather a proof verifier.  Developing a database can be tedious, hard work,
especially if you want to make the proofs as short as possible, but it becomes
easier as you build up a collection of useful theorems.  The purpose of
Metamath is simply to document existing mathematics in an absolutely rigorous,
computer-verifiable way, not to aid directly in the creation of new
mathematics.  It also is not a magic solution for learning abstract
mathematics, although it may be helpful to be able to actually see the implied
rigor behind what you are learning from textbooks, as well as providing hints
to work out proofs that you are stumped on.

As of this writing, a sizable set theory database has been developed to
provide a foundation for many fields of mathematics, but much more work would
be required to develop useful databases for specific fields.

Metamath\index{Metamath} ``knows no math;'' it just provides a framework in
which to express mathematics.  Its language is very small.  You can define two
kinds of symbols, constants\index{constant} and variables\index{variable}.
The only thing Metamath knows how to do is to substitute strings of symbols
for the variables\index{substitution!variable}\index{variable substitution} in
an expression based on instructions you provide it in a proof, subject to
certain constraints you specify for the variables.  Even the decimal
representation of a number is merely a string of certain constants (digits)
which together, in a specific context, correspond to whatever mathematical
object you choose to define for it; unlike other computer languages, there is
no actual number stored inside the computer.  In a proof, you in effect
instruct Metamath what symbol substitutions to make in previous axioms or
theorems and join a sequence of them together to result in the desired
theorem.  This kind of symbol manipulation captures the essence of mathematics
at a preaxiomatic level.

\subsubsection{Metamath and Mathematical Literature}

In advanced mathematical literature, proofs are usually presented in the form
of short outlines that often only an expert can follow.  This is partly out of
a desire for brevity, but it would also be unwise (even if it were practical)
to present proofs in complete formal detail, since the overall picture would
be lost.\index{formal proof}

A solution I envision\label{envision} that would allow mathematics to remain
acceptable to the expert, yet increase its accessibility to non-specialists,
consists of a combination of the traditional short, informal proof in print
accompanied by a complete formal proof stored in a computer database.  In an
analogy with a computer program, the informal proof is like a set of comments
that describe the overall reasoning and content of the proof, whereas the
computer database is like the actual program and provides a means for anyone,
even a non-expert, to follow the proof in as much detail as desired, exploring
it back through layers of theorems (like subroutines that call other
subroutines) all the way back to the axioms of the theory.  In addition, the
computer database would have the advantage of providing absolute assurance
that the proof is correct, since each step can be verified automatically.

There are several other approaches besides Metamath to a project such
as this.  Section~\ref{proofverifiers} discusses some of these.

To us, a noble goal would be a database with hundreds of thousands of
theorems and their computer-verifiable proofs, encompassing a significant
fraction of known mathematics and available for instant access.
These would be fully verified by multiple independently-implemented verifiers,
to provide extremely high confidence that the proofs are completely correct.
The database would allow people to investigate whatever details they were
interested in, so that they could confirm whatever portions they wished.
Whether or not Metamath is an appropriate choice remains to be seen, but in
principle we believe it is sufficient.

\subsubsection{Formalism}

Over the past fifty years, a group of French mathematicians working
collectively under the pseudonym of Bourbaki\index{Bourbaki, Nicolas} have
co-authored a series of monographs that attempt to rigorously and
consistently formalize large bodies of mathematics from foundations.  On the
one hand, certainly such an effort has its merits; on the other hand, the
Bourbaki project has been criticized for its ``scholasticism'' and
``hyperaxiomatics'' that hide the intuitive steps that lead to the results
\cite[p.~191]{Barrow}\index{Barrow, John D.}.

Metamath unabashedly carries this philosophy to its extreme and no doubt is
subject to the same kind of criticism.  Nonetheless I think that in
conjunction with conventional approaches to mathematics Metamath can serve a
useful purpose.  The Bourbaki approach is essentially pedagogic, requiring the
reader to become intimately familiar with each detail in a very large
hierarchy before he or she can proceed to the next step.  The difference with
Metamath is that the ``reader'' (user) knows that all details are contained in
its computer database, available as needed; it does not demand that the user
know everything but conveniently makes available those portions that are of
interest.  As the body of all mathematical knowledge grows larger and larger,
no one individual can have a thorough grasp of its entirety.  Metamath
can finalize and put to rest any questions about the validity of any part of it
and can make any part of it accessible, in principle, to a non-specialist.

\subsubsection{A Personal Note}
Why did I develop Metamath\index{Metamath}?  I enjoy abstract mathematics, but
I sometimes get lost in a barrage of definitions and start to lose confidence
that my proofs are correct.  Or I reach a point where I lose sight of how
anything I'm doing relates to the axioms that a theory is based on and am
sometimes suspicious that there may be some overlooked implicit axiom
accidentally introduced along the way (as happened historically with Euclidean
geometry\index{Euclidean geometry}, whose omission of Pasch's
axiom\index{Pasch's axiom} went unnoticed for 2000 years
\cite[p.~160]{Davis}!). I'm also somewhat lazy and wish to avoid the effort
involved in re-verifying the gaps in informal proofs ``left to the reader;'' I
prefer to figure them out just once and not have to go through the same
frustration a year from now when I've forgotten what I did.  Metamath provides
better recovery of my efforts than scraps of paper that I can't
decipher anymore.  But mostly I find very appealing the idea of rigorously
archiving mathematical knowledge in a computer database, providing precision,
certainty, and elimination of human error.

\subsubsection{Note on Bibliography and Index}

The Bibliography usually includes the Library of Congress classification
for a work to make it easier for you to find it in on a university
library shelf.  The Index has author references to pages where their works
are cited, even though the authors' names may not appear on those pages.

\subsubsection{Acknowledgments}

Acknowledgments are first due to my wife, Deborah (who passed away on
September 4, 1998), for critiquing the manu\-script but most of all for
her patience and support.  I also wish to thank Joe Wright, Richard
Becker, Clarke Evans, Buddha Buck, and Jeremy Henty for helpful
comments.  Any errors, omissions, and other shortcomings are of course
my responsibility.

\subsubsection{Note Added June 22, 2005}\label{note2002}

The original, unpublished version of this book was written in 1997 and
distributed via the web.  The present edition has been updated to
reflect the current Metamath program and databases, as well as more
current {\sc url}s for Internet sites.  Thanks to Josh
Purinton\index{Purinton, Josh}, One Hand
Clapping, Mel L.\ O'Cat, and Roy F. Longton for pointing out
typographical and other errors.  I have also benefitted from numerous
discussions with Raph Levien\index{Levien, Raph}, who has extended
Metamath's philosophy of rigor to result in his {\em
Ghilbert}\index{Ghilbert} proof language (\url{http://ghilbert.org}).

Robert (Bob) Solovay\index{Solovay, Robert} communicated a new result of
A.~R.~D.~Mathias on the system of Bourbaki, and the text has been
updated accordingly (p.~\pageref{bourbaki}).

Bob also pointed out a clarification of the literature regarding
category theory and inaccessible cardinals\index{category
theory}\index{cardinal, inaccessible} (p.~\pageref{categoryth}),
and a misleading statement was removed from the text.  Specifically,
contrary to a statement in previous editions, it is possible to express
``There is a proper class of inaccessible cardinals'' in the language of
ZFC.  This can be done as follows:  ``For every set $x$ there is an
inaccessible cardinal $\kappa$ such that $\kappa$ is not in $x$.''
Bob writes:\footnote{Private communication, Nov.~30, 2002.}
\begin{quotation}
     This axiom is how Grothendieck presents category theory.  To each
inaccessible cardinal $\kappa$ one associates a Grothendieck universe
\index{Grothendieck, Alexander} $U(\kappa)$.  $U(\kappa)$ consists of
those sets which lie in a transitive set of cardinality less than
$\kappa$.  Instead of the ``category of all groups,'' one works relative
to a universe [considering the category of groups of cardinality less
than $\kappa$].  Now the category whose objects are all categories
``relative to the universe $U(\kappa)$'' will be a category not
relative to this universe but to the next universe.

     All of the things category theorists like to do can be done in this
framework.  The only controversial point is whether the Grothen\-dieck
axiom is too strong for the needs of category theorists.  Mac Lane
\index{Mac Lane, Saunders} argues that ``one universe is enough'' and
Feferman\index{Feferman, Solomon} has argued that one can get by with
ordinary ZFC.  I don't find Feferman's arguments persuasive.  Mac Lane
may be right, but when I think about category theory I do it \`{a} la
Grothendieck.

        By the way Mizar\index{Mizar} adds the axiom ``there is a proper
class of inaccessibles'' precisely so as to do category theory.
\end{quotation}

The most current information on the Metamath program and databases can
always be found at \url{http://metamath.org}.


\subsubsection{Note Added June 24, 2006}\label{note2006}

The Metamath spec was restricted slightly to make parsers easier to
write.  See the footnote on p.~\pageref{namespace}.

%\subsubsection{Note Added July 24, 2006}\label{note2006b}
\subsubsection{Note Added March 10, 2007}\label{note2006b}

I am grateful to Anthony Williams\index{Williams, Anthony} for writing
the \LaTeX\ package called {\tt realref.sty} and contributing it to the
public domain.  This package allows the internal hyperlinks in a {\sc
pdf} file to anchor to specific page numbers instead of just section
titles, making the navigation of the {\sc pdf} file for this book much
more pleasant and ``logical.''

A typographical error found by Martin Kiselkov was corrected.
A confusing remark about unification was deleted per suggestion of
Mel O'Cat.

\subsubsection{Note Added May 27, 2009}\label{note2009}

Several typos found by Kim Sparre were corrected.  A note was added that
the Poincar\'{e} conjecture has been proved (p.~\pageref{poincare}).

\subsubsection{Note Added Nov. 17, 2014}\label{note2014}

The statement of the Schr\"{o}der--Bernstein theorem was corrected in
Section~\ref{trust}.  Thanks to Bob Solovay for pointing out the error.

\subsubsection{Note Added May 25, 2016}\label{note2016}

Thanks to Jerry James for correcting 16 typos.

\subsubsection{Note Added February 25, 2019}\label{note201902}

David A. Wheeler\index{Wheeler, David A.}
made a large number of improvements and updates,
in coordination with Norman Megill.
The predicate calculus axioms were renumbered, and the text makes
it clear that they are based on Tarski's system S2;
the one slight deviation in axiom ax-6 is explained and justified.
The real and complex number axioms were modified to be consistent with
\texttt{set.mm}\index{set theory database (\texttt{set.mm})}%
\index{Metamath Proof Explorer}.
Long-awaited specification changes ``1--8'' were made,
which clarified previously ambiguous points.
Some errors in the text involving \texttt{\$f} and
\texttt{\$d} statements were corrected (the spec was correct, but
the in-book explanations unintentionally contradicted the spec).
We now have a system for automatically generating narrow PDFs,
so that those with smartphones can have easy access to the current
version of this document.
A new section on deduction was added;
it discusses the standard deduction theorem,
the weak deduction theorem,
deduction style, and natural deduction.
Many minor corrections (too numerous to list here) were also made.

\subsubsection{Note Added March 7, 2019}\label{note201903}

This added a description of the Matamath language syntax in
Extended Backus--Naur Form (EBNF)\index{Extended Backus--Naur Form}\index{EBNF}
in Appendix \ref{BNF}, added a brief explanation about typecodes,
inserted more examples in the deduction section,
and added a variety of smaller improvements.

\subsubsection{Note Added April 7, 2019}\label{note201904}

This version clarified the proper substitution notation, improved the
discussion on the weak deduction theorem and natural deduction,
documented the \texttt{undo} command, updated the information on
\texttt{write source}, changed the typecode
from \texttt{set} to \texttt{setvar} to be consistent with the current
version of \texttt{set.mm}, added more documentation about comment markup
(e.g., documented how to create headings), and clarified the
differences between various assertion forms (in particular deduction form).

\subsubsection{Note Added June 2, 2019}\label{note201906}

This version fixes a large number of small issues reported by
Beno\^{i}t Jubin\index{Jubin, Beno\^{i}t}, such as editorial issues
and the need to document \texttt{verify markup} (thank you!).
This version also includes specific examples
of forms (deduction form, inference form, and closed form).
We call this version the ``second edition'';
the previous edition formally published in 2007 had a slightly different title
(\textit{Metamath: A Computer Language for Pure Mathematics}).

\chapter{Introduction}
\pagenumbering{arabic}

\begin{quotation}
  {\em {\em I.M.:}  No, no.  There's nothing subjective about it!  Everybody
knows what a proof is.  Just read some books, take courses from a competent
mathematician, and you'll catch on.

{\em Student:}  Are you sure?

{\em I.M.:}  Well---it is possible that you won't, if you don't have any
aptitude for it.  That can happen, too.

{\em Student:}  Then {\em you} decide what a proof is, and if I don't learn
to decide in the same way, you decide I don't have any aptitude.

{\em I.M.:}  If not me, then who?}
    \flushright\sc  ``The Ideal Mathematician''
    \index{Davis, Phillip J.}
    \footnote{\cite{Davis}, p.~40.}\\
\end{quotation}

Brilliant mathematicians have discovered almost
unimaginably profound results that rank among the crowning intellectual
achievements of mankind.  However, there is a sense in which modern abstract
mathematics is behind the times, stuck in an era before computers existed.
While no one disputes the remarkable results that have been achieved,
communicating these results in a precise way to the uninitiated is virtually
impossible.  To describe these results, a terse informal language is used which
despite its elegance is very difficult to learn.  This informal language is not
imprecise, far from it, but rather it often has omitted detail
and symbols with hidden context that are
implicitly understood by an expert but few others.  Extremely complex technical
meanings are associated with innocent-sounding English words such as
``compact'' and ``measurable'' that barely hint at what is actually being
said.  Anyone who does not keep the precise technical meaning constantly in
mind is bound to fail, and acquiring the ability to do this can be achieved
only through much practice and hard work.  Only the few who complete the
painful learning experience can join the small in-group of pure
mathematicians.  The informal language effectively cuts off the true nature of
their knowledge from most everyone else.

Metamath\index{Metamath} makes abstract mathematics more concrete.  It allows
a computer to keep track of the complexity associated with each word or symbol
with absolute rigor.  You can explore this complexity at your leisure, to
whatever degree you desire.  Whether or not you believe that concepts such as
infinity actually ``exist'' outside of the mind, Metamath lets you get to the
foundation for what's really being said.

Metamath also enables completely rigorous and thorough proof verification.
Its language is simple enough so that you
don't have to rely on the authority of experts but can verify the results
yourself, step by step.  If you want to attempt to derive your own results,
Metamath will not let you make a mistake in reasoning.
Even professional mathematicians make mistakes; Metamath makes it possible
to thoroughly verify that proofs are correct.

Metamath\index{Metamath} is a computer language and an associated computer
program for archiving, verifying, and studying mathematical proofs at a very
detailed level.
The Metamath language
describes formal\index{formal system} mathematical
systems and expresses proofs of theorems in those systems.  Such a language
is called a metalanguage\index{metalanguage} by mathematicians.
The Metamath program is a computer program that verifies
proofs expressed in the Metamath language.
The Metamath program does not have the built-in
ability to make logical inferences; it just makes a series of symbol
substitutions according to instructions given to it in a proof
and verifies that the result matches the expected theorem.  It makes logical
inferences based only on rules of logic that are contained in a set of
axioms\index{axiom}, or first principles, that you provide to it as the
starting point for proofs.

The complete specification of the Metamath language is only four pages long
(Section~\ref{spec}, p.~\pageref{spec}).  Its simplicity may at first make you
wonder how it can do much of anything at all.  But in fact the kinds of
symbol manipulations it performs are the ones that are implicitly done in all
mathematical systems at the lowest level.  You can learn it relatively quickly
and have complete confidence in any mathematical proof that it verifies.  On
the other hand, it is powerful and general enough so that virtually any
mathematical theory, from the most basic to the deeply abstract, can be
described with it.

Although in principle Metamath can be used with any
kind of mathematics, it is best suited for abstract or ``pure'' mathematics
that is mostly concerned with theorems and their proofs, as opposed to the
kind of mathematics that deals with the practical manipulation of numbers.
Examples of branches of pure mathematics are logic\index{logic},\footnote{Logic
is the study of statements that are universally true regardless of the objects
being described by the statements.  An example is the statement, ``if $P$
implies $Q$, then either $P$ is false or $Q$ is true.''} set theory\index{set
theory},\footnote{Set theory is the study of general-purpose mathematical objects called
``sets,'' and from it essentially all of mathematics can be derived.  For
example, numbers can be defined as specific sets, and their properties
can be explored using the tools of set theory.} number theory\index{number
theory},\footnote{Number theory deals with the properties of positive and
negative integers (whole numbers).} group theory\index{group
theory},\footnote{Group theory studies the properties of mathematical objects
called groups that obey a simple set of axioms and have properties of symmetry
that make them useful in many other fields.} abstract algebra\index{abstract
algebra},\footnote{Abstract algebra includes group theory and also studies
groups with additional properties that qualify them as ``rings'' and
``fields.''  The set of real numbers is a familiar example of a field.},
analysis\index{analysis} \index{real and complex numbers}\footnote{Analysis is
the study of real and complex numbers.} and
topology\index{topology}.\footnote{One area studied by topology are properties
that remain unchanged when geometrical objects undergo stretching
deformations; for example a doughnut and a coffee cup each have one hole (the
cup's hole is in its handle) and are thus considered topologically
equivalent.  In general, though, topology is the study of abstract
mathematical objects that obey a certain (surprisingly simple) set of axioms.
See, for example, Munkres \cite{Munkres}\index{Munkres, James R.}.} Even in
physics, Metamath could be applied to certain branches that make use of
abstract mathematics, such as quantum logic\index{quantum logic} (used to study
aspects of quantum mechanics\index{quantum mechanics}).

On the other hand, Metamath\index{Metamath} is less suited to applications
that deal primarily with intensive numeric computations.  Metamath does not
have any built-in representation of numbers\index{Metamath!representation of
numbers}; instead, a specific string of symbols (digits) must be syntactically
constructed as part of any proof in which an ordinary number is used.  For
this reason, numbers in Metamath are best limited to specific constants that
arise during the course of a theorem or its proof.  Numbers are only a tiny
part of the world of abstract mathematics.  The exclusion of built-in numbers
was a conscious decision to help achieve Metamath's simplicity, and there are
other software tools if you have different mathematical needs.
If you wish to quickly solve algebraic problems, the computer algebra
programs\index{computer algebra system} {\sc
macsyma}\index{macsyma@{\sc macsyma}}, Mathematica\index{Mathematica}, and
Maple\index{Maple} are specifically suited to handling numbers and
algebra efficiently.
If you wish to simply calculate numeric or matrix expressions easily,
tools such as Octave\index{Octave} may be a better choice.

After learning Metamath's basic statement types, any
tech\-ni\-cal\-ly ori\-ent\-ed person, mathematician or not, can
immediately trace
any theorem proved in the language as far back as he or she wants, all the way
to the axioms on which the theorem is based.  This ability suggests a
non-traditional way of learning about pure mathematics.  Used in conjunction
with traditional methods, Metamath could make pure mathematics accessible to
people who are not sufficiently skilled to figure out the implicit detail in
ordinary textbook proofs.  Once you learn the axioms of a theory, you can have
complete confidence that everything you need to understand a proof you are
studying is all there, at your beck and call, allowing you to focus in on any
proof step you don't understand in as much depth as you need, without worrying
about getting stuck on a step you can't figure out.\footnote{On the other
hand, writing proofs in the Metamath language is challenging, requiring
a degree of rigor far in excess of that normally taught to students.  In a
classroom setting, I doubt that writing Metamath proofs would ever replace
traditional homework exercises involving informal proofs, because the time
needed to work out the details would not allow a course to
cover much material.  For students who have trouble grasping the implied rigor
in traditional material, writing a few simple proofs in the Metamath language
might help clarify fuzzy thought processes.  Although somewhat difficult at
first, it eventually becomes fun to do, like solving a puzzle, because of the
instant feedback provided by the computer.}

Metamath\index{Metamath} is probably unlike anything you have
encountered before.  In this first chapter we will look at the philosophy and
use of computers in mathematics in order to better understand the motivation
behind Metamath.  The material in this chapter is not required in order to use
Metamath.  You may skip it if you are impatient, but I hope you will find it
educational and enjoyable.  If you want to start experimenting with the
Metamath program right away, proceed directly to Chapter~\ref{using}
(p.~\pageref{using}).  To
learn the Metamath language, skim Chapter~\ref{using} then proceed to
Chapter~\ref{languagespec} (p.~\pageref{languagespec}).

\section{Mathematics as a Computer Language}

\begin{quote}
  {\em The study of mathematics is apt to commence in
dis\-ap\-point\-ment.\ldots \\
We are told that by its aid the stars are weighted
and the billions of molecules in a drop of water are counted.  Yet, like the
ghost of Hamlet's father, this great science eludes the efforts of our mental
weapons to grasp it.}
  \flushright\sc  Alfred North Whitehead\footnote{\cite{Whitehead}, ch.\ 1.}\\
\end{quote}\index{Whitehead, Alfred North}

\subsection{Is Mathematics ``User-Friendly''?}

Suppose you have no formal training in abstract mathematics.  But popular
books you've read offer tempting glimpses of this world filled with profound
ideas that have stirred the human spirit.  You are not satisfied with the
informal, watered-down descriptions you've read but feel it is important to
grasp the underlying mathematics itself to understand its true meaning. It's
not practical to go back to school to learn it, though; you don't want to
dedicate years of your life to it.  There are many important things in life,
and you have to set priorities for what's important to you.  What would happen
if you tried to pursue it on your own, in your spare time?

After all, you were able to learn a computer programming language such as
Pascal on your own without too much difficulty, even though you had no formal
training in computers.  You don't claim to be an expert in software design,
but you can write a passable program when necessary to suit your needs.  Even
more important, you know that you can look at anyone else's Pascal program, no
matter how complex, and with enough patience figure out exactly how it works,
even though you are not a specialist.  Pascal allows you do anything that a
computer can do, at least in principle.  Thus you know you have the ability,
in principle, to follow anything that a computer program can do:  you just
have to break it down into small enough pieces.

Here's an imaginary scenario of what might happen if you na\-ive\-ly a\-dopted
this same view of abstract mathematics and tried to pick it up on your own, in
a period of time comparable to, say, learning a computer programming
language.

\subsubsection{A Non-Mathematician's Quest for Truth}

\begin{quote}
  {\em \ldots my daughters have been studying (chemistry) for several
se\-mes\-ters, think they have learned differential and integral calculus in
school, and yet even today don't know why $x\cdot y=y\cdot x$ is true.}
  \flushright\sc  Edmund Landau\footnote{\cite{Landau}, p.~vi.}\\
\end{quote}\index{Landau, Edmund}

\begin{quote}
  {\em Minus times minus is plus,\\
The reason for this we need not discuss.}
  \flushright\sc W.\ H.\ Auden\footnote{As quoted in \cite{Guillen}, p.~64.}\\
\end{quote}\index{Auden, W.\ H.}\index{Guillen, Michael}

We'll suppose you are a technically oriented professional, perhaps an engineer, a
computer programmer, or a physicist, but probably not a mathematician.  You
consider yourself reasonably intelligent.  You did well in school, learning a
variety of methods and techniques in practical mathematics such as calculus and
differential equations.  But rarely did your courses get into anything
resembling modern abstract mathematics, and proofs were something that appeared
only occasionally in your textbooks, a kind of necessary evil that was
supposed to convince you of a certain key result.  Most of your
homework consisted of exercises that gave you practice in the techniques, and
you were hardly ever asked to come up with a proof of your own.

You find yourself curious about advanced, abstract mathematics.  You are
driven by an inner conviction that it is important to understand and
appreciate some of the most profound knowledge discovered by mankind.  But it
seems very hard to learn, something that only certain gifted longhairs can
access and understand.  You are frustrated that it seems forever cut off from
you.

Eventually your curiosity drives you to do something about it.
You set for yourself a goal of ``really'' understanding mathematics:  not just
how to manipulate equations in algebra or calculus according to cookbook
rules, but rather to gain a deep understanding of where those rules come from.
In fact, you're not thinking about this kind of ordinary mathematics at all,
but about a much more abstract, ethereal realm of pure mathematics, where
famous results such as G\"{o}del's incompleteness theorem\index{G\"{o}del's
incompleteness theorem} and Cantor's different kinds of infinities
reside.

You have probably read a number of popular books, with titles like {\em
Infinity and the Mind} \cite{Rucker}\index{Rucker, Rudy}, on topics such as
these.  You found them inspiring but at the same time somewhat
unsatisfactory.  They gave you a general idea of what these results are about,
but if someone asked you to prove them, you wouldn't have the faintest idea of
where to begin.   Sure, you could give the same overall outline that you
learned from the popular books; and in a general sort of way, you do have an
understanding.  But deep down inside, you know that there is a rigor that is
missing, that probably there are many subtle steps and pitfalls along the way,
and ultimately it seems you have to place your trust in the experts in the
field.  You don't like this; you want to be able to verify these results for
yourself.

So where do you go next?  As a first step, you decide to look up some of the
original papers on the theorems you are curious about, or better, obtain some
standard textbooks in the field.  You look up a theorem you want to
understand.  Sure enough, it's there, but it's expressed with strange
terms and odd symbols that mean absolutely nothing to you.  It might as well be written in
a foreign language you've never seen before, whose symbols are totally alien.
You look at the proof, and you haven't the foggiest notion what each step
means, much less how one step follows from another.  Well, obviously you have
a lot to learn if you want to understand this stuff.

You feel that you could probably understand it by
going back to college for another three to six years and getting a math
degree.  But that does not fit in with your career and the other things in
your life and would serve no practical purpose.  You decide to seek a quicker
path.  You figure you'll just trace your way back to the beginning, step by
step, as you would do with a computer program, until you understand it.  But
you quickly find that this is not possible, since you can't even understand
enough to know what you have to trace back to.

Maybe a different approach is in order---maybe you should start at the
beginning and work your way up.  First, you read the introduction to the book
to find out what the prerequisites are.  In a similar fashion, you trace your
way back through two or three more books, finally arriving at one that seems
to start at a beginning:  it lists the axioms of arithmetic.  ``Aha!'' you
naively think, ``This must be the starting point, the source of all mathematical
knowledge.'' Or at least the starting point for mathematics dealing with
numbers; you have to start somewhere and have no idea what the starting point
for other mathematics would be.  But the word ``axioms'' looks promising.  So
you eagerly read along and work through some elementary exercises at the
beginning of the book.  You feel vaguely bothered:  these
don't seem like axioms at all, at least not in the sense that you want to
think of axioms.  Axioms imply a starting point from which everything else can
be built up, according to precise rules specified in the axiom system.  Even
though you can understand the first few proofs in an informal way,
and are able to do some of the
exercises, it's hard to pin down precisely what the
rules are.   Sure, each step seems to follow logically from the others, but
exactly what does that mean?  Is the ``logic'' just a matter of common sense,
something vague that we all understand but can never quite state precisely?

You've spent a number of years, off and on, programming computers, and you
know that in the case of computer languages there is no question of what the
rules are---they are precise and crystal clear.  If you follow them, your
program will work, and if you don't, it won't.  No matter how complex a
program, it can always be broken down into simpler and simpler pieces, until
you can ultimately identify the bits that are moved around to perform a
specific function.  Some programs might require a lot of perseverance to
accomplish this, but if you focus on a specific portion of it, you don't even
necessarily have to know how the rest of it works. Shouldn't there be an
analogy in mathematics?

You decide to apply the ultimate test:  you ask yourself how a computer could
verify or ensure that the steps in these proofs follow from one another.
Certainly mathematics must be at least as precisely defined as a computer
language, if not more so; after all, computer science itself is based on it.
If you can get a computer to verify these proofs, then you should also be
able, in principle, to understand them yourself in a very crystal clear,
precise way.

You're in for a surprise:  you can conceive of no way to convert the
proofs, which are in English, to a form that the computer can understand.
The proofs are filled with phrases such as ``assume there exists a unique
$x$\ldots'' and ``given any $y$, let $z$ be the number such that\ldots''  This
isn't the kind of logic you are used to in computer programming, where
everything, even arithmetic, reduces to Boolean ones and zeroes if you care to
break it down sufficiently.  Even though you think you understand the proofs,
there seems to be some kind of higher reasoning involved rather than precise
rules that define how you manipulate the symbols in the axioms.  Whatever it
is, it just isn't obvious how you would express it to a computer, and the more
you think about it, the more puzzled and confused you get, to the point where
you even wonder whether {\em you} really understand it.  There's a lot more to
these axioms of arithmetic than meets the eye.

Nobody ever talked about this in school in your applied math and engineering
courses.  You just learned the rules they gave you, not quite understanding
how or why they worked, sometimes vaguely suspicious or uncertain of them, and
through homework problems and osmosis learned how to present solutions that
satisfied the instructor and earned you an ``A.''  Rarely did you actually
``prove'' anything in a rigorous way, and the math majors who did do stuff
like that seemed to be in a different world.

Of course, there are computer algebra programs that can do mathematics, and
rather impressively.  They can instantly solve the integrals that you
struggled with in freshman calculus, and do much, much more.  But when you
look at these programs, what you see is a big collection of algorithms and
techniques that evolved and were added to over time, along with some basic
software that manipulates symbols.  Each algorithm that is built in is the
result of someone's theorem whose proof is omitted; you just have to trust the
person who proved it and the person who programmed it in and hope there are no
bugs.\index{computer program bugs}  Somehow this doesn't seem to be the
essence of mathematics.  Although computer algebra systems can generate
theorems with amazing speed, they can't actually prove a single one of them.

After some puzzlement, you revisit some popular books on what mathematics is
all about.  Somewhere you read that all of mathematics is actually derived
from something called ``set theory.''  This is a little confusing, because
nowhere in the book that presented the axioms of arithmetic was there any
mention of set theory, or if there was, it seemed to be just a tool that helps
you describe things better---the set of even numbers, that sort of thing.  If
set theory is the basis for all mathematics, then why are additional axioms
needed for arithmetic?

Something is wrong but you're not sure what.  One of your friends is a pure
mathematician.  He knows he is unable to communicate to you what he does for a
living and seems to have little interest in trying.  You do know that for him,
proofs are what mathematics is all about. You ask him what a proof is, and he
essentially tells you that, while of course it's based on logic, really it's
something you learn by doing it over and over until you pick it up.  He refers
you to a book, {\em How to Read and Do Proofs} \cite{Solow}.\index{Solow,
Daniel}  Although this book helps you understand traditional informal proofs,
there is still something missing you can't seem to pin down yet.

You ask your friend how you would go about having a computer verify a proof.
At first he seems puzzled by the question; why would you want to do that?
Then he says it's not something that would make any sense to do, but he's
heard that you'd have to break the proof down into thousands or even millions
of individual steps to do such a thing, because the reasoning involved is at
such a high level of abstraction.  He says that maybe it's something you could
do up to a point, but the computer would be completely impractical once you
get into any meaningful mathematics.  There, the only way you can verify a
proof is by hand, and you can only acquire the ability to do this by
specializing in the field for a couple of years in grad school.  Anyway, he
thinks it all has to do with set theory, although he has never taken a formal
course in set theory but just learned what he needed as he went along.

You are intrigued and amazed.  Apparently a mathematician can grasp as a
single concept something that would take a computer a thousand or a million
steps to verify, and have complete confidence in it.  Each one of these
thousand or million steps must be absolutely correct, or else the whole proof
is meaningless.  If you added a million numbers by hand, would you trust the
result?  How do you really know that all these steps are correct, that there
isn't some subtle pitfall in one of these million steps, like a bug in a
computer program?\index{computer program bugs}  After all, you've read that
famous mathematicians have occasionally made mistakes, and you certainly know
you've made your share on your math homework problems in school.

You recall the analogy with a computer program.  Sure, you can understand what
a large computer program such as a word processor does, as a single high-level
concept or a small set of such concepts, but your ability to understand it in
no way ensures that the program is correct and doesn't have hidden bugs.  Even
if you wrote the program yourself you can't really know this; most large
programs that you've written have had bugs that crop up at some later date, no
matter how careful you tried to be while writing them.

OK, so now it seems the reason you can't figure out how to make a
computer verify proofs is because each step really corresponds to a
million small steps.  Well, you say, a computer can do a million
calculations in a second, so maybe it's still practical to do.  Now the
puzzle becomes how to figure out what the million steps are that each
English-language step corresponds to.  Your mathematician friend hasn't
a clue, but suggests that maybe you would find the answer by studying
set theory.  Actually, your friend thinks you're a little off the wall
for even wondering such a thing.  For him, this is not what mathematics
is all about.

The subject of set theory keeps popping up, so you decide it's
time to look it up.

You decide to start off on a careful footing, so you start reading a couple of
very elementary books on set theory.  A lot of it seems pretty obvious, like
intersections, subsets, and Venn diagrams.  You thumb through one of the
books; nowhere is anything about axioms mentioned. The other book relegates to
an appendix a brief discussion that mentions a set of axioms called
``Zermelo--Fraenkel set theory''\index{Zermelo--Fraenkel set theory} and states
them in English.  You look at them and have no idea what they really mean or
what you can do with them.  The comments in this appendix say that the purpose
of mentioning them is to expose you to the idea, but imply that they are not
necessary for basic understanding and that they are really the subject matter
of advanced treatments where fine points such as a certain paradox (Russell's
paradox\index{Russell's paradox}\footnote{Russell's paradox assumes that there
exists a set $S$ that is a collection of all sets that don't contain
themselves.  Now, either $S$ contains itself or it doesn't.  If it contains
itself, it contradicts its definition.  But if it doesn't contain itself, it
also contradicts its definition.  Russell's paradox is resolved in ZF set
theory by denying that such a set $S$ exists.}) are resolved.  Wait a
minute---shouldn't the axioms be a starting point, not an ending point?  If
there are paradoxes that arise without the axioms, how do you know you won't
stumble across one accidentally when using the informal approach?

And nowhere do these books describe how ``all of mathematics can be
derived from set theory'' which by now you've heard a few times.

You find a more advanced book on set theory.  This one actually lists the
axioms of ZF set theory in plain English on page one.  {\em Now} you think
your quest has ended and you've finally found the source of all mathematical
knowledge; you just have to understand what it means.  Here, in one place, is
the basis for all of mathematics!  You stare at the axioms in awe, puzzle over
them, memorize them, hoping that if you just meditate on them long enough they
will become clear.  Of course, you haven't the slightest idea how the rest of
mathematics is ``derived'' from them; in particular, if these are the axioms
of mathematics, then why do arithmetic, group theory, and so on need their own
axioms?

You start reading this advanced book carefully, pondering the meaning of every
word, because by now you're really determined to get to the bottom of this.
The first thing the book does is explain how the axioms came about, which was
to resolve Russell's paradox.\index{Russell's paradox}  In fact that seems to
be the main purpose of their existence; that they supposedly can be used to
derive all of mathematics seems irrelevant and is not even mentioned.  Well,
you go on.  You hope the book will explain to you clearly, step by step, how
to derive things from the axioms.  After all, this is the starting point of
mathematics, like a book that explains the basics of a computer programming
language.  But something is missing.  You find you can't even understand the
first proof or do the first exercise.  Symbols such as $\exists$ and $\forall$
permeate the page without any mention of where they came from or how to
manipulate them; the author assumes you are totally familiar with them and
doesn't even tell you what they mean.  By now you know that $\exists$ means
``there exists'' and $\forall$ means ``for all,'' but shouldn't the rules for
manipulating these symbols be part of the axioms?  You still have no idea
how you could even describe the axioms to a computer.

Certainly there is something much different here from the technical
literature you're used to reading.  A computer language manual almost
always explains very clearly what all the symbols mean, precisely what
they do, and the rules used for combining them, and you work your way up
from there.

After glancing at four or five other such books, you come to the realization
that there is another whole field of study that you need just to get to the
point at which you can understand the axioms of set theory.  The field is
called ``logic.''  In fact, some of the books did recommend it as a
prerequisite, but it just didn't sink in.  You assumed logic was, well, just
logic, something that a person with common sense intuitively understood.  Why
waste your time reading boring treatises on symbolic logic, the manipulation
of 1's and 0's that computers do, when you already know that?  But this is a
different kind of logic, quite alien to you.  The subject of {\sc nand} and
{\sc nor} gates is not even touched upon or in any case has to do with only a
very small part of this field.

So your quest continues.  Skimming through the first couple of introductory
books, you get a general idea of what logic is about and what quantifiers
(``for all,'' ``there exists'') mean, but you find their examples somewhat
trivial and mildly annoying (``all dogs are animals,'' ``some animals are
dogs,'' and such).  But all you want to know is what the rules are for
manipulating the symbols so you can apply them to set theory.  Some formulas
describing the relationships among quantifiers ($\exists$ and $\forall$) are
listed in tables, along with some verbal reasoning to justify them.
Presumably, if you want to find out if a formula is correct, you go through
this same kind of mental reasoning process, possibly using images of dogs and
animals. Intuitively, the formulas seem to make sense.  But when you ask
yourself, ``What are the rules I need to get a computer to figure out whether
this formula is correct?'', you still don't know.  Certainly you don't ask the
computer to imagine dogs and animals.

You look at some more advanced logic books.  Many of them have an introductory
chapter summarizing set theory, which turns out to be a prerequisite.  You
need logic to understand set theory, but it seems you also need set theory to
understand logic!  These books jump right into proving rather advanced
theorems about logic, without offering the faintest clue about where the logic
came from that allows them to prove these theorems.

Luckily, you come across an elementary book of logic that, halfway through,
after the usual truth tables and metaphors, presents in a clear, precise way
what you've been looking for all along: the axioms!  They're divided into
propositional calculus (also called sentential logic) and predicate calculus
(also called first-order logic),\index{first-order logic} with rules so simple
and crystal clear that now you can finally program a computer to understand
them.  Indeed, they're no harder than learning how to play a game of chess.
As far as what you seem to need is concerned, the whole book could have been
written in five pages!

{\em Now} you think you've found the ultimate source of mathematical
truth.  So---the axioms of mathematics consist of these axioms of logic,
together with the axioms of ZF set theory. (By now you've also been able to
figure out how to translate the ZF axioms from English into the
actual symbols of logic which you can now manipulate according to
precise, easy-to-understand rules.)

Of course, you still don't understand how ``all of mathematics can be
derived from set theory,'' but maybe this will reveal itself in due
course.

You eagerly set out to program the axioms and rules into a computer and start
to look at the theorems you will have to prove as the logic is developed.  All
sorts of important theorems start popping up:  the deduction
theorem,\index{deduction theorem} the substitution theorem,\index{substitution
theorem} the completeness theorem of propositional calculus,\index{first-order
logic!completeness} the completeness theorem of predicate calculus.  Uh-oh,
there seems to be trouble.  They all get harder and harder, and not one of
them can be derived with the axioms and rules of logic you've just been
handed.  Instead, they all require ``metalogic'' for their proofs, a kind of
mixture of logic and set theory that allows you to prove things {\em about}
the axioms and theorems of logic rather than {\em with} them.

You plow ahead anyway.  A month later, you've spent much of your
free time getting the computer to verify proofs in propositional calculus.
You've programmed in the axioms, but you've also had to program in the
deduction theorem, the substitution theorem, and the completeness theorem of
propositional calculus, which by now you've resigned yourself to treating as
rather complex additional axioms, since they can't be proved from the axioms
you were given.  You can now get the computer to verify and even generate
complete, rigorous, formal proofs\index{formal proof}.  Never mind that they
may have 100,000 steps---at least now you can have complete, absolute
confidence in them.  Unfortunately, the only theorems you have proved are
pretty trivial and you can easily verify them in a few minutes with truth
tables, if not by inspection.

It looks like your mathematician friend was right.  Getting the computer to do
serious mathematics with this kind of rigor seems almost hopeless.  Even
worse, it seems that the further along you get, the more ``axioms'' you have
to add, as each new theorem seems to involve additional ``metamathematical''
reasoning that hasn't been formalized, and none of it can be derived from the
axioms of logic.  Not only do the proofs keep growing exponentially as you get
further along, but the program to verify them keeps getting bigger and bigger
as you program in more ``metatheorems.''\index{metatheorem}\footnote{A
metatheorem is usually a statement that is too general to be directly provable
in a theory.  For example, ``if $n_1$, $n_2$, and $n_3$ are integers, then
$n_1+n_2+n_3$ is an integer'' is a theorem of number theory.  But ``for any
integer $k > 1$, if $n_1, \ldots, n_k$ are integers, then $n_1+\ldots +n_k$ is
an integer'' is a metatheorem, in other words a family of theorems, one for
every $k$.  The reason it is not a theorem is that the general sum $n_1+\ldots
+n_k$ (as a function of $k$) is not an operation that can be defined directly
in number theory.} The bugs\index{computer program bugs} that have cropped up
so far have already made you start to lose faith in the rigor you seem to have
achieved, and you know it's just going to get worse as your program gets larger.

\subsection{Mathematics and the Non-Specialist}

\begin{quote}
  {\em A real proof is not checkable by a machine, or even by any mathematician
not privy to the gestalt, the mode of thought of the particular field of
mathematics in which the proof is located.}
  \flushright\sc  Davis and Hersh\index{Davis, Phillip J.}
  \footnote{\cite{Davis}, p.~354.}\\
\end{quote}

The bulk of abstract or theoretical mathematics is ordinarily outside
the reach of anyone but a few specialists in each field who have completed
the necessary difficult internship in order to enter its coterie.  The
typical intelligent layperson has no reasonable hope of understanding much of
it, nor even the specialist mathematician of understanding other fields.  It
is like a foreign language that has no dictionary to look up the translation;
the only way you can learn it is by living in the country for a few years.  It
is argued that the effort involved in learning a specialty is a necessary
process for acquiring a deep understanding.  Of course, this is almost certainly
true if one is to make significant contributions to a field; in particular,
``doing'' proofs is probably the most important part of a mathematician's
training.  But is it also necessary to deny outsiders access to it?  Is it
necessary that abstract mathematics be so hard for a layperson to grasp?

A computer normally is of no help whatsoever.  Most published proofs are
actually just series of hints written in an informal style that requires
considerable knowledge of the field to understand.  These are the ``real
proofs'' referred to by Davis and Hersh.\index{informal proof}  There is an
implicit understanding that, in principle, such a proof could be converted to
a complete formal proof\index{formal proof}.  However, it is said that no one
would ever attempt such a conversion, even if they could, because that would
presumably require millions of steps (Section~\ref{dream}).  Unfortunately the
informal style automatically excludes the understanding of the proof
by anyone who hasn't gone through the necessary apprenticeship. The
best that the intelligent layperson can do is to read popular books about deep
and famous results; while this can be helpful, it can also be misleading, and
the lack of detail usually leaves the reader with no ability whatsoever to
explore any aspect of the field being described.

The statements of theorems often use sophisticated notation that makes them
inaccessible to the non-specialist.  For a non-specialist who wants to achieve
a deeper understanding of a proof, the process of tracing definitions and
lemmas back through their hierarchy\index{hierarchy} quickly becomes confusing
and discouraging.  Textbooks are usually written to train mathematicians or to
communicate to people who are already mathematicians, and large gaps in proofs
are often left as exercises to the reader who is left at an impasse if he or
she becomes stuck.

I believe that eventually computers will enable non-specialists and even
intelligent laypersons to follow almost any mathematical proof in any field.
Metamath is an attempt in that direction.  If all of mathematics were as
easily accessible as a computer programming language, I could envision
computer programmers and hobbyists who otherwise lack mathematical
sophistication exploring and being amazed by the world of theorems and proofs
in obscure specialties, perhaps even coming up with results of their own.  A
tremendous advantage would be that anyone could experiment with conjectures in
any field---the computer would offer instant feedback as to whether
an inference step was correct.

Mathematicians sometimes have to put up with the annoyance of
cranks\index{cranks} who lack a fundamental understanding of mathematics but
insist that their ``proofs'' of, say, Fermat's Last Theorem\index{Fermat's
Last Theorem} be taken seriously.  I think part of the problem is that these
people are misled by informal mathematical language, treating it as if they
were reading ordinary expository English and failing to appreciate the
implicit underlying rigor.  Such cranks are rare in the field of computers,
because computer languages are much more explicit, and ultimately the proof is
in whether a computer program works or not.  With easily accessible
computer-based abstract mathematics, a mathematician could say to a crank,
``don't bother me until you've demonstrated your claim on the computer!''

% 22-May-04 nm
% Attempt to move De Millo quote so it doesn't separate from attribution
% CHANGE THIS NUMBER (AND ELIMINATE IF POSSIBLE) WHEN ABOVE TEXT CHANGES
\vspace{-0.5em}

\subsection{An Impossible Dream?}\label{dream}

\begin{quote}
  {\em Even quite basic theorems would demand almost unbelievably vast
  books to display their proofs.}
    \flushright\sc  Robert E. Edwards\footnote{\cite{Edwards}, p.~68.}\\
\end{quote}\index{Edwards, Robert E.}

\begin{quote}
  {\em Oh, of course no one ever really {\em does} it.  It would take
  forever!  You just show that you could do it, that's sufficient.}
    \flushright\sc  ``The Ideal Mathematician''
    \index{Davis, Phillip J.}\footnote{\cite{Davis},
p.~40.}\\
\end{quote}

\begin{quote}
  {\em There is a theorem in the primitive notation of set theory that
  corresponds to the arithmetic theorem `$1000+2000=3000$'.  The formula
  would be forbiddingly long\ldots even if [one] knows the definitions
  and is asked to simplify the long formula according to them, chances are
  he will make errors and arrive at some incorrect result.}
    \flushright\sc  Hao Wang\footnote{\cite{Wang}, p.~140.}\\
\end{quote}\index{Wang, Hao}

% 22-May-04 nm
% Attempt to move De Millo quote so it doesn't separate from attribution
% CHANGE THIS NUMBER (AND ELIMINATE IF POSSIBLE) WHEN ABOVE TEXT CHANGES
\vspace{-0.5em}

\begin{quote}
  {\em The {\em Principia Mathematica} was the crowning achievement of the
  formalists.  It was also the deathblow of the formalist view.\ldots
  {[Rus\-sell]} failed, in three enormous volumes, to get beyond the elementary
  facts of arithmetic.  He showed what can be done in principle and what
  cannot be done in practice.  If the mathematical process were really
  one of strict, logical progression, we would still be counting our
  fingers.\ldots
  One theoretician estimates\ldots that a demonstration of one of
  Ramanujan's conjectures assuming set theory and elementary analysis would
  take about two thousand pages; the length of a deduction from first principles
  is nearly in\-con\-ceiv\-a\-ble\ldots The probabilists argue that\ldots any
  very long proof can at best be viewed as only probably correct\ldots}
  \flushright\sc Richard de Millo et. al.\footnote{\cite{deMillo}, pp.~269,
  271.}\\
\end{quote}\index{de Millo, Richard}

A number of writers have conveyed the impression that the kind of absolute
rigor provided by Metamath\index{Metamath} is an impossible dream, suggesting
that a complete, formal verification\index{formal proof} of a typical theorem
would take millions of steps in untold volumes of books.  Even if it could be
done, the thinking sometimes goes, all meaning would be lost in such a
monstrous, tedious verification.\index{informal proof}\index{proof length}

These writers assume, however, that in order to achieve the kind of complete
formal verification they desire one must break down a proof into individual
primitive steps that make direct reference to the axioms.  This is
not necessary.  There is no reason not to make use of previously proved
theorems rather than proving them over and over.

Just as important, definitions\index{definition} can be introduced along
the way, allowing very complex formulas to be represented with few
symbols.  Not doing this can lead to absurdly long formulas.  For
example, the mere statement of
G\"{o}del's incompleteness theorem\index{G\"{o}del's
incompleteness theorem}, which can be expressed with a small number of
defined symbols, would require about 20,000 primitive symbols to express
it.\index{Boolos, George S.}\footnote{George S.\ Boolos, lecture at
Massachusetts Institute of Technology, spring 1990.} An extreme example
is Bourbaki's\label{bourbaki} language for set theory, which requires
4,523,659,424,929 symbols plus 1,179,618,517,981 disambiguatory links
(lines connecting symbol pairs, usually drawn below or above the
formula) to express the number
``one'' \cite{Mathias}.\index{Mathias, Adrian R. D.}\index{Bourbaki,
Nicolas}
% http://www.dpmms.cam.ac.uk/~ardm/

A hierarchy\index{hierarchy} of theorems and definitions permits an
exponential growth in the formula sizes and primitive proof steps to be
described with only a linear growth in the number of symbols used.  Of course,
this is how ordinary informal mathematics is normally done anyway, but with
Metamath\index{Metamath} it can be done with absolute rigor and precision.

\subsection{Beauty}


\begin{quote}
  {\em No one shall be able to drive us from the paradise that Cantor has
created for us.}
   \flushright\sc  David Hilbert\footnote{As quoted in \cite{Moore}, p.~131.}\\
\end{quote}\index{Hilbert, David}

\needspace{3\baselineskip}
\begin{quote}
  {\em Mathematics possesses not only truth, but some supreme beauty ---a
  beauty cold and austere, like that of a sculpture.}
    \flushright\sc  Bertrand
    Russell\footnote{\cite{Russell}.}\\
\end{quote}\index{Russell, Bertrand}

\begin{quote}
  {\em Euclid alone has looked on Beauty bare.}
  \flushright\sc Edna Millay\footnote{As quoted in \cite{Davis}, p.~150.}\\
\end{quote}\index{Millay, Edna}

For most people, abstract mathematics is distant, strange, and
incomprehensible.  Many popular books have tried to convey some of the sense
of beauty in famous theorems.  But even an intelligent layperson is left with
only a general idea of what a theorem is about and is hardly given the tools
needed to make use of it.  Traditionally, it is only after years of arduous
study that one can grasp the concepts needed for deep understanding.
Metamath\index{Metamath} allows you to approach the proof of the theorem from
a quite different perspective, peeling apart the formulas and definitions
layer by layer until an entirely different kind of understanding is achieved.
Every step of the proof is there, pieced together with absolute precision and
instantly available for inspection through a microscope with a magnification
as powerful as you desire.

A proof in itself can be considered an object of beauty.  Constructing an
elegant proof is an art.  Once a famous theorem has been proved, often
considerable effort is made to find simpler and more easily understood
proofs.  Creating and communicating elegant proofs is a major concern of
mathematicians.  Metamath is one way of providing a common language for
archiving and preserving this information.

The length of a proof can, to a certain extent, be considered an
objective measure of its ``beauty,'' since shorter proofs are usually
considered more elegant.  In the set theory database
\texttt{set.mm}\index{set theory database (\texttt{set.mm})}%
\index{Metamath Proof Explorer}
provided with Metamath, one goal was to make all proofs as short as possible.

\needspace{4\baselineskip}
\subsection{Simplicity}

\begin{quote}
  {\em God made man simple; man's complex problems are of his own
  devising.}
    \flushright\sc Eccles. 7:29\footnote{Jerusalem Bible.}\\
\end{quote}\index{Bible}

\needspace{3\baselineskip}
\begin{quote}
  {\em God made integers, all else is the work of man.}
    \flushright\sc Leopold Kronecker\footnote{{\em Jahresbericht
	der Deutschen Mathematiker-Vereinigung }, vol. 2, p. 19.}\\
\end{quote}\index{Kronecker, Leopold}

\needspace{3\baselineskip}
\begin{quote}
  {\em For what is clear and easily comprehended attracts; the
  complicated repels.}
    \flushright\sc David Hilbert\footnote{As quoted in \cite{deMillo},
p.~273.}\\
\end{quote}\index{Hilbert, David}

The Metamath\index{Metamath} language is simple and Spartan.  Metamath treats
all mathematical expressions as simple sequences of symbols, devoid of meaning.
The higher-level or ``metamathematical'' notions underlying Metamath are about
as simple as they could possibly be.  Each individual step in a proof involves
a single basic concept, the substitution of an expression for a variable, so
that in principle almost anyone, whether mathematician or not, can
completely understand how it was arrived at.

In one of its most basic applications, Metamath\index{Metamath} can be used to
develop the foundations of mathematics\index{foundations of mathematics} from
the very beginning.  This is done in the set theory database that is provided
with the Metamath package and is the subject matter
of Chapter~\ref{fol}. Any language (a metalanguage\index{metalanguage})
used to describe mathematics (an object language\index{object language}) must
have a mathematical content of its own, but it is desirable to keep this
content down to a bare minimum, namely that needed to make use of the
inference rules specified by the axioms.  With any metalanguage there is a
``chicken and egg'' problem somewhat like circular reasoning:  you must assume
the validity of the mathematics of the metalanguage in order to prove the
validity of the mathematics of the object language.  The mathematical content
of Metamath itself is quite limited.  Like the rules of a game of chess, the
essential concepts are simple enough so that virtually anyone should be able to
understand them (although that in itself will not let you play like
a master).  The symbols that Metamath manipulates do not in themselves
have any intrinsic meaning.  Your interpretation of the axioms that you supply
to Metamath is what gives them meaning.  Metamath is an attempt to strip down
mathematical thought to its bare essence and show you exactly how the symbols
are manipulated.

Philosophers and logicians, with various motivations, have often thought it
important to study ``weak'' fragments of logic\index{weak logic}
\cite{Anderson}\index{Anderson, Alan Ross} \cite{MegillBunder}\index{Megill,
Norman}\index{Bunder, Martin}, other unconventional systems of logic (such as
``modal'' logic\index{modal logic} \cite[ch.\ 27]{Boolos}\index{Boolos, George
S.}), and quantum logic\index{quantum logic} in physics
\cite{Pavicic}\index{Pavi{\v{c}}i{\'{c}}, M.}.  Metamath\index{Metamath}
provides a framework in which such systems can be expressed, with an absolute
precision that makes all underlying metamathematical assumptions rigorous and
crystal clear.

Some schools of philosophical thought, for example
intuitionism\index{intuitionism} and constructivism\index{constructivism},
demand that the notions underlying any mathematical system be as simple and
concrete as possible.  Metamath should meet the requirements of these
philosophies.  Metamath must be taught the symbols, axioms\index{axiom}, and
rules\index{rule} for a specific theory, from the skeptical (such as
intuitionism\index{intuitionism}\footnote{Intuitionism does not accept the law
of excluded middle (``either something is true or it is not true'').  See
\cite[p.~xi]{Tymoczko}\index{Tymoczko, Thomas} for discussion and references
on this topic.  Consider the theorem, ``There exist irrational numbers $a$ and
$b$ such that $a^b$ is rational.''  An intuitionist would reject the following
proof:  If $\sqrt{2}^{\sqrt{2}}$ is rational, we are done.  Otherwise, let
$a=\sqrt{2}^{\sqrt{2}}$ and $b=\sqrt{2}$. Then $a^b=2$, which is rational.})
to the bold (such as the axiom of choice in set theory\footnote{The axiom of
choice\index{Axiom of Choice} asserts that given any collection of pairwise
disjoint nonempty sets, there exists a set that has exactly one element in
common with each set of the collection.  It is used to prove many important
theorems in standard mathematics.  Some philosophers object to it because it
asserts the existence of a set without specifying what the set contains
\cite[p.~154]{Enderton}\index{Enderton, Herbert B.}.  In one foundation for
mathematics due to Quine\index{Quine, Willard Van Orman}, that has not been
otherwise shown to be inconsistent, the axiom of choice turns out to be false
\cite[p.~23]{Curry}\index{Curry, Haskell B.}.  The \texttt{show
trace{\char`\_}back} command of the Metamath program allows you to find out
whether the axiom of choice, or any other axiom, was assumed by a
proof.}\index{\texttt{show trace{\char`\_}back} command}).

The simplicity of the Metamath language lets the algorithm (computer program)
that verifies the validity of a Metamath proof be straightforward and
robust.  You can have confidence that the theorems it verifies really can be
derived from your axioms.

\subsection{Rigor}

\begin{quote}
  {\em Rigor became a goal with the Greeks\ldots But the efforts to
  pursue rigor to the utmost have led to an impasse in which there is
  no longer any agreement on what it really means.  Mathematics remains
  alive and vital, but only on a pragmatic basis.}
    \flushright\sc  Morris Kline\footnote{\cite{Kline}, p.~1209.}\\
\end{quote}\index{Kline, Morris}

Kline refers to a much deeper kind of rigor than that which we will discuss in
this section.  G\"{o}del's incompleteness theorem\index{G\"{o}del's
incompleteness theorem} showed that it is impossible to achieve absolute rigor
in standard mathematics because we can never prove that mathematics is
consistent (free from contradictions).\index{consistent theory}  If
mathematics is consistent, we will never know it, but must rely on faith.  If
mathematics is inconsistent, the best we can hope for is that some clever
future mathematician will discover the inconsistency.  In this case, the
axioms would probably be revised slightly to eliminate the inconsistency, as
was done in the case of Russell's paradox,\index{Russell's paradox} but the
bulk of mathematics would probably not be affected by such a discovery.
Russell's paradox, for example, did not affect most of the remarkable results
achieved by 19th-century and earlier mathematicians.  It mainly invalidated
some of Gottlob Frege's\index{Frege, Gottlob} work on the foundations of
mathematics in the late 1800's; in fact Frege's work inspired Russell's
discovery.  Despite the paradox, Frege's work contains important concepts that
have significantly influenced modern logic.  Kline's {\em Mathematics, The
Loss of Certainty} \cite{Klinel}\index{Kline, Morris} has an interesting
discussion of this topic.

What {\em can} be achieved with absolute certainty\index{certainty} is the
knowledge that if we assume the axioms are consistent and true, then the
results derived from them are true.  Part of the beauty of mathematics is that
it is the one area of human endeavor where absolute certainty can be achieved
in this sense.  A mathematical truth will remain such for eternity.  However,
our actual knowledge of whether a particular statement is a mathematical truth
is only as certain as the correctness of the proof that establishes it.  If
the proof of a statement is questionable or vague, we can't have absolute
confidence in the truth that the statement claims.

Let us look at some traditional ways of expressing proofs.

Except in the field of formal logic\index{formal logic}, almost all
traditional proofs in mathematics are really not proofs at all, but rather
proof outlines or hints as to how to go about constructing the proof.  Many
gaps\index{gaps in proofs} are left for the reader to fill in. There are
several reasons for this.  First, it is usually assumed in mathematical
literature that the person reading the proof is a mathematician familiar with
the specialty being described, and that the missing steps are obvious to such
a reader or at least that the reader is capable of filling them in.  This
attitude is fine for professional mathematicians in the specialty, but
unfortunately it often has the drawback of cutting off the rest of the world,
including mathematicians in other specialties, from understanding the proof.
We discussed one possible resolution to this on p.~\pageref{envision}.
Second, it is often assumed that a complete formal proof\index{formal proof}
would require countless millions of symbols (Section~\ref{dream}). This might
be true if the proof were to be expressed directly in terms of the axioms of
logic and set theory,\index{set theory} but it is usually not true if we allow
ourselves a hierarchy\index{hierarchy} of definitions and theorems to build
upon, using a notation that allows us to introduce new symbols, definitions,
and theorems in a precisely specified way.

Even in formal logic,\index{formal logic} formal proofs\index{formal proof}
that are considered complete still contain hidden or implicit information.
For example, a ``proof'' is usually defined as a sequence of
wffs,\index{well-formed formula (wff)}\footnote{A {\em wff} or well-formed
formula is a mathematical expression (string of symbols) constructed according
to some precise rules.  A formal mathematical system\index{formal system}
contains (1) the rules for constructing syntactically correct
wffs,\index{syntax rules} (2) a list of starting wffs called
axioms,\index{axiom} and (3) one or more rules prescribing how to derive new
wffs, called theorems\index{theorem}, from the axioms or previously derived
theorems.  An example of such a system is contained in
Metamath's\index{Metamath} set theory database, which defines a formal
system\index{formal system} from which all of standard mathematics can be
derived.  Section~\ref{startf} steps you through a complete example of a formal
system, and you may want to skim it now if you are unfamiliar with the
concept.} each of which is an axiom or follows from a rule applied to previous
wffs in the sequence.  The implicit part of the proof is the algorithm by
which a sequence of symbols is verified to be a valid wff, given the
definition of a wff.  The algorithm in this case is rather simple, but for a
computer to verify the proof,\index{automated proof verification} it must have
the algorithm built into its verification program.\footnote{It is possible, of
course, to specify wff construction syntax outside of the program itself
with a suitable input language (the Metamath language being an example), but
some proof-verification or theorem-proving programs lack the ability to extend
wff syntax in such a fashion.} If one deals exclusively with axioms and
elementary wffs, it is straightforward to implement such an algorithm.  But as
more and more definitions are added to the theory in order to make the
expression of wffs more compact, the algorithm becomes more and more
complicated.  A computer program that implements the algorithm becomes larger
and harder to understand as each definition is introduced, and thus more prone
to bugs.\index{computer program bugs}  The larger the program, the
more suspicious the mathematician may be about
the validity of its algorithms.  This is especially true because
computer programs are inherently hard to follow to begin with, and few people
enjoy verifying them manually in detail.

Metamath\index{Metamath} takes a different approach.  Metamath's ``knowledge''
is limited to the ability to substitute variables for expressions, subject to
some simple constraints.  Once the basic algorithm of Metamath is assumed to
be debugged, and perhaps independently confirmed, it
can be trusted once and for all.  The information that Metamath needs to
``understand'' mathematics is contained entirely in the body of knowledge
presented to Metamath.  Any errors in reasoning can only be errors in the
axioms or definitions contained in this body of knowledge.  As a
``constructive'' language\index{constructive language} Metamath has no
conditional branches or loops like the ones that make computer programs hard
to decipher; instead, the language can only build new sequences of symbols
from earlier sequences  of symbols.

The simplicity of the rules that underlie Metamath not only makes Metamath
easy to learn but also gives Metamath a great deal of flexibility. For
example, Metamath is not limited to describing standard first-order
logic\index{first-order logic}; higher-order logics\index{higher-order logic}
and fragments of logic\index{weak logic} can be described just as easily.
Metamath gives you the freedom to define whatever wff notation you prefer; it
has no built-in conception of the syntax of a wff.\index{well-formed formula
(wff)}  With suitable axioms and definitions, Metamath can even describe and
prove things about itself.\index{Metamath!self-description}  (John
Harrison\index{Harrison, John} discusses the ``reflection''
principle\index{reflection principle} involved in self-descriptive systems in
\cite{Harrison}.)

The flexibility of Metamath requires that its proofs specify a lot of detail,
much more than in an ordinary ``formal'' proof.\index{formal proof}  For
example, in an ordinary formal proof, a single step consists of displaying the
wff that constitutes that step.  In order for a computer program to verify
that the step is acceptable, it first must verify that the symbol sequence
being displayed is an acceptable wff.\index{automated proof verification} Most
proof verifiers have at least basic wff syntax built into their programs.
Metamath has no hard-wired knowledge of what constitutes a wff built into it;
instead every wff must be explicitly constructed based on rules defining wffs
that are present in a database.  Thus a single step in an ordinary formal
proof may be correspond to many steps in a Metamath proof. Despite the larger
number of steps, though, this does not mean that a Metamath proof must be
significantly larger than an ordinary formal proof. The reason is that since
we have constructed the wff from scratch, we know what the wff is, so there is
no reason to display it.  We only need to refer to a sequence of statements
that construct it.  In a sense, the display of the wff in an ordinary formal
proof is an implicit proof of its own validity as a wff; Metamath just makes
the proof explicit. (Section~\ref{proof} describes Metamath's proof notation.)

\section{Computers and Mathematicians}

\begin{quote}
  {\em The computer is important, but not to mathematics.}
    \flushright\sc  Paul Halmos\footnote{As quoted in \cite{Albers}, p.~121.}\\
\end{quote}\index{Halmos, Paul}

Pure mathematicians have traditionally been indifferent to computers, even to
the point of disdain.\index{computers and pure mathematics}  Computer science
itself is sometimes considered to fall in the mundane realm of ``applied''
mathematics, perhaps essential for the real world but intellectually unexciting
to those who seek the deepest truths in mathematics.  Perhaps a reason for this
attitude towards computers is that there is little or no computer software that
meets their needs, and there may be a general feeling that such software could
not even exist.  On the one hand, there are the practical computer algebra
systems, which can perform amazing symbolic manipulations in algebra and
calculus,\index{computer algebra system} yet can't prove the simplest
existence theorem, if the idea of a proof is present at all.  On the other
hand, there are specialized automated theorem provers that technically speaking
may generate correct proofs.\index{automated theorem proving}  But sometimes
their specialized input notation may be cryptic and their output perceived to
be long, inelegant, incomprehensible proofs.    The output
may be viewed with suspicion, since the program that generates it tends to be
very large, and its size increases the potential for bugs\index{computer
program bugs}.  Such a proof may be considered trustworthy only if
independently verified and ``understood'' by a human, but no one wants to
waste their time on such a boring, unrewarding chore.



\needspace{4\baselineskip}
\subsection{Trusting the Computer}

\begin{quote}
  {\em \ldots I continue to find the quasi-empirical interpretation of
  computer proofs to be the more plausible.\ldots Since not
  everything that claims to be a computer proof can be
  accepted as valid, what are the mathematical criteria for acceptable
  computer proofs?}
    \flushright\sc  Thomas Tymoczko\footnote{\cite{Tymoczko}, p.~245.}\\
\end{quote}\index{Tymoczko, Thomas}

In some cases, computers have been essential tools for proving famous
theorems.  But if a proof is so long and obscure that it can be verified in a
practical way only with a computer, it is vaguely felt to be suspicious.  For
example, proving the famous four-color theorem\index{four-color
theorem}\index{proof length} (``a map needs no more than four colors to
prevent any two adjacent countries from having the same color'') can presently
only be done with the aid of a very complex computer program which originally
required 1200 hours of computer time. There has been considerable debate about
whether such a proof can be trusted and whether such a proof is ``real''
mathematics \cite{Swart}\index{Swart, E. R.}.\index{trusting computers}

However, under normal circumstances even a skeptical mathematician would have a
great deal of confidence in the result of multiplying two numbers on a pocket
calculator, even though the precise details of what goes on are hidden from its
user.  Even the verification on a supercomputer that a huge number is prime is
trusted, especially if there is independent verification; no one bothers to
debate the philosophical significance of its ``proof,'' even though the actual
proof would be so large that it would be completely impractical to ever write
it down on paper.  It seems that if the algorithm used by the computer is
simple enough to be readily understood, then the computer can be trusted.

Metamath\index{Metamath} adopts this philosophy.  The simplicity of its
language makes it easy to learn, and because of its simplicity one can have
essentially absolute confidence that a proof is correct. All axioms, rules, and
definitions are available for inspection at any time because they are defined
by the user; there are no hidden or built-in rules that may be prone to subtle
bugs\index{computer program bugs}.  The basic algorithm at the heart of
Metamath is simple and fixed, and it can be assumed to be bug-free and robust
with a degree of confidence approaching certainty.
Independently written implementations of the Metamath verifier
can reduce any residual doubt on the part of a skeptic even further;
there are now over a dozen such implementations, written by many people.

\subsection{Trusting the Mathematician}\label{trust}

\begin{quote}
  {\em There is no Algebraist nor Mathematician so expert in his science, as
  to place entire confidence in any truth immediately upon his discovery of it,
  or regard it as any thing, but a mere probability.  Every time he runs over
  his proofs, his confidence encreases; but still more by the approbation of
  his friends; and is rais'd to its utmost perfection by the universal assent
  and applauses of the learned world.}
  \flushright\sc David Hume\footnote{{\em A Treatise of Human Nature}, as
  quoted in \cite{deMillo}, p.~267.}\\
\end{quote}\index{Hume, David}

\begin{quote}
  {\em Stanislaw Ulam estimates that mathematicians publish 200,000 theorems
  every year.  A number of these are subsequently contradicted or otherwise
  disallowed, others are thrown into doubt, and most are ignored.}
  \flushright\sc Richard de Millo et. al.\footnote{\cite{deMillo}, p.~269.}\\
\end{quote}\index{Ulam, Stanislaw}

Whether or not the computer can be trusted, humans  of course will occasionally
err. Only the most memorable proofs get independently verified, and of these
only a handful of truly great ones achieve the status of being ``known''
mathematical truths that are used without giving a second thought to their
correctness.

There are many famous examples of incorrect theorems and proofs in
mathematical literature.\index{errors in proofs}

\begin{itemize}
\item There have been thousands of purported proofs of Fermat's Last
Theorem\index{Fermat's Last Theorem} (``no integer solutions exist to $x^n +
y^n = z^n$ for $n > 2$''), by amateurs, cranks, and well-regarded
mathematicians \cite[p.~5]{Stark}\index{Stark, Harold M}.  Fermat wrote a note
in his copy of Bachet's {\em Diophantus} that he found ``a truly marvelous
proof of this theorem but this margin is too narrow to contain it''
\cite[p.~507]{Kramer}.  A recent, much publicized proof by Yoichi
Miyaoka\index{Miyaoka, Yoichi} was shown to be incorrect ({\em Science News},
April 9, 1988, p.~230).  The theorem was finally proved by Andrew
Wiles\index{Wiles, Andrew} ({\em Science News}, July 3, 1993, p.~5), but it
initially had some gaps and took over a year after its announcement to be
checked thoroughly by experts.  On Oct. 25, 1994, Wiles announced that the last
gap found in his proof had been filled in.
  \item In 1882, M. Pasch discovered that an axiom was omitted from Euclid's
formulation of geometry\index{Euclidean geometry}; without it, the proofs of
important theorems of Euclid are not valid.  Pasch's axiom\index{Pasch's
axiom} states that a line that intersects one side of a triangle must also
intersect another side, provided that it does not touch any of the triangle's
vertices.  The omission of Pasch's axiom went unnoticed for 2000
years \cite[p.~160]{Davis}, in spite of (one presumes) the thousands of
students, instructors, and mathematicians who studied Euclid.
  \item The first published proof of the famous Schr\"{o}der--Bernstein
theorem\index{Schr\"{o}der--Bernstein theorem} in set theory was incorrect
\cite[p.~148]{Enderton}\index{Enderton, Herbert B.}.  This theorem states
that if there exists a 1-to-1 function\footnote{A {\em set}\index{set} is any
collection of objects. A {\em function}\index{function} or {\em
mapping}\index{mapping} is a rule that assigns to each element of one set
(called the function's {\em domain}\index{domain}) an element from another
set.} from set $A$ into set $B$ and vice-versa, then sets $A$ and $B$ have
a 1-to-1 correspondence.  Although it sounds simple and obvious,
the standard proof is quite long and complex.
  \item In the early 1900's, Hilbert\index{Hilbert, David} published a
purported proof of the continuum hypothesis\index{continuum hypothesis}, which
was eventually established as unprovable by Cohen\index{Cohen, Paul} in 1963
\cite[p.~166]{Enderton}.  The continuum hypothesis states that no
infinity\index{infinity} (``transfinite cardinal number'')\index{cardinal,
transfinite} exists whose size (or ``cardinality''\index{cardinality}) is
between the size of the set of integers and the size of the set of real
numbers.  This hypothesis originated with German mathematician Georg
Cantor\index{Cantor, Georg} in the late 1800's, and his inability to prove it
is said to have contributed to mental illness that afflicted him in his later
years.
  \item An incorrect proof of the four-color theorem\index{four-color theorem}
was published by Kempe\index{Kempe, A. B.} in 1879
\cite[p.~582]{Courant}\index{Courant, Richard}; it stood for 11 years before
its flaw was discovered.  This theorem states that any map can be colored
using only four colors, so that no two adjacent countries have the same
color.  In 1976 the theorem was finally proved by the famous computer-assisted
proof of Haken, Appel, and Koch \cite{Swart}\index{Appel, K.}\index{Haken,
W.}\index{Koch, K.}.  Or at least it seems that way.  Mathematician
H.~S.~M.~Coxeter\index{Coxeter, H. S. M.} has doubts \cite[p.~58]{Davis}:  ``I
have a feeling that that is an untidy kind of use of the computers, and the more
you correspond with Haken and Appel, the more shaky you seem to be.''
  \item Many false ``proofs'' of the Poincar\'{e}
conjecture\index{Poincar\'{e} conjecture} have been proposed over the years.
This conjecture states that any object that mathematically behaves like a
three-dimensional sphere is a three-dimensional sphere topologically,
regardless of how it is distorted.  In March 1986, mathematicians Colin
Rourke\index{Rourke, Colin} and Eduardo R\^{e}go\index{R\^{e}go, Eduardo}
caused  a stir in the mathematical community by announcing that they had found
a proof; in November of that year the proof was found to be false \cite[p.
218]{PetersonI}.  It was finally proved in 2003 by Grigory Perelman
\label{poincare}\index{Szpiro, George}\index{Perelman, Grigory}\cite{Szpiro}.
 \end{itemize}

Many counterexamples to ``theorems'' in recent mathematical
literature related to Clifford algebras\index{Clifford algebras}
 have been found by Pertti
Lounesto (who passed away in 2002).\index{Lounesto, Pertti}
See the web page \url{http://mathforum.org/library/view/4933.html}.
% http://users.tkk.fi/~ppuska/mirror/Lounesto/counterexamples.htm

One of the purposes of Metamath\index{Metamath} is to allow proofs to be
expressed with absolute precision.  Developing a proof in the Metamath
language can be challenging, because Metamath will not permit even the
tiniest mistake.\index{errors in proofs}  But once the proof is created, its
correctness can be trusted immediately, without having to depend on the
process of peer review for confirmation.

\section{The Use of Computers in Mathematics}

\subsection{Computer Algebra Systems}

For the most part, you will find that Metamath\index{Metamath} is not a
practical tool for manipulating numbers.  (Even proving that $2 + 2 = 4$, if
you start with set theory, can be quite complex!)  Several commercial
mathematics packages are quite good at arithmetic, algebra, and calculus, and
as practical tools they are invaluable.\index{computer algebra system} But
they have no notion of proof, and cannot understand statements starting with
``there exists such and such...''.

Software packages such as Mathematica \cite{Wolfram}\index{Mathematica} do not
concern themselves with proofs but instead work directly with known results.
These packages primarily emphasize heuristic rules such as the substitution of
equals for equals to achieve simpler expressions or expressions in a different
form.  Starting with a rich collection of built-in rules and algorithms, users
can add to the collection by means of a powerful programming language.
However, results such as, say, the existence of a certain abstract object
without displaying the actual object cannot be expressed (directly) in their
languages.  The idea of a proof from a small set of axioms is absent.  Instead
this software simply assumes that each fact or rule you add to the built-in
collection of algorithms is valid.  One way to view the software is as a large
collection of axioms from which the software, with certain goals, attempts to
derive new theorems, for example equating a complex expression with a simpler
equivalent. But the terms ``theorem''\index{theorem} and
``proof,''\index{proof} for example, are not even mentioned in the index of
the user's manual for Mathematica.\index{Mathematica and proofs}  What is also
unsatisfactory from a philosophical point of view is that there is no way to
ensure the validity of the results other than by trusting the writer of each
application module or tediously checking each module by hand, similar to
checking a computer program for bugs.\index{computer program
bugs}\footnote{Two examples illustrate why the knowledge database of computer
algebra systems should sometimes be regarded with a certain caution.  If you
ask Mathematica (version 3.0) to \texttt{Solve[x\^{ }n + y\^{ }n == z\^{ }n , n]}
it will respond with \texttt{\{\{n-\char`\>-2\}, \{n-\char`\>-1\},
\{n-\char`\>1\}, \{n-\char`\>2\}\}}. In other words, Mathematica seems to
``know'' that Fermat's Last Theorem\index{Fermat's Last Theorem} is true!  (At
the time this version of Mathematica was released this fact was unknown.)  If
you ask Maple\index{Maple} to \texttt{solve(x\^{ }2 = 2\^{ }x)} then
\texttt{simplify(\{"\})}, it returns the solution set \texttt{\{2, 4\}}, apparently
unaware that $-0.7666647$\ldots is also a solution.} While of course extremely
valuable in applied mathematics, computer algebra systems tend to be of little
interest to the theoretical mathematician except as aids for exploring certain
specific problems.

Because of possible bugs, trusting the output of a computer algebra system for
use as theorems in a proof-verifier would defeat the latter's goal of rigor.
On the other hand, a fact such that a certain relatively large number is
prime, while easy for a computer algebra system to derive, might have a long,
tedious proof that could overwhelm a proof-verifier. One approach for linking
computer algebra systems to a proof-verifier while retaining the advantages of
both is to add a hypothesis to each such theorem indicating its source.  For
example, a constant {\sc maple} could indicate the theorem came from the Maple
package, and would mean ``assuming Maple is consistent, then\ldots''  This and
many other topics concerning the formalization of mathematics are discussed in
John Harrison's\index{Harrison, John} very interesting
PhD thesis~\cite{Harrison-thesis}.

\subsection{Automated Theorem Provers}\label{theoremprovers}

A mathematical theory is ``decidable''\index{decidable theory} if a mechanical
method or algorithm exists that is guaranteed to determine whether or not a
particular formula is a theorem.  Among the few theories that are decidable is
elementary geometry,\index{Euclidean geometry} as was shown by a classic
result of logician Alfred Tarski\index{Tarski, Alfred} in 1948
\cite{Tarski}.\footnote{Tarski's result actually applies to a subset of the
geometry discussed in elementary textbooks.  This subset includes most of what
would be considered elementary geometry but it is not powerful enough to
express, among other things, the notions of the circumference and area of a
circle.  Extending the theory in a way that includes notions such as these
makes the theory undecidable, as was also shown by Tarski.  Tarski's algorithm
is far too inefficient to implement practically on a computer.  A practical
algorithm for proving a smaller subset of geometry theorems (those not
involving concepts of ``order'' or ``continuity'') was discovered by Chinese
mathematician Wu Wen-ts\"{u}n in 1977 \cite{Chou}\index{Chou,
Shang-Ching}.}\index{Wen-ts{\"{u}}n, Wu}  But most theories, including
elementary arithmetic, are undecidable.  This fact contributes to keeping
mathematics alive and well, since many mathematicians believe
that they will never be
replaced by computers (if they believe Roger Penrose's argument that a
computer can never replace the brain \cite{Penrose}\index{Penrose, Roger}).
In fact,  elementary geometry is often considered a ``dead'' field
for the simple reason that it is decidable.

On the other hand, the undecidability of a theory does not mean that one cannot
use a computer to search for proofs, providing one is willing to give up if a
proof is not found after a reasonable amount of time.  The field of automated
theorem proving\index{automated theorem proving} specializes in pursuing such
computer searches.  Among the more successful results to date are those based
on an algorithm known as Robinson's resolution principle
\cite{Robinson}\index{Robinson's resolution principle}.

Automated theorem provers can be excellent tools for those willing to learn
how to use them.  But they are not widely used in mainstream pure
mathematics, even though they could probably be useful in many areas.  There
are several reasons for this.  Probably most important, the main goal in pure
mathematics is to arrive at results that are considered to be deep or
important; proving them is essential but secondary.  Usually, an automated
theorem prover cannot assist in this main goal, and by the time the main goal
is achieved, the mathematician may have already figured out the proof as a
by-product.  There is also a notational problem.  Mathematicians are used to
using very compact syntax where one or two symbols (heavily dependent on
context) can represent very complex concepts; this is part of the
hierarchy\index{hierarchy} they have built up to tackle difficult problems.  A
theorem prover on the other hand might require that a theorem be expressed in
``first-order logic,''\index{first-order logic} which is the logic on which
most of mathematics is ultimately based but which is not ordinarily used
directly because expressions can become very long.  Some automated theorem
provers are experimental programs, limited in their use to very specialized
areas, and the goal of many is simply research into the nature of automated
theorem proving itself.  Finally, much research remains to be done to enable
them to prove very deep theorems.  One significant result was a
computer proof by Larry Wos\index{Wos, Larry} and colleagues that every Robbins
algebra\index{Robbins algebra} is a Boolean  algebra\index{Boolean algebra}
({\em New York Times}, Dec. 10, 1996).\footnote{In 1933, E.~V.\
Huntington\index{Huntington, E. V.}
presented the following axiom system for
Boolean algebra with a unary operation $n$ and a binary operation $+$:
\begin{center}
    $x + y = y + x$ \\
    $(x + y) + z = x + (y + z)$ \\
    $n(n(x) + y) + n(n(x) + n(y)) = x$
\end{center}
Herbert Robbins\index{Robbins, Herbert}, a student of Huntington, conjectured
that the last equation can be replaced with a simpler one:
\begin{center}
    $n(n(x + y) + n(x + n(y))) = x$
\end{center}
Robbins and Huntington could not find a proof.  The problem was
later studied unsuccessfully by Tarski\index{Tarski, Alfred} and his
students, and it remained an unsolved problem until a
computer found the proof in 1996.  For more information on
the Robbins algebra problem see \cite{Wos}.}

How does Metamath\index{Metamath} relate to automated theorem provers?  A
theorem prover is primarily concerned with one theorem at a time (perhaps
tapping into a small database of known theorems) whereas Metamath is more like
a theorem archiving system, storing both the theorem and its proof in a
database for access and verification.  Metamath is one answer to ``what do you
do with the output of a theorem prover?''  and could be viewed as the
next step in the process.  Automated theorem provers could be useful tools for
helping develop its database.
Note that very long, automatically
generated proofs can make your database fat and ugly and cause Metamath's proof
verification to take a long time to run.  Unless you have a particularly good
program that generates very concise proofs, it might be best to consider the
use of automatically generated proofs as a quick-and-dirty approach, to be
manually rewritten at some later date.

The program {\sc otter}\index{otter@{\sc otter}}\footnote{\url{http://www.cs.unm.edu/\~mccune/otter/}.}, later succeeded by
prover9\index{prover9}\footnote{\url{https://www.cs.unm.edu/~mccune/mace4/}.},
have been historically influential.
The E prover\index{E prover}\footnote{\url{https://github.com/eprover/eprover}.}
is a maintained automated theorem prover
for full first-order logic with equality.
There are many other automated theorem provers as well.

If you want to combine automated theorem provers with Metamath
consider investigating
the book {\em Automated Reasoning:  Introduction and Applications}
\cite{Wos}\index{Wos, Larry}.  This book discusses
how to use {\sc otter} in a way that can
not only able to generate
relatively efficient proofs, it can even be instructed to search for
shorter proofs.  The effective use of {\sc otter} (and similar tools)
does require a certain
amount of experience, skill, and patience.  The axiom system used in the
\texttt{set.mm}\index{set theory database (\texttt{set.mm})} set theory
database can be expressed to {\sc otter} using a method described in
\cite{Megill}.\index{Megill, Norman}\footnote{To use those axioms with
{\sc otter}, they must be restated in a way that eliminates the need for
``dummy variables.''\index{dummy variable!eliminating} See the Comment
on p.~\pageref{nodd}.} When successful, this method tends to generate
short and clever proofs, but my experiments with it indicate that the
method will find proofs within a reasonable time only for relatively
easy theorems.  It is still fun to experiment with.

Reference \cite{Bledsoe}\index{Bledsoe, W. W.} surveys a number of approaches
people have explored in the field of automated theorem proving\index{automated
theorem proving}.

\subsection{Interactive Theorem Provers}\label{interactivetheoremprovers}

Finding proofs completely automatically is difficult, so there
are some interactive theorem provers that allow a human to guide the
computer to find a proof.
Examples include
HOL Light\index{HOL light}%
\footnote{\url{https://www.cl.cam.ac.uk/~jrh13/hol-light/}.},
Isabelle\index{Isabelle}%
\footnote{\url{http://www.cl.cam.ac.uk/Research/HVG/Isabelle}.},
{\sc hol}\index{hol@{\sc hol}}%
\footnote{\url{https://hol-theorem-prover.org/}.},
and
Coq\index{Coq}\footnote{\url{https://coq.inria.fr/}.}.

A major difference between most of these tools and Metamath is that the
``proofs'' are actually programs that guide the program to find a proof,
and not the proof itself.
For example, an Isabelle/HOL proof might apply a step
\texttt{apply (blast dest: rearrange reduction)}. The \texttt{blast}
instruction applies
an automatic tableux prover and returns if it found a sequence of proof
steps that work... but the sequence is not considered part of the proof.

A good overview of
higher-level proof verification languages (such as {\sc lcf}\index{lcf@{\sc
lcf}} and {\sc hol}\index{hol@{\sc hol}})
is given in \cite{Harrison}.  All of these languages are fundamentally
different from Metamath in that much of the mathematical foundational
knowledge is embedded in the underlying proof-verification program, rather
than placed directly in the database that is being verified.
These can have a steep learning curve for those without a mathematical
background.  For example, one usually must have a fair understanding of
mathematical logic in order to follow their proofs.

\subsection{Proof Verifiers}\label{proofverifiers}

A proof verifier is a program that doesn't generate proofs but instead
verifies proofs that you give it.  Many proof verifiers have limited built-in
automated proof capabilities, such as figuring out simple logical inferences
(while still being guided by a person who provides the overall proof).  Metamath
has no built-in automated proof capability other than the limited
capability of its Proof Assistant.

Proof-verification languages are not used as frequently as they might be.
Pure mathematicians are more concerned with producing new results, and such
detail and rigor would interfere with that goal.  The use of computers in pure
mathematics is primarily focused on automated theorem provers (not verifiers),
again with the ultimate goal of aiding the creation of new mathematics.
Automated theorem provers are usually concerned with attacking one theorem at
time rather than making a large, organized database easily available to the
user.  Metamath is one way to help close this gap.

By itself Metamath is a mostly a proof verifier.
This does not mean that other approaches can't be used; the difference
is that in Metamath, the results of various provers must be recorded
step-by-step so that they can be verified.

Another proof-verification language is Mizar,\index{Mizar} which can display
its proofs in the informal language that mathematicians are accustomed to.
Information on the Mizar language is available at \url{http://mizar.org}.

For the working mathematician, Mizar is an excellent tool for rigorously
documenting proofs. Mizar typesets its proofs in the informal English used by
mathematicians (and, while fine for them, are just as inscrutable by
laypersons!). A price paid for Mizar is a relatively steep learning curve of a
couple of weeks.  Several mathematicians are actively formalizing different
areas of mathematics using Mizar and publishing the proofs in a dedicated
journal. Unfortunately the task of formalizing mathematics is still looked
down upon to a certain extent since it doesn't involve the creation of ``new''
mathematics.

The closest system to Metamath is
the {\em Ghilbert}\index{Ghilbert} proof language (\url{http://ghilbert.org})
system developed by
Raph Levien\index{Levien, Raph}.
Ghilbert is a formal proof checker heavily inspired by Metamath.
Ghilbert statements are s-expressions (a la Lisp), which is easy
for computers to parse but many people find them hard to read.
There are a number of differences in their specific constructs, but
there is at least one tool to translate some Metamath materials into Ghilbert.
As of 2019 the Ghilbert community is smaller and less active than the
Metamath community.
That said, the Metamath and Ghilbert communities overlap, and fruitful
conversations between them have occurred many times over the years.

\subsection{Creating a Database of Formalized Mathematics}\label{mathdatabase}

Besides Metamath, there are several other ongoing projects with the goal of
formalizing mathematics into computer-verifiable databases.
Understanding some history will help.

The {\sc qed}\index{qed project@{\sc qed} project}%
\footnote{\url{http://www-unix.mcs.anl.gov/qed}.}
project arose in 1993 and its goals were outlined in the
{\sc qed} manifesto.
The {\sc qed} manifesto was
a proposal for a computer-based database of all mathematical knowledge,
strictly formalized and with all proofs having been checked automatically.
The project had a conference in 1994 and another in 1995;
there was also a ``twenty years of the {\sc qed} manifesto'' workshop
in 2014.
Its ideals are regularly reraised.

In a 2007 paper, Freek Wiedijk identified two reasons
for the failure of the {\sc qed} project as originally envisioned:%
\cite{Wiedijk-revisited}\index{Wiedijk, Freek}

\begin{itemize}
\item Very few people are working on formalization of mathematics. There is no compelling application for fully mechanized mathematics.
\item Formalized mathematics does not yet resemble traditional mathematics. This is partly due to the complexity of mathematical notation, and partly to the limitations of existing theorem provers and proof assistants.
\end{itemize}

But this did not end the dream of
formalizing mathematics into computer-verifiable databases.
The problems that led to the {\sc qed} manifesto are still with us,
even though the challenges were harder than originally considered.
What has happened instead is that various independent projects have
worked towards formalizing mathematics into computer-verifiable databases,
each simultaneously competing and cooperating with each other.

A concrete way to see this is
Freek Wiedijk's ``Formalizing 100 Theorems'' list%
\footnote{\url{http://www.cs.ru.nl/\%7Efreek/100/}.}
which shows the progress different systems have made on a challenge list
of 100 mathematical theorems.%
\footnote{ This is not the only list of ``interesting'' theorems.
Another interesting list was posted by Oliver Knill's list
\cite{Knill}\index{Knill, Oliver}.}
The top systems as of February 2019
(in order of the number of challenges completed) are
HOL Light, Isabelle, Metamath, Coq, and Mizar.

The Metamath 100%
\footnote{\url{http://us.metamath.org/mm\_100.html}}
page (maintained by David A. Wheeler\index{Wheeler, David A.})
shows the progress of Metamath (specifically its \texttt{set.mm} database)
against this challenge list maintained by Freek Wiedijk.
The Metamath \texttt{set.mm} database
has made a lot of progress over the years,
in part because working to prove those challenge theorems required
defining various terms and proving their properties as a prerequisite.
Here are just a few of the many statements that have been
formally proven with Metamath:

% The entries of this cause the narrow display to break poorly,
% since the short amount of text means LaTeX doesn't get a lot to work with
% and the itemize format gives it even *less* margin than usual.
% No one will mind if we make just this list flushleft, since this list
% will be internally consistent.
\begin{flushleft}
\begin{itemize}
\item 1. The Irrationality of the Square Root of 2
  (\texttt{sqr2irr}, by Norman Megill, 2001-08-20)
\item 2. The Fundamental Theorem of Algebra
  (\texttt{fta}, by Mario Carneiro, 2014-09-15)
\item 22. The Non-Denumerability of the Continuum
  (\texttt{ruc}, by Norman Megill, 2004-08-13)
\item 54. The Konigsberg Bridge Problem
  (\texttt{konigsberg}, by Mario Carneiro, 2015-04-16)
\item 83. The Friendship Theorem
  (\texttt{friendship}, by Alexander W. van der Vekens, 2018-10-09)
\end{itemize}
\end{flushleft}

We thank all of those who have developed at least one of the Metamath 100
proofs, and we particularly thank
Mario Carneiro\index{Carneiro, Mario}
who has contributed the most Metamath 100 proofs as of 2019.
The Metamath 100 page shows the list of all people who have contributed a
proof, and links to graphs and charts showing progress over time.
We encourage others to work on proving theorems not yet proven in Metamath,
since doing so improves the work as a whole.

Each of the math formalization systems (including Metamath)
has different strengths and weaknesses, depending on what you value.
Key aspects that differentiate Metamath from the other top systems are:

\begin{itemize}
\item Metamath is not tied to any particular set of axioms.
\item Metamath can show every step of every proof, no exceptions.
  Most other provers only assert that a proof can be found, and do not
  show every step. This also makes verification fast, because
  the system does not need to rediscover proof details.
\item The Metamath verifier has been re-implemented in many different
  programming languages, so verification can be done by multiple
  implementations.  In particular, the
  \texttt{set.mm}\index{set theory database (\texttt{set.mm})}%
  \index{Metamath Proof Explorer} database is verified by
  four different verifiers
  written in four different languages by four different authors.
  This greatly reduces the risk of accepting an invalid
  proof due to an error in the verifier.
\item Proofs stay proven.  In some systems, changes to the system's
  syntax or how a tactic works causes proofs to fail in later versions,
  causing older work to become essentially lost.
  Metamath's language is
  extremely small and fixed, so once a proof is added to a database,
  the database can be rechecked with later versions of the Metamath program
  and with other verifiers of Metamath databases.
  If an axiom or key definition needs to be changed, it is easy to
  manipulate the database as a whole to handle the change
  without touching the underlying verifier.
  Since re-verification of an entire database takes seconds, there
  is never a reason to delay complete verification.
  This aspect is especially compelling if your
  goal is to have a long-term database of proofs.
\item Licensing is generous.  The main Metamath databases are released to
  the public domain, and the main Metamath program is open source software
  under a standard, widely-used license.
\item Substitutions are easy to understand, even by those who are not
  professional mathematicians.
\end{itemize}

Of course, other systems may have advantages over Metamath
that are more compelling, depending on what you value.
In any case, we hope this helps you understand Metamath
within a wider context.

\subsection{In Summary}\label{computers-summary}

To summarize our discussions of computers and mathematics, computer algebra
systems can be viewed as theorem generators focusing on a narrow realm of
mathematics (numbers and their properties), automated theorem provers as proof
generators for specific theorems in a much broader realm covered by a built-in
formal system such as first-order logic, interactive theorem
provers require human guidance, proof verifiers verify proofs but
historically they have been
restricted to first-order logic.
Metamath, in contrast,
is a proof verifier and documenter whose realm is essentially unlimited.

\section{Mathematics and Metamath}

\subsection{Standard Mathematics}

There are a number of ways that Metamath\index{Metamath} can be used with
standard mathematics.  The most satisfying way philosophically is to start at
the very beginning, and develop the desired mathematics from the axioms of
logic and set theory.\index{set theory}  This is the approach taken in the
\texttt{set.mm}\index{set theory database (\texttt{set.mm})}%
\index{Metamath Proof Explorer}
database (also known as the Metamath Proof Explorer).
Among other things, this database builds up to the
axioms of real and complex numbers\index{analysis}\index{real and complex
numbers} (see Section~\ref{real}), and a standard development of analysis, for
example, could start at that point, using it as a basis.   Besides this
philosophical advantage, there are practical advantages to having all of the
tools of set theory available in the supporting infrastructure.

On the other hand, you may wish to start with the standard axioms of a
mathematical theory without going through the set theoretical proofs of those
axioms.  You will need mathematical logic to make inferences, but if you wish
you can simply introduce theorems\index{theorem} of logic as
``axioms''\index{axiom} wherever you need them, with the implicit assumption
that in principle they can be proved, if they are obvious to you.  If you
choose this approach, you will probably want to review the notation used in
\texttt{set.mm}\index{set theory database (\texttt{set.mm})} so that your own
notation will be consistent with it.

\subsection{Other Formal Systems}
\index{formal system}

Unlike some programs, Metamath\index{Metamath} is not limited to any specific
area of mathematics, nor committed to any particular mathematical philosophy
such as classical logic versus intuitionism, nor limited, say, to expressions
in first-order logic.  Although the database \texttt{set.mm}
describes standard logic and set theory, Meta\-math
is actually a general-purpose language for describing a wide variety of formal
systems.\index{formal system}  Non-standard systems such as modal
logic,\index{modal logic} intuitionist logic\index{intuitionism}, higher-order
logic\index{higher-order logic}, quantum logic\index{quantum logic}, and
category theory\index{category theory} can all be described with the Metamath
language.  You define the symbols you prefer and tell Metamath the axioms and
rules you want to start from, and Metamath will verify any inferences you make
from those axioms and rules.  A simple example of a non-standard formal system
is Hofstadter's\index{Hofstadter, Douglas R.} MIU system,\index{MIU-system}
whose Metamath description is presented in Appendix~\ref{MIU}.

This is not hypothetical.
The largest Metamath database is
\texttt{set.mm}\index{set theory database (\texttt{set.mm}}%
\index{Metamath Proof Explorer}), aka the Metamath Proof Explorer,
which uses the most common axioms for mathematical foundations
(specifically classical logic combined with Zermelo--Fraenkel
set theory\index{Zermelo--Fraenkel set theory} with the Axiom of Choice).
But other Metamath databases are available:

\begin{itemize}
\item The database
  \texttt{iset.mm}\index{intuitionistic logic database (\texttt{iset.mm})},
  aka the
  Intuitionistic Logic Explorer\index{Intuitionistic Logic Explorer},
  uses intuitionistic logic (a constructivist point of view)
  instead of classical logic.
\item The database
  \texttt{nf.mm}\index{New Foundations database (\texttt{nf.mm})},
  aka the
  New Foundations Explorer\index{New Foundations Explorer},
  constructs mathematics from scratch,
  starting from Quine's New Foundations (NF) set theory axioms.
\item The database
  \texttt{hol.mm}\index{Higher-order Logic database (\texttt{hol.mm})},
  aka the
  Higher-Order Logic (HOL) Explorer\index{Higher-Order Logic (HOL) Explorer},
  starts with HOL (also called simple type theory) and derives
  equivalents to ZFC axioms, connecting the two approaches.
\end{itemize}

Since the days of David Hilbert,\index{Hilbert, David} mathematicians have
been concerned with the fact that the metalanguage\index{metalanguage} used to
describe mathematics may be stronger than the mathematics being described.
Metamath\index{Metamath}'s underlying finitary\index{finitary proof},
constructive nature provides a good philosophical basis for studying even the
weakest logics.\index{weak logic}

The usual treatment of many non-standard formal systems\index{formal
system} uses model theory\index{model theory} or proof theory\index{proof
theory} to describe these systems; these theories, in turn, are based on
standard set theory.  In other words, a non-standard formal system is defined
as a set with certain properties, and standard set theory is used to derive
additional properties of this set.  The standard set theory database provided
with Metamath can be used for this purpose, and when used this way
the development of a special
axiom system for the non-standard formal system becomes unnecessary.  The
model- or proof-theoretic approach often allows you to prove much deeper
results with less effort.

Metamath supports both approaches.  You can define the non-standard
formal system directly, or define the non-standard formal system as
a set with certain properties, whichever you find most helpful.

%\section{Additional Remarks}

\subsection{Metamath and Its Philosophy}

Closely related to Metamath\index{Metamath} is a philosophy or way of looking
at mathematics. This philosophy is related to the formalist
philosophy\index{formalism} of Hilbert\index{Hilbert, David} and his followers
\cite[pp.~1203--1208]{Kline}\index{Kline, Morris}
\cite[p.~6]{Behnke}\index{Behnke, H.}. In this philosophy, mathematics is
viewed as nothing more than a set of rules that manipulate symbols, together
with the consequences of those rules.  While the mathematics being described
may be complex, the rules used to describe it (the
``metamathematics''\index{metamathematics}) should be as simple as possible.
In particular, proofs should be restricted to dealing with concrete objects
(the symbols we write on paper rather than the abstract concepts they
represent) in a constructive manner; these are called ``finitary''
proofs\index{finitary proof} \cite[pp.~2--3]{Shoenfield}\index{Shoenfield,
Joseph R.}.

Whether or not you find Metamath interesting or useful will in part depend on
the appeal you find in its philosophy, and this appeal will probably depend on
your particular goals with respect to mathematics.  For example, if you are a
pure mathematician at the forefront of discovering new mathematical knowledge,
you will probably find that the rigid formality of Metamath stifles your
creativity.  On the other hand, we would argue that once this knowledge is
discovered, there are advantages to documenting it in a standard format that
will make it accessible to others.  Sixty years from now, your field may be
dormant, and as Davis and Hersh put it, your ``writings would become less
translatable than those of the Maya'' \cite[p.~37]{Davis}\index{Davis, Phillip
J.}.


\subsection{A History of the Approach Behind Metamath}

Probably the one work that has had the most motivating influence on
Metamath\index{Metamath} is Whitehead and Russell's monumental {\em Principia
Mathematica} \cite{PM}\index{Whitehead, Alfred North}\index{Russell,
Bertrand}\index{principia mathematica@{\em Principia Mathematica}}, whose aim
was to deduce all of mathematics from a small number of primitive ideas, in a
very explicit way that in principle anyone could understand and follow.  While
this work was tremendously influential in its time, from a modern perspective
it suffers from several drawbacks.  Both its notation and its underlying
axioms are now considered dated and are no longer used.  From our point of
view, its development is not really as accessible as we would like to see; for
practical reasons, proofs become more and more sketchy as its mathematics
progresses, and working them out in fine detail requires a degree of
mathematical skill and patience that many people don't have.  There are
numerous small errors, which is understandable given the tedious, technical
nature of the proofs and the lack of a computer to verify the details.
However, even today {\em Principia Mathematica} stands out as the work closest
in spirit to Metamath.  It remains a mind-boggling work, and one can't help
but be amazed at seeing ``$1+1=2$'' finally appear on page 83 of Volume II
(Theorem *110.643).

The origin of the proof notation used by Metamath dates back to the 1950's,
when the logician C.~A.~Meredith expressed his proofs in a compact notation
called ``condensed detachment''\index{condensed detachment}
\cite{Hindley}\index{Hindley, J. Roger} \cite{Kalman}\index{Kalman, J. A.}
\cite{Meredith}\index{Meredith, C. A.} \cite{Peterson}\index{Peterson, Jeremy
George}.  This notation allows proofs to be communicated unambiguously by
merely referencing the axiom\index{axiom}, rule\index{rule}, or
theorem\index{theorem} used at each step, without explicitly indicating the
substitutions\index{substitution!variable}\index{variable substitution} that
have to be made to the variables in that axiom, rule, or theorem.  Ordinarily,
condensed detachment is more or less limited to propositional
calculus\index{propositional calculus}.  The concept has been extended to
first-order logic\index{first-order logic} in \cite{Megill}\index{Megill,
Norman}, making it is easy to write a small computer program to verify proofs
of simple first-order logic theorems.\index{condensed detachment!and
first-order logic}

A key concept behind the notation of condensed detachment is called
``unification,''\index{unification} which is an algorithm for determining what
substitutions\index{substitution!variable}\index{variable substitution} to
variables have to be made to make two expressions match each other.
Unification was first precisely defined by the logician J.~A.~Robinson, who
used it in the development of a powerful
theorem-proving technique called the ``resolution principle''
\cite{Robinson}\index{Robinson's resolution principle}. Metamath does not make
use of the resolution principle, which is intended for systems of first-order
logic.\index{first-order logic}  Metamath's use is not restricted to
first-order logic, and as we have mentioned it does not automatically discover
proofs.  However, unification is a key idea behind Metamath's proof
notation, and Metamath makes use of a very simple version of it
(Section~\ref{unify}).

\subsection{Metamath and First-Order Logic}

First-order logic\index{first-order logic} is the supporting structure
for standard mathematics.  On top of it is set theory, which contains
the axioms from which virtually all of mathematics can be derived---a
remarkable fact.\index{category
theory}\index{cardinal, inaccessible}\label{categoryth}\footnote{An exception seems
to be category theory.  There are several schools of thought on whether
category theory is derivable from set theory.  At a minimum, it appears
that an additional axiom is needed that asserts the existence of an
``inaccessible cardinal'' (a type of infinity so large that standard set
theory can't prove or deny that it exists).
%
%%%% (I took this out that was in previous editions:)
% But it is also argued that not just one but a ``proper class'' of them
% is needed, and the existence of proper classes is impossible in standard
% set theory.  (A proper class is a collection of sets so huge that no set
% can contain it as an element.  Proper classes can lead to
% inconsistencies such as ``Russell's paradox.''  The axioms of standard
% set theory are devised so as to deny the existence of proper classes.)
%
For more information, see
\cite[pp.~328--331]{Herrlich}\index{Herrlich, Horst} and
\cite{Blass}\index{Blass, Andrea}.}

One of the things that makes Metamath\index{Metamath} more practical for
first-order theories is a set of axioms for first-order logic designed
specifically with Metamath's approach in mind.  These are included in
the database \texttt{set.mm}\index{set theory database (\texttt{set.mm})}.
See Chapter~\ref{fol} for a detailed
description; the axioms are shown in Section~\ref{metaaxioms}.  While
logically equivalent to standard axiom systems, our axiom system breaks
up the standard axioms into smaller pieces such that from them, you can
directly derive what in other systems can only be derived as higher-level
``metatheorems.''\index{metatheorem}  In other words, it is more powerful than
the standard axioms from a metalogical point of view.  A rigorous
justification for this system and its ``metalogical
completeness''\index{metalogical completeness} is found in
\cite{Megill}\index{Megill, Norman}.  The system is closely related to a
system developed by Monk\index{Monk, J. Donald} and Tarski\index{Tarski,
Alfred} in 1965 \cite{Monks}.

For example, the formula $\exists x \, x = y $ (given $y$, there exists some
$x$ equal to it) is a theorem of logic,\footnote{Specifically, it is a theorem
of those systems of logic that assume non-empty domains.  It is not a theorem
of more general systems that include the empty domain\index{empty domain}, in
which nothing exists, period!  Such systems are called ``free
logics.''\index{free logic} For a discussion of these systems, see
\cite{Leblanc}\index{Leblanc, Hugues}.  Since our use for logic is as a basis
for set theory, which has a non-empty domain, it is more convenient (and more
traditional) to use a less general system.  An interesting curiosity is that,
using a free logic as a basis for Zermelo--Fraenkel set
theory\index{Zermelo--Fraenkel set theory} (with the redundant Axiom of the
Null Set omitted),\index{Axiom of the Null Set} we cannot even prove the
existence of a single set without assuming the axiom of infinity!\index{Axiom
of Infinity}} whether or not $x$ and $y$ are distinct variables\index{distinct
variables}.  In many systems of logic, we would have to prove two theorems to
arrive at this result.  First we would prove ``$\exists x \, x = x $,'' then
we would separately prove ``$\exists x \, x = y $, where $x$ and $y$ are
distinct variables.''  We would then combine these two special cases ``outside
of the system'' (i.e.\ in our heads) to be able to claim, ``$\exists x \, x =
y $, regardless of whether $x$ and $y$ are distinct.''  In other words, the
combination of the two special cases is a
metatheorem.  In the system of logic
used in Metamath's set theory\index{set theory database (\texttt{set.mm})}
database, the axioms of logic are broken down into small pieces that allow
them to be reassembled in such a way that theorems such as these can be proved
directly.

Breaking down the axioms in this way makes them look peculiar and not very
intuitive at first, but rest assured that they are correct and complete.  Their
correctness is ensured because they are theorem schemes of standard first-order
logic (which you can easily verify if you are a logician).  Their completeness
follows from the fact that we explicitly derive the standard axioms of
first-order logic as theorems.  Deriving the standard axioms is somewhat
tricky, but once we're there, we have at our disposal a system that is less
awkward to work with in formal proofs\index{formal proof}.  In technical terms
that logicians understand, we eliminate the cumbersome concepts of ``free
variable,''\index{free variable} ``bound variable,''\index{bound variable} and
``proper substitution''\index{proper substitution}\index{substitution!proper}
as primitive notions.  These concepts are present in our system but are
defined in terms of concepts expressed by the axioms and can be eliminated in
principle.  In standard systems, these concepts are really like additional,
implicit axioms\index{implicit axiom} that are somewhat complex and cannot be
eliminated.

The traditional approach to logic, wherein free variables and proper
substitution is defined, is also possible to do directly in the Metamath
language.  However, the notation tends to become awkward, and there are
disadvantages:  for example, extending the definition of a wff with a
definition is awkward, because the free variable and proper substitution
concepts have to have their definitions also extended.  Our choice of
axioms for \texttt{set.mm} is to a certain extent a matter of style, in
an attempt to achieve overall simplicity, but you should also be aware
that the traditional approach is possible as well if you should choose
it.

\chapter{Using the Metamath Program}
\label{using}

\section{Installation}

The way that you install Metamath\index{Metamath!installation} on your
computer system will vary for different computers.  Current
instructions are provided with the Metamath program download at
\url{http://metamath.org}.  In general, the installation is simple.
There is one file containing the Metamath program itself.  This file is
usually called \texttt{metamath} or \texttt{metamath.}{\em xxx} where
{\em xxx} is the convention (such as \texttt{exe}) for an executable
program on your operating system.  There are several additional files
containing samples of the Metamath language, all ending with
\texttt{.mm}.  The file \texttt{set.mm}\index{set theory database
(\texttt{set.mm})} contains logic and set theory and can be used as a
starting point for other areas of mathematics.

You will also need a text editor\index{text editor} capable of editing plain
{\sc ascii}\footnote{American Standard Code for Information Interchange.} text
in order to prepare your input files.\index{ascii@{\sc ascii}}  Most computers
have this capability built in.  Note that plain text is not necessarily the
default for some word processing programs\index{word processor}, especially if
they can handle different fonts; for example, with Microsoft Word\index{Word
(Microsoft)}, you must save the file in the format ``Text Only With Line
Breaks'' to get a plain text\index{plain text} file.\footnote{It is
recommended that all lines in a Metamath source file be 79 characters or less
in length for compatibility among different computer terminals.  When creating
a source file on an editor such as Word, select a monospaced
font\index{monospaced font} such as Courier\index{Courier font} or
Monaco\index{Monaco font} to make this easier to achieve.  Better yet,
just use a plain text editor such as Notepad.}

On some computer systems, Metamath does not have the capability to print
its output directly; instead, you send its output to a file (using the
\texttt{open} commands described later).  The way you print this output
file depends on your computer.\index{printers} Some computers have a
print command, whereas with others, you may have to read the file into
an editor and print it from there.

If you want to print your Metamath source files with typeset formulas
containing standard mathematical symbols, you will need the \LaTeX\
typesetting program\index{latex@{\LaTeX}}, which is widely and freely
available for most operating systems.  It runs natively on Unix and
Linux, and can be installed on Windows as part of the free Cygwin
package (\url{http://cygwin.com}).

You can also produce {\sc html}\footnote{HyperText Markup Language.}
web pages.  The {\tt help html} command in the Metamath program will
assist you with this feature.

\section{Your First Formal System}\label{start}
\subsection{From Nothing to Zero}\label{startf}

To give you a feel for what the Metamath\index{Metamath} language looks like,
we will take a look at a very simple example from formal number
theory\index{number theory}.  This example is taken from
Mendelson\index{Mendelson, Elliot} \cite[p. 123]{Mendelson}.\footnote{To keep
the example simple, we have changed the formalism slightly, and what we call
axioms\index{axiom} are strictly speaking theorems\index{theorem} in
\cite{Mendelson}.}  We will look at a small subset of this theory, namely that
part needed for the first number theory theorem proved in \cite{Mendelson}.

First we will look at a standard formal proof\index{formal proof} for the
example we have picked, then we will look at the Metamath version.  If you
have never been exposed to formal proofs, the notation may seem to be such
overkill to express such simple notions that you may wonder if you are missing
something.  You aren't.  The concepts involved are in fact very simple, and a
detailed breakdown in this fashion is necessary to express the proof in a way
that can be verified mechanically.  And as you will see, Metamath breaks the
proof down into even finer pieces so that the mechanical verification process
can be about as simple as possible.

Before we can introduce the axioms\index{axiom} of the theory, we must define
the syntax rules for forming legal expressions\index{syntax rules}
(combinations of symbols) with which those axioms can be used. The number 0 is
a {\bf term}\index{term}; and if $ t$ and $r$ are terms, so is $(t+r)$. Here,
$ t$ and $r$ are ``metavariables''\index{metavariable} ranging over terms; they
themselves do not appear as symbols in an actual term.  Some examples of
actual terms are $(0 + 0)$ and $((0+0)+0)$.  (Note that our theory describes
only the number zero and sums of zeroes.  Of course, not much can be done with
such a trivial theory, but remember that we have picked a very small subset of
complete number theory for our example.  The important thing for you to focus
on is our definitions that describe how symbols are combined to form valid
expressions, and not on the content or meaning of those expressions.) If $ t$
and $r$ are terms, an expression of the form $ t=r$ is a {\bf wff}
(well-formed formula)\index{well-formed formula (wff)}; and if $P$ and $Q$ are
wffs, so is $(P\rightarrow Q)$ (which means ``$P$ implies
$Q$''\index{implication ($\rightarrow$)} or ``if $P$ then $Q$'').
Here $P$ and $Q$ are metavariables ranging over wffs.  Examples of actual
wffs are $0=0$, $(0+0)=0$, $(0=0 \rightarrow (0+0)=0)$, and $(0=0\rightarrow
(0=0\rightarrow 0=(0+0)))$.  (Our notation makes use of more parentheses than
are customary, but the elimination of ambiguity this way simplifies our
example by avoiding the need to define operator precedence\index{operator
precedence}.)

The {\bf axioms}\index{axiom} of our theory are all wffs of the following
form, where $ t$, $r$, and $s$ are any terms:

%Latex p. 92
\renewcommand{\theequation}{A\arabic{equation}}

\begin{equation}
(t=r\rightarrow (t=s\rightarrow r=s))
\end{equation}
\begin{equation}
(t+0)=t
\end{equation}

Note that there are an infinite number of axioms since there are an infinite
number of possible terms.  A1 and A2 are properly called ``axiom
schemes,''\index{axiom scheme} but we will refer to them as ``axioms'' for
brevity.

An axiom is a {\bf theorem}; and if $P$ and $(P\rightarrow Q)$ are theorems
(where $P$ and $Q$ are wffs), then $Q$ is also a theorem.\index{theorem}  The
second part of this definition is called the modus ponens (MP) rule of
inference\index{inference rule}\index{modus ponens}.  It allows us to obtain
new theorems from old ones.

The {\bf proof}\index{proof} of a theorem is a sequence of one or more
theorems, each of which is either an axiom or the result of modus ponens
applied to two previous theorems in the sequence, and the last of which is the
theorem being proved.

The theorem we will prove for our example is very simple:  $ t=t$.  The proof of
our theorem follows.  Study it carefully until you feel sure you
understand it.\label{zeroproof}

% Use tabu so that lines will wrap automatically as needed.
\begin{tabu} { l X X }
1. & $(t+0)=t$ & (by axiom A2) \\
2. & $(t+0)=t$ & (by axiom A2) \\
3. & $((t+0)=t \rightarrow ((t+0)=t\rightarrow t=t))$ & (by axiom A1) \\
4. & $((t+0)=t\rightarrow t=t)$ & (by MP applied to steps 2 and 3) \\
5. & $t=t$ & (by MP applied to steps 1 and 4) \\
\end{tabu}

(You may wonder why step 1 is repeated twice.  This is not necessary in the
formal language we have defined, but in Metamath's ``reverse Polish
notation''\index{reverse Polish notation (RPN)} for proofs, a previous step
can be referred to only once.  The repetition of step~1 here will enable you
to see more clearly the correspondence of this proof with the
Metamath\index{Metamath} version on p.~\pageref{demoproof}.)

Our theorem is more properly called a ``theorem scheme,''\index{theorem
scheme} for it represents an infinite number of theorems, one for each
possible term $ t$.  Two examples of actual theorems would be $0=0$ and
$(0+0)=(0+0)$.  Rarely do we prove actual theorems, since by proving schemes
we can prove an infinite number of theorems in one fell swoop.  Similarly, our
proof should really be called a ``proof scheme.''\index{proof scheme}  To
obtain an actual proof, pick an actual term to use in place of $ t$, and
substitute it for $ t$ throughout the proof.

Let's discuss what we have done here.  The axioms\index{axiom} of our theory,
A1 and A2, are trivial and obvious.  Everyone knows that adding zero to
something doesn't change it, and also that if two things are equal to a third,
then they are equal to each other. In fact, stating the trivial and obvious is
a goal to strive for in any axiomatic system.  From trivial and obvious truths
that everyone agrees upon, we can prove results that are not so obvious yet
have absolute faith in them.  If we trust the axioms and the rules, we must,
by definition, trust the consequences of those axioms and rules, if logic is
to mean anything at all.

Our rule of inference\index{rule}, modus ponens\index{modus ponens}, is also
pretty obvious once you understand what it means.  If we prove a fact $P$, and
we also prove that $P$ implies $Q$, then $Q$ necessarily follows as a new
fact.  The rule provides us with a means for obtaining new facts (i.e.\
theorems\index{theorem}) from old ones.

The theorem that we have proved, $ t=t$, is so fundamental that you may wonder
why it isn't one of the axioms\index{axiom}.  In some axiom systems of
arithmetic, it {\em is} an axiom.  The choice of axioms in a theory is to some
extent arbitrary and even an art form, constrained only by the requirement
that any two equivalent axiom systems be able to derive each other as
theorems.  We could imagine that the inventor of our axiom system originally
included $ t=t$ as an axiom, then discovered that it could be derived as a
theorem from the other axioms.  Because of this, it was not necessary to
keep it as an axiom.  By eliminating it, the final set of axioms became
that much simpler.

Unless you have worked with formal proofs\index{formal proof} before, it
probably wasn't apparent to you that $ t=t$ could be derived from our two
axioms until you saw the proof. While you certainly believe that $ t=t$ is
true, you might not be able to convince an imaginary skeptic who believes only
in our two axioms until you produce the proof.  Formal proofs such as this are
hard to come up with when you first start working with them, but after you get
used to them they can become interesting and fun.  Once you understand the
idea behind formal proofs you will have grasped the fundamental principle that
underlies all of mathematics.  As the mathematics becomes more sophisticated,
its proofs become more challenging, but ultimately they all can be broken down
into individual steps as simple as the ones in our proof above.

Mendelson's\index{Mendelson, Elliot} book, from which our example was taken,
contains a number of detailed formal proofs such as these, and you may be
interested in looking it up.  The book is intended for mathematicians,
however, and most of it is rather advanced.  Popular literature describing
formal proofs\index{formal proof} include \cite[p.~296]{Rucker}\index{Rucker,
Rudy} and \cite[pp.~204--230]{Hofstadter}\index{Hofstadter, Douglas R.}.

\subsection{Converting It to Metamath}\label{convert}

Formal proofs\index{formal proof} such as the one in our example break down
logical reasoning into small, precise steps that leave little doubt that the
results follow from the axioms\index{axiom}.  You might think that this would
be the finest breakdown we can achieve in mathematics.  However, there is more
to the proof than meets the eye. Although our axioms were rather simple, a lot
of verbiage was needed before we could even state them:  we needed to define
``term,'' ``wff,'' and so on.  In addition, there are a number of implied
rules that we haven't even mentioned. For example, how do we know that step 3
of our proof follows from axiom A1? There is some hidden reasoning involved in
determining this.  Axiom A1 has two occurrences of the letter $ t$.  One of
the implied rules states that whatever we substitute for $ t$ must be a legal
term\index{term}.\footnote{Some authors make this implied rule explicit by
stating, ``only expressions of the above form are terms,'' after defining
``term.''}  The expression $ t+0$ is pretty obviously a legal term whenever $
t$ is, but suppose we wanted to substitute a huge term with thousands of
symbols?  Certainly a lot of work would be involved in determining that it
really is a term, but in ordinary formal proofs all of this work would be
considered a single ``step.''

To express our axiom system in the Metamath\index{Metamath} language, we must
describe this auxiliary information in addition to the axioms themselves.
Metamath does not know what a ``term'' or a ``wff''\index{well-formed formula
(wff)} is.  In Metamath, the specification of the ways in which we can combine
symbols to obtain terms and wffs are like little axioms in themselves.  These
auxiliary axioms are expressed in the same notation as the ``real''
axioms\index{axiom}, and Metamath does not distinguish between the two.  The
distinction is made by you, i.e.\ by the way in which you interpret the
notation you have chosen to express these two kinds of axioms.

The Metamath language breaks down mathematical proofs into tiny pieces, much
more so than in ordinary formal proofs\index{formal proof}.  If a single
step\index{proof step} involves the
substitution\index{substitution!variable}\index{variable substitution} of a
complex term for one of its variables, Metamath must see this single step
broken down into many small steps.  This fine-grained breakdown is what gives
Metamath generality and flexibility as it lets it not be limited to any
particular mathematical notation.

Metamath's proof notation is not, in itself, intended to be read by humans but
rather is in a compact format intended for a machine.  The Metamath program
will convert this notation to a form you can understand, using the \texttt{show
proof}\index{\texttt{show proof} command} command.  You can tell the program what
level of detail of the proof you want to look at.  You may want to look at
just the logical inference steps that correspond
to ordinary formal proof steps,
or you may want to see the fine-grained steps that prove that an expression is
a term.

Here, without further ado, is our example converted to the
Metamath\index{Metamath} language:\index{metavariable}\label{demo0}

\begin{verbatim}
$( Declare the constant symbols we will use $)
    $c 0 + = -> ( ) term wff |- $.
$( Declare the metavariables we will use $)
    $v t r s P Q $.
$( Specify properties of the metavariables $)
    tt $f term t $.
    tr $f term r $.
    ts $f term s $.
    wp $f wff P $.
    wq $f wff Q $.
$( Define "term" and "wff" $)
    tze $a term 0 $.
    tpl $a term ( t + r ) $.
    weq $a wff t = r $.
    wim $a wff ( P -> Q ) $.
$( State the axioms $)
    a1 $a |- ( t = r -> ( t = s -> r = s ) ) $.
    a2 $a |- ( t + 0 ) = t $.
$( Define the modus ponens inference rule $)
    ${
       min $e |- P $.
       maj $e |- ( P -> Q ) $.
       mp  $a |- Q $.
    $}
$( Prove a theorem $)
    th1 $p |- t = t $=
  $( Here is its proof: $)
       tt tze tpl tt weq tt tt weq tt a2 tt tze tpl
       tt weq tt tze tpl tt weq tt tt weq wim tt a2
       tt tze tpl tt tt a1 mp mp
     $.
\end{verbatim}\index{metavariable}

A ``database''\index{database} is a set of one or more {\sc ascii} source
files.  Here's a brief description of this Metamath\index{Metamath} database
(which consists of this single source file), so that you can understand in
general terms what is going on.  To understand the source file in detail, you
should read Chapter~\ref{languagespec}.

The database is a sequence of ``tokens,''\index{token} which are normally
separated by spaces or line breaks.  The only tokens that are built into
the Metamath language are those beginning with \texttt{\$}.  These tokens
are called ``keywords.''\index{keyword}  All other tokens are
user-defined, and their names are arbitrary.

As you might have guessed, the Metamath token \texttt{\$(}\index{\texttt{\$(} and
\texttt{\$)} auxiliary keywords} starts a comment and \texttt{\$)} ends a comment.

The Metamath tokens \texttt{\$c}\index{\texttt{\$c} statement},
\texttt{\$v}\index{\texttt{\$v} statement},
\texttt{\$e}\index{\texttt{\$e} statement},
\texttt{\$f}\index{\texttt{\$f} statement},
\texttt{\$a}\index{\texttt{\$a} statement}, and
\texttt{\$p}\index{\texttt{\$p} statement} specify ``statements'' that
end with \texttt{\$.}\,.\index{\texttt{\$.}\ keyword}

The Metamath tokens \texttt{\$c} and \texttt{\$v} each declare\index{constant
declaration}\index{variable declaration} a list of user-defined tokens, called
``math symbols,''\index{math symbol} that the database will reference later
on.  All of the math symbols they define you have seen earlier except the
turnstile symbol \texttt{|-} ($\vdash$)\index{turnstile ({$\,\vdash$})}, which is
commonly used by logicians to mean ``a proof exists for.''  For us
the turnstile is just a
convenient symbol that distinguishes expressions that are axioms\index{axiom}
or theorems\index{theorem} from expressions that are terms or wffs.

The \texttt{\$c} statement declares ``constants''\index{constant} and
the \texttt{\$v} statement declares
``variables''\index{variable}\index{constant declaration}\index{variable
declaration} (or more precisely, metavariables\index{metavariable}).  A
variable may be substituted\index{substitution!variable}\index{variable
substitution} with sequences of math symbols whereas a constant may not
be substituted with anything.

It may seem redundant to require both \texttt{\$c}\index{\texttt{\$c} statement} and
\texttt{\$v}\index{\texttt{\$v} statement} statements (since any math
symbol\index{math symbol} not specified with a \texttt{\$c} statement could be
presumed to be a variable), but this provides for better error checking and
also allows math symbols to be redeclared\index{redeclaration of symbols}
(Section~\ref{scoping}).

The token \texttt{\$f}\index{\texttt{\$f} statement} specifies a
statement called a ``variable-type hypothesis'' (also called a
``floating hypothesis'') and \texttt{\$e}\index{\texttt{\$e} statement}
specifies a ``logical hypothesis'' (also called an ``essential
hypothesis'').\index{hypothesis}\index{variable-type
hypothesis}\index{logical hypothesis}\index{floating
hypothesis}\index{essential hypothesis} The token
\texttt{\$a}\index{\texttt{\$a} statement} specifies an ``axiomatic
assertion,''\index{axiomatic assertion} and
\texttt{\$p}\index{\texttt{\$p} statement} specifies a ``provable
assertion.''\index{provable assertion} To the left of each occurrence of
these four tokens is a ``label''\index{label} that identifies the
hypothesis or assertion for later reference.  For example, the label of
the first axiomatic assertion is \texttt{tze}.  A \texttt{\$f} statement
must contain exactly two math symbols, a constant followed by a
variable.  The \texttt{\$e}, \texttt{\$a}, and \texttt{\$p} statements
each start with a constant followed by, in general, an arbitrary
sequence of math symbols.

Associated with each assertion\index{assertion} is a set of hypotheses
that must be satisfied in order for the assertion to be used in a proof.
These are called the ``mandatory hypotheses''\index{mandatory
hypothesis} of the assertion.  Among those hypotheses whose ``scope''
(described below) includes the assertion, \texttt{\$e} hypotheses are
always mandatory and \texttt{\$f}\index{\texttt{\$f} statement}
hypotheses are mandatory when they share their variable with the
assertion or its \texttt{\$e} hypotheses.  The exact rules for
determining which hypotheses are mandatory are described in detail in
Sections~\ref{frames} and \ref{scoping}.  For example, the mandatory
hypotheses of assertion \texttt{tpl} are \texttt{tt} and \texttt{tr},
whereas assertion \texttt{tze} has no mandatory hypotheses because it
contains no variables and has no \texttt{\$e}\index{\texttt{\$e}
statement} hypothesis.  Metamath's \texttt{show statement}
command\index{\texttt{show statement} command}, described in the next
section, will show you a statement's mandatory hypotheses.

Sometimes we need to make a hypothesis relevant to only certain
assertions.  The set of statements to which a hypothesis is relevant is
called its ``scope.''  The Metamath brackets,
\texttt{\$\char`\{}\index{\texttt{\$\char`\{} and \texttt{\$\char`\}}
keywords} and \texttt{\$\char`\}}, define a ``block''\index{block} that
delimits the scope of any hypothesis contained between them.  The
assertion \texttt{mp} has mandatory hypotheses \texttt{wp}, \texttt{wq},
\texttt{min}, and \texttt{maj}.  The only mandatory hypothesis of
\texttt{th1}, on the other hand, is \texttt{tt}, since \texttt{th1}
occurs outside of the block containing \texttt{min} and \texttt{maj}.

Note that \texttt{\$\char`\{} and \texttt{\$\char`\}} do not affect the
scope of assertions (\texttt{\$a} and \texttt{\$p}).  Assertions are always
available to be referenced by any later proof in the source file.

Each provable assertion (\texttt{\$p}\index{\texttt{\$p} statement}
statement) has two parts.  The first part is the
assertion\index{assertion} itself, which is a sequence of math
symbol\index{math symbol} tokens placed between the \texttt{\$p} token
and a \texttt{\$=}\index{\texttt{\$=} keyword} token.  The second part
is a ``proof,'' which is a list of label tokens placed between the
\texttt{\$=} token and the \texttt{\$.}\index{\texttt{\$.}\ keyword}\
token that ends the statement.\footnote{If you've looked at the
\texttt{set.mm} database, you may have noticed another notation used for
proofs.  The other notation is called ``compressed.''\index{compressed
proof}\index{proof!compressed} It reduces the amount of space needed to
store a proof in the database and is described in
Appendix~\ref{compressed}.  In the example above, we use
``normal''\index{normal proof}\index{proof!normal} notation.} The proof
acts as a series of instructions to the Metamath program, telling it how
to build up the sequence of math symbols contained in the assertion part of
the \texttt{\$p} statement, making use of the hypotheses of the
\texttt{\$p} statement and previous assertions.  The construction takes
place according to precise rules.  If the list of labels in the proof
causes these rules to be violated, or if the final sequence that results
does not match the assertion, the Metamath program will notify you with
an error message.

If you are familiar with reverse Polish notation (RPN), which is sometimes used
on pocket calculators, here in a nutshell is how a proof works.  Each
hypothesis label\index{hypothesis label} in the proof is pushed\index{push}
onto the RPN stack\index{stack}\index{RPN stack} as it is encountered. Each
assertion label\index{assertion label} pops\index{pop} off the stack as many
entries as the referenced assertion has mandatory hypotheses.  Variable
substitutions\index{substitution!variable}\index{variable substitution} are
computed which, when made to the referenced assertion's mandatory hypotheses,
cause these hypotheses to match the stack entries. These same substitutions
are then made to the variables in the referenced assertion itself, which is
then pushed onto the stack.  At the end of the proof, there should be one
stack entry, namely the assertion being proved.  This process is explained in
detail in Section~\ref{proof}.

Metamath's proof notation is not very readable for humans, but it allows the
proof to be stored compactly in a file.  The Metamath\index{Metamath} program
has proof display features that let you see what's going on in a more
readable way, as you will see in the next section.

The rules used in verifying a proof are not based on any built-in syntax of the
symbol sequence in an assertion\index{assertion} nor on any built-in meanings
attached to specific symbol names.  They are based strictly on symbol
matching:  constants\index{constant} must match themselves, and
variables\index{variable} may be replaced with anything that allows a match to
occur.  For example, instead of \texttt{term}, \texttt{0}, and \verb$|-$ we could
have just as well used \texttt{yellow}, \texttt{zero}, and \texttt{provable}, as long
as we did so consistently throughout the database.  Also, we could have used
\texttt{is provable} (two tokens) instead of \verb$|-$ (one token) throughout the
database.  In each of these cases, the proof would be exactly the same.  The
independence of proofs and notation means that you have a lot of flexibility to
change the notation you use without having to change any proofs.

\section{A Trial Run}\label{trialrun}

Now you are ready to try out the Metamath\index{Metamath} program.

On all computer systems, Metamath has a standard ``command line
interface'' (CLI)\index{command line interface (CLI)} that allows you to
interact with it.  You supply commands to the CLI by typing them on the
keyboard and pressing your keyboard's {\em return} key after each line
you enter.  The CLI is designed to be easy to use and has built-in help
features.

The first thing you should do is to use a text editor to create a file
called \texttt{demo0.mm} and type into it the Metamath source shown on
p.~\pageref{demo0}.  Actually, this file is included with your Metamath
software package, so check that first.  If you type it in, make sure
that you save it in the form of ``plain {\sc ascii} text with line
breaks.''  Most word processors will have this feature.

Next you must run the Metamath program.  Depending on your computer
system and how Metamath is installed, this could range from clicking the
mouse on the Metamath icon to typing \texttt{run metamath} to typing
simply \texttt{metamath}.  (Metamath's {\tt help invoke} command describes
alternate ways of invoking the Metamath program.)

When you first enter Metamath\index{Metamath}, it will be at the CLI, waiting
for your input. You will see something like the following on your screen:
\begin{verbatim}
Metamath - Version 0.177 27-Apr-2019
Type HELP for help, EXIT to exit.
MM>
\end{verbatim}
The \texttt{MM>} prompt means that Metamath is waiting for a command.
Command keywords\index{command keyword} are not case sensitive;
we will use lower-case commands in our examples.
The version number and its release date will probably be different on your
system from the one we show above.

The first thing that you need to do is to read in your
database:\index{\texttt{read} command}\footnote{If a directory path is
needed on Unix,\index{Unix file names}\index{file names!Unix} you should
enclose the path/file name in quotes to prevent Metamath from thinking
that the \texttt{/} in the path name is a command qualifier, e.g.,
\texttt{read \char`\"db/set.mm\char`\"}.  Quotes are optional when there
is no ambiguity.}
\begin{verbatim}
MM> read demo0.mm
\end{verbatim}
Remember to press the {\em return} key after entering this command.  If
you omit the file name, Metamath will prompt you for one.   The syntax for
specifying a Macintosh file name path is given in a footnote on
p.~\pageref{includef}.\index{Macintosh file names}\index{file
names!Macintosh}

If there are any syntax errors in the database, Metamath will let you know
when it reads in the file.  The one thing that Metamath does not check when
reading in a database is that all proofs are correct, because this would
slow it down too much.  It is a good idea to periodically verify the proofs in
a database you are making changes to.  To do this, use the following command
(and do it for your \texttt{demo0.mm} file now).  Note that the \texttt{*} is a
``wild card'' meaning all proofs in the file.\index{\texttt{verify proof} command}
\begin{verbatim}
MM> verify proof *
\end{verbatim}
Metamath will report any proofs that are incorrect.

It is often useful to save the information that the Metamath program displays
on the screen. You can save everything that happens on the screen by opening a
log file. You may want to do this before you read in a database so that you
can examine any errors later on.  To open a log file, type
\begin{verbatim}
MM> open log abc.log
\end{verbatim}
This will open a file called \texttt{abc.log}, and everything that appears on the
screen from this point on will be stored in this file.  The name of the log file
is arbitrary. To close the log file, type
\begin{verbatim}
MM> close log
\end{verbatim}

Several commands let you examine what's inside your database.
Section~\ref{exploring} has an overview of some useful ones.  The
\texttt{show labels} command lets you see what statement
labels\index{label} exist.  A \texttt{*} matches any combination of
characters, and \texttt{t*} refers to all labels starting with the
letter \texttt{t}.\index{\texttt{show labels} command} The \texttt{/all}
is a ``command qualifier''\index{command qualifier} that tells Metamath
to include labels of hypotheses.  (To see the syntax explained, type
\texttt{help show labels}.)  Type
\begin{verbatim}
MM> show labels t* /all
\end{verbatim}
Metamath will respond with
\begin{verbatim}
The statement number, label, and type are shown.
3 tt $f       4 tr $f       5 ts $f       8 tze $a
9 tpl $a      19 th1 $p
\end{verbatim}

You can use the \texttt{show statement} command to get information about a
particular statement.\index{\texttt{show statement} command}
For example, you can get information about the statement with label \texttt{mp}
by typing
\begin{verbatim}
MM> show statement mp /full
\end{verbatim}
Metamath will respond with
\begin{verbatim}
Statement 17 is located on line 43 of the file
"demo0.mm".
"Define the modus ponens inference rule"
17 mp $a |- Q $.
Its mandatory hypotheses in RPN order are:
  wp $f wff P $.
  wq $f wff Q $.
  min $e |- P $.
  maj $e |- ( P -> Q ) $.
The statement and its hypotheses require the
      variables:  Q P
The variables it contains are:  Q P
\end{verbatim}
The mandatory hypotheses\index{mandatory hypothesis} and their
order\index{RPN order} are
useful to know when you are trying to understand or debug a proof.

Now you are ready to look at what's really inside our proof.  First, here is
how to look at every step in the proof---not just the ones corresponding to an
ordinary formal proof\index{formal proof}, but also the ones that build up the
formulas that appear in each ordinary formal proof step.\index{\texttt{show
proof} command}
\begin{verbatim}
MM> show proof th1 /lemmon /all
\end{verbatim}

This will display the proof on the screen in the following format:
\begin{verbatim}
 1 tt            $f term t
 2 tze           $a term 0
 3 1,2 tpl       $a term ( t + 0 )
 4 tt            $f term t
 5 3,4 weq       $a wff ( t + 0 ) = t
 6 tt            $f term t
 7 tt            $f term t
 8 6,7 weq       $a wff t = t
 9 tt            $f term t
10 9 a2          $a |- ( t + 0 ) = t
11 tt            $f term t
12 tze           $a term 0
13 11,12 tpl     $a term ( t + 0 )
14 tt            $f term t
15 13,14 weq     $a wff ( t + 0 ) = t
16 tt            $f term t
17 tze           $a term 0
18 16,17 tpl     $a term ( t + 0 )
19 tt            $f term t
20 18,19 weq     $a wff ( t + 0 ) = t
21 tt            $f term t
22 tt            $f term t
23 21,22 weq     $a wff t = t
24 20,23 wim     $a wff ( ( t + 0 ) = t -> t = t )
25 tt            $f term t
26 25 a2         $a |- ( t + 0 ) = t
27 tt            $f term t
28 tze           $a term 0
29 27,28 tpl     $a term ( t + 0 )
30 tt            $f term t
31 tt            $f term t
32 29,30,31 a1   $a |- ( ( t + 0 ) = t -> ( ( t + 0 )
                                     = t -> t = t ) )
33 15,24,26,32 mp  $a |- ( ( t + 0 ) = t -> t = t )
34 5,8,10,33 mp  $a |- t = t
\end{verbatim}

The \texttt{/lemmon} command qualifier specifies what is known as a Lemmon-style
display\index{Lemmon-style proof}\index{proof!Lemmon-style}.  Omitting the
\texttt{/lemmon} qualifier results in a tree-style proof (see
p.~\pageref{treeproof} for an example) that is somewhat less explicit but
easier to follow once you get used to it.\index{tree-style
proof}\index{proof!tree-style}

The first number on each line is the step
number of the proof.  Any numbers that follow are step numbers assigned to the
hypotheses of the statement referenced by that step.  Next is the label of
the statement referenced by the step.  The statement type of the statement
referenced comes next, followed by the math symbol\index{math symbol} string
constructed by the proof up to that step.

The last step, 34, contains the statement that is being proved.

Looking at a small piece of the proof, notice that steps 3 and 4 have
established that
\texttt{( t + 0 )} and \texttt{t} are \texttt{term}\,s, and step 5 makes use of steps 3 and
4 to establish that \texttt{( t + 0 ) = t} is a \texttt{wff}.  Let Metamath
itself tell us in detail what is happening in step 5.  Note that the
``target hypothesis'' refers to where step 5 is eventually used, i.e., in step
34.
\begin{verbatim}
MM> show proof th1 /detailed_step 5
Proof step 5:  wp=weq $a wff ( t + 0 ) = t
This step assigns source "weq" ($a) to target "wp"
($f).  The source assertion requires the hypotheses
"tt" ($f, step 3) and "tr" ($f, step 4).  The parent
assertion of the target hypothesis is "mp" ($a,
step 34).
The source assertion before substitution was:
    weq $a wff t = r
The following substitutions were made to the source
assertion:
    Variable  Substituted with
     t         ( t + 0 )
     r         t
The target hypothesis before substitution was:
    wp $f wff P
The following substitution was made to the target
hypothesis:
    Variable  Substituted with
     P         ( t + 0 ) = t
\end{verbatim}

The full proof just shown is useful to understand what is going on in detail.
However, most of the time you will just be interested in
the ``essential'' or logical steps of a proof, i.e.\ those steps
that correspond to an
ordinary formal proof\index{formal proof}.  If you type
\begin{verbatim}
MM> show proof th1 /lemmon /renumber
\end{verbatim}
you will see\label{demoproof}
\begin{verbatim}
1 a2             $a |- ( t + 0 ) = t
2 a2             $a |- ( t + 0 ) = t
3 a1             $a |- ( ( t + 0 ) = t -> ( ( t + 0 )
                                     = t -> t = t ) )
4 2,3 mp         $a |- ( ( t + 0 ) = t -> t = t )
5 1,4 mp         $a |- t = t
\end{verbatim}
Compare this to the formal proof on p.~\pageref{zeroproof} and
notice the resemblance.
By default Metamath
does not show \texttt{\$f}\index{\texttt{\$f}
statement} hypotheses and everything branching off of them in the proof tree
when the proof is displayed; this makes the proof look more like an ordinary
mathematical proof, which does not normally incorporate the explicit
construction of expressions.
This is called the ``essential'' view
(at one time you had to add the
\texttt{/essential} qualifier in the \texttt{show proof}
command to get this view, but this is now the default).
You can could use the \texttt{/all} qualifier in the \texttt{show
proof} command to also show the explicit construction of expressions.
The \texttt{/renumber} qualifier means to renumber
the steps to correspond only to what is displayed.\index{\texttt{show proof}
command}

To exit Metamath, type\index{\texttt{exit} command}
\begin{verbatim}
MM> exit
\end{verbatim}

\subsection{Some Hints for Using the Command Line Interface}

We will conclude this quick introduction to Metamath\index{Metamath} with some
helpful hints on how to navigate your way through the commands.
\index{command line interface (CLI)}

When you type commands into Metamath's CLI, you only have to type as many
characters of a command keyword\index{command keyword} as are needed to make
it unambiguous.  If you type too few characters, Metamath will tell you what
the choices are.  In the case of the \texttt{read} command, only the \texttt{r} is
needed to specify it unambiguously, so you could have typed\index{\texttt{read}
command}
\begin{verbatim}
MM> r demo0.mm
\end{verbatim}
instead of
\begin{verbatim}
MM> read demo0.mm
\end{verbatim}
In our description, we always show the full command words.  When using the
Metamath CLI commands in a command file (to be read with the \texttt{submit}
command)\index{\texttt{submit} command}, it is good practice to use
the unabbreviated command to ensure your instructions will not become ambiguous
if more commands are added to the Metamath program in the future.

The command keywords\index{command
keyword} are not case sensitive; you may type either \texttt{read} or
\texttt{ReAd}.  File names may or may not be case sensitive, depending on your
computer's operating system.  Metamath label\index{label} and math
symbol\index{math symbol} tokens\index{token} are case-sensitive.

The \texttt{help} command\index{\texttt{help} command} will provide you
with a list of topics you can get help on.  You can then type
\texttt{help} {\em topic} to get help on that topic.

If you are uncertain of a command's spelling, just type as many characters
as you remember of the command.  If you have not typed enough characters to
specify it unambiguously, Metamath will tell you what choices you have.

\begin{verbatim}
MM> show s
         ^
?Ambiguous keyword - please specify SETTINGS,
STATEMENT, or SOURCE.
\end{verbatim}

If you don't know what argument to use as part of a command, type a
\texttt{?}\index{\texttt{]}@\texttt{?}\ in command lines}\ at the
argument position.  Metamath will tell you what it expected there.

\begin{verbatim}
MM> show ?
         ^
?Expected SETTINGS, LABELS, STATEMENT, SOURCE, PROOF,
MEMORY, TRACE_BACK, or USAGE.
\end{verbatim}

Finally, you may type just the first word or words of a command followed
by {\em return}.  Metamath will prompt you for the remaining part of the
command, showing you the choices at each step.  For example, instead of
typing \texttt{show statement th1 /full} you could interact in the
following manner:
\begin{verbatim}
MM> show
SETTINGS, LABELS, STATEMENT, SOURCE, PROOF,
MEMORY, TRACE_BACK, or USAGE <SETTINGS>? st
What is the statement label <th1>?
/ or nothing <nothing>? /
TEX, COMMENT_ONLY, or FULL <TEX>? f
/ or nothing <nothing>?
19 th1 $p |- t = t $= ... $.
\end{verbatim}
After each \texttt{?}\ in this mode, you must give Metamath the
information it requests.  Sometimes Metamath gives you a list of choices
with the default choice indicated by brackets \texttt{< > }. Pressing
{\em return} after the \texttt{?}\ will select the default choice.
Answering anything else will override the default.  Note that the
\texttt{/} in command qualifiers is considered a separate
token\index{token} by the parser, and this is why it is asked for
separately.

\section{Your First Proof}\label{frstprf}

Proofs are developed with the aid of the Proof Assistant\index{Proof
Assistant}.  We will now show you how the proof of theorem \texttt{th1}
was built.  So that you can repeat these steps, we will first have the
Proof Assistant erase the proof in Metamath's source buffer\index{source
buffer}, then reconstruct it.  (The source buffer is the place in memory
where Metamath stores the information in the database when it is
\texttt{read}\index{\texttt{read} command} in.  New or modified proofs
are kept in the source buffer until a \texttt{write source}
command\index{\texttt{write source} command} is issued.)  In practice, you
would place a \texttt{?}\index{\texttt{]}@\texttt{?}\ inside proofs}\
between \texttt{\$=}\index{\texttt{\$=} keyword} and
\texttt{\$.}\index{\texttt{\$.}\ keyword}\ in the database to indicate
to Metamath\index{Metamath} that the proof is unknown, and that would be
your starting point.  Whenever the \texttt{verify proof} command encounters
a proof with a \texttt{?}\ in place of a proof step, the statement is
identified as not proved.

When I first started creating Metamath proofs, I would write down
on a piece of paper the complete
formal proof\index{formal proof} as it would appear
in a \texttt{show proof} command\index{\texttt{show proof} command}; see
the display of \texttt{show proof th1 /lemmon /re\-num\-ber} above as an
example.  After you get used to using the Proof Assistant\index{Proof
Assistant} you may get to a point where you can ``see'' the proof in your mind
and let the Proof Assistant guide you in filling in the details, at least for
simpler proofs, but until you gain that experience you may find it very useful
to write down all the details in advance.
Otherwise you may waste a lot of time as you let it take you down a wrong path.
However, others do not find this approach as helpful.
For example, Thomas Brendan Leahy\index{Leahy, Thomas Brendan}
finds that it is more helpful to him to interactively
work backward from a machine-readable statement.
David A. Wheeler\index{Wheeler, David A.}
writes down a general approach, but develops the proof
interactively by switching between
working forwards (from hypotheses and facts likely to be useful) and
backwards (from the goal) until the forwards and backwards approaches meet.
In the end, use whatever approach works for you.

A proof is developed with the Proof Assistant by working backwards, starting
with the theorem\index{theorem} to be proved, and assigning each unknown step
with a theorem or hypothesis until no more unknown steps remain.  The Proof
Assistant will not let you make an assignment unless it can be ``unified''
with the unknown step.  This means that a
substitution\index{substitution!variable}\index{variable substitution} of
variables exists that will make the assignment match the unknown step.  On the
other hand, in the middle of a proof, when working backwards, often more than
one unification\index{unification} (set of substitutions) is possible, since
there is not enough information available at that point to uniquely establish
it.  In this case you can tell Metamath which unification to choose, or you
can continue to assign unknown steps until enough information is available to
make the unification unique.

We will assume you have entered Metamath and read in the database as described
above.  The following dialog shows how the proof was developed.  For more
details on what some of the commands do, refer to Section~\ref{pfcommands}.
\index{\texttt{prove} command}

\begin{verbatim}
MM> prove th1
Entering the Proof Assistant.  Type HELP for help, EXIT
to exit.  You will be working on the proof of statement th1:
  $p |- t = t
Note:  The proof you are starting with is already complete.
MM-PA>
\end{verbatim}

The \verb/MM-PA>/ prompt means we are inside the Proof
Assistant.\index{Proof Assistant} Most of the regular Metamath commands
(\texttt{show statement}, etc.) are still available if you need them.

\begin{verbatim}
MM-PA> delete all
The entire proof was deleted.
\end{verbatim}

We have deleted the whole proof so we can start from scratch.

\begin{verbatim}
MM-PA> show new_proof/lemmon/all
1 ?              $? |- t = t
\end{verbatim}

The \texttt{show new{\char`\_}proof} command\index{\texttt{show
new{\char`\_}proof} command} is like \texttt{show proof} except that we
don't specify a statement; instead, the proof we're working on is
displayed.

\begin{verbatim}
MM-PA> assign 1 mp
To undo the assignment, DELETE STEP 5 and INITIALIZE, UNIFY
if needed.
3   min=?  $? |- $2
4   maj=?  $? |- ( $2 -> t = t )
\end{verbatim}

The \texttt{assign} command\index{\texttt{assign} command} above means
``assign step 1 with the statement whose label is \texttt{mp}.''  Note
that step renumbering will constantly occur as you assign steps in the
middle of a proof; in general all steps from the step you assign until
the end of the proof will get moved up.  In this case, what used to be
step 1 is now step 5, because the (partial) proof now has five steps:
the four hypotheses of the \texttt{mp} statement and the \texttt{mp}
statement itself.  Let's look at all the steps in our partial proof:

\begin{verbatim}
MM-PA> show new_proof/lemmon/all
1 ?              $? wff $2
2 ?              $? wff t = t
3 ?              $? |- $2
4 ?              $? |- ( $2 -> t = t )
5 1,2,3,4 mp     $a |- t = t
\end{verbatim}

The symbol \texttt{\$2} is a temporary variable\index{temporary
variable} that represents a symbol sequence not yet known.  In the final
proof, all temporary variables will be eliminated.  The general format
for a temporary variable is \texttt{\$} followed by an integer.  Note
that \texttt{\$} is not a legal character in a math symbol (see
Section~\ref{dollardollar}, p.~\pageref{dollardollar}), so there will
never be a naming conflict between real symbols and temporary variables.

Unknown steps 1 and 2 are constructions of the two wffs used by the
modus ponens rule.  As you will see at the end of this section, the
Proof Assistant\index{Proof Assistant} can usually figure these steps
out by itself, and we will not have to worry about them.  Therefore from
here on we will display only the ``essential'' hypotheses, i.e.\ those
steps that correspond to traditional formal proofs\index{formal proof}.

\begin{verbatim}
MM-PA> show new_proof/lemmon
3 ?              $? |- $2
4 ?              $? |- ( $2 -> t = t )
5 3,4 mp         $a |- t = t
\end{verbatim}

Unknown steps 3 and 4 are the ones we must focus on.  They correspond to the
minor and major premises of the modus ponens rule.  We will assign them as
follows.  Notice that because of the step renumbering that takes place
after an assignment, it is advantageous to assign unknown steps in reverse
order, because earlier steps will not get renumbered.

\begin{verbatim}
MM-PA> assign 4 mp
To undo the assignment, DELETE STEP 8 and INITIALIZE, UNIFY
if needed.
3   min=?  $? |- $2
6     min=?  $? |- $4
7     maj=?  $? |- ( $4 -> ( $2 -> t = t ) )
\end{verbatim}

We are now going to describe an obscure feature that you will probably
never use but should be aware of.  The Metamath language allows empty
symbol sequences to be substituted for variables, but in most formal
systems this feature is never used.  One of the few examples where is it
used is the MIU-system\index{MIU-system} described in
Appendix~\ref{MIU}.  But such systems are rare, and by default this
feature is turned off in the Proof Assistant.  (It is always allowed for
{\tt verify proof}.)  Let us turn it on and see what
happens.\index{\texttt{set empty{\char`\_}substitution} command}

\begin{verbatim}
MM-PA> set empty_substitution on
Substitutions with empty symbol sequences is now allowed.
\end{verbatim}

With this feature enabled, more unifications will be
ambiguous\index{ambiguous unification}\index{unification!ambiguous} in
the middle of a proof, because
substitution\index{substitution!variable}\index{variable substitution}
of variables with empty symbol sequences will become an additional
possibility.  Let's see what happens when we make our next assignment.

\begin{verbatim}
MM-PA> assign 3 a2
There are 2 possible unifications.  Please select the correct
    one or Q if you want to UNIFY later.
Unify:  |- $6
 with:  |- ( $9 + 0 ) = $9
Unification #1 of 2 (weight = 7):
  Replace "$6" with "( + 0 ) ="
  Replace "$9" with ""
  Accept (A), reject (R), or quit (Q) <A>? r
\end{verbatim}

The first choice presented is the wrong one.  If we had selected it,
temporary variable \texttt{\$6} would have been assigned a truncated
wff, and temporary variable \texttt{\$9} would have been assigned an
empty sequence (which is not allowed in our system).  With this choice,
eventually we would reach a point where we would get stuck because
we would end up with steps impossible to prove.  (You may want to
try it.)  We typed \texttt{r} to reject the choice.

\begin{verbatim}
Unification #2 of 2 (weight = 21):
  Replace "$6" with "( $9 + 0 ) = $9"
  Accept (A), reject (R), or quit (Q) <A>? q
To undo the assignment, DELETE STEP 4 and INITIALIZE, UNIFY
if needed.
 7     min=?  $? |- $8
 8     maj=?  $? |- ( $8 -> ( $6 -> t = t ) )
\end{verbatim}

The second choice is correct, and normally we would type \texttt{a}
to accept it.  But instead we typed \texttt{q} to show what will happen:
it will leave the step with an unknown unification, which can be
seen as follows:

\begin{verbatim}
MM-PA> show new_proof/not_unified
 4   min    $a |- $6
        =a2  = |- ( $9 + 0 ) = $9
\end{verbatim}

Later we can unify this with the \texttt{unify}
\texttt{all/interactive} command.

The important point to remember is that occasionally you will be
presented with several unification choices while entering a proof, when
the program determines that there is not enough information yet to make
an unambiguous choice automatically (and this can happen even with
\texttt{set empty{\char`\_}substitution} turned off).  Usually it is
obvious by inspection which choice is correct, since incorrect ones will
tend to be meaningless fragments of wffs.  In addition, the correct
choice will usually be the first one presented, unlike our example
above.

Enough of this digression.  Let us go back to the default setting.

\begin{verbatim}
MM-PA> set empty_substitution off
The ability to substitute empty expressions for variables
has been turned off.  Note that this may make the Proof
Assistant too restrictive in some cases.
\end{verbatim}

If we delete the proof, start over, and get to the point where
we digressed above, there will no longer be an ambiguous unification.

\begin{verbatim}
MM-PA> assign 3 a2
To undo the assignment, DELETE STEP 4 and INITIALIZE, UNIFY
if needed.
 7     min=?  $? |- $4
 8     maj=?  $? |- ( $4 -> ( ( $5 + 0 ) = $5 -> t = t ) )
\end{verbatim}

Let us look at our proof so far, and continue.

\begin{verbatim}
MM-PA> show new_proof/lemmon
 4 a2            $a |- ( $5 + 0 ) = $5
 7 ?             $? |- $4
 8 ?             $? |- ( $4 -> ( ( $5 + 0 ) = $5 -> t = t ) )
 9 7,8 mp        $a |- ( ( $5 + 0 ) = $5 -> t = t )
10 4,9 mp        $a |- t = t
MM-PA> assign 8 a1
To undo the assignment, DELETE STEP 11 and INITIALIZE, UNIFY
if needed.
 7     min=?  $? |- ( t + 0 ) = t
MM-PA> assign 7 a2
To undo the assignment, DELETE STEP 8 and INITIALIZE, UNIFY
if needed.
MM-PA> show new_proof/lemmon
 4 a2            $a |- ( t + 0 ) = t
 8 a2            $a |- ( t + 0 ) = t
12 a1            $a |- ( ( t + 0 ) = t -> ( ( t + 0 ) = t ->
                                                    t = t ) )
13 8,12 mp       $a |- ( ( t + 0 ) = t -> t = t )
14 4,13 mp       $a |- t = t
\end{verbatim}

Now all temporary variables and unknown steps have been eliminated from the
``essential'' part of the proof.  When this is achieved, the Proof
Assistant\index{Proof Assistant} can usually figure out the rest of the proof
automatically.  (Note that the \texttt{improve} command can occasionally be
useful for filling in essential steps as well, but it only tries to make use
of statements that introduce no new variables in their hypotheses, which is
not the case for \texttt{mp}. Also it will not try to improve steps containing
temporary variables.)  Let's look at the complete proof, then run
the \texttt{improve} command, then look at it again.

\begin{verbatim}
MM-PA> show new_proof/lemmon/all
 1 ?             $? wff ( t + 0 ) = t
 2 ?             $? wff t = t
 3 ?             $? term t
 4 3 a2          $a |- ( t + 0 ) = t
 5 ?             $? wff ( t + 0 ) = t
 6 ?             $? wff ( ( t + 0 ) = t -> t = t )
 7 ?             $? term t
 8 7 a2          $a |- ( t + 0 ) = t
 9 ?             $? term ( t + 0 )
10 ?             $? term t
11 ?             $? term t
12 9,10,11 a1    $a |- ( ( t + 0 ) = t -> ( ( t + 0 ) = t ->
                                                    t = t ) )
13 5,6,8,12 mp   $a |- ( ( t + 0 ) = t -> t = t )
14 1,2,4,13 mp   $a |- t = t
\end{verbatim}

\begin{verbatim}
MM-PA> improve all
A proof of length 1 was found for step 11.
A proof of length 1 was found for step 10.
A proof of length 3 was found for step 9.
A proof of length 1 was found for step 7.
A proof of length 9 was found for step 6.
A proof of length 5 was found for step 5.
A proof of length 1 was found for step 3.
A proof of length 3 was found for step 2.
A proof of length 5 was found for step 1.
Steps 1 and above have been renumbered.
CONGRATULATIONS!  The proof is complete.  Use SAVE
NEW_PROOF to save it.  Note:  The Proof Assistant does
not detect $d violations.  After saving the proof, you
should verify it with VERIFY PROOF.
\end{verbatim}

The \texttt{save new{\char`\_}proof} command\index{\texttt{save
new{\char`\_}proof} command} will save the proof in the database.  Here
we will just display it in a form that can be clipped out of a log file
and inserted manually into the database source file with a text
editor.\index{normal proof}\index{proof!normal}

\begin{verbatim}
MM-PA> show new_proof/normal
---------Clip out the proof below this line:
      tt tze tpl tt weq tt tt weq tt a2 tt tze tpl tt weq
      tt tze tpl tt weq tt tt weq wim tt a2 tt tze tpl tt
      tt a1 mp mp $.
---------The proof of 'th1' to clip out ends above this line.
\end{verbatim}

There is another proof format called ``compressed''\index{compressed
proof}\index{proof!compressed} that you will see in databases.  It is
not important to understand how it is encoded but only to recognize it
when you see it.  Its only purpose is to reduce storage requirements for
large proofs.  A compressed proof can always be converted to a normal
one and vice-versa, and the Metamath \texttt{show proof}
commands\index{\texttt{show proof} command} work equally well with
compressed proofs.  The compressed proof format is described in
Appendix~\ref{compressed}.

\begin{verbatim}
MM-PA> show new_proof/compressed
---------Clip out the proof below this line:
      ( tze tpl weq a2 wim a1 mp ) ABCZADZAADZAEZJJKFLIA
      AGHH $.
---------The proof of 'th1' to clip out ends above this line.
\end{verbatim}

Now we will exit the Proof Assistant.  Since we made changes to the proof,
it will warn us that we have not saved it.  In this case, we don't care.

\begin{verbatim}
MM-PA> exit
Warning:  You have not saved changes to the proof.
Do you want to EXIT anyway (Y, N) <N>? y
Exiting the Proof Assistant.
Type EXIT again to exit Metamath.
\end{verbatim}

The Proof Assistant\index{Proof Assistant} has several other commands
that can help you while creating proofs.  See Section~\ref{pfcommands}
for a list of them.

A command that is often useful is \texttt{minimize{\char`\_}with
*/brief}, which tries to shorten the proof.  It can make the process
more efficient by letting you write a somewhat ``sloppy'' proof then
clean up some of the fine details of optimization for you (although it
can't perform miracles such as restructuring the overall proof).

\section{A Note About Editing a Data\-base File}

Once your source file contains proofs, there are some restrictions on
how you can edit it so that the proofs remain valid.  Pay particular
attention to these rules, since otherwise you can lose a lot of work.
It is a good idea to periodically verify all proofs with \texttt{verify
proof *} to ensure their integrity.

If your file contains only normal (as opposed to compressed) proofs, the
main rule is that you may not change the order of the mandatory
hypotheses\index{mandatory hypothesis} of any statement referenced in a
later proof.  For example, if you swap the order of the major and minor
premise in the modus ponens rule, all proofs making use of that rule
will become incorrect.  The \texttt{show statement}
command\index{\texttt{show statement} command} will show you the
mandatory hypotheses of a statement and their order.

If a statement has a compressed proof, you also must not change the
order of {\em its} mandatory hypotheses.  The compressed proof format
makes use of this information as part of the compression technique.
Note that swapping the names of two variables in a theorem will change
the order of its mandatory hypotheses.

The safest way to edit a statement, say \texttt{mytheorem}, is to
duplicate it then rename the original to \texttt{mytheoremOLD}
throughout the database.  Once the edited version is re-proved, all
statements referencing \texttt{mytheoremOLD} can be updated in the Proof
Assistant using \texttt{minimize{\char`\_}with
mytheorem
/allow{\char`\_}growth}.\index{\texttt{minimize{\char`\_}with} command}
% 3/10/07 Note: line-breaking the above results in duplicate index entries

\chapter{Abstract Mathematics Revealed}\label{fol}

\section{Logic and Set Theory}\label{logicandsettheory}

\begin{quote}
  {\em Set theory can be viewed as a form of exact theology.}
  \flushright\sc  Rudy Rucker\footnote{\cite{Barrow}, p.~31.}\\
\end{quote}\index{Rucker, Rudy}

Despite its seeming complexity, all of standard mathematics, no matter how
deep or abstract, can amazingly enough be derived from a relatively small set
of axioms\index{axiom} or first principles. The development of these axioms is
among the most impressive and important accomplishments of mathematics in the
20th century. Ultimately, these axioms can be broken down into a set of rules
for manipulating symbols that any technically oriented person can follow.

We will not spend much time trying to convey a deep, higher-level
understanding of the meaning of the axioms. This kind of understanding
requires some mathematical sophistication as well as an understanding of the
philosophy underlying the foundations of mathematics and typically develops
over time as you work with mathematics.  Our goal, instead, is to give you the
immediate ability to follow how theorems\index{theorem} are derived from the
axioms and from other theorems.  This will be similar to learning the syntax
of a computer language, which lets you follow the details in a program but
does not necessarily give you the ability to write non-trivial programs on
your own, an ability that comes with practice. For now don't be alarmed by
abstract-sounding names of the axioms; just focus on the rules for
manipulating the symbols, which follow the simple conventions of the
Metamath\index{Metamath} language.

The axioms that underlie all of standard mathematics consist of axioms of logic
and axioms of set theory. The axioms of logic are divided into two
subcategories, propositional calculus\index{propositional calculus} (sometimes
called sentential logic\index{sentential logic}) and predicate calculus
(sometimes called first-order logic\index{first-order logic}\index{quantifier
theory}\index{predicate calculus} or quantifier theory).  Propositional
calculus is a prerequisite for predicate calculus, and predicate calculus is a
prerequisite for set theory.  The version of set theory most commonly used is
Zermelo--Fraenkel set theory\index{Zermelo--Fraenkel set theory}\index{set theory}
with the axiom of choice,
often abbreviated as ZFC\index{ZFC}.

Here in a nutshell is what the axioms are all about in an informal way. The
connection between this description and symbols we will show you won't be
immediately apparent and in principle needn't ever be.  Our description just
tries to summarize what mathematicians think about when they work with the
axioms.

Logic is a set of rules that allow us determine truths given other truths.
Put another way,
logic is more or less the translation of what we would consider common sense
into a rigorous set of axioms.\index{axioms of logic}  Suppose $\varphi$,
$\psi$, and $\chi$ (the Greek letters phi, psi, and chi) represent statements
that are either true or false, and $x$ is a variable\index{variable!in predicate
calculus} ranging over some group of mathematical objects (sets, integers,
real numbers, etc.). In mathematics, a ``statement'' really means a formula,
and $\psi$ could be for example ``$x = 2$.''
Propositional calculus\index{propositional calculus}
allows us to use variables that are either true or false
and make deductions such as
``if $\varphi$ implies $\psi$ and $\psi$ implies $\chi$, then $\varphi$
implies $\chi$.''
Predicate calculus\index{predicate calculus}
extends propositional calculus by also allowing us
to discuss statements about objects (not just true and false values), including
statements about ``all'' or ``at least one'' object.
For example, predicate calculus allows to say,
``if $\varphi$ is true for all $x$, then $\varphi$ is true for some $x$.''
The logic used in \texttt{set.mm} is standard classical logic
(as opposed to other logic systems like intuitionistic logic).

Set theory\index{set theory} has to do with the manipulation of objects and
collections of objects, specifically the abstract, imaginary objects that
mathematics deals with, such as numbers. Everything that is claimed to exist
in mathematics is considered to be a set.  A set called the empty
set\index{empty set} contains nothing.  We represent the empty set by
$\varnothing$.  Many sets can be built up from the empty set.  There is a set
represented by $\{\varnothing\}$ that contains the empty set, another set
represented by $\{\varnothing,\{\varnothing\}\}$ that contains this set as
well as the empty set, another set represented by $\{\{\varnothing\}\}$ that
contains just the set that contains the empty set, and so on ad infinitum. All
mathematical objects, no matter how complex, are defined as being identical to
certain sets: the integer\index{integer} 0 is defined as the empty set, the
integer 1 is defined as $\{\varnothing\}$, the integer 2 is defined as
$\{\varnothing,\{\varnothing\}\}$.  (How these definitions were chosen doesn't
matter now, but the idea behind it is that these sets have the properties we
expect of integers once suitable operations are defined.)  Mathematical
operations, such as addition, are defined in terms of operations on
sets---their union\index{set union}, intersection\index{set intersection}, and
so on---operations you may have used in elementary school when you worked
with groups of apples and oranges.

With a leap of faith, the axioms also postulate the existence of infinite
sets\index{infinite set}, such as the set of all non-negative integers ($0, 1,
2,\ldots$, also called ``natural numbers''\index{natural number}).  This set
can't be represented with the brace notation\index{brace notation} we just
showed you, but requires a more complicated notation called ``class
abstraction.''\index{class abstraction}\index{abstraction class}  For
example, the infinite set $\{ x |
\mbox{``$x$ is a natural number''} \} $ means the ``set of all objects $x$
such that $x$ is a natural number'' i.e.\ the set of natural numbers; here,
``$x$ is a natural number'' is a rather complicated formula when broken down
into the primitive symbols.\label{expandom}\footnote{The statement ``$x$ is a
natural number'' is formally expressed as ``$x \in \omega$,'' where $\in$
(stylized epsilon) means ``is in'' or ``is an element of'' and $\omega$
(omega) means ``the set of natural numbers.''  When ``$x\in\omega$'' is
completely expanded in terms of the primitive symbols of set theory, the
result is  $\lnot$ $($ $\lnot$ $($ $\forall$ $z$ $($ $\lnot$ $\forall$ $w$ $($
$z$ $\in$ $w$ $\rightarrow$ $\lnot$ $w$ $\in$ $x$ $)$ $\rightarrow$ $z$ $\in$
$x$ $)$ $\rightarrow$ $($ $\forall$ $z$ $($ $\lnot$ $($ $\forall$ $w$ $($ $w$
$\in$ $z$ $\rightarrow$ $w$ $\in$ $x$ $)$ $\rightarrow$ $\forall$ $w$ $\lnot$
$w$ $\in$ $z$ $)$ $\rightarrow$ $\lnot$ $\forall$ $w$ $($ $w$ $\in$ $z$
$\rightarrow$ $\lnot$ $\forall$ $v$ $($ $v$ $\in$ $z$ $\rightarrow$ $\lnot$
$v$ $\in$ $w$ $)$ $)$ $)$ $\rightarrow$ $\lnot$ $\forall$ $z$ $\forall$ $w$
$($ $\lnot$ $($ $z$ $\in$ $x$ $\rightarrow$ $\lnot$ $w$ $\in$ $x$ $)$
$\rightarrow$ $($ $\lnot$ $z$ $\in$ $w$ $\rightarrow$ $($ $\lnot$ $z$ $=$ $w$
$\rightarrow$ $w$ $\in$ $z$ $)$ $)$ $)$ $)$ $)$ $\rightarrow$ $\lnot$
$\forall$ $y$ $($ $\lnot$ $($ $\lnot$ $($ $\forall$ $z$ $($ $\lnot$ $\forall$
$w$ $($ $z$ $\in$ $w$ $\rightarrow$ $\lnot$ $w$ $\in$ $y$ $)$ $\rightarrow$
$z$ $\in$ $y$ $)$ $\rightarrow$ $($ $\forall$ $z$ $($ $\lnot$ $($ $\forall$
$w$ $($ $w$ $\in$ $z$ $\rightarrow$ $w$ $\in$ $y$ $)$ $\rightarrow$ $\forall$
$w$ $\lnot$ $w$ $\in$ $z$ $)$ $\rightarrow$ $\lnot$ $\forall$ $w$ $($ $w$
$\in$ $z$ $\rightarrow$ $\lnot$ $\forall$ $v$ $($ $v$ $\in$ $z$ $\rightarrow$
$\lnot$ $v$ $\in$ $w$ $)$ $)$ $)$ $\rightarrow$ $\lnot$ $\forall$ $z$
$\forall$ $w$ $($ $\lnot$ $($ $z$ $\in$ $y$ $\rightarrow$ $\lnot$ $w$ $\in$
$y$ $)$ $\rightarrow$ $($ $\lnot$ $z$ $\in$ $w$ $\rightarrow$ $($ $\lnot$ $z$
$=$ $w$ $\rightarrow$ $w$ $\in$ $z$ $)$ $)$ $)$ $)$ $\rightarrow$ $($
$\forall$ $z$ $\lnot$ $z$ $\in$ $y$ $\rightarrow$ $\lnot$ $\forall$ $w$ $($
$\lnot$ $($ $w$ $\in$ $y$ $\rightarrow$ $\lnot$ $\forall$ $z$ $($ $w$ $\in$
$z$ $\rightarrow$ $\lnot$ $z$ $\in$ $y$ $)$ $)$ $\rightarrow$ $\lnot$ $($
$\lnot$ $\forall$ $z$ $($ $w$ $\in$ $z$ $\rightarrow$ $\lnot$ $z$ $\in$ $y$
$)$ $\rightarrow$ $w$ $\in$ $y$ $)$ $)$ $)$ $)$ $\rightarrow$ $x$ $\in$ $y$
$)$ $)$ $)$. Section~\ref{hierarchy} shows the hierarchy of definitions that
leads up to this expression.}\index{stylized epsilon ($\in$)}\index{omega
($\omega$)}  Actually, the primitive symbols don't even include the brace
notation.  The brace notation is a high-level definition, which you can find in
Section~\ref{hierarchy}.

Interestingly, the arithmetic of integers\index{integer} and
rationals\index{rational number} can be developed without appealing to the
existence of an infinite set, whereas the arithmetic of real
numbers\index{real number} requires it.

Each variable\index{variable!in set theory} in the axioms of set theory
represents an arbitrary set, and the axioms specify the legal kinds of things
you can do with these variables at a very primitive level.

Now, you may think that numbers and arithmetic are a lot more intuitive and
fundamental than sets and therefore should be the foundation of mathematics.
What is really the case is that you've dealt with numbers all your life and
are comfortable with a few rules for manipulating them such as addition and
multiplication.  Those rules only cover a small portion of what can be done
with numbers and only a very tiny fraction of the rest of mathematics.  If you
look at any elementary book on number theory, you will quickly become lost if
these are the only rules that you know.  Even though such books may present a
list of ``axioms''\index{axiom} for arithmetic, the ability to use the axioms
and to understand proofs of theorems\index{theorem} (facts) about numbers
requires an implicit mathematical talent that frustrates many people
from studying abstract mathematics.  The kind of mathematics that most people
know limits them to the practical, everyday usage of blindly manipulating
numbers and formulas, without any understanding of why those rules are correct
nor any ability to go any further.  For example, do you know why multiplying
two negative numbers yields a positive number?  Starting with set theory, you
will also start off blindly manipulating symbols according to the rules we give
you, but with the advantage that these rules will allow you, in principle, to
access {\em all} of mathematics, not just a tiny part of it.

Of course, concrete examples are often helpful in the learning process. For
example, you can verify that $2\cdot 3=3 \cdot 2$ by actually grouping
objects and can easily ``see'' how it generalizes to $x\cdot y = y\cdot x$,
even though you might not be able to rigorously prove it.  Similarly, in set
theory it can be helpful to understand how the axioms of set theory apply to
(and are correct for) small finite collections of objects.  You should be aware
that in set theory intuition can be misleading for infinite collections, and
rigorous proofs become more important.  For example, while $x\cdot y = y\cdot
x$ is correct for finite ordinals (which are the natural numbers), it is not
usually true for infinite ordinals.

\section{The Axioms for All of Mathematics}

In this section\index{axioms for mathematics}, we will show you the axioms
for all of standard mathematics (i.e.\ logic and set theory) as they are
traditionally presented.  The traditional presentation is useful for someone
with the mathematical experience needed to correctly manipulate high-level
abstract concepts.  For someone without this talent, knowing how to actually
make use of these axioms can be difficult.  The purpose of this section is to
allow you to see how the version of the axioms used in the standard
Metamath\index{Metamath} database \texttt{set.mm}\index{set
theory database (\texttt{set.mm})} relates to  the typical version
in textbooks, and also to give you an informal feel for them.

\subsection{Propositional Calculus}

Propositional calculus\index{propositional calculus} concerns itself with
statements that can be interpreted as either true or false.  Some examples of
statements (outside of mathematics) that are either true or false are ``It is
raining today'' and ``The United States has a female president.'' In
mathematics, as we mentioned, statements are really formulas.

In propositional calculus, we don't care what the statements are.  We also
treat a logical combination of statements, such as ``It is raining today and
the United States has a female president,'' no differently from a single
statement.  Statements and their combinations are called well-formed formulas
(wffs)\index{well-formed formula (wff)}.  We define wffs only in terms of
other wffs and don't define what a ``starting'' wff is.  As is common practice
in the literature, we use Greek letters to represent wffs.

Specifically, suppose $\varphi$ and $\psi$ are wffs.  Then the combinations
$\varphi\rightarrow\psi$ (``$\varphi$ implies $\psi$,'' also read ``if
$\varphi$ then $\psi$'')\index{implication ($\rightarrow$)} and $\lnot\varphi$
(``not $\varphi$'')\index{negation ($\lnot$)} are also wffs.

The three axioms of propositional calculus\index{axioms of propositional
calculus} are all wffs of the following form:\footnote{A remarkable result of
C.~A.~Meredith\index{Meredith, C. A.} squeezes these three axioms into the
single axiom $((((\varphi\rightarrow \psi)\rightarrow(\neg \chi\rightarrow\neg
\theta))\rightarrow \chi )\rightarrow \tau)\rightarrow((\tau\rightarrow
\varphi)\rightarrow(\theta\rightarrow \varphi))$ \cite{CAMeredith},
which is believed to be the shortest possible.}
\begin{center}
     $\varphi\rightarrow(\psi\rightarrow \varphi)$\\

     $(\varphi\rightarrow (\psi\rightarrow \chi))\rightarrow
((\varphi\rightarrow  \psi)\rightarrow (\varphi\rightarrow \chi))$\\

     $(\neg \varphi\rightarrow \neg\psi)\rightarrow (\psi\rightarrow
\varphi)$
\end{center}

These three axioms are widely used.
They are attributed to Jan {\L}ukasiewicz
(pronounced woo-kah-SHAY-vitch) and was popularized by Alonzo Church,
who called it system P2. (Thanks to Ted Ulrich for this information.)

There are an infinite number of axioms, one for each possible
wff\index{well-formed formula (wff)} of the above form.  (For this reason,
axioms such as the above are often called ``axiom schemes.''\index{axiom
scheme})  Each Greek letter in the axioms may be substituted with a more
complex wff to result in another axiom.  For example, substituting
$\neg(\varphi\rightarrow\chi)$ for $\varphi$ in the first axiom yields
$\neg(\varphi\rightarrow\chi)\rightarrow(\psi\rightarrow
\neg(\varphi\rightarrow\chi))$, which is still an axiom.

To deduce new true statements (theorems\index{theorem}) from the axioms, a
rule\index{rule} called ``modus ponens''\index{modus ponens} is used.  This
rule states that if the wff $\varphi$ is an axiom or a theorem, and the wff
$\varphi\rightarrow\psi$ is an axiom or a theorem, then the wff $\psi$ is also
a theorem\index{theorem}.

As a non-mathematical example of modus ponens, suppose we have proved (or
taken as an axiom) ``Bob is a man'' and separately have proved (or taken as
an axiom) ``If Bob is a man, then Bob is a human.''  Using the rule of modus
ponens, we can logically deduce, ``Bob is a human.''

From Metamath's\index{Metamath} point of view, the axioms and the rule of
modus ponens just define a mechanical means for deducing new true statements
from existing true statements, and that is the complete content of
propositional calculus as far as Metamath is concerned.  You can read a logic
textbook to gain a better understanding of their meaning, or you can just let
their meaning slowly become apparent to you after you use them for a while.

It is actually rather easy to check to see if a formula is a theorem of
propositional calculus.  Theorems of propositional calculus are also called
``tautologies.''\index{tautology}  The technique to check whether a formula is
a tautology is called the ``truth table method,''\index{truth table} and it
works like this.  A wff $\varphi\rightarrow\psi$ is false whenever $\varphi$ is true
and $\psi$ is false.  Otherwise it is true.  A wff $\lnot\varphi$ is false
whenever $\varphi$ is true and false otherwise. To verify a tautology such as
$\varphi\rightarrow(\psi\rightarrow \varphi)$, you break it down into sub-wffs and
construct a truth table that accounts for all possible combinations of true
and false assigned to the wff metavariables:
\begin{center}\begin{tabular}{|c|c|c|c|}\hline
\mbox{$\varphi$} & \mbox{$\psi$} & \mbox{$\psi\rightarrow\varphi$}
    & \mbox{$\varphi\rightarrow(\psi\rightarrow \varphi)$} \\ \hline \hline
              T   &  T    &      T       &        T    \\ \hline
              T   &  F    &      T       &        T    \\ \hline
              F   &  T    &      F       &        T    \\ \hline
              F   &  F    &      T       &        T    \\ \hline
\end{tabular}\end{center}
If all entries in the last column are true, the formula is a tautology.

Now, the truth table method doesn't tell you how to prove the tautology from
the axioms, but only that a proof exists.  Finding an actual proof (especially
one that is short and elegant) can be challenging.  Methods do exist for
automatically generating proofs in propositional calculus, but the proofs that
result can sometimes be very long.  In the Metamath \texttt{set.mm}\index{set
theory database (\texttt{set.mm})} database, most
or all proofs were created manually.

Section \ref{metadefprop} discusses various definitions
that make propositional calculus easier to use.
For example, we define:

\begin{itemize}
\item $\varphi \vee \psi$
  is true if either $\varphi$ or $\psi$ (or both) are true
  (this is disjunction\index{disjunction ($\vee$)}
  aka logical {\sc or}\index{logical {\sc or} ($\vee$)}).

\item $\varphi \wedge \psi$
  is true if both $\varphi$ and $\psi$ are true
  (this is conjunction\index{conjunction ($\wedge$)}
  aka logical {\sc and}\index{logical {\sc and} ($\wedge$)}).

\item $\varphi \leftrightarrow \psi$
  is true if $\varphi$ and $\psi$ have the same value, that is,
  they are both true or both false
  (this is the biconditional\index{biconditional ($\leftrightarrow$)}).
\end{itemize}

\subsection{Predicate Calculus}

Predicate calculus\index{predicate calculus} introduces the concept of
``individual variables,''\index{variable!in predicate calculus}\index{individual
variable} which
we will usually just call ``variables.''
These variables can represent something other than true or false (wffs),
and will always represent sets when we get to set theory.  There are also
three new symbols $\forall$\index{universal quantifier ($\forall$)},
$=$\index{equality ($=$)}, and $\in$\index{stylized epsilon ($\in$)},
read ``for all,'' ``equals,'' and ``is an element of''
respectively.  We will represent variables with the letters $x$, $y$, $z$, and
$w$, as is common practice in the literature.
For example, $\forall x \varphi$ means ``for all possible values of
$x$, $\varphi$ is true.''

In predicate calculus, we extend the definition of a wff\index{well-formed
formula (wff)}.  If $\varphi$ is a wff and $x$ and $y$ are variables, then
$\forall x \, \varphi$, $x=y$, and $x\in y$ are wffs. Note that these three new
types of wffs can be considered ``starting'' wffs from which we can build
other wffs with $\rightarrow$ and $\neg$ .  The concept of a starting wff was
absent in propositional calculus.  But starting wff or not, all we are really
concerned with is whether our wffs are correctly constructed according to
these mechanical rules.

A quick aside:
To prevent confusion, it might be best at this point to think of the variables
of Metamath\index{Metamath} as ``metavariables,''\index{metavariable} because
they are not quite the same as the variables we are introducing here.  A
(meta)variable in Metamath can be a wff or an individual variable, as well
as many other things; in general, it represents a kind of place holder for an
unspecified sequence of math symbols\index{math symbol}.

Unlike propositional calculus, no decision procedure\index{decision procedure}
analogous to the truth table method exists (nor theoretically can exist) that
will definitely determine whether a formula is a theorem of predicate
calculus.  Much of the work in the field of automated theorem
proving\index{automated theorem proving} has been dedicated to coming up with
clever heuristics for proving theorems of predicate calculus, but they can
never be guaranteed to work always.

Section \ref{metadefpred} discusses various definitions
that make predicate calculus easier to use.
For example, we define
$\exists x \varphi$ to mean
``there exists at least one possible value of $x$ where $\varphi$ is true.''

We now turn to looking at how predicate calculus can be formally
represented.

\subsubsection{Common Axioms}

There is a new rule of inference in predicate calculus:  if $\varphi$ is
an axiom or a theorem, then $\forall x \,\varphi$ is also a
theorem\index{theorem}.  This is called the rule of
``generalization.''\index{rule of generalization}
This is easily represented in Metamath.

In standard texts of logic, there are often two axioms of predicate
calculus\index{axioms of predicate calculus}:
\begin{center}
  $\forall x \,\varphi ( x ) \rightarrow \varphi ( y )$,
      where ``$y$ is properly substituted for $x$.''\\
  $\forall x ( \varphi \rightarrow \psi )\rightarrow ( \varphi \rightarrow
    \forall x\, \psi )$,
    where ``$x$ is not free in $\varphi$.''
\end{center}

Now at first glance, this seems simple:  just two axioms.  However,
conditional clauses are attached to each axiom describing requirements that
may seem puzzling to you.  In addition, the first axiom puts a variable symbol
in parentheses after each wff, seemingly violating our definition of a
wff\index{well-formed formula (wff)}; this is just an informal way of
referring to some arbitrary variable that may occur in the wff.  The
conditional clauses do, of course, have a precise meaning, but as it turns out
the precise meaning is somewhat complicated and awkward to formalize in a
way that a computer can handle easily.  Unlike propositional calculus, a
certain amount of mathematical sophistication and practice is needed to be
able to easily grasp and manipulate these concepts correctly.

Predicate calculus may be presented with or without axioms for
equality\index{axioms of equality}\index{equality ($=$)}. We will require the
axioms of equality as a prerequisite for the version of set theory we will
use.  The axioms for equality, when included, are often represented using these
two axioms:
\begin{center}
$x=x$\\ \ \\
$x=y\rightarrow (\varphi(x,x)\rightarrow\varphi(x,y))$ where ``$\varphi(x,y)$
   arises from $\varphi(x,x)$ by replacing some, but not necessarily all,
   free\index{free variable}
   occurrences of $x$ by $y$,\\ provided that $y$ is free for $x$
   in $\varphi(x,x)$.'' \end{center}
% (Mendelson p. 95)
The first equality axiom is simple, but again,
the condition on the second one is
somewhat awkward to implement on a computer.

\subsubsection{Tarski System S2}

Of course, we are not the first to notice the complications of these
predicate calculus axioms when being rigorous.

Well-known logician Alfred Tarski published in 1965
a system he called system S2\cite[p.~77]{Tarski1965}.
Tarski's system is \textit{exactly equivalent} to the traditional textbook
formalization, but (by clever use of equality axioms) it eliminates the
latter's primitive notions of ``proper substitution'' and ``free variable,''
replacing them with direct substitution and the notion of a variable
not occurring in a formula (which we express with distinct variable
constraints).

In advocating his system, Tarski wrote, ``The relatively complicated
character of [free variables and proper substitution] is a source
of certain inconveniences of both practical and theoretical nature;
this is clearly experienced both in teaching an elementary course of
mathematical logic and in formalizing the syntax of predicate logic for
some theoretical purposes''\cite[p.~61]{Tarski1965}\index{Tarski, Alfred}.

\subsubsection{Developing a Metamath Representation}

The standard textbook axioms of predicate calculus are somewhat
cumbersome to implement on a computer because of the complex notions of
``free variable''\index{free variable} and ``proper
substitution.''\index{proper substitution}\index{substitution!proper}
While it is possible to use the Metamath\index{Metamath} language to
implement these concepts, we have chosen not to implement them
as primitive constructs in the
\texttt{set.mm} set theory database.  Instead, we have eliminated them
within the axioms
by carefully crafting the axioms so as to avoid them,
building on Tarski's system S2.  This makes it
easy for a beginner to follow the steps in a proof without knowing any
advanced concepts other than the simple concept of
replacing\index{substitution!variable}\index{variable substitution}
variables with expressions.

In order to develop the concepts of free variable and proper
substitution from the axioms, we use an additional
Metamath statement type called ``disjoint variable
restriction''\index{disjoint variables} that we have not encountered
before.  In the context of the axioms, the statement \texttt{\$d} $ x\,
y$\index{\texttt{\$d} statement} simply means that $x$ and $y$ must be
distinct\index{distinct variables}, i.e.\ they may not be simultaneously
substituted\index{substitution!variable}\index{variable substitution}
with the same variable.  The statement \texttt{\$d} $ x\, \varphi$ means
variable $x$ must not occur in wff $\varphi$.  For the precise
definition of \texttt{\$d}, see Section~\ref{dollard}.

\subsubsection{Metamath representation}

The Metamath axiom system for predicate calculus
defined in set.mm uses Tarski's system S2.
As noted above, this has a different representation
than the traditional textbook formalization,
but it is \textit{exactly equivalent} to the textbook formalization,
and it is \textit{much} easier to work with.
This is reproduced as system S3 in Section 6 of
Megill's formalization \cite{Megill}\index{Megill, Norman}.

There is one exception, Tarski's axiom of existence,
which we label as axiom ax-6.
In the case of ax-6, Tarski's version is weaker because it includes a
distinct variable proviso. If we wish, we can also weaken our version
in this way and still have a metalogically complete system. Theorem
ax6 shows this by deriving, in the presence of the other axioms, our
ax-6 from Tarski's weaker version ax6v. However, we chose the stronger
version for our system because it is simpler to state and easier to use.

Tarski's system was designed for proving specific theorems rather than
more general theorem schemes. However, theorem schemes are much more
efficient than specific theorems for building a body of mathematical
knowledge, since they can be reused with different instances as
needed. While Tarski does derive some theorem schemes from his axioms,
their proofs require concepts that are ``outside'' of the system, such as
induction on formula length. The verification of such proofs is difficult
to automate in a proof verifier. (Specifically, Tarski treats the formulas
of his system as set-theoretical objects. In order to verify the proofs
of his theorem schemes, a proof verifier would need a significant amount
of set theory built into it.)

The Metamath axiom system for predicate calculus extends
Tarski's system to eliminate this difficulty. The additional
``auxilliary'' axiom
schemes (as we will call them in this section; see below) endow Tarski's
system with a nice property we call
metalogical completeness \cite[Remark 9.6]{Megill}\index{Megill, Norman}.
As a result, we can prove any theorem scheme
expressable in the ``simple metalogic'' of Tarski's system by using
only Metamath's direct substitution rule applied to the axiom system
(and no other metalogical or set-theoretical notions ``outside'' of the
system). Simple metalogic consists of schemes containing wff metavariables
(with no arguments) and/or set (also called ``individual'') metavariables,
accompanied by optional provisos each stating that two specified set
metavariables must be distinct or that a specified set metavariable may
not occur in a specified wff metavariable. Metamath's logic and set theory
axiom and rule schemes are all examples of simple metalogic. The schemes
of traditional predicate calculus with equality are examples which are
not simple metalogic, because they use wff metavariables with arguments
and have ``free for'' and ``not free in'' side conditions.

A rigorous justification for this system, using an older but
exactly equivalent set of axioms, can be
found in \cite{Megill}\index{Megill, Norman}.

This allows us to
take a different approach in the Metamath\index{Metamath} database
\texttt{set.mm}\index{set theory database (\texttt{set.mm})}.  We do not
directly use the primitive notions of ``free variable''\index{free variable}
and ``proper substitution''\index{proper
substitution}\index{substitution!proper} at all as primitive constructs.
Instead, we use a set
of axioms that are almost as simple to manipulate as those of
propositional calculus.  Our axiom system avoids complex primitive
notions by effectively embedding the complexity into the axioms
themselves.  As a result, we will end up with a larger number of axioms,
but they are ideally suited for a computer language such as Metamath.
(Section~\ref{metaaxioms} shows these axioms.)

We will not elaborate further
on the ``free variable'' and ``proper substitution''
concepts here.  You may consult
\cite[ch.\ 3--4]{Hamilton}\index{Hamilton, Alan G.} (as well as
many other books) for a precise explanation
of these concepts.  If you intend to do serious mathematical work, it is wise
to become familiar with the traditional textbook approach; even though the
concepts embedded in their axioms require a higher level of sophistication,
they can be more practical to deal with on an everyday, informal basis.  Even
if you are just developing Metamath proofs, familiarity with the traditional
approach can help you arrive at a proof outline much faster, which you can
then convert to the detail required by Metamath.

We do develop proper substitution rules later on, but in set.mm
they are defined as derived constructs; they are not primitives.

You should also note that our system of predicate calculus is specifically
tailored for set theory; thus there are only two specific predicates $=$ and
$\in$ and no functions\index{function!in predicate calculus}
or constants\index{constant!in predicate calculus} unlike more general systems.
We later add these.

\subsection{Set Theory}

Traditional Zermelo--Fraenkel set theory\index{Zermelo--Fraenkel set
theory}\index{set theory} with the Axiom of Choice
has 10 axioms, which can be expressed in the
language of predicate calculus.  In this section, we will list only the
names and brief English descriptions of these axioms, since we will give
you the precise formulas used by the Metamath\index{Metamath} set theory
database \texttt{set.mm} later on.

In the descriptions of the axioms, we assume that $x$, $y$, $z$, $w$, and $v$
represent sets.  These are the same as the variables\index{variable!in set
theory} in our predicate calculus system above, except that now we informally
think of the variables as ranging over sets.  Note that the terms
``object,''\index{object} ``set,''\index{set} ``element,''\index{element}
``collection,''\index{collection} and ``family''\index{family} are synonymous,
as are ``is an element of,'' ``is a member of,''\index{member} ``is contained
in,'' and ``belongs to.''  The different terms are used for convenience; for
example, ``a collection of sets'' is less confusing than ``a set of sets.''
A set $x$ is said to be a ``subset''\index{subset} of $y$ if every element of
$x$ is also an element of $y$; we also say $x$ is ``included in''
$y$.

The axioms are very general and apply to almost any conceivable mathematical
object, and this level of abstraction can be overwhelming at first.  To gain an
intuitive feel, it can be helpful to draw a picture illustrating the concept;
for example, a circle containing dots could represent a collection of sets,
and a smaller circle drawn inside the circle could represent a subset.
Overlapping circles can illustrate intersection and union.  Circles that
illustrate the concepts of set theory are frequently used in elementary
textbooks and are called Venn diagrams\index{Venn diagram}.\index{axioms of
set theory}

1. Axiom of Extensionality:  Two sets are identical if they contain the same
   elements.\index{Axiom of Extensionality}

2. Axiom of Pairing:  The set $\{ x , y \}$ exists.\index{Axiom of Pairing}

3. Axiom of Power Sets:  The power set of a set (the collection of all of
   its subsets) exists.  For example, the power set of $\{x,y\}$ is
   $\{\varnothing,\{x\},\{y\},\{x,y\}\}$ and it exists.\index{Axiom
of Power Sets}

4. Axiom of the Null Set:  The empty set $\varnothing$ exists.\index{Axiom of
the Null Set}

5. Axiom of Union:  The union of a set (the set containing the elements of
   its members) exists.  For example, the union of $\{\{x,y\},\{z\}\}$ is
 $\{x,y,z\}$ and
   it exists.\index{Axiom of Union}

6. Axiom of Regularity:  Roughly, no set can contain itself, nor can there
   be membership ``loops,'' such as a set being an
   element of one of its members.\index{Axiom of Regularity}

7. Axiom of Infinity:  An infinite set exists.  An example of an infinite
   set is the set of all
   integers.\index{Axiom of Infinity}

8. Axiom of Separation:  The set exists that is obtained by restricting $x$
   with some property.  For example, if the set of all integers exists,
   then the set of all even integers exists.\index{Axiom of Separation}

9. Axiom of Replacement:  The range of a function whose domain is restricted
   to the elements of a set $x$, is also a set.  For example, there
   is a function
   from integers (the function's domain) to their squares (its
   range).  If we
   restrict the domain to even integers, its range will become the set of
   squares of even integers, so this axiom asserts that the set of
    squares of even numbers exists.  Technical note:  In general, the
   ``function'' need not be a set but can be a proper class.
   \index{Axiom of Replacement}

10. Axiom of Choice:  Let $x$ be a set whose members are pairwise
  disjoint\index{disjoint sets} (i.e,
  whose members contain no elements in common).  Then there exists another
  set containing one element from each member of $x$.  For
  example, if $x$ is
  $\{\{y,z\},\{w,v\}\}$, where $y$, $z$, $w$, and $v$ are
  different sets, then a set such as $\{z,w\}$
  exists (but the axiom doesn't tell
  us which one).  (Actually the Axiom
  of Choice is redundant if the set $x$, as in this example, has a finite
  number of elements.)\index{Axiom of Choice}

The Axiom of Choice is usually considered an extension of ZF set theory rather
than a proper part of it.  It is sometimes considered philosophically
controversial because it specifies the existence of a set without specifying
what the set is. Constructive logics, including intuitionistic logic,
do not accept the axiom of choice.
Since there is some lingering controversy, we often prefer proofs that do
not use the axiom of choice (where there is a known alternative), and
in some cases we will use weaker axioms than the full axiom of choice.
That said, the axiom of choice is a powerful and widely-accepted tool,
so we do use it when needed.
ZF set theory that includes the Axiom of Choice is
called Zermelo--Fraenkel set theory with choice (ZFC\index{ZFC set theory}).

When expressed symbolically, the Axiom of Separation and the Axiom of
Replacement contain wff symbols and therefore each represent infinitely many
axioms, one for each possible wff. For this reason, they are often called
axiom schemes\index{axiom scheme}\index{well-formed formula (wff)}.

It turns out that the Axiom of the Null Set, the Axiom of Pairing, and the
Axiom of Separation can be derived from the other axioms and are therefore
unnecessary, although they tend to be included in standard texts for various
reasons (historical, philosophical, and possibly because some authors may not
know this).  In the Metamath\index{Metamath} set theory database, these
redundant axioms are derived from the other ones instead of truly
being considered axioms.
This is in keeping with our general goal of minimizing the number of
axioms we must depend on.

\subsection{Other Axioms}

Above we qualified the phrase ``all of mathematics'' with ``essentially.''
The main important missing piece is the ability to do category theory,
which requires huge sets (inaccessible cardinals) larger than those
postulated by the ZFC axioms. The Tarski--Grothendieck Axiom postulates
the existence of such sets.
Note that this is the same axiom used by Mizar for supporting
category theory.
The Tarski--Grothendieck axiom
can be viewed as a very strong replacement of the Axiom of Infinity,
the Axiom of Choice, and the Axiom of Power Sets.
The \texttt{set.mm} database includes this axiom; see the database
for details about it.
Again, we only use this axiom when we need to.
You are only likely to encounter or use this axiom if you are doing
category theory, since its use is highly specialized,
so we will not list the Tarsky-Grothendieck axiom
in the short list of axioms below.

Can there be even more axioms?
Of course.
G\"{o}del showed that no finite set of axioms or axiom schemes can completely
describe any consistent theory strong enough to include arithmetic.
But practically speaking, the ones above are the accepted foundation that
almost all mathematicians explicitly or implicitly base their work on.

\section{The Axioms in the Metamath Language}\label{metaaxioms}

Here we list the axioms as they appear in
\texttt{set.mm}\index{set theory database (\texttt{set.mm})} so you can
look them up there easily.  Incidentally, the \texttt{show statement
/tex} command\index{\texttt{show statement} command} was used to
typeset them.

%macros from show statement /tex
\newbox\mlinebox
\newbox\mtrialbox
\newbox\startprefix  % Prefix for first line of a formula
\newbox\contprefix  % Prefix for continuation line of a formula
\def\startm{  % Initialize formula line
  \setbox\mlinebox=\hbox{\unhcopy\startprefix}
}
\def\m#1{  % Add a symbol to the formula
  \setbox\mtrialbox=\hbox{\unhcopy\mlinebox $\,#1$}
  \ifdim\wd\mtrialbox>\hsize
    \box\mlinebox
    \setbox\mlinebox=\hbox{\unhcopy\contprefix $\,#1$}
  \else
    \setbox\mlinebox=\hbox{\unhbox\mtrialbox}
  \fi
}
\def\endm{  % Output the last line of a formula
  \box\mlinebox
}

% \SLASH for \ , \TOR for \/ (text OR), \TAND for /\ (text and)
% This embeds a following forced space to force the space.
\newcommand\SLASH{\char`\\~}
\newcommand\TOR{\char`\\/~}
\newcommand\TAND{/\char`\\~}
%
% Macro to output metamath raw text.
% This assumes \startprefix and \contprefix are set.
% NOTE: "\" is tricky to escape, use \SLASH, \TOR, and \TAND inside.
% Any use of "$ { ~ ^" must be escaped; ~ and ^ must be escaped specially.
% We escape { and } for consistency.
% For more about how this macro written, see:
% https://stackoverflow.com/questions/4073674/
% how-to-disable-indentation-in-particular-section-in-latex/4075706
% Use frenchspacing, or "e." will get an extra space after it.
\newlength\mystoreparindent
\newlength\mystorehangindent
\newenvironment{mmraw}{%
\setlength{\mystoreparindent}{\the\parindent}
\setlength{\mystorehangindent}{\the\hangindent}
\setlength{\parindent}{0pt} % TODO - we'll put in the \startprefix instead
\setlength{\hangindent}{\wd\the\contprefix}
\begin{flushleft}
\begin{frenchspacing}
\begin{tt}
{\unhcopy\startprefix}%
}{%
\end{tt}
\end{frenchspacing}
\end{flushleft}
\setlength{\parindent}{\mystoreparindent}
\setlength{\hangindent}{\mystorehangindent}
\vskip 1ex
}

\needspace{5\baselineskip}
\subsection{Propositional Calculus}\label{propcalc}\index{axioms of
propositional calculus}

\needspace{2\baselineskip}
Axiom of Simplification.\label{ax1}

\setbox\startprefix=\hbox{\tt \ \ ax-1\ \$a\ }
\setbox\contprefix=\hbox{\tt \ \ \ \ \ \ \ \ \ \ }
\startm
\m{\vdash}\m{(}\m{\varphi}\m{\rightarrow}\m{(}\m{\psi}\m{\rightarrow}\m{\varphi}\m{)}
\m{)}
\endm

\needspace{3\baselineskip}
\noindent Axiom of Distribution.

\setbox\startprefix=\hbox{\tt \ \ ax-2\ \$a\ }
\setbox\contprefix=\hbox{\tt \ \ \ \ \ \ \ \ \ \ }
\startm
\m{\vdash}\m{(}\m{(}\m{\varphi}\m{\rightarrow}\m{(}\m{\psi}\m{\rightarrow}\m{\chi}
\m{)}\m{)}\m{\rightarrow}\m{(}\m{(}\m{\varphi}\m{\rightarrow}\m{\psi}\m{)}\m{
\rightarrow}\m{(}\m{\varphi}\m{\rightarrow}\m{\chi}\m{)}\m{)}\m{)}
\endm

\needspace{2\baselineskip}
\noindent Axiom of Contraposition.

\setbox\startprefix=\hbox{\tt \ \ ax-3\ \$a\ }
\setbox\contprefix=\hbox{\tt \ \ \ \ \ \ \ \ \ \ }
\startm
\m{\vdash}\m{(}\m{(}\m{\lnot}\m{\varphi}\m{\rightarrow}\m{\lnot}\m{\psi}\m{)}\m{
\rightarrow}\m{(}\m{\psi}\m{\rightarrow}\m{\varphi}\m{)}\m{)}
\endm


\needspace{4\baselineskip}
\noindent Rule of Modus Ponens.\label{axmp}\index{modus ponens}

\setbox\startprefix=\hbox{\tt \ \ min\ \$e\ }
\setbox\contprefix=\hbox{\tt \ \ \ \ \ \ \ \ \ }
\startm
\m{\vdash}\m{\varphi}
\endm

\setbox\startprefix=\hbox{\tt \ \ maj\ \$e\ }
\setbox\contprefix=\hbox{\tt \ \ \ \ \ \ \ \ \ }
\startm
\m{\vdash}\m{(}\m{\varphi}\m{\rightarrow}\m{\psi}\m{)}
\endm

\setbox\startprefix=\hbox{\tt \ \ ax-mp\ \$a\ }
\setbox\contprefix=\hbox{\tt \ \ \ \ \ \ \ \ \ \ \ }
\startm
\m{\vdash}\m{\psi}
\endm


\needspace{7\baselineskip}
\subsection{Axioms of Predicate Calculus with Equality---Tarski's S2}\index{axioms of predicate calculus}

\needspace{3\baselineskip}
\noindent Rule of Generalization.\index{rule of generalization}

\setbox\startprefix=\hbox{\tt \ \ ax-g.1\ \$e\ }
\setbox\contprefix=\hbox{\tt \ \ \ \ \ \ \ \ \ \ \ \ }
\startm
\m{\vdash}\m{\varphi}
\endm

\setbox\startprefix=\hbox{\tt \ \ ax-gen\ \$a\ }
\setbox\contprefix=\hbox{\tt \ \ \ \ \ \ \ \ \ \ \ \ }
\startm
\m{\vdash}\m{\forall}\m{x}\m{\varphi}
\endm

\needspace{2\baselineskip}
\noindent Axiom of Quantified Implication.

\setbox\startprefix=\hbox{\tt \ \ ax-4\ \$a\ }
\setbox\contprefix=\hbox{\tt \ \ \ \ \ \ \ \ \ \ }
\startm
\m{\vdash}\m{(}\m{\forall}\m{x}\m{(}\m{\forall}\m{x}\m{\varphi}\m{\rightarrow}\m{
\psi}\m{)}\m{\rightarrow}\m{(}\m{\forall}\m{x}\m{\varphi}\m{\rightarrow}\m{
\forall}\m{x}\m{\psi}\m{)}\m{)}
\endm

\needspace{3\baselineskip}
\noindent Axiom of Distinctness.

% Aka: Add $d x ph $.
\setbox\startprefix=\hbox{\tt \ \ ax-5\ \$a\ }
\setbox\contprefix=\hbox{\tt \ \ \ \ \ \ \ \ \ \ }
\startm
\m{\vdash}\m{(}\m{\varphi}\m{\rightarrow}\m{\forall}\m{x}\m{\varphi}\m{)}\m{where}\m{ }\m{\$d}\m{ }\m{x}\m{ }\m{\varphi}\m{ }\m{(}\m{x}\m{ }\m{does}\m{ }\m{not}\m{ }\m{occur}\m{ }\m{in}\m{ }\m{\varphi}\m{)}
\endm

\needspace{2\baselineskip}
\noindent Axiom of Existence.

\setbox\startprefix=\hbox{\tt \ \ ax-6\ \$a\ }
\setbox\contprefix=\hbox{\tt \ \ \ \ \ \ \ \ \ \ }
\startm
\m{\vdash}\m{(}\m{\forall}\m{x}\m{(}\m{x}\m{=}\m{y}\m{\rightarrow}\m{\forall}
\m{x}\m{\varphi}\m{)}\m{\rightarrow}\m{\varphi}\m{)}
\endm

\needspace{2\baselineskip}
\noindent Axiom of Equality.

\setbox\startprefix=\hbox{\tt \ \ ax-7\ \$a\ }
\setbox\contprefix=\hbox{\tt \ \ \ \ \ \ \ \ \ \ }
\startm
\m{\vdash}\m{(}\m{x}\m{=}\m{y}\m{\rightarrow}\m{(}\m{x}\m{=}\m{z}\m{
\rightarrow}\m{y}\m{=}\m{z}\m{)}\m{)}
\endm

\needspace{2\baselineskip}
\noindent Axiom of Left Equality for Binary Predicate.

\setbox\startprefix=\hbox{\tt \ \ ax-8\ \$a\ }
\setbox\contprefix=\hbox{\tt \ \ \ \ \ \ \ \ \ \ \ }
\startm
\m{\vdash}\m{(}\m{x}\m{=}\m{y}\m{\rightarrow}\m{(}\m{x}\m{\in}\m{z}\m{
\rightarrow}\m{y}\m{\in}\m{z}\m{)}\m{)}
\endm

\needspace{2\baselineskip}
\noindent Axiom of Right Equality for Binary Predicate.

\setbox\startprefix=\hbox{\tt \ \ ax-9\ \$a\ }
\setbox\contprefix=\hbox{\tt \ \ \ \ \ \ \ \ \ \ \ }
\startm
\m{\vdash}\m{(}\m{x}\m{=}\m{y}\m{\rightarrow}\m{(}\m{z}\m{\in}\m{x}\m{
\rightarrow}\m{z}\m{\in}\m{y}\m{)}\m{)}
\endm


\needspace{4\baselineskip}
\subsection{Axioms of Predicate Calculus with Equality---Auxiliary}\index{axioms of predicate calculus - auxiliary}

\needspace{2\baselineskip}
\noindent Axiom of Quantified Negation.

\setbox\startprefix=\hbox{\tt \ \ ax-10\ \$a\ }
\setbox\contprefix=\hbox{\tt \ \ \ \ \ \ \ \ \ \ }
\startm
\m{\vdash}\m{(}\m{\lnot}\m{\forall}\m{x}\m{\lnot}\m{\forall}\m{x}\m{\varphi}\m{
\rightarrow}\m{\varphi}\m{)}
\endm

\needspace{2\baselineskip}
\noindent Axiom of Quantifier Commutation.

\setbox\startprefix=\hbox{\tt \ \ ax-11\ \$a\ }
\setbox\contprefix=\hbox{\tt \ \ \ \ \ \ \ \ \ \ }
\startm
\m{\vdash}\m{(}\m{\forall}\m{x}\m{\forall}\m{y}\m{\varphi}\m{\rightarrow}\m{
\forall}\m{y}\m{\forall}\m{x}\m{\varphi}\m{)}
\endm

\needspace{3\baselineskip}
\noindent Axiom of Substitution.

\setbox\startprefix=\hbox{\tt \ \ ax-12\ \$a\ }
\setbox\contprefix=\hbox{\tt \ \ \ \ \ \ \ \ \ \ \ }
\startm
\m{\vdash}\m{(}\m{\lnot}\m{\forall}\m{x}\m{\,x}\m{=}\m{y}\m{\rightarrow}\m{(}
\m{x}\m{=}\m{y}\m{\rightarrow}\m{(}\m{\varphi}\m{\rightarrow}\m{\forall}\m{x}\m{(}
\m{x}\m{=}\m{y}\m{\rightarrow}\m{\varphi}\m{)}\m{)}\m{)}\m{)}
\endm

\needspace{3\baselineskip}
\noindent Axiom of Quantified Equality.

\setbox\startprefix=\hbox{\tt \ \ ax-13\ \$a\ }
\setbox\contprefix=\hbox{\tt \ \ \ \ \ \ \ \ \ \ \ }
\startm
\m{\vdash}\m{(}\m{\lnot}\m{\forall}\m{z}\m{\,z}\m{=}\m{x}\m{\rightarrow}\m{(}
\m{\lnot}\m{\forall}\m{z}\m{\,z}\m{=}\m{y}\m{\rightarrow}\m{(}\m{x}\m{=}\m{y}
\m{\rightarrow}\m{\forall}\m{z}\m{\,x}\m{=}\m{y}\m{)}\m{)}\m{)}
\endm

% \noindent Axiom of Quantifier Substitution
%
% \setbox\startprefix=\hbox{\tt \ \ ax-c11n\ \$a\ }
% \setbox\contprefix=\hbox{\tt \ \ \ \ \ \ \ \ \ \ \ }
% \startm
% \m{\vdash}\m{(}\m{\forall}\m{x}\m{\,x}\m{=}\m{y}\m{\rightarrow}\m{(}\m{\forall}
% \m{x}\m{\varphi}\m{\rightarrow}\m{\forall}\m{y}\m{\varphi}\m{)}\m{)}
% \endm
%
% \noindent Axiom of Distinct Variables. (This axiom requires
% that two individual variables
% be distinct\index{\texttt{\$d} statement}\index{distinct
% variables}.)
%
% \setbox\startprefix=\hbox{\tt \ \ \ \ \ \ \ \ \$d\ }
% \setbox\contprefix=\hbox{\tt \ \ \ \ \ \ \ \ \ \ \ }
% \startm
% \m{x}\m{\,}\m{y}
% \endm
%
% \setbox\startprefix=\hbox{\tt \ \ ax-c16\ \$a\ }
% \setbox\contprefix=\hbox{\tt \ \ \ \ \ \ \ \ \ \ \ }
% \startm
% \m{\vdash}\m{(}\m{\forall}\m{x}\m{\,x}\m{=}\m{y}\m{\rightarrow}\m{(}\m{\varphi}\m{
% \rightarrow}\m{\forall}\m{x}\m{\varphi}\m{)}\m{)}
% \endm

% \noindent Axiom of Quantifier Introduction (2).  (This axiom requires
% that the individual variable not occur in the
% wff\index{\texttt{\$d} statement}\index{distinct variables}.)
%
% \setbox\startprefix=\hbox{\tt \ \ \ \ \ \ \ \ \$d\ }
% \setbox\contprefix=\hbox{\tt \ \ \ \ \ \ \ \ \ \ \ }
% \startm
% \m{x}\m{\,}\m{\varphi}
% \endm
% \setbox\startprefix=\hbox{\tt \ \ ax-5\ \$a\ }
% \setbox\contprefix=\hbox{\tt \ \ \ \ \ \ \ \ \ \ \ }
% \startm
% \m{\vdash}\m{(}\m{\varphi}\m{\rightarrow}\m{\forall}\m{x}\m{\varphi}\m{)}
% \endm

\subsection{Set Theory}\label{mmsettheoryaxioms}

In order to make the axioms of set theory\index{axioms of set theory} a little
more compact, there are several definitions from logic that we make use of
implicitly, namely, ``logical {\sc and},''\index{conjunction ($\wedge$)}
\index{logical {\sc and} ($\wedge$)} ``logical equivalence,''\index{logical
equivalence ($\leftrightarrow$)}\index{biconditional ($\leftrightarrow$)} and
``there exists.''\index{existential quantifier ($\exists$)}

\begin{center}\begin{tabular}{rcl}
  $( \varphi \wedge \psi )$ &\mbox{stands for}& $\neg ( \varphi
     \rightarrow \neg \psi )$\\
  $( \varphi \leftrightarrow \psi )$& \mbox{stands
     for}& $( ( \varphi \rightarrow \psi ) \wedge
     ( \psi \rightarrow \varphi ) )$\\
  $\exists x \,\varphi$ &\mbox{stands for}& $\neg \forall x \neg \varphi$
\end{tabular}\end{center}

In addition, the axioms of set theory require that all variables be
dis\-tinct,\index{distinct variables}\footnote{Set theory axioms can be
devised so that {\em no} variables are required to be distinct,
provided we replace \texttt{ax-c16} with an axiom stating that ``at
least two things exist,'' thus
making \texttt{ax-5} the only other axiom requiring the
\texttt{\$d} statement.  These axioms are unconventional and are not
presented here, but they can be found on the \url{http://metamath.org}
web site.  See also the Comment on
p.~\pageref{nodd}.}\index{\texttt{\$d} statement} thus we also assume:
\begin{center}
  \texttt{\$d }$x\,y\,z\,w$
\end{center}

\needspace{2\baselineskip}
\noindent Axiom of Extensionality.\index{Axiom of Extensionality}

\setbox\startprefix=\hbox{\tt \ \ ax-ext\ \$a\ }
\setbox\contprefix=\hbox{\tt \ \ \ \ \ \ \ \ \ \ \ \ }
\startm
\m{\vdash}\m{(}\m{\forall}\m{x}\m{(}\m{x}\m{\in}\m{y}\m{\leftrightarrow}\m{x}
\m{\in}\m{z}\m{)}\m{\rightarrow}\m{y}\m{=}\m{z}\m{)}
\endm

\needspace{3\baselineskip}
\noindent Axiom of Replacement.\index{Axiom of Replacement}

\setbox\startprefix=\hbox{\tt \ \ ax-rep\ \$a\ }
\setbox\contprefix=\hbox{\tt \ \ \ \ \ \ \ \ \ \ \ \ }
\startm
\m{\vdash}\m{(}\m{\forall}\m{w}\m{\exists}\m{y}\m{\forall}\m{z}\m{(}\m{%
\forall}\m{y}\m{\varphi}\m{\rightarrow}\m{z}\m{=}\m{y}\m{)}\m{\rightarrow}\m{%
\exists}\m{y}\m{\forall}\m{z}\m{(}\m{z}\m{\in}\m{y}\m{\leftrightarrow}\m{%
\exists}\m{w}\m{(}\m{w}\m{\in}\m{x}\m{\wedge}\m{\forall}\m{y}\m{\varphi}\m{)}%
\m{)}\m{)}
\endm

\needspace{2\baselineskip}
\noindent Axiom of Union.\index{Axiom of Union}

\setbox\startprefix=\hbox{\tt \ \ ax-un\ \$a\ }
\setbox\contprefix=\hbox{\tt \ \ \ \ \ \ \ \ \ \ \ }
\startm
\m{\vdash}\m{\exists}\m{x}\m{\forall}\m{y}\m{(}\m{\exists}\m{x}\m{(}\m{y}\m{
\in}\m{x}\m{\wedge}\m{x}\m{\in}\m{z}\m{)}\m{\rightarrow}\m{y}\m{\in}\m{x}\m{)}
\endm

\needspace{2\baselineskip}
\noindent Axiom of Power Sets.\index{Axiom of Power Sets}

\setbox\startprefix=\hbox{\tt \ \ ax-pow\ \$a\ }
\setbox\contprefix=\hbox{\tt \ \ \ \ \ \ \ \ \ \ \ \ }
\startm
\m{\vdash}\m{\exists}\m{x}\m{\forall}\m{y}\m{(}\m{\forall}\m{x}\m{(}\m{x}\m{
\in}\m{y}\m{\rightarrow}\m{x}\m{\in}\m{z}\m{)}\m{\rightarrow}\m{y}\m{\in}\m{x}
\m{)}
\endm

\needspace{3\baselineskip}
\noindent Axiom of Regularity.\index{Axiom of Regularity}

\setbox\startprefix=\hbox{\tt \ \ ax-reg\ \$a\ }
\setbox\contprefix=\hbox{\tt \ \ \ \ \ \ \ \ \ \ \ \ }
\startm
\m{\vdash}\m{(}\m{\exists}\m{x}\m{\,x}\m{\in}\m{y}\m{\rightarrow}\m{\exists}
\m{x}\m{(}\m{x}\m{\in}\m{y}\m{\wedge}\m{\forall}\m{z}\m{(}\m{z}\m{\in}\m{x}\m{
\rightarrow}\m{\lnot}\m{z}\m{\in}\m{y}\m{)}\m{)}\m{)}
\endm

\needspace{3\baselineskip}
\noindent Axiom of Infinity.\index{Axiom of Infinity}

\setbox\startprefix=\hbox{\tt \ \ ax-inf\ \$a\ }
\setbox\contprefix=\hbox{\tt \ \ \ \ \ \ \ \ \ \ \ \ \ \ \ }
\startm
\m{\vdash}\m{\exists}\m{x}\m{(}\m{y}\m{\in}\m{x}\m{\wedge}\m{\forall}\m{y}%
\m{(}\m{y}\m{\in}\m{x}\m{\rightarrow}\m{\exists}\m{z}\m{(}\m{y}\m{\in}\m{z}\m{%
\wedge}\m{z}\m{\in}\m{x}\m{)}\m{)}\m{)}
\endm

\needspace{4\baselineskip}
\noindent Axiom of Choice.\index{Axiom of Choice}

\setbox\startprefix=\hbox{\tt \ \ ax-ac\ \$a\ }
\setbox\contprefix=\hbox{\tt \ \ \ \ \ \ \ \ \ \ \ \ \ \ }
\startm
\m{\vdash}\m{\exists}\m{x}\m{\forall}\m{y}\m{\forall}\m{z}\m{(}\m{(}\m{y}\m{%
\in}\m{z}\m{\wedge}\m{z}\m{\in}\m{w}\m{)}\m{\rightarrow}\m{\exists}\m{w}\m{%
\forall}\m{y}\m{(}\m{\exists}\m{w}\m{(}\m{(}\m{y}\m{\in}\m{z}\m{\wedge}\m{z}%
\m{\in}\m{w}\m{)}\m{\wedge}\m{(}\m{y}\m{\in}\m{w}\m{\wedge}\m{w}\m{\in}\m{x}%
\m{)}\m{)}\m{\leftrightarrow}\m{y}\m{=}\m{w}\m{)}\m{)}
\endm

\subsection{That's It}

There you have it, the axioms for (essentially) all of mathematics!
Wonder at them and stare at them in awe.  Put a copy in your wallet, and
you will carry in your pocket the encoding for all theorems ever proved
and that ever will be proved, from the most mundane to the most
profound.

\section{A Hierarchy of Definitions}\label{hierarchy}

The axioms in the previous section in principle embody everything that can be
done within standard mathematics.  However, it is impractical to accomplish
very much by using them directly, for even simple concepts (from a human
perspective) can involve extremely long, incomprehensible formulas.
Mathematics is made practical by introducing definitions\index{definition}.
Definitions usually introduce new symbols, or at least new relationships among
existing symbols, to abbreviate more complex formulas.  An important
requirement for a definition is that there exist a straightforward
(algorithmic) method for eliminating the abbreviation by expanding it into the
more primitive symbol string that it represents.  Some
important definitions included in
the file \texttt{set.mm} are listed in this section for reference, and also to
give you a feel for why something like $\omega$\index{omega ($\omega$)} (the
set of natural numbers\index{natural number} 0, 1, 2,\ldots) becomes very
complicated when completely expanded into primitive symbols.

What is the motivation for definitions, aside from allowing complicated
expressions to be expressed more simply?  In the case of  $\omega$, one goal is
to provide a basis for the theory of natural numbers.\index{natural number}
Before set theory was invented, a set of axioms for arithmetic, called Peano's
postulates\index{Peano's postulates}, was devised and shown to have the
properties one expects for natural numbers.  Now anyone can postulate a
set of axioms, but if the axioms are inconsistent contradictions can be derived
from them.  Once a contradiction is derived, anything can be trivially
proved, including
all the facts of arithmetic and their negations.  To ensure that an
axiom system is at least as reliable as the axioms for set theory, we can
define sets and operations on those sets that satisfy the new axioms. In the
\texttt{set.mm} Metamath database, we prove that the elements of $\omega$ satisfy
Peano's postulates, and it's a long and hard journey to get there directly
from the axioms of set theory.  But the result is confidence in the
foundations of arithmetic.  And there is another advantage:  we now have all
the tools of set theory at our disposal for manipulating objects that obey the
axioms for arithmetic.

What are the criteria we use for definitions?  First, and of utmost importance,
the definition should not be {\em creative}\index{creative
definition}\index{definition!creative}, that
is it should not allow an expression that previously qualified as a wff but
was not provable, to become provable.   Second, the definition should be {\em
eliminable}\index{definition!eliminability}, that is, there should exist an
algorithmic method for converting any expression using the definition into
a logically equivalent expression that previously qualified as a wff.

In almost all cases below, definitions connect two expressions with either
$\leftrightarrow$ or $=$.  Eliminating\footnote{Here we mean the
elimination that a human might do in his or her head.  To eliminate them as
part of a Metamath proof we would invoke one of a number of
theorems that deal with transitivity of equivalence or equality; there are
many such examples in the proofs in \texttt{set.mm}.} such a definition is a
simple matter of substituting the expression on the left-hand side ({\em
definiendum}\index{definiendum} or thing being defined) with the equivalent,
more primitive expression on the right-hand side ({\em
definiens}\index{definiens} or definition).

Often a definition has variables on the right-hand side which do not appear on
the left-hand side; these are called {\em dummy variables}.\index{dummy
variable!in definitions}  In this case, any
allowable substitution (such as a new, distinct
variable) can be used when the definition is eliminated.  Dummy variables may
be used only if they are {\em effectively bound}\index{effectively bound
variable}, meaning that the definition will remain logically equivalent upon
any substitution of a dummy variable with any other {\em qualifying
expression}\index{qualifying expression}, i.e.\ any symbol string (such as
another variable) that
meets the restrictions on the dummy variable imposed by \texttt{\$d} and
\texttt{\$f} statements.  For example, we could define a constant $\perp$
(inverted tee, meaning logical ``false'') as $( \varphi \wedge \lnot \varphi
)$, i.e.\ ``phi and not phi.''  Here $\varphi$ is effectively bound because the
definition remains logically equivalent when we replace $\varphi$ with any
other wff.  (It is actually \texttt{df-fal}
in \texttt{set.mm}, which defines $\perp$.)

There are two cases where eliminating definitions is a little more
complex.  These cases are the definitions \texttt{df-bi} and
\texttt{df-cleq}.  The first stretches the concept of a definition a
little, as in effect it ``defines a definition;'' however, it meets our
requirements for a definition in that it is eliminable and does not
strengthen the language.  Theorem \texttt{bii} shows the substitution
needed to eliminate the $\leftrightarrow$\index{logical equivalence
($\leftrightarrow$)}\index{biconditional ($\leftrightarrow$)} symbol.

Definition \texttt{df-cleq}\index{equality ($=$)} extends the usage of
the equality symbol to include ``classes''\index{class} in set theory.  The
reason it is potentially problematic is that it can lead to statements which
do not follow from logic alone but presuppose the Axiom of
Extensionality\index{Axiom of Extensionality}, so we include this axiom
as a hypothesis for the definition.  We could have made \texttt{df-cleq} directly
eliminable by introducing a new equality symbol, but have chosen not to do so
in keeping with standard textbook practice.  Definitions such as \texttt{df-cleq}
that extend the meaning of existing symbols must be introduced carefully so
that they do not lead to contradictions.  Definition \texttt{df-clel} also
extends the meaning of an existing symbol ($\in$); while it doesn't strengthen
the language like \texttt{df-cleq}, this is not obvious and it must also be
subject to the same scrutiny.

Exercise:  Study how the wff $x\in\omega$, meaning ``$x$ is a natural
number,'' could be expanded in terms of primitive symbols, starting with the
definitions \texttt{df-clel} on p.~\pageref{dfclel} and \texttt{df-om} on
p.~\pageref{dfom} and working your way back.  Don't bother to work out the
details; just make sure that you understand how you could do it in principle.
The answer is shown in the footnote on p.~\pageref{expandom}.  If you
actually do work it out, you won't get exactly the same answer because we used
a few simplifications such as discarding occurrences of $\lnot\lnot$ (double
negation).

In the definitions below, we have placed the {\sc ascii} Metamath source
below each of the formulas to help you become familiar with the
notation in the database.  For simplicity, the necessary \texttt{\$f}
and \texttt{\$d} statements are not shown.  If you are in doubt, use the
\texttt{show statement}\index{\texttt{show statement} command} command
in the Metamath program to see the full statement.
A selection of this notation is summarized in Appendix~\ref{ASCII}.

To understand the motivation for these definitions, you should consult the
references indicated:  Takeuti and Zaring \cite{Takeuti}\index{Takeuti, Gaisi},
Quine \cite{Quine}\index{Quine, Willard Van Orman}, Bell and Machover
\cite{Bell}\index{Bell, J. L.}, and Enderton \cite{Enderton}\index{Enderton,
Herbert B.}.  Our list of definitions is provided more for reference than as a
learning aid.  However, by looking at a few of them you can gain a feel for
how the hierarchy is built up.  The definitions are a representative sample of
the many definitions
in \texttt{set.mm}, but they are complete with respect to the
theorem examples we will present in Section~\ref{sometheorems}.  Also, some are
slightly different from, but logically equivalent to, the ones in \texttt{set.mm}
(some of which have been revised over time to shorten them, for example).

\subsection{Definitions for Propositional Calculus}\label{metadefprop}

The symbols $\varphi$, $\psi$, and $\chi$ represent wffs.

Our first definition introduces the biconditional
connective\footnote{The term ``connective'' is informally used to mean a
symbol that is placed between two variables or adjacent to a variable,
whereas a mathematical ``constant'' usually indicates a symbol such as
the number 0 that may replace a variable or metavariable.  From
Metamath's point of view, there is no distinction between a connective
and a constant; both are constants in the Metamath
language.}\index{connective}\index{constant} (also called logical
equivalence)\index{logical equivalence
($\leftrightarrow$)}\index{biconditional ($\leftrightarrow$)}.  Unlike
most traditional developments, we have chosen not to have a separate
symbol such as ``Df.'' to mean ``is defined as.''  Instead, we will use
the biconditional connective for this purpose, as it lets us use
logic to manipulate definitions directly.  Here we state the properties
of the biconditional connective with a carefully crafted \texttt{\$a}
statement, which effectively uses the biconditional connective to define
itself.  The $\leftrightarrow$ symbol can be eliminated from a formula
using theorem \texttt{bii}, which is derived later.

\vskip 2ex
\noindent Define the biconditional connective.\label{df-bi}

\vskip 0.5ex
\setbox\startprefix=\hbox{\tt \ \ df-bi\ \$a\ }
\setbox\contprefix=\hbox{\tt \ \ \ \ \ \ \ \ \ \ \ }
\startm
\m{\vdash}\m{\lnot}\m{(}\m{(}\m{(}\m{\varphi}\m{\leftrightarrow}\m{\psi}\m{)}%
\m{\rightarrow}\m{\lnot}\m{(}\m{(}\m{\varphi}\m{\rightarrow}\m{\psi}\m{)}\m{%
\rightarrow}\m{\lnot}\m{(}\m{\psi}\m{\rightarrow}\m{\varphi}\m{)}\m{)}\m{)}\m{%
\rightarrow}\m{\lnot}\m{(}\m{\lnot}\m{(}\m{(}\m{\varphi}\m{\rightarrow}\m{%
\psi}\m{)}\m{\rightarrow}\m{\lnot}\m{(}\m{\psi}\m{\rightarrow}\m{\varphi}\m{)}%
\m{)}\m{\rightarrow}\m{(}\m{\varphi}\m{\leftrightarrow}\m{\psi}\m{)}\m{)}\m{)}
\endm
\begin{mmraw}%
|- -. ( ( ( ph <-> ps ) -> -. ( ( ph -> ps ) ->
-. ( ps -> ph ) ) ) -> -. ( -. ( ( ph -> ps ) -> -. (
ps -> ph ) ) -> ( ph <-> ps ) ) ) \$.
\end{mmraw}

\noindent This theorem relates the biconditional connective to primitive
connectives and can be used to eliminate the $\leftrightarrow$ symbol from any
wff.

\vskip 0.5ex
\setbox\startprefix=\hbox{\tt \ \ bii\ \$p\ }
\setbox\contprefix=\hbox{\tt \ \ \ \ \ \ \ \ \ }
\startm
\m{\vdash}\m{(}\m{(}\m{\varphi}\m{\leftrightarrow}\m{\psi}\m{)}\m{\leftrightarrow}
\m{\lnot}\m{(}\m{(}\m{\varphi}\m{\rightarrow}\m{\psi}\m{)}\m{\rightarrow}\m{\lnot}
\m{(}\m{\psi}\m{\rightarrow}\m{\varphi}\m{)}\m{)}\m{)}
\endm
\begin{mmraw}%
|- ( ( ph <-> ps ) <-> -. ( ( ph -> ps ) -> -. ( ps -> ph ) ) ) \$= ... \$.
\end{mmraw}

\noindent Define disjunction ({\sc or}).\index{disjunction ($\vee$)}%
\index{logical or (vee)@logical {\sc or} ($\vee$)}%
\index{df-or@\texttt{df-or}}\label{df-or}

\vskip 0.5ex
\setbox\startprefix=\hbox{\tt \ \ df-or\ \$a\ }
\setbox\contprefix=\hbox{\tt \ \ \ \ \ \ \ \ \ \ \ }
\startm
\m{\vdash}\m{(}\m{(}\m{\varphi}\m{\vee}\m{\psi}\m{)}\m{\leftrightarrow}\m{(}\m{
\lnot}\m{\varphi}\m{\rightarrow}\m{\psi}\m{)}\m{)}
\endm
\begin{mmraw}%
|- ( ( ph \TOR ps ) <-> ( -. ph -> ps ) ) \$.
\end{mmraw}

\noindent Define conjunction ({\sc and}).\index{conjunction ($\wedge$)}%
\index{logical {\sc and} ($\wedge$)}%
\index{df-an@\texttt{df-an}}\label{df-an}

\vskip 0.5ex
\setbox\startprefix=\hbox{\tt \ \ df-an\ \$a\ }
\setbox\contprefix=\hbox{\tt \ \ \ \ \ \ \ \ \ \ \ }
\startm
\m{\vdash}\m{(}\m{(}\m{\varphi}\m{\wedge}\m{\psi}\m{)}\m{\leftrightarrow}\m{\lnot}
\m{(}\m{\varphi}\m{\rightarrow}\m{\lnot}\m{\psi}\m{)}\m{)}
\endm
\begin{mmraw}%
|- ( ( ph \TAND ps ) <-> -. ( ph -> -. ps ) ) \$.
\end{mmraw}

\noindent Define disjunction ({\sc or}) of 3 wffs.%
\index{df-3or@\texttt{df-3or}}\label{df-3or}

\vskip 0.5ex
\setbox\startprefix=\hbox{\tt \ \ df-3or\ \$a\ }
\setbox\contprefix=\hbox{\tt \ \ \ \ \ \ \ \ \ \ \ \ }
\startm
\m{\vdash}\m{(}\m{(}\m{\varphi}\m{\vee}\m{\psi}\m{\vee}\m{\chi}\m{)}\m{
\leftrightarrow}\m{(}\m{(}\m{\varphi}\m{\vee}\m{\psi}\m{)}\m{\vee}\m{\chi}\m{)}
\m{)}
\endm
\begin{mmraw}%
|- ( ( ph \TOR ps \TOR ch ) <-> ( ( ph \TOR ps ) \TOR ch ) ) \$.
\end{mmraw}

\noindent Define conjunction ({\sc and}) of 3 wffs.%
\index{df-3an}\label{df-3an}

\vskip 0.5ex
\setbox\startprefix=\hbox{\tt \ \ df-3an\ \$a\ }
\setbox\contprefix=\hbox{\tt \ \ \ \ \ \ \ \ \ \ \ \ }
\startm
\m{\vdash}\m{(}\m{(}\m{\varphi}\m{\wedge}\m{\psi}\m{\wedge}\m{\chi}\m{)}\m{
\leftrightarrow}\m{(}\m{(}\m{\varphi}\m{\wedge}\m{\psi}\m{)}\m{\wedge}\m{\chi}
\m{)}\m{)}
\endm

\begin{mmraw}%
|- ( ( ph \TAND ps \TAND ch ) <-> ( ( ph \TAND ps ) \TAND ch ) ) \$.
\end{mmraw}

\subsection{Definitions for Predicate Calculus}\label{metadefpred}

The symbols $x$, $y$, and $z$ represent individual variables of predicate
calculus.  In this section, they are not necessarily distinct unless it is
explicitly
mentioned.

\vskip 2ex
\noindent Define existential quantification.
The expression $\exists x \varphi$ means
``there exists an $x$ where $\varphi$ is true.''\index{existential quantifier
($\exists$)}\label{df-ex}

\vskip 0.5ex
\setbox\startprefix=\hbox{\tt \ \ df-ex\ \$a\ }
\setbox\contprefix=\hbox{\tt \ \ \ \ \ \ \ \ \ \ \ }
\startm
\m{\vdash}\m{(}\m{\exists}\m{x}\m{\varphi}\m{\leftrightarrow}\m{\lnot}\m{\forall}
\m{x}\m{\lnot}\m{\varphi}\m{)}
\endm
\begin{mmraw}%
|- ( E. x ph <-> -. A. x -. ph ) \$.
\end{mmraw}

\noindent Define proper substitution.\index{proper
substitution}\index{substitution!proper}\label{df-sb}
In our notation, we use $[ y / x ] \varphi$ to mean ``the wff that
results when $y$ is properly substituted for $x$ in the wff
$\varphi$.''\footnote{
This can also be described
as substituting $x$ with $y$, $y$ properly replaces $x$, or
$x$ is properly replaced by $y$.}
% This is elsb4, though it currently says: ( [ x / y ] z e. y <-> z e. x )
For example,
$[ y / x ] z \in x$ is the same as $z \in y$.
One way to remember this notation is to notice that it looks like division
and recall that $( y / x ) \cdot x $ is $y$ (when $x \neq 0$).
The notation is different from the notation $\varphi ( x | y )$
that is sometimes used, because the latter notation is ambiguous for us:
for example, we don't know whether $\lnot \varphi ( x | y )$ is to be
interpreted as $\lnot ( \varphi ( x | y ) )$ or
$( \lnot \varphi ) ( x | y )$.\footnote{Because of the way
we initially defined wffs, this is the case
with any postfix connective\index{postfix connective} (one occurring after the
symbols being connected) or infix connective\index{infix connective} (one
occurring between the symbols being connected).  Metamath does not have a
built-in notion of operator binding strength that could eliminate the
ambiguity.  The initial parenthesis effectively provides a prefix
connective\index{prefix connective} to eliminate ambiguity.  Some conventions,
such as Polish notation\index{Polish notation} used in the 1930's and 1940's
by Polish logicians, use only prefix connectives and thus allow the total
elimination of parentheses, at the expense of readability.  In Metamath we
could actually redefine all notation to be Polish if we wanted to without
having to change any proofs!}  Other texts often use $\varphi(y)$ to indicate
our $[ y / x ] \varphi$, but this notation is even more ambiguous since there is
no explicit indication of what is being substituted.
Note that this
definition is valid even when
$x$ and $y$ are the same variable.  The first conjunct is a ``trick'' used to
achieve this property, making the definition look somewhat peculiar at
first.

\vskip 0.5ex
\setbox\startprefix=\hbox{\tt \ \ df-sb\ \$a\ }
\setbox\contprefix=\hbox{\tt \ \ \ \ \ \ \ \ \ \ \ }
\startm
\m{\vdash}\m{(}\m{[}\m{y}\m{/}\m{x}\m{]}\m{\varphi}\m{\leftrightarrow}\m{(}%
\m{(}\m{x}\m{=}\m{y}\m{\rightarrow}\m{\varphi}\m{)}\m{\wedge}\m{\exists}\m{x}%
\m{(}\m{x}\m{=}\m{y}\m{\wedge}\m{\varphi}\m{)}\m{)}\m{)}
\endm
\begin{mmraw}%
|- ( [ y / x ] ph <-> ( ( x = y -> ph ) \TAND E. x ( x = y \TAND ph ) ) ) \$.
\end{mmraw}


\noindent Define existential uniqueness\index{existential uniqueness
quantifier ($\exists "!$)} (``there exists exactly one'').  Note that $y$ is a
variable distinct from $x$ and not occurring in $\varphi$.\label{df-eu}

\vskip 0.5ex
\setbox\startprefix=\hbox{\tt \ \ df-eu\ \$a\ }
\setbox\contprefix=\hbox{\tt \ \ \ \ \ \ \ \ \ \ \ }
\startm
\m{\vdash}\m{(}\m{\exists}\m{{!}}\m{x}\m{\varphi}\m{\leftrightarrow}\m{\exists}
\m{y}\m{\forall}\m{x}\m{(}\m{\varphi}\m{\leftrightarrow}\m{x}\m{=}\m{y}\m{)}\m{)}
\endm

\begin{mmraw}%
|- ( E! x ph <-> E. y A. x ( ph <-> x = y ) ) \$.
\end{mmraw}

\subsection{Definitions for Set Theory}\label{setdefinitions}

The symbols $x$, $y$, $z$, and $w$ represent individual variables of
predicate calculus, which in set theory are understood to be sets.
However, using only the constructs shown so far would be very inconvenient.

To make set theory more practical, we introduce the notion of a ``class.''
A class\index{class} is either a set variable (such as $x$) or an
expression of the form $\{ x | \varphi\}$ (called an ``abstraction
class''\index{abstraction class}\index{class abstraction}).  Note that
sets (i.e.\ individual variables) always exist (this is a theorem of
logic, namely $\exists y \, y = x$ for any set $x$), whereas classes may
or may not exist (i.e.\ $\exists y \, y = A$ may or may not be true).
If a class does not exist it is called a ``proper class.''\index{proper
class}\index{class!proper} Definitions \texttt{df-clab},
\texttt{df-cleq}, and \texttt{df-clel} can be used to convert an
expression containing classes into one containing only set variables and
wff metavariables.

The symbols $A$, $B$, $C$, $D$, $ F$, $G$, and $R$ are metavariables that range
over classes.  A class metavariable $A$ may be eliminated from a wff by
replacing it with $\{ x|\varphi\}$ where neither $x$ nor $\varphi$ occur in
the wff.

The theory of classes can be shown to be an eliminable and conservative
extension of set theory. The \textbf{eliminability}
property shows that for every
formula in the extended language we can build a logically equivalent
formula in the basic language; so that even if the extended language
provides more ease to convey and formulate mathematical ideas for set
theory, its expressive power does not in fact strengthen the basic
language's expressive power.
The \textbf{conservation} property shows that for
every proof of a formula of the basic language in the extended system
we can build another proof of the same formula in the basic system;
so that, concerning theorems on sets only, the deductive powers of
the extended system and of the basic system are identical. Together,
these properties mean that the extended language can be treated as a
definitional extension that is \textbf{sound}.

A rigorous justification, which we will not give here, can be found in
Levy \cite[pp.~357-366]{Levy} supplementing his informal introduction to class
theory on pp.~7-17. Two other good treatments of class theory are provided
by Quine \cite[pp.~15-21]{Quine}\index{Quine, Willard Van Orman}
and also \cite[pp.~10-14]{Takeuti}\index{Takeuti, Gaisi}.
Quine's exposition (he calls them virtual classes)
is nicely written and very readable.

In the rest of this
section, individual variables are always assumed to be distinct from
each other unless otherwise indicated.  In addition, dummy variables on the
right-hand side of a definition do not occur in the class and wff
metavariables in the definition.

The definitions we present here are a partial but self-contained
collection selected from several hundred that appear in the current
\texttt{set.mm} database.  They are adequate for a basic development of
elementary set theory.

\vskip 2ex
\noindent Define the abstraction class.\index{abstraction class}\index{class
abstraction}\label{df-clab}  $x$ and $y$
need not be distinct.  Definition 2.1 of Quine, p.~16.  This definition may
seem puzzling since it is shorter than the expression being defined and does not
buy us anything in terms of brevity.  The reason we introduce this definition
is because it fits in neatly with the extension of the $\in$ connective
provided by \texttt{df-clel}.

\vskip 0.5ex
\setbox\startprefix=\hbox{\tt \ \ df-clab\ \$a\ }
\setbox\contprefix=\hbox{\tt \ \ \ \ \ \ \ \ \ \ \ \ \ }
\startm
\m{\vdash}\m{(}\m{x}\m{\in}\m{\{}\m{y}\m{|}\m{\varphi}\m{\}}\m{%
\leftrightarrow}\m{[}\m{x}\m{/}\m{y}\m{]}\m{\varphi}\m{)}
\endm
\begin{mmraw}%
|- ( x e. \{ y | ph \} <-> [ x / y ] ph ) \$.
\end{mmraw}

\noindent Define the equality connective between classes\index{class
equality}\label{df-cleq}.  See Quine or Chapter 4 of Takeuti and Zaring for its
justification and methods for eliminating it.  This is an example of a
somewhat ``dangerous'' definition, because it extends the use of the
existing equality symbol rather than introducing a new symbol, allowing
us to make statements in the original language that may not be true.
For example, it permits us to deduce $y = z \leftrightarrow \forall x (
x \in y \leftrightarrow x \in z )$ which is not a theorem of logic but
rather presupposes the Axiom of Extensionality,\index{Axiom of
Extensionality} which we include as a hypothesis so that we can know
when this axiom is assumed in a proof (with the \texttt{show
trace{\char`\_}back} command).  We could avoid the danger by introducing
another symbol, say $\eqcirc$, in place of $=$; this
would also have the advantage of making elimination of the definition
straightforward and would eliminate the need for Extensionality as a
hypothesis.  We would then also have the advantage of being able to
identify exactly where Extensionality truly comes into play.  One of our
theorems would be $x \eqcirc y \leftrightarrow x = y$
by invoking Extensionality.  However in keeping with standard practice
we retain the ``dangerous'' definition.

\vskip 0.5ex
\setbox\startprefix=\hbox{\tt \ \ df-cleq.1\ \$e\ }
\setbox\contprefix=\hbox{\tt \ \ \ \ \ \ \ \ \ \ \ \ \ \ \ }
\startm
\m{\vdash}\m{(}\m{\forall}\m{x}\m{(}\m{x}\m{\in}\m{y}\m{\leftrightarrow}\m{x}
\m{\in}\m{z}\m{)}\m{\rightarrow}\m{y}\m{=}\m{z}\m{)}
\endm
\setbox\startprefix=\hbox{\tt \ \ df-cleq\ \$a\ }
\setbox\contprefix=\hbox{\tt \ \ \ \ \ \ \ \ \ \ \ \ \ }
\startm
\m{\vdash}\m{(}\m{A}\m{=}\m{B}\m{\leftrightarrow}\m{\forall}\m{x}\m{(}\m{x}\m{
\in}\m{A}\m{\leftrightarrow}\m{x}\m{\in}\m{B}\m{)}\m{)}
\endm
% We need to reset the startprefix and contprefix.
\setbox\startprefix=\hbox{\tt \ \ df-cleq.1\ \$e\ }
\setbox\contprefix=\hbox{\tt \ \ \ \ \ \ \ \ \ \ \ \ \ \ \ }
\begin{mmraw}%
|- ( A. x ( x e. y <-> x e. z ) -> y = z ) \$.
\end{mmraw}
\setbox\startprefix=\hbox{\tt \ \ df-cleq\ \$a\ }
\setbox\contprefix=\hbox{\tt \ \ \ \ \ \ \ \ \ \ \ \ \ }
\begin{mmraw}%
|- ( A = B <-> A. x ( x e. A <-> x e. B ) ) \$.
\end{mmraw}

\noindent Define the membership connective between classes\index{class
membership}.  Theorem 6.3 of Quine, p.~41, which we adopt as a definition.
Note that it extends the use of the existing membership symbol, but unlike
{\tt df-cleq} it does not extend the set of valid wffs of logic when the class
metavariables are replaced with set variables.\label{dfclel}\label{df-clel}

\vskip 0.5ex
\setbox\startprefix=\hbox{\tt \ \ df-clel\ \$a\ }
\setbox\contprefix=\hbox{\tt \ \ \ \ \ \ \ \ \ \ \ \ \ }
\startm
\m{\vdash}\m{(}\m{A}\m{\in}\m{B}\m{\leftrightarrow}\m{\exists}\m{x}\m{(}\m{x}
\m{=}\m{A}\m{\wedge}\m{x}\m{\in}\m{B}\m{)}\m{)}
\endm
\begin{mmraw}%
|- ( A e. B <-> E. x ( x = A \TAND x e. B ) ) \$.?
\end{mmraw}

\noindent Define inequality.

\vskip 0.5ex
\setbox\startprefix=\hbox{\tt \ \ df-ne\ \$a\ }
\setbox\contprefix=\hbox{\tt \ \ \ \ \ \ \ \ \ \ \ }
\startm
\m{\vdash}\m{(}\m{A}\m{\ne}\m{B}\m{\leftrightarrow}\m{\lnot}\m{A}\m{=}\m{B}%
\m{)}
\endm
\begin{mmraw}%
|- ( A =/= B <-> -. A = B ) \$.
\end{mmraw}

\noindent Define restricted universal quantification.\index{universal
quantifier ($\forall$)!restricted}  Enderton, p.~22.

\vskip 0.5ex
\setbox\startprefix=\hbox{\tt \ \ df-ral\ \$a\ }
\setbox\contprefix=\hbox{\tt \ \ \ \ \ \ \ \ \ \ \ \ }
\startm
\m{\vdash}\m{(}\m{\forall}\m{x}\m{\in}\m{A}\m{\varphi}\m{\leftrightarrow}\m{%
\forall}\m{x}\m{(}\m{x}\m{\in}\m{A}\m{\rightarrow}\m{\varphi}\m{)}\m{)}
\endm
\begin{mmraw}%
|- ( A. x e. A ph <-> A. x ( x e. A -> ph ) ) \$.
\end{mmraw}

\noindent Define restricted existential quantification.\index{existential
quantifier ($\exists$)!restricted}  Enderton, p.~22.

\vskip 0.5ex
\setbox\startprefix=\hbox{\tt \ \ df-rex\ \$a\ }
\setbox\contprefix=\hbox{\tt \ \ \ \ \ \ \ \ \ \ \ \ }
\startm
\m{\vdash}\m{(}\m{\exists}\m{x}\m{\in}\m{A}\m{\varphi}\m{\leftrightarrow}\m{%
\exists}\m{x}\m{(}\m{x}\m{\in}\m{A}\m{\wedge}\m{\varphi}\m{)}\m{)}
\endm
\begin{mmraw}%
|- ( E. x e. A ph <-> E. x ( x e. A \TAND ph ) ) \$.
\end{mmraw}

\noindent Define the universal class\index{universal class ($V$)}.  Definition
5.20, p.~21, of Takeuti and Zaring.\label{df-v}

\vskip 0.5ex
\setbox\startprefix=\hbox{\tt \ \ df-v\ \$a\ }
\setbox\contprefix=\hbox{\tt \ \ \ \ \ \ \ \ \ \ }
\startm
\m{\vdash}\m{{\rm V}}\m{=}\m{\{}\m{x}\m{|}\m{x}\m{=}\m{x}\m{\}}
\endm
\begin{mmraw}%
|- {\char`\_}V = \{ x | x = x \} \$.
\end{mmraw}

\noindent Define the subclass\index{subclass}\index{subset} relationship
between two classes (called the subset relation if the classes are sets i.e.\
are not proper).  Definition 5.9 of Takeuti and Zaring, p.~17.\label{df-ss}

\vskip 0.5ex
\setbox\startprefix=\hbox{\tt \ \ df-ss\ \$a\ }
\setbox\contprefix=\hbox{\tt \ \ \ \ \ \ \ \ \ \ \ }
\startm
\m{\vdash}\m{(}\m{A}\m{\subseteq}\m{B}\m{\leftrightarrow}\m{\forall}\m{x}\m{(}
\m{x}\m{\in}\m{A}\m{\rightarrow}\m{x}\m{\in}\m{B}\m{)}\m{)}
\endm
\begin{mmraw}%
|- ( A C\_ B <-> A. x ( x e. A -> x e. B ) ) \$.
\end{mmraw}

\noindent Define the union\index{union} of two classes.  Definition 5.6 of Takeuti and Zaring,
p.~16.\label{df-un}

\vskip 0.5ex
\setbox\startprefix=\hbox{\tt \ \ df-un\ \$a\ }
\setbox\contprefix=\hbox{\tt \ \ \ \ \ \ \ \ \ \ \ }
\startm
\m{\vdash}\m{(}\m{A}\m{\cup}\m{B}\m{)}\m{=}\m{\{}\m{x}\m{|}\m{(}\m{x}\m{\in}
\m{A}\m{\vee}\m{x}\m{\in}\m{B}\m{)}\m{\}}
\endm
\begin{mmraw}%
( A u. B ) = \{ x | ( x e. A \TOR x e. B ) \} \$.
\end{mmraw}

\noindent Define the intersection\index{intersection} of two classes.  Definition 5.6 of
Takeuti and Zaring, p.~16.\label{df-in}

\vskip 0.5ex
\setbox\startprefix=\hbox{\tt \ \ df-in\ \$a\ }
\setbox\contprefix=\hbox{\tt \ \ \ \ \ \ \ \ \ \ \ }
\startm
\m{\vdash}\m{(}\m{A}\m{\cap}\m{B}\m{)}\m{=}\m{\{}\m{x}\m{|}\m{(}\m{x}\m{\in}
\m{A}\m{\wedge}\m{x}\m{\in}\m{B}\m{)}\m{\}}
\endm
% Caret ^ requires special treatment
\begin{mmraw}%
|- ( A i\^{}i B ) = \{ x | ( x e. A \TAND x e. B ) \} \$.
\end{mmraw}

\noindent Define class difference\index{class difference}\index{set difference}.
Definition 5.12 of Takeuti and Zaring, p.~20.  Several notations are used in
the literature; we chose the $\setminus$ convention instead of a minus sign to
reserve the latter for later use in, e.g., arithmetic.\label{df-dif}

\vskip 0.5ex
\setbox\startprefix=\hbox{\tt \ \ df-dif\ \$a\ }
\setbox\contprefix=\hbox{\tt \ \ \ \ \ \ \ \ \ \ \ \ }
\startm
\m{\vdash}\m{(}\m{A}\m{\setminus}\m{B}\m{)}\m{=}\m{\{}\m{x}\m{|}\m{(}\m{x}\m{
\in}\m{A}\m{\wedge}\m{\lnot}\m{x}\m{\in}\m{B}\m{)}\m{\}}
\endm
\begin{mmraw}%
( A \SLASH B ) = \{ x | ( x e. A \TAND -. x e. B ) \} \$.
\end{mmraw}

\noindent Define the empty or null set\index{empty set}\index{null set}.
Compare  Definition 5.14 of Takeuti and Zaring, p.~20.\label{df-nul}

\vskip 0.5ex
\setbox\startprefix=\hbox{\tt \ \ df-nul\ \$a\ }
\setbox\contprefix=\hbox{\tt \ \ \ \ \ \ \ \ \ \ }
\startm
\m{\vdash}\m{\varnothing}\m{=}\m{(}\m{{\rm V}}\m{\setminus}\m{{\rm V}}\m{)}
\endm
\begin{mmraw}%
|- (/) = ( {\char`\_}V \SLASH {\char`\_}V ) \$.
\end{mmraw}

\noindent Define power class\index{power set}\index{power class}.  Definition 5.10 of
Takeuti and Zaring, p.~17, but we also let it apply to proper classes.  (Note
that \verb$~P$ is the symbol for calligraphic P, the tilde
suggesting ``curly;'' see Appendix~\ref{ASCII}.)\label{df-pw}

\vskip 0.5ex
\setbox\startprefix=\hbox{\tt \ \ df-pw\ \$a\ }
\setbox\contprefix=\hbox{\tt \ \ \ \ \ \ \ \ \ \ \ }
\startm
\m{\vdash}\m{{\cal P}}\m{A}\m{=}\m{\{}\m{x}\m{|}\m{x}\m{\subseteq}\m{A}\m{\}}
\endm
% Special incantation required to put ~ into the text
\begin{mmraw}%
|- \char`\~P~A = \{ x | x C\_ A \} \$.
\end{mmraw}

\noindent Define the singleton of a class\index{singleton}.  Definition 7.1 of
Quine, p.~48.  It is well-defined for proper classes, although
it is not very meaningful in this case, where it evaluates to the empty
set.

\vskip 0.5ex
\setbox\startprefix=\hbox{\tt \ \ df-sn\ \$a\ }
\setbox\contprefix=\hbox{\tt \ \ \ \ \ \ \ \ \ \ \ }
\startm
\m{\vdash}\m{\{}\m{A}\m{\}}\m{=}\m{\{}\m{x}\m{|}\m{x}\m{=}\m{A}\m{\}}
\endm
\begin{mmraw}%
|- \{ A \} = \{ x | x = A \} \$.
\end{mmraw}%

\noindent Define an unordered pair of classes\index{unordered pair}\index{pair}.  Definition
7.1 of Quine, p.~48.

\vskip 0.5ex
\setbox\startprefix=\hbox{\tt \ \ df-pr\ \$a\ }
\setbox\contprefix=\hbox{\tt \ \ \ \ \ \ \ \ \ \ \ }
\startm
\m{\vdash}\m{\{}\m{A}\m{,}\m{B}\m{\}}\m{=}\m{(}\m{\{}\m{A}\m{\}}\m{\cup}\m{\{}
\m{B}\m{\}}\m{)}
\endm
\begin{mmraw}%
|- \{ A , B \} = ( \{ A \} u. \{ B \} ) \$.
\end{mmraw}

\noindent Define an unordered triple of classes\index{unordered triple}.  Definition of
Enderton, p.~19.

\vskip 0.5ex
\setbox\startprefix=\hbox{\tt \ \ df-tp\ \$a\ }
\setbox\contprefix=\hbox{\tt \ \ \ \ \ \ \ \ \ \ \ }
\startm
\m{\vdash}\m{\{}\m{A}\m{,}\m{B}\m{,}\m{C}\m{\}}\m{=}\m{(}\m{\{}\m{A}\m{,}\m{B}
\m{\}}\m{\cup}\m{\{}\m{C}\m{\}}\m{)}
\endm
\begin{mmraw}%
|- \{ A , B , C \} = ( \{ A , B \} u. \{ C \} ) \$.
\end{mmraw}%

\noindent Kuratowski's\index{Kuratowski, Kazimierz} ordered pair\index{ordered
pair} definition.  Definition 9.1 of Quine, p.~58. For proper classes it is
not meaningful but is well-defined for convenience.  (Note that \verb$<.$
stands for $\langle$ whereas \verb$<$ stands for $<$, and similarly for
\verb$>.$\,.)\label{df-op}

\vskip 0.5ex
\setbox\startprefix=\hbox{\tt \ \ df-op\ \$a\ }
\setbox\contprefix=\hbox{\tt \ \ \ \ \ \ \ \ \ \ \ }
\startm
\m{\vdash}\m{\langle}\m{A}\m{,}\m{B}\m{\rangle}\m{=}\m{\{}\m{\{}\m{A}\m{\}}
\m{,}\m{\{}\m{A}\m{,}\m{B}\m{\}}\m{\}}
\endm
\begin{mmraw}%
|- <. A , B >. = \{ \{ A \} , \{ A , B \} \} \$.
\end{mmraw}

\noindent Define the union of a class\index{union}.  Definition 5.5, p.~16,
of Takeuti and Zaring.

\vskip 0.5ex
\setbox\startprefix=\hbox{\tt \ \ df-uni\ \$a\ }
\setbox\contprefix=\hbox{\tt \ \ \ \ \ \ \ \ \ \ \ \ }
\startm
\m{\vdash}\m{\bigcup}\m{A}\m{=}\m{\{}\m{x}\m{|}\m{\exists}\m{y}\m{(}\m{x}\m{
\in}\m{y}\m{\wedge}\m{y}\m{\in}\m{A}\m{)}\m{\}}
\endm
\begin{mmraw}%
|- U. A = \{ x | E. y ( x e. y \TAND y e. A ) \} \$.
\end{mmraw}

\noindent Define the intersection\index{intersection} of a class.  Definition 7.35,
p.~44, of Takeuti and Zaring.

\vskip 0.5ex
\setbox\startprefix=\hbox{\tt \ \ df-int\ \$a\ }
\setbox\contprefix=\hbox{\tt \ \ \ \ \ \ \ \ \ \ \ \ }
\startm
\m{\vdash}\m{\bigcap}\m{A}\m{=}\m{\{}\m{x}\m{|}\m{\forall}\m{y}\m{(}\m{y}\m{
\in}\m{A}\m{\rightarrow}\m{x}\m{\in}\m{y}\m{)}\m{\}}
\endm
\begin{mmraw}%
|- |\^{}| A = \{ x | A. y ( y e. A -> x e. y ) \} \$.
\end{mmraw}

\noindent Define a transitive class\index{transitive class}\index{transitive
set}.  This should not be confused with a transitive relation, which is a different
concept.  Definition from p.~71 of Enderton, extended to classes.

\vskip 0.5ex
\setbox\startprefix=\hbox{\tt \ \ df-tr\ \$a\ }
\setbox\contprefix=\hbox{\tt \ \ \ \ \ \ \ \ \ \ \ }
\startm
\m{\vdash}\m{(}\m{\mbox{\rm Tr}}\m{A}\m{\leftrightarrow}\m{\bigcup}\m{A}\m{
\subseteq}\m{A}\m{)}
\endm
\begin{mmraw}%
|- ( Tr A <-> U. A C\_ A ) \$.
\end{mmraw}

\noindent Define a notation for a general binary relation\index{binary
relation}.  Definition 6.18, p.~29, of Takeuti and Zaring, generalized to
arbitrary classes.  This definition is well-defined, although not very
meaningful, when classes $A$ and/or $B$ are proper.\label{dfbr}  The lack of
parentheses (or any other connective) creates no ambiguity since we are defining
an atomic wff.

\vskip 0.5ex
\setbox\startprefix=\hbox{\tt \ \ df-br\ \$a\ }
\setbox\contprefix=\hbox{\tt \ \ \ \ \ \ \ \ \ \ \ }
\startm
\m{\vdash}\m{(}\m{A}\m{\,R}\m{\,B}\m{\leftrightarrow}\m{\langle}\m{A}\m{,}\m{B}
\m{\rangle}\m{\in}\m{R}\m{)}
\endm
\begin{mmraw}%
|- ( A R B <-> <. A , B >. e. R ) \$.
\end{mmraw}

\noindent Define an abstraction class of ordered pairs\index{abstraction
class!of ordered
pairs}.  A special case of Definition 4.16, p.~14, of Takeuti and Zaring.
Note that $ z $ must be distinct from $ x $ and $ y $,
and $ z $ must not occur in $\varphi$, but $ x $ and $ y $ may be
identical and may appear in $\varphi$.

\vskip 0.5ex
\setbox\startprefix=\hbox{\tt \ \ df-opab\ \$a\ }
\setbox\contprefix=\hbox{\tt \ \ \ \ \ \ \ \ \ \ \ \ \ }
\startm
\m{\vdash}\m{\{}\m{\langle}\m{x}\m{,}\m{y}\m{\rangle}\m{|}\m{\varphi}\m{\}}\m{=}
\m{\{}\m{z}\m{|}\m{\exists}\m{x}\m{\exists}\m{y}\m{(}\m{z}\m{=}\m{\langle}\m{x}
\m{,}\m{y}\m{\rangle}\m{\wedge}\m{\varphi}\m{)}\m{\}}
\endm

\begin{mmraw}%
|- \{ <. x , y >. | ph \} = \{ z | E. x E. y ( z =
<. x , y >. /\ ph ) \} \$.
\end{mmraw}

\noindent Define the epsilon relation\index{epsilon relation}.  Similar to Definition
6.22, p.~30, of Takeuti and Zaring.

\vskip 0.5ex
\setbox\startprefix=\hbox{\tt \ \ df-eprel\ \$a\ }
\setbox\contprefix=\hbox{\tt \ \ \ \ \ \ \ \ \ \ \ \ \ \ }
\startm
\m{\vdash}\m{{\rm E}}\m{=}\m{\{}\m{\langle}\m{x}\m{,}\m{y}\m{\rangle}\m{|}\m{x}\m{
\in}\m{y}\m{\}}
\endm
\begin{mmraw}%
|- \_E = \{ <. x , y >. | x e. y \} \$.
\end{mmraw}

\noindent Define a founded relation\index{founded relation}.  $R$ is a founded
relation on $A$ iff\index{iff} (if and only if) each nonempty subset of $A$
has an ``$R$-minimal element.''  Similar to Definition 6.21, p.~30, of
Takeuti and Zaring.

\vskip 0.5ex
\setbox\startprefix=\hbox{\tt \ \ df-fr\ \$a\ }
\setbox\contprefix=\hbox{\tt \ \ \ \ \ \ \ \ \ \ \ }
\startm
\m{\vdash}\m{(}\m{R}\m{\,\mbox{\rm Fr}}\m{\,A}\m{\leftrightarrow}\m{\forall}\m{x}
\m{(}\m{(}\m{x}\m{\subseteq}\m{A}\m{\wedge}\m{\lnot}\m{x}\m{=}\m{\varnothing}
\m{)}\m{\rightarrow}\m{\exists}\m{y}\m{(}\m{y}\m{\in}\m{x}\m{\wedge}\m{(}\m{x}
\m{\cap}\m{\{}\m{z}\m{|}\m{z}\m{\,R}\m{\,y}\m{\}}\m{)}\m{=}\m{\varnothing}\m{)}
\m{)}\m{)}
\endm
\begin{mmraw}%
|- ( R Fr A <-> A. x ( ( x C\_ A \TAND -. x = (/) ) ->
E. y ( y e. x \TAND ( x i\^{}i \{ z | z R y \} ) = (/) ) ) ) \$.
\end{mmraw}

\noindent Define a well-ordering\index{well-ordering}.  $R$ is a well-ordering of $A$ iff
it is founded on $A$ and the elements of $A$ are pairwise $R$-comparable.
Similar to Definition 6.24(2), p.~30, of Takeuti and Zaring.

\vskip 0.5ex
\setbox\startprefix=\hbox{\tt \ \ df-we\ \$a\ }
\setbox\contprefix=\hbox{\tt \ \ \ \ \ \ \ \ \ \ \ }
\startm
\m{\vdash}\m{(}\m{R}\m{\,\mbox{\rm We}}\m{\,A}\m{\leftrightarrow}\m{(}\m{R}\m{\,
\mbox{\rm Fr}}\m{\,A}\m{\wedge}\m{\forall}\m{x}\m{\forall}\m{y}\m{(}\m{(}\m{x}\m{
\in}\m{A}\m{\wedge}\m{y}\m{\in}\m{A}\m{)}\m{\rightarrow}\m{(}\m{x}\m{\,R}\m{\,y}
\m{\vee}\m{x}\m{=}\m{y}\m{\vee}\m{y}\m{\,R}\m{\,x}\m{)}\m{)}\m{)}\m{)}
\endm
\begin{mmraw}%
( R We A <-> ( R Fr A \TAND A. x A. y ( ( x e.
A \TAND y e. A ) -> ( x R y \TOR x = y \TOR y R x ) ) ) ) \$.
\end{mmraw}

\noindent Define the ordinal predicate\index{ordinal predicate}, which is true for a class
that is transitive and is well-ordered by the epsilon relation.  Similar to
definition on p.~468, Bell and Machover.

\vskip 0.5ex
\setbox\startprefix=\hbox{\tt \ \ df-ord\ \$a\ }
\setbox\contprefix=\hbox{\tt \ \ \ \ \ \ \ \ \ \ \ \ }
\startm
\m{\vdash}\m{(}\m{\mbox{\rm Ord}}\m{\,A}\m{\leftrightarrow}\m{(}
\m{\mbox{\rm Tr}}\m{\,A}\m{\wedge}\m{E}\m{\,\mbox{\rm We}}\m{\,A}\m{)}\m{)}
\endm
\begin{mmraw}%
|- ( Ord A <-> ( Tr A \TAND E We A ) ) \$.
\end{mmraw}

\noindent Define the class of all ordinal numbers\index{ordinal number}.  An ordinal number is
a set that satisfies the ordinal predicate.  Definition 7.11 of Takeuti and
Zaring, p.~38.

\vskip 0.5ex
\setbox\startprefix=\hbox{\tt \ \ df-on\ \$a\ }
\setbox\contprefix=\hbox{\tt \ \ \ \ \ \ \ \ \ \ \ }
\startm
\m{\vdash}\m{\,\mbox{\rm On}}\m{=}\m{\{}\m{x}\m{|}\m{\mbox{\rm Ord}}\m{\,x}
\m{\}}
\endm
\begin{mmraw}%
|- On = \{ x | Ord x \} \$.
\end{mmraw}

\noindent Define the limit ordinal predicate\index{limit ordinal}, which is true for a
non-empty ordinal that is not a successor (i.e.\ that is the union of itself).
Compare Bell and Machover, p.~471 and Exercise (1), p.~42 of Takeuti and
Zaring.

\vskip 0.5ex
\setbox\startprefix=\hbox{\tt \ \ df-lim\ \$a\ }
\setbox\contprefix=\hbox{\tt \ \ \ \ \ \ \ \ \ \ \ \ }
\startm
\m{\vdash}\m{(}\m{\mbox{\rm Lim}}\m{\,A}\m{\leftrightarrow}\m{(}\m{\mbox{
\rm Ord}}\m{\,A}\m{\wedge}\m{\lnot}\m{A}\m{=}\m{\varnothing}\m{\wedge}\m{A}
\m{=}\m{\bigcup}\m{A}\m{)}\m{)}
\endm
\begin{mmraw}%
|- ( Lim A <-> ( Ord A \TAND -. A = (/) \TAND A = U. A ) ) \$.
\end{mmraw}

\noindent Define the successor\index{successor} of a class.  Definition 7.22 of Takeuti
and Zaring, p.~41.  Our definition is a generalization to classes, although it
is meaningless when classes are proper.

\vskip 0.5ex
\setbox\startprefix=\hbox{\tt \ \ df-suc\ \$a\ }
\setbox\contprefix=\hbox{\tt \ \ \ \ \ \ \ \ \ \ \ \ }
\startm
\m{\vdash}\m{\,\mbox{\rm suc}}\m{\,A}\m{=}\m{(}\m{A}\m{\cup}\m{\{}\m{A}\m{\}}
\m{)}
\endm
\begin{mmraw}%
|- suc A = ( A u. \{ A \} ) \$.
\end{mmraw}

\noindent Define the class of natural numbers\index{natural number}\index{omega
($\omega$)}.  Compare Bell and Machover, p.~471.\label{dfom}

\vskip 0.5ex
\setbox\startprefix=\hbox{\tt \ \ df-om\ \$a\ }
\setbox\contprefix=\hbox{\tt \ \ \ \ \ \ \ \ \ \ \ }
\startm
\m{\vdash}\m{\omega}\m{=}\m{\{}\m{x}\m{|}\m{(}\m{\mbox{\rm Ord}}\m{\,x}\m{
\wedge}\m{\forall}\m{y}\m{(}\m{\mbox{\rm Lim}}\m{\,y}\m{\rightarrow}\m{x}\m{
\in}\m{y}\m{)}\m{)}\m{\}}
\endm
\begin{mmraw}%
|- om = \{ x | ( Ord x \TAND A. y ( Lim y -> x e. y ) ) \} \$.
\end{mmraw}

\noindent Define the Cartesian product (also called the
cross product)\index{Cartesian product}\index{cross product}
of two classes.  Definition 9.11 of Quine, p.~64.

\vskip 0.5ex
\setbox\startprefix=\hbox{\tt \ \ df-xp\ \$a\ }
\setbox\contprefix=\hbox{\tt \ \ \ \ \ \ \ \ \ \ \ }
\startm
\m{\vdash}\m{(}\m{A}\m{\times}\m{B}\m{)}\m{=}\m{\{}\m{\langle}\m{x}\m{,}\m{y}
\m{\rangle}\m{|}\m{(}\m{x}\m{\in}\m{A}\m{\wedge}\m{y}\m{\in}\m{B}\m{)}\m{\}}
\endm
\begin{mmraw}%
|- ( A X. B ) = \{ <. x , y >. | ( x e. A \TAND y e. B) \} \$.
\end{mmraw}

\noindent Define a relation\index{relation}.  Definition 6.4(1) of Takeuti and
Zaring, p.~23.

\vskip 0.5ex
\setbox\startprefix=\hbox{\tt \ \ df-rel\ \$a\ }
\setbox\contprefix=\hbox{\tt \ \ \ \ \ \ \ \ \ \ \ \ }
\startm
\m{\vdash}\m{(}\m{\mbox{\rm Rel}}\m{\,A}\m{\leftrightarrow}\m{A}\m{\subseteq}
\m{(}\m{{\rm V}}\m{\times}\m{{\rm V}}\m{)}\m{)}
\endm
\begin{mmraw}%
|- ( Rel A <-> A C\_ ( {\char`\_}V X. {\char`\_}V ) ) \$.
\end{mmraw}

\noindent Define the domain\index{domain} of a class.  Definition 6.5(1) of
Takeuti and Zaring, p.~24.

\vskip 0.5ex
\setbox\startprefix=\hbox{\tt \ \ df-dm\ \$a\ }
\setbox\contprefix=\hbox{\tt \ \ \ \ \ \ \ \ \ \ \ }
\startm
\m{\vdash}\m{\,\mbox{\rm dom}}\m{A}\m{=}\m{\{}\m{x}\m{|}\m{\exists}\m{y}\m{
\langle}\m{x}\m{,}\m{y}\m{\rangle}\m{\in}\m{A}\m{\}}
\endm
\begin{mmraw}%
|- dom A = \{ x | E. y <. x , y >. e. A \} \$.
\end{mmraw}

\noindent Define the range\index{range} of a class.  Definition 6.5(2) of
Takeuti and Zaring, p.~24.

\vskip 0.5ex
\setbox\startprefix=\hbox{\tt \ \ df-rn\ \$a\ }
\setbox\contprefix=\hbox{\tt \ \ \ \ \ \ \ \ \ \ \ }
\startm
\m{\vdash}\m{\,\mbox{\rm ran}}\m{A}\m{=}\m{\{}\m{y}\m{|}\m{\exists}\m{x}\m{
\langle}\m{x}\m{,}\m{y}\m{\rangle}\m{\in}\m{A}\m{\}}
\endm
\begin{mmraw}%
|- ran A = \{ y | E. x <. x , y >. e. A \} \$.
\end{mmraw}

\noindent Define the restriction\index{restriction} of a class.  Definition
6.6(1) of Takeuti and Zaring, p.~24.

\vskip 0.5ex
\setbox\startprefix=\hbox{\tt \ \ df-res\ \$a\ }
\setbox\contprefix=\hbox{\tt \ \ \ \ \ \ \ \ \ \ \ \ }
\startm
\m{\vdash}\m{(}\m{A}\m{\restriction}\m{B}\m{)}\m{=}\m{(}\m{A}\m{\cap}\m{(}\m{B}
\m{\times}\m{{\rm V}}\m{)}\m{)}
\endm
\begin{mmraw}%
|- ( A |` B ) = ( A i\^{}i ( B X. {\char`\_}V ) ) \$.
\end{mmraw}

\noindent Define the image\index{image} of a class.  Definition 6.6(2) of
Takeuti and Zaring, p.~24.

\vskip 0.5ex
\setbox\startprefix=\hbox{\tt \ \ df-ima\ \$a\ }
\setbox\contprefix=\hbox{\tt \ \ \ \ \ \ \ \ \ \ \ \ }
\startm
\m{\vdash}\m{(}\m{A}\m{``}\m{B}\m{)}\m{=}\m{\,\mbox{\rm ran}}\m{\,(}\m{A}\m{
\restriction}\m{B}\m{)}
\endm
\begin{mmraw}%
|- ( A " B ) = ran ( A |` B ) \$.
\end{mmraw}

\noindent Define the composition\index{composition} of two classes.  Definition 6.6(3) of
Takeuti and Zaring, p.~24.

\vskip 0.5ex
\setbox\startprefix=\hbox{\tt \ \ df-co\ \$a\ }
\setbox\contprefix=\hbox{\tt \ \ \ \ \ \ \ \ \ \ \ \ }
\startm
\m{\vdash}\m{(}\m{A}\m{\circ}\m{B}\m{)}\m{=}\m{\{}\m{\langle}\m{x}\m{,}\m{y}\m{
\rangle}\m{|}\m{\exists}\m{z}\m{(}\m{\langle}\m{x}\m{,}\m{z}\m{\rangle}\m{\in}
\m{B}\m{\wedge}\m{\langle}\m{z}\m{,}\m{y}\m{\rangle}\m{\in}\m{A}\m{)}\m{\}}
\endm
\begin{mmraw}%
|- ( A o. B ) = \{ <. x , y >. | E. z ( <. x , z
>. e. B \TAND <. z , y >. e. A ) \} \$.
\end{mmraw}

\noindent Define a function\index{function}.  Definition 6.4(4) of Takeuti and
Zaring, p.~24.

\vskip 0.5ex
\setbox\startprefix=\hbox{\tt \ \ df-fun\ \$a\ }
\setbox\contprefix=\hbox{\tt \ \ \ \ \ \ \ \ \ \ \ \ }
\startm
\m{\vdash}\m{(}\m{\mbox{\rm Fun}}\m{\,A}\m{\leftrightarrow}\m{(}
\m{\mbox{\rm Rel}}\m{\,A}\m{\wedge}
\m{\forall}\m{x}\m{\exists}\m{z}\m{\forall}\m{y}\m{(}
\m{\langle}\m{x}\m{,}\m{y}\m{\rangle}\m{\in}\m{A}\m{\rightarrow}\m{y}\m{=}\m{z}
\m{)}\m{)}\m{)}
\endm
\begin{mmraw}%
|- ( Fun A <-> ( Rel A /\ A. x E. z A. y ( <. x
   , y >. e. A -> y = z ) ) ) \$.
\end{mmraw}

\noindent Define a function with domain.  Definition 6.15(1) of Takeuti and
Zaring, p.~27.

\vskip 0.5ex
\setbox\startprefix=\hbox{\tt \ \ df-fn\ \$a\ }
\setbox\contprefix=\hbox{\tt \ \ \ \ \ \ \ \ \ \ \ }
\startm
\m{\vdash}\m{(}\m{A}\m{\,\mbox{\rm Fn}}\m{\,B}\m{\leftrightarrow}\m{(}
\m{\mbox{\rm Fun}}\m{\,A}\m{\wedge}\m{\mbox{\rm dom}}\m{\,A}\m{=}\m{B}\m{)}
\m{)}
\endm
\begin{mmraw}%
|- ( A Fn B <-> ( Fun A \TAND dom A = B ) ) \$.
\end{mmraw}

\noindent Define a function with domain and co-domain.  Definition 6.15(3)
of Takeuti and Zaring, p.~27.

\vskip 0.5ex
\setbox\startprefix=\hbox{\tt \ \ df-f\ \$a\ }
\setbox\contprefix=\hbox{\tt \ \ \ \ \ \ \ \ \ \ }
\startm
\m{\vdash}\m{(}\m{F}\m{:}\m{A}\m{\longrightarrow}\m{B}\m{
\leftrightarrow}\m{(}\m{F}\m{\,\mbox{\rm Fn}}\m{\,A}\m{\wedge}\m{
\mbox{\rm ran}}\m{\,F}\m{\subseteq}\m{B}\m{)}\m{)}
\endm
\begin{mmraw}%
|- ( F : A --> B <-> ( F Fn A \TAND ran F C\_ B ) ) \$.
\end{mmraw}

\noindent Define a one-to-one function\index{one-to-one function}.  Compare
Definition 6.15(5) of Takeuti and Zaring, p.~27.

\vskip 0.5ex
\setbox\startprefix=\hbox{\tt \ \ df-f1\ \$a\ }
\setbox\contprefix=\hbox{\tt \ \ \ \ \ \ \ \ \ \ \ }
\startm
\m{\vdash}\m{(}\m{F}\m{:}\m{A}\m{
\raisebox{.5ex}{${\textstyle{\:}_{\mbox{\footnotesize\rm
1\tt -\rm 1}}}\atop{\textstyle{
\longrightarrow}\atop{\textstyle{}^{\mbox{\footnotesize\rm {\ }}}}}$}
}\m{B}
\m{\leftrightarrow}\m{(}\m{F}\m{:}\m{A}\m{\longrightarrow}\m{B}
\m{\wedge}\m{\forall}\m{y}\m{\exists}\m{z}\m{\forall}\m{x}\m{(}\m{\langle}\m{x}
\m{,}\m{y}\m{\rangle}\m{\in}\m{F}\m{\rightarrow}\m{x}\m{=}\m{z}\m{)}\m{)}\m{)}
\endm
\begin{mmraw}%
|- ( F : A -1-1-> B <-> ( F : A --> B \TAND
   A. y E. z A. x ( <. x , y >. e. F -> x = z ) ) ) \$.
\end{mmraw}

\noindent Define an onto function\index{onto function}.  Definition 6.15(4) of Takeuti and
Zaring, p.~27.

\vskip 0.5ex
\setbox\startprefix=\hbox{\tt \ \ df-fo\ \$a\ }
\setbox\contprefix=\hbox{\tt \ \ \ \ \ \ \ \ \ \ \ }
\startm
\m{\vdash}\m{(}\m{F}\m{:}\m{A}\m{
\raisebox{.5ex}{${\textstyle{\:}_{\mbox{\footnotesize\rm
{\ }}}}\atop{\textstyle{
\longrightarrow}\atop{\textstyle{}^{\mbox{\footnotesize\rm onto}}}}$}
}\m{B}
\m{\leftrightarrow}\m{(}\m{F}\m{\,\mbox{\rm Fn}}\m{\,A}\m{\wedge}
\m{\mbox{\rm ran}}\m{\,F}\m{=}\m{B}\m{)}\m{)}
\endm
\begin{mmraw}%
|- ( F : A -onto-> B <-> ( F Fn A /\ ran F = B ) ) \$.
\end{mmraw}

\noindent Define a one-to-one, onto function.  Compare Definition 6.15(6) of
Takeuti and Zaring, p.~27.

\vskip 0.5ex
\setbox\startprefix=\hbox{\tt \ \ df-f1o\ \$a\ }
\setbox\contprefix=\hbox{\tt \ \ \ \ \ \ \ \ \ \ \ \ }
\startm
\m{\vdash}\m{(}\m{F}\m{:}\m{A}
\m{
\raisebox{.5ex}{${\textstyle{\:}_{\mbox{\footnotesize\rm
1\tt -\rm 1}}}\atop{\textstyle{
\longrightarrow}\atop{\textstyle{}^{\mbox{\footnotesize\rm onto}}}}$}
}
\m{B}
\m{\leftrightarrow}\m{(}\m{F}\m{:}\m{A}
\m{
\raisebox{.5ex}{${\textstyle{\:}_{\mbox{\footnotesize\rm
1\tt -\rm 1}}}\atop{\textstyle{
\longrightarrow}\atop{\textstyle{}^{\mbox{\footnotesize\rm {\ }}}}}$}
}
\m{B}\m{\wedge}\m{F}\m{:}\m{A}
\m{
\raisebox{.5ex}{${\textstyle{\:}_{\mbox{\footnotesize\rm
{\ }}}}\atop{\textstyle{
\longrightarrow}\atop{\textstyle{}^{\mbox{\footnotesize\rm onto}}}}$}
}
\m{B}\m{)}\m{)}
\endm
\begin{mmraw}%
|- ( F : A -1-1-onto-> B <-> ( F : A -1-1-> B? \TAND F : A -onto-> B ) ) \$.?
\end{mmraw}

\noindent Define the value of a function\index{function value}.  This
definition applies to any class and evaluates to the empty set when it is not
meaningful. Note that $ F`A$ means the same thing as the more familiar $ F(A)$
notation for a function's value at $A$.  The $ F`A$ notation is common in
formal set theory.\label{df-fv}

\vskip 0.5ex
\setbox\startprefix=\hbox{\tt \ \ df-fv\ \$a\ }
\setbox\contprefix=\hbox{\tt \ \ \ \ \ \ \ \ \ \ \ }
\startm
\m{\vdash}\m{(}\m{F}\m{`}\m{A}\m{)}\m{=}\m{\bigcup}\m{\{}\m{x}\m{|}\m{(}\m{F}%
\m{``}\m{\{}\m{A}\m{\}}\m{)}\m{=}\m{\{}\m{x}\m{\}}\m{\}}
\endm
\begin{mmraw}%
|- ( F ` A ) = U. \{ x | ( F " \{ A \} ) = \{ x \} \} \$.
\end{mmraw}

\noindent Define the result of an operation.\index{operation}  Here, $F$ is
     an operation on two
     values (such as $+$ for real numbers).   This is defined for proper
     classes $A$ and $B$ even though not meaningful in that case.  However,
     the definition can be meaningful when $F$ is a proper class.\label{dfopr}

\vskip 0.5ex
\setbox\startprefix=\hbox{\tt \ \ df-opr\ \$a\ }
\setbox\contprefix=\hbox{\tt \ \ \ \ \ \ \ \ \ \ \ \ }
\startm
\m{\vdash}\m{(}\m{A}\m{\,F}\m{\,B}\m{)}\m{=}\m{(}\m{F}\m{`}\m{\langle}\m{A}%
\m{,}\m{B}\m{\rangle}\m{)}
\endm
\begin{mmraw}%
|- ( A F B ) = ( F ` <. A , B >. ) \$.
\end{mmraw}

\section{Tricks of the Trade}\label{tricks}

In the \texttt{set.mm}\index{set theory database (\texttt{set.mm})} database our goal
was usually to conform to modern notation.  However in some cases the
relationship to standard textbook language may be obscured by several
unconventional devices we used to simplify the development and to take
advantage of the Metamath language.  In this section we will describe some
common conventions used in \texttt{set.mm}.

\begin{itemize}
\item
The turnstile\index{turnstile ({$\,\vdash$})} symbol, $\vdash$, meaning ``it
is provable that,'' is the first token of all assertions and hypotheses that
aren't syntax constructions.  This is a standard convention in logic.  (We
mentioned this earlier, but this symbol is bothersome to some people without a
logic background.  It has no deeper meaning but just provides us with a way to
distinguish syntax constructions from ordinary mathematical statements.)

\item
A hypothesis of the form

\vskip 1ex
\setbox\startprefix=\hbox{\tt \ \ \ \ \ \ \ \ \ \$e\ }
\setbox\contprefix=\hbox{\tt \ \ \ \ \ \ \ \ \ \ }
\startm
\m{\vdash}\m{(}\m{\varphi}\m{\rightarrow}\m{\forall}\m{x}\m{\varphi}\m{)}
\endm
\vskip 1ex

should be read ``assume variable $x$ is (effectively) not free in wff
$\varphi$.''\index{effectively not free}
Literally, this says ``assume it is provable that $\varphi \rightarrow \forall
x\, \varphi$.''  This device lets us avoid the complexities associated with
the standard treatment of free and bound variables.
%% Uncomment this when uncommenting section {formalspec} below
The footnote on p.~\pageref{effectivelybound} discusses this further.

\item
A statement of one of the forms

\vskip 1ex
\setbox\startprefix=\hbox{\tt \ \ \ \ \ \ \ \ \ \$a\ }
\setbox\contprefix=\hbox{\tt \ \ \ \ \ \ \ \ \ \ }
\startm
\m{\vdash}\m{(}\m{\lnot}\m{\forall}\m{x}\m{\,x}\m{=}\m{y}\m{\rightarrow}
\m{\ldots}\m{)}
\endm
\setbox\startprefix=\hbox{\tt \ \ \ \ \ \ \ \ \ \$p\ }
\setbox\contprefix=\hbox{\tt \ \ \ \ \ \ \ \ \ \ }
\startm
\m{\vdash}\m{(}\m{\lnot}\m{\forall}\m{x}\m{\,x}\m{=}\m{y}\m{\rightarrow}
\m{\ldots}\m{)}
\endm
\vskip 1ex

should be read ``if $x$ and $y$ are distinct variables, then...''  This
antecedent provides us with a technical device to avoid the need for the
\texttt{\$d} statement early in our development of predicate calculus,
permitting symbol manipulations to be as conceptually simple as those in
propositional calculus.  However, the \texttt{\$d} statement eventually
becomes a requirement, and after that this device is rarely used.

\item
The statement

\vskip 1ex
\setbox\startprefix=\hbox{\tt \ \ \ \ \ \ \ \ \ \$d\ }
\setbox\contprefix=\hbox{\tt \ \ \ \ \ \ \ \ \ \ }
\startm
\m{x}\m{\,y}
\endm
\vskip 1ex

should be read ``assume $x$ and $y$ are distinct variables.''

\item
The statement

\vskip 1ex
\setbox\startprefix=\hbox{\tt \ \ \ \ \ \ \ \ \ \$d\ }
\setbox\contprefix=\hbox{\tt \ \ \ \ \ \ \ \ \ \ }
\startm
\m{x}\m{\,\varphi}
\endm
\vskip 1ex

should be read ``assume $x$ does not occur in $\varphi$.''

\item
The statement

\vskip 1ex
\setbox\startprefix=\hbox{\tt \ \ \ \ \ \ \ \ \ \$d\ }
\setbox\contprefix=\hbox{\tt \ \ \ \ \ \ \ \ \ \ }
\startm
\m{x}\m{\,A}
\endm
\vskip 1ex

should be read ``assume variable $x$ does not occur in class $A$.''

\item
The restriction and hypothesis group

\vskip 1ex
\setbox\startprefix=\hbox{\tt \ \ \ \ \ \ \ \ \ \$d\ }
\setbox\contprefix=\hbox{\tt \ \ \ \ \ \ \ \ \ \ }
\startm
\m{x}\m{\,A}
\endm
\setbox\startprefix=\hbox{\tt \ \ \ \ \ \ \ \ \ \$d\ }
\setbox\contprefix=\hbox{\tt \ \ \ \ \ \ \ \ \ \ }
\startm
\m{x}\m{\,\psi}
\endm
\setbox\startprefix=\hbox{\tt \ \ \ \ \ \ \ \ \ \$e\ }
\setbox\contprefix=\hbox{\tt \ \ \ \ \ \ \ \ \ \ }
\startm
\m{\vdash}\m{(}\m{x}\m{=}\m{A}\m{\rightarrow}\m{(}\m{\varphi}\m{\leftrightarrow}
\m{\psi}\m{)}\m{)}
\endm
\vskip 1ex

is frequently used in place of explicit substitution, meaning ``assume
$\psi$ results from the proper substitution of $A$ for $x$ in
$\varphi$.''  Sometimes ``\texttt{\$e} $\vdash ( \psi \rightarrow
\forall x \, \psi )$'' is used instead of ``\texttt{\$d} $x\, \psi $,''
which requires only that $x$ be effectively not free in $\varphi$ but
not necessarily absent from it.  The use of implicit
substitution\index{substitution!implicit} is partly a matter of personal
style, although it may make proofs somewhat shorter than would be the
case with explicit substitution.

\item
The hypothesis


\vskip 1ex
\setbox\startprefix=\hbox{\tt \ \ \ \ \ \ \ \ \ \$e\ }
\setbox\contprefix=\hbox{\tt \ \ \ \ \ \ \ \ \ \ }
\startm
\m{\vdash}\m{A}\m{\in}\m{{\rm V}}
\endm
\vskip 1ex

should be read ``assume class $A$ is a set (i.e.\ exists).''
This is a convenient convention used by Quine.

\item
The restriction and hypothesis

\vskip 1ex
\setbox\startprefix=\hbox{\tt \ \ \ \ \ \ \ \ \ \$d\ }
\setbox\contprefix=\hbox{\tt \ \ \ \ \ \ \ \ \ \ }
\startm
\m{x}\m{\,y}
\endm
\setbox\startprefix=\hbox{\tt \ \ \ \ \ \ \ \ \ \$e\ }
\setbox\contprefix=\hbox{\tt \ \ \ \ \ \ \ \ \ \ }
\startm
\m{\vdash}\m{(}\m{y}\m{\in}\m{A}\m{\rightarrow}\m{\forall}\m{x}\m{\,y}
\m{\in}\m{A}\m{)}
\endm
\vskip 1ex

should be read ``assume variable $x$ is
(effectively) not free in class $A$.''

\end{itemize}

\section{A Theorem Sampler}\label{sometheorems}

In this section we list some of the more important theorems that are proved in
the \texttt{set.mm} database, and they illustrate the kinds of things that can be
done with Metamath.  While all of these facts are well-known results,
Metamath offers the advantage of easily allowing you to trace their
derivation back to axioms.  Our intent here is not to try to explain the
details or motivation; for this we refer you to the textbooks that are
mentioned in the descriptions.  (The \texttt{set.mm} file has bibliographic
references for the text references.)  Their proofs often embody important
concepts you may wish to explore with the Metamath program (see
Section~\ref{exploring}).  All the symbols that are used here are defined in
Section~\ref{hierarchy}.  For brevity we haven't included the \texttt{\$d}
restrictions or \texttt{\$f} hypotheses for these theorems; when you are
uncertain consult the \texttt{set.mm} database.

We start with \texttt{syl} (principle of the syllogism).
In \textit{Principia Mathematica}
Whitehead and Russell call this ``the principle of the
syllogism... because... the syllogism in Barbara is derived from them''
\cite[quote after Theorem *2.06 p.~101]{PM}.
Some authors call this law a ``hypothetical syllogism.''
As of 2019 \texttt{syl} is the most commonly referenced proven
assertion in the \texttt{set.mm} database.\footnote{
The Metamath program command \texttt{show usage}
shows the number of references.
On 2019-04-29 (commit 71cbbbdb387e) \texttt{syl} was directly referenced
10,819 times. The second most commonly referenced proven assertion
was \texttt{eqid}, which was directly referenced 7,738 times.
}

\vskip 2ex
\noindent Theorem syl (principle of the syllogism)\index{Syllogism}%
\index{\texttt{syl}}\label{syl}.

\vskip 0.5ex
\setbox\startprefix=\hbox{\tt \ \ syl.1\ \$e\ }
\setbox\contprefix=\hbox{\tt \ \ \ \ \ \ \ \ \ \ \ }
\startm
\m{\vdash}\m{(}\m{\varphi}\m{ \rightarrow }\m{\psi}\m{)}
\endm
\setbox\startprefix=\hbox{\tt \ \ syl.2\ \$e\ }
\setbox\contprefix=\hbox{\tt \ \ \ \ \ \ \ \ \ \ \ }
\startm
\m{\vdash}\m{(}\m{\psi}\m{ \rightarrow }\m{\chi}\m{)}
\endm
\setbox\startprefix=\hbox{\tt \ \ syl\ \$p\ }
\setbox\contprefix=\hbox{\tt \ \ \ \ \ \ \ \ \ }
\startm
\m{\vdash}\m{(}\m{\varphi}\m{ \rightarrow }\m{\chi}\m{)}
\endm
\vskip 2ex

The following theorem is not very deep but provides us with a notational device
that is frequently used.  It allows us to use the expression ``$A \in V$'' as
a compact way of saying that class $A$ exists, i.e.\ is a set.

\vskip 2ex
\noindent Two ways to say ``$A$ is a set'':  $A$ is a member of the universe
$V$ if and only if $A$ exists (i.e.\ there exists a set equal to $A$).
Theorem 6.9 of Quine, p. 43.

\vskip 0.5ex
\setbox\startprefix=\hbox{\tt \ \ isset\ \$p\ }
\setbox\contprefix=\hbox{\tt \ \ \ \ \ \ \ \ \ \ \ }
\startm
\m{\vdash}\m{(}\m{A}\m{\in}\m{{\rm V}}\m{\leftrightarrow}\m{\exists}\m{x}\m{\,x}\m{=}
\m{A}\m{)}
\endm
\vskip 1ex

Next we prove the axioms of standard ZF set theory that were missing from our
axiom system.  From our point of view they are theorems since they
can be derived from the other axioms.

\vskip 2ex
\noindent Axiom of Separation\index{Axiom of Separation}
(Aussonderung)\index{Aussonderung} proved from the other axioms of ZF set
theory.  Compare Exercise 4 of Takeuti and Zaring, p.~22.

\vskip 0.5ex
\setbox\startprefix=\hbox{\tt \ \ inex1.1\ \$e\ }
\setbox\contprefix=\hbox{\tt \ \ \ \ \ \ \ \ \ \ \ \ \ \ \ }
\startm
\m{\vdash}\m{A}\m{\in}\m{{\rm V}}
\endm
\setbox\startprefix=\hbox{\tt \ \ inex\ \$p\ }
\setbox\contprefix=\hbox{\tt \ \ \ \ \ \ \ \ \ \ \ \ \ }
\startm
\m{\vdash}\m{(}\m{A}\m{\cap}\m{B}\m{)}\m{\in}\m{{\rm V}}
\endm
\vskip 1ex

\noindent Axiom of the Null Set\index{Axiom of the Null Set} proved from the
other axioms of ZF set theory. Corollary 5.16 of Takeuti and Zaring, p.~20.

\vskip 0.5ex
\setbox\startprefix=\hbox{\tt \ \ 0ex\ \$p\ }
\setbox\contprefix=\hbox{\tt \ \ \ \ \ \ \ \ \ \ \ \ }
\startm
\m{\vdash}\m{\varnothing}\m{\in}\m{{\rm V}}
\endm
\vskip 1ex

\noindent The Axiom of Pairing\index{Axiom of Pairing} proved from the other
axioms of ZF set theory.  Theorem 7.13 of Quine, p.~51.
\vskip 0.5ex
\setbox\startprefix=\hbox{\tt \ \ prex\ \$p\ }
\setbox\contprefix=\hbox{\tt \ \ \ \ \ \ \ \ \ \ \ \ \ \ }
\startm
\m{\vdash}\m{\{}\m{A}\m{,}\m{B}\m{\}}\m{\in}\m{{\rm V}}
\endm
\vskip 2ex

Next we will list some famous or important theorems that are proved in
the \texttt{set.mm} database.  None of them except \texttt{omex}
require the Axiom of Infinity, as you can verify with the \texttt{show
trace{\char`\_}back} Metamath command.

\vskip 2ex
\noindent The resolution of Russell's paradox\index{Russell's paradox}.  There
exists no set containing the set of all sets which are not members of
themselves.  Proposition 4.14 of Takeuti and Zaring, p.~14.

\vskip 0.5ex
\setbox\startprefix=\hbox{\tt \ \ ru\ \$p\ }
\setbox\contprefix=\hbox{\tt \ \ \ \ \ \ \ \ }
\startm
\m{\vdash}\m{\lnot}\m{\exists}\m{x}\m{\,x}\m{=}\m{\{}\m{y}\m{|}\m{\lnot}\m{y}
\m{\in}\m{y}\m{\}}
\endm
\vskip 1ex

\noindent Cantor's theorem\index{Cantor's theorem}.  No set can be mapped onto
its power set.  Compare Theorem 6B(b) of Enderton, p.~132.

\vskip 0.5ex
\setbox\startprefix=\hbox{\tt \ \ canth.1\ \$e\ }
\setbox\contprefix=\hbox{\tt \ \ \ \ \ \ \ \ \ \ \ \ \ }
\startm
\m{\vdash}\m{A}\m{\in}\m{{\rm V}}
\endm
\setbox\startprefix=\hbox{\tt \ \ canth\ \$p\ }
\setbox\contprefix=\hbox{\tt \ \ \ \ \ \ \ \ \ \ \ }
\startm
\m{\vdash}\m{\lnot}\m{F}\m{:}\m{A}\m{\raisebox{.5ex}{${\textstyle{\:}_{
\mbox{\footnotesize\rm {\ }}}}\atop{\textstyle{\longrightarrow}\atop{
\textstyle{}^{\mbox{\footnotesize\rm onto}}}}$}}\m{{\cal P}}\m{A}
\endm
\vskip 1ex

\noindent The Burali-Forti paradox\index{Burali-Forti paradox}.  No set
contains all ordinal numbers. Enderton, p.~194.  (Burali-Forti was one person,
not two.)

\vskip 0.5ex
\setbox\startprefix=\hbox{\tt \ \ onprc\ \$p\ }
\setbox\contprefix=\hbox{\tt \ \ \ \ \ \ \ \ \ \ \ \ }
\startm
\m{\vdash}\m{\lnot}\m{\mbox{\rm On}}\m{\in}\m{{\rm V}}
\endm
\vskip 1ex

\noindent Peano's postulates\index{Peano's postulates} for arithmetic.
Proposition 7.30 of Takeuti and Zaring, pp.~42--43.  The objects being
described are the members of $\omega$ i.e.\ the natural numbers 0, 1,
2,\ldots.  The successor\index{successor} operation suc means ``plus
one.''  \texttt{peano1} says that 0 (which is defined as the empty set)
is a natural number.  \texttt{peano2} says that if $A$ is a natural
number, so is $A+1$.  \texttt{peano3} says that 0 is not the successor
of any natural number.  \texttt{peano4} says that two natural numbers
are equal if and only if their successors are equal.  \texttt{peano5} is
essentially the same as mathematical induction.

\vskip 1ex
\setbox\startprefix=\hbox{\tt \ \ peano1\ \$p\ }
\setbox\contprefix=\hbox{\tt \ \ \ \ \ \ \ \ \ \ \ \ }
\startm
\m{\vdash}\m{\varnothing}\m{\in}\m{\omega}
\endm
\vskip 1.5ex

\setbox\startprefix=\hbox{\tt \ \ peano2\ \$p\ }
\setbox\contprefix=\hbox{\tt \ \ \ \ \ \ \ \ \ \ \ \ }
\startm
\m{\vdash}\m{(}\m{A}\m{\in}\m{\omega}\m{\rightarrow}\m{{\rm suc}}\m{A}\m{\in}%
\m{\omega}\m{)}
\endm
\vskip 1.5ex

\setbox\startprefix=\hbox{\tt \ \ peano3\ \$p\ }
\setbox\contprefix=\hbox{\tt \ \ \ \ \ \ \ \ \ \ \ \ }
\startm
\m{\vdash}\m{(}\m{A}\m{\in}\m{\omega}\m{\rightarrow}\m{\lnot}\m{{\rm suc}}%
\m{A}\m{=}\m{\varnothing}\m{)}
\endm
\vskip 1.5ex

\setbox\startprefix=\hbox{\tt \ \ peano4\ \$p\ }
\setbox\contprefix=\hbox{\tt \ \ \ \ \ \ \ \ \ \ \ \ }
\startm
\m{\vdash}\m{(}\m{(}\m{A}\m{\in}\m{\omega}\m{\wedge}\m{B}\m{\in}\m{\omega}%
\m{)}\m{\rightarrow}\m{(}\m{{\rm suc}}\m{A}\m{=}\m{{\rm suc}}\m{B}\m{%
\leftrightarrow}\m{A}\m{=}\m{B}\m{)}\m{)}
\endm
\vskip 1.5ex

\setbox\startprefix=\hbox{\tt \ \ peano5\ \$p\ }
\setbox\contprefix=\hbox{\tt \ \ \ \ \ \ \ \ \ \ \ \ }
\startm
\m{\vdash}\m{(}\m{(}\m{\varnothing}\m{\in}\m{A}\m{\wedge}\m{\forall}\m{x}\m{%
\in}\m{\omega}\m{(}\m{x}\m{\in}\m{A}\m{\rightarrow}\m{{\rm suc}}\m{x}\m{\in}%
\m{A}\m{)}\m{)}\m{\rightarrow}\m{\omega}\m{\subseteq}\m{A}\m{)}
\endm
\vskip 1.5ex

\noindent Finite Induction (mathematical induction).\index{finite
induction}\index{mathematical induction} The first hypothesis is the
basis and the second is the induction hypothesis.  Theorem Schema 22 of
Suppes, p.~136.

\vskip 0.5ex
\setbox\startprefix=\hbox{\tt \ \ findes.1\ \$e\ }
\setbox\contprefix=\hbox{\tt \ \ \ \ \ \ \ \ \ \ \ \ \ \ }
\startm
\m{\vdash}\m{[}\m{\varnothing}\m{/}\m{x}\m{]}\m{\varphi}
\endm
\setbox\startprefix=\hbox{\tt \ \ findes.2\ \$e\ }
\setbox\contprefix=\hbox{\tt \ \ \ \ \ \ \ \ \ \ \ \ \ \ }
\startm
\m{\vdash}\m{(}\m{x}\m{\in}\m{\omega}\m{\rightarrow}\m{(}\m{\varphi}\m{%
\rightarrow}\m{[}\m{{\rm suc}}\m{x}\m{/}\m{x}\m{]}\m{\varphi}\m{)}\m{)}
\endm
\setbox\startprefix=\hbox{\tt \ \ findes\ \$p\ }
\setbox\contprefix=\hbox{\tt \ \ \ \ \ \ \ \ \ \ \ \ }
\startm
\m{\vdash}\m{(}\m{x}\m{\in}\m{\omega}\m{\rightarrow}\m{\varphi}\m{)}
\endm
\vskip 1ex

\noindent Transfinite Induction with explicit substitution.  The first
hypothesis is the basis, the second is the induction hypothesis for
successors, and the third is the induction hypothesis for limit
ordinals.  Theorem Schema 4 of Suppes, p. 197.

\vskip 0.5ex
\setbox\startprefix=\hbox{\tt \ \ tfindes.1\ \$e\ }
\setbox\contprefix=\hbox{\tt \ \ \ \ \ \ \ \ \ \ \ \ \ \ \ }
\startm
\m{\vdash}\m{[}\m{\varnothing}\m{/}\m{x}\m{]}\m{\varphi}
\endm
\setbox\startprefix=\hbox{\tt \ \ tfindes.2\ \$e\ }
\setbox\contprefix=\hbox{\tt \ \ \ \ \ \ \ \ \ \ \ \ \ \ \ }
\startm
\m{\vdash}\m{(}\m{x}\m{\in}\m{{\rm On}}\m{\rightarrow}\m{(}\m{\varphi}\m{%
\rightarrow}\m{[}\m{{\rm suc}}\m{x}\m{/}\m{x}\m{]}\m{\varphi}\m{)}\m{)}
\endm
\setbox\startprefix=\hbox{\tt \ \ tfindes.3\ \$e\ }
\setbox\contprefix=\hbox{\tt \ \ \ \ \ \ \ \ \ \ \ \ \ \ \ }
\startm
\m{\vdash}\m{(}\m{{\rm Lim}}\m{y}\m{\rightarrow}\m{(}\m{\forall}\m{x}\m{\in}%
\m{y}\m{\varphi}\m{\rightarrow}\m{[}\m{y}\m{/}\m{x}\m{]}\m{\varphi}\m{)}\m{)}
\endm
\setbox\startprefix=\hbox{\tt \ \ tfindes\ \$p\ }
\setbox\contprefix=\hbox{\tt \ \ \ \ \ \ \ \ \ \ \ \ \ }
\startm
\m{\vdash}\m{(}\m{x}\m{\in}\m{{\rm On}}\m{\rightarrow}\m{\varphi}\m{)}
\endm
\vskip 1ex

\noindent Principle of Transfinite Recursion.\index{transfinite
recursion} Theorem 7.41 of Takeuti and Zaring, p.~47.  Transfinite
recursion is the key theorem that allows arithmetic of ordinals to be
rigorously defined, and has many other important uses as well.
Hypotheses \texttt{tfr.1} and \texttt{tfr.2} specify a certain (proper)
class $ F$.  The complicated definition of $ F$ is not important in
itself; what is important is that there be such an $ F$ with the
required properties, and we show this by displaying $ F$ explicitly.
\texttt{tfr1} states that $ F$ is a function whose domain is the set of
ordinal numbers.  \texttt{tfr2} states that any value of $ F$ is
completely determined by its previous values and the values of an
auxiliary function, $G$.  \texttt{tfr3} states that $ F$ is unique,
i.e.\ it is the only function that satisfies \texttt{tfr1} and
\texttt{tfr2}.  Note that $ f$ is an individual variable like $x$ and
$y$; it is just a mnemonic to remind us that $A$ is a collection of
functions.

\vskip 0.5ex
\setbox\startprefix=\hbox{\tt \ \ tfr.1\ \$e\ }
\setbox\contprefix=\hbox{\tt \ \ \ \ \ \ \ \ \ \ \ }
\startm
\m{\vdash}\m{A}\m{=}\m{\{}\m{f}\m{|}\m{\exists}\m{x}\m{\in}\m{{\rm On}}\m{(}%
\m{f}\m{{\rm Fn}}\m{x}\m{\wedge}\m{\forall}\m{y}\m{\in}\m{x}\m{(}\m{f}\m{`}%
\m{y}\m{)}\m{=}\m{(}\m{G}\m{`}\m{(}\m{f}\m{\restriction}\m{y}\m{)}\m{)}\m{)}%
\m{\}}
\endm
\setbox\startprefix=\hbox{\tt \ \ tfr.2\ \$e\ }
\setbox\contprefix=\hbox{\tt \ \ \ \ \ \ \ \ \ \ \ }
\startm
\m{\vdash}\m{F}\m{=}\m{\bigcup}\m{A}
\endm
\setbox\startprefix=\hbox{\tt \ \ tfr1\ \$p\ }
\setbox\contprefix=\hbox{\tt \ \ \ \ \ \ \ \ \ \ }
\startm
\m{\vdash}\m{F}\m{{\rm Fn}}\m{{\rm On}}
\endm
\setbox\startprefix=\hbox{\tt \ \ tfr2\ \$p\ }
\setbox\contprefix=\hbox{\tt \ \ \ \ \ \ \ \ \ \ }
\startm
\m{\vdash}\m{(}\m{z}\m{\in}\m{{\rm On}}\m{\rightarrow}\m{(}\m{F}\m{`}\m{z}%
\m{)}\m{=}\m{(}\m{G}\m{`}\m{(}\m{F}\m{\restriction}\m{z}\m{)}\m{)}\m{)}
\endm
\setbox\startprefix=\hbox{\tt \ \ tfr3\ \$p\ }
\setbox\contprefix=\hbox{\tt \ \ \ \ \ \ \ \ \ \ }
\startm
\m{\vdash}\m{(}\m{(}\m{B}\m{{\rm Fn}}\m{{\rm On}}\m{\wedge}\m{\forall}\m{x}\m{%
\in}\m{{\rm On}}\m{(}\m{B}\m{`}\m{x}\m{)}\m{=}\m{(}\m{G}\m{`}\m{(}\m{B}\m{%
\restriction}\m{x}\m{)}\m{)}\m{)}\m{\rightarrow}\m{B}\m{=}\m{F}\m{)}
\endm
\vskip 1ex

\noindent The existence of omega (the class of natural numbers).\index{natural
number}\index{omega ($\omega$)}\index{Axiom of Infinity}  Axiom 7 of Takeuti
and Zaring, p.~43.  (This is the only theorem in this section requiring the
Axiom of Infinity.)

\vskip 0.5ex
\setbox\startprefix=\hbox{\tt \
\ omex\ \$p\ }
\setbox\contprefix=\hbox{\tt \ \ \ \ \ \ \ \ \ \ }
\startm
\m{\vdash}\m{\omega}\m{\in}\m{{\rm V}}
\endm
%\vskip 2ex


\section{Axioms for Real and Complex Numbers}\label{real}
\index{real and complex numbers!axioms for}

This section presents the axioms for real and complex numbers, along
with some commentary about them.  Analysis
textbooks implicitly or explicitly use these axioms or their equivalents
as their starting point.  In the database \texttt{set.mm}, we define real
and complex numbers as (rather complicated) specific sets and derive these
axioms as {\em theorems} from the axioms of ZF set theory, using a method
called Dedekind cuts.  We omit the details of this construction, which you can
follow if you wish using the \texttt{set.mm} database in conjunction with the
textbooks referenced therein.

Once we prove those theorems, we then restate these proven theorems as axioms.
This lets us easily identify which axioms are needed for a particular complex number proof, without the obfuscation of the set theory used to derive them.
As a result,
the construction is actually unimportant other
than to show that sets exist that satisfy the axioms, and thus that the axioms
are consistent if ZF set theory is consistent.  When working with real numbers
you can think of them as being the actual sets resulting
from the construction (for definiteness), or you can
think of them as otherwise unspecified sets that happen to satisfy the axioms.
The derivation is not easy, but the fact that it works is quite remarkable
and lends support to the idea that ZFC set theory is all we need to
provide a foundation for essentially all of mathematics.

\needspace{3\baselineskip}
\subsection{The Axioms for Real and Complex Numbers Themselves}\label{realactual}

For the axioms we are given (or postulate) 8 classes:  $\mathbb{C}$ (the
set of complex numbers), $\mathbb{R}$ (the set of real numbers, a subset
of $\mathbb{C}$), $0$ (zero), $1$ (one), $i$ (square root of
$-1$), $+$ (plus), $\cdot$ (times), and
$<_{\mathbb{R}}$ (less than for just the real numbers).
Subtraction and division are defined terms and are not part of the
axioms; for their definitions see \texttt{set.mm}.

Note that the notation $(A+B)$ (and similarly $(A\cdot B)$) specifies a class
called an {\em operation},\index{operation} and is the function value of the
class $+$ at ordered pair $\langle A,B \rangle$.  An operation is defined by
statement \texttt{df-opr} on p.~\pageref{dfopr}.
The notation $A <_{\mathbb{R}} B$ specifies a
wff called a {\em binary relation}\index{binary relation} and means $\langle A,B \rangle \in \,<_{\mathbb{R}}$, as defined by statement \texttt{df-br} on p.~\pageref{dfbr}.

Our set of 8 given classes is assumed to satisfy the following 22 axioms
(in the axioms listed below, $<$ really means $<_{\mathbb{R}}$).

\vskip 2ex

\noindent 1. The real numbers are a subset of the complex numbers.

%\vskip 0.5ex
\setbox\startprefix=\hbox{\tt \ \ ax-resscn\ \$p\ }
\setbox\contprefix=\hbox{\tt \ \ \ \ \ \ \ \ \ \ \ \ \ \ }
\startm
\m{\vdash}\m{\mathbb{R}}\m{\subseteq}\m{\mathbb{C}}
\endm
%\vskip 1ex

\noindent 2. One is a complex number.

%\vskip 0.5ex
\setbox\startprefix=\hbox{\tt \ \ ax-1cn\ \$p\ }
\setbox\contprefix=\hbox{\tt \ \ \ \ \ \ \ \ \ \ \ }
\startm
\m{\vdash}\m{1}\m{\in}\m{\mathbb{C}}
\endm
%\vskip 1ex

\noindent 3. The imaginary number $i$ is a complex number.

%\vskip 0.5ex
\setbox\startprefix=\hbox{\tt \ \ ax-icn\ \$p\ }
\setbox\contprefix=\hbox{\tt \ \ \ \ \ \ \ \ \ \ \ }
\startm
\m{\vdash}\m{i}\m{\in}\m{\mathbb{C}}
\endm
%\vskip 1ex

\noindent 4. Complex numbers are closed under addition.

%\vskip 0.5ex
\setbox\startprefix=\hbox{\tt \ \ ax-addcl\ \$p\ }
\setbox\contprefix=\hbox{\tt \ \ \ \ \ \ \ \ \ \ \ \ \ }
\startm
\m{\vdash}\m{(}\m{(}\m{A}\m{\in}\m{\mathbb{C}}\m{\wedge}\m{B}\m{\in}\m{\mathbb{C}}%
\m{)}\m{\rightarrow}\m{(}\m{A}\m{+}\m{B}\m{)}\m{\in}\m{\mathbb{C}}\m{)}
\endm
%\vskip 1ex

\noindent 5. Real numbers are closed under addition.

%\vskip 0.5ex
\setbox\startprefix=\hbox{\tt \ \ ax-addrcl\ \$p\ }
\setbox\contprefix=\hbox{\tt \ \ \ \ \ \ \ \ \ \ \ \ \ \ }
\startm
\m{\vdash}\m{(}\m{(}\m{A}\m{\in}\m{\mathbb{R}}\m{\wedge}\m{B}\m{\in}\m{\mathbb{R}}%
\m{)}\m{\rightarrow}\m{(}\m{A}\m{+}\m{B}\m{)}\m{\in}\m{\mathbb{R}}\m{)}
\endm
%\vskip 1ex

\noindent 6. Complex numbers are closed under multiplication.

%\vskip 0.5ex
\setbox\startprefix=\hbox{\tt \ \ ax-mulcl\ \$p\ }
\setbox\contprefix=\hbox{\tt \ \ \ \ \ \ \ \ \ \ \ \ \ }
\startm
\m{\vdash}\m{(}\m{(}\m{A}\m{\in}\m{\mathbb{C}}\m{\wedge}\m{B}\m{\in}\m{\mathbb{C}}%
\m{)}\m{\rightarrow}\m{(}\m{A}\m{\cdot}\m{B}\m{)}\m{\in}\m{\mathbb{C}}\m{)}
\endm
%\vskip 1ex

\noindent 7. Real numbers are closed under multiplication.

%\vskip 0.5ex
\setbox\startprefix=\hbox{\tt \ \ ax-mulrcl\ \$p\ }
\setbox\contprefix=\hbox{\tt \ \ \ \ \ \ \ \ \ \ \ \ \ \ }
\startm
\m{\vdash}\m{(}\m{(}\m{A}\m{\in}\m{\mathbb{R}}\m{\wedge}\m{B}\m{\in}\m{\mathbb{R}}%
\m{)}\m{\rightarrow}\m{(}\m{A}\m{\cdot}\m{B}\m{)}\m{\in}\m{\mathbb{R}}\m{)}
\endm
%\vskip 1ex

\noindent 8. Multiplication of complex numbers is commutative.

%\vskip 0.5ex
\setbox\startprefix=\hbox{\tt \ \ ax-mulcom\ \$p\ }
\setbox\contprefix=\hbox{\tt \ \ \ \ \ \ \ \ \ \ \ \ \ \ }
\startm
\m{\vdash}\m{(}\m{(}\m{A}\m{\in}\m{\mathbb{C}}\m{\wedge}\m{B}\m{\in}\m{\mathbb{C}}%
\m{)}\m{\rightarrow}\m{(}\m{A}\m{\cdot}\m{B}\m{)}\m{=}\m{(}\m{B}\m{\cdot}\m{A}%
\m{)}\m{)}
\endm
%\vskip 1ex

\noindent 9. Addition of complex numbers is associative.

%\vskip 0.5ex
\setbox\startprefix=\hbox{\tt \ \ ax-addass\ \$p\ }
\setbox\contprefix=\hbox{\tt \ \ \ \ \ \ \ \ \ \ \ \ \ \ }
\startm
\m{\vdash}\m{(}\m{(}\m{A}\m{\in}\m{\mathbb{C}}\m{\wedge}\m{B}\m{\in}\m{\mathbb{C}}%
\m{\wedge}\m{C}\m{\in}\m{\mathbb{C}}\m{)}\m{\rightarrow}\m{(}\m{(}\m{A}\m{+}%
\m{B}\m{)}\m{+}\m{C}\m{)}\m{=}\m{(}\m{A}\m{+}\m{(}\m{B}\m{+}\m{C}\m{)}\m{)}%
\m{)}
\endm
%\vskip 1ex

\noindent 10. Multiplication of complex numbers is associative.

%\vskip 0.5ex
\setbox\startprefix=\hbox{\tt \ \ ax-mulass\ \$p\ }
\setbox\contprefix=\hbox{\tt \ \ \ \ \ \ \ \ \ \ \ \ \ \ }
\startm
\m{\vdash}\m{(}\m{(}\m{A}\m{\in}\m{\mathbb{C}}\m{\wedge}\m{B}\m{\in}\m{\mathbb{C}}%
\m{\wedge}\m{C}\m{\in}\m{\mathbb{C}}\m{)}\m{\rightarrow}\m{(}\m{(}\m{A}\m{\cdot}%
\m{B}\m{)}\m{\cdot}\m{C}\m{)}\m{=}\m{(}\m{A}\m{\cdot}\m{(}\m{B}\m{\cdot}\m{C}%
\m{)}\m{)}\m{)}
\endm
%\vskip 1ex

\noindent 11. Multiplication distributes over addition for complex numbers.

%\vskip 0.5ex
\setbox\startprefix=\hbox{\tt \ \ ax-distr\ \$p\ }
\setbox\contprefix=\hbox{\tt \ \ \ \ \ \ \ \ \ \ \ \ \ }
\startm
\m{\vdash}\m{(}\m{(}\m{A}\m{\in}\m{\mathbb{C}}\m{\wedge}\m{B}\m{\in}\m{\mathbb{C}}%
\m{\wedge}\m{C}\m{\in}\m{\mathbb{C}}\m{)}\m{\rightarrow}\m{(}\m{A}\m{\cdot}\m{(}%
\m{B}\m{+}\m{C}\m{)}\m{)}\m{=}\m{(}\m{(}\m{A}\m{\cdot}\m{B}\m{)}\m{+}\m{(}%
\m{A}\m{\cdot}\m{C}\m{)}\m{)}\m{)}
\endm
%\vskip 1ex

\noindent 12. The square of $i$ equals $-1$ (expressed as $i$-squared plus 1 is
0).

%\vskip 0.5ex
\setbox\startprefix=\hbox{\tt \ \ ax-i2m1\ \$p\ }
\setbox\contprefix=\hbox{\tt \ \ \ \ \ \ \ \ \ \ \ \ }
\startm
\m{\vdash}\m{(}\m{(}\m{i}\m{\cdot}\m{i}\m{)}\m{+}\m{1}\m{)}\m{=}\m{0}
\endm
%\vskip 1ex

\noindent 13. One and zero are distinct.

%\vskip 0.5ex
\setbox\startprefix=\hbox{\tt \ \ ax-1ne0\ \$p\ }
\setbox\contprefix=\hbox{\tt \ \ \ \ \ \ \ \ \ \ \ \ }
\startm
\m{\vdash}\m{1}\m{\ne}\m{0}
\endm
%\vskip 1ex

\noindent 14. One is an identity element for real multiplication.

%\vskip 0.5ex
\setbox\startprefix=\hbox{\tt \ \ ax-1rid\ \$p\ }
\setbox\contprefix=\hbox{\tt \ \ \ \ \ \ \ \ \ \ \ }
\startm
\m{\vdash}\m{(}\m{A}\m{\in}\m{\mathbb{R}}\m{\rightarrow}\m{(}\m{A}\m{\cdot}\m{1}%
\m{)}\m{=}\m{A}\m{)}
\endm
%\vskip 1ex

\noindent 15. Every real number has a negative.

%\vskip 0.5ex
\setbox\startprefix=\hbox{\tt \ \ ax-rnegex\ \$p\ }
\setbox\contprefix=\hbox{\tt \ \ \ \ \ \ \ \ \ \ \ \ \ \ }
\startm
\m{\vdash}\m{(}\m{A}\m{\in}\m{\mathbb{R}}\m{\rightarrow}\m{\exists}\m{x}\m{\in}%
\m{\mathbb{R}}\m{(}\m{A}\m{+}\m{x}\m{)}\m{=}\m{0}\m{)}
\endm
%\vskip 1ex

\noindent 16. Every nonzero real number has a reciprocal.

%\vskip 0.5ex
\setbox\startprefix=\hbox{\tt \ \ ax-rrecex\ \$p\ }
\setbox\contprefix=\hbox{\tt \ \ \ \ \ \ \ \ \ \ \ \ \ \ }
\startm
\m{\vdash}\m{(}\m{A}\m{\in}\m{\mathbb{R}}\m{\rightarrow}\m{(}\m{A}\m{\ne}\m{0}%
\m{\rightarrow}\m{\exists}\m{x}\m{\in}\m{\mathbb{R}}\m{(}\m{A}\m{\cdot}%
\m{x}\m{)}\m{=}\m{1}\m{)}\m{)}
\endm
%\vskip 1ex

\noindent 17. A complex number can be expressed in terms of two reals.

%\vskip 0.5ex
\setbox\startprefix=\hbox{\tt \ \ ax-cnre\ \$p\ }
\setbox\contprefix=\hbox{\tt \ \ \ \ \ \ \ \ \ \ \ \ }
\startm
\m{\vdash}\m{(}\m{A}\m{\in}\m{\mathbb{C}}\m{\rightarrow}\m{\exists}\m{x}\m{\in}%
\m{\mathbb{R}}\m{\exists}\m{y}\m{\in}\m{\mathbb{R}}\m{A}\m{=}\m{(}\m{x}\m{+}\m{(}%
\m{y}\m{\cdot}\m{i}\m{)}\m{)}\m{)}
\endm
%\vskip 1ex

\noindent 18. Ordering on reals satisfies strict trichotomy.

%\vskip 0.5ex
\setbox\startprefix=\hbox{\tt \ \ ax-pre-lttri\ \$p\ }
\setbox\contprefix=\hbox{\tt \ \ \ \ \ \ \ \ \ \ \ \ \ }
\startm
\m{\vdash}\m{(}\m{(}\m{A}\m{\in}\m{\mathbb{R}}\m{\wedge}\m{B}\m{\in}\m{\mathbb{R}}%
\m{)}\m{\rightarrow}\m{(}\m{A}\m{<}\m{B}\m{\leftrightarrow}\m{\lnot}\m{(}\m{A}%
\m{=}\m{B}\m{\vee}\m{B}\m{<}\m{A}\m{)}\m{)}\m{)}
\endm
%\vskip 1ex

\noindent 19. Ordering on reals is transitive.

%\vskip 0.5ex
\setbox\startprefix=\hbox{\tt \ \ ax-pre-lttrn\ \$p\ }
\setbox\contprefix=\hbox{\tt \ \ \ \ \ \ \ \ \ \ \ \ \ }
\startm
\m{\vdash}\m{(}\m{(}\m{A}\m{\in}\m{\mathbb{R}}\m{\wedge}\m{B}\m{\in}\m{\mathbb{R}}%
\m{\wedge}\m{C}\m{\in}\m{\mathbb{R}}\m{)}\m{\rightarrow}\m{(}\m{(}\m{A}\m{<}%
\m{B}\m{\wedge}\m{B}\m{<}\m{C}\m{)}\m{\rightarrow}\m{A}\m{<}\m{C}\m{)}\m{)}
\endm
%\vskip 1ex

\noindent 20. Ordering on reals is preserved after addition to both sides.

%\vskip 0.5ex
\setbox\startprefix=\hbox{\tt \ \ ax-pre-ltadd\ \$p\ }
\setbox\contprefix=\hbox{\tt \ \ \ \ \ \ \ \ \ \ \ \ \ }
\startm
\m{\vdash}\m{(}\m{(}\m{A}\m{\in}\m{\mathbb{R}}\m{\wedge}\m{B}\m{\in}\m{\mathbb{R}}%
\m{\wedge}\m{C}\m{\in}\m{\mathbb{R}}\m{)}\m{\rightarrow}\m{(}\m{A}\m{<}\m{B}\m{%
\rightarrow}\m{(}\m{C}\m{+}\m{A}\m{)}\m{<}\m{(}\m{C}\m{+}\m{B}\m{)}\m{)}\m{)}
\endm
%\vskip 1ex

\noindent 21. The product of two positive reals is positive.

%\vskip 0.5ex
\setbox\startprefix=\hbox{\tt \ \ ax-pre-mulgt0\ \$p\ }
\setbox\contprefix=\hbox{\tt \ \ \ \ \ \ \ \ \ \ \ \ \ \ }
\startm
\m{\vdash}\m{(}\m{(}\m{A}\m{\in}\m{\mathbb{R}}\m{\wedge}\m{B}\m{\in}\m{\mathbb{R}}%
\m{)}\m{\rightarrow}\m{(}\m{(}\m{0}\m{<}\m{A}\m{\wedge}\m{0}%
\m{<}\m{B}\m{)}\m{\rightarrow}\m{0}\m{<}\m{(}\m{A}\m{\cdot}\m{B}\m{)}%
\m{)}\m{)}
\endm
%\vskip 1ex

\noindent 22. A non-empty, bounded-above set of reals has a supremum.

%\vskip 0.5ex
\setbox\startprefix=\hbox{\tt \ \ ax-pre-sup\ \$p\ }
\setbox\contprefix=\hbox{\tt \ \ \ \ \ \ \ \ \ \ \ }
\startm
\m{\vdash}\m{(}\m{(}\m{A}\m{\subseteq}\m{\mathbb{R}}\m{\wedge}\m{A}\m{\ne}\m{%
\varnothing}\m{\wedge}\m{\exists}\m{x}\m{\in}\m{\mathbb{R}}\m{\forall}\m{y}\m{%
\in}\m{A}\m{\,y}\m{<}\m{x}\m{)}\m{\rightarrow}\m{\exists}\m{x}\m{\in}\m{%
\mathbb{R}}\m{(}\m{\forall}\m{y}\m{\in}\m{A}\m{\lnot}\m{x}\m{<}\m{y}\m{\wedge}\m{%
\forall}\m{y}\m{\in}\m{\mathbb{R}}\m{(}\m{y}\m{<}\m{x}\m{\rightarrow}\m{\exists}%
\m{z}\m{\in}\m{A}\m{\,y}\m{<}\m{z}\m{)}\m{)}\m{)}
\endm

% NOTE: The \m{...} expressions above could be represented as
% $ \vdash ( ( A \subseteq \mathbb{R} \wedge A \ne \varnothing \wedge \exists x \in \mathbb{R} \forall y \in A \,y < x ) \rightarrow \exists x \in \mathbb{R} ( \forall y \in A \lnot x < y \wedge \forall y \in \mathbb{R} ( y < x \rightarrow \exists z \in A \,y < z ) ) ) $

\vskip 2ex

This completes the set of axioms for real and complex numbers.  You may
wish to look at how subtraction, division, and decimal numbers
are defined in \texttt{set.mm}, and for fun look at the proof of $2+
2 = 4$ (theorem \texttt{2p2e4} in \texttt{set.mm})
as discussed in section \ref{2p2e4}.

In \texttt{set.mm} we define the non-negative integers $\mathbb{N}$, the integers
$\mathbb{Z}$, and the rationals $\mathbb{Q}$ as subsets of $\mathbb{R}$.  This leads
to the nice inclusion $\mathbb{N} \subseteq \mathbb{Z} \subseteq \mathbb{Q} \subseteq
\mathbb{R} \subseteq \mathbb{C}$, giving us a uniform framework in which, for
example, a property such as commutativity of complex number addition
automatically applies to integers.  The natural numbers $\mathbb{N}$
are different from the set $\omega$ we defined earlier, but both satisfy
Peano's postulates.

\subsection{Complex Number Axioms in Analysis Texts}

Most analysis texts construct complex numbers as ordered pairs of reals,
leading to construction-dependent properties that satisfy these axioms
but are not stated in their pure form.  (This is also done in
\texttt{set.mm} but our axioms are extracted from that construction.)
Other texts will simply state that $\mathbb{R}$ is a ``complete ordered
subfield of $\mathbb{C}$,'' leading to redundant axioms when this phrase
is completely expanded out.  In fact I have not seen a text with the
axioms in the explicit form above.
None of these axioms is unique individually, but this carefully worked out
collection of axioms is the result of years of work
by the Metamath community.

\subsection{Eliminating Unnecessary Complex Number Axioms}

We once had more axioms for real and complex numbers, but over years of time
we (the Metamath community)
have found ways to eliminate them (by proving them from other axioms)
or weaken them (by making weaker claims without reducing what
can be proved).
In particular, here are statements that used to be complex number
axioms but have since been formally proven (with Metamath) to be redundant:

\begin{itemize}
\item
  $\mathbb{C} \in V$.
  At one time this was listed as a ``complex number axiom.''
  However, this is not properly speaking a complex number axiom,
  and in any case its proof uses axioms of set theory.
  Proven redundant by Mario Carneiro\index{Carneiro, Mario} on
  17-Nov-2014 (see \texttt{axcnex}).
\item
  $((A \in \mathbb{C} \land B \in \mathbb{C}$) $\rightarrow$
  $(A + B) = (B + A))$.
  Proved redundant by Eric Schmidt\index{Schmidt, Eric} on 19-Jun-2012,
  and formalized by Scott Fenton\index{Fenton, Scott} on 3-Jan-2013
  (see \texttt{addcom}).
\item
  $(A \in \mathbb{C} \rightarrow (A + 0) = A)$.
  Proved redundant by Eric Schmidt on 19-Jun-2012,
  and formalized by Scott Fenton on 3-Jan-2013
  (see \texttt{addid1}).
\item
  $(A \in \mathbb{C} \rightarrow \exists x \in \mathbb{C} (A + x) = 0)$.
  Proved redundant by Eric Schmidt and formalized on 21-May-2007
  (see \texttt{cnegex}).
\item
  $((A \in \mathbb{C} \land A \ne 0) \rightarrow \exists x \in \mathbb{C} (A \cdot x) = 1)$.
  Proved redundant by Eric Schmidt and formalized on 22-May-2007
  (see \texttt{recex}).
\item
  $0 \in \mathbb{R}$.
  Proved redundant by Eric Schmidt on 19-Feb-2005 and formalized 21-May-2007
  (see \texttt{0re}).
\end{itemize}

We could eliminate 0 as an axiomatic object by defining it as
$( ( i \cdot i ) + 1 )$
and replacing it with this expression throughout the axioms. If this
is done, axiom ax-i2m1 becomes redundant. However, the remaining axioms
would become longer and less intuitive.

Eric Schmidt's paper analyzing this axiom system \cite{Schmidt}
presented a proof that these remaining axioms,
with the possible exception of ax-mulcom, are independent of the others.
It is currently an open question if ax-mulcom is independent of the others.

\section{Two Plus Two Equals Four}\label{2p2e4}

Here is a proof that $2 + 2 = 4$, as proven in the theorem \texttt{2p2e4}
in the database \texttt{set.mm}.
This is a useful demonstration of what a Metamath proof can look like.
This proof may have more steps than you're used to, but each step is rigorously
proven all the way back to the axioms of logic and set theory.
This display was originally generated by the Metamath program
as an {\sc HTML} file.

In the table showing the proof ``Step'' is the sequential step number,
while its associated ``Expression'' is an expression that we have proved.
``Ref'' is the name of a theorem or axiom that justifies that expression,
and ``Hyp'' refers to previous steps (if any) that the theorem or axiom
needs so that we can use it.  Expressions are indented further than
the expressions that depend on them to show their interdependencies.

\begin{table}[!htbp]
\caption{Two plus two equals four}
\begin{tabular}{lllll}
\textbf{Step} & \textbf{Hyp} & \textbf{Ref} & \textbf{Expression} & \\
1  &       & df-2    & $ \; \; \vdash 2 = 1 + 1$  & \\
2  & 1     & oveq2i  & $ \; \vdash (2 + 2) = (2 + (1 + 1))$ & \\
3  &       & df-4    & $ \; \; \vdash 4 = (3 + 1)$ & \\
4  &       & df-3    & $ \; \; \; \vdash 3 = (2 + 1)$ & \\
5  & 4     & oveq1i  & $ \; \; \vdash (3 + 1) = ((2 + 1) + 1)$ & \\
6  &       & 2cn     & $ \; \; \; \vdash 2 \in \mathbb{C}$ & \\
7  &       & ax-1cn  & $ \; \; \; \vdash 1 \in \mathbb{C}$ & \\
8  & 6,7,7 & addassi & $ \; \; \vdash ((2 + 1) + 1) = (2 + (1 + 1))$ & \\
9  & 3,5,8 & 3eqtri  & $ \; \vdash 4 = (2 + (1 + 1))$ & \\
10 & 2,9   & eqtr4i  & $ \vdash (2 + 2) = 4$ & \\
\end{tabular}
\end{table}

Step 1 says that we can assert that $2 = 1 + 1$ because it is
justified by \texttt{df-2}.
What is \texttt{df-2}?
It is simply the definition of $2$, which in our system is defined as being
equal to $1 + 1$.  This shows how we can use definitions in proofs.

Look at Step 2 of the proof. In the Ref column, we see that it references
a previously proved theorem, \texttt{oveq2i}.
It turns out that
theorem \texttt{oveq2i} requires a
hypothesis, and in the Hyp column of Step 2 we indicate that Step 1 will
satisfy (match) this hypothesis.
If we looked at \texttt{oveq2i}
we would find that it proves that given some hypothesis
$A = B$, we can prove that $( C F A ) = ( C F B )$.
If we use \texttt{oveq2i} and apply step 1's result as the hypothesis,
that will mean that $A = 2$ and $B = ( 1 + 1 )$ within this use of
\texttt{oveq2i}.
We are free to select any value of $C$ and $F$ (subject to syntax constraints),
so we are free to select $C = 2$ and $F = +$,
producing our desired result,
$ (2 + 2) = (2 + (1 + 1))$.

Step 2 is an example of substitution.
In the end, every step in every proof uses only this one substitution rule.
All the rules of logic, and all the axioms, are expressed so that
they can be used via this one substitution rule.
So once you master substitution, you can master every Metamath proof,
no exceptions.

Each step is clear and can be immediately checked.
In the {\sc HTML} display you can even click on each reference to see why it is
justified, making it easy to see why the proof works.

\section{Deduction}\label{deduction}

Strictly speaking,
a deduction (also called an inference) is a kind of statement that needs
some hypotheses to be true in order for its conclusion to be true.
A theorem, on the other hand, has no hypotheses.
Informally we often call both of them theorems, but in this section we
will stick to the strict definitions.

It sometimes happens that we have proved a deduction of the form
$\varphi \Rightarrow \psi$\index{$\Rightarrow$}
(given hypothesis $\varphi$ we can prove $\psi$)
and we want to then prove a theorem of the form
$\varphi \rightarrow \psi$.

Converting a deduction (which uses a hypothesis) into a theorem
(which does not) is not as simple as you might think.
The deduction says, ``if we can prove $\varphi$ then we can prove $\psi$,''
which is in some sense weaker than saying
``$\varphi$ implies $\psi$.''
There is no axiom of logic that permits us to directly obtain the theorem
given the deduction.\footnote{
The conversion of a deduction to a theorem does not even hold in general
for quantum propositional calculus,
which is a weak subset of classical propositional calculus.
It has been shown that adding the Standard Deduction Theorem (discussed below)
to quantum propositional calculus turns it into classical
propositional calculus!
}

This is in contrast to going the other way.
If we have the theorem ($\varphi \rightarrow \psi$),
it is easy to recover the deduction
($\varphi \Rightarrow \psi$)
using modus ponens\index{modus ponens}
(\texttt{ax-mp}; see section \ref{axmp}).

In the following subsections we first discuss the standard deduction theorem
(the traditional but awkward way to convert deductions into theorems) and
the weak deduction theorem (a limited version of the standard deduction
theorem that is easier to use and was once widely used in
\texttt{set.mm}\index{set theory database (\texttt{set.mm})}).
In section \ref{deductionstyle} we discuss
deduction style, the newer approach we now recommend in most cases.
Deduction style uses ``deduction form,'' a form that
prefixes each hypothesis (other than definitions) and the
conclusion with a universal antecedent (``$\varphi \rightarrow$'').
Deduction style is widely used in \texttt{set.mm},
so it is useful to understand it and \textit{why} it is widely used.
Section \ref{naturaldeduction}
briefly discusses our approach for using natural deduction
within \texttt{set.mm},
as that approach is deeply related to deduction style.
We conclude with a summary of the strengths of
our approach, which we believe are compelling.

\subsection{The Standard Deduction Theorem}\label{standarddeductiontheorem}

It is possible to make use of information
contained in the deduction or its proof to assist us with the proof of
the related theorem.
In traditional logic books, there is a metatheorem called the
Deduction Theorem\index{Deduction Theorem}\index{Standard Deduction Theorem},
discovered independently by Herbrand and Tarski around 1930.
The Deduction Theorem, which we often call the Standard Deduction Theorem,
provides an algorithm for constructing a proof of a theorem from the
proof of its corresponding deduction. See, for example,
\cite[p.~56]{Margaris}\index{Margaris, Angelo}.
To construct a proof for a theorem, the
algorithm looks at each step in the proof of the original deduction and
rewrites the step with several steps wherein the hypothesis is eliminated
and becomes an antecedent.

In ordinary mathematics, no one actually carries out the algorithm,
because (in its most basic form) it involves an exponential explosion of
the number of proof steps as more hypotheses are eliminated. Instead,
the Standard Deduction Theorem is invoked simply to claim that it can
be done in principle, without actually doing it.
What's more, the algorithm is not as simple as it might first appear
when applying it rigorously.
There is a subtle restriction on the Standard Deduction Theorem
that must be taken into account involving the axiom of generalization
when working with predicate calculus (see the literature for more detail).

One of the goals of Metamath is to let you plainly see, with as few
underlying concepts as possible, how mathematics can be derived directly
from the axioms, and not indirectly according to some hidden rules
buried inside a program or understood only by logicians. If we added
the Standard Deduction Theorem to the language and proof verifier,
that would greatly complicate both and largely defeat Metamath's goal
of simplicity. In principle, we could show direct proofs by expanding
out the proof steps generated by the algorithm of the Standard Deduction
Theorem, but that is not feasible in practice because the number of proof
steps quickly becomes huge, even astronomical.
Since the algorithm of the Standard Deduction Theorem is driven by the proof,
we would have to go through that proof
all over again---starting from axioms---in order to obtain the theorem form.
In terms of proof length, there would be no savings over just
proving the theorem directly instead of first proving the deduction form.

\subsection{Weak Deduction Theorem}\label{weakdeductiontheorem}

We have developed
a more efficient method for proving a theorem from a deduction
that can be used instead of the Standard Deduction Theorem
in many (but not all) cases.
We call this more efficient method the
Weak Deduction Theorem\index{Weak Deduction Theorem}.\footnote{
There is also an unrelated ``Weak Deduction Theorem''
in the field of relevance logic, so to avoid confusion we could call
ours the ``Weak Deduction Theorem for Classical Logic.''}
Unlike the Standard Deduction Theorem, the Weak Deduction Theorem produces the
theorem directly from a special substitution instance of the deduction,
using a small, fixed number of steps roughly proportional to the length
of the final theorem.

If you come to a proof referencing the Weak Deduction Theorem
\texttt{dedth} (or one of its variants \texttt{dedthxx}),
here is how to follow the proof without getting into the details:
just click on the theorem referenced in the step
just before the reference to \texttt{dedth} and ignore everything else.
Theorem \texttt{dedth} simply turns a hypothesis into an antecedent
(i.e. the hypothesis followed by $\rightarrow$
is placed in front of the assertion, and the hypothesis
itself is eliminated) given certain conditions.

The Weak Deduction Theorem
eliminates a hypothesis $\varphi$, making it become an antecedent.
It does this by proving an expression
$ \varphi \rightarrow \psi $ given two hypotheses:
(1)
$ ( A = {\rm if} ( \varphi , A , B ) \rightarrow ( \varphi \leftrightarrow \chi ) ) $
and
(2) $\chi$.
Note that it requires that a proof exists for $\varphi$ when the class variable
$A$ is replaced with a specific class $B$. The hypothesis $\chi$
should be assigned to the inference.
You can see the details of the proof of the Weak Deduction Theorem
in theorem \texttt{dedth}.

The Weak Deduction Theorem
is probably easier to understand by studying proofs that make use of it.
For example, let's look at the proof of \texttt{renegcl}, which proves that
$ \vdash ( A \in \mathbb{R} \rightarrow - A \in \mathbb{R} )$:

\needspace{4\baselineskip}
\begin{longtabu} {l l l X}
\textbf{Step} & \textbf{Hyp} & \textbf{Ref} & \textbf{Expression} \\
  1 &  & negeq &
  $\vdash$ $($ $A$ $=$ ${\rm if}$ $($ $A$ $\in$ $\mathbb{R}$ $,$ $A$ $,$ $1$ $)$ $\rightarrow$
  $\textrm{-}$ $A$ $=$ $\textrm{-}$ ${\rm if}$ $($ $A$ $\in$ $\mathbb{R}$
  $,$ $A$ $,$ $1$ $)$ $)$ \\
 2 & 1 & eleq1d &
    $\vdash$ $($ $A$ $=$ ${\rm if}$ $($ $A$ $\in$ $\mathbb{R}$ $,$ $A$ $,$ $1$ $)$ $\rightarrow$ $($
    $\textrm{-}$ $A$ $\in$ $\mathbb{R}$ $\leftrightarrow$
    $\textrm{-}$ ${\rm if}$ $($ $A$ $\in$ $\mathbb{R}$ $,$ $A$ $,$ $1$ $)$ $\in$
    $\mathbb{R}$ $)$ $)$ \\
 3 &  & 1re & $\vdash 1 \in \mathbb{R}$ \\
 4 & 3 & elimel &
   $\vdash {\rm if} ( A \in \mathbb{R} , A , 1 ) \in \mathbb{R}$ \\
 5 & 4 & renegcli &
   $\vdash \textrm{-} {\rm if} ( A \in \mathbb{R} , A , 1 ) \in \mathbb{R}$ \\
 6 & 2,5 & dedth &
   $\vdash ( A \in \mathbb{R} \rightarrow \textrm{-} A \in \mathbb{R}$ ) \\
\end{longtabu}

The somewhat strange-looking steps in \texttt{renegcl} before step 5 are
technical stuff that makes this magic work, and they can be ignored
for a quick overview of the proof. To continue following the ``important''
part of the proof of \texttt{renegcl},
you can look at the reference to \texttt{renegcli} at step 5.

That said, let's briefly look at how
\texttt{renegcl} uses the
Weak Deduction Theorem (\texttt{dedth}) to do its job,
in case you want to do something similar or want understand it more deeply.
Let's work backwards in the proof of \texttt{renegcl}.
Step 6 applies \texttt{dedth} to produce our goal result
$ \vdash ( A \in \mathbb{R} \rightarrow\, - A \in \mathbb{R} )$.
This requires on the one hand the (substituted) deduction
\texttt{renegcli} in step 5.
By itself \texttt{renegcli} proves the deduction
$ \vdash A \in \mathbb{R} \Rightarrow\, \vdash - A \in \mathbb{R}$;
this is the deduction form we are trying to turn into theorem form,
and thus
\texttt{renegcli} has a separate hypothesis that must be fulfilled.
To fulfill the hypothesis of the invocation of
\texttt{renegcli} in step 5, it is eventually
reduced to the already proven theorem $1 \in \mathbb{R}$ in step 3.
Step 4 connects steps 3 and 5; step 4 invokes
\texttt{elimel}, a special case of \texttt{elimhyp} that eliminates
a membership hypothesis for the weak deduction theorem.
On the other hand, the equivalence of the conclusion of
\texttt{renegcl}
$( - A \in \mathbb{R} )$ and the substituted conclusion of
\texttt{renegcli} must be proven, which is done in steps 2 and 1.

The weak deduction theorem has limitations.
In particular, we must be able to prove a special case of the deduction's
hypothesis as a stand-alone theorem.
For example, we used $1 \in \mathbb{R}$ in step 3 of \texttt{renegcl}.

We used to use the weak deduction theorem
extensively within \texttt{set.mm}.
However, we now recommend applying ``deduction style''
instead in most cases, as deduction style is
often an easier and clearer approach.
Therefore, we will now describe deduction style.

\subsection{Deduction Style}\label{deductionstyle}

We now prefer to write assertions in ``deduction form''
instead of writing a proof that would require use of the standard or
weak deduction theorem.
We call this appraoch
``deduction style.''\index{deduction style}

It will be easier to explain this by first defining some terms:

\begin{itemize}
\item \textbf{closed form}\index{closed form}\index{forms!closed}:
A kind of assertion (theorem) with no hypotheses.
Typically its label has no special suffix.
An example is \texttt{unss}, which states:
$\vdash ( ( A \subseteq C \wedge B \subseteq C ) \leftrightarrow ( A \cup B )
\subseteq C )\label{eq:unss}$
\item \textbf{deduction form}\index{deduction form}\index{forms!deduction}:
A kind of assertion with one or more hypotheses
where the conclusion is an implication with
a wff variable as the antecedent (usually $\varphi$), and every hypothesis
(\$e statement)
is either (1) an implication with the same antecedent as the conclusion or
(2) a definition.
A definition
can be for a class variable (this is a class variable followed by ``='')
or a wff variable (this is a wff variable followed by $\leftrightarrow$);
class variable definitions are more common.
In practice, a proof
in deduction form will also contain many steps that are implications
where the antecedent is either that wff variable (normally $\varphi$)
or is
a conjunction (...$\land$...) including that wff variable ($\varphi$).
If an assertion is in deduction form, and other forms are also available,
then we suffix its label with ``d.''
An example is \texttt{unssd}, which states\footnote{
For brevity we show here (and in other places)
a $\&$\index{$\&$} between hypotheses\index{hypotheses}
and a $\Rightarrow$\index{$\Rightarrow$}\index{conclusion}
between the hypotheses and the conclusion.
This notation is technically not part of the Metamath language, but is
instead a convenient abbreviation to show both the hypotheses and conclusion.}:
$\vdash ( \varphi \rightarrow A \subseteq C )\quad\&\quad \vdash ( \varphi
    \rightarrow B \subseteq C )\quad\Rightarrow\quad \vdash ( \varphi
    \rightarrow ( A \cup B ) \subseteq C )\label{eq:unssd}$
\item \textbf{inference form}\index{inference form}\index{forms!inference}:
A kind of assertion with one or more hypotheses that is not in deduction form
(e.g., there is no common antecedent).
If an assertion is in inference form, and other forms are also available,
then we suffix its label with ``i.''
An example is \texttt{unssi}, which states:
$\vdash A \subseteq C\quad\&\quad \vdash B \subseteq C\quad\Rightarrow\quad
    \vdash ( A \cup B ) \subseteq C\label{eq:unssi}$
\end{itemize}

When using deduction style we express an assertion in deduction form.
This form prefixes each hypothesis (other than definitions) and the
conclusion with a universal antecedent (``$\varphi \rightarrow$'').
The antecedent (e.g., $\varphi$)
mimics the context handled in the deduction theorem, eliminating
the need to directly use the deduction theorem.

Once you have an assertion in deduction form, you can easily convert it
to inference form or closed form:

\begin{itemize}
\item To
prove some assertion Ti in inference form, given assertion Td in deduction
form, there is a simple mechanical process you can use. First take each
Ti hypothesis and insert a \texttt{T.} $\rightarrow$ prefix (``true implies'')
using \texttt{a1i}. You
can then use the existing assertion Td to prove the resulting conclusion
with a \texttt{T.} $\rightarrow$ prefix.
Finally, you can remove that prefix using \texttt{mptru},
resulting in the conclusion you wanted to prove.
\item To
prove some assertion T in closed form, given assertion Td in deduction
form, there is another simple mechanical process you can use. First,
select an expression that is the conjunction (...$\land$...) of all of the
consequents of every hypothesis of Td. Next, prove that this expression
implies each of the separate hypotheses of Td in turn by eliminating
conjuncts (there are a variety of proven assertions to do this, including
\texttt{simpl},
\texttt{simpr},
\texttt{3simpa},
\texttt{3simpb},
\texttt{3simpc},
\texttt{simp1},
\texttt{simp2},
and
\texttt{simp3}).
If the
expression has nested conjunctions, inner conjuncts can be broken out by
chaining the above theorems with \texttt{syl}
(see section \ref{syl}).\footnote{
There are actually many theorems
(labeled simp* such as \texttt{simp333}) that break out inner conjuncts in one
step, but rather than learning them you can just use the chaining we
just described to prove them, and then let the Metamath program command
\texttt{minimize{\char`\_}with}\index{\texttt{minimize{\char`\_}with} command}
figure out the right ones needed to collapse them.}
As your final step, you can then apply the already-proven assertion Td
(which is in deduction form), proving assertion T in closed form.
\end{itemize}

We can also easily convert any assertion T in closed form to its related
assertion Ti in inference form by applying
modus ponens\index{modus ponens} (see section \ref{axmp}).

The deduction form antecedent can also be used to represent the context
necessary to support natural deduction systems, so we will now
discuss natural deduction.

\subsection{Natural Deduction}\label{naturaldeduction}

Natural deduction\index{natural deduction}
(ND) systems, as such, were originally introduced in
1934 by two logicians working independently: Ja\'skowski and Gentzen. ND
systems are supposed to reconstruct, in a formally proper way, traditional
ways of mathematical reasoning (such as conditional proof, indirect proof,
and proof by cases). As reconstructions they were naturally influenced
by previous work, and many specific ND systems and notations have been
developed since their original work.

There are many ND variants, but
Indrzejczak \cite[p.~31-32]{Indrzejczak}\index{Indrzejczak, Andrzej}
suggests that any natural deductive system must satisfy at
least these three criteria:

\begin{itemize}
\item ``There are some means for entering assumptions into a proof and
also for eliminating them. Usually it requires some bookkeeping devices
for indicating the scope of an assumption, and showing that a part of
a proof depending on eliminated assumption is discharged.
\item There are no (or, at least, a very limited set of) axioms, because
their role is taken over by the set of primitive rules for introduction
and elimination of logical constants which means that elementary
inferences instead of formulae are taken as primitive.
\item (A genuine) ND system admits a lot of freedom in proof construction
and possibility of applying several proof search strategies, like
conditional proof, proof by cases, proof by reductio ad absurdum etc.''
\end{itemize}

The Metamath Proof Explorer (MPE) as defined in \texttt{set.mm}
is fundamentally a Hilbert-style system.
That is, MPE is based on a larger number of axioms (compared
to natural deduction systems), a very small set of rules of inference
(modus ponens), and the context is not changed by the rules of inference
in the middle of a proof. That said, MPE proofs can be developed using
the natural deduction (ND) approach as originally developed by Ja\'skowski
and Gentzen.

The most common and recommended approach for applying ND in MPE is to use
deduction form\index{deduction form}%
\index{forms!deduction}
and apply the MPE proven assertions that are equivalent to ND rules.
For example, MPE's \texttt{jca} is equivalent to ND rule $\land$-I
(and-insertion).
We maintain a list of equivalences that you may consult.
This approach for applying an ND approach within MPE relies on Metamath's
wff metavariables in an essential way, and is described in more detail
in the presentation ``Natural Deductions in the Metamath Proof Language''
by Mario Carneiro \cite{CarneiroND}\index{Carneiro, Mario}.

In this style many steps are an implication, whose antecedent mimics
the context ($\Gamma$) of most ND systems. To add an assumption, simply add
it to the implication antecedent (typically using
\texttt{simpr}),
and use that
new antecedent for all later claims in the same scope. If you wish to
use an assertion in an ND hypothesis scope that is outside the current
ND hypothesis scope, modify the assertion so that the ND hypothesis
assumption is added to its antecedent (typically using \texttt{adantr}). Most
proof steps will be proved using rules that have hypotheses and results
of the form $\varphi \rightarrow$ ...

An example may make this clearer.
Let's show theorem 5.5 of
\cite[p.~18]{Clemente}\index{Clemente Laboreo, Daniel}
along with a line by line translation using the usual
translation of natural deduction (ND) in the Metamath Proof Explorer
(MPE) notation (this is proof \texttt{ex-natded5.5}).
The proof's original goal was to prove
$\lnot \psi$ given two hypotheses,
$( \psi \rightarrow \chi )$ and $ \lnot \chi$.
We will translate these statements into MPE deduction form
by prefixing them all with $\varphi \rightarrow$.
As a result, in MPE the goal is stated as
$( \varphi \rightarrow \lnot \psi )$, and the two hypotheses are stated as
$( \varphi \rightarrow ( \psi \rightarrow \chi ) )$ and
$( \varphi \rightarrow \lnot \chi )$.

The following table shows the proof in Fitch natural deduction style
and its MPE equivalent.
The \textit{\#} column shows the original numbering,
\textit{MPE\#} shows the number in the equivalent MPE proof
(which we will show later),
\textit{ND Expression} shows the original proof claim in ND notation,
and \textit{MPE Translation} shows its translation into MPE
as discussed in this section.
The final columns show the rationale in ND and MPE respectively.

\needspace{4\baselineskip}
{\setlength{\extrarowsep}{4pt} % Keep rows from being too close together
\begin{longtabu}   { @{} c c X X X X }
\textbf{\#} & \textbf{MPE\#} & \textbf{ND Ex\-pres\-sion} &
\textbf{MPE Trans\-lation} & \textbf{ND Ration\-ale} &
\textbf{MPE Ra\-tio\-nale} \\
\endhead

1 & 2;3 &
$( \psi \rightarrow \chi )$ &
$( \varphi \rightarrow ( \psi \rightarrow \chi ) )$ &
Given &
\$e; \texttt{adantr} to put in ND hypothesis \\

2 & 5 &
$ \lnot \chi$ &
$( \varphi \rightarrow \lnot \chi )$ &
Given &
\$e; \texttt{adantr} to put in ND hypothesis \\

3 & 1 &
... $\vert$ $\psi$ &
$( \varphi \rightarrow \psi )$ &
ND hypothesis assumption &
\texttt{simpr} \\

4 & 4 &
... $\chi$ &
$( ( \varphi \land \psi ) \rightarrow \chi )$ &
$\rightarrow$\,E 1,3 &
\texttt{mpd} 1,3 \\

5 & 6 &
... $\lnot \chi$ &
$( ( \varphi \land \psi ) \rightarrow \lnot \chi )$ &
IT 2 &
\texttt{adantr} 5 \\

6 & 7 &
$\lnot \psi$ &
$( \varphi \rightarrow \lnot \psi )$ &
$\land$\,I 3,4,5 &
\texttt{pm2.65da} 4,6 \\

\end{longtabu}
}


The original used Latin letters; we have replaced them with Greek letters
to follow Metamath naming conventions and so that it is easier to follow
the Metamath translation. The Metamath line-for-line translation of
this natural deduction approach precedes every line with an antecedent
including $\varphi$ and uses the Metamath equivalents of the natural deduction
rules. To add an assumption, the antecedent is modified to include it
(typically by using \texttt{adantr};
\texttt{simpr} is useful when you want to
depend directly on the new assumption, as is shown here).

In Metamath we can represent the two given statements as these hypotheses:

\needspace{2\baselineskip}
\begin{itemize}
\item ex-natded5.5.1 $\vdash ( \varphi \rightarrow ( \psi \rightarrow \chi ) )$
\item ex-natded5.5.2 $\vdash ( \varphi \rightarrow \lnot \chi )$
\end{itemize}

\needspace{4\baselineskip}
Here is the proof in Metamath as a line-by-line translation:

\begin{longtabu}   { l l l X }
\textbf{Step} & \textbf{Hyp} & \textbf{Ref} & \textbf{Ex\-pres\-sion} \\
\endhead
1 & & simpr & $\vdash ( ( \varphi \land \psi ) \rightarrow \psi )$ \\
2 & & ex-natded5.5.1 &
  $\vdash ( \varphi \rightarrow ( \psi \rightarrow \chi ) )$ \\
3 & 2 & adantr &
 $\vdash ( ( \varphi \land \psi ) \rightarrow ( \psi \rightarrow \chi ) )$ \\
4 & 1, 3 & mpd &
 $\vdash ( ( \varphi \land \psi ) \rightarrow \chi ) $ \\
5 & & ex-natded5.5.2 &
 $\vdash ( \varphi \rightarrow \lnot \chi )$ \\
6 & 5 & adantr &
 $\vdash ( ( \varphi \land \psi ) \rightarrow \lnot \chi )$ \\
7 & 4, 6 & pm2.65da &
 $\vdash ( \varphi \rightarrow \lnot \psi )$ \\
\end{longtabu}

Only using specific natural deduction rules directly can lead to very
long proofs, for exactly the same reason that only using axioms directly
in Hilbert-style proofs can lead to very long proofs.
If the goal is short and clear proofs,
then it is better to reuse already-proven assertions
in deduction form than to start from scratch each time
and using only basic natural deduction rules.

\subsection{Strengths of Our Approach}

As far as we know there is nothing else in the literature like either the
weak deduction theorem or Mario Carneiro\index{Carneiro, Mario}'s
natural deduction method.
In order to
transform a hypothesis into an antecedent, the literature's standard
``Deduction Theorem''\index{Deduction Theorem}\index{Standard Deduction Theorem}
requires metalogic outside of the notions provided
by the axiom system. We instead generally prefer to use Mario Carneiro's
natural deduction method, then use the weak deduction theorem in cases
where that is difficult to apply, and only then use the full standard
deduction theorem as a last resort.

The weak deduction theorem\index{Weak Deduction Theorem}
does not require any additional metalogic
but converts an inference directly into a closed form theorem, with
a rigorous proof that uses only the axiom system. Unlike the standard
Deduction Theorem, there is no implicit external justification that we
have to trust in order to use it.

Mario Carneiro's natural deduction\index{natural deduction}
method also does not require any new metalogical
notions. It avoids the Deduction Theorem's metalogic by prefixing the
hypotheses and conclusion of every would-be inference with a universal
antecedent (``$\varphi \rightarrow$'') from the very start.

We think it is impressive and satisfying that we can do so much in a
practical sense without stepping outside of our Hilbert-style axiom system.
Of course our axiomatization, which is in the form of schemes,
contains a metalogic of its own that we exploit. But this metalogic
is relatively simple, and for our Deduction Theorem alternatives,
we primarily use just the direct substitution of expressions for
metavariables.

\begin{sloppy}
\section{Exploring the Set The\-o\-ry Data\-base}\label{exploring}
\end{sloppy}
% NOTE: All examples performed in this section are
% recorded wtih "set width 61" % on set.mm as of 2019-05-28
% commit c1e7849557661260f77cfdf0f97ac4354fbb4f4d.

At this point you may wish to study the \texttt{set.mm}\index{set theory
database (\texttt{set.mm})} file in more detail.  Pay particular
attention to the assumptions needed to define wffs\index{well-formed
formula (wff)} (which are not included above), the variable types
(\texttt{\$f}\index{\texttt{\$f} statement} statements), and the
definitions that are introduced.  Start with some simple theorems in
propositional calculus, making sure you understand in detail each step
of a proof.  Once you get past the first few proofs and become familiar
with the Metamath language, any part of the \texttt{set.mm} database
will be as easy to follow, step by step, as any other part---you won't
have to undergo a ``quantum leap'' in mathematical sophistication to be
able to follow a deep proof in set theory.

Next, you may want to explore how concepts such as natural numbers are
defined and described.  This is probably best done in conjunction with
standard set theory textbooks, which can help give you a higher-level
understanding.  The \texttt{set.mm} database provides references that will get
you started.  From there, you will be on your way towards a very deep,
rigorous understanding of abstract mathematics.

The Metamath\index{Metamath} program can help you peruse a Metamath data\-base,
wheth\-er you are trying to figure out how a certain step follows in a proof or
just have a general curiosity.  We will go through some examples of the
commands, using the \texttt{set.mm}\index{set theory database (\texttt{set.mm})}
database provided with the Metamath software.  These should help get you
started.  See Chapter~\ref{commands} for a more detailed description of
the commands.  Note that we have included the full spelling of all commands to
prevent ambiguity with future commands.  In practice you may type just the
characters needed to specify each command keyword\index{command keyword}
unambiguously, often just one or two characters per keyword, and you don't
need to type them in upper case.

First run the Metamath program as described earlier.  You should see the
\verb/MM>/ prompt.  Read in the \texttt{set.mm} file:\index{\texttt{read}
command}

\begin{verbatim}
MM> read set.mm
Reading source file "set.mm"... 34554442 bytes
34554442 bytes were read into the source buffer.
The source has 155711 statements; 2254 are $a and 32250 are $p.
No errors were found.  However, proofs were not checked.
Type VERIFY PROOF * if you want to check them.
\end{verbatim}

As with most examples in this book, what you will see
will be slightly different because we are continuously
improving our databases (including \texttt{set.mm}).

Let's check the database integrity.  This may take a minute or two to run if
your computer is slow.

\begin{verbatim}
MM> verify proof *
0 10%  20%  30%  40%  50%  60%  70%  80%  90% 100%
..................................................
All proofs in the database were verified in 2.84 s.
\end{verbatim}

No errors were reported, so every proof is correct.

You need to know the names (labels) of theorems before you can look at them.
Often just examining the database file(s) with a text editor is the best
approach.  In \texttt{set.mm} there are many detailed comments, especially near
the beginning, that can help guide you. The \texttt{search} command in the
Metamath program is also handy.  The \texttt{comments} qualifier will list the
statements whose associated comment (the one immediately before it) contain a
string you give it.  For example, if you are studying Enderton's {\em Elements
of Set Theory} \cite{Enderton}\index{Enderton, Herbert B.} you may want to see
the references to it in the database.  The search string \texttt{enderton} is not
case sensitive.  (This will not show you all the database theorems that are in
Enderton's book because there is usually only one citation for a given
theorem, which may appear in several textbooks.)\index{\texttt{search}
command}

\begin{verbatim}
MM> search * "enderton" / comments
12067 unineq $p "... Exercise 20 of [Enderton] p. 32 and ..."
12459 undif2 $p "...Corollary 6K of [Enderton] p. 144. (C..."
12953 df-tp $a "...s. Definition of [Enderton] p. 19. (Co..."
13689 unissb $p ".... Exercise 5 of [Enderton] p. 26 and ..."
\end{verbatim}
\begin{center}
(etc.)
\end{center}

Or you may want to see what theorems have something to do with
conjunction (logical {\sc and}).  The quotes around the search
string are optional when there's no ambiguity.\index{\texttt{search}
command}

\begin{verbatim}
MM> search * conjunction / comments
120 a1d $p "...be replaced with a conjunction ( ~ df-an )..."
662 df-bi $a "...viated form after conjunction is introdu..."
1319 wa $a "...ff definition to include conjunction ('and')."
1321 df-an $a "Define conjunction (logical 'and'). Defini..."
1420 imnan $p "...tion in terms of conjunction. (Contribu..."
\end{verbatim}
\begin{center}
(etc.)
\end{center}


Now we will start to look at some details.  Let's look at the first
axiom of propositional calculus
(we could use \texttt{sh st} to abbreviate
\texttt{show statement}).\index{\texttt{show statement} command}

\begin{verbatim}
MM> show statement ax-1/full
Statement 19 is located on line 881 of the file "set.mm".
"Axiom _Simp_.  Axiom A1 of [Margaris] p. 49.  One of the 3
axioms of propositional calculus.  The 3 axioms are also
        ...
19 ax-1 $a |- ( ph -> ( ps -> ph ) ) $.
Its mandatory hypotheses in RPN order are:
  wph $f wff ph $.
  wps $f wff ps $.
The statement and its hypotheses require the variables:  ph
      ps
The variables it contains are:  ph ps


Statement 49 is located on line 11182 of the file "set.mm".
Its statement number for HTML pages is 6.
"Axiom _Simp_.  Axiom A1 of [Margaris] p. 49.  One of the 3
axioms of propositional calculus.  The 3 axioms are also
given as Definition 2.1 of [Hamilton] p. 28.
...
49 ax-1 $a |- ( ph -> ( ps -> ph ) ) $.
Its mandatory hypotheses in RPN order are:
  wph $f wff ph $.
  wps $f wff ps $.
The statement and its hypotheses require the variables:
  ph ps
The variables it contains are:  ph ps
\end{verbatim}

Compare this to \texttt{ax-1} on p.~\pageref{ax1}.  You can see that the
symbol \texttt{ph} is the {\sc ascii} notation for $\varphi$, etc.  To
see the mathematical symbols for any expression you may typeset it in
\LaTeX\ (type \texttt{help tex} for instructions)\index{latex@{\LaTeX}}
or, easier, just use a text editor to look at the comments where symbols
are first introduced in \texttt{set.mm}.  The hypotheses \texttt{wph}
and \texttt{wps} required by \texttt{ax-1} mean that variables
\texttt{ph} and \texttt{ps} must be wffs.

Next we'll pick a simple theorem of propositional calculus, the Principle of
Identity, which is proved directly from the axioms.  We'll look at the
statement then its proof.\index{\texttt{show statement}
command}

\begin{verbatim}
MM> show statement id1/full
Statement 116 is located on line 11371 of the file "set.mm".
Its statement number for HTML pages is 22.
"Principle of identity.  Theorem *2.08 of [WhiteheadRussell]
p. 101.  This version is proved directly from the axioms for
demonstration purposes.
...
116 id1 $p |- ( ph -> ph ) $= ... $.
Its mandatory hypotheses in RPN order are:
  wph $f wff ph $.
Its optional hypotheses are:  wps wch wth wta wet
      wze wsi wrh wmu wla wka
The statement and its hypotheses require the variables:  ph
These additional variables are allowed in its proof:
      ps ch th ta et ze si rh mu la ka
The variables it contains are:  ph
\end{verbatim}

The optional variables\index{optional variable} \texttt{ps}, \texttt{ch}, etc.\ are
available for use in a proof of this statement if we wish, and were we to do
so we would make use of optional hypotheses \texttt{wps}, \texttt{wch}, etc.  (See
Section~\ref{dollaref} for the meaning of ``optional
hypothesis.''\index{optional hypothesis}) The reason these show up in the
statement display is that statement \texttt{id1} happens to be in their scope
(see Section~\ref{scoping} for the definition of ``scope''\index{scope}), but
in fact in propositional calculus we will never make use of optional
hypotheses or variables.  This becomes important after quantifiers are
introduced, where ``dummy'' variables\index{dummy variable} are often needed
in the middle of a proof.

Let's look at the proof of statement \texttt{id1}.  We'll use the
\texttt{show proof} command, which by default suppresses the
``non-essential'' steps that construct the wffs.\index{\texttt{show proof}
command}
We will display the proof in ``lemmon' format (a non-indented format
with explicit previous step number references) and renumber the
displayed steps:

\begin{verbatim}
MM> show proof id1 /lemmon/renumber
1 ax-1           $a |- ( ph -> ( ph -> ph ) )
2 ax-1           $a |- ( ph -> ( ( ph -> ph ) -> ph ) )
3 ax-2           $a |- ( ( ph -> ( ( ph -> ph ) -> ph ) ) ->
                     ( ( ph -> ( ph -> ph ) ) -> ( ph -> ph )
                                                          ) )
4 2,3 ax-mp      $a |- ( ( ph -> ( ph -> ph ) ) -> ( ph -> ph
                                                          ) )
5 1,4 ax-mp      $a |- ( ph -> ph )
\end{verbatim}

If you have read Section~\ref{trialrun}, you'll know how to interpret this
proof.  Step~2, for example, is an application of axiom \texttt{ax-1}.  This
proof is identical to the one in Hamilton's {\em Logic for Mathematicians}
\cite[p.~32]{Hamilton}\index{Hamilton, Alan G.}.

You may want to look at what
substitutions\index{substitution!variable}\index{variable substitution} are
made into \texttt{ax-1} to arrive at step~2.  The command to do this needs to
know the ``real'' step number, so we'll display the proof again without
the \texttt{renumber} qualifier.\index{\texttt{show proof}
command}

\begin{verbatim}
MM> show proof id1 /lemmon
 9 ax-1          $a |- ( ph -> ( ph -> ph ) )
20 ax-1          $a |- ( ph -> ( ( ph -> ph ) -> ph ) )
24 ax-2          $a |- ( ( ph -> ( ( ph -> ph ) -> ph ) ) ->
                     ( ( ph -> ( ph -> ph ) ) -> ( ph -> ph )
                                                          ) )
25 20,24 ax-mp   $a |- ( ( ph -> ( ph -> ph ) ) -> ( ph -> ph
                                                          ) )
26 9,25 ax-mp    $a |- ( ph -> ph )
\end{verbatim}

The ``real'' step number is 20.  Let's look at its details.

\begin{verbatim}
MM> show proof id1 /detailed_step 20
Proof step 20:  min=ax-1 $a |- ( ph -> ( ( ph -> ph ) -> ph )
  )
This step assigns source "ax-1" ($a) to target "min" ($e).
The source assertion requires the hypotheses "wph" ($f, step
18) and "wps" ($f, step 19).  The parent assertion of the
target hypothesis is "ax-mp" ($a, step 25).
The source assertion before substitution was:
    ax-1 $a |- ( ph -> ( ps -> ph ) )
The following substitutions were made to the source
assertion:
    Variable  Substituted with
     ph        ph
     ps        ( ph -> ph )
The target hypothesis before substitution was:
    min $e |- ph
The following substitution was made to the target hypothesis:
    Variable  Substituted with
     ph        ( ph -> ( ( ph -> ph ) -> ph ) )
\end{verbatim}

This shows the substitutions\index{substitution!variable}\index{variable
substitution} made to the variables in \texttt{ax-1}.  References are made to
steps 18 and 19 which are not shown in our proof display.  To see these steps,
you can display the proof with the \texttt{all} qualifier.

Let's look at a slightly more advanced proof of propositional calculus.  Note
that \verb+/\+ is the symbol for $\wedge$ (logical {\sc and}, also
called conjunction).\index{conjunction ($\wedge$)}
\index{logical {\sc and} ($\wedge$)}

\begin{verbatim}
MM> show statement prth/full
Statement 1791 is located on line 15503 of the file "set.mm".
Its statement number for HTML pages is 559.
"Conjoin antecedents and consequents of two premises.  This
is the closed theorem form of ~ anim12d .  Theorem *3.47 of
[WhiteheadRussell] p. 113.  It was proved by Leibniz,
and it evidently pleased him enough to call it
_praeclarum theorema_ (splendid theorem).
...
1791 prth $p |- ( ( ( ph -> ps ) /\ ( ch -> th ) ) -> ( ( ph
      /\ ch ) -> ( ps /\ th ) ) ) $= ... $.
Its mandatory hypotheses in RPN order are:
  wph $f wff ph $.
  wps $f wff ps $.
  wch $f wff ch $.
  wth $f wff th $.
Its optional hypotheses are:  wta wet wze wsi wrh wmu wla wka
The statement and its hypotheses require the variables:  ph
      ps ch th
These additional variables are allowed in its proof:  ta et
      ze si rh mu la ka
The variables it contains are:  ph ps ch th


MM> show proof prth /lemmon/renumber
1 simpl          $p |- ( ( ( ph -> ps ) /\ ( ch -> th ) ) ->
                                               ( ph -> ps ) )
2 simpr          $p |- ( ( ( ph -> ps ) /\ ( ch -> th ) ) ->
                                               ( ch -> th ) )
3 1,2 anim12d    $p |- ( ( ( ph -> ps ) /\ ( ch -> th ) ) ->
                           ( ( ph /\ ch ) -> ( ps /\ th ) ) )
\end{verbatim}

There are references to a lot of unfamiliar statements.  To see what they are,
you may type the following:

\begin{verbatim}
MM> show proof prth /statement_summary
Summary of statements used in the proof of "prth":

Statement simpl is located on line 14748 of the file
"set.mm".
"Elimination of a conjunct.  Theorem *3.26 (Simp) of
[WhiteheadRussell] p. 112. ..."
  simpl $p |- ( ( ph /\ ps ) -> ph ) $= ... $.

Statement simpr is located on line 14777 of the file
"set.mm".
"Elimination of a conjunct.  Theorem *3.27 (Simp) of
[WhiteheadRussell] ..."
  simpr $p |- ( ( ph /\ ps ) -> ps ) $= ... $.

Statement anim12d is located on line 15445 of the file
"set.mm".
"Conjoin antecedents and consequents in a deduction.
..."
  anim12d.1 $e |- ( ph -> ( ps -> ch ) ) $.
  anim12d.2 $e |- ( ph -> ( th -> ta ) ) $.
  anim12d $p |- ( ph -> ( ( ps /\ th ) -> ( ch /\ ta ) ) )
      $= ... $.
\end{verbatim}
\begin{center}
(etc.)
\end{center}

Of course you can look at each of these statements and their proofs, and
so on, back to the axioms of propositional calculus if you wish.

The \texttt{search} command is useful for finding statements when you
know all or part of their contents.  The following example finds all
statements containing \verb@ph -> ps@ followed by \verb@ch -> th@.  The
\verb@$*@ is a wildcard that matches anything; the \texttt{\$} before the
\verb$*$ prevents conflicts with math symbol token names.  The \verb@*@ after
\texttt{SEARCH} is also a wildcard that in this case means ``match any label.''
\index{\texttt{search} command}

% I'm omitting this one, since readers are unlikely to see it:
% 1096 bisymOLD $p |- ( ( ( ph -> ps ) -> ( ch -> th ) ) -> ( (
%   ( ps -> ph ) -> ( th -> ch ) ) -> ( ( ph <-> ps ) -> ( ch
%    <-> th ) ) ) )
\begin{verbatim}
MM> search * "ph -> ps $* ch -> th"
1791 prth $p |- ( ( ( ph -> ps ) /\ ( ch -> th ) ) -> ( ( ph
    /\ ch ) -> ( ps /\ th ) ) )
2455 pm3.48 $p |- ( ( ( ph -> ps ) /\ ( ch -> th ) ) -> ( (
    ph \/ ch ) -> ( ps \/ th ) ) )
117859 pm11.71 $p |- ( ( E. x ph /\ E. y ch ) -> ( ( A. x (
    ph -> ps ) /\ A. y ( ch -> th ) ) <-> A. x A. y ( ( ph /\
    ch ) -> ( ps /\ th ) ) ) )
\end{verbatim}

Three statements, \texttt{prth}, \texttt{pm3.48},
 and \texttt{pm11.71}, were found to match.

To see what axioms\index{axiom} and definitions\index{definition}
\texttt{prth} ultimately depends on for its proof, you can have the
program backtrack through the hierarchy\index{hierarchy} of theorems and
definitions.\index{\texttt{show trace{\char`\_}back} command}

\begin{verbatim}
MM> show trace_back prth /essential/axioms
Statement "prth" assumes the following axioms ($a
statements):
  ax-1 ax-2 ax-3 ax-mp df-bi df-an
\end{verbatim}

Note that the 3 axioms of propositional calculus and the modus ponens rule are
needed (as expected); in addition, there are a couple of definitions that are used
along the way.  Note that Metamath makes no distinction\index{axiom vs.\
definition} between axioms\index{axiom} and definitions\index{definition}.  In
\texttt{set.mm} they have been distinguished artificially by prefixing their
labels\index{labels in \texttt{set.mm}} with \texttt{ax-} and \texttt{df-}
respectively.  For example, \texttt{df-an} defines conjunction (logical {\sc
and}), which is represented by the symbol \verb+/\+.
Section~\ref{definitions} discusses the philosophy of definitions, and the
Metamath language takes a particularly simple, conservative approach by using
the \texttt{\$a}\index{\texttt{\$a} statement} statement for both axioms and
definitions.

You can also have the program compute how many steps a proof
has\index{proof length} if we were to follow it all the way back to
\texttt{\$a} statements.

\begin{verbatim}
MM> show trace_back prth /essential/count_steps
The statement's actual proof has 3 steps.  Backtracking, a
total of 79 different subtheorems are used.  The statement
and subtheorems have a total of 274 actual steps.  If
subtheorems used only once were eliminated, there would be a
total of 38 subtheorems, and the statement and subtheorems
would have a total of 185 steps.  The proof would have 28349
steps if fully expanded back to axiom references.  The
maximum path length is 38.  A longest path is:  prth <-
anim12d <- syl2and <- sylan2d <- ancomsd <- ancom <- pm3.22
<- pm3.21 <- pm3.2 <- ex <- sylbir <- biimpri <- bicomi <-
bicom1 <- bi2 <- dfbi1 <- impbii <- bi3 <- simprim <- impi <-
con1i <- nsyl2 <- mt3d <- con1d <- notnot1 <- con2i <- nsyl3
<- mt2d <- con2d <- notnot2 <- pm2.18d <- pm2.18 <- pm2.21 <-
pm2.21d <- a1d <- syl <- mpd <- a2i <- a2i.1 .
\end{verbatim}

This tells us that we would have to inspect 274 steps if we want to
verify the proof completely starting from the axioms.  A few more
statistics are also shown.  There are one or more paths back to axioms
that are the longest; this command ferrets out one of them and shows it
to you.  There may be a sense in which the longest path length is
related to how ``deep'' the theorem is.

We might also be curious about what proofs depend on the theorem
\texttt{prth}.  If it is never used later on, we could eliminate it as
redundant if it has no intrinsic interest by itself.\index{\texttt{show
usage} command}

% I decided to show the OLD values here.
\begin{verbatim}
MM> show usage prth
Statement "prth" is directly referenced in the proofs of 18
statements:
  mo3 moOLD 2mo 2moOLD euind reuind reuss2 reusv3i opelopabt
  wemaplem2 rexanre rlimcn2 o1of2 o1rlimmul 2sqlem6 spanuni
  heicant pm11.71
\end{verbatim}

Thus \texttt{prth} is directly used by 18 proofs.
We can use the \texttt{/recursive} qualifier to include indirect use:

\begin{verbatim}
MM> show usage prth /recursive
Statement "prth" directly or indirectly affects the proofs of
24214 statements:
  mo3 mo mo3OLD eu2 moOLD eu2OLD eu3OLD mo4f mo4 eu4 mopick
...
\end{verbatim}

\subsection{A Note on the ``Compact'' Proof Format}

The Metamath program will display proofs in a ``compact''\index{compact proof}
format whenever the proof is stored in compressed format in the database.  It
may be be slightly confusing unless you know how to interpret it.
For example,
if you display the complete proof of theorem \texttt{id1} it will start
off as follows:

\begin{verbatim}
MM> show proof id1 /lemmon/all
 1 wph           $f wff ph
 2 wph           $f wff ph
 3 wph           $f wff ph
 4 2,3 wi    @4: $a wff ( ph -> ph )
 5 1,4 wi    @5: $a wff ( ph -> ( ph -> ph ) )
 6 @4            $a wff ( ph -> ph )
\end{verbatim}

\begin{center}
{etc.}
\end{center}

Step 4 has a ``local label,''\index{local label} \texttt{@4}, assigned to it.
Later on, at step 6, the label \texttt{@1} is referenced instead of
displaying the explicit proof for that step.  This technique takes advantage
of the fact that steps in a proof often repeat, especially during the
construction of wffs.  The compact format reduces the number of steps in the
proof display and may be preferred by some people.

If you want to see the normal format with the ``true'' step numbers, you can
use the following workaround:\index{\texttt{save proof} command}

\begin{verbatim}
MM> save proof id1 /normal
The proof of "id1" has been reformatted and saved internally.
Remember to use WRITE SOURCE to save it permanently.
MM> show proof id1 /lemmon/all
 1 wph           $f wff ph
 2 wph           $f wff ph
 3 wph           $f wff ph
 4 2,3 wi        $a wff ( ph -> ph )
 5 1,4 wi        $a wff ( ph -> ( ph -> ph ) )
 6 wph           $f wff ph
 7 wph           $f wff ph
 8 6,7 wi        $a wff ( ph -> ph )
\end{verbatim}

\begin{center}
{etc.}
\end{center}

Note that the original 6 steps are now 8 steps.  However, the format is
now the same as that described in Chapter~\ref{using}.

\chapter{The Metamath Language}
\label{languagespec}

\begin{quote}
  {\em Thus mathematics may be defined as the subject in which we never know
what we are talking about, nor whether what we are saying is true.}
    \flushright\sc  Bertrand Russell\footnote{\cite[p.~84]{Russell2}.}\\
\end{quote}\index{Russell, Bertrand}

Probably the most striking feature of the Metamath language is its almost
complete absence of hard-wired syntax. Metamath\index{Metamath} does not
understand any mathematics or logic other than that needed to construct finite
sequences of symbols according to a small set of simple, built-in rules.  The
only rule it uses in a proof is the substitution of an expression (symbol
sequence) for a variable, subject to a simple constraint to prevent
bound-variable clashes.  The primitive notions built into Metamath involve the
simple manipulation of finite objects (symbols) that we as humans can easily
visualize and that computers can easily deal with.  They seem to be just
about the simplest notions possible that are required to do standard
mathematics.

This chapter serves as a reference manual for the Metamath\index{Metamath}
language. It covers the tedious technical details of the language, some of
which you may wish to skim in a first reading.  On the other hand, you should
pay close attention to the defined terms in {\bf boldface}; they have precise
meanings that are important to keep in mind for later understanding.  It may
be best to first become familiar with the examples in Chapter~\ref{using} to
gain some motivation for the language.

%% Uncomment this when uncommenting section {formalspec} below
If you have some knowledge of set theory, you may wish to study this
chapter in conjunction with the formal set-theoretical description of the
Metamath language in Appendix~\ref{formalspec}.

We will use the name ``Metamath''\index{Metamath} to mean either the Metamath
computer language or the Metamath software associated with the computer
language.  We will not distinguish these two when the context is clear.

The next section contains the complete specification of the Metamath
language.
It serves as an
authoritative reference and presents the syntax in enough detail to
write a parser\index{parsing Metamath} and proof verifier.  The
specification is terse and it is probably hard to learn the language
directly from it, but we include it here for those impatient people who
prefer to see everything up front before looking at verbose expository
material.  Later sections explain this material and provide examples.
We will repeat the definitions in those sections, and you may skip the
next section at first reading and proceed to Section~\ref{tut1}
(p.~\pageref{tut1}).

\section{Specification of the Metamath Language}\label{spec}
\index{Metamath!specification}

\begin{quote}
  {\em Sometimes one has to say difficult things, but one ought to say
them as simply as one knows how.}
    \flushright\sc  G. H. Hardy\footnote{As quoted in
    \cite{deMillo}, p.~273.}\\
\end{quote}\index{Hardy, G. H.}

\subsection{Preliminaries}\label{spec1}

% Space is technically a printable character, so we'll word things
% carefully so it's unambiguous.
A Metamath {\bf database}\index{database} is built up from a top-level source
file together with any source files that are brought in through file inclusion
commands (see below).  The only characters that are allowed to appear in a
Metamath source file are the 94 non-whitespace printable {\sc
ascii}\index{ascii@{\sc ascii}} characters, which are digits, upper and lower
case letters, and the following 32 special
characters\index{special characters}:\label{spec1chars}

\begin{verbatim}
! " # $ % & ' ( ) * + , - . / :
; < = > ? @ [ \ ] ^ _ ` { | } ~
\end{verbatim}
plus the following characters which are the ``white space'' characters:
space (a printable character),
tab, carriage return, line feed, and form feed.\label{whitespace}
We will use \texttt{typewriter}
font to display the printable characters.

A Metamath database consists of a sequence of three kinds of {\bf
tokens}\index{token} separated by {\bf white space}\index{white space}
(which is any sequence of one or more white space characters).  The set
of {\bf keyword}\index{keyword} tokens is \texttt{\$\char`\{},
\texttt{\$\char`\}}, \texttt{\$c}, \texttt{\$v}, \texttt{\$f},
\texttt{\$e}, \texttt{\$d}, \texttt{\$a}, \texttt{\$p}, \texttt{\$.},
\texttt{\$=}, \texttt{\$(}, \texttt{\$)}, \texttt{\$[}, and
\texttt{\$]}.  The last four are called {\bf auxiliary}\index{auxiliary
keyword} or preprocessing keywords.  A {\bf label}\index{label} token
consists of any combination of letters, digits, and the characters
hyphen, underscore, and period.  A {\bf math symbol}\index{math symbol}
token may consist of any combination of the 93 printable standard {\sc
ascii} characters other than space or \texttt{\$}~. All tokens are
case-sensitive.

\subsection{Preprocessing}

The token \texttt{\$(} begins a {\bf comment} and
\texttt{\$)} ends a comment.\index{\texttt{\$(}
and \texttt{\$)} auxiliary keywords}\index{comment}
Comments may contain any of
the 94 non-whitespace printable characters and white space,
except they may not contain the
2-character sequences \texttt{\$(} or \texttt{\$)} (comments do not nest).
Comments are ignored (treated
like white space) for the purpose of parsing, e.g.,
\texttt{\$( \$[ \$)} is a comment.
See p.~\pageref{mathcomments} for comment typesetting conventions; these
conventions may be ignored for the purpose of parsing.

A {\bf file inclusion command} consists of \texttt{\$[} followed by a file name
followed by \texttt{\$]}.\index{\texttt{\$[} and \texttt{\$]} auxiliary
keywords}\index{included file}\index{file inclusion}
It is only allowed in the outermost scope (i.e., not between
\texttt{\$\char`\{} and \texttt{\$\char`\}})
and must not be inside a statement (e.g., it may not occur
between the label of a \texttt{\$a} statement and its \texttt{\$.}).
The file name may not
contain a \texttt{\$} or white space.  The file must exist.
The case-sensitivity
of its name follows the conventions of the operating system.  The contents of
the file replace the inclusion command.
Included files may include other files.
Only the first reference to a given file is included; any later
references to the same file (whether in the top-level file or in included
files) cause the inclusion command to be ignored (treated like white space).
A verifier may assume that file names with different strings
refer to different files for the purpose of ignoring later references.
A file self-reference is ignored, as is any reference to the top-level file
(to avoid loops).
Included files may not include a \texttt{\$(} without a matching \texttt{\$)},
may not include a \texttt{\$[} without a matching \texttt{\$]}, and may
not include incomplete statements (e.g., a \texttt{\$a} without a matching
\texttt{\$.}).
It is currently unspecified if path references are relative to the process'
current directory or the file's containing directory, so databases should
avoid using pathname separators (e.g., ``/'') in file names.

Like all tokens, the \texttt{\$(}, \texttt{\$)}, \texttt{\$[}, and \texttt{\$]} keywords
must be surrounded by white space.

\subsection{Basic Syntax}

After preprocessing, a database will consist of a sequence of {\bf
statements}.
These are the scoping statements \texttt{\$\char`\{} and
\texttt{\$\char`\}}, along with the \texttt{\$c}, \texttt{\$v},
\texttt{\$f}, \texttt{\$e}, \texttt{\$d}, \texttt{\$a}, and \texttt{\$p}
statements.

A {\bf scoping statement}\index{scoping statement} consists only of its
keyword, \texttt{\$\char`\{} or \texttt{\$\char`\}}.
A \texttt{\$\char`\{} begins a {\bf
block}\index{block} and a matching \texttt{\$\char`\}} ends the block.
Every \texttt{\$\char`\{}
must have a matching \texttt{\$\char`\}}.
Defining it recursively, we say a block
contains a sequence of zero or more tokens other
than \texttt{\$\char`\{} and \texttt{\$\char`\}} and
possibly other blocks.  There is an {\bf outermost
block}\index{block!outermost} not bracketed by \texttt{\$\char`\{} \texttt{\$\char`\}}; the end
of the outermost block is the end of the database.

% LaTeX bug? can't do \bf\tt

A {\bf \$v} or {\bf \$c statement}\index{\texttt{\$v} statement}\index{\texttt{\$c}
statement} consists of the keyword token \texttt{\$v} or \texttt{\$c} respectively,
followed by one or more math symbols,
% The word "token" is used to distinguish "$." from the sentence-ending period.
followed by the \texttt{\$.}\ token.
These
statements {\bf declare}\index{declaration} the math symbols to be {\bf
variables}\index{variable!Metamath} or {\bf constants}\index{constant}
respectively. The same math symbol may not occur twice in a given \texttt{\$v} or
\texttt{\$c} statement.

%c%A math symbol becomes an {\bf active}\index{active math symbol}
%c%when declared and stays active until the end of the block in which it is
%c%declared.  A math symbol may not be declared a second time while it is active,
%c%but it may be declared again after it becomes inactive.

A math symbol becomes {\bf active}\index{active math symbol} when declared
and stays active until the end of the block in which it is declared.  A
variable may not be declared a second time while it is active, but it
may be declared again (as a variable, but not as a constant) after it
becomes inactive.  A constant must be declared in the outermost block and may
not be declared a second time.\index{redeclaration of symbols}

A {\bf \$f statement}\index{\texttt{\$f} statement} consists of a label,
followed by \texttt{\$f}, followed by its typecode (an active constant),
followed by an
active variable, followed by the \texttt{\$.}\ token.  A {\bf \$e
statement}\index{\texttt{\$e} statement} consists of a label, followed
by \texttt{\$e}, followed by its typecode (an active constant),
followed by zero or more
active math symbols, followed by the \texttt{\$.}\ token.  A {\bf
hypothesis}\index{hypothesis} is a \texttt{\$f} or \texttt{\$e}
statement.
The type declared by a \texttt{\$f} statement for a given label
is global even if the variable is not
(e.g., a database may not have \texttt{wff P} in one local scope
and \texttt{class P} in another).

A {\bf simple \$d statement}\index{\texttt{\$d} statement!simple}
consists of \texttt{\$d}, followed by two different active variables,
followed by the \texttt{\$.}\ token.  A {\bf compound \$d
statement}\index{\texttt{\$d} statement!compound} consists of
\texttt{\$d}, followed by three or more variables (all different),
followed by the \texttt{\$.}\ token.  The order of the variables in a
\texttt{\$d} statement is unimportant.  A compound \texttt{\$d}
statement is equivalent to a set of simple \texttt{\$d} statements, one
for each possible pair of variables occurring in the compound
\texttt{\$d} statement.  Henceforth in this specification we shall
assume all \texttt{\$d} statements are simple.  A \texttt{\$d} statement
is also called a {\bf disjoint} (or {\bf distinct}) {\bf variable
restriction}.\index{disjoint-variable restriction}

A {\bf \$a statement}\index{\texttt{\$a} statement} consists of a label,
followed by \texttt{\$a}, followed by its typecode (an active constant),
followed by
zero or more active math symbols, followed by the \texttt{\$.}\ token.  A {\bf
\$p statement}\index{\texttt{\$p} statement} consists of a label,
followed by \texttt{\$p}, followed by its typecode (an active constant),
followed by
zero or more active math symbols, followed by \texttt{\$=}, followed by
a sequence of labels, followed by the \texttt{\$.}\ token.  An {\bf
assertion}\index{assertion} is a \texttt{\$a} or \texttt{\$p} statement.

A \texttt{\$f}, \texttt{\$e}, or \texttt{\$d} statement is {\bf active}\index{active
statement} from the place it occurs until the end of the block it occurs in.
A \texttt{\$a} or \texttt{\$p} statement is {\bf active} from the place it occurs
through the end of the database.
There may not be two active \texttt{\$f} statements containing the same
variable.  Each variable in a \texttt{\$e}, \texttt{\$a}, or
\texttt{\$p} statement must exist in an active \texttt{\$f}
statement.\footnote{This requirement can greatly simplify the
unification algorithm (substitution calculation) required by proof
verification.}

%The label that begins each \texttt{\$f}, \texttt{\$e}, \texttt{\$a}, and
%\texttt{\$p} statement must be unique.
Each label token must be unique, and
no label token may match any math symbol
token.\label{namespace}\footnote{This
restriction was added on June 24, 2006.
It is not theoretically necessary but is imposed to make it easier to
write certain parsers.}

The set of {\bf mandatory variables}\index{mandatory variable} associated with
an assertion is the set of (zero or more) variables in the assertion and in any
active \texttt{\$e} statements.  The (possibly empty) set of {\bf mandatory
hypotheses}\index{mandatory hypothesis} is the set of all active \texttt{\$f}
statements containing mandatory variables, together with all active \texttt{\$e}
statements.
The set of {\bf mandatory {\bf \$d} statements}\index{mandatory
disjoint-variable restriction} associated with an assertion are those active
\texttt{\$d} statements whose variables are both among the assertion's
mandatory variables.

\subsection{Proof Verification}\label{spec4}

The sequence of labels between the \texttt{\$=} and \texttt{\$.}\ tokens
in a \texttt{\$p} statement is a {\bf proof}.\index{proof!Metamath} Each
label in a proof must be the label of an active statement other than the
\texttt{\$p} statement itself; thus a label must refer either to an
active hypothesis of the \texttt{\$p} statement or to an earlier
assertion.

An {\bf expression}\index{expression} is a sequence of math symbols. A {\bf
substitution map}\index{substitution map} associates a set of variables with a
set of expressions.  It is acceptable for a variable to be mapped to an
expression containing it.  A {\bf
substitution}\index{substitution!variable}\index{variable substitution} is the
simultaneous replacement of all variables in one or more expressions with the
expressions that the variables map to.

A proof is scanned in order of its label sequence.  If the label refers to an
active hypothesis, the expression in the hypothesis is pushed onto a
stack.\index{stack}\index{RPN stack}  If the label refers to an assertion, a
(unique) substitution must exist that, when made to the mandatory hypotheses
of the referenced assertion, causes them to match the topmost (i.e.\ most
recent) entries of the stack, in order of occurrence of the mandatory
hypotheses, with the topmost stack entry matching the last mandatory
hypothesis of the referenced assertion.  As many stack entries as there are
mandatory hypotheses are then popped from the stack.  The same substitution is
made to the referenced assertion, and the result is pushed onto the stack.
After the last label in the proof is processed, the stack must have a single
entry that matches the expression in the \texttt{\$p} statement containing the
proof.

%c%{\footnotesize\begin{quotation}\index{redeclaration of symbols}
%c%{{\em Comment.}\label{spec4comment} Whenever a math symbol token occurs in a
%c%{\texttt{\$c} or \texttt{\$v} statement, it is considered to designate a distinct new
%c%{symbol, even if the same token was previously declared (and is now inactive).
%c%{Thus a math token declared as a constant in two different blocks is considered
%c%{to designate two distinct constants (even though they have the same name).
%c%{The two constants will not match in a proof that references both blocks.
%c%{However, a proof referencing both blocks is acceptable as long as it doesn't
%c%{require that the constants match.  Similarly, a token declared to be a
%c%{constant for a referenced assertion will not match the same token declared to
%c%{be a variable for the \texttt{\$p} statement containing the proof.  In the case
%c%{of a token declared to be a variable for a referenced assertion, this is not
%c%{an issue since the variable can be substituted with whatever expression is
%c%{needed to achieve the required match.
%c%{\end{quotation}}
%c2%A proof may reference an assertion that contains or whose hypotheses contain a
%c2%constant that is not active for the \texttt{\$p} statement containing the proof.
%c2%However, the final result of the proof may not contain that constant. A proof
%c2%may also reference an assertion that contains or whose hypotheses contain a
%c2%variable that is not active for the \texttt{\$p} statement containing the proof.
%c2%That variable, of course, will be substituted with whatever expression is
%c2%needed to achieve the required match.

A proof may contain a \texttt{?}\ in place of a label to indicate an unknown step
(Section~\ref{unknown}).  A proof verifier may ignore any proof containing
\texttt{?}\ but should warn the user that the proof is incomplete.

A {\bf compressed proof}\index{compressed proof}\index{proof!compressed} is an
alternate proof notation described in Appen\-dix~\ref{compressed}; also see
references to ``compressed proof'' in the Index.  Compressed proofs are a
Metamath language extension which a complete proof verifier should be able to
parse and verify.

\subsubsection{Verifying Disjoint Variable Restrictions}

Each substitution made in a proof must be checked to verify that any
disjoint variable restrictions are satisfied, as follows.

If two variables replaced by a substitution exist in a mandatory \texttt{\$d}
statement\index{\texttt{\$d} statement} of the assertion referenced, the two
expressions resulting from the substitution must satisfy the following
conditions.  First, the two expressions must have no variables in common.
Second, each possible pair of variables, one from each expression, must exist
in an active \texttt{\$d} statement of the \texttt{\$p} statement containing the
proof.

\vskip 1ex

This ends the specification of the Metamath language;
see Appendix \ref{BNF} for its syntax in
Extended Backus--Naur Form (EBNF)\index{Extended Backus--Naur Form}\index{EBNF}.

\section{The Basic Keywords}\label{tut1}

Our expository material begins here.

Like most computer languages, Metamath\index{Metamath} takes its input from
one or more {\bf source files}\index{source file} which contain characters
expressed in the standard {\sc ascii} (American Standard Code for Information
Interchange)\index{ascii@{\sc ascii}} code for computers.  A source file
consists of a series of {\bf tokens}\index{token}, which are strings of
non-whitespace
printable characters (from the set of 94 shown on p.~\pageref{spec1chars})
separated by {\bf white space}\index{white space} (spaces, tabs, carriage
returns, line feeds, and form feeds). Any string consisting only of these
characters is treated the same as a single space.  The non-whitespace printable
characters\index{printable character} that Metamath recognizes are the 94
characters on standard {\sc ascii} keyboards.

Metamath has the ability to join several files together to form its
input (Section~\ref{include}).  We call the aggregate contents of all
the files after they have been joined together a {\bf
database}\index{database} to distinguish it from an individual source
file.  The tokens in a database consist of {\bf
keywords}\index{keyword}, which are built into the language, together
with two kinds of user-defined tokens called {\bf labels}\index{label}
and {\bf math symbols}\index{math symbol}.  (Often we will simply say
{\bf symbol}\index{symbol} instead of math symbol for brevity).  The set
of {\bf basic keywords}\index{basic keyword} is
\texttt{\$c}\index{\texttt{\$c} statement},
\texttt{\$v}\index{\texttt{\$v} statement},
\texttt{\$e}\index{\texttt{\$e} statement},
\texttt{\$f}\index{\texttt{\$f} statement},
\texttt{\$d}\index{\texttt{\$d} statement},
\texttt{\$a}\index{\texttt{\$a} statement},
\texttt{\$p}\index{\texttt{\$p} statement},
\texttt{\$=}\index{\texttt{\$=} keyword},
\texttt{\$.}\index{\texttt{\$.}\ keyword},
\texttt{\$\char`\{}\index{\texttt{\$\char`\{} and \texttt{\$\char`\}}
keywords}, and \texttt{\$\char`\}}.  This is the complete set of
syntactical elements of what we call the {\bf basic
language}\index{basic language} of Metamath, and with them you can
express all of the mathematics that were intended by the design of
Metamath.  You should make it a point to become very familiar with them.
Table~\ref{basickeywords} lists the basic keywords along with a brief
description of their functions.  For now, this description will give you
only a vague notion of what the keywords are for; later we will describe
the keywords in detail.


\begin{table}[htp] \caption{Summary of the basic Metamath
keywords} \label{basickeywords}
\begin{center}
\begin{tabular}{|p{4pc}|l|}
\hline
\em \centering Keyword&\em Description\\
\hline
\hline
\centering
   \texttt{\$c}&Constant symbol declaration\\
\hline
\centering
   \texttt{\$v}&Variable symbol declaration\\
\hline
\centering
   \texttt{\$d}&Disjoint variable restriction\\
\hline
\centering
   \texttt{\$f}&Variable-type (``floating'') hypothesis\\
\hline
\centering
   \texttt{\$e}&Logical (``essential'') hypothesis\\
\hline
\centering
   \texttt{\$a}&Axiomatic assertion\\
\hline
\centering
   \texttt{\$p}&Provable assertion\\
\hline
\centering
   \texttt{\$=}&Start of proof in \texttt{\$p} statement\\
\hline
\centering
   \texttt{\$.}&End of the above statement types\\
\hline
\centering
   \texttt{\$\char`\{}&Start of block\\
\hline
\centering
   \texttt{\$\char`\}}&End of block\\
\hline
\end{tabular}
\end{center}
\end{table}

%For LaTeX bug(?) where it puts tables on blank page instead of btwn text
%May have to adjust if text changes
%\newpage

There are some additional keywords, called {\bf auxiliary
keywords}\index{auxiliary keyword} that help make Metamath\index{Metamath}
more practical. These are part of the {\bf extended language}\index{extended
language}. They provide you with a means to put comments into a Metamath
source file\index{source file} and reference other source files.  We will
introduce these in later sections. Table~\ref{otherkeywords} summarizes them
so that you can recognize them now if you want to peruse some source
files while learning the basic keywords.


\begin{table}[htp] \caption{Auxiliary Metamath
keywords} \label{otherkeywords}
\begin{center}
\begin{tabular}{|p{4pc}|l|}
\hline
\em \centering Keyword&\em Description\\
\hline
\hline
\centering
   \texttt{\$(}&Start of comment\\
\hline
\centering
   \texttt{\$)}&End of comment\\
\hline
\centering
   \texttt{\$[}&Start of included source file name\\
\hline
\centering
   \texttt{\$]}&End of included source file name\\
\hline
\end{tabular}
\end{center}
\end{table}
\index{\texttt{\$(} and \texttt{\$)} auxiliary keywords}
\index{\texttt{\$[} and \texttt{\$]} auxiliary keywords}


Unlike those in some computer languages, the keywords\index{keyword} are short
two-character sequences rather than English-like words.  While this may make
them slightly more difficult to remember at first, their brevity allows
them to blend in with the mathematics being described, not
distract from it, like punctuation marks.


\subsection{User-Defined Tokens}\label{dollardollar}\index{token}

As you may have noticed, all keywords\index{keyword} begin with the \texttt{\$}
character.  This mundane monetary symbol is not ordinarily used in higher
mathematics (outside of grant proposals), so we have appropriated it to
distinguish the Metamath\index{Metamath} keywords from ordinary mathematical
symbols. The \texttt{\$} character is thus considered special and may not be
used as a character in a user-defined token.  All tokens and keywords are
case-sensitive; for example, \texttt{n} is considered to be a different character
from \texttt{N}.  Case-sensitivity makes the available {\sc ascii} character set
as rich as possible.

\subsubsection{Math Symbol Tokens}\index{token}

Math symbols\index{math symbol} are tokens used to represent the symbols
that appear in ordinary mathematical formulas.  They may consist of any
combination of the 93 non-whitespace printable {\sc ascii} characters other than
\texttt{\$}~. Some examples are \texttt{x}, \texttt{+}, \texttt{(},
\texttt{|-}, \verb$!%@?&$, and \texttt{bounded}.  For readability, it is
best to try to make these look as similar to actual mathematical symbols
as possible, within the constraints of the {\sc ascii} character set, in
order to make the resulting mathematical expressions more readable.

In the Metamath\index{Metamath} language, you express ordinary
mathematical formulas and statements as sequences of math symbols such
as \texttt{2 + 2 = 4} (five symbols, all constants).\footnote{To
eliminate ambiguity with other expressions, this is expressed in the set
theory database \texttt{set.mm} as \texttt{|- ( 2 + 2
 ) = 4 }, whose \LaTeX\ equivalent is $\vdash
(2+2)=4$.  The \,$\vdash$ means ``is a theorem'' and the
parentheses allow explicit associative grouping.}\index{turnstile
({$\,\vdash$})} They may even be English
sentences, as in \texttt{E is closed and bounded} (five symbols)---here
\texttt{E} would be a variable and the other four symbols constants.  In
principle, a Metamath database could be constructed to work with almost
any unambiguous English-language mathematical statement, but as a
practical matter the definitions needed to provide for all possible
syntax variations would be cumbersome and distracting and possibly have
subtle pitfalls accidentally built in.  We generally recommend that you
express mathematical statements with compact standard mathematical
symbols whenever possible and put their English-language descriptions in
comments.  Axioms\index{axiom} and definitions\index{definition}
(\texttt{\$a}\index{\texttt{\$a} statement} statements) are the only
places where Metamath will not detect an error, and doing this will help
reduce the number of definitions needed.

You are free to use any tokens\index{token} you like for math
symbols\index{math symbol}.  Appendix~\ref{ASCII} recommends token names to
use for symbols in set theory, and we suggest you adopt these in order to be
able to include the \texttt{set.mm} set theory database in your database.  For
printouts, you can convert the tokens in a database
to standard mathematical symbols with the \LaTeX\ typesetting program.  The
Metamath command \texttt{open tex} {\em filename}\index{\texttt{open tex} command}
produces output that can be read by \LaTeX.\index{latex@{\LaTeX}}
The correspondence
between tokens and the actual symbols is made by \texttt{latexdef}
statements inside a special database comment tagged
with \texttt{\$t}.\index{\texttt{\$t} comment}\index{typesetting comment}
  You can edit
this comment to change the definitions or add new ones.
Appendix~\ref{ASCII} describes how to do this in more detail.

% White space\index{white space} is normally used to separate math
% symbol\index{math symbol} tokens, but they may be juxtaposed without white
% space in \texttt{\$d}\index{\texttt{\$d} statement}, \texttt{\$e}\index{\texttt{\$e}
% statement}, \texttt{\$f}\index{\texttt{\$f} statement}, \texttt{\$a}\index{\texttt{\$a}
% statement}, and \texttt{\$p}\index{\texttt{\$p} statement} statements when no
% ambiguity will result.  Specifically, Metamath parses the math symbol sequence
% in one of these statements in the following manner:  when the math symbol
% sequence has been broken up into tokens\index{token} up to a given character,
% the next token is the longest string of characters that could constitute a
% math symbol that is active\index{active
% math symbol} at that point.  (See Section~\ref{scoping} for the
% definition of an active math symbol.)  For example, if \texttt{-}, \texttt{>}, and
% \texttt{->} are the only active math symbols, the juxtaposition \texttt{>-} will be
% interpreted as the two symbols \texttt{>} and \texttt{-}, whereas \texttt{->} will
% always be interpreted as that single symbol.\footnote{For better readability we
% recommend a white space between each token.  This also makes searching for a
% symbol easier to do with an editor.  Omission of optional white space is useful
% for reducing typing when assigning an expression to a temporary
% variable\index{temporary variable} with the \texttt{let variable} Metamath
% program command.}\index{\texttt{let variable} command}
%
% Keywords\index{keyword} may be placed next to math symbols without white
% space\index{white space} between them.\footnote{Again, we do not recommend
% this for readability.}
%
% The math symbols\index{math symbol} in \texttt{\$c}\index{\texttt{\$c} statement}
% and \texttt{\$v}\index{\texttt{\$v} statement} statements must always be separated
% by white space\index{white
% space}, for the obvious reason that these statements define the names
% of the symbols.
%
% Math symbols referred to in comments (see Section~\ref{comments}) must also be
% separated by white space.  This allows you to make comments about symbols that
% are not yet active\index{active
% math symbol}.  (The ``math mode'' feature of comments is also a quick and
% easy way to obtain word processing text with embedded mathematical symbols,
% independently of the main purpose of Metamath; the way to do this is described
% in Section~\ref{comments})

\subsubsection{Label Tokens}\index{token}\index{label}

Label tokens are used to identify Metamath\index{Metamath} statements for
later reference. Label tokens may contain only letters, digits, and the three
characters period, hyphen, and underscore:
\begin{verbatim}
. - _
\end{verbatim}

A label is {\bf declared}\index{label declaration} by placing it immediately
before the keyword of the statement it identifies.  For example, the label
\texttt{axiom.1} might be declared as follows:
\begin{verbatim}
axiom.1 $a |- x = x $.
\end{verbatim}

Each \texttt{\$e}\index{\texttt{\$e} statement},
\texttt{\$f}\index{\texttt{\$f} statement},
\texttt{\$a}\index{\texttt{\$a} statement}, and
\texttt{\$p}\index{\texttt{\$p} statement} statement in a database must
have a label declared for it.  No other statement types may have label
declarations.  Every label must be unique.

A label (and the statement it identifies) is {\bf referenced}\index{label
reference} by including the label between the \texttt{\$=}\index{\texttt{\$=}
keyword} and \texttt{\$.}\index{\texttt{\$.}\ keyword}\ keywords in a \texttt{\$p}
statement.  The sequence of labels\index{label sequence} between these two
keywords is called a {\bf proof}\index{proof}.  An example of a statement with
a proof that we will encounter later (Section~\ref{proof}) is
\begin{verbatim}
wnew $p wff ( s -> ( r -> p ) )
     $= ws wr wp w2 w2 $.
\end{verbatim}

You don't have to know what this means just yet, but you should know that the
label \texttt{wnew} is declared by this \texttt{\$p} statement and that the labels
\texttt{ws}, \texttt{wr}, \texttt{wp}, and \texttt{w2} are assumed to have been declared
earlier in the database and are referenced here.

\subsection{Constants and Variables}
\index{constant}
\index{variable}

An {\bf expression}\index{expression} is any sequence of math
symbols, possibly empty.

The basic Metamath\index{Metamath} language\index{basic language} has two
kinds of math symbols\index{math symbol}:  {\bf constants}\index{constant} and
{\bf variables}\index{variable}.  In a Metamath proof, a constant may not be
substituted with any expression.  A variable can be
substituted\index{substitution!variable}\index{variable substitution} with any
expression.  This sequence may include other variables and may even include
the variable being substituted.  This substitution takes place when proofs are
verified, and it will be described in Section~\ref{proof}.  The \texttt{\$f}
statement (described later in Section~\ref{dollaref}) is used to specify the
{\bf type} of a variable (i.e.\ what kind of
variable it is)\index{variable type}\index{type} and
give it a meaning typically
associated with a ``metavariable''\index{metavariable}\footnote{A metavariable
is a variable that ranges over the syntactical elements of the object language
being discussed; for example, one metavariable might represent a variable of
the object language and another metavariable might represent a formula in the
object language.} in ordinary mathematics; for example, a variable may be
specified to be a wff or well-formed formula (in logic), a set (in set
theory), or a non-negative integer (in number theory).

%\subsection{The \texttt{\$c} and \texttt{\$v} Declaration Statements}
\subsection{The \texttt{\$c} and \texttt{\$v} Declaration Statements}
\index{\texttt{\$c} statement}
\index{constant declaration}
\index{\texttt{\$v} statement}
\index{variable declaration}

Constants are introduced or {\bf declared}\index{constant declaration}
with \texttt{\$c}\index{\texttt{\$c} statement} statements, and
variables are declared\index{variable declaration} with
\texttt{\$v}\index{\texttt{\$v} statement} statements.  A {\bf simple}
declaration\index{simple declaration} statement introduces a single
constant or variable.  Its syntax is one of the following:
\begin{center}
  \texttt{\$c} {\em math-symbol} \texttt{\$.}\\
  \texttt{\$v} {\em math-symbol} \texttt{\$.}
\end{center}
The notation {\em math-symbol} means any math symbol token\index{token}.

Some examples of simple declaration statements are:
\begin{center}
  \texttt{\$c + \$.}\\
  \texttt{\$c -> \$.}\\
  \texttt{\$c ( \$.}\\
  \texttt{\$v x \$.}\\
  \texttt{\$v y2 \$.}
\end{center}

The characters in a math symbol\index{math symbol} being declared are
irrelevant to Meta\-math; for example, we could declare a right parenthesis to
be a variable,
\begin{center}
  \texttt{\$v ) \$.}\\
\end{center}
although this would be unconventional.

A {\bf compound} declaration\index{compound declaration} statement is a
shorthand for declaring several symbols at once.  Its syntax is one of the
following:
\begin{center}
  \texttt{\$c} {\em math-symbol}\ \,$\cdots$\ {\em math-symbol} \texttt{\$.}\\
  \texttt{\$v} {\em math-symbol}\ \,$\cdots$\ {\em math-symbol} \texttt{\$.}
\end{center}\index{\texttt{\$c} statement}
Here, the ellipsis (\ldots) means any number of {\em math-symbol}\,s.

An example of a compound declaration statement is:
\begin{center}
  \texttt{\$v x y mu \$.}\\
\end{center}
This is equivalent to the three simple declaration statements
\begin{center}
  \texttt{\$v x \$.}\\
  \texttt{\$v y \$.}\\
  \texttt{\$v mu \$.}\\
\end{center}
\index{\texttt{\$v} statement}

There are certain rules on where in the database math symbols may be declared,
what sections of the database are aware of them (i.e.\ where they are
``active''), and when they may be declared more than once.  These will be
discussed in Section~\ref{scoping} and specifically on
p.~\pageref{redeclaration}.

\subsection{The \texttt{\$d} Statement}\label{dollard}
\index{\texttt{\$d} statement}

The \texttt{\$d} statement is called a {\bf disjoint-variable restriction}.  The
syntax of the {\bf simple} version of this statement is
\begin{center}
  \texttt{\$d} {\em variable variable} \texttt{\$.}
\end{center}
where each {\em variable} is a previously declared variable and the two {\em
variable}\,s are different.  (More specifically, each  {\em variable} must be
an {\bf active} variable\index{active math symbol}, which means there must be
a previous \texttt{\$v} statement whose {\bf scope}\index{scope} includes the
\texttt{\$d} statement.  These terms will be defined when we discuss scoping
statements in Section~\ref{scoping}.)

In ordinary mathematics, formulas may arise that are true if the variables in
them are distinct\index{distinct variables}, but become false when those
variables are made identical. For example, the formula in logic $\exists x\,x
\neq y$, which means ``for a given $y$, there exists an $x$ that is not equal
to $y$,'' is true in most mathematical theories (namely all non-trivial
theories\index{non-trivial theory}, i.e.\ those that describe more than one
individual, such as arithmetic).  However, if we substitute $y$ with $x$, we
obtain $\exists x\,x \neq x$, which is always false, as it means ``there
exists something that is not equal to itself.''\footnote{If you are a
logician, you will recognize this as the improper substitution\index{proper
substitution}\index{substitution!proper} of a free variable\index{free
variable} with a bound variable\index{bound variable}.  Metamath makes no
inherent distinction between free and bound variables; instead, you let
Metamath know what substitutions are permissible by using \texttt{\$d} statements
in the right way in your axiom system.}\index{free vs.\ bound variable}  The
\texttt{\$d} statement allows you to specify a restriction that forbids the
substitution of one variable with another.  In
this case, we would use the statement
\begin{center}
  \texttt{\$d x y \$.}
\end{center}\index{\texttt{\$d} statement}
to specify this restriction.

The order in which the variables appear in a \texttt{\$d} statement is not
important.  We could also use
\begin{center}
  \texttt{\$d y x \$.}
\end{center}

The \texttt{\$d} statement is actually more general than this, as the
``disjoint''\index{disjoint variables} in its name suggests.  The full meaning
is that if any substitution is made to its two variables (during the
course of a proof that references a \texttt{\$a} or \texttt{\$p} statement
associated with the \texttt{\$d}), the two expressions that result from the
substitution must have no variables in common.  In addition, each possible
pair of variables, one from each expression, must be in a \texttt{\$d} statement
associated with the statement being proved.  (This requirement forces the
statement being proved to ``inherit'' the original disjoint variable
restriction.)

For example, suppose \texttt{u} is a variable.  If the restriction
\begin{center}
  \texttt{\$d A B \$.}
\end{center}
has been specified for a theorem referenced in a
proof, we may not substitute \texttt{A} with \mbox{\tt a + u} and
\texttt{B} with \mbox{\tt b + u} because these two symbol sequences have the
variable \texttt{u} in common.  Furthermore, if \texttt{a} and \texttt{b} are
variables, we may not substitute \texttt{A} with \texttt{a} and \texttt{B} with \texttt{b}
unless we have also specified \texttt{\$d a b} for the theorem being proved; in
other words, the \texttt{\$d} property associated with a pair of variables must
be effectively preserved after substitution.

The \texttt{\$d}\index{\texttt{\$d} statement} statement does {\em not} mean ``the
two variables may not be substituted with the same thing,'' as you might think
at first.  For example, substituting each of \texttt{A} and \texttt{B} in the above
example with identical symbol sequences consisting only of constants does not
cause a disjoint variable conflict, because two symbol sequences have no
variables in common (since they have no variables, period).  Similarly, a
conflict will not occur by substituting the two variables in a \texttt{\$d}
statement with the empty symbol sequence\index{empty substitution}.

The \texttt{\$d} statement does not have a direct counterpart in
ordinary mathematics, partly because the variables\index{variable} of
Metamath are not really the same as the variables\index{variable!in
ordinary mathematics} of ordinary mathematics but rather are
metavariables\index{metavariable} ranging over them (as well as over
other kinds of symbols and groups of symbols).  Depending on the
situation, we may informally interpret the \texttt{\$d} statement in
different ways.  Suppose, for example, that \texttt{x} and \texttt{y}
are variables ranging over numbers (more precisely, that \texttt{x} and
\texttt{y} are metavariables ranging over variables that range over
numbers), and that \texttt{ph} ($\varphi$) and \texttt{ps} ($\psi$) are
variables (more precisely, metavariables) ranging over formulas.  We can
make the following interpretations that correspond to the informal
language of ordinary mathematics:
\begin{quote}
\begin{tabbing}
\texttt{\$d x y \$.} means ``assume $x$ and $y$ are
distinct variables.''\\
\texttt{\$d x ph \$.} means ``assume $x$ does not
occur in $\varphi$.''\\
\texttt{\$d ph ps \$.} \=means ``assume $\varphi$ and
$\psi$ have no variables\\ \>in common.''
\end{tabbing}
\end{quote}\index{\texttt{\$d} statement}

\subsubsection{Compound \texttt{\$d} Statements}

The {\bf compound} version of the \texttt{\$d} statement is a shorthand for
specifying several variables whose substitutions must be pairwise disjoint.
Its syntax is:
\begin{center}
  \texttt{\$d} {\em variable}\ \,$\cdots$\ {\em variable} \texttt{\$.}
\end{center}\index{\texttt{\$d} statement}
Here, {\em variable} represents the token of a previously declared
variable (specifically, an active variable) and all {\em variable}\,s are
different.  The compound \texttt{\$d}
statement is internally broken up by Metamath into one simple \texttt{\$d}
statement for each possible pair of variables in the original \texttt{\$d}
statement.  For example,
\begin{center}
  \texttt{\$d w x y z \$.}
\end{center}
is equivalent to
\begin{center}
  \texttt{\$d w x \$.}\\
  \texttt{\$d w y \$.}\\
  \texttt{\$d w z \$.}\\
  \texttt{\$d x y \$.}\\
  \texttt{\$d x z \$.}\\
  \texttt{\$d y z \$.}
\end{center}

Two or more simple \texttt{\$d} statements specifying the same variable pair are
internally combined into a single \texttt{\$d} statement.  Thus the set of three
statements
\begin{center}
  \texttt{\$d x y \$.}
  \texttt{\$d x y \$.}
  \texttt{\$d y x \$.}
\end{center}
is equivalent to
\begin{center}
  \texttt{\$d x y \$.}
\end{center}

Similarly, compound \texttt{\$d} statements, after being internally broken up,
internally have their common variable pairs combined.  For example the
set of statements
\begin{center}
  \texttt{\$d x y A \$.}
  \texttt{\$d x y B \$.}
\end{center}
is equivalent to
\begin{center}
  \texttt{\$d x y \$.}
  \texttt{\$d x A \$.}
  \texttt{\$d y A \$.}
  \texttt{\$d x y \$.}
  \texttt{\$d x B \$.}
  \texttt{\$d y B \$.}
\end{center}
which is equivalent to
\begin{center}
  \texttt{\$d x y \$.}
  \texttt{\$d x A \$.}
  \texttt{\$d y A \$.}
  \texttt{\$d x B \$.}
  \texttt{\$d y B \$.}
\end{center}

Metamath\index{Metamath} automatically verifies that all \texttt{\$d}
restrictions are met whenever it verifies proofs.  \texttt{\$d} statements are
never referenced directly in proofs (this is why they do not have
labels\index{label}), but Metamath is always aware of which ones must be
satisfied (i.e.\ are active) and will notify you with an error message if any
violation occurs.

To illustrate how Metamath detects a missing \texttt{\$d}
statement, we will look at the following example from the
\texttt{set.mm} database.

\begin{verbatim}
$d x z $.  $d y z $.
$( Theorem to add distinct quantifier to atomic formula. $)
ax17eq $p |- ( x = y -> A. z x = y ) $=...
\end{verbatim}

This statement has the obvious requirement that $z$ must be
distinct\index{distinct variables} from $x$ in theorem \texttt{ax17eq} that
states $x=y \rightarrow \forall z \, x=y$ (well, obvious if you're a logician,
for otherwise we could conclude  $x=y \rightarrow \forall x \, x=y$, which is
false when the free variables $x$ and $y$ are equal).

Let's look at what happens if we edit the database to comment out this
requirement.

\begin{verbatim}
$( $d x z $. $) $d y z $.
$( Theorem to add distinct quantifier to atomic formula. $)
ax17eq $p |- ( x = y -> A. z x = y ) $=...
\end{verbatim}

When it tries to verify the proof, Metamath will tell you that \texttt{x} and
\texttt{z} must be disjoint, because one of its steps references an axiom or
theorem that has this requirement.

\begin{verbatim}
MM> verify proof ax17eq
ax17eq ?Error at statement 1918, label "ax17eq", type "$p":
      vz wal wi vx vy vz ax-13 vx vy weq vz vx ax-c16 vx vy
                                               ^^^^^
There is a disjoint variable ($d) violation at proof step 29.
Assertion "ax-c16" requires that variables "x" and "y" be
disjoint.  But "x" was substituted with "z" and "y" was
substituted with "x".  The assertion being proved, "ax17eq",
does not require that variables "z" and "x" be disjoint.
\end{verbatim}

We can see the substitutions into \texttt{ax-c16} with the following command.

\begin{verbatim}
MM> show proof ax17eq / detailed_step 29
Proof step 29:  pm2.61dd.2=ax-c16 $a |- ( A. z z = x -> ( x =
  y -> A. z x = y ) )
This step assigns source "ax-c16" ($a) to target "pm2.61dd.2"
($e).  The source assertion requires the hypotheses "wph"
($f, step 26), "vx" ($f, step 27), and "vy" ($f, step 28).
The parent assertion of the target hypothesis is "pm2.61dd"
($p, step 36).
The source assertion before substitution was:
    ax-c16 $a |- ( A. x x = y -> ( ph -> A. x ph ) )
The following substitutions were made to the source
assertion:
    Variable  Substituted with
     x         z
     y         x
     ph        x = y
The target hypothesis before substitution was:
    pm2.61dd.2 $e |- ( ph -> ch )
The following substitutions were made to the target
hypothesis:
    Variable  Substituted with
     ph        A. z z = x
     ch        ( x = y -> A. z x = y )
\end{verbatim}

The disjoint variable restrictions of \texttt{ax-c16} can be seen from the
\texttt{show state\-ment} command.  The line that begins ``\texttt{Its mandatory
dis\-joint var\-i\-able pairs are:}\ldots'' lists any \texttt{\$d} variable
pairs in brackets.

\begin{verbatim}
MM> show statement ax-c16/full
Statement 3033 is located on line 9338 of the file "set.mm".
"Axiom of Distinct Variables. ..."
  ax-c16 $a |- ( A. x x = y -> ( ph -> A. x ph ) ) $.
Its mandatory hypotheses in RPN order are:
  wph $f wff ph $.
  vx $f setvar x $.
  vy $f setvar y $.
Its mandatory disjoint variable pairs are:  <x,y>
The statement and its hypotheses require the variables:  x y
      ph
The variables it contains are:  x y ph
\end{verbatim}

Since Metamath will always detect when \texttt{\$d}\index{\texttt{\$d} statement}
statements are needed for a proof, you don't have to worry too much about
forgetting to put one in; it can always be added if you see the error message
above.  If you put in unnecessary \texttt{\$d} statements, the worst that could
happen is that your theorem might not be as general as it could be, and this
may limit its use later on.

On the other hand, when you introduce axioms (\texttt{\$a}\index{\texttt{\$a}
statement} statements), you must be very careful to properly specify the
necessary associated \texttt{\$d} statements since Metamath has no way of knowing
whether your axioms are correct.  For example, Metamath would have no idea
that \texttt{ax-c16}, which we are telling it is an axiom of logic, would lead to
contradictions if we omitted its associated \texttt{\$d} statement.

% This was previously a comment in footnote-sized type, but it can be
% hard to read this much text in a small size.
% As a result, it's been changed to normally-sized text.
\label{nodd}
You may wonder if it is possible to develop standard
mathematics in the Metamath language without the \texttt{\$d}\index{\texttt{\$d}
statement} statement, since it seems like a nuisance that complicates proof
verification. The \texttt{\$d} statement is not needed in certain subsets of
mathematics such as propositional calculus.  However, dummy
variables\index{dummy variable!eliminating} and their associated \texttt{\$d}
statements are impossible to avoid in proofs in standard first-order logic as
well as in the variant used in \texttt{set.mm}.  In fact, there is no upper bound to
the number of dummy variables that might be needed in a proof of a theorem of
first-order logic containing 3 or more variables, as shown by H.\
Andr\'{e}ka\index{Andr{\'{e}}ka, H.} \cite{Nemeti}.  A first-order system that
avoids them entirely is given in \cite{Megill}\index{Megill, Norman}; the
trick there is simply to embed harmlessly the necessary dummy variables into a
theorem being proved so that they aren't ``dummy'' anymore, then interpret the
resulting longer theorem so as to ignore the embedded dummy variables.  If
this interests you, the system in \texttt{set.mm} obtained from \texttt{ax-1}
through \texttt{ax-c14} in \texttt{set.mm}, and deleting \texttt{ax-c16} and \texttt{ax-5},
requires no \texttt{\$d} statements but is logically complete in the sense
described in \cite{Megill}.  This means it can prove any theorem of
first-order logic as long as we add to the theorem an antecedent that embeds
dummy and any other variables that must be distinct.  In a similar fashion,
axioms for set theory can be devised that
do not require distinct variable
provisos\index{Set theory without distinct variable provisos},
as explained at
\url{http://us.metamath.org/mpeuni/mmzfcnd.html}.
Together, these in principle allow all of
mathematics to be developed under Metamath without a \texttt{\$d} statement,
although the length of the resulting theorems will grow as more and
more dummy variables become required in their proofs.

\subsection{The \texttt{\$f}
and \texttt{\$e} Statements}\label{dollaref}
\index{\texttt{\$e} statement}
\index{\texttt{\$f} statement}
\index{floating hypothesis}
\index{essential hypothesis}
\index{variable-type hypothesis}
\index{logical hypothesis}
\index{hypothesis}

Metamath has two kinds of hypo\-theses, the \texttt{\$f}\index{\texttt{\$f}
statement} or {\bf variable-type} hypothesis and the \texttt{\$e} or {\bf logical}
hypo\-the\-sis.\index{\texttt{\$d} statement}\footnote{Strictly speaking, the
\texttt{\$d} statement is also a hypothesis, but it is never directly referenced
in a proof, so we call it a restriction rather than a hypothesis to lessen
confusion.  The checking for violations of \texttt{\$d} restrictions is automatic
and built into Metamath's proof-checking algorithm.} The letters \texttt{f} and
\texttt{e} stand for ``floating''\index{floating hypothesis} (roughly meaning
used only if relevant) and ``essential''\index{essential hypothesis} (meaning
always used) respectively, for reasons that will become apparent
when we discuss frames in
Section~\ref{frames} and scoping in Section~\ref{scoping}. The syntax of these
are as follows:
\begin{center}
  {\em label} \texttt{\$f} {\em typecode} {\em variable} \texttt{\$.}\\
  {\em label} \texttt{\$e} {\em typecode}
      {\em math-symbol}\ \,$\cdots$\ {\em math-symbol} \texttt{\$.}\\
\end{center}
\index{\texttt{\$e} statement}
\index{\texttt{\$f} statement}
A hypothesis must have a {\em label}\index{label}.  The expression in a
\texttt{\$e} hypothesis consists of a typecode (an active constant math symbol)
followed by a sequence
of zero or more math symbols. Each math symbol (including {\em constant}
and {\em variable}) must be a previously declared constant or variable.  (In
addition, each math symbol must be active, which will be covered when we
discuss scoping statements in Section~\ref{scoping}.)  You use a \texttt{\$f}
hypothesis to specify the
nature or {\bf type}\index{variable type}\index{type} of a variable (such as ``let $x$ be an
integer'') and use a \texttt{\$e} hypothesis to express a logical truth (such as
``assume $x$ is prime'') that must be established in order for an assertion
requiring it to also be true.

A variable must have its type specified in a \texttt{\$f} statement before
it may be used in a \texttt{\$e}, \texttt{\$a}, or \texttt{\$p}
statement.  There may be only one (active) \texttt{\$f} statement for a
given variable.  (``Active'' is defined in Section~\ref{scoping}.)

In ordinary mathematics, theorems\index{theorem} are often expressed in the
form ``Assume $P$; then $Q$,'' where $Q$ is a statement that you can derive
if you start with statement $P$.\index{free variable}\footnote{A stronger
version of a theorem like this would be the {\em single} formula $P\rightarrow
Q$ ($P$ implies $Q$) from which the weaker version above follows by the rule
of modus ponens in logic.  We are not discussing this stronger form here.  In
the weaker form, we are saying only that if we can {\em prove} $P$, then we can
{\em prove} $Q$.  In a logician's language, if $x$ is the only free variable
in $P$ and $Q$, the stronger form is equivalent to $\forall x ( P \rightarrow
Q)$ (for all $x$, $P$ implies $Q$), whereas the weaker form is equivalent to
$\forall x P \rightarrow \forall x Q$. The stronger form implies the weaker,
but not vice-versa.  To be precise, the weaker form of the theorem is more
properly called an ``inference'' rather than a theorem.}\index{inference}
In the
Metamath\index{Metamath} language, you would express mathematical statement
$P$ as a hypothesis (a \texttt{\$e} Metamath language statement in this case) and
statement $Q$ as a provable assertion (a \texttt{\$p}\index{\texttt{\$p} statement}
statement).

Some examples of hypotheses you might encounter in logic and set theory are
\begin{center}
  \texttt{stmt1 \$f wff P \$.}\\
  \texttt{stmt2 \$f setvar x \$.}\\
  \texttt{stmt3 \$e |- ( P -> Q ) \$.}
\end{center}
\index{\texttt{\$e} statement}
\index{\texttt{\$f} statement}
Informally, these would be read, ``Let $P$ be a well-formed-formula,'' ``Let
$x$ be an (individual) variable,'' and ``Assume we have proved $P \rightarrow
Q$.''  The turnstile symbol \,$\vdash$\index{turnstile ({$\,\vdash$})} is
commonly used in logic texts to mean ``a proof exists for.''

To summarize:
\begin{itemize}
\item A \texttt{\$f} hypothesis tells Metamath the type or kind of its variable.
It is analogous to a variable declaration in a computer language that
tells the compiler that a variable is an integer or a floating-point
number.
\item The \texttt{\$e} hypothesis corresponds to what you would usually call a
``hypothesis'' in ordinary mathematics.
\end{itemize}

Before an assertion\index{assertion} (\texttt{\$a} or \texttt{\$p} statement) can be
referenced in a proof, all of its associated \texttt{\$f} and \texttt{\$e} hypotheses
(i.e.\ those \texttt{\$e} hypotheses that are active) must be satisfied (i.e.
established by the proof).  The meaning of ``associated'' (which we will call
{\bf mandatory} in Section~\ref{frames}) will become clear when we discuss
scoping later.

Note that after any \texttt{\$f}, \texttt{\$e},
\texttt{\$a}, or \texttt{\$p} token there is a required
\textit{typecode}\index{typecode}.
The typecode is a constant used to enforce types of expressions.
This will become clearer once we learn more about
assertions (\texttt{\$a} and \texttt{\$p} statements).
An example may also clarify their purpose.
In the
\texttt{set.mm}\index{set theory database (\texttt{set.mm})}%
\index{Metamath Proof Explorer}
database,
the following typecodes are used:

\begin{itemize}
\item \texttt{wff} :
  Well-formed formula (wff) symbol
  (read: ``the following symbol sequence is a wff'').
% The *textual* typecode for turnstile is "|-", but when read it's a little
% confusing, so I intentionally display the mathematical symbol here instead
% (I think it's clearer in this context).
\item \texttt{$\vdash$} :
  Turnstile (read: ``the following symbol sequence is provable'' or
  ``a proof exists for'').
\item \texttt{setvar} :
  Individual set variable type (read: ``the following is an
  individual set variable'').
  Note that this is \textit{not} the type of an arbitrary set expression,
  instead, it is used to ensure that there is only a single symbol used
  after quantifiers like for-all ($\forall$) and there-exists ($\exists$).
\item \texttt{class} :
  An expression that is a syntactically valid class expression.
  All valid set expressions are also valid class expression, so expressions
  of sets normally have the \texttt{class} typecode.
  Use the \texttt{class} typecode,
  \textit{not} the \texttt{setvar} typecode,
  for the type of set expressions unless you are specifically identifying
  a single set variable.
\end{itemize}

\subsection{Assertions (\texttt{\$a} and \texttt{\$p} Statements)}
\index{\texttt{\$a} statement}
\index{\texttt{\$p} statement}\index{assertion}\index{axiomatic assertion}
\index{provable assertion}

There are two types of assertions, \texttt{\$a}\index{\texttt{\$a} statement}
statements ({\bf axiomatic assertions}) and \texttt{\$p} statements ({\bf
provable assertions}).  Their syntax is as follows:
\begin{center}
  {\em label} \texttt{\$a} {\em typecode} {\em math-symbol} \ldots
         {\em math-symbol} \texttt{\$.}\\
  {\em label} \texttt{\$p} {\em typecode} {\em math-symbol} \ldots
        {\em math-symbol} \texttt{\$=} {\em proof} \texttt{\$.}
\end{center}
\index{\texttt{\$a} statement}
\index{\texttt{\$p} statement}
\index{\texttt{\$=} keyword}
An assertion always requires a {\em label}\index{label}. The expression in an
assertion consists of a typecode (an active constant)
followed by a sequence of zero
or more math symbols.  Each math symbol, including any {\em constant}, must be a
previously declared constant or variable.  (In addition, each math symbol
must be active, which will be covered when we discuss scoping statements in
Section~\ref{scoping}.)

A \texttt{\$a} statement is usually a definition of syntax (for example, if $P$
and $Q$ are wffs then so is $(P\to Q)$), an axiom\index{axiom} of ordinary
mathematics (for example, $x=x$), or a definition\index{definition} of
ordinary mathematics (for example, $x\ne y$ means $\lnot x=y$). A \texttt{\$p}
statement is a claim that a certain combination of math symbols follows from
previous assertions and is accompanied by a proof that demonstrates it.

Assertions can also be referenced in (later) proofs in order to derive new
assertions from them. The label of an assertion is used to refer to it in a
proof. Section~\ref{proof} will describe the proof in detail.

Assertions also provide the primary means for communicating the mathematical
results in the database to people.  Proofs (when conveniently displayed)
communicate to people how the results were arrived at.

\subsubsection{The \texttt{\$a} Statement}
\index{\texttt{\$a} statement}

Axiomatic assertions (\texttt{\$a} statements) represent the starting points from
which other assertions (\texttt{\$p}\index{\texttt{\$p} statement} statements) are
derived.  Their most obvious use is for specifying ordinary mathematical
axioms\index{axiom}, but they are also used for two other purposes.

First, Metamath\index{Metamath} needs to know the syntax of symbol
sequences that constitute valid mathematical statements.  A Metamath
proof must be broken down into much more detail than ordinary
mathematical proofs that you may be used to thinking of (even the
``complete'' proofs of formal logic\index{formal logic}).  This is one
of the things that makes Metamath a general-purpose language,
independent of any system of logic or even syntax.  If you want to use a
substitution instance of an assertion as a step in a proof, you must
first prove that the substitution is syntactically correct (or if you
prefer, you must ``construct'' it), showing for example that the
expression you are substituting for a wff metavariable is a valid wff.
The \texttt{\$a}\index{\texttt{\$a} statement} statement is used to
specify those combinations of symbols that are considered syntactically
valid, such as the legal forms of wffs.

Second, \texttt{\$a} statements are used to specify what are ordinarily thought of
as definitions, i.e.\ new combinations of symbols that abbreviate other
combinations of symbols.  Metamath makes no distinction\index{axiom vs.\
definition} between axioms\index{axiom} and definitions\index{definition}.
Indeed, it has been argued that such distinction should not be made even in
ordinary mathematics; see Section~\ref{definitions}, which discusses the
philosophy of definitions.  Section~\ref{hierarchy} discusses some
technical requirements for definitions.  In \texttt{set.mm} we adopt the
convention of prefixing axiom labels with \texttt{ax-} and definition labels with
\texttt{df-}\index{label}.

The results that can be derived with the Metamath language are only as good as
the \texttt{\$a}\index{\texttt{\$a} statement} statements used as their starting
point.  We cannot stress this too strongly.  For example, Metamath will
not prevent you from specifying $x\neq x$ as an axiom of logic.  It is
essential that you scrutinize all \texttt{\$a} statements with great care.
Because they are a source of potential pitfalls, it is best not to add new
ones (usually new definitions) casually; rather you should carefully evaluate
each one's necessity and advantages.

Once you have in place all of the basic axioms\index{axiom} and
rules\index{rule} of a mathematical theory, the only \texttt{\$a} statements that
you will be adding will be what are ordinarily called definitions.  In
principle, definitions should be in some sense eliminable from the language of
a theory according to some convention (usually involving logical equivalence
or equality).  The most common convention is that any formula that was
syntactically valid but not provable before the definition was introduced will
not become provable after the definition is introduced.  In an ideal world,
definitions should not be present at all if one is to have absolute confidence
in a mathematical result.  However, they are necessary to make
mathematics practical, for otherwise the resulting formulas would be
extremely long and incomprehensible.  Since the nature of definitions (in the
most general sense) does not permit them to automatically be verified as
``proper,''\index{proper definition}\index{definition!proper} the judgment of
the mathematician is required to ensure it.  (In \texttt{set.mm} effort was made
to make almost all definitions directly eliminable and thus minimize the need
for such judgment.)

If you are not a mathematician, it may be best not to add or change any
\texttt{\$a}\index{\texttt{\$a} statement} statements but instead use
the mathematical language already provided in standard databases.  This
way Metamath will not allow you to make a mistake (i.e.\ prove a false
result).


\subsection{Frames}\label{frames}

We now introduce the concept of a collection of related Metamath statements
called a frame.  Every assertion (\texttt{\$a} or \texttt{\$p} statement) in the database has
an associated frame.

A {\bf frame}\index{frame} is a sequence of \texttt{\$d}, \texttt{\$f},
and \texttt{\$e} statements (zero or more of each) followed by one
\texttt{\$a} or \texttt{\$p} statement, subject to certain conditions we
will describe.  For simplicity we will assume that all math symbol
tokens used are declared at the beginning of the database with
\texttt{\$c} and \texttt{\$v} statements (which are not properly part of
a frame).  Also for simplicity we will assume there are only simple
\texttt{\$d} statements (those with only two variables) and imagine any
compound \texttt{\$d} statements (those with more than two variables) as
broken up into simple ones.

A frame groups together those hypotheses (and \texttt{\$d} statements) relevant
to an assertion (\texttt{\$a} or \texttt{\$p} statement).  The statements in a frame
may or may not be physically adjacent in a database; we will cover
this in our discussion of scoping statements
in Section~\ref{scoping}.

A frame has the following properties:
\begin{enumerate}
 \item The set of variables contained in its \texttt{\$f} statements must
be identical to the set of variables contained in its \texttt{\$e},
\texttt{\$a}, and/or \texttt{\$p} statements.  In other words, each
variable in a \texttt{\$e}, \texttt{\$a}, or \texttt{\$p} statement must
have an associated ``variable type'' defined for it in a \texttt{\$f}
statement.
  \item No two \texttt{\$f} statements may contain the same variable.
  \item Any \texttt{\$f} statement
must occur before a \texttt{\$e} statement in which its variable occurs.
\end{enumerate}

The first property determines the set of variables occurring in a frame.
These are the {\bf mandatory
variables}\index{mandatory variable} of the frame.  The second property
tells us there must be only one type specified for a variable.
The last property is not a theoretical requirement but it
makes parsing of the database easier.

For our examples, we assume our database has the following declarations:

\begin{verbatim}
$v P Q R $.
$c -> ( ) |- wff $.
\end{verbatim}

The following sequence of statements, describing the modus ponens inference
rule, is an example of a frame:

\begin{verbatim}
wp  $f wff P $.
wq  $f wff Q $.
maj $e |- ( P -> Q ) $.
min $e |- P $.
mp  $a |- Q $.
\end{verbatim}

The following sequence of statements is not a frame because \texttt{R} does not
occur in the \texttt{\$e}'s or the \texttt{\$a}:

\begin{verbatim}
wp  $f wff P $.
wq  $f wff Q $.
wr  $f wff R $.
maj $e |- ( P -> Q ) $.
min $e |- P $.
mp  $a |- Q $.
\end{verbatim}

The following sequence of statements is not a frame because \texttt{Q} does not
occur in a \texttt{\$f}:

\begin{verbatim}
wp  $f wff P $.
maj $e |- ( P -> Q ) $.
min $e |- P $.
mp  $a |- Q $.
\end{verbatim}

The following sequence of statements is not a frame because the \texttt{\$a} statement is
not the last one:

\begin{verbatim}
wp  $f wff P $.
wq  $f wff Q $.
maj $e |- ( P -> Q ) $.
mp  $a |- Q $.
min $e |- P $.
\end{verbatim}

Associated with a frame is a sequence of {\bf mandatory
hypotheses}\index{mandatory hypothesis}.  This is simply the set of all
\texttt{\$f} and \texttt{\$e} statements in the frame, in the order they
appear.  A frame can be referenced in a later proof using the label of
the \texttt{\$a} or \texttt{\$p} assertion statement, and the proof
makes an assignment to each mandatory hypothesis in the order in which
it appears.  This means the order of the hypotheses, once chosen, must
not be changed so as not to affect later proofs referencing the frame's
assertion statement.  (The Metamath proof verifier will, of course, flag
an error if a proof becomes incorrect by doing this.)  Since proofs make
use of ``Reverse Polish notation,'' described in Section~\ref{proof}, we
call this order the {\bf RPN order}\index{RPN order} of the hypotheses.

Note that \texttt{\$d} statements are not part of the set of mandatory
hypotheses, and their order doesn't matter (as long as they satisfy the
fourth property for a frame described above).  The \texttt{\$d}
statements specify restrictions on variables that must be satisfied (and
are checked by the proof verifier) when expressions are substituted for
them in a proof, and the \texttt{\$d} statements themselves are never
referenced directly in a proof.

A frame with a \texttt{\$p} (provable) statement requires a proof as part of the
\texttt{\$p} statement.  Sometimes in a proof we want to make use of temporary or
dummy variables\index{dummy variable} that do not occur in the \texttt{\$p}
statement or its mandatory hypotheses.  To accommodate this we define an {\bf
extended frame}\index{extended frame} as a frame together with zero or more
\texttt{\$d} and \texttt{\$f} statements that reference variables not among the
mandatory variables of the frame.  Any new variables referenced are called the
{\bf optional variables}\index{optional variable} of the extended frame. If a
\texttt{\$f} statement references an optional variable it is called an {\bf
optional hypothesis}\index{optional hypothesis}, and if one or both of the
variables in a \texttt{\$d} statement are optional variables it is called an {\bf
optional disjoint-variable restriction}\index{optional disjoint-variable
restriction}.  Properties 2 and 3 for a frame also apply to an extended
frame.

The concept of optional variables is not meaningful for frames with \texttt{\$a}
statements, since those statements have no proofs that might make use of them.
There is no restriction on including optional hypotheses in the extended frame
for a \texttt{\$a} statement, but they serve no purpose.

The following set of statements is an example of an extended frame, which
contains an optional variable \texttt{R} and an optional hypothesis \texttt{wr}.  In
this example, we suppose the rule of modus ponens is not an axiom but is
derived as a theorem from earlier statements (we omit its presumed proof).
Variable \texttt{R} may be used in its proof if desired (although this would
probably have no advantage in propositional calculus).  Note that the sequence
of mandatory hypotheses in RPN order is still \texttt{wp}, \texttt{wq}, \texttt{maj},
\texttt{min} (i.e.\ \texttt{wr} is omitted), and this sequence is still assumed
whenever the assertion \texttt{mp} is referenced in a subsequent proof.

\begin{verbatim}
wp  $f wff P $.
wq  $f wff Q $.
wr  $f wff R $.
maj $e |- ( P -> Q ) $.
min $e |- P $.
mp  $p |- Q $= ... $.
\end{verbatim}

Every frame is an extended frame, but not every extended frame is a frame, as
this example shows.  The underlying frame for an extended frame is
obtained by simply removing all statements containing optional variables.
Any proof referencing an assertion will ignore any extensions to its
frame, which means we may add or delete optional hypotheses at will without
affecting subsequent proofs.

The conceptually simplest way of organizing a Metamath database is as a
sequence of extended frames.  The scoping statements
\texttt{\$\char`\{}\index{\texttt{\$\char`\{} and \texttt{\$\char`\}}
keywords} and \texttt{\$\char`\}} can be used to delimit the start and
end of an extended frame, leading to the following possible structure for a
database.  \label{framelist}

\vskip 2ex
\setbox\startprefix=\hbox{\tt \ \ \ \ \ \ \ \ }
\setbox\contprefix=\hbox{}
\startm
\m{\mbox{(\texttt{\$v} {\em and} \texttt{\$c}\,{\em statements})}}
\endm
\startm
\m{\mbox{\texttt{\$\char`\{}}}
\endm
\startm
\m{\mbox{\texttt{\ \ } {\em extended frame}}}
\endm
\startm
\m{\mbox{\texttt{\$\char`\}}}}
\endm
\startm
\m{\mbox{\texttt{\$\char`\{}}}
\endm
\startm
\m{\mbox{\texttt{\ \ } {\em extended frame}}}
\endm
\startm
\m{\mbox{\texttt{\$\char`\}}}}
\endm
\startm
\m{\mbox{\texttt{\ \ \ \ \ \ \ \ \ }}\vdots}
\endm
\vskip 2ex

In practice, this structure is inconvenient because we have to repeat
any \texttt{\$f}, \texttt{\$e}, and \texttt{\$d} statements over and
over again rather than stating them once for use by several assertions.
The scoping statements, which we will discuss next, allow this to be
done.  In principle, any Metamath database can be converted to the above
format, and the above format is the most convenient to use when studying
a Metamath database as a formal system%
%% Uncomment this when uncommenting section {formalspec} below
   (Appendix \ref{formalspec})%
.
In fact, Metamath internally converts the database to the above format.
The command \texttt{show statement} in the Metamath program will show
you the contents of the frame for any \texttt{\$a} or \texttt{\$p}
statement, as well as its extension in the case of a \texttt{\$p}
statement.

%c%(provided that all ``local'' variables and constants with limited scope have
%c%unique names),

During our discussion of scoping statements, it may be helpful to
think in terms of the equivalent sequence of frames that will result when
the database is parsed.  Scoping (other than the limited
use above to delimit frames) is not a theoretical requirement for
Metamath but makes it more convenient.


\subsection{Scoping Statements (\texttt{\$\{} and \texttt{\$\}})}\label{scoping}
\index{\texttt{\$\char`\{} and \texttt{\$\char`\}} keywords}\index{scoping statement}

%c%Some Metamath statements may be needed only temporarily to
%c%serve a specific purpose, and after we're done with them we would like to
%c%disregard or ignore them.  For example, when we're finished using a variable,
%c%we might want to
%c%we might want to free up the token\index{token} used to name it so that the
%c%token can be used for other purposes later on, such as a different kind of
%c%variable or even a constant.  In the terminology of computer programming, we
%c%might want to let some symbol declarations be ``local'' rather than ``global.''
%c%\index{local symbol}\index{global symbol}

The {\bf scoping} statements, \texttt{\$\char`\{} ({\bf start of block}) and \texttt{\$\char`\}}
({\bf end of block})\index{block}, provide a means for controlling the portion
of a database over which certain statement types are recognized.  The
syntax of a scoping statement is very simple; it just consists of the
statement's keyword:
\begin{center}
\texttt{\$\char`\{}\\
\texttt{\$\char`\}}
\end{center}
\index{\texttt{\$\char`\{} and \texttt{\$\char`\}} keywords}

For example, consider the following database where we have stripped out
all tokens except the scoping statement keywords.  For the purpose of the
discussion, we have added subscripts to the scoping statements; these subscripts
do not appear in the actual database.
\[
 \mbox{\tt \ \$\char`\{}_1
 \mbox{\tt \ \$\char`\{}_2
 \mbox{\tt \ \$\char`\}}_2
 \mbox{\tt \ \$\char`\{}_3
 \mbox{\tt \ \$\char`\{}_4
 \mbox{\tt \ \$\char`\}}_4
 \mbox{\tt \ \$\char`\}}_3
 \mbox{\tt \ \$\char`\}}_1
\]
Each \texttt{\$\char`\{} statement in this example is said to be {\bf
matched} with the \texttt{\$\char`\}} statement that has the same
subscript.  Each pair of matched scoping statements defines a region of
the database called a {\bf block}.\index{block} Blocks can be {\bf
nested}\index{nested block} inside other blocks; in the example, the
block defined by $\mbox{\tt \$\char`\{}_4$ and $\mbox{\tt \$\char`\}}_4$
is nested inside the block defined by $\mbox{\tt \$\char`\{}_3$ and
$\mbox{\tt \$\char`\}}_3$ as well as inside the block defined by
$\mbox{\tt \$\char`\{}_1$ and $\mbox{\tt \$\char`\}}_1$.  In general, a
block may be empty, it may contain only non-scoping
statements,\footnote{Those statements other than \texttt{\$\char`\{} and
\texttt{\$\char`\}}.}\index{non-scoping statement} or it may contain any
mixture of other blocks and non-scoping statements.  (This is called a
``recursive'' definition\index{recursive definition} of a block.)

Associated with each block is a number called its {\bf nesting
level}\index{nesting level} that indicates how deeply the block is nested.
The nesting levels of the blocks in our example are as follows:
\[
  \underbrace{
    \mbox{\tt \ }
    \underbrace{
     \mbox{\tt \$\char`\{\ }
     \underbrace{
       \mbox{\tt \$\char`\{\ }
       \mbox{\tt \$\char`\}}
     }_{2}
     \mbox{\tt \ }
     \underbrace{
       \mbox{\tt \$\char`\{\ }
       \underbrace{
         \mbox{\tt \$\char`\{\ }
         \mbox{\tt \$\char`\}}
       }_{3}
       \mbox{\tt \ \$\char`\}}
     }_{2}
     \mbox{\tt \ \$\char`\}}
   }_{1}
   \mbox{\tt \ }
 }_{0}
\]
\index{\texttt{\$\char`\{} and \texttt{\$\char`\}} keywords}
The entire database is considered to be one big block (the {\bf outermost}
block) with a nesting level of 0.  The outermost block is {\em not} bracketed
by scoping statements.\footnote{The language was designed this way so that
several source files can be joined together more easily.}\index{outermost
block}

All non-scoping Metamath statements become recognized or {\bf
active}\index{active statement} at the place where they appear.\footnote{To
keep things slightly simpler, we do not bother to define the concept of
``active'' for the scoping statements.}  Certain of these statement types
become inactive at the end of the block in which they appear; these statement
types are:
\begin{center}
  \texttt{\$c}, \texttt{\$v}, \texttt{\$d}, \texttt{\$e}, and \texttt{\$f}.
%  \texttt{\$v}, \texttt{\$f}, \texttt{\$e}, and \texttt{\$d}.
\end{center}
\index{\texttt{\$c} statement}
\index{\texttt{\$d} statement}
\index{\texttt{\$e} statement}
\index{\texttt{\$f} statement}
\index{\texttt{\$v} statement}
The other statement types remain active forever (i.e.\ through the end of the
database); they are:
\begin{center}
  \texttt{\$a} and \texttt{\$p}.
%  \texttt{\$c}, \texttt{\$a}, and \texttt{\$p}.
\end{center}
\index{\texttt{\$a} statement}
\index{\texttt{\$p} statement}
Any statement (of these 7 types) located in the outermost
block\index{outermost block} will remain active through the end of the
database and thus are effectively ``global'' statements.\index{global
statement}

All \texttt{\$c} statements must be placed in the outermost block.  Since they are
therefore always global, they could be considered as belonging to both of the
above categories.

The {\bf scope}\index{scope} of a statement is the set of statements that
recognize it as active.

%c%The concept of ``active'' is also defined for math symbols\index{math
%c%symbol}.  Math symbols (constants\index{constant} and
%c%variables\index{variable}) become {\bf active}\index{active
%c%math symbol} in the \texttt{\$c}\index{\texttt{\$c}
%c%statement} and \texttt{\$v}\index{\texttt{\$v} statement} statements that
%c%declare them.  They become inactive when their declaration statements become
%c%inactive.

The concept of ``active'' is also defined for math symbols\index{math
symbol}.  Math symbols (constants\index{constant} and
variables\index{variable}) become {\bf active}\index{active math symbol}
in the \texttt{\$c}\index{\texttt{\$c} statement} and
\texttt{\$v}\index{\texttt{\$v} statement} statements that declare them.
A variable becomes inactive when its declaration statement becomes
inactive.  Because all \texttt{\$c} statements must be in the outermost
block, a constant will never become inactive after it is declared.

\subsubsection{Redeclaration of Math Symbols}
\index{redeclaration of symbols}\label{redeclaration}

%c%A math symbol may not be declared a second time while it is active, but it may
%c%be declared again after it becomes inactive.

A variable may not be declared a second time while it is active, but it may be
declared again after it becomes inactive.  This provides a convenient way to
introduce ``local'' variables,\index{local variable} i.e.\ temporary variables
for use in the frame of an assertion or in a proof without keeping them around
forever.  A previously declared variable may not be redeclared as a constant.

A constant may not be redeclared.  And, as mentioned above, constants must be
declared in the outermost block.

The reason variables may have limited scope but not constants is that an
assertion (\texttt{\$a} or \texttt{\$p} statement) remains available for use in
proofs through the end of the database.  Variables in an assertion's frame may
be substituted with whatever is needed in a proof step that references the
assertion, whereas constants remain fixed and may not be substituted with
anything.  The particular token used for a variable in an assertion's frame is
irrelevant when the assertion is referenced in a proof, and it doesn't matter
if that token is not available outside of the referenced assertion's frame.
Constants, however, must be globally fixed.

There is no theoretical
benefit for the feature allowing variables to be active for limited scopes
rather than global. It is just a convenience that allows them, for example, to
be locally grouped together with their corresponding \texttt{\$f} variable-type
declarations.

%c%If you declare a math symbol more than once, internally Metamath considers it a
%c%new distinct symbol, even though it has the same name.  If you are unaware of
%c%this, you may find that what you think are correct proofs are incorrectly
%c%rejected as invalid, because Metamath may tell you that a constant you
%c%previously declared does not match a newly declared math symbol with the same
%c%name.  For details on this subtle point, see the Comment on
%c%p.~\pageref{spec4comment}.  This is done purposely to allow temporary
%c%constants to be introduced while developing a subtheory, then allow their math
%c%symbol tokens to be reused later on; in general they will not refer to the
%c%same thing.  In practice, you would not ordinarily reuse the names of
%c%constants because it would tend to be confusing to the reader.  The reuse of
%c%names of variables, on the other hand, is something that is often useful to do
%c%(for example it is done frequently in \texttt{set.mm}).  Since variables in an
%c%assertion referenced in a proof can be substituted as needed to achieve a
%c%symbol match, this is not an issue.

% (This section covers a somewhat advanced topic you may want to skip
% at first reading.)
%
% Under certain circumstances, math symbol\index{math symbol}
% tokens\index{token} may be redeclared (i.e.\ the token
% may appear in more than
% one \texttt{\$c}\index{\texttt{\$c} statement} or \texttt{\$v}\index{\texttt{\$v}
% statement} statement).  You might want to do this say, to make temporary use
% of a variable name without having to worry about its affect elsewhere,
% somewhat analogous to declaring a local variable in a standard computer
% language.  Understanding what goes on when math symbol tokens are redeclared
% is a little tricky to understand at first, since it requires that we
% distinguish the token itself from the math symbol that it names.  It will help
% if we first take a peek at the internal workings of the
% Metamath\index{Metamath} program.
%
% Metamath reserves a memory location for each occurrence of a
% token\index{token} in a declaration statement (\texttt{\$c}\index{\texttt{\$c}
% statement} or \texttt{\$v}\index{\texttt{\$v} statement}).  If a given token appears
% in more than one declaration statement, it will refer to more than one memory
% locations.  A math symbol\index{math symbol} may be thought of as being one of
% these memory locations rather than as the token itself.  Only one of the
% memory locations associated with a given token may be active at any one time.
% The math symbol (memory location) that gets looked up when the token appears
% in a non-declaration statement is the one that happens to be active at that
% time.
%
% We now look at the rules for the redeclaration\index{redeclaration of symbols}
% of math symbol tokens.
% \begin{itemize}
% \item A math symbol token may not be declared twice in the
% same block.\footnote{While there is no theoretical reason for disallowing
% this, it was decided in the design of Metamath that allowing it would offer no
% advantage and might cause confusion.}
% \item An inactive math symbol may always be
% redeclared.
% \item  An active math symbol may be redeclared in a different (i.e.\
% inner) block\index{block} from the one it became active in.
% \end{itemize}
%
% When a math symbol token is redeclared, it conceptually refers to a different
% math symbol, just as it would be if it were called a different name.  In
% addition, the original math symbol that it referred to, if it was active,
% temporarily becomes inactive.  At the end of the block in which the
% redeclaration occurred, the new math symbol\index{math symbol} becomes
% inactive and the original symbol becomes active again.  This concept is
% illustrated in the following example, where the symbol \texttt{e} is
% ordinarily a constant (say Euler's constant, 2.71828...) but
% temporarily we want to use it as a ``local'' variable, say as a coefficient
% in the equation $a x^4 + b x^3 + c x^2 + d x + e$:
% \[
%   \mbox{\tt \$\char`\{\ \$c e \$.}
%   \underbrace{
%     \ \ldots\ %
%     \mbox{\tt \$\char`\{}\ \ldots\ %
%   }_{\mbox{\rm region A}}
%   \mbox{\tt \$v e \$.}
%   \underbrace{
%     \mbox{\ \ \ \ldots\ \ \ }
%   }_{\mbox{\rm region B}}
%   \mbox{\tt \$\char`\}}
%   \underbrace{
%     \mbox{\ \ \ \ldots\ \ \ }
%   }_{\mbox{\rm region C}}
%   \mbox{\tt \$\char`\}}
% \]
% \index{\texttt{\$\char`\{} and \texttt{\$\char`\}} keywords}
% In region A, the token \texttt{e} refers to a constant.  It is redeclared as a
% variable in region B, and any reference to it in this region will refer to this
% variable.  In region C, the redeclaration becomes inactive, and the original
% declaration becomes active again.  In region C, the token \texttt{x} refers to the
% original constant.
%
% As a practical matter, overuse of math symbol\index{math symbol}
% redeclarations\index{redeclaration of symbols} can be confusing (even though
% it is well-defined) and is best avoided when possible.  Here are some good
% general guidelines you can follow.  Usually, you should declare all
% constants\index{constant} in the outermost block\index{outermost block},
% especially if they are general-purpose (such as the token \verb$A.$, meaning
% $\forall$ or ``for all'').  This will make them ``globally'' active (although
% as in the example above local redeclarations will temporarily make them
% inactive.)  Most or all variables\index{variable}, on the other hand, could be
% declared in inner blocks, so that the token for them can be used later for a
% different type of variable or a constant.  (The names of the variables you
% choose are not used when you refer to an assertion\index{assertion} in a
% proof, whereas constants must match exactly.  A locally declared constant will
% not match a globally declared constant in a proof, even if they use the same
% token, because Metamath internally considers them to be different math
% symbols.)  To avoid confusion, you should generally avoid redeclaring active
% variables.  If you must redeclare them, do so at the beginning of a block.
% The temporary declaration of constants in inner blocks might be occasionally
% appropriate when you make use of a temporary definition to prove lemmas
% leading to a main result that does not make direct use of the definition.
% This way, you will not clutter up your database with a large number of
% seldom-used global constant symbols.  You might want to note that while
% inactive constants may not appear directly in an assertion (a \texttt{\$a}\index{\texttt{\$a}
% statement} or \texttt{\$p}\index{\texttt{\$p} statement}
% statement), they may be indirectly used in the proof of a \texttt{\$p} statement
% so long as they do not appear in the final math symbol sequence constructed by
% the proof.  In the end, you will have to use your best judgment, taking into
% account standard mathematical usage of the symbols as well as consideration
% for the reader of your work.
%
% \subsubsection{Reuse of Labels}\index{reuse of labels}\index{label}
%
% The \texttt{\$e}\index{\texttt{\$e} statement}, \texttt{\$f}\index{\texttt{\$f}
% statement}, \texttt{\$a}\index{\texttt{\$a} statement}, and
% \texttt{\$p}\index{\texttt{\$p}
% statement} statement types require labels, which allow them to be
% referenced later inside proofs.  A label is considered {\bf
% active}\index{active label} when the statement it is associated with is
% active.  The token\index{token} for a label may be reused
% (redeclared)\index{redeclaration of labels} provided that it is not being used
% for a currently active label.  (Unlike the tokens for math symbols, active
% label tokens may not be redeclared in an inner scope.)  Note that the labels
% of \texttt{\$a} and \texttt{\$p} statements can never be reused after these
% statements appear, because these statements remain active through the end of
% the database.
%
% You might find the reuse of labels a convenient way to have standard names for
% temporary hypotheses, such as \texttt{h1}, \texttt{h2}, etc.  This way you don't have
% to invent unique names for each of them, and in some cases it may be less
% confusing to the reader (although in other cases it might be more confusing, if
% the hypothesis is located far away from the assertion that uses
% it).\footnote{The current implementation requires that all labels, even
% inactive ones, be unique.}

\subsubsection{Frames Revisited}\index{frames and scoping statements}

Now that we have covered scoping, we will look at how an arbitrary
Metamath database can be converted to the simple sequence of extended
frames described on p.~\pageref{framelist}.  This is also how Metamath
stores the database internally when it reads in the database
source.\label{frameconvert} The method is simple.  First, we collect all
constant and variable (\texttt{\$c} and \texttt{\$v}) declarations in
the database, ignoring duplicate declarations of the same variable in
different scopes.  We then put our collected \texttt{\$c} and
\texttt{\$v} declarations at the beginning of the database, so that
their scope is the entire database.  Next, for each assertion in the
database, we determine its frame and extended frame.  The extended frame
is simply the \texttt{\$f}, \texttt{\$e}, and \texttt{\$d} statements
that are active.  The frame is the extended frame with all optional
hypotheses removed.

An equivalent way of saying this is that the extended frame of an assertion
is the collection of all \texttt{\$f}, \texttt{\$e}, and \texttt{\$d} statements
whose scope includes the assertion.
The \texttt{\$f} and \texttt{\$e} statements
occur in the order they appear
(order is irrelevant for \texttt{\$d} statements).

%c%, renaming any
%c%redeclared variables as needed so that all of them have unique names.  (The
%c%exact renaming convention is unimportant.  You might imagine renaming
%c%different declarations of math symbol \texttt{a} as \texttt{a\$1}, \texttt{a\$2}, etc.\
%c%which would prevent any conflicts since \texttt{\$} is not a legal character in a
%c%math symbol token.)

\section{The Anatomy of a Proof} \label{proof}
\index{proof!Metamath, description of}

Each provable assertion (\texttt{\$p}\index{\texttt{\$p} statement} statement) in a
database must include a {\bf proof}\index{proof}.  The proof is located
between the \texttt{\$=}\index{\texttt{\$=} keyword} and \texttt{\$.}\ keywords in the
\texttt{\$p} statement.

In the basic Metamath language\index{basic language}, a proof is a
sequence of statement labels.  This label sequence\index{label sequence}
serves as a set of instructions that the Metamath program uses to
construct a series of math symbol sequences.  The construction must
ultimately result in the math symbol sequence contained between the
\texttt{\$p}\index{\texttt{\$p} statement} and
\texttt{\$=}\index{\texttt{\$=} keyword} keywords of the \texttt{\$p}
statement.  Otherwise, the Metamath program will consider the proof
incorrect, and it will notify you with an appropriate error message when
you ask it to verify the proof.\footnote{To make the loading faster, the
Metamath program does not automatically verify proofs when you
\texttt{read} in a database unless you use the \texttt{/verify}
qualifier.  After a database has been read in, you may use the
\texttt{verify proof *} command to verify proofs.}\index{\texttt{verify
proof} command} Each label in a proof is said to {\bf
reference}\index{label reference} its corresponding statement.

Associated with any assertion\index{assertion} (\texttt{\$p} or
\texttt{\$a}\index{\texttt{\$a} statement} statement) is a set of
hypotheses (\texttt{\$f}\index{\texttt{\$f} statement} or
\texttt{\$e}\index{\texttt{\$e} statement} statements) that are active
with respect to that assertion.  Some are mandatory and the others are
optional.  You should review these concepts if necessary.

Each label\index{label} in a proof must be either the label of a
previous assertion (\texttt{\$a}\index{\texttt{\$a} statement} or
\texttt{\$p}\index{\texttt{\$p} statement} statement) or the label of an
active hypothesis (\texttt{\$e} or \texttt{\$f}\index{\texttt{\$f}
statement} statement) of the \texttt{\$p} statement containing the
proof.  Hypothesis labels may reference both the
mandatory\index{mandatory hypothesis} and the optional hypotheses of the
\texttt{\$p} statement.

The label sequence in a proof specifies a construction in {\bf reverse Polish
notation}\index{reverse Polish notation (RPN)} (RPN).  You may be familiar
with RPN if you have used older
Hewlett--Packard or similar hand-held calculators.
In the calculator analogy, a hypothesis label\index{hypothesis label} is like
a number and an assertion label\index{assertion label} is like an operation
(more precisely, an $n$-ary operation when the
assertion has $n$ \texttt{\$e}-hypotheses).
On an RPN calculator, an operation takes one or more previous numbers in an
input sequence, performs a calculation on them, and replaces those numbers and
itself with the result of the calculation.  For example, the input sequence
$2,3,+$ on an RPN calculator results in $5$, and the input sequence
$2,3,5,{\times},+$ results in $2,15,+$ which results in $17$.

Understanding how RPN is processed involves the concept of a {\bf
stack}\index{stack}\index{RPN stack}, which can be thought of as a set of
temporary memory locations that hold intermediate results.  When Metamath
encounters a hypothesis label it places or {\bf pushes}\index{push} the math
symbol sequence of the hypothesis onto the stack.  When Metamath encounters an
assertion label, it associates the most recent stack entries with the {\em
mandatory} hypotheses\index{mandatory hypothesis} of the assertion, in the
order where the most recent stack entry is associated with the last mandatory
hypothesis of the assertion.  It then determines what
substitutions\index{substitution!variable}\index{variable substitution} have
to be made into the variables of the assertion's mandatory hypotheses to make
them identical to the associated stack entries.  It then makes those same
substitutions into the assertion itself.  Finally, Metamath removes or {\bf
pops}\index{pop} the matched hypotheses from the stack and pushes the
substituted assertion onto the stack.

For the purpose of matching the mandatory hypothesis to the most recent stack
entries, whether a hypothesis is a \texttt{\$e} or \texttt{\$f} statement is
irrelevant.  The only important thing is that a set of
substitutions\footnote{In the Metamath spec (Section~\ref{spec}), we use the
singular term ``substitution'' to refer to the set of substitutions we talk
about here.} exist that allow a match (and if they don't, the proof verifier
will let you know with an error message).  The Metamath language is specified
in such a way that if a set of substitutions exists, it will be unique.
Specifically, the requirement that each variable have a type specified for it
with a \texttt{\$f} statement ensures the uniqueness.

We will illustrate this with an example.
Consider the following Metamath source file:
\begin{verbatim}
$c ( ) -> wff $.
$v p q r s $.
wp $f wff p $.
wq $f wff q $.
wr $f wff r $.
ws $f wff s $.
w2 $a wff ( p -> q ) $.
wnew $p wff ( s -> ( r -> p ) ) $= ws wr wp w2 w2 $.
\end{verbatim}
This Metamath source example shows the definition and ``proof'' (i.e.,
construction) of a well-formed formula (wff)\index{well-formed formula (wff)}
in propositional calculus.  (You may wish to type this example into a file to
experiment with the Metamath program.)  The first two statements declare
(introduce the names of) four constants and four variables.  The next four
statements specify the variable types, namely that
each variable is assumed to be a wff.  Statement \texttt{w2} defines (postulates)
a way to produce a new wff, \texttt{( p -> q )}, from two given wffs \texttt{p} and
\texttt{q}. The mandatory hypotheses of \texttt{w2} are \texttt{wp} and \texttt{wq}.
Statement \texttt{wnew} claims that \texttt{( s -> ( r -> p ) )} is a wff given
three wffs \texttt{s}, \texttt{r}, and \texttt{p}.  More precisely, \texttt{wnew} claims
that the sequence of ten symbols \texttt{wff ( s -> ( r -> p ) )} is provable from
previous assertions and the hypotheses of \texttt{wnew}.  Metamath does not know
or care what a wff is, and as far as it is concerned
the typecode \texttt{wff} is just an
arbitrary constant symbol in a math symbol sequence.  The mandatory hypotheses
of \texttt{wnew} are \texttt{wp}, \texttt{wr}, and \texttt{ws}; \texttt{wq} is an optional
hypothesis.  In our particular proof, the optional hypothesis is not
referenced, but in general, any combination of active (i.e.\ optional and
mandatory) hypotheses could be referenced.  The proof of statement \texttt{wnew}
is the sequence of five labels starting with \texttt{ws} (step~1) and ending with
\texttt{w2} (step~5).

When Metamath verifies the proof, it scans the proof from left to right.  We
will examine what happens at each step of the proof.  The stack starts off
empty.  At step 1, Metamath looks up label \texttt{ws} and determines that it is a
hypothesis, so it pushes the symbol sequence of statement \texttt{ws} onto the
stack:

\begin{center}\begin{tabular}{|l|l|}\hline
{Stack location} & {Contents} \\ \hline \hline
1 & \texttt{wff s} \\ \hline
\end{tabular}\end{center}

Metamath sees that the labels \texttt{wr} and \texttt{wp} in steps~2 and 3 are also
hypotheses, so it pushes them onto the stack.  After step~3, the stack looks
like
this:

\begin{center}\begin{tabular}{|l|l|}\hline
{Stack location} & {Contents} \\ \hline \hline
3 & \texttt{wff p} \\ \hline
2 & \texttt{wff r} \\ \hline
1 & \texttt{wff s} \\ \hline
\end{tabular}\end{center}

At step 4, Metamath sees that label \texttt{w2} is an assertion, so it must do
some processing.  First, it associates the mandatory hypotheses of \texttt{w2},
which are \texttt{wp} and \texttt{wq}, with stack locations~2 and 3, {\em in that
order}. Metamath determines that the only possible way
to make hypothesis \texttt{wp} match (become identical to) stack location~2 and
\texttt{wq} match stack location 3 is to substitute variable \texttt{p} with \texttt{r}
and \texttt{q} with \texttt{p}.  Metamath makes these substitutions into \texttt{w2} and
obtains the symbol sequence \texttt{wff ( r -> p )}.  It removes the hypotheses
from stack locations~2 and 3, then places the result into stack location~2:

\begin{center}\begin{tabular}{|l|l|}\hline
{Stack location} & {Contents} \\ \hline \hline
2 & \texttt{wff ( r -> p )} \\ \hline
1 & \texttt{wff s} \\ \hline
\end{tabular}\end{center}

At step 5, Metamath sees that label \texttt{w2} is an assertion, so it must again
do some processing.  First, it matches the mandatory hypotheses of \texttt{w2},
which are \texttt{wp} and \texttt{wq}, to stack locations 1 and 2.
Metamath determines that the only possible way to make the
hypotheses match is to substitute variable \texttt{p} with \texttt{s} and \texttt{q} with
\texttt{( r -> p )}.  Metamath makes these substitutions into \texttt{w2} and obtains
the symbol
sequence \texttt{wff ( s -> ( r -> p ) )}.  It removes stack
locations 1 and 2, then places the result into stack location~1:

\begin{center}\begin{tabular}{|l|l|}\hline
{Stack location} & {Contents} \\ \hline \hline
1 & \texttt{wff ( s -> ( r -> p ) )} \\ \hline
\end{tabular}\end{center}

After Metamath finishes processing the proof, it checks to see that the
stack contains exactly one element and that this element is
the same as the math symbol sequence in the
\texttt{\$p}\index{\texttt{\$p} statement} statement.  This is the case for our
proof of \texttt{wnew},
so we have proved \texttt{wnew} successfully.  If the result
differs, Metamath will notify you with an error message.  An error message
will also result if the stack contains more than one entry at the end of the
proof, or if the stack did not contain enough entries at any point in the
proof to match all of the mandatory hypotheses\index{mandatory hypothesis} of
an assertion.  Finally, Metamath will notify you with an error message if no
substitution is possible that will make a referenced assertion's hypothesis
match the
stack entries.  You may want to experiment with the different kinds of errors
that Metamath will detect by making some small changes in the proof of our
example.

Metamath's proof notation was designed primarily to express proofs in a
relatively compact manner, not for readability by humans.  Metamath can display
proofs in a number of different ways with the \texttt{show proof}\index{\texttt{show
proof} command} command.  The
\texttt{/lemmon} qualifier displays it in a format that is easier to read when the
proofs are short, and you saw examples of its use in Chapter~\ref{using}.  For
longer proofs, it is useful to see the tree structure of the proof.  A tree
structure is displayed when the \texttt{/lemmon} qualifier is omitted.  You will
probably find this display more convenient as you get used to it. The tree
display of the proof in our example looks like
this:\label{treeproof}\index{tree-style proof}\index{proof!tree-style}
\begin{verbatim}
1     wp=ws    $f wff s
2        wp=wr    $f wff r
3        wq=wp    $f wff p
4     wq=w2    $a wff ( r -> p )
5  wnew=w2  $a wff ( s -> ( r -> p ) )
\end{verbatim}
The number to the left of each line is the step number.  Following it is a
{\bf hypothesis association}\index{hypothesis association}, consisting of two
labels\index{label} separated by \texttt{=}.  To the left of the \texttt{=} (except
in the last step) is the label of a hypothesis of an assertion referenced
later in the proof; here, steps 1 and 4 are the hypothesis associations for
the assertion \texttt{w2} that is referenced in step 5.  A hypothesis association
is indented one level more than the assertion that uses it, so it is easy to
find the corresponding assertion by moving directly down until the indentation
level decreases to one less than where you started from.  To the right of each
\texttt{=} is the proof step label for that proof step.  The statement keyword of
the proof step label is listed next, followed by the content of the top of the
stack (the most recent stack entry) as it exists after that proof step is
processed.  With a little practice, you should have no trouble reading proofs
displayed in this format.

Metamath proofs include the syntax construction of a formula.
In standard mathematics, this kind of
construction is not considered a proper part of the proof at all, and it
certainly becomes rather boring after a while.
Therefore,
by default the \texttt{show proof}\index{\texttt{show proof}
command} command does not show the syntax construction.
Historically \texttt{show proof} command
\textit{did} show the syntax construction, and you needed to add the
\texttt{/essential} option to hide, them, but today
\texttt{/essential} is the default and you need to use
\texttt{/all} to see the syntax constructions.

When verifying a proof, Metamath will check that no mandatory
\texttt{\$d}\index{\texttt{\$d} statement}\index{mandatory \texttt{\$d}
statement} statement of an assertion referenced in a proof is violated
when substitutions\index{substitution!variable}\index{variable
substitution} are made to the variables in the assertion.  For details
see Section~\ref{spec4} or \ref{dollard}.

\subsection{The Concept of Unification} \label{unify}

During the course of verifying a proof, when Metamath\index{Metamath}
encounters an assertion label\index{assertion label}, it associates the
mandatory hypotheses\index{mandatory hypothesis} of the assertion with the top
entries of the RPN stack\index{stack}\index{RPN stack}.  Metamath then
determines what substitutions\index{substitution!variable}\index{variable
substitution} it must make to the variables in the assertion's mandatory
hypotheses in order for these hypotheses to become identical to their
corresponding stack entries.  This process is called {\bf
unification}\index{unification}.  (We also informally use the term
``unification'' to refer to a set of substitutions that results from the
process, as in ``two unifications are possible.'')  After the substitutions
are made, the hypotheses are said to be {\bf unified}.

If no such substitutions are possible, Metamath will consider the proof
incorrect and notify you with an error message.
% (deleted 3/10/07, per suggestion of Mel O'Cat:)
% The syntax of the
% Metamath language ensures that if a set of substitutions exists, it
% will be unique.

The general algorithm for unification described in the literature is
somewhat complex.
However, in the case of Metamath it is intentionally trivial.
Mandatory hypotheses must be
pushed on the proof stack in the order in which they appear.
In addition, each variable must have its type specified
with a \texttt{\$f} hypothesis before it is used
and that each \texttt{\$f} hypothesis
have the restricted syntax of a typecode (a constant) followed by a variable.
The typecode in the \texttt{\$f} hypothesis must match the first symbol of
the corresponding RPN stack entry (which will also be a constant), so
the only possible match for the variable in the \texttt{\$f} hypothesis is
the sequence of symbols in the stack entry after the initial constant.

In the Proof Assistant\index{Proof Assistant}, a more general unification
algorithm is used.  While a proof is being developed, sometimes not enough
information is available to determine a unique unification.  In this case
Metamath will ask you to pick the correct one.\index{ambiguous
unification}\index{unification!ambiguous}

\section{Extensions to the Metamath Language}\index{extended
language}

\subsection{Comments in the Metamath Language}\label{comments}
\index{markup notation}
\index{comments!markup notation}

The commenting feature allows you to annotate the contents of
a database.  Just as with most
computer languages, comments are ignored for the purpose of interpreting the
contents of the database. Comments effectively act as
additional white space\index{white
space} between tokens
when a database is parsed.

A comment may be placed at the beginning, end, or
between any two tokens\index{token} in a source file.

Comments have the following syntax:
\begin{center}
 \texttt{\$(} {\em text} \texttt{\$)}
\end{center}
Here,\index{\texttt{\$(} and \texttt{\$)} auxiliary
keywords}\index{comment} {\em text} is a string, possibly empty, of any
characters in Metamath's character set (p.~\pageref{spec1chars}), except
that the character strings \texttt{\$(} and \texttt{\$)} may not appear
in {\em text}.  Thus nested comments are not
permitted:\footnote{Computer languages have differing standards for
nested comments, and rather than picking one it was felt simplest not to
allow them at all, at least in the current version (0.177) of
Metamath\index{Metamath!limitations of version 0.177}.} Metamath will
complain if you give it
\begin{center}
 \texttt{\$( This is a \$( nested \$) comment.\ \$)}
\end{center}
To compensate for this non-nesting behavior, I often change all \texttt{\$}'s
to \texttt{@}'s in sections of Metamath code I wish to comment out.

The Metamath program supports a number of markup mechanisms and conventions
to generate good-looking results in \LaTeX\ and {\sc html},
as discussed below.
These markup features have to do only with how the comments are typeset,
and have no effect on how Metamath verifies the proofs in the database.
The improper
use of them may result in incorrectly typeset output, but no Metamath
error messages will result during the \texttt{read} and \texttt{verify
proof} commands.  (However, the \texttt{write
theorem\texttt{\char`\_}list} command
will check for markup errors as a side-effect of its
{\sc html} generation.)
Section~\ref{texout} has instructions for creating \LaTeX\ output, and
section~\ref{htmlout} has instructions for creating
{\sc html}\index{HTML} output.

\subsubsection{Headings}\label{commentheadings}

If the \texttt{\$(} is immediately followed by a new line
starting with a heading marker, it is a header.
This can start with:

\begin{itemize}
 \item[] \texttt{\#\#\#\#} - major part header
 \item[] \texttt{\#*\#*} - section header
 \item[] \texttt{=-=-} - subsection header
 \item[] \texttt{-.-.} - subsubsection header
\end{itemize}

The line following the marker line
will be used for the table of contents entry, after trimming spaces.
The next line should be another (closing) matching marker line.
Any text after that
but before the closing \texttt{\$}, such as an extended description of the
section, will be included on the \texttt{mmtheoremsNNN.html} page.

For more information, run
\texttt{help write theorem\char`\_list}.

\subsubsection{Math mode}
\label{mathcomments}
\index{\texttt{`} inside comments}
\index{\texttt{\char`\~} inside comments}
\index{math mode}

Inside of comments, a string of tokens\index{token} enclosed in
grave accents\index{grave accent (\texttt{`})} (\texttt{`}) will be converted
to standard mathematical symbols during
{\sc HTML}\index{HTML} or \LaTeX\ output
typesetting,\index{latex@{\LaTeX}} according to the information in the
special \texttt{\$t}\index{\texttt{\$t} comment}\index{typesetting
comment} comment in the database
(see section~\ref{tcomment} for information about the typesetting
comment, and Appendix~\ref{ASCII} to see examples of its results).

The first grave accent\index{grave accent (\texttt{`})} \texttt{`}
causes the output processor to enter {\bf math mode}\index{math mode}
and the second one exits it.
In this
mode, the characters following the \texttt{`} are interpreted as a
sequence of math symbol tokens separated by white space\index{white
space}.  The tokens are looked up in the \texttt{\$t}
comment\index{\texttt{\$t} comment}\index{typesetting comment} and if
found, they will be replaced by the standard mathematical symbols that
they correspond to before being placed in the typeset output file.  If
not found, the symbol will be output as is and a warning will be issued.
The tokens do not have to be active in the database, although a warning
will be issued if they are not declared with \texttt{\$c} or
\texttt{\$v} statements.

Two consecutive
grave accents \texttt{``} are treated as a single actual grave accent
(both inside and outside of math mode) and will not cause the output
processor to enter or exit math mode.

Here is an example of its use\index{Pierce's axiom}:
\begin{center}
\texttt{\$( Pierce's axiom, ` ( ( ph -> ps ) -> ph ) -> ph ` ,\\
         is not very intuitive. \$)}
\end{center}
becomes
\begin{center}
   \texttt{\$(} Pierce's axiom, $((\varphi \rightarrow \psi)\rightarrow
\varphi)\rightarrow \varphi$, is not very intuitive. \texttt{\$)}
\end{center}

Note that the math symbol tokens\index{token} must be surrounded by white
space\index{white space}.
%, since there is no context that allows ambiguity to be
%resolved, as is the case with math symbol sequences in some of the Metamath
%statements.
White space should also surround the \texttt{`}
delimiters.

The math mode feature also gives you a quick and easy way to generate
text containing mathematical symbols, independently of the intended
purpose of Metamath.\index{Metamath!using as a math editor} To do this,
simply create your text with grave accents surrounding your formulas,
after making sure that your math symbols are mapped to \LaTeX\ symbols
as described in Appendix~\ref{ASCII}.  It is easier if you start with a
database with predefined symbols such as \texttt{set.mm}.  Use your
grave-quoted math string to replace an existing comment, then typeset
the statement corresponding to that comment following the instructions
from the \texttt{help tex} command in the Metamath program.  You will
then probably want to edit the resulting file with a text editor to fine
tune it to your exact needs.

\subsubsection{Label Mode}\index{label mode}

Outside of math mode, a tilde\index{tilde (\texttt{\char`\~})} \verb/~/
indicates to Metamath's\index{Metamath} output processor that the
token\index{token} that follows (i.e.\ the characters up to the next
white space\index{white space}) represents a statement label or URL.
This formatting mode is called {\bf label mode}\index{label mode}.
If a literal tilde
is desired (outside of math mode) instead of label mode,
use two tildes in a row to represent it.

When generating a \LaTeX\ output file,
the following token will be formatted in \texttt{typewriter}
font, and the tilde removed, to make it stand out from the rest of the text.
This formatting will be applied to all characters after the
tilde up to the first white space\index{white space}.
Whether
or not the token is an actual statement label is not checked, and the
token does not have to have the correct syntax for a label; no error
messages will be produced.  The only effect of the label mode on the
output is that typewriter font will be used for the tokens that are
placed in the \LaTeX\ output file.

When generating {\sc html},
the tokens after the tilde {\em must} be a URL (either http: or https:)
or a valid label.
Error messages will be issued during that output if they aren't.
A hyperlink will be generated to that URL or label.

\subsubsection{Link to bibliographical reference}\index{citation}%
\index{link to bibliographical reference}

Bibliographical references are handled specially when generating
{\sc html} if formatted specially.
Text in the form \texttt{[}{\em author}\texttt{]}
is considered a link to a bibliographical reference.
See \texttt{help html} and \texttt{help write
bibliography} in the Metamath program for more
information.
% \index{\texttt{\char`\[}\ldots\texttt{]} inside comments}
See also Sections~\ref{tcomment} and \ref{wrbib}.

The \texttt{[}{\em author}\texttt{]} notation will also create an entry in
the bibliography cross-reference file generated by \texttt{write
bibliography} (Section~\ref{wrbib}) for {\sc HTML}.
For this to work properly, the
surrounding comment must be formatted as follows:
\begin{quote}
    {\em keyword} {\em label} {\em noise-word}
     \texttt{[}{\em author}\texttt{] p.} {\em number}
\end{quote}
for example
\begin{verbatim}
     Theorem 5.2 of [Monk] p. 223
\end{verbatim}
The {\em keyword} is not case sensitive and must be one of the following:
\begin{verbatim}
     theorem lemma definition compare proposition corollary
     axiom rule remark exercise problem notation example
     property figure postulate equation scheme chapter
\end{verbatim}
The optional {\em label} may consist of more than one
(non-{\em keyword} and non-{\em noise-word}) word.
The optional {\em noise-word} is one of:
\begin{verbatim}
     of in from on
\end{verbatim}
and is  ignored when the cross-reference file is created.  The
\texttt{write
biblio\-graphy} command will perform error checking to verify the
above format.\index{error checking}

\subsubsection{Parentheticals}\label{parentheticals}

The end of a comment may include one or more parenthicals, that is,
statements enclosed in parentheses.
The Metamath program looks for certain parentheticals and can issue
warnings based on them.
They are:

\begin{itemize}
 \item[] \texttt{(Contributed by }
   \textit{NAME}\texttt{,} \textit{DATE}\texttt{.)} -
   document the original contributor's name and the date it was created.
 \item[] \texttt{(Revised by }
   \textit{NAME}\texttt{,} \textit{DATE}\texttt{.)} -
   document the contributor's name and creation date
   that resulted in significant revision
   (not just an automated minimization or shortening).
 \item[] \texttt{(Proof shortened by }
   \textit{NAME}\texttt{,} \textit{DATE}\texttt{.)} -
   document the contributor's name and date that developed a significant
   shortening of the proof (not just an automated minimization).
 \item[] \texttt{(Proof modification is discouraged.)} -
   Note that this proof should normally not be modified.
 \item[] \texttt{(New usage is discouraged.)} -
   Note that this assertion should normally not be used.
\end{itemize}

The \textit{DATE} must be in form YYYY-MMM-DD, where MMM is the
English abbreviation of that month.

\subsubsection{Other markup}\label{othermarkup}
\index{markup notation}

There are other markup notations for generating good-looking results
beyond math mode and label mode:

\begin{itemize}
 \item[]
         \texttt{\char`\_} (underscore)\index{\texttt{\char`\_} inside comments} -
             Italicize text starting from
              {\em space}\texttt{\char`\_}{\em non-space} (i.e.\ \texttt{\char`\_}
              with a space before it and a non-space character after it) until
             the next
             {\em non-space}\texttt{\char`\_}{\em space}.  Normal
             punctuation (e.g.\ a trailing
             comma or period) is ignored when determining {\em space}.
 \item[]
         \texttt{\char`\_} (underscore) - {\em
         non-space}\texttt{\char`\_}{\em non-space-string}, where
          {\em non-space-string} is a string of non-space characters,
         will make {\em non-space-string} become a subscript.
 \item[]
         \texttt{<HTML>}...\texttt{</HTML>} - do not convert
         ``\texttt{<}'' and ``\texttt{>}''
         in the enclosed text when generating {\sc HTML},
         otherwise process markup normally. This allows direct insertion
         of {\sc html} commands.
 \item[]
       ``\texttt{\&}ref\texttt{;}'' - insert an {\sc HTML}
         character reference.
         This is how to insert arbitrary Unicode characters
         (such as accented characters).  Currently only directly supported
         when generating {\sc HTML}.
\end{itemize}

It is recommended that spaces surround any \texttt{\char`\~} and
\texttt{`} tokens in the comment and that a space follow the {\em label}
after a \texttt{\char`\~} token.  This will make global substitutions
to change labels and symbol names much easier and also eliminate any
future chance of ambiguity.  Spaces around these tokens are automatically
removed in the final output to conform with normal rules of punctuation;
for example, a space between a trailing \texttt{`} and a left parenthesis
will be removed.

A good way to become familiar with the markup notation is to look at
the extensive examples in the \texttt{set.mm} database.

\subsection{The Typesetting Comment (\texttt{\$t})}\label{tcomment}

The typesetting comment \texttt{\$t} in the input database file
provides the information necessary to produce good-looking results.
It provides \LaTeX\ and {\sc html}
definitions for math symbols,
as well supporting as some
customization of the generated web page.
If you add a new token to a database, you should also
update the \texttt{\$t} comment information if you want to eventually
create output in \LaTeX\ or {\sc HTML}.
See the
\texttt{set.mm}\index{set theory database (\texttt{set.mm})} database
file for an extensive example of a \texttt{\$t} comment illustrating
many of the features described below.

Programs that do not need to generate good-looking presentation results,
such as programs that only verify Metamath databases,
can completely ignore typesetting comments
and just treat them as normal comments.
Even the Metamath program only consults the
\texttt{\$t} comment information when it needs to generate typeset output
in \LaTeX\ or {\sc HTML}
(e.g., when you open a \LaTeX\ output file with the \texttt{open tex} command).

We will first discuss the syntax of typesetting comments, and then
briefly discuss how this can be used within the Metamath program.

\subsubsection{Typesetting Comment Syntax Overview}

The typesetting comment is identified by the token
\texttt{\$t}\index{\texttt{\$t} comment}\index{typesetting comment} in
the comment, and the typesetting comment ends at the matching
\texttt{\$)}:
\[
  \mbox{\tt \$(\ }
  \mbox{\tt \$t\ }
  \underbrace{
    \mbox{\tt \ \ \ \ \ \ \ \ \ \ \ }
    \cdots
    \mbox{\tt \ \ \ \ \ \ \ \ \ \ \ }
  }_{\mbox{Typesetting definitions go here}}
  \mbox{\tt \ \$)}
\]

There must be one or more white space characters, and only white space
characters, between the \texttt{\$(} that starts the comment
and the \texttt{\$t} symbol,
and the \texttt{\$t} must be followed by one
or more white space characters
(see section \ref{whitespace} for the definition of white space characters).
The typesetting comment continues until the comment end token \texttt{\$)}
(which must be preceded by one or more white space characters).

In version 0.177\index{Metamath!limitations of version 0.177} of the
Metamath program, there may be only one \texttt{\$t} comment in a
database.  This restriction may be lifted in the future to allow
many \texttt{\$t} comments in a database.

Between the \texttt{\$t} symbol (and its following white space) and the
comment end token \texttt{\$)} (and its preceding white space)
is a sequence of one or more typesetting definitions, where
each definition has the form
\textit{definition-type arg arg ... ;}.
Each of the zero or more \textit{arg} values
can be either a typesetting data or a keyword
(what keywords are allowed, and where, depends on the specific
\textit{definition-type}).
The \textit{definition-type}, and each argument \textit{arg},
are separated by one or more white space characters.
Every definition ends in an unquoted semicolon;
white space is not required before the terminating semicolon of a definition.
Each definition should start on a new line.\footnote{This
restriction of the current version of Metamath
(0.177)\index{Metamath!limitations of version 0.177} may be removed
in a future version, but you should do it anyway for readability.}

For example, this typesetting definition:
\begin{center}
 \verb$latexdef "C_" as "\subseteq";$
\end{center}
defines the token \verb$C_$ as the \LaTeX\ symbol $\subseteq$ (which means
``subset'').

Typesetting data is a sequence of one or more quoted strings
(if there is more than one, they are connected by \texttt{\char`\+}).
Often a single quoted string is used to provide data for a definition, using
either double (\texttt{\char`\"}) or single (\texttt{'}) quotation marks.
However,
{\em a quoted string (enclosed in quotation marks) may not include
line breaks.}
A quoted string
may include a quotation mark that matches the enclosing quotes by repeating
the quotation mark twice.  Here are some examples:

\begin{tabu}   { l l }
\textbf{Example} & \textbf{Meaning} \\
\texttt{\char`\"a\char`\"\char`\"b\char`\"} & \texttt{a\char`\"b} \\
\texttt{'c''d'} & \texttt{c'd} \\
\texttt{\char`\"e''f\char`\"} & \texttt{e''f} \\
\texttt{'g\char`\"\char`\"h'} & \texttt{g\char`\"\char`\"h} \\
\end{tabu}

Finally, a long quoted string
may be broken up into multiple quoted strings (considered, as a whole,
a single quoted string) and joined with \texttt{\char`\+}.
You can even use multiple lines as long as a
'+' is at the end of every line except the last one.
The \texttt{\char`\+} should be preceded and followed by at least one
white space character.
Thus, for example,
\begin{center}
 \texttt{\char`\"ab\char`\"\ \char`\+\ \char`\"cd\char`\"
    \ \char`\+\ \\ 'ef'}
\end{center}
is the same as
\begin{center}
 \texttt{\char`\"abcdef\char`\"}
\end{center}

{\sc c}-style comments \texttt{/*}\ldots\texttt{*/} are also supported.

In practice, whenever you add a new math token you will often want to add
typesetting definitions using
\texttt{latexdef}, \texttt{htmldef}, and
\texttt{althtmldef}, as described below.
That way, they will all be up to date.
Of course, whether or not you want to use all three definitions will
depend on how the database is intended to be used.

Below we discuss the different possible \textit{definition-kind} options.
We will show data surrounded by double quotes (in practice they can also use
single quotes and/or be a sequence joined by \texttt{+}s).
We will use specific names for the \textit{data} to make clear what
the data is used for, such as
{\em math-token} (for a Metamath math token,
{\em latex-string} (for string to be placed in a \LaTeX\ stream),
{\em {\sc html}-code} (for {\sc html} code),
and {\em filename} (for a filename).

\subsubsection{Typesetting Comment - \LaTeX}

The syntax for a \LaTeX\ definition is:
\begin{center}
 \texttt{latexdef "}{\em math-token}\texttt{" as "}{\em latex-string}\texttt{";}
\end{center}
\index{latex definitions@\LaTeX\ definitions}%
\index{\texttt{latexdef} statement}

The {\em token-string} and {\em latex-string} are the data
(character strings) for
the token and the \LaTeX\ definition of the token, respectively,

These \LaTeX\ definitions are used by the Metamath program
when it is asked to product \LaTeX output using
the \texttt{write tex} command.

\subsubsection{Typesetting Comment - {\sc html}}

The key kinds of {\sc HTML} definitions have the following syntax:

\vskip 1ex
    \texttt{htmldef "}{\em math-token}\texttt{" as "}{\em
    {\sc html}-code}\texttt{";}\index{\texttt{htmldef} statement}
                    \ \ \ \ \ \ldots

    \texttt{althtmldef "}{\em math-token}\texttt{" as "}{\em
{\sc html}-code}\texttt{";}\index{\texttt{althtmldef} statement}

                    \ \ \ \ \ \ldots

Note that in {\sc HTML} there are two possible definitions for math tokens.
This feature is useful when
an alternate representation of symbols is desired, for example one that
uses Unicode entities and another uses {\sc gif} images.

There are many other typesetting definitions that can control {\sc HTML}.
These include:

\vskip 1ex

    \texttt{htmldef "}{\em math-token}\texttt{" as "}{\em {\sc
    html}-code}\texttt{";}

    \texttt{htmltitle "}{\em {\sc html}-code}\texttt{";}%
\index{\texttt{htmltitle} statement}

    \texttt{htmlhome "}{\em {\sc html}-code}\texttt{";}%
\index{\texttt{htmlhome} statement}

    \texttt{htmlvarcolor "}{\em {\sc html}-code}\texttt{";}%
\index{\texttt{htmlvarcolor} statement}

    \texttt{htmlbibliography "}{\em filename}\texttt{";}%
\index{\texttt{htmlbibliography} statement}

\vskip 1ex

\noindent The \texttt{htmltitle} is the {\sc html} code for a common
title, such as ``Metamath Proof Explorer.''  The \texttt{htmlhome} is
code for a link back to the home page.  The \texttt{htmlvarcolor} is
code for a color key that appears at the bottom of each proof.  The file
specified by {\em filename} is an {\sc html} file that is assumed to
have a \texttt{<A NAME=}\ldots\texttt{>} tag for each bibiographic
reference in the database comments.  For example, if
\texttt{[Monk]}\index{\texttt{\char`\[}\ldots\texttt{]} inside comments}
occurs in the comment for a theorem, then \texttt{<A NAME='Monk'>} must
be present in the file; if not, a warning message is given.

Associated with
\texttt{althtmldef}
are the statements
\vskip 1ex

    \texttt{htmldir "}{\em
      directoryname}\texttt{";}\index{\texttt{htmldir} statement}

    \texttt{althtmldir "}{\em
     directoryname}\texttt{";}\index{\texttt{althtmldir} statement}

\vskip 1ex
\noindent giving the directories of the {\sc gif} and Unicode versions
respectively; their purpose is to provide cross-linking between the
two versions in the generated web pages.

When two different types of pages need to be produced from a single
database, such as the Hilbert Space Explorer that extends the Metamath
Proof Explorer, ``extended'' variables may be declared in the
\texttt{\$t} comment:
\vskip 1ex

    \texttt{exthtmltitle "}{\em {\sc html}-code}\texttt{";}%
\index{\texttt{exthtmltitle} statement}

    \texttt{exthtmlhome "}{\em {\sc html}-code}\texttt{";}%
\index{\texttt{exthtmlhome} statement}

    \texttt{exthtmlbibliography "}{\em filename}\texttt{";}%
\index{\texttt{exthtmlbibliography} statement}

\vskip 1ex
\noindent When these are declared, you also must declare
\vskip 1ex

    \texttt{exthtmllabel "}{\em label}\texttt{";}%
\index{\texttt{exthtmllabel} statement}

\vskip 1ex \noindent that identifies the database statement where the
``extended'' section of the database starts (in our example, where the
Hilbert Space Explorer starts).  During the generation of web pages for
that starting statement and the statements after it, the {\sc html} code
assigned to \texttt{exthtmltitle} and \texttt{exthtmlhome} is used
instead of that assigned to \texttt{htmltitle} and \texttt{htmlhome},
respectively.

\begin{sloppy}
\subsection{Additional Information Com\-ment (\texttt{\$j})} \label{jcomment}
\end{sloppy}

The additional information comment, aka the
\texttt{\$j}\index{\texttt{\$j} comment}\index{additional information comment}
comment,
provides a way to add additional structured information that can
be optionally parsed by systems.

The additional information comment is parsed the same way as the
typesetting comment (\texttt{\$t}) (see section \ref{tcomment}).
That is,
the additional information comment begins with the token
\texttt{\$j} within a comment,
and continues until the comment close \texttt{\$)}.
Within an additional information comment is a sequence of one or more
commands of the form \texttt{command arg arg ... ;}
where each of the zero or more \texttt{arg} values
can be either a quoted string or a keyword.
Note that every command ends in an unquoted semicolon.
If a verifier is parsing an additional information comment, but
doesn't recognize a particular command, it must skip the command
by finding the end of the command (an unquoted semicolon).

A database may have 0 or more additional information comments.
Note, however, that a verifier may ignore these comments entirely or only
process certain commands in an additional information comment.
The \texttt{mmj2} verifier supports many commands in additional information
comments.
We encourage systems that process additional information comments
to coordinate so that they will use the same command for the same effect.

Examples of additional information comments with various commands
(from the \texttt{set.mm} database) are:

\begin{itemize}
   \item Define the syntax and logical typecodes,
     and declare that our grammar is
     unambiguous (verifiable using the KLR parser, with compositing depth 5).
\begin{verbatim}
  $( $j
    syntax 'wff';
    syntax '|-' as 'wff';
    unambiguous 'klr 5';
  $)
\end{verbatim}

   \item Register $\lnot$ and $\rightarrow$ as primitive expressions
           (lacking definitions).
\begin{verbatim}
  $( $j primitive 'wn' 'wi'; $)
\end{verbatim}

   \item There is a special justification for \texttt{df-bi}.
\begin{verbatim}
  $( $j justification 'bijust' for 'df-bi'; $)
\end{verbatim}

   \item Register $\leftrightarrow$ as an equality for its type (wff).
\begin{verbatim}
  $( $j
    equality 'wb' from 'biid' 'bicomi' 'bitri';
    definition 'dfbi1' for 'wb';
  $)
\end{verbatim}

   \item Theorem \texttt{notbii} is the congruence law for negation.
\begin{verbatim}
  $( $j congruence 'notbii'; $)
\end{verbatim}

   \item Add \texttt{setvar} as a typecode.
\begin{verbatim}
  $( $j syntax 'setvar'; $)
\end{verbatim}

   \item Register $=$ as an equality for its type (\texttt{class}).
\begin{verbatim}
  $( $j equality 'wceq' from 'eqid' 'eqcomi' 'eqtri'; $)
\end{verbatim}

\end{itemize}


\subsection{Including Other Files in a Metamath Source File} \label{include}
\index{\texttt{\$[} and \texttt{\$]} auxiliary keywords}

The keywords \texttt{\$[} and \texttt{\$]} specify a file to be
included\index{included file}\index{file inclusion} at that point in a
Metamath\index{Metamath} source file\index{source file}.  The syntax for
including a file is as follows:
\begin{center}
\texttt{\$[} {\em file-name} \texttt{\$]}
\end{center}

The {\em file-name} should be a single token\index{token} with the same syntax
as a math symbol (i.e., all 93 non-whitespace
printable characters other than \texttt{\$} are
allowed, subject to the file-naming limitations of your operating system).
Comments may appear between the \texttt{\$[} and \texttt{\$]} keywords.  Included
files may include other files, which may in turn include other files, and so
on.

For example, suppose you want to use the set theory database as the starting
point for your own theory.  The first line in your file could be
\begin{center}
\texttt{\$[ set.mm \$]}
\end{center} All of the information (axioms, theorems,
etc.) in \texttt{set.mm} and any files that {\em it} includes will become
available for you to reference in your file. This can help make your work more
modular. A drawback to including files is that if you change the name of a
symbol or the label of a statement, you must also remember to update any
references in any file that includes it.


The naming conventions for included files are the same as those of your
operating system.\footnote{On the Macintosh, prior to Mac OS X,
 a colon is used to separate disk
and folder names from your file name.  For example, {\em volume}\texttt{:}{\em
file-name} refers to the root directory, {\em volume}\texttt{:}{\em
folder-name}\texttt{:}{\em file-name} refers to a folder in root, and {\em
volume}\texttt{:}{\em folder-name}\texttt{:}\ldots\texttt{:}{\em file-name} refers to a
deeper folder.  A simple {\em file-name} refers to a file in the folder from
which you launch the Metamath application.  Under Mac OS X and later,
the Metamath program is run under the Terminal application, which
conforms to Unix naming conventions.}\index{Macintosh file
names}\index{file names!Macintosh}\label{includef} For compatibility among
operating systems, you should keep the file names as simple as possible.  A
good convention to use is {\em file}\texttt{.mm} where {\em file} is eight
characters or less, in lower case.

There is no limit to the nesting depth of included files.  One thing that you
should be aware of is that if two included files themselves include a common
third file, only the {\em first} reference to this common file will be read
in.  This allows you to include two or more files that build on a common
starting file without having to worry about label and symbol conflicts that
would occur if the common file were read in more than once.  (In fact, if a
file includes itself, the self-reference will be ignored, although of course
it would not make any sense to do that.)  This feature also means, however,
that if you try to include a common file in several inner blocks, the result
might not be what you expect, since only the first reference will be replaced
with the included file (unlike the include statement in most other computer
languages).  Thus you would normally include common files only in the
outermost block\index{outermost block}.

\subsection{Compressed Proof Format}\label{compressed1}\index{compressed
proof}\index{proof!compressed}

The proof notation presented in Section~\ref{proof} is called a
{\bf normal proof}\index{normal proof}\index{proof!normal} and in principle is
sufficient to express any proof.  However, proofs often contain steps and
subproofs that are identical.  This is particularly true in typical
Metamath\index{Metamath} applications, because Metamath requires that the math
symbol sequence (usually containing a formula) at each step be separately
constructed, that is, built up piece by piece. As a result, a lot of
repetition often results.  The {\bf compressed proof} format allows Metamath
to take advantage of this redundancy to shorten proofs.

The specification for the compressed proof format is given in
Appen\-dix~\ref{compressed}.

Normally you need not concern yourself with the details of the compressed
proof format, since the Metamath program will allow you to convert from
the normal format to the compressed format with ease, and will also
automatically convert from the compressed format when proofs are displayed.
The overall structure of the compressed format is as follows:
\begin{center}
  \texttt{\$= ( } {\em label-list} \texttt{) } {\em compressed-proof\ }\ \texttt{\$.}
\end{center}
\index{\texttt{\$=} keyword}
The first \texttt{(} serves as a flag to Metamath that a compressed proof
follows.  The {\em label-list} includes all statements referred to by the
proof except the mandatory hypotheses\index{mandatory hypothesis}.  The {\em
compressed-proof} is a compact encoding of the proof, using upper-case
letters, and can be thought of as a large integer in base 26.  White
space\index{white space} inside a {\em compressed-proof} is
optional and is ignored.

It is important to note that the order of the mandatory hypotheses of
the statement being proved must not be changed if the compressed proof
format is used, otherwise the proof will become incorrect.  The reason
for this is that the mandatory hypotheses are not mentioned explicitly
in the compressed proof in order to make the compression more efficient.
If you wish to change the order of mandatory hypotheses, you must first
convert the proof back to normal format using the \texttt{save proof
{\em statement} /normal}\index{\texttt{save proof} command} command.
Later, you can go back to compressed format with \texttt{save proof {\em
statement} /compressed}.

During error checking with the \texttt{verify proof} command, an error
found in a compressed proof may point to a character in {\em
compressed-proof}, which may not be very meaningful to you.  In this
case, try to \texttt{save proof /normal} first, then do the
\texttt{verify proof} again.  In general, it is best to make sure a
proof is correct before saving it in compressed format, because severe
errors are less likely to be recoverable than in normal format.

\subsection{Specifying Unknown Proofs or Subproofs}\label{unknown}

In a proof under development, any step or subproof that is not yet known
may be represented with a single \texttt{?}.  For the purposes of
parsing the proof, the \texttt{?}\ \index{\texttt{]}@\texttt{?}\ inside
proofs} will push a single entry onto the RPN stack just as if it were a
hypothesis.  While developing a proof with the Proof
Assistant\index{Proof Assistant}, a partially developed proof may be
saved with the \texttt{save new{\char`\_}proof}\index{\texttt{save
new{\char`\_}proof} command} command, and \texttt{?}'s will be placed at
the appropriate places.

All \texttt{\$p}\index{\texttt{\$p} statement} statements must have
proofs, even if they are entirely unknown.  Before creating a proof with
the Proof Assistant, you should specify a completely unknown proof as
follows:
\begin{center}
  {\em label} \texttt{\$p} {\em statement} \texttt{\$= ?\ \$.}
\end{center}
\index{\texttt{\$=} keyword}
\index{\texttt{]}@\texttt{?}\ inside proofs}

The \texttt{verify proof}\index{\texttt{verify proof} command} command
will check the known portions of a partial proof for errors, but will
warn you that the statement has not been proved.

Note that partially developed proofs may be saved in compressed format
if desired.  In this case, you will see one or more \texttt{?}'s in the
{\em compressed-proof} part.\index{compressed
proof}\index{proof!compressed}

\section{Axioms vs.\ Definitions}\label{definitions}

The \textit{basic}
Metamath\index{Metamath} language and program
make no distinction\index{axiom vs.\
definition} between axioms\index{axiom} and
definitions.\index{definition} The \texttt{\$a}\index{\texttt{\$a}
statement} statement is used for both.  At first, this may seem
puzzling.  In the minds of many mathematicians, the distinction is
clear, even obvious, and hardly worth discussing.  A definition is
considered to be merely an abbreviation that can be replaced by the
expression for which it stands; although unless one actually does this,
to be precise then one should say that a theorem\index{theorem} is a
consequence of the axioms {\em and} the definitions that are used in the
formulation of the theorem \cite[p.~20]{Behnke}.\index{Behnke, H.}

\subsection{What is a Definition?}

What is a definition?  In its simplest form, a definition introduces a new
symbol and provides an unambiguous rule to transform an expression containing
the new symbol to one without it.  The concept of a ``proper
definition''\index{proper definition}\index{definition!proper} (as opposed to
a creative definition)\index{creative definition}\index{definition!creative}
that is usually agreed upon is (1) the definition should not strengthen the
language and (2) any symbols introduced by the definition should be eliminable
from the language \cite{Nemesszeghy}\index{Nemesszeghy, E. Z.}.  In other
words, they are mere typographical conveniences that do not belong to the
system and are theoretically superfluous.  This may seem obvious, but in fact
the nature of definitions can be subtle, sometimes requiring difficult
metatheorems to establish that they are not creative.

A more conservative stance was taken by logician S.
Le\'{s}niewski.\index{Le\'{s}niewski, S.}
\begin{quote}
Le\'{s}niewski
regards definitions as theses of the system.  In this respect they do
not differ either from the axioms or from theorems, i.e.\ from the
theses added to the system on the basis of the rule of substitution or
the rule of detachment [modus ponens].  Once definitions have been
accepted as theses of the system, it becomes necessary to consider them
as true propositions in the same sense in which axioms are true
\cite{Lejewski}.
\end{quote}\index{Lejewski, Czeslaw}

Let us look at some simple examples of definitions in propositional
calculus.  Consider the definition of logical {\sc or}
(disjunction):\index{disjunction ($\vee$)} ``$P\vee Q$ denotes $\neg P
\rightarrow Q$ (not $P$ implies $Q$).''  It is very easy to recognize a
statement making use of this definition, because it introduces the new
symbol $\vee$ that did not previously exist in the language.  It is easy
to see that no new theorems of the original language will result from
this definition.

Next, consider a definition that eliminates parentheses:  ``$P
\rightarrow Q\rightarrow R$ denotes $P\rightarrow (Q \rightarrow R)$.''
This is more subtle, because no new symbols are introduced.  The reason
this definition is considered proper is that no new symbol sequences
that are valid wffs (well-formed formulas)\index{well-formed formula
(wff)} in the original language will result from the definition, since
``$P \rightarrow Q\rightarrow R$'' is not a wff in the original
language.  Here, we implicitly make use of the fact that there is a
decision procedure that allows us to determine whether or not a symbol
sequence is a wff, and this fact allows us to use symbol sequences that
are not wffs to represent other things (such as wffs) by means of the
definition.  However, to justify the definition as not being creative we
need to prove that ``$P \rightarrow Q\rightarrow R$'' is in fact not a
wff in the original language, and this is more difficult than in the
case where we simply introduce a new symbol.

%Now let's take this reasoning to an extreme.  Propositional calculus is a
%decidable theory,\footnote{This means that a mechanical algorithm exists to
%determine whether or not a wff is a theorem.} so in principle we could make use
%of symbol sequences that are not theorems to represent other things (say, to
%encode actual theorems in a more compact way).  For example, let us extend the
%language by defining a wff ``$P$'' in the extended language as the theorem
%``$P\rightarrow P$''\footnote{This is one of the first theorems proved in the
%Metamath database \texttt{set.mm}.}\index{set
%theory database (\texttt{set.mm})} in the original language whenever ``$P$'' is
%not a theorem in the original language.  In the extended language, any wff
%``$Q$'' thus represents a theorem; to find out what theorem (in the original
%language) ``$Q$'' represents, we determine whether ``$Q$'' is a theorem in the
%original language (before the definition was introduced).  If so, we're done; if
%not, we replace ``$Q$'' by ``$Q\rightarrow Q$'' to eliminate the definition.
%This definition is therefore eliminable, and it does not ``strengthen'' the
%language because any wff that is not a theorem is not in the set of statements
%provable in the original language and thus is available for use by definitions.
%
%Of course, a definition such as this would render practically useless the
%communication of theorems of propositional calculus; but
%this is just a human shortcoming, since we can't always easily discern what is
%and is not a theorem by inspection.  In fact, the extended theory with this
%definition has no more and no less information than the original theory; it just
%expresses certain theorems of the form ``$P\rightarrow P$''
%in a more compact way.
%
%The point here is that what constitutes a proper definition is a matter of
%judgment about whether a symbol sequence can easily be recognized by a human
%as invalid in some sense (for example, not a wff); if so, the symbol sequence
%can be appropriated for use by a definition in order to make the extended
%language more compact.  Metamath\index{Metamath} lacks the ability to make this
%judgment, since as far as Metamath is concerned the definition of a wff, for
%example, is arbitrary.  You define for Metamath how wffs\index{well-formed
%formula (wff)} are constructed according to your own preferred style.  The
%concept of a wff may not even exist in a given formal system\index{formal
%system}.  Metamath treats all definitions as if they were new axioms, and it
%is up to the human mathematician to judge whether the definition is ``proper''
%'\index{proper definition}\index{definition!proper} in some agreed-upon way.

What constitutes a definition\index{definition} versus\index{axiom vs.\
definition} an axiom\index{axiom} is sometimes arbitrary in mathematical
literature.  For example, the connectives $\vee$ ({\sc or}), $\wedge$
({\sc and}), and $\leftrightarrow$ (equivalent to) in propositional
calculus are usually considered defined symbols that can be used as
abbreviations for expressions containing the ``primitive'' connectives
$\rightarrow$ and $\neg$.  This is the way we treat them in the standard
logic and set theory database \texttt{set.mm}\index{set theory database
(\texttt{set.mm})}.  However, the first three connectives can also be
considered ``primitive,'' and axiom systems have been devised that treat
all of them as such.  For example,
\cite[p.~35]{Goodstein}\index{Goodstein, R. L.} presents one with 15
axioms, some of which in fact coincide with what we have chosen to call
definitions in \texttt{set.mm}.  In certain subsets of classical
propositional calculus, such as the intuitionist
fragment\index{intuitionism}, it can be shown that one cannot make do
with just $\rightarrow$ and $\neg$ but must treat additional connectives
as primitive in order for the system to make sense.\footnote{Two nice
systems that make the transition from intuitionistic and other weak
fragments to classical logic just by adding axioms are given in
\cite{Robinsont}\index{Robinson, T. Thacher}.}

\subsection{The Approach to Definitions in \texttt{set.mm}}

In set theory, recursive definitions define a newly introduced symbol in
terms of itself.
The justification of recursive definitions, using
several ``recursion theorems,'' is usually one of the first
sophisticated proofs a student encounters when learning set theory, and
there is a significant amount of implicit metalogic behind a recursive
definition even though the definition itself is typically simple to
state.

Metamath itself has no built-in technical limitation that prevents
multiple-part recursive definitions in the traditional textbook style.
However, because the recursive definition requires advanced metalogic
to justify, eliminating a recursive definition is very difficult and
often not even shown in textbooks.

\subsubsection{Direct definitions instead of recursive definitions}

It is, however, possible to substitute one kind of complexity
for another.  We can eliminate the need for metalogical justification by
defining the operation directly with an explicit (but complicated)
expression, then deriving the recursive definition directly as a
theorem, using a recursion theorem ``in reverse.''
The elimination
of a direct definition is a matter of simple mechanical substitution.
We do this in
\texttt{set.mm}, as follows.

In \texttt{set.mm} our goal was to introduce almost all definitions in
the form of two expressions connected by either $\leftrightarrow$ or
$=$, where the thing being defined does not appear on the right hand
side.  Quine calls this form ``a genuine or direct definition'' \cite[p.
174]{Quine}\index{Quine, Willard Van Orman}, which makes the definitions
very easy to eliminate and the metalogic\index{metalogic} needed to
justify them as simple as possible.
Put another way, we had a goal of being able to
eliminate all definitions with direct mechanical substitution and to
verify easily the soundness of the definitions.

\subsubsection{Example of direct definitions}

We achieved this goal in almost all cases in \texttt{set.mm}.
Sometimes this makes the definitions more complex and less
intuitive.
For example, the traditional way to define addition of
natural numbers is to define an operation called {\em
successor}\index{successor} (which means ``plus one'' and is denoted by
``${\rm suc}$''), then define addition recursively\index{recursive
definition} with the two definitions $n + 0 = n$ and $m + {\rm suc}\,n =
{\rm suc} (m + n)$.  Although this definition seems simple and obvious,
the method to eliminate the definition is not obvious:  in the second
part of the definition, addition is defined in terms of itself.  By
eliminating the definition, we don't mean repeatedly applying it to
specific $m$ and $n$ but rather showing the explicit, closed-form
set-theoretical expression that $m + n$ represents, that will work for
any $m$ and $n$ and that does not have a $+$ sign on its right-hand
side.  For a recursive definition like this not to be circular
(creative), there are some hidden, underlying assumptions we must make,
for example that the natural numbers have a certain kind of order.

In \texttt{set.mm} we chose to start with the direct (though complex and
nonintuitive) definition then derive from it the standard recursive
definition.
For example, the closed-form definition used in \texttt{set.mm}
for the addition operation on ordinals\index{ordinal
addition}\index{addition!of ordinals} (of which natural numbers are a
subset) is

\setbox\startprefix=\hbox{\tt \ \ df-oadd\ \$a\ }
\setbox\contprefix=\hbox{\tt \ \ \ \ \ \ \ \ \ \ \ \ \ }
\startm
\m{\vdash}\m{+_o}\m{=}\m{(}\m{x}\m{\in}\m{{\rm On}}\m{,}\m{y}\m{\in}\m{{\rm
On}}\m{\mapsto}\m{(}\m{{\rm rec}}\m{(}\m{(}\m{z}\m{\in}\m{{\rm
V}}\m{\mapsto}\m{{\rm suc}}\m{z}\m{)}\m{,}\m{x}\m{)}\m{`}\m{y}\m{)}\m{)}
\endm
\noindent which depends on ${\rm rec}$.

\subsubsection{Recursion operators}

The above definition of \texttt{df-oadd} depends on the definition of
${\rm rec}$, a ``recursion operator''\index{recursion operator} with
the definition \texttt{df-rdg}:

\setbox\startprefix=\hbox{\tt \ \ df-rdg\ \$a\ }
\setbox\contprefix=\hbox{\tt \ \ \ \ \ \ \ \ \ \ \ \ }
\startm
\m{\vdash}\m{{\rm
rec}}\m{(}\m{F}\m{,}\m{I}\m{)}\m{=}\m{\mathrm{recs}}\m{(}\m{(}\m{g}\m{\in}\m{{\rm
V}}\m{\mapsto}\m{{\rm if}}\m{(}\m{g}\m{=}\m{\varnothing}\m{,}\m{I}\m{,}\m{{\rm
if}}\m{(}\m{{\rm Lim}}\m{{\rm dom}}\m{g}\m{,}\m{\bigcup}\m{{\rm
ran}}\m{g}\m{,}\m{(}\m{F}\m{`}\m{(}\m{g}\m{`}\m{\bigcup}\m{{\rm
dom}}\m{g}\m{)}\m{)}\m{)}\m{)}\m{)}\m{)}
\endm

\noindent which can be further broken down with definitions shown in
Section~\ref{setdefinitions}.

This definition of ${\rm rec}$
defines a recursive definition generator on ${\rm On}$ (the class of ordinal
numbers) with characteristic function $F$ and initial value $I$.
This operation allows us to define, with
compact direct definitions, functions that are usually defined in
textbooks with recursive definitions.
The price paid with our approach
is the complexity of our ${\rm rec}$ operation
(especially when {\tt df-recs} that it is built on is also eliminated).
But once we get past this hurdle, definitions that would otherwise be
recursive become relatively simple, as in for example {\tt oav}, from
which we prove the recursive textbook definition as theorems {\tt oa0}, {\tt
oasuc}, and {\tt oalim} (with the help of theorems {\tt rdg0}, {\tt rdgsuc},
and {\tt rdglim2a}).  We can also restrict the ${\rm rec}$ operation to
define otherwise recursive functions on the natural numbers $\omega$; see {\tt
fr0g} and {\tt frsuc}.  Our ${\rm rec}$ operation apparently does not appear
in published literature, although closely related is Definition 25.2 of
[Quine] p. 177, which he uses to ``turn...a recursion into a genuine or
direct definition" (p. 174).  Note that the ${\rm if}$ operations (see
{\tt df-if}) select cases based on whether the domain of $g$ is zero, a
successor, or a limit ordinal.

An important use of this definition ${\rm rec}$ is in the recursive sequence
generator {\tt df-seq} on the natural numbers (as a subset of the
complex infinite sequences such as the factorial function {\tt df-fac} and
integer powers {\tt df-exp}).

The definition of ${\rm rec}$ depends on ${\rm recs}$.
New direct usage of the more powerful (and more primitive) ${\rm recs}$
construct is discouraged, but it is available when needed.
This
defines a function $\mathrm{recs} ( F )$ on ${\rm On}$, the class of ordinal
numbers, by transfinite recursion given a rule $F$ which sets the next
value given all values so far.
Unlike {\tt df-rdg} which restricts the
update rule to use only the previous value, this version allows the
update rule to use all previous values, which is why it is described
as ``strong,'' although it is actually more primitive.  See {\tt
recsfnon} and {\tt recsval} for the primary contract of this definition.
It is defined as:

\setbox\startprefix=\hbox{\tt \ \ df-recs\ \$a\ }
\setbox\contprefix=\hbox{\tt \ \ \ \ \ \ \ \ \ \ \ \ \ }
\startm
\m{\vdash}\m{\mathrm{recs}}\m{(}\m{F}\m{)}\m{=}\m{\bigcup}\m{\{}\m{f}\m{|}\m{\exists}\m{x}\m{\in}\m{{\rm
On}}\m{(}\m{f}\m{{\rm
Fn}}\m{x}\m{\wedge}\m{\forall}\m{y}\m{\in}\m{x}\m{(}\m{f}\m{`}\m{y}\m{)}\m{=}\m{(}\m{F}\m{`}\m{(}\m{f}\m{\restriction}\m{y}\m{)}\m{)}\m{)}\m{\}}
\endm

\subsubsection{Closing comments on direct definitions}

From these direct definitions the simpler, more
intuitive recursive definition is derived as a set of theorems.\index{natural
number}\index{addition}\index{recursive definition}\index{ordinal addition}
The end result is the same, but we completely eliminate the rather complex
metalogic that justifies the recursive definition.

Recursive definitions are often considered more efficient and intuitive than
direct ones once the metalogic has been learned or possibly just accepted as
correct.  However, it was felt that direct definition in \texttt{set.mm}
maximizes rigor by minimizing metalogic.  It can be eliminated effortlessly,
something that is difficult to do with a recursive definition.

Again, Metamath itself has no built-in technical limitation that prevents
multiple-part recursive definitions in the traditional textbook style.
Instead, our goal is to eliminate all definitions with
direct mechanical substitution and to verify easily the soundness of
definitions.

\subsection{Adding Constraints on Definitions}

The basic Metamath language and the Metamath program do
not have any built-in constraints on definitions, since they are just
\$a statements.

However, nothing prevents a verification system from verifying
additional rules to impose further limitations on definitions.
For example, the \texttt{mmj2}\index{mmj2} program
supports various kinds of
additional information comments (see section \ref{jcomment}).
One of their uses is to optionally verify additional constraints,
including constraints to verify that definitions meet certain
requirements.
These additional checks are required by the
continuous integration (CI)\index{continuous integration (CI)}
checks of the
\texttt{set.mm}\index{set theory database (\texttt{set.mm})}%
\index{Metamath Proof Explorer}
database.
This approach enables us to optionally impose additional requirements
on definitions if we wish, without necessarily imposing those rules on
all databases or requiring all verification systems to implement them.
In addition, this allows us to impose specialized constraints tailored
to one database while not requiring other systems to implement
those specialized constraints.

We impose two constraints on the
\texttt{set.mm}\index{set theory database (\texttt{set.mm})}%
\index{Metamath Proof Explorer} database
via the \texttt{mmj2}\index{mmj2} program that are worth discussing here:
a parse check and a definition soundness check.

% On February 11, 2019 8:32:32 PM EST, saueran@oregonstate.edu wrote:
% The following addition to the end of set.mm is accepted by the mmj2
% parser and definition checker and the metamath verifier(at least it was
% when I checked, you should check it too), and creates a contradiction by
% proving the theorem |- ph.
% ${
% wleftp $a wff ( ( ph ) $.
% wbothp $a wff ( ph ) $.
% df-leftp $a |- ( ( ( ph ) <-> -. ph ) $.
% df-bothp $a |- ( ( ph ) <-> ph ) $.
% anything $p |- ph $=
%   ( wbothp wn wi wleftp df-leftp biimpi df-bothp mpbir mpbi simplim ax-mp)
%   ABZAMACZDZCZMOEZOCQAEZNDZRNAFGSHIOFJMNKLAHJ $.
% $}
%
% This particular problem is countered by enabling, within mmj2,
% SetParser,mmj.verify.LRParser

First,
we enable a parse check in \texttt{mmj2} (through its
\texttt{SetParser} command) that requires that all new definitions
must \textit{not} create an ambiguous parse for a KLR(5) parser.
This prevents some errors such as definitions with imbalanced parentheses.

Second, we run a definition soundness check specific to
\texttt{set.mm} or databases similar to it.
(through the \texttt{definitionCheck} macro).
Some \texttt{\$a} statements (including all ax-* statemnets)
are excluded from these checks, as they will
always fail this simple check,
but they are appropriate for most definitions.
This check imposes a set of additional rules:

\begin{enumerate}

\item New definitions must be introduced using $=$ or $\leftrightarrow$.

\item No \texttt{\$a} statement introduced before this one is allowed to use the
symbol being defined in this definition, and the definition is not
allowed to use itself (except once, in the definiendum).

\item Every variable in the definiens should not be distinct

\item Every dummy variable in the definiendum
are required to be distinct from each other and from variables in
the definiendum.
To determine this, the system will look for a "justification" theorem
in the database, and if it is not there, attempt to briefly prove
$( \varphi \rightarrow \forall x \varphi )$  for each dummy variable x.

\item Every dummy variable should be a set variable,
unless there is a justification theorem available.

\item Every dummy variable must be bound
(if the system cannot determine this a justification theorem must be
provided).

\end{enumerate}

\subsection{Summary of Approach to Definitions}

In short, when being rigorous it turns out that
definitions can be subtle, sometimes requiring difficult
metatheorems to establish that they are not creative.

Instead of building such complications into the Metamath language itself,
the basic Metmath language and program simply treat traditional
axioms and definitions as the same kind of \texttt{\$a} statement.
We have then built various tools to enable people to
verify additional conditions as their creators believe is appropriate
for those specific databases, without complicating the Metamath foundations.

\chapter{The Metamath Program}\label{commands}

This chapter provides a reference manual for the
Metamath program.\index{Metamath!commands}

Current instructions for obtaining and installing the Metamath program
can be found at the \url{http://metamath.org} web site.
For Windows, there is a pre-compiled version called
\texttt{metamath.exe}.  For Unix, Linux, and Mac OS X
(which we will refer to collectively as ``Unix''), the Metamath program
can be compiled from its source code with the command
\begin{verbatim}
gcc *.c -o metamath
\end{verbatim}
using the \texttt{gcc} {\sc c} compiler available on those systems.

In the command syntax descriptions below, fields enclosed in square
brackets [\ ] are optional.  File names may be optionally enclosed in
single or double quotes.  This is useful if the file name contains
spaces or
slashes (\texttt{/}), such as in Unix path names, \index{Unix file
names}\index{file names!Unix} that might be confused with Metamath
command qualifiers.\index{Unix file names}\index{file names!Unix}

\section{Invoking Metamath}

Unix, Linux, and Mac OS X
have a command-line interface called the {\em
bash shell}.  (In Mac OS X, select the Terminal application from
Applications/Utilities.)  To invoke Metamath from the bash shell prompt,
assuming that the Metamath program is in the current directory, type
\begin{verbatim}
bash$ ./metamath
\end{verbatim}

To invoke Metamath from a Windows DOS or Command Prompt, assuming that
the Metamath program is in the current directory (or in a directory
included in the Path system environment variable), type
\begin{verbatim}
C:\metamath>metamath
\end{verbatim}

To use command-line arguments at invocation, the command-line arguments
should be a list of Metamath commands, surrounded by quotes if they
contain spaces.  In Windows, the surrounding quotes must be double (not
single) quotes.  For example, to read the database file \texttt{set.mm},
verify all proofs, and exit the program, type (under Unix)
\begin{verbatim}
bash$ ./metamath 'read set.mm' 'verify proof *' exit
\end{verbatim}
Note that in Unix, any directory path with \texttt{/}'s must be
surrounded by quotes so Metamath will not interpret the \texttt{/} as a
command qualifier.  So if \texttt{set.mm} is in the \texttt{/tmp}
directory, use for the above example
\begin{verbatim}
bash$ ./metamath 'read "/tmp/set.mm"' 'verify proof *' exit
\end{verbatim}

For convenience, if the command-line has one argument and no spaces in
the argument, the command is implicitly assumed to be \texttt{read}.  In
this one special case, \texttt{/}'s are not interpreted as command
qualifiers, so you don't need quotes around a Unix file name.  Thus
\begin{verbatim}
bash$ ./metamath /tmp/set.mm
\end{verbatim}
and
\begin{verbatim}
bash$ ./metamath "read '/tmp/set.mm'"
\end{verbatim}
are equivalent.


\section{Controlling Metamath}

The Metamath program was first developed on a {\sc vax/vms} system, and
some aspects of its command line behavior reflect this heritage.
Hopefully you will find it reasonably user-friendly once you get used to
it.

Each command line is a sequence of English-like words separated by
spaces, as in \texttt{show settings}.  Command words are not case
sensitive, and only as many letters are needed as are necessary to
eliminate ambiguity; for example, \texttt{sh se} would work for the
command \texttt{show settings}.  In some cases arguments such as file
names, statement labels, or symbol names are required; these are
case-sensitive (although file names may not be on some operating
systems).

A command line is entered by typing it in then pressing the {\em return}
({\em enter}) key.  To find out what commands are available, type
\texttt{?} at the \texttt{MM>} prompt.  To find out the choices at any
point in a command, press {\em return} and you will be prompted for
them.  The default choice (the one selected if you just press {\em
return}) is shown in brackets (\texttt{<>}).

You may also type \texttt{?} in place of a command word to force
Metamath to tell you what the choices are.  The \texttt{?} method won't
work, though, if a non-keyword argument such as a file name is expected
at that point, because the program will think that \texttt{?} is the
value of the argument.

Some commands have one or more optional qualifiers which modify the
behavior of the command.  Qualifiers are preceded by a slash
(\texttt{/}), such as in \texttt{read set.mm / verify}.  Spaces are
optional around the \texttt{/}.  If you need to use a space or
slash in a command
argument, as in a Unix file name, put single or double quotes around the
command argument.

The \texttt{open log} command will save everything you see on the
screen and is useful to help you recover should something go wrong in a
proof, or if you want to document a bug.

If a command responds with more than a screenful, you will be
prompted to \texttt{<return> to continue, Q to quit, or S to scroll to
end}.  \texttt{Q} or \texttt{q} (not case-sensitive) will complete the
command internally but will suppress further output until the next
\texttt{MM>} prompt.  \texttt{s} will suppress further pausing until the
next \texttt{MM>} prompt.  After the first screen, you are also
presented with the choice of \texttt{b} to go back a screenful.  Note
that \texttt{b} may also be entered at the \texttt{MM>} prompt
immediately after a command to scroll back through the output of that
command.

A command line enclosed in quotes is executed by your operating system.
See Section~\ref{oscmd}.

{\em Warning:} Pressing {\sc ctrl-c} will abort the Metamath program
unconditionally.  This means any unsaved work will be lost.


\subsection{\texttt{exit} Command}\index{\texttt{exit} command}

Syntax:  \texttt{exit} [\texttt{/force}]

This command exits from Metamath.  If there have been changes to the
source with the \texttt{save proof} or \texttt{save new{\char`\_}proof}
commands, you will be given an opportunity to \texttt{write source} to
permanently save the changes.

In Proof Assistant\index{Proof Assistant} mode, the \texttt{exit} command will
return to the \verb/MM>/ prompt. If there were changes to the proof, you will
be given an opportunity to \texttt{save new{\char`\_}proof}.

The \texttt{quit} command is a synonym for \texttt{exit}.

Optional qualifier:
    \texttt{/force} - Do not prompt if changes were not saved.  This qualifier is
        useful in \texttt{submit} command files (Section~\ref{sbmt})
        to ensure predictable behavior.





\subsection{\texttt{open log} Command}\index{\texttt{open log} command}
Syntax:  \texttt{open log} {\em file-name}

This command will open a log file that will store everything you see on
the screen.  It is useful to help recovery from a mistake in a long Proof
Assistant session, or to document bugs.\index{Metamath!bugs}

The log file can be closed with \texttt{close log}.  It will automatically be
closed upon exiting Metamath.



\subsection{\texttt{close log} Command}\index{\texttt{close log} command}
Syntax:  \texttt{close log}

The \texttt{close log} command closes a log file if one is open.  See
also \texttt{open log}.




\subsection{\texttt{submit} Command}\index{\texttt{submit} command}\label{sbmt}
Syntax:  \texttt{submit} {\em filename}

This command causes further command lines to be taken from the specified
file.  Note that any line beginning with an exclamation point (\texttt{!}) is
treated as a comment (i.e.\ ignored).  Also note that the scrolling
of the screen output is continuous, so you may want to open a log file
(see \texttt{open log}) to record the results that fly by on the screen.
After the lines in the file are exhausted, Metamath returns to its
normal user interface mode.

The \texttt{submit} command can be called recursively (i.e. from inside
of a \texttt{submit} command file).


Optional command qualifier:

    \texttt{/silent} - suppresses the screen output but still
        records the output in a log file if one is open.


\subsection{\texttt{erase} Command}\index{\texttt{erase} command}
Syntax:  \texttt{erase}

This command will reset Metamath to its starting state, deleting any
data\-base that was \texttt{read} in.
 If there have been changes to the
source with the \texttt{save proof} or \texttt{save new{\char`\_}proof}
commands, you will be given an opportunity to \texttt{write source} to
permanently save the changes.



\subsection{\texttt{set echo} Command}\index{\texttt{set echo} command}
Syntax:  \texttt{set echo on} or \texttt{set echo off}

The \texttt{set echo on} command will cause command lines to be echoed with any
abbreviations expanded.  While learning the Metamath commands, this
feature will show you the exact command that your abbreviated input
corresponds to.



\subsection{\texttt{set scroll} Command}\index{\texttt{set scroll} command}
Syntax:  \texttt{set scroll prompted} or \texttt{set scroll continuous}

The Metamath command line interface starts off in the \texttt{prompted} mode,
which means that you will be prompted to continue or quit after each
full screen in a long listing.  In \texttt{continuous} mode, long listings will be
scrolled without pausing.

% LaTeX bug? (1) \texttt{\_} puts out different character than
% \texttt{{\char`\_}}
%  = \verb$_$  (2) \texttt{{\char`\_}} puts out garbage in \subsection
%  argument
\subsection{\texttt{set width} Command}\index{\texttt{set
width} command}
Syntax:  \texttt{set width} {\em number}

Metamath assumes the width of your screen is 79 characters (chosen
because the Command Prompt in Windows XP has a wrapping bug at column
80).  If your screen is wider or narrower, this command allows you to
change this default screen width.  A larger width is advantageous for
logging proofs to an output file to be printed on a wide printer.  A
smaller width may be necessary on some terminals; in this case, the
wrapping of the information messages may sometimes seem somewhat
unnatural, however.  In \LaTeX\index{latex@{\LaTeX}!characters per line},
there is normally a maximum of 61 characters per line with typewriter
font.  (The examples in this book were produced with 61 characters per
line.)

\subsection{\texttt{set height} Command}\index{\texttt{set
height} command}
Syntax:  \texttt{set height} {\em number}

Metamath assumes your screen height is 24 lines of characters.  If your
screen is taller or shorter, this command lets you to change the number
of lines at which the display pauses and prompts you to continue.

\subsection{\texttt{beep} Command}\index{\texttt{beep} command}

Syntax:  \texttt{beep}

This command will produce a beep.  By typing it ahead after a
long-running command has started, it will alert you that the command is
finished.  For convenience, \texttt{b} is an abbreviation for
\texttt{beep}.

Note:  If \texttt{b} is typed at the \texttt{MM>} prompt immediately
after the end of a multiple-page display paged with ``\texttt{Press
<return> for more}...'' prompts, then the \texttt{b} will back up to the
previous page rather than perform the \texttt{beep} command.
In that case you must type the unabbreviated \texttt{beep} form
of the command.

\subsection{\texttt{more} Command}\index{\texttt{more} command}

Syntax:  \texttt{more} {\em filename}

This command will display the contents of an {\sc ascii} file on your
screen.  (This command is provided for convenience but is not very
powerful.  See Section~\ref{oscmd} to invoke your operating system's
command to do this, such as the \texttt{more} command in Unix.)

\subsection{Operating System Commands}\index{operating system
command}\label{oscmd}

A line enclosed in single or double quotes will be executed by your
computer's operating system if it has a command line interface.  For
example, on a {\sc vax/vms} system,
\verb/MM> 'dir'/
will print disk directory contents.  Note that this feature will not
work on the Macintosh prior to Mac OS X, which does not have a command
line interface.

For your convenience, the trailing quote is optional.

\subsection{Size Limitations in Metamath}

In general, there are no fixed, predefined limits\index{Metamath!memory
limits} on how many labels, tokens\index{token}, statements, etc.\ that
you may have in a database file.  The Metamath program uses 32-bit
variables (64-bit on 64-bit CPUs) as indices for almost all internal
arrays, which are allocated dynamically as needed.



\section{Reading and Writing Files}

The following commands create new files:  the \texttt{open} commands;
the \texttt{write} commands; the \texttt{/html},
\texttt{/alt{\char`\_}html}, \texttt{/brief{\char`\_}html},
\texttt{/brief{\char`\_}alt{\char`\_}html} qualifiers of \texttt{show
statement}, and \texttt{midi}.  The following commands append to files
previously opened:  the \texttt{/tex} qualifier of \texttt{show proof}
and \texttt{show new{\char`\_}proof}; the \texttt{/tex} and
\texttt{/simple{\char`\_}tex} qualifiers of \texttt{show statement}; the
\texttt{close} commands; and all screen dialog between \texttt{open log}
and \texttt{close log}.

The commands that create new files will not overwrite an existing {\em
filename} but will rename the existing one to {\em
filename}\texttt{{\char`\~}1}.  An existing {\em
filename}\texttt{{\char`\~}1} is renamed {\em
filename}\texttt{{\char`\~}2}, etc.\ up to {\em
filename}\texttt{{\char`\~}9}.  An existing {\em
filename}\texttt{{\char`\~}9} is deleted.  This makes recovery from
mistakes easier but also will clutter up your directory, so occasionally
you may want to clean up (delete) these \texttt{{\char`\~}}$n$ files.


\subsection{\texttt{read} Command}\index{\texttt{read} command}
Syntax:  \texttt{read} {\em file-name} [\texttt{/verify}]

This command will read in a Metamath language source file and any included
files.  Normally it will be the first thing you do when entering Metamath.
Statement syntax is checked, but proof syntax is not checked.
Note that the file name may be enclosed in single or double quotes;
this is useful if the file name contains slashes, as might be the case
under Unix.

If you are getting an ``\texttt{?Expected VERIFY}'' error
when trying to read a Unix file name with slashes, you probably haven't
quoted it.\index{Unix file names}\index{file names!Unix}

If you are prompted for the file name (by pressing {\em return}
 after \texttt{read})
you should {\em not} put quotes around it, even if it is a Unix file name
with slashes.

Optional command qualifier:

    \texttt{/verify} - Verify all proofs as the database is read in.  This
         qualifier will slow down reading in the file.  See \texttt{verify
         proof} for more information on file error-checking.

See also \texttt{erase}.



\subsection{\texttt{write source} Command}\index{\texttt{write source} command}
Syntax:  \texttt{write source} {\em filename}
[\texttt{/rewrap}]
[\texttt{/split}]
% TeX doesn't handle this long line with tt text very well,
% so force a line break here.
[\texttt{/keep\_includes}] {\\}
[\texttt{/no\_versioning}]

This command will write the contents of a Metamath\index{database}
database into a file.\index{source file}

Optional command qualifiers:

\texttt{/rewrap} -
Reformats statements and comments according to the
convention used in the set.mm database.
It unwraps the
lines in the comment before each \texttt{\$a} and \texttt{\$p} statement,
then it
rewraps the line.  You should compare the output to the original
to make sure that the desired effect results; if not, go back to
the original source.  The wrapped line length honors the
\texttt{set width}
parameter currently in effect.  Note:  Text
enclosed in \texttt{<HTML>}...\texttt{</HTML>} tags is not modified by the
\texttt{/rewrap} qualifier.
Proofs themselves are not reformatted;
use \texttt{save proof * / compressed} to do that.
An isolated tilde (\~{}) is always kept on the same line as the following
symbol, so you can find all comment references to a symbol by
searching for \~{} followed by a space and that symbol
(this is useful for finding cross-references).
Incidentally, \texttt{save proof} also honors the \texttt{set width}
parameter currently in effect.

\texttt{/split} - Files included in the source using the expression
\$[ \textit{inclfile} \$] will be
written into separate files instead of being included in a single output
file.  The name of each separately written included file will be the
\textit{inclfile} argument of its inclusion command.

\texttt{/keep\_includes} - If a source file has includes but is written as a
single file by omitting \texttt{/split}, by default the included files will
be deleted (actually just renamed with a \char`\~1 suffix unless
\texttt{/no\_versioning} is specified) to prevent the possibly confusing
source duplication in both the output file and the included file.
The \texttt{/keep\_includes} qualifier will prevent this deletion.

\texttt{/no\_versioning} - Backup files suffixed with \char`\~1 are not created.


\section{Showing Status and Statements}



\subsection{\texttt{show settings} Command}\index{\texttt{show settings} command}
Syntax:  \texttt{show settings}

This command shows the state of various parameters.

\subsection{\texttt{show memory} Command}\index{\texttt{show memory} command}
Syntax:  \texttt{show memory}

This command shows the available memory left.  It is not meaningful
on most modern operating systems,
which have virtual memory.\index{Metamath!memory usage}


\subsection{\texttt{show labels} Command}\index{\texttt{show labels} command}
Syntax:  \texttt{show labels} {\em label-match} [\texttt{/all}]
   [\texttt{/linear}]

This command shows the labels of \texttt{\$a} and \texttt{\$p}
statements that match {\em label-match}.  A \verb$*$ in {label-match}
matches zero or more characters.  For example, \verb$*abc*def$ will match all
labels containing \verb$abc$ and ending with \verb$def$.

Optional command qualifiers:

   \texttt{/all} - Include matches for \texttt{\$e} and \texttt{\$f}
   statement labels.

   \texttt{/linear} - Display only one label per line.  This can be useful for
       building scripts in conjunction with the utilities under the
       \texttt{tools}\index{\texttt{tools} command} command.



\subsection{\texttt{show statement} Command}\index{\texttt{show statement} command}
Syntax:  \texttt{show statement} {\em label-match} [{\em qualifiers} (see below)]

This command provides information about a statement.  Only statements
that have labels (\texttt{\$f}\index{\texttt{\$f} statement},
\texttt{\$e}\index{\texttt{\$e} statement},
\texttt{\$a}\index{\texttt{\$a} statement}, and
\texttt{\$p}\index{\texttt{\$p} statement}) may be specified.
If {\em label-match}
contains wildcard (\verb$*$) characters, all matching statements will be
displayed in the order they occur in the database.

Optional qualifiers (only one qualifier at a time is allowed):

    \texttt{/comment} - This qualifier includes the comment that immediately
       precedes the statement.

    \texttt{/full} - Show complete information about each statement,
       and show all
       statements matching {\em label} (including \texttt{\$e}
       and \texttt{\$f} statements).

    \texttt{/tex} - This qualifier will write the statement information to the
       \LaTeX\ file previously opened with \texttt{open tex}.  See
       Section~\ref{texout}.

    \texttt{/simple{\char`\_}tex} - The same as \texttt{/tex}, except that
       \LaTeX\ macros are not used for formatting equations, allowing easier
       manual edits of the output for slide presentations, etc.

    \texttt{/html}\index{html generation@{\sc html} generation},
       \texttt{/alt{\char`\_}html}, \texttt{/brief{\char`\_}html},
       \texttt{/brief{\char`\_}alt{\char`\_}html} -
       These qualifiers invoke a special mode of
       \texttt{show statement} that
       creates a web page for the statement.  They may not be used with
       any other qualifier.  See Section~\ref{htmlout} or
       \texttt{help html} in the program.


\subsection{\texttt{search} Command}\index{\texttt{search} command}
Syntax:  search {\em label-match}
\texttt{"}{\em symbol-match}{\tt}" [\texttt{/all}] [\texttt{/comments}]
[\texttt{/join}]

This command searches all \texttt{\$a} and \texttt{\$p} statements
matching {\em label-match} for occurrences of {\em symbol-match}.  A
\verb@*@ in {\em label-match} matches any label character.  A \verb@$*@
in {\em symbol-match} matches any sequence of symbols.  The symbols in
{\em symbol-match} must be separated by white space.  The quotes
surrounding {\em symbol-match} may be single or double quotes.  For
example, \texttt{search b}\verb@* "-> $* ch"@ will list all statements
whose labels begin with \texttt{b} and contain the symbols \verb@->@ and
\texttt{ch} surrounding any symbol sequence (including no symbol
sequence).  The wildcards \texttt{?} and \texttt{\$?} are also available
to match individual characters in labels and symbols respectively; see
\texttt{help search} in the Metamath program for details on their usage.

Optional command qualifiers:

    \texttt{/all} - Also search \texttt{\$e} and \texttt{\$f} statements.

    \texttt{/comments} - Search the comment that immediately precedes each
        label-matched statement for {\em symbol-match}.  In this case
        {\em symbol-match} is an arbitrary, non-case-sensitive character
        string.  Quotes around {\em symbol-match} are optional if there
        is no ambiguity.

    \texttt{/join} - In the case of a \texttt{\$a} or \texttt{\$p} statement,
	prepend its \texttt{\$e}
	hypotheses for searching. The
	\texttt{/join} qualifier has no effect in \texttt{/comments} mode.

\section{Displaying and Verifying Proofs}


\subsection{\texttt{show proof} Command}\index{\texttt{show proof} command}
Syntax:  \texttt{show proof} {\em label-match} [{\em qualifiers} (see below)]

This command displays the proof of the specified
\texttt{\$p}\index{\texttt{\$p} statement} statement in various formats.
The {\em label-match} may contain wildcard (\verb@$*@) characters to match
multiple statements.  Without any qualifiers, only the logical steps
will be shown (i.e.\ syntax construction steps will be omitted), in an
indented format.

Most of the time, you will use
    \texttt{show proof} {\em label}
to see just the proof steps corresponding to logical inferences.

Optional command qualifiers:

    \texttt{/essential} - The proof tree is trimmed of all
        \texttt{\$f}\index{\texttt{\$f} statement} hypotheses before
        being displayed.  (This is the default, and it is redundant to
        specify it.)

    \texttt{/all} - the proof tree is not trimmed of all \texttt{\$f} hypotheses before
        being displayed.  \texttt{/essential} and \texttt{/all} are mutually exclusive.

    \texttt{/from{\char`\_}step} {\em step} - The display starts at the specified
        step.  If
        this qualifier is omitted, the display starts at the first step.

    \texttt{/to{\char`\_}step} {\em step} - The display ends at the specified
        step.  If this
        qualifier is omitted, the display ends at the last step.

    \texttt{/tree{\char`\_}depth} {\em number} - Only
         steps at less than the specified proof
        tree depth are displayed.  Sometimes useful for obtaining an overview of
        the proof.

    \texttt{/reverse} - The steps are displayed in reverse order.

    \texttt{/renumber} - When used with \texttt{/essential}, the steps are renumbered
        to correspond only to the essential steps.

    \texttt{/tex} - The proof is converted to \LaTeX\ and\index{latex@{\LaTeX}}
        stored in the file opened
        with \texttt{open tex}.  See Section~\ref{texout} or
        \texttt{help tex} in the program.

    \texttt{/lemmon} - The proof is displayed in a non-indented format known
        as Lemmon style, with explicit previous step number references.
        If this qualifier is omitted, steps are indented in a tree format.

    \texttt{/start{\char`\_}column} {\em number} - Overrides the default column
        (16)
        at which the formula display starts in a Lemmon-style display.  May be
        used only in conjunction with \texttt{/lemmon}.

    \texttt{/normal} - The proof is displayed in normal format suitable for
        inclusion in a Metamath source file.  May not be used with any other
        qualifier.

    \texttt{/compressed} - The proof is displayed in compressed format
        suitable for inclusion in a Metamath source file.  May not be used with
        any other qualifier.

    \texttt{/statement{\char`\_}summary} - Summarizes all statements (like a
        brief \texttt{show statement})
        used by the proof.  It may not be used with any other qualifier
        except \texttt{/essential}.

    \texttt{/detailed{\char`\_}step} {\em step} - Shows the details of what is
        happening at
        a specific proof step.  May not be used with any other qualifier.
        The {\em step} is the step number shown when displaying a
        proof without the \texttt{/renumber} qualifier.


\subsection{\texttt{show usage} Command}\index{\texttt{show usage} command}
Syntax:  \texttt{show usage} {\em label-match} [\texttt{/recursive}]

This command lists the statements whose proofs make direct reference to
the statement specified.

Optional command qualifier:

    \texttt{/recursive} - Also include statements whose proofs ultimately
        depend on the statement specified.



\subsection{\texttt{show trace\_back} Command}\index{\texttt{show
       trace{\char`\_}back} command}
Syntax:  \texttt{show trace{\char`\_}back} {\em label-match} [\texttt{/essential}] [\texttt{/axioms}]
    [\texttt{/tree}] [\texttt{/depth} {\em number}]

This command lists all statements that the proof of the \texttt{\$p}
statement(s) specified by {\em label-match} depends on.

Optional command qualifiers:

    \texttt{/essential} - Restrict the trace-back to \texttt{\$e}
        \index{\texttt{\$e} statement} hypotheses of proof trees.

    \texttt{/axioms} - List only the axioms that the proof ultimately depends on.

    \texttt{/tree} - Display the trace-back in an indented tree format.

    \texttt{/depth} {\em number} - Restrict the \texttt{/tree} trace-back to the
        specified indentation depth.

    \texttt{/count{\char`\_}steps} - Count the number of steps the proof has
       all the way back to axioms.  If \texttt{/essential} is specified,
       expansions of variable-type hypotheses (syntax constructions) are not counted.

\subsection{\texttt{verify proof} Command}\index{\texttt{verify proof} command}
Syntax:  \texttt{verify proof} {\em label-match} [\texttt{/syntax{\char`\_}only}]

This command verifies the proofs of the specified statements.  {\em
label-match} may contain wild card characters (\texttt{*}) to verify
more than one proof; for example \verb/*abc*def/ will match all labels
containing \texttt{abc} and ending with \texttt{def}.
The command \texttt{verify proof *} will verify all proofs in the database.

Optional command qualifier:

    \texttt{/syntax{\char`\_}only} - This qualifier will perform a check of syntax
        and RPN
        stack violations only.  It will not verify that the proof is
        correct.  This qualifier is useful for quickly determining which
        proofs are incomplete (i.e.\ are under development and have \texttt{?}'s
        in them).

{\em Note:} \texttt{read}, followed by \texttt{verify proof *}, will ensure
 the database is free
from errors in the Metamath language but will not check the markup notation
in comments.
You can also check the markup notation by running \texttt{verify markup *}
as discussed in Section~\ref{verifymarkup}; see also the discussion
on generating {\sc HTML} in Section~\ref{htmlout}.

\subsection{\texttt{verify markup} Command}\index{\texttt{verify markup} command}\label{verifymarkup}
Syntax:  \texttt{verify markup} {\em label-match}
[\texttt{/date{\char`\_}skip}]
[\texttt{/top{\char`\_}date{\char`\_}skip}] {\\}
[\texttt{/file{\char`\_}skip}]
[\texttt{/verbose}]

This command checks comment markup and other informal conventions we have
adopted.  It error-checks the latexdef, htmldef, and althtmldef statements
in the \texttt{\$t} statement of a Metamath source file.\index{error checking}
It error-checks any \texttt{`}...\texttt{`},
\texttt{\char`\~}~\textit{label},
and bibliographic markups in statement descriptions.
It checks that
\texttt{\$p} and \texttt{\$a} statements
have the same content when their labels start with
``ax'' and ``ax-'' respectively but are otherwise identical, for example
ax4 and ax-4.
It also verifies the date consistency of ``(Contributed by...),''
``(Revised by...),'' and ``(Proof shortened by...)'' tags in the comment
above each \texttt{\$a} and \texttt{\$p} statement.

Optional command qualifiers:

    \texttt{/date{\char`\_}skip} - This qualifier will
        skip date consistency checking,
        which is usually not required for databases other than
	\texttt{set.mm}.

    \texttt{/top{\char`\_}date{\char`\_}skip} - This qualifier will check date consistency except
        that the version date at the top of the database file will not
        be checked.  Only one of
        \texttt{/date{\char`\_}skip} and
        \texttt{/top{\char`\_}date{\char`\_}skip} may be
        specified.

    \texttt{/file{\char`\_}skip} - This qualifier will skip checks that require
        external files to be present, such as checking GIF existence and
        bibliographic links to mmset.html or equivalent.  It is useful
        for doing a quick check from a directory without these files.

    \texttt{/verbose} - Provides more information.  Currently it provides a list
        of axXXX vs. ax-XXX matches.

\subsection{\texttt{save proof} Command}\index{\texttt{save proof} command}
Syntax:  \texttt{save proof} {\em label-match} [\texttt{/normal}]
   [\texttt{/compressed}]

The \texttt{save proof} command will reformat a proof in one of two formats and
replace the existing proof in the source buffer\index{source
buffer}.  It is useful for
converting between proof formats.  Note that a proof will not be
permanently saved until a \texttt{write source} command is issued.

Optional command qualifiers:

    \texttt{/normal} - The proof is saved in the normal format (i.e., as a
        sequence
        of labels, which is the defined format of the basic Metamath
        language).\index{basic language}  This is the default format that
        is used if a qualifier
        is omitted.

    \texttt{/compressed} - The proof is saved in the compressed format which
        reduces storage requirements for a database.
        See Appendix~\ref{compressed}.




\section{Creating Proofs}\label{pfcommands}\index{Proof Assistant}

Before using the Proof Assistant, you must add a \texttt{\$p} to your
source file (using a text editor) containing the statement you want to
prove.  Its proof should consist of a single \texttt{?}, meaning
``unknown step.''  Example:
\begin{verbatim}
equid $p x = x $= ? $.
\end{verbatim}

To enter the Proof assistant, type \texttt{prove} {\em label}, e.g.
\texttt{prove equid}.  Metamath will respond with the \texttt{MM-PA>}
prompt.

Proofs are created working backwards from the statement being proved,
primarily using a series of \texttt{assign} commands.  A proof is
complete when all steps are assigned to statements and all steps are
unified and completely known.  During the creation of a proof, Metamath
will allow only operations that are legal based on what is known up to
that point.  For example, it will not allow an \texttt{assign} of a
statement that cannot be unified with the unknown proof step being
assigned.

{\em Important:}
The Proof Assistant is
{\em not} a tool to help you discover proofs.  It is just a tool to help
you add them to the database.  For a tutorial read
Section~\ref{frstprf}.
To practice using the Proof Assistant, you may
want to \texttt{prove} an existing theorem, then delete all steps with
\texttt{delete all}, then re-create it with the Proof Assistant while
looking at its proof display (before deletion).
You might want to figure out your first few proofs completely
and write them down by hand, before using the Proof Assistant, though
not everyone finds that effective.

{\em Important:}
The \texttt{undo} command if very helpful when entering a proof, because
it allows you to undo a previously-entered step.
In addition, we suggest that you
keep track of your work with a log file (\texttt{open
log}) and save it frequently (\texttt{save new{\char`\_}proof},
\texttt{write source}).
You can use \texttt{delete} to reverse an \texttt{assign}.
You can also do \texttt{delete floating{\char`\_}hypotheses}, then
\texttt{initialize all}, then \texttt{unify all /interactive} to
reinitialize bad unifications made accidentally or by bad
\texttt{assign}s.  You cannot reverse a \texttt{delete} except by
a relevant \texttt{undo} or using
\texttt{exit /force} then reentering the Proof Assistant to recover from
the last \texttt{save new{\char`\_}proof}.

The following commands are available in the Proof Assistant (at the
\texttt{MM-PA>} prompt) to help you create your proof.  See the
individual commands for more detail.

\begin{itemize}
\item[]
    \texttt{show new{\char`\_}proof} [\texttt{/all},...] - Displays the
        proof in progress.  You will use this command a lot; see \texttt{help
        show new{\char`\_}proof} to become familiar with its qualifiers.  The
        qualifiers \texttt{/unknown} and \texttt{/not{\char`\_}unified} are
        useful for seeing the work remaining to be done.  The combination
        \texttt{/all/unknown} is useful for identifying dummy variables that must be
        assigned, or attempts to use illegal syntax, when \texttt{improve all}
        is unable to complete the syntax constructions.  Unknown variables are
        shown as \texttt{\$1}, \texttt{\$2},...
\item[]
    \texttt{assign} {\em step} {\em label} - Assigns an unknown {\em step}
        number with the statement
        specified by {\em label}.
\item[]
    \texttt{let variable} {\em variable}
        \texttt{= "}{\em symbol sequence}\texttt{"}
          - Forces a symbol
        sequence to replace an unknown variable (such as \texttt{\$1}) in a proof.
        It is useful
        for helping difficult unifications, and it is necessary when you have
        dummy variables that eventually must be assigned a name.
\item[]
    \texttt{let step} {\em step} \texttt{= "}{\em symbol sequence}\texttt{"} -
          Forces a symbol sequence
        to replace the contents of a proof step, provided it can be
        unified with the existing step contents.  (I rarely use this.)
\item[]
    \texttt{unify step} {\em step} (or \texttt{unify all}) - Unifies
        the source and target of
        a step.  If you specify a specific step, you will be prompted
        to select among the unifications that are possible.  If you
        specify \texttt{all}, all steps with unique unifications, but only
        those steps, will be
        unified.  \texttt{unify all /interactive} goes through all non-unified
        steps.
\item[]
    \texttt{initialize} {\em step} (or \texttt{all}) - De-unifies the target and source of
        a step (or all steps), as well as the hypotheses of the source,
        and makes all variables in the source unknown.  Useful to recover from
        an \texttt{assign} or \texttt{let} mistake that
        resulted in incorrect unifications.
\item[]
    \texttt{delete} {\em step} (or \texttt{all} or \texttt{floating{\char`\_}hypotheses}) -
        Deletes the specified
        step(s).  \texttt{delete floating{\char`\_}hypotheses}, then \texttt{initialize all}, then
        \texttt{unify all /interactive} is useful for recovering from mistakes
        where incorrect unifications assigned wrong math symbol strings to
        variables.
\item[]
    \texttt{improve} {\em step} (or \texttt{all}) -
      Automatically creates a proof for steps (with no unknown variables)
      whose proof requires no statements with \texttt{\$e} hypotheses.  Useful
      for filling in proofs of \texttt{\$f} hypotheses.  The \texttt{/depth}
      qualifier will also try statements whose \texttt{\$e} hypotheses contain
      no new variables.  {\em Warning:} Save your work (with \texttt{save
      new{\char`\_}proof} then \texttt{write source}) before using
      \texttt{/depth = 2} or greater, since the search time grows
      exponentially and may never terminate in a reasonable time, and you
      cannot interrupt the search.  I have found that it is rare for
      \texttt{/depth = 3} or greater to be useful.
 \item[]
    \texttt{save new{\char`\_}proof} - Saves the proof in progress in the program's
        internal database buffer.  To save it permanently into the database file,
        use \texttt{write source} after
        \texttt{save new{\char`\_}proof}.  To revert to the last
        \texttt{save new{\char`\_}proof},
        \texttt{exit /force} from the Proof Assistant then re-enter the Proof
        Assistant.
 \item[]
    \texttt{match step} {\em step} (or \texttt{match all}) - Shows what
        statements are
        possibilities for the \texttt{assign} statement. (This command
        is not very
        useful in its present form and hopefully will be improved
        eventually.  In the meantime, use the \texttt{search} statement for
        candidates matching specific math token combinations.)
 \item[]
 \texttt{minimize{\char`\_}with}\index{\texttt{minimize{\char`\_}with} command}
% 3/10/07 Note: line-breaking the above results in duplicate index entries
     - After a proof is complete, this command will attempt
        to match other database theorems to the proof to see if the proof
        size can be reduced as a result.  See \texttt{help
        minimize{\char`\_}with} in the
        Metamath program for its usage.
 \item[]
 \texttt{undo}\index{\texttt{undo} command}
    - Undo the effect of a proof-changing command (all but the
      \texttt{show} and \texttt{save} commands above).
 \item[]
 \texttt{redo}\index{\texttt{redo} command}
    - Reverse the previous \texttt{undo}.
\end{itemize}

The following commands set parameters that may be relevant to your proof.
Consult the individual \texttt{help set}... commands.
\begin{itemize}
   \item[] \texttt{set unification{\char`\_}timeout}
 \item[]
    \texttt{set search{\char`\_}limit}
  \item[]
    \texttt{set empty{\char`\_}substitution} - note that default is \texttt{off}
\end{itemize}

Type \texttt{exit} to exit the \texttt{MM-PA>}
 prompt and get back to the \texttt{MM>} prompt.
Another \texttt{exit} will then get you out of Metamath.



\subsection{\texttt{prove} Command}\index{\texttt{prove} command}
Syntax:  \texttt{prove} {\em label}

This command will enter the Proof Assistant\index{Proof Assistant}, which will
allow you to create or edit the proof of the specified statement.
The command-line prompt will change from \texttt{MM>} to \texttt{MM-PA>}.

Note:  In the present version (0.177) of
Metamath\index{Metamath!limitations of version 0.177}, the Proof
Assistant does not verify that \texttt{\$d}\index{\texttt{\$d}
statement} restrictions are met as a proof is being built.  After you
have completed a proof, you should type \texttt{save new{\char`\_}proof}
followed by \texttt{verify proof} {\em label} (where {\em label} is the
statement you are proving with the \texttt{prove} command) to verify the
\texttt{\$d} restrictions.

See also: \texttt{exit}

\subsection{\texttt{set unification\_timeout} Command}\index{\texttt{set
unification{\char`\_}timeout} command}
Syntax:  \texttt{set unification{\char`\_}timeout} {\em number}

(This command is available outside the Proof Assistant but affects the
Proof Assistant\index{Proof Assistant} only.)

Sometimes the Proof Assistant will inform you that a unification
time-out occurred.  This may happen when you try to \texttt{unify}
formulas with many temporary variables\index{temporary variable}
(\texttt{\$1}, \texttt{\$2}, etc.), since the time to compute all possible
unifications may grow exponentially with the number of variables.  If
you want Metamath to try harder (and you're willing to wait longer) you
may increase this parameter.  \texttt{show settings} will show you the
current value.

\subsection{\texttt{set empty\_substitution} Command}\index{\texttt{set
empty{\char`\_}substitution} command}
% These long names can't break well in narrow mode, and even "sloppy"
% is not enough. Work around this by not demanding justification.
\begin{flushleft}
Syntax:  \texttt{set empty{\char`\_}substitution on} or \texttt{set
empty{\char`\_}substitution off}
\end{flushleft}

(This command is available outside the Proof Assistant but affects the
Proof Assistant\index{Proof Assistant} only.)

The Metamath language allows variables to be
substituted\index{substitution!variable}\index{variable substitution}
with empty symbol sequences\index{empty substitution}.  However, in many
formal systems\index{formal system} this will never happen in a valid
proof.  Allowing for this possibility increases the likelihood of
ambiguous unifications\index{ambiguous
unification}\index{unification!ambiguous} during proof creation.
The default is that
empty substitutions are not allowed; for formal systems requiring them,
you must \texttt{set empty{\char`\_}substitution on}.
(An example where this must be \texttt{on}
would be a system that implements a Deduction Rule and in
which deductions from empty assumption lists would be permissible.  The
MIU-system\index{MIU-system} described in Appendix~\ref{MIU} is another
example.)
Note that empty substitutions are
always permissible in proof verification (VERIFY PROOF...) outside the
Proof Assistant.  (See the MIU system in the Metamath book for an example
of a system needing empty substitutions; another example would be a
system that implements a Deduction Rule and in which deductions from
empty assumption lists would be permissible.)

It is better to leave this \texttt{off} when working with \texttt{set.mm}.
Note that this command does not affect the way proofs are verified with
the \texttt{verify proof} command.  Outside of the Proof Assistant,
substitution of empty sequences for math symbols is always allowed.

\subsection{\texttt{set search\_limit} Command}\index{\texttt{set
search{\char`\_}limit} command} Syntax:  \texttt{set search{\char`\_}limit} {\em
number}

(This command is available outside the Proof Assistant but affects the
Proof Assistant\index{Proof Assistant} only.)

This command sets a parameter that determines when the \texttt{improve} command
in Proof Assistant mode gives up.  If you want \texttt{improve} to search harder,
you may increase it.  The \texttt{show settings} command tells you its current
value.


\subsection{\texttt{show new\_proof} Command}\index{\texttt{show
new{\char`\_}proof} command}
Syntax:  \texttt{show new{\char`\_}proof} [{\em
qualifiers} (see below)]

This command (available only in Proof Assistant mode) displays the proof
in progress.  It is identical to the \texttt{show proof} command, except that
there is no statement argument (since it is the statement being proved) and
the following qualifiers are not available:

    \texttt{/statement{\char`\_}summary}

    \texttt{/detailed{\char`\_}step}

Also, the following additional qualifiers are available:

    \texttt{/unknown} - Shows only steps that have no statement assigned.

    \texttt{/not{\char`\_}unified} - Shows only steps that have not been unified.

Note that \texttt{/essential}, \texttt{/depth}, \texttt{/unknown}, and
\texttt{/not{\char`\_}unified} may be
used in any combination; each of them effectively filters out additional
steps from the proof display.

See also:  \texttt{show proof}






\subsection{\texttt{assign} Command}\index{\texttt{assign} command}
Syntax:   \texttt{assign} {\em step} {\em label} [\texttt{/no{\char`\_}unify}]

   and:   \texttt{assign first} {\em label}

   and:   \texttt{assign last} {\em label}


This command, available in the Proof Assistant only, assigns an unknown
step (one with \texttt{?} in the \texttt{show new{\char`\_}proof}
listing) with the statement specified by {\em label}.  The assignment
will not be allowed if the statement cannot be unified with the step.

If \texttt{last} is specified instead of {\em step} number, the last
step that is shown by \texttt{show new{\char`\_}proof /unknown} will be
used.  This can be useful for building a proof with a command file (see
\texttt{help submit}).  It also makes building proofs faster when you know
the assignment for the last step.

If \texttt{first} is specified instead of {\em step} number, the first
step that is shown by \texttt{show new{\char`\_}proof /unknown} will be
used.

If {\em step} is zero or negative, the -{\em step}th from last unknown
step, as shown by \texttt{show new{\char`\_}proof /unknown}, will be
used.  \texttt{assign -1} {\em label} will assign the penultimate
unknown step, \texttt{assign -2} {\em label} the antepenultimate, and
\texttt{assign 0} {\em label} is the same as \texttt{assign last} {\em
label}.

Optional command qualifier:

    \texttt{/no{\char`\_}unify} - do not prompt user to select a unification if there is
        more than one possibility.  This is useful for noninteractive
        command files.  Later, the user can \texttt{unify all /interactive}.
        (The assignment will still be automatically unified if there is only
        one possibility and will be refused if unification is not possible.)



\subsection{\texttt{match} Command}\index{\texttt{match} command}
Syntax:  \texttt{match step} {\em step} [\texttt{/max{\char`\_}essential{\char`\_}hyp}
{\em number}]

    and:  \texttt{match all} [\texttt{/essential}]
          [\texttt{/max{\char`\_}essential{\char`\_}hyp} {\em number}]

This command, available in the Proof Assistant only, shows what
statements can be unified with the specified step(s).  {\em Note:} In
its current form, this command is not very useful because of the large
number of matches it reports.
It may be enhanced in the future.  In the meantime, the \texttt{search}
command can often provide finer control over locating theorems of interest.

Optional command qualifiers:

    \texttt{/max{\char`\_}essential{\char`\_}hyp} {\em number} - filters out
        of the list any statements
        with more than the specified number of
        \texttt{\$e}\index{\texttt{\$e} statement} hypotheses.

    \texttt{/essential{\char`\_}only} - in the \texttt{match all} statement, only
        the steps that
        would be listed in the \texttt{show new{\char`\_}proof /essential} display are
        matched.



\subsection{\texttt{let} Command}\index{\texttt{let} command}
Syntax: \texttt{let variable} {\em variable} = \verb/"/{\em symbol-sequence}\verb/"/

  and: \texttt{let step} {\em step} = \verb/"/{\em symbol-sequence}\verb/"/

These commands, available in the Proof Assistant\index{Proof Assistant}
only, assign a temporary variable\index{temporary variable} or unknown
step with a specific symbol sequence.  They are useful in the middle of
creating a proof, when you know what should be in the proof step but the
unification algorithm doesn't yet have enough information to completely
specify the temporary variables.  A ``temporary variable'' is one that
has the form \texttt{\$}{\em nn} in the proof display, such as
\texttt{\$1}, \texttt{\$2}, etc.  The {\em symbol-sequence} may contain
other unknown variables if desired.  Examples:

    \verb/let variable $32 = "A = B"/

    \verb/let variable $32 = "A = $35"/

    \verb/let step 10 = '|- x = x'/

    \verb/let step -2 = "|- ( $7 = ph )"/

Any symbol sequence will be accepted for the \texttt{let variable}
command.  Only those symbol sequences that can be unified with the step
will be accepted for \texttt{let step}.

The \texttt{let} commands ``zap'' the proof with information that can
only be verified when the proof is built up further.  If you make an
error, the command sequence \texttt{delete
floating{\char`\_}hypotheses}, \texttt{initialize all}, and
\texttt{unify all /interactive} will undo a bad \texttt{let} assignment.

If {\em step} is zero or negative, the -{\em step}th from last unknown
step, as shown by \texttt{show new{\char`\_}proof /unknown}, will be
used.  The command \texttt{let step 0} = \verb/"/{\em
symbol-sequence}\verb/"/ will use the last unknown step, \texttt{let
step -1} = \verb/"/{\em symbol-sequence}\verb/"/ the penultimate, etc.
If {\em step} is positive, \texttt{let step} may be used to assign known
(in the sense of having previously been assigned a label with
\texttt{assign}) as well as unknown steps.

Either single or double quotes can surround the {\em symbol-sequence} as
long as they are different from any quotes inside a {\em
symbol-sequence}.  If {\em symbol-sequence} contains both kinds of
quotes, see the instructions at the end of \texttt{help let} in the
Metamath program.


\subsection{\texttt{unify} Command}\index{\texttt{unify} command}
Syntax:  \texttt{unify step} {\em step}

      and:   \texttt{unify all} [\texttt{/interactive}]

These commands, available in the Proof Assistant only, unify the source
and target of the specified step(s). If you specify a specific step, you
will be prompted to select among the unifications that are possible.  If
you specify \texttt{all}, only those steps with unique unifications will be
unified.

Optional command qualifier for \texttt{unify all}:

    \texttt{/interactive} - You will be prompted to select among the
        unifications
        that are possible for any steps that do not have unique
        unifications.  (Otherwise \texttt{unify all} will bypass these.)

See also \texttt{set unification{\char`\_}timeout}.  The default is
100000, but increasing it to 1000000 can help difficult cases.  Manually
assigning some or all of the unknown variables with the \texttt{let
variable} command also helps difficult cases.



\subsection{\texttt{initialize} Command}\index{\texttt{initialize} command}
Syntax:  \texttt{initialize step} {\em step}

    and: \texttt{initialize all}

These commands, available in the Proof Assistant\index{Proof Assistant}
only, ``de-unify'' the target and source of a step (or all steps), as
well as the hypotheses of the source, and makes all variables in the
source and the source's hypotheses unknown.  This command is useful to
help recover from incorrect unifications that resulted from an incorrect
\texttt{assign}, \texttt{let}, or unification choice.  Part or all of
the command sequence \texttt{delete floating{\char`\_}hypotheses},
\texttt{initialize all}, and \texttt{unify all /interactive} will recover
from incorrect unifications.

See also:  \texttt{unify} and \texttt{delete}



\subsection{\texttt{delete} Command}\index{\texttt{delete} command}
Syntax:  \texttt{delete step} {\em step}

   and:      \texttt{delete all} -- {\em Warning: dangerous!}

   and:      \texttt{delete floating{\char`\_}hypotheses}

These commands are available in the Proof Assistant only.  The
\texttt{delete step} command deletes the proof tree section that
branches off of the specified step and makes the step become unknown.
\texttt{delete all} is equivalent to \texttt{delete step} {\em step}
where {\em step} is the last step in the proof (i.e.\ the beginning of
the proof tree).

In most cases the \texttt{undo} command is the best way to undo
a previous step.
An alternative is to salvage your last \texttt{save
new{\char`\_}proof} by exiting and reentering the Proof Assistant.
For this to work, keep a log file open to record your work
and to do \texttt{save new{\char`\_}proof} frequently, especially before
\texttt{delete}.

\texttt{delete floating{\char`\_}hypotheses} will delete all sections of
the proof that branch off of \texttt{\$f}\index{\texttt{\$f} statement}
statements.  It is sometimes useful to do this before an
\texttt{initialize} command to recover from an error.  Note that once a
proof step with a \texttt{\$f} hypothesis as the target is completely
known, the \texttt{improve} command can usually fill in the proof for
that step.  Unlike the deletion of logical steps, \texttt{delete
floating{\char`\_}hypotheses} is a relatively safe command that is
usually easy to recover from.



\subsection{\texttt{improve} Command}\index{\texttt{improve} command}
\label{improve}
Syntax:  \texttt{improve} {\em step} [\texttt{/depth} {\em number}]
                                               [\texttt{/no{\char`\_}distinct}]

   and:   \texttt{improve first} [\texttt{/depth} {\em number}]
                                              [\texttt{/no{\char`\_}distinct}]

   and:   \texttt{improve last} [\texttt{/depth} {\em number}]
                                              [\texttt{/no{\char`\_}distinct}]

   and:   \texttt{improve all} [\texttt{/depth} {\em number}]
                                              [\texttt{/no{\char`\_}distinct}]

These commands, available in the Proof Assistant\index{Proof Assistant}
only, try to find proofs automatically for unknown steps whose symbol
sequences are completely known.  They are primarily useful for filling in
proofs of \texttt{\$f}\index{\texttt{\$f} statement} hypotheses.  The
search will be restricted to statements having no
\texttt{\$e}\index{\texttt{\$e} statement} hypotheses.

\begin{sloppypar} % narrow
Note:  If memory is limited, \texttt{improve all} on a large proof may
overflow memory.  If you use \texttt{set unification{\char`\_}timeout 1}
before \texttt{improve all}, there will usually be sufficient
improvement to easily recover and completely \texttt{improve} the proof
later on a larger computer.  Warning:  Once memory has overflowed, there
is no recovery.  If in doubt, save the intermediate proof (\texttt{save
new{\char`\_}proof} then \texttt{write source}) before \texttt{improve
all}.
\end{sloppypar}

If \texttt{last} is specified instead of {\em step} number, the last
step that is shown by \texttt{show new{\char`\_}proof /unknown} will be
used.

If \texttt{first} is specified instead of {\em step} number, the first
step that is shown by \texttt{show new{\char`\_}proof /unknown} will be
used.

If {\em step} is zero or negative, the -{\em step}th from last unknown
step, as shown by \texttt{show new{\char`\_}proof /unknown}, will be
used.  \texttt{improve -1} will use the penultimate
unknown step, \texttt{improve -2} {\em label} the antepenultimate, and
\texttt{improve 0} is the same as \texttt{improve last}.

Optional command qualifier:

    \texttt{/depth} {\em number} - This qualifier will cause the search
        to include
        statements with \texttt{\$e} hypotheses (but no new variables in
        the \texttt{\$e}
        hypotheses), provided that the backtracking has not exceeded the
        specified depth. {\em Warning:}  Try \texttt{/depth 1},
        then \texttt{2}, then \texttt{3}, etc.
        in sequence because of possible exponential blowups.  Save your
        work before trying \texttt{/depth} greater than \texttt{1}!

    \texttt{/no{\char`\_}distinct} - Skip trial statements that have
        \texttt{\$d}\index{\texttt{\$d} statement} requirements.
        This qualifier will prevent assignments that might violate \texttt{\$d}
        requirements but it also could miss possible legal assignments.

See also: \texttt{set search{\char`\_}limit}

\subsection{\texttt{save new\_proof} Command}\index{\texttt{save
new{\char`\_}proof} command}
Syntax:  \texttt{save new{\char`\_}proof} {\em label} [\texttt{/normal}]
   [\texttt{/compressed}]

The \texttt{save new{\char`\_}proof} command is available in the Proof
Assistant only.  It saves the proof in progress in the source
buffer\index{source buffer}.  \texttt{save new{\char`\_}proof} may be
used to save a completed proof, or it may be used to save a proof in
progress in order to work on it later.  If an incomplete proof is saved,
any user assignments with \texttt{let step} or \texttt{let variable}
will be lost, as will any ambiguous unifications\index{ambiguous
unification}\index{unification!ambiguous} that were resolved manually.
To help make recovery easier, it can be helpful to \texttt{improve all}
before \texttt{save new{\char`\_}proof} so that the incomplete proof
will have as much information as possible.

Note that the proof will not be permanently saved until a \texttt{write
source} command is issued.

Optional command qualifiers:

    \texttt{/normal} - The proof is saved in the normal format (i.e., as a
        sequence of labels, which is the defined format of the basic Metamath
        language).\index{basic language}  This is the default format that
        is used if a qualifier is omitted.

    \texttt{/compressed} - The proof is saved in the compressed format, which
        reduces storage requirements for a database.
        (See Appendix~\ref{compressed}.)


\section{Creating \LaTeX\ Output}\label{texout}\index{latex@{\LaTeX}}

You can generate \LaTeX\ output given the
information in a database.
The database must already include the necessary typesetting information
(see section \ref{tcomment} for how to provide this information).

The \texttt{show statement} and \texttt{show proof} commands each have a
special \texttt{/tex} command qualifier that produces \LaTeX\ output.
(The \texttt{show statement} command also has the
\texttt{/simple{\char`\_}tex} qualifier for output that is easier to
edit by hand.)  Before you can use them, you must open a \LaTeX\ file to
which to send their output.  A typical complete session will use this
sequence of Metamath commands:

\begin{verbatim}
read set.mm
open tex example.tex
show statement a1i /tex
show proof a1i /all/lemmon/renumber/tex
show statement uneq2 /tex
show proof uneq2 /all/lemmon/renumber/tex
close tex
\end{verbatim}

See Section~\ref{mathcomments} for information on comment markup and
Appendix~\ref{ASCII} for information on how math symbol translation is
specified.

To format and print the \LaTeX\ source, you will need the \LaTeX\
program, which is standard on most Linux installations and available for
Windows.  On Linux, in order to create a {\sc pdf} file, you will
typically type at the shell prompt
\begin{verbatim}
$ pdflatex example.tex
\end{verbatim}

\subsection{\texttt{open tex} Command}\index{\texttt{open tex} command}
Syntax:  \texttt{open tex} {\em file-name} [\texttt{/no{\char`\_}header}]

This command opens a file for writing \LaTeX\
source\index{latex@{\LaTeX}} and writes a \LaTeX\ header to the file.
\LaTeX\ source can be written with the \texttt{show proof}, \texttt{show
new{\char`\_}proof}, and \texttt{show statement} commands using the
\texttt{/tex} qualifier.

The mapping to \LaTeX\ symbols is defined in a special comment
containing a \texttt{\$t} token, described in Appendix~\ref{ASCII}.

There is an optional command qualifier:

    \texttt{/no{\char`\_}header} - This qualifier prevents a standard
        \LaTeX\ header and trailer
        from being included with the output \LaTeX\ code.


\subsection{\texttt{close tex} Command}\index{\texttt{close tex} command}
Syntax:  \texttt{close tex}

This command writes a trailer to any \LaTeX\ file\index{latex@{\LaTeX}}
that was opened with \texttt{open tex} (unless
\texttt{/no{\char`\_}header} was used with \texttt{open tex}) and closes
the \LaTeX\ file.


\section{Creating {\sc HTML} Output}\label{htmlout}

You can generate {\sc html} web pages given the
information in a database.
The database must already include the necessary typesetting information
(see section \ref{tcomment} for how to provide this information).
The ability to produce {\sc html} web pages was added in Metamath version
0.07.30.

To create an {\sc html} output file(s) for \texttt{\$a} or \texttt{\$p}
statement(s), use
\begin{quote}
    \texttt{show statement} {\em label-match} \texttt{/html}
\end{quote}
The output file will be named {\em label-match}\texttt{.html}
for each match.  When {\em
label-match} has wildcard (\texttt{*}) characters, all statements with
matching labels will have {\sc html} files produced for them.  Also,
when {\em label-match} has a wildcard (\texttt{*}) character, two additional
files, \texttt{mmdefinitions.html} and \texttt{mmascii.html} will be
produced.  To produce {\em only} these two additional files, you can use
\texttt{?*}, which will not match any statement label, in place of {\em
label-match}.

There are three other qualifiers for \texttt{show statement} that also
generate {\sc HTML} code.  These are \texttt{/alt{\char`\_}html},
\texttt{/brief{\char`\_}html}, and
\texttt{/brief{\char`\_}alt{\char`\_}html}, and are described in the
next section.

The command
\begin{quote}
    \texttt{show statement} {\em label-match} \texttt{/alt{\char`\_}html}
\end{quote}
does the same as \texttt{show statement} {\em label-match} \texttt{/html},
except that the {\sc html} code for the symbols is taken from
\texttt{althtmldef} statements instead of \texttt{htmldef} statements in
the \texttt{\$t} comment.

The command
\begin{verbatim}
show statement * /brief_html
\end{verbatim}
invokes a special mode that just produces definition and theorem lists
accompanied by their symbol strings, in a format suitable for copying and
pasting into another web page (such as the tutorial pages on the
Metamath web site).

Finally, the command
\begin{verbatim}
show statement * /brief_alt_html
\end{verbatim}
does the same as \texttt{show statement * / brief{\char`\_}html}
for the alternate {\sc html}
symbol representation.

A statement's comment can include a special notation that provides a
certain amount of control over the {\sc HTML} version of the comment.  See
Section~\ref{mathcomments} (p.~\pageref{mathcomments}) for the comment
markup features.

The \texttt{write theorem{\char`\_}list} and \texttt{write bibliography}
commands, which are described below, provide as a side effect complete
error checking for all of the features described in this
section.\index{error checking}

\subsection{\texttt{write theorem\_list}
Command}\index{\texttt{write theorem{\char`\_}list} command}
Syntax:  \texttt{write theorem{\char`\_}list}
[\texttt{/theorems{\char`\_}per{\char`\_}page} {\em number}]

This command writes a list of all of the \texttt{\$a} and \texttt{\$p}
statements in the database into a web page file
 called \texttt{mmtheorems.html}.
When additional files are needed, they are called
\texttt{mmtheorems2.html}, \texttt{mmtheorems3.html}, etc.

Optional command qualifier:

    \texttt{/theorems{\char`\_}per{\char`\_}page} {\em number} -
 This qualifier specifies the number of statements to
        write per web page.  The default is 100.

{\em Note:} In version 0.177\index{Metamath!limitations of version
0.177} of Metamath, the ``Nearby theorems'' links on the individual
web pages presuppose 100 theorems per page when linking to the theorem
list pages.  Therefore the \texttt{/theorems{\char`\_}per{\char`\_}page}
qualifier, if it specifies a number other than 100, will cause the
individual web pages to be out of sync and should not be used to
generate the main theorem list for the web site.  This may be
fixed in a future version.


\subsection{\texttt{write bibliography}\label{wrbib}
Command}\index{\texttt{write bibliography} command}
Syntax:  \texttt{write bibliography} {\em filename}

This command reads an existing {\sc html} bibliographic cross-reference
file, normally called \texttt{mmbiblio.html}, and updates it per the
bibliographic links in the database comments.  The file is updated
between the {\sc html} comment lines \texttt{<!--
{\char`\#}START{\char`\#} -->} and \texttt{<!-- {\char`\#}END{\char`\#}
-->}.  The original input file is renamed to {\em
filename}\texttt{{\char`\~}1}.

A bibliographic reference is indicated with the reference name
in brackets, such as  \texttt{Theorem 3.1 of
[Monk] p.\ 22}.
See Section~\ref{htmlout} (p.~\pageref{htmlout}) for
syntax details.


\subsection{\texttt{write recent\_additions}
Command}\index{\texttt{write recent{\char`\_}additions} command}
Syntax:  \texttt{write recent{\char`\_}additions} {\em filename}
[\texttt{/limit} {\em number}]

This command reads an existing ``Recent Additions'' {\sc html} file,
normally called \texttt{mmrecent.html}, and updates it with the
descriptions of the most recently added theorems to the database.
 The file is updated between
the {\sc html} comment lines \texttt{<!-- {\char`\#}START{\char`\#} -->}
and \texttt{<!-- {\char`\#}END{\char`\#} -->}.  The original input file
is renamed to {\em filename}\texttt{{\char`\~}1}.

Optional command qualifier:

    \texttt{/limit} {\em number} -
 This qualifier specifies the number of most recent theorems to
   write to the output file.  The default is 100.


\section{Text File Utilities}

\subsection{\texttt{tools} Command}\index{\texttt{tools} command}
Syntax:  \texttt{tools}

This command invokes an easy-to-use, general purpose utility for
manipulating the contents of {\sc ascii} text files.  Upon typing
\texttt{tools}, the command-line prompt will change to \texttt{TOOLS>}
until you type \texttt{exit}.  The \texttt{tools} commands can be used
to perform simple, global edits on an input/output file,
such as making a character string substitution on each line, adding a
string to each line, and so on.  A typical use of this utility is
to build a \texttt{submit} input file to perform a common operation on a
list of statements obtained from \texttt{show label} or \texttt{show
usage}.

The actions of most of the \texttt{tools} commands can also be
performed with equivalent (and more powerful) Unix shell commands, and
some users may find those more efficient.  But for Windows users or
users not comfortable with Unix, \texttt{tools} provides an
easy-to-learn alternative that is adequate for most of the
script-building tasks needed to use the Metamath program effectively.

\subsection{\texttt{help} Command (in \texttt{tools})}
Syntax:  \texttt{help}

The \texttt{help} command lists the commands available in the
\texttt{tools} utility, along with a brief description.  Each command,
in turn, has its own help, such as \texttt{help add}.  As with
Metamath's \texttt{MM>} prompt, a complete command can be entered at
once, or just the command word can be typed, causing the program to
prompt for each argument.

\vskip 1ex
\noindent Line-by-line editing commands:

  \texttt{add} - Add a specified string to each line in a file.

  \texttt{clean} - Trim spaces and tabs on each line in a file; convert
         characters.

  \texttt{delete} - Delete a section of each line in a file.

  \texttt{insert} - Insert a string at a specified column in each line of
        a file.

  \texttt{substitute} - Make a simple substitution on each line of the file.

  \texttt{tag} - Like \texttt{add}, but restricted to a range of lines.

  \texttt{swap} - Swap the two halves of each line in a file.

\vskip 1ex
\noindent Other file-processing commands:

  \texttt{break} - Break up (tokenize) a file into a list of tokens (one per
        line).

  \texttt{build} - Build a file with multiple tokens per line from a list.

  \texttt{count} - Count the occurrences in a file of a specified string.

  \texttt{number} - Create a list of numbers.

  \texttt{parallel} - Put two files in parallel.

  \texttt{reverse} - Reverse the order of the lines in a file.

  \texttt{right} - Right-justify lines in a file (useful before sorting
         numbers).

%  \texttt{tag} - Tag edit updates in a program for revision control.

  \texttt{sort} - Sort the lines in a file with key starting at
         specified string.

  \texttt{match} - Extract lines containing (or not) a specified string.

  \texttt{unduplicate} - Eliminate duplicate occurrences of lines in a file.

  \texttt{duplicate} - Extract first occurrence of any line occurring
         more than

   \ \ \    once in a file, discarding lines occurring exactly once.

  \texttt{unique} - Extract lines occurring exactly once in a file.

  \texttt{type} (10 lines) - Display the first few lines in a file.
                  Similar to Unix \texttt{head}.

  \texttt{copy} - Similar to Unix \texttt{cat} but safe (same input
         and output file allowed).

  \texttt{submit} - Run a script containing \texttt{tools} commands.

\vskip 1ex

\noindent Note:
  \texttt{unduplicate}, \texttt{duplicate}, and \texttt{unique} also
 sort the lines as a side effect.


\subsection{Using \texttt{tools} to Build Metamath \texttt{submit}
Scripts}

The \texttt{break} command is typically used to break up a series of
statement labels, such as the output of Metamath's \texttt{show usage},
into one label per line.  The other \texttt{tools} commands can then be
used to add strings before and after each statement label to specify
commands to be performed on the statement.  The \texttt{parallel}
command is useful when a statement label must be mentioned more than
once on a line.

Very often a \texttt{submit} script for Metamath will require multiple
command lines for each statement being processed.  For example, you may
want to enter the Proof Assistant, \texttt{minimize{\char`\_}with} your
latest theorem, \texttt{save} the new proof, and \texttt{exit} the Proof
Assistant.  To accomplish this, you can build a file with these four
commands for each statement on a single line, separating each command
with a designated character such as \texttt{@}.  Then at the end you can
\texttt{substitute} each \texttt{@} with \texttt{{\char`\\}n} to break
up the lines into individual command lines (see \texttt{help
substitute}).


\subsection{Example of a \texttt{tools} Session}

To give you a quick feel for the \texttt{tools} utility, we show a
simple session where we create a file \texttt{n.txt} with 3 lines, add
strings before and after each line, and display the lines on the screen.
You can experiment with the various commands to gain experience with the
\texttt{tools} utility.

\begin{verbatim}
MM> tools
Entering the Text Tools utilities.
Type HELP for help, EXIT to exit.
TOOLS> number
Output file <n.tmp>? n.txt
First number <1>?
Last number <10>? 3
Increment <1>?
TOOLS> add
Input/output file? n.txt
String to add to beginning of each line <>? This is line
String to add to end of each line <>? .
The file n.txt has 3 lines; 3 were changed.
First change is on line 1:
This is line 1.
TOOLS> type n.txt
This is line 1.
This is line 2.
This is line 3.
TOOLS> exit
Exiting the Text Tools.
Type EXIT again to exit Metamath.
MM>
\end{verbatim}



\appendix
\chapter{Sample Representations}
\label{ASCII}

This Appendix provides a sample of {\sc ASCII} representations,
their corresponding traditional mathematical symbols,
and a discussion of their meanings
in the \texttt{set.mm} database.
These are provided in order of appearance.
This is only a partial list, and new definitions are routinely added.
A complete list is available at \url{http://metamath.org}.

These {\sc ASCII} representations, along
with information on how to display them,
are defined in the \texttt{set.mm} database file inside
a special comment called a \texttt{\$t} {\em
comment}\index{\texttt{\$t} comment} or {\em typesetting
comment.}\index{typesetting comment}
A typesetting comment
is indicated by the appearance of the
two-character string \texttt{\$t} at the beginning of the comment.
For more information,
see Section~\ref{tcomment}, p.~\pageref{tcomment}.

In the following table the ``{\sc ASCII}'' column shows the {\sc ASCII}
representation,
``Symbol'' shows the mathematical symbolic display
that corresponds to that {\sc ASCII} representation, ``Labels'' shows
the key label(s) that define the representation, and
``Description'' provides a description about the symbol.
As usual, ``iff'' is short for ``if and only if.''\index{iff}
In most cases the ``{\sc ASCII}'' column only shows
the key token, but it will sometimes show a sequence of tokens
if that is necessary for clarity.

{\setlength{\extrarowsep}{4pt} % Keep rows from being too close together
\begin{longtabu}   { @{} c c l X }
\textbf{ASCII} & \textbf{Symbol} & \textbf{Labels} & \textbf{Description} \\
\endhead
\texttt{|-} & $\vdash$ & &
  ``It is provable that...'' \\
\texttt{ph} & $\varphi$ & \texttt{wph} &
  The wff (boolean) variable phi,
  conventionally the first wff variable. \\
\texttt{ps} & $\psi$ & \texttt{wps} &
  The wff (boolean) variable psi,
  conventionally the second wff variable. \\
\texttt{ch} & $\chi$ & \texttt{wch} &
  The wff (boolean) variable chi,
  conventionally the third wff variable. \\
\texttt{-.} & $\lnot$ & \texttt{wn} &
  Logical not. E.g., if $\varphi$ is true, then $\lnot \varphi$ is false. \\
\texttt{->} & $\rightarrow$ & \texttt{wi} &
  Implies, also known as material implication.
  In classical logic the expression $\varphi \rightarrow \psi$ is true
  if either $\varphi$ is false or $\psi$ is true (or both), that is,
  $\varphi \rightarrow \psi$ has the same meaning as
  $\lnot \varphi \lor \psi$ (as proven in theorem \texttt{imor}). \\
\texttt{<->} & $\leftrightarrow$ &
  \hyperref[df-bi]{\texttt{df-bi}} &
  Biconditional (aka is-equals for boolean values).
  $\varphi \leftrightarrow \psi$ is true iff
  $\varphi$ and $\psi$ have the same value. \\
\texttt{\char`\\/} & $\lor$ &
  \makecell[tl]{{\hyperref[df-or]{\texttt{df-or}}}, \\
	         \hyperref[df-3or]{\texttt{df-3or}}} &
  Disjunction (logical ``or''). $\varphi \lor \psi$ is true iff
  $\varphi$, $\psi$, or both are true. \\
\texttt{/\char`\\} & $\land$ &
  \makecell[tl]{{\hyperref[df-an]{\texttt{df-an}}}, \\
                 \hyperref[df-3an]{\texttt{df-3an}}} &
  Conjunction (logical ``and''). $\varphi \land \psi$ is true iff
  both $\varphi$ and $\psi$ are true. \\
\texttt{A.} & $\forall$ &
  \texttt{wal} &
  For all; the wff $\forall x \varphi$ is true iff
  $\varphi$ is true for all values of $x$. \\
\texttt{E.} & $\exists$ &
  \hyperref[df-ex]{\texttt{df-ex}} &
  There exists; the wff
  $\exists x \varphi$ is true iff
  there is at least one $x$ where $\varphi$ is true. \\
\texttt{[ y / x ]} & $[ y / x ]$ &
  \hyperref[df-sb]{\texttt{df-sb}} &
  The wff $[ y / x ] \varphi$ produces
  the result when $y$ is properly substituted for $x$ in $\varphi$
  ($y$ replaces $x$).
  % This is elsb4
  % ( [ x / y ] z e. y <-> z e. x )
  For example,
  $[ x / y ] z \in y$ is the same as $z \in x$. \\
\texttt{E!} & $\exists !$ &
  \hyperref[df-eu]{\texttt{df-eu}} &
  There exists exactly one;
  $\exists ! x \varphi$ is true iff
  there is at least one $x$ where $\varphi$ is true. \\
\texttt{\{ y | phi \}}  & $ \{ y | \varphi \}$ &
  \hyperref[df-clab]{\texttt{df-clab}} &
  The class of all sets where $\varphi$ is true. \\
\texttt{=} & $ = $ &
  \hyperref[df-cleq]{\texttt{df-cleq}} &
  Class equality; $A = B$ iff $A$ equals $B$. \\
\texttt{e.} & $ \in $ &
  \hyperref[df-clel]{\texttt{df-clel}} &
  Class membership; $A \in B$ if $A$ is a member of $B$. \\
\texttt{{\char`\_}V} & {\rm V} &
  \hyperref[df-v]{\texttt{df-v}} &
  Class of all sets (not itself a set). \\
\texttt{C\_} & $ \subseteq $ &
  \hyperref[df-ss]{\texttt{df-ss}} &
  Subclass (subset); $A \subseteq B$ is true iff
  $A$ is a subclass of $B$. \\
\texttt{u.} & $ \cup $ &
  \hyperref[df-un]{\texttt{df-un}} &
  $A \cup B$ is the union of classes $A$ and $B$. \\
\texttt{i^i} & $ \cap $ &
  \hyperref[df-in]{\texttt{df-in}} &
  $A \cap B$ is the intersection of classes $A$ and $B$. \\
\texttt{\char`\\} & $ \setminus $ &
  \hyperref[df-dif]{\texttt{df-dif}} &
  $A \setminus B$ (set difference)
  is the class of all sets in $A$ except for those in $B$. \\
\texttt{(/)} & $ \varnothing $ &
  \hyperref[df-nul]{\texttt{df-nul}} &
  $ \varnothing $ is the empty set (aka null set). \\
\texttt{\char`\~P} & $ \cal P $ &
  \hyperref[df-pw]{\texttt{df-pw}} &
  Power class. \\
\texttt{<.\ A , B >.} & $\langle A , B \rangle$ &
  \hyperref[df-op]{\texttt{df-op}} &
  The ordered pair $\langle A , B \rangle$. \\
\texttt{( F ` A )} & $ ( F ` A ) $ &
  \hyperref[df-fv]{\texttt{df-fv}} &
  The value of function $F$ when applied to $A$. \\
\texttt{\_i} & $ i $ &
  \texttt{df-i} &
  The square root of negative one. \\
\texttt{x.} & $ \cdot $ &
  \texttt{df-mul} &
  Complex number multiplication; $2~\cdot~3~=~6$. \\
\texttt{CC} & $ \mathbb{C} $ &
  \texttt{df-c} &
  The set of complex numbers. \\
\texttt{RR} & $ \mathbb{R} $ &
  \texttt{df-r} &
  The set of real numbers. \\
\end{longtabu}
} % end of extrarowsep

\chapter{Compressed Proofs}
\label{compressed}\index{compressed proof}\index{proof!compressed}

The proofs in the \texttt{set.mm} set theory database are stored in compressed
format for efficiency.  Normally you needn't concern yourself with the
compressed format, since you can display it with the usual proof display tools
in the Metamath program (\texttt{show proof}\ldots) or convert it to the normal
RPN proof format described in Section~\ref{proof} (with \texttt{save proof}
{\em label} \texttt{/normal}).  However for sake of completeness we describe the
format here and show how it maps to the normal RPN proof format.

A compressed proof, located between \texttt{\$=} and \texttt{\$.}\ keywords, consists
of a left parenthesis, a sequence of statement labels, a right parenthesis,
and a sequence of upper-case letters \texttt{A} through \texttt{Z} (with optional
white space between them).  White space must surround the parentheses
and the labels.  The left parenthesis tells Metamath that a
compressed proof follows.  (A normal RPN proof consists of just a sequence of
labels, and a parenthesis is not a legal character in a label.)

The sequence of upper-case letters corresponds to a sequence of integers
with the following mapping.  Each integer corresponds to a proof step as
described later.
\begin{center}
  \texttt{A} = 1 \\
  \texttt{B} = 2 \\
   \ldots \\
  \texttt{T} = 20 \\
  \texttt{UA} = 21 \\
  \texttt{UB} = 22 \\
   \ldots \\
  \texttt{UT} = 40 \\
  \texttt{VA} = 41 \\
  \texttt{VB} = 42 \\
   \ldots \\
  \texttt{YT} = 120 \\
  \texttt{UUA} = 121 \\
   \ldots \\
  \texttt{YYT} = 620 \\
  \texttt{UUUA} = 621 \\
   etc.
\end{center}

In other words, \texttt{A} through \texttt{T} represent the
least-significant digit in base 20, and \texttt{U} through \texttt{Y}
represent zero or more most-significant digits in base 5, where the
digits start counting at 1 instead of the usual 0. With this scheme, we
don't need white space between these ``numbers.''

(In the design of the compressed proof format, only upper-case letters,
as opposed to say all non-whitespace printable {\sc ascii} characters other than
%\texttt{\$}, was chosen to make the compressed proof a little less
%displeasing to the eye, at the expense of a typical 20\% compression
\texttt{\$}, were chosen so as not to collide with most text editor
searches, at the expense of a typical 20\% compression
loss.  The base 5/base 20 grouping, as opposed to say base 6/base 19,
was chosen by experimentally determining the grouping that resulted in
best typical compression.)

The letter \texttt{Z} identifies (tags) a proof step that is identical to one
that occurs later on in the proof; it helps shorten the proof by not requiring
that identical proof steps be proved over and over again (which happens often
when building wff's).  The \texttt{Z} is placed immediately after the
least-significant digit (letters \texttt{A} through \texttt{T}) that ends the integer
corresponding to the step to later be referenced.

The integers that the upper-case letters correspond to are mapped to labels as
follows.  If the statement being proved has $m$ mandatory hypotheses, integers
1 through $m$ correspond to the labels of these hypotheses in the order shown
by the \texttt{show statement ... / full} command, i.e., the RPN order\index{RPN
order} of the mandatory
hypotheses.  Integers $m+1$ through $m+n$ correspond to the labels enclosed in
the parentheses of the compressed proof, in the order that they appear, where
$n$ is the number of those labels.  Integers $m+n+1$ on up don't directly
correspond to statement labels but point to proof steps identified with the
letter \texttt{Z}, so that these proof steps can be referenced later in the
proof.  Integer $m+n+1$ corresponds to the first step tagged with a \texttt{Z},
$m+n+2$ to the second step tagged with a \texttt{Z}, etc.  When the compressed
proof is converted to a normal proof, the entire subproof of a step tagged
with \texttt{Z} replaces the reference to that step.

For efficiency, Metamath works with compressed proofs directly, without
converting them internally to normal proofs.  In addition to the usual
error-checking, an error message is given if (1) a label in the label list in
parentheses does not refer to a previous \texttt{\$p} or \texttt{\$a} statement or a
non-mandatory hypothesis of the statement being proved and (2) a proof step
tagged with \texttt{Z} is referenced before the step tagged with the \texttt{Z}.

Just as in a normal proof under development (Section~\ref{unknown}), any step
or subproof that is not yet known may be represented with a single \texttt{?}.
White space does not have to appear between the \texttt{?}\ and the upper-case
letters (or other \texttt{?}'s) representing the remainder of the proof.

% April 1, 2004 Appendix C has been added back in with corrections.
%
% May 20, 2003 Appendix C was removed for now because there was a problem found
% by Bob Solovay
%
% Also, removed earlier \ref{formalspec} 's (3 cases above)
%
% Bob Solovay wrote on 30 Nov 2002:
%%%%%%%%%%%%% (start of email comment )
%      3. My next set of comments concern appendix C. I read this before I
% read Chapter 4. So I first noted that the system as presented in the
% Appendix lacked a certain formal property that I thought desirable. I
% then came up with a revised formal system that had this property. Upon
% reading Chapter 4, I noticed that the revised system was closer to the
% treatment in Chapter 4 than the system in Appendix C.
%
%         First a very minor correction:
%
%         On page 142 line 2: The condition that V(e) != V(f) should only be
% required of e, f in T such that e != f.
%
%         Here is a natural property [transitivity] that one would like
% the formal system to have:
%
%         Let Gamma be a set of statements. Suppose that the statement Phi
% is provable from Gamma and that the statement Psi is provable from Gamma
% \cup {Phi}. Then Psi is provable from Gamma.
%
%         I shall present an example to show that this property does not
% hold for the formal systems of Appendix C:
%
%         I write the example in metamath style:
%
% $c A B C D E $.
% $v x y
%
% ${
% tx $f A x $.
% ty $f B y $.
% ax1 $a C x y $.
% $}
%
% ${
% tx $f A x $.
% ty $f B y $.
% ax2-h1 $e C x y $.
% ax2 $a D y $.
% $}
%
% ${
% ty $f B y $.
% ax3-h1 $e D y $.
% ax3 $a E y $.
% $}
%
% $(These three axioms are Gamma $)
%
% ${
% tx $f A x $.
% ty $f B y $.
% Phi $p D y $=
% tx ty tx ty ax1 ax2 $.
% $}
%
% ${
% ty $f B y $.
% Psi $p E y $=
% ty ty Phi ax3 $.
% $}
%
%
% I omit the formal proofs of the following claims. [I will be glad to
% supply them upon request.]
%
% 1) Psi is not provable from Gamma;
%
% 2) Psi is provable from Gamma + Phi.
%
% Here "provable" refers to the formalism of Appendix C.
%
% The trouble of course is that Psi is lacking the variable declaration
%
% $f Ax $.
%
% In the Metamath system there is no trouble proving Psi. I attach a
% metamath file that shows this and which has been checked by the
% metamath program.
%
% I next want to indicate how I think the treatment in Appendix C should
% be revised so as to conform more closely to the metamath system of the
% main text. The revised system *does* have the transitivity property.
%
% We want to give revised definitions of "statement" and
% "provable". [cf. sections C.2.4. and C.2.5] Our new definitions will
% use the definitions given in Appendix C. So we take the following
% tack. We refer to the original notions as o-statement and o-provable. And
% we refer to the notions we are defining as n-statement and n-provable.
%
%         A n-statement is an o-statement in which the only variables
% that appear in the T component are mandatory.
%
%         To any o-statement we can associate its reduct which is a
% n-statement by dropping all the elements of T or D which contain
% non-mandatory variables.
%
%         An n-statement gamma is n-provable if there is an o-statement
% gamma' which has gamma as its reduct andf such that gamma' is
% o-provable.
%
%         It seems to me [though I am not completely sure on this point]
% that n-provability corresponds to metamath provability as discussed
% say in Chapter 4.
%
%         Attached to this letter is the metamath proof of Phi and Psi
% from Gamma discussed above.
%
%         I am still brooding over the question of whether metamath
% correctly formalizes set-theory. No doubt I will have some questions
% re this after my thoughts become clearer.
%%%%%%%%%%%%%%%% (end of email comment)

%%%%%%%%%%%%%%%% (start of 2nd email comment from Bob Solovay 1-Apr-04)
%
%         I hope that Appendix C is the one that gives a "formal" treatment
% of Metamath. At any rate, thats the appendix I want to comment on.
%
%         I'm going to suggest two changes in the definition.
%
%         First change (in the definition of statement): Require that the
% sets D, T, and E be finite.
%
%         Probably things are fine as you give them. But in the applications
% to the main metamath system they will always be finite, and its useful in
% thinking about things [at least for me] to stick to the finite case.
%
%         Second change:
%
%         First let me give an approximate description. Remove the dummy
% variables from the statement. Instead, include them in the proof.
%
%         More formally: Require that T consists of type declarations only
% for mandatory variables. Require that all the pairs in D consist of
% mandatory variables.
%
%         At the start of a proof we are allowed to declare a finite number
% of dummy variables [provided that none of them appear in any of the
% statements in E \cup {A}. We have to supply type declarations for all the
% dummy variables. We are allowed to add new $d statements referring to
% either the mandatory or dummy variables. But we require that no new $d
% statement references only mandatory variables.
%
%         I find this way of doing things more conceptual than the treatment
% in Appendix C. But the change [which I will use implicitly in later
% letters about doing Peano] is mainly aesthetic. I definitely claim that my
% results on doing Peano all apply to Metamath as it is presented in your
% book.
%
%         --Bob
%
%%%%%%%%%%%%%%%% (end of 2nd email comment)

%%
%% When uncommenting the below, also uncomment references above to {formalspec}
%%
\chapter{Metamath's Formal System}\label{formalspec}\index{Metamath!as a formal
system}

\section{Introduction}

\begin{quote}
  {\em Perfection is when there is no longer anything more to take away.}
    \flushright\sc Antoine de
     Saint-Exupery\footnote{\cite[p.~3-25]{Campbell}.}\\
\end{quote}\index{de Saint-Exupery, Antoine}

This appendix describes the theory behind the Metamath language in an abstract
way intended for mathematicians.  Specifically, we construct two
set-theo\-ret\-i\-cal objects:  a ``formal system'' (roughly, a set of syntax
rules, axioms, and logical rules) and its ``universe'' (roughly, the set of
theorems derivable in the formal system).  The Metamath computer language
provides us with a way to describe specific formal systems and, with the aid of
a proof provided by the user, to verify that given theorems
belong to their universes.

To understand this appendix, you need a basic knowledge of informal set theory.
It should be sufficient to understand, for example, Ch.\ 1 of Munkres' {\em
Topology} \cite{Munkres}\index{Munkres, James R.} or the
introductory set theory chapter
in many textbooks that introduce abstract mathematics. (Note that there are
minor notational differences among authors; e.g.\ Munkres uses $\subset$ instead
of our $\subseteq$ for ``subset.''  We use ``included in'' to mean ``a subset
of,'' and ``belongs to'' or ``is contained in'' to mean ``is an element of.'')
What we call a ``formal'' description here, unlike earlier, is actually an
informal description in the ordinary language of mathematicians.  However we
provide sufficient detail so that a mathematician could easily formalize it,
even in the language of Metamath itself if desired.  To understand the logic
examples at the end of this appendix, familiarity with an introductory book on
mathematical logic would be helpful.

\section{The Formal Description}

\subsection[Preliminaries]{Preliminaries\protect\footnotemark}%
\footnotetext{This section is taken mostly verbatim
from Tarski \cite[p.~63]{Tarski1965}\index{Tarski, Alfred}.}

By $\omega$ we denote the set of all natural numbers (non-negative integers).
Each natural number $n$ is identified with the set of all smaller numbers: $n =
\{ m | m < n \}$.  The formula $m < n$ is thus equivalent to the condition: $m
\in n$ and $m,n \in \omega$. In particular, 0 is the number zero and at the
same time the empty set $\varnothing$, $1=\{0\}$, $2=\{0,1\}$, etc. ${}^B A$
denotes the set of all functions on $B$ to $A$ (i.e.\ with domain $B$ and range
included in $A$).  The members of ${}^\omega A$ are what are called {\em simple
infinite sequences},\index{simple infinite sequence}
with all {\em terms}\index{term} in $A$.  In case $n \in \omega$, the
members of ${}^n A$ are referred to as {\em finite $n$-termed
sequences},\index{finite $n$-termed
sequence} again
with terms in $A$.  The consecutive terms (function values) of a finite or
infinite sequence $f$ are denoted by $f_0, f_1, \ldots ,f_n,\ldots$.  Every
finite sequence $f \in \bigcup _{n \in \omega} {}^n A$ uniquely determines the
number $n$ such that $f \in {}^n A$; $n$ is called the {\em
length}\index{length of a sequence ({$"|\ "|$})} of $f$ and
is denoted by $|f|$.  $\langle a \rangle$ is the sequence $f$ with $|f|=1$ and
$f_0=a$; $\langle a,b \rangle$ is the sequence $f$ with $|f|=2$, $f_0=a$,
$f_1=b$; etc.  Given two finite sequences $f$ and $g$, we denote by $f\frown g$
their {\em concatenation},\index{concatenation} i.e., the
finite sequence $h$ determined by the
conditions:
\begin{eqnarray*}
& |h| = |f|+|g|;&  \\
& h_n = f_n & \mbox{\ for\ } n < |f|;  \\
& h_{|f|+n} = g_n & \mbox{\ for\ } n < |g|.
\end{eqnarray*}

\subsection{Constants, Variables, and Expressions}

A formal system has a set of {\em symbols}\index{symbol!in
a formal system} denoted
by $\mbox{\em SM}$.  A
precise set-theo\-ret\-i\-cal definition of this set is unimportant; a symbol
could be considered a primitive or atomic element if we wish.  We assume this
set is divided into two disjoint subsets:  a set $\mbox{\em CN}$ of {\em
constants}\index{constant!in a formal system} and a set $\mbox{\em VR}$ of
{\em variables}.\index{variable!in a formal system}  $\mbox{\em CN}$ and
$\mbox{\em VR}$ are each assumed to consist of countably many symbols which
may be arranged in finite or simple infinite sequences $c_0, c_1, \ldots$ and
$v_0, v_1, \ldots$ respectively, without repeating terms.  We will represent
arbitrary symbols by metavariables $\alpha$, $\beta$, etc.

{\footnotesize\begin{quotation}
{\em Comment.} The variables $v_0, v_1, \ldots$ of our formal system
correspond to what are usually considered ``metavariables'' in
descriptions of specific formal systems in the literature.  Typically,
when describing a specific formal system a book will postulate a set of
primitive objects called variables, then proceed to describe their
properties using metavariables that range over them, never mentioning
again the actual variables themselves.  Our formal system does not
mention these primitive variable objects at all but deals directly with
metavariables, as its primitive objects, from the start.  This is a
subtle but key distinction you should keep in mind, and it makes our
definition of ``formal system'' somewhat different from that typically
found in the literature.  (So, the $\alpha$, $\beta$, etc.\ above are
actually ``metametavariables'' when used to represent $v_0, v_1,
\ldots$.)
\end{quotation}}

Finite sequences all terms of which are symbols are called {\em
expressions}.\index{expression!in a formal system}  $\mbox{\em EX}$ is
the set of all expressions; thus
\begin{displaymath}
\mbox{\em EX} = \bigcup _{n \in \omega} {}^n \mbox{\em SM}.
\end{displaymath}

A {\em constant-prefixed expression}\index{constant-prefixed expression}
is an expression of non-zero length
whose first term is a constant.  We denote the set of all constant-prefixed
expressions by $\mbox{\em EX}_C = \{ e \in \mbox{\em EX} | ( |e| > 0 \wedge
e_0 \in \mbox{\em CN} ) \}$.

A {\em constant-variable pair}\index{constant-variable pair}
is an expression of length 2 whose first term
is a constant and whose second term is a variable.  We denote the set of all
constant-variable pairs by $\mbox{\em EX}_2 = \{ e \in \mbox{\em EX}_C | ( |e|
= 2 \wedge e_1 \in \mbox{\em VR} ) \}$.


{\footnotesize\begin{quotation}
{\em Relationship to Metamath.} In general, the set $\mbox{\em SM}$
corresponds to the set of declared math symbols in a Metamath database, the
set $\mbox{\em CN}$ to those declared with \texttt{\$c} statements, and the set
$\mbox{\em VR}$ to those declared with \texttt{\$v} statements.  Of course a
Metamath database can only have a finite number of math symbols, whereas
formal systems in general can have an infinite number, although the number of
Metamath math symbols available is in principle unlimited.

The set $\mbox{\em EX}_C$ corresponds to the set of permissible expressions
for \texttt{\$e}, \texttt{\$a}, and \texttt{\$p} statements.  The set $\mbox{\em EX}_2$
corresponds to the set of permissible expressions for \texttt{\$f} statements.
\end{quotation}}

We denote by ${\cal V}(e)$ the set of all variables in an expression $e \in
\mbox{\em EX}$, i.e.\ the set of all $\alpha \in \mbox{\em VR}$ such that
$\alpha = e_n$ for some $n < |e|$.  We also denote (with abuse of notation) by
${\cal V}(E)$ the set of all variables in a collection of expressions $E
\subseteq \mbox{\em EX}$, i.e.\ $\bigcup _{e \in E} {\cal V}(e)$.


\subsection{Substitution}

Given a function $F$ from $\mbox{\em VR}$ to
$\mbox{\em EX}$, we
denote by $\sigma_{F}$ or just $\sigma$ the function from $\mbox{\em EX}$ to
$\mbox{\em EX}$ defined recursively for nonempty sequences by
\begin{eqnarray*}
& \sigma(<\alpha>) = F(\alpha) & \mbox{for\ } \alpha \in \mbox{\em VR}; \\
& \sigma(<\alpha>) = <\alpha> & \mbox{for\ } \alpha \not\in \mbox{\em VR}; \\
& \sigma(g \frown h) = \sigma(g) \frown
    \sigma(h) & \mbox{for\ } g,h \in \mbox{\em EX}.
\end{eqnarray*}
We also define $\sigma(\varnothing)=\varnothing$.  We call $\sigma$ a {\em
simultaneous substitution}\index{substitution!variable}\index{variable
substitution} (or just {\em substitution}) with {\em substitution
map}\index{substitution map} $F$.

We also denote (with abuse of notation) by $\sigma(E)$ a substitution on a
collection of expressions $E \subseteq \mbox{\em EX}$, i.e.\ the set $\{
\sigma(e) | e \in E \}$.  The collection $\sigma(E)$ may of course contain
fewer expressions than $E$ because duplicate expressions could result from the
substitution.

\subsection{Statements}

We denote by $\mbox{\em DV}$ the set of all
unordered pairs $\{\alpha, \beta \} \subseteq \mbox{\em VR}$ such that $\alpha
\neq \beta$.  $\mbox{\em DV}$ stands for ``distinct variables.''

A {\em pre-statement}\index{pre-statement!in a formal system} is a
quadruple $\langle D,T,H,A \rangle$ such that
$D\subseteq \mbox{\em DV}$, $T\subseteq \mbox{\em EX}_2$, $H\subseteq
\mbox{\em EX}_C$ and $H$ is finite,
$A\in \mbox{\em EX}_C$, ${\cal V}(H\cup\{A\}) \subseteq
{\cal V}(T)$, and $\forall e,f\in T {\ } {\cal V}(e) \neq {\cal V}(f)$ (or
equivalently, $e_1 \ne f_1$) whenever $e \neq f$. The terms of the quadruple are called {\em
distinct-variable restrictions},\index{disjoint-variable restriction!in a
formal system} {\em variable-type hypotheses},\index{variable-type
hypothesis!in a formal system} {\em logical hypotheses},\index{logical
hypothesis!in a formal system} and the {\em assertion}\index{assertion!in a
formal system} respectively.  We denote by $T_M$ ({\em mandatory variable-type
hypotheses}\index{mandatory variable-type hypothesis!in a formal system}) the
subset of $T$ such that ${\cal V}(T_M) ={\cal V}(H \cup \{A\})$.  We denote by
$D_M=\{\{\alpha,\beta\}\in D|\{\alpha,\beta\}\subseteq {\cal V}(T_M)\}$ the
{\em mandatory distinct-variable restrictions}\index{mandatory
disjoint-variable restriction!in a formal system} of the pre-statement.
The set
of {\em mandatory hypotheses}\index{mandatory hypothesis!in a formal system}
is $T_M\cup H$.  We call the quadruple $\langle D_M,T_M,H,A \rangle$
the {\em reduct}\index{reduct!in a formal system} of
the pre-statement $\langle D,T,H,A \rangle$.

A {\em statement} is the reduct of some pre-statement\index{statement!in a
formal system}.  A statement is therefore a special kind of pre-statement;
in particular, a statement is the reduct of itself.

{\footnotesize\begin{quotation}
{\em Comment.}  $T$ is a set of expressions, each of length 2, that associate
a set of constants (``variable types'') with a set of variables.  The
condition ${\cal V}(H\cup\{A\}) \subseteq {\cal V}(T) $
means that each variable occurring in a statement's logical
hypotheses or assertion must have an associated variable-type hypothesis or
``type declaration,'' in  analogy to a computer programming language, where a
variable must be declared to be say, a string or an integer.  The requirement
that $\forall e,f\in T \, e_1 \ne f_1$ for $e\neq f$
means that each variable must be
associated with a unique constant designating its variable type; e.g., a
variable might be a ``wff'' or a ``set'' but not both.

Distinct-variable restrictions are used to specify what variable substitutions
are permissible to make for the statement to remain valid.  For example, in
the theorem scheme of set theory $\lnot\forall x\,x=y$ we may not substitute
the same variable for both $x$ and $y$.  On the other hand, the theorem scheme
$x=y\to y=x$ does not require that $x$ and $y$ be distinct, so we do not
require a distinct-variable restriction, although having one
would cause no harm other than making the scheme less general.

A mandatory variable-type hypothesis is one whose variable exists in a logical
hypothesis or the assertion.  A provable pre-statement
(defined below) may require
non-mandatory variable-type hypotheses that effectively introduce ``dummy''
variables for use in its proof.  Any number of dummy variables might
be required by a specific proof; indeed, it has been shown by H.\
Andr\'{e}ka\index{Andr{\'{e}}ka, H.} \cite{Nemeti} that there is no finite
upper bound to the number of dummy variables needed to prove an arbitrary
theorem in first-order logic (with equality) having a fixed number $n>2$ of
individual variables.  (See also the Comment on p.~\pageref{nodd}.)
For this reason we do not set a finite size bound on the collections $D$ and
$T$, although in an actual application (Metamath database) these will of
course be finite, increased to whatever size is necessary as more
proofs are added.
\end{quotation}}

{\footnotesize\begin{quotation}
{\em Relationship to Metamath.} A pre-statement of a formal system
corresponds to an extended frame in a Metamath database
(Section~\ref{frames}).  The collections $D$, $T$, and $H$ correspond
respectively to the \texttt{\$d}, \texttt{\$f}, and \texttt{\$e}
statement collections in an extended frame.  The expression $A$
corresponds to the \texttt{\$a} (or \texttt{\$p}) statement in an
extended frame.

A statement of a formal system corresponds to a frame in a Metamath
database.
\end{quotation}}

\subsection{Formal Systems}

A {\em formal system}\index{formal system} is a
triple $\langle \mbox{\em CN},\mbox{\em
VR},\Gamma\rangle$ where $\Gamma$ is a set of statements.  The members of
$\Gamma$ are called {\em axiomatic statements}.\index{axiomatic
statement!in a formal system}  Sometimes we will refer to a
formal system by just $\Gamma$ when $\mbox{\em CN}$ and $\mbox{\em VR}$ are
understood.

Given a formal system $\Gamma$, the {\em closure}\index{closure}\footnote{This
definition of closure incorporates a simplification due to
Josh Purinton.\index{Purinton, Josh}.} of a
pre-statement
$\langle D,T,H,A \rangle$ is the smallest set $C$ of expressions
such that:
%\begin{enumerate}
%  \item $T\cup H\subseteq C$; and
%  \item If for some axiomatic statement
%    $\langle D_M',T_M',H',A' \rangle \in \Gamma_A$, for
%    some $E \subseteq C$, some $F \subseteq C-T$ (where ``-'' denotes
%    set difference), and some substitution
%    $\sigma$ we have
%    \begin{enumerate}
%       \item $\sigma(T_M') = E$ (where, as above, the $M$ denotes the
%           mandatory variable-type hypotheses of $T^A$);
%       \item $\sigma(H') = F$;
%       \item for all $\{\alpha,\beta\}\in D^A$ and $\subseteq
%         {\cal V}(T_M')$, for all $\gamma\in {\cal V}(\sigma(\langle \alpha
%         \rangle))$, and for all $\delta\in  {\cal V}(\sigma(\langle \beta
%         \rangle))$, we have $\{\gamma, \delta\} \in D$;
%   \end{enumerate}
%   then $\sigma(A') \in C$.
%\end{enumerate}
\begin{list}{}{\itemsep 0.0pt}
  \item[1.] $T\cup H\subseteq C$; and
  \item[2.] If for some axiomatic statement
    $\langle D_M',T_M',H',A' \rangle \in
       \Gamma$ and for some substitution
    $\sigma$ we have
    \begin{enumerate}
       \item[a.] $\sigma(T_M' \cup H') \subseteq C$; and
       \item[b.] for all $\{\alpha,\beta\}\in D_M'$, for all $\gamma\in
         {\cal V}(\sigma(\langle \alpha
         \rangle))$, and for all $\delta\in  {\cal V}(\sigma(\langle \beta
         \rangle))$, we have $\{\gamma, \delta\} \in D$;
   \end{enumerate}
   then $\sigma(A') \in C$.
\end{list}
A pre-statement $\langle D,T,H,A
\rangle$ is {\em provable}\index{provable statement!in a formal
system} if $A\in C$ i.e.\ if its assertion belongs to its
closure.  A statement is {\em provable} if it is
the reduct of a provable pre-statement.
The {\em universe}\index{universe of a formal system}
of a formal system is
the collection of all of its provable statements.  Note that the
set of axiomatic statements $\Gamma$ in a formal system is a subset of its
universe.

{\footnotesize\begin{quotation}
{\em Comment.} The first condition in the definition of closure simply says
that the hypotheses of the pre-statement are in its closure.

Condition 2(a) says that a substitution exists that makes the
mandatory hypotheses of an axiomatic statement exactly match some members of
the closure.  This is what we explicitly demonstrate in a Metamath language
proof.

%Conditions 2(a) and 2(b) say that a substitution exists that makes the
%(mandatory) hypotheses of an axiomatic statement exactly match some members of
%the closure.  This is what we explicitly demonstrate with a Metamath language
%proof.
%
%The set of expressions $F$ in condition 2(b) excludes the variable-type
%hypotheses; this is done because non-mandatory variable-type hypotheses are
%effectively ``dropped'' as irrelevant whereas logical hypotheses must be
%retained to achieve a consistent logical system.

Condition 2(b) describes how distinct-variable restrictions in the axiomatic
statement must be met.  It means that after a substitution for two variables
that must be distinct, the resulting two expressions must either contain no
variables, or if they do, they may not have variables in common, and each pair
of any variables they do have, with one variable from each expression, must be
specified as distinct in the original statement.
\end{quotation}}

{\footnotesize\begin{quotation}
{\em Relationship to Metamath.} Axiomatic statements
 and provable statements in a formal
system correspond to the frames for \texttt{\$a} and \texttt{\$p} statements
respectively in a Metamath database.  The set of axiomatic statements is a
subset of the set of provable statements in a formal system, although in a
Metamath database a \texttt{\$a} statement is distinguished by not having a
proof.  A Metamath language proof for a \texttt{\$p} statement tells the computer
how to explicitly construct a series of members of the closure ultimately
leading to a demonstration that the assertion
being proved is in the closure.  The actual closure typically contains
an infinite number of expressions.  A formal system itself does not have
an explicit object called a ``proof'' but rather the existence of a proof
is implied indirectly by membership of an assertion in a provable
statement's closure.  We do this to make the formal system easier
to describe in the language of set theory.

We also note that once established as provable, a statement may be considered
to acquire the same status as an axiomatic statement, because if the set of
axiomatic statements is extended with a provable statement, the universe of
the formal system remains unchanged (provided that $\mbox{\em VR}$ is
infinite).
In practice, this means we can build a hierarchy of provable statements to
more efficiently establish additional provable statements.  This is
what we do in Metamath when we allow proofs to reference previous
\texttt{\$p} statements as well as previous \texttt{\$a} statements.
\end{quotation}}

\section{Examples of Formal Systems}

{\footnotesize\begin{quotation}
{\em Relationship to Metamath.} The examples in this section, except Example~2,
are for the most part exact equivalents of the development in the set
theory database \texttt{set.mm}.  You may want to compare Examples~1, 3, and 5
to Section~\ref{metaaxioms}, Example 4 to Sections~\ref{metadefprop} and
\ref{metadefpred}, and Example 6 to
Section~\ref{setdefinitions}.\label{exampleref}
\end{quotation}}

\subsection{Example~1---Propositional Calculus}\index{propositional calculus}

Classical propositional calculus can be described by the following formal
system.  We assume the set of variables is infinite.  Rather than denoting the
constants and variables by $c_0, c_1, \ldots$ and $v_0, v_1, \ldots$, for
readability we will instead use more conventional symbols, with the
understanding of course that they denote distinct primitive objects.
Also for readability we may omit commas between successive terms of a
sequence; thus $\langle \mbox{wff\ } \varphi\rangle$ denotes
$\langle \mbox{wff}, \varphi\rangle$.

Let
\begin{itemize}
  \item[] $\mbox{\em CN}=\{\mbox{wff}, \vdash, \to, \lnot, (,)\}$
  \item[] $\mbox{\em VR}=\{\varphi,\psi,\chi,\ldots\}$
  \item[] $T = \{\langle \mbox{wff\ } \varphi\rangle,
             \langle \mbox{wff\ } \psi\rangle,
             \langle \mbox{wff\ } \chi\rangle,\ldots\}$, i.e.\ those
             expressions of length 2 whose first member is $\mbox{\rm wff}$
             and whose second member belongs to $\mbox{\em VR}$.\footnote{For
convenience we let $T$ be an infinite set; the definition of a statement
permits this in principle.  Since a Metamath source file has a finite size, in
practice we must of course use appropriate finite subsets of this $T$,
specifically ones containing at least the mandatory variable-type
hypotheses.  Similarly, in the source file we introduce new variables as
required, with the understanding that a potentially infinite number of
them are available.}
\noindent Then $\Gamma$ consists of the axiomatic statements that
are the reducts of the following pre-statements:
    \begin{itemize}
      \item[] $\langle\varnothing,T,\varnothing,
               \langle \mbox{wff\ }(\varphi\to\psi)\rangle\rangle$
      \item[] $\langle\varnothing,T,\varnothing,
               \langle \mbox{wff\ }\lnot\varphi\rangle\rangle$
      \item[] $\langle\varnothing,T,\varnothing,
               \langle \vdash(\varphi\to(\psi\to\varphi))
               \rangle\rangle$
      \item[] $\langle\varnothing,T,
               \varnothing,
               \langle \vdash((\varphi\to(\psi\to\chi))\to
               ((\varphi\to\psi)\to(\varphi\to\chi)))
               \rangle\rangle$
      \item[] $\langle\varnothing,T,
               \varnothing,
               \langle \vdash((\lnot\varphi\to\lnot\psi)\to
               (\psi\to\varphi))\rangle\rangle$
      \item[] $\langle\varnothing,T,
               \{\langle\vdash(\varphi\to\psi)\rangle,
                 \langle\vdash\varphi\rangle\},
               \langle\vdash\psi\rangle\rangle$
    \end{itemize}
\end{itemize}

(For example, the reduct of $\langle\varnothing,T,\varnothing,
               \langle \mbox{wff\ }(\varphi\to\psi)\rangle\rangle$
is
\begin{itemize}
\item[] $\langle\varnothing,
\{\langle \mbox{wff\ } \varphi\rangle,
             \langle \mbox{wff\ } \psi\rangle\},
             \varnothing,
               \langle \mbox{wff\ }(\varphi\to\psi)\rangle\rangle$,
\end{itemize}
which is the first axiomatic statement.)

We call the members of $\mbox{\em VR}$ {\em wff variables} or (in the context
of first-order logic which we will describe shortly) {\em wff metavariables}.
Note that the symbols $\phi$, $\psi$, etc.\ denote actual specific members of
$\mbox{\em VR}$; they are not metavariables of our expository language (which
we denote with $\alpha$, $\beta$, etc.) but are instead (meta)constant symbols
(members of $\mbox{\em SM}$) from the point of view of our expository
language.  The equivalent system of propositional calculus described in
\cite{Tarski1965} also uses the symbols $\phi$, $\psi$, etc.\ to denote wff
metavariables, but in \cite{Tarski1965} unlike here those are metavariables of
the expository language and not primitive symbols of the formal system.

The first two statements define wffs: if $\varphi$ and $\psi$ are wffs, so is
$(\varphi \to \psi)$; if $\varphi$ is a wff, so is $\lnot\varphi$. The next
three are the axioms of propositional calculus: if $\varphi$ and $\psi$ are
wffs, then $\vdash (\varphi \to (\psi \to \varphi))$ is an (axiomatic)
theorem; etc. The
last is the rule of modus ponens: if $\varphi$ and $\psi$ are wffs, and
$\vdash (\varphi\to\psi)$ and $\vdash \varphi$ are theorems, then $\vdash
\psi$ is a theorem.

The correspondence to ordinary propositional calculus is as follows.  We
consider only provable statements of the form $\langle\varnothing,
T,\varnothing,A\rangle$ with $T$ defined as above.  The first term of the
assertion $A$ of any such statement is either ``wff'' or ``$\vdash$''.  A
statement for which the first term is ``wff'' is a {\em wff} of propositional
calculus, and one where the first term is ``$\vdash$'' is a {\em
theorem (scheme)} of propositional calculus.

The universe of this formal system also contains many other provable
statements.  Those with distinct-variable restrictions are irrelevant because
propositional calculus has no constraints on substitutions.  Those that have
logical hypotheses we call {\em inferences}\index{inference} when
the logical hypotheses are of the form
$\langle\vdash\rangle\frown w$ where $w$ is a wff (with the leading constant
term ``wff'' removed).  Inferences (other than the modus ponens rule) are not a
proper part of propositional calculus but are convenient to use when building a
hierarchy of provable statements.  A provable statement with a nonsense
hypothesis such as $\langle \to,\vdash,\lnot\rangle$, and this same expression
as its assertion, we consider irrelevant; no use can be made of it in
proving theorems, since there is no way to eliminate the nonsense hypothesis.

{\footnotesize\begin{quotation}
{\em Comment.} Our use of parentheses in the definition of a wff illustrates
how axiomatic statements should be carefully stated in a way that
ties in unambiguously with the substitutions allowed by the formal system.
There are many ways we could have defined wffs---for example, Polish
prefix notation would have allowed us to omit parentheses entirely, at
the expense of readability---but we must define them in a way that is
unambiguous.  For example, if we had omitted parentheses from the
definition of $(\varphi\to \psi)$, the wff $\lnot\varphi\to \psi$ could
be interpreted as either $\lnot(\varphi\to\psi)$ or $(\lnot\varphi\to\psi)$
and would have allowed us to prove nonsense.  Note that there is no
concept of operator binding precedence built into our formal system.
\end{quotation}}

\begin{sloppy}
\subsection{Example~2---Predicate Calculus with Equality}\index{predicate
calculus}
\end{sloppy}

Here we extend Example~1 to include predicate calculus with equality,
illustrating the use of distinct-variable restrictions.  This system is the
same as Tarski's system $\mathfrak{S}_2$ in \cite{Tarski1965} (except that the
axioms of propositional calculus are different but equivalent, and a redundant
axiom is omitted).  We extend $\mbox{\em CN}$ with the constants
$\{\mbox{var},\forall,=\}$.  We extend $\mbox{\em VR}$ with an infinite set of
{\em individual metavariables}\index{individual
metavariable} $\{x,y,z,\ldots\}$ and denote this subset
$\mbox{\em Vr}$.

We also join to $\mbox{\em CN}$ a possibly infinite set $\mbox{\em Pr}$ of {\em
predicates} $\{R,S,\ldots\}$.  We associate with $\mbox{\em Pr}$ a function
$\mbox{rnk}$ from $\mbox{\em Pr}$ to $\omega$, and for $\alpha\in \mbox{\em
Pr}$ we call $\mbox{rnk}(\alpha)$ the {\em rank} of the predicate $\alpha$,
which is simply the number of ``arguments'' that the predicate has.  (Most
applications of predicate calculus will have a finite number of predicates;
for example, set theory has the single two-argument or binary predicate $\in$,
which is usually written with its arguments surrounding the predicate symbol
rather than with the prefix notation we will use for the general case.)  As a
device to facilitate our discussion, we will let $\mbox{\em Vs}$ be any fixed
one-to-one function from $\omega$ to $\mbox{\em Vr}$; thus $\mbox{\em Vs}$ is
any simple infinite sequence of individual metavariables with no repeating
terms.

In this example we will not include the function symbols that are often part of
formalizations of predicate calculus.  Using metalogical arguments that are
beyond the scope of our discussion, it can be shown that our formalization is
equivalent when functions are introduced via appropriate definitions.

We extend the set $T$ defined in Example~1 with the expressions
$\{\langle \mbox{var\ } x\rangle,$ $ \langle \mbox{var\ } y\rangle, \langle
\mbox{var\ } z\rangle,\ldots\}$.  We extend the $\Gamma$ above
with the axiomatic statements that are the reducts of the following
pre-statements:
\begin{list}{}{\itemsep 0.0pt}
      \item[] $\langle\varnothing,T,\varnothing,
               \langle \mbox{wff\ }\forall x\,\varphi\rangle\rangle$
      \item[] $\langle\varnothing,T,\varnothing,
               \langle \mbox{wff\ }x=y\rangle\rangle$
      \item[] $\langle\varnothing,T,
               \{\langle\vdash\varphi\rangle\},
               \langle\vdash\forall x\,\varphi\rangle\rangle$
      \item[] $\langle\varnothing,T,\varnothing,
               \langle \vdash((\forall x(\varphi\to\psi)
                  \to(\forall x\,\varphi\to\forall x\,\psi))
               \rangle\rangle$
      \item[] $\langle\{\{x,\varphi\}\},T,\varnothing,
               \langle \vdash(\varphi\to\forall x\,\varphi)
               \rangle\rangle$
      \item[] $\langle\{\{x,y\}\},T,\varnothing,
               \langle \vdash\lnot\forall x\lnot x=y
               \rangle\rangle$
      \item[] $\langle\varnothing,T,\varnothing,
               \langle \vdash(x=z
                  \to(x=y\to z=y))
               \rangle\rangle$
      \item[] $\langle\varnothing,T,\varnothing,
               \langle \vdash(y=z
                  \to(x=y\to x=z))
               \rangle\rangle$
\end{list}
These are the axioms not involving predicate symbols. The first two statements
extend the definition of a wff.  The third is the rule of generalization.  The
fifth states, in effect, ``For a wff $\varphi$ and variable $x$,
$\vdash(\varphi\to\forall x\,\varphi)$, provided that $x$ does not occur in
$\varphi$.''  The sixth states ``For variables $x$ and $y$,
$\vdash\lnot\forall x\lnot x = y$, provided that $x$ and $y$ are distinct.''
(This proviso is not necessary but was included by Tarski to
weaken the axiom and still show that the system is logically complete.)

Finally, for each predicate symbol $\alpha\in \mbox{\em Pr}$, we add to
$\Gamma$ an axiomatic statement, extending the definition of wff,
that is the reduct of the following pre-statement:
\begin{displaymath}
    \langle\varnothing,T,\varnothing,
            \langle \mbox{wff},\alpha\rangle\
            \frown \mbox{\em Vs}\restriction\mbox{rnk}(\alpha)\rangle
\end{displaymath}
and for each $\alpha\in \mbox{\em Pr}$ and each $n < \mbox{rnk}(\alpha)$
we add to $\Gamma$ an equality axiom that is the reduct of the
following pre-statement:
\begin{eqnarray*}
    \lefteqn{\langle\varnothing,T,\varnothing,
            \langle
      \vdash,(,\mbox{\em Vs}_n,=,\mbox{\em Vs}_{\mbox{rnk}(\alpha)},\to,
            (,\alpha\rangle\frown \mbox{\em Vs}\restriction\mbox{rnk}(\alpha)} \\
  & & \frown
            \langle\to,\alpha\rangle\frown \mbox{\em Vs}\restriction n\frown
            \langle \mbox{\em Vs}_{\mbox{rnk}(\alpha)}\rangle \\
 & & \frown
            \mbox{\em Vs}\restriction(\mbox{rnk}(\alpha)\setminus(n+1))\frown
            \langle),)\rangle\rangle
\end{eqnarray*}
where $\restriction$ denotes function domain restriction and $\setminus$
denotes set difference.  Recall that a subscript on $\mbox{\em Vs}$
denotes one of its terms.  (In the above two axiom sets commas are placed
between successive terms of sequences to prevent ambiguity, and if you examine
them with care you will be able to distinguish those parentheses that denote
constant symbols from those of our expository language that delimit function
arguments.  Although it might have been better to use boldface for our
primitive symbols, unfortunately boldface was not available for all characters
on the \LaTeX\ system used to typeset this text.)  These seemingly forbidding
axioms can be understood by analogy to concatenation of substrings in a
computer language.  They are actually relatively simple for each specific case
and will become clearer by looking at the special case of a binary predicate
$\alpha = R$ where $\mbox{rnk}(R)=2$.  Letting $\mbox{\em Vs}$ be the sequence
$\langle x,y,z,\ldots\rangle$, the axioms we would add to $\Gamma$ for this
case would be the wff extension and two equality axioms that are the
reducts of the pre-statements:
\begin{list}{}{\itemsep 0.0pt}
      \item[] $\langle\varnothing,T,\varnothing,
               \langle \mbox{wff\ }R x y\rangle\rangle$
      \item[] $\langle\varnothing,T,\varnothing,
               \langle \vdash(x=z
                  \to(R x y \to R z y))
               \rangle\rangle$
      \item[] $\langle\varnothing,T,\varnothing,
               \langle \vdash(y=z
                  \to(R x y \to R x z))
               \rangle\rangle$
\end{list}
Study these carefully to see how the general axioms above evaluate to
them.  In practice, typically only a few special cases such as this would be
needed, and in any case the Metamath language will only permit us to describe
a finite number of predicates, as opposed to the infinite number permitted by
the formal system.  (If an infinite number should be needed for some reason,
we could not define the formal system directly in the Metamath language but
could instead define it metalogically under set theory as we
do in this appendix, and only the underlying set theory, with its single
binary predicate, would be defined directly in the Metamath language.)


{\footnotesize\begin{quotation}
{\em Comment.}  As we noted earlier, the specific variables denoted by the
symbols $x,y,z,\ldots\in \mbox{\em Vr}\subseteq \mbox{\em VR}\subseteq
\mbox{\em SM}$ in Example~2 are not the actual variables of ordinary predicate
calculus but should be thought of as metavariables ranging over them.  For
example, a distinct-variable restriction would be meaningless for actual
variables of ordinary predicate calculus since two different actual variables
are by definition distinct.  And when we talk about an arbitrary
representative $\alpha\in \mbox{\em Vr}$, $\alpha$ is a metavariable (in our
expository language) that ranges over metavariables (which are primitives of
our formal system) each of which ranges over the actual individual variables
of predicate calculus (which are never mentioned in our formal system).

The constant called ``var'' above is called \texttt{setvar} in the
\texttt{set.mm} database file, but it means the same thing.  I felt
that ``var'' is a more meaningful name in the context of predicate
calculus, whose use is not limited to set theory.  For consistency we
stick with the name ``var'' throughout this Appendix, even after set
theory is introduced.
\end{quotation}}

\subsection{Free Variables and Proper Substitution}\index{free variable}
\index{proper substitution}\index{substitution!proper}

Typical representations of mathematical axioms use concepts such
as ``free variable,'' ``bound variable,'' and ``proper substitution''
as primitive notions.
A free variable is a variable that
is not a parameter of any container expression.
A bound variable is the opposite of a free variable; it is a
a variable that has been bound in a container expression.
For example, in the expression $\forall x \varphi$ (for all $x$, $\varphi$
is true), the variable $x$
is bound within the for-all ($\forall$) expression.
It is possible to change one variable to another, and that process is called
``proper substitution.''
In most books, proper substitution has a somewhat complicated recursive
definition with multiple cases based on the occurrences of free and
bound variables.
You may consult
\cite[ch.\ 3--4]{Hamilton}\index{Hamilton, Alan G.} (as well as
many other texts) for more formal details about these terms.

Using these concepts as \texttt{primitives} creates complications
for computer implementations.

In the system of Example~2, there are no primitive notions of free variable
and proper substitution.  Tarski \cite{Tarski1965} shows that this system is
logically equivalent to the more typical textbook systems that do have these
primitive notions, if we introduce these notions with appropriate definitions
and metalogic.  We could also define axioms for such systems directly,
although the recursive definitions of free variable and proper substitution
would be messy and awkward to work with.  Instead, we mention two devices that
can be used in practice to mimic these notions.  (1) Instead of introducing
special notation to express (as a logical hypothesis) ``where $x$ is not free
in $\varphi$'' we can use the logical hypothesis $\vdash(\varphi\to\forall
x\,\varphi)$.\label{effectivelybound}\index{effectively
not free}\footnote{This is a slightly weaker requirement than ``where $x$ is
not free in $\varphi$.''  If we let $\varphi$ be $x=x$, we have the theorem
$(x=x\to\forall x\,x=x)$ which satisfies the hypothesis, even though $x$ is
free in $x=x$ .  In a case like this we say that $x$ is {\em effectively not
free}\index{effectively not free} in $x=x$, since $x=x$ is logically
equivalent to $\forall x\,x=x$ in which $x$ is bound.} (2) It can be shown
that the wff $((x=y\to\varphi)\wedge\exists x(x=y\wedge\varphi))$ (with the
usual definitions of $\wedge$ and $\exists$; see Example~4 below) is logically
equivalent to ``the wff that results from proper substitution of $y$ for $x$
in $\varphi$.''  This works whether or not $x$ and $y$ are distinct.

\subsection{Metalogical Completeness}\index{metalogical completeness}

In the system of Example~2, the
following are provable pre-statements (and their reducts are
provable statements):
\begin{eqnarray*}
      & \langle\{\{x,y\}\},T,\varnothing,
               \langle \vdash\lnot\forall x\lnot x=y
               \rangle\rangle & \\
     &  \langle\varnothing,T,\varnothing,
               \langle \vdash\lnot\forall x\lnot x=x
               \rangle\rangle &
\end{eqnarray*}
whereas the following pre-statement is not to my knowledge provable (but
in any case we will pretend it's not for sake of illustration):
\begin{eqnarray*}
     &  \langle\varnothing,T,\varnothing,
               \langle \vdash\lnot\forall x\lnot x=y
               \rangle\rangle &
\end{eqnarray*}
In other words, we can prove ``$\lnot\forall x\lnot x=y$ where $x$ and $y$ are
distinct'' and separately prove ``$\lnot\forall x\lnot x=x$'', but we can't
prove the combined general case ``$\lnot\forall x\lnot x=y$'' that has no
proviso.  Now this does not compromise logical completeness, because the
variables are really metavariables and the two provable cases together cover
all possible cases.  The third case can be considered a metatheorem whose
direct proof, using the system of Example~2, lies outside the capability of the
formal system.

Also, in the system of Example~2 the following pre-statement is not to my
knowledge provable (again, a conjecture that we will pretend to be the case):
\begin{eqnarray*}
     & \langle\varnothing,T,\varnothing,
               \langle \vdash(\forall x\, \varphi\to\varphi)
               \rangle\rangle &
\end{eqnarray*}
Instead, we can only prove specific cases of $\varphi$ involving individual
metavariables, and by induction on formula length, prove as a metatheorem
outside of our formal system the general statement above.  The details of this
proof are found in \cite{Kalish}.

There does, however, exist a system of predicate calculus in which all such
``simple metatheorems'' as those above can be proved directly, and we present
it in Example~3. A {\em simple metatheorem}\index{simple metatheorem}
is any statement of the formal
system of Example~2 where all distinct variable restrictions consist of either
two individual metavariables or an individual metavariable and a wff
metavariable, and which is provable by combining cases outside the system as
above.  A system is {\em metalogically complete}\index{metalogical
completeness} if all of its simple
metatheorems are (directly) provable statements. The precise definition of
``simple metatheorem'' and the proof of the ``metalogical completeness'' of
Example~3 is found in Remark 9.6 and Theorem 9.7 of \cite{Megill}.\index{Megill,
Norman}

\begin{sloppy}
\subsection{Example~3---Metalogically Complete Predicate
Calculus with
Equality}
\end{sloppy}

For simplicity we will assume there is one binary predicate $R$;
this system suffices for set theory, where the $R$ is of course the $\in$
predicate.  We label the axioms as they appear in \cite{Megill}.  This
system is logically equivalent to that of Example~2 (when the latter is
restricted to this single binary predicate) but is also metalogically
complete.\index{metalogical completeness}

Let
\begin{itemize}
  \item[] $\mbox{\em CN}=\{\mbox{wff}, \mbox{var}, \vdash, \to, \lnot, (,),\forall,=,R\}$.
  \item[] $\mbox{\em VR}=\{\varphi,\psi,\chi,\ldots\}\cup\{x,y,z,\ldots\}$.
  \item[] $T = \{\langle \mbox{wff\ } \varphi\rangle,
             \langle \mbox{wff\ } \psi\rangle,
             \langle \mbox{wff\ } \chi\rangle,\ldots\}\cup
       \{\langle \mbox{var\ } x\rangle, \langle \mbox{var\ } y\rangle, \langle
       \mbox{var\ }z\rangle,\ldots\}$.

\noindent Then
  $\Gamma$ consists of the reducts of the following pre-statements:
    \begin{itemize}
      \item[] $\langle\varnothing,T,\varnothing,
               \langle \mbox{wff\ }(\varphi\to\psi)\rangle\rangle$
      \item[] $\langle\varnothing,T,\varnothing,
               \langle \mbox{wff\ }\lnot\varphi\rangle\rangle$
      \item[] $\langle\varnothing,T,\varnothing,
               \langle \mbox{wff\ }\forall x\,\varphi\rangle\rangle$
      \item[] $\langle\varnothing,T,\varnothing,
               \langle \mbox{wff\ }x=y\rangle\rangle$
      \item[] $\langle\varnothing,T,\varnothing,
               \langle \mbox{wff\ }Rxy\rangle\rangle$
      \item[(C1$'$)] $\langle\varnothing,T,\varnothing,
               \langle \vdash(\varphi\to(\psi\to\varphi))
               \rangle\rangle$
      \item[(C2$'$)] $\langle\varnothing,T,
               \varnothing,
               \langle \vdash((\varphi\to(\psi\to\chi))\to
               ((\varphi\to\psi)\to(\varphi\to\chi)))
               \rangle\rangle$
      \item[(C3$'$)] $\langle\varnothing,T,
               \varnothing,
               \langle \vdash((\lnot\varphi\to\lnot\psi)\to
               (\psi\to\varphi))\rangle\rangle$
      \item[(C4$'$)] $\langle\varnothing,T,
               \varnothing,
               \langle \vdash(\forall x(\forall x\,\varphi\to\psi)\to
                 (\forall x\,\varphi\to\forall x\,\psi))\rangle\rangle$
      \item[(C5$'$)] $\langle\varnothing,T,
               \varnothing,
               \langle \vdash(\forall x\,\varphi\to\varphi)\rangle\rangle$
      \item[(C6$'$)] $\langle\varnothing,T,
               \varnothing,
               \langle \vdash(\forall x\forall y\,\varphi\to
                 \forall y\forall x\,\varphi)\rangle\rangle$
      \item[(C7$'$)] $\langle\varnothing,T,
               \varnothing,
               \langle \vdash(\lnot\varphi\to\forall x\lnot\forall x\,\varphi
                 )\rangle\rangle$
      \item[(C8$'$)] $\langle\varnothing,T,
               \varnothing,
               \langle \vdash(x=y\to(x=z\to y=z))\rangle\rangle$
      \item[(C9$'$)] $\langle\varnothing,T,
               \varnothing,
               \langle \vdash(\lnot\forall x\, x=y\to(\lnot\forall x\, x=z\to
                 (y=z\to\forall x\, y=z)))\rangle\rangle$
      \item[(C10$'$)] $\langle\varnothing,T,
               \varnothing,
               \langle \vdash(\forall x(x=y\to\forall x\,\varphi)\to
                 \varphi))\rangle\rangle$
      \item[(C11$'$)] $\langle\varnothing,T,
               \varnothing,
               \langle \vdash(\forall x\, x=y\to(\forall x\,\varphi
               \to\forall y\,\varphi))\rangle\rangle$
      \item[(C12$'$)] $\langle\varnothing,T,
               \varnothing,
               \langle \vdash(x=y\to(Rxz\to Ryz))\rangle\rangle$
      \item[(C13$'$)] $\langle\varnothing,T,
               \varnothing,
               \langle \vdash(x=y\to(Rzx\to Rzy))\rangle\rangle$
      \item[(C15$'$)] $\langle\varnothing,T,
               \varnothing,
               \langle \vdash(\lnot\forall x\, x=y\to(x=y\to(\varphi
                 \to\forall x(x=y\to\varphi))))\rangle\rangle$
      \item[(C16$'$)] $\langle\{\{x,y\}\},T,
               \varnothing,
               \langle \vdash(\forall x\, x=y\to(\varphi\to\forall x\,\varphi)
                 )\rangle\rangle$
      \item[(C5)] $\langle\{\{x,\varphi\}\},T,\varnothing,
               \langle \vdash(\varphi\to\forall x\,\varphi)
               \rangle\rangle$
      \item[(MP)] $\langle\varnothing,T,
               \{\langle\vdash(\varphi\to\psi)\rangle,
                 \langle\vdash\varphi\rangle\},
               \langle\vdash\psi\rangle\rangle$
      \item[(Gen)] $\langle\varnothing,T,
               \{\langle\vdash\varphi\rangle\},
               \langle\vdash\forall x\,\varphi\rangle\rangle$
    \end{itemize}
\end{itemize}

While it is known that these axioms are ``metalogically complete,'' it is
not known whether they are independent (i.e.\ none is
redundant) in the metalogical sense; specifically, whether any axiom (possibly
with additional non-mandatory distinct-variable restrictions, for use with any
dummy variables in its proof) is provable from the others.  Note that
metalogical independence is a weaker requirement than independence in the
usual logical sense.  Not all of the above axioms are logically independent:
for example, C9$'$ can be proved as a metatheorem from the others, outside the
formal system, by combining the possible cases of distinct variables.

\subsection{Example~4---Adding Definitions}\index{definition}
There are several ways to add definitions to a formal system.  Probably the
most proper way is to consider definitions not as part of the formal system at
all but rather as abbreviations that are part of the expository metalogic
outside the formal system.  For convenience, though, we may use the formal
system itself to incorporate definitions, adding them as axiomatic extensions
to the system.  This could be done by adding a constant representing the
concept ``is defined as'' along with axioms for it. But there is a nicer way,
at least in this writer's opinion, that introduces definitions as direct
extensions to the language rather than as extralogical primitive notions.  We
introduce additional logical connectives and provide axioms for them.  For
systems of logic such as Examples 1 through 3, the additional axioms must be
conservative in the sense that no wff of the original system that was not a
theorem (when the initial term ``wff'' is replaced by ``$\vdash$'' of course)
becomes a theorem of the extended system.  In this example we extend Example~3
(or 2) with standard abbreviations of logic.

We extend $\mbox{\em CN}$ of Example~3 with new constants $\{\leftrightarrow,
\wedge,\vee,\exists\}$, corresponding to logical equivalence,\index{logical
equivalence ($\leftrightarrow$)}\index{biconditional ($\leftrightarrow$)}
conjunction,\index{conjunction ($\wedge$)} disjunction,\index{disjunction
($\vee$)} and the existential quantifier.\index{existential quantifier
($\exists$)}  We extend $\Gamma$ with the axiomatic statements that are
the reducts of the following pre-statements:
\begin{list}{}{\itemsep 0.0pt}
      \item[] $\langle\varnothing,T,\varnothing,
               \langle \mbox{wff\ }(\varphi\leftrightarrow\psi)\rangle\rangle$
      \item[] $\langle\varnothing,T,\varnothing,
               \langle \mbox{wff\ }(\varphi\vee\psi)\rangle\rangle$
      \item[] $\langle\varnothing,T,\varnothing,
               \langle \mbox{wff\ }(\varphi\wedge\psi)\rangle\rangle$
      \item[] $\langle\varnothing,T,\varnothing,
               \langle \mbox{wff\ }\exists x\, \varphi\rangle\rangle$
  \item[] $\langle\varnothing,T,\varnothing,
     \langle\vdash ( ( \varphi \leftrightarrow \psi ) \to
     ( \varphi \to \psi ) )\rangle\rangle$
  \item[] $\langle\varnothing,T,\varnothing,
     \langle\vdash ((\varphi\leftrightarrow\psi)\to
    (\psi\to\varphi))\rangle\rangle$
  \item[] $\langle\varnothing,T,\varnothing,
     \langle\vdash ((\varphi\to\psi)\to(
     (\psi\to\varphi)\to(\varphi
     \leftrightarrow\psi)))\rangle\rangle$
  \item[] $\langle\varnothing,T,\varnothing,
     \langle\vdash (( \varphi \wedge \psi ) \leftrightarrow\neg ( \varphi
     \to \neg \psi )) \rangle\rangle$
  \item[] $\langle\varnothing,T,\varnothing,
     \langle\vdash (( \varphi \vee \psi ) \leftrightarrow (\neg \varphi
     \to \psi )) \rangle\rangle$
  \item[] $\langle\varnothing,T,\varnothing,
     \langle\vdash (\exists x \,\varphi\leftrightarrow
     \lnot \forall x \lnot \varphi)\rangle\rangle$
\end{list}
The first three logical axioms (statements containing ``$\vdash$'') introduce
and effectively define logical equivalence, ``$\leftrightarrow$''.  The last
three use ``$\leftrightarrow$'' to effectively mean ``is defined as.''

\subsection{Example~5---ZFC Set Theory}\index{ZFC set theory}

Here we add to the system of Example~4 the axioms of Zermelo--Fraenkel set
theory with Choice.  For convenience we make use of the
definitions in Example~4.

In the $\mbox{\em CN}$ of Example~4 (which extends Example~3), we replace the symbol $R$
with the symbol $\in$.
More explicitly, we remove from $\Gamma$ of Example~4 the three
axiomatic statements containing $R$ and replace them with the
reducts of the following:
\begin{list}{}{\itemsep 0.0pt}
      \item[] $\langle\varnothing,T,\varnothing,
               \langle \mbox{wff\ }x\in y\rangle\rangle$
      \item[] $\langle\varnothing,T,
               \varnothing,
               \langle \vdash(x=y\to(x\in z\to y\in z))\rangle\rangle$
      \item[] $\langle\varnothing,T,
               \varnothing,
               \langle \vdash(x=y\to(z\in x\to z\in y))\rangle\rangle$
\end{list}
Letting $D=\{\{\alpha,\beta\}\in \mbox{\em DV}\,|\alpha,\beta\in \mbox{\em
Vr}\}$ (in other words all individual variables must be distinct), we extend
$\Gamma$ with the ZFC axioms, called
\index{Axiom of Extensionality}
\index{Axiom of Replacement}
\index{Axiom of Union}
\index{Axiom of Power Sets}
\index{Axiom of Regularity}
\index{Axiom of Infinity}
\index{Axiom of Choice}
Extensionality, Replacement, Union, Power
Set, Regularity, Infinity, and Choice, that are the reducts of:
\begin{list}{}{\itemsep 0.0pt}
      \item[Ext] $\langle D,T,
               \varnothing,
               \langle\vdash (\forall x(x\in y\leftrightarrow x \in z)\to y
               =z) \rangle\rangle$
      \item[Rep] $\langle D,T,
               \varnothing,
               \langle\vdash\exists x ( \exists y \forall z (\varphi \to z = y
                        ) \to
                        \forall z ( z \in x \leftrightarrow \exists x ( x \in
                        y \wedge \forall y\,\varphi ) ) )\rangle\rangle$
      \item[Un] $\langle D,T,
               \varnothing,
               \langle\vdash \exists x \forall y ( \exists x ( y \in x \wedge
               x \in z ) \to y \in x ) \rangle\rangle$
      \item[Pow] $\langle D,T,
               \varnothing,
               \langle\vdash \exists x \forall y ( \forall x ( x \in y \to x
               \in z ) \to y \in x ) \rangle\rangle$
      \item[Reg] $\langle D,T,
               \varnothing,
               \langle\vdash (  x \in y \to
                 \exists x ( x \in y \wedge \forall z ( z \in x \to \lnot z
                \in y ) ) ) \rangle\rangle$
      \item[Inf] $\langle D,T,
               \varnothing,
               \langle\vdash \exists x(y\in x\wedge\forall y(y\in
               x\to
               \exists z(y \in z\wedge z\in x))) \rangle\rangle$
      \item[AC] $\langle D,T,
               \varnothing,
               \langle\vdash \exists x \forall y \forall z ( ( y \in z
               \wedge z \in w ) \to \exists w \forall y ( \exists w
              ( ( y \in z \wedge z \in w ) \wedge ( y \in w \wedge w \in x
              ) ) \leftrightarrow y = w ) ) \rangle\rangle$
\end{list}

\subsection{Example~6---Class Notation in Set Theory}\label{class}

A powerful device that makes set theory easier (and that we have
been using all along in our informal expository language) is {\em class
abstraction notation}.\index{class abstraction}\index{abstraction class}  The
definitions we introduce are rigorously justified
as conservative by Takeuti and Zaring \cite{Takeuti}\index{Takeuti, Gaisi} or
Quine \cite{Quine}\index{Quine, Willard Van Orman}.  The key idea is to
introduce the notation $\{x|\mbox{---}\}$ which means ``the class of all $x$
such that ---'' for abstraction classes and introduce (meta)variables that
range over them.  An abstraction class may or may not be a set, depending on
whether it exists (as a set).  A class that does not exist is
called a {\em proper class}.\index{proper class}\index{class!proper}

To illustrate the use of abstraction classes we will provide some examples
of definitions that make use of them:  the empty set, class union, and
unordered pair.  Many other such definitions can be found in the
Metamath set theory database,
\texttt{set.mm}.\index{set theory database (\texttt{set.mm})}

% We intentionally break up the sequence of math symbols here
% because otherwise the overlong line goes beyond the page in narrow mode.
We extend $\mbox{\em CN}$ of Example~5 with new symbols $\{$
$\mbox{class},$ $\{,$ $|,$ $\},$ $\varnothing,$ $\cup,$ $,$ $\}$
where the inner braces and last comma are
constant symbols. (As before,
our dual use of some mathematical symbols for both our expository
language and as primitives of the formal system should be clear from context.)

We extend $\mbox{\em VR}$ of Example~5 with a set of {\em class
variables}\index{class variable}
$\{A,B,C,\ldots\}$. We extend the $T$ of Example~5 with $\{\langle
\mbox{class\ } A\rangle, \langle \mbox{class\ }B\rangle, \langle \mbox{class\ }
C\rangle,\ldots\}$.

To
introduce our definitions,
we add to $\Gamma$ of Example~5 the axiomatic statements
that are the reducts of the following pre-statements:
\begin{list}{}{\itemsep 0.0pt}
      \item[] $\langle\varnothing,T,\varnothing,
               \langle \mbox{class\ }x\rangle\rangle$
      \item[] $\langle\varnothing,T,\varnothing,
               \langle \mbox{class\ }\{x|\varphi\}\rangle\rangle$
      \item[] $\langle\varnothing,T,\varnothing,
               \langle \mbox{wff\ }A=B\rangle\rangle$
      \item[] $\langle\varnothing,T,\varnothing,
               \langle \mbox{wff\ }A\in B\rangle\rangle$
      \item[Ab] $\langle\varnothing,T,\varnothing,
               \langle \vdash ( y \in \{ x |\varphi\} \leftrightarrow
                  ( ( x = y \to\varphi) \wedge \exists x ( x = y
                  \wedge\varphi) ))
               \rangle\rangle$
      \item[Eq] $\langle\{\{x,A\},\{x,B\}\},T,\varnothing,
               \langle \vdash ( A = B \leftrightarrow
               \forall x ( x \in A \leftrightarrow x \in B ) )
               \rangle\rangle$
      \item[El] $\langle\{\{x,A\},\{x,B\}\},T,\varnothing,
               \langle \vdash ( A \in B \leftrightarrow \exists x
               ( x = A \wedge x \in B ) )
               \rangle\rangle$
\end{list}
Here we say that an individual variable is a class; $\{x|\varphi\}$ is a
class; and we extend the definition of a wff to include class equality and
membership.  Axiom Ab defines membership of a variable in a class abstraction;
the right-hand side can be read as ``the wff that results from proper
substitution of $y$ for $x$ in $\varphi$.''\footnote{Note that this definition
makes unnecessary the introduction of a separate notation similar to
$\varphi(x|y)$ for proper substitution, although we may choose to do so to be
conventional.  Incidentally, $\varphi(x|y)$ as it stands would be ambiguous in
the formal systems of our examples, since we wouldn't know whether
$\lnot\varphi(x|y)$ meant $\lnot(\varphi(x|y))$ or $(\lnot\varphi)(x|y)$.
Instead, we would have to use an unambiguous variant such as $(\varphi\,
x|y)$.}  Axioms Eq and El extend the meaning of the existing equality and
membership connectives.  This is potentially dangerous and requires careful
justification.  For example, from Eq we can derive the Axiom of Extensionality
with predicate logic alone; thus in principle we should include the Axiom of
Extensionality as a logical hypothesis.  However we do not bother to do this
since we have already presupposed that axiom earlier. The distinct variable
restrictions should be read ``where $x$ does not occur in $A$ or $B$.''  We
typically do this when the right-hand side of a definition involves an
individual variable not in the expression being defined; it is done so that
the right-hand side remains independent of the particular ``dummy'' variable
we use.

We continue to add to $\Gamma$ the following definitions
(i.e. the reducts of the following pre-statements) for empty
set,\index{empty set} class union,\index{union} and unordered
pair.\index{unordered pair}  They should be self-explanatory.  Analogous to our
use of ``$\leftrightarrow$'' to define new wffs in Example~4, we use ``$=$''
to define new abstraction terms, and both may be read informally as ``is
defined as'' in this context.
\begin{list}{}{\itemsep 0.0pt}
      \item[] $\langle\varnothing,T,\varnothing,
               \langle \mbox{class\ }\varnothing\rangle\rangle$
      \item[] $\langle\varnothing,T,\varnothing,
               \langle \vdash \varnothing = \{ x | \lnot x = x \}
               \rangle\rangle$
      \item[] $\langle\varnothing,T,\varnothing,
               \langle \mbox{class\ }(A\cup B)\rangle\rangle$
      \item[] $\langle\{\{x,A\},\{x,B\}\},T,\varnothing,
               \langle \vdash ( A \cup B ) = \{ x | ( x \in A \vee x \in B ) \}
               \rangle\rangle$
      \item[] $\langle\varnothing,T,\varnothing,
               \langle \mbox{class\ }\{A,B\}\rangle\rangle$
      \item[] $\langle\{\{x,A\},\{x,B\}\},T,\varnothing,
               \langle \vdash \{ A , B \} = \{ x | ( x = A \vee x = B ) \}
               \rangle\rangle$
\end{list}

\section{Metamath as a Formal System}\label{theorymm}

This section presupposes a familiarity with the Metamath computer language.

Our theory describes formal systems and their universes.  The Metamath
language provides a way of representing these set-theoretical objects to
a computer.  A Metamath database, being a finite set of {\sc ascii}
characters, can usually describe only a subset of a formal system and
its universe, which are typically infinite.  However the database can
contain as large a finite subset of the formal system and its universe
as we wish.  (Of course a Metamath set theory database can, in
principle, indirectly describe an entire infinite formal system by
formalizing the expository language in this Appendix.)

For purpose of our discussion, we assume the Metamath database
is in the simple form described on p.~\pageref{framelist},
consisting of all constant and variable declarations at the beginning,
followed by a sequence of extended frames each
delimited by \texttt{\$\char`\{} and \texttt{\$\char`\}}.  Any Metamath database can
be converted to this form, as described on p.~\pageref{frameconvert}.

The math symbol tokens of a Metamath source file, which are declared
with \texttt{\$c} and \texttt{\$v} statements, are names we assign to
representatives of $\mbox{\em CN}$ and $\mbox{\em VR}$.  For
definiteness we could assume that the first math symbol declared as a
variable corresponds to $v_0$, the second to $v_1$, etc., although the
exact correspondence we choose is not important.

In the Metamath language, each \texttt{\$d}, \texttt{\$f}, and
 \texttt{\$e} source
statement in an extended frame (Section~\ref{frames})
corresponds respectively to a member of the
collections $D$, $T$, and $H$ in a formal system statement $\langle
D_M,T_M,H,A\rangle$.  The math symbol strings following these Metamath keywords
correspond to a variable pair (in the case of \texttt{\$d}) or an expression (for
the other two keywords). The math symbol string following a \texttt{\$a} source
statement corresponds to expression $A$ in an axiomatic statement of the
formal system; the one following a \texttt{\$p} source statement corresponds to
$A$ in a provable statement that is not axiomatic.  In other words, each
extended frame in a Metamath database corresponds to
a pre-statement of the formal system, and a frame corresponds to
a statement of the formal system.  (Don't confuse the two meanings of
``statement'' here.  A statement of the formal system corresponds to the
several statements in a Metamath database that may constitute a
frame.)

In order for the computer to verify that a formal system statement is
provable, each \texttt{\$p} source statement is accompanied by a proof.
However, the proof does not correspond to anything in the formal system
but is simply a way of communicating to the computer the information
needed for its verification.  The proof tells the computer {\em how to
construct} specific members of closure of the formal system
pre-statement corresponding to the extended frame of the \texttt{\$p}
statement.  The final result of the construction is the member of the
closure that matches the \texttt{\$p} statement.  The abstract formal
system, on the other hand, is concerned only with the {\em existence} of
members of the closure.

As mentioned on p.~\pageref{exampleref}, Examples 1 and 3--6 in the
previous Section parallel the development of logic and set theory in the
Metamath database
\texttt{set.mm}.\index{set theory database (\texttt{set.mm})} You may
find it instructive to compare them.


\chapter{The MIU System}
\label{MIU}
\index{formal system}
\index{MIU-system}

The following is a listing of the file \texttt{miu.mm}.  It is self-explanatory.

%%%%%%%%%%%%%%%%%%%%%%%%%%%%%%%%%%%%%%%%%%%%%%%%%%%%%%%%%%%%

\begin{verbatim}
$( The MIU-system:  A simple formal system $)

$( Note:  This formal system is unusual in that it allows
empty wffs.  To work with a proof, you must type
SET EMPTY_SUBSTITUTION ON before using the PROVE command.
By default, this is OFF in order to reduce the number of
ambiguous unification possibilities that have to be selected
during the construction of a proof.  $)

$(
Hofstadter's MIU-system is a simple example of a formal
system that illustrates some concepts of Metamath.  See
Douglas R. Hofstadter, _Goedel, Escher, Bach:  An Eternal
Golden Braid_ (Vintage Books, New York, 1979), pp. 33ff. for
a description of the MIU-system.

The system has 3 constant symbols, M, I, and U.  The sole
axiom of the system is MI. There are 4 rules:
     Rule I:  If you possess a string whose last letter is I,
     you can add on a U at the end.
     Rule II:  Suppose you have Mx.  Then you may add Mxx to
     your collection.
     Rule III:  If III occurs in one of the strings in your
     collection, you may make a new string with U in place
     of III.
     Rule IV:  If UU occurs inside one of your strings, you
     can drop it.
Unfortunately, Rules III and IV do not have unique results:
strings could have more than one occurrence of III or UU.
This requires that we introduce the concept of an "MIU
well-formed formula" or wff, which allows us to construct
unique symbol sequences to which Rules III and IV can be
applied.
$)

$( First, we declare the constant symbols of the language.
Note that we need two symbols to distinguish the assertion
that a sequence is a wff from the assertion that it is a
theorem; we have arbitrarily chosen "wff" and "|-". $)
      $c M I U |- wff $. $( Declare constants $)

$( Next, we declare some variables. $)
     $v x y $.

$( Throughout our theory, we shall assume that these
variables represent wffs. $)
 wx   $f wff x $.
 wy   $f wff y $.

$( Define MIU-wffs.  We allow the empty sequence to be a
wff. $)

$( The empty sequence is a wff. $)
 we   $a wff $.
$( "M" after any wff is a wff. $)
 wM   $a wff x M $.
$( "I" after any wff is a wff. $)
 wI   $a wff x I $.
$( "U" after any wff is a wff. $)
 wU   $a wff x U $.

$( Assert the axiom. $)
 ax   $a |- M I $.

$( Assert the rules. $)
 ${
   Ia   $e |- x I $.
$( Given any theorem ending with "I", it remains a theorem
if "U" is added after it.  (We distinguish the label I_
from the math symbol I to conform to the 24-Jun-2006
Metamath spec.) $)
   I_    $a |- x I U $.
 $}
 ${
IIa  $e |- M x $.
$( Given any theorem starting with "M", it remains a theorem
if the part after the "M" is added again after it. $)
   II   $a |- M x x $.
 $}
 ${
   IIIa $e |- x I I I y $.
$( Given any theorem with "III" in the middle, it remains a
theorem if the "III" is replaced with "U". $)
   III  $a |- x U y $.
 $}
 ${
   IVa  $e |- x U U y $.
$( Given any theorem with "UU" in the middle, it remains a
theorem if the "UU" is deleted. $)
   IV   $a |- x y $.
  $}

$( Now we prove the theorem MUIIU.  You may be interested in
comparing this proof with that of Hofstadter (pp. 35 - 36).
$)
 theorem1  $p |- M U I I U $=
      we wM wU wI we wI wU we wU wI wU we wM we wI wU we wM
      wI wI wI we wI wI we wI ax II II I_ III II IV $.
\end{verbatim}\index{well-formed formula (wff)}

The \texttt{show proof /lemmon/renumber} command
yields the following display.  It is very similar
to the one in \cite[pp.~35--36]{Hofstadter}.\index{Hofstadter, Douglas R.}

\begin{verbatim}
1 ax             $a |- M I
2 1 II           $a |- M I I
3 2 II           $a |- M I I I I
4 3 I_           $a |- M I I I I U
5 4 III          $a |- M U I U
6 5 II           $a |- M U I U U I U
7 6 IV           $a |- M U I I U
\end{verbatim}

We note that Hofstadter's ``MU-puzzle,'' which asks whether
MU is a theorem of the MIU-system, cannot be answered using
the system above because the MU-puzzle is a question {\em
about} the system.  To prove the answer to the MU-puzzle,
a much more elaborate system is needed, namely one that
models the MIU-system within set theory.  (Incidentally, the
answer to the MU-puzzle is no.)

\chapter{Metamath Language EBNF}%
\label{BNF}%
\index{Metamath Language EBNF}

The following is a formal description of the basic Metamath language syntax
(with compressed proofs and support for unknown proof steps).
It is defined using the
Extended Backus--Naur Form (EBNF)\index{Extended Backus--Naur Form}\index{EBNF}
notation from W3C\index{W3C}
\textit{Extensible Markup Language (XML) 1.0 (Fifth Edition)}
(W3C Recommendation 26 November 2008) at
\url{https://www.w3.org/TR/xml/#sec-notation}.

The \texttt{database}
rule is processed until the end of the file (\texttt{EOF}).
The rules eventually require reading whitespace-separated tokens.
A token has an upper-case definition (see below)
or is a string constant in a non-token (such as \texttt{'\$a'}).
We intend for this to be correct, but if there is a conflict the
rules of section \ref{spec} govern. That section also discusses
non-syntax restrictions not shown here
(e.g., that each new label token
defined in a \texttt{hypothesis-stmt} or \texttt{assert-stmt}
must be unique).

\begin{verbatim}
database ::= outermost-scope-stmt*

outermost-scope-stmt ::=
  include-stmt | constant-stmt | stmt

/* File inclusion command; process file as a database.
   Databases should NOT have a comment in the filename. */
include-stmt ::= '$[' filename '$]'

/* Constant symbols declaration. */
constant-stmt ::= '$c' constant+ '$.'

/* A normal statement can occur in any scope. */
stmt ::= block | variable-stmt | disjoint-stmt |
  hypothesis-stmt | assert-stmt

/* A block. You can have 0 statements in a block. */
block ::= '${' stmt* '$}'

/* Variable symbols declaration. */
variable-stmt ::= '$v' variable+ '$.'

/* Disjoint variables. Simple disjoint statements have
   2 variables, i.e., "variable*" is empty for them. */
disjoint-stmt ::= '$d' variable variable variable* '$.'

hypothesis-stmt ::= floating-stmt | essential-stmt

/* Floating (variable-type) hypothesis. */
floating-stmt ::= LABEL '$f' typecode variable '$.'

/* Essential (logical) hypothesis. */
essential-stmt ::= LABEL '$e' typecode MATH-SYMBOL* '$.'

assert-stmt ::= axiom-stmt | provable-stmt

/* Axiomatic assertion. */
axiom-stmt ::= LABEL '$a' typecode MATH-SYMBOL* '$.'

/* Provable assertion. */
provable-stmt ::= LABEL '$p' typecode MATH-SYMBOL*
  '$=' proof '$.'

/* A proof. Proofs may be interspersed by comments.
   If '?' is in a proof it's an "incomplete" proof. */
proof ::= uncompressed-proof | compressed-proof
uncompressed-proof ::= (LABEL | '?')+
compressed-proof ::= '(' LABEL* ')' COMPRESSED-PROOF-BLOCK+

typecode ::= constant

filename ::= MATH-SYMBOL /* No whitespace or '$' */
constant ::= MATH-SYMBOL
variable ::= MATH-SYMBOL
\end{verbatim}

\needspace{2\baselineskip}
A \texttt{frame} is a sequence of 0 or more
\texttt{disjoint-{\allowbreak}stmt} and
\texttt{hypotheses-{\allowbreak}stmt} statements
(possibly interleaved with other non-\texttt{assert-stmt} statements)
followed by one \texttt{assert-stmt}.

\needspace{3\baselineskip}
Here are the rules for lexical processing (tokenization) beyond
the constant tokens shown above.
By convention these tokenization rules have upper-case names.
Every token is read for the longest possible length.
Whitespace-separated tokens are read sequentially;
note that the separating whitespace and \texttt{\$(} ... \texttt{\$)}
comments are skipped.

If a token definition uses another token definition, the whole thing
is considered a single token.
A pattern that is only part of a full token has a name beginning
with an underscore (``\_'').
An implementation could tokenize many tokens as a
\texttt{PRINTABLE-SEQUENCE}
and then check if it meets the more specific rule shown here.

Comments do not nest, and both \texttt{\$(} and \texttt{\$)}
have to be surrounded
by at least one whitespace character (\texttt{\_WHITECHAR}).
Technically comments end without consuming the trailing
\texttt{\_WHITECHAR}, but the trailing
\texttt{\_WHITECHAR} gets ignored anyway so we ignore that detail here.
Metamath language processors
are not required to support \texttt{\$)} followed
immediately by a bare end-of-file, because the closing
comment symbol is supposed to be followed by a
\texttt{\_WHITECHAR} such as a newline.

\begin{verbatim}
PRINTABLE-SEQUENCE ::= _PRINTABLE-CHARACTER+

MATH-SYMBOL ::= (_PRINTABLE-CHARACTER - '$')+

/* ASCII non-whitespace printable characters */
_PRINTABLE-CHARACTER ::= [#x21-#x7e]

LABEL ::= ( _LETTER-OR-DIGIT | '.' | '-' | '_' )+

_LETTER-OR-DIGIT ::= [A-Za-z0-9]

COMPRESSED-PROOF-BLOCK ::= ([A-Z] | '?')+

/* Define whitespace between tokens. The -> SKIP
   means that when whitespace is seen, it is
   skipped and we simply read again. */
WHITESPACE ::= (_WHITECHAR+ | _COMMENT) -> SKIP

/* Comments. $( ... $) and do not nest. */
_COMMENT ::= '$(' (_WHITECHAR+ (PRINTABLE-SEQUENCE - '$)'))*
  _WHITECHAR+ '$)' _WHITECHAR

/* Whitespace: (' ' | '\t' | '\r' | '\n' | '\f') */
_WHITECHAR ::= [#x20#x09#x0d#x0a#x0c]
\end{verbatim}
% This EBNF was developed as a collaboration between
% David A. Wheeler\index{Wheeler, David A.},
% Mario Carneiro\index{Carneiro, Mario}, and
% Benoit Jubin\index{Jubin, Benoit}, inspired by a request
% (and a lot of initial work) by Benoit Jubin.
%
% \chapter{Disclaimer and Trademarks}
%
% Information in this document is subject to change without notice and does not
% represent a commitment on the part of Norman Megill.
% \vspace{2ex}
%
% \noindent Norman D. Megill makes no warranties, either express or implied,
% regarding the Metamath computer software package.
%
% \vspace{2ex}
%
% \noindent Any trademarks mentioned in this book are the property of
% their respective owners.  The name ``Metamath'' is a trademark of
% Norman Megill.
%
\cleardoublepage
\phantomsection  % fixes the link anchor
\addcontentsline{toc}{chapter}{\bibname}

\bibliography{metamath}
%% metamath.tex - Version of 2-Jun-2019
% If you change the date above, also change the "Printed date" below.
% SPDX-License-Identifier: CC0-1.0
%
%                              PUBLIC DOMAIN
%
% This file (specifically, the version of this file with the above date)
% has been released into the Public Domain per the
% Creative Commons CC0 1.0 Universal (CC0 1.0) Public Domain Dedication
% https://creativecommons.org/publicdomain/zero/1.0/
%
% The public domain release applies worldwide.  In case this is not
% legally possible, the right is granted to use the work for any purpose,
% without any conditions, unless such conditions are required by law.
%
% Several short, attributed quotations from copyrighted works
% appear in this file under the ``fair use'' provision of Section 107 of
% the United States Copyright Act (Title 17 of the {\em United States
% Code}).  The public-domain status of this file is not applicable to
% those quotations.
%
% Norman Megill - email: nm(at)alum(dot)mit(dot)edu
%
% David A. Wheeler also donates his improvements to this file to the
% public domain per the CC0.  He works at the Institute for Defense Analyses
% (IDA), but IDA has agreed that this Metamath work is outside its "lane"
% and is not a work by IDA.  This was specifically confirmed by
% Margaret E. Myers (Division Director of the Information Technology
% and Systems Division) on 2019-05-24 and by Ben Lindorf (General Counsel)
% on 2019-05-22.

% This file, 'metamath.tex', is self-contained with everything needed to
% generate the the PDF file 'metamath.pdf' (the _Metamath_ book) on
% standard LaTeX 2e installations.  The auxiliary files are embedded with
% "filecontents" commands.  To generate metamath.pdf file, run these
% commands under Linux or Cygwin in the directory that contains
% 'metamath.tex':
%
%   rm -f realref.sty metamath.bib
%   touch metamath.ind
%   pdflatex metamath
%   pdflatex metamath
%   bibtex metamath
%   makeindex metamath
%   pdflatex metamath
%   pdflatex metamath
%
% The warnings that occur in the initial runs of pdflatex can be ignored.
% For the final run,
%
%   egrep -i 'error|warn' metamath.log
%
% should show exactly these 5 warnings:
%
%   LaTeX Warning: File `realref.sty' already exists on the system.
%   LaTeX Warning: File `metamath.bib' already exists on the system.
%   LaTeX Font Warning: Font shape `OMS/cmtt/m/n' undefined
%   LaTeX Font Warning: Font shape `OMS/cmtt/bx/n' undefined
%   LaTeX Font Warning: Some font shapes were not available, defaults
%       substituted.
%
% Search for "Uncomment" below if you want to suppress hyperlink boxes
% in the PDF output file
%
% TYPOGRAPHICAL NOTES:
% * It is customary to use an en dash (--) to "connect" names of different
%   people (and to denote ranges), and use a hyphen (-) for a
%   single compound name. Examples of connected multiple people are
%   Zermelo--Fraenkel, Schr\"{o}der--Bernstein, Tarski--Grothendieck,
%   Hewlett--Packard, and Backus--Naur.  Examples of a single person with
%   a compound name include Levi-Civita, Mittag-Leffler, and Burali-Forti.
% * Use non-breaking spaces after page abbreviations, e.g.,
%   p.~\pageref{note2002}.
%
% --------------------------- Start of realref.sty -----------------------------
\begin{filecontents}{realref.sty}
% Save the following as realref.sty.
% You can then use it with \usepackage{realref}
%
% This has \pageref jumping to the page on which the ref appears,
% \ref jumping to the point of the anchor, and \sectionref
% jumping to the start of section.
%
% Author:  Anthony Williams
%          Software Engineer
%          Nortel Networks Optical Components Ltd
% Date:    9 Nov 2001 (posted to comp.text.tex)
%
% The following declaration was made by Anthony Williams on
% 24 Jul 2006 (private email to Norman Megill):
%
%   ``I hereby donate the code for realref.sty posted on the
%   comp.text.tex newsgroup on 9th November 2001, accessible from
%   http://groups.google.com/group/comp.text.tex/msg/5a0e1cc13ea7fbb2
%   to the public domain.''
%
\ProvidesPackage{realref}
\RequirePackage[plainpages=false,pdfpagelabels=true]{hyperref}
\def\realref@anchorname{}
\AtBeginDocument{%
% ensure every label is a possible hyperlink target
\let\realref@oldrefstepcounter\refstepcounter%
\DeclareRobustCommand{\refstepcounter}[1]{\realref@oldrefstepcounter{#1}
\edef\realref@anchorname{\string #1.\@currentlabel}%
}%
\let\realref@oldlabel\label%
\DeclareRobustCommand{\label}[1]{\realref@oldlabel{#1}\hypertarget{#1}{}%
\@bsphack\protected@write\@auxout{}{%
    \string\expandafter\gdef\protect\csname
    page@num.#1\string\endcsname{\thepage}%
    \string\expandafter\gdef\protect\csname
    ref@num.#1\string\endcsname{\@currentlabel}%
    \string\expandafter\gdef\protect\csname
    sectionref@name.#1\string\endcsname{\realref@anchorname}%
}\@esphack}%
\DeclareRobustCommand\pageref[1]{{\edef\a{\csname
            page@num.#1\endcsname}\expandafter\hyperlink{page.\a}{\a}}}%
\DeclareRobustCommand\ref[1]{{\edef\a{\csname
            ref@num.#1\endcsname}\hyperlink{#1}{\a}}}%
\DeclareRobustCommand\sectionref[1]{{\edef\a{\csname
            ref@num.#1\endcsname}\edef\b{\csname
            sectionref@name.#1\endcsname}\hyperlink{\b}{\a}}}%
}
\end{filecontents}
% ---------------------------- End of realref.sty ------------------------------

% --------------------------- Start of metamath.bib -----------------------------
\begin{filecontents}{metamath.bib}
@book{Albers, editor = "Donald J. Albers and G. L. Alexanderson",
  title = "Mathematical People",
  publisher = "Contemporary Books, Inc.",
  address = "Chicago",
  note = "[QA28.M37]",
  year = 1985 }
@book{Anderson, author = "Alan Ross Anderson and Nuel D. Belnap",
  title = "Entailment",
  publisher = "Princeton University Press",
  address = "Princeton",
  volume = 1,
  note = "[QA9.A634 1975 v.1]",
  year = 1975}
@book{Barrow, author = "John D. Barrow",
  title = "Theories of Everything:  The Quest for Ultimate Explanation",
  publisher = "Oxford University Press",
  address = "Oxford",
  note = "[Q175.B225]",
  year = 1991 }
@book{Behnke,
  editor = "H. Behnke and F. Backmann and K. Fladt and W. S{\"{u}}ss",
  title = "Fundamentals of Mathematics",
  volume = "I",
  publisher = "The MIT Press",
  address = "Cambridge, Massachusetts",
  note = "[QA37.2.B413]",
  year = 1974 }
@book{Bell, author = "J. L. Bell and M. Machover",
  title = "A Course in Mathematical Logic",
  publisher = "North-Holland",
  address = "Amsterdam",
  note = "[QA9.B3953]",
  year = 1977 }
@inproceedings{Blass, author = "Andrea Blass",
  title = "The Interaction Between Category Theory and Set Theory",
  pages = "5--29",
  booktitle = "Mathematical Applications of Category Theory (Proceedings
     of the Special Session on Mathematical Applications
     Category Theory, 89th Annual Meeting of the American Mathematical
     Society, held in Denver, Colorado January 5--9, 1983)",
  editor = "John Walter Gray",
  year = 1983,
  note = "[QA169.A47 1983]",
  publisher = "American Mathematical Society",
  address = "Providence, Rhode Island"}
@proceedings{Bledsoe, editor = "W. W. Bledsoe and D. W. Loveland",
  title = "Automated Theorem Proving:  After 25 Years (Proceedings
     of the Special Session on Automatic Theorem Proving,
     89th Annual Meeting of the American Mathematical
     Society, held in Denver, Colorado January 5--9, 1983)",
  year = 1983,
  note = "[QA76.9.A96.S64 1983]",
  publisher = "American Mathematical Society",
  address = "Providence, Rhode Island" }
@book{Boolos, author = "George S. Boolos and Richard C. Jeffrey",
  title = "Computability and Log\-ic",
  publisher = "Cambridge University Press",
  edition = "third",
  address = "Cambridge",
  note = "[QA9.59.B66 1989]",
  year = 1989 }
@book{Campbell, author = "John Campbell",
  title = "Programmer's Progress",
  publisher = "White Star Software",
  address = "Box 51623, Palo Alto, CA 94303",
  year = 1991 }
@article{DBLP:journals/corr/Carneiro14,
  author    = {Mario Carneiro},
  title     = {Conversion of {HOL} Light proofs into Metamath},
  journal   = {CoRR},
  volume    = {abs/1412.8091},
  year      = {2014},
  url       = {http://arxiv.org/abs/1412.8091},
  archivePrefix = {arXiv},
  eprint    = {1412.8091},
  timestamp = {Mon, 13 Aug 2018 16:47:05 +0200},
  biburl    = {https://dblp.org/rec/bib/journals/corr/Carneiro14},
  bibsource = {dblp computer science bibliography, https://dblp.org}
}
@article{CarneiroND,
  author    = {Mario Carneiro},
  title     = {Natural Deductions in the Metamath Proof Language},
  url       = {http://us.metamath.org/ocat/natded.pdf},
  year      = 2014
}
@inproceedings{Chou, author = "Shang-Ching Chou",
  title = "Proving Elementary Geometry Theorems Using {W}u's Algorithm",
  pages = "243--286",
  booktitle = "Automated Theorem Proving:  After 25 Years (Proceedings
     of the Special Session on Automatic Theorem Proving,
     89th Annual Meeting of the American Mathematical
     Society, held in Denver, Colorado January 5--9, 1983)",
  editor = "W. W. Bledsoe and D. W. Loveland",
  year = 1983,
  note = "[QA76.9.A96.S64 1983]",
  publisher = "American Mathematical Society",
  address = "Providence, Rhode Island" }
@book{Clemente, author = "Daniel Clemente Laboreo",
  title = "Introduction to natural deduction",
  year = 2014,
  url = "http://www.danielclemente.com/logica/dn.en.pdf" }
@incollection{Courant, author = "Richard Courant and Herbert Robbins",
  title = "Topology",
  pages = "573--590",
  booktitle = "The World of Mathematics, Volume One",
  editor = "James R. Newman",
  publisher = "Simon and Schuster",
  address = "New York",
  note = "[QA3.W67 1988]",
  year = 1956 }
@book{Curry, author = "Haskell B. Curry",
  title = "Foundations of Mathematical Logic",
  publisher = "Dover Publications, Inc.",
  address = "New York",
  note = "[QA9.C976 1977]",
  year = 1977 }
@book{Davis, author = "Philip J. Davis and Reuben Hersh",
  title = "The Mathematical Experience",
  publisher = "Birkh{\"{a}}user Boston",
  address = "Boston",
  note = "[QA8.4.D37 1982]",
  year = 1981 }
@incollection{deMillo,
  author = "Richard de Millo and Richard Lipton and Alan Perlis",
  title = "Social Processes and Proofs of Theorems and Programs",
  pages = "267--285",
  booktitle = "New Directions in the Philosophy of Mathematics",
  editor = "Thomas Tymoczko",
  publisher = "Birkh{\"{a}}user Boston, Inc.",
  address = "Boston",
  note = "[QA8.6.N48 1986]",
  year = 1986 }
@book{Edwards, author = "Robert E. Edwards",
  title = "A Formal Background to Mathematics",
  publisher = "Springer-Verlag",
  address = "New York",
  note = "[QA37.2.E38 v.1a]",
  year = 1979 }
@book{Enderton, author = "Herbert B. Enderton",
  title = "Elements of Set Theory",
  publisher = "Academic Press, Inc.",
  address = "San Diego",
  note = "[QA248.E5]",
  year = 1977 }
@book{Goodstein, author = "R. L. Goodstein",
  title = "Development of Mathematical Logic",
  publisher = "Springer-Verlag New York Inc.",
  address = "New York",
  note = "[QA9.G6554]",
  year = 1971 }
@book{Guillen, author = "Michael Guillen",
  title = "Bridges to Infinity",
  publisher = "Jeremy P. Tarcher, Inc.",
  address = "Los Angeles",
  note = "[QA93.G8]",
  year = 1983 }
@book{Hamilton, author = "Alan G. Hamilton",
  title = "Logic for Mathematicians",
  edition = "revised",
  publisher = "Cambridge University Press",
  address = "Cambridge",
  note = "[QA9.H298]",
  year = 1988 }
@unpublished{Harrison, author = "John Robert Harrison",
  title = "Metatheory and Reflection in Theorem Proving:
    A Survey and Critique",
  note = "Technical Report
    CRC-053.
    SRI Cambridge,
    Millers Yard, Cambridge, UK,
    1995.
    Available on the Web as
{\verb+http:+}\-{\verb+//www.cl.cam.ac.uk/users/jrh/papers/reflect.html+}"}
@TECHREPORT{Harrison-thesis,
        author          = "John Robert Harrison",
        title           = "Theorem Proving with the Real Numbers",
        institution   = "University of Cambridge Computer
                         Lab\-o\-ra\-to\-ry",
        address         = "New Museums Site, Pembroke Street, Cambridge,
                           CB2 3QG, UK",
        year            = 1996,
        number          = 408,
        type            = "Technical Report",
        note            = "Author's PhD thesis,
   available on the Web at
{\verb+http:+}\-{\verb+//www.cl.cam.ac.uk+}\-{\verb+/users+}\-{\verb+/jrh+}%
\-{\verb+/papers+}\-{\verb+/thesis.html+}"}
@book{Herrlich, author = "Horst Herrlich and George E. Strecker",
  title = "Category Theory:  An Introduction",
  publisher = "Allyn and Bacon Inc.",
  address = "Boston",
  note = "[QA169.H567]",
  year = 1973 }
@article{Hindley, author = "J. Roger Hindley and David Meredith",
  title = "Principal Type-Schemes and Condensed Detachment",
  journal = "The Journal of Symbolic Logic",
  volume = 55,
  year = 1990,
  note = "[QA.J87]",
  pages = "90--105" }
@book{Hofstadter, author = "Douglas R. Hofstadter",
  title = "G{\"{o}}del, Escher, Bach",
  publisher = "Basic Books, Inc.",
  address = "New York",
  note = "[QA9.H63 1980]",
  year = 1979 }
@article{Indrzejczak, author= "Andrzej Indrzejczak",
  title = "Natural Deduction, Hybrid Systems and Modal Logic",
  journal = "Trends in Logic",
  volume = 30,
  publisher = "Springer",
  year = 2010 }
@article{Kalish, author = "D. Kalish and R. Montague",
  title = "On {T}arski's Formalization of Predicate Logic with Identity",
  journal = "Archiv f{\"{u}}r Mathematische Logik und Grundlagenfor\-schung",
  volume = 7,
  year = 1965,
  note = "[QA.A673]",
  pages = "81--101" }
@article{Kalman, author = "J. A. Kalman",
  title = "Condensed Detachment as a Rule of Inference",
  journal = "Studia Logica",
  volume = 42,
  number = 4,
  year = 1983,
  note = "[B18.P6.S933]",
  pages = "443-451" }
@book{Kline, author = "Morris Kline",
  title = "Mathematical Thought from Ancient to Modern Times",
  publisher = "Oxford University Press",
  address = "New York",
  note = "[QA21.K516 1990 v.3]",
  year = 1972 }
@book{Klinel, author = "Morris Kline",
  title = "Mathematics, The Loss of Certainty",
  publisher = "Oxford University Press",
  address = "New York",
  note = "[QA21.K525]",
  year = 1980 }
@book{Kramer, author = "Edna E. Kramer",
  title = "The Nature and Growth of Modern Mathematics",
  publisher = "Princeton University Press",
  address = "Princeton, New Jersey",
  note = "[QA93.K89 1981]",
  year = 1981 }
@article{Knill, author = "Oliver Knill",
  title = "Some Fundamental Theorems in Mathematics",
  year = "2018",
  url = "https://arxiv.org/abs/1807.08416" }
@book{Landau, author = "Edmund Landau",
  title = "Foundations of Analysis",
  publisher = "Chelsea Publishing Company",
  address = "New York",
  edition = "second",
  note = "[QA241.L2541 1960]",
  year = 1960 }
@article{Leblanc, author = "Hugues Leblanc",
  title = "On {M}eyer and {L}ambert's Quantificational Calculus {FQ}",
  journal = "The Journal of Symbolic Logic",
  volume = 33,
  year = 1968,
  note = "[QA.J87]",
  pages = "275--280" }
@article{Lejewski, author = "Czeslaw Lejewski",
  title = "On Implicational Definitions",
  journal = "Studia Logica",
  volume = 8,
  year = 1958,
  note = "[B18.P6.S933]",
  pages = "189--208" }
@book{Levy, author = "Azriel Levy",
  title = "Basic Set Theory",
  publisher = "Dover Publications",
  address = "Mineola, NY",
  year = "2002"
}
@book{Margaris, author = "Angelo Margaris",
  title = "First Order Mathematical Logic",
  publisher = "Blaisdell Publishing Company",
  address = "Waltham, Massachusetts",
  note = "[QA9.M327]",
  year = 1967}
@book{Manin, author = "Yu I. Manin",
  title = "A Course in Mathematical Logic",
  publisher = "Springer-Verlag",
  address = "New York",
  note = "[QA9.M29613]",
  year = "1977" }
@article{Mathias, author = "Adrian R. D. Mathias",
  title = "A Term of Length 4,523,659,424,929",
  journal = "Synthese",
  volume = 133,
  year = 2002,
  note = "[Q.S993]",
  pages = "75--86" }
@article{Megill, author = "Norman D. Megill",
  title = "A Finitely Axiomatized Formalization of Predicate Calculus
     with Equality",
  journal = "Notre Dame Journal of Formal Logic",
  volume = 36,
  year = 1995,
  note = "[QA.N914]",
  pages = "435--453" }
@unpublished{Megillc, author = "Norman D. Megill",
  title = "A Shorter Equivalent of the Axiom of Choice",
  month = "June",
  note = "Unpublished",
  year = 1991 }
@article{MegillBunder, author = "Norman D. Megill and Martin W.
    Bunder",
  title = "Weaker {D}-Complete Logics",
  journal = "Journal of the IGPL",
  volume = 4,
  year = 1996,
  pages = "215--225",
  note = "Available on the Web at
{\verb+http:+}\-{\verb+//www.mpi-sb.mpg.de+}\-{\verb+/igpl+}%
\-{\verb+/Journal+}\-{\verb+/V4-2+}\-{\verb+/#Megill+}"}
}
@book{Mendelson, author = "Elliott Mendelson",
  title = "Introduction to Mathematical Logic",
  edition = "second",
  publisher = "D. Van Nostrand Company, Inc.",
  address = "New York",
  note = "[QA9.M537 1979]",
  year = 1979 }
@article{Meredith, author = "David Meredith",
  title = "In Memoriam {C}arew {A}rthur {M}eredith (1904-1976)",
  journal = "Notre Dame Journal of Formal Logic",
  volume = 18,
  year = 1977,
  note = "[QA.N914]",
  pages = "513--516" }
@article{CAMeredith, author = "C. A. Meredith",
  title = "Single Axioms for the Systems ({C},{N}), ({C},{O}) and ({A},{N})
      of the Two-Valued Propositional Calculus",
  journal = "The Journal of Computing Systems",
  volume = 3,
  year = 1953,
  pages = "155--164" }
@article{Monk, author = "J. Donald Monk",
  title = "Provability With Finitely Many Variables",
  journal = "The Journal of Symbolic Logic",
  volume = 27,
  year = 1971,
  note = "[QA.J87]",
  pages = "353--358" }
@article{Monks, author = "J. Donald Monk",
  title = "Substitutionless Predicate Logic With Identity",
  journal = "Archiv f{\"{u}}r Mathematische Logik und Grundlagenfor\-schung",
  volume = 7,
  year = 1965,
  pages = "103--121" }
  %% Took out this from above to prevent LaTeX underfull warning:
  % note = "[QA.A673]",
@book{Moore, author = "A. W. Moore",
  title = "The Infinite",
  publisher = "Routledge",
  address = "New York",
  note = "[BD411.M59]",
  year = 1989}
@book{Munkres, author = "James R. Munkres",
  title = "Topology: A First Course",
  publisher = "Prentice-Hall, Inc.",
  address = "Englewood Cliffs, New Jersey",
  note = "[QA611.M82]",
  year = 1975}
@article{Nemesszeghy, author = "E. Z. Nemesszeghy and E. A. Nemesszeghy",
  title = "On Strongly Creative Definitions:  A Reply to {V}. {F}. {R}ickey",
  journal = "Logique et Analyse (N.\ S.)",
  year = 1977,
  volume = 20,
  note = "[BC.L832]",
  pages = "111--115" }
@unpublished{Nemeti, author = "N{\'{e}}meti, I.",
  title = "Algebraizations of Quantifier Logics, an Overview",
  note = "Version 11.4, preprint, Mathematical Institute, Budapest,
    1994.  A shortened version without proofs appeared in
    ``Algebraizations of quantifier logics, an introductory overview,''
   {\em Studia Logica}, 50:485--569, 1991 [B18.P6.S933]"}
@article{Pavicic, author = "M. Pavi{\v{c}}i{\'{c}}",
  title = "A New Axiomatization of Unified Quantum Logic",
  journal = "International Journal of Theoretical Physics",
  year = 1992,
  volume = 31,
  note = "[QC.I626]",
  pages = "1753 --1766" }
@book{Penrose, author = "Roger Penrose",
  title = "The Emperor's New Mind",
  publisher = "Oxford University Press",
  address = "New York",
  note = "[Q335.P415]",
  year = 1989 }
@book{PetersonI, author = "Ivars Peterson",
  title = "The Mathematical Tourist",
  publisher = "W. H. Freeman and Company",
  address = "New York",
  note = "[QA93.P475]",
  year = 1988 }
@article{Peterson, author = "Jeremy George Peterson",
  title = "An automatic theorem prover for substitution and detachment systems",
  journal = "Notre Dame Journal of Formal Logic",
  volume = 19,
  year = 1978,
  note = "[QA.N914]",
  pages = "119--122" }
@book{Quine, author = "Willard Van Orman Quine",
  title = "Set Theory and Its Logic",
  edition = "revised",
  publisher = "The Belknap Press of Harvard University Press",
  address = "Cambridge, Massachusetts",
  note = "[QA248.Q7 1969]",
  year = 1969 }
@article{Robinson, author = "J. A. Robinson",
  title = "A Machine-Oriented Logic Based on the Resolution Principle",
  journal = "Journal of the Association for Computing Machinery",
  year = 1965,
  volume = 12,
  pages = "23--41" }
@article{RobinsonT, author = "T. Thacher Robinson",
  title = "Independence of Two Nice Sets of Axioms for the Propositional
    Calculus",
  journal = "The Journal of Symbolic Logic",
  volume = 33,
  year = 1968,
  note = "[QA.J87]",
  pages = "265--270" }
@book{Rucker, author = "Rudy Rucker",
  title = "Infinity and the Mind:  The Science and Philosophy of the
    Infinite",
  publisher = "Bantam Books, Inc.",
  address = "New York",
  note = "[QA9.R79 1982]",
  year = 1982 }
@book{Russell, author = "Bertrand Russell",
  title = "Mysticism and Logic, and Other Essays",
  publisher = "Barnes \& Noble Books",
  address = "Totowa, New Jersey",
  note = "[B1649.R963.M9 1981]",
  year = 1981 }
@article{Russell2, author = "Bertrand Russell",
  title = "Recent Work on the Principles of Mathematics",
  journal = "International Monthly",
  volume = 4,
  year = 1901,
  pages = "84"}
@article{Schmidt, author = "Eric Schmidt",
  title = "Reductions in Norman Megill's axiom system for complex numbers",
  url = "http://us.metamath.org/downloads/schmidt-cnaxioms.pdf",
  year = "2012" }
@book{Shoenfield, author = "Joseph R. Shoenfield",
  title = "Mathematical Logic",
  publisher = "Addison-Wesley Publishing Company, Inc.",
  address = "Reading, Massachusetts",
  year = 1967,
  note = "[QA9.S52]" }
@book{Smullyan, author = "Raymond M. Smullyan",
  title = "Theory of Formal Systems",
  publisher = "Princeton University Press",
  address = "Princeton, New Jersey",
  year = 1961,
  note = "[QA248.5.S55]" }
@book{Solow, author = "Daniel Solow",
  title = "How to Read and Do Proofs:  An Introduction to Mathematical
    Thought Process",
  publisher = "John Wiley \& Sons",
  address = "New York",
  year = 1982,
  note = "[QA9.S577]" }
@book{Stark, author = "Harold M. Stark",
  title = "An Introduction to Number Theory",
  publisher = "Markham Publishing Company",
  address = "Chicago",
  note = "[QA241.S72 1978]",
  year = 1970 }
@article{Swart, author = "E. R. Swart",
  title = "The Philosophical Implications of the Four-Color Problem",
  journal = "American Mathematical Monthly",
  year = 1980,
  volume = 87,
  month = "November",
  note = "[QA.A5125]",
  pages = "697--707" }
@book{Szpiro, author = "George G. Szpiro",
  title = "Poincar{\'{e}}'s Prize: The Hundred-Year Quest to Solve One
    of Math's Greatest Puzzles",
  publisher = "Penguin Books Ltd",
  address = "London",
  note = "[QA43.S985 2007]",
  year = 2007}
@book{Takeuti, author = "Gaisi Takeuti and Wilson M. Zaring",
  title = "Introduction to Axiomatic Set Theory",
  edition = "second",
  publisher = "Springer-Verlag New York Inc.",
  address = "New York",
  note = "[QA248.T136 1982]",
  year = 1982}
@inproceedings{Tarski, author = "Alfred Tarski",
  title = "What is Elementary Geometry",
  pages = "16--29",
  booktitle = "The Axiomatic Method, with Special Reference to Geometry and
     Physics (Proceedings of an International Symposium held at the University
     of California, Berkeley, December 26, 1957 --- January 4, 1958)",
  editor = "Leon Henkin and Patrick Suppes and Alfred Tarski",
  year = 1959,
  publisher = "North-Holland Publishing Company",
  address = "Amsterdam"}
@article{Tarski1965, author = "Alfred Tarski",
  title = "A Simplified Formalization of Predicate Logic with Identity",
  journal = "Archiv f{\"{u}}r Mathematische Logik und Grundlagenforschung",
  volume = 7,
  year = 1965,
  note = "[QA.A673]",
  pages = "61--79" }
@book{Tymoczko,
  title = "New Directions in the Philosophy of Mathematics",
  editor = "Thomas Tymoczko",
  publisher = "Birkh{\"{a}}user Boston, Inc.",
  address = "Boston",
  note = "[QA8.6.N48 1986]",
  year = 1986 }
@incollection{Wang,
  author = "Hao Wang",
  title = "Theory and Practice in Mathematics",
  pages = "129--152",
  booktitle = "New Directions in the Philosophy of Mathematics",
  editor = "Thomas Tymoczko",
  publisher = "Birkh{\"{a}}user Boston, Inc.",
  address = "Boston",
  note = "[QA8.6.N48 1986]",
  year = 1986 }
@manual{Webster,
  title = "Webster's New Collegiate Dictionary",
  organization = "G. \& C. Merriam Co.",
  address = "Springfield, Massachusetts",
  note = "[PE1628.W4M4 1977]",
  year = 1977 }
@manual{Whitehead, author = "Alfred North Whitehead",
  title = "An Introduction to Mathematics",
  year = 1911 }
@book{PM, author = "Alfred North Whitehead and Bertrand Russell",
  title = "Principia Mathematica",
  edition = "second",
  publisher = "Cambridge University Press",
  address = "Cambridge",
  year = "1927",
  note = "(3 vols.) [QA9.W592 1927]" }
@article{DBLP:journals/corr/Whalen16,
  author    = {Daniel Whalen},
  title     = {Holophrasm: a neural Automated Theorem Prover for higher-order logic},
  journal   = {CoRR},
  volume    = {abs/1608.02644},
  year      = {2016},
  url       = {http://arxiv.org/abs/1608.02644},
  archivePrefix = {arXiv},
  eprint    = {1608.02644},
  timestamp = {Mon, 13 Aug 2018 16:46:19 +0200},
  biburl    = {https://dblp.org/rec/bib/journals/corr/Whalen16},
  bibsource = {dblp computer science bibliography, https://dblp.org} }
@article{Wiedijk-revisited,
  author = {Freek Wiedijk},
  title = {The QED Manifesto Revisited},
  year = {2007},
  url = {http://mizar.org/trybulec65/8.pdf} }
@book{Wolfram,
  author = "Stephen Wolfram",
  title = "Mathematica:  A System for Doing Mathematics by Computer",
  edition = "second",
  publisher = "Addison-Wesley Publishing Co.",
  address = "Redwood City, California",
  note = "[QA76.95.W65 1991]",
  year = 1991 }
@book{Wos, author = "Larry Wos and Ross Overbeek and Ewing Lusk and Jim Boyle",
  title = "Automated Reasoning:  Introduction and Applications",
  edition = "second",
  publisher = "McGraw-Hill, Inc.",
  address = "New York",
  note = "[QA76.9.A96.A93 1992]",
  year = 1992 }

%
%
%[1] Church, Alonzo, Introduction to Mathematical Logic,
% Volume 1, Princeton University Press, Princeton, N. J., 1956.
%
%[2] Cohen, Paul J., Set Theory and the Continuum Hypothesis,
% W. A. Benjamin, Inc., Reading, Mass., 1966.
%
%[3] Hamilton, Alan G., Logic for Mathematicians, Cambridge
% University Press,
% Cambridge, 1988.

%[6] Kleene, Stephen Cole, Introduction to Metamathematics, D.  Van
% Nostrand Company, Inc., Princeton (1952).

%[13] Tarski, Alfred, "A simplified formalization of predicate
% logic with identity," Archiv fur Mathematische Logik und
% Grundlagenforschung, vol. 7 (1965), pp. 61-79.

%[14] Tarski, Alfred and Steven Givant, A Formalization of Set
% Theory Without Variables, American Mathematical Society Colloquium
% Publications, vol. 41, American Mathematical Society,
% Providence, R. I., 1987.

%[15] Zeman, J. J., Modal Logic, Oxford University Press, Oxford, 1973.
\end{filecontents}
% --------------------------- End of metamath.bib -----------------------------


%Book: Metamath
%Author:  Norman Megill Email:  nm at alum.mit.edu
%Author:  David A. Wheeler Email:  dwheeler at dwheeler.com

% A book template example
% http://www.stsci.edu/ftp/software/tex/bookstuff/book.template

\documentclass[leqno]{book} % LaTeX 2e. 10pt. Use [leqno,12pt] for 12pt
% hyperref 2002/05/27 v6.72r  (couldn't get pagebackref to work)
\usepackage[plainpages=false,pdfpagelabels=true]{hyperref}

\usepackage{needspace}     % Enable control over page breaks
\usepackage{breqn}         % automatic equation breaking
\usepackage{microtype}     % microtypography, reduces hyphenation

% Packages for flexible tables.  We need to be able to
% wrap text within a cell (with automatically-determined widths) AND
% split a table automatically across multiple pages.
% * "tabularx" wraps text in cells but only 1 page
% * "longtable" goes across pages but by itself is incompatible with tabularx
% * "ltxtable" combines longtable and tabularx, but table contents
%    must be in a separate file.
% * "ltablex" combines tabularx and longtable - must install specially
% * "booktabs" is recommended as a way to improve the look of tables,
%   but doesn't add these capabilities.
% * "tabu" much more capable and seems to be recommended. So use that.

\usepackage{makecell}      % Enable forced line splits within a table cell
% v4.13 needed for tabu: https://tex.stackexchange.com/questions/600724/dimension-too-large-after-recent-longtable-update
\usepackage{longtable}[=v4.13] % Enable multi-page tables  
\usepackage{tabu}          % Multi-page tables with wrapped text in a cell

% You can find more Tex packages using commands like:
% tlmgr search --file tabu.sty
% find /usr/share/texmf-dist/ -name '*tab*'
%
%%%%%%%%%%%%%%%%%%%%%%%%%%%%%%%%%%%%%%%%%%%%%%%%%%%%%%%%%%%%%%%%%%%%%%%%%%%%
% Uncomment the next 3 lines to suppress boxes and colors on the hyperlinks
%%%%%%%%%%%%%%%%%%%%%%%%%%%%%%%%%%%%%%%%%%%%%%%%%%%%%%%%%%%%%%%%%%%%%%%%%%%%
%\hypersetup{
%colorlinks,citecolor=black,filecolor=black,linkcolor=black,urlcolor=black
%}
%
\usepackage{realref}

% Restarting page numbers: try?
%   \printglossary
%   \cleardoublepage
%   \pagenumbering{arabic}
%   \setcounter{page}{1}    ???needed
%   \include{chap1}

% not used:
% \def\R2Lurl#1#2{\mbox{\href{#1}\texttt{#2}}}

\usepackage{amssymb}

% Version 1 of book: margins: t=.4, b=.2, ll=.4, rr=.55
% \usepackage{anysize}
% % \papersize{<height>}{<width>}
% % \marginsize{<left>}{<right>}{<top>}{<bottom>}
% \papersize{9in}{6in}
% % l/r 0.6124-0.6170 works t/b 0.2418-0.3411 = 192pp. 0.2926-03118=exact
% \marginsize{0.7147in}{0.5147in}{0.4012in}{0.2012in}

\usepackage{anysize}
% \papersize{<height>}{<width>}
% \marginsize{<left>}{<right>}{<top>}{<bottom>}
\papersize{9in}{6in}
% l/r 0.85in&0.6431-0.6539 works t/b ?-?
%\marginsize{0.85in}{0.6485in}{0.55in}{0.35in}
\marginsize{0.8in}{0.65in}{0.5in}{0.3in}

% \usepackage[papersize={3.6in,4.8in},hmargin=0.1in,vmargin={0.1in,0.1in}]{geometry}  % page geometry
\usepackage{special-settings}

\raggedbottom
\makeindex

\begin{document}
% Discourage page widows and orphans:
\clubpenalty=300
\widowpenalty=300

%%%%%%% load in AMS fonts %%%%%%% % LaTeX 2.09 - obsolete in LaTeX 2e
%\input{amssym.def}
%\input{amssym.tex}
%\input{c:/texmf/tex/plain/amsfonts/amssym.def}
%\input{c:/texmf/tex/plain/amsfonts/amssym.tex}

\bibliographystyle{plain}
\pagenumbering{roman}
\pagestyle{headings}

\thispagestyle{empty}

\hfill
\vfill

\begin{center}
{\LARGE\bf Metamath} \\
\vspace{1ex}
{\large A Computer Language for Mathematical Proofs} \\
\vspace{7ex}
{\large Norman Megill} \\
\vspace{7ex}
with extensive revisions by \\
\vspace{1ex}
{\large David A. Wheeler} \\
\vspace{7ex}
% Printed date. If changing the date below, also fix the date at the beginning.
2019-06-02
\end{center}

\vfill
\hfill

\newpage
\thispagestyle{empty}

\hfill
\vfill

\begin{center}
$\sim$\ {\sc Public Domain}\ $\sim$

\vspace{2ex}
This book (including its later revisions)
has been released into the Public Domain by Norman Megill per the
Creative Commons CC0 1.0 Universal (CC0 1.0) Public Domain Dedication.
David A. Wheeler has done the same.
This public domain release applies worldwide.  In case this is not
legally possible, the right is granted to use the work for any purpose,
without any conditions, unless such conditions are required by law.
See \url{https://creativecommons.org/publicdomain/zero/1.0/}.

\vspace{3ex}
Several short, attributed quotations from copyrighted works
appear in this book under the ``fair use'' provision of Section 107 of
the United States Copyright Act (Title 17 of the {\em United States
Code}).  The public-domain status of this book is not applicable to
those quotations.

\vspace{3ex}
Any trademarks used in this book are the property of their owners.

% QA76.9.L63.M??

% \vspace{1ex}
%
% \vspace{1ex}
% {\small Permission is granted to make and distribute verbatim copies of this
% book
% provided the copyright notice and this
% permission notice are preserved on all copies.}
%
% \vspace{1ex}
% {\small Permission is granted to copy and distribute modified versions of this
% book under the conditions for verbatim copying, provided that the
% entire
% resulting derived work is distributed under the terms of a permission
% notice
% identical to this one.}
%
% \vspace{1ex}
% {\small Permission is granted to copy and distribute translations of this
% book into another language, under the above conditions for modified
% versions,
% except that this permission notice may be stated in a translation
% approved by the
% author.}
%
% \vspace{1ex}
% %{\small   For a copy of the \LaTeX\ source files for this book, contact
% %the author.} \\
% \ \\
% \ \\

\vspace{7ex}
% ISBN: 1-4116-3724-0 \\
% ISBN: 978-1-4116-3724-5 \\
ISBN: 978-0-359-70223-7 \\
{\ } \\
Lulu Press \\
Morrisville, North Carolina\\
USA


\hfill
\vfill

Norman Megill\\ 93 Bridge St., Lexington, MA 02421 \\
E-mail address: \texttt{nm{\char`\@}alum.mit.edu} \\
\vspace{7ex}
David A. Wheeler \\
E-mail address: \texttt{dwheeler{\char`\@}dwheeler.com} \\
% See notes added at end of Preface for revision history. \\
% For current information on the Metamath software see \\
\vspace{7ex}
\url{http://metamath.org}
\end{center}

\hfill
\vfill

{\parindent0pt%
\footnotesize{%
Cover: Aleph null ($\aleph_0$) is the symbol for the
first infinite cardinal number, discovered by Georg Cantor in 1873.
We use a red aleph null (with dark outline and gold glow) as the Metamath logo.
Credit: Norman Megill (1994) and Giovanni Mascellani (2019),
public domain.%
\index{aleph null}%
\index{Metamath!logo}\index{Cantor, Georg}\index{Mascellani, Giovanni}}}

% \newpage
% \thispagestyle{empty}
%
% \hfill
% \vfill
%
% \begin{center}
% {\it To my son Robin Dwight Megill}
% \end{center}
%
% \vfill
% \hfill
%
% \newpage

\tableofcontents
%\listoftables

\chapter*{Preface}
\markboth{PREFACE}{PREFACE}
\addcontentsline{toc}{section}{Preface}


% (For current information, see the notes added at the
% end of this preface on p.~\pageref{note2002}.)

\subsubsection{Overview}

Metamath\index{Metamath} is a computer language and an associated computer
program for archiving, verifying, and studying mathematical proofs at a very
detailed level.  The Metamath language incorporates no mathematics per se but
treats all mathematical statements as mere sequences of symbols.  You provide
Metamath with certain special sequences (axioms) that tell it what rules
of inference are allowed.  Metamath is not limited to any specific field of
mathematics.  The Metamath language is simple and robust, with an
almost total absence of hard-wired syntax, and
we\footnote{Unless otherwise noted, the words
``I,'' ``me,'' and ``my'' refer to Norman Megill\index{Megill, Norman}, while
``we,'' ``us,'' and ``our'' refer to Norman Megill and
David A. Wheeler\index{Wheeler, David A.}.}
believe that it
provides about the simplest possible framework that allows essentially all of
mathematics to be expressed with absolute rigor.

% index test
%\newcommand{\nn}[1]{#1n}
%\index{aaa@bbb}
%\index{abc!def}
%\index{abd|see{qqq}}
%\index{abe|nn}
%\index{abf|emph}
%\index{abg|(}
%\index{abg|)}

Using the Metamath language, you can build formal or mathematical
systems\index{formal system}\footnote{A formal or mathematical system consists
of a collection of symbols (such as $2$, $4$, $+$ and $=$), syntax rules that
describe how symbols may be combined to form a legal expression (called a
well-formed formula or {\em wff}, pronounced ``whiff''), some starting wffs
called axioms, and inference rules that describe how theorems may be derived
(proved) from the axioms.  A theorem is a mathematical fact such as $2+2=4$.
Strictly speaking, even an obvious fact such as this must be proved from
axioms to be formally acceptable to a mathematician.}\index{theorem}
\index{axiom}\index{rule}\index{well-formed formula (wff)} that involve
inferences from axioms.  Although a database is provided
that includes a recommended set of axioms for standard mathematics, if you
wish you can supply your own symbols, syntax, axioms, rules, and definitions.

The name ``Metamath'' was chosen to suggest that the language provides a
means for {\em describing} mathematics rather than {\em being} the
mathematics itself.  Actually in some sense any mathematical language is
metamathematical.  Symbols written on paper, or stored in a computer,
are not mathematics itself but rather a way of expressing mathematics.
For example ``7'' and ``VII'' are symbols for denoting the number seven
in Arabic and Roman numerals; neither {\em is} the number seven.

If you are able to understand and write computer programs, you should be able
to follow abstract mathematics with the aid of Metamath.  Used in conjunction
with standard textbooks, Metamath can guide you step-by-step towards an
understanding of abstract mathematics from a very rigorous viewpoint, even if
you have no formal abstract mathematics background.  By using a single,
consistent notation to express proofs, once you grasp its basic concepts
Metamath provides you with the ability to immediately follow and dissect
proofs even in totally unfamiliar areas.

Of course, just being able follow a proof will not necessarily give you an
intuitive familiarity with mathematics.  Memorizing the rules of chess does not
give you the ability to appreciate the game of a master, and knowing how the
notes on a musical score map to piano keys does not give you the ability to
hear in your head how it would sound.  But each of these can be a first step.

Metamath allows you to explore proofs in the sense that you can see the
theorem referenced at any step expanded in as much detail as you want, right
down to the underlying axioms of logic and set theory (in the case of the set
theory database provided).  While Metamath will not replace the higher-level
understanding that can only be acquired through exercises and hard work, being
able to see how gaps in a proof are filled in can give you increased
confidence that can speed up the learning process and save you time when you
get stuck.

The Metamath language breaks down a mathematical proof into its tiniest
possible parts.  These can be pieced together, like interlocking
pieces in a puzzle, only in a way that produces correct and absolutely rigorous
mathematics.

The nature of Metamath\index{Metamath} enforces very precise mathematical
thinking, similar to that involved in writing a computer program.  A crucial
difference, though, is that once a proof is verified (by the Metamath program)
to be correct, it is definitely correct; it can never have a hidden
``bug.''\index{computer program bugs}  After getting used to the kind of rigor
and accuracy provided by Metamath, you might even be tempted to
adopt the attitude that a proof should never be considered correct until it
has been verified by a computer, just as you would not completely trust a
manual calculation until you have verified it on a
calculator.

My goal
for Metamath was a system for describing and verifying
mathematics that is completely universal yet conceptually as simple as
possible.  In approaching mathematics from an axiomatic, formal viewpoint, I
wanted Metamath to be able to handle almost any mathematical system, not
necessarily with ease, but at least in principle and hopefully in practice. I
wanted it to verify proofs with absolute rigor, and for this reason Metamath
is what might be thought of as a ``compile-only'' language rather than an
algorithmic or Turing-machine language (Pascal, C, Prolog, Mathematica,
etc.).  In other words, a database written in the Metamath
language doesn't ``do'' anything; it merely exhibits mathematical knowledge
and permits this knowledge to be verified as being correct.  A program in an
algorithmic language can potentially have hidden bugs\index{computer program
bugs} as well as possibly being hard to understand.  But each token in a
Metamath database must be consistent with the database's earlier
contents according to simple, fixed rules.
If a database is verified
to be correct,\footnote{This includes
verification that a sequential list of proof steps results in the specified
theorem.} then the mathematical content is correct if the
verifier is correct and the axioms are correct.
The verification program could be incorrect, but the verification algorithm
is relatively simple (making it unlikely to be implemented incorrectly
by the Metamath program),
and there are over a dozen Metamath database verifiers
written by different people in different programming languages
(so these different verifiers can act as multiple reviewers of a database).
The most-used Metamath database, the Metamath Proof Explorer
(aka \texttt{set.mm}\index{set theory database (\texttt{set.mm})}%
\index{Metamath Proof Explorer}),
is currently verified by four different Metamath verifiers written by
four different people in four different languages, including the
original Metamath program described in this book.
The only ``bugs'' that can exist are in the statement of the axioms,
for example if the axioms are inconsistent (a famous problem shown to be
unsolvable by G\"{o}del's incompleteness theorem\index{G\"{o}del's
incompleteness theorem}).
However, real mathematical systems have very few axioms, and these can
be carefully studied.
All of this provides extraordinarily high confidence that the verified database
is in fact correct.

The Metamath program
doesn't prove theorems automatically but is designed to verify proofs
that you supply to it.
The underlying Metamath language is completely general and has no built-in,
preconceived notions about your formal system\index{formal system}, its logic
or its syntax.
For constructing proofs, the Metamath program has a Proof Assistant\index{Proof
Assistant} which helps you fill in some of a proof step's details, shows you
what choices you have at any step, and verifies the proof as you build it; but
you are still expected to provide the proof.

There are many other programs that can process or generate information
in the Metamath language, and more continue to be written.
This is in part because the Metamath language itself is very simple
and intentionally easy to automatically process.
Some programs, such as \texttt{mmj2}\index{mmj2}, include a proof assistant
that can automate some steps beyond what the Metamath program can do.
Mario Carneiro has developed an algorithm for converting proofs from
the OpenTheory interchange format, which can be translated to and from
any of the HOL family of proof languages (HOL4, HOL Light, ProofPower,
and Isabelle), into the
Metamath language \cite{DBLP:journals/corr/Carneiro14}\index{Carneiro, Mario}.
Daniel Whalen has developed Holophrasm, which can automatically
prove many Metamath proofs using
machine learning\index{machine learning}\index{artificial intelligence}
approaches
(including multiple neural networks\index{neural networks})\cite{DBLP:journals/corr/Whalen16}\index{Whalen, Daniel}.
However,
a discussion of these other programs is beyond the scope of this book.

Like most computer languages, the Metamath\index{Metamath} language uses the
standard ({\sc ascii}) characters on a computer keyboard, so it cannot
directly represent many of the special symbols that mathematicians use.  A
useful feature of the Metamath program is its ability to convert its notation
into the \LaTeX\ typesetting language.\index{latex@{\LaTeX}}  This feature
lets you convert the {\sc ascii} tokens you've defined into standard
mathematical symbols, so you end up with symbols and formulas you are familiar
with instead of somewhat cryptic {\sc ascii} representations of them.
The Metamath program can also generate HTML\index{HTML}, making it easy
to view results on the web and to see related information by using
hypertext links.

Metamath is probably conceptually different from anything you've seen
before and some aspects may take some getting used to.  This book will
help you decide whether Metamath suits your specific needs.

\subsubsection{Setting Your Expectations}
It is important for you to understand what Metamath\index{Metamath} is and is
not.  As mentioned, the Metamath program
is {\em not} an automated theorem prover but
rather a proof verifier.  Developing a database can be tedious, hard work,
especially if you want to make the proofs as short as possible, but it becomes
easier as you build up a collection of useful theorems.  The purpose of
Metamath is simply to document existing mathematics in an absolutely rigorous,
computer-verifiable way, not to aid directly in the creation of new
mathematics.  It also is not a magic solution for learning abstract
mathematics, although it may be helpful to be able to actually see the implied
rigor behind what you are learning from textbooks, as well as providing hints
to work out proofs that you are stumped on.

As of this writing, a sizable set theory database has been developed to
provide a foundation for many fields of mathematics, but much more work would
be required to develop useful databases for specific fields.

Metamath\index{Metamath} ``knows no math;'' it just provides a framework in
which to express mathematics.  Its language is very small.  You can define two
kinds of symbols, constants\index{constant} and variables\index{variable}.
The only thing Metamath knows how to do is to substitute strings of symbols
for the variables\index{substitution!variable}\index{variable substitution} in
an expression based on instructions you provide it in a proof, subject to
certain constraints you specify for the variables.  Even the decimal
representation of a number is merely a string of certain constants (digits)
which together, in a specific context, correspond to whatever mathematical
object you choose to define for it; unlike other computer languages, there is
no actual number stored inside the computer.  In a proof, you in effect
instruct Metamath what symbol substitutions to make in previous axioms or
theorems and join a sequence of them together to result in the desired
theorem.  This kind of symbol manipulation captures the essence of mathematics
at a preaxiomatic level.

\subsubsection{Metamath and Mathematical Literature}

In advanced mathematical literature, proofs are usually presented in the form
of short outlines that often only an expert can follow.  This is partly out of
a desire for brevity, but it would also be unwise (even if it were practical)
to present proofs in complete formal detail, since the overall picture would
be lost.\index{formal proof}

A solution I envision\label{envision} that would allow mathematics to remain
acceptable to the expert, yet increase its accessibility to non-specialists,
consists of a combination of the traditional short, informal proof in print
accompanied by a complete formal proof stored in a computer database.  In an
analogy with a computer program, the informal proof is like a set of comments
that describe the overall reasoning and content of the proof, whereas the
computer database is like the actual program and provides a means for anyone,
even a non-expert, to follow the proof in as much detail as desired, exploring
it back through layers of theorems (like subroutines that call other
subroutines) all the way back to the axioms of the theory.  In addition, the
computer database would have the advantage of providing absolute assurance
that the proof is correct, since each step can be verified automatically.

There are several other approaches besides Metamath to a project such
as this.  Section~\ref{proofverifiers} discusses some of these.

To us, a noble goal would be a database with hundreds of thousands of
theorems and their computer-verifiable proofs, encompassing a significant
fraction of known mathematics and available for instant access.
These would be fully verified by multiple independently-implemented verifiers,
to provide extremely high confidence that the proofs are completely correct.
The database would allow people to investigate whatever details they were
interested in, so that they could confirm whatever portions they wished.
Whether or not Metamath is an appropriate choice remains to be seen, but in
principle we believe it is sufficient.

\subsubsection{Formalism}

Over the past fifty years, a group of French mathematicians working
collectively under the pseudonym of Bourbaki\index{Bourbaki, Nicolas} have
co-authored a series of monographs that attempt to rigorously and
consistently formalize large bodies of mathematics from foundations.  On the
one hand, certainly such an effort has its merits; on the other hand, the
Bourbaki project has been criticized for its ``scholasticism'' and
``hyperaxiomatics'' that hide the intuitive steps that lead to the results
\cite[p.~191]{Barrow}\index{Barrow, John D.}.

Metamath unabashedly carries this philosophy to its extreme and no doubt is
subject to the same kind of criticism.  Nonetheless I think that in
conjunction with conventional approaches to mathematics Metamath can serve a
useful purpose.  The Bourbaki approach is essentially pedagogic, requiring the
reader to become intimately familiar with each detail in a very large
hierarchy before he or she can proceed to the next step.  The difference with
Metamath is that the ``reader'' (user) knows that all details are contained in
its computer database, available as needed; it does not demand that the user
know everything but conveniently makes available those portions that are of
interest.  As the body of all mathematical knowledge grows larger and larger,
no one individual can have a thorough grasp of its entirety.  Metamath
can finalize and put to rest any questions about the validity of any part of it
and can make any part of it accessible, in principle, to a non-specialist.

\subsubsection{A Personal Note}
Why did I develop Metamath\index{Metamath}?  I enjoy abstract mathematics, but
I sometimes get lost in a barrage of definitions and start to lose confidence
that my proofs are correct.  Or I reach a point where I lose sight of how
anything I'm doing relates to the axioms that a theory is based on and am
sometimes suspicious that there may be some overlooked implicit axiom
accidentally introduced along the way (as happened historically with Euclidean
geometry\index{Euclidean geometry}, whose omission of Pasch's
axiom\index{Pasch's axiom} went unnoticed for 2000 years
\cite[p.~160]{Davis}!). I'm also somewhat lazy and wish to avoid the effort
involved in re-verifying the gaps in informal proofs ``left to the reader;'' I
prefer to figure them out just once and not have to go through the same
frustration a year from now when I've forgotten what I did.  Metamath provides
better recovery of my efforts than scraps of paper that I can't
decipher anymore.  But mostly I find very appealing the idea of rigorously
archiving mathematical knowledge in a computer database, providing precision,
certainty, and elimination of human error.

\subsubsection{Note on Bibliography and Index}

The Bibliography usually includes the Library of Congress classification
for a work to make it easier for you to find it in on a university
library shelf.  The Index has author references to pages where their works
are cited, even though the authors' names may not appear on those pages.

\subsubsection{Acknowledgments}

Acknowledgments are first due to my wife, Deborah (who passed away on
September 4, 1998), for critiquing the manu\-script but most of all for
her patience and support.  I also wish to thank Joe Wright, Richard
Becker, Clarke Evans, Buddha Buck, and Jeremy Henty for helpful
comments.  Any errors, omissions, and other shortcomings are of course
my responsibility.

\subsubsection{Note Added June 22, 2005}\label{note2002}

The original, unpublished version of this book was written in 1997 and
distributed via the web.  The present edition has been updated to
reflect the current Metamath program and databases, as well as more
current {\sc url}s for Internet sites.  Thanks to Josh
Purinton\index{Purinton, Josh}, One Hand
Clapping, Mel L.\ O'Cat, and Roy F. Longton for pointing out
typographical and other errors.  I have also benefitted from numerous
discussions with Raph Levien\index{Levien, Raph}, who has extended
Metamath's philosophy of rigor to result in his {\em
Ghilbert}\index{Ghilbert} proof language (\url{http://ghilbert.org}).

Robert (Bob) Solovay\index{Solovay, Robert} communicated a new result of
A.~R.~D.~Mathias on the system of Bourbaki, and the text has been
updated accordingly (p.~\pageref{bourbaki}).

Bob also pointed out a clarification of the literature regarding
category theory and inaccessible cardinals\index{category
theory}\index{cardinal, inaccessible} (p.~\pageref{categoryth}),
and a misleading statement was removed from the text.  Specifically,
contrary to a statement in previous editions, it is possible to express
``There is a proper class of inaccessible cardinals'' in the language of
ZFC.  This can be done as follows:  ``For every set $x$ there is an
inaccessible cardinal $\kappa$ such that $\kappa$ is not in $x$.''
Bob writes:\footnote{Private communication, Nov.~30, 2002.}
\begin{quotation}
     This axiom is how Grothendieck presents category theory.  To each
inaccessible cardinal $\kappa$ one associates a Grothendieck universe
\index{Grothendieck, Alexander} $U(\kappa)$.  $U(\kappa)$ consists of
those sets which lie in a transitive set of cardinality less than
$\kappa$.  Instead of the ``category of all groups,'' one works relative
to a universe [considering the category of groups of cardinality less
than $\kappa$].  Now the category whose objects are all categories
``relative to the universe $U(\kappa)$'' will be a category not
relative to this universe but to the next universe.

     All of the things category theorists like to do can be done in this
framework.  The only controversial point is whether the Grothen\-dieck
axiom is too strong for the needs of category theorists.  Mac Lane
\index{Mac Lane, Saunders} argues that ``one universe is enough'' and
Feferman\index{Feferman, Solomon} has argued that one can get by with
ordinary ZFC.  I don't find Feferman's arguments persuasive.  Mac Lane
may be right, but when I think about category theory I do it \`{a} la
Grothendieck.

        By the way Mizar\index{Mizar} adds the axiom ``there is a proper
class of inaccessibles'' precisely so as to do category theory.
\end{quotation}

The most current information on the Metamath program and databases can
always be found at \url{http://metamath.org}.


\subsubsection{Note Added June 24, 2006}\label{note2006}

The Metamath spec was restricted slightly to make parsers easier to
write.  See the footnote on p.~\pageref{namespace}.

%\subsubsection{Note Added July 24, 2006}\label{note2006b}
\subsubsection{Note Added March 10, 2007}\label{note2006b}

I am grateful to Anthony Williams\index{Williams, Anthony} for writing
the \LaTeX\ package called {\tt realref.sty} and contributing it to the
public domain.  This package allows the internal hyperlinks in a {\sc
pdf} file to anchor to specific page numbers instead of just section
titles, making the navigation of the {\sc pdf} file for this book much
more pleasant and ``logical.''

A typographical error found by Martin Kiselkov was corrected.
A confusing remark about unification was deleted per suggestion of
Mel O'Cat.

\subsubsection{Note Added May 27, 2009}\label{note2009}

Several typos found by Kim Sparre were corrected.  A note was added that
the Poincar\'{e} conjecture has been proved (p.~\pageref{poincare}).

\subsubsection{Note Added Nov. 17, 2014}\label{note2014}

The statement of the Schr\"{o}der--Bernstein theorem was corrected in
Section~\ref{trust}.  Thanks to Bob Solovay for pointing out the error.

\subsubsection{Note Added May 25, 2016}\label{note2016}

Thanks to Jerry James for correcting 16 typos.

\subsubsection{Note Added February 25, 2019}\label{note201902}

David A. Wheeler\index{Wheeler, David A.}
made a large number of improvements and updates,
in coordination with Norman Megill.
The predicate calculus axioms were renumbered, and the text makes
it clear that they are based on Tarski's system S2;
the one slight deviation in axiom ax-6 is explained and justified.
The real and complex number axioms were modified to be consistent with
\texttt{set.mm}\index{set theory database (\texttt{set.mm})}%
\index{Metamath Proof Explorer}.
Long-awaited specification changes ``1--8'' were made,
which clarified previously ambiguous points.
Some errors in the text involving \texttt{\$f} and
\texttt{\$d} statements were corrected (the spec was correct, but
the in-book explanations unintentionally contradicted the spec).
We now have a system for automatically generating narrow PDFs,
so that those with smartphones can have easy access to the current
version of this document.
A new section on deduction was added;
it discusses the standard deduction theorem,
the weak deduction theorem,
deduction style, and natural deduction.
Many minor corrections (too numerous to list here) were also made.

\subsubsection{Note Added March 7, 2019}\label{note201903}

This added a description of the Matamath language syntax in
Extended Backus--Naur Form (EBNF)\index{Extended Backus--Naur Form}\index{EBNF}
in Appendix \ref{BNF}, added a brief explanation about typecodes,
inserted more examples in the deduction section,
and added a variety of smaller improvements.

\subsubsection{Note Added April 7, 2019}\label{note201904}

This version clarified the proper substitution notation, improved the
discussion on the weak deduction theorem and natural deduction,
documented the \texttt{undo} command, updated the information on
\texttt{write source}, changed the typecode
from \texttt{set} to \texttt{setvar} to be consistent with the current
version of \texttt{set.mm}, added more documentation about comment markup
(e.g., documented how to create headings), and clarified the
differences between various assertion forms (in particular deduction form).

\subsubsection{Note Added June 2, 2019}\label{note201906}

This version fixes a large number of small issues reported by
Beno\^{i}t Jubin\index{Jubin, Beno\^{i}t}, such as editorial issues
and the need to document \texttt{verify markup} (thank you!).
This version also includes specific examples
of forms (deduction form, inference form, and closed form).
We call this version the ``second edition'';
the previous edition formally published in 2007 had a slightly different title
(\textit{Metamath: A Computer Language for Pure Mathematics}).

\chapter{Introduction}
\pagenumbering{arabic}

\begin{quotation}
  {\em {\em I.M.:}  No, no.  There's nothing subjective about it!  Everybody
knows what a proof is.  Just read some books, take courses from a competent
mathematician, and you'll catch on.

{\em Student:}  Are you sure?

{\em I.M.:}  Well---it is possible that you won't, if you don't have any
aptitude for it.  That can happen, too.

{\em Student:}  Then {\em you} decide what a proof is, and if I don't learn
to decide in the same way, you decide I don't have any aptitude.

{\em I.M.:}  If not me, then who?}
    \flushright\sc  ``The Ideal Mathematician''
    \index{Davis, Phillip J.}
    \footnote{\cite{Davis}, p.~40.}\\
\end{quotation}

Brilliant mathematicians have discovered almost
unimaginably profound results that rank among the crowning intellectual
achievements of mankind.  However, there is a sense in which modern abstract
mathematics is behind the times, stuck in an era before computers existed.
While no one disputes the remarkable results that have been achieved,
communicating these results in a precise way to the uninitiated is virtually
impossible.  To describe these results, a terse informal language is used which
despite its elegance is very difficult to learn.  This informal language is not
imprecise, far from it, but rather it often has omitted detail
and symbols with hidden context that are
implicitly understood by an expert but few others.  Extremely complex technical
meanings are associated with innocent-sounding English words such as
``compact'' and ``measurable'' that barely hint at what is actually being
said.  Anyone who does not keep the precise technical meaning constantly in
mind is bound to fail, and acquiring the ability to do this can be achieved
only through much practice and hard work.  Only the few who complete the
painful learning experience can join the small in-group of pure
mathematicians.  The informal language effectively cuts off the true nature of
their knowledge from most everyone else.

Metamath\index{Metamath} makes abstract mathematics more concrete.  It allows
a computer to keep track of the complexity associated with each word or symbol
with absolute rigor.  You can explore this complexity at your leisure, to
whatever degree you desire.  Whether or not you believe that concepts such as
infinity actually ``exist'' outside of the mind, Metamath lets you get to the
foundation for what's really being said.

Metamath also enables completely rigorous and thorough proof verification.
Its language is simple enough so that you
don't have to rely on the authority of experts but can verify the results
yourself, step by step.  If you want to attempt to derive your own results,
Metamath will not let you make a mistake in reasoning.
Even professional mathematicians make mistakes; Metamath makes it possible
to thoroughly verify that proofs are correct.

Metamath\index{Metamath} is a computer language and an associated computer
program for archiving, verifying, and studying mathematical proofs at a very
detailed level.
The Metamath language
describes formal\index{formal system} mathematical
systems and expresses proofs of theorems in those systems.  Such a language
is called a metalanguage\index{metalanguage} by mathematicians.
The Metamath program is a computer program that verifies
proofs expressed in the Metamath language.
The Metamath program does not have the built-in
ability to make logical inferences; it just makes a series of symbol
substitutions according to instructions given to it in a proof
and verifies that the result matches the expected theorem.  It makes logical
inferences based only on rules of logic that are contained in a set of
axioms\index{axiom}, or first principles, that you provide to it as the
starting point for proofs.

The complete specification of the Metamath language is only four pages long
(Section~\ref{spec}, p.~\pageref{spec}).  Its simplicity may at first make you
wonder how it can do much of anything at all.  But in fact the kinds of
symbol manipulations it performs are the ones that are implicitly done in all
mathematical systems at the lowest level.  You can learn it relatively quickly
and have complete confidence in any mathematical proof that it verifies.  On
the other hand, it is powerful and general enough so that virtually any
mathematical theory, from the most basic to the deeply abstract, can be
described with it.

Although in principle Metamath can be used with any
kind of mathematics, it is best suited for abstract or ``pure'' mathematics
that is mostly concerned with theorems and their proofs, as opposed to the
kind of mathematics that deals with the practical manipulation of numbers.
Examples of branches of pure mathematics are logic\index{logic},\footnote{Logic
is the study of statements that are universally true regardless of the objects
being described by the statements.  An example is the statement, ``if $P$
implies $Q$, then either $P$ is false or $Q$ is true.''} set theory\index{set
theory},\footnote{Set theory is the study of general-purpose mathematical objects called
``sets,'' and from it essentially all of mathematics can be derived.  For
example, numbers can be defined as specific sets, and their properties
can be explored using the tools of set theory.} number theory\index{number
theory},\footnote{Number theory deals with the properties of positive and
negative integers (whole numbers).} group theory\index{group
theory},\footnote{Group theory studies the properties of mathematical objects
called groups that obey a simple set of axioms and have properties of symmetry
that make them useful in many other fields.} abstract algebra\index{abstract
algebra},\footnote{Abstract algebra includes group theory and also studies
groups with additional properties that qualify them as ``rings'' and
``fields.''  The set of real numbers is a familiar example of a field.},
analysis\index{analysis} \index{real and complex numbers}\footnote{Analysis is
the study of real and complex numbers.} and
topology\index{topology}.\footnote{One area studied by topology are properties
that remain unchanged when geometrical objects undergo stretching
deformations; for example a doughnut and a coffee cup each have one hole (the
cup's hole is in its handle) and are thus considered topologically
equivalent.  In general, though, topology is the study of abstract
mathematical objects that obey a certain (surprisingly simple) set of axioms.
See, for example, Munkres \cite{Munkres}\index{Munkres, James R.}.} Even in
physics, Metamath could be applied to certain branches that make use of
abstract mathematics, such as quantum logic\index{quantum logic} (used to study
aspects of quantum mechanics\index{quantum mechanics}).

On the other hand, Metamath\index{Metamath} is less suited to applications
that deal primarily with intensive numeric computations.  Metamath does not
have any built-in representation of numbers\index{Metamath!representation of
numbers}; instead, a specific string of symbols (digits) must be syntactically
constructed as part of any proof in which an ordinary number is used.  For
this reason, numbers in Metamath are best limited to specific constants that
arise during the course of a theorem or its proof.  Numbers are only a tiny
part of the world of abstract mathematics.  The exclusion of built-in numbers
was a conscious decision to help achieve Metamath's simplicity, and there are
other software tools if you have different mathematical needs.
If you wish to quickly solve algebraic problems, the computer algebra
programs\index{computer algebra system} {\sc
macsyma}\index{macsyma@{\sc macsyma}}, Mathematica\index{Mathematica}, and
Maple\index{Maple} are specifically suited to handling numbers and
algebra efficiently.
If you wish to simply calculate numeric or matrix expressions easily,
tools such as Octave\index{Octave} may be a better choice.

After learning Metamath's basic statement types, any
tech\-ni\-cal\-ly ori\-ent\-ed person, mathematician or not, can
immediately trace
any theorem proved in the language as far back as he or she wants, all the way
to the axioms on which the theorem is based.  This ability suggests a
non-traditional way of learning about pure mathematics.  Used in conjunction
with traditional methods, Metamath could make pure mathematics accessible to
people who are not sufficiently skilled to figure out the implicit detail in
ordinary textbook proofs.  Once you learn the axioms of a theory, you can have
complete confidence that everything you need to understand a proof you are
studying is all there, at your beck and call, allowing you to focus in on any
proof step you don't understand in as much depth as you need, without worrying
about getting stuck on a step you can't figure out.\footnote{On the other
hand, writing proofs in the Metamath language is challenging, requiring
a degree of rigor far in excess of that normally taught to students.  In a
classroom setting, I doubt that writing Metamath proofs would ever replace
traditional homework exercises involving informal proofs, because the time
needed to work out the details would not allow a course to
cover much material.  For students who have trouble grasping the implied rigor
in traditional material, writing a few simple proofs in the Metamath language
might help clarify fuzzy thought processes.  Although somewhat difficult at
first, it eventually becomes fun to do, like solving a puzzle, because of the
instant feedback provided by the computer.}

Metamath\index{Metamath} is probably unlike anything you have
encountered before.  In this first chapter we will look at the philosophy and
use of computers in mathematics in order to better understand the motivation
behind Metamath.  The material in this chapter is not required in order to use
Metamath.  You may skip it if you are impatient, but I hope you will find it
educational and enjoyable.  If you want to start experimenting with the
Metamath program right away, proceed directly to Chapter~\ref{using}
(p.~\pageref{using}).  To
learn the Metamath language, skim Chapter~\ref{using} then proceed to
Chapter~\ref{languagespec} (p.~\pageref{languagespec}).

\section{Mathematics as a Computer Language}

\begin{quote}
  {\em The study of mathematics is apt to commence in
dis\-ap\-point\-ment.\ldots \\
We are told that by its aid the stars are weighted
and the billions of molecules in a drop of water are counted.  Yet, like the
ghost of Hamlet's father, this great science eludes the efforts of our mental
weapons to grasp it.}
  \flushright\sc  Alfred North Whitehead\footnote{\cite{Whitehead}, ch.\ 1.}\\
\end{quote}\index{Whitehead, Alfred North}

\subsection{Is Mathematics ``User-Friendly''?}

Suppose you have no formal training in abstract mathematics.  But popular
books you've read offer tempting glimpses of this world filled with profound
ideas that have stirred the human spirit.  You are not satisfied with the
informal, watered-down descriptions you've read but feel it is important to
grasp the underlying mathematics itself to understand its true meaning. It's
not practical to go back to school to learn it, though; you don't want to
dedicate years of your life to it.  There are many important things in life,
and you have to set priorities for what's important to you.  What would happen
if you tried to pursue it on your own, in your spare time?

After all, you were able to learn a computer programming language such as
Pascal on your own without too much difficulty, even though you had no formal
training in computers.  You don't claim to be an expert in software design,
but you can write a passable program when necessary to suit your needs.  Even
more important, you know that you can look at anyone else's Pascal program, no
matter how complex, and with enough patience figure out exactly how it works,
even though you are not a specialist.  Pascal allows you do anything that a
computer can do, at least in principle.  Thus you know you have the ability,
in principle, to follow anything that a computer program can do:  you just
have to break it down into small enough pieces.

Here's an imaginary scenario of what might happen if you na\-ive\-ly a\-dopted
this same view of abstract mathematics and tried to pick it up on your own, in
a period of time comparable to, say, learning a computer programming
language.

\subsubsection{A Non-Mathematician's Quest for Truth}

\begin{quote}
  {\em \ldots my daughters have been studying (chemistry) for several
se\-mes\-ters, think they have learned differential and integral calculus in
school, and yet even today don't know why $x\cdot y=y\cdot x$ is true.}
  \flushright\sc  Edmund Landau\footnote{\cite{Landau}, p.~vi.}\\
\end{quote}\index{Landau, Edmund}

\begin{quote}
  {\em Minus times minus is plus,\\
The reason for this we need not discuss.}
  \flushright\sc W.\ H.\ Auden\footnote{As quoted in \cite{Guillen}, p.~64.}\\
\end{quote}\index{Auden, W.\ H.}\index{Guillen, Michael}

We'll suppose you are a technically oriented professional, perhaps an engineer, a
computer programmer, or a physicist, but probably not a mathematician.  You
consider yourself reasonably intelligent.  You did well in school, learning a
variety of methods and techniques in practical mathematics such as calculus and
differential equations.  But rarely did your courses get into anything
resembling modern abstract mathematics, and proofs were something that appeared
only occasionally in your textbooks, a kind of necessary evil that was
supposed to convince you of a certain key result.  Most of your
homework consisted of exercises that gave you practice in the techniques, and
you were hardly ever asked to come up with a proof of your own.

You find yourself curious about advanced, abstract mathematics.  You are
driven by an inner conviction that it is important to understand and
appreciate some of the most profound knowledge discovered by mankind.  But it
seems very hard to learn, something that only certain gifted longhairs can
access and understand.  You are frustrated that it seems forever cut off from
you.

Eventually your curiosity drives you to do something about it.
You set for yourself a goal of ``really'' understanding mathematics:  not just
how to manipulate equations in algebra or calculus according to cookbook
rules, but rather to gain a deep understanding of where those rules come from.
In fact, you're not thinking about this kind of ordinary mathematics at all,
but about a much more abstract, ethereal realm of pure mathematics, where
famous results such as G\"{o}del's incompleteness theorem\index{G\"{o}del's
incompleteness theorem} and Cantor's different kinds of infinities
reside.

You have probably read a number of popular books, with titles like {\em
Infinity and the Mind} \cite{Rucker}\index{Rucker, Rudy}, on topics such as
these.  You found them inspiring but at the same time somewhat
unsatisfactory.  They gave you a general idea of what these results are about,
but if someone asked you to prove them, you wouldn't have the faintest idea of
where to begin.   Sure, you could give the same overall outline that you
learned from the popular books; and in a general sort of way, you do have an
understanding.  But deep down inside, you know that there is a rigor that is
missing, that probably there are many subtle steps and pitfalls along the way,
and ultimately it seems you have to place your trust in the experts in the
field.  You don't like this; you want to be able to verify these results for
yourself.

So where do you go next?  As a first step, you decide to look up some of the
original papers on the theorems you are curious about, or better, obtain some
standard textbooks in the field.  You look up a theorem you want to
understand.  Sure enough, it's there, but it's expressed with strange
terms and odd symbols that mean absolutely nothing to you.  It might as well be written in
a foreign language you've never seen before, whose symbols are totally alien.
You look at the proof, and you haven't the foggiest notion what each step
means, much less how one step follows from another.  Well, obviously you have
a lot to learn if you want to understand this stuff.

You feel that you could probably understand it by
going back to college for another three to six years and getting a math
degree.  But that does not fit in with your career and the other things in
your life and would serve no practical purpose.  You decide to seek a quicker
path.  You figure you'll just trace your way back to the beginning, step by
step, as you would do with a computer program, until you understand it.  But
you quickly find that this is not possible, since you can't even understand
enough to know what you have to trace back to.

Maybe a different approach is in order---maybe you should start at the
beginning and work your way up.  First, you read the introduction to the book
to find out what the prerequisites are.  In a similar fashion, you trace your
way back through two or three more books, finally arriving at one that seems
to start at a beginning:  it lists the axioms of arithmetic.  ``Aha!'' you
naively think, ``This must be the starting point, the source of all mathematical
knowledge.'' Or at least the starting point for mathematics dealing with
numbers; you have to start somewhere and have no idea what the starting point
for other mathematics would be.  But the word ``axioms'' looks promising.  So
you eagerly read along and work through some elementary exercises at the
beginning of the book.  You feel vaguely bothered:  these
don't seem like axioms at all, at least not in the sense that you want to
think of axioms.  Axioms imply a starting point from which everything else can
be built up, according to precise rules specified in the axiom system.  Even
though you can understand the first few proofs in an informal way,
and are able to do some of the
exercises, it's hard to pin down precisely what the
rules are.   Sure, each step seems to follow logically from the others, but
exactly what does that mean?  Is the ``logic'' just a matter of common sense,
something vague that we all understand but can never quite state precisely?

You've spent a number of years, off and on, programming computers, and you
know that in the case of computer languages there is no question of what the
rules are---they are precise and crystal clear.  If you follow them, your
program will work, and if you don't, it won't.  No matter how complex a
program, it can always be broken down into simpler and simpler pieces, until
you can ultimately identify the bits that are moved around to perform a
specific function.  Some programs might require a lot of perseverance to
accomplish this, but if you focus on a specific portion of it, you don't even
necessarily have to know how the rest of it works. Shouldn't there be an
analogy in mathematics?

You decide to apply the ultimate test:  you ask yourself how a computer could
verify or ensure that the steps in these proofs follow from one another.
Certainly mathematics must be at least as precisely defined as a computer
language, if not more so; after all, computer science itself is based on it.
If you can get a computer to verify these proofs, then you should also be
able, in principle, to understand them yourself in a very crystal clear,
precise way.

You're in for a surprise:  you can conceive of no way to convert the
proofs, which are in English, to a form that the computer can understand.
The proofs are filled with phrases such as ``assume there exists a unique
$x$\ldots'' and ``given any $y$, let $z$ be the number such that\ldots''  This
isn't the kind of logic you are used to in computer programming, where
everything, even arithmetic, reduces to Boolean ones and zeroes if you care to
break it down sufficiently.  Even though you think you understand the proofs,
there seems to be some kind of higher reasoning involved rather than precise
rules that define how you manipulate the symbols in the axioms.  Whatever it
is, it just isn't obvious how you would express it to a computer, and the more
you think about it, the more puzzled and confused you get, to the point where
you even wonder whether {\em you} really understand it.  There's a lot more to
these axioms of arithmetic than meets the eye.

Nobody ever talked about this in school in your applied math and engineering
courses.  You just learned the rules they gave you, not quite understanding
how or why they worked, sometimes vaguely suspicious or uncertain of them, and
through homework problems and osmosis learned how to present solutions that
satisfied the instructor and earned you an ``A.''  Rarely did you actually
``prove'' anything in a rigorous way, and the math majors who did do stuff
like that seemed to be in a different world.

Of course, there are computer algebra programs that can do mathematics, and
rather impressively.  They can instantly solve the integrals that you
struggled with in freshman calculus, and do much, much more.  But when you
look at these programs, what you see is a big collection of algorithms and
techniques that evolved and were added to over time, along with some basic
software that manipulates symbols.  Each algorithm that is built in is the
result of someone's theorem whose proof is omitted; you just have to trust the
person who proved it and the person who programmed it in and hope there are no
bugs.\index{computer program bugs}  Somehow this doesn't seem to be the
essence of mathematics.  Although computer algebra systems can generate
theorems with amazing speed, they can't actually prove a single one of them.

After some puzzlement, you revisit some popular books on what mathematics is
all about.  Somewhere you read that all of mathematics is actually derived
from something called ``set theory.''  This is a little confusing, because
nowhere in the book that presented the axioms of arithmetic was there any
mention of set theory, or if there was, it seemed to be just a tool that helps
you describe things better---the set of even numbers, that sort of thing.  If
set theory is the basis for all mathematics, then why are additional axioms
needed for arithmetic?

Something is wrong but you're not sure what.  One of your friends is a pure
mathematician.  He knows he is unable to communicate to you what he does for a
living and seems to have little interest in trying.  You do know that for him,
proofs are what mathematics is all about. You ask him what a proof is, and he
essentially tells you that, while of course it's based on logic, really it's
something you learn by doing it over and over until you pick it up.  He refers
you to a book, {\em How to Read and Do Proofs} \cite{Solow}.\index{Solow,
Daniel}  Although this book helps you understand traditional informal proofs,
there is still something missing you can't seem to pin down yet.

You ask your friend how you would go about having a computer verify a proof.
At first he seems puzzled by the question; why would you want to do that?
Then he says it's not something that would make any sense to do, but he's
heard that you'd have to break the proof down into thousands or even millions
of individual steps to do such a thing, because the reasoning involved is at
such a high level of abstraction.  He says that maybe it's something you could
do up to a point, but the computer would be completely impractical once you
get into any meaningful mathematics.  There, the only way you can verify a
proof is by hand, and you can only acquire the ability to do this by
specializing in the field for a couple of years in grad school.  Anyway, he
thinks it all has to do with set theory, although he has never taken a formal
course in set theory but just learned what he needed as he went along.

You are intrigued and amazed.  Apparently a mathematician can grasp as a
single concept something that would take a computer a thousand or a million
steps to verify, and have complete confidence in it.  Each one of these
thousand or million steps must be absolutely correct, or else the whole proof
is meaningless.  If you added a million numbers by hand, would you trust the
result?  How do you really know that all these steps are correct, that there
isn't some subtle pitfall in one of these million steps, like a bug in a
computer program?\index{computer program bugs}  After all, you've read that
famous mathematicians have occasionally made mistakes, and you certainly know
you've made your share on your math homework problems in school.

You recall the analogy with a computer program.  Sure, you can understand what
a large computer program such as a word processor does, as a single high-level
concept or a small set of such concepts, but your ability to understand it in
no way ensures that the program is correct and doesn't have hidden bugs.  Even
if you wrote the program yourself you can't really know this; most large
programs that you've written have had bugs that crop up at some later date, no
matter how careful you tried to be while writing them.

OK, so now it seems the reason you can't figure out how to make a
computer verify proofs is because each step really corresponds to a
million small steps.  Well, you say, a computer can do a million
calculations in a second, so maybe it's still practical to do.  Now the
puzzle becomes how to figure out what the million steps are that each
English-language step corresponds to.  Your mathematician friend hasn't
a clue, but suggests that maybe you would find the answer by studying
set theory.  Actually, your friend thinks you're a little off the wall
for even wondering such a thing.  For him, this is not what mathematics
is all about.

The subject of set theory keeps popping up, so you decide it's
time to look it up.

You decide to start off on a careful footing, so you start reading a couple of
very elementary books on set theory.  A lot of it seems pretty obvious, like
intersections, subsets, and Venn diagrams.  You thumb through one of the
books; nowhere is anything about axioms mentioned. The other book relegates to
an appendix a brief discussion that mentions a set of axioms called
``Zermelo--Fraenkel set theory''\index{Zermelo--Fraenkel set theory} and states
them in English.  You look at them and have no idea what they really mean or
what you can do with them.  The comments in this appendix say that the purpose
of mentioning them is to expose you to the idea, but imply that they are not
necessary for basic understanding and that they are really the subject matter
of advanced treatments where fine points such as a certain paradox (Russell's
paradox\index{Russell's paradox}\footnote{Russell's paradox assumes that there
exists a set $S$ that is a collection of all sets that don't contain
themselves.  Now, either $S$ contains itself or it doesn't.  If it contains
itself, it contradicts its definition.  But if it doesn't contain itself, it
also contradicts its definition.  Russell's paradox is resolved in ZF set
theory by denying that such a set $S$ exists.}) are resolved.  Wait a
minute---shouldn't the axioms be a starting point, not an ending point?  If
there are paradoxes that arise without the axioms, how do you know you won't
stumble across one accidentally when using the informal approach?

And nowhere do these books describe how ``all of mathematics can be
derived from set theory'' which by now you've heard a few times.

You find a more advanced book on set theory.  This one actually lists the
axioms of ZF set theory in plain English on page one.  {\em Now} you think
your quest has ended and you've finally found the source of all mathematical
knowledge; you just have to understand what it means.  Here, in one place, is
the basis for all of mathematics!  You stare at the axioms in awe, puzzle over
them, memorize them, hoping that if you just meditate on them long enough they
will become clear.  Of course, you haven't the slightest idea how the rest of
mathematics is ``derived'' from them; in particular, if these are the axioms
of mathematics, then why do arithmetic, group theory, and so on need their own
axioms?

You start reading this advanced book carefully, pondering the meaning of every
word, because by now you're really determined to get to the bottom of this.
The first thing the book does is explain how the axioms came about, which was
to resolve Russell's paradox.\index{Russell's paradox}  In fact that seems to
be the main purpose of their existence; that they supposedly can be used to
derive all of mathematics seems irrelevant and is not even mentioned.  Well,
you go on.  You hope the book will explain to you clearly, step by step, how
to derive things from the axioms.  After all, this is the starting point of
mathematics, like a book that explains the basics of a computer programming
language.  But something is missing.  You find you can't even understand the
first proof or do the first exercise.  Symbols such as $\exists$ and $\forall$
permeate the page without any mention of where they came from or how to
manipulate them; the author assumes you are totally familiar with them and
doesn't even tell you what they mean.  By now you know that $\exists$ means
``there exists'' and $\forall$ means ``for all,'' but shouldn't the rules for
manipulating these symbols be part of the axioms?  You still have no idea
how you could even describe the axioms to a computer.

Certainly there is something much different here from the technical
literature you're used to reading.  A computer language manual almost
always explains very clearly what all the symbols mean, precisely what
they do, and the rules used for combining them, and you work your way up
from there.

After glancing at four or five other such books, you come to the realization
that there is another whole field of study that you need just to get to the
point at which you can understand the axioms of set theory.  The field is
called ``logic.''  In fact, some of the books did recommend it as a
prerequisite, but it just didn't sink in.  You assumed logic was, well, just
logic, something that a person with common sense intuitively understood.  Why
waste your time reading boring treatises on symbolic logic, the manipulation
of 1's and 0's that computers do, when you already know that?  But this is a
different kind of logic, quite alien to you.  The subject of {\sc nand} and
{\sc nor} gates is not even touched upon or in any case has to do with only a
very small part of this field.

So your quest continues.  Skimming through the first couple of introductory
books, you get a general idea of what logic is about and what quantifiers
(``for all,'' ``there exists'') mean, but you find their examples somewhat
trivial and mildly annoying (``all dogs are animals,'' ``some animals are
dogs,'' and such).  But all you want to know is what the rules are for
manipulating the symbols so you can apply them to set theory.  Some formulas
describing the relationships among quantifiers ($\exists$ and $\forall$) are
listed in tables, along with some verbal reasoning to justify them.
Presumably, if you want to find out if a formula is correct, you go through
this same kind of mental reasoning process, possibly using images of dogs and
animals. Intuitively, the formulas seem to make sense.  But when you ask
yourself, ``What are the rules I need to get a computer to figure out whether
this formula is correct?'', you still don't know.  Certainly you don't ask the
computer to imagine dogs and animals.

You look at some more advanced logic books.  Many of them have an introductory
chapter summarizing set theory, which turns out to be a prerequisite.  You
need logic to understand set theory, but it seems you also need set theory to
understand logic!  These books jump right into proving rather advanced
theorems about logic, without offering the faintest clue about where the logic
came from that allows them to prove these theorems.

Luckily, you come across an elementary book of logic that, halfway through,
after the usual truth tables and metaphors, presents in a clear, precise way
what you've been looking for all along: the axioms!  They're divided into
propositional calculus (also called sentential logic) and predicate calculus
(also called first-order logic),\index{first-order logic} with rules so simple
and crystal clear that now you can finally program a computer to understand
them.  Indeed, they're no harder than learning how to play a game of chess.
As far as what you seem to need is concerned, the whole book could have been
written in five pages!

{\em Now} you think you've found the ultimate source of mathematical
truth.  So---the axioms of mathematics consist of these axioms of logic,
together with the axioms of ZF set theory. (By now you've also been able to
figure out how to translate the ZF axioms from English into the
actual symbols of logic which you can now manipulate according to
precise, easy-to-understand rules.)

Of course, you still don't understand how ``all of mathematics can be
derived from set theory,'' but maybe this will reveal itself in due
course.

You eagerly set out to program the axioms and rules into a computer and start
to look at the theorems you will have to prove as the logic is developed.  All
sorts of important theorems start popping up:  the deduction
theorem,\index{deduction theorem} the substitution theorem,\index{substitution
theorem} the completeness theorem of propositional calculus,\index{first-order
logic!completeness} the completeness theorem of predicate calculus.  Uh-oh,
there seems to be trouble.  They all get harder and harder, and not one of
them can be derived with the axioms and rules of logic you've just been
handed.  Instead, they all require ``metalogic'' for their proofs, a kind of
mixture of logic and set theory that allows you to prove things {\em about}
the axioms and theorems of logic rather than {\em with} them.

You plow ahead anyway.  A month later, you've spent much of your
free time getting the computer to verify proofs in propositional calculus.
You've programmed in the axioms, but you've also had to program in the
deduction theorem, the substitution theorem, and the completeness theorem of
propositional calculus, which by now you've resigned yourself to treating as
rather complex additional axioms, since they can't be proved from the axioms
you were given.  You can now get the computer to verify and even generate
complete, rigorous, formal proofs\index{formal proof}.  Never mind that they
may have 100,000 steps---at least now you can have complete, absolute
confidence in them.  Unfortunately, the only theorems you have proved are
pretty trivial and you can easily verify them in a few minutes with truth
tables, if not by inspection.

It looks like your mathematician friend was right.  Getting the computer to do
serious mathematics with this kind of rigor seems almost hopeless.  Even
worse, it seems that the further along you get, the more ``axioms'' you have
to add, as each new theorem seems to involve additional ``metamathematical''
reasoning that hasn't been formalized, and none of it can be derived from the
axioms of logic.  Not only do the proofs keep growing exponentially as you get
further along, but the program to verify them keeps getting bigger and bigger
as you program in more ``metatheorems.''\index{metatheorem}\footnote{A
metatheorem is usually a statement that is too general to be directly provable
in a theory.  For example, ``if $n_1$, $n_2$, and $n_3$ are integers, then
$n_1+n_2+n_3$ is an integer'' is a theorem of number theory.  But ``for any
integer $k > 1$, if $n_1, \ldots, n_k$ are integers, then $n_1+\ldots +n_k$ is
an integer'' is a metatheorem, in other words a family of theorems, one for
every $k$.  The reason it is not a theorem is that the general sum $n_1+\ldots
+n_k$ (as a function of $k$) is not an operation that can be defined directly
in number theory.} The bugs\index{computer program bugs} that have cropped up
so far have already made you start to lose faith in the rigor you seem to have
achieved, and you know it's just going to get worse as your program gets larger.

\subsection{Mathematics and the Non-Specialist}

\begin{quote}
  {\em A real proof is not checkable by a machine, or even by any mathematician
not privy to the gestalt, the mode of thought of the particular field of
mathematics in which the proof is located.}
  \flushright\sc  Davis and Hersh\index{Davis, Phillip J.}
  \footnote{\cite{Davis}, p.~354.}\\
\end{quote}

The bulk of abstract or theoretical mathematics is ordinarily outside
the reach of anyone but a few specialists in each field who have completed
the necessary difficult internship in order to enter its coterie.  The
typical intelligent layperson has no reasonable hope of understanding much of
it, nor even the specialist mathematician of understanding other fields.  It
is like a foreign language that has no dictionary to look up the translation;
the only way you can learn it is by living in the country for a few years.  It
is argued that the effort involved in learning a specialty is a necessary
process for acquiring a deep understanding.  Of course, this is almost certainly
true if one is to make significant contributions to a field; in particular,
``doing'' proofs is probably the most important part of a mathematician's
training.  But is it also necessary to deny outsiders access to it?  Is it
necessary that abstract mathematics be so hard for a layperson to grasp?

A computer normally is of no help whatsoever.  Most published proofs are
actually just series of hints written in an informal style that requires
considerable knowledge of the field to understand.  These are the ``real
proofs'' referred to by Davis and Hersh.\index{informal proof}  There is an
implicit understanding that, in principle, such a proof could be converted to
a complete formal proof\index{formal proof}.  However, it is said that no one
would ever attempt such a conversion, even if they could, because that would
presumably require millions of steps (Section~\ref{dream}).  Unfortunately the
informal style automatically excludes the understanding of the proof
by anyone who hasn't gone through the necessary apprenticeship. The
best that the intelligent layperson can do is to read popular books about deep
and famous results; while this can be helpful, it can also be misleading, and
the lack of detail usually leaves the reader with no ability whatsoever to
explore any aspect of the field being described.

The statements of theorems often use sophisticated notation that makes them
inaccessible to the non-specialist.  For a non-specialist who wants to achieve
a deeper understanding of a proof, the process of tracing definitions and
lemmas back through their hierarchy\index{hierarchy} quickly becomes confusing
and discouraging.  Textbooks are usually written to train mathematicians or to
communicate to people who are already mathematicians, and large gaps in proofs
are often left as exercises to the reader who is left at an impasse if he or
she becomes stuck.

I believe that eventually computers will enable non-specialists and even
intelligent laypersons to follow almost any mathematical proof in any field.
Metamath is an attempt in that direction.  If all of mathematics were as
easily accessible as a computer programming language, I could envision
computer programmers and hobbyists who otherwise lack mathematical
sophistication exploring and being amazed by the world of theorems and proofs
in obscure specialties, perhaps even coming up with results of their own.  A
tremendous advantage would be that anyone could experiment with conjectures in
any field---the computer would offer instant feedback as to whether
an inference step was correct.

Mathematicians sometimes have to put up with the annoyance of
cranks\index{cranks} who lack a fundamental understanding of mathematics but
insist that their ``proofs'' of, say, Fermat's Last Theorem\index{Fermat's
Last Theorem} be taken seriously.  I think part of the problem is that these
people are misled by informal mathematical language, treating it as if they
were reading ordinary expository English and failing to appreciate the
implicit underlying rigor.  Such cranks are rare in the field of computers,
because computer languages are much more explicit, and ultimately the proof is
in whether a computer program works or not.  With easily accessible
computer-based abstract mathematics, a mathematician could say to a crank,
``don't bother me until you've demonstrated your claim on the computer!''

% 22-May-04 nm
% Attempt to move De Millo quote so it doesn't separate from attribution
% CHANGE THIS NUMBER (AND ELIMINATE IF POSSIBLE) WHEN ABOVE TEXT CHANGES
\vspace{-0.5em}

\subsection{An Impossible Dream?}\label{dream}

\begin{quote}
  {\em Even quite basic theorems would demand almost unbelievably vast
  books to display their proofs.}
    \flushright\sc  Robert E. Edwards\footnote{\cite{Edwards}, p.~68.}\\
\end{quote}\index{Edwards, Robert E.}

\begin{quote}
  {\em Oh, of course no one ever really {\em does} it.  It would take
  forever!  You just show that you could do it, that's sufficient.}
    \flushright\sc  ``The Ideal Mathematician''
    \index{Davis, Phillip J.}\footnote{\cite{Davis},
p.~40.}\\
\end{quote}

\begin{quote}
  {\em There is a theorem in the primitive notation of set theory that
  corresponds to the arithmetic theorem `$1000+2000=3000$'.  The formula
  would be forbiddingly long\ldots even if [one] knows the definitions
  and is asked to simplify the long formula according to them, chances are
  he will make errors and arrive at some incorrect result.}
    \flushright\sc  Hao Wang\footnote{\cite{Wang}, p.~140.}\\
\end{quote}\index{Wang, Hao}

% 22-May-04 nm
% Attempt to move De Millo quote so it doesn't separate from attribution
% CHANGE THIS NUMBER (AND ELIMINATE IF POSSIBLE) WHEN ABOVE TEXT CHANGES
\vspace{-0.5em}

\begin{quote}
  {\em The {\em Principia Mathematica} was the crowning achievement of the
  formalists.  It was also the deathblow of the formalist view.\ldots
  {[Rus\-sell]} failed, in three enormous volumes, to get beyond the elementary
  facts of arithmetic.  He showed what can be done in principle and what
  cannot be done in practice.  If the mathematical process were really
  one of strict, logical progression, we would still be counting our
  fingers.\ldots
  One theoretician estimates\ldots that a demonstration of one of
  Ramanujan's conjectures assuming set theory and elementary analysis would
  take about two thousand pages; the length of a deduction from first principles
  is nearly in\-con\-ceiv\-a\-ble\ldots The probabilists argue that\ldots any
  very long proof can at best be viewed as only probably correct\ldots}
  \flushright\sc Richard de Millo et. al.\footnote{\cite{deMillo}, pp.~269,
  271.}\\
\end{quote}\index{de Millo, Richard}

A number of writers have conveyed the impression that the kind of absolute
rigor provided by Metamath\index{Metamath} is an impossible dream, suggesting
that a complete, formal verification\index{formal proof} of a typical theorem
would take millions of steps in untold volumes of books.  Even if it could be
done, the thinking sometimes goes, all meaning would be lost in such a
monstrous, tedious verification.\index{informal proof}\index{proof length}

These writers assume, however, that in order to achieve the kind of complete
formal verification they desire one must break down a proof into individual
primitive steps that make direct reference to the axioms.  This is
not necessary.  There is no reason not to make use of previously proved
theorems rather than proving them over and over.

Just as important, definitions\index{definition} can be introduced along
the way, allowing very complex formulas to be represented with few
symbols.  Not doing this can lead to absurdly long formulas.  For
example, the mere statement of
G\"{o}del's incompleteness theorem\index{G\"{o}del's
incompleteness theorem}, which can be expressed with a small number of
defined symbols, would require about 20,000 primitive symbols to express
it.\index{Boolos, George S.}\footnote{George S.\ Boolos, lecture at
Massachusetts Institute of Technology, spring 1990.} An extreme example
is Bourbaki's\label{bourbaki} language for set theory, which requires
4,523,659,424,929 symbols plus 1,179,618,517,981 disambiguatory links
(lines connecting symbol pairs, usually drawn below or above the
formula) to express the number
``one'' \cite{Mathias}.\index{Mathias, Adrian R. D.}\index{Bourbaki,
Nicolas}
% http://www.dpmms.cam.ac.uk/~ardm/

A hierarchy\index{hierarchy} of theorems and definitions permits an
exponential growth in the formula sizes and primitive proof steps to be
described with only a linear growth in the number of symbols used.  Of course,
this is how ordinary informal mathematics is normally done anyway, but with
Metamath\index{Metamath} it can be done with absolute rigor and precision.

\subsection{Beauty}


\begin{quote}
  {\em No one shall be able to drive us from the paradise that Cantor has
created for us.}
   \flushright\sc  David Hilbert\footnote{As quoted in \cite{Moore}, p.~131.}\\
\end{quote}\index{Hilbert, David}

\needspace{3\baselineskip}
\begin{quote}
  {\em Mathematics possesses not only truth, but some supreme beauty ---a
  beauty cold and austere, like that of a sculpture.}
    \flushright\sc  Bertrand
    Russell\footnote{\cite{Russell}.}\\
\end{quote}\index{Russell, Bertrand}

\begin{quote}
  {\em Euclid alone has looked on Beauty bare.}
  \flushright\sc Edna Millay\footnote{As quoted in \cite{Davis}, p.~150.}\\
\end{quote}\index{Millay, Edna}

For most people, abstract mathematics is distant, strange, and
incomprehensible.  Many popular books have tried to convey some of the sense
of beauty in famous theorems.  But even an intelligent layperson is left with
only a general idea of what a theorem is about and is hardly given the tools
needed to make use of it.  Traditionally, it is only after years of arduous
study that one can grasp the concepts needed for deep understanding.
Metamath\index{Metamath} allows you to approach the proof of the theorem from
a quite different perspective, peeling apart the formulas and definitions
layer by layer until an entirely different kind of understanding is achieved.
Every step of the proof is there, pieced together with absolute precision and
instantly available for inspection through a microscope with a magnification
as powerful as you desire.

A proof in itself can be considered an object of beauty.  Constructing an
elegant proof is an art.  Once a famous theorem has been proved, often
considerable effort is made to find simpler and more easily understood
proofs.  Creating and communicating elegant proofs is a major concern of
mathematicians.  Metamath is one way of providing a common language for
archiving and preserving this information.

The length of a proof can, to a certain extent, be considered an
objective measure of its ``beauty,'' since shorter proofs are usually
considered more elegant.  In the set theory database
\texttt{set.mm}\index{set theory database (\texttt{set.mm})}%
\index{Metamath Proof Explorer}
provided with Metamath, one goal was to make all proofs as short as possible.

\needspace{4\baselineskip}
\subsection{Simplicity}

\begin{quote}
  {\em God made man simple; man's complex problems are of his own
  devising.}
    \flushright\sc Eccles. 7:29\footnote{Jerusalem Bible.}\\
\end{quote}\index{Bible}

\needspace{3\baselineskip}
\begin{quote}
  {\em God made integers, all else is the work of man.}
    \flushright\sc Leopold Kronecker\footnote{{\em Jahresbericht
	der Deutschen Mathematiker-Vereinigung }, vol. 2, p. 19.}\\
\end{quote}\index{Kronecker, Leopold}

\needspace{3\baselineskip}
\begin{quote}
  {\em For what is clear and easily comprehended attracts; the
  complicated repels.}
    \flushright\sc David Hilbert\footnote{As quoted in \cite{deMillo},
p.~273.}\\
\end{quote}\index{Hilbert, David}

The Metamath\index{Metamath} language is simple and Spartan.  Metamath treats
all mathematical expressions as simple sequences of symbols, devoid of meaning.
The higher-level or ``metamathematical'' notions underlying Metamath are about
as simple as they could possibly be.  Each individual step in a proof involves
a single basic concept, the substitution of an expression for a variable, so
that in principle almost anyone, whether mathematician or not, can
completely understand how it was arrived at.

In one of its most basic applications, Metamath\index{Metamath} can be used to
develop the foundations of mathematics\index{foundations of mathematics} from
the very beginning.  This is done in the set theory database that is provided
with the Metamath package and is the subject matter
of Chapter~\ref{fol}. Any language (a metalanguage\index{metalanguage})
used to describe mathematics (an object language\index{object language}) must
have a mathematical content of its own, but it is desirable to keep this
content down to a bare minimum, namely that needed to make use of the
inference rules specified by the axioms.  With any metalanguage there is a
``chicken and egg'' problem somewhat like circular reasoning:  you must assume
the validity of the mathematics of the metalanguage in order to prove the
validity of the mathematics of the object language.  The mathematical content
of Metamath itself is quite limited.  Like the rules of a game of chess, the
essential concepts are simple enough so that virtually anyone should be able to
understand them (although that in itself will not let you play like
a master).  The symbols that Metamath manipulates do not in themselves
have any intrinsic meaning.  Your interpretation of the axioms that you supply
to Metamath is what gives them meaning.  Metamath is an attempt to strip down
mathematical thought to its bare essence and show you exactly how the symbols
are manipulated.

Philosophers and logicians, with various motivations, have often thought it
important to study ``weak'' fragments of logic\index{weak logic}
\cite{Anderson}\index{Anderson, Alan Ross} \cite{MegillBunder}\index{Megill,
Norman}\index{Bunder, Martin}, other unconventional systems of logic (such as
``modal'' logic\index{modal logic} \cite[ch.\ 27]{Boolos}\index{Boolos, George
S.}), and quantum logic\index{quantum logic} in physics
\cite{Pavicic}\index{Pavi{\v{c}}i{\'{c}}, M.}.  Metamath\index{Metamath}
provides a framework in which such systems can be expressed, with an absolute
precision that makes all underlying metamathematical assumptions rigorous and
crystal clear.

Some schools of philosophical thought, for example
intuitionism\index{intuitionism} and constructivism\index{constructivism},
demand that the notions underlying any mathematical system be as simple and
concrete as possible.  Metamath should meet the requirements of these
philosophies.  Metamath must be taught the symbols, axioms\index{axiom}, and
rules\index{rule} for a specific theory, from the skeptical (such as
intuitionism\index{intuitionism}\footnote{Intuitionism does not accept the law
of excluded middle (``either something is true or it is not true'').  See
\cite[p.~xi]{Tymoczko}\index{Tymoczko, Thomas} for discussion and references
on this topic.  Consider the theorem, ``There exist irrational numbers $a$ and
$b$ such that $a^b$ is rational.''  An intuitionist would reject the following
proof:  If $\sqrt{2}^{\sqrt{2}}$ is rational, we are done.  Otherwise, let
$a=\sqrt{2}^{\sqrt{2}}$ and $b=\sqrt{2}$. Then $a^b=2$, which is rational.})
to the bold (such as the axiom of choice in set theory\footnote{The axiom of
choice\index{Axiom of Choice} asserts that given any collection of pairwise
disjoint nonempty sets, there exists a set that has exactly one element in
common with each set of the collection.  It is used to prove many important
theorems in standard mathematics.  Some philosophers object to it because it
asserts the existence of a set without specifying what the set contains
\cite[p.~154]{Enderton}\index{Enderton, Herbert B.}.  In one foundation for
mathematics due to Quine\index{Quine, Willard Van Orman}, that has not been
otherwise shown to be inconsistent, the axiom of choice turns out to be false
\cite[p.~23]{Curry}\index{Curry, Haskell B.}.  The \texttt{show
trace{\char`\_}back} command of the Metamath program allows you to find out
whether the axiom of choice, or any other axiom, was assumed by a
proof.}\index{\texttt{show trace{\char`\_}back} command}).

The simplicity of the Metamath language lets the algorithm (computer program)
that verifies the validity of a Metamath proof be straightforward and
robust.  You can have confidence that the theorems it verifies really can be
derived from your axioms.

\subsection{Rigor}

\begin{quote}
  {\em Rigor became a goal with the Greeks\ldots But the efforts to
  pursue rigor to the utmost have led to an impasse in which there is
  no longer any agreement on what it really means.  Mathematics remains
  alive and vital, but only on a pragmatic basis.}
    \flushright\sc  Morris Kline\footnote{\cite{Kline}, p.~1209.}\\
\end{quote}\index{Kline, Morris}

Kline refers to a much deeper kind of rigor than that which we will discuss in
this section.  G\"{o}del's incompleteness theorem\index{G\"{o}del's
incompleteness theorem} showed that it is impossible to achieve absolute rigor
in standard mathematics because we can never prove that mathematics is
consistent (free from contradictions).\index{consistent theory}  If
mathematics is consistent, we will never know it, but must rely on faith.  If
mathematics is inconsistent, the best we can hope for is that some clever
future mathematician will discover the inconsistency.  In this case, the
axioms would probably be revised slightly to eliminate the inconsistency, as
was done in the case of Russell's paradox,\index{Russell's paradox} but the
bulk of mathematics would probably not be affected by such a discovery.
Russell's paradox, for example, did not affect most of the remarkable results
achieved by 19th-century and earlier mathematicians.  It mainly invalidated
some of Gottlob Frege's\index{Frege, Gottlob} work on the foundations of
mathematics in the late 1800's; in fact Frege's work inspired Russell's
discovery.  Despite the paradox, Frege's work contains important concepts that
have significantly influenced modern logic.  Kline's {\em Mathematics, The
Loss of Certainty} \cite{Klinel}\index{Kline, Morris} has an interesting
discussion of this topic.

What {\em can} be achieved with absolute certainty\index{certainty} is the
knowledge that if we assume the axioms are consistent and true, then the
results derived from them are true.  Part of the beauty of mathematics is that
it is the one area of human endeavor where absolute certainty can be achieved
in this sense.  A mathematical truth will remain such for eternity.  However,
our actual knowledge of whether a particular statement is a mathematical truth
is only as certain as the correctness of the proof that establishes it.  If
the proof of a statement is questionable or vague, we can't have absolute
confidence in the truth that the statement claims.

Let us look at some traditional ways of expressing proofs.

Except in the field of formal logic\index{formal logic}, almost all
traditional proofs in mathematics are really not proofs at all, but rather
proof outlines or hints as to how to go about constructing the proof.  Many
gaps\index{gaps in proofs} are left for the reader to fill in. There are
several reasons for this.  First, it is usually assumed in mathematical
literature that the person reading the proof is a mathematician familiar with
the specialty being described, and that the missing steps are obvious to such
a reader or at least that the reader is capable of filling them in.  This
attitude is fine for professional mathematicians in the specialty, but
unfortunately it often has the drawback of cutting off the rest of the world,
including mathematicians in other specialties, from understanding the proof.
We discussed one possible resolution to this on p.~\pageref{envision}.
Second, it is often assumed that a complete formal proof\index{formal proof}
would require countless millions of symbols (Section~\ref{dream}). This might
be true if the proof were to be expressed directly in terms of the axioms of
logic and set theory,\index{set theory} but it is usually not true if we allow
ourselves a hierarchy\index{hierarchy} of definitions and theorems to build
upon, using a notation that allows us to introduce new symbols, definitions,
and theorems in a precisely specified way.

Even in formal logic,\index{formal logic} formal proofs\index{formal proof}
that are considered complete still contain hidden or implicit information.
For example, a ``proof'' is usually defined as a sequence of
wffs,\index{well-formed formula (wff)}\footnote{A {\em wff} or well-formed
formula is a mathematical expression (string of symbols) constructed according
to some precise rules.  A formal mathematical system\index{formal system}
contains (1) the rules for constructing syntactically correct
wffs,\index{syntax rules} (2) a list of starting wffs called
axioms,\index{axiom} and (3) one or more rules prescribing how to derive new
wffs, called theorems\index{theorem}, from the axioms or previously derived
theorems.  An example of such a system is contained in
Metamath's\index{Metamath} set theory database, which defines a formal
system\index{formal system} from which all of standard mathematics can be
derived.  Section~\ref{startf} steps you through a complete example of a formal
system, and you may want to skim it now if you are unfamiliar with the
concept.} each of which is an axiom or follows from a rule applied to previous
wffs in the sequence.  The implicit part of the proof is the algorithm by
which a sequence of symbols is verified to be a valid wff, given the
definition of a wff.  The algorithm in this case is rather simple, but for a
computer to verify the proof,\index{automated proof verification} it must have
the algorithm built into its verification program.\footnote{It is possible, of
course, to specify wff construction syntax outside of the program itself
with a suitable input language (the Metamath language being an example), but
some proof-verification or theorem-proving programs lack the ability to extend
wff syntax in such a fashion.} If one deals exclusively with axioms and
elementary wffs, it is straightforward to implement such an algorithm.  But as
more and more definitions are added to the theory in order to make the
expression of wffs more compact, the algorithm becomes more and more
complicated.  A computer program that implements the algorithm becomes larger
and harder to understand as each definition is introduced, and thus more prone
to bugs.\index{computer program bugs}  The larger the program, the
more suspicious the mathematician may be about
the validity of its algorithms.  This is especially true because
computer programs are inherently hard to follow to begin with, and few people
enjoy verifying them manually in detail.

Metamath\index{Metamath} takes a different approach.  Metamath's ``knowledge''
is limited to the ability to substitute variables for expressions, subject to
some simple constraints.  Once the basic algorithm of Metamath is assumed to
be debugged, and perhaps independently confirmed, it
can be trusted once and for all.  The information that Metamath needs to
``understand'' mathematics is contained entirely in the body of knowledge
presented to Metamath.  Any errors in reasoning can only be errors in the
axioms or definitions contained in this body of knowledge.  As a
``constructive'' language\index{constructive language} Metamath has no
conditional branches or loops like the ones that make computer programs hard
to decipher; instead, the language can only build new sequences of symbols
from earlier sequences  of symbols.

The simplicity of the rules that underlie Metamath not only makes Metamath
easy to learn but also gives Metamath a great deal of flexibility. For
example, Metamath is not limited to describing standard first-order
logic\index{first-order logic}; higher-order logics\index{higher-order logic}
and fragments of logic\index{weak logic} can be described just as easily.
Metamath gives you the freedom to define whatever wff notation you prefer; it
has no built-in conception of the syntax of a wff.\index{well-formed formula
(wff)}  With suitable axioms and definitions, Metamath can even describe and
prove things about itself.\index{Metamath!self-description}  (John
Harrison\index{Harrison, John} discusses the ``reflection''
principle\index{reflection principle} involved in self-descriptive systems in
\cite{Harrison}.)

The flexibility of Metamath requires that its proofs specify a lot of detail,
much more than in an ordinary ``formal'' proof.\index{formal proof}  For
example, in an ordinary formal proof, a single step consists of displaying the
wff that constitutes that step.  In order for a computer program to verify
that the step is acceptable, it first must verify that the symbol sequence
being displayed is an acceptable wff.\index{automated proof verification} Most
proof verifiers have at least basic wff syntax built into their programs.
Metamath has no hard-wired knowledge of what constitutes a wff built into it;
instead every wff must be explicitly constructed based on rules defining wffs
that are present in a database.  Thus a single step in an ordinary formal
proof may be correspond to many steps in a Metamath proof. Despite the larger
number of steps, though, this does not mean that a Metamath proof must be
significantly larger than an ordinary formal proof. The reason is that since
we have constructed the wff from scratch, we know what the wff is, so there is
no reason to display it.  We only need to refer to a sequence of statements
that construct it.  In a sense, the display of the wff in an ordinary formal
proof is an implicit proof of its own validity as a wff; Metamath just makes
the proof explicit. (Section~\ref{proof} describes Metamath's proof notation.)

\section{Computers and Mathematicians}

\begin{quote}
  {\em The computer is important, but not to mathematics.}
    \flushright\sc  Paul Halmos\footnote{As quoted in \cite{Albers}, p.~121.}\\
\end{quote}\index{Halmos, Paul}

Pure mathematicians have traditionally been indifferent to computers, even to
the point of disdain.\index{computers and pure mathematics}  Computer science
itself is sometimes considered to fall in the mundane realm of ``applied''
mathematics, perhaps essential for the real world but intellectually unexciting
to those who seek the deepest truths in mathematics.  Perhaps a reason for this
attitude towards computers is that there is little or no computer software that
meets their needs, and there may be a general feeling that such software could
not even exist.  On the one hand, there are the practical computer algebra
systems, which can perform amazing symbolic manipulations in algebra and
calculus,\index{computer algebra system} yet can't prove the simplest
existence theorem, if the idea of a proof is present at all.  On the other
hand, there are specialized automated theorem provers that technically speaking
may generate correct proofs.\index{automated theorem proving}  But sometimes
their specialized input notation may be cryptic and their output perceived to
be long, inelegant, incomprehensible proofs.    The output
may be viewed with suspicion, since the program that generates it tends to be
very large, and its size increases the potential for bugs\index{computer
program bugs}.  Such a proof may be considered trustworthy only if
independently verified and ``understood'' by a human, but no one wants to
waste their time on such a boring, unrewarding chore.



\needspace{4\baselineskip}
\subsection{Trusting the Computer}

\begin{quote}
  {\em \ldots I continue to find the quasi-empirical interpretation of
  computer proofs to be the more plausible.\ldots Since not
  everything that claims to be a computer proof can be
  accepted as valid, what are the mathematical criteria for acceptable
  computer proofs?}
    \flushright\sc  Thomas Tymoczko\footnote{\cite{Tymoczko}, p.~245.}\\
\end{quote}\index{Tymoczko, Thomas}

In some cases, computers have been essential tools for proving famous
theorems.  But if a proof is so long and obscure that it can be verified in a
practical way only with a computer, it is vaguely felt to be suspicious.  For
example, proving the famous four-color theorem\index{four-color
theorem}\index{proof length} (``a map needs no more than four colors to
prevent any two adjacent countries from having the same color'') can presently
only be done with the aid of a very complex computer program which originally
required 1200 hours of computer time. There has been considerable debate about
whether such a proof can be trusted and whether such a proof is ``real''
mathematics \cite{Swart}\index{Swart, E. R.}.\index{trusting computers}

However, under normal circumstances even a skeptical mathematician would have a
great deal of confidence in the result of multiplying two numbers on a pocket
calculator, even though the precise details of what goes on are hidden from its
user.  Even the verification on a supercomputer that a huge number is prime is
trusted, especially if there is independent verification; no one bothers to
debate the philosophical significance of its ``proof,'' even though the actual
proof would be so large that it would be completely impractical to ever write
it down on paper.  It seems that if the algorithm used by the computer is
simple enough to be readily understood, then the computer can be trusted.

Metamath\index{Metamath} adopts this philosophy.  The simplicity of its
language makes it easy to learn, and because of its simplicity one can have
essentially absolute confidence that a proof is correct. All axioms, rules, and
definitions are available for inspection at any time because they are defined
by the user; there are no hidden or built-in rules that may be prone to subtle
bugs\index{computer program bugs}.  The basic algorithm at the heart of
Metamath is simple and fixed, and it can be assumed to be bug-free and robust
with a degree of confidence approaching certainty.
Independently written implementations of the Metamath verifier
can reduce any residual doubt on the part of a skeptic even further;
there are now over a dozen such implementations, written by many people.

\subsection{Trusting the Mathematician}\label{trust}

\begin{quote}
  {\em There is no Algebraist nor Mathematician so expert in his science, as
  to place entire confidence in any truth immediately upon his discovery of it,
  or regard it as any thing, but a mere probability.  Every time he runs over
  his proofs, his confidence encreases; but still more by the approbation of
  his friends; and is rais'd to its utmost perfection by the universal assent
  and applauses of the learned world.}
  \flushright\sc David Hume\footnote{{\em A Treatise of Human Nature}, as
  quoted in \cite{deMillo}, p.~267.}\\
\end{quote}\index{Hume, David}

\begin{quote}
  {\em Stanislaw Ulam estimates that mathematicians publish 200,000 theorems
  every year.  A number of these are subsequently contradicted or otherwise
  disallowed, others are thrown into doubt, and most are ignored.}
  \flushright\sc Richard de Millo et. al.\footnote{\cite{deMillo}, p.~269.}\\
\end{quote}\index{Ulam, Stanislaw}

Whether or not the computer can be trusted, humans  of course will occasionally
err. Only the most memorable proofs get independently verified, and of these
only a handful of truly great ones achieve the status of being ``known''
mathematical truths that are used without giving a second thought to their
correctness.

There are many famous examples of incorrect theorems and proofs in
mathematical literature.\index{errors in proofs}

\begin{itemize}
\item There have been thousands of purported proofs of Fermat's Last
Theorem\index{Fermat's Last Theorem} (``no integer solutions exist to $x^n +
y^n = z^n$ for $n > 2$''), by amateurs, cranks, and well-regarded
mathematicians \cite[p.~5]{Stark}\index{Stark, Harold M}.  Fermat wrote a note
in his copy of Bachet's {\em Diophantus} that he found ``a truly marvelous
proof of this theorem but this margin is too narrow to contain it''
\cite[p.~507]{Kramer}.  A recent, much publicized proof by Yoichi
Miyaoka\index{Miyaoka, Yoichi} was shown to be incorrect ({\em Science News},
April 9, 1988, p.~230).  The theorem was finally proved by Andrew
Wiles\index{Wiles, Andrew} ({\em Science News}, July 3, 1993, p.~5), but it
initially had some gaps and took over a year after its announcement to be
checked thoroughly by experts.  On Oct. 25, 1994, Wiles announced that the last
gap found in his proof had been filled in.
  \item In 1882, M. Pasch discovered that an axiom was omitted from Euclid's
formulation of geometry\index{Euclidean geometry}; without it, the proofs of
important theorems of Euclid are not valid.  Pasch's axiom\index{Pasch's
axiom} states that a line that intersects one side of a triangle must also
intersect another side, provided that it does not touch any of the triangle's
vertices.  The omission of Pasch's axiom went unnoticed for 2000
years \cite[p.~160]{Davis}, in spite of (one presumes) the thousands of
students, instructors, and mathematicians who studied Euclid.
  \item The first published proof of the famous Schr\"{o}der--Bernstein
theorem\index{Schr\"{o}der--Bernstein theorem} in set theory was incorrect
\cite[p.~148]{Enderton}\index{Enderton, Herbert B.}.  This theorem states
that if there exists a 1-to-1 function\footnote{A {\em set}\index{set} is any
collection of objects. A {\em function}\index{function} or {\em
mapping}\index{mapping} is a rule that assigns to each element of one set
(called the function's {\em domain}\index{domain}) an element from another
set.} from set $A$ into set $B$ and vice-versa, then sets $A$ and $B$ have
a 1-to-1 correspondence.  Although it sounds simple and obvious,
the standard proof is quite long and complex.
  \item In the early 1900's, Hilbert\index{Hilbert, David} published a
purported proof of the continuum hypothesis\index{continuum hypothesis}, which
was eventually established as unprovable by Cohen\index{Cohen, Paul} in 1963
\cite[p.~166]{Enderton}.  The continuum hypothesis states that no
infinity\index{infinity} (``transfinite cardinal number'')\index{cardinal,
transfinite} exists whose size (or ``cardinality''\index{cardinality}) is
between the size of the set of integers and the size of the set of real
numbers.  This hypothesis originated with German mathematician Georg
Cantor\index{Cantor, Georg} in the late 1800's, and his inability to prove it
is said to have contributed to mental illness that afflicted him in his later
years.
  \item An incorrect proof of the four-color theorem\index{four-color theorem}
was published by Kempe\index{Kempe, A. B.} in 1879
\cite[p.~582]{Courant}\index{Courant, Richard}; it stood for 11 years before
its flaw was discovered.  This theorem states that any map can be colored
using only four colors, so that no two adjacent countries have the same
color.  In 1976 the theorem was finally proved by the famous computer-assisted
proof of Haken, Appel, and Koch \cite{Swart}\index{Appel, K.}\index{Haken,
W.}\index{Koch, K.}.  Or at least it seems that way.  Mathematician
H.~S.~M.~Coxeter\index{Coxeter, H. S. M.} has doubts \cite[p.~58]{Davis}:  ``I
have a feeling that that is an untidy kind of use of the computers, and the more
you correspond with Haken and Appel, the more shaky you seem to be.''
  \item Many false ``proofs'' of the Poincar\'{e}
conjecture\index{Poincar\'{e} conjecture} have been proposed over the years.
This conjecture states that any object that mathematically behaves like a
three-dimensional sphere is a three-dimensional sphere topologically,
regardless of how it is distorted.  In March 1986, mathematicians Colin
Rourke\index{Rourke, Colin} and Eduardo R\^{e}go\index{R\^{e}go, Eduardo}
caused  a stir in the mathematical community by announcing that they had found
a proof; in November of that year the proof was found to be false \cite[p.
218]{PetersonI}.  It was finally proved in 2003 by Grigory Perelman
\label{poincare}\index{Szpiro, George}\index{Perelman, Grigory}\cite{Szpiro}.
 \end{itemize}

Many counterexamples to ``theorems'' in recent mathematical
literature related to Clifford algebras\index{Clifford algebras}
 have been found by Pertti
Lounesto (who passed away in 2002).\index{Lounesto, Pertti}
See the web page \url{http://mathforum.org/library/view/4933.html}.
% http://users.tkk.fi/~ppuska/mirror/Lounesto/counterexamples.htm

One of the purposes of Metamath\index{Metamath} is to allow proofs to be
expressed with absolute precision.  Developing a proof in the Metamath
language can be challenging, because Metamath will not permit even the
tiniest mistake.\index{errors in proofs}  But once the proof is created, its
correctness can be trusted immediately, without having to depend on the
process of peer review for confirmation.

\section{The Use of Computers in Mathematics}

\subsection{Computer Algebra Systems}

For the most part, you will find that Metamath\index{Metamath} is not a
practical tool for manipulating numbers.  (Even proving that $2 + 2 = 4$, if
you start with set theory, can be quite complex!)  Several commercial
mathematics packages are quite good at arithmetic, algebra, and calculus, and
as practical tools they are invaluable.\index{computer algebra system} But
they have no notion of proof, and cannot understand statements starting with
``there exists such and such...''.

Software packages such as Mathematica \cite{Wolfram}\index{Mathematica} do not
concern themselves with proofs but instead work directly with known results.
These packages primarily emphasize heuristic rules such as the substitution of
equals for equals to achieve simpler expressions or expressions in a different
form.  Starting with a rich collection of built-in rules and algorithms, users
can add to the collection by means of a powerful programming language.
However, results such as, say, the existence of a certain abstract object
without displaying the actual object cannot be expressed (directly) in their
languages.  The idea of a proof from a small set of axioms is absent.  Instead
this software simply assumes that each fact or rule you add to the built-in
collection of algorithms is valid.  One way to view the software is as a large
collection of axioms from which the software, with certain goals, attempts to
derive new theorems, for example equating a complex expression with a simpler
equivalent. But the terms ``theorem''\index{theorem} and
``proof,''\index{proof} for example, are not even mentioned in the index of
the user's manual for Mathematica.\index{Mathematica and proofs}  What is also
unsatisfactory from a philosophical point of view is that there is no way to
ensure the validity of the results other than by trusting the writer of each
application module or tediously checking each module by hand, similar to
checking a computer program for bugs.\index{computer program
bugs}\footnote{Two examples illustrate why the knowledge database of computer
algebra systems should sometimes be regarded with a certain caution.  If you
ask Mathematica (version 3.0) to \texttt{Solve[x\^{ }n + y\^{ }n == z\^{ }n , n]}
it will respond with \texttt{\{\{n-\char`\>-2\}, \{n-\char`\>-1\},
\{n-\char`\>1\}, \{n-\char`\>2\}\}}. In other words, Mathematica seems to
``know'' that Fermat's Last Theorem\index{Fermat's Last Theorem} is true!  (At
the time this version of Mathematica was released this fact was unknown.)  If
you ask Maple\index{Maple} to \texttt{solve(x\^{ }2 = 2\^{ }x)} then
\texttt{simplify(\{"\})}, it returns the solution set \texttt{\{2, 4\}}, apparently
unaware that $-0.7666647$\ldots is also a solution.} While of course extremely
valuable in applied mathematics, computer algebra systems tend to be of little
interest to the theoretical mathematician except as aids for exploring certain
specific problems.

Because of possible bugs, trusting the output of a computer algebra system for
use as theorems in a proof-verifier would defeat the latter's goal of rigor.
On the other hand, a fact such that a certain relatively large number is
prime, while easy for a computer algebra system to derive, might have a long,
tedious proof that could overwhelm a proof-verifier. One approach for linking
computer algebra systems to a proof-verifier while retaining the advantages of
both is to add a hypothesis to each such theorem indicating its source.  For
example, a constant {\sc maple} could indicate the theorem came from the Maple
package, and would mean ``assuming Maple is consistent, then\ldots''  This and
many other topics concerning the formalization of mathematics are discussed in
John Harrison's\index{Harrison, John} very interesting
PhD thesis~\cite{Harrison-thesis}.

\subsection{Automated Theorem Provers}\label{theoremprovers}

A mathematical theory is ``decidable''\index{decidable theory} if a mechanical
method or algorithm exists that is guaranteed to determine whether or not a
particular formula is a theorem.  Among the few theories that are decidable is
elementary geometry,\index{Euclidean geometry} as was shown by a classic
result of logician Alfred Tarski\index{Tarski, Alfred} in 1948
\cite{Tarski}.\footnote{Tarski's result actually applies to a subset of the
geometry discussed in elementary textbooks.  This subset includes most of what
would be considered elementary geometry but it is not powerful enough to
express, among other things, the notions of the circumference and area of a
circle.  Extending the theory in a way that includes notions such as these
makes the theory undecidable, as was also shown by Tarski.  Tarski's algorithm
is far too inefficient to implement practically on a computer.  A practical
algorithm for proving a smaller subset of geometry theorems (those not
involving concepts of ``order'' or ``continuity'') was discovered by Chinese
mathematician Wu Wen-ts\"{u}n in 1977 \cite{Chou}\index{Chou,
Shang-Ching}.}\index{Wen-ts{\"{u}}n, Wu}  But most theories, including
elementary arithmetic, are undecidable.  This fact contributes to keeping
mathematics alive and well, since many mathematicians believe
that they will never be
replaced by computers (if they believe Roger Penrose's argument that a
computer can never replace the brain \cite{Penrose}\index{Penrose, Roger}).
In fact,  elementary geometry is often considered a ``dead'' field
for the simple reason that it is decidable.

On the other hand, the undecidability of a theory does not mean that one cannot
use a computer to search for proofs, providing one is willing to give up if a
proof is not found after a reasonable amount of time.  The field of automated
theorem proving\index{automated theorem proving} specializes in pursuing such
computer searches.  Among the more successful results to date are those based
on an algorithm known as Robinson's resolution principle
\cite{Robinson}\index{Robinson's resolution principle}.

Automated theorem provers can be excellent tools for those willing to learn
how to use them.  But they are not widely used in mainstream pure
mathematics, even though they could probably be useful in many areas.  There
are several reasons for this.  Probably most important, the main goal in pure
mathematics is to arrive at results that are considered to be deep or
important; proving them is essential but secondary.  Usually, an automated
theorem prover cannot assist in this main goal, and by the time the main goal
is achieved, the mathematician may have already figured out the proof as a
by-product.  There is also a notational problem.  Mathematicians are used to
using very compact syntax where one or two symbols (heavily dependent on
context) can represent very complex concepts; this is part of the
hierarchy\index{hierarchy} they have built up to tackle difficult problems.  A
theorem prover on the other hand might require that a theorem be expressed in
``first-order logic,''\index{first-order logic} which is the logic on which
most of mathematics is ultimately based but which is not ordinarily used
directly because expressions can become very long.  Some automated theorem
provers are experimental programs, limited in their use to very specialized
areas, and the goal of many is simply research into the nature of automated
theorem proving itself.  Finally, much research remains to be done to enable
them to prove very deep theorems.  One significant result was a
computer proof by Larry Wos\index{Wos, Larry} and colleagues that every Robbins
algebra\index{Robbins algebra} is a Boolean  algebra\index{Boolean algebra}
({\em New York Times}, Dec. 10, 1996).\footnote{In 1933, E.~V.\
Huntington\index{Huntington, E. V.}
presented the following axiom system for
Boolean algebra with a unary operation $n$ and a binary operation $+$:
\begin{center}
    $x + y = y + x$ \\
    $(x + y) + z = x + (y + z)$ \\
    $n(n(x) + y) + n(n(x) + n(y)) = x$
\end{center}
Herbert Robbins\index{Robbins, Herbert}, a student of Huntington, conjectured
that the last equation can be replaced with a simpler one:
\begin{center}
    $n(n(x + y) + n(x + n(y))) = x$
\end{center}
Robbins and Huntington could not find a proof.  The problem was
later studied unsuccessfully by Tarski\index{Tarski, Alfred} and his
students, and it remained an unsolved problem until a
computer found the proof in 1996.  For more information on
the Robbins algebra problem see \cite{Wos}.}

How does Metamath\index{Metamath} relate to automated theorem provers?  A
theorem prover is primarily concerned with one theorem at a time (perhaps
tapping into a small database of known theorems) whereas Metamath is more like
a theorem archiving system, storing both the theorem and its proof in a
database for access and verification.  Metamath is one answer to ``what do you
do with the output of a theorem prover?''  and could be viewed as the
next step in the process.  Automated theorem provers could be useful tools for
helping develop its database.
Note that very long, automatically
generated proofs can make your database fat and ugly and cause Metamath's proof
verification to take a long time to run.  Unless you have a particularly good
program that generates very concise proofs, it might be best to consider the
use of automatically generated proofs as a quick-and-dirty approach, to be
manually rewritten at some later date.

The program {\sc otter}\index{otter@{\sc otter}}\footnote{\url{http://www.cs.unm.edu/\~mccune/otter/}.}, later succeeded by
prover9\index{prover9}\footnote{\url{https://www.cs.unm.edu/~mccune/mace4/}.},
have been historically influential.
The E prover\index{E prover}\footnote{\url{https://github.com/eprover/eprover}.}
is a maintained automated theorem prover
for full first-order logic with equality.
There are many other automated theorem provers as well.

If you want to combine automated theorem provers with Metamath
consider investigating
the book {\em Automated Reasoning:  Introduction and Applications}
\cite{Wos}\index{Wos, Larry}.  This book discusses
how to use {\sc otter} in a way that can
not only able to generate
relatively efficient proofs, it can even be instructed to search for
shorter proofs.  The effective use of {\sc otter} (and similar tools)
does require a certain
amount of experience, skill, and patience.  The axiom system used in the
\texttt{set.mm}\index{set theory database (\texttt{set.mm})} set theory
database can be expressed to {\sc otter} using a method described in
\cite{Megill}.\index{Megill, Norman}\footnote{To use those axioms with
{\sc otter}, they must be restated in a way that eliminates the need for
``dummy variables.''\index{dummy variable!eliminating} See the Comment
on p.~\pageref{nodd}.} When successful, this method tends to generate
short and clever proofs, but my experiments with it indicate that the
method will find proofs within a reasonable time only for relatively
easy theorems.  It is still fun to experiment with.

Reference \cite{Bledsoe}\index{Bledsoe, W. W.} surveys a number of approaches
people have explored in the field of automated theorem proving\index{automated
theorem proving}.

\subsection{Interactive Theorem Provers}\label{interactivetheoremprovers}

Finding proofs completely automatically is difficult, so there
are some interactive theorem provers that allow a human to guide the
computer to find a proof.
Examples include
HOL Light\index{HOL light}%
\footnote{\url{https://www.cl.cam.ac.uk/~jrh13/hol-light/}.},
Isabelle\index{Isabelle}%
\footnote{\url{http://www.cl.cam.ac.uk/Research/HVG/Isabelle}.},
{\sc hol}\index{hol@{\sc hol}}%
\footnote{\url{https://hol-theorem-prover.org/}.},
and
Coq\index{Coq}\footnote{\url{https://coq.inria.fr/}.}.

A major difference between most of these tools and Metamath is that the
``proofs'' are actually programs that guide the program to find a proof,
and not the proof itself.
For example, an Isabelle/HOL proof might apply a step
\texttt{apply (blast dest: rearrange reduction)}. The \texttt{blast}
instruction applies
an automatic tableux prover and returns if it found a sequence of proof
steps that work... but the sequence is not considered part of the proof.

A good overview of
higher-level proof verification languages (such as {\sc lcf}\index{lcf@{\sc
lcf}} and {\sc hol}\index{hol@{\sc hol}})
is given in \cite{Harrison}.  All of these languages are fundamentally
different from Metamath in that much of the mathematical foundational
knowledge is embedded in the underlying proof-verification program, rather
than placed directly in the database that is being verified.
These can have a steep learning curve for those without a mathematical
background.  For example, one usually must have a fair understanding of
mathematical logic in order to follow their proofs.

\subsection{Proof Verifiers}\label{proofverifiers}

A proof verifier is a program that doesn't generate proofs but instead
verifies proofs that you give it.  Many proof verifiers have limited built-in
automated proof capabilities, such as figuring out simple logical inferences
(while still being guided by a person who provides the overall proof).  Metamath
has no built-in automated proof capability other than the limited
capability of its Proof Assistant.

Proof-verification languages are not used as frequently as they might be.
Pure mathematicians are more concerned with producing new results, and such
detail and rigor would interfere with that goal.  The use of computers in pure
mathematics is primarily focused on automated theorem provers (not verifiers),
again with the ultimate goal of aiding the creation of new mathematics.
Automated theorem provers are usually concerned with attacking one theorem at
time rather than making a large, organized database easily available to the
user.  Metamath is one way to help close this gap.

By itself Metamath is a mostly a proof verifier.
This does not mean that other approaches can't be used; the difference
is that in Metamath, the results of various provers must be recorded
step-by-step so that they can be verified.

Another proof-verification language is Mizar,\index{Mizar} which can display
its proofs in the informal language that mathematicians are accustomed to.
Information on the Mizar language is available at \url{http://mizar.org}.

For the working mathematician, Mizar is an excellent tool for rigorously
documenting proofs. Mizar typesets its proofs in the informal English used by
mathematicians (and, while fine for them, are just as inscrutable by
laypersons!). A price paid for Mizar is a relatively steep learning curve of a
couple of weeks.  Several mathematicians are actively formalizing different
areas of mathematics using Mizar and publishing the proofs in a dedicated
journal. Unfortunately the task of formalizing mathematics is still looked
down upon to a certain extent since it doesn't involve the creation of ``new''
mathematics.

The closest system to Metamath is
the {\em Ghilbert}\index{Ghilbert} proof language (\url{http://ghilbert.org})
system developed by
Raph Levien\index{Levien, Raph}.
Ghilbert is a formal proof checker heavily inspired by Metamath.
Ghilbert statements are s-expressions (a la Lisp), which is easy
for computers to parse but many people find them hard to read.
There are a number of differences in their specific constructs, but
there is at least one tool to translate some Metamath materials into Ghilbert.
As of 2019 the Ghilbert community is smaller and less active than the
Metamath community.
That said, the Metamath and Ghilbert communities overlap, and fruitful
conversations between them have occurred many times over the years.

\subsection{Creating a Database of Formalized Mathematics}\label{mathdatabase}

Besides Metamath, there are several other ongoing projects with the goal of
formalizing mathematics into computer-verifiable databases.
Understanding some history will help.

The {\sc qed}\index{qed project@{\sc qed} project}%
\footnote{\url{http://www-unix.mcs.anl.gov/qed}.}
project arose in 1993 and its goals were outlined in the
{\sc qed} manifesto.
The {\sc qed} manifesto was
a proposal for a computer-based database of all mathematical knowledge,
strictly formalized and with all proofs having been checked automatically.
The project had a conference in 1994 and another in 1995;
there was also a ``twenty years of the {\sc qed} manifesto'' workshop
in 2014.
Its ideals are regularly reraised.

In a 2007 paper, Freek Wiedijk identified two reasons
for the failure of the {\sc qed} project as originally envisioned:%
\cite{Wiedijk-revisited}\index{Wiedijk, Freek}

\begin{itemize}
\item Very few people are working on formalization of mathematics. There is no compelling application for fully mechanized mathematics.
\item Formalized mathematics does not yet resemble traditional mathematics. This is partly due to the complexity of mathematical notation, and partly to the limitations of existing theorem provers and proof assistants.
\end{itemize}

But this did not end the dream of
formalizing mathematics into computer-verifiable databases.
The problems that led to the {\sc qed} manifesto are still with us,
even though the challenges were harder than originally considered.
What has happened instead is that various independent projects have
worked towards formalizing mathematics into computer-verifiable databases,
each simultaneously competing and cooperating with each other.

A concrete way to see this is
Freek Wiedijk's ``Formalizing 100 Theorems'' list%
\footnote{\url{http://www.cs.ru.nl/\%7Efreek/100/}.}
which shows the progress different systems have made on a challenge list
of 100 mathematical theorems.%
\footnote{ This is not the only list of ``interesting'' theorems.
Another interesting list was posted by Oliver Knill's list
\cite{Knill}\index{Knill, Oliver}.}
The top systems as of February 2019
(in order of the number of challenges completed) are
HOL Light, Isabelle, Metamath, Coq, and Mizar.

The Metamath 100%
\footnote{\url{http://us.metamath.org/mm\_100.html}}
page (maintained by David A. Wheeler\index{Wheeler, David A.})
shows the progress of Metamath (specifically its \texttt{set.mm} database)
against this challenge list maintained by Freek Wiedijk.
The Metamath \texttt{set.mm} database
has made a lot of progress over the years,
in part because working to prove those challenge theorems required
defining various terms and proving their properties as a prerequisite.
Here are just a few of the many statements that have been
formally proven with Metamath:

% The entries of this cause the narrow display to break poorly,
% since the short amount of text means LaTeX doesn't get a lot to work with
% and the itemize format gives it even *less* margin than usual.
% No one will mind if we make just this list flushleft, since this list
% will be internally consistent.
\begin{flushleft}
\begin{itemize}
\item 1. The Irrationality of the Square Root of 2
  (\texttt{sqr2irr}, by Norman Megill, 2001-08-20)
\item 2. The Fundamental Theorem of Algebra
  (\texttt{fta}, by Mario Carneiro, 2014-09-15)
\item 22. The Non-Denumerability of the Continuum
  (\texttt{ruc}, by Norman Megill, 2004-08-13)
\item 54. The Konigsberg Bridge Problem
  (\texttt{konigsberg}, by Mario Carneiro, 2015-04-16)
\item 83. The Friendship Theorem
  (\texttt{friendship}, by Alexander W. van der Vekens, 2018-10-09)
\end{itemize}
\end{flushleft}

We thank all of those who have developed at least one of the Metamath 100
proofs, and we particularly thank
Mario Carneiro\index{Carneiro, Mario}
who has contributed the most Metamath 100 proofs as of 2019.
The Metamath 100 page shows the list of all people who have contributed a
proof, and links to graphs and charts showing progress over time.
We encourage others to work on proving theorems not yet proven in Metamath,
since doing so improves the work as a whole.

Each of the math formalization systems (including Metamath)
has different strengths and weaknesses, depending on what you value.
Key aspects that differentiate Metamath from the other top systems are:

\begin{itemize}
\item Metamath is not tied to any particular set of axioms.
\item Metamath can show every step of every proof, no exceptions.
  Most other provers only assert that a proof can be found, and do not
  show every step. This also makes verification fast, because
  the system does not need to rediscover proof details.
\item The Metamath verifier has been re-implemented in many different
  programming languages, so verification can be done by multiple
  implementations.  In particular, the
  \texttt{set.mm}\index{set theory database (\texttt{set.mm})}%
  \index{Metamath Proof Explorer} database is verified by
  four different verifiers
  written in four different languages by four different authors.
  This greatly reduces the risk of accepting an invalid
  proof due to an error in the verifier.
\item Proofs stay proven.  In some systems, changes to the system's
  syntax or how a tactic works causes proofs to fail in later versions,
  causing older work to become essentially lost.
  Metamath's language is
  extremely small and fixed, so once a proof is added to a database,
  the database can be rechecked with later versions of the Metamath program
  and with other verifiers of Metamath databases.
  If an axiom or key definition needs to be changed, it is easy to
  manipulate the database as a whole to handle the change
  without touching the underlying verifier.
  Since re-verification of an entire database takes seconds, there
  is never a reason to delay complete verification.
  This aspect is especially compelling if your
  goal is to have a long-term database of proofs.
\item Licensing is generous.  The main Metamath databases are released to
  the public domain, and the main Metamath program is open source software
  under a standard, widely-used license.
\item Substitutions are easy to understand, even by those who are not
  professional mathematicians.
\end{itemize}

Of course, other systems may have advantages over Metamath
that are more compelling, depending on what you value.
In any case, we hope this helps you understand Metamath
within a wider context.

\subsection{In Summary}\label{computers-summary}

To summarize our discussions of computers and mathematics, computer algebra
systems can be viewed as theorem generators focusing on a narrow realm of
mathematics (numbers and their properties), automated theorem provers as proof
generators for specific theorems in a much broader realm covered by a built-in
formal system such as first-order logic, interactive theorem
provers require human guidance, proof verifiers verify proofs but
historically they have been
restricted to first-order logic.
Metamath, in contrast,
is a proof verifier and documenter whose realm is essentially unlimited.

\section{Mathematics and Metamath}

\subsection{Standard Mathematics}

There are a number of ways that Metamath\index{Metamath} can be used with
standard mathematics.  The most satisfying way philosophically is to start at
the very beginning, and develop the desired mathematics from the axioms of
logic and set theory.\index{set theory}  This is the approach taken in the
\texttt{set.mm}\index{set theory database (\texttt{set.mm})}%
\index{Metamath Proof Explorer}
database (also known as the Metamath Proof Explorer).
Among other things, this database builds up to the
axioms of real and complex numbers\index{analysis}\index{real and complex
numbers} (see Section~\ref{real}), and a standard development of analysis, for
example, could start at that point, using it as a basis.   Besides this
philosophical advantage, there are practical advantages to having all of the
tools of set theory available in the supporting infrastructure.

On the other hand, you may wish to start with the standard axioms of a
mathematical theory without going through the set theoretical proofs of those
axioms.  You will need mathematical logic to make inferences, but if you wish
you can simply introduce theorems\index{theorem} of logic as
``axioms''\index{axiom} wherever you need them, with the implicit assumption
that in principle they can be proved, if they are obvious to you.  If you
choose this approach, you will probably want to review the notation used in
\texttt{set.mm}\index{set theory database (\texttt{set.mm})} so that your own
notation will be consistent with it.

\subsection{Other Formal Systems}
\index{formal system}

Unlike some programs, Metamath\index{Metamath} is not limited to any specific
area of mathematics, nor committed to any particular mathematical philosophy
such as classical logic versus intuitionism, nor limited, say, to expressions
in first-order logic.  Although the database \texttt{set.mm}
describes standard logic and set theory, Meta\-math
is actually a general-purpose language for describing a wide variety of formal
systems.\index{formal system}  Non-standard systems such as modal
logic,\index{modal logic} intuitionist logic\index{intuitionism}, higher-order
logic\index{higher-order logic}, quantum logic\index{quantum logic}, and
category theory\index{category theory} can all be described with the Metamath
language.  You define the symbols you prefer and tell Metamath the axioms and
rules you want to start from, and Metamath will verify any inferences you make
from those axioms and rules.  A simple example of a non-standard formal system
is Hofstadter's\index{Hofstadter, Douglas R.} MIU system,\index{MIU-system}
whose Metamath description is presented in Appendix~\ref{MIU}.

This is not hypothetical.
The largest Metamath database is
\texttt{set.mm}\index{set theory database (\texttt{set.mm}}%
\index{Metamath Proof Explorer}), aka the Metamath Proof Explorer,
which uses the most common axioms for mathematical foundations
(specifically classical logic combined with Zermelo--Fraenkel
set theory\index{Zermelo--Fraenkel set theory} with the Axiom of Choice).
But other Metamath databases are available:

\begin{itemize}
\item The database
  \texttt{iset.mm}\index{intuitionistic logic database (\texttt{iset.mm})},
  aka the
  Intuitionistic Logic Explorer\index{Intuitionistic Logic Explorer},
  uses intuitionistic logic (a constructivist point of view)
  instead of classical logic.
\item The database
  \texttt{nf.mm}\index{New Foundations database (\texttt{nf.mm})},
  aka the
  New Foundations Explorer\index{New Foundations Explorer},
  constructs mathematics from scratch,
  starting from Quine's New Foundations (NF) set theory axioms.
\item The database
  \texttt{hol.mm}\index{Higher-order Logic database (\texttt{hol.mm})},
  aka the
  Higher-Order Logic (HOL) Explorer\index{Higher-Order Logic (HOL) Explorer},
  starts with HOL (also called simple type theory) and derives
  equivalents to ZFC axioms, connecting the two approaches.
\end{itemize}

Since the days of David Hilbert,\index{Hilbert, David} mathematicians have
been concerned with the fact that the metalanguage\index{metalanguage} used to
describe mathematics may be stronger than the mathematics being described.
Metamath\index{Metamath}'s underlying finitary\index{finitary proof},
constructive nature provides a good philosophical basis for studying even the
weakest logics.\index{weak logic}

The usual treatment of many non-standard formal systems\index{formal
system} uses model theory\index{model theory} or proof theory\index{proof
theory} to describe these systems; these theories, in turn, are based on
standard set theory.  In other words, a non-standard formal system is defined
as a set with certain properties, and standard set theory is used to derive
additional properties of this set.  The standard set theory database provided
with Metamath can be used for this purpose, and when used this way
the development of a special
axiom system for the non-standard formal system becomes unnecessary.  The
model- or proof-theoretic approach often allows you to prove much deeper
results with less effort.

Metamath supports both approaches.  You can define the non-standard
formal system directly, or define the non-standard formal system as
a set with certain properties, whichever you find most helpful.

%\section{Additional Remarks}

\subsection{Metamath and Its Philosophy}

Closely related to Metamath\index{Metamath} is a philosophy or way of looking
at mathematics. This philosophy is related to the formalist
philosophy\index{formalism} of Hilbert\index{Hilbert, David} and his followers
\cite[pp.~1203--1208]{Kline}\index{Kline, Morris}
\cite[p.~6]{Behnke}\index{Behnke, H.}. In this philosophy, mathematics is
viewed as nothing more than a set of rules that manipulate symbols, together
with the consequences of those rules.  While the mathematics being described
may be complex, the rules used to describe it (the
``metamathematics''\index{metamathematics}) should be as simple as possible.
In particular, proofs should be restricted to dealing with concrete objects
(the symbols we write on paper rather than the abstract concepts they
represent) in a constructive manner; these are called ``finitary''
proofs\index{finitary proof} \cite[pp.~2--3]{Shoenfield}\index{Shoenfield,
Joseph R.}.

Whether or not you find Metamath interesting or useful will in part depend on
the appeal you find in its philosophy, and this appeal will probably depend on
your particular goals with respect to mathematics.  For example, if you are a
pure mathematician at the forefront of discovering new mathematical knowledge,
you will probably find that the rigid formality of Metamath stifles your
creativity.  On the other hand, we would argue that once this knowledge is
discovered, there are advantages to documenting it in a standard format that
will make it accessible to others.  Sixty years from now, your field may be
dormant, and as Davis and Hersh put it, your ``writings would become less
translatable than those of the Maya'' \cite[p.~37]{Davis}\index{Davis, Phillip
J.}.


\subsection{A History of the Approach Behind Metamath}

Probably the one work that has had the most motivating influence on
Metamath\index{Metamath} is Whitehead and Russell's monumental {\em Principia
Mathematica} \cite{PM}\index{Whitehead, Alfred North}\index{Russell,
Bertrand}\index{principia mathematica@{\em Principia Mathematica}}, whose aim
was to deduce all of mathematics from a small number of primitive ideas, in a
very explicit way that in principle anyone could understand and follow.  While
this work was tremendously influential in its time, from a modern perspective
it suffers from several drawbacks.  Both its notation and its underlying
axioms are now considered dated and are no longer used.  From our point of
view, its development is not really as accessible as we would like to see; for
practical reasons, proofs become more and more sketchy as its mathematics
progresses, and working them out in fine detail requires a degree of
mathematical skill and patience that many people don't have.  There are
numerous small errors, which is understandable given the tedious, technical
nature of the proofs and the lack of a computer to verify the details.
However, even today {\em Principia Mathematica} stands out as the work closest
in spirit to Metamath.  It remains a mind-boggling work, and one can't help
but be amazed at seeing ``$1+1=2$'' finally appear on page 83 of Volume II
(Theorem *110.643).

The origin of the proof notation used by Metamath dates back to the 1950's,
when the logician C.~A.~Meredith expressed his proofs in a compact notation
called ``condensed detachment''\index{condensed detachment}
\cite{Hindley}\index{Hindley, J. Roger} \cite{Kalman}\index{Kalman, J. A.}
\cite{Meredith}\index{Meredith, C. A.} \cite{Peterson}\index{Peterson, Jeremy
George}.  This notation allows proofs to be communicated unambiguously by
merely referencing the axiom\index{axiom}, rule\index{rule}, or
theorem\index{theorem} used at each step, without explicitly indicating the
substitutions\index{substitution!variable}\index{variable substitution} that
have to be made to the variables in that axiom, rule, or theorem.  Ordinarily,
condensed detachment is more or less limited to propositional
calculus\index{propositional calculus}.  The concept has been extended to
first-order logic\index{first-order logic} in \cite{Megill}\index{Megill,
Norman}, making it is easy to write a small computer program to verify proofs
of simple first-order logic theorems.\index{condensed detachment!and
first-order logic}

A key concept behind the notation of condensed detachment is called
``unification,''\index{unification} which is an algorithm for determining what
substitutions\index{substitution!variable}\index{variable substitution} to
variables have to be made to make two expressions match each other.
Unification was first precisely defined by the logician J.~A.~Robinson, who
used it in the development of a powerful
theorem-proving technique called the ``resolution principle''
\cite{Robinson}\index{Robinson's resolution principle}. Metamath does not make
use of the resolution principle, which is intended for systems of first-order
logic.\index{first-order logic}  Metamath's use is not restricted to
first-order logic, and as we have mentioned it does not automatically discover
proofs.  However, unification is a key idea behind Metamath's proof
notation, and Metamath makes use of a very simple version of it
(Section~\ref{unify}).

\subsection{Metamath and First-Order Logic}

First-order logic\index{first-order logic} is the supporting structure
for standard mathematics.  On top of it is set theory, which contains
the axioms from which virtually all of mathematics can be derived---a
remarkable fact.\index{category
theory}\index{cardinal, inaccessible}\label{categoryth}\footnote{An exception seems
to be category theory.  There are several schools of thought on whether
category theory is derivable from set theory.  At a minimum, it appears
that an additional axiom is needed that asserts the existence of an
``inaccessible cardinal'' (a type of infinity so large that standard set
theory can't prove or deny that it exists).
%
%%%% (I took this out that was in previous editions:)
% But it is also argued that not just one but a ``proper class'' of them
% is needed, and the existence of proper classes is impossible in standard
% set theory.  (A proper class is a collection of sets so huge that no set
% can contain it as an element.  Proper classes can lead to
% inconsistencies such as ``Russell's paradox.''  The axioms of standard
% set theory are devised so as to deny the existence of proper classes.)
%
For more information, see
\cite[pp.~328--331]{Herrlich}\index{Herrlich, Horst} and
\cite{Blass}\index{Blass, Andrea}.}

One of the things that makes Metamath\index{Metamath} more practical for
first-order theories is a set of axioms for first-order logic designed
specifically with Metamath's approach in mind.  These are included in
the database \texttt{set.mm}\index{set theory database (\texttt{set.mm})}.
See Chapter~\ref{fol} for a detailed
description; the axioms are shown in Section~\ref{metaaxioms}.  While
logically equivalent to standard axiom systems, our axiom system breaks
up the standard axioms into smaller pieces such that from them, you can
directly derive what in other systems can only be derived as higher-level
``metatheorems.''\index{metatheorem}  In other words, it is more powerful than
the standard axioms from a metalogical point of view.  A rigorous
justification for this system and its ``metalogical
completeness''\index{metalogical completeness} is found in
\cite{Megill}\index{Megill, Norman}.  The system is closely related to a
system developed by Monk\index{Monk, J. Donald} and Tarski\index{Tarski,
Alfred} in 1965 \cite{Monks}.

For example, the formula $\exists x \, x = y $ (given $y$, there exists some
$x$ equal to it) is a theorem of logic,\footnote{Specifically, it is a theorem
of those systems of logic that assume non-empty domains.  It is not a theorem
of more general systems that include the empty domain\index{empty domain}, in
which nothing exists, period!  Such systems are called ``free
logics.''\index{free logic} For a discussion of these systems, see
\cite{Leblanc}\index{Leblanc, Hugues}.  Since our use for logic is as a basis
for set theory, which has a non-empty domain, it is more convenient (and more
traditional) to use a less general system.  An interesting curiosity is that,
using a free logic as a basis for Zermelo--Fraenkel set
theory\index{Zermelo--Fraenkel set theory} (with the redundant Axiom of the
Null Set omitted),\index{Axiom of the Null Set} we cannot even prove the
existence of a single set without assuming the axiom of infinity!\index{Axiom
of Infinity}} whether or not $x$ and $y$ are distinct variables\index{distinct
variables}.  In many systems of logic, we would have to prove two theorems to
arrive at this result.  First we would prove ``$\exists x \, x = x $,'' then
we would separately prove ``$\exists x \, x = y $, where $x$ and $y$ are
distinct variables.''  We would then combine these two special cases ``outside
of the system'' (i.e.\ in our heads) to be able to claim, ``$\exists x \, x =
y $, regardless of whether $x$ and $y$ are distinct.''  In other words, the
combination of the two special cases is a
metatheorem.  In the system of logic
used in Metamath's set theory\index{set theory database (\texttt{set.mm})}
database, the axioms of logic are broken down into small pieces that allow
them to be reassembled in such a way that theorems such as these can be proved
directly.

Breaking down the axioms in this way makes them look peculiar and not very
intuitive at first, but rest assured that they are correct and complete.  Their
correctness is ensured because they are theorem schemes of standard first-order
logic (which you can easily verify if you are a logician).  Their completeness
follows from the fact that we explicitly derive the standard axioms of
first-order logic as theorems.  Deriving the standard axioms is somewhat
tricky, but once we're there, we have at our disposal a system that is less
awkward to work with in formal proofs\index{formal proof}.  In technical terms
that logicians understand, we eliminate the cumbersome concepts of ``free
variable,''\index{free variable} ``bound variable,''\index{bound variable} and
``proper substitution''\index{proper substitution}\index{substitution!proper}
as primitive notions.  These concepts are present in our system but are
defined in terms of concepts expressed by the axioms and can be eliminated in
principle.  In standard systems, these concepts are really like additional,
implicit axioms\index{implicit axiom} that are somewhat complex and cannot be
eliminated.

The traditional approach to logic, wherein free variables and proper
substitution is defined, is also possible to do directly in the Metamath
language.  However, the notation tends to become awkward, and there are
disadvantages:  for example, extending the definition of a wff with a
definition is awkward, because the free variable and proper substitution
concepts have to have their definitions also extended.  Our choice of
axioms for \texttt{set.mm} is to a certain extent a matter of style, in
an attempt to achieve overall simplicity, but you should also be aware
that the traditional approach is possible as well if you should choose
it.

\chapter{Using the Metamath Program}
\label{using}

\section{Installation}

The way that you install Metamath\index{Metamath!installation} on your
computer system will vary for different computers.  Current
instructions are provided with the Metamath program download at
\url{http://metamath.org}.  In general, the installation is simple.
There is one file containing the Metamath program itself.  This file is
usually called \texttt{metamath} or \texttt{metamath.}{\em xxx} where
{\em xxx} is the convention (such as \texttt{exe}) for an executable
program on your operating system.  There are several additional files
containing samples of the Metamath language, all ending with
\texttt{.mm}.  The file \texttt{set.mm}\index{set theory database
(\texttt{set.mm})} contains logic and set theory and can be used as a
starting point for other areas of mathematics.

You will also need a text editor\index{text editor} capable of editing plain
{\sc ascii}\footnote{American Standard Code for Information Interchange.} text
in order to prepare your input files.\index{ascii@{\sc ascii}}  Most computers
have this capability built in.  Note that plain text is not necessarily the
default for some word processing programs\index{word processor}, especially if
they can handle different fonts; for example, with Microsoft Word\index{Word
(Microsoft)}, you must save the file in the format ``Text Only With Line
Breaks'' to get a plain text\index{plain text} file.\footnote{It is
recommended that all lines in a Metamath source file be 79 characters or less
in length for compatibility among different computer terminals.  When creating
a source file on an editor such as Word, select a monospaced
font\index{monospaced font} such as Courier\index{Courier font} or
Monaco\index{Monaco font} to make this easier to achieve.  Better yet,
just use a plain text editor such as Notepad.}

On some computer systems, Metamath does not have the capability to print
its output directly; instead, you send its output to a file (using the
\texttt{open} commands described later).  The way you print this output
file depends on your computer.\index{printers} Some computers have a
print command, whereas with others, you may have to read the file into
an editor and print it from there.

If you want to print your Metamath source files with typeset formulas
containing standard mathematical symbols, you will need the \LaTeX\
typesetting program\index{latex@{\LaTeX}}, which is widely and freely
available for most operating systems.  It runs natively on Unix and
Linux, and can be installed on Windows as part of the free Cygwin
package (\url{http://cygwin.com}).

You can also produce {\sc html}\footnote{HyperText Markup Language.}
web pages.  The {\tt help html} command in the Metamath program will
assist you with this feature.

\section{Your First Formal System}\label{start}
\subsection{From Nothing to Zero}\label{startf}

To give you a feel for what the Metamath\index{Metamath} language looks like,
we will take a look at a very simple example from formal number
theory\index{number theory}.  This example is taken from
Mendelson\index{Mendelson, Elliot} \cite[p. 123]{Mendelson}.\footnote{To keep
the example simple, we have changed the formalism slightly, and what we call
axioms\index{axiom} are strictly speaking theorems\index{theorem} in
\cite{Mendelson}.}  We will look at a small subset of this theory, namely that
part needed for the first number theory theorem proved in \cite{Mendelson}.

First we will look at a standard formal proof\index{formal proof} for the
example we have picked, then we will look at the Metamath version.  If you
have never been exposed to formal proofs, the notation may seem to be such
overkill to express such simple notions that you may wonder if you are missing
something.  You aren't.  The concepts involved are in fact very simple, and a
detailed breakdown in this fashion is necessary to express the proof in a way
that can be verified mechanically.  And as you will see, Metamath breaks the
proof down into even finer pieces so that the mechanical verification process
can be about as simple as possible.

Before we can introduce the axioms\index{axiom} of the theory, we must define
the syntax rules for forming legal expressions\index{syntax rules}
(combinations of symbols) with which those axioms can be used. The number 0 is
a {\bf term}\index{term}; and if $ t$ and $r$ are terms, so is $(t+r)$. Here,
$ t$ and $r$ are ``metavariables''\index{metavariable} ranging over terms; they
themselves do not appear as symbols in an actual term.  Some examples of
actual terms are $(0 + 0)$ and $((0+0)+0)$.  (Note that our theory describes
only the number zero and sums of zeroes.  Of course, not much can be done with
such a trivial theory, but remember that we have picked a very small subset of
complete number theory for our example.  The important thing for you to focus
on is our definitions that describe how symbols are combined to form valid
expressions, and not on the content or meaning of those expressions.) If $ t$
and $r$ are terms, an expression of the form $ t=r$ is a {\bf wff}
(well-formed formula)\index{well-formed formula (wff)}; and if $P$ and $Q$ are
wffs, so is $(P\rightarrow Q)$ (which means ``$P$ implies
$Q$''\index{implication ($\rightarrow$)} or ``if $P$ then $Q$'').
Here $P$ and $Q$ are metavariables ranging over wffs.  Examples of actual
wffs are $0=0$, $(0+0)=0$, $(0=0 \rightarrow (0+0)=0)$, and $(0=0\rightarrow
(0=0\rightarrow 0=(0+0)))$.  (Our notation makes use of more parentheses than
are customary, but the elimination of ambiguity this way simplifies our
example by avoiding the need to define operator precedence\index{operator
precedence}.)

The {\bf axioms}\index{axiom} of our theory are all wffs of the following
form, where $ t$, $r$, and $s$ are any terms:

%Latex p. 92
\renewcommand{\theequation}{A\arabic{equation}}

\begin{equation}
(t=r\rightarrow (t=s\rightarrow r=s))
\end{equation}
\begin{equation}
(t+0)=t
\end{equation}

Note that there are an infinite number of axioms since there are an infinite
number of possible terms.  A1 and A2 are properly called ``axiom
schemes,''\index{axiom scheme} but we will refer to them as ``axioms'' for
brevity.

An axiom is a {\bf theorem}; and if $P$ and $(P\rightarrow Q)$ are theorems
(where $P$ and $Q$ are wffs), then $Q$ is also a theorem.\index{theorem}  The
second part of this definition is called the modus ponens (MP) rule of
inference\index{inference rule}\index{modus ponens}.  It allows us to obtain
new theorems from old ones.

The {\bf proof}\index{proof} of a theorem is a sequence of one or more
theorems, each of which is either an axiom or the result of modus ponens
applied to two previous theorems in the sequence, and the last of which is the
theorem being proved.

The theorem we will prove for our example is very simple:  $ t=t$.  The proof of
our theorem follows.  Study it carefully until you feel sure you
understand it.\label{zeroproof}

% Use tabu so that lines will wrap automatically as needed.
\begin{tabu} { l X X }
1. & $(t+0)=t$ & (by axiom A2) \\
2. & $(t+0)=t$ & (by axiom A2) \\
3. & $((t+0)=t \rightarrow ((t+0)=t\rightarrow t=t))$ & (by axiom A1) \\
4. & $((t+0)=t\rightarrow t=t)$ & (by MP applied to steps 2 and 3) \\
5. & $t=t$ & (by MP applied to steps 1 and 4) \\
\end{tabu}

(You may wonder why step 1 is repeated twice.  This is not necessary in the
formal language we have defined, but in Metamath's ``reverse Polish
notation''\index{reverse Polish notation (RPN)} for proofs, a previous step
can be referred to only once.  The repetition of step~1 here will enable you
to see more clearly the correspondence of this proof with the
Metamath\index{Metamath} version on p.~\pageref{demoproof}.)

Our theorem is more properly called a ``theorem scheme,''\index{theorem
scheme} for it represents an infinite number of theorems, one for each
possible term $ t$.  Two examples of actual theorems would be $0=0$ and
$(0+0)=(0+0)$.  Rarely do we prove actual theorems, since by proving schemes
we can prove an infinite number of theorems in one fell swoop.  Similarly, our
proof should really be called a ``proof scheme.''\index{proof scheme}  To
obtain an actual proof, pick an actual term to use in place of $ t$, and
substitute it for $ t$ throughout the proof.

Let's discuss what we have done here.  The axioms\index{axiom} of our theory,
A1 and A2, are trivial and obvious.  Everyone knows that adding zero to
something doesn't change it, and also that if two things are equal to a third,
then they are equal to each other. In fact, stating the trivial and obvious is
a goal to strive for in any axiomatic system.  From trivial and obvious truths
that everyone agrees upon, we can prove results that are not so obvious yet
have absolute faith in them.  If we trust the axioms and the rules, we must,
by definition, trust the consequences of those axioms and rules, if logic is
to mean anything at all.

Our rule of inference\index{rule}, modus ponens\index{modus ponens}, is also
pretty obvious once you understand what it means.  If we prove a fact $P$, and
we also prove that $P$ implies $Q$, then $Q$ necessarily follows as a new
fact.  The rule provides us with a means for obtaining new facts (i.e.\
theorems\index{theorem}) from old ones.

The theorem that we have proved, $ t=t$, is so fundamental that you may wonder
why it isn't one of the axioms\index{axiom}.  In some axiom systems of
arithmetic, it {\em is} an axiom.  The choice of axioms in a theory is to some
extent arbitrary and even an art form, constrained only by the requirement
that any two equivalent axiom systems be able to derive each other as
theorems.  We could imagine that the inventor of our axiom system originally
included $ t=t$ as an axiom, then discovered that it could be derived as a
theorem from the other axioms.  Because of this, it was not necessary to
keep it as an axiom.  By eliminating it, the final set of axioms became
that much simpler.

Unless you have worked with formal proofs\index{formal proof} before, it
probably wasn't apparent to you that $ t=t$ could be derived from our two
axioms until you saw the proof. While you certainly believe that $ t=t$ is
true, you might not be able to convince an imaginary skeptic who believes only
in our two axioms until you produce the proof.  Formal proofs such as this are
hard to come up with when you first start working with them, but after you get
used to them they can become interesting and fun.  Once you understand the
idea behind formal proofs you will have grasped the fundamental principle that
underlies all of mathematics.  As the mathematics becomes more sophisticated,
its proofs become more challenging, but ultimately they all can be broken down
into individual steps as simple as the ones in our proof above.

Mendelson's\index{Mendelson, Elliot} book, from which our example was taken,
contains a number of detailed formal proofs such as these, and you may be
interested in looking it up.  The book is intended for mathematicians,
however, and most of it is rather advanced.  Popular literature describing
formal proofs\index{formal proof} include \cite[p.~296]{Rucker}\index{Rucker,
Rudy} and \cite[pp.~204--230]{Hofstadter}\index{Hofstadter, Douglas R.}.

\subsection{Converting It to Metamath}\label{convert}

Formal proofs\index{formal proof} such as the one in our example break down
logical reasoning into small, precise steps that leave little doubt that the
results follow from the axioms\index{axiom}.  You might think that this would
be the finest breakdown we can achieve in mathematics.  However, there is more
to the proof than meets the eye. Although our axioms were rather simple, a lot
of verbiage was needed before we could even state them:  we needed to define
``term,'' ``wff,'' and so on.  In addition, there are a number of implied
rules that we haven't even mentioned. For example, how do we know that step 3
of our proof follows from axiom A1? There is some hidden reasoning involved in
determining this.  Axiom A1 has two occurrences of the letter $ t$.  One of
the implied rules states that whatever we substitute for $ t$ must be a legal
term\index{term}.\footnote{Some authors make this implied rule explicit by
stating, ``only expressions of the above form are terms,'' after defining
``term.''}  The expression $ t+0$ is pretty obviously a legal term whenever $
t$ is, but suppose we wanted to substitute a huge term with thousands of
symbols?  Certainly a lot of work would be involved in determining that it
really is a term, but in ordinary formal proofs all of this work would be
considered a single ``step.''

To express our axiom system in the Metamath\index{Metamath} language, we must
describe this auxiliary information in addition to the axioms themselves.
Metamath does not know what a ``term'' or a ``wff''\index{well-formed formula
(wff)} is.  In Metamath, the specification of the ways in which we can combine
symbols to obtain terms and wffs are like little axioms in themselves.  These
auxiliary axioms are expressed in the same notation as the ``real''
axioms\index{axiom}, and Metamath does not distinguish between the two.  The
distinction is made by you, i.e.\ by the way in which you interpret the
notation you have chosen to express these two kinds of axioms.

The Metamath language breaks down mathematical proofs into tiny pieces, much
more so than in ordinary formal proofs\index{formal proof}.  If a single
step\index{proof step} involves the
substitution\index{substitution!variable}\index{variable substitution} of a
complex term for one of its variables, Metamath must see this single step
broken down into many small steps.  This fine-grained breakdown is what gives
Metamath generality and flexibility as it lets it not be limited to any
particular mathematical notation.

Metamath's proof notation is not, in itself, intended to be read by humans but
rather is in a compact format intended for a machine.  The Metamath program
will convert this notation to a form you can understand, using the \texttt{show
proof}\index{\texttt{show proof} command} command.  You can tell the program what
level of detail of the proof you want to look at.  You may want to look at
just the logical inference steps that correspond
to ordinary formal proof steps,
or you may want to see the fine-grained steps that prove that an expression is
a term.

Here, without further ado, is our example converted to the
Metamath\index{Metamath} language:\index{metavariable}\label{demo0}

\begin{verbatim}
$( Declare the constant symbols we will use $)
    $c 0 + = -> ( ) term wff |- $.
$( Declare the metavariables we will use $)
    $v t r s P Q $.
$( Specify properties of the metavariables $)
    tt $f term t $.
    tr $f term r $.
    ts $f term s $.
    wp $f wff P $.
    wq $f wff Q $.
$( Define "term" and "wff" $)
    tze $a term 0 $.
    tpl $a term ( t + r ) $.
    weq $a wff t = r $.
    wim $a wff ( P -> Q ) $.
$( State the axioms $)
    a1 $a |- ( t = r -> ( t = s -> r = s ) ) $.
    a2 $a |- ( t + 0 ) = t $.
$( Define the modus ponens inference rule $)
    ${
       min $e |- P $.
       maj $e |- ( P -> Q ) $.
       mp  $a |- Q $.
    $}
$( Prove a theorem $)
    th1 $p |- t = t $=
  $( Here is its proof: $)
       tt tze tpl tt weq tt tt weq tt a2 tt tze tpl
       tt weq tt tze tpl tt weq tt tt weq wim tt a2
       tt tze tpl tt tt a1 mp mp
     $.
\end{verbatim}\index{metavariable}

A ``database''\index{database} is a set of one or more {\sc ascii} source
files.  Here's a brief description of this Metamath\index{Metamath} database
(which consists of this single source file), so that you can understand in
general terms what is going on.  To understand the source file in detail, you
should read Chapter~\ref{languagespec}.

The database is a sequence of ``tokens,''\index{token} which are normally
separated by spaces or line breaks.  The only tokens that are built into
the Metamath language are those beginning with \texttt{\$}.  These tokens
are called ``keywords.''\index{keyword}  All other tokens are
user-defined, and their names are arbitrary.

As you might have guessed, the Metamath token \texttt{\$(}\index{\texttt{\$(} and
\texttt{\$)} auxiliary keywords} starts a comment and \texttt{\$)} ends a comment.

The Metamath tokens \texttt{\$c}\index{\texttt{\$c} statement},
\texttt{\$v}\index{\texttt{\$v} statement},
\texttt{\$e}\index{\texttt{\$e} statement},
\texttt{\$f}\index{\texttt{\$f} statement},
\texttt{\$a}\index{\texttt{\$a} statement}, and
\texttt{\$p}\index{\texttt{\$p} statement} specify ``statements'' that
end with \texttt{\$.}\,.\index{\texttt{\$.}\ keyword}

The Metamath tokens \texttt{\$c} and \texttt{\$v} each declare\index{constant
declaration}\index{variable declaration} a list of user-defined tokens, called
``math symbols,''\index{math symbol} that the database will reference later
on.  All of the math symbols they define you have seen earlier except the
turnstile symbol \texttt{|-} ($\vdash$)\index{turnstile ({$\,\vdash$})}, which is
commonly used by logicians to mean ``a proof exists for.''  For us
the turnstile is just a
convenient symbol that distinguishes expressions that are axioms\index{axiom}
or theorems\index{theorem} from expressions that are terms or wffs.

The \texttt{\$c} statement declares ``constants''\index{constant} and
the \texttt{\$v} statement declares
``variables''\index{variable}\index{constant declaration}\index{variable
declaration} (or more precisely, metavariables\index{metavariable}).  A
variable may be substituted\index{substitution!variable}\index{variable
substitution} with sequences of math symbols whereas a constant may not
be substituted with anything.

It may seem redundant to require both \texttt{\$c}\index{\texttt{\$c} statement} and
\texttt{\$v}\index{\texttt{\$v} statement} statements (since any math
symbol\index{math symbol} not specified with a \texttt{\$c} statement could be
presumed to be a variable), but this provides for better error checking and
also allows math symbols to be redeclared\index{redeclaration of symbols}
(Section~\ref{scoping}).

The token \texttt{\$f}\index{\texttt{\$f} statement} specifies a
statement called a ``variable-type hypothesis'' (also called a
``floating hypothesis'') and \texttt{\$e}\index{\texttt{\$e} statement}
specifies a ``logical hypothesis'' (also called an ``essential
hypothesis'').\index{hypothesis}\index{variable-type
hypothesis}\index{logical hypothesis}\index{floating
hypothesis}\index{essential hypothesis} The token
\texttt{\$a}\index{\texttt{\$a} statement} specifies an ``axiomatic
assertion,''\index{axiomatic assertion} and
\texttt{\$p}\index{\texttt{\$p} statement} specifies a ``provable
assertion.''\index{provable assertion} To the left of each occurrence of
these four tokens is a ``label''\index{label} that identifies the
hypothesis or assertion for later reference.  For example, the label of
the first axiomatic assertion is \texttt{tze}.  A \texttt{\$f} statement
must contain exactly two math symbols, a constant followed by a
variable.  The \texttt{\$e}, \texttt{\$a}, and \texttt{\$p} statements
each start with a constant followed by, in general, an arbitrary
sequence of math symbols.

Associated with each assertion\index{assertion} is a set of hypotheses
that must be satisfied in order for the assertion to be used in a proof.
These are called the ``mandatory hypotheses''\index{mandatory
hypothesis} of the assertion.  Among those hypotheses whose ``scope''
(described below) includes the assertion, \texttt{\$e} hypotheses are
always mandatory and \texttt{\$f}\index{\texttt{\$f} statement}
hypotheses are mandatory when they share their variable with the
assertion or its \texttt{\$e} hypotheses.  The exact rules for
determining which hypotheses are mandatory are described in detail in
Sections~\ref{frames} and \ref{scoping}.  For example, the mandatory
hypotheses of assertion \texttt{tpl} are \texttt{tt} and \texttt{tr},
whereas assertion \texttt{tze} has no mandatory hypotheses because it
contains no variables and has no \texttt{\$e}\index{\texttt{\$e}
statement} hypothesis.  Metamath's \texttt{show statement}
command\index{\texttt{show statement} command}, described in the next
section, will show you a statement's mandatory hypotheses.

Sometimes we need to make a hypothesis relevant to only certain
assertions.  The set of statements to which a hypothesis is relevant is
called its ``scope.''  The Metamath brackets,
\texttt{\$\char`\{}\index{\texttt{\$\char`\{} and \texttt{\$\char`\}}
keywords} and \texttt{\$\char`\}}, define a ``block''\index{block} that
delimits the scope of any hypothesis contained between them.  The
assertion \texttt{mp} has mandatory hypotheses \texttt{wp}, \texttt{wq},
\texttt{min}, and \texttt{maj}.  The only mandatory hypothesis of
\texttt{th1}, on the other hand, is \texttt{tt}, since \texttt{th1}
occurs outside of the block containing \texttt{min} and \texttt{maj}.

Note that \texttt{\$\char`\{} and \texttt{\$\char`\}} do not affect the
scope of assertions (\texttt{\$a} and \texttt{\$p}).  Assertions are always
available to be referenced by any later proof in the source file.

Each provable assertion (\texttt{\$p}\index{\texttt{\$p} statement}
statement) has two parts.  The first part is the
assertion\index{assertion} itself, which is a sequence of math
symbol\index{math symbol} tokens placed between the \texttt{\$p} token
and a \texttt{\$=}\index{\texttt{\$=} keyword} token.  The second part
is a ``proof,'' which is a list of label tokens placed between the
\texttt{\$=} token and the \texttt{\$.}\index{\texttt{\$.}\ keyword}\
token that ends the statement.\footnote{If you've looked at the
\texttt{set.mm} database, you may have noticed another notation used for
proofs.  The other notation is called ``compressed.''\index{compressed
proof}\index{proof!compressed} It reduces the amount of space needed to
store a proof in the database and is described in
Appendix~\ref{compressed}.  In the example above, we use
``normal''\index{normal proof}\index{proof!normal} notation.} The proof
acts as a series of instructions to the Metamath program, telling it how
to build up the sequence of math symbols contained in the assertion part of
the \texttt{\$p} statement, making use of the hypotheses of the
\texttt{\$p} statement and previous assertions.  The construction takes
place according to precise rules.  If the list of labels in the proof
causes these rules to be violated, or if the final sequence that results
does not match the assertion, the Metamath program will notify you with
an error message.

If you are familiar with reverse Polish notation (RPN), which is sometimes used
on pocket calculators, here in a nutshell is how a proof works.  Each
hypothesis label\index{hypothesis label} in the proof is pushed\index{push}
onto the RPN stack\index{stack}\index{RPN stack} as it is encountered. Each
assertion label\index{assertion label} pops\index{pop} off the stack as many
entries as the referenced assertion has mandatory hypotheses.  Variable
substitutions\index{substitution!variable}\index{variable substitution} are
computed which, when made to the referenced assertion's mandatory hypotheses,
cause these hypotheses to match the stack entries. These same substitutions
are then made to the variables in the referenced assertion itself, which is
then pushed onto the stack.  At the end of the proof, there should be one
stack entry, namely the assertion being proved.  This process is explained in
detail in Section~\ref{proof}.

Metamath's proof notation is not very readable for humans, but it allows the
proof to be stored compactly in a file.  The Metamath\index{Metamath} program
has proof display features that let you see what's going on in a more
readable way, as you will see in the next section.

The rules used in verifying a proof are not based on any built-in syntax of the
symbol sequence in an assertion\index{assertion} nor on any built-in meanings
attached to specific symbol names.  They are based strictly on symbol
matching:  constants\index{constant} must match themselves, and
variables\index{variable} may be replaced with anything that allows a match to
occur.  For example, instead of \texttt{term}, \texttt{0}, and \verb$|-$ we could
have just as well used \texttt{yellow}, \texttt{zero}, and \texttt{provable}, as long
as we did so consistently throughout the database.  Also, we could have used
\texttt{is provable} (two tokens) instead of \verb$|-$ (one token) throughout the
database.  In each of these cases, the proof would be exactly the same.  The
independence of proofs and notation means that you have a lot of flexibility to
change the notation you use without having to change any proofs.

\section{A Trial Run}\label{trialrun}

Now you are ready to try out the Metamath\index{Metamath} program.

On all computer systems, Metamath has a standard ``command line
interface'' (CLI)\index{command line interface (CLI)} that allows you to
interact with it.  You supply commands to the CLI by typing them on the
keyboard and pressing your keyboard's {\em return} key after each line
you enter.  The CLI is designed to be easy to use and has built-in help
features.

The first thing you should do is to use a text editor to create a file
called \texttt{demo0.mm} and type into it the Metamath source shown on
p.~\pageref{demo0}.  Actually, this file is included with your Metamath
software package, so check that first.  If you type it in, make sure
that you save it in the form of ``plain {\sc ascii} text with line
breaks.''  Most word processors will have this feature.

Next you must run the Metamath program.  Depending on your computer
system and how Metamath is installed, this could range from clicking the
mouse on the Metamath icon to typing \texttt{run metamath} to typing
simply \texttt{metamath}.  (Metamath's {\tt help invoke} command describes
alternate ways of invoking the Metamath program.)

When you first enter Metamath\index{Metamath}, it will be at the CLI, waiting
for your input. You will see something like the following on your screen:
\begin{verbatim}
Metamath - Version 0.177 27-Apr-2019
Type HELP for help, EXIT to exit.
MM>
\end{verbatim}
The \texttt{MM>} prompt means that Metamath is waiting for a command.
Command keywords\index{command keyword} are not case sensitive;
we will use lower-case commands in our examples.
The version number and its release date will probably be different on your
system from the one we show above.

The first thing that you need to do is to read in your
database:\index{\texttt{read} command}\footnote{If a directory path is
needed on Unix,\index{Unix file names}\index{file names!Unix} you should
enclose the path/file name in quotes to prevent Metamath from thinking
that the \texttt{/} in the path name is a command qualifier, e.g.,
\texttt{read \char`\"db/set.mm\char`\"}.  Quotes are optional when there
is no ambiguity.}
\begin{verbatim}
MM> read demo0.mm
\end{verbatim}
Remember to press the {\em return} key after entering this command.  If
you omit the file name, Metamath will prompt you for one.   The syntax for
specifying a Macintosh file name path is given in a footnote on
p.~\pageref{includef}.\index{Macintosh file names}\index{file
names!Macintosh}

If there are any syntax errors in the database, Metamath will let you know
when it reads in the file.  The one thing that Metamath does not check when
reading in a database is that all proofs are correct, because this would
slow it down too much.  It is a good idea to periodically verify the proofs in
a database you are making changes to.  To do this, use the following command
(and do it for your \texttt{demo0.mm} file now).  Note that the \texttt{*} is a
``wild card'' meaning all proofs in the file.\index{\texttt{verify proof} command}
\begin{verbatim}
MM> verify proof *
\end{verbatim}
Metamath will report any proofs that are incorrect.

It is often useful to save the information that the Metamath program displays
on the screen. You can save everything that happens on the screen by opening a
log file. You may want to do this before you read in a database so that you
can examine any errors later on.  To open a log file, type
\begin{verbatim}
MM> open log abc.log
\end{verbatim}
This will open a file called \texttt{abc.log}, and everything that appears on the
screen from this point on will be stored in this file.  The name of the log file
is arbitrary. To close the log file, type
\begin{verbatim}
MM> close log
\end{verbatim}

Several commands let you examine what's inside your database.
Section~\ref{exploring} has an overview of some useful ones.  The
\texttt{show labels} command lets you see what statement
labels\index{label} exist.  A \texttt{*} matches any combination of
characters, and \texttt{t*} refers to all labels starting with the
letter \texttt{t}.\index{\texttt{show labels} command} The \texttt{/all}
is a ``command qualifier''\index{command qualifier} that tells Metamath
to include labels of hypotheses.  (To see the syntax explained, type
\texttt{help show labels}.)  Type
\begin{verbatim}
MM> show labels t* /all
\end{verbatim}
Metamath will respond with
\begin{verbatim}
The statement number, label, and type are shown.
3 tt $f       4 tr $f       5 ts $f       8 tze $a
9 tpl $a      19 th1 $p
\end{verbatim}

You can use the \texttt{show statement} command to get information about a
particular statement.\index{\texttt{show statement} command}
For example, you can get information about the statement with label \texttt{mp}
by typing
\begin{verbatim}
MM> show statement mp /full
\end{verbatim}
Metamath will respond with
\begin{verbatim}
Statement 17 is located on line 43 of the file
"demo0.mm".
"Define the modus ponens inference rule"
17 mp $a |- Q $.
Its mandatory hypotheses in RPN order are:
  wp $f wff P $.
  wq $f wff Q $.
  min $e |- P $.
  maj $e |- ( P -> Q ) $.
The statement and its hypotheses require the
      variables:  Q P
The variables it contains are:  Q P
\end{verbatim}
The mandatory hypotheses\index{mandatory hypothesis} and their
order\index{RPN order} are
useful to know when you are trying to understand or debug a proof.

Now you are ready to look at what's really inside our proof.  First, here is
how to look at every step in the proof---not just the ones corresponding to an
ordinary formal proof\index{formal proof}, but also the ones that build up the
formulas that appear in each ordinary formal proof step.\index{\texttt{show
proof} command}
\begin{verbatim}
MM> show proof th1 /lemmon /all
\end{verbatim}

This will display the proof on the screen in the following format:
\begin{verbatim}
 1 tt            $f term t
 2 tze           $a term 0
 3 1,2 tpl       $a term ( t + 0 )
 4 tt            $f term t
 5 3,4 weq       $a wff ( t + 0 ) = t
 6 tt            $f term t
 7 tt            $f term t
 8 6,7 weq       $a wff t = t
 9 tt            $f term t
10 9 a2          $a |- ( t + 0 ) = t
11 tt            $f term t
12 tze           $a term 0
13 11,12 tpl     $a term ( t + 0 )
14 tt            $f term t
15 13,14 weq     $a wff ( t + 0 ) = t
16 tt            $f term t
17 tze           $a term 0
18 16,17 tpl     $a term ( t + 0 )
19 tt            $f term t
20 18,19 weq     $a wff ( t + 0 ) = t
21 tt            $f term t
22 tt            $f term t
23 21,22 weq     $a wff t = t
24 20,23 wim     $a wff ( ( t + 0 ) = t -> t = t )
25 tt            $f term t
26 25 a2         $a |- ( t + 0 ) = t
27 tt            $f term t
28 tze           $a term 0
29 27,28 tpl     $a term ( t + 0 )
30 tt            $f term t
31 tt            $f term t
32 29,30,31 a1   $a |- ( ( t + 0 ) = t -> ( ( t + 0 )
                                     = t -> t = t ) )
33 15,24,26,32 mp  $a |- ( ( t + 0 ) = t -> t = t )
34 5,8,10,33 mp  $a |- t = t
\end{verbatim}

The \texttt{/lemmon} command qualifier specifies what is known as a Lemmon-style
display\index{Lemmon-style proof}\index{proof!Lemmon-style}.  Omitting the
\texttt{/lemmon} qualifier results in a tree-style proof (see
p.~\pageref{treeproof} for an example) that is somewhat less explicit but
easier to follow once you get used to it.\index{tree-style
proof}\index{proof!tree-style}

The first number on each line is the step
number of the proof.  Any numbers that follow are step numbers assigned to the
hypotheses of the statement referenced by that step.  Next is the label of
the statement referenced by the step.  The statement type of the statement
referenced comes next, followed by the math symbol\index{math symbol} string
constructed by the proof up to that step.

The last step, 34, contains the statement that is being proved.

Looking at a small piece of the proof, notice that steps 3 and 4 have
established that
\texttt{( t + 0 )} and \texttt{t} are \texttt{term}\,s, and step 5 makes use of steps 3 and
4 to establish that \texttt{( t + 0 ) = t} is a \texttt{wff}.  Let Metamath
itself tell us in detail what is happening in step 5.  Note that the
``target hypothesis'' refers to where step 5 is eventually used, i.e., in step
34.
\begin{verbatim}
MM> show proof th1 /detailed_step 5
Proof step 5:  wp=weq $a wff ( t + 0 ) = t
This step assigns source "weq" ($a) to target "wp"
($f).  The source assertion requires the hypotheses
"tt" ($f, step 3) and "tr" ($f, step 4).  The parent
assertion of the target hypothesis is "mp" ($a,
step 34).
The source assertion before substitution was:
    weq $a wff t = r
The following substitutions were made to the source
assertion:
    Variable  Substituted with
     t         ( t + 0 )
     r         t
The target hypothesis before substitution was:
    wp $f wff P
The following substitution was made to the target
hypothesis:
    Variable  Substituted with
     P         ( t + 0 ) = t
\end{verbatim}

The full proof just shown is useful to understand what is going on in detail.
However, most of the time you will just be interested in
the ``essential'' or logical steps of a proof, i.e.\ those steps
that correspond to an
ordinary formal proof\index{formal proof}.  If you type
\begin{verbatim}
MM> show proof th1 /lemmon /renumber
\end{verbatim}
you will see\label{demoproof}
\begin{verbatim}
1 a2             $a |- ( t + 0 ) = t
2 a2             $a |- ( t + 0 ) = t
3 a1             $a |- ( ( t + 0 ) = t -> ( ( t + 0 )
                                     = t -> t = t ) )
4 2,3 mp         $a |- ( ( t + 0 ) = t -> t = t )
5 1,4 mp         $a |- t = t
\end{verbatim}
Compare this to the formal proof on p.~\pageref{zeroproof} and
notice the resemblance.
By default Metamath
does not show \texttt{\$f}\index{\texttt{\$f}
statement} hypotheses and everything branching off of them in the proof tree
when the proof is displayed; this makes the proof look more like an ordinary
mathematical proof, which does not normally incorporate the explicit
construction of expressions.
This is called the ``essential'' view
(at one time you had to add the
\texttt{/essential} qualifier in the \texttt{show proof}
command to get this view, but this is now the default).
You can could use the \texttt{/all} qualifier in the \texttt{show
proof} command to also show the explicit construction of expressions.
The \texttt{/renumber} qualifier means to renumber
the steps to correspond only to what is displayed.\index{\texttt{show proof}
command}

To exit Metamath, type\index{\texttt{exit} command}
\begin{verbatim}
MM> exit
\end{verbatim}

\subsection{Some Hints for Using the Command Line Interface}

We will conclude this quick introduction to Metamath\index{Metamath} with some
helpful hints on how to navigate your way through the commands.
\index{command line interface (CLI)}

When you type commands into Metamath's CLI, you only have to type as many
characters of a command keyword\index{command keyword} as are needed to make
it unambiguous.  If you type too few characters, Metamath will tell you what
the choices are.  In the case of the \texttt{read} command, only the \texttt{r} is
needed to specify it unambiguously, so you could have typed\index{\texttt{read}
command}
\begin{verbatim}
MM> r demo0.mm
\end{verbatim}
instead of
\begin{verbatim}
MM> read demo0.mm
\end{verbatim}
In our description, we always show the full command words.  When using the
Metamath CLI commands in a command file (to be read with the \texttt{submit}
command)\index{\texttt{submit} command}, it is good practice to use
the unabbreviated command to ensure your instructions will not become ambiguous
if more commands are added to the Metamath program in the future.

The command keywords\index{command
keyword} are not case sensitive; you may type either \texttt{read} or
\texttt{ReAd}.  File names may or may not be case sensitive, depending on your
computer's operating system.  Metamath label\index{label} and math
symbol\index{math symbol} tokens\index{token} are case-sensitive.

The \texttt{help} command\index{\texttt{help} command} will provide you
with a list of topics you can get help on.  You can then type
\texttt{help} {\em topic} to get help on that topic.

If you are uncertain of a command's spelling, just type as many characters
as you remember of the command.  If you have not typed enough characters to
specify it unambiguously, Metamath will tell you what choices you have.

\begin{verbatim}
MM> show s
         ^
?Ambiguous keyword - please specify SETTINGS,
STATEMENT, or SOURCE.
\end{verbatim}

If you don't know what argument to use as part of a command, type a
\texttt{?}\index{\texttt{]}@\texttt{?}\ in command lines}\ at the
argument position.  Metamath will tell you what it expected there.

\begin{verbatim}
MM> show ?
         ^
?Expected SETTINGS, LABELS, STATEMENT, SOURCE, PROOF,
MEMORY, TRACE_BACK, or USAGE.
\end{verbatim}

Finally, you may type just the first word or words of a command followed
by {\em return}.  Metamath will prompt you for the remaining part of the
command, showing you the choices at each step.  For example, instead of
typing \texttt{show statement th1 /full} you could interact in the
following manner:
\begin{verbatim}
MM> show
SETTINGS, LABELS, STATEMENT, SOURCE, PROOF,
MEMORY, TRACE_BACK, or USAGE <SETTINGS>? st
What is the statement label <th1>?
/ or nothing <nothing>? /
TEX, COMMENT_ONLY, or FULL <TEX>? f
/ or nothing <nothing>?
19 th1 $p |- t = t $= ... $.
\end{verbatim}
After each \texttt{?}\ in this mode, you must give Metamath the
information it requests.  Sometimes Metamath gives you a list of choices
with the default choice indicated by brackets \texttt{< > }. Pressing
{\em return} after the \texttt{?}\ will select the default choice.
Answering anything else will override the default.  Note that the
\texttt{/} in command qualifiers is considered a separate
token\index{token} by the parser, and this is why it is asked for
separately.

\section{Your First Proof}\label{frstprf}

Proofs are developed with the aid of the Proof Assistant\index{Proof
Assistant}.  We will now show you how the proof of theorem \texttt{th1}
was built.  So that you can repeat these steps, we will first have the
Proof Assistant erase the proof in Metamath's source buffer\index{source
buffer}, then reconstruct it.  (The source buffer is the place in memory
where Metamath stores the information in the database when it is
\texttt{read}\index{\texttt{read} command} in.  New or modified proofs
are kept in the source buffer until a \texttt{write source}
command\index{\texttt{write source} command} is issued.)  In practice, you
would place a \texttt{?}\index{\texttt{]}@\texttt{?}\ inside proofs}\
between \texttt{\$=}\index{\texttt{\$=} keyword} and
\texttt{\$.}\index{\texttt{\$.}\ keyword}\ in the database to indicate
to Metamath\index{Metamath} that the proof is unknown, and that would be
your starting point.  Whenever the \texttt{verify proof} command encounters
a proof with a \texttt{?}\ in place of a proof step, the statement is
identified as not proved.

When I first started creating Metamath proofs, I would write down
on a piece of paper the complete
formal proof\index{formal proof} as it would appear
in a \texttt{show proof} command\index{\texttt{show proof} command}; see
the display of \texttt{show proof th1 /lemmon /re\-num\-ber} above as an
example.  After you get used to using the Proof Assistant\index{Proof
Assistant} you may get to a point where you can ``see'' the proof in your mind
and let the Proof Assistant guide you in filling in the details, at least for
simpler proofs, but until you gain that experience you may find it very useful
to write down all the details in advance.
Otherwise you may waste a lot of time as you let it take you down a wrong path.
However, others do not find this approach as helpful.
For example, Thomas Brendan Leahy\index{Leahy, Thomas Brendan}
finds that it is more helpful to him to interactively
work backward from a machine-readable statement.
David A. Wheeler\index{Wheeler, David A.}
writes down a general approach, but develops the proof
interactively by switching between
working forwards (from hypotheses and facts likely to be useful) and
backwards (from the goal) until the forwards and backwards approaches meet.
In the end, use whatever approach works for you.

A proof is developed with the Proof Assistant by working backwards, starting
with the theorem\index{theorem} to be proved, and assigning each unknown step
with a theorem or hypothesis until no more unknown steps remain.  The Proof
Assistant will not let you make an assignment unless it can be ``unified''
with the unknown step.  This means that a
substitution\index{substitution!variable}\index{variable substitution} of
variables exists that will make the assignment match the unknown step.  On the
other hand, in the middle of a proof, when working backwards, often more than
one unification\index{unification} (set of substitutions) is possible, since
there is not enough information available at that point to uniquely establish
it.  In this case you can tell Metamath which unification to choose, or you
can continue to assign unknown steps until enough information is available to
make the unification unique.

We will assume you have entered Metamath and read in the database as described
above.  The following dialog shows how the proof was developed.  For more
details on what some of the commands do, refer to Section~\ref{pfcommands}.
\index{\texttt{prove} command}

\begin{verbatim}
MM> prove th1
Entering the Proof Assistant.  Type HELP for help, EXIT
to exit.  You will be working on the proof of statement th1:
  $p |- t = t
Note:  The proof you are starting with is already complete.
MM-PA>
\end{verbatim}

The \verb/MM-PA>/ prompt means we are inside the Proof
Assistant.\index{Proof Assistant} Most of the regular Metamath commands
(\texttt{show statement}, etc.) are still available if you need them.

\begin{verbatim}
MM-PA> delete all
The entire proof was deleted.
\end{verbatim}

We have deleted the whole proof so we can start from scratch.

\begin{verbatim}
MM-PA> show new_proof/lemmon/all
1 ?              $? |- t = t
\end{verbatim}

The \texttt{show new{\char`\_}proof} command\index{\texttt{show
new{\char`\_}proof} command} is like \texttt{show proof} except that we
don't specify a statement; instead, the proof we're working on is
displayed.

\begin{verbatim}
MM-PA> assign 1 mp
To undo the assignment, DELETE STEP 5 and INITIALIZE, UNIFY
if needed.
3   min=?  $? |- $2
4   maj=?  $? |- ( $2 -> t = t )
\end{verbatim}

The \texttt{assign} command\index{\texttt{assign} command} above means
``assign step 1 with the statement whose label is \texttt{mp}.''  Note
that step renumbering will constantly occur as you assign steps in the
middle of a proof; in general all steps from the step you assign until
the end of the proof will get moved up.  In this case, what used to be
step 1 is now step 5, because the (partial) proof now has five steps:
the four hypotheses of the \texttt{mp} statement and the \texttt{mp}
statement itself.  Let's look at all the steps in our partial proof:

\begin{verbatim}
MM-PA> show new_proof/lemmon/all
1 ?              $? wff $2
2 ?              $? wff t = t
3 ?              $? |- $2
4 ?              $? |- ( $2 -> t = t )
5 1,2,3,4 mp     $a |- t = t
\end{verbatim}

The symbol \texttt{\$2} is a temporary variable\index{temporary
variable} that represents a symbol sequence not yet known.  In the final
proof, all temporary variables will be eliminated.  The general format
for a temporary variable is \texttt{\$} followed by an integer.  Note
that \texttt{\$} is not a legal character in a math symbol (see
Section~\ref{dollardollar}, p.~\pageref{dollardollar}), so there will
never be a naming conflict between real symbols and temporary variables.

Unknown steps 1 and 2 are constructions of the two wffs used by the
modus ponens rule.  As you will see at the end of this section, the
Proof Assistant\index{Proof Assistant} can usually figure these steps
out by itself, and we will not have to worry about them.  Therefore from
here on we will display only the ``essential'' hypotheses, i.e.\ those
steps that correspond to traditional formal proofs\index{formal proof}.

\begin{verbatim}
MM-PA> show new_proof/lemmon
3 ?              $? |- $2
4 ?              $? |- ( $2 -> t = t )
5 3,4 mp         $a |- t = t
\end{verbatim}

Unknown steps 3 and 4 are the ones we must focus on.  They correspond to the
minor and major premises of the modus ponens rule.  We will assign them as
follows.  Notice that because of the step renumbering that takes place
after an assignment, it is advantageous to assign unknown steps in reverse
order, because earlier steps will not get renumbered.

\begin{verbatim}
MM-PA> assign 4 mp
To undo the assignment, DELETE STEP 8 and INITIALIZE, UNIFY
if needed.
3   min=?  $? |- $2
6     min=?  $? |- $4
7     maj=?  $? |- ( $4 -> ( $2 -> t = t ) )
\end{verbatim}

We are now going to describe an obscure feature that you will probably
never use but should be aware of.  The Metamath language allows empty
symbol sequences to be substituted for variables, but in most formal
systems this feature is never used.  One of the few examples where is it
used is the MIU-system\index{MIU-system} described in
Appendix~\ref{MIU}.  But such systems are rare, and by default this
feature is turned off in the Proof Assistant.  (It is always allowed for
{\tt verify proof}.)  Let us turn it on and see what
happens.\index{\texttt{set empty{\char`\_}substitution} command}

\begin{verbatim}
MM-PA> set empty_substitution on
Substitutions with empty symbol sequences is now allowed.
\end{verbatim}

With this feature enabled, more unifications will be
ambiguous\index{ambiguous unification}\index{unification!ambiguous} in
the middle of a proof, because
substitution\index{substitution!variable}\index{variable substitution}
of variables with empty symbol sequences will become an additional
possibility.  Let's see what happens when we make our next assignment.

\begin{verbatim}
MM-PA> assign 3 a2
There are 2 possible unifications.  Please select the correct
    one or Q if you want to UNIFY later.
Unify:  |- $6
 with:  |- ( $9 + 0 ) = $9
Unification #1 of 2 (weight = 7):
  Replace "$6" with "( + 0 ) ="
  Replace "$9" with ""
  Accept (A), reject (R), or quit (Q) <A>? r
\end{verbatim}

The first choice presented is the wrong one.  If we had selected it,
temporary variable \texttt{\$6} would have been assigned a truncated
wff, and temporary variable \texttt{\$9} would have been assigned an
empty sequence (which is not allowed in our system).  With this choice,
eventually we would reach a point where we would get stuck because
we would end up with steps impossible to prove.  (You may want to
try it.)  We typed \texttt{r} to reject the choice.

\begin{verbatim}
Unification #2 of 2 (weight = 21):
  Replace "$6" with "( $9 + 0 ) = $9"
  Accept (A), reject (R), or quit (Q) <A>? q
To undo the assignment, DELETE STEP 4 and INITIALIZE, UNIFY
if needed.
 7     min=?  $? |- $8
 8     maj=?  $? |- ( $8 -> ( $6 -> t = t ) )
\end{verbatim}

The second choice is correct, and normally we would type \texttt{a}
to accept it.  But instead we typed \texttt{q} to show what will happen:
it will leave the step with an unknown unification, which can be
seen as follows:

\begin{verbatim}
MM-PA> show new_proof/not_unified
 4   min    $a |- $6
        =a2  = |- ( $9 + 0 ) = $9
\end{verbatim}

Later we can unify this with the \texttt{unify}
\texttt{all/interactive} command.

The important point to remember is that occasionally you will be
presented with several unification choices while entering a proof, when
the program determines that there is not enough information yet to make
an unambiguous choice automatically (and this can happen even with
\texttt{set empty{\char`\_}substitution} turned off).  Usually it is
obvious by inspection which choice is correct, since incorrect ones will
tend to be meaningless fragments of wffs.  In addition, the correct
choice will usually be the first one presented, unlike our example
above.

Enough of this digression.  Let us go back to the default setting.

\begin{verbatim}
MM-PA> set empty_substitution off
The ability to substitute empty expressions for variables
has been turned off.  Note that this may make the Proof
Assistant too restrictive in some cases.
\end{verbatim}

If we delete the proof, start over, and get to the point where
we digressed above, there will no longer be an ambiguous unification.

\begin{verbatim}
MM-PA> assign 3 a2
To undo the assignment, DELETE STEP 4 and INITIALIZE, UNIFY
if needed.
 7     min=?  $? |- $4
 8     maj=?  $? |- ( $4 -> ( ( $5 + 0 ) = $5 -> t = t ) )
\end{verbatim}

Let us look at our proof so far, and continue.

\begin{verbatim}
MM-PA> show new_proof/lemmon
 4 a2            $a |- ( $5 + 0 ) = $5
 7 ?             $? |- $4
 8 ?             $? |- ( $4 -> ( ( $5 + 0 ) = $5 -> t = t ) )
 9 7,8 mp        $a |- ( ( $5 + 0 ) = $5 -> t = t )
10 4,9 mp        $a |- t = t
MM-PA> assign 8 a1
To undo the assignment, DELETE STEP 11 and INITIALIZE, UNIFY
if needed.
 7     min=?  $? |- ( t + 0 ) = t
MM-PA> assign 7 a2
To undo the assignment, DELETE STEP 8 and INITIALIZE, UNIFY
if needed.
MM-PA> show new_proof/lemmon
 4 a2            $a |- ( t + 0 ) = t
 8 a2            $a |- ( t + 0 ) = t
12 a1            $a |- ( ( t + 0 ) = t -> ( ( t + 0 ) = t ->
                                                    t = t ) )
13 8,12 mp       $a |- ( ( t + 0 ) = t -> t = t )
14 4,13 mp       $a |- t = t
\end{verbatim}

Now all temporary variables and unknown steps have been eliminated from the
``essential'' part of the proof.  When this is achieved, the Proof
Assistant\index{Proof Assistant} can usually figure out the rest of the proof
automatically.  (Note that the \texttt{improve} command can occasionally be
useful for filling in essential steps as well, but it only tries to make use
of statements that introduce no new variables in their hypotheses, which is
not the case for \texttt{mp}. Also it will not try to improve steps containing
temporary variables.)  Let's look at the complete proof, then run
the \texttt{improve} command, then look at it again.

\begin{verbatim}
MM-PA> show new_proof/lemmon/all
 1 ?             $? wff ( t + 0 ) = t
 2 ?             $? wff t = t
 3 ?             $? term t
 4 3 a2          $a |- ( t + 0 ) = t
 5 ?             $? wff ( t + 0 ) = t
 6 ?             $? wff ( ( t + 0 ) = t -> t = t )
 7 ?             $? term t
 8 7 a2          $a |- ( t + 0 ) = t
 9 ?             $? term ( t + 0 )
10 ?             $? term t
11 ?             $? term t
12 9,10,11 a1    $a |- ( ( t + 0 ) = t -> ( ( t + 0 ) = t ->
                                                    t = t ) )
13 5,6,8,12 mp   $a |- ( ( t + 0 ) = t -> t = t )
14 1,2,4,13 mp   $a |- t = t
\end{verbatim}

\begin{verbatim}
MM-PA> improve all
A proof of length 1 was found for step 11.
A proof of length 1 was found for step 10.
A proof of length 3 was found for step 9.
A proof of length 1 was found for step 7.
A proof of length 9 was found for step 6.
A proof of length 5 was found for step 5.
A proof of length 1 was found for step 3.
A proof of length 3 was found for step 2.
A proof of length 5 was found for step 1.
Steps 1 and above have been renumbered.
CONGRATULATIONS!  The proof is complete.  Use SAVE
NEW_PROOF to save it.  Note:  The Proof Assistant does
not detect $d violations.  After saving the proof, you
should verify it with VERIFY PROOF.
\end{verbatim}

The \texttt{save new{\char`\_}proof} command\index{\texttt{save
new{\char`\_}proof} command} will save the proof in the database.  Here
we will just display it in a form that can be clipped out of a log file
and inserted manually into the database source file with a text
editor.\index{normal proof}\index{proof!normal}

\begin{verbatim}
MM-PA> show new_proof/normal
---------Clip out the proof below this line:
      tt tze tpl tt weq tt tt weq tt a2 tt tze tpl tt weq
      tt tze tpl tt weq tt tt weq wim tt a2 tt tze tpl tt
      tt a1 mp mp $.
---------The proof of 'th1' to clip out ends above this line.
\end{verbatim}

There is another proof format called ``compressed''\index{compressed
proof}\index{proof!compressed} that you will see in databases.  It is
not important to understand how it is encoded but only to recognize it
when you see it.  Its only purpose is to reduce storage requirements for
large proofs.  A compressed proof can always be converted to a normal
one and vice-versa, and the Metamath \texttt{show proof}
commands\index{\texttt{show proof} command} work equally well with
compressed proofs.  The compressed proof format is described in
Appendix~\ref{compressed}.

\begin{verbatim}
MM-PA> show new_proof/compressed
---------Clip out the proof below this line:
      ( tze tpl weq a2 wim a1 mp ) ABCZADZAADZAEZJJKFLIA
      AGHH $.
---------The proof of 'th1' to clip out ends above this line.
\end{verbatim}

Now we will exit the Proof Assistant.  Since we made changes to the proof,
it will warn us that we have not saved it.  In this case, we don't care.

\begin{verbatim}
MM-PA> exit
Warning:  You have not saved changes to the proof.
Do you want to EXIT anyway (Y, N) <N>? y
Exiting the Proof Assistant.
Type EXIT again to exit Metamath.
\end{verbatim}

The Proof Assistant\index{Proof Assistant} has several other commands
that can help you while creating proofs.  See Section~\ref{pfcommands}
for a list of them.

A command that is often useful is \texttt{minimize{\char`\_}with
*/brief}, which tries to shorten the proof.  It can make the process
more efficient by letting you write a somewhat ``sloppy'' proof then
clean up some of the fine details of optimization for you (although it
can't perform miracles such as restructuring the overall proof).

\section{A Note About Editing a Data\-base File}

Once your source file contains proofs, there are some restrictions on
how you can edit it so that the proofs remain valid.  Pay particular
attention to these rules, since otherwise you can lose a lot of work.
It is a good idea to periodically verify all proofs with \texttt{verify
proof *} to ensure their integrity.

If your file contains only normal (as opposed to compressed) proofs, the
main rule is that you may not change the order of the mandatory
hypotheses\index{mandatory hypothesis} of any statement referenced in a
later proof.  For example, if you swap the order of the major and minor
premise in the modus ponens rule, all proofs making use of that rule
will become incorrect.  The \texttt{show statement}
command\index{\texttt{show statement} command} will show you the
mandatory hypotheses of a statement and their order.

If a statement has a compressed proof, you also must not change the
order of {\em its} mandatory hypotheses.  The compressed proof format
makes use of this information as part of the compression technique.
Note that swapping the names of two variables in a theorem will change
the order of its mandatory hypotheses.

The safest way to edit a statement, say \texttt{mytheorem}, is to
duplicate it then rename the original to \texttt{mytheoremOLD}
throughout the database.  Once the edited version is re-proved, all
statements referencing \texttt{mytheoremOLD} can be updated in the Proof
Assistant using \texttt{minimize{\char`\_}with
mytheorem
/allow{\char`\_}growth}.\index{\texttt{minimize{\char`\_}with} command}
% 3/10/07 Note: line-breaking the above results in duplicate index entries

\chapter{Abstract Mathematics Revealed}\label{fol}

\section{Logic and Set Theory}\label{logicandsettheory}

\begin{quote}
  {\em Set theory can be viewed as a form of exact theology.}
  \flushright\sc  Rudy Rucker\footnote{\cite{Barrow}, p.~31.}\\
\end{quote}\index{Rucker, Rudy}

Despite its seeming complexity, all of standard mathematics, no matter how
deep or abstract, can amazingly enough be derived from a relatively small set
of axioms\index{axiom} or first principles. The development of these axioms is
among the most impressive and important accomplishments of mathematics in the
20th century. Ultimately, these axioms can be broken down into a set of rules
for manipulating symbols that any technically oriented person can follow.

We will not spend much time trying to convey a deep, higher-level
understanding of the meaning of the axioms. This kind of understanding
requires some mathematical sophistication as well as an understanding of the
philosophy underlying the foundations of mathematics and typically develops
over time as you work with mathematics.  Our goal, instead, is to give you the
immediate ability to follow how theorems\index{theorem} are derived from the
axioms and from other theorems.  This will be similar to learning the syntax
of a computer language, which lets you follow the details in a program but
does not necessarily give you the ability to write non-trivial programs on
your own, an ability that comes with practice. For now don't be alarmed by
abstract-sounding names of the axioms; just focus on the rules for
manipulating the symbols, which follow the simple conventions of the
Metamath\index{Metamath} language.

The axioms that underlie all of standard mathematics consist of axioms of logic
and axioms of set theory. The axioms of logic are divided into two
subcategories, propositional calculus\index{propositional calculus} (sometimes
called sentential logic\index{sentential logic}) and predicate calculus
(sometimes called first-order logic\index{first-order logic}\index{quantifier
theory}\index{predicate calculus} or quantifier theory).  Propositional
calculus is a prerequisite for predicate calculus, and predicate calculus is a
prerequisite for set theory.  The version of set theory most commonly used is
Zermelo--Fraenkel set theory\index{Zermelo--Fraenkel set theory}\index{set theory}
with the axiom of choice,
often abbreviated as ZFC\index{ZFC}.

Here in a nutshell is what the axioms are all about in an informal way. The
connection between this description and symbols we will show you won't be
immediately apparent and in principle needn't ever be.  Our description just
tries to summarize what mathematicians think about when they work with the
axioms.

Logic is a set of rules that allow us determine truths given other truths.
Put another way,
logic is more or less the translation of what we would consider common sense
into a rigorous set of axioms.\index{axioms of logic}  Suppose $\varphi$,
$\psi$, and $\chi$ (the Greek letters phi, psi, and chi) represent statements
that are either true or false, and $x$ is a variable\index{variable!in predicate
calculus} ranging over some group of mathematical objects (sets, integers,
real numbers, etc.). In mathematics, a ``statement'' really means a formula,
and $\psi$ could be for example ``$x = 2$.''
Propositional calculus\index{propositional calculus}
allows us to use variables that are either true or false
and make deductions such as
``if $\varphi$ implies $\psi$ and $\psi$ implies $\chi$, then $\varphi$
implies $\chi$.''
Predicate calculus\index{predicate calculus}
extends propositional calculus by also allowing us
to discuss statements about objects (not just true and false values), including
statements about ``all'' or ``at least one'' object.
For example, predicate calculus allows to say,
``if $\varphi$ is true for all $x$, then $\varphi$ is true for some $x$.''
The logic used in \texttt{set.mm} is standard classical logic
(as opposed to other logic systems like intuitionistic logic).

Set theory\index{set theory} has to do with the manipulation of objects and
collections of objects, specifically the abstract, imaginary objects that
mathematics deals with, such as numbers. Everything that is claimed to exist
in mathematics is considered to be a set.  A set called the empty
set\index{empty set} contains nothing.  We represent the empty set by
$\varnothing$.  Many sets can be built up from the empty set.  There is a set
represented by $\{\varnothing\}$ that contains the empty set, another set
represented by $\{\varnothing,\{\varnothing\}\}$ that contains this set as
well as the empty set, another set represented by $\{\{\varnothing\}\}$ that
contains just the set that contains the empty set, and so on ad infinitum. All
mathematical objects, no matter how complex, are defined as being identical to
certain sets: the integer\index{integer} 0 is defined as the empty set, the
integer 1 is defined as $\{\varnothing\}$, the integer 2 is defined as
$\{\varnothing,\{\varnothing\}\}$.  (How these definitions were chosen doesn't
matter now, but the idea behind it is that these sets have the properties we
expect of integers once suitable operations are defined.)  Mathematical
operations, such as addition, are defined in terms of operations on
sets---their union\index{set union}, intersection\index{set intersection}, and
so on---operations you may have used in elementary school when you worked
with groups of apples and oranges.

With a leap of faith, the axioms also postulate the existence of infinite
sets\index{infinite set}, such as the set of all non-negative integers ($0, 1,
2,\ldots$, also called ``natural numbers''\index{natural number}).  This set
can't be represented with the brace notation\index{brace notation} we just
showed you, but requires a more complicated notation called ``class
abstraction.''\index{class abstraction}\index{abstraction class}  For
example, the infinite set $\{ x |
\mbox{``$x$ is a natural number''} \} $ means the ``set of all objects $x$
such that $x$ is a natural number'' i.e.\ the set of natural numbers; here,
``$x$ is a natural number'' is a rather complicated formula when broken down
into the primitive symbols.\label{expandom}\footnote{The statement ``$x$ is a
natural number'' is formally expressed as ``$x \in \omega$,'' where $\in$
(stylized epsilon) means ``is in'' or ``is an element of'' and $\omega$
(omega) means ``the set of natural numbers.''  When ``$x\in\omega$'' is
completely expanded in terms of the primitive symbols of set theory, the
result is  $\lnot$ $($ $\lnot$ $($ $\forall$ $z$ $($ $\lnot$ $\forall$ $w$ $($
$z$ $\in$ $w$ $\rightarrow$ $\lnot$ $w$ $\in$ $x$ $)$ $\rightarrow$ $z$ $\in$
$x$ $)$ $\rightarrow$ $($ $\forall$ $z$ $($ $\lnot$ $($ $\forall$ $w$ $($ $w$
$\in$ $z$ $\rightarrow$ $w$ $\in$ $x$ $)$ $\rightarrow$ $\forall$ $w$ $\lnot$
$w$ $\in$ $z$ $)$ $\rightarrow$ $\lnot$ $\forall$ $w$ $($ $w$ $\in$ $z$
$\rightarrow$ $\lnot$ $\forall$ $v$ $($ $v$ $\in$ $z$ $\rightarrow$ $\lnot$
$v$ $\in$ $w$ $)$ $)$ $)$ $\rightarrow$ $\lnot$ $\forall$ $z$ $\forall$ $w$
$($ $\lnot$ $($ $z$ $\in$ $x$ $\rightarrow$ $\lnot$ $w$ $\in$ $x$ $)$
$\rightarrow$ $($ $\lnot$ $z$ $\in$ $w$ $\rightarrow$ $($ $\lnot$ $z$ $=$ $w$
$\rightarrow$ $w$ $\in$ $z$ $)$ $)$ $)$ $)$ $)$ $\rightarrow$ $\lnot$
$\forall$ $y$ $($ $\lnot$ $($ $\lnot$ $($ $\forall$ $z$ $($ $\lnot$ $\forall$
$w$ $($ $z$ $\in$ $w$ $\rightarrow$ $\lnot$ $w$ $\in$ $y$ $)$ $\rightarrow$
$z$ $\in$ $y$ $)$ $\rightarrow$ $($ $\forall$ $z$ $($ $\lnot$ $($ $\forall$
$w$ $($ $w$ $\in$ $z$ $\rightarrow$ $w$ $\in$ $y$ $)$ $\rightarrow$ $\forall$
$w$ $\lnot$ $w$ $\in$ $z$ $)$ $\rightarrow$ $\lnot$ $\forall$ $w$ $($ $w$
$\in$ $z$ $\rightarrow$ $\lnot$ $\forall$ $v$ $($ $v$ $\in$ $z$ $\rightarrow$
$\lnot$ $v$ $\in$ $w$ $)$ $)$ $)$ $\rightarrow$ $\lnot$ $\forall$ $z$
$\forall$ $w$ $($ $\lnot$ $($ $z$ $\in$ $y$ $\rightarrow$ $\lnot$ $w$ $\in$
$y$ $)$ $\rightarrow$ $($ $\lnot$ $z$ $\in$ $w$ $\rightarrow$ $($ $\lnot$ $z$
$=$ $w$ $\rightarrow$ $w$ $\in$ $z$ $)$ $)$ $)$ $)$ $\rightarrow$ $($
$\forall$ $z$ $\lnot$ $z$ $\in$ $y$ $\rightarrow$ $\lnot$ $\forall$ $w$ $($
$\lnot$ $($ $w$ $\in$ $y$ $\rightarrow$ $\lnot$ $\forall$ $z$ $($ $w$ $\in$
$z$ $\rightarrow$ $\lnot$ $z$ $\in$ $y$ $)$ $)$ $\rightarrow$ $\lnot$ $($
$\lnot$ $\forall$ $z$ $($ $w$ $\in$ $z$ $\rightarrow$ $\lnot$ $z$ $\in$ $y$
$)$ $\rightarrow$ $w$ $\in$ $y$ $)$ $)$ $)$ $)$ $\rightarrow$ $x$ $\in$ $y$
$)$ $)$ $)$. Section~\ref{hierarchy} shows the hierarchy of definitions that
leads up to this expression.}\index{stylized epsilon ($\in$)}\index{omega
($\omega$)}  Actually, the primitive symbols don't even include the brace
notation.  The brace notation is a high-level definition, which you can find in
Section~\ref{hierarchy}.

Interestingly, the arithmetic of integers\index{integer} and
rationals\index{rational number} can be developed without appealing to the
existence of an infinite set, whereas the arithmetic of real
numbers\index{real number} requires it.

Each variable\index{variable!in set theory} in the axioms of set theory
represents an arbitrary set, and the axioms specify the legal kinds of things
you can do with these variables at a very primitive level.

Now, you may think that numbers and arithmetic are a lot more intuitive and
fundamental than sets and therefore should be the foundation of mathematics.
What is really the case is that you've dealt with numbers all your life and
are comfortable with a few rules for manipulating them such as addition and
multiplication.  Those rules only cover a small portion of what can be done
with numbers and only a very tiny fraction of the rest of mathematics.  If you
look at any elementary book on number theory, you will quickly become lost if
these are the only rules that you know.  Even though such books may present a
list of ``axioms''\index{axiom} for arithmetic, the ability to use the axioms
and to understand proofs of theorems\index{theorem} (facts) about numbers
requires an implicit mathematical talent that frustrates many people
from studying abstract mathematics.  The kind of mathematics that most people
know limits them to the practical, everyday usage of blindly manipulating
numbers and formulas, without any understanding of why those rules are correct
nor any ability to go any further.  For example, do you know why multiplying
two negative numbers yields a positive number?  Starting with set theory, you
will also start off blindly manipulating symbols according to the rules we give
you, but with the advantage that these rules will allow you, in principle, to
access {\em all} of mathematics, not just a tiny part of it.

Of course, concrete examples are often helpful in the learning process. For
example, you can verify that $2\cdot 3=3 \cdot 2$ by actually grouping
objects and can easily ``see'' how it generalizes to $x\cdot y = y\cdot x$,
even though you might not be able to rigorously prove it.  Similarly, in set
theory it can be helpful to understand how the axioms of set theory apply to
(and are correct for) small finite collections of objects.  You should be aware
that in set theory intuition can be misleading for infinite collections, and
rigorous proofs become more important.  For example, while $x\cdot y = y\cdot
x$ is correct for finite ordinals (which are the natural numbers), it is not
usually true for infinite ordinals.

\section{The Axioms for All of Mathematics}

In this section\index{axioms for mathematics}, we will show you the axioms
for all of standard mathematics (i.e.\ logic and set theory) as they are
traditionally presented.  The traditional presentation is useful for someone
with the mathematical experience needed to correctly manipulate high-level
abstract concepts.  For someone without this talent, knowing how to actually
make use of these axioms can be difficult.  The purpose of this section is to
allow you to see how the version of the axioms used in the standard
Metamath\index{Metamath} database \texttt{set.mm}\index{set
theory database (\texttt{set.mm})} relates to  the typical version
in textbooks, and also to give you an informal feel for them.

\subsection{Propositional Calculus}

Propositional calculus\index{propositional calculus} concerns itself with
statements that can be interpreted as either true or false.  Some examples of
statements (outside of mathematics) that are either true or false are ``It is
raining today'' and ``The United States has a female president.'' In
mathematics, as we mentioned, statements are really formulas.

In propositional calculus, we don't care what the statements are.  We also
treat a logical combination of statements, such as ``It is raining today and
the United States has a female president,'' no differently from a single
statement.  Statements and their combinations are called well-formed formulas
(wffs)\index{well-formed formula (wff)}.  We define wffs only in terms of
other wffs and don't define what a ``starting'' wff is.  As is common practice
in the literature, we use Greek letters to represent wffs.

Specifically, suppose $\varphi$ and $\psi$ are wffs.  Then the combinations
$\varphi\rightarrow\psi$ (``$\varphi$ implies $\psi$,'' also read ``if
$\varphi$ then $\psi$'')\index{implication ($\rightarrow$)} and $\lnot\varphi$
(``not $\varphi$'')\index{negation ($\lnot$)} are also wffs.

The three axioms of propositional calculus\index{axioms of propositional
calculus} are all wffs of the following form:\footnote{A remarkable result of
C.~A.~Meredith\index{Meredith, C. A.} squeezes these three axioms into the
single axiom $((((\varphi\rightarrow \psi)\rightarrow(\neg \chi\rightarrow\neg
\theta))\rightarrow \chi )\rightarrow \tau)\rightarrow((\tau\rightarrow
\varphi)\rightarrow(\theta\rightarrow \varphi))$ \cite{CAMeredith},
which is believed to be the shortest possible.}
\begin{center}
     $\varphi\rightarrow(\psi\rightarrow \varphi)$\\

     $(\varphi\rightarrow (\psi\rightarrow \chi))\rightarrow
((\varphi\rightarrow  \psi)\rightarrow (\varphi\rightarrow \chi))$\\

     $(\neg \varphi\rightarrow \neg\psi)\rightarrow (\psi\rightarrow
\varphi)$
\end{center}

These three axioms are widely used.
They are attributed to Jan {\L}ukasiewicz
(pronounced woo-kah-SHAY-vitch) and was popularized by Alonzo Church,
who called it system P2. (Thanks to Ted Ulrich for this information.)

There are an infinite number of axioms, one for each possible
wff\index{well-formed formula (wff)} of the above form.  (For this reason,
axioms such as the above are often called ``axiom schemes.''\index{axiom
scheme})  Each Greek letter in the axioms may be substituted with a more
complex wff to result in another axiom.  For example, substituting
$\neg(\varphi\rightarrow\chi)$ for $\varphi$ in the first axiom yields
$\neg(\varphi\rightarrow\chi)\rightarrow(\psi\rightarrow
\neg(\varphi\rightarrow\chi))$, which is still an axiom.

To deduce new true statements (theorems\index{theorem}) from the axioms, a
rule\index{rule} called ``modus ponens''\index{modus ponens} is used.  This
rule states that if the wff $\varphi$ is an axiom or a theorem, and the wff
$\varphi\rightarrow\psi$ is an axiom or a theorem, then the wff $\psi$ is also
a theorem\index{theorem}.

As a non-mathematical example of modus ponens, suppose we have proved (or
taken as an axiom) ``Bob is a man'' and separately have proved (or taken as
an axiom) ``If Bob is a man, then Bob is a human.''  Using the rule of modus
ponens, we can logically deduce, ``Bob is a human.''

From Metamath's\index{Metamath} point of view, the axioms and the rule of
modus ponens just define a mechanical means for deducing new true statements
from existing true statements, and that is the complete content of
propositional calculus as far as Metamath is concerned.  You can read a logic
textbook to gain a better understanding of their meaning, or you can just let
their meaning slowly become apparent to you after you use them for a while.

It is actually rather easy to check to see if a formula is a theorem of
propositional calculus.  Theorems of propositional calculus are also called
``tautologies.''\index{tautology}  The technique to check whether a formula is
a tautology is called the ``truth table method,''\index{truth table} and it
works like this.  A wff $\varphi\rightarrow\psi$ is false whenever $\varphi$ is true
and $\psi$ is false.  Otherwise it is true.  A wff $\lnot\varphi$ is false
whenever $\varphi$ is true and false otherwise. To verify a tautology such as
$\varphi\rightarrow(\psi\rightarrow \varphi)$, you break it down into sub-wffs and
construct a truth table that accounts for all possible combinations of true
and false assigned to the wff metavariables:
\begin{center}\begin{tabular}{|c|c|c|c|}\hline
\mbox{$\varphi$} & \mbox{$\psi$} & \mbox{$\psi\rightarrow\varphi$}
    & \mbox{$\varphi\rightarrow(\psi\rightarrow \varphi)$} \\ \hline \hline
              T   &  T    &      T       &        T    \\ \hline
              T   &  F    &      T       &        T    \\ \hline
              F   &  T    &      F       &        T    \\ \hline
              F   &  F    &      T       &        T    \\ \hline
\end{tabular}\end{center}
If all entries in the last column are true, the formula is a tautology.

Now, the truth table method doesn't tell you how to prove the tautology from
the axioms, but only that a proof exists.  Finding an actual proof (especially
one that is short and elegant) can be challenging.  Methods do exist for
automatically generating proofs in propositional calculus, but the proofs that
result can sometimes be very long.  In the Metamath \texttt{set.mm}\index{set
theory database (\texttt{set.mm})} database, most
or all proofs were created manually.

Section \ref{metadefprop} discusses various definitions
that make propositional calculus easier to use.
For example, we define:

\begin{itemize}
\item $\varphi \vee \psi$
  is true if either $\varphi$ or $\psi$ (or both) are true
  (this is disjunction\index{disjunction ($\vee$)}
  aka logical {\sc or}\index{logical {\sc or} ($\vee$)}).

\item $\varphi \wedge \psi$
  is true if both $\varphi$ and $\psi$ are true
  (this is conjunction\index{conjunction ($\wedge$)}
  aka logical {\sc and}\index{logical {\sc and} ($\wedge$)}).

\item $\varphi \leftrightarrow \psi$
  is true if $\varphi$ and $\psi$ have the same value, that is,
  they are both true or both false
  (this is the biconditional\index{biconditional ($\leftrightarrow$)}).
\end{itemize}

\subsection{Predicate Calculus}

Predicate calculus\index{predicate calculus} introduces the concept of
``individual variables,''\index{variable!in predicate calculus}\index{individual
variable} which
we will usually just call ``variables.''
These variables can represent something other than true or false (wffs),
and will always represent sets when we get to set theory.  There are also
three new symbols $\forall$\index{universal quantifier ($\forall$)},
$=$\index{equality ($=$)}, and $\in$\index{stylized epsilon ($\in$)},
read ``for all,'' ``equals,'' and ``is an element of''
respectively.  We will represent variables with the letters $x$, $y$, $z$, and
$w$, as is common practice in the literature.
For example, $\forall x \varphi$ means ``for all possible values of
$x$, $\varphi$ is true.''

In predicate calculus, we extend the definition of a wff\index{well-formed
formula (wff)}.  If $\varphi$ is a wff and $x$ and $y$ are variables, then
$\forall x \, \varphi$, $x=y$, and $x\in y$ are wffs. Note that these three new
types of wffs can be considered ``starting'' wffs from which we can build
other wffs with $\rightarrow$ and $\neg$ .  The concept of a starting wff was
absent in propositional calculus.  But starting wff or not, all we are really
concerned with is whether our wffs are correctly constructed according to
these mechanical rules.

A quick aside:
To prevent confusion, it might be best at this point to think of the variables
of Metamath\index{Metamath} as ``metavariables,''\index{metavariable} because
they are not quite the same as the variables we are introducing here.  A
(meta)variable in Metamath can be a wff or an individual variable, as well
as many other things; in general, it represents a kind of place holder for an
unspecified sequence of math symbols\index{math symbol}.

Unlike propositional calculus, no decision procedure\index{decision procedure}
analogous to the truth table method exists (nor theoretically can exist) that
will definitely determine whether a formula is a theorem of predicate
calculus.  Much of the work in the field of automated theorem
proving\index{automated theorem proving} has been dedicated to coming up with
clever heuristics for proving theorems of predicate calculus, but they can
never be guaranteed to work always.

Section \ref{metadefpred} discusses various definitions
that make predicate calculus easier to use.
For example, we define
$\exists x \varphi$ to mean
``there exists at least one possible value of $x$ where $\varphi$ is true.''

We now turn to looking at how predicate calculus can be formally
represented.

\subsubsection{Common Axioms}

There is a new rule of inference in predicate calculus:  if $\varphi$ is
an axiom or a theorem, then $\forall x \,\varphi$ is also a
theorem\index{theorem}.  This is called the rule of
``generalization.''\index{rule of generalization}
This is easily represented in Metamath.

In standard texts of logic, there are often two axioms of predicate
calculus\index{axioms of predicate calculus}:
\begin{center}
  $\forall x \,\varphi ( x ) \rightarrow \varphi ( y )$,
      where ``$y$ is properly substituted for $x$.''\\
  $\forall x ( \varphi \rightarrow \psi )\rightarrow ( \varphi \rightarrow
    \forall x\, \psi )$,
    where ``$x$ is not free in $\varphi$.''
\end{center}

Now at first glance, this seems simple:  just two axioms.  However,
conditional clauses are attached to each axiom describing requirements that
may seem puzzling to you.  In addition, the first axiom puts a variable symbol
in parentheses after each wff, seemingly violating our definition of a
wff\index{well-formed formula (wff)}; this is just an informal way of
referring to some arbitrary variable that may occur in the wff.  The
conditional clauses do, of course, have a precise meaning, but as it turns out
the precise meaning is somewhat complicated and awkward to formalize in a
way that a computer can handle easily.  Unlike propositional calculus, a
certain amount of mathematical sophistication and practice is needed to be
able to easily grasp and manipulate these concepts correctly.

Predicate calculus may be presented with or without axioms for
equality\index{axioms of equality}\index{equality ($=$)}. We will require the
axioms of equality as a prerequisite for the version of set theory we will
use.  The axioms for equality, when included, are often represented using these
two axioms:
\begin{center}
$x=x$\\ \ \\
$x=y\rightarrow (\varphi(x,x)\rightarrow\varphi(x,y))$ where ``$\varphi(x,y)$
   arises from $\varphi(x,x)$ by replacing some, but not necessarily all,
   free\index{free variable}
   occurrences of $x$ by $y$,\\ provided that $y$ is free for $x$
   in $\varphi(x,x)$.'' \end{center}
% (Mendelson p. 95)
The first equality axiom is simple, but again,
the condition on the second one is
somewhat awkward to implement on a computer.

\subsubsection{Tarski System S2}

Of course, we are not the first to notice the complications of these
predicate calculus axioms when being rigorous.

Well-known logician Alfred Tarski published in 1965
a system he called system S2\cite[p.~77]{Tarski1965}.
Tarski's system is \textit{exactly equivalent} to the traditional textbook
formalization, but (by clever use of equality axioms) it eliminates the
latter's primitive notions of ``proper substitution'' and ``free variable,''
replacing them with direct substitution and the notion of a variable
not occurring in a formula (which we express with distinct variable
constraints).

In advocating his system, Tarski wrote, ``The relatively complicated
character of [free variables and proper substitution] is a source
of certain inconveniences of both practical and theoretical nature;
this is clearly experienced both in teaching an elementary course of
mathematical logic and in formalizing the syntax of predicate logic for
some theoretical purposes''\cite[p.~61]{Tarski1965}\index{Tarski, Alfred}.

\subsubsection{Developing a Metamath Representation}

The standard textbook axioms of predicate calculus are somewhat
cumbersome to implement on a computer because of the complex notions of
``free variable''\index{free variable} and ``proper
substitution.''\index{proper substitution}\index{substitution!proper}
While it is possible to use the Metamath\index{Metamath} language to
implement these concepts, we have chosen not to implement them
as primitive constructs in the
\texttt{set.mm} set theory database.  Instead, we have eliminated them
within the axioms
by carefully crafting the axioms so as to avoid them,
building on Tarski's system S2.  This makes it
easy for a beginner to follow the steps in a proof without knowing any
advanced concepts other than the simple concept of
replacing\index{substitution!variable}\index{variable substitution}
variables with expressions.

In order to develop the concepts of free variable and proper
substitution from the axioms, we use an additional
Metamath statement type called ``disjoint variable
restriction''\index{disjoint variables} that we have not encountered
before.  In the context of the axioms, the statement \texttt{\$d} $ x\,
y$\index{\texttt{\$d} statement} simply means that $x$ and $y$ must be
distinct\index{distinct variables}, i.e.\ they may not be simultaneously
substituted\index{substitution!variable}\index{variable substitution}
with the same variable.  The statement \texttt{\$d} $ x\, \varphi$ means
variable $x$ must not occur in wff $\varphi$.  For the precise
definition of \texttt{\$d}, see Section~\ref{dollard}.

\subsubsection{Metamath representation}

The Metamath axiom system for predicate calculus
defined in set.mm uses Tarski's system S2.
As noted above, this has a different representation
than the traditional textbook formalization,
but it is \textit{exactly equivalent} to the textbook formalization,
and it is \textit{much} easier to work with.
This is reproduced as system S3 in Section 6 of
Megill's formalization \cite{Megill}\index{Megill, Norman}.

There is one exception, Tarski's axiom of existence,
which we label as axiom ax-6.
In the case of ax-6, Tarski's version is weaker because it includes a
distinct variable proviso. If we wish, we can also weaken our version
in this way and still have a metalogically complete system. Theorem
ax6 shows this by deriving, in the presence of the other axioms, our
ax-6 from Tarski's weaker version ax6v. However, we chose the stronger
version for our system because it is simpler to state and easier to use.

Tarski's system was designed for proving specific theorems rather than
more general theorem schemes. However, theorem schemes are much more
efficient than specific theorems for building a body of mathematical
knowledge, since they can be reused with different instances as
needed. While Tarski does derive some theorem schemes from his axioms,
their proofs require concepts that are ``outside'' of the system, such as
induction on formula length. The verification of such proofs is difficult
to automate in a proof verifier. (Specifically, Tarski treats the formulas
of his system as set-theoretical objects. In order to verify the proofs
of his theorem schemes, a proof verifier would need a significant amount
of set theory built into it.)

The Metamath axiom system for predicate calculus extends
Tarski's system to eliminate this difficulty. The additional
``auxilliary'' axiom
schemes (as we will call them in this section; see below) endow Tarski's
system with a nice property we call
metalogical completeness \cite[Remark 9.6]{Megill}\index{Megill, Norman}.
As a result, we can prove any theorem scheme
expressable in the ``simple metalogic'' of Tarski's system by using
only Metamath's direct substitution rule applied to the axiom system
(and no other metalogical or set-theoretical notions ``outside'' of the
system). Simple metalogic consists of schemes containing wff metavariables
(with no arguments) and/or set (also called ``individual'') metavariables,
accompanied by optional provisos each stating that two specified set
metavariables must be distinct or that a specified set metavariable may
not occur in a specified wff metavariable. Metamath's logic and set theory
axiom and rule schemes are all examples of simple metalogic. The schemes
of traditional predicate calculus with equality are examples which are
not simple metalogic, because they use wff metavariables with arguments
and have ``free for'' and ``not free in'' side conditions.

A rigorous justification for this system, using an older but
exactly equivalent set of axioms, can be
found in \cite{Megill}\index{Megill, Norman}.

This allows us to
take a different approach in the Metamath\index{Metamath} database
\texttt{set.mm}\index{set theory database (\texttt{set.mm})}.  We do not
directly use the primitive notions of ``free variable''\index{free variable}
and ``proper substitution''\index{proper
substitution}\index{substitution!proper} at all as primitive constructs.
Instead, we use a set
of axioms that are almost as simple to manipulate as those of
propositional calculus.  Our axiom system avoids complex primitive
notions by effectively embedding the complexity into the axioms
themselves.  As a result, we will end up with a larger number of axioms,
but they are ideally suited for a computer language such as Metamath.
(Section~\ref{metaaxioms} shows these axioms.)

We will not elaborate further
on the ``free variable'' and ``proper substitution''
concepts here.  You may consult
\cite[ch.\ 3--4]{Hamilton}\index{Hamilton, Alan G.} (as well as
many other books) for a precise explanation
of these concepts.  If you intend to do serious mathematical work, it is wise
to become familiar with the traditional textbook approach; even though the
concepts embedded in their axioms require a higher level of sophistication,
they can be more practical to deal with on an everyday, informal basis.  Even
if you are just developing Metamath proofs, familiarity with the traditional
approach can help you arrive at a proof outline much faster, which you can
then convert to the detail required by Metamath.

We do develop proper substitution rules later on, but in set.mm
they are defined as derived constructs; they are not primitives.

You should also note that our system of predicate calculus is specifically
tailored for set theory; thus there are only two specific predicates $=$ and
$\in$ and no functions\index{function!in predicate calculus}
or constants\index{constant!in predicate calculus} unlike more general systems.
We later add these.

\subsection{Set Theory}

Traditional Zermelo--Fraenkel set theory\index{Zermelo--Fraenkel set
theory}\index{set theory} with the Axiom of Choice
has 10 axioms, which can be expressed in the
language of predicate calculus.  In this section, we will list only the
names and brief English descriptions of these axioms, since we will give
you the precise formulas used by the Metamath\index{Metamath} set theory
database \texttt{set.mm} later on.

In the descriptions of the axioms, we assume that $x$, $y$, $z$, $w$, and $v$
represent sets.  These are the same as the variables\index{variable!in set
theory} in our predicate calculus system above, except that now we informally
think of the variables as ranging over sets.  Note that the terms
``object,''\index{object} ``set,''\index{set} ``element,''\index{element}
``collection,''\index{collection} and ``family''\index{family} are synonymous,
as are ``is an element of,'' ``is a member of,''\index{member} ``is contained
in,'' and ``belongs to.''  The different terms are used for convenience; for
example, ``a collection of sets'' is less confusing than ``a set of sets.''
A set $x$ is said to be a ``subset''\index{subset} of $y$ if every element of
$x$ is also an element of $y$; we also say $x$ is ``included in''
$y$.

The axioms are very general and apply to almost any conceivable mathematical
object, and this level of abstraction can be overwhelming at first.  To gain an
intuitive feel, it can be helpful to draw a picture illustrating the concept;
for example, a circle containing dots could represent a collection of sets,
and a smaller circle drawn inside the circle could represent a subset.
Overlapping circles can illustrate intersection and union.  Circles that
illustrate the concepts of set theory are frequently used in elementary
textbooks and are called Venn diagrams\index{Venn diagram}.\index{axioms of
set theory}

1. Axiom of Extensionality:  Two sets are identical if they contain the same
   elements.\index{Axiom of Extensionality}

2. Axiom of Pairing:  The set $\{ x , y \}$ exists.\index{Axiom of Pairing}

3. Axiom of Power Sets:  The power set of a set (the collection of all of
   its subsets) exists.  For example, the power set of $\{x,y\}$ is
   $\{\varnothing,\{x\},\{y\},\{x,y\}\}$ and it exists.\index{Axiom
of Power Sets}

4. Axiom of the Null Set:  The empty set $\varnothing$ exists.\index{Axiom of
the Null Set}

5. Axiom of Union:  The union of a set (the set containing the elements of
   its members) exists.  For example, the union of $\{\{x,y\},\{z\}\}$ is
 $\{x,y,z\}$ and
   it exists.\index{Axiom of Union}

6. Axiom of Regularity:  Roughly, no set can contain itself, nor can there
   be membership ``loops,'' such as a set being an
   element of one of its members.\index{Axiom of Regularity}

7. Axiom of Infinity:  An infinite set exists.  An example of an infinite
   set is the set of all
   integers.\index{Axiom of Infinity}

8. Axiom of Separation:  The set exists that is obtained by restricting $x$
   with some property.  For example, if the set of all integers exists,
   then the set of all even integers exists.\index{Axiom of Separation}

9. Axiom of Replacement:  The range of a function whose domain is restricted
   to the elements of a set $x$, is also a set.  For example, there
   is a function
   from integers (the function's domain) to their squares (its
   range).  If we
   restrict the domain to even integers, its range will become the set of
   squares of even integers, so this axiom asserts that the set of
    squares of even numbers exists.  Technical note:  In general, the
   ``function'' need not be a set but can be a proper class.
   \index{Axiom of Replacement}

10. Axiom of Choice:  Let $x$ be a set whose members are pairwise
  disjoint\index{disjoint sets} (i.e,
  whose members contain no elements in common).  Then there exists another
  set containing one element from each member of $x$.  For
  example, if $x$ is
  $\{\{y,z\},\{w,v\}\}$, where $y$, $z$, $w$, and $v$ are
  different sets, then a set such as $\{z,w\}$
  exists (but the axiom doesn't tell
  us which one).  (Actually the Axiom
  of Choice is redundant if the set $x$, as in this example, has a finite
  number of elements.)\index{Axiom of Choice}

The Axiom of Choice is usually considered an extension of ZF set theory rather
than a proper part of it.  It is sometimes considered philosophically
controversial because it specifies the existence of a set without specifying
what the set is. Constructive logics, including intuitionistic logic,
do not accept the axiom of choice.
Since there is some lingering controversy, we often prefer proofs that do
not use the axiom of choice (where there is a known alternative), and
in some cases we will use weaker axioms than the full axiom of choice.
That said, the axiom of choice is a powerful and widely-accepted tool,
so we do use it when needed.
ZF set theory that includes the Axiom of Choice is
called Zermelo--Fraenkel set theory with choice (ZFC\index{ZFC set theory}).

When expressed symbolically, the Axiom of Separation and the Axiom of
Replacement contain wff symbols and therefore each represent infinitely many
axioms, one for each possible wff. For this reason, they are often called
axiom schemes\index{axiom scheme}\index{well-formed formula (wff)}.

It turns out that the Axiom of the Null Set, the Axiom of Pairing, and the
Axiom of Separation can be derived from the other axioms and are therefore
unnecessary, although they tend to be included in standard texts for various
reasons (historical, philosophical, and possibly because some authors may not
know this).  In the Metamath\index{Metamath} set theory database, these
redundant axioms are derived from the other ones instead of truly
being considered axioms.
This is in keeping with our general goal of minimizing the number of
axioms we must depend on.

\subsection{Other Axioms}

Above we qualified the phrase ``all of mathematics'' with ``essentially.''
The main important missing piece is the ability to do category theory,
which requires huge sets (inaccessible cardinals) larger than those
postulated by the ZFC axioms. The Tarski--Grothendieck Axiom postulates
the existence of such sets.
Note that this is the same axiom used by Mizar for supporting
category theory.
The Tarski--Grothendieck axiom
can be viewed as a very strong replacement of the Axiom of Infinity,
the Axiom of Choice, and the Axiom of Power Sets.
The \texttt{set.mm} database includes this axiom; see the database
for details about it.
Again, we only use this axiom when we need to.
You are only likely to encounter or use this axiom if you are doing
category theory, since its use is highly specialized,
so we will not list the Tarsky-Grothendieck axiom
in the short list of axioms below.

Can there be even more axioms?
Of course.
G\"{o}del showed that no finite set of axioms or axiom schemes can completely
describe any consistent theory strong enough to include arithmetic.
But practically speaking, the ones above are the accepted foundation that
almost all mathematicians explicitly or implicitly base their work on.

\section{The Axioms in the Metamath Language}\label{metaaxioms}

Here we list the axioms as they appear in
\texttt{set.mm}\index{set theory database (\texttt{set.mm})} so you can
look them up there easily.  Incidentally, the \texttt{show statement
/tex} command\index{\texttt{show statement} command} was used to
typeset them.

%macros from show statement /tex
\newbox\mlinebox
\newbox\mtrialbox
\newbox\startprefix  % Prefix for first line of a formula
\newbox\contprefix  % Prefix for continuation line of a formula
\def\startm{  % Initialize formula line
  \setbox\mlinebox=\hbox{\unhcopy\startprefix}
}
\def\m#1{  % Add a symbol to the formula
  \setbox\mtrialbox=\hbox{\unhcopy\mlinebox $\,#1$}
  \ifdim\wd\mtrialbox>\hsize
    \box\mlinebox
    \setbox\mlinebox=\hbox{\unhcopy\contprefix $\,#1$}
  \else
    \setbox\mlinebox=\hbox{\unhbox\mtrialbox}
  \fi
}
\def\endm{  % Output the last line of a formula
  \box\mlinebox
}

% \SLASH for \ , \TOR for \/ (text OR), \TAND for /\ (text and)
% This embeds a following forced space to force the space.
\newcommand\SLASH{\char`\\~}
\newcommand\TOR{\char`\\/~}
\newcommand\TAND{/\char`\\~}
%
% Macro to output metamath raw text.
% This assumes \startprefix and \contprefix are set.
% NOTE: "\" is tricky to escape, use \SLASH, \TOR, and \TAND inside.
% Any use of "$ { ~ ^" must be escaped; ~ and ^ must be escaped specially.
% We escape { and } for consistency.
% For more about how this macro written, see:
% https://stackoverflow.com/questions/4073674/
% how-to-disable-indentation-in-particular-section-in-latex/4075706
% Use frenchspacing, or "e." will get an extra space after it.
\newlength\mystoreparindent
\newlength\mystorehangindent
\newenvironment{mmraw}{%
\setlength{\mystoreparindent}{\the\parindent}
\setlength{\mystorehangindent}{\the\hangindent}
\setlength{\parindent}{0pt} % TODO - we'll put in the \startprefix instead
\setlength{\hangindent}{\wd\the\contprefix}
\begin{flushleft}
\begin{frenchspacing}
\begin{tt}
{\unhcopy\startprefix}%
}{%
\end{tt}
\end{frenchspacing}
\end{flushleft}
\setlength{\parindent}{\mystoreparindent}
\setlength{\hangindent}{\mystorehangindent}
\vskip 1ex
}

\needspace{5\baselineskip}
\subsection{Propositional Calculus}\label{propcalc}\index{axioms of
propositional calculus}

\needspace{2\baselineskip}
Axiom of Simplification.\label{ax1}

\setbox\startprefix=\hbox{\tt \ \ ax-1\ \$a\ }
\setbox\contprefix=\hbox{\tt \ \ \ \ \ \ \ \ \ \ }
\startm
\m{\vdash}\m{(}\m{\varphi}\m{\rightarrow}\m{(}\m{\psi}\m{\rightarrow}\m{\varphi}\m{)}
\m{)}
\endm

\needspace{3\baselineskip}
\noindent Axiom of Distribution.

\setbox\startprefix=\hbox{\tt \ \ ax-2\ \$a\ }
\setbox\contprefix=\hbox{\tt \ \ \ \ \ \ \ \ \ \ }
\startm
\m{\vdash}\m{(}\m{(}\m{\varphi}\m{\rightarrow}\m{(}\m{\psi}\m{\rightarrow}\m{\chi}
\m{)}\m{)}\m{\rightarrow}\m{(}\m{(}\m{\varphi}\m{\rightarrow}\m{\psi}\m{)}\m{
\rightarrow}\m{(}\m{\varphi}\m{\rightarrow}\m{\chi}\m{)}\m{)}\m{)}
\endm

\needspace{2\baselineskip}
\noindent Axiom of Contraposition.

\setbox\startprefix=\hbox{\tt \ \ ax-3\ \$a\ }
\setbox\contprefix=\hbox{\tt \ \ \ \ \ \ \ \ \ \ }
\startm
\m{\vdash}\m{(}\m{(}\m{\lnot}\m{\varphi}\m{\rightarrow}\m{\lnot}\m{\psi}\m{)}\m{
\rightarrow}\m{(}\m{\psi}\m{\rightarrow}\m{\varphi}\m{)}\m{)}
\endm


\needspace{4\baselineskip}
\noindent Rule of Modus Ponens.\label{axmp}\index{modus ponens}

\setbox\startprefix=\hbox{\tt \ \ min\ \$e\ }
\setbox\contprefix=\hbox{\tt \ \ \ \ \ \ \ \ \ }
\startm
\m{\vdash}\m{\varphi}
\endm

\setbox\startprefix=\hbox{\tt \ \ maj\ \$e\ }
\setbox\contprefix=\hbox{\tt \ \ \ \ \ \ \ \ \ }
\startm
\m{\vdash}\m{(}\m{\varphi}\m{\rightarrow}\m{\psi}\m{)}
\endm

\setbox\startprefix=\hbox{\tt \ \ ax-mp\ \$a\ }
\setbox\contprefix=\hbox{\tt \ \ \ \ \ \ \ \ \ \ \ }
\startm
\m{\vdash}\m{\psi}
\endm


\needspace{7\baselineskip}
\subsection{Axioms of Predicate Calculus with Equality---Tarski's S2}\index{axioms of predicate calculus}

\needspace{3\baselineskip}
\noindent Rule of Generalization.\index{rule of generalization}

\setbox\startprefix=\hbox{\tt \ \ ax-g.1\ \$e\ }
\setbox\contprefix=\hbox{\tt \ \ \ \ \ \ \ \ \ \ \ \ }
\startm
\m{\vdash}\m{\varphi}
\endm

\setbox\startprefix=\hbox{\tt \ \ ax-gen\ \$a\ }
\setbox\contprefix=\hbox{\tt \ \ \ \ \ \ \ \ \ \ \ \ }
\startm
\m{\vdash}\m{\forall}\m{x}\m{\varphi}
\endm

\needspace{2\baselineskip}
\noindent Axiom of Quantified Implication.

\setbox\startprefix=\hbox{\tt \ \ ax-4\ \$a\ }
\setbox\contprefix=\hbox{\tt \ \ \ \ \ \ \ \ \ \ }
\startm
\m{\vdash}\m{(}\m{\forall}\m{x}\m{(}\m{\forall}\m{x}\m{\varphi}\m{\rightarrow}\m{
\psi}\m{)}\m{\rightarrow}\m{(}\m{\forall}\m{x}\m{\varphi}\m{\rightarrow}\m{
\forall}\m{x}\m{\psi}\m{)}\m{)}
\endm

\needspace{3\baselineskip}
\noindent Axiom of Distinctness.

% Aka: Add $d x ph $.
\setbox\startprefix=\hbox{\tt \ \ ax-5\ \$a\ }
\setbox\contprefix=\hbox{\tt \ \ \ \ \ \ \ \ \ \ }
\startm
\m{\vdash}\m{(}\m{\varphi}\m{\rightarrow}\m{\forall}\m{x}\m{\varphi}\m{)}\m{where}\m{ }\m{\$d}\m{ }\m{x}\m{ }\m{\varphi}\m{ }\m{(}\m{x}\m{ }\m{does}\m{ }\m{not}\m{ }\m{occur}\m{ }\m{in}\m{ }\m{\varphi}\m{)}
\endm

\needspace{2\baselineskip}
\noindent Axiom of Existence.

\setbox\startprefix=\hbox{\tt \ \ ax-6\ \$a\ }
\setbox\contprefix=\hbox{\tt \ \ \ \ \ \ \ \ \ \ }
\startm
\m{\vdash}\m{(}\m{\forall}\m{x}\m{(}\m{x}\m{=}\m{y}\m{\rightarrow}\m{\forall}
\m{x}\m{\varphi}\m{)}\m{\rightarrow}\m{\varphi}\m{)}
\endm

\needspace{2\baselineskip}
\noindent Axiom of Equality.

\setbox\startprefix=\hbox{\tt \ \ ax-7\ \$a\ }
\setbox\contprefix=\hbox{\tt \ \ \ \ \ \ \ \ \ \ }
\startm
\m{\vdash}\m{(}\m{x}\m{=}\m{y}\m{\rightarrow}\m{(}\m{x}\m{=}\m{z}\m{
\rightarrow}\m{y}\m{=}\m{z}\m{)}\m{)}
\endm

\needspace{2\baselineskip}
\noindent Axiom of Left Equality for Binary Predicate.

\setbox\startprefix=\hbox{\tt \ \ ax-8\ \$a\ }
\setbox\contprefix=\hbox{\tt \ \ \ \ \ \ \ \ \ \ \ }
\startm
\m{\vdash}\m{(}\m{x}\m{=}\m{y}\m{\rightarrow}\m{(}\m{x}\m{\in}\m{z}\m{
\rightarrow}\m{y}\m{\in}\m{z}\m{)}\m{)}
\endm

\needspace{2\baselineskip}
\noindent Axiom of Right Equality for Binary Predicate.

\setbox\startprefix=\hbox{\tt \ \ ax-9\ \$a\ }
\setbox\contprefix=\hbox{\tt \ \ \ \ \ \ \ \ \ \ \ }
\startm
\m{\vdash}\m{(}\m{x}\m{=}\m{y}\m{\rightarrow}\m{(}\m{z}\m{\in}\m{x}\m{
\rightarrow}\m{z}\m{\in}\m{y}\m{)}\m{)}
\endm


\needspace{4\baselineskip}
\subsection{Axioms of Predicate Calculus with Equality---Auxiliary}\index{axioms of predicate calculus - auxiliary}

\needspace{2\baselineskip}
\noindent Axiom of Quantified Negation.

\setbox\startprefix=\hbox{\tt \ \ ax-10\ \$a\ }
\setbox\contprefix=\hbox{\tt \ \ \ \ \ \ \ \ \ \ }
\startm
\m{\vdash}\m{(}\m{\lnot}\m{\forall}\m{x}\m{\lnot}\m{\forall}\m{x}\m{\varphi}\m{
\rightarrow}\m{\varphi}\m{)}
\endm

\needspace{2\baselineskip}
\noindent Axiom of Quantifier Commutation.

\setbox\startprefix=\hbox{\tt \ \ ax-11\ \$a\ }
\setbox\contprefix=\hbox{\tt \ \ \ \ \ \ \ \ \ \ }
\startm
\m{\vdash}\m{(}\m{\forall}\m{x}\m{\forall}\m{y}\m{\varphi}\m{\rightarrow}\m{
\forall}\m{y}\m{\forall}\m{x}\m{\varphi}\m{)}
\endm

\needspace{3\baselineskip}
\noindent Axiom of Substitution.

\setbox\startprefix=\hbox{\tt \ \ ax-12\ \$a\ }
\setbox\contprefix=\hbox{\tt \ \ \ \ \ \ \ \ \ \ \ }
\startm
\m{\vdash}\m{(}\m{\lnot}\m{\forall}\m{x}\m{\,x}\m{=}\m{y}\m{\rightarrow}\m{(}
\m{x}\m{=}\m{y}\m{\rightarrow}\m{(}\m{\varphi}\m{\rightarrow}\m{\forall}\m{x}\m{(}
\m{x}\m{=}\m{y}\m{\rightarrow}\m{\varphi}\m{)}\m{)}\m{)}\m{)}
\endm

\needspace{3\baselineskip}
\noindent Axiom of Quantified Equality.

\setbox\startprefix=\hbox{\tt \ \ ax-13\ \$a\ }
\setbox\contprefix=\hbox{\tt \ \ \ \ \ \ \ \ \ \ \ }
\startm
\m{\vdash}\m{(}\m{\lnot}\m{\forall}\m{z}\m{\,z}\m{=}\m{x}\m{\rightarrow}\m{(}
\m{\lnot}\m{\forall}\m{z}\m{\,z}\m{=}\m{y}\m{\rightarrow}\m{(}\m{x}\m{=}\m{y}
\m{\rightarrow}\m{\forall}\m{z}\m{\,x}\m{=}\m{y}\m{)}\m{)}\m{)}
\endm

% \noindent Axiom of Quantifier Substitution
%
% \setbox\startprefix=\hbox{\tt \ \ ax-c11n\ \$a\ }
% \setbox\contprefix=\hbox{\tt \ \ \ \ \ \ \ \ \ \ \ }
% \startm
% \m{\vdash}\m{(}\m{\forall}\m{x}\m{\,x}\m{=}\m{y}\m{\rightarrow}\m{(}\m{\forall}
% \m{x}\m{\varphi}\m{\rightarrow}\m{\forall}\m{y}\m{\varphi}\m{)}\m{)}
% \endm
%
% \noindent Axiom of Distinct Variables. (This axiom requires
% that two individual variables
% be distinct\index{\texttt{\$d} statement}\index{distinct
% variables}.)
%
% \setbox\startprefix=\hbox{\tt \ \ \ \ \ \ \ \ \$d\ }
% \setbox\contprefix=\hbox{\tt \ \ \ \ \ \ \ \ \ \ \ }
% \startm
% \m{x}\m{\,}\m{y}
% \endm
%
% \setbox\startprefix=\hbox{\tt \ \ ax-c16\ \$a\ }
% \setbox\contprefix=\hbox{\tt \ \ \ \ \ \ \ \ \ \ \ }
% \startm
% \m{\vdash}\m{(}\m{\forall}\m{x}\m{\,x}\m{=}\m{y}\m{\rightarrow}\m{(}\m{\varphi}\m{
% \rightarrow}\m{\forall}\m{x}\m{\varphi}\m{)}\m{)}
% \endm

% \noindent Axiom of Quantifier Introduction (2).  (This axiom requires
% that the individual variable not occur in the
% wff\index{\texttt{\$d} statement}\index{distinct variables}.)
%
% \setbox\startprefix=\hbox{\tt \ \ \ \ \ \ \ \ \$d\ }
% \setbox\contprefix=\hbox{\tt \ \ \ \ \ \ \ \ \ \ \ }
% \startm
% \m{x}\m{\,}\m{\varphi}
% \endm
% \setbox\startprefix=\hbox{\tt \ \ ax-5\ \$a\ }
% \setbox\contprefix=\hbox{\tt \ \ \ \ \ \ \ \ \ \ \ }
% \startm
% \m{\vdash}\m{(}\m{\varphi}\m{\rightarrow}\m{\forall}\m{x}\m{\varphi}\m{)}
% \endm

\subsection{Set Theory}\label{mmsettheoryaxioms}

In order to make the axioms of set theory\index{axioms of set theory} a little
more compact, there are several definitions from logic that we make use of
implicitly, namely, ``logical {\sc and},''\index{conjunction ($\wedge$)}
\index{logical {\sc and} ($\wedge$)} ``logical equivalence,''\index{logical
equivalence ($\leftrightarrow$)}\index{biconditional ($\leftrightarrow$)} and
``there exists.''\index{existential quantifier ($\exists$)}

\begin{center}\begin{tabular}{rcl}
  $( \varphi \wedge \psi )$ &\mbox{stands for}& $\neg ( \varphi
     \rightarrow \neg \psi )$\\
  $( \varphi \leftrightarrow \psi )$& \mbox{stands
     for}& $( ( \varphi \rightarrow \psi ) \wedge
     ( \psi \rightarrow \varphi ) )$\\
  $\exists x \,\varphi$ &\mbox{stands for}& $\neg \forall x \neg \varphi$
\end{tabular}\end{center}

In addition, the axioms of set theory require that all variables be
dis\-tinct,\index{distinct variables}\footnote{Set theory axioms can be
devised so that {\em no} variables are required to be distinct,
provided we replace \texttt{ax-c16} with an axiom stating that ``at
least two things exist,'' thus
making \texttt{ax-5} the only other axiom requiring the
\texttt{\$d} statement.  These axioms are unconventional and are not
presented here, but they can be found on the \url{http://metamath.org}
web site.  See also the Comment on
p.~\pageref{nodd}.}\index{\texttt{\$d} statement} thus we also assume:
\begin{center}
  \texttt{\$d }$x\,y\,z\,w$
\end{center}

\needspace{2\baselineskip}
\noindent Axiom of Extensionality.\index{Axiom of Extensionality}

\setbox\startprefix=\hbox{\tt \ \ ax-ext\ \$a\ }
\setbox\contprefix=\hbox{\tt \ \ \ \ \ \ \ \ \ \ \ \ }
\startm
\m{\vdash}\m{(}\m{\forall}\m{x}\m{(}\m{x}\m{\in}\m{y}\m{\leftrightarrow}\m{x}
\m{\in}\m{z}\m{)}\m{\rightarrow}\m{y}\m{=}\m{z}\m{)}
\endm

\needspace{3\baselineskip}
\noindent Axiom of Replacement.\index{Axiom of Replacement}

\setbox\startprefix=\hbox{\tt \ \ ax-rep\ \$a\ }
\setbox\contprefix=\hbox{\tt \ \ \ \ \ \ \ \ \ \ \ \ }
\startm
\m{\vdash}\m{(}\m{\forall}\m{w}\m{\exists}\m{y}\m{\forall}\m{z}\m{(}\m{%
\forall}\m{y}\m{\varphi}\m{\rightarrow}\m{z}\m{=}\m{y}\m{)}\m{\rightarrow}\m{%
\exists}\m{y}\m{\forall}\m{z}\m{(}\m{z}\m{\in}\m{y}\m{\leftrightarrow}\m{%
\exists}\m{w}\m{(}\m{w}\m{\in}\m{x}\m{\wedge}\m{\forall}\m{y}\m{\varphi}\m{)}%
\m{)}\m{)}
\endm

\needspace{2\baselineskip}
\noindent Axiom of Union.\index{Axiom of Union}

\setbox\startprefix=\hbox{\tt \ \ ax-un\ \$a\ }
\setbox\contprefix=\hbox{\tt \ \ \ \ \ \ \ \ \ \ \ }
\startm
\m{\vdash}\m{\exists}\m{x}\m{\forall}\m{y}\m{(}\m{\exists}\m{x}\m{(}\m{y}\m{
\in}\m{x}\m{\wedge}\m{x}\m{\in}\m{z}\m{)}\m{\rightarrow}\m{y}\m{\in}\m{x}\m{)}
\endm

\needspace{2\baselineskip}
\noindent Axiom of Power Sets.\index{Axiom of Power Sets}

\setbox\startprefix=\hbox{\tt \ \ ax-pow\ \$a\ }
\setbox\contprefix=\hbox{\tt \ \ \ \ \ \ \ \ \ \ \ \ }
\startm
\m{\vdash}\m{\exists}\m{x}\m{\forall}\m{y}\m{(}\m{\forall}\m{x}\m{(}\m{x}\m{
\in}\m{y}\m{\rightarrow}\m{x}\m{\in}\m{z}\m{)}\m{\rightarrow}\m{y}\m{\in}\m{x}
\m{)}
\endm

\needspace{3\baselineskip}
\noindent Axiom of Regularity.\index{Axiom of Regularity}

\setbox\startprefix=\hbox{\tt \ \ ax-reg\ \$a\ }
\setbox\contprefix=\hbox{\tt \ \ \ \ \ \ \ \ \ \ \ \ }
\startm
\m{\vdash}\m{(}\m{\exists}\m{x}\m{\,x}\m{\in}\m{y}\m{\rightarrow}\m{\exists}
\m{x}\m{(}\m{x}\m{\in}\m{y}\m{\wedge}\m{\forall}\m{z}\m{(}\m{z}\m{\in}\m{x}\m{
\rightarrow}\m{\lnot}\m{z}\m{\in}\m{y}\m{)}\m{)}\m{)}
\endm

\needspace{3\baselineskip}
\noindent Axiom of Infinity.\index{Axiom of Infinity}

\setbox\startprefix=\hbox{\tt \ \ ax-inf\ \$a\ }
\setbox\contprefix=\hbox{\tt \ \ \ \ \ \ \ \ \ \ \ \ \ \ \ }
\startm
\m{\vdash}\m{\exists}\m{x}\m{(}\m{y}\m{\in}\m{x}\m{\wedge}\m{\forall}\m{y}%
\m{(}\m{y}\m{\in}\m{x}\m{\rightarrow}\m{\exists}\m{z}\m{(}\m{y}\m{\in}\m{z}\m{%
\wedge}\m{z}\m{\in}\m{x}\m{)}\m{)}\m{)}
\endm

\needspace{4\baselineskip}
\noindent Axiom of Choice.\index{Axiom of Choice}

\setbox\startprefix=\hbox{\tt \ \ ax-ac\ \$a\ }
\setbox\contprefix=\hbox{\tt \ \ \ \ \ \ \ \ \ \ \ \ \ \ }
\startm
\m{\vdash}\m{\exists}\m{x}\m{\forall}\m{y}\m{\forall}\m{z}\m{(}\m{(}\m{y}\m{%
\in}\m{z}\m{\wedge}\m{z}\m{\in}\m{w}\m{)}\m{\rightarrow}\m{\exists}\m{w}\m{%
\forall}\m{y}\m{(}\m{\exists}\m{w}\m{(}\m{(}\m{y}\m{\in}\m{z}\m{\wedge}\m{z}%
\m{\in}\m{w}\m{)}\m{\wedge}\m{(}\m{y}\m{\in}\m{w}\m{\wedge}\m{w}\m{\in}\m{x}%
\m{)}\m{)}\m{\leftrightarrow}\m{y}\m{=}\m{w}\m{)}\m{)}
\endm

\subsection{That's It}

There you have it, the axioms for (essentially) all of mathematics!
Wonder at them and stare at them in awe.  Put a copy in your wallet, and
you will carry in your pocket the encoding for all theorems ever proved
and that ever will be proved, from the most mundane to the most
profound.

\section{A Hierarchy of Definitions}\label{hierarchy}

The axioms in the previous section in principle embody everything that can be
done within standard mathematics.  However, it is impractical to accomplish
very much by using them directly, for even simple concepts (from a human
perspective) can involve extremely long, incomprehensible formulas.
Mathematics is made practical by introducing definitions\index{definition}.
Definitions usually introduce new symbols, or at least new relationships among
existing symbols, to abbreviate more complex formulas.  An important
requirement for a definition is that there exist a straightforward
(algorithmic) method for eliminating the abbreviation by expanding it into the
more primitive symbol string that it represents.  Some
important definitions included in
the file \texttt{set.mm} are listed in this section for reference, and also to
give you a feel for why something like $\omega$\index{omega ($\omega$)} (the
set of natural numbers\index{natural number} 0, 1, 2,\ldots) becomes very
complicated when completely expanded into primitive symbols.

What is the motivation for definitions, aside from allowing complicated
expressions to be expressed more simply?  In the case of  $\omega$, one goal is
to provide a basis for the theory of natural numbers.\index{natural number}
Before set theory was invented, a set of axioms for arithmetic, called Peano's
postulates\index{Peano's postulates}, was devised and shown to have the
properties one expects for natural numbers.  Now anyone can postulate a
set of axioms, but if the axioms are inconsistent contradictions can be derived
from them.  Once a contradiction is derived, anything can be trivially
proved, including
all the facts of arithmetic and their negations.  To ensure that an
axiom system is at least as reliable as the axioms for set theory, we can
define sets and operations on those sets that satisfy the new axioms. In the
\texttt{set.mm} Metamath database, we prove that the elements of $\omega$ satisfy
Peano's postulates, and it's a long and hard journey to get there directly
from the axioms of set theory.  But the result is confidence in the
foundations of arithmetic.  And there is another advantage:  we now have all
the tools of set theory at our disposal for manipulating objects that obey the
axioms for arithmetic.

What are the criteria we use for definitions?  First, and of utmost importance,
the definition should not be {\em creative}\index{creative
definition}\index{definition!creative}, that
is it should not allow an expression that previously qualified as a wff but
was not provable, to become provable.   Second, the definition should be {\em
eliminable}\index{definition!eliminability}, that is, there should exist an
algorithmic method for converting any expression using the definition into
a logically equivalent expression that previously qualified as a wff.

In almost all cases below, definitions connect two expressions with either
$\leftrightarrow$ or $=$.  Eliminating\footnote{Here we mean the
elimination that a human might do in his or her head.  To eliminate them as
part of a Metamath proof we would invoke one of a number of
theorems that deal with transitivity of equivalence or equality; there are
many such examples in the proofs in \texttt{set.mm}.} such a definition is a
simple matter of substituting the expression on the left-hand side ({\em
definiendum}\index{definiendum} or thing being defined) with the equivalent,
more primitive expression on the right-hand side ({\em
definiens}\index{definiens} or definition).

Often a definition has variables on the right-hand side which do not appear on
the left-hand side; these are called {\em dummy variables}.\index{dummy
variable!in definitions}  In this case, any
allowable substitution (such as a new, distinct
variable) can be used when the definition is eliminated.  Dummy variables may
be used only if they are {\em effectively bound}\index{effectively bound
variable}, meaning that the definition will remain logically equivalent upon
any substitution of a dummy variable with any other {\em qualifying
expression}\index{qualifying expression}, i.e.\ any symbol string (such as
another variable) that
meets the restrictions on the dummy variable imposed by \texttt{\$d} and
\texttt{\$f} statements.  For example, we could define a constant $\perp$
(inverted tee, meaning logical ``false'') as $( \varphi \wedge \lnot \varphi
)$, i.e.\ ``phi and not phi.''  Here $\varphi$ is effectively bound because the
definition remains logically equivalent when we replace $\varphi$ with any
other wff.  (It is actually \texttt{df-fal}
in \texttt{set.mm}, which defines $\perp$.)

There are two cases where eliminating definitions is a little more
complex.  These cases are the definitions \texttt{df-bi} and
\texttt{df-cleq}.  The first stretches the concept of a definition a
little, as in effect it ``defines a definition;'' however, it meets our
requirements for a definition in that it is eliminable and does not
strengthen the language.  Theorem \texttt{bii} shows the substitution
needed to eliminate the $\leftrightarrow$\index{logical equivalence
($\leftrightarrow$)}\index{biconditional ($\leftrightarrow$)} symbol.

Definition \texttt{df-cleq}\index{equality ($=$)} extends the usage of
the equality symbol to include ``classes''\index{class} in set theory.  The
reason it is potentially problematic is that it can lead to statements which
do not follow from logic alone but presuppose the Axiom of
Extensionality\index{Axiom of Extensionality}, so we include this axiom
as a hypothesis for the definition.  We could have made \texttt{df-cleq} directly
eliminable by introducing a new equality symbol, but have chosen not to do so
in keeping with standard textbook practice.  Definitions such as \texttt{df-cleq}
that extend the meaning of existing symbols must be introduced carefully so
that they do not lead to contradictions.  Definition \texttt{df-clel} also
extends the meaning of an existing symbol ($\in$); while it doesn't strengthen
the language like \texttt{df-cleq}, this is not obvious and it must also be
subject to the same scrutiny.

Exercise:  Study how the wff $x\in\omega$, meaning ``$x$ is a natural
number,'' could be expanded in terms of primitive symbols, starting with the
definitions \texttt{df-clel} on p.~\pageref{dfclel} and \texttt{df-om} on
p.~\pageref{dfom} and working your way back.  Don't bother to work out the
details; just make sure that you understand how you could do it in principle.
The answer is shown in the footnote on p.~\pageref{expandom}.  If you
actually do work it out, you won't get exactly the same answer because we used
a few simplifications such as discarding occurrences of $\lnot\lnot$ (double
negation).

In the definitions below, we have placed the {\sc ascii} Metamath source
below each of the formulas to help you become familiar with the
notation in the database.  For simplicity, the necessary \texttt{\$f}
and \texttt{\$d} statements are not shown.  If you are in doubt, use the
\texttt{show statement}\index{\texttt{show statement} command} command
in the Metamath program to see the full statement.
A selection of this notation is summarized in Appendix~\ref{ASCII}.

To understand the motivation for these definitions, you should consult the
references indicated:  Takeuti and Zaring \cite{Takeuti}\index{Takeuti, Gaisi},
Quine \cite{Quine}\index{Quine, Willard Van Orman}, Bell and Machover
\cite{Bell}\index{Bell, J. L.}, and Enderton \cite{Enderton}\index{Enderton,
Herbert B.}.  Our list of definitions is provided more for reference than as a
learning aid.  However, by looking at a few of them you can gain a feel for
how the hierarchy is built up.  The definitions are a representative sample of
the many definitions
in \texttt{set.mm}, but they are complete with respect to the
theorem examples we will present in Section~\ref{sometheorems}.  Also, some are
slightly different from, but logically equivalent to, the ones in \texttt{set.mm}
(some of which have been revised over time to shorten them, for example).

\subsection{Definitions for Propositional Calculus}\label{metadefprop}

The symbols $\varphi$, $\psi$, and $\chi$ represent wffs.

Our first definition introduces the biconditional
connective\footnote{The term ``connective'' is informally used to mean a
symbol that is placed between two variables or adjacent to a variable,
whereas a mathematical ``constant'' usually indicates a symbol such as
the number 0 that may replace a variable or metavariable.  From
Metamath's point of view, there is no distinction between a connective
and a constant; both are constants in the Metamath
language.}\index{connective}\index{constant} (also called logical
equivalence)\index{logical equivalence
($\leftrightarrow$)}\index{biconditional ($\leftrightarrow$)}.  Unlike
most traditional developments, we have chosen not to have a separate
symbol such as ``Df.'' to mean ``is defined as.''  Instead, we will use
the biconditional connective for this purpose, as it lets us use
logic to manipulate definitions directly.  Here we state the properties
of the biconditional connective with a carefully crafted \texttt{\$a}
statement, which effectively uses the biconditional connective to define
itself.  The $\leftrightarrow$ symbol can be eliminated from a formula
using theorem \texttt{bii}, which is derived later.

\vskip 2ex
\noindent Define the biconditional connective.\label{df-bi}

\vskip 0.5ex
\setbox\startprefix=\hbox{\tt \ \ df-bi\ \$a\ }
\setbox\contprefix=\hbox{\tt \ \ \ \ \ \ \ \ \ \ \ }
\startm
\m{\vdash}\m{\lnot}\m{(}\m{(}\m{(}\m{\varphi}\m{\leftrightarrow}\m{\psi}\m{)}%
\m{\rightarrow}\m{\lnot}\m{(}\m{(}\m{\varphi}\m{\rightarrow}\m{\psi}\m{)}\m{%
\rightarrow}\m{\lnot}\m{(}\m{\psi}\m{\rightarrow}\m{\varphi}\m{)}\m{)}\m{)}\m{%
\rightarrow}\m{\lnot}\m{(}\m{\lnot}\m{(}\m{(}\m{\varphi}\m{\rightarrow}\m{%
\psi}\m{)}\m{\rightarrow}\m{\lnot}\m{(}\m{\psi}\m{\rightarrow}\m{\varphi}\m{)}%
\m{)}\m{\rightarrow}\m{(}\m{\varphi}\m{\leftrightarrow}\m{\psi}\m{)}\m{)}\m{)}
\endm
\begin{mmraw}%
|- -. ( ( ( ph <-> ps ) -> -. ( ( ph -> ps ) ->
-. ( ps -> ph ) ) ) -> -. ( -. ( ( ph -> ps ) -> -. (
ps -> ph ) ) -> ( ph <-> ps ) ) ) \$.
\end{mmraw}

\noindent This theorem relates the biconditional connective to primitive
connectives and can be used to eliminate the $\leftrightarrow$ symbol from any
wff.

\vskip 0.5ex
\setbox\startprefix=\hbox{\tt \ \ bii\ \$p\ }
\setbox\contprefix=\hbox{\tt \ \ \ \ \ \ \ \ \ }
\startm
\m{\vdash}\m{(}\m{(}\m{\varphi}\m{\leftrightarrow}\m{\psi}\m{)}\m{\leftrightarrow}
\m{\lnot}\m{(}\m{(}\m{\varphi}\m{\rightarrow}\m{\psi}\m{)}\m{\rightarrow}\m{\lnot}
\m{(}\m{\psi}\m{\rightarrow}\m{\varphi}\m{)}\m{)}\m{)}
\endm
\begin{mmraw}%
|- ( ( ph <-> ps ) <-> -. ( ( ph -> ps ) -> -. ( ps -> ph ) ) ) \$= ... \$.
\end{mmraw}

\noindent Define disjunction ({\sc or}).\index{disjunction ($\vee$)}%
\index{logical or (vee)@logical {\sc or} ($\vee$)}%
\index{df-or@\texttt{df-or}}\label{df-or}

\vskip 0.5ex
\setbox\startprefix=\hbox{\tt \ \ df-or\ \$a\ }
\setbox\contprefix=\hbox{\tt \ \ \ \ \ \ \ \ \ \ \ }
\startm
\m{\vdash}\m{(}\m{(}\m{\varphi}\m{\vee}\m{\psi}\m{)}\m{\leftrightarrow}\m{(}\m{
\lnot}\m{\varphi}\m{\rightarrow}\m{\psi}\m{)}\m{)}
\endm
\begin{mmraw}%
|- ( ( ph \TOR ps ) <-> ( -. ph -> ps ) ) \$.
\end{mmraw}

\noindent Define conjunction ({\sc and}).\index{conjunction ($\wedge$)}%
\index{logical {\sc and} ($\wedge$)}%
\index{df-an@\texttt{df-an}}\label{df-an}

\vskip 0.5ex
\setbox\startprefix=\hbox{\tt \ \ df-an\ \$a\ }
\setbox\contprefix=\hbox{\tt \ \ \ \ \ \ \ \ \ \ \ }
\startm
\m{\vdash}\m{(}\m{(}\m{\varphi}\m{\wedge}\m{\psi}\m{)}\m{\leftrightarrow}\m{\lnot}
\m{(}\m{\varphi}\m{\rightarrow}\m{\lnot}\m{\psi}\m{)}\m{)}
\endm
\begin{mmraw}%
|- ( ( ph \TAND ps ) <-> -. ( ph -> -. ps ) ) \$.
\end{mmraw}

\noindent Define disjunction ({\sc or}) of 3 wffs.%
\index{df-3or@\texttt{df-3or}}\label{df-3or}

\vskip 0.5ex
\setbox\startprefix=\hbox{\tt \ \ df-3or\ \$a\ }
\setbox\contprefix=\hbox{\tt \ \ \ \ \ \ \ \ \ \ \ \ }
\startm
\m{\vdash}\m{(}\m{(}\m{\varphi}\m{\vee}\m{\psi}\m{\vee}\m{\chi}\m{)}\m{
\leftrightarrow}\m{(}\m{(}\m{\varphi}\m{\vee}\m{\psi}\m{)}\m{\vee}\m{\chi}\m{)}
\m{)}
\endm
\begin{mmraw}%
|- ( ( ph \TOR ps \TOR ch ) <-> ( ( ph \TOR ps ) \TOR ch ) ) \$.
\end{mmraw}

\noindent Define conjunction ({\sc and}) of 3 wffs.%
\index{df-3an}\label{df-3an}

\vskip 0.5ex
\setbox\startprefix=\hbox{\tt \ \ df-3an\ \$a\ }
\setbox\contprefix=\hbox{\tt \ \ \ \ \ \ \ \ \ \ \ \ }
\startm
\m{\vdash}\m{(}\m{(}\m{\varphi}\m{\wedge}\m{\psi}\m{\wedge}\m{\chi}\m{)}\m{
\leftrightarrow}\m{(}\m{(}\m{\varphi}\m{\wedge}\m{\psi}\m{)}\m{\wedge}\m{\chi}
\m{)}\m{)}
\endm

\begin{mmraw}%
|- ( ( ph \TAND ps \TAND ch ) <-> ( ( ph \TAND ps ) \TAND ch ) ) \$.
\end{mmraw}

\subsection{Definitions for Predicate Calculus}\label{metadefpred}

The symbols $x$, $y$, and $z$ represent individual variables of predicate
calculus.  In this section, they are not necessarily distinct unless it is
explicitly
mentioned.

\vskip 2ex
\noindent Define existential quantification.
The expression $\exists x \varphi$ means
``there exists an $x$ where $\varphi$ is true.''\index{existential quantifier
($\exists$)}\label{df-ex}

\vskip 0.5ex
\setbox\startprefix=\hbox{\tt \ \ df-ex\ \$a\ }
\setbox\contprefix=\hbox{\tt \ \ \ \ \ \ \ \ \ \ \ }
\startm
\m{\vdash}\m{(}\m{\exists}\m{x}\m{\varphi}\m{\leftrightarrow}\m{\lnot}\m{\forall}
\m{x}\m{\lnot}\m{\varphi}\m{)}
\endm
\begin{mmraw}%
|- ( E. x ph <-> -. A. x -. ph ) \$.
\end{mmraw}

\noindent Define proper substitution.\index{proper
substitution}\index{substitution!proper}\label{df-sb}
In our notation, we use $[ y / x ] \varphi$ to mean ``the wff that
results when $y$ is properly substituted for $x$ in the wff
$\varphi$.''\footnote{
This can also be described
as substituting $x$ with $y$, $y$ properly replaces $x$, or
$x$ is properly replaced by $y$.}
% This is elsb4, though it currently says: ( [ x / y ] z e. y <-> z e. x )
For example,
$[ y / x ] z \in x$ is the same as $z \in y$.
One way to remember this notation is to notice that it looks like division
and recall that $( y / x ) \cdot x $ is $y$ (when $x \neq 0$).
The notation is different from the notation $\varphi ( x | y )$
that is sometimes used, because the latter notation is ambiguous for us:
for example, we don't know whether $\lnot \varphi ( x | y )$ is to be
interpreted as $\lnot ( \varphi ( x | y ) )$ or
$( \lnot \varphi ) ( x | y )$.\footnote{Because of the way
we initially defined wffs, this is the case
with any postfix connective\index{postfix connective} (one occurring after the
symbols being connected) or infix connective\index{infix connective} (one
occurring between the symbols being connected).  Metamath does not have a
built-in notion of operator binding strength that could eliminate the
ambiguity.  The initial parenthesis effectively provides a prefix
connective\index{prefix connective} to eliminate ambiguity.  Some conventions,
such as Polish notation\index{Polish notation} used in the 1930's and 1940's
by Polish logicians, use only prefix connectives and thus allow the total
elimination of parentheses, at the expense of readability.  In Metamath we
could actually redefine all notation to be Polish if we wanted to without
having to change any proofs!}  Other texts often use $\varphi(y)$ to indicate
our $[ y / x ] \varphi$, but this notation is even more ambiguous since there is
no explicit indication of what is being substituted.
Note that this
definition is valid even when
$x$ and $y$ are the same variable.  The first conjunct is a ``trick'' used to
achieve this property, making the definition look somewhat peculiar at
first.

\vskip 0.5ex
\setbox\startprefix=\hbox{\tt \ \ df-sb\ \$a\ }
\setbox\contprefix=\hbox{\tt \ \ \ \ \ \ \ \ \ \ \ }
\startm
\m{\vdash}\m{(}\m{[}\m{y}\m{/}\m{x}\m{]}\m{\varphi}\m{\leftrightarrow}\m{(}%
\m{(}\m{x}\m{=}\m{y}\m{\rightarrow}\m{\varphi}\m{)}\m{\wedge}\m{\exists}\m{x}%
\m{(}\m{x}\m{=}\m{y}\m{\wedge}\m{\varphi}\m{)}\m{)}\m{)}
\endm
\begin{mmraw}%
|- ( [ y / x ] ph <-> ( ( x = y -> ph ) \TAND E. x ( x = y \TAND ph ) ) ) \$.
\end{mmraw}


\noindent Define existential uniqueness\index{existential uniqueness
quantifier ($\exists "!$)} (``there exists exactly one'').  Note that $y$ is a
variable distinct from $x$ and not occurring in $\varphi$.\label{df-eu}

\vskip 0.5ex
\setbox\startprefix=\hbox{\tt \ \ df-eu\ \$a\ }
\setbox\contprefix=\hbox{\tt \ \ \ \ \ \ \ \ \ \ \ }
\startm
\m{\vdash}\m{(}\m{\exists}\m{{!}}\m{x}\m{\varphi}\m{\leftrightarrow}\m{\exists}
\m{y}\m{\forall}\m{x}\m{(}\m{\varphi}\m{\leftrightarrow}\m{x}\m{=}\m{y}\m{)}\m{)}
\endm

\begin{mmraw}%
|- ( E! x ph <-> E. y A. x ( ph <-> x = y ) ) \$.
\end{mmraw}

\subsection{Definitions for Set Theory}\label{setdefinitions}

The symbols $x$, $y$, $z$, and $w$ represent individual variables of
predicate calculus, which in set theory are understood to be sets.
However, using only the constructs shown so far would be very inconvenient.

To make set theory more practical, we introduce the notion of a ``class.''
A class\index{class} is either a set variable (such as $x$) or an
expression of the form $\{ x | \varphi\}$ (called an ``abstraction
class''\index{abstraction class}\index{class abstraction}).  Note that
sets (i.e.\ individual variables) always exist (this is a theorem of
logic, namely $\exists y \, y = x$ for any set $x$), whereas classes may
or may not exist (i.e.\ $\exists y \, y = A$ may or may not be true).
If a class does not exist it is called a ``proper class.''\index{proper
class}\index{class!proper} Definitions \texttt{df-clab},
\texttt{df-cleq}, and \texttt{df-clel} can be used to convert an
expression containing classes into one containing only set variables and
wff metavariables.

The symbols $A$, $B$, $C$, $D$, $ F$, $G$, and $R$ are metavariables that range
over classes.  A class metavariable $A$ may be eliminated from a wff by
replacing it with $\{ x|\varphi\}$ where neither $x$ nor $\varphi$ occur in
the wff.

The theory of classes can be shown to be an eliminable and conservative
extension of set theory. The \textbf{eliminability}
property shows that for every
formula in the extended language we can build a logically equivalent
formula in the basic language; so that even if the extended language
provides more ease to convey and formulate mathematical ideas for set
theory, its expressive power does not in fact strengthen the basic
language's expressive power.
The \textbf{conservation} property shows that for
every proof of a formula of the basic language in the extended system
we can build another proof of the same formula in the basic system;
so that, concerning theorems on sets only, the deductive powers of
the extended system and of the basic system are identical. Together,
these properties mean that the extended language can be treated as a
definitional extension that is \textbf{sound}.

A rigorous justification, which we will not give here, can be found in
Levy \cite[pp.~357-366]{Levy} supplementing his informal introduction to class
theory on pp.~7-17. Two other good treatments of class theory are provided
by Quine \cite[pp.~15-21]{Quine}\index{Quine, Willard Van Orman}
and also \cite[pp.~10-14]{Takeuti}\index{Takeuti, Gaisi}.
Quine's exposition (he calls them virtual classes)
is nicely written and very readable.

In the rest of this
section, individual variables are always assumed to be distinct from
each other unless otherwise indicated.  In addition, dummy variables on the
right-hand side of a definition do not occur in the class and wff
metavariables in the definition.

The definitions we present here are a partial but self-contained
collection selected from several hundred that appear in the current
\texttt{set.mm} database.  They are adequate for a basic development of
elementary set theory.

\vskip 2ex
\noindent Define the abstraction class.\index{abstraction class}\index{class
abstraction}\label{df-clab}  $x$ and $y$
need not be distinct.  Definition 2.1 of Quine, p.~16.  This definition may
seem puzzling since it is shorter than the expression being defined and does not
buy us anything in terms of brevity.  The reason we introduce this definition
is because it fits in neatly with the extension of the $\in$ connective
provided by \texttt{df-clel}.

\vskip 0.5ex
\setbox\startprefix=\hbox{\tt \ \ df-clab\ \$a\ }
\setbox\contprefix=\hbox{\tt \ \ \ \ \ \ \ \ \ \ \ \ \ }
\startm
\m{\vdash}\m{(}\m{x}\m{\in}\m{\{}\m{y}\m{|}\m{\varphi}\m{\}}\m{%
\leftrightarrow}\m{[}\m{x}\m{/}\m{y}\m{]}\m{\varphi}\m{)}
\endm
\begin{mmraw}%
|- ( x e. \{ y | ph \} <-> [ x / y ] ph ) \$.
\end{mmraw}

\noindent Define the equality connective between classes\index{class
equality}\label{df-cleq}.  See Quine or Chapter 4 of Takeuti and Zaring for its
justification and methods for eliminating it.  This is an example of a
somewhat ``dangerous'' definition, because it extends the use of the
existing equality symbol rather than introducing a new symbol, allowing
us to make statements in the original language that may not be true.
For example, it permits us to deduce $y = z \leftrightarrow \forall x (
x \in y \leftrightarrow x \in z )$ which is not a theorem of logic but
rather presupposes the Axiom of Extensionality,\index{Axiom of
Extensionality} which we include as a hypothesis so that we can know
when this axiom is assumed in a proof (with the \texttt{show
trace{\char`\_}back} command).  We could avoid the danger by introducing
another symbol, say $\eqcirc$, in place of $=$; this
would also have the advantage of making elimination of the definition
straightforward and would eliminate the need for Extensionality as a
hypothesis.  We would then also have the advantage of being able to
identify exactly where Extensionality truly comes into play.  One of our
theorems would be $x \eqcirc y \leftrightarrow x = y$
by invoking Extensionality.  However in keeping with standard practice
we retain the ``dangerous'' definition.

\vskip 0.5ex
\setbox\startprefix=\hbox{\tt \ \ df-cleq.1\ \$e\ }
\setbox\contprefix=\hbox{\tt \ \ \ \ \ \ \ \ \ \ \ \ \ \ \ }
\startm
\m{\vdash}\m{(}\m{\forall}\m{x}\m{(}\m{x}\m{\in}\m{y}\m{\leftrightarrow}\m{x}
\m{\in}\m{z}\m{)}\m{\rightarrow}\m{y}\m{=}\m{z}\m{)}
\endm
\setbox\startprefix=\hbox{\tt \ \ df-cleq\ \$a\ }
\setbox\contprefix=\hbox{\tt \ \ \ \ \ \ \ \ \ \ \ \ \ }
\startm
\m{\vdash}\m{(}\m{A}\m{=}\m{B}\m{\leftrightarrow}\m{\forall}\m{x}\m{(}\m{x}\m{
\in}\m{A}\m{\leftrightarrow}\m{x}\m{\in}\m{B}\m{)}\m{)}
\endm
% We need to reset the startprefix and contprefix.
\setbox\startprefix=\hbox{\tt \ \ df-cleq.1\ \$e\ }
\setbox\contprefix=\hbox{\tt \ \ \ \ \ \ \ \ \ \ \ \ \ \ \ }
\begin{mmraw}%
|- ( A. x ( x e. y <-> x e. z ) -> y = z ) \$.
\end{mmraw}
\setbox\startprefix=\hbox{\tt \ \ df-cleq\ \$a\ }
\setbox\contprefix=\hbox{\tt \ \ \ \ \ \ \ \ \ \ \ \ \ }
\begin{mmraw}%
|- ( A = B <-> A. x ( x e. A <-> x e. B ) ) \$.
\end{mmraw}

\noindent Define the membership connective between classes\index{class
membership}.  Theorem 6.3 of Quine, p.~41, which we adopt as a definition.
Note that it extends the use of the existing membership symbol, but unlike
{\tt df-cleq} it does not extend the set of valid wffs of logic when the class
metavariables are replaced with set variables.\label{dfclel}\label{df-clel}

\vskip 0.5ex
\setbox\startprefix=\hbox{\tt \ \ df-clel\ \$a\ }
\setbox\contprefix=\hbox{\tt \ \ \ \ \ \ \ \ \ \ \ \ \ }
\startm
\m{\vdash}\m{(}\m{A}\m{\in}\m{B}\m{\leftrightarrow}\m{\exists}\m{x}\m{(}\m{x}
\m{=}\m{A}\m{\wedge}\m{x}\m{\in}\m{B}\m{)}\m{)}
\endm
\begin{mmraw}%
|- ( A e. B <-> E. x ( x = A \TAND x e. B ) ) \$.?
\end{mmraw}

\noindent Define inequality.

\vskip 0.5ex
\setbox\startprefix=\hbox{\tt \ \ df-ne\ \$a\ }
\setbox\contprefix=\hbox{\tt \ \ \ \ \ \ \ \ \ \ \ }
\startm
\m{\vdash}\m{(}\m{A}\m{\ne}\m{B}\m{\leftrightarrow}\m{\lnot}\m{A}\m{=}\m{B}%
\m{)}
\endm
\begin{mmraw}%
|- ( A =/= B <-> -. A = B ) \$.
\end{mmraw}

\noindent Define restricted universal quantification.\index{universal
quantifier ($\forall$)!restricted}  Enderton, p.~22.

\vskip 0.5ex
\setbox\startprefix=\hbox{\tt \ \ df-ral\ \$a\ }
\setbox\contprefix=\hbox{\tt \ \ \ \ \ \ \ \ \ \ \ \ }
\startm
\m{\vdash}\m{(}\m{\forall}\m{x}\m{\in}\m{A}\m{\varphi}\m{\leftrightarrow}\m{%
\forall}\m{x}\m{(}\m{x}\m{\in}\m{A}\m{\rightarrow}\m{\varphi}\m{)}\m{)}
\endm
\begin{mmraw}%
|- ( A. x e. A ph <-> A. x ( x e. A -> ph ) ) \$.
\end{mmraw}

\noindent Define restricted existential quantification.\index{existential
quantifier ($\exists$)!restricted}  Enderton, p.~22.

\vskip 0.5ex
\setbox\startprefix=\hbox{\tt \ \ df-rex\ \$a\ }
\setbox\contprefix=\hbox{\tt \ \ \ \ \ \ \ \ \ \ \ \ }
\startm
\m{\vdash}\m{(}\m{\exists}\m{x}\m{\in}\m{A}\m{\varphi}\m{\leftrightarrow}\m{%
\exists}\m{x}\m{(}\m{x}\m{\in}\m{A}\m{\wedge}\m{\varphi}\m{)}\m{)}
\endm
\begin{mmraw}%
|- ( E. x e. A ph <-> E. x ( x e. A \TAND ph ) ) \$.
\end{mmraw}

\noindent Define the universal class\index{universal class ($V$)}.  Definition
5.20, p.~21, of Takeuti and Zaring.\label{df-v}

\vskip 0.5ex
\setbox\startprefix=\hbox{\tt \ \ df-v\ \$a\ }
\setbox\contprefix=\hbox{\tt \ \ \ \ \ \ \ \ \ \ }
\startm
\m{\vdash}\m{{\rm V}}\m{=}\m{\{}\m{x}\m{|}\m{x}\m{=}\m{x}\m{\}}
\endm
\begin{mmraw}%
|- {\char`\_}V = \{ x | x = x \} \$.
\end{mmraw}

\noindent Define the subclass\index{subclass}\index{subset} relationship
between two classes (called the subset relation if the classes are sets i.e.\
are not proper).  Definition 5.9 of Takeuti and Zaring, p.~17.\label{df-ss}

\vskip 0.5ex
\setbox\startprefix=\hbox{\tt \ \ df-ss\ \$a\ }
\setbox\contprefix=\hbox{\tt \ \ \ \ \ \ \ \ \ \ \ }
\startm
\m{\vdash}\m{(}\m{A}\m{\subseteq}\m{B}\m{\leftrightarrow}\m{\forall}\m{x}\m{(}
\m{x}\m{\in}\m{A}\m{\rightarrow}\m{x}\m{\in}\m{B}\m{)}\m{)}
\endm
\begin{mmraw}%
|- ( A C\_ B <-> A. x ( x e. A -> x e. B ) ) \$.
\end{mmraw}

\noindent Define the union\index{union} of two classes.  Definition 5.6 of Takeuti and Zaring,
p.~16.\label{df-un}

\vskip 0.5ex
\setbox\startprefix=\hbox{\tt \ \ df-un\ \$a\ }
\setbox\contprefix=\hbox{\tt \ \ \ \ \ \ \ \ \ \ \ }
\startm
\m{\vdash}\m{(}\m{A}\m{\cup}\m{B}\m{)}\m{=}\m{\{}\m{x}\m{|}\m{(}\m{x}\m{\in}
\m{A}\m{\vee}\m{x}\m{\in}\m{B}\m{)}\m{\}}
\endm
\begin{mmraw}%
( A u. B ) = \{ x | ( x e. A \TOR x e. B ) \} \$.
\end{mmraw}

\noindent Define the intersection\index{intersection} of two classes.  Definition 5.6 of
Takeuti and Zaring, p.~16.\label{df-in}

\vskip 0.5ex
\setbox\startprefix=\hbox{\tt \ \ df-in\ \$a\ }
\setbox\contprefix=\hbox{\tt \ \ \ \ \ \ \ \ \ \ \ }
\startm
\m{\vdash}\m{(}\m{A}\m{\cap}\m{B}\m{)}\m{=}\m{\{}\m{x}\m{|}\m{(}\m{x}\m{\in}
\m{A}\m{\wedge}\m{x}\m{\in}\m{B}\m{)}\m{\}}
\endm
% Caret ^ requires special treatment
\begin{mmraw}%
|- ( A i\^{}i B ) = \{ x | ( x e. A \TAND x e. B ) \} \$.
\end{mmraw}

\noindent Define class difference\index{class difference}\index{set difference}.
Definition 5.12 of Takeuti and Zaring, p.~20.  Several notations are used in
the literature; we chose the $\setminus$ convention instead of a minus sign to
reserve the latter for later use in, e.g., arithmetic.\label{df-dif}

\vskip 0.5ex
\setbox\startprefix=\hbox{\tt \ \ df-dif\ \$a\ }
\setbox\contprefix=\hbox{\tt \ \ \ \ \ \ \ \ \ \ \ \ }
\startm
\m{\vdash}\m{(}\m{A}\m{\setminus}\m{B}\m{)}\m{=}\m{\{}\m{x}\m{|}\m{(}\m{x}\m{
\in}\m{A}\m{\wedge}\m{\lnot}\m{x}\m{\in}\m{B}\m{)}\m{\}}
\endm
\begin{mmraw}%
( A \SLASH B ) = \{ x | ( x e. A \TAND -. x e. B ) \} \$.
\end{mmraw}

\noindent Define the empty or null set\index{empty set}\index{null set}.
Compare  Definition 5.14 of Takeuti and Zaring, p.~20.\label{df-nul}

\vskip 0.5ex
\setbox\startprefix=\hbox{\tt \ \ df-nul\ \$a\ }
\setbox\contprefix=\hbox{\tt \ \ \ \ \ \ \ \ \ \ }
\startm
\m{\vdash}\m{\varnothing}\m{=}\m{(}\m{{\rm V}}\m{\setminus}\m{{\rm V}}\m{)}
\endm
\begin{mmraw}%
|- (/) = ( {\char`\_}V \SLASH {\char`\_}V ) \$.
\end{mmraw}

\noindent Define power class\index{power set}\index{power class}.  Definition 5.10 of
Takeuti and Zaring, p.~17, but we also let it apply to proper classes.  (Note
that \verb$~P$ is the symbol for calligraphic P, the tilde
suggesting ``curly;'' see Appendix~\ref{ASCII}.)\label{df-pw}

\vskip 0.5ex
\setbox\startprefix=\hbox{\tt \ \ df-pw\ \$a\ }
\setbox\contprefix=\hbox{\tt \ \ \ \ \ \ \ \ \ \ \ }
\startm
\m{\vdash}\m{{\cal P}}\m{A}\m{=}\m{\{}\m{x}\m{|}\m{x}\m{\subseteq}\m{A}\m{\}}
\endm
% Special incantation required to put ~ into the text
\begin{mmraw}%
|- \char`\~P~A = \{ x | x C\_ A \} \$.
\end{mmraw}

\noindent Define the singleton of a class\index{singleton}.  Definition 7.1 of
Quine, p.~48.  It is well-defined for proper classes, although
it is not very meaningful in this case, where it evaluates to the empty
set.

\vskip 0.5ex
\setbox\startprefix=\hbox{\tt \ \ df-sn\ \$a\ }
\setbox\contprefix=\hbox{\tt \ \ \ \ \ \ \ \ \ \ \ }
\startm
\m{\vdash}\m{\{}\m{A}\m{\}}\m{=}\m{\{}\m{x}\m{|}\m{x}\m{=}\m{A}\m{\}}
\endm
\begin{mmraw}%
|- \{ A \} = \{ x | x = A \} \$.
\end{mmraw}%

\noindent Define an unordered pair of classes\index{unordered pair}\index{pair}.  Definition
7.1 of Quine, p.~48.

\vskip 0.5ex
\setbox\startprefix=\hbox{\tt \ \ df-pr\ \$a\ }
\setbox\contprefix=\hbox{\tt \ \ \ \ \ \ \ \ \ \ \ }
\startm
\m{\vdash}\m{\{}\m{A}\m{,}\m{B}\m{\}}\m{=}\m{(}\m{\{}\m{A}\m{\}}\m{\cup}\m{\{}
\m{B}\m{\}}\m{)}
\endm
\begin{mmraw}%
|- \{ A , B \} = ( \{ A \} u. \{ B \} ) \$.
\end{mmraw}

\noindent Define an unordered triple of classes\index{unordered triple}.  Definition of
Enderton, p.~19.

\vskip 0.5ex
\setbox\startprefix=\hbox{\tt \ \ df-tp\ \$a\ }
\setbox\contprefix=\hbox{\tt \ \ \ \ \ \ \ \ \ \ \ }
\startm
\m{\vdash}\m{\{}\m{A}\m{,}\m{B}\m{,}\m{C}\m{\}}\m{=}\m{(}\m{\{}\m{A}\m{,}\m{B}
\m{\}}\m{\cup}\m{\{}\m{C}\m{\}}\m{)}
\endm
\begin{mmraw}%
|- \{ A , B , C \} = ( \{ A , B \} u. \{ C \} ) \$.
\end{mmraw}%

\noindent Kuratowski's\index{Kuratowski, Kazimierz} ordered pair\index{ordered
pair} definition.  Definition 9.1 of Quine, p.~58. For proper classes it is
not meaningful but is well-defined for convenience.  (Note that \verb$<.$
stands for $\langle$ whereas \verb$<$ stands for $<$, and similarly for
\verb$>.$\,.)\label{df-op}

\vskip 0.5ex
\setbox\startprefix=\hbox{\tt \ \ df-op\ \$a\ }
\setbox\contprefix=\hbox{\tt \ \ \ \ \ \ \ \ \ \ \ }
\startm
\m{\vdash}\m{\langle}\m{A}\m{,}\m{B}\m{\rangle}\m{=}\m{\{}\m{\{}\m{A}\m{\}}
\m{,}\m{\{}\m{A}\m{,}\m{B}\m{\}}\m{\}}
\endm
\begin{mmraw}%
|- <. A , B >. = \{ \{ A \} , \{ A , B \} \} \$.
\end{mmraw}

\noindent Define the union of a class\index{union}.  Definition 5.5, p.~16,
of Takeuti and Zaring.

\vskip 0.5ex
\setbox\startprefix=\hbox{\tt \ \ df-uni\ \$a\ }
\setbox\contprefix=\hbox{\tt \ \ \ \ \ \ \ \ \ \ \ \ }
\startm
\m{\vdash}\m{\bigcup}\m{A}\m{=}\m{\{}\m{x}\m{|}\m{\exists}\m{y}\m{(}\m{x}\m{
\in}\m{y}\m{\wedge}\m{y}\m{\in}\m{A}\m{)}\m{\}}
\endm
\begin{mmraw}%
|- U. A = \{ x | E. y ( x e. y \TAND y e. A ) \} \$.
\end{mmraw}

\noindent Define the intersection\index{intersection} of a class.  Definition 7.35,
p.~44, of Takeuti and Zaring.

\vskip 0.5ex
\setbox\startprefix=\hbox{\tt \ \ df-int\ \$a\ }
\setbox\contprefix=\hbox{\tt \ \ \ \ \ \ \ \ \ \ \ \ }
\startm
\m{\vdash}\m{\bigcap}\m{A}\m{=}\m{\{}\m{x}\m{|}\m{\forall}\m{y}\m{(}\m{y}\m{
\in}\m{A}\m{\rightarrow}\m{x}\m{\in}\m{y}\m{)}\m{\}}
\endm
\begin{mmraw}%
|- |\^{}| A = \{ x | A. y ( y e. A -> x e. y ) \} \$.
\end{mmraw}

\noindent Define a transitive class\index{transitive class}\index{transitive
set}.  This should not be confused with a transitive relation, which is a different
concept.  Definition from p.~71 of Enderton, extended to classes.

\vskip 0.5ex
\setbox\startprefix=\hbox{\tt \ \ df-tr\ \$a\ }
\setbox\contprefix=\hbox{\tt \ \ \ \ \ \ \ \ \ \ \ }
\startm
\m{\vdash}\m{(}\m{\mbox{\rm Tr}}\m{A}\m{\leftrightarrow}\m{\bigcup}\m{A}\m{
\subseteq}\m{A}\m{)}
\endm
\begin{mmraw}%
|- ( Tr A <-> U. A C\_ A ) \$.
\end{mmraw}

\noindent Define a notation for a general binary relation\index{binary
relation}.  Definition 6.18, p.~29, of Takeuti and Zaring, generalized to
arbitrary classes.  This definition is well-defined, although not very
meaningful, when classes $A$ and/or $B$ are proper.\label{dfbr}  The lack of
parentheses (or any other connective) creates no ambiguity since we are defining
an atomic wff.

\vskip 0.5ex
\setbox\startprefix=\hbox{\tt \ \ df-br\ \$a\ }
\setbox\contprefix=\hbox{\tt \ \ \ \ \ \ \ \ \ \ \ }
\startm
\m{\vdash}\m{(}\m{A}\m{\,R}\m{\,B}\m{\leftrightarrow}\m{\langle}\m{A}\m{,}\m{B}
\m{\rangle}\m{\in}\m{R}\m{)}
\endm
\begin{mmraw}%
|- ( A R B <-> <. A , B >. e. R ) \$.
\end{mmraw}

\noindent Define an abstraction class of ordered pairs\index{abstraction
class!of ordered
pairs}.  A special case of Definition 4.16, p.~14, of Takeuti and Zaring.
Note that $ z $ must be distinct from $ x $ and $ y $,
and $ z $ must not occur in $\varphi$, but $ x $ and $ y $ may be
identical and may appear in $\varphi$.

\vskip 0.5ex
\setbox\startprefix=\hbox{\tt \ \ df-opab\ \$a\ }
\setbox\contprefix=\hbox{\tt \ \ \ \ \ \ \ \ \ \ \ \ \ }
\startm
\m{\vdash}\m{\{}\m{\langle}\m{x}\m{,}\m{y}\m{\rangle}\m{|}\m{\varphi}\m{\}}\m{=}
\m{\{}\m{z}\m{|}\m{\exists}\m{x}\m{\exists}\m{y}\m{(}\m{z}\m{=}\m{\langle}\m{x}
\m{,}\m{y}\m{\rangle}\m{\wedge}\m{\varphi}\m{)}\m{\}}
\endm

\begin{mmraw}%
|- \{ <. x , y >. | ph \} = \{ z | E. x E. y ( z =
<. x , y >. /\ ph ) \} \$.
\end{mmraw}

\noindent Define the epsilon relation\index{epsilon relation}.  Similar to Definition
6.22, p.~30, of Takeuti and Zaring.

\vskip 0.5ex
\setbox\startprefix=\hbox{\tt \ \ df-eprel\ \$a\ }
\setbox\contprefix=\hbox{\tt \ \ \ \ \ \ \ \ \ \ \ \ \ \ }
\startm
\m{\vdash}\m{{\rm E}}\m{=}\m{\{}\m{\langle}\m{x}\m{,}\m{y}\m{\rangle}\m{|}\m{x}\m{
\in}\m{y}\m{\}}
\endm
\begin{mmraw}%
|- \_E = \{ <. x , y >. | x e. y \} \$.
\end{mmraw}

\noindent Define a founded relation\index{founded relation}.  $R$ is a founded
relation on $A$ iff\index{iff} (if and only if) each nonempty subset of $A$
has an ``$R$-minimal element.''  Similar to Definition 6.21, p.~30, of
Takeuti and Zaring.

\vskip 0.5ex
\setbox\startprefix=\hbox{\tt \ \ df-fr\ \$a\ }
\setbox\contprefix=\hbox{\tt \ \ \ \ \ \ \ \ \ \ \ }
\startm
\m{\vdash}\m{(}\m{R}\m{\,\mbox{\rm Fr}}\m{\,A}\m{\leftrightarrow}\m{\forall}\m{x}
\m{(}\m{(}\m{x}\m{\subseteq}\m{A}\m{\wedge}\m{\lnot}\m{x}\m{=}\m{\varnothing}
\m{)}\m{\rightarrow}\m{\exists}\m{y}\m{(}\m{y}\m{\in}\m{x}\m{\wedge}\m{(}\m{x}
\m{\cap}\m{\{}\m{z}\m{|}\m{z}\m{\,R}\m{\,y}\m{\}}\m{)}\m{=}\m{\varnothing}\m{)}
\m{)}\m{)}
\endm
\begin{mmraw}%
|- ( R Fr A <-> A. x ( ( x C\_ A \TAND -. x = (/) ) ->
E. y ( y e. x \TAND ( x i\^{}i \{ z | z R y \} ) = (/) ) ) ) \$.
\end{mmraw}

\noindent Define a well-ordering\index{well-ordering}.  $R$ is a well-ordering of $A$ iff
it is founded on $A$ and the elements of $A$ are pairwise $R$-comparable.
Similar to Definition 6.24(2), p.~30, of Takeuti and Zaring.

\vskip 0.5ex
\setbox\startprefix=\hbox{\tt \ \ df-we\ \$a\ }
\setbox\contprefix=\hbox{\tt \ \ \ \ \ \ \ \ \ \ \ }
\startm
\m{\vdash}\m{(}\m{R}\m{\,\mbox{\rm We}}\m{\,A}\m{\leftrightarrow}\m{(}\m{R}\m{\,
\mbox{\rm Fr}}\m{\,A}\m{\wedge}\m{\forall}\m{x}\m{\forall}\m{y}\m{(}\m{(}\m{x}\m{
\in}\m{A}\m{\wedge}\m{y}\m{\in}\m{A}\m{)}\m{\rightarrow}\m{(}\m{x}\m{\,R}\m{\,y}
\m{\vee}\m{x}\m{=}\m{y}\m{\vee}\m{y}\m{\,R}\m{\,x}\m{)}\m{)}\m{)}\m{)}
\endm
\begin{mmraw}%
( R We A <-> ( R Fr A \TAND A. x A. y ( ( x e.
A \TAND y e. A ) -> ( x R y \TOR x = y \TOR y R x ) ) ) ) \$.
\end{mmraw}

\noindent Define the ordinal predicate\index{ordinal predicate}, which is true for a class
that is transitive and is well-ordered by the epsilon relation.  Similar to
definition on p.~468, Bell and Machover.

\vskip 0.5ex
\setbox\startprefix=\hbox{\tt \ \ df-ord\ \$a\ }
\setbox\contprefix=\hbox{\tt \ \ \ \ \ \ \ \ \ \ \ \ }
\startm
\m{\vdash}\m{(}\m{\mbox{\rm Ord}}\m{\,A}\m{\leftrightarrow}\m{(}
\m{\mbox{\rm Tr}}\m{\,A}\m{\wedge}\m{E}\m{\,\mbox{\rm We}}\m{\,A}\m{)}\m{)}
\endm
\begin{mmraw}%
|- ( Ord A <-> ( Tr A \TAND E We A ) ) \$.
\end{mmraw}

\noindent Define the class of all ordinal numbers\index{ordinal number}.  An ordinal number is
a set that satisfies the ordinal predicate.  Definition 7.11 of Takeuti and
Zaring, p.~38.

\vskip 0.5ex
\setbox\startprefix=\hbox{\tt \ \ df-on\ \$a\ }
\setbox\contprefix=\hbox{\tt \ \ \ \ \ \ \ \ \ \ \ }
\startm
\m{\vdash}\m{\,\mbox{\rm On}}\m{=}\m{\{}\m{x}\m{|}\m{\mbox{\rm Ord}}\m{\,x}
\m{\}}
\endm
\begin{mmraw}%
|- On = \{ x | Ord x \} \$.
\end{mmraw}

\noindent Define the limit ordinal predicate\index{limit ordinal}, which is true for a
non-empty ordinal that is not a successor (i.e.\ that is the union of itself).
Compare Bell and Machover, p.~471 and Exercise (1), p.~42 of Takeuti and
Zaring.

\vskip 0.5ex
\setbox\startprefix=\hbox{\tt \ \ df-lim\ \$a\ }
\setbox\contprefix=\hbox{\tt \ \ \ \ \ \ \ \ \ \ \ \ }
\startm
\m{\vdash}\m{(}\m{\mbox{\rm Lim}}\m{\,A}\m{\leftrightarrow}\m{(}\m{\mbox{
\rm Ord}}\m{\,A}\m{\wedge}\m{\lnot}\m{A}\m{=}\m{\varnothing}\m{\wedge}\m{A}
\m{=}\m{\bigcup}\m{A}\m{)}\m{)}
\endm
\begin{mmraw}%
|- ( Lim A <-> ( Ord A \TAND -. A = (/) \TAND A = U. A ) ) \$.
\end{mmraw}

\noindent Define the successor\index{successor} of a class.  Definition 7.22 of Takeuti
and Zaring, p.~41.  Our definition is a generalization to classes, although it
is meaningless when classes are proper.

\vskip 0.5ex
\setbox\startprefix=\hbox{\tt \ \ df-suc\ \$a\ }
\setbox\contprefix=\hbox{\tt \ \ \ \ \ \ \ \ \ \ \ \ }
\startm
\m{\vdash}\m{\,\mbox{\rm suc}}\m{\,A}\m{=}\m{(}\m{A}\m{\cup}\m{\{}\m{A}\m{\}}
\m{)}
\endm
\begin{mmraw}%
|- suc A = ( A u. \{ A \} ) \$.
\end{mmraw}

\noindent Define the class of natural numbers\index{natural number}\index{omega
($\omega$)}.  Compare Bell and Machover, p.~471.\label{dfom}

\vskip 0.5ex
\setbox\startprefix=\hbox{\tt \ \ df-om\ \$a\ }
\setbox\contprefix=\hbox{\tt \ \ \ \ \ \ \ \ \ \ \ }
\startm
\m{\vdash}\m{\omega}\m{=}\m{\{}\m{x}\m{|}\m{(}\m{\mbox{\rm Ord}}\m{\,x}\m{
\wedge}\m{\forall}\m{y}\m{(}\m{\mbox{\rm Lim}}\m{\,y}\m{\rightarrow}\m{x}\m{
\in}\m{y}\m{)}\m{)}\m{\}}
\endm
\begin{mmraw}%
|- om = \{ x | ( Ord x \TAND A. y ( Lim y -> x e. y ) ) \} \$.
\end{mmraw}

\noindent Define the Cartesian product (also called the
cross product)\index{Cartesian product}\index{cross product}
of two classes.  Definition 9.11 of Quine, p.~64.

\vskip 0.5ex
\setbox\startprefix=\hbox{\tt \ \ df-xp\ \$a\ }
\setbox\contprefix=\hbox{\tt \ \ \ \ \ \ \ \ \ \ \ }
\startm
\m{\vdash}\m{(}\m{A}\m{\times}\m{B}\m{)}\m{=}\m{\{}\m{\langle}\m{x}\m{,}\m{y}
\m{\rangle}\m{|}\m{(}\m{x}\m{\in}\m{A}\m{\wedge}\m{y}\m{\in}\m{B}\m{)}\m{\}}
\endm
\begin{mmraw}%
|- ( A X. B ) = \{ <. x , y >. | ( x e. A \TAND y e. B) \} \$.
\end{mmraw}

\noindent Define a relation\index{relation}.  Definition 6.4(1) of Takeuti and
Zaring, p.~23.

\vskip 0.5ex
\setbox\startprefix=\hbox{\tt \ \ df-rel\ \$a\ }
\setbox\contprefix=\hbox{\tt \ \ \ \ \ \ \ \ \ \ \ \ }
\startm
\m{\vdash}\m{(}\m{\mbox{\rm Rel}}\m{\,A}\m{\leftrightarrow}\m{A}\m{\subseteq}
\m{(}\m{{\rm V}}\m{\times}\m{{\rm V}}\m{)}\m{)}
\endm
\begin{mmraw}%
|- ( Rel A <-> A C\_ ( {\char`\_}V X. {\char`\_}V ) ) \$.
\end{mmraw}

\noindent Define the domain\index{domain} of a class.  Definition 6.5(1) of
Takeuti and Zaring, p.~24.

\vskip 0.5ex
\setbox\startprefix=\hbox{\tt \ \ df-dm\ \$a\ }
\setbox\contprefix=\hbox{\tt \ \ \ \ \ \ \ \ \ \ \ }
\startm
\m{\vdash}\m{\,\mbox{\rm dom}}\m{A}\m{=}\m{\{}\m{x}\m{|}\m{\exists}\m{y}\m{
\langle}\m{x}\m{,}\m{y}\m{\rangle}\m{\in}\m{A}\m{\}}
\endm
\begin{mmraw}%
|- dom A = \{ x | E. y <. x , y >. e. A \} \$.
\end{mmraw}

\noindent Define the range\index{range} of a class.  Definition 6.5(2) of
Takeuti and Zaring, p.~24.

\vskip 0.5ex
\setbox\startprefix=\hbox{\tt \ \ df-rn\ \$a\ }
\setbox\contprefix=\hbox{\tt \ \ \ \ \ \ \ \ \ \ \ }
\startm
\m{\vdash}\m{\,\mbox{\rm ran}}\m{A}\m{=}\m{\{}\m{y}\m{|}\m{\exists}\m{x}\m{
\langle}\m{x}\m{,}\m{y}\m{\rangle}\m{\in}\m{A}\m{\}}
\endm
\begin{mmraw}%
|- ran A = \{ y | E. x <. x , y >. e. A \} \$.
\end{mmraw}

\noindent Define the restriction\index{restriction} of a class.  Definition
6.6(1) of Takeuti and Zaring, p.~24.

\vskip 0.5ex
\setbox\startprefix=\hbox{\tt \ \ df-res\ \$a\ }
\setbox\contprefix=\hbox{\tt \ \ \ \ \ \ \ \ \ \ \ \ }
\startm
\m{\vdash}\m{(}\m{A}\m{\restriction}\m{B}\m{)}\m{=}\m{(}\m{A}\m{\cap}\m{(}\m{B}
\m{\times}\m{{\rm V}}\m{)}\m{)}
\endm
\begin{mmraw}%
|- ( A |` B ) = ( A i\^{}i ( B X. {\char`\_}V ) ) \$.
\end{mmraw}

\noindent Define the image\index{image} of a class.  Definition 6.6(2) of
Takeuti and Zaring, p.~24.

\vskip 0.5ex
\setbox\startprefix=\hbox{\tt \ \ df-ima\ \$a\ }
\setbox\contprefix=\hbox{\tt \ \ \ \ \ \ \ \ \ \ \ \ }
\startm
\m{\vdash}\m{(}\m{A}\m{``}\m{B}\m{)}\m{=}\m{\,\mbox{\rm ran}}\m{\,(}\m{A}\m{
\restriction}\m{B}\m{)}
\endm
\begin{mmraw}%
|- ( A " B ) = ran ( A |` B ) \$.
\end{mmraw}

\noindent Define the composition\index{composition} of two classes.  Definition 6.6(3) of
Takeuti and Zaring, p.~24.

\vskip 0.5ex
\setbox\startprefix=\hbox{\tt \ \ df-co\ \$a\ }
\setbox\contprefix=\hbox{\tt \ \ \ \ \ \ \ \ \ \ \ \ }
\startm
\m{\vdash}\m{(}\m{A}\m{\circ}\m{B}\m{)}\m{=}\m{\{}\m{\langle}\m{x}\m{,}\m{y}\m{
\rangle}\m{|}\m{\exists}\m{z}\m{(}\m{\langle}\m{x}\m{,}\m{z}\m{\rangle}\m{\in}
\m{B}\m{\wedge}\m{\langle}\m{z}\m{,}\m{y}\m{\rangle}\m{\in}\m{A}\m{)}\m{\}}
\endm
\begin{mmraw}%
|- ( A o. B ) = \{ <. x , y >. | E. z ( <. x , z
>. e. B \TAND <. z , y >. e. A ) \} \$.
\end{mmraw}

\noindent Define a function\index{function}.  Definition 6.4(4) of Takeuti and
Zaring, p.~24.

\vskip 0.5ex
\setbox\startprefix=\hbox{\tt \ \ df-fun\ \$a\ }
\setbox\contprefix=\hbox{\tt \ \ \ \ \ \ \ \ \ \ \ \ }
\startm
\m{\vdash}\m{(}\m{\mbox{\rm Fun}}\m{\,A}\m{\leftrightarrow}\m{(}
\m{\mbox{\rm Rel}}\m{\,A}\m{\wedge}
\m{\forall}\m{x}\m{\exists}\m{z}\m{\forall}\m{y}\m{(}
\m{\langle}\m{x}\m{,}\m{y}\m{\rangle}\m{\in}\m{A}\m{\rightarrow}\m{y}\m{=}\m{z}
\m{)}\m{)}\m{)}
\endm
\begin{mmraw}%
|- ( Fun A <-> ( Rel A /\ A. x E. z A. y ( <. x
   , y >. e. A -> y = z ) ) ) \$.
\end{mmraw}

\noindent Define a function with domain.  Definition 6.15(1) of Takeuti and
Zaring, p.~27.

\vskip 0.5ex
\setbox\startprefix=\hbox{\tt \ \ df-fn\ \$a\ }
\setbox\contprefix=\hbox{\tt \ \ \ \ \ \ \ \ \ \ \ }
\startm
\m{\vdash}\m{(}\m{A}\m{\,\mbox{\rm Fn}}\m{\,B}\m{\leftrightarrow}\m{(}
\m{\mbox{\rm Fun}}\m{\,A}\m{\wedge}\m{\mbox{\rm dom}}\m{\,A}\m{=}\m{B}\m{)}
\m{)}
\endm
\begin{mmraw}%
|- ( A Fn B <-> ( Fun A \TAND dom A = B ) ) \$.
\end{mmraw}

\noindent Define a function with domain and co-domain.  Definition 6.15(3)
of Takeuti and Zaring, p.~27.

\vskip 0.5ex
\setbox\startprefix=\hbox{\tt \ \ df-f\ \$a\ }
\setbox\contprefix=\hbox{\tt \ \ \ \ \ \ \ \ \ \ }
\startm
\m{\vdash}\m{(}\m{F}\m{:}\m{A}\m{\longrightarrow}\m{B}\m{
\leftrightarrow}\m{(}\m{F}\m{\,\mbox{\rm Fn}}\m{\,A}\m{\wedge}\m{
\mbox{\rm ran}}\m{\,F}\m{\subseteq}\m{B}\m{)}\m{)}
\endm
\begin{mmraw}%
|- ( F : A --> B <-> ( F Fn A \TAND ran F C\_ B ) ) \$.
\end{mmraw}

\noindent Define a one-to-one function\index{one-to-one function}.  Compare
Definition 6.15(5) of Takeuti and Zaring, p.~27.

\vskip 0.5ex
\setbox\startprefix=\hbox{\tt \ \ df-f1\ \$a\ }
\setbox\contprefix=\hbox{\tt \ \ \ \ \ \ \ \ \ \ \ }
\startm
\m{\vdash}\m{(}\m{F}\m{:}\m{A}\m{
\raisebox{.5ex}{${\textstyle{\:}_{\mbox{\footnotesize\rm
1\tt -\rm 1}}}\atop{\textstyle{
\longrightarrow}\atop{\textstyle{}^{\mbox{\footnotesize\rm {\ }}}}}$}
}\m{B}
\m{\leftrightarrow}\m{(}\m{F}\m{:}\m{A}\m{\longrightarrow}\m{B}
\m{\wedge}\m{\forall}\m{y}\m{\exists}\m{z}\m{\forall}\m{x}\m{(}\m{\langle}\m{x}
\m{,}\m{y}\m{\rangle}\m{\in}\m{F}\m{\rightarrow}\m{x}\m{=}\m{z}\m{)}\m{)}\m{)}
\endm
\begin{mmraw}%
|- ( F : A -1-1-> B <-> ( F : A --> B \TAND
   A. y E. z A. x ( <. x , y >. e. F -> x = z ) ) ) \$.
\end{mmraw}

\noindent Define an onto function\index{onto function}.  Definition 6.15(4) of Takeuti and
Zaring, p.~27.

\vskip 0.5ex
\setbox\startprefix=\hbox{\tt \ \ df-fo\ \$a\ }
\setbox\contprefix=\hbox{\tt \ \ \ \ \ \ \ \ \ \ \ }
\startm
\m{\vdash}\m{(}\m{F}\m{:}\m{A}\m{
\raisebox{.5ex}{${\textstyle{\:}_{\mbox{\footnotesize\rm
{\ }}}}\atop{\textstyle{
\longrightarrow}\atop{\textstyle{}^{\mbox{\footnotesize\rm onto}}}}$}
}\m{B}
\m{\leftrightarrow}\m{(}\m{F}\m{\,\mbox{\rm Fn}}\m{\,A}\m{\wedge}
\m{\mbox{\rm ran}}\m{\,F}\m{=}\m{B}\m{)}\m{)}
\endm
\begin{mmraw}%
|- ( F : A -onto-> B <-> ( F Fn A /\ ran F = B ) ) \$.
\end{mmraw}

\noindent Define a one-to-one, onto function.  Compare Definition 6.15(6) of
Takeuti and Zaring, p.~27.

\vskip 0.5ex
\setbox\startprefix=\hbox{\tt \ \ df-f1o\ \$a\ }
\setbox\contprefix=\hbox{\tt \ \ \ \ \ \ \ \ \ \ \ \ }
\startm
\m{\vdash}\m{(}\m{F}\m{:}\m{A}
\m{
\raisebox{.5ex}{${\textstyle{\:}_{\mbox{\footnotesize\rm
1\tt -\rm 1}}}\atop{\textstyle{
\longrightarrow}\atop{\textstyle{}^{\mbox{\footnotesize\rm onto}}}}$}
}
\m{B}
\m{\leftrightarrow}\m{(}\m{F}\m{:}\m{A}
\m{
\raisebox{.5ex}{${\textstyle{\:}_{\mbox{\footnotesize\rm
1\tt -\rm 1}}}\atop{\textstyle{
\longrightarrow}\atop{\textstyle{}^{\mbox{\footnotesize\rm {\ }}}}}$}
}
\m{B}\m{\wedge}\m{F}\m{:}\m{A}
\m{
\raisebox{.5ex}{${\textstyle{\:}_{\mbox{\footnotesize\rm
{\ }}}}\atop{\textstyle{
\longrightarrow}\atop{\textstyle{}^{\mbox{\footnotesize\rm onto}}}}$}
}
\m{B}\m{)}\m{)}
\endm
\begin{mmraw}%
|- ( F : A -1-1-onto-> B <-> ( F : A -1-1-> B? \TAND F : A -onto-> B ) ) \$.?
\end{mmraw}

\noindent Define the value of a function\index{function value}.  This
definition applies to any class and evaluates to the empty set when it is not
meaningful. Note that $ F`A$ means the same thing as the more familiar $ F(A)$
notation for a function's value at $A$.  The $ F`A$ notation is common in
formal set theory.\label{df-fv}

\vskip 0.5ex
\setbox\startprefix=\hbox{\tt \ \ df-fv\ \$a\ }
\setbox\contprefix=\hbox{\tt \ \ \ \ \ \ \ \ \ \ \ }
\startm
\m{\vdash}\m{(}\m{F}\m{`}\m{A}\m{)}\m{=}\m{\bigcup}\m{\{}\m{x}\m{|}\m{(}\m{F}%
\m{``}\m{\{}\m{A}\m{\}}\m{)}\m{=}\m{\{}\m{x}\m{\}}\m{\}}
\endm
\begin{mmraw}%
|- ( F ` A ) = U. \{ x | ( F " \{ A \} ) = \{ x \} \} \$.
\end{mmraw}

\noindent Define the result of an operation.\index{operation}  Here, $F$ is
     an operation on two
     values (such as $+$ for real numbers).   This is defined for proper
     classes $A$ and $B$ even though not meaningful in that case.  However,
     the definition can be meaningful when $F$ is a proper class.\label{dfopr}

\vskip 0.5ex
\setbox\startprefix=\hbox{\tt \ \ df-opr\ \$a\ }
\setbox\contprefix=\hbox{\tt \ \ \ \ \ \ \ \ \ \ \ \ }
\startm
\m{\vdash}\m{(}\m{A}\m{\,F}\m{\,B}\m{)}\m{=}\m{(}\m{F}\m{`}\m{\langle}\m{A}%
\m{,}\m{B}\m{\rangle}\m{)}
\endm
\begin{mmraw}%
|- ( A F B ) = ( F ` <. A , B >. ) \$.
\end{mmraw}

\section{Tricks of the Trade}\label{tricks}

In the \texttt{set.mm}\index{set theory database (\texttt{set.mm})} database our goal
was usually to conform to modern notation.  However in some cases the
relationship to standard textbook language may be obscured by several
unconventional devices we used to simplify the development and to take
advantage of the Metamath language.  In this section we will describe some
common conventions used in \texttt{set.mm}.

\begin{itemize}
\item
The turnstile\index{turnstile ({$\,\vdash$})} symbol, $\vdash$, meaning ``it
is provable that,'' is the first token of all assertions and hypotheses that
aren't syntax constructions.  This is a standard convention in logic.  (We
mentioned this earlier, but this symbol is bothersome to some people without a
logic background.  It has no deeper meaning but just provides us with a way to
distinguish syntax constructions from ordinary mathematical statements.)

\item
A hypothesis of the form

\vskip 1ex
\setbox\startprefix=\hbox{\tt \ \ \ \ \ \ \ \ \ \$e\ }
\setbox\contprefix=\hbox{\tt \ \ \ \ \ \ \ \ \ \ }
\startm
\m{\vdash}\m{(}\m{\varphi}\m{\rightarrow}\m{\forall}\m{x}\m{\varphi}\m{)}
\endm
\vskip 1ex

should be read ``assume variable $x$ is (effectively) not free in wff
$\varphi$.''\index{effectively not free}
Literally, this says ``assume it is provable that $\varphi \rightarrow \forall
x\, \varphi$.''  This device lets us avoid the complexities associated with
the standard treatment of free and bound variables.
%% Uncomment this when uncommenting section {formalspec} below
The footnote on p.~\pageref{effectivelybound} discusses this further.

\item
A statement of one of the forms

\vskip 1ex
\setbox\startprefix=\hbox{\tt \ \ \ \ \ \ \ \ \ \$a\ }
\setbox\contprefix=\hbox{\tt \ \ \ \ \ \ \ \ \ \ }
\startm
\m{\vdash}\m{(}\m{\lnot}\m{\forall}\m{x}\m{\,x}\m{=}\m{y}\m{\rightarrow}
\m{\ldots}\m{)}
\endm
\setbox\startprefix=\hbox{\tt \ \ \ \ \ \ \ \ \ \$p\ }
\setbox\contprefix=\hbox{\tt \ \ \ \ \ \ \ \ \ \ }
\startm
\m{\vdash}\m{(}\m{\lnot}\m{\forall}\m{x}\m{\,x}\m{=}\m{y}\m{\rightarrow}
\m{\ldots}\m{)}
\endm
\vskip 1ex

should be read ``if $x$ and $y$ are distinct variables, then...''  This
antecedent provides us with a technical device to avoid the need for the
\texttt{\$d} statement early in our development of predicate calculus,
permitting symbol manipulations to be as conceptually simple as those in
propositional calculus.  However, the \texttt{\$d} statement eventually
becomes a requirement, and after that this device is rarely used.

\item
The statement

\vskip 1ex
\setbox\startprefix=\hbox{\tt \ \ \ \ \ \ \ \ \ \$d\ }
\setbox\contprefix=\hbox{\tt \ \ \ \ \ \ \ \ \ \ }
\startm
\m{x}\m{\,y}
\endm
\vskip 1ex

should be read ``assume $x$ and $y$ are distinct variables.''

\item
The statement

\vskip 1ex
\setbox\startprefix=\hbox{\tt \ \ \ \ \ \ \ \ \ \$d\ }
\setbox\contprefix=\hbox{\tt \ \ \ \ \ \ \ \ \ \ }
\startm
\m{x}\m{\,\varphi}
\endm
\vskip 1ex

should be read ``assume $x$ does not occur in $\varphi$.''

\item
The statement

\vskip 1ex
\setbox\startprefix=\hbox{\tt \ \ \ \ \ \ \ \ \ \$d\ }
\setbox\contprefix=\hbox{\tt \ \ \ \ \ \ \ \ \ \ }
\startm
\m{x}\m{\,A}
\endm
\vskip 1ex

should be read ``assume variable $x$ does not occur in class $A$.''

\item
The restriction and hypothesis group

\vskip 1ex
\setbox\startprefix=\hbox{\tt \ \ \ \ \ \ \ \ \ \$d\ }
\setbox\contprefix=\hbox{\tt \ \ \ \ \ \ \ \ \ \ }
\startm
\m{x}\m{\,A}
\endm
\setbox\startprefix=\hbox{\tt \ \ \ \ \ \ \ \ \ \$d\ }
\setbox\contprefix=\hbox{\tt \ \ \ \ \ \ \ \ \ \ }
\startm
\m{x}\m{\,\psi}
\endm
\setbox\startprefix=\hbox{\tt \ \ \ \ \ \ \ \ \ \$e\ }
\setbox\contprefix=\hbox{\tt \ \ \ \ \ \ \ \ \ \ }
\startm
\m{\vdash}\m{(}\m{x}\m{=}\m{A}\m{\rightarrow}\m{(}\m{\varphi}\m{\leftrightarrow}
\m{\psi}\m{)}\m{)}
\endm
\vskip 1ex

is frequently used in place of explicit substitution, meaning ``assume
$\psi$ results from the proper substitution of $A$ for $x$ in
$\varphi$.''  Sometimes ``\texttt{\$e} $\vdash ( \psi \rightarrow
\forall x \, \psi )$'' is used instead of ``\texttt{\$d} $x\, \psi $,''
which requires only that $x$ be effectively not free in $\varphi$ but
not necessarily absent from it.  The use of implicit
substitution\index{substitution!implicit} is partly a matter of personal
style, although it may make proofs somewhat shorter than would be the
case with explicit substitution.

\item
The hypothesis


\vskip 1ex
\setbox\startprefix=\hbox{\tt \ \ \ \ \ \ \ \ \ \$e\ }
\setbox\contprefix=\hbox{\tt \ \ \ \ \ \ \ \ \ \ }
\startm
\m{\vdash}\m{A}\m{\in}\m{{\rm V}}
\endm
\vskip 1ex

should be read ``assume class $A$ is a set (i.e.\ exists).''
This is a convenient convention used by Quine.

\item
The restriction and hypothesis

\vskip 1ex
\setbox\startprefix=\hbox{\tt \ \ \ \ \ \ \ \ \ \$d\ }
\setbox\contprefix=\hbox{\tt \ \ \ \ \ \ \ \ \ \ }
\startm
\m{x}\m{\,y}
\endm
\setbox\startprefix=\hbox{\tt \ \ \ \ \ \ \ \ \ \$e\ }
\setbox\contprefix=\hbox{\tt \ \ \ \ \ \ \ \ \ \ }
\startm
\m{\vdash}\m{(}\m{y}\m{\in}\m{A}\m{\rightarrow}\m{\forall}\m{x}\m{\,y}
\m{\in}\m{A}\m{)}
\endm
\vskip 1ex

should be read ``assume variable $x$ is
(effectively) not free in class $A$.''

\end{itemize}

\section{A Theorem Sampler}\label{sometheorems}

In this section we list some of the more important theorems that are proved in
the \texttt{set.mm} database, and they illustrate the kinds of things that can be
done with Metamath.  While all of these facts are well-known results,
Metamath offers the advantage of easily allowing you to trace their
derivation back to axioms.  Our intent here is not to try to explain the
details or motivation; for this we refer you to the textbooks that are
mentioned in the descriptions.  (The \texttt{set.mm} file has bibliographic
references for the text references.)  Their proofs often embody important
concepts you may wish to explore with the Metamath program (see
Section~\ref{exploring}).  All the symbols that are used here are defined in
Section~\ref{hierarchy}.  For brevity we haven't included the \texttt{\$d}
restrictions or \texttt{\$f} hypotheses for these theorems; when you are
uncertain consult the \texttt{set.mm} database.

We start with \texttt{syl} (principle of the syllogism).
In \textit{Principia Mathematica}
Whitehead and Russell call this ``the principle of the
syllogism... because... the syllogism in Barbara is derived from them''
\cite[quote after Theorem *2.06 p.~101]{PM}.
Some authors call this law a ``hypothetical syllogism.''
As of 2019 \texttt{syl} is the most commonly referenced proven
assertion in the \texttt{set.mm} database.\footnote{
The Metamath program command \texttt{show usage}
shows the number of references.
On 2019-04-29 (commit 71cbbbdb387e) \texttt{syl} was directly referenced
10,819 times. The second most commonly referenced proven assertion
was \texttt{eqid}, which was directly referenced 7,738 times.
}

\vskip 2ex
\noindent Theorem syl (principle of the syllogism)\index{Syllogism}%
\index{\texttt{syl}}\label{syl}.

\vskip 0.5ex
\setbox\startprefix=\hbox{\tt \ \ syl.1\ \$e\ }
\setbox\contprefix=\hbox{\tt \ \ \ \ \ \ \ \ \ \ \ }
\startm
\m{\vdash}\m{(}\m{\varphi}\m{ \rightarrow }\m{\psi}\m{)}
\endm
\setbox\startprefix=\hbox{\tt \ \ syl.2\ \$e\ }
\setbox\contprefix=\hbox{\tt \ \ \ \ \ \ \ \ \ \ \ }
\startm
\m{\vdash}\m{(}\m{\psi}\m{ \rightarrow }\m{\chi}\m{)}
\endm
\setbox\startprefix=\hbox{\tt \ \ syl\ \$p\ }
\setbox\contprefix=\hbox{\tt \ \ \ \ \ \ \ \ \ }
\startm
\m{\vdash}\m{(}\m{\varphi}\m{ \rightarrow }\m{\chi}\m{)}
\endm
\vskip 2ex

The following theorem is not very deep but provides us with a notational device
that is frequently used.  It allows us to use the expression ``$A \in V$'' as
a compact way of saying that class $A$ exists, i.e.\ is a set.

\vskip 2ex
\noindent Two ways to say ``$A$ is a set'':  $A$ is a member of the universe
$V$ if and only if $A$ exists (i.e.\ there exists a set equal to $A$).
Theorem 6.9 of Quine, p. 43.

\vskip 0.5ex
\setbox\startprefix=\hbox{\tt \ \ isset\ \$p\ }
\setbox\contprefix=\hbox{\tt \ \ \ \ \ \ \ \ \ \ \ }
\startm
\m{\vdash}\m{(}\m{A}\m{\in}\m{{\rm V}}\m{\leftrightarrow}\m{\exists}\m{x}\m{\,x}\m{=}
\m{A}\m{)}
\endm
\vskip 1ex

Next we prove the axioms of standard ZF set theory that were missing from our
axiom system.  From our point of view they are theorems since they
can be derived from the other axioms.

\vskip 2ex
\noindent Axiom of Separation\index{Axiom of Separation}
(Aussonderung)\index{Aussonderung} proved from the other axioms of ZF set
theory.  Compare Exercise 4 of Takeuti and Zaring, p.~22.

\vskip 0.5ex
\setbox\startprefix=\hbox{\tt \ \ inex1.1\ \$e\ }
\setbox\contprefix=\hbox{\tt \ \ \ \ \ \ \ \ \ \ \ \ \ \ \ }
\startm
\m{\vdash}\m{A}\m{\in}\m{{\rm V}}
\endm
\setbox\startprefix=\hbox{\tt \ \ inex\ \$p\ }
\setbox\contprefix=\hbox{\tt \ \ \ \ \ \ \ \ \ \ \ \ \ }
\startm
\m{\vdash}\m{(}\m{A}\m{\cap}\m{B}\m{)}\m{\in}\m{{\rm V}}
\endm
\vskip 1ex

\noindent Axiom of the Null Set\index{Axiom of the Null Set} proved from the
other axioms of ZF set theory. Corollary 5.16 of Takeuti and Zaring, p.~20.

\vskip 0.5ex
\setbox\startprefix=\hbox{\tt \ \ 0ex\ \$p\ }
\setbox\contprefix=\hbox{\tt \ \ \ \ \ \ \ \ \ \ \ \ }
\startm
\m{\vdash}\m{\varnothing}\m{\in}\m{{\rm V}}
\endm
\vskip 1ex

\noindent The Axiom of Pairing\index{Axiom of Pairing} proved from the other
axioms of ZF set theory.  Theorem 7.13 of Quine, p.~51.
\vskip 0.5ex
\setbox\startprefix=\hbox{\tt \ \ prex\ \$p\ }
\setbox\contprefix=\hbox{\tt \ \ \ \ \ \ \ \ \ \ \ \ \ \ }
\startm
\m{\vdash}\m{\{}\m{A}\m{,}\m{B}\m{\}}\m{\in}\m{{\rm V}}
\endm
\vskip 2ex

Next we will list some famous or important theorems that are proved in
the \texttt{set.mm} database.  None of them except \texttt{omex}
require the Axiom of Infinity, as you can verify with the \texttt{show
trace{\char`\_}back} Metamath command.

\vskip 2ex
\noindent The resolution of Russell's paradox\index{Russell's paradox}.  There
exists no set containing the set of all sets which are not members of
themselves.  Proposition 4.14 of Takeuti and Zaring, p.~14.

\vskip 0.5ex
\setbox\startprefix=\hbox{\tt \ \ ru\ \$p\ }
\setbox\contprefix=\hbox{\tt \ \ \ \ \ \ \ \ }
\startm
\m{\vdash}\m{\lnot}\m{\exists}\m{x}\m{\,x}\m{=}\m{\{}\m{y}\m{|}\m{\lnot}\m{y}
\m{\in}\m{y}\m{\}}
\endm
\vskip 1ex

\noindent Cantor's theorem\index{Cantor's theorem}.  No set can be mapped onto
its power set.  Compare Theorem 6B(b) of Enderton, p.~132.

\vskip 0.5ex
\setbox\startprefix=\hbox{\tt \ \ canth.1\ \$e\ }
\setbox\contprefix=\hbox{\tt \ \ \ \ \ \ \ \ \ \ \ \ \ }
\startm
\m{\vdash}\m{A}\m{\in}\m{{\rm V}}
\endm
\setbox\startprefix=\hbox{\tt \ \ canth\ \$p\ }
\setbox\contprefix=\hbox{\tt \ \ \ \ \ \ \ \ \ \ \ }
\startm
\m{\vdash}\m{\lnot}\m{F}\m{:}\m{A}\m{\raisebox{.5ex}{${\textstyle{\:}_{
\mbox{\footnotesize\rm {\ }}}}\atop{\textstyle{\longrightarrow}\atop{
\textstyle{}^{\mbox{\footnotesize\rm onto}}}}$}}\m{{\cal P}}\m{A}
\endm
\vskip 1ex

\noindent The Burali-Forti paradox\index{Burali-Forti paradox}.  No set
contains all ordinal numbers. Enderton, p.~194.  (Burali-Forti was one person,
not two.)

\vskip 0.5ex
\setbox\startprefix=\hbox{\tt \ \ onprc\ \$p\ }
\setbox\contprefix=\hbox{\tt \ \ \ \ \ \ \ \ \ \ \ \ }
\startm
\m{\vdash}\m{\lnot}\m{\mbox{\rm On}}\m{\in}\m{{\rm V}}
\endm
\vskip 1ex

\noindent Peano's postulates\index{Peano's postulates} for arithmetic.
Proposition 7.30 of Takeuti and Zaring, pp.~42--43.  The objects being
described are the members of $\omega$ i.e.\ the natural numbers 0, 1,
2,\ldots.  The successor\index{successor} operation suc means ``plus
one.''  \texttt{peano1} says that 0 (which is defined as the empty set)
is a natural number.  \texttt{peano2} says that if $A$ is a natural
number, so is $A+1$.  \texttt{peano3} says that 0 is not the successor
of any natural number.  \texttt{peano4} says that two natural numbers
are equal if and only if their successors are equal.  \texttt{peano5} is
essentially the same as mathematical induction.

\vskip 1ex
\setbox\startprefix=\hbox{\tt \ \ peano1\ \$p\ }
\setbox\contprefix=\hbox{\tt \ \ \ \ \ \ \ \ \ \ \ \ }
\startm
\m{\vdash}\m{\varnothing}\m{\in}\m{\omega}
\endm
\vskip 1.5ex

\setbox\startprefix=\hbox{\tt \ \ peano2\ \$p\ }
\setbox\contprefix=\hbox{\tt \ \ \ \ \ \ \ \ \ \ \ \ }
\startm
\m{\vdash}\m{(}\m{A}\m{\in}\m{\omega}\m{\rightarrow}\m{{\rm suc}}\m{A}\m{\in}%
\m{\omega}\m{)}
\endm
\vskip 1.5ex

\setbox\startprefix=\hbox{\tt \ \ peano3\ \$p\ }
\setbox\contprefix=\hbox{\tt \ \ \ \ \ \ \ \ \ \ \ \ }
\startm
\m{\vdash}\m{(}\m{A}\m{\in}\m{\omega}\m{\rightarrow}\m{\lnot}\m{{\rm suc}}%
\m{A}\m{=}\m{\varnothing}\m{)}
\endm
\vskip 1.5ex

\setbox\startprefix=\hbox{\tt \ \ peano4\ \$p\ }
\setbox\contprefix=\hbox{\tt \ \ \ \ \ \ \ \ \ \ \ \ }
\startm
\m{\vdash}\m{(}\m{(}\m{A}\m{\in}\m{\omega}\m{\wedge}\m{B}\m{\in}\m{\omega}%
\m{)}\m{\rightarrow}\m{(}\m{{\rm suc}}\m{A}\m{=}\m{{\rm suc}}\m{B}\m{%
\leftrightarrow}\m{A}\m{=}\m{B}\m{)}\m{)}
\endm
\vskip 1.5ex

\setbox\startprefix=\hbox{\tt \ \ peano5\ \$p\ }
\setbox\contprefix=\hbox{\tt \ \ \ \ \ \ \ \ \ \ \ \ }
\startm
\m{\vdash}\m{(}\m{(}\m{\varnothing}\m{\in}\m{A}\m{\wedge}\m{\forall}\m{x}\m{%
\in}\m{\omega}\m{(}\m{x}\m{\in}\m{A}\m{\rightarrow}\m{{\rm suc}}\m{x}\m{\in}%
\m{A}\m{)}\m{)}\m{\rightarrow}\m{\omega}\m{\subseteq}\m{A}\m{)}
\endm
\vskip 1.5ex

\noindent Finite Induction (mathematical induction).\index{finite
induction}\index{mathematical induction} The first hypothesis is the
basis and the second is the induction hypothesis.  Theorem Schema 22 of
Suppes, p.~136.

\vskip 0.5ex
\setbox\startprefix=\hbox{\tt \ \ findes.1\ \$e\ }
\setbox\contprefix=\hbox{\tt \ \ \ \ \ \ \ \ \ \ \ \ \ \ }
\startm
\m{\vdash}\m{[}\m{\varnothing}\m{/}\m{x}\m{]}\m{\varphi}
\endm
\setbox\startprefix=\hbox{\tt \ \ findes.2\ \$e\ }
\setbox\contprefix=\hbox{\tt \ \ \ \ \ \ \ \ \ \ \ \ \ \ }
\startm
\m{\vdash}\m{(}\m{x}\m{\in}\m{\omega}\m{\rightarrow}\m{(}\m{\varphi}\m{%
\rightarrow}\m{[}\m{{\rm suc}}\m{x}\m{/}\m{x}\m{]}\m{\varphi}\m{)}\m{)}
\endm
\setbox\startprefix=\hbox{\tt \ \ findes\ \$p\ }
\setbox\contprefix=\hbox{\tt \ \ \ \ \ \ \ \ \ \ \ \ }
\startm
\m{\vdash}\m{(}\m{x}\m{\in}\m{\omega}\m{\rightarrow}\m{\varphi}\m{)}
\endm
\vskip 1ex

\noindent Transfinite Induction with explicit substitution.  The first
hypothesis is the basis, the second is the induction hypothesis for
successors, and the third is the induction hypothesis for limit
ordinals.  Theorem Schema 4 of Suppes, p. 197.

\vskip 0.5ex
\setbox\startprefix=\hbox{\tt \ \ tfindes.1\ \$e\ }
\setbox\contprefix=\hbox{\tt \ \ \ \ \ \ \ \ \ \ \ \ \ \ \ }
\startm
\m{\vdash}\m{[}\m{\varnothing}\m{/}\m{x}\m{]}\m{\varphi}
\endm
\setbox\startprefix=\hbox{\tt \ \ tfindes.2\ \$e\ }
\setbox\contprefix=\hbox{\tt \ \ \ \ \ \ \ \ \ \ \ \ \ \ \ }
\startm
\m{\vdash}\m{(}\m{x}\m{\in}\m{{\rm On}}\m{\rightarrow}\m{(}\m{\varphi}\m{%
\rightarrow}\m{[}\m{{\rm suc}}\m{x}\m{/}\m{x}\m{]}\m{\varphi}\m{)}\m{)}
\endm
\setbox\startprefix=\hbox{\tt \ \ tfindes.3\ \$e\ }
\setbox\contprefix=\hbox{\tt \ \ \ \ \ \ \ \ \ \ \ \ \ \ \ }
\startm
\m{\vdash}\m{(}\m{{\rm Lim}}\m{y}\m{\rightarrow}\m{(}\m{\forall}\m{x}\m{\in}%
\m{y}\m{\varphi}\m{\rightarrow}\m{[}\m{y}\m{/}\m{x}\m{]}\m{\varphi}\m{)}\m{)}
\endm
\setbox\startprefix=\hbox{\tt \ \ tfindes\ \$p\ }
\setbox\contprefix=\hbox{\tt \ \ \ \ \ \ \ \ \ \ \ \ \ }
\startm
\m{\vdash}\m{(}\m{x}\m{\in}\m{{\rm On}}\m{\rightarrow}\m{\varphi}\m{)}
\endm
\vskip 1ex

\noindent Principle of Transfinite Recursion.\index{transfinite
recursion} Theorem 7.41 of Takeuti and Zaring, p.~47.  Transfinite
recursion is the key theorem that allows arithmetic of ordinals to be
rigorously defined, and has many other important uses as well.
Hypotheses \texttt{tfr.1} and \texttt{tfr.2} specify a certain (proper)
class $ F$.  The complicated definition of $ F$ is not important in
itself; what is important is that there be such an $ F$ with the
required properties, and we show this by displaying $ F$ explicitly.
\texttt{tfr1} states that $ F$ is a function whose domain is the set of
ordinal numbers.  \texttt{tfr2} states that any value of $ F$ is
completely determined by its previous values and the values of an
auxiliary function, $G$.  \texttt{tfr3} states that $ F$ is unique,
i.e.\ it is the only function that satisfies \texttt{tfr1} and
\texttt{tfr2}.  Note that $ f$ is an individual variable like $x$ and
$y$; it is just a mnemonic to remind us that $A$ is a collection of
functions.

\vskip 0.5ex
\setbox\startprefix=\hbox{\tt \ \ tfr.1\ \$e\ }
\setbox\contprefix=\hbox{\tt \ \ \ \ \ \ \ \ \ \ \ }
\startm
\m{\vdash}\m{A}\m{=}\m{\{}\m{f}\m{|}\m{\exists}\m{x}\m{\in}\m{{\rm On}}\m{(}%
\m{f}\m{{\rm Fn}}\m{x}\m{\wedge}\m{\forall}\m{y}\m{\in}\m{x}\m{(}\m{f}\m{`}%
\m{y}\m{)}\m{=}\m{(}\m{G}\m{`}\m{(}\m{f}\m{\restriction}\m{y}\m{)}\m{)}\m{)}%
\m{\}}
\endm
\setbox\startprefix=\hbox{\tt \ \ tfr.2\ \$e\ }
\setbox\contprefix=\hbox{\tt \ \ \ \ \ \ \ \ \ \ \ }
\startm
\m{\vdash}\m{F}\m{=}\m{\bigcup}\m{A}
\endm
\setbox\startprefix=\hbox{\tt \ \ tfr1\ \$p\ }
\setbox\contprefix=\hbox{\tt \ \ \ \ \ \ \ \ \ \ }
\startm
\m{\vdash}\m{F}\m{{\rm Fn}}\m{{\rm On}}
\endm
\setbox\startprefix=\hbox{\tt \ \ tfr2\ \$p\ }
\setbox\contprefix=\hbox{\tt \ \ \ \ \ \ \ \ \ \ }
\startm
\m{\vdash}\m{(}\m{z}\m{\in}\m{{\rm On}}\m{\rightarrow}\m{(}\m{F}\m{`}\m{z}%
\m{)}\m{=}\m{(}\m{G}\m{`}\m{(}\m{F}\m{\restriction}\m{z}\m{)}\m{)}\m{)}
\endm
\setbox\startprefix=\hbox{\tt \ \ tfr3\ \$p\ }
\setbox\contprefix=\hbox{\tt \ \ \ \ \ \ \ \ \ \ }
\startm
\m{\vdash}\m{(}\m{(}\m{B}\m{{\rm Fn}}\m{{\rm On}}\m{\wedge}\m{\forall}\m{x}\m{%
\in}\m{{\rm On}}\m{(}\m{B}\m{`}\m{x}\m{)}\m{=}\m{(}\m{G}\m{`}\m{(}\m{B}\m{%
\restriction}\m{x}\m{)}\m{)}\m{)}\m{\rightarrow}\m{B}\m{=}\m{F}\m{)}
\endm
\vskip 1ex

\noindent The existence of omega (the class of natural numbers).\index{natural
number}\index{omega ($\omega$)}\index{Axiom of Infinity}  Axiom 7 of Takeuti
and Zaring, p.~43.  (This is the only theorem in this section requiring the
Axiom of Infinity.)

\vskip 0.5ex
\setbox\startprefix=\hbox{\tt \
\ omex\ \$p\ }
\setbox\contprefix=\hbox{\tt \ \ \ \ \ \ \ \ \ \ }
\startm
\m{\vdash}\m{\omega}\m{\in}\m{{\rm V}}
\endm
%\vskip 2ex


\section{Axioms for Real and Complex Numbers}\label{real}
\index{real and complex numbers!axioms for}

This section presents the axioms for real and complex numbers, along
with some commentary about them.  Analysis
textbooks implicitly or explicitly use these axioms or their equivalents
as their starting point.  In the database \texttt{set.mm}, we define real
and complex numbers as (rather complicated) specific sets and derive these
axioms as {\em theorems} from the axioms of ZF set theory, using a method
called Dedekind cuts.  We omit the details of this construction, which you can
follow if you wish using the \texttt{set.mm} database in conjunction with the
textbooks referenced therein.

Once we prove those theorems, we then restate these proven theorems as axioms.
This lets us easily identify which axioms are needed for a particular complex number proof, without the obfuscation of the set theory used to derive them.
As a result,
the construction is actually unimportant other
than to show that sets exist that satisfy the axioms, and thus that the axioms
are consistent if ZF set theory is consistent.  When working with real numbers
you can think of them as being the actual sets resulting
from the construction (for definiteness), or you can
think of them as otherwise unspecified sets that happen to satisfy the axioms.
The derivation is not easy, but the fact that it works is quite remarkable
and lends support to the idea that ZFC set theory is all we need to
provide a foundation for essentially all of mathematics.

\needspace{3\baselineskip}
\subsection{The Axioms for Real and Complex Numbers Themselves}\label{realactual}

For the axioms we are given (or postulate) 8 classes:  $\mathbb{C}$ (the
set of complex numbers), $\mathbb{R}$ (the set of real numbers, a subset
of $\mathbb{C}$), $0$ (zero), $1$ (one), $i$ (square root of
$-1$), $+$ (plus), $\cdot$ (times), and
$<_{\mathbb{R}}$ (less than for just the real numbers).
Subtraction and division are defined terms and are not part of the
axioms; for their definitions see \texttt{set.mm}.

Note that the notation $(A+B)$ (and similarly $(A\cdot B)$) specifies a class
called an {\em operation},\index{operation} and is the function value of the
class $+$ at ordered pair $\langle A,B \rangle$.  An operation is defined by
statement \texttt{df-opr} on p.~\pageref{dfopr}.
The notation $A <_{\mathbb{R}} B$ specifies a
wff called a {\em binary relation}\index{binary relation} and means $\langle A,B \rangle \in \,<_{\mathbb{R}}$, as defined by statement \texttt{df-br} on p.~\pageref{dfbr}.

Our set of 8 given classes is assumed to satisfy the following 22 axioms
(in the axioms listed below, $<$ really means $<_{\mathbb{R}}$).

\vskip 2ex

\noindent 1. The real numbers are a subset of the complex numbers.

%\vskip 0.5ex
\setbox\startprefix=\hbox{\tt \ \ ax-resscn\ \$p\ }
\setbox\contprefix=\hbox{\tt \ \ \ \ \ \ \ \ \ \ \ \ \ \ }
\startm
\m{\vdash}\m{\mathbb{R}}\m{\subseteq}\m{\mathbb{C}}
\endm
%\vskip 1ex

\noindent 2. One is a complex number.

%\vskip 0.5ex
\setbox\startprefix=\hbox{\tt \ \ ax-1cn\ \$p\ }
\setbox\contprefix=\hbox{\tt \ \ \ \ \ \ \ \ \ \ \ }
\startm
\m{\vdash}\m{1}\m{\in}\m{\mathbb{C}}
\endm
%\vskip 1ex

\noindent 3. The imaginary number $i$ is a complex number.

%\vskip 0.5ex
\setbox\startprefix=\hbox{\tt \ \ ax-icn\ \$p\ }
\setbox\contprefix=\hbox{\tt \ \ \ \ \ \ \ \ \ \ \ }
\startm
\m{\vdash}\m{i}\m{\in}\m{\mathbb{C}}
\endm
%\vskip 1ex

\noindent 4. Complex numbers are closed under addition.

%\vskip 0.5ex
\setbox\startprefix=\hbox{\tt \ \ ax-addcl\ \$p\ }
\setbox\contprefix=\hbox{\tt \ \ \ \ \ \ \ \ \ \ \ \ \ }
\startm
\m{\vdash}\m{(}\m{(}\m{A}\m{\in}\m{\mathbb{C}}\m{\wedge}\m{B}\m{\in}\m{\mathbb{C}}%
\m{)}\m{\rightarrow}\m{(}\m{A}\m{+}\m{B}\m{)}\m{\in}\m{\mathbb{C}}\m{)}
\endm
%\vskip 1ex

\noindent 5. Real numbers are closed under addition.

%\vskip 0.5ex
\setbox\startprefix=\hbox{\tt \ \ ax-addrcl\ \$p\ }
\setbox\contprefix=\hbox{\tt \ \ \ \ \ \ \ \ \ \ \ \ \ \ }
\startm
\m{\vdash}\m{(}\m{(}\m{A}\m{\in}\m{\mathbb{R}}\m{\wedge}\m{B}\m{\in}\m{\mathbb{R}}%
\m{)}\m{\rightarrow}\m{(}\m{A}\m{+}\m{B}\m{)}\m{\in}\m{\mathbb{R}}\m{)}
\endm
%\vskip 1ex

\noindent 6. Complex numbers are closed under multiplication.

%\vskip 0.5ex
\setbox\startprefix=\hbox{\tt \ \ ax-mulcl\ \$p\ }
\setbox\contprefix=\hbox{\tt \ \ \ \ \ \ \ \ \ \ \ \ \ }
\startm
\m{\vdash}\m{(}\m{(}\m{A}\m{\in}\m{\mathbb{C}}\m{\wedge}\m{B}\m{\in}\m{\mathbb{C}}%
\m{)}\m{\rightarrow}\m{(}\m{A}\m{\cdot}\m{B}\m{)}\m{\in}\m{\mathbb{C}}\m{)}
\endm
%\vskip 1ex

\noindent 7. Real numbers are closed under multiplication.

%\vskip 0.5ex
\setbox\startprefix=\hbox{\tt \ \ ax-mulrcl\ \$p\ }
\setbox\contprefix=\hbox{\tt \ \ \ \ \ \ \ \ \ \ \ \ \ \ }
\startm
\m{\vdash}\m{(}\m{(}\m{A}\m{\in}\m{\mathbb{R}}\m{\wedge}\m{B}\m{\in}\m{\mathbb{R}}%
\m{)}\m{\rightarrow}\m{(}\m{A}\m{\cdot}\m{B}\m{)}\m{\in}\m{\mathbb{R}}\m{)}
\endm
%\vskip 1ex

\noindent 8. Multiplication of complex numbers is commutative.

%\vskip 0.5ex
\setbox\startprefix=\hbox{\tt \ \ ax-mulcom\ \$p\ }
\setbox\contprefix=\hbox{\tt \ \ \ \ \ \ \ \ \ \ \ \ \ \ }
\startm
\m{\vdash}\m{(}\m{(}\m{A}\m{\in}\m{\mathbb{C}}\m{\wedge}\m{B}\m{\in}\m{\mathbb{C}}%
\m{)}\m{\rightarrow}\m{(}\m{A}\m{\cdot}\m{B}\m{)}\m{=}\m{(}\m{B}\m{\cdot}\m{A}%
\m{)}\m{)}
\endm
%\vskip 1ex

\noindent 9. Addition of complex numbers is associative.

%\vskip 0.5ex
\setbox\startprefix=\hbox{\tt \ \ ax-addass\ \$p\ }
\setbox\contprefix=\hbox{\tt \ \ \ \ \ \ \ \ \ \ \ \ \ \ }
\startm
\m{\vdash}\m{(}\m{(}\m{A}\m{\in}\m{\mathbb{C}}\m{\wedge}\m{B}\m{\in}\m{\mathbb{C}}%
\m{\wedge}\m{C}\m{\in}\m{\mathbb{C}}\m{)}\m{\rightarrow}\m{(}\m{(}\m{A}\m{+}%
\m{B}\m{)}\m{+}\m{C}\m{)}\m{=}\m{(}\m{A}\m{+}\m{(}\m{B}\m{+}\m{C}\m{)}\m{)}%
\m{)}
\endm
%\vskip 1ex

\noindent 10. Multiplication of complex numbers is associative.

%\vskip 0.5ex
\setbox\startprefix=\hbox{\tt \ \ ax-mulass\ \$p\ }
\setbox\contprefix=\hbox{\tt \ \ \ \ \ \ \ \ \ \ \ \ \ \ }
\startm
\m{\vdash}\m{(}\m{(}\m{A}\m{\in}\m{\mathbb{C}}\m{\wedge}\m{B}\m{\in}\m{\mathbb{C}}%
\m{\wedge}\m{C}\m{\in}\m{\mathbb{C}}\m{)}\m{\rightarrow}\m{(}\m{(}\m{A}\m{\cdot}%
\m{B}\m{)}\m{\cdot}\m{C}\m{)}\m{=}\m{(}\m{A}\m{\cdot}\m{(}\m{B}\m{\cdot}\m{C}%
\m{)}\m{)}\m{)}
\endm
%\vskip 1ex

\noindent 11. Multiplication distributes over addition for complex numbers.

%\vskip 0.5ex
\setbox\startprefix=\hbox{\tt \ \ ax-distr\ \$p\ }
\setbox\contprefix=\hbox{\tt \ \ \ \ \ \ \ \ \ \ \ \ \ }
\startm
\m{\vdash}\m{(}\m{(}\m{A}\m{\in}\m{\mathbb{C}}\m{\wedge}\m{B}\m{\in}\m{\mathbb{C}}%
\m{\wedge}\m{C}\m{\in}\m{\mathbb{C}}\m{)}\m{\rightarrow}\m{(}\m{A}\m{\cdot}\m{(}%
\m{B}\m{+}\m{C}\m{)}\m{)}\m{=}\m{(}\m{(}\m{A}\m{\cdot}\m{B}\m{)}\m{+}\m{(}%
\m{A}\m{\cdot}\m{C}\m{)}\m{)}\m{)}
\endm
%\vskip 1ex

\noindent 12. The square of $i$ equals $-1$ (expressed as $i$-squared plus 1 is
0).

%\vskip 0.5ex
\setbox\startprefix=\hbox{\tt \ \ ax-i2m1\ \$p\ }
\setbox\contprefix=\hbox{\tt \ \ \ \ \ \ \ \ \ \ \ \ }
\startm
\m{\vdash}\m{(}\m{(}\m{i}\m{\cdot}\m{i}\m{)}\m{+}\m{1}\m{)}\m{=}\m{0}
\endm
%\vskip 1ex

\noindent 13. One and zero are distinct.

%\vskip 0.5ex
\setbox\startprefix=\hbox{\tt \ \ ax-1ne0\ \$p\ }
\setbox\contprefix=\hbox{\tt \ \ \ \ \ \ \ \ \ \ \ \ }
\startm
\m{\vdash}\m{1}\m{\ne}\m{0}
\endm
%\vskip 1ex

\noindent 14. One is an identity element for real multiplication.

%\vskip 0.5ex
\setbox\startprefix=\hbox{\tt \ \ ax-1rid\ \$p\ }
\setbox\contprefix=\hbox{\tt \ \ \ \ \ \ \ \ \ \ \ }
\startm
\m{\vdash}\m{(}\m{A}\m{\in}\m{\mathbb{R}}\m{\rightarrow}\m{(}\m{A}\m{\cdot}\m{1}%
\m{)}\m{=}\m{A}\m{)}
\endm
%\vskip 1ex

\noindent 15. Every real number has a negative.

%\vskip 0.5ex
\setbox\startprefix=\hbox{\tt \ \ ax-rnegex\ \$p\ }
\setbox\contprefix=\hbox{\tt \ \ \ \ \ \ \ \ \ \ \ \ \ \ }
\startm
\m{\vdash}\m{(}\m{A}\m{\in}\m{\mathbb{R}}\m{\rightarrow}\m{\exists}\m{x}\m{\in}%
\m{\mathbb{R}}\m{(}\m{A}\m{+}\m{x}\m{)}\m{=}\m{0}\m{)}
\endm
%\vskip 1ex

\noindent 16. Every nonzero real number has a reciprocal.

%\vskip 0.5ex
\setbox\startprefix=\hbox{\tt \ \ ax-rrecex\ \$p\ }
\setbox\contprefix=\hbox{\tt \ \ \ \ \ \ \ \ \ \ \ \ \ \ }
\startm
\m{\vdash}\m{(}\m{A}\m{\in}\m{\mathbb{R}}\m{\rightarrow}\m{(}\m{A}\m{\ne}\m{0}%
\m{\rightarrow}\m{\exists}\m{x}\m{\in}\m{\mathbb{R}}\m{(}\m{A}\m{\cdot}%
\m{x}\m{)}\m{=}\m{1}\m{)}\m{)}
\endm
%\vskip 1ex

\noindent 17. A complex number can be expressed in terms of two reals.

%\vskip 0.5ex
\setbox\startprefix=\hbox{\tt \ \ ax-cnre\ \$p\ }
\setbox\contprefix=\hbox{\tt \ \ \ \ \ \ \ \ \ \ \ \ }
\startm
\m{\vdash}\m{(}\m{A}\m{\in}\m{\mathbb{C}}\m{\rightarrow}\m{\exists}\m{x}\m{\in}%
\m{\mathbb{R}}\m{\exists}\m{y}\m{\in}\m{\mathbb{R}}\m{A}\m{=}\m{(}\m{x}\m{+}\m{(}%
\m{y}\m{\cdot}\m{i}\m{)}\m{)}\m{)}
\endm
%\vskip 1ex

\noindent 18. Ordering on reals satisfies strict trichotomy.

%\vskip 0.5ex
\setbox\startprefix=\hbox{\tt \ \ ax-pre-lttri\ \$p\ }
\setbox\contprefix=\hbox{\tt \ \ \ \ \ \ \ \ \ \ \ \ \ }
\startm
\m{\vdash}\m{(}\m{(}\m{A}\m{\in}\m{\mathbb{R}}\m{\wedge}\m{B}\m{\in}\m{\mathbb{R}}%
\m{)}\m{\rightarrow}\m{(}\m{A}\m{<}\m{B}\m{\leftrightarrow}\m{\lnot}\m{(}\m{A}%
\m{=}\m{B}\m{\vee}\m{B}\m{<}\m{A}\m{)}\m{)}\m{)}
\endm
%\vskip 1ex

\noindent 19. Ordering on reals is transitive.

%\vskip 0.5ex
\setbox\startprefix=\hbox{\tt \ \ ax-pre-lttrn\ \$p\ }
\setbox\contprefix=\hbox{\tt \ \ \ \ \ \ \ \ \ \ \ \ \ }
\startm
\m{\vdash}\m{(}\m{(}\m{A}\m{\in}\m{\mathbb{R}}\m{\wedge}\m{B}\m{\in}\m{\mathbb{R}}%
\m{\wedge}\m{C}\m{\in}\m{\mathbb{R}}\m{)}\m{\rightarrow}\m{(}\m{(}\m{A}\m{<}%
\m{B}\m{\wedge}\m{B}\m{<}\m{C}\m{)}\m{\rightarrow}\m{A}\m{<}\m{C}\m{)}\m{)}
\endm
%\vskip 1ex

\noindent 20. Ordering on reals is preserved after addition to both sides.

%\vskip 0.5ex
\setbox\startprefix=\hbox{\tt \ \ ax-pre-ltadd\ \$p\ }
\setbox\contprefix=\hbox{\tt \ \ \ \ \ \ \ \ \ \ \ \ \ }
\startm
\m{\vdash}\m{(}\m{(}\m{A}\m{\in}\m{\mathbb{R}}\m{\wedge}\m{B}\m{\in}\m{\mathbb{R}}%
\m{\wedge}\m{C}\m{\in}\m{\mathbb{R}}\m{)}\m{\rightarrow}\m{(}\m{A}\m{<}\m{B}\m{%
\rightarrow}\m{(}\m{C}\m{+}\m{A}\m{)}\m{<}\m{(}\m{C}\m{+}\m{B}\m{)}\m{)}\m{)}
\endm
%\vskip 1ex

\noindent 21. The product of two positive reals is positive.

%\vskip 0.5ex
\setbox\startprefix=\hbox{\tt \ \ ax-pre-mulgt0\ \$p\ }
\setbox\contprefix=\hbox{\tt \ \ \ \ \ \ \ \ \ \ \ \ \ \ }
\startm
\m{\vdash}\m{(}\m{(}\m{A}\m{\in}\m{\mathbb{R}}\m{\wedge}\m{B}\m{\in}\m{\mathbb{R}}%
\m{)}\m{\rightarrow}\m{(}\m{(}\m{0}\m{<}\m{A}\m{\wedge}\m{0}%
\m{<}\m{B}\m{)}\m{\rightarrow}\m{0}\m{<}\m{(}\m{A}\m{\cdot}\m{B}\m{)}%
\m{)}\m{)}
\endm
%\vskip 1ex

\noindent 22. A non-empty, bounded-above set of reals has a supremum.

%\vskip 0.5ex
\setbox\startprefix=\hbox{\tt \ \ ax-pre-sup\ \$p\ }
\setbox\contprefix=\hbox{\tt \ \ \ \ \ \ \ \ \ \ \ }
\startm
\m{\vdash}\m{(}\m{(}\m{A}\m{\subseteq}\m{\mathbb{R}}\m{\wedge}\m{A}\m{\ne}\m{%
\varnothing}\m{\wedge}\m{\exists}\m{x}\m{\in}\m{\mathbb{R}}\m{\forall}\m{y}\m{%
\in}\m{A}\m{\,y}\m{<}\m{x}\m{)}\m{\rightarrow}\m{\exists}\m{x}\m{\in}\m{%
\mathbb{R}}\m{(}\m{\forall}\m{y}\m{\in}\m{A}\m{\lnot}\m{x}\m{<}\m{y}\m{\wedge}\m{%
\forall}\m{y}\m{\in}\m{\mathbb{R}}\m{(}\m{y}\m{<}\m{x}\m{\rightarrow}\m{\exists}%
\m{z}\m{\in}\m{A}\m{\,y}\m{<}\m{z}\m{)}\m{)}\m{)}
\endm

% NOTE: The \m{...} expressions above could be represented as
% $ \vdash ( ( A \subseteq \mathbb{R} \wedge A \ne \varnothing \wedge \exists x \in \mathbb{R} \forall y \in A \,y < x ) \rightarrow \exists x \in \mathbb{R} ( \forall y \in A \lnot x < y \wedge \forall y \in \mathbb{R} ( y < x \rightarrow \exists z \in A \,y < z ) ) ) $

\vskip 2ex

This completes the set of axioms for real and complex numbers.  You may
wish to look at how subtraction, division, and decimal numbers
are defined in \texttt{set.mm}, and for fun look at the proof of $2+
2 = 4$ (theorem \texttt{2p2e4} in \texttt{set.mm})
as discussed in section \ref{2p2e4}.

In \texttt{set.mm} we define the non-negative integers $\mathbb{N}$, the integers
$\mathbb{Z}$, and the rationals $\mathbb{Q}$ as subsets of $\mathbb{R}$.  This leads
to the nice inclusion $\mathbb{N} \subseteq \mathbb{Z} \subseteq \mathbb{Q} \subseteq
\mathbb{R} \subseteq \mathbb{C}$, giving us a uniform framework in which, for
example, a property such as commutativity of complex number addition
automatically applies to integers.  The natural numbers $\mathbb{N}$
are different from the set $\omega$ we defined earlier, but both satisfy
Peano's postulates.

\subsection{Complex Number Axioms in Analysis Texts}

Most analysis texts construct complex numbers as ordered pairs of reals,
leading to construction-dependent properties that satisfy these axioms
but are not stated in their pure form.  (This is also done in
\texttt{set.mm} but our axioms are extracted from that construction.)
Other texts will simply state that $\mathbb{R}$ is a ``complete ordered
subfield of $\mathbb{C}$,'' leading to redundant axioms when this phrase
is completely expanded out.  In fact I have not seen a text with the
axioms in the explicit form above.
None of these axioms is unique individually, but this carefully worked out
collection of axioms is the result of years of work
by the Metamath community.

\subsection{Eliminating Unnecessary Complex Number Axioms}

We once had more axioms for real and complex numbers, but over years of time
we (the Metamath community)
have found ways to eliminate them (by proving them from other axioms)
or weaken them (by making weaker claims without reducing what
can be proved).
In particular, here are statements that used to be complex number
axioms but have since been formally proven (with Metamath) to be redundant:

\begin{itemize}
\item
  $\mathbb{C} \in V$.
  At one time this was listed as a ``complex number axiom.''
  However, this is not properly speaking a complex number axiom,
  and in any case its proof uses axioms of set theory.
  Proven redundant by Mario Carneiro\index{Carneiro, Mario} on
  17-Nov-2014 (see \texttt{axcnex}).
\item
  $((A \in \mathbb{C} \land B \in \mathbb{C}$) $\rightarrow$
  $(A + B) = (B + A))$.
  Proved redundant by Eric Schmidt\index{Schmidt, Eric} on 19-Jun-2012,
  and formalized by Scott Fenton\index{Fenton, Scott} on 3-Jan-2013
  (see \texttt{addcom}).
\item
  $(A \in \mathbb{C} \rightarrow (A + 0) = A)$.
  Proved redundant by Eric Schmidt on 19-Jun-2012,
  and formalized by Scott Fenton on 3-Jan-2013
  (see \texttt{addid1}).
\item
  $(A \in \mathbb{C} \rightarrow \exists x \in \mathbb{C} (A + x) = 0)$.
  Proved redundant by Eric Schmidt and formalized on 21-May-2007
  (see \texttt{cnegex}).
\item
  $((A \in \mathbb{C} \land A \ne 0) \rightarrow \exists x \in \mathbb{C} (A \cdot x) = 1)$.
  Proved redundant by Eric Schmidt and formalized on 22-May-2007
  (see \texttt{recex}).
\item
  $0 \in \mathbb{R}$.
  Proved redundant by Eric Schmidt on 19-Feb-2005 and formalized 21-May-2007
  (see \texttt{0re}).
\end{itemize}

We could eliminate 0 as an axiomatic object by defining it as
$( ( i \cdot i ) + 1 )$
and replacing it with this expression throughout the axioms. If this
is done, axiom ax-i2m1 becomes redundant. However, the remaining axioms
would become longer and less intuitive.

Eric Schmidt's paper analyzing this axiom system \cite{Schmidt}
presented a proof that these remaining axioms,
with the possible exception of ax-mulcom, are independent of the others.
It is currently an open question if ax-mulcom is independent of the others.

\section{Two Plus Two Equals Four}\label{2p2e4}

Here is a proof that $2 + 2 = 4$, as proven in the theorem \texttt{2p2e4}
in the database \texttt{set.mm}.
This is a useful demonstration of what a Metamath proof can look like.
This proof may have more steps than you're used to, but each step is rigorously
proven all the way back to the axioms of logic and set theory.
This display was originally generated by the Metamath program
as an {\sc HTML} file.

In the table showing the proof ``Step'' is the sequential step number,
while its associated ``Expression'' is an expression that we have proved.
``Ref'' is the name of a theorem or axiom that justifies that expression,
and ``Hyp'' refers to previous steps (if any) that the theorem or axiom
needs so that we can use it.  Expressions are indented further than
the expressions that depend on them to show their interdependencies.

\begin{table}[!htbp]
\caption{Two plus two equals four}
\begin{tabular}{lllll}
\textbf{Step} & \textbf{Hyp} & \textbf{Ref} & \textbf{Expression} & \\
1  &       & df-2    & $ \; \; \vdash 2 = 1 + 1$  & \\
2  & 1     & oveq2i  & $ \; \vdash (2 + 2) = (2 + (1 + 1))$ & \\
3  &       & df-4    & $ \; \; \vdash 4 = (3 + 1)$ & \\
4  &       & df-3    & $ \; \; \; \vdash 3 = (2 + 1)$ & \\
5  & 4     & oveq1i  & $ \; \; \vdash (3 + 1) = ((2 + 1) + 1)$ & \\
6  &       & 2cn     & $ \; \; \; \vdash 2 \in \mathbb{C}$ & \\
7  &       & ax-1cn  & $ \; \; \; \vdash 1 \in \mathbb{C}$ & \\
8  & 6,7,7 & addassi & $ \; \; \vdash ((2 + 1) + 1) = (2 + (1 + 1))$ & \\
9  & 3,5,8 & 3eqtri  & $ \; \vdash 4 = (2 + (1 + 1))$ & \\
10 & 2,9   & eqtr4i  & $ \vdash (2 + 2) = 4$ & \\
\end{tabular}
\end{table}

Step 1 says that we can assert that $2 = 1 + 1$ because it is
justified by \texttt{df-2}.
What is \texttt{df-2}?
It is simply the definition of $2$, which in our system is defined as being
equal to $1 + 1$.  This shows how we can use definitions in proofs.

Look at Step 2 of the proof. In the Ref column, we see that it references
a previously proved theorem, \texttt{oveq2i}.
It turns out that
theorem \texttt{oveq2i} requires a
hypothesis, and in the Hyp column of Step 2 we indicate that Step 1 will
satisfy (match) this hypothesis.
If we looked at \texttt{oveq2i}
we would find that it proves that given some hypothesis
$A = B$, we can prove that $( C F A ) = ( C F B )$.
If we use \texttt{oveq2i} and apply step 1's result as the hypothesis,
that will mean that $A = 2$ and $B = ( 1 + 1 )$ within this use of
\texttt{oveq2i}.
We are free to select any value of $C$ and $F$ (subject to syntax constraints),
so we are free to select $C = 2$ and $F = +$,
producing our desired result,
$ (2 + 2) = (2 + (1 + 1))$.

Step 2 is an example of substitution.
In the end, every step in every proof uses only this one substitution rule.
All the rules of logic, and all the axioms, are expressed so that
they can be used via this one substitution rule.
So once you master substitution, you can master every Metamath proof,
no exceptions.

Each step is clear and can be immediately checked.
In the {\sc HTML} display you can even click on each reference to see why it is
justified, making it easy to see why the proof works.

\section{Deduction}\label{deduction}

Strictly speaking,
a deduction (also called an inference) is a kind of statement that needs
some hypotheses to be true in order for its conclusion to be true.
A theorem, on the other hand, has no hypotheses.
Informally we often call both of them theorems, but in this section we
will stick to the strict definitions.

It sometimes happens that we have proved a deduction of the form
$\varphi \Rightarrow \psi$\index{$\Rightarrow$}
(given hypothesis $\varphi$ we can prove $\psi$)
and we want to then prove a theorem of the form
$\varphi \rightarrow \psi$.

Converting a deduction (which uses a hypothesis) into a theorem
(which does not) is not as simple as you might think.
The deduction says, ``if we can prove $\varphi$ then we can prove $\psi$,''
which is in some sense weaker than saying
``$\varphi$ implies $\psi$.''
There is no axiom of logic that permits us to directly obtain the theorem
given the deduction.\footnote{
The conversion of a deduction to a theorem does not even hold in general
for quantum propositional calculus,
which is a weak subset of classical propositional calculus.
It has been shown that adding the Standard Deduction Theorem (discussed below)
to quantum propositional calculus turns it into classical
propositional calculus!
}

This is in contrast to going the other way.
If we have the theorem ($\varphi \rightarrow \psi$),
it is easy to recover the deduction
($\varphi \Rightarrow \psi$)
using modus ponens\index{modus ponens}
(\texttt{ax-mp}; see section \ref{axmp}).

In the following subsections we first discuss the standard deduction theorem
(the traditional but awkward way to convert deductions into theorems) and
the weak deduction theorem (a limited version of the standard deduction
theorem that is easier to use and was once widely used in
\texttt{set.mm}\index{set theory database (\texttt{set.mm})}).
In section \ref{deductionstyle} we discuss
deduction style, the newer approach we now recommend in most cases.
Deduction style uses ``deduction form,'' a form that
prefixes each hypothesis (other than definitions) and the
conclusion with a universal antecedent (``$\varphi \rightarrow$'').
Deduction style is widely used in \texttt{set.mm},
so it is useful to understand it and \textit{why} it is widely used.
Section \ref{naturaldeduction}
briefly discusses our approach for using natural deduction
within \texttt{set.mm},
as that approach is deeply related to deduction style.
We conclude with a summary of the strengths of
our approach, which we believe are compelling.

\subsection{The Standard Deduction Theorem}\label{standarddeductiontheorem}

It is possible to make use of information
contained in the deduction or its proof to assist us with the proof of
the related theorem.
In traditional logic books, there is a metatheorem called the
Deduction Theorem\index{Deduction Theorem}\index{Standard Deduction Theorem},
discovered independently by Herbrand and Tarski around 1930.
The Deduction Theorem, which we often call the Standard Deduction Theorem,
provides an algorithm for constructing a proof of a theorem from the
proof of its corresponding deduction. See, for example,
\cite[p.~56]{Margaris}\index{Margaris, Angelo}.
To construct a proof for a theorem, the
algorithm looks at each step in the proof of the original deduction and
rewrites the step with several steps wherein the hypothesis is eliminated
and becomes an antecedent.

In ordinary mathematics, no one actually carries out the algorithm,
because (in its most basic form) it involves an exponential explosion of
the number of proof steps as more hypotheses are eliminated. Instead,
the Standard Deduction Theorem is invoked simply to claim that it can
be done in principle, without actually doing it.
What's more, the algorithm is not as simple as it might first appear
when applying it rigorously.
There is a subtle restriction on the Standard Deduction Theorem
that must be taken into account involving the axiom of generalization
when working with predicate calculus (see the literature for more detail).

One of the goals of Metamath is to let you plainly see, with as few
underlying concepts as possible, how mathematics can be derived directly
from the axioms, and not indirectly according to some hidden rules
buried inside a program or understood only by logicians. If we added
the Standard Deduction Theorem to the language and proof verifier,
that would greatly complicate both and largely defeat Metamath's goal
of simplicity. In principle, we could show direct proofs by expanding
out the proof steps generated by the algorithm of the Standard Deduction
Theorem, but that is not feasible in practice because the number of proof
steps quickly becomes huge, even astronomical.
Since the algorithm of the Standard Deduction Theorem is driven by the proof,
we would have to go through that proof
all over again---starting from axioms---in order to obtain the theorem form.
In terms of proof length, there would be no savings over just
proving the theorem directly instead of first proving the deduction form.

\subsection{Weak Deduction Theorem}\label{weakdeductiontheorem}

We have developed
a more efficient method for proving a theorem from a deduction
that can be used instead of the Standard Deduction Theorem
in many (but not all) cases.
We call this more efficient method the
Weak Deduction Theorem\index{Weak Deduction Theorem}.\footnote{
There is also an unrelated ``Weak Deduction Theorem''
in the field of relevance logic, so to avoid confusion we could call
ours the ``Weak Deduction Theorem for Classical Logic.''}
Unlike the Standard Deduction Theorem, the Weak Deduction Theorem produces the
theorem directly from a special substitution instance of the deduction,
using a small, fixed number of steps roughly proportional to the length
of the final theorem.

If you come to a proof referencing the Weak Deduction Theorem
\texttt{dedth} (or one of its variants \texttt{dedthxx}),
here is how to follow the proof without getting into the details:
just click on the theorem referenced in the step
just before the reference to \texttt{dedth} and ignore everything else.
Theorem \texttt{dedth} simply turns a hypothesis into an antecedent
(i.e. the hypothesis followed by $\rightarrow$
is placed in front of the assertion, and the hypothesis
itself is eliminated) given certain conditions.

The Weak Deduction Theorem
eliminates a hypothesis $\varphi$, making it become an antecedent.
It does this by proving an expression
$ \varphi \rightarrow \psi $ given two hypotheses:
(1)
$ ( A = {\rm if} ( \varphi , A , B ) \rightarrow ( \varphi \leftrightarrow \chi ) ) $
and
(2) $\chi$.
Note that it requires that a proof exists for $\varphi$ when the class variable
$A$ is replaced with a specific class $B$. The hypothesis $\chi$
should be assigned to the inference.
You can see the details of the proof of the Weak Deduction Theorem
in theorem \texttt{dedth}.

The Weak Deduction Theorem
is probably easier to understand by studying proofs that make use of it.
For example, let's look at the proof of \texttt{renegcl}, which proves that
$ \vdash ( A \in \mathbb{R} \rightarrow - A \in \mathbb{R} )$:

\needspace{4\baselineskip}
\begin{longtabu} {l l l X}
\textbf{Step} & \textbf{Hyp} & \textbf{Ref} & \textbf{Expression} \\
  1 &  & negeq &
  $\vdash$ $($ $A$ $=$ ${\rm if}$ $($ $A$ $\in$ $\mathbb{R}$ $,$ $A$ $,$ $1$ $)$ $\rightarrow$
  $\textrm{-}$ $A$ $=$ $\textrm{-}$ ${\rm if}$ $($ $A$ $\in$ $\mathbb{R}$
  $,$ $A$ $,$ $1$ $)$ $)$ \\
 2 & 1 & eleq1d &
    $\vdash$ $($ $A$ $=$ ${\rm if}$ $($ $A$ $\in$ $\mathbb{R}$ $,$ $A$ $,$ $1$ $)$ $\rightarrow$ $($
    $\textrm{-}$ $A$ $\in$ $\mathbb{R}$ $\leftrightarrow$
    $\textrm{-}$ ${\rm if}$ $($ $A$ $\in$ $\mathbb{R}$ $,$ $A$ $,$ $1$ $)$ $\in$
    $\mathbb{R}$ $)$ $)$ \\
 3 &  & 1re & $\vdash 1 \in \mathbb{R}$ \\
 4 & 3 & elimel &
   $\vdash {\rm if} ( A \in \mathbb{R} , A , 1 ) \in \mathbb{R}$ \\
 5 & 4 & renegcli &
   $\vdash \textrm{-} {\rm if} ( A \in \mathbb{R} , A , 1 ) \in \mathbb{R}$ \\
 6 & 2,5 & dedth &
   $\vdash ( A \in \mathbb{R} \rightarrow \textrm{-} A \in \mathbb{R}$ ) \\
\end{longtabu}

The somewhat strange-looking steps in \texttt{renegcl} before step 5 are
technical stuff that makes this magic work, and they can be ignored
for a quick overview of the proof. To continue following the ``important''
part of the proof of \texttt{renegcl},
you can look at the reference to \texttt{renegcli} at step 5.

That said, let's briefly look at how
\texttt{renegcl} uses the
Weak Deduction Theorem (\texttt{dedth}) to do its job,
in case you want to do something similar or want understand it more deeply.
Let's work backwards in the proof of \texttt{renegcl}.
Step 6 applies \texttt{dedth} to produce our goal result
$ \vdash ( A \in \mathbb{R} \rightarrow\, - A \in \mathbb{R} )$.
This requires on the one hand the (substituted) deduction
\texttt{renegcli} in step 5.
By itself \texttt{renegcli} proves the deduction
$ \vdash A \in \mathbb{R} \Rightarrow\, \vdash - A \in \mathbb{R}$;
this is the deduction form we are trying to turn into theorem form,
and thus
\texttt{renegcli} has a separate hypothesis that must be fulfilled.
To fulfill the hypothesis of the invocation of
\texttt{renegcli} in step 5, it is eventually
reduced to the already proven theorem $1 \in \mathbb{R}$ in step 3.
Step 4 connects steps 3 and 5; step 4 invokes
\texttt{elimel}, a special case of \texttt{elimhyp} that eliminates
a membership hypothesis for the weak deduction theorem.
On the other hand, the equivalence of the conclusion of
\texttt{renegcl}
$( - A \in \mathbb{R} )$ and the substituted conclusion of
\texttt{renegcli} must be proven, which is done in steps 2 and 1.

The weak deduction theorem has limitations.
In particular, we must be able to prove a special case of the deduction's
hypothesis as a stand-alone theorem.
For example, we used $1 \in \mathbb{R}$ in step 3 of \texttt{renegcl}.

We used to use the weak deduction theorem
extensively within \texttt{set.mm}.
However, we now recommend applying ``deduction style''
instead in most cases, as deduction style is
often an easier and clearer approach.
Therefore, we will now describe deduction style.

\subsection{Deduction Style}\label{deductionstyle}

We now prefer to write assertions in ``deduction form''
instead of writing a proof that would require use of the standard or
weak deduction theorem.
We call this appraoch
``deduction style.''\index{deduction style}

It will be easier to explain this by first defining some terms:

\begin{itemize}
\item \textbf{closed form}\index{closed form}\index{forms!closed}:
A kind of assertion (theorem) with no hypotheses.
Typically its label has no special suffix.
An example is \texttt{unss}, which states:
$\vdash ( ( A \subseteq C \wedge B \subseteq C ) \leftrightarrow ( A \cup B )
\subseteq C )\label{eq:unss}$
\item \textbf{deduction form}\index{deduction form}\index{forms!deduction}:
A kind of assertion with one or more hypotheses
where the conclusion is an implication with
a wff variable as the antecedent (usually $\varphi$), and every hypothesis
(\$e statement)
is either (1) an implication with the same antecedent as the conclusion or
(2) a definition.
A definition
can be for a class variable (this is a class variable followed by ``='')
or a wff variable (this is a wff variable followed by $\leftrightarrow$);
class variable definitions are more common.
In practice, a proof
in deduction form will also contain many steps that are implications
where the antecedent is either that wff variable (normally $\varphi$)
or is
a conjunction (...$\land$...) including that wff variable ($\varphi$).
If an assertion is in deduction form, and other forms are also available,
then we suffix its label with ``d.''
An example is \texttt{unssd}, which states\footnote{
For brevity we show here (and in other places)
a $\&$\index{$\&$} between hypotheses\index{hypotheses}
and a $\Rightarrow$\index{$\Rightarrow$}\index{conclusion}
between the hypotheses and the conclusion.
This notation is technically not part of the Metamath language, but is
instead a convenient abbreviation to show both the hypotheses and conclusion.}:
$\vdash ( \varphi \rightarrow A \subseteq C )\quad\&\quad \vdash ( \varphi
    \rightarrow B \subseteq C )\quad\Rightarrow\quad \vdash ( \varphi
    \rightarrow ( A \cup B ) \subseteq C )\label{eq:unssd}$
\item \textbf{inference form}\index{inference form}\index{forms!inference}:
A kind of assertion with one or more hypotheses that is not in deduction form
(e.g., there is no common antecedent).
If an assertion is in inference form, and other forms are also available,
then we suffix its label with ``i.''
An example is \texttt{unssi}, which states:
$\vdash A \subseteq C\quad\&\quad \vdash B \subseteq C\quad\Rightarrow\quad
    \vdash ( A \cup B ) \subseteq C\label{eq:unssi}$
\end{itemize}

When using deduction style we express an assertion in deduction form.
This form prefixes each hypothesis (other than definitions) and the
conclusion with a universal antecedent (``$\varphi \rightarrow$'').
The antecedent (e.g., $\varphi$)
mimics the context handled in the deduction theorem, eliminating
the need to directly use the deduction theorem.

Once you have an assertion in deduction form, you can easily convert it
to inference form or closed form:

\begin{itemize}
\item To
prove some assertion Ti in inference form, given assertion Td in deduction
form, there is a simple mechanical process you can use. First take each
Ti hypothesis and insert a \texttt{T.} $\rightarrow$ prefix (``true implies'')
using \texttt{a1i}. You
can then use the existing assertion Td to prove the resulting conclusion
with a \texttt{T.} $\rightarrow$ prefix.
Finally, you can remove that prefix using \texttt{mptru},
resulting in the conclusion you wanted to prove.
\item To
prove some assertion T in closed form, given assertion Td in deduction
form, there is another simple mechanical process you can use. First,
select an expression that is the conjunction (...$\land$...) of all of the
consequents of every hypothesis of Td. Next, prove that this expression
implies each of the separate hypotheses of Td in turn by eliminating
conjuncts (there are a variety of proven assertions to do this, including
\texttt{simpl},
\texttt{simpr},
\texttt{3simpa},
\texttt{3simpb},
\texttt{3simpc},
\texttt{simp1},
\texttt{simp2},
and
\texttt{simp3}).
If the
expression has nested conjunctions, inner conjuncts can be broken out by
chaining the above theorems with \texttt{syl}
(see section \ref{syl}).\footnote{
There are actually many theorems
(labeled simp* such as \texttt{simp333}) that break out inner conjuncts in one
step, but rather than learning them you can just use the chaining we
just described to prove them, and then let the Metamath program command
\texttt{minimize{\char`\_}with}\index{\texttt{minimize{\char`\_}with} command}
figure out the right ones needed to collapse them.}
As your final step, you can then apply the already-proven assertion Td
(which is in deduction form), proving assertion T in closed form.
\end{itemize}

We can also easily convert any assertion T in closed form to its related
assertion Ti in inference form by applying
modus ponens\index{modus ponens} (see section \ref{axmp}).

The deduction form antecedent can also be used to represent the context
necessary to support natural deduction systems, so we will now
discuss natural deduction.

\subsection{Natural Deduction}\label{naturaldeduction}

Natural deduction\index{natural deduction}
(ND) systems, as such, were originally introduced in
1934 by two logicians working independently: Ja\'skowski and Gentzen. ND
systems are supposed to reconstruct, in a formally proper way, traditional
ways of mathematical reasoning (such as conditional proof, indirect proof,
and proof by cases). As reconstructions they were naturally influenced
by previous work, and many specific ND systems and notations have been
developed since their original work.

There are many ND variants, but
Indrzejczak \cite[p.~31-32]{Indrzejczak}\index{Indrzejczak, Andrzej}
suggests that any natural deductive system must satisfy at
least these three criteria:

\begin{itemize}
\item ``There are some means for entering assumptions into a proof and
also for eliminating them. Usually it requires some bookkeeping devices
for indicating the scope of an assumption, and showing that a part of
a proof depending on eliminated assumption is discharged.
\item There are no (or, at least, a very limited set of) axioms, because
their role is taken over by the set of primitive rules for introduction
and elimination of logical constants which means that elementary
inferences instead of formulae are taken as primitive.
\item (A genuine) ND system admits a lot of freedom in proof construction
and possibility of applying several proof search strategies, like
conditional proof, proof by cases, proof by reductio ad absurdum etc.''
\end{itemize}

The Metamath Proof Explorer (MPE) as defined in \texttt{set.mm}
is fundamentally a Hilbert-style system.
That is, MPE is based on a larger number of axioms (compared
to natural deduction systems), a very small set of rules of inference
(modus ponens), and the context is not changed by the rules of inference
in the middle of a proof. That said, MPE proofs can be developed using
the natural deduction (ND) approach as originally developed by Ja\'skowski
and Gentzen.

The most common and recommended approach for applying ND in MPE is to use
deduction form\index{deduction form}%
\index{forms!deduction}
and apply the MPE proven assertions that are equivalent to ND rules.
For example, MPE's \texttt{jca} is equivalent to ND rule $\land$-I
(and-insertion).
We maintain a list of equivalences that you may consult.
This approach for applying an ND approach within MPE relies on Metamath's
wff metavariables in an essential way, and is described in more detail
in the presentation ``Natural Deductions in the Metamath Proof Language''
by Mario Carneiro \cite{CarneiroND}\index{Carneiro, Mario}.

In this style many steps are an implication, whose antecedent mimics
the context ($\Gamma$) of most ND systems. To add an assumption, simply add
it to the implication antecedent (typically using
\texttt{simpr}),
and use that
new antecedent for all later claims in the same scope. If you wish to
use an assertion in an ND hypothesis scope that is outside the current
ND hypothesis scope, modify the assertion so that the ND hypothesis
assumption is added to its antecedent (typically using \texttt{adantr}). Most
proof steps will be proved using rules that have hypotheses and results
of the form $\varphi \rightarrow$ ...

An example may make this clearer.
Let's show theorem 5.5 of
\cite[p.~18]{Clemente}\index{Clemente Laboreo, Daniel}
along with a line by line translation using the usual
translation of natural deduction (ND) in the Metamath Proof Explorer
(MPE) notation (this is proof \texttt{ex-natded5.5}).
The proof's original goal was to prove
$\lnot \psi$ given two hypotheses,
$( \psi \rightarrow \chi )$ and $ \lnot \chi$.
We will translate these statements into MPE deduction form
by prefixing them all with $\varphi \rightarrow$.
As a result, in MPE the goal is stated as
$( \varphi \rightarrow \lnot \psi )$, and the two hypotheses are stated as
$( \varphi \rightarrow ( \psi \rightarrow \chi ) )$ and
$( \varphi \rightarrow \lnot \chi )$.

The following table shows the proof in Fitch natural deduction style
and its MPE equivalent.
The \textit{\#} column shows the original numbering,
\textit{MPE\#} shows the number in the equivalent MPE proof
(which we will show later),
\textit{ND Expression} shows the original proof claim in ND notation,
and \textit{MPE Translation} shows its translation into MPE
as discussed in this section.
The final columns show the rationale in ND and MPE respectively.

\needspace{4\baselineskip}
{\setlength{\extrarowsep}{4pt} % Keep rows from being too close together
\begin{longtabu}   { @{} c c X X X X }
\textbf{\#} & \textbf{MPE\#} & \textbf{ND Ex\-pres\-sion} &
\textbf{MPE Trans\-lation} & \textbf{ND Ration\-ale} &
\textbf{MPE Ra\-tio\-nale} \\
\endhead

1 & 2;3 &
$( \psi \rightarrow \chi )$ &
$( \varphi \rightarrow ( \psi \rightarrow \chi ) )$ &
Given &
\$e; \texttt{adantr} to put in ND hypothesis \\

2 & 5 &
$ \lnot \chi$ &
$( \varphi \rightarrow \lnot \chi )$ &
Given &
\$e; \texttt{adantr} to put in ND hypothesis \\

3 & 1 &
... $\vert$ $\psi$ &
$( \varphi \rightarrow \psi )$ &
ND hypothesis assumption &
\texttt{simpr} \\

4 & 4 &
... $\chi$ &
$( ( \varphi \land \psi ) \rightarrow \chi )$ &
$\rightarrow$\,E 1,3 &
\texttt{mpd} 1,3 \\

5 & 6 &
... $\lnot \chi$ &
$( ( \varphi \land \psi ) \rightarrow \lnot \chi )$ &
IT 2 &
\texttt{adantr} 5 \\

6 & 7 &
$\lnot \psi$ &
$( \varphi \rightarrow \lnot \psi )$ &
$\land$\,I 3,4,5 &
\texttt{pm2.65da} 4,6 \\

\end{longtabu}
}


The original used Latin letters; we have replaced them with Greek letters
to follow Metamath naming conventions and so that it is easier to follow
the Metamath translation. The Metamath line-for-line translation of
this natural deduction approach precedes every line with an antecedent
including $\varphi$ and uses the Metamath equivalents of the natural deduction
rules. To add an assumption, the antecedent is modified to include it
(typically by using \texttt{adantr};
\texttt{simpr} is useful when you want to
depend directly on the new assumption, as is shown here).

In Metamath we can represent the two given statements as these hypotheses:

\needspace{2\baselineskip}
\begin{itemize}
\item ex-natded5.5.1 $\vdash ( \varphi \rightarrow ( \psi \rightarrow \chi ) )$
\item ex-natded5.5.2 $\vdash ( \varphi \rightarrow \lnot \chi )$
\end{itemize}

\needspace{4\baselineskip}
Here is the proof in Metamath as a line-by-line translation:

\begin{longtabu}   { l l l X }
\textbf{Step} & \textbf{Hyp} & \textbf{Ref} & \textbf{Ex\-pres\-sion} \\
\endhead
1 & & simpr & $\vdash ( ( \varphi \land \psi ) \rightarrow \psi )$ \\
2 & & ex-natded5.5.1 &
  $\vdash ( \varphi \rightarrow ( \psi \rightarrow \chi ) )$ \\
3 & 2 & adantr &
 $\vdash ( ( \varphi \land \psi ) \rightarrow ( \psi \rightarrow \chi ) )$ \\
4 & 1, 3 & mpd &
 $\vdash ( ( \varphi \land \psi ) \rightarrow \chi ) $ \\
5 & & ex-natded5.5.2 &
 $\vdash ( \varphi \rightarrow \lnot \chi )$ \\
6 & 5 & adantr &
 $\vdash ( ( \varphi \land \psi ) \rightarrow \lnot \chi )$ \\
7 & 4, 6 & pm2.65da &
 $\vdash ( \varphi \rightarrow \lnot \psi )$ \\
\end{longtabu}

Only using specific natural deduction rules directly can lead to very
long proofs, for exactly the same reason that only using axioms directly
in Hilbert-style proofs can lead to very long proofs.
If the goal is short and clear proofs,
then it is better to reuse already-proven assertions
in deduction form than to start from scratch each time
and using only basic natural deduction rules.

\subsection{Strengths of Our Approach}

As far as we know there is nothing else in the literature like either the
weak deduction theorem or Mario Carneiro\index{Carneiro, Mario}'s
natural deduction method.
In order to
transform a hypothesis into an antecedent, the literature's standard
``Deduction Theorem''\index{Deduction Theorem}\index{Standard Deduction Theorem}
requires metalogic outside of the notions provided
by the axiom system. We instead generally prefer to use Mario Carneiro's
natural deduction method, then use the weak deduction theorem in cases
where that is difficult to apply, and only then use the full standard
deduction theorem as a last resort.

The weak deduction theorem\index{Weak Deduction Theorem}
does not require any additional metalogic
but converts an inference directly into a closed form theorem, with
a rigorous proof that uses only the axiom system. Unlike the standard
Deduction Theorem, there is no implicit external justification that we
have to trust in order to use it.

Mario Carneiro's natural deduction\index{natural deduction}
method also does not require any new metalogical
notions. It avoids the Deduction Theorem's metalogic by prefixing the
hypotheses and conclusion of every would-be inference with a universal
antecedent (``$\varphi \rightarrow$'') from the very start.

We think it is impressive and satisfying that we can do so much in a
practical sense without stepping outside of our Hilbert-style axiom system.
Of course our axiomatization, which is in the form of schemes,
contains a metalogic of its own that we exploit. But this metalogic
is relatively simple, and for our Deduction Theorem alternatives,
we primarily use just the direct substitution of expressions for
metavariables.

\begin{sloppy}
\section{Exploring the Set The\-o\-ry Data\-base}\label{exploring}
\end{sloppy}
% NOTE: All examples performed in this section are
% recorded wtih "set width 61" % on set.mm as of 2019-05-28
% commit c1e7849557661260f77cfdf0f97ac4354fbb4f4d.

At this point you may wish to study the \texttt{set.mm}\index{set theory
database (\texttt{set.mm})} file in more detail.  Pay particular
attention to the assumptions needed to define wffs\index{well-formed
formula (wff)} (which are not included above), the variable types
(\texttt{\$f}\index{\texttt{\$f} statement} statements), and the
definitions that are introduced.  Start with some simple theorems in
propositional calculus, making sure you understand in detail each step
of a proof.  Once you get past the first few proofs and become familiar
with the Metamath language, any part of the \texttt{set.mm} database
will be as easy to follow, step by step, as any other part---you won't
have to undergo a ``quantum leap'' in mathematical sophistication to be
able to follow a deep proof in set theory.

Next, you may want to explore how concepts such as natural numbers are
defined and described.  This is probably best done in conjunction with
standard set theory textbooks, which can help give you a higher-level
understanding.  The \texttt{set.mm} database provides references that will get
you started.  From there, you will be on your way towards a very deep,
rigorous understanding of abstract mathematics.

The Metamath\index{Metamath} program can help you peruse a Metamath data\-base,
wheth\-er you are trying to figure out how a certain step follows in a proof or
just have a general curiosity.  We will go through some examples of the
commands, using the \texttt{set.mm}\index{set theory database (\texttt{set.mm})}
database provided with the Metamath software.  These should help get you
started.  See Chapter~\ref{commands} for a more detailed description of
the commands.  Note that we have included the full spelling of all commands to
prevent ambiguity with future commands.  In practice you may type just the
characters needed to specify each command keyword\index{command keyword}
unambiguously, often just one or two characters per keyword, and you don't
need to type them in upper case.

First run the Metamath program as described earlier.  You should see the
\verb/MM>/ prompt.  Read in the \texttt{set.mm} file:\index{\texttt{read}
command}

\begin{verbatim}
MM> read set.mm
Reading source file "set.mm"... 34554442 bytes
34554442 bytes were read into the source buffer.
The source has 155711 statements; 2254 are $a and 32250 are $p.
No errors were found.  However, proofs were not checked.
Type VERIFY PROOF * if you want to check them.
\end{verbatim}

As with most examples in this book, what you will see
will be slightly different because we are continuously
improving our databases (including \texttt{set.mm}).

Let's check the database integrity.  This may take a minute or two to run if
your computer is slow.

\begin{verbatim}
MM> verify proof *
0 10%  20%  30%  40%  50%  60%  70%  80%  90% 100%
..................................................
All proofs in the database were verified in 2.84 s.
\end{verbatim}

No errors were reported, so every proof is correct.

You need to know the names (labels) of theorems before you can look at them.
Often just examining the database file(s) with a text editor is the best
approach.  In \texttt{set.mm} there are many detailed comments, especially near
the beginning, that can help guide you. The \texttt{search} command in the
Metamath program is also handy.  The \texttt{comments} qualifier will list the
statements whose associated comment (the one immediately before it) contain a
string you give it.  For example, if you are studying Enderton's {\em Elements
of Set Theory} \cite{Enderton}\index{Enderton, Herbert B.} you may want to see
the references to it in the database.  The search string \texttt{enderton} is not
case sensitive.  (This will not show you all the database theorems that are in
Enderton's book because there is usually only one citation for a given
theorem, which may appear in several textbooks.)\index{\texttt{search}
command}

\begin{verbatim}
MM> search * "enderton" / comments
12067 unineq $p "... Exercise 20 of [Enderton] p. 32 and ..."
12459 undif2 $p "...Corollary 6K of [Enderton] p. 144. (C..."
12953 df-tp $a "...s. Definition of [Enderton] p. 19. (Co..."
13689 unissb $p ".... Exercise 5 of [Enderton] p. 26 and ..."
\end{verbatim}
\begin{center}
(etc.)
\end{center}

Or you may want to see what theorems have something to do with
conjunction (logical {\sc and}).  The quotes around the search
string are optional when there's no ambiguity.\index{\texttt{search}
command}

\begin{verbatim}
MM> search * conjunction / comments
120 a1d $p "...be replaced with a conjunction ( ~ df-an )..."
662 df-bi $a "...viated form after conjunction is introdu..."
1319 wa $a "...ff definition to include conjunction ('and')."
1321 df-an $a "Define conjunction (logical 'and'). Defini..."
1420 imnan $p "...tion in terms of conjunction. (Contribu..."
\end{verbatim}
\begin{center}
(etc.)
\end{center}


Now we will start to look at some details.  Let's look at the first
axiom of propositional calculus
(we could use \texttt{sh st} to abbreviate
\texttt{show statement}).\index{\texttt{show statement} command}

\begin{verbatim}
MM> show statement ax-1/full
Statement 19 is located on line 881 of the file "set.mm".
"Axiom _Simp_.  Axiom A1 of [Margaris] p. 49.  One of the 3
axioms of propositional calculus.  The 3 axioms are also
        ...
19 ax-1 $a |- ( ph -> ( ps -> ph ) ) $.
Its mandatory hypotheses in RPN order are:
  wph $f wff ph $.
  wps $f wff ps $.
The statement and its hypotheses require the variables:  ph
      ps
The variables it contains are:  ph ps


Statement 49 is located on line 11182 of the file "set.mm".
Its statement number for HTML pages is 6.
"Axiom _Simp_.  Axiom A1 of [Margaris] p. 49.  One of the 3
axioms of propositional calculus.  The 3 axioms are also
given as Definition 2.1 of [Hamilton] p. 28.
...
49 ax-1 $a |- ( ph -> ( ps -> ph ) ) $.
Its mandatory hypotheses in RPN order are:
  wph $f wff ph $.
  wps $f wff ps $.
The statement and its hypotheses require the variables:
  ph ps
The variables it contains are:  ph ps
\end{verbatim}

Compare this to \texttt{ax-1} on p.~\pageref{ax1}.  You can see that the
symbol \texttt{ph} is the {\sc ascii} notation for $\varphi$, etc.  To
see the mathematical symbols for any expression you may typeset it in
\LaTeX\ (type \texttt{help tex} for instructions)\index{latex@{\LaTeX}}
or, easier, just use a text editor to look at the comments where symbols
are first introduced in \texttt{set.mm}.  The hypotheses \texttt{wph}
and \texttt{wps} required by \texttt{ax-1} mean that variables
\texttt{ph} and \texttt{ps} must be wffs.

Next we'll pick a simple theorem of propositional calculus, the Principle of
Identity, which is proved directly from the axioms.  We'll look at the
statement then its proof.\index{\texttt{show statement}
command}

\begin{verbatim}
MM> show statement id1/full
Statement 116 is located on line 11371 of the file "set.mm".
Its statement number for HTML pages is 22.
"Principle of identity.  Theorem *2.08 of [WhiteheadRussell]
p. 101.  This version is proved directly from the axioms for
demonstration purposes.
...
116 id1 $p |- ( ph -> ph ) $= ... $.
Its mandatory hypotheses in RPN order are:
  wph $f wff ph $.
Its optional hypotheses are:  wps wch wth wta wet
      wze wsi wrh wmu wla wka
The statement and its hypotheses require the variables:  ph
These additional variables are allowed in its proof:
      ps ch th ta et ze si rh mu la ka
The variables it contains are:  ph
\end{verbatim}

The optional variables\index{optional variable} \texttt{ps}, \texttt{ch}, etc.\ are
available for use in a proof of this statement if we wish, and were we to do
so we would make use of optional hypotheses \texttt{wps}, \texttt{wch}, etc.  (See
Section~\ref{dollaref} for the meaning of ``optional
hypothesis.''\index{optional hypothesis}) The reason these show up in the
statement display is that statement \texttt{id1} happens to be in their scope
(see Section~\ref{scoping} for the definition of ``scope''\index{scope}), but
in fact in propositional calculus we will never make use of optional
hypotheses or variables.  This becomes important after quantifiers are
introduced, where ``dummy'' variables\index{dummy variable} are often needed
in the middle of a proof.

Let's look at the proof of statement \texttt{id1}.  We'll use the
\texttt{show proof} command, which by default suppresses the
``non-essential'' steps that construct the wffs.\index{\texttt{show proof}
command}
We will display the proof in ``lemmon' format (a non-indented format
with explicit previous step number references) and renumber the
displayed steps:

\begin{verbatim}
MM> show proof id1 /lemmon/renumber
1 ax-1           $a |- ( ph -> ( ph -> ph ) )
2 ax-1           $a |- ( ph -> ( ( ph -> ph ) -> ph ) )
3 ax-2           $a |- ( ( ph -> ( ( ph -> ph ) -> ph ) ) ->
                     ( ( ph -> ( ph -> ph ) ) -> ( ph -> ph )
                                                          ) )
4 2,3 ax-mp      $a |- ( ( ph -> ( ph -> ph ) ) -> ( ph -> ph
                                                          ) )
5 1,4 ax-mp      $a |- ( ph -> ph )
\end{verbatim}

If you have read Section~\ref{trialrun}, you'll know how to interpret this
proof.  Step~2, for example, is an application of axiom \texttt{ax-1}.  This
proof is identical to the one in Hamilton's {\em Logic for Mathematicians}
\cite[p.~32]{Hamilton}\index{Hamilton, Alan G.}.

You may want to look at what
substitutions\index{substitution!variable}\index{variable substitution} are
made into \texttt{ax-1} to arrive at step~2.  The command to do this needs to
know the ``real'' step number, so we'll display the proof again without
the \texttt{renumber} qualifier.\index{\texttt{show proof}
command}

\begin{verbatim}
MM> show proof id1 /lemmon
 9 ax-1          $a |- ( ph -> ( ph -> ph ) )
20 ax-1          $a |- ( ph -> ( ( ph -> ph ) -> ph ) )
24 ax-2          $a |- ( ( ph -> ( ( ph -> ph ) -> ph ) ) ->
                     ( ( ph -> ( ph -> ph ) ) -> ( ph -> ph )
                                                          ) )
25 20,24 ax-mp   $a |- ( ( ph -> ( ph -> ph ) ) -> ( ph -> ph
                                                          ) )
26 9,25 ax-mp    $a |- ( ph -> ph )
\end{verbatim}

The ``real'' step number is 20.  Let's look at its details.

\begin{verbatim}
MM> show proof id1 /detailed_step 20
Proof step 20:  min=ax-1 $a |- ( ph -> ( ( ph -> ph ) -> ph )
  )
This step assigns source "ax-1" ($a) to target "min" ($e).
The source assertion requires the hypotheses "wph" ($f, step
18) and "wps" ($f, step 19).  The parent assertion of the
target hypothesis is "ax-mp" ($a, step 25).
The source assertion before substitution was:
    ax-1 $a |- ( ph -> ( ps -> ph ) )
The following substitutions were made to the source
assertion:
    Variable  Substituted with
     ph        ph
     ps        ( ph -> ph )
The target hypothesis before substitution was:
    min $e |- ph
The following substitution was made to the target hypothesis:
    Variable  Substituted with
     ph        ( ph -> ( ( ph -> ph ) -> ph ) )
\end{verbatim}

This shows the substitutions\index{substitution!variable}\index{variable
substitution} made to the variables in \texttt{ax-1}.  References are made to
steps 18 and 19 which are not shown in our proof display.  To see these steps,
you can display the proof with the \texttt{all} qualifier.

Let's look at a slightly more advanced proof of propositional calculus.  Note
that \verb+/\+ is the symbol for $\wedge$ (logical {\sc and}, also
called conjunction).\index{conjunction ($\wedge$)}
\index{logical {\sc and} ($\wedge$)}

\begin{verbatim}
MM> show statement prth/full
Statement 1791 is located on line 15503 of the file "set.mm".
Its statement number for HTML pages is 559.
"Conjoin antecedents and consequents of two premises.  This
is the closed theorem form of ~ anim12d .  Theorem *3.47 of
[WhiteheadRussell] p. 113.  It was proved by Leibniz,
and it evidently pleased him enough to call it
_praeclarum theorema_ (splendid theorem).
...
1791 prth $p |- ( ( ( ph -> ps ) /\ ( ch -> th ) ) -> ( ( ph
      /\ ch ) -> ( ps /\ th ) ) ) $= ... $.
Its mandatory hypotheses in RPN order are:
  wph $f wff ph $.
  wps $f wff ps $.
  wch $f wff ch $.
  wth $f wff th $.
Its optional hypotheses are:  wta wet wze wsi wrh wmu wla wka
The statement and its hypotheses require the variables:  ph
      ps ch th
These additional variables are allowed in its proof:  ta et
      ze si rh mu la ka
The variables it contains are:  ph ps ch th


MM> show proof prth /lemmon/renumber
1 simpl          $p |- ( ( ( ph -> ps ) /\ ( ch -> th ) ) ->
                                               ( ph -> ps ) )
2 simpr          $p |- ( ( ( ph -> ps ) /\ ( ch -> th ) ) ->
                                               ( ch -> th ) )
3 1,2 anim12d    $p |- ( ( ( ph -> ps ) /\ ( ch -> th ) ) ->
                           ( ( ph /\ ch ) -> ( ps /\ th ) ) )
\end{verbatim}

There are references to a lot of unfamiliar statements.  To see what they are,
you may type the following:

\begin{verbatim}
MM> show proof prth /statement_summary
Summary of statements used in the proof of "prth":

Statement simpl is located on line 14748 of the file
"set.mm".
"Elimination of a conjunct.  Theorem *3.26 (Simp) of
[WhiteheadRussell] p. 112. ..."
  simpl $p |- ( ( ph /\ ps ) -> ph ) $= ... $.

Statement simpr is located on line 14777 of the file
"set.mm".
"Elimination of a conjunct.  Theorem *3.27 (Simp) of
[WhiteheadRussell] ..."
  simpr $p |- ( ( ph /\ ps ) -> ps ) $= ... $.

Statement anim12d is located on line 15445 of the file
"set.mm".
"Conjoin antecedents and consequents in a deduction.
..."
  anim12d.1 $e |- ( ph -> ( ps -> ch ) ) $.
  anim12d.2 $e |- ( ph -> ( th -> ta ) ) $.
  anim12d $p |- ( ph -> ( ( ps /\ th ) -> ( ch /\ ta ) ) )
      $= ... $.
\end{verbatim}
\begin{center}
(etc.)
\end{center}

Of course you can look at each of these statements and their proofs, and
so on, back to the axioms of propositional calculus if you wish.

The \texttt{search} command is useful for finding statements when you
know all or part of their contents.  The following example finds all
statements containing \verb@ph -> ps@ followed by \verb@ch -> th@.  The
\verb@$*@ is a wildcard that matches anything; the \texttt{\$} before the
\verb$*$ prevents conflicts with math symbol token names.  The \verb@*@ after
\texttt{SEARCH} is also a wildcard that in this case means ``match any label.''
\index{\texttt{search} command}

% I'm omitting this one, since readers are unlikely to see it:
% 1096 bisymOLD $p |- ( ( ( ph -> ps ) -> ( ch -> th ) ) -> ( (
%   ( ps -> ph ) -> ( th -> ch ) ) -> ( ( ph <-> ps ) -> ( ch
%    <-> th ) ) ) )
\begin{verbatim}
MM> search * "ph -> ps $* ch -> th"
1791 prth $p |- ( ( ( ph -> ps ) /\ ( ch -> th ) ) -> ( ( ph
    /\ ch ) -> ( ps /\ th ) ) )
2455 pm3.48 $p |- ( ( ( ph -> ps ) /\ ( ch -> th ) ) -> ( (
    ph \/ ch ) -> ( ps \/ th ) ) )
117859 pm11.71 $p |- ( ( E. x ph /\ E. y ch ) -> ( ( A. x (
    ph -> ps ) /\ A. y ( ch -> th ) ) <-> A. x A. y ( ( ph /\
    ch ) -> ( ps /\ th ) ) ) )
\end{verbatim}

Three statements, \texttt{prth}, \texttt{pm3.48},
 and \texttt{pm11.71}, were found to match.

To see what axioms\index{axiom} and definitions\index{definition}
\texttt{prth} ultimately depends on for its proof, you can have the
program backtrack through the hierarchy\index{hierarchy} of theorems and
definitions.\index{\texttt{show trace{\char`\_}back} command}

\begin{verbatim}
MM> show trace_back prth /essential/axioms
Statement "prth" assumes the following axioms ($a
statements):
  ax-1 ax-2 ax-3 ax-mp df-bi df-an
\end{verbatim}

Note that the 3 axioms of propositional calculus and the modus ponens rule are
needed (as expected); in addition, there are a couple of definitions that are used
along the way.  Note that Metamath makes no distinction\index{axiom vs.\
definition} between axioms\index{axiom} and definitions\index{definition}.  In
\texttt{set.mm} they have been distinguished artificially by prefixing their
labels\index{labels in \texttt{set.mm}} with \texttt{ax-} and \texttt{df-}
respectively.  For example, \texttt{df-an} defines conjunction (logical {\sc
and}), which is represented by the symbol \verb+/\+.
Section~\ref{definitions} discusses the philosophy of definitions, and the
Metamath language takes a particularly simple, conservative approach by using
the \texttt{\$a}\index{\texttt{\$a} statement} statement for both axioms and
definitions.

You can also have the program compute how many steps a proof
has\index{proof length} if we were to follow it all the way back to
\texttt{\$a} statements.

\begin{verbatim}
MM> show trace_back prth /essential/count_steps
The statement's actual proof has 3 steps.  Backtracking, a
total of 79 different subtheorems are used.  The statement
and subtheorems have a total of 274 actual steps.  If
subtheorems used only once were eliminated, there would be a
total of 38 subtheorems, and the statement and subtheorems
would have a total of 185 steps.  The proof would have 28349
steps if fully expanded back to axiom references.  The
maximum path length is 38.  A longest path is:  prth <-
anim12d <- syl2and <- sylan2d <- ancomsd <- ancom <- pm3.22
<- pm3.21 <- pm3.2 <- ex <- sylbir <- biimpri <- bicomi <-
bicom1 <- bi2 <- dfbi1 <- impbii <- bi3 <- simprim <- impi <-
con1i <- nsyl2 <- mt3d <- con1d <- notnot1 <- con2i <- nsyl3
<- mt2d <- con2d <- notnot2 <- pm2.18d <- pm2.18 <- pm2.21 <-
pm2.21d <- a1d <- syl <- mpd <- a2i <- a2i.1 .
\end{verbatim}

This tells us that we would have to inspect 274 steps if we want to
verify the proof completely starting from the axioms.  A few more
statistics are also shown.  There are one or more paths back to axioms
that are the longest; this command ferrets out one of them and shows it
to you.  There may be a sense in which the longest path length is
related to how ``deep'' the theorem is.

We might also be curious about what proofs depend on the theorem
\texttt{prth}.  If it is never used later on, we could eliminate it as
redundant if it has no intrinsic interest by itself.\index{\texttt{show
usage} command}

% I decided to show the OLD values here.
\begin{verbatim}
MM> show usage prth
Statement "prth" is directly referenced in the proofs of 18
statements:
  mo3 moOLD 2mo 2moOLD euind reuind reuss2 reusv3i opelopabt
  wemaplem2 rexanre rlimcn2 o1of2 o1rlimmul 2sqlem6 spanuni
  heicant pm11.71
\end{verbatim}

Thus \texttt{prth} is directly used by 18 proofs.
We can use the \texttt{/recursive} qualifier to include indirect use:

\begin{verbatim}
MM> show usage prth /recursive
Statement "prth" directly or indirectly affects the proofs of
24214 statements:
  mo3 mo mo3OLD eu2 moOLD eu2OLD eu3OLD mo4f mo4 eu4 mopick
...
\end{verbatim}

\subsection{A Note on the ``Compact'' Proof Format}

The Metamath program will display proofs in a ``compact''\index{compact proof}
format whenever the proof is stored in compressed format in the database.  It
may be be slightly confusing unless you know how to interpret it.
For example,
if you display the complete proof of theorem \texttt{id1} it will start
off as follows:

\begin{verbatim}
MM> show proof id1 /lemmon/all
 1 wph           $f wff ph
 2 wph           $f wff ph
 3 wph           $f wff ph
 4 2,3 wi    @4: $a wff ( ph -> ph )
 5 1,4 wi    @5: $a wff ( ph -> ( ph -> ph ) )
 6 @4            $a wff ( ph -> ph )
\end{verbatim}

\begin{center}
{etc.}
\end{center}

Step 4 has a ``local label,''\index{local label} \texttt{@4}, assigned to it.
Later on, at step 6, the label \texttt{@1} is referenced instead of
displaying the explicit proof for that step.  This technique takes advantage
of the fact that steps in a proof often repeat, especially during the
construction of wffs.  The compact format reduces the number of steps in the
proof display and may be preferred by some people.

If you want to see the normal format with the ``true'' step numbers, you can
use the following workaround:\index{\texttt{save proof} command}

\begin{verbatim}
MM> save proof id1 /normal
The proof of "id1" has been reformatted and saved internally.
Remember to use WRITE SOURCE to save it permanently.
MM> show proof id1 /lemmon/all
 1 wph           $f wff ph
 2 wph           $f wff ph
 3 wph           $f wff ph
 4 2,3 wi        $a wff ( ph -> ph )
 5 1,4 wi        $a wff ( ph -> ( ph -> ph ) )
 6 wph           $f wff ph
 7 wph           $f wff ph
 8 6,7 wi        $a wff ( ph -> ph )
\end{verbatim}

\begin{center}
{etc.}
\end{center}

Note that the original 6 steps are now 8 steps.  However, the format is
now the same as that described in Chapter~\ref{using}.

\chapter{The Metamath Language}
\label{languagespec}

\begin{quote}
  {\em Thus mathematics may be defined as the subject in which we never know
what we are talking about, nor whether what we are saying is true.}
    \flushright\sc  Bertrand Russell\footnote{\cite[p.~84]{Russell2}.}\\
\end{quote}\index{Russell, Bertrand}

Probably the most striking feature of the Metamath language is its almost
complete absence of hard-wired syntax. Metamath\index{Metamath} does not
understand any mathematics or logic other than that needed to construct finite
sequences of symbols according to a small set of simple, built-in rules.  The
only rule it uses in a proof is the substitution of an expression (symbol
sequence) for a variable, subject to a simple constraint to prevent
bound-variable clashes.  The primitive notions built into Metamath involve the
simple manipulation of finite objects (symbols) that we as humans can easily
visualize and that computers can easily deal with.  They seem to be just
about the simplest notions possible that are required to do standard
mathematics.

This chapter serves as a reference manual for the Metamath\index{Metamath}
language. It covers the tedious technical details of the language, some of
which you may wish to skim in a first reading.  On the other hand, you should
pay close attention to the defined terms in {\bf boldface}; they have precise
meanings that are important to keep in mind for later understanding.  It may
be best to first become familiar with the examples in Chapter~\ref{using} to
gain some motivation for the language.

%% Uncomment this when uncommenting section {formalspec} below
If you have some knowledge of set theory, you may wish to study this
chapter in conjunction with the formal set-theoretical description of the
Metamath language in Appendix~\ref{formalspec}.

We will use the name ``Metamath''\index{Metamath} to mean either the Metamath
computer language or the Metamath software associated with the computer
language.  We will not distinguish these two when the context is clear.

The next section contains the complete specification of the Metamath
language.
It serves as an
authoritative reference and presents the syntax in enough detail to
write a parser\index{parsing Metamath} and proof verifier.  The
specification is terse and it is probably hard to learn the language
directly from it, but we include it here for those impatient people who
prefer to see everything up front before looking at verbose expository
material.  Later sections explain this material and provide examples.
We will repeat the definitions in those sections, and you may skip the
next section at first reading and proceed to Section~\ref{tut1}
(p.~\pageref{tut1}).

\section{Specification of the Metamath Language}\label{spec}
\index{Metamath!specification}

\begin{quote}
  {\em Sometimes one has to say difficult things, but one ought to say
them as simply as one knows how.}
    \flushright\sc  G. H. Hardy\footnote{As quoted in
    \cite{deMillo}, p.~273.}\\
\end{quote}\index{Hardy, G. H.}

\subsection{Preliminaries}\label{spec1}

% Space is technically a printable character, so we'll word things
% carefully so it's unambiguous.
A Metamath {\bf database}\index{database} is built up from a top-level source
file together with any source files that are brought in through file inclusion
commands (see below).  The only characters that are allowed to appear in a
Metamath source file are the 94 non-whitespace printable {\sc
ascii}\index{ascii@{\sc ascii}} characters, which are digits, upper and lower
case letters, and the following 32 special
characters\index{special characters}:\label{spec1chars}

\begin{verbatim}
! " # $ % & ' ( ) * + , - . / :
; < = > ? @ [ \ ] ^ _ ` { | } ~
\end{verbatim}
plus the following characters which are the ``white space'' characters:
space (a printable character),
tab, carriage return, line feed, and form feed.\label{whitespace}
We will use \texttt{typewriter}
font to display the printable characters.

A Metamath database consists of a sequence of three kinds of {\bf
tokens}\index{token} separated by {\bf white space}\index{white space}
(which is any sequence of one or more white space characters).  The set
of {\bf keyword}\index{keyword} tokens is \texttt{\$\char`\{},
\texttt{\$\char`\}}, \texttt{\$c}, \texttt{\$v}, \texttt{\$f},
\texttt{\$e}, \texttt{\$d}, \texttt{\$a}, \texttt{\$p}, \texttt{\$.},
\texttt{\$=}, \texttt{\$(}, \texttt{\$)}, \texttt{\$[}, and
\texttt{\$]}.  The last four are called {\bf auxiliary}\index{auxiliary
keyword} or preprocessing keywords.  A {\bf label}\index{label} token
consists of any combination of letters, digits, and the characters
hyphen, underscore, and period.  A {\bf math symbol}\index{math symbol}
token may consist of any combination of the 93 printable standard {\sc
ascii} characters other than space or \texttt{\$}~. All tokens are
case-sensitive.

\subsection{Preprocessing}

The token \texttt{\$(} begins a {\bf comment} and
\texttt{\$)} ends a comment.\index{\texttt{\$(}
and \texttt{\$)} auxiliary keywords}\index{comment}
Comments may contain any of
the 94 non-whitespace printable characters and white space,
except they may not contain the
2-character sequences \texttt{\$(} or \texttt{\$)} (comments do not nest).
Comments are ignored (treated
like white space) for the purpose of parsing, e.g.,
\texttt{\$( \$[ \$)} is a comment.
See p.~\pageref{mathcomments} for comment typesetting conventions; these
conventions may be ignored for the purpose of parsing.

A {\bf file inclusion command} consists of \texttt{\$[} followed by a file name
followed by \texttt{\$]}.\index{\texttt{\$[} and \texttt{\$]} auxiliary
keywords}\index{included file}\index{file inclusion}
It is only allowed in the outermost scope (i.e., not between
\texttt{\$\char`\{} and \texttt{\$\char`\}})
and must not be inside a statement (e.g., it may not occur
between the label of a \texttt{\$a} statement and its \texttt{\$.}).
The file name may not
contain a \texttt{\$} or white space.  The file must exist.
The case-sensitivity
of its name follows the conventions of the operating system.  The contents of
the file replace the inclusion command.
Included files may include other files.
Only the first reference to a given file is included; any later
references to the same file (whether in the top-level file or in included
files) cause the inclusion command to be ignored (treated like white space).
A verifier may assume that file names with different strings
refer to different files for the purpose of ignoring later references.
A file self-reference is ignored, as is any reference to the top-level file
(to avoid loops).
Included files may not include a \texttt{\$(} without a matching \texttt{\$)},
may not include a \texttt{\$[} without a matching \texttt{\$]}, and may
not include incomplete statements (e.g., a \texttt{\$a} without a matching
\texttt{\$.}).
It is currently unspecified if path references are relative to the process'
current directory or the file's containing directory, so databases should
avoid using pathname separators (e.g., ``/'') in file names.

Like all tokens, the \texttt{\$(}, \texttt{\$)}, \texttt{\$[}, and \texttt{\$]} keywords
must be surrounded by white space.

\subsection{Basic Syntax}

After preprocessing, a database will consist of a sequence of {\bf
statements}.
These are the scoping statements \texttt{\$\char`\{} and
\texttt{\$\char`\}}, along with the \texttt{\$c}, \texttt{\$v},
\texttt{\$f}, \texttt{\$e}, \texttt{\$d}, \texttt{\$a}, and \texttt{\$p}
statements.

A {\bf scoping statement}\index{scoping statement} consists only of its
keyword, \texttt{\$\char`\{} or \texttt{\$\char`\}}.
A \texttt{\$\char`\{} begins a {\bf
block}\index{block} and a matching \texttt{\$\char`\}} ends the block.
Every \texttt{\$\char`\{}
must have a matching \texttt{\$\char`\}}.
Defining it recursively, we say a block
contains a sequence of zero or more tokens other
than \texttt{\$\char`\{} and \texttt{\$\char`\}} and
possibly other blocks.  There is an {\bf outermost
block}\index{block!outermost} not bracketed by \texttt{\$\char`\{} \texttt{\$\char`\}}; the end
of the outermost block is the end of the database.

% LaTeX bug? can't do \bf\tt

A {\bf \$v} or {\bf \$c statement}\index{\texttt{\$v} statement}\index{\texttt{\$c}
statement} consists of the keyword token \texttt{\$v} or \texttt{\$c} respectively,
followed by one or more math symbols,
% The word "token" is used to distinguish "$." from the sentence-ending period.
followed by the \texttt{\$.}\ token.
These
statements {\bf declare}\index{declaration} the math symbols to be {\bf
variables}\index{variable!Metamath} or {\bf constants}\index{constant}
respectively. The same math symbol may not occur twice in a given \texttt{\$v} or
\texttt{\$c} statement.

%c%A math symbol becomes an {\bf active}\index{active math symbol}
%c%when declared and stays active until the end of the block in which it is
%c%declared.  A math symbol may not be declared a second time while it is active,
%c%but it may be declared again after it becomes inactive.

A math symbol becomes {\bf active}\index{active math symbol} when declared
and stays active until the end of the block in which it is declared.  A
variable may not be declared a second time while it is active, but it
may be declared again (as a variable, but not as a constant) after it
becomes inactive.  A constant must be declared in the outermost block and may
not be declared a second time.\index{redeclaration of symbols}

A {\bf \$f statement}\index{\texttt{\$f} statement} consists of a label,
followed by \texttt{\$f}, followed by its typecode (an active constant),
followed by an
active variable, followed by the \texttt{\$.}\ token.  A {\bf \$e
statement}\index{\texttt{\$e} statement} consists of a label, followed
by \texttt{\$e}, followed by its typecode (an active constant),
followed by zero or more
active math symbols, followed by the \texttt{\$.}\ token.  A {\bf
hypothesis}\index{hypothesis} is a \texttt{\$f} or \texttt{\$e}
statement.
The type declared by a \texttt{\$f} statement for a given label
is global even if the variable is not
(e.g., a database may not have \texttt{wff P} in one local scope
and \texttt{class P} in another).

A {\bf simple \$d statement}\index{\texttt{\$d} statement!simple}
consists of \texttt{\$d}, followed by two different active variables,
followed by the \texttt{\$.}\ token.  A {\bf compound \$d
statement}\index{\texttt{\$d} statement!compound} consists of
\texttt{\$d}, followed by three or more variables (all different),
followed by the \texttt{\$.}\ token.  The order of the variables in a
\texttt{\$d} statement is unimportant.  A compound \texttt{\$d}
statement is equivalent to a set of simple \texttt{\$d} statements, one
for each possible pair of variables occurring in the compound
\texttt{\$d} statement.  Henceforth in this specification we shall
assume all \texttt{\$d} statements are simple.  A \texttt{\$d} statement
is also called a {\bf disjoint} (or {\bf distinct}) {\bf variable
restriction}.\index{disjoint-variable restriction}

A {\bf \$a statement}\index{\texttt{\$a} statement} consists of a label,
followed by \texttt{\$a}, followed by its typecode (an active constant),
followed by
zero or more active math symbols, followed by the \texttt{\$.}\ token.  A {\bf
\$p statement}\index{\texttt{\$p} statement} consists of a label,
followed by \texttt{\$p}, followed by its typecode (an active constant),
followed by
zero or more active math symbols, followed by \texttt{\$=}, followed by
a sequence of labels, followed by the \texttt{\$.}\ token.  An {\bf
assertion}\index{assertion} is a \texttt{\$a} or \texttt{\$p} statement.

A \texttt{\$f}, \texttt{\$e}, or \texttt{\$d} statement is {\bf active}\index{active
statement} from the place it occurs until the end of the block it occurs in.
A \texttt{\$a} or \texttt{\$p} statement is {\bf active} from the place it occurs
through the end of the database.
There may not be two active \texttt{\$f} statements containing the same
variable.  Each variable in a \texttt{\$e}, \texttt{\$a}, or
\texttt{\$p} statement must exist in an active \texttt{\$f}
statement.\footnote{This requirement can greatly simplify the
unification algorithm (substitution calculation) required by proof
verification.}

%The label that begins each \texttt{\$f}, \texttt{\$e}, \texttt{\$a}, and
%\texttt{\$p} statement must be unique.
Each label token must be unique, and
no label token may match any math symbol
token.\label{namespace}\footnote{This
restriction was added on June 24, 2006.
It is not theoretically necessary but is imposed to make it easier to
write certain parsers.}

The set of {\bf mandatory variables}\index{mandatory variable} associated with
an assertion is the set of (zero or more) variables in the assertion and in any
active \texttt{\$e} statements.  The (possibly empty) set of {\bf mandatory
hypotheses}\index{mandatory hypothesis} is the set of all active \texttt{\$f}
statements containing mandatory variables, together with all active \texttt{\$e}
statements.
The set of {\bf mandatory {\bf \$d} statements}\index{mandatory
disjoint-variable restriction} associated with an assertion are those active
\texttt{\$d} statements whose variables are both among the assertion's
mandatory variables.

\subsection{Proof Verification}\label{spec4}

The sequence of labels between the \texttt{\$=} and \texttt{\$.}\ tokens
in a \texttt{\$p} statement is a {\bf proof}.\index{proof!Metamath} Each
label in a proof must be the label of an active statement other than the
\texttt{\$p} statement itself; thus a label must refer either to an
active hypothesis of the \texttt{\$p} statement or to an earlier
assertion.

An {\bf expression}\index{expression} is a sequence of math symbols. A {\bf
substitution map}\index{substitution map} associates a set of variables with a
set of expressions.  It is acceptable for a variable to be mapped to an
expression containing it.  A {\bf
substitution}\index{substitution!variable}\index{variable substitution} is the
simultaneous replacement of all variables in one or more expressions with the
expressions that the variables map to.

A proof is scanned in order of its label sequence.  If the label refers to an
active hypothesis, the expression in the hypothesis is pushed onto a
stack.\index{stack}\index{RPN stack}  If the label refers to an assertion, a
(unique) substitution must exist that, when made to the mandatory hypotheses
of the referenced assertion, causes them to match the topmost (i.e.\ most
recent) entries of the stack, in order of occurrence of the mandatory
hypotheses, with the topmost stack entry matching the last mandatory
hypothesis of the referenced assertion.  As many stack entries as there are
mandatory hypotheses are then popped from the stack.  The same substitution is
made to the referenced assertion, and the result is pushed onto the stack.
After the last label in the proof is processed, the stack must have a single
entry that matches the expression in the \texttt{\$p} statement containing the
proof.

%c%{\footnotesize\begin{quotation}\index{redeclaration of symbols}
%c%{{\em Comment.}\label{spec4comment} Whenever a math symbol token occurs in a
%c%{\texttt{\$c} or \texttt{\$v} statement, it is considered to designate a distinct new
%c%{symbol, even if the same token was previously declared (and is now inactive).
%c%{Thus a math token declared as a constant in two different blocks is considered
%c%{to designate two distinct constants (even though they have the same name).
%c%{The two constants will not match in a proof that references both blocks.
%c%{However, a proof referencing both blocks is acceptable as long as it doesn't
%c%{require that the constants match.  Similarly, a token declared to be a
%c%{constant for a referenced assertion will not match the same token declared to
%c%{be a variable for the \texttt{\$p} statement containing the proof.  In the case
%c%{of a token declared to be a variable for a referenced assertion, this is not
%c%{an issue since the variable can be substituted with whatever expression is
%c%{needed to achieve the required match.
%c%{\end{quotation}}
%c2%A proof may reference an assertion that contains or whose hypotheses contain a
%c2%constant that is not active for the \texttt{\$p} statement containing the proof.
%c2%However, the final result of the proof may not contain that constant. A proof
%c2%may also reference an assertion that contains or whose hypotheses contain a
%c2%variable that is not active for the \texttt{\$p} statement containing the proof.
%c2%That variable, of course, will be substituted with whatever expression is
%c2%needed to achieve the required match.

A proof may contain a \texttt{?}\ in place of a label to indicate an unknown step
(Section~\ref{unknown}).  A proof verifier may ignore any proof containing
\texttt{?}\ but should warn the user that the proof is incomplete.

A {\bf compressed proof}\index{compressed proof}\index{proof!compressed} is an
alternate proof notation described in Appen\-dix~\ref{compressed}; also see
references to ``compressed proof'' in the Index.  Compressed proofs are a
Metamath language extension which a complete proof verifier should be able to
parse and verify.

\subsubsection{Verifying Disjoint Variable Restrictions}

Each substitution made in a proof must be checked to verify that any
disjoint variable restrictions are satisfied, as follows.

If two variables replaced by a substitution exist in a mandatory \texttt{\$d}
statement\index{\texttt{\$d} statement} of the assertion referenced, the two
expressions resulting from the substitution must satisfy the following
conditions.  First, the two expressions must have no variables in common.
Second, each possible pair of variables, one from each expression, must exist
in an active \texttt{\$d} statement of the \texttt{\$p} statement containing the
proof.

\vskip 1ex

This ends the specification of the Metamath language;
see Appendix \ref{BNF} for its syntax in
Extended Backus--Naur Form (EBNF)\index{Extended Backus--Naur Form}\index{EBNF}.

\section{The Basic Keywords}\label{tut1}

Our expository material begins here.

Like most computer languages, Metamath\index{Metamath} takes its input from
one or more {\bf source files}\index{source file} which contain characters
expressed in the standard {\sc ascii} (American Standard Code for Information
Interchange)\index{ascii@{\sc ascii}} code for computers.  A source file
consists of a series of {\bf tokens}\index{token}, which are strings of
non-whitespace
printable characters (from the set of 94 shown on p.~\pageref{spec1chars})
separated by {\bf white space}\index{white space} (spaces, tabs, carriage
returns, line feeds, and form feeds). Any string consisting only of these
characters is treated the same as a single space.  The non-whitespace printable
characters\index{printable character} that Metamath recognizes are the 94
characters on standard {\sc ascii} keyboards.

Metamath has the ability to join several files together to form its
input (Section~\ref{include}).  We call the aggregate contents of all
the files after they have been joined together a {\bf
database}\index{database} to distinguish it from an individual source
file.  The tokens in a database consist of {\bf
keywords}\index{keyword}, which are built into the language, together
with two kinds of user-defined tokens called {\bf labels}\index{label}
and {\bf math symbols}\index{math symbol}.  (Often we will simply say
{\bf symbol}\index{symbol} instead of math symbol for brevity).  The set
of {\bf basic keywords}\index{basic keyword} is
\texttt{\$c}\index{\texttt{\$c} statement},
\texttt{\$v}\index{\texttt{\$v} statement},
\texttt{\$e}\index{\texttt{\$e} statement},
\texttt{\$f}\index{\texttt{\$f} statement},
\texttt{\$d}\index{\texttt{\$d} statement},
\texttt{\$a}\index{\texttt{\$a} statement},
\texttt{\$p}\index{\texttt{\$p} statement},
\texttt{\$=}\index{\texttt{\$=} keyword},
\texttt{\$.}\index{\texttt{\$.}\ keyword},
\texttt{\$\char`\{}\index{\texttt{\$\char`\{} and \texttt{\$\char`\}}
keywords}, and \texttt{\$\char`\}}.  This is the complete set of
syntactical elements of what we call the {\bf basic
language}\index{basic language} of Metamath, and with them you can
express all of the mathematics that were intended by the design of
Metamath.  You should make it a point to become very familiar with them.
Table~\ref{basickeywords} lists the basic keywords along with a brief
description of their functions.  For now, this description will give you
only a vague notion of what the keywords are for; later we will describe
the keywords in detail.


\begin{table}[htp] \caption{Summary of the basic Metamath
keywords} \label{basickeywords}
\begin{center}
\begin{tabular}{|p{4pc}|l|}
\hline
\em \centering Keyword&\em Description\\
\hline
\hline
\centering
   \texttt{\$c}&Constant symbol declaration\\
\hline
\centering
   \texttt{\$v}&Variable symbol declaration\\
\hline
\centering
   \texttt{\$d}&Disjoint variable restriction\\
\hline
\centering
   \texttt{\$f}&Variable-type (``floating'') hypothesis\\
\hline
\centering
   \texttt{\$e}&Logical (``essential'') hypothesis\\
\hline
\centering
   \texttt{\$a}&Axiomatic assertion\\
\hline
\centering
   \texttt{\$p}&Provable assertion\\
\hline
\centering
   \texttt{\$=}&Start of proof in \texttt{\$p} statement\\
\hline
\centering
   \texttt{\$.}&End of the above statement types\\
\hline
\centering
   \texttt{\$\char`\{}&Start of block\\
\hline
\centering
   \texttt{\$\char`\}}&End of block\\
\hline
\end{tabular}
\end{center}
\end{table}

%For LaTeX bug(?) where it puts tables on blank page instead of btwn text
%May have to adjust if text changes
%\newpage

There are some additional keywords, called {\bf auxiliary
keywords}\index{auxiliary keyword} that help make Metamath\index{Metamath}
more practical. These are part of the {\bf extended language}\index{extended
language}. They provide you with a means to put comments into a Metamath
source file\index{source file} and reference other source files.  We will
introduce these in later sections. Table~\ref{otherkeywords} summarizes them
so that you can recognize them now if you want to peruse some source
files while learning the basic keywords.


\begin{table}[htp] \caption{Auxiliary Metamath
keywords} \label{otherkeywords}
\begin{center}
\begin{tabular}{|p{4pc}|l|}
\hline
\em \centering Keyword&\em Description\\
\hline
\hline
\centering
   \texttt{\$(}&Start of comment\\
\hline
\centering
   \texttt{\$)}&End of comment\\
\hline
\centering
   \texttt{\$[}&Start of included source file name\\
\hline
\centering
   \texttt{\$]}&End of included source file name\\
\hline
\end{tabular}
\end{center}
\end{table}
\index{\texttt{\$(} and \texttt{\$)} auxiliary keywords}
\index{\texttt{\$[} and \texttt{\$]} auxiliary keywords}


Unlike those in some computer languages, the keywords\index{keyword} are short
two-character sequences rather than English-like words.  While this may make
them slightly more difficult to remember at first, their brevity allows
them to blend in with the mathematics being described, not
distract from it, like punctuation marks.


\subsection{User-Defined Tokens}\label{dollardollar}\index{token}

As you may have noticed, all keywords\index{keyword} begin with the \texttt{\$}
character.  This mundane monetary symbol is not ordinarily used in higher
mathematics (outside of grant proposals), so we have appropriated it to
distinguish the Metamath\index{Metamath} keywords from ordinary mathematical
symbols. The \texttt{\$} character is thus considered special and may not be
used as a character in a user-defined token.  All tokens and keywords are
case-sensitive; for example, \texttt{n} is considered to be a different character
from \texttt{N}.  Case-sensitivity makes the available {\sc ascii} character set
as rich as possible.

\subsubsection{Math Symbol Tokens}\index{token}

Math symbols\index{math symbol} are tokens used to represent the symbols
that appear in ordinary mathematical formulas.  They may consist of any
combination of the 93 non-whitespace printable {\sc ascii} characters other than
\texttt{\$}~. Some examples are \texttt{x}, \texttt{+}, \texttt{(},
\texttt{|-}, \verb$!%@?&$, and \texttt{bounded}.  For readability, it is
best to try to make these look as similar to actual mathematical symbols
as possible, within the constraints of the {\sc ascii} character set, in
order to make the resulting mathematical expressions more readable.

In the Metamath\index{Metamath} language, you express ordinary
mathematical formulas and statements as sequences of math symbols such
as \texttt{2 + 2 = 4} (five symbols, all constants).\footnote{To
eliminate ambiguity with other expressions, this is expressed in the set
theory database \texttt{set.mm} as \texttt{|- ( 2 + 2
 ) = 4 }, whose \LaTeX\ equivalent is $\vdash
(2+2)=4$.  The \,$\vdash$ means ``is a theorem'' and the
parentheses allow explicit associative grouping.}\index{turnstile
({$\,\vdash$})} They may even be English
sentences, as in \texttt{E is closed and bounded} (five symbols)---here
\texttt{E} would be a variable and the other four symbols constants.  In
principle, a Metamath database could be constructed to work with almost
any unambiguous English-language mathematical statement, but as a
practical matter the definitions needed to provide for all possible
syntax variations would be cumbersome and distracting and possibly have
subtle pitfalls accidentally built in.  We generally recommend that you
express mathematical statements with compact standard mathematical
symbols whenever possible and put their English-language descriptions in
comments.  Axioms\index{axiom} and definitions\index{definition}
(\texttt{\$a}\index{\texttt{\$a} statement} statements) are the only
places where Metamath will not detect an error, and doing this will help
reduce the number of definitions needed.

You are free to use any tokens\index{token} you like for math
symbols\index{math symbol}.  Appendix~\ref{ASCII} recommends token names to
use for symbols in set theory, and we suggest you adopt these in order to be
able to include the \texttt{set.mm} set theory database in your database.  For
printouts, you can convert the tokens in a database
to standard mathematical symbols with the \LaTeX\ typesetting program.  The
Metamath command \texttt{open tex} {\em filename}\index{\texttt{open tex} command}
produces output that can be read by \LaTeX.\index{latex@{\LaTeX}}
The correspondence
between tokens and the actual symbols is made by \texttt{latexdef}
statements inside a special database comment tagged
with \texttt{\$t}.\index{\texttt{\$t} comment}\index{typesetting comment}
  You can edit
this comment to change the definitions or add new ones.
Appendix~\ref{ASCII} describes how to do this in more detail.

% White space\index{white space} is normally used to separate math
% symbol\index{math symbol} tokens, but they may be juxtaposed without white
% space in \texttt{\$d}\index{\texttt{\$d} statement}, \texttt{\$e}\index{\texttt{\$e}
% statement}, \texttt{\$f}\index{\texttt{\$f} statement}, \texttt{\$a}\index{\texttt{\$a}
% statement}, and \texttt{\$p}\index{\texttt{\$p} statement} statements when no
% ambiguity will result.  Specifically, Metamath parses the math symbol sequence
% in one of these statements in the following manner:  when the math symbol
% sequence has been broken up into tokens\index{token} up to a given character,
% the next token is the longest string of characters that could constitute a
% math symbol that is active\index{active
% math symbol} at that point.  (See Section~\ref{scoping} for the
% definition of an active math symbol.)  For example, if \texttt{-}, \texttt{>}, and
% \texttt{->} are the only active math symbols, the juxtaposition \texttt{>-} will be
% interpreted as the two symbols \texttt{>} and \texttt{-}, whereas \texttt{->} will
% always be interpreted as that single symbol.\footnote{For better readability we
% recommend a white space between each token.  This also makes searching for a
% symbol easier to do with an editor.  Omission of optional white space is useful
% for reducing typing when assigning an expression to a temporary
% variable\index{temporary variable} with the \texttt{let variable} Metamath
% program command.}\index{\texttt{let variable} command}
%
% Keywords\index{keyword} may be placed next to math symbols without white
% space\index{white space} between them.\footnote{Again, we do not recommend
% this for readability.}
%
% The math symbols\index{math symbol} in \texttt{\$c}\index{\texttt{\$c} statement}
% and \texttt{\$v}\index{\texttt{\$v} statement} statements must always be separated
% by white space\index{white
% space}, for the obvious reason that these statements define the names
% of the symbols.
%
% Math symbols referred to in comments (see Section~\ref{comments}) must also be
% separated by white space.  This allows you to make comments about symbols that
% are not yet active\index{active
% math symbol}.  (The ``math mode'' feature of comments is also a quick and
% easy way to obtain word processing text with embedded mathematical symbols,
% independently of the main purpose of Metamath; the way to do this is described
% in Section~\ref{comments})

\subsubsection{Label Tokens}\index{token}\index{label}

Label tokens are used to identify Metamath\index{Metamath} statements for
later reference. Label tokens may contain only letters, digits, and the three
characters period, hyphen, and underscore:
\begin{verbatim}
. - _
\end{verbatim}

A label is {\bf declared}\index{label declaration} by placing it immediately
before the keyword of the statement it identifies.  For example, the label
\texttt{axiom.1} might be declared as follows:
\begin{verbatim}
axiom.1 $a |- x = x $.
\end{verbatim}

Each \texttt{\$e}\index{\texttt{\$e} statement},
\texttt{\$f}\index{\texttt{\$f} statement},
\texttt{\$a}\index{\texttt{\$a} statement}, and
\texttt{\$p}\index{\texttt{\$p} statement} statement in a database must
have a label declared for it.  No other statement types may have label
declarations.  Every label must be unique.

A label (and the statement it identifies) is {\bf referenced}\index{label
reference} by including the label between the \texttt{\$=}\index{\texttt{\$=}
keyword} and \texttt{\$.}\index{\texttt{\$.}\ keyword}\ keywords in a \texttt{\$p}
statement.  The sequence of labels\index{label sequence} between these two
keywords is called a {\bf proof}\index{proof}.  An example of a statement with
a proof that we will encounter later (Section~\ref{proof}) is
\begin{verbatim}
wnew $p wff ( s -> ( r -> p ) )
     $= ws wr wp w2 w2 $.
\end{verbatim}

You don't have to know what this means just yet, but you should know that the
label \texttt{wnew} is declared by this \texttt{\$p} statement and that the labels
\texttt{ws}, \texttt{wr}, \texttt{wp}, and \texttt{w2} are assumed to have been declared
earlier in the database and are referenced here.

\subsection{Constants and Variables}
\index{constant}
\index{variable}

An {\bf expression}\index{expression} is any sequence of math
symbols, possibly empty.

The basic Metamath\index{Metamath} language\index{basic language} has two
kinds of math symbols\index{math symbol}:  {\bf constants}\index{constant} and
{\bf variables}\index{variable}.  In a Metamath proof, a constant may not be
substituted with any expression.  A variable can be
substituted\index{substitution!variable}\index{variable substitution} with any
expression.  This sequence may include other variables and may even include
the variable being substituted.  This substitution takes place when proofs are
verified, and it will be described in Section~\ref{proof}.  The \texttt{\$f}
statement (described later in Section~\ref{dollaref}) is used to specify the
{\bf type} of a variable (i.e.\ what kind of
variable it is)\index{variable type}\index{type} and
give it a meaning typically
associated with a ``metavariable''\index{metavariable}\footnote{A metavariable
is a variable that ranges over the syntactical elements of the object language
being discussed; for example, one metavariable might represent a variable of
the object language and another metavariable might represent a formula in the
object language.} in ordinary mathematics; for example, a variable may be
specified to be a wff or well-formed formula (in logic), a set (in set
theory), or a non-negative integer (in number theory).

%\subsection{The \texttt{\$c} and \texttt{\$v} Declaration Statements}
\subsection{The \texttt{\$c} and \texttt{\$v} Declaration Statements}
\index{\texttt{\$c} statement}
\index{constant declaration}
\index{\texttt{\$v} statement}
\index{variable declaration}

Constants are introduced or {\bf declared}\index{constant declaration}
with \texttt{\$c}\index{\texttt{\$c} statement} statements, and
variables are declared\index{variable declaration} with
\texttt{\$v}\index{\texttt{\$v} statement} statements.  A {\bf simple}
declaration\index{simple declaration} statement introduces a single
constant or variable.  Its syntax is one of the following:
\begin{center}
  \texttt{\$c} {\em math-symbol} \texttt{\$.}\\
  \texttt{\$v} {\em math-symbol} \texttt{\$.}
\end{center}
The notation {\em math-symbol} means any math symbol token\index{token}.

Some examples of simple declaration statements are:
\begin{center}
  \texttt{\$c + \$.}\\
  \texttt{\$c -> \$.}\\
  \texttt{\$c ( \$.}\\
  \texttt{\$v x \$.}\\
  \texttt{\$v y2 \$.}
\end{center}

The characters in a math symbol\index{math symbol} being declared are
irrelevant to Meta\-math; for example, we could declare a right parenthesis to
be a variable,
\begin{center}
  \texttt{\$v ) \$.}\\
\end{center}
although this would be unconventional.

A {\bf compound} declaration\index{compound declaration} statement is a
shorthand for declaring several symbols at once.  Its syntax is one of the
following:
\begin{center}
  \texttt{\$c} {\em math-symbol}\ \,$\cdots$\ {\em math-symbol} \texttt{\$.}\\
  \texttt{\$v} {\em math-symbol}\ \,$\cdots$\ {\em math-symbol} \texttt{\$.}
\end{center}\index{\texttt{\$c} statement}
Here, the ellipsis (\ldots) means any number of {\em math-symbol}\,s.

An example of a compound declaration statement is:
\begin{center}
  \texttt{\$v x y mu \$.}\\
\end{center}
This is equivalent to the three simple declaration statements
\begin{center}
  \texttt{\$v x \$.}\\
  \texttt{\$v y \$.}\\
  \texttt{\$v mu \$.}\\
\end{center}
\index{\texttt{\$v} statement}

There are certain rules on where in the database math symbols may be declared,
what sections of the database are aware of them (i.e.\ where they are
``active''), and when they may be declared more than once.  These will be
discussed in Section~\ref{scoping} and specifically on
p.~\pageref{redeclaration}.

\subsection{The \texttt{\$d} Statement}\label{dollard}
\index{\texttt{\$d} statement}

The \texttt{\$d} statement is called a {\bf disjoint-variable restriction}.  The
syntax of the {\bf simple} version of this statement is
\begin{center}
  \texttt{\$d} {\em variable variable} \texttt{\$.}
\end{center}
where each {\em variable} is a previously declared variable and the two {\em
variable}\,s are different.  (More specifically, each  {\em variable} must be
an {\bf active} variable\index{active math symbol}, which means there must be
a previous \texttt{\$v} statement whose {\bf scope}\index{scope} includes the
\texttt{\$d} statement.  These terms will be defined when we discuss scoping
statements in Section~\ref{scoping}.)

In ordinary mathematics, formulas may arise that are true if the variables in
them are distinct\index{distinct variables}, but become false when those
variables are made identical. For example, the formula in logic $\exists x\,x
\neq y$, which means ``for a given $y$, there exists an $x$ that is not equal
to $y$,'' is true in most mathematical theories (namely all non-trivial
theories\index{non-trivial theory}, i.e.\ those that describe more than one
individual, such as arithmetic).  However, if we substitute $y$ with $x$, we
obtain $\exists x\,x \neq x$, which is always false, as it means ``there
exists something that is not equal to itself.''\footnote{If you are a
logician, you will recognize this as the improper substitution\index{proper
substitution}\index{substitution!proper} of a free variable\index{free
variable} with a bound variable\index{bound variable}.  Metamath makes no
inherent distinction between free and bound variables; instead, you let
Metamath know what substitutions are permissible by using \texttt{\$d} statements
in the right way in your axiom system.}\index{free vs.\ bound variable}  The
\texttt{\$d} statement allows you to specify a restriction that forbids the
substitution of one variable with another.  In
this case, we would use the statement
\begin{center}
  \texttt{\$d x y \$.}
\end{center}\index{\texttt{\$d} statement}
to specify this restriction.

The order in which the variables appear in a \texttt{\$d} statement is not
important.  We could also use
\begin{center}
  \texttt{\$d y x \$.}
\end{center}

The \texttt{\$d} statement is actually more general than this, as the
``disjoint''\index{disjoint variables} in its name suggests.  The full meaning
is that if any substitution is made to its two variables (during the
course of a proof that references a \texttt{\$a} or \texttt{\$p} statement
associated with the \texttt{\$d}), the two expressions that result from the
substitution must have no variables in common.  In addition, each possible
pair of variables, one from each expression, must be in a \texttt{\$d} statement
associated with the statement being proved.  (This requirement forces the
statement being proved to ``inherit'' the original disjoint variable
restriction.)

For example, suppose \texttt{u} is a variable.  If the restriction
\begin{center}
  \texttt{\$d A B \$.}
\end{center}
has been specified for a theorem referenced in a
proof, we may not substitute \texttt{A} with \mbox{\tt a + u} and
\texttt{B} with \mbox{\tt b + u} because these two symbol sequences have the
variable \texttt{u} in common.  Furthermore, if \texttt{a} and \texttt{b} are
variables, we may not substitute \texttt{A} with \texttt{a} and \texttt{B} with \texttt{b}
unless we have also specified \texttt{\$d a b} for the theorem being proved; in
other words, the \texttt{\$d} property associated with a pair of variables must
be effectively preserved after substitution.

The \texttt{\$d}\index{\texttt{\$d} statement} statement does {\em not} mean ``the
two variables may not be substituted with the same thing,'' as you might think
at first.  For example, substituting each of \texttt{A} and \texttt{B} in the above
example with identical symbol sequences consisting only of constants does not
cause a disjoint variable conflict, because two symbol sequences have no
variables in common (since they have no variables, period).  Similarly, a
conflict will not occur by substituting the two variables in a \texttt{\$d}
statement with the empty symbol sequence\index{empty substitution}.

The \texttt{\$d} statement does not have a direct counterpart in
ordinary mathematics, partly because the variables\index{variable} of
Metamath are not really the same as the variables\index{variable!in
ordinary mathematics} of ordinary mathematics but rather are
metavariables\index{metavariable} ranging over them (as well as over
other kinds of symbols and groups of symbols).  Depending on the
situation, we may informally interpret the \texttt{\$d} statement in
different ways.  Suppose, for example, that \texttt{x} and \texttt{y}
are variables ranging over numbers (more precisely, that \texttt{x} and
\texttt{y} are metavariables ranging over variables that range over
numbers), and that \texttt{ph} ($\varphi$) and \texttt{ps} ($\psi$) are
variables (more precisely, metavariables) ranging over formulas.  We can
make the following interpretations that correspond to the informal
language of ordinary mathematics:
\begin{quote}
\begin{tabbing}
\texttt{\$d x y \$.} means ``assume $x$ and $y$ are
distinct variables.''\\
\texttt{\$d x ph \$.} means ``assume $x$ does not
occur in $\varphi$.''\\
\texttt{\$d ph ps \$.} \=means ``assume $\varphi$ and
$\psi$ have no variables\\ \>in common.''
\end{tabbing}
\end{quote}\index{\texttt{\$d} statement}

\subsubsection{Compound \texttt{\$d} Statements}

The {\bf compound} version of the \texttt{\$d} statement is a shorthand for
specifying several variables whose substitutions must be pairwise disjoint.
Its syntax is:
\begin{center}
  \texttt{\$d} {\em variable}\ \,$\cdots$\ {\em variable} \texttt{\$.}
\end{center}\index{\texttt{\$d} statement}
Here, {\em variable} represents the token of a previously declared
variable (specifically, an active variable) and all {\em variable}\,s are
different.  The compound \texttt{\$d}
statement is internally broken up by Metamath into one simple \texttt{\$d}
statement for each possible pair of variables in the original \texttt{\$d}
statement.  For example,
\begin{center}
  \texttt{\$d w x y z \$.}
\end{center}
is equivalent to
\begin{center}
  \texttt{\$d w x \$.}\\
  \texttt{\$d w y \$.}\\
  \texttt{\$d w z \$.}\\
  \texttt{\$d x y \$.}\\
  \texttt{\$d x z \$.}\\
  \texttt{\$d y z \$.}
\end{center}

Two or more simple \texttt{\$d} statements specifying the same variable pair are
internally combined into a single \texttt{\$d} statement.  Thus the set of three
statements
\begin{center}
  \texttt{\$d x y \$.}
  \texttt{\$d x y \$.}
  \texttt{\$d y x \$.}
\end{center}
is equivalent to
\begin{center}
  \texttt{\$d x y \$.}
\end{center}

Similarly, compound \texttt{\$d} statements, after being internally broken up,
internally have their common variable pairs combined.  For example the
set of statements
\begin{center}
  \texttt{\$d x y A \$.}
  \texttt{\$d x y B \$.}
\end{center}
is equivalent to
\begin{center}
  \texttt{\$d x y \$.}
  \texttt{\$d x A \$.}
  \texttt{\$d y A \$.}
  \texttt{\$d x y \$.}
  \texttt{\$d x B \$.}
  \texttt{\$d y B \$.}
\end{center}
which is equivalent to
\begin{center}
  \texttt{\$d x y \$.}
  \texttt{\$d x A \$.}
  \texttt{\$d y A \$.}
  \texttt{\$d x B \$.}
  \texttt{\$d y B \$.}
\end{center}

Metamath\index{Metamath} automatically verifies that all \texttt{\$d}
restrictions are met whenever it verifies proofs.  \texttt{\$d} statements are
never referenced directly in proofs (this is why they do not have
labels\index{label}), but Metamath is always aware of which ones must be
satisfied (i.e.\ are active) and will notify you with an error message if any
violation occurs.

To illustrate how Metamath detects a missing \texttt{\$d}
statement, we will look at the following example from the
\texttt{set.mm} database.

\begin{verbatim}
$d x z $.  $d y z $.
$( Theorem to add distinct quantifier to atomic formula. $)
ax17eq $p |- ( x = y -> A. z x = y ) $=...
\end{verbatim}

This statement has the obvious requirement that $z$ must be
distinct\index{distinct variables} from $x$ in theorem \texttt{ax17eq} that
states $x=y \rightarrow \forall z \, x=y$ (well, obvious if you're a logician,
for otherwise we could conclude  $x=y \rightarrow \forall x \, x=y$, which is
false when the free variables $x$ and $y$ are equal).

Let's look at what happens if we edit the database to comment out this
requirement.

\begin{verbatim}
$( $d x z $. $) $d y z $.
$( Theorem to add distinct quantifier to atomic formula. $)
ax17eq $p |- ( x = y -> A. z x = y ) $=...
\end{verbatim}

When it tries to verify the proof, Metamath will tell you that \texttt{x} and
\texttt{z} must be disjoint, because one of its steps references an axiom or
theorem that has this requirement.

\begin{verbatim}
MM> verify proof ax17eq
ax17eq ?Error at statement 1918, label "ax17eq", type "$p":
      vz wal wi vx vy vz ax-13 vx vy weq vz vx ax-c16 vx vy
                                               ^^^^^
There is a disjoint variable ($d) violation at proof step 29.
Assertion "ax-c16" requires that variables "x" and "y" be
disjoint.  But "x" was substituted with "z" and "y" was
substituted with "x".  The assertion being proved, "ax17eq",
does not require that variables "z" and "x" be disjoint.
\end{verbatim}

We can see the substitutions into \texttt{ax-c16} with the following command.

\begin{verbatim}
MM> show proof ax17eq / detailed_step 29
Proof step 29:  pm2.61dd.2=ax-c16 $a |- ( A. z z = x -> ( x =
  y -> A. z x = y ) )
This step assigns source "ax-c16" ($a) to target "pm2.61dd.2"
($e).  The source assertion requires the hypotheses "wph"
($f, step 26), "vx" ($f, step 27), and "vy" ($f, step 28).
The parent assertion of the target hypothesis is "pm2.61dd"
($p, step 36).
The source assertion before substitution was:
    ax-c16 $a |- ( A. x x = y -> ( ph -> A. x ph ) )
The following substitutions were made to the source
assertion:
    Variable  Substituted with
     x         z
     y         x
     ph        x = y
The target hypothesis before substitution was:
    pm2.61dd.2 $e |- ( ph -> ch )
The following substitutions were made to the target
hypothesis:
    Variable  Substituted with
     ph        A. z z = x
     ch        ( x = y -> A. z x = y )
\end{verbatim}

The disjoint variable restrictions of \texttt{ax-c16} can be seen from the
\texttt{show state\-ment} command.  The line that begins ``\texttt{Its mandatory
dis\-joint var\-i\-able pairs are:}\ldots'' lists any \texttt{\$d} variable
pairs in brackets.

\begin{verbatim}
MM> show statement ax-c16/full
Statement 3033 is located on line 9338 of the file "set.mm".
"Axiom of Distinct Variables. ..."
  ax-c16 $a |- ( A. x x = y -> ( ph -> A. x ph ) ) $.
Its mandatory hypotheses in RPN order are:
  wph $f wff ph $.
  vx $f setvar x $.
  vy $f setvar y $.
Its mandatory disjoint variable pairs are:  <x,y>
The statement and its hypotheses require the variables:  x y
      ph
The variables it contains are:  x y ph
\end{verbatim}

Since Metamath will always detect when \texttt{\$d}\index{\texttt{\$d} statement}
statements are needed for a proof, you don't have to worry too much about
forgetting to put one in; it can always be added if you see the error message
above.  If you put in unnecessary \texttt{\$d} statements, the worst that could
happen is that your theorem might not be as general as it could be, and this
may limit its use later on.

On the other hand, when you introduce axioms (\texttt{\$a}\index{\texttt{\$a}
statement} statements), you must be very careful to properly specify the
necessary associated \texttt{\$d} statements since Metamath has no way of knowing
whether your axioms are correct.  For example, Metamath would have no idea
that \texttt{ax-c16}, which we are telling it is an axiom of logic, would lead to
contradictions if we omitted its associated \texttt{\$d} statement.

% This was previously a comment in footnote-sized type, but it can be
% hard to read this much text in a small size.
% As a result, it's been changed to normally-sized text.
\label{nodd}
You may wonder if it is possible to develop standard
mathematics in the Metamath language without the \texttt{\$d}\index{\texttt{\$d}
statement} statement, since it seems like a nuisance that complicates proof
verification. The \texttt{\$d} statement is not needed in certain subsets of
mathematics such as propositional calculus.  However, dummy
variables\index{dummy variable!eliminating} and their associated \texttt{\$d}
statements are impossible to avoid in proofs in standard first-order logic as
well as in the variant used in \texttt{set.mm}.  In fact, there is no upper bound to
the number of dummy variables that might be needed in a proof of a theorem of
first-order logic containing 3 or more variables, as shown by H.\
Andr\'{e}ka\index{Andr{\'{e}}ka, H.} \cite{Nemeti}.  A first-order system that
avoids them entirely is given in \cite{Megill}\index{Megill, Norman}; the
trick there is simply to embed harmlessly the necessary dummy variables into a
theorem being proved so that they aren't ``dummy'' anymore, then interpret the
resulting longer theorem so as to ignore the embedded dummy variables.  If
this interests you, the system in \texttt{set.mm} obtained from \texttt{ax-1}
through \texttt{ax-c14} in \texttt{set.mm}, and deleting \texttt{ax-c16} and \texttt{ax-5},
requires no \texttt{\$d} statements but is logically complete in the sense
described in \cite{Megill}.  This means it can prove any theorem of
first-order logic as long as we add to the theorem an antecedent that embeds
dummy and any other variables that must be distinct.  In a similar fashion,
axioms for set theory can be devised that
do not require distinct variable
provisos\index{Set theory without distinct variable provisos},
as explained at
\url{http://us.metamath.org/mpeuni/mmzfcnd.html}.
Together, these in principle allow all of
mathematics to be developed under Metamath without a \texttt{\$d} statement,
although the length of the resulting theorems will grow as more and
more dummy variables become required in their proofs.

\subsection{The \texttt{\$f}
and \texttt{\$e} Statements}\label{dollaref}
\index{\texttt{\$e} statement}
\index{\texttt{\$f} statement}
\index{floating hypothesis}
\index{essential hypothesis}
\index{variable-type hypothesis}
\index{logical hypothesis}
\index{hypothesis}

Metamath has two kinds of hypo\-theses, the \texttt{\$f}\index{\texttt{\$f}
statement} or {\bf variable-type} hypothesis and the \texttt{\$e} or {\bf logical}
hypo\-the\-sis.\index{\texttt{\$d} statement}\footnote{Strictly speaking, the
\texttt{\$d} statement is also a hypothesis, but it is never directly referenced
in a proof, so we call it a restriction rather than a hypothesis to lessen
confusion.  The checking for violations of \texttt{\$d} restrictions is automatic
and built into Metamath's proof-checking algorithm.} The letters \texttt{f} and
\texttt{e} stand for ``floating''\index{floating hypothesis} (roughly meaning
used only if relevant) and ``essential''\index{essential hypothesis} (meaning
always used) respectively, for reasons that will become apparent
when we discuss frames in
Section~\ref{frames} and scoping in Section~\ref{scoping}. The syntax of these
are as follows:
\begin{center}
  {\em label} \texttt{\$f} {\em typecode} {\em variable} \texttt{\$.}\\
  {\em label} \texttt{\$e} {\em typecode}
      {\em math-symbol}\ \,$\cdots$\ {\em math-symbol} \texttt{\$.}\\
\end{center}
\index{\texttt{\$e} statement}
\index{\texttt{\$f} statement}
A hypothesis must have a {\em label}\index{label}.  The expression in a
\texttt{\$e} hypothesis consists of a typecode (an active constant math symbol)
followed by a sequence
of zero or more math symbols. Each math symbol (including {\em constant}
and {\em variable}) must be a previously declared constant or variable.  (In
addition, each math symbol must be active, which will be covered when we
discuss scoping statements in Section~\ref{scoping}.)  You use a \texttt{\$f}
hypothesis to specify the
nature or {\bf type}\index{variable type}\index{type} of a variable (such as ``let $x$ be an
integer'') and use a \texttt{\$e} hypothesis to express a logical truth (such as
``assume $x$ is prime'') that must be established in order for an assertion
requiring it to also be true.

A variable must have its type specified in a \texttt{\$f} statement before
it may be used in a \texttt{\$e}, \texttt{\$a}, or \texttt{\$p}
statement.  There may be only one (active) \texttt{\$f} statement for a
given variable.  (``Active'' is defined in Section~\ref{scoping}.)

In ordinary mathematics, theorems\index{theorem} are often expressed in the
form ``Assume $P$; then $Q$,'' where $Q$ is a statement that you can derive
if you start with statement $P$.\index{free variable}\footnote{A stronger
version of a theorem like this would be the {\em single} formula $P\rightarrow
Q$ ($P$ implies $Q$) from which the weaker version above follows by the rule
of modus ponens in logic.  We are not discussing this stronger form here.  In
the weaker form, we are saying only that if we can {\em prove} $P$, then we can
{\em prove} $Q$.  In a logician's language, if $x$ is the only free variable
in $P$ and $Q$, the stronger form is equivalent to $\forall x ( P \rightarrow
Q)$ (for all $x$, $P$ implies $Q$), whereas the weaker form is equivalent to
$\forall x P \rightarrow \forall x Q$. The stronger form implies the weaker,
but not vice-versa.  To be precise, the weaker form of the theorem is more
properly called an ``inference'' rather than a theorem.}\index{inference}
In the
Metamath\index{Metamath} language, you would express mathematical statement
$P$ as a hypothesis (a \texttt{\$e} Metamath language statement in this case) and
statement $Q$ as a provable assertion (a \texttt{\$p}\index{\texttt{\$p} statement}
statement).

Some examples of hypotheses you might encounter in logic and set theory are
\begin{center}
  \texttt{stmt1 \$f wff P \$.}\\
  \texttt{stmt2 \$f setvar x \$.}\\
  \texttt{stmt3 \$e |- ( P -> Q ) \$.}
\end{center}
\index{\texttt{\$e} statement}
\index{\texttt{\$f} statement}
Informally, these would be read, ``Let $P$ be a well-formed-formula,'' ``Let
$x$ be an (individual) variable,'' and ``Assume we have proved $P \rightarrow
Q$.''  The turnstile symbol \,$\vdash$\index{turnstile ({$\,\vdash$})} is
commonly used in logic texts to mean ``a proof exists for.''

To summarize:
\begin{itemize}
\item A \texttt{\$f} hypothesis tells Metamath the type or kind of its variable.
It is analogous to a variable declaration in a computer language that
tells the compiler that a variable is an integer or a floating-point
number.
\item The \texttt{\$e} hypothesis corresponds to what you would usually call a
``hypothesis'' in ordinary mathematics.
\end{itemize}

Before an assertion\index{assertion} (\texttt{\$a} or \texttt{\$p} statement) can be
referenced in a proof, all of its associated \texttt{\$f} and \texttt{\$e} hypotheses
(i.e.\ those \texttt{\$e} hypotheses that are active) must be satisfied (i.e.
established by the proof).  The meaning of ``associated'' (which we will call
{\bf mandatory} in Section~\ref{frames}) will become clear when we discuss
scoping later.

Note that after any \texttt{\$f}, \texttt{\$e},
\texttt{\$a}, or \texttt{\$p} token there is a required
\textit{typecode}\index{typecode}.
The typecode is a constant used to enforce types of expressions.
This will become clearer once we learn more about
assertions (\texttt{\$a} and \texttt{\$p} statements).
An example may also clarify their purpose.
In the
\texttt{set.mm}\index{set theory database (\texttt{set.mm})}%
\index{Metamath Proof Explorer}
database,
the following typecodes are used:

\begin{itemize}
\item \texttt{wff} :
  Well-formed formula (wff) symbol
  (read: ``the following symbol sequence is a wff'').
% The *textual* typecode for turnstile is "|-", but when read it's a little
% confusing, so I intentionally display the mathematical symbol here instead
% (I think it's clearer in this context).
\item \texttt{$\vdash$} :
  Turnstile (read: ``the following symbol sequence is provable'' or
  ``a proof exists for'').
\item \texttt{setvar} :
  Individual set variable type (read: ``the following is an
  individual set variable'').
  Note that this is \textit{not} the type of an arbitrary set expression,
  instead, it is used to ensure that there is only a single symbol used
  after quantifiers like for-all ($\forall$) and there-exists ($\exists$).
\item \texttt{class} :
  An expression that is a syntactically valid class expression.
  All valid set expressions are also valid class expression, so expressions
  of sets normally have the \texttt{class} typecode.
  Use the \texttt{class} typecode,
  \textit{not} the \texttt{setvar} typecode,
  for the type of set expressions unless you are specifically identifying
  a single set variable.
\end{itemize}

\subsection{Assertions (\texttt{\$a} and \texttt{\$p} Statements)}
\index{\texttt{\$a} statement}
\index{\texttt{\$p} statement}\index{assertion}\index{axiomatic assertion}
\index{provable assertion}

There are two types of assertions, \texttt{\$a}\index{\texttt{\$a} statement}
statements ({\bf axiomatic assertions}) and \texttt{\$p} statements ({\bf
provable assertions}).  Their syntax is as follows:
\begin{center}
  {\em label} \texttt{\$a} {\em typecode} {\em math-symbol} \ldots
         {\em math-symbol} \texttt{\$.}\\
  {\em label} \texttt{\$p} {\em typecode} {\em math-symbol} \ldots
        {\em math-symbol} \texttt{\$=} {\em proof} \texttt{\$.}
\end{center}
\index{\texttt{\$a} statement}
\index{\texttt{\$p} statement}
\index{\texttt{\$=} keyword}
An assertion always requires a {\em label}\index{label}. The expression in an
assertion consists of a typecode (an active constant)
followed by a sequence of zero
or more math symbols.  Each math symbol, including any {\em constant}, must be a
previously declared constant or variable.  (In addition, each math symbol
must be active, which will be covered when we discuss scoping statements in
Section~\ref{scoping}.)

A \texttt{\$a} statement is usually a definition of syntax (for example, if $P$
and $Q$ are wffs then so is $(P\to Q)$), an axiom\index{axiom} of ordinary
mathematics (for example, $x=x$), or a definition\index{definition} of
ordinary mathematics (for example, $x\ne y$ means $\lnot x=y$). A \texttt{\$p}
statement is a claim that a certain combination of math symbols follows from
previous assertions and is accompanied by a proof that demonstrates it.

Assertions can also be referenced in (later) proofs in order to derive new
assertions from them. The label of an assertion is used to refer to it in a
proof. Section~\ref{proof} will describe the proof in detail.

Assertions also provide the primary means for communicating the mathematical
results in the database to people.  Proofs (when conveniently displayed)
communicate to people how the results were arrived at.

\subsubsection{The \texttt{\$a} Statement}
\index{\texttt{\$a} statement}

Axiomatic assertions (\texttt{\$a} statements) represent the starting points from
which other assertions (\texttt{\$p}\index{\texttt{\$p} statement} statements) are
derived.  Their most obvious use is for specifying ordinary mathematical
axioms\index{axiom}, but they are also used for two other purposes.

First, Metamath\index{Metamath} needs to know the syntax of symbol
sequences that constitute valid mathematical statements.  A Metamath
proof must be broken down into much more detail than ordinary
mathematical proofs that you may be used to thinking of (even the
``complete'' proofs of formal logic\index{formal logic}).  This is one
of the things that makes Metamath a general-purpose language,
independent of any system of logic or even syntax.  If you want to use a
substitution instance of an assertion as a step in a proof, you must
first prove that the substitution is syntactically correct (or if you
prefer, you must ``construct'' it), showing for example that the
expression you are substituting for a wff metavariable is a valid wff.
The \texttt{\$a}\index{\texttt{\$a} statement} statement is used to
specify those combinations of symbols that are considered syntactically
valid, such as the legal forms of wffs.

Second, \texttt{\$a} statements are used to specify what are ordinarily thought of
as definitions, i.e.\ new combinations of symbols that abbreviate other
combinations of symbols.  Metamath makes no distinction\index{axiom vs.\
definition} between axioms\index{axiom} and definitions\index{definition}.
Indeed, it has been argued that such distinction should not be made even in
ordinary mathematics; see Section~\ref{definitions}, which discusses the
philosophy of definitions.  Section~\ref{hierarchy} discusses some
technical requirements for definitions.  In \texttt{set.mm} we adopt the
convention of prefixing axiom labels with \texttt{ax-} and definition labels with
\texttt{df-}\index{label}.

The results that can be derived with the Metamath language are only as good as
the \texttt{\$a}\index{\texttt{\$a} statement} statements used as their starting
point.  We cannot stress this too strongly.  For example, Metamath will
not prevent you from specifying $x\neq x$ as an axiom of logic.  It is
essential that you scrutinize all \texttt{\$a} statements with great care.
Because they are a source of potential pitfalls, it is best not to add new
ones (usually new definitions) casually; rather you should carefully evaluate
each one's necessity and advantages.

Once you have in place all of the basic axioms\index{axiom} and
rules\index{rule} of a mathematical theory, the only \texttt{\$a} statements that
you will be adding will be what are ordinarily called definitions.  In
principle, definitions should be in some sense eliminable from the language of
a theory according to some convention (usually involving logical equivalence
or equality).  The most common convention is that any formula that was
syntactically valid but not provable before the definition was introduced will
not become provable after the definition is introduced.  In an ideal world,
definitions should not be present at all if one is to have absolute confidence
in a mathematical result.  However, they are necessary to make
mathematics practical, for otherwise the resulting formulas would be
extremely long and incomprehensible.  Since the nature of definitions (in the
most general sense) does not permit them to automatically be verified as
``proper,''\index{proper definition}\index{definition!proper} the judgment of
the mathematician is required to ensure it.  (In \texttt{set.mm} effort was made
to make almost all definitions directly eliminable and thus minimize the need
for such judgment.)

If you are not a mathematician, it may be best not to add or change any
\texttt{\$a}\index{\texttt{\$a} statement} statements but instead use
the mathematical language already provided in standard databases.  This
way Metamath will not allow you to make a mistake (i.e.\ prove a false
result).


\subsection{Frames}\label{frames}

We now introduce the concept of a collection of related Metamath statements
called a frame.  Every assertion (\texttt{\$a} or \texttt{\$p} statement) in the database has
an associated frame.

A {\bf frame}\index{frame} is a sequence of \texttt{\$d}, \texttt{\$f},
and \texttt{\$e} statements (zero or more of each) followed by one
\texttt{\$a} or \texttt{\$p} statement, subject to certain conditions we
will describe.  For simplicity we will assume that all math symbol
tokens used are declared at the beginning of the database with
\texttt{\$c} and \texttt{\$v} statements (which are not properly part of
a frame).  Also for simplicity we will assume there are only simple
\texttt{\$d} statements (those with only two variables) and imagine any
compound \texttt{\$d} statements (those with more than two variables) as
broken up into simple ones.

A frame groups together those hypotheses (and \texttt{\$d} statements) relevant
to an assertion (\texttt{\$a} or \texttt{\$p} statement).  The statements in a frame
may or may not be physically adjacent in a database; we will cover
this in our discussion of scoping statements
in Section~\ref{scoping}.

A frame has the following properties:
\begin{enumerate}
 \item The set of variables contained in its \texttt{\$f} statements must
be identical to the set of variables contained in its \texttt{\$e},
\texttt{\$a}, and/or \texttt{\$p} statements.  In other words, each
variable in a \texttt{\$e}, \texttt{\$a}, or \texttt{\$p} statement must
have an associated ``variable type'' defined for it in a \texttt{\$f}
statement.
  \item No two \texttt{\$f} statements may contain the same variable.
  \item Any \texttt{\$f} statement
must occur before a \texttt{\$e} statement in which its variable occurs.
\end{enumerate}

The first property determines the set of variables occurring in a frame.
These are the {\bf mandatory
variables}\index{mandatory variable} of the frame.  The second property
tells us there must be only one type specified for a variable.
The last property is not a theoretical requirement but it
makes parsing of the database easier.

For our examples, we assume our database has the following declarations:

\begin{verbatim}
$v P Q R $.
$c -> ( ) |- wff $.
\end{verbatim}

The following sequence of statements, describing the modus ponens inference
rule, is an example of a frame:

\begin{verbatim}
wp  $f wff P $.
wq  $f wff Q $.
maj $e |- ( P -> Q ) $.
min $e |- P $.
mp  $a |- Q $.
\end{verbatim}

The following sequence of statements is not a frame because \texttt{R} does not
occur in the \texttt{\$e}'s or the \texttt{\$a}:

\begin{verbatim}
wp  $f wff P $.
wq  $f wff Q $.
wr  $f wff R $.
maj $e |- ( P -> Q ) $.
min $e |- P $.
mp  $a |- Q $.
\end{verbatim}

The following sequence of statements is not a frame because \texttt{Q} does not
occur in a \texttt{\$f}:

\begin{verbatim}
wp  $f wff P $.
maj $e |- ( P -> Q ) $.
min $e |- P $.
mp  $a |- Q $.
\end{verbatim}

The following sequence of statements is not a frame because the \texttt{\$a} statement is
not the last one:

\begin{verbatim}
wp  $f wff P $.
wq  $f wff Q $.
maj $e |- ( P -> Q ) $.
mp  $a |- Q $.
min $e |- P $.
\end{verbatim}

Associated with a frame is a sequence of {\bf mandatory
hypotheses}\index{mandatory hypothesis}.  This is simply the set of all
\texttt{\$f} and \texttt{\$e} statements in the frame, in the order they
appear.  A frame can be referenced in a later proof using the label of
the \texttt{\$a} or \texttt{\$p} assertion statement, and the proof
makes an assignment to each mandatory hypothesis in the order in which
it appears.  This means the order of the hypotheses, once chosen, must
not be changed so as not to affect later proofs referencing the frame's
assertion statement.  (The Metamath proof verifier will, of course, flag
an error if a proof becomes incorrect by doing this.)  Since proofs make
use of ``Reverse Polish notation,'' described in Section~\ref{proof}, we
call this order the {\bf RPN order}\index{RPN order} of the hypotheses.

Note that \texttt{\$d} statements are not part of the set of mandatory
hypotheses, and their order doesn't matter (as long as they satisfy the
fourth property for a frame described above).  The \texttt{\$d}
statements specify restrictions on variables that must be satisfied (and
are checked by the proof verifier) when expressions are substituted for
them in a proof, and the \texttt{\$d} statements themselves are never
referenced directly in a proof.

A frame with a \texttt{\$p} (provable) statement requires a proof as part of the
\texttt{\$p} statement.  Sometimes in a proof we want to make use of temporary or
dummy variables\index{dummy variable} that do not occur in the \texttt{\$p}
statement or its mandatory hypotheses.  To accommodate this we define an {\bf
extended frame}\index{extended frame} as a frame together with zero or more
\texttt{\$d} and \texttt{\$f} statements that reference variables not among the
mandatory variables of the frame.  Any new variables referenced are called the
{\bf optional variables}\index{optional variable} of the extended frame. If a
\texttt{\$f} statement references an optional variable it is called an {\bf
optional hypothesis}\index{optional hypothesis}, and if one or both of the
variables in a \texttt{\$d} statement are optional variables it is called an {\bf
optional disjoint-variable restriction}\index{optional disjoint-variable
restriction}.  Properties 2 and 3 for a frame also apply to an extended
frame.

The concept of optional variables is not meaningful for frames with \texttt{\$a}
statements, since those statements have no proofs that might make use of them.
There is no restriction on including optional hypotheses in the extended frame
for a \texttt{\$a} statement, but they serve no purpose.

The following set of statements is an example of an extended frame, which
contains an optional variable \texttt{R} and an optional hypothesis \texttt{wr}.  In
this example, we suppose the rule of modus ponens is not an axiom but is
derived as a theorem from earlier statements (we omit its presumed proof).
Variable \texttt{R} may be used in its proof if desired (although this would
probably have no advantage in propositional calculus).  Note that the sequence
of mandatory hypotheses in RPN order is still \texttt{wp}, \texttt{wq}, \texttt{maj},
\texttt{min} (i.e.\ \texttt{wr} is omitted), and this sequence is still assumed
whenever the assertion \texttt{mp} is referenced in a subsequent proof.

\begin{verbatim}
wp  $f wff P $.
wq  $f wff Q $.
wr  $f wff R $.
maj $e |- ( P -> Q ) $.
min $e |- P $.
mp  $p |- Q $= ... $.
\end{verbatim}

Every frame is an extended frame, but not every extended frame is a frame, as
this example shows.  The underlying frame for an extended frame is
obtained by simply removing all statements containing optional variables.
Any proof referencing an assertion will ignore any extensions to its
frame, which means we may add or delete optional hypotheses at will without
affecting subsequent proofs.

The conceptually simplest way of organizing a Metamath database is as a
sequence of extended frames.  The scoping statements
\texttt{\$\char`\{}\index{\texttt{\$\char`\{} and \texttt{\$\char`\}}
keywords} and \texttt{\$\char`\}} can be used to delimit the start and
end of an extended frame, leading to the following possible structure for a
database.  \label{framelist}

\vskip 2ex
\setbox\startprefix=\hbox{\tt \ \ \ \ \ \ \ \ }
\setbox\contprefix=\hbox{}
\startm
\m{\mbox{(\texttt{\$v} {\em and} \texttt{\$c}\,{\em statements})}}
\endm
\startm
\m{\mbox{\texttt{\$\char`\{}}}
\endm
\startm
\m{\mbox{\texttt{\ \ } {\em extended frame}}}
\endm
\startm
\m{\mbox{\texttt{\$\char`\}}}}
\endm
\startm
\m{\mbox{\texttt{\$\char`\{}}}
\endm
\startm
\m{\mbox{\texttt{\ \ } {\em extended frame}}}
\endm
\startm
\m{\mbox{\texttt{\$\char`\}}}}
\endm
\startm
\m{\mbox{\texttt{\ \ \ \ \ \ \ \ \ }}\vdots}
\endm
\vskip 2ex

In practice, this structure is inconvenient because we have to repeat
any \texttt{\$f}, \texttt{\$e}, and \texttt{\$d} statements over and
over again rather than stating them once for use by several assertions.
The scoping statements, which we will discuss next, allow this to be
done.  In principle, any Metamath database can be converted to the above
format, and the above format is the most convenient to use when studying
a Metamath database as a formal system%
%% Uncomment this when uncommenting section {formalspec} below
   (Appendix \ref{formalspec})%
.
In fact, Metamath internally converts the database to the above format.
The command \texttt{show statement} in the Metamath program will show
you the contents of the frame for any \texttt{\$a} or \texttt{\$p}
statement, as well as its extension in the case of a \texttt{\$p}
statement.

%c%(provided that all ``local'' variables and constants with limited scope have
%c%unique names),

During our discussion of scoping statements, it may be helpful to
think in terms of the equivalent sequence of frames that will result when
the database is parsed.  Scoping (other than the limited
use above to delimit frames) is not a theoretical requirement for
Metamath but makes it more convenient.


\subsection{Scoping Statements (\texttt{\$\{} and \texttt{\$\}})}\label{scoping}
\index{\texttt{\$\char`\{} and \texttt{\$\char`\}} keywords}\index{scoping statement}

%c%Some Metamath statements may be needed only temporarily to
%c%serve a specific purpose, and after we're done with them we would like to
%c%disregard or ignore them.  For example, when we're finished using a variable,
%c%we might want to
%c%we might want to free up the token\index{token} used to name it so that the
%c%token can be used for other purposes later on, such as a different kind of
%c%variable or even a constant.  In the terminology of computer programming, we
%c%might want to let some symbol declarations be ``local'' rather than ``global.''
%c%\index{local symbol}\index{global symbol}

The {\bf scoping} statements, \texttt{\$\char`\{} ({\bf start of block}) and \texttt{\$\char`\}}
({\bf end of block})\index{block}, provide a means for controlling the portion
of a database over which certain statement types are recognized.  The
syntax of a scoping statement is very simple; it just consists of the
statement's keyword:
\begin{center}
\texttt{\$\char`\{}\\
\texttt{\$\char`\}}
\end{center}
\index{\texttt{\$\char`\{} and \texttt{\$\char`\}} keywords}

For example, consider the following database where we have stripped out
all tokens except the scoping statement keywords.  For the purpose of the
discussion, we have added subscripts to the scoping statements; these subscripts
do not appear in the actual database.
\[
 \mbox{\tt \ \$\char`\{}_1
 \mbox{\tt \ \$\char`\{}_2
 \mbox{\tt \ \$\char`\}}_2
 \mbox{\tt \ \$\char`\{}_3
 \mbox{\tt \ \$\char`\{}_4
 \mbox{\tt \ \$\char`\}}_4
 \mbox{\tt \ \$\char`\}}_3
 \mbox{\tt \ \$\char`\}}_1
\]
Each \texttt{\$\char`\{} statement in this example is said to be {\bf
matched} with the \texttt{\$\char`\}} statement that has the same
subscript.  Each pair of matched scoping statements defines a region of
the database called a {\bf block}.\index{block} Blocks can be {\bf
nested}\index{nested block} inside other blocks; in the example, the
block defined by $\mbox{\tt \$\char`\{}_4$ and $\mbox{\tt \$\char`\}}_4$
is nested inside the block defined by $\mbox{\tt \$\char`\{}_3$ and
$\mbox{\tt \$\char`\}}_3$ as well as inside the block defined by
$\mbox{\tt \$\char`\{}_1$ and $\mbox{\tt \$\char`\}}_1$.  In general, a
block may be empty, it may contain only non-scoping
statements,\footnote{Those statements other than \texttt{\$\char`\{} and
\texttt{\$\char`\}}.}\index{non-scoping statement} or it may contain any
mixture of other blocks and non-scoping statements.  (This is called a
``recursive'' definition\index{recursive definition} of a block.)

Associated with each block is a number called its {\bf nesting
level}\index{nesting level} that indicates how deeply the block is nested.
The nesting levels of the blocks in our example are as follows:
\[
  \underbrace{
    \mbox{\tt \ }
    \underbrace{
     \mbox{\tt \$\char`\{\ }
     \underbrace{
       \mbox{\tt \$\char`\{\ }
       \mbox{\tt \$\char`\}}
     }_{2}
     \mbox{\tt \ }
     \underbrace{
       \mbox{\tt \$\char`\{\ }
       \underbrace{
         \mbox{\tt \$\char`\{\ }
         \mbox{\tt \$\char`\}}
       }_{3}
       \mbox{\tt \ \$\char`\}}
     }_{2}
     \mbox{\tt \ \$\char`\}}
   }_{1}
   \mbox{\tt \ }
 }_{0}
\]
\index{\texttt{\$\char`\{} and \texttt{\$\char`\}} keywords}
The entire database is considered to be one big block (the {\bf outermost}
block) with a nesting level of 0.  The outermost block is {\em not} bracketed
by scoping statements.\footnote{The language was designed this way so that
several source files can be joined together more easily.}\index{outermost
block}

All non-scoping Metamath statements become recognized or {\bf
active}\index{active statement} at the place where they appear.\footnote{To
keep things slightly simpler, we do not bother to define the concept of
``active'' for the scoping statements.}  Certain of these statement types
become inactive at the end of the block in which they appear; these statement
types are:
\begin{center}
  \texttt{\$c}, \texttt{\$v}, \texttt{\$d}, \texttt{\$e}, and \texttt{\$f}.
%  \texttt{\$v}, \texttt{\$f}, \texttt{\$e}, and \texttt{\$d}.
\end{center}
\index{\texttt{\$c} statement}
\index{\texttt{\$d} statement}
\index{\texttt{\$e} statement}
\index{\texttt{\$f} statement}
\index{\texttt{\$v} statement}
The other statement types remain active forever (i.e.\ through the end of the
database); they are:
\begin{center}
  \texttt{\$a} and \texttt{\$p}.
%  \texttt{\$c}, \texttt{\$a}, and \texttt{\$p}.
\end{center}
\index{\texttt{\$a} statement}
\index{\texttt{\$p} statement}
Any statement (of these 7 types) located in the outermost
block\index{outermost block} will remain active through the end of the
database and thus are effectively ``global'' statements.\index{global
statement}

All \texttt{\$c} statements must be placed in the outermost block.  Since they are
therefore always global, they could be considered as belonging to both of the
above categories.

The {\bf scope}\index{scope} of a statement is the set of statements that
recognize it as active.

%c%The concept of ``active'' is also defined for math symbols\index{math
%c%symbol}.  Math symbols (constants\index{constant} and
%c%variables\index{variable}) become {\bf active}\index{active
%c%math symbol} in the \texttt{\$c}\index{\texttt{\$c}
%c%statement} and \texttt{\$v}\index{\texttt{\$v} statement} statements that
%c%declare them.  They become inactive when their declaration statements become
%c%inactive.

The concept of ``active'' is also defined for math symbols\index{math
symbol}.  Math symbols (constants\index{constant} and
variables\index{variable}) become {\bf active}\index{active math symbol}
in the \texttt{\$c}\index{\texttt{\$c} statement} and
\texttt{\$v}\index{\texttt{\$v} statement} statements that declare them.
A variable becomes inactive when its declaration statement becomes
inactive.  Because all \texttt{\$c} statements must be in the outermost
block, a constant will never become inactive after it is declared.

\subsubsection{Redeclaration of Math Symbols}
\index{redeclaration of symbols}\label{redeclaration}

%c%A math symbol may not be declared a second time while it is active, but it may
%c%be declared again after it becomes inactive.

A variable may not be declared a second time while it is active, but it may be
declared again after it becomes inactive.  This provides a convenient way to
introduce ``local'' variables,\index{local variable} i.e.\ temporary variables
for use in the frame of an assertion or in a proof without keeping them around
forever.  A previously declared variable may not be redeclared as a constant.

A constant may not be redeclared.  And, as mentioned above, constants must be
declared in the outermost block.

The reason variables may have limited scope but not constants is that an
assertion (\texttt{\$a} or \texttt{\$p} statement) remains available for use in
proofs through the end of the database.  Variables in an assertion's frame may
be substituted with whatever is needed in a proof step that references the
assertion, whereas constants remain fixed and may not be substituted with
anything.  The particular token used for a variable in an assertion's frame is
irrelevant when the assertion is referenced in a proof, and it doesn't matter
if that token is not available outside of the referenced assertion's frame.
Constants, however, must be globally fixed.

There is no theoretical
benefit for the feature allowing variables to be active for limited scopes
rather than global. It is just a convenience that allows them, for example, to
be locally grouped together with their corresponding \texttt{\$f} variable-type
declarations.

%c%If you declare a math symbol more than once, internally Metamath considers it a
%c%new distinct symbol, even though it has the same name.  If you are unaware of
%c%this, you may find that what you think are correct proofs are incorrectly
%c%rejected as invalid, because Metamath may tell you that a constant you
%c%previously declared does not match a newly declared math symbol with the same
%c%name.  For details on this subtle point, see the Comment on
%c%p.~\pageref{spec4comment}.  This is done purposely to allow temporary
%c%constants to be introduced while developing a subtheory, then allow their math
%c%symbol tokens to be reused later on; in general they will not refer to the
%c%same thing.  In practice, you would not ordinarily reuse the names of
%c%constants because it would tend to be confusing to the reader.  The reuse of
%c%names of variables, on the other hand, is something that is often useful to do
%c%(for example it is done frequently in \texttt{set.mm}).  Since variables in an
%c%assertion referenced in a proof can be substituted as needed to achieve a
%c%symbol match, this is not an issue.

% (This section covers a somewhat advanced topic you may want to skip
% at first reading.)
%
% Under certain circumstances, math symbol\index{math symbol}
% tokens\index{token} may be redeclared (i.e.\ the token
% may appear in more than
% one \texttt{\$c}\index{\texttt{\$c} statement} or \texttt{\$v}\index{\texttt{\$v}
% statement} statement).  You might want to do this say, to make temporary use
% of a variable name without having to worry about its affect elsewhere,
% somewhat analogous to declaring a local variable in a standard computer
% language.  Understanding what goes on when math symbol tokens are redeclared
% is a little tricky to understand at first, since it requires that we
% distinguish the token itself from the math symbol that it names.  It will help
% if we first take a peek at the internal workings of the
% Metamath\index{Metamath} program.
%
% Metamath reserves a memory location for each occurrence of a
% token\index{token} in a declaration statement (\texttt{\$c}\index{\texttt{\$c}
% statement} or \texttt{\$v}\index{\texttt{\$v} statement}).  If a given token appears
% in more than one declaration statement, it will refer to more than one memory
% locations.  A math symbol\index{math symbol} may be thought of as being one of
% these memory locations rather than as the token itself.  Only one of the
% memory locations associated with a given token may be active at any one time.
% The math symbol (memory location) that gets looked up when the token appears
% in a non-declaration statement is the one that happens to be active at that
% time.
%
% We now look at the rules for the redeclaration\index{redeclaration of symbols}
% of math symbol tokens.
% \begin{itemize}
% \item A math symbol token may not be declared twice in the
% same block.\footnote{While there is no theoretical reason for disallowing
% this, it was decided in the design of Metamath that allowing it would offer no
% advantage and might cause confusion.}
% \item An inactive math symbol may always be
% redeclared.
% \item  An active math symbol may be redeclared in a different (i.e.\
% inner) block\index{block} from the one it became active in.
% \end{itemize}
%
% When a math symbol token is redeclared, it conceptually refers to a different
% math symbol, just as it would be if it were called a different name.  In
% addition, the original math symbol that it referred to, if it was active,
% temporarily becomes inactive.  At the end of the block in which the
% redeclaration occurred, the new math symbol\index{math symbol} becomes
% inactive and the original symbol becomes active again.  This concept is
% illustrated in the following example, where the symbol \texttt{e} is
% ordinarily a constant (say Euler's constant, 2.71828...) but
% temporarily we want to use it as a ``local'' variable, say as a coefficient
% in the equation $a x^4 + b x^3 + c x^2 + d x + e$:
% \[
%   \mbox{\tt \$\char`\{\ \$c e \$.}
%   \underbrace{
%     \ \ldots\ %
%     \mbox{\tt \$\char`\{}\ \ldots\ %
%   }_{\mbox{\rm region A}}
%   \mbox{\tt \$v e \$.}
%   \underbrace{
%     \mbox{\ \ \ \ldots\ \ \ }
%   }_{\mbox{\rm region B}}
%   \mbox{\tt \$\char`\}}
%   \underbrace{
%     \mbox{\ \ \ \ldots\ \ \ }
%   }_{\mbox{\rm region C}}
%   \mbox{\tt \$\char`\}}
% \]
% \index{\texttt{\$\char`\{} and \texttt{\$\char`\}} keywords}
% In region A, the token \texttt{e} refers to a constant.  It is redeclared as a
% variable in region B, and any reference to it in this region will refer to this
% variable.  In region C, the redeclaration becomes inactive, and the original
% declaration becomes active again.  In region C, the token \texttt{x} refers to the
% original constant.
%
% As a practical matter, overuse of math symbol\index{math symbol}
% redeclarations\index{redeclaration of symbols} can be confusing (even though
% it is well-defined) and is best avoided when possible.  Here are some good
% general guidelines you can follow.  Usually, you should declare all
% constants\index{constant} in the outermost block\index{outermost block},
% especially if they are general-purpose (such as the token \verb$A.$, meaning
% $\forall$ or ``for all'').  This will make them ``globally'' active (although
% as in the example above local redeclarations will temporarily make them
% inactive.)  Most or all variables\index{variable}, on the other hand, could be
% declared in inner blocks, so that the token for them can be used later for a
% different type of variable or a constant.  (The names of the variables you
% choose are not used when you refer to an assertion\index{assertion} in a
% proof, whereas constants must match exactly.  A locally declared constant will
% not match a globally declared constant in a proof, even if they use the same
% token, because Metamath internally considers them to be different math
% symbols.)  To avoid confusion, you should generally avoid redeclaring active
% variables.  If you must redeclare them, do so at the beginning of a block.
% The temporary declaration of constants in inner blocks might be occasionally
% appropriate when you make use of a temporary definition to prove lemmas
% leading to a main result that does not make direct use of the definition.
% This way, you will not clutter up your database with a large number of
% seldom-used global constant symbols.  You might want to note that while
% inactive constants may not appear directly in an assertion (a \texttt{\$a}\index{\texttt{\$a}
% statement} or \texttt{\$p}\index{\texttt{\$p} statement}
% statement), they may be indirectly used in the proof of a \texttt{\$p} statement
% so long as they do not appear in the final math symbol sequence constructed by
% the proof.  In the end, you will have to use your best judgment, taking into
% account standard mathematical usage of the symbols as well as consideration
% for the reader of your work.
%
% \subsubsection{Reuse of Labels}\index{reuse of labels}\index{label}
%
% The \texttt{\$e}\index{\texttt{\$e} statement}, \texttt{\$f}\index{\texttt{\$f}
% statement}, \texttt{\$a}\index{\texttt{\$a} statement}, and
% \texttt{\$p}\index{\texttt{\$p}
% statement} statement types require labels, which allow them to be
% referenced later inside proofs.  A label is considered {\bf
% active}\index{active label} when the statement it is associated with is
% active.  The token\index{token} for a label may be reused
% (redeclared)\index{redeclaration of labels} provided that it is not being used
% for a currently active label.  (Unlike the tokens for math symbols, active
% label tokens may not be redeclared in an inner scope.)  Note that the labels
% of \texttt{\$a} and \texttt{\$p} statements can never be reused after these
% statements appear, because these statements remain active through the end of
% the database.
%
% You might find the reuse of labels a convenient way to have standard names for
% temporary hypotheses, such as \texttt{h1}, \texttt{h2}, etc.  This way you don't have
% to invent unique names for each of them, and in some cases it may be less
% confusing to the reader (although in other cases it might be more confusing, if
% the hypothesis is located far away from the assertion that uses
% it).\footnote{The current implementation requires that all labels, even
% inactive ones, be unique.}

\subsubsection{Frames Revisited}\index{frames and scoping statements}

Now that we have covered scoping, we will look at how an arbitrary
Metamath database can be converted to the simple sequence of extended
frames described on p.~\pageref{framelist}.  This is also how Metamath
stores the database internally when it reads in the database
source.\label{frameconvert} The method is simple.  First, we collect all
constant and variable (\texttt{\$c} and \texttt{\$v}) declarations in
the database, ignoring duplicate declarations of the same variable in
different scopes.  We then put our collected \texttt{\$c} and
\texttt{\$v} declarations at the beginning of the database, so that
their scope is the entire database.  Next, for each assertion in the
database, we determine its frame and extended frame.  The extended frame
is simply the \texttt{\$f}, \texttt{\$e}, and \texttt{\$d} statements
that are active.  The frame is the extended frame with all optional
hypotheses removed.

An equivalent way of saying this is that the extended frame of an assertion
is the collection of all \texttt{\$f}, \texttt{\$e}, and \texttt{\$d} statements
whose scope includes the assertion.
The \texttt{\$f} and \texttt{\$e} statements
occur in the order they appear
(order is irrelevant for \texttt{\$d} statements).

%c%, renaming any
%c%redeclared variables as needed so that all of them have unique names.  (The
%c%exact renaming convention is unimportant.  You might imagine renaming
%c%different declarations of math symbol \texttt{a} as \texttt{a\$1}, \texttt{a\$2}, etc.\
%c%which would prevent any conflicts since \texttt{\$} is not a legal character in a
%c%math symbol token.)

\section{The Anatomy of a Proof} \label{proof}
\index{proof!Metamath, description of}

Each provable assertion (\texttt{\$p}\index{\texttt{\$p} statement} statement) in a
database must include a {\bf proof}\index{proof}.  The proof is located
between the \texttt{\$=}\index{\texttt{\$=} keyword} and \texttt{\$.}\ keywords in the
\texttt{\$p} statement.

In the basic Metamath language\index{basic language}, a proof is a
sequence of statement labels.  This label sequence\index{label sequence}
serves as a set of instructions that the Metamath program uses to
construct a series of math symbol sequences.  The construction must
ultimately result in the math symbol sequence contained between the
\texttt{\$p}\index{\texttt{\$p} statement} and
\texttt{\$=}\index{\texttt{\$=} keyword} keywords of the \texttt{\$p}
statement.  Otherwise, the Metamath program will consider the proof
incorrect, and it will notify you with an appropriate error message when
you ask it to verify the proof.\footnote{To make the loading faster, the
Metamath program does not automatically verify proofs when you
\texttt{read} in a database unless you use the \texttt{/verify}
qualifier.  After a database has been read in, you may use the
\texttt{verify proof *} command to verify proofs.}\index{\texttt{verify
proof} command} Each label in a proof is said to {\bf
reference}\index{label reference} its corresponding statement.

Associated with any assertion\index{assertion} (\texttt{\$p} or
\texttt{\$a}\index{\texttt{\$a} statement} statement) is a set of
hypotheses (\texttt{\$f}\index{\texttt{\$f} statement} or
\texttt{\$e}\index{\texttt{\$e} statement} statements) that are active
with respect to that assertion.  Some are mandatory and the others are
optional.  You should review these concepts if necessary.

Each label\index{label} in a proof must be either the label of a
previous assertion (\texttt{\$a}\index{\texttt{\$a} statement} or
\texttt{\$p}\index{\texttt{\$p} statement} statement) or the label of an
active hypothesis (\texttt{\$e} or \texttt{\$f}\index{\texttt{\$f}
statement} statement) of the \texttt{\$p} statement containing the
proof.  Hypothesis labels may reference both the
mandatory\index{mandatory hypothesis} and the optional hypotheses of the
\texttt{\$p} statement.

The label sequence in a proof specifies a construction in {\bf reverse Polish
notation}\index{reverse Polish notation (RPN)} (RPN).  You may be familiar
with RPN if you have used older
Hewlett--Packard or similar hand-held calculators.
In the calculator analogy, a hypothesis label\index{hypothesis label} is like
a number and an assertion label\index{assertion label} is like an operation
(more precisely, an $n$-ary operation when the
assertion has $n$ \texttt{\$e}-hypotheses).
On an RPN calculator, an operation takes one or more previous numbers in an
input sequence, performs a calculation on them, and replaces those numbers and
itself with the result of the calculation.  For example, the input sequence
$2,3,+$ on an RPN calculator results in $5$, and the input sequence
$2,3,5,{\times},+$ results in $2,15,+$ which results in $17$.

Understanding how RPN is processed involves the concept of a {\bf
stack}\index{stack}\index{RPN stack}, which can be thought of as a set of
temporary memory locations that hold intermediate results.  When Metamath
encounters a hypothesis label it places or {\bf pushes}\index{push} the math
symbol sequence of the hypothesis onto the stack.  When Metamath encounters an
assertion label, it associates the most recent stack entries with the {\em
mandatory} hypotheses\index{mandatory hypothesis} of the assertion, in the
order where the most recent stack entry is associated with the last mandatory
hypothesis of the assertion.  It then determines what
substitutions\index{substitution!variable}\index{variable substitution} have
to be made into the variables of the assertion's mandatory hypotheses to make
them identical to the associated stack entries.  It then makes those same
substitutions into the assertion itself.  Finally, Metamath removes or {\bf
pops}\index{pop} the matched hypotheses from the stack and pushes the
substituted assertion onto the stack.

For the purpose of matching the mandatory hypothesis to the most recent stack
entries, whether a hypothesis is a \texttt{\$e} or \texttt{\$f} statement is
irrelevant.  The only important thing is that a set of
substitutions\footnote{In the Metamath spec (Section~\ref{spec}), we use the
singular term ``substitution'' to refer to the set of substitutions we talk
about here.} exist that allow a match (and if they don't, the proof verifier
will let you know with an error message).  The Metamath language is specified
in such a way that if a set of substitutions exists, it will be unique.
Specifically, the requirement that each variable have a type specified for it
with a \texttt{\$f} statement ensures the uniqueness.

We will illustrate this with an example.
Consider the following Metamath source file:
\begin{verbatim}
$c ( ) -> wff $.
$v p q r s $.
wp $f wff p $.
wq $f wff q $.
wr $f wff r $.
ws $f wff s $.
w2 $a wff ( p -> q ) $.
wnew $p wff ( s -> ( r -> p ) ) $= ws wr wp w2 w2 $.
\end{verbatim}
This Metamath source example shows the definition and ``proof'' (i.e.,
construction) of a well-formed formula (wff)\index{well-formed formula (wff)}
in propositional calculus.  (You may wish to type this example into a file to
experiment with the Metamath program.)  The first two statements declare
(introduce the names of) four constants and four variables.  The next four
statements specify the variable types, namely that
each variable is assumed to be a wff.  Statement \texttt{w2} defines (postulates)
a way to produce a new wff, \texttt{( p -> q )}, from two given wffs \texttt{p} and
\texttt{q}. The mandatory hypotheses of \texttt{w2} are \texttt{wp} and \texttt{wq}.
Statement \texttt{wnew} claims that \texttt{( s -> ( r -> p ) )} is a wff given
three wffs \texttt{s}, \texttt{r}, and \texttt{p}.  More precisely, \texttt{wnew} claims
that the sequence of ten symbols \texttt{wff ( s -> ( r -> p ) )} is provable from
previous assertions and the hypotheses of \texttt{wnew}.  Metamath does not know
or care what a wff is, and as far as it is concerned
the typecode \texttt{wff} is just an
arbitrary constant symbol in a math symbol sequence.  The mandatory hypotheses
of \texttt{wnew} are \texttt{wp}, \texttt{wr}, and \texttt{ws}; \texttt{wq} is an optional
hypothesis.  In our particular proof, the optional hypothesis is not
referenced, but in general, any combination of active (i.e.\ optional and
mandatory) hypotheses could be referenced.  The proof of statement \texttt{wnew}
is the sequence of five labels starting with \texttt{ws} (step~1) and ending with
\texttt{w2} (step~5).

When Metamath verifies the proof, it scans the proof from left to right.  We
will examine what happens at each step of the proof.  The stack starts off
empty.  At step 1, Metamath looks up label \texttt{ws} and determines that it is a
hypothesis, so it pushes the symbol sequence of statement \texttt{ws} onto the
stack:

\begin{center}\begin{tabular}{|l|l|}\hline
{Stack location} & {Contents} \\ \hline \hline
1 & \texttt{wff s} \\ \hline
\end{tabular}\end{center}

Metamath sees that the labels \texttt{wr} and \texttt{wp} in steps~2 and 3 are also
hypotheses, so it pushes them onto the stack.  After step~3, the stack looks
like
this:

\begin{center}\begin{tabular}{|l|l|}\hline
{Stack location} & {Contents} \\ \hline \hline
3 & \texttt{wff p} \\ \hline
2 & \texttt{wff r} \\ \hline
1 & \texttt{wff s} \\ \hline
\end{tabular}\end{center}

At step 4, Metamath sees that label \texttt{w2} is an assertion, so it must do
some processing.  First, it associates the mandatory hypotheses of \texttt{w2},
which are \texttt{wp} and \texttt{wq}, with stack locations~2 and 3, {\em in that
order}. Metamath determines that the only possible way
to make hypothesis \texttt{wp} match (become identical to) stack location~2 and
\texttt{wq} match stack location 3 is to substitute variable \texttt{p} with \texttt{r}
and \texttt{q} with \texttt{p}.  Metamath makes these substitutions into \texttt{w2} and
obtains the symbol sequence \texttt{wff ( r -> p )}.  It removes the hypotheses
from stack locations~2 and 3, then places the result into stack location~2:

\begin{center}\begin{tabular}{|l|l|}\hline
{Stack location} & {Contents} \\ \hline \hline
2 & \texttt{wff ( r -> p )} \\ \hline
1 & \texttt{wff s} \\ \hline
\end{tabular}\end{center}

At step 5, Metamath sees that label \texttt{w2} is an assertion, so it must again
do some processing.  First, it matches the mandatory hypotheses of \texttt{w2},
which are \texttt{wp} and \texttt{wq}, to stack locations 1 and 2.
Metamath determines that the only possible way to make the
hypotheses match is to substitute variable \texttt{p} with \texttt{s} and \texttt{q} with
\texttt{( r -> p )}.  Metamath makes these substitutions into \texttt{w2} and obtains
the symbol
sequence \texttt{wff ( s -> ( r -> p ) )}.  It removes stack
locations 1 and 2, then places the result into stack location~1:

\begin{center}\begin{tabular}{|l|l|}\hline
{Stack location} & {Contents} \\ \hline \hline
1 & \texttt{wff ( s -> ( r -> p ) )} \\ \hline
\end{tabular}\end{center}

After Metamath finishes processing the proof, it checks to see that the
stack contains exactly one element and that this element is
the same as the math symbol sequence in the
\texttt{\$p}\index{\texttt{\$p} statement} statement.  This is the case for our
proof of \texttt{wnew},
so we have proved \texttt{wnew} successfully.  If the result
differs, Metamath will notify you with an error message.  An error message
will also result if the stack contains more than one entry at the end of the
proof, or if the stack did not contain enough entries at any point in the
proof to match all of the mandatory hypotheses\index{mandatory hypothesis} of
an assertion.  Finally, Metamath will notify you with an error message if no
substitution is possible that will make a referenced assertion's hypothesis
match the
stack entries.  You may want to experiment with the different kinds of errors
that Metamath will detect by making some small changes in the proof of our
example.

Metamath's proof notation was designed primarily to express proofs in a
relatively compact manner, not for readability by humans.  Metamath can display
proofs in a number of different ways with the \texttt{show proof}\index{\texttt{show
proof} command} command.  The
\texttt{/lemmon} qualifier displays it in a format that is easier to read when the
proofs are short, and you saw examples of its use in Chapter~\ref{using}.  For
longer proofs, it is useful to see the tree structure of the proof.  A tree
structure is displayed when the \texttt{/lemmon} qualifier is omitted.  You will
probably find this display more convenient as you get used to it. The tree
display of the proof in our example looks like
this:\label{treeproof}\index{tree-style proof}\index{proof!tree-style}
\begin{verbatim}
1     wp=ws    $f wff s
2        wp=wr    $f wff r
3        wq=wp    $f wff p
4     wq=w2    $a wff ( r -> p )
5  wnew=w2  $a wff ( s -> ( r -> p ) )
\end{verbatim}
The number to the left of each line is the step number.  Following it is a
{\bf hypothesis association}\index{hypothesis association}, consisting of two
labels\index{label} separated by \texttt{=}.  To the left of the \texttt{=} (except
in the last step) is the label of a hypothesis of an assertion referenced
later in the proof; here, steps 1 and 4 are the hypothesis associations for
the assertion \texttt{w2} that is referenced in step 5.  A hypothesis association
is indented one level more than the assertion that uses it, so it is easy to
find the corresponding assertion by moving directly down until the indentation
level decreases to one less than where you started from.  To the right of each
\texttt{=} is the proof step label for that proof step.  The statement keyword of
the proof step label is listed next, followed by the content of the top of the
stack (the most recent stack entry) as it exists after that proof step is
processed.  With a little practice, you should have no trouble reading proofs
displayed in this format.

Metamath proofs include the syntax construction of a formula.
In standard mathematics, this kind of
construction is not considered a proper part of the proof at all, and it
certainly becomes rather boring after a while.
Therefore,
by default the \texttt{show proof}\index{\texttt{show proof}
command} command does not show the syntax construction.
Historically \texttt{show proof} command
\textit{did} show the syntax construction, and you needed to add the
\texttt{/essential} option to hide, them, but today
\texttt{/essential} is the default and you need to use
\texttt{/all} to see the syntax constructions.

When verifying a proof, Metamath will check that no mandatory
\texttt{\$d}\index{\texttt{\$d} statement}\index{mandatory \texttt{\$d}
statement} statement of an assertion referenced in a proof is violated
when substitutions\index{substitution!variable}\index{variable
substitution} are made to the variables in the assertion.  For details
see Section~\ref{spec4} or \ref{dollard}.

\subsection{The Concept of Unification} \label{unify}

During the course of verifying a proof, when Metamath\index{Metamath}
encounters an assertion label\index{assertion label}, it associates the
mandatory hypotheses\index{mandatory hypothesis} of the assertion with the top
entries of the RPN stack\index{stack}\index{RPN stack}.  Metamath then
determines what substitutions\index{substitution!variable}\index{variable
substitution} it must make to the variables in the assertion's mandatory
hypotheses in order for these hypotheses to become identical to their
corresponding stack entries.  This process is called {\bf
unification}\index{unification}.  (We also informally use the term
``unification'' to refer to a set of substitutions that results from the
process, as in ``two unifications are possible.'')  After the substitutions
are made, the hypotheses are said to be {\bf unified}.

If no such substitutions are possible, Metamath will consider the proof
incorrect and notify you with an error message.
% (deleted 3/10/07, per suggestion of Mel O'Cat:)
% The syntax of the
% Metamath language ensures that if a set of substitutions exists, it
% will be unique.

The general algorithm for unification described in the literature is
somewhat complex.
However, in the case of Metamath it is intentionally trivial.
Mandatory hypotheses must be
pushed on the proof stack in the order in which they appear.
In addition, each variable must have its type specified
with a \texttt{\$f} hypothesis before it is used
and that each \texttt{\$f} hypothesis
have the restricted syntax of a typecode (a constant) followed by a variable.
The typecode in the \texttt{\$f} hypothesis must match the first symbol of
the corresponding RPN stack entry (which will also be a constant), so
the only possible match for the variable in the \texttt{\$f} hypothesis is
the sequence of symbols in the stack entry after the initial constant.

In the Proof Assistant\index{Proof Assistant}, a more general unification
algorithm is used.  While a proof is being developed, sometimes not enough
information is available to determine a unique unification.  In this case
Metamath will ask you to pick the correct one.\index{ambiguous
unification}\index{unification!ambiguous}

\section{Extensions to the Metamath Language}\index{extended
language}

\subsection{Comments in the Metamath Language}\label{comments}
\index{markup notation}
\index{comments!markup notation}

The commenting feature allows you to annotate the contents of
a database.  Just as with most
computer languages, comments are ignored for the purpose of interpreting the
contents of the database. Comments effectively act as
additional white space\index{white
space} between tokens
when a database is parsed.

A comment may be placed at the beginning, end, or
between any two tokens\index{token} in a source file.

Comments have the following syntax:
\begin{center}
 \texttt{\$(} {\em text} \texttt{\$)}
\end{center}
Here,\index{\texttt{\$(} and \texttt{\$)} auxiliary
keywords}\index{comment} {\em text} is a string, possibly empty, of any
characters in Metamath's character set (p.~\pageref{spec1chars}), except
that the character strings \texttt{\$(} and \texttt{\$)} may not appear
in {\em text}.  Thus nested comments are not
permitted:\footnote{Computer languages have differing standards for
nested comments, and rather than picking one it was felt simplest not to
allow them at all, at least in the current version (0.177) of
Metamath\index{Metamath!limitations of version 0.177}.} Metamath will
complain if you give it
\begin{center}
 \texttt{\$( This is a \$( nested \$) comment.\ \$)}
\end{center}
To compensate for this non-nesting behavior, I often change all \texttt{\$}'s
to \texttt{@}'s in sections of Metamath code I wish to comment out.

The Metamath program supports a number of markup mechanisms and conventions
to generate good-looking results in \LaTeX\ and {\sc html},
as discussed below.
These markup features have to do only with how the comments are typeset,
and have no effect on how Metamath verifies the proofs in the database.
The improper
use of them may result in incorrectly typeset output, but no Metamath
error messages will result during the \texttt{read} and \texttt{verify
proof} commands.  (However, the \texttt{write
theorem\texttt{\char`\_}list} command
will check for markup errors as a side-effect of its
{\sc html} generation.)
Section~\ref{texout} has instructions for creating \LaTeX\ output, and
section~\ref{htmlout} has instructions for creating
{\sc html}\index{HTML} output.

\subsubsection{Headings}\label{commentheadings}

If the \texttt{\$(} is immediately followed by a new line
starting with a heading marker, it is a header.
This can start with:

\begin{itemize}
 \item[] \texttt{\#\#\#\#} - major part header
 \item[] \texttt{\#*\#*} - section header
 \item[] \texttt{=-=-} - subsection header
 \item[] \texttt{-.-.} - subsubsection header
\end{itemize}

The line following the marker line
will be used for the table of contents entry, after trimming spaces.
The next line should be another (closing) matching marker line.
Any text after that
but before the closing \texttt{\$}, such as an extended description of the
section, will be included on the \texttt{mmtheoremsNNN.html} page.

For more information, run
\texttt{help write theorem\char`\_list}.

\subsubsection{Math mode}
\label{mathcomments}
\index{\texttt{`} inside comments}
\index{\texttt{\char`\~} inside comments}
\index{math mode}

Inside of comments, a string of tokens\index{token} enclosed in
grave accents\index{grave accent (\texttt{`})} (\texttt{`}) will be converted
to standard mathematical symbols during
{\sc HTML}\index{HTML} or \LaTeX\ output
typesetting,\index{latex@{\LaTeX}} according to the information in the
special \texttt{\$t}\index{\texttt{\$t} comment}\index{typesetting
comment} comment in the database
(see section~\ref{tcomment} for information about the typesetting
comment, and Appendix~\ref{ASCII} to see examples of its results).

The first grave accent\index{grave accent (\texttt{`})} \texttt{`}
causes the output processor to enter {\bf math mode}\index{math mode}
and the second one exits it.
In this
mode, the characters following the \texttt{`} are interpreted as a
sequence of math symbol tokens separated by white space\index{white
space}.  The tokens are looked up in the \texttt{\$t}
comment\index{\texttt{\$t} comment}\index{typesetting comment} and if
found, they will be replaced by the standard mathematical symbols that
they correspond to before being placed in the typeset output file.  If
not found, the symbol will be output as is and a warning will be issued.
The tokens do not have to be active in the database, although a warning
will be issued if they are not declared with \texttt{\$c} or
\texttt{\$v} statements.

Two consecutive
grave accents \texttt{``} are treated as a single actual grave accent
(both inside and outside of math mode) and will not cause the output
processor to enter or exit math mode.

Here is an example of its use\index{Pierce's axiom}:
\begin{center}
\texttt{\$( Pierce's axiom, ` ( ( ph -> ps ) -> ph ) -> ph ` ,\\
         is not very intuitive. \$)}
\end{center}
becomes
\begin{center}
   \texttt{\$(} Pierce's axiom, $((\varphi \rightarrow \psi)\rightarrow
\varphi)\rightarrow \varphi$, is not very intuitive. \texttt{\$)}
\end{center}

Note that the math symbol tokens\index{token} must be surrounded by white
space\index{white space}.
%, since there is no context that allows ambiguity to be
%resolved, as is the case with math symbol sequences in some of the Metamath
%statements.
White space should also surround the \texttt{`}
delimiters.

The math mode feature also gives you a quick and easy way to generate
text containing mathematical symbols, independently of the intended
purpose of Metamath.\index{Metamath!using as a math editor} To do this,
simply create your text with grave accents surrounding your formulas,
after making sure that your math symbols are mapped to \LaTeX\ symbols
as described in Appendix~\ref{ASCII}.  It is easier if you start with a
database with predefined symbols such as \texttt{set.mm}.  Use your
grave-quoted math string to replace an existing comment, then typeset
the statement corresponding to that comment following the instructions
from the \texttt{help tex} command in the Metamath program.  You will
then probably want to edit the resulting file with a text editor to fine
tune it to your exact needs.

\subsubsection{Label Mode}\index{label mode}

Outside of math mode, a tilde\index{tilde (\texttt{\char`\~})} \verb/~/
indicates to Metamath's\index{Metamath} output processor that the
token\index{token} that follows (i.e.\ the characters up to the next
white space\index{white space}) represents a statement label or URL.
This formatting mode is called {\bf label mode}\index{label mode}.
If a literal tilde
is desired (outside of math mode) instead of label mode,
use two tildes in a row to represent it.

When generating a \LaTeX\ output file,
the following token will be formatted in \texttt{typewriter}
font, and the tilde removed, to make it stand out from the rest of the text.
This formatting will be applied to all characters after the
tilde up to the first white space\index{white space}.
Whether
or not the token is an actual statement label is not checked, and the
token does not have to have the correct syntax for a label; no error
messages will be produced.  The only effect of the label mode on the
output is that typewriter font will be used for the tokens that are
placed in the \LaTeX\ output file.

When generating {\sc html},
the tokens after the tilde {\em must} be a URL (either http: or https:)
or a valid label.
Error messages will be issued during that output if they aren't.
A hyperlink will be generated to that URL or label.

\subsubsection{Link to bibliographical reference}\index{citation}%
\index{link to bibliographical reference}

Bibliographical references are handled specially when generating
{\sc html} if formatted specially.
Text in the form \texttt{[}{\em author}\texttt{]}
is considered a link to a bibliographical reference.
See \texttt{help html} and \texttt{help write
bibliography} in the Metamath program for more
information.
% \index{\texttt{\char`\[}\ldots\texttt{]} inside comments}
See also Sections~\ref{tcomment} and \ref{wrbib}.

The \texttt{[}{\em author}\texttt{]} notation will also create an entry in
the bibliography cross-reference file generated by \texttt{write
bibliography} (Section~\ref{wrbib}) for {\sc HTML}.
For this to work properly, the
surrounding comment must be formatted as follows:
\begin{quote}
    {\em keyword} {\em label} {\em noise-word}
     \texttt{[}{\em author}\texttt{] p.} {\em number}
\end{quote}
for example
\begin{verbatim}
     Theorem 5.2 of [Monk] p. 223
\end{verbatim}
The {\em keyword} is not case sensitive and must be one of the following:
\begin{verbatim}
     theorem lemma definition compare proposition corollary
     axiom rule remark exercise problem notation example
     property figure postulate equation scheme chapter
\end{verbatim}
The optional {\em label} may consist of more than one
(non-{\em keyword} and non-{\em noise-word}) word.
The optional {\em noise-word} is one of:
\begin{verbatim}
     of in from on
\end{verbatim}
and is  ignored when the cross-reference file is created.  The
\texttt{write
biblio\-graphy} command will perform error checking to verify the
above format.\index{error checking}

\subsubsection{Parentheticals}\label{parentheticals}

The end of a comment may include one or more parenthicals, that is,
statements enclosed in parentheses.
The Metamath program looks for certain parentheticals and can issue
warnings based on them.
They are:

\begin{itemize}
 \item[] \texttt{(Contributed by }
   \textit{NAME}\texttt{,} \textit{DATE}\texttt{.)} -
   document the original contributor's name and the date it was created.
 \item[] \texttt{(Revised by }
   \textit{NAME}\texttt{,} \textit{DATE}\texttt{.)} -
   document the contributor's name and creation date
   that resulted in significant revision
   (not just an automated minimization or shortening).
 \item[] \texttt{(Proof shortened by }
   \textit{NAME}\texttt{,} \textit{DATE}\texttt{.)} -
   document the contributor's name and date that developed a significant
   shortening of the proof (not just an automated minimization).
 \item[] \texttt{(Proof modification is discouraged.)} -
   Note that this proof should normally not be modified.
 \item[] \texttt{(New usage is discouraged.)} -
   Note that this assertion should normally not be used.
\end{itemize}

The \textit{DATE} must be in form YYYY-MMM-DD, where MMM is the
English abbreviation of that month.

\subsubsection{Other markup}\label{othermarkup}
\index{markup notation}

There are other markup notations for generating good-looking results
beyond math mode and label mode:

\begin{itemize}
 \item[]
         \texttt{\char`\_} (underscore)\index{\texttt{\char`\_} inside comments} -
             Italicize text starting from
              {\em space}\texttt{\char`\_}{\em non-space} (i.e.\ \texttt{\char`\_}
              with a space before it and a non-space character after it) until
             the next
             {\em non-space}\texttt{\char`\_}{\em space}.  Normal
             punctuation (e.g.\ a trailing
             comma or period) is ignored when determining {\em space}.
 \item[]
         \texttt{\char`\_} (underscore) - {\em
         non-space}\texttt{\char`\_}{\em non-space-string}, where
          {\em non-space-string} is a string of non-space characters,
         will make {\em non-space-string} become a subscript.
 \item[]
         \texttt{<HTML>}...\texttt{</HTML>} - do not convert
         ``\texttt{<}'' and ``\texttt{>}''
         in the enclosed text when generating {\sc HTML},
         otherwise process markup normally. This allows direct insertion
         of {\sc html} commands.
 \item[]
       ``\texttt{\&}ref\texttt{;}'' - insert an {\sc HTML}
         character reference.
         This is how to insert arbitrary Unicode characters
         (such as accented characters).  Currently only directly supported
         when generating {\sc HTML}.
\end{itemize}

It is recommended that spaces surround any \texttt{\char`\~} and
\texttt{`} tokens in the comment and that a space follow the {\em label}
after a \texttt{\char`\~} token.  This will make global substitutions
to change labels and symbol names much easier and also eliminate any
future chance of ambiguity.  Spaces around these tokens are automatically
removed in the final output to conform with normal rules of punctuation;
for example, a space between a trailing \texttt{`} and a left parenthesis
will be removed.

A good way to become familiar with the markup notation is to look at
the extensive examples in the \texttt{set.mm} database.

\subsection{The Typesetting Comment (\texttt{\$t})}\label{tcomment}

The typesetting comment \texttt{\$t} in the input database file
provides the information necessary to produce good-looking results.
It provides \LaTeX\ and {\sc html}
definitions for math symbols,
as well supporting as some
customization of the generated web page.
If you add a new token to a database, you should also
update the \texttt{\$t} comment information if you want to eventually
create output in \LaTeX\ or {\sc HTML}.
See the
\texttt{set.mm}\index{set theory database (\texttt{set.mm})} database
file for an extensive example of a \texttt{\$t} comment illustrating
many of the features described below.

Programs that do not need to generate good-looking presentation results,
such as programs that only verify Metamath databases,
can completely ignore typesetting comments
and just treat them as normal comments.
Even the Metamath program only consults the
\texttt{\$t} comment information when it needs to generate typeset output
in \LaTeX\ or {\sc HTML}
(e.g., when you open a \LaTeX\ output file with the \texttt{open tex} command).

We will first discuss the syntax of typesetting comments, and then
briefly discuss how this can be used within the Metamath program.

\subsubsection{Typesetting Comment Syntax Overview}

The typesetting comment is identified by the token
\texttt{\$t}\index{\texttt{\$t} comment}\index{typesetting comment} in
the comment, and the typesetting comment ends at the matching
\texttt{\$)}:
\[
  \mbox{\tt \$(\ }
  \mbox{\tt \$t\ }
  \underbrace{
    \mbox{\tt \ \ \ \ \ \ \ \ \ \ \ }
    \cdots
    \mbox{\tt \ \ \ \ \ \ \ \ \ \ \ }
  }_{\mbox{Typesetting definitions go here}}
  \mbox{\tt \ \$)}
\]

There must be one or more white space characters, and only white space
characters, between the \texttt{\$(} that starts the comment
and the \texttt{\$t} symbol,
and the \texttt{\$t} must be followed by one
or more white space characters
(see section \ref{whitespace} for the definition of white space characters).
The typesetting comment continues until the comment end token \texttt{\$)}
(which must be preceded by one or more white space characters).

In version 0.177\index{Metamath!limitations of version 0.177} of the
Metamath program, there may be only one \texttt{\$t} comment in a
database.  This restriction may be lifted in the future to allow
many \texttt{\$t} comments in a database.

Between the \texttt{\$t} symbol (and its following white space) and the
comment end token \texttt{\$)} (and its preceding white space)
is a sequence of one or more typesetting definitions, where
each definition has the form
\textit{definition-type arg arg ... ;}.
Each of the zero or more \textit{arg} values
can be either a typesetting data or a keyword
(what keywords are allowed, and where, depends on the specific
\textit{definition-type}).
The \textit{definition-type}, and each argument \textit{arg},
are separated by one or more white space characters.
Every definition ends in an unquoted semicolon;
white space is not required before the terminating semicolon of a definition.
Each definition should start on a new line.\footnote{This
restriction of the current version of Metamath
(0.177)\index{Metamath!limitations of version 0.177} may be removed
in a future version, but you should do it anyway for readability.}

For example, this typesetting definition:
\begin{center}
 \verb$latexdef "C_" as "\subseteq";$
\end{center}
defines the token \verb$C_$ as the \LaTeX\ symbol $\subseteq$ (which means
``subset'').

Typesetting data is a sequence of one or more quoted strings
(if there is more than one, they are connected by \texttt{\char`\+}).
Often a single quoted string is used to provide data for a definition, using
either double (\texttt{\char`\"}) or single (\texttt{'}) quotation marks.
However,
{\em a quoted string (enclosed in quotation marks) may not include
line breaks.}
A quoted string
may include a quotation mark that matches the enclosing quotes by repeating
the quotation mark twice.  Here are some examples:

\begin{tabu}   { l l }
\textbf{Example} & \textbf{Meaning} \\
\texttt{\char`\"a\char`\"\char`\"b\char`\"} & \texttt{a\char`\"b} \\
\texttt{'c''d'} & \texttt{c'd} \\
\texttt{\char`\"e''f\char`\"} & \texttt{e''f} \\
\texttt{'g\char`\"\char`\"h'} & \texttt{g\char`\"\char`\"h} \\
\end{tabu}

Finally, a long quoted string
may be broken up into multiple quoted strings (considered, as a whole,
a single quoted string) and joined with \texttt{\char`\+}.
You can even use multiple lines as long as a
'+' is at the end of every line except the last one.
The \texttt{\char`\+} should be preceded and followed by at least one
white space character.
Thus, for example,
\begin{center}
 \texttt{\char`\"ab\char`\"\ \char`\+\ \char`\"cd\char`\"
    \ \char`\+\ \\ 'ef'}
\end{center}
is the same as
\begin{center}
 \texttt{\char`\"abcdef\char`\"}
\end{center}

{\sc c}-style comments \texttt{/*}\ldots\texttt{*/} are also supported.

In practice, whenever you add a new math token you will often want to add
typesetting definitions using
\texttt{latexdef}, \texttt{htmldef}, and
\texttt{althtmldef}, as described below.
That way, they will all be up to date.
Of course, whether or not you want to use all three definitions will
depend on how the database is intended to be used.

Below we discuss the different possible \textit{definition-kind} options.
We will show data surrounded by double quotes (in practice they can also use
single quotes and/or be a sequence joined by \texttt{+}s).
We will use specific names for the \textit{data} to make clear what
the data is used for, such as
{\em math-token} (for a Metamath math token,
{\em latex-string} (for string to be placed in a \LaTeX\ stream),
{\em {\sc html}-code} (for {\sc html} code),
and {\em filename} (for a filename).

\subsubsection{Typesetting Comment - \LaTeX}

The syntax for a \LaTeX\ definition is:
\begin{center}
 \texttt{latexdef "}{\em math-token}\texttt{" as "}{\em latex-string}\texttt{";}
\end{center}
\index{latex definitions@\LaTeX\ definitions}%
\index{\texttt{latexdef} statement}

The {\em token-string} and {\em latex-string} are the data
(character strings) for
the token and the \LaTeX\ definition of the token, respectively,

These \LaTeX\ definitions are used by the Metamath program
when it is asked to product \LaTeX output using
the \texttt{write tex} command.

\subsubsection{Typesetting Comment - {\sc html}}

The key kinds of {\sc HTML} definitions have the following syntax:

\vskip 1ex
    \texttt{htmldef "}{\em math-token}\texttt{" as "}{\em
    {\sc html}-code}\texttt{";}\index{\texttt{htmldef} statement}
                    \ \ \ \ \ \ldots

    \texttt{althtmldef "}{\em math-token}\texttt{" as "}{\em
{\sc html}-code}\texttt{";}\index{\texttt{althtmldef} statement}

                    \ \ \ \ \ \ldots

Note that in {\sc HTML} there are two possible definitions for math tokens.
This feature is useful when
an alternate representation of symbols is desired, for example one that
uses Unicode entities and another uses {\sc gif} images.

There are many other typesetting definitions that can control {\sc HTML}.
These include:

\vskip 1ex

    \texttt{htmldef "}{\em math-token}\texttt{" as "}{\em {\sc
    html}-code}\texttt{";}

    \texttt{htmltitle "}{\em {\sc html}-code}\texttt{";}%
\index{\texttt{htmltitle} statement}

    \texttt{htmlhome "}{\em {\sc html}-code}\texttt{";}%
\index{\texttt{htmlhome} statement}

    \texttt{htmlvarcolor "}{\em {\sc html}-code}\texttt{";}%
\index{\texttt{htmlvarcolor} statement}

    \texttt{htmlbibliography "}{\em filename}\texttt{";}%
\index{\texttt{htmlbibliography} statement}

\vskip 1ex

\noindent The \texttt{htmltitle} is the {\sc html} code for a common
title, such as ``Metamath Proof Explorer.''  The \texttt{htmlhome} is
code for a link back to the home page.  The \texttt{htmlvarcolor} is
code for a color key that appears at the bottom of each proof.  The file
specified by {\em filename} is an {\sc html} file that is assumed to
have a \texttt{<A NAME=}\ldots\texttt{>} tag for each bibiographic
reference in the database comments.  For example, if
\texttt{[Monk]}\index{\texttt{\char`\[}\ldots\texttt{]} inside comments}
occurs in the comment for a theorem, then \texttt{<A NAME='Monk'>} must
be present in the file; if not, a warning message is given.

Associated with
\texttt{althtmldef}
are the statements
\vskip 1ex

    \texttt{htmldir "}{\em
      directoryname}\texttt{";}\index{\texttt{htmldir} statement}

    \texttt{althtmldir "}{\em
     directoryname}\texttt{";}\index{\texttt{althtmldir} statement}

\vskip 1ex
\noindent giving the directories of the {\sc gif} and Unicode versions
respectively; their purpose is to provide cross-linking between the
two versions in the generated web pages.

When two different types of pages need to be produced from a single
database, such as the Hilbert Space Explorer that extends the Metamath
Proof Explorer, ``extended'' variables may be declared in the
\texttt{\$t} comment:
\vskip 1ex

    \texttt{exthtmltitle "}{\em {\sc html}-code}\texttt{";}%
\index{\texttt{exthtmltitle} statement}

    \texttt{exthtmlhome "}{\em {\sc html}-code}\texttt{";}%
\index{\texttt{exthtmlhome} statement}

    \texttt{exthtmlbibliography "}{\em filename}\texttt{";}%
\index{\texttt{exthtmlbibliography} statement}

\vskip 1ex
\noindent When these are declared, you also must declare
\vskip 1ex

    \texttt{exthtmllabel "}{\em label}\texttt{";}%
\index{\texttt{exthtmllabel} statement}

\vskip 1ex \noindent that identifies the database statement where the
``extended'' section of the database starts (in our example, where the
Hilbert Space Explorer starts).  During the generation of web pages for
that starting statement and the statements after it, the {\sc html} code
assigned to \texttt{exthtmltitle} and \texttt{exthtmlhome} is used
instead of that assigned to \texttt{htmltitle} and \texttt{htmlhome},
respectively.

\begin{sloppy}
\subsection{Additional Information Com\-ment (\texttt{\$j})} \label{jcomment}
\end{sloppy}

The additional information comment, aka the
\texttt{\$j}\index{\texttt{\$j} comment}\index{additional information comment}
comment,
provides a way to add additional structured information that can
be optionally parsed by systems.

The additional information comment is parsed the same way as the
typesetting comment (\texttt{\$t}) (see section \ref{tcomment}).
That is,
the additional information comment begins with the token
\texttt{\$j} within a comment,
and continues until the comment close \texttt{\$)}.
Within an additional information comment is a sequence of one or more
commands of the form \texttt{command arg arg ... ;}
where each of the zero or more \texttt{arg} values
can be either a quoted string or a keyword.
Note that every command ends in an unquoted semicolon.
If a verifier is parsing an additional information comment, but
doesn't recognize a particular command, it must skip the command
by finding the end of the command (an unquoted semicolon).

A database may have 0 or more additional information comments.
Note, however, that a verifier may ignore these comments entirely or only
process certain commands in an additional information comment.
The \texttt{mmj2} verifier supports many commands in additional information
comments.
We encourage systems that process additional information comments
to coordinate so that they will use the same command for the same effect.

Examples of additional information comments with various commands
(from the \texttt{set.mm} database) are:

\begin{itemize}
   \item Define the syntax and logical typecodes,
     and declare that our grammar is
     unambiguous (verifiable using the KLR parser, with compositing depth 5).
\begin{verbatim}
  $( $j
    syntax 'wff';
    syntax '|-' as 'wff';
    unambiguous 'klr 5';
  $)
\end{verbatim}

   \item Register $\lnot$ and $\rightarrow$ as primitive expressions
           (lacking definitions).
\begin{verbatim}
  $( $j primitive 'wn' 'wi'; $)
\end{verbatim}

   \item There is a special justification for \texttt{df-bi}.
\begin{verbatim}
  $( $j justification 'bijust' for 'df-bi'; $)
\end{verbatim}

   \item Register $\leftrightarrow$ as an equality for its type (wff).
\begin{verbatim}
  $( $j
    equality 'wb' from 'biid' 'bicomi' 'bitri';
    definition 'dfbi1' for 'wb';
  $)
\end{verbatim}

   \item Theorem \texttt{notbii} is the congruence law for negation.
\begin{verbatim}
  $( $j congruence 'notbii'; $)
\end{verbatim}

   \item Add \texttt{setvar} as a typecode.
\begin{verbatim}
  $( $j syntax 'setvar'; $)
\end{verbatim}

   \item Register $=$ as an equality for its type (\texttt{class}).
\begin{verbatim}
  $( $j equality 'wceq' from 'eqid' 'eqcomi' 'eqtri'; $)
\end{verbatim}

\end{itemize}


\subsection{Including Other Files in a Metamath Source File} \label{include}
\index{\texttt{\$[} and \texttt{\$]} auxiliary keywords}

The keywords \texttt{\$[} and \texttt{\$]} specify a file to be
included\index{included file}\index{file inclusion} at that point in a
Metamath\index{Metamath} source file\index{source file}.  The syntax for
including a file is as follows:
\begin{center}
\texttt{\$[} {\em file-name} \texttt{\$]}
\end{center}

The {\em file-name} should be a single token\index{token} with the same syntax
as a math symbol (i.e., all 93 non-whitespace
printable characters other than \texttt{\$} are
allowed, subject to the file-naming limitations of your operating system).
Comments may appear between the \texttt{\$[} and \texttt{\$]} keywords.  Included
files may include other files, which may in turn include other files, and so
on.

For example, suppose you want to use the set theory database as the starting
point for your own theory.  The first line in your file could be
\begin{center}
\texttt{\$[ set.mm \$]}
\end{center} All of the information (axioms, theorems,
etc.) in \texttt{set.mm} and any files that {\em it} includes will become
available for you to reference in your file. This can help make your work more
modular. A drawback to including files is that if you change the name of a
symbol or the label of a statement, you must also remember to update any
references in any file that includes it.


The naming conventions for included files are the same as those of your
operating system.\footnote{On the Macintosh, prior to Mac OS X,
 a colon is used to separate disk
and folder names from your file name.  For example, {\em volume}\texttt{:}{\em
file-name} refers to the root directory, {\em volume}\texttt{:}{\em
folder-name}\texttt{:}{\em file-name} refers to a folder in root, and {\em
volume}\texttt{:}{\em folder-name}\texttt{:}\ldots\texttt{:}{\em file-name} refers to a
deeper folder.  A simple {\em file-name} refers to a file in the folder from
which you launch the Metamath application.  Under Mac OS X and later,
the Metamath program is run under the Terminal application, which
conforms to Unix naming conventions.}\index{Macintosh file
names}\index{file names!Macintosh}\label{includef} For compatibility among
operating systems, you should keep the file names as simple as possible.  A
good convention to use is {\em file}\texttt{.mm} where {\em file} is eight
characters or less, in lower case.

There is no limit to the nesting depth of included files.  One thing that you
should be aware of is that if two included files themselves include a common
third file, only the {\em first} reference to this common file will be read
in.  This allows you to include two or more files that build on a common
starting file without having to worry about label and symbol conflicts that
would occur if the common file were read in more than once.  (In fact, if a
file includes itself, the self-reference will be ignored, although of course
it would not make any sense to do that.)  This feature also means, however,
that if you try to include a common file in several inner blocks, the result
might not be what you expect, since only the first reference will be replaced
with the included file (unlike the include statement in most other computer
languages).  Thus you would normally include common files only in the
outermost block\index{outermost block}.

\subsection{Compressed Proof Format}\label{compressed1}\index{compressed
proof}\index{proof!compressed}

The proof notation presented in Section~\ref{proof} is called a
{\bf normal proof}\index{normal proof}\index{proof!normal} and in principle is
sufficient to express any proof.  However, proofs often contain steps and
subproofs that are identical.  This is particularly true in typical
Metamath\index{Metamath} applications, because Metamath requires that the math
symbol sequence (usually containing a formula) at each step be separately
constructed, that is, built up piece by piece. As a result, a lot of
repetition often results.  The {\bf compressed proof} format allows Metamath
to take advantage of this redundancy to shorten proofs.

The specification for the compressed proof format is given in
Appen\-dix~\ref{compressed}.

Normally you need not concern yourself with the details of the compressed
proof format, since the Metamath program will allow you to convert from
the normal format to the compressed format with ease, and will also
automatically convert from the compressed format when proofs are displayed.
The overall structure of the compressed format is as follows:
\begin{center}
  \texttt{\$= ( } {\em label-list} \texttt{) } {\em compressed-proof\ }\ \texttt{\$.}
\end{center}
\index{\texttt{\$=} keyword}
The first \texttt{(} serves as a flag to Metamath that a compressed proof
follows.  The {\em label-list} includes all statements referred to by the
proof except the mandatory hypotheses\index{mandatory hypothesis}.  The {\em
compressed-proof} is a compact encoding of the proof, using upper-case
letters, and can be thought of as a large integer in base 26.  White
space\index{white space} inside a {\em compressed-proof} is
optional and is ignored.

It is important to note that the order of the mandatory hypotheses of
the statement being proved must not be changed if the compressed proof
format is used, otherwise the proof will become incorrect.  The reason
for this is that the mandatory hypotheses are not mentioned explicitly
in the compressed proof in order to make the compression more efficient.
If you wish to change the order of mandatory hypotheses, you must first
convert the proof back to normal format using the \texttt{save proof
{\em statement} /normal}\index{\texttt{save proof} command} command.
Later, you can go back to compressed format with \texttt{save proof {\em
statement} /compressed}.

During error checking with the \texttt{verify proof} command, an error
found in a compressed proof may point to a character in {\em
compressed-proof}, which may not be very meaningful to you.  In this
case, try to \texttt{save proof /normal} first, then do the
\texttt{verify proof} again.  In general, it is best to make sure a
proof is correct before saving it in compressed format, because severe
errors are less likely to be recoverable than in normal format.

\subsection{Specifying Unknown Proofs or Subproofs}\label{unknown}

In a proof under development, any step or subproof that is not yet known
may be represented with a single \texttt{?}.  For the purposes of
parsing the proof, the \texttt{?}\ \index{\texttt{]}@\texttt{?}\ inside
proofs} will push a single entry onto the RPN stack just as if it were a
hypothesis.  While developing a proof with the Proof
Assistant\index{Proof Assistant}, a partially developed proof may be
saved with the \texttt{save new{\char`\_}proof}\index{\texttt{save
new{\char`\_}proof} command} command, and \texttt{?}'s will be placed at
the appropriate places.

All \texttt{\$p}\index{\texttt{\$p} statement} statements must have
proofs, even if they are entirely unknown.  Before creating a proof with
the Proof Assistant, you should specify a completely unknown proof as
follows:
\begin{center}
  {\em label} \texttt{\$p} {\em statement} \texttt{\$= ?\ \$.}
\end{center}
\index{\texttt{\$=} keyword}
\index{\texttt{]}@\texttt{?}\ inside proofs}

The \texttt{verify proof}\index{\texttt{verify proof} command} command
will check the known portions of a partial proof for errors, but will
warn you that the statement has not been proved.

Note that partially developed proofs may be saved in compressed format
if desired.  In this case, you will see one or more \texttt{?}'s in the
{\em compressed-proof} part.\index{compressed
proof}\index{proof!compressed}

\section{Axioms vs.\ Definitions}\label{definitions}

The \textit{basic}
Metamath\index{Metamath} language and program
make no distinction\index{axiom vs.\
definition} between axioms\index{axiom} and
definitions.\index{definition} The \texttt{\$a}\index{\texttt{\$a}
statement} statement is used for both.  At first, this may seem
puzzling.  In the minds of many mathematicians, the distinction is
clear, even obvious, and hardly worth discussing.  A definition is
considered to be merely an abbreviation that can be replaced by the
expression for which it stands; although unless one actually does this,
to be precise then one should say that a theorem\index{theorem} is a
consequence of the axioms {\em and} the definitions that are used in the
formulation of the theorem \cite[p.~20]{Behnke}.\index{Behnke, H.}

\subsection{What is a Definition?}

What is a definition?  In its simplest form, a definition introduces a new
symbol and provides an unambiguous rule to transform an expression containing
the new symbol to one without it.  The concept of a ``proper
definition''\index{proper definition}\index{definition!proper} (as opposed to
a creative definition)\index{creative definition}\index{definition!creative}
that is usually agreed upon is (1) the definition should not strengthen the
language and (2) any symbols introduced by the definition should be eliminable
from the language \cite{Nemesszeghy}\index{Nemesszeghy, E. Z.}.  In other
words, they are mere typographical conveniences that do not belong to the
system and are theoretically superfluous.  This may seem obvious, but in fact
the nature of definitions can be subtle, sometimes requiring difficult
metatheorems to establish that they are not creative.

A more conservative stance was taken by logician S.
Le\'{s}niewski.\index{Le\'{s}niewski, S.}
\begin{quote}
Le\'{s}niewski
regards definitions as theses of the system.  In this respect they do
not differ either from the axioms or from theorems, i.e.\ from the
theses added to the system on the basis of the rule of substitution or
the rule of detachment [modus ponens].  Once definitions have been
accepted as theses of the system, it becomes necessary to consider them
as true propositions in the same sense in which axioms are true
\cite{Lejewski}.
\end{quote}\index{Lejewski, Czeslaw}

Let us look at some simple examples of definitions in propositional
calculus.  Consider the definition of logical {\sc or}
(disjunction):\index{disjunction ($\vee$)} ``$P\vee Q$ denotes $\neg P
\rightarrow Q$ (not $P$ implies $Q$).''  It is very easy to recognize a
statement making use of this definition, because it introduces the new
symbol $\vee$ that did not previously exist in the language.  It is easy
to see that no new theorems of the original language will result from
this definition.

Next, consider a definition that eliminates parentheses:  ``$P
\rightarrow Q\rightarrow R$ denotes $P\rightarrow (Q \rightarrow R)$.''
This is more subtle, because no new symbols are introduced.  The reason
this definition is considered proper is that no new symbol sequences
that are valid wffs (well-formed formulas)\index{well-formed formula
(wff)} in the original language will result from the definition, since
``$P \rightarrow Q\rightarrow R$'' is not a wff in the original
language.  Here, we implicitly make use of the fact that there is a
decision procedure that allows us to determine whether or not a symbol
sequence is a wff, and this fact allows us to use symbol sequences that
are not wffs to represent other things (such as wffs) by means of the
definition.  However, to justify the definition as not being creative we
need to prove that ``$P \rightarrow Q\rightarrow R$'' is in fact not a
wff in the original language, and this is more difficult than in the
case where we simply introduce a new symbol.

%Now let's take this reasoning to an extreme.  Propositional calculus is a
%decidable theory,\footnote{This means that a mechanical algorithm exists to
%determine whether or not a wff is a theorem.} so in principle we could make use
%of symbol sequences that are not theorems to represent other things (say, to
%encode actual theorems in a more compact way).  For example, let us extend the
%language by defining a wff ``$P$'' in the extended language as the theorem
%``$P\rightarrow P$''\footnote{This is one of the first theorems proved in the
%Metamath database \texttt{set.mm}.}\index{set
%theory database (\texttt{set.mm})} in the original language whenever ``$P$'' is
%not a theorem in the original language.  In the extended language, any wff
%``$Q$'' thus represents a theorem; to find out what theorem (in the original
%language) ``$Q$'' represents, we determine whether ``$Q$'' is a theorem in the
%original language (before the definition was introduced).  If so, we're done; if
%not, we replace ``$Q$'' by ``$Q\rightarrow Q$'' to eliminate the definition.
%This definition is therefore eliminable, and it does not ``strengthen'' the
%language because any wff that is not a theorem is not in the set of statements
%provable in the original language and thus is available for use by definitions.
%
%Of course, a definition such as this would render practically useless the
%communication of theorems of propositional calculus; but
%this is just a human shortcoming, since we can't always easily discern what is
%and is not a theorem by inspection.  In fact, the extended theory with this
%definition has no more and no less information than the original theory; it just
%expresses certain theorems of the form ``$P\rightarrow P$''
%in a more compact way.
%
%The point here is that what constitutes a proper definition is a matter of
%judgment about whether a symbol sequence can easily be recognized by a human
%as invalid in some sense (for example, not a wff); if so, the symbol sequence
%can be appropriated for use by a definition in order to make the extended
%language more compact.  Metamath\index{Metamath} lacks the ability to make this
%judgment, since as far as Metamath is concerned the definition of a wff, for
%example, is arbitrary.  You define for Metamath how wffs\index{well-formed
%formula (wff)} are constructed according to your own preferred style.  The
%concept of a wff may not even exist in a given formal system\index{formal
%system}.  Metamath treats all definitions as if they were new axioms, and it
%is up to the human mathematician to judge whether the definition is ``proper''
%'\index{proper definition}\index{definition!proper} in some agreed-upon way.

What constitutes a definition\index{definition} versus\index{axiom vs.\
definition} an axiom\index{axiom} is sometimes arbitrary in mathematical
literature.  For example, the connectives $\vee$ ({\sc or}), $\wedge$
({\sc and}), and $\leftrightarrow$ (equivalent to) in propositional
calculus are usually considered defined symbols that can be used as
abbreviations for expressions containing the ``primitive'' connectives
$\rightarrow$ and $\neg$.  This is the way we treat them in the standard
logic and set theory database \texttt{set.mm}\index{set theory database
(\texttt{set.mm})}.  However, the first three connectives can also be
considered ``primitive,'' and axiom systems have been devised that treat
all of them as such.  For example,
\cite[p.~35]{Goodstein}\index{Goodstein, R. L.} presents one with 15
axioms, some of which in fact coincide with what we have chosen to call
definitions in \texttt{set.mm}.  In certain subsets of classical
propositional calculus, such as the intuitionist
fragment\index{intuitionism}, it can be shown that one cannot make do
with just $\rightarrow$ and $\neg$ but must treat additional connectives
as primitive in order for the system to make sense.\footnote{Two nice
systems that make the transition from intuitionistic and other weak
fragments to classical logic just by adding axioms are given in
\cite{Robinsont}\index{Robinson, T. Thacher}.}

\subsection{The Approach to Definitions in \texttt{set.mm}}

In set theory, recursive definitions define a newly introduced symbol in
terms of itself.
The justification of recursive definitions, using
several ``recursion theorems,'' is usually one of the first
sophisticated proofs a student encounters when learning set theory, and
there is a significant amount of implicit metalogic behind a recursive
definition even though the definition itself is typically simple to
state.

Metamath itself has no built-in technical limitation that prevents
multiple-part recursive definitions in the traditional textbook style.
However, because the recursive definition requires advanced metalogic
to justify, eliminating a recursive definition is very difficult and
often not even shown in textbooks.

\subsubsection{Direct definitions instead of recursive definitions}

It is, however, possible to substitute one kind of complexity
for another.  We can eliminate the need for metalogical justification by
defining the operation directly with an explicit (but complicated)
expression, then deriving the recursive definition directly as a
theorem, using a recursion theorem ``in reverse.''
The elimination
of a direct definition is a matter of simple mechanical substitution.
We do this in
\texttt{set.mm}, as follows.

In \texttt{set.mm} our goal was to introduce almost all definitions in
the form of two expressions connected by either $\leftrightarrow$ or
$=$, where the thing being defined does not appear on the right hand
side.  Quine calls this form ``a genuine or direct definition'' \cite[p.
174]{Quine}\index{Quine, Willard Van Orman}, which makes the definitions
very easy to eliminate and the metalogic\index{metalogic} needed to
justify them as simple as possible.
Put another way, we had a goal of being able to
eliminate all definitions with direct mechanical substitution and to
verify easily the soundness of the definitions.

\subsubsection{Example of direct definitions}

We achieved this goal in almost all cases in \texttt{set.mm}.
Sometimes this makes the definitions more complex and less
intuitive.
For example, the traditional way to define addition of
natural numbers is to define an operation called {\em
successor}\index{successor} (which means ``plus one'' and is denoted by
``${\rm suc}$''), then define addition recursively\index{recursive
definition} with the two definitions $n + 0 = n$ and $m + {\rm suc}\,n =
{\rm suc} (m + n)$.  Although this definition seems simple and obvious,
the method to eliminate the definition is not obvious:  in the second
part of the definition, addition is defined in terms of itself.  By
eliminating the definition, we don't mean repeatedly applying it to
specific $m$ and $n$ but rather showing the explicit, closed-form
set-theoretical expression that $m + n$ represents, that will work for
any $m$ and $n$ and that does not have a $+$ sign on its right-hand
side.  For a recursive definition like this not to be circular
(creative), there are some hidden, underlying assumptions we must make,
for example that the natural numbers have a certain kind of order.

In \texttt{set.mm} we chose to start with the direct (though complex and
nonintuitive) definition then derive from it the standard recursive
definition.
For example, the closed-form definition used in \texttt{set.mm}
for the addition operation on ordinals\index{ordinal
addition}\index{addition!of ordinals} (of which natural numbers are a
subset) is

\setbox\startprefix=\hbox{\tt \ \ df-oadd\ \$a\ }
\setbox\contprefix=\hbox{\tt \ \ \ \ \ \ \ \ \ \ \ \ \ }
\startm
\m{\vdash}\m{+_o}\m{=}\m{(}\m{x}\m{\in}\m{{\rm On}}\m{,}\m{y}\m{\in}\m{{\rm
On}}\m{\mapsto}\m{(}\m{{\rm rec}}\m{(}\m{(}\m{z}\m{\in}\m{{\rm
V}}\m{\mapsto}\m{{\rm suc}}\m{z}\m{)}\m{,}\m{x}\m{)}\m{`}\m{y}\m{)}\m{)}
\endm
\noindent which depends on ${\rm rec}$.

\subsubsection{Recursion operators}

The above definition of \texttt{df-oadd} depends on the definition of
${\rm rec}$, a ``recursion operator''\index{recursion operator} with
the definition \texttt{df-rdg}:

\setbox\startprefix=\hbox{\tt \ \ df-rdg\ \$a\ }
\setbox\contprefix=\hbox{\tt \ \ \ \ \ \ \ \ \ \ \ \ }
\startm
\m{\vdash}\m{{\rm
rec}}\m{(}\m{F}\m{,}\m{I}\m{)}\m{=}\m{\mathrm{recs}}\m{(}\m{(}\m{g}\m{\in}\m{{\rm
V}}\m{\mapsto}\m{{\rm if}}\m{(}\m{g}\m{=}\m{\varnothing}\m{,}\m{I}\m{,}\m{{\rm
if}}\m{(}\m{{\rm Lim}}\m{{\rm dom}}\m{g}\m{,}\m{\bigcup}\m{{\rm
ran}}\m{g}\m{,}\m{(}\m{F}\m{`}\m{(}\m{g}\m{`}\m{\bigcup}\m{{\rm
dom}}\m{g}\m{)}\m{)}\m{)}\m{)}\m{)}\m{)}
\endm

\noindent which can be further broken down with definitions shown in
Section~\ref{setdefinitions}.

This definition of ${\rm rec}$
defines a recursive definition generator on ${\rm On}$ (the class of ordinal
numbers) with characteristic function $F$ and initial value $I$.
This operation allows us to define, with
compact direct definitions, functions that are usually defined in
textbooks with recursive definitions.
The price paid with our approach
is the complexity of our ${\rm rec}$ operation
(especially when {\tt df-recs} that it is built on is also eliminated).
But once we get past this hurdle, definitions that would otherwise be
recursive become relatively simple, as in for example {\tt oav}, from
which we prove the recursive textbook definition as theorems {\tt oa0}, {\tt
oasuc}, and {\tt oalim} (with the help of theorems {\tt rdg0}, {\tt rdgsuc},
and {\tt rdglim2a}).  We can also restrict the ${\rm rec}$ operation to
define otherwise recursive functions on the natural numbers $\omega$; see {\tt
fr0g} and {\tt frsuc}.  Our ${\rm rec}$ operation apparently does not appear
in published literature, although closely related is Definition 25.2 of
[Quine] p. 177, which he uses to ``turn...a recursion into a genuine or
direct definition" (p. 174).  Note that the ${\rm if}$ operations (see
{\tt df-if}) select cases based on whether the domain of $g$ is zero, a
successor, or a limit ordinal.

An important use of this definition ${\rm rec}$ is in the recursive sequence
generator {\tt df-seq} on the natural numbers (as a subset of the
complex infinite sequences such as the factorial function {\tt df-fac} and
integer powers {\tt df-exp}).

The definition of ${\rm rec}$ depends on ${\rm recs}$.
New direct usage of the more powerful (and more primitive) ${\rm recs}$
construct is discouraged, but it is available when needed.
This
defines a function $\mathrm{recs} ( F )$ on ${\rm On}$, the class of ordinal
numbers, by transfinite recursion given a rule $F$ which sets the next
value given all values so far.
Unlike {\tt df-rdg} which restricts the
update rule to use only the previous value, this version allows the
update rule to use all previous values, which is why it is described
as ``strong,'' although it is actually more primitive.  See {\tt
recsfnon} and {\tt recsval} for the primary contract of this definition.
It is defined as:

\setbox\startprefix=\hbox{\tt \ \ df-recs\ \$a\ }
\setbox\contprefix=\hbox{\tt \ \ \ \ \ \ \ \ \ \ \ \ \ }
\startm
\m{\vdash}\m{\mathrm{recs}}\m{(}\m{F}\m{)}\m{=}\m{\bigcup}\m{\{}\m{f}\m{|}\m{\exists}\m{x}\m{\in}\m{{\rm
On}}\m{(}\m{f}\m{{\rm
Fn}}\m{x}\m{\wedge}\m{\forall}\m{y}\m{\in}\m{x}\m{(}\m{f}\m{`}\m{y}\m{)}\m{=}\m{(}\m{F}\m{`}\m{(}\m{f}\m{\restriction}\m{y}\m{)}\m{)}\m{)}\m{\}}
\endm

\subsubsection{Closing comments on direct definitions}

From these direct definitions the simpler, more
intuitive recursive definition is derived as a set of theorems.\index{natural
number}\index{addition}\index{recursive definition}\index{ordinal addition}
The end result is the same, but we completely eliminate the rather complex
metalogic that justifies the recursive definition.

Recursive definitions are often considered more efficient and intuitive than
direct ones once the metalogic has been learned or possibly just accepted as
correct.  However, it was felt that direct definition in \texttt{set.mm}
maximizes rigor by minimizing metalogic.  It can be eliminated effortlessly,
something that is difficult to do with a recursive definition.

Again, Metamath itself has no built-in technical limitation that prevents
multiple-part recursive definitions in the traditional textbook style.
Instead, our goal is to eliminate all definitions with
direct mechanical substitution and to verify easily the soundness of
definitions.

\subsection{Adding Constraints on Definitions}

The basic Metamath language and the Metamath program do
not have any built-in constraints on definitions, since they are just
\$a statements.

However, nothing prevents a verification system from verifying
additional rules to impose further limitations on definitions.
For example, the \texttt{mmj2}\index{mmj2} program
supports various kinds of
additional information comments (see section \ref{jcomment}).
One of their uses is to optionally verify additional constraints,
including constraints to verify that definitions meet certain
requirements.
These additional checks are required by the
continuous integration (CI)\index{continuous integration (CI)}
checks of the
\texttt{set.mm}\index{set theory database (\texttt{set.mm})}%
\index{Metamath Proof Explorer}
database.
This approach enables us to optionally impose additional requirements
on definitions if we wish, without necessarily imposing those rules on
all databases or requiring all verification systems to implement them.
In addition, this allows us to impose specialized constraints tailored
to one database while not requiring other systems to implement
those specialized constraints.

We impose two constraints on the
\texttt{set.mm}\index{set theory database (\texttt{set.mm})}%
\index{Metamath Proof Explorer} database
via the \texttt{mmj2}\index{mmj2} program that are worth discussing here:
a parse check and a definition soundness check.

% On February 11, 2019 8:32:32 PM EST, saueran@oregonstate.edu wrote:
% The following addition to the end of set.mm is accepted by the mmj2
% parser and definition checker and the metamath verifier(at least it was
% when I checked, you should check it too), and creates a contradiction by
% proving the theorem |- ph.
% ${
% wleftp $a wff ( ( ph ) $.
% wbothp $a wff ( ph ) $.
% df-leftp $a |- ( ( ( ph ) <-> -. ph ) $.
% df-bothp $a |- ( ( ph ) <-> ph ) $.
% anything $p |- ph $=
%   ( wbothp wn wi wleftp df-leftp biimpi df-bothp mpbir mpbi simplim ax-mp)
%   ABZAMACZDZCZMOEZOCQAEZNDZRNAFGSHIOFJMNKLAHJ $.
% $}
%
% This particular problem is countered by enabling, within mmj2,
% SetParser,mmj.verify.LRParser

First,
we enable a parse check in \texttt{mmj2} (through its
\texttt{SetParser} command) that requires that all new definitions
must \textit{not} create an ambiguous parse for a KLR(5) parser.
This prevents some errors such as definitions with imbalanced parentheses.

Second, we run a definition soundness check specific to
\texttt{set.mm} or databases similar to it.
(through the \texttt{definitionCheck} macro).
Some \texttt{\$a} statements (including all ax-* statemnets)
are excluded from these checks, as they will
always fail this simple check,
but they are appropriate for most definitions.
This check imposes a set of additional rules:

\begin{enumerate}

\item New definitions must be introduced using $=$ or $\leftrightarrow$.

\item No \texttt{\$a} statement introduced before this one is allowed to use the
symbol being defined in this definition, and the definition is not
allowed to use itself (except once, in the definiendum).

\item Every variable in the definiens should not be distinct

\item Every dummy variable in the definiendum
are required to be distinct from each other and from variables in
the definiendum.
To determine this, the system will look for a "justification" theorem
in the database, and if it is not there, attempt to briefly prove
$( \varphi \rightarrow \forall x \varphi )$  for each dummy variable x.

\item Every dummy variable should be a set variable,
unless there is a justification theorem available.

\item Every dummy variable must be bound
(if the system cannot determine this a justification theorem must be
provided).

\end{enumerate}

\subsection{Summary of Approach to Definitions}

In short, when being rigorous it turns out that
definitions can be subtle, sometimes requiring difficult
metatheorems to establish that they are not creative.

Instead of building such complications into the Metamath language itself,
the basic Metmath language and program simply treat traditional
axioms and definitions as the same kind of \texttt{\$a} statement.
We have then built various tools to enable people to
verify additional conditions as their creators believe is appropriate
for those specific databases, without complicating the Metamath foundations.

\chapter{The Metamath Program}\label{commands}

This chapter provides a reference manual for the
Metamath program.\index{Metamath!commands}

Current instructions for obtaining and installing the Metamath program
can be found at the \url{http://metamath.org} web site.
For Windows, there is a pre-compiled version called
\texttt{metamath.exe}.  For Unix, Linux, and Mac OS X
(which we will refer to collectively as ``Unix''), the Metamath program
can be compiled from its source code with the command
\begin{verbatim}
gcc *.c -o metamath
\end{verbatim}
using the \texttt{gcc} {\sc c} compiler available on those systems.

In the command syntax descriptions below, fields enclosed in square
brackets [\ ] are optional.  File names may be optionally enclosed in
single or double quotes.  This is useful if the file name contains
spaces or
slashes (\texttt{/}), such as in Unix path names, \index{Unix file
names}\index{file names!Unix} that might be confused with Metamath
command qualifiers.\index{Unix file names}\index{file names!Unix}

\section{Invoking Metamath}

Unix, Linux, and Mac OS X
have a command-line interface called the {\em
bash shell}.  (In Mac OS X, select the Terminal application from
Applications/Utilities.)  To invoke Metamath from the bash shell prompt,
assuming that the Metamath program is in the current directory, type
\begin{verbatim}
bash$ ./metamath
\end{verbatim}

To invoke Metamath from a Windows DOS or Command Prompt, assuming that
the Metamath program is in the current directory (or in a directory
included in the Path system environment variable), type
\begin{verbatim}
C:\metamath>metamath
\end{verbatim}

To use command-line arguments at invocation, the command-line arguments
should be a list of Metamath commands, surrounded by quotes if they
contain spaces.  In Windows, the surrounding quotes must be double (not
single) quotes.  For example, to read the database file \texttt{set.mm},
verify all proofs, and exit the program, type (under Unix)
\begin{verbatim}
bash$ ./metamath 'read set.mm' 'verify proof *' exit
\end{verbatim}
Note that in Unix, any directory path with \texttt{/}'s must be
surrounded by quotes so Metamath will not interpret the \texttt{/} as a
command qualifier.  So if \texttt{set.mm} is in the \texttt{/tmp}
directory, use for the above example
\begin{verbatim}
bash$ ./metamath 'read "/tmp/set.mm"' 'verify proof *' exit
\end{verbatim}

For convenience, if the command-line has one argument and no spaces in
the argument, the command is implicitly assumed to be \texttt{read}.  In
this one special case, \texttt{/}'s are not interpreted as command
qualifiers, so you don't need quotes around a Unix file name.  Thus
\begin{verbatim}
bash$ ./metamath /tmp/set.mm
\end{verbatim}
and
\begin{verbatim}
bash$ ./metamath "read '/tmp/set.mm'"
\end{verbatim}
are equivalent.


\section{Controlling Metamath}

The Metamath program was first developed on a {\sc vax/vms} system, and
some aspects of its command line behavior reflect this heritage.
Hopefully you will find it reasonably user-friendly once you get used to
it.

Each command line is a sequence of English-like words separated by
spaces, as in \texttt{show settings}.  Command words are not case
sensitive, and only as many letters are needed as are necessary to
eliminate ambiguity; for example, \texttt{sh se} would work for the
command \texttt{show settings}.  In some cases arguments such as file
names, statement labels, or symbol names are required; these are
case-sensitive (although file names may not be on some operating
systems).

A command line is entered by typing it in then pressing the {\em return}
({\em enter}) key.  To find out what commands are available, type
\texttt{?} at the \texttt{MM>} prompt.  To find out the choices at any
point in a command, press {\em return} and you will be prompted for
them.  The default choice (the one selected if you just press {\em
return}) is shown in brackets (\texttt{<>}).

You may also type \texttt{?} in place of a command word to force
Metamath to tell you what the choices are.  The \texttt{?} method won't
work, though, if a non-keyword argument such as a file name is expected
at that point, because the program will think that \texttt{?} is the
value of the argument.

Some commands have one or more optional qualifiers which modify the
behavior of the command.  Qualifiers are preceded by a slash
(\texttt{/}), such as in \texttt{read set.mm / verify}.  Spaces are
optional around the \texttt{/}.  If you need to use a space or
slash in a command
argument, as in a Unix file name, put single or double quotes around the
command argument.

The \texttt{open log} command will save everything you see on the
screen and is useful to help you recover should something go wrong in a
proof, or if you want to document a bug.

If a command responds with more than a screenful, you will be
prompted to \texttt{<return> to continue, Q to quit, or S to scroll to
end}.  \texttt{Q} or \texttt{q} (not case-sensitive) will complete the
command internally but will suppress further output until the next
\texttt{MM>} prompt.  \texttt{s} will suppress further pausing until the
next \texttt{MM>} prompt.  After the first screen, you are also
presented with the choice of \texttt{b} to go back a screenful.  Note
that \texttt{b} may also be entered at the \texttt{MM>} prompt
immediately after a command to scroll back through the output of that
command.

A command line enclosed in quotes is executed by your operating system.
See Section~\ref{oscmd}.

{\em Warning:} Pressing {\sc ctrl-c} will abort the Metamath program
unconditionally.  This means any unsaved work will be lost.


\subsection{\texttt{exit} Command}\index{\texttt{exit} command}

Syntax:  \texttt{exit} [\texttt{/force}]

This command exits from Metamath.  If there have been changes to the
source with the \texttt{save proof} or \texttt{save new{\char`\_}proof}
commands, you will be given an opportunity to \texttt{write source} to
permanently save the changes.

In Proof Assistant\index{Proof Assistant} mode, the \texttt{exit} command will
return to the \verb/MM>/ prompt. If there were changes to the proof, you will
be given an opportunity to \texttt{save new{\char`\_}proof}.

The \texttt{quit} command is a synonym for \texttt{exit}.

Optional qualifier:
    \texttt{/force} - Do not prompt if changes were not saved.  This qualifier is
        useful in \texttt{submit} command files (Section~\ref{sbmt})
        to ensure predictable behavior.





\subsection{\texttt{open log} Command}\index{\texttt{open log} command}
Syntax:  \texttt{open log} {\em file-name}

This command will open a log file that will store everything you see on
the screen.  It is useful to help recovery from a mistake in a long Proof
Assistant session, or to document bugs.\index{Metamath!bugs}

The log file can be closed with \texttt{close log}.  It will automatically be
closed upon exiting Metamath.



\subsection{\texttt{close log} Command}\index{\texttt{close log} command}
Syntax:  \texttt{close log}

The \texttt{close log} command closes a log file if one is open.  See
also \texttt{open log}.




\subsection{\texttt{submit} Command}\index{\texttt{submit} command}\label{sbmt}
Syntax:  \texttt{submit} {\em filename}

This command causes further command lines to be taken from the specified
file.  Note that any line beginning with an exclamation point (\texttt{!}) is
treated as a comment (i.e.\ ignored).  Also note that the scrolling
of the screen output is continuous, so you may want to open a log file
(see \texttt{open log}) to record the results that fly by on the screen.
After the lines in the file are exhausted, Metamath returns to its
normal user interface mode.

The \texttt{submit} command can be called recursively (i.e. from inside
of a \texttt{submit} command file).


Optional command qualifier:

    \texttt{/silent} - suppresses the screen output but still
        records the output in a log file if one is open.


\subsection{\texttt{erase} Command}\index{\texttt{erase} command}
Syntax:  \texttt{erase}

This command will reset Metamath to its starting state, deleting any
data\-base that was \texttt{read} in.
 If there have been changes to the
source with the \texttt{save proof} or \texttt{save new{\char`\_}proof}
commands, you will be given an opportunity to \texttt{write source} to
permanently save the changes.



\subsection{\texttt{set echo} Command}\index{\texttt{set echo} command}
Syntax:  \texttt{set echo on} or \texttt{set echo off}

The \texttt{set echo on} command will cause command lines to be echoed with any
abbreviations expanded.  While learning the Metamath commands, this
feature will show you the exact command that your abbreviated input
corresponds to.



\subsection{\texttt{set scroll} Command}\index{\texttt{set scroll} command}
Syntax:  \texttt{set scroll prompted} or \texttt{set scroll continuous}

The Metamath command line interface starts off in the \texttt{prompted} mode,
which means that you will be prompted to continue or quit after each
full screen in a long listing.  In \texttt{continuous} mode, long listings will be
scrolled without pausing.

% LaTeX bug? (1) \texttt{\_} puts out different character than
% \texttt{{\char`\_}}
%  = \verb$_$  (2) \texttt{{\char`\_}} puts out garbage in \subsection
%  argument
\subsection{\texttt{set width} Command}\index{\texttt{set
width} command}
Syntax:  \texttt{set width} {\em number}

Metamath assumes the width of your screen is 79 characters (chosen
because the Command Prompt in Windows XP has a wrapping bug at column
80).  If your screen is wider or narrower, this command allows you to
change this default screen width.  A larger width is advantageous for
logging proofs to an output file to be printed on a wide printer.  A
smaller width may be necessary on some terminals; in this case, the
wrapping of the information messages may sometimes seem somewhat
unnatural, however.  In \LaTeX\index{latex@{\LaTeX}!characters per line},
there is normally a maximum of 61 characters per line with typewriter
font.  (The examples in this book were produced with 61 characters per
line.)

\subsection{\texttt{set height} Command}\index{\texttt{set
height} command}
Syntax:  \texttt{set height} {\em number}

Metamath assumes your screen height is 24 lines of characters.  If your
screen is taller or shorter, this command lets you to change the number
of lines at which the display pauses and prompts you to continue.

\subsection{\texttt{beep} Command}\index{\texttt{beep} command}

Syntax:  \texttt{beep}

This command will produce a beep.  By typing it ahead after a
long-running command has started, it will alert you that the command is
finished.  For convenience, \texttt{b} is an abbreviation for
\texttt{beep}.

Note:  If \texttt{b} is typed at the \texttt{MM>} prompt immediately
after the end of a multiple-page display paged with ``\texttt{Press
<return> for more}...'' prompts, then the \texttt{b} will back up to the
previous page rather than perform the \texttt{beep} command.
In that case you must type the unabbreviated \texttt{beep} form
of the command.

\subsection{\texttt{more} Command}\index{\texttt{more} command}

Syntax:  \texttt{more} {\em filename}

This command will display the contents of an {\sc ascii} file on your
screen.  (This command is provided for convenience but is not very
powerful.  See Section~\ref{oscmd} to invoke your operating system's
command to do this, such as the \texttt{more} command in Unix.)

\subsection{Operating System Commands}\index{operating system
command}\label{oscmd}

A line enclosed in single or double quotes will be executed by your
computer's operating system if it has a command line interface.  For
example, on a {\sc vax/vms} system,
\verb/MM> 'dir'/
will print disk directory contents.  Note that this feature will not
work on the Macintosh prior to Mac OS X, which does not have a command
line interface.

For your convenience, the trailing quote is optional.

\subsection{Size Limitations in Metamath}

In general, there are no fixed, predefined limits\index{Metamath!memory
limits} on how many labels, tokens\index{token}, statements, etc.\ that
you may have in a database file.  The Metamath program uses 32-bit
variables (64-bit on 64-bit CPUs) as indices for almost all internal
arrays, which are allocated dynamically as needed.



\section{Reading and Writing Files}

The following commands create new files:  the \texttt{open} commands;
the \texttt{write} commands; the \texttt{/html},
\texttt{/alt{\char`\_}html}, \texttt{/brief{\char`\_}html},
\texttt{/brief{\char`\_}alt{\char`\_}html} qualifiers of \texttt{show
statement}, and \texttt{midi}.  The following commands append to files
previously opened:  the \texttt{/tex} qualifier of \texttt{show proof}
and \texttt{show new{\char`\_}proof}; the \texttt{/tex} and
\texttt{/simple{\char`\_}tex} qualifiers of \texttt{show statement}; the
\texttt{close} commands; and all screen dialog between \texttt{open log}
and \texttt{close log}.

The commands that create new files will not overwrite an existing {\em
filename} but will rename the existing one to {\em
filename}\texttt{{\char`\~}1}.  An existing {\em
filename}\texttt{{\char`\~}1} is renamed {\em
filename}\texttt{{\char`\~}2}, etc.\ up to {\em
filename}\texttt{{\char`\~}9}.  An existing {\em
filename}\texttt{{\char`\~}9} is deleted.  This makes recovery from
mistakes easier but also will clutter up your directory, so occasionally
you may want to clean up (delete) these \texttt{{\char`\~}}$n$ files.


\subsection{\texttt{read} Command}\index{\texttt{read} command}
Syntax:  \texttt{read} {\em file-name} [\texttt{/verify}]

This command will read in a Metamath language source file and any included
files.  Normally it will be the first thing you do when entering Metamath.
Statement syntax is checked, but proof syntax is not checked.
Note that the file name may be enclosed in single or double quotes;
this is useful if the file name contains slashes, as might be the case
under Unix.

If you are getting an ``\texttt{?Expected VERIFY}'' error
when trying to read a Unix file name with slashes, you probably haven't
quoted it.\index{Unix file names}\index{file names!Unix}

If you are prompted for the file name (by pressing {\em return}
 after \texttt{read})
you should {\em not} put quotes around it, even if it is a Unix file name
with slashes.

Optional command qualifier:

    \texttt{/verify} - Verify all proofs as the database is read in.  This
         qualifier will slow down reading in the file.  See \texttt{verify
         proof} for more information on file error-checking.

See also \texttt{erase}.



\subsection{\texttt{write source} Command}\index{\texttt{write source} command}
Syntax:  \texttt{write source} {\em filename}
[\texttt{/rewrap}]
[\texttt{/split}]
% TeX doesn't handle this long line with tt text very well,
% so force a line break here.
[\texttt{/keep\_includes}] {\\}
[\texttt{/no\_versioning}]

This command will write the contents of a Metamath\index{database}
database into a file.\index{source file}

Optional command qualifiers:

\texttt{/rewrap} -
Reformats statements and comments according to the
convention used in the set.mm database.
It unwraps the
lines in the comment before each \texttt{\$a} and \texttt{\$p} statement,
then it
rewraps the line.  You should compare the output to the original
to make sure that the desired effect results; if not, go back to
the original source.  The wrapped line length honors the
\texttt{set width}
parameter currently in effect.  Note:  Text
enclosed in \texttt{<HTML>}...\texttt{</HTML>} tags is not modified by the
\texttt{/rewrap} qualifier.
Proofs themselves are not reformatted;
use \texttt{save proof * / compressed} to do that.
An isolated tilde (\~{}) is always kept on the same line as the following
symbol, so you can find all comment references to a symbol by
searching for \~{} followed by a space and that symbol
(this is useful for finding cross-references).
Incidentally, \texttt{save proof} also honors the \texttt{set width}
parameter currently in effect.

\texttt{/split} - Files included in the source using the expression
\$[ \textit{inclfile} \$] will be
written into separate files instead of being included in a single output
file.  The name of each separately written included file will be the
\textit{inclfile} argument of its inclusion command.

\texttt{/keep\_includes} - If a source file has includes but is written as a
single file by omitting \texttt{/split}, by default the included files will
be deleted (actually just renamed with a \char`\~1 suffix unless
\texttt{/no\_versioning} is specified) to prevent the possibly confusing
source duplication in both the output file and the included file.
The \texttt{/keep\_includes} qualifier will prevent this deletion.

\texttt{/no\_versioning} - Backup files suffixed with \char`\~1 are not created.


\section{Showing Status and Statements}



\subsection{\texttt{show settings} Command}\index{\texttt{show settings} command}
Syntax:  \texttt{show settings}

This command shows the state of various parameters.

\subsection{\texttt{show memory} Command}\index{\texttt{show memory} command}
Syntax:  \texttt{show memory}

This command shows the available memory left.  It is not meaningful
on most modern operating systems,
which have virtual memory.\index{Metamath!memory usage}


\subsection{\texttt{show labels} Command}\index{\texttt{show labels} command}
Syntax:  \texttt{show labels} {\em label-match} [\texttt{/all}]
   [\texttt{/linear}]

This command shows the labels of \texttt{\$a} and \texttt{\$p}
statements that match {\em label-match}.  A \verb$*$ in {label-match}
matches zero or more characters.  For example, \verb$*abc*def$ will match all
labels containing \verb$abc$ and ending with \verb$def$.

Optional command qualifiers:

   \texttt{/all} - Include matches for \texttt{\$e} and \texttt{\$f}
   statement labels.

   \texttt{/linear} - Display only one label per line.  This can be useful for
       building scripts in conjunction with the utilities under the
       \texttt{tools}\index{\texttt{tools} command} command.



\subsection{\texttt{show statement} Command}\index{\texttt{show statement} command}
Syntax:  \texttt{show statement} {\em label-match} [{\em qualifiers} (see below)]

This command provides information about a statement.  Only statements
that have labels (\texttt{\$f}\index{\texttt{\$f} statement},
\texttt{\$e}\index{\texttt{\$e} statement},
\texttt{\$a}\index{\texttt{\$a} statement}, and
\texttt{\$p}\index{\texttt{\$p} statement}) may be specified.
If {\em label-match}
contains wildcard (\verb$*$) characters, all matching statements will be
displayed in the order they occur in the database.

Optional qualifiers (only one qualifier at a time is allowed):

    \texttt{/comment} - This qualifier includes the comment that immediately
       precedes the statement.

    \texttt{/full} - Show complete information about each statement,
       and show all
       statements matching {\em label} (including \texttt{\$e}
       and \texttt{\$f} statements).

    \texttt{/tex} - This qualifier will write the statement information to the
       \LaTeX\ file previously opened with \texttt{open tex}.  See
       Section~\ref{texout}.

    \texttt{/simple{\char`\_}tex} - The same as \texttt{/tex}, except that
       \LaTeX\ macros are not used for formatting equations, allowing easier
       manual edits of the output for slide presentations, etc.

    \texttt{/html}\index{html generation@{\sc html} generation},
       \texttt{/alt{\char`\_}html}, \texttt{/brief{\char`\_}html},
       \texttt{/brief{\char`\_}alt{\char`\_}html} -
       These qualifiers invoke a special mode of
       \texttt{show statement} that
       creates a web page for the statement.  They may not be used with
       any other qualifier.  See Section~\ref{htmlout} or
       \texttt{help html} in the program.


\subsection{\texttt{search} Command}\index{\texttt{search} command}
Syntax:  search {\em label-match}
\texttt{"}{\em symbol-match}{\tt}" [\texttt{/all}] [\texttt{/comments}]
[\texttt{/join}]

This command searches all \texttt{\$a} and \texttt{\$p} statements
matching {\em label-match} for occurrences of {\em symbol-match}.  A
\verb@*@ in {\em label-match} matches any label character.  A \verb@$*@
in {\em symbol-match} matches any sequence of symbols.  The symbols in
{\em symbol-match} must be separated by white space.  The quotes
surrounding {\em symbol-match} may be single or double quotes.  For
example, \texttt{search b}\verb@* "-> $* ch"@ will list all statements
whose labels begin with \texttt{b} and contain the symbols \verb@->@ and
\texttt{ch} surrounding any symbol sequence (including no symbol
sequence).  The wildcards \texttt{?} and \texttt{\$?} are also available
to match individual characters in labels and symbols respectively; see
\texttt{help search} in the Metamath program for details on their usage.

Optional command qualifiers:

    \texttt{/all} - Also search \texttt{\$e} and \texttt{\$f} statements.

    \texttt{/comments} - Search the comment that immediately precedes each
        label-matched statement for {\em symbol-match}.  In this case
        {\em symbol-match} is an arbitrary, non-case-sensitive character
        string.  Quotes around {\em symbol-match} are optional if there
        is no ambiguity.

    \texttt{/join} - In the case of a \texttt{\$a} or \texttt{\$p} statement,
	prepend its \texttt{\$e}
	hypotheses for searching. The
	\texttt{/join} qualifier has no effect in \texttt{/comments} mode.

\section{Displaying and Verifying Proofs}


\subsection{\texttt{show proof} Command}\index{\texttt{show proof} command}
Syntax:  \texttt{show proof} {\em label-match} [{\em qualifiers} (see below)]

This command displays the proof of the specified
\texttt{\$p}\index{\texttt{\$p} statement} statement in various formats.
The {\em label-match} may contain wildcard (\verb@$*@) characters to match
multiple statements.  Without any qualifiers, only the logical steps
will be shown (i.e.\ syntax construction steps will be omitted), in an
indented format.

Most of the time, you will use
    \texttt{show proof} {\em label}
to see just the proof steps corresponding to logical inferences.

Optional command qualifiers:

    \texttt{/essential} - The proof tree is trimmed of all
        \texttt{\$f}\index{\texttt{\$f} statement} hypotheses before
        being displayed.  (This is the default, and it is redundant to
        specify it.)

    \texttt{/all} - the proof tree is not trimmed of all \texttt{\$f} hypotheses before
        being displayed.  \texttt{/essential} and \texttt{/all} are mutually exclusive.

    \texttt{/from{\char`\_}step} {\em step} - The display starts at the specified
        step.  If
        this qualifier is omitted, the display starts at the first step.

    \texttt{/to{\char`\_}step} {\em step} - The display ends at the specified
        step.  If this
        qualifier is omitted, the display ends at the last step.

    \texttt{/tree{\char`\_}depth} {\em number} - Only
         steps at less than the specified proof
        tree depth are displayed.  Sometimes useful for obtaining an overview of
        the proof.

    \texttt{/reverse} - The steps are displayed in reverse order.

    \texttt{/renumber} - When used with \texttt{/essential}, the steps are renumbered
        to correspond only to the essential steps.

    \texttt{/tex} - The proof is converted to \LaTeX\ and\index{latex@{\LaTeX}}
        stored in the file opened
        with \texttt{open tex}.  See Section~\ref{texout} or
        \texttt{help tex} in the program.

    \texttt{/lemmon} - The proof is displayed in a non-indented format known
        as Lemmon style, with explicit previous step number references.
        If this qualifier is omitted, steps are indented in a tree format.

    \texttt{/start{\char`\_}column} {\em number} - Overrides the default column
        (16)
        at which the formula display starts in a Lemmon-style display.  May be
        used only in conjunction with \texttt{/lemmon}.

    \texttt{/normal} - The proof is displayed in normal format suitable for
        inclusion in a Metamath source file.  May not be used with any other
        qualifier.

    \texttt{/compressed} - The proof is displayed in compressed format
        suitable for inclusion in a Metamath source file.  May not be used with
        any other qualifier.

    \texttt{/statement{\char`\_}summary} - Summarizes all statements (like a
        brief \texttt{show statement})
        used by the proof.  It may not be used with any other qualifier
        except \texttt{/essential}.

    \texttt{/detailed{\char`\_}step} {\em step} - Shows the details of what is
        happening at
        a specific proof step.  May not be used with any other qualifier.
        The {\em step} is the step number shown when displaying a
        proof without the \texttt{/renumber} qualifier.


\subsection{\texttt{show usage} Command}\index{\texttt{show usage} command}
Syntax:  \texttt{show usage} {\em label-match} [\texttt{/recursive}]

This command lists the statements whose proofs make direct reference to
the statement specified.

Optional command qualifier:

    \texttt{/recursive} - Also include statements whose proofs ultimately
        depend on the statement specified.



\subsection{\texttt{show trace\_back} Command}\index{\texttt{show
       trace{\char`\_}back} command}
Syntax:  \texttt{show trace{\char`\_}back} {\em label-match} [\texttt{/essential}] [\texttt{/axioms}]
    [\texttt{/tree}] [\texttt{/depth} {\em number}]

This command lists all statements that the proof of the \texttt{\$p}
statement(s) specified by {\em label-match} depends on.

Optional command qualifiers:

    \texttt{/essential} - Restrict the trace-back to \texttt{\$e}
        \index{\texttt{\$e} statement} hypotheses of proof trees.

    \texttt{/axioms} - List only the axioms that the proof ultimately depends on.

    \texttt{/tree} - Display the trace-back in an indented tree format.

    \texttt{/depth} {\em number} - Restrict the \texttt{/tree} trace-back to the
        specified indentation depth.

    \texttt{/count{\char`\_}steps} - Count the number of steps the proof has
       all the way back to axioms.  If \texttt{/essential} is specified,
       expansions of variable-type hypotheses (syntax constructions) are not counted.

\subsection{\texttt{verify proof} Command}\index{\texttt{verify proof} command}
Syntax:  \texttt{verify proof} {\em label-match} [\texttt{/syntax{\char`\_}only}]

This command verifies the proofs of the specified statements.  {\em
label-match} may contain wild card characters (\texttt{*}) to verify
more than one proof; for example \verb/*abc*def/ will match all labels
containing \texttt{abc} and ending with \texttt{def}.
The command \texttt{verify proof *} will verify all proofs in the database.

Optional command qualifier:

    \texttt{/syntax{\char`\_}only} - This qualifier will perform a check of syntax
        and RPN
        stack violations only.  It will not verify that the proof is
        correct.  This qualifier is useful for quickly determining which
        proofs are incomplete (i.e.\ are under development and have \texttt{?}'s
        in them).

{\em Note:} \texttt{read}, followed by \texttt{verify proof *}, will ensure
 the database is free
from errors in the Metamath language but will not check the markup notation
in comments.
You can also check the markup notation by running \texttt{verify markup *}
as discussed in Section~\ref{verifymarkup}; see also the discussion
on generating {\sc HTML} in Section~\ref{htmlout}.

\subsection{\texttt{verify markup} Command}\index{\texttt{verify markup} command}\label{verifymarkup}
Syntax:  \texttt{verify markup} {\em label-match}
[\texttt{/date{\char`\_}skip}]
[\texttt{/top{\char`\_}date{\char`\_}skip}] {\\}
[\texttt{/file{\char`\_}skip}]
[\texttt{/verbose}]

This command checks comment markup and other informal conventions we have
adopted.  It error-checks the latexdef, htmldef, and althtmldef statements
in the \texttt{\$t} statement of a Metamath source file.\index{error checking}
It error-checks any \texttt{`}...\texttt{`},
\texttt{\char`\~}~\textit{label},
and bibliographic markups in statement descriptions.
It checks that
\texttt{\$p} and \texttt{\$a} statements
have the same content when their labels start with
``ax'' and ``ax-'' respectively but are otherwise identical, for example
ax4 and ax-4.
It also verifies the date consistency of ``(Contributed by...),''
``(Revised by...),'' and ``(Proof shortened by...)'' tags in the comment
above each \texttt{\$a} and \texttt{\$p} statement.

Optional command qualifiers:

    \texttt{/date{\char`\_}skip} - This qualifier will
        skip date consistency checking,
        which is usually not required for databases other than
	\texttt{set.mm}.

    \texttt{/top{\char`\_}date{\char`\_}skip} - This qualifier will check date consistency except
        that the version date at the top of the database file will not
        be checked.  Only one of
        \texttt{/date{\char`\_}skip} and
        \texttt{/top{\char`\_}date{\char`\_}skip} may be
        specified.

    \texttt{/file{\char`\_}skip} - This qualifier will skip checks that require
        external files to be present, such as checking GIF existence and
        bibliographic links to mmset.html or equivalent.  It is useful
        for doing a quick check from a directory without these files.

    \texttt{/verbose} - Provides more information.  Currently it provides a list
        of axXXX vs. ax-XXX matches.

\subsection{\texttt{save proof} Command}\index{\texttt{save proof} command}
Syntax:  \texttt{save proof} {\em label-match} [\texttt{/normal}]
   [\texttt{/compressed}]

The \texttt{save proof} command will reformat a proof in one of two formats and
replace the existing proof in the source buffer\index{source
buffer}.  It is useful for
converting between proof formats.  Note that a proof will not be
permanently saved until a \texttt{write source} command is issued.

Optional command qualifiers:

    \texttt{/normal} - The proof is saved in the normal format (i.e., as a
        sequence
        of labels, which is the defined format of the basic Metamath
        language).\index{basic language}  This is the default format that
        is used if a qualifier
        is omitted.

    \texttt{/compressed} - The proof is saved in the compressed format which
        reduces storage requirements for a database.
        See Appendix~\ref{compressed}.




\section{Creating Proofs}\label{pfcommands}\index{Proof Assistant}

Before using the Proof Assistant, you must add a \texttt{\$p} to your
source file (using a text editor) containing the statement you want to
prove.  Its proof should consist of a single \texttt{?}, meaning
``unknown step.''  Example:
\begin{verbatim}
equid $p x = x $= ? $.
\end{verbatim}

To enter the Proof assistant, type \texttt{prove} {\em label}, e.g.
\texttt{prove equid}.  Metamath will respond with the \texttt{MM-PA>}
prompt.

Proofs are created working backwards from the statement being proved,
primarily using a series of \texttt{assign} commands.  A proof is
complete when all steps are assigned to statements and all steps are
unified and completely known.  During the creation of a proof, Metamath
will allow only operations that are legal based on what is known up to
that point.  For example, it will not allow an \texttt{assign} of a
statement that cannot be unified with the unknown proof step being
assigned.

{\em Important:}
The Proof Assistant is
{\em not} a tool to help you discover proofs.  It is just a tool to help
you add them to the database.  For a tutorial read
Section~\ref{frstprf}.
To practice using the Proof Assistant, you may
want to \texttt{prove} an existing theorem, then delete all steps with
\texttt{delete all}, then re-create it with the Proof Assistant while
looking at its proof display (before deletion).
You might want to figure out your first few proofs completely
and write them down by hand, before using the Proof Assistant, though
not everyone finds that effective.

{\em Important:}
The \texttt{undo} command if very helpful when entering a proof, because
it allows you to undo a previously-entered step.
In addition, we suggest that you
keep track of your work with a log file (\texttt{open
log}) and save it frequently (\texttt{save new{\char`\_}proof},
\texttt{write source}).
You can use \texttt{delete} to reverse an \texttt{assign}.
You can also do \texttt{delete floating{\char`\_}hypotheses}, then
\texttt{initialize all}, then \texttt{unify all /interactive} to
reinitialize bad unifications made accidentally or by bad
\texttt{assign}s.  You cannot reverse a \texttt{delete} except by
a relevant \texttt{undo} or using
\texttt{exit /force} then reentering the Proof Assistant to recover from
the last \texttt{save new{\char`\_}proof}.

The following commands are available in the Proof Assistant (at the
\texttt{MM-PA>} prompt) to help you create your proof.  See the
individual commands for more detail.

\begin{itemize}
\item[]
    \texttt{show new{\char`\_}proof} [\texttt{/all},...] - Displays the
        proof in progress.  You will use this command a lot; see \texttt{help
        show new{\char`\_}proof} to become familiar with its qualifiers.  The
        qualifiers \texttt{/unknown} and \texttt{/not{\char`\_}unified} are
        useful for seeing the work remaining to be done.  The combination
        \texttt{/all/unknown} is useful for identifying dummy variables that must be
        assigned, or attempts to use illegal syntax, when \texttt{improve all}
        is unable to complete the syntax constructions.  Unknown variables are
        shown as \texttt{\$1}, \texttt{\$2},...
\item[]
    \texttt{assign} {\em step} {\em label} - Assigns an unknown {\em step}
        number with the statement
        specified by {\em label}.
\item[]
    \texttt{let variable} {\em variable}
        \texttt{= "}{\em symbol sequence}\texttt{"}
          - Forces a symbol
        sequence to replace an unknown variable (such as \texttt{\$1}) in a proof.
        It is useful
        for helping difficult unifications, and it is necessary when you have
        dummy variables that eventually must be assigned a name.
\item[]
    \texttt{let step} {\em step} \texttt{= "}{\em symbol sequence}\texttt{"} -
          Forces a symbol sequence
        to replace the contents of a proof step, provided it can be
        unified with the existing step contents.  (I rarely use this.)
\item[]
    \texttt{unify step} {\em step} (or \texttt{unify all}) - Unifies
        the source and target of
        a step.  If you specify a specific step, you will be prompted
        to select among the unifications that are possible.  If you
        specify \texttt{all}, all steps with unique unifications, but only
        those steps, will be
        unified.  \texttt{unify all /interactive} goes through all non-unified
        steps.
\item[]
    \texttt{initialize} {\em step} (or \texttt{all}) - De-unifies the target and source of
        a step (or all steps), as well as the hypotheses of the source,
        and makes all variables in the source unknown.  Useful to recover from
        an \texttt{assign} or \texttt{let} mistake that
        resulted in incorrect unifications.
\item[]
    \texttt{delete} {\em step} (or \texttt{all} or \texttt{floating{\char`\_}hypotheses}) -
        Deletes the specified
        step(s).  \texttt{delete floating{\char`\_}hypotheses}, then \texttt{initialize all}, then
        \texttt{unify all /interactive} is useful for recovering from mistakes
        where incorrect unifications assigned wrong math symbol strings to
        variables.
\item[]
    \texttt{improve} {\em step} (or \texttt{all}) -
      Automatically creates a proof for steps (with no unknown variables)
      whose proof requires no statements with \texttt{\$e} hypotheses.  Useful
      for filling in proofs of \texttt{\$f} hypotheses.  The \texttt{/depth}
      qualifier will also try statements whose \texttt{\$e} hypotheses contain
      no new variables.  {\em Warning:} Save your work (with \texttt{save
      new{\char`\_}proof} then \texttt{write source}) before using
      \texttt{/depth = 2} or greater, since the search time grows
      exponentially and may never terminate in a reasonable time, and you
      cannot interrupt the search.  I have found that it is rare for
      \texttt{/depth = 3} or greater to be useful.
 \item[]
    \texttt{save new{\char`\_}proof} - Saves the proof in progress in the program's
        internal database buffer.  To save it permanently into the database file,
        use \texttt{write source} after
        \texttt{save new{\char`\_}proof}.  To revert to the last
        \texttt{save new{\char`\_}proof},
        \texttt{exit /force} from the Proof Assistant then re-enter the Proof
        Assistant.
 \item[]
    \texttt{match step} {\em step} (or \texttt{match all}) - Shows what
        statements are
        possibilities for the \texttt{assign} statement. (This command
        is not very
        useful in its present form and hopefully will be improved
        eventually.  In the meantime, use the \texttt{search} statement for
        candidates matching specific math token combinations.)
 \item[]
 \texttt{minimize{\char`\_}with}\index{\texttt{minimize{\char`\_}with} command}
% 3/10/07 Note: line-breaking the above results in duplicate index entries
     - After a proof is complete, this command will attempt
        to match other database theorems to the proof to see if the proof
        size can be reduced as a result.  See \texttt{help
        minimize{\char`\_}with} in the
        Metamath program for its usage.
 \item[]
 \texttt{undo}\index{\texttt{undo} command}
    - Undo the effect of a proof-changing command (all but the
      \texttt{show} and \texttt{save} commands above).
 \item[]
 \texttt{redo}\index{\texttt{redo} command}
    - Reverse the previous \texttt{undo}.
\end{itemize}

The following commands set parameters that may be relevant to your proof.
Consult the individual \texttt{help set}... commands.
\begin{itemize}
   \item[] \texttt{set unification{\char`\_}timeout}
 \item[]
    \texttt{set search{\char`\_}limit}
  \item[]
    \texttt{set empty{\char`\_}substitution} - note that default is \texttt{off}
\end{itemize}

Type \texttt{exit} to exit the \texttt{MM-PA>}
 prompt and get back to the \texttt{MM>} prompt.
Another \texttt{exit} will then get you out of Metamath.



\subsection{\texttt{prove} Command}\index{\texttt{prove} command}
Syntax:  \texttt{prove} {\em label}

This command will enter the Proof Assistant\index{Proof Assistant}, which will
allow you to create or edit the proof of the specified statement.
The command-line prompt will change from \texttt{MM>} to \texttt{MM-PA>}.

Note:  In the present version (0.177) of
Metamath\index{Metamath!limitations of version 0.177}, the Proof
Assistant does not verify that \texttt{\$d}\index{\texttt{\$d}
statement} restrictions are met as a proof is being built.  After you
have completed a proof, you should type \texttt{save new{\char`\_}proof}
followed by \texttt{verify proof} {\em label} (where {\em label} is the
statement you are proving with the \texttt{prove} command) to verify the
\texttt{\$d} restrictions.

See also: \texttt{exit}

\subsection{\texttt{set unification\_timeout} Command}\index{\texttt{set
unification{\char`\_}timeout} command}
Syntax:  \texttt{set unification{\char`\_}timeout} {\em number}

(This command is available outside the Proof Assistant but affects the
Proof Assistant\index{Proof Assistant} only.)

Sometimes the Proof Assistant will inform you that a unification
time-out occurred.  This may happen when you try to \texttt{unify}
formulas with many temporary variables\index{temporary variable}
(\texttt{\$1}, \texttt{\$2}, etc.), since the time to compute all possible
unifications may grow exponentially with the number of variables.  If
you want Metamath to try harder (and you're willing to wait longer) you
may increase this parameter.  \texttt{show settings} will show you the
current value.

\subsection{\texttt{set empty\_substitution} Command}\index{\texttt{set
empty{\char`\_}substitution} command}
% These long names can't break well in narrow mode, and even "sloppy"
% is not enough. Work around this by not demanding justification.
\begin{flushleft}
Syntax:  \texttt{set empty{\char`\_}substitution on} or \texttt{set
empty{\char`\_}substitution off}
\end{flushleft}

(This command is available outside the Proof Assistant but affects the
Proof Assistant\index{Proof Assistant} only.)

The Metamath language allows variables to be
substituted\index{substitution!variable}\index{variable substitution}
with empty symbol sequences\index{empty substitution}.  However, in many
formal systems\index{formal system} this will never happen in a valid
proof.  Allowing for this possibility increases the likelihood of
ambiguous unifications\index{ambiguous
unification}\index{unification!ambiguous} during proof creation.
The default is that
empty substitutions are not allowed; for formal systems requiring them,
you must \texttt{set empty{\char`\_}substitution on}.
(An example where this must be \texttt{on}
would be a system that implements a Deduction Rule and in
which deductions from empty assumption lists would be permissible.  The
MIU-system\index{MIU-system} described in Appendix~\ref{MIU} is another
example.)
Note that empty substitutions are
always permissible in proof verification (VERIFY PROOF...) outside the
Proof Assistant.  (See the MIU system in the Metamath book for an example
of a system needing empty substitutions; another example would be a
system that implements a Deduction Rule and in which deductions from
empty assumption lists would be permissible.)

It is better to leave this \texttt{off} when working with \texttt{set.mm}.
Note that this command does not affect the way proofs are verified with
the \texttt{verify proof} command.  Outside of the Proof Assistant,
substitution of empty sequences for math symbols is always allowed.

\subsection{\texttt{set search\_limit} Command}\index{\texttt{set
search{\char`\_}limit} command} Syntax:  \texttt{set search{\char`\_}limit} {\em
number}

(This command is available outside the Proof Assistant but affects the
Proof Assistant\index{Proof Assistant} only.)

This command sets a parameter that determines when the \texttt{improve} command
in Proof Assistant mode gives up.  If you want \texttt{improve} to search harder,
you may increase it.  The \texttt{show settings} command tells you its current
value.


\subsection{\texttt{show new\_proof} Command}\index{\texttt{show
new{\char`\_}proof} command}
Syntax:  \texttt{show new{\char`\_}proof} [{\em
qualifiers} (see below)]

This command (available only in Proof Assistant mode) displays the proof
in progress.  It is identical to the \texttt{show proof} command, except that
there is no statement argument (since it is the statement being proved) and
the following qualifiers are not available:

    \texttt{/statement{\char`\_}summary}

    \texttt{/detailed{\char`\_}step}

Also, the following additional qualifiers are available:

    \texttt{/unknown} - Shows only steps that have no statement assigned.

    \texttt{/not{\char`\_}unified} - Shows only steps that have not been unified.

Note that \texttt{/essential}, \texttt{/depth}, \texttt{/unknown}, and
\texttt{/not{\char`\_}unified} may be
used in any combination; each of them effectively filters out additional
steps from the proof display.

See also:  \texttt{show proof}






\subsection{\texttt{assign} Command}\index{\texttt{assign} command}
Syntax:   \texttt{assign} {\em step} {\em label} [\texttt{/no{\char`\_}unify}]

   and:   \texttt{assign first} {\em label}

   and:   \texttt{assign last} {\em label}


This command, available in the Proof Assistant only, assigns an unknown
step (one with \texttt{?} in the \texttt{show new{\char`\_}proof}
listing) with the statement specified by {\em label}.  The assignment
will not be allowed if the statement cannot be unified with the step.

If \texttt{last} is specified instead of {\em step} number, the last
step that is shown by \texttt{show new{\char`\_}proof /unknown} will be
used.  This can be useful for building a proof with a command file (see
\texttt{help submit}).  It also makes building proofs faster when you know
the assignment for the last step.

If \texttt{first} is specified instead of {\em step} number, the first
step that is shown by \texttt{show new{\char`\_}proof /unknown} will be
used.

If {\em step} is zero or negative, the -{\em step}th from last unknown
step, as shown by \texttt{show new{\char`\_}proof /unknown}, will be
used.  \texttt{assign -1} {\em label} will assign the penultimate
unknown step, \texttt{assign -2} {\em label} the antepenultimate, and
\texttt{assign 0} {\em label} is the same as \texttt{assign last} {\em
label}.

Optional command qualifier:

    \texttt{/no{\char`\_}unify} - do not prompt user to select a unification if there is
        more than one possibility.  This is useful for noninteractive
        command files.  Later, the user can \texttt{unify all /interactive}.
        (The assignment will still be automatically unified if there is only
        one possibility and will be refused if unification is not possible.)



\subsection{\texttt{match} Command}\index{\texttt{match} command}
Syntax:  \texttt{match step} {\em step} [\texttt{/max{\char`\_}essential{\char`\_}hyp}
{\em number}]

    and:  \texttt{match all} [\texttt{/essential}]
          [\texttt{/max{\char`\_}essential{\char`\_}hyp} {\em number}]

This command, available in the Proof Assistant only, shows what
statements can be unified with the specified step(s).  {\em Note:} In
its current form, this command is not very useful because of the large
number of matches it reports.
It may be enhanced in the future.  In the meantime, the \texttt{search}
command can often provide finer control over locating theorems of interest.

Optional command qualifiers:

    \texttt{/max{\char`\_}essential{\char`\_}hyp} {\em number} - filters out
        of the list any statements
        with more than the specified number of
        \texttt{\$e}\index{\texttt{\$e} statement} hypotheses.

    \texttt{/essential{\char`\_}only} - in the \texttt{match all} statement, only
        the steps that
        would be listed in the \texttt{show new{\char`\_}proof /essential} display are
        matched.



\subsection{\texttt{let} Command}\index{\texttt{let} command}
Syntax: \texttt{let variable} {\em variable} = \verb/"/{\em symbol-sequence}\verb/"/

  and: \texttt{let step} {\em step} = \verb/"/{\em symbol-sequence}\verb/"/

These commands, available in the Proof Assistant\index{Proof Assistant}
only, assign a temporary variable\index{temporary variable} or unknown
step with a specific symbol sequence.  They are useful in the middle of
creating a proof, when you know what should be in the proof step but the
unification algorithm doesn't yet have enough information to completely
specify the temporary variables.  A ``temporary variable'' is one that
has the form \texttt{\$}{\em nn} in the proof display, such as
\texttt{\$1}, \texttt{\$2}, etc.  The {\em symbol-sequence} may contain
other unknown variables if desired.  Examples:

    \verb/let variable $32 = "A = B"/

    \verb/let variable $32 = "A = $35"/

    \verb/let step 10 = '|- x = x'/

    \verb/let step -2 = "|- ( $7 = ph )"/

Any symbol sequence will be accepted for the \texttt{let variable}
command.  Only those symbol sequences that can be unified with the step
will be accepted for \texttt{let step}.

The \texttt{let} commands ``zap'' the proof with information that can
only be verified when the proof is built up further.  If you make an
error, the command sequence \texttt{delete
floating{\char`\_}hypotheses}, \texttt{initialize all}, and
\texttt{unify all /interactive} will undo a bad \texttt{let} assignment.

If {\em step} is zero or negative, the -{\em step}th from last unknown
step, as shown by \texttt{show new{\char`\_}proof /unknown}, will be
used.  The command \texttt{let step 0} = \verb/"/{\em
symbol-sequence}\verb/"/ will use the last unknown step, \texttt{let
step -1} = \verb/"/{\em symbol-sequence}\verb/"/ the penultimate, etc.
If {\em step} is positive, \texttt{let step} may be used to assign known
(in the sense of having previously been assigned a label with
\texttt{assign}) as well as unknown steps.

Either single or double quotes can surround the {\em symbol-sequence} as
long as they are different from any quotes inside a {\em
symbol-sequence}.  If {\em symbol-sequence} contains both kinds of
quotes, see the instructions at the end of \texttt{help let} in the
Metamath program.


\subsection{\texttt{unify} Command}\index{\texttt{unify} command}
Syntax:  \texttt{unify step} {\em step}

      and:   \texttt{unify all} [\texttt{/interactive}]

These commands, available in the Proof Assistant only, unify the source
and target of the specified step(s). If you specify a specific step, you
will be prompted to select among the unifications that are possible.  If
you specify \texttt{all}, only those steps with unique unifications will be
unified.

Optional command qualifier for \texttt{unify all}:

    \texttt{/interactive} - You will be prompted to select among the
        unifications
        that are possible for any steps that do not have unique
        unifications.  (Otherwise \texttt{unify all} will bypass these.)

See also \texttt{set unification{\char`\_}timeout}.  The default is
100000, but increasing it to 1000000 can help difficult cases.  Manually
assigning some or all of the unknown variables with the \texttt{let
variable} command also helps difficult cases.



\subsection{\texttt{initialize} Command}\index{\texttt{initialize} command}
Syntax:  \texttt{initialize step} {\em step}

    and: \texttt{initialize all}

These commands, available in the Proof Assistant\index{Proof Assistant}
only, ``de-unify'' the target and source of a step (or all steps), as
well as the hypotheses of the source, and makes all variables in the
source and the source's hypotheses unknown.  This command is useful to
help recover from incorrect unifications that resulted from an incorrect
\texttt{assign}, \texttt{let}, or unification choice.  Part or all of
the command sequence \texttt{delete floating{\char`\_}hypotheses},
\texttt{initialize all}, and \texttt{unify all /interactive} will recover
from incorrect unifications.

See also:  \texttt{unify} and \texttt{delete}



\subsection{\texttt{delete} Command}\index{\texttt{delete} command}
Syntax:  \texttt{delete step} {\em step}

   and:      \texttt{delete all} -- {\em Warning: dangerous!}

   and:      \texttt{delete floating{\char`\_}hypotheses}

These commands are available in the Proof Assistant only.  The
\texttt{delete step} command deletes the proof tree section that
branches off of the specified step and makes the step become unknown.
\texttt{delete all} is equivalent to \texttt{delete step} {\em step}
where {\em step} is the last step in the proof (i.e.\ the beginning of
the proof tree).

In most cases the \texttt{undo} command is the best way to undo
a previous step.
An alternative is to salvage your last \texttt{save
new{\char`\_}proof} by exiting and reentering the Proof Assistant.
For this to work, keep a log file open to record your work
and to do \texttt{save new{\char`\_}proof} frequently, especially before
\texttt{delete}.

\texttt{delete floating{\char`\_}hypotheses} will delete all sections of
the proof that branch off of \texttt{\$f}\index{\texttt{\$f} statement}
statements.  It is sometimes useful to do this before an
\texttt{initialize} command to recover from an error.  Note that once a
proof step with a \texttt{\$f} hypothesis as the target is completely
known, the \texttt{improve} command can usually fill in the proof for
that step.  Unlike the deletion of logical steps, \texttt{delete
floating{\char`\_}hypotheses} is a relatively safe command that is
usually easy to recover from.



\subsection{\texttt{improve} Command}\index{\texttt{improve} command}
\label{improve}
Syntax:  \texttt{improve} {\em step} [\texttt{/depth} {\em number}]
                                               [\texttt{/no{\char`\_}distinct}]

   and:   \texttt{improve first} [\texttt{/depth} {\em number}]
                                              [\texttt{/no{\char`\_}distinct}]

   and:   \texttt{improve last} [\texttt{/depth} {\em number}]
                                              [\texttt{/no{\char`\_}distinct}]

   and:   \texttt{improve all} [\texttt{/depth} {\em number}]
                                              [\texttt{/no{\char`\_}distinct}]

These commands, available in the Proof Assistant\index{Proof Assistant}
only, try to find proofs automatically for unknown steps whose symbol
sequences are completely known.  They are primarily useful for filling in
proofs of \texttt{\$f}\index{\texttt{\$f} statement} hypotheses.  The
search will be restricted to statements having no
\texttt{\$e}\index{\texttt{\$e} statement} hypotheses.

\begin{sloppypar} % narrow
Note:  If memory is limited, \texttt{improve all} on a large proof may
overflow memory.  If you use \texttt{set unification{\char`\_}timeout 1}
before \texttt{improve all}, there will usually be sufficient
improvement to easily recover and completely \texttt{improve} the proof
later on a larger computer.  Warning:  Once memory has overflowed, there
is no recovery.  If in doubt, save the intermediate proof (\texttt{save
new{\char`\_}proof} then \texttt{write source}) before \texttt{improve
all}.
\end{sloppypar}

If \texttt{last} is specified instead of {\em step} number, the last
step that is shown by \texttt{show new{\char`\_}proof /unknown} will be
used.

If \texttt{first} is specified instead of {\em step} number, the first
step that is shown by \texttt{show new{\char`\_}proof /unknown} will be
used.

If {\em step} is zero or negative, the -{\em step}th from last unknown
step, as shown by \texttt{show new{\char`\_}proof /unknown}, will be
used.  \texttt{improve -1} will use the penultimate
unknown step, \texttt{improve -2} {\em label} the antepenultimate, and
\texttt{improve 0} is the same as \texttt{improve last}.

Optional command qualifier:

    \texttt{/depth} {\em number} - This qualifier will cause the search
        to include
        statements with \texttt{\$e} hypotheses (but no new variables in
        the \texttt{\$e}
        hypotheses), provided that the backtracking has not exceeded the
        specified depth. {\em Warning:}  Try \texttt{/depth 1},
        then \texttt{2}, then \texttt{3}, etc.
        in sequence because of possible exponential blowups.  Save your
        work before trying \texttt{/depth} greater than \texttt{1}!

    \texttt{/no{\char`\_}distinct} - Skip trial statements that have
        \texttt{\$d}\index{\texttt{\$d} statement} requirements.
        This qualifier will prevent assignments that might violate \texttt{\$d}
        requirements but it also could miss possible legal assignments.

See also: \texttt{set search{\char`\_}limit}

\subsection{\texttt{save new\_proof} Command}\index{\texttt{save
new{\char`\_}proof} command}
Syntax:  \texttt{save new{\char`\_}proof} {\em label} [\texttt{/normal}]
   [\texttt{/compressed}]

The \texttt{save new{\char`\_}proof} command is available in the Proof
Assistant only.  It saves the proof in progress in the source
buffer\index{source buffer}.  \texttt{save new{\char`\_}proof} may be
used to save a completed proof, or it may be used to save a proof in
progress in order to work on it later.  If an incomplete proof is saved,
any user assignments with \texttt{let step} or \texttt{let variable}
will be lost, as will any ambiguous unifications\index{ambiguous
unification}\index{unification!ambiguous} that were resolved manually.
To help make recovery easier, it can be helpful to \texttt{improve all}
before \texttt{save new{\char`\_}proof} so that the incomplete proof
will have as much information as possible.

Note that the proof will not be permanently saved until a \texttt{write
source} command is issued.

Optional command qualifiers:

    \texttt{/normal} - The proof is saved in the normal format (i.e., as a
        sequence of labels, which is the defined format of the basic Metamath
        language).\index{basic language}  This is the default format that
        is used if a qualifier is omitted.

    \texttt{/compressed} - The proof is saved in the compressed format, which
        reduces storage requirements for a database.
        (See Appendix~\ref{compressed}.)


\section{Creating \LaTeX\ Output}\label{texout}\index{latex@{\LaTeX}}

You can generate \LaTeX\ output given the
information in a database.
The database must already include the necessary typesetting information
(see section \ref{tcomment} for how to provide this information).

The \texttt{show statement} and \texttt{show proof} commands each have a
special \texttt{/tex} command qualifier that produces \LaTeX\ output.
(The \texttt{show statement} command also has the
\texttt{/simple{\char`\_}tex} qualifier for output that is easier to
edit by hand.)  Before you can use them, you must open a \LaTeX\ file to
which to send their output.  A typical complete session will use this
sequence of Metamath commands:

\begin{verbatim}
read set.mm
open tex example.tex
show statement a1i /tex
show proof a1i /all/lemmon/renumber/tex
show statement uneq2 /tex
show proof uneq2 /all/lemmon/renumber/tex
close tex
\end{verbatim}

See Section~\ref{mathcomments} for information on comment markup and
Appendix~\ref{ASCII} for information on how math symbol translation is
specified.

To format and print the \LaTeX\ source, you will need the \LaTeX\
program, which is standard on most Linux installations and available for
Windows.  On Linux, in order to create a {\sc pdf} file, you will
typically type at the shell prompt
\begin{verbatim}
$ pdflatex example.tex
\end{verbatim}

\subsection{\texttt{open tex} Command}\index{\texttt{open tex} command}
Syntax:  \texttt{open tex} {\em file-name} [\texttt{/no{\char`\_}header}]

This command opens a file for writing \LaTeX\
source\index{latex@{\LaTeX}} and writes a \LaTeX\ header to the file.
\LaTeX\ source can be written with the \texttt{show proof}, \texttt{show
new{\char`\_}proof}, and \texttt{show statement} commands using the
\texttt{/tex} qualifier.

The mapping to \LaTeX\ symbols is defined in a special comment
containing a \texttt{\$t} token, described in Appendix~\ref{ASCII}.

There is an optional command qualifier:

    \texttt{/no{\char`\_}header} - This qualifier prevents a standard
        \LaTeX\ header and trailer
        from being included with the output \LaTeX\ code.


\subsection{\texttt{close tex} Command}\index{\texttt{close tex} command}
Syntax:  \texttt{close tex}

This command writes a trailer to any \LaTeX\ file\index{latex@{\LaTeX}}
that was opened with \texttt{open tex} (unless
\texttt{/no{\char`\_}header} was used with \texttt{open tex}) and closes
the \LaTeX\ file.


\section{Creating {\sc HTML} Output}\label{htmlout}

You can generate {\sc html} web pages given the
information in a database.
The database must already include the necessary typesetting information
(see section \ref{tcomment} for how to provide this information).
The ability to produce {\sc html} web pages was added in Metamath version
0.07.30.

To create an {\sc html} output file(s) for \texttt{\$a} or \texttt{\$p}
statement(s), use
\begin{quote}
    \texttt{show statement} {\em label-match} \texttt{/html}
\end{quote}
The output file will be named {\em label-match}\texttt{.html}
for each match.  When {\em
label-match} has wildcard (\texttt{*}) characters, all statements with
matching labels will have {\sc html} files produced for them.  Also,
when {\em label-match} has a wildcard (\texttt{*}) character, two additional
files, \texttt{mmdefinitions.html} and \texttt{mmascii.html} will be
produced.  To produce {\em only} these two additional files, you can use
\texttt{?*}, which will not match any statement label, in place of {\em
label-match}.

There are three other qualifiers for \texttt{show statement} that also
generate {\sc HTML} code.  These are \texttt{/alt{\char`\_}html},
\texttt{/brief{\char`\_}html}, and
\texttt{/brief{\char`\_}alt{\char`\_}html}, and are described in the
next section.

The command
\begin{quote}
    \texttt{show statement} {\em label-match} \texttt{/alt{\char`\_}html}
\end{quote}
does the same as \texttt{show statement} {\em label-match} \texttt{/html},
except that the {\sc html} code for the symbols is taken from
\texttt{althtmldef} statements instead of \texttt{htmldef} statements in
the \texttt{\$t} comment.

The command
\begin{verbatim}
show statement * /brief_html
\end{verbatim}
invokes a special mode that just produces definition and theorem lists
accompanied by their symbol strings, in a format suitable for copying and
pasting into another web page (such as the tutorial pages on the
Metamath web site).

Finally, the command
\begin{verbatim}
show statement * /brief_alt_html
\end{verbatim}
does the same as \texttt{show statement * / brief{\char`\_}html}
for the alternate {\sc html}
symbol representation.

A statement's comment can include a special notation that provides a
certain amount of control over the {\sc HTML} version of the comment.  See
Section~\ref{mathcomments} (p.~\pageref{mathcomments}) for the comment
markup features.

The \texttt{write theorem{\char`\_}list} and \texttt{write bibliography}
commands, which are described below, provide as a side effect complete
error checking for all of the features described in this
section.\index{error checking}

\subsection{\texttt{write theorem\_list}
Command}\index{\texttt{write theorem{\char`\_}list} command}
Syntax:  \texttt{write theorem{\char`\_}list}
[\texttt{/theorems{\char`\_}per{\char`\_}page} {\em number}]

This command writes a list of all of the \texttt{\$a} and \texttt{\$p}
statements in the database into a web page file
 called \texttt{mmtheorems.html}.
When additional files are needed, they are called
\texttt{mmtheorems2.html}, \texttt{mmtheorems3.html}, etc.

Optional command qualifier:

    \texttt{/theorems{\char`\_}per{\char`\_}page} {\em number} -
 This qualifier specifies the number of statements to
        write per web page.  The default is 100.

{\em Note:} In version 0.177\index{Metamath!limitations of version
0.177} of Metamath, the ``Nearby theorems'' links on the individual
web pages presuppose 100 theorems per page when linking to the theorem
list pages.  Therefore the \texttt{/theorems{\char`\_}per{\char`\_}page}
qualifier, if it specifies a number other than 100, will cause the
individual web pages to be out of sync and should not be used to
generate the main theorem list for the web site.  This may be
fixed in a future version.


\subsection{\texttt{write bibliography}\label{wrbib}
Command}\index{\texttt{write bibliography} command}
Syntax:  \texttt{write bibliography} {\em filename}

This command reads an existing {\sc html} bibliographic cross-reference
file, normally called \texttt{mmbiblio.html}, and updates it per the
bibliographic links in the database comments.  The file is updated
between the {\sc html} comment lines \texttt{<!--
{\char`\#}START{\char`\#} -->} and \texttt{<!-- {\char`\#}END{\char`\#}
-->}.  The original input file is renamed to {\em
filename}\texttt{{\char`\~}1}.

A bibliographic reference is indicated with the reference name
in brackets, such as  \texttt{Theorem 3.1 of
[Monk] p.\ 22}.
See Section~\ref{htmlout} (p.~\pageref{htmlout}) for
syntax details.


\subsection{\texttt{write recent\_additions}
Command}\index{\texttt{write recent{\char`\_}additions} command}
Syntax:  \texttt{write recent{\char`\_}additions} {\em filename}
[\texttt{/limit} {\em number}]

This command reads an existing ``Recent Additions'' {\sc html} file,
normally called \texttt{mmrecent.html}, and updates it with the
descriptions of the most recently added theorems to the database.
 The file is updated between
the {\sc html} comment lines \texttt{<!-- {\char`\#}START{\char`\#} -->}
and \texttt{<!-- {\char`\#}END{\char`\#} -->}.  The original input file
is renamed to {\em filename}\texttt{{\char`\~}1}.

Optional command qualifier:

    \texttt{/limit} {\em number} -
 This qualifier specifies the number of most recent theorems to
   write to the output file.  The default is 100.


\section{Text File Utilities}

\subsection{\texttt{tools} Command}\index{\texttt{tools} command}
Syntax:  \texttt{tools}

This command invokes an easy-to-use, general purpose utility for
manipulating the contents of {\sc ascii} text files.  Upon typing
\texttt{tools}, the command-line prompt will change to \texttt{TOOLS>}
until you type \texttt{exit}.  The \texttt{tools} commands can be used
to perform simple, global edits on an input/output file,
such as making a character string substitution on each line, adding a
string to each line, and so on.  A typical use of this utility is
to build a \texttt{submit} input file to perform a common operation on a
list of statements obtained from \texttt{show label} or \texttt{show
usage}.

The actions of most of the \texttt{tools} commands can also be
performed with equivalent (and more powerful) Unix shell commands, and
some users may find those more efficient.  But for Windows users or
users not comfortable with Unix, \texttt{tools} provides an
easy-to-learn alternative that is adequate for most of the
script-building tasks needed to use the Metamath program effectively.

\subsection{\texttt{help} Command (in \texttt{tools})}
Syntax:  \texttt{help}

The \texttt{help} command lists the commands available in the
\texttt{tools} utility, along with a brief description.  Each command,
in turn, has its own help, such as \texttt{help add}.  As with
Metamath's \texttt{MM>} prompt, a complete command can be entered at
once, or just the command word can be typed, causing the program to
prompt for each argument.

\vskip 1ex
\noindent Line-by-line editing commands:

  \texttt{add} - Add a specified string to each line in a file.

  \texttt{clean} - Trim spaces and tabs on each line in a file; convert
         characters.

  \texttt{delete} - Delete a section of each line in a file.

  \texttt{insert} - Insert a string at a specified column in each line of
        a file.

  \texttt{substitute} - Make a simple substitution on each line of the file.

  \texttt{tag} - Like \texttt{add}, but restricted to a range of lines.

  \texttt{swap} - Swap the two halves of each line in a file.

\vskip 1ex
\noindent Other file-processing commands:

  \texttt{break} - Break up (tokenize) a file into a list of tokens (one per
        line).

  \texttt{build} - Build a file with multiple tokens per line from a list.

  \texttt{count} - Count the occurrences in a file of a specified string.

  \texttt{number} - Create a list of numbers.

  \texttt{parallel} - Put two files in parallel.

  \texttt{reverse} - Reverse the order of the lines in a file.

  \texttt{right} - Right-justify lines in a file (useful before sorting
         numbers).

%  \texttt{tag} - Tag edit updates in a program for revision control.

  \texttt{sort} - Sort the lines in a file with key starting at
         specified string.

  \texttt{match} - Extract lines containing (or not) a specified string.

  \texttt{unduplicate} - Eliminate duplicate occurrences of lines in a file.

  \texttt{duplicate} - Extract first occurrence of any line occurring
         more than

   \ \ \    once in a file, discarding lines occurring exactly once.

  \texttt{unique} - Extract lines occurring exactly once in a file.

  \texttt{type} (10 lines) - Display the first few lines in a file.
                  Similar to Unix \texttt{head}.

  \texttt{copy} - Similar to Unix \texttt{cat} but safe (same input
         and output file allowed).

  \texttt{submit} - Run a script containing \texttt{tools} commands.

\vskip 1ex

\noindent Note:
  \texttt{unduplicate}, \texttt{duplicate}, and \texttt{unique} also
 sort the lines as a side effect.


\subsection{Using \texttt{tools} to Build Metamath \texttt{submit}
Scripts}

The \texttt{break} command is typically used to break up a series of
statement labels, such as the output of Metamath's \texttt{show usage},
into one label per line.  The other \texttt{tools} commands can then be
used to add strings before and after each statement label to specify
commands to be performed on the statement.  The \texttt{parallel}
command is useful when a statement label must be mentioned more than
once on a line.

Very often a \texttt{submit} script for Metamath will require multiple
command lines for each statement being processed.  For example, you may
want to enter the Proof Assistant, \texttt{minimize{\char`\_}with} your
latest theorem, \texttt{save} the new proof, and \texttt{exit} the Proof
Assistant.  To accomplish this, you can build a file with these four
commands for each statement on a single line, separating each command
with a designated character such as \texttt{@}.  Then at the end you can
\texttt{substitute} each \texttt{@} with \texttt{{\char`\\}n} to break
up the lines into individual command lines (see \texttt{help
substitute}).


\subsection{Example of a \texttt{tools} Session}

To give you a quick feel for the \texttt{tools} utility, we show a
simple session where we create a file \texttt{n.txt} with 3 lines, add
strings before and after each line, and display the lines on the screen.
You can experiment with the various commands to gain experience with the
\texttt{tools} utility.

\begin{verbatim}
MM> tools
Entering the Text Tools utilities.
Type HELP for help, EXIT to exit.
TOOLS> number
Output file <n.tmp>? n.txt
First number <1>?
Last number <10>? 3
Increment <1>?
TOOLS> add
Input/output file? n.txt
String to add to beginning of each line <>? This is line
String to add to end of each line <>? .
The file n.txt has 3 lines; 3 were changed.
First change is on line 1:
This is line 1.
TOOLS> type n.txt
This is line 1.
This is line 2.
This is line 3.
TOOLS> exit
Exiting the Text Tools.
Type EXIT again to exit Metamath.
MM>
\end{verbatim}



\appendix
\chapter{Sample Representations}
\label{ASCII}

This Appendix provides a sample of {\sc ASCII} representations,
their corresponding traditional mathematical symbols,
and a discussion of their meanings
in the \texttt{set.mm} database.
These are provided in order of appearance.
This is only a partial list, and new definitions are routinely added.
A complete list is available at \url{http://metamath.org}.

These {\sc ASCII} representations, along
with information on how to display them,
are defined in the \texttt{set.mm} database file inside
a special comment called a \texttt{\$t} {\em
comment}\index{\texttt{\$t} comment} or {\em typesetting
comment.}\index{typesetting comment}
A typesetting comment
is indicated by the appearance of the
two-character string \texttt{\$t} at the beginning of the comment.
For more information,
see Section~\ref{tcomment}, p.~\pageref{tcomment}.

In the following table the ``{\sc ASCII}'' column shows the {\sc ASCII}
representation,
``Symbol'' shows the mathematical symbolic display
that corresponds to that {\sc ASCII} representation, ``Labels'' shows
the key label(s) that define the representation, and
``Description'' provides a description about the symbol.
As usual, ``iff'' is short for ``if and only if.''\index{iff}
In most cases the ``{\sc ASCII}'' column only shows
the key token, but it will sometimes show a sequence of tokens
if that is necessary for clarity.

{\setlength{\extrarowsep}{4pt} % Keep rows from being too close together
\begin{longtabu}   { @{} c c l X }
\textbf{ASCII} & \textbf{Symbol} & \textbf{Labels} & \textbf{Description} \\
\endhead
\texttt{|-} & $\vdash$ & &
  ``It is provable that...'' \\
\texttt{ph} & $\varphi$ & \texttt{wph} &
  The wff (boolean) variable phi,
  conventionally the first wff variable. \\
\texttt{ps} & $\psi$ & \texttt{wps} &
  The wff (boolean) variable psi,
  conventionally the second wff variable. \\
\texttt{ch} & $\chi$ & \texttt{wch} &
  The wff (boolean) variable chi,
  conventionally the third wff variable. \\
\texttt{-.} & $\lnot$ & \texttt{wn} &
  Logical not. E.g., if $\varphi$ is true, then $\lnot \varphi$ is false. \\
\texttt{->} & $\rightarrow$ & \texttt{wi} &
  Implies, also known as material implication.
  In classical logic the expression $\varphi \rightarrow \psi$ is true
  if either $\varphi$ is false or $\psi$ is true (or both), that is,
  $\varphi \rightarrow \psi$ has the same meaning as
  $\lnot \varphi \lor \psi$ (as proven in theorem \texttt{imor}). \\
\texttt{<->} & $\leftrightarrow$ &
  \hyperref[df-bi]{\texttt{df-bi}} &
  Biconditional (aka is-equals for boolean values).
  $\varphi \leftrightarrow \psi$ is true iff
  $\varphi$ and $\psi$ have the same value. \\
\texttt{\char`\\/} & $\lor$ &
  \makecell[tl]{{\hyperref[df-or]{\texttt{df-or}}}, \\
	         \hyperref[df-3or]{\texttt{df-3or}}} &
  Disjunction (logical ``or''). $\varphi \lor \psi$ is true iff
  $\varphi$, $\psi$, or both are true. \\
\texttt{/\char`\\} & $\land$ &
  \makecell[tl]{{\hyperref[df-an]{\texttt{df-an}}}, \\
                 \hyperref[df-3an]{\texttt{df-3an}}} &
  Conjunction (logical ``and''). $\varphi \land \psi$ is true iff
  both $\varphi$ and $\psi$ are true. \\
\texttt{A.} & $\forall$ &
  \texttt{wal} &
  For all; the wff $\forall x \varphi$ is true iff
  $\varphi$ is true for all values of $x$. \\
\texttt{E.} & $\exists$ &
  \hyperref[df-ex]{\texttt{df-ex}} &
  There exists; the wff
  $\exists x \varphi$ is true iff
  there is at least one $x$ where $\varphi$ is true. \\
\texttt{[ y / x ]} & $[ y / x ]$ &
  \hyperref[df-sb]{\texttt{df-sb}} &
  The wff $[ y / x ] \varphi$ produces
  the result when $y$ is properly substituted for $x$ in $\varphi$
  ($y$ replaces $x$).
  % This is elsb4
  % ( [ x / y ] z e. y <-> z e. x )
  For example,
  $[ x / y ] z \in y$ is the same as $z \in x$. \\
\texttt{E!} & $\exists !$ &
  \hyperref[df-eu]{\texttt{df-eu}} &
  There exists exactly one;
  $\exists ! x \varphi$ is true iff
  there is at least one $x$ where $\varphi$ is true. \\
\texttt{\{ y | phi \}}  & $ \{ y | \varphi \}$ &
  \hyperref[df-clab]{\texttt{df-clab}} &
  The class of all sets where $\varphi$ is true. \\
\texttt{=} & $ = $ &
  \hyperref[df-cleq]{\texttt{df-cleq}} &
  Class equality; $A = B$ iff $A$ equals $B$. \\
\texttt{e.} & $ \in $ &
  \hyperref[df-clel]{\texttt{df-clel}} &
  Class membership; $A \in B$ if $A$ is a member of $B$. \\
\texttt{{\char`\_}V} & {\rm V} &
  \hyperref[df-v]{\texttt{df-v}} &
  Class of all sets (not itself a set). \\
\texttt{C\_} & $ \subseteq $ &
  \hyperref[df-ss]{\texttt{df-ss}} &
  Subclass (subset); $A \subseteq B$ is true iff
  $A$ is a subclass of $B$. \\
\texttt{u.} & $ \cup $ &
  \hyperref[df-un]{\texttt{df-un}} &
  $A \cup B$ is the union of classes $A$ and $B$. \\
\texttt{i^i} & $ \cap $ &
  \hyperref[df-in]{\texttt{df-in}} &
  $A \cap B$ is the intersection of classes $A$ and $B$. \\
\texttt{\char`\\} & $ \setminus $ &
  \hyperref[df-dif]{\texttt{df-dif}} &
  $A \setminus B$ (set difference)
  is the class of all sets in $A$ except for those in $B$. \\
\texttt{(/)} & $ \varnothing $ &
  \hyperref[df-nul]{\texttt{df-nul}} &
  $ \varnothing $ is the empty set (aka null set). \\
\texttt{\char`\~P} & $ \cal P $ &
  \hyperref[df-pw]{\texttt{df-pw}} &
  Power class. \\
\texttt{<.\ A , B >.} & $\langle A , B \rangle$ &
  \hyperref[df-op]{\texttt{df-op}} &
  The ordered pair $\langle A , B \rangle$. \\
\texttt{( F ` A )} & $ ( F ` A ) $ &
  \hyperref[df-fv]{\texttt{df-fv}} &
  The value of function $F$ when applied to $A$. \\
\texttt{\_i} & $ i $ &
  \texttt{df-i} &
  The square root of negative one. \\
\texttt{x.} & $ \cdot $ &
  \texttt{df-mul} &
  Complex number multiplication; $2~\cdot~3~=~6$. \\
\texttt{CC} & $ \mathbb{C} $ &
  \texttt{df-c} &
  The set of complex numbers. \\
\texttt{RR} & $ \mathbb{R} $ &
  \texttt{df-r} &
  The set of real numbers. \\
\end{longtabu}
} % end of extrarowsep

\chapter{Compressed Proofs}
\label{compressed}\index{compressed proof}\index{proof!compressed}

The proofs in the \texttt{set.mm} set theory database are stored in compressed
format for efficiency.  Normally you needn't concern yourself with the
compressed format, since you can display it with the usual proof display tools
in the Metamath program (\texttt{show proof}\ldots) or convert it to the normal
RPN proof format described in Section~\ref{proof} (with \texttt{save proof}
{\em label} \texttt{/normal}).  However for sake of completeness we describe the
format here and show how it maps to the normal RPN proof format.

A compressed proof, located between \texttt{\$=} and \texttt{\$.}\ keywords, consists
of a left parenthesis, a sequence of statement labels, a right parenthesis,
and a sequence of upper-case letters \texttt{A} through \texttt{Z} (with optional
white space between them).  White space must surround the parentheses
and the labels.  The left parenthesis tells Metamath that a
compressed proof follows.  (A normal RPN proof consists of just a sequence of
labels, and a parenthesis is not a legal character in a label.)

The sequence of upper-case letters corresponds to a sequence of integers
with the following mapping.  Each integer corresponds to a proof step as
described later.
\begin{center}
  \texttt{A} = 1 \\
  \texttt{B} = 2 \\
   \ldots \\
  \texttt{T} = 20 \\
  \texttt{UA} = 21 \\
  \texttt{UB} = 22 \\
   \ldots \\
  \texttt{UT} = 40 \\
  \texttt{VA} = 41 \\
  \texttt{VB} = 42 \\
   \ldots \\
  \texttt{YT} = 120 \\
  \texttt{UUA} = 121 \\
   \ldots \\
  \texttt{YYT} = 620 \\
  \texttt{UUUA} = 621 \\
   etc.
\end{center}

In other words, \texttt{A} through \texttt{T} represent the
least-significant digit in base 20, and \texttt{U} through \texttt{Y}
represent zero or more most-significant digits in base 5, where the
digits start counting at 1 instead of the usual 0. With this scheme, we
don't need white space between these ``numbers.''

(In the design of the compressed proof format, only upper-case letters,
as opposed to say all non-whitespace printable {\sc ascii} characters other than
%\texttt{\$}, was chosen to make the compressed proof a little less
%displeasing to the eye, at the expense of a typical 20\% compression
\texttt{\$}, were chosen so as not to collide with most text editor
searches, at the expense of a typical 20\% compression
loss.  The base 5/base 20 grouping, as opposed to say base 6/base 19,
was chosen by experimentally determining the grouping that resulted in
best typical compression.)

The letter \texttt{Z} identifies (tags) a proof step that is identical to one
that occurs later on in the proof; it helps shorten the proof by not requiring
that identical proof steps be proved over and over again (which happens often
when building wff's).  The \texttt{Z} is placed immediately after the
least-significant digit (letters \texttt{A} through \texttt{T}) that ends the integer
corresponding to the step to later be referenced.

The integers that the upper-case letters correspond to are mapped to labels as
follows.  If the statement being proved has $m$ mandatory hypotheses, integers
1 through $m$ correspond to the labels of these hypotheses in the order shown
by the \texttt{show statement ... / full} command, i.e., the RPN order\index{RPN
order} of the mandatory
hypotheses.  Integers $m+1$ through $m+n$ correspond to the labels enclosed in
the parentheses of the compressed proof, in the order that they appear, where
$n$ is the number of those labels.  Integers $m+n+1$ on up don't directly
correspond to statement labels but point to proof steps identified with the
letter \texttt{Z}, so that these proof steps can be referenced later in the
proof.  Integer $m+n+1$ corresponds to the first step tagged with a \texttt{Z},
$m+n+2$ to the second step tagged with a \texttt{Z}, etc.  When the compressed
proof is converted to a normal proof, the entire subproof of a step tagged
with \texttt{Z} replaces the reference to that step.

For efficiency, Metamath works with compressed proofs directly, without
converting them internally to normal proofs.  In addition to the usual
error-checking, an error message is given if (1) a label in the label list in
parentheses does not refer to a previous \texttt{\$p} or \texttt{\$a} statement or a
non-mandatory hypothesis of the statement being proved and (2) a proof step
tagged with \texttt{Z} is referenced before the step tagged with the \texttt{Z}.

Just as in a normal proof under development (Section~\ref{unknown}), any step
or subproof that is not yet known may be represented with a single \texttt{?}.
White space does not have to appear between the \texttt{?}\ and the upper-case
letters (or other \texttt{?}'s) representing the remainder of the proof.

% April 1, 2004 Appendix C has been added back in with corrections.
%
% May 20, 2003 Appendix C was removed for now because there was a problem found
% by Bob Solovay
%
% Also, removed earlier \ref{formalspec} 's (3 cases above)
%
% Bob Solovay wrote on 30 Nov 2002:
%%%%%%%%%%%%% (start of email comment )
%      3. My next set of comments concern appendix C. I read this before I
% read Chapter 4. So I first noted that the system as presented in the
% Appendix lacked a certain formal property that I thought desirable. I
% then came up with a revised formal system that had this property. Upon
% reading Chapter 4, I noticed that the revised system was closer to the
% treatment in Chapter 4 than the system in Appendix C.
%
%         First a very minor correction:
%
%         On page 142 line 2: The condition that V(e) != V(f) should only be
% required of e, f in T such that e != f.
%
%         Here is a natural property [transitivity] that one would like
% the formal system to have:
%
%         Let Gamma be a set of statements. Suppose that the statement Phi
% is provable from Gamma and that the statement Psi is provable from Gamma
% \cup {Phi}. Then Psi is provable from Gamma.
%
%         I shall present an example to show that this property does not
% hold for the formal systems of Appendix C:
%
%         I write the example in metamath style:
%
% $c A B C D E $.
% $v x y
%
% ${
% tx $f A x $.
% ty $f B y $.
% ax1 $a C x y $.
% $}
%
% ${
% tx $f A x $.
% ty $f B y $.
% ax2-h1 $e C x y $.
% ax2 $a D y $.
% $}
%
% ${
% ty $f B y $.
% ax3-h1 $e D y $.
% ax3 $a E y $.
% $}
%
% $(These three axioms are Gamma $)
%
% ${
% tx $f A x $.
% ty $f B y $.
% Phi $p D y $=
% tx ty tx ty ax1 ax2 $.
% $}
%
% ${
% ty $f B y $.
% Psi $p E y $=
% ty ty Phi ax3 $.
% $}
%
%
% I omit the formal proofs of the following claims. [I will be glad to
% supply them upon request.]
%
% 1) Psi is not provable from Gamma;
%
% 2) Psi is provable from Gamma + Phi.
%
% Here "provable" refers to the formalism of Appendix C.
%
% The trouble of course is that Psi is lacking the variable declaration
%
% $f Ax $.
%
% In the Metamath system there is no trouble proving Psi. I attach a
% metamath file that shows this and which has been checked by the
% metamath program.
%
% I next want to indicate how I think the treatment in Appendix C should
% be revised so as to conform more closely to the metamath system of the
% main text. The revised system *does* have the transitivity property.
%
% We want to give revised definitions of "statement" and
% "provable". [cf. sections C.2.4. and C.2.5] Our new definitions will
% use the definitions given in Appendix C. So we take the following
% tack. We refer to the original notions as o-statement and o-provable. And
% we refer to the notions we are defining as n-statement and n-provable.
%
%         A n-statement is an o-statement in which the only variables
% that appear in the T component are mandatory.
%
%         To any o-statement we can associate its reduct which is a
% n-statement by dropping all the elements of T or D which contain
% non-mandatory variables.
%
%         An n-statement gamma is n-provable if there is an o-statement
% gamma' which has gamma as its reduct andf such that gamma' is
% o-provable.
%
%         It seems to me [though I am not completely sure on this point]
% that n-provability corresponds to metamath provability as discussed
% say in Chapter 4.
%
%         Attached to this letter is the metamath proof of Phi and Psi
% from Gamma discussed above.
%
%         I am still brooding over the question of whether metamath
% correctly formalizes set-theory. No doubt I will have some questions
% re this after my thoughts become clearer.
%%%%%%%%%%%%%%%% (end of email comment)

%%%%%%%%%%%%%%%% (start of 2nd email comment from Bob Solovay 1-Apr-04)
%
%         I hope that Appendix C is the one that gives a "formal" treatment
% of Metamath. At any rate, thats the appendix I want to comment on.
%
%         I'm going to suggest two changes in the definition.
%
%         First change (in the definition of statement): Require that the
% sets D, T, and E be finite.
%
%         Probably things are fine as you give them. But in the applications
% to the main metamath system they will always be finite, and its useful in
% thinking about things [at least for me] to stick to the finite case.
%
%         Second change:
%
%         First let me give an approximate description. Remove the dummy
% variables from the statement. Instead, include them in the proof.
%
%         More formally: Require that T consists of type declarations only
% for mandatory variables. Require that all the pairs in D consist of
% mandatory variables.
%
%         At the start of a proof we are allowed to declare a finite number
% of dummy variables [provided that none of them appear in any of the
% statements in E \cup {A}. We have to supply type declarations for all the
% dummy variables. We are allowed to add new $d statements referring to
% either the mandatory or dummy variables. But we require that no new $d
% statement references only mandatory variables.
%
%         I find this way of doing things more conceptual than the treatment
% in Appendix C. But the change [which I will use implicitly in later
% letters about doing Peano] is mainly aesthetic. I definitely claim that my
% results on doing Peano all apply to Metamath as it is presented in your
% book.
%
%         --Bob
%
%%%%%%%%%%%%%%%% (end of 2nd email comment)

%%
%% When uncommenting the below, also uncomment references above to {formalspec}
%%
\chapter{Metamath's Formal System}\label{formalspec}\index{Metamath!as a formal
system}

\section{Introduction}

\begin{quote}
  {\em Perfection is when there is no longer anything more to take away.}
    \flushright\sc Antoine de
     Saint-Exupery\footnote{\cite[p.~3-25]{Campbell}.}\\
\end{quote}\index{de Saint-Exupery, Antoine}

This appendix describes the theory behind the Metamath language in an abstract
way intended for mathematicians.  Specifically, we construct two
set-theo\-ret\-i\-cal objects:  a ``formal system'' (roughly, a set of syntax
rules, axioms, and logical rules) and its ``universe'' (roughly, the set of
theorems derivable in the formal system).  The Metamath computer language
provides us with a way to describe specific formal systems and, with the aid of
a proof provided by the user, to verify that given theorems
belong to their universes.

To understand this appendix, you need a basic knowledge of informal set theory.
It should be sufficient to understand, for example, Ch.\ 1 of Munkres' {\em
Topology} \cite{Munkres}\index{Munkres, James R.} or the
introductory set theory chapter
in many textbooks that introduce abstract mathematics. (Note that there are
minor notational differences among authors; e.g.\ Munkres uses $\subset$ instead
of our $\subseteq$ for ``subset.''  We use ``included in'' to mean ``a subset
of,'' and ``belongs to'' or ``is contained in'' to mean ``is an element of.'')
What we call a ``formal'' description here, unlike earlier, is actually an
informal description in the ordinary language of mathematicians.  However we
provide sufficient detail so that a mathematician could easily formalize it,
even in the language of Metamath itself if desired.  To understand the logic
examples at the end of this appendix, familiarity with an introductory book on
mathematical logic would be helpful.

\section{The Formal Description}

\subsection[Preliminaries]{Preliminaries\protect\footnotemark}%
\footnotetext{This section is taken mostly verbatim
from Tarski \cite[p.~63]{Tarski1965}\index{Tarski, Alfred}.}

By $\omega$ we denote the set of all natural numbers (non-negative integers).
Each natural number $n$ is identified with the set of all smaller numbers: $n =
\{ m | m < n \}$.  The formula $m < n$ is thus equivalent to the condition: $m
\in n$ and $m,n \in \omega$. In particular, 0 is the number zero and at the
same time the empty set $\varnothing$, $1=\{0\}$, $2=\{0,1\}$, etc. ${}^B A$
denotes the set of all functions on $B$ to $A$ (i.e.\ with domain $B$ and range
included in $A$).  The members of ${}^\omega A$ are what are called {\em simple
infinite sequences},\index{simple infinite sequence}
with all {\em terms}\index{term} in $A$.  In case $n \in \omega$, the
members of ${}^n A$ are referred to as {\em finite $n$-termed
sequences},\index{finite $n$-termed
sequence} again
with terms in $A$.  The consecutive terms (function values) of a finite or
infinite sequence $f$ are denoted by $f_0, f_1, \ldots ,f_n,\ldots$.  Every
finite sequence $f \in \bigcup _{n \in \omega} {}^n A$ uniquely determines the
number $n$ such that $f \in {}^n A$; $n$ is called the {\em
length}\index{length of a sequence ({$"|\ "|$})} of $f$ and
is denoted by $|f|$.  $\langle a \rangle$ is the sequence $f$ with $|f|=1$ and
$f_0=a$; $\langle a,b \rangle$ is the sequence $f$ with $|f|=2$, $f_0=a$,
$f_1=b$; etc.  Given two finite sequences $f$ and $g$, we denote by $f\frown g$
their {\em concatenation},\index{concatenation} i.e., the
finite sequence $h$ determined by the
conditions:
\begin{eqnarray*}
& |h| = |f|+|g|;&  \\
& h_n = f_n & \mbox{\ for\ } n < |f|;  \\
& h_{|f|+n} = g_n & \mbox{\ for\ } n < |g|.
\end{eqnarray*}

\subsection{Constants, Variables, and Expressions}

A formal system has a set of {\em symbols}\index{symbol!in
a formal system} denoted
by $\mbox{\em SM}$.  A
precise set-theo\-ret\-i\-cal definition of this set is unimportant; a symbol
could be considered a primitive or atomic element if we wish.  We assume this
set is divided into two disjoint subsets:  a set $\mbox{\em CN}$ of {\em
constants}\index{constant!in a formal system} and a set $\mbox{\em VR}$ of
{\em variables}.\index{variable!in a formal system}  $\mbox{\em CN}$ and
$\mbox{\em VR}$ are each assumed to consist of countably many symbols which
may be arranged in finite or simple infinite sequences $c_0, c_1, \ldots$ and
$v_0, v_1, \ldots$ respectively, without repeating terms.  We will represent
arbitrary symbols by metavariables $\alpha$, $\beta$, etc.

{\footnotesize\begin{quotation}
{\em Comment.} The variables $v_0, v_1, \ldots$ of our formal system
correspond to what are usually considered ``metavariables'' in
descriptions of specific formal systems in the literature.  Typically,
when describing a specific formal system a book will postulate a set of
primitive objects called variables, then proceed to describe their
properties using metavariables that range over them, never mentioning
again the actual variables themselves.  Our formal system does not
mention these primitive variable objects at all but deals directly with
metavariables, as its primitive objects, from the start.  This is a
subtle but key distinction you should keep in mind, and it makes our
definition of ``formal system'' somewhat different from that typically
found in the literature.  (So, the $\alpha$, $\beta$, etc.\ above are
actually ``metametavariables'' when used to represent $v_0, v_1,
\ldots$.)
\end{quotation}}

Finite sequences all terms of which are symbols are called {\em
expressions}.\index{expression!in a formal system}  $\mbox{\em EX}$ is
the set of all expressions; thus
\begin{displaymath}
\mbox{\em EX} = \bigcup _{n \in \omega} {}^n \mbox{\em SM}.
\end{displaymath}

A {\em constant-prefixed expression}\index{constant-prefixed expression}
is an expression of non-zero length
whose first term is a constant.  We denote the set of all constant-prefixed
expressions by $\mbox{\em EX}_C = \{ e \in \mbox{\em EX} | ( |e| > 0 \wedge
e_0 \in \mbox{\em CN} ) \}$.

A {\em constant-variable pair}\index{constant-variable pair}
is an expression of length 2 whose first term
is a constant and whose second term is a variable.  We denote the set of all
constant-variable pairs by $\mbox{\em EX}_2 = \{ e \in \mbox{\em EX}_C | ( |e|
= 2 \wedge e_1 \in \mbox{\em VR} ) \}$.


{\footnotesize\begin{quotation}
{\em Relationship to Metamath.} In general, the set $\mbox{\em SM}$
corresponds to the set of declared math symbols in a Metamath database, the
set $\mbox{\em CN}$ to those declared with \texttt{\$c} statements, and the set
$\mbox{\em VR}$ to those declared with \texttt{\$v} statements.  Of course a
Metamath database can only have a finite number of math symbols, whereas
formal systems in general can have an infinite number, although the number of
Metamath math symbols available is in principle unlimited.

The set $\mbox{\em EX}_C$ corresponds to the set of permissible expressions
for \texttt{\$e}, \texttt{\$a}, and \texttt{\$p} statements.  The set $\mbox{\em EX}_2$
corresponds to the set of permissible expressions for \texttt{\$f} statements.
\end{quotation}}

We denote by ${\cal V}(e)$ the set of all variables in an expression $e \in
\mbox{\em EX}$, i.e.\ the set of all $\alpha \in \mbox{\em VR}$ such that
$\alpha = e_n$ for some $n < |e|$.  We also denote (with abuse of notation) by
${\cal V}(E)$ the set of all variables in a collection of expressions $E
\subseteq \mbox{\em EX}$, i.e.\ $\bigcup _{e \in E} {\cal V}(e)$.


\subsection{Substitution}

Given a function $F$ from $\mbox{\em VR}$ to
$\mbox{\em EX}$, we
denote by $\sigma_{F}$ or just $\sigma$ the function from $\mbox{\em EX}$ to
$\mbox{\em EX}$ defined recursively for nonempty sequences by
\begin{eqnarray*}
& \sigma(<\alpha>) = F(\alpha) & \mbox{for\ } \alpha \in \mbox{\em VR}; \\
& \sigma(<\alpha>) = <\alpha> & \mbox{for\ } \alpha \not\in \mbox{\em VR}; \\
& \sigma(g \frown h) = \sigma(g) \frown
    \sigma(h) & \mbox{for\ } g,h \in \mbox{\em EX}.
\end{eqnarray*}
We also define $\sigma(\varnothing)=\varnothing$.  We call $\sigma$ a {\em
simultaneous substitution}\index{substitution!variable}\index{variable
substitution} (or just {\em substitution}) with {\em substitution
map}\index{substitution map} $F$.

We also denote (with abuse of notation) by $\sigma(E)$ a substitution on a
collection of expressions $E \subseteq \mbox{\em EX}$, i.e.\ the set $\{
\sigma(e) | e \in E \}$.  The collection $\sigma(E)$ may of course contain
fewer expressions than $E$ because duplicate expressions could result from the
substitution.

\subsection{Statements}

We denote by $\mbox{\em DV}$ the set of all
unordered pairs $\{\alpha, \beta \} \subseteq \mbox{\em VR}$ such that $\alpha
\neq \beta$.  $\mbox{\em DV}$ stands for ``distinct variables.''

A {\em pre-statement}\index{pre-statement!in a formal system} is a
quadruple $\langle D,T,H,A \rangle$ such that
$D\subseteq \mbox{\em DV}$, $T\subseteq \mbox{\em EX}_2$, $H\subseteq
\mbox{\em EX}_C$ and $H$ is finite,
$A\in \mbox{\em EX}_C$, ${\cal V}(H\cup\{A\}) \subseteq
{\cal V}(T)$, and $\forall e,f\in T {\ } {\cal V}(e) \neq {\cal V}(f)$ (or
equivalently, $e_1 \ne f_1$) whenever $e \neq f$. The terms of the quadruple are called {\em
distinct-variable restrictions},\index{disjoint-variable restriction!in a
formal system} {\em variable-type hypotheses},\index{variable-type
hypothesis!in a formal system} {\em logical hypotheses},\index{logical
hypothesis!in a formal system} and the {\em assertion}\index{assertion!in a
formal system} respectively.  We denote by $T_M$ ({\em mandatory variable-type
hypotheses}\index{mandatory variable-type hypothesis!in a formal system}) the
subset of $T$ such that ${\cal V}(T_M) ={\cal V}(H \cup \{A\})$.  We denote by
$D_M=\{\{\alpha,\beta\}\in D|\{\alpha,\beta\}\subseteq {\cal V}(T_M)\}$ the
{\em mandatory distinct-variable restrictions}\index{mandatory
disjoint-variable restriction!in a formal system} of the pre-statement.
The set
of {\em mandatory hypotheses}\index{mandatory hypothesis!in a formal system}
is $T_M\cup H$.  We call the quadruple $\langle D_M,T_M,H,A \rangle$
the {\em reduct}\index{reduct!in a formal system} of
the pre-statement $\langle D,T,H,A \rangle$.

A {\em statement} is the reduct of some pre-statement\index{statement!in a
formal system}.  A statement is therefore a special kind of pre-statement;
in particular, a statement is the reduct of itself.

{\footnotesize\begin{quotation}
{\em Comment.}  $T$ is a set of expressions, each of length 2, that associate
a set of constants (``variable types'') with a set of variables.  The
condition ${\cal V}(H\cup\{A\}) \subseteq {\cal V}(T) $
means that each variable occurring in a statement's logical
hypotheses or assertion must have an associated variable-type hypothesis or
``type declaration,'' in  analogy to a computer programming language, where a
variable must be declared to be say, a string or an integer.  The requirement
that $\forall e,f\in T \, e_1 \ne f_1$ for $e\neq f$
means that each variable must be
associated with a unique constant designating its variable type; e.g., a
variable might be a ``wff'' or a ``set'' but not both.

Distinct-variable restrictions are used to specify what variable substitutions
are permissible to make for the statement to remain valid.  For example, in
the theorem scheme of set theory $\lnot\forall x\,x=y$ we may not substitute
the same variable for both $x$ and $y$.  On the other hand, the theorem scheme
$x=y\to y=x$ does not require that $x$ and $y$ be distinct, so we do not
require a distinct-variable restriction, although having one
would cause no harm other than making the scheme less general.

A mandatory variable-type hypothesis is one whose variable exists in a logical
hypothesis or the assertion.  A provable pre-statement
(defined below) may require
non-mandatory variable-type hypotheses that effectively introduce ``dummy''
variables for use in its proof.  Any number of dummy variables might
be required by a specific proof; indeed, it has been shown by H.\
Andr\'{e}ka\index{Andr{\'{e}}ka, H.} \cite{Nemeti} that there is no finite
upper bound to the number of dummy variables needed to prove an arbitrary
theorem in first-order logic (with equality) having a fixed number $n>2$ of
individual variables.  (See also the Comment on p.~\pageref{nodd}.)
For this reason we do not set a finite size bound on the collections $D$ and
$T$, although in an actual application (Metamath database) these will of
course be finite, increased to whatever size is necessary as more
proofs are added.
\end{quotation}}

{\footnotesize\begin{quotation}
{\em Relationship to Metamath.} A pre-statement of a formal system
corresponds to an extended frame in a Metamath database
(Section~\ref{frames}).  The collections $D$, $T$, and $H$ correspond
respectively to the \texttt{\$d}, \texttt{\$f}, and \texttt{\$e}
statement collections in an extended frame.  The expression $A$
corresponds to the \texttt{\$a} (or \texttt{\$p}) statement in an
extended frame.

A statement of a formal system corresponds to a frame in a Metamath
database.
\end{quotation}}

\subsection{Formal Systems}

A {\em formal system}\index{formal system} is a
triple $\langle \mbox{\em CN},\mbox{\em
VR},\Gamma\rangle$ where $\Gamma$ is a set of statements.  The members of
$\Gamma$ are called {\em axiomatic statements}.\index{axiomatic
statement!in a formal system}  Sometimes we will refer to a
formal system by just $\Gamma$ when $\mbox{\em CN}$ and $\mbox{\em VR}$ are
understood.

Given a formal system $\Gamma$, the {\em closure}\index{closure}\footnote{This
definition of closure incorporates a simplification due to
Josh Purinton.\index{Purinton, Josh}.} of a
pre-statement
$\langle D,T,H,A \rangle$ is the smallest set $C$ of expressions
such that:
%\begin{enumerate}
%  \item $T\cup H\subseteq C$; and
%  \item If for some axiomatic statement
%    $\langle D_M',T_M',H',A' \rangle \in \Gamma_A$, for
%    some $E \subseteq C$, some $F \subseteq C-T$ (where ``-'' denotes
%    set difference), and some substitution
%    $\sigma$ we have
%    \begin{enumerate}
%       \item $\sigma(T_M') = E$ (where, as above, the $M$ denotes the
%           mandatory variable-type hypotheses of $T^A$);
%       \item $\sigma(H') = F$;
%       \item for all $\{\alpha,\beta\}\in D^A$ and $\subseteq
%         {\cal V}(T_M')$, for all $\gamma\in {\cal V}(\sigma(\langle \alpha
%         \rangle))$, and for all $\delta\in  {\cal V}(\sigma(\langle \beta
%         \rangle))$, we have $\{\gamma, \delta\} \in D$;
%   \end{enumerate}
%   then $\sigma(A') \in C$.
%\end{enumerate}
\begin{list}{}{\itemsep 0.0pt}
  \item[1.] $T\cup H\subseteq C$; and
  \item[2.] If for some axiomatic statement
    $\langle D_M',T_M',H',A' \rangle \in
       \Gamma$ and for some substitution
    $\sigma$ we have
    \begin{enumerate}
       \item[a.] $\sigma(T_M' \cup H') \subseteq C$; and
       \item[b.] for all $\{\alpha,\beta\}\in D_M'$, for all $\gamma\in
         {\cal V}(\sigma(\langle \alpha
         \rangle))$, and for all $\delta\in  {\cal V}(\sigma(\langle \beta
         \rangle))$, we have $\{\gamma, \delta\} \in D$;
   \end{enumerate}
   then $\sigma(A') \in C$.
\end{list}
A pre-statement $\langle D,T,H,A
\rangle$ is {\em provable}\index{provable statement!in a formal
system} if $A\in C$ i.e.\ if its assertion belongs to its
closure.  A statement is {\em provable} if it is
the reduct of a provable pre-statement.
The {\em universe}\index{universe of a formal system}
of a formal system is
the collection of all of its provable statements.  Note that the
set of axiomatic statements $\Gamma$ in a formal system is a subset of its
universe.

{\footnotesize\begin{quotation}
{\em Comment.} The first condition in the definition of closure simply says
that the hypotheses of the pre-statement are in its closure.

Condition 2(a) says that a substitution exists that makes the
mandatory hypotheses of an axiomatic statement exactly match some members of
the closure.  This is what we explicitly demonstrate in a Metamath language
proof.

%Conditions 2(a) and 2(b) say that a substitution exists that makes the
%(mandatory) hypotheses of an axiomatic statement exactly match some members of
%the closure.  This is what we explicitly demonstrate with a Metamath language
%proof.
%
%The set of expressions $F$ in condition 2(b) excludes the variable-type
%hypotheses; this is done because non-mandatory variable-type hypotheses are
%effectively ``dropped'' as irrelevant whereas logical hypotheses must be
%retained to achieve a consistent logical system.

Condition 2(b) describes how distinct-variable restrictions in the axiomatic
statement must be met.  It means that after a substitution for two variables
that must be distinct, the resulting two expressions must either contain no
variables, or if they do, they may not have variables in common, and each pair
of any variables they do have, with one variable from each expression, must be
specified as distinct in the original statement.
\end{quotation}}

{\footnotesize\begin{quotation}
{\em Relationship to Metamath.} Axiomatic statements
 and provable statements in a formal
system correspond to the frames for \texttt{\$a} and \texttt{\$p} statements
respectively in a Metamath database.  The set of axiomatic statements is a
subset of the set of provable statements in a formal system, although in a
Metamath database a \texttt{\$a} statement is distinguished by not having a
proof.  A Metamath language proof for a \texttt{\$p} statement tells the computer
how to explicitly construct a series of members of the closure ultimately
leading to a demonstration that the assertion
being proved is in the closure.  The actual closure typically contains
an infinite number of expressions.  A formal system itself does not have
an explicit object called a ``proof'' but rather the existence of a proof
is implied indirectly by membership of an assertion in a provable
statement's closure.  We do this to make the formal system easier
to describe in the language of set theory.

We also note that once established as provable, a statement may be considered
to acquire the same status as an axiomatic statement, because if the set of
axiomatic statements is extended with a provable statement, the universe of
the formal system remains unchanged (provided that $\mbox{\em VR}$ is
infinite).
In practice, this means we can build a hierarchy of provable statements to
more efficiently establish additional provable statements.  This is
what we do in Metamath when we allow proofs to reference previous
\texttt{\$p} statements as well as previous \texttt{\$a} statements.
\end{quotation}}

\section{Examples of Formal Systems}

{\footnotesize\begin{quotation}
{\em Relationship to Metamath.} The examples in this section, except Example~2,
are for the most part exact equivalents of the development in the set
theory database \texttt{set.mm}.  You may want to compare Examples~1, 3, and 5
to Section~\ref{metaaxioms}, Example 4 to Sections~\ref{metadefprop} and
\ref{metadefpred}, and Example 6 to
Section~\ref{setdefinitions}.\label{exampleref}
\end{quotation}}

\subsection{Example~1---Propositional Calculus}\index{propositional calculus}

Classical propositional calculus can be described by the following formal
system.  We assume the set of variables is infinite.  Rather than denoting the
constants and variables by $c_0, c_1, \ldots$ and $v_0, v_1, \ldots$, for
readability we will instead use more conventional symbols, with the
understanding of course that they denote distinct primitive objects.
Also for readability we may omit commas between successive terms of a
sequence; thus $\langle \mbox{wff\ } \varphi\rangle$ denotes
$\langle \mbox{wff}, \varphi\rangle$.

Let
\begin{itemize}
  \item[] $\mbox{\em CN}=\{\mbox{wff}, \vdash, \to, \lnot, (,)\}$
  \item[] $\mbox{\em VR}=\{\varphi,\psi,\chi,\ldots\}$
  \item[] $T = \{\langle \mbox{wff\ } \varphi\rangle,
             \langle \mbox{wff\ } \psi\rangle,
             \langle \mbox{wff\ } \chi\rangle,\ldots\}$, i.e.\ those
             expressions of length 2 whose first member is $\mbox{\rm wff}$
             and whose second member belongs to $\mbox{\em VR}$.\footnote{For
convenience we let $T$ be an infinite set; the definition of a statement
permits this in principle.  Since a Metamath source file has a finite size, in
practice we must of course use appropriate finite subsets of this $T$,
specifically ones containing at least the mandatory variable-type
hypotheses.  Similarly, in the source file we introduce new variables as
required, with the understanding that a potentially infinite number of
them are available.}
\noindent Then $\Gamma$ consists of the axiomatic statements that
are the reducts of the following pre-statements:
    \begin{itemize}
      \item[] $\langle\varnothing,T,\varnothing,
               \langle \mbox{wff\ }(\varphi\to\psi)\rangle\rangle$
      \item[] $\langle\varnothing,T,\varnothing,
               \langle \mbox{wff\ }\lnot\varphi\rangle\rangle$
      \item[] $\langle\varnothing,T,\varnothing,
               \langle \vdash(\varphi\to(\psi\to\varphi))
               \rangle\rangle$
      \item[] $\langle\varnothing,T,
               \varnothing,
               \langle \vdash((\varphi\to(\psi\to\chi))\to
               ((\varphi\to\psi)\to(\varphi\to\chi)))
               \rangle\rangle$
      \item[] $\langle\varnothing,T,
               \varnothing,
               \langle \vdash((\lnot\varphi\to\lnot\psi)\to
               (\psi\to\varphi))\rangle\rangle$
      \item[] $\langle\varnothing,T,
               \{\langle\vdash(\varphi\to\psi)\rangle,
                 \langle\vdash\varphi\rangle\},
               \langle\vdash\psi\rangle\rangle$
    \end{itemize}
\end{itemize}

(For example, the reduct of $\langle\varnothing,T,\varnothing,
               \langle \mbox{wff\ }(\varphi\to\psi)\rangle\rangle$
is
\begin{itemize}
\item[] $\langle\varnothing,
\{\langle \mbox{wff\ } \varphi\rangle,
             \langle \mbox{wff\ } \psi\rangle\},
             \varnothing,
               \langle \mbox{wff\ }(\varphi\to\psi)\rangle\rangle$,
\end{itemize}
which is the first axiomatic statement.)

We call the members of $\mbox{\em VR}$ {\em wff variables} or (in the context
of first-order logic which we will describe shortly) {\em wff metavariables}.
Note that the symbols $\phi$, $\psi$, etc.\ denote actual specific members of
$\mbox{\em VR}$; they are not metavariables of our expository language (which
we denote with $\alpha$, $\beta$, etc.) but are instead (meta)constant symbols
(members of $\mbox{\em SM}$) from the point of view of our expository
language.  The equivalent system of propositional calculus described in
\cite{Tarski1965} also uses the symbols $\phi$, $\psi$, etc.\ to denote wff
metavariables, but in \cite{Tarski1965} unlike here those are metavariables of
the expository language and not primitive symbols of the formal system.

The first two statements define wffs: if $\varphi$ and $\psi$ are wffs, so is
$(\varphi \to \psi)$; if $\varphi$ is a wff, so is $\lnot\varphi$. The next
three are the axioms of propositional calculus: if $\varphi$ and $\psi$ are
wffs, then $\vdash (\varphi \to (\psi \to \varphi))$ is an (axiomatic)
theorem; etc. The
last is the rule of modus ponens: if $\varphi$ and $\psi$ are wffs, and
$\vdash (\varphi\to\psi)$ and $\vdash \varphi$ are theorems, then $\vdash
\psi$ is a theorem.

The correspondence to ordinary propositional calculus is as follows.  We
consider only provable statements of the form $\langle\varnothing,
T,\varnothing,A\rangle$ with $T$ defined as above.  The first term of the
assertion $A$ of any such statement is either ``wff'' or ``$\vdash$''.  A
statement for which the first term is ``wff'' is a {\em wff} of propositional
calculus, and one where the first term is ``$\vdash$'' is a {\em
theorem (scheme)} of propositional calculus.

The universe of this formal system also contains many other provable
statements.  Those with distinct-variable restrictions are irrelevant because
propositional calculus has no constraints on substitutions.  Those that have
logical hypotheses we call {\em inferences}\index{inference} when
the logical hypotheses are of the form
$\langle\vdash\rangle\frown w$ where $w$ is a wff (with the leading constant
term ``wff'' removed).  Inferences (other than the modus ponens rule) are not a
proper part of propositional calculus but are convenient to use when building a
hierarchy of provable statements.  A provable statement with a nonsense
hypothesis such as $\langle \to,\vdash,\lnot\rangle$, and this same expression
as its assertion, we consider irrelevant; no use can be made of it in
proving theorems, since there is no way to eliminate the nonsense hypothesis.

{\footnotesize\begin{quotation}
{\em Comment.} Our use of parentheses in the definition of a wff illustrates
how axiomatic statements should be carefully stated in a way that
ties in unambiguously with the substitutions allowed by the formal system.
There are many ways we could have defined wffs---for example, Polish
prefix notation would have allowed us to omit parentheses entirely, at
the expense of readability---but we must define them in a way that is
unambiguous.  For example, if we had omitted parentheses from the
definition of $(\varphi\to \psi)$, the wff $\lnot\varphi\to \psi$ could
be interpreted as either $\lnot(\varphi\to\psi)$ or $(\lnot\varphi\to\psi)$
and would have allowed us to prove nonsense.  Note that there is no
concept of operator binding precedence built into our formal system.
\end{quotation}}

\begin{sloppy}
\subsection{Example~2---Predicate Calculus with Equality}\index{predicate
calculus}
\end{sloppy}

Here we extend Example~1 to include predicate calculus with equality,
illustrating the use of distinct-variable restrictions.  This system is the
same as Tarski's system $\mathfrak{S}_2$ in \cite{Tarski1965} (except that the
axioms of propositional calculus are different but equivalent, and a redundant
axiom is omitted).  We extend $\mbox{\em CN}$ with the constants
$\{\mbox{var},\forall,=\}$.  We extend $\mbox{\em VR}$ with an infinite set of
{\em individual metavariables}\index{individual
metavariable} $\{x,y,z,\ldots\}$ and denote this subset
$\mbox{\em Vr}$.

We also join to $\mbox{\em CN}$ a possibly infinite set $\mbox{\em Pr}$ of {\em
predicates} $\{R,S,\ldots\}$.  We associate with $\mbox{\em Pr}$ a function
$\mbox{rnk}$ from $\mbox{\em Pr}$ to $\omega$, and for $\alpha\in \mbox{\em
Pr}$ we call $\mbox{rnk}(\alpha)$ the {\em rank} of the predicate $\alpha$,
which is simply the number of ``arguments'' that the predicate has.  (Most
applications of predicate calculus will have a finite number of predicates;
for example, set theory has the single two-argument or binary predicate $\in$,
which is usually written with its arguments surrounding the predicate symbol
rather than with the prefix notation we will use for the general case.)  As a
device to facilitate our discussion, we will let $\mbox{\em Vs}$ be any fixed
one-to-one function from $\omega$ to $\mbox{\em Vr}$; thus $\mbox{\em Vs}$ is
any simple infinite sequence of individual metavariables with no repeating
terms.

In this example we will not include the function symbols that are often part of
formalizations of predicate calculus.  Using metalogical arguments that are
beyond the scope of our discussion, it can be shown that our formalization is
equivalent when functions are introduced via appropriate definitions.

We extend the set $T$ defined in Example~1 with the expressions
$\{\langle \mbox{var\ } x\rangle,$ $ \langle \mbox{var\ } y\rangle, \langle
\mbox{var\ } z\rangle,\ldots\}$.  We extend the $\Gamma$ above
with the axiomatic statements that are the reducts of the following
pre-statements:
\begin{list}{}{\itemsep 0.0pt}
      \item[] $\langle\varnothing,T,\varnothing,
               \langle \mbox{wff\ }\forall x\,\varphi\rangle\rangle$
      \item[] $\langle\varnothing,T,\varnothing,
               \langle \mbox{wff\ }x=y\rangle\rangle$
      \item[] $\langle\varnothing,T,
               \{\langle\vdash\varphi\rangle\},
               \langle\vdash\forall x\,\varphi\rangle\rangle$
      \item[] $\langle\varnothing,T,\varnothing,
               \langle \vdash((\forall x(\varphi\to\psi)
                  \to(\forall x\,\varphi\to\forall x\,\psi))
               \rangle\rangle$
      \item[] $\langle\{\{x,\varphi\}\},T,\varnothing,
               \langle \vdash(\varphi\to\forall x\,\varphi)
               \rangle\rangle$
      \item[] $\langle\{\{x,y\}\},T,\varnothing,
               \langle \vdash\lnot\forall x\lnot x=y
               \rangle\rangle$
      \item[] $\langle\varnothing,T,\varnothing,
               \langle \vdash(x=z
                  \to(x=y\to z=y))
               \rangle\rangle$
      \item[] $\langle\varnothing,T,\varnothing,
               \langle \vdash(y=z
                  \to(x=y\to x=z))
               \rangle\rangle$
\end{list}
These are the axioms not involving predicate symbols. The first two statements
extend the definition of a wff.  The third is the rule of generalization.  The
fifth states, in effect, ``For a wff $\varphi$ and variable $x$,
$\vdash(\varphi\to\forall x\,\varphi)$, provided that $x$ does not occur in
$\varphi$.''  The sixth states ``For variables $x$ and $y$,
$\vdash\lnot\forall x\lnot x = y$, provided that $x$ and $y$ are distinct.''
(This proviso is not necessary but was included by Tarski to
weaken the axiom and still show that the system is logically complete.)

Finally, for each predicate symbol $\alpha\in \mbox{\em Pr}$, we add to
$\Gamma$ an axiomatic statement, extending the definition of wff,
that is the reduct of the following pre-statement:
\begin{displaymath}
    \langle\varnothing,T,\varnothing,
            \langle \mbox{wff},\alpha\rangle\
            \frown \mbox{\em Vs}\restriction\mbox{rnk}(\alpha)\rangle
\end{displaymath}
and for each $\alpha\in \mbox{\em Pr}$ and each $n < \mbox{rnk}(\alpha)$
we add to $\Gamma$ an equality axiom that is the reduct of the
following pre-statement:
\begin{eqnarray*}
    \lefteqn{\langle\varnothing,T,\varnothing,
            \langle
      \vdash,(,\mbox{\em Vs}_n,=,\mbox{\em Vs}_{\mbox{rnk}(\alpha)},\to,
            (,\alpha\rangle\frown \mbox{\em Vs}\restriction\mbox{rnk}(\alpha)} \\
  & & \frown
            \langle\to,\alpha\rangle\frown \mbox{\em Vs}\restriction n\frown
            \langle \mbox{\em Vs}_{\mbox{rnk}(\alpha)}\rangle \\
 & & \frown
            \mbox{\em Vs}\restriction(\mbox{rnk}(\alpha)\setminus(n+1))\frown
            \langle),)\rangle\rangle
\end{eqnarray*}
where $\restriction$ denotes function domain restriction and $\setminus$
denotes set difference.  Recall that a subscript on $\mbox{\em Vs}$
denotes one of its terms.  (In the above two axiom sets commas are placed
between successive terms of sequences to prevent ambiguity, and if you examine
them with care you will be able to distinguish those parentheses that denote
constant symbols from those of our expository language that delimit function
arguments.  Although it might have been better to use boldface for our
primitive symbols, unfortunately boldface was not available for all characters
on the \LaTeX\ system used to typeset this text.)  These seemingly forbidding
axioms can be understood by analogy to concatenation of substrings in a
computer language.  They are actually relatively simple for each specific case
and will become clearer by looking at the special case of a binary predicate
$\alpha = R$ where $\mbox{rnk}(R)=2$.  Letting $\mbox{\em Vs}$ be the sequence
$\langle x,y,z,\ldots\rangle$, the axioms we would add to $\Gamma$ for this
case would be the wff extension and two equality axioms that are the
reducts of the pre-statements:
\begin{list}{}{\itemsep 0.0pt}
      \item[] $\langle\varnothing,T,\varnothing,
               \langle \mbox{wff\ }R x y\rangle\rangle$
      \item[] $\langle\varnothing,T,\varnothing,
               \langle \vdash(x=z
                  \to(R x y \to R z y))
               \rangle\rangle$
      \item[] $\langle\varnothing,T,\varnothing,
               \langle \vdash(y=z
                  \to(R x y \to R x z))
               \rangle\rangle$
\end{list}
Study these carefully to see how the general axioms above evaluate to
them.  In practice, typically only a few special cases such as this would be
needed, and in any case the Metamath language will only permit us to describe
a finite number of predicates, as opposed to the infinite number permitted by
the formal system.  (If an infinite number should be needed for some reason,
we could not define the formal system directly in the Metamath language but
could instead define it metalogically under set theory as we
do in this appendix, and only the underlying set theory, with its single
binary predicate, would be defined directly in the Metamath language.)


{\footnotesize\begin{quotation}
{\em Comment.}  As we noted earlier, the specific variables denoted by the
symbols $x,y,z,\ldots\in \mbox{\em Vr}\subseteq \mbox{\em VR}\subseteq
\mbox{\em SM}$ in Example~2 are not the actual variables of ordinary predicate
calculus but should be thought of as metavariables ranging over them.  For
example, a distinct-variable restriction would be meaningless for actual
variables of ordinary predicate calculus since two different actual variables
are by definition distinct.  And when we talk about an arbitrary
representative $\alpha\in \mbox{\em Vr}$, $\alpha$ is a metavariable (in our
expository language) that ranges over metavariables (which are primitives of
our formal system) each of which ranges over the actual individual variables
of predicate calculus (which are never mentioned in our formal system).

The constant called ``var'' above is called \texttt{setvar} in the
\texttt{set.mm} database file, but it means the same thing.  I felt
that ``var'' is a more meaningful name in the context of predicate
calculus, whose use is not limited to set theory.  For consistency we
stick with the name ``var'' throughout this Appendix, even after set
theory is introduced.
\end{quotation}}

\subsection{Free Variables and Proper Substitution}\index{free variable}
\index{proper substitution}\index{substitution!proper}

Typical representations of mathematical axioms use concepts such
as ``free variable,'' ``bound variable,'' and ``proper substitution''
as primitive notions.
A free variable is a variable that
is not a parameter of any container expression.
A bound variable is the opposite of a free variable; it is a
a variable that has been bound in a container expression.
For example, in the expression $\forall x \varphi$ (for all $x$, $\varphi$
is true), the variable $x$
is bound within the for-all ($\forall$) expression.
It is possible to change one variable to another, and that process is called
``proper substitution.''
In most books, proper substitution has a somewhat complicated recursive
definition with multiple cases based on the occurrences of free and
bound variables.
You may consult
\cite[ch.\ 3--4]{Hamilton}\index{Hamilton, Alan G.} (as well as
many other texts) for more formal details about these terms.

Using these concepts as \texttt{primitives} creates complications
for computer implementations.

In the system of Example~2, there are no primitive notions of free variable
and proper substitution.  Tarski \cite{Tarski1965} shows that this system is
logically equivalent to the more typical textbook systems that do have these
primitive notions, if we introduce these notions with appropriate definitions
and metalogic.  We could also define axioms for such systems directly,
although the recursive definitions of free variable and proper substitution
would be messy and awkward to work with.  Instead, we mention two devices that
can be used in practice to mimic these notions.  (1) Instead of introducing
special notation to express (as a logical hypothesis) ``where $x$ is not free
in $\varphi$'' we can use the logical hypothesis $\vdash(\varphi\to\forall
x\,\varphi)$.\label{effectivelybound}\index{effectively
not free}\footnote{This is a slightly weaker requirement than ``where $x$ is
not free in $\varphi$.''  If we let $\varphi$ be $x=x$, we have the theorem
$(x=x\to\forall x\,x=x)$ which satisfies the hypothesis, even though $x$ is
free in $x=x$ .  In a case like this we say that $x$ is {\em effectively not
free}\index{effectively not free} in $x=x$, since $x=x$ is logically
equivalent to $\forall x\,x=x$ in which $x$ is bound.} (2) It can be shown
that the wff $((x=y\to\varphi)\wedge\exists x(x=y\wedge\varphi))$ (with the
usual definitions of $\wedge$ and $\exists$; see Example~4 below) is logically
equivalent to ``the wff that results from proper substitution of $y$ for $x$
in $\varphi$.''  This works whether or not $x$ and $y$ are distinct.

\subsection{Metalogical Completeness}\index{metalogical completeness}

In the system of Example~2, the
following are provable pre-statements (and their reducts are
provable statements):
\begin{eqnarray*}
      & \langle\{\{x,y\}\},T,\varnothing,
               \langle \vdash\lnot\forall x\lnot x=y
               \rangle\rangle & \\
     &  \langle\varnothing,T,\varnothing,
               \langle \vdash\lnot\forall x\lnot x=x
               \rangle\rangle &
\end{eqnarray*}
whereas the following pre-statement is not to my knowledge provable (but
in any case we will pretend it's not for sake of illustration):
\begin{eqnarray*}
     &  \langle\varnothing,T,\varnothing,
               \langle \vdash\lnot\forall x\lnot x=y
               \rangle\rangle &
\end{eqnarray*}
In other words, we can prove ``$\lnot\forall x\lnot x=y$ where $x$ and $y$ are
distinct'' and separately prove ``$\lnot\forall x\lnot x=x$'', but we can't
prove the combined general case ``$\lnot\forall x\lnot x=y$'' that has no
proviso.  Now this does not compromise logical completeness, because the
variables are really metavariables and the two provable cases together cover
all possible cases.  The third case can be considered a metatheorem whose
direct proof, using the system of Example~2, lies outside the capability of the
formal system.

Also, in the system of Example~2 the following pre-statement is not to my
knowledge provable (again, a conjecture that we will pretend to be the case):
\begin{eqnarray*}
     & \langle\varnothing,T,\varnothing,
               \langle \vdash(\forall x\, \varphi\to\varphi)
               \rangle\rangle &
\end{eqnarray*}
Instead, we can only prove specific cases of $\varphi$ involving individual
metavariables, and by induction on formula length, prove as a metatheorem
outside of our formal system the general statement above.  The details of this
proof are found in \cite{Kalish}.

There does, however, exist a system of predicate calculus in which all such
``simple metatheorems'' as those above can be proved directly, and we present
it in Example~3. A {\em simple metatheorem}\index{simple metatheorem}
is any statement of the formal
system of Example~2 where all distinct variable restrictions consist of either
two individual metavariables or an individual metavariable and a wff
metavariable, and which is provable by combining cases outside the system as
above.  A system is {\em metalogically complete}\index{metalogical
completeness} if all of its simple
metatheorems are (directly) provable statements. The precise definition of
``simple metatheorem'' and the proof of the ``metalogical completeness'' of
Example~3 is found in Remark 9.6 and Theorem 9.7 of \cite{Megill}.\index{Megill,
Norman}

\begin{sloppy}
\subsection{Example~3---Metalogically Complete Predicate
Calculus with
Equality}
\end{sloppy}

For simplicity we will assume there is one binary predicate $R$;
this system suffices for set theory, where the $R$ is of course the $\in$
predicate.  We label the axioms as they appear in \cite{Megill}.  This
system is logically equivalent to that of Example~2 (when the latter is
restricted to this single binary predicate) but is also metalogically
complete.\index{metalogical completeness}

Let
\begin{itemize}
  \item[] $\mbox{\em CN}=\{\mbox{wff}, \mbox{var}, \vdash, \to, \lnot, (,),\forall,=,R\}$.
  \item[] $\mbox{\em VR}=\{\varphi,\psi,\chi,\ldots\}\cup\{x,y,z,\ldots\}$.
  \item[] $T = \{\langle \mbox{wff\ } \varphi\rangle,
             \langle \mbox{wff\ } \psi\rangle,
             \langle \mbox{wff\ } \chi\rangle,\ldots\}\cup
       \{\langle \mbox{var\ } x\rangle, \langle \mbox{var\ } y\rangle, \langle
       \mbox{var\ }z\rangle,\ldots\}$.

\noindent Then
  $\Gamma$ consists of the reducts of the following pre-statements:
    \begin{itemize}
      \item[] $\langle\varnothing,T,\varnothing,
               \langle \mbox{wff\ }(\varphi\to\psi)\rangle\rangle$
      \item[] $\langle\varnothing,T,\varnothing,
               \langle \mbox{wff\ }\lnot\varphi\rangle\rangle$
      \item[] $\langle\varnothing,T,\varnothing,
               \langle \mbox{wff\ }\forall x\,\varphi\rangle\rangle$
      \item[] $\langle\varnothing,T,\varnothing,
               \langle \mbox{wff\ }x=y\rangle\rangle$
      \item[] $\langle\varnothing,T,\varnothing,
               \langle \mbox{wff\ }Rxy\rangle\rangle$
      \item[(C1$'$)] $\langle\varnothing,T,\varnothing,
               \langle \vdash(\varphi\to(\psi\to\varphi))
               \rangle\rangle$
      \item[(C2$'$)] $\langle\varnothing,T,
               \varnothing,
               \langle \vdash((\varphi\to(\psi\to\chi))\to
               ((\varphi\to\psi)\to(\varphi\to\chi)))
               \rangle\rangle$
      \item[(C3$'$)] $\langle\varnothing,T,
               \varnothing,
               \langle \vdash((\lnot\varphi\to\lnot\psi)\to
               (\psi\to\varphi))\rangle\rangle$
      \item[(C4$'$)] $\langle\varnothing,T,
               \varnothing,
               \langle \vdash(\forall x(\forall x\,\varphi\to\psi)\to
                 (\forall x\,\varphi\to\forall x\,\psi))\rangle\rangle$
      \item[(C5$'$)] $\langle\varnothing,T,
               \varnothing,
               \langle \vdash(\forall x\,\varphi\to\varphi)\rangle\rangle$
      \item[(C6$'$)] $\langle\varnothing,T,
               \varnothing,
               \langle \vdash(\forall x\forall y\,\varphi\to
                 \forall y\forall x\,\varphi)\rangle\rangle$
      \item[(C7$'$)] $\langle\varnothing,T,
               \varnothing,
               \langle \vdash(\lnot\varphi\to\forall x\lnot\forall x\,\varphi
                 )\rangle\rangle$
      \item[(C8$'$)] $\langle\varnothing,T,
               \varnothing,
               \langle \vdash(x=y\to(x=z\to y=z))\rangle\rangle$
      \item[(C9$'$)] $\langle\varnothing,T,
               \varnothing,
               \langle \vdash(\lnot\forall x\, x=y\to(\lnot\forall x\, x=z\to
                 (y=z\to\forall x\, y=z)))\rangle\rangle$
      \item[(C10$'$)] $\langle\varnothing,T,
               \varnothing,
               \langle \vdash(\forall x(x=y\to\forall x\,\varphi)\to
                 \varphi))\rangle\rangle$
      \item[(C11$'$)] $\langle\varnothing,T,
               \varnothing,
               \langle \vdash(\forall x\, x=y\to(\forall x\,\varphi
               \to\forall y\,\varphi))\rangle\rangle$
      \item[(C12$'$)] $\langle\varnothing,T,
               \varnothing,
               \langle \vdash(x=y\to(Rxz\to Ryz))\rangle\rangle$
      \item[(C13$'$)] $\langle\varnothing,T,
               \varnothing,
               \langle \vdash(x=y\to(Rzx\to Rzy))\rangle\rangle$
      \item[(C15$'$)] $\langle\varnothing,T,
               \varnothing,
               \langle \vdash(\lnot\forall x\, x=y\to(x=y\to(\varphi
                 \to\forall x(x=y\to\varphi))))\rangle\rangle$
      \item[(C16$'$)] $\langle\{\{x,y\}\},T,
               \varnothing,
               \langle \vdash(\forall x\, x=y\to(\varphi\to\forall x\,\varphi)
                 )\rangle\rangle$
      \item[(C5)] $\langle\{\{x,\varphi\}\},T,\varnothing,
               \langle \vdash(\varphi\to\forall x\,\varphi)
               \rangle\rangle$
      \item[(MP)] $\langle\varnothing,T,
               \{\langle\vdash(\varphi\to\psi)\rangle,
                 \langle\vdash\varphi\rangle\},
               \langle\vdash\psi\rangle\rangle$
      \item[(Gen)] $\langle\varnothing,T,
               \{\langle\vdash\varphi\rangle\},
               \langle\vdash\forall x\,\varphi\rangle\rangle$
    \end{itemize}
\end{itemize}

While it is known that these axioms are ``metalogically complete,'' it is
not known whether they are independent (i.e.\ none is
redundant) in the metalogical sense; specifically, whether any axiom (possibly
with additional non-mandatory distinct-variable restrictions, for use with any
dummy variables in its proof) is provable from the others.  Note that
metalogical independence is a weaker requirement than independence in the
usual logical sense.  Not all of the above axioms are logically independent:
for example, C9$'$ can be proved as a metatheorem from the others, outside the
formal system, by combining the possible cases of distinct variables.

\subsection{Example~4---Adding Definitions}\index{definition}
There are several ways to add definitions to a formal system.  Probably the
most proper way is to consider definitions not as part of the formal system at
all but rather as abbreviations that are part of the expository metalogic
outside the formal system.  For convenience, though, we may use the formal
system itself to incorporate definitions, adding them as axiomatic extensions
to the system.  This could be done by adding a constant representing the
concept ``is defined as'' along with axioms for it. But there is a nicer way,
at least in this writer's opinion, that introduces definitions as direct
extensions to the language rather than as extralogical primitive notions.  We
introduce additional logical connectives and provide axioms for them.  For
systems of logic such as Examples 1 through 3, the additional axioms must be
conservative in the sense that no wff of the original system that was not a
theorem (when the initial term ``wff'' is replaced by ``$\vdash$'' of course)
becomes a theorem of the extended system.  In this example we extend Example~3
(or 2) with standard abbreviations of logic.

We extend $\mbox{\em CN}$ of Example~3 with new constants $\{\leftrightarrow,
\wedge,\vee,\exists\}$, corresponding to logical equivalence,\index{logical
equivalence ($\leftrightarrow$)}\index{biconditional ($\leftrightarrow$)}
conjunction,\index{conjunction ($\wedge$)} disjunction,\index{disjunction
($\vee$)} and the existential quantifier.\index{existential quantifier
($\exists$)}  We extend $\Gamma$ with the axiomatic statements that are
the reducts of the following pre-statements:
\begin{list}{}{\itemsep 0.0pt}
      \item[] $\langle\varnothing,T,\varnothing,
               \langle \mbox{wff\ }(\varphi\leftrightarrow\psi)\rangle\rangle$
      \item[] $\langle\varnothing,T,\varnothing,
               \langle \mbox{wff\ }(\varphi\vee\psi)\rangle\rangle$
      \item[] $\langle\varnothing,T,\varnothing,
               \langle \mbox{wff\ }(\varphi\wedge\psi)\rangle\rangle$
      \item[] $\langle\varnothing,T,\varnothing,
               \langle \mbox{wff\ }\exists x\, \varphi\rangle\rangle$
  \item[] $\langle\varnothing,T,\varnothing,
     \langle\vdash ( ( \varphi \leftrightarrow \psi ) \to
     ( \varphi \to \psi ) )\rangle\rangle$
  \item[] $\langle\varnothing,T,\varnothing,
     \langle\vdash ((\varphi\leftrightarrow\psi)\to
    (\psi\to\varphi))\rangle\rangle$
  \item[] $\langle\varnothing,T,\varnothing,
     \langle\vdash ((\varphi\to\psi)\to(
     (\psi\to\varphi)\to(\varphi
     \leftrightarrow\psi)))\rangle\rangle$
  \item[] $\langle\varnothing,T,\varnothing,
     \langle\vdash (( \varphi \wedge \psi ) \leftrightarrow\neg ( \varphi
     \to \neg \psi )) \rangle\rangle$
  \item[] $\langle\varnothing,T,\varnothing,
     \langle\vdash (( \varphi \vee \psi ) \leftrightarrow (\neg \varphi
     \to \psi )) \rangle\rangle$
  \item[] $\langle\varnothing,T,\varnothing,
     \langle\vdash (\exists x \,\varphi\leftrightarrow
     \lnot \forall x \lnot \varphi)\rangle\rangle$
\end{list}
The first three logical axioms (statements containing ``$\vdash$'') introduce
and effectively define logical equivalence, ``$\leftrightarrow$''.  The last
three use ``$\leftrightarrow$'' to effectively mean ``is defined as.''

\subsection{Example~5---ZFC Set Theory}\index{ZFC set theory}

Here we add to the system of Example~4 the axioms of Zermelo--Fraenkel set
theory with Choice.  For convenience we make use of the
definitions in Example~4.

In the $\mbox{\em CN}$ of Example~4 (which extends Example~3), we replace the symbol $R$
with the symbol $\in$.
More explicitly, we remove from $\Gamma$ of Example~4 the three
axiomatic statements containing $R$ and replace them with the
reducts of the following:
\begin{list}{}{\itemsep 0.0pt}
      \item[] $\langle\varnothing,T,\varnothing,
               \langle \mbox{wff\ }x\in y\rangle\rangle$
      \item[] $\langle\varnothing,T,
               \varnothing,
               \langle \vdash(x=y\to(x\in z\to y\in z))\rangle\rangle$
      \item[] $\langle\varnothing,T,
               \varnothing,
               \langle \vdash(x=y\to(z\in x\to z\in y))\rangle\rangle$
\end{list}
Letting $D=\{\{\alpha,\beta\}\in \mbox{\em DV}\,|\alpha,\beta\in \mbox{\em
Vr}\}$ (in other words all individual variables must be distinct), we extend
$\Gamma$ with the ZFC axioms, called
\index{Axiom of Extensionality}
\index{Axiom of Replacement}
\index{Axiom of Union}
\index{Axiom of Power Sets}
\index{Axiom of Regularity}
\index{Axiom of Infinity}
\index{Axiom of Choice}
Extensionality, Replacement, Union, Power
Set, Regularity, Infinity, and Choice, that are the reducts of:
\begin{list}{}{\itemsep 0.0pt}
      \item[Ext] $\langle D,T,
               \varnothing,
               \langle\vdash (\forall x(x\in y\leftrightarrow x \in z)\to y
               =z) \rangle\rangle$
      \item[Rep] $\langle D,T,
               \varnothing,
               \langle\vdash\exists x ( \exists y \forall z (\varphi \to z = y
                        ) \to
                        \forall z ( z \in x \leftrightarrow \exists x ( x \in
                        y \wedge \forall y\,\varphi ) ) )\rangle\rangle$
      \item[Un] $\langle D,T,
               \varnothing,
               \langle\vdash \exists x \forall y ( \exists x ( y \in x \wedge
               x \in z ) \to y \in x ) \rangle\rangle$
      \item[Pow] $\langle D,T,
               \varnothing,
               \langle\vdash \exists x \forall y ( \forall x ( x \in y \to x
               \in z ) \to y \in x ) \rangle\rangle$
      \item[Reg] $\langle D,T,
               \varnothing,
               \langle\vdash (  x \in y \to
                 \exists x ( x \in y \wedge \forall z ( z \in x \to \lnot z
                \in y ) ) ) \rangle\rangle$
      \item[Inf] $\langle D,T,
               \varnothing,
               \langle\vdash \exists x(y\in x\wedge\forall y(y\in
               x\to
               \exists z(y \in z\wedge z\in x))) \rangle\rangle$
      \item[AC] $\langle D,T,
               \varnothing,
               \langle\vdash \exists x \forall y \forall z ( ( y \in z
               \wedge z \in w ) \to \exists w \forall y ( \exists w
              ( ( y \in z \wedge z \in w ) \wedge ( y \in w \wedge w \in x
              ) ) \leftrightarrow y = w ) ) \rangle\rangle$
\end{list}

\subsection{Example~6---Class Notation in Set Theory}\label{class}

A powerful device that makes set theory easier (and that we have
been using all along in our informal expository language) is {\em class
abstraction notation}.\index{class abstraction}\index{abstraction class}  The
definitions we introduce are rigorously justified
as conservative by Takeuti and Zaring \cite{Takeuti}\index{Takeuti, Gaisi} or
Quine \cite{Quine}\index{Quine, Willard Van Orman}.  The key idea is to
introduce the notation $\{x|\mbox{---}\}$ which means ``the class of all $x$
such that ---'' for abstraction classes and introduce (meta)variables that
range over them.  An abstraction class may or may not be a set, depending on
whether it exists (as a set).  A class that does not exist is
called a {\em proper class}.\index{proper class}\index{class!proper}

To illustrate the use of abstraction classes we will provide some examples
of definitions that make use of them:  the empty set, class union, and
unordered pair.  Many other such definitions can be found in the
Metamath set theory database,
\texttt{set.mm}.\index{set theory database (\texttt{set.mm})}

% We intentionally break up the sequence of math symbols here
% because otherwise the overlong line goes beyond the page in narrow mode.
We extend $\mbox{\em CN}$ of Example~5 with new symbols $\{$
$\mbox{class},$ $\{,$ $|,$ $\},$ $\varnothing,$ $\cup,$ $,$ $\}$
where the inner braces and last comma are
constant symbols. (As before,
our dual use of some mathematical symbols for both our expository
language and as primitives of the formal system should be clear from context.)

We extend $\mbox{\em VR}$ of Example~5 with a set of {\em class
variables}\index{class variable}
$\{A,B,C,\ldots\}$. We extend the $T$ of Example~5 with $\{\langle
\mbox{class\ } A\rangle, \langle \mbox{class\ }B\rangle, \langle \mbox{class\ }
C\rangle,\ldots\}$.

To
introduce our definitions,
we add to $\Gamma$ of Example~5 the axiomatic statements
that are the reducts of the following pre-statements:
\begin{list}{}{\itemsep 0.0pt}
      \item[] $\langle\varnothing,T,\varnothing,
               \langle \mbox{class\ }x\rangle\rangle$
      \item[] $\langle\varnothing,T,\varnothing,
               \langle \mbox{class\ }\{x|\varphi\}\rangle\rangle$
      \item[] $\langle\varnothing,T,\varnothing,
               \langle \mbox{wff\ }A=B\rangle\rangle$
      \item[] $\langle\varnothing,T,\varnothing,
               \langle \mbox{wff\ }A\in B\rangle\rangle$
      \item[Ab] $\langle\varnothing,T,\varnothing,
               \langle \vdash ( y \in \{ x |\varphi\} \leftrightarrow
                  ( ( x = y \to\varphi) \wedge \exists x ( x = y
                  \wedge\varphi) ))
               \rangle\rangle$
      \item[Eq] $\langle\{\{x,A\},\{x,B\}\},T,\varnothing,
               \langle \vdash ( A = B \leftrightarrow
               \forall x ( x \in A \leftrightarrow x \in B ) )
               \rangle\rangle$
      \item[El] $\langle\{\{x,A\},\{x,B\}\},T,\varnothing,
               \langle \vdash ( A \in B \leftrightarrow \exists x
               ( x = A \wedge x \in B ) )
               \rangle\rangle$
\end{list}
Here we say that an individual variable is a class; $\{x|\varphi\}$ is a
class; and we extend the definition of a wff to include class equality and
membership.  Axiom Ab defines membership of a variable in a class abstraction;
the right-hand side can be read as ``the wff that results from proper
substitution of $y$ for $x$ in $\varphi$.''\footnote{Note that this definition
makes unnecessary the introduction of a separate notation similar to
$\varphi(x|y)$ for proper substitution, although we may choose to do so to be
conventional.  Incidentally, $\varphi(x|y)$ as it stands would be ambiguous in
the formal systems of our examples, since we wouldn't know whether
$\lnot\varphi(x|y)$ meant $\lnot(\varphi(x|y))$ or $(\lnot\varphi)(x|y)$.
Instead, we would have to use an unambiguous variant such as $(\varphi\,
x|y)$.}  Axioms Eq and El extend the meaning of the existing equality and
membership connectives.  This is potentially dangerous and requires careful
justification.  For example, from Eq we can derive the Axiom of Extensionality
with predicate logic alone; thus in principle we should include the Axiom of
Extensionality as a logical hypothesis.  However we do not bother to do this
since we have already presupposed that axiom earlier. The distinct variable
restrictions should be read ``where $x$ does not occur in $A$ or $B$.''  We
typically do this when the right-hand side of a definition involves an
individual variable not in the expression being defined; it is done so that
the right-hand side remains independent of the particular ``dummy'' variable
we use.

We continue to add to $\Gamma$ the following definitions
(i.e. the reducts of the following pre-statements) for empty
set,\index{empty set} class union,\index{union} and unordered
pair.\index{unordered pair}  They should be self-explanatory.  Analogous to our
use of ``$\leftrightarrow$'' to define new wffs in Example~4, we use ``$=$''
to define new abstraction terms, and both may be read informally as ``is
defined as'' in this context.
\begin{list}{}{\itemsep 0.0pt}
      \item[] $\langle\varnothing,T,\varnothing,
               \langle \mbox{class\ }\varnothing\rangle\rangle$
      \item[] $\langle\varnothing,T,\varnothing,
               \langle \vdash \varnothing = \{ x | \lnot x = x \}
               \rangle\rangle$
      \item[] $\langle\varnothing,T,\varnothing,
               \langle \mbox{class\ }(A\cup B)\rangle\rangle$
      \item[] $\langle\{\{x,A\},\{x,B\}\},T,\varnothing,
               \langle \vdash ( A \cup B ) = \{ x | ( x \in A \vee x \in B ) \}
               \rangle\rangle$
      \item[] $\langle\varnothing,T,\varnothing,
               \langle \mbox{class\ }\{A,B\}\rangle\rangle$
      \item[] $\langle\{\{x,A\},\{x,B\}\},T,\varnothing,
               \langle \vdash \{ A , B \} = \{ x | ( x = A \vee x = B ) \}
               \rangle\rangle$
\end{list}

\section{Metamath as a Formal System}\label{theorymm}

This section presupposes a familiarity with the Metamath computer language.

Our theory describes formal systems and their universes.  The Metamath
language provides a way of representing these set-theoretical objects to
a computer.  A Metamath database, being a finite set of {\sc ascii}
characters, can usually describe only a subset of a formal system and
its universe, which are typically infinite.  However the database can
contain as large a finite subset of the formal system and its universe
as we wish.  (Of course a Metamath set theory database can, in
principle, indirectly describe an entire infinite formal system by
formalizing the expository language in this Appendix.)

For purpose of our discussion, we assume the Metamath database
is in the simple form described on p.~\pageref{framelist},
consisting of all constant and variable declarations at the beginning,
followed by a sequence of extended frames each
delimited by \texttt{\$\char`\{} and \texttt{\$\char`\}}.  Any Metamath database can
be converted to this form, as described on p.~\pageref{frameconvert}.

The math symbol tokens of a Metamath source file, which are declared
with \texttt{\$c} and \texttt{\$v} statements, are names we assign to
representatives of $\mbox{\em CN}$ and $\mbox{\em VR}$.  For
definiteness we could assume that the first math symbol declared as a
variable corresponds to $v_0$, the second to $v_1$, etc., although the
exact correspondence we choose is not important.

In the Metamath language, each \texttt{\$d}, \texttt{\$f}, and
 \texttt{\$e} source
statement in an extended frame (Section~\ref{frames})
corresponds respectively to a member of the
collections $D$, $T$, and $H$ in a formal system statement $\langle
D_M,T_M,H,A\rangle$.  The math symbol strings following these Metamath keywords
correspond to a variable pair (in the case of \texttt{\$d}) or an expression (for
the other two keywords). The math symbol string following a \texttt{\$a} source
statement corresponds to expression $A$ in an axiomatic statement of the
formal system; the one following a \texttt{\$p} source statement corresponds to
$A$ in a provable statement that is not axiomatic.  In other words, each
extended frame in a Metamath database corresponds to
a pre-statement of the formal system, and a frame corresponds to
a statement of the formal system.  (Don't confuse the two meanings of
``statement'' here.  A statement of the formal system corresponds to the
several statements in a Metamath database that may constitute a
frame.)

In order for the computer to verify that a formal system statement is
provable, each \texttt{\$p} source statement is accompanied by a proof.
However, the proof does not correspond to anything in the formal system
but is simply a way of communicating to the computer the information
needed for its verification.  The proof tells the computer {\em how to
construct} specific members of closure of the formal system
pre-statement corresponding to the extended frame of the \texttt{\$p}
statement.  The final result of the construction is the member of the
closure that matches the \texttt{\$p} statement.  The abstract formal
system, on the other hand, is concerned only with the {\em existence} of
members of the closure.

As mentioned on p.~\pageref{exampleref}, Examples 1 and 3--6 in the
previous Section parallel the development of logic and set theory in the
Metamath database
\texttt{set.mm}.\index{set theory database (\texttt{set.mm})} You may
find it instructive to compare them.


\chapter{The MIU System}
\label{MIU}
\index{formal system}
\index{MIU-system}

The following is a listing of the file \texttt{miu.mm}.  It is self-explanatory.

%%%%%%%%%%%%%%%%%%%%%%%%%%%%%%%%%%%%%%%%%%%%%%%%%%%%%%%%%%%%

\begin{verbatim}
$( The MIU-system:  A simple formal system $)

$( Note:  This formal system is unusual in that it allows
empty wffs.  To work with a proof, you must type
SET EMPTY_SUBSTITUTION ON before using the PROVE command.
By default, this is OFF in order to reduce the number of
ambiguous unification possibilities that have to be selected
during the construction of a proof.  $)

$(
Hofstadter's MIU-system is a simple example of a formal
system that illustrates some concepts of Metamath.  See
Douglas R. Hofstadter, _Goedel, Escher, Bach:  An Eternal
Golden Braid_ (Vintage Books, New York, 1979), pp. 33ff. for
a description of the MIU-system.

The system has 3 constant symbols, M, I, and U.  The sole
axiom of the system is MI. There are 4 rules:
     Rule I:  If you possess a string whose last letter is I,
     you can add on a U at the end.
     Rule II:  Suppose you have Mx.  Then you may add Mxx to
     your collection.
     Rule III:  If III occurs in one of the strings in your
     collection, you may make a new string with U in place
     of III.
     Rule IV:  If UU occurs inside one of your strings, you
     can drop it.
Unfortunately, Rules III and IV do not have unique results:
strings could have more than one occurrence of III or UU.
This requires that we introduce the concept of an "MIU
well-formed formula" or wff, which allows us to construct
unique symbol sequences to which Rules III and IV can be
applied.
$)

$( First, we declare the constant symbols of the language.
Note that we need two symbols to distinguish the assertion
that a sequence is a wff from the assertion that it is a
theorem; we have arbitrarily chosen "wff" and "|-". $)
      $c M I U |- wff $. $( Declare constants $)

$( Next, we declare some variables. $)
     $v x y $.

$( Throughout our theory, we shall assume that these
variables represent wffs. $)
 wx   $f wff x $.
 wy   $f wff y $.

$( Define MIU-wffs.  We allow the empty sequence to be a
wff. $)

$( The empty sequence is a wff. $)
 we   $a wff $.
$( "M" after any wff is a wff. $)
 wM   $a wff x M $.
$( "I" after any wff is a wff. $)
 wI   $a wff x I $.
$( "U" after any wff is a wff. $)
 wU   $a wff x U $.

$( Assert the axiom. $)
 ax   $a |- M I $.

$( Assert the rules. $)
 ${
   Ia   $e |- x I $.
$( Given any theorem ending with "I", it remains a theorem
if "U" is added after it.  (We distinguish the label I_
from the math symbol I to conform to the 24-Jun-2006
Metamath spec.) $)
   I_    $a |- x I U $.
 $}
 ${
IIa  $e |- M x $.
$( Given any theorem starting with "M", it remains a theorem
if the part after the "M" is added again after it. $)
   II   $a |- M x x $.
 $}
 ${
   IIIa $e |- x I I I y $.
$( Given any theorem with "III" in the middle, it remains a
theorem if the "III" is replaced with "U". $)
   III  $a |- x U y $.
 $}
 ${
   IVa  $e |- x U U y $.
$( Given any theorem with "UU" in the middle, it remains a
theorem if the "UU" is deleted. $)
   IV   $a |- x y $.
  $}

$( Now we prove the theorem MUIIU.  You may be interested in
comparing this proof with that of Hofstadter (pp. 35 - 36).
$)
 theorem1  $p |- M U I I U $=
      we wM wU wI we wI wU we wU wI wU we wM we wI wU we wM
      wI wI wI we wI wI we wI ax II II I_ III II IV $.
\end{verbatim}\index{well-formed formula (wff)}

The \texttt{show proof /lemmon/renumber} command
yields the following display.  It is very similar
to the one in \cite[pp.~35--36]{Hofstadter}.\index{Hofstadter, Douglas R.}

\begin{verbatim}
1 ax             $a |- M I
2 1 II           $a |- M I I
3 2 II           $a |- M I I I I
4 3 I_           $a |- M I I I I U
5 4 III          $a |- M U I U
6 5 II           $a |- M U I U U I U
7 6 IV           $a |- M U I I U
\end{verbatim}

We note that Hofstadter's ``MU-puzzle,'' which asks whether
MU is a theorem of the MIU-system, cannot be answered using
the system above because the MU-puzzle is a question {\em
about} the system.  To prove the answer to the MU-puzzle,
a much more elaborate system is needed, namely one that
models the MIU-system within set theory.  (Incidentally, the
answer to the MU-puzzle is no.)

\chapter{Metamath Language EBNF}%
\label{BNF}%
\index{Metamath Language EBNF}

The following is a formal description of the basic Metamath language syntax
(with compressed proofs and support for unknown proof steps).
It is defined using the
Extended Backus--Naur Form (EBNF)\index{Extended Backus--Naur Form}\index{EBNF}
notation from W3C\index{W3C}
\textit{Extensible Markup Language (XML) 1.0 (Fifth Edition)}
(W3C Recommendation 26 November 2008) at
\url{https://www.w3.org/TR/xml/#sec-notation}.

The \texttt{database}
rule is processed until the end of the file (\texttt{EOF}).
The rules eventually require reading whitespace-separated tokens.
A token has an upper-case definition (see below)
or is a string constant in a non-token (such as \texttt{'\$a'}).
We intend for this to be correct, but if there is a conflict the
rules of section \ref{spec} govern. That section also discusses
non-syntax restrictions not shown here
(e.g., that each new label token
defined in a \texttt{hypothesis-stmt} or \texttt{assert-stmt}
must be unique).

\begin{verbatim}
database ::= outermost-scope-stmt*

outermost-scope-stmt ::=
  include-stmt | constant-stmt | stmt

/* File inclusion command; process file as a database.
   Databases should NOT have a comment in the filename. */
include-stmt ::= '$[' filename '$]'

/* Constant symbols declaration. */
constant-stmt ::= '$c' constant+ '$.'

/* A normal statement can occur in any scope. */
stmt ::= block | variable-stmt | disjoint-stmt |
  hypothesis-stmt | assert-stmt

/* A block. You can have 0 statements in a block. */
block ::= '${' stmt* '$}'

/* Variable symbols declaration. */
variable-stmt ::= '$v' variable+ '$.'

/* Disjoint variables. Simple disjoint statements have
   2 variables, i.e., "variable*" is empty for them. */
disjoint-stmt ::= '$d' variable variable variable* '$.'

hypothesis-stmt ::= floating-stmt | essential-stmt

/* Floating (variable-type) hypothesis. */
floating-stmt ::= LABEL '$f' typecode variable '$.'

/* Essential (logical) hypothesis. */
essential-stmt ::= LABEL '$e' typecode MATH-SYMBOL* '$.'

assert-stmt ::= axiom-stmt | provable-stmt

/* Axiomatic assertion. */
axiom-stmt ::= LABEL '$a' typecode MATH-SYMBOL* '$.'

/* Provable assertion. */
provable-stmt ::= LABEL '$p' typecode MATH-SYMBOL*
  '$=' proof '$.'

/* A proof. Proofs may be interspersed by comments.
   If '?' is in a proof it's an "incomplete" proof. */
proof ::= uncompressed-proof | compressed-proof
uncompressed-proof ::= (LABEL | '?')+
compressed-proof ::= '(' LABEL* ')' COMPRESSED-PROOF-BLOCK+

typecode ::= constant

filename ::= MATH-SYMBOL /* No whitespace or '$' */
constant ::= MATH-SYMBOL
variable ::= MATH-SYMBOL
\end{verbatim}

\needspace{2\baselineskip}
A \texttt{frame} is a sequence of 0 or more
\texttt{disjoint-{\allowbreak}stmt} and
\texttt{hypotheses-{\allowbreak}stmt} statements
(possibly interleaved with other non-\texttt{assert-stmt} statements)
followed by one \texttt{assert-stmt}.

\needspace{3\baselineskip}
Here are the rules for lexical processing (tokenization) beyond
the constant tokens shown above.
By convention these tokenization rules have upper-case names.
Every token is read for the longest possible length.
Whitespace-separated tokens are read sequentially;
note that the separating whitespace and \texttt{\$(} ... \texttt{\$)}
comments are skipped.

If a token definition uses another token definition, the whole thing
is considered a single token.
A pattern that is only part of a full token has a name beginning
with an underscore (``\_'').
An implementation could tokenize many tokens as a
\texttt{PRINTABLE-SEQUENCE}
and then check if it meets the more specific rule shown here.

Comments do not nest, and both \texttt{\$(} and \texttt{\$)}
have to be surrounded
by at least one whitespace character (\texttt{\_WHITECHAR}).
Technically comments end without consuming the trailing
\texttt{\_WHITECHAR}, but the trailing
\texttt{\_WHITECHAR} gets ignored anyway so we ignore that detail here.
Metamath language processors
are not required to support \texttt{\$)} followed
immediately by a bare end-of-file, because the closing
comment symbol is supposed to be followed by a
\texttt{\_WHITECHAR} such as a newline.

\begin{verbatim}
PRINTABLE-SEQUENCE ::= _PRINTABLE-CHARACTER+

MATH-SYMBOL ::= (_PRINTABLE-CHARACTER - '$')+

/* ASCII non-whitespace printable characters */
_PRINTABLE-CHARACTER ::= [#x21-#x7e]

LABEL ::= ( _LETTER-OR-DIGIT | '.' | '-' | '_' )+

_LETTER-OR-DIGIT ::= [A-Za-z0-9]

COMPRESSED-PROOF-BLOCK ::= ([A-Z] | '?')+

/* Define whitespace between tokens. The -> SKIP
   means that when whitespace is seen, it is
   skipped and we simply read again. */
WHITESPACE ::= (_WHITECHAR+ | _COMMENT) -> SKIP

/* Comments. $( ... $) and do not nest. */
_COMMENT ::= '$(' (_WHITECHAR+ (PRINTABLE-SEQUENCE - '$)'))*
  _WHITECHAR+ '$)' _WHITECHAR

/* Whitespace: (' ' | '\t' | '\r' | '\n' | '\f') */
_WHITECHAR ::= [#x20#x09#x0d#x0a#x0c]
\end{verbatim}
% This EBNF was developed as a collaboration between
% David A. Wheeler\index{Wheeler, David A.},
% Mario Carneiro\index{Carneiro, Mario}, and
% Benoit Jubin\index{Jubin, Benoit}, inspired by a request
% (and a lot of initial work) by Benoit Jubin.
%
% \chapter{Disclaimer and Trademarks}
%
% Information in this document is subject to change without notice and does not
% represent a commitment on the part of Norman Megill.
% \vspace{2ex}
%
% \noindent Norman D. Megill makes no warranties, either express or implied,
% regarding the Metamath computer software package.
%
% \vspace{2ex}
%
% \noindent Any trademarks mentioned in this book are the property of
% their respective owners.  The name ``Metamath'' is a trademark of
% Norman Megill.
%
\cleardoublepage
\phantomsection  % fixes the link anchor
\addcontentsline{toc}{chapter}{\bibname}

\bibliography{metamath}
%\input{metamath.bbl}

\raggedright
\cleardoublepage
\phantomsection % fixes the link anchor
\addcontentsline{toc}{chapter}{\indexname}
%\printindex   ??
\input{metamath.ind}

\end{document}



\raggedright
\cleardoublepage
\phantomsection % fixes the link anchor
\addcontentsline{toc}{chapter}{\indexname}
%\printindex   ??
% metamath.tex - Version of 2-Jun-2019
% If you change the date above, also change the "Printed date" below.
% SPDX-License-Identifier: CC0-1.0
%
%                              PUBLIC DOMAIN
%
% This file (specifically, the version of this file with the above date)
% has been released into the Public Domain per the
% Creative Commons CC0 1.0 Universal (CC0 1.0) Public Domain Dedication
% https://creativecommons.org/publicdomain/zero/1.0/
%
% The public domain release applies worldwide.  In case this is not
% legally possible, the right is granted to use the work for any purpose,
% without any conditions, unless such conditions are required by law.
%
% Several short, attributed quotations from copyrighted works
% appear in this file under the ``fair use'' provision of Section 107 of
% the United States Copyright Act (Title 17 of the {\em United States
% Code}).  The public-domain status of this file is not applicable to
% those quotations.
%
% Norman Megill - email: nm(at)alum(dot)mit(dot)edu
%
% David A. Wheeler also donates his improvements to this file to the
% public domain per the CC0.  He works at the Institute for Defense Analyses
% (IDA), but IDA has agreed that this Metamath work is outside its "lane"
% and is not a work by IDA.  This was specifically confirmed by
% Margaret E. Myers (Division Director of the Information Technology
% and Systems Division) on 2019-05-24 and by Ben Lindorf (General Counsel)
% on 2019-05-22.

% This file, 'metamath.tex', is self-contained with everything needed to
% generate the the PDF file 'metamath.pdf' (the _Metamath_ book) on
% standard LaTeX 2e installations.  The auxiliary files are embedded with
% "filecontents" commands.  To generate metamath.pdf file, run these
% commands under Linux or Cygwin in the directory that contains
% 'metamath.tex':
%
%   rm -f realref.sty metamath.bib
%   touch metamath.ind
%   pdflatex metamath
%   pdflatex metamath
%   bibtex metamath
%   makeindex metamath
%   pdflatex metamath
%   pdflatex metamath
%
% The warnings that occur in the initial runs of pdflatex can be ignored.
% For the final run,
%
%   egrep -i 'error|warn' metamath.log
%
% should show exactly these 5 warnings:
%
%   LaTeX Warning: File `realref.sty' already exists on the system.
%   LaTeX Warning: File `metamath.bib' already exists on the system.
%   LaTeX Font Warning: Font shape `OMS/cmtt/m/n' undefined
%   LaTeX Font Warning: Font shape `OMS/cmtt/bx/n' undefined
%   LaTeX Font Warning: Some font shapes were not available, defaults
%       substituted.
%
% Search for "Uncomment" below if you want to suppress hyperlink boxes
% in the PDF output file
%
% TYPOGRAPHICAL NOTES:
% * It is customary to use an en dash (--) to "connect" names of different
%   people (and to denote ranges), and use a hyphen (-) for a
%   single compound name. Examples of connected multiple people are
%   Zermelo--Fraenkel, Schr\"{o}der--Bernstein, Tarski--Grothendieck,
%   Hewlett--Packard, and Backus--Naur.  Examples of a single person with
%   a compound name include Levi-Civita, Mittag-Leffler, and Burali-Forti.
% * Use non-breaking spaces after page abbreviations, e.g.,
%   p.~\pageref{note2002}.
%
% --------------------------- Start of realref.sty -----------------------------
\begin{filecontents}{realref.sty}
% Save the following as realref.sty.
% You can then use it with \usepackage{realref}
%
% This has \pageref jumping to the page on which the ref appears,
% \ref jumping to the point of the anchor, and \sectionref
% jumping to the start of section.
%
% Author:  Anthony Williams
%          Software Engineer
%          Nortel Networks Optical Components Ltd
% Date:    9 Nov 2001 (posted to comp.text.tex)
%
% The following declaration was made by Anthony Williams on
% 24 Jul 2006 (private email to Norman Megill):
%
%   ``I hereby donate the code for realref.sty posted on the
%   comp.text.tex newsgroup on 9th November 2001, accessible from
%   http://groups.google.com/group/comp.text.tex/msg/5a0e1cc13ea7fbb2
%   to the public domain.''
%
\ProvidesPackage{realref}
\RequirePackage[plainpages=false,pdfpagelabels=true]{hyperref}
\def\realref@anchorname{}
\AtBeginDocument{%
% ensure every label is a possible hyperlink target
\let\realref@oldrefstepcounter\refstepcounter%
\DeclareRobustCommand{\refstepcounter}[1]{\realref@oldrefstepcounter{#1}
\edef\realref@anchorname{\string #1.\@currentlabel}%
}%
\let\realref@oldlabel\label%
\DeclareRobustCommand{\label}[1]{\realref@oldlabel{#1}\hypertarget{#1}{}%
\@bsphack\protected@write\@auxout{}{%
    \string\expandafter\gdef\protect\csname
    page@num.#1\string\endcsname{\thepage}%
    \string\expandafter\gdef\protect\csname
    ref@num.#1\string\endcsname{\@currentlabel}%
    \string\expandafter\gdef\protect\csname
    sectionref@name.#1\string\endcsname{\realref@anchorname}%
}\@esphack}%
\DeclareRobustCommand\pageref[1]{{\edef\a{\csname
            page@num.#1\endcsname}\expandafter\hyperlink{page.\a}{\a}}}%
\DeclareRobustCommand\ref[1]{{\edef\a{\csname
            ref@num.#1\endcsname}\hyperlink{#1}{\a}}}%
\DeclareRobustCommand\sectionref[1]{{\edef\a{\csname
            ref@num.#1\endcsname}\edef\b{\csname
            sectionref@name.#1\endcsname}\hyperlink{\b}{\a}}}%
}
\end{filecontents}
% ---------------------------- End of realref.sty ------------------------------

% --------------------------- Start of metamath.bib -----------------------------
\begin{filecontents}{metamath.bib}
@book{Albers, editor = "Donald J. Albers and G. L. Alexanderson",
  title = "Mathematical People",
  publisher = "Contemporary Books, Inc.",
  address = "Chicago",
  note = "[QA28.M37]",
  year = 1985 }
@book{Anderson, author = "Alan Ross Anderson and Nuel D. Belnap",
  title = "Entailment",
  publisher = "Princeton University Press",
  address = "Princeton",
  volume = 1,
  note = "[QA9.A634 1975 v.1]",
  year = 1975}
@book{Barrow, author = "John D. Barrow",
  title = "Theories of Everything:  The Quest for Ultimate Explanation",
  publisher = "Oxford University Press",
  address = "Oxford",
  note = "[Q175.B225]",
  year = 1991 }
@book{Behnke,
  editor = "H. Behnke and F. Backmann and K. Fladt and W. S{\"{u}}ss",
  title = "Fundamentals of Mathematics",
  volume = "I",
  publisher = "The MIT Press",
  address = "Cambridge, Massachusetts",
  note = "[QA37.2.B413]",
  year = 1974 }
@book{Bell, author = "J. L. Bell and M. Machover",
  title = "A Course in Mathematical Logic",
  publisher = "North-Holland",
  address = "Amsterdam",
  note = "[QA9.B3953]",
  year = 1977 }
@inproceedings{Blass, author = "Andrea Blass",
  title = "The Interaction Between Category Theory and Set Theory",
  pages = "5--29",
  booktitle = "Mathematical Applications of Category Theory (Proceedings
     of the Special Session on Mathematical Applications
     Category Theory, 89th Annual Meeting of the American Mathematical
     Society, held in Denver, Colorado January 5--9, 1983)",
  editor = "John Walter Gray",
  year = 1983,
  note = "[QA169.A47 1983]",
  publisher = "American Mathematical Society",
  address = "Providence, Rhode Island"}
@proceedings{Bledsoe, editor = "W. W. Bledsoe and D. W. Loveland",
  title = "Automated Theorem Proving:  After 25 Years (Proceedings
     of the Special Session on Automatic Theorem Proving,
     89th Annual Meeting of the American Mathematical
     Society, held in Denver, Colorado January 5--9, 1983)",
  year = 1983,
  note = "[QA76.9.A96.S64 1983]",
  publisher = "American Mathematical Society",
  address = "Providence, Rhode Island" }
@book{Boolos, author = "George S. Boolos and Richard C. Jeffrey",
  title = "Computability and Log\-ic",
  publisher = "Cambridge University Press",
  edition = "third",
  address = "Cambridge",
  note = "[QA9.59.B66 1989]",
  year = 1989 }
@book{Campbell, author = "John Campbell",
  title = "Programmer's Progress",
  publisher = "White Star Software",
  address = "Box 51623, Palo Alto, CA 94303",
  year = 1991 }
@article{DBLP:journals/corr/Carneiro14,
  author    = {Mario Carneiro},
  title     = {Conversion of {HOL} Light proofs into Metamath},
  journal   = {CoRR},
  volume    = {abs/1412.8091},
  year      = {2014},
  url       = {http://arxiv.org/abs/1412.8091},
  archivePrefix = {arXiv},
  eprint    = {1412.8091},
  timestamp = {Mon, 13 Aug 2018 16:47:05 +0200},
  biburl    = {https://dblp.org/rec/bib/journals/corr/Carneiro14},
  bibsource = {dblp computer science bibliography, https://dblp.org}
}
@article{CarneiroND,
  author    = {Mario Carneiro},
  title     = {Natural Deductions in the Metamath Proof Language},
  url       = {http://us.metamath.org/ocat/natded.pdf},
  year      = 2014
}
@inproceedings{Chou, author = "Shang-Ching Chou",
  title = "Proving Elementary Geometry Theorems Using {W}u's Algorithm",
  pages = "243--286",
  booktitle = "Automated Theorem Proving:  After 25 Years (Proceedings
     of the Special Session on Automatic Theorem Proving,
     89th Annual Meeting of the American Mathematical
     Society, held in Denver, Colorado January 5--9, 1983)",
  editor = "W. W. Bledsoe and D. W. Loveland",
  year = 1983,
  note = "[QA76.9.A96.S64 1983]",
  publisher = "American Mathematical Society",
  address = "Providence, Rhode Island" }
@book{Clemente, author = "Daniel Clemente Laboreo",
  title = "Introduction to natural deduction",
  year = 2014,
  url = "http://www.danielclemente.com/logica/dn.en.pdf" }
@incollection{Courant, author = "Richard Courant and Herbert Robbins",
  title = "Topology",
  pages = "573--590",
  booktitle = "The World of Mathematics, Volume One",
  editor = "James R. Newman",
  publisher = "Simon and Schuster",
  address = "New York",
  note = "[QA3.W67 1988]",
  year = 1956 }
@book{Curry, author = "Haskell B. Curry",
  title = "Foundations of Mathematical Logic",
  publisher = "Dover Publications, Inc.",
  address = "New York",
  note = "[QA9.C976 1977]",
  year = 1977 }
@book{Davis, author = "Philip J. Davis and Reuben Hersh",
  title = "The Mathematical Experience",
  publisher = "Birkh{\"{a}}user Boston",
  address = "Boston",
  note = "[QA8.4.D37 1982]",
  year = 1981 }
@incollection{deMillo,
  author = "Richard de Millo and Richard Lipton and Alan Perlis",
  title = "Social Processes and Proofs of Theorems and Programs",
  pages = "267--285",
  booktitle = "New Directions in the Philosophy of Mathematics",
  editor = "Thomas Tymoczko",
  publisher = "Birkh{\"{a}}user Boston, Inc.",
  address = "Boston",
  note = "[QA8.6.N48 1986]",
  year = 1986 }
@book{Edwards, author = "Robert E. Edwards",
  title = "A Formal Background to Mathematics",
  publisher = "Springer-Verlag",
  address = "New York",
  note = "[QA37.2.E38 v.1a]",
  year = 1979 }
@book{Enderton, author = "Herbert B. Enderton",
  title = "Elements of Set Theory",
  publisher = "Academic Press, Inc.",
  address = "San Diego",
  note = "[QA248.E5]",
  year = 1977 }
@book{Goodstein, author = "R. L. Goodstein",
  title = "Development of Mathematical Logic",
  publisher = "Springer-Verlag New York Inc.",
  address = "New York",
  note = "[QA9.G6554]",
  year = 1971 }
@book{Guillen, author = "Michael Guillen",
  title = "Bridges to Infinity",
  publisher = "Jeremy P. Tarcher, Inc.",
  address = "Los Angeles",
  note = "[QA93.G8]",
  year = 1983 }
@book{Hamilton, author = "Alan G. Hamilton",
  title = "Logic for Mathematicians",
  edition = "revised",
  publisher = "Cambridge University Press",
  address = "Cambridge",
  note = "[QA9.H298]",
  year = 1988 }
@unpublished{Harrison, author = "John Robert Harrison",
  title = "Metatheory and Reflection in Theorem Proving:
    A Survey and Critique",
  note = "Technical Report
    CRC-053.
    SRI Cambridge,
    Millers Yard, Cambridge, UK,
    1995.
    Available on the Web as
{\verb+http:+}\-{\verb+//www.cl.cam.ac.uk/users/jrh/papers/reflect.html+}"}
@TECHREPORT{Harrison-thesis,
        author          = "John Robert Harrison",
        title           = "Theorem Proving with the Real Numbers",
        institution   = "University of Cambridge Computer
                         Lab\-o\-ra\-to\-ry",
        address         = "New Museums Site, Pembroke Street, Cambridge,
                           CB2 3QG, UK",
        year            = 1996,
        number          = 408,
        type            = "Technical Report",
        note            = "Author's PhD thesis,
   available on the Web at
{\verb+http:+}\-{\verb+//www.cl.cam.ac.uk+}\-{\verb+/users+}\-{\verb+/jrh+}%
\-{\verb+/papers+}\-{\verb+/thesis.html+}"}
@book{Herrlich, author = "Horst Herrlich and George E. Strecker",
  title = "Category Theory:  An Introduction",
  publisher = "Allyn and Bacon Inc.",
  address = "Boston",
  note = "[QA169.H567]",
  year = 1973 }
@article{Hindley, author = "J. Roger Hindley and David Meredith",
  title = "Principal Type-Schemes and Condensed Detachment",
  journal = "The Journal of Symbolic Logic",
  volume = 55,
  year = 1990,
  note = "[QA.J87]",
  pages = "90--105" }
@book{Hofstadter, author = "Douglas R. Hofstadter",
  title = "G{\"{o}}del, Escher, Bach",
  publisher = "Basic Books, Inc.",
  address = "New York",
  note = "[QA9.H63 1980]",
  year = 1979 }
@article{Indrzejczak, author= "Andrzej Indrzejczak",
  title = "Natural Deduction, Hybrid Systems and Modal Logic",
  journal = "Trends in Logic",
  volume = 30,
  publisher = "Springer",
  year = 2010 }
@article{Kalish, author = "D. Kalish and R. Montague",
  title = "On {T}arski's Formalization of Predicate Logic with Identity",
  journal = "Archiv f{\"{u}}r Mathematische Logik und Grundlagenfor\-schung",
  volume = 7,
  year = 1965,
  note = "[QA.A673]",
  pages = "81--101" }
@article{Kalman, author = "J. A. Kalman",
  title = "Condensed Detachment as a Rule of Inference",
  journal = "Studia Logica",
  volume = 42,
  number = 4,
  year = 1983,
  note = "[B18.P6.S933]",
  pages = "443-451" }
@book{Kline, author = "Morris Kline",
  title = "Mathematical Thought from Ancient to Modern Times",
  publisher = "Oxford University Press",
  address = "New York",
  note = "[QA21.K516 1990 v.3]",
  year = 1972 }
@book{Klinel, author = "Morris Kline",
  title = "Mathematics, The Loss of Certainty",
  publisher = "Oxford University Press",
  address = "New York",
  note = "[QA21.K525]",
  year = 1980 }
@book{Kramer, author = "Edna E. Kramer",
  title = "The Nature and Growth of Modern Mathematics",
  publisher = "Princeton University Press",
  address = "Princeton, New Jersey",
  note = "[QA93.K89 1981]",
  year = 1981 }
@article{Knill, author = "Oliver Knill",
  title = "Some Fundamental Theorems in Mathematics",
  year = "2018",
  url = "https://arxiv.org/abs/1807.08416" }
@book{Landau, author = "Edmund Landau",
  title = "Foundations of Analysis",
  publisher = "Chelsea Publishing Company",
  address = "New York",
  edition = "second",
  note = "[QA241.L2541 1960]",
  year = 1960 }
@article{Leblanc, author = "Hugues Leblanc",
  title = "On {M}eyer and {L}ambert's Quantificational Calculus {FQ}",
  journal = "The Journal of Symbolic Logic",
  volume = 33,
  year = 1968,
  note = "[QA.J87]",
  pages = "275--280" }
@article{Lejewski, author = "Czeslaw Lejewski",
  title = "On Implicational Definitions",
  journal = "Studia Logica",
  volume = 8,
  year = 1958,
  note = "[B18.P6.S933]",
  pages = "189--208" }
@book{Levy, author = "Azriel Levy",
  title = "Basic Set Theory",
  publisher = "Dover Publications",
  address = "Mineola, NY",
  year = "2002"
}
@book{Margaris, author = "Angelo Margaris",
  title = "First Order Mathematical Logic",
  publisher = "Blaisdell Publishing Company",
  address = "Waltham, Massachusetts",
  note = "[QA9.M327]",
  year = 1967}
@book{Manin, author = "Yu I. Manin",
  title = "A Course in Mathematical Logic",
  publisher = "Springer-Verlag",
  address = "New York",
  note = "[QA9.M29613]",
  year = "1977" }
@article{Mathias, author = "Adrian R. D. Mathias",
  title = "A Term of Length 4,523,659,424,929",
  journal = "Synthese",
  volume = 133,
  year = 2002,
  note = "[Q.S993]",
  pages = "75--86" }
@article{Megill, author = "Norman D. Megill",
  title = "A Finitely Axiomatized Formalization of Predicate Calculus
     with Equality",
  journal = "Notre Dame Journal of Formal Logic",
  volume = 36,
  year = 1995,
  note = "[QA.N914]",
  pages = "435--453" }
@unpublished{Megillc, author = "Norman D. Megill",
  title = "A Shorter Equivalent of the Axiom of Choice",
  month = "June",
  note = "Unpublished",
  year = 1991 }
@article{MegillBunder, author = "Norman D. Megill and Martin W.
    Bunder",
  title = "Weaker {D}-Complete Logics",
  journal = "Journal of the IGPL",
  volume = 4,
  year = 1996,
  pages = "215--225",
  note = "Available on the Web at
{\verb+http:+}\-{\verb+//www.mpi-sb.mpg.de+}\-{\verb+/igpl+}%
\-{\verb+/Journal+}\-{\verb+/V4-2+}\-{\verb+/#Megill+}"}
}
@book{Mendelson, author = "Elliott Mendelson",
  title = "Introduction to Mathematical Logic",
  edition = "second",
  publisher = "D. Van Nostrand Company, Inc.",
  address = "New York",
  note = "[QA9.M537 1979]",
  year = 1979 }
@article{Meredith, author = "David Meredith",
  title = "In Memoriam {C}arew {A}rthur {M}eredith (1904-1976)",
  journal = "Notre Dame Journal of Formal Logic",
  volume = 18,
  year = 1977,
  note = "[QA.N914]",
  pages = "513--516" }
@article{CAMeredith, author = "C. A. Meredith",
  title = "Single Axioms for the Systems ({C},{N}), ({C},{O}) and ({A},{N})
      of the Two-Valued Propositional Calculus",
  journal = "The Journal of Computing Systems",
  volume = 3,
  year = 1953,
  pages = "155--164" }
@article{Monk, author = "J. Donald Monk",
  title = "Provability With Finitely Many Variables",
  journal = "The Journal of Symbolic Logic",
  volume = 27,
  year = 1971,
  note = "[QA.J87]",
  pages = "353--358" }
@article{Monks, author = "J. Donald Monk",
  title = "Substitutionless Predicate Logic With Identity",
  journal = "Archiv f{\"{u}}r Mathematische Logik und Grundlagenfor\-schung",
  volume = 7,
  year = 1965,
  pages = "103--121" }
  %% Took out this from above to prevent LaTeX underfull warning:
  % note = "[QA.A673]",
@book{Moore, author = "A. W. Moore",
  title = "The Infinite",
  publisher = "Routledge",
  address = "New York",
  note = "[BD411.M59]",
  year = 1989}
@book{Munkres, author = "James R. Munkres",
  title = "Topology: A First Course",
  publisher = "Prentice-Hall, Inc.",
  address = "Englewood Cliffs, New Jersey",
  note = "[QA611.M82]",
  year = 1975}
@article{Nemesszeghy, author = "E. Z. Nemesszeghy and E. A. Nemesszeghy",
  title = "On Strongly Creative Definitions:  A Reply to {V}. {F}. {R}ickey",
  journal = "Logique et Analyse (N.\ S.)",
  year = 1977,
  volume = 20,
  note = "[BC.L832]",
  pages = "111--115" }
@unpublished{Nemeti, author = "N{\'{e}}meti, I.",
  title = "Algebraizations of Quantifier Logics, an Overview",
  note = "Version 11.4, preprint, Mathematical Institute, Budapest,
    1994.  A shortened version without proofs appeared in
    ``Algebraizations of quantifier logics, an introductory overview,''
   {\em Studia Logica}, 50:485--569, 1991 [B18.P6.S933]"}
@article{Pavicic, author = "M. Pavi{\v{c}}i{\'{c}}",
  title = "A New Axiomatization of Unified Quantum Logic",
  journal = "International Journal of Theoretical Physics",
  year = 1992,
  volume = 31,
  note = "[QC.I626]",
  pages = "1753 --1766" }
@book{Penrose, author = "Roger Penrose",
  title = "The Emperor's New Mind",
  publisher = "Oxford University Press",
  address = "New York",
  note = "[Q335.P415]",
  year = 1989 }
@book{PetersonI, author = "Ivars Peterson",
  title = "The Mathematical Tourist",
  publisher = "W. H. Freeman and Company",
  address = "New York",
  note = "[QA93.P475]",
  year = 1988 }
@article{Peterson, author = "Jeremy George Peterson",
  title = "An automatic theorem prover for substitution and detachment systems",
  journal = "Notre Dame Journal of Formal Logic",
  volume = 19,
  year = 1978,
  note = "[QA.N914]",
  pages = "119--122" }
@book{Quine, author = "Willard Van Orman Quine",
  title = "Set Theory and Its Logic",
  edition = "revised",
  publisher = "The Belknap Press of Harvard University Press",
  address = "Cambridge, Massachusetts",
  note = "[QA248.Q7 1969]",
  year = 1969 }
@article{Robinson, author = "J. A. Robinson",
  title = "A Machine-Oriented Logic Based on the Resolution Principle",
  journal = "Journal of the Association for Computing Machinery",
  year = 1965,
  volume = 12,
  pages = "23--41" }
@article{RobinsonT, author = "T. Thacher Robinson",
  title = "Independence of Two Nice Sets of Axioms for the Propositional
    Calculus",
  journal = "The Journal of Symbolic Logic",
  volume = 33,
  year = 1968,
  note = "[QA.J87]",
  pages = "265--270" }
@book{Rucker, author = "Rudy Rucker",
  title = "Infinity and the Mind:  The Science and Philosophy of the
    Infinite",
  publisher = "Bantam Books, Inc.",
  address = "New York",
  note = "[QA9.R79 1982]",
  year = 1982 }
@book{Russell, author = "Bertrand Russell",
  title = "Mysticism and Logic, and Other Essays",
  publisher = "Barnes \& Noble Books",
  address = "Totowa, New Jersey",
  note = "[B1649.R963.M9 1981]",
  year = 1981 }
@article{Russell2, author = "Bertrand Russell",
  title = "Recent Work on the Principles of Mathematics",
  journal = "International Monthly",
  volume = 4,
  year = 1901,
  pages = "84"}
@article{Schmidt, author = "Eric Schmidt",
  title = "Reductions in Norman Megill's axiom system for complex numbers",
  url = "http://us.metamath.org/downloads/schmidt-cnaxioms.pdf",
  year = "2012" }
@book{Shoenfield, author = "Joseph R. Shoenfield",
  title = "Mathematical Logic",
  publisher = "Addison-Wesley Publishing Company, Inc.",
  address = "Reading, Massachusetts",
  year = 1967,
  note = "[QA9.S52]" }
@book{Smullyan, author = "Raymond M. Smullyan",
  title = "Theory of Formal Systems",
  publisher = "Princeton University Press",
  address = "Princeton, New Jersey",
  year = 1961,
  note = "[QA248.5.S55]" }
@book{Solow, author = "Daniel Solow",
  title = "How to Read and Do Proofs:  An Introduction to Mathematical
    Thought Process",
  publisher = "John Wiley \& Sons",
  address = "New York",
  year = 1982,
  note = "[QA9.S577]" }
@book{Stark, author = "Harold M. Stark",
  title = "An Introduction to Number Theory",
  publisher = "Markham Publishing Company",
  address = "Chicago",
  note = "[QA241.S72 1978]",
  year = 1970 }
@article{Swart, author = "E. R. Swart",
  title = "The Philosophical Implications of the Four-Color Problem",
  journal = "American Mathematical Monthly",
  year = 1980,
  volume = 87,
  month = "November",
  note = "[QA.A5125]",
  pages = "697--707" }
@book{Szpiro, author = "George G. Szpiro",
  title = "Poincar{\'{e}}'s Prize: The Hundred-Year Quest to Solve One
    of Math's Greatest Puzzles",
  publisher = "Penguin Books Ltd",
  address = "London",
  note = "[QA43.S985 2007]",
  year = 2007}
@book{Takeuti, author = "Gaisi Takeuti and Wilson M. Zaring",
  title = "Introduction to Axiomatic Set Theory",
  edition = "second",
  publisher = "Springer-Verlag New York Inc.",
  address = "New York",
  note = "[QA248.T136 1982]",
  year = 1982}
@inproceedings{Tarski, author = "Alfred Tarski",
  title = "What is Elementary Geometry",
  pages = "16--29",
  booktitle = "The Axiomatic Method, with Special Reference to Geometry and
     Physics (Proceedings of an International Symposium held at the University
     of California, Berkeley, December 26, 1957 --- January 4, 1958)",
  editor = "Leon Henkin and Patrick Suppes and Alfred Tarski",
  year = 1959,
  publisher = "North-Holland Publishing Company",
  address = "Amsterdam"}
@article{Tarski1965, author = "Alfred Tarski",
  title = "A Simplified Formalization of Predicate Logic with Identity",
  journal = "Archiv f{\"{u}}r Mathematische Logik und Grundlagenforschung",
  volume = 7,
  year = 1965,
  note = "[QA.A673]",
  pages = "61--79" }
@book{Tymoczko,
  title = "New Directions in the Philosophy of Mathematics",
  editor = "Thomas Tymoczko",
  publisher = "Birkh{\"{a}}user Boston, Inc.",
  address = "Boston",
  note = "[QA8.6.N48 1986]",
  year = 1986 }
@incollection{Wang,
  author = "Hao Wang",
  title = "Theory and Practice in Mathematics",
  pages = "129--152",
  booktitle = "New Directions in the Philosophy of Mathematics",
  editor = "Thomas Tymoczko",
  publisher = "Birkh{\"{a}}user Boston, Inc.",
  address = "Boston",
  note = "[QA8.6.N48 1986]",
  year = 1986 }
@manual{Webster,
  title = "Webster's New Collegiate Dictionary",
  organization = "G. \& C. Merriam Co.",
  address = "Springfield, Massachusetts",
  note = "[PE1628.W4M4 1977]",
  year = 1977 }
@manual{Whitehead, author = "Alfred North Whitehead",
  title = "An Introduction to Mathematics",
  year = 1911 }
@book{PM, author = "Alfred North Whitehead and Bertrand Russell",
  title = "Principia Mathematica",
  edition = "second",
  publisher = "Cambridge University Press",
  address = "Cambridge",
  year = "1927",
  note = "(3 vols.) [QA9.W592 1927]" }
@article{DBLP:journals/corr/Whalen16,
  author    = {Daniel Whalen},
  title     = {Holophrasm: a neural Automated Theorem Prover for higher-order logic},
  journal   = {CoRR},
  volume    = {abs/1608.02644},
  year      = {2016},
  url       = {http://arxiv.org/abs/1608.02644},
  archivePrefix = {arXiv},
  eprint    = {1608.02644},
  timestamp = {Mon, 13 Aug 2018 16:46:19 +0200},
  biburl    = {https://dblp.org/rec/bib/journals/corr/Whalen16},
  bibsource = {dblp computer science bibliography, https://dblp.org} }
@article{Wiedijk-revisited,
  author = {Freek Wiedijk},
  title = {The QED Manifesto Revisited},
  year = {2007},
  url = {http://mizar.org/trybulec65/8.pdf} }
@book{Wolfram,
  author = "Stephen Wolfram",
  title = "Mathematica:  A System for Doing Mathematics by Computer",
  edition = "second",
  publisher = "Addison-Wesley Publishing Co.",
  address = "Redwood City, California",
  note = "[QA76.95.W65 1991]",
  year = 1991 }
@book{Wos, author = "Larry Wos and Ross Overbeek and Ewing Lusk and Jim Boyle",
  title = "Automated Reasoning:  Introduction and Applications",
  edition = "second",
  publisher = "McGraw-Hill, Inc.",
  address = "New York",
  note = "[QA76.9.A96.A93 1992]",
  year = 1992 }

%
%
%[1] Church, Alonzo, Introduction to Mathematical Logic,
% Volume 1, Princeton University Press, Princeton, N. J., 1956.
%
%[2] Cohen, Paul J., Set Theory and the Continuum Hypothesis,
% W. A. Benjamin, Inc., Reading, Mass., 1966.
%
%[3] Hamilton, Alan G., Logic for Mathematicians, Cambridge
% University Press,
% Cambridge, 1988.

%[6] Kleene, Stephen Cole, Introduction to Metamathematics, D.  Van
% Nostrand Company, Inc., Princeton (1952).

%[13] Tarski, Alfred, "A simplified formalization of predicate
% logic with identity," Archiv fur Mathematische Logik und
% Grundlagenforschung, vol. 7 (1965), pp. 61-79.

%[14] Tarski, Alfred and Steven Givant, A Formalization of Set
% Theory Without Variables, American Mathematical Society Colloquium
% Publications, vol. 41, American Mathematical Society,
% Providence, R. I., 1987.

%[15] Zeman, J. J., Modal Logic, Oxford University Press, Oxford, 1973.
\end{filecontents}
% --------------------------- End of metamath.bib -----------------------------


%Book: Metamath
%Author:  Norman Megill Email:  nm at alum.mit.edu
%Author:  David A. Wheeler Email:  dwheeler at dwheeler.com

% A book template example
% http://www.stsci.edu/ftp/software/tex/bookstuff/book.template

\documentclass[leqno]{book} % LaTeX 2e. 10pt. Use [leqno,12pt] for 12pt
% hyperref 2002/05/27 v6.72r  (couldn't get pagebackref to work)
\usepackage[plainpages=false,pdfpagelabels=true]{hyperref}

\usepackage{needspace}     % Enable control over page breaks
\usepackage{breqn}         % automatic equation breaking
\usepackage{microtype}     % microtypography, reduces hyphenation

% Packages for flexible tables.  We need to be able to
% wrap text within a cell (with automatically-determined widths) AND
% split a table automatically across multiple pages.
% * "tabularx" wraps text in cells but only 1 page
% * "longtable" goes across pages but by itself is incompatible with tabularx
% * "ltxtable" combines longtable and tabularx, but table contents
%    must be in a separate file.
% * "ltablex" combines tabularx and longtable - must install specially
% * "booktabs" is recommended as a way to improve the look of tables,
%   but doesn't add these capabilities.
% * "tabu" much more capable and seems to be recommended. So use that.

\usepackage{makecell}      % Enable forced line splits within a table cell
% v4.13 needed for tabu: https://tex.stackexchange.com/questions/600724/dimension-too-large-after-recent-longtable-update
\usepackage{longtable}[=v4.13] % Enable multi-page tables  
\usepackage{tabu}          % Multi-page tables with wrapped text in a cell

% You can find more Tex packages using commands like:
% tlmgr search --file tabu.sty
% find /usr/share/texmf-dist/ -name '*tab*'
%
%%%%%%%%%%%%%%%%%%%%%%%%%%%%%%%%%%%%%%%%%%%%%%%%%%%%%%%%%%%%%%%%%%%%%%%%%%%%
% Uncomment the next 3 lines to suppress boxes and colors on the hyperlinks
%%%%%%%%%%%%%%%%%%%%%%%%%%%%%%%%%%%%%%%%%%%%%%%%%%%%%%%%%%%%%%%%%%%%%%%%%%%%
%\hypersetup{
%colorlinks,citecolor=black,filecolor=black,linkcolor=black,urlcolor=black
%}
%
\usepackage{realref}

% Restarting page numbers: try?
%   \printglossary
%   \cleardoublepage
%   \pagenumbering{arabic}
%   \setcounter{page}{1}    ???needed
%   \include{chap1}

% not used:
% \def\R2Lurl#1#2{\mbox{\href{#1}\texttt{#2}}}

\usepackage{amssymb}

% Version 1 of book: margins: t=.4, b=.2, ll=.4, rr=.55
% \usepackage{anysize}
% % \papersize{<height>}{<width>}
% % \marginsize{<left>}{<right>}{<top>}{<bottom>}
% \papersize{9in}{6in}
% % l/r 0.6124-0.6170 works t/b 0.2418-0.3411 = 192pp. 0.2926-03118=exact
% \marginsize{0.7147in}{0.5147in}{0.4012in}{0.2012in}

\usepackage{anysize}
% \papersize{<height>}{<width>}
% \marginsize{<left>}{<right>}{<top>}{<bottom>}
\papersize{9in}{6in}
% l/r 0.85in&0.6431-0.6539 works t/b ?-?
%\marginsize{0.85in}{0.6485in}{0.55in}{0.35in}
\marginsize{0.8in}{0.65in}{0.5in}{0.3in}

% \usepackage[papersize={3.6in,4.8in},hmargin=0.1in,vmargin={0.1in,0.1in}]{geometry}  % page geometry
\usepackage{special-settings}

\raggedbottom
\makeindex

\begin{document}
% Discourage page widows and orphans:
\clubpenalty=300
\widowpenalty=300

%%%%%%% load in AMS fonts %%%%%%% % LaTeX 2.09 - obsolete in LaTeX 2e
%\input{amssym.def}
%\input{amssym.tex}
%\input{c:/texmf/tex/plain/amsfonts/amssym.def}
%\input{c:/texmf/tex/plain/amsfonts/amssym.tex}

\bibliographystyle{plain}
\pagenumbering{roman}
\pagestyle{headings}

\thispagestyle{empty}

\hfill
\vfill

\begin{center}
{\LARGE\bf Metamath} \\
\vspace{1ex}
{\large A Computer Language for Mathematical Proofs} \\
\vspace{7ex}
{\large Norman Megill} \\
\vspace{7ex}
with extensive revisions by \\
\vspace{1ex}
{\large David A. Wheeler} \\
\vspace{7ex}
% Printed date. If changing the date below, also fix the date at the beginning.
2019-06-02
\end{center}

\vfill
\hfill

\newpage
\thispagestyle{empty}

\hfill
\vfill

\begin{center}
$\sim$\ {\sc Public Domain}\ $\sim$

\vspace{2ex}
This book (including its later revisions)
has been released into the Public Domain by Norman Megill per the
Creative Commons CC0 1.0 Universal (CC0 1.0) Public Domain Dedication.
David A. Wheeler has done the same.
This public domain release applies worldwide.  In case this is not
legally possible, the right is granted to use the work for any purpose,
without any conditions, unless such conditions are required by law.
See \url{https://creativecommons.org/publicdomain/zero/1.0/}.

\vspace{3ex}
Several short, attributed quotations from copyrighted works
appear in this book under the ``fair use'' provision of Section 107 of
the United States Copyright Act (Title 17 of the {\em United States
Code}).  The public-domain status of this book is not applicable to
those quotations.

\vspace{3ex}
Any trademarks used in this book are the property of their owners.

% QA76.9.L63.M??

% \vspace{1ex}
%
% \vspace{1ex}
% {\small Permission is granted to make and distribute verbatim copies of this
% book
% provided the copyright notice and this
% permission notice are preserved on all copies.}
%
% \vspace{1ex}
% {\small Permission is granted to copy and distribute modified versions of this
% book under the conditions for verbatim copying, provided that the
% entire
% resulting derived work is distributed under the terms of a permission
% notice
% identical to this one.}
%
% \vspace{1ex}
% {\small Permission is granted to copy and distribute translations of this
% book into another language, under the above conditions for modified
% versions,
% except that this permission notice may be stated in a translation
% approved by the
% author.}
%
% \vspace{1ex}
% %{\small   For a copy of the \LaTeX\ source files for this book, contact
% %the author.} \\
% \ \\
% \ \\

\vspace{7ex}
% ISBN: 1-4116-3724-0 \\
% ISBN: 978-1-4116-3724-5 \\
ISBN: 978-0-359-70223-7 \\
{\ } \\
Lulu Press \\
Morrisville, North Carolina\\
USA


\hfill
\vfill

Norman Megill\\ 93 Bridge St., Lexington, MA 02421 \\
E-mail address: \texttt{nm{\char`\@}alum.mit.edu} \\
\vspace{7ex}
David A. Wheeler \\
E-mail address: \texttt{dwheeler{\char`\@}dwheeler.com} \\
% See notes added at end of Preface for revision history. \\
% For current information on the Metamath software see \\
\vspace{7ex}
\url{http://metamath.org}
\end{center}

\hfill
\vfill

{\parindent0pt%
\footnotesize{%
Cover: Aleph null ($\aleph_0$) is the symbol for the
first infinite cardinal number, discovered by Georg Cantor in 1873.
We use a red aleph null (with dark outline and gold glow) as the Metamath logo.
Credit: Norman Megill (1994) and Giovanni Mascellani (2019),
public domain.%
\index{aleph null}%
\index{Metamath!logo}\index{Cantor, Georg}\index{Mascellani, Giovanni}}}

% \newpage
% \thispagestyle{empty}
%
% \hfill
% \vfill
%
% \begin{center}
% {\it To my son Robin Dwight Megill}
% \end{center}
%
% \vfill
% \hfill
%
% \newpage

\tableofcontents
%\listoftables

\chapter*{Preface}
\markboth{PREFACE}{PREFACE}
\addcontentsline{toc}{section}{Preface}


% (For current information, see the notes added at the
% end of this preface on p.~\pageref{note2002}.)

\subsubsection{Overview}

Metamath\index{Metamath} is a computer language and an associated computer
program for archiving, verifying, and studying mathematical proofs at a very
detailed level.  The Metamath language incorporates no mathematics per se but
treats all mathematical statements as mere sequences of symbols.  You provide
Metamath with certain special sequences (axioms) that tell it what rules
of inference are allowed.  Metamath is not limited to any specific field of
mathematics.  The Metamath language is simple and robust, with an
almost total absence of hard-wired syntax, and
we\footnote{Unless otherwise noted, the words
``I,'' ``me,'' and ``my'' refer to Norman Megill\index{Megill, Norman}, while
``we,'' ``us,'' and ``our'' refer to Norman Megill and
David A. Wheeler\index{Wheeler, David A.}.}
believe that it
provides about the simplest possible framework that allows essentially all of
mathematics to be expressed with absolute rigor.

% index test
%\newcommand{\nn}[1]{#1n}
%\index{aaa@bbb}
%\index{abc!def}
%\index{abd|see{qqq}}
%\index{abe|nn}
%\index{abf|emph}
%\index{abg|(}
%\index{abg|)}

Using the Metamath language, you can build formal or mathematical
systems\index{formal system}\footnote{A formal or mathematical system consists
of a collection of symbols (such as $2$, $4$, $+$ and $=$), syntax rules that
describe how symbols may be combined to form a legal expression (called a
well-formed formula or {\em wff}, pronounced ``whiff''), some starting wffs
called axioms, and inference rules that describe how theorems may be derived
(proved) from the axioms.  A theorem is a mathematical fact such as $2+2=4$.
Strictly speaking, even an obvious fact such as this must be proved from
axioms to be formally acceptable to a mathematician.}\index{theorem}
\index{axiom}\index{rule}\index{well-formed formula (wff)} that involve
inferences from axioms.  Although a database is provided
that includes a recommended set of axioms for standard mathematics, if you
wish you can supply your own symbols, syntax, axioms, rules, and definitions.

The name ``Metamath'' was chosen to suggest that the language provides a
means for {\em describing} mathematics rather than {\em being} the
mathematics itself.  Actually in some sense any mathematical language is
metamathematical.  Symbols written on paper, or stored in a computer,
are not mathematics itself but rather a way of expressing mathematics.
For example ``7'' and ``VII'' are symbols for denoting the number seven
in Arabic and Roman numerals; neither {\em is} the number seven.

If you are able to understand and write computer programs, you should be able
to follow abstract mathematics with the aid of Metamath.  Used in conjunction
with standard textbooks, Metamath can guide you step-by-step towards an
understanding of abstract mathematics from a very rigorous viewpoint, even if
you have no formal abstract mathematics background.  By using a single,
consistent notation to express proofs, once you grasp its basic concepts
Metamath provides you with the ability to immediately follow and dissect
proofs even in totally unfamiliar areas.

Of course, just being able follow a proof will not necessarily give you an
intuitive familiarity with mathematics.  Memorizing the rules of chess does not
give you the ability to appreciate the game of a master, and knowing how the
notes on a musical score map to piano keys does not give you the ability to
hear in your head how it would sound.  But each of these can be a first step.

Metamath allows you to explore proofs in the sense that you can see the
theorem referenced at any step expanded in as much detail as you want, right
down to the underlying axioms of logic and set theory (in the case of the set
theory database provided).  While Metamath will not replace the higher-level
understanding that can only be acquired through exercises and hard work, being
able to see how gaps in a proof are filled in can give you increased
confidence that can speed up the learning process and save you time when you
get stuck.

The Metamath language breaks down a mathematical proof into its tiniest
possible parts.  These can be pieced together, like interlocking
pieces in a puzzle, only in a way that produces correct and absolutely rigorous
mathematics.

The nature of Metamath\index{Metamath} enforces very precise mathematical
thinking, similar to that involved in writing a computer program.  A crucial
difference, though, is that once a proof is verified (by the Metamath program)
to be correct, it is definitely correct; it can never have a hidden
``bug.''\index{computer program bugs}  After getting used to the kind of rigor
and accuracy provided by Metamath, you might even be tempted to
adopt the attitude that a proof should never be considered correct until it
has been verified by a computer, just as you would not completely trust a
manual calculation until you have verified it on a
calculator.

My goal
for Metamath was a system for describing and verifying
mathematics that is completely universal yet conceptually as simple as
possible.  In approaching mathematics from an axiomatic, formal viewpoint, I
wanted Metamath to be able to handle almost any mathematical system, not
necessarily with ease, but at least in principle and hopefully in practice. I
wanted it to verify proofs with absolute rigor, and for this reason Metamath
is what might be thought of as a ``compile-only'' language rather than an
algorithmic or Turing-machine language (Pascal, C, Prolog, Mathematica,
etc.).  In other words, a database written in the Metamath
language doesn't ``do'' anything; it merely exhibits mathematical knowledge
and permits this knowledge to be verified as being correct.  A program in an
algorithmic language can potentially have hidden bugs\index{computer program
bugs} as well as possibly being hard to understand.  But each token in a
Metamath database must be consistent with the database's earlier
contents according to simple, fixed rules.
If a database is verified
to be correct,\footnote{This includes
verification that a sequential list of proof steps results in the specified
theorem.} then the mathematical content is correct if the
verifier is correct and the axioms are correct.
The verification program could be incorrect, but the verification algorithm
is relatively simple (making it unlikely to be implemented incorrectly
by the Metamath program),
and there are over a dozen Metamath database verifiers
written by different people in different programming languages
(so these different verifiers can act as multiple reviewers of a database).
The most-used Metamath database, the Metamath Proof Explorer
(aka \texttt{set.mm}\index{set theory database (\texttt{set.mm})}%
\index{Metamath Proof Explorer}),
is currently verified by four different Metamath verifiers written by
four different people in four different languages, including the
original Metamath program described in this book.
The only ``bugs'' that can exist are in the statement of the axioms,
for example if the axioms are inconsistent (a famous problem shown to be
unsolvable by G\"{o}del's incompleteness theorem\index{G\"{o}del's
incompleteness theorem}).
However, real mathematical systems have very few axioms, and these can
be carefully studied.
All of this provides extraordinarily high confidence that the verified database
is in fact correct.

The Metamath program
doesn't prove theorems automatically but is designed to verify proofs
that you supply to it.
The underlying Metamath language is completely general and has no built-in,
preconceived notions about your formal system\index{formal system}, its logic
or its syntax.
For constructing proofs, the Metamath program has a Proof Assistant\index{Proof
Assistant} which helps you fill in some of a proof step's details, shows you
what choices you have at any step, and verifies the proof as you build it; but
you are still expected to provide the proof.

There are many other programs that can process or generate information
in the Metamath language, and more continue to be written.
This is in part because the Metamath language itself is very simple
and intentionally easy to automatically process.
Some programs, such as \texttt{mmj2}\index{mmj2}, include a proof assistant
that can automate some steps beyond what the Metamath program can do.
Mario Carneiro has developed an algorithm for converting proofs from
the OpenTheory interchange format, which can be translated to and from
any of the HOL family of proof languages (HOL4, HOL Light, ProofPower,
and Isabelle), into the
Metamath language \cite{DBLP:journals/corr/Carneiro14}\index{Carneiro, Mario}.
Daniel Whalen has developed Holophrasm, which can automatically
prove many Metamath proofs using
machine learning\index{machine learning}\index{artificial intelligence}
approaches
(including multiple neural networks\index{neural networks})\cite{DBLP:journals/corr/Whalen16}\index{Whalen, Daniel}.
However,
a discussion of these other programs is beyond the scope of this book.

Like most computer languages, the Metamath\index{Metamath} language uses the
standard ({\sc ascii}) characters on a computer keyboard, so it cannot
directly represent many of the special symbols that mathematicians use.  A
useful feature of the Metamath program is its ability to convert its notation
into the \LaTeX\ typesetting language.\index{latex@{\LaTeX}}  This feature
lets you convert the {\sc ascii} tokens you've defined into standard
mathematical symbols, so you end up with symbols and formulas you are familiar
with instead of somewhat cryptic {\sc ascii} representations of them.
The Metamath program can also generate HTML\index{HTML}, making it easy
to view results on the web and to see related information by using
hypertext links.

Metamath is probably conceptually different from anything you've seen
before and some aspects may take some getting used to.  This book will
help you decide whether Metamath suits your specific needs.

\subsubsection{Setting Your Expectations}
It is important for you to understand what Metamath\index{Metamath} is and is
not.  As mentioned, the Metamath program
is {\em not} an automated theorem prover but
rather a proof verifier.  Developing a database can be tedious, hard work,
especially if you want to make the proofs as short as possible, but it becomes
easier as you build up a collection of useful theorems.  The purpose of
Metamath is simply to document existing mathematics in an absolutely rigorous,
computer-verifiable way, not to aid directly in the creation of new
mathematics.  It also is not a magic solution for learning abstract
mathematics, although it may be helpful to be able to actually see the implied
rigor behind what you are learning from textbooks, as well as providing hints
to work out proofs that you are stumped on.

As of this writing, a sizable set theory database has been developed to
provide a foundation for many fields of mathematics, but much more work would
be required to develop useful databases for specific fields.

Metamath\index{Metamath} ``knows no math;'' it just provides a framework in
which to express mathematics.  Its language is very small.  You can define two
kinds of symbols, constants\index{constant} and variables\index{variable}.
The only thing Metamath knows how to do is to substitute strings of symbols
for the variables\index{substitution!variable}\index{variable substitution} in
an expression based on instructions you provide it in a proof, subject to
certain constraints you specify for the variables.  Even the decimal
representation of a number is merely a string of certain constants (digits)
which together, in a specific context, correspond to whatever mathematical
object you choose to define for it; unlike other computer languages, there is
no actual number stored inside the computer.  In a proof, you in effect
instruct Metamath what symbol substitutions to make in previous axioms or
theorems and join a sequence of them together to result in the desired
theorem.  This kind of symbol manipulation captures the essence of mathematics
at a preaxiomatic level.

\subsubsection{Metamath and Mathematical Literature}

In advanced mathematical literature, proofs are usually presented in the form
of short outlines that often only an expert can follow.  This is partly out of
a desire for brevity, but it would also be unwise (even if it were practical)
to present proofs in complete formal detail, since the overall picture would
be lost.\index{formal proof}

A solution I envision\label{envision} that would allow mathematics to remain
acceptable to the expert, yet increase its accessibility to non-specialists,
consists of a combination of the traditional short, informal proof in print
accompanied by a complete formal proof stored in a computer database.  In an
analogy with a computer program, the informal proof is like a set of comments
that describe the overall reasoning and content of the proof, whereas the
computer database is like the actual program and provides a means for anyone,
even a non-expert, to follow the proof in as much detail as desired, exploring
it back through layers of theorems (like subroutines that call other
subroutines) all the way back to the axioms of the theory.  In addition, the
computer database would have the advantage of providing absolute assurance
that the proof is correct, since each step can be verified automatically.

There are several other approaches besides Metamath to a project such
as this.  Section~\ref{proofverifiers} discusses some of these.

To us, a noble goal would be a database with hundreds of thousands of
theorems and their computer-verifiable proofs, encompassing a significant
fraction of known mathematics and available for instant access.
These would be fully verified by multiple independently-implemented verifiers,
to provide extremely high confidence that the proofs are completely correct.
The database would allow people to investigate whatever details they were
interested in, so that they could confirm whatever portions they wished.
Whether or not Metamath is an appropriate choice remains to be seen, but in
principle we believe it is sufficient.

\subsubsection{Formalism}

Over the past fifty years, a group of French mathematicians working
collectively under the pseudonym of Bourbaki\index{Bourbaki, Nicolas} have
co-authored a series of monographs that attempt to rigorously and
consistently formalize large bodies of mathematics from foundations.  On the
one hand, certainly such an effort has its merits; on the other hand, the
Bourbaki project has been criticized for its ``scholasticism'' and
``hyperaxiomatics'' that hide the intuitive steps that lead to the results
\cite[p.~191]{Barrow}\index{Barrow, John D.}.

Metamath unabashedly carries this philosophy to its extreme and no doubt is
subject to the same kind of criticism.  Nonetheless I think that in
conjunction with conventional approaches to mathematics Metamath can serve a
useful purpose.  The Bourbaki approach is essentially pedagogic, requiring the
reader to become intimately familiar with each detail in a very large
hierarchy before he or she can proceed to the next step.  The difference with
Metamath is that the ``reader'' (user) knows that all details are contained in
its computer database, available as needed; it does not demand that the user
know everything but conveniently makes available those portions that are of
interest.  As the body of all mathematical knowledge grows larger and larger,
no one individual can have a thorough grasp of its entirety.  Metamath
can finalize and put to rest any questions about the validity of any part of it
and can make any part of it accessible, in principle, to a non-specialist.

\subsubsection{A Personal Note}
Why did I develop Metamath\index{Metamath}?  I enjoy abstract mathematics, but
I sometimes get lost in a barrage of definitions and start to lose confidence
that my proofs are correct.  Or I reach a point where I lose sight of how
anything I'm doing relates to the axioms that a theory is based on and am
sometimes suspicious that there may be some overlooked implicit axiom
accidentally introduced along the way (as happened historically with Euclidean
geometry\index{Euclidean geometry}, whose omission of Pasch's
axiom\index{Pasch's axiom} went unnoticed for 2000 years
\cite[p.~160]{Davis}!). I'm also somewhat lazy and wish to avoid the effort
involved in re-verifying the gaps in informal proofs ``left to the reader;'' I
prefer to figure them out just once and not have to go through the same
frustration a year from now when I've forgotten what I did.  Metamath provides
better recovery of my efforts than scraps of paper that I can't
decipher anymore.  But mostly I find very appealing the idea of rigorously
archiving mathematical knowledge in a computer database, providing precision,
certainty, and elimination of human error.

\subsubsection{Note on Bibliography and Index}

The Bibliography usually includes the Library of Congress classification
for a work to make it easier for you to find it in on a university
library shelf.  The Index has author references to pages where their works
are cited, even though the authors' names may not appear on those pages.

\subsubsection{Acknowledgments}

Acknowledgments are first due to my wife, Deborah (who passed away on
September 4, 1998), for critiquing the manu\-script but most of all for
her patience and support.  I also wish to thank Joe Wright, Richard
Becker, Clarke Evans, Buddha Buck, and Jeremy Henty for helpful
comments.  Any errors, omissions, and other shortcomings are of course
my responsibility.

\subsubsection{Note Added June 22, 2005}\label{note2002}

The original, unpublished version of this book was written in 1997 and
distributed via the web.  The present edition has been updated to
reflect the current Metamath program and databases, as well as more
current {\sc url}s for Internet sites.  Thanks to Josh
Purinton\index{Purinton, Josh}, One Hand
Clapping, Mel L.\ O'Cat, and Roy F. Longton for pointing out
typographical and other errors.  I have also benefitted from numerous
discussions with Raph Levien\index{Levien, Raph}, who has extended
Metamath's philosophy of rigor to result in his {\em
Ghilbert}\index{Ghilbert} proof language (\url{http://ghilbert.org}).

Robert (Bob) Solovay\index{Solovay, Robert} communicated a new result of
A.~R.~D.~Mathias on the system of Bourbaki, and the text has been
updated accordingly (p.~\pageref{bourbaki}).

Bob also pointed out a clarification of the literature regarding
category theory and inaccessible cardinals\index{category
theory}\index{cardinal, inaccessible} (p.~\pageref{categoryth}),
and a misleading statement was removed from the text.  Specifically,
contrary to a statement in previous editions, it is possible to express
``There is a proper class of inaccessible cardinals'' in the language of
ZFC.  This can be done as follows:  ``For every set $x$ there is an
inaccessible cardinal $\kappa$ such that $\kappa$ is not in $x$.''
Bob writes:\footnote{Private communication, Nov.~30, 2002.}
\begin{quotation}
     This axiom is how Grothendieck presents category theory.  To each
inaccessible cardinal $\kappa$ one associates a Grothendieck universe
\index{Grothendieck, Alexander} $U(\kappa)$.  $U(\kappa)$ consists of
those sets which lie in a transitive set of cardinality less than
$\kappa$.  Instead of the ``category of all groups,'' one works relative
to a universe [considering the category of groups of cardinality less
than $\kappa$].  Now the category whose objects are all categories
``relative to the universe $U(\kappa)$'' will be a category not
relative to this universe but to the next universe.

     All of the things category theorists like to do can be done in this
framework.  The only controversial point is whether the Grothen\-dieck
axiom is too strong for the needs of category theorists.  Mac Lane
\index{Mac Lane, Saunders} argues that ``one universe is enough'' and
Feferman\index{Feferman, Solomon} has argued that one can get by with
ordinary ZFC.  I don't find Feferman's arguments persuasive.  Mac Lane
may be right, but when I think about category theory I do it \`{a} la
Grothendieck.

        By the way Mizar\index{Mizar} adds the axiom ``there is a proper
class of inaccessibles'' precisely so as to do category theory.
\end{quotation}

The most current information on the Metamath program and databases can
always be found at \url{http://metamath.org}.


\subsubsection{Note Added June 24, 2006}\label{note2006}

The Metamath spec was restricted slightly to make parsers easier to
write.  See the footnote on p.~\pageref{namespace}.

%\subsubsection{Note Added July 24, 2006}\label{note2006b}
\subsubsection{Note Added March 10, 2007}\label{note2006b}

I am grateful to Anthony Williams\index{Williams, Anthony} for writing
the \LaTeX\ package called {\tt realref.sty} and contributing it to the
public domain.  This package allows the internal hyperlinks in a {\sc
pdf} file to anchor to specific page numbers instead of just section
titles, making the navigation of the {\sc pdf} file for this book much
more pleasant and ``logical.''

A typographical error found by Martin Kiselkov was corrected.
A confusing remark about unification was deleted per suggestion of
Mel O'Cat.

\subsubsection{Note Added May 27, 2009}\label{note2009}

Several typos found by Kim Sparre were corrected.  A note was added that
the Poincar\'{e} conjecture has been proved (p.~\pageref{poincare}).

\subsubsection{Note Added Nov. 17, 2014}\label{note2014}

The statement of the Schr\"{o}der--Bernstein theorem was corrected in
Section~\ref{trust}.  Thanks to Bob Solovay for pointing out the error.

\subsubsection{Note Added May 25, 2016}\label{note2016}

Thanks to Jerry James for correcting 16 typos.

\subsubsection{Note Added February 25, 2019}\label{note201902}

David A. Wheeler\index{Wheeler, David A.}
made a large number of improvements and updates,
in coordination with Norman Megill.
The predicate calculus axioms were renumbered, and the text makes
it clear that they are based on Tarski's system S2;
the one slight deviation in axiom ax-6 is explained and justified.
The real and complex number axioms were modified to be consistent with
\texttt{set.mm}\index{set theory database (\texttt{set.mm})}%
\index{Metamath Proof Explorer}.
Long-awaited specification changes ``1--8'' were made,
which clarified previously ambiguous points.
Some errors in the text involving \texttt{\$f} and
\texttt{\$d} statements were corrected (the spec was correct, but
the in-book explanations unintentionally contradicted the spec).
We now have a system for automatically generating narrow PDFs,
so that those with smartphones can have easy access to the current
version of this document.
A new section on deduction was added;
it discusses the standard deduction theorem,
the weak deduction theorem,
deduction style, and natural deduction.
Many minor corrections (too numerous to list here) were also made.

\subsubsection{Note Added March 7, 2019}\label{note201903}

This added a description of the Matamath language syntax in
Extended Backus--Naur Form (EBNF)\index{Extended Backus--Naur Form}\index{EBNF}
in Appendix \ref{BNF}, added a brief explanation about typecodes,
inserted more examples in the deduction section,
and added a variety of smaller improvements.

\subsubsection{Note Added April 7, 2019}\label{note201904}

This version clarified the proper substitution notation, improved the
discussion on the weak deduction theorem and natural deduction,
documented the \texttt{undo} command, updated the information on
\texttt{write source}, changed the typecode
from \texttt{set} to \texttt{setvar} to be consistent with the current
version of \texttt{set.mm}, added more documentation about comment markup
(e.g., documented how to create headings), and clarified the
differences between various assertion forms (in particular deduction form).

\subsubsection{Note Added June 2, 2019}\label{note201906}

This version fixes a large number of small issues reported by
Beno\^{i}t Jubin\index{Jubin, Beno\^{i}t}, such as editorial issues
and the need to document \texttt{verify markup} (thank you!).
This version also includes specific examples
of forms (deduction form, inference form, and closed form).
We call this version the ``second edition'';
the previous edition formally published in 2007 had a slightly different title
(\textit{Metamath: A Computer Language for Pure Mathematics}).

\chapter{Introduction}
\pagenumbering{arabic}

\begin{quotation}
  {\em {\em I.M.:}  No, no.  There's nothing subjective about it!  Everybody
knows what a proof is.  Just read some books, take courses from a competent
mathematician, and you'll catch on.

{\em Student:}  Are you sure?

{\em I.M.:}  Well---it is possible that you won't, if you don't have any
aptitude for it.  That can happen, too.

{\em Student:}  Then {\em you} decide what a proof is, and if I don't learn
to decide in the same way, you decide I don't have any aptitude.

{\em I.M.:}  If not me, then who?}
    \flushright\sc  ``The Ideal Mathematician''
    \index{Davis, Phillip J.}
    \footnote{\cite{Davis}, p.~40.}\\
\end{quotation}

Brilliant mathematicians have discovered almost
unimaginably profound results that rank among the crowning intellectual
achievements of mankind.  However, there is a sense in which modern abstract
mathematics is behind the times, stuck in an era before computers existed.
While no one disputes the remarkable results that have been achieved,
communicating these results in a precise way to the uninitiated is virtually
impossible.  To describe these results, a terse informal language is used which
despite its elegance is very difficult to learn.  This informal language is not
imprecise, far from it, but rather it often has omitted detail
and symbols with hidden context that are
implicitly understood by an expert but few others.  Extremely complex technical
meanings are associated with innocent-sounding English words such as
``compact'' and ``measurable'' that barely hint at what is actually being
said.  Anyone who does not keep the precise technical meaning constantly in
mind is bound to fail, and acquiring the ability to do this can be achieved
only through much practice and hard work.  Only the few who complete the
painful learning experience can join the small in-group of pure
mathematicians.  The informal language effectively cuts off the true nature of
their knowledge from most everyone else.

Metamath\index{Metamath} makes abstract mathematics more concrete.  It allows
a computer to keep track of the complexity associated with each word or symbol
with absolute rigor.  You can explore this complexity at your leisure, to
whatever degree you desire.  Whether or not you believe that concepts such as
infinity actually ``exist'' outside of the mind, Metamath lets you get to the
foundation for what's really being said.

Metamath also enables completely rigorous and thorough proof verification.
Its language is simple enough so that you
don't have to rely on the authority of experts but can verify the results
yourself, step by step.  If you want to attempt to derive your own results,
Metamath will not let you make a mistake in reasoning.
Even professional mathematicians make mistakes; Metamath makes it possible
to thoroughly verify that proofs are correct.

Metamath\index{Metamath} is a computer language and an associated computer
program for archiving, verifying, and studying mathematical proofs at a very
detailed level.
The Metamath language
describes formal\index{formal system} mathematical
systems and expresses proofs of theorems in those systems.  Such a language
is called a metalanguage\index{metalanguage} by mathematicians.
The Metamath program is a computer program that verifies
proofs expressed in the Metamath language.
The Metamath program does not have the built-in
ability to make logical inferences; it just makes a series of symbol
substitutions according to instructions given to it in a proof
and verifies that the result matches the expected theorem.  It makes logical
inferences based only on rules of logic that are contained in a set of
axioms\index{axiom}, or first principles, that you provide to it as the
starting point for proofs.

The complete specification of the Metamath language is only four pages long
(Section~\ref{spec}, p.~\pageref{spec}).  Its simplicity may at first make you
wonder how it can do much of anything at all.  But in fact the kinds of
symbol manipulations it performs are the ones that are implicitly done in all
mathematical systems at the lowest level.  You can learn it relatively quickly
and have complete confidence in any mathematical proof that it verifies.  On
the other hand, it is powerful and general enough so that virtually any
mathematical theory, from the most basic to the deeply abstract, can be
described with it.

Although in principle Metamath can be used with any
kind of mathematics, it is best suited for abstract or ``pure'' mathematics
that is mostly concerned with theorems and their proofs, as opposed to the
kind of mathematics that deals with the practical manipulation of numbers.
Examples of branches of pure mathematics are logic\index{logic},\footnote{Logic
is the study of statements that are universally true regardless of the objects
being described by the statements.  An example is the statement, ``if $P$
implies $Q$, then either $P$ is false or $Q$ is true.''} set theory\index{set
theory},\footnote{Set theory is the study of general-purpose mathematical objects called
``sets,'' and from it essentially all of mathematics can be derived.  For
example, numbers can be defined as specific sets, and their properties
can be explored using the tools of set theory.} number theory\index{number
theory},\footnote{Number theory deals with the properties of positive and
negative integers (whole numbers).} group theory\index{group
theory},\footnote{Group theory studies the properties of mathematical objects
called groups that obey a simple set of axioms and have properties of symmetry
that make them useful in many other fields.} abstract algebra\index{abstract
algebra},\footnote{Abstract algebra includes group theory and also studies
groups with additional properties that qualify them as ``rings'' and
``fields.''  The set of real numbers is a familiar example of a field.},
analysis\index{analysis} \index{real and complex numbers}\footnote{Analysis is
the study of real and complex numbers.} and
topology\index{topology}.\footnote{One area studied by topology are properties
that remain unchanged when geometrical objects undergo stretching
deformations; for example a doughnut and a coffee cup each have one hole (the
cup's hole is in its handle) and are thus considered topologically
equivalent.  In general, though, topology is the study of abstract
mathematical objects that obey a certain (surprisingly simple) set of axioms.
See, for example, Munkres \cite{Munkres}\index{Munkres, James R.}.} Even in
physics, Metamath could be applied to certain branches that make use of
abstract mathematics, such as quantum logic\index{quantum logic} (used to study
aspects of quantum mechanics\index{quantum mechanics}).

On the other hand, Metamath\index{Metamath} is less suited to applications
that deal primarily with intensive numeric computations.  Metamath does not
have any built-in representation of numbers\index{Metamath!representation of
numbers}; instead, a specific string of symbols (digits) must be syntactically
constructed as part of any proof in which an ordinary number is used.  For
this reason, numbers in Metamath are best limited to specific constants that
arise during the course of a theorem or its proof.  Numbers are only a tiny
part of the world of abstract mathematics.  The exclusion of built-in numbers
was a conscious decision to help achieve Metamath's simplicity, and there are
other software tools if you have different mathematical needs.
If you wish to quickly solve algebraic problems, the computer algebra
programs\index{computer algebra system} {\sc
macsyma}\index{macsyma@{\sc macsyma}}, Mathematica\index{Mathematica}, and
Maple\index{Maple} are specifically suited to handling numbers and
algebra efficiently.
If you wish to simply calculate numeric or matrix expressions easily,
tools such as Octave\index{Octave} may be a better choice.

After learning Metamath's basic statement types, any
tech\-ni\-cal\-ly ori\-ent\-ed person, mathematician or not, can
immediately trace
any theorem proved in the language as far back as he or she wants, all the way
to the axioms on which the theorem is based.  This ability suggests a
non-traditional way of learning about pure mathematics.  Used in conjunction
with traditional methods, Metamath could make pure mathematics accessible to
people who are not sufficiently skilled to figure out the implicit detail in
ordinary textbook proofs.  Once you learn the axioms of a theory, you can have
complete confidence that everything you need to understand a proof you are
studying is all there, at your beck and call, allowing you to focus in on any
proof step you don't understand in as much depth as you need, without worrying
about getting stuck on a step you can't figure out.\footnote{On the other
hand, writing proofs in the Metamath language is challenging, requiring
a degree of rigor far in excess of that normally taught to students.  In a
classroom setting, I doubt that writing Metamath proofs would ever replace
traditional homework exercises involving informal proofs, because the time
needed to work out the details would not allow a course to
cover much material.  For students who have trouble grasping the implied rigor
in traditional material, writing a few simple proofs in the Metamath language
might help clarify fuzzy thought processes.  Although somewhat difficult at
first, it eventually becomes fun to do, like solving a puzzle, because of the
instant feedback provided by the computer.}

Metamath\index{Metamath} is probably unlike anything you have
encountered before.  In this first chapter we will look at the philosophy and
use of computers in mathematics in order to better understand the motivation
behind Metamath.  The material in this chapter is not required in order to use
Metamath.  You may skip it if you are impatient, but I hope you will find it
educational and enjoyable.  If you want to start experimenting with the
Metamath program right away, proceed directly to Chapter~\ref{using}
(p.~\pageref{using}).  To
learn the Metamath language, skim Chapter~\ref{using} then proceed to
Chapter~\ref{languagespec} (p.~\pageref{languagespec}).

\section{Mathematics as a Computer Language}

\begin{quote}
  {\em The study of mathematics is apt to commence in
dis\-ap\-point\-ment.\ldots \\
We are told that by its aid the stars are weighted
and the billions of molecules in a drop of water are counted.  Yet, like the
ghost of Hamlet's father, this great science eludes the efforts of our mental
weapons to grasp it.}
  \flushright\sc  Alfred North Whitehead\footnote{\cite{Whitehead}, ch.\ 1.}\\
\end{quote}\index{Whitehead, Alfred North}

\subsection{Is Mathematics ``User-Friendly''?}

Suppose you have no formal training in abstract mathematics.  But popular
books you've read offer tempting glimpses of this world filled with profound
ideas that have stirred the human spirit.  You are not satisfied with the
informal, watered-down descriptions you've read but feel it is important to
grasp the underlying mathematics itself to understand its true meaning. It's
not practical to go back to school to learn it, though; you don't want to
dedicate years of your life to it.  There are many important things in life,
and you have to set priorities for what's important to you.  What would happen
if you tried to pursue it on your own, in your spare time?

After all, you were able to learn a computer programming language such as
Pascal on your own without too much difficulty, even though you had no formal
training in computers.  You don't claim to be an expert in software design,
but you can write a passable program when necessary to suit your needs.  Even
more important, you know that you can look at anyone else's Pascal program, no
matter how complex, and with enough patience figure out exactly how it works,
even though you are not a specialist.  Pascal allows you do anything that a
computer can do, at least in principle.  Thus you know you have the ability,
in principle, to follow anything that a computer program can do:  you just
have to break it down into small enough pieces.

Here's an imaginary scenario of what might happen if you na\-ive\-ly a\-dopted
this same view of abstract mathematics and tried to pick it up on your own, in
a period of time comparable to, say, learning a computer programming
language.

\subsubsection{A Non-Mathematician's Quest for Truth}

\begin{quote}
  {\em \ldots my daughters have been studying (chemistry) for several
se\-mes\-ters, think they have learned differential and integral calculus in
school, and yet even today don't know why $x\cdot y=y\cdot x$ is true.}
  \flushright\sc  Edmund Landau\footnote{\cite{Landau}, p.~vi.}\\
\end{quote}\index{Landau, Edmund}

\begin{quote}
  {\em Minus times minus is plus,\\
The reason for this we need not discuss.}
  \flushright\sc W.\ H.\ Auden\footnote{As quoted in \cite{Guillen}, p.~64.}\\
\end{quote}\index{Auden, W.\ H.}\index{Guillen, Michael}

We'll suppose you are a technically oriented professional, perhaps an engineer, a
computer programmer, or a physicist, but probably not a mathematician.  You
consider yourself reasonably intelligent.  You did well in school, learning a
variety of methods and techniques in practical mathematics such as calculus and
differential equations.  But rarely did your courses get into anything
resembling modern abstract mathematics, and proofs were something that appeared
only occasionally in your textbooks, a kind of necessary evil that was
supposed to convince you of a certain key result.  Most of your
homework consisted of exercises that gave you practice in the techniques, and
you were hardly ever asked to come up with a proof of your own.

You find yourself curious about advanced, abstract mathematics.  You are
driven by an inner conviction that it is important to understand and
appreciate some of the most profound knowledge discovered by mankind.  But it
seems very hard to learn, something that only certain gifted longhairs can
access and understand.  You are frustrated that it seems forever cut off from
you.

Eventually your curiosity drives you to do something about it.
You set for yourself a goal of ``really'' understanding mathematics:  not just
how to manipulate equations in algebra or calculus according to cookbook
rules, but rather to gain a deep understanding of where those rules come from.
In fact, you're not thinking about this kind of ordinary mathematics at all,
but about a much more abstract, ethereal realm of pure mathematics, where
famous results such as G\"{o}del's incompleteness theorem\index{G\"{o}del's
incompleteness theorem} and Cantor's different kinds of infinities
reside.

You have probably read a number of popular books, with titles like {\em
Infinity and the Mind} \cite{Rucker}\index{Rucker, Rudy}, on topics such as
these.  You found them inspiring but at the same time somewhat
unsatisfactory.  They gave you a general idea of what these results are about,
but if someone asked you to prove them, you wouldn't have the faintest idea of
where to begin.   Sure, you could give the same overall outline that you
learned from the popular books; and in a general sort of way, you do have an
understanding.  But deep down inside, you know that there is a rigor that is
missing, that probably there are many subtle steps and pitfalls along the way,
and ultimately it seems you have to place your trust in the experts in the
field.  You don't like this; you want to be able to verify these results for
yourself.

So where do you go next?  As a first step, you decide to look up some of the
original papers on the theorems you are curious about, or better, obtain some
standard textbooks in the field.  You look up a theorem you want to
understand.  Sure enough, it's there, but it's expressed with strange
terms and odd symbols that mean absolutely nothing to you.  It might as well be written in
a foreign language you've never seen before, whose symbols are totally alien.
You look at the proof, and you haven't the foggiest notion what each step
means, much less how one step follows from another.  Well, obviously you have
a lot to learn if you want to understand this stuff.

You feel that you could probably understand it by
going back to college for another three to six years and getting a math
degree.  But that does not fit in with your career and the other things in
your life and would serve no practical purpose.  You decide to seek a quicker
path.  You figure you'll just trace your way back to the beginning, step by
step, as you would do with a computer program, until you understand it.  But
you quickly find that this is not possible, since you can't even understand
enough to know what you have to trace back to.

Maybe a different approach is in order---maybe you should start at the
beginning and work your way up.  First, you read the introduction to the book
to find out what the prerequisites are.  In a similar fashion, you trace your
way back through two or three more books, finally arriving at one that seems
to start at a beginning:  it lists the axioms of arithmetic.  ``Aha!'' you
naively think, ``This must be the starting point, the source of all mathematical
knowledge.'' Or at least the starting point for mathematics dealing with
numbers; you have to start somewhere and have no idea what the starting point
for other mathematics would be.  But the word ``axioms'' looks promising.  So
you eagerly read along and work through some elementary exercises at the
beginning of the book.  You feel vaguely bothered:  these
don't seem like axioms at all, at least not in the sense that you want to
think of axioms.  Axioms imply a starting point from which everything else can
be built up, according to precise rules specified in the axiom system.  Even
though you can understand the first few proofs in an informal way,
and are able to do some of the
exercises, it's hard to pin down precisely what the
rules are.   Sure, each step seems to follow logically from the others, but
exactly what does that mean?  Is the ``logic'' just a matter of common sense,
something vague that we all understand but can never quite state precisely?

You've spent a number of years, off and on, programming computers, and you
know that in the case of computer languages there is no question of what the
rules are---they are precise and crystal clear.  If you follow them, your
program will work, and if you don't, it won't.  No matter how complex a
program, it can always be broken down into simpler and simpler pieces, until
you can ultimately identify the bits that are moved around to perform a
specific function.  Some programs might require a lot of perseverance to
accomplish this, but if you focus on a specific portion of it, you don't even
necessarily have to know how the rest of it works. Shouldn't there be an
analogy in mathematics?

You decide to apply the ultimate test:  you ask yourself how a computer could
verify or ensure that the steps in these proofs follow from one another.
Certainly mathematics must be at least as precisely defined as a computer
language, if not more so; after all, computer science itself is based on it.
If you can get a computer to verify these proofs, then you should also be
able, in principle, to understand them yourself in a very crystal clear,
precise way.

You're in for a surprise:  you can conceive of no way to convert the
proofs, which are in English, to a form that the computer can understand.
The proofs are filled with phrases such as ``assume there exists a unique
$x$\ldots'' and ``given any $y$, let $z$ be the number such that\ldots''  This
isn't the kind of logic you are used to in computer programming, where
everything, even arithmetic, reduces to Boolean ones and zeroes if you care to
break it down sufficiently.  Even though you think you understand the proofs,
there seems to be some kind of higher reasoning involved rather than precise
rules that define how you manipulate the symbols in the axioms.  Whatever it
is, it just isn't obvious how you would express it to a computer, and the more
you think about it, the more puzzled and confused you get, to the point where
you even wonder whether {\em you} really understand it.  There's a lot more to
these axioms of arithmetic than meets the eye.

Nobody ever talked about this in school in your applied math and engineering
courses.  You just learned the rules they gave you, not quite understanding
how or why they worked, sometimes vaguely suspicious or uncertain of them, and
through homework problems and osmosis learned how to present solutions that
satisfied the instructor and earned you an ``A.''  Rarely did you actually
``prove'' anything in a rigorous way, and the math majors who did do stuff
like that seemed to be in a different world.

Of course, there are computer algebra programs that can do mathematics, and
rather impressively.  They can instantly solve the integrals that you
struggled with in freshman calculus, and do much, much more.  But when you
look at these programs, what you see is a big collection of algorithms and
techniques that evolved and were added to over time, along with some basic
software that manipulates symbols.  Each algorithm that is built in is the
result of someone's theorem whose proof is omitted; you just have to trust the
person who proved it and the person who programmed it in and hope there are no
bugs.\index{computer program bugs}  Somehow this doesn't seem to be the
essence of mathematics.  Although computer algebra systems can generate
theorems with amazing speed, they can't actually prove a single one of them.

After some puzzlement, you revisit some popular books on what mathematics is
all about.  Somewhere you read that all of mathematics is actually derived
from something called ``set theory.''  This is a little confusing, because
nowhere in the book that presented the axioms of arithmetic was there any
mention of set theory, or if there was, it seemed to be just a tool that helps
you describe things better---the set of even numbers, that sort of thing.  If
set theory is the basis for all mathematics, then why are additional axioms
needed for arithmetic?

Something is wrong but you're not sure what.  One of your friends is a pure
mathematician.  He knows he is unable to communicate to you what he does for a
living and seems to have little interest in trying.  You do know that for him,
proofs are what mathematics is all about. You ask him what a proof is, and he
essentially tells you that, while of course it's based on logic, really it's
something you learn by doing it over and over until you pick it up.  He refers
you to a book, {\em How to Read and Do Proofs} \cite{Solow}.\index{Solow,
Daniel}  Although this book helps you understand traditional informal proofs,
there is still something missing you can't seem to pin down yet.

You ask your friend how you would go about having a computer verify a proof.
At first he seems puzzled by the question; why would you want to do that?
Then he says it's not something that would make any sense to do, but he's
heard that you'd have to break the proof down into thousands or even millions
of individual steps to do such a thing, because the reasoning involved is at
such a high level of abstraction.  He says that maybe it's something you could
do up to a point, but the computer would be completely impractical once you
get into any meaningful mathematics.  There, the only way you can verify a
proof is by hand, and you can only acquire the ability to do this by
specializing in the field for a couple of years in grad school.  Anyway, he
thinks it all has to do with set theory, although he has never taken a formal
course in set theory but just learned what he needed as he went along.

You are intrigued and amazed.  Apparently a mathematician can grasp as a
single concept something that would take a computer a thousand or a million
steps to verify, and have complete confidence in it.  Each one of these
thousand or million steps must be absolutely correct, or else the whole proof
is meaningless.  If you added a million numbers by hand, would you trust the
result?  How do you really know that all these steps are correct, that there
isn't some subtle pitfall in one of these million steps, like a bug in a
computer program?\index{computer program bugs}  After all, you've read that
famous mathematicians have occasionally made mistakes, and you certainly know
you've made your share on your math homework problems in school.

You recall the analogy with a computer program.  Sure, you can understand what
a large computer program such as a word processor does, as a single high-level
concept or a small set of such concepts, but your ability to understand it in
no way ensures that the program is correct and doesn't have hidden bugs.  Even
if you wrote the program yourself you can't really know this; most large
programs that you've written have had bugs that crop up at some later date, no
matter how careful you tried to be while writing them.

OK, so now it seems the reason you can't figure out how to make a
computer verify proofs is because each step really corresponds to a
million small steps.  Well, you say, a computer can do a million
calculations in a second, so maybe it's still practical to do.  Now the
puzzle becomes how to figure out what the million steps are that each
English-language step corresponds to.  Your mathematician friend hasn't
a clue, but suggests that maybe you would find the answer by studying
set theory.  Actually, your friend thinks you're a little off the wall
for even wondering such a thing.  For him, this is not what mathematics
is all about.

The subject of set theory keeps popping up, so you decide it's
time to look it up.

You decide to start off on a careful footing, so you start reading a couple of
very elementary books on set theory.  A lot of it seems pretty obvious, like
intersections, subsets, and Venn diagrams.  You thumb through one of the
books; nowhere is anything about axioms mentioned. The other book relegates to
an appendix a brief discussion that mentions a set of axioms called
``Zermelo--Fraenkel set theory''\index{Zermelo--Fraenkel set theory} and states
them in English.  You look at them and have no idea what they really mean or
what you can do with them.  The comments in this appendix say that the purpose
of mentioning them is to expose you to the idea, but imply that they are not
necessary for basic understanding and that they are really the subject matter
of advanced treatments where fine points such as a certain paradox (Russell's
paradox\index{Russell's paradox}\footnote{Russell's paradox assumes that there
exists a set $S$ that is a collection of all sets that don't contain
themselves.  Now, either $S$ contains itself or it doesn't.  If it contains
itself, it contradicts its definition.  But if it doesn't contain itself, it
also contradicts its definition.  Russell's paradox is resolved in ZF set
theory by denying that such a set $S$ exists.}) are resolved.  Wait a
minute---shouldn't the axioms be a starting point, not an ending point?  If
there are paradoxes that arise without the axioms, how do you know you won't
stumble across one accidentally when using the informal approach?

And nowhere do these books describe how ``all of mathematics can be
derived from set theory'' which by now you've heard a few times.

You find a more advanced book on set theory.  This one actually lists the
axioms of ZF set theory in plain English on page one.  {\em Now} you think
your quest has ended and you've finally found the source of all mathematical
knowledge; you just have to understand what it means.  Here, in one place, is
the basis for all of mathematics!  You stare at the axioms in awe, puzzle over
them, memorize them, hoping that if you just meditate on them long enough they
will become clear.  Of course, you haven't the slightest idea how the rest of
mathematics is ``derived'' from them; in particular, if these are the axioms
of mathematics, then why do arithmetic, group theory, and so on need their own
axioms?

You start reading this advanced book carefully, pondering the meaning of every
word, because by now you're really determined to get to the bottom of this.
The first thing the book does is explain how the axioms came about, which was
to resolve Russell's paradox.\index{Russell's paradox}  In fact that seems to
be the main purpose of their existence; that they supposedly can be used to
derive all of mathematics seems irrelevant and is not even mentioned.  Well,
you go on.  You hope the book will explain to you clearly, step by step, how
to derive things from the axioms.  After all, this is the starting point of
mathematics, like a book that explains the basics of a computer programming
language.  But something is missing.  You find you can't even understand the
first proof or do the first exercise.  Symbols such as $\exists$ and $\forall$
permeate the page without any mention of where they came from or how to
manipulate them; the author assumes you are totally familiar with them and
doesn't even tell you what they mean.  By now you know that $\exists$ means
``there exists'' and $\forall$ means ``for all,'' but shouldn't the rules for
manipulating these symbols be part of the axioms?  You still have no idea
how you could even describe the axioms to a computer.

Certainly there is something much different here from the technical
literature you're used to reading.  A computer language manual almost
always explains very clearly what all the symbols mean, precisely what
they do, and the rules used for combining them, and you work your way up
from there.

After glancing at four or five other such books, you come to the realization
that there is another whole field of study that you need just to get to the
point at which you can understand the axioms of set theory.  The field is
called ``logic.''  In fact, some of the books did recommend it as a
prerequisite, but it just didn't sink in.  You assumed logic was, well, just
logic, something that a person with common sense intuitively understood.  Why
waste your time reading boring treatises on symbolic logic, the manipulation
of 1's and 0's that computers do, when you already know that?  But this is a
different kind of logic, quite alien to you.  The subject of {\sc nand} and
{\sc nor} gates is not even touched upon or in any case has to do with only a
very small part of this field.

So your quest continues.  Skimming through the first couple of introductory
books, you get a general idea of what logic is about and what quantifiers
(``for all,'' ``there exists'') mean, but you find their examples somewhat
trivial and mildly annoying (``all dogs are animals,'' ``some animals are
dogs,'' and such).  But all you want to know is what the rules are for
manipulating the symbols so you can apply them to set theory.  Some formulas
describing the relationships among quantifiers ($\exists$ and $\forall$) are
listed in tables, along with some verbal reasoning to justify them.
Presumably, if you want to find out if a formula is correct, you go through
this same kind of mental reasoning process, possibly using images of dogs and
animals. Intuitively, the formulas seem to make sense.  But when you ask
yourself, ``What are the rules I need to get a computer to figure out whether
this formula is correct?'', you still don't know.  Certainly you don't ask the
computer to imagine dogs and animals.

You look at some more advanced logic books.  Many of them have an introductory
chapter summarizing set theory, which turns out to be a prerequisite.  You
need logic to understand set theory, but it seems you also need set theory to
understand logic!  These books jump right into proving rather advanced
theorems about logic, without offering the faintest clue about where the logic
came from that allows them to prove these theorems.

Luckily, you come across an elementary book of logic that, halfway through,
after the usual truth tables and metaphors, presents in a clear, precise way
what you've been looking for all along: the axioms!  They're divided into
propositional calculus (also called sentential logic) and predicate calculus
(also called first-order logic),\index{first-order logic} with rules so simple
and crystal clear that now you can finally program a computer to understand
them.  Indeed, they're no harder than learning how to play a game of chess.
As far as what you seem to need is concerned, the whole book could have been
written in five pages!

{\em Now} you think you've found the ultimate source of mathematical
truth.  So---the axioms of mathematics consist of these axioms of logic,
together with the axioms of ZF set theory. (By now you've also been able to
figure out how to translate the ZF axioms from English into the
actual symbols of logic which you can now manipulate according to
precise, easy-to-understand rules.)

Of course, you still don't understand how ``all of mathematics can be
derived from set theory,'' but maybe this will reveal itself in due
course.

You eagerly set out to program the axioms and rules into a computer and start
to look at the theorems you will have to prove as the logic is developed.  All
sorts of important theorems start popping up:  the deduction
theorem,\index{deduction theorem} the substitution theorem,\index{substitution
theorem} the completeness theorem of propositional calculus,\index{first-order
logic!completeness} the completeness theorem of predicate calculus.  Uh-oh,
there seems to be trouble.  They all get harder and harder, and not one of
them can be derived with the axioms and rules of logic you've just been
handed.  Instead, they all require ``metalogic'' for their proofs, a kind of
mixture of logic and set theory that allows you to prove things {\em about}
the axioms and theorems of logic rather than {\em with} them.

You plow ahead anyway.  A month later, you've spent much of your
free time getting the computer to verify proofs in propositional calculus.
You've programmed in the axioms, but you've also had to program in the
deduction theorem, the substitution theorem, and the completeness theorem of
propositional calculus, which by now you've resigned yourself to treating as
rather complex additional axioms, since they can't be proved from the axioms
you were given.  You can now get the computer to verify and even generate
complete, rigorous, formal proofs\index{formal proof}.  Never mind that they
may have 100,000 steps---at least now you can have complete, absolute
confidence in them.  Unfortunately, the only theorems you have proved are
pretty trivial and you can easily verify them in a few minutes with truth
tables, if not by inspection.

It looks like your mathematician friend was right.  Getting the computer to do
serious mathematics with this kind of rigor seems almost hopeless.  Even
worse, it seems that the further along you get, the more ``axioms'' you have
to add, as each new theorem seems to involve additional ``metamathematical''
reasoning that hasn't been formalized, and none of it can be derived from the
axioms of logic.  Not only do the proofs keep growing exponentially as you get
further along, but the program to verify them keeps getting bigger and bigger
as you program in more ``metatheorems.''\index{metatheorem}\footnote{A
metatheorem is usually a statement that is too general to be directly provable
in a theory.  For example, ``if $n_1$, $n_2$, and $n_3$ are integers, then
$n_1+n_2+n_3$ is an integer'' is a theorem of number theory.  But ``for any
integer $k > 1$, if $n_1, \ldots, n_k$ are integers, then $n_1+\ldots +n_k$ is
an integer'' is a metatheorem, in other words a family of theorems, one for
every $k$.  The reason it is not a theorem is that the general sum $n_1+\ldots
+n_k$ (as a function of $k$) is not an operation that can be defined directly
in number theory.} The bugs\index{computer program bugs} that have cropped up
so far have already made you start to lose faith in the rigor you seem to have
achieved, and you know it's just going to get worse as your program gets larger.

\subsection{Mathematics and the Non-Specialist}

\begin{quote}
  {\em A real proof is not checkable by a machine, or even by any mathematician
not privy to the gestalt, the mode of thought of the particular field of
mathematics in which the proof is located.}
  \flushright\sc  Davis and Hersh\index{Davis, Phillip J.}
  \footnote{\cite{Davis}, p.~354.}\\
\end{quote}

The bulk of abstract or theoretical mathematics is ordinarily outside
the reach of anyone but a few specialists in each field who have completed
the necessary difficult internship in order to enter its coterie.  The
typical intelligent layperson has no reasonable hope of understanding much of
it, nor even the specialist mathematician of understanding other fields.  It
is like a foreign language that has no dictionary to look up the translation;
the only way you can learn it is by living in the country for a few years.  It
is argued that the effort involved in learning a specialty is a necessary
process for acquiring a deep understanding.  Of course, this is almost certainly
true if one is to make significant contributions to a field; in particular,
``doing'' proofs is probably the most important part of a mathematician's
training.  But is it also necessary to deny outsiders access to it?  Is it
necessary that abstract mathematics be so hard for a layperson to grasp?

A computer normally is of no help whatsoever.  Most published proofs are
actually just series of hints written in an informal style that requires
considerable knowledge of the field to understand.  These are the ``real
proofs'' referred to by Davis and Hersh.\index{informal proof}  There is an
implicit understanding that, in principle, such a proof could be converted to
a complete formal proof\index{formal proof}.  However, it is said that no one
would ever attempt such a conversion, even if they could, because that would
presumably require millions of steps (Section~\ref{dream}).  Unfortunately the
informal style automatically excludes the understanding of the proof
by anyone who hasn't gone through the necessary apprenticeship. The
best that the intelligent layperson can do is to read popular books about deep
and famous results; while this can be helpful, it can also be misleading, and
the lack of detail usually leaves the reader with no ability whatsoever to
explore any aspect of the field being described.

The statements of theorems often use sophisticated notation that makes them
inaccessible to the non-specialist.  For a non-specialist who wants to achieve
a deeper understanding of a proof, the process of tracing definitions and
lemmas back through their hierarchy\index{hierarchy} quickly becomes confusing
and discouraging.  Textbooks are usually written to train mathematicians or to
communicate to people who are already mathematicians, and large gaps in proofs
are often left as exercises to the reader who is left at an impasse if he or
she becomes stuck.

I believe that eventually computers will enable non-specialists and even
intelligent laypersons to follow almost any mathematical proof in any field.
Metamath is an attempt in that direction.  If all of mathematics were as
easily accessible as a computer programming language, I could envision
computer programmers and hobbyists who otherwise lack mathematical
sophistication exploring and being amazed by the world of theorems and proofs
in obscure specialties, perhaps even coming up with results of their own.  A
tremendous advantage would be that anyone could experiment with conjectures in
any field---the computer would offer instant feedback as to whether
an inference step was correct.

Mathematicians sometimes have to put up with the annoyance of
cranks\index{cranks} who lack a fundamental understanding of mathematics but
insist that their ``proofs'' of, say, Fermat's Last Theorem\index{Fermat's
Last Theorem} be taken seriously.  I think part of the problem is that these
people are misled by informal mathematical language, treating it as if they
were reading ordinary expository English and failing to appreciate the
implicit underlying rigor.  Such cranks are rare in the field of computers,
because computer languages are much more explicit, and ultimately the proof is
in whether a computer program works or not.  With easily accessible
computer-based abstract mathematics, a mathematician could say to a crank,
``don't bother me until you've demonstrated your claim on the computer!''

% 22-May-04 nm
% Attempt to move De Millo quote so it doesn't separate from attribution
% CHANGE THIS NUMBER (AND ELIMINATE IF POSSIBLE) WHEN ABOVE TEXT CHANGES
\vspace{-0.5em}

\subsection{An Impossible Dream?}\label{dream}

\begin{quote}
  {\em Even quite basic theorems would demand almost unbelievably vast
  books to display their proofs.}
    \flushright\sc  Robert E. Edwards\footnote{\cite{Edwards}, p.~68.}\\
\end{quote}\index{Edwards, Robert E.}

\begin{quote}
  {\em Oh, of course no one ever really {\em does} it.  It would take
  forever!  You just show that you could do it, that's sufficient.}
    \flushright\sc  ``The Ideal Mathematician''
    \index{Davis, Phillip J.}\footnote{\cite{Davis},
p.~40.}\\
\end{quote}

\begin{quote}
  {\em There is a theorem in the primitive notation of set theory that
  corresponds to the arithmetic theorem `$1000+2000=3000$'.  The formula
  would be forbiddingly long\ldots even if [one] knows the definitions
  and is asked to simplify the long formula according to them, chances are
  he will make errors and arrive at some incorrect result.}
    \flushright\sc  Hao Wang\footnote{\cite{Wang}, p.~140.}\\
\end{quote}\index{Wang, Hao}

% 22-May-04 nm
% Attempt to move De Millo quote so it doesn't separate from attribution
% CHANGE THIS NUMBER (AND ELIMINATE IF POSSIBLE) WHEN ABOVE TEXT CHANGES
\vspace{-0.5em}

\begin{quote}
  {\em The {\em Principia Mathematica} was the crowning achievement of the
  formalists.  It was also the deathblow of the formalist view.\ldots
  {[Rus\-sell]} failed, in three enormous volumes, to get beyond the elementary
  facts of arithmetic.  He showed what can be done in principle and what
  cannot be done in practice.  If the mathematical process were really
  one of strict, logical progression, we would still be counting our
  fingers.\ldots
  One theoretician estimates\ldots that a demonstration of one of
  Ramanujan's conjectures assuming set theory and elementary analysis would
  take about two thousand pages; the length of a deduction from first principles
  is nearly in\-con\-ceiv\-a\-ble\ldots The probabilists argue that\ldots any
  very long proof can at best be viewed as only probably correct\ldots}
  \flushright\sc Richard de Millo et. al.\footnote{\cite{deMillo}, pp.~269,
  271.}\\
\end{quote}\index{de Millo, Richard}

A number of writers have conveyed the impression that the kind of absolute
rigor provided by Metamath\index{Metamath} is an impossible dream, suggesting
that a complete, formal verification\index{formal proof} of a typical theorem
would take millions of steps in untold volumes of books.  Even if it could be
done, the thinking sometimes goes, all meaning would be lost in such a
monstrous, tedious verification.\index{informal proof}\index{proof length}

These writers assume, however, that in order to achieve the kind of complete
formal verification they desire one must break down a proof into individual
primitive steps that make direct reference to the axioms.  This is
not necessary.  There is no reason not to make use of previously proved
theorems rather than proving them over and over.

Just as important, definitions\index{definition} can be introduced along
the way, allowing very complex formulas to be represented with few
symbols.  Not doing this can lead to absurdly long formulas.  For
example, the mere statement of
G\"{o}del's incompleteness theorem\index{G\"{o}del's
incompleteness theorem}, which can be expressed with a small number of
defined symbols, would require about 20,000 primitive symbols to express
it.\index{Boolos, George S.}\footnote{George S.\ Boolos, lecture at
Massachusetts Institute of Technology, spring 1990.} An extreme example
is Bourbaki's\label{bourbaki} language for set theory, which requires
4,523,659,424,929 symbols plus 1,179,618,517,981 disambiguatory links
(lines connecting symbol pairs, usually drawn below or above the
formula) to express the number
``one'' \cite{Mathias}.\index{Mathias, Adrian R. D.}\index{Bourbaki,
Nicolas}
% http://www.dpmms.cam.ac.uk/~ardm/

A hierarchy\index{hierarchy} of theorems and definitions permits an
exponential growth in the formula sizes and primitive proof steps to be
described with only a linear growth in the number of symbols used.  Of course,
this is how ordinary informal mathematics is normally done anyway, but with
Metamath\index{Metamath} it can be done with absolute rigor and precision.

\subsection{Beauty}


\begin{quote}
  {\em No one shall be able to drive us from the paradise that Cantor has
created for us.}
   \flushright\sc  David Hilbert\footnote{As quoted in \cite{Moore}, p.~131.}\\
\end{quote}\index{Hilbert, David}

\needspace{3\baselineskip}
\begin{quote}
  {\em Mathematics possesses not only truth, but some supreme beauty ---a
  beauty cold and austere, like that of a sculpture.}
    \flushright\sc  Bertrand
    Russell\footnote{\cite{Russell}.}\\
\end{quote}\index{Russell, Bertrand}

\begin{quote}
  {\em Euclid alone has looked on Beauty bare.}
  \flushright\sc Edna Millay\footnote{As quoted in \cite{Davis}, p.~150.}\\
\end{quote}\index{Millay, Edna}

For most people, abstract mathematics is distant, strange, and
incomprehensible.  Many popular books have tried to convey some of the sense
of beauty in famous theorems.  But even an intelligent layperson is left with
only a general idea of what a theorem is about and is hardly given the tools
needed to make use of it.  Traditionally, it is only after years of arduous
study that one can grasp the concepts needed for deep understanding.
Metamath\index{Metamath} allows you to approach the proof of the theorem from
a quite different perspective, peeling apart the formulas and definitions
layer by layer until an entirely different kind of understanding is achieved.
Every step of the proof is there, pieced together with absolute precision and
instantly available for inspection through a microscope with a magnification
as powerful as you desire.

A proof in itself can be considered an object of beauty.  Constructing an
elegant proof is an art.  Once a famous theorem has been proved, often
considerable effort is made to find simpler and more easily understood
proofs.  Creating and communicating elegant proofs is a major concern of
mathematicians.  Metamath is one way of providing a common language for
archiving and preserving this information.

The length of a proof can, to a certain extent, be considered an
objective measure of its ``beauty,'' since shorter proofs are usually
considered more elegant.  In the set theory database
\texttt{set.mm}\index{set theory database (\texttt{set.mm})}%
\index{Metamath Proof Explorer}
provided with Metamath, one goal was to make all proofs as short as possible.

\needspace{4\baselineskip}
\subsection{Simplicity}

\begin{quote}
  {\em God made man simple; man's complex problems are of his own
  devising.}
    \flushright\sc Eccles. 7:29\footnote{Jerusalem Bible.}\\
\end{quote}\index{Bible}

\needspace{3\baselineskip}
\begin{quote}
  {\em God made integers, all else is the work of man.}
    \flushright\sc Leopold Kronecker\footnote{{\em Jahresbericht
	der Deutschen Mathematiker-Vereinigung }, vol. 2, p. 19.}\\
\end{quote}\index{Kronecker, Leopold}

\needspace{3\baselineskip}
\begin{quote}
  {\em For what is clear and easily comprehended attracts; the
  complicated repels.}
    \flushright\sc David Hilbert\footnote{As quoted in \cite{deMillo},
p.~273.}\\
\end{quote}\index{Hilbert, David}

The Metamath\index{Metamath} language is simple and Spartan.  Metamath treats
all mathematical expressions as simple sequences of symbols, devoid of meaning.
The higher-level or ``metamathematical'' notions underlying Metamath are about
as simple as they could possibly be.  Each individual step in a proof involves
a single basic concept, the substitution of an expression for a variable, so
that in principle almost anyone, whether mathematician or not, can
completely understand how it was arrived at.

In one of its most basic applications, Metamath\index{Metamath} can be used to
develop the foundations of mathematics\index{foundations of mathematics} from
the very beginning.  This is done in the set theory database that is provided
with the Metamath package and is the subject matter
of Chapter~\ref{fol}. Any language (a metalanguage\index{metalanguage})
used to describe mathematics (an object language\index{object language}) must
have a mathematical content of its own, but it is desirable to keep this
content down to a bare minimum, namely that needed to make use of the
inference rules specified by the axioms.  With any metalanguage there is a
``chicken and egg'' problem somewhat like circular reasoning:  you must assume
the validity of the mathematics of the metalanguage in order to prove the
validity of the mathematics of the object language.  The mathematical content
of Metamath itself is quite limited.  Like the rules of a game of chess, the
essential concepts are simple enough so that virtually anyone should be able to
understand them (although that in itself will not let you play like
a master).  The symbols that Metamath manipulates do not in themselves
have any intrinsic meaning.  Your interpretation of the axioms that you supply
to Metamath is what gives them meaning.  Metamath is an attempt to strip down
mathematical thought to its bare essence and show you exactly how the symbols
are manipulated.

Philosophers and logicians, with various motivations, have often thought it
important to study ``weak'' fragments of logic\index{weak logic}
\cite{Anderson}\index{Anderson, Alan Ross} \cite{MegillBunder}\index{Megill,
Norman}\index{Bunder, Martin}, other unconventional systems of logic (such as
``modal'' logic\index{modal logic} \cite[ch.\ 27]{Boolos}\index{Boolos, George
S.}), and quantum logic\index{quantum logic} in physics
\cite{Pavicic}\index{Pavi{\v{c}}i{\'{c}}, M.}.  Metamath\index{Metamath}
provides a framework in which such systems can be expressed, with an absolute
precision that makes all underlying metamathematical assumptions rigorous and
crystal clear.

Some schools of philosophical thought, for example
intuitionism\index{intuitionism} and constructivism\index{constructivism},
demand that the notions underlying any mathematical system be as simple and
concrete as possible.  Metamath should meet the requirements of these
philosophies.  Metamath must be taught the symbols, axioms\index{axiom}, and
rules\index{rule} for a specific theory, from the skeptical (such as
intuitionism\index{intuitionism}\footnote{Intuitionism does not accept the law
of excluded middle (``either something is true or it is not true'').  See
\cite[p.~xi]{Tymoczko}\index{Tymoczko, Thomas} for discussion and references
on this topic.  Consider the theorem, ``There exist irrational numbers $a$ and
$b$ such that $a^b$ is rational.''  An intuitionist would reject the following
proof:  If $\sqrt{2}^{\sqrt{2}}$ is rational, we are done.  Otherwise, let
$a=\sqrt{2}^{\sqrt{2}}$ and $b=\sqrt{2}$. Then $a^b=2$, which is rational.})
to the bold (such as the axiom of choice in set theory\footnote{The axiom of
choice\index{Axiom of Choice} asserts that given any collection of pairwise
disjoint nonempty sets, there exists a set that has exactly one element in
common with each set of the collection.  It is used to prove many important
theorems in standard mathematics.  Some philosophers object to it because it
asserts the existence of a set without specifying what the set contains
\cite[p.~154]{Enderton}\index{Enderton, Herbert B.}.  In one foundation for
mathematics due to Quine\index{Quine, Willard Van Orman}, that has not been
otherwise shown to be inconsistent, the axiom of choice turns out to be false
\cite[p.~23]{Curry}\index{Curry, Haskell B.}.  The \texttt{show
trace{\char`\_}back} command of the Metamath program allows you to find out
whether the axiom of choice, or any other axiom, was assumed by a
proof.}\index{\texttt{show trace{\char`\_}back} command}).

The simplicity of the Metamath language lets the algorithm (computer program)
that verifies the validity of a Metamath proof be straightforward and
robust.  You can have confidence that the theorems it verifies really can be
derived from your axioms.

\subsection{Rigor}

\begin{quote}
  {\em Rigor became a goal with the Greeks\ldots But the efforts to
  pursue rigor to the utmost have led to an impasse in which there is
  no longer any agreement on what it really means.  Mathematics remains
  alive and vital, but only on a pragmatic basis.}
    \flushright\sc  Morris Kline\footnote{\cite{Kline}, p.~1209.}\\
\end{quote}\index{Kline, Morris}

Kline refers to a much deeper kind of rigor than that which we will discuss in
this section.  G\"{o}del's incompleteness theorem\index{G\"{o}del's
incompleteness theorem} showed that it is impossible to achieve absolute rigor
in standard mathematics because we can never prove that mathematics is
consistent (free from contradictions).\index{consistent theory}  If
mathematics is consistent, we will never know it, but must rely on faith.  If
mathematics is inconsistent, the best we can hope for is that some clever
future mathematician will discover the inconsistency.  In this case, the
axioms would probably be revised slightly to eliminate the inconsistency, as
was done in the case of Russell's paradox,\index{Russell's paradox} but the
bulk of mathematics would probably not be affected by such a discovery.
Russell's paradox, for example, did not affect most of the remarkable results
achieved by 19th-century and earlier mathematicians.  It mainly invalidated
some of Gottlob Frege's\index{Frege, Gottlob} work on the foundations of
mathematics in the late 1800's; in fact Frege's work inspired Russell's
discovery.  Despite the paradox, Frege's work contains important concepts that
have significantly influenced modern logic.  Kline's {\em Mathematics, The
Loss of Certainty} \cite{Klinel}\index{Kline, Morris} has an interesting
discussion of this topic.

What {\em can} be achieved with absolute certainty\index{certainty} is the
knowledge that if we assume the axioms are consistent and true, then the
results derived from them are true.  Part of the beauty of mathematics is that
it is the one area of human endeavor where absolute certainty can be achieved
in this sense.  A mathematical truth will remain such for eternity.  However,
our actual knowledge of whether a particular statement is a mathematical truth
is only as certain as the correctness of the proof that establishes it.  If
the proof of a statement is questionable or vague, we can't have absolute
confidence in the truth that the statement claims.

Let us look at some traditional ways of expressing proofs.

Except in the field of formal logic\index{formal logic}, almost all
traditional proofs in mathematics are really not proofs at all, but rather
proof outlines or hints as to how to go about constructing the proof.  Many
gaps\index{gaps in proofs} are left for the reader to fill in. There are
several reasons for this.  First, it is usually assumed in mathematical
literature that the person reading the proof is a mathematician familiar with
the specialty being described, and that the missing steps are obvious to such
a reader or at least that the reader is capable of filling them in.  This
attitude is fine for professional mathematicians in the specialty, but
unfortunately it often has the drawback of cutting off the rest of the world,
including mathematicians in other specialties, from understanding the proof.
We discussed one possible resolution to this on p.~\pageref{envision}.
Second, it is often assumed that a complete formal proof\index{formal proof}
would require countless millions of symbols (Section~\ref{dream}). This might
be true if the proof were to be expressed directly in terms of the axioms of
logic and set theory,\index{set theory} but it is usually not true if we allow
ourselves a hierarchy\index{hierarchy} of definitions and theorems to build
upon, using a notation that allows us to introduce new symbols, definitions,
and theorems in a precisely specified way.

Even in formal logic,\index{formal logic} formal proofs\index{formal proof}
that are considered complete still contain hidden or implicit information.
For example, a ``proof'' is usually defined as a sequence of
wffs,\index{well-formed formula (wff)}\footnote{A {\em wff} or well-formed
formula is a mathematical expression (string of symbols) constructed according
to some precise rules.  A formal mathematical system\index{formal system}
contains (1) the rules for constructing syntactically correct
wffs,\index{syntax rules} (2) a list of starting wffs called
axioms,\index{axiom} and (3) one or more rules prescribing how to derive new
wffs, called theorems\index{theorem}, from the axioms or previously derived
theorems.  An example of such a system is contained in
Metamath's\index{Metamath} set theory database, which defines a formal
system\index{formal system} from which all of standard mathematics can be
derived.  Section~\ref{startf} steps you through a complete example of a formal
system, and you may want to skim it now if you are unfamiliar with the
concept.} each of which is an axiom or follows from a rule applied to previous
wffs in the sequence.  The implicit part of the proof is the algorithm by
which a sequence of symbols is verified to be a valid wff, given the
definition of a wff.  The algorithm in this case is rather simple, but for a
computer to verify the proof,\index{automated proof verification} it must have
the algorithm built into its verification program.\footnote{It is possible, of
course, to specify wff construction syntax outside of the program itself
with a suitable input language (the Metamath language being an example), but
some proof-verification or theorem-proving programs lack the ability to extend
wff syntax in such a fashion.} If one deals exclusively with axioms and
elementary wffs, it is straightforward to implement such an algorithm.  But as
more and more definitions are added to the theory in order to make the
expression of wffs more compact, the algorithm becomes more and more
complicated.  A computer program that implements the algorithm becomes larger
and harder to understand as each definition is introduced, and thus more prone
to bugs.\index{computer program bugs}  The larger the program, the
more suspicious the mathematician may be about
the validity of its algorithms.  This is especially true because
computer programs are inherently hard to follow to begin with, and few people
enjoy verifying them manually in detail.

Metamath\index{Metamath} takes a different approach.  Metamath's ``knowledge''
is limited to the ability to substitute variables for expressions, subject to
some simple constraints.  Once the basic algorithm of Metamath is assumed to
be debugged, and perhaps independently confirmed, it
can be trusted once and for all.  The information that Metamath needs to
``understand'' mathematics is contained entirely in the body of knowledge
presented to Metamath.  Any errors in reasoning can only be errors in the
axioms or definitions contained in this body of knowledge.  As a
``constructive'' language\index{constructive language} Metamath has no
conditional branches or loops like the ones that make computer programs hard
to decipher; instead, the language can only build new sequences of symbols
from earlier sequences  of symbols.

The simplicity of the rules that underlie Metamath not only makes Metamath
easy to learn but also gives Metamath a great deal of flexibility. For
example, Metamath is not limited to describing standard first-order
logic\index{first-order logic}; higher-order logics\index{higher-order logic}
and fragments of logic\index{weak logic} can be described just as easily.
Metamath gives you the freedom to define whatever wff notation you prefer; it
has no built-in conception of the syntax of a wff.\index{well-formed formula
(wff)}  With suitable axioms and definitions, Metamath can even describe and
prove things about itself.\index{Metamath!self-description}  (John
Harrison\index{Harrison, John} discusses the ``reflection''
principle\index{reflection principle} involved in self-descriptive systems in
\cite{Harrison}.)

The flexibility of Metamath requires that its proofs specify a lot of detail,
much more than in an ordinary ``formal'' proof.\index{formal proof}  For
example, in an ordinary formal proof, a single step consists of displaying the
wff that constitutes that step.  In order for a computer program to verify
that the step is acceptable, it first must verify that the symbol sequence
being displayed is an acceptable wff.\index{automated proof verification} Most
proof verifiers have at least basic wff syntax built into their programs.
Metamath has no hard-wired knowledge of what constitutes a wff built into it;
instead every wff must be explicitly constructed based on rules defining wffs
that are present in a database.  Thus a single step in an ordinary formal
proof may be correspond to many steps in a Metamath proof. Despite the larger
number of steps, though, this does not mean that a Metamath proof must be
significantly larger than an ordinary formal proof. The reason is that since
we have constructed the wff from scratch, we know what the wff is, so there is
no reason to display it.  We only need to refer to a sequence of statements
that construct it.  In a sense, the display of the wff in an ordinary formal
proof is an implicit proof of its own validity as a wff; Metamath just makes
the proof explicit. (Section~\ref{proof} describes Metamath's proof notation.)

\section{Computers and Mathematicians}

\begin{quote}
  {\em The computer is important, but not to mathematics.}
    \flushright\sc  Paul Halmos\footnote{As quoted in \cite{Albers}, p.~121.}\\
\end{quote}\index{Halmos, Paul}

Pure mathematicians have traditionally been indifferent to computers, even to
the point of disdain.\index{computers and pure mathematics}  Computer science
itself is sometimes considered to fall in the mundane realm of ``applied''
mathematics, perhaps essential for the real world but intellectually unexciting
to those who seek the deepest truths in mathematics.  Perhaps a reason for this
attitude towards computers is that there is little or no computer software that
meets their needs, and there may be a general feeling that such software could
not even exist.  On the one hand, there are the practical computer algebra
systems, which can perform amazing symbolic manipulations in algebra and
calculus,\index{computer algebra system} yet can't prove the simplest
existence theorem, if the idea of a proof is present at all.  On the other
hand, there are specialized automated theorem provers that technically speaking
may generate correct proofs.\index{automated theorem proving}  But sometimes
their specialized input notation may be cryptic and their output perceived to
be long, inelegant, incomprehensible proofs.    The output
may be viewed with suspicion, since the program that generates it tends to be
very large, and its size increases the potential for bugs\index{computer
program bugs}.  Such a proof may be considered trustworthy only if
independently verified and ``understood'' by a human, but no one wants to
waste their time on such a boring, unrewarding chore.



\needspace{4\baselineskip}
\subsection{Trusting the Computer}

\begin{quote}
  {\em \ldots I continue to find the quasi-empirical interpretation of
  computer proofs to be the more plausible.\ldots Since not
  everything that claims to be a computer proof can be
  accepted as valid, what are the mathematical criteria for acceptable
  computer proofs?}
    \flushright\sc  Thomas Tymoczko\footnote{\cite{Tymoczko}, p.~245.}\\
\end{quote}\index{Tymoczko, Thomas}

In some cases, computers have been essential tools for proving famous
theorems.  But if a proof is so long and obscure that it can be verified in a
practical way only with a computer, it is vaguely felt to be suspicious.  For
example, proving the famous four-color theorem\index{four-color
theorem}\index{proof length} (``a map needs no more than four colors to
prevent any two adjacent countries from having the same color'') can presently
only be done with the aid of a very complex computer program which originally
required 1200 hours of computer time. There has been considerable debate about
whether such a proof can be trusted and whether such a proof is ``real''
mathematics \cite{Swart}\index{Swart, E. R.}.\index{trusting computers}

However, under normal circumstances even a skeptical mathematician would have a
great deal of confidence in the result of multiplying two numbers on a pocket
calculator, even though the precise details of what goes on are hidden from its
user.  Even the verification on a supercomputer that a huge number is prime is
trusted, especially if there is independent verification; no one bothers to
debate the philosophical significance of its ``proof,'' even though the actual
proof would be so large that it would be completely impractical to ever write
it down on paper.  It seems that if the algorithm used by the computer is
simple enough to be readily understood, then the computer can be trusted.

Metamath\index{Metamath} adopts this philosophy.  The simplicity of its
language makes it easy to learn, and because of its simplicity one can have
essentially absolute confidence that a proof is correct. All axioms, rules, and
definitions are available for inspection at any time because they are defined
by the user; there are no hidden or built-in rules that may be prone to subtle
bugs\index{computer program bugs}.  The basic algorithm at the heart of
Metamath is simple and fixed, and it can be assumed to be bug-free and robust
with a degree of confidence approaching certainty.
Independently written implementations of the Metamath verifier
can reduce any residual doubt on the part of a skeptic even further;
there are now over a dozen such implementations, written by many people.

\subsection{Trusting the Mathematician}\label{trust}

\begin{quote}
  {\em There is no Algebraist nor Mathematician so expert in his science, as
  to place entire confidence in any truth immediately upon his discovery of it,
  or regard it as any thing, but a mere probability.  Every time he runs over
  his proofs, his confidence encreases; but still more by the approbation of
  his friends; and is rais'd to its utmost perfection by the universal assent
  and applauses of the learned world.}
  \flushright\sc David Hume\footnote{{\em A Treatise of Human Nature}, as
  quoted in \cite{deMillo}, p.~267.}\\
\end{quote}\index{Hume, David}

\begin{quote}
  {\em Stanislaw Ulam estimates that mathematicians publish 200,000 theorems
  every year.  A number of these are subsequently contradicted or otherwise
  disallowed, others are thrown into doubt, and most are ignored.}
  \flushright\sc Richard de Millo et. al.\footnote{\cite{deMillo}, p.~269.}\\
\end{quote}\index{Ulam, Stanislaw}

Whether or not the computer can be trusted, humans  of course will occasionally
err. Only the most memorable proofs get independently verified, and of these
only a handful of truly great ones achieve the status of being ``known''
mathematical truths that are used without giving a second thought to their
correctness.

There are many famous examples of incorrect theorems and proofs in
mathematical literature.\index{errors in proofs}

\begin{itemize}
\item There have been thousands of purported proofs of Fermat's Last
Theorem\index{Fermat's Last Theorem} (``no integer solutions exist to $x^n +
y^n = z^n$ for $n > 2$''), by amateurs, cranks, and well-regarded
mathematicians \cite[p.~5]{Stark}\index{Stark, Harold M}.  Fermat wrote a note
in his copy of Bachet's {\em Diophantus} that he found ``a truly marvelous
proof of this theorem but this margin is too narrow to contain it''
\cite[p.~507]{Kramer}.  A recent, much publicized proof by Yoichi
Miyaoka\index{Miyaoka, Yoichi} was shown to be incorrect ({\em Science News},
April 9, 1988, p.~230).  The theorem was finally proved by Andrew
Wiles\index{Wiles, Andrew} ({\em Science News}, July 3, 1993, p.~5), but it
initially had some gaps and took over a year after its announcement to be
checked thoroughly by experts.  On Oct. 25, 1994, Wiles announced that the last
gap found in his proof had been filled in.
  \item In 1882, M. Pasch discovered that an axiom was omitted from Euclid's
formulation of geometry\index{Euclidean geometry}; without it, the proofs of
important theorems of Euclid are not valid.  Pasch's axiom\index{Pasch's
axiom} states that a line that intersects one side of a triangle must also
intersect another side, provided that it does not touch any of the triangle's
vertices.  The omission of Pasch's axiom went unnoticed for 2000
years \cite[p.~160]{Davis}, in spite of (one presumes) the thousands of
students, instructors, and mathematicians who studied Euclid.
  \item The first published proof of the famous Schr\"{o}der--Bernstein
theorem\index{Schr\"{o}der--Bernstein theorem} in set theory was incorrect
\cite[p.~148]{Enderton}\index{Enderton, Herbert B.}.  This theorem states
that if there exists a 1-to-1 function\footnote{A {\em set}\index{set} is any
collection of objects. A {\em function}\index{function} or {\em
mapping}\index{mapping} is a rule that assigns to each element of one set
(called the function's {\em domain}\index{domain}) an element from another
set.} from set $A$ into set $B$ and vice-versa, then sets $A$ and $B$ have
a 1-to-1 correspondence.  Although it sounds simple and obvious,
the standard proof is quite long and complex.
  \item In the early 1900's, Hilbert\index{Hilbert, David} published a
purported proof of the continuum hypothesis\index{continuum hypothesis}, which
was eventually established as unprovable by Cohen\index{Cohen, Paul} in 1963
\cite[p.~166]{Enderton}.  The continuum hypothesis states that no
infinity\index{infinity} (``transfinite cardinal number'')\index{cardinal,
transfinite} exists whose size (or ``cardinality''\index{cardinality}) is
between the size of the set of integers and the size of the set of real
numbers.  This hypothesis originated with German mathematician Georg
Cantor\index{Cantor, Georg} in the late 1800's, and his inability to prove it
is said to have contributed to mental illness that afflicted him in his later
years.
  \item An incorrect proof of the four-color theorem\index{four-color theorem}
was published by Kempe\index{Kempe, A. B.} in 1879
\cite[p.~582]{Courant}\index{Courant, Richard}; it stood for 11 years before
its flaw was discovered.  This theorem states that any map can be colored
using only four colors, so that no two adjacent countries have the same
color.  In 1976 the theorem was finally proved by the famous computer-assisted
proof of Haken, Appel, and Koch \cite{Swart}\index{Appel, K.}\index{Haken,
W.}\index{Koch, K.}.  Or at least it seems that way.  Mathematician
H.~S.~M.~Coxeter\index{Coxeter, H. S. M.} has doubts \cite[p.~58]{Davis}:  ``I
have a feeling that that is an untidy kind of use of the computers, and the more
you correspond with Haken and Appel, the more shaky you seem to be.''
  \item Many false ``proofs'' of the Poincar\'{e}
conjecture\index{Poincar\'{e} conjecture} have been proposed over the years.
This conjecture states that any object that mathematically behaves like a
three-dimensional sphere is a three-dimensional sphere topologically,
regardless of how it is distorted.  In March 1986, mathematicians Colin
Rourke\index{Rourke, Colin} and Eduardo R\^{e}go\index{R\^{e}go, Eduardo}
caused  a stir in the mathematical community by announcing that they had found
a proof; in November of that year the proof was found to be false \cite[p.
218]{PetersonI}.  It was finally proved in 2003 by Grigory Perelman
\label{poincare}\index{Szpiro, George}\index{Perelman, Grigory}\cite{Szpiro}.
 \end{itemize}

Many counterexamples to ``theorems'' in recent mathematical
literature related to Clifford algebras\index{Clifford algebras}
 have been found by Pertti
Lounesto (who passed away in 2002).\index{Lounesto, Pertti}
See the web page \url{http://mathforum.org/library/view/4933.html}.
% http://users.tkk.fi/~ppuska/mirror/Lounesto/counterexamples.htm

One of the purposes of Metamath\index{Metamath} is to allow proofs to be
expressed with absolute precision.  Developing a proof in the Metamath
language can be challenging, because Metamath will not permit even the
tiniest mistake.\index{errors in proofs}  But once the proof is created, its
correctness can be trusted immediately, without having to depend on the
process of peer review for confirmation.

\section{The Use of Computers in Mathematics}

\subsection{Computer Algebra Systems}

For the most part, you will find that Metamath\index{Metamath} is not a
practical tool for manipulating numbers.  (Even proving that $2 + 2 = 4$, if
you start with set theory, can be quite complex!)  Several commercial
mathematics packages are quite good at arithmetic, algebra, and calculus, and
as practical tools they are invaluable.\index{computer algebra system} But
they have no notion of proof, and cannot understand statements starting with
``there exists such and such...''.

Software packages such as Mathematica \cite{Wolfram}\index{Mathematica} do not
concern themselves with proofs but instead work directly with known results.
These packages primarily emphasize heuristic rules such as the substitution of
equals for equals to achieve simpler expressions or expressions in a different
form.  Starting with a rich collection of built-in rules and algorithms, users
can add to the collection by means of a powerful programming language.
However, results such as, say, the existence of a certain abstract object
without displaying the actual object cannot be expressed (directly) in their
languages.  The idea of a proof from a small set of axioms is absent.  Instead
this software simply assumes that each fact or rule you add to the built-in
collection of algorithms is valid.  One way to view the software is as a large
collection of axioms from which the software, with certain goals, attempts to
derive new theorems, for example equating a complex expression with a simpler
equivalent. But the terms ``theorem''\index{theorem} and
``proof,''\index{proof} for example, are not even mentioned in the index of
the user's manual for Mathematica.\index{Mathematica and proofs}  What is also
unsatisfactory from a philosophical point of view is that there is no way to
ensure the validity of the results other than by trusting the writer of each
application module or tediously checking each module by hand, similar to
checking a computer program for bugs.\index{computer program
bugs}\footnote{Two examples illustrate why the knowledge database of computer
algebra systems should sometimes be regarded with a certain caution.  If you
ask Mathematica (version 3.0) to \texttt{Solve[x\^{ }n + y\^{ }n == z\^{ }n , n]}
it will respond with \texttt{\{\{n-\char`\>-2\}, \{n-\char`\>-1\},
\{n-\char`\>1\}, \{n-\char`\>2\}\}}. In other words, Mathematica seems to
``know'' that Fermat's Last Theorem\index{Fermat's Last Theorem} is true!  (At
the time this version of Mathematica was released this fact was unknown.)  If
you ask Maple\index{Maple} to \texttt{solve(x\^{ }2 = 2\^{ }x)} then
\texttt{simplify(\{"\})}, it returns the solution set \texttt{\{2, 4\}}, apparently
unaware that $-0.7666647$\ldots is also a solution.} While of course extremely
valuable in applied mathematics, computer algebra systems tend to be of little
interest to the theoretical mathematician except as aids for exploring certain
specific problems.

Because of possible bugs, trusting the output of a computer algebra system for
use as theorems in a proof-verifier would defeat the latter's goal of rigor.
On the other hand, a fact such that a certain relatively large number is
prime, while easy for a computer algebra system to derive, might have a long,
tedious proof that could overwhelm a proof-verifier. One approach for linking
computer algebra systems to a proof-verifier while retaining the advantages of
both is to add a hypothesis to each such theorem indicating its source.  For
example, a constant {\sc maple} could indicate the theorem came from the Maple
package, and would mean ``assuming Maple is consistent, then\ldots''  This and
many other topics concerning the formalization of mathematics are discussed in
John Harrison's\index{Harrison, John} very interesting
PhD thesis~\cite{Harrison-thesis}.

\subsection{Automated Theorem Provers}\label{theoremprovers}

A mathematical theory is ``decidable''\index{decidable theory} if a mechanical
method or algorithm exists that is guaranteed to determine whether or not a
particular formula is a theorem.  Among the few theories that are decidable is
elementary geometry,\index{Euclidean geometry} as was shown by a classic
result of logician Alfred Tarski\index{Tarski, Alfred} in 1948
\cite{Tarski}.\footnote{Tarski's result actually applies to a subset of the
geometry discussed in elementary textbooks.  This subset includes most of what
would be considered elementary geometry but it is not powerful enough to
express, among other things, the notions of the circumference and area of a
circle.  Extending the theory in a way that includes notions such as these
makes the theory undecidable, as was also shown by Tarski.  Tarski's algorithm
is far too inefficient to implement practically on a computer.  A practical
algorithm for proving a smaller subset of geometry theorems (those not
involving concepts of ``order'' or ``continuity'') was discovered by Chinese
mathematician Wu Wen-ts\"{u}n in 1977 \cite{Chou}\index{Chou,
Shang-Ching}.}\index{Wen-ts{\"{u}}n, Wu}  But most theories, including
elementary arithmetic, are undecidable.  This fact contributes to keeping
mathematics alive and well, since many mathematicians believe
that they will never be
replaced by computers (if they believe Roger Penrose's argument that a
computer can never replace the brain \cite{Penrose}\index{Penrose, Roger}).
In fact,  elementary geometry is often considered a ``dead'' field
for the simple reason that it is decidable.

On the other hand, the undecidability of a theory does not mean that one cannot
use a computer to search for proofs, providing one is willing to give up if a
proof is not found after a reasonable amount of time.  The field of automated
theorem proving\index{automated theorem proving} specializes in pursuing such
computer searches.  Among the more successful results to date are those based
on an algorithm known as Robinson's resolution principle
\cite{Robinson}\index{Robinson's resolution principle}.

Automated theorem provers can be excellent tools for those willing to learn
how to use them.  But they are not widely used in mainstream pure
mathematics, even though they could probably be useful in many areas.  There
are several reasons for this.  Probably most important, the main goal in pure
mathematics is to arrive at results that are considered to be deep or
important; proving them is essential but secondary.  Usually, an automated
theorem prover cannot assist in this main goal, and by the time the main goal
is achieved, the mathematician may have already figured out the proof as a
by-product.  There is also a notational problem.  Mathematicians are used to
using very compact syntax where one or two symbols (heavily dependent on
context) can represent very complex concepts; this is part of the
hierarchy\index{hierarchy} they have built up to tackle difficult problems.  A
theorem prover on the other hand might require that a theorem be expressed in
``first-order logic,''\index{first-order logic} which is the logic on which
most of mathematics is ultimately based but which is not ordinarily used
directly because expressions can become very long.  Some automated theorem
provers are experimental programs, limited in their use to very specialized
areas, and the goal of many is simply research into the nature of automated
theorem proving itself.  Finally, much research remains to be done to enable
them to prove very deep theorems.  One significant result was a
computer proof by Larry Wos\index{Wos, Larry} and colleagues that every Robbins
algebra\index{Robbins algebra} is a Boolean  algebra\index{Boolean algebra}
({\em New York Times}, Dec. 10, 1996).\footnote{In 1933, E.~V.\
Huntington\index{Huntington, E. V.}
presented the following axiom system for
Boolean algebra with a unary operation $n$ and a binary operation $+$:
\begin{center}
    $x + y = y + x$ \\
    $(x + y) + z = x + (y + z)$ \\
    $n(n(x) + y) + n(n(x) + n(y)) = x$
\end{center}
Herbert Robbins\index{Robbins, Herbert}, a student of Huntington, conjectured
that the last equation can be replaced with a simpler one:
\begin{center}
    $n(n(x + y) + n(x + n(y))) = x$
\end{center}
Robbins and Huntington could not find a proof.  The problem was
later studied unsuccessfully by Tarski\index{Tarski, Alfred} and his
students, and it remained an unsolved problem until a
computer found the proof in 1996.  For more information on
the Robbins algebra problem see \cite{Wos}.}

How does Metamath\index{Metamath} relate to automated theorem provers?  A
theorem prover is primarily concerned with one theorem at a time (perhaps
tapping into a small database of known theorems) whereas Metamath is more like
a theorem archiving system, storing both the theorem and its proof in a
database for access and verification.  Metamath is one answer to ``what do you
do with the output of a theorem prover?''  and could be viewed as the
next step in the process.  Automated theorem provers could be useful tools for
helping develop its database.
Note that very long, automatically
generated proofs can make your database fat and ugly and cause Metamath's proof
verification to take a long time to run.  Unless you have a particularly good
program that generates very concise proofs, it might be best to consider the
use of automatically generated proofs as a quick-and-dirty approach, to be
manually rewritten at some later date.

The program {\sc otter}\index{otter@{\sc otter}}\footnote{\url{http://www.cs.unm.edu/\~mccune/otter/}.}, later succeeded by
prover9\index{prover9}\footnote{\url{https://www.cs.unm.edu/~mccune/mace4/}.},
have been historically influential.
The E prover\index{E prover}\footnote{\url{https://github.com/eprover/eprover}.}
is a maintained automated theorem prover
for full first-order logic with equality.
There are many other automated theorem provers as well.

If you want to combine automated theorem provers with Metamath
consider investigating
the book {\em Automated Reasoning:  Introduction and Applications}
\cite{Wos}\index{Wos, Larry}.  This book discusses
how to use {\sc otter} in a way that can
not only able to generate
relatively efficient proofs, it can even be instructed to search for
shorter proofs.  The effective use of {\sc otter} (and similar tools)
does require a certain
amount of experience, skill, and patience.  The axiom system used in the
\texttt{set.mm}\index{set theory database (\texttt{set.mm})} set theory
database can be expressed to {\sc otter} using a method described in
\cite{Megill}.\index{Megill, Norman}\footnote{To use those axioms with
{\sc otter}, they must be restated in a way that eliminates the need for
``dummy variables.''\index{dummy variable!eliminating} See the Comment
on p.~\pageref{nodd}.} When successful, this method tends to generate
short and clever proofs, but my experiments with it indicate that the
method will find proofs within a reasonable time only for relatively
easy theorems.  It is still fun to experiment with.

Reference \cite{Bledsoe}\index{Bledsoe, W. W.} surveys a number of approaches
people have explored in the field of automated theorem proving\index{automated
theorem proving}.

\subsection{Interactive Theorem Provers}\label{interactivetheoremprovers}

Finding proofs completely automatically is difficult, so there
are some interactive theorem provers that allow a human to guide the
computer to find a proof.
Examples include
HOL Light\index{HOL light}%
\footnote{\url{https://www.cl.cam.ac.uk/~jrh13/hol-light/}.},
Isabelle\index{Isabelle}%
\footnote{\url{http://www.cl.cam.ac.uk/Research/HVG/Isabelle}.},
{\sc hol}\index{hol@{\sc hol}}%
\footnote{\url{https://hol-theorem-prover.org/}.},
and
Coq\index{Coq}\footnote{\url{https://coq.inria.fr/}.}.

A major difference between most of these tools and Metamath is that the
``proofs'' are actually programs that guide the program to find a proof,
and not the proof itself.
For example, an Isabelle/HOL proof might apply a step
\texttt{apply (blast dest: rearrange reduction)}. The \texttt{blast}
instruction applies
an automatic tableux prover and returns if it found a sequence of proof
steps that work... but the sequence is not considered part of the proof.

A good overview of
higher-level proof verification languages (such as {\sc lcf}\index{lcf@{\sc
lcf}} and {\sc hol}\index{hol@{\sc hol}})
is given in \cite{Harrison}.  All of these languages are fundamentally
different from Metamath in that much of the mathematical foundational
knowledge is embedded in the underlying proof-verification program, rather
than placed directly in the database that is being verified.
These can have a steep learning curve for those without a mathematical
background.  For example, one usually must have a fair understanding of
mathematical logic in order to follow their proofs.

\subsection{Proof Verifiers}\label{proofverifiers}

A proof verifier is a program that doesn't generate proofs but instead
verifies proofs that you give it.  Many proof verifiers have limited built-in
automated proof capabilities, such as figuring out simple logical inferences
(while still being guided by a person who provides the overall proof).  Metamath
has no built-in automated proof capability other than the limited
capability of its Proof Assistant.

Proof-verification languages are not used as frequently as they might be.
Pure mathematicians are more concerned with producing new results, and such
detail and rigor would interfere with that goal.  The use of computers in pure
mathematics is primarily focused on automated theorem provers (not verifiers),
again with the ultimate goal of aiding the creation of new mathematics.
Automated theorem provers are usually concerned with attacking one theorem at
time rather than making a large, organized database easily available to the
user.  Metamath is one way to help close this gap.

By itself Metamath is a mostly a proof verifier.
This does not mean that other approaches can't be used; the difference
is that in Metamath, the results of various provers must be recorded
step-by-step so that they can be verified.

Another proof-verification language is Mizar,\index{Mizar} which can display
its proofs in the informal language that mathematicians are accustomed to.
Information on the Mizar language is available at \url{http://mizar.org}.

For the working mathematician, Mizar is an excellent tool for rigorously
documenting proofs. Mizar typesets its proofs in the informal English used by
mathematicians (and, while fine for them, are just as inscrutable by
laypersons!). A price paid for Mizar is a relatively steep learning curve of a
couple of weeks.  Several mathematicians are actively formalizing different
areas of mathematics using Mizar and publishing the proofs in a dedicated
journal. Unfortunately the task of formalizing mathematics is still looked
down upon to a certain extent since it doesn't involve the creation of ``new''
mathematics.

The closest system to Metamath is
the {\em Ghilbert}\index{Ghilbert} proof language (\url{http://ghilbert.org})
system developed by
Raph Levien\index{Levien, Raph}.
Ghilbert is a formal proof checker heavily inspired by Metamath.
Ghilbert statements are s-expressions (a la Lisp), which is easy
for computers to parse but many people find them hard to read.
There are a number of differences in their specific constructs, but
there is at least one tool to translate some Metamath materials into Ghilbert.
As of 2019 the Ghilbert community is smaller and less active than the
Metamath community.
That said, the Metamath and Ghilbert communities overlap, and fruitful
conversations between them have occurred many times over the years.

\subsection{Creating a Database of Formalized Mathematics}\label{mathdatabase}

Besides Metamath, there are several other ongoing projects with the goal of
formalizing mathematics into computer-verifiable databases.
Understanding some history will help.

The {\sc qed}\index{qed project@{\sc qed} project}%
\footnote{\url{http://www-unix.mcs.anl.gov/qed}.}
project arose in 1993 and its goals were outlined in the
{\sc qed} manifesto.
The {\sc qed} manifesto was
a proposal for a computer-based database of all mathematical knowledge,
strictly formalized and with all proofs having been checked automatically.
The project had a conference in 1994 and another in 1995;
there was also a ``twenty years of the {\sc qed} manifesto'' workshop
in 2014.
Its ideals are regularly reraised.

In a 2007 paper, Freek Wiedijk identified two reasons
for the failure of the {\sc qed} project as originally envisioned:%
\cite{Wiedijk-revisited}\index{Wiedijk, Freek}

\begin{itemize}
\item Very few people are working on formalization of mathematics. There is no compelling application for fully mechanized mathematics.
\item Formalized mathematics does not yet resemble traditional mathematics. This is partly due to the complexity of mathematical notation, and partly to the limitations of existing theorem provers and proof assistants.
\end{itemize}

But this did not end the dream of
formalizing mathematics into computer-verifiable databases.
The problems that led to the {\sc qed} manifesto are still with us,
even though the challenges were harder than originally considered.
What has happened instead is that various independent projects have
worked towards formalizing mathematics into computer-verifiable databases,
each simultaneously competing and cooperating with each other.

A concrete way to see this is
Freek Wiedijk's ``Formalizing 100 Theorems'' list%
\footnote{\url{http://www.cs.ru.nl/\%7Efreek/100/}.}
which shows the progress different systems have made on a challenge list
of 100 mathematical theorems.%
\footnote{ This is not the only list of ``interesting'' theorems.
Another interesting list was posted by Oliver Knill's list
\cite{Knill}\index{Knill, Oliver}.}
The top systems as of February 2019
(in order of the number of challenges completed) are
HOL Light, Isabelle, Metamath, Coq, and Mizar.

The Metamath 100%
\footnote{\url{http://us.metamath.org/mm\_100.html}}
page (maintained by David A. Wheeler\index{Wheeler, David A.})
shows the progress of Metamath (specifically its \texttt{set.mm} database)
against this challenge list maintained by Freek Wiedijk.
The Metamath \texttt{set.mm} database
has made a lot of progress over the years,
in part because working to prove those challenge theorems required
defining various terms and proving their properties as a prerequisite.
Here are just a few of the many statements that have been
formally proven with Metamath:

% The entries of this cause the narrow display to break poorly,
% since the short amount of text means LaTeX doesn't get a lot to work with
% and the itemize format gives it even *less* margin than usual.
% No one will mind if we make just this list flushleft, since this list
% will be internally consistent.
\begin{flushleft}
\begin{itemize}
\item 1. The Irrationality of the Square Root of 2
  (\texttt{sqr2irr}, by Norman Megill, 2001-08-20)
\item 2. The Fundamental Theorem of Algebra
  (\texttt{fta}, by Mario Carneiro, 2014-09-15)
\item 22. The Non-Denumerability of the Continuum
  (\texttt{ruc}, by Norman Megill, 2004-08-13)
\item 54. The Konigsberg Bridge Problem
  (\texttt{konigsberg}, by Mario Carneiro, 2015-04-16)
\item 83. The Friendship Theorem
  (\texttt{friendship}, by Alexander W. van der Vekens, 2018-10-09)
\end{itemize}
\end{flushleft}

We thank all of those who have developed at least one of the Metamath 100
proofs, and we particularly thank
Mario Carneiro\index{Carneiro, Mario}
who has contributed the most Metamath 100 proofs as of 2019.
The Metamath 100 page shows the list of all people who have contributed a
proof, and links to graphs and charts showing progress over time.
We encourage others to work on proving theorems not yet proven in Metamath,
since doing so improves the work as a whole.

Each of the math formalization systems (including Metamath)
has different strengths and weaknesses, depending on what you value.
Key aspects that differentiate Metamath from the other top systems are:

\begin{itemize}
\item Metamath is not tied to any particular set of axioms.
\item Metamath can show every step of every proof, no exceptions.
  Most other provers only assert that a proof can be found, and do not
  show every step. This also makes verification fast, because
  the system does not need to rediscover proof details.
\item The Metamath verifier has been re-implemented in many different
  programming languages, so verification can be done by multiple
  implementations.  In particular, the
  \texttt{set.mm}\index{set theory database (\texttt{set.mm})}%
  \index{Metamath Proof Explorer} database is verified by
  four different verifiers
  written in four different languages by four different authors.
  This greatly reduces the risk of accepting an invalid
  proof due to an error in the verifier.
\item Proofs stay proven.  In some systems, changes to the system's
  syntax or how a tactic works causes proofs to fail in later versions,
  causing older work to become essentially lost.
  Metamath's language is
  extremely small and fixed, so once a proof is added to a database,
  the database can be rechecked with later versions of the Metamath program
  and with other verifiers of Metamath databases.
  If an axiom or key definition needs to be changed, it is easy to
  manipulate the database as a whole to handle the change
  without touching the underlying verifier.
  Since re-verification of an entire database takes seconds, there
  is never a reason to delay complete verification.
  This aspect is especially compelling if your
  goal is to have a long-term database of proofs.
\item Licensing is generous.  The main Metamath databases are released to
  the public domain, and the main Metamath program is open source software
  under a standard, widely-used license.
\item Substitutions are easy to understand, even by those who are not
  professional mathematicians.
\end{itemize}

Of course, other systems may have advantages over Metamath
that are more compelling, depending on what you value.
In any case, we hope this helps you understand Metamath
within a wider context.

\subsection{In Summary}\label{computers-summary}

To summarize our discussions of computers and mathematics, computer algebra
systems can be viewed as theorem generators focusing on a narrow realm of
mathematics (numbers and their properties), automated theorem provers as proof
generators for specific theorems in a much broader realm covered by a built-in
formal system such as first-order logic, interactive theorem
provers require human guidance, proof verifiers verify proofs but
historically they have been
restricted to first-order logic.
Metamath, in contrast,
is a proof verifier and documenter whose realm is essentially unlimited.

\section{Mathematics and Metamath}

\subsection{Standard Mathematics}

There are a number of ways that Metamath\index{Metamath} can be used with
standard mathematics.  The most satisfying way philosophically is to start at
the very beginning, and develop the desired mathematics from the axioms of
logic and set theory.\index{set theory}  This is the approach taken in the
\texttt{set.mm}\index{set theory database (\texttt{set.mm})}%
\index{Metamath Proof Explorer}
database (also known as the Metamath Proof Explorer).
Among other things, this database builds up to the
axioms of real and complex numbers\index{analysis}\index{real and complex
numbers} (see Section~\ref{real}), and a standard development of analysis, for
example, could start at that point, using it as a basis.   Besides this
philosophical advantage, there are practical advantages to having all of the
tools of set theory available in the supporting infrastructure.

On the other hand, you may wish to start with the standard axioms of a
mathematical theory without going through the set theoretical proofs of those
axioms.  You will need mathematical logic to make inferences, but if you wish
you can simply introduce theorems\index{theorem} of logic as
``axioms''\index{axiom} wherever you need them, with the implicit assumption
that in principle they can be proved, if they are obvious to you.  If you
choose this approach, you will probably want to review the notation used in
\texttt{set.mm}\index{set theory database (\texttt{set.mm})} so that your own
notation will be consistent with it.

\subsection{Other Formal Systems}
\index{formal system}

Unlike some programs, Metamath\index{Metamath} is not limited to any specific
area of mathematics, nor committed to any particular mathematical philosophy
such as classical logic versus intuitionism, nor limited, say, to expressions
in first-order logic.  Although the database \texttt{set.mm}
describes standard logic and set theory, Meta\-math
is actually a general-purpose language for describing a wide variety of formal
systems.\index{formal system}  Non-standard systems such as modal
logic,\index{modal logic} intuitionist logic\index{intuitionism}, higher-order
logic\index{higher-order logic}, quantum logic\index{quantum logic}, and
category theory\index{category theory} can all be described with the Metamath
language.  You define the symbols you prefer and tell Metamath the axioms and
rules you want to start from, and Metamath will verify any inferences you make
from those axioms and rules.  A simple example of a non-standard formal system
is Hofstadter's\index{Hofstadter, Douglas R.} MIU system,\index{MIU-system}
whose Metamath description is presented in Appendix~\ref{MIU}.

This is not hypothetical.
The largest Metamath database is
\texttt{set.mm}\index{set theory database (\texttt{set.mm}}%
\index{Metamath Proof Explorer}), aka the Metamath Proof Explorer,
which uses the most common axioms for mathematical foundations
(specifically classical logic combined with Zermelo--Fraenkel
set theory\index{Zermelo--Fraenkel set theory} with the Axiom of Choice).
But other Metamath databases are available:

\begin{itemize}
\item The database
  \texttt{iset.mm}\index{intuitionistic logic database (\texttt{iset.mm})},
  aka the
  Intuitionistic Logic Explorer\index{Intuitionistic Logic Explorer},
  uses intuitionistic logic (a constructivist point of view)
  instead of classical logic.
\item The database
  \texttt{nf.mm}\index{New Foundations database (\texttt{nf.mm})},
  aka the
  New Foundations Explorer\index{New Foundations Explorer},
  constructs mathematics from scratch,
  starting from Quine's New Foundations (NF) set theory axioms.
\item The database
  \texttt{hol.mm}\index{Higher-order Logic database (\texttt{hol.mm})},
  aka the
  Higher-Order Logic (HOL) Explorer\index{Higher-Order Logic (HOL) Explorer},
  starts with HOL (also called simple type theory) and derives
  equivalents to ZFC axioms, connecting the two approaches.
\end{itemize}

Since the days of David Hilbert,\index{Hilbert, David} mathematicians have
been concerned with the fact that the metalanguage\index{metalanguage} used to
describe mathematics may be stronger than the mathematics being described.
Metamath\index{Metamath}'s underlying finitary\index{finitary proof},
constructive nature provides a good philosophical basis for studying even the
weakest logics.\index{weak logic}

The usual treatment of many non-standard formal systems\index{formal
system} uses model theory\index{model theory} or proof theory\index{proof
theory} to describe these systems; these theories, in turn, are based on
standard set theory.  In other words, a non-standard formal system is defined
as a set with certain properties, and standard set theory is used to derive
additional properties of this set.  The standard set theory database provided
with Metamath can be used for this purpose, and when used this way
the development of a special
axiom system for the non-standard formal system becomes unnecessary.  The
model- or proof-theoretic approach often allows you to prove much deeper
results with less effort.

Metamath supports both approaches.  You can define the non-standard
formal system directly, or define the non-standard formal system as
a set with certain properties, whichever you find most helpful.

%\section{Additional Remarks}

\subsection{Metamath and Its Philosophy}

Closely related to Metamath\index{Metamath} is a philosophy or way of looking
at mathematics. This philosophy is related to the formalist
philosophy\index{formalism} of Hilbert\index{Hilbert, David} and his followers
\cite[pp.~1203--1208]{Kline}\index{Kline, Morris}
\cite[p.~6]{Behnke}\index{Behnke, H.}. In this philosophy, mathematics is
viewed as nothing more than a set of rules that manipulate symbols, together
with the consequences of those rules.  While the mathematics being described
may be complex, the rules used to describe it (the
``metamathematics''\index{metamathematics}) should be as simple as possible.
In particular, proofs should be restricted to dealing with concrete objects
(the symbols we write on paper rather than the abstract concepts they
represent) in a constructive manner; these are called ``finitary''
proofs\index{finitary proof} \cite[pp.~2--3]{Shoenfield}\index{Shoenfield,
Joseph R.}.

Whether or not you find Metamath interesting or useful will in part depend on
the appeal you find in its philosophy, and this appeal will probably depend on
your particular goals with respect to mathematics.  For example, if you are a
pure mathematician at the forefront of discovering new mathematical knowledge,
you will probably find that the rigid formality of Metamath stifles your
creativity.  On the other hand, we would argue that once this knowledge is
discovered, there are advantages to documenting it in a standard format that
will make it accessible to others.  Sixty years from now, your field may be
dormant, and as Davis and Hersh put it, your ``writings would become less
translatable than those of the Maya'' \cite[p.~37]{Davis}\index{Davis, Phillip
J.}.


\subsection{A History of the Approach Behind Metamath}

Probably the one work that has had the most motivating influence on
Metamath\index{Metamath} is Whitehead and Russell's monumental {\em Principia
Mathematica} \cite{PM}\index{Whitehead, Alfred North}\index{Russell,
Bertrand}\index{principia mathematica@{\em Principia Mathematica}}, whose aim
was to deduce all of mathematics from a small number of primitive ideas, in a
very explicit way that in principle anyone could understand and follow.  While
this work was tremendously influential in its time, from a modern perspective
it suffers from several drawbacks.  Both its notation and its underlying
axioms are now considered dated and are no longer used.  From our point of
view, its development is not really as accessible as we would like to see; for
practical reasons, proofs become more and more sketchy as its mathematics
progresses, and working them out in fine detail requires a degree of
mathematical skill and patience that many people don't have.  There are
numerous small errors, which is understandable given the tedious, technical
nature of the proofs and the lack of a computer to verify the details.
However, even today {\em Principia Mathematica} stands out as the work closest
in spirit to Metamath.  It remains a mind-boggling work, and one can't help
but be amazed at seeing ``$1+1=2$'' finally appear on page 83 of Volume II
(Theorem *110.643).

The origin of the proof notation used by Metamath dates back to the 1950's,
when the logician C.~A.~Meredith expressed his proofs in a compact notation
called ``condensed detachment''\index{condensed detachment}
\cite{Hindley}\index{Hindley, J. Roger} \cite{Kalman}\index{Kalman, J. A.}
\cite{Meredith}\index{Meredith, C. A.} \cite{Peterson}\index{Peterson, Jeremy
George}.  This notation allows proofs to be communicated unambiguously by
merely referencing the axiom\index{axiom}, rule\index{rule}, or
theorem\index{theorem} used at each step, without explicitly indicating the
substitutions\index{substitution!variable}\index{variable substitution} that
have to be made to the variables in that axiom, rule, or theorem.  Ordinarily,
condensed detachment is more or less limited to propositional
calculus\index{propositional calculus}.  The concept has been extended to
first-order logic\index{first-order logic} in \cite{Megill}\index{Megill,
Norman}, making it is easy to write a small computer program to verify proofs
of simple first-order logic theorems.\index{condensed detachment!and
first-order logic}

A key concept behind the notation of condensed detachment is called
``unification,''\index{unification} which is an algorithm for determining what
substitutions\index{substitution!variable}\index{variable substitution} to
variables have to be made to make two expressions match each other.
Unification was first precisely defined by the logician J.~A.~Robinson, who
used it in the development of a powerful
theorem-proving technique called the ``resolution principle''
\cite{Robinson}\index{Robinson's resolution principle}. Metamath does not make
use of the resolution principle, which is intended for systems of first-order
logic.\index{first-order logic}  Metamath's use is not restricted to
first-order logic, and as we have mentioned it does not automatically discover
proofs.  However, unification is a key idea behind Metamath's proof
notation, and Metamath makes use of a very simple version of it
(Section~\ref{unify}).

\subsection{Metamath and First-Order Logic}

First-order logic\index{first-order logic} is the supporting structure
for standard mathematics.  On top of it is set theory, which contains
the axioms from which virtually all of mathematics can be derived---a
remarkable fact.\index{category
theory}\index{cardinal, inaccessible}\label{categoryth}\footnote{An exception seems
to be category theory.  There are several schools of thought on whether
category theory is derivable from set theory.  At a minimum, it appears
that an additional axiom is needed that asserts the existence of an
``inaccessible cardinal'' (a type of infinity so large that standard set
theory can't prove or deny that it exists).
%
%%%% (I took this out that was in previous editions:)
% But it is also argued that not just one but a ``proper class'' of them
% is needed, and the existence of proper classes is impossible in standard
% set theory.  (A proper class is a collection of sets so huge that no set
% can contain it as an element.  Proper classes can lead to
% inconsistencies such as ``Russell's paradox.''  The axioms of standard
% set theory are devised so as to deny the existence of proper classes.)
%
For more information, see
\cite[pp.~328--331]{Herrlich}\index{Herrlich, Horst} and
\cite{Blass}\index{Blass, Andrea}.}

One of the things that makes Metamath\index{Metamath} more practical for
first-order theories is a set of axioms for first-order logic designed
specifically with Metamath's approach in mind.  These are included in
the database \texttt{set.mm}\index{set theory database (\texttt{set.mm})}.
See Chapter~\ref{fol} for a detailed
description; the axioms are shown in Section~\ref{metaaxioms}.  While
logically equivalent to standard axiom systems, our axiom system breaks
up the standard axioms into smaller pieces such that from them, you can
directly derive what in other systems can only be derived as higher-level
``metatheorems.''\index{metatheorem}  In other words, it is more powerful than
the standard axioms from a metalogical point of view.  A rigorous
justification for this system and its ``metalogical
completeness''\index{metalogical completeness} is found in
\cite{Megill}\index{Megill, Norman}.  The system is closely related to a
system developed by Monk\index{Monk, J. Donald} and Tarski\index{Tarski,
Alfred} in 1965 \cite{Monks}.

For example, the formula $\exists x \, x = y $ (given $y$, there exists some
$x$ equal to it) is a theorem of logic,\footnote{Specifically, it is a theorem
of those systems of logic that assume non-empty domains.  It is not a theorem
of more general systems that include the empty domain\index{empty domain}, in
which nothing exists, period!  Such systems are called ``free
logics.''\index{free logic} For a discussion of these systems, see
\cite{Leblanc}\index{Leblanc, Hugues}.  Since our use for logic is as a basis
for set theory, which has a non-empty domain, it is more convenient (and more
traditional) to use a less general system.  An interesting curiosity is that,
using a free logic as a basis for Zermelo--Fraenkel set
theory\index{Zermelo--Fraenkel set theory} (with the redundant Axiom of the
Null Set omitted),\index{Axiom of the Null Set} we cannot even prove the
existence of a single set without assuming the axiom of infinity!\index{Axiom
of Infinity}} whether or not $x$ and $y$ are distinct variables\index{distinct
variables}.  In many systems of logic, we would have to prove two theorems to
arrive at this result.  First we would prove ``$\exists x \, x = x $,'' then
we would separately prove ``$\exists x \, x = y $, where $x$ and $y$ are
distinct variables.''  We would then combine these two special cases ``outside
of the system'' (i.e.\ in our heads) to be able to claim, ``$\exists x \, x =
y $, regardless of whether $x$ and $y$ are distinct.''  In other words, the
combination of the two special cases is a
metatheorem.  In the system of logic
used in Metamath's set theory\index{set theory database (\texttt{set.mm})}
database, the axioms of logic are broken down into small pieces that allow
them to be reassembled in such a way that theorems such as these can be proved
directly.

Breaking down the axioms in this way makes them look peculiar and not very
intuitive at first, but rest assured that they are correct and complete.  Their
correctness is ensured because they are theorem schemes of standard first-order
logic (which you can easily verify if you are a logician).  Their completeness
follows from the fact that we explicitly derive the standard axioms of
first-order logic as theorems.  Deriving the standard axioms is somewhat
tricky, but once we're there, we have at our disposal a system that is less
awkward to work with in formal proofs\index{formal proof}.  In technical terms
that logicians understand, we eliminate the cumbersome concepts of ``free
variable,''\index{free variable} ``bound variable,''\index{bound variable} and
``proper substitution''\index{proper substitution}\index{substitution!proper}
as primitive notions.  These concepts are present in our system but are
defined in terms of concepts expressed by the axioms and can be eliminated in
principle.  In standard systems, these concepts are really like additional,
implicit axioms\index{implicit axiom} that are somewhat complex and cannot be
eliminated.

The traditional approach to logic, wherein free variables and proper
substitution is defined, is also possible to do directly in the Metamath
language.  However, the notation tends to become awkward, and there are
disadvantages:  for example, extending the definition of a wff with a
definition is awkward, because the free variable and proper substitution
concepts have to have their definitions also extended.  Our choice of
axioms for \texttt{set.mm} is to a certain extent a matter of style, in
an attempt to achieve overall simplicity, but you should also be aware
that the traditional approach is possible as well if you should choose
it.

\chapter{Using the Metamath Program}
\label{using}

\section{Installation}

The way that you install Metamath\index{Metamath!installation} on your
computer system will vary for different computers.  Current
instructions are provided with the Metamath program download at
\url{http://metamath.org}.  In general, the installation is simple.
There is one file containing the Metamath program itself.  This file is
usually called \texttt{metamath} or \texttt{metamath.}{\em xxx} where
{\em xxx} is the convention (such as \texttt{exe}) for an executable
program on your operating system.  There are several additional files
containing samples of the Metamath language, all ending with
\texttt{.mm}.  The file \texttt{set.mm}\index{set theory database
(\texttt{set.mm})} contains logic and set theory and can be used as a
starting point for other areas of mathematics.

You will also need a text editor\index{text editor} capable of editing plain
{\sc ascii}\footnote{American Standard Code for Information Interchange.} text
in order to prepare your input files.\index{ascii@{\sc ascii}}  Most computers
have this capability built in.  Note that plain text is not necessarily the
default for some word processing programs\index{word processor}, especially if
they can handle different fonts; for example, with Microsoft Word\index{Word
(Microsoft)}, you must save the file in the format ``Text Only With Line
Breaks'' to get a plain text\index{plain text} file.\footnote{It is
recommended that all lines in a Metamath source file be 79 characters or less
in length for compatibility among different computer terminals.  When creating
a source file on an editor such as Word, select a monospaced
font\index{monospaced font} such as Courier\index{Courier font} or
Monaco\index{Monaco font} to make this easier to achieve.  Better yet,
just use a plain text editor such as Notepad.}

On some computer systems, Metamath does not have the capability to print
its output directly; instead, you send its output to a file (using the
\texttt{open} commands described later).  The way you print this output
file depends on your computer.\index{printers} Some computers have a
print command, whereas with others, you may have to read the file into
an editor and print it from there.

If you want to print your Metamath source files with typeset formulas
containing standard mathematical symbols, you will need the \LaTeX\
typesetting program\index{latex@{\LaTeX}}, which is widely and freely
available for most operating systems.  It runs natively on Unix and
Linux, and can be installed on Windows as part of the free Cygwin
package (\url{http://cygwin.com}).

You can also produce {\sc html}\footnote{HyperText Markup Language.}
web pages.  The {\tt help html} command in the Metamath program will
assist you with this feature.

\section{Your First Formal System}\label{start}
\subsection{From Nothing to Zero}\label{startf}

To give you a feel for what the Metamath\index{Metamath} language looks like,
we will take a look at a very simple example from formal number
theory\index{number theory}.  This example is taken from
Mendelson\index{Mendelson, Elliot} \cite[p. 123]{Mendelson}.\footnote{To keep
the example simple, we have changed the formalism slightly, and what we call
axioms\index{axiom} are strictly speaking theorems\index{theorem} in
\cite{Mendelson}.}  We will look at a small subset of this theory, namely that
part needed for the first number theory theorem proved in \cite{Mendelson}.

First we will look at a standard formal proof\index{formal proof} for the
example we have picked, then we will look at the Metamath version.  If you
have never been exposed to formal proofs, the notation may seem to be such
overkill to express such simple notions that you may wonder if you are missing
something.  You aren't.  The concepts involved are in fact very simple, and a
detailed breakdown in this fashion is necessary to express the proof in a way
that can be verified mechanically.  And as you will see, Metamath breaks the
proof down into even finer pieces so that the mechanical verification process
can be about as simple as possible.

Before we can introduce the axioms\index{axiom} of the theory, we must define
the syntax rules for forming legal expressions\index{syntax rules}
(combinations of symbols) with which those axioms can be used. The number 0 is
a {\bf term}\index{term}; and if $ t$ and $r$ are terms, so is $(t+r)$. Here,
$ t$ and $r$ are ``metavariables''\index{metavariable} ranging over terms; they
themselves do not appear as symbols in an actual term.  Some examples of
actual terms are $(0 + 0)$ and $((0+0)+0)$.  (Note that our theory describes
only the number zero and sums of zeroes.  Of course, not much can be done with
such a trivial theory, but remember that we have picked a very small subset of
complete number theory for our example.  The important thing for you to focus
on is our definitions that describe how symbols are combined to form valid
expressions, and not on the content or meaning of those expressions.) If $ t$
and $r$ are terms, an expression of the form $ t=r$ is a {\bf wff}
(well-formed formula)\index{well-formed formula (wff)}; and if $P$ and $Q$ are
wffs, so is $(P\rightarrow Q)$ (which means ``$P$ implies
$Q$''\index{implication ($\rightarrow$)} or ``if $P$ then $Q$'').
Here $P$ and $Q$ are metavariables ranging over wffs.  Examples of actual
wffs are $0=0$, $(0+0)=0$, $(0=0 \rightarrow (0+0)=0)$, and $(0=0\rightarrow
(0=0\rightarrow 0=(0+0)))$.  (Our notation makes use of more parentheses than
are customary, but the elimination of ambiguity this way simplifies our
example by avoiding the need to define operator precedence\index{operator
precedence}.)

The {\bf axioms}\index{axiom} of our theory are all wffs of the following
form, where $ t$, $r$, and $s$ are any terms:

%Latex p. 92
\renewcommand{\theequation}{A\arabic{equation}}

\begin{equation}
(t=r\rightarrow (t=s\rightarrow r=s))
\end{equation}
\begin{equation}
(t+0)=t
\end{equation}

Note that there are an infinite number of axioms since there are an infinite
number of possible terms.  A1 and A2 are properly called ``axiom
schemes,''\index{axiom scheme} but we will refer to them as ``axioms'' for
brevity.

An axiom is a {\bf theorem}; and if $P$ and $(P\rightarrow Q)$ are theorems
(where $P$ and $Q$ are wffs), then $Q$ is also a theorem.\index{theorem}  The
second part of this definition is called the modus ponens (MP) rule of
inference\index{inference rule}\index{modus ponens}.  It allows us to obtain
new theorems from old ones.

The {\bf proof}\index{proof} of a theorem is a sequence of one or more
theorems, each of which is either an axiom or the result of modus ponens
applied to two previous theorems in the sequence, and the last of which is the
theorem being proved.

The theorem we will prove for our example is very simple:  $ t=t$.  The proof of
our theorem follows.  Study it carefully until you feel sure you
understand it.\label{zeroproof}

% Use tabu so that lines will wrap automatically as needed.
\begin{tabu} { l X X }
1. & $(t+0)=t$ & (by axiom A2) \\
2. & $(t+0)=t$ & (by axiom A2) \\
3. & $((t+0)=t \rightarrow ((t+0)=t\rightarrow t=t))$ & (by axiom A1) \\
4. & $((t+0)=t\rightarrow t=t)$ & (by MP applied to steps 2 and 3) \\
5. & $t=t$ & (by MP applied to steps 1 and 4) \\
\end{tabu}

(You may wonder why step 1 is repeated twice.  This is not necessary in the
formal language we have defined, but in Metamath's ``reverse Polish
notation''\index{reverse Polish notation (RPN)} for proofs, a previous step
can be referred to only once.  The repetition of step~1 here will enable you
to see more clearly the correspondence of this proof with the
Metamath\index{Metamath} version on p.~\pageref{demoproof}.)

Our theorem is more properly called a ``theorem scheme,''\index{theorem
scheme} for it represents an infinite number of theorems, one for each
possible term $ t$.  Two examples of actual theorems would be $0=0$ and
$(0+0)=(0+0)$.  Rarely do we prove actual theorems, since by proving schemes
we can prove an infinite number of theorems in one fell swoop.  Similarly, our
proof should really be called a ``proof scheme.''\index{proof scheme}  To
obtain an actual proof, pick an actual term to use in place of $ t$, and
substitute it for $ t$ throughout the proof.

Let's discuss what we have done here.  The axioms\index{axiom} of our theory,
A1 and A2, are trivial and obvious.  Everyone knows that adding zero to
something doesn't change it, and also that if two things are equal to a third,
then they are equal to each other. In fact, stating the trivial and obvious is
a goal to strive for in any axiomatic system.  From trivial and obvious truths
that everyone agrees upon, we can prove results that are not so obvious yet
have absolute faith in them.  If we trust the axioms and the rules, we must,
by definition, trust the consequences of those axioms and rules, if logic is
to mean anything at all.

Our rule of inference\index{rule}, modus ponens\index{modus ponens}, is also
pretty obvious once you understand what it means.  If we prove a fact $P$, and
we also prove that $P$ implies $Q$, then $Q$ necessarily follows as a new
fact.  The rule provides us with a means for obtaining new facts (i.e.\
theorems\index{theorem}) from old ones.

The theorem that we have proved, $ t=t$, is so fundamental that you may wonder
why it isn't one of the axioms\index{axiom}.  In some axiom systems of
arithmetic, it {\em is} an axiom.  The choice of axioms in a theory is to some
extent arbitrary and even an art form, constrained only by the requirement
that any two equivalent axiom systems be able to derive each other as
theorems.  We could imagine that the inventor of our axiom system originally
included $ t=t$ as an axiom, then discovered that it could be derived as a
theorem from the other axioms.  Because of this, it was not necessary to
keep it as an axiom.  By eliminating it, the final set of axioms became
that much simpler.

Unless you have worked with formal proofs\index{formal proof} before, it
probably wasn't apparent to you that $ t=t$ could be derived from our two
axioms until you saw the proof. While you certainly believe that $ t=t$ is
true, you might not be able to convince an imaginary skeptic who believes only
in our two axioms until you produce the proof.  Formal proofs such as this are
hard to come up with when you first start working with them, but after you get
used to them they can become interesting and fun.  Once you understand the
idea behind formal proofs you will have grasped the fundamental principle that
underlies all of mathematics.  As the mathematics becomes more sophisticated,
its proofs become more challenging, but ultimately they all can be broken down
into individual steps as simple as the ones in our proof above.

Mendelson's\index{Mendelson, Elliot} book, from which our example was taken,
contains a number of detailed formal proofs such as these, and you may be
interested in looking it up.  The book is intended for mathematicians,
however, and most of it is rather advanced.  Popular literature describing
formal proofs\index{formal proof} include \cite[p.~296]{Rucker}\index{Rucker,
Rudy} and \cite[pp.~204--230]{Hofstadter}\index{Hofstadter, Douglas R.}.

\subsection{Converting It to Metamath}\label{convert}

Formal proofs\index{formal proof} such as the one in our example break down
logical reasoning into small, precise steps that leave little doubt that the
results follow from the axioms\index{axiom}.  You might think that this would
be the finest breakdown we can achieve in mathematics.  However, there is more
to the proof than meets the eye. Although our axioms were rather simple, a lot
of verbiage was needed before we could even state them:  we needed to define
``term,'' ``wff,'' and so on.  In addition, there are a number of implied
rules that we haven't even mentioned. For example, how do we know that step 3
of our proof follows from axiom A1? There is some hidden reasoning involved in
determining this.  Axiom A1 has two occurrences of the letter $ t$.  One of
the implied rules states that whatever we substitute for $ t$ must be a legal
term\index{term}.\footnote{Some authors make this implied rule explicit by
stating, ``only expressions of the above form are terms,'' after defining
``term.''}  The expression $ t+0$ is pretty obviously a legal term whenever $
t$ is, but suppose we wanted to substitute a huge term with thousands of
symbols?  Certainly a lot of work would be involved in determining that it
really is a term, but in ordinary formal proofs all of this work would be
considered a single ``step.''

To express our axiom system in the Metamath\index{Metamath} language, we must
describe this auxiliary information in addition to the axioms themselves.
Metamath does not know what a ``term'' or a ``wff''\index{well-formed formula
(wff)} is.  In Metamath, the specification of the ways in which we can combine
symbols to obtain terms and wffs are like little axioms in themselves.  These
auxiliary axioms are expressed in the same notation as the ``real''
axioms\index{axiom}, and Metamath does not distinguish between the two.  The
distinction is made by you, i.e.\ by the way in which you interpret the
notation you have chosen to express these two kinds of axioms.

The Metamath language breaks down mathematical proofs into tiny pieces, much
more so than in ordinary formal proofs\index{formal proof}.  If a single
step\index{proof step} involves the
substitution\index{substitution!variable}\index{variable substitution} of a
complex term for one of its variables, Metamath must see this single step
broken down into many small steps.  This fine-grained breakdown is what gives
Metamath generality and flexibility as it lets it not be limited to any
particular mathematical notation.

Metamath's proof notation is not, in itself, intended to be read by humans but
rather is in a compact format intended for a machine.  The Metamath program
will convert this notation to a form you can understand, using the \texttt{show
proof}\index{\texttt{show proof} command} command.  You can tell the program what
level of detail of the proof you want to look at.  You may want to look at
just the logical inference steps that correspond
to ordinary formal proof steps,
or you may want to see the fine-grained steps that prove that an expression is
a term.

Here, without further ado, is our example converted to the
Metamath\index{Metamath} language:\index{metavariable}\label{demo0}

\begin{verbatim}
$( Declare the constant symbols we will use $)
    $c 0 + = -> ( ) term wff |- $.
$( Declare the metavariables we will use $)
    $v t r s P Q $.
$( Specify properties of the metavariables $)
    tt $f term t $.
    tr $f term r $.
    ts $f term s $.
    wp $f wff P $.
    wq $f wff Q $.
$( Define "term" and "wff" $)
    tze $a term 0 $.
    tpl $a term ( t + r ) $.
    weq $a wff t = r $.
    wim $a wff ( P -> Q ) $.
$( State the axioms $)
    a1 $a |- ( t = r -> ( t = s -> r = s ) ) $.
    a2 $a |- ( t + 0 ) = t $.
$( Define the modus ponens inference rule $)
    ${
       min $e |- P $.
       maj $e |- ( P -> Q ) $.
       mp  $a |- Q $.
    $}
$( Prove a theorem $)
    th1 $p |- t = t $=
  $( Here is its proof: $)
       tt tze tpl tt weq tt tt weq tt a2 tt tze tpl
       tt weq tt tze tpl tt weq tt tt weq wim tt a2
       tt tze tpl tt tt a1 mp mp
     $.
\end{verbatim}\index{metavariable}

A ``database''\index{database} is a set of one or more {\sc ascii} source
files.  Here's a brief description of this Metamath\index{Metamath} database
(which consists of this single source file), so that you can understand in
general terms what is going on.  To understand the source file in detail, you
should read Chapter~\ref{languagespec}.

The database is a sequence of ``tokens,''\index{token} which are normally
separated by spaces or line breaks.  The only tokens that are built into
the Metamath language are those beginning with \texttt{\$}.  These tokens
are called ``keywords.''\index{keyword}  All other tokens are
user-defined, and their names are arbitrary.

As you might have guessed, the Metamath token \texttt{\$(}\index{\texttt{\$(} and
\texttt{\$)} auxiliary keywords} starts a comment and \texttt{\$)} ends a comment.

The Metamath tokens \texttt{\$c}\index{\texttt{\$c} statement},
\texttt{\$v}\index{\texttt{\$v} statement},
\texttt{\$e}\index{\texttt{\$e} statement},
\texttt{\$f}\index{\texttt{\$f} statement},
\texttt{\$a}\index{\texttt{\$a} statement}, and
\texttt{\$p}\index{\texttt{\$p} statement} specify ``statements'' that
end with \texttt{\$.}\,.\index{\texttt{\$.}\ keyword}

The Metamath tokens \texttt{\$c} and \texttt{\$v} each declare\index{constant
declaration}\index{variable declaration} a list of user-defined tokens, called
``math symbols,''\index{math symbol} that the database will reference later
on.  All of the math symbols they define you have seen earlier except the
turnstile symbol \texttt{|-} ($\vdash$)\index{turnstile ({$\,\vdash$})}, which is
commonly used by logicians to mean ``a proof exists for.''  For us
the turnstile is just a
convenient symbol that distinguishes expressions that are axioms\index{axiom}
or theorems\index{theorem} from expressions that are terms or wffs.

The \texttt{\$c} statement declares ``constants''\index{constant} and
the \texttt{\$v} statement declares
``variables''\index{variable}\index{constant declaration}\index{variable
declaration} (or more precisely, metavariables\index{metavariable}).  A
variable may be substituted\index{substitution!variable}\index{variable
substitution} with sequences of math symbols whereas a constant may not
be substituted with anything.

It may seem redundant to require both \texttt{\$c}\index{\texttt{\$c} statement} and
\texttt{\$v}\index{\texttt{\$v} statement} statements (since any math
symbol\index{math symbol} not specified with a \texttt{\$c} statement could be
presumed to be a variable), but this provides for better error checking and
also allows math symbols to be redeclared\index{redeclaration of symbols}
(Section~\ref{scoping}).

The token \texttt{\$f}\index{\texttt{\$f} statement} specifies a
statement called a ``variable-type hypothesis'' (also called a
``floating hypothesis'') and \texttt{\$e}\index{\texttt{\$e} statement}
specifies a ``logical hypothesis'' (also called an ``essential
hypothesis'').\index{hypothesis}\index{variable-type
hypothesis}\index{logical hypothesis}\index{floating
hypothesis}\index{essential hypothesis} The token
\texttt{\$a}\index{\texttt{\$a} statement} specifies an ``axiomatic
assertion,''\index{axiomatic assertion} and
\texttt{\$p}\index{\texttt{\$p} statement} specifies a ``provable
assertion.''\index{provable assertion} To the left of each occurrence of
these four tokens is a ``label''\index{label} that identifies the
hypothesis or assertion for later reference.  For example, the label of
the first axiomatic assertion is \texttt{tze}.  A \texttt{\$f} statement
must contain exactly two math symbols, a constant followed by a
variable.  The \texttt{\$e}, \texttt{\$a}, and \texttt{\$p} statements
each start with a constant followed by, in general, an arbitrary
sequence of math symbols.

Associated with each assertion\index{assertion} is a set of hypotheses
that must be satisfied in order for the assertion to be used in a proof.
These are called the ``mandatory hypotheses''\index{mandatory
hypothesis} of the assertion.  Among those hypotheses whose ``scope''
(described below) includes the assertion, \texttt{\$e} hypotheses are
always mandatory and \texttt{\$f}\index{\texttt{\$f} statement}
hypotheses are mandatory when they share their variable with the
assertion or its \texttt{\$e} hypotheses.  The exact rules for
determining which hypotheses are mandatory are described in detail in
Sections~\ref{frames} and \ref{scoping}.  For example, the mandatory
hypotheses of assertion \texttt{tpl} are \texttt{tt} and \texttt{tr},
whereas assertion \texttt{tze} has no mandatory hypotheses because it
contains no variables and has no \texttt{\$e}\index{\texttt{\$e}
statement} hypothesis.  Metamath's \texttt{show statement}
command\index{\texttt{show statement} command}, described in the next
section, will show you a statement's mandatory hypotheses.

Sometimes we need to make a hypothesis relevant to only certain
assertions.  The set of statements to which a hypothesis is relevant is
called its ``scope.''  The Metamath brackets,
\texttt{\$\char`\{}\index{\texttt{\$\char`\{} and \texttt{\$\char`\}}
keywords} and \texttt{\$\char`\}}, define a ``block''\index{block} that
delimits the scope of any hypothesis contained between them.  The
assertion \texttt{mp} has mandatory hypotheses \texttt{wp}, \texttt{wq},
\texttt{min}, and \texttt{maj}.  The only mandatory hypothesis of
\texttt{th1}, on the other hand, is \texttt{tt}, since \texttt{th1}
occurs outside of the block containing \texttt{min} and \texttt{maj}.

Note that \texttt{\$\char`\{} and \texttt{\$\char`\}} do not affect the
scope of assertions (\texttt{\$a} and \texttt{\$p}).  Assertions are always
available to be referenced by any later proof in the source file.

Each provable assertion (\texttt{\$p}\index{\texttt{\$p} statement}
statement) has two parts.  The first part is the
assertion\index{assertion} itself, which is a sequence of math
symbol\index{math symbol} tokens placed between the \texttt{\$p} token
and a \texttt{\$=}\index{\texttt{\$=} keyword} token.  The second part
is a ``proof,'' which is a list of label tokens placed between the
\texttt{\$=} token and the \texttt{\$.}\index{\texttt{\$.}\ keyword}\
token that ends the statement.\footnote{If you've looked at the
\texttt{set.mm} database, you may have noticed another notation used for
proofs.  The other notation is called ``compressed.''\index{compressed
proof}\index{proof!compressed} It reduces the amount of space needed to
store a proof in the database and is described in
Appendix~\ref{compressed}.  In the example above, we use
``normal''\index{normal proof}\index{proof!normal} notation.} The proof
acts as a series of instructions to the Metamath program, telling it how
to build up the sequence of math symbols contained in the assertion part of
the \texttt{\$p} statement, making use of the hypotheses of the
\texttt{\$p} statement and previous assertions.  The construction takes
place according to precise rules.  If the list of labels in the proof
causes these rules to be violated, or if the final sequence that results
does not match the assertion, the Metamath program will notify you with
an error message.

If you are familiar with reverse Polish notation (RPN), which is sometimes used
on pocket calculators, here in a nutshell is how a proof works.  Each
hypothesis label\index{hypothesis label} in the proof is pushed\index{push}
onto the RPN stack\index{stack}\index{RPN stack} as it is encountered. Each
assertion label\index{assertion label} pops\index{pop} off the stack as many
entries as the referenced assertion has mandatory hypotheses.  Variable
substitutions\index{substitution!variable}\index{variable substitution} are
computed which, when made to the referenced assertion's mandatory hypotheses,
cause these hypotheses to match the stack entries. These same substitutions
are then made to the variables in the referenced assertion itself, which is
then pushed onto the stack.  At the end of the proof, there should be one
stack entry, namely the assertion being proved.  This process is explained in
detail in Section~\ref{proof}.

Metamath's proof notation is not very readable for humans, but it allows the
proof to be stored compactly in a file.  The Metamath\index{Metamath} program
has proof display features that let you see what's going on in a more
readable way, as you will see in the next section.

The rules used in verifying a proof are not based on any built-in syntax of the
symbol sequence in an assertion\index{assertion} nor on any built-in meanings
attached to specific symbol names.  They are based strictly on symbol
matching:  constants\index{constant} must match themselves, and
variables\index{variable} may be replaced with anything that allows a match to
occur.  For example, instead of \texttt{term}, \texttt{0}, and \verb$|-$ we could
have just as well used \texttt{yellow}, \texttt{zero}, and \texttt{provable}, as long
as we did so consistently throughout the database.  Also, we could have used
\texttt{is provable} (two tokens) instead of \verb$|-$ (one token) throughout the
database.  In each of these cases, the proof would be exactly the same.  The
independence of proofs and notation means that you have a lot of flexibility to
change the notation you use without having to change any proofs.

\section{A Trial Run}\label{trialrun}

Now you are ready to try out the Metamath\index{Metamath} program.

On all computer systems, Metamath has a standard ``command line
interface'' (CLI)\index{command line interface (CLI)} that allows you to
interact with it.  You supply commands to the CLI by typing them on the
keyboard and pressing your keyboard's {\em return} key after each line
you enter.  The CLI is designed to be easy to use and has built-in help
features.

The first thing you should do is to use a text editor to create a file
called \texttt{demo0.mm} and type into it the Metamath source shown on
p.~\pageref{demo0}.  Actually, this file is included with your Metamath
software package, so check that first.  If you type it in, make sure
that you save it in the form of ``plain {\sc ascii} text with line
breaks.''  Most word processors will have this feature.

Next you must run the Metamath program.  Depending on your computer
system and how Metamath is installed, this could range from clicking the
mouse on the Metamath icon to typing \texttt{run metamath} to typing
simply \texttt{metamath}.  (Metamath's {\tt help invoke} command describes
alternate ways of invoking the Metamath program.)

When you first enter Metamath\index{Metamath}, it will be at the CLI, waiting
for your input. You will see something like the following on your screen:
\begin{verbatim}
Metamath - Version 0.177 27-Apr-2019
Type HELP for help, EXIT to exit.
MM>
\end{verbatim}
The \texttt{MM>} prompt means that Metamath is waiting for a command.
Command keywords\index{command keyword} are not case sensitive;
we will use lower-case commands in our examples.
The version number and its release date will probably be different on your
system from the one we show above.

The first thing that you need to do is to read in your
database:\index{\texttt{read} command}\footnote{If a directory path is
needed on Unix,\index{Unix file names}\index{file names!Unix} you should
enclose the path/file name in quotes to prevent Metamath from thinking
that the \texttt{/} in the path name is a command qualifier, e.g.,
\texttt{read \char`\"db/set.mm\char`\"}.  Quotes are optional when there
is no ambiguity.}
\begin{verbatim}
MM> read demo0.mm
\end{verbatim}
Remember to press the {\em return} key after entering this command.  If
you omit the file name, Metamath will prompt you for one.   The syntax for
specifying a Macintosh file name path is given in a footnote on
p.~\pageref{includef}.\index{Macintosh file names}\index{file
names!Macintosh}

If there are any syntax errors in the database, Metamath will let you know
when it reads in the file.  The one thing that Metamath does not check when
reading in a database is that all proofs are correct, because this would
slow it down too much.  It is a good idea to periodically verify the proofs in
a database you are making changes to.  To do this, use the following command
(and do it for your \texttt{demo0.mm} file now).  Note that the \texttt{*} is a
``wild card'' meaning all proofs in the file.\index{\texttt{verify proof} command}
\begin{verbatim}
MM> verify proof *
\end{verbatim}
Metamath will report any proofs that are incorrect.

It is often useful to save the information that the Metamath program displays
on the screen. You can save everything that happens on the screen by opening a
log file. You may want to do this before you read in a database so that you
can examine any errors later on.  To open a log file, type
\begin{verbatim}
MM> open log abc.log
\end{verbatim}
This will open a file called \texttt{abc.log}, and everything that appears on the
screen from this point on will be stored in this file.  The name of the log file
is arbitrary. To close the log file, type
\begin{verbatim}
MM> close log
\end{verbatim}

Several commands let you examine what's inside your database.
Section~\ref{exploring} has an overview of some useful ones.  The
\texttt{show labels} command lets you see what statement
labels\index{label} exist.  A \texttt{*} matches any combination of
characters, and \texttt{t*} refers to all labels starting with the
letter \texttt{t}.\index{\texttt{show labels} command} The \texttt{/all}
is a ``command qualifier''\index{command qualifier} that tells Metamath
to include labels of hypotheses.  (To see the syntax explained, type
\texttt{help show labels}.)  Type
\begin{verbatim}
MM> show labels t* /all
\end{verbatim}
Metamath will respond with
\begin{verbatim}
The statement number, label, and type are shown.
3 tt $f       4 tr $f       5 ts $f       8 tze $a
9 tpl $a      19 th1 $p
\end{verbatim}

You can use the \texttt{show statement} command to get information about a
particular statement.\index{\texttt{show statement} command}
For example, you can get information about the statement with label \texttt{mp}
by typing
\begin{verbatim}
MM> show statement mp /full
\end{verbatim}
Metamath will respond with
\begin{verbatim}
Statement 17 is located on line 43 of the file
"demo0.mm".
"Define the modus ponens inference rule"
17 mp $a |- Q $.
Its mandatory hypotheses in RPN order are:
  wp $f wff P $.
  wq $f wff Q $.
  min $e |- P $.
  maj $e |- ( P -> Q ) $.
The statement and its hypotheses require the
      variables:  Q P
The variables it contains are:  Q P
\end{verbatim}
The mandatory hypotheses\index{mandatory hypothesis} and their
order\index{RPN order} are
useful to know when you are trying to understand or debug a proof.

Now you are ready to look at what's really inside our proof.  First, here is
how to look at every step in the proof---not just the ones corresponding to an
ordinary formal proof\index{formal proof}, but also the ones that build up the
formulas that appear in each ordinary formal proof step.\index{\texttt{show
proof} command}
\begin{verbatim}
MM> show proof th1 /lemmon /all
\end{verbatim}

This will display the proof on the screen in the following format:
\begin{verbatim}
 1 tt            $f term t
 2 tze           $a term 0
 3 1,2 tpl       $a term ( t + 0 )
 4 tt            $f term t
 5 3,4 weq       $a wff ( t + 0 ) = t
 6 tt            $f term t
 7 tt            $f term t
 8 6,7 weq       $a wff t = t
 9 tt            $f term t
10 9 a2          $a |- ( t + 0 ) = t
11 tt            $f term t
12 tze           $a term 0
13 11,12 tpl     $a term ( t + 0 )
14 tt            $f term t
15 13,14 weq     $a wff ( t + 0 ) = t
16 tt            $f term t
17 tze           $a term 0
18 16,17 tpl     $a term ( t + 0 )
19 tt            $f term t
20 18,19 weq     $a wff ( t + 0 ) = t
21 tt            $f term t
22 tt            $f term t
23 21,22 weq     $a wff t = t
24 20,23 wim     $a wff ( ( t + 0 ) = t -> t = t )
25 tt            $f term t
26 25 a2         $a |- ( t + 0 ) = t
27 tt            $f term t
28 tze           $a term 0
29 27,28 tpl     $a term ( t + 0 )
30 tt            $f term t
31 tt            $f term t
32 29,30,31 a1   $a |- ( ( t + 0 ) = t -> ( ( t + 0 )
                                     = t -> t = t ) )
33 15,24,26,32 mp  $a |- ( ( t + 0 ) = t -> t = t )
34 5,8,10,33 mp  $a |- t = t
\end{verbatim}

The \texttt{/lemmon} command qualifier specifies what is known as a Lemmon-style
display\index{Lemmon-style proof}\index{proof!Lemmon-style}.  Omitting the
\texttt{/lemmon} qualifier results in a tree-style proof (see
p.~\pageref{treeproof} for an example) that is somewhat less explicit but
easier to follow once you get used to it.\index{tree-style
proof}\index{proof!tree-style}

The first number on each line is the step
number of the proof.  Any numbers that follow are step numbers assigned to the
hypotheses of the statement referenced by that step.  Next is the label of
the statement referenced by the step.  The statement type of the statement
referenced comes next, followed by the math symbol\index{math symbol} string
constructed by the proof up to that step.

The last step, 34, contains the statement that is being proved.

Looking at a small piece of the proof, notice that steps 3 and 4 have
established that
\texttt{( t + 0 )} and \texttt{t} are \texttt{term}\,s, and step 5 makes use of steps 3 and
4 to establish that \texttt{( t + 0 ) = t} is a \texttt{wff}.  Let Metamath
itself tell us in detail what is happening in step 5.  Note that the
``target hypothesis'' refers to where step 5 is eventually used, i.e., in step
34.
\begin{verbatim}
MM> show proof th1 /detailed_step 5
Proof step 5:  wp=weq $a wff ( t + 0 ) = t
This step assigns source "weq" ($a) to target "wp"
($f).  The source assertion requires the hypotheses
"tt" ($f, step 3) and "tr" ($f, step 4).  The parent
assertion of the target hypothesis is "mp" ($a,
step 34).
The source assertion before substitution was:
    weq $a wff t = r
The following substitutions were made to the source
assertion:
    Variable  Substituted with
     t         ( t + 0 )
     r         t
The target hypothesis before substitution was:
    wp $f wff P
The following substitution was made to the target
hypothesis:
    Variable  Substituted with
     P         ( t + 0 ) = t
\end{verbatim}

The full proof just shown is useful to understand what is going on in detail.
However, most of the time you will just be interested in
the ``essential'' or logical steps of a proof, i.e.\ those steps
that correspond to an
ordinary formal proof\index{formal proof}.  If you type
\begin{verbatim}
MM> show proof th1 /lemmon /renumber
\end{verbatim}
you will see\label{demoproof}
\begin{verbatim}
1 a2             $a |- ( t + 0 ) = t
2 a2             $a |- ( t + 0 ) = t
3 a1             $a |- ( ( t + 0 ) = t -> ( ( t + 0 )
                                     = t -> t = t ) )
4 2,3 mp         $a |- ( ( t + 0 ) = t -> t = t )
5 1,4 mp         $a |- t = t
\end{verbatim}
Compare this to the formal proof on p.~\pageref{zeroproof} and
notice the resemblance.
By default Metamath
does not show \texttt{\$f}\index{\texttt{\$f}
statement} hypotheses and everything branching off of them in the proof tree
when the proof is displayed; this makes the proof look more like an ordinary
mathematical proof, which does not normally incorporate the explicit
construction of expressions.
This is called the ``essential'' view
(at one time you had to add the
\texttt{/essential} qualifier in the \texttt{show proof}
command to get this view, but this is now the default).
You can could use the \texttt{/all} qualifier in the \texttt{show
proof} command to also show the explicit construction of expressions.
The \texttt{/renumber} qualifier means to renumber
the steps to correspond only to what is displayed.\index{\texttt{show proof}
command}

To exit Metamath, type\index{\texttt{exit} command}
\begin{verbatim}
MM> exit
\end{verbatim}

\subsection{Some Hints for Using the Command Line Interface}

We will conclude this quick introduction to Metamath\index{Metamath} with some
helpful hints on how to navigate your way through the commands.
\index{command line interface (CLI)}

When you type commands into Metamath's CLI, you only have to type as many
characters of a command keyword\index{command keyword} as are needed to make
it unambiguous.  If you type too few characters, Metamath will tell you what
the choices are.  In the case of the \texttt{read} command, only the \texttt{r} is
needed to specify it unambiguously, so you could have typed\index{\texttt{read}
command}
\begin{verbatim}
MM> r demo0.mm
\end{verbatim}
instead of
\begin{verbatim}
MM> read demo0.mm
\end{verbatim}
In our description, we always show the full command words.  When using the
Metamath CLI commands in a command file (to be read with the \texttt{submit}
command)\index{\texttt{submit} command}, it is good practice to use
the unabbreviated command to ensure your instructions will not become ambiguous
if more commands are added to the Metamath program in the future.

The command keywords\index{command
keyword} are not case sensitive; you may type either \texttt{read} or
\texttt{ReAd}.  File names may or may not be case sensitive, depending on your
computer's operating system.  Metamath label\index{label} and math
symbol\index{math symbol} tokens\index{token} are case-sensitive.

The \texttt{help} command\index{\texttt{help} command} will provide you
with a list of topics you can get help on.  You can then type
\texttt{help} {\em topic} to get help on that topic.

If you are uncertain of a command's spelling, just type as many characters
as you remember of the command.  If you have not typed enough characters to
specify it unambiguously, Metamath will tell you what choices you have.

\begin{verbatim}
MM> show s
         ^
?Ambiguous keyword - please specify SETTINGS,
STATEMENT, or SOURCE.
\end{verbatim}

If you don't know what argument to use as part of a command, type a
\texttt{?}\index{\texttt{]}@\texttt{?}\ in command lines}\ at the
argument position.  Metamath will tell you what it expected there.

\begin{verbatim}
MM> show ?
         ^
?Expected SETTINGS, LABELS, STATEMENT, SOURCE, PROOF,
MEMORY, TRACE_BACK, or USAGE.
\end{verbatim}

Finally, you may type just the first word or words of a command followed
by {\em return}.  Metamath will prompt you for the remaining part of the
command, showing you the choices at each step.  For example, instead of
typing \texttt{show statement th1 /full} you could interact in the
following manner:
\begin{verbatim}
MM> show
SETTINGS, LABELS, STATEMENT, SOURCE, PROOF,
MEMORY, TRACE_BACK, or USAGE <SETTINGS>? st
What is the statement label <th1>?
/ or nothing <nothing>? /
TEX, COMMENT_ONLY, or FULL <TEX>? f
/ or nothing <nothing>?
19 th1 $p |- t = t $= ... $.
\end{verbatim}
After each \texttt{?}\ in this mode, you must give Metamath the
information it requests.  Sometimes Metamath gives you a list of choices
with the default choice indicated by brackets \texttt{< > }. Pressing
{\em return} after the \texttt{?}\ will select the default choice.
Answering anything else will override the default.  Note that the
\texttt{/} in command qualifiers is considered a separate
token\index{token} by the parser, and this is why it is asked for
separately.

\section{Your First Proof}\label{frstprf}

Proofs are developed with the aid of the Proof Assistant\index{Proof
Assistant}.  We will now show you how the proof of theorem \texttt{th1}
was built.  So that you can repeat these steps, we will first have the
Proof Assistant erase the proof in Metamath's source buffer\index{source
buffer}, then reconstruct it.  (The source buffer is the place in memory
where Metamath stores the information in the database when it is
\texttt{read}\index{\texttt{read} command} in.  New or modified proofs
are kept in the source buffer until a \texttt{write source}
command\index{\texttt{write source} command} is issued.)  In practice, you
would place a \texttt{?}\index{\texttt{]}@\texttt{?}\ inside proofs}\
between \texttt{\$=}\index{\texttt{\$=} keyword} and
\texttt{\$.}\index{\texttt{\$.}\ keyword}\ in the database to indicate
to Metamath\index{Metamath} that the proof is unknown, and that would be
your starting point.  Whenever the \texttt{verify proof} command encounters
a proof with a \texttt{?}\ in place of a proof step, the statement is
identified as not proved.

When I first started creating Metamath proofs, I would write down
on a piece of paper the complete
formal proof\index{formal proof} as it would appear
in a \texttt{show proof} command\index{\texttt{show proof} command}; see
the display of \texttt{show proof th1 /lemmon /re\-num\-ber} above as an
example.  After you get used to using the Proof Assistant\index{Proof
Assistant} you may get to a point where you can ``see'' the proof in your mind
and let the Proof Assistant guide you in filling in the details, at least for
simpler proofs, but until you gain that experience you may find it very useful
to write down all the details in advance.
Otherwise you may waste a lot of time as you let it take you down a wrong path.
However, others do not find this approach as helpful.
For example, Thomas Brendan Leahy\index{Leahy, Thomas Brendan}
finds that it is more helpful to him to interactively
work backward from a machine-readable statement.
David A. Wheeler\index{Wheeler, David A.}
writes down a general approach, but develops the proof
interactively by switching between
working forwards (from hypotheses and facts likely to be useful) and
backwards (from the goal) until the forwards and backwards approaches meet.
In the end, use whatever approach works for you.

A proof is developed with the Proof Assistant by working backwards, starting
with the theorem\index{theorem} to be proved, and assigning each unknown step
with a theorem or hypothesis until no more unknown steps remain.  The Proof
Assistant will not let you make an assignment unless it can be ``unified''
with the unknown step.  This means that a
substitution\index{substitution!variable}\index{variable substitution} of
variables exists that will make the assignment match the unknown step.  On the
other hand, in the middle of a proof, when working backwards, often more than
one unification\index{unification} (set of substitutions) is possible, since
there is not enough information available at that point to uniquely establish
it.  In this case you can tell Metamath which unification to choose, or you
can continue to assign unknown steps until enough information is available to
make the unification unique.

We will assume you have entered Metamath and read in the database as described
above.  The following dialog shows how the proof was developed.  For more
details on what some of the commands do, refer to Section~\ref{pfcommands}.
\index{\texttt{prove} command}

\begin{verbatim}
MM> prove th1
Entering the Proof Assistant.  Type HELP for help, EXIT
to exit.  You will be working on the proof of statement th1:
  $p |- t = t
Note:  The proof you are starting with is already complete.
MM-PA>
\end{verbatim}

The \verb/MM-PA>/ prompt means we are inside the Proof
Assistant.\index{Proof Assistant} Most of the regular Metamath commands
(\texttt{show statement}, etc.) are still available if you need them.

\begin{verbatim}
MM-PA> delete all
The entire proof was deleted.
\end{verbatim}

We have deleted the whole proof so we can start from scratch.

\begin{verbatim}
MM-PA> show new_proof/lemmon/all
1 ?              $? |- t = t
\end{verbatim}

The \texttt{show new{\char`\_}proof} command\index{\texttt{show
new{\char`\_}proof} command} is like \texttt{show proof} except that we
don't specify a statement; instead, the proof we're working on is
displayed.

\begin{verbatim}
MM-PA> assign 1 mp
To undo the assignment, DELETE STEP 5 and INITIALIZE, UNIFY
if needed.
3   min=?  $? |- $2
4   maj=?  $? |- ( $2 -> t = t )
\end{verbatim}

The \texttt{assign} command\index{\texttt{assign} command} above means
``assign step 1 with the statement whose label is \texttt{mp}.''  Note
that step renumbering will constantly occur as you assign steps in the
middle of a proof; in general all steps from the step you assign until
the end of the proof will get moved up.  In this case, what used to be
step 1 is now step 5, because the (partial) proof now has five steps:
the four hypotheses of the \texttt{mp} statement and the \texttt{mp}
statement itself.  Let's look at all the steps in our partial proof:

\begin{verbatim}
MM-PA> show new_proof/lemmon/all
1 ?              $? wff $2
2 ?              $? wff t = t
3 ?              $? |- $2
4 ?              $? |- ( $2 -> t = t )
5 1,2,3,4 mp     $a |- t = t
\end{verbatim}

The symbol \texttt{\$2} is a temporary variable\index{temporary
variable} that represents a symbol sequence not yet known.  In the final
proof, all temporary variables will be eliminated.  The general format
for a temporary variable is \texttt{\$} followed by an integer.  Note
that \texttt{\$} is not a legal character in a math symbol (see
Section~\ref{dollardollar}, p.~\pageref{dollardollar}), so there will
never be a naming conflict between real symbols and temporary variables.

Unknown steps 1 and 2 are constructions of the two wffs used by the
modus ponens rule.  As you will see at the end of this section, the
Proof Assistant\index{Proof Assistant} can usually figure these steps
out by itself, and we will not have to worry about them.  Therefore from
here on we will display only the ``essential'' hypotheses, i.e.\ those
steps that correspond to traditional formal proofs\index{formal proof}.

\begin{verbatim}
MM-PA> show new_proof/lemmon
3 ?              $? |- $2
4 ?              $? |- ( $2 -> t = t )
5 3,4 mp         $a |- t = t
\end{verbatim}

Unknown steps 3 and 4 are the ones we must focus on.  They correspond to the
minor and major premises of the modus ponens rule.  We will assign them as
follows.  Notice that because of the step renumbering that takes place
after an assignment, it is advantageous to assign unknown steps in reverse
order, because earlier steps will not get renumbered.

\begin{verbatim}
MM-PA> assign 4 mp
To undo the assignment, DELETE STEP 8 and INITIALIZE, UNIFY
if needed.
3   min=?  $? |- $2
6     min=?  $? |- $4
7     maj=?  $? |- ( $4 -> ( $2 -> t = t ) )
\end{verbatim}

We are now going to describe an obscure feature that you will probably
never use but should be aware of.  The Metamath language allows empty
symbol sequences to be substituted for variables, but in most formal
systems this feature is never used.  One of the few examples where is it
used is the MIU-system\index{MIU-system} described in
Appendix~\ref{MIU}.  But such systems are rare, and by default this
feature is turned off in the Proof Assistant.  (It is always allowed for
{\tt verify proof}.)  Let us turn it on and see what
happens.\index{\texttt{set empty{\char`\_}substitution} command}

\begin{verbatim}
MM-PA> set empty_substitution on
Substitutions with empty symbol sequences is now allowed.
\end{verbatim}

With this feature enabled, more unifications will be
ambiguous\index{ambiguous unification}\index{unification!ambiguous} in
the middle of a proof, because
substitution\index{substitution!variable}\index{variable substitution}
of variables with empty symbol sequences will become an additional
possibility.  Let's see what happens when we make our next assignment.

\begin{verbatim}
MM-PA> assign 3 a2
There are 2 possible unifications.  Please select the correct
    one or Q if you want to UNIFY later.
Unify:  |- $6
 with:  |- ( $9 + 0 ) = $9
Unification #1 of 2 (weight = 7):
  Replace "$6" with "( + 0 ) ="
  Replace "$9" with ""
  Accept (A), reject (R), or quit (Q) <A>? r
\end{verbatim}

The first choice presented is the wrong one.  If we had selected it,
temporary variable \texttt{\$6} would have been assigned a truncated
wff, and temporary variable \texttt{\$9} would have been assigned an
empty sequence (which is not allowed in our system).  With this choice,
eventually we would reach a point where we would get stuck because
we would end up with steps impossible to prove.  (You may want to
try it.)  We typed \texttt{r} to reject the choice.

\begin{verbatim}
Unification #2 of 2 (weight = 21):
  Replace "$6" with "( $9 + 0 ) = $9"
  Accept (A), reject (R), or quit (Q) <A>? q
To undo the assignment, DELETE STEP 4 and INITIALIZE, UNIFY
if needed.
 7     min=?  $? |- $8
 8     maj=?  $? |- ( $8 -> ( $6 -> t = t ) )
\end{verbatim}

The second choice is correct, and normally we would type \texttt{a}
to accept it.  But instead we typed \texttt{q} to show what will happen:
it will leave the step with an unknown unification, which can be
seen as follows:

\begin{verbatim}
MM-PA> show new_proof/not_unified
 4   min    $a |- $6
        =a2  = |- ( $9 + 0 ) = $9
\end{verbatim}

Later we can unify this with the \texttt{unify}
\texttt{all/interactive} command.

The important point to remember is that occasionally you will be
presented with several unification choices while entering a proof, when
the program determines that there is not enough information yet to make
an unambiguous choice automatically (and this can happen even with
\texttt{set empty{\char`\_}substitution} turned off).  Usually it is
obvious by inspection which choice is correct, since incorrect ones will
tend to be meaningless fragments of wffs.  In addition, the correct
choice will usually be the first one presented, unlike our example
above.

Enough of this digression.  Let us go back to the default setting.

\begin{verbatim}
MM-PA> set empty_substitution off
The ability to substitute empty expressions for variables
has been turned off.  Note that this may make the Proof
Assistant too restrictive in some cases.
\end{verbatim}

If we delete the proof, start over, and get to the point where
we digressed above, there will no longer be an ambiguous unification.

\begin{verbatim}
MM-PA> assign 3 a2
To undo the assignment, DELETE STEP 4 and INITIALIZE, UNIFY
if needed.
 7     min=?  $? |- $4
 8     maj=?  $? |- ( $4 -> ( ( $5 + 0 ) = $5 -> t = t ) )
\end{verbatim}

Let us look at our proof so far, and continue.

\begin{verbatim}
MM-PA> show new_proof/lemmon
 4 a2            $a |- ( $5 + 0 ) = $5
 7 ?             $? |- $4
 8 ?             $? |- ( $4 -> ( ( $5 + 0 ) = $5 -> t = t ) )
 9 7,8 mp        $a |- ( ( $5 + 0 ) = $5 -> t = t )
10 4,9 mp        $a |- t = t
MM-PA> assign 8 a1
To undo the assignment, DELETE STEP 11 and INITIALIZE, UNIFY
if needed.
 7     min=?  $? |- ( t + 0 ) = t
MM-PA> assign 7 a2
To undo the assignment, DELETE STEP 8 and INITIALIZE, UNIFY
if needed.
MM-PA> show new_proof/lemmon
 4 a2            $a |- ( t + 0 ) = t
 8 a2            $a |- ( t + 0 ) = t
12 a1            $a |- ( ( t + 0 ) = t -> ( ( t + 0 ) = t ->
                                                    t = t ) )
13 8,12 mp       $a |- ( ( t + 0 ) = t -> t = t )
14 4,13 mp       $a |- t = t
\end{verbatim}

Now all temporary variables and unknown steps have been eliminated from the
``essential'' part of the proof.  When this is achieved, the Proof
Assistant\index{Proof Assistant} can usually figure out the rest of the proof
automatically.  (Note that the \texttt{improve} command can occasionally be
useful for filling in essential steps as well, but it only tries to make use
of statements that introduce no new variables in their hypotheses, which is
not the case for \texttt{mp}. Also it will not try to improve steps containing
temporary variables.)  Let's look at the complete proof, then run
the \texttt{improve} command, then look at it again.

\begin{verbatim}
MM-PA> show new_proof/lemmon/all
 1 ?             $? wff ( t + 0 ) = t
 2 ?             $? wff t = t
 3 ?             $? term t
 4 3 a2          $a |- ( t + 0 ) = t
 5 ?             $? wff ( t + 0 ) = t
 6 ?             $? wff ( ( t + 0 ) = t -> t = t )
 7 ?             $? term t
 8 7 a2          $a |- ( t + 0 ) = t
 9 ?             $? term ( t + 0 )
10 ?             $? term t
11 ?             $? term t
12 9,10,11 a1    $a |- ( ( t + 0 ) = t -> ( ( t + 0 ) = t ->
                                                    t = t ) )
13 5,6,8,12 mp   $a |- ( ( t + 0 ) = t -> t = t )
14 1,2,4,13 mp   $a |- t = t
\end{verbatim}

\begin{verbatim}
MM-PA> improve all
A proof of length 1 was found for step 11.
A proof of length 1 was found for step 10.
A proof of length 3 was found for step 9.
A proof of length 1 was found for step 7.
A proof of length 9 was found for step 6.
A proof of length 5 was found for step 5.
A proof of length 1 was found for step 3.
A proof of length 3 was found for step 2.
A proof of length 5 was found for step 1.
Steps 1 and above have been renumbered.
CONGRATULATIONS!  The proof is complete.  Use SAVE
NEW_PROOF to save it.  Note:  The Proof Assistant does
not detect $d violations.  After saving the proof, you
should verify it with VERIFY PROOF.
\end{verbatim}

The \texttt{save new{\char`\_}proof} command\index{\texttt{save
new{\char`\_}proof} command} will save the proof in the database.  Here
we will just display it in a form that can be clipped out of a log file
and inserted manually into the database source file with a text
editor.\index{normal proof}\index{proof!normal}

\begin{verbatim}
MM-PA> show new_proof/normal
---------Clip out the proof below this line:
      tt tze tpl tt weq tt tt weq tt a2 tt tze tpl tt weq
      tt tze tpl tt weq tt tt weq wim tt a2 tt tze tpl tt
      tt a1 mp mp $.
---------The proof of 'th1' to clip out ends above this line.
\end{verbatim}

There is another proof format called ``compressed''\index{compressed
proof}\index{proof!compressed} that you will see in databases.  It is
not important to understand how it is encoded but only to recognize it
when you see it.  Its only purpose is to reduce storage requirements for
large proofs.  A compressed proof can always be converted to a normal
one and vice-versa, and the Metamath \texttt{show proof}
commands\index{\texttt{show proof} command} work equally well with
compressed proofs.  The compressed proof format is described in
Appendix~\ref{compressed}.

\begin{verbatim}
MM-PA> show new_proof/compressed
---------Clip out the proof below this line:
      ( tze tpl weq a2 wim a1 mp ) ABCZADZAADZAEZJJKFLIA
      AGHH $.
---------The proof of 'th1' to clip out ends above this line.
\end{verbatim}

Now we will exit the Proof Assistant.  Since we made changes to the proof,
it will warn us that we have not saved it.  In this case, we don't care.

\begin{verbatim}
MM-PA> exit
Warning:  You have not saved changes to the proof.
Do you want to EXIT anyway (Y, N) <N>? y
Exiting the Proof Assistant.
Type EXIT again to exit Metamath.
\end{verbatim}

The Proof Assistant\index{Proof Assistant} has several other commands
that can help you while creating proofs.  See Section~\ref{pfcommands}
for a list of them.

A command that is often useful is \texttt{minimize{\char`\_}with
*/brief}, which tries to shorten the proof.  It can make the process
more efficient by letting you write a somewhat ``sloppy'' proof then
clean up some of the fine details of optimization for you (although it
can't perform miracles such as restructuring the overall proof).

\section{A Note About Editing a Data\-base File}

Once your source file contains proofs, there are some restrictions on
how you can edit it so that the proofs remain valid.  Pay particular
attention to these rules, since otherwise you can lose a lot of work.
It is a good idea to periodically verify all proofs with \texttt{verify
proof *} to ensure their integrity.

If your file contains only normal (as opposed to compressed) proofs, the
main rule is that you may not change the order of the mandatory
hypotheses\index{mandatory hypothesis} of any statement referenced in a
later proof.  For example, if you swap the order of the major and minor
premise in the modus ponens rule, all proofs making use of that rule
will become incorrect.  The \texttt{show statement}
command\index{\texttt{show statement} command} will show you the
mandatory hypotheses of a statement and their order.

If a statement has a compressed proof, you also must not change the
order of {\em its} mandatory hypotheses.  The compressed proof format
makes use of this information as part of the compression technique.
Note that swapping the names of two variables in a theorem will change
the order of its mandatory hypotheses.

The safest way to edit a statement, say \texttt{mytheorem}, is to
duplicate it then rename the original to \texttt{mytheoremOLD}
throughout the database.  Once the edited version is re-proved, all
statements referencing \texttt{mytheoremOLD} can be updated in the Proof
Assistant using \texttt{minimize{\char`\_}with
mytheorem
/allow{\char`\_}growth}.\index{\texttt{minimize{\char`\_}with} command}
% 3/10/07 Note: line-breaking the above results in duplicate index entries

\chapter{Abstract Mathematics Revealed}\label{fol}

\section{Logic and Set Theory}\label{logicandsettheory}

\begin{quote}
  {\em Set theory can be viewed as a form of exact theology.}
  \flushright\sc  Rudy Rucker\footnote{\cite{Barrow}, p.~31.}\\
\end{quote}\index{Rucker, Rudy}

Despite its seeming complexity, all of standard mathematics, no matter how
deep or abstract, can amazingly enough be derived from a relatively small set
of axioms\index{axiom} or first principles. The development of these axioms is
among the most impressive and important accomplishments of mathematics in the
20th century. Ultimately, these axioms can be broken down into a set of rules
for manipulating symbols that any technically oriented person can follow.

We will not spend much time trying to convey a deep, higher-level
understanding of the meaning of the axioms. This kind of understanding
requires some mathematical sophistication as well as an understanding of the
philosophy underlying the foundations of mathematics and typically develops
over time as you work with mathematics.  Our goal, instead, is to give you the
immediate ability to follow how theorems\index{theorem} are derived from the
axioms and from other theorems.  This will be similar to learning the syntax
of a computer language, which lets you follow the details in a program but
does not necessarily give you the ability to write non-trivial programs on
your own, an ability that comes with practice. For now don't be alarmed by
abstract-sounding names of the axioms; just focus on the rules for
manipulating the symbols, which follow the simple conventions of the
Metamath\index{Metamath} language.

The axioms that underlie all of standard mathematics consist of axioms of logic
and axioms of set theory. The axioms of logic are divided into two
subcategories, propositional calculus\index{propositional calculus} (sometimes
called sentential logic\index{sentential logic}) and predicate calculus
(sometimes called first-order logic\index{first-order logic}\index{quantifier
theory}\index{predicate calculus} or quantifier theory).  Propositional
calculus is a prerequisite for predicate calculus, and predicate calculus is a
prerequisite for set theory.  The version of set theory most commonly used is
Zermelo--Fraenkel set theory\index{Zermelo--Fraenkel set theory}\index{set theory}
with the axiom of choice,
often abbreviated as ZFC\index{ZFC}.

Here in a nutshell is what the axioms are all about in an informal way. The
connection between this description and symbols we will show you won't be
immediately apparent and in principle needn't ever be.  Our description just
tries to summarize what mathematicians think about when they work with the
axioms.

Logic is a set of rules that allow us determine truths given other truths.
Put another way,
logic is more or less the translation of what we would consider common sense
into a rigorous set of axioms.\index{axioms of logic}  Suppose $\varphi$,
$\psi$, and $\chi$ (the Greek letters phi, psi, and chi) represent statements
that are either true or false, and $x$ is a variable\index{variable!in predicate
calculus} ranging over some group of mathematical objects (sets, integers,
real numbers, etc.). In mathematics, a ``statement'' really means a formula,
and $\psi$ could be for example ``$x = 2$.''
Propositional calculus\index{propositional calculus}
allows us to use variables that are either true or false
and make deductions such as
``if $\varphi$ implies $\psi$ and $\psi$ implies $\chi$, then $\varphi$
implies $\chi$.''
Predicate calculus\index{predicate calculus}
extends propositional calculus by also allowing us
to discuss statements about objects (not just true and false values), including
statements about ``all'' or ``at least one'' object.
For example, predicate calculus allows to say,
``if $\varphi$ is true for all $x$, then $\varphi$ is true for some $x$.''
The logic used in \texttt{set.mm} is standard classical logic
(as opposed to other logic systems like intuitionistic logic).

Set theory\index{set theory} has to do with the manipulation of objects and
collections of objects, specifically the abstract, imaginary objects that
mathematics deals with, such as numbers. Everything that is claimed to exist
in mathematics is considered to be a set.  A set called the empty
set\index{empty set} contains nothing.  We represent the empty set by
$\varnothing$.  Many sets can be built up from the empty set.  There is a set
represented by $\{\varnothing\}$ that contains the empty set, another set
represented by $\{\varnothing,\{\varnothing\}\}$ that contains this set as
well as the empty set, another set represented by $\{\{\varnothing\}\}$ that
contains just the set that contains the empty set, and so on ad infinitum. All
mathematical objects, no matter how complex, are defined as being identical to
certain sets: the integer\index{integer} 0 is defined as the empty set, the
integer 1 is defined as $\{\varnothing\}$, the integer 2 is defined as
$\{\varnothing,\{\varnothing\}\}$.  (How these definitions were chosen doesn't
matter now, but the idea behind it is that these sets have the properties we
expect of integers once suitable operations are defined.)  Mathematical
operations, such as addition, are defined in terms of operations on
sets---their union\index{set union}, intersection\index{set intersection}, and
so on---operations you may have used in elementary school when you worked
with groups of apples and oranges.

With a leap of faith, the axioms also postulate the existence of infinite
sets\index{infinite set}, such as the set of all non-negative integers ($0, 1,
2,\ldots$, also called ``natural numbers''\index{natural number}).  This set
can't be represented with the brace notation\index{brace notation} we just
showed you, but requires a more complicated notation called ``class
abstraction.''\index{class abstraction}\index{abstraction class}  For
example, the infinite set $\{ x |
\mbox{``$x$ is a natural number''} \} $ means the ``set of all objects $x$
such that $x$ is a natural number'' i.e.\ the set of natural numbers; here,
``$x$ is a natural number'' is a rather complicated formula when broken down
into the primitive symbols.\label{expandom}\footnote{The statement ``$x$ is a
natural number'' is formally expressed as ``$x \in \omega$,'' where $\in$
(stylized epsilon) means ``is in'' or ``is an element of'' and $\omega$
(omega) means ``the set of natural numbers.''  When ``$x\in\omega$'' is
completely expanded in terms of the primitive symbols of set theory, the
result is  $\lnot$ $($ $\lnot$ $($ $\forall$ $z$ $($ $\lnot$ $\forall$ $w$ $($
$z$ $\in$ $w$ $\rightarrow$ $\lnot$ $w$ $\in$ $x$ $)$ $\rightarrow$ $z$ $\in$
$x$ $)$ $\rightarrow$ $($ $\forall$ $z$ $($ $\lnot$ $($ $\forall$ $w$ $($ $w$
$\in$ $z$ $\rightarrow$ $w$ $\in$ $x$ $)$ $\rightarrow$ $\forall$ $w$ $\lnot$
$w$ $\in$ $z$ $)$ $\rightarrow$ $\lnot$ $\forall$ $w$ $($ $w$ $\in$ $z$
$\rightarrow$ $\lnot$ $\forall$ $v$ $($ $v$ $\in$ $z$ $\rightarrow$ $\lnot$
$v$ $\in$ $w$ $)$ $)$ $)$ $\rightarrow$ $\lnot$ $\forall$ $z$ $\forall$ $w$
$($ $\lnot$ $($ $z$ $\in$ $x$ $\rightarrow$ $\lnot$ $w$ $\in$ $x$ $)$
$\rightarrow$ $($ $\lnot$ $z$ $\in$ $w$ $\rightarrow$ $($ $\lnot$ $z$ $=$ $w$
$\rightarrow$ $w$ $\in$ $z$ $)$ $)$ $)$ $)$ $)$ $\rightarrow$ $\lnot$
$\forall$ $y$ $($ $\lnot$ $($ $\lnot$ $($ $\forall$ $z$ $($ $\lnot$ $\forall$
$w$ $($ $z$ $\in$ $w$ $\rightarrow$ $\lnot$ $w$ $\in$ $y$ $)$ $\rightarrow$
$z$ $\in$ $y$ $)$ $\rightarrow$ $($ $\forall$ $z$ $($ $\lnot$ $($ $\forall$
$w$ $($ $w$ $\in$ $z$ $\rightarrow$ $w$ $\in$ $y$ $)$ $\rightarrow$ $\forall$
$w$ $\lnot$ $w$ $\in$ $z$ $)$ $\rightarrow$ $\lnot$ $\forall$ $w$ $($ $w$
$\in$ $z$ $\rightarrow$ $\lnot$ $\forall$ $v$ $($ $v$ $\in$ $z$ $\rightarrow$
$\lnot$ $v$ $\in$ $w$ $)$ $)$ $)$ $\rightarrow$ $\lnot$ $\forall$ $z$
$\forall$ $w$ $($ $\lnot$ $($ $z$ $\in$ $y$ $\rightarrow$ $\lnot$ $w$ $\in$
$y$ $)$ $\rightarrow$ $($ $\lnot$ $z$ $\in$ $w$ $\rightarrow$ $($ $\lnot$ $z$
$=$ $w$ $\rightarrow$ $w$ $\in$ $z$ $)$ $)$ $)$ $)$ $\rightarrow$ $($
$\forall$ $z$ $\lnot$ $z$ $\in$ $y$ $\rightarrow$ $\lnot$ $\forall$ $w$ $($
$\lnot$ $($ $w$ $\in$ $y$ $\rightarrow$ $\lnot$ $\forall$ $z$ $($ $w$ $\in$
$z$ $\rightarrow$ $\lnot$ $z$ $\in$ $y$ $)$ $)$ $\rightarrow$ $\lnot$ $($
$\lnot$ $\forall$ $z$ $($ $w$ $\in$ $z$ $\rightarrow$ $\lnot$ $z$ $\in$ $y$
$)$ $\rightarrow$ $w$ $\in$ $y$ $)$ $)$ $)$ $)$ $\rightarrow$ $x$ $\in$ $y$
$)$ $)$ $)$. Section~\ref{hierarchy} shows the hierarchy of definitions that
leads up to this expression.}\index{stylized epsilon ($\in$)}\index{omega
($\omega$)}  Actually, the primitive symbols don't even include the brace
notation.  The brace notation is a high-level definition, which you can find in
Section~\ref{hierarchy}.

Interestingly, the arithmetic of integers\index{integer} and
rationals\index{rational number} can be developed without appealing to the
existence of an infinite set, whereas the arithmetic of real
numbers\index{real number} requires it.

Each variable\index{variable!in set theory} in the axioms of set theory
represents an arbitrary set, and the axioms specify the legal kinds of things
you can do with these variables at a very primitive level.

Now, you may think that numbers and arithmetic are a lot more intuitive and
fundamental than sets and therefore should be the foundation of mathematics.
What is really the case is that you've dealt with numbers all your life and
are comfortable with a few rules for manipulating them such as addition and
multiplication.  Those rules only cover a small portion of what can be done
with numbers and only a very tiny fraction of the rest of mathematics.  If you
look at any elementary book on number theory, you will quickly become lost if
these are the only rules that you know.  Even though such books may present a
list of ``axioms''\index{axiom} for arithmetic, the ability to use the axioms
and to understand proofs of theorems\index{theorem} (facts) about numbers
requires an implicit mathematical talent that frustrates many people
from studying abstract mathematics.  The kind of mathematics that most people
know limits them to the practical, everyday usage of blindly manipulating
numbers and formulas, without any understanding of why those rules are correct
nor any ability to go any further.  For example, do you know why multiplying
two negative numbers yields a positive number?  Starting with set theory, you
will also start off blindly manipulating symbols according to the rules we give
you, but with the advantage that these rules will allow you, in principle, to
access {\em all} of mathematics, not just a tiny part of it.

Of course, concrete examples are often helpful in the learning process. For
example, you can verify that $2\cdot 3=3 \cdot 2$ by actually grouping
objects and can easily ``see'' how it generalizes to $x\cdot y = y\cdot x$,
even though you might not be able to rigorously prove it.  Similarly, in set
theory it can be helpful to understand how the axioms of set theory apply to
(and are correct for) small finite collections of objects.  You should be aware
that in set theory intuition can be misleading for infinite collections, and
rigorous proofs become more important.  For example, while $x\cdot y = y\cdot
x$ is correct for finite ordinals (which are the natural numbers), it is not
usually true for infinite ordinals.

\section{The Axioms for All of Mathematics}

In this section\index{axioms for mathematics}, we will show you the axioms
for all of standard mathematics (i.e.\ logic and set theory) as they are
traditionally presented.  The traditional presentation is useful for someone
with the mathematical experience needed to correctly manipulate high-level
abstract concepts.  For someone without this talent, knowing how to actually
make use of these axioms can be difficult.  The purpose of this section is to
allow you to see how the version of the axioms used in the standard
Metamath\index{Metamath} database \texttt{set.mm}\index{set
theory database (\texttt{set.mm})} relates to  the typical version
in textbooks, and also to give you an informal feel for them.

\subsection{Propositional Calculus}

Propositional calculus\index{propositional calculus} concerns itself with
statements that can be interpreted as either true or false.  Some examples of
statements (outside of mathematics) that are either true or false are ``It is
raining today'' and ``The United States has a female president.'' In
mathematics, as we mentioned, statements are really formulas.

In propositional calculus, we don't care what the statements are.  We also
treat a logical combination of statements, such as ``It is raining today and
the United States has a female president,'' no differently from a single
statement.  Statements and their combinations are called well-formed formulas
(wffs)\index{well-formed formula (wff)}.  We define wffs only in terms of
other wffs and don't define what a ``starting'' wff is.  As is common practice
in the literature, we use Greek letters to represent wffs.

Specifically, suppose $\varphi$ and $\psi$ are wffs.  Then the combinations
$\varphi\rightarrow\psi$ (``$\varphi$ implies $\psi$,'' also read ``if
$\varphi$ then $\psi$'')\index{implication ($\rightarrow$)} and $\lnot\varphi$
(``not $\varphi$'')\index{negation ($\lnot$)} are also wffs.

The three axioms of propositional calculus\index{axioms of propositional
calculus} are all wffs of the following form:\footnote{A remarkable result of
C.~A.~Meredith\index{Meredith, C. A.} squeezes these three axioms into the
single axiom $((((\varphi\rightarrow \psi)\rightarrow(\neg \chi\rightarrow\neg
\theta))\rightarrow \chi )\rightarrow \tau)\rightarrow((\tau\rightarrow
\varphi)\rightarrow(\theta\rightarrow \varphi))$ \cite{CAMeredith},
which is believed to be the shortest possible.}
\begin{center}
     $\varphi\rightarrow(\psi\rightarrow \varphi)$\\

     $(\varphi\rightarrow (\psi\rightarrow \chi))\rightarrow
((\varphi\rightarrow  \psi)\rightarrow (\varphi\rightarrow \chi))$\\

     $(\neg \varphi\rightarrow \neg\psi)\rightarrow (\psi\rightarrow
\varphi)$
\end{center}

These three axioms are widely used.
They are attributed to Jan {\L}ukasiewicz
(pronounced woo-kah-SHAY-vitch) and was popularized by Alonzo Church,
who called it system P2. (Thanks to Ted Ulrich for this information.)

There are an infinite number of axioms, one for each possible
wff\index{well-formed formula (wff)} of the above form.  (For this reason,
axioms such as the above are often called ``axiom schemes.''\index{axiom
scheme})  Each Greek letter in the axioms may be substituted with a more
complex wff to result in another axiom.  For example, substituting
$\neg(\varphi\rightarrow\chi)$ for $\varphi$ in the first axiom yields
$\neg(\varphi\rightarrow\chi)\rightarrow(\psi\rightarrow
\neg(\varphi\rightarrow\chi))$, which is still an axiom.

To deduce new true statements (theorems\index{theorem}) from the axioms, a
rule\index{rule} called ``modus ponens''\index{modus ponens} is used.  This
rule states that if the wff $\varphi$ is an axiom or a theorem, and the wff
$\varphi\rightarrow\psi$ is an axiom or a theorem, then the wff $\psi$ is also
a theorem\index{theorem}.

As a non-mathematical example of modus ponens, suppose we have proved (or
taken as an axiom) ``Bob is a man'' and separately have proved (or taken as
an axiom) ``If Bob is a man, then Bob is a human.''  Using the rule of modus
ponens, we can logically deduce, ``Bob is a human.''

From Metamath's\index{Metamath} point of view, the axioms and the rule of
modus ponens just define a mechanical means for deducing new true statements
from existing true statements, and that is the complete content of
propositional calculus as far as Metamath is concerned.  You can read a logic
textbook to gain a better understanding of their meaning, or you can just let
their meaning slowly become apparent to you after you use them for a while.

It is actually rather easy to check to see if a formula is a theorem of
propositional calculus.  Theorems of propositional calculus are also called
``tautologies.''\index{tautology}  The technique to check whether a formula is
a tautology is called the ``truth table method,''\index{truth table} and it
works like this.  A wff $\varphi\rightarrow\psi$ is false whenever $\varphi$ is true
and $\psi$ is false.  Otherwise it is true.  A wff $\lnot\varphi$ is false
whenever $\varphi$ is true and false otherwise. To verify a tautology such as
$\varphi\rightarrow(\psi\rightarrow \varphi)$, you break it down into sub-wffs and
construct a truth table that accounts for all possible combinations of true
and false assigned to the wff metavariables:
\begin{center}\begin{tabular}{|c|c|c|c|}\hline
\mbox{$\varphi$} & \mbox{$\psi$} & \mbox{$\psi\rightarrow\varphi$}
    & \mbox{$\varphi\rightarrow(\psi\rightarrow \varphi)$} \\ \hline \hline
              T   &  T    &      T       &        T    \\ \hline
              T   &  F    &      T       &        T    \\ \hline
              F   &  T    &      F       &        T    \\ \hline
              F   &  F    &      T       &        T    \\ \hline
\end{tabular}\end{center}
If all entries in the last column are true, the formula is a tautology.

Now, the truth table method doesn't tell you how to prove the tautology from
the axioms, but only that a proof exists.  Finding an actual proof (especially
one that is short and elegant) can be challenging.  Methods do exist for
automatically generating proofs in propositional calculus, but the proofs that
result can sometimes be very long.  In the Metamath \texttt{set.mm}\index{set
theory database (\texttt{set.mm})} database, most
or all proofs were created manually.

Section \ref{metadefprop} discusses various definitions
that make propositional calculus easier to use.
For example, we define:

\begin{itemize}
\item $\varphi \vee \psi$
  is true if either $\varphi$ or $\psi$ (or both) are true
  (this is disjunction\index{disjunction ($\vee$)}
  aka logical {\sc or}\index{logical {\sc or} ($\vee$)}).

\item $\varphi \wedge \psi$
  is true if both $\varphi$ and $\psi$ are true
  (this is conjunction\index{conjunction ($\wedge$)}
  aka logical {\sc and}\index{logical {\sc and} ($\wedge$)}).

\item $\varphi \leftrightarrow \psi$
  is true if $\varphi$ and $\psi$ have the same value, that is,
  they are both true or both false
  (this is the biconditional\index{biconditional ($\leftrightarrow$)}).
\end{itemize}

\subsection{Predicate Calculus}

Predicate calculus\index{predicate calculus} introduces the concept of
``individual variables,''\index{variable!in predicate calculus}\index{individual
variable} which
we will usually just call ``variables.''
These variables can represent something other than true or false (wffs),
and will always represent sets when we get to set theory.  There are also
three new symbols $\forall$\index{universal quantifier ($\forall$)},
$=$\index{equality ($=$)}, and $\in$\index{stylized epsilon ($\in$)},
read ``for all,'' ``equals,'' and ``is an element of''
respectively.  We will represent variables with the letters $x$, $y$, $z$, and
$w$, as is common practice in the literature.
For example, $\forall x \varphi$ means ``for all possible values of
$x$, $\varphi$ is true.''

In predicate calculus, we extend the definition of a wff\index{well-formed
formula (wff)}.  If $\varphi$ is a wff and $x$ and $y$ are variables, then
$\forall x \, \varphi$, $x=y$, and $x\in y$ are wffs. Note that these three new
types of wffs can be considered ``starting'' wffs from which we can build
other wffs with $\rightarrow$ and $\neg$ .  The concept of a starting wff was
absent in propositional calculus.  But starting wff or not, all we are really
concerned with is whether our wffs are correctly constructed according to
these mechanical rules.

A quick aside:
To prevent confusion, it might be best at this point to think of the variables
of Metamath\index{Metamath} as ``metavariables,''\index{metavariable} because
they are not quite the same as the variables we are introducing here.  A
(meta)variable in Metamath can be a wff or an individual variable, as well
as many other things; in general, it represents a kind of place holder for an
unspecified sequence of math symbols\index{math symbol}.

Unlike propositional calculus, no decision procedure\index{decision procedure}
analogous to the truth table method exists (nor theoretically can exist) that
will definitely determine whether a formula is a theorem of predicate
calculus.  Much of the work in the field of automated theorem
proving\index{automated theorem proving} has been dedicated to coming up with
clever heuristics for proving theorems of predicate calculus, but they can
never be guaranteed to work always.

Section \ref{metadefpred} discusses various definitions
that make predicate calculus easier to use.
For example, we define
$\exists x \varphi$ to mean
``there exists at least one possible value of $x$ where $\varphi$ is true.''

We now turn to looking at how predicate calculus can be formally
represented.

\subsubsection{Common Axioms}

There is a new rule of inference in predicate calculus:  if $\varphi$ is
an axiom or a theorem, then $\forall x \,\varphi$ is also a
theorem\index{theorem}.  This is called the rule of
``generalization.''\index{rule of generalization}
This is easily represented in Metamath.

In standard texts of logic, there are often two axioms of predicate
calculus\index{axioms of predicate calculus}:
\begin{center}
  $\forall x \,\varphi ( x ) \rightarrow \varphi ( y )$,
      where ``$y$ is properly substituted for $x$.''\\
  $\forall x ( \varphi \rightarrow \psi )\rightarrow ( \varphi \rightarrow
    \forall x\, \psi )$,
    where ``$x$ is not free in $\varphi$.''
\end{center}

Now at first glance, this seems simple:  just two axioms.  However,
conditional clauses are attached to each axiom describing requirements that
may seem puzzling to you.  In addition, the first axiom puts a variable symbol
in parentheses after each wff, seemingly violating our definition of a
wff\index{well-formed formula (wff)}; this is just an informal way of
referring to some arbitrary variable that may occur in the wff.  The
conditional clauses do, of course, have a precise meaning, but as it turns out
the precise meaning is somewhat complicated and awkward to formalize in a
way that a computer can handle easily.  Unlike propositional calculus, a
certain amount of mathematical sophistication and practice is needed to be
able to easily grasp and manipulate these concepts correctly.

Predicate calculus may be presented with or without axioms for
equality\index{axioms of equality}\index{equality ($=$)}. We will require the
axioms of equality as a prerequisite for the version of set theory we will
use.  The axioms for equality, when included, are often represented using these
two axioms:
\begin{center}
$x=x$\\ \ \\
$x=y\rightarrow (\varphi(x,x)\rightarrow\varphi(x,y))$ where ``$\varphi(x,y)$
   arises from $\varphi(x,x)$ by replacing some, but not necessarily all,
   free\index{free variable}
   occurrences of $x$ by $y$,\\ provided that $y$ is free for $x$
   in $\varphi(x,x)$.'' \end{center}
% (Mendelson p. 95)
The first equality axiom is simple, but again,
the condition on the second one is
somewhat awkward to implement on a computer.

\subsubsection{Tarski System S2}

Of course, we are not the first to notice the complications of these
predicate calculus axioms when being rigorous.

Well-known logician Alfred Tarski published in 1965
a system he called system S2\cite[p.~77]{Tarski1965}.
Tarski's system is \textit{exactly equivalent} to the traditional textbook
formalization, but (by clever use of equality axioms) it eliminates the
latter's primitive notions of ``proper substitution'' and ``free variable,''
replacing them with direct substitution and the notion of a variable
not occurring in a formula (which we express with distinct variable
constraints).

In advocating his system, Tarski wrote, ``The relatively complicated
character of [free variables and proper substitution] is a source
of certain inconveniences of both practical and theoretical nature;
this is clearly experienced both in teaching an elementary course of
mathematical logic and in formalizing the syntax of predicate logic for
some theoretical purposes''\cite[p.~61]{Tarski1965}\index{Tarski, Alfred}.

\subsubsection{Developing a Metamath Representation}

The standard textbook axioms of predicate calculus are somewhat
cumbersome to implement on a computer because of the complex notions of
``free variable''\index{free variable} and ``proper
substitution.''\index{proper substitution}\index{substitution!proper}
While it is possible to use the Metamath\index{Metamath} language to
implement these concepts, we have chosen not to implement them
as primitive constructs in the
\texttt{set.mm} set theory database.  Instead, we have eliminated them
within the axioms
by carefully crafting the axioms so as to avoid them,
building on Tarski's system S2.  This makes it
easy for a beginner to follow the steps in a proof without knowing any
advanced concepts other than the simple concept of
replacing\index{substitution!variable}\index{variable substitution}
variables with expressions.

In order to develop the concepts of free variable and proper
substitution from the axioms, we use an additional
Metamath statement type called ``disjoint variable
restriction''\index{disjoint variables} that we have not encountered
before.  In the context of the axioms, the statement \texttt{\$d} $ x\,
y$\index{\texttt{\$d} statement} simply means that $x$ and $y$ must be
distinct\index{distinct variables}, i.e.\ they may not be simultaneously
substituted\index{substitution!variable}\index{variable substitution}
with the same variable.  The statement \texttt{\$d} $ x\, \varphi$ means
variable $x$ must not occur in wff $\varphi$.  For the precise
definition of \texttt{\$d}, see Section~\ref{dollard}.

\subsubsection{Metamath representation}

The Metamath axiom system for predicate calculus
defined in set.mm uses Tarski's system S2.
As noted above, this has a different representation
than the traditional textbook formalization,
but it is \textit{exactly equivalent} to the textbook formalization,
and it is \textit{much} easier to work with.
This is reproduced as system S3 in Section 6 of
Megill's formalization \cite{Megill}\index{Megill, Norman}.

There is one exception, Tarski's axiom of existence,
which we label as axiom ax-6.
In the case of ax-6, Tarski's version is weaker because it includes a
distinct variable proviso. If we wish, we can also weaken our version
in this way and still have a metalogically complete system. Theorem
ax6 shows this by deriving, in the presence of the other axioms, our
ax-6 from Tarski's weaker version ax6v. However, we chose the stronger
version for our system because it is simpler to state and easier to use.

Tarski's system was designed for proving specific theorems rather than
more general theorem schemes. However, theorem schemes are much more
efficient than specific theorems for building a body of mathematical
knowledge, since they can be reused with different instances as
needed. While Tarski does derive some theorem schemes from his axioms,
their proofs require concepts that are ``outside'' of the system, such as
induction on formula length. The verification of such proofs is difficult
to automate in a proof verifier. (Specifically, Tarski treats the formulas
of his system as set-theoretical objects. In order to verify the proofs
of his theorem schemes, a proof verifier would need a significant amount
of set theory built into it.)

The Metamath axiom system for predicate calculus extends
Tarski's system to eliminate this difficulty. The additional
``auxilliary'' axiom
schemes (as we will call them in this section; see below) endow Tarski's
system with a nice property we call
metalogical completeness \cite[Remark 9.6]{Megill}\index{Megill, Norman}.
As a result, we can prove any theorem scheme
expressable in the ``simple metalogic'' of Tarski's system by using
only Metamath's direct substitution rule applied to the axiom system
(and no other metalogical or set-theoretical notions ``outside'' of the
system). Simple metalogic consists of schemes containing wff metavariables
(with no arguments) and/or set (also called ``individual'') metavariables,
accompanied by optional provisos each stating that two specified set
metavariables must be distinct or that a specified set metavariable may
not occur in a specified wff metavariable. Metamath's logic and set theory
axiom and rule schemes are all examples of simple metalogic. The schemes
of traditional predicate calculus with equality are examples which are
not simple metalogic, because they use wff metavariables with arguments
and have ``free for'' and ``not free in'' side conditions.

A rigorous justification for this system, using an older but
exactly equivalent set of axioms, can be
found in \cite{Megill}\index{Megill, Norman}.

This allows us to
take a different approach in the Metamath\index{Metamath} database
\texttt{set.mm}\index{set theory database (\texttt{set.mm})}.  We do not
directly use the primitive notions of ``free variable''\index{free variable}
and ``proper substitution''\index{proper
substitution}\index{substitution!proper} at all as primitive constructs.
Instead, we use a set
of axioms that are almost as simple to manipulate as those of
propositional calculus.  Our axiom system avoids complex primitive
notions by effectively embedding the complexity into the axioms
themselves.  As a result, we will end up with a larger number of axioms,
but they are ideally suited for a computer language such as Metamath.
(Section~\ref{metaaxioms} shows these axioms.)

We will not elaborate further
on the ``free variable'' and ``proper substitution''
concepts here.  You may consult
\cite[ch.\ 3--4]{Hamilton}\index{Hamilton, Alan G.} (as well as
many other books) for a precise explanation
of these concepts.  If you intend to do serious mathematical work, it is wise
to become familiar with the traditional textbook approach; even though the
concepts embedded in their axioms require a higher level of sophistication,
they can be more practical to deal with on an everyday, informal basis.  Even
if you are just developing Metamath proofs, familiarity with the traditional
approach can help you arrive at a proof outline much faster, which you can
then convert to the detail required by Metamath.

We do develop proper substitution rules later on, but in set.mm
they are defined as derived constructs; they are not primitives.

You should also note that our system of predicate calculus is specifically
tailored for set theory; thus there are only two specific predicates $=$ and
$\in$ and no functions\index{function!in predicate calculus}
or constants\index{constant!in predicate calculus} unlike more general systems.
We later add these.

\subsection{Set Theory}

Traditional Zermelo--Fraenkel set theory\index{Zermelo--Fraenkel set
theory}\index{set theory} with the Axiom of Choice
has 10 axioms, which can be expressed in the
language of predicate calculus.  In this section, we will list only the
names and brief English descriptions of these axioms, since we will give
you the precise formulas used by the Metamath\index{Metamath} set theory
database \texttt{set.mm} later on.

In the descriptions of the axioms, we assume that $x$, $y$, $z$, $w$, and $v$
represent sets.  These are the same as the variables\index{variable!in set
theory} in our predicate calculus system above, except that now we informally
think of the variables as ranging over sets.  Note that the terms
``object,''\index{object} ``set,''\index{set} ``element,''\index{element}
``collection,''\index{collection} and ``family''\index{family} are synonymous,
as are ``is an element of,'' ``is a member of,''\index{member} ``is contained
in,'' and ``belongs to.''  The different terms are used for convenience; for
example, ``a collection of sets'' is less confusing than ``a set of sets.''
A set $x$ is said to be a ``subset''\index{subset} of $y$ if every element of
$x$ is also an element of $y$; we also say $x$ is ``included in''
$y$.

The axioms are very general and apply to almost any conceivable mathematical
object, and this level of abstraction can be overwhelming at first.  To gain an
intuitive feel, it can be helpful to draw a picture illustrating the concept;
for example, a circle containing dots could represent a collection of sets,
and a smaller circle drawn inside the circle could represent a subset.
Overlapping circles can illustrate intersection and union.  Circles that
illustrate the concepts of set theory are frequently used in elementary
textbooks and are called Venn diagrams\index{Venn diagram}.\index{axioms of
set theory}

1. Axiom of Extensionality:  Two sets are identical if they contain the same
   elements.\index{Axiom of Extensionality}

2. Axiom of Pairing:  The set $\{ x , y \}$ exists.\index{Axiom of Pairing}

3. Axiom of Power Sets:  The power set of a set (the collection of all of
   its subsets) exists.  For example, the power set of $\{x,y\}$ is
   $\{\varnothing,\{x\},\{y\},\{x,y\}\}$ and it exists.\index{Axiom
of Power Sets}

4. Axiom of the Null Set:  The empty set $\varnothing$ exists.\index{Axiom of
the Null Set}

5. Axiom of Union:  The union of a set (the set containing the elements of
   its members) exists.  For example, the union of $\{\{x,y\},\{z\}\}$ is
 $\{x,y,z\}$ and
   it exists.\index{Axiom of Union}

6. Axiom of Regularity:  Roughly, no set can contain itself, nor can there
   be membership ``loops,'' such as a set being an
   element of one of its members.\index{Axiom of Regularity}

7. Axiom of Infinity:  An infinite set exists.  An example of an infinite
   set is the set of all
   integers.\index{Axiom of Infinity}

8. Axiom of Separation:  The set exists that is obtained by restricting $x$
   with some property.  For example, if the set of all integers exists,
   then the set of all even integers exists.\index{Axiom of Separation}

9. Axiom of Replacement:  The range of a function whose domain is restricted
   to the elements of a set $x$, is also a set.  For example, there
   is a function
   from integers (the function's domain) to their squares (its
   range).  If we
   restrict the domain to even integers, its range will become the set of
   squares of even integers, so this axiom asserts that the set of
    squares of even numbers exists.  Technical note:  In general, the
   ``function'' need not be a set but can be a proper class.
   \index{Axiom of Replacement}

10. Axiom of Choice:  Let $x$ be a set whose members are pairwise
  disjoint\index{disjoint sets} (i.e,
  whose members contain no elements in common).  Then there exists another
  set containing one element from each member of $x$.  For
  example, if $x$ is
  $\{\{y,z\},\{w,v\}\}$, where $y$, $z$, $w$, and $v$ are
  different sets, then a set such as $\{z,w\}$
  exists (but the axiom doesn't tell
  us which one).  (Actually the Axiom
  of Choice is redundant if the set $x$, as in this example, has a finite
  number of elements.)\index{Axiom of Choice}

The Axiom of Choice is usually considered an extension of ZF set theory rather
than a proper part of it.  It is sometimes considered philosophically
controversial because it specifies the existence of a set without specifying
what the set is. Constructive logics, including intuitionistic logic,
do not accept the axiom of choice.
Since there is some lingering controversy, we often prefer proofs that do
not use the axiom of choice (where there is a known alternative), and
in some cases we will use weaker axioms than the full axiom of choice.
That said, the axiom of choice is a powerful and widely-accepted tool,
so we do use it when needed.
ZF set theory that includes the Axiom of Choice is
called Zermelo--Fraenkel set theory with choice (ZFC\index{ZFC set theory}).

When expressed symbolically, the Axiom of Separation and the Axiom of
Replacement contain wff symbols and therefore each represent infinitely many
axioms, one for each possible wff. For this reason, they are often called
axiom schemes\index{axiom scheme}\index{well-formed formula (wff)}.

It turns out that the Axiom of the Null Set, the Axiom of Pairing, and the
Axiom of Separation can be derived from the other axioms and are therefore
unnecessary, although they tend to be included in standard texts for various
reasons (historical, philosophical, and possibly because some authors may not
know this).  In the Metamath\index{Metamath} set theory database, these
redundant axioms are derived from the other ones instead of truly
being considered axioms.
This is in keeping with our general goal of minimizing the number of
axioms we must depend on.

\subsection{Other Axioms}

Above we qualified the phrase ``all of mathematics'' with ``essentially.''
The main important missing piece is the ability to do category theory,
which requires huge sets (inaccessible cardinals) larger than those
postulated by the ZFC axioms. The Tarski--Grothendieck Axiom postulates
the existence of such sets.
Note that this is the same axiom used by Mizar for supporting
category theory.
The Tarski--Grothendieck axiom
can be viewed as a very strong replacement of the Axiom of Infinity,
the Axiom of Choice, and the Axiom of Power Sets.
The \texttt{set.mm} database includes this axiom; see the database
for details about it.
Again, we only use this axiom when we need to.
You are only likely to encounter or use this axiom if you are doing
category theory, since its use is highly specialized,
so we will not list the Tarsky-Grothendieck axiom
in the short list of axioms below.

Can there be even more axioms?
Of course.
G\"{o}del showed that no finite set of axioms or axiom schemes can completely
describe any consistent theory strong enough to include arithmetic.
But practically speaking, the ones above are the accepted foundation that
almost all mathematicians explicitly or implicitly base their work on.

\section{The Axioms in the Metamath Language}\label{metaaxioms}

Here we list the axioms as they appear in
\texttt{set.mm}\index{set theory database (\texttt{set.mm})} so you can
look them up there easily.  Incidentally, the \texttt{show statement
/tex} command\index{\texttt{show statement} command} was used to
typeset them.

%macros from show statement /tex
\newbox\mlinebox
\newbox\mtrialbox
\newbox\startprefix  % Prefix for first line of a formula
\newbox\contprefix  % Prefix for continuation line of a formula
\def\startm{  % Initialize formula line
  \setbox\mlinebox=\hbox{\unhcopy\startprefix}
}
\def\m#1{  % Add a symbol to the formula
  \setbox\mtrialbox=\hbox{\unhcopy\mlinebox $\,#1$}
  \ifdim\wd\mtrialbox>\hsize
    \box\mlinebox
    \setbox\mlinebox=\hbox{\unhcopy\contprefix $\,#1$}
  \else
    \setbox\mlinebox=\hbox{\unhbox\mtrialbox}
  \fi
}
\def\endm{  % Output the last line of a formula
  \box\mlinebox
}

% \SLASH for \ , \TOR for \/ (text OR), \TAND for /\ (text and)
% This embeds a following forced space to force the space.
\newcommand\SLASH{\char`\\~}
\newcommand\TOR{\char`\\/~}
\newcommand\TAND{/\char`\\~}
%
% Macro to output metamath raw text.
% This assumes \startprefix and \contprefix are set.
% NOTE: "\" is tricky to escape, use \SLASH, \TOR, and \TAND inside.
% Any use of "$ { ~ ^" must be escaped; ~ and ^ must be escaped specially.
% We escape { and } for consistency.
% For more about how this macro written, see:
% https://stackoverflow.com/questions/4073674/
% how-to-disable-indentation-in-particular-section-in-latex/4075706
% Use frenchspacing, or "e." will get an extra space after it.
\newlength\mystoreparindent
\newlength\mystorehangindent
\newenvironment{mmraw}{%
\setlength{\mystoreparindent}{\the\parindent}
\setlength{\mystorehangindent}{\the\hangindent}
\setlength{\parindent}{0pt} % TODO - we'll put in the \startprefix instead
\setlength{\hangindent}{\wd\the\contprefix}
\begin{flushleft}
\begin{frenchspacing}
\begin{tt}
{\unhcopy\startprefix}%
}{%
\end{tt}
\end{frenchspacing}
\end{flushleft}
\setlength{\parindent}{\mystoreparindent}
\setlength{\hangindent}{\mystorehangindent}
\vskip 1ex
}

\needspace{5\baselineskip}
\subsection{Propositional Calculus}\label{propcalc}\index{axioms of
propositional calculus}

\needspace{2\baselineskip}
Axiom of Simplification.\label{ax1}

\setbox\startprefix=\hbox{\tt \ \ ax-1\ \$a\ }
\setbox\contprefix=\hbox{\tt \ \ \ \ \ \ \ \ \ \ }
\startm
\m{\vdash}\m{(}\m{\varphi}\m{\rightarrow}\m{(}\m{\psi}\m{\rightarrow}\m{\varphi}\m{)}
\m{)}
\endm

\needspace{3\baselineskip}
\noindent Axiom of Distribution.

\setbox\startprefix=\hbox{\tt \ \ ax-2\ \$a\ }
\setbox\contprefix=\hbox{\tt \ \ \ \ \ \ \ \ \ \ }
\startm
\m{\vdash}\m{(}\m{(}\m{\varphi}\m{\rightarrow}\m{(}\m{\psi}\m{\rightarrow}\m{\chi}
\m{)}\m{)}\m{\rightarrow}\m{(}\m{(}\m{\varphi}\m{\rightarrow}\m{\psi}\m{)}\m{
\rightarrow}\m{(}\m{\varphi}\m{\rightarrow}\m{\chi}\m{)}\m{)}\m{)}
\endm

\needspace{2\baselineskip}
\noindent Axiom of Contraposition.

\setbox\startprefix=\hbox{\tt \ \ ax-3\ \$a\ }
\setbox\contprefix=\hbox{\tt \ \ \ \ \ \ \ \ \ \ }
\startm
\m{\vdash}\m{(}\m{(}\m{\lnot}\m{\varphi}\m{\rightarrow}\m{\lnot}\m{\psi}\m{)}\m{
\rightarrow}\m{(}\m{\psi}\m{\rightarrow}\m{\varphi}\m{)}\m{)}
\endm


\needspace{4\baselineskip}
\noindent Rule of Modus Ponens.\label{axmp}\index{modus ponens}

\setbox\startprefix=\hbox{\tt \ \ min\ \$e\ }
\setbox\contprefix=\hbox{\tt \ \ \ \ \ \ \ \ \ }
\startm
\m{\vdash}\m{\varphi}
\endm

\setbox\startprefix=\hbox{\tt \ \ maj\ \$e\ }
\setbox\contprefix=\hbox{\tt \ \ \ \ \ \ \ \ \ }
\startm
\m{\vdash}\m{(}\m{\varphi}\m{\rightarrow}\m{\psi}\m{)}
\endm

\setbox\startprefix=\hbox{\tt \ \ ax-mp\ \$a\ }
\setbox\contprefix=\hbox{\tt \ \ \ \ \ \ \ \ \ \ \ }
\startm
\m{\vdash}\m{\psi}
\endm


\needspace{7\baselineskip}
\subsection{Axioms of Predicate Calculus with Equality---Tarski's S2}\index{axioms of predicate calculus}

\needspace{3\baselineskip}
\noindent Rule of Generalization.\index{rule of generalization}

\setbox\startprefix=\hbox{\tt \ \ ax-g.1\ \$e\ }
\setbox\contprefix=\hbox{\tt \ \ \ \ \ \ \ \ \ \ \ \ }
\startm
\m{\vdash}\m{\varphi}
\endm

\setbox\startprefix=\hbox{\tt \ \ ax-gen\ \$a\ }
\setbox\contprefix=\hbox{\tt \ \ \ \ \ \ \ \ \ \ \ \ }
\startm
\m{\vdash}\m{\forall}\m{x}\m{\varphi}
\endm

\needspace{2\baselineskip}
\noindent Axiom of Quantified Implication.

\setbox\startprefix=\hbox{\tt \ \ ax-4\ \$a\ }
\setbox\contprefix=\hbox{\tt \ \ \ \ \ \ \ \ \ \ }
\startm
\m{\vdash}\m{(}\m{\forall}\m{x}\m{(}\m{\forall}\m{x}\m{\varphi}\m{\rightarrow}\m{
\psi}\m{)}\m{\rightarrow}\m{(}\m{\forall}\m{x}\m{\varphi}\m{\rightarrow}\m{
\forall}\m{x}\m{\psi}\m{)}\m{)}
\endm

\needspace{3\baselineskip}
\noindent Axiom of Distinctness.

% Aka: Add $d x ph $.
\setbox\startprefix=\hbox{\tt \ \ ax-5\ \$a\ }
\setbox\contprefix=\hbox{\tt \ \ \ \ \ \ \ \ \ \ }
\startm
\m{\vdash}\m{(}\m{\varphi}\m{\rightarrow}\m{\forall}\m{x}\m{\varphi}\m{)}\m{where}\m{ }\m{\$d}\m{ }\m{x}\m{ }\m{\varphi}\m{ }\m{(}\m{x}\m{ }\m{does}\m{ }\m{not}\m{ }\m{occur}\m{ }\m{in}\m{ }\m{\varphi}\m{)}
\endm

\needspace{2\baselineskip}
\noindent Axiom of Existence.

\setbox\startprefix=\hbox{\tt \ \ ax-6\ \$a\ }
\setbox\contprefix=\hbox{\tt \ \ \ \ \ \ \ \ \ \ }
\startm
\m{\vdash}\m{(}\m{\forall}\m{x}\m{(}\m{x}\m{=}\m{y}\m{\rightarrow}\m{\forall}
\m{x}\m{\varphi}\m{)}\m{\rightarrow}\m{\varphi}\m{)}
\endm

\needspace{2\baselineskip}
\noindent Axiom of Equality.

\setbox\startprefix=\hbox{\tt \ \ ax-7\ \$a\ }
\setbox\contprefix=\hbox{\tt \ \ \ \ \ \ \ \ \ \ }
\startm
\m{\vdash}\m{(}\m{x}\m{=}\m{y}\m{\rightarrow}\m{(}\m{x}\m{=}\m{z}\m{
\rightarrow}\m{y}\m{=}\m{z}\m{)}\m{)}
\endm

\needspace{2\baselineskip}
\noindent Axiom of Left Equality for Binary Predicate.

\setbox\startprefix=\hbox{\tt \ \ ax-8\ \$a\ }
\setbox\contprefix=\hbox{\tt \ \ \ \ \ \ \ \ \ \ \ }
\startm
\m{\vdash}\m{(}\m{x}\m{=}\m{y}\m{\rightarrow}\m{(}\m{x}\m{\in}\m{z}\m{
\rightarrow}\m{y}\m{\in}\m{z}\m{)}\m{)}
\endm

\needspace{2\baselineskip}
\noindent Axiom of Right Equality for Binary Predicate.

\setbox\startprefix=\hbox{\tt \ \ ax-9\ \$a\ }
\setbox\contprefix=\hbox{\tt \ \ \ \ \ \ \ \ \ \ \ }
\startm
\m{\vdash}\m{(}\m{x}\m{=}\m{y}\m{\rightarrow}\m{(}\m{z}\m{\in}\m{x}\m{
\rightarrow}\m{z}\m{\in}\m{y}\m{)}\m{)}
\endm


\needspace{4\baselineskip}
\subsection{Axioms of Predicate Calculus with Equality---Auxiliary}\index{axioms of predicate calculus - auxiliary}

\needspace{2\baselineskip}
\noindent Axiom of Quantified Negation.

\setbox\startprefix=\hbox{\tt \ \ ax-10\ \$a\ }
\setbox\contprefix=\hbox{\tt \ \ \ \ \ \ \ \ \ \ }
\startm
\m{\vdash}\m{(}\m{\lnot}\m{\forall}\m{x}\m{\lnot}\m{\forall}\m{x}\m{\varphi}\m{
\rightarrow}\m{\varphi}\m{)}
\endm

\needspace{2\baselineskip}
\noindent Axiom of Quantifier Commutation.

\setbox\startprefix=\hbox{\tt \ \ ax-11\ \$a\ }
\setbox\contprefix=\hbox{\tt \ \ \ \ \ \ \ \ \ \ }
\startm
\m{\vdash}\m{(}\m{\forall}\m{x}\m{\forall}\m{y}\m{\varphi}\m{\rightarrow}\m{
\forall}\m{y}\m{\forall}\m{x}\m{\varphi}\m{)}
\endm

\needspace{3\baselineskip}
\noindent Axiom of Substitution.

\setbox\startprefix=\hbox{\tt \ \ ax-12\ \$a\ }
\setbox\contprefix=\hbox{\tt \ \ \ \ \ \ \ \ \ \ \ }
\startm
\m{\vdash}\m{(}\m{\lnot}\m{\forall}\m{x}\m{\,x}\m{=}\m{y}\m{\rightarrow}\m{(}
\m{x}\m{=}\m{y}\m{\rightarrow}\m{(}\m{\varphi}\m{\rightarrow}\m{\forall}\m{x}\m{(}
\m{x}\m{=}\m{y}\m{\rightarrow}\m{\varphi}\m{)}\m{)}\m{)}\m{)}
\endm

\needspace{3\baselineskip}
\noindent Axiom of Quantified Equality.

\setbox\startprefix=\hbox{\tt \ \ ax-13\ \$a\ }
\setbox\contprefix=\hbox{\tt \ \ \ \ \ \ \ \ \ \ \ }
\startm
\m{\vdash}\m{(}\m{\lnot}\m{\forall}\m{z}\m{\,z}\m{=}\m{x}\m{\rightarrow}\m{(}
\m{\lnot}\m{\forall}\m{z}\m{\,z}\m{=}\m{y}\m{\rightarrow}\m{(}\m{x}\m{=}\m{y}
\m{\rightarrow}\m{\forall}\m{z}\m{\,x}\m{=}\m{y}\m{)}\m{)}\m{)}
\endm

% \noindent Axiom of Quantifier Substitution
%
% \setbox\startprefix=\hbox{\tt \ \ ax-c11n\ \$a\ }
% \setbox\contprefix=\hbox{\tt \ \ \ \ \ \ \ \ \ \ \ }
% \startm
% \m{\vdash}\m{(}\m{\forall}\m{x}\m{\,x}\m{=}\m{y}\m{\rightarrow}\m{(}\m{\forall}
% \m{x}\m{\varphi}\m{\rightarrow}\m{\forall}\m{y}\m{\varphi}\m{)}\m{)}
% \endm
%
% \noindent Axiom of Distinct Variables. (This axiom requires
% that two individual variables
% be distinct\index{\texttt{\$d} statement}\index{distinct
% variables}.)
%
% \setbox\startprefix=\hbox{\tt \ \ \ \ \ \ \ \ \$d\ }
% \setbox\contprefix=\hbox{\tt \ \ \ \ \ \ \ \ \ \ \ }
% \startm
% \m{x}\m{\,}\m{y}
% \endm
%
% \setbox\startprefix=\hbox{\tt \ \ ax-c16\ \$a\ }
% \setbox\contprefix=\hbox{\tt \ \ \ \ \ \ \ \ \ \ \ }
% \startm
% \m{\vdash}\m{(}\m{\forall}\m{x}\m{\,x}\m{=}\m{y}\m{\rightarrow}\m{(}\m{\varphi}\m{
% \rightarrow}\m{\forall}\m{x}\m{\varphi}\m{)}\m{)}
% \endm

% \noindent Axiom of Quantifier Introduction (2).  (This axiom requires
% that the individual variable not occur in the
% wff\index{\texttt{\$d} statement}\index{distinct variables}.)
%
% \setbox\startprefix=\hbox{\tt \ \ \ \ \ \ \ \ \$d\ }
% \setbox\contprefix=\hbox{\tt \ \ \ \ \ \ \ \ \ \ \ }
% \startm
% \m{x}\m{\,}\m{\varphi}
% \endm
% \setbox\startprefix=\hbox{\tt \ \ ax-5\ \$a\ }
% \setbox\contprefix=\hbox{\tt \ \ \ \ \ \ \ \ \ \ \ }
% \startm
% \m{\vdash}\m{(}\m{\varphi}\m{\rightarrow}\m{\forall}\m{x}\m{\varphi}\m{)}
% \endm

\subsection{Set Theory}\label{mmsettheoryaxioms}

In order to make the axioms of set theory\index{axioms of set theory} a little
more compact, there are several definitions from logic that we make use of
implicitly, namely, ``logical {\sc and},''\index{conjunction ($\wedge$)}
\index{logical {\sc and} ($\wedge$)} ``logical equivalence,''\index{logical
equivalence ($\leftrightarrow$)}\index{biconditional ($\leftrightarrow$)} and
``there exists.''\index{existential quantifier ($\exists$)}

\begin{center}\begin{tabular}{rcl}
  $( \varphi \wedge \psi )$ &\mbox{stands for}& $\neg ( \varphi
     \rightarrow \neg \psi )$\\
  $( \varphi \leftrightarrow \psi )$& \mbox{stands
     for}& $( ( \varphi \rightarrow \psi ) \wedge
     ( \psi \rightarrow \varphi ) )$\\
  $\exists x \,\varphi$ &\mbox{stands for}& $\neg \forall x \neg \varphi$
\end{tabular}\end{center}

In addition, the axioms of set theory require that all variables be
dis\-tinct,\index{distinct variables}\footnote{Set theory axioms can be
devised so that {\em no} variables are required to be distinct,
provided we replace \texttt{ax-c16} with an axiom stating that ``at
least two things exist,'' thus
making \texttt{ax-5} the only other axiom requiring the
\texttt{\$d} statement.  These axioms are unconventional and are not
presented here, but they can be found on the \url{http://metamath.org}
web site.  See also the Comment on
p.~\pageref{nodd}.}\index{\texttt{\$d} statement} thus we also assume:
\begin{center}
  \texttt{\$d }$x\,y\,z\,w$
\end{center}

\needspace{2\baselineskip}
\noindent Axiom of Extensionality.\index{Axiom of Extensionality}

\setbox\startprefix=\hbox{\tt \ \ ax-ext\ \$a\ }
\setbox\contprefix=\hbox{\tt \ \ \ \ \ \ \ \ \ \ \ \ }
\startm
\m{\vdash}\m{(}\m{\forall}\m{x}\m{(}\m{x}\m{\in}\m{y}\m{\leftrightarrow}\m{x}
\m{\in}\m{z}\m{)}\m{\rightarrow}\m{y}\m{=}\m{z}\m{)}
\endm

\needspace{3\baselineskip}
\noindent Axiom of Replacement.\index{Axiom of Replacement}

\setbox\startprefix=\hbox{\tt \ \ ax-rep\ \$a\ }
\setbox\contprefix=\hbox{\tt \ \ \ \ \ \ \ \ \ \ \ \ }
\startm
\m{\vdash}\m{(}\m{\forall}\m{w}\m{\exists}\m{y}\m{\forall}\m{z}\m{(}\m{%
\forall}\m{y}\m{\varphi}\m{\rightarrow}\m{z}\m{=}\m{y}\m{)}\m{\rightarrow}\m{%
\exists}\m{y}\m{\forall}\m{z}\m{(}\m{z}\m{\in}\m{y}\m{\leftrightarrow}\m{%
\exists}\m{w}\m{(}\m{w}\m{\in}\m{x}\m{\wedge}\m{\forall}\m{y}\m{\varphi}\m{)}%
\m{)}\m{)}
\endm

\needspace{2\baselineskip}
\noindent Axiom of Union.\index{Axiom of Union}

\setbox\startprefix=\hbox{\tt \ \ ax-un\ \$a\ }
\setbox\contprefix=\hbox{\tt \ \ \ \ \ \ \ \ \ \ \ }
\startm
\m{\vdash}\m{\exists}\m{x}\m{\forall}\m{y}\m{(}\m{\exists}\m{x}\m{(}\m{y}\m{
\in}\m{x}\m{\wedge}\m{x}\m{\in}\m{z}\m{)}\m{\rightarrow}\m{y}\m{\in}\m{x}\m{)}
\endm

\needspace{2\baselineskip}
\noindent Axiom of Power Sets.\index{Axiom of Power Sets}

\setbox\startprefix=\hbox{\tt \ \ ax-pow\ \$a\ }
\setbox\contprefix=\hbox{\tt \ \ \ \ \ \ \ \ \ \ \ \ }
\startm
\m{\vdash}\m{\exists}\m{x}\m{\forall}\m{y}\m{(}\m{\forall}\m{x}\m{(}\m{x}\m{
\in}\m{y}\m{\rightarrow}\m{x}\m{\in}\m{z}\m{)}\m{\rightarrow}\m{y}\m{\in}\m{x}
\m{)}
\endm

\needspace{3\baselineskip}
\noindent Axiom of Regularity.\index{Axiom of Regularity}

\setbox\startprefix=\hbox{\tt \ \ ax-reg\ \$a\ }
\setbox\contprefix=\hbox{\tt \ \ \ \ \ \ \ \ \ \ \ \ }
\startm
\m{\vdash}\m{(}\m{\exists}\m{x}\m{\,x}\m{\in}\m{y}\m{\rightarrow}\m{\exists}
\m{x}\m{(}\m{x}\m{\in}\m{y}\m{\wedge}\m{\forall}\m{z}\m{(}\m{z}\m{\in}\m{x}\m{
\rightarrow}\m{\lnot}\m{z}\m{\in}\m{y}\m{)}\m{)}\m{)}
\endm

\needspace{3\baselineskip}
\noindent Axiom of Infinity.\index{Axiom of Infinity}

\setbox\startprefix=\hbox{\tt \ \ ax-inf\ \$a\ }
\setbox\contprefix=\hbox{\tt \ \ \ \ \ \ \ \ \ \ \ \ \ \ \ }
\startm
\m{\vdash}\m{\exists}\m{x}\m{(}\m{y}\m{\in}\m{x}\m{\wedge}\m{\forall}\m{y}%
\m{(}\m{y}\m{\in}\m{x}\m{\rightarrow}\m{\exists}\m{z}\m{(}\m{y}\m{\in}\m{z}\m{%
\wedge}\m{z}\m{\in}\m{x}\m{)}\m{)}\m{)}
\endm

\needspace{4\baselineskip}
\noindent Axiom of Choice.\index{Axiom of Choice}

\setbox\startprefix=\hbox{\tt \ \ ax-ac\ \$a\ }
\setbox\contprefix=\hbox{\tt \ \ \ \ \ \ \ \ \ \ \ \ \ \ }
\startm
\m{\vdash}\m{\exists}\m{x}\m{\forall}\m{y}\m{\forall}\m{z}\m{(}\m{(}\m{y}\m{%
\in}\m{z}\m{\wedge}\m{z}\m{\in}\m{w}\m{)}\m{\rightarrow}\m{\exists}\m{w}\m{%
\forall}\m{y}\m{(}\m{\exists}\m{w}\m{(}\m{(}\m{y}\m{\in}\m{z}\m{\wedge}\m{z}%
\m{\in}\m{w}\m{)}\m{\wedge}\m{(}\m{y}\m{\in}\m{w}\m{\wedge}\m{w}\m{\in}\m{x}%
\m{)}\m{)}\m{\leftrightarrow}\m{y}\m{=}\m{w}\m{)}\m{)}
\endm

\subsection{That's It}

There you have it, the axioms for (essentially) all of mathematics!
Wonder at them and stare at them in awe.  Put a copy in your wallet, and
you will carry in your pocket the encoding for all theorems ever proved
and that ever will be proved, from the most mundane to the most
profound.

\section{A Hierarchy of Definitions}\label{hierarchy}

The axioms in the previous section in principle embody everything that can be
done within standard mathematics.  However, it is impractical to accomplish
very much by using them directly, for even simple concepts (from a human
perspective) can involve extremely long, incomprehensible formulas.
Mathematics is made practical by introducing definitions\index{definition}.
Definitions usually introduce new symbols, or at least new relationships among
existing symbols, to abbreviate more complex formulas.  An important
requirement for a definition is that there exist a straightforward
(algorithmic) method for eliminating the abbreviation by expanding it into the
more primitive symbol string that it represents.  Some
important definitions included in
the file \texttt{set.mm} are listed in this section for reference, and also to
give you a feel for why something like $\omega$\index{omega ($\omega$)} (the
set of natural numbers\index{natural number} 0, 1, 2,\ldots) becomes very
complicated when completely expanded into primitive symbols.

What is the motivation for definitions, aside from allowing complicated
expressions to be expressed more simply?  In the case of  $\omega$, one goal is
to provide a basis for the theory of natural numbers.\index{natural number}
Before set theory was invented, a set of axioms for arithmetic, called Peano's
postulates\index{Peano's postulates}, was devised and shown to have the
properties one expects for natural numbers.  Now anyone can postulate a
set of axioms, but if the axioms are inconsistent contradictions can be derived
from them.  Once a contradiction is derived, anything can be trivially
proved, including
all the facts of arithmetic and their negations.  To ensure that an
axiom system is at least as reliable as the axioms for set theory, we can
define sets and operations on those sets that satisfy the new axioms. In the
\texttt{set.mm} Metamath database, we prove that the elements of $\omega$ satisfy
Peano's postulates, and it's a long and hard journey to get there directly
from the axioms of set theory.  But the result is confidence in the
foundations of arithmetic.  And there is another advantage:  we now have all
the tools of set theory at our disposal for manipulating objects that obey the
axioms for arithmetic.

What are the criteria we use for definitions?  First, and of utmost importance,
the definition should not be {\em creative}\index{creative
definition}\index{definition!creative}, that
is it should not allow an expression that previously qualified as a wff but
was not provable, to become provable.   Second, the definition should be {\em
eliminable}\index{definition!eliminability}, that is, there should exist an
algorithmic method for converting any expression using the definition into
a logically equivalent expression that previously qualified as a wff.

In almost all cases below, definitions connect two expressions with either
$\leftrightarrow$ or $=$.  Eliminating\footnote{Here we mean the
elimination that a human might do in his or her head.  To eliminate them as
part of a Metamath proof we would invoke one of a number of
theorems that deal with transitivity of equivalence or equality; there are
many such examples in the proofs in \texttt{set.mm}.} such a definition is a
simple matter of substituting the expression on the left-hand side ({\em
definiendum}\index{definiendum} or thing being defined) with the equivalent,
more primitive expression on the right-hand side ({\em
definiens}\index{definiens} or definition).

Often a definition has variables on the right-hand side which do not appear on
the left-hand side; these are called {\em dummy variables}.\index{dummy
variable!in definitions}  In this case, any
allowable substitution (such as a new, distinct
variable) can be used when the definition is eliminated.  Dummy variables may
be used only if they are {\em effectively bound}\index{effectively bound
variable}, meaning that the definition will remain logically equivalent upon
any substitution of a dummy variable with any other {\em qualifying
expression}\index{qualifying expression}, i.e.\ any symbol string (such as
another variable) that
meets the restrictions on the dummy variable imposed by \texttt{\$d} and
\texttt{\$f} statements.  For example, we could define a constant $\perp$
(inverted tee, meaning logical ``false'') as $( \varphi \wedge \lnot \varphi
)$, i.e.\ ``phi and not phi.''  Here $\varphi$ is effectively bound because the
definition remains logically equivalent when we replace $\varphi$ with any
other wff.  (It is actually \texttt{df-fal}
in \texttt{set.mm}, which defines $\perp$.)

There are two cases where eliminating definitions is a little more
complex.  These cases are the definitions \texttt{df-bi} and
\texttt{df-cleq}.  The first stretches the concept of a definition a
little, as in effect it ``defines a definition;'' however, it meets our
requirements for a definition in that it is eliminable and does not
strengthen the language.  Theorem \texttt{bii} shows the substitution
needed to eliminate the $\leftrightarrow$\index{logical equivalence
($\leftrightarrow$)}\index{biconditional ($\leftrightarrow$)} symbol.

Definition \texttt{df-cleq}\index{equality ($=$)} extends the usage of
the equality symbol to include ``classes''\index{class} in set theory.  The
reason it is potentially problematic is that it can lead to statements which
do not follow from logic alone but presuppose the Axiom of
Extensionality\index{Axiom of Extensionality}, so we include this axiom
as a hypothesis for the definition.  We could have made \texttt{df-cleq} directly
eliminable by introducing a new equality symbol, but have chosen not to do so
in keeping with standard textbook practice.  Definitions such as \texttt{df-cleq}
that extend the meaning of existing symbols must be introduced carefully so
that they do not lead to contradictions.  Definition \texttt{df-clel} also
extends the meaning of an existing symbol ($\in$); while it doesn't strengthen
the language like \texttt{df-cleq}, this is not obvious and it must also be
subject to the same scrutiny.

Exercise:  Study how the wff $x\in\omega$, meaning ``$x$ is a natural
number,'' could be expanded in terms of primitive symbols, starting with the
definitions \texttt{df-clel} on p.~\pageref{dfclel} and \texttt{df-om} on
p.~\pageref{dfom} and working your way back.  Don't bother to work out the
details; just make sure that you understand how you could do it in principle.
The answer is shown in the footnote on p.~\pageref{expandom}.  If you
actually do work it out, you won't get exactly the same answer because we used
a few simplifications such as discarding occurrences of $\lnot\lnot$ (double
negation).

In the definitions below, we have placed the {\sc ascii} Metamath source
below each of the formulas to help you become familiar with the
notation in the database.  For simplicity, the necessary \texttt{\$f}
and \texttt{\$d} statements are not shown.  If you are in doubt, use the
\texttt{show statement}\index{\texttt{show statement} command} command
in the Metamath program to see the full statement.
A selection of this notation is summarized in Appendix~\ref{ASCII}.

To understand the motivation for these definitions, you should consult the
references indicated:  Takeuti and Zaring \cite{Takeuti}\index{Takeuti, Gaisi},
Quine \cite{Quine}\index{Quine, Willard Van Orman}, Bell and Machover
\cite{Bell}\index{Bell, J. L.}, and Enderton \cite{Enderton}\index{Enderton,
Herbert B.}.  Our list of definitions is provided more for reference than as a
learning aid.  However, by looking at a few of them you can gain a feel for
how the hierarchy is built up.  The definitions are a representative sample of
the many definitions
in \texttt{set.mm}, but they are complete with respect to the
theorem examples we will present in Section~\ref{sometheorems}.  Also, some are
slightly different from, but logically equivalent to, the ones in \texttt{set.mm}
(some of which have been revised over time to shorten them, for example).

\subsection{Definitions for Propositional Calculus}\label{metadefprop}

The symbols $\varphi$, $\psi$, and $\chi$ represent wffs.

Our first definition introduces the biconditional
connective\footnote{The term ``connective'' is informally used to mean a
symbol that is placed between two variables or adjacent to a variable,
whereas a mathematical ``constant'' usually indicates a symbol such as
the number 0 that may replace a variable or metavariable.  From
Metamath's point of view, there is no distinction between a connective
and a constant; both are constants in the Metamath
language.}\index{connective}\index{constant} (also called logical
equivalence)\index{logical equivalence
($\leftrightarrow$)}\index{biconditional ($\leftrightarrow$)}.  Unlike
most traditional developments, we have chosen not to have a separate
symbol such as ``Df.'' to mean ``is defined as.''  Instead, we will use
the biconditional connective for this purpose, as it lets us use
logic to manipulate definitions directly.  Here we state the properties
of the biconditional connective with a carefully crafted \texttt{\$a}
statement, which effectively uses the biconditional connective to define
itself.  The $\leftrightarrow$ symbol can be eliminated from a formula
using theorem \texttt{bii}, which is derived later.

\vskip 2ex
\noindent Define the biconditional connective.\label{df-bi}

\vskip 0.5ex
\setbox\startprefix=\hbox{\tt \ \ df-bi\ \$a\ }
\setbox\contprefix=\hbox{\tt \ \ \ \ \ \ \ \ \ \ \ }
\startm
\m{\vdash}\m{\lnot}\m{(}\m{(}\m{(}\m{\varphi}\m{\leftrightarrow}\m{\psi}\m{)}%
\m{\rightarrow}\m{\lnot}\m{(}\m{(}\m{\varphi}\m{\rightarrow}\m{\psi}\m{)}\m{%
\rightarrow}\m{\lnot}\m{(}\m{\psi}\m{\rightarrow}\m{\varphi}\m{)}\m{)}\m{)}\m{%
\rightarrow}\m{\lnot}\m{(}\m{\lnot}\m{(}\m{(}\m{\varphi}\m{\rightarrow}\m{%
\psi}\m{)}\m{\rightarrow}\m{\lnot}\m{(}\m{\psi}\m{\rightarrow}\m{\varphi}\m{)}%
\m{)}\m{\rightarrow}\m{(}\m{\varphi}\m{\leftrightarrow}\m{\psi}\m{)}\m{)}\m{)}
\endm
\begin{mmraw}%
|- -. ( ( ( ph <-> ps ) -> -. ( ( ph -> ps ) ->
-. ( ps -> ph ) ) ) -> -. ( -. ( ( ph -> ps ) -> -. (
ps -> ph ) ) -> ( ph <-> ps ) ) ) \$.
\end{mmraw}

\noindent This theorem relates the biconditional connective to primitive
connectives and can be used to eliminate the $\leftrightarrow$ symbol from any
wff.

\vskip 0.5ex
\setbox\startprefix=\hbox{\tt \ \ bii\ \$p\ }
\setbox\contprefix=\hbox{\tt \ \ \ \ \ \ \ \ \ }
\startm
\m{\vdash}\m{(}\m{(}\m{\varphi}\m{\leftrightarrow}\m{\psi}\m{)}\m{\leftrightarrow}
\m{\lnot}\m{(}\m{(}\m{\varphi}\m{\rightarrow}\m{\psi}\m{)}\m{\rightarrow}\m{\lnot}
\m{(}\m{\psi}\m{\rightarrow}\m{\varphi}\m{)}\m{)}\m{)}
\endm
\begin{mmraw}%
|- ( ( ph <-> ps ) <-> -. ( ( ph -> ps ) -> -. ( ps -> ph ) ) ) \$= ... \$.
\end{mmraw}

\noindent Define disjunction ({\sc or}).\index{disjunction ($\vee$)}%
\index{logical or (vee)@logical {\sc or} ($\vee$)}%
\index{df-or@\texttt{df-or}}\label{df-or}

\vskip 0.5ex
\setbox\startprefix=\hbox{\tt \ \ df-or\ \$a\ }
\setbox\contprefix=\hbox{\tt \ \ \ \ \ \ \ \ \ \ \ }
\startm
\m{\vdash}\m{(}\m{(}\m{\varphi}\m{\vee}\m{\psi}\m{)}\m{\leftrightarrow}\m{(}\m{
\lnot}\m{\varphi}\m{\rightarrow}\m{\psi}\m{)}\m{)}
\endm
\begin{mmraw}%
|- ( ( ph \TOR ps ) <-> ( -. ph -> ps ) ) \$.
\end{mmraw}

\noindent Define conjunction ({\sc and}).\index{conjunction ($\wedge$)}%
\index{logical {\sc and} ($\wedge$)}%
\index{df-an@\texttt{df-an}}\label{df-an}

\vskip 0.5ex
\setbox\startprefix=\hbox{\tt \ \ df-an\ \$a\ }
\setbox\contprefix=\hbox{\tt \ \ \ \ \ \ \ \ \ \ \ }
\startm
\m{\vdash}\m{(}\m{(}\m{\varphi}\m{\wedge}\m{\psi}\m{)}\m{\leftrightarrow}\m{\lnot}
\m{(}\m{\varphi}\m{\rightarrow}\m{\lnot}\m{\psi}\m{)}\m{)}
\endm
\begin{mmraw}%
|- ( ( ph \TAND ps ) <-> -. ( ph -> -. ps ) ) \$.
\end{mmraw}

\noindent Define disjunction ({\sc or}) of 3 wffs.%
\index{df-3or@\texttt{df-3or}}\label{df-3or}

\vskip 0.5ex
\setbox\startprefix=\hbox{\tt \ \ df-3or\ \$a\ }
\setbox\contprefix=\hbox{\tt \ \ \ \ \ \ \ \ \ \ \ \ }
\startm
\m{\vdash}\m{(}\m{(}\m{\varphi}\m{\vee}\m{\psi}\m{\vee}\m{\chi}\m{)}\m{
\leftrightarrow}\m{(}\m{(}\m{\varphi}\m{\vee}\m{\psi}\m{)}\m{\vee}\m{\chi}\m{)}
\m{)}
\endm
\begin{mmraw}%
|- ( ( ph \TOR ps \TOR ch ) <-> ( ( ph \TOR ps ) \TOR ch ) ) \$.
\end{mmraw}

\noindent Define conjunction ({\sc and}) of 3 wffs.%
\index{df-3an}\label{df-3an}

\vskip 0.5ex
\setbox\startprefix=\hbox{\tt \ \ df-3an\ \$a\ }
\setbox\contprefix=\hbox{\tt \ \ \ \ \ \ \ \ \ \ \ \ }
\startm
\m{\vdash}\m{(}\m{(}\m{\varphi}\m{\wedge}\m{\psi}\m{\wedge}\m{\chi}\m{)}\m{
\leftrightarrow}\m{(}\m{(}\m{\varphi}\m{\wedge}\m{\psi}\m{)}\m{\wedge}\m{\chi}
\m{)}\m{)}
\endm

\begin{mmraw}%
|- ( ( ph \TAND ps \TAND ch ) <-> ( ( ph \TAND ps ) \TAND ch ) ) \$.
\end{mmraw}

\subsection{Definitions for Predicate Calculus}\label{metadefpred}

The symbols $x$, $y$, and $z$ represent individual variables of predicate
calculus.  In this section, they are not necessarily distinct unless it is
explicitly
mentioned.

\vskip 2ex
\noindent Define existential quantification.
The expression $\exists x \varphi$ means
``there exists an $x$ where $\varphi$ is true.''\index{existential quantifier
($\exists$)}\label{df-ex}

\vskip 0.5ex
\setbox\startprefix=\hbox{\tt \ \ df-ex\ \$a\ }
\setbox\contprefix=\hbox{\tt \ \ \ \ \ \ \ \ \ \ \ }
\startm
\m{\vdash}\m{(}\m{\exists}\m{x}\m{\varphi}\m{\leftrightarrow}\m{\lnot}\m{\forall}
\m{x}\m{\lnot}\m{\varphi}\m{)}
\endm
\begin{mmraw}%
|- ( E. x ph <-> -. A. x -. ph ) \$.
\end{mmraw}

\noindent Define proper substitution.\index{proper
substitution}\index{substitution!proper}\label{df-sb}
In our notation, we use $[ y / x ] \varphi$ to mean ``the wff that
results when $y$ is properly substituted for $x$ in the wff
$\varphi$.''\footnote{
This can also be described
as substituting $x$ with $y$, $y$ properly replaces $x$, or
$x$ is properly replaced by $y$.}
% This is elsb4, though it currently says: ( [ x / y ] z e. y <-> z e. x )
For example,
$[ y / x ] z \in x$ is the same as $z \in y$.
One way to remember this notation is to notice that it looks like division
and recall that $( y / x ) \cdot x $ is $y$ (when $x \neq 0$).
The notation is different from the notation $\varphi ( x | y )$
that is sometimes used, because the latter notation is ambiguous for us:
for example, we don't know whether $\lnot \varphi ( x | y )$ is to be
interpreted as $\lnot ( \varphi ( x | y ) )$ or
$( \lnot \varphi ) ( x | y )$.\footnote{Because of the way
we initially defined wffs, this is the case
with any postfix connective\index{postfix connective} (one occurring after the
symbols being connected) or infix connective\index{infix connective} (one
occurring between the symbols being connected).  Metamath does not have a
built-in notion of operator binding strength that could eliminate the
ambiguity.  The initial parenthesis effectively provides a prefix
connective\index{prefix connective} to eliminate ambiguity.  Some conventions,
such as Polish notation\index{Polish notation} used in the 1930's and 1940's
by Polish logicians, use only prefix connectives and thus allow the total
elimination of parentheses, at the expense of readability.  In Metamath we
could actually redefine all notation to be Polish if we wanted to without
having to change any proofs!}  Other texts often use $\varphi(y)$ to indicate
our $[ y / x ] \varphi$, but this notation is even more ambiguous since there is
no explicit indication of what is being substituted.
Note that this
definition is valid even when
$x$ and $y$ are the same variable.  The first conjunct is a ``trick'' used to
achieve this property, making the definition look somewhat peculiar at
first.

\vskip 0.5ex
\setbox\startprefix=\hbox{\tt \ \ df-sb\ \$a\ }
\setbox\contprefix=\hbox{\tt \ \ \ \ \ \ \ \ \ \ \ }
\startm
\m{\vdash}\m{(}\m{[}\m{y}\m{/}\m{x}\m{]}\m{\varphi}\m{\leftrightarrow}\m{(}%
\m{(}\m{x}\m{=}\m{y}\m{\rightarrow}\m{\varphi}\m{)}\m{\wedge}\m{\exists}\m{x}%
\m{(}\m{x}\m{=}\m{y}\m{\wedge}\m{\varphi}\m{)}\m{)}\m{)}
\endm
\begin{mmraw}%
|- ( [ y / x ] ph <-> ( ( x = y -> ph ) \TAND E. x ( x = y \TAND ph ) ) ) \$.
\end{mmraw}


\noindent Define existential uniqueness\index{existential uniqueness
quantifier ($\exists "!$)} (``there exists exactly one'').  Note that $y$ is a
variable distinct from $x$ and not occurring in $\varphi$.\label{df-eu}

\vskip 0.5ex
\setbox\startprefix=\hbox{\tt \ \ df-eu\ \$a\ }
\setbox\contprefix=\hbox{\tt \ \ \ \ \ \ \ \ \ \ \ }
\startm
\m{\vdash}\m{(}\m{\exists}\m{{!}}\m{x}\m{\varphi}\m{\leftrightarrow}\m{\exists}
\m{y}\m{\forall}\m{x}\m{(}\m{\varphi}\m{\leftrightarrow}\m{x}\m{=}\m{y}\m{)}\m{)}
\endm

\begin{mmraw}%
|- ( E! x ph <-> E. y A. x ( ph <-> x = y ) ) \$.
\end{mmraw}

\subsection{Definitions for Set Theory}\label{setdefinitions}

The symbols $x$, $y$, $z$, and $w$ represent individual variables of
predicate calculus, which in set theory are understood to be sets.
However, using only the constructs shown so far would be very inconvenient.

To make set theory more practical, we introduce the notion of a ``class.''
A class\index{class} is either a set variable (such as $x$) or an
expression of the form $\{ x | \varphi\}$ (called an ``abstraction
class''\index{abstraction class}\index{class abstraction}).  Note that
sets (i.e.\ individual variables) always exist (this is a theorem of
logic, namely $\exists y \, y = x$ for any set $x$), whereas classes may
or may not exist (i.e.\ $\exists y \, y = A$ may or may not be true).
If a class does not exist it is called a ``proper class.''\index{proper
class}\index{class!proper} Definitions \texttt{df-clab},
\texttt{df-cleq}, and \texttt{df-clel} can be used to convert an
expression containing classes into one containing only set variables and
wff metavariables.

The symbols $A$, $B$, $C$, $D$, $ F$, $G$, and $R$ are metavariables that range
over classes.  A class metavariable $A$ may be eliminated from a wff by
replacing it with $\{ x|\varphi\}$ where neither $x$ nor $\varphi$ occur in
the wff.

The theory of classes can be shown to be an eliminable and conservative
extension of set theory. The \textbf{eliminability}
property shows that for every
formula in the extended language we can build a logically equivalent
formula in the basic language; so that even if the extended language
provides more ease to convey and formulate mathematical ideas for set
theory, its expressive power does not in fact strengthen the basic
language's expressive power.
The \textbf{conservation} property shows that for
every proof of a formula of the basic language in the extended system
we can build another proof of the same formula in the basic system;
so that, concerning theorems on sets only, the deductive powers of
the extended system and of the basic system are identical. Together,
these properties mean that the extended language can be treated as a
definitional extension that is \textbf{sound}.

A rigorous justification, which we will not give here, can be found in
Levy \cite[pp.~357-366]{Levy} supplementing his informal introduction to class
theory on pp.~7-17. Two other good treatments of class theory are provided
by Quine \cite[pp.~15-21]{Quine}\index{Quine, Willard Van Orman}
and also \cite[pp.~10-14]{Takeuti}\index{Takeuti, Gaisi}.
Quine's exposition (he calls them virtual classes)
is nicely written and very readable.

In the rest of this
section, individual variables are always assumed to be distinct from
each other unless otherwise indicated.  In addition, dummy variables on the
right-hand side of a definition do not occur in the class and wff
metavariables in the definition.

The definitions we present here are a partial but self-contained
collection selected from several hundred that appear in the current
\texttt{set.mm} database.  They are adequate for a basic development of
elementary set theory.

\vskip 2ex
\noindent Define the abstraction class.\index{abstraction class}\index{class
abstraction}\label{df-clab}  $x$ and $y$
need not be distinct.  Definition 2.1 of Quine, p.~16.  This definition may
seem puzzling since it is shorter than the expression being defined and does not
buy us anything in terms of brevity.  The reason we introduce this definition
is because it fits in neatly with the extension of the $\in$ connective
provided by \texttt{df-clel}.

\vskip 0.5ex
\setbox\startprefix=\hbox{\tt \ \ df-clab\ \$a\ }
\setbox\contprefix=\hbox{\tt \ \ \ \ \ \ \ \ \ \ \ \ \ }
\startm
\m{\vdash}\m{(}\m{x}\m{\in}\m{\{}\m{y}\m{|}\m{\varphi}\m{\}}\m{%
\leftrightarrow}\m{[}\m{x}\m{/}\m{y}\m{]}\m{\varphi}\m{)}
\endm
\begin{mmraw}%
|- ( x e. \{ y | ph \} <-> [ x / y ] ph ) \$.
\end{mmraw}

\noindent Define the equality connective between classes\index{class
equality}\label{df-cleq}.  See Quine or Chapter 4 of Takeuti and Zaring for its
justification and methods for eliminating it.  This is an example of a
somewhat ``dangerous'' definition, because it extends the use of the
existing equality symbol rather than introducing a new symbol, allowing
us to make statements in the original language that may not be true.
For example, it permits us to deduce $y = z \leftrightarrow \forall x (
x \in y \leftrightarrow x \in z )$ which is not a theorem of logic but
rather presupposes the Axiom of Extensionality,\index{Axiom of
Extensionality} which we include as a hypothesis so that we can know
when this axiom is assumed in a proof (with the \texttt{show
trace{\char`\_}back} command).  We could avoid the danger by introducing
another symbol, say $\eqcirc$, in place of $=$; this
would also have the advantage of making elimination of the definition
straightforward and would eliminate the need for Extensionality as a
hypothesis.  We would then also have the advantage of being able to
identify exactly where Extensionality truly comes into play.  One of our
theorems would be $x \eqcirc y \leftrightarrow x = y$
by invoking Extensionality.  However in keeping with standard practice
we retain the ``dangerous'' definition.

\vskip 0.5ex
\setbox\startprefix=\hbox{\tt \ \ df-cleq.1\ \$e\ }
\setbox\contprefix=\hbox{\tt \ \ \ \ \ \ \ \ \ \ \ \ \ \ \ }
\startm
\m{\vdash}\m{(}\m{\forall}\m{x}\m{(}\m{x}\m{\in}\m{y}\m{\leftrightarrow}\m{x}
\m{\in}\m{z}\m{)}\m{\rightarrow}\m{y}\m{=}\m{z}\m{)}
\endm
\setbox\startprefix=\hbox{\tt \ \ df-cleq\ \$a\ }
\setbox\contprefix=\hbox{\tt \ \ \ \ \ \ \ \ \ \ \ \ \ }
\startm
\m{\vdash}\m{(}\m{A}\m{=}\m{B}\m{\leftrightarrow}\m{\forall}\m{x}\m{(}\m{x}\m{
\in}\m{A}\m{\leftrightarrow}\m{x}\m{\in}\m{B}\m{)}\m{)}
\endm
% We need to reset the startprefix and contprefix.
\setbox\startprefix=\hbox{\tt \ \ df-cleq.1\ \$e\ }
\setbox\contprefix=\hbox{\tt \ \ \ \ \ \ \ \ \ \ \ \ \ \ \ }
\begin{mmraw}%
|- ( A. x ( x e. y <-> x e. z ) -> y = z ) \$.
\end{mmraw}
\setbox\startprefix=\hbox{\tt \ \ df-cleq\ \$a\ }
\setbox\contprefix=\hbox{\tt \ \ \ \ \ \ \ \ \ \ \ \ \ }
\begin{mmraw}%
|- ( A = B <-> A. x ( x e. A <-> x e. B ) ) \$.
\end{mmraw}

\noindent Define the membership connective between classes\index{class
membership}.  Theorem 6.3 of Quine, p.~41, which we adopt as a definition.
Note that it extends the use of the existing membership symbol, but unlike
{\tt df-cleq} it does not extend the set of valid wffs of logic when the class
metavariables are replaced with set variables.\label{dfclel}\label{df-clel}

\vskip 0.5ex
\setbox\startprefix=\hbox{\tt \ \ df-clel\ \$a\ }
\setbox\contprefix=\hbox{\tt \ \ \ \ \ \ \ \ \ \ \ \ \ }
\startm
\m{\vdash}\m{(}\m{A}\m{\in}\m{B}\m{\leftrightarrow}\m{\exists}\m{x}\m{(}\m{x}
\m{=}\m{A}\m{\wedge}\m{x}\m{\in}\m{B}\m{)}\m{)}
\endm
\begin{mmraw}%
|- ( A e. B <-> E. x ( x = A \TAND x e. B ) ) \$.?
\end{mmraw}

\noindent Define inequality.

\vskip 0.5ex
\setbox\startprefix=\hbox{\tt \ \ df-ne\ \$a\ }
\setbox\contprefix=\hbox{\tt \ \ \ \ \ \ \ \ \ \ \ }
\startm
\m{\vdash}\m{(}\m{A}\m{\ne}\m{B}\m{\leftrightarrow}\m{\lnot}\m{A}\m{=}\m{B}%
\m{)}
\endm
\begin{mmraw}%
|- ( A =/= B <-> -. A = B ) \$.
\end{mmraw}

\noindent Define restricted universal quantification.\index{universal
quantifier ($\forall$)!restricted}  Enderton, p.~22.

\vskip 0.5ex
\setbox\startprefix=\hbox{\tt \ \ df-ral\ \$a\ }
\setbox\contprefix=\hbox{\tt \ \ \ \ \ \ \ \ \ \ \ \ }
\startm
\m{\vdash}\m{(}\m{\forall}\m{x}\m{\in}\m{A}\m{\varphi}\m{\leftrightarrow}\m{%
\forall}\m{x}\m{(}\m{x}\m{\in}\m{A}\m{\rightarrow}\m{\varphi}\m{)}\m{)}
\endm
\begin{mmraw}%
|- ( A. x e. A ph <-> A. x ( x e. A -> ph ) ) \$.
\end{mmraw}

\noindent Define restricted existential quantification.\index{existential
quantifier ($\exists$)!restricted}  Enderton, p.~22.

\vskip 0.5ex
\setbox\startprefix=\hbox{\tt \ \ df-rex\ \$a\ }
\setbox\contprefix=\hbox{\tt \ \ \ \ \ \ \ \ \ \ \ \ }
\startm
\m{\vdash}\m{(}\m{\exists}\m{x}\m{\in}\m{A}\m{\varphi}\m{\leftrightarrow}\m{%
\exists}\m{x}\m{(}\m{x}\m{\in}\m{A}\m{\wedge}\m{\varphi}\m{)}\m{)}
\endm
\begin{mmraw}%
|- ( E. x e. A ph <-> E. x ( x e. A \TAND ph ) ) \$.
\end{mmraw}

\noindent Define the universal class\index{universal class ($V$)}.  Definition
5.20, p.~21, of Takeuti and Zaring.\label{df-v}

\vskip 0.5ex
\setbox\startprefix=\hbox{\tt \ \ df-v\ \$a\ }
\setbox\contprefix=\hbox{\tt \ \ \ \ \ \ \ \ \ \ }
\startm
\m{\vdash}\m{{\rm V}}\m{=}\m{\{}\m{x}\m{|}\m{x}\m{=}\m{x}\m{\}}
\endm
\begin{mmraw}%
|- {\char`\_}V = \{ x | x = x \} \$.
\end{mmraw}

\noindent Define the subclass\index{subclass}\index{subset} relationship
between two classes (called the subset relation if the classes are sets i.e.\
are not proper).  Definition 5.9 of Takeuti and Zaring, p.~17.\label{df-ss}

\vskip 0.5ex
\setbox\startprefix=\hbox{\tt \ \ df-ss\ \$a\ }
\setbox\contprefix=\hbox{\tt \ \ \ \ \ \ \ \ \ \ \ }
\startm
\m{\vdash}\m{(}\m{A}\m{\subseteq}\m{B}\m{\leftrightarrow}\m{\forall}\m{x}\m{(}
\m{x}\m{\in}\m{A}\m{\rightarrow}\m{x}\m{\in}\m{B}\m{)}\m{)}
\endm
\begin{mmraw}%
|- ( A C\_ B <-> A. x ( x e. A -> x e. B ) ) \$.
\end{mmraw}

\noindent Define the union\index{union} of two classes.  Definition 5.6 of Takeuti and Zaring,
p.~16.\label{df-un}

\vskip 0.5ex
\setbox\startprefix=\hbox{\tt \ \ df-un\ \$a\ }
\setbox\contprefix=\hbox{\tt \ \ \ \ \ \ \ \ \ \ \ }
\startm
\m{\vdash}\m{(}\m{A}\m{\cup}\m{B}\m{)}\m{=}\m{\{}\m{x}\m{|}\m{(}\m{x}\m{\in}
\m{A}\m{\vee}\m{x}\m{\in}\m{B}\m{)}\m{\}}
\endm
\begin{mmraw}%
( A u. B ) = \{ x | ( x e. A \TOR x e. B ) \} \$.
\end{mmraw}

\noindent Define the intersection\index{intersection} of two classes.  Definition 5.6 of
Takeuti and Zaring, p.~16.\label{df-in}

\vskip 0.5ex
\setbox\startprefix=\hbox{\tt \ \ df-in\ \$a\ }
\setbox\contprefix=\hbox{\tt \ \ \ \ \ \ \ \ \ \ \ }
\startm
\m{\vdash}\m{(}\m{A}\m{\cap}\m{B}\m{)}\m{=}\m{\{}\m{x}\m{|}\m{(}\m{x}\m{\in}
\m{A}\m{\wedge}\m{x}\m{\in}\m{B}\m{)}\m{\}}
\endm
% Caret ^ requires special treatment
\begin{mmraw}%
|- ( A i\^{}i B ) = \{ x | ( x e. A \TAND x e. B ) \} \$.
\end{mmraw}

\noindent Define class difference\index{class difference}\index{set difference}.
Definition 5.12 of Takeuti and Zaring, p.~20.  Several notations are used in
the literature; we chose the $\setminus$ convention instead of a minus sign to
reserve the latter for later use in, e.g., arithmetic.\label{df-dif}

\vskip 0.5ex
\setbox\startprefix=\hbox{\tt \ \ df-dif\ \$a\ }
\setbox\contprefix=\hbox{\tt \ \ \ \ \ \ \ \ \ \ \ \ }
\startm
\m{\vdash}\m{(}\m{A}\m{\setminus}\m{B}\m{)}\m{=}\m{\{}\m{x}\m{|}\m{(}\m{x}\m{
\in}\m{A}\m{\wedge}\m{\lnot}\m{x}\m{\in}\m{B}\m{)}\m{\}}
\endm
\begin{mmraw}%
( A \SLASH B ) = \{ x | ( x e. A \TAND -. x e. B ) \} \$.
\end{mmraw}

\noindent Define the empty or null set\index{empty set}\index{null set}.
Compare  Definition 5.14 of Takeuti and Zaring, p.~20.\label{df-nul}

\vskip 0.5ex
\setbox\startprefix=\hbox{\tt \ \ df-nul\ \$a\ }
\setbox\contprefix=\hbox{\tt \ \ \ \ \ \ \ \ \ \ }
\startm
\m{\vdash}\m{\varnothing}\m{=}\m{(}\m{{\rm V}}\m{\setminus}\m{{\rm V}}\m{)}
\endm
\begin{mmraw}%
|- (/) = ( {\char`\_}V \SLASH {\char`\_}V ) \$.
\end{mmraw}

\noindent Define power class\index{power set}\index{power class}.  Definition 5.10 of
Takeuti and Zaring, p.~17, but we also let it apply to proper classes.  (Note
that \verb$~P$ is the symbol for calligraphic P, the tilde
suggesting ``curly;'' see Appendix~\ref{ASCII}.)\label{df-pw}

\vskip 0.5ex
\setbox\startprefix=\hbox{\tt \ \ df-pw\ \$a\ }
\setbox\contprefix=\hbox{\tt \ \ \ \ \ \ \ \ \ \ \ }
\startm
\m{\vdash}\m{{\cal P}}\m{A}\m{=}\m{\{}\m{x}\m{|}\m{x}\m{\subseteq}\m{A}\m{\}}
\endm
% Special incantation required to put ~ into the text
\begin{mmraw}%
|- \char`\~P~A = \{ x | x C\_ A \} \$.
\end{mmraw}

\noindent Define the singleton of a class\index{singleton}.  Definition 7.1 of
Quine, p.~48.  It is well-defined for proper classes, although
it is not very meaningful in this case, where it evaluates to the empty
set.

\vskip 0.5ex
\setbox\startprefix=\hbox{\tt \ \ df-sn\ \$a\ }
\setbox\contprefix=\hbox{\tt \ \ \ \ \ \ \ \ \ \ \ }
\startm
\m{\vdash}\m{\{}\m{A}\m{\}}\m{=}\m{\{}\m{x}\m{|}\m{x}\m{=}\m{A}\m{\}}
\endm
\begin{mmraw}%
|- \{ A \} = \{ x | x = A \} \$.
\end{mmraw}%

\noindent Define an unordered pair of classes\index{unordered pair}\index{pair}.  Definition
7.1 of Quine, p.~48.

\vskip 0.5ex
\setbox\startprefix=\hbox{\tt \ \ df-pr\ \$a\ }
\setbox\contprefix=\hbox{\tt \ \ \ \ \ \ \ \ \ \ \ }
\startm
\m{\vdash}\m{\{}\m{A}\m{,}\m{B}\m{\}}\m{=}\m{(}\m{\{}\m{A}\m{\}}\m{\cup}\m{\{}
\m{B}\m{\}}\m{)}
\endm
\begin{mmraw}%
|- \{ A , B \} = ( \{ A \} u. \{ B \} ) \$.
\end{mmraw}

\noindent Define an unordered triple of classes\index{unordered triple}.  Definition of
Enderton, p.~19.

\vskip 0.5ex
\setbox\startprefix=\hbox{\tt \ \ df-tp\ \$a\ }
\setbox\contprefix=\hbox{\tt \ \ \ \ \ \ \ \ \ \ \ }
\startm
\m{\vdash}\m{\{}\m{A}\m{,}\m{B}\m{,}\m{C}\m{\}}\m{=}\m{(}\m{\{}\m{A}\m{,}\m{B}
\m{\}}\m{\cup}\m{\{}\m{C}\m{\}}\m{)}
\endm
\begin{mmraw}%
|- \{ A , B , C \} = ( \{ A , B \} u. \{ C \} ) \$.
\end{mmraw}%

\noindent Kuratowski's\index{Kuratowski, Kazimierz} ordered pair\index{ordered
pair} definition.  Definition 9.1 of Quine, p.~58. For proper classes it is
not meaningful but is well-defined for convenience.  (Note that \verb$<.$
stands for $\langle$ whereas \verb$<$ stands for $<$, and similarly for
\verb$>.$\,.)\label{df-op}

\vskip 0.5ex
\setbox\startprefix=\hbox{\tt \ \ df-op\ \$a\ }
\setbox\contprefix=\hbox{\tt \ \ \ \ \ \ \ \ \ \ \ }
\startm
\m{\vdash}\m{\langle}\m{A}\m{,}\m{B}\m{\rangle}\m{=}\m{\{}\m{\{}\m{A}\m{\}}
\m{,}\m{\{}\m{A}\m{,}\m{B}\m{\}}\m{\}}
\endm
\begin{mmraw}%
|- <. A , B >. = \{ \{ A \} , \{ A , B \} \} \$.
\end{mmraw}

\noindent Define the union of a class\index{union}.  Definition 5.5, p.~16,
of Takeuti and Zaring.

\vskip 0.5ex
\setbox\startprefix=\hbox{\tt \ \ df-uni\ \$a\ }
\setbox\contprefix=\hbox{\tt \ \ \ \ \ \ \ \ \ \ \ \ }
\startm
\m{\vdash}\m{\bigcup}\m{A}\m{=}\m{\{}\m{x}\m{|}\m{\exists}\m{y}\m{(}\m{x}\m{
\in}\m{y}\m{\wedge}\m{y}\m{\in}\m{A}\m{)}\m{\}}
\endm
\begin{mmraw}%
|- U. A = \{ x | E. y ( x e. y \TAND y e. A ) \} \$.
\end{mmraw}

\noindent Define the intersection\index{intersection} of a class.  Definition 7.35,
p.~44, of Takeuti and Zaring.

\vskip 0.5ex
\setbox\startprefix=\hbox{\tt \ \ df-int\ \$a\ }
\setbox\contprefix=\hbox{\tt \ \ \ \ \ \ \ \ \ \ \ \ }
\startm
\m{\vdash}\m{\bigcap}\m{A}\m{=}\m{\{}\m{x}\m{|}\m{\forall}\m{y}\m{(}\m{y}\m{
\in}\m{A}\m{\rightarrow}\m{x}\m{\in}\m{y}\m{)}\m{\}}
\endm
\begin{mmraw}%
|- |\^{}| A = \{ x | A. y ( y e. A -> x e. y ) \} \$.
\end{mmraw}

\noindent Define a transitive class\index{transitive class}\index{transitive
set}.  This should not be confused with a transitive relation, which is a different
concept.  Definition from p.~71 of Enderton, extended to classes.

\vskip 0.5ex
\setbox\startprefix=\hbox{\tt \ \ df-tr\ \$a\ }
\setbox\contprefix=\hbox{\tt \ \ \ \ \ \ \ \ \ \ \ }
\startm
\m{\vdash}\m{(}\m{\mbox{\rm Tr}}\m{A}\m{\leftrightarrow}\m{\bigcup}\m{A}\m{
\subseteq}\m{A}\m{)}
\endm
\begin{mmraw}%
|- ( Tr A <-> U. A C\_ A ) \$.
\end{mmraw}

\noindent Define a notation for a general binary relation\index{binary
relation}.  Definition 6.18, p.~29, of Takeuti and Zaring, generalized to
arbitrary classes.  This definition is well-defined, although not very
meaningful, when classes $A$ and/or $B$ are proper.\label{dfbr}  The lack of
parentheses (or any other connective) creates no ambiguity since we are defining
an atomic wff.

\vskip 0.5ex
\setbox\startprefix=\hbox{\tt \ \ df-br\ \$a\ }
\setbox\contprefix=\hbox{\tt \ \ \ \ \ \ \ \ \ \ \ }
\startm
\m{\vdash}\m{(}\m{A}\m{\,R}\m{\,B}\m{\leftrightarrow}\m{\langle}\m{A}\m{,}\m{B}
\m{\rangle}\m{\in}\m{R}\m{)}
\endm
\begin{mmraw}%
|- ( A R B <-> <. A , B >. e. R ) \$.
\end{mmraw}

\noindent Define an abstraction class of ordered pairs\index{abstraction
class!of ordered
pairs}.  A special case of Definition 4.16, p.~14, of Takeuti and Zaring.
Note that $ z $ must be distinct from $ x $ and $ y $,
and $ z $ must not occur in $\varphi$, but $ x $ and $ y $ may be
identical and may appear in $\varphi$.

\vskip 0.5ex
\setbox\startprefix=\hbox{\tt \ \ df-opab\ \$a\ }
\setbox\contprefix=\hbox{\tt \ \ \ \ \ \ \ \ \ \ \ \ \ }
\startm
\m{\vdash}\m{\{}\m{\langle}\m{x}\m{,}\m{y}\m{\rangle}\m{|}\m{\varphi}\m{\}}\m{=}
\m{\{}\m{z}\m{|}\m{\exists}\m{x}\m{\exists}\m{y}\m{(}\m{z}\m{=}\m{\langle}\m{x}
\m{,}\m{y}\m{\rangle}\m{\wedge}\m{\varphi}\m{)}\m{\}}
\endm

\begin{mmraw}%
|- \{ <. x , y >. | ph \} = \{ z | E. x E. y ( z =
<. x , y >. /\ ph ) \} \$.
\end{mmraw}

\noindent Define the epsilon relation\index{epsilon relation}.  Similar to Definition
6.22, p.~30, of Takeuti and Zaring.

\vskip 0.5ex
\setbox\startprefix=\hbox{\tt \ \ df-eprel\ \$a\ }
\setbox\contprefix=\hbox{\tt \ \ \ \ \ \ \ \ \ \ \ \ \ \ }
\startm
\m{\vdash}\m{{\rm E}}\m{=}\m{\{}\m{\langle}\m{x}\m{,}\m{y}\m{\rangle}\m{|}\m{x}\m{
\in}\m{y}\m{\}}
\endm
\begin{mmraw}%
|- \_E = \{ <. x , y >. | x e. y \} \$.
\end{mmraw}

\noindent Define a founded relation\index{founded relation}.  $R$ is a founded
relation on $A$ iff\index{iff} (if and only if) each nonempty subset of $A$
has an ``$R$-minimal element.''  Similar to Definition 6.21, p.~30, of
Takeuti and Zaring.

\vskip 0.5ex
\setbox\startprefix=\hbox{\tt \ \ df-fr\ \$a\ }
\setbox\contprefix=\hbox{\tt \ \ \ \ \ \ \ \ \ \ \ }
\startm
\m{\vdash}\m{(}\m{R}\m{\,\mbox{\rm Fr}}\m{\,A}\m{\leftrightarrow}\m{\forall}\m{x}
\m{(}\m{(}\m{x}\m{\subseteq}\m{A}\m{\wedge}\m{\lnot}\m{x}\m{=}\m{\varnothing}
\m{)}\m{\rightarrow}\m{\exists}\m{y}\m{(}\m{y}\m{\in}\m{x}\m{\wedge}\m{(}\m{x}
\m{\cap}\m{\{}\m{z}\m{|}\m{z}\m{\,R}\m{\,y}\m{\}}\m{)}\m{=}\m{\varnothing}\m{)}
\m{)}\m{)}
\endm
\begin{mmraw}%
|- ( R Fr A <-> A. x ( ( x C\_ A \TAND -. x = (/) ) ->
E. y ( y e. x \TAND ( x i\^{}i \{ z | z R y \} ) = (/) ) ) ) \$.
\end{mmraw}

\noindent Define a well-ordering\index{well-ordering}.  $R$ is a well-ordering of $A$ iff
it is founded on $A$ and the elements of $A$ are pairwise $R$-comparable.
Similar to Definition 6.24(2), p.~30, of Takeuti and Zaring.

\vskip 0.5ex
\setbox\startprefix=\hbox{\tt \ \ df-we\ \$a\ }
\setbox\contprefix=\hbox{\tt \ \ \ \ \ \ \ \ \ \ \ }
\startm
\m{\vdash}\m{(}\m{R}\m{\,\mbox{\rm We}}\m{\,A}\m{\leftrightarrow}\m{(}\m{R}\m{\,
\mbox{\rm Fr}}\m{\,A}\m{\wedge}\m{\forall}\m{x}\m{\forall}\m{y}\m{(}\m{(}\m{x}\m{
\in}\m{A}\m{\wedge}\m{y}\m{\in}\m{A}\m{)}\m{\rightarrow}\m{(}\m{x}\m{\,R}\m{\,y}
\m{\vee}\m{x}\m{=}\m{y}\m{\vee}\m{y}\m{\,R}\m{\,x}\m{)}\m{)}\m{)}\m{)}
\endm
\begin{mmraw}%
( R We A <-> ( R Fr A \TAND A. x A. y ( ( x e.
A \TAND y e. A ) -> ( x R y \TOR x = y \TOR y R x ) ) ) ) \$.
\end{mmraw}

\noindent Define the ordinal predicate\index{ordinal predicate}, which is true for a class
that is transitive and is well-ordered by the epsilon relation.  Similar to
definition on p.~468, Bell and Machover.

\vskip 0.5ex
\setbox\startprefix=\hbox{\tt \ \ df-ord\ \$a\ }
\setbox\contprefix=\hbox{\tt \ \ \ \ \ \ \ \ \ \ \ \ }
\startm
\m{\vdash}\m{(}\m{\mbox{\rm Ord}}\m{\,A}\m{\leftrightarrow}\m{(}
\m{\mbox{\rm Tr}}\m{\,A}\m{\wedge}\m{E}\m{\,\mbox{\rm We}}\m{\,A}\m{)}\m{)}
\endm
\begin{mmraw}%
|- ( Ord A <-> ( Tr A \TAND E We A ) ) \$.
\end{mmraw}

\noindent Define the class of all ordinal numbers\index{ordinal number}.  An ordinal number is
a set that satisfies the ordinal predicate.  Definition 7.11 of Takeuti and
Zaring, p.~38.

\vskip 0.5ex
\setbox\startprefix=\hbox{\tt \ \ df-on\ \$a\ }
\setbox\contprefix=\hbox{\tt \ \ \ \ \ \ \ \ \ \ \ }
\startm
\m{\vdash}\m{\,\mbox{\rm On}}\m{=}\m{\{}\m{x}\m{|}\m{\mbox{\rm Ord}}\m{\,x}
\m{\}}
\endm
\begin{mmraw}%
|- On = \{ x | Ord x \} \$.
\end{mmraw}

\noindent Define the limit ordinal predicate\index{limit ordinal}, which is true for a
non-empty ordinal that is not a successor (i.e.\ that is the union of itself).
Compare Bell and Machover, p.~471 and Exercise (1), p.~42 of Takeuti and
Zaring.

\vskip 0.5ex
\setbox\startprefix=\hbox{\tt \ \ df-lim\ \$a\ }
\setbox\contprefix=\hbox{\tt \ \ \ \ \ \ \ \ \ \ \ \ }
\startm
\m{\vdash}\m{(}\m{\mbox{\rm Lim}}\m{\,A}\m{\leftrightarrow}\m{(}\m{\mbox{
\rm Ord}}\m{\,A}\m{\wedge}\m{\lnot}\m{A}\m{=}\m{\varnothing}\m{\wedge}\m{A}
\m{=}\m{\bigcup}\m{A}\m{)}\m{)}
\endm
\begin{mmraw}%
|- ( Lim A <-> ( Ord A \TAND -. A = (/) \TAND A = U. A ) ) \$.
\end{mmraw}

\noindent Define the successor\index{successor} of a class.  Definition 7.22 of Takeuti
and Zaring, p.~41.  Our definition is a generalization to classes, although it
is meaningless when classes are proper.

\vskip 0.5ex
\setbox\startprefix=\hbox{\tt \ \ df-suc\ \$a\ }
\setbox\contprefix=\hbox{\tt \ \ \ \ \ \ \ \ \ \ \ \ }
\startm
\m{\vdash}\m{\,\mbox{\rm suc}}\m{\,A}\m{=}\m{(}\m{A}\m{\cup}\m{\{}\m{A}\m{\}}
\m{)}
\endm
\begin{mmraw}%
|- suc A = ( A u. \{ A \} ) \$.
\end{mmraw}

\noindent Define the class of natural numbers\index{natural number}\index{omega
($\omega$)}.  Compare Bell and Machover, p.~471.\label{dfom}

\vskip 0.5ex
\setbox\startprefix=\hbox{\tt \ \ df-om\ \$a\ }
\setbox\contprefix=\hbox{\tt \ \ \ \ \ \ \ \ \ \ \ }
\startm
\m{\vdash}\m{\omega}\m{=}\m{\{}\m{x}\m{|}\m{(}\m{\mbox{\rm Ord}}\m{\,x}\m{
\wedge}\m{\forall}\m{y}\m{(}\m{\mbox{\rm Lim}}\m{\,y}\m{\rightarrow}\m{x}\m{
\in}\m{y}\m{)}\m{)}\m{\}}
\endm
\begin{mmraw}%
|- om = \{ x | ( Ord x \TAND A. y ( Lim y -> x e. y ) ) \} \$.
\end{mmraw}

\noindent Define the Cartesian product (also called the
cross product)\index{Cartesian product}\index{cross product}
of two classes.  Definition 9.11 of Quine, p.~64.

\vskip 0.5ex
\setbox\startprefix=\hbox{\tt \ \ df-xp\ \$a\ }
\setbox\contprefix=\hbox{\tt \ \ \ \ \ \ \ \ \ \ \ }
\startm
\m{\vdash}\m{(}\m{A}\m{\times}\m{B}\m{)}\m{=}\m{\{}\m{\langle}\m{x}\m{,}\m{y}
\m{\rangle}\m{|}\m{(}\m{x}\m{\in}\m{A}\m{\wedge}\m{y}\m{\in}\m{B}\m{)}\m{\}}
\endm
\begin{mmraw}%
|- ( A X. B ) = \{ <. x , y >. | ( x e. A \TAND y e. B) \} \$.
\end{mmraw}

\noindent Define a relation\index{relation}.  Definition 6.4(1) of Takeuti and
Zaring, p.~23.

\vskip 0.5ex
\setbox\startprefix=\hbox{\tt \ \ df-rel\ \$a\ }
\setbox\contprefix=\hbox{\tt \ \ \ \ \ \ \ \ \ \ \ \ }
\startm
\m{\vdash}\m{(}\m{\mbox{\rm Rel}}\m{\,A}\m{\leftrightarrow}\m{A}\m{\subseteq}
\m{(}\m{{\rm V}}\m{\times}\m{{\rm V}}\m{)}\m{)}
\endm
\begin{mmraw}%
|- ( Rel A <-> A C\_ ( {\char`\_}V X. {\char`\_}V ) ) \$.
\end{mmraw}

\noindent Define the domain\index{domain} of a class.  Definition 6.5(1) of
Takeuti and Zaring, p.~24.

\vskip 0.5ex
\setbox\startprefix=\hbox{\tt \ \ df-dm\ \$a\ }
\setbox\contprefix=\hbox{\tt \ \ \ \ \ \ \ \ \ \ \ }
\startm
\m{\vdash}\m{\,\mbox{\rm dom}}\m{A}\m{=}\m{\{}\m{x}\m{|}\m{\exists}\m{y}\m{
\langle}\m{x}\m{,}\m{y}\m{\rangle}\m{\in}\m{A}\m{\}}
\endm
\begin{mmraw}%
|- dom A = \{ x | E. y <. x , y >. e. A \} \$.
\end{mmraw}

\noindent Define the range\index{range} of a class.  Definition 6.5(2) of
Takeuti and Zaring, p.~24.

\vskip 0.5ex
\setbox\startprefix=\hbox{\tt \ \ df-rn\ \$a\ }
\setbox\contprefix=\hbox{\tt \ \ \ \ \ \ \ \ \ \ \ }
\startm
\m{\vdash}\m{\,\mbox{\rm ran}}\m{A}\m{=}\m{\{}\m{y}\m{|}\m{\exists}\m{x}\m{
\langle}\m{x}\m{,}\m{y}\m{\rangle}\m{\in}\m{A}\m{\}}
\endm
\begin{mmraw}%
|- ran A = \{ y | E. x <. x , y >. e. A \} \$.
\end{mmraw}

\noindent Define the restriction\index{restriction} of a class.  Definition
6.6(1) of Takeuti and Zaring, p.~24.

\vskip 0.5ex
\setbox\startprefix=\hbox{\tt \ \ df-res\ \$a\ }
\setbox\contprefix=\hbox{\tt \ \ \ \ \ \ \ \ \ \ \ \ }
\startm
\m{\vdash}\m{(}\m{A}\m{\restriction}\m{B}\m{)}\m{=}\m{(}\m{A}\m{\cap}\m{(}\m{B}
\m{\times}\m{{\rm V}}\m{)}\m{)}
\endm
\begin{mmraw}%
|- ( A |` B ) = ( A i\^{}i ( B X. {\char`\_}V ) ) \$.
\end{mmraw}

\noindent Define the image\index{image} of a class.  Definition 6.6(2) of
Takeuti and Zaring, p.~24.

\vskip 0.5ex
\setbox\startprefix=\hbox{\tt \ \ df-ima\ \$a\ }
\setbox\contprefix=\hbox{\tt \ \ \ \ \ \ \ \ \ \ \ \ }
\startm
\m{\vdash}\m{(}\m{A}\m{``}\m{B}\m{)}\m{=}\m{\,\mbox{\rm ran}}\m{\,(}\m{A}\m{
\restriction}\m{B}\m{)}
\endm
\begin{mmraw}%
|- ( A " B ) = ran ( A |` B ) \$.
\end{mmraw}

\noindent Define the composition\index{composition} of two classes.  Definition 6.6(3) of
Takeuti and Zaring, p.~24.

\vskip 0.5ex
\setbox\startprefix=\hbox{\tt \ \ df-co\ \$a\ }
\setbox\contprefix=\hbox{\tt \ \ \ \ \ \ \ \ \ \ \ \ }
\startm
\m{\vdash}\m{(}\m{A}\m{\circ}\m{B}\m{)}\m{=}\m{\{}\m{\langle}\m{x}\m{,}\m{y}\m{
\rangle}\m{|}\m{\exists}\m{z}\m{(}\m{\langle}\m{x}\m{,}\m{z}\m{\rangle}\m{\in}
\m{B}\m{\wedge}\m{\langle}\m{z}\m{,}\m{y}\m{\rangle}\m{\in}\m{A}\m{)}\m{\}}
\endm
\begin{mmraw}%
|- ( A o. B ) = \{ <. x , y >. | E. z ( <. x , z
>. e. B \TAND <. z , y >. e. A ) \} \$.
\end{mmraw}

\noindent Define a function\index{function}.  Definition 6.4(4) of Takeuti and
Zaring, p.~24.

\vskip 0.5ex
\setbox\startprefix=\hbox{\tt \ \ df-fun\ \$a\ }
\setbox\contprefix=\hbox{\tt \ \ \ \ \ \ \ \ \ \ \ \ }
\startm
\m{\vdash}\m{(}\m{\mbox{\rm Fun}}\m{\,A}\m{\leftrightarrow}\m{(}
\m{\mbox{\rm Rel}}\m{\,A}\m{\wedge}
\m{\forall}\m{x}\m{\exists}\m{z}\m{\forall}\m{y}\m{(}
\m{\langle}\m{x}\m{,}\m{y}\m{\rangle}\m{\in}\m{A}\m{\rightarrow}\m{y}\m{=}\m{z}
\m{)}\m{)}\m{)}
\endm
\begin{mmraw}%
|- ( Fun A <-> ( Rel A /\ A. x E. z A. y ( <. x
   , y >. e. A -> y = z ) ) ) \$.
\end{mmraw}

\noindent Define a function with domain.  Definition 6.15(1) of Takeuti and
Zaring, p.~27.

\vskip 0.5ex
\setbox\startprefix=\hbox{\tt \ \ df-fn\ \$a\ }
\setbox\contprefix=\hbox{\tt \ \ \ \ \ \ \ \ \ \ \ }
\startm
\m{\vdash}\m{(}\m{A}\m{\,\mbox{\rm Fn}}\m{\,B}\m{\leftrightarrow}\m{(}
\m{\mbox{\rm Fun}}\m{\,A}\m{\wedge}\m{\mbox{\rm dom}}\m{\,A}\m{=}\m{B}\m{)}
\m{)}
\endm
\begin{mmraw}%
|- ( A Fn B <-> ( Fun A \TAND dom A = B ) ) \$.
\end{mmraw}

\noindent Define a function with domain and co-domain.  Definition 6.15(3)
of Takeuti and Zaring, p.~27.

\vskip 0.5ex
\setbox\startprefix=\hbox{\tt \ \ df-f\ \$a\ }
\setbox\contprefix=\hbox{\tt \ \ \ \ \ \ \ \ \ \ }
\startm
\m{\vdash}\m{(}\m{F}\m{:}\m{A}\m{\longrightarrow}\m{B}\m{
\leftrightarrow}\m{(}\m{F}\m{\,\mbox{\rm Fn}}\m{\,A}\m{\wedge}\m{
\mbox{\rm ran}}\m{\,F}\m{\subseteq}\m{B}\m{)}\m{)}
\endm
\begin{mmraw}%
|- ( F : A --> B <-> ( F Fn A \TAND ran F C\_ B ) ) \$.
\end{mmraw}

\noindent Define a one-to-one function\index{one-to-one function}.  Compare
Definition 6.15(5) of Takeuti and Zaring, p.~27.

\vskip 0.5ex
\setbox\startprefix=\hbox{\tt \ \ df-f1\ \$a\ }
\setbox\contprefix=\hbox{\tt \ \ \ \ \ \ \ \ \ \ \ }
\startm
\m{\vdash}\m{(}\m{F}\m{:}\m{A}\m{
\raisebox{.5ex}{${\textstyle{\:}_{\mbox{\footnotesize\rm
1\tt -\rm 1}}}\atop{\textstyle{
\longrightarrow}\atop{\textstyle{}^{\mbox{\footnotesize\rm {\ }}}}}$}
}\m{B}
\m{\leftrightarrow}\m{(}\m{F}\m{:}\m{A}\m{\longrightarrow}\m{B}
\m{\wedge}\m{\forall}\m{y}\m{\exists}\m{z}\m{\forall}\m{x}\m{(}\m{\langle}\m{x}
\m{,}\m{y}\m{\rangle}\m{\in}\m{F}\m{\rightarrow}\m{x}\m{=}\m{z}\m{)}\m{)}\m{)}
\endm
\begin{mmraw}%
|- ( F : A -1-1-> B <-> ( F : A --> B \TAND
   A. y E. z A. x ( <. x , y >. e. F -> x = z ) ) ) \$.
\end{mmraw}

\noindent Define an onto function\index{onto function}.  Definition 6.15(4) of Takeuti and
Zaring, p.~27.

\vskip 0.5ex
\setbox\startprefix=\hbox{\tt \ \ df-fo\ \$a\ }
\setbox\contprefix=\hbox{\tt \ \ \ \ \ \ \ \ \ \ \ }
\startm
\m{\vdash}\m{(}\m{F}\m{:}\m{A}\m{
\raisebox{.5ex}{${\textstyle{\:}_{\mbox{\footnotesize\rm
{\ }}}}\atop{\textstyle{
\longrightarrow}\atop{\textstyle{}^{\mbox{\footnotesize\rm onto}}}}$}
}\m{B}
\m{\leftrightarrow}\m{(}\m{F}\m{\,\mbox{\rm Fn}}\m{\,A}\m{\wedge}
\m{\mbox{\rm ran}}\m{\,F}\m{=}\m{B}\m{)}\m{)}
\endm
\begin{mmraw}%
|- ( F : A -onto-> B <-> ( F Fn A /\ ran F = B ) ) \$.
\end{mmraw}

\noindent Define a one-to-one, onto function.  Compare Definition 6.15(6) of
Takeuti and Zaring, p.~27.

\vskip 0.5ex
\setbox\startprefix=\hbox{\tt \ \ df-f1o\ \$a\ }
\setbox\contprefix=\hbox{\tt \ \ \ \ \ \ \ \ \ \ \ \ }
\startm
\m{\vdash}\m{(}\m{F}\m{:}\m{A}
\m{
\raisebox{.5ex}{${\textstyle{\:}_{\mbox{\footnotesize\rm
1\tt -\rm 1}}}\atop{\textstyle{
\longrightarrow}\atop{\textstyle{}^{\mbox{\footnotesize\rm onto}}}}$}
}
\m{B}
\m{\leftrightarrow}\m{(}\m{F}\m{:}\m{A}
\m{
\raisebox{.5ex}{${\textstyle{\:}_{\mbox{\footnotesize\rm
1\tt -\rm 1}}}\atop{\textstyle{
\longrightarrow}\atop{\textstyle{}^{\mbox{\footnotesize\rm {\ }}}}}$}
}
\m{B}\m{\wedge}\m{F}\m{:}\m{A}
\m{
\raisebox{.5ex}{${\textstyle{\:}_{\mbox{\footnotesize\rm
{\ }}}}\atop{\textstyle{
\longrightarrow}\atop{\textstyle{}^{\mbox{\footnotesize\rm onto}}}}$}
}
\m{B}\m{)}\m{)}
\endm
\begin{mmraw}%
|- ( F : A -1-1-onto-> B <-> ( F : A -1-1-> B? \TAND F : A -onto-> B ) ) \$.?
\end{mmraw}

\noindent Define the value of a function\index{function value}.  This
definition applies to any class and evaluates to the empty set when it is not
meaningful. Note that $ F`A$ means the same thing as the more familiar $ F(A)$
notation for a function's value at $A$.  The $ F`A$ notation is common in
formal set theory.\label{df-fv}

\vskip 0.5ex
\setbox\startprefix=\hbox{\tt \ \ df-fv\ \$a\ }
\setbox\contprefix=\hbox{\tt \ \ \ \ \ \ \ \ \ \ \ }
\startm
\m{\vdash}\m{(}\m{F}\m{`}\m{A}\m{)}\m{=}\m{\bigcup}\m{\{}\m{x}\m{|}\m{(}\m{F}%
\m{``}\m{\{}\m{A}\m{\}}\m{)}\m{=}\m{\{}\m{x}\m{\}}\m{\}}
\endm
\begin{mmraw}%
|- ( F ` A ) = U. \{ x | ( F " \{ A \} ) = \{ x \} \} \$.
\end{mmraw}

\noindent Define the result of an operation.\index{operation}  Here, $F$ is
     an operation on two
     values (such as $+$ for real numbers).   This is defined for proper
     classes $A$ and $B$ even though not meaningful in that case.  However,
     the definition can be meaningful when $F$ is a proper class.\label{dfopr}

\vskip 0.5ex
\setbox\startprefix=\hbox{\tt \ \ df-opr\ \$a\ }
\setbox\contprefix=\hbox{\tt \ \ \ \ \ \ \ \ \ \ \ \ }
\startm
\m{\vdash}\m{(}\m{A}\m{\,F}\m{\,B}\m{)}\m{=}\m{(}\m{F}\m{`}\m{\langle}\m{A}%
\m{,}\m{B}\m{\rangle}\m{)}
\endm
\begin{mmraw}%
|- ( A F B ) = ( F ` <. A , B >. ) \$.
\end{mmraw}

\section{Tricks of the Trade}\label{tricks}

In the \texttt{set.mm}\index{set theory database (\texttt{set.mm})} database our goal
was usually to conform to modern notation.  However in some cases the
relationship to standard textbook language may be obscured by several
unconventional devices we used to simplify the development and to take
advantage of the Metamath language.  In this section we will describe some
common conventions used in \texttt{set.mm}.

\begin{itemize}
\item
The turnstile\index{turnstile ({$\,\vdash$})} symbol, $\vdash$, meaning ``it
is provable that,'' is the first token of all assertions and hypotheses that
aren't syntax constructions.  This is a standard convention in logic.  (We
mentioned this earlier, but this symbol is bothersome to some people without a
logic background.  It has no deeper meaning but just provides us with a way to
distinguish syntax constructions from ordinary mathematical statements.)

\item
A hypothesis of the form

\vskip 1ex
\setbox\startprefix=\hbox{\tt \ \ \ \ \ \ \ \ \ \$e\ }
\setbox\contprefix=\hbox{\tt \ \ \ \ \ \ \ \ \ \ }
\startm
\m{\vdash}\m{(}\m{\varphi}\m{\rightarrow}\m{\forall}\m{x}\m{\varphi}\m{)}
\endm
\vskip 1ex

should be read ``assume variable $x$ is (effectively) not free in wff
$\varphi$.''\index{effectively not free}
Literally, this says ``assume it is provable that $\varphi \rightarrow \forall
x\, \varphi$.''  This device lets us avoid the complexities associated with
the standard treatment of free and bound variables.
%% Uncomment this when uncommenting section {formalspec} below
The footnote on p.~\pageref{effectivelybound} discusses this further.

\item
A statement of one of the forms

\vskip 1ex
\setbox\startprefix=\hbox{\tt \ \ \ \ \ \ \ \ \ \$a\ }
\setbox\contprefix=\hbox{\tt \ \ \ \ \ \ \ \ \ \ }
\startm
\m{\vdash}\m{(}\m{\lnot}\m{\forall}\m{x}\m{\,x}\m{=}\m{y}\m{\rightarrow}
\m{\ldots}\m{)}
\endm
\setbox\startprefix=\hbox{\tt \ \ \ \ \ \ \ \ \ \$p\ }
\setbox\contprefix=\hbox{\tt \ \ \ \ \ \ \ \ \ \ }
\startm
\m{\vdash}\m{(}\m{\lnot}\m{\forall}\m{x}\m{\,x}\m{=}\m{y}\m{\rightarrow}
\m{\ldots}\m{)}
\endm
\vskip 1ex

should be read ``if $x$ and $y$ are distinct variables, then...''  This
antecedent provides us with a technical device to avoid the need for the
\texttt{\$d} statement early in our development of predicate calculus,
permitting symbol manipulations to be as conceptually simple as those in
propositional calculus.  However, the \texttt{\$d} statement eventually
becomes a requirement, and after that this device is rarely used.

\item
The statement

\vskip 1ex
\setbox\startprefix=\hbox{\tt \ \ \ \ \ \ \ \ \ \$d\ }
\setbox\contprefix=\hbox{\tt \ \ \ \ \ \ \ \ \ \ }
\startm
\m{x}\m{\,y}
\endm
\vskip 1ex

should be read ``assume $x$ and $y$ are distinct variables.''

\item
The statement

\vskip 1ex
\setbox\startprefix=\hbox{\tt \ \ \ \ \ \ \ \ \ \$d\ }
\setbox\contprefix=\hbox{\tt \ \ \ \ \ \ \ \ \ \ }
\startm
\m{x}\m{\,\varphi}
\endm
\vskip 1ex

should be read ``assume $x$ does not occur in $\varphi$.''

\item
The statement

\vskip 1ex
\setbox\startprefix=\hbox{\tt \ \ \ \ \ \ \ \ \ \$d\ }
\setbox\contprefix=\hbox{\tt \ \ \ \ \ \ \ \ \ \ }
\startm
\m{x}\m{\,A}
\endm
\vskip 1ex

should be read ``assume variable $x$ does not occur in class $A$.''

\item
The restriction and hypothesis group

\vskip 1ex
\setbox\startprefix=\hbox{\tt \ \ \ \ \ \ \ \ \ \$d\ }
\setbox\contprefix=\hbox{\tt \ \ \ \ \ \ \ \ \ \ }
\startm
\m{x}\m{\,A}
\endm
\setbox\startprefix=\hbox{\tt \ \ \ \ \ \ \ \ \ \$d\ }
\setbox\contprefix=\hbox{\tt \ \ \ \ \ \ \ \ \ \ }
\startm
\m{x}\m{\,\psi}
\endm
\setbox\startprefix=\hbox{\tt \ \ \ \ \ \ \ \ \ \$e\ }
\setbox\contprefix=\hbox{\tt \ \ \ \ \ \ \ \ \ \ }
\startm
\m{\vdash}\m{(}\m{x}\m{=}\m{A}\m{\rightarrow}\m{(}\m{\varphi}\m{\leftrightarrow}
\m{\psi}\m{)}\m{)}
\endm
\vskip 1ex

is frequently used in place of explicit substitution, meaning ``assume
$\psi$ results from the proper substitution of $A$ for $x$ in
$\varphi$.''  Sometimes ``\texttt{\$e} $\vdash ( \psi \rightarrow
\forall x \, \psi )$'' is used instead of ``\texttt{\$d} $x\, \psi $,''
which requires only that $x$ be effectively not free in $\varphi$ but
not necessarily absent from it.  The use of implicit
substitution\index{substitution!implicit} is partly a matter of personal
style, although it may make proofs somewhat shorter than would be the
case with explicit substitution.

\item
The hypothesis


\vskip 1ex
\setbox\startprefix=\hbox{\tt \ \ \ \ \ \ \ \ \ \$e\ }
\setbox\contprefix=\hbox{\tt \ \ \ \ \ \ \ \ \ \ }
\startm
\m{\vdash}\m{A}\m{\in}\m{{\rm V}}
\endm
\vskip 1ex

should be read ``assume class $A$ is a set (i.e.\ exists).''
This is a convenient convention used by Quine.

\item
The restriction and hypothesis

\vskip 1ex
\setbox\startprefix=\hbox{\tt \ \ \ \ \ \ \ \ \ \$d\ }
\setbox\contprefix=\hbox{\tt \ \ \ \ \ \ \ \ \ \ }
\startm
\m{x}\m{\,y}
\endm
\setbox\startprefix=\hbox{\tt \ \ \ \ \ \ \ \ \ \$e\ }
\setbox\contprefix=\hbox{\tt \ \ \ \ \ \ \ \ \ \ }
\startm
\m{\vdash}\m{(}\m{y}\m{\in}\m{A}\m{\rightarrow}\m{\forall}\m{x}\m{\,y}
\m{\in}\m{A}\m{)}
\endm
\vskip 1ex

should be read ``assume variable $x$ is
(effectively) not free in class $A$.''

\end{itemize}

\section{A Theorem Sampler}\label{sometheorems}

In this section we list some of the more important theorems that are proved in
the \texttt{set.mm} database, and they illustrate the kinds of things that can be
done with Metamath.  While all of these facts are well-known results,
Metamath offers the advantage of easily allowing you to trace their
derivation back to axioms.  Our intent here is not to try to explain the
details or motivation; for this we refer you to the textbooks that are
mentioned in the descriptions.  (The \texttt{set.mm} file has bibliographic
references for the text references.)  Their proofs often embody important
concepts you may wish to explore with the Metamath program (see
Section~\ref{exploring}).  All the symbols that are used here are defined in
Section~\ref{hierarchy}.  For brevity we haven't included the \texttt{\$d}
restrictions or \texttt{\$f} hypotheses for these theorems; when you are
uncertain consult the \texttt{set.mm} database.

We start with \texttt{syl} (principle of the syllogism).
In \textit{Principia Mathematica}
Whitehead and Russell call this ``the principle of the
syllogism... because... the syllogism in Barbara is derived from them''
\cite[quote after Theorem *2.06 p.~101]{PM}.
Some authors call this law a ``hypothetical syllogism.''
As of 2019 \texttt{syl} is the most commonly referenced proven
assertion in the \texttt{set.mm} database.\footnote{
The Metamath program command \texttt{show usage}
shows the number of references.
On 2019-04-29 (commit 71cbbbdb387e) \texttt{syl} was directly referenced
10,819 times. The second most commonly referenced proven assertion
was \texttt{eqid}, which was directly referenced 7,738 times.
}

\vskip 2ex
\noindent Theorem syl (principle of the syllogism)\index{Syllogism}%
\index{\texttt{syl}}\label{syl}.

\vskip 0.5ex
\setbox\startprefix=\hbox{\tt \ \ syl.1\ \$e\ }
\setbox\contprefix=\hbox{\tt \ \ \ \ \ \ \ \ \ \ \ }
\startm
\m{\vdash}\m{(}\m{\varphi}\m{ \rightarrow }\m{\psi}\m{)}
\endm
\setbox\startprefix=\hbox{\tt \ \ syl.2\ \$e\ }
\setbox\contprefix=\hbox{\tt \ \ \ \ \ \ \ \ \ \ \ }
\startm
\m{\vdash}\m{(}\m{\psi}\m{ \rightarrow }\m{\chi}\m{)}
\endm
\setbox\startprefix=\hbox{\tt \ \ syl\ \$p\ }
\setbox\contprefix=\hbox{\tt \ \ \ \ \ \ \ \ \ }
\startm
\m{\vdash}\m{(}\m{\varphi}\m{ \rightarrow }\m{\chi}\m{)}
\endm
\vskip 2ex

The following theorem is not very deep but provides us with a notational device
that is frequently used.  It allows us to use the expression ``$A \in V$'' as
a compact way of saying that class $A$ exists, i.e.\ is a set.

\vskip 2ex
\noindent Two ways to say ``$A$ is a set'':  $A$ is a member of the universe
$V$ if and only if $A$ exists (i.e.\ there exists a set equal to $A$).
Theorem 6.9 of Quine, p. 43.

\vskip 0.5ex
\setbox\startprefix=\hbox{\tt \ \ isset\ \$p\ }
\setbox\contprefix=\hbox{\tt \ \ \ \ \ \ \ \ \ \ \ }
\startm
\m{\vdash}\m{(}\m{A}\m{\in}\m{{\rm V}}\m{\leftrightarrow}\m{\exists}\m{x}\m{\,x}\m{=}
\m{A}\m{)}
\endm
\vskip 1ex

Next we prove the axioms of standard ZF set theory that were missing from our
axiom system.  From our point of view they are theorems since they
can be derived from the other axioms.

\vskip 2ex
\noindent Axiom of Separation\index{Axiom of Separation}
(Aussonderung)\index{Aussonderung} proved from the other axioms of ZF set
theory.  Compare Exercise 4 of Takeuti and Zaring, p.~22.

\vskip 0.5ex
\setbox\startprefix=\hbox{\tt \ \ inex1.1\ \$e\ }
\setbox\contprefix=\hbox{\tt \ \ \ \ \ \ \ \ \ \ \ \ \ \ \ }
\startm
\m{\vdash}\m{A}\m{\in}\m{{\rm V}}
\endm
\setbox\startprefix=\hbox{\tt \ \ inex\ \$p\ }
\setbox\contprefix=\hbox{\tt \ \ \ \ \ \ \ \ \ \ \ \ \ }
\startm
\m{\vdash}\m{(}\m{A}\m{\cap}\m{B}\m{)}\m{\in}\m{{\rm V}}
\endm
\vskip 1ex

\noindent Axiom of the Null Set\index{Axiom of the Null Set} proved from the
other axioms of ZF set theory. Corollary 5.16 of Takeuti and Zaring, p.~20.

\vskip 0.5ex
\setbox\startprefix=\hbox{\tt \ \ 0ex\ \$p\ }
\setbox\contprefix=\hbox{\tt \ \ \ \ \ \ \ \ \ \ \ \ }
\startm
\m{\vdash}\m{\varnothing}\m{\in}\m{{\rm V}}
\endm
\vskip 1ex

\noindent The Axiom of Pairing\index{Axiom of Pairing} proved from the other
axioms of ZF set theory.  Theorem 7.13 of Quine, p.~51.
\vskip 0.5ex
\setbox\startprefix=\hbox{\tt \ \ prex\ \$p\ }
\setbox\contprefix=\hbox{\tt \ \ \ \ \ \ \ \ \ \ \ \ \ \ }
\startm
\m{\vdash}\m{\{}\m{A}\m{,}\m{B}\m{\}}\m{\in}\m{{\rm V}}
\endm
\vskip 2ex

Next we will list some famous or important theorems that are proved in
the \texttt{set.mm} database.  None of them except \texttt{omex}
require the Axiom of Infinity, as you can verify with the \texttt{show
trace{\char`\_}back} Metamath command.

\vskip 2ex
\noindent The resolution of Russell's paradox\index{Russell's paradox}.  There
exists no set containing the set of all sets which are not members of
themselves.  Proposition 4.14 of Takeuti and Zaring, p.~14.

\vskip 0.5ex
\setbox\startprefix=\hbox{\tt \ \ ru\ \$p\ }
\setbox\contprefix=\hbox{\tt \ \ \ \ \ \ \ \ }
\startm
\m{\vdash}\m{\lnot}\m{\exists}\m{x}\m{\,x}\m{=}\m{\{}\m{y}\m{|}\m{\lnot}\m{y}
\m{\in}\m{y}\m{\}}
\endm
\vskip 1ex

\noindent Cantor's theorem\index{Cantor's theorem}.  No set can be mapped onto
its power set.  Compare Theorem 6B(b) of Enderton, p.~132.

\vskip 0.5ex
\setbox\startprefix=\hbox{\tt \ \ canth.1\ \$e\ }
\setbox\contprefix=\hbox{\tt \ \ \ \ \ \ \ \ \ \ \ \ \ }
\startm
\m{\vdash}\m{A}\m{\in}\m{{\rm V}}
\endm
\setbox\startprefix=\hbox{\tt \ \ canth\ \$p\ }
\setbox\contprefix=\hbox{\tt \ \ \ \ \ \ \ \ \ \ \ }
\startm
\m{\vdash}\m{\lnot}\m{F}\m{:}\m{A}\m{\raisebox{.5ex}{${\textstyle{\:}_{
\mbox{\footnotesize\rm {\ }}}}\atop{\textstyle{\longrightarrow}\atop{
\textstyle{}^{\mbox{\footnotesize\rm onto}}}}$}}\m{{\cal P}}\m{A}
\endm
\vskip 1ex

\noindent The Burali-Forti paradox\index{Burali-Forti paradox}.  No set
contains all ordinal numbers. Enderton, p.~194.  (Burali-Forti was one person,
not two.)

\vskip 0.5ex
\setbox\startprefix=\hbox{\tt \ \ onprc\ \$p\ }
\setbox\contprefix=\hbox{\tt \ \ \ \ \ \ \ \ \ \ \ \ }
\startm
\m{\vdash}\m{\lnot}\m{\mbox{\rm On}}\m{\in}\m{{\rm V}}
\endm
\vskip 1ex

\noindent Peano's postulates\index{Peano's postulates} for arithmetic.
Proposition 7.30 of Takeuti and Zaring, pp.~42--43.  The objects being
described are the members of $\omega$ i.e.\ the natural numbers 0, 1,
2,\ldots.  The successor\index{successor} operation suc means ``plus
one.''  \texttt{peano1} says that 0 (which is defined as the empty set)
is a natural number.  \texttt{peano2} says that if $A$ is a natural
number, so is $A+1$.  \texttt{peano3} says that 0 is not the successor
of any natural number.  \texttt{peano4} says that two natural numbers
are equal if and only if their successors are equal.  \texttt{peano5} is
essentially the same as mathematical induction.

\vskip 1ex
\setbox\startprefix=\hbox{\tt \ \ peano1\ \$p\ }
\setbox\contprefix=\hbox{\tt \ \ \ \ \ \ \ \ \ \ \ \ }
\startm
\m{\vdash}\m{\varnothing}\m{\in}\m{\omega}
\endm
\vskip 1.5ex

\setbox\startprefix=\hbox{\tt \ \ peano2\ \$p\ }
\setbox\contprefix=\hbox{\tt \ \ \ \ \ \ \ \ \ \ \ \ }
\startm
\m{\vdash}\m{(}\m{A}\m{\in}\m{\omega}\m{\rightarrow}\m{{\rm suc}}\m{A}\m{\in}%
\m{\omega}\m{)}
\endm
\vskip 1.5ex

\setbox\startprefix=\hbox{\tt \ \ peano3\ \$p\ }
\setbox\contprefix=\hbox{\tt \ \ \ \ \ \ \ \ \ \ \ \ }
\startm
\m{\vdash}\m{(}\m{A}\m{\in}\m{\omega}\m{\rightarrow}\m{\lnot}\m{{\rm suc}}%
\m{A}\m{=}\m{\varnothing}\m{)}
\endm
\vskip 1.5ex

\setbox\startprefix=\hbox{\tt \ \ peano4\ \$p\ }
\setbox\contprefix=\hbox{\tt \ \ \ \ \ \ \ \ \ \ \ \ }
\startm
\m{\vdash}\m{(}\m{(}\m{A}\m{\in}\m{\omega}\m{\wedge}\m{B}\m{\in}\m{\omega}%
\m{)}\m{\rightarrow}\m{(}\m{{\rm suc}}\m{A}\m{=}\m{{\rm suc}}\m{B}\m{%
\leftrightarrow}\m{A}\m{=}\m{B}\m{)}\m{)}
\endm
\vskip 1.5ex

\setbox\startprefix=\hbox{\tt \ \ peano5\ \$p\ }
\setbox\contprefix=\hbox{\tt \ \ \ \ \ \ \ \ \ \ \ \ }
\startm
\m{\vdash}\m{(}\m{(}\m{\varnothing}\m{\in}\m{A}\m{\wedge}\m{\forall}\m{x}\m{%
\in}\m{\omega}\m{(}\m{x}\m{\in}\m{A}\m{\rightarrow}\m{{\rm suc}}\m{x}\m{\in}%
\m{A}\m{)}\m{)}\m{\rightarrow}\m{\omega}\m{\subseteq}\m{A}\m{)}
\endm
\vskip 1.5ex

\noindent Finite Induction (mathematical induction).\index{finite
induction}\index{mathematical induction} The first hypothesis is the
basis and the second is the induction hypothesis.  Theorem Schema 22 of
Suppes, p.~136.

\vskip 0.5ex
\setbox\startprefix=\hbox{\tt \ \ findes.1\ \$e\ }
\setbox\contprefix=\hbox{\tt \ \ \ \ \ \ \ \ \ \ \ \ \ \ }
\startm
\m{\vdash}\m{[}\m{\varnothing}\m{/}\m{x}\m{]}\m{\varphi}
\endm
\setbox\startprefix=\hbox{\tt \ \ findes.2\ \$e\ }
\setbox\contprefix=\hbox{\tt \ \ \ \ \ \ \ \ \ \ \ \ \ \ }
\startm
\m{\vdash}\m{(}\m{x}\m{\in}\m{\omega}\m{\rightarrow}\m{(}\m{\varphi}\m{%
\rightarrow}\m{[}\m{{\rm suc}}\m{x}\m{/}\m{x}\m{]}\m{\varphi}\m{)}\m{)}
\endm
\setbox\startprefix=\hbox{\tt \ \ findes\ \$p\ }
\setbox\contprefix=\hbox{\tt \ \ \ \ \ \ \ \ \ \ \ \ }
\startm
\m{\vdash}\m{(}\m{x}\m{\in}\m{\omega}\m{\rightarrow}\m{\varphi}\m{)}
\endm
\vskip 1ex

\noindent Transfinite Induction with explicit substitution.  The first
hypothesis is the basis, the second is the induction hypothesis for
successors, and the third is the induction hypothesis for limit
ordinals.  Theorem Schema 4 of Suppes, p. 197.

\vskip 0.5ex
\setbox\startprefix=\hbox{\tt \ \ tfindes.1\ \$e\ }
\setbox\contprefix=\hbox{\tt \ \ \ \ \ \ \ \ \ \ \ \ \ \ \ }
\startm
\m{\vdash}\m{[}\m{\varnothing}\m{/}\m{x}\m{]}\m{\varphi}
\endm
\setbox\startprefix=\hbox{\tt \ \ tfindes.2\ \$e\ }
\setbox\contprefix=\hbox{\tt \ \ \ \ \ \ \ \ \ \ \ \ \ \ \ }
\startm
\m{\vdash}\m{(}\m{x}\m{\in}\m{{\rm On}}\m{\rightarrow}\m{(}\m{\varphi}\m{%
\rightarrow}\m{[}\m{{\rm suc}}\m{x}\m{/}\m{x}\m{]}\m{\varphi}\m{)}\m{)}
\endm
\setbox\startprefix=\hbox{\tt \ \ tfindes.3\ \$e\ }
\setbox\contprefix=\hbox{\tt \ \ \ \ \ \ \ \ \ \ \ \ \ \ \ }
\startm
\m{\vdash}\m{(}\m{{\rm Lim}}\m{y}\m{\rightarrow}\m{(}\m{\forall}\m{x}\m{\in}%
\m{y}\m{\varphi}\m{\rightarrow}\m{[}\m{y}\m{/}\m{x}\m{]}\m{\varphi}\m{)}\m{)}
\endm
\setbox\startprefix=\hbox{\tt \ \ tfindes\ \$p\ }
\setbox\contprefix=\hbox{\tt \ \ \ \ \ \ \ \ \ \ \ \ \ }
\startm
\m{\vdash}\m{(}\m{x}\m{\in}\m{{\rm On}}\m{\rightarrow}\m{\varphi}\m{)}
\endm
\vskip 1ex

\noindent Principle of Transfinite Recursion.\index{transfinite
recursion} Theorem 7.41 of Takeuti and Zaring, p.~47.  Transfinite
recursion is the key theorem that allows arithmetic of ordinals to be
rigorously defined, and has many other important uses as well.
Hypotheses \texttt{tfr.1} and \texttt{tfr.2} specify a certain (proper)
class $ F$.  The complicated definition of $ F$ is not important in
itself; what is important is that there be such an $ F$ with the
required properties, and we show this by displaying $ F$ explicitly.
\texttt{tfr1} states that $ F$ is a function whose domain is the set of
ordinal numbers.  \texttt{tfr2} states that any value of $ F$ is
completely determined by its previous values and the values of an
auxiliary function, $G$.  \texttt{tfr3} states that $ F$ is unique,
i.e.\ it is the only function that satisfies \texttt{tfr1} and
\texttt{tfr2}.  Note that $ f$ is an individual variable like $x$ and
$y$; it is just a mnemonic to remind us that $A$ is a collection of
functions.

\vskip 0.5ex
\setbox\startprefix=\hbox{\tt \ \ tfr.1\ \$e\ }
\setbox\contprefix=\hbox{\tt \ \ \ \ \ \ \ \ \ \ \ }
\startm
\m{\vdash}\m{A}\m{=}\m{\{}\m{f}\m{|}\m{\exists}\m{x}\m{\in}\m{{\rm On}}\m{(}%
\m{f}\m{{\rm Fn}}\m{x}\m{\wedge}\m{\forall}\m{y}\m{\in}\m{x}\m{(}\m{f}\m{`}%
\m{y}\m{)}\m{=}\m{(}\m{G}\m{`}\m{(}\m{f}\m{\restriction}\m{y}\m{)}\m{)}\m{)}%
\m{\}}
\endm
\setbox\startprefix=\hbox{\tt \ \ tfr.2\ \$e\ }
\setbox\contprefix=\hbox{\tt \ \ \ \ \ \ \ \ \ \ \ }
\startm
\m{\vdash}\m{F}\m{=}\m{\bigcup}\m{A}
\endm
\setbox\startprefix=\hbox{\tt \ \ tfr1\ \$p\ }
\setbox\contprefix=\hbox{\tt \ \ \ \ \ \ \ \ \ \ }
\startm
\m{\vdash}\m{F}\m{{\rm Fn}}\m{{\rm On}}
\endm
\setbox\startprefix=\hbox{\tt \ \ tfr2\ \$p\ }
\setbox\contprefix=\hbox{\tt \ \ \ \ \ \ \ \ \ \ }
\startm
\m{\vdash}\m{(}\m{z}\m{\in}\m{{\rm On}}\m{\rightarrow}\m{(}\m{F}\m{`}\m{z}%
\m{)}\m{=}\m{(}\m{G}\m{`}\m{(}\m{F}\m{\restriction}\m{z}\m{)}\m{)}\m{)}
\endm
\setbox\startprefix=\hbox{\tt \ \ tfr3\ \$p\ }
\setbox\contprefix=\hbox{\tt \ \ \ \ \ \ \ \ \ \ }
\startm
\m{\vdash}\m{(}\m{(}\m{B}\m{{\rm Fn}}\m{{\rm On}}\m{\wedge}\m{\forall}\m{x}\m{%
\in}\m{{\rm On}}\m{(}\m{B}\m{`}\m{x}\m{)}\m{=}\m{(}\m{G}\m{`}\m{(}\m{B}\m{%
\restriction}\m{x}\m{)}\m{)}\m{)}\m{\rightarrow}\m{B}\m{=}\m{F}\m{)}
\endm
\vskip 1ex

\noindent The existence of omega (the class of natural numbers).\index{natural
number}\index{omega ($\omega$)}\index{Axiom of Infinity}  Axiom 7 of Takeuti
and Zaring, p.~43.  (This is the only theorem in this section requiring the
Axiom of Infinity.)

\vskip 0.5ex
\setbox\startprefix=\hbox{\tt \
\ omex\ \$p\ }
\setbox\contprefix=\hbox{\tt \ \ \ \ \ \ \ \ \ \ }
\startm
\m{\vdash}\m{\omega}\m{\in}\m{{\rm V}}
\endm
%\vskip 2ex


\section{Axioms for Real and Complex Numbers}\label{real}
\index{real and complex numbers!axioms for}

This section presents the axioms for real and complex numbers, along
with some commentary about them.  Analysis
textbooks implicitly or explicitly use these axioms or their equivalents
as their starting point.  In the database \texttt{set.mm}, we define real
and complex numbers as (rather complicated) specific sets and derive these
axioms as {\em theorems} from the axioms of ZF set theory, using a method
called Dedekind cuts.  We omit the details of this construction, which you can
follow if you wish using the \texttt{set.mm} database in conjunction with the
textbooks referenced therein.

Once we prove those theorems, we then restate these proven theorems as axioms.
This lets us easily identify which axioms are needed for a particular complex number proof, without the obfuscation of the set theory used to derive them.
As a result,
the construction is actually unimportant other
than to show that sets exist that satisfy the axioms, and thus that the axioms
are consistent if ZF set theory is consistent.  When working with real numbers
you can think of them as being the actual sets resulting
from the construction (for definiteness), or you can
think of them as otherwise unspecified sets that happen to satisfy the axioms.
The derivation is not easy, but the fact that it works is quite remarkable
and lends support to the idea that ZFC set theory is all we need to
provide a foundation for essentially all of mathematics.

\needspace{3\baselineskip}
\subsection{The Axioms for Real and Complex Numbers Themselves}\label{realactual}

For the axioms we are given (or postulate) 8 classes:  $\mathbb{C}$ (the
set of complex numbers), $\mathbb{R}$ (the set of real numbers, a subset
of $\mathbb{C}$), $0$ (zero), $1$ (one), $i$ (square root of
$-1$), $+$ (plus), $\cdot$ (times), and
$<_{\mathbb{R}}$ (less than for just the real numbers).
Subtraction and division are defined terms and are not part of the
axioms; for their definitions see \texttt{set.mm}.

Note that the notation $(A+B)$ (and similarly $(A\cdot B)$) specifies a class
called an {\em operation},\index{operation} and is the function value of the
class $+$ at ordered pair $\langle A,B \rangle$.  An operation is defined by
statement \texttt{df-opr} on p.~\pageref{dfopr}.
The notation $A <_{\mathbb{R}} B$ specifies a
wff called a {\em binary relation}\index{binary relation} and means $\langle A,B \rangle \in \,<_{\mathbb{R}}$, as defined by statement \texttt{df-br} on p.~\pageref{dfbr}.

Our set of 8 given classes is assumed to satisfy the following 22 axioms
(in the axioms listed below, $<$ really means $<_{\mathbb{R}}$).

\vskip 2ex

\noindent 1. The real numbers are a subset of the complex numbers.

%\vskip 0.5ex
\setbox\startprefix=\hbox{\tt \ \ ax-resscn\ \$p\ }
\setbox\contprefix=\hbox{\tt \ \ \ \ \ \ \ \ \ \ \ \ \ \ }
\startm
\m{\vdash}\m{\mathbb{R}}\m{\subseteq}\m{\mathbb{C}}
\endm
%\vskip 1ex

\noindent 2. One is a complex number.

%\vskip 0.5ex
\setbox\startprefix=\hbox{\tt \ \ ax-1cn\ \$p\ }
\setbox\contprefix=\hbox{\tt \ \ \ \ \ \ \ \ \ \ \ }
\startm
\m{\vdash}\m{1}\m{\in}\m{\mathbb{C}}
\endm
%\vskip 1ex

\noindent 3. The imaginary number $i$ is a complex number.

%\vskip 0.5ex
\setbox\startprefix=\hbox{\tt \ \ ax-icn\ \$p\ }
\setbox\contprefix=\hbox{\tt \ \ \ \ \ \ \ \ \ \ \ }
\startm
\m{\vdash}\m{i}\m{\in}\m{\mathbb{C}}
\endm
%\vskip 1ex

\noindent 4. Complex numbers are closed under addition.

%\vskip 0.5ex
\setbox\startprefix=\hbox{\tt \ \ ax-addcl\ \$p\ }
\setbox\contprefix=\hbox{\tt \ \ \ \ \ \ \ \ \ \ \ \ \ }
\startm
\m{\vdash}\m{(}\m{(}\m{A}\m{\in}\m{\mathbb{C}}\m{\wedge}\m{B}\m{\in}\m{\mathbb{C}}%
\m{)}\m{\rightarrow}\m{(}\m{A}\m{+}\m{B}\m{)}\m{\in}\m{\mathbb{C}}\m{)}
\endm
%\vskip 1ex

\noindent 5. Real numbers are closed under addition.

%\vskip 0.5ex
\setbox\startprefix=\hbox{\tt \ \ ax-addrcl\ \$p\ }
\setbox\contprefix=\hbox{\tt \ \ \ \ \ \ \ \ \ \ \ \ \ \ }
\startm
\m{\vdash}\m{(}\m{(}\m{A}\m{\in}\m{\mathbb{R}}\m{\wedge}\m{B}\m{\in}\m{\mathbb{R}}%
\m{)}\m{\rightarrow}\m{(}\m{A}\m{+}\m{B}\m{)}\m{\in}\m{\mathbb{R}}\m{)}
\endm
%\vskip 1ex

\noindent 6. Complex numbers are closed under multiplication.

%\vskip 0.5ex
\setbox\startprefix=\hbox{\tt \ \ ax-mulcl\ \$p\ }
\setbox\contprefix=\hbox{\tt \ \ \ \ \ \ \ \ \ \ \ \ \ }
\startm
\m{\vdash}\m{(}\m{(}\m{A}\m{\in}\m{\mathbb{C}}\m{\wedge}\m{B}\m{\in}\m{\mathbb{C}}%
\m{)}\m{\rightarrow}\m{(}\m{A}\m{\cdot}\m{B}\m{)}\m{\in}\m{\mathbb{C}}\m{)}
\endm
%\vskip 1ex

\noindent 7. Real numbers are closed under multiplication.

%\vskip 0.5ex
\setbox\startprefix=\hbox{\tt \ \ ax-mulrcl\ \$p\ }
\setbox\contprefix=\hbox{\tt \ \ \ \ \ \ \ \ \ \ \ \ \ \ }
\startm
\m{\vdash}\m{(}\m{(}\m{A}\m{\in}\m{\mathbb{R}}\m{\wedge}\m{B}\m{\in}\m{\mathbb{R}}%
\m{)}\m{\rightarrow}\m{(}\m{A}\m{\cdot}\m{B}\m{)}\m{\in}\m{\mathbb{R}}\m{)}
\endm
%\vskip 1ex

\noindent 8. Multiplication of complex numbers is commutative.

%\vskip 0.5ex
\setbox\startprefix=\hbox{\tt \ \ ax-mulcom\ \$p\ }
\setbox\contprefix=\hbox{\tt \ \ \ \ \ \ \ \ \ \ \ \ \ \ }
\startm
\m{\vdash}\m{(}\m{(}\m{A}\m{\in}\m{\mathbb{C}}\m{\wedge}\m{B}\m{\in}\m{\mathbb{C}}%
\m{)}\m{\rightarrow}\m{(}\m{A}\m{\cdot}\m{B}\m{)}\m{=}\m{(}\m{B}\m{\cdot}\m{A}%
\m{)}\m{)}
\endm
%\vskip 1ex

\noindent 9. Addition of complex numbers is associative.

%\vskip 0.5ex
\setbox\startprefix=\hbox{\tt \ \ ax-addass\ \$p\ }
\setbox\contprefix=\hbox{\tt \ \ \ \ \ \ \ \ \ \ \ \ \ \ }
\startm
\m{\vdash}\m{(}\m{(}\m{A}\m{\in}\m{\mathbb{C}}\m{\wedge}\m{B}\m{\in}\m{\mathbb{C}}%
\m{\wedge}\m{C}\m{\in}\m{\mathbb{C}}\m{)}\m{\rightarrow}\m{(}\m{(}\m{A}\m{+}%
\m{B}\m{)}\m{+}\m{C}\m{)}\m{=}\m{(}\m{A}\m{+}\m{(}\m{B}\m{+}\m{C}\m{)}\m{)}%
\m{)}
\endm
%\vskip 1ex

\noindent 10. Multiplication of complex numbers is associative.

%\vskip 0.5ex
\setbox\startprefix=\hbox{\tt \ \ ax-mulass\ \$p\ }
\setbox\contprefix=\hbox{\tt \ \ \ \ \ \ \ \ \ \ \ \ \ \ }
\startm
\m{\vdash}\m{(}\m{(}\m{A}\m{\in}\m{\mathbb{C}}\m{\wedge}\m{B}\m{\in}\m{\mathbb{C}}%
\m{\wedge}\m{C}\m{\in}\m{\mathbb{C}}\m{)}\m{\rightarrow}\m{(}\m{(}\m{A}\m{\cdot}%
\m{B}\m{)}\m{\cdot}\m{C}\m{)}\m{=}\m{(}\m{A}\m{\cdot}\m{(}\m{B}\m{\cdot}\m{C}%
\m{)}\m{)}\m{)}
\endm
%\vskip 1ex

\noindent 11. Multiplication distributes over addition for complex numbers.

%\vskip 0.5ex
\setbox\startprefix=\hbox{\tt \ \ ax-distr\ \$p\ }
\setbox\contprefix=\hbox{\tt \ \ \ \ \ \ \ \ \ \ \ \ \ }
\startm
\m{\vdash}\m{(}\m{(}\m{A}\m{\in}\m{\mathbb{C}}\m{\wedge}\m{B}\m{\in}\m{\mathbb{C}}%
\m{\wedge}\m{C}\m{\in}\m{\mathbb{C}}\m{)}\m{\rightarrow}\m{(}\m{A}\m{\cdot}\m{(}%
\m{B}\m{+}\m{C}\m{)}\m{)}\m{=}\m{(}\m{(}\m{A}\m{\cdot}\m{B}\m{)}\m{+}\m{(}%
\m{A}\m{\cdot}\m{C}\m{)}\m{)}\m{)}
\endm
%\vskip 1ex

\noindent 12. The square of $i$ equals $-1$ (expressed as $i$-squared plus 1 is
0).

%\vskip 0.5ex
\setbox\startprefix=\hbox{\tt \ \ ax-i2m1\ \$p\ }
\setbox\contprefix=\hbox{\tt \ \ \ \ \ \ \ \ \ \ \ \ }
\startm
\m{\vdash}\m{(}\m{(}\m{i}\m{\cdot}\m{i}\m{)}\m{+}\m{1}\m{)}\m{=}\m{0}
\endm
%\vskip 1ex

\noindent 13. One and zero are distinct.

%\vskip 0.5ex
\setbox\startprefix=\hbox{\tt \ \ ax-1ne0\ \$p\ }
\setbox\contprefix=\hbox{\tt \ \ \ \ \ \ \ \ \ \ \ \ }
\startm
\m{\vdash}\m{1}\m{\ne}\m{0}
\endm
%\vskip 1ex

\noindent 14. One is an identity element for real multiplication.

%\vskip 0.5ex
\setbox\startprefix=\hbox{\tt \ \ ax-1rid\ \$p\ }
\setbox\contprefix=\hbox{\tt \ \ \ \ \ \ \ \ \ \ \ }
\startm
\m{\vdash}\m{(}\m{A}\m{\in}\m{\mathbb{R}}\m{\rightarrow}\m{(}\m{A}\m{\cdot}\m{1}%
\m{)}\m{=}\m{A}\m{)}
\endm
%\vskip 1ex

\noindent 15. Every real number has a negative.

%\vskip 0.5ex
\setbox\startprefix=\hbox{\tt \ \ ax-rnegex\ \$p\ }
\setbox\contprefix=\hbox{\tt \ \ \ \ \ \ \ \ \ \ \ \ \ \ }
\startm
\m{\vdash}\m{(}\m{A}\m{\in}\m{\mathbb{R}}\m{\rightarrow}\m{\exists}\m{x}\m{\in}%
\m{\mathbb{R}}\m{(}\m{A}\m{+}\m{x}\m{)}\m{=}\m{0}\m{)}
\endm
%\vskip 1ex

\noindent 16. Every nonzero real number has a reciprocal.

%\vskip 0.5ex
\setbox\startprefix=\hbox{\tt \ \ ax-rrecex\ \$p\ }
\setbox\contprefix=\hbox{\tt \ \ \ \ \ \ \ \ \ \ \ \ \ \ }
\startm
\m{\vdash}\m{(}\m{A}\m{\in}\m{\mathbb{R}}\m{\rightarrow}\m{(}\m{A}\m{\ne}\m{0}%
\m{\rightarrow}\m{\exists}\m{x}\m{\in}\m{\mathbb{R}}\m{(}\m{A}\m{\cdot}%
\m{x}\m{)}\m{=}\m{1}\m{)}\m{)}
\endm
%\vskip 1ex

\noindent 17. A complex number can be expressed in terms of two reals.

%\vskip 0.5ex
\setbox\startprefix=\hbox{\tt \ \ ax-cnre\ \$p\ }
\setbox\contprefix=\hbox{\tt \ \ \ \ \ \ \ \ \ \ \ \ }
\startm
\m{\vdash}\m{(}\m{A}\m{\in}\m{\mathbb{C}}\m{\rightarrow}\m{\exists}\m{x}\m{\in}%
\m{\mathbb{R}}\m{\exists}\m{y}\m{\in}\m{\mathbb{R}}\m{A}\m{=}\m{(}\m{x}\m{+}\m{(}%
\m{y}\m{\cdot}\m{i}\m{)}\m{)}\m{)}
\endm
%\vskip 1ex

\noindent 18. Ordering on reals satisfies strict trichotomy.

%\vskip 0.5ex
\setbox\startprefix=\hbox{\tt \ \ ax-pre-lttri\ \$p\ }
\setbox\contprefix=\hbox{\tt \ \ \ \ \ \ \ \ \ \ \ \ \ }
\startm
\m{\vdash}\m{(}\m{(}\m{A}\m{\in}\m{\mathbb{R}}\m{\wedge}\m{B}\m{\in}\m{\mathbb{R}}%
\m{)}\m{\rightarrow}\m{(}\m{A}\m{<}\m{B}\m{\leftrightarrow}\m{\lnot}\m{(}\m{A}%
\m{=}\m{B}\m{\vee}\m{B}\m{<}\m{A}\m{)}\m{)}\m{)}
\endm
%\vskip 1ex

\noindent 19. Ordering on reals is transitive.

%\vskip 0.5ex
\setbox\startprefix=\hbox{\tt \ \ ax-pre-lttrn\ \$p\ }
\setbox\contprefix=\hbox{\tt \ \ \ \ \ \ \ \ \ \ \ \ \ }
\startm
\m{\vdash}\m{(}\m{(}\m{A}\m{\in}\m{\mathbb{R}}\m{\wedge}\m{B}\m{\in}\m{\mathbb{R}}%
\m{\wedge}\m{C}\m{\in}\m{\mathbb{R}}\m{)}\m{\rightarrow}\m{(}\m{(}\m{A}\m{<}%
\m{B}\m{\wedge}\m{B}\m{<}\m{C}\m{)}\m{\rightarrow}\m{A}\m{<}\m{C}\m{)}\m{)}
\endm
%\vskip 1ex

\noindent 20. Ordering on reals is preserved after addition to both sides.

%\vskip 0.5ex
\setbox\startprefix=\hbox{\tt \ \ ax-pre-ltadd\ \$p\ }
\setbox\contprefix=\hbox{\tt \ \ \ \ \ \ \ \ \ \ \ \ \ }
\startm
\m{\vdash}\m{(}\m{(}\m{A}\m{\in}\m{\mathbb{R}}\m{\wedge}\m{B}\m{\in}\m{\mathbb{R}}%
\m{\wedge}\m{C}\m{\in}\m{\mathbb{R}}\m{)}\m{\rightarrow}\m{(}\m{A}\m{<}\m{B}\m{%
\rightarrow}\m{(}\m{C}\m{+}\m{A}\m{)}\m{<}\m{(}\m{C}\m{+}\m{B}\m{)}\m{)}\m{)}
\endm
%\vskip 1ex

\noindent 21. The product of two positive reals is positive.

%\vskip 0.5ex
\setbox\startprefix=\hbox{\tt \ \ ax-pre-mulgt0\ \$p\ }
\setbox\contprefix=\hbox{\tt \ \ \ \ \ \ \ \ \ \ \ \ \ \ }
\startm
\m{\vdash}\m{(}\m{(}\m{A}\m{\in}\m{\mathbb{R}}\m{\wedge}\m{B}\m{\in}\m{\mathbb{R}}%
\m{)}\m{\rightarrow}\m{(}\m{(}\m{0}\m{<}\m{A}\m{\wedge}\m{0}%
\m{<}\m{B}\m{)}\m{\rightarrow}\m{0}\m{<}\m{(}\m{A}\m{\cdot}\m{B}\m{)}%
\m{)}\m{)}
\endm
%\vskip 1ex

\noindent 22. A non-empty, bounded-above set of reals has a supremum.

%\vskip 0.5ex
\setbox\startprefix=\hbox{\tt \ \ ax-pre-sup\ \$p\ }
\setbox\contprefix=\hbox{\tt \ \ \ \ \ \ \ \ \ \ \ }
\startm
\m{\vdash}\m{(}\m{(}\m{A}\m{\subseteq}\m{\mathbb{R}}\m{\wedge}\m{A}\m{\ne}\m{%
\varnothing}\m{\wedge}\m{\exists}\m{x}\m{\in}\m{\mathbb{R}}\m{\forall}\m{y}\m{%
\in}\m{A}\m{\,y}\m{<}\m{x}\m{)}\m{\rightarrow}\m{\exists}\m{x}\m{\in}\m{%
\mathbb{R}}\m{(}\m{\forall}\m{y}\m{\in}\m{A}\m{\lnot}\m{x}\m{<}\m{y}\m{\wedge}\m{%
\forall}\m{y}\m{\in}\m{\mathbb{R}}\m{(}\m{y}\m{<}\m{x}\m{\rightarrow}\m{\exists}%
\m{z}\m{\in}\m{A}\m{\,y}\m{<}\m{z}\m{)}\m{)}\m{)}
\endm

% NOTE: The \m{...} expressions above could be represented as
% $ \vdash ( ( A \subseteq \mathbb{R} \wedge A \ne \varnothing \wedge \exists x \in \mathbb{R} \forall y \in A \,y < x ) \rightarrow \exists x \in \mathbb{R} ( \forall y \in A \lnot x < y \wedge \forall y \in \mathbb{R} ( y < x \rightarrow \exists z \in A \,y < z ) ) ) $

\vskip 2ex

This completes the set of axioms for real and complex numbers.  You may
wish to look at how subtraction, division, and decimal numbers
are defined in \texttt{set.mm}, and for fun look at the proof of $2+
2 = 4$ (theorem \texttt{2p2e4} in \texttt{set.mm})
as discussed in section \ref{2p2e4}.

In \texttt{set.mm} we define the non-negative integers $\mathbb{N}$, the integers
$\mathbb{Z}$, and the rationals $\mathbb{Q}$ as subsets of $\mathbb{R}$.  This leads
to the nice inclusion $\mathbb{N} \subseteq \mathbb{Z} \subseteq \mathbb{Q} \subseteq
\mathbb{R} \subseteq \mathbb{C}$, giving us a uniform framework in which, for
example, a property such as commutativity of complex number addition
automatically applies to integers.  The natural numbers $\mathbb{N}$
are different from the set $\omega$ we defined earlier, but both satisfy
Peano's postulates.

\subsection{Complex Number Axioms in Analysis Texts}

Most analysis texts construct complex numbers as ordered pairs of reals,
leading to construction-dependent properties that satisfy these axioms
but are not stated in their pure form.  (This is also done in
\texttt{set.mm} but our axioms are extracted from that construction.)
Other texts will simply state that $\mathbb{R}$ is a ``complete ordered
subfield of $\mathbb{C}$,'' leading to redundant axioms when this phrase
is completely expanded out.  In fact I have not seen a text with the
axioms in the explicit form above.
None of these axioms is unique individually, but this carefully worked out
collection of axioms is the result of years of work
by the Metamath community.

\subsection{Eliminating Unnecessary Complex Number Axioms}

We once had more axioms for real and complex numbers, but over years of time
we (the Metamath community)
have found ways to eliminate them (by proving them from other axioms)
or weaken them (by making weaker claims without reducing what
can be proved).
In particular, here are statements that used to be complex number
axioms but have since been formally proven (with Metamath) to be redundant:

\begin{itemize}
\item
  $\mathbb{C} \in V$.
  At one time this was listed as a ``complex number axiom.''
  However, this is not properly speaking a complex number axiom,
  and in any case its proof uses axioms of set theory.
  Proven redundant by Mario Carneiro\index{Carneiro, Mario} on
  17-Nov-2014 (see \texttt{axcnex}).
\item
  $((A \in \mathbb{C} \land B \in \mathbb{C}$) $\rightarrow$
  $(A + B) = (B + A))$.
  Proved redundant by Eric Schmidt\index{Schmidt, Eric} on 19-Jun-2012,
  and formalized by Scott Fenton\index{Fenton, Scott} on 3-Jan-2013
  (see \texttt{addcom}).
\item
  $(A \in \mathbb{C} \rightarrow (A + 0) = A)$.
  Proved redundant by Eric Schmidt on 19-Jun-2012,
  and formalized by Scott Fenton on 3-Jan-2013
  (see \texttt{addid1}).
\item
  $(A \in \mathbb{C} \rightarrow \exists x \in \mathbb{C} (A + x) = 0)$.
  Proved redundant by Eric Schmidt and formalized on 21-May-2007
  (see \texttt{cnegex}).
\item
  $((A \in \mathbb{C} \land A \ne 0) \rightarrow \exists x \in \mathbb{C} (A \cdot x) = 1)$.
  Proved redundant by Eric Schmidt and formalized on 22-May-2007
  (see \texttt{recex}).
\item
  $0 \in \mathbb{R}$.
  Proved redundant by Eric Schmidt on 19-Feb-2005 and formalized 21-May-2007
  (see \texttt{0re}).
\end{itemize}

We could eliminate 0 as an axiomatic object by defining it as
$( ( i \cdot i ) + 1 )$
and replacing it with this expression throughout the axioms. If this
is done, axiom ax-i2m1 becomes redundant. However, the remaining axioms
would become longer and less intuitive.

Eric Schmidt's paper analyzing this axiom system \cite{Schmidt}
presented a proof that these remaining axioms,
with the possible exception of ax-mulcom, are independent of the others.
It is currently an open question if ax-mulcom is independent of the others.

\section{Two Plus Two Equals Four}\label{2p2e4}

Here is a proof that $2 + 2 = 4$, as proven in the theorem \texttt{2p2e4}
in the database \texttt{set.mm}.
This is a useful demonstration of what a Metamath proof can look like.
This proof may have more steps than you're used to, but each step is rigorously
proven all the way back to the axioms of logic and set theory.
This display was originally generated by the Metamath program
as an {\sc HTML} file.

In the table showing the proof ``Step'' is the sequential step number,
while its associated ``Expression'' is an expression that we have proved.
``Ref'' is the name of a theorem or axiom that justifies that expression,
and ``Hyp'' refers to previous steps (if any) that the theorem or axiom
needs so that we can use it.  Expressions are indented further than
the expressions that depend on them to show their interdependencies.

\begin{table}[!htbp]
\caption{Two plus two equals four}
\begin{tabular}{lllll}
\textbf{Step} & \textbf{Hyp} & \textbf{Ref} & \textbf{Expression} & \\
1  &       & df-2    & $ \; \; \vdash 2 = 1 + 1$  & \\
2  & 1     & oveq2i  & $ \; \vdash (2 + 2) = (2 + (1 + 1))$ & \\
3  &       & df-4    & $ \; \; \vdash 4 = (3 + 1)$ & \\
4  &       & df-3    & $ \; \; \; \vdash 3 = (2 + 1)$ & \\
5  & 4     & oveq1i  & $ \; \; \vdash (3 + 1) = ((2 + 1) + 1)$ & \\
6  &       & 2cn     & $ \; \; \; \vdash 2 \in \mathbb{C}$ & \\
7  &       & ax-1cn  & $ \; \; \; \vdash 1 \in \mathbb{C}$ & \\
8  & 6,7,7 & addassi & $ \; \; \vdash ((2 + 1) + 1) = (2 + (1 + 1))$ & \\
9  & 3,5,8 & 3eqtri  & $ \; \vdash 4 = (2 + (1 + 1))$ & \\
10 & 2,9   & eqtr4i  & $ \vdash (2 + 2) = 4$ & \\
\end{tabular}
\end{table}

Step 1 says that we can assert that $2 = 1 + 1$ because it is
justified by \texttt{df-2}.
What is \texttt{df-2}?
It is simply the definition of $2$, which in our system is defined as being
equal to $1 + 1$.  This shows how we can use definitions in proofs.

Look at Step 2 of the proof. In the Ref column, we see that it references
a previously proved theorem, \texttt{oveq2i}.
It turns out that
theorem \texttt{oveq2i} requires a
hypothesis, and in the Hyp column of Step 2 we indicate that Step 1 will
satisfy (match) this hypothesis.
If we looked at \texttt{oveq2i}
we would find that it proves that given some hypothesis
$A = B$, we can prove that $( C F A ) = ( C F B )$.
If we use \texttt{oveq2i} and apply step 1's result as the hypothesis,
that will mean that $A = 2$ and $B = ( 1 + 1 )$ within this use of
\texttt{oveq2i}.
We are free to select any value of $C$ and $F$ (subject to syntax constraints),
so we are free to select $C = 2$ and $F = +$,
producing our desired result,
$ (2 + 2) = (2 + (1 + 1))$.

Step 2 is an example of substitution.
In the end, every step in every proof uses only this one substitution rule.
All the rules of logic, and all the axioms, are expressed so that
they can be used via this one substitution rule.
So once you master substitution, you can master every Metamath proof,
no exceptions.

Each step is clear and can be immediately checked.
In the {\sc HTML} display you can even click on each reference to see why it is
justified, making it easy to see why the proof works.

\section{Deduction}\label{deduction}

Strictly speaking,
a deduction (also called an inference) is a kind of statement that needs
some hypotheses to be true in order for its conclusion to be true.
A theorem, on the other hand, has no hypotheses.
Informally we often call both of them theorems, but in this section we
will stick to the strict definitions.

It sometimes happens that we have proved a deduction of the form
$\varphi \Rightarrow \psi$\index{$\Rightarrow$}
(given hypothesis $\varphi$ we can prove $\psi$)
and we want to then prove a theorem of the form
$\varphi \rightarrow \psi$.

Converting a deduction (which uses a hypothesis) into a theorem
(which does not) is not as simple as you might think.
The deduction says, ``if we can prove $\varphi$ then we can prove $\psi$,''
which is in some sense weaker than saying
``$\varphi$ implies $\psi$.''
There is no axiom of logic that permits us to directly obtain the theorem
given the deduction.\footnote{
The conversion of a deduction to a theorem does not even hold in general
for quantum propositional calculus,
which is a weak subset of classical propositional calculus.
It has been shown that adding the Standard Deduction Theorem (discussed below)
to quantum propositional calculus turns it into classical
propositional calculus!
}

This is in contrast to going the other way.
If we have the theorem ($\varphi \rightarrow \psi$),
it is easy to recover the deduction
($\varphi \Rightarrow \psi$)
using modus ponens\index{modus ponens}
(\texttt{ax-mp}; see section \ref{axmp}).

In the following subsections we first discuss the standard deduction theorem
(the traditional but awkward way to convert deductions into theorems) and
the weak deduction theorem (a limited version of the standard deduction
theorem that is easier to use and was once widely used in
\texttt{set.mm}\index{set theory database (\texttt{set.mm})}).
In section \ref{deductionstyle} we discuss
deduction style, the newer approach we now recommend in most cases.
Deduction style uses ``deduction form,'' a form that
prefixes each hypothesis (other than definitions) and the
conclusion with a universal antecedent (``$\varphi \rightarrow$'').
Deduction style is widely used in \texttt{set.mm},
so it is useful to understand it and \textit{why} it is widely used.
Section \ref{naturaldeduction}
briefly discusses our approach for using natural deduction
within \texttt{set.mm},
as that approach is deeply related to deduction style.
We conclude with a summary of the strengths of
our approach, which we believe are compelling.

\subsection{The Standard Deduction Theorem}\label{standarddeductiontheorem}

It is possible to make use of information
contained in the deduction or its proof to assist us with the proof of
the related theorem.
In traditional logic books, there is a metatheorem called the
Deduction Theorem\index{Deduction Theorem}\index{Standard Deduction Theorem},
discovered independently by Herbrand and Tarski around 1930.
The Deduction Theorem, which we often call the Standard Deduction Theorem,
provides an algorithm for constructing a proof of a theorem from the
proof of its corresponding deduction. See, for example,
\cite[p.~56]{Margaris}\index{Margaris, Angelo}.
To construct a proof for a theorem, the
algorithm looks at each step in the proof of the original deduction and
rewrites the step with several steps wherein the hypothesis is eliminated
and becomes an antecedent.

In ordinary mathematics, no one actually carries out the algorithm,
because (in its most basic form) it involves an exponential explosion of
the number of proof steps as more hypotheses are eliminated. Instead,
the Standard Deduction Theorem is invoked simply to claim that it can
be done in principle, without actually doing it.
What's more, the algorithm is not as simple as it might first appear
when applying it rigorously.
There is a subtle restriction on the Standard Deduction Theorem
that must be taken into account involving the axiom of generalization
when working with predicate calculus (see the literature for more detail).

One of the goals of Metamath is to let you plainly see, with as few
underlying concepts as possible, how mathematics can be derived directly
from the axioms, and not indirectly according to some hidden rules
buried inside a program or understood only by logicians. If we added
the Standard Deduction Theorem to the language and proof verifier,
that would greatly complicate both and largely defeat Metamath's goal
of simplicity. In principle, we could show direct proofs by expanding
out the proof steps generated by the algorithm of the Standard Deduction
Theorem, but that is not feasible in practice because the number of proof
steps quickly becomes huge, even astronomical.
Since the algorithm of the Standard Deduction Theorem is driven by the proof,
we would have to go through that proof
all over again---starting from axioms---in order to obtain the theorem form.
In terms of proof length, there would be no savings over just
proving the theorem directly instead of first proving the deduction form.

\subsection{Weak Deduction Theorem}\label{weakdeductiontheorem}

We have developed
a more efficient method for proving a theorem from a deduction
that can be used instead of the Standard Deduction Theorem
in many (but not all) cases.
We call this more efficient method the
Weak Deduction Theorem\index{Weak Deduction Theorem}.\footnote{
There is also an unrelated ``Weak Deduction Theorem''
in the field of relevance logic, so to avoid confusion we could call
ours the ``Weak Deduction Theorem for Classical Logic.''}
Unlike the Standard Deduction Theorem, the Weak Deduction Theorem produces the
theorem directly from a special substitution instance of the deduction,
using a small, fixed number of steps roughly proportional to the length
of the final theorem.

If you come to a proof referencing the Weak Deduction Theorem
\texttt{dedth} (or one of its variants \texttt{dedthxx}),
here is how to follow the proof without getting into the details:
just click on the theorem referenced in the step
just before the reference to \texttt{dedth} and ignore everything else.
Theorem \texttt{dedth} simply turns a hypothesis into an antecedent
(i.e. the hypothesis followed by $\rightarrow$
is placed in front of the assertion, and the hypothesis
itself is eliminated) given certain conditions.

The Weak Deduction Theorem
eliminates a hypothesis $\varphi$, making it become an antecedent.
It does this by proving an expression
$ \varphi \rightarrow \psi $ given two hypotheses:
(1)
$ ( A = {\rm if} ( \varphi , A , B ) \rightarrow ( \varphi \leftrightarrow \chi ) ) $
and
(2) $\chi$.
Note that it requires that a proof exists for $\varphi$ when the class variable
$A$ is replaced with a specific class $B$. The hypothesis $\chi$
should be assigned to the inference.
You can see the details of the proof of the Weak Deduction Theorem
in theorem \texttt{dedth}.

The Weak Deduction Theorem
is probably easier to understand by studying proofs that make use of it.
For example, let's look at the proof of \texttt{renegcl}, which proves that
$ \vdash ( A \in \mathbb{R} \rightarrow - A \in \mathbb{R} )$:

\needspace{4\baselineskip}
\begin{longtabu} {l l l X}
\textbf{Step} & \textbf{Hyp} & \textbf{Ref} & \textbf{Expression} \\
  1 &  & negeq &
  $\vdash$ $($ $A$ $=$ ${\rm if}$ $($ $A$ $\in$ $\mathbb{R}$ $,$ $A$ $,$ $1$ $)$ $\rightarrow$
  $\textrm{-}$ $A$ $=$ $\textrm{-}$ ${\rm if}$ $($ $A$ $\in$ $\mathbb{R}$
  $,$ $A$ $,$ $1$ $)$ $)$ \\
 2 & 1 & eleq1d &
    $\vdash$ $($ $A$ $=$ ${\rm if}$ $($ $A$ $\in$ $\mathbb{R}$ $,$ $A$ $,$ $1$ $)$ $\rightarrow$ $($
    $\textrm{-}$ $A$ $\in$ $\mathbb{R}$ $\leftrightarrow$
    $\textrm{-}$ ${\rm if}$ $($ $A$ $\in$ $\mathbb{R}$ $,$ $A$ $,$ $1$ $)$ $\in$
    $\mathbb{R}$ $)$ $)$ \\
 3 &  & 1re & $\vdash 1 \in \mathbb{R}$ \\
 4 & 3 & elimel &
   $\vdash {\rm if} ( A \in \mathbb{R} , A , 1 ) \in \mathbb{R}$ \\
 5 & 4 & renegcli &
   $\vdash \textrm{-} {\rm if} ( A \in \mathbb{R} , A , 1 ) \in \mathbb{R}$ \\
 6 & 2,5 & dedth &
   $\vdash ( A \in \mathbb{R} \rightarrow \textrm{-} A \in \mathbb{R}$ ) \\
\end{longtabu}

The somewhat strange-looking steps in \texttt{renegcl} before step 5 are
technical stuff that makes this magic work, and they can be ignored
for a quick overview of the proof. To continue following the ``important''
part of the proof of \texttt{renegcl},
you can look at the reference to \texttt{renegcli} at step 5.

That said, let's briefly look at how
\texttt{renegcl} uses the
Weak Deduction Theorem (\texttt{dedth}) to do its job,
in case you want to do something similar or want understand it more deeply.
Let's work backwards in the proof of \texttt{renegcl}.
Step 6 applies \texttt{dedth} to produce our goal result
$ \vdash ( A \in \mathbb{R} \rightarrow\, - A \in \mathbb{R} )$.
This requires on the one hand the (substituted) deduction
\texttt{renegcli} in step 5.
By itself \texttt{renegcli} proves the deduction
$ \vdash A \in \mathbb{R} \Rightarrow\, \vdash - A \in \mathbb{R}$;
this is the deduction form we are trying to turn into theorem form,
and thus
\texttt{renegcli} has a separate hypothesis that must be fulfilled.
To fulfill the hypothesis of the invocation of
\texttt{renegcli} in step 5, it is eventually
reduced to the already proven theorem $1 \in \mathbb{R}$ in step 3.
Step 4 connects steps 3 and 5; step 4 invokes
\texttt{elimel}, a special case of \texttt{elimhyp} that eliminates
a membership hypothesis for the weak deduction theorem.
On the other hand, the equivalence of the conclusion of
\texttt{renegcl}
$( - A \in \mathbb{R} )$ and the substituted conclusion of
\texttt{renegcli} must be proven, which is done in steps 2 and 1.

The weak deduction theorem has limitations.
In particular, we must be able to prove a special case of the deduction's
hypothesis as a stand-alone theorem.
For example, we used $1 \in \mathbb{R}$ in step 3 of \texttt{renegcl}.

We used to use the weak deduction theorem
extensively within \texttt{set.mm}.
However, we now recommend applying ``deduction style''
instead in most cases, as deduction style is
often an easier and clearer approach.
Therefore, we will now describe deduction style.

\subsection{Deduction Style}\label{deductionstyle}

We now prefer to write assertions in ``deduction form''
instead of writing a proof that would require use of the standard or
weak deduction theorem.
We call this appraoch
``deduction style.''\index{deduction style}

It will be easier to explain this by first defining some terms:

\begin{itemize}
\item \textbf{closed form}\index{closed form}\index{forms!closed}:
A kind of assertion (theorem) with no hypotheses.
Typically its label has no special suffix.
An example is \texttt{unss}, which states:
$\vdash ( ( A \subseteq C \wedge B \subseteq C ) \leftrightarrow ( A \cup B )
\subseteq C )\label{eq:unss}$
\item \textbf{deduction form}\index{deduction form}\index{forms!deduction}:
A kind of assertion with one or more hypotheses
where the conclusion is an implication with
a wff variable as the antecedent (usually $\varphi$), and every hypothesis
(\$e statement)
is either (1) an implication with the same antecedent as the conclusion or
(2) a definition.
A definition
can be for a class variable (this is a class variable followed by ``='')
or a wff variable (this is a wff variable followed by $\leftrightarrow$);
class variable definitions are more common.
In practice, a proof
in deduction form will also contain many steps that are implications
where the antecedent is either that wff variable (normally $\varphi$)
or is
a conjunction (...$\land$...) including that wff variable ($\varphi$).
If an assertion is in deduction form, and other forms are also available,
then we suffix its label with ``d.''
An example is \texttt{unssd}, which states\footnote{
For brevity we show here (and in other places)
a $\&$\index{$\&$} between hypotheses\index{hypotheses}
and a $\Rightarrow$\index{$\Rightarrow$}\index{conclusion}
between the hypotheses and the conclusion.
This notation is technically not part of the Metamath language, but is
instead a convenient abbreviation to show both the hypotheses and conclusion.}:
$\vdash ( \varphi \rightarrow A \subseteq C )\quad\&\quad \vdash ( \varphi
    \rightarrow B \subseteq C )\quad\Rightarrow\quad \vdash ( \varphi
    \rightarrow ( A \cup B ) \subseteq C )\label{eq:unssd}$
\item \textbf{inference form}\index{inference form}\index{forms!inference}:
A kind of assertion with one or more hypotheses that is not in deduction form
(e.g., there is no common antecedent).
If an assertion is in inference form, and other forms are also available,
then we suffix its label with ``i.''
An example is \texttt{unssi}, which states:
$\vdash A \subseteq C\quad\&\quad \vdash B \subseteq C\quad\Rightarrow\quad
    \vdash ( A \cup B ) \subseteq C\label{eq:unssi}$
\end{itemize}

When using deduction style we express an assertion in deduction form.
This form prefixes each hypothesis (other than definitions) and the
conclusion with a universal antecedent (``$\varphi \rightarrow$'').
The antecedent (e.g., $\varphi$)
mimics the context handled in the deduction theorem, eliminating
the need to directly use the deduction theorem.

Once you have an assertion in deduction form, you can easily convert it
to inference form or closed form:

\begin{itemize}
\item To
prove some assertion Ti in inference form, given assertion Td in deduction
form, there is a simple mechanical process you can use. First take each
Ti hypothesis and insert a \texttt{T.} $\rightarrow$ prefix (``true implies'')
using \texttt{a1i}. You
can then use the existing assertion Td to prove the resulting conclusion
with a \texttt{T.} $\rightarrow$ prefix.
Finally, you can remove that prefix using \texttt{mptru},
resulting in the conclusion you wanted to prove.
\item To
prove some assertion T in closed form, given assertion Td in deduction
form, there is another simple mechanical process you can use. First,
select an expression that is the conjunction (...$\land$...) of all of the
consequents of every hypothesis of Td. Next, prove that this expression
implies each of the separate hypotheses of Td in turn by eliminating
conjuncts (there are a variety of proven assertions to do this, including
\texttt{simpl},
\texttt{simpr},
\texttt{3simpa},
\texttt{3simpb},
\texttt{3simpc},
\texttt{simp1},
\texttt{simp2},
and
\texttt{simp3}).
If the
expression has nested conjunctions, inner conjuncts can be broken out by
chaining the above theorems with \texttt{syl}
(see section \ref{syl}).\footnote{
There are actually many theorems
(labeled simp* such as \texttt{simp333}) that break out inner conjuncts in one
step, but rather than learning them you can just use the chaining we
just described to prove them, and then let the Metamath program command
\texttt{minimize{\char`\_}with}\index{\texttt{minimize{\char`\_}with} command}
figure out the right ones needed to collapse them.}
As your final step, you can then apply the already-proven assertion Td
(which is in deduction form), proving assertion T in closed form.
\end{itemize}

We can also easily convert any assertion T in closed form to its related
assertion Ti in inference form by applying
modus ponens\index{modus ponens} (see section \ref{axmp}).

The deduction form antecedent can also be used to represent the context
necessary to support natural deduction systems, so we will now
discuss natural deduction.

\subsection{Natural Deduction}\label{naturaldeduction}

Natural deduction\index{natural deduction}
(ND) systems, as such, were originally introduced in
1934 by two logicians working independently: Ja\'skowski and Gentzen. ND
systems are supposed to reconstruct, in a formally proper way, traditional
ways of mathematical reasoning (such as conditional proof, indirect proof,
and proof by cases). As reconstructions they were naturally influenced
by previous work, and many specific ND systems and notations have been
developed since their original work.

There are many ND variants, but
Indrzejczak \cite[p.~31-32]{Indrzejczak}\index{Indrzejczak, Andrzej}
suggests that any natural deductive system must satisfy at
least these three criteria:

\begin{itemize}
\item ``There are some means for entering assumptions into a proof and
also for eliminating them. Usually it requires some bookkeeping devices
for indicating the scope of an assumption, and showing that a part of
a proof depending on eliminated assumption is discharged.
\item There are no (or, at least, a very limited set of) axioms, because
their role is taken over by the set of primitive rules for introduction
and elimination of logical constants which means that elementary
inferences instead of formulae are taken as primitive.
\item (A genuine) ND system admits a lot of freedom in proof construction
and possibility of applying several proof search strategies, like
conditional proof, proof by cases, proof by reductio ad absurdum etc.''
\end{itemize}

The Metamath Proof Explorer (MPE) as defined in \texttt{set.mm}
is fundamentally a Hilbert-style system.
That is, MPE is based on a larger number of axioms (compared
to natural deduction systems), a very small set of rules of inference
(modus ponens), and the context is not changed by the rules of inference
in the middle of a proof. That said, MPE proofs can be developed using
the natural deduction (ND) approach as originally developed by Ja\'skowski
and Gentzen.

The most common and recommended approach for applying ND in MPE is to use
deduction form\index{deduction form}%
\index{forms!deduction}
and apply the MPE proven assertions that are equivalent to ND rules.
For example, MPE's \texttt{jca} is equivalent to ND rule $\land$-I
(and-insertion).
We maintain a list of equivalences that you may consult.
This approach for applying an ND approach within MPE relies on Metamath's
wff metavariables in an essential way, and is described in more detail
in the presentation ``Natural Deductions in the Metamath Proof Language''
by Mario Carneiro \cite{CarneiroND}\index{Carneiro, Mario}.

In this style many steps are an implication, whose antecedent mimics
the context ($\Gamma$) of most ND systems. To add an assumption, simply add
it to the implication antecedent (typically using
\texttt{simpr}),
and use that
new antecedent for all later claims in the same scope. If you wish to
use an assertion in an ND hypothesis scope that is outside the current
ND hypothesis scope, modify the assertion so that the ND hypothesis
assumption is added to its antecedent (typically using \texttt{adantr}). Most
proof steps will be proved using rules that have hypotheses and results
of the form $\varphi \rightarrow$ ...

An example may make this clearer.
Let's show theorem 5.5 of
\cite[p.~18]{Clemente}\index{Clemente Laboreo, Daniel}
along with a line by line translation using the usual
translation of natural deduction (ND) in the Metamath Proof Explorer
(MPE) notation (this is proof \texttt{ex-natded5.5}).
The proof's original goal was to prove
$\lnot \psi$ given two hypotheses,
$( \psi \rightarrow \chi )$ and $ \lnot \chi$.
We will translate these statements into MPE deduction form
by prefixing them all with $\varphi \rightarrow$.
As a result, in MPE the goal is stated as
$( \varphi \rightarrow \lnot \psi )$, and the two hypotheses are stated as
$( \varphi \rightarrow ( \psi \rightarrow \chi ) )$ and
$( \varphi \rightarrow \lnot \chi )$.

The following table shows the proof in Fitch natural deduction style
and its MPE equivalent.
The \textit{\#} column shows the original numbering,
\textit{MPE\#} shows the number in the equivalent MPE proof
(which we will show later),
\textit{ND Expression} shows the original proof claim in ND notation,
and \textit{MPE Translation} shows its translation into MPE
as discussed in this section.
The final columns show the rationale in ND and MPE respectively.

\needspace{4\baselineskip}
{\setlength{\extrarowsep}{4pt} % Keep rows from being too close together
\begin{longtabu}   { @{} c c X X X X }
\textbf{\#} & \textbf{MPE\#} & \textbf{ND Ex\-pres\-sion} &
\textbf{MPE Trans\-lation} & \textbf{ND Ration\-ale} &
\textbf{MPE Ra\-tio\-nale} \\
\endhead

1 & 2;3 &
$( \psi \rightarrow \chi )$ &
$( \varphi \rightarrow ( \psi \rightarrow \chi ) )$ &
Given &
\$e; \texttt{adantr} to put in ND hypothesis \\

2 & 5 &
$ \lnot \chi$ &
$( \varphi \rightarrow \lnot \chi )$ &
Given &
\$e; \texttt{adantr} to put in ND hypothesis \\

3 & 1 &
... $\vert$ $\psi$ &
$( \varphi \rightarrow \psi )$ &
ND hypothesis assumption &
\texttt{simpr} \\

4 & 4 &
... $\chi$ &
$( ( \varphi \land \psi ) \rightarrow \chi )$ &
$\rightarrow$\,E 1,3 &
\texttt{mpd} 1,3 \\

5 & 6 &
... $\lnot \chi$ &
$( ( \varphi \land \psi ) \rightarrow \lnot \chi )$ &
IT 2 &
\texttt{adantr} 5 \\

6 & 7 &
$\lnot \psi$ &
$( \varphi \rightarrow \lnot \psi )$ &
$\land$\,I 3,4,5 &
\texttt{pm2.65da} 4,6 \\

\end{longtabu}
}


The original used Latin letters; we have replaced them with Greek letters
to follow Metamath naming conventions and so that it is easier to follow
the Metamath translation. The Metamath line-for-line translation of
this natural deduction approach precedes every line with an antecedent
including $\varphi$ and uses the Metamath equivalents of the natural deduction
rules. To add an assumption, the antecedent is modified to include it
(typically by using \texttt{adantr};
\texttt{simpr} is useful when you want to
depend directly on the new assumption, as is shown here).

In Metamath we can represent the two given statements as these hypotheses:

\needspace{2\baselineskip}
\begin{itemize}
\item ex-natded5.5.1 $\vdash ( \varphi \rightarrow ( \psi \rightarrow \chi ) )$
\item ex-natded5.5.2 $\vdash ( \varphi \rightarrow \lnot \chi )$
\end{itemize}

\needspace{4\baselineskip}
Here is the proof in Metamath as a line-by-line translation:

\begin{longtabu}   { l l l X }
\textbf{Step} & \textbf{Hyp} & \textbf{Ref} & \textbf{Ex\-pres\-sion} \\
\endhead
1 & & simpr & $\vdash ( ( \varphi \land \psi ) \rightarrow \psi )$ \\
2 & & ex-natded5.5.1 &
  $\vdash ( \varphi \rightarrow ( \psi \rightarrow \chi ) )$ \\
3 & 2 & adantr &
 $\vdash ( ( \varphi \land \psi ) \rightarrow ( \psi \rightarrow \chi ) )$ \\
4 & 1, 3 & mpd &
 $\vdash ( ( \varphi \land \psi ) \rightarrow \chi ) $ \\
5 & & ex-natded5.5.2 &
 $\vdash ( \varphi \rightarrow \lnot \chi )$ \\
6 & 5 & adantr &
 $\vdash ( ( \varphi \land \psi ) \rightarrow \lnot \chi )$ \\
7 & 4, 6 & pm2.65da &
 $\vdash ( \varphi \rightarrow \lnot \psi )$ \\
\end{longtabu}

Only using specific natural deduction rules directly can lead to very
long proofs, for exactly the same reason that only using axioms directly
in Hilbert-style proofs can lead to very long proofs.
If the goal is short and clear proofs,
then it is better to reuse already-proven assertions
in deduction form than to start from scratch each time
and using only basic natural deduction rules.

\subsection{Strengths of Our Approach}

As far as we know there is nothing else in the literature like either the
weak deduction theorem or Mario Carneiro\index{Carneiro, Mario}'s
natural deduction method.
In order to
transform a hypothesis into an antecedent, the literature's standard
``Deduction Theorem''\index{Deduction Theorem}\index{Standard Deduction Theorem}
requires metalogic outside of the notions provided
by the axiom system. We instead generally prefer to use Mario Carneiro's
natural deduction method, then use the weak deduction theorem in cases
where that is difficult to apply, and only then use the full standard
deduction theorem as a last resort.

The weak deduction theorem\index{Weak Deduction Theorem}
does not require any additional metalogic
but converts an inference directly into a closed form theorem, with
a rigorous proof that uses only the axiom system. Unlike the standard
Deduction Theorem, there is no implicit external justification that we
have to trust in order to use it.

Mario Carneiro's natural deduction\index{natural deduction}
method also does not require any new metalogical
notions. It avoids the Deduction Theorem's metalogic by prefixing the
hypotheses and conclusion of every would-be inference with a universal
antecedent (``$\varphi \rightarrow$'') from the very start.

We think it is impressive and satisfying that we can do so much in a
practical sense without stepping outside of our Hilbert-style axiom system.
Of course our axiomatization, which is in the form of schemes,
contains a metalogic of its own that we exploit. But this metalogic
is relatively simple, and for our Deduction Theorem alternatives,
we primarily use just the direct substitution of expressions for
metavariables.

\begin{sloppy}
\section{Exploring the Set The\-o\-ry Data\-base}\label{exploring}
\end{sloppy}
% NOTE: All examples performed in this section are
% recorded wtih "set width 61" % on set.mm as of 2019-05-28
% commit c1e7849557661260f77cfdf0f97ac4354fbb4f4d.

At this point you may wish to study the \texttt{set.mm}\index{set theory
database (\texttt{set.mm})} file in more detail.  Pay particular
attention to the assumptions needed to define wffs\index{well-formed
formula (wff)} (which are not included above), the variable types
(\texttt{\$f}\index{\texttt{\$f} statement} statements), and the
definitions that are introduced.  Start with some simple theorems in
propositional calculus, making sure you understand in detail each step
of a proof.  Once you get past the first few proofs and become familiar
with the Metamath language, any part of the \texttt{set.mm} database
will be as easy to follow, step by step, as any other part---you won't
have to undergo a ``quantum leap'' in mathematical sophistication to be
able to follow a deep proof in set theory.

Next, you may want to explore how concepts such as natural numbers are
defined and described.  This is probably best done in conjunction with
standard set theory textbooks, which can help give you a higher-level
understanding.  The \texttt{set.mm} database provides references that will get
you started.  From there, you will be on your way towards a very deep,
rigorous understanding of abstract mathematics.

The Metamath\index{Metamath} program can help you peruse a Metamath data\-base,
wheth\-er you are trying to figure out how a certain step follows in a proof or
just have a general curiosity.  We will go through some examples of the
commands, using the \texttt{set.mm}\index{set theory database (\texttt{set.mm})}
database provided with the Metamath software.  These should help get you
started.  See Chapter~\ref{commands} for a more detailed description of
the commands.  Note that we have included the full spelling of all commands to
prevent ambiguity with future commands.  In practice you may type just the
characters needed to specify each command keyword\index{command keyword}
unambiguously, often just one or two characters per keyword, and you don't
need to type them in upper case.

First run the Metamath program as described earlier.  You should see the
\verb/MM>/ prompt.  Read in the \texttt{set.mm} file:\index{\texttt{read}
command}

\begin{verbatim}
MM> read set.mm
Reading source file "set.mm"... 34554442 bytes
34554442 bytes were read into the source buffer.
The source has 155711 statements; 2254 are $a and 32250 are $p.
No errors were found.  However, proofs were not checked.
Type VERIFY PROOF * if you want to check them.
\end{verbatim}

As with most examples in this book, what you will see
will be slightly different because we are continuously
improving our databases (including \texttt{set.mm}).

Let's check the database integrity.  This may take a minute or two to run if
your computer is slow.

\begin{verbatim}
MM> verify proof *
0 10%  20%  30%  40%  50%  60%  70%  80%  90% 100%
..................................................
All proofs in the database were verified in 2.84 s.
\end{verbatim}

No errors were reported, so every proof is correct.

You need to know the names (labels) of theorems before you can look at them.
Often just examining the database file(s) with a text editor is the best
approach.  In \texttt{set.mm} there are many detailed comments, especially near
the beginning, that can help guide you. The \texttt{search} command in the
Metamath program is also handy.  The \texttt{comments} qualifier will list the
statements whose associated comment (the one immediately before it) contain a
string you give it.  For example, if you are studying Enderton's {\em Elements
of Set Theory} \cite{Enderton}\index{Enderton, Herbert B.} you may want to see
the references to it in the database.  The search string \texttt{enderton} is not
case sensitive.  (This will not show you all the database theorems that are in
Enderton's book because there is usually only one citation for a given
theorem, which may appear in several textbooks.)\index{\texttt{search}
command}

\begin{verbatim}
MM> search * "enderton" / comments
12067 unineq $p "... Exercise 20 of [Enderton] p. 32 and ..."
12459 undif2 $p "...Corollary 6K of [Enderton] p. 144. (C..."
12953 df-tp $a "...s. Definition of [Enderton] p. 19. (Co..."
13689 unissb $p ".... Exercise 5 of [Enderton] p. 26 and ..."
\end{verbatim}
\begin{center}
(etc.)
\end{center}

Or you may want to see what theorems have something to do with
conjunction (logical {\sc and}).  The quotes around the search
string are optional when there's no ambiguity.\index{\texttt{search}
command}

\begin{verbatim}
MM> search * conjunction / comments
120 a1d $p "...be replaced with a conjunction ( ~ df-an )..."
662 df-bi $a "...viated form after conjunction is introdu..."
1319 wa $a "...ff definition to include conjunction ('and')."
1321 df-an $a "Define conjunction (logical 'and'). Defini..."
1420 imnan $p "...tion in terms of conjunction. (Contribu..."
\end{verbatim}
\begin{center}
(etc.)
\end{center}


Now we will start to look at some details.  Let's look at the first
axiom of propositional calculus
(we could use \texttt{sh st} to abbreviate
\texttt{show statement}).\index{\texttt{show statement} command}

\begin{verbatim}
MM> show statement ax-1/full
Statement 19 is located on line 881 of the file "set.mm".
"Axiom _Simp_.  Axiom A1 of [Margaris] p. 49.  One of the 3
axioms of propositional calculus.  The 3 axioms are also
        ...
19 ax-1 $a |- ( ph -> ( ps -> ph ) ) $.
Its mandatory hypotheses in RPN order are:
  wph $f wff ph $.
  wps $f wff ps $.
The statement and its hypotheses require the variables:  ph
      ps
The variables it contains are:  ph ps


Statement 49 is located on line 11182 of the file "set.mm".
Its statement number for HTML pages is 6.
"Axiom _Simp_.  Axiom A1 of [Margaris] p. 49.  One of the 3
axioms of propositional calculus.  The 3 axioms are also
given as Definition 2.1 of [Hamilton] p. 28.
...
49 ax-1 $a |- ( ph -> ( ps -> ph ) ) $.
Its mandatory hypotheses in RPN order are:
  wph $f wff ph $.
  wps $f wff ps $.
The statement and its hypotheses require the variables:
  ph ps
The variables it contains are:  ph ps
\end{verbatim}

Compare this to \texttt{ax-1} on p.~\pageref{ax1}.  You can see that the
symbol \texttt{ph} is the {\sc ascii} notation for $\varphi$, etc.  To
see the mathematical symbols for any expression you may typeset it in
\LaTeX\ (type \texttt{help tex} for instructions)\index{latex@{\LaTeX}}
or, easier, just use a text editor to look at the comments where symbols
are first introduced in \texttt{set.mm}.  The hypotheses \texttt{wph}
and \texttt{wps} required by \texttt{ax-1} mean that variables
\texttt{ph} and \texttt{ps} must be wffs.

Next we'll pick a simple theorem of propositional calculus, the Principle of
Identity, which is proved directly from the axioms.  We'll look at the
statement then its proof.\index{\texttt{show statement}
command}

\begin{verbatim}
MM> show statement id1/full
Statement 116 is located on line 11371 of the file "set.mm".
Its statement number for HTML pages is 22.
"Principle of identity.  Theorem *2.08 of [WhiteheadRussell]
p. 101.  This version is proved directly from the axioms for
demonstration purposes.
...
116 id1 $p |- ( ph -> ph ) $= ... $.
Its mandatory hypotheses in RPN order are:
  wph $f wff ph $.
Its optional hypotheses are:  wps wch wth wta wet
      wze wsi wrh wmu wla wka
The statement and its hypotheses require the variables:  ph
These additional variables are allowed in its proof:
      ps ch th ta et ze si rh mu la ka
The variables it contains are:  ph
\end{verbatim}

The optional variables\index{optional variable} \texttt{ps}, \texttt{ch}, etc.\ are
available for use in a proof of this statement if we wish, and were we to do
so we would make use of optional hypotheses \texttt{wps}, \texttt{wch}, etc.  (See
Section~\ref{dollaref} for the meaning of ``optional
hypothesis.''\index{optional hypothesis}) The reason these show up in the
statement display is that statement \texttt{id1} happens to be in their scope
(see Section~\ref{scoping} for the definition of ``scope''\index{scope}), but
in fact in propositional calculus we will never make use of optional
hypotheses or variables.  This becomes important after quantifiers are
introduced, where ``dummy'' variables\index{dummy variable} are often needed
in the middle of a proof.

Let's look at the proof of statement \texttt{id1}.  We'll use the
\texttt{show proof} command, which by default suppresses the
``non-essential'' steps that construct the wffs.\index{\texttt{show proof}
command}
We will display the proof in ``lemmon' format (a non-indented format
with explicit previous step number references) and renumber the
displayed steps:

\begin{verbatim}
MM> show proof id1 /lemmon/renumber
1 ax-1           $a |- ( ph -> ( ph -> ph ) )
2 ax-1           $a |- ( ph -> ( ( ph -> ph ) -> ph ) )
3 ax-2           $a |- ( ( ph -> ( ( ph -> ph ) -> ph ) ) ->
                     ( ( ph -> ( ph -> ph ) ) -> ( ph -> ph )
                                                          ) )
4 2,3 ax-mp      $a |- ( ( ph -> ( ph -> ph ) ) -> ( ph -> ph
                                                          ) )
5 1,4 ax-mp      $a |- ( ph -> ph )
\end{verbatim}

If you have read Section~\ref{trialrun}, you'll know how to interpret this
proof.  Step~2, for example, is an application of axiom \texttt{ax-1}.  This
proof is identical to the one in Hamilton's {\em Logic for Mathematicians}
\cite[p.~32]{Hamilton}\index{Hamilton, Alan G.}.

You may want to look at what
substitutions\index{substitution!variable}\index{variable substitution} are
made into \texttt{ax-1} to arrive at step~2.  The command to do this needs to
know the ``real'' step number, so we'll display the proof again without
the \texttt{renumber} qualifier.\index{\texttt{show proof}
command}

\begin{verbatim}
MM> show proof id1 /lemmon
 9 ax-1          $a |- ( ph -> ( ph -> ph ) )
20 ax-1          $a |- ( ph -> ( ( ph -> ph ) -> ph ) )
24 ax-2          $a |- ( ( ph -> ( ( ph -> ph ) -> ph ) ) ->
                     ( ( ph -> ( ph -> ph ) ) -> ( ph -> ph )
                                                          ) )
25 20,24 ax-mp   $a |- ( ( ph -> ( ph -> ph ) ) -> ( ph -> ph
                                                          ) )
26 9,25 ax-mp    $a |- ( ph -> ph )
\end{verbatim}

The ``real'' step number is 20.  Let's look at its details.

\begin{verbatim}
MM> show proof id1 /detailed_step 20
Proof step 20:  min=ax-1 $a |- ( ph -> ( ( ph -> ph ) -> ph )
  )
This step assigns source "ax-1" ($a) to target "min" ($e).
The source assertion requires the hypotheses "wph" ($f, step
18) and "wps" ($f, step 19).  The parent assertion of the
target hypothesis is "ax-mp" ($a, step 25).
The source assertion before substitution was:
    ax-1 $a |- ( ph -> ( ps -> ph ) )
The following substitutions were made to the source
assertion:
    Variable  Substituted with
     ph        ph
     ps        ( ph -> ph )
The target hypothesis before substitution was:
    min $e |- ph
The following substitution was made to the target hypothesis:
    Variable  Substituted with
     ph        ( ph -> ( ( ph -> ph ) -> ph ) )
\end{verbatim}

This shows the substitutions\index{substitution!variable}\index{variable
substitution} made to the variables in \texttt{ax-1}.  References are made to
steps 18 and 19 which are not shown in our proof display.  To see these steps,
you can display the proof with the \texttt{all} qualifier.

Let's look at a slightly more advanced proof of propositional calculus.  Note
that \verb+/\+ is the symbol for $\wedge$ (logical {\sc and}, also
called conjunction).\index{conjunction ($\wedge$)}
\index{logical {\sc and} ($\wedge$)}

\begin{verbatim}
MM> show statement prth/full
Statement 1791 is located on line 15503 of the file "set.mm".
Its statement number for HTML pages is 559.
"Conjoin antecedents and consequents of two premises.  This
is the closed theorem form of ~ anim12d .  Theorem *3.47 of
[WhiteheadRussell] p. 113.  It was proved by Leibniz,
and it evidently pleased him enough to call it
_praeclarum theorema_ (splendid theorem).
...
1791 prth $p |- ( ( ( ph -> ps ) /\ ( ch -> th ) ) -> ( ( ph
      /\ ch ) -> ( ps /\ th ) ) ) $= ... $.
Its mandatory hypotheses in RPN order are:
  wph $f wff ph $.
  wps $f wff ps $.
  wch $f wff ch $.
  wth $f wff th $.
Its optional hypotheses are:  wta wet wze wsi wrh wmu wla wka
The statement and its hypotheses require the variables:  ph
      ps ch th
These additional variables are allowed in its proof:  ta et
      ze si rh mu la ka
The variables it contains are:  ph ps ch th


MM> show proof prth /lemmon/renumber
1 simpl          $p |- ( ( ( ph -> ps ) /\ ( ch -> th ) ) ->
                                               ( ph -> ps ) )
2 simpr          $p |- ( ( ( ph -> ps ) /\ ( ch -> th ) ) ->
                                               ( ch -> th ) )
3 1,2 anim12d    $p |- ( ( ( ph -> ps ) /\ ( ch -> th ) ) ->
                           ( ( ph /\ ch ) -> ( ps /\ th ) ) )
\end{verbatim}

There are references to a lot of unfamiliar statements.  To see what they are,
you may type the following:

\begin{verbatim}
MM> show proof prth /statement_summary
Summary of statements used in the proof of "prth":

Statement simpl is located on line 14748 of the file
"set.mm".
"Elimination of a conjunct.  Theorem *3.26 (Simp) of
[WhiteheadRussell] p. 112. ..."
  simpl $p |- ( ( ph /\ ps ) -> ph ) $= ... $.

Statement simpr is located on line 14777 of the file
"set.mm".
"Elimination of a conjunct.  Theorem *3.27 (Simp) of
[WhiteheadRussell] ..."
  simpr $p |- ( ( ph /\ ps ) -> ps ) $= ... $.

Statement anim12d is located on line 15445 of the file
"set.mm".
"Conjoin antecedents and consequents in a deduction.
..."
  anim12d.1 $e |- ( ph -> ( ps -> ch ) ) $.
  anim12d.2 $e |- ( ph -> ( th -> ta ) ) $.
  anim12d $p |- ( ph -> ( ( ps /\ th ) -> ( ch /\ ta ) ) )
      $= ... $.
\end{verbatim}
\begin{center}
(etc.)
\end{center}

Of course you can look at each of these statements and their proofs, and
so on, back to the axioms of propositional calculus if you wish.

The \texttt{search} command is useful for finding statements when you
know all or part of their contents.  The following example finds all
statements containing \verb@ph -> ps@ followed by \verb@ch -> th@.  The
\verb@$*@ is a wildcard that matches anything; the \texttt{\$} before the
\verb$*$ prevents conflicts with math symbol token names.  The \verb@*@ after
\texttt{SEARCH} is also a wildcard that in this case means ``match any label.''
\index{\texttt{search} command}

% I'm omitting this one, since readers are unlikely to see it:
% 1096 bisymOLD $p |- ( ( ( ph -> ps ) -> ( ch -> th ) ) -> ( (
%   ( ps -> ph ) -> ( th -> ch ) ) -> ( ( ph <-> ps ) -> ( ch
%    <-> th ) ) ) )
\begin{verbatim}
MM> search * "ph -> ps $* ch -> th"
1791 prth $p |- ( ( ( ph -> ps ) /\ ( ch -> th ) ) -> ( ( ph
    /\ ch ) -> ( ps /\ th ) ) )
2455 pm3.48 $p |- ( ( ( ph -> ps ) /\ ( ch -> th ) ) -> ( (
    ph \/ ch ) -> ( ps \/ th ) ) )
117859 pm11.71 $p |- ( ( E. x ph /\ E. y ch ) -> ( ( A. x (
    ph -> ps ) /\ A. y ( ch -> th ) ) <-> A. x A. y ( ( ph /\
    ch ) -> ( ps /\ th ) ) ) )
\end{verbatim}

Three statements, \texttt{prth}, \texttt{pm3.48},
 and \texttt{pm11.71}, were found to match.

To see what axioms\index{axiom} and definitions\index{definition}
\texttt{prth} ultimately depends on for its proof, you can have the
program backtrack through the hierarchy\index{hierarchy} of theorems and
definitions.\index{\texttt{show trace{\char`\_}back} command}

\begin{verbatim}
MM> show trace_back prth /essential/axioms
Statement "prth" assumes the following axioms ($a
statements):
  ax-1 ax-2 ax-3 ax-mp df-bi df-an
\end{verbatim}

Note that the 3 axioms of propositional calculus and the modus ponens rule are
needed (as expected); in addition, there are a couple of definitions that are used
along the way.  Note that Metamath makes no distinction\index{axiom vs.\
definition} between axioms\index{axiom} and definitions\index{definition}.  In
\texttt{set.mm} they have been distinguished artificially by prefixing their
labels\index{labels in \texttt{set.mm}} with \texttt{ax-} and \texttt{df-}
respectively.  For example, \texttt{df-an} defines conjunction (logical {\sc
and}), which is represented by the symbol \verb+/\+.
Section~\ref{definitions} discusses the philosophy of definitions, and the
Metamath language takes a particularly simple, conservative approach by using
the \texttt{\$a}\index{\texttt{\$a} statement} statement for both axioms and
definitions.

You can also have the program compute how many steps a proof
has\index{proof length} if we were to follow it all the way back to
\texttt{\$a} statements.

\begin{verbatim}
MM> show trace_back prth /essential/count_steps
The statement's actual proof has 3 steps.  Backtracking, a
total of 79 different subtheorems are used.  The statement
and subtheorems have a total of 274 actual steps.  If
subtheorems used only once were eliminated, there would be a
total of 38 subtheorems, and the statement and subtheorems
would have a total of 185 steps.  The proof would have 28349
steps if fully expanded back to axiom references.  The
maximum path length is 38.  A longest path is:  prth <-
anim12d <- syl2and <- sylan2d <- ancomsd <- ancom <- pm3.22
<- pm3.21 <- pm3.2 <- ex <- sylbir <- biimpri <- bicomi <-
bicom1 <- bi2 <- dfbi1 <- impbii <- bi3 <- simprim <- impi <-
con1i <- nsyl2 <- mt3d <- con1d <- notnot1 <- con2i <- nsyl3
<- mt2d <- con2d <- notnot2 <- pm2.18d <- pm2.18 <- pm2.21 <-
pm2.21d <- a1d <- syl <- mpd <- a2i <- a2i.1 .
\end{verbatim}

This tells us that we would have to inspect 274 steps if we want to
verify the proof completely starting from the axioms.  A few more
statistics are also shown.  There are one or more paths back to axioms
that are the longest; this command ferrets out one of them and shows it
to you.  There may be a sense in which the longest path length is
related to how ``deep'' the theorem is.

We might also be curious about what proofs depend on the theorem
\texttt{prth}.  If it is never used later on, we could eliminate it as
redundant if it has no intrinsic interest by itself.\index{\texttt{show
usage} command}

% I decided to show the OLD values here.
\begin{verbatim}
MM> show usage prth
Statement "prth" is directly referenced in the proofs of 18
statements:
  mo3 moOLD 2mo 2moOLD euind reuind reuss2 reusv3i opelopabt
  wemaplem2 rexanre rlimcn2 o1of2 o1rlimmul 2sqlem6 spanuni
  heicant pm11.71
\end{verbatim}

Thus \texttt{prth} is directly used by 18 proofs.
We can use the \texttt{/recursive} qualifier to include indirect use:

\begin{verbatim}
MM> show usage prth /recursive
Statement "prth" directly or indirectly affects the proofs of
24214 statements:
  mo3 mo mo3OLD eu2 moOLD eu2OLD eu3OLD mo4f mo4 eu4 mopick
...
\end{verbatim}

\subsection{A Note on the ``Compact'' Proof Format}

The Metamath program will display proofs in a ``compact''\index{compact proof}
format whenever the proof is stored in compressed format in the database.  It
may be be slightly confusing unless you know how to interpret it.
For example,
if you display the complete proof of theorem \texttt{id1} it will start
off as follows:

\begin{verbatim}
MM> show proof id1 /lemmon/all
 1 wph           $f wff ph
 2 wph           $f wff ph
 3 wph           $f wff ph
 4 2,3 wi    @4: $a wff ( ph -> ph )
 5 1,4 wi    @5: $a wff ( ph -> ( ph -> ph ) )
 6 @4            $a wff ( ph -> ph )
\end{verbatim}

\begin{center}
{etc.}
\end{center}

Step 4 has a ``local label,''\index{local label} \texttt{@4}, assigned to it.
Later on, at step 6, the label \texttt{@1} is referenced instead of
displaying the explicit proof for that step.  This technique takes advantage
of the fact that steps in a proof often repeat, especially during the
construction of wffs.  The compact format reduces the number of steps in the
proof display and may be preferred by some people.

If you want to see the normal format with the ``true'' step numbers, you can
use the following workaround:\index{\texttt{save proof} command}

\begin{verbatim}
MM> save proof id1 /normal
The proof of "id1" has been reformatted and saved internally.
Remember to use WRITE SOURCE to save it permanently.
MM> show proof id1 /lemmon/all
 1 wph           $f wff ph
 2 wph           $f wff ph
 3 wph           $f wff ph
 4 2,3 wi        $a wff ( ph -> ph )
 5 1,4 wi        $a wff ( ph -> ( ph -> ph ) )
 6 wph           $f wff ph
 7 wph           $f wff ph
 8 6,7 wi        $a wff ( ph -> ph )
\end{verbatim}

\begin{center}
{etc.}
\end{center}

Note that the original 6 steps are now 8 steps.  However, the format is
now the same as that described in Chapter~\ref{using}.

\chapter{The Metamath Language}
\label{languagespec}

\begin{quote}
  {\em Thus mathematics may be defined as the subject in which we never know
what we are talking about, nor whether what we are saying is true.}
    \flushright\sc  Bertrand Russell\footnote{\cite[p.~84]{Russell2}.}\\
\end{quote}\index{Russell, Bertrand}

Probably the most striking feature of the Metamath language is its almost
complete absence of hard-wired syntax. Metamath\index{Metamath} does not
understand any mathematics or logic other than that needed to construct finite
sequences of symbols according to a small set of simple, built-in rules.  The
only rule it uses in a proof is the substitution of an expression (symbol
sequence) for a variable, subject to a simple constraint to prevent
bound-variable clashes.  The primitive notions built into Metamath involve the
simple manipulation of finite objects (symbols) that we as humans can easily
visualize and that computers can easily deal with.  They seem to be just
about the simplest notions possible that are required to do standard
mathematics.

This chapter serves as a reference manual for the Metamath\index{Metamath}
language. It covers the tedious technical details of the language, some of
which you may wish to skim in a first reading.  On the other hand, you should
pay close attention to the defined terms in {\bf boldface}; they have precise
meanings that are important to keep in mind for later understanding.  It may
be best to first become familiar with the examples in Chapter~\ref{using} to
gain some motivation for the language.

%% Uncomment this when uncommenting section {formalspec} below
If you have some knowledge of set theory, you may wish to study this
chapter in conjunction with the formal set-theoretical description of the
Metamath language in Appendix~\ref{formalspec}.

We will use the name ``Metamath''\index{Metamath} to mean either the Metamath
computer language or the Metamath software associated with the computer
language.  We will not distinguish these two when the context is clear.

The next section contains the complete specification of the Metamath
language.
It serves as an
authoritative reference and presents the syntax in enough detail to
write a parser\index{parsing Metamath} and proof verifier.  The
specification is terse and it is probably hard to learn the language
directly from it, but we include it here for those impatient people who
prefer to see everything up front before looking at verbose expository
material.  Later sections explain this material and provide examples.
We will repeat the definitions in those sections, and you may skip the
next section at first reading and proceed to Section~\ref{tut1}
(p.~\pageref{tut1}).

\section{Specification of the Metamath Language}\label{spec}
\index{Metamath!specification}

\begin{quote}
  {\em Sometimes one has to say difficult things, but one ought to say
them as simply as one knows how.}
    \flushright\sc  G. H. Hardy\footnote{As quoted in
    \cite{deMillo}, p.~273.}\\
\end{quote}\index{Hardy, G. H.}

\subsection{Preliminaries}\label{spec1}

% Space is technically a printable character, so we'll word things
% carefully so it's unambiguous.
A Metamath {\bf database}\index{database} is built up from a top-level source
file together with any source files that are brought in through file inclusion
commands (see below).  The only characters that are allowed to appear in a
Metamath source file are the 94 non-whitespace printable {\sc
ascii}\index{ascii@{\sc ascii}} characters, which are digits, upper and lower
case letters, and the following 32 special
characters\index{special characters}:\label{spec1chars}

\begin{verbatim}
! " # $ % & ' ( ) * + , - . / :
; < = > ? @ [ \ ] ^ _ ` { | } ~
\end{verbatim}
plus the following characters which are the ``white space'' characters:
space (a printable character),
tab, carriage return, line feed, and form feed.\label{whitespace}
We will use \texttt{typewriter}
font to display the printable characters.

A Metamath database consists of a sequence of three kinds of {\bf
tokens}\index{token} separated by {\bf white space}\index{white space}
(which is any sequence of one or more white space characters).  The set
of {\bf keyword}\index{keyword} tokens is \texttt{\$\char`\{},
\texttt{\$\char`\}}, \texttt{\$c}, \texttt{\$v}, \texttt{\$f},
\texttt{\$e}, \texttt{\$d}, \texttt{\$a}, \texttt{\$p}, \texttt{\$.},
\texttt{\$=}, \texttt{\$(}, \texttt{\$)}, \texttt{\$[}, and
\texttt{\$]}.  The last four are called {\bf auxiliary}\index{auxiliary
keyword} or preprocessing keywords.  A {\bf label}\index{label} token
consists of any combination of letters, digits, and the characters
hyphen, underscore, and period.  A {\bf math symbol}\index{math symbol}
token may consist of any combination of the 93 printable standard {\sc
ascii} characters other than space or \texttt{\$}~. All tokens are
case-sensitive.

\subsection{Preprocessing}

The token \texttt{\$(} begins a {\bf comment} and
\texttt{\$)} ends a comment.\index{\texttt{\$(}
and \texttt{\$)} auxiliary keywords}\index{comment}
Comments may contain any of
the 94 non-whitespace printable characters and white space,
except they may not contain the
2-character sequences \texttt{\$(} or \texttt{\$)} (comments do not nest).
Comments are ignored (treated
like white space) for the purpose of parsing, e.g.,
\texttt{\$( \$[ \$)} is a comment.
See p.~\pageref{mathcomments} for comment typesetting conventions; these
conventions may be ignored for the purpose of parsing.

A {\bf file inclusion command} consists of \texttt{\$[} followed by a file name
followed by \texttt{\$]}.\index{\texttt{\$[} and \texttt{\$]} auxiliary
keywords}\index{included file}\index{file inclusion}
It is only allowed in the outermost scope (i.e., not between
\texttt{\$\char`\{} and \texttt{\$\char`\}})
and must not be inside a statement (e.g., it may not occur
between the label of a \texttt{\$a} statement and its \texttt{\$.}).
The file name may not
contain a \texttt{\$} or white space.  The file must exist.
The case-sensitivity
of its name follows the conventions of the operating system.  The contents of
the file replace the inclusion command.
Included files may include other files.
Only the first reference to a given file is included; any later
references to the same file (whether in the top-level file or in included
files) cause the inclusion command to be ignored (treated like white space).
A verifier may assume that file names with different strings
refer to different files for the purpose of ignoring later references.
A file self-reference is ignored, as is any reference to the top-level file
(to avoid loops).
Included files may not include a \texttt{\$(} without a matching \texttt{\$)},
may not include a \texttt{\$[} without a matching \texttt{\$]}, and may
not include incomplete statements (e.g., a \texttt{\$a} without a matching
\texttt{\$.}).
It is currently unspecified if path references are relative to the process'
current directory or the file's containing directory, so databases should
avoid using pathname separators (e.g., ``/'') in file names.

Like all tokens, the \texttt{\$(}, \texttt{\$)}, \texttt{\$[}, and \texttt{\$]} keywords
must be surrounded by white space.

\subsection{Basic Syntax}

After preprocessing, a database will consist of a sequence of {\bf
statements}.
These are the scoping statements \texttt{\$\char`\{} and
\texttt{\$\char`\}}, along with the \texttt{\$c}, \texttt{\$v},
\texttt{\$f}, \texttt{\$e}, \texttt{\$d}, \texttt{\$a}, and \texttt{\$p}
statements.

A {\bf scoping statement}\index{scoping statement} consists only of its
keyword, \texttt{\$\char`\{} or \texttt{\$\char`\}}.
A \texttt{\$\char`\{} begins a {\bf
block}\index{block} and a matching \texttt{\$\char`\}} ends the block.
Every \texttt{\$\char`\{}
must have a matching \texttt{\$\char`\}}.
Defining it recursively, we say a block
contains a sequence of zero or more tokens other
than \texttt{\$\char`\{} and \texttt{\$\char`\}} and
possibly other blocks.  There is an {\bf outermost
block}\index{block!outermost} not bracketed by \texttt{\$\char`\{} \texttt{\$\char`\}}; the end
of the outermost block is the end of the database.

% LaTeX bug? can't do \bf\tt

A {\bf \$v} or {\bf \$c statement}\index{\texttt{\$v} statement}\index{\texttt{\$c}
statement} consists of the keyword token \texttt{\$v} or \texttt{\$c} respectively,
followed by one or more math symbols,
% The word "token" is used to distinguish "$." from the sentence-ending period.
followed by the \texttt{\$.}\ token.
These
statements {\bf declare}\index{declaration} the math symbols to be {\bf
variables}\index{variable!Metamath} or {\bf constants}\index{constant}
respectively. The same math symbol may not occur twice in a given \texttt{\$v} or
\texttt{\$c} statement.

%c%A math symbol becomes an {\bf active}\index{active math symbol}
%c%when declared and stays active until the end of the block in which it is
%c%declared.  A math symbol may not be declared a second time while it is active,
%c%but it may be declared again after it becomes inactive.

A math symbol becomes {\bf active}\index{active math symbol} when declared
and stays active until the end of the block in which it is declared.  A
variable may not be declared a second time while it is active, but it
may be declared again (as a variable, but not as a constant) after it
becomes inactive.  A constant must be declared in the outermost block and may
not be declared a second time.\index{redeclaration of symbols}

A {\bf \$f statement}\index{\texttt{\$f} statement} consists of a label,
followed by \texttt{\$f}, followed by its typecode (an active constant),
followed by an
active variable, followed by the \texttt{\$.}\ token.  A {\bf \$e
statement}\index{\texttt{\$e} statement} consists of a label, followed
by \texttt{\$e}, followed by its typecode (an active constant),
followed by zero or more
active math symbols, followed by the \texttt{\$.}\ token.  A {\bf
hypothesis}\index{hypothesis} is a \texttt{\$f} or \texttt{\$e}
statement.
The type declared by a \texttt{\$f} statement for a given label
is global even if the variable is not
(e.g., a database may not have \texttt{wff P} in one local scope
and \texttt{class P} in another).

A {\bf simple \$d statement}\index{\texttt{\$d} statement!simple}
consists of \texttt{\$d}, followed by two different active variables,
followed by the \texttt{\$.}\ token.  A {\bf compound \$d
statement}\index{\texttt{\$d} statement!compound} consists of
\texttt{\$d}, followed by three or more variables (all different),
followed by the \texttt{\$.}\ token.  The order of the variables in a
\texttt{\$d} statement is unimportant.  A compound \texttt{\$d}
statement is equivalent to a set of simple \texttt{\$d} statements, one
for each possible pair of variables occurring in the compound
\texttt{\$d} statement.  Henceforth in this specification we shall
assume all \texttt{\$d} statements are simple.  A \texttt{\$d} statement
is also called a {\bf disjoint} (or {\bf distinct}) {\bf variable
restriction}.\index{disjoint-variable restriction}

A {\bf \$a statement}\index{\texttt{\$a} statement} consists of a label,
followed by \texttt{\$a}, followed by its typecode (an active constant),
followed by
zero or more active math symbols, followed by the \texttt{\$.}\ token.  A {\bf
\$p statement}\index{\texttt{\$p} statement} consists of a label,
followed by \texttt{\$p}, followed by its typecode (an active constant),
followed by
zero or more active math symbols, followed by \texttt{\$=}, followed by
a sequence of labels, followed by the \texttt{\$.}\ token.  An {\bf
assertion}\index{assertion} is a \texttt{\$a} or \texttt{\$p} statement.

A \texttt{\$f}, \texttt{\$e}, or \texttt{\$d} statement is {\bf active}\index{active
statement} from the place it occurs until the end of the block it occurs in.
A \texttt{\$a} or \texttt{\$p} statement is {\bf active} from the place it occurs
through the end of the database.
There may not be two active \texttt{\$f} statements containing the same
variable.  Each variable in a \texttt{\$e}, \texttt{\$a}, or
\texttt{\$p} statement must exist in an active \texttt{\$f}
statement.\footnote{This requirement can greatly simplify the
unification algorithm (substitution calculation) required by proof
verification.}

%The label that begins each \texttt{\$f}, \texttt{\$e}, \texttt{\$a}, and
%\texttt{\$p} statement must be unique.
Each label token must be unique, and
no label token may match any math symbol
token.\label{namespace}\footnote{This
restriction was added on June 24, 2006.
It is not theoretically necessary but is imposed to make it easier to
write certain parsers.}

The set of {\bf mandatory variables}\index{mandatory variable} associated with
an assertion is the set of (zero or more) variables in the assertion and in any
active \texttt{\$e} statements.  The (possibly empty) set of {\bf mandatory
hypotheses}\index{mandatory hypothesis} is the set of all active \texttt{\$f}
statements containing mandatory variables, together with all active \texttt{\$e}
statements.
The set of {\bf mandatory {\bf \$d} statements}\index{mandatory
disjoint-variable restriction} associated with an assertion are those active
\texttt{\$d} statements whose variables are both among the assertion's
mandatory variables.

\subsection{Proof Verification}\label{spec4}

The sequence of labels between the \texttt{\$=} and \texttt{\$.}\ tokens
in a \texttt{\$p} statement is a {\bf proof}.\index{proof!Metamath} Each
label in a proof must be the label of an active statement other than the
\texttt{\$p} statement itself; thus a label must refer either to an
active hypothesis of the \texttt{\$p} statement or to an earlier
assertion.

An {\bf expression}\index{expression} is a sequence of math symbols. A {\bf
substitution map}\index{substitution map} associates a set of variables with a
set of expressions.  It is acceptable for a variable to be mapped to an
expression containing it.  A {\bf
substitution}\index{substitution!variable}\index{variable substitution} is the
simultaneous replacement of all variables in one or more expressions with the
expressions that the variables map to.

A proof is scanned in order of its label sequence.  If the label refers to an
active hypothesis, the expression in the hypothesis is pushed onto a
stack.\index{stack}\index{RPN stack}  If the label refers to an assertion, a
(unique) substitution must exist that, when made to the mandatory hypotheses
of the referenced assertion, causes them to match the topmost (i.e.\ most
recent) entries of the stack, in order of occurrence of the mandatory
hypotheses, with the topmost stack entry matching the last mandatory
hypothesis of the referenced assertion.  As many stack entries as there are
mandatory hypotheses are then popped from the stack.  The same substitution is
made to the referenced assertion, and the result is pushed onto the stack.
After the last label in the proof is processed, the stack must have a single
entry that matches the expression in the \texttt{\$p} statement containing the
proof.

%c%{\footnotesize\begin{quotation}\index{redeclaration of symbols}
%c%{{\em Comment.}\label{spec4comment} Whenever a math symbol token occurs in a
%c%{\texttt{\$c} or \texttt{\$v} statement, it is considered to designate a distinct new
%c%{symbol, even if the same token was previously declared (and is now inactive).
%c%{Thus a math token declared as a constant in two different blocks is considered
%c%{to designate two distinct constants (even though they have the same name).
%c%{The two constants will not match in a proof that references both blocks.
%c%{However, a proof referencing both blocks is acceptable as long as it doesn't
%c%{require that the constants match.  Similarly, a token declared to be a
%c%{constant for a referenced assertion will not match the same token declared to
%c%{be a variable for the \texttt{\$p} statement containing the proof.  In the case
%c%{of a token declared to be a variable for a referenced assertion, this is not
%c%{an issue since the variable can be substituted with whatever expression is
%c%{needed to achieve the required match.
%c%{\end{quotation}}
%c2%A proof may reference an assertion that contains or whose hypotheses contain a
%c2%constant that is not active for the \texttt{\$p} statement containing the proof.
%c2%However, the final result of the proof may not contain that constant. A proof
%c2%may also reference an assertion that contains or whose hypotheses contain a
%c2%variable that is not active for the \texttt{\$p} statement containing the proof.
%c2%That variable, of course, will be substituted with whatever expression is
%c2%needed to achieve the required match.

A proof may contain a \texttt{?}\ in place of a label to indicate an unknown step
(Section~\ref{unknown}).  A proof verifier may ignore any proof containing
\texttt{?}\ but should warn the user that the proof is incomplete.

A {\bf compressed proof}\index{compressed proof}\index{proof!compressed} is an
alternate proof notation described in Appen\-dix~\ref{compressed}; also see
references to ``compressed proof'' in the Index.  Compressed proofs are a
Metamath language extension which a complete proof verifier should be able to
parse and verify.

\subsubsection{Verifying Disjoint Variable Restrictions}

Each substitution made in a proof must be checked to verify that any
disjoint variable restrictions are satisfied, as follows.

If two variables replaced by a substitution exist in a mandatory \texttt{\$d}
statement\index{\texttt{\$d} statement} of the assertion referenced, the two
expressions resulting from the substitution must satisfy the following
conditions.  First, the two expressions must have no variables in common.
Second, each possible pair of variables, one from each expression, must exist
in an active \texttt{\$d} statement of the \texttt{\$p} statement containing the
proof.

\vskip 1ex

This ends the specification of the Metamath language;
see Appendix \ref{BNF} for its syntax in
Extended Backus--Naur Form (EBNF)\index{Extended Backus--Naur Form}\index{EBNF}.

\section{The Basic Keywords}\label{tut1}

Our expository material begins here.

Like most computer languages, Metamath\index{Metamath} takes its input from
one or more {\bf source files}\index{source file} which contain characters
expressed in the standard {\sc ascii} (American Standard Code for Information
Interchange)\index{ascii@{\sc ascii}} code for computers.  A source file
consists of a series of {\bf tokens}\index{token}, which are strings of
non-whitespace
printable characters (from the set of 94 shown on p.~\pageref{spec1chars})
separated by {\bf white space}\index{white space} (spaces, tabs, carriage
returns, line feeds, and form feeds). Any string consisting only of these
characters is treated the same as a single space.  The non-whitespace printable
characters\index{printable character} that Metamath recognizes are the 94
characters on standard {\sc ascii} keyboards.

Metamath has the ability to join several files together to form its
input (Section~\ref{include}).  We call the aggregate contents of all
the files after they have been joined together a {\bf
database}\index{database} to distinguish it from an individual source
file.  The tokens in a database consist of {\bf
keywords}\index{keyword}, which are built into the language, together
with two kinds of user-defined tokens called {\bf labels}\index{label}
and {\bf math symbols}\index{math symbol}.  (Often we will simply say
{\bf symbol}\index{symbol} instead of math symbol for brevity).  The set
of {\bf basic keywords}\index{basic keyword} is
\texttt{\$c}\index{\texttt{\$c} statement},
\texttt{\$v}\index{\texttt{\$v} statement},
\texttt{\$e}\index{\texttt{\$e} statement},
\texttt{\$f}\index{\texttt{\$f} statement},
\texttt{\$d}\index{\texttt{\$d} statement},
\texttt{\$a}\index{\texttt{\$a} statement},
\texttt{\$p}\index{\texttt{\$p} statement},
\texttt{\$=}\index{\texttt{\$=} keyword},
\texttt{\$.}\index{\texttt{\$.}\ keyword},
\texttt{\$\char`\{}\index{\texttt{\$\char`\{} and \texttt{\$\char`\}}
keywords}, and \texttt{\$\char`\}}.  This is the complete set of
syntactical elements of what we call the {\bf basic
language}\index{basic language} of Metamath, and with them you can
express all of the mathematics that were intended by the design of
Metamath.  You should make it a point to become very familiar with them.
Table~\ref{basickeywords} lists the basic keywords along with a brief
description of their functions.  For now, this description will give you
only a vague notion of what the keywords are for; later we will describe
the keywords in detail.


\begin{table}[htp] \caption{Summary of the basic Metamath
keywords} \label{basickeywords}
\begin{center}
\begin{tabular}{|p{4pc}|l|}
\hline
\em \centering Keyword&\em Description\\
\hline
\hline
\centering
   \texttt{\$c}&Constant symbol declaration\\
\hline
\centering
   \texttt{\$v}&Variable symbol declaration\\
\hline
\centering
   \texttt{\$d}&Disjoint variable restriction\\
\hline
\centering
   \texttt{\$f}&Variable-type (``floating'') hypothesis\\
\hline
\centering
   \texttt{\$e}&Logical (``essential'') hypothesis\\
\hline
\centering
   \texttt{\$a}&Axiomatic assertion\\
\hline
\centering
   \texttt{\$p}&Provable assertion\\
\hline
\centering
   \texttt{\$=}&Start of proof in \texttt{\$p} statement\\
\hline
\centering
   \texttt{\$.}&End of the above statement types\\
\hline
\centering
   \texttt{\$\char`\{}&Start of block\\
\hline
\centering
   \texttt{\$\char`\}}&End of block\\
\hline
\end{tabular}
\end{center}
\end{table}

%For LaTeX bug(?) where it puts tables on blank page instead of btwn text
%May have to adjust if text changes
%\newpage

There are some additional keywords, called {\bf auxiliary
keywords}\index{auxiliary keyword} that help make Metamath\index{Metamath}
more practical. These are part of the {\bf extended language}\index{extended
language}. They provide you with a means to put comments into a Metamath
source file\index{source file} and reference other source files.  We will
introduce these in later sections. Table~\ref{otherkeywords} summarizes them
so that you can recognize them now if you want to peruse some source
files while learning the basic keywords.


\begin{table}[htp] \caption{Auxiliary Metamath
keywords} \label{otherkeywords}
\begin{center}
\begin{tabular}{|p{4pc}|l|}
\hline
\em \centering Keyword&\em Description\\
\hline
\hline
\centering
   \texttt{\$(}&Start of comment\\
\hline
\centering
   \texttt{\$)}&End of comment\\
\hline
\centering
   \texttt{\$[}&Start of included source file name\\
\hline
\centering
   \texttt{\$]}&End of included source file name\\
\hline
\end{tabular}
\end{center}
\end{table}
\index{\texttt{\$(} and \texttt{\$)} auxiliary keywords}
\index{\texttt{\$[} and \texttt{\$]} auxiliary keywords}


Unlike those in some computer languages, the keywords\index{keyword} are short
two-character sequences rather than English-like words.  While this may make
them slightly more difficult to remember at first, their brevity allows
them to blend in with the mathematics being described, not
distract from it, like punctuation marks.


\subsection{User-Defined Tokens}\label{dollardollar}\index{token}

As you may have noticed, all keywords\index{keyword} begin with the \texttt{\$}
character.  This mundane monetary symbol is not ordinarily used in higher
mathematics (outside of grant proposals), so we have appropriated it to
distinguish the Metamath\index{Metamath} keywords from ordinary mathematical
symbols. The \texttt{\$} character is thus considered special and may not be
used as a character in a user-defined token.  All tokens and keywords are
case-sensitive; for example, \texttt{n} is considered to be a different character
from \texttt{N}.  Case-sensitivity makes the available {\sc ascii} character set
as rich as possible.

\subsubsection{Math Symbol Tokens}\index{token}

Math symbols\index{math symbol} are tokens used to represent the symbols
that appear in ordinary mathematical formulas.  They may consist of any
combination of the 93 non-whitespace printable {\sc ascii} characters other than
\texttt{\$}~. Some examples are \texttt{x}, \texttt{+}, \texttt{(},
\texttt{|-}, \verb$!%@?&$, and \texttt{bounded}.  For readability, it is
best to try to make these look as similar to actual mathematical symbols
as possible, within the constraints of the {\sc ascii} character set, in
order to make the resulting mathematical expressions more readable.

In the Metamath\index{Metamath} language, you express ordinary
mathematical formulas and statements as sequences of math symbols such
as \texttt{2 + 2 = 4} (five symbols, all constants).\footnote{To
eliminate ambiguity with other expressions, this is expressed in the set
theory database \texttt{set.mm} as \texttt{|- ( 2 + 2
 ) = 4 }, whose \LaTeX\ equivalent is $\vdash
(2+2)=4$.  The \,$\vdash$ means ``is a theorem'' and the
parentheses allow explicit associative grouping.}\index{turnstile
({$\,\vdash$})} They may even be English
sentences, as in \texttt{E is closed and bounded} (five symbols)---here
\texttt{E} would be a variable and the other four symbols constants.  In
principle, a Metamath database could be constructed to work with almost
any unambiguous English-language mathematical statement, but as a
practical matter the definitions needed to provide for all possible
syntax variations would be cumbersome and distracting and possibly have
subtle pitfalls accidentally built in.  We generally recommend that you
express mathematical statements with compact standard mathematical
symbols whenever possible and put their English-language descriptions in
comments.  Axioms\index{axiom} and definitions\index{definition}
(\texttt{\$a}\index{\texttt{\$a} statement} statements) are the only
places where Metamath will not detect an error, and doing this will help
reduce the number of definitions needed.

You are free to use any tokens\index{token} you like for math
symbols\index{math symbol}.  Appendix~\ref{ASCII} recommends token names to
use for symbols in set theory, and we suggest you adopt these in order to be
able to include the \texttt{set.mm} set theory database in your database.  For
printouts, you can convert the tokens in a database
to standard mathematical symbols with the \LaTeX\ typesetting program.  The
Metamath command \texttt{open tex} {\em filename}\index{\texttt{open tex} command}
produces output that can be read by \LaTeX.\index{latex@{\LaTeX}}
The correspondence
between tokens and the actual symbols is made by \texttt{latexdef}
statements inside a special database comment tagged
with \texttt{\$t}.\index{\texttt{\$t} comment}\index{typesetting comment}
  You can edit
this comment to change the definitions or add new ones.
Appendix~\ref{ASCII} describes how to do this in more detail.

% White space\index{white space} is normally used to separate math
% symbol\index{math symbol} tokens, but they may be juxtaposed without white
% space in \texttt{\$d}\index{\texttt{\$d} statement}, \texttt{\$e}\index{\texttt{\$e}
% statement}, \texttt{\$f}\index{\texttt{\$f} statement}, \texttt{\$a}\index{\texttt{\$a}
% statement}, and \texttt{\$p}\index{\texttt{\$p} statement} statements when no
% ambiguity will result.  Specifically, Metamath parses the math symbol sequence
% in one of these statements in the following manner:  when the math symbol
% sequence has been broken up into tokens\index{token} up to a given character,
% the next token is the longest string of characters that could constitute a
% math symbol that is active\index{active
% math symbol} at that point.  (See Section~\ref{scoping} for the
% definition of an active math symbol.)  For example, if \texttt{-}, \texttt{>}, and
% \texttt{->} are the only active math symbols, the juxtaposition \texttt{>-} will be
% interpreted as the two symbols \texttt{>} and \texttt{-}, whereas \texttt{->} will
% always be interpreted as that single symbol.\footnote{For better readability we
% recommend a white space between each token.  This also makes searching for a
% symbol easier to do with an editor.  Omission of optional white space is useful
% for reducing typing when assigning an expression to a temporary
% variable\index{temporary variable} with the \texttt{let variable} Metamath
% program command.}\index{\texttt{let variable} command}
%
% Keywords\index{keyword} may be placed next to math symbols without white
% space\index{white space} between them.\footnote{Again, we do not recommend
% this for readability.}
%
% The math symbols\index{math symbol} in \texttt{\$c}\index{\texttt{\$c} statement}
% and \texttt{\$v}\index{\texttt{\$v} statement} statements must always be separated
% by white space\index{white
% space}, for the obvious reason that these statements define the names
% of the symbols.
%
% Math symbols referred to in comments (see Section~\ref{comments}) must also be
% separated by white space.  This allows you to make comments about symbols that
% are not yet active\index{active
% math symbol}.  (The ``math mode'' feature of comments is also a quick and
% easy way to obtain word processing text with embedded mathematical symbols,
% independently of the main purpose of Metamath; the way to do this is described
% in Section~\ref{comments})

\subsubsection{Label Tokens}\index{token}\index{label}

Label tokens are used to identify Metamath\index{Metamath} statements for
later reference. Label tokens may contain only letters, digits, and the three
characters period, hyphen, and underscore:
\begin{verbatim}
. - _
\end{verbatim}

A label is {\bf declared}\index{label declaration} by placing it immediately
before the keyword of the statement it identifies.  For example, the label
\texttt{axiom.1} might be declared as follows:
\begin{verbatim}
axiom.1 $a |- x = x $.
\end{verbatim}

Each \texttt{\$e}\index{\texttt{\$e} statement},
\texttt{\$f}\index{\texttt{\$f} statement},
\texttt{\$a}\index{\texttt{\$a} statement}, and
\texttt{\$p}\index{\texttt{\$p} statement} statement in a database must
have a label declared for it.  No other statement types may have label
declarations.  Every label must be unique.

A label (and the statement it identifies) is {\bf referenced}\index{label
reference} by including the label between the \texttt{\$=}\index{\texttt{\$=}
keyword} and \texttt{\$.}\index{\texttt{\$.}\ keyword}\ keywords in a \texttt{\$p}
statement.  The sequence of labels\index{label sequence} between these two
keywords is called a {\bf proof}\index{proof}.  An example of a statement with
a proof that we will encounter later (Section~\ref{proof}) is
\begin{verbatim}
wnew $p wff ( s -> ( r -> p ) )
     $= ws wr wp w2 w2 $.
\end{verbatim}

You don't have to know what this means just yet, but you should know that the
label \texttt{wnew} is declared by this \texttt{\$p} statement and that the labels
\texttt{ws}, \texttt{wr}, \texttt{wp}, and \texttt{w2} are assumed to have been declared
earlier in the database and are referenced here.

\subsection{Constants and Variables}
\index{constant}
\index{variable}

An {\bf expression}\index{expression} is any sequence of math
symbols, possibly empty.

The basic Metamath\index{Metamath} language\index{basic language} has two
kinds of math symbols\index{math symbol}:  {\bf constants}\index{constant} and
{\bf variables}\index{variable}.  In a Metamath proof, a constant may not be
substituted with any expression.  A variable can be
substituted\index{substitution!variable}\index{variable substitution} with any
expression.  This sequence may include other variables and may even include
the variable being substituted.  This substitution takes place when proofs are
verified, and it will be described in Section~\ref{proof}.  The \texttt{\$f}
statement (described later in Section~\ref{dollaref}) is used to specify the
{\bf type} of a variable (i.e.\ what kind of
variable it is)\index{variable type}\index{type} and
give it a meaning typically
associated with a ``metavariable''\index{metavariable}\footnote{A metavariable
is a variable that ranges over the syntactical elements of the object language
being discussed; for example, one metavariable might represent a variable of
the object language and another metavariable might represent a formula in the
object language.} in ordinary mathematics; for example, a variable may be
specified to be a wff or well-formed formula (in logic), a set (in set
theory), or a non-negative integer (in number theory).

%\subsection{The \texttt{\$c} and \texttt{\$v} Declaration Statements}
\subsection{The \texttt{\$c} and \texttt{\$v} Declaration Statements}
\index{\texttt{\$c} statement}
\index{constant declaration}
\index{\texttt{\$v} statement}
\index{variable declaration}

Constants are introduced or {\bf declared}\index{constant declaration}
with \texttt{\$c}\index{\texttt{\$c} statement} statements, and
variables are declared\index{variable declaration} with
\texttt{\$v}\index{\texttt{\$v} statement} statements.  A {\bf simple}
declaration\index{simple declaration} statement introduces a single
constant or variable.  Its syntax is one of the following:
\begin{center}
  \texttt{\$c} {\em math-symbol} \texttt{\$.}\\
  \texttt{\$v} {\em math-symbol} \texttt{\$.}
\end{center}
The notation {\em math-symbol} means any math symbol token\index{token}.

Some examples of simple declaration statements are:
\begin{center}
  \texttt{\$c + \$.}\\
  \texttt{\$c -> \$.}\\
  \texttt{\$c ( \$.}\\
  \texttt{\$v x \$.}\\
  \texttt{\$v y2 \$.}
\end{center}

The characters in a math symbol\index{math symbol} being declared are
irrelevant to Meta\-math; for example, we could declare a right parenthesis to
be a variable,
\begin{center}
  \texttt{\$v ) \$.}\\
\end{center}
although this would be unconventional.

A {\bf compound} declaration\index{compound declaration} statement is a
shorthand for declaring several symbols at once.  Its syntax is one of the
following:
\begin{center}
  \texttt{\$c} {\em math-symbol}\ \,$\cdots$\ {\em math-symbol} \texttt{\$.}\\
  \texttt{\$v} {\em math-symbol}\ \,$\cdots$\ {\em math-symbol} \texttt{\$.}
\end{center}\index{\texttt{\$c} statement}
Here, the ellipsis (\ldots) means any number of {\em math-symbol}\,s.

An example of a compound declaration statement is:
\begin{center}
  \texttt{\$v x y mu \$.}\\
\end{center}
This is equivalent to the three simple declaration statements
\begin{center}
  \texttt{\$v x \$.}\\
  \texttt{\$v y \$.}\\
  \texttt{\$v mu \$.}\\
\end{center}
\index{\texttt{\$v} statement}

There are certain rules on where in the database math symbols may be declared,
what sections of the database are aware of them (i.e.\ where they are
``active''), and when they may be declared more than once.  These will be
discussed in Section~\ref{scoping} and specifically on
p.~\pageref{redeclaration}.

\subsection{The \texttt{\$d} Statement}\label{dollard}
\index{\texttt{\$d} statement}

The \texttt{\$d} statement is called a {\bf disjoint-variable restriction}.  The
syntax of the {\bf simple} version of this statement is
\begin{center}
  \texttt{\$d} {\em variable variable} \texttt{\$.}
\end{center}
where each {\em variable} is a previously declared variable and the two {\em
variable}\,s are different.  (More specifically, each  {\em variable} must be
an {\bf active} variable\index{active math symbol}, which means there must be
a previous \texttt{\$v} statement whose {\bf scope}\index{scope} includes the
\texttt{\$d} statement.  These terms will be defined when we discuss scoping
statements in Section~\ref{scoping}.)

In ordinary mathematics, formulas may arise that are true if the variables in
them are distinct\index{distinct variables}, but become false when those
variables are made identical. For example, the formula in logic $\exists x\,x
\neq y$, which means ``for a given $y$, there exists an $x$ that is not equal
to $y$,'' is true in most mathematical theories (namely all non-trivial
theories\index{non-trivial theory}, i.e.\ those that describe more than one
individual, such as arithmetic).  However, if we substitute $y$ with $x$, we
obtain $\exists x\,x \neq x$, which is always false, as it means ``there
exists something that is not equal to itself.''\footnote{If you are a
logician, you will recognize this as the improper substitution\index{proper
substitution}\index{substitution!proper} of a free variable\index{free
variable} with a bound variable\index{bound variable}.  Metamath makes no
inherent distinction between free and bound variables; instead, you let
Metamath know what substitutions are permissible by using \texttt{\$d} statements
in the right way in your axiom system.}\index{free vs.\ bound variable}  The
\texttt{\$d} statement allows you to specify a restriction that forbids the
substitution of one variable with another.  In
this case, we would use the statement
\begin{center}
  \texttt{\$d x y \$.}
\end{center}\index{\texttt{\$d} statement}
to specify this restriction.

The order in which the variables appear in a \texttt{\$d} statement is not
important.  We could also use
\begin{center}
  \texttt{\$d y x \$.}
\end{center}

The \texttt{\$d} statement is actually more general than this, as the
``disjoint''\index{disjoint variables} in its name suggests.  The full meaning
is that if any substitution is made to its two variables (during the
course of a proof that references a \texttt{\$a} or \texttt{\$p} statement
associated with the \texttt{\$d}), the two expressions that result from the
substitution must have no variables in common.  In addition, each possible
pair of variables, one from each expression, must be in a \texttt{\$d} statement
associated with the statement being proved.  (This requirement forces the
statement being proved to ``inherit'' the original disjoint variable
restriction.)

For example, suppose \texttt{u} is a variable.  If the restriction
\begin{center}
  \texttt{\$d A B \$.}
\end{center}
has been specified for a theorem referenced in a
proof, we may not substitute \texttt{A} with \mbox{\tt a + u} and
\texttt{B} with \mbox{\tt b + u} because these two symbol sequences have the
variable \texttt{u} in common.  Furthermore, if \texttt{a} and \texttt{b} are
variables, we may not substitute \texttt{A} with \texttt{a} and \texttt{B} with \texttt{b}
unless we have also specified \texttt{\$d a b} for the theorem being proved; in
other words, the \texttt{\$d} property associated with a pair of variables must
be effectively preserved after substitution.

The \texttt{\$d}\index{\texttt{\$d} statement} statement does {\em not} mean ``the
two variables may not be substituted with the same thing,'' as you might think
at first.  For example, substituting each of \texttt{A} and \texttt{B} in the above
example with identical symbol sequences consisting only of constants does not
cause a disjoint variable conflict, because two symbol sequences have no
variables in common (since they have no variables, period).  Similarly, a
conflict will not occur by substituting the two variables in a \texttt{\$d}
statement with the empty symbol sequence\index{empty substitution}.

The \texttt{\$d} statement does not have a direct counterpart in
ordinary mathematics, partly because the variables\index{variable} of
Metamath are not really the same as the variables\index{variable!in
ordinary mathematics} of ordinary mathematics but rather are
metavariables\index{metavariable} ranging over them (as well as over
other kinds of symbols and groups of symbols).  Depending on the
situation, we may informally interpret the \texttt{\$d} statement in
different ways.  Suppose, for example, that \texttt{x} and \texttt{y}
are variables ranging over numbers (more precisely, that \texttt{x} and
\texttt{y} are metavariables ranging over variables that range over
numbers), and that \texttt{ph} ($\varphi$) and \texttt{ps} ($\psi$) are
variables (more precisely, metavariables) ranging over formulas.  We can
make the following interpretations that correspond to the informal
language of ordinary mathematics:
\begin{quote}
\begin{tabbing}
\texttt{\$d x y \$.} means ``assume $x$ and $y$ are
distinct variables.''\\
\texttt{\$d x ph \$.} means ``assume $x$ does not
occur in $\varphi$.''\\
\texttt{\$d ph ps \$.} \=means ``assume $\varphi$ and
$\psi$ have no variables\\ \>in common.''
\end{tabbing}
\end{quote}\index{\texttt{\$d} statement}

\subsubsection{Compound \texttt{\$d} Statements}

The {\bf compound} version of the \texttt{\$d} statement is a shorthand for
specifying several variables whose substitutions must be pairwise disjoint.
Its syntax is:
\begin{center}
  \texttt{\$d} {\em variable}\ \,$\cdots$\ {\em variable} \texttt{\$.}
\end{center}\index{\texttt{\$d} statement}
Here, {\em variable} represents the token of a previously declared
variable (specifically, an active variable) and all {\em variable}\,s are
different.  The compound \texttt{\$d}
statement is internally broken up by Metamath into one simple \texttt{\$d}
statement for each possible pair of variables in the original \texttt{\$d}
statement.  For example,
\begin{center}
  \texttt{\$d w x y z \$.}
\end{center}
is equivalent to
\begin{center}
  \texttt{\$d w x \$.}\\
  \texttt{\$d w y \$.}\\
  \texttt{\$d w z \$.}\\
  \texttt{\$d x y \$.}\\
  \texttt{\$d x z \$.}\\
  \texttt{\$d y z \$.}
\end{center}

Two or more simple \texttt{\$d} statements specifying the same variable pair are
internally combined into a single \texttt{\$d} statement.  Thus the set of three
statements
\begin{center}
  \texttt{\$d x y \$.}
  \texttt{\$d x y \$.}
  \texttt{\$d y x \$.}
\end{center}
is equivalent to
\begin{center}
  \texttt{\$d x y \$.}
\end{center}

Similarly, compound \texttt{\$d} statements, after being internally broken up,
internally have their common variable pairs combined.  For example the
set of statements
\begin{center}
  \texttt{\$d x y A \$.}
  \texttt{\$d x y B \$.}
\end{center}
is equivalent to
\begin{center}
  \texttt{\$d x y \$.}
  \texttt{\$d x A \$.}
  \texttt{\$d y A \$.}
  \texttt{\$d x y \$.}
  \texttt{\$d x B \$.}
  \texttt{\$d y B \$.}
\end{center}
which is equivalent to
\begin{center}
  \texttt{\$d x y \$.}
  \texttt{\$d x A \$.}
  \texttt{\$d y A \$.}
  \texttt{\$d x B \$.}
  \texttt{\$d y B \$.}
\end{center}

Metamath\index{Metamath} automatically verifies that all \texttt{\$d}
restrictions are met whenever it verifies proofs.  \texttt{\$d} statements are
never referenced directly in proofs (this is why they do not have
labels\index{label}), but Metamath is always aware of which ones must be
satisfied (i.e.\ are active) and will notify you with an error message if any
violation occurs.

To illustrate how Metamath detects a missing \texttt{\$d}
statement, we will look at the following example from the
\texttt{set.mm} database.

\begin{verbatim}
$d x z $.  $d y z $.
$( Theorem to add distinct quantifier to atomic formula. $)
ax17eq $p |- ( x = y -> A. z x = y ) $=...
\end{verbatim}

This statement has the obvious requirement that $z$ must be
distinct\index{distinct variables} from $x$ in theorem \texttt{ax17eq} that
states $x=y \rightarrow \forall z \, x=y$ (well, obvious if you're a logician,
for otherwise we could conclude  $x=y \rightarrow \forall x \, x=y$, which is
false when the free variables $x$ and $y$ are equal).

Let's look at what happens if we edit the database to comment out this
requirement.

\begin{verbatim}
$( $d x z $. $) $d y z $.
$( Theorem to add distinct quantifier to atomic formula. $)
ax17eq $p |- ( x = y -> A. z x = y ) $=...
\end{verbatim}

When it tries to verify the proof, Metamath will tell you that \texttt{x} and
\texttt{z} must be disjoint, because one of its steps references an axiom or
theorem that has this requirement.

\begin{verbatim}
MM> verify proof ax17eq
ax17eq ?Error at statement 1918, label "ax17eq", type "$p":
      vz wal wi vx vy vz ax-13 vx vy weq vz vx ax-c16 vx vy
                                               ^^^^^
There is a disjoint variable ($d) violation at proof step 29.
Assertion "ax-c16" requires that variables "x" and "y" be
disjoint.  But "x" was substituted with "z" and "y" was
substituted with "x".  The assertion being proved, "ax17eq",
does not require that variables "z" and "x" be disjoint.
\end{verbatim}

We can see the substitutions into \texttt{ax-c16} with the following command.

\begin{verbatim}
MM> show proof ax17eq / detailed_step 29
Proof step 29:  pm2.61dd.2=ax-c16 $a |- ( A. z z = x -> ( x =
  y -> A. z x = y ) )
This step assigns source "ax-c16" ($a) to target "pm2.61dd.2"
($e).  The source assertion requires the hypotheses "wph"
($f, step 26), "vx" ($f, step 27), and "vy" ($f, step 28).
The parent assertion of the target hypothesis is "pm2.61dd"
($p, step 36).
The source assertion before substitution was:
    ax-c16 $a |- ( A. x x = y -> ( ph -> A. x ph ) )
The following substitutions were made to the source
assertion:
    Variable  Substituted with
     x         z
     y         x
     ph        x = y
The target hypothesis before substitution was:
    pm2.61dd.2 $e |- ( ph -> ch )
The following substitutions were made to the target
hypothesis:
    Variable  Substituted with
     ph        A. z z = x
     ch        ( x = y -> A. z x = y )
\end{verbatim}

The disjoint variable restrictions of \texttt{ax-c16} can be seen from the
\texttt{show state\-ment} command.  The line that begins ``\texttt{Its mandatory
dis\-joint var\-i\-able pairs are:}\ldots'' lists any \texttt{\$d} variable
pairs in brackets.

\begin{verbatim}
MM> show statement ax-c16/full
Statement 3033 is located on line 9338 of the file "set.mm".
"Axiom of Distinct Variables. ..."
  ax-c16 $a |- ( A. x x = y -> ( ph -> A. x ph ) ) $.
Its mandatory hypotheses in RPN order are:
  wph $f wff ph $.
  vx $f setvar x $.
  vy $f setvar y $.
Its mandatory disjoint variable pairs are:  <x,y>
The statement and its hypotheses require the variables:  x y
      ph
The variables it contains are:  x y ph
\end{verbatim}

Since Metamath will always detect when \texttt{\$d}\index{\texttt{\$d} statement}
statements are needed for a proof, you don't have to worry too much about
forgetting to put one in; it can always be added if you see the error message
above.  If you put in unnecessary \texttt{\$d} statements, the worst that could
happen is that your theorem might not be as general as it could be, and this
may limit its use later on.

On the other hand, when you introduce axioms (\texttt{\$a}\index{\texttt{\$a}
statement} statements), you must be very careful to properly specify the
necessary associated \texttt{\$d} statements since Metamath has no way of knowing
whether your axioms are correct.  For example, Metamath would have no idea
that \texttt{ax-c16}, which we are telling it is an axiom of logic, would lead to
contradictions if we omitted its associated \texttt{\$d} statement.

% This was previously a comment in footnote-sized type, but it can be
% hard to read this much text in a small size.
% As a result, it's been changed to normally-sized text.
\label{nodd}
You may wonder if it is possible to develop standard
mathematics in the Metamath language without the \texttt{\$d}\index{\texttt{\$d}
statement} statement, since it seems like a nuisance that complicates proof
verification. The \texttt{\$d} statement is not needed in certain subsets of
mathematics such as propositional calculus.  However, dummy
variables\index{dummy variable!eliminating} and their associated \texttt{\$d}
statements are impossible to avoid in proofs in standard first-order logic as
well as in the variant used in \texttt{set.mm}.  In fact, there is no upper bound to
the number of dummy variables that might be needed in a proof of a theorem of
first-order logic containing 3 or more variables, as shown by H.\
Andr\'{e}ka\index{Andr{\'{e}}ka, H.} \cite{Nemeti}.  A first-order system that
avoids them entirely is given in \cite{Megill}\index{Megill, Norman}; the
trick there is simply to embed harmlessly the necessary dummy variables into a
theorem being proved so that they aren't ``dummy'' anymore, then interpret the
resulting longer theorem so as to ignore the embedded dummy variables.  If
this interests you, the system in \texttt{set.mm} obtained from \texttt{ax-1}
through \texttt{ax-c14} in \texttt{set.mm}, and deleting \texttt{ax-c16} and \texttt{ax-5},
requires no \texttt{\$d} statements but is logically complete in the sense
described in \cite{Megill}.  This means it can prove any theorem of
first-order logic as long as we add to the theorem an antecedent that embeds
dummy and any other variables that must be distinct.  In a similar fashion,
axioms for set theory can be devised that
do not require distinct variable
provisos\index{Set theory without distinct variable provisos},
as explained at
\url{http://us.metamath.org/mpeuni/mmzfcnd.html}.
Together, these in principle allow all of
mathematics to be developed under Metamath without a \texttt{\$d} statement,
although the length of the resulting theorems will grow as more and
more dummy variables become required in their proofs.

\subsection{The \texttt{\$f}
and \texttt{\$e} Statements}\label{dollaref}
\index{\texttt{\$e} statement}
\index{\texttt{\$f} statement}
\index{floating hypothesis}
\index{essential hypothesis}
\index{variable-type hypothesis}
\index{logical hypothesis}
\index{hypothesis}

Metamath has two kinds of hypo\-theses, the \texttt{\$f}\index{\texttt{\$f}
statement} or {\bf variable-type} hypothesis and the \texttt{\$e} or {\bf logical}
hypo\-the\-sis.\index{\texttt{\$d} statement}\footnote{Strictly speaking, the
\texttt{\$d} statement is also a hypothesis, but it is never directly referenced
in a proof, so we call it a restriction rather than a hypothesis to lessen
confusion.  The checking for violations of \texttt{\$d} restrictions is automatic
and built into Metamath's proof-checking algorithm.} The letters \texttt{f} and
\texttt{e} stand for ``floating''\index{floating hypothesis} (roughly meaning
used only if relevant) and ``essential''\index{essential hypothesis} (meaning
always used) respectively, for reasons that will become apparent
when we discuss frames in
Section~\ref{frames} and scoping in Section~\ref{scoping}. The syntax of these
are as follows:
\begin{center}
  {\em label} \texttt{\$f} {\em typecode} {\em variable} \texttt{\$.}\\
  {\em label} \texttt{\$e} {\em typecode}
      {\em math-symbol}\ \,$\cdots$\ {\em math-symbol} \texttt{\$.}\\
\end{center}
\index{\texttt{\$e} statement}
\index{\texttt{\$f} statement}
A hypothesis must have a {\em label}\index{label}.  The expression in a
\texttt{\$e} hypothesis consists of a typecode (an active constant math symbol)
followed by a sequence
of zero or more math symbols. Each math symbol (including {\em constant}
and {\em variable}) must be a previously declared constant or variable.  (In
addition, each math symbol must be active, which will be covered when we
discuss scoping statements in Section~\ref{scoping}.)  You use a \texttt{\$f}
hypothesis to specify the
nature or {\bf type}\index{variable type}\index{type} of a variable (such as ``let $x$ be an
integer'') and use a \texttt{\$e} hypothesis to express a logical truth (such as
``assume $x$ is prime'') that must be established in order for an assertion
requiring it to also be true.

A variable must have its type specified in a \texttt{\$f} statement before
it may be used in a \texttt{\$e}, \texttt{\$a}, or \texttt{\$p}
statement.  There may be only one (active) \texttt{\$f} statement for a
given variable.  (``Active'' is defined in Section~\ref{scoping}.)

In ordinary mathematics, theorems\index{theorem} are often expressed in the
form ``Assume $P$; then $Q$,'' where $Q$ is a statement that you can derive
if you start with statement $P$.\index{free variable}\footnote{A stronger
version of a theorem like this would be the {\em single} formula $P\rightarrow
Q$ ($P$ implies $Q$) from which the weaker version above follows by the rule
of modus ponens in logic.  We are not discussing this stronger form here.  In
the weaker form, we are saying only that if we can {\em prove} $P$, then we can
{\em prove} $Q$.  In a logician's language, if $x$ is the only free variable
in $P$ and $Q$, the stronger form is equivalent to $\forall x ( P \rightarrow
Q)$ (for all $x$, $P$ implies $Q$), whereas the weaker form is equivalent to
$\forall x P \rightarrow \forall x Q$. The stronger form implies the weaker,
but not vice-versa.  To be precise, the weaker form of the theorem is more
properly called an ``inference'' rather than a theorem.}\index{inference}
In the
Metamath\index{Metamath} language, you would express mathematical statement
$P$ as a hypothesis (a \texttt{\$e} Metamath language statement in this case) and
statement $Q$ as a provable assertion (a \texttt{\$p}\index{\texttt{\$p} statement}
statement).

Some examples of hypotheses you might encounter in logic and set theory are
\begin{center}
  \texttt{stmt1 \$f wff P \$.}\\
  \texttt{stmt2 \$f setvar x \$.}\\
  \texttt{stmt3 \$e |- ( P -> Q ) \$.}
\end{center}
\index{\texttt{\$e} statement}
\index{\texttt{\$f} statement}
Informally, these would be read, ``Let $P$ be a well-formed-formula,'' ``Let
$x$ be an (individual) variable,'' and ``Assume we have proved $P \rightarrow
Q$.''  The turnstile symbol \,$\vdash$\index{turnstile ({$\,\vdash$})} is
commonly used in logic texts to mean ``a proof exists for.''

To summarize:
\begin{itemize}
\item A \texttt{\$f} hypothesis tells Metamath the type or kind of its variable.
It is analogous to a variable declaration in a computer language that
tells the compiler that a variable is an integer or a floating-point
number.
\item The \texttt{\$e} hypothesis corresponds to what you would usually call a
``hypothesis'' in ordinary mathematics.
\end{itemize}

Before an assertion\index{assertion} (\texttt{\$a} or \texttt{\$p} statement) can be
referenced in a proof, all of its associated \texttt{\$f} and \texttt{\$e} hypotheses
(i.e.\ those \texttt{\$e} hypotheses that are active) must be satisfied (i.e.
established by the proof).  The meaning of ``associated'' (which we will call
{\bf mandatory} in Section~\ref{frames}) will become clear when we discuss
scoping later.

Note that after any \texttt{\$f}, \texttt{\$e},
\texttt{\$a}, or \texttt{\$p} token there is a required
\textit{typecode}\index{typecode}.
The typecode is a constant used to enforce types of expressions.
This will become clearer once we learn more about
assertions (\texttt{\$a} and \texttt{\$p} statements).
An example may also clarify their purpose.
In the
\texttt{set.mm}\index{set theory database (\texttt{set.mm})}%
\index{Metamath Proof Explorer}
database,
the following typecodes are used:

\begin{itemize}
\item \texttt{wff} :
  Well-formed formula (wff) symbol
  (read: ``the following symbol sequence is a wff'').
% The *textual* typecode for turnstile is "|-", but when read it's a little
% confusing, so I intentionally display the mathematical symbol here instead
% (I think it's clearer in this context).
\item \texttt{$\vdash$} :
  Turnstile (read: ``the following symbol sequence is provable'' or
  ``a proof exists for'').
\item \texttt{setvar} :
  Individual set variable type (read: ``the following is an
  individual set variable'').
  Note that this is \textit{not} the type of an arbitrary set expression,
  instead, it is used to ensure that there is only a single symbol used
  after quantifiers like for-all ($\forall$) and there-exists ($\exists$).
\item \texttt{class} :
  An expression that is a syntactically valid class expression.
  All valid set expressions are also valid class expression, so expressions
  of sets normally have the \texttt{class} typecode.
  Use the \texttt{class} typecode,
  \textit{not} the \texttt{setvar} typecode,
  for the type of set expressions unless you are specifically identifying
  a single set variable.
\end{itemize}

\subsection{Assertions (\texttt{\$a} and \texttt{\$p} Statements)}
\index{\texttt{\$a} statement}
\index{\texttt{\$p} statement}\index{assertion}\index{axiomatic assertion}
\index{provable assertion}

There are two types of assertions, \texttt{\$a}\index{\texttt{\$a} statement}
statements ({\bf axiomatic assertions}) and \texttt{\$p} statements ({\bf
provable assertions}).  Their syntax is as follows:
\begin{center}
  {\em label} \texttt{\$a} {\em typecode} {\em math-symbol} \ldots
         {\em math-symbol} \texttt{\$.}\\
  {\em label} \texttt{\$p} {\em typecode} {\em math-symbol} \ldots
        {\em math-symbol} \texttt{\$=} {\em proof} \texttt{\$.}
\end{center}
\index{\texttt{\$a} statement}
\index{\texttt{\$p} statement}
\index{\texttt{\$=} keyword}
An assertion always requires a {\em label}\index{label}. The expression in an
assertion consists of a typecode (an active constant)
followed by a sequence of zero
or more math symbols.  Each math symbol, including any {\em constant}, must be a
previously declared constant or variable.  (In addition, each math symbol
must be active, which will be covered when we discuss scoping statements in
Section~\ref{scoping}.)

A \texttt{\$a} statement is usually a definition of syntax (for example, if $P$
and $Q$ are wffs then so is $(P\to Q)$), an axiom\index{axiom} of ordinary
mathematics (for example, $x=x$), or a definition\index{definition} of
ordinary mathematics (for example, $x\ne y$ means $\lnot x=y$). A \texttt{\$p}
statement is a claim that a certain combination of math symbols follows from
previous assertions and is accompanied by a proof that demonstrates it.

Assertions can also be referenced in (later) proofs in order to derive new
assertions from them. The label of an assertion is used to refer to it in a
proof. Section~\ref{proof} will describe the proof in detail.

Assertions also provide the primary means for communicating the mathematical
results in the database to people.  Proofs (when conveniently displayed)
communicate to people how the results were arrived at.

\subsubsection{The \texttt{\$a} Statement}
\index{\texttt{\$a} statement}

Axiomatic assertions (\texttt{\$a} statements) represent the starting points from
which other assertions (\texttt{\$p}\index{\texttt{\$p} statement} statements) are
derived.  Their most obvious use is for specifying ordinary mathematical
axioms\index{axiom}, but they are also used for two other purposes.

First, Metamath\index{Metamath} needs to know the syntax of symbol
sequences that constitute valid mathematical statements.  A Metamath
proof must be broken down into much more detail than ordinary
mathematical proofs that you may be used to thinking of (even the
``complete'' proofs of formal logic\index{formal logic}).  This is one
of the things that makes Metamath a general-purpose language,
independent of any system of logic or even syntax.  If you want to use a
substitution instance of an assertion as a step in a proof, you must
first prove that the substitution is syntactically correct (or if you
prefer, you must ``construct'' it), showing for example that the
expression you are substituting for a wff metavariable is a valid wff.
The \texttt{\$a}\index{\texttt{\$a} statement} statement is used to
specify those combinations of symbols that are considered syntactically
valid, such as the legal forms of wffs.

Second, \texttt{\$a} statements are used to specify what are ordinarily thought of
as definitions, i.e.\ new combinations of symbols that abbreviate other
combinations of symbols.  Metamath makes no distinction\index{axiom vs.\
definition} between axioms\index{axiom} and definitions\index{definition}.
Indeed, it has been argued that such distinction should not be made even in
ordinary mathematics; see Section~\ref{definitions}, which discusses the
philosophy of definitions.  Section~\ref{hierarchy} discusses some
technical requirements for definitions.  In \texttt{set.mm} we adopt the
convention of prefixing axiom labels with \texttt{ax-} and definition labels with
\texttt{df-}\index{label}.

The results that can be derived with the Metamath language are only as good as
the \texttt{\$a}\index{\texttt{\$a} statement} statements used as their starting
point.  We cannot stress this too strongly.  For example, Metamath will
not prevent you from specifying $x\neq x$ as an axiom of logic.  It is
essential that you scrutinize all \texttt{\$a} statements with great care.
Because they are a source of potential pitfalls, it is best not to add new
ones (usually new definitions) casually; rather you should carefully evaluate
each one's necessity and advantages.

Once you have in place all of the basic axioms\index{axiom} and
rules\index{rule} of a mathematical theory, the only \texttt{\$a} statements that
you will be adding will be what are ordinarily called definitions.  In
principle, definitions should be in some sense eliminable from the language of
a theory according to some convention (usually involving logical equivalence
or equality).  The most common convention is that any formula that was
syntactically valid but not provable before the definition was introduced will
not become provable after the definition is introduced.  In an ideal world,
definitions should not be present at all if one is to have absolute confidence
in a mathematical result.  However, they are necessary to make
mathematics practical, for otherwise the resulting formulas would be
extremely long and incomprehensible.  Since the nature of definitions (in the
most general sense) does not permit them to automatically be verified as
``proper,''\index{proper definition}\index{definition!proper} the judgment of
the mathematician is required to ensure it.  (In \texttt{set.mm} effort was made
to make almost all definitions directly eliminable and thus minimize the need
for such judgment.)

If you are not a mathematician, it may be best not to add or change any
\texttt{\$a}\index{\texttt{\$a} statement} statements but instead use
the mathematical language already provided in standard databases.  This
way Metamath will not allow you to make a mistake (i.e.\ prove a false
result).


\subsection{Frames}\label{frames}

We now introduce the concept of a collection of related Metamath statements
called a frame.  Every assertion (\texttt{\$a} or \texttt{\$p} statement) in the database has
an associated frame.

A {\bf frame}\index{frame} is a sequence of \texttt{\$d}, \texttt{\$f},
and \texttt{\$e} statements (zero or more of each) followed by one
\texttt{\$a} or \texttt{\$p} statement, subject to certain conditions we
will describe.  For simplicity we will assume that all math symbol
tokens used are declared at the beginning of the database with
\texttt{\$c} and \texttt{\$v} statements (which are not properly part of
a frame).  Also for simplicity we will assume there are only simple
\texttt{\$d} statements (those with only two variables) and imagine any
compound \texttt{\$d} statements (those with more than two variables) as
broken up into simple ones.

A frame groups together those hypotheses (and \texttt{\$d} statements) relevant
to an assertion (\texttt{\$a} or \texttt{\$p} statement).  The statements in a frame
may or may not be physically adjacent in a database; we will cover
this in our discussion of scoping statements
in Section~\ref{scoping}.

A frame has the following properties:
\begin{enumerate}
 \item The set of variables contained in its \texttt{\$f} statements must
be identical to the set of variables contained in its \texttt{\$e},
\texttt{\$a}, and/or \texttt{\$p} statements.  In other words, each
variable in a \texttt{\$e}, \texttt{\$a}, or \texttt{\$p} statement must
have an associated ``variable type'' defined for it in a \texttt{\$f}
statement.
  \item No two \texttt{\$f} statements may contain the same variable.
  \item Any \texttt{\$f} statement
must occur before a \texttt{\$e} statement in which its variable occurs.
\end{enumerate}

The first property determines the set of variables occurring in a frame.
These are the {\bf mandatory
variables}\index{mandatory variable} of the frame.  The second property
tells us there must be only one type specified for a variable.
The last property is not a theoretical requirement but it
makes parsing of the database easier.

For our examples, we assume our database has the following declarations:

\begin{verbatim}
$v P Q R $.
$c -> ( ) |- wff $.
\end{verbatim}

The following sequence of statements, describing the modus ponens inference
rule, is an example of a frame:

\begin{verbatim}
wp  $f wff P $.
wq  $f wff Q $.
maj $e |- ( P -> Q ) $.
min $e |- P $.
mp  $a |- Q $.
\end{verbatim}

The following sequence of statements is not a frame because \texttt{R} does not
occur in the \texttt{\$e}'s or the \texttt{\$a}:

\begin{verbatim}
wp  $f wff P $.
wq  $f wff Q $.
wr  $f wff R $.
maj $e |- ( P -> Q ) $.
min $e |- P $.
mp  $a |- Q $.
\end{verbatim}

The following sequence of statements is not a frame because \texttt{Q} does not
occur in a \texttt{\$f}:

\begin{verbatim}
wp  $f wff P $.
maj $e |- ( P -> Q ) $.
min $e |- P $.
mp  $a |- Q $.
\end{verbatim}

The following sequence of statements is not a frame because the \texttt{\$a} statement is
not the last one:

\begin{verbatim}
wp  $f wff P $.
wq  $f wff Q $.
maj $e |- ( P -> Q ) $.
mp  $a |- Q $.
min $e |- P $.
\end{verbatim}

Associated with a frame is a sequence of {\bf mandatory
hypotheses}\index{mandatory hypothesis}.  This is simply the set of all
\texttt{\$f} and \texttt{\$e} statements in the frame, in the order they
appear.  A frame can be referenced in a later proof using the label of
the \texttt{\$a} or \texttt{\$p} assertion statement, and the proof
makes an assignment to each mandatory hypothesis in the order in which
it appears.  This means the order of the hypotheses, once chosen, must
not be changed so as not to affect later proofs referencing the frame's
assertion statement.  (The Metamath proof verifier will, of course, flag
an error if a proof becomes incorrect by doing this.)  Since proofs make
use of ``Reverse Polish notation,'' described in Section~\ref{proof}, we
call this order the {\bf RPN order}\index{RPN order} of the hypotheses.

Note that \texttt{\$d} statements are not part of the set of mandatory
hypotheses, and their order doesn't matter (as long as they satisfy the
fourth property for a frame described above).  The \texttt{\$d}
statements specify restrictions on variables that must be satisfied (and
are checked by the proof verifier) when expressions are substituted for
them in a proof, and the \texttt{\$d} statements themselves are never
referenced directly in a proof.

A frame with a \texttt{\$p} (provable) statement requires a proof as part of the
\texttt{\$p} statement.  Sometimes in a proof we want to make use of temporary or
dummy variables\index{dummy variable} that do not occur in the \texttt{\$p}
statement or its mandatory hypotheses.  To accommodate this we define an {\bf
extended frame}\index{extended frame} as a frame together with zero or more
\texttt{\$d} and \texttt{\$f} statements that reference variables not among the
mandatory variables of the frame.  Any new variables referenced are called the
{\bf optional variables}\index{optional variable} of the extended frame. If a
\texttt{\$f} statement references an optional variable it is called an {\bf
optional hypothesis}\index{optional hypothesis}, and if one or both of the
variables in a \texttt{\$d} statement are optional variables it is called an {\bf
optional disjoint-variable restriction}\index{optional disjoint-variable
restriction}.  Properties 2 and 3 for a frame also apply to an extended
frame.

The concept of optional variables is not meaningful for frames with \texttt{\$a}
statements, since those statements have no proofs that might make use of them.
There is no restriction on including optional hypotheses in the extended frame
for a \texttt{\$a} statement, but they serve no purpose.

The following set of statements is an example of an extended frame, which
contains an optional variable \texttt{R} and an optional hypothesis \texttt{wr}.  In
this example, we suppose the rule of modus ponens is not an axiom but is
derived as a theorem from earlier statements (we omit its presumed proof).
Variable \texttt{R} may be used in its proof if desired (although this would
probably have no advantage in propositional calculus).  Note that the sequence
of mandatory hypotheses in RPN order is still \texttt{wp}, \texttt{wq}, \texttt{maj},
\texttt{min} (i.e.\ \texttt{wr} is omitted), and this sequence is still assumed
whenever the assertion \texttt{mp} is referenced in a subsequent proof.

\begin{verbatim}
wp  $f wff P $.
wq  $f wff Q $.
wr  $f wff R $.
maj $e |- ( P -> Q ) $.
min $e |- P $.
mp  $p |- Q $= ... $.
\end{verbatim}

Every frame is an extended frame, but not every extended frame is a frame, as
this example shows.  The underlying frame for an extended frame is
obtained by simply removing all statements containing optional variables.
Any proof referencing an assertion will ignore any extensions to its
frame, which means we may add or delete optional hypotheses at will without
affecting subsequent proofs.

The conceptually simplest way of organizing a Metamath database is as a
sequence of extended frames.  The scoping statements
\texttt{\$\char`\{}\index{\texttt{\$\char`\{} and \texttt{\$\char`\}}
keywords} and \texttt{\$\char`\}} can be used to delimit the start and
end of an extended frame, leading to the following possible structure for a
database.  \label{framelist}

\vskip 2ex
\setbox\startprefix=\hbox{\tt \ \ \ \ \ \ \ \ }
\setbox\contprefix=\hbox{}
\startm
\m{\mbox{(\texttt{\$v} {\em and} \texttt{\$c}\,{\em statements})}}
\endm
\startm
\m{\mbox{\texttt{\$\char`\{}}}
\endm
\startm
\m{\mbox{\texttt{\ \ } {\em extended frame}}}
\endm
\startm
\m{\mbox{\texttt{\$\char`\}}}}
\endm
\startm
\m{\mbox{\texttt{\$\char`\{}}}
\endm
\startm
\m{\mbox{\texttt{\ \ } {\em extended frame}}}
\endm
\startm
\m{\mbox{\texttt{\$\char`\}}}}
\endm
\startm
\m{\mbox{\texttt{\ \ \ \ \ \ \ \ \ }}\vdots}
\endm
\vskip 2ex

In practice, this structure is inconvenient because we have to repeat
any \texttt{\$f}, \texttt{\$e}, and \texttt{\$d} statements over and
over again rather than stating them once for use by several assertions.
The scoping statements, which we will discuss next, allow this to be
done.  In principle, any Metamath database can be converted to the above
format, and the above format is the most convenient to use when studying
a Metamath database as a formal system%
%% Uncomment this when uncommenting section {formalspec} below
   (Appendix \ref{formalspec})%
.
In fact, Metamath internally converts the database to the above format.
The command \texttt{show statement} in the Metamath program will show
you the contents of the frame for any \texttt{\$a} or \texttt{\$p}
statement, as well as its extension in the case of a \texttt{\$p}
statement.

%c%(provided that all ``local'' variables and constants with limited scope have
%c%unique names),

During our discussion of scoping statements, it may be helpful to
think in terms of the equivalent sequence of frames that will result when
the database is parsed.  Scoping (other than the limited
use above to delimit frames) is not a theoretical requirement for
Metamath but makes it more convenient.


\subsection{Scoping Statements (\texttt{\$\{} and \texttt{\$\}})}\label{scoping}
\index{\texttt{\$\char`\{} and \texttt{\$\char`\}} keywords}\index{scoping statement}

%c%Some Metamath statements may be needed only temporarily to
%c%serve a specific purpose, and after we're done with them we would like to
%c%disregard or ignore them.  For example, when we're finished using a variable,
%c%we might want to
%c%we might want to free up the token\index{token} used to name it so that the
%c%token can be used for other purposes later on, such as a different kind of
%c%variable or even a constant.  In the terminology of computer programming, we
%c%might want to let some symbol declarations be ``local'' rather than ``global.''
%c%\index{local symbol}\index{global symbol}

The {\bf scoping} statements, \texttt{\$\char`\{} ({\bf start of block}) and \texttt{\$\char`\}}
({\bf end of block})\index{block}, provide a means for controlling the portion
of a database over which certain statement types are recognized.  The
syntax of a scoping statement is very simple; it just consists of the
statement's keyword:
\begin{center}
\texttt{\$\char`\{}\\
\texttt{\$\char`\}}
\end{center}
\index{\texttt{\$\char`\{} and \texttt{\$\char`\}} keywords}

For example, consider the following database where we have stripped out
all tokens except the scoping statement keywords.  For the purpose of the
discussion, we have added subscripts to the scoping statements; these subscripts
do not appear in the actual database.
\[
 \mbox{\tt \ \$\char`\{}_1
 \mbox{\tt \ \$\char`\{}_2
 \mbox{\tt \ \$\char`\}}_2
 \mbox{\tt \ \$\char`\{}_3
 \mbox{\tt \ \$\char`\{}_4
 \mbox{\tt \ \$\char`\}}_4
 \mbox{\tt \ \$\char`\}}_3
 \mbox{\tt \ \$\char`\}}_1
\]
Each \texttt{\$\char`\{} statement in this example is said to be {\bf
matched} with the \texttt{\$\char`\}} statement that has the same
subscript.  Each pair of matched scoping statements defines a region of
the database called a {\bf block}.\index{block} Blocks can be {\bf
nested}\index{nested block} inside other blocks; in the example, the
block defined by $\mbox{\tt \$\char`\{}_4$ and $\mbox{\tt \$\char`\}}_4$
is nested inside the block defined by $\mbox{\tt \$\char`\{}_3$ and
$\mbox{\tt \$\char`\}}_3$ as well as inside the block defined by
$\mbox{\tt \$\char`\{}_1$ and $\mbox{\tt \$\char`\}}_1$.  In general, a
block may be empty, it may contain only non-scoping
statements,\footnote{Those statements other than \texttt{\$\char`\{} and
\texttt{\$\char`\}}.}\index{non-scoping statement} or it may contain any
mixture of other blocks and non-scoping statements.  (This is called a
``recursive'' definition\index{recursive definition} of a block.)

Associated with each block is a number called its {\bf nesting
level}\index{nesting level} that indicates how deeply the block is nested.
The nesting levels of the blocks in our example are as follows:
\[
  \underbrace{
    \mbox{\tt \ }
    \underbrace{
     \mbox{\tt \$\char`\{\ }
     \underbrace{
       \mbox{\tt \$\char`\{\ }
       \mbox{\tt \$\char`\}}
     }_{2}
     \mbox{\tt \ }
     \underbrace{
       \mbox{\tt \$\char`\{\ }
       \underbrace{
         \mbox{\tt \$\char`\{\ }
         \mbox{\tt \$\char`\}}
       }_{3}
       \mbox{\tt \ \$\char`\}}
     }_{2}
     \mbox{\tt \ \$\char`\}}
   }_{1}
   \mbox{\tt \ }
 }_{0}
\]
\index{\texttt{\$\char`\{} and \texttt{\$\char`\}} keywords}
The entire database is considered to be one big block (the {\bf outermost}
block) with a nesting level of 0.  The outermost block is {\em not} bracketed
by scoping statements.\footnote{The language was designed this way so that
several source files can be joined together more easily.}\index{outermost
block}

All non-scoping Metamath statements become recognized or {\bf
active}\index{active statement} at the place where they appear.\footnote{To
keep things slightly simpler, we do not bother to define the concept of
``active'' for the scoping statements.}  Certain of these statement types
become inactive at the end of the block in which they appear; these statement
types are:
\begin{center}
  \texttt{\$c}, \texttt{\$v}, \texttt{\$d}, \texttt{\$e}, and \texttt{\$f}.
%  \texttt{\$v}, \texttt{\$f}, \texttt{\$e}, and \texttt{\$d}.
\end{center}
\index{\texttt{\$c} statement}
\index{\texttt{\$d} statement}
\index{\texttt{\$e} statement}
\index{\texttt{\$f} statement}
\index{\texttt{\$v} statement}
The other statement types remain active forever (i.e.\ through the end of the
database); they are:
\begin{center}
  \texttt{\$a} and \texttt{\$p}.
%  \texttt{\$c}, \texttt{\$a}, and \texttt{\$p}.
\end{center}
\index{\texttt{\$a} statement}
\index{\texttt{\$p} statement}
Any statement (of these 7 types) located in the outermost
block\index{outermost block} will remain active through the end of the
database and thus are effectively ``global'' statements.\index{global
statement}

All \texttt{\$c} statements must be placed in the outermost block.  Since they are
therefore always global, they could be considered as belonging to both of the
above categories.

The {\bf scope}\index{scope} of a statement is the set of statements that
recognize it as active.

%c%The concept of ``active'' is also defined for math symbols\index{math
%c%symbol}.  Math symbols (constants\index{constant} and
%c%variables\index{variable}) become {\bf active}\index{active
%c%math symbol} in the \texttt{\$c}\index{\texttt{\$c}
%c%statement} and \texttt{\$v}\index{\texttt{\$v} statement} statements that
%c%declare them.  They become inactive when their declaration statements become
%c%inactive.

The concept of ``active'' is also defined for math symbols\index{math
symbol}.  Math symbols (constants\index{constant} and
variables\index{variable}) become {\bf active}\index{active math symbol}
in the \texttt{\$c}\index{\texttt{\$c} statement} and
\texttt{\$v}\index{\texttt{\$v} statement} statements that declare them.
A variable becomes inactive when its declaration statement becomes
inactive.  Because all \texttt{\$c} statements must be in the outermost
block, a constant will never become inactive after it is declared.

\subsubsection{Redeclaration of Math Symbols}
\index{redeclaration of symbols}\label{redeclaration}

%c%A math symbol may not be declared a second time while it is active, but it may
%c%be declared again after it becomes inactive.

A variable may not be declared a second time while it is active, but it may be
declared again after it becomes inactive.  This provides a convenient way to
introduce ``local'' variables,\index{local variable} i.e.\ temporary variables
for use in the frame of an assertion or in a proof without keeping them around
forever.  A previously declared variable may not be redeclared as a constant.

A constant may not be redeclared.  And, as mentioned above, constants must be
declared in the outermost block.

The reason variables may have limited scope but not constants is that an
assertion (\texttt{\$a} or \texttt{\$p} statement) remains available for use in
proofs through the end of the database.  Variables in an assertion's frame may
be substituted with whatever is needed in a proof step that references the
assertion, whereas constants remain fixed and may not be substituted with
anything.  The particular token used for a variable in an assertion's frame is
irrelevant when the assertion is referenced in a proof, and it doesn't matter
if that token is not available outside of the referenced assertion's frame.
Constants, however, must be globally fixed.

There is no theoretical
benefit for the feature allowing variables to be active for limited scopes
rather than global. It is just a convenience that allows them, for example, to
be locally grouped together with their corresponding \texttt{\$f} variable-type
declarations.

%c%If you declare a math symbol more than once, internally Metamath considers it a
%c%new distinct symbol, even though it has the same name.  If you are unaware of
%c%this, you may find that what you think are correct proofs are incorrectly
%c%rejected as invalid, because Metamath may tell you that a constant you
%c%previously declared does not match a newly declared math symbol with the same
%c%name.  For details on this subtle point, see the Comment on
%c%p.~\pageref{spec4comment}.  This is done purposely to allow temporary
%c%constants to be introduced while developing a subtheory, then allow their math
%c%symbol tokens to be reused later on; in general they will not refer to the
%c%same thing.  In practice, you would not ordinarily reuse the names of
%c%constants because it would tend to be confusing to the reader.  The reuse of
%c%names of variables, on the other hand, is something that is often useful to do
%c%(for example it is done frequently in \texttt{set.mm}).  Since variables in an
%c%assertion referenced in a proof can be substituted as needed to achieve a
%c%symbol match, this is not an issue.

% (This section covers a somewhat advanced topic you may want to skip
% at first reading.)
%
% Under certain circumstances, math symbol\index{math symbol}
% tokens\index{token} may be redeclared (i.e.\ the token
% may appear in more than
% one \texttt{\$c}\index{\texttt{\$c} statement} or \texttt{\$v}\index{\texttt{\$v}
% statement} statement).  You might want to do this say, to make temporary use
% of a variable name without having to worry about its affect elsewhere,
% somewhat analogous to declaring a local variable in a standard computer
% language.  Understanding what goes on when math symbol tokens are redeclared
% is a little tricky to understand at first, since it requires that we
% distinguish the token itself from the math symbol that it names.  It will help
% if we first take a peek at the internal workings of the
% Metamath\index{Metamath} program.
%
% Metamath reserves a memory location for each occurrence of a
% token\index{token} in a declaration statement (\texttt{\$c}\index{\texttt{\$c}
% statement} or \texttt{\$v}\index{\texttt{\$v} statement}).  If a given token appears
% in more than one declaration statement, it will refer to more than one memory
% locations.  A math symbol\index{math symbol} may be thought of as being one of
% these memory locations rather than as the token itself.  Only one of the
% memory locations associated with a given token may be active at any one time.
% The math symbol (memory location) that gets looked up when the token appears
% in a non-declaration statement is the one that happens to be active at that
% time.
%
% We now look at the rules for the redeclaration\index{redeclaration of symbols}
% of math symbol tokens.
% \begin{itemize}
% \item A math symbol token may not be declared twice in the
% same block.\footnote{While there is no theoretical reason for disallowing
% this, it was decided in the design of Metamath that allowing it would offer no
% advantage and might cause confusion.}
% \item An inactive math symbol may always be
% redeclared.
% \item  An active math symbol may be redeclared in a different (i.e.\
% inner) block\index{block} from the one it became active in.
% \end{itemize}
%
% When a math symbol token is redeclared, it conceptually refers to a different
% math symbol, just as it would be if it were called a different name.  In
% addition, the original math symbol that it referred to, if it was active,
% temporarily becomes inactive.  At the end of the block in which the
% redeclaration occurred, the new math symbol\index{math symbol} becomes
% inactive and the original symbol becomes active again.  This concept is
% illustrated in the following example, where the symbol \texttt{e} is
% ordinarily a constant (say Euler's constant, 2.71828...) but
% temporarily we want to use it as a ``local'' variable, say as a coefficient
% in the equation $a x^4 + b x^3 + c x^2 + d x + e$:
% \[
%   \mbox{\tt \$\char`\{\ \$c e \$.}
%   \underbrace{
%     \ \ldots\ %
%     \mbox{\tt \$\char`\{}\ \ldots\ %
%   }_{\mbox{\rm region A}}
%   \mbox{\tt \$v e \$.}
%   \underbrace{
%     \mbox{\ \ \ \ldots\ \ \ }
%   }_{\mbox{\rm region B}}
%   \mbox{\tt \$\char`\}}
%   \underbrace{
%     \mbox{\ \ \ \ldots\ \ \ }
%   }_{\mbox{\rm region C}}
%   \mbox{\tt \$\char`\}}
% \]
% \index{\texttt{\$\char`\{} and \texttt{\$\char`\}} keywords}
% In region A, the token \texttt{e} refers to a constant.  It is redeclared as a
% variable in region B, and any reference to it in this region will refer to this
% variable.  In region C, the redeclaration becomes inactive, and the original
% declaration becomes active again.  In region C, the token \texttt{x} refers to the
% original constant.
%
% As a practical matter, overuse of math symbol\index{math symbol}
% redeclarations\index{redeclaration of symbols} can be confusing (even though
% it is well-defined) and is best avoided when possible.  Here are some good
% general guidelines you can follow.  Usually, you should declare all
% constants\index{constant} in the outermost block\index{outermost block},
% especially if they are general-purpose (such as the token \verb$A.$, meaning
% $\forall$ or ``for all'').  This will make them ``globally'' active (although
% as in the example above local redeclarations will temporarily make them
% inactive.)  Most or all variables\index{variable}, on the other hand, could be
% declared in inner blocks, so that the token for them can be used later for a
% different type of variable or a constant.  (The names of the variables you
% choose are not used when you refer to an assertion\index{assertion} in a
% proof, whereas constants must match exactly.  A locally declared constant will
% not match a globally declared constant in a proof, even if they use the same
% token, because Metamath internally considers them to be different math
% symbols.)  To avoid confusion, you should generally avoid redeclaring active
% variables.  If you must redeclare them, do so at the beginning of a block.
% The temporary declaration of constants in inner blocks might be occasionally
% appropriate when you make use of a temporary definition to prove lemmas
% leading to a main result that does not make direct use of the definition.
% This way, you will not clutter up your database with a large number of
% seldom-used global constant symbols.  You might want to note that while
% inactive constants may not appear directly in an assertion (a \texttt{\$a}\index{\texttt{\$a}
% statement} or \texttt{\$p}\index{\texttt{\$p} statement}
% statement), they may be indirectly used in the proof of a \texttt{\$p} statement
% so long as they do not appear in the final math symbol sequence constructed by
% the proof.  In the end, you will have to use your best judgment, taking into
% account standard mathematical usage of the symbols as well as consideration
% for the reader of your work.
%
% \subsubsection{Reuse of Labels}\index{reuse of labels}\index{label}
%
% The \texttt{\$e}\index{\texttt{\$e} statement}, \texttt{\$f}\index{\texttt{\$f}
% statement}, \texttt{\$a}\index{\texttt{\$a} statement}, and
% \texttt{\$p}\index{\texttt{\$p}
% statement} statement types require labels, which allow them to be
% referenced later inside proofs.  A label is considered {\bf
% active}\index{active label} when the statement it is associated with is
% active.  The token\index{token} for a label may be reused
% (redeclared)\index{redeclaration of labels} provided that it is not being used
% for a currently active label.  (Unlike the tokens for math symbols, active
% label tokens may not be redeclared in an inner scope.)  Note that the labels
% of \texttt{\$a} and \texttt{\$p} statements can never be reused after these
% statements appear, because these statements remain active through the end of
% the database.
%
% You might find the reuse of labels a convenient way to have standard names for
% temporary hypotheses, such as \texttt{h1}, \texttt{h2}, etc.  This way you don't have
% to invent unique names for each of them, and in some cases it may be less
% confusing to the reader (although in other cases it might be more confusing, if
% the hypothesis is located far away from the assertion that uses
% it).\footnote{The current implementation requires that all labels, even
% inactive ones, be unique.}

\subsubsection{Frames Revisited}\index{frames and scoping statements}

Now that we have covered scoping, we will look at how an arbitrary
Metamath database can be converted to the simple sequence of extended
frames described on p.~\pageref{framelist}.  This is also how Metamath
stores the database internally when it reads in the database
source.\label{frameconvert} The method is simple.  First, we collect all
constant and variable (\texttt{\$c} and \texttt{\$v}) declarations in
the database, ignoring duplicate declarations of the same variable in
different scopes.  We then put our collected \texttt{\$c} and
\texttt{\$v} declarations at the beginning of the database, so that
their scope is the entire database.  Next, for each assertion in the
database, we determine its frame and extended frame.  The extended frame
is simply the \texttt{\$f}, \texttt{\$e}, and \texttt{\$d} statements
that are active.  The frame is the extended frame with all optional
hypotheses removed.

An equivalent way of saying this is that the extended frame of an assertion
is the collection of all \texttt{\$f}, \texttt{\$e}, and \texttt{\$d} statements
whose scope includes the assertion.
The \texttt{\$f} and \texttt{\$e} statements
occur in the order they appear
(order is irrelevant for \texttt{\$d} statements).

%c%, renaming any
%c%redeclared variables as needed so that all of them have unique names.  (The
%c%exact renaming convention is unimportant.  You might imagine renaming
%c%different declarations of math symbol \texttt{a} as \texttt{a\$1}, \texttt{a\$2}, etc.\
%c%which would prevent any conflicts since \texttt{\$} is not a legal character in a
%c%math symbol token.)

\section{The Anatomy of a Proof} \label{proof}
\index{proof!Metamath, description of}

Each provable assertion (\texttt{\$p}\index{\texttt{\$p} statement} statement) in a
database must include a {\bf proof}\index{proof}.  The proof is located
between the \texttt{\$=}\index{\texttt{\$=} keyword} and \texttt{\$.}\ keywords in the
\texttt{\$p} statement.

In the basic Metamath language\index{basic language}, a proof is a
sequence of statement labels.  This label sequence\index{label sequence}
serves as a set of instructions that the Metamath program uses to
construct a series of math symbol sequences.  The construction must
ultimately result in the math symbol sequence contained between the
\texttt{\$p}\index{\texttt{\$p} statement} and
\texttt{\$=}\index{\texttt{\$=} keyword} keywords of the \texttt{\$p}
statement.  Otherwise, the Metamath program will consider the proof
incorrect, and it will notify you with an appropriate error message when
you ask it to verify the proof.\footnote{To make the loading faster, the
Metamath program does not automatically verify proofs when you
\texttt{read} in a database unless you use the \texttt{/verify}
qualifier.  After a database has been read in, you may use the
\texttt{verify proof *} command to verify proofs.}\index{\texttt{verify
proof} command} Each label in a proof is said to {\bf
reference}\index{label reference} its corresponding statement.

Associated with any assertion\index{assertion} (\texttt{\$p} or
\texttt{\$a}\index{\texttt{\$a} statement} statement) is a set of
hypotheses (\texttt{\$f}\index{\texttt{\$f} statement} or
\texttt{\$e}\index{\texttt{\$e} statement} statements) that are active
with respect to that assertion.  Some are mandatory and the others are
optional.  You should review these concepts if necessary.

Each label\index{label} in a proof must be either the label of a
previous assertion (\texttt{\$a}\index{\texttt{\$a} statement} or
\texttt{\$p}\index{\texttt{\$p} statement} statement) or the label of an
active hypothesis (\texttt{\$e} or \texttt{\$f}\index{\texttt{\$f}
statement} statement) of the \texttt{\$p} statement containing the
proof.  Hypothesis labels may reference both the
mandatory\index{mandatory hypothesis} and the optional hypotheses of the
\texttt{\$p} statement.

The label sequence in a proof specifies a construction in {\bf reverse Polish
notation}\index{reverse Polish notation (RPN)} (RPN).  You may be familiar
with RPN if you have used older
Hewlett--Packard or similar hand-held calculators.
In the calculator analogy, a hypothesis label\index{hypothesis label} is like
a number and an assertion label\index{assertion label} is like an operation
(more precisely, an $n$-ary operation when the
assertion has $n$ \texttt{\$e}-hypotheses).
On an RPN calculator, an operation takes one or more previous numbers in an
input sequence, performs a calculation on them, and replaces those numbers and
itself with the result of the calculation.  For example, the input sequence
$2,3,+$ on an RPN calculator results in $5$, and the input sequence
$2,3,5,{\times},+$ results in $2,15,+$ which results in $17$.

Understanding how RPN is processed involves the concept of a {\bf
stack}\index{stack}\index{RPN stack}, which can be thought of as a set of
temporary memory locations that hold intermediate results.  When Metamath
encounters a hypothesis label it places or {\bf pushes}\index{push} the math
symbol sequence of the hypothesis onto the stack.  When Metamath encounters an
assertion label, it associates the most recent stack entries with the {\em
mandatory} hypotheses\index{mandatory hypothesis} of the assertion, in the
order where the most recent stack entry is associated with the last mandatory
hypothesis of the assertion.  It then determines what
substitutions\index{substitution!variable}\index{variable substitution} have
to be made into the variables of the assertion's mandatory hypotheses to make
them identical to the associated stack entries.  It then makes those same
substitutions into the assertion itself.  Finally, Metamath removes or {\bf
pops}\index{pop} the matched hypotheses from the stack and pushes the
substituted assertion onto the stack.

For the purpose of matching the mandatory hypothesis to the most recent stack
entries, whether a hypothesis is a \texttt{\$e} or \texttt{\$f} statement is
irrelevant.  The only important thing is that a set of
substitutions\footnote{In the Metamath spec (Section~\ref{spec}), we use the
singular term ``substitution'' to refer to the set of substitutions we talk
about here.} exist that allow a match (and if they don't, the proof verifier
will let you know with an error message).  The Metamath language is specified
in such a way that if a set of substitutions exists, it will be unique.
Specifically, the requirement that each variable have a type specified for it
with a \texttt{\$f} statement ensures the uniqueness.

We will illustrate this with an example.
Consider the following Metamath source file:
\begin{verbatim}
$c ( ) -> wff $.
$v p q r s $.
wp $f wff p $.
wq $f wff q $.
wr $f wff r $.
ws $f wff s $.
w2 $a wff ( p -> q ) $.
wnew $p wff ( s -> ( r -> p ) ) $= ws wr wp w2 w2 $.
\end{verbatim}
This Metamath source example shows the definition and ``proof'' (i.e.,
construction) of a well-formed formula (wff)\index{well-formed formula (wff)}
in propositional calculus.  (You may wish to type this example into a file to
experiment with the Metamath program.)  The first two statements declare
(introduce the names of) four constants and four variables.  The next four
statements specify the variable types, namely that
each variable is assumed to be a wff.  Statement \texttt{w2} defines (postulates)
a way to produce a new wff, \texttt{( p -> q )}, from two given wffs \texttt{p} and
\texttt{q}. The mandatory hypotheses of \texttt{w2} are \texttt{wp} and \texttt{wq}.
Statement \texttt{wnew} claims that \texttt{( s -> ( r -> p ) )} is a wff given
three wffs \texttt{s}, \texttt{r}, and \texttt{p}.  More precisely, \texttt{wnew} claims
that the sequence of ten symbols \texttt{wff ( s -> ( r -> p ) )} is provable from
previous assertions and the hypotheses of \texttt{wnew}.  Metamath does not know
or care what a wff is, and as far as it is concerned
the typecode \texttt{wff} is just an
arbitrary constant symbol in a math symbol sequence.  The mandatory hypotheses
of \texttt{wnew} are \texttt{wp}, \texttt{wr}, and \texttt{ws}; \texttt{wq} is an optional
hypothesis.  In our particular proof, the optional hypothesis is not
referenced, but in general, any combination of active (i.e.\ optional and
mandatory) hypotheses could be referenced.  The proof of statement \texttt{wnew}
is the sequence of five labels starting with \texttt{ws} (step~1) and ending with
\texttt{w2} (step~5).

When Metamath verifies the proof, it scans the proof from left to right.  We
will examine what happens at each step of the proof.  The stack starts off
empty.  At step 1, Metamath looks up label \texttt{ws} and determines that it is a
hypothesis, so it pushes the symbol sequence of statement \texttt{ws} onto the
stack:

\begin{center}\begin{tabular}{|l|l|}\hline
{Stack location} & {Contents} \\ \hline \hline
1 & \texttt{wff s} \\ \hline
\end{tabular}\end{center}

Metamath sees that the labels \texttt{wr} and \texttt{wp} in steps~2 and 3 are also
hypotheses, so it pushes them onto the stack.  After step~3, the stack looks
like
this:

\begin{center}\begin{tabular}{|l|l|}\hline
{Stack location} & {Contents} \\ \hline \hline
3 & \texttt{wff p} \\ \hline
2 & \texttt{wff r} \\ \hline
1 & \texttt{wff s} \\ \hline
\end{tabular}\end{center}

At step 4, Metamath sees that label \texttt{w2} is an assertion, so it must do
some processing.  First, it associates the mandatory hypotheses of \texttt{w2},
which are \texttt{wp} and \texttt{wq}, with stack locations~2 and 3, {\em in that
order}. Metamath determines that the only possible way
to make hypothesis \texttt{wp} match (become identical to) stack location~2 and
\texttt{wq} match stack location 3 is to substitute variable \texttt{p} with \texttt{r}
and \texttt{q} with \texttt{p}.  Metamath makes these substitutions into \texttt{w2} and
obtains the symbol sequence \texttt{wff ( r -> p )}.  It removes the hypotheses
from stack locations~2 and 3, then places the result into stack location~2:

\begin{center}\begin{tabular}{|l|l|}\hline
{Stack location} & {Contents} \\ \hline \hline
2 & \texttt{wff ( r -> p )} \\ \hline
1 & \texttt{wff s} \\ \hline
\end{tabular}\end{center}

At step 5, Metamath sees that label \texttt{w2} is an assertion, so it must again
do some processing.  First, it matches the mandatory hypotheses of \texttt{w2},
which are \texttt{wp} and \texttt{wq}, to stack locations 1 and 2.
Metamath determines that the only possible way to make the
hypotheses match is to substitute variable \texttt{p} with \texttt{s} and \texttt{q} with
\texttt{( r -> p )}.  Metamath makes these substitutions into \texttt{w2} and obtains
the symbol
sequence \texttt{wff ( s -> ( r -> p ) )}.  It removes stack
locations 1 and 2, then places the result into stack location~1:

\begin{center}\begin{tabular}{|l|l|}\hline
{Stack location} & {Contents} \\ \hline \hline
1 & \texttt{wff ( s -> ( r -> p ) )} \\ \hline
\end{tabular}\end{center}

After Metamath finishes processing the proof, it checks to see that the
stack contains exactly one element and that this element is
the same as the math symbol sequence in the
\texttt{\$p}\index{\texttt{\$p} statement} statement.  This is the case for our
proof of \texttt{wnew},
so we have proved \texttt{wnew} successfully.  If the result
differs, Metamath will notify you with an error message.  An error message
will also result if the stack contains more than one entry at the end of the
proof, or if the stack did not contain enough entries at any point in the
proof to match all of the mandatory hypotheses\index{mandatory hypothesis} of
an assertion.  Finally, Metamath will notify you with an error message if no
substitution is possible that will make a referenced assertion's hypothesis
match the
stack entries.  You may want to experiment with the different kinds of errors
that Metamath will detect by making some small changes in the proof of our
example.

Metamath's proof notation was designed primarily to express proofs in a
relatively compact manner, not for readability by humans.  Metamath can display
proofs in a number of different ways with the \texttt{show proof}\index{\texttt{show
proof} command} command.  The
\texttt{/lemmon} qualifier displays it in a format that is easier to read when the
proofs are short, and you saw examples of its use in Chapter~\ref{using}.  For
longer proofs, it is useful to see the tree structure of the proof.  A tree
structure is displayed when the \texttt{/lemmon} qualifier is omitted.  You will
probably find this display more convenient as you get used to it. The tree
display of the proof in our example looks like
this:\label{treeproof}\index{tree-style proof}\index{proof!tree-style}
\begin{verbatim}
1     wp=ws    $f wff s
2        wp=wr    $f wff r
3        wq=wp    $f wff p
4     wq=w2    $a wff ( r -> p )
5  wnew=w2  $a wff ( s -> ( r -> p ) )
\end{verbatim}
The number to the left of each line is the step number.  Following it is a
{\bf hypothesis association}\index{hypothesis association}, consisting of two
labels\index{label} separated by \texttt{=}.  To the left of the \texttt{=} (except
in the last step) is the label of a hypothesis of an assertion referenced
later in the proof; here, steps 1 and 4 are the hypothesis associations for
the assertion \texttt{w2} that is referenced in step 5.  A hypothesis association
is indented one level more than the assertion that uses it, so it is easy to
find the corresponding assertion by moving directly down until the indentation
level decreases to one less than where you started from.  To the right of each
\texttt{=} is the proof step label for that proof step.  The statement keyword of
the proof step label is listed next, followed by the content of the top of the
stack (the most recent stack entry) as it exists after that proof step is
processed.  With a little practice, you should have no trouble reading proofs
displayed in this format.

Metamath proofs include the syntax construction of a formula.
In standard mathematics, this kind of
construction is not considered a proper part of the proof at all, and it
certainly becomes rather boring after a while.
Therefore,
by default the \texttt{show proof}\index{\texttt{show proof}
command} command does not show the syntax construction.
Historically \texttt{show proof} command
\textit{did} show the syntax construction, and you needed to add the
\texttt{/essential} option to hide, them, but today
\texttt{/essential} is the default and you need to use
\texttt{/all} to see the syntax constructions.

When verifying a proof, Metamath will check that no mandatory
\texttt{\$d}\index{\texttt{\$d} statement}\index{mandatory \texttt{\$d}
statement} statement of an assertion referenced in a proof is violated
when substitutions\index{substitution!variable}\index{variable
substitution} are made to the variables in the assertion.  For details
see Section~\ref{spec4} or \ref{dollard}.

\subsection{The Concept of Unification} \label{unify}

During the course of verifying a proof, when Metamath\index{Metamath}
encounters an assertion label\index{assertion label}, it associates the
mandatory hypotheses\index{mandatory hypothesis} of the assertion with the top
entries of the RPN stack\index{stack}\index{RPN stack}.  Metamath then
determines what substitutions\index{substitution!variable}\index{variable
substitution} it must make to the variables in the assertion's mandatory
hypotheses in order for these hypotheses to become identical to their
corresponding stack entries.  This process is called {\bf
unification}\index{unification}.  (We also informally use the term
``unification'' to refer to a set of substitutions that results from the
process, as in ``two unifications are possible.'')  After the substitutions
are made, the hypotheses are said to be {\bf unified}.

If no such substitutions are possible, Metamath will consider the proof
incorrect and notify you with an error message.
% (deleted 3/10/07, per suggestion of Mel O'Cat:)
% The syntax of the
% Metamath language ensures that if a set of substitutions exists, it
% will be unique.

The general algorithm for unification described in the literature is
somewhat complex.
However, in the case of Metamath it is intentionally trivial.
Mandatory hypotheses must be
pushed on the proof stack in the order in which they appear.
In addition, each variable must have its type specified
with a \texttt{\$f} hypothesis before it is used
and that each \texttt{\$f} hypothesis
have the restricted syntax of a typecode (a constant) followed by a variable.
The typecode in the \texttt{\$f} hypothesis must match the first symbol of
the corresponding RPN stack entry (which will also be a constant), so
the only possible match for the variable in the \texttt{\$f} hypothesis is
the sequence of symbols in the stack entry after the initial constant.

In the Proof Assistant\index{Proof Assistant}, a more general unification
algorithm is used.  While a proof is being developed, sometimes not enough
information is available to determine a unique unification.  In this case
Metamath will ask you to pick the correct one.\index{ambiguous
unification}\index{unification!ambiguous}

\section{Extensions to the Metamath Language}\index{extended
language}

\subsection{Comments in the Metamath Language}\label{comments}
\index{markup notation}
\index{comments!markup notation}

The commenting feature allows you to annotate the contents of
a database.  Just as with most
computer languages, comments are ignored for the purpose of interpreting the
contents of the database. Comments effectively act as
additional white space\index{white
space} between tokens
when a database is parsed.

A comment may be placed at the beginning, end, or
between any two tokens\index{token} in a source file.

Comments have the following syntax:
\begin{center}
 \texttt{\$(} {\em text} \texttt{\$)}
\end{center}
Here,\index{\texttt{\$(} and \texttt{\$)} auxiliary
keywords}\index{comment} {\em text} is a string, possibly empty, of any
characters in Metamath's character set (p.~\pageref{spec1chars}), except
that the character strings \texttt{\$(} and \texttt{\$)} may not appear
in {\em text}.  Thus nested comments are not
permitted:\footnote{Computer languages have differing standards for
nested comments, and rather than picking one it was felt simplest not to
allow them at all, at least in the current version (0.177) of
Metamath\index{Metamath!limitations of version 0.177}.} Metamath will
complain if you give it
\begin{center}
 \texttt{\$( This is a \$( nested \$) comment.\ \$)}
\end{center}
To compensate for this non-nesting behavior, I often change all \texttt{\$}'s
to \texttt{@}'s in sections of Metamath code I wish to comment out.

The Metamath program supports a number of markup mechanisms and conventions
to generate good-looking results in \LaTeX\ and {\sc html},
as discussed below.
These markup features have to do only with how the comments are typeset,
and have no effect on how Metamath verifies the proofs in the database.
The improper
use of them may result in incorrectly typeset output, but no Metamath
error messages will result during the \texttt{read} and \texttt{verify
proof} commands.  (However, the \texttt{write
theorem\texttt{\char`\_}list} command
will check for markup errors as a side-effect of its
{\sc html} generation.)
Section~\ref{texout} has instructions for creating \LaTeX\ output, and
section~\ref{htmlout} has instructions for creating
{\sc html}\index{HTML} output.

\subsubsection{Headings}\label{commentheadings}

If the \texttt{\$(} is immediately followed by a new line
starting with a heading marker, it is a header.
This can start with:

\begin{itemize}
 \item[] \texttt{\#\#\#\#} - major part header
 \item[] \texttt{\#*\#*} - section header
 \item[] \texttt{=-=-} - subsection header
 \item[] \texttt{-.-.} - subsubsection header
\end{itemize}

The line following the marker line
will be used for the table of contents entry, after trimming spaces.
The next line should be another (closing) matching marker line.
Any text after that
but before the closing \texttt{\$}, such as an extended description of the
section, will be included on the \texttt{mmtheoremsNNN.html} page.

For more information, run
\texttt{help write theorem\char`\_list}.

\subsubsection{Math mode}
\label{mathcomments}
\index{\texttt{`} inside comments}
\index{\texttt{\char`\~} inside comments}
\index{math mode}

Inside of comments, a string of tokens\index{token} enclosed in
grave accents\index{grave accent (\texttt{`})} (\texttt{`}) will be converted
to standard mathematical symbols during
{\sc HTML}\index{HTML} or \LaTeX\ output
typesetting,\index{latex@{\LaTeX}} according to the information in the
special \texttt{\$t}\index{\texttt{\$t} comment}\index{typesetting
comment} comment in the database
(see section~\ref{tcomment} for information about the typesetting
comment, and Appendix~\ref{ASCII} to see examples of its results).

The first grave accent\index{grave accent (\texttt{`})} \texttt{`}
causes the output processor to enter {\bf math mode}\index{math mode}
and the second one exits it.
In this
mode, the characters following the \texttt{`} are interpreted as a
sequence of math symbol tokens separated by white space\index{white
space}.  The tokens are looked up in the \texttt{\$t}
comment\index{\texttt{\$t} comment}\index{typesetting comment} and if
found, they will be replaced by the standard mathematical symbols that
they correspond to before being placed in the typeset output file.  If
not found, the symbol will be output as is and a warning will be issued.
The tokens do not have to be active in the database, although a warning
will be issued if they are not declared with \texttt{\$c} or
\texttt{\$v} statements.

Two consecutive
grave accents \texttt{``} are treated as a single actual grave accent
(both inside and outside of math mode) and will not cause the output
processor to enter or exit math mode.

Here is an example of its use\index{Pierce's axiom}:
\begin{center}
\texttt{\$( Pierce's axiom, ` ( ( ph -> ps ) -> ph ) -> ph ` ,\\
         is not very intuitive. \$)}
\end{center}
becomes
\begin{center}
   \texttt{\$(} Pierce's axiom, $((\varphi \rightarrow \psi)\rightarrow
\varphi)\rightarrow \varphi$, is not very intuitive. \texttt{\$)}
\end{center}

Note that the math symbol tokens\index{token} must be surrounded by white
space\index{white space}.
%, since there is no context that allows ambiguity to be
%resolved, as is the case with math symbol sequences in some of the Metamath
%statements.
White space should also surround the \texttt{`}
delimiters.

The math mode feature also gives you a quick and easy way to generate
text containing mathematical symbols, independently of the intended
purpose of Metamath.\index{Metamath!using as a math editor} To do this,
simply create your text with grave accents surrounding your formulas,
after making sure that your math symbols are mapped to \LaTeX\ symbols
as described in Appendix~\ref{ASCII}.  It is easier if you start with a
database with predefined symbols such as \texttt{set.mm}.  Use your
grave-quoted math string to replace an existing comment, then typeset
the statement corresponding to that comment following the instructions
from the \texttt{help tex} command in the Metamath program.  You will
then probably want to edit the resulting file with a text editor to fine
tune it to your exact needs.

\subsubsection{Label Mode}\index{label mode}

Outside of math mode, a tilde\index{tilde (\texttt{\char`\~})} \verb/~/
indicates to Metamath's\index{Metamath} output processor that the
token\index{token} that follows (i.e.\ the characters up to the next
white space\index{white space}) represents a statement label or URL.
This formatting mode is called {\bf label mode}\index{label mode}.
If a literal tilde
is desired (outside of math mode) instead of label mode,
use two tildes in a row to represent it.

When generating a \LaTeX\ output file,
the following token will be formatted in \texttt{typewriter}
font, and the tilde removed, to make it stand out from the rest of the text.
This formatting will be applied to all characters after the
tilde up to the first white space\index{white space}.
Whether
or not the token is an actual statement label is not checked, and the
token does not have to have the correct syntax for a label; no error
messages will be produced.  The only effect of the label mode on the
output is that typewriter font will be used for the tokens that are
placed in the \LaTeX\ output file.

When generating {\sc html},
the tokens after the tilde {\em must} be a URL (either http: or https:)
or a valid label.
Error messages will be issued during that output if they aren't.
A hyperlink will be generated to that URL or label.

\subsubsection{Link to bibliographical reference}\index{citation}%
\index{link to bibliographical reference}

Bibliographical references are handled specially when generating
{\sc html} if formatted specially.
Text in the form \texttt{[}{\em author}\texttt{]}
is considered a link to a bibliographical reference.
See \texttt{help html} and \texttt{help write
bibliography} in the Metamath program for more
information.
% \index{\texttt{\char`\[}\ldots\texttt{]} inside comments}
See also Sections~\ref{tcomment} and \ref{wrbib}.

The \texttt{[}{\em author}\texttt{]} notation will also create an entry in
the bibliography cross-reference file generated by \texttt{write
bibliography} (Section~\ref{wrbib}) for {\sc HTML}.
For this to work properly, the
surrounding comment must be formatted as follows:
\begin{quote}
    {\em keyword} {\em label} {\em noise-word}
     \texttt{[}{\em author}\texttt{] p.} {\em number}
\end{quote}
for example
\begin{verbatim}
     Theorem 5.2 of [Monk] p. 223
\end{verbatim}
The {\em keyword} is not case sensitive and must be one of the following:
\begin{verbatim}
     theorem lemma definition compare proposition corollary
     axiom rule remark exercise problem notation example
     property figure postulate equation scheme chapter
\end{verbatim}
The optional {\em label} may consist of more than one
(non-{\em keyword} and non-{\em noise-word}) word.
The optional {\em noise-word} is one of:
\begin{verbatim}
     of in from on
\end{verbatim}
and is  ignored when the cross-reference file is created.  The
\texttt{write
biblio\-graphy} command will perform error checking to verify the
above format.\index{error checking}

\subsubsection{Parentheticals}\label{parentheticals}

The end of a comment may include one or more parenthicals, that is,
statements enclosed in parentheses.
The Metamath program looks for certain parentheticals and can issue
warnings based on them.
They are:

\begin{itemize}
 \item[] \texttt{(Contributed by }
   \textit{NAME}\texttt{,} \textit{DATE}\texttt{.)} -
   document the original contributor's name and the date it was created.
 \item[] \texttt{(Revised by }
   \textit{NAME}\texttt{,} \textit{DATE}\texttt{.)} -
   document the contributor's name and creation date
   that resulted in significant revision
   (not just an automated minimization or shortening).
 \item[] \texttt{(Proof shortened by }
   \textit{NAME}\texttt{,} \textit{DATE}\texttt{.)} -
   document the contributor's name and date that developed a significant
   shortening of the proof (not just an automated minimization).
 \item[] \texttt{(Proof modification is discouraged.)} -
   Note that this proof should normally not be modified.
 \item[] \texttt{(New usage is discouraged.)} -
   Note that this assertion should normally not be used.
\end{itemize}

The \textit{DATE} must be in form YYYY-MMM-DD, where MMM is the
English abbreviation of that month.

\subsubsection{Other markup}\label{othermarkup}
\index{markup notation}

There are other markup notations for generating good-looking results
beyond math mode and label mode:

\begin{itemize}
 \item[]
         \texttt{\char`\_} (underscore)\index{\texttt{\char`\_} inside comments} -
             Italicize text starting from
              {\em space}\texttt{\char`\_}{\em non-space} (i.e.\ \texttt{\char`\_}
              with a space before it and a non-space character after it) until
             the next
             {\em non-space}\texttt{\char`\_}{\em space}.  Normal
             punctuation (e.g.\ a trailing
             comma or period) is ignored when determining {\em space}.
 \item[]
         \texttt{\char`\_} (underscore) - {\em
         non-space}\texttt{\char`\_}{\em non-space-string}, where
          {\em non-space-string} is a string of non-space characters,
         will make {\em non-space-string} become a subscript.
 \item[]
         \texttt{<HTML>}...\texttt{</HTML>} - do not convert
         ``\texttt{<}'' and ``\texttt{>}''
         in the enclosed text when generating {\sc HTML},
         otherwise process markup normally. This allows direct insertion
         of {\sc html} commands.
 \item[]
       ``\texttt{\&}ref\texttt{;}'' - insert an {\sc HTML}
         character reference.
         This is how to insert arbitrary Unicode characters
         (such as accented characters).  Currently only directly supported
         when generating {\sc HTML}.
\end{itemize}

It is recommended that spaces surround any \texttt{\char`\~} and
\texttt{`} tokens in the comment and that a space follow the {\em label}
after a \texttt{\char`\~} token.  This will make global substitutions
to change labels and symbol names much easier and also eliminate any
future chance of ambiguity.  Spaces around these tokens are automatically
removed in the final output to conform with normal rules of punctuation;
for example, a space between a trailing \texttt{`} and a left parenthesis
will be removed.

A good way to become familiar with the markup notation is to look at
the extensive examples in the \texttt{set.mm} database.

\subsection{The Typesetting Comment (\texttt{\$t})}\label{tcomment}

The typesetting comment \texttt{\$t} in the input database file
provides the information necessary to produce good-looking results.
It provides \LaTeX\ and {\sc html}
definitions for math symbols,
as well supporting as some
customization of the generated web page.
If you add a new token to a database, you should also
update the \texttt{\$t} comment information if you want to eventually
create output in \LaTeX\ or {\sc HTML}.
See the
\texttt{set.mm}\index{set theory database (\texttt{set.mm})} database
file for an extensive example of a \texttt{\$t} comment illustrating
many of the features described below.

Programs that do not need to generate good-looking presentation results,
such as programs that only verify Metamath databases,
can completely ignore typesetting comments
and just treat them as normal comments.
Even the Metamath program only consults the
\texttt{\$t} comment information when it needs to generate typeset output
in \LaTeX\ or {\sc HTML}
(e.g., when you open a \LaTeX\ output file with the \texttt{open tex} command).

We will first discuss the syntax of typesetting comments, and then
briefly discuss how this can be used within the Metamath program.

\subsubsection{Typesetting Comment Syntax Overview}

The typesetting comment is identified by the token
\texttt{\$t}\index{\texttt{\$t} comment}\index{typesetting comment} in
the comment, and the typesetting comment ends at the matching
\texttt{\$)}:
\[
  \mbox{\tt \$(\ }
  \mbox{\tt \$t\ }
  \underbrace{
    \mbox{\tt \ \ \ \ \ \ \ \ \ \ \ }
    \cdots
    \mbox{\tt \ \ \ \ \ \ \ \ \ \ \ }
  }_{\mbox{Typesetting definitions go here}}
  \mbox{\tt \ \$)}
\]

There must be one or more white space characters, and only white space
characters, between the \texttt{\$(} that starts the comment
and the \texttt{\$t} symbol,
and the \texttt{\$t} must be followed by one
or more white space characters
(see section \ref{whitespace} for the definition of white space characters).
The typesetting comment continues until the comment end token \texttt{\$)}
(which must be preceded by one or more white space characters).

In version 0.177\index{Metamath!limitations of version 0.177} of the
Metamath program, there may be only one \texttt{\$t} comment in a
database.  This restriction may be lifted in the future to allow
many \texttt{\$t} comments in a database.

Between the \texttt{\$t} symbol (and its following white space) and the
comment end token \texttt{\$)} (and its preceding white space)
is a sequence of one or more typesetting definitions, where
each definition has the form
\textit{definition-type arg arg ... ;}.
Each of the zero or more \textit{arg} values
can be either a typesetting data or a keyword
(what keywords are allowed, and where, depends on the specific
\textit{definition-type}).
The \textit{definition-type}, and each argument \textit{arg},
are separated by one or more white space characters.
Every definition ends in an unquoted semicolon;
white space is not required before the terminating semicolon of a definition.
Each definition should start on a new line.\footnote{This
restriction of the current version of Metamath
(0.177)\index{Metamath!limitations of version 0.177} may be removed
in a future version, but you should do it anyway for readability.}

For example, this typesetting definition:
\begin{center}
 \verb$latexdef "C_" as "\subseteq";$
\end{center}
defines the token \verb$C_$ as the \LaTeX\ symbol $\subseteq$ (which means
``subset'').

Typesetting data is a sequence of one or more quoted strings
(if there is more than one, they are connected by \texttt{\char`\+}).
Often a single quoted string is used to provide data for a definition, using
either double (\texttt{\char`\"}) or single (\texttt{'}) quotation marks.
However,
{\em a quoted string (enclosed in quotation marks) may not include
line breaks.}
A quoted string
may include a quotation mark that matches the enclosing quotes by repeating
the quotation mark twice.  Here are some examples:

\begin{tabu}   { l l }
\textbf{Example} & \textbf{Meaning} \\
\texttt{\char`\"a\char`\"\char`\"b\char`\"} & \texttt{a\char`\"b} \\
\texttt{'c''d'} & \texttt{c'd} \\
\texttt{\char`\"e''f\char`\"} & \texttt{e''f} \\
\texttt{'g\char`\"\char`\"h'} & \texttt{g\char`\"\char`\"h} \\
\end{tabu}

Finally, a long quoted string
may be broken up into multiple quoted strings (considered, as a whole,
a single quoted string) and joined with \texttt{\char`\+}.
You can even use multiple lines as long as a
'+' is at the end of every line except the last one.
The \texttt{\char`\+} should be preceded and followed by at least one
white space character.
Thus, for example,
\begin{center}
 \texttt{\char`\"ab\char`\"\ \char`\+\ \char`\"cd\char`\"
    \ \char`\+\ \\ 'ef'}
\end{center}
is the same as
\begin{center}
 \texttt{\char`\"abcdef\char`\"}
\end{center}

{\sc c}-style comments \texttt{/*}\ldots\texttt{*/} are also supported.

In practice, whenever you add a new math token you will often want to add
typesetting definitions using
\texttt{latexdef}, \texttt{htmldef}, and
\texttt{althtmldef}, as described below.
That way, they will all be up to date.
Of course, whether or not you want to use all three definitions will
depend on how the database is intended to be used.

Below we discuss the different possible \textit{definition-kind} options.
We will show data surrounded by double quotes (in practice they can also use
single quotes and/or be a sequence joined by \texttt{+}s).
We will use specific names for the \textit{data} to make clear what
the data is used for, such as
{\em math-token} (for a Metamath math token,
{\em latex-string} (for string to be placed in a \LaTeX\ stream),
{\em {\sc html}-code} (for {\sc html} code),
and {\em filename} (for a filename).

\subsubsection{Typesetting Comment - \LaTeX}

The syntax for a \LaTeX\ definition is:
\begin{center}
 \texttt{latexdef "}{\em math-token}\texttt{" as "}{\em latex-string}\texttt{";}
\end{center}
\index{latex definitions@\LaTeX\ definitions}%
\index{\texttt{latexdef} statement}

The {\em token-string} and {\em latex-string} are the data
(character strings) for
the token and the \LaTeX\ definition of the token, respectively,

These \LaTeX\ definitions are used by the Metamath program
when it is asked to product \LaTeX output using
the \texttt{write tex} command.

\subsubsection{Typesetting Comment - {\sc html}}

The key kinds of {\sc HTML} definitions have the following syntax:

\vskip 1ex
    \texttt{htmldef "}{\em math-token}\texttt{" as "}{\em
    {\sc html}-code}\texttt{";}\index{\texttt{htmldef} statement}
                    \ \ \ \ \ \ldots

    \texttt{althtmldef "}{\em math-token}\texttt{" as "}{\em
{\sc html}-code}\texttt{";}\index{\texttt{althtmldef} statement}

                    \ \ \ \ \ \ldots

Note that in {\sc HTML} there are two possible definitions for math tokens.
This feature is useful when
an alternate representation of symbols is desired, for example one that
uses Unicode entities and another uses {\sc gif} images.

There are many other typesetting definitions that can control {\sc HTML}.
These include:

\vskip 1ex

    \texttt{htmldef "}{\em math-token}\texttt{" as "}{\em {\sc
    html}-code}\texttt{";}

    \texttt{htmltitle "}{\em {\sc html}-code}\texttt{";}%
\index{\texttt{htmltitle} statement}

    \texttt{htmlhome "}{\em {\sc html}-code}\texttt{";}%
\index{\texttt{htmlhome} statement}

    \texttt{htmlvarcolor "}{\em {\sc html}-code}\texttt{";}%
\index{\texttt{htmlvarcolor} statement}

    \texttt{htmlbibliography "}{\em filename}\texttt{";}%
\index{\texttt{htmlbibliography} statement}

\vskip 1ex

\noindent The \texttt{htmltitle} is the {\sc html} code for a common
title, such as ``Metamath Proof Explorer.''  The \texttt{htmlhome} is
code for a link back to the home page.  The \texttt{htmlvarcolor} is
code for a color key that appears at the bottom of each proof.  The file
specified by {\em filename} is an {\sc html} file that is assumed to
have a \texttt{<A NAME=}\ldots\texttt{>} tag for each bibiographic
reference in the database comments.  For example, if
\texttt{[Monk]}\index{\texttt{\char`\[}\ldots\texttt{]} inside comments}
occurs in the comment for a theorem, then \texttt{<A NAME='Monk'>} must
be present in the file; if not, a warning message is given.

Associated with
\texttt{althtmldef}
are the statements
\vskip 1ex

    \texttt{htmldir "}{\em
      directoryname}\texttt{";}\index{\texttt{htmldir} statement}

    \texttt{althtmldir "}{\em
     directoryname}\texttt{";}\index{\texttt{althtmldir} statement}

\vskip 1ex
\noindent giving the directories of the {\sc gif} and Unicode versions
respectively; their purpose is to provide cross-linking between the
two versions in the generated web pages.

When two different types of pages need to be produced from a single
database, such as the Hilbert Space Explorer that extends the Metamath
Proof Explorer, ``extended'' variables may be declared in the
\texttt{\$t} comment:
\vskip 1ex

    \texttt{exthtmltitle "}{\em {\sc html}-code}\texttt{";}%
\index{\texttt{exthtmltitle} statement}

    \texttt{exthtmlhome "}{\em {\sc html}-code}\texttt{";}%
\index{\texttt{exthtmlhome} statement}

    \texttt{exthtmlbibliography "}{\em filename}\texttt{";}%
\index{\texttt{exthtmlbibliography} statement}

\vskip 1ex
\noindent When these are declared, you also must declare
\vskip 1ex

    \texttt{exthtmllabel "}{\em label}\texttt{";}%
\index{\texttt{exthtmllabel} statement}

\vskip 1ex \noindent that identifies the database statement where the
``extended'' section of the database starts (in our example, where the
Hilbert Space Explorer starts).  During the generation of web pages for
that starting statement and the statements after it, the {\sc html} code
assigned to \texttt{exthtmltitle} and \texttt{exthtmlhome} is used
instead of that assigned to \texttt{htmltitle} and \texttt{htmlhome},
respectively.

\begin{sloppy}
\subsection{Additional Information Com\-ment (\texttt{\$j})} \label{jcomment}
\end{sloppy}

The additional information comment, aka the
\texttt{\$j}\index{\texttt{\$j} comment}\index{additional information comment}
comment,
provides a way to add additional structured information that can
be optionally parsed by systems.

The additional information comment is parsed the same way as the
typesetting comment (\texttt{\$t}) (see section \ref{tcomment}).
That is,
the additional information comment begins with the token
\texttt{\$j} within a comment,
and continues until the comment close \texttt{\$)}.
Within an additional information comment is a sequence of one or more
commands of the form \texttt{command arg arg ... ;}
where each of the zero or more \texttt{arg} values
can be either a quoted string or a keyword.
Note that every command ends in an unquoted semicolon.
If a verifier is parsing an additional information comment, but
doesn't recognize a particular command, it must skip the command
by finding the end of the command (an unquoted semicolon).

A database may have 0 or more additional information comments.
Note, however, that a verifier may ignore these comments entirely or only
process certain commands in an additional information comment.
The \texttt{mmj2} verifier supports many commands in additional information
comments.
We encourage systems that process additional information comments
to coordinate so that they will use the same command for the same effect.

Examples of additional information comments with various commands
(from the \texttt{set.mm} database) are:

\begin{itemize}
   \item Define the syntax and logical typecodes,
     and declare that our grammar is
     unambiguous (verifiable using the KLR parser, with compositing depth 5).
\begin{verbatim}
  $( $j
    syntax 'wff';
    syntax '|-' as 'wff';
    unambiguous 'klr 5';
  $)
\end{verbatim}

   \item Register $\lnot$ and $\rightarrow$ as primitive expressions
           (lacking definitions).
\begin{verbatim}
  $( $j primitive 'wn' 'wi'; $)
\end{verbatim}

   \item There is a special justification for \texttt{df-bi}.
\begin{verbatim}
  $( $j justification 'bijust' for 'df-bi'; $)
\end{verbatim}

   \item Register $\leftrightarrow$ as an equality for its type (wff).
\begin{verbatim}
  $( $j
    equality 'wb' from 'biid' 'bicomi' 'bitri';
    definition 'dfbi1' for 'wb';
  $)
\end{verbatim}

   \item Theorem \texttt{notbii} is the congruence law for negation.
\begin{verbatim}
  $( $j congruence 'notbii'; $)
\end{verbatim}

   \item Add \texttt{setvar} as a typecode.
\begin{verbatim}
  $( $j syntax 'setvar'; $)
\end{verbatim}

   \item Register $=$ as an equality for its type (\texttt{class}).
\begin{verbatim}
  $( $j equality 'wceq' from 'eqid' 'eqcomi' 'eqtri'; $)
\end{verbatim}

\end{itemize}


\subsection{Including Other Files in a Metamath Source File} \label{include}
\index{\texttt{\$[} and \texttt{\$]} auxiliary keywords}

The keywords \texttt{\$[} and \texttt{\$]} specify a file to be
included\index{included file}\index{file inclusion} at that point in a
Metamath\index{Metamath} source file\index{source file}.  The syntax for
including a file is as follows:
\begin{center}
\texttt{\$[} {\em file-name} \texttt{\$]}
\end{center}

The {\em file-name} should be a single token\index{token} with the same syntax
as a math symbol (i.e., all 93 non-whitespace
printable characters other than \texttt{\$} are
allowed, subject to the file-naming limitations of your operating system).
Comments may appear between the \texttt{\$[} and \texttt{\$]} keywords.  Included
files may include other files, which may in turn include other files, and so
on.

For example, suppose you want to use the set theory database as the starting
point for your own theory.  The first line in your file could be
\begin{center}
\texttt{\$[ set.mm \$]}
\end{center} All of the information (axioms, theorems,
etc.) in \texttt{set.mm} and any files that {\em it} includes will become
available for you to reference in your file. This can help make your work more
modular. A drawback to including files is that if you change the name of a
symbol or the label of a statement, you must also remember to update any
references in any file that includes it.


The naming conventions for included files are the same as those of your
operating system.\footnote{On the Macintosh, prior to Mac OS X,
 a colon is used to separate disk
and folder names from your file name.  For example, {\em volume}\texttt{:}{\em
file-name} refers to the root directory, {\em volume}\texttt{:}{\em
folder-name}\texttt{:}{\em file-name} refers to a folder in root, and {\em
volume}\texttt{:}{\em folder-name}\texttt{:}\ldots\texttt{:}{\em file-name} refers to a
deeper folder.  A simple {\em file-name} refers to a file in the folder from
which you launch the Metamath application.  Under Mac OS X and later,
the Metamath program is run under the Terminal application, which
conforms to Unix naming conventions.}\index{Macintosh file
names}\index{file names!Macintosh}\label{includef} For compatibility among
operating systems, you should keep the file names as simple as possible.  A
good convention to use is {\em file}\texttt{.mm} where {\em file} is eight
characters or less, in lower case.

There is no limit to the nesting depth of included files.  One thing that you
should be aware of is that if two included files themselves include a common
third file, only the {\em first} reference to this common file will be read
in.  This allows you to include two or more files that build on a common
starting file without having to worry about label and symbol conflicts that
would occur if the common file were read in more than once.  (In fact, if a
file includes itself, the self-reference will be ignored, although of course
it would not make any sense to do that.)  This feature also means, however,
that if you try to include a common file in several inner blocks, the result
might not be what you expect, since only the first reference will be replaced
with the included file (unlike the include statement in most other computer
languages).  Thus you would normally include common files only in the
outermost block\index{outermost block}.

\subsection{Compressed Proof Format}\label{compressed1}\index{compressed
proof}\index{proof!compressed}

The proof notation presented in Section~\ref{proof} is called a
{\bf normal proof}\index{normal proof}\index{proof!normal} and in principle is
sufficient to express any proof.  However, proofs often contain steps and
subproofs that are identical.  This is particularly true in typical
Metamath\index{Metamath} applications, because Metamath requires that the math
symbol sequence (usually containing a formula) at each step be separately
constructed, that is, built up piece by piece. As a result, a lot of
repetition often results.  The {\bf compressed proof} format allows Metamath
to take advantage of this redundancy to shorten proofs.

The specification for the compressed proof format is given in
Appen\-dix~\ref{compressed}.

Normally you need not concern yourself with the details of the compressed
proof format, since the Metamath program will allow you to convert from
the normal format to the compressed format with ease, and will also
automatically convert from the compressed format when proofs are displayed.
The overall structure of the compressed format is as follows:
\begin{center}
  \texttt{\$= ( } {\em label-list} \texttt{) } {\em compressed-proof\ }\ \texttt{\$.}
\end{center}
\index{\texttt{\$=} keyword}
The first \texttt{(} serves as a flag to Metamath that a compressed proof
follows.  The {\em label-list} includes all statements referred to by the
proof except the mandatory hypotheses\index{mandatory hypothesis}.  The {\em
compressed-proof} is a compact encoding of the proof, using upper-case
letters, and can be thought of as a large integer in base 26.  White
space\index{white space} inside a {\em compressed-proof} is
optional and is ignored.

It is important to note that the order of the mandatory hypotheses of
the statement being proved must not be changed if the compressed proof
format is used, otherwise the proof will become incorrect.  The reason
for this is that the mandatory hypotheses are not mentioned explicitly
in the compressed proof in order to make the compression more efficient.
If you wish to change the order of mandatory hypotheses, you must first
convert the proof back to normal format using the \texttt{save proof
{\em statement} /normal}\index{\texttt{save proof} command} command.
Later, you can go back to compressed format with \texttt{save proof {\em
statement} /compressed}.

During error checking with the \texttt{verify proof} command, an error
found in a compressed proof may point to a character in {\em
compressed-proof}, which may not be very meaningful to you.  In this
case, try to \texttt{save proof /normal} first, then do the
\texttt{verify proof} again.  In general, it is best to make sure a
proof is correct before saving it in compressed format, because severe
errors are less likely to be recoverable than in normal format.

\subsection{Specifying Unknown Proofs or Subproofs}\label{unknown}

In a proof under development, any step or subproof that is not yet known
may be represented with a single \texttt{?}.  For the purposes of
parsing the proof, the \texttt{?}\ \index{\texttt{]}@\texttt{?}\ inside
proofs} will push a single entry onto the RPN stack just as if it were a
hypothesis.  While developing a proof with the Proof
Assistant\index{Proof Assistant}, a partially developed proof may be
saved with the \texttt{save new{\char`\_}proof}\index{\texttt{save
new{\char`\_}proof} command} command, and \texttt{?}'s will be placed at
the appropriate places.

All \texttt{\$p}\index{\texttt{\$p} statement} statements must have
proofs, even if they are entirely unknown.  Before creating a proof with
the Proof Assistant, you should specify a completely unknown proof as
follows:
\begin{center}
  {\em label} \texttt{\$p} {\em statement} \texttt{\$= ?\ \$.}
\end{center}
\index{\texttt{\$=} keyword}
\index{\texttt{]}@\texttt{?}\ inside proofs}

The \texttt{verify proof}\index{\texttt{verify proof} command} command
will check the known portions of a partial proof for errors, but will
warn you that the statement has not been proved.

Note that partially developed proofs may be saved in compressed format
if desired.  In this case, you will see one or more \texttt{?}'s in the
{\em compressed-proof} part.\index{compressed
proof}\index{proof!compressed}

\section{Axioms vs.\ Definitions}\label{definitions}

The \textit{basic}
Metamath\index{Metamath} language and program
make no distinction\index{axiom vs.\
definition} between axioms\index{axiom} and
definitions.\index{definition} The \texttt{\$a}\index{\texttt{\$a}
statement} statement is used for both.  At first, this may seem
puzzling.  In the minds of many mathematicians, the distinction is
clear, even obvious, and hardly worth discussing.  A definition is
considered to be merely an abbreviation that can be replaced by the
expression for which it stands; although unless one actually does this,
to be precise then one should say that a theorem\index{theorem} is a
consequence of the axioms {\em and} the definitions that are used in the
formulation of the theorem \cite[p.~20]{Behnke}.\index{Behnke, H.}

\subsection{What is a Definition?}

What is a definition?  In its simplest form, a definition introduces a new
symbol and provides an unambiguous rule to transform an expression containing
the new symbol to one without it.  The concept of a ``proper
definition''\index{proper definition}\index{definition!proper} (as opposed to
a creative definition)\index{creative definition}\index{definition!creative}
that is usually agreed upon is (1) the definition should not strengthen the
language and (2) any symbols introduced by the definition should be eliminable
from the language \cite{Nemesszeghy}\index{Nemesszeghy, E. Z.}.  In other
words, they are mere typographical conveniences that do not belong to the
system and are theoretically superfluous.  This may seem obvious, but in fact
the nature of definitions can be subtle, sometimes requiring difficult
metatheorems to establish that they are not creative.

A more conservative stance was taken by logician S.
Le\'{s}niewski.\index{Le\'{s}niewski, S.}
\begin{quote}
Le\'{s}niewski
regards definitions as theses of the system.  In this respect they do
not differ either from the axioms or from theorems, i.e.\ from the
theses added to the system on the basis of the rule of substitution or
the rule of detachment [modus ponens].  Once definitions have been
accepted as theses of the system, it becomes necessary to consider them
as true propositions in the same sense in which axioms are true
\cite{Lejewski}.
\end{quote}\index{Lejewski, Czeslaw}

Let us look at some simple examples of definitions in propositional
calculus.  Consider the definition of logical {\sc or}
(disjunction):\index{disjunction ($\vee$)} ``$P\vee Q$ denotes $\neg P
\rightarrow Q$ (not $P$ implies $Q$).''  It is very easy to recognize a
statement making use of this definition, because it introduces the new
symbol $\vee$ that did not previously exist in the language.  It is easy
to see that no new theorems of the original language will result from
this definition.

Next, consider a definition that eliminates parentheses:  ``$P
\rightarrow Q\rightarrow R$ denotes $P\rightarrow (Q \rightarrow R)$.''
This is more subtle, because no new symbols are introduced.  The reason
this definition is considered proper is that no new symbol sequences
that are valid wffs (well-formed formulas)\index{well-formed formula
(wff)} in the original language will result from the definition, since
``$P \rightarrow Q\rightarrow R$'' is not a wff in the original
language.  Here, we implicitly make use of the fact that there is a
decision procedure that allows us to determine whether or not a symbol
sequence is a wff, and this fact allows us to use symbol sequences that
are not wffs to represent other things (such as wffs) by means of the
definition.  However, to justify the definition as not being creative we
need to prove that ``$P \rightarrow Q\rightarrow R$'' is in fact not a
wff in the original language, and this is more difficult than in the
case where we simply introduce a new symbol.

%Now let's take this reasoning to an extreme.  Propositional calculus is a
%decidable theory,\footnote{This means that a mechanical algorithm exists to
%determine whether or not a wff is a theorem.} so in principle we could make use
%of symbol sequences that are not theorems to represent other things (say, to
%encode actual theorems in a more compact way).  For example, let us extend the
%language by defining a wff ``$P$'' in the extended language as the theorem
%``$P\rightarrow P$''\footnote{This is one of the first theorems proved in the
%Metamath database \texttt{set.mm}.}\index{set
%theory database (\texttt{set.mm})} in the original language whenever ``$P$'' is
%not a theorem in the original language.  In the extended language, any wff
%``$Q$'' thus represents a theorem; to find out what theorem (in the original
%language) ``$Q$'' represents, we determine whether ``$Q$'' is a theorem in the
%original language (before the definition was introduced).  If so, we're done; if
%not, we replace ``$Q$'' by ``$Q\rightarrow Q$'' to eliminate the definition.
%This definition is therefore eliminable, and it does not ``strengthen'' the
%language because any wff that is not a theorem is not in the set of statements
%provable in the original language and thus is available for use by definitions.
%
%Of course, a definition such as this would render practically useless the
%communication of theorems of propositional calculus; but
%this is just a human shortcoming, since we can't always easily discern what is
%and is not a theorem by inspection.  In fact, the extended theory with this
%definition has no more and no less information than the original theory; it just
%expresses certain theorems of the form ``$P\rightarrow P$''
%in a more compact way.
%
%The point here is that what constitutes a proper definition is a matter of
%judgment about whether a symbol sequence can easily be recognized by a human
%as invalid in some sense (for example, not a wff); if so, the symbol sequence
%can be appropriated for use by a definition in order to make the extended
%language more compact.  Metamath\index{Metamath} lacks the ability to make this
%judgment, since as far as Metamath is concerned the definition of a wff, for
%example, is arbitrary.  You define for Metamath how wffs\index{well-formed
%formula (wff)} are constructed according to your own preferred style.  The
%concept of a wff may not even exist in a given formal system\index{formal
%system}.  Metamath treats all definitions as if they were new axioms, and it
%is up to the human mathematician to judge whether the definition is ``proper''
%'\index{proper definition}\index{definition!proper} in some agreed-upon way.

What constitutes a definition\index{definition} versus\index{axiom vs.\
definition} an axiom\index{axiom} is sometimes arbitrary in mathematical
literature.  For example, the connectives $\vee$ ({\sc or}), $\wedge$
({\sc and}), and $\leftrightarrow$ (equivalent to) in propositional
calculus are usually considered defined symbols that can be used as
abbreviations for expressions containing the ``primitive'' connectives
$\rightarrow$ and $\neg$.  This is the way we treat them in the standard
logic and set theory database \texttt{set.mm}\index{set theory database
(\texttt{set.mm})}.  However, the first three connectives can also be
considered ``primitive,'' and axiom systems have been devised that treat
all of them as such.  For example,
\cite[p.~35]{Goodstein}\index{Goodstein, R. L.} presents one with 15
axioms, some of which in fact coincide with what we have chosen to call
definitions in \texttt{set.mm}.  In certain subsets of classical
propositional calculus, such as the intuitionist
fragment\index{intuitionism}, it can be shown that one cannot make do
with just $\rightarrow$ and $\neg$ but must treat additional connectives
as primitive in order for the system to make sense.\footnote{Two nice
systems that make the transition from intuitionistic and other weak
fragments to classical logic just by adding axioms are given in
\cite{Robinsont}\index{Robinson, T. Thacher}.}

\subsection{The Approach to Definitions in \texttt{set.mm}}

In set theory, recursive definitions define a newly introduced symbol in
terms of itself.
The justification of recursive definitions, using
several ``recursion theorems,'' is usually one of the first
sophisticated proofs a student encounters when learning set theory, and
there is a significant amount of implicit metalogic behind a recursive
definition even though the definition itself is typically simple to
state.

Metamath itself has no built-in technical limitation that prevents
multiple-part recursive definitions in the traditional textbook style.
However, because the recursive definition requires advanced metalogic
to justify, eliminating a recursive definition is very difficult and
often not even shown in textbooks.

\subsubsection{Direct definitions instead of recursive definitions}

It is, however, possible to substitute one kind of complexity
for another.  We can eliminate the need for metalogical justification by
defining the operation directly with an explicit (but complicated)
expression, then deriving the recursive definition directly as a
theorem, using a recursion theorem ``in reverse.''
The elimination
of a direct definition is a matter of simple mechanical substitution.
We do this in
\texttt{set.mm}, as follows.

In \texttt{set.mm} our goal was to introduce almost all definitions in
the form of two expressions connected by either $\leftrightarrow$ or
$=$, where the thing being defined does not appear on the right hand
side.  Quine calls this form ``a genuine or direct definition'' \cite[p.
174]{Quine}\index{Quine, Willard Van Orman}, which makes the definitions
very easy to eliminate and the metalogic\index{metalogic} needed to
justify them as simple as possible.
Put another way, we had a goal of being able to
eliminate all definitions with direct mechanical substitution and to
verify easily the soundness of the definitions.

\subsubsection{Example of direct definitions}

We achieved this goal in almost all cases in \texttt{set.mm}.
Sometimes this makes the definitions more complex and less
intuitive.
For example, the traditional way to define addition of
natural numbers is to define an operation called {\em
successor}\index{successor} (which means ``plus one'' and is denoted by
``${\rm suc}$''), then define addition recursively\index{recursive
definition} with the two definitions $n + 0 = n$ and $m + {\rm suc}\,n =
{\rm suc} (m + n)$.  Although this definition seems simple and obvious,
the method to eliminate the definition is not obvious:  in the second
part of the definition, addition is defined in terms of itself.  By
eliminating the definition, we don't mean repeatedly applying it to
specific $m$ and $n$ but rather showing the explicit, closed-form
set-theoretical expression that $m + n$ represents, that will work for
any $m$ and $n$ and that does not have a $+$ sign on its right-hand
side.  For a recursive definition like this not to be circular
(creative), there are some hidden, underlying assumptions we must make,
for example that the natural numbers have a certain kind of order.

In \texttt{set.mm} we chose to start with the direct (though complex and
nonintuitive) definition then derive from it the standard recursive
definition.
For example, the closed-form definition used in \texttt{set.mm}
for the addition operation on ordinals\index{ordinal
addition}\index{addition!of ordinals} (of which natural numbers are a
subset) is

\setbox\startprefix=\hbox{\tt \ \ df-oadd\ \$a\ }
\setbox\contprefix=\hbox{\tt \ \ \ \ \ \ \ \ \ \ \ \ \ }
\startm
\m{\vdash}\m{+_o}\m{=}\m{(}\m{x}\m{\in}\m{{\rm On}}\m{,}\m{y}\m{\in}\m{{\rm
On}}\m{\mapsto}\m{(}\m{{\rm rec}}\m{(}\m{(}\m{z}\m{\in}\m{{\rm
V}}\m{\mapsto}\m{{\rm suc}}\m{z}\m{)}\m{,}\m{x}\m{)}\m{`}\m{y}\m{)}\m{)}
\endm
\noindent which depends on ${\rm rec}$.

\subsubsection{Recursion operators}

The above definition of \texttt{df-oadd} depends on the definition of
${\rm rec}$, a ``recursion operator''\index{recursion operator} with
the definition \texttt{df-rdg}:

\setbox\startprefix=\hbox{\tt \ \ df-rdg\ \$a\ }
\setbox\contprefix=\hbox{\tt \ \ \ \ \ \ \ \ \ \ \ \ }
\startm
\m{\vdash}\m{{\rm
rec}}\m{(}\m{F}\m{,}\m{I}\m{)}\m{=}\m{\mathrm{recs}}\m{(}\m{(}\m{g}\m{\in}\m{{\rm
V}}\m{\mapsto}\m{{\rm if}}\m{(}\m{g}\m{=}\m{\varnothing}\m{,}\m{I}\m{,}\m{{\rm
if}}\m{(}\m{{\rm Lim}}\m{{\rm dom}}\m{g}\m{,}\m{\bigcup}\m{{\rm
ran}}\m{g}\m{,}\m{(}\m{F}\m{`}\m{(}\m{g}\m{`}\m{\bigcup}\m{{\rm
dom}}\m{g}\m{)}\m{)}\m{)}\m{)}\m{)}\m{)}
\endm

\noindent which can be further broken down with definitions shown in
Section~\ref{setdefinitions}.

This definition of ${\rm rec}$
defines a recursive definition generator on ${\rm On}$ (the class of ordinal
numbers) with characteristic function $F$ and initial value $I$.
This operation allows us to define, with
compact direct definitions, functions that are usually defined in
textbooks with recursive definitions.
The price paid with our approach
is the complexity of our ${\rm rec}$ operation
(especially when {\tt df-recs} that it is built on is also eliminated).
But once we get past this hurdle, definitions that would otherwise be
recursive become relatively simple, as in for example {\tt oav}, from
which we prove the recursive textbook definition as theorems {\tt oa0}, {\tt
oasuc}, and {\tt oalim} (with the help of theorems {\tt rdg0}, {\tt rdgsuc},
and {\tt rdglim2a}).  We can also restrict the ${\rm rec}$ operation to
define otherwise recursive functions on the natural numbers $\omega$; see {\tt
fr0g} and {\tt frsuc}.  Our ${\rm rec}$ operation apparently does not appear
in published literature, although closely related is Definition 25.2 of
[Quine] p. 177, which he uses to ``turn...a recursion into a genuine or
direct definition" (p. 174).  Note that the ${\rm if}$ operations (see
{\tt df-if}) select cases based on whether the domain of $g$ is zero, a
successor, or a limit ordinal.

An important use of this definition ${\rm rec}$ is in the recursive sequence
generator {\tt df-seq} on the natural numbers (as a subset of the
complex infinite sequences such as the factorial function {\tt df-fac} and
integer powers {\tt df-exp}).

The definition of ${\rm rec}$ depends on ${\rm recs}$.
New direct usage of the more powerful (and more primitive) ${\rm recs}$
construct is discouraged, but it is available when needed.
This
defines a function $\mathrm{recs} ( F )$ on ${\rm On}$, the class of ordinal
numbers, by transfinite recursion given a rule $F$ which sets the next
value given all values so far.
Unlike {\tt df-rdg} which restricts the
update rule to use only the previous value, this version allows the
update rule to use all previous values, which is why it is described
as ``strong,'' although it is actually more primitive.  See {\tt
recsfnon} and {\tt recsval} for the primary contract of this definition.
It is defined as:

\setbox\startprefix=\hbox{\tt \ \ df-recs\ \$a\ }
\setbox\contprefix=\hbox{\tt \ \ \ \ \ \ \ \ \ \ \ \ \ }
\startm
\m{\vdash}\m{\mathrm{recs}}\m{(}\m{F}\m{)}\m{=}\m{\bigcup}\m{\{}\m{f}\m{|}\m{\exists}\m{x}\m{\in}\m{{\rm
On}}\m{(}\m{f}\m{{\rm
Fn}}\m{x}\m{\wedge}\m{\forall}\m{y}\m{\in}\m{x}\m{(}\m{f}\m{`}\m{y}\m{)}\m{=}\m{(}\m{F}\m{`}\m{(}\m{f}\m{\restriction}\m{y}\m{)}\m{)}\m{)}\m{\}}
\endm

\subsubsection{Closing comments on direct definitions}

From these direct definitions the simpler, more
intuitive recursive definition is derived as a set of theorems.\index{natural
number}\index{addition}\index{recursive definition}\index{ordinal addition}
The end result is the same, but we completely eliminate the rather complex
metalogic that justifies the recursive definition.

Recursive definitions are often considered more efficient and intuitive than
direct ones once the metalogic has been learned or possibly just accepted as
correct.  However, it was felt that direct definition in \texttt{set.mm}
maximizes rigor by minimizing metalogic.  It can be eliminated effortlessly,
something that is difficult to do with a recursive definition.

Again, Metamath itself has no built-in technical limitation that prevents
multiple-part recursive definitions in the traditional textbook style.
Instead, our goal is to eliminate all definitions with
direct mechanical substitution and to verify easily the soundness of
definitions.

\subsection{Adding Constraints on Definitions}

The basic Metamath language and the Metamath program do
not have any built-in constraints on definitions, since they are just
\$a statements.

However, nothing prevents a verification system from verifying
additional rules to impose further limitations on definitions.
For example, the \texttt{mmj2}\index{mmj2} program
supports various kinds of
additional information comments (see section \ref{jcomment}).
One of their uses is to optionally verify additional constraints,
including constraints to verify that definitions meet certain
requirements.
These additional checks are required by the
continuous integration (CI)\index{continuous integration (CI)}
checks of the
\texttt{set.mm}\index{set theory database (\texttt{set.mm})}%
\index{Metamath Proof Explorer}
database.
This approach enables us to optionally impose additional requirements
on definitions if we wish, without necessarily imposing those rules on
all databases or requiring all verification systems to implement them.
In addition, this allows us to impose specialized constraints tailored
to one database while not requiring other systems to implement
those specialized constraints.

We impose two constraints on the
\texttt{set.mm}\index{set theory database (\texttt{set.mm})}%
\index{Metamath Proof Explorer} database
via the \texttt{mmj2}\index{mmj2} program that are worth discussing here:
a parse check and a definition soundness check.

% On February 11, 2019 8:32:32 PM EST, saueran@oregonstate.edu wrote:
% The following addition to the end of set.mm is accepted by the mmj2
% parser and definition checker and the metamath verifier(at least it was
% when I checked, you should check it too), and creates a contradiction by
% proving the theorem |- ph.
% ${
% wleftp $a wff ( ( ph ) $.
% wbothp $a wff ( ph ) $.
% df-leftp $a |- ( ( ( ph ) <-> -. ph ) $.
% df-bothp $a |- ( ( ph ) <-> ph ) $.
% anything $p |- ph $=
%   ( wbothp wn wi wleftp df-leftp biimpi df-bothp mpbir mpbi simplim ax-mp)
%   ABZAMACZDZCZMOEZOCQAEZNDZRNAFGSHIOFJMNKLAHJ $.
% $}
%
% This particular problem is countered by enabling, within mmj2,
% SetParser,mmj.verify.LRParser

First,
we enable a parse check in \texttt{mmj2} (through its
\texttt{SetParser} command) that requires that all new definitions
must \textit{not} create an ambiguous parse for a KLR(5) parser.
This prevents some errors such as definitions with imbalanced parentheses.

Second, we run a definition soundness check specific to
\texttt{set.mm} or databases similar to it.
(through the \texttt{definitionCheck} macro).
Some \texttt{\$a} statements (including all ax-* statemnets)
are excluded from these checks, as they will
always fail this simple check,
but they are appropriate for most definitions.
This check imposes a set of additional rules:

\begin{enumerate}

\item New definitions must be introduced using $=$ or $\leftrightarrow$.

\item No \texttt{\$a} statement introduced before this one is allowed to use the
symbol being defined in this definition, and the definition is not
allowed to use itself (except once, in the definiendum).

\item Every variable in the definiens should not be distinct

\item Every dummy variable in the definiendum
are required to be distinct from each other and from variables in
the definiendum.
To determine this, the system will look for a "justification" theorem
in the database, and if it is not there, attempt to briefly prove
$( \varphi \rightarrow \forall x \varphi )$  for each dummy variable x.

\item Every dummy variable should be a set variable,
unless there is a justification theorem available.

\item Every dummy variable must be bound
(if the system cannot determine this a justification theorem must be
provided).

\end{enumerate}

\subsection{Summary of Approach to Definitions}

In short, when being rigorous it turns out that
definitions can be subtle, sometimes requiring difficult
metatheorems to establish that they are not creative.

Instead of building such complications into the Metamath language itself,
the basic Metmath language and program simply treat traditional
axioms and definitions as the same kind of \texttt{\$a} statement.
We have then built various tools to enable people to
verify additional conditions as their creators believe is appropriate
for those specific databases, without complicating the Metamath foundations.

\chapter{The Metamath Program}\label{commands}

This chapter provides a reference manual for the
Metamath program.\index{Metamath!commands}

Current instructions for obtaining and installing the Metamath program
can be found at the \url{http://metamath.org} web site.
For Windows, there is a pre-compiled version called
\texttt{metamath.exe}.  For Unix, Linux, and Mac OS X
(which we will refer to collectively as ``Unix''), the Metamath program
can be compiled from its source code with the command
\begin{verbatim}
gcc *.c -o metamath
\end{verbatim}
using the \texttt{gcc} {\sc c} compiler available on those systems.

In the command syntax descriptions below, fields enclosed in square
brackets [\ ] are optional.  File names may be optionally enclosed in
single or double quotes.  This is useful if the file name contains
spaces or
slashes (\texttt{/}), such as in Unix path names, \index{Unix file
names}\index{file names!Unix} that might be confused with Metamath
command qualifiers.\index{Unix file names}\index{file names!Unix}

\section{Invoking Metamath}

Unix, Linux, and Mac OS X
have a command-line interface called the {\em
bash shell}.  (In Mac OS X, select the Terminal application from
Applications/Utilities.)  To invoke Metamath from the bash shell prompt,
assuming that the Metamath program is in the current directory, type
\begin{verbatim}
bash$ ./metamath
\end{verbatim}

To invoke Metamath from a Windows DOS or Command Prompt, assuming that
the Metamath program is in the current directory (or in a directory
included in the Path system environment variable), type
\begin{verbatim}
C:\metamath>metamath
\end{verbatim}

To use command-line arguments at invocation, the command-line arguments
should be a list of Metamath commands, surrounded by quotes if they
contain spaces.  In Windows, the surrounding quotes must be double (not
single) quotes.  For example, to read the database file \texttt{set.mm},
verify all proofs, and exit the program, type (under Unix)
\begin{verbatim}
bash$ ./metamath 'read set.mm' 'verify proof *' exit
\end{verbatim}
Note that in Unix, any directory path with \texttt{/}'s must be
surrounded by quotes so Metamath will not interpret the \texttt{/} as a
command qualifier.  So if \texttt{set.mm} is in the \texttt{/tmp}
directory, use for the above example
\begin{verbatim}
bash$ ./metamath 'read "/tmp/set.mm"' 'verify proof *' exit
\end{verbatim}

For convenience, if the command-line has one argument and no spaces in
the argument, the command is implicitly assumed to be \texttt{read}.  In
this one special case, \texttt{/}'s are not interpreted as command
qualifiers, so you don't need quotes around a Unix file name.  Thus
\begin{verbatim}
bash$ ./metamath /tmp/set.mm
\end{verbatim}
and
\begin{verbatim}
bash$ ./metamath "read '/tmp/set.mm'"
\end{verbatim}
are equivalent.


\section{Controlling Metamath}

The Metamath program was first developed on a {\sc vax/vms} system, and
some aspects of its command line behavior reflect this heritage.
Hopefully you will find it reasonably user-friendly once you get used to
it.

Each command line is a sequence of English-like words separated by
spaces, as in \texttt{show settings}.  Command words are not case
sensitive, and only as many letters are needed as are necessary to
eliminate ambiguity; for example, \texttt{sh se} would work for the
command \texttt{show settings}.  In some cases arguments such as file
names, statement labels, or symbol names are required; these are
case-sensitive (although file names may not be on some operating
systems).

A command line is entered by typing it in then pressing the {\em return}
({\em enter}) key.  To find out what commands are available, type
\texttt{?} at the \texttt{MM>} prompt.  To find out the choices at any
point in a command, press {\em return} and you will be prompted for
them.  The default choice (the one selected if you just press {\em
return}) is shown in brackets (\texttt{<>}).

You may also type \texttt{?} in place of a command word to force
Metamath to tell you what the choices are.  The \texttt{?} method won't
work, though, if a non-keyword argument such as a file name is expected
at that point, because the program will think that \texttt{?} is the
value of the argument.

Some commands have one or more optional qualifiers which modify the
behavior of the command.  Qualifiers are preceded by a slash
(\texttt{/}), such as in \texttt{read set.mm / verify}.  Spaces are
optional around the \texttt{/}.  If you need to use a space or
slash in a command
argument, as in a Unix file name, put single or double quotes around the
command argument.

The \texttt{open log} command will save everything you see on the
screen and is useful to help you recover should something go wrong in a
proof, or if you want to document a bug.

If a command responds with more than a screenful, you will be
prompted to \texttt{<return> to continue, Q to quit, or S to scroll to
end}.  \texttt{Q} or \texttt{q} (not case-sensitive) will complete the
command internally but will suppress further output until the next
\texttt{MM>} prompt.  \texttt{s} will suppress further pausing until the
next \texttt{MM>} prompt.  After the first screen, you are also
presented with the choice of \texttt{b} to go back a screenful.  Note
that \texttt{b} may also be entered at the \texttt{MM>} prompt
immediately after a command to scroll back through the output of that
command.

A command line enclosed in quotes is executed by your operating system.
See Section~\ref{oscmd}.

{\em Warning:} Pressing {\sc ctrl-c} will abort the Metamath program
unconditionally.  This means any unsaved work will be lost.


\subsection{\texttt{exit} Command}\index{\texttt{exit} command}

Syntax:  \texttt{exit} [\texttt{/force}]

This command exits from Metamath.  If there have been changes to the
source with the \texttt{save proof} or \texttt{save new{\char`\_}proof}
commands, you will be given an opportunity to \texttt{write source} to
permanently save the changes.

In Proof Assistant\index{Proof Assistant} mode, the \texttt{exit} command will
return to the \verb/MM>/ prompt. If there were changes to the proof, you will
be given an opportunity to \texttt{save new{\char`\_}proof}.

The \texttt{quit} command is a synonym for \texttt{exit}.

Optional qualifier:
    \texttt{/force} - Do not prompt if changes were not saved.  This qualifier is
        useful in \texttt{submit} command files (Section~\ref{sbmt})
        to ensure predictable behavior.





\subsection{\texttt{open log} Command}\index{\texttt{open log} command}
Syntax:  \texttt{open log} {\em file-name}

This command will open a log file that will store everything you see on
the screen.  It is useful to help recovery from a mistake in a long Proof
Assistant session, or to document bugs.\index{Metamath!bugs}

The log file can be closed with \texttt{close log}.  It will automatically be
closed upon exiting Metamath.



\subsection{\texttt{close log} Command}\index{\texttt{close log} command}
Syntax:  \texttt{close log}

The \texttt{close log} command closes a log file if one is open.  See
also \texttt{open log}.




\subsection{\texttt{submit} Command}\index{\texttt{submit} command}\label{sbmt}
Syntax:  \texttt{submit} {\em filename}

This command causes further command lines to be taken from the specified
file.  Note that any line beginning with an exclamation point (\texttt{!}) is
treated as a comment (i.e.\ ignored).  Also note that the scrolling
of the screen output is continuous, so you may want to open a log file
(see \texttt{open log}) to record the results that fly by on the screen.
After the lines in the file are exhausted, Metamath returns to its
normal user interface mode.

The \texttt{submit} command can be called recursively (i.e. from inside
of a \texttt{submit} command file).


Optional command qualifier:

    \texttt{/silent} - suppresses the screen output but still
        records the output in a log file if one is open.


\subsection{\texttt{erase} Command}\index{\texttt{erase} command}
Syntax:  \texttt{erase}

This command will reset Metamath to its starting state, deleting any
data\-base that was \texttt{read} in.
 If there have been changes to the
source with the \texttt{save proof} or \texttt{save new{\char`\_}proof}
commands, you will be given an opportunity to \texttt{write source} to
permanently save the changes.



\subsection{\texttt{set echo} Command}\index{\texttt{set echo} command}
Syntax:  \texttt{set echo on} or \texttt{set echo off}

The \texttt{set echo on} command will cause command lines to be echoed with any
abbreviations expanded.  While learning the Metamath commands, this
feature will show you the exact command that your abbreviated input
corresponds to.



\subsection{\texttt{set scroll} Command}\index{\texttt{set scroll} command}
Syntax:  \texttt{set scroll prompted} or \texttt{set scroll continuous}

The Metamath command line interface starts off in the \texttt{prompted} mode,
which means that you will be prompted to continue or quit after each
full screen in a long listing.  In \texttt{continuous} mode, long listings will be
scrolled without pausing.

% LaTeX bug? (1) \texttt{\_} puts out different character than
% \texttt{{\char`\_}}
%  = \verb$_$  (2) \texttt{{\char`\_}} puts out garbage in \subsection
%  argument
\subsection{\texttt{set width} Command}\index{\texttt{set
width} command}
Syntax:  \texttt{set width} {\em number}

Metamath assumes the width of your screen is 79 characters (chosen
because the Command Prompt in Windows XP has a wrapping bug at column
80).  If your screen is wider or narrower, this command allows you to
change this default screen width.  A larger width is advantageous for
logging proofs to an output file to be printed on a wide printer.  A
smaller width may be necessary on some terminals; in this case, the
wrapping of the information messages may sometimes seem somewhat
unnatural, however.  In \LaTeX\index{latex@{\LaTeX}!characters per line},
there is normally a maximum of 61 characters per line with typewriter
font.  (The examples in this book were produced with 61 characters per
line.)

\subsection{\texttt{set height} Command}\index{\texttt{set
height} command}
Syntax:  \texttt{set height} {\em number}

Metamath assumes your screen height is 24 lines of characters.  If your
screen is taller or shorter, this command lets you to change the number
of lines at which the display pauses and prompts you to continue.

\subsection{\texttt{beep} Command}\index{\texttt{beep} command}

Syntax:  \texttt{beep}

This command will produce a beep.  By typing it ahead after a
long-running command has started, it will alert you that the command is
finished.  For convenience, \texttt{b} is an abbreviation for
\texttt{beep}.

Note:  If \texttt{b} is typed at the \texttt{MM>} prompt immediately
after the end of a multiple-page display paged with ``\texttt{Press
<return> for more}...'' prompts, then the \texttt{b} will back up to the
previous page rather than perform the \texttt{beep} command.
In that case you must type the unabbreviated \texttt{beep} form
of the command.

\subsection{\texttt{more} Command}\index{\texttt{more} command}

Syntax:  \texttt{more} {\em filename}

This command will display the contents of an {\sc ascii} file on your
screen.  (This command is provided for convenience but is not very
powerful.  See Section~\ref{oscmd} to invoke your operating system's
command to do this, such as the \texttt{more} command in Unix.)

\subsection{Operating System Commands}\index{operating system
command}\label{oscmd}

A line enclosed in single or double quotes will be executed by your
computer's operating system if it has a command line interface.  For
example, on a {\sc vax/vms} system,
\verb/MM> 'dir'/
will print disk directory contents.  Note that this feature will not
work on the Macintosh prior to Mac OS X, which does not have a command
line interface.

For your convenience, the trailing quote is optional.

\subsection{Size Limitations in Metamath}

In general, there are no fixed, predefined limits\index{Metamath!memory
limits} on how many labels, tokens\index{token}, statements, etc.\ that
you may have in a database file.  The Metamath program uses 32-bit
variables (64-bit on 64-bit CPUs) as indices for almost all internal
arrays, which are allocated dynamically as needed.



\section{Reading and Writing Files}

The following commands create new files:  the \texttt{open} commands;
the \texttt{write} commands; the \texttt{/html},
\texttt{/alt{\char`\_}html}, \texttt{/brief{\char`\_}html},
\texttt{/brief{\char`\_}alt{\char`\_}html} qualifiers of \texttt{show
statement}, and \texttt{midi}.  The following commands append to files
previously opened:  the \texttt{/tex} qualifier of \texttt{show proof}
and \texttt{show new{\char`\_}proof}; the \texttt{/tex} and
\texttt{/simple{\char`\_}tex} qualifiers of \texttt{show statement}; the
\texttt{close} commands; and all screen dialog between \texttt{open log}
and \texttt{close log}.

The commands that create new files will not overwrite an existing {\em
filename} but will rename the existing one to {\em
filename}\texttt{{\char`\~}1}.  An existing {\em
filename}\texttt{{\char`\~}1} is renamed {\em
filename}\texttt{{\char`\~}2}, etc.\ up to {\em
filename}\texttt{{\char`\~}9}.  An existing {\em
filename}\texttt{{\char`\~}9} is deleted.  This makes recovery from
mistakes easier but also will clutter up your directory, so occasionally
you may want to clean up (delete) these \texttt{{\char`\~}}$n$ files.


\subsection{\texttt{read} Command}\index{\texttt{read} command}
Syntax:  \texttt{read} {\em file-name} [\texttt{/verify}]

This command will read in a Metamath language source file and any included
files.  Normally it will be the first thing you do when entering Metamath.
Statement syntax is checked, but proof syntax is not checked.
Note that the file name may be enclosed in single or double quotes;
this is useful if the file name contains slashes, as might be the case
under Unix.

If you are getting an ``\texttt{?Expected VERIFY}'' error
when trying to read a Unix file name with slashes, you probably haven't
quoted it.\index{Unix file names}\index{file names!Unix}

If you are prompted for the file name (by pressing {\em return}
 after \texttt{read})
you should {\em not} put quotes around it, even if it is a Unix file name
with slashes.

Optional command qualifier:

    \texttt{/verify} - Verify all proofs as the database is read in.  This
         qualifier will slow down reading in the file.  See \texttt{verify
         proof} for more information on file error-checking.

See also \texttt{erase}.



\subsection{\texttt{write source} Command}\index{\texttt{write source} command}
Syntax:  \texttt{write source} {\em filename}
[\texttt{/rewrap}]
[\texttt{/split}]
% TeX doesn't handle this long line with tt text very well,
% so force a line break here.
[\texttt{/keep\_includes}] {\\}
[\texttt{/no\_versioning}]

This command will write the contents of a Metamath\index{database}
database into a file.\index{source file}

Optional command qualifiers:

\texttt{/rewrap} -
Reformats statements and comments according to the
convention used in the set.mm database.
It unwraps the
lines in the comment before each \texttt{\$a} and \texttt{\$p} statement,
then it
rewraps the line.  You should compare the output to the original
to make sure that the desired effect results; if not, go back to
the original source.  The wrapped line length honors the
\texttt{set width}
parameter currently in effect.  Note:  Text
enclosed in \texttt{<HTML>}...\texttt{</HTML>} tags is not modified by the
\texttt{/rewrap} qualifier.
Proofs themselves are not reformatted;
use \texttt{save proof * / compressed} to do that.
An isolated tilde (\~{}) is always kept on the same line as the following
symbol, so you can find all comment references to a symbol by
searching for \~{} followed by a space and that symbol
(this is useful for finding cross-references).
Incidentally, \texttt{save proof} also honors the \texttt{set width}
parameter currently in effect.

\texttt{/split} - Files included in the source using the expression
\$[ \textit{inclfile} \$] will be
written into separate files instead of being included in a single output
file.  The name of each separately written included file will be the
\textit{inclfile} argument of its inclusion command.

\texttt{/keep\_includes} - If a source file has includes but is written as a
single file by omitting \texttt{/split}, by default the included files will
be deleted (actually just renamed with a \char`\~1 suffix unless
\texttt{/no\_versioning} is specified) to prevent the possibly confusing
source duplication in both the output file and the included file.
The \texttt{/keep\_includes} qualifier will prevent this deletion.

\texttt{/no\_versioning} - Backup files suffixed with \char`\~1 are not created.


\section{Showing Status and Statements}



\subsection{\texttt{show settings} Command}\index{\texttt{show settings} command}
Syntax:  \texttt{show settings}

This command shows the state of various parameters.

\subsection{\texttt{show memory} Command}\index{\texttt{show memory} command}
Syntax:  \texttt{show memory}

This command shows the available memory left.  It is not meaningful
on most modern operating systems,
which have virtual memory.\index{Metamath!memory usage}


\subsection{\texttt{show labels} Command}\index{\texttt{show labels} command}
Syntax:  \texttt{show labels} {\em label-match} [\texttt{/all}]
   [\texttt{/linear}]

This command shows the labels of \texttt{\$a} and \texttt{\$p}
statements that match {\em label-match}.  A \verb$*$ in {label-match}
matches zero or more characters.  For example, \verb$*abc*def$ will match all
labels containing \verb$abc$ and ending with \verb$def$.

Optional command qualifiers:

   \texttt{/all} - Include matches for \texttt{\$e} and \texttt{\$f}
   statement labels.

   \texttt{/linear} - Display only one label per line.  This can be useful for
       building scripts in conjunction with the utilities under the
       \texttt{tools}\index{\texttt{tools} command} command.



\subsection{\texttt{show statement} Command}\index{\texttt{show statement} command}
Syntax:  \texttt{show statement} {\em label-match} [{\em qualifiers} (see below)]

This command provides information about a statement.  Only statements
that have labels (\texttt{\$f}\index{\texttt{\$f} statement},
\texttt{\$e}\index{\texttt{\$e} statement},
\texttt{\$a}\index{\texttt{\$a} statement}, and
\texttt{\$p}\index{\texttt{\$p} statement}) may be specified.
If {\em label-match}
contains wildcard (\verb$*$) characters, all matching statements will be
displayed in the order they occur in the database.

Optional qualifiers (only one qualifier at a time is allowed):

    \texttt{/comment} - This qualifier includes the comment that immediately
       precedes the statement.

    \texttt{/full} - Show complete information about each statement,
       and show all
       statements matching {\em label} (including \texttt{\$e}
       and \texttt{\$f} statements).

    \texttt{/tex} - This qualifier will write the statement information to the
       \LaTeX\ file previously opened with \texttt{open tex}.  See
       Section~\ref{texout}.

    \texttt{/simple{\char`\_}tex} - The same as \texttt{/tex}, except that
       \LaTeX\ macros are not used for formatting equations, allowing easier
       manual edits of the output for slide presentations, etc.

    \texttt{/html}\index{html generation@{\sc html} generation},
       \texttt{/alt{\char`\_}html}, \texttt{/brief{\char`\_}html},
       \texttt{/brief{\char`\_}alt{\char`\_}html} -
       These qualifiers invoke a special mode of
       \texttt{show statement} that
       creates a web page for the statement.  They may not be used with
       any other qualifier.  See Section~\ref{htmlout} or
       \texttt{help html} in the program.


\subsection{\texttt{search} Command}\index{\texttt{search} command}
Syntax:  search {\em label-match}
\texttt{"}{\em symbol-match}{\tt}" [\texttt{/all}] [\texttt{/comments}]
[\texttt{/join}]

This command searches all \texttt{\$a} and \texttt{\$p} statements
matching {\em label-match} for occurrences of {\em symbol-match}.  A
\verb@*@ in {\em label-match} matches any label character.  A \verb@$*@
in {\em symbol-match} matches any sequence of symbols.  The symbols in
{\em symbol-match} must be separated by white space.  The quotes
surrounding {\em symbol-match} may be single or double quotes.  For
example, \texttt{search b}\verb@* "-> $* ch"@ will list all statements
whose labels begin with \texttt{b} and contain the symbols \verb@->@ and
\texttt{ch} surrounding any symbol sequence (including no symbol
sequence).  The wildcards \texttt{?} and \texttt{\$?} are also available
to match individual characters in labels and symbols respectively; see
\texttt{help search} in the Metamath program for details on their usage.

Optional command qualifiers:

    \texttt{/all} - Also search \texttt{\$e} and \texttt{\$f} statements.

    \texttt{/comments} - Search the comment that immediately precedes each
        label-matched statement for {\em symbol-match}.  In this case
        {\em symbol-match} is an arbitrary, non-case-sensitive character
        string.  Quotes around {\em symbol-match} are optional if there
        is no ambiguity.

    \texttt{/join} - In the case of a \texttt{\$a} or \texttt{\$p} statement,
	prepend its \texttt{\$e}
	hypotheses for searching. The
	\texttt{/join} qualifier has no effect in \texttt{/comments} mode.

\section{Displaying and Verifying Proofs}


\subsection{\texttt{show proof} Command}\index{\texttt{show proof} command}
Syntax:  \texttt{show proof} {\em label-match} [{\em qualifiers} (see below)]

This command displays the proof of the specified
\texttt{\$p}\index{\texttt{\$p} statement} statement in various formats.
The {\em label-match} may contain wildcard (\verb@$*@) characters to match
multiple statements.  Without any qualifiers, only the logical steps
will be shown (i.e.\ syntax construction steps will be omitted), in an
indented format.

Most of the time, you will use
    \texttt{show proof} {\em label}
to see just the proof steps corresponding to logical inferences.

Optional command qualifiers:

    \texttt{/essential} - The proof tree is trimmed of all
        \texttt{\$f}\index{\texttt{\$f} statement} hypotheses before
        being displayed.  (This is the default, and it is redundant to
        specify it.)

    \texttt{/all} - the proof tree is not trimmed of all \texttt{\$f} hypotheses before
        being displayed.  \texttt{/essential} and \texttt{/all} are mutually exclusive.

    \texttt{/from{\char`\_}step} {\em step} - The display starts at the specified
        step.  If
        this qualifier is omitted, the display starts at the first step.

    \texttt{/to{\char`\_}step} {\em step} - The display ends at the specified
        step.  If this
        qualifier is omitted, the display ends at the last step.

    \texttt{/tree{\char`\_}depth} {\em number} - Only
         steps at less than the specified proof
        tree depth are displayed.  Sometimes useful for obtaining an overview of
        the proof.

    \texttt{/reverse} - The steps are displayed in reverse order.

    \texttt{/renumber} - When used with \texttt{/essential}, the steps are renumbered
        to correspond only to the essential steps.

    \texttt{/tex} - The proof is converted to \LaTeX\ and\index{latex@{\LaTeX}}
        stored in the file opened
        with \texttt{open tex}.  See Section~\ref{texout} or
        \texttt{help tex} in the program.

    \texttt{/lemmon} - The proof is displayed in a non-indented format known
        as Lemmon style, with explicit previous step number references.
        If this qualifier is omitted, steps are indented in a tree format.

    \texttt{/start{\char`\_}column} {\em number} - Overrides the default column
        (16)
        at which the formula display starts in a Lemmon-style display.  May be
        used only in conjunction with \texttt{/lemmon}.

    \texttt{/normal} - The proof is displayed in normal format suitable for
        inclusion in a Metamath source file.  May not be used with any other
        qualifier.

    \texttt{/compressed} - The proof is displayed in compressed format
        suitable for inclusion in a Metamath source file.  May not be used with
        any other qualifier.

    \texttt{/statement{\char`\_}summary} - Summarizes all statements (like a
        brief \texttt{show statement})
        used by the proof.  It may not be used with any other qualifier
        except \texttt{/essential}.

    \texttt{/detailed{\char`\_}step} {\em step} - Shows the details of what is
        happening at
        a specific proof step.  May not be used with any other qualifier.
        The {\em step} is the step number shown when displaying a
        proof without the \texttt{/renumber} qualifier.


\subsection{\texttt{show usage} Command}\index{\texttt{show usage} command}
Syntax:  \texttt{show usage} {\em label-match} [\texttt{/recursive}]

This command lists the statements whose proofs make direct reference to
the statement specified.

Optional command qualifier:

    \texttt{/recursive} - Also include statements whose proofs ultimately
        depend on the statement specified.



\subsection{\texttt{show trace\_back} Command}\index{\texttt{show
       trace{\char`\_}back} command}
Syntax:  \texttt{show trace{\char`\_}back} {\em label-match} [\texttt{/essential}] [\texttt{/axioms}]
    [\texttt{/tree}] [\texttt{/depth} {\em number}]

This command lists all statements that the proof of the \texttt{\$p}
statement(s) specified by {\em label-match} depends on.

Optional command qualifiers:

    \texttt{/essential} - Restrict the trace-back to \texttt{\$e}
        \index{\texttt{\$e} statement} hypotheses of proof trees.

    \texttt{/axioms} - List only the axioms that the proof ultimately depends on.

    \texttt{/tree} - Display the trace-back in an indented tree format.

    \texttt{/depth} {\em number} - Restrict the \texttt{/tree} trace-back to the
        specified indentation depth.

    \texttt{/count{\char`\_}steps} - Count the number of steps the proof has
       all the way back to axioms.  If \texttt{/essential} is specified,
       expansions of variable-type hypotheses (syntax constructions) are not counted.

\subsection{\texttt{verify proof} Command}\index{\texttt{verify proof} command}
Syntax:  \texttt{verify proof} {\em label-match} [\texttt{/syntax{\char`\_}only}]

This command verifies the proofs of the specified statements.  {\em
label-match} may contain wild card characters (\texttt{*}) to verify
more than one proof; for example \verb/*abc*def/ will match all labels
containing \texttt{abc} and ending with \texttt{def}.
The command \texttt{verify proof *} will verify all proofs in the database.

Optional command qualifier:

    \texttt{/syntax{\char`\_}only} - This qualifier will perform a check of syntax
        and RPN
        stack violations only.  It will not verify that the proof is
        correct.  This qualifier is useful for quickly determining which
        proofs are incomplete (i.e.\ are under development and have \texttt{?}'s
        in them).

{\em Note:} \texttt{read}, followed by \texttt{verify proof *}, will ensure
 the database is free
from errors in the Metamath language but will not check the markup notation
in comments.
You can also check the markup notation by running \texttt{verify markup *}
as discussed in Section~\ref{verifymarkup}; see also the discussion
on generating {\sc HTML} in Section~\ref{htmlout}.

\subsection{\texttt{verify markup} Command}\index{\texttt{verify markup} command}\label{verifymarkup}
Syntax:  \texttt{verify markup} {\em label-match}
[\texttt{/date{\char`\_}skip}]
[\texttt{/top{\char`\_}date{\char`\_}skip}] {\\}
[\texttt{/file{\char`\_}skip}]
[\texttt{/verbose}]

This command checks comment markup and other informal conventions we have
adopted.  It error-checks the latexdef, htmldef, and althtmldef statements
in the \texttt{\$t} statement of a Metamath source file.\index{error checking}
It error-checks any \texttt{`}...\texttt{`},
\texttt{\char`\~}~\textit{label},
and bibliographic markups in statement descriptions.
It checks that
\texttt{\$p} and \texttt{\$a} statements
have the same content when their labels start with
``ax'' and ``ax-'' respectively but are otherwise identical, for example
ax4 and ax-4.
It also verifies the date consistency of ``(Contributed by...),''
``(Revised by...),'' and ``(Proof shortened by...)'' tags in the comment
above each \texttt{\$a} and \texttt{\$p} statement.

Optional command qualifiers:

    \texttt{/date{\char`\_}skip} - This qualifier will
        skip date consistency checking,
        which is usually not required for databases other than
	\texttt{set.mm}.

    \texttt{/top{\char`\_}date{\char`\_}skip} - This qualifier will check date consistency except
        that the version date at the top of the database file will not
        be checked.  Only one of
        \texttt{/date{\char`\_}skip} and
        \texttt{/top{\char`\_}date{\char`\_}skip} may be
        specified.

    \texttt{/file{\char`\_}skip} - This qualifier will skip checks that require
        external files to be present, such as checking GIF existence and
        bibliographic links to mmset.html or equivalent.  It is useful
        for doing a quick check from a directory without these files.

    \texttt{/verbose} - Provides more information.  Currently it provides a list
        of axXXX vs. ax-XXX matches.

\subsection{\texttt{save proof} Command}\index{\texttt{save proof} command}
Syntax:  \texttt{save proof} {\em label-match} [\texttt{/normal}]
   [\texttt{/compressed}]

The \texttt{save proof} command will reformat a proof in one of two formats and
replace the existing proof in the source buffer\index{source
buffer}.  It is useful for
converting between proof formats.  Note that a proof will not be
permanently saved until a \texttt{write source} command is issued.

Optional command qualifiers:

    \texttt{/normal} - The proof is saved in the normal format (i.e., as a
        sequence
        of labels, which is the defined format of the basic Metamath
        language).\index{basic language}  This is the default format that
        is used if a qualifier
        is omitted.

    \texttt{/compressed} - The proof is saved in the compressed format which
        reduces storage requirements for a database.
        See Appendix~\ref{compressed}.




\section{Creating Proofs}\label{pfcommands}\index{Proof Assistant}

Before using the Proof Assistant, you must add a \texttt{\$p} to your
source file (using a text editor) containing the statement you want to
prove.  Its proof should consist of a single \texttt{?}, meaning
``unknown step.''  Example:
\begin{verbatim}
equid $p x = x $= ? $.
\end{verbatim}

To enter the Proof assistant, type \texttt{prove} {\em label}, e.g.
\texttt{prove equid}.  Metamath will respond with the \texttt{MM-PA>}
prompt.

Proofs are created working backwards from the statement being proved,
primarily using a series of \texttt{assign} commands.  A proof is
complete when all steps are assigned to statements and all steps are
unified and completely known.  During the creation of a proof, Metamath
will allow only operations that are legal based on what is known up to
that point.  For example, it will not allow an \texttt{assign} of a
statement that cannot be unified with the unknown proof step being
assigned.

{\em Important:}
The Proof Assistant is
{\em not} a tool to help you discover proofs.  It is just a tool to help
you add them to the database.  For a tutorial read
Section~\ref{frstprf}.
To practice using the Proof Assistant, you may
want to \texttt{prove} an existing theorem, then delete all steps with
\texttt{delete all}, then re-create it with the Proof Assistant while
looking at its proof display (before deletion).
You might want to figure out your first few proofs completely
and write them down by hand, before using the Proof Assistant, though
not everyone finds that effective.

{\em Important:}
The \texttt{undo} command if very helpful when entering a proof, because
it allows you to undo a previously-entered step.
In addition, we suggest that you
keep track of your work with a log file (\texttt{open
log}) and save it frequently (\texttt{save new{\char`\_}proof},
\texttt{write source}).
You can use \texttt{delete} to reverse an \texttt{assign}.
You can also do \texttt{delete floating{\char`\_}hypotheses}, then
\texttt{initialize all}, then \texttt{unify all /interactive} to
reinitialize bad unifications made accidentally or by bad
\texttt{assign}s.  You cannot reverse a \texttt{delete} except by
a relevant \texttt{undo} or using
\texttt{exit /force} then reentering the Proof Assistant to recover from
the last \texttt{save new{\char`\_}proof}.

The following commands are available in the Proof Assistant (at the
\texttt{MM-PA>} prompt) to help you create your proof.  See the
individual commands for more detail.

\begin{itemize}
\item[]
    \texttt{show new{\char`\_}proof} [\texttt{/all},...] - Displays the
        proof in progress.  You will use this command a lot; see \texttt{help
        show new{\char`\_}proof} to become familiar with its qualifiers.  The
        qualifiers \texttt{/unknown} and \texttt{/not{\char`\_}unified} are
        useful for seeing the work remaining to be done.  The combination
        \texttt{/all/unknown} is useful for identifying dummy variables that must be
        assigned, or attempts to use illegal syntax, when \texttt{improve all}
        is unable to complete the syntax constructions.  Unknown variables are
        shown as \texttt{\$1}, \texttt{\$2},...
\item[]
    \texttt{assign} {\em step} {\em label} - Assigns an unknown {\em step}
        number with the statement
        specified by {\em label}.
\item[]
    \texttt{let variable} {\em variable}
        \texttt{= "}{\em symbol sequence}\texttt{"}
          - Forces a symbol
        sequence to replace an unknown variable (such as \texttt{\$1}) in a proof.
        It is useful
        for helping difficult unifications, and it is necessary when you have
        dummy variables that eventually must be assigned a name.
\item[]
    \texttt{let step} {\em step} \texttt{= "}{\em symbol sequence}\texttt{"} -
          Forces a symbol sequence
        to replace the contents of a proof step, provided it can be
        unified with the existing step contents.  (I rarely use this.)
\item[]
    \texttt{unify step} {\em step} (or \texttt{unify all}) - Unifies
        the source and target of
        a step.  If you specify a specific step, you will be prompted
        to select among the unifications that are possible.  If you
        specify \texttt{all}, all steps with unique unifications, but only
        those steps, will be
        unified.  \texttt{unify all /interactive} goes through all non-unified
        steps.
\item[]
    \texttt{initialize} {\em step} (or \texttt{all}) - De-unifies the target and source of
        a step (or all steps), as well as the hypotheses of the source,
        and makes all variables in the source unknown.  Useful to recover from
        an \texttt{assign} or \texttt{let} mistake that
        resulted in incorrect unifications.
\item[]
    \texttt{delete} {\em step} (or \texttt{all} or \texttt{floating{\char`\_}hypotheses}) -
        Deletes the specified
        step(s).  \texttt{delete floating{\char`\_}hypotheses}, then \texttt{initialize all}, then
        \texttt{unify all /interactive} is useful for recovering from mistakes
        where incorrect unifications assigned wrong math symbol strings to
        variables.
\item[]
    \texttt{improve} {\em step} (or \texttt{all}) -
      Automatically creates a proof for steps (with no unknown variables)
      whose proof requires no statements with \texttt{\$e} hypotheses.  Useful
      for filling in proofs of \texttt{\$f} hypotheses.  The \texttt{/depth}
      qualifier will also try statements whose \texttt{\$e} hypotheses contain
      no new variables.  {\em Warning:} Save your work (with \texttt{save
      new{\char`\_}proof} then \texttt{write source}) before using
      \texttt{/depth = 2} or greater, since the search time grows
      exponentially and may never terminate in a reasonable time, and you
      cannot interrupt the search.  I have found that it is rare for
      \texttt{/depth = 3} or greater to be useful.
 \item[]
    \texttt{save new{\char`\_}proof} - Saves the proof in progress in the program's
        internal database buffer.  To save it permanently into the database file,
        use \texttt{write source} after
        \texttt{save new{\char`\_}proof}.  To revert to the last
        \texttt{save new{\char`\_}proof},
        \texttt{exit /force} from the Proof Assistant then re-enter the Proof
        Assistant.
 \item[]
    \texttt{match step} {\em step} (or \texttt{match all}) - Shows what
        statements are
        possibilities for the \texttt{assign} statement. (This command
        is not very
        useful in its present form and hopefully will be improved
        eventually.  In the meantime, use the \texttt{search} statement for
        candidates matching specific math token combinations.)
 \item[]
 \texttt{minimize{\char`\_}with}\index{\texttt{minimize{\char`\_}with} command}
% 3/10/07 Note: line-breaking the above results in duplicate index entries
     - After a proof is complete, this command will attempt
        to match other database theorems to the proof to see if the proof
        size can be reduced as a result.  See \texttt{help
        minimize{\char`\_}with} in the
        Metamath program for its usage.
 \item[]
 \texttt{undo}\index{\texttt{undo} command}
    - Undo the effect of a proof-changing command (all but the
      \texttt{show} and \texttt{save} commands above).
 \item[]
 \texttt{redo}\index{\texttt{redo} command}
    - Reverse the previous \texttt{undo}.
\end{itemize}

The following commands set parameters that may be relevant to your proof.
Consult the individual \texttt{help set}... commands.
\begin{itemize}
   \item[] \texttt{set unification{\char`\_}timeout}
 \item[]
    \texttt{set search{\char`\_}limit}
  \item[]
    \texttt{set empty{\char`\_}substitution} - note that default is \texttt{off}
\end{itemize}

Type \texttt{exit} to exit the \texttt{MM-PA>}
 prompt and get back to the \texttt{MM>} prompt.
Another \texttt{exit} will then get you out of Metamath.



\subsection{\texttt{prove} Command}\index{\texttt{prove} command}
Syntax:  \texttt{prove} {\em label}

This command will enter the Proof Assistant\index{Proof Assistant}, which will
allow you to create or edit the proof of the specified statement.
The command-line prompt will change from \texttt{MM>} to \texttt{MM-PA>}.

Note:  In the present version (0.177) of
Metamath\index{Metamath!limitations of version 0.177}, the Proof
Assistant does not verify that \texttt{\$d}\index{\texttt{\$d}
statement} restrictions are met as a proof is being built.  After you
have completed a proof, you should type \texttt{save new{\char`\_}proof}
followed by \texttt{verify proof} {\em label} (where {\em label} is the
statement you are proving with the \texttt{prove} command) to verify the
\texttt{\$d} restrictions.

See also: \texttt{exit}

\subsection{\texttt{set unification\_timeout} Command}\index{\texttt{set
unification{\char`\_}timeout} command}
Syntax:  \texttt{set unification{\char`\_}timeout} {\em number}

(This command is available outside the Proof Assistant but affects the
Proof Assistant\index{Proof Assistant} only.)

Sometimes the Proof Assistant will inform you that a unification
time-out occurred.  This may happen when you try to \texttt{unify}
formulas with many temporary variables\index{temporary variable}
(\texttt{\$1}, \texttt{\$2}, etc.), since the time to compute all possible
unifications may grow exponentially with the number of variables.  If
you want Metamath to try harder (and you're willing to wait longer) you
may increase this parameter.  \texttt{show settings} will show you the
current value.

\subsection{\texttt{set empty\_substitution} Command}\index{\texttt{set
empty{\char`\_}substitution} command}
% These long names can't break well in narrow mode, and even "sloppy"
% is not enough. Work around this by not demanding justification.
\begin{flushleft}
Syntax:  \texttt{set empty{\char`\_}substitution on} or \texttt{set
empty{\char`\_}substitution off}
\end{flushleft}

(This command is available outside the Proof Assistant but affects the
Proof Assistant\index{Proof Assistant} only.)

The Metamath language allows variables to be
substituted\index{substitution!variable}\index{variable substitution}
with empty symbol sequences\index{empty substitution}.  However, in many
formal systems\index{formal system} this will never happen in a valid
proof.  Allowing for this possibility increases the likelihood of
ambiguous unifications\index{ambiguous
unification}\index{unification!ambiguous} during proof creation.
The default is that
empty substitutions are not allowed; for formal systems requiring them,
you must \texttt{set empty{\char`\_}substitution on}.
(An example where this must be \texttt{on}
would be a system that implements a Deduction Rule and in
which deductions from empty assumption lists would be permissible.  The
MIU-system\index{MIU-system} described in Appendix~\ref{MIU} is another
example.)
Note that empty substitutions are
always permissible in proof verification (VERIFY PROOF...) outside the
Proof Assistant.  (See the MIU system in the Metamath book for an example
of a system needing empty substitutions; another example would be a
system that implements a Deduction Rule and in which deductions from
empty assumption lists would be permissible.)

It is better to leave this \texttt{off} when working with \texttt{set.mm}.
Note that this command does not affect the way proofs are verified with
the \texttt{verify proof} command.  Outside of the Proof Assistant,
substitution of empty sequences for math symbols is always allowed.

\subsection{\texttt{set search\_limit} Command}\index{\texttt{set
search{\char`\_}limit} command} Syntax:  \texttt{set search{\char`\_}limit} {\em
number}

(This command is available outside the Proof Assistant but affects the
Proof Assistant\index{Proof Assistant} only.)

This command sets a parameter that determines when the \texttt{improve} command
in Proof Assistant mode gives up.  If you want \texttt{improve} to search harder,
you may increase it.  The \texttt{show settings} command tells you its current
value.


\subsection{\texttt{show new\_proof} Command}\index{\texttt{show
new{\char`\_}proof} command}
Syntax:  \texttt{show new{\char`\_}proof} [{\em
qualifiers} (see below)]

This command (available only in Proof Assistant mode) displays the proof
in progress.  It is identical to the \texttt{show proof} command, except that
there is no statement argument (since it is the statement being proved) and
the following qualifiers are not available:

    \texttt{/statement{\char`\_}summary}

    \texttt{/detailed{\char`\_}step}

Also, the following additional qualifiers are available:

    \texttt{/unknown} - Shows only steps that have no statement assigned.

    \texttt{/not{\char`\_}unified} - Shows only steps that have not been unified.

Note that \texttt{/essential}, \texttt{/depth}, \texttt{/unknown}, and
\texttt{/not{\char`\_}unified} may be
used in any combination; each of them effectively filters out additional
steps from the proof display.

See also:  \texttt{show proof}






\subsection{\texttt{assign} Command}\index{\texttt{assign} command}
Syntax:   \texttt{assign} {\em step} {\em label} [\texttt{/no{\char`\_}unify}]

   and:   \texttt{assign first} {\em label}

   and:   \texttt{assign last} {\em label}


This command, available in the Proof Assistant only, assigns an unknown
step (one with \texttt{?} in the \texttt{show new{\char`\_}proof}
listing) with the statement specified by {\em label}.  The assignment
will not be allowed if the statement cannot be unified with the step.

If \texttt{last} is specified instead of {\em step} number, the last
step that is shown by \texttt{show new{\char`\_}proof /unknown} will be
used.  This can be useful for building a proof with a command file (see
\texttt{help submit}).  It also makes building proofs faster when you know
the assignment for the last step.

If \texttt{first} is specified instead of {\em step} number, the first
step that is shown by \texttt{show new{\char`\_}proof /unknown} will be
used.

If {\em step} is zero or negative, the -{\em step}th from last unknown
step, as shown by \texttt{show new{\char`\_}proof /unknown}, will be
used.  \texttt{assign -1} {\em label} will assign the penultimate
unknown step, \texttt{assign -2} {\em label} the antepenultimate, and
\texttt{assign 0} {\em label} is the same as \texttt{assign last} {\em
label}.

Optional command qualifier:

    \texttt{/no{\char`\_}unify} - do not prompt user to select a unification if there is
        more than one possibility.  This is useful for noninteractive
        command files.  Later, the user can \texttt{unify all /interactive}.
        (The assignment will still be automatically unified if there is only
        one possibility and will be refused if unification is not possible.)



\subsection{\texttt{match} Command}\index{\texttt{match} command}
Syntax:  \texttt{match step} {\em step} [\texttt{/max{\char`\_}essential{\char`\_}hyp}
{\em number}]

    and:  \texttt{match all} [\texttt{/essential}]
          [\texttt{/max{\char`\_}essential{\char`\_}hyp} {\em number}]

This command, available in the Proof Assistant only, shows what
statements can be unified with the specified step(s).  {\em Note:} In
its current form, this command is not very useful because of the large
number of matches it reports.
It may be enhanced in the future.  In the meantime, the \texttt{search}
command can often provide finer control over locating theorems of interest.

Optional command qualifiers:

    \texttt{/max{\char`\_}essential{\char`\_}hyp} {\em number} - filters out
        of the list any statements
        with more than the specified number of
        \texttt{\$e}\index{\texttt{\$e} statement} hypotheses.

    \texttt{/essential{\char`\_}only} - in the \texttt{match all} statement, only
        the steps that
        would be listed in the \texttt{show new{\char`\_}proof /essential} display are
        matched.



\subsection{\texttt{let} Command}\index{\texttt{let} command}
Syntax: \texttt{let variable} {\em variable} = \verb/"/{\em symbol-sequence}\verb/"/

  and: \texttt{let step} {\em step} = \verb/"/{\em symbol-sequence}\verb/"/

These commands, available in the Proof Assistant\index{Proof Assistant}
only, assign a temporary variable\index{temporary variable} or unknown
step with a specific symbol sequence.  They are useful in the middle of
creating a proof, when you know what should be in the proof step but the
unification algorithm doesn't yet have enough information to completely
specify the temporary variables.  A ``temporary variable'' is one that
has the form \texttt{\$}{\em nn} in the proof display, such as
\texttt{\$1}, \texttt{\$2}, etc.  The {\em symbol-sequence} may contain
other unknown variables if desired.  Examples:

    \verb/let variable $32 = "A = B"/

    \verb/let variable $32 = "A = $35"/

    \verb/let step 10 = '|- x = x'/

    \verb/let step -2 = "|- ( $7 = ph )"/

Any symbol sequence will be accepted for the \texttt{let variable}
command.  Only those symbol sequences that can be unified with the step
will be accepted for \texttt{let step}.

The \texttt{let} commands ``zap'' the proof with information that can
only be verified when the proof is built up further.  If you make an
error, the command sequence \texttt{delete
floating{\char`\_}hypotheses}, \texttt{initialize all}, and
\texttt{unify all /interactive} will undo a bad \texttt{let} assignment.

If {\em step} is zero or negative, the -{\em step}th from last unknown
step, as shown by \texttt{show new{\char`\_}proof /unknown}, will be
used.  The command \texttt{let step 0} = \verb/"/{\em
symbol-sequence}\verb/"/ will use the last unknown step, \texttt{let
step -1} = \verb/"/{\em symbol-sequence}\verb/"/ the penultimate, etc.
If {\em step} is positive, \texttt{let step} may be used to assign known
(in the sense of having previously been assigned a label with
\texttt{assign}) as well as unknown steps.

Either single or double quotes can surround the {\em symbol-sequence} as
long as they are different from any quotes inside a {\em
symbol-sequence}.  If {\em symbol-sequence} contains both kinds of
quotes, see the instructions at the end of \texttt{help let} in the
Metamath program.


\subsection{\texttt{unify} Command}\index{\texttt{unify} command}
Syntax:  \texttt{unify step} {\em step}

      and:   \texttt{unify all} [\texttt{/interactive}]

These commands, available in the Proof Assistant only, unify the source
and target of the specified step(s). If you specify a specific step, you
will be prompted to select among the unifications that are possible.  If
you specify \texttt{all}, only those steps with unique unifications will be
unified.

Optional command qualifier for \texttt{unify all}:

    \texttt{/interactive} - You will be prompted to select among the
        unifications
        that are possible for any steps that do not have unique
        unifications.  (Otherwise \texttt{unify all} will bypass these.)

See also \texttt{set unification{\char`\_}timeout}.  The default is
100000, but increasing it to 1000000 can help difficult cases.  Manually
assigning some or all of the unknown variables with the \texttt{let
variable} command also helps difficult cases.



\subsection{\texttt{initialize} Command}\index{\texttt{initialize} command}
Syntax:  \texttt{initialize step} {\em step}

    and: \texttt{initialize all}

These commands, available in the Proof Assistant\index{Proof Assistant}
only, ``de-unify'' the target and source of a step (or all steps), as
well as the hypotheses of the source, and makes all variables in the
source and the source's hypotheses unknown.  This command is useful to
help recover from incorrect unifications that resulted from an incorrect
\texttt{assign}, \texttt{let}, or unification choice.  Part or all of
the command sequence \texttt{delete floating{\char`\_}hypotheses},
\texttt{initialize all}, and \texttt{unify all /interactive} will recover
from incorrect unifications.

See also:  \texttt{unify} and \texttt{delete}



\subsection{\texttt{delete} Command}\index{\texttt{delete} command}
Syntax:  \texttt{delete step} {\em step}

   and:      \texttt{delete all} -- {\em Warning: dangerous!}

   and:      \texttt{delete floating{\char`\_}hypotheses}

These commands are available in the Proof Assistant only.  The
\texttt{delete step} command deletes the proof tree section that
branches off of the specified step and makes the step become unknown.
\texttt{delete all} is equivalent to \texttt{delete step} {\em step}
where {\em step} is the last step in the proof (i.e.\ the beginning of
the proof tree).

In most cases the \texttt{undo} command is the best way to undo
a previous step.
An alternative is to salvage your last \texttt{save
new{\char`\_}proof} by exiting and reentering the Proof Assistant.
For this to work, keep a log file open to record your work
and to do \texttt{save new{\char`\_}proof} frequently, especially before
\texttt{delete}.

\texttt{delete floating{\char`\_}hypotheses} will delete all sections of
the proof that branch off of \texttt{\$f}\index{\texttt{\$f} statement}
statements.  It is sometimes useful to do this before an
\texttt{initialize} command to recover from an error.  Note that once a
proof step with a \texttt{\$f} hypothesis as the target is completely
known, the \texttt{improve} command can usually fill in the proof for
that step.  Unlike the deletion of logical steps, \texttt{delete
floating{\char`\_}hypotheses} is a relatively safe command that is
usually easy to recover from.



\subsection{\texttt{improve} Command}\index{\texttt{improve} command}
\label{improve}
Syntax:  \texttt{improve} {\em step} [\texttt{/depth} {\em number}]
                                               [\texttt{/no{\char`\_}distinct}]

   and:   \texttt{improve first} [\texttt{/depth} {\em number}]
                                              [\texttt{/no{\char`\_}distinct}]

   and:   \texttt{improve last} [\texttt{/depth} {\em number}]
                                              [\texttt{/no{\char`\_}distinct}]

   and:   \texttt{improve all} [\texttt{/depth} {\em number}]
                                              [\texttt{/no{\char`\_}distinct}]

These commands, available in the Proof Assistant\index{Proof Assistant}
only, try to find proofs automatically for unknown steps whose symbol
sequences are completely known.  They are primarily useful for filling in
proofs of \texttt{\$f}\index{\texttt{\$f} statement} hypotheses.  The
search will be restricted to statements having no
\texttt{\$e}\index{\texttt{\$e} statement} hypotheses.

\begin{sloppypar} % narrow
Note:  If memory is limited, \texttt{improve all} on a large proof may
overflow memory.  If you use \texttt{set unification{\char`\_}timeout 1}
before \texttt{improve all}, there will usually be sufficient
improvement to easily recover and completely \texttt{improve} the proof
later on a larger computer.  Warning:  Once memory has overflowed, there
is no recovery.  If in doubt, save the intermediate proof (\texttt{save
new{\char`\_}proof} then \texttt{write source}) before \texttt{improve
all}.
\end{sloppypar}

If \texttt{last} is specified instead of {\em step} number, the last
step that is shown by \texttt{show new{\char`\_}proof /unknown} will be
used.

If \texttt{first} is specified instead of {\em step} number, the first
step that is shown by \texttt{show new{\char`\_}proof /unknown} will be
used.

If {\em step} is zero or negative, the -{\em step}th from last unknown
step, as shown by \texttt{show new{\char`\_}proof /unknown}, will be
used.  \texttt{improve -1} will use the penultimate
unknown step, \texttt{improve -2} {\em label} the antepenultimate, and
\texttt{improve 0} is the same as \texttt{improve last}.

Optional command qualifier:

    \texttt{/depth} {\em number} - This qualifier will cause the search
        to include
        statements with \texttt{\$e} hypotheses (but no new variables in
        the \texttt{\$e}
        hypotheses), provided that the backtracking has not exceeded the
        specified depth. {\em Warning:}  Try \texttt{/depth 1},
        then \texttt{2}, then \texttt{3}, etc.
        in sequence because of possible exponential blowups.  Save your
        work before trying \texttt{/depth} greater than \texttt{1}!

    \texttt{/no{\char`\_}distinct} - Skip trial statements that have
        \texttt{\$d}\index{\texttt{\$d} statement} requirements.
        This qualifier will prevent assignments that might violate \texttt{\$d}
        requirements but it also could miss possible legal assignments.

See also: \texttt{set search{\char`\_}limit}

\subsection{\texttt{save new\_proof} Command}\index{\texttt{save
new{\char`\_}proof} command}
Syntax:  \texttt{save new{\char`\_}proof} {\em label} [\texttt{/normal}]
   [\texttt{/compressed}]

The \texttt{save new{\char`\_}proof} command is available in the Proof
Assistant only.  It saves the proof in progress in the source
buffer\index{source buffer}.  \texttt{save new{\char`\_}proof} may be
used to save a completed proof, or it may be used to save a proof in
progress in order to work on it later.  If an incomplete proof is saved,
any user assignments with \texttt{let step} or \texttt{let variable}
will be lost, as will any ambiguous unifications\index{ambiguous
unification}\index{unification!ambiguous} that were resolved manually.
To help make recovery easier, it can be helpful to \texttt{improve all}
before \texttt{save new{\char`\_}proof} so that the incomplete proof
will have as much information as possible.

Note that the proof will not be permanently saved until a \texttt{write
source} command is issued.

Optional command qualifiers:

    \texttt{/normal} - The proof is saved in the normal format (i.e., as a
        sequence of labels, which is the defined format of the basic Metamath
        language).\index{basic language}  This is the default format that
        is used if a qualifier is omitted.

    \texttt{/compressed} - The proof is saved in the compressed format, which
        reduces storage requirements for a database.
        (See Appendix~\ref{compressed}.)


\section{Creating \LaTeX\ Output}\label{texout}\index{latex@{\LaTeX}}

You can generate \LaTeX\ output given the
information in a database.
The database must already include the necessary typesetting information
(see section \ref{tcomment} for how to provide this information).

The \texttt{show statement} and \texttt{show proof} commands each have a
special \texttt{/tex} command qualifier that produces \LaTeX\ output.
(The \texttt{show statement} command also has the
\texttt{/simple{\char`\_}tex} qualifier for output that is easier to
edit by hand.)  Before you can use them, you must open a \LaTeX\ file to
which to send their output.  A typical complete session will use this
sequence of Metamath commands:

\begin{verbatim}
read set.mm
open tex example.tex
show statement a1i /tex
show proof a1i /all/lemmon/renumber/tex
show statement uneq2 /tex
show proof uneq2 /all/lemmon/renumber/tex
close tex
\end{verbatim}

See Section~\ref{mathcomments} for information on comment markup and
Appendix~\ref{ASCII} for information on how math symbol translation is
specified.

To format and print the \LaTeX\ source, you will need the \LaTeX\
program, which is standard on most Linux installations and available for
Windows.  On Linux, in order to create a {\sc pdf} file, you will
typically type at the shell prompt
\begin{verbatim}
$ pdflatex example.tex
\end{verbatim}

\subsection{\texttt{open tex} Command}\index{\texttt{open tex} command}
Syntax:  \texttt{open tex} {\em file-name} [\texttt{/no{\char`\_}header}]

This command opens a file for writing \LaTeX\
source\index{latex@{\LaTeX}} and writes a \LaTeX\ header to the file.
\LaTeX\ source can be written with the \texttt{show proof}, \texttt{show
new{\char`\_}proof}, and \texttt{show statement} commands using the
\texttt{/tex} qualifier.

The mapping to \LaTeX\ symbols is defined in a special comment
containing a \texttt{\$t} token, described in Appendix~\ref{ASCII}.

There is an optional command qualifier:

    \texttt{/no{\char`\_}header} - This qualifier prevents a standard
        \LaTeX\ header and trailer
        from being included with the output \LaTeX\ code.


\subsection{\texttt{close tex} Command}\index{\texttt{close tex} command}
Syntax:  \texttt{close tex}

This command writes a trailer to any \LaTeX\ file\index{latex@{\LaTeX}}
that was opened with \texttt{open tex} (unless
\texttt{/no{\char`\_}header} was used with \texttt{open tex}) and closes
the \LaTeX\ file.


\section{Creating {\sc HTML} Output}\label{htmlout}

You can generate {\sc html} web pages given the
information in a database.
The database must already include the necessary typesetting information
(see section \ref{tcomment} for how to provide this information).
The ability to produce {\sc html} web pages was added in Metamath version
0.07.30.

To create an {\sc html} output file(s) for \texttt{\$a} or \texttt{\$p}
statement(s), use
\begin{quote}
    \texttt{show statement} {\em label-match} \texttt{/html}
\end{quote}
The output file will be named {\em label-match}\texttt{.html}
for each match.  When {\em
label-match} has wildcard (\texttt{*}) characters, all statements with
matching labels will have {\sc html} files produced for them.  Also,
when {\em label-match} has a wildcard (\texttt{*}) character, two additional
files, \texttt{mmdefinitions.html} and \texttt{mmascii.html} will be
produced.  To produce {\em only} these two additional files, you can use
\texttt{?*}, which will not match any statement label, in place of {\em
label-match}.

There are three other qualifiers for \texttt{show statement} that also
generate {\sc HTML} code.  These are \texttt{/alt{\char`\_}html},
\texttt{/brief{\char`\_}html}, and
\texttt{/brief{\char`\_}alt{\char`\_}html}, and are described in the
next section.

The command
\begin{quote}
    \texttt{show statement} {\em label-match} \texttt{/alt{\char`\_}html}
\end{quote}
does the same as \texttt{show statement} {\em label-match} \texttt{/html},
except that the {\sc html} code for the symbols is taken from
\texttt{althtmldef} statements instead of \texttt{htmldef} statements in
the \texttt{\$t} comment.

The command
\begin{verbatim}
show statement * /brief_html
\end{verbatim}
invokes a special mode that just produces definition and theorem lists
accompanied by their symbol strings, in a format suitable for copying and
pasting into another web page (such as the tutorial pages on the
Metamath web site).

Finally, the command
\begin{verbatim}
show statement * /brief_alt_html
\end{verbatim}
does the same as \texttt{show statement * / brief{\char`\_}html}
for the alternate {\sc html}
symbol representation.

A statement's comment can include a special notation that provides a
certain amount of control over the {\sc HTML} version of the comment.  See
Section~\ref{mathcomments} (p.~\pageref{mathcomments}) for the comment
markup features.

The \texttt{write theorem{\char`\_}list} and \texttt{write bibliography}
commands, which are described below, provide as a side effect complete
error checking for all of the features described in this
section.\index{error checking}

\subsection{\texttt{write theorem\_list}
Command}\index{\texttt{write theorem{\char`\_}list} command}
Syntax:  \texttt{write theorem{\char`\_}list}
[\texttt{/theorems{\char`\_}per{\char`\_}page} {\em number}]

This command writes a list of all of the \texttt{\$a} and \texttt{\$p}
statements in the database into a web page file
 called \texttt{mmtheorems.html}.
When additional files are needed, they are called
\texttt{mmtheorems2.html}, \texttt{mmtheorems3.html}, etc.

Optional command qualifier:

    \texttt{/theorems{\char`\_}per{\char`\_}page} {\em number} -
 This qualifier specifies the number of statements to
        write per web page.  The default is 100.

{\em Note:} In version 0.177\index{Metamath!limitations of version
0.177} of Metamath, the ``Nearby theorems'' links on the individual
web pages presuppose 100 theorems per page when linking to the theorem
list pages.  Therefore the \texttt{/theorems{\char`\_}per{\char`\_}page}
qualifier, if it specifies a number other than 100, will cause the
individual web pages to be out of sync and should not be used to
generate the main theorem list for the web site.  This may be
fixed in a future version.


\subsection{\texttt{write bibliography}\label{wrbib}
Command}\index{\texttt{write bibliography} command}
Syntax:  \texttt{write bibliography} {\em filename}

This command reads an existing {\sc html} bibliographic cross-reference
file, normally called \texttt{mmbiblio.html}, and updates it per the
bibliographic links in the database comments.  The file is updated
between the {\sc html} comment lines \texttt{<!--
{\char`\#}START{\char`\#} -->} and \texttt{<!-- {\char`\#}END{\char`\#}
-->}.  The original input file is renamed to {\em
filename}\texttt{{\char`\~}1}.

A bibliographic reference is indicated with the reference name
in brackets, such as  \texttt{Theorem 3.1 of
[Monk] p.\ 22}.
See Section~\ref{htmlout} (p.~\pageref{htmlout}) for
syntax details.


\subsection{\texttt{write recent\_additions}
Command}\index{\texttt{write recent{\char`\_}additions} command}
Syntax:  \texttt{write recent{\char`\_}additions} {\em filename}
[\texttt{/limit} {\em number}]

This command reads an existing ``Recent Additions'' {\sc html} file,
normally called \texttt{mmrecent.html}, and updates it with the
descriptions of the most recently added theorems to the database.
 The file is updated between
the {\sc html} comment lines \texttt{<!-- {\char`\#}START{\char`\#} -->}
and \texttt{<!-- {\char`\#}END{\char`\#} -->}.  The original input file
is renamed to {\em filename}\texttt{{\char`\~}1}.

Optional command qualifier:

    \texttt{/limit} {\em number} -
 This qualifier specifies the number of most recent theorems to
   write to the output file.  The default is 100.


\section{Text File Utilities}

\subsection{\texttt{tools} Command}\index{\texttt{tools} command}
Syntax:  \texttt{tools}

This command invokes an easy-to-use, general purpose utility for
manipulating the contents of {\sc ascii} text files.  Upon typing
\texttt{tools}, the command-line prompt will change to \texttt{TOOLS>}
until you type \texttt{exit}.  The \texttt{tools} commands can be used
to perform simple, global edits on an input/output file,
such as making a character string substitution on each line, adding a
string to each line, and so on.  A typical use of this utility is
to build a \texttt{submit} input file to perform a common operation on a
list of statements obtained from \texttt{show label} or \texttt{show
usage}.

The actions of most of the \texttt{tools} commands can also be
performed with equivalent (and more powerful) Unix shell commands, and
some users may find those more efficient.  But for Windows users or
users not comfortable with Unix, \texttt{tools} provides an
easy-to-learn alternative that is adequate for most of the
script-building tasks needed to use the Metamath program effectively.

\subsection{\texttt{help} Command (in \texttt{tools})}
Syntax:  \texttt{help}

The \texttt{help} command lists the commands available in the
\texttt{tools} utility, along with a brief description.  Each command,
in turn, has its own help, such as \texttt{help add}.  As with
Metamath's \texttt{MM>} prompt, a complete command can be entered at
once, or just the command word can be typed, causing the program to
prompt for each argument.

\vskip 1ex
\noindent Line-by-line editing commands:

  \texttt{add} - Add a specified string to each line in a file.

  \texttt{clean} - Trim spaces and tabs on each line in a file; convert
         characters.

  \texttt{delete} - Delete a section of each line in a file.

  \texttt{insert} - Insert a string at a specified column in each line of
        a file.

  \texttt{substitute} - Make a simple substitution on each line of the file.

  \texttt{tag} - Like \texttt{add}, but restricted to a range of lines.

  \texttt{swap} - Swap the two halves of each line in a file.

\vskip 1ex
\noindent Other file-processing commands:

  \texttt{break} - Break up (tokenize) a file into a list of tokens (one per
        line).

  \texttt{build} - Build a file with multiple tokens per line from a list.

  \texttt{count} - Count the occurrences in a file of a specified string.

  \texttt{number} - Create a list of numbers.

  \texttt{parallel} - Put two files in parallel.

  \texttt{reverse} - Reverse the order of the lines in a file.

  \texttt{right} - Right-justify lines in a file (useful before sorting
         numbers).

%  \texttt{tag} - Tag edit updates in a program for revision control.

  \texttt{sort} - Sort the lines in a file with key starting at
         specified string.

  \texttt{match} - Extract lines containing (or not) a specified string.

  \texttt{unduplicate} - Eliminate duplicate occurrences of lines in a file.

  \texttt{duplicate} - Extract first occurrence of any line occurring
         more than

   \ \ \    once in a file, discarding lines occurring exactly once.

  \texttt{unique} - Extract lines occurring exactly once in a file.

  \texttt{type} (10 lines) - Display the first few lines in a file.
                  Similar to Unix \texttt{head}.

  \texttt{copy} - Similar to Unix \texttt{cat} but safe (same input
         and output file allowed).

  \texttt{submit} - Run a script containing \texttt{tools} commands.

\vskip 1ex

\noindent Note:
  \texttt{unduplicate}, \texttt{duplicate}, and \texttt{unique} also
 sort the lines as a side effect.


\subsection{Using \texttt{tools} to Build Metamath \texttt{submit}
Scripts}

The \texttt{break} command is typically used to break up a series of
statement labels, such as the output of Metamath's \texttt{show usage},
into one label per line.  The other \texttt{tools} commands can then be
used to add strings before and after each statement label to specify
commands to be performed on the statement.  The \texttt{parallel}
command is useful when a statement label must be mentioned more than
once on a line.

Very often a \texttt{submit} script for Metamath will require multiple
command lines for each statement being processed.  For example, you may
want to enter the Proof Assistant, \texttt{minimize{\char`\_}with} your
latest theorem, \texttt{save} the new proof, and \texttt{exit} the Proof
Assistant.  To accomplish this, you can build a file with these four
commands for each statement on a single line, separating each command
with a designated character such as \texttt{@}.  Then at the end you can
\texttt{substitute} each \texttt{@} with \texttt{{\char`\\}n} to break
up the lines into individual command lines (see \texttt{help
substitute}).


\subsection{Example of a \texttt{tools} Session}

To give you a quick feel for the \texttt{tools} utility, we show a
simple session where we create a file \texttt{n.txt} with 3 lines, add
strings before and after each line, and display the lines on the screen.
You can experiment with the various commands to gain experience with the
\texttt{tools} utility.

\begin{verbatim}
MM> tools
Entering the Text Tools utilities.
Type HELP for help, EXIT to exit.
TOOLS> number
Output file <n.tmp>? n.txt
First number <1>?
Last number <10>? 3
Increment <1>?
TOOLS> add
Input/output file? n.txt
String to add to beginning of each line <>? This is line
String to add to end of each line <>? .
The file n.txt has 3 lines; 3 were changed.
First change is on line 1:
This is line 1.
TOOLS> type n.txt
This is line 1.
This is line 2.
This is line 3.
TOOLS> exit
Exiting the Text Tools.
Type EXIT again to exit Metamath.
MM>
\end{verbatim}



\appendix
\chapter{Sample Representations}
\label{ASCII}

This Appendix provides a sample of {\sc ASCII} representations,
their corresponding traditional mathematical symbols,
and a discussion of their meanings
in the \texttt{set.mm} database.
These are provided in order of appearance.
This is only a partial list, and new definitions are routinely added.
A complete list is available at \url{http://metamath.org}.

These {\sc ASCII} representations, along
with information on how to display them,
are defined in the \texttt{set.mm} database file inside
a special comment called a \texttt{\$t} {\em
comment}\index{\texttt{\$t} comment} or {\em typesetting
comment.}\index{typesetting comment}
A typesetting comment
is indicated by the appearance of the
two-character string \texttt{\$t} at the beginning of the comment.
For more information,
see Section~\ref{tcomment}, p.~\pageref{tcomment}.

In the following table the ``{\sc ASCII}'' column shows the {\sc ASCII}
representation,
``Symbol'' shows the mathematical symbolic display
that corresponds to that {\sc ASCII} representation, ``Labels'' shows
the key label(s) that define the representation, and
``Description'' provides a description about the symbol.
As usual, ``iff'' is short for ``if and only if.''\index{iff}
In most cases the ``{\sc ASCII}'' column only shows
the key token, but it will sometimes show a sequence of tokens
if that is necessary for clarity.

{\setlength{\extrarowsep}{4pt} % Keep rows from being too close together
\begin{longtabu}   { @{} c c l X }
\textbf{ASCII} & \textbf{Symbol} & \textbf{Labels} & \textbf{Description} \\
\endhead
\texttt{|-} & $\vdash$ & &
  ``It is provable that...'' \\
\texttt{ph} & $\varphi$ & \texttt{wph} &
  The wff (boolean) variable phi,
  conventionally the first wff variable. \\
\texttt{ps} & $\psi$ & \texttt{wps} &
  The wff (boolean) variable psi,
  conventionally the second wff variable. \\
\texttt{ch} & $\chi$ & \texttt{wch} &
  The wff (boolean) variable chi,
  conventionally the third wff variable. \\
\texttt{-.} & $\lnot$ & \texttt{wn} &
  Logical not. E.g., if $\varphi$ is true, then $\lnot \varphi$ is false. \\
\texttt{->} & $\rightarrow$ & \texttt{wi} &
  Implies, also known as material implication.
  In classical logic the expression $\varphi \rightarrow \psi$ is true
  if either $\varphi$ is false or $\psi$ is true (or both), that is,
  $\varphi \rightarrow \psi$ has the same meaning as
  $\lnot \varphi \lor \psi$ (as proven in theorem \texttt{imor}). \\
\texttt{<->} & $\leftrightarrow$ &
  \hyperref[df-bi]{\texttt{df-bi}} &
  Biconditional (aka is-equals for boolean values).
  $\varphi \leftrightarrow \psi$ is true iff
  $\varphi$ and $\psi$ have the same value. \\
\texttt{\char`\\/} & $\lor$ &
  \makecell[tl]{{\hyperref[df-or]{\texttt{df-or}}}, \\
	         \hyperref[df-3or]{\texttt{df-3or}}} &
  Disjunction (logical ``or''). $\varphi \lor \psi$ is true iff
  $\varphi$, $\psi$, or both are true. \\
\texttt{/\char`\\} & $\land$ &
  \makecell[tl]{{\hyperref[df-an]{\texttt{df-an}}}, \\
                 \hyperref[df-3an]{\texttt{df-3an}}} &
  Conjunction (logical ``and''). $\varphi \land \psi$ is true iff
  both $\varphi$ and $\psi$ are true. \\
\texttt{A.} & $\forall$ &
  \texttt{wal} &
  For all; the wff $\forall x \varphi$ is true iff
  $\varphi$ is true for all values of $x$. \\
\texttt{E.} & $\exists$ &
  \hyperref[df-ex]{\texttt{df-ex}} &
  There exists; the wff
  $\exists x \varphi$ is true iff
  there is at least one $x$ where $\varphi$ is true. \\
\texttt{[ y / x ]} & $[ y / x ]$ &
  \hyperref[df-sb]{\texttt{df-sb}} &
  The wff $[ y / x ] \varphi$ produces
  the result when $y$ is properly substituted for $x$ in $\varphi$
  ($y$ replaces $x$).
  % This is elsb4
  % ( [ x / y ] z e. y <-> z e. x )
  For example,
  $[ x / y ] z \in y$ is the same as $z \in x$. \\
\texttt{E!} & $\exists !$ &
  \hyperref[df-eu]{\texttt{df-eu}} &
  There exists exactly one;
  $\exists ! x \varphi$ is true iff
  there is at least one $x$ where $\varphi$ is true. \\
\texttt{\{ y | phi \}}  & $ \{ y | \varphi \}$ &
  \hyperref[df-clab]{\texttt{df-clab}} &
  The class of all sets where $\varphi$ is true. \\
\texttt{=} & $ = $ &
  \hyperref[df-cleq]{\texttt{df-cleq}} &
  Class equality; $A = B$ iff $A$ equals $B$. \\
\texttt{e.} & $ \in $ &
  \hyperref[df-clel]{\texttt{df-clel}} &
  Class membership; $A \in B$ if $A$ is a member of $B$. \\
\texttt{{\char`\_}V} & {\rm V} &
  \hyperref[df-v]{\texttt{df-v}} &
  Class of all sets (not itself a set). \\
\texttt{C\_} & $ \subseteq $ &
  \hyperref[df-ss]{\texttt{df-ss}} &
  Subclass (subset); $A \subseteq B$ is true iff
  $A$ is a subclass of $B$. \\
\texttt{u.} & $ \cup $ &
  \hyperref[df-un]{\texttt{df-un}} &
  $A \cup B$ is the union of classes $A$ and $B$. \\
\texttt{i^i} & $ \cap $ &
  \hyperref[df-in]{\texttt{df-in}} &
  $A \cap B$ is the intersection of classes $A$ and $B$. \\
\texttt{\char`\\} & $ \setminus $ &
  \hyperref[df-dif]{\texttt{df-dif}} &
  $A \setminus B$ (set difference)
  is the class of all sets in $A$ except for those in $B$. \\
\texttt{(/)} & $ \varnothing $ &
  \hyperref[df-nul]{\texttt{df-nul}} &
  $ \varnothing $ is the empty set (aka null set). \\
\texttt{\char`\~P} & $ \cal P $ &
  \hyperref[df-pw]{\texttt{df-pw}} &
  Power class. \\
\texttt{<.\ A , B >.} & $\langle A , B \rangle$ &
  \hyperref[df-op]{\texttt{df-op}} &
  The ordered pair $\langle A , B \rangle$. \\
\texttt{( F ` A )} & $ ( F ` A ) $ &
  \hyperref[df-fv]{\texttt{df-fv}} &
  The value of function $F$ when applied to $A$. \\
\texttt{\_i} & $ i $ &
  \texttt{df-i} &
  The square root of negative one. \\
\texttt{x.} & $ \cdot $ &
  \texttt{df-mul} &
  Complex number multiplication; $2~\cdot~3~=~6$. \\
\texttt{CC} & $ \mathbb{C} $ &
  \texttt{df-c} &
  The set of complex numbers. \\
\texttt{RR} & $ \mathbb{R} $ &
  \texttt{df-r} &
  The set of real numbers. \\
\end{longtabu}
} % end of extrarowsep

\chapter{Compressed Proofs}
\label{compressed}\index{compressed proof}\index{proof!compressed}

The proofs in the \texttt{set.mm} set theory database are stored in compressed
format for efficiency.  Normally you needn't concern yourself with the
compressed format, since you can display it with the usual proof display tools
in the Metamath program (\texttt{show proof}\ldots) or convert it to the normal
RPN proof format described in Section~\ref{proof} (with \texttt{save proof}
{\em label} \texttt{/normal}).  However for sake of completeness we describe the
format here and show how it maps to the normal RPN proof format.

A compressed proof, located between \texttt{\$=} and \texttt{\$.}\ keywords, consists
of a left parenthesis, a sequence of statement labels, a right parenthesis,
and a sequence of upper-case letters \texttt{A} through \texttt{Z} (with optional
white space between them).  White space must surround the parentheses
and the labels.  The left parenthesis tells Metamath that a
compressed proof follows.  (A normal RPN proof consists of just a sequence of
labels, and a parenthesis is not a legal character in a label.)

The sequence of upper-case letters corresponds to a sequence of integers
with the following mapping.  Each integer corresponds to a proof step as
described later.
\begin{center}
  \texttt{A} = 1 \\
  \texttt{B} = 2 \\
   \ldots \\
  \texttt{T} = 20 \\
  \texttt{UA} = 21 \\
  \texttt{UB} = 22 \\
   \ldots \\
  \texttt{UT} = 40 \\
  \texttt{VA} = 41 \\
  \texttt{VB} = 42 \\
   \ldots \\
  \texttt{YT} = 120 \\
  \texttt{UUA} = 121 \\
   \ldots \\
  \texttt{YYT} = 620 \\
  \texttt{UUUA} = 621 \\
   etc.
\end{center}

In other words, \texttt{A} through \texttt{T} represent the
least-significant digit in base 20, and \texttt{U} through \texttt{Y}
represent zero or more most-significant digits in base 5, where the
digits start counting at 1 instead of the usual 0. With this scheme, we
don't need white space between these ``numbers.''

(In the design of the compressed proof format, only upper-case letters,
as opposed to say all non-whitespace printable {\sc ascii} characters other than
%\texttt{\$}, was chosen to make the compressed proof a little less
%displeasing to the eye, at the expense of a typical 20\% compression
\texttt{\$}, were chosen so as not to collide with most text editor
searches, at the expense of a typical 20\% compression
loss.  The base 5/base 20 grouping, as opposed to say base 6/base 19,
was chosen by experimentally determining the grouping that resulted in
best typical compression.)

The letter \texttt{Z} identifies (tags) a proof step that is identical to one
that occurs later on in the proof; it helps shorten the proof by not requiring
that identical proof steps be proved over and over again (which happens often
when building wff's).  The \texttt{Z} is placed immediately after the
least-significant digit (letters \texttt{A} through \texttt{T}) that ends the integer
corresponding to the step to later be referenced.

The integers that the upper-case letters correspond to are mapped to labels as
follows.  If the statement being proved has $m$ mandatory hypotheses, integers
1 through $m$ correspond to the labels of these hypotheses in the order shown
by the \texttt{show statement ... / full} command, i.e., the RPN order\index{RPN
order} of the mandatory
hypotheses.  Integers $m+1$ through $m+n$ correspond to the labels enclosed in
the parentheses of the compressed proof, in the order that they appear, where
$n$ is the number of those labels.  Integers $m+n+1$ on up don't directly
correspond to statement labels but point to proof steps identified with the
letter \texttt{Z}, so that these proof steps can be referenced later in the
proof.  Integer $m+n+1$ corresponds to the first step tagged with a \texttt{Z},
$m+n+2$ to the second step tagged with a \texttt{Z}, etc.  When the compressed
proof is converted to a normal proof, the entire subproof of a step tagged
with \texttt{Z} replaces the reference to that step.

For efficiency, Metamath works with compressed proofs directly, without
converting them internally to normal proofs.  In addition to the usual
error-checking, an error message is given if (1) a label in the label list in
parentheses does not refer to a previous \texttt{\$p} or \texttt{\$a} statement or a
non-mandatory hypothesis of the statement being proved and (2) a proof step
tagged with \texttt{Z} is referenced before the step tagged with the \texttt{Z}.

Just as in a normal proof under development (Section~\ref{unknown}), any step
or subproof that is not yet known may be represented with a single \texttt{?}.
White space does not have to appear between the \texttt{?}\ and the upper-case
letters (or other \texttt{?}'s) representing the remainder of the proof.

% April 1, 2004 Appendix C has been added back in with corrections.
%
% May 20, 2003 Appendix C was removed for now because there was a problem found
% by Bob Solovay
%
% Also, removed earlier \ref{formalspec} 's (3 cases above)
%
% Bob Solovay wrote on 30 Nov 2002:
%%%%%%%%%%%%% (start of email comment )
%      3. My next set of comments concern appendix C. I read this before I
% read Chapter 4. So I first noted that the system as presented in the
% Appendix lacked a certain formal property that I thought desirable. I
% then came up with a revised formal system that had this property. Upon
% reading Chapter 4, I noticed that the revised system was closer to the
% treatment in Chapter 4 than the system in Appendix C.
%
%         First a very minor correction:
%
%         On page 142 line 2: The condition that V(e) != V(f) should only be
% required of e, f in T such that e != f.
%
%         Here is a natural property [transitivity] that one would like
% the formal system to have:
%
%         Let Gamma be a set of statements. Suppose that the statement Phi
% is provable from Gamma and that the statement Psi is provable from Gamma
% \cup {Phi}. Then Psi is provable from Gamma.
%
%         I shall present an example to show that this property does not
% hold for the formal systems of Appendix C:
%
%         I write the example in metamath style:
%
% $c A B C D E $.
% $v x y
%
% ${
% tx $f A x $.
% ty $f B y $.
% ax1 $a C x y $.
% $}
%
% ${
% tx $f A x $.
% ty $f B y $.
% ax2-h1 $e C x y $.
% ax2 $a D y $.
% $}
%
% ${
% ty $f B y $.
% ax3-h1 $e D y $.
% ax3 $a E y $.
% $}
%
% $(These three axioms are Gamma $)
%
% ${
% tx $f A x $.
% ty $f B y $.
% Phi $p D y $=
% tx ty tx ty ax1 ax2 $.
% $}
%
% ${
% ty $f B y $.
% Psi $p E y $=
% ty ty Phi ax3 $.
% $}
%
%
% I omit the formal proofs of the following claims. [I will be glad to
% supply them upon request.]
%
% 1) Psi is not provable from Gamma;
%
% 2) Psi is provable from Gamma + Phi.
%
% Here "provable" refers to the formalism of Appendix C.
%
% The trouble of course is that Psi is lacking the variable declaration
%
% $f Ax $.
%
% In the Metamath system there is no trouble proving Psi. I attach a
% metamath file that shows this and which has been checked by the
% metamath program.
%
% I next want to indicate how I think the treatment in Appendix C should
% be revised so as to conform more closely to the metamath system of the
% main text. The revised system *does* have the transitivity property.
%
% We want to give revised definitions of "statement" and
% "provable". [cf. sections C.2.4. and C.2.5] Our new definitions will
% use the definitions given in Appendix C. So we take the following
% tack. We refer to the original notions as o-statement and o-provable. And
% we refer to the notions we are defining as n-statement and n-provable.
%
%         A n-statement is an o-statement in which the only variables
% that appear in the T component are mandatory.
%
%         To any o-statement we can associate its reduct which is a
% n-statement by dropping all the elements of T or D which contain
% non-mandatory variables.
%
%         An n-statement gamma is n-provable if there is an o-statement
% gamma' which has gamma as its reduct andf such that gamma' is
% o-provable.
%
%         It seems to me [though I am not completely sure on this point]
% that n-provability corresponds to metamath provability as discussed
% say in Chapter 4.
%
%         Attached to this letter is the metamath proof of Phi and Psi
% from Gamma discussed above.
%
%         I am still brooding over the question of whether metamath
% correctly formalizes set-theory. No doubt I will have some questions
% re this after my thoughts become clearer.
%%%%%%%%%%%%%%%% (end of email comment)

%%%%%%%%%%%%%%%% (start of 2nd email comment from Bob Solovay 1-Apr-04)
%
%         I hope that Appendix C is the one that gives a "formal" treatment
% of Metamath. At any rate, thats the appendix I want to comment on.
%
%         I'm going to suggest two changes in the definition.
%
%         First change (in the definition of statement): Require that the
% sets D, T, and E be finite.
%
%         Probably things are fine as you give them. But in the applications
% to the main metamath system they will always be finite, and its useful in
% thinking about things [at least for me] to stick to the finite case.
%
%         Second change:
%
%         First let me give an approximate description. Remove the dummy
% variables from the statement. Instead, include them in the proof.
%
%         More formally: Require that T consists of type declarations only
% for mandatory variables. Require that all the pairs in D consist of
% mandatory variables.
%
%         At the start of a proof we are allowed to declare a finite number
% of dummy variables [provided that none of them appear in any of the
% statements in E \cup {A}. We have to supply type declarations for all the
% dummy variables. We are allowed to add new $d statements referring to
% either the mandatory or dummy variables. But we require that no new $d
% statement references only mandatory variables.
%
%         I find this way of doing things more conceptual than the treatment
% in Appendix C. But the change [which I will use implicitly in later
% letters about doing Peano] is mainly aesthetic. I definitely claim that my
% results on doing Peano all apply to Metamath as it is presented in your
% book.
%
%         --Bob
%
%%%%%%%%%%%%%%%% (end of 2nd email comment)

%%
%% When uncommenting the below, also uncomment references above to {formalspec}
%%
\chapter{Metamath's Formal System}\label{formalspec}\index{Metamath!as a formal
system}

\section{Introduction}

\begin{quote}
  {\em Perfection is when there is no longer anything more to take away.}
    \flushright\sc Antoine de
     Saint-Exupery\footnote{\cite[p.~3-25]{Campbell}.}\\
\end{quote}\index{de Saint-Exupery, Antoine}

This appendix describes the theory behind the Metamath language in an abstract
way intended for mathematicians.  Specifically, we construct two
set-theo\-ret\-i\-cal objects:  a ``formal system'' (roughly, a set of syntax
rules, axioms, and logical rules) and its ``universe'' (roughly, the set of
theorems derivable in the formal system).  The Metamath computer language
provides us with a way to describe specific formal systems and, with the aid of
a proof provided by the user, to verify that given theorems
belong to their universes.

To understand this appendix, you need a basic knowledge of informal set theory.
It should be sufficient to understand, for example, Ch.\ 1 of Munkres' {\em
Topology} \cite{Munkres}\index{Munkres, James R.} or the
introductory set theory chapter
in many textbooks that introduce abstract mathematics. (Note that there are
minor notational differences among authors; e.g.\ Munkres uses $\subset$ instead
of our $\subseteq$ for ``subset.''  We use ``included in'' to mean ``a subset
of,'' and ``belongs to'' or ``is contained in'' to mean ``is an element of.'')
What we call a ``formal'' description here, unlike earlier, is actually an
informal description in the ordinary language of mathematicians.  However we
provide sufficient detail so that a mathematician could easily formalize it,
even in the language of Metamath itself if desired.  To understand the logic
examples at the end of this appendix, familiarity with an introductory book on
mathematical logic would be helpful.

\section{The Formal Description}

\subsection[Preliminaries]{Preliminaries\protect\footnotemark}%
\footnotetext{This section is taken mostly verbatim
from Tarski \cite[p.~63]{Tarski1965}\index{Tarski, Alfred}.}

By $\omega$ we denote the set of all natural numbers (non-negative integers).
Each natural number $n$ is identified with the set of all smaller numbers: $n =
\{ m | m < n \}$.  The formula $m < n$ is thus equivalent to the condition: $m
\in n$ and $m,n \in \omega$. In particular, 0 is the number zero and at the
same time the empty set $\varnothing$, $1=\{0\}$, $2=\{0,1\}$, etc. ${}^B A$
denotes the set of all functions on $B$ to $A$ (i.e.\ with domain $B$ and range
included in $A$).  The members of ${}^\omega A$ are what are called {\em simple
infinite sequences},\index{simple infinite sequence}
with all {\em terms}\index{term} in $A$.  In case $n \in \omega$, the
members of ${}^n A$ are referred to as {\em finite $n$-termed
sequences},\index{finite $n$-termed
sequence} again
with terms in $A$.  The consecutive terms (function values) of a finite or
infinite sequence $f$ are denoted by $f_0, f_1, \ldots ,f_n,\ldots$.  Every
finite sequence $f \in \bigcup _{n \in \omega} {}^n A$ uniquely determines the
number $n$ such that $f \in {}^n A$; $n$ is called the {\em
length}\index{length of a sequence ({$"|\ "|$})} of $f$ and
is denoted by $|f|$.  $\langle a \rangle$ is the sequence $f$ with $|f|=1$ and
$f_0=a$; $\langle a,b \rangle$ is the sequence $f$ with $|f|=2$, $f_0=a$,
$f_1=b$; etc.  Given two finite sequences $f$ and $g$, we denote by $f\frown g$
their {\em concatenation},\index{concatenation} i.e., the
finite sequence $h$ determined by the
conditions:
\begin{eqnarray*}
& |h| = |f|+|g|;&  \\
& h_n = f_n & \mbox{\ for\ } n < |f|;  \\
& h_{|f|+n} = g_n & \mbox{\ for\ } n < |g|.
\end{eqnarray*}

\subsection{Constants, Variables, and Expressions}

A formal system has a set of {\em symbols}\index{symbol!in
a formal system} denoted
by $\mbox{\em SM}$.  A
precise set-theo\-ret\-i\-cal definition of this set is unimportant; a symbol
could be considered a primitive or atomic element if we wish.  We assume this
set is divided into two disjoint subsets:  a set $\mbox{\em CN}$ of {\em
constants}\index{constant!in a formal system} and a set $\mbox{\em VR}$ of
{\em variables}.\index{variable!in a formal system}  $\mbox{\em CN}$ and
$\mbox{\em VR}$ are each assumed to consist of countably many symbols which
may be arranged in finite or simple infinite sequences $c_0, c_1, \ldots$ and
$v_0, v_1, \ldots$ respectively, without repeating terms.  We will represent
arbitrary symbols by metavariables $\alpha$, $\beta$, etc.

{\footnotesize\begin{quotation}
{\em Comment.} The variables $v_0, v_1, \ldots$ of our formal system
correspond to what are usually considered ``metavariables'' in
descriptions of specific formal systems in the literature.  Typically,
when describing a specific formal system a book will postulate a set of
primitive objects called variables, then proceed to describe their
properties using metavariables that range over them, never mentioning
again the actual variables themselves.  Our formal system does not
mention these primitive variable objects at all but deals directly with
metavariables, as its primitive objects, from the start.  This is a
subtle but key distinction you should keep in mind, and it makes our
definition of ``formal system'' somewhat different from that typically
found in the literature.  (So, the $\alpha$, $\beta$, etc.\ above are
actually ``metametavariables'' when used to represent $v_0, v_1,
\ldots$.)
\end{quotation}}

Finite sequences all terms of which are symbols are called {\em
expressions}.\index{expression!in a formal system}  $\mbox{\em EX}$ is
the set of all expressions; thus
\begin{displaymath}
\mbox{\em EX} = \bigcup _{n \in \omega} {}^n \mbox{\em SM}.
\end{displaymath}

A {\em constant-prefixed expression}\index{constant-prefixed expression}
is an expression of non-zero length
whose first term is a constant.  We denote the set of all constant-prefixed
expressions by $\mbox{\em EX}_C = \{ e \in \mbox{\em EX} | ( |e| > 0 \wedge
e_0 \in \mbox{\em CN} ) \}$.

A {\em constant-variable pair}\index{constant-variable pair}
is an expression of length 2 whose first term
is a constant and whose second term is a variable.  We denote the set of all
constant-variable pairs by $\mbox{\em EX}_2 = \{ e \in \mbox{\em EX}_C | ( |e|
= 2 \wedge e_1 \in \mbox{\em VR} ) \}$.


{\footnotesize\begin{quotation}
{\em Relationship to Metamath.} In general, the set $\mbox{\em SM}$
corresponds to the set of declared math symbols in a Metamath database, the
set $\mbox{\em CN}$ to those declared with \texttt{\$c} statements, and the set
$\mbox{\em VR}$ to those declared with \texttt{\$v} statements.  Of course a
Metamath database can only have a finite number of math symbols, whereas
formal systems in general can have an infinite number, although the number of
Metamath math symbols available is in principle unlimited.

The set $\mbox{\em EX}_C$ corresponds to the set of permissible expressions
for \texttt{\$e}, \texttt{\$a}, and \texttt{\$p} statements.  The set $\mbox{\em EX}_2$
corresponds to the set of permissible expressions for \texttt{\$f} statements.
\end{quotation}}

We denote by ${\cal V}(e)$ the set of all variables in an expression $e \in
\mbox{\em EX}$, i.e.\ the set of all $\alpha \in \mbox{\em VR}$ such that
$\alpha = e_n$ for some $n < |e|$.  We also denote (with abuse of notation) by
${\cal V}(E)$ the set of all variables in a collection of expressions $E
\subseteq \mbox{\em EX}$, i.e.\ $\bigcup _{e \in E} {\cal V}(e)$.


\subsection{Substitution}

Given a function $F$ from $\mbox{\em VR}$ to
$\mbox{\em EX}$, we
denote by $\sigma_{F}$ or just $\sigma$ the function from $\mbox{\em EX}$ to
$\mbox{\em EX}$ defined recursively for nonempty sequences by
\begin{eqnarray*}
& \sigma(<\alpha>) = F(\alpha) & \mbox{for\ } \alpha \in \mbox{\em VR}; \\
& \sigma(<\alpha>) = <\alpha> & \mbox{for\ } \alpha \not\in \mbox{\em VR}; \\
& \sigma(g \frown h) = \sigma(g) \frown
    \sigma(h) & \mbox{for\ } g,h \in \mbox{\em EX}.
\end{eqnarray*}
We also define $\sigma(\varnothing)=\varnothing$.  We call $\sigma$ a {\em
simultaneous substitution}\index{substitution!variable}\index{variable
substitution} (or just {\em substitution}) with {\em substitution
map}\index{substitution map} $F$.

We also denote (with abuse of notation) by $\sigma(E)$ a substitution on a
collection of expressions $E \subseteq \mbox{\em EX}$, i.e.\ the set $\{
\sigma(e) | e \in E \}$.  The collection $\sigma(E)$ may of course contain
fewer expressions than $E$ because duplicate expressions could result from the
substitution.

\subsection{Statements}

We denote by $\mbox{\em DV}$ the set of all
unordered pairs $\{\alpha, \beta \} \subseteq \mbox{\em VR}$ such that $\alpha
\neq \beta$.  $\mbox{\em DV}$ stands for ``distinct variables.''

A {\em pre-statement}\index{pre-statement!in a formal system} is a
quadruple $\langle D,T,H,A \rangle$ such that
$D\subseteq \mbox{\em DV}$, $T\subseteq \mbox{\em EX}_2$, $H\subseteq
\mbox{\em EX}_C$ and $H$ is finite,
$A\in \mbox{\em EX}_C$, ${\cal V}(H\cup\{A\}) \subseteq
{\cal V}(T)$, and $\forall e,f\in T {\ } {\cal V}(e) \neq {\cal V}(f)$ (or
equivalently, $e_1 \ne f_1$) whenever $e \neq f$. The terms of the quadruple are called {\em
distinct-variable restrictions},\index{disjoint-variable restriction!in a
formal system} {\em variable-type hypotheses},\index{variable-type
hypothesis!in a formal system} {\em logical hypotheses},\index{logical
hypothesis!in a formal system} and the {\em assertion}\index{assertion!in a
formal system} respectively.  We denote by $T_M$ ({\em mandatory variable-type
hypotheses}\index{mandatory variable-type hypothesis!in a formal system}) the
subset of $T$ such that ${\cal V}(T_M) ={\cal V}(H \cup \{A\})$.  We denote by
$D_M=\{\{\alpha,\beta\}\in D|\{\alpha,\beta\}\subseteq {\cal V}(T_M)\}$ the
{\em mandatory distinct-variable restrictions}\index{mandatory
disjoint-variable restriction!in a formal system} of the pre-statement.
The set
of {\em mandatory hypotheses}\index{mandatory hypothesis!in a formal system}
is $T_M\cup H$.  We call the quadruple $\langle D_M,T_M,H,A \rangle$
the {\em reduct}\index{reduct!in a formal system} of
the pre-statement $\langle D,T,H,A \rangle$.

A {\em statement} is the reduct of some pre-statement\index{statement!in a
formal system}.  A statement is therefore a special kind of pre-statement;
in particular, a statement is the reduct of itself.

{\footnotesize\begin{quotation}
{\em Comment.}  $T$ is a set of expressions, each of length 2, that associate
a set of constants (``variable types'') with a set of variables.  The
condition ${\cal V}(H\cup\{A\}) \subseteq {\cal V}(T) $
means that each variable occurring in a statement's logical
hypotheses or assertion must have an associated variable-type hypothesis or
``type declaration,'' in  analogy to a computer programming language, where a
variable must be declared to be say, a string or an integer.  The requirement
that $\forall e,f\in T \, e_1 \ne f_1$ for $e\neq f$
means that each variable must be
associated with a unique constant designating its variable type; e.g., a
variable might be a ``wff'' or a ``set'' but not both.

Distinct-variable restrictions are used to specify what variable substitutions
are permissible to make for the statement to remain valid.  For example, in
the theorem scheme of set theory $\lnot\forall x\,x=y$ we may not substitute
the same variable for both $x$ and $y$.  On the other hand, the theorem scheme
$x=y\to y=x$ does not require that $x$ and $y$ be distinct, so we do not
require a distinct-variable restriction, although having one
would cause no harm other than making the scheme less general.

A mandatory variable-type hypothesis is one whose variable exists in a logical
hypothesis or the assertion.  A provable pre-statement
(defined below) may require
non-mandatory variable-type hypotheses that effectively introduce ``dummy''
variables for use in its proof.  Any number of dummy variables might
be required by a specific proof; indeed, it has been shown by H.\
Andr\'{e}ka\index{Andr{\'{e}}ka, H.} \cite{Nemeti} that there is no finite
upper bound to the number of dummy variables needed to prove an arbitrary
theorem in first-order logic (with equality) having a fixed number $n>2$ of
individual variables.  (See also the Comment on p.~\pageref{nodd}.)
For this reason we do not set a finite size bound on the collections $D$ and
$T$, although in an actual application (Metamath database) these will of
course be finite, increased to whatever size is necessary as more
proofs are added.
\end{quotation}}

{\footnotesize\begin{quotation}
{\em Relationship to Metamath.} A pre-statement of a formal system
corresponds to an extended frame in a Metamath database
(Section~\ref{frames}).  The collections $D$, $T$, and $H$ correspond
respectively to the \texttt{\$d}, \texttt{\$f}, and \texttt{\$e}
statement collections in an extended frame.  The expression $A$
corresponds to the \texttt{\$a} (or \texttt{\$p}) statement in an
extended frame.

A statement of a formal system corresponds to a frame in a Metamath
database.
\end{quotation}}

\subsection{Formal Systems}

A {\em formal system}\index{formal system} is a
triple $\langle \mbox{\em CN},\mbox{\em
VR},\Gamma\rangle$ where $\Gamma$ is a set of statements.  The members of
$\Gamma$ are called {\em axiomatic statements}.\index{axiomatic
statement!in a formal system}  Sometimes we will refer to a
formal system by just $\Gamma$ when $\mbox{\em CN}$ and $\mbox{\em VR}$ are
understood.

Given a formal system $\Gamma$, the {\em closure}\index{closure}\footnote{This
definition of closure incorporates a simplification due to
Josh Purinton.\index{Purinton, Josh}.} of a
pre-statement
$\langle D,T,H,A \rangle$ is the smallest set $C$ of expressions
such that:
%\begin{enumerate}
%  \item $T\cup H\subseteq C$; and
%  \item If for some axiomatic statement
%    $\langle D_M',T_M',H',A' \rangle \in \Gamma_A$, for
%    some $E \subseteq C$, some $F \subseteq C-T$ (where ``-'' denotes
%    set difference), and some substitution
%    $\sigma$ we have
%    \begin{enumerate}
%       \item $\sigma(T_M') = E$ (where, as above, the $M$ denotes the
%           mandatory variable-type hypotheses of $T^A$);
%       \item $\sigma(H') = F$;
%       \item for all $\{\alpha,\beta\}\in D^A$ and $\subseteq
%         {\cal V}(T_M')$, for all $\gamma\in {\cal V}(\sigma(\langle \alpha
%         \rangle))$, and for all $\delta\in  {\cal V}(\sigma(\langle \beta
%         \rangle))$, we have $\{\gamma, \delta\} \in D$;
%   \end{enumerate}
%   then $\sigma(A') \in C$.
%\end{enumerate}
\begin{list}{}{\itemsep 0.0pt}
  \item[1.] $T\cup H\subseteq C$; and
  \item[2.] If for some axiomatic statement
    $\langle D_M',T_M',H',A' \rangle \in
       \Gamma$ and for some substitution
    $\sigma$ we have
    \begin{enumerate}
       \item[a.] $\sigma(T_M' \cup H') \subseteq C$; and
       \item[b.] for all $\{\alpha,\beta\}\in D_M'$, for all $\gamma\in
         {\cal V}(\sigma(\langle \alpha
         \rangle))$, and for all $\delta\in  {\cal V}(\sigma(\langle \beta
         \rangle))$, we have $\{\gamma, \delta\} \in D$;
   \end{enumerate}
   then $\sigma(A') \in C$.
\end{list}
A pre-statement $\langle D,T,H,A
\rangle$ is {\em provable}\index{provable statement!in a formal
system} if $A\in C$ i.e.\ if its assertion belongs to its
closure.  A statement is {\em provable} if it is
the reduct of a provable pre-statement.
The {\em universe}\index{universe of a formal system}
of a formal system is
the collection of all of its provable statements.  Note that the
set of axiomatic statements $\Gamma$ in a formal system is a subset of its
universe.

{\footnotesize\begin{quotation}
{\em Comment.} The first condition in the definition of closure simply says
that the hypotheses of the pre-statement are in its closure.

Condition 2(a) says that a substitution exists that makes the
mandatory hypotheses of an axiomatic statement exactly match some members of
the closure.  This is what we explicitly demonstrate in a Metamath language
proof.

%Conditions 2(a) and 2(b) say that a substitution exists that makes the
%(mandatory) hypotheses of an axiomatic statement exactly match some members of
%the closure.  This is what we explicitly demonstrate with a Metamath language
%proof.
%
%The set of expressions $F$ in condition 2(b) excludes the variable-type
%hypotheses; this is done because non-mandatory variable-type hypotheses are
%effectively ``dropped'' as irrelevant whereas logical hypotheses must be
%retained to achieve a consistent logical system.

Condition 2(b) describes how distinct-variable restrictions in the axiomatic
statement must be met.  It means that after a substitution for two variables
that must be distinct, the resulting two expressions must either contain no
variables, or if they do, they may not have variables in common, and each pair
of any variables they do have, with one variable from each expression, must be
specified as distinct in the original statement.
\end{quotation}}

{\footnotesize\begin{quotation}
{\em Relationship to Metamath.} Axiomatic statements
 and provable statements in a formal
system correspond to the frames for \texttt{\$a} and \texttt{\$p} statements
respectively in a Metamath database.  The set of axiomatic statements is a
subset of the set of provable statements in a formal system, although in a
Metamath database a \texttt{\$a} statement is distinguished by not having a
proof.  A Metamath language proof for a \texttt{\$p} statement tells the computer
how to explicitly construct a series of members of the closure ultimately
leading to a demonstration that the assertion
being proved is in the closure.  The actual closure typically contains
an infinite number of expressions.  A formal system itself does not have
an explicit object called a ``proof'' but rather the existence of a proof
is implied indirectly by membership of an assertion in a provable
statement's closure.  We do this to make the formal system easier
to describe in the language of set theory.

We also note that once established as provable, a statement may be considered
to acquire the same status as an axiomatic statement, because if the set of
axiomatic statements is extended with a provable statement, the universe of
the formal system remains unchanged (provided that $\mbox{\em VR}$ is
infinite).
In practice, this means we can build a hierarchy of provable statements to
more efficiently establish additional provable statements.  This is
what we do in Metamath when we allow proofs to reference previous
\texttt{\$p} statements as well as previous \texttt{\$a} statements.
\end{quotation}}

\section{Examples of Formal Systems}

{\footnotesize\begin{quotation}
{\em Relationship to Metamath.} The examples in this section, except Example~2,
are for the most part exact equivalents of the development in the set
theory database \texttt{set.mm}.  You may want to compare Examples~1, 3, and 5
to Section~\ref{metaaxioms}, Example 4 to Sections~\ref{metadefprop} and
\ref{metadefpred}, and Example 6 to
Section~\ref{setdefinitions}.\label{exampleref}
\end{quotation}}

\subsection{Example~1---Propositional Calculus}\index{propositional calculus}

Classical propositional calculus can be described by the following formal
system.  We assume the set of variables is infinite.  Rather than denoting the
constants and variables by $c_0, c_1, \ldots$ and $v_0, v_1, \ldots$, for
readability we will instead use more conventional symbols, with the
understanding of course that they denote distinct primitive objects.
Also for readability we may omit commas between successive terms of a
sequence; thus $\langle \mbox{wff\ } \varphi\rangle$ denotes
$\langle \mbox{wff}, \varphi\rangle$.

Let
\begin{itemize}
  \item[] $\mbox{\em CN}=\{\mbox{wff}, \vdash, \to, \lnot, (,)\}$
  \item[] $\mbox{\em VR}=\{\varphi,\psi,\chi,\ldots\}$
  \item[] $T = \{\langle \mbox{wff\ } \varphi\rangle,
             \langle \mbox{wff\ } \psi\rangle,
             \langle \mbox{wff\ } \chi\rangle,\ldots\}$, i.e.\ those
             expressions of length 2 whose first member is $\mbox{\rm wff}$
             and whose second member belongs to $\mbox{\em VR}$.\footnote{For
convenience we let $T$ be an infinite set; the definition of a statement
permits this in principle.  Since a Metamath source file has a finite size, in
practice we must of course use appropriate finite subsets of this $T$,
specifically ones containing at least the mandatory variable-type
hypotheses.  Similarly, in the source file we introduce new variables as
required, with the understanding that a potentially infinite number of
them are available.}
\noindent Then $\Gamma$ consists of the axiomatic statements that
are the reducts of the following pre-statements:
    \begin{itemize}
      \item[] $\langle\varnothing,T,\varnothing,
               \langle \mbox{wff\ }(\varphi\to\psi)\rangle\rangle$
      \item[] $\langle\varnothing,T,\varnothing,
               \langle \mbox{wff\ }\lnot\varphi\rangle\rangle$
      \item[] $\langle\varnothing,T,\varnothing,
               \langle \vdash(\varphi\to(\psi\to\varphi))
               \rangle\rangle$
      \item[] $\langle\varnothing,T,
               \varnothing,
               \langle \vdash((\varphi\to(\psi\to\chi))\to
               ((\varphi\to\psi)\to(\varphi\to\chi)))
               \rangle\rangle$
      \item[] $\langle\varnothing,T,
               \varnothing,
               \langle \vdash((\lnot\varphi\to\lnot\psi)\to
               (\psi\to\varphi))\rangle\rangle$
      \item[] $\langle\varnothing,T,
               \{\langle\vdash(\varphi\to\psi)\rangle,
                 \langle\vdash\varphi\rangle\},
               \langle\vdash\psi\rangle\rangle$
    \end{itemize}
\end{itemize}

(For example, the reduct of $\langle\varnothing,T,\varnothing,
               \langle \mbox{wff\ }(\varphi\to\psi)\rangle\rangle$
is
\begin{itemize}
\item[] $\langle\varnothing,
\{\langle \mbox{wff\ } \varphi\rangle,
             \langle \mbox{wff\ } \psi\rangle\},
             \varnothing,
               \langle \mbox{wff\ }(\varphi\to\psi)\rangle\rangle$,
\end{itemize}
which is the first axiomatic statement.)

We call the members of $\mbox{\em VR}$ {\em wff variables} or (in the context
of first-order logic which we will describe shortly) {\em wff metavariables}.
Note that the symbols $\phi$, $\psi$, etc.\ denote actual specific members of
$\mbox{\em VR}$; they are not metavariables of our expository language (which
we denote with $\alpha$, $\beta$, etc.) but are instead (meta)constant symbols
(members of $\mbox{\em SM}$) from the point of view of our expository
language.  The equivalent system of propositional calculus described in
\cite{Tarski1965} also uses the symbols $\phi$, $\psi$, etc.\ to denote wff
metavariables, but in \cite{Tarski1965} unlike here those are metavariables of
the expository language and not primitive symbols of the formal system.

The first two statements define wffs: if $\varphi$ and $\psi$ are wffs, so is
$(\varphi \to \psi)$; if $\varphi$ is a wff, so is $\lnot\varphi$. The next
three are the axioms of propositional calculus: if $\varphi$ and $\psi$ are
wffs, then $\vdash (\varphi \to (\psi \to \varphi))$ is an (axiomatic)
theorem; etc. The
last is the rule of modus ponens: if $\varphi$ and $\psi$ are wffs, and
$\vdash (\varphi\to\psi)$ and $\vdash \varphi$ are theorems, then $\vdash
\psi$ is a theorem.

The correspondence to ordinary propositional calculus is as follows.  We
consider only provable statements of the form $\langle\varnothing,
T,\varnothing,A\rangle$ with $T$ defined as above.  The first term of the
assertion $A$ of any such statement is either ``wff'' or ``$\vdash$''.  A
statement for which the first term is ``wff'' is a {\em wff} of propositional
calculus, and one where the first term is ``$\vdash$'' is a {\em
theorem (scheme)} of propositional calculus.

The universe of this formal system also contains many other provable
statements.  Those with distinct-variable restrictions are irrelevant because
propositional calculus has no constraints on substitutions.  Those that have
logical hypotheses we call {\em inferences}\index{inference} when
the logical hypotheses are of the form
$\langle\vdash\rangle\frown w$ where $w$ is a wff (with the leading constant
term ``wff'' removed).  Inferences (other than the modus ponens rule) are not a
proper part of propositional calculus but are convenient to use when building a
hierarchy of provable statements.  A provable statement with a nonsense
hypothesis such as $\langle \to,\vdash,\lnot\rangle$, and this same expression
as its assertion, we consider irrelevant; no use can be made of it in
proving theorems, since there is no way to eliminate the nonsense hypothesis.

{\footnotesize\begin{quotation}
{\em Comment.} Our use of parentheses in the definition of a wff illustrates
how axiomatic statements should be carefully stated in a way that
ties in unambiguously with the substitutions allowed by the formal system.
There are many ways we could have defined wffs---for example, Polish
prefix notation would have allowed us to omit parentheses entirely, at
the expense of readability---but we must define them in a way that is
unambiguous.  For example, if we had omitted parentheses from the
definition of $(\varphi\to \psi)$, the wff $\lnot\varphi\to \psi$ could
be interpreted as either $\lnot(\varphi\to\psi)$ or $(\lnot\varphi\to\psi)$
and would have allowed us to prove nonsense.  Note that there is no
concept of operator binding precedence built into our formal system.
\end{quotation}}

\begin{sloppy}
\subsection{Example~2---Predicate Calculus with Equality}\index{predicate
calculus}
\end{sloppy}

Here we extend Example~1 to include predicate calculus with equality,
illustrating the use of distinct-variable restrictions.  This system is the
same as Tarski's system $\mathfrak{S}_2$ in \cite{Tarski1965} (except that the
axioms of propositional calculus are different but equivalent, and a redundant
axiom is omitted).  We extend $\mbox{\em CN}$ with the constants
$\{\mbox{var},\forall,=\}$.  We extend $\mbox{\em VR}$ with an infinite set of
{\em individual metavariables}\index{individual
metavariable} $\{x,y,z,\ldots\}$ and denote this subset
$\mbox{\em Vr}$.

We also join to $\mbox{\em CN}$ a possibly infinite set $\mbox{\em Pr}$ of {\em
predicates} $\{R,S,\ldots\}$.  We associate with $\mbox{\em Pr}$ a function
$\mbox{rnk}$ from $\mbox{\em Pr}$ to $\omega$, and for $\alpha\in \mbox{\em
Pr}$ we call $\mbox{rnk}(\alpha)$ the {\em rank} of the predicate $\alpha$,
which is simply the number of ``arguments'' that the predicate has.  (Most
applications of predicate calculus will have a finite number of predicates;
for example, set theory has the single two-argument or binary predicate $\in$,
which is usually written with its arguments surrounding the predicate symbol
rather than with the prefix notation we will use for the general case.)  As a
device to facilitate our discussion, we will let $\mbox{\em Vs}$ be any fixed
one-to-one function from $\omega$ to $\mbox{\em Vr}$; thus $\mbox{\em Vs}$ is
any simple infinite sequence of individual metavariables with no repeating
terms.

In this example we will not include the function symbols that are often part of
formalizations of predicate calculus.  Using metalogical arguments that are
beyond the scope of our discussion, it can be shown that our formalization is
equivalent when functions are introduced via appropriate definitions.

We extend the set $T$ defined in Example~1 with the expressions
$\{\langle \mbox{var\ } x\rangle,$ $ \langle \mbox{var\ } y\rangle, \langle
\mbox{var\ } z\rangle,\ldots\}$.  We extend the $\Gamma$ above
with the axiomatic statements that are the reducts of the following
pre-statements:
\begin{list}{}{\itemsep 0.0pt}
      \item[] $\langle\varnothing,T,\varnothing,
               \langle \mbox{wff\ }\forall x\,\varphi\rangle\rangle$
      \item[] $\langle\varnothing,T,\varnothing,
               \langle \mbox{wff\ }x=y\rangle\rangle$
      \item[] $\langle\varnothing,T,
               \{\langle\vdash\varphi\rangle\},
               \langle\vdash\forall x\,\varphi\rangle\rangle$
      \item[] $\langle\varnothing,T,\varnothing,
               \langle \vdash((\forall x(\varphi\to\psi)
                  \to(\forall x\,\varphi\to\forall x\,\psi))
               \rangle\rangle$
      \item[] $\langle\{\{x,\varphi\}\},T,\varnothing,
               \langle \vdash(\varphi\to\forall x\,\varphi)
               \rangle\rangle$
      \item[] $\langle\{\{x,y\}\},T,\varnothing,
               \langle \vdash\lnot\forall x\lnot x=y
               \rangle\rangle$
      \item[] $\langle\varnothing,T,\varnothing,
               \langle \vdash(x=z
                  \to(x=y\to z=y))
               \rangle\rangle$
      \item[] $\langle\varnothing,T,\varnothing,
               \langle \vdash(y=z
                  \to(x=y\to x=z))
               \rangle\rangle$
\end{list}
These are the axioms not involving predicate symbols. The first two statements
extend the definition of a wff.  The third is the rule of generalization.  The
fifth states, in effect, ``For a wff $\varphi$ and variable $x$,
$\vdash(\varphi\to\forall x\,\varphi)$, provided that $x$ does not occur in
$\varphi$.''  The sixth states ``For variables $x$ and $y$,
$\vdash\lnot\forall x\lnot x = y$, provided that $x$ and $y$ are distinct.''
(This proviso is not necessary but was included by Tarski to
weaken the axiom and still show that the system is logically complete.)

Finally, for each predicate symbol $\alpha\in \mbox{\em Pr}$, we add to
$\Gamma$ an axiomatic statement, extending the definition of wff,
that is the reduct of the following pre-statement:
\begin{displaymath}
    \langle\varnothing,T,\varnothing,
            \langle \mbox{wff},\alpha\rangle\
            \frown \mbox{\em Vs}\restriction\mbox{rnk}(\alpha)\rangle
\end{displaymath}
and for each $\alpha\in \mbox{\em Pr}$ and each $n < \mbox{rnk}(\alpha)$
we add to $\Gamma$ an equality axiom that is the reduct of the
following pre-statement:
\begin{eqnarray*}
    \lefteqn{\langle\varnothing,T,\varnothing,
            \langle
      \vdash,(,\mbox{\em Vs}_n,=,\mbox{\em Vs}_{\mbox{rnk}(\alpha)},\to,
            (,\alpha\rangle\frown \mbox{\em Vs}\restriction\mbox{rnk}(\alpha)} \\
  & & \frown
            \langle\to,\alpha\rangle\frown \mbox{\em Vs}\restriction n\frown
            \langle \mbox{\em Vs}_{\mbox{rnk}(\alpha)}\rangle \\
 & & \frown
            \mbox{\em Vs}\restriction(\mbox{rnk}(\alpha)\setminus(n+1))\frown
            \langle),)\rangle\rangle
\end{eqnarray*}
where $\restriction$ denotes function domain restriction and $\setminus$
denotes set difference.  Recall that a subscript on $\mbox{\em Vs}$
denotes one of its terms.  (In the above two axiom sets commas are placed
between successive terms of sequences to prevent ambiguity, and if you examine
them with care you will be able to distinguish those parentheses that denote
constant symbols from those of our expository language that delimit function
arguments.  Although it might have been better to use boldface for our
primitive symbols, unfortunately boldface was not available for all characters
on the \LaTeX\ system used to typeset this text.)  These seemingly forbidding
axioms can be understood by analogy to concatenation of substrings in a
computer language.  They are actually relatively simple for each specific case
and will become clearer by looking at the special case of a binary predicate
$\alpha = R$ where $\mbox{rnk}(R)=2$.  Letting $\mbox{\em Vs}$ be the sequence
$\langle x,y,z,\ldots\rangle$, the axioms we would add to $\Gamma$ for this
case would be the wff extension and two equality axioms that are the
reducts of the pre-statements:
\begin{list}{}{\itemsep 0.0pt}
      \item[] $\langle\varnothing,T,\varnothing,
               \langle \mbox{wff\ }R x y\rangle\rangle$
      \item[] $\langle\varnothing,T,\varnothing,
               \langle \vdash(x=z
                  \to(R x y \to R z y))
               \rangle\rangle$
      \item[] $\langle\varnothing,T,\varnothing,
               \langle \vdash(y=z
                  \to(R x y \to R x z))
               \rangle\rangle$
\end{list}
Study these carefully to see how the general axioms above evaluate to
them.  In practice, typically only a few special cases such as this would be
needed, and in any case the Metamath language will only permit us to describe
a finite number of predicates, as opposed to the infinite number permitted by
the formal system.  (If an infinite number should be needed for some reason,
we could not define the formal system directly in the Metamath language but
could instead define it metalogically under set theory as we
do in this appendix, and only the underlying set theory, with its single
binary predicate, would be defined directly in the Metamath language.)


{\footnotesize\begin{quotation}
{\em Comment.}  As we noted earlier, the specific variables denoted by the
symbols $x,y,z,\ldots\in \mbox{\em Vr}\subseteq \mbox{\em VR}\subseteq
\mbox{\em SM}$ in Example~2 are not the actual variables of ordinary predicate
calculus but should be thought of as metavariables ranging over them.  For
example, a distinct-variable restriction would be meaningless for actual
variables of ordinary predicate calculus since two different actual variables
are by definition distinct.  And when we talk about an arbitrary
representative $\alpha\in \mbox{\em Vr}$, $\alpha$ is a metavariable (in our
expository language) that ranges over metavariables (which are primitives of
our formal system) each of which ranges over the actual individual variables
of predicate calculus (which are never mentioned in our formal system).

The constant called ``var'' above is called \texttt{setvar} in the
\texttt{set.mm} database file, but it means the same thing.  I felt
that ``var'' is a more meaningful name in the context of predicate
calculus, whose use is not limited to set theory.  For consistency we
stick with the name ``var'' throughout this Appendix, even after set
theory is introduced.
\end{quotation}}

\subsection{Free Variables and Proper Substitution}\index{free variable}
\index{proper substitution}\index{substitution!proper}

Typical representations of mathematical axioms use concepts such
as ``free variable,'' ``bound variable,'' and ``proper substitution''
as primitive notions.
A free variable is a variable that
is not a parameter of any container expression.
A bound variable is the opposite of a free variable; it is a
a variable that has been bound in a container expression.
For example, in the expression $\forall x \varphi$ (for all $x$, $\varphi$
is true), the variable $x$
is bound within the for-all ($\forall$) expression.
It is possible to change one variable to another, and that process is called
``proper substitution.''
In most books, proper substitution has a somewhat complicated recursive
definition with multiple cases based on the occurrences of free and
bound variables.
You may consult
\cite[ch.\ 3--4]{Hamilton}\index{Hamilton, Alan G.} (as well as
many other texts) for more formal details about these terms.

Using these concepts as \texttt{primitives} creates complications
for computer implementations.

In the system of Example~2, there are no primitive notions of free variable
and proper substitution.  Tarski \cite{Tarski1965} shows that this system is
logically equivalent to the more typical textbook systems that do have these
primitive notions, if we introduce these notions with appropriate definitions
and metalogic.  We could also define axioms for such systems directly,
although the recursive definitions of free variable and proper substitution
would be messy and awkward to work with.  Instead, we mention two devices that
can be used in practice to mimic these notions.  (1) Instead of introducing
special notation to express (as a logical hypothesis) ``where $x$ is not free
in $\varphi$'' we can use the logical hypothesis $\vdash(\varphi\to\forall
x\,\varphi)$.\label{effectivelybound}\index{effectively
not free}\footnote{This is a slightly weaker requirement than ``where $x$ is
not free in $\varphi$.''  If we let $\varphi$ be $x=x$, we have the theorem
$(x=x\to\forall x\,x=x)$ which satisfies the hypothesis, even though $x$ is
free in $x=x$ .  In a case like this we say that $x$ is {\em effectively not
free}\index{effectively not free} in $x=x$, since $x=x$ is logically
equivalent to $\forall x\,x=x$ in which $x$ is bound.} (2) It can be shown
that the wff $((x=y\to\varphi)\wedge\exists x(x=y\wedge\varphi))$ (with the
usual definitions of $\wedge$ and $\exists$; see Example~4 below) is logically
equivalent to ``the wff that results from proper substitution of $y$ for $x$
in $\varphi$.''  This works whether or not $x$ and $y$ are distinct.

\subsection{Metalogical Completeness}\index{metalogical completeness}

In the system of Example~2, the
following are provable pre-statements (and their reducts are
provable statements):
\begin{eqnarray*}
      & \langle\{\{x,y\}\},T,\varnothing,
               \langle \vdash\lnot\forall x\lnot x=y
               \rangle\rangle & \\
     &  \langle\varnothing,T,\varnothing,
               \langle \vdash\lnot\forall x\lnot x=x
               \rangle\rangle &
\end{eqnarray*}
whereas the following pre-statement is not to my knowledge provable (but
in any case we will pretend it's not for sake of illustration):
\begin{eqnarray*}
     &  \langle\varnothing,T,\varnothing,
               \langle \vdash\lnot\forall x\lnot x=y
               \rangle\rangle &
\end{eqnarray*}
In other words, we can prove ``$\lnot\forall x\lnot x=y$ where $x$ and $y$ are
distinct'' and separately prove ``$\lnot\forall x\lnot x=x$'', but we can't
prove the combined general case ``$\lnot\forall x\lnot x=y$'' that has no
proviso.  Now this does not compromise logical completeness, because the
variables are really metavariables and the two provable cases together cover
all possible cases.  The third case can be considered a metatheorem whose
direct proof, using the system of Example~2, lies outside the capability of the
formal system.

Also, in the system of Example~2 the following pre-statement is not to my
knowledge provable (again, a conjecture that we will pretend to be the case):
\begin{eqnarray*}
     & \langle\varnothing,T,\varnothing,
               \langle \vdash(\forall x\, \varphi\to\varphi)
               \rangle\rangle &
\end{eqnarray*}
Instead, we can only prove specific cases of $\varphi$ involving individual
metavariables, and by induction on formula length, prove as a metatheorem
outside of our formal system the general statement above.  The details of this
proof are found in \cite{Kalish}.

There does, however, exist a system of predicate calculus in which all such
``simple metatheorems'' as those above can be proved directly, and we present
it in Example~3. A {\em simple metatheorem}\index{simple metatheorem}
is any statement of the formal
system of Example~2 where all distinct variable restrictions consist of either
two individual metavariables or an individual metavariable and a wff
metavariable, and which is provable by combining cases outside the system as
above.  A system is {\em metalogically complete}\index{metalogical
completeness} if all of its simple
metatheorems are (directly) provable statements. The precise definition of
``simple metatheorem'' and the proof of the ``metalogical completeness'' of
Example~3 is found in Remark 9.6 and Theorem 9.7 of \cite{Megill}.\index{Megill,
Norman}

\begin{sloppy}
\subsection{Example~3---Metalogically Complete Predicate
Calculus with
Equality}
\end{sloppy}

For simplicity we will assume there is one binary predicate $R$;
this system suffices for set theory, where the $R$ is of course the $\in$
predicate.  We label the axioms as they appear in \cite{Megill}.  This
system is logically equivalent to that of Example~2 (when the latter is
restricted to this single binary predicate) but is also metalogically
complete.\index{metalogical completeness}

Let
\begin{itemize}
  \item[] $\mbox{\em CN}=\{\mbox{wff}, \mbox{var}, \vdash, \to, \lnot, (,),\forall,=,R\}$.
  \item[] $\mbox{\em VR}=\{\varphi,\psi,\chi,\ldots\}\cup\{x,y,z,\ldots\}$.
  \item[] $T = \{\langle \mbox{wff\ } \varphi\rangle,
             \langle \mbox{wff\ } \psi\rangle,
             \langle \mbox{wff\ } \chi\rangle,\ldots\}\cup
       \{\langle \mbox{var\ } x\rangle, \langle \mbox{var\ } y\rangle, \langle
       \mbox{var\ }z\rangle,\ldots\}$.

\noindent Then
  $\Gamma$ consists of the reducts of the following pre-statements:
    \begin{itemize}
      \item[] $\langle\varnothing,T,\varnothing,
               \langle \mbox{wff\ }(\varphi\to\psi)\rangle\rangle$
      \item[] $\langle\varnothing,T,\varnothing,
               \langle \mbox{wff\ }\lnot\varphi\rangle\rangle$
      \item[] $\langle\varnothing,T,\varnothing,
               \langle \mbox{wff\ }\forall x\,\varphi\rangle\rangle$
      \item[] $\langle\varnothing,T,\varnothing,
               \langle \mbox{wff\ }x=y\rangle\rangle$
      \item[] $\langle\varnothing,T,\varnothing,
               \langle \mbox{wff\ }Rxy\rangle\rangle$
      \item[(C1$'$)] $\langle\varnothing,T,\varnothing,
               \langle \vdash(\varphi\to(\psi\to\varphi))
               \rangle\rangle$
      \item[(C2$'$)] $\langle\varnothing,T,
               \varnothing,
               \langle \vdash((\varphi\to(\psi\to\chi))\to
               ((\varphi\to\psi)\to(\varphi\to\chi)))
               \rangle\rangle$
      \item[(C3$'$)] $\langle\varnothing,T,
               \varnothing,
               \langle \vdash((\lnot\varphi\to\lnot\psi)\to
               (\psi\to\varphi))\rangle\rangle$
      \item[(C4$'$)] $\langle\varnothing,T,
               \varnothing,
               \langle \vdash(\forall x(\forall x\,\varphi\to\psi)\to
                 (\forall x\,\varphi\to\forall x\,\psi))\rangle\rangle$
      \item[(C5$'$)] $\langle\varnothing,T,
               \varnothing,
               \langle \vdash(\forall x\,\varphi\to\varphi)\rangle\rangle$
      \item[(C6$'$)] $\langle\varnothing,T,
               \varnothing,
               \langle \vdash(\forall x\forall y\,\varphi\to
                 \forall y\forall x\,\varphi)\rangle\rangle$
      \item[(C7$'$)] $\langle\varnothing,T,
               \varnothing,
               \langle \vdash(\lnot\varphi\to\forall x\lnot\forall x\,\varphi
                 )\rangle\rangle$
      \item[(C8$'$)] $\langle\varnothing,T,
               \varnothing,
               \langle \vdash(x=y\to(x=z\to y=z))\rangle\rangle$
      \item[(C9$'$)] $\langle\varnothing,T,
               \varnothing,
               \langle \vdash(\lnot\forall x\, x=y\to(\lnot\forall x\, x=z\to
                 (y=z\to\forall x\, y=z)))\rangle\rangle$
      \item[(C10$'$)] $\langle\varnothing,T,
               \varnothing,
               \langle \vdash(\forall x(x=y\to\forall x\,\varphi)\to
                 \varphi))\rangle\rangle$
      \item[(C11$'$)] $\langle\varnothing,T,
               \varnothing,
               \langle \vdash(\forall x\, x=y\to(\forall x\,\varphi
               \to\forall y\,\varphi))\rangle\rangle$
      \item[(C12$'$)] $\langle\varnothing,T,
               \varnothing,
               \langle \vdash(x=y\to(Rxz\to Ryz))\rangle\rangle$
      \item[(C13$'$)] $\langle\varnothing,T,
               \varnothing,
               \langle \vdash(x=y\to(Rzx\to Rzy))\rangle\rangle$
      \item[(C15$'$)] $\langle\varnothing,T,
               \varnothing,
               \langle \vdash(\lnot\forall x\, x=y\to(x=y\to(\varphi
                 \to\forall x(x=y\to\varphi))))\rangle\rangle$
      \item[(C16$'$)] $\langle\{\{x,y\}\},T,
               \varnothing,
               \langle \vdash(\forall x\, x=y\to(\varphi\to\forall x\,\varphi)
                 )\rangle\rangle$
      \item[(C5)] $\langle\{\{x,\varphi\}\},T,\varnothing,
               \langle \vdash(\varphi\to\forall x\,\varphi)
               \rangle\rangle$
      \item[(MP)] $\langle\varnothing,T,
               \{\langle\vdash(\varphi\to\psi)\rangle,
                 \langle\vdash\varphi\rangle\},
               \langle\vdash\psi\rangle\rangle$
      \item[(Gen)] $\langle\varnothing,T,
               \{\langle\vdash\varphi\rangle\},
               \langle\vdash\forall x\,\varphi\rangle\rangle$
    \end{itemize}
\end{itemize}

While it is known that these axioms are ``metalogically complete,'' it is
not known whether they are independent (i.e.\ none is
redundant) in the metalogical sense; specifically, whether any axiom (possibly
with additional non-mandatory distinct-variable restrictions, for use with any
dummy variables in its proof) is provable from the others.  Note that
metalogical independence is a weaker requirement than independence in the
usual logical sense.  Not all of the above axioms are logically independent:
for example, C9$'$ can be proved as a metatheorem from the others, outside the
formal system, by combining the possible cases of distinct variables.

\subsection{Example~4---Adding Definitions}\index{definition}
There are several ways to add definitions to a formal system.  Probably the
most proper way is to consider definitions not as part of the formal system at
all but rather as abbreviations that are part of the expository metalogic
outside the formal system.  For convenience, though, we may use the formal
system itself to incorporate definitions, adding them as axiomatic extensions
to the system.  This could be done by adding a constant representing the
concept ``is defined as'' along with axioms for it. But there is a nicer way,
at least in this writer's opinion, that introduces definitions as direct
extensions to the language rather than as extralogical primitive notions.  We
introduce additional logical connectives and provide axioms for them.  For
systems of logic such as Examples 1 through 3, the additional axioms must be
conservative in the sense that no wff of the original system that was not a
theorem (when the initial term ``wff'' is replaced by ``$\vdash$'' of course)
becomes a theorem of the extended system.  In this example we extend Example~3
(or 2) with standard abbreviations of logic.

We extend $\mbox{\em CN}$ of Example~3 with new constants $\{\leftrightarrow,
\wedge,\vee,\exists\}$, corresponding to logical equivalence,\index{logical
equivalence ($\leftrightarrow$)}\index{biconditional ($\leftrightarrow$)}
conjunction,\index{conjunction ($\wedge$)} disjunction,\index{disjunction
($\vee$)} and the existential quantifier.\index{existential quantifier
($\exists$)}  We extend $\Gamma$ with the axiomatic statements that are
the reducts of the following pre-statements:
\begin{list}{}{\itemsep 0.0pt}
      \item[] $\langle\varnothing,T,\varnothing,
               \langle \mbox{wff\ }(\varphi\leftrightarrow\psi)\rangle\rangle$
      \item[] $\langle\varnothing,T,\varnothing,
               \langle \mbox{wff\ }(\varphi\vee\psi)\rangle\rangle$
      \item[] $\langle\varnothing,T,\varnothing,
               \langle \mbox{wff\ }(\varphi\wedge\psi)\rangle\rangle$
      \item[] $\langle\varnothing,T,\varnothing,
               \langle \mbox{wff\ }\exists x\, \varphi\rangle\rangle$
  \item[] $\langle\varnothing,T,\varnothing,
     \langle\vdash ( ( \varphi \leftrightarrow \psi ) \to
     ( \varphi \to \psi ) )\rangle\rangle$
  \item[] $\langle\varnothing,T,\varnothing,
     \langle\vdash ((\varphi\leftrightarrow\psi)\to
    (\psi\to\varphi))\rangle\rangle$
  \item[] $\langle\varnothing,T,\varnothing,
     \langle\vdash ((\varphi\to\psi)\to(
     (\psi\to\varphi)\to(\varphi
     \leftrightarrow\psi)))\rangle\rangle$
  \item[] $\langle\varnothing,T,\varnothing,
     \langle\vdash (( \varphi \wedge \psi ) \leftrightarrow\neg ( \varphi
     \to \neg \psi )) \rangle\rangle$
  \item[] $\langle\varnothing,T,\varnothing,
     \langle\vdash (( \varphi \vee \psi ) \leftrightarrow (\neg \varphi
     \to \psi )) \rangle\rangle$
  \item[] $\langle\varnothing,T,\varnothing,
     \langle\vdash (\exists x \,\varphi\leftrightarrow
     \lnot \forall x \lnot \varphi)\rangle\rangle$
\end{list}
The first three logical axioms (statements containing ``$\vdash$'') introduce
and effectively define logical equivalence, ``$\leftrightarrow$''.  The last
three use ``$\leftrightarrow$'' to effectively mean ``is defined as.''

\subsection{Example~5---ZFC Set Theory}\index{ZFC set theory}

Here we add to the system of Example~4 the axioms of Zermelo--Fraenkel set
theory with Choice.  For convenience we make use of the
definitions in Example~4.

In the $\mbox{\em CN}$ of Example~4 (which extends Example~3), we replace the symbol $R$
with the symbol $\in$.
More explicitly, we remove from $\Gamma$ of Example~4 the three
axiomatic statements containing $R$ and replace them with the
reducts of the following:
\begin{list}{}{\itemsep 0.0pt}
      \item[] $\langle\varnothing,T,\varnothing,
               \langle \mbox{wff\ }x\in y\rangle\rangle$
      \item[] $\langle\varnothing,T,
               \varnothing,
               \langle \vdash(x=y\to(x\in z\to y\in z))\rangle\rangle$
      \item[] $\langle\varnothing,T,
               \varnothing,
               \langle \vdash(x=y\to(z\in x\to z\in y))\rangle\rangle$
\end{list}
Letting $D=\{\{\alpha,\beta\}\in \mbox{\em DV}\,|\alpha,\beta\in \mbox{\em
Vr}\}$ (in other words all individual variables must be distinct), we extend
$\Gamma$ with the ZFC axioms, called
\index{Axiom of Extensionality}
\index{Axiom of Replacement}
\index{Axiom of Union}
\index{Axiom of Power Sets}
\index{Axiom of Regularity}
\index{Axiom of Infinity}
\index{Axiom of Choice}
Extensionality, Replacement, Union, Power
Set, Regularity, Infinity, and Choice, that are the reducts of:
\begin{list}{}{\itemsep 0.0pt}
      \item[Ext] $\langle D,T,
               \varnothing,
               \langle\vdash (\forall x(x\in y\leftrightarrow x \in z)\to y
               =z) \rangle\rangle$
      \item[Rep] $\langle D,T,
               \varnothing,
               \langle\vdash\exists x ( \exists y \forall z (\varphi \to z = y
                        ) \to
                        \forall z ( z \in x \leftrightarrow \exists x ( x \in
                        y \wedge \forall y\,\varphi ) ) )\rangle\rangle$
      \item[Un] $\langle D,T,
               \varnothing,
               \langle\vdash \exists x \forall y ( \exists x ( y \in x \wedge
               x \in z ) \to y \in x ) \rangle\rangle$
      \item[Pow] $\langle D,T,
               \varnothing,
               \langle\vdash \exists x \forall y ( \forall x ( x \in y \to x
               \in z ) \to y \in x ) \rangle\rangle$
      \item[Reg] $\langle D,T,
               \varnothing,
               \langle\vdash (  x \in y \to
                 \exists x ( x \in y \wedge \forall z ( z \in x \to \lnot z
                \in y ) ) ) \rangle\rangle$
      \item[Inf] $\langle D,T,
               \varnothing,
               \langle\vdash \exists x(y\in x\wedge\forall y(y\in
               x\to
               \exists z(y \in z\wedge z\in x))) \rangle\rangle$
      \item[AC] $\langle D,T,
               \varnothing,
               \langle\vdash \exists x \forall y \forall z ( ( y \in z
               \wedge z \in w ) \to \exists w \forall y ( \exists w
              ( ( y \in z \wedge z \in w ) \wedge ( y \in w \wedge w \in x
              ) ) \leftrightarrow y = w ) ) \rangle\rangle$
\end{list}

\subsection{Example~6---Class Notation in Set Theory}\label{class}

A powerful device that makes set theory easier (and that we have
been using all along in our informal expository language) is {\em class
abstraction notation}.\index{class abstraction}\index{abstraction class}  The
definitions we introduce are rigorously justified
as conservative by Takeuti and Zaring \cite{Takeuti}\index{Takeuti, Gaisi} or
Quine \cite{Quine}\index{Quine, Willard Van Orman}.  The key idea is to
introduce the notation $\{x|\mbox{---}\}$ which means ``the class of all $x$
such that ---'' for abstraction classes and introduce (meta)variables that
range over them.  An abstraction class may or may not be a set, depending on
whether it exists (as a set).  A class that does not exist is
called a {\em proper class}.\index{proper class}\index{class!proper}

To illustrate the use of abstraction classes we will provide some examples
of definitions that make use of them:  the empty set, class union, and
unordered pair.  Many other such definitions can be found in the
Metamath set theory database,
\texttt{set.mm}.\index{set theory database (\texttt{set.mm})}

% We intentionally break up the sequence of math symbols here
% because otherwise the overlong line goes beyond the page in narrow mode.
We extend $\mbox{\em CN}$ of Example~5 with new symbols $\{$
$\mbox{class},$ $\{,$ $|,$ $\},$ $\varnothing,$ $\cup,$ $,$ $\}$
where the inner braces and last comma are
constant symbols. (As before,
our dual use of some mathematical symbols for both our expository
language and as primitives of the formal system should be clear from context.)

We extend $\mbox{\em VR}$ of Example~5 with a set of {\em class
variables}\index{class variable}
$\{A,B,C,\ldots\}$. We extend the $T$ of Example~5 with $\{\langle
\mbox{class\ } A\rangle, \langle \mbox{class\ }B\rangle, \langle \mbox{class\ }
C\rangle,\ldots\}$.

To
introduce our definitions,
we add to $\Gamma$ of Example~5 the axiomatic statements
that are the reducts of the following pre-statements:
\begin{list}{}{\itemsep 0.0pt}
      \item[] $\langle\varnothing,T,\varnothing,
               \langle \mbox{class\ }x\rangle\rangle$
      \item[] $\langle\varnothing,T,\varnothing,
               \langle \mbox{class\ }\{x|\varphi\}\rangle\rangle$
      \item[] $\langle\varnothing,T,\varnothing,
               \langle \mbox{wff\ }A=B\rangle\rangle$
      \item[] $\langle\varnothing,T,\varnothing,
               \langle \mbox{wff\ }A\in B\rangle\rangle$
      \item[Ab] $\langle\varnothing,T,\varnothing,
               \langle \vdash ( y \in \{ x |\varphi\} \leftrightarrow
                  ( ( x = y \to\varphi) \wedge \exists x ( x = y
                  \wedge\varphi) ))
               \rangle\rangle$
      \item[Eq] $\langle\{\{x,A\},\{x,B\}\},T,\varnothing,
               \langle \vdash ( A = B \leftrightarrow
               \forall x ( x \in A \leftrightarrow x \in B ) )
               \rangle\rangle$
      \item[El] $\langle\{\{x,A\},\{x,B\}\},T,\varnothing,
               \langle \vdash ( A \in B \leftrightarrow \exists x
               ( x = A \wedge x \in B ) )
               \rangle\rangle$
\end{list}
Here we say that an individual variable is a class; $\{x|\varphi\}$ is a
class; and we extend the definition of a wff to include class equality and
membership.  Axiom Ab defines membership of a variable in a class abstraction;
the right-hand side can be read as ``the wff that results from proper
substitution of $y$ for $x$ in $\varphi$.''\footnote{Note that this definition
makes unnecessary the introduction of a separate notation similar to
$\varphi(x|y)$ for proper substitution, although we may choose to do so to be
conventional.  Incidentally, $\varphi(x|y)$ as it stands would be ambiguous in
the formal systems of our examples, since we wouldn't know whether
$\lnot\varphi(x|y)$ meant $\lnot(\varphi(x|y))$ or $(\lnot\varphi)(x|y)$.
Instead, we would have to use an unambiguous variant such as $(\varphi\,
x|y)$.}  Axioms Eq and El extend the meaning of the existing equality and
membership connectives.  This is potentially dangerous and requires careful
justification.  For example, from Eq we can derive the Axiom of Extensionality
with predicate logic alone; thus in principle we should include the Axiom of
Extensionality as a logical hypothesis.  However we do not bother to do this
since we have already presupposed that axiom earlier. The distinct variable
restrictions should be read ``where $x$ does not occur in $A$ or $B$.''  We
typically do this when the right-hand side of a definition involves an
individual variable not in the expression being defined; it is done so that
the right-hand side remains independent of the particular ``dummy'' variable
we use.

We continue to add to $\Gamma$ the following definitions
(i.e. the reducts of the following pre-statements) for empty
set,\index{empty set} class union,\index{union} and unordered
pair.\index{unordered pair}  They should be self-explanatory.  Analogous to our
use of ``$\leftrightarrow$'' to define new wffs in Example~4, we use ``$=$''
to define new abstraction terms, and both may be read informally as ``is
defined as'' in this context.
\begin{list}{}{\itemsep 0.0pt}
      \item[] $\langle\varnothing,T,\varnothing,
               \langle \mbox{class\ }\varnothing\rangle\rangle$
      \item[] $\langle\varnothing,T,\varnothing,
               \langle \vdash \varnothing = \{ x | \lnot x = x \}
               \rangle\rangle$
      \item[] $\langle\varnothing,T,\varnothing,
               \langle \mbox{class\ }(A\cup B)\rangle\rangle$
      \item[] $\langle\{\{x,A\},\{x,B\}\},T,\varnothing,
               \langle \vdash ( A \cup B ) = \{ x | ( x \in A \vee x \in B ) \}
               \rangle\rangle$
      \item[] $\langle\varnothing,T,\varnothing,
               \langle \mbox{class\ }\{A,B\}\rangle\rangle$
      \item[] $\langle\{\{x,A\},\{x,B\}\},T,\varnothing,
               \langle \vdash \{ A , B \} = \{ x | ( x = A \vee x = B ) \}
               \rangle\rangle$
\end{list}

\section{Metamath as a Formal System}\label{theorymm}

This section presupposes a familiarity with the Metamath computer language.

Our theory describes formal systems and their universes.  The Metamath
language provides a way of representing these set-theoretical objects to
a computer.  A Metamath database, being a finite set of {\sc ascii}
characters, can usually describe only a subset of a formal system and
its universe, which are typically infinite.  However the database can
contain as large a finite subset of the formal system and its universe
as we wish.  (Of course a Metamath set theory database can, in
principle, indirectly describe an entire infinite formal system by
formalizing the expository language in this Appendix.)

For purpose of our discussion, we assume the Metamath database
is in the simple form described on p.~\pageref{framelist},
consisting of all constant and variable declarations at the beginning,
followed by a sequence of extended frames each
delimited by \texttt{\$\char`\{} and \texttt{\$\char`\}}.  Any Metamath database can
be converted to this form, as described on p.~\pageref{frameconvert}.

The math symbol tokens of a Metamath source file, which are declared
with \texttt{\$c} and \texttt{\$v} statements, are names we assign to
representatives of $\mbox{\em CN}$ and $\mbox{\em VR}$.  For
definiteness we could assume that the first math symbol declared as a
variable corresponds to $v_0$, the second to $v_1$, etc., although the
exact correspondence we choose is not important.

In the Metamath language, each \texttt{\$d}, \texttt{\$f}, and
 \texttt{\$e} source
statement in an extended frame (Section~\ref{frames})
corresponds respectively to a member of the
collections $D$, $T$, and $H$ in a formal system statement $\langle
D_M,T_M,H,A\rangle$.  The math symbol strings following these Metamath keywords
correspond to a variable pair (in the case of \texttt{\$d}) or an expression (for
the other two keywords). The math symbol string following a \texttt{\$a} source
statement corresponds to expression $A$ in an axiomatic statement of the
formal system; the one following a \texttt{\$p} source statement corresponds to
$A$ in a provable statement that is not axiomatic.  In other words, each
extended frame in a Metamath database corresponds to
a pre-statement of the formal system, and a frame corresponds to
a statement of the formal system.  (Don't confuse the two meanings of
``statement'' here.  A statement of the formal system corresponds to the
several statements in a Metamath database that may constitute a
frame.)

In order for the computer to verify that a formal system statement is
provable, each \texttt{\$p} source statement is accompanied by a proof.
However, the proof does not correspond to anything in the formal system
but is simply a way of communicating to the computer the information
needed for its verification.  The proof tells the computer {\em how to
construct} specific members of closure of the formal system
pre-statement corresponding to the extended frame of the \texttt{\$p}
statement.  The final result of the construction is the member of the
closure that matches the \texttt{\$p} statement.  The abstract formal
system, on the other hand, is concerned only with the {\em existence} of
members of the closure.

As mentioned on p.~\pageref{exampleref}, Examples 1 and 3--6 in the
previous Section parallel the development of logic and set theory in the
Metamath database
\texttt{set.mm}.\index{set theory database (\texttt{set.mm})} You may
find it instructive to compare them.


\chapter{The MIU System}
\label{MIU}
\index{formal system}
\index{MIU-system}

The following is a listing of the file \texttt{miu.mm}.  It is self-explanatory.

%%%%%%%%%%%%%%%%%%%%%%%%%%%%%%%%%%%%%%%%%%%%%%%%%%%%%%%%%%%%

\begin{verbatim}
$( The MIU-system:  A simple formal system $)

$( Note:  This formal system is unusual in that it allows
empty wffs.  To work with a proof, you must type
SET EMPTY_SUBSTITUTION ON before using the PROVE command.
By default, this is OFF in order to reduce the number of
ambiguous unification possibilities that have to be selected
during the construction of a proof.  $)

$(
Hofstadter's MIU-system is a simple example of a formal
system that illustrates some concepts of Metamath.  See
Douglas R. Hofstadter, _Goedel, Escher, Bach:  An Eternal
Golden Braid_ (Vintage Books, New York, 1979), pp. 33ff. for
a description of the MIU-system.

The system has 3 constant symbols, M, I, and U.  The sole
axiom of the system is MI. There are 4 rules:
     Rule I:  If you possess a string whose last letter is I,
     you can add on a U at the end.
     Rule II:  Suppose you have Mx.  Then you may add Mxx to
     your collection.
     Rule III:  If III occurs in one of the strings in your
     collection, you may make a new string with U in place
     of III.
     Rule IV:  If UU occurs inside one of your strings, you
     can drop it.
Unfortunately, Rules III and IV do not have unique results:
strings could have more than one occurrence of III or UU.
This requires that we introduce the concept of an "MIU
well-formed formula" or wff, which allows us to construct
unique symbol sequences to which Rules III and IV can be
applied.
$)

$( First, we declare the constant symbols of the language.
Note that we need two symbols to distinguish the assertion
that a sequence is a wff from the assertion that it is a
theorem; we have arbitrarily chosen "wff" and "|-". $)
      $c M I U |- wff $. $( Declare constants $)

$( Next, we declare some variables. $)
     $v x y $.

$( Throughout our theory, we shall assume that these
variables represent wffs. $)
 wx   $f wff x $.
 wy   $f wff y $.

$( Define MIU-wffs.  We allow the empty sequence to be a
wff. $)

$( The empty sequence is a wff. $)
 we   $a wff $.
$( "M" after any wff is a wff. $)
 wM   $a wff x M $.
$( "I" after any wff is a wff. $)
 wI   $a wff x I $.
$( "U" after any wff is a wff. $)
 wU   $a wff x U $.

$( Assert the axiom. $)
 ax   $a |- M I $.

$( Assert the rules. $)
 ${
   Ia   $e |- x I $.
$( Given any theorem ending with "I", it remains a theorem
if "U" is added after it.  (We distinguish the label I_
from the math symbol I to conform to the 24-Jun-2006
Metamath spec.) $)
   I_    $a |- x I U $.
 $}
 ${
IIa  $e |- M x $.
$( Given any theorem starting with "M", it remains a theorem
if the part after the "M" is added again after it. $)
   II   $a |- M x x $.
 $}
 ${
   IIIa $e |- x I I I y $.
$( Given any theorem with "III" in the middle, it remains a
theorem if the "III" is replaced with "U". $)
   III  $a |- x U y $.
 $}
 ${
   IVa  $e |- x U U y $.
$( Given any theorem with "UU" in the middle, it remains a
theorem if the "UU" is deleted. $)
   IV   $a |- x y $.
  $}

$( Now we prove the theorem MUIIU.  You may be interested in
comparing this proof with that of Hofstadter (pp. 35 - 36).
$)
 theorem1  $p |- M U I I U $=
      we wM wU wI we wI wU we wU wI wU we wM we wI wU we wM
      wI wI wI we wI wI we wI ax II II I_ III II IV $.
\end{verbatim}\index{well-formed formula (wff)}

The \texttt{show proof /lemmon/renumber} command
yields the following display.  It is very similar
to the one in \cite[pp.~35--36]{Hofstadter}.\index{Hofstadter, Douglas R.}

\begin{verbatim}
1 ax             $a |- M I
2 1 II           $a |- M I I
3 2 II           $a |- M I I I I
4 3 I_           $a |- M I I I I U
5 4 III          $a |- M U I U
6 5 II           $a |- M U I U U I U
7 6 IV           $a |- M U I I U
\end{verbatim}

We note that Hofstadter's ``MU-puzzle,'' which asks whether
MU is a theorem of the MIU-system, cannot be answered using
the system above because the MU-puzzle is a question {\em
about} the system.  To prove the answer to the MU-puzzle,
a much more elaborate system is needed, namely one that
models the MIU-system within set theory.  (Incidentally, the
answer to the MU-puzzle is no.)

\chapter{Metamath Language EBNF}%
\label{BNF}%
\index{Metamath Language EBNF}

The following is a formal description of the basic Metamath language syntax
(with compressed proofs and support for unknown proof steps).
It is defined using the
Extended Backus--Naur Form (EBNF)\index{Extended Backus--Naur Form}\index{EBNF}
notation from W3C\index{W3C}
\textit{Extensible Markup Language (XML) 1.0 (Fifth Edition)}
(W3C Recommendation 26 November 2008) at
\url{https://www.w3.org/TR/xml/#sec-notation}.

The \texttt{database}
rule is processed until the end of the file (\texttt{EOF}).
The rules eventually require reading whitespace-separated tokens.
A token has an upper-case definition (see below)
or is a string constant in a non-token (such as \texttt{'\$a'}).
We intend for this to be correct, but if there is a conflict the
rules of section \ref{spec} govern. That section also discusses
non-syntax restrictions not shown here
(e.g., that each new label token
defined in a \texttt{hypothesis-stmt} or \texttt{assert-stmt}
must be unique).

\begin{verbatim}
database ::= outermost-scope-stmt*

outermost-scope-stmt ::=
  include-stmt | constant-stmt | stmt

/* File inclusion command; process file as a database.
   Databases should NOT have a comment in the filename. */
include-stmt ::= '$[' filename '$]'

/* Constant symbols declaration. */
constant-stmt ::= '$c' constant+ '$.'

/* A normal statement can occur in any scope. */
stmt ::= block | variable-stmt | disjoint-stmt |
  hypothesis-stmt | assert-stmt

/* A block. You can have 0 statements in a block. */
block ::= '${' stmt* '$}'

/* Variable symbols declaration. */
variable-stmt ::= '$v' variable+ '$.'

/* Disjoint variables. Simple disjoint statements have
   2 variables, i.e., "variable*" is empty for them. */
disjoint-stmt ::= '$d' variable variable variable* '$.'

hypothesis-stmt ::= floating-stmt | essential-stmt

/* Floating (variable-type) hypothesis. */
floating-stmt ::= LABEL '$f' typecode variable '$.'

/* Essential (logical) hypothesis. */
essential-stmt ::= LABEL '$e' typecode MATH-SYMBOL* '$.'

assert-stmt ::= axiom-stmt | provable-stmt

/* Axiomatic assertion. */
axiom-stmt ::= LABEL '$a' typecode MATH-SYMBOL* '$.'

/* Provable assertion. */
provable-stmt ::= LABEL '$p' typecode MATH-SYMBOL*
  '$=' proof '$.'

/* A proof. Proofs may be interspersed by comments.
   If '?' is in a proof it's an "incomplete" proof. */
proof ::= uncompressed-proof | compressed-proof
uncompressed-proof ::= (LABEL | '?')+
compressed-proof ::= '(' LABEL* ')' COMPRESSED-PROOF-BLOCK+

typecode ::= constant

filename ::= MATH-SYMBOL /* No whitespace or '$' */
constant ::= MATH-SYMBOL
variable ::= MATH-SYMBOL
\end{verbatim}

\needspace{2\baselineskip}
A \texttt{frame} is a sequence of 0 or more
\texttt{disjoint-{\allowbreak}stmt} and
\texttt{hypotheses-{\allowbreak}stmt} statements
(possibly interleaved with other non-\texttt{assert-stmt} statements)
followed by one \texttt{assert-stmt}.

\needspace{3\baselineskip}
Here are the rules for lexical processing (tokenization) beyond
the constant tokens shown above.
By convention these tokenization rules have upper-case names.
Every token is read for the longest possible length.
Whitespace-separated tokens are read sequentially;
note that the separating whitespace and \texttt{\$(} ... \texttt{\$)}
comments are skipped.

If a token definition uses another token definition, the whole thing
is considered a single token.
A pattern that is only part of a full token has a name beginning
with an underscore (``\_'').
An implementation could tokenize many tokens as a
\texttt{PRINTABLE-SEQUENCE}
and then check if it meets the more specific rule shown here.

Comments do not nest, and both \texttt{\$(} and \texttt{\$)}
have to be surrounded
by at least one whitespace character (\texttt{\_WHITECHAR}).
Technically comments end without consuming the trailing
\texttt{\_WHITECHAR}, but the trailing
\texttt{\_WHITECHAR} gets ignored anyway so we ignore that detail here.
Metamath language processors
are not required to support \texttt{\$)} followed
immediately by a bare end-of-file, because the closing
comment symbol is supposed to be followed by a
\texttt{\_WHITECHAR} such as a newline.

\begin{verbatim}
PRINTABLE-SEQUENCE ::= _PRINTABLE-CHARACTER+

MATH-SYMBOL ::= (_PRINTABLE-CHARACTER - '$')+

/* ASCII non-whitespace printable characters */
_PRINTABLE-CHARACTER ::= [#x21-#x7e]

LABEL ::= ( _LETTER-OR-DIGIT | '.' | '-' | '_' )+

_LETTER-OR-DIGIT ::= [A-Za-z0-9]

COMPRESSED-PROOF-BLOCK ::= ([A-Z] | '?')+

/* Define whitespace between tokens. The -> SKIP
   means that when whitespace is seen, it is
   skipped and we simply read again. */
WHITESPACE ::= (_WHITECHAR+ | _COMMENT) -> SKIP

/* Comments. $( ... $) and do not nest. */
_COMMENT ::= '$(' (_WHITECHAR+ (PRINTABLE-SEQUENCE - '$)'))*
  _WHITECHAR+ '$)' _WHITECHAR

/* Whitespace: (' ' | '\t' | '\r' | '\n' | '\f') */
_WHITECHAR ::= [#x20#x09#x0d#x0a#x0c]
\end{verbatim}
% This EBNF was developed as a collaboration between
% David A. Wheeler\index{Wheeler, David A.},
% Mario Carneiro\index{Carneiro, Mario}, and
% Benoit Jubin\index{Jubin, Benoit}, inspired by a request
% (and a lot of initial work) by Benoit Jubin.
%
% \chapter{Disclaimer and Trademarks}
%
% Information in this document is subject to change without notice and does not
% represent a commitment on the part of Norman Megill.
% \vspace{2ex}
%
% \noindent Norman D. Megill makes no warranties, either express or implied,
% regarding the Metamath computer software package.
%
% \vspace{2ex}
%
% \noindent Any trademarks mentioned in this book are the property of
% their respective owners.  The name ``Metamath'' is a trademark of
% Norman Megill.
%
\cleardoublepage
\phantomsection  % fixes the link anchor
\addcontentsline{toc}{chapter}{\bibname}

\bibliography{metamath}
%\input{metamath.bbl}

\raggedright
\cleardoublepage
\phantomsection % fixes the link anchor
\addcontentsline{toc}{chapter}{\indexname}
%\printindex   ??
\input{metamath.ind}

\end{document}



\end{document}



\raggedright
\cleardoublepage
\phantomsection % fixes the link anchor
\addcontentsline{toc}{chapter}{\indexname}
%\printindex   ??
% metamath.tex - Version of 2-Jun-2019
% If you change the date above, also change the "Printed date" below.
% SPDX-License-Identifier: CC0-1.0
%
%                              PUBLIC DOMAIN
%
% This file (specifically, the version of this file with the above date)
% has been released into the Public Domain per the
% Creative Commons CC0 1.0 Universal (CC0 1.0) Public Domain Dedication
% https://creativecommons.org/publicdomain/zero/1.0/
%
% The public domain release applies worldwide.  In case this is not
% legally possible, the right is granted to use the work for any purpose,
% without any conditions, unless such conditions are required by law.
%
% Several short, attributed quotations from copyrighted works
% appear in this file under the ``fair use'' provision of Section 107 of
% the United States Copyright Act (Title 17 of the {\em United States
% Code}).  The public-domain status of this file is not applicable to
% those quotations.
%
% Norman Megill - email: nm(at)alum(dot)mit(dot)edu
%
% David A. Wheeler also donates his improvements to this file to the
% public domain per the CC0.  He works at the Institute for Defense Analyses
% (IDA), but IDA has agreed that this Metamath work is outside its "lane"
% and is not a work by IDA.  This was specifically confirmed by
% Margaret E. Myers (Division Director of the Information Technology
% and Systems Division) on 2019-05-24 and by Ben Lindorf (General Counsel)
% on 2019-05-22.

% This file, 'metamath.tex', is self-contained with everything needed to
% generate the the PDF file 'metamath.pdf' (the _Metamath_ book) on
% standard LaTeX 2e installations.  The auxiliary files are embedded with
% "filecontents" commands.  To generate metamath.pdf file, run these
% commands under Linux or Cygwin in the directory that contains
% 'metamath.tex':
%
%   rm -f realref.sty metamath.bib
%   touch metamath.ind
%   pdflatex metamath
%   pdflatex metamath
%   bibtex metamath
%   makeindex metamath
%   pdflatex metamath
%   pdflatex metamath
%
% The warnings that occur in the initial runs of pdflatex can be ignored.
% For the final run,
%
%   egrep -i 'error|warn' metamath.log
%
% should show exactly these 5 warnings:
%
%   LaTeX Warning: File `realref.sty' already exists on the system.
%   LaTeX Warning: File `metamath.bib' already exists on the system.
%   LaTeX Font Warning: Font shape `OMS/cmtt/m/n' undefined
%   LaTeX Font Warning: Font shape `OMS/cmtt/bx/n' undefined
%   LaTeX Font Warning: Some font shapes were not available, defaults
%       substituted.
%
% Search for "Uncomment" below if you want to suppress hyperlink boxes
% in the PDF output file
%
% TYPOGRAPHICAL NOTES:
% * It is customary to use an en dash (--) to "connect" names of different
%   people (and to denote ranges), and use a hyphen (-) for a
%   single compound name. Examples of connected multiple people are
%   Zermelo--Fraenkel, Schr\"{o}der--Bernstein, Tarski--Grothendieck,
%   Hewlett--Packard, and Backus--Naur.  Examples of a single person with
%   a compound name include Levi-Civita, Mittag-Leffler, and Burali-Forti.
% * Use non-breaking spaces after page abbreviations, e.g.,
%   p.~\pageref{note2002}.
%
% --------------------------- Start of realref.sty -----------------------------
\begin{filecontents}{realref.sty}
% Save the following as realref.sty.
% You can then use it with \usepackage{realref}
%
% This has \pageref jumping to the page on which the ref appears,
% \ref jumping to the point of the anchor, and \sectionref
% jumping to the start of section.
%
% Author:  Anthony Williams
%          Software Engineer
%          Nortel Networks Optical Components Ltd
% Date:    9 Nov 2001 (posted to comp.text.tex)
%
% The following declaration was made by Anthony Williams on
% 24 Jul 2006 (private email to Norman Megill):
%
%   ``I hereby donate the code for realref.sty posted on the
%   comp.text.tex newsgroup on 9th November 2001, accessible from
%   http://groups.google.com/group/comp.text.tex/msg/5a0e1cc13ea7fbb2
%   to the public domain.''
%
\ProvidesPackage{realref}
\RequirePackage[plainpages=false,pdfpagelabels=true]{hyperref}
\def\realref@anchorname{}
\AtBeginDocument{%
% ensure every label is a possible hyperlink target
\let\realref@oldrefstepcounter\refstepcounter%
\DeclareRobustCommand{\refstepcounter}[1]{\realref@oldrefstepcounter{#1}
\edef\realref@anchorname{\string #1.\@currentlabel}%
}%
\let\realref@oldlabel\label%
\DeclareRobustCommand{\label}[1]{\realref@oldlabel{#1}\hypertarget{#1}{}%
\@bsphack\protected@write\@auxout{}{%
    \string\expandafter\gdef\protect\csname
    page@num.#1\string\endcsname{\thepage}%
    \string\expandafter\gdef\protect\csname
    ref@num.#1\string\endcsname{\@currentlabel}%
    \string\expandafter\gdef\protect\csname
    sectionref@name.#1\string\endcsname{\realref@anchorname}%
}\@esphack}%
\DeclareRobustCommand\pageref[1]{{\edef\a{\csname
            page@num.#1\endcsname}\expandafter\hyperlink{page.\a}{\a}}}%
\DeclareRobustCommand\ref[1]{{\edef\a{\csname
            ref@num.#1\endcsname}\hyperlink{#1}{\a}}}%
\DeclareRobustCommand\sectionref[1]{{\edef\a{\csname
            ref@num.#1\endcsname}\edef\b{\csname
            sectionref@name.#1\endcsname}\hyperlink{\b}{\a}}}%
}
\end{filecontents}
% ---------------------------- End of realref.sty ------------------------------

% --------------------------- Start of metamath.bib -----------------------------
\begin{filecontents}{metamath.bib}
@book{Albers, editor = "Donald J. Albers and G. L. Alexanderson",
  title = "Mathematical People",
  publisher = "Contemporary Books, Inc.",
  address = "Chicago",
  note = "[QA28.M37]",
  year = 1985 }
@book{Anderson, author = "Alan Ross Anderson and Nuel D. Belnap",
  title = "Entailment",
  publisher = "Princeton University Press",
  address = "Princeton",
  volume = 1,
  note = "[QA9.A634 1975 v.1]",
  year = 1975}
@book{Barrow, author = "John D. Barrow",
  title = "Theories of Everything:  The Quest for Ultimate Explanation",
  publisher = "Oxford University Press",
  address = "Oxford",
  note = "[Q175.B225]",
  year = 1991 }
@book{Behnke,
  editor = "H. Behnke and F. Backmann and K. Fladt and W. S{\"{u}}ss",
  title = "Fundamentals of Mathematics",
  volume = "I",
  publisher = "The MIT Press",
  address = "Cambridge, Massachusetts",
  note = "[QA37.2.B413]",
  year = 1974 }
@book{Bell, author = "J. L. Bell and M. Machover",
  title = "A Course in Mathematical Logic",
  publisher = "North-Holland",
  address = "Amsterdam",
  note = "[QA9.B3953]",
  year = 1977 }
@inproceedings{Blass, author = "Andrea Blass",
  title = "The Interaction Between Category Theory and Set Theory",
  pages = "5--29",
  booktitle = "Mathematical Applications of Category Theory (Proceedings
     of the Special Session on Mathematical Applications
     Category Theory, 89th Annual Meeting of the American Mathematical
     Society, held in Denver, Colorado January 5--9, 1983)",
  editor = "John Walter Gray",
  year = 1983,
  note = "[QA169.A47 1983]",
  publisher = "American Mathematical Society",
  address = "Providence, Rhode Island"}
@proceedings{Bledsoe, editor = "W. W. Bledsoe and D. W. Loveland",
  title = "Automated Theorem Proving:  After 25 Years (Proceedings
     of the Special Session on Automatic Theorem Proving,
     89th Annual Meeting of the American Mathematical
     Society, held in Denver, Colorado January 5--9, 1983)",
  year = 1983,
  note = "[QA76.9.A96.S64 1983]",
  publisher = "American Mathematical Society",
  address = "Providence, Rhode Island" }
@book{Boolos, author = "George S. Boolos and Richard C. Jeffrey",
  title = "Computability and Log\-ic",
  publisher = "Cambridge University Press",
  edition = "third",
  address = "Cambridge",
  note = "[QA9.59.B66 1989]",
  year = 1989 }
@book{Campbell, author = "John Campbell",
  title = "Programmer's Progress",
  publisher = "White Star Software",
  address = "Box 51623, Palo Alto, CA 94303",
  year = 1991 }
@article{DBLP:journals/corr/Carneiro14,
  author    = {Mario Carneiro},
  title     = {Conversion of {HOL} Light proofs into Metamath},
  journal   = {CoRR},
  volume    = {abs/1412.8091},
  year      = {2014},
  url       = {http://arxiv.org/abs/1412.8091},
  archivePrefix = {arXiv},
  eprint    = {1412.8091},
  timestamp = {Mon, 13 Aug 2018 16:47:05 +0200},
  biburl    = {https://dblp.org/rec/bib/journals/corr/Carneiro14},
  bibsource = {dblp computer science bibliography, https://dblp.org}
}
@article{CarneiroND,
  author    = {Mario Carneiro},
  title     = {Natural Deductions in the Metamath Proof Language},
  url       = {http://us.metamath.org/ocat/natded.pdf},
  year      = 2014
}
@inproceedings{Chou, author = "Shang-Ching Chou",
  title = "Proving Elementary Geometry Theorems Using {W}u's Algorithm",
  pages = "243--286",
  booktitle = "Automated Theorem Proving:  After 25 Years (Proceedings
     of the Special Session on Automatic Theorem Proving,
     89th Annual Meeting of the American Mathematical
     Society, held in Denver, Colorado January 5--9, 1983)",
  editor = "W. W. Bledsoe and D. W. Loveland",
  year = 1983,
  note = "[QA76.9.A96.S64 1983]",
  publisher = "American Mathematical Society",
  address = "Providence, Rhode Island" }
@book{Clemente, author = "Daniel Clemente Laboreo",
  title = "Introduction to natural deduction",
  year = 2014,
  url = "http://www.danielclemente.com/logica/dn.en.pdf" }
@incollection{Courant, author = "Richard Courant and Herbert Robbins",
  title = "Topology",
  pages = "573--590",
  booktitle = "The World of Mathematics, Volume One",
  editor = "James R. Newman",
  publisher = "Simon and Schuster",
  address = "New York",
  note = "[QA3.W67 1988]",
  year = 1956 }
@book{Curry, author = "Haskell B. Curry",
  title = "Foundations of Mathematical Logic",
  publisher = "Dover Publications, Inc.",
  address = "New York",
  note = "[QA9.C976 1977]",
  year = 1977 }
@book{Davis, author = "Philip J. Davis and Reuben Hersh",
  title = "The Mathematical Experience",
  publisher = "Birkh{\"{a}}user Boston",
  address = "Boston",
  note = "[QA8.4.D37 1982]",
  year = 1981 }
@incollection{deMillo,
  author = "Richard de Millo and Richard Lipton and Alan Perlis",
  title = "Social Processes and Proofs of Theorems and Programs",
  pages = "267--285",
  booktitle = "New Directions in the Philosophy of Mathematics",
  editor = "Thomas Tymoczko",
  publisher = "Birkh{\"{a}}user Boston, Inc.",
  address = "Boston",
  note = "[QA8.6.N48 1986]",
  year = 1986 }
@book{Edwards, author = "Robert E. Edwards",
  title = "A Formal Background to Mathematics",
  publisher = "Springer-Verlag",
  address = "New York",
  note = "[QA37.2.E38 v.1a]",
  year = 1979 }
@book{Enderton, author = "Herbert B. Enderton",
  title = "Elements of Set Theory",
  publisher = "Academic Press, Inc.",
  address = "San Diego",
  note = "[QA248.E5]",
  year = 1977 }
@book{Goodstein, author = "R. L. Goodstein",
  title = "Development of Mathematical Logic",
  publisher = "Springer-Verlag New York Inc.",
  address = "New York",
  note = "[QA9.G6554]",
  year = 1971 }
@book{Guillen, author = "Michael Guillen",
  title = "Bridges to Infinity",
  publisher = "Jeremy P. Tarcher, Inc.",
  address = "Los Angeles",
  note = "[QA93.G8]",
  year = 1983 }
@book{Hamilton, author = "Alan G. Hamilton",
  title = "Logic for Mathematicians",
  edition = "revised",
  publisher = "Cambridge University Press",
  address = "Cambridge",
  note = "[QA9.H298]",
  year = 1988 }
@unpublished{Harrison, author = "John Robert Harrison",
  title = "Metatheory and Reflection in Theorem Proving:
    A Survey and Critique",
  note = "Technical Report
    CRC-053.
    SRI Cambridge,
    Millers Yard, Cambridge, UK,
    1995.
    Available on the Web as
{\verb+http:+}\-{\verb+//www.cl.cam.ac.uk/users/jrh/papers/reflect.html+}"}
@TECHREPORT{Harrison-thesis,
        author          = "John Robert Harrison",
        title           = "Theorem Proving with the Real Numbers",
        institution   = "University of Cambridge Computer
                         Lab\-o\-ra\-to\-ry",
        address         = "New Museums Site, Pembroke Street, Cambridge,
                           CB2 3QG, UK",
        year            = 1996,
        number          = 408,
        type            = "Technical Report",
        note            = "Author's PhD thesis,
   available on the Web at
{\verb+http:+}\-{\verb+//www.cl.cam.ac.uk+}\-{\verb+/users+}\-{\verb+/jrh+}%
\-{\verb+/papers+}\-{\verb+/thesis.html+}"}
@book{Herrlich, author = "Horst Herrlich and George E. Strecker",
  title = "Category Theory:  An Introduction",
  publisher = "Allyn and Bacon Inc.",
  address = "Boston",
  note = "[QA169.H567]",
  year = 1973 }
@article{Hindley, author = "J. Roger Hindley and David Meredith",
  title = "Principal Type-Schemes and Condensed Detachment",
  journal = "The Journal of Symbolic Logic",
  volume = 55,
  year = 1990,
  note = "[QA.J87]",
  pages = "90--105" }
@book{Hofstadter, author = "Douglas R. Hofstadter",
  title = "G{\"{o}}del, Escher, Bach",
  publisher = "Basic Books, Inc.",
  address = "New York",
  note = "[QA9.H63 1980]",
  year = 1979 }
@article{Indrzejczak, author= "Andrzej Indrzejczak",
  title = "Natural Deduction, Hybrid Systems and Modal Logic",
  journal = "Trends in Logic",
  volume = 30,
  publisher = "Springer",
  year = 2010 }
@article{Kalish, author = "D. Kalish and R. Montague",
  title = "On {T}arski's Formalization of Predicate Logic with Identity",
  journal = "Archiv f{\"{u}}r Mathematische Logik und Grundlagenfor\-schung",
  volume = 7,
  year = 1965,
  note = "[QA.A673]",
  pages = "81--101" }
@article{Kalman, author = "J. A. Kalman",
  title = "Condensed Detachment as a Rule of Inference",
  journal = "Studia Logica",
  volume = 42,
  number = 4,
  year = 1983,
  note = "[B18.P6.S933]",
  pages = "443-451" }
@book{Kline, author = "Morris Kline",
  title = "Mathematical Thought from Ancient to Modern Times",
  publisher = "Oxford University Press",
  address = "New York",
  note = "[QA21.K516 1990 v.3]",
  year = 1972 }
@book{Klinel, author = "Morris Kline",
  title = "Mathematics, The Loss of Certainty",
  publisher = "Oxford University Press",
  address = "New York",
  note = "[QA21.K525]",
  year = 1980 }
@book{Kramer, author = "Edna E. Kramer",
  title = "The Nature and Growth of Modern Mathematics",
  publisher = "Princeton University Press",
  address = "Princeton, New Jersey",
  note = "[QA93.K89 1981]",
  year = 1981 }
@article{Knill, author = "Oliver Knill",
  title = "Some Fundamental Theorems in Mathematics",
  year = "2018",
  url = "https://arxiv.org/abs/1807.08416" }
@book{Landau, author = "Edmund Landau",
  title = "Foundations of Analysis",
  publisher = "Chelsea Publishing Company",
  address = "New York",
  edition = "second",
  note = "[QA241.L2541 1960]",
  year = 1960 }
@article{Leblanc, author = "Hugues Leblanc",
  title = "On {M}eyer and {L}ambert's Quantificational Calculus {FQ}",
  journal = "The Journal of Symbolic Logic",
  volume = 33,
  year = 1968,
  note = "[QA.J87]",
  pages = "275--280" }
@article{Lejewski, author = "Czeslaw Lejewski",
  title = "On Implicational Definitions",
  journal = "Studia Logica",
  volume = 8,
  year = 1958,
  note = "[B18.P6.S933]",
  pages = "189--208" }
@book{Levy, author = "Azriel Levy",
  title = "Basic Set Theory",
  publisher = "Dover Publications",
  address = "Mineola, NY",
  year = "2002"
}
@book{Margaris, author = "Angelo Margaris",
  title = "First Order Mathematical Logic",
  publisher = "Blaisdell Publishing Company",
  address = "Waltham, Massachusetts",
  note = "[QA9.M327]",
  year = 1967}
@book{Manin, author = "Yu I. Manin",
  title = "A Course in Mathematical Logic",
  publisher = "Springer-Verlag",
  address = "New York",
  note = "[QA9.M29613]",
  year = "1977" }
@article{Mathias, author = "Adrian R. D. Mathias",
  title = "A Term of Length 4,523,659,424,929",
  journal = "Synthese",
  volume = 133,
  year = 2002,
  note = "[Q.S993]",
  pages = "75--86" }
@article{Megill, author = "Norman D. Megill",
  title = "A Finitely Axiomatized Formalization of Predicate Calculus
     with Equality",
  journal = "Notre Dame Journal of Formal Logic",
  volume = 36,
  year = 1995,
  note = "[QA.N914]",
  pages = "435--453" }
@unpublished{Megillc, author = "Norman D. Megill",
  title = "A Shorter Equivalent of the Axiom of Choice",
  month = "June",
  note = "Unpublished",
  year = 1991 }
@article{MegillBunder, author = "Norman D. Megill and Martin W.
    Bunder",
  title = "Weaker {D}-Complete Logics",
  journal = "Journal of the IGPL",
  volume = 4,
  year = 1996,
  pages = "215--225",
  note = "Available on the Web at
{\verb+http:+}\-{\verb+//www.mpi-sb.mpg.de+}\-{\verb+/igpl+}%
\-{\verb+/Journal+}\-{\verb+/V4-2+}\-{\verb+/#Megill+}"}
}
@book{Mendelson, author = "Elliott Mendelson",
  title = "Introduction to Mathematical Logic",
  edition = "second",
  publisher = "D. Van Nostrand Company, Inc.",
  address = "New York",
  note = "[QA9.M537 1979]",
  year = 1979 }
@article{Meredith, author = "David Meredith",
  title = "In Memoriam {C}arew {A}rthur {M}eredith (1904-1976)",
  journal = "Notre Dame Journal of Formal Logic",
  volume = 18,
  year = 1977,
  note = "[QA.N914]",
  pages = "513--516" }
@article{CAMeredith, author = "C. A. Meredith",
  title = "Single Axioms for the Systems ({C},{N}), ({C},{O}) and ({A},{N})
      of the Two-Valued Propositional Calculus",
  journal = "The Journal of Computing Systems",
  volume = 3,
  year = 1953,
  pages = "155--164" }
@article{Monk, author = "J. Donald Monk",
  title = "Provability With Finitely Many Variables",
  journal = "The Journal of Symbolic Logic",
  volume = 27,
  year = 1971,
  note = "[QA.J87]",
  pages = "353--358" }
@article{Monks, author = "J. Donald Monk",
  title = "Substitutionless Predicate Logic With Identity",
  journal = "Archiv f{\"{u}}r Mathematische Logik und Grundlagenfor\-schung",
  volume = 7,
  year = 1965,
  pages = "103--121" }
  %% Took out this from above to prevent LaTeX underfull warning:
  % note = "[QA.A673]",
@book{Moore, author = "A. W. Moore",
  title = "The Infinite",
  publisher = "Routledge",
  address = "New York",
  note = "[BD411.M59]",
  year = 1989}
@book{Munkres, author = "James R. Munkres",
  title = "Topology: A First Course",
  publisher = "Prentice-Hall, Inc.",
  address = "Englewood Cliffs, New Jersey",
  note = "[QA611.M82]",
  year = 1975}
@article{Nemesszeghy, author = "E. Z. Nemesszeghy and E. A. Nemesszeghy",
  title = "On Strongly Creative Definitions:  A Reply to {V}. {F}. {R}ickey",
  journal = "Logique et Analyse (N.\ S.)",
  year = 1977,
  volume = 20,
  note = "[BC.L832]",
  pages = "111--115" }
@unpublished{Nemeti, author = "N{\'{e}}meti, I.",
  title = "Algebraizations of Quantifier Logics, an Overview",
  note = "Version 11.4, preprint, Mathematical Institute, Budapest,
    1994.  A shortened version without proofs appeared in
    ``Algebraizations of quantifier logics, an introductory overview,''
   {\em Studia Logica}, 50:485--569, 1991 [B18.P6.S933]"}
@article{Pavicic, author = "M. Pavi{\v{c}}i{\'{c}}",
  title = "A New Axiomatization of Unified Quantum Logic",
  journal = "International Journal of Theoretical Physics",
  year = 1992,
  volume = 31,
  note = "[QC.I626]",
  pages = "1753 --1766" }
@book{Penrose, author = "Roger Penrose",
  title = "The Emperor's New Mind",
  publisher = "Oxford University Press",
  address = "New York",
  note = "[Q335.P415]",
  year = 1989 }
@book{PetersonI, author = "Ivars Peterson",
  title = "The Mathematical Tourist",
  publisher = "W. H. Freeman and Company",
  address = "New York",
  note = "[QA93.P475]",
  year = 1988 }
@article{Peterson, author = "Jeremy George Peterson",
  title = "An automatic theorem prover for substitution and detachment systems",
  journal = "Notre Dame Journal of Formal Logic",
  volume = 19,
  year = 1978,
  note = "[QA.N914]",
  pages = "119--122" }
@book{Quine, author = "Willard Van Orman Quine",
  title = "Set Theory and Its Logic",
  edition = "revised",
  publisher = "The Belknap Press of Harvard University Press",
  address = "Cambridge, Massachusetts",
  note = "[QA248.Q7 1969]",
  year = 1969 }
@article{Robinson, author = "J. A. Robinson",
  title = "A Machine-Oriented Logic Based on the Resolution Principle",
  journal = "Journal of the Association for Computing Machinery",
  year = 1965,
  volume = 12,
  pages = "23--41" }
@article{RobinsonT, author = "T. Thacher Robinson",
  title = "Independence of Two Nice Sets of Axioms for the Propositional
    Calculus",
  journal = "The Journal of Symbolic Logic",
  volume = 33,
  year = 1968,
  note = "[QA.J87]",
  pages = "265--270" }
@book{Rucker, author = "Rudy Rucker",
  title = "Infinity and the Mind:  The Science and Philosophy of the
    Infinite",
  publisher = "Bantam Books, Inc.",
  address = "New York",
  note = "[QA9.R79 1982]",
  year = 1982 }
@book{Russell, author = "Bertrand Russell",
  title = "Mysticism and Logic, and Other Essays",
  publisher = "Barnes \& Noble Books",
  address = "Totowa, New Jersey",
  note = "[B1649.R963.M9 1981]",
  year = 1981 }
@article{Russell2, author = "Bertrand Russell",
  title = "Recent Work on the Principles of Mathematics",
  journal = "International Monthly",
  volume = 4,
  year = 1901,
  pages = "84"}
@article{Schmidt, author = "Eric Schmidt",
  title = "Reductions in Norman Megill's axiom system for complex numbers",
  url = "http://us.metamath.org/downloads/schmidt-cnaxioms.pdf",
  year = "2012" }
@book{Shoenfield, author = "Joseph R. Shoenfield",
  title = "Mathematical Logic",
  publisher = "Addison-Wesley Publishing Company, Inc.",
  address = "Reading, Massachusetts",
  year = 1967,
  note = "[QA9.S52]" }
@book{Smullyan, author = "Raymond M. Smullyan",
  title = "Theory of Formal Systems",
  publisher = "Princeton University Press",
  address = "Princeton, New Jersey",
  year = 1961,
  note = "[QA248.5.S55]" }
@book{Solow, author = "Daniel Solow",
  title = "How to Read and Do Proofs:  An Introduction to Mathematical
    Thought Process",
  publisher = "John Wiley \& Sons",
  address = "New York",
  year = 1982,
  note = "[QA9.S577]" }
@book{Stark, author = "Harold M. Stark",
  title = "An Introduction to Number Theory",
  publisher = "Markham Publishing Company",
  address = "Chicago",
  note = "[QA241.S72 1978]",
  year = 1970 }
@article{Swart, author = "E. R. Swart",
  title = "The Philosophical Implications of the Four-Color Problem",
  journal = "American Mathematical Monthly",
  year = 1980,
  volume = 87,
  month = "November",
  note = "[QA.A5125]",
  pages = "697--707" }
@book{Szpiro, author = "George G. Szpiro",
  title = "Poincar{\'{e}}'s Prize: The Hundred-Year Quest to Solve One
    of Math's Greatest Puzzles",
  publisher = "Penguin Books Ltd",
  address = "London",
  note = "[QA43.S985 2007]",
  year = 2007}
@book{Takeuti, author = "Gaisi Takeuti and Wilson M. Zaring",
  title = "Introduction to Axiomatic Set Theory",
  edition = "second",
  publisher = "Springer-Verlag New York Inc.",
  address = "New York",
  note = "[QA248.T136 1982]",
  year = 1982}
@inproceedings{Tarski, author = "Alfred Tarski",
  title = "What is Elementary Geometry",
  pages = "16--29",
  booktitle = "The Axiomatic Method, with Special Reference to Geometry and
     Physics (Proceedings of an International Symposium held at the University
     of California, Berkeley, December 26, 1957 --- January 4, 1958)",
  editor = "Leon Henkin and Patrick Suppes and Alfred Tarski",
  year = 1959,
  publisher = "North-Holland Publishing Company",
  address = "Amsterdam"}
@article{Tarski1965, author = "Alfred Tarski",
  title = "A Simplified Formalization of Predicate Logic with Identity",
  journal = "Archiv f{\"{u}}r Mathematische Logik und Grundlagenforschung",
  volume = 7,
  year = 1965,
  note = "[QA.A673]",
  pages = "61--79" }
@book{Tymoczko,
  title = "New Directions in the Philosophy of Mathematics",
  editor = "Thomas Tymoczko",
  publisher = "Birkh{\"{a}}user Boston, Inc.",
  address = "Boston",
  note = "[QA8.6.N48 1986]",
  year = 1986 }
@incollection{Wang,
  author = "Hao Wang",
  title = "Theory and Practice in Mathematics",
  pages = "129--152",
  booktitle = "New Directions in the Philosophy of Mathematics",
  editor = "Thomas Tymoczko",
  publisher = "Birkh{\"{a}}user Boston, Inc.",
  address = "Boston",
  note = "[QA8.6.N48 1986]",
  year = 1986 }
@manual{Webster,
  title = "Webster's New Collegiate Dictionary",
  organization = "G. \& C. Merriam Co.",
  address = "Springfield, Massachusetts",
  note = "[PE1628.W4M4 1977]",
  year = 1977 }
@manual{Whitehead, author = "Alfred North Whitehead",
  title = "An Introduction to Mathematics",
  year = 1911 }
@book{PM, author = "Alfred North Whitehead and Bertrand Russell",
  title = "Principia Mathematica",
  edition = "second",
  publisher = "Cambridge University Press",
  address = "Cambridge",
  year = "1927",
  note = "(3 vols.) [QA9.W592 1927]" }
@article{DBLP:journals/corr/Whalen16,
  author    = {Daniel Whalen},
  title     = {Holophrasm: a neural Automated Theorem Prover for higher-order logic},
  journal   = {CoRR},
  volume    = {abs/1608.02644},
  year      = {2016},
  url       = {http://arxiv.org/abs/1608.02644},
  archivePrefix = {arXiv},
  eprint    = {1608.02644},
  timestamp = {Mon, 13 Aug 2018 16:46:19 +0200},
  biburl    = {https://dblp.org/rec/bib/journals/corr/Whalen16},
  bibsource = {dblp computer science bibliography, https://dblp.org} }
@article{Wiedijk-revisited,
  author = {Freek Wiedijk},
  title = {The QED Manifesto Revisited},
  year = {2007},
  url = {http://mizar.org/trybulec65/8.pdf} }
@book{Wolfram,
  author = "Stephen Wolfram",
  title = "Mathematica:  A System for Doing Mathematics by Computer",
  edition = "second",
  publisher = "Addison-Wesley Publishing Co.",
  address = "Redwood City, California",
  note = "[QA76.95.W65 1991]",
  year = 1991 }
@book{Wos, author = "Larry Wos and Ross Overbeek and Ewing Lusk and Jim Boyle",
  title = "Automated Reasoning:  Introduction and Applications",
  edition = "second",
  publisher = "McGraw-Hill, Inc.",
  address = "New York",
  note = "[QA76.9.A96.A93 1992]",
  year = 1992 }

%
%
%[1] Church, Alonzo, Introduction to Mathematical Logic,
% Volume 1, Princeton University Press, Princeton, N. J., 1956.
%
%[2] Cohen, Paul J., Set Theory and the Continuum Hypothesis,
% W. A. Benjamin, Inc., Reading, Mass., 1966.
%
%[3] Hamilton, Alan G., Logic for Mathematicians, Cambridge
% University Press,
% Cambridge, 1988.

%[6] Kleene, Stephen Cole, Introduction to Metamathematics, D.  Van
% Nostrand Company, Inc., Princeton (1952).

%[13] Tarski, Alfred, "A simplified formalization of predicate
% logic with identity," Archiv fur Mathematische Logik und
% Grundlagenforschung, vol. 7 (1965), pp. 61-79.

%[14] Tarski, Alfred and Steven Givant, A Formalization of Set
% Theory Without Variables, American Mathematical Society Colloquium
% Publications, vol. 41, American Mathematical Society,
% Providence, R. I., 1987.

%[15] Zeman, J. J., Modal Logic, Oxford University Press, Oxford, 1973.
\end{filecontents}
% --------------------------- End of metamath.bib -----------------------------


%Book: Metamath
%Author:  Norman Megill Email:  nm at alum.mit.edu
%Author:  David A. Wheeler Email:  dwheeler at dwheeler.com

% A book template example
% http://www.stsci.edu/ftp/software/tex/bookstuff/book.template

\documentclass[leqno]{book} % LaTeX 2e. 10pt. Use [leqno,12pt] for 12pt
% hyperref 2002/05/27 v6.72r  (couldn't get pagebackref to work)
\usepackage[plainpages=false,pdfpagelabels=true]{hyperref}

\usepackage{needspace}     % Enable control over page breaks
\usepackage{breqn}         % automatic equation breaking
\usepackage{microtype}     % microtypography, reduces hyphenation

% Packages for flexible tables.  We need to be able to
% wrap text within a cell (with automatically-determined widths) AND
% split a table automatically across multiple pages.
% * "tabularx" wraps text in cells but only 1 page
% * "longtable" goes across pages but by itself is incompatible with tabularx
% * "ltxtable" combines longtable and tabularx, but table contents
%    must be in a separate file.
% * "ltablex" combines tabularx and longtable - must install specially
% * "booktabs" is recommended as a way to improve the look of tables,
%   but doesn't add these capabilities.
% * "tabu" much more capable and seems to be recommended. So use that.

\usepackage{makecell}      % Enable forced line splits within a table cell
% v4.13 needed for tabu: https://tex.stackexchange.com/questions/600724/dimension-too-large-after-recent-longtable-update
\usepackage{longtable}[=v4.13] % Enable multi-page tables  
\usepackage{tabu}          % Multi-page tables with wrapped text in a cell

% You can find more Tex packages using commands like:
% tlmgr search --file tabu.sty
% find /usr/share/texmf-dist/ -name '*tab*'
%
%%%%%%%%%%%%%%%%%%%%%%%%%%%%%%%%%%%%%%%%%%%%%%%%%%%%%%%%%%%%%%%%%%%%%%%%%%%%
% Uncomment the next 3 lines to suppress boxes and colors on the hyperlinks
%%%%%%%%%%%%%%%%%%%%%%%%%%%%%%%%%%%%%%%%%%%%%%%%%%%%%%%%%%%%%%%%%%%%%%%%%%%%
%\hypersetup{
%colorlinks,citecolor=black,filecolor=black,linkcolor=black,urlcolor=black
%}
%
\usepackage{realref}

% Restarting page numbers: try?
%   \printglossary
%   \cleardoublepage
%   \pagenumbering{arabic}
%   \setcounter{page}{1}    ???needed
%   \include{chap1}

% not used:
% \def\R2Lurl#1#2{\mbox{\href{#1}\texttt{#2}}}

\usepackage{amssymb}

% Version 1 of book: margins: t=.4, b=.2, ll=.4, rr=.55
% \usepackage{anysize}
% % \papersize{<height>}{<width>}
% % \marginsize{<left>}{<right>}{<top>}{<bottom>}
% \papersize{9in}{6in}
% % l/r 0.6124-0.6170 works t/b 0.2418-0.3411 = 192pp. 0.2926-03118=exact
% \marginsize{0.7147in}{0.5147in}{0.4012in}{0.2012in}

\usepackage{anysize}
% \papersize{<height>}{<width>}
% \marginsize{<left>}{<right>}{<top>}{<bottom>}
\papersize{9in}{6in}
% l/r 0.85in&0.6431-0.6539 works t/b ?-?
%\marginsize{0.85in}{0.6485in}{0.55in}{0.35in}
\marginsize{0.8in}{0.65in}{0.5in}{0.3in}

% \usepackage[papersize={3.6in,4.8in},hmargin=0.1in,vmargin={0.1in,0.1in}]{geometry}  % page geometry
\usepackage{special-settings}

\raggedbottom
\makeindex

\begin{document}
% Discourage page widows and orphans:
\clubpenalty=300
\widowpenalty=300

%%%%%%% load in AMS fonts %%%%%%% % LaTeX 2.09 - obsolete in LaTeX 2e
%\input{amssym.def}
%\input{amssym.tex}
%\input{c:/texmf/tex/plain/amsfonts/amssym.def}
%\input{c:/texmf/tex/plain/amsfonts/amssym.tex}

\bibliographystyle{plain}
\pagenumbering{roman}
\pagestyle{headings}

\thispagestyle{empty}

\hfill
\vfill

\begin{center}
{\LARGE\bf Metamath} \\
\vspace{1ex}
{\large A Computer Language for Mathematical Proofs} \\
\vspace{7ex}
{\large Norman Megill} \\
\vspace{7ex}
with extensive revisions by \\
\vspace{1ex}
{\large David A. Wheeler} \\
\vspace{7ex}
% Printed date. If changing the date below, also fix the date at the beginning.
2019-06-02
\end{center}

\vfill
\hfill

\newpage
\thispagestyle{empty}

\hfill
\vfill

\begin{center}
$\sim$\ {\sc Public Domain}\ $\sim$

\vspace{2ex}
This book (including its later revisions)
has been released into the Public Domain by Norman Megill per the
Creative Commons CC0 1.0 Universal (CC0 1.0) Public Domain Dedication.
David A. Wheeler has done the same.
This public domain release applies worldwide.  In case this is not
legally possible, the right is granted to use the work for any purpose,
without any conditions, unless such conditions are required by law.
See \url{https://creativecommons.org/publicdomain/zero/1.0/}.

\vspace{3ex}
Several short, attributed quotations from copyrighted works
appear in this book under the ``fair use'' provision of Section 107 of
the United States Copyright Act (Title 17 of the {\em United States
Code}).  The public-domain status of this book is not applicable to
those quotations.

\vspace{3ex}
Any trademarks used in this book are the property of their owners.

% QA76.9.L63.M??

% \vspace{1ex}
%
% \vspace{1ex}
% {\small Permission is granted to make and distribute verbatim copies of this
% book
% provided the copyright notice and this
% permission notice are preserved on all copies.}
%
% \vspace{1ex}
% {\small Permission is granted to copy and distribute modified versions of this
% book under the conditions for verbatim copying, provided that the
% entire
% resulting derived work is distributed under the terms of a permission
% notice
% identical to this one.}
%
% \vspace{1ex}
% {\small Permission is granted to copy and distribute translations of this
% book into another language, under the above conditions for modified
% versions,
% except that this permission notice may be stated in a translation
% approved by the
% author.}
%
% \vspace{1ex}
% %{\small   For a copy of the \LaTeX\ source files for this book, contact
% %the author.} \\
% \ \\
% \ \\

\vspace{7ex}
% ISBN: 1-4116-3724-0 \\
% ISBN: 978-1-4116-3724-5 \\
ISBN: 978-0-359-70223-7 \\
{\ } \\
Lulu Press \\
Morrisville, North Carolina\\
USA


\hfill
\vfill

Norman Megill\\ 93 Bridge St., Lexington, MA 02421 \\
E-mail address: \texttt{nm{\char`\@}alum.mit.edu} \\
\vspace{7ex}
David A. Wheeler \\
E-mail address: \texttt{dwheeler{\char`\@}dwheeler.com} \\
% See notes added at end of Preface for revision history. \\
% For current information on the Metamath software see \\
\vspace{7ex}
\url{http://metamath.org}
\end{center}

\hfill
\vfill

{\parindent0pt%
\footnotesize{%
Cover: Aleph null ($\aleph_0$) is the symbol for the
first infinite cardinal number, discovered by Georg Cantor in 1873.
We use a red aleph null (with dark outline and gold glow) as the Metamath logo.
Credit: Norman Megill (1994) and Giovanni Mascellani (2019),
public domain.%
\index{aleph null}%
\index{Metamath!logo}\index{Cantor, Georg}\index{Mascellani, Giovanni}}}

% \newpage
% \thispagestyle{empty}
%
% \hfill
% \vfill
%
% \begin{center}
% {\it To my son Robin Dwight Megill}
% \end{center}
%
% \vfill
% \hfill
%
% \newpage

\tableofcontents
%\listoftables

\chapter*{Preface}
\markboth{PREFACE}{PREFACE}
\addcontentsline{toc}{section}{Preface}


% (For current information, see the notes added at the
% end of this preface on p.~\pageref{note2002}.)

\subsubsection{Overview}

Metamath\index{Metamath} is a computer language and an associated computer
program for archiving, verifying, and studying mathematical proofs at a very
detailed level.  The Metamath language incorporates no mathematics per se but
treats all mathematical statements as mere sequences of symbols.  You provide
Metamath with certain special sequences (axioms) that tell it what rules
of inference are allowed.  Metamath is not limited to any specific field of
mathematics.  The Metamath language is simple and robust, with an
almost total absence of hard-wired syntax, and
we\footnote{Unless otherwise noted, the words
``I,'' ``me,'' and ``my'' refer to Norman Megill\index{Megill, Norman}, while
``we,'' ``us,'' and ``our'' refer to Norman Megill and
David A. Wheeler\index{Wheeler, David A.}.}
believe that it
provides about the simplest possible framework that allows essentially all of
mathematics to be expressed with absolute rigor.

% index test
%\newcommand{\nn}[1]{#1n}
%\index{aaa@bbb}
%\index{abc!def}
%\index{abd|see{qqq}}
%\index{abe|nn}
%\index{abf|emph}
%\index{abg|(}
%\index{abg|)}

Using the Metamath language, you can build formal or mathematical
systems\index{formal system}\footnote{A formal or mathematical system consists
of a collection of symbols (such as $2$, $4$, $+$ and $=$), syntax rules that
describe how symbols may be combined to form a legal expression (called a
well-formed formula or {\em wff}, pronounced ``whiff''), some starting wffs
called axioms, and inference rules that describe how theorems may be derived
(proved) from the axioms.  A theorem is a mathematical fact such as $2+2=4$.
Strictly speaking, even an obvious fact such as this must be proved from
axioms to be formally acceptable to a mathematician.}\index{theorem}
\index{axiom}\index{rule}\index{well-formed formula (wff)} that involve
inferences from axioms.  Although a database is provided
that includes a recommended set of axioms for standard mathematics, if you
wish you can supply your own symbols, syntax, axioms, rules, and definitions.

The name ``Metamath'' was chosen to suggest that the language provides a
means for {\em describing} mathematics rather than {\em being} the
mathematics itself.  Actually in some sense any mathematical language is
metamathematical.  Symbols written on paper, or stored in a computer,
are not mathematics itself but rather a way of expressing mathematics.
For example ``7'' and ``VII'' are symbols for denoting the number seven
in Arabic and Roman numerals; neither {\em is} the number seven.

If you are able to understand and write computer programs, you should be able
to follow abstract mathematics with the aid of Metamath.  Used in conjunction
with standard textbooks, Metamath can guide you step-by-step towards an
understanding of abstract mathematics from a very rigorous viewpoint, even if
you have no formal abstract mathematics background.  By using a single,
consistent notation to express proofs, once you grasp its basic concepts
Metamath provides you with the ability to immediately follow and dissect
proofs even in totally unfamiliar areas.

Of course, just being able follow a proof will not necessarily give you an
intuitive familiarity with mathematics.  Memorizing the rules of chess does not
give you the ability to appreciate the game of a master, and knowing how the
notes on a musical score map to piano keys does not give you the ability to
hear in your head how it would sound.  But each of these can be a first step.

Metamath allows you to explore proofs in the sense that you can see the
theorem referenced at any step expanded in as much detail as you want, right
down to the underlying axioms of logic and set theory (in the case of the set
theory database provided).  While Metamath will not replace the higher-level
understanding that can only be acquired through exercises and hard work, being
able to see how gaps in a proof are filled in can give you increased
confidence that can speed up the learning process and save you time when you
get stuck.

The Metamath language breaks down a mathematical proof into its tiniest
possible parts.  These can be pieced together, like interlocking
pieces in a puzzle, only in a way that produces correct and absolutely rigorous
mathematics.

The nature of Metamath\index{Metamath} enforces very precise mathematical
thinking, similar to that involved in writing a computer program.  A crucial
difference, though, is that once a proof is verified (by the Metamath program)
to be correct, it is definitely correct; it can never have a hidden
``bug.''\index{computer program bugs}  After getting used to the kind of rigor
and accuracy provided by Metamath, you might even be tempted to
adopt the attitude that a proof should never be considered correct until it
has been verified by a computer, just as you would not completely trust a
manual calculation until you have verified it on a
calculator.

My goal
for Metamath was a system for describing and verifying
mathematics that is completely universal yet conceptually as simple as
possible.  In approaching mathematics from an axiomatic, formal viewpoint, I
wanted Metamath to be able to handle almost any mathematical system, not
necessarily with ease, but at least in principle and hopefully in practice. I
wanted it to verify proofs with absolute rigor, and for this reason Metamath
is what might be thought of as a ``compile-only'' language rather than an
algorithmic or Turing-machine language (Pascal, C, Prolog, Mathematica,
etc.).  In other words, a database written in the Metamath
language doesn't ``do'' anything; it merely exhibits mathematical knowledge
and permits this knowledge to be verified as being correct.  A program in an
algorithmic language can potentially have hidden bugs\index{computer program
bugs} as well as possibly being hard to understand.  But each token in a
Metamath database must be consistent with the database's earlier
contents according to simple, fixed rules.
If a database is verified
to be correct,\footnote{This includes
verification that a sequential list of proof steps results in the specified
theorem.} then the mathematical content is correct if the
verifier is correct and the axioms are correct.
The verification program could be incorrect, but the verification algorithm
is relatively simple (making it unlikely to be implemented incorrectly
by the Metamath program),
and there are over a dozen Metamath database verifiers
written by different people in different programming languages
(so these different verifiers can act as multiple reviewers of a database).
The most-used Metamath database, the Metamath Proof Explorer
(aka \texttt{set.mm}\index{set theory database (\texttt{set.mm})}%
\index{Metamath Proof Explorer}),
is currently verified by four different Metamath verifiers written by
four different people in four different languages, including the
original Metamath program described in this book.
The only ``bugs'' that can exist are in the statement of the axioms,
for example if the axioms are inconsistent (a famous problem shown to be
unsolvable by G\"{o}del's incompleteness theorem\index{G\"{o}del's
incompleteness theorem}).
However, real mathematical systems have very few axioms, and these can
be carefully studied.
All of this provides extraordinarily high confidence that the verified database
is in fact correct.

The Metamath program
doesn't prove theorems automatically but is designed to verify proofs
that you supply to it.
The underlying Metamath language is completely general and has no built-in,
preconceived notions about your formal system\index{formal system}, its logic
or its syntax.
For constructing proofs, the Metamath program has a Proof Assistant\index{Proof
Assistant} which helps you fill in some of a proof step's details, shows you
what choices you have at any step, and verifies the proof as you build it; but
you are still expected to provide the proof.

There are many other programs that can process or generate information
in the Metamath language, and more continue to be written.
This is in part because the Metamath language itself is very simple
and intentionally easy to automatically process.
Some programs, such as \texttt{mmj2}\index{mmj2}, include a proof assistant
that can automate some steps beyond what the Metamath program can do.
Mario Carneiro has developed an algorithm for converting proofs from
the OpenTheory interchange format, which can be translated to and from
any of the HOL family of proof languages (HOL4, HOL Light, ProofPower,
and Isabelle), into the
Metamath language \cite{DBLP:journals/corr/Carneiro14}\index{Carneiro, Mario}.
Daniel Whalen has developed Holophrasm, which can automatically
prove many Metamath proofs using
machine learning\index{machine learning}\index{artificial intelligence}
approaches
(including multiple neural networks\index{neural networks})\cite{DBLP:journals/corr/Whalen16}\index{Whalen, Daniel}.
However,
a discussion of these other programs is beyond the scope of this book.

Like most computer languages, the Metamath\index{Metamath} language uses the
standard ({\sc ascii}) characters on a computer keyboard, so it cannot
directly represent many of the special symbols that mathematicians use.  A
useful feature of the Metamath program is its ability to convert its notation
into the \LaTeX\ typesetting language.\index{latex@{\LaTeX}}  This feature
lets you convert the {\sc ascii} tokens you've defined into standard
mathematical symbols, so you end up with symbols and formulas you are familiar
with instead of somewhat cryptic {\sc ascii} representations of them.
The Metamath program can also generate HTML\index{HTML}, making it easy
to view results on the web and to see related information by using
hypertext links.

Metamath is probably conceptually different from anything you've seen
before and some aspects may take some getting used to.  This book will
help you decide whether Metamath suits your specific needs.

\subsubsection{Setting Your Expectations}
It is important for you to understand what Metamath\index{Metamath} is and is
not.  As mentioned, the Metamath program
is {\em not} an automated theorem prover but
rather a proof verifier.  Developing a database can be tedious, hard work,
especially if you want to make the proofs as short as possible, but it becomes
easier as you build up a collection of useful theorems.  The purpose of
Metamath is simply to document existing mathematics in an absolutely rigorous,
computer-verifiable way, not to aid directly in the creation of new
mathematics.  It also is not a magic solution for learning abstract
mathematics, although it may be helpful to be able to actually see the implied
rigor behind what you are learning from textbooks, as well as providing hints
to work out proofs that you are stumped on.

As of this writing, a sizable set theory database has been developed to
provide a foundation for many fields of mathematics, but much more work would
be required to develop useful databases for specific fields.

Metamath\index{Metamath} ``knows no math;'' it just provides a framework in
which to express mathematics.  Its language is very small.  You can define two
kinds of symbols, constants\index{constant} and variables\index{variable}.
The only thing Metamath knows how to do is to substitute strings of symbols
for the variables\index{substitution!variable}\index{variable substitution} in
an expression based on instructions you provide it in a proof, subject to
certain constraints you specify for the variables.  Even the decimal
representation of a number is merely a string of certain constants (digits)
which together, in a specific context, correspond to whatever mathematical
object you choose to define for it; unlike other computer languages, there is
no actual number stored inside the computer.  In a proof, you in effect
instruct Metamath what symbol substitutions to make in previous axioms or
theorems and join a sequence of them together to result in the desired
theorem.  This kind of symbol manipulation captures the essence of mathematics
at a preaxiomatic level.

\subsubsection{Metamath and Mathematical Literature}

In advanced mathematical literature, proofs are usually presented in the form
of short outlines that often only an expert can follow.  This is partly out of
a desire for brevity, but it would also be unwise (even if it were practical)
to present proofs in complete formal detail, since the overall picture would
be lost.\index{formal proof}

A solution I envision\label{envision} that would allow mathematics to remain
acceptable to the expert, yet increase its accessibility to non-specialists,
consists of a combination of the traditional short, informal proof in print
accompanied by a complete formal proof stored in a computer database.  In an
analogy with a computer program, the informal proof is like a set of comments
that describe the overall reasoning and content of the proof, whereas the
computer database is like the actual program and provides a means for anyone,
even a non-expert, to follow the proof in as much detail as desired, exploring
it back through layers of theorems (like subroutines that call other
subroutines) all the way back to the axioms of the theory.  In addition, the
computer database would have the advantage of providing absolute assurance
that the proof is correct, since each step can be verified automatically.

There are several other approaches besides Metamath to a project such
as this.  Section~\ref{proofverifiers} discusses some of these.

To us, a noble goal would be a database with hundreds of thousands of
theorems and their computer-verifiable proofs, encompassing a significant
fraction of known mathematics and available for instant access.
These would be fully verified by multiple independently-implemented verifiers,
to provide extremely high confidence that the proofs are completely correct.
The database would allow people to investigate whatever details they were
interested in, so that they could confirm whatever portions they wished.
Whether or not Metamath is an appropriate choice remains to be seen, but in
principle we believe it is sufficient.

\subsubsection{Formalism}

Over the past fifty years, a group of French mathematicians working
collectively under the pseudonym of Bourbaki\index{Bourbaki, Nicolas} have
co-authored a series of monographs that attempt to rigorously and
consistently formalize large bodies of mathematics from foundations.  On the
one hand, certainly such an effort has its merits; on the other hand, the
Bourbaki project has been criticized for its ``scholasticism'' and
``hyperaxiomatics'' that hide the intuitive steps that lead to the results
\cite[p.~191]{Barrow}\index{Barrow, John D.}.

Metamath unabashedly carries this philosophy to its extreme and no doubt is
subject to the same kind of criticism.  Nonetheless I think that in
conjunction with conventional approaches to mathematics Metamath can serve a
useful purpose.  The Bourbaki approach is essentially pedagogic, requiring the
reader to become intimately familiar with each detail in a very large
hierarchy before he or she can proceed to the next step.  The difference with
Metamath is that the ``reader'' (user) knows that all details are contained in
its computer database, available as needed; it does not demand that the user
know everything but conveniently makes available those portions that are of
interest.  As the body of all mathematical knowledge grows larger and larger,
no one individual can have a thorough grasp of its entirety.  Metamath
can finalize and put to rest any questions about the validity of any part of it
and can make any part of it accessible, in principle, to a non-specialist.

\subsubsection{A Personal Note}
Why did I develop Metamath\index{Metamath}?  I enjoy abstract mathematics, but
I sometimes get lost in a barrage of definitions and start to lose confidence
that my proofs are correct.  Or I reach a point where I lose sight of how
anything I'm doing relates to the axioms that a theory is based on and am
sometimes suspicious that there may be some overlooked implicit axiom
accidentally introduced along the way (as happened historically with Euclidean
geometry\index{Euclidean geometry}, whose omission of Pasch's
axiom\index{Pasch's axiom} went unnoticed for 2000 years
\cite[p.~160]{Davis}!). I'm also somewhat lazy and wish to avoid the effort
involved in re-verifying the gaps in informal proofs ``left to the reader;'' I
prefer to figure them out just once and not have to go through the same
frustration a year from now when I've forgotten what I did.  Metamath provides
better recovery of my efforts than scraps of paper that I can't
decipher anymore.  But mostly I find very appealing the idea of rigorously
archiving mathematical knowledge in a computer database, providing precision,
certainty, and elimination of human error.

\subsubsection{Note on Bibliography and Index}

The Bibliography usually includes the Library of Congress classification
for a work to make it easier for you to find it in on a university
library shelf.  The Index has author references to pages where their works
are cited, even though the authors' names may not appear on those pages.

\subsubsection{Acknowledgments}

Acknowledgments are first due to my wife, Deborah (who passed away on
September 4, 1998), for critiquing the manu\-script but most of all for
her patience and support.  I also wish to thank Joe Wright, Richard
Becker, Clarke Evans, Buddha Buck, and Jeremy Henty for helpful
comments.  Any errors, omissions, and other shortcomings are of course
my responsibility.

\subsubsection{Note Added June 22, 2005}\label{note2002}

The original, unpublished version of this book was written in 1997 and
distributed via the web.  The present edition has been updated to
reflect the current Metamath program and databases, as well as more
current {\sc url}s for Internet sites.  Thanks to Josh
Purinton\index{Purinton, Josh}, One Hand
Clapping, Mel L.\ O'Cat, and Roy F. Longton for pointing out
typographical and other errors.  I have also benefitted from numerous
discussions with Raph Levien\index{Levien, Raph}, who has extended
Metamath's philosophy of rigor to result in his {\em
Ghilbert}\index{Ghilbert} proof language (\url{http://ghilbert.org}).

Robert (Bob) Solovay\index{Solovay, Robert} communicated a new result of
A.~R.~D.~Mathias on the system of Bourbaki, and the text has been
updated accordingly (p.~\pageref{bourbaki}).

Bob also pointed out a clarification of the literature regarding
category theory and inaccessible cardinals\index{category
theory}\index{cardinal, inaccessible} (p.~\pageref{categoryth}),
and a misleading statement was removed from the text.  Specifically,
contrary to a statement in previous editions, it is possible to express
``There is a proper class of inaccessible cardinals'' in the language of
ZFC.  This can be done as follows:  ``For every set $x$ there is an
inaccessible cardinal $\kappa$ such that $\kappa$ is not in $x$.''
Bob writes:\footnote{Private communication, Nov.~30, 2002.}
\begin{quotation}
     This axiom is how Grothendieck presents category theory.  To each
inaccessible cardinal $\kappa$ one associates a Grothendieck universe
\index{Grothendieck, Alexander} $U(\kappa)$.  $U(\kappa)$ consists of
those sets which lie in a transitive set of cardinality less than
$\kappa$.  Instead of the ``category of all groups,'' one works relative
to a universe [considering the category of groups of cardinality less
than $\kappa$].  Now the category whose objects are all categories
``relative to the universe $U(\kappa)$'' will be a category not
relative to this universe but to the next universe.

     All of the things category theorists like to do can be done in this
framework.  The only controversial point is whether the Grothen\-dieck
axiom is too strong for the needs of category theorists.  Mac Lane
\index{Mac Lane, Saunders} argues that ``one universe is enough'' and
Feferman\index{Feferman, Solomon} has argued that one can get by with
ordinary ZFC.  I don't find Feferman's arguments persuasive.  Mac Lane
may be right, but when I think about category theory I do it \`{a} la
Grothendieck.

        By the way Mizar\index{Mizar} adds the axiom ``there is a proper
class of inaccessibles'' precisely so as to do category theory.
\end{quotation}

The most current information on the Metamath program and databases can
always be found at \url{http://metamath.org}.


\subsubsection{Note Added June 24, 2006}\label{note2006}

The Metamath spec was restricted slightly to make parsers easier to
write.  See the footnote on p.~\pageref{namespace}.

%\subsubsection{Note Added July 24, 2006}\label{note2006b}
\subsubsection{Note Added March 10, 2007}\label{note2006b}

I am grateful to Anthony Williams\index{Williams, Anthony} for writing
the \LaTeX\ package called {\tt realref.sty} and contributing it to the
public domain.  This package allows the internal hyperlinks in a {\sc
pdf} file to anchor to specific page numbers instead of just section
titles, making the navigation of the {\sc pdf} file for this book much
more pleasant and ``logical.''

A typographical error found by Martin Kiselkov was corrected.
A confusing remark about unification was deleted per suggestion of
Mel O'Cat.

\subsubsection{Note Added May 27, 2009}\label{note2009}

Several typos found by Kim Sparre were corrected.  A note was added that
the Poincar\'{e} conjecture has been proved (p.~\pageref{poincare}).

\subsubsection{Note Added Nov. 17, 2014}\label{note2014}

The statement of the Schr\"{o}der--Bernstein theorem was corrected in
Section~\ref{trust}.  Thanks to Bob Solovay for pointing out the error.

\subsubsection{Note Added May 25, 2016}\label{note2016}

Thanks to Jerry James for correcting 16 typos.

\subsubsection{Note Added February 25, 2019}\label{note201902}

David A. Wheeler\index{Wheeler, David A.}
made a large number of improvements and updates,
in coordination with Norman Megill.
The predicate calculus axioms were renumbered, and the text makes
it clear that they are based on Tarski's system S2;
the one slight deviation in axiom ax-6 is explained and justified.
The real and complex number axioms were modified to be consistent with
\texttt{set.mm}\index{set theory database (\texttt{set.mm})}%
\index{Metamath Proof Explorer}.
Long-awaited specification changes ``1--8'' were made,
which clarified previously ambiguous points.
Some errors in the text involving \texttt{\$f} and
\texttt{\$d} statements were corrected (the spec was correct, but
the in-book explanations unintentionally contradicted the spec).
We now have a system for automatically generating narrow PDFs,
so that those with smartphones can have easy access to the current
version of this document.
A new section on deduction was added;
it discusses the standard deduction theorem,
the weak deduction theorem,
deduction style, and natural deduction.
Many minor corrections (too numerous to list here) were also made.

\subsubsection{Note Added March 7, 2019}\label{note201903}

This added a description of the Matamath language syntax in
Extended Backus--Naur Form (EBNF)\index{Extended Backus--Naur Form}\index{EBNF}
in Appendix \ref{BNF}, added a brief explanation about typecodes,
inserted more examples in the deduction section,
and added a variety of smaller improvements.

\subsubsection{Note Added April 7, 2019}\label{note201904}

This version clarified the proper substitution notation, improved the
discussion on the weak deduction theorem and natural deduction,
documented the \texttt{undo} command, updated the information on
\texttt{write source}, changed the typecode
from \texttt{set} to \texttt{setvar} to be consistent with the current
version of \texttt{set.mm}, added more documentation about comment markup
(e.g., documented how to create headings), and clarified the
differences between various assertion forms (in particular deduction form).

\subsubsection{Note Added June 2, 2019}\label{note201906}

This version fixes a large number of small issues reported by
Beno\^{i}t Jubin\index{Jubin, Beno\^{i}t}, such as editorial issues
and the need to document \texttt{verify markup} (thank you!).
This version also includes specific examples
of forms (deduction form, inference form, and closed form).
We call this version the ``second edition'';
the previous edition formally published in 2007 had a slightly different title
(\textit{Metamath: A Computer Language for Pure Mathematics}).

\chapter{Introduction}
\pagenumbering{arabic}

\begin{quotation}
  {\em {\em I.M.:}  No, no.  There's nothing subjective about it!  Everybody
knows what a proof is.  Just read some books, take courses from a competent
mathematician, and you'll catch on.

{\em Student:}  Are you sure?

{\em I.M.:}  Well---it is possible that you won't, if you don't have any
aptitude for it.  That can happen, too.

{\em Student:}  Then {\em you} decide what a proof is, and if I don't learn
to decide in the same way, you decide I don't have any aptitude.

{\em I.M.:}  If not me, then who?}
    \flushright\sc  ``The Ideal Mathematician''
    \index{Davis, Phillip J.}
    \footnote{\cite{Davis}, p.~40.}\\
\end{quotation}

Brilliant mathematicians have discovered almost
unimaginably profound results that rank among the crowning intellectual
achievements of mankind.  However, there is a sense in which modern abstract
mathematics is behind the times, stuck in an era before computers existed.
While no one disputes the remarkable results that have been achieved,
communicating these results in a precise way to the uninitiated is virtually
impossible.  To describe these results, a terse informal language is used which
despite its elegance is very difficult to learn.  This informal language is not
imprecise, far from it, but rather it often has omitted detail
and symbols with hidden context that are
implicitly understood by an expert but few others.  Extremely complex technical
meanings are associated with innocent-sounding English words such as
``compact'' and ``measurable'' that barely hint at what is actually being
said.  Anyone who does not keep the precise technical meaning constantly in
mind is bound to fail, and acquiring the ability to do this can be achieved
only through much practice and hard work.  Only the few who complete the
painful learning experience can join the small in-group of pure
mathematicians.  The informal language effectively cuts off the true nature of
their knowledge from most everyone else.

Metamath\index{Metamath} makes abstract mathematics more concrete.  It allows
a computer to keep track of the complexity associated with each word or symbol
with absolute rigor.  You can explore this complexity at your leisure, to
whatever degree you desire.  Whether or not you believe that concepts such as
infinity actually ``exist'' outside of the mind, Metamath lets you get to the
foundation for what's really being said.

Metamath also enables completely rigorous and thorough proof verification.
Its language is simple enough so that you
don't have to rely on the authority of experts but can verify the results
yourself, step by step.  If you want to attempt to derive your own results,
Metamath will not let you make a mistake in reasoning.
Even professional mathematicians make mistakes; Metamath makes it possible
to thoroughly verify that proofs are correct.

Metamath\index{Metamath} is a computer language and an associated computer
program for archiving, verifying, and studying mathematical proofs at a very
detailed level.
The Metamath language
describes formal\index{formal system} mathematical
systems and expresses proofs of theorems in those systems.  Such a language
is called a metalanguage\index{metalanguage} by mathematicians.
The Metamath program is a computer program that verifies
proofs expressed in the Metamath language.
The Metamath program does not have the built-in
ability to make logical inferences; it just makes a series of symbol
substitutions according to instructions given to it in a proof
and verifies that the result matches the expected theorem.  It makes logical
inferences based only on rules of logic that are contained in a set of
axioms\index{axiom}, or first principles, that you provide to it as the
starting point for proofs.

The complete specification of the Metamath language is only four pages long
(Section~\ref{spec}, p.~\pageref{spec}).  Its simplicity may at first make you
wonder how it can do much of anything at all.  But in fact the kinds of
symbol manipulations it performs are the ones that are implicitly done in all
mathematical systems at the lowest level.  You can learn it relatively quickly
and have complete confidence in any mathematical proof that it verifies.  On
the other hand, it is powerful and general enough so that virtually any
mathematical theory, from the most basic to the deeply abstract, can be
described with it.

Although in principle Metamath can be used with any
kind of mathematics, it is best suited for abstract or ``pure'' mathematics
that is mostly concerned with theorems and their proofs, as opposed to the
kind of mathematics that deals with the practical manipulation of numbers.
Examples of branches of pure mathematics are logic\index{logic},\footnote{Logic
is the study of statements that are universally true regardless of the objects
being described by the statements.  An example is the statement, ``if $P$
implies $Q$, then either $P$ is false or $Q$ is true.''} set theory\index{set
theory},\footnote{Set theory is the study of general-purpose mathematical objects called
``sets,'' and from it essentially all of mathematics can be derived.  For
example, numbers can be defined as specific sets, and their properties
can be explored using the tools of set theory.} number theory\index{number
theory},\footnote{Number theory deals with the properties of positive and
negative integers (whole numbers).} group theory\index{group
theory},\footnote{Group theory studies the properties of mathematical objects
called groups that obey a simple set of axioms and have properties of symmetry
that make them useful in many other fields.} abstract algebra\index{abstract
algebra},\footnote{Abstract algebra includes group theory and also studies
groups with additional properties that qualify them as ``rings'' and
``fields.''  The set of real numbers is a familiar example of a field.},
analysis\index{analysis} \index{real and complex numbers}\footnote{Analysis is
the study of real and complex numbers.} and
topology\index{topology}.\footnote{One area studied by topology are properties
that remain unchanged when geometrical objects undergo stretching
deformations; for example a doughnut and a coffee cup each have one hole (the
cup's hole is in its handle) and are thus considered topologically
equivalent.  In general, though, topology is the study of abstract
mathematical objects that obey a certain (surprisingly simple) set of axioms.
See, for example, Munkres \cite{Munkres}\index{Munkres, James R.}.} Even in
physics, Metamath could be applied to certain branches that make use of
abstract mathematics, such as quantum logic\index{quantum logic} (used to study
aspects of quantum mechanics\index{quantum mechanics}).

On the other hand, Metamath\index{Metamath} is less suited to applications
that deal primarily with intensive numeric computations.  Metamath does not
have any built-in representation of numbers\index{Metamath!representation of
numbers}; instead, a specific string of symbols (digits) must be syntactically
constructed as part of any proof in which an ordinary number is used.  For
this reason, numbers in Metamath are best limited to specific constants that
arise during the course of a theorem or its proof.  Numbers are only a tiny
part of the world of abstract mathematics.  The exclusion of built-in numbers
was a conscious decision to help achieve Metamath's simplicity, and there are
other software tools if you have different mathematical needs.
If you wish to quickly solve algebraic problems, the computer algebra
programs\index{computer algebra system} {\sc
macsyma}\index{macsyma@{\sc macsyma}}, Mathematica\index{Mathematica}, and
Maple\index{Maple} are specifically suited to handling numbers and
algebra efficiently.
If you wish to simply calculate numeric or matrix expressions easily,
tools such as Octave\index{Octave} may be a better choice.

After learning Metamath's basic statement types, any
tech\-ni\-cal\-ly ori\-ent\-ed person, mathematician or not, can
immediately trace
any theorem proved in the language as far back as he or she wants, all the way
to the axioms on which the theorem is based.  This ability suggests a
non-traditional way of learning about pure mathematics.  Used in conjunction
with traditional methods, Metamath could make pure mathematics accessible to
people who are not sufficiently skilled to figure out the implicit detail in
ordinary textbook proofs.  Once you learn the axioms of a theory, you can have
complete confidence that everything you need to understand a proof you are
studying is all there, at your beck and call, allowing you to focus in on any
proof step you don't understand in as much depth as you need, without worrying
about getting stuck on a step you can't figure out.\footnote{On the other
hand, writing proofs in the Metamath language is challenging, requiring
a degree of rigor far in excess of that normally taught to students.  In a
classroom setting, I doubt that writing Metamath proofs would ever replace
traditional homework exercises involving informal proofs, because the time
needed to work out the details would not allow a course to
cover much material.  For students who have trouble grasping the implied rigor
in traditional material, writing a few simple proofs in the Metamath language
might help clarify fuzzy thought processes.  Although somewhat difficult at
first, it eventually becomes fun to do, like solving a puzzle, because of the
instant feedback provided by the computer.}

Metamath\index{Metamath} is probably unlike anything you have
encountered before.  In this first chapter we will look at the philosophy and
use of computers in mathematics in order to better understand the motivation
behind Metamath.  The material in this chapter is not required in order to use
Metamath.  You may skip it if you are impatient, but I hope you will find it
educational and enjoyable.  If you want to start experimenting with the
Metamath program right away, proceed directly to Chapter~\ref{using}
(p.~\pageref{using}).  To
learn the Metamath language, skim Chapter~\ref{using} then proceed to
Chapter~\ref{languagespec} (p.~\pageref{languagespec}).

\section{Mathematics as a Computer Language}

\begin{quote}
  {\em The study of mathematics is apt to commence in
dis\-ap\-point\-ment.\ldots \\
We are told that by its aid the stars are weighted
and the billions of molecules in a drop of water are counted.  Yet, like the
ghost of Hamlet's father, this great science eludes the efforts of our mental
weapons to grasp it.}
  \flushright\sc  Alfred North Whitehead\footnote{\cite{Whitehead}, ch.\ 1.}\\
\end{quote}\index{Whitehead, Alfred North}

\subsection{Is Mathematics ``User-Friendly''?}

Suppose you have no formal training in abstract mathematics.  But popular
books you've read offer tempting glimpses of this world filled with profound
ideas that have stirred the human spirit.  You are not satisfied with the
informal, watered-down descriptions you've read but feel it is important to
grasp the underlying mathematics itself to understand its true meaning. It's
not practical to go back to school to learn it, though; you don't want to
dedicate years of your life to it.  There are many important things in life,
and you have to set priorities for what's important to you.  What would happen
if you tried to pursue it on your own, in your spare time?

After all, you were able to learn a computer programming language such as
Pascal on your own without too much difficulty, even though you had no formal
training in computers.  You don't claim to be an expert in software design,
but you can write a passable program when necessary to suit your needs.  Even
more important, you know that you can look at anyone else's Pascal program, no
matter how complex, and with enough patience figure out exactly how it works,
even though you are not a specialist.  Pascal allows you do anything that a
computer can do, at least in principle.  Thus you know you have the ability,
in principle, to follow anything that a computer program can do:  you just
have to break it down into small enough pieces.

Here's an imaginary scenario of what might happen if you na\-ive\-ly a\-dopted
this same view of abstract mathematics and tried to pick it up on your own, in
a period of time comparable to, say, learning a computer programming
language.

\subsubsection{A Non-Mathematician's Quest for Truth}

\begin{quote}
  {\em \ldots my daughters have been studying (chemistry) for several
se\-mes\-ters, think they have learned differential and integral calculus in
school, and yet even today don't know why $x\cdot y=y\cdot x$ is true.}
  \flushright\sc  Edmund Landau\footnote{\cite{Landau}, p.~vi.}\\
\end{quote}\index{Landau, Edmund}

\begin{quote}
  {\em Minus times minus is plus,\\
The reason for this we need not discuss.}
  \flushright\sc W.\ H.\ Auden\footnote{As quoted in \cite{Guillen}, p.~64.}\\
\end{quote}\index{Auden, W.\ H.}\index{Guillen, Michael}

We'll suppose you are a technically oriented professional, perhaps an engineer, a
computer programmer, or a physicist, but probably not a mathematician.  You
consider yourself reasonably intelligent.  You did well in school, learning a
variety of methods and techniques in practical mathematics such as calculus and
differential equations.  But rarely did your courses get into anything
resembling modern abstract mathematics, and proofs were something that appeared
only occasionally in your textbooks, a kind of necessary evil that was
supposed to convince you of a certain key result.  Most of your
homework consisted of exercises that gave you practice in the techniques, and
you were hardly ever asked to come up with a proof of your own.

You find yourself curious about advanced, abstract mathematics.  You are
driven by an inner conviction that it is important to understand and
appreciate some of the most profound knowledge discovered by mankind.  But it
seems very hard to learn, something that only certain gifted longhairs can
access and understand.  You are frustrated that it seems forever cut off from
you.

Eventually your curiosity drives you to do something about it.
You set for yourself a goal of ``really'' understanding mathematics:  not just
how to manipulate equations in algebra or calculus according to cookbook
rules, but rather to gain a deep understanding of where those rules come from.
In fact, you're not thinking about this kind of ordinary mathematics at all,
but about a much more abstract, ethereal realm of pure mathematics, where
famous results such as G\"{o}del's incompleteness theorem\index{G\"{o}del's
incompleteness theorem} and Cantor's different kinds of infinities
reside.

You have probably read a number of popular books, with titles like {\em
Infinity and the Mind} \cite{Rucker}\index{Rucker, Rudy}, on topics such as
these.  You found them inspiring but at the same time somewhat
unsatisfactory.  They gave you a general idea of what these results are about,
but if someone asked you to prove them, you wouldn't have the faintest idea of
where to begin.   Sure, you could give the same overall outline that you
learned from the popular books; and in a general sort of way, you do have an
understanding.  But deep down inside, you know that there is a rigor that is
missing, that probably there are many subtle steps and pitfalls along the way,
and ultimately it seems you have to place your trust in the experts in the
field.  You don't like this; you want to be able to verify these results for
yourself.

So where do you go next?  As a first step, you decide to look up some of the
original papers on the theorems you are curious about, or better, obtain some
standard textbooks in the field.  You look up a theorem you want to
understand.  Sure enough, it's there, but it's expressed with strange
terms and odd symbols that mean absolutely nothing to you.  It might as well be written in
a foreign language you've never seen before, whose symbols are totally alien.
You look at the proof, and you haven't the foggiest notion what each step
means, much less how one step follows from another.  Well, obviously you have
a lot to learn if you want to understand this stuff.

You feel that you could probably understand it by
going back to college for another three to six years and getting a math
degree.  But that does not fit in with your career and the other things in
your life and would serve no practical purpose.  You decide to seek a quicker
path.  You figure you'll just trace your way back to the beginning, step by
step, as you would do with a computer program, until you understand it.  But
you quickly find that this is not possible, since you can't even understand
enough to know what you have to trace back to.

Maybe a different approach is in order---maybe you should start at the
beginning and work your way up.  First, you read the introduction to the book
to find out what the prerequisites are.  In a similar fashion, you trace your
way back through two or three more books, finally arriving at one that seems
to start at a beginning:  it lists the axioms of arithmetic.  ``Aha!'' you
naively think, ``This must be the starting point, the source of all mathematical
knowledge.'' Or at least the starting point for mathematics dealing with
numbers; you have to start somewhere and have no idea what the starting point
for other mathematics would be.  But the word ``axioms'' looks promising.  So
you eagerly read along and work through some elementary exercises at the
beginning of the book.  You feel vaguely bothered:  these
don't seem like axioms at all, at least not in the sense that you want to
think of axioms.  Axioms imply a starting point from which everything else can
be built up, according to precise rules specified in the axiom system.  Even
though you can understand the first few proofs in an informal way,
and are able to do some of the
exercises, it's hard to pin down precisely what the
rules are.   Sure, each step seems to follow logically from the others, but
exactly what does that mean?  Is the ``logic'' just a matter of common sense,
something vague that we all understand but can never quite state precisely?

You've spent a number of years, off and on, programming computers, and you
know that in the case of computer languages there is no question of what the
rules are---they are precise and crystal clear.  If you follow them, your
program will work, and if you don't, it won't.  No matter how complex a
program, it can always be broken down into simpler and simpler pieces, until
you can ultimately identify the bits that are moved around to perform a
specific function.  Some programs might require a lot of perseverance to
accomplish this, but if you focus on a specific portion of it, you don't even
necessarily have to know how the rest of it works. Shouldn't there be an
analogy in mathematics?

You decide to apply the ultimate test:  you ask yourself how a computer could
verify or ensure that the steps in these proofs follow from one another.
Certainly mathematics must be at least as precisely defined as a computer
language, if not more so; after all, computer science itself is based on it.
If you can get a computer to verify these proofs, then you should also be
able, in principle, to understand them yourself in a very crystal clear,
precise way.

You're in for a surprise:  you can conceive of no way to convert the
proofs, which are in English, to a form that the computer can understand.
The proofs are filled with phrases such as ``assume there exists a unique
$x$\ldots'' and ``given any $y$, let $z$ be the number such that\ldots''  This
isn't the kind of logic you are used to in computer programming, where
everything, even arithmetic, reduces to Boolean ones and zeroes if you care to
break it down sufficiently.  Even though you think you understand the proofs,
there seems to be some kind of higher reasoning involved rather than precise
rules that define how you manipulate the symbols in the axioms.  Whatever it
is, it just isn't obvious how you would express it to a computer, and the more
you think about it, the more puzzled and confused you get, to the point where
you even wonder whether {\em you} really understand it.  There's a lot more to
these axioms of arithmetic than meets the eye.

Nobody ever talked about this in school in your applied math and engineering
courses.  You just learned the rules they gave you, not quite understanding
how or why they worked, sometimes vaguely suspicious or uncertain of them, and
through homework problems and osmosis learned how to present solutions that
satisfied the instructor and earned you an ``A.''  Rarely did you actually
``prove'' anything in a rigorous way, and the math majors who did do stuff
like that seemed to be in a different world.

Of course, there are computer algebra programs that can do mathematics, and
rather impressively.  They can instantly solve the integrals that you
struggled with in freshman calculus, and do much, much more.  But when you
look at these programs, what you see is a big collection of algorithms and
techniques that evolved and were added to over time, along with some basic
software that manipulates symbols.  Each algorithm that is built in is the
result of someone's theorem whose proof is omitted; you just have to trust the
person who proved it and the person who programmed it in and hope there are no
bugs.\index{computer program bugs}  Somehow this doesn't seem to be the
essence of mathematics.  Although computer algebra systems can generate
theorems with amazing speed, they can't actually prove a single one of them.

After some puzzlement, you revisit some popular books on what mathematics is
all about.  Somewhere you read that all of mathematics is actually derived
from something called ``set theory.''  This is a little confusing, because
nowhere in the book that presented the axioms of arithmetic was there any
mention of set theory, or if there was, it seemed to be just a tool that helps
you describe things better---the set of even numbers, that sort of thing.  If
set theory is the basis for all mathematics, then why are additional axioms
needed for arithmetic?

Something is wrong but you're not sure what.  One of your friends is a pure
mathematician.  He knows he is unable to communicate to you what he does for a
living and seems to have little interest in trying.  You do know that for him,
proofs are what mathematics is all about. You ask him what a proof is, and he
essentially tells you that, while of course it's based on logic, really it's
something you learn by doing it over and over until you pick it up.  He refers
you to a book, {\em How to Read and Do Proofs} \cite{Solow}.\index{Solow,
Daniel}  Although this book helps you understand traditional informal proofs,
there is still something missing you can't seem to pin down yet.

You ask your friend how you would go about having a computer verify a proof.
At first he seems puzzled by the question; why would you want to do that?
Then he says it's not something that would make any sense to do, but he's
heard that you'd have to break the proof down into thousands or even millions
of individual steps to do such a thing, because the reasoning involved is at
such a high level of abstraction.  He says that maybe it's something you could
do up to a point, but the computer would be completely impractical once you
get into any meaningful mathematics.  There, the only way you can verify a
proof is by hand, and you can only acquire the ability to do this by
specializing in the field for a couple of years in grad school.  Anyway, he
thinks it all has to do with set theory, although he has never taken a formal
course in set theory but just learned what he needed as he went along.

You are intrigued and amazed.  Apparently a mathematician can grasp as a
single concept something that would take a computer a thousand or a million
steps to verify, and have complete confidence in it.  Each one of these
thousand or million steps must be absolutely correct, or else the whole proof
is meaningless.  If you added a million numbers by hand, would you trust the
result?  How do you really know that all these steps are correct, that there
isn't some subtle pitfall in one of these million steps, like a bug in a
computer program?\index{computer program bugs}  After all, you've read that
famous mathematicians have occasionally made mistakes, and you certainly know
you've made your share on your math homework problems in school.

You recall the analogy with a computer program.  Sure, you can understand what
a large computer program such as a word processor does, as a single high-level
concept or a small set of such concepts, but your ability to understand it in
no way ensures that the program is correct and doesn't have hidden bugs.  Even
if you wrote the program yourself you can't really know this; most large
programs that you've written have had bugs that crop up at some later date, no
matter how careful you tried to be while writing them.

OK, so now it seems the reason you can't figure out how to make a
computer verify proofs is because each step really corresponds to a
million small steps.  Well, you say, a computer can do a million
calculations in a second, so maybe it's still practical to do.  Now the
puzzle becomes how to figure out what the million steps are that each
English-language step corresponds to.  Your mathematician friend hasn't
a clue, but suggests that maybe you would find the answer by studying
set theory.  Actually, your friend thinks you're a little off the wall
for even wondering such a thing.  For him, this is not what mathematics
is all about.

The subject of set theory keeps popping up, so you decide it's
time to look it up.

You decide to start off on a careful footing, so you start reading a couple of
very elementary books on set theory.  A lot of it seems pretty obvious, like
intersections, subsets, and Venn diagrams.  You thumb through one of the
books; nowhere is anything about axioms mentioned. The other book relegates to
an appendix a brief discussion that mentions a set of axioms called
``Zermelo--Fraenkel set theory''\index{Zermelo--Fraenkel set theory} and states
them in English.  You look at them and have no idea what they really mean or
what you can do with them.  The comments in this appendix say that the purpose
of mentioning them is to expose you to the idea, but imply that they are not
necessary for basic understanding and that they are really the subject matter
of advanced treatments where fine points such as a certain paradox (Russell's
paradox\index{Russell's paradox}\footnote{Russell's paradox assumes that there
exists a set $S$ that is a collection of all sets that don't contain
themselves.  Now, either $S$ contains itself or it doesn't.  If it contains
itself, it contradicts its definition.  But if it doesn't contain itself, it
also contradicts its definition.  Russell's paradox is resolved in ZF set
theory by denying that such a set $S$ exists.}) are resolved.  Wait a
minute---shouldn't the axioms be a starting point, not an ending point?  If
there are paradoxes that arise without the axioms, how do you know you won't
stumble across one accidentally when using the informal approach?

And nowhere do these books describe how ``all of mathematics can be
derived from set theory'' which by now you've heard a few times.

You find a more advanced book on set theory.  This one actually lists the
axioms of ZF set theory in plain English on page one.  {\em Now} you think
your quest has ended and you've finally found the source of all mathematical
knowledge; you just have to understand what it means.  Here, in one place, is
the basis for all of mathematics!  You stare at the axioms in awe, puzzle over
them, memorize them, hoping that if you just meditate on them long enough they
will become clear.  Of course, you haven't the slightest idea how the rest of
mathematics is ``derived'' from them; in particular, if these are the axioms
of mathematics, then why do arithmetic, group theory, and so on need their own
axioms?

You start reading this advanced book carefully, pondering the meaning of every
word, because by now you're really determined to get to the bottom of this.
The first thing the book does is explain how the axioms came about, which was
to resolve Russell's paradox.\index{Russell's paradox}  In fact that seems to
be the main purpose of their existence; that they supposedly can be used to
derive all of mathematics seems irrelevant and is not even mentioned.  Well,
you go on.  You hope the book will explain to you clearly, step by step, how
to derive things from the axioms.  After all, this is the starting point of
mathematics, like a book that explains the basics of a computer programming
language.  But something is missing.  You find you can't even understand the
first proof or do the first exercise.  Symbols such as $\exists$ and $\forall$
permeate the page without any mention of where they came from or how to
manipulate them; the author assumes you are totally familiar with them and
doesn't even tell you what they mean.  By now you know that $\exists$ means
``there exists'' and $\forall$ means ``for all,'' but shouldn't the rules for
manipulating these symbols be part of the axioms?  You still have no idea
how you could even describe the axioms to a computer.

Certainly there is something much different here from the technical
literature you're used to reading.  A computer language manual almost
always explains very clearly what all the symbols mean, precisely what
they do, and the rules used for combining them, and you work your way up
from there.

After glancing at four or five other such books, you come to the realization
that there is another whole field of study that you need just to get to the
point at which you can understand the axioms of set theory.  The field is
called ``logic.''  In fact, some of the books did recommend it as a
prerequisite, but it just didn't sink in.  You assumed logic was, well, just
logic, something that a person with common sense intuitively understood.  Why
waste your time reading boring treatises on symbolic logic, the manipulation
of 1's and 0's that computers do, when you already know that?  But this is a
different kind of logic, quite alien to you.  The subject of {\sc nand} and
{\sc nor} gates is not even touched upon or in any case has to do with only a
very small part of this field.

So your quest continues.  Skimming through the first couple of introductory
books, you get a general idea of what logic is about and what quantifiers
(``for all,'' ``there exists'') mean, but you find their examples somewhat
trivial and mildly annoying (``all dogs are animals,'' ``some animals are
dogs,'' and such).  But all you want to know is what the rules are for
manipulating the symbols so you can apply them to set theory.  Some formulas
describing the relationships among quantifiers ($\exists$ and $\forall$) are
listed in tables, along with some verbal reasoning to justify them.
Presumably, if you want to find out if a formula is correct, you go through
this same kind of mental reasoning process, possibly using images of dogs and
animals. Intuitively, the formulas seem to make sense.  But when you ask
yourself, ``What are the rules I need to get a computer to figure out whether
this formula is correct?'', you still don't know.  Certainly you don't ask the
computer to imagine dogs and animals.

You look at some more advanced logic books.  Many of them have an introductory
chapter summarizing set theory, which turns out to be a prerequisite.  You
need logic to understand set theory, but it seems you also need set theory to
understand logic!  These books jump right into proving rather advanced
theorems about logic, without offering the faintest clue about where the logic
came from that allows them to prove these theorems.

Luckily, you come across an elementary book of logic that, halfway through,
after the usual truth tables and metaphors, presents in a clear, precise way
what you've been looking for all along: the axioms!  They're divided into
propositional calculus (also called sentential logic) and predicate calculus
(also called first-order logic),\index{first-order logic} with rules so simple
and crystal clear that now you can finally program a computer to understand
them.  Indeed, they're no harder than learning how to play a game of chess.
As far as what you seem to need is concerned, the whole book could have been
written in five pages!

{\em Now} you think you've found the ultimate source of mathematical
truth.  So---the axioms of mathematics consist of these axioms of logic,
together with the axioms of ZF set theory. (By now you've also been able to
figure out how to translate the ZF axioms from English into the
actual symbols of logic which you can now manipulate according to
precise, easy-to-understand rules.)

Of course, you still don't understand how ``all of mathematics can be
derived from set theory,'' but maybe this will reveal itself in due
course.

You eagerly set out to program the axioms and rules into a computer and start
to look at the theorems you will have to prove as the logic is developed.  All
sorts of important theorems start popping up:  the deduction
theorem,\index{deduction theorem} the substitution theorem,\index{substitution
theorem} the completeness theorem of propositional calculus,\index{first-order
logic!completeness} the completeness theorem of predicate calculus.  Uh-oh,
there seems to be trouble.  They all get harder and harder, and not one of
them can be derived with the axioms and rules of logic you've just been
handed.  Instead, they all require ``metalogic'' for their proofs, a kind of
mixture of logic and set theory that allows you to prove things {\em about}
the axioms and theorems of logic rather than {\em with} them.

You plow ahead anyway.  A month later, you've spent much of your
free time getting the computer to verify proofs in propositional calculus.
You've programmed in the axioms, but you've also had to program in the
deduction theorem, the substitution theorem, and the completeness theorem of
propositional calculus, which by now you've resigned yourself to treating as
rather complex additional axioms, since they can't be proved from the axioms
you were given.  You can now get the computer to verify and even generate
complete, rigorous, formal proofs\index{formal proof}.  Never mind that they
may have 100,000 steps---at least now you can have complete, absolute
confidence in them.  Unfortunately, the only theorems you have proved are
pretty trivial and you can easily verify them in a few minutes with truth
tables, if not by inspection.

It looks like your mathematician friend was right.  Getting the computer to do
serious mathematics with this kind of rigor seems almost hopeless.  Even
worse, it seems that the further along you get, the more ``axioms'' you have
to add, as each new theorem seems to involve additional ``metamathematical''
reasoning that hasn't been formalized, and none of it can be derived from the
axioms of logic.  Not only do the proofs keep growing exponentially as you get
further along, but the program to verify them keeps getting bigger and bigger
as you program in more ``metatheorems.''\index{metatheorem}\footnote{A
metatheorem is usually a statement that is too general to be directly provable
in a theory.  For example, ``if $n_1$, $n_2$, and $n_3$ are integers, then
$n_1+n_2+n_3$ is an integer'' is a theorem of number theory.  But ``for any
integer $k > 1$, if $n_1, \ldots, n_k$ are integers, then $n_1+\ldots +n_k$ is
an integer'' is a metatheorem, in other words a family of theorems, one for
every $k$.  The reason it is not a theorem is that the general sum $n_1+\ldots
+n_k$ (as a function of $k$) is not an operation that can be defined directly
in number theory.} The bugs\index{computer program bugs} that have cropped up
so far have already made you start to lose faith in the rigor you seem to have
achieved, and you know it's just going to get worse as your program gets larger.

\subsection{Mathematics and the Non-Specialist}

\begin{quote}
  {\em A real proof is not checkable by a machine, or even by any mathematician
not privy to the gestalt, the mode of thought of the particular field of
mathematics in which the proof is located.}
  \flushright\sc  Davis and Hersh\index{Davis, Phillip J.}
  \footnote{\cite{Davis}, p.~354.}\\
\end{quote}

The bulk of abstract or theoretical mathematics is ordinarily outside
the reach of anyone but a few specialists in each field who have completed
the necessary difficult internship in order to enter its coterie.  The
typical intelligent layperson has no reasonable hope of understanding much of
it, nor even the specialist mathematician of understanding other fields.  It
is like a foreign language that has no dictionary to look up the translation;
the only way you can learn it is by living in the country for a few years.  It
is argued that the effort involved in learning a specialty is a necessary
process for acquiring a deep understanding.  Of course, this is almost certainly
true if one is to make significant contributions to a field; in particular,
``doing'' proofs is probably the most important part of a mathematician's
training.  But is it also necessary to deny outsiders access to it?  Is it
necessary that abstract mathematics be so hard for a layperson to grasp?

A computer normally is of no help whatsoever.  Most published proofs are
actually just series of hints written in an informal style that requires
considerable knowledge of the field to understand.  These are the ``real
proofs'' referred to by Davis and Hersh.\index{informal proof}  There is an
implicit understanding that, in principle, such a proof could be converted to
a complete formal proof\index{formal proof}.  However, it is said that no one
would ever attempt such a conversion, even if they could, because that would
presumably require millions of steps (Section~\ref{dream}).  Unfortunately the
informal style automatically excludes the understanding of the proof
by anyone who hasn't gone through the necessary apprenticeship. The
best that the intelligent layperson can do is to read popular books about deep
and famous results; while this can be helpful, it can also be misleading, and
the lack of detail usually leaves the reader with no ability whatsoever to
explore any aspect of the field being described.

The statements of theorems often use sophisticated notation that makes them
inaccessible to the non-specialist.  For a non-specialist who wants to achieve
a deeper understanding of a proof, the process of tracing definitions and
lemmas back through their hierarchy\index{hierarchy} quickly becomes confusing
and discouraging.  Textbooks are usually written to train mathematicians or to
communicate to people who are already mathematicians, and large gaps in proofs
are often left as exercises to the reader who is left at an impasse if he or
she becomes stuck.

I believe that eventually computers will enable non-specialists and even
intelligent laypersons to follow almost any mathematical proof in any field.
Metamath is an attempt in that direction.  If all of mathematics were as
easily accessible as a computer programming language, I could envision
computer programmers and hobbyists who otherwise lack mathematical
sophistication exploring and being amazed by the world of theorems and proofs
in obscure specialties, perhaps even coming up with results of their own.  A
tremendous advantage would be that anyone could experiment with conjectures in
any field---the computer would offer instant feedback as to whether
an inference step was correct.

Mathematicians sometimes have to put up with the annoyance of
cranks\index{cranks} who lack a fundamental understanding of mathematics but
insist that their ``proofs'' of, say, Fermat's Last Theorem\index{Fermat's
Last Theorem} be taken seriously.  I think part of the problem is that these
people are misled by informal mathematical language, treating it as if they
were reading ordinary expository English and failing to appreciate the
implicit underlying rigor.  Such cranks are rare in the field of computers,
because computer languages are much more explicit, and ultimately the proof is
in whether a computer program works or not.  With easily accessible
computer-based abstract mathematics, a mathematician could say to a crank,
``don't bother me until you've demonstrated your claim on the computer!''

% 22-May-04 nm
% Attempt to move De Millo quote so it doesn't separate from attribution
% CHANGE THIS NUMBER (AND ELIMINATE IF POSSIBLE) WHEN ABOVE TEXT CHANGES
\vspace{-0.5em}

\subsection{An Impossible Dream?}\label{dream}

\begin{quote}
  {\em Even quite basic theorems would demand almost unbelievably vast
  books to display their proofs.}
    \flushright\sc  Robert E. Edwards\footnote{\cite{Edwards}, p.~68.}\\
\end{quote}\index{Edwards, Robert E.}

\begin{quote}
  {\em Oh, of course no one ever really {\em does} it.  It would take
  forever!  You just show that you could do it, that's sufficient.}
    \flushright\sc  ``The Ideal Mathematician''
    \index{Davis, Phillip J.}\footnote{\cite{Davis},
p.~40.}\\
\end{quote}

\begin{quote}
  {\em There is a theorem in the primitive notation of set theory that
  corresponds to the arithmetic theorem `$1000+2000=3000$'.  The formula
  would be forbiddingly long\ldots even if [one] knows the definitions
  and is asked to simplify the long formula according to them, chances are
  he will make errors and arrive at some incorrect result.}
    \flushright\sc  Hao Wang\footnote{\cite{Wang}, p.~140.}\\
\end{quote}\index{Wang, Hao}

% 22-May-04 nm
% Attempt to move De Millo quote so it doesn't separate from attribution
% CHANGE THIS NUMBER (AND ELIMINATE IF POSSIBLE) WHEN ABOVE TEXT CHANGES
\vspace{-0.5em}

\begin{quote}
  {\em The {\em Principia Mathematica} was the crowning achievement of the
  formalists.  It was also the deathblow of the formalist view.\ldots
  {[Rus\-sell]} failed, in three enormous volumes, to get beyond the elementary
  facts of arithmetic.  He showed what can be done in principle and what
  cannot be done in practice.  If the mathematical process were really
  one of strict, logical progression, we would still be counting our
  fingers.\ldots
  One theoretician estimates\ldots that a demonstration of one of
  Ramanujan's conjectures assuming set theory and elementary analysis would
  take about two thousand pages; the length of a deduction from first principles
  is nearly in\-con\-ceiv\-a\-ble\ldots The probabilists argue that\ldots any
  very long proof can at best be viewed as only probably correct\ldots}
  \flushright\sc Richard de Millo et. al.\footnote{\cite{deMillo}, pp.~269,
  271.}\\
\end{quote}\index{de Millo, Richard}

A number of writers have conveyed the impression that the kind of absolute
rigor provided by Metamath\index{Metamath} is an impossible dream, suggesting
that a complete, formal verification\index{formal proof} of a typical theorem
would take millions of steps in untold volumes of books.  Even if it could be
done, the thinking sometimes goes, all meaning would be lost in such a
monstrous, tedious verification.\index{informal proof}\index{proof length}

These writers assume, however, that in order to achieve the kind of complete
formal verification they desire one must break down a proof into individual
primitive steps that make direct reference to the axioms.  This is
not necessary.  There is no reason not to make use of previously proved
theorems rather than proving them over and over.

Just as important, definitions\index{definition} can be introduced along
the way, allowing very complex formulas to be represented with few
symbols.  Not doing this can lead to absurdly long formulas.  For
example, the mere statement of
G\"{o}del's incompleteness theorem\index{G\"{o}del's
incompleteness theorem}, which can be expressed with a small number of
defined symbols, would require about 20,000 primitive symbols to express
it.\index{Boolos, George S.}\footnote{George S.\ Boolos, lecture at
Massachusetts Institute of Technology, spring 1990.} An extreme example
is Bourbaki's\label{bourbaki} language for set theory, which requires
4,523,659,424,929 symbols plus 1,179,618,517,981 disambiguatory links
(lines connecting symbol pairs, usually drawn below or above the
formula) to express the number
``one'' \cite{Mathias}.\index{Mathias, Adrian R. D.}\index{Bourbaki,
Nicolas}
% http://www.dpmms.cam.ac.uk/~ardm/

A hierarchy\index{hierarchy} of theorems and definitions permits an
exponential growth in the formula sizes and primitive proof steps to be
described with only a linear growth in the number of symbols used.  Of course,
this is how ordinary informal mathematics is normally done anyway, but with
Metamath\index{Metamath} it can be done with absolute rigor and precision.

\subsection{Beauty}


\begin{quote}
  {\em No one shall be able to drive us from the paradise that Cantor has
created for us.}
   \flushright\sc  David Hilbert\footnote{As quoted in \cite{Moore}, p.~131.}\\
\end{quote}\index{Hilbert, David}

\needspace{3\baselineskip}
\begin{quote}
  {\em Mathematics possesses not only truth, but some supreme beauty ---a
  beauty cold and austere, like that of a sculpture.}
    \flushright\sc  Bertrand
    Russell\footnote{\cite{Russell}.}\\
\end{quote}\index{Russell, Bertrand}

\begin{quote}
  {\em Euclid alone has looked on Beauty bare.}
  \flushright\sc Edna Millay\footnote{As quoted in \cite{Davis}, p.~150.}\\
\end{quote}\index{Millay, Edna}

For most people, abstract mathematics is distant, strange, and
incomprehensible.  Many popular books have tried to convey some of the sense
of beauty in famous theorems.  But even an intelligent layperson is left with
only a general idea of what a theorem is about and is hardly given the tools
needed to make use of it.  Traditionally, it is only after years of arduous
study that one can grasp the concepts needed for deep understanding.
Metamath\index{Metamath} allows you to approach the proof of the theorem from
a quite different perspective, peeling apart the formulas and definitions
layer by layer until an entirely different kind of understanding is achieved.
Every step of the proof is there, pieced together with absolute precision and
instantly available for inspection through a microscope with a magnification
as powerful as you desire.

A proof in itself can be considered an object of beauty.  Constructing an
elegant proof is an art.  Once a famous theorem has been proved, often
considerable effort is made to find simpler and more easily understood
proofs.  Creating and communicating elegant proofs is a major concern of
mathematicians.  Metamath is one way of providing a common language for
archiving and preserving this information.

The length of a proof can, to a certain extent, be considered an
objective measure of its ``beauty,'' since shorter proofs are usually
considered more elegant.  In the set theory database
\texttt{set.mm}\index{set theory database (\texttt{set.mm})}%
\index{Metamath Proof Explorer}
provided with Metamath, one goal was to make all proofs as short as possible.

\needspace{4\baselineskip}
\subsection{Simplicity}

\begin{quote}
  {\em God made man simple; man's complex problems are of his own
  devising.}
    \flushright\sc Eccles. 7:29\footnote{Jerusalem Bible.}\\
\end{quote}\index{Bible}

\needspace{3\baselineskip}
\begin{quote}
  {\em God made integers, all else is the work of man.}
    \flushright\sc Leopold Kronecker\footnote{{\em Jahresbericht
	der Deutschen Mathematiker-Vereinigung }, vol. 2, p. 19.}\\
\end{quote}\index{Kronecker, Leopold}

\needspace{3\baselineskip}
\begin{quote}
  {\em For what is clear and easily comprehended attracts; the
  complicated repels.}
    \flushright\sc David Hilbert\footnote{As quoted in \cite{deMillo},
p.~273.}\\
\end{quote}\index{Hilbert, David}

The Metamath\index{Metamath} language is simple and Spartan.  Metamath treats
all mathematical expressions as simple sequences of symbols, devoid of meaning.
The higher-level or ``metamathematical'' notions underlying Metamath are about
as simple as they could possibly be.  Each individual step in a proof involves
a single basic concept, the substitution of an expression for a variable, so
that in principle almost anyone, whether mathematician or not, can
completely understand how it was arrived at.

In one of its most basic applications, Metamath\index{Metamath} can be used to
develop the foundations of mathematics\index{foundations of mathematics} from
the very beginning.  This is done in the set theory database that is provided
with the Metamath package and is the subject matter
of Chapter~\ref{fol}. Any language (a metalanguage\index{metalanguage})
used to describe mathematics (an object language\index{object language}) must
have a mathematical content of its own, but it is desirable to keep this
content down to a bare minimum, namely that needed to make use of the
inference rules specified by the axioms.  With any metalanguage there is a
``chicken and egg'' problem somewhat like circular reasoning:  you must assume
the validity of the mathematics of the metalanguage in order to prove the
validity of the mathematics of the object language.  The mathematical content
of Metamath itself is quite limited.  Like the rules of a game of chess, the
essential concepts are simple enough so that virtually anyone should be able to
understand them (although that in itself will not let you play like
a master).  The symbols that Metamath manipulates do not in themselves
have any intrinsic meaning.  Your interpretation of the axioms that you supply
to Metamath is what gives them meaning.  Metamath is an attempt to strip down
mathematical thought to its bare essence and show you exactly how the symbols
are manipulated.

Philosophers and logicians, with various motivations, have often thought it
important to study ``weak'' fragments of logic\index{weak logic}
\cite{Anderson}\index{Anderson, Alan Ross} \cite{MegillBunder}\index{Megill,
Norman}\index{Bunder, Martin}, other unconventional systems of logic (such as
``modal'' logic\index{modal logic} \cite[ch.\ 27]{Boolos}\index{Boolos, George
S.}), and quantum logic\index{quantum logic} in physics
\cite{Pavicic}\index{Pavi{\v{c}}i{\'{c}}, M.}.  Metamath\index{Metamath}
provides a framework in which such systems can be expressed, with an absolute
precision that makes all underlying metamathematical assumptions rigorous and
crystal clear.

Some schools of philosophical thought, for example
intuitionism\index{intuitionism} and constructivism\index{constructivism},
demand that the notions underlying any mathematical system be as simple and
concrete as possible.  Metamath should meet the requirements of these
philosophies.  Metamath must be taught the symbols, axioms\index{axiom}, and
rules\index{rule} for a specific theory, from the skeptical (such as
intuitionism\index{intuitionism}\footnote{Intuitionism does not accept the law
of excluded middle (``either something is true or it is not true'').  See
\cite[p.~xi]{Tymoczko}\index{Tymoczko, Thomas} for discussion and references
on this topic.  Consider the theorem, ``There exist irrational numbers $a$ and
$b$ such that $a^b$ is rational.''  An intuitionist would reject the following
proof:  If $\sqrt{2}^{\sqrt{2}}$ is rational, we are done.  Otherwise, let
$a=\sqrt{2}^{\sqrt{2}}$ and $b=\sqrt{2}$. Then $a^b=2$, which is rational.})
to the bold (such as the axiom of choice in set theory\footnote{The axiom of
choice\index{Axiom of Choice} asserts that given any collection of pairwise
disjoint nonempty sets, there exists a set that has exactly one element in
common with each set of the collection.  It is used to prove many important
theorems in standard mathematics.  Some philosophers object to it because it
asserts the existence of a set without specifying what the set contains
\cite[p.~154]{Enderton}\index{Enderton, Herbert B.}.  In one foundation for
mathematics due to Quine\index{Quine, Willard Van Orman}, that has not been
otherwise shown to be inconsistent, the axiom of choice turns out to be false
\cite[p.~23]{Curry}\index{Curry, Haskell B.}.  The \texttt{show
trace{\char`\_}back} command of the Metamath program allows you to find out
whether the axiom of choice, or any other axiom, was assumed by a
proof.}\index{\texttt{show trace{\char`\_}back} command}).

The simplicity of the Metamath language lets the algorithm (computer program)
that verifies the validity of a Metamath proof be straightforward and
robust.  You can have confidence that the theorems it verifies really can be
derived from your axioms.

\subsection{Rigor}

\begin{quote}
  {\em Rigor became a goal with the Greeks\ldots But the efforts to
  pursue rigor to the utmost have led to an impasse in which there is
  no longer any agreement on what it really means.  Mathematics remains
  alive and vital, but only on a pragmatic basis.}
    \flushright\sc  Morris Kline\footnote{\cite{Kline}, p.~1209.}\\
\end{quote}\index{Kline, Morris}

Kline refers to a much deeper kind of rigor than that which we will discuss in
this section.  G\"{o}del's incompleteness theorem\index{G\"{o}del's
incompleteness theorem} showed that it is impossible to achieve absolute rigor
in standard mathematics because we can never prove that mathematics is
consistent (free from contradictions).\index{consistent theory}  If
mathematics is consistent, we will never know it, but must rely on faith.  If
mathematics is inconsistent, the best we can hope for is that some clever
future mathematician will discover the inconsistency.  In this case, the
axioms would probably be revised slightly to eliminate the inconsistency, as
was done in the case of Russell's paradox,\index{Russell's paradox} but the
bulk of mathematics would probably not be affected by such a discovery.
Russell's paradox, for example, did not affect most of the remarkable results
achieved by 19th-century and earlier mathematicians.  It mainly invalidated
some of Gottlob Frege's\index{Frege, Gottlob} work on the foundations of
mathematics in the late 1800's; in fact Frege's work inspired Russell's
discovery.  Despite the paradox, Frege's work contains important concepts that
have significantly influenced modern logic.  Kline's {\em Mathematics, The
Loss of Certainty} \cite{Klinel}\index{Kline, Morris} has an interesting
discussion of this topic.

What {\em can} be achieved with absolute certainty\index{certainty} is the
knowledge that if we assume the axioms are consistent and true, then the
results derived from them are true.  Part of the beauty of mathematics is that
it is the one area of human endeavor where absolute certainty can be achieved
in this sense.  A mathematical truth will remain such for eternity.  However,
our actual knowledge of whether a particular statement is a mathematical truth
is only as certain as the correctness of the proof that establishes it.  If
the proof of a statement is questionable or vague, we can't have absolute
confidence in the truth that the statement claims.

Let us look at some traditional ways of expressing proofs.

Except in the field of formal logic\index{formal logic}, almost all
traditional proofs in mathematics are really not proofs at all, but rather
proof outlines or hints as to how to go about constructing the proof.  Many
gaps\index{gaps in proofs} are left for the reader to fill in. There are
several reasons for this.  First, it is usually assumed in mathematical
literature that the person reading the proof is a mathematician familiar with
the specialty being described, and that the missing steps are obvious to such
a reader or at least that the reader is capable of filling them in.  This
attitude is fine for professional mathematicians in the specialty, but
unfortunately it often has the drawback of cutting off the rest of the world,
including mathematicians in other specialties, from understanding the proof.
We discussed one possible resolution to this on p.~\pageref{envision}.
Second, it is often assumed that a complete formal proof\index{formal proof}
would require countless millions of symbols (Section~\ref{dream}). This might
be true if the proof were to be expressed directly in terms of the axioms of
logic and set theory,\index{set theory} but it is usually not true if we allow
ourselves a hierarchy\index{hierarchy} of definitions and theorems to build
upon, using a notation that allows us to introduce new symbols, definitions,
and theorems in a precisely specified way.

Even in formal logic,\index{formal logic} formal proofs\index{formal proof}
that are considered complete still contain hidden or implicit information.
For example, a ``proof'' is usually defined as a sequence of
wffs,\index{well-formed formula (wff)}\footnote{A {\em wff} or well-formed
formula is a mathematical expression (string of symbols) constructed according
to some precise rules.  A formal mathematical system\index{formal system}
contains (1) the rules for constructing syntactically correct
wffs,\index{syntax rules} (2) a list of starting wffs called
axioms,\index{axiom} and (3) one or more rules prescribing how to derive new
wffs, called theorems\index{theorem}, from the axioms or previously derived
theorems.  An example of such a system is contained in
Metamath's\index{Metamath} set theory database, which defines a formal
system\index{formal system} from which all of standard mathematics can be
derived.  Section~\ref{startf} steps you through a complete example of a formal
system, and you may want to skim it now if you are unfamiliar with the
concept.} each of which is an axiom or follows from a rule applied to previous
wffs in the sequence.  The implicit part of the proof is the algorithm by
which a sequence of symbols is verified to be a valid wff, given the
definition of a wff.  The algorithm in this case is rather simple, but for a
computer to verify the proof,\index{automated proof verification} it must have
the algorithm built into its verification program.\footnote{It is possible, of
course, to specify wff construction syntax outside of the program itself
with a suitable input language (the Metamath language being an example), but
some proof-verification or theorem-proving programs lack the ability to extend
wff syntax in such a fashion.} If one deals exclusively with axioms and
elementary wffs, it is straightforward to implement such an algorithm.  But as
more and more definitions are added to the theory in order to make the
expression of wffs more compact, the algorithm becomes more and more
complicated.  A computer program that implements the algorithm becomes larger
and harder to understand as each definition is introduced, and thus more prone
to bugs.\index{computer program bugs}  The larger the program, the
more suspicious the mathematician may be about
the validity of its algorithms.  This is especially true because
computer programs are inherently hard to follow to begin with, and few people
enjoy verifying them manually in detail.

Metamath\index{Metamath} takes a different approach.  Metamath's ``knowledge''
is limited to the ability to substitute variables for expressions, subject to
some simple constraints.  Once the basic algorithm of Metamath is assumed to
be debugged, and perhaps independently confirmed, it
can be trusted once and for all.  The information that Metamath needs to
``understand'' mathematics is contained entirely in the body of knowledge
presented to Metamath.  Any errors in reasoning can only be errors in the
axioms or definitions contained in this body of knowledge.  As a
``constructive'' language\index{constructive language} Metamath has no
conditional branches or loops like the ones that make computer programs hard
to decipher; instead, the language can only build new sequences of symbols
from earlier sequences  of symbols.

The simplicity of the rules that underlie Metamath not only makes Metamath
easy to learn but also gives Metamath a great deal of flexibility. For
example, Metamath is not limited to describing standard first-order
logic\index{first-order logic}; higher-order logics\index{higher-order logic}
and fragments of logic\index{weak logic} can be described just as easily.
Metamath gives you the freedom to define whatever wff notation you prefer; it
has no built-in conception of the syntax of a wff.\index{well-formed formula
(wff)}  With suitable axioms and definitions, Metamath can even describe and
prove things about itself.\index{Metamath!self-description}  (John
Harrison\index{Harrison, John} discusses the ``reflection''
principle\index{reflection principle} involved in self-descriptive systems in
\cite{Harrison}.)

The flexibility of Metamath requires that its proofs specify a lot of detail,
much more than in an ordinary ``formal'' proof.\index{formal proof}  For
example, in an ordinary formal proof, a single step consists of displaying the
wff that constitutes that step.  In order for a computer program to verify
that the step is acceptable, it first must verify that the symbol sequence
being displayed is an acceptable wff.\index{automated proof verification} Most
proof verifiers have at least basic wff syntax built into their programs.
Metamath has no hard-wired knowledge of what constitutes a wff built into it;
instead every wff must be explicitly constructed based on rules defining wffs
that are present in a database.  Thus a single step in an ordinary formal
proof may be correspond to many steps in a Metamath proof. Despite the larger
number of steps, though, this does not mean that a Metamath proof must be
significantly larger than an ordinary formal proof. The reason is that since
we have constructed the wff from scratch, we know what the wff is, so there is
no reason to display it.  We only need to refer to a sequence of statements
that construct it.  In a sense, the display of the wff in an ordinary formal
proof is an implicit proof of its own validity as a wff; Metamath just makes
the proof explicit. (Section~\ref{proof} describes Metamath's proof notation.)

\section{Computers and Mathematicians}

\begin{quote}
  {\em The computer is important, but not to mathematics.}
    \flushright\sc  Paul Halmos\footnote{As quoted in \cite{Albers}, p.~121.}\\
\end{quote}\index{Halmos, Paul}

Pure mathematicians have traditionally been indifferent to computers, even to
the point of disdain.\index{computers and pure mathematics}  Computer science
itself is sometimes considered to fall in the mundane realm of ``applied''
mathematics, perhaps essential for the real world but intellectually unexciting
to those who seek the deepest truths in mathematics.  Perhaps a reason for this
attitude towards computers is that there is little or no computer software that
meets their needs, and there may be a general feeling that such software could
not even exist.  On the one hand, there are the practical computer algebra
systems, which can perform amazing symbolic manipulations in algebra and
calculus,\index{computer algebra system} yet can't prove the simplest
existence theorem, if the idea of a proof is present at all.  On the other
hand, there are specialized automated theorem provers that technically speaking
may generate correct proofs.\index{automated theorem proving}  But sometimes
their specialized input notation may be cryptic and their output perceived to
be long, inelegant, incomprehensible proofs.    The output
may be viewed with suspicion, since the program that generates it tends to be
very large, and its size increases the potential for bugs\index{computer
program bugs}.  Such a proof may be considered trustworthy only if
independently verified and ``understood'' by a human, but no one wants to
waste their time on such a boring, unrewarding chore.



\needspace{4\baselineskip}
\subsection{Trusting the Computer}

\begin{quote}
  {\em \ldots I continue to find the quasi-empirical interpretation of
  computer proofs to be the more plausible.\ldots Since not
  everything that claims to be a computer proof can be
  accepted as valid, what are the mathematical criteria for acceptable
  computer proofs?}
    \flushright\sc  Thomas Tymoczko\footnote{\cite{Tymoczko}, p.~245.}\\
\end{quote}\index{Tymoczko, Thomas}

In some cases, computers have been essential tools for proving famous
theorems.  But if a proof is so long and obscure that it can be verified in a
practical way only with a computer, it is vaguely felt to be suspicious.  For
example, proving the famous four-color theorem\index{four-color
theorem}\index{proof length} (``a map needs no more than four colors to
prevent any two adjacent countries from having the same color'') can presently
only be done with the aid of a very complex computer program which originally
required 1200 hours of computer time. There has been considerable debate about
whether such a proof can be trusted and whether such a proof is ``real''
mathematics \cite{Swart}\index{Swart, E. R.}.\index{trusting computers}

However, under normal circumstances even a skeptical mathematician would have a
great deal of confidence in the result of multiplying two numbers on a pocket
calculator, even though the precise details of what goes on are hidden from its
user.  Even the verification on a supercomputer that a huge number is prime is
trusted, especially if there is independent verification; no one bothers to
debate the philosophical significance of its ``proof,'' even though the actual
proof would be so large that it would be completely impractical to ever write
it down on paper.  It seems that if the algorithm used by the computer is
simple enough to be readily understood, then the computer can be trusted.

Metamath\index{Metamath} adopts this philosophy.  The simplicity of its
language makes it easy to learn, and because of its simplicity one can have
essentially absolute confidence that a proof is correct. All axioms, rules, and
definitions are available for inspection at any time because they are defined
by the user; there are no hidden or built-in rules that may be prone to subtle
bugs\index{computer program bugs}.  The basic algorithm at the heart of
Metamath is simple and fixed, and it can be assumed to be bug-free and robust
with a degree of confidence approaching certainty.
Independently written implementations of the Metamath verifier
can reduce any residual doubt on the part of a skeptic even further;
there are now over a dozen such implementations, written by many people.

\subsection{Trusting the Mathematician}\label{trust}

\begin{quote}
  {\em There is no Algebraist nor Mathematician so expert in his science, as
  to place entire confidence in any truth immediately upon his discovery of it,
  or regard it as any thing, but a mere probability.  Every time he runs over
  his proofs, his confidence encreases; but still more by the approbation of
  his friends; and is rais'd to its utmost perfection by the universal assent
  and applauses of the learned world.}
  \flushright\sc David Hume\footnote{{\em A Treatise of Human Nature}, as
  quoted in \cite{deMillo}, p.~267.}\\
\end{quote}\index{Hume, David}

\begin{quote}
  {\em Stanislaw Ulam estimates that mathematicians publish 200,000 theorems
  every year.  A number of these are subsequently contradicted or otherwise
  disallowed, others are thrown into doubt, and most are ignored.}
  \flushright\sc Richard de Millo et. al.\footnote{\cite{deMillo}, p.~269.}\\
\end{quote}\index{Ulam, Stanislaw}

Whether or not the computer can be trusted, humans  of course will occasionally
err. Only the most memorable proofs get independently verified, and of these
only a handful of truly great ones achieve the status of being ``known''
mathematical truths that are used without giving a second thought to their
correctness.

There are many famous examples of incorrect theorems and proofs in
mathematical literature.\index{errors in proofs}

\begin{itemize}
\item There have been thousands of purported proofs of Fermat's Last
Theorem\index{Fermat's Last Theorem} (``no integer solutions exist to $x^n +
y^n = z^n$ for $n > 2$''), by amateurs, cranks, and well-regarded
mathematicians \cite[p.~5]{Stark}\index{Stark, Harold M}.  Fermat wrote a note
in his copy of Bachet's {\em Diophantus} that he found ``a truly marvelous
proof of this theorem but this margin is too narrow to contain it''
\cite[p.~507]{Kramer}.  A recent, much publicized proof by Yoichi
Miyaoka\index{Miyaoka, Yoichi} was shown to be incorrect ({\em Science News},
April 9, 1988, p.~230).  The theorem was finally proved by Andrew
Wiles\index{Wiles, Andrew} ({\em Science News}, July 3, 1993, p.~5), but it
initially had some gaps and took over a year after its announcement to be
checked thoroughly by experts.  On Oct. 25, 1994, Wiles announced that the last
gap found in his proof had been filled in.
  \item In 1882, M. Pasch discovered that an axiom was omitted from Euclid's
formulation of geometry\index{Euclidean geometry}; without it, the proofs of
important theorems of Euclid are not valid.  Pasch's axiom\index{Pasch's
axiom} states that a line that intersects one side of a triangle must also
intersect another side, provided that it does not touch any of the triangle's
vertices.  The omission of Pasch's axiom went unnoticed for 2000
years \cite[p.~160]{Davis}, in spite of (one presumes) the thousands of
students, instructors, and mathematicians who studied Euclid.
  \item The first published proof of the famous Schr\"{o}der--Bernstein
theorem\index{Schr\"{o}der--Bernstein theorem} in set theory was incorrect
\cite[p.~148]{Enderton}\index{Enderton, Herbert B.}.  This theorem states
that if there exists a 1-to-1 function\footnote{A {\em set}\index{set} is any
collection of objects. A {\em function}\index{function} or {\em
mapping}\index{mapping} is a rule that assigns to each element of one set
(called the function's {\em domain}\index{domain}) an element from another
set.} from set $A$ into set $B$ and vice-versa, then sets $A$ and $B$ have
a 1-to-1 correspondence.  Although it sounds simple and obvious,
the standard proof is quite long and complex.
  \item In the early 1900's, Hilbert\index{Hilbert, David} published a
purported proof of the continuum hypothesis\index{continuum hypothesis}, which
was eventually established as unprovable by Cohen\index{Cohen, Paul} in 1963
\cite[p.~166]{Enderton}.  The continuum hypothesis states that no
infinity\index{infinity} (``transfinite cardinal number'')\index{cardinal,
transfinite} exists whose size (or ``cardinality''\index{cardinality}) is
between the size of the set of integers and the size of the set of real
numbers.  This hypothesis originated with German mathematician Georg
Cantor\index{Cantor, Georg} in the late 1800's, and his inability to prove it
is said to have contributed to mental illness that afflicted him in his later
years.
  \item An incorrect proof of the four-color theorem\index{four-color theorem}
was published by Kempe\index{Kempe, A. B.} in 1879
\cite[p.~582]{Courant}\index{Courant, Richard}; it stood for 11 years before
its flaw was discovered.  This theorem states that any map can be colored
using only four colors, so that no two adjacent countries have the same
color.  In 1976 the theorem was finally proved by the famous computer-assisted
proof of Haken, Appel, and Koch \cite{Swart}\index{Appel, K.}\index{Haken,
W.}\index{Koch, K.}.  Or at least it seems that way.  Mathematician
H.~S.~M.~Coxeter\index{Coxeter, H. S. M.} has doubts \cite[p.~58]{Davis}:  ``I
have a feeling that that is an untidy kind of use of the computers, and the more
you correspond with Haken and Appel, the more shaky you seem to be.''
  \item Many false ``proofs'' of the Poincar\'{e}
conjecture\index{Poincar\'{e} conjecture} have been proposed over the years.
This conjecture states that any object that mathematically behaves like a
three-dimensional sphere is a three-dimensional sphere topologically,
regardless of how it is distorted.  In March 1986, mathematicians Colin
Rourke\index{Rourke, Colin} and Eduardo R\^{e}go\index{R\^{e}go, Eduardo}
caused  a stir in the mathematical community by announcing that they had found
a proof; in November of that year the proof was found to be false \cite[p.
218]{PetersonI}.  It was finally proved in 2003 by Grigory Perelman
\label{poincare}\index{Szpiro, George}\index{Perelman, Grigory}\cite{Szpiro}.
 \end{itemize}

Many counterexamples to ``theorems'' in recent mathematical
literature related to Clifford algebras\index{Clifford algebras}
 have been found by Pertti
Lounesto (who passed away in 2002).\index{Lounesto, Pertti}
See the web page \url{http://mathforum.org/library/view/4933.html}.
% http://users.tkk.fi/~ppuska/mirror/Lounesto/counterexamples.htm

One of the purposes of Metamath\index{Metamath} is to allow proofs to be
expressed with absolute precision.  Developing a proof in the Metamath
language can be challenging, because Metamath will not permit even the
tiniest mistake.\index{errors in proofs}  But once the proof is created, its
correctness can be trusted immediately, without having to depend on the
process of peer review for confirmation.

\section{The Use of Computers in Mathematics}

\subsection{Computer Algebra Systems}

For the most part, you will find that Metamath\index{Metamath} is not a
practical tool for manipulating numbers.  (Even proving that $2 + 2 = 4$, if
you start with set theory, can be quite complex!)  Several commercial
mathematics packages are quite good at arithmetic, algebra, and calculus, and
as practical tools they are invaluable.\index{computer algebra system} But
they have no notion of proof, and cannot understand statements starting with
``there exists such and such...''.

Software packages such as Mathematica \cite{Wolfram}\index{Mathematica} do not
concern themselves with proofs but instead work directly with known results.
These packages primarily emphasize heuristic rules such as the substitution of
equals for equals to achieve simpler expressions or expressions in a different
form.  Starting with a rich collection of built-in rules and algorithms, users
can add to the collection by means of a powerful programming language.
However, results such as, say, the existence of a certain abstract object
without displaying the actual object cannot be expressed (directly) in their
languages.  The idea of a proof from a small set of axioms is absent.  Instead
this software simply assumes that each fact or rule you add to the built-in
collection of algorithms is valid.  One way to view the software is as a large
collection of axioms from which the software, with certain goals, attempts to
derive new theorems, for example equating a complex expression with a simpler
equivalent. But the terms ``theorem''\index{theorem} and
``proof,''\index{proof} for example, are not even mentioned in the index of
the user's manual for Mathematica.\index{Mathematica and proofs}  What is also
unsatisfactory from a philosophical point of view is that there is no way to
ensure the validity of the results other than by trusting the writer of each
application module or tediously checking each module by hand, similar to
checking a computer program for bugs.\index{computer program
bugs}\footnote{Two examples illustrate why the knowledge database of computer
algebra systems should sometimes be regarded with a certain caution.  If you
ask Mathematica (version 3.0) to \texttt{Solve[x\^{ }n + y\^{ }n == z\^{ }n , n]}
it will respond with \texttt{\{\{n-\char`\>-2\}, \{n-\char`\>-1\},
\{n-\char`\>1\}, \{n-\char`\>2\}\}}. In other words, Mathematica seems to
``know'' that Fermat's Last Theorem\index{Fermat's Last Theorem} is true!  (At
the time this version of Mathematica was released this fact was unknown.)  If
you ask Maple\index{Maple} to \texttt{solve(x\^{ }2 = 2\^{ }x)} then
\texttt{simplify(\{"\})}, it returns the solution set \texttt{\{2, 4\}}, apparently
unaware that $-0.7666647$\ldots is also a solution.} While of course extremely
valuable in applied mathematics, computer algebra systems tend to be of little
interest to the theoretical mathematician except as aids for exploring certain
specific problems.

Because of possible bugs, trusting the output of a computer algebra system for
use as theorems in a proof-verifier would defeat the latter's goal of rigor.
On the other hand, a fact such that a certain relatively large number is
prime, while easy for a computer algebra system to derive, might have a long,
tedious proof that could overwhelm a proof-verifier. One approach for linking
computer algebra systems to a proof-verifier while retaining the advantages of
both is to add a hypothesis to each such theorem indicating its source.  For
example, a constant {\sc maple} could indicate the theorem came from the Maple
package, and would mean ``assuming Maple is consistent, then\ldots''  This and
many other topics concerning the formalization of mathematics are discussed in
John Harrison's\index{Harrison, John} very interesting
PhD thesis~\cite{Harrison-thesis}.

\subsection{Automated Theorem Provers}\label{theoremprovers}

A mathematical theory is ``decidable''\index{decidable theory} if a mechanical
method or algorithm exists that is guaranteed to determine whether or not a
particular formula is a theorem.  Among the few theories that are decidable is
elementary geometry,\index{Euclidean geometry} as was shown by a classic
result of logician Alfred Tarski\index{Tarski, Alfred} in 1948
\cite{Tarski}.\footnote{Tarski's result actually applies to a subset of the
geometry discussed in elementary textbooks.  This subset includes most of what
would be considered elementary geometry but it is not powerful enough to
express, among other things, the notions of the circumference and area of a
circle.  Extending the theory in a way that includes notions such as these
makes the theory undecidable, as was also shown by Tarski.  Tarski's algorithm
is far too inefficient to implement practically on a computer.  A practical
algorithm for proving a smaller subset of geometry theorems (those not
involving concepts of ``order'' or ``continuity'') was discovered by Chinese
mathematician Wu Wen-ts\"{u}n in 1977 \cite{Chou}\index{Chou,
Shang-Ching}.}\index{Wen-ts{\"{u}}n, Wu}  But most theories, including
elementary arithmetic, are undecidable.  This fact contributes to keeping
mathematics alive and well, since many mathematicians believe
that they will never be
replaced by computers (if they believe Roger Penrose's argument that a
computer can never replace the brain \cite{Penrose}\index{Penrose, Roger}).
In fact,  elementary geometry is often considered a ``dead'' field
for the simple reason that it is decidable.

On the other hand, the undecidability of a theory does not mean that one cannot
use a computer to search for proofs, providing one is willing to give up if a
proof is not found after a reasonable amount of time.  The field of automated
theorem proving\index{automated theorem proving} specializes in pursuing such
computer searches.  Among the more successful results to date are those based
on an algorithm known as Robinson's resolution principle
\cite{Robinson}\index{Robinson's resolution principle}.

Automated theorem provers can be excellent tools for those willing to learn
how to use them.  But they are not widely used in mainstream pure
mathematics, even though they could probably be useful in many areas.  There
are several reasons for this.  Probably most important, the main goal in pure
mathematics is to arrive at results that are considered to be deep or
important; proving them is essential but secondary.  Usually, an automated
theorem prover cannot assist in this main goal, and by the time the main goal
is achieved, the mathematician may have already figured out the proof as a
by-product.  There is also a notational problem.  Mathematicians are used to
using very compact syntax where one or two symbols (heavily dependent on
context) can represent very complex concepts; this is part of the
hierarchy\index{hierarchy} they have built up to tackle difficult problems.  A
theorem prover on the other hand might require that a theorem be expressed in
``first-order logic,''\index{first-order logic} which is the logic on which
most of mathematics is ultimately based but which is not ordinarily used
directly because expressions can become very long.  Some automated theorem
provers are experimental programs, limited in their use to very specialized
areas, and the goal of many is simply research into the nature of automated
theorem proving itself.  Finally, much research remains to be done to enable
them to prove very deep theorems.  One significant result was a
computer proof by Larry Wos\index{Wos, Larry} and colleagues that every Robbins
algebra\index{Robbins algebra} is a Boolean  algebra\index{Boolean algebra}
({\em New York Times}, Dec. 10, 1996).\footnote{In 1933, E.~V.\
Huntington\index{Huntington, E. V.}
presented the following axiom system for
Boolean algebra with a unary operation $n$ and a binary operation $+$:
\begin{center}
    $x + y = y + x$ \\
    $(x + y) + z = x + (y + z)$ \\
    $n(n(x) + y) + n(n(x) + n(y)) = x$
\end{center}
Herbert Robbins\index{Robbins, Herbert}, a student of Huntington, conjectured
that the last equation can be replaced with a simpler one:
\begin{center}
    $n(n(x + y) + n(x + n(y))) = x$
\end{center}
Robbins and Huntington could not find a proof.  The problem was
later studied unsuccessfully by Tarski\index{Tarski, Alfred} and his
students, and it remained an unsolved problem until a
computer found the proof in 1996.  For more information on
the Robbins algebra problem see \cite{Wos}.}

How does Metamath\index{Metamath} relate to automated theorem provers?  A
theorem prover is primarily concerned with one theorem at a time (perhaps
tapping into a small database of known theorems) whereas Metamath is more like
a theorem archiving system, storing both the theorem and its proof in a
database for access and verification.  Metamath is one answer to ``what do you
do with the output of a theorem prover?''  and could be viewed as the
next step in the process.  Automated theorem provers could be useful tools for
helping develop its database.
Note that very long, automatically
generated proofs can make your database fat and ugly and cause Metamath's proof
verification to take a long time to run.  Unless you have a particularly good
program that generates very concise proofs, it might be best to consider the
use of automatically generated proofs as a quick-and-dirty approach, to be
manually rewritten at some later date.

The program {\sc otter}\index{otter@{\sc otter}}\footnote{\url{http://www.cs.unm.edu/\~mccune/otter/}.}, later succeeded by
prover9\index{prover9}\footnote{\url{https://www.cs.unm.edu/~mccune/mace4/}.},
have been historically influential.
The E prover\index{E prover}\footnote{\url{https://github.com/eprover/eprover}.}
is a maintained automated theorem prover
for full first-order logic with equality.
There are many other automated theorem provers as well.

If you want to combine automated theorem provers with Metamath
consider investigating
the book {\em Automated Reasoning:  Introduction and Applications}
\cite{Wos}\index{Wos, Larry}.  This book discusses
how to use {\sc otter} in a way that can
not only able to generate
relatively efficient proofs, it can even be instructed to search for
shorter proofs.  The effective use of {\sc otter} (and similar tools)
does require a certain
amount of experience, skill, and patience.  The axiom system used in the
\texttt{set.mm}\index{set theory database (\texttt{set.mm})} set theory
database can be expressed to {\sc otter} using a method described in
\cite{Megill}.\index{Megill, Norman}\footnote{To use those axioms with
{\sc otter}, they must be restated in a way that eliminates the need for
``dummy variables.''\index{dummy variable!eliminating} See the Comment
on p.~\pageref{nodd}.} When successful, this method tends to generate
short and clever proofs, but my experiments with it indicate that the
method will find proofs within a reasonable time only for relatively
easy theorems.  It is still fun to experiment with.

Reference \cite{Bledsoe}\index{Bledsoe, W. W.} surveys a number of approaches
people have explored in the field of automated theorem proving\index{automated
theorem proving}.

\subsection{Interactive Theorem Provers}\label{interactivetheoremprovers}

Finding proofs completely automatically is difficult, so there
are some interactive theorem provers that allow a human to guide the
computer to find a proof.
Examples include
HOL Light\index{HOL light}%
\footnote{\url{https://www.cl.cam.ac.uk/~jrh13/hol-light/}.},
Isabelle\index{Isabelle}%
\footnote{\url{http://www.cl.cam.ac.uk/Research/HVG/Isabelle}.},
{\sc hol}\index{hol@{\sc hol}}%
\footnote{\url{https://hol-theorem-prover.org/}.},
and
Coq\index{Coq}\footnote{\url{https://coq.inria.fr/}.}.

A major difference between most of these tools and Metamath is that the
``proofs'' are actually programs that guide the program to find a proof,
and not the proof itself.
For example, an Isabelle/HOL proof might apply a step
\texttt{apply (blast dest: rearrange reduction)}. The \texttt{blast}
instruction applies
an automatic tableux prover and returns if it found a sequence of proof
steps that work... but the sequence is not considered part of the proof.

A good overview of
higher-level proof verification languages (such as {\sc lcf}\index{lcf@{\sc
lcf}} and {\sc hol}\index{hol@{\sc hol}})
is given in \cite{Harrison}.  All of these languages are fundamentally
different from Metamath in that much of the mathematical foundational
knowledge is embedded in the underlying proof-verification program, rather
than placed directly in the database that is being verified.
These can have a steep learning curve for those without a mathematical
background.  For example, one usually must have a fair understanding of
mathematical logic in order to follow their proofs.

\subsection{Proof Verifiers}\label{proofverifiers}

A proof verifier is a program that doesn't generate proofs but instead
verifies proofs that you give it.  Many proof verifiers have limited built-in
automated proof capabilities, such as figuring out simple logical inferences
(while still being guided by a person who provides the overall proof).  Metamath
has no built-in automated proof capability other than the limited
capability of its Proof Assistant.

Proof-verification languages are not used as frequently as they might be.
Pure mathematicians are more concerned with producing new results, and such
detail and rigor would interfere with that goal.  The use of computers in pure
mathematics is primarily focused on automated theorem provers (not verifiers),
again with the ultimate goal of aiding the creation of new mathematics.
Automated theorem provers are usually concerned with attacking one theorem at
time rather than making a large, organized database easily available to the
user.  Metamath is one way to help close this gap.

By itself Metamath is a mostly a proof verifier.
This does not mean that other approaches can't be used; the difference
is that in Metamath, the results of various provers must be recorded
step-by-step so that they can be verified.

Another proof-verification language is Mizar,\index{Mizar} which can display
its proofs in the informal language that mathematicians are accustomed to.
Information on the Mizar language is available at \url{http://mizar.org}.

For the working mathematician, Mizar is an excellent tool for rigorously
documenting proofs. Mizar typesets its proofs in the informal English used by
mathematicians (and, while fine for them, are just as inscrutable by
laypersons!). A price paid for Mizar is a relatively steep learning curve of a
couple of weeks.  Several mathematicians are actively formalizing different
areas of mathematics using Mizar and publishing the proofs in a dedicated
journal. Unfortunately the task of formalizing mathematics is still looked
down upon to a certain extent since it doesn't involve the creation of ``new''
mathematics.

The closest system to Metamath is
the {\em Ghilbert}\index{Ghilbert} proof language (\url{http://ghilbert.org})
system developed by
Raph Levien\index{Levien, Raph}.
Ghilbert is a formal proof checker heavily inspired by Metamath.
Ghilbert statements are s-expressions (a la Lisp), which is easy
for computers to parse but many people find them hard to read.
There are a number of differences in their specific constructs, but
there is at least one tool to translate some Metamath materials into Ghilbert.
As of 2019 the Ghilbert community is smaller and less active than the
Metamath community.
That said, the Metamath and Ghilbert communities overlap, and fruitful
conversations between them have occurred many times over the years.

\subsection{Creating a Database of Formalized Mathematics}\label{mathdatabase}

Besides Metamath, there are several other ongoing projects with the goal of
formalizing mathematics into computer-verifiable databases.
Understanding some history will help.

The {\sc qed}\index{qed project@{\sc qed} project}%
\footnote{\url{http://www-unix.mcs.anl.gov/qed}.}
project arose in 1993 and its goals were outlined in the
{\sc qed} manifesto.
The {\sc qed} manifesto was
a proposal for a computer-based database of all mathematical knowledge,
strictly formalized and with all proofs having been checked automatically.
The project had a conference in 1994 and another in 1995;
there was also a ``twenty years of the {\sc qed} manifesto'' workshop
in 2014.
Its ideals are regularly reraised.

In a 2007 paper, Freek Wiedijk identified two reasons
for the failure of the {\sc qed} project as originally envisioned:%
\cite{Wiedijk-revisited}\index{Wiedijk, Freek}

\begin{itemize}
\item Very few people are working on formalization of mathematics. There is no compelling application for fully mechanized mathematics.
\item Formalized mathematics does not yet resemble traditional mathematics. This is partly due to the complexity of mathematical notation, and partly to the limitations of existing theorem provers and proof assistants.
\end{itemize}

But this did not end the dream of
formalizing mathematics into computer-verifiable databases.
The problems that led to the {\sc qed} manifesto are still with us,
even though the challenges were harder than originally considered.
What has happened instead is that various independent projects have
worked towards formalizing mathematics into computer-verifiable databases,
each simultaneously competing and cooperating with each other.

A concrete way to see this is
Freek Wiedijk's ``Formalizing 100 Theorems'' list%
\footnote{\url{http://www.cs.ru.nl/\%7Efreek/100/}.}
which shows the progress different systems have made on a challenge list
of 100 mathematical theorems.%
\footnote{ This is not the only list of ``interesting'' theorems.
Another interesting list was posted by Oliver Knill's list
\cite{Knill}\index{Knill, Oliver}.}
The top systems as of February 2019
(in order of the number of challenges completed) are
HOL Light, Isabelle, Metamath, Coq, and Mizar.

The Metamath 100%
\footnote{\url{http://us.metamath.org/mm\_100.html}}
page (maintained by David A. Wheeler\index{Wheeler, David A.})
shows the progress of Metamath (specifically its \texttt{set.mm} database)
against this challenge list maintained by Freek Wiedijk.
The Metamath \texttt{set.mm} database
has made a lot of progress over the years,
in part because working to prove those challenge theorems required
defining various terms and proving their properties as a prerequisite.
Here are just a few of the many statements that have been
formally proven with Metamath:

% The entries of this cause the narrow display to break poorly,
% since the short amount of text means LaTeX doesn't get a lot to work with
% and the itemize format gives it even *less* margin than usual.
% No one will mind if we make just this list flushleft, since this list
% will be internally consistent.
\begin{flushleft}
\begin{itemize}
\item 1. The Irrationality of the Square Root of 2
  (\texttt{sqr2irr}, by Norman Megill, 2001-08-20)
\item 2. The Fundamental Theorem of Algebra
  (\texttt{fta}, by Mario Carneiro, 2014-09-15)
\item 22. The Non-Denumerability of the Continuum
  (\texttt{ruc}, by Norman Megill, 2004-08-13)
\item 54. The Konigsberg Bridge Problem
  (\texttt{konigsberg}, by Mario Carneiro, 2015-04-16)
\item 83. The Friendship Theorem
  (\texttt{friendship}, by Alexander W. van der Vekens, 2018-10-09)
\end{itemize}
\end{flushleft}

We thank all of those who have developed at least one of the Metamath 100
proofs, and we particularly thank
Mario Carneiro\index{Carneiro, Mario}
who has contributed the most Metamath 100 proofs as of 2019.
The Metamath 100 page shows the list of all people who have contributed a
proof, and links to graphs and charts showing progress over time.
We encourage others to work on proving theorems not yet proven in Metamath,
since doing so improves the work as a whole.

Each of the math formalization systems (including Metamath)
has different strengths and weaknesses, depending on what you value.
Key aspects that differentiate Metamath from the other top systems are:

\begin{itemize}
\item Metamath is not tied to any particular set of axioms.
\item Metamath can show every step of every proof, no exceptions.
  Most other provers only assert that a proof can be found, and do not
  show every step. This also makes verification fast, because
  the system does not need to rediscover proof details.
\item The Metamath verifier has been re-implemented in many different
  programming languages, so verification can be done by multiple
  implementations.  In particular, the
  \texttt{set.mm}\index{set theory database (\texttt{set.mm})}%
  \index{Metamath Proof Explorer} database is verified by
  four different verifiers
  written in four different languages by four different authors.
  This greatly reduces the risk of accepting an invalid
  proof due to an error in the verifier.
\item Proofs stay proven.  In some systems, changes to the system's
  syntax or how a tactic works causes proofs to fail in later versions,
  causing older work to become essentially lost.
  Metamath's language is
  extremely small and fixed, so once a proof is added to a database,
  the database can be rechecked with later versions of the Metamath program
  and with other verifiers of Metamath databases.
  If an axiom or key definition needs to be changed, it is easy to
  manipulate the database as a whole to handle the change
  without touching the underlying verifier.
  Since re-verification of an entire database takes seconds, there
  is never a reason to delay complete verification.
  This aspect is especially compelling if your
  goal is to have a long-term database of proofs.
\item Licensing is generous.  The main Metamath databases are released to
  the public domain, and the main Metamath program is open source software
  under a standard, widely-used license.
\item Substitutions are easy to understand, even by those who are not
  professional mathematicians.
\end{itemize}

Of course, other systems may have advantages over Metamath
that are more compelling, depending on what you value.
In any case, we hope this helps you understand Metamath
within a wider context.

\subsection{In Summary}\label{computers-summary}

To summarize our discussions of computers and mathematics, computer algebra
systems can be viewed as theorem generators focusing on a narrow realm of
mathematics (numbers and their properties), automated theorem provers as proof
generators for specific theorems in a much broader realm covered by a built-in
formal system such as first-order logic, interactive theorem
provers require human guidance, proof verifiers verify proofs but
historically they have been
restricted to first-order logic.
Metamath, in contrast,
is a proof verifier and documenter whose realm is essentially unlimited.

\section{Mathematics and Metamath}

\subsection{Standard Mathematics}

There are a number of ways that Metamath\index{Metamath} can be used with
standard mathematics.  The most satisfying way philosophically is to start at
the very beginning, and develop the desired mathematics from the axioms of
logic and set theory.\index{set theory}  This is the approach taken in the
\texttt{set.mm}\index{set theory database (\texttt{set.mm})}%
\index{Metamath Proof Explorer}
database (also known as the Metamath Proof Explorer).
Among other things, this database builds up to the
axioms of real and complex numbers\index{analysis}\index{real and complex
numbers} (see Section~\ref{real}), and a standard development of analysis, for
example, could start at that point, using it as a basis.   Besides this
philosophical advantage, there are practical advantages to having all of the
tools of set theory available in the supporting infrastructure.

On the other hand, you may wish to start with the standard axioms of a
mathematical theory without going through the set theoretical proofs of those
axioms.  You will need mathematical logic to make inferences, but if you wish
you can simply introduce theorems\index{theorem} of logic as
``axioms''\index{axiom} wherever you need them, with the implicit assumption
that in principle they can be proved, if they are obvious to you.  If you
choose this approach, you will probably want to review the notation used in
\texttt{set.mm}\index{set theory database (\texttt{set.mm})} so that your own
notation will be consistent with it.

\subsection{Other Formal Systems}
\index{formal system}

Unlike some programs, Metamath\index{Metamath} is not limited to any specific
area of mathematics, nor committed to any particular mathematical philosophy
such as classical logic versus intuitionism, nor limited, say, to expressions
in first-order logic.  Although the database \texttt{set.mm}
describes standard logic and set theory, Meta\-math
is actually a general-purpose language for describing a wide variety of formal
systems.\index{formal system}  Non-standard systems such as modal
logic,\index{modal logic} intuitionist logic\index{intuitionism}, higher-order
logic\index{higher-order logic}, quantum logic\index{quantum logic}, and
category theory\index{category theory} can all be described with the Metamath
language.  You define the symbols you prefer and tell Metamath the axioms and
rules you want to start from, and Metamath will verify any inferences you make
from those axioms and rules.  A simple example of a non-standard formal system
is Hofstadter's\index{Hofstadter, Douglas R.} MIU system,\index{MIU-system}
whose Metamath description is presented in Appendix~\ref{MIU}.

This is not hypothetical.
The largest Metamath database is
\texttt{set.mm}\index{set theory database (\texttt{set.mm}}%
\index{Metamath Proof Explorer}), aka the Metamath Proof Explorer,
which uses the most common axioms for mathematical foundations
(specifically classical logic combined with Zermelo--Fraenkel
set theory\index{Zermelo--Fraenkel set theory} with the Axiom of Choice).
But other Metamath databases are available:

\begin{itemize}
\item The database
  \texttt{iset.mm}\index{intuitionistic logic database (\texttt{iset.mm})},
  aka the
  Intuitionistic Logic Explorer\index{Intuitionistic Logic Explorer},
  uses intuitionistic logic (a constructivist point of view)
  instead of classical logic.
\item The database
  \texttt{nf.mm}\index{New Foundations database (\texttt{nf.mm})},
  aka the
  New Foundations Explorer\index{New Foundations Explorer},
  constructs mathematics from scratch,
  starting from Quine's New Foundations (NF) set theory axioms.
\item The database
  \texttt{hol.mm}\index{Higher-order Logic database (\texttt{hol.mm})},
  aka the
  Higher-Order Logic (HOL) Explorer\index{Higher-Order Logic (HOL) Explorer},
  starts with HOL (also called simple type theory) and derives
  equivalents to ZFC axioms, connecting the two approaches.
\end{itemize}

Since the days of David Hilbert,\index{Hilbert, David} mathematicians have
been concerned with the fact that the metalanguage\index{metalanguage} used to
describe mathematics may be stronger than the mathematics being described.
Metamath\index{Metamath}'s underlying finitary\index{finitary proof},
constructive nature provides a good philosophical basis for studying even the
weakest logics.\index{weak logic}

The usual treatment of many non-standard formal systems\index{formal
system} uses model theory\index{model theory} or proof theory\index{proof
theory} to describe these systems; these theories, in turn, are based on
standard set theory.  In other words, a non-standard formal system is defined
as a set with certain properties, and standard set theory is used to derive
additional properties of this set.  The standard set theory database provided
with Metamath can be used for this purpose, and when used this way
the development of a special
axiom system for the non-standard formal system becomes unnecessary.  The
model- or proof-theoretic approach often allows you to prove much deeper
results with less effort.

Metamath supports both approaches.  You can define the non-standard
formal system directly, or define the non-standard formal system as
a set with certain properties, whichever you find most helpful.

%\section{Additional Remarks}

\subsection{Metamath and Its Philosophy}

Closely related to Metamath\index{Metamath} is a philosophy or way of looking
at mathematics. This philosophy is related to the formalist
philosophy\index{formalism} of Hilbert\index{Hilbert, David} and his followers
\cite[pp.~1203--1208]{Kline}\index{Kline, Morris}
\cite[p.~6]{Behnke}\index{Behnke, H.}. In this philosophy, mathematics is
viewed as nothing more than a set of rules that manipulate symbols, together
with the consequences of those rules.  While the mathematics being described
may be complex, the rules used to describe it (the
``metamathematics''\index{metamathematics}) should be as simple as possible.
In particular, proofs should be restricted to dealing with concrete objects
(the symbols we write on paper rather than the abstract concepts they
represent) in a constructive manner; these are called ``finitary''
proofs\index{finitary proof} \cite[pp.~2--3]{Shoenfield}\index{Shoenfield,
Joseph R.}.

Whether or not you find Metamath interesting or useful will in part depend on
the appeal you find in its philosophy, and this appeal will probably depend on
your particular goals with respect to mathematics.  For example, if you are a
pure mathematician at the forefront of discovering new mathematical knowledge,
you will probably find that the rigid formality of Metamath stifles your
creativity.  On the other hand, we would argue that once this knowledge is
discovered, there are advantages to documenting it in a standard format that
will make it accessible to others.  Sixty years from now, your field may be
dormant, and as Davis and Hersh put it, your ``writings would become less
translatable than those of the Maya'' \cite[p.~37]{Davis}\index{Davis, Phillip
J.}.


\subsection{A History of the Approach Behind Metamath}

Probably the one work that has had the most motivating influence on
Metamath\index{Metamath} is Whitehead and Russell's monumental {\em Principia
Mathematica} \cite{PM}\index{Whitehead, Alfred North}\index{Russell,
Bertrand}\index{principia mathematica@{\em Principia Mathematica}}, whose aim
was to deduce all of mathematics from a small number of primitive ideas, in a
very explicit way that in principle anyone could understand and follow.  While
this work was tremendously influential in its time, from a modern perspective
it suffers from several drawbacks.  Both its notation and its underlying
axioms are now considered dated and are no longer used.  From our point of
view, its development is not really as accessible as we would like to see; for
practical reasons, proofs become more and more sketchy as its mathematics
progresses, and working them out in fine detail requires a degree of
mathematical skill and patience that many people don't have.  There are
numerous small errors, which is understandable given the tedious, technical
nature of the proofs and the lack of a computer to verify the details.
However, even today {\em Principia Mathematica} stands out as the work closest
in spirit to Metamath.  It remains a mind-boggling work, and one can't help
but be amazed at seeing ``$1+1=2$'' finally appear on page 83 of Volume II
(Theorem *110.643).

The origin of the proof notation used by Metamath dates back to the 1950's,
when the logician C.~A.~Meredith expressed his proofs in a compact notation
called ``condensed detachment''\index{condensed detachment}
\cite{Hindley}\index{Hindley, J. Roger} \cite{Kalman}\index{Kalman, J. A.}
\cite{Meredith}\index{Meredith, C. A.} \cite{Peterson}\index{Peterson, Jeremy
George}.  This notation allows proofs to be communicated unambiguously by
merely referencing the axiom\index{axiom}, rule\index{rule}, or
theorem\index{theorem} used at each step, without explicitly indicating the
substitutions\index{substitution!variable}\index{variable substitution} that
have to be made to the variables in that axiom, rule, or theorem.  Ordinarily,
condensed detachment is more or less limited to propositional
calculus\index{propositional calculus}.  The concept has been extended to
first-order logic\index{first-order logic} in \cite{Megill}\index{Megill,
Norman}, making it is easy to write a small computer program to verify proofs
of simple first-order logic theorems.\index{condensed detachment!and
first-order logic}

A key concept behind the notation of condensed detachment is called
``unification,''\index{unification} which is an algorithm for determining what
substitutions\index{substitution!variable}\index{variable substitution} to
variables have to be made to make two expressions match each other.
Unification was first precisely defined by the logician J.~A.~Robinson, who
used it in the development of a powerful
theorem-proving technique called the ``resolution principle''
\cite{Robinson}\index{Robinson's resolution principle}. Metamath does not make
use of the resolution principle, which is intended for systems of first-order
logic.\index{first-order logic}  Metamath's use is not restricted to
first-order logic, and as we have mentioned it does not automatically discover
proofs.  However, unification is a key idea behind Metamath's proof
notation, and Metamath makes use of a very simple version of it
(Section~\ref{unify}).

\subsection{Metamath and First-Order Logic}

First-order logic\index{first-order logic} is the supporting structure
for standard mathematics.  On top of it is set theory, which contains
the axioms from which virtually all of mathematics can be derived---a
remarkable fact.\index{category
theory}\index{cardinal, inaccessible}\label{categoryth}\footnote{An exception seems
to be category theory.  There are several schools of thought on whether
category theory is derivable from set theory.  At a minimum, it appears
that an additional axiom is needed that asserts the existence of an
``inaccessible cardinal'' (a type of infinity so large that standard set
theory can't prove or deny that it exists).
%
%%%% (I took this out that was in previous editions:)
% But it is also argued that not just one but a ``proper class'' of them
% is needed, and the existence of proper classes is impossible in standard
% set theory.  (A proper class is a collection of sets so huge that no set
% can contain it as an element.  Proper classes can lead to
% inconsistencies such as ``Russell's paradox.''  The axioms of standard
% set theory are devised so as to deny the existence of proper classes.)
%
For more information, see
\cite[pp.~328--331]{Herrlich}\index{Herrlich, Horst} and
\cite{Blass}\index{Blass, Andrea}.}

One of the things that makes Metamath\index{Metamath} more practical for
first-order theories is a set of axioms for first-order logic designed
specifically with Metamath's approach in mind.  These are included in
the database \texttt{set.mm}\index{set theory database (\texttt{set.mm})}.
See Chapter~\ref{fol} for a detailed
description; the axioms are shown in Section~\ref{metaaxioms}.  While
logically equivalent to standard axiom systems, our axiom system breaks
up the standard axioms into smaller pieces such that from them, you can
directly derive what in other systems can only be derived as higher-level
``metatheorems.''\index{metatheorem}  In other words, it is more powerful than
the standard axioms from a metalogical point of view.  A rigorous
justification for this system and its ``metalogical
completeness''\index{metalogical completeness} is found in
\cite{Megill}\index{Megill, Norman}.  The system is closely related to a
system developed by Monk\index{Monk, J. Donald} and Tarski\index{Tarski,
Alfred} in 1965 \cite{Monks}.

For example, the formula $\exists x \, x = y $ (given $y$, there exists some
$x$ equal to it) is a theorem of logic,\footnote{Specifically, it is a theorem
of those systems of logic that assume non-empty domains.  It is not a theorem
of more general systems that include the empty domain\index{empty domain}, in
which nothing exists, period!  Such systems are called ``free
logics.''\index{free logic} For a discussion of these systems, see
\cite{Leblanc}\index{Leblanc, Hugues}.  Since our use for logic is as a basis
for set theory, which has a non-empty domain, it is more convenient (and more
traditional) to use a less general system.  An interesting curiosity is that,
using a free logic as a basis for Zermelo--Fraenkel set
theory\index{Zermelo--Fraenkel set theory} (with the redundant Axiom of the
Null Set omitted),\index{Axiom of the Null Set} we cannot even prove the
existence of a single set without assuming the axiom of infinity!\index{Axiom
of Infinity}} whether or not $x$ and $y$ are distinct variables\index{distinct
variables}.  In many systems of logic, we would have to prove two theorems to
arrive at this result.  First we would prove ``$\exists x \, x = x $,'' then
we would separately prove ``$\exists x \, x = y $, where $x$ and $y$ are
distinct variables.''  We would then combine these two special cases ``outside
of the system'' (i.e.\ in our heads) to be able to claim, ``$\exists x \, x =
y $, regardless of whether $x$ and $y$ are distinct.''  In other words, the
combination of the two special cases is a
metatheorem.  In the system of logic
used in Metamath's set theory\index{set theory database (\texttt{set.mm})}
database, the axioms of logic are broken down into small pieces that allow
them to be reassembled in such a way that theorems such as these can be proved
directly.

Breaking down the axioms in this way makes them look peculiar and not very
intuitive at first, but rest assured that they are correct and complete.  Their
correctness is ensured because they are theorem schemes of standard first-order
logic (which you can easily verify if you are a logician).  Their completeness
follows from the fact that we explicitly derive the standard axioms of
first-order logic as theorems.  Deriving the standard axioms is somewhat
tricky, but once we're there, we have at our disposal a system that is less
awkward to work with in formal proofs\index{formal proof}.  In technical terms
that logicians understand, we eliminate the cumbersome concepts of ``free
variable,''\index{free variable} ``bound variable,''\index{bound variable} and
``proper substitution''\index{proper substitution}\index{substitution!proper}
as primitive notions.  These concepts are present in our system but are
defined in terms of concepts expressed by the axioms and can be eliminated in
principle.  In standard systems, these concepts are really like additional,
implicit axioms\index{implicit axiom} that are somewhat complex and cannot be
eliminated.

The traditional approach to logic, wherein free variables and proper
substitution is defined, is also possible to do directly in the Metamath
language.  However, the notation tends to become awkward, and there are
disadvantages:  for example, extending the definition of a wff with a
definition is awkward, because the free variable and proper substitution
concepts have to have their definitions also extended.  Our choice of
axioms for \texttt{set.mm} is to a certain extent a matter of style, in
an attempt to achieve overall simplicity, but you should also be aware
that the traditional approach is possible as well if you should choose
it.

\chapter{Using the Metamath Program}
\label{using}

\section{Installation}

The way that you install Metamath\index{Metamath!installation} on your
computer system will vary for different computers.  Current
instructions are provided with the Metamath program download at
\url{http://metamath.org}.  In general, the installation is simple.
There is one file containing the Metamath program itself.  This file is
usually called \texttt{metamath} or \texttt{metamath.}{\em xxx} where
{\em xxx} is the convention (such as \texttt{exe}) for an executable
program on your operating system.  There are several additional files
containing samples of the Metamath language, all ending with
\texttt{.mm}.  The file \texttt{set.mm}\index{set theory database
(\texttt{set.mm})} contains logic and set theory and can be used as a
starting point for other areas of mathematics.

You will also need a text editor\index{text editor} capable of editing plain
{\sc ascii}\footnote{American Standard Code for Information Interchange.} text
in order to prepare your input files.\index{ascii@{\sc ascii}}  Most computers
have this capability built in.  Note that plain text is not necessarily the
default for some word processing programs\index{word processor}, especially if
they can handle different fonts; for example, with Microsoft Word\index{Word
(Microsoft)}, you must save the file in the format ``Text Only With Line
Breaks'' to get a plain text\index{plain text} file.\footnote{It is
recommended that all lines in a Metamath source file be 79 characters or less
in length for compatibility among different computer terminals.  When creating
a source file on an editor such as Word, select a monospaced
font\index{monospaced font} such as Courier\index{Courier font} or
Monaco\index{Monaco font} to make this easier to achieve.  Better yet,
just use a plain text editor such as Notepad.}

On some computer systems, Metamath does not have the capability to print
its output directly; instead, you send its output to a file (using the
\texttt{open} commands described later).  The way you print this output
file depends on your computer.\index{printers} Some computers have a
print command, whereas with others, you may have to read the file into
an editor and print it from there.

If you want to print your Metamath source files with typeset formulas
containing standard mathematical symbols, you will need the \LaTeX\
typesetting program\index{latex@{\LaTeX}}, which is widely and freely
available for most operating systems.  It runs natively on Unix and
Linux, and can be installed on Windows as part of the free Cygwin
package (\url{http://cygwin.com}).

You can also produce {\sc html}\footnote{HyperText Markup Language.}
web pages.  The {\tt help html} command in the Metamath program will
assist you with this feature.

\section{Your First Formal System}\label{start}
\subsection{From Nothing to Zero}\label{startf}

To give you a feel for what the Metamath\index{Metamath} language looks like,
we will take a look at a very simple example from formal number
theory\index{number theory}.  This example is taken from
Mendelson\index{Mendelson, Elliot} \cite[p. 123]{Mendelson}.\footnote{To keep
the example simple, we have changed the formalism slightly, and what we call
axioms\index{axiom} are strictly speaking theorems\index{theorem} in
\cite{Mendelson}.}  We will look at a small subset of this theory, namely that
part needed for the first number theory theorem proved in \cite{Mendelson}.

First we will look at a standard formal proof\index{formal proof} for the
example we have picked, then we will look at the Metamath version.  If you
have never been exposed to formal proofs, the notation may seem to be such
overkill to express such simple notions that you may wonder if you are missing
something.  You aren't.  The concepts involved are in fact very simple, and a
detailed breakdown in this fashion is necessary to express the proof in a way
that can be verified mechanically.  And as you will see, Metamath breaks the
proof down into even finer pieces so that the mechanical verification process
can be about as simple as possible.

Before we can introduce the axioms\index{axiom} of the theory, we must define
the syntax rules for forming legal expressions\index{syntax rules}
(combinations of symbols) with which those axioms can be used. The number 0 is
a {\bf term}\index{term}; and if $ t$ and $r$ are terms, so is $(t+r)$. Here,
$ t$ and $r$ are ``metavariables''\index{metavariable} ranging over terms; they
themselves do not appear as symbols in an actual term.  Some examples of
actual terms are $(0 + 0)$ and $((0+0)+0)$.  (Note that our theory describes
only the number zero and sums of zeroes.  Of course, not much can be done with
such a trivial theory, but remember that we have picked a very small subset of
complete number theory for our example.  The important thing for you to focus
on is our definitions that describe how symbols are combined to form valid
expressions, and not on the content or meaning of those expressions.) If $ t$
and $r$ are terms, an expression of the form $ t=r$ is a {\bf wff}
(well-formed formula)\index{well-formed formula (wff)}; and if $P$ and $Q$ are
wffs, so is $(P\rightarrow Q)$ (which means ``$P$ implies
$Q$''\index{implication ($\rightarrow$)} or ``if $P$ then $Q$'').
Here $P$ and $Q$ are metavariables ranging over wffs.  Examples of actual
wffs are $0=0$, $(0+0)=0$, $(0=0 \rightarrow (0+0)=0)$, and $(0=0\rightarrow
(0=0\rightarrow 0=(0+0)))$.  (Our notation makes use of more parentheses than
are customary, but the elimination of ambiguity this way simplifies our
example by avoiding the need to define operator precedence\index{operator
precedence}.)

The {\bf axioms}\index{axiom} of our theory are all wffs of the following
form, where $ t$, $r$, and $s$ are any terms:

%Latex p. 92
\renewcommand{\theequation}{A\arabic{equation}}

\begin{equation}
(t=r\rightarrow (t=s\rightarrow r=s))
\end{equation}
\begin{equation}
(t+0)=t
\end{equation}

Note that there are an infinite number of axioms since there are an infinite
number of possible terms.  A1 and A2 are properly called ``axiom
schemes,''\index{axiom scheme} but we will refer to them as ``axioms'' for
brevity.

An axiom is a {\bf theorem}; and if $P$ and $(P\rightarrow Q)$ are theorems
(where $P$ and $Q$ are wffs), then $Q$ is also a theorem.\index{theorem}  The
second part of this definition is called the modus ponens (MP) rule of
inference\index{inference rule}\index{modus ponens}.  It allows us to obtain
new theorems from old ones.

The {\bf proof}\index{proof} of a theorem is a sequence of one or more
theorems, each of which is either an axiom or the result of modus ponens
applied to two previous theorems in the sequence, and the last of which is the
theorem being proved.

The theorem we will prove for our example is very simple:  $ t=t$.  The proof of
our theorem follows.  Study it carefully until you feel sure you
understand it.\label{zeroproof}

% Use tabu so that lines will wrap automatically as needed.
\begin{tabu} { l X X }
1. & $(t+0)=t$ & (by axiom A2) \\
2. & $(t+0)=t$ & (by axiom A2) \\
3. & $((t+0)=t \rightarrow ((t+0)=t\rightarrow t=t))$ & (by axiom A1) \\
4. & $((t+0)=t\rightarrow t=t)$ & (by MP applied to steps 2 and 3) \\
5. & $t=t$ & (by MP applied to steps 1 and 4) \\
\end{tabu}

(You may wonder why step 1 is repeated twice.  This is not necessary in the
formal language we have defined, but in Metamath's ``reverse Polish
notation''\index{reverse Polish notation (RPN)} for proofs, a previous step
can be referred to only once.  The repetition of step~1 here will enable you
to see more clearly the correspondence of this proof with the
Metamath\index{Metamath} version on p.~\pageref{demoproof}.)

Our theorem is more properly called a ``theorem scheme,''\index{theorem
scheme} for it represents an infinite number of theorems, one for each
possible term $ t$.  Two examples of actual theorems would be $0=0$ and
$(0+0)=(0+0)$.  Rarely do we prove actual theorems, since by proving schemes
we can prove an infinite number of theorems in one fell swoop.  Similarly, our
proof should really be called a ``proof scheme.''\index{proof scheme}  To
obtain an actual proof, pick an actual term to use in place of $ t$, and
substitute it for $ t$ throughout the proof.

Let's discuss what we have done here.  The axioms\index{axiom} of our theory,
A1 and A2, are trivial and obvious.  Everyone knows that adding zero to
something doesn't change it, and also that if two things are equal to a third,
then they are equal to each other. In fact, stating the trivial and obvious is
a goal to strive for in any axiomatic system.  From trivial and obvious truths
that everyone agrees upon, we can prove results that are not so obvious yet
have absolute faith in them.  If we trust the axioms and the rules, we must,
by definition, trust the consequences of those axioms and rules, if logic is
to mean anything at all.

Our rule of inference\index{rule}, modus ponens\index{modus ponens}, is also
pretty obvious once you understand what it means.  If we prove a fact $P$, and
we also prove that $P$ implies $Q$, then $Q$ necessarily follows as a new
fact.  The rule provides us with a means for obtaining new facts (i.e.\
theorems\index{theorem}) from old ones.

The theorem that we have proved, $ t=t$, is so fundamental that you may wonder
why it isn't one of the axioms\index{axiom}.  In some axiom systems of
arithmetic, it {\em is} an axiom.  The choice of axioms in a theory is to some
extent arbitrary and even an art form, constrained only by the requirement
that any two equivalent axiom systems be able to derive each other as
theorems.  We could imagine that the inventor of our axiom system originally
included $ t=t$ as an axiom, then discovered that it could be derived as a
theorem from the other axioms.  Because of this, it was not necessary to
keep it as an axiom.  By eliminating it, the final set of axioms became
that much simpler.

Unless you have worked with formal proofs\index{formal proof} before, it
probably wasn't apparent to you that $ t=t$ could be derived from our two
axioms until you saw the proof. While you certainly believe that $ t=t$ is
true, you might not be able to convince an imaginary skeptic who believes only
in our two axioms until you produce the proof.  Formal proofs such as this are
hard to come up with when you first start working with them, but after you get
used to them they can become interesting and fun.  Once you understand the
idea behind formal proofs you will have grasped the fundamental principle that
underlies all of mathematics.  As the mathematics becomes more sophisticated,
its proofs become more challenging, but ultimately they all can be broken down
into individual steps as simple as the ones in our proof above.

Mendelson's\index{Mendelson, Elliot} book, from which our example was taken,
contains a number of detailed formal proofs such as these, and you may be
interested in looking it up.  The book is intended for mathematicians,
however, and most of it is rather advanced.  Popular literature describing
formal proofs\index{formal proof} include \cite[p.~296]{Rucker}\index{Rucker,
Rudy} and \cite[pp.~204--230]{Hofstadter}\index{Hofstadter, Douglas R.}.

\subsection{Converting It to Metamath}\label{convert}

Formal proofs\index{formal proof} such as the one in our example break down
logical reasoning into small, precise steps that leave little doubt that the
results follow from the axioms\index{axiom}.  You might think that this would
be the finest breakdown we can achieve in mathematics.  However, there is more
to the proof than meets the eye. Although our axioms were rather simple, a lot
of verbiage was needed before we could even state them:  we needed to define
``term,'' ``wff,'' and so on.  In addition, there are a number of implied
rules that we haven't even mentioned. For example, how do we know that step 3
of our proof follows from axiom A1? There is some hidden reasoning involved in
determining this.  Axiom A1 has two occurrences of the letter $ t$.  One of
the implied rules states that whatever we substitute for $ t$ must be a legal
term\index{term}.\footnote{Some authors make this implied rule explicit by
stating, ``only expressions of the above form are terms,'' after defining
``term.''}  The expression $ t+0$ is pretty obviously a legal term whenever $
t$ is, but suppose we wanted to substitute a huge term with thousands of
symbols?  Certainly a lot of work would be involved in determining that it
really is a term, but in ordinary formal proofs all of this work would be
considered a single ``step.''

To express our axiom system in the Metamath\index{Metamath} language, we must
describe this auxiliary information in addition to the axioms themselves.
Metamath does not know what a ``term'' or a ``wff''\index{well-formed formula
(wff)} is.  In Metamath, the specification of the ways in which we can combine
symbols to obtain terms and wffs are like little axioms in themselves.  These
auxiliary axioms are expressed in the same notation as the ``real''
axioms\index{axiom}, and Metamath does not distinguish between the two.  The
distinction is made by you, i.e.\ by the way in which you interpret the
notation you have chosen to express these two kinds of axioms.

The Metamath language breaks down mathematical proofs into tiny pieces, much
more so than in ordinary formal proofs\index{formal proof}.  If a single
step\index{proof step} involves the
substitution\index{substitution!variable}\index{variable substitution} of a
complex term for one of its variables, Metamath must see this single step
broken down into many small steps.  This fine-grained breakdown is what gives
Metamath generality and flexibility as it lets it not be limited to any
particular mathematical notation.

Metamath's proof notation is not, in itself, intended to be read by humans but
rather is in a compact format intended for a machine.  The Metamath program
will convert this notation to a form you can understand, using the \texttt{show
proof}\index{\texttt{show proof} command} command.  You can tell the program what
level of detail of the proof you want to look at.  You may want to look at
just the logical inference steps that correspond
to ordinary formal proof steps,
or you may want to see the fine-grained steps that prove that an expression is
a term.

Here, without further ado, is our example converted to the
Metamath\index{Metamath} language:\index{metavariable}\label{demo0}

\begin{verbatim}
$( Declare the constant symbols we will use $)
    $c 0 + = -> ( ) term wff |- $.
$( Declare the metavariables we will use $)
    $v t r s P Q $.
$( Specify properties of the metavariables $)
    tt $f term t $.
    tr $f term r $.
    ts $f term s $.
    wp $f wff P $.
    wq $f wff Q $.
$( Define "term" and "wff" $)
    tze $a term 0 $.
    tpl $a term ( t + r ) $.
    weq $a wff t = r $.
    wim $a wff ( P -> Q ) $.
$( State the axioms $)
    a1 $a |- ( t = r -> ( t = s -> r = s ) ) $.
    a2 $a |- ( t + 0 ) = t $.
$( Define the modus ponens inference rule $)
    ${
       min $e |- P $.
       maj $e |- ( P -> Q ) $.
       mp  $a |- Q $.
    $}
$( Prove a theorem $)
    th1 $p |- t = t $=
  $( Here is its proof: $)
       tt tze tpl tt weq tt tt weq tt a2 tt tze tpl
       tt weq tt tze tpl tt weq tt tt weq wim tt a2
       tt tze tpl tt tt a1 mp mp
     $.
\end{verbatim}\index{metavariable}

A ``database''\index{database} is a set of one or more {\sc ascii} source
files.  Here's a brief description of this Metamath\index{Metamath} database
(which consists of this single source file), so that you can understand in
general terms what is going on.  To understand the source file in detail, you
should read Chapter~\ref{languagespec}.

The database is a sequence of ``tokens,''\index{token} which are normally
separated by spaces or line breaks.  The only tokens that are built into
the Metamath language are those beginning with \texttt{\$}.  These tokens
are called ``keywords.''\index{keyword}  All other tokens are
user-defined, and their names are arbitrary.

As you might have guessed, the Metamath token \texttt{\$(}\index{\texttt{\$(} and
\texttt{\$)} auxiliary keywords} starts a comment and \texttt{\$)} ends a comment.

The Metamath tokens \texttt{\$c}\index{\texttt{\$c} statement},
\texttt{\$v}\index{\texttt{\$v} statement},
\texttt{\$e}\index{\texttt{\$e} statement},
\texttt{\$f}\index{\texttt{\$f} statement},
\texttt{\$a}\index{\texttt{\$a} statement}, and
\texttt{\$p}\index{\texttt{\$p} statement} specify ``statements'' that
end with \texttt{\$.}\,.\index{\texttt{\$.}\ keyword}

The Metamath tokens \texttt{\$c} and \texttt{\$v} each declare\index{constant
declaration}\index{variable declaration} a list of user-defined tokens, called
``math symbols,''\index{math symbol} that the database will reference later
on.  All of the math symbols they define you have seen earlier except the
turnstile symbol \texttt{|-} ($\vdash$)\index{turnstile ({$\,\vdash$})}, which is
commonly used by logicians to mean ``a proof exists for.''  For us
the turnstile is just a
convenient symbol that distinguishes expressions that are axioms\index{axiom}
or theorems\index{theorem} from expressions that are terms or wffs.

The \texttt{\$c} statement declares ``constants''\index{constant} and
the \texttt{\$v} statement declares
``variables''\index{variable}\index{constant declaration}\index{variable
declaration} (or more precisely, metavariables\index{metavariable}).  A
variable may be substituted\index{substitution!variable}\index{variable
substitution} with sequences of math symbols whereas a constant may not
be substituted with anything.

It may seem redundant to require both \texttt{\$c}\index{\texttt{\$c} statement} and
\texttt{\$v}\index{\texttt{\$v} statement} statements (since any math
symbol\index{math symbol} not specified with a \texttt{\$c} statement could be
presumed to be a variable), but this provides for better error checking and
also allows math symbols to be redeclared\index{redeclaration of symbols}
(Section~\ref{scoping}).

The token \texttt{\$f}\index{\texttt{\$f} statement} specifies a
statement called a ``variable-type hypothesis'' (also called a
``floating hypothesis'') and \texttt{\$e}\index{\texttt{\$e} statement}
specifies a ``logical hypothesis'' (also called an ``essential
hypothesis'').\index{hypothesis}\index{variable-type
hypothesis}\index{logical hypothesis}\index{floating
hypothesis}\index{essential hypothesis} The token
\texttt{\$a}\index{\texttt{\$a} statement} specifies an ``axiomatic
assertion,''\index{axiomatic assertion} and
\texttt{\$p}\index{\texttt{\$p} statement} specifies a ``provable
assertion.''\index{provable assertion} To the left of each occurrence of
these four tokens is a ``label''\index{label} that identifies the
hypothesis or assertion for later reference.  For example, the label of
the first axiomatic assertion is \texttt{tze}.  A \texttt{\$f} statement
must contain exactly two math symbols, a constant followed by a
variable.  The \texttt{\$e}, \texttt{\$a}, and \texttt{\$p} statements
each start with a constant followed by, in general, an arbitrary
sequence of math symbols.

Associated with each assertion\index{assertion} is a set of hypotheses
that must be satisfied in order for the assertion to be used in a proof.
These are called the ``mandatory hypotheses''\index{mandatory
hypothesis} of the assertion.  Among those hypotheses whose ``scope''
(described below) includes the assertion, \texttt{\$e} hypotheses are
always mandatory and \texttt{\$f}\index{\texttt{\$f} statement}
hypotheses are mandatory when they share their variable with the
assertion or its \texttt{\$e} hypotheses.  The exact rules for
determining which hypotheses are mandatory are described in detail in
Sections~\ref{frames} and \ref{scoping}.  For example, the mandatory
hypotheses of assertion \texttt{tpl} are \texttt{tt} and \texttt{tr},
whereas assertion \texttt{tze} has no mandatory hypotheses because it
contains no variables and has no \texttt{\$e}\index{\texttt{\$e}
statement} hypothesis.  Metamath's \texttt{show statement}
command\index{\texttt{show statement} command}, described in the next
section, will show you a statement's mandatory hypotheses.

Sometimes we need to make a hypothesis relevant to only certain
assertions.  The set of statements to which a hypothesis is relevant is
called its ``scope.''  The Metamath brackets,
\texttt{\$\char`\{}\index{\texttt{\$\char`\{} and \texttt{\$\char`\}}
keywords} and \texttt{\$\char`\}}, define a ``block''\index{block} that
delimits the scope of any hypothesis contained between them.  The
assertion \texttt{mp} has mandatory hypotheses \texttt{wp}, \texttt{wq},
\texttt{min}, and \texttt{maj}.  The only mandatory hypothesis of
\texttt{th1}, on the other hand, is \texttt{tt}, since \texttt{th1}
occurs outside of the block containing \texttt{min} and \texttt{maj}.

Note that \texttt{\$\char`\{} and \texttt{\$\char`\}} do not affect the
scope of assertions (\texttt{\$a} and \texttt{\$p}).  Assertions are always
available to be referenced by any later proof in the source file.

Each provable assertion (\texttt{\$p}\index{\texttt{\$p} statement}
statement) has two parts.  The first part is the
assertion\index{assertion} itself, which is a sequence of math
symbol\index{math symbol} tokens placed between the \texttt{\$p} token
and a \texttt{\$=}\index{\texttt{\$=} keyword} token.  The second part
is a ``proof,'' which is a list of label tokens placed between the
\texttt{\$=} token and the \texttt{\$.}\index{\texttt{\$.}\ keyword}\
token that ends the statement.\footnote{If you've looked at the
\texttt{set.mm} database, you may have noticed another notation used for
proofs.  The other notation is called ``compressed.''\index{compressed
proof}\index{proof!compressed} It reduces the amount of space needed to
store a proof in the database and is described in
Appendix~\ref{compressed}.  In the example above, we use
``normal''\index{normal proof}\index{proof!normal} notation.} The proof
acts as a series of instructions to the Metamath program, telling it how
to build up the sequence of math symbols contained in the assertion part of
the \texttt{\$p} statement, making use of the hypotheses of the
\texttt{\$p} statement and previous assertions.  The construction takes
place according to precise rules.  If the list of labels in the proof
causes these rules to be violated, or if the final sequence that results
does not match the assertion, the Metamath program will notify you with
an error message.

If you are familiar with reverse Polish notation (RPN), which is sometimes used
on pocket calculators, here in a nutshell is how a proof works.  Each
hypothesis label\index{hypothesis label} in the proof is pushed\index{push}
onto the RPN stack\index{stack}\index{RPN stack} as it is encountered. Each
assertion label\index{assertion label} pops\index{pop} off the stack as many
entries as the referenced assertion has mandatory hypotheses.  Variable
substitutions\index{substitution!variable}\index{variable substitution} are
computed which, when made to the referenced assertion's mandatory hypotheses,
cause these hypotheses to match the stack entries. These same substitutions
are then made to the variables in the referenced assertion itself, which is
then pushed onto the stack.  At the end of the proof, there should be one
stack entry, namely the assertion being proved.  This process is explained in
detail in Section~\ref{proof}.

Metamath's proof notation is not very readable for humans, but it allows the
proof to be stored compactly in a file.  The Metamath\index{Metamath} program
has proof display features that let you see what's going on in a more
readable way, as you will see in the next section.

The rules used in verifying a proof are not based on any built-in syntax of the
symbol sequence in an assertion\index{assertion} nor on any built-in meanings
attached to specific symbol names.  They are based strictly on symbol
matching:  constants\index{constant} must match themselves, and
variables\index{variable} may be replaced with anything that allows a match to
occur.  For example, instead of \texttt{term}, \texttt{0}, and \verb$|-$ we could
have just as well used \texttt{yellow}, \texttt{zero}, and \texttt{provable}, as long
as we did so consistently throughout the database.  Also, we could have used
\texttt{is provable} (two tokens) instead of \verb$|-$ (one token) throughout the
database.  In each of these cases, the proof would be exactly the same.  The
independence of proofs and notation means that you have a lot of flexibility to
change the notation you use without having to change any proofs.

\section{A Trial Run}\label{trialrun}

Now you are ready to try out the Metamath\index{Metamath} program.

On all computer systems, Metamath has a standard ``command line
interface'' (CLI)\index{command line interface (CLI)} that allows you to
interact with it.  You supply commands to the CLI by typing them on the
keyboard and pressing your keyboard's {\em return} key after each line
you enter.  The CLI is designed to be easy to use and has built-in help
features.

The first thing you should do is to use a text editor to create a file
called \texttt{demo0.mm} and type into it the Metamath source shown on
p.~\pageref{demo0}.  Actually, this file is included with your Metamath
software package, so check that first.  If you type it in, make sure
that you save it in the form of ``plain {\sc ascii} text with line
breaks.''  Most word processors will have this feature.

Next you must run the Metamath program.  Depending on your computer
system and how Metamath is installed, this could range from clicking the
mouse on the Metamath icon to typing \texttt{run metamath} to typing
simply \texttt{metamath}.  (Metamath's {\tt help invoke} command describes
alternate ways of invoking the Metamath program.)

When you first enter Metamath\index{Metamath}, it will be at the CLI, waiting
for your input. You will see something like the following on your screen:
\begin{verbatim}
Metamath - Version 0.177 27-Apr-2019
Type HELP for help, EXIT to exit.
MM>
\end{verbatim}
The \texttt{MM>} prompt means that Metamath is waiting for a command.
Command keywords\index{command keyword} are not case sensitive;
we will use lower-case commands in our examples.
The version number and its release date will probably be different on your
system from the one we show above.

The first thing that you need to do is to read in your
database:\index{\texttt{read} command}\footnote{If a directory path is
needed on Unix,\index{Unix file names}\index{file names!Unix} you should
enclose the path/file name in quotes to prevent Metamath from thinking
that the \texttt{/} in the path name is a command qualifier, e.g.,
\texttt{read \char`\"db/set.mm\char`\"}.  Quotes are optional when there
is no ambiguity.}
\begin{verbatim}
MM> read demo0.mm
\end{verbatim}
Remember to press the {\em return} key after entering this command.  If
you omit the file name, Metamath will prompt you for one.   The syntax for
specifying a Macintosh file name path is given in a footnote on
p.~\pageref{includef}.\index{Macintosh file names}\index{file
names!Macintosh}

If there are any syntax errors in the database, Metamath will let you know
when it reads in the file.  The one thing that Metamath does not check when
reading in a database is that all proofs are correct, because this would
slow it down too much.  It is a good idea to periodically verify the proofs in
a database you are making changes to.  To do this, use the following command
(and do it for your \texttt{demo0.mm} file now).  Note that the \texttt{*} is a
``wild card'' meaning all proofs in the file.\index{\texttt{verify proof} command}
\begin{verbatim}
MM> verify proof *
\end{verbatim}
Metamath will report any proofs that are incorrect.

It is often useful to save the information that the Metamath program displays
on the screen. You can save everything that happens on the screen by opening a
log file. You may want to do this before you read in a database so that you
can examine any errors later on.  To open a log file, type
\begin{verbatim}
MM> open log abc.log
\end{verbatim}
This will open a file called \texttt{abc.log}, and everything that appears on the
screen from this point on will be stored in this file.  The name of the log file
is arbitrary. To close the log file, type
\begin{verbatim}
MM> close log
\end{verbatim}

Several commands let you examine what's inside your database.
Section~\ref{exploring} has an overview of some useful ones.  The
\texttt{show labels} command lets you see what statement
labels\index{label} exist.  A \texttt{*} matches any combination of
characters, and \texttt{t*} refers to all labels starting with the
letter \texttt{t}.\index{\texttt{show labels} command} The \texttt{/all}
is a ``command qualifier''\index{command qualifier} that tells Metamath
to include labels of hypotheses.  (To see the syntax explained, type
\texttt{help show labels}.)  Type
\begin{verbatim}
MM> show labels t* /all
\end{verbatim}
Metamath will respond with
\begin{verbatim}
The statement number, label, and type are shown.
3 tt $f       4 tr $f       5 ts $f       8 tze $a
9 tpl $a      19 th1 $p
\end{verbatim}

You can use the \texttt{show statement} command to get information about a
particular statement.\index{\texttt{show statement} command}
For example, you can get information about the statement with label \texttt{mp}
by typing
\begin{verbatim}
MM> show statement mp /full
\end{verbatim}
Metamath will respond with
\begin{verbatim}
Statement 17 is located on line 43 of the file
"demo0.mm".
"Define the modus ponens inference rule"
17 mp $a |- Q $.
Its mandatory hypotheses in RPN order are:
  wp $f wff P $.
  wq $f wff Q $.
  min $e |- P $.
  maj $e |- ( P -> Q ) $.
The statement and its hypotheses require the
      variables:  Q P
The variables it contains are:  Q P
\end{verbatim}
The mandatory hypotheses\index{mandatory hypothesis} and their
order\index{RPN order} are
useful to know when you are trying to understand or debug a proof.

Now you are ready to look at what's really inside our proof.  First, here is
how to look at every step in the proof---not just the ones corresponding to an
ordinary formal proof\index{formal proof}, but also the ones that build up the
formulas that appear in each ordinary formal proof step.\index{\texttt{show
proof} command}
\begin{verbatim}
MM> show proof th1 /lemmon /all
\end{verbatim}

This will display the proof on the screen in the following format:
\begin{verbatim}
 1 tt            $f term t
 2 tze           $a term 0
 3 1,2 tpl       $a term ( t + 0 )
 4 tt            $f term t
 5 3,4 weq       $a wff ( t + 0 ) = t
 6 tt            $f term t
 7 tt            $f term t
 8 6,7 weq       $a wff t = t
 9 tt            $f term t
10 9 a2          $a |- ( t + 0 ) = t
11 tt            $f term t
12 tze           $a term 0
13 11,12 tpl     $a term ( t + 0 )
14 tt            $f term t
15 13,14 weq     $a wff ( t + 0 ) = t
16 tt            $f term t
17 tze           $a term 0
18 16,17 tpl     $a term ( t + 0 )
19 tt            $f term t
20 18,19 weq     $a wff ( t + 0 ) = t
21 tt            $f term t
22 tt            $f term t
23 21,22 weq     $a wff t = t
24 20,23 wim     $a wff ( ( t + 0 ) = t -> t = t )
25 tt            $f term t
26 25 a2         $a |- ( t + 0 ) = t
27 tt            $f term t
28 tze           $a term 0
29 27,28 tpl     $a term ( t + 0 )
30 tt            $f term t
31 tt            $f term t
32 29,30,31 a1   $a |- ( ( t + 0 ) = t -> ( ( t + 0 )
                                     = t -> t = t ) )
33 15,24,26,32 mp  $a |- ( ( t + 0 ) = t -> t = t )
34 5,8,10,33 mp  $a |- t = t
\end{verbatim}

The \texttt{/lemmon} command qualifier specifies what is known as a Lemmon-style
display\index{Lemmon-style proof}\index{proof!Lemmon-style}.  Omitting the
\texttt{/lemmon} qualifier results in a tree-style proof (see
p.~\pageref{treeproof} for an example) that is somewhat less explicit but
easier to follow once you get used to it.\index{tree-style
proof}\index{proof!tree-style}

The first number on each line is the step
number of the proof.  Any numbers that follow are step numbers assigned to the
hypotheses of the statement referenced by that step.  Next is the label of
the statement referenced by the step.  The statement type of the statement
referenced comes next, followed by the math symbol\index{math symbol} string
constructed by the proof up to that step.

The last step, 34, contains the statement that is being proved.

Looking at a small piece of the proof, notice that steps 3 and 4 have
established that
\texttt{( t + 0 )} and \texttt{t} are \texttt{term}\,s, and step 5 makes use of steps 3 and
4 to establish that \texttt{( t + 0 ) = t} is a \texttt{wff}.  Let Metamath
itself tell us in detail what is happening in step 5.  Note that the
``target hypothesis'' refers to where step 5 is eventually used, i.e., in step
34.
\begin{verbatim}
MM> show proof th1 /detailed_step 5
Proof step 5:  wp=weq $a wff ( t + 0 ) = t
This step assigns source "weq" ($a) to target "wp"
($f).  The source assertion requires the hypotheses
"tt" ($f, step 3) and "tr" ($f, step 4).  The parent
assertion of the target hypothesis is "mp" ($a,
step 34).
The source assertion before substitution was:
    weq $a wff t = r
The following substitutions were made to the source
assertion:
    Variable  Substituted with
     t         ( t + 0 )
     r         t
The target hypothesis before substitution was:
    wp $f wff P
The following substitution was made to the target
hypothesis:
    Variable  Substituted with
     P         ( t + 0 ) = t
\end{verbatim}

The full proof just shown is useful to understand what is going on in detail.
However, most of the time you will just be interested in
the ``essential'' or logical steps of a proof, i.e.\ those steps
that correspond to an
ordinary formal proof\index{formal proof}.  If you type
\begin{verbatim}
MM> show proof th1 /lemmon /renumber
\end{verbatim}
you will see\label{demoproof}
\begin{verbatim}
1 a2             $a |- ( t + 0 ) = t
2 a2             $a |- ( t + 0 ) = t
3 a1             $a |- ( ( t + 0 ) = t -> ( ( t + 0 )
                                     = t -> t = t ) )
4 2,3 mp         $a |- ( ( t + 0 ) = t -> t = t )
5 1,4 mp         $a |- t = t
\end{verbatim}
Compare this to the formal proof on p.~\pageref{zeroproof} and
notice the resemblance.
By default Metamath
does not show \texttt{\$f}\index{\texttt{\$f}
statement} hypotheses and everything branching off of them in the proof tree
when the proof is displayed; this makes the proof look more like an ordinary
mathematical proof, which does not normally incorporate the explicit
construction of expressions.
This is called the ``essential'' view
(at one time you had to add the
\texttt{/essential} qualifier in the \texttt{show proof}
command to get this view, but this is now the default).
You can could use the \texttt{/all} qualifier in the \texttt{show
proof} command to also show the explicit construction of expressions.
The \texttt{/renumber} qualifier means to renumber
the steps to correspond only to what is displayed.\index{\texttt{show proof}
command}

To exit Metamath, type\index{\texttt{exit} command}
\begin{verbatim}
MM> exit
\end{verbatim}

\subsection{Some Hints for Using the Command Line Interface}

We will conclude this quick introduction to Metamath\index{Metamath} with some
helpful hints on how to navigate your way through the commands.
\index{command line interface (CLI)}

When you type commands into Metamath's CLI, you only have to type as many
characters of a command keyword\index{command keyword} as are needed to make
it unambiguous.  If you type too few characters, Metamath will tell you what
the choices are.  In the case of the \texttt{read} command, only the \texttt{r} is
needed to specify it unambiguously, so you could have typed\index{\texttt{read}
command}
\begin{verbatim}
MM> r demo0.mm
\end{verbatim}
instead of
\begin{verbatim}
MM> read demo0.mm
\end{verbatim}
In our description, we always show the full command words.  When using the
Metamath CLI commands in a command file (to be read with the \texttt{submit}
command)\index{\texttt{submit} command}, it is good practice to use
the unabbreviated command to ensure your instructions will not become ambiguous
if more commands are added to the Metamath program in the future.

The command keywords\index{command
keyword} are not case sensitive; you may type either \texttt{read} or
\texttt{ReAd}.  File names may or may not be case sensitive, depending on your
computer's operating system.  Metamath label\index{label} and math
symbol\index{math symbol} tokens\index{token} are case-sensitive.

The \texttt{help} command\index{\texttt{help} command} will provide you
with a list of topics you can get help on.  You can then type
\texttt{help} {\em topic} to get help on that topic.

If you are uncertain of a command's spelling, just type as many characters
as you remember of the command.  If you have not typed enough characters to
specify it unambiguously, Metamath will tell you what choices you have.

\begin{verbatim}
MM> show s
         ^
?Ambiguous keyword - please specify SETTINGS,
STATEMENT, or SOURCE.
\end{verbatim}

If you don't know what argument to use as part of a command, type a
\texttt{?}\index{\texttt{]}@\texttt{?}\ in command lines}\ at the
argument position.  Metamath will tell you what it expected there.

\begin{verbatim}
MM> show ?
         ^
?Expected SETTINGS, LABELS, STATEMENT, SOURCE, PROOF,
MEMORY, TRACE_BACK, or USAGE.
\end{verbatim}

Finally, you may type just the first word or words of a command followed
by {\em return}.  Metamath will prompt you for the remaining part of the
command, showing you the choices at each step.  For example, instead of
typing \texttt{show statement th1 /full} you could interact in the
following manner:
\begin{verbatim}
MM> show
SETTINGS, LABELS, STATEMENT, SOURCE, PROOF,
MEMORY, TRACE_BACK, or USAGE <SETTINGS>? st
What is the statement label <th1>?
/ or nothing <nothing>? /
TEX, COMMENT_ONLY, or FULL <TEX>? f
/ or nothing <nothing>?
19 th1 $p |- t = t $= ... $.
\end{verbatim}
After each \texttt{?}\ in this mode, you must give Metamath the
information it requests.  Sometimes Metamath gives you a list of choices
with the default choice indicated by brackets \texttt{< > }. Pressing
{\em return} after the \texttt{?}\ will select the default choice.
Answering anything else will override the default.  Note that the
\texttt{/} in command qualifiers is considered a separate
token\index{token} by the parser, and this is why it is asked for
separately.

\section{Your First Proof}\label{frstprf}

Proofs are developed with the aid of the Proof Assistant\index{Proof
Assistant}.  We will now show you how the proof of theorem \texttt{th1}
was built.  So that you can repeat these steps, we will first have the
Proof Assistant erase the proof in Metamath's source buffer\index{source
buffer}, then reconstruct it.  (The source buffer is the place in memory
where Metamath stores the information in the database when it is
\texttt{read}\index{\texttt{read} command} in.  New or modified proofs
are kept in the source buffer until a \texttt{write source}
command\index{\texttt{write source} command} is issued.)  In practice, you
would place a \texttt{?}\index{\texttt{]}@\texttt{?}\ inside proofs}\
between \texttt{\$=}\index{\texttt{\$=} keyword} and
\texttt{\$.}\index{\texttt{\$.}\ keyword}\ in the database to indicate
to Metamath\index{Metamath} that the proof is unknown, and that would be
your starting point.  Whenever the \texttt{verify proof} command encounters
a proof with a \texttt{?}\ in place of a proof step, the statement is
identified as not proved.

When I first started creating Metamath proofs, I would write down
on a piece of paper the complete
formal proof\index{formal proof} as it would appear
in a \texttt{show proof} command\index{\texttt{show proof} command}; see
the display of \texttt{show proof th1 /lemmon /re\-num\-ber} above as an
example.  After you get used to using the Proof Assistant\index{Proof
Assistant} you may get to a point where you can ``see'' the proof in your mind
and let the Proof Assistant guide you in filling in the details, at least for
simpler proofs, but until you gain that experience you may find it very useful
to write down all the details in advance.
Otherwise you may waste a lot of time as you let it take you down a wrong path.
However, others do not find this approach as helpful.
For example, Thomas Brendan Leahy\index{Leahy, Thomas Brendan}
finds that it is more helpful to him to interactively
work backward from a machine-readable statement.
David A. Wheeler\index{Wheeler, David A.}
writes down a general approach, but develops the proof
interactively by switching between
working forwards (from hypotheses and facts likely to be useful) and
backwards (from the goal) until the forwards and backwards approaches meet.
In the end, use whatever approach works for you.

A proof is developed with the Proof Assistant by working backwards, starting
with the theorem\index{theorem} to be proved, and assigning each unknown step
with a theorem or hypothesis until no more unknown steps remain.  The Proof
Assistant will not let you make an assignment unless it can be ``unified''
with the unknown step.  This means that a
substitution\index{substitution!variable}\index{variable substitution} of
variables exists that will make the assignment match the unknown step.  On the
other hand, in the middle of a proof, when working backwards, often more than
one unification\index{unification} (set of substitutions) is possible, since
there is not enough information available at that point to uniquely establish
it.  In this case you can tell Metamath which unification to choose, or you
can continue to assign unknown steps until enough information is available to
make the unification unique.

We will assume you have entered Metamath and read in the database as described
above.  The following dialog shows how the proof was developed.  For more
details on what some of the commands do, refer to Section~\ref{pfcommands}.
\index{\texttt{prove} command}

\begin{verbatim}
MM> prove th1
Entering the Proof Assistant.  Type HELP for help, EXIT
to exit.  You will be working on the proof of statement th1:
  $p |- t = t
Note:  The proof you are starting with is already complete.
MM-PA>
\end{verbatim}

The \verb/MM-PA>/ prompt means we are inside the Proof
Assistant.\index{Proof Assistant} Most of the regular Metamath commands
(\texttt{show statement}, etc.) are still available if you need them.

\begin{verbatim}
MM-PA> delete all
The entire proof was deleted.
\end{verbatim}

We have deleted the whole proof so we can start from scratch.

\begin{verbatim}
MM-PA> show new_proof/lemmon/all
1 ?              $? |- t = t
\end{verbatim}

The \texttt{show new{\char`\_}proof} command\index{\texttt{show
new{\char`\_}proof} command} is like \texttt{show proof} except that we
don't specify a statement; instead, the proof we're working on is
displayed.

\begin{verbatim}
MM-PA> assign 1 mp
To undo the assignment, DELETE STEP 5 and INITIALIZE, UNIFY
if needed.
3   min=?  $? |- $2
4   maj=?  $? |- ( $2 -> t = t )
\end{verbatim}

The \texttt{assign} command\index{\texttt{assign} command} above means
``assign step 1 with the statement whose label is \texttt{mp}.''  Note
that step renumbering will constantly occur as you assign steps in the
middle of a proof; in general all steps from the step you assign until
the end of the proof will get moved up.  In this case, what used to be
step 1 is now step 5, because the (partial) proof now has five steps:
the four hypotheses of the \texttt{mp} statement and the \texttt{mp}
statement itself.  Let's look at all the steps in our partial proof:

\begin{verbatim}
MM-PA> show new_proof/lemmon/all
1 ?              $? wff $2
2 ?              $? wff t = t
3 ?              $? |- $2
4 ?              $? |- ( $2 -> t = t )
5 1,2,3,4 mp     $a |- t = t
\end{verbatim}

The symbol \texttt{\$2} is a temporary variable\index{temporary
variable} that represents a symbol sequence not yet known.  In the final
proof, all temporary variables will be eliminated.  The general format
for a temporary variable is \texttt{\$} followed by an integer.  Note
that \texttt{\$} is not a legal character in a math symbol (see
Section~\ref{dollardollar}, p.~\pageref{dollardollar}), so there will
never be a naming conflict between real symbols and temporary variables.

Unknown steps 1 and 2 are constructions of the two wffs used by the
modus ponens rule.  As you will see at the end of this section, the
Proof Assistant\index{Proof Assistant} can usually figure these steps
out by itself, and we will not have to worry about them.  Therefore from
here on we will display only the ``essential'' hypotheses, i.e.\ those
steps that correspond to traditional formal proofs\index{formal proof}.

\begin{verbatim}
MM-PA> show new_proof/lemmon
3 ?              $? |- $2
4 ?              $? |- ( $2 -> t = t )
5 3,4 mp         $a |- t = t
\end{verbatim}

Unknown steps 3 and 4 are the ones we must focus on.  They correspond to the
minor and major premises of the modus ponens rule.  We will assign them as
follows.  Notice that because of the step renumbering that takes place
after an assignment, it is advantageous to assign unknown steps in reverse
order, because earlier steps will not get renumbered.

\begin{verbatim}
MM-PA> assign 4 mp
To undo the assignment, DELETE STEP 8 and INITIALIZE, UNIFY
if needed.
3   min=?  $? |- $2
6     min=?  $? |- $4
7     maj=?  $? |- ( $4 -> ( $2 -> t = t ) )
\end{verbatim}

We are now going to describe an obscure feature that you will probably
never use but should be aware of.  The Metamath language allows empty
symbol sequences to be substituted for variables, but in most formal
systems this feature is never used.  One of the few examples where is it
used is the MIU-system\index{MIU-system} described in
Appendix~\ref{MIU}.  But such systems are rare, and by default this
feature is turned off in the Proof Assistant.  (It is always allowed for
{\tt verify proof}.)  Let us turn it on and see what
happens.\index{\texttt{set empty{\char`\_}substitution} command}

\begin{verbatim}
MM-PA> set empty_substitution on
Substitutions with empty symbol sequences is now allowed.
\end{verbatim}

With this feature enabled, more unifications will be
ambiguous\index{ambiguous unification}\index{unification!ambiguous} in
the middle of a proof, because
substitution\index{substitution!variable}\index{variable substitution}
of variables with empty symbol sequences will become an additional
possibility.  Let's see what happens when we make our next assignment.

\begin{verbatim}
MM-PA> assign 3 a2
There are 2 possible unifications.  Please select the correct
    one or Q if you want to UNIFY later.
Unify:  |- $6
 with:  |- ( $9 + 0 ) = $9
Unification #1 of 2 (weight = 7):
  Replace "$6" with "( + 0 ) ="
  Replace "$9" with ""
  Accept (A), reject (R), or quit (Q) <A>? r
\end{verbatim}

The first choice presented is the wrong one.  If we had selected it,
temporary variable \texttt{\$6} would have been assigned a truncated
wff, and temporary variable \texttt{\$9} would have been assigned an
empty sequence (which is not allowed in our system).  With this choice,
eventually we would reach a point where we would get stuck because
we would end up with steps impossible to prove.  (You may want to
try it.)  We typed \texttt{r} to reject the choice.

\begin{verbatim}
Unification #2 of 2 (weight = 21):
  Replace "$6" with "( $9 + 0 ) = $9"
  Accept (A), reject (R), or quit (Q) <A>? q
To undo the assignment, DELETE STEP 4 and INITIALIZE, UNIFY
if needed.
 7     min=?  $? |- $8
 8     maj=?  $? |- ( $8 -> ( $6 -> t = t ) )
\end{verbatim}

The second choice is correct, and normally we would type \texttt{a}
to accept it.  But instead we typed \texttt{q} to show what will happen:
it will leave the step with an unknown unification, which can be
seen as follows:

\begin{verbatim}
MM-PA> show new_proof/not_unified
 4   min    $a |- $6
        =a2  = |- ( $9 + 0 ) = $9
\end{verbatim}

Later we can unify this with the \texttt{unify}
\texttt{all/interactive} command.

The important point to remember is that occasionally you will be
presented with several unification choices while entering a proof, when
the program determines that there is not enough information yet to make
an unambiguous choice automatically (and this can happen even with
\texttt{set empty{\char`\_}substitution} turned off).  Usually it is
obvious by inspection which choice is correct, since incorrect ones will
tend to be meaningless fragments of wffs.  In addition, the correct
choice will usually be the first one presented, unlike our example
above.

Enough of this digression.  Let us go back to the default setting.

\begin{verbatim}
MM-PA> set empty_substitution off
The ability to substitute empty expressions for variables
has been turned off.  Note that this may make the Proof
Assistant too restrictive in some cases.
\end{verbatim}

If we delete the proof, start over, and get to the point where
we digressed above, there will no longer be an ambiguous unification.

\begin{verbatim}
MM-PA> assign 3 a2
To undo the assignment, DELETE STEP 4 and INITIALIZE, UNIFY
if needed.
 7     min=?  $? |- $4
 8     maj=?  $? |- ( $4 -> ( ( $5 + 0 ) = $5 -> t = t ) )
\end{verbatim}

Let us look at our proof so far, and continue.

\begin{verbatim}
MM-PA> show new_proof/lemmon
 4 a2            $a |- ( $5 + 0 ) = $5
 7 ?             $? |- $4
 8 ?             $? |- ( $4 -> ( ( $5 + 0 ) = $5 -> t = t ) )
 9 7,8 mp        $a |- ( ( $5 + 0 ) = $5 -> t = t )
10 4,9 mp        $a |- t = t
MM-PA> assign 8 a1
To undo the assignment, DELETE STEP 11 and INITIALIZE, UNIFY
if needed.
 7     min=?  $? |- ( t + 0 ) = t
MM-PA> assign 7 a2
To undo the assignment, DELETE STEP 8 and INITIALIZE, UNIFY
if needed.
MM-PA> show new_proof/lemmon
 4 a2            $a |- ( t + 0 ) = t
 8 a2            $a |- ( t + 0 ) = t
12 a1            $a |- ( ( t + 0 ) = t -> ( ( t + 0 ) = t ->
                                                    t = t ) )
13 8,12 mp       $a |- ( ( t + 0 ) = t -> t = t )
14 4,13 mp       $a |- t = t
\end{verbatim}

Now all temporary variables and unknown steps have been eliminated from the
``essential'' part of the proof.  When this is achieved, the Proof
Assistant\index{Proof Assistant} can usually figure out the rest of the proof
automatically.  (Note that the \texttt{improve} command can occasionally be
useful for filling in essential steps as well, but it only tries to make use
of statements that introduce no new variables in their hypotheses, which is
not the case for \texttt{mp}. Also it will not try to improve steps containing
temporary variables.)  Let's look at the complete proof, then run
the \texttt{improve} command, then look at it again.

\begin{verbatim}
MM-PA> show new_proof/lemmon/all
 1 ?             $? wff ( t + 0 ) = t
 2 ?             $? wff t = t
 3 ?             $? term t
 4 3 a2          $a |- ( t + 0 ) = t
 5 ?             $? wff ( t + 0 ) = t
 6 ?             $? wff ( ( t + 0 ) = t -> t = t )
 7 ?             $? term t
 8 7 a2          $a |- ( t + 0 ) = t
 9 ?             $? term ( t + 0 )
10 ?             $? term t
11 ?             $? term t
12 9,10,11 a1    $a |- ( ( t + 0 ) = t -> ( ( t + 0 ) = t ->
                                                    t = t ) )
13 5,6,8,12 mp   $a |- ( ( t + 0 ) = t -> t = t )
14 1,2,4,13 mp   $a |- t = t
\end{verbatim}

\begin{verbatim}
MM-PA> improve all
A proof of length 1 was found for step 11.
A proof of length 1 was found for step 10.
A proof of length 3 was found for step 9.
A proof of length 1 was found for step 7.
A proof of length 9 was found for step 6.
A proof of length 5 was found for step 5.
A proof of length 1 was found for step 3.
A proof of length 3 was found for step 2.
A proof of length 5 was found for step 1.
Steps 1 and above have been renumbered.
CONGRATULATIONS!  The proof is complete.  Use SAVE
NEW_PROOF to save it.  Note:  The Proof Assistant does
not detect $d violations.  After saving the proof, you
should verify it with VERIFY PROOF.
\end{verbatim}

The \texttt{save new{\char`\_}proof} command\index{\texttt{save
new{\char`\_}proof} command} will save the proof in the database.  Here
we will just display it in a form that can be clipped out of a log file
and inserted manually into the database source file with a text
editor.\index{normal proof}\index{proof!normal}

\begin{verbatim}
MM-PA> show new_proof/normal
---------Clip out the proof below this line:
      tt tze tpl tt weq tt tt weq tt a2 tt tze tpl tt weq
      tt tze tpl tt weq tt tt weq wim tt a2 tt tze tpl tt
      tt a1 mp mp $.
---------The proof of 'th1' to clip out ends above this line.
\end{verbatim}

There is another proof format called ``compressed''\index{compressed
proof}\index{proof!compressed} that you will see in databases.  It is
not important to understand how it is encoded but only to recognize it
when you see it.  Its only purpose is to reduce storage requirements for
large proofs.  A compressed proof can always be converted to a normal
one and vice-versa, and the Metamath \texttt{show proof}
commands\index{\texttt{show proof} command} work equally well with
compressed proofs.  The compressed proof format is described in
Appendix~\ref{compressed}.

\begin{verbatim}
MM-PA> show new_proof/compressed
---------Clip out the proof below this line:
      ( tze tpl weq a2 wim a1 mp ) ABCZADZAADZAEZJJKFLIA
      AGHH $.
---------The proof of 'th1' to clip out ends above this line.
\end{verbatim}

Now we will exit the Proof Assistant.  Since we made changes to the proof,
it will warn us that we have not saved it.  In this case, we don't care.

\begin{verbatim}
MM-PA> exit
Warning:  You have not saved changes to the proof.
Do you want to EXIT anyway (Y, N) <N>? y
Exiting the Proof Assistant.
Type EXIT again to exit Metamath.
\end{verbatim}

The Proof Assistant\index{Proof Assistant} has several other commands
that can help you while creating proofs.  See Section~\ref{pfcommands}
for a list of them.

A command that is often useful is \texttt{minimize{\char`\_}with
*/brief}, which tries to shorten the proof.  It can make the process
more efficient by letting you write a somewhat ``sloppy'' proof then
clean up some of the fine details of optimization for you (although it
can't perform miracles such as restructuring the overall proof).

\section{A Note About Editing a Data\-base File}

Once your source file contains proofs, there are some restrictions on
how you can edit it so that the proofs remain valid.  Pay particular
attention to these rules, since otherwise you can lose a lot of work.
It is a good idea to periodically verify all proofs with \texttt{verify
proof *} to ensure their integrity.

If your file contains only normal (as opposed to compressed) proofs, the
main rule is that you may not change the order of the mandatory
hypotheses\index{mandatory hypothesis} of any statement referenced in a
later proof.  For example, if you swap the order of the major and minor
premise in the modus ponens rule, all proofs making use of that rule
will become incorrect.  The \texttt{show statement}
command\index{\texttt{show statement} command} will show you the
mandatory hypotheses of a statement and their order.

If a statement has a compressed proof, you also must not change the
order of {\em its} mandatory hypotheses.  The compressed proof format
makes use of this information as part of the compression technique.
Note that swapping the names of two variables in a theorem will change
the order of its mandatory hypotheses.

The safest way to edit a statement, say \texttt{mytheorem}, is to
duplicate it then rename the original to \texttt{mytheoremOLD}
throughout the database.  Once the edited version is re-proved, all
statements referencing \texttt{mytheoremOLD} can be updated in the Proof
Assistant using \texttt{minimize{\char`\_}with
mytheorem
/allow{\char`\_}growth}.\index{\texttt{minimize{\char`\_}with} command}
% 3/10/07 Note: line-breaking the above results in duplicate index entries

\chapter{Abstract Mathematics Revealed}\label{fol}

\section{Logic and Set Theory}\label{logicandsettheory}

\begin{quote}
  {\em Set theory can be viewed as a form of exact theology.}
  \flushright\sc  Rudy Rucker\footnote{\cite{Barrow}, p.~31.}\\
\end{quote}\index{Rucker, Rudy}

Despite its seeming complexity, all of standard mathematics, no matter how
deep or abstract, can amazingly enough be derived from a relatively small set
of axioms\index{axiom} or first principles. The development of these axioms is
among the most impressive and important accomplishments of mathematics in the
20th century. Ultimately, these axioms can be broken down into a set of rules
for manipulating symbols that any technically oriented person can follow.

We will not spend much time trying to convey a deep, higher-level
understanding of the meaning of the axioms. This kind of understanding
requires some mathematical sophistication as well as an understanding of the
philosophy underlying the foundations of mathematics and typically develops
over time as you work with mathematics.  Our goal, instead, is to give you the
immediate ability to follow how theorems\index{theorem} are derived from the
axioms and from other theorems.  This will be similar to learning the syntax
of a computer language, which lets you follow the details in a program but
does not necessarily give you the ability to write non-trivial programs on
your own, an ability that comes with practice. For now don't be alarmed by
abstract-sounding names of the axioms; just focus on the rules for
manipulating the symbols, which follow the simple conventions of the
Metamath\index{Metamath} language.

The axioms that underlie all of standard mathematics consist of axioms of logic
and axioms of set theory. The axioms of logic are divided into two
subcategories, propositional calculus\index{propositional calculus} (sometimes
called sentential logic\index{sentential logic}) and predicate calculus
(sometimes called first-order logic\index{first-order logic}\index{quantifier
theory}\index{predicate calculus} or quantifier theory).  Propositional
calculus is a prerequisite for predicate calculus, and predicate calculus is a
prerequisite for set theory.  The version of set theory most commonly used is
Zermelo--Fraenkel set theory\index{Zermelo--Fraenkel set theory}\index{set theory}
with the axiom of choice,
often abbreviated as ZFC\index{ZFC}.

Here in a nutshell is what the axioms are all about in an informal way. The
connection between this description and symbols we will show you won't be
immediately apparent and in principle needn't ever be.  Our description just
tries to summarize what mathematicians think about when they work with the
axioms.

Logic is a set of rules that allow us determine truths given other truths.
Put another way,
logic is more or less the translation of what we would consider common sense
into a rigorous set of axioms.\index{axioms of logic}  Suppose $\varphi$,
$\psi$, and $\chi$ (the Greek letters phi, psi, and chi) represent statements
that are either true or false, and $x$ is a variable\index{variable!in predicate
calculus} ranging over some group of mathematical objects (sets, integers,
real numbers, etc.). In mathematics, a ``statement'' really means a formula,
and $\psi$ could be for example ``$x = 2$.''
Propositional calculus\index{propositional calculus}
allows us to use variables that are either true or false
and make deductions such as
``if $\varphi$ implies $\psi$ and $\psi$ implies $\chi$, then $\varphi$
implies $\chi$.''
Predicate calculus\index{predicate calculus}
extends propositional calculus by also allowing us
to discuss statements about objects (not just true and false values), including
statements about ``all'' or ``at least one'' object.
For example, predicate calculus allows to say,
``if $\varphi$ is true for all $x$, then $\varphi$ is true for some $x$.''
The logic used in \texttt{set.mm} is standard classical logic
(as opposed to other logic systems like intuitionistic logic).

Set theory\index{set theory} has to do with the manipulation of objects and
collections of objects, specifically the abstract, imaginary objects that
mathematics deals with, such as numbers. Everything that is claimed to exist
in mathematics is considered to be a set.  A set called the empty
set\index{empty set} contains nothing.  We represent the empty set by
$\varnothing$.  Many sets can be built up from the empty set.  There is a set
represented by $\{\varnothing\}$ that contains the empty set, another set
represented by $\{\varnothing,\{\varnothing\}\}$ that contains this set as
well as the empty set, another set represented by $\{\{\varnothing\}\}$ that
contains just the set that contains the empty set, and so on ad infinitum. All
mathematical objects, no matter how complex, are defined as being identical to
certain sets: the integer\index{integer} 0 is defined as the empty set, the
integer 1 is defined as $\{\varnothing\}$, the integer 2 is defined as
$\{\varnothing,\{\varnothing\}\}$.  (How these definitions were chosen doesn't
matter now, but the idea behind it is that these sets have the properties we
expect of integers once suitable operations are defined.)  Mathematical
operations, such as addition, are defined in terms of operations on
sets---their union\index{set union}, intersection\index{set intersection}, and
so on---operations you may have used in elementary school when you worked
with groups of apples and oranges.

With a leap of faith, the axioms also postulate the existence of infinite
sets\index{infinite set}, such as the set of all non-negative integers ($0, 1,
2,\ldots$, also called ``natural numbers''\index{natural number}).  This set
can't be represented with the brace notation\index{brace notation} we just
showed you, but requires a more complicated notation called ``class
abstraction.''\index{class abstraction}\index{abstraction class}  For
example, the infinite set $\{ x |
\mbox{``$x$ is a natural number''} \} $ means the ``set of all objects $x$
such that $x$ is a natural number'' i.e.\ the set of natural numbers; here,
``$x$ is a natural number'' is a rather complicated formula when broken down
into the primitive symbols.\label{expandom}\footnote{The statement ``$x$ is a
natural number'' is formally expressed as ``$x \in \omega$,'' where $\in$
(stylized epsilon) means ``is in'' or ``is an element of'' and $\omega$
(omega) means ``the set of natural numbers.''  When ``$x\in\omega$'' is
completely expanded in terms of the primitive symbols of set theory, the
result is  $\lnot$ $($ $\lnot$ $($ $\forall$ $z$ $($ $\lnot$ $\forall$ $w$ $($
$z$ $\in$ $w$ $\rightarrow$ $\lnot$ $w$ $\in$ $x$ $)$ $\rightarrow$ $z$ $\in$
$x$ $)$ $\rightarrow$ $($ $\forall$ $z$ $($ $\lnot$ $($ $\forall$ $w$ $($ $w$
$\in$ $z$ $\rightarrow$ $w$ $\in$ $x$ $)$ $\rightarrow$ $\forall$ $w$ $\lnot$
$w$ $\in$ $z$ $)$ $\rightarrow$ $\lnot$ $\forall$ $w$ $($ $w$ $\in$ $z$
$\rightarrow$ $\lnot$ $\forall$ $v$ $($ $v$ $\in$ $z$ $\rightarrow$ $\lnot$
$v$ $\in$ $w$ $)$ $)$ $)$ $\rightarrow$ $\lnot$ $\forall$ $z$ $\forall$ $w$
$($ $\lnot$ $($ $z$ $\in$ $x$ $\rightarrow$ $\lnot$ $w$ $\in$ $x$ $)$
$\rightarrow$ $($ $\lnot$ $z$ $\in$ $w$ $\rightarrow$ $($ $\lnot$ $z$ $=$ $w$
$\rightarrow$ $w$ $\in$ $z$ $)$ $)$ $)$ $)$ $)$ $\rightarrow$ $\lnot$
$\forall$ $y$ $($ $\lnot$ $($ $\lnot$ $($ $\forall$ $z$ $($ $\lnot$ $\forall$
$w$ $($ $z$ $\in$ $w$ $\rightarrow$ $\lnot$ $w$ $\in$ $y$ $)$ $\rightarrow$
$z$ $\in$ $y$ $)$ $\rightarrow$ $($ $\forall$ $z$ $($ $\lnot$ $($ $\forall$
$w$ $($ $w$ $\in$ $z$ $\rightarrow$ $w$ $\in$ $y$ $)$ $\rightarrow$ $\forall$
$w$ $\lnot$ $w$ $\in$ $z$ $)$ $\rightarrow$ $\lnot$ $\forall$ $w$ $($ $w$
$\in$ $z$ $\rightarrow$ $\lnot$ $\forall$ $v$ $($ $v$ $\in$ $z$ $\rightarrow$
$\lnot$ $v$ $\in$ $w$ $)$ $)$ $)$ $\rightarrow$ $\lnot$ $\forall$ $z$
$\forall$ $w$ $($ $\lnot$ $($ $z$ $\in$ $y$ $\rightarrow$ $\lnot$ $w$ $\in$
$y$ $)$ $\rightarrow$ $($ $\lnot$ $z$ $\in$ $w$ $\rightarrow$ $($ $\lnot$ $z$
$=$ $w$ $\rightarrow$ $w$ $\in$ $z$ $)$ $)$ $)$ $)$ $\rightarrow$ $($
$\forall$ $z$ $\lnot$ $z$ $\in$ $y$ $\rightarrow$ $\lnot$ $\forall$ $w$ $($
$\lnot$ $($ $w$ $\in$ $y$ $\rightarrow$ $\lnot$ $\forall$ $z$ $($ $w$ $\in$
$z$ $\rightarrow$ $\lnot$ $z$ $\in$ $y$ $)$ $)$ $\rightarrow$ $\lnot$ $($
$\lnot$ $\forall$ $z$ $($ $w$ $\in$ $z$ $\rightarrow$ $\lnot$ $z$ $\in$ $y$
$)$ $\rightarrow$ $w$ $\in$ $y$ $)$ $)$ $)$ $)$ $\rightarrow$ $x$ $\in$ $y$
$)$ $)$ $)$. Section~\ref{hierarchy} shows the hierarchy of definitions that
leads up to this expression.}\index{stylized epsilon ($\in$)}\index{omega
($\omega$)}  Actually, the primitive symbols don't even include the brace
notation.  The brace notation is a high-level definition, which you can find in
Section~\ref{hierarchy}.

Interestingly, the arithmetic of integers\index{integer} and
rationals\index{rational number} can be developed without appealing to the
existence of an infinite set, whereas the arithmetic of real
numbers\index{real number} requires it.

Each variable\index{variable!in set theory} in the axioms of set theory
represents an arbitrary set, and the axioms specify the legal kinds of things
you can do with these variables at a very primitive level.

Now, you may think that numbers and arithmetic are a lot more intuitive and
fundamental than sets and therefore should be the foundation of mathematics.
What is really the case is that you've dealt with numbers all your life and
are comfortable with a few rules for manipulating them such as addition and
multiplication.  Those rules only cover a small portion of what can be done
with numbers and only a very tiny fraction of the rest of mathematics.  If you
look at any elementary book on number theory, you will quickly become lost if
these are the only rules that you know.  Even though such books may present a
list of ``axioms''\index{axiom} for arithmetic, the ability to use the axioms
and to understand proofs of theorems\index{theorem} (facts) about numbers
requires an implicit mathematical talent that frustrates many people
from studying abstract mathematics.  The kind of mathematics that most people
know limits them to the practical, everyday usage of blindly manipulating
numbers and formulas, without any understanding of why those rules are correct
nor any ability to go any further.  For example, do you know why multiplying
two negative numbers yields a positive number?  Starting with set theory, you
will also start off blindly manipulating symbols according to the rules we give
you, but with the advantage that these rules will allow you, in principle, to
access {\em all} of mathematics, not just a tiny part of it.

Of course, concrete examples are often helpful in the learning process. For
example, you can verify that $2\cdot 3=3 \cdot 2$ by actually grouping
objects and can easily ``see'' how it generalizes to $x\cdot y = y\cdot x$,
even though you might not be able to rigorously prove it.  Similarly, in set
theory it can be helpful to understand how the axioms of set theory apply to
(and are correct for) small finite collections of objects.  You should be aware
that in set theory intuition can be misleading for infinite collections, and
rigorous proofs become more important.  For example, while $x\cdot y = y\cdot
x$ is correct for finite ordinals (which are the natural numbers), it is not
usually true for infinite ordinals.

\section{The Axioms for All of Mathematics}

In this section\index{axioms for mathematics}, we will show you the axioms
for all of standard mathematics (i.e.\ logic and set theory) as they are
traditionally presented.  The traditional presentation is useful for someone
with the mathematical experience needed to correctly manipulate high-level
abstract concepts.  For someone without this talent, knowing how to actually
make use of these axioms can be difficult.  The purpose of this section is to
allow you to see how the version of the axioms used in the standard
Metamath\index{Metamath} database \texttt{set.mm}\index{set
theory database (\texttt{set.mm})} relates to  the typical version
in textbooks, and also to give you an informal feel for them.

\subsection{Propositional Calculus}

Propositional calculus\index{propositional calculus} concerns itself with
statements that can be interpreted as either true or false.  Some examples of
statements (outside of mathematics) that are either true or false are ``It is
raining today'' and ``The United States has a female president.'' In
mathematics, as we mentioned, statements are really formulas.

In propositional calculus, we don't care what the statements are.  We also
treat a logical combination of statements, such as ``It is raining today and
the United States has a female president,'' no differently from a single
statement.  Statements and their combinations are called well-formed formulas
(wffs)\index{well-formed formula (wff)}.  We define wffs only in terms of
other wffs and don't define what a ``starting'' wff is.  As is common practice
in the literature, we use Greek letters to represent wffs.

Specifically, suppose $\varphi$ and $\psi$ are wffs.  Then the combinations
$\varphi\rightarrow\psi$ (``$\varphi$ implies $\psi$,'' also read ``if
$\varphi$ then $\psi$'')\index{implication ($\rightarrow$)} and $\lnot\varphi$
(``not $\varphi$'')\index{negation ($\lnot$)} are also wffs.

The three axioms of propositional calculus\index{axioms of propositional
calculus} are all wffs of the following form:\footnote{A remarkable result of
C.~A.~Meredith\index{Meredith, C. A.} squeezes these three axioms into the
single axiom $((((\varphi\rightarrow \psi)\rightarrow(\neg \chi\rightarrow\neg
\theta))\rightarrow \chi )\rightarrow \tau)\rightarrow((\tau\rightarrow
\varphi)\rightarrow(\theta\rightarrow \varphi))$ \cite{CAMeredith},
which is believed to be the shortest possible.}
\begin{center}
     $\varphi\rightarrow(\psi\rightarrow \varphi)$\\

     $(\varphi\rightarrow (\psi\rightarrow \chi))\rightarrow
((\varphi\rightarrow  \psi)\rightarrow (\varphi\rightarrow \chi))$\\

     $(\neg \varphi\rightarrow \neg\psi)\rightarrow (\psi\rightarrow
\varphi)$
\end{center}

These three axioms are widely used.
They are attributed to Jan {\L}ukasiewicz
(pronounced woo-kah-SHAY-vitch) and was popularized by Alonzo Church,
who called it system P2. (Thanks to Ted Ulrich for this information.)

There are an infinite number of axioms, one for each possible
wff\index{well-formed formula (wff)} of the above form.  (For this reason,
axioms such as the above are often called ``axiom schemes.''\index{axiom
scheme})  Each Greek letter in the axioms may be substituted with a more
complex wff to result in another axiom.  For example, substituting
$\neg(\varphi\rightarrow\chi)$ for $\varphi$ in the first axiom yields
$\neg(\varphi\rightarrow\chi)\rightarrow(\psi\rightarrow
\neg(\varphi\rightarrow\chi))$, which is still an axiom.

To deduce new true statements (theorems\index{theorem}) from the axioms, a
rule\index{rule} called ``modus ponens''\index{modus ponens} is used.  This
rule states that if the wff $\varphi$ is an axiom or a theorem, and the wff
$\varphi\rightarrow\psi$ is an axiom or a theorem, then the wff $\psi$ is also
a theorem\index{theorem}.

As a non-mathematical example of modus ponens, suppose we have proved (or
taken as an axiom) ``Bob is a man'' and separately have proved (or taken as
an axiom) ``If Bob is a man, then Bob is a human.''  Using the rule of modus
ponens, we can logically deduce, ``Bob is a human.''

From Metamath's\index{Metamath} point of view, the axioms and the rule of
modus ponens just define a mechanical means for deducing new true statements
from existing true statements, and that is the complete content of
propositional calculus as far as Metamath is concerned.  You can read a logic
textbook to gain a better understanding of their meaning, or you can just let
their meaning slowly become apparent to you after you use them for a while.

It is actually rather easy to check to see if a formula is a theorem of
propositional calculus.  Theorems of propositional calculus are also called
``tautologies.''\index{tautology}  The technique to check whether a formula is
a tautology is called the ``truth table method,''\index{truth table} and it
works like this.  A wff $\varphi\rightarrow\psi$ is false whenever $\varphi$ is true
and $\psi$ is false.  Otherwise it is true.  A wff $\lnot\varphi$ is false
whenever $\varphi$ is true and false otherwise. To verify a tautology such as
$\varphi\rightarrow(\psi\rightarrow \varphi)$, you break it down into sub-wffs and
construct a truth table that accounts for all possible combinations of true
and false assigned to the wff metavariables:
\begin{center}\begin{tabular}{|c|c|c|c|}\hline
\mbox{$\varphi$} & \mbox{$\psi$} & \mbox{$\psi\rightarrow\varphi$}
    & \mbox{$\varphi\rightarrow(\psi\rightarrow \varphi)$} \\ \hline \hline
              T   &  T    &      T       &        T    \\ \hline
              T   &  F    &      T       &        T    \\ \hline
              F   &  T    &      F       &        T    \\ \hline
              F   &  F    &      T       &        T    \\ \hline
\end{tabular}\end{center}
If all entries in the last column are true, the formula is a tautology.

Now, the truth table method doesn't tell you how to prove the tautology from
the axioms, but only that a proof exists.  Finding an actual proof (especially
one that is short and elegant) can be challenging.  Methods do exist for
automatically generating proofs in propositional calculus, but the proofs that
result can sometimes be very long.  In the Metamath \texttt{set.mm}\index{set
theory database (\texttt{set.mm})} database, most
or all proofs were created manually.

Section \ref{metadefprop} discusses various definitions
that make propositional calculus easier to use.
For example, we define:

\begin{itemize}
\item $\varphi \vee \psi$
  is true if either $\varphi$ or $\psi$ (or both) are true
  (this is disjunction\index{disjunction ($\vee$)}
  aka logical {\sc or}\index{logical {\sc or} ($\vee$)}).

\item $\varphi \wedge \psi$
  is true if both $\varphi$ and $\psi$ are true
  (this is conjunction\index{conjunction ($\wedge$)}
  aka logical {\sc and}\index{logical {\sc and} ($\wedge$)}).

\item $\varphi \leftrightarrow \psi$
  is true if $\varphi$ and $\psi$ have the same value, that is,
  they are both true or both false
  (this is the biconditional\index{biconditional ($\leftrightarrow$)}).
\end{itemize}

\subsection{Predicate Calculus}

Predicate calculus\index{predicate calculus} introduces the concept of
``individual variables,''\index{variable!in predicate calculus}\index{individual
variable} which
we will usually just call ``variables.''
These variables can represent something other than true or false (wffs),
and will always represent sets when we get to set theory.  There are also
three new symbols $\forall$\index{universal quantifier ($\forall$)},
$=$\index{equality ($=$)}, and $\in$\index{stylized epsilon ($\in$)},
read ``for all,'' ``equals,'' and ``is an element of''
respectively.  We will represent variables with the letters $x$, $y$, $z$, and
$w$, as is common practice in the literature.
For example, $\forall x \varphi$ means ``for all possible values of
$x$, $\varphi$ is true.''

In predicate calculus, we extend the definition of a wff\index{well-formed
formula (wff)}.  If $\varphi$ is a wff and $x$ and $y$ are variables, then
$\forall x \, \varphi$, $x=y$, and $x\in y$ are wffs. Note that these three new
types of wffs can be considered ``starting'' wffs from which we can build
other wffs with $\rightarrow$ and $\neg$ .  The concept of a starting wff was
absent in propositional calculus.  But starting wff or not, all we are really
concerned with is whether our wffs are correctly constructed according to
these mechanical rules.

A quick aside:
To prevent confusion, it might be best at this point to think of the variables
of Metamath\index{Metamath} as ``metavariables,''\index{metavariable} because
they are not quite the same as the variables we are introducing here.  A
(meta)variable in Metamath can be a wff or an individual variable, as well
as many other things; in general, it represents a kind of place holder for an
unspecified sequence of math symbols\index{math symbol}.

Unlike propositional calculus, no decision procedure\index{decision procedure}
analogous to the truth table method exists (nor theoretically can exist) that
will definitely determine whether a formula is a theorem of predicate
calculus.  Much of the work in the field of automated theorem
proving\index{automated theorem proving} has been dedicated to coming up with
clever heuristics for proving theorems of predicate calculus, but they can
never be guaranteed to work always.

Section \ref{metadefpred} discusses various definitions
that make predicate calculus easier to use.
For example, we define
$\exists x \varphi$ to mean
``there exists at least one possible value of $x$ where $\varphi$ is true.''

We now turn to looking at how predicate calculus can be formally
represented.

\subsubsection{Common Axioms}

There is a new rule of inference in predicate calculus:  if $\varphi$ is
an axiom or a theorem, then $\forall x \,\varphi$ is also a
theorem\index{theorem}.  This is called the rule of
``generalization.''\index{rule of generalization}
This is easily represented in Metamath.

In standard texts of logic, there are often two axioms of predicate
calculus\index{axioms of predicate calculus}:
\begin{center}
  $\forall x \,\varphi ( x ) \rightarrow \varphi ( y )$,
      where ``$y$ is properly substituted for $x$.''\\
  $\forall x ( \varphi \rightarrow \psi )\rightarrow ( \varphi \rightarrow
    \forall x\, \psi )$,
    where ``$x$ is not free in $\varphi$.''
\end{center}

Now at first glance, this seems simple:  just two axioms.  However,
conditional clauses are attached to each axiom describing requirements that
may seem puzzling to you.  In addition, the first axiom puts a variable symbol
in parentheses after each wff, seemingly violating our definition of a
wff\index{well-formed formula (wff)}; this is just an informal way of
referring to some arbitrary variable that may occur in the wff.  The
conditional clauses do, of course, have a precise meaning, but as it turns out
the precise meaning is somewhat complicated and awkward to formalize in a
way that a computer can handle easily.  Unlike propositional calculus, a
certain amount of mathematical sophistication and practice is needed to be
able to easily grasp and manipulate these concepts correctly.

Predicate calculus may be presented with or without axioms for
equality\index{axioms of equality}\index{equality ($=$)}. We will require the
axioms of equality as a prerequisite for the version of set theory we will
use.  The axioms for equality, when included, are often represented using these
two axioms:
\begin{center}
$x=x$\\ \ \\
$x=y\rightarrow (\varphi(x,x)\rightarrow\varphi(x,y))$ where ``$\varphi(x,y)$
   arises from $\varphi(x,x)$ by replacing some, but not necessarily all,
   free\index{free variable}
   occurrences of $x$ by $y$,\\ provided that $y$ is free for $x$
   in $\varphi(x,x)$.'' \end{center}
% (Mendelson p. 95)
The first equality axiom is simple, but again,
the condition on the second one is
somewhat awkward to implement on a computer.

\subsubsection{Tarski System S2}

Of course, we are not the first to notice the complications of these
predicate calculus axioms when being rigorous.

Well-known logician Alfred Tarski published in 1965
a system he called system S2\cite[p.~77]{Tarski1965}.
Tarski's system is \textit{exactly equivalent} to the traditional textbook
formalization, but (by clever use of equality axioms) it eliminates the
latter's primitive notions of ``proper substitution'' and ``free variable,''
replacing them with direct substitution and the notion of a variable
not occurring in a formula (which we express with distinct variable
constraints).

In advocating his system, Tarski wrote, ``The relatively complicated
character of [free variables and proper substitution] is a source
of certain inconveniences of both practical and theoretical nature;
this is clearly experienced both in teaching an elementary course of
mathematical logic and in formalizing the syntax of predicate logic for
some theoretical purposes''\cite[p.~61]{Tarski1965}\index{Tarski, Alfred}.

\subsubsection{Developing a Metamath Representation}

The standard textbook axioms of predicate calculus are somewhat
cumbersome to implement on a computer because of the complex notions of
``free variable''\index{free variable} and ``proper
substitution.''\index{proper substitution}\index{substitution!proper}
While it is possible to use the Metamath\index{Metamath} language to
implement these concepts, we have chosen not to implement them
as primitive constructs in the
\texttt{set.mm} set theory database.  Instead, we have eliminated them
within the axioms
by carefully crafting the axioms so as to avoid them,
building on Tarski's system S2.  This makes it
easy for a beginner to follow the steps in a proof without knowing any
advanced concepts other than the simple concept of
replacing\index{substitution!variable}\index{variable substitution}
variables with expressions.

In order to develop the concepts of free variable and proper
substitution from the axioms, we use an additional
Metamath statement type called ``disjoint variable
restriction''\index{disjoint variables} that we have not encountered
before.  In the context of the axioms, the statement \texttt{\$d} $ x\,
y$\index{\texttt{\$d} statement} simply means that $x$ and $y$ must be
distinct\index{distinct variables}, i.e.\ they may not be simultaneously
substituted\index{substitution!variable}\index{variable substitution}
with the same variable.  The statement \texttt{\$d} $ x\, \varphi$ means
variable $x$ must not occur in wff $\varphi$.  For the precise
definition of \texttt{\$d}, see Section~\ref{dollard}.

\subsubsection{Metamath representation}

The Metamath axiom system for predicate calculus
defined in set.mm uses Tarski's system S2.
As noted above, this has a different representation
than the traditional textbook formalization,
but it is \textit{exactly equivalent} to the textbook formalization,
and it is \textit{much} easier to work with.
This is reproduced as system S3 in Section 6 of
Megill's formalization \cite{Megill}\index{Megill, Norman}.

There is one exception, Tarski's axiom of existence,
which we label as axiom ax-6.
In the case of ax-6, Tarski's version is weaker because it includes a
distinct variable proviso. If we wish, we can also weaken our version
in this way and still have a metalogically complete system. Theorem
ax6 shows this by deriving, in the presence of the other axioms, our
ax-6 from Tarski's weaker version ax6v. However, we chose the stronger
version for our system because it is simpler to state and easier to use.

Tarski's system was designed for proving specific theorems rather than
more general theorem schemes. However, theorem schemes are much more
efficient than specific theorems for building a body of mathematical
knowledge, since they can be reused with different instances as
needed. While Tarski does derive some theorem schemes from his axioms,
their proofs require concepts that are ``outside'' of the system, such as
induction on formula length. The verification of such proofs is difficult
to automate in a proof verifier. (Specifically, Tarski treats the formulas
of his system as set-theoretical objects. In order to verify the proofs
of his theorem schemes, a proof verifier would need a significant amount
of set theory built into it.)

The Metamath axiom system for predicate calculus extends
Tarski's system to eliminate this difficulty. The additional
``auxilliary'' axiom
schemes (as we will call them in this section; see below) endow Tarski's
system with a nice property we call
metalogical completeness \cite[Remark 9.6]{Megill}\index{Megill, Norman}.
As a result, we can prove any theorem scheme
expressable in the ``simple metalogic'' of Tarski's system by using
only Metamath's direct substitution rule applied to the axiom system
(and no other metalogical or set-theoretical notions ``outside'' of the
system). Simple metalogic consists of schemes containing wff metavariables
(with no arguments) and/or set (also called ``individual'') metavariables,
accompanied by optional provisos each stating that two specified set
metavariables must be distinct or that a specified set metavariable may
not occur in a specified wff metavariable. Metamath's logic and set theory
axiom and rule schemes are all examples of simple metalogic. The schemes
of traditional predicate calculus with equality are examples which are
not simple metalogic, because they use wff metavariables with arguments
and have ``free for'' and ``not free in'' side conditions.

A rigorous justification for this system, using an older but
exactly equivalent set of axioms, can be
found in \cite{Megill}\index{Megill, Norman}.

This allows us to
take a different approach in the Metamath\index{Metamath} database
\texttt{set.mm}\index{set theory database (\texttt{set.mm})}.  We do not
directly use the primitive notions of ``free variable''\index{free variable}
and ``proper substitution''\index{proper
substitution}\index{substitution!proper} at all as primitive constructs.
Instead, we use a set
of axioms that are almost as simple to manipulate as those of
propositional calculus.  Our axiom system avoids complex primitive
notions by effectively embedding the complexity into the axioms
themselves.  As a result, we will end up with a larger number of axioms,
but they are ideally suited for a computer language such as Metamath.
(Section~\ref{metaaxioms} shows these axioms.)

We will not elaborate further
on the ``free variable'' and ``proper substitution''
concepts here.  You may consult
\cite[ch.\ 3--4]{Hamilton}\index{Hamilton, Alan G.} (as well as
many other books) for a precise explanation
of these concepts.  If you intend to do serious mathematical work, it is wise
to become familiar with the traditional textbook approach; even though the
concepts embedded in their axioms require a higher level of sophistication,
they can be more practical to deal with on an everyday, informal basis.  Even
if you are just developing Metamath proofs, familiarity with the traditional
approach can help you arrive at a proof outline much faster, which you can
then convert to the detail required by Metamath.

We do develop proper substitution rules later on, but in set.mm
they are defined as derived constructs; they are not primitives.

You should also note that our system of predicate calculus is specifically
tailored for set theory; thus there are only two specific predicates $=$ and
$\in$ and no functions\index{function!in predicate calculus}
or constants\index{constant!in predicate calculus} unlike more general systems.
We later add these.

\subsection{Set Theory}

Traditional Zermelo--Fraenkel set theory\index{Zermelo--Fraenkel set
theory}\index{set theory} with the Axiom of Choice
has 10 axioms, which can be expressed in the
language of predicate calculus.  In this section, we will list only the
names and brief English descriptions of these axioms, since we will give
you the precise formulas used by the Metamath\index{Metamath} set theory
database \texttt{set.mm} later on.

In the descriptions of the axioms, we assume that $x$, $y$, $z$, $w$, and $v$
represent sets.  These are the same as the variables\index{variable!in set
theory} in our predicate calculus system above, except that now we informally
think of the variables as ranging over sets.  Note that the terms
``object,''\index{object} ``set,''\index{set} ``element,''\index{element}
``collection,''\index{collection} and ``family''\index{family} are synonymous,
as are ``is an element of,'' ``is a member of,''\index{member} ``is contained
in,'' and ``belongs to.''  The different terms are used for convenience; for
example, ``a collection of sets'' is less confusing than ``a set of sets.''
A set $x$ is said to be a ``subset''\index{subset} of $y$ if every element of
$x$ is also an element of $y$; we also say $x$ is ``included in''
$y$.

The axioms are very general and apply to almost any conceivable mathematical
object, and this level of abstraction can be overwhelming at first.  To gain an
intuitive feel, it can be helpful to draw a picture illustrating the concept;
for example, a circle containing dots could represent a collection of sets,
and a smaller circle drawn inside the circle could represent a subset.
Overlapping circles can illustrate intersection and union.  Circles that
illustrate the concepts of set theory are frequently used in elementary
textbooks and are called Venn diagrams\index{Venn diagram}.\index{axioms of
set theory}

1. Axiom of Extensionality:  Two sets are identical if they contain the same
   elements.\index{Axiom of Extensionality}

2. Axiom of Pairing:  The set $\{ x , y \}$ exists.\index{Axiom of Pairing}

3. Axiom of Power Sets:  The power set of a set (the collection of all of
   its subsets) exists.  For example, the power set of $\{x,y\}$ is
   $\{\varnothing,\{x\},\{y\},\{x,y\}\}$ and it exists.\index{Axiom
of Power Sets}

4. Axiom of the Null Set:  The empty set $\varnothing$ exists.\index{Axiom of
the Null Set}

5. Axiom of Union:  The union of a set (the set containing the elements of
   its members) exists.  For example, the union of $\{\{x,y\},\{z\}\}$ is
 $\{x,y,z\}$ and
   it exists.\index{Axiom of Union}

6. Axiom of Regularity:  Roughly, no set can contain itself, nor can there
   be membership ``loops,'' such as a set being an
   element of one of its members.\index{Axiom of Regularity}

7. Axiom of Infinity:  An infinite set exists.  An example of an infinite
   set is the set of all
   integers.\index{Axiom of Infinity}

8. Axiom of Separation:  The set exists that is obtained by restricting $x$
   with some property.  For example, if the set of all integers exists,
   then the set of all even integers exists.\index{Axiom of Separation}

9. Axiom of Replacement:  The range of a function whose domain is restricted
   to the elements of a set $x$, is also a set.  For example, there
   is a function
   from integers (the function's domain) to their squares (its
   range).  If we
   restrict the domain to even integers, its range will become the set of
   squares of even integers, so this axiom asserts that the set of
    squares of even numbers exists.  Technical note:  In general, the
   ``function'' need not be a set but can be a proper class.
   \index{Axiom of Replacement}

10. Axiom of Choice:  Let $x$ be a set whose members are pairwise
  disjoint\index{disjoint sets} (i.e,
  whose members contain no elements in common).  Then there exists another
  set containing one element from each member of $x$.  For
  example, if $x$ is
  $\{\{y,z\},\{w,v\}\}$, where $y$, $z$, $w$, and $v$ are
  different sets, then a set such as $\{z,w\}$
  exists (but the axiom doesn't tell
  us which one).  (Actually the Axiom
  of Choice is redundant if the set $x$, as in this example, has a finite
  number of elements.)\index{Axiom of Choice}

The Axiom of Choice is usually considered an extension of ZF set theory rather
than a proper part of it.  It is sometimes considered philosophically
controversial because it specifies the existence of a set without specifying
what the set is. Constructive logics, including intuitionistic logic,
do not accept the axiom of choice.
Since there is some lingering controversy, we often prefer proofs that do
not use the axiom of choice (where there is a known alternative), and
in some cases we will use weaker axioms than the full axiom of choice.
That said, the axiom of choice is a powerful and widely-accepted tool,
so we do use it when needed.
ZF set theory that includes the Axiom of Choice is
called Zermelo--Fraenkel set theory with choice (ZFC\index{ZFC set theory}).

When expressed symbolically, the Axiom of Separation and the Axiom of
Replacement contain wff symbols and therefore each represent infinitely many
axioms, one for each possible wff. For this reason, they are often called
axiom schemes\index{axiom scheme}\index{well-formed formula (wff)}.

It turns out that the Axiom of the Null Set, the Axiom of Pairing, and the
Axiom of Separation can be derived from the other axioms and are therefore
unnecessary, although they tend to be included in standard texts for various
reasons (historical, philosophical, and possibly because some authors may not
know this).  In the Metamath\index{Metamath} set theory database, these
redundant axioms are derived from the other ones instead of truly
being considered axioms.
This is in keeping with our general goal of minimizing the number of
axioms we must depend on.

\subsection{Other Axioms}

Above we qualified the phrase ``all of mathematics'' with ``essentially.''
The main important missing piece is the ability to do category theory,
which requires huge sets (inaccessible cardinals) larger than those
postulated by the ZFC axioms. The Tarski--Grothendieck Axiom postulates
the existence of such sets.
Note that this is the same axiom used by Mizar for supporting
category theory.
The Tarski--Grothendieck axiom
can be viewed as a very strong replacement of the Axiom of Infinity,
the Axiom of Choice, and the Axiom of Power Sets.
The \texttt{set.mm} database includes this axiom; see the database
for details about it.
Again, we only use this axiom when we need to.
You are only likely to encounter or use this axiom if you are doing
category theory, since its use is highly specialized,
so we will not list the Tarsky-Grothendieck axiom
in the short list of axioms below.

Can there be even more axioms?
Of course.
G\"{o}del showed that no finite set of axioms or axiom schemes can completely
describe any consistent theory strong enough to include arithmetic.
But practically speaking, the ones above are the accepted foundation that
almost all mathematicians explicitly or implicitly base their work on.

\section{The Axioms in the Metamath Language}\label{metaaxioms}

Here we list the axioms as they appear in
\texttt{set.mm}\index{set theory database (\texttt{set.mm})} so you can
look them up there easily.  Incidentally, the \texttt{show statement
/tex} command\index{\texttt{show statement} command} was used to
typeset them.

%macros from show statement /tex
\newbox\mlinebox
\newbox\mtrialbox
\newbox\startprefix  % Prefix for first line of a formula
\newbox\contprefix  % Prefix for continuation line of a formula
\def\startm{  % Initialize formula line
  \setbox\mlinebox=\hbox{\unhcopy\startprefix}
}
\def\m#1{  % Add a symbol to the formula
  \setbox\mtrialbox=\hbox{\unhcopy\mlinebox $\,#1$}
  \ifdim\wd\mtrialbox>\hsize
    \box\mlinebox
    \setbox\mlinebox=\hbox{\unhcopy\contprefix $\,#1$}
  \else
    \setbox\mlinebox=\hbox{\unhbox\mtrialbox}
  \fi
}
\def\endm{  % Output the last line of a formula
  \box\mlinebox
}

% \SLASH for \ , \TOR for \/ (text OR), \TAND for /\ (text and)
% This embeds a following forced space to force the space.
\newcommand\SLASH{\char`\\~}
\newcommand\TOR{\char`\\/~}
\newcommand\TAND{/\char`\\~}
%
% Macro to output metamath raw text.
% This assumes \startprefix and \contprefix are set.
% NOTE: "\" is tricky to escape, use \SLASH, \TOR, and \TAND inside.
% Any use of "$ { ~ ^" must be escaped; ~ and ^ must be escaped specially.
% We escape { and } for consistency.
% For more about how this macro written, see:
% https://stackoverflow.com/questions/4073674/
% how-to-disable-indentation-in-particular-section-in-latex/4075706
% Use frenchspacing, or "e." will get an extra space after it.
\newlength\mystoreparindent
\newlength\mystorehangindent
\newenvironment{mmraw}{%
\setlength{\mystoreparindent}{\the\parindent}
\setlength{\mystorehangindent}{\the\hangindent}
\setlength{\parindent}{0pt} % TODO - we'll put in the \startprefix instead
\setlength{\hangindent}{\wd\the\contprefix}
\begin{flushleft}
\begin{frenchspacing}
\begin{tt}
{\unhcopy\startprefix}%
}{%
\end{tt}
\end{frenchspacing}
\end{flushleft}
\setlength{\parindent}{\mystoreparindent}
\setlength{\hangindent}{\mystorehangindent}
\vskip 1ex
}

\needspace{5\baselineskip}
\subsection{Propositional Calculus}\label{propcalc}\index{axioms of
propositional calculus}

\needspace{2\baselineskip}
Axiom of Simplification.\label{ax1}

\setbox\startprefix=\hbox{\tt \ \ ax-1\ \$a\ }
\setbox\contprefix=\hbox{\tt \ \ \ \ \ \ \ \ \ \ }
\startm
\m{\vdash}\m{(}\m{\varphi}\m{\rightarrow}\m{(}\m{\psi}\m{\rightarrow}\m{\varphi}\m{)}
\m{)}
\endm

\needspace{3\baselineskip}
\noindent Axiom of Distribution.

\setbox\startprefix=\hbox{\tt \ \ ax-2\ \$a\ }
\setbox\contprefix=\hbox{\tt \ \ \ \ \ \ \ \ \ \ }
\startm
\m{\vdash}\m{(}\m{(}\m{\varphi}\m{\rightarrow}\m{(}\m{\psi}\m{\rightarrow}\m{\chi}
\m{)}\m{)}\m{\rightarrow}\m{(}\m{(}\m{\varphi}\m{\rightarrow}\m{\psi}\m{)}\m{
\rightarrow}\m{(}\m{\varphi}\m{\rightarrow}\m{\chi}\m{)}\m{)}\m{)}
\endm

\needspace{2\baselineskip}
\noindent Axiom of Contraposition.

\setbox\startprefix=\hbox{\tt \ \ ax-3\ \$a\ }
\setbox\contprefix=\hbox{\tt \ \ \ \ \ \ \ \ \ \ }
\startm
\m{\vdash}\m{(}\m{(}\m{\lnot}\m{\varphi}\m{\rightarrow}\m{\lnot}\m{\psi}\m{)}\m{
\rightarrow}\m{(}\m{\psi}\m{\rightarrow}\m{\varphi}\m{)}\m{)}
\endm


\needspace{4\baselineskip}
\noindent Rule of Modus Ponens.\label{axmp}\index{modus ponens}

\setbox\startprefix=\hbox{\tt \ \ min\ \$e\ }
\setbox\contprefix=\hbox{\tt \ \ \ \ \ \ \ \ \ }
\startm
\m{\vdash}\m{\varphi}
\endm

\setbox\startprefix=\hbox{\tt \ \ maj\ \$e\ }
\setbox\contprefix=\hbox{\tt \ \ \ \ \ \ \ \ \ }
\startm
\m{\vdash}\m{(}\m{\varphi}\m{\rightarrow}\m{\psi}\m{)}
\endm

\setbox\startprefix=\hbox{\tt \ \ ax-mp\ \$a\ }
\setbox\contprefix=\hbox{\tt \ \ \ \ \ \ \ \ \ \ \ }
\startm
\m{\vdash}\m{\psi}
\endm


\needspace{7\baselineskip}
\subsection{Axioms of Predicate Calculus with Equality---Tarski's S2}\index{axioms of predicate calculus}

\needspace{3\baselineskip}
\noindent Rule of Generalization.\index{rule of generalization}

\setbox\startprefix=\hbox{\tt \ \ ax-g.1\ \$e\ }
\setbox\contprefix=\hbox{\tt \ \ \ \ \ \ \ \ \ \ \ \ }
\startm
\m{\vdash}\m{\varphi}
\endm

\setbox\startprefix=\hbox{\tt \ \ ax-gen\ \$a\ }
\setbox\contprefix=\hbox{\tt \ \ \ \ \ \ \ \ \ \ \ \ }
\startm
\m{\vdash}\m{\forall}\m{x}\m{\varphi}
\endm

\needspace{2\baselineskip}
\noindent Axiom of Quantified Implication.

\setbox\startprefix=\hbox{\tt \ \ ax-4\ \$a\ }
\setbox\contprefix=\hbox{\tt \ \ \ \ \ \ \ \ \ \ }
\startm
\m{\vdash}\m{(}\m{\forall}\m{x}\m{(}\m{\forall}\m{x}\m{\varphi}\m{\rightarrow}\m{
\psi}\m{)}\m{\rightarrow}\m{(}\m{\forall}\m{x}\m{\varphi}\m{\rightarrow}\m{
\forall}\m{x}\m{\psi}\m{)}\m{)}
\endm

\needspace{3\baselineskip}
\noindent Axiom of Distinctness.

% Aka: Add $d x ph $.
\setbox\startprefix=\hbox{\tt \ \ ax-5\ \$a\ }
\setbox\contprefix=\hbox{\tt \ \ \ \ \ \ \ \ \ \ }
\startm
\m{\vdash}\m{(}\m{\varphi}\m{\rightarrow}\m{\forall}\m{x}\m{\varphi}\m{)}\m{where}\m{ }\m{\$d}\m{ }\m{x}\m{ }\m{\varphi}\m{ }\m{(}\m{x}\m{ }\m{does}\m{ }\m{not}\m{ }\m{occur}\m{ }\m{in}\m{ }\m{\varphi}\m{)}
\endm

\needspace{2\baselineskip}
\noindent Axiom of Existence.

\setbox\startprefix=\hbox{\tt \ \ ax-6\ \$a\ }
\setbox\contprefix=\hbox{\tt \ \ \ \ \ \ \ \ \ \ }
\startm
\m{\vdash}\m{(}\m{\forall}\m{x}\m{(}\m{x}\m{=}\m{y}\m{\rightarrow}\m{\forall}
\m{x}\m{\varphi}\m{)}\m{\rightarrow}\m{\varphi}\m{)}
\endm

\needspace{2\baselineskip}
\noindent Axiom of Equality.

\setbox\startprefix=\hbox{\tt \ \ ax-7\ \$a\ }
\setbox\contprefix=\hbox{\tt \ \ \ \ \ \ \ \ \ \ }
\startm
\m{\vdash}\m{(}\m{x}\m{=}\m{y}\m{\rightarrow}\m{(}\m{x}\m{=}\m{z}\m{
\rightarrow}\m{y}\m{=}\m{z}\m{)}\m{)}
\endm

\needspace{2\baselineskip}
\noindent Axiom of Left Equality for Binary Predicate.

\setbox\startprefix=\hbox{\tt \ \ ax-8\ \$a\ }
\setbox\contprefix=\hbox{\tt \ \ \ \ \ \ \ \ \ \ \ }
\startm
\m{\vdash}\m{(}\m{x}\m{=}\m{y}\m{\rightarrow}\m{(}\m{x}\m{\in}\m{z}\m{
\rightarrow}\m{y}\m{\in}\m{z}\m{)}\m{)}
\endm

\needspace{2\baselineskip}
\noindent Axiom of Right Equality for Binary Predicate.

\setbox\startprefix=\hbox{\tt \ \ ax-9\ \$a\ }
\setbox\contprefix=\hbox{\tt \ \ \ \ \ \ \ \ \ \ \ }
\startm
\m{\vdash}\m{(}\m{x}\m{=}\m{y}\m{\rightarrow}\m{(}\m{z}\m{\in}\m{x}\m{
\rightarrow}\m{z}\m{\in}\m{y}\m{)}\m{)}
\endm


\needspace{4\baselineskip}
\subsection{Axioms of Predicate Calculus with Equality---Auxiliary}\index{axioms of predicate calculus - auxiliary}

\needspace{2\baselineskip}
\noindent Axiom of Quantified Negation.

\setbox\startprefix=\hbox{\tt \ \ ax-10\ \$a\ }
\setbox\contprefix=\hbox{\tt \ \ \ \ \ \ \ \ \ \ }
\startm
\m{\vdash}\m{(}\m{\lnot}\m{\forall}\m{x}\m{\lnot}\m{\forall}\m{x}\m{\varphi}\m{
\rightarrow}\m{\varphi}\m{)}
\endm

\needspace{2\baselineskip}
\noindent Axiom of Quantifier Commutation.

\setbox\startprefix=\hbox{\tt \ \ ax-11\ \$a\ }
\setbox\contprefix=\hbox{\tt \ \ \ \ \ \ \ \ \ \ }
\startm
\m{\vdash}\m{(}\m{\forall}\m{x}\m{\forall}\m{y}\m{\varphi}\m{\rightarrow}\m{
\forall}\m{y}\m{\forall}\m{x}\m{\varphi}\m{)}
\endm

\needspace{3\baselineskip}
\noindent Axiom of Substitution.

\setbox\startprefix=\hbox{\tt \ \ ax-12\ \$a\ }
\setbox\contprefix=\hbox{\tt \ \ \ \ \ \ \ \ \ \ \ }
\startm
\m{\vdash}\m{(}\m{\lnot}\m{\forall}\m{x}\m{\,x}\m{=}\m{y}\m{\rightarrow}\m{(}
\m{x}\m{=}\m{y}\m{\rightarrow}\m{(}\m{\varphi}\m{\rightarrow}\m{\forall}\m{x}\m{(}
\m{x}\m{=}\m{y}\m{\rightarrow}\m{\varphi}\m{)}\m{)}\m{)}\m{)}
\endm

\needspace{3\baselineskip}
\noindent Axiom of Quantified Equality.

\setbox\startprefix=\hbox{\tt \ \ ax-13\ \$a\ }
\setbox\contprefix=\hbox{\tt \ \ \ \ \ \ \ \ \ \ \ }
\startm
\m{\vdash}\m{(}\m{\lnot}\m{\forall}\m{z}\m{\,z}\m{=}\m{x}\m{\rightarrow}\m{(}
\m{\lnot}\m{\forall}\m{z}\m{\,z}\m{=}\m{y}\m{\rightarrow}\m{(}\m{x}\m{=}\m{y}
\m{\rightarrow}\m{\forall}\m{z}\m{\,x}\m{=}\m{y}\m{)}\m{)}\m{)}
\endm

% \noindent Axiom of Quantifier Substitution
%
% \setbox\startprefix=\hbox{\tt \ \ ax-c11n\ \$a\ }
% \setbox\contprefix=\hbox{\tt \ \ \ \ \ \ \ \ \ \ \ }
% \startm
% \m{\vdash}\m{(}\m{\forall}\m{x}\m{\,x}\m{=}\m{y}\m{\rightarrow}\m{(}\m{\forall}
% \m{x}\m{\varphi}\m{\rightarrow}\m{\forall}\m{y}\m{\varphi}\m{)}\m{)}
% \endm
%
% \noindent Axiom of Distinct Variables. (This axiom requires
% that two individual variables
% be distinct\index{\texttt{\$d} statement}\index{distinct
% variables}.)
%
% \setbox\startprefix=\hbox{\tt \ \ \ \ \ \ \ \ \$d\ }
% \setbox\contprefix=\hbox{\tt \ \ \ \ \ \ \ \ \ \ \ }
% \startm
% \m{x}\m{\,}\m{y}
% \endm
%
% \setbox\startprefix=\hbox{\tt \ \ ax-c16\ \$a\ }
% \setbox\contprefix=\hbox{\tt \ \ \ \ \ \ \ \ \ \ \ }
% \startm
% \m{\vdash}\m{(}\m{\forall}\m{x}\m{\,x}\m{=}\m{y}\m{\rightarrow}\m{(}\m{\varphi}\m{
% \rightarrow}\m{\forall}\m{x}\m{\varphi}\m{)}\m{)}
% \endm

% \noindent Axiom of Quantifier Introduction (2).  (This axiom requires
% that the individual variable not occur in the
% wff\index{\texttt{\$d} statement}\index{distinct variables}.)
%
% \setbox\startprefix=\hbox{\tt \ \ \ \ \ \ \ \ \$d\ }
% \setbox\contprefix=\hbox{\tt \ \ \ \ \ \ \ \ \ \ \ }
% \startm
% \m{x}\m{\,}\m{\varphi}
% \endm
% \setbox\startprefix=\hbox{\tt \ \ ax-5\ \$a\ }
% \setbox\contprefix=\hbox{\tt \ \ \ \ \ \ \ \ \ \ \ }
% \startm
% \m{\vdash}\m{(}\m{\varphi}\m{\rightarrow}\m{\forall}\m{x}\m{\varphi}\m{)}
% \endm

\subsection{Set Theory}\label{mmsettheoryaxioms}

In order to make the axioms of set theory\index{axioms of set theory} a little
more compact, there are several definitions from logic that we make use of
implicitly, namely, ``logical {\sc and},''\index{conjunction ($\wedge$)}
\index{logical {\sc and} ($\wedge$)} ``logical equivalence,''\index{logical
equivalence ($\leftrightarrow$)}\index{biconditional ($\leftrightarrow$)} and
``there exists.''\index{existential quantifier ($\exists$)}

\begin{center}\begin{tabular}{rcl}
  $( \varphi \wedge \psi )$ &\mbox{stands for}& $\neg ( \varphi
     \rightarrow \neg \psi )$\\
  $( \varphi \leftrightarrow \psi )$& \mbox{stands
     for}& $( ( \varphi \rightarrow \psi ) \wedge
     ( \psi \rightarrow \varphi ) )$\\
  $\exists x \,\varphi$ &\mbox{stands for}& $\neg \forall x \neg \varphi$
\end{tabular}\end{center}

In addition, the axioms of set theory require that all variables be
dis\-tinct,\index{distinct variables}\footnote{Set theory axioms can be
devised so that {\em no} variables are required to be distinct,
provided we replace \texttt{ax-c16} with an axiom stating that ``at
least two things exist,'' thus
making \texttt{ax-5} the only other axiom requiring the
\texttt{\$d} statement.  These axioms are unconventional and are not
presented here, but they can be found on the \url{http://metamath.org}
web site.  See also the Comment on
p.~\pageref{nodd}.}\index{\texttt{\$d} statement} thus we also assume:
\begin{center}
  \texttt{\$d }$x\,y\,z\,w$
\end{center}

\needspace{2\baselineskip}
\noindent Axiom of Extensionality.\index{Axiom of Extensionality}

\setbox\startprefix=\hbox{\tt \ \ ax-ext\ \$a\ }
\setbox\contprefix=\hbox{\tt \ \ \ \ \ \ \ \ \ \ \ \ }
\startm
\m{\vdash}\m{(}\m{\forall}\m{x}\m{(}\m{x}\m{\in}\m{y}\m{\leftrightarrow}\m{x}
\m{\in}\m{z}\m{)}\m{\rightarrow}\m{y}\m{=}\m{z}\m{)}
\endm

\needspace{3\baselineskip}
\noindent Axiom of Replacement.\index{Axiom of Replacement}

\setbox\startprefix=\hbox{\tt \ \ ax-rep\ \$a\ }
\setbox\contprefix=\hbox{\tt \ \ \ \ \ \ \ \ \ \ \ \ }
\startm
\m{\vdash}\m{(}\m{\forall}\m{w}\m{\exists}\m{y}\m{\forall}\m{z}\m{(}\m{%
\forall}\m{y}\m{\varphi}\m{\rightarrow}\m{z}\m{=}\m{y}\m{)}\m{\rightarrow}\m{%
\exists}\m{y}\m{\forall}\m{z}\m{(}\m{z}\m{\in}\m{y}\m{\leftrightarrow}\m{%
\exists}\m{w}\m{(}\m{w}\m{\in}\m{x}\m{\wedge}\m{\forall}\m{y}\m{\varphi}\m{)}%
\m{)}\m{)}
\endm

\needspace{2\baselineskip}
\noindent Axiom of Union.\index{Axiom of Union}

\setbox\startprefix=\hbox{\tt \ \ ax-un\ \$a\ }
\setbox\contprefix=\hbox{\tt \ \ \ \ \ \ \ \ \ \ \ }
\startm
\m{\vdash}\m{\exists}\m{x}\m{\forall}\m{y}\m{(}\m{\exists}\m{x}\m{(}\m{y}\m{
\in}\m{x}\m{\wedge}\m{x}\m{\in}\m{z}\m{)}\m{\rightarrow}\m{y}\m{\in}\m{x}\m{)}
\endm

\needspace{2\baselineskip}
\noindent Axiom of Power Sets.\index{Axiom of Power Sets}

\setbox\startprefix=\hbox{\tt \ \ ax-pow\ \$a\ }
\setbox\contprefix=\hbox{\tt \ \ \ \ \ \ \ \ \ \ \ \ }
\startm
\m{\vdash}\m{\exists}\m{x}\m{\forall}\m{y}\m{(}\m{\forall}\m{x}\m{(}\m{x}\m{
\in}\m{y}\m{\rightarrow}\m{x}\m{\in}\m{z}\m{)}\m{\rightarrow}\m{y}\m{\in}\m{x}
\m{)}
\endm

\needspace{3\baselineskip}
\noindent Axiom of Regularity.\index{Axiom of Regularity}

\setbox\startprefix=\hbox{\tt \ \ ax-reg\ \$a\ }
\setbox\contprefix=\hbox{\tt \ \ \ \ \ \ \ \ \ \ \ \ }
\startm
\m{\vdash}\m{(}\m{\exists}\m{x}\m{\,x}\m{\in}\m{y}\m{\rightarrow}\m{\exists}
\m{x}\m{(}\m{x}\m{\in}\m{y}\m{\wedge}\m{\forall}\m{z}\m{(}\m{z}\m{\in}\m{x}\m{
\rightarrow}\m{\lnot}\m{z}\m{\in}\m{y}\m{)}\m{)}\m{)}
\endm

\needspace{3\baselineskip}
\noindent Axiom of Infinity.\index{Axiom of Infinity}

\setbox\startprefix=\hbox{\tt \ \ ax-inf\ \$a\ }
\setbox\contprefix=\hbox{\tt \ \ \ \ \ \ \ \ \ \ \ \ \ \ \ }
\startm
\m{\vdash}\m{\exists}\m{x}\m{(}\m{y}\m{\in}\m{x}\m{\wedge}\m{\forall}\m{y}%
\m{(}\m{y}\m{\in}\m{x}\m{\rightarrow}\m{\exists}\m{z}\m{(}\m{y}\m{\in}\m{z}\m{%
\wedge}\m{z}\m{\in}\m{x}\m{)}\m{)}\m{)}
\endm

\needspace{4\baselineskip}
\noindent Axiom of Choice.\index{Axiom of Choice}

\setbox\startprefix=\hbox{\tt \ \ ax-ac\ \$a\ }
\setbox\contprefix=\hbox{\tt \ \ \ \ \ \ \ \ \ \ \ \ \ \ }
\startm
\m{\vdash}\m{\exists}\m{x}\m{\forall}\m{y}\m{\forall}\m{z}\m{(}\m{(}\m{y}\m{%
\in}\m{z}\m{\wedge}\m{z}\m{\in}\m{w}\m{)}\m{\rightarrow}\m{\exists}\m{w}\m{%
\forall}\m{y}\m{(}\m{\exists}\m{w}\m{(}\m{(}\m{y}\m{\in}\m{z}\m{\wedge}\m{z}%
\m{\in}\m{w}\m{)}\m{\wedge}\m{(}\m{y}\m{\in}\m{w}\m{\wedge}\m{w}\m{\in}\m{x}%
\m{)}\m{)}\m{\leftrightarrow}\m{y}\m{=}\m{w}\m{)}\m{)}
\endm

\subsection{That's It}

There you have it, the axioms for (essentially) all of mathematics!
Wonder at them and stare at them in awe.  Put a copy in your wallet, and
you will carry in your pocket the encoding for all theorems ever proved
and that ever will be proved, from the most mundane to the most
profound.

\section{A Hierarchy of Definitions}\label{hierarchy}

The axioms in the previous section in principle embody everything that can be
done within standard mathematics.  However, it is impractical to accomplish
very much by using them directly, for even simple concepts (from a human
perspective) can involve extremely long, incomprehensible formulas.
Mathematics is made practical by introducing definitions\index{definition}.
Definitions usually introduce new symbols, or at least new relationships among
existing symbols, to abbreviate more complex formulas.  An important
requirement for a definition is that there exist a straightforward
(algorithmic) method for eliminating the abbreviation by expanding it into the
more primitive symbol string that it represents.  Some
important definitions included in
the file \texttt{set.mm} are listed in this section for reference, and also to
give you a feel for why something like $\omega$\index{omega ($\omega$)} (the
set of natural numbers\index{natural number} 0, 1, 2,\ldots) becomes very
complicated when completely expanded into primitive symbols.

What is the motivation for definitions, aside from allowing complicated
expressions to be expressed more simply?  In the case of  $\omega$, one goal is
to provide a basis for the theory of natural numbers.\index{natural number}
Before set theory was invented, a set of axioms for arithmetic, called Peano's
postulates\index{Peano's postulates}, was devised and shown to have the
properties one expects for natural numbers.  Now anyone can postulate a
set of axioms, but if the axioms are inconsistent contradictions can be derived
from them.  Once a contradiction is derived, anything can be trivially
proved, including
all the facts of arithmetic and their negations.  To ensure that an
axiom system is at least as reliable as the axioms for set theory, we can
define sets and operations on those sets that satisfy the new axioms. In the
\texttt{set.mm} Metamath database, we prove that the elements of $\omega$ satisfy
Peano's postulates, and it's a long and hard journey to get there directly
from the axioms of set theory.  But the result is confidence in the
foundations of arithmetic.  And there is another advantage:  we now have all
the tools of set theory at our disposal for manipulating objects that obey the
axioms for arithmetic.

What are the criteria we use for definitions?  First, and of utmost importance,
the definition should not be {\em creative}\index{creative
definition}\index{definition!creative}, that
is it should not allow an expression that previously qualified as a wff but
was not provable, to become provable.   Second, the definition should be {\em
eliminable}\index{definition!eliminability}, that is, there should exist an
algorithmic method for converting any expression using the definition into
a logically equivalent expression that previously qualified as a wff.

In almost all cases below, definitions connect two expressions with either
$\leftrightarrow$ or $=$.  Eliminating\footnote{Here we mean the
elimination that a human might do in his or her head.  To eliminate them as
part of a Metamath proof we would invoke one of a number of
theorems that deal with transitivity of equivalence or equality; there are
many such examples in the proofs in \texttt{set.mm}.} such a definition is a
simple matter of substituting the expression on the left-hand side ({\em
definiendum}\index{definiendum} or thing being defined) with the equivalent,
more primitive expression on the right-hand side ({\em
definiens}\index{definiens} or definition).

Often a definition has variables on the right-hand side which do not appear on
the left-hand side; these are called {\em dummy variables}.\index{dummy
variable!in definitions}  In this case, any
allowable substitution (such as a new, distinct
variable) can be used when the definition is eliminated.  Dummy variables may
be used only if they are {\em effectively bound}\index{effectively bound
variable}, meaning that the definition will remain logically equivalent upon
any substitution of a dummy variable with any other {\em qualifying
expression}\index{qualifying expression}, i.e.\ any symbol string (such as
another variable) that
meets the restrictions on the dummy variable imposed by \texttt{\$d} and
\texttt{\$f} statements.  For example, we could define a constant $\perp$
(inverted tee, meaning logical ``false'') as $( \varphi \wedge \lnot \varphi
)$, i.e.\ ``phi and not phi.''  Here $\varphi$ is effectively bound because the
definition remains logically equivalent when we replace $\varphi$ with any
other wff.  (It is actually \texttt{df-fal}
in \texttt{set.mm}, which defines $\perp$.)

There are two cases where eliminating definitions is a little more
complex.  These cases are the definitions \texttt{df-bi} and
\texttt{df-cleq}.  The first stretches the concept of a definition a
little, as in effect it ``defines a definition;'' however, it meets our
requirements for a definition in that it is eliminable and does not
strengthen the language.  Theorem \texttt{bii} shows the substitution
needed to eliminate the $\leftrightarrow$\index{logical equivalence
($\leftrightarrow$)}\index{biconditional ($\leftrightarrow$)} symbol.

Definition \texttt{df-cleq}\index{equality ($=$)} extends the usage of
the equality symbol to include ``classes''\index{class} in set theory.  The
reason it is potentially problematic is that it can lead to statements which
do not follow from logic alone but presuppose the Axiom of
Extensionality\index{Axiom of Extensionality}, so we include this axiom
as a hypothesis for the definition.  We could have made \texttt{df-cleq} directly
eliminable by introducing a new equality symbol, but have chosen not to do so
in keeping with standard textbook practice.  Definitions such as \texttt{df-cleq}
that extend the meaning of existing symbols must be introduced carefully so
that they do not lead to contradictions.  Definition \texttt{df-clel} also
extends the meaning of an existing symbol ($\in$); while it doesn't strengthen
the language like \texttt{df-cleq}, this is not obvious and it must also be
subject to the same scrutiny.

Exercise:  Study how the wff $x\in\omega$, meaning ``$x$ is a natural
number,'' could be expanded in terms of primitive symbols, starting with the
definitions \texttt{df-clel} on p.~\pageref{dfclel} and \texttt{df-om} on
p.~\pageref{dfom} and working your way back.  Don't bother to work out the
details; just make sure that you understand how you could do it in principle.
The answer is shown in the footnote on p.~\pageref{expandom}.  If you
actually do work it out, you won't get exactly the same answer because we used
a few simplifications such as discarding occurrences of $\lnot\lnot$ (double
negation).

In the definitions below, we have placed the {\sc ascii} Metamath source
below each of the formulas to help you become familiar with the
notation in the database.  For simplicity, the necessary \texttt{\$f}
and \texttt{\$d} statements are not shown.  If you are in doubt, use the
\texttt{show statement}\index{\texttt{show statement} command} command
in the Metamath program to see the full statement.
A selection of this notation is summarized in Appendix~\ref{ASCII}.

To understand the motivation for these definitions, you should consult the
references indicated:  Takeuti and Zaring \cite{Takeuti}\index{Takeuti, Gaisi},
Quine \cite{Quine}\index{Quine, Willard Van Orman}, Bell and Machover
\cite{Bell}\index{Bell, J. L.}, and Enderton \cite{Enderton}\index{Enderton,
Herbert B.}.  Our list of definitions is provided more for reference than as a
learning aid.  However, by looking at a few of them you can gain a feel for
how the hierarchy is built up.  The definitions are a representative sample of
the many definitions
in \texttt{set.mm}, but they are complete with respect to the
theorem examples we will present in Section~\ref{sometheorems}.  Also, some are
slightly different from, but logically equivalent to, the ones in \texttt{set.mm}
(some of which have been revised over time to shorten them, for example).

\subsection{Definitions for Propositional Calculus}\label{metadefprop}

The symbols $\varphi$, $\psi$, and $\chi$ represent wffs.

Our first definition introduces the biconditional
connective\footnote{The term ``connective'' is informally used to mean a
symbol that is placed between two variables or adjacent to a variable,
whereas a mathematical ``constant'' usually indicates a symbol such as
the number 0 that may replace a variable or metavariable.  From
Metamath's point of view, there is no distinction between a connective
and a constant; both are constants in the Metamath
language.}\index{connective}\index{constant} (also called logical
equivalence)\index{logical equivalence
($\leftrightarrow$)}\index{biconditional ($\leftrightarrow$)}.  Unlike
most traditional developments, we have chosen not to have a separate
symbol such as ``Df.'' to mean ``is defined as.''  Instead, we will use
the biconditional connective for this purpose, as it lets us use
logic to manipulate definitions directly.  Here we state the properties
of the biconditional connective with a carefully crafted \texttt{\$a}
statement, which effectively uses the biconditional connective to define
itself.  The $\leftrightarrow$ symbol can be eliminated from a formula
using theorem \texttt{bii}, which is derived later.

\vskip 2ex
\noindent Define the biconditional connective.\label{df-bi}

\vskip 0.5ex
\setbox\startprefix=\hbox{\tt \ \ df-bi\ \$a\ }
\setbox\contprefix=\hbox{\tt \ \ \ \ \ \ \ \ \ \ \ }
\startm
\m{\vdash}\m{\lnot}\m{(}\m{(}\m{(}\m{\varphi}\m{\leftrightarrow}\m{\psi}\m{)}%
\m{\rightarrow}\m{\lnot}\m{(}\m{(}\m{\varphi}\m{\rightarrow}\m{\psi}\m{)}\m{%
\rightarrow}\m{\lnot}\m{(}\m{\psi}\m{\rightarrow}\m{\varphi}\m{)}\m{)}\m{)}\m{%
\rightarrow}\m{\lnot}\m{(}\m{\lnot}\m{(}\m{(}\m{\varphi}\m{\rightarrow}\m{%
\psi}\m{)}\m{\rightarrow}\m{\lnot}\m{(}\m{\psi}\m{\rightarrow}\m{\varphi}\m{)}%
\m{)}\m{\rightarrow}\m{(}\m{\varphi}\m{\leftrightarrow}\m{\psi}\m{)}\m{)}\m{)}
\endm
\begin{mmraw}%
|- -. ( ( ( ph <-> ps ) -> -. ( ( ph -> ps ) ->
-. ( ps -> ph ) ) ) -> -. ( -. ( ( ph -> ps ) -> -. (
ps -> ph ) ) -> ( ph <-> ps ) ) ) \$.
\end{mmraw}

\noindent This theorem relates the biconditional connective to primitive
connectives and can be used to eliminate the $\leftrightarrow$ symbol from any
wff.

\vskip 0.5ex
\setbox\startprefix=\hbox{\tt \ \ bii\ \$p\ }
\setbox\contprefix=\hbox{\tt \ \ \ \ \ \ \ \ \ }
\startm
\m{\vdash}\m{(}\m{(}\m{\varphi}\m{\leftrightarrow}\m{\psi}\m{)}\m{\leftrightarrow}
\m{\lnot}\m{(}\m{(}\m{\varphi}\m{\rightarrow}\m{\psi}\m{)}\m{\rightarrow}\m{\lnot}
\m{(}\m{\psi}\m{\rightarrow}\m{\varphi}\m{)}\m{)}\m{)}
\endm
\begin{mmraw}%
|- ( ( ph <-> ps ) <-> -. ( ( ph -> ps ) -> -. ( ps -> ph ) ) ) \$= ... \$.
\end{mmraw}

\noindent Define disjunction ({\sc or}).\index{disjunction ($\vee$)}%
\index{logical or (vee)@logical {\sc or} ($\vee$)}%
\index{df-or@\texttt{df-or}}\label{df-or}

\vskip 0.5ex
\setbox\startprefix=\hbox{\tt \ \ df-or\ \$a\ }
\setbox\contprefix=\hbox{\tt \ \ \ \ \ \ \ \ \ \ \ }
\startm
\m{\vdash}\m{(}\m{(}\m{\varphi}\m{\vee}\m{\psi}\m{)}\m{\leftrightarrow}\m{(}\m{
\lnot}\m{\varphi}\m{\rightarrow}\m{\psi}\m{)}\m{)}
\endm
\begin{mmraw}%
|- ( ( ph \TOR ps ) <-> ( -. ph -> ps ) ) \$.
\end{mmraw}

\noindent Define conjunction ({\sc and}).\index{conjunction ($\wedge$)}%
\index{logical {\sc and} ($\wedge$)}%
\index{df-an@\texttt{df-an}}\label{df-an}

\vskip 0.5ex
\setbox\startprefix=\hbox{\tt \ \ df-an\ \$a\ }
\setbox\contprefix=\hbox{\tt \ \ \ \ \ \ \ \ \ \ \ }
\startm
\m{\vdash}\m{(}\m{(}\m{\varphi}\m{\wedge}\m{\psi}\m{)}\m{\leftrightarrow}\m{\lnot}
\m{(}\m{\varphi}\m{\rightarrow}\m{\lnot}\m{\psi}\m{)}\m{)}
\endm
\begin{mmraw}%
|- ( ( ph \TAND ps ) <-> -. ( ph -> -. ps ) ) \$.
\end{mmraw}

\noindent Define disjunction ({\sc or}) of 3 wffs.%
\index{df-3or@\texttt{df-3or}}\label{df-3or}

\vskip 0.5ex
\setbox\startprefix=\hbox{\tt \ \ df-3or\ \$a\ }
\setbox\contprefix=\hbox{\tt \ \ \ \ \ \ \ \ \ \ \ \ }
\startm
\m{\vdash}\m{(}\m{(}\m{\varphi}\m{\vee}\m{\psi}\m{\vee}\m{\chi}\m{)}\m{
\leftrightarrow}\m{(}\m{(}\m{\varphi}\m{\vee}\m{\psi}\m{)}\m{\vee}\m{\chi}\m{)}
\m{)}
\endm
\begin{mmraw}%
|- ( ( ph \TOR ps \TOR ch ) <-> ( ( ph \TOR ps ) \TOR ch ) ) \$.
\end{mmraw}

\noindent Define conjunction ({\sc and}) of 3 wffs.%
\index{df-3an}\label{df-3an}

\vskip 0.5ex
\setbox\startprefix=\hbox{\tt \ \ df-3an\ \$a\ }
\setbox\contprefix=\hbox{\tt \ \ \ \ \ \ \ \ \ \ \ \ }
\startm
\m{\vdash}\m{(}\m{(}\m{\varphi}\m{\wedge}\m{\psi}\m{\wedge}\m{\chi}\m{)}\m{
\leftrightarrow}\m{(}\m{(}\m{\varphi}\m{\wedge}\m{\psi}\m{)}\m{\wedge}\m{\chi}
\m{)}\m{)}
\endm

\begin{mmraw}%
|- ( ( ph \TAND ps \TAND ch ) <-> ( ( ph \TAND ps ) \TAND ch ) ) \$.
\end{mmraw}

\subsection{Definitions for Predicate Calculus}\label{metadefpred}

The symbols $x$, $y$, and $z$ represent individual variables of predicate
calculus.  In this section, they are not necessarily distinct unless it is
explicitly
mentioned.

\vskip 2ex
\noindent Define existential quantification.
The expression $\exists x \varphi$ means
``there exists an $x$ where $\varphi$ is true.''\index{existential quantifier
($\exists$)}\label{df-ex}

\vskip 0.5ex
\setbox\startprefix=\hbox{\tt \ \ df-ex\ \$a\ }
\setbox\contprefix=\hbox{\tt \ \ \ \ \ \ \ \ \ \ \ }
\startm
\m{\vdash}\m{(}\m{\exists}\m{x}\m{\varphi}\m{\leftrightarrow}\m{\lnot}\m{\forall}
\m{x}\m{\lnot}\m{\varphi}\m{)}
\endm
\begin{mmraw}%
|- ( E. x ph <-> -. A. x -. ph ) \$.
\end{mmraw}

\noindent Define proper substitution.\index{proper
substitution}\index{substitution!proper}\label{df-sb}
In our notation, we use $[ y / x ] \varphi$ to mean ``the wff that
results when $y$ is properly substituted for $x$ in the wff
$\varphi$.''\footnote{
This can also be described
as substituting $x$ with $y$, $y$ properly replaces $x$, or
$x$ is properly replaced by $y$.}
% This is elsb4, though it currently says: ( [ x / y ] z e. y <-> z e. x )
For example,
$[ y / x ] z \in x$ is the same as $z \in y$.
One way to remember this notation is to notice that it looks like division
and recall that $( y / x ) \cdot x $ is $y$ (when $x \neq 0$).
The notation is different from the notation $\varphi ( x | y )$
that is sometimes used, because the latter notation is ambiguous for us:
for example, we don't know whether $\lnot \varphi ( x | y )$ is to be
interpreted as $\lnot ( \varphi ( x | y ) )$ or
$( \lnot \varphi ) ( x | y )$.\footnote{Because of the way
we initially defined wffs, this is the case
with any postfix connective\index{postfix connective} (one occurring after the
symbols being connected) or infix connective\index{infix connective} (one
occurring between the symbols being connected).  Metamath does not have a
built-in notion of operator binding strength that could eliminate the
ambiguity.  The initial parenthesis effectively provides a prefix
connective\index{prefix connective} to eliminate ambiguity.  Some conventions,
such as Polish notation\index{Polish notation} used in the 1930's and 1940's
by Polish logicians, use only prefix connectives and thus allow the total
elimination of parentheses, at the expense of readability.  In Metamath we
could actually redefine all notation to be Polish if we wanted to without
having to change any proofs!}  Other texts often use $\varphi(y)$ to indicate
our $[ y / x ] \varphi$, but this notation is even more ambiguous since there is
no explicit indication of what is being substituted.
Note that this
definition is valid even when
$x$ and $y$ are the same variable.  The first conjunct is a ``trick'' used to
achieve this property, making the definition look somewhat peculiar at
first.

\vskip 0.5ex
\setbox\startprefix=\hbox{\tt \ \ df-sb\ \$a\ }
\setbox\contprefix=\hbox{\tt \ \ \ \ \ \ \ \ \ \ \ }
\startm
\m{\vdash}\m{(}\m{[}\m{y}\m{/}\m{x}\m{]}\m{\varphi}\m{\leftrightarrow}\m{(}%
\m{(}\m{x}\m{=}\m{y}\m{\rightarrow}\m{\varphi}\m{)}\m{\wedge}\m{\exists}\m{x}%
\m{(}\m{x}\m{=}\m{y}\m{\wedge}\m{\varphi}\m{)}\m{)}\m{)}
\endm
\begin{mmraw}%
|- ( [ y / x ] ph <-> ( ( x = y -> ph ) \TAND E. x ( x = y \TAND ph ) ) ) \$.
\end{mmraw}


\noindent Define existential uniqueness\index{existential uniqueness
quantifier ($\exists "!$)} (``there exists exactly one'').  Note that $y$ is a
variable distinct from $x$ and not occurring in $\varphi$.\label{df-eu}

\vskip 0.5ex
\setbox\startprefix=\hbox{\tt \ \ df-eu\ \$a\ }
\setbox\contprefix=\hbox{\tt \ \ \ \ \ \ \ \ \ \ \ }
\startm
\m{\vdash}\m{(}\m{\exists}\m{{!}}\m{x}\m{\varphi}\m{\leftrightarrow}\m{\exists}
\m{y}\m{\forall}\m{x}\m{(}\m{\varphi}\m{\leftrightarrow}\m{x}\m{=}\m{y}\m{)}\m{)}
\endm

\begin{mmraw}%
|- ( E! x ph <-> E. y A. x ( ph <-> x = y ) ) \$.
\end{mmraw}

\subsection{Definitions for Set Theory}\label{setdefinitions}

The symbols $x$, $y$, $z$, and $w$ represent individual variables of
predicate calculus, which in set theory are understood to be sets.
However, using only the constructs shown so far would be very inconvenient.

To make set theory more practical, we introduce the notion of a ``class.''
A class\index{class} is either a set variable (such as $x$) or an
expression of the form $\{ x | \varphi\}$ (called an ``abstraction
class''\index{abstraction class}\index{class abstraction}).  Note that
sets (i.e.\ individual variables) always exist (this is a theorem of
logic, namely $\exists y \, y = x$ for any set $x$), whereas classes may
or may not exist (i.e.\ $\exists y \, y = A$ may or may not be true).
If a class does not exist it is called a ``proper class.''\index{proper
class}\index{class!proper} Definitions \texttt{df-clab},
\texttt{df-cleq}, and \texttt{df-clel} can be used to convert an
expression containing classes into one containing only set variables and
wff metavariables.

The symbols $A$, $B$, $C$, $D$, $ F$, $G$, and $R$ are metavariables that range
over classes.  A class metavariable $A$ may be eliminated from a wff by
replacing it with $\{ x|\varphi\}$ where neither $x$ nor $\varphi$ occur in
the wff.

The theory of classes can be shown to be an eliminable and conservative
extension of set theory. The \textbf{eliminability}
property shows that for every
formula in the extended language we can build a logically equivalent
formula in the basic language; so that even if the extended language
provides more ease to convey and formulate mathematical ideas for set
theory, its expressive power does not in fact strengthen the basic
language's expressive power.
The \textbf{conservation} property shows that for
every proof of a formula of the basic language in the extended system
we can build another proof of the same formula in the basic system;
so that, concerning theorems on sets only, the deductive powers of
the extended system and of the basic system are identical. Together,
these properties mean that the extended language can be treated as a
definitional extension that is \textbf{sound}.

A rigorous justification, which we will not give here, can be found in
Levy \cite[pp.~357-366]{Levy} supplementing his informal introduction to class
theory on pp.~7-17. Two other good treatments of class theory are provided
by Quine \cite[pp.~15-21]{Quine}\index{Quine, Willard Van Orman}
and also \cite[pp.~10-14]{Takeuti}\index{Takeuti, Gaisi}.
Quine's exposition (he calls them virtual classes)
is nicely written and very readable.

In the rest of this
section, individual variables are always assumed to be distinct from
each other unless otherwise indicated.  In addition, dummy variables on the
right-hand side of a definition do not occur in the class and wff
metavariables in the definition.

The definitions we present here are a partial but self-contained
collection selected from several hundred that appear in the current
\texttt{set.mm} database.  They are adequate for a basic development of
elementary set theory.

\vskip 2ex
\noindent Define the abstraction class.\index{abstraction class}\index{class
abstraction}\label{df-clab}  $x$ and $y$
need not be distinct.  Definition 2.1 of Quine, p.~16.  This definition may
seem puzzling since it is shorter than the expression being defined and does not
buy us anything in terms of brevity.  The reason we introduce this definition
is because it fits in neatly with the extension of the $\in$ connective
provided by \texttt{df-clel}.

\vskip 0.5ex
\setbox\startprefix=\hbox{\tt \ \ df-clab\ \$a\ }
\setbox\contprefix=\hbox{\tt \ \ \ \ \ \ \ \ \ \ \ \ \ }
\startm
\m{\vdash}\m{(}\m{x}\m{\in}\m{\{}\m{y}\m{|}\m{\varphi}\m{\}}\m{%
\leftrightarrow}\m{[}\m{x}\m{/}\m{y}\m{]}\m{\varphi}\m{)}
\endm
\begin{mmraw}%
|- ( x e. \{ y | ph \} <-> [ x / y ] ph ) \$.
\end{mmraw}

\noindent Define the equality connective between classes\index{class
equality}\label{df-cleq}.  See Quine or Chapter 4 of Takeuti and Zaring for its
justification and methods for eliminating it.  This is an example of a
somewhat ``dangerous'' definition, because it extends the use of the
existing equality symbol rather than introducing a new symbol, allowing
us to make statements in the original language that may not be true.
For example, it permits us to deduce $y = z \leftrightarrow \forall x (
x \in y \leftrightarrow x \in z )$ which is not a theorem of logic but
rather presupposes the Axiom of Extensionality,\index{Axiom of
Extensionality} which we include as a hypothesis so that we can know
when this axiom is assumed in a proof (with the \texttt{show
trace{\char`\_}back} command).  We could avoid the danger by introducing
another symbol, say $\eqcirc$, in place of $=$; this
would also have the advantage of making elimination of the definition
straightforward and would eliminate the need for Extensionality as a
hypothesis.  We would then also have the advantage of being able to
identify exactly where Extensionality truly comes into play.  One of our
theorems would be $x \eqcirc y \leftrightarrow x = y$
by invoking Extensionality.  However in keeping with standard practice
we retain the ``dangerous'' definition.

\vskip 0.5ex
\setbox\startprefix=\hbox{\tt \ \ df-cleq.1\ \$e\ }
\setbox\contprefix=\hbox{\tt \ \ \ \ \ \ \ \ \ \ \ \ \ \ \ }
\startm
\m{\vdash}\m{(}\m{\forall}\m{x}\m{(}\m{x}\m{\in}\m{y}\m{\leftrightarrow}\m{x}
\m{\in}\m{z}\m{)}\m{\rightarrow}\m{y}\m{=}\m{z}\m{)}
\endm
\setbox\startprefix=\hbox{\tt \ \ df-cleq\ \$a\ }
\setbox\contprefix=\hbox{\tt \ \ \ \ \ \ \ \ \ \ \ \ \ }
\startm
\m{\vdash}\m{(}\m{A}\m{=}\m{B}\m{\leftrightarrow}\m{\forall}\m{x}\m{(}\m{x}\m{
\in}\m{A}\m{\leftrightarrow}\m{x}\m{\in}\m{B}\m{)}\m{)}
\endm
% We need to reset the startprefix and contprefix.
\setbox\startprefix=\hbox{\tt \ \ df-cleq.1\ \$e\ }
\setbox\contprefix=\hbox{\tt \ \ \ \ \ \ \ \ \ \ \ \ \ \ \ }
\begin{mmraw}%
|- ( A. x ( x e. y <-> x e. z ) -> y = z ) \$.
\end{mmraw}
\setbox\startprefix=\hbox{\tt \ \ df-cleq\ \$a\ }
\setbox\contprefix=\hbox{\tt \ \ \ \ \ \ \ \ \ \ \ \ \ }
\begin{mmraw}%
|- ( A = B <-> A. x ( x e. A <-> x e. B ) ) \$.
\end{mmraw}

\noindent Define the membership connective between classes\index{class
membership}.  Theorem 6.3 of Quine, p.~41, which we adopt as a definition.
Note that it extends the use of the existing membership symbol, but unlike
{\tt df-cleq} it does not extend the set of valid wffs of logic when the class
metavariables are replaced with set variables.\label{dfclel}\label{df-clel}

\vskip 0.5ex
\setbox\startprefix=\hbox{\tt \ \ df-clel\ \$a\ }
\setbox\contprefix=\hbox{\tt \ \ \ \ \ \ \ \ \ \ \ \ \ }
\startm
\m{\vdash}\m{(}\m{A}\m{\in}\m{B}\m{\leftrightarrow}\m{\exists}\m{x}\m{(}\m{x}
\m{=}\m{A}\m{\wedge}\m{x}\m{\in}\m{B}\m{)}\m{)}
\endm
\begin{mmraw}%
|- ( A e. B <-> E. x ( x = A \TAND x e. B ) ) \$.?
\end{mmraw}

\noindent Define inequality.

\vskip 0.5ex
\setbox\startprefix=\hbox{\tt \ \ df-ne\ \$a\ }
\setbox\contprefix=\hbox{\tt \ \ \ \ \ \ \ \ \ \ \ }
\startm
\m{\vdash}\m{(}\m{A}\m{\ne}\m{B}\m{\leftrightarrow}\m{\lnot}\m{A}\m{=}\m{B}%
\m{)}
\endm
\begin{mmraw}%
|- ( A =/= B <-> -. A = B ) \$.
\end{mmraw}

\noindent Define restricted universal quantification.\index{universal
quantifier ($\forall$)!restricted}  Enderton, p.~22.

\vskip 0.5ex
\setbox\startprefix=\hbox{\tt \ \ df-ral\ \$a\ }
\setbox\contprefix=\hbox{\tt \ \ \ \ \ \ \ \ \ \ \ \ }
\startm
\m{\vdash}\m{(}\m{\forall}\m{x}\m{\in}\m{A}\m{\varphi}\m{\leftrightarrow}\m{%
\forall}\m{x}\m{(}\m{x}\m{\in}\m{A}\m{\rightarrow}\m{\varphi}\m{)}\m{)}
\endm
\begin{mmraw}%
|- ( A. x e. A ph <-> A. x ( x e. A -> ph ) ) \$.
\end{mmraw}

\noindent Define restricted existential quantification.\index{existential
quantifier ($\exists$)!restricted}  Enderton, p.~22.

\vskip 0.5ex
\setbox\startprefix=\hbox{\tt \ \ df-rex\ \$a\ }
\setbox\contprefix=\hbox{\tt \ \ \ \ \ \ \ \ \ \ \ \ }
\startm
\m{\vdash}\m{(}\m{\exists}\m{x}\m{\in}\m{A}\m{\varphi}\m{\leftrightarrow}\m{%
\exists}\m{x}\m{(}\m{x}\m{\in}\m{A}\m{\wedge}\m{\varphi}\m{)}\m{)}
\endm
\begin{mmraw}%
|- ( E. x e. A ph <-> E. x ( x e. A \TAND ph ) ) \$.
\end{mmraw}

\noindent Define the universal class\index{universal class ($V$)}.  Definition
5.20, p.~21, of Takeuti and Zaring.\label{df-v}

\vskip 0.5ex
\setbox\startprefix=\hbox{\tt \ \ df-v\ \$a\ }
\setbox\contprefix=\hbox{\tt \ \ \ \ \ \ \ \ \ \ }
\startm
\m{\vdash}\m{{\rm V}}\m{=}\m{\{}\m{x}\m{|}\m{x}\m{=}\m{x}\m{\}}
\endm
\begin{mmraw}%
|- {\char`\_}V = \{ x | x = x \} \$.
\end{mmraw}

\noindent Define the subclass\index{subclass}\index{subset} relationship
between two classes (called the subset relation if the classes are sets i.e.\
are not proper).  Definition 5.9 of Takeuti and Zaring, p.~17.\label{df-ss}

\vskip 0.5ex
\setbox\startprefix=\hbox{\tt \ \ df-ss\ \$a\ }
\setbox\contprefix=\hbox{\tt \ \ \ \ \ \ \ \ \ \ \ }
\startm
\m{\vdash}\m{(}\m{A}\m{\subseteq}\m{B}\m{\leftrightarrow}\m{\forall}\m{x}\m{(}
\m{x}\m{\in}\m{A}\m{\rightarrow}\m{x}\m{\in}\m{B}\m{)}\m{)}
\endm
\begin{mmraw}%
|- ( A C\_ B <-> A. x ( x e. A -> x e. B ) ) \$.
\end{mmraw}

\noindent Define the union\index{union} of two classes.  Definition 5.6 of Takeuti and Zaring,
p.~16.\label{df-un}

\vskip 0.5ex
\setbox\startprefix=\hbox{\tt \ \ df-un\ \$a\ }
\setbox\contprefix=\hbox{\tt \ \ \ \ \ \ \ \ \ \ \ }
\startm
\m{\vdash}\m{(}\m{A}\m{\cup}\m{B}\m{)}\m{=}\m{\{}\m{x}\m{|}\m{(}\m{x}\m{\in}
\m{A}\m{\vee}\m{x}\m{\in}\m{B}\m{)}\m{\}}
\endm
\begin{mmraw}%
( A u. B ) = \{ x | ( x e. A \TOR x e. B ) \} \$.
\end{mmraw}

\noindent Define the intersection\index{intersection} of two classes.  Definition 5.6 of
Takeuti and Zaring, p.~16.\label{df-in}

\vskip 0.5ex
\setbox\startprefix=\hbox{\tt \ \ df-in\ \$a\ }
\setbox\contprefix=\hbox{\tt \ \ \ \ \ \ \ \ \ \ \ }
\startm
\m{\vdash}\m{(}\m{A}\m{\cap}\m{B}\m{)}\m{=}\m{\{}\m{x}\m{|}\m{(}\m{x}\m{\in}
\m{A}\m{\wedge}\m{x}\m{\in}\m{B}\m{)}\m{\}}
\endm
% Caret ^ requires special treatment
\begin{mmraw}%
|- ( A i\^{}i B ) = \{ x | ( x e. A \TAND x e. B ) \} \$.
\end{mmraw}

\noindent Define class difference\index{class difference}\index{set difference}.
Definition 5.12 of Takeuti and Zaring, p.~20.  Several notations are used in
the literature; we chose the $\setminus$ convention instead of a minus sign to
reserve the latter for later use in, e.g., arithmetic.\label{df-dif}

\vskip 0.5ex
\setbox\startprefix=\hbox{\tt \ \ df-dif\ \$a\ }
\setbox\contprefix=\hbox{\tt \ \ \ \ \ \ \ \ \ \ \ \ }
\startm
\m{\vdash}\m{(}\m{A}\m{\setminus}\m{B}\m{)}\m{=}\m{\{}\m{x}\m{|}\m{(}\m{x}\m{
\in}\m{A}\m{\wedge}\m{\lnot}\m{x}\m{\in}\m{B}\m{)}\m{\}}
\endm
\begin{mmraw}%
( A \SLASH B ) = \{ x | ( x e. A \TAND -. x e. B ) \} \$.
\end{mmraw}

\noindent Define the empty or null set\index{empty set}\index{null set}.
Compare  Definition 5.14 of Takeuti and Zaring, p.~20.\label{df-nul}

\vskip 0.5ex
\setbox\startprefix=\hbox{\tt \ \ df-nul\ \$a\ }
\setbox\contprefix=\hbox{\tt \ \ \ \ \ \ \ \ \ \ }
\startm
\m{\vdash}\m{\varnothing}\m{=}\m{(}\m{{\rm V}}\m{\setminus}\m{{\rm V}}\m{)}
\endm
\begin{mmraw}%
|- (/) = ( {\char`\_}V \SLASH {\char`\_}V ) \$.
\end{mmraw}

\noindent Define power class\index{power set}\index{power class}.  Definition 5.10 of
Takeuti and Zaring, p.~17, but we also let it apply to proper classes.  (Note
that \verb$~P$ is the symbol for calligraphic P, the tilde
suggesting ``curly;'' see Appendix~\ref{ASCII}.)\label{df-pw}

\vskip 0.5ex
\setbox\startprefix=\hbox{\tt \ \ df-pw\ \$a\ }
\setbox\contprefix=\hbox{\tt \ \ \ \ \ \ \ \ \ \ \ }
\startm
\m{\vdash}\m{{\cal P}}\m{A}\m{=}\m{\{}\m{x}\m{|}\m{x}\m{\subseteq}\m{A}\m{\}}
\endm
% Special incantation required to put ~ into the text
\begin{mmraw}%
|- \char`\~P~A = \{ x | x C\_ A \} \$.
\end{mmraw}

\noindent Define the singleton of a class\index{singleton}.  Definition 7.1 of
Quine, p.~48.  It is well-defined for proper classes, although
it is not very meaningful in this case, where it evaluates to the empty
set.

\vskip 0.5ex
\setbox\startprefix=\hbox{\tt \ \ df-sn\ \$a\ }
\setbox\contprefix=\hbox{\tt \ \ \ \ \ \ \ \ \ \ \ }
\startm
\m{\vdash}\m{\{}\m{A}\m{\}}\m{=}\m{\{}\m{x}\m{|}\m{x}\m{=}\m{A}\m{\}}
\endm
\begin{mmraw}%
|- \{ A \} = \{ x | x = A \} \$.
\end{mmraw}%

\noindent Define an unordered pair of classes\index{unordered pair}\index{pair}.  Definition
7.1 of Quine, p.~48.

\vskip 0.5ex
\setbox\startprefix=\hbox{\tt \ \ df-pr\ \$a\ }
\setbox\contprefix=\hbox{\tt \ \ \ \ \ \ \ \ \ \ \ }
\startm
\m{\vdash}\m{\{}\m{A}\m{,}\m{B}\m{\}}\m{=}\m{(}\m{\{}\m{A}\m{\}}\m{\cup}\m{\{}
\m{B}\m{\}}\m{)}
\endm
\begin{mmraw}%
|- \{ A , B \} = ( \{ A \} u. \{ B \} ) \$.
\end{mmraw}

\noindent Define an unordered triple of classes\index{unordered triple}.  Definition of
Enderton, p.~19.

\vskip 0.5ex
\setbox\startprefix=\hbox{\tt \ \ df-tp\ \$a\ }
\setbox\contprefix=\hbox{\tt \ \ \ \ \ \ \ \ \ \ \ }
\startm
\m{\vdash}\m{\{}\m{A}\m{,}\m{B}\m{,}\m{C}\m{\}}\m{=}\m{(}\m{\{}\m{A}\m{,}\m{B}
\m{\}}\m{\cup}\m{\{}\m{C}\m{\}}\m{)}
\endm
\begin{mmraw}%
|- \{ A , B , C \} = ( \{ A , B \} u. \{ C \} ) \$.
\end{mmraw}%

\noindent Kuratowski's\index{Kuratowski, Kazimierz} ordered pair\index{ordered
pair} definition.  Definition 9.1 of Quine, p.~58. For proper classes it is
not meaningful but is well-defined for convenience.  (Note that \verb$<.$
stands for $\langle$ whereas \verb$<$ stands for $<$, and similarly for
\verb$>.$\,.)\label{df-op}

\vskip 0.5ex
\setbox\startprefix=\hbox{\tt \ \ df-op\ \$a\ }
\setbox\contprefix=\hbox{\tt \ \ \ \ \ \ \ \ \ \ \ }
\startm
\m{\vdash}\m{\langle}\m{A}\m{,}\m{B}\m{\rangle}\m{=}\m{\{}\m{\{}\m{A}\m{\}}
\m{,}\m{\{}\m{A}\m{,}\m{B}\m{\}}\m{\}}
\endm
\begin{mmraw}%
|- <. A , B >. = \{ \{ A \} , \{ A , B \} \} \$.
\end{mmraw}

\noindent Define the union of a class\index{union}.  Definition 5.5, p.~16,
of Takeuti and Zaring.

\vskip 0.5ex
\setbox\startprefix=\hbox{\tt \ \ df-uni\ \$a\ }
\setbox\contprefix=\hbox{\tt \ \ \ \ \ \ \ \ \ \ \ \ }
\startm
\m{\vdash}\m{\bigcup}\m{A}\m{=}\m{\{}\m{x}\m{|}\m{\exists}\m{y}\m{(}\m{x}\m{
\in}\m{y}\m{\wedge}\m{y}\m{\in}\m{A}\m{)}\m{\}}
\endm
\begin{mmraw}%
|- U. A = \{ x | E. y ( x e. y \TAND y e. A ) \} \$.
\end{mmraw}

\noindent Define the intersection\index{intersection} of a class.  Definition 7.35,
p.~44, of Takeuti and Zaring.

\vskip 0.5ex
\setbox\startprefix=\hbox{\tt \ \ df-int\ \$a\ }
\setbox\contprefix=\hbox{\tt \ \ \ \ \ \ \ \ \ \ \ \ }
\startm
\m{\vdash}\m{\bigcap}\m{A}\m{=}\m{\{}\m{x}\m{|}\m{\forall}\m{y}\m{(}\m{y}\m{
\in}\m{A}\m{\rightarrow}\m{x}\m{\in}\m{y}\m{)}\m{\}}
\endm
\begin{mmraw}%
|- |\^{}| A = \{ x | A. y ( y e. A -> x e. y ) \} \$.
\end{mmraw}

\noindent Define a transitive class\index{transitive class}\index{transitive
set}.  This should not be confused with a transitive relation, which is a different
concept.  Definition from p.~71 of Enderton, extended to classes.

\vskip 0.5ex
\setbox\startprefix=\hbox{\tt \ \ df-tr\ \$a\ }
\setbox\contprefix=\hbox{\tt \ \ \ \ \ \ \ \ \ \ \ }
\startm
\m{\vdash}\m{(}\m{\mbox{\rm Tr}}\m{A}\m{\leftrightarrow}\m{\bigcup}\m{A}\m{
\subseteq}\m{A}\m{)}
\endm
\begin{mmraw}%
|- ( Tr A <-> U. A C\_ A ) \$.
\end{mmraw}

\noindent Define a notation for a general binary relation\index{binary
relation}.  Definition 6.18, p.~29, of Takeuti and Zaring, generalized to
arbitrary classes.  This definition is well-defined, although not very
meaningful, when classes $A$ and/or $B$ are proper.\label{dfbr}  The lack of
parentheses (or any other connective) creates no ambiguity since we are defining
an atomic wff.

\vskip 0.5ex
\setbox\startprefix=\hbox{\tt \ \ df-br\ \$a\ }
\setbox\contprefix=\hbox{\tt \ \ \ \ \ \ \ \ \ \ \ }
\startm
\m{\vdash}\m{(}\m{A}\m{\,R}\m{\,B}\m{\leftrightarrow}\m{\langle}\m{A}\m{,}\m{B}
\m{\rangle}\m{\in}\m{R}\m{)}
\endm
\begin{mmraw}%
|- ( A R B <-> <. A , B >. e. R ) \$.
\end{mmraw}

\noindent Define an abstraction class of ordered pairs\index{abstraction
class!of ordered
pairs}.  A special case of Definition 4.16, p.~14, of Takeuti and Zaring.
Note that $ z $ must be distinct from $ x $ and $ y $,
and $ z $ must not occur in $\varphi$, but $ x $ and $ y $ may be
identical and may appear in $\varphi$.

\vskip 0.5ex
\setbox\startprefix=\hbox{\tt \ \ df-opab\ \$a\ }
\setbox\contprefix=\hbox{\tt \ \ \ \ \ \ \ \ \ \ \ \ \ }
\startm
\m{\vdash}\m{\{}\m{\langle}\m{x}\m{,}\m{y}\m{\rangle}\m{|}\m{\varphi}\m{\}}\m{=}
\m{\{}\m{z}\m{|}\m{\exists}\m{x}\m{\exists}\m{y}\m{(}\m{z}\m{=}\m{\langle}\m{x}
\m{,}\m{y}\m{\rangle}\m{\wedge}\m{\varphi}\m{)}\m{\}}
\endm

\begin{mmraw}%
|- \{ <. x , y >. | ph \} = \{ z | E. x E. y ( z =
<. x , y >. /\ ph ) \} \$.
\end{mmraw}

\noindent Define the epsilon relation\index{epsilon relation}.  Similar to Definition
6.22, p.~30, of Takeuti and Zaring.

\vskip 0.5ex
\setbox\startprefix=\hbox{\tt \ \ df-eprel\ \$a\ }
\setbox\contprefix=\hbox{\tt \ \ \ \ \ \ \ \ \ \ \ \ \ \ }
\startm
\m{\vdash}\m{{\rm E}}\m{=}\m{\{}\m{\langle}\m{x}\m{,}\m{y}\m{\rangle}\m{|}\m{x}\m{
\in}\m{y}\m{\}}
\endm
\begin{mmraw}%
|- \_E = \{ <. x , y >. | x e. y \} \$.
\end{mmraw}

\noindent Define a founded relation\index{founded relation}.  $R$ is a founded
relation on $A$ iff\index{iff} (if and only if) each nonempty subset of $A$
has an ``$R$-minimal element.''  Similar to Definition 6.21, p.~30, of
Takeuti and Zaring.

\vskip 0.5ex
\setbox\startprefix=\hbox{\tt \ \ df-fr\ \$a\ }
\setbox\contprefix=\hbox{\tt \ \ \ \ \ \ \ \ \ \ \ }
\startm
\m{\vdash}\m{(}\m{R}\m{\,\mbox{\rm Fr}}\m{\,A}\m{\leftrightarrow}\m{\forall}\m{x}
\m{(}\m{(}\m{x}\m{\subseteq}\m{A}\m{\wedge}\m{\lnot}\m{x}\m{=}\m{\varnothing}
\m{)}\m{\rightarrow}\m{\exists}\m{y}\m{(}\m{y}\m{\in}\m{x}\m{\wedge}\m{(}\m{x}
\m{\cap}\m{\{}\m{z}\m{|}\m{z}\m{\,R}\m{\,y}\m{\}}\m{)}\m{=}\m{\varnothing}\m{)}
\m{)}\m{)}
\endm
\begin{mmraw}%
|- ( R Fr A <-> A. x ( ( x C\_ A \TAND -. x = (/) ) ->
E. y ( y e. x \TAND ( x i\^{}i \{ z | z R y \} ) = (/) ) ) ) \$.
\end{mmraw}

\noindent Define a well-ordering\index{well-ordering}.  $R$ is a well-ordering of $A$ iff
it is founded on $A$ and the elements of $A$ are pairwise $R$-comparable.
Similar to Definition 6.24(2), p.~30, of Takeuti and Zaring.

\vskip 0.5ex
\setbox\startprefix=\hbox{\tt \ \ df-we\ \$a\ }
\setbox\contprefix=\hbox{\tt \ \ \ \ \ \ \ \ \ \ \ }
\startm
\m{\vdash}\m{(}\m{R}\m{\,\mbox{\rm We}}\m{\,A}\m{\leftrightarrow}\m{(}\m{R}\m{\,
\mbox{\rm Fr}}\m{\,A}\m{\wedge}\m{\forall}\m{x}\m{\forall}\m{y}\m{(}\m{(}\m{x}\m{
\in}\m{A}\m{\wedge}\m{y}\m{\in}\m{A}\m{)}\m{\rightarrow}\m{(}\m{x}\m{\,R}\m{\,y}
\m{\vee}\m{x}\m{=}\m{y}\m{\vee}\m{y}\m{\,R}\m{\,x}\m{)}\m{)}\m{)}\m{)}
\endm
\begin{mmraw}%
( R We A <-> ( R Fr A \TAND A. x A. y ( ( x e.
A \TAND y e. A ) -> ( x R y \TOR x = y \TOR y R x ) ) ) ) \$.
\end{mmraw}

\noindent Define the ordinal predicate\index{ordinal predicate}, which is true for a class
that is transitive and is well-ordered by the epsilon relation.  Similar to
definition on p.~468, Bell and Machover.

\vskip 0.5ex
\setbox\startprefix=\hbox{\tt \ \ df-ord\ \$a\ }
\setbox\contprefix=\hbox{\tt \ \ \ \ \ \ \ \ \ \ \ \ }
\startm
\m{\vdash}\m{(}\m{\mbox{\rm Ord}}\m{\,A}\m{\leftrightarrow}\m{(}
\m{\mbox{\rm Tr}}\m{\,A}\m{\wedge}\m{E}\m{\,\mbox{\rm We}}\m{\,A}\m{)}\m{)}
\endm
\begin{mmraw}%
|- ( Ord A <-> ( Tr A \TAND E We A ) ) \$.
\end{mmraw}

\noindent Define the class of all ordinal numbers\index{ordinal number}.  An ordinal number is
a set that satisfies the ordinal predicate.  Definition 7.11 of Takeuti and
Zaring, p.~38.

\vskip 0.5ex
\setbox\startprefix=\hbox{\tt \ \ df-on\ \$a\ }
\setbox\contprefix=\hbox{\tt \ \ \ \ \ \ \ \ \ \ \ }
\startm
\m{\vdash}\m{\,\mbox{\rm On}}\m{=}\m{\{}\m{x}\m{|}\m{\mbox{\rm Ord}}\m{\,x}
\m{\}}
\endm
\begin{mmraw}%
|- On = \{ x | Ord x \} \$.
\end{mmraw}

\noindent Define the limit ordinal predicate\index{limit ordinal}, which is true for a
non-empty ordinal that is not a successor (i.e.\ that is the union of itself).
Compare Bell and Machover, p.~471 and Exercise (1), p.~42 of Takeuti and
Zaring.

\vskip 0.5ex
\setbox\startprefix=\hbox{\tt \ \ df-lim\ \$a\ }
\setbox\contprefix=\hbox{\tt \ \ \ \ \ \ \ \ \ \ \ \ }
\startm
\m{\vdash}\m{(}\m{\mbox{\rm Lim}}\m{\,A}\m{\leftrightarrow}\m{(}\m{\mbox{
\rm Ord}}\m{\,A}\m{\wedge}\m{\lnot}\m{A}\m{=}\m{\varnothing}\m{\wedge}\m{A}
\m{=}\m{\bigcup}\m{A}\m{)}\m{)}
\endm
\begin{mmraw}%
|- ( Lim A <-> ( Ord A \TAND -. A = (/) \TAND A = U. A ) ) \$.
\end{mmraw}

\noindent Define the successor\index{successor} of a class.  Definition 7.22 of Takeuti
and Zaring, p.~41.  Our definition is a generalization to classes, although it
is meaningless when classes are proper.

\vskip 0.5ex
\setbox\startprefix=\hbox{\tt \ \ df-suc\ \$a\ }
\setbox\contprefix=\hbox{\tt \ \ \ \ \ \ \ \ \ \ \ \ }
\startm
\m{\vdash}\m{\,\mbox{\rm suc}}\m{\,A}\m{=}\m{(}\m{A}\m{\cup}\m{\{}\m{A}\m{\}}
\m{)}
\endm
\begin{mmraw}%
|- suc A = ( A u. \{ A \} ) \$.
\end{mmraw}

\noindent Define the class of natural numbers\index{natural number}\index{omega
($\omega$)}.  Compare Bell and Machover, p.~471.\label{dfom}

\vskip 0.5ex
\setbox\startprefix=\hbox{\tt \ \ df-om\ \$a\ }
\setbox\contprefix=\hbox{\tt \ \ \ \ \ \ \ \ \ \ \ }
\startm
\m{\vdash}\m{\omega}\m{=}\m{\{}\m{x}\m{|}\m{(}\m{\mbox{\rm Ord}}\m{\,x}\m{
\wedge}\m{\forall}\m{y}\m{(}\m{\mbox{\rm Lim}}\m{\,y}\m{\rightarrow}\m{x}\m{
\in}\m{y}\m{)}\m{)}\m{\}}
\endm
\begin{mmraw}%
|- om = \{ x | ( Ord x \TAND A. y ( Lim y -> x e. y ) ) \} \$.
\end{mmraw}

\noindent Define the Cartesian product (also called the
cross product)\index{Cartesian product}\index{cross product}
of two classes.  Definition 9.11 of Quine, p.~64.

\vskip 0.5ex
\setbox\startprefix=\hbox{\tt \ \ df-xp\ \$a\ }
\setbox\contprefix=\hbox{\tt \ \ \ \ \ \ \ \ \ \ \ }
\startm
\m{\vdash}\m{(}\m{A}\m{\times}\m{B}\m{)}\m{=}\m{\{}\m{\langle}\m{x}\m{,}\m{y}
\m{\rangle}\m{|}\m{(}\m{x}\m{\in}\m{A}\m{\wedge}\m{y}\m{\in}\m{B}\m{)}\m{\}}
\endm
\begin{mmraw}%
|- ( A X. B ) = \{ <. x , y >. | ( x e. A \TAND y e. B) \} \$.
\end{mmraw}

\noindent Define a relation\index{relation}.  Definition 6.4(1) of Takeuti and
Zaring, p.~23.

\vskip 0.5ex
\setbox\startprefix=\hbox{\tt \ \ df-rel\ \$a\ }
\setbox\contprefix=\hbox{\tt \ \ \ \ \ \ \ \ \ \ \ \ }
\startm
\m{\vdash}\m{(}\m{\mbox{\rm Rel}}\m{\,A}\m{\leftrightarrow}\m{A}\m{\subseteq}
\m{(}\m{{\rm V}}\m{\times}\m{{\rm V}}\m{)}\m{)}
\endm
\begin{mmraw}%
|- ( Rel A <-> A C\_ ( {\char`\_}V X. {\char`\_}V ) ) \$.
\end{mmraw}

\noindent Define the domain\index{domain} of a class.  Definition 6.5(1) of
Takeuti and Zaring, p.~24.

\vskip 0.5ex
\setbox\startprefix=\hbox{\tt \ \ df-dm\ \$a\ }
\setbox\contprefix=\hbox{\tt \ \ \ \ \ \ \ \ \ \ \ }
\startm
\m{\vdash}\m{\,\mbox{\rm dom}}\m{A}\m{=}\m{\{}\m{x}\m{|}\m{\exists}\m{y}\m{
\langle}\m{x}\m{,}\m{y}\m{\rangle}\m{\in}\m{A}\m{\}}
\endm
\begin{mmraw}%
|- dom A = \{ x | E. y <. x , y >. e. A \} \$.
\end{mmraw}

\noindent Define the range\index{range} of a class.  Definition 6.5(2) of
Takeuti and Zaring, p.~24.

\vskip 0.5ex
\setbox\startprefix=\hbox{\tt \ \ df-rn\ \$a\ }
\setbox\contprefix=\hbox{\tt \ \ \ \ \ \ \ \ \ \ \ }
\startm
\m{\vdash}\m{\,\mbox{\rm ran}}\m{A}\m{=}\m{\{}\m{y}\m{|}\m{\exists}\m{x}\m{
\langle}\m{x}\m{,}\m{y}\m{\rangle}\m{\in}\m{A}\m{\}}
\endm
\begin{mmraw}%
|- ran A = \{ y | E. x <. x , y >. e. A \} \$.
\end{mmraw}

\noindent Define the restriction\index{restriction} of a class.  Definition
6.6(1) of Takeuti and Zaring, p.~24.

\vskip 0.5ex
\setbox\startprefix=\hbox{\tt \ \ df-res\ \$a\ }
\setbox\contprefix=\hbox{\tt \ \ \ \ \ \ \ \ \ \ \ \ }
\startm
\m{\vdash}\m{(}\m{A}\m{\restriction}\m{B}\m{)}\m{=}\m{(}\m{A}\m{\cap}\m{(}\m{B}
\m{\times}\m{{\rm V}}\m{)}\m{)}
\endm
\begin{mmraw}%
|- ( A |` B ) = ( A i\^{}i ( B X. {\char`\_}V ) ) \$.
\end{mmraw}

\noindent Define the image\index{image} of a class.  Definition 6.6(2) of
Takeuti and Zaring, p.~24.

\vskip 0.5ex
\setbox\startprefix=\hbox{\tt \ \ df-ima\ \$a\ }
\setbox\contprefix=\hbox{\tt \ \ \ \ \ \ \ \ \ \ \ \ }
\startm
\m{\vdash}\m{(}\m{A}\m{``}\m{B}\m{)}\m{=}\m{\,\mbox{\rm ran}}\m{\,(}\m{A}\m{
\restriction}\m{B}\m{)}
\endm
\begin{mmraw}%
|- ( A " B ) = ran ( A |` B ) \$.
\end{mmraw}

\noindent Define the composition\index{composition} of two classes.  Definition 6.6(3) of
Takeuti and Zaring, p.~24.

\vskip 0.5ex
\setbox\startprefix=\hbox{\tt \ \ df-co\ \$a\ }
\setbox\contprefix=\hbox{\tt \ \ \ \ \ \ \ \ \ \ \ \ }
\startm
\m{\vdash}\m{(}\m{A}\m{\circ}\m{B}\m{)}\m{=}\m{\{}\m{\langle}\m{x}\m{,}\m{y}\m{
\rangle}\m{|}\m{\exists}\m{z}\m{(}\m{\langle}\m{x}\m{,}\m{z}\m{\rangle}\m{\in}
\m{B}\m{\wedge}\m{\langle}\m{z}\m{,}\m{y}\m{\rangle}\m{\in}\m{A}\m{)}\m{\}}
\endm
\begin{mmraw}%
|- ( A o. B ) = \{ <. x , y >. | E. z ( <. x , z
>. e. B \TAND <. z , y >. e. A ) \} \$.
\end{mmraw}

\noindent Define a function\index{function}.  Definition 6.4(4) of Takeuti and
Zaring, p.~24.

\vskip 0.5ex
\setbox\startprefix=\hbox{\tt \ \ df-fun\ \$a\ }
\setbox\contprefix=\hbox{\tt \ \ \ \ \ \ \ \ \ \ \ \ }
\startm
\m{\vdash}\m{(}\m{\mbox{\rm Fun}}\m{\,A}\m{\leftrightarrow}\m{(}
\m{\mbox{\rm Rel}}\m{\,A}\m{\wedge}
\m{\forall}\m{x}\m{\exists}\m{z}\m{\forall}\m{y}\m{(}
\m{\langle}\m{x}\m{,}\m{y}\m{\rangle}\m{\in}\m{A}\m{\rightarrow}\m{y}\m{=}\m{z}
\m{)}\m{)}\m{)}
\endm
\begin{mmraw}%
|- ( Fun A <-> ( Rel A /\ A. x E. z A. y ( <. x
   , y >. e. A -> y = z ) ) ) \$.
\end{mmraw}

\noindent Define a function with domain.  Definition 6.15(1) of Takeuti and
Zaring, p.~27.

\vskip 0.5ex
\setbox\startprefix=\hbox{\tt \ \ df-fn\ \$a\ }
\setbox\contprefix=\hbox{\tt \ \ \ \ \ \ \ \ \ \ \ }
\startm
\m{\vdash}\m{(}\m{A}\m{\,\mbox{\rm Fn}}\m{\,B}\m{\leftrightarrow}\m{(}
\m{\mbox{\rm Fun}}\m{\,A}\m{\wedge}\m{\mbox{\rm dom}}\m{\,A}\m{=}\m{B}\m{)}
\m{)}
\endm
\begin{mmraw}%
|- ( A Fn B <-> ( Fun A \TAND dom A = B ) ) \$.
\end{mmraw}

\noindent Define a function with domain and co-domain.  Definition 6.15(3)
of Takeuti and Zaring, p.~27.

\vskip 0.5ex
\setbox\startprefix=\hbox{\tt \ \ df-f\ \$a\ }
\setbox\contprefix=\hbox{\tt \ \ \ \ \ \ \ \ \ \ }
\startm
\m{\vdash}\m{(}\m{F}\m{:}\m{A}\m{\longrightarrow}\m{B}\m{
\leftrightarrow}\m{(}\m{F}\m{\,\mbox{\rm Fn}}\m{\,A}\m{\wedge}\m{
\mbox{\rm ran}}\m{\,F}\m{\subseteq}\m{B}\m{)}\m{)}
\endm
\begin{mmraw}%
|- ( F : A --> B <-> ( F Fn A \TAND ran F C\_ B ) ) \$.
\end{mmraw}

\noindent Define a one-to-one function\index{one-to-one function}.  Compare
Definition 6.15(5) of Takeuti and Zaring, p.~27.

\vskip 0.5ex
\setbox\startprefix=\hbox{\tt \ \ df-f1\ \$a\ }
\setbox\contprefix=\hbox{\tt \ \ \ \ \ \ \ \ \ \ \ }
\startm
\m{\vdash}\m{(}\m{F}\m{:}\m{A}\m{
\raisebox{.5ex}{${\textstyle{\:}_{\mbox{\footnotesize\rm
1\tt -\rm 1}}}\atop{\textstyle{
\longrightarrow}\atop{\textstyle{}^{\mbox{\footnotesize\rm {\ }}}}}$}
}\m{B}
\m{\leftrightarrow}\m{(}\m{F}\m{:}\m{A}\m{\longrightarrow}\m{B}
\m{\wedge}\m{\forall}\m{y}\m{\exists}\m{z}\m{\forall}\m{x}\m{(}\m{\langle}\m{x}
\m{,}\m{y}\m{\rangle}\m{\in}\m{F}\m{\rightarrow}\m{x}\m{=}\m{z}\m{)}\m{)}\m{)}
\endm
\begin{mmraw}%
|- ( F : A -1-1-> B <-> ( F : A --> B \TAND
   A. y E. z A. x ( <. x , y >. e. F -> x = z ) ) ) \$.
\end{mmraw}

\noindent Define an onto function\index{onto function}.  Definition 6.15(4) of Takeuti and
Zaring, p.~27.

\vskip 0.5ex
\setbox\startprefix=\hbox{\tt \ \ df-fo\ \$a\ }
\setbox\contprefix=\hbox{\tt \ \ \ \ \ \ \ \ \ \ \ }
\startm
\m{\vdash}\m{(}\m{F}\m{:}\m{A}\m{
\raisebox{.5ex}{${\textstyle{\:}_{\mbox{\footnotesize\rm
{\ }}}}\atop{\textstyle{
\longrightarrow}\atop{\textstyle{}^{\mbox{\footnotesize\rm onto}}}}$}
}\m{B}
\m{\leftrightarrow}\m{(}\m{F}\m{\,\mbox{\rm Fn}}\m{\,A}\m{\wedge}
\m{\mbox{\rm ran}}\m{\,F}\m{=}\m{B}\m{)}\m{)}
\endm
\begin{mmraw}%
|- ( F : A -onto-> B <-> ( F Fn A /\ ran F = B ) ) \$.
\end{mmraw}

\noindent Define a one-to-one, onto function.  Compare Definition 6.15(6) of
Takeuti and Zaring, p.~27.

\vskip 0.5ex
\setbox\startprefix=\hbox{\tt \ \ df-f1o\ \$a\ }
\setbox\contprefix=\hbox{\tt \ \ \ \ \ \ \ \ \ \ \ \ }
\startm
\m{\vdash}\m{(}\m{F}\m{:}\m{A}
\m{
\raisebox{.5ex}{${\textstyle{\:}_{\mbox{\footnotesize\rm
1\tt -\rm 1}}}\atop{\textstyle{
\longrightarrow}\atop{\textstyle{}^{\mbox{\footnotesize\rm onto}}}}$}
}
\m{B}
\m{\leftrightarrow}\m{(}\m{F}\m{:}\m{A}
\m{
\raisebox{.5ex}{${\textstyle{\:}_{\mbox{\footnotesize\rm
1\tt -\rm 1}}}\atop{\textstyle{
\longrightarrow}\atop{\textstyle{}^{\mbox{\footnotesize\rm {\ }}}}}$}
}
\m{B}\m{\wedge}\m{F}\m{:}\m{A}
\m{
\raisebox{.5ex}{${\textstyle{\:}_{\mbox{\footnotesize\rm
{\ }}}}\atop{\textstyle{
\longrightarrow}\atop{\textstyle{}^{\mbox{\footnotesize\rm onto}}}}$}
}
\m{B}\m{)}\m{)}
\endm
\begin{mmraw}%
|- ( F : A -1-1-onto-> B <-> ( F : A -1-1-> B? \TAND F : A -onto-> B ) ) \$.?
\end{mmraw}

\noindent Define the value of a function\index{function value}.  This
definition applies to any class and evaluates to the empty set when it is not
meaningful. Note that $ F`A$ means the same thing as the more familiar $ F(A)$
notation for a function's value at $A$.  The $ F`A$ notation is common in
formal set theory.\label{df-fv}

\vskip 0.5ex
\setbox\startprefix=\hbox{\tt \ \ df-fv\ \$a\ }
\setbox\contprefix=\hbox{\tt \ \ \ \ \ \ \ \ \ \ \ }
\startm
\m{\vdash}\m{(}\m{F}\m{`}\m{A}\m{)}\m{=}\m{\bigcup}\m{\{}\m{x}\m{|}\m{(}\m{F}%
\m{``}\m{\{}\m{A}\m{\}}\m{)}\m{=}\m{\{}\m{x}\m{\}}\m{\}}
\endm
\begin{mmraw}%
|- ( F ` A ) = U. \{ x | ( F " \{ A \} ) = \{ x \} \} \$.
\end{mmraw}

\noindent Define the result of an operation.\index{operation}  Here, $F$ is
     an operation on two
     values (such as $+$ for real numbers).   This is defined for proper
     classes $A$ and $B$ even though not meaningful in that case.  However,
     the definition can be meaningful when $F$ is a proper class.\label{dfopr}

\vskip 0.5ex
\setbox\startprefix=\hbox{\tt \ \ df-opr\ \$a\ }
\setbox\contprefix=\hbox{\tt \ \ \ \ \ \ \ \ \ \ \ \ }
\startm
\m{\vdash}\m{(}\m{A}\m{\,F}\m{\,B}\m{)}\m{=}\m{(}\m{F}\m{`}\m{\langle}\m{A}%
\m{,}\m{B}\m{\rangle}\m{)}
\endm
\begin{mmraw}%
|- ( A F B ) = ( F ` <. A , B >. ) \$.
\end{mmraw}

\section{Tricks of the Trade}\label{tricks}

In the \texttt{set.mm}\index{set theory database (\texttt{set.mm})} database our goal
was usually to conform to modern notation.  However in some cases the
relationship to standard textbook language may be obscured by several
unconventional devices we used to simplify the development and to take
advantage of the Metamath language.  In this section we will describe some
common conventions used in \texttt{set.mm}.

\begin{itemize}
\item
The turnstile\index{turnstile ({$\,\vdash$})} symbol, $\vdash$, meaning ``it
is provable that,'' is the first token of all assertions and hypotheses that
aren't syntax constructions.  This is a standard convention in logic.  (We
mentioned this earlier, but this symbol is bothersome to some people without a
logic background.  It has no deeper meaning but just provides us with a way to
distinguish syntax constructions from ordinary mathematical statements.)

\item
A hypothesis of the form

\vskip 1ex
\setbox\startprefix=\hbox{\tt \ \ \ \ \ \ \ \ \ \$e\ }
\setbox\contprefix=\hbox{\tt \ \ \ \ \ \ \ \ \ \ }
\startm
\m{\vdash}\m{(}\m{\varphi}\m{\rightarrow}\m{\forall}\m{x}\m{\varphi}\m{)}
\endm
\vskip 1ex

should be read ``assume variable $x$ is (effectively) not free in wff
$\varphi$.''\index{effectively not free}
Literally, this says ``assume it is provable that $\varphi \rightarrow \forall
x\, \varphi$.''  This device lets us avoid the complexities associated with
the standard treatment of free and bound variables.
%% Uncomment this when uncommenting section {formalspec} below
The footnote on p.~\pageref{effectivelybound} discusses this further.

\item
A statement of one of the forms

\vskip 1ex
\setbox\startprefix=\hbox{\tt \ \ \ \ \ \ \ \ \ \$a\ }
\setbox\contprefix=\hbox{\tt \ \ \ \ \ \ \ \ \ \ }
\startm
\m{\vdash}\m{(}\m{\lnot}\m{\forall}\m{x}\m{\,x}\m{=}\m{y}\m{\rightarrow}
\m{\ldots}\m{)}
\endm
\setbox\startprefix=\hbox{\tt \ \ \ \ \ \ \ \ \ \$p\ }
\setbox\contprefix=\hbox{\tt \ \ \ \ \ \ \ \ \ \ }
\startm
\m{\vdash}\m{(}\m{\lnot}\m{\forall}\m{x}\m{\,x}\m{=}\m{y}\m{\rightarrow}
\m{\ldots}\m{)}
\endm
\vskip 1ex

should be read ``if $x$ and $y$ are distinct variables, then...''  This
antecedent provides us with a technical device to avoid the need for the
\texttt{\$d} statement early in our development of predicate calculus,
permitting symbol manipulations to be as conceptually simple as those in
propositional calculus.  However, the \texttt{\$d} statement eventually
becomes a requirement, and after that this device is rarely used.

\item
The statement

\vskip 1ex
\setbox\startprefix=\hbox{\tt \ \ \ \ \ \ \ \ \ \$d\ }
\setbox\contprefix=\hbox{\tt \ \ \ \ \ \ \ \ \ \ }
\startm
\m{x}\m{\,y}
\endm
\vskip 1ex

should be read ``assume $x$ and $y$ are distinct variables.''

\item
The statement

\vskip 1ex
\setbox\startprefix=\hbox{\tt \ \ \ \ \ \ \ \ \ \$d\ }
\setbox\contprefix=\hbox{\tt \ \ \ \ \ \ \ \ \ \ }
\startm
\m{x}\m{\,\varphi}
\endm
\vskip 1ex

should be read ``assume $x$ does not occur in $\varphi$.''

\item
The statement

\vskip 1ex
\setbox\startprefix=\hbox{\tt \ \ \ \ \ \ \ \ \ \$d\ }
\setbox\contprefix=\hbox{\tt \ \ \ \ \ \ \ \ \ \ }
\startm
\m{x}\m{\,A}
\endm
\vskip 1ex

should be read ``assume variable $x$ does not occur in class $A$.''

\item
The restriction and hypothesis group

\vskip 1ex
\setbox\startprefix=\hbox{\tt \ \ \ \ \ \ \ \ \ \$d\ }
\setbox\contprefix=\hbox{\tt \ \ \ \ \ \ \ \ \ \ }
\startm
\m{x}\m{\,A}
\endm
\setbox\startprefix=\hbox{\tt \ \ \ \ \ \ \ \ \ \$d\ }
\setbox\contprefix=\hbox{\tt \ \ \ \ \ \ \ \ \ \ }
\startm
\m{x}\m{\,\psi}
\endm
\setbox\startprefix=\hbox{\tt \ \ \ \ \ \ \ \ \ \$e\ }
\setbox\contprefix=\hbox{\tt \ \ \ \ \ \ \ \ \ \ }
\startm
\m{\vdash}\m{(}\m{x}\m{=}\m{A}\m{\rightarrow}\m{(}\m{\varphi}\m{\leftrightarrow}
\m{\psi}\m{)}\m{)}
\endm
\vskip 1ex

is frequently used in place of explicit substitution, meaning ``assume
$\psi$ results from the proper substitution of $A$ for $x$ in
$\varphi$.''  Sometimes ``\texttt{\$e} $\vdash ( \psi \rightarrow
\forall x \, \psi )$'' is used instead of ``\texttt{\$d} $x\, \psi $,''
which requires only that $x$ be effectively not free in $\varphi$ but
not necessarily absent from it.  The use of implicit
substitution\index{substitution!implicit} is partly a matter of personal
style, although it may make proofs somewhat shorter than would be the
case with explicit substitution.

\item
The hypothesis


\vskip 1ex
\setbox\startprefix=\hbox{\tt \ \ \ \ \ \ \ \ \ \$e\ }
\setbox\contprefix=\hbox{\tt \ \ \ \ \ \ \ \ \ \ }
\startm
\m{\vdash}\m{A}\m{\in}\m{{\rm V}}
\endm
\vskip 1ex

should be read ``assume class $A$ is a set (i.e.\ exists).''
This is a convenient convention used by Quine.

\item
The restriction and hypothesis

\vskip 1ex
\setbox\startprefix=\hbox{\tt \ \ \ \ \ \ \ \ \ \$d\ }
\setbox\contprefix=\hbox{\tt \ \ \ \ \ \ \ \ \ \ }
\startm
\m{x}\m{\,y}
\endm
\setbox\startprefix=\hbox{\tt \ \ \ \ \ \ \ \ \ \$e\ }
\setbox\contprefix=\hbox{\tt \ \ \ \ \ \ \ \ \ \ }
\startm
\m{\vdash}\m{(}\m{y}\m{\in}\m{A}\m{\rightarrow}\m{\forall}\m{x}\m{\,y}
\m{\in}\m{A}\m{)}
\endm
\vskip 1ex

should be read ``assume variable $x$ is
(effectively) not free in class $A$.''

\end{itemize}

\section{A Theorem Sampler}\label{sometheorems}

In this section we list some of the more important theorems that are proved in
the \texttt{set.mm} database, and they illustrate the kinds of things that can be
done with Metamath.  While all of these facts are well-known results,
Metamath offers the advantage of easily allowing you to trace their
derivation back to axioms.  Our intent here is not to try to explain the
details or motivation; for this we refer you to the textbooks that are
mentioned in the descriptions.  (The \texttt{set.mm} file has bibliographic
references for the text references.)  Their proofs often embody important
concepts you may wish to explore with the Metamath program (see
Section~\ref{exploring}).  All the symbols that are used here are defined in
Section~\ref{hierarchy}.  For brevity we haven't included the \texttt{\$d}
restrictions or \texttt{\$f} hypotheses for these theorems; when you are
uncertain consult the \texttt{set.mm} database.

We start with \texttt{syl} (principle of the syllogism).
In \textit{Principia Mathematica}
Whitehead and Russell call this ``the principle of the
syllogism... because... the syllogism in Barbara is derived from them''
\cite[quote after Theorem *2.06 p.~101]{PM}.
Some authors call this law a ``hypothetical syllogism.''
As of 2019 \texttt{syl} is the most commonly referenced proven
assertion in the \texttt{set.mm} database.\footnote{
The Metamath program command \texttt{show usage}
shows the number of references.
On 2019-04-29 (commit 71cbbbdb387e) \texttt{syl} was directly referenced
10,819 times. The second most commonly referenced proven assertion
was \texttt{eqid}, which was directly referenced 7,738 times.
}

\vskip 2ex
\noindent Theorem syl (principle of the syllogism)\index{Syllogism}%
\index{\texttt{syl}}\label{syl}.

\vskip 0.5ex
\setbox\startprefix=\hbox{\tt \ \ syl.1\ \$e\ }
\setbox\contprefix=\hbox{\tt \ \ \ \ \ \ \ \ \ \ \ }
\startm
\m{\vdash}\m{(}\m{\varphi}\m{ \rightarrow }\m{\psi}\m{)}
\endm
\setbox\startprefix=\hbox{\tt \ \ syl.2\ \$e\ }
\setbox\contprefix=\hbox{\tt \ \ \ \ \ \ \ \ \ \ \ }
\startm
\m{\vdash}\m{(}\m{\psi}\m{ \rightarrow }\m{\chi}\m{)}
\endm
\setbox\startprefix=\hbox{\tt \ \ syl\ \$p\ }
\setbox\contprefix=\hbox{\tt \ \ \ \ \ \ \ \ \ }
\startm
\m{\vdash}\m{(}\m{\varphi}\m{ \rightarrow }\m{\chi}\m{)}
\endm
\vskip 2ex

The following theorem is not very deep but provides us with a notational device
that is frequently used.  It allows us to use the expression ``$A \in V$'' as
a compact way of saying that class $A$ exists, i.e.\ is a set.

\vskip 2ex
\noindent Two ways to say ``$A$ is a set'':  $A$ is a member of the universe
$V$ if and only if $A$ exists (i.e.\ there exists a set equal to $A$).
Theorem 6.9 of Quine, p. 43.

\vskip 0.5ex
\setbox\startprefix=\hbox{\tt \ \ isset\ \$p\ }
\setbox\contprefix=\hbox{\tt \ \ \ \ \ \ \ \ \ \ \ }
\startm
\m{\vdash}\m{(}\m{A}\m{\in}\m{{\rm V}}\m{\leftrightarrow}\m{\exists}\m{x}\m{\,x}\m{=}
\m{A}\m{)}
\endm
\vskip 1ex

Next we prove the axioms of standard ZF set theory that were missing from our
axiom system.  From our point of view they are theorems since they
can be derived from the other axioms.

\vskip 2ex
\noindent Axiom of Separation\index{Axiom of Separation}
(Aussonderung)\index{Aussonderung} proved from the other axioms of ZF set
theory.  Compare Exercise 4 of Takeuti and Zaring, p.~22.

\vskip 0.5ex
\setbox\startprefix=\hbox{\tt \ \ inex1.1\ \$e\ }
\setbox\contprefix=\hbox{\tt \ \ \ \ \ \ \ \ \ \ \ \ \ \ \ }
\startm
\m{\vdash}\m{A}\m{\in}\m{{\rm V}}
\endm
\setbox\startprefix=\hbox{\tt \ \ inex\ \$p\ }
\setbox\contprefix=\hbox{\tt \ \ \ \ \ \ \ \ \ \ \ \ \ }
\startm
\m{\vdash}\m{(}\m{A}\m{\cap}\m{B}\m{)}\m{\in}\m{{\rm V}}
\endm
\vskip 1ex

\noindent Axiom of the Null Set\index{Axiom of the Null Set} proved from the
other axioms of ZF set theory. Corollary 5.16 of Takeuti and Zaring, p.~20.

\vskip 0.5ex
\setbox\startprefix=\hbox{\tt \ \ 0ex\ \$p\ }
\setbox\contprefix=\hbox{\tt \ \ \ \ \ \ \ \ \ \ \ \ }
\startm
\m{\vdash}\m{\varnothing}\m{\in}\m{{\rm V}}
\endm
\vskip 1ex

\noindent The Axiom of Pairing\index{Axiom of Pairing} proved from the other
axioms of ZF set theory.  Theorem 7.13 of Quine, p.~51.
\vskip 0.5ex
\setbox\startprefix=\hbox{\tt \ \ prex\ \$p\ }
\setbox\contprefix=\hbox{\tt \ \ \ \ \ \ \ \ \ \ \ \ \ \ }
\startm
\m{\vdash}\m{\{}\m{A}\m{,}\m{B}\m{\}}\m{\in}\m{{\rm V}}
\endm
\vskip 2ex

Next we will list some famous or important theorems that are proved in
the \texttt{set.mm} database.  None of them except \texttt{omex}
require the Axiom of Infinity, as you can verify with the \texttt{show
trace{\char`\_}back} Metamath command.

\vskip 2ex
\noindent The resolution of Russell's paradox\index{Russell's paradox}.  There
exists no set containing the set of all sets which are not members of
themselves.  Proposition 4.14 of Takeuti and Zaring, p.~14.

\vskip 0.5ex
\setbox\startprefix=\hbox{\tt \ \ ru\ \$p\ }
\setbox\contprefix=\hbox{\tt \ \ \ \ \ \ \ \ }
\startm
\m{\vdash}\m{\lnot}\m{\exists}\m{x}\m{\,x}\m{=}\m{\{}\m{y}\m{|}\m{\lnot}\m{y}
\m{\in}\m{y}\m{\}}
\endm
\vskip 1ex

\noindent Cantor's theorem\index{Cantor's theorem}.  No set can be mapped onto
its power set.  Compare Theorem 6B(b) of Enderton, p.~132.

\vskip 0.5ex
\setbox\startprefix=\hbox{\tt \ \ canth.1\ \$e\ }
\setbox\contprefix=\hbox{\tt \ \ \ \ \ \ \ \ \ \ \ \ \ }
\startm
\m{\vdash}\m{A}\m{\in}\m{{\rm V}}
\endm
\setbox\startprefix=\hbox{\tt \ \ canth\ \$p\ }
\setbox\contprefix=\hbox{\tt \ \ \ \ \ \ \ \ \ \ \ }
\startm
\m{\vdash}\m{\lnot}\m{F}\m{:}\m{A}\m{\raisebox{.5ex}{${\textstyle{\:}_{
\mbox{\footnotesize\rm {\ }}}}\atop{\textstyle{\longrightarrow}\atop{
\textstyle{}^{\mbox{\footnotesize\rm onto}}}}$}}\m{{\cal P}}\m{A}
\endm
\vskip 1ex

\noindent The Burali-Forti paradox\index{Burali-Forti paradox}.  No set
contains all ordinal numbers. Enderton, p.~194.  (Burali-Forti was one person,
not two.)

\vskip 0.5ex
\setbox\startprefix=\hbox{\tt \ \ onprc\ \$p\ }
\setbox\contprefix=\hbox{\tt \ \ \ \ \ \ \ \ \ \ \ \ }
\startm
\m{\vdash}\m{\lnot}\m{\mbox{\rm On}}\m{\in}\m{{\rm V}}
\endm
\vskip 1ex

\noindent Peano's postulates\index{Peano's postulates} for arithmetic.
Proposition 7.30 of Takeuti and Zaring, pp.~42--43.  The objects being
described are the members of $\omega$ i.e.\ the natural numbers 0, 1,
2,\ldots.  The successor\index{successor} operation suc means ``plus
one.''  \texttt{peano1} says that 0 (which is defined as the empty set)
is a natural number.  \texttt{peano2} says that if $A$ is a natural
number, so is $A+1$.  \texttt{peano3} says that 0 is not the successor
of any natural number.  \texttt{peano4} says that two natural numbers
are equal if and only if their successors are equal.  \texttt{peano5} is
essentially the same as mathematical induction.

\vskip 1ex
\setbox\startprefix=\hbox{\tt \ \ peano1\ \$p\ }
\setbox\contprefix=\hbox{\tt \ \ \ \ \ \ \ \ \ \ \ \ }
\startm
\m{\vdash}\m{\varnothing}\m{\in}\m{\omega}
\endm
\vskip 1.5ex

\setbox\startprefix=\hbox{\tt \ \ peano2\ \$p\ }
\setbox\contprefix=\hbox{\tt \ \ \ \ \ \ \ \ \ \ \ \ }
\startm
\m{\vdash}\m{(}\m{A}\m{\in}\m{\omega}\m{\rightarrow}\m{{\rm suc}}\m{A}\m{\in}%
\m{\omega}\m{)}
\endm
\vskip 1.5ex

\setbox\startprefix=\hbox{\tt \ \ peano3\ \$p\ }
\setbox\contprefix=\hbox{\tt \ \ \ \ \ \ \ \ \ \ \ \ }
\startm
\m{\vdash}\m{(}\m{A}\m{\in}\m{\omega}\m{\rightarrow}\m{\lnot}\m{{\rm suc}}%
\m{A}\m{=}\m{\varnothing}\m{)}
\endm
\vskip 1.5ex

\setbox\startprefix=\hbox{\tt \ \ peano4\ \$p\ }
\setbox\contprefix=\hbox{\tt \ \ \ \ \ \ \ \ \ \ \ \ }
\startm
\m{\vdash}\m{(}\m{(}\m{A}\m{\in}\m{\omega}\m{\wedge}\m{B}\m{\in}\m{\omega}%
\m{)}\m{\rightarrow}\m{(}\m{{\rm suc}}\m{A}\m{=}\m{{\rm suc}}\m{B}\m{%
\leftrightarrow}\m{A}\m{=}\m{B}\m{)}\m{)}
\endm
\vskip 1.5ex

\setbox\startprefix=\hbox{\tt \ \ peano5\ \$p\ }
\setbox\contprefix=\hbox{\tt \ \ \ \ \ \ \ \ \ \ \ \ }
\startm
\m{\vdash}\m{(}\m{(}\m{\varnothing}\m{\in}\m{A}\m{\wedge}\m{\forall}\m{x}\m{%
\in}\m{\omega}\m{(}\m{x}\m{\in}\m{A}\m{\rightarrow}\m{{\rm suc}}\m{x}\m{\in}%
\m{A}\m{)}\m{)}\m{\rightarrow}\m{\omega}\m{\subseteq}\m{A}\m{)}
\endm
\vskip 1.5ex

\noindent Finite Induction (mathematical induction).\index{finite
induction}\index{mathematical induction} The first hypothesis is the
basis and the second is the induction hypothesis.  Theorem Schema 22 of
Suppes, p.~136.

\vskip 0.5ex
\setbox\startprefix=\hbox{\tt \ \ findes.1\ \$e\ }
\setbox\contprefix=\hbox{\tt \ \ \ \ \ \ \ \ \ \ \ \ \ \ }
\startm
\m{\vdash}\m{[}\m{\varnothing}\m{/}\m{x}\m{]}\m{\varphi}
\endm
\setbox\startprefix=\hbox{\tt \ \ findes.2\ \$e\ }
\setbox\contprefix=\hbox{\tt \ \ \ \ \ \ \ \ \ \ \ \ \ \ }
\startm
\m{\vdash}\m{(}\m{x}\m{\in}\m{\omega}\m{\rightarrow}\m{(}\m{\varphi}\m{%
\rightarrow}\m{[}\m{{\rm suc}}\m{x}\m{/}\m{x}\m{]}\m{\varphi}\m{)}\m{)}
\endm
\setbox\startprefix=\hbox{\tt \ \ findes\ \$p\ }
\setbox\contprefix=\hbox{\tt \ \ \ \ \ \ \ \ \ \ \ \ }
\startm
\m{\vdash}\m{(}\m{x}\m{\in}\m{\omega}\m{\rightarrow}\m{\varphi}\m{)}
\endm
\vskip 1ex

\noindent Transfinite Induction with explicit substitution.  The first
hypothesis is the basis, the second is the induction hypothesis for
successors, and the third is the induction hypothesis for limit
ordinals.  Theorem Schema 4 of Suppes, p. 197.

\vskip 0.5ex
\setbox\startprefix=\hbox{\tt \ \ tfindes.1\ \$e\ }
\setbox\contprefix=\hbox{\tt \ \ \ \ \ \ \ \ \ \ \ \ \ \ \ }
\startm
\m{\vdash}\m{[}\m{\varnothing}\m{/}\m{x}\m{]}\m{\varphi}
\endm
\setbox\startprefix=\hbox{\tt \ \ tfindes.2\ \$e\ }
\setbox\contprefix=\hbox{\tt \ \ \ \ \ \ \ \ \ \ \ \ \ \ \ }
\startm
\m{\vdash}\m{(}\m{x}\m{\in}\m{{\rm On}}\m{\rightarrow}\m{(}\m{\varphi}\m{%
\rightarrow}\m{[}\m{{\rm suc}}\m{x}\m{/}\m{x}\m{]}\m{\varphi}\m{)}\m{)}
\endm
\setbox\startprefix=\hbox{\tt \ \ tfindes.3\ \$e\ }
\setbox\contprefix=\hbox{\tt \ \ \ \ \ \ \ \ \ \ \ \ \ \ \ }
\startm
\m{\vdash}\m{(}\m{{\rm Lim}}\m{y}\m{\rightarrow}\m{(}\m{\forall}\m{x}\m{\in}%
\m{y}\m{\varphi}\m{\rightarrow}\m{[}\m{y}\m{/}\m{x}\m{]}\m{\varphi}\m{)}\m{)}
\endm
\setbox\startprefix=\hbox{\tt \ \ tfindes\ \$p\ }
\setbox\contprefix=\hbox{\tt \ \ \ \ \ \ \ \ \ \ \ \ \ }
\startm
\m{\vdash}\m{(}\m{x}\m{\in}\m{{\rm On}}\m{\rightarrow}\m{\varphi}\m{)}
\endm
\vskip 1ex

\noindent Principle of Transfinite Recursion.\index{transfinite
recursion} Theorem 7.41 of Takeuti and Zaring, p.~47.  Transfinite
recursion is the key theorem that allows arithmetic of ordinals to be
rigorously defined, and has many other important uses as well.
Hypotheses \texttt{tfr.1} and \texttt{tfr.2} specify a certain (proper)
class $ F$.  The complicated definition of $ F$ is not important in
itself; what is important is that there be such an $ F$ with the
required properties, and we show this by displaying $ F$ explicitly.
\texttt{tfr1} states that $ F$ is a function whose domain is the set of
ordinal numbers.  \texttt{tfr2} states that any value of $ F$ is
completely determined by its previous values and the values of an
auxiliary function, $G$.  \texttt{tfr3} states that $ F$ is unique,
i.e.\ it is the only function that satisfies \texttt{tfr1} and
\texttt{tfr2}.  Note that $ f$ is an individual variable like $x$ and
$y$; it is just a mnemonic to remind us that $A$ is a collection of
functions.

\vskip 0.5ex
\setbox\startprefix=\hbox{\tt \ \ tfr.1\ \$e\ }
\setbox\contprefix=\hbox{\tt \ \ \ \ \ \ \ \ \ \ \ }
\startm
\m{\vdash}\m{A}\m{=}\m{\{}\m{f}\m{|}\m{\exists}\m{x}\m{\in}\m{{\rm On}}\m{(}%
\m{f}\m{{\rm Fn}}\m{x}\m{\wedge}\m{\forall}\m{y}\m{\in}\m{x}\m{(}\m{f}\m{`}%
\m{y}\m{)}\m{=}\m{(}\m{G}\m{`}\m{(}\m{f}\m{\restriction}\m{y}\m{)}\m{)}\m{)}%
\m{\}}
\endm
\setbox\startprefix=\hbox{\tt \ \ tfr.2\ \$e\ }
\setbox\contprefix=\hbox{\tt \ \ \ \ \ \ \ \ \ \ \ }
\startm
\m{\vdash}\m{F}\m{=}\m{\bigcup}\m{A}
\endm
\setbox\startprefix=\hbox{\tt \ \ tfr1\ \$p\ }
\setbox\contprefix=\hbox{\tt \ \ \ \ \ \ \ \ \ \ }
\startm
\m{\vdash}\m{F}\m{{\rm Fn}}\m{{\rm On}}
\endm
\setbox\startprefix=\hbox{\tt \ \ tfr2\ \$p\ }
\setbox\contprefix=\hbox{\tt \ \ \ \ \ \ \ \ \ \ }
\startm
\m{\vdash}\m{(}\m{z}\m{\in}\m{{\rm On}}\m{\rightarrow}\m{(}\m{F}\m{`}\m{z}%
\m{)}\m{=}\m{(}\m{G}\m{`}\m{(}\m{F}\m{\restriction}\m{z}\m{)}\m{)}\m{)}
\endm
\setbox\startprefix=\hbox{\tt \ \ tfr3\ \$p\ }
\setbox\contprefix=\hbox{\tt \ \ \ \ \ \ \ \ \ \ }
\startm
\m{\vdash}\m{(}\m{(}\m{B}\m{{\rm Fn}}\m{{\rm On}}\m{\wedge}\m{\forall}\m{x}\m{%
\in}\m{{\rm On}}\m{(}\m{B}\m{`}\m{x}\m{)}\m{=}\m{(}\m{G}\m{`}\m{(}\m{B}\m{%
\restriction}\m{x}\m{)}\m{)}\m{)}\m{\rightarrow}\m{B}\m{=}\m{F}\m{)}
\endm
\vskip 1ex

\noindent The existence of omega (the class of natural numbers).\index{natural
number}\index{omega ($\omega$)}\index{Axiom of Infinity}  Axiom 7 of Takeuti
and Zaring, p.~43.  (This is the only theorem in this section requiring the
Axiom of Infinity.)

\vskip 0.5ex
\setbox\startprefix=\hbox{\tt \
\ omex\ \$p\ }
\setbox\contprefix=\hbox{\tt \ \ \ \ \ \ \ \ \ \ }
\startm
\m{\vdash}\m{\omega}\m{\in}\m{{\rm V}}
\endm
%\vskip 2ex


\section{Axioms for Real and Complex Numbers}\label{real}
\index{real and complex numbers!axioms for}

This section presents the axioms for real and complex numbers, along
with some commentary about them.  Analysis
textbooks implicitly or explicitly use these axioms or their equivalents
as their starting point.  In the database \texttt{set.mm}, we define real
and complex numbers as (rather complicated) specific sets and derive these
axioms as {\em theorems} from the axioms of ZF set theory, using a method
called Dedekind cuts.  We omit the details of this construction, which you can
follow if you wish using the \texttt{set.mm} database in conjunction with the
textbooks referenced therein.

Once we prove those theorems, we then restate these proven theorems as axioms.
This lets us easily identify which axioms are needed for a particular complex number proof, without the obfuscation of the set theory used to derive them.
As a result,
the construction is actually unimportant other
than to show that sets exist that satisfy the axioms, and thus that the axioms
are consistent if ZF set theory is consistent.  When working with real numbers
you can think of them as being the actual sets resulting
from the construction (for definiteness), or you can
think of them as otherwise unspecified sets that happen to satisfy the axioms.
The derivation is not easy, but the fact that it works is quite remarkable
and lends support to the idea that ZFC set theory is all we need to
provide a foundation for essentially all of mathematics.

\needspace{3\baselineskip}
\subsection{The Axioms for Real and Complex Numbers Themselves}\label{realactual}

For the axioms we are given (or postulate) 8 classes:  $\mathbb{C}$ (the
set of complex numbers), $\mathbb{R}$ (the set of real numbers, a subset
of $\mathbb{C}$), $0$ (zero), $1$ (one), $i$ (square root of
$-1$), $+$ (plus), $\cdot$ (times), and
$<_{\mathbb{R}}$ (less than for just the real numbers).
Subtraction and division are defined terms and are not part of the
axioms; for their definitions see \texttt{set.mm}.

Note that the notation $(A+B)$ (and similarly $(A\cdot B)$) specifies a class
called an {\em operation},\index{operation} and is the function value of the
class $+$ at ordered pair $\langle A,B \rangle$.  An operation is defined by
statement \texttt{df-opr} on p.~\pageref{dfopr}.
The notation $A <_{\mathbb{R}} B$ specifies a
wff called a {\em binary relation}\index{binary relation} and means $\langle A,B \rangle \in \,<_{\mathbb{R}}$, as defined by statement \texttt{df-br} on p.~\pageref{dfbr}.

Our set of 8 given classes is assumed to satisfy the following 22 axioms
(in the axioms listed below, $<$ really means $<_{\mathbb{R}}$).

\vskip 2ex

\noindent 1. The real numbers are a subset of the complex numbers.

%\vskip 0.5ex
\setbox\startprefix=\hbox{\tt \ \ ax-resscn\ \$p\ }
\setbox\contprefix=\hbox{\tt \ \ \ \ \ \ \ \ \ \ \ \ \ \ }
\startm
\m{\vdash}\m{\mathbb{R}}\m{\subseteq}\m{\mathbb{C}}
\endm
%\vskip 1ex

\noindent 2. One is a complex number.

%\vskip 0.5ex
\setbox\startprefix=\hbox{\tt \ \ ax-1cn\ \$p\ }
\setbox\contprefix=\hbox{\tt \ \ \ \ \ \ \ \ \ \ \ }
\startm
\m{\vdash}\m{1}\m{\in}\m{\mathbb{C}}
\endm
%\vskip 1ex

\noindent 3. The imaginary number $i$ is a complex number.

%\vskip 0.5ex
\setbox\startprefix=\hbox{\tt \ \ ax-icn\ \$p\ }
\setbox\contprefix=\hbox{\tt \ \ \ \ \ \ \ \ \ \ \ }
\startm
\m{\vdash}\m{i}\m{\in}\m{\mathbb{C}}
\endm
%\vskip 1ex

\noindent 4. Complex numbers are closed under addition.

%\vskip 0.5ex
\setbox\startprefix=\hbox{\tt \ \ ax-addcl\ \$p\ }
\setbox\contprefix=\hbox{\tt \ \ \ \ \ \ \ \ \ \ \ \ \ }
\startm
\m{\vdash}\m{(}\m{(}\m{A}\m{\in}\m{\mathbb{C}}\m{\wedge}\m{B}\m{\in}\m{\mathbb{C}}%
\m{)}\m{\rightarrow}\m{(}\m{A}\m{+}\m{B}\m{)}\m{\in}\m{\mathbb{C}}\m{)}
\endm
%\vskip 1ex

\noindent 5. Real numbers are closed under addition.

%\vskip 0.5ex
\setbox\startprefix=\hbox{\tt \ \ ax-addrcl\ \$p\ }
\setbox\contprefix=\hbox{\tt \ \ \ \ \ \ \ \ \ \ \ \ \ \ }
\startm
\m{\vdash}\m{(}\m{(}\m{A}\m{\in}\m{\mathbb{R}}\m{\wedge}\m{B}\m{\in}\m{\mathbb{R}}%
\m{)}\m{\rightarrow}\m{(}\m{A}\m{+}\m{B}\m{)}\m{\in}\m{\mathbb{R}}\m{)}
\endm
%\vskip 1ex

\noindent 6. Complex numbers are closed under multiplication.

%\vskip 0.5ex
\setbox\startprefix=\hbox{\tt \ \ ax-mulcl\ \$p\ }
\setbox\contprefix=\hbox{\tt \ \ \ \ \ \ \ \ \ \ \ \ \ }
\startm
\m{\vdash}\m{(}\m{(}\m{A}\m{\in}\m{\mathbb{C}}\m{\wedge}\m{B}\m{\in}\m{\mathbb{C}}%
\m{)}\m{\rightarrow}\m{(}\m{A}\m{\cdot}\m{B}\m{)}\m{\in}\m{\mathbb{C}}\m{)}
\endm
%\vskip 1ex

\noindent 7. Real numbers are closed under multiplication.

%\vskip 0.5ex
\setbox\startprefix=\hbox{\tt \ \ ax-mulrcl\ \$p\ }
\setbox\contprefix=\hbox{\tt \ \ \ \ \ \ \ \ \ \ \ \ \ \ }
\startm
\m{\vdash}\m{(}\m{(}\m{A}\m{\in}\m{\mathbb{R}}\m{\wedge}\m{B}\m{\in}\m{\mathbb{R}}%
\m{)}\m{\rightarrow}\m{(}\m{A}\m{\cdot}\m{B}\m{)}\m{\in}\m{\mathbb{R}}\m{)}
\endm
%\vskip 1ex

\noindent 8. Multiplication of complex numbers is commutative.

%\vskip 0.5ex
\setbox\startprefix=\hbox{\tt \ \ ax-mulcom\ \$p\ }
\setbox\contprefix=\hbox{\tt \ \ \ \ \ \ \ \ \ \ \ \ \ \ }
\startm
\m{\vdash}\m{(}\m{(}\m{A}\m{\in}\m{\mathbb{C}}\m{\wedge}\m{B}\m{\in}\m{\mathbb{C}}%
\m{)}\m{\rightarrow}\m{(}\m{A}\m{\cdot}\m{B}\m{)}\m{=}\m{(}\m{B}\m{\cdot}\m{A}%
\m{)}\m{)}
\endm
%\vskip 1ex

\noindent 9. Addition of complex numbers is associative.

%\vskip 0.5ex
\setbox\startprefix=\hbox{\tt \ \ ax-addass\ \$p\ }
\setbox\contprefix=\hbox{\tt \ \ \ \ \ \ \ \ \ \ \ \ \ \ }
\startm
\m{\vdash}\m{(}\m{(}\m{A}\m{\in}\m{\mathbb{C}}\m{\wedge}\m{B}\m{\in}\m{\mathbb{C}}%
\m{\wedge}\m{C}\m{\in}\m{\mathbb{C}}\m{)}\m{\rightarrow}\m{(}\m{(}\m{A}\m{+}%
\m{B}\m{)}\m{+}\m{C}\m{)}\m{=}\m{(}\m{A}\m{+}\m{(}\m{B}\m{+}\m{C}\m{)}\m{)}%
\m{)}
\endm
%\vskip 1ex

\noindent 10. Multiplication of complex numbers is associative.

%\vskip 0.5ex
\setbox\startprefix=\hbox{\tt \ \ ax-mulass\ \$p\ }
\setbox\contprefix=\hbox{\tt \ \ \ \ \ \ \ \ \ \ \ \ \ \ }
\startm
\m{\vdash}\m{(}\m{(}\m{A}\m{\in}\m{\mathbb{C}}\m{\wedge}\m{B}\m{\in}\m{\mathbb{C}}%
\m{\wedge}\m{C}\m{\in}\m{\mathbb{C}}\m{)}\m{\rightarrow}\m{(}\m{(}\m{A}\m{\cdot}%
\m{B}\m{)}\m{\cdot}\m{C}\m{)}\m{=}\m{(}\m{A}\m{\cdot}\m{(}\m{B}\m{\cdot}\m{C}%
\m{)}\m{)}\m{)}
\endm
%\vskip 1ex

\noindent 11. Multiplication distributes over addition for complex numbers.

%\vskip 0.5ex
\setbox\startprefix=\hbox{\tt \ \ ax-distr\ \$p\ }
\setbox\contprefix=\hbox{\tt \ \ \ \ \ \ \ \ \ \ \ \ \ }
\startm
\m{\vdash}\m{(}\m{(}\m{A}\m{\in}\m{\mathbb{C}}\m{\wedge}\m{B}\m{\in}\m{\mathbb{C}}%
\m{\wedge}\m{C}\m{\in}\m{\mathbb{C}}\m{)}\m{\rightarrow}\m{(}\m{A}\m{\cdot}\m{(}%
\m{B}\m{+}\m{C}\m{)}\m{)}\m{=}\m{(}\m{(}\m{A}\m{\cdot}\m{B}\m{)}\m{+}\m{(}%
\m{A}\m{\cdot}\m{C}\m{)}\m{)}\m{)}
\endm
%\vskip 1ex

\noindent 12. The square of $i$ equals $-1$ (expressed as $i$-squared plus 1 is
0).

%\vskip 0.5ex
\setbox\startprefix=\hbox{\tt \ \ ax-i2m1\ \$p\ }
\setbox\contprefix=\hbox{\tt \ \ \ \ \ \ \ \ \ \ \ \ }
\startm
\m{\vdash}\m{(}\m{(}\m{i}\m{\cdot}\m{i}\m{)}\m{+}\m{1}\m{)}\m{=}\m{0}
\endm
%\vskip 1ex

\noindent 13. One and zero are distinct.

%\vskip 0.5ex
\setbox\startprefix=\hbox{\tt \ \ ax-1ne0\ \$p\ }
\setbox\contprefix=\hbox{\tt \ \ \ \ \ \ \ \ \ \ \ \ }
\startm
\m{\vdash}\m{1}\m{\ne}\m{0}
\endm
%\vskip 1ex

\noindent 14. One is an identity element for real multiplication.

%\vskip 0.5ex
\setbox\startprefix=\hbox{\tt \ \ ax-1rid\ \$p\ }
\setbox\contprefix=\hbox{\tt \ \ \ \ \ \ \ \ \ \ \ }
\startm
\m{\vdash}\m{(}\m{A}\m{\in}\m{\mathbb{R}}\m{\rightarrow}\m{(}\m{A}\m{\cdot}\m{1}%
\m{)}\m{=}\m{A}\m{)}
\endm
%\vskip 1ex

\noindent 15. Every real number has a negative.

%\vskip 0.5ex
\setbox\startprefix=\hbox{\tt \ \ ax-rnegex\ \$p\ }
\setbox\contprefix=\hbox{\tt \ \ \ \ \ \ \ \ \ \ \ \ \ \ }
\startm
\m{\vdash}\m{(}\m{A}\m{\in}\m{\mathbb{R}}\m{\rightarrow}\m{\exists}\m{x}\m{\in}%
\m{\mathbb{R}}\m{(}\m{A}\m{+}\m{x}\m{)}\m{=}\m{0}\m{)}
\endm
%\vskip 1ex

\noindent 16. Every nonzero real number has a reciprocal.

%\vskip 0.5ex
\setbox\startprefix=\hbox{\tt \ \ ax-rrecex\ \$p\ }
\setbox\contprefix=\hbox{\tt \ \ \ \ \ \ \ \ \ \ \ \ \ \ }
\startm
\m{\vdash}\m{(}\m{A}\m{\in}\m{\mathbb{R}}\m{\rightarrow}\m{(}\m{A}\m{\ne}\m{0}%
\m{\rightarrow}\m{\exists}\m{x}\m{\in}\m{\mathbb{R}}\m{(}\m{A}\m{\cdot}%
\m{x}\m{)}\m{=}\m{1}\m{)}\m{)}
\endm
%\vskip 1ex

\noindent 17. A complex number can be expressed in terms of two reals.

%\vskip 0.5ex
\setbox\startprefix=\hbox{\tt \ \ ax-cnre\ \$p\ }
\setbox\contprefix=\hbox{\tt \ \ \ \ \ \ \ \ \ \ \ \ }
\startm
\m{\vdash}\m{(}\m{A}\m{\in}\m{\mathbb{C}}\m{\rightarrow}\m{\exists}\m{x}\m{\in}%
\m{\mathbb{R}}\m{\exists}\m{y}\m{\in}\m{\mathbb{R}}\m{A}\m{=}\m{(}\m{x}\m{+}\m{(}%
\m{y}\m{\cdot}\m{i}\m{)}\m{)}\m{)}
\endm
%\vskip 1ex

\noindent 18. Ordering on reals satisfies strict trichotomy.

%\vskip 0.5ex
\setbox\startprefix=\hbox{\tt \ \ ax-pre-lttri\ \$p\ }
\setbox\contprefix=\hbox{\tt \ \ \ \ \ \ \ \ \ \ \ \ \ }
\startm
\m{\vdash}\m{(}\m{(}\m{A}\m{\in}\m{\mathbb{R}}\m{\wedge}\m{B}\m{\in}\m{\mathbb{R}}%
\m{)}\m{\rightarrow}\m{(}\m{A}\m{<}\m{B}\m{\leftrightarrow}\m{\lnot}\m{(}\m{A}%
\m{=}\m{B}\m{\vee}\m{B}\m{<}\m{A}\m{)}\m{)}\m{)}
\endm
%\vskip 1ex

\noindent 19. Ordering on reals is transitive.

%\vskip 0.5ex
\setbox\startprefix=\hbox{\tt \ \ ax-pre-lttrn\ \$p\ }
\setbox\contprefix=\hbox{\tt \ \ \ \ \ \ \ \ \ \ \ \ \ }
\startm
\m{\vdash}\m{(}\m{(}\m{A}\m{\in}\m{\mathbb{R}}\m{\wedge}\m{B}\m{\in}\m{\mathbb{R}}%
\m{\wedge}\m{C}\m{\in}\m{\mathbb{R}}\m{)}\m{\rightarrow}\m{(}\m{(}\m{A}\m{<}%
\m{B}\m{\wedge}\m{B}\m{<}\m{C}\m{)}\m{\rightarrow}\m{A}\m{<}\m{C}\m{)}\m{)}
\endm
%\vskip 1ex

\noindent 20. Ordering on reals is preserved after addition to both sides.

%\vskip 0.5ex
\setbox\startprefix=\hbox{\tt \ \ ax-pre-ltadd\ \$p\ }
\setbox\contprefix=\hbox{\tt \ \ \ \ \ \ \ \ \ \ \ \ \ }
\startm
\m{\vdash}\m{(}\m{(}\m{A}\m{\in}\m{\mathbb{R}}\m{\wedge}\m{B}\m{\in}\m{\mathbb{R}}%
\m{\wedge}\m{C}\m{\in}\m{\mathbb{R}}\m{)}\m{\rightarrow}\m{(}\m{A}\m{<}\m{B}\m{%
\rightarrow}\m{(}\m{C}\m{+}\m{A}\m{)}\m{<}\m{(}\m{C}\m{+}\m{B}\m{)}\m{)}\m{)}
\endm
%\vskip 1ex

\noindent 21. The product of two positive reals is positive.

%\vskip 0.5ex
\setbox\startprefix=\hbox{\tt \ \ ax-pre-mulgt0\ \$p\ }
\setbox\contprefix=\hbox{\tt \ \ \ \ \ \ \ \ \ \ \ \ \ \ }
\startm
\m{\vdash}\m{(}\m{(}\m{A}\m{\in}\m{\mathbb{R}}\m{\wedge}\m{B}\m{\in}\m{\mathbb{R}}%
\m{)}\m{\rightarrow}\m{(}\m{(}\m{0}\m{<}\m{A}\m{\wedge}\m{0}%
\m{<}\m{B}\m{)}\m{\rightarrow}\m{0}\m{<}\m{(}\m{A}\m{\cdot}\m{B}\m{)}%
\m{)}\m{)}
\endm
%\vskip 1ex

\noindent 22. A non-empty, bounded-above set of reals has a supremum.

%\vskip 0.5ex
\setbox\startprefix=\hbox{\tt \ \ ax-pre-sup\ \$p\ }
\setbox\contprefix=\hbox{\tt \ \ \ \ \ \ \ \ \ \ \ }
\startm
\m{\vdash}\m{(}\m{(}\m{A}\m{\subseteq}\m{\mathbb{R}}\m{\wedge}\m{A}\m{\ne}\m{%
\varnothing}\m{\wedge}\m{\exists}\m{x}\m{\in}\m{\mathbb{R}}\m{\forall}\m{y}\m{%
\in}\m{A}\m{\,y}\m{<}\m{x}\m{)}\m{\rightarrow}\m{\exists}\m{x}\m{\in}\m{%
\mathbb{R}}\m{(}\m{\forall}\m{y}\m{\in}\m{A}\m{\lnot}\m{x}\m{<}\m{y}\m{\wedge}\m{%
\forall}\m{y}\m{\in}\m{\mathbb{R}}\m{(}\m{y}\m{<}\m{x}\m{\rightarrow}\m{\exists}%
\m{z}\m{\in}\m{A}\m{\,y}\m{<}\m{z}\m{)}\m{)}\m{)}
\endm

% NOTE: The \m{...} expressions above could be represented as
% $ \vdash ( ( A \subseteq \mathbb{R} \wedge A \ne \varnothing \wedge \exists x \in \mathbb{R} \forall y \in A \,y < x ) \rightarrow \exists x \in \mathbb{R} ( \forall y \in A \lnot x < y \wedge \forall y \in \mathbb{R} ( y < x \rightarrow \exists z \in A \,y < z ) ) ) $

\vskip 2ex

This completes the set of axioms for real and complex numbers.  You may
wish to look at how subtraction, division, and decimal numbers
are defined in \texttt{set.mm}, and for fun look at the proof of $2+
2 = 4$ (theorem \texttt{2p2e4} in \texttt{set.mm})
as discussed in section \ref{2p2e4}.

In \texttt{set.mm} we define the non-negative integers $\mathbb{N}$, the integers
$\mathbb{Z}$, and the rationals $\mathbb{Q}$ as subsets of $\mathbb{R}$.  This leads
to the nice inclusion $\mathbb{N} \subseteq \mathbb{Z} \subseteq \mathbb{Q} \subseteq
\mathbb{R} \subseteq \mathbb{C}$, giving us a uniform framework in which, for
example, a property such as commutativity of complex number addition
automatically applies to integers.  The natural numbers $\mathbb{N}$
are different from the set $\omega$ we defined earlier, but both satisfy
Peano's postulates.

\subsection{Complex Number Axioms in Analysis Texts}

Most analysis texts construct complex numbers as ordered pairs of reals,
leading to construction-dependent properties that satisfy these axioms
but are not stated in their pure form.  (This is also done in
\texttt{set.mm} but our axioms are extracted from that construction.)
Other texts will simply state that $\mathbb{R}$ is a ``complete ordered
subfield of $\mathbb{C}$,'' leading to redundant axioms when this phrase
is completely expanded out.  In fact I have not seen a text with the
axioms in the explicit form above.
None of these axioms is unique individually, but this carefully worked out
collection of axioms is the result of years of work
by the Metamath community.

\subsection{Eliminating Unnecessary Complex Number Axioms}

We once had more axioms for real and complex numbers, but over years of time
we (the Metamath community)
have found ways to eliminate them (by proving them from other axioms)
or weaken them (by making weaker claims without reducing what
can be proved).
In particular, here are statements that used to be complex number
axioms but have since been formally proven (with Metamath) to be redundant:

\begin{itemize}
\item
  $\mathbb{C} \in V$.
  At one time this was listed as a ``complex number axiom.''
  However, this is not properly speaking a complex number axiom,
  and in any case its proof uses axioms of set theory.
  Proven redundant by Mario Carneiro\index{Carneiro, Mario} on
  17-Nov-2014 (see \texttt{axcnex}).
\item
  $((A \in \mathbb{C} \land B \in \mathbb{C}$) $\rightarrow$
  $(A + B) = (B + A))$.
  Proved redundant by Eric Schmidt\index{Schmidt, Eric} on 19-Jun-2012,
  and formalized by Scott Fenton\index{Fenton, Scott} on 3-Jan-2013
  (see \texttt{addcom}).
\item
  $(A \in \mathbb{C} \rightarrow (A + 0) = A)$.
  Proved redundant by Eric Schmidt on 19-Jun-2012,
  and formalized by Scott Fenton on 3-Jan-2013
  (see \texttt{addid1}).
\item
  $(A \in \mathbb{C} \rightarrow \exists x \in \mathbb{C} (A + x) = 0)$.
  Proved redundant by Eric Schmidt and formalized on 21-May-2007
  (see \texttt{cnegex}).
\item
  $((A \in \mathbb{C} \land A \ne 0) \rightarrow \exists x \in \mathbb{C} (A \cdot x) = 1)$.
  Proved redundant by Eric Schmidt and formalized on 22-May-2007
  (see \texttt{recex}).
\item
  $0 \in \mathbb{R}$.
  Proved redundant by Eric Schmidt on 19-Feb-2005 and formalized 21-May-2007
  (see \texttt{0re}).
\end{itemize}

We could eliminate 0 as an axiomatic object by defining it as
$( ( i \cdot i ) + 1 )$
and replacing it with this expression throughout the axioms. If this
is done, axiom ax-i2m1 becomes redundant. However, the remaining axioms
would become longer and less intuitive.

Eric Schmidt's paper analyzing this axiom system \cite{Schmidt}
presented a proof that these remaining axioms,
with the possible exception of ax-mulcom, are independent of the others.
It is currently an open question if ax-mulcom is independent of the others.

\section{Two Plus Two Equals Four}\label{2p2e4}

Here is a proof that $2 + 2 = 4$, as proven in the theorem \texttt{2p2e4}
in the database \texttt{set.mm}.
This is a useful demonstration of what a Metamath proof can look like.
This proof may have more steps than you're used to, but each step is rigorously
proven all the way back to the axioms of logic and set theory.
This display was originally generated by the Metamath program
as an {\sc HTML} file.

In the table showing the proof ``Step'' is the sequential step number,
while its associated ``Expression'' is an expression that we have proved.
``Ref'' is the name of a theorem or axiom that justifies that expression,
and ``Hyp'' refers to previous steps (if any) that the theorem or axiom
needs so that we can use it.  Expressions are indented further than
the expressions that depend on them to show their interdependencies.

\begin{table}[!htbp]
\caption{Two plus two equals four}
\begin{tabular}{lllll}
\textbf{Step} & \textbf{Hyp} & \textbf{Ref} & \textbf{Expression} & \\
1  &       & df-2    & $ \; \; \vdash 2 = 1 + 1$  & \\
2  & 1     & oveq2i  & $ \; \vdash (2 + 2) = (2 + (1 + 1))$ & \\
3  &       & df-4    & $ \; \; \vdash 4 = (3 + 1)$ & \\
4  &       & df-3    & $ \; \; \; \vdash 3 = (2 + 1)$ & \\
5  & 4     & oveq1i  & $ \; \; \vdash (3 + 1) = ((2 + 1) + 1)$ & \\
6  &       & 2cn     & $ \; \; \; \vdash 2 \in \mathbb{C}$ & \\
7  &       & ax-1cn  & $ \; \; \; \vdash 1 \in \mathbb{C}$ & \\
8  & 6,7,7 & addassi & $ \; \; \vdash ((2 + 1) + 1) = (2 + (1 + 1))$ & \\
9  & 3,5,8 & 3eqtri  & $ \; \vdash 4 = (2 + (1 + 1))$ & \\
10 & 2,9   & eqtr4i  & $ \vdash (2 + 2) = 4$ & \\
\end{tabular}
\end{table}

Step 1 says that we can assert that $2 = 1 + 1$ because it is
justified by \texttt{df-2}.
What is \texttt{df-2}?
It is simply the definition of $2$, which in our system is defined as being
equal to $1 + 1$.  This shows how we can use definitions in proofs.

Look at Step 2 of the proof. In the Ref column, we see that it references
a previously proved theorem, \texttt{oveq2i}.
It turns out that
theorem \texttt{oveq2i} requires a
hypothesis, and in the Hyp column of Step 2 we indicate that Step 1 will
satisfy (match) this hypothesis.
If we looked at \texttt{oveq2i}
we would find that it proves that given some hypothesis
$A = B$, we can prove that $( C F A ) = ( C F B )$.
If we use \texttt{oveq2i} and apply step 1's result as the hypothesis,
that will mean that $A = 2$ and $B = ( 1 + 1 )$ within this use of
\texttt{oveq2i}.
We are free to select any value of $C$ and $F$ (subject to syntax constraints),
so we are free to select $C = 2$ and $F = +$,
producing our desired result,
$ (2 + 2) = (2 + (1 + 1))$.

Step 2 is an example of substitution.
In the end, every step in every proof uses only this one substitution rule.
All the rules of logic, and all the axioms, are expressed so that
they can be used via this one substitution rule.
So once you master substitution, you can master every Metamath proof,
no exceptions.

Each step is clear and can be immediately checked.
In the {\sc HTML} display you can even click on each reference to see why it is
justified, making it easy to see why the proof works.

\section{Deduction}\label{deduction}

Strictly speaking,
a deduction (also called an inference) is a kind of statement that needs
some hypotheses to be true in order for its conclusion to be true.
A theorem, on the other hand, has no hypotheses.
Informally we often call both of them theorems, but in this section we
will stick to the strict definitions.

It sometimes happens that we have proved a deduction of the form
$\varphi \Rightarrow \psi$\index{$\Rightarrow$}
(given hypothesis $\varphi$ we can prove $\psi$)
and we want to then prove a theorem of the form
$\varphi \rightarrow \psi$.

Converting a deduction (which uses a hypothesis) into a theorem
(which does not) is not as simple as you might think.
The deduction says, ``if we can prove $\varphi$ then we can prove $\psi$,''
which is in some sense weaker than saying
``$\varphi$ implies $\psi$.''
There is no axiom of logic that permits us to directly obtain the theorem
given the deduction.\footnote{
The conversion of a deduction to a theorem does not even hold in general
for quantum propositional calculus,
which is a weak subset of classical propositional calculus.
It has been shown that adding the Standard Deduction Theorem (discussed below)
to quantum propositional calculus turns it into classical
propositional calculus!
}

This is in contrast to going the other way.
If we have the theorem ($\varphi \rightarrow \psi$),
it is easy to recover the deduction
($\varphi \Rightarrow \psi$)
using modus ponens\index{modus ponens}
(\texttt{ax-mp}; see section \ref{axmp}).

In the following subsections we first discuss the standard deduction theorem
(the traditional but awkward way to convert deductions into theorems) and
the weak deduction theorem (a limited version of the standard deduction
theorem that is easier to use and was once widely used in
\texttt{set.mm}\index{set theory database (\texttt{set.mm})}).
In section \ref{deductionstyle} we discuss
deduction style, the newer approach we now recommend in most cases.
Deduction style uses ``deduction form,'' a form that
prefixes each hypothesis (other than definitions) and the
conclusion with a universal antecedent (``$\varphi \rightarrow$'').
Deduction style is widely used in \texttt{set.mm},
so it is useful to understand it and \textit{why} it is widely used.
Section \ref{naturaldeduction}
briefly discusses our approach for using natural deduction
within \texttt{set.mm},
as that approach is deeply related to deduction style.
We conclude with a summary of the strengths of
our approach, which we believe are compelling.

\subsection{The Standard Deduction Theorem}\label{standarddeductiontheorem}

It is possible to make use of information
contained in the deduction or its proof to assist us with the proof of
the related theorem.
In traditional logic books, there is a metatheorem called the
Deduction Theorem\index{Deduction Theorem}\index{Standard Deduction Theorem},
discovered independently by Herbrand and Tarski around 1930.
The Deduction Theorem, which we often call the Standard Deduction Theorem,
provides an algorithm for constructing a proof of a theorem from the
proof of its corresponding deduction. See, for example,
\cite[p.~56]{Margaris}\index{Margaris, Angelo}.
To construct a proof for a theorem, the
algorithm looks at each step in the proof of the original deduction and
rewrites the step with several steps wherein the hypothesis is eliminated
and becomes an antecedent.

In ordinary mathematics, no one actually carries out the algorithm,
because (in its most basic form) it involves an exponential explosion of
the number of proof steps as more hypotheses are eliminated. Instead,
the Standard Deduction Theorem is invoked simply to claim that it can
be done in principle, without actually doing it.
What's more, the algorithm is not as simple as it might first appear
when applying it rigorously.
There is a subtle restriction on the Standard Deduction Theorem
that must be taken into account involving the axiom of generalization
when working with predicate calculus (see the literature for more detail).

One of the goals of Metamath is to let you plainly see, with as few
underlying concepts as possible, how mathematics can be derived directly
from the axioms, and not indirectly according to some hidden rules
buried inside a program or understood only by logicians. If we added
the Standard Deduction Theorem to the language and proof verifier,
that would greatly complicate both and largely defeat Metamath's goal
of simplicity. In principle, we could show direct proofs by expanding
out the proof steps generated by the algorithm of the Standard Deduction
Theorem, but that is not feasible in practice because the number of proof
steps quickly becomes huge, even astronomical.
Since the algorithm of the Standard Deduction Theorem is driven by the proof,
we would have to go through that proof
all over again---starting from axioms---in order to obtain the theorem form.
In terms of proof length, there would be no savings over just
proving the theorem directly instead of first proving the deduction form.

\subsection{Weak Deduction Theorem}\label{weakdeductiontheorem}

We have developed
a more efficient method for proving a theorem from a deduction
that can be used instead of the Standard Deduction Theorem
in many (but not all) cases.
We call this more efficient method the
Weak Deduction Theorem\index{Weak Deduction Theorem}.\footnote{
There is also an unrelated ``Weak Deduction Theorem''
in the field of relevance logic, so to avoid confusion we could call
ours the ``Weak Deduction Theorem for Classical Logic.''}
Unlike the Standard Deduction Theorem, the Weak Deduction Theorem produces the
theorem directly from a special substitution instance of the deduction,
using a small, fixed number of steps roughly proportional to the length
of the final theorem.

If you come to a proof referencing the Weak Deduction Theorem
\texttt{dedth} (or one of its variants \texttt{dedthxx}),
here is how to follow the proof without getting into the details:
just click on the theorem referenced in the step
just before the reference to \texttt{dedth} and ignore everything else.
Theorem \texttt{dedth} simply turns a hypothesis into an antecedent
(i.e. the hypothesis followed by $\rightarrow$
is placed in front of the assertion, and the hypothesis
itself is eliminated) given certain conditions.

The Weak Deduction Theorem
eliminates a hypothesis $\varphi$, making it become an antecedent.
It does this by proving an expression
$ \varphi \rightarrow \psi $ given two hypotheses:
(1)
$ ( A = {\rm if} ( \varphi , A , B ) \rightarrow ( \varphi \leftrightarrow \chi ) ) $
and
(2) $\chi$.
Note that it requires that a proof exists for $\varphi$ when the class variable
$A$ is replaced with a specific class $B$. The hypothesis $\chi$
should be assigned to the inference.
You can see the details of the proof of the Weak Deduction Theorem
in theorem \texttt{dedth}.

The Weak Deduction Theorem
is probably easier to understand by studying proofs that make use of it.
For example, let's look at the proof of \texttt{renegcl}, which proves that
$ \vdash ( A \in \mathbb{R} \rightarrow - A \in \mathbb{R} )$:

\needspace{4\baselineskip}
\begin{longtabu} {l l l X}
\textbf{Step} & \textbf{Hyp} & \textbf{Ref} & \textbf{Expression} \\
  1 &  & negeq &
  $\vdash$ $($ $A$ $=$ ${\rm if}$ $($ $A$ $\in$ $\mathbb{R}$ $,$ $A$ $,$ $1$ $)$ $\rightarrow$
  $\textrm{-}$ $A$ $=$ $\textrm{-}$ ${\rm if}$ $($ $A$ $\in$ $\mathbb{R}$
  $,$ $A$ $,$ $1$ $)$ $)$ \\
 2 & 1 & eleq1d &
    $\vdash$ $($ $A$ $=$ ${\rm if}$ $($ $A$ $\in$ $\mathbb{R}$ $,$ $A$ $,$ $1$ $)$ $\rightarrow$ $($
    $\textrm{-}$ $A$ $\in$ $\mathbb{R}$ $\leftrightarrow$
    $\textrm{-}$ ${\rm if}$ $($ $A$ $\in$ $\mathbb{R}$ $,$ $A$ $,$ $1$ $)$ $\in$
    $\mathbb{R}$ $)$ $)$ \\
 3 &  & 1re & $\vdash 1 \in \mathbb{R}$ \\
 4 & 3 & elimel &
   $\vdash {\rm if} ( A \in \mathbb{R} , A , 1 ) \in \mathbb{R}$ \\
 5 & 4 & renegcli &
   $\vdash \textrm{-} {\rm if} ( A \in \mathbb{R} , A , 1 ) \in \mathbb{R}$ \\
 6 & 2,5 & dedth &
   $\vdash ( A \in \mathbb{R} \rightarrow \textrm{-} A \in \mathbb{R}$ ) \\
\end{longtabu}

The somewhat strange-looking steps in \texttt{renegcl} before step 5 are
technical stuff that makes this magic work, and they can be ignored
for a quick overview of the proof. To continue following the ``important''
part of the proof of \texttt{renegcl},
you can look at the reference to \texttt{renegcli} at step 5.

That said, let's briefly look at how
\texttt{renegcl} uses the
Weak Deduction Theorem (\texttt{dedth}) to do its job,
in case you want to do something similar or want understand it more deeply.
Let's work backwards in the proof of \texttt{renegcl}.
Step 6 applies \texttt{dedth} to produce our goal result
$ \vdash ( A \in \mathbb{R} \rightarrow\, - A \in \mathbb{R} )$.
This requires on the one hand the (substituted) deduction
\texttt{renegcli} in step 5.
By itself \texttt{renegcli} proves the deduction
$ \vdash A \in \mathbb{R} \Rightarrow\, \vdash - A \in \mathbb{R}$;
this is the deduction form we are trying to turn into theorem form,
and thus
\texttt{renegcli} has a separate hypothesis that must be fulfilled.
To fulfill the hypothesis of the invocation of
\texttt{renegcli} in step 5, it is eventually
reduced to the already proven theorem $1 \in \mathbb{R}$ in step 3.
Step 4 connects steps 3 and 5; step 4 invokes
\texttt{elimel}, a special case of \texttt{elimhyp} that eliminates
a membership hypothesis for the weak deduction theorem.
On the other hand, the equivalence of the conclusion of
\texttt{renegcl}
$( - A \in \mathbb{R} )$ and the substituted conclusion of
\texttt{renegcli} must be proven, which is done in steps 2 and 1.

The weak deduction theorem has limitations.
In particular, we must be able to prove a special case of the deduction's
hypothesis as a stand-alone theorem.
For example, we used $1 \in \mathbb{R}$ in step 3 of \texttt{renegcl}.

We used to use the weak deduction theorem
extensively within \texttt{set.mm}.
However, we now recommend applying ``deduction style''
instead in most cases, as deduction style is
often an easier and clearer approach.
Therefore, we will now describe deduction style.

\subsection{Deduction Style}\label{deductionstyle}

We now prefer to write assertions in ``deduction form''
instead of writing a proof that would require use of the standard or
weak deduction theorem.
We call this appraoch
``deduction style.''\index{deduction style}

It will be easier to explain this by first defining some terms:

\begin{itemize}
\item \textbf{closed form}\index{closed form}\index{forms!closed}:
A kind of assertion (theorem) with no hypotheses.
Typically its label has no special suffix.
An example is \texttt{unss}, which states:
$\vdash ( ( A \subseteq C \wedge B \subseteq C ) \leftrightarrow ( A \cup B )
\subseteq C )\label{eq:unss}$
\item \textbf{deduction form}\index{deduction form}\index{forms!deduction}:
A kind of assertion with one or more hypotheses
where the conclusion is an implication with
a wff variable as the antecedent (usually $\varphi$), and every hypothesis
(\$e statement)
is either (1) an implication with the same antecedent as the conclusion or
(2) a definition.
A definition
can be for a class variable (this is a class variable followed by ``='')
or a wff variable (this is a wff variable followed by $\leftrightarrow$);
class variable definitions are more common.
In practice, a proof
in deduction form will also contain many steps that are implications
where the antecedent is either that wff variable (normally $\varphi$)
or is
a conjunction (...$\land$...) including that wff variable ($\varphi$).
If an assertion is in deduction form, and other forms are also available,
then we suffix its label with ``d.''
An example is \texttt{unssd}, which states\footnote{
For brevity we show here (and in other places)
a $\&$\index{$\&$} between hypotheses\index{hypotheses}
and a $\Rightarrow$\index{$\Rightarrow$}\index{conclusion}
between the hypotheses and the conclusion.
This notation is technically not part of the Metamath language, but is
instead a convenient abbreviation to show both the hypotheses and conclusion.}:
$\vdash ( \varphi \rightarrow A \subseteq C )\quad\&\quad \vdash ( \varphi
    \rightarrow B \subseteq C )\quad\Rightarrow\quad \vdash ( \varphi
    \rightarrow ( A \cup B ) \subseteq C )\label{eq:unssd}$
\item \textbf{inference form}\index{inference form}\index{forms!inference}:
A kind of assertion with one or more hypotheses that is not in deduction form
(e.g., there is no common antecedent).
If an assertion is in inference form, and other forms are also available,
then we suffix its label with ``i.''
An example is \texttt{unssi}, which states:
$\vdash A \subseteq C\quad\&\quad \vdash B \subseteq C\quad\Rightarrow\quad
    \vdash ( A \cup B ) \subseteq C\label{eq:unssi}$
\end{itemize}

When using deduction style we express an assertion in deduction form.
This form prefixes each hypothesis (other than definitions) and the
conclusion with a universal antecedent (``$\varphi \rightarrow$'').
The antecedent (e.g., $\varphi$)
mimics the context handled in the deduction theorem, eliminating
the need to directly use the deduction theorem.

Once you have an assertion in deduction form, you can easily convert it
to inference form or closed form:

\begin{itemize}
\item To
prove some assertion Ti in inference form, given assertion Td in deduction
form, there is a simple mechanical process you can use. First take each
Ti hypothesis and insert a \texttt{T.} $\rightarrow$ prefix (``true implies'')
using \texttt{a1i}. You
can then use the existing assertion Td to prove the resulting conclusion
with a \texttt{T.} $\rightarrow$ prefix.
Finally, you can remove that prefix using \texttt{mptru},
resulting in the conclusion you wanted to prove.
\item To
prove some assertion T in closed form, given assertion Td in deduction
form, there is another simple mechanical process you can use. First,
select an expression that is the conjunction (...$\land$...) of all of the
consequents of every hypothesis of Td. Next, prove that this expression
implies each of the separate hypotheses of Td in turn by eliminating
conjuncts (there are a variety of proven assertions to do this, including
\texttt{simpl},
\texttt{simpr},
\texttt{3simpa},
\texttt{3simpb},
\texttt{3simpc},
\texttt{simp1},
\texttt{simp2},
and
\texttt{simp3}).
If the
expression has nested conjunctions, inner conjuncts can be broken out by
chaining the above theorems with \texttt{syl}
(see section \ref{syl}).\footnote{
There are actually many theorems
(labeled simp* such as \texttt{simp333}) that break out inner conjuncts in one
step, but rather than learning them you can just use the chaining we
just described to prove them, and then let the Metamath program command
\texttt{minimize{\char`\_}with}\index{\texttt{minimize{\char`\_}with} command}
figure out the right ones needed to collapse them.}
As your final step, you can then apply the already-proven assertion Td
(which is in deduction form), proving assertion T in closed form.
\end{itemize}

We can also easily convert any assertion T in closed form to its related
assertion Ti in inference form by applying
modus ponens\index{modus ponens} (see section \ref{axmp}).

The deduction form antecedent can also be used to represent the context
necessary to support natural deduction systems, so we will now
discuss natural deduction.

\subsection{Natural Deduction}\label{naturaldeduction}

Natural deduction\index{natural deduction}
(ND) systems, as such, were originally introduced in
1934 by two logicians working independently: Ja\'skowski and Gentzen. ND
systems are supposed to reconstruct, in a formally proper way, traditional
ways of mathematical reasoning (such as conditional proof, indirect proof,
and proof by cases). As reconstructions they were naturally influenced
by previous work, and many specific ND systems and notations have been
developed since their original work.

There are many ND variants, but
Indrzejczak \cite[p.~31-32]{Indrzejczak}\index{Indrzejczak, Andrzej}
suggests that any natural deductive system must satisfy at
least these three criteria:

\begin{itemize}
\item ``There are some means for entering assumptions into a proof and
also for eliminating them. Usually it requires some bookkeeping devices
for indicating the scope of an assumption, and showing that a part of
a proof depending on eliminated assumption is discharged.
\item There are no (or, at least, a very limited set of) axioms, because
their role is taken over by the set of primitive rules for introduction
and elimination of logical constants which means that elementary
inferences instead of formulae are taken as primitive.
\item (A genuine) ND system admits a lot of freedom in proof construction
and possibility of applying several proof search strategies, like
conditional proof, proof by cases, proof by reductio ad absurdum etc.''
\end{itemize}

The Metamath Proof Explorer (MPE) as defined in \texttt{set.mm}
is fundamentally a Hilbert-style system.
That is, MPE is based on a larger number of axioms (compared
to natural deduction systems), a very small set of rules of inference
(modus ponens), and the context is not changed by the rules of inference
in the middle of a proof. That said, MPE proofs can be developed using
the natural deduction (ND) approach as originally developed by Ja\'skowski
and Gentzen.

The most common and recommended approach for applying ND in MPE is to use
deduction form\index{deduction form}%
\index{forms!deduction}
and apply the MPE proven assertions that are equivalent to ND rules.
For example, MPE's \texttt{jca} is equivalent to ND rule $\land$-I
(and-insertion).
We maintain a list of equivalences that you may consult.
This approach for applying an ND approach within MPE relies on Metamath's
wff metavariables in an essential way, and is described in more detail
in the presentation ``Natural Deductions in the Metamath Proof Language''
by Mario Carneiro \cite{CarneiroND}\index{Carneiro, Mario}.

In this style many steps are an implication, whose antecedent mimics
the context ($\Gamma$) of most ND systems. To add an assumption, simply add
it to the implication antecedent (typically using
\texttt{simpr}),
and use that
new antecedent for all later claims in the same scope. If you wish to
use an assertion in an ND hypothesis scope that is outside the current
ND hypothesis scope, modify the assertion so that the ND hypothesis
assumption is added to its antecedent (typically using \texttt{adantr}). Most
proof steps will be proved using rules that have hypotheses and results
of the form $\varphi \rightarrow$ ...

An example may make this clearer.
Let's show theorem 5.5 of
\cite[p.~18]{Clemente}\index{Clemente Laboreo, Daniel}
along with a line by line translation using the usual
translation of natural deduction (ND) in the Metamath Proof Explorer
(MPE) notation (this is proof \texttt{ex-natded5.5}).
The proof's original goal was to prove
$\lnot \psi$ given two hypotheses,
$( \psi \rightarrow \chi )$ and $ \lnot \chi$.
We will translate these statements into MPE deduction form
by prefixing them all with $\varphi \rightarrow$.
As a result, in MPE the goal is stated as
$( \varphi \rightarrow \lnot \psi )$, and the two hypotheses are stated as
$( \varphi \rightarrow ( \psi \rightarrow \chi ) )$ and
$( \varphi \rightarrow \lnot \chi )$.

The following table shows the proof in Fitch natural deduction style
and its MPE equivalent.
The \textit{\#} column shows the original numbering,
\textit{MPE\#} shows the number in the equivalent MPE proof
(which we will show later),
\textit{ND Expression} shows the original proof claim in ND notation,
and \textit{MPE Translation} shows its translation into MPE
as discussed in this section.
The final columns show the rationale in ND and MPE respectively.

\needspace{4\baselineskip}
{\setlength{\extrarowsep}{4pt} % Keep rows from being too close together
\begin{longtabu}   { @{} c c X X X X }
\textbf{\#} & \textbf{MPE\#} & \textbf{ND Ex\-pres\-sion} &
\textbf{MPE Trans\-lation} & \textbf{ND Ration\-ale} &
\textbf{MPE Ra\-tio\-nale} \\
\endhead

1 & 2;3 &
$( \psi \rightarrow \chi )$ &
$( \varphi \rightarrow ( \psi \rightarrow \chi ) )$ &
Given &
\$e; \texttt{adantr} to put in ND hypothesis \\

2 & 5 &
$ \lnot \chi$ &
$( \varphi \rightarrow \lnot \chi )$ &
Given &
\$e; \texttt{adantr} to put in ND hypothesis \\

3 & 1 &
... $\vert$ $\psi$ &
$( \varphi \rightarrow \psi )$ &
ND hypothesis assumption &
\texttt{simpr} \\

4 & 4 &
... $\chi$ &
$( ( \varphi \land \psi ) \rightarrow \chi )$ &
$\rightarrow$\,E 1,3 &
\texttt{mpd} 1,3 \\

5 & 6 &
... $\lnot \chi$ &
$( ( \varphi \land \psi ) \rightarrow \lnot \chi )$ &
IT 2 &
\texttt{adantr} 5 \\

6 & 7 &
$\lnot \psi$ &
$( \varphi \rightarrow \lnot \psi )$ &
$\land$\,I 3,4,5 &
\texttt{pm2.65da} 4,6 \\

\end{longtabu}
}


The original used Latin letters; we have replaced them with Greek letters
to follow Metamath naming conventions and so that it is easier to follow
the Metamath translation. The Metamath line-for-line translation of
this natural deduction approach precedes every line with an antecedent
including $\varphi$ and uses the Metamath equivalents of the natural deduction
rules. To add an assumption, the antecedent is modified to include it
(typically by using \texttt{adantr};
\texttt{simpr} is useful when you want to
depend directly on the new assumption, as is shown here).

In Metamath we can represent the two given statements as these hypotheses:

\needspace{2\baselineskip}
\begin{itemize}
\item ex-natded5.5.1 $\vdash ( \varphi \rightarrow ( \psi \rightarrow \chi ) )$
\item ex-natded5.5.2 $\vdash ( \varphi \rightarrow \lnot \chi )$
\end{itemize}

\needspace{4\baselineskip}
Here is the proof in Metamath as a line-by-line translation:

\begin{longtabu}   { l l l X }
\textbf{Step} & \textbf{Hyp} & \textbf{Ref} & \textbf{Ex\-pres\-sion} \\
\endhead
1 & & simpr & $\vdash ( ( \varphi \land \psi ) \rightarrow \psi )$ \\
2 & & ex-natded5.5.1 &
  $\vdash ( \varphi \rightarrow ( \psi \rightarrow \chi ) )$ \\
3 & 2 & adantr &
 $\vdash ( ( \varphi \land \psi ) \rightarrow ( \psi \rightarrow \chi ) )$ \\
4 & 1, 3 & mpd &
 $\vdash ( ( \varphi \land \psi ) \rightarrow \chi ) $ \\
5 & & ex-natded5.5.2 &
 $\vdash ( \varphi \rightarrow \lnot \chi )$ \\
6 & 5 & adantr &
 $\vdash ( ( \varphi \land \psi ) \rightarrow \lnot \chi )$ \\
7 & 4, 6 & pm2.65da &
 $\vdash ( \varphi \rightarrow \lnot \psi )$ \\
\end{longtabu}

Only using specific natural deduction rules directly can lead to very
long proofs, for exactly the same reason that only using axioms directly
in Hilbert-style proofs can lead to very long proofs.
If the goal is short and clear proofs,
then it is better to reuse already-proven assertions
in deduction form than to start from scratch each time
and using only basic natural deduction rules.

\subsection{Strengths of Our Approach}

As far as we know there is nothing else in the literature like either the
weak deduction theorem or Mario Carneiro\index{Carneiro, Mario}'s
natural deduction method.
In order to
transform a hypothesis into an antecedent, the literature's standard
``Deduction Theorem''\index{Deduction Theorem}\index{Standard Deduction Theorem}
requires metalogic outside of the notions provided
by the axiom system. We instead generally prefer to use Mario Carneiro's
natural deduction method, then use the weak deduction theorem in cases
where that is difficult to apply, and only then use the full standard
deduction theorem as a last resort.

The weak deduction theorem\index{Weak Deduction Theorem}
does not require any additional metalogic
but converts an inference directly into a closed form theorem, with
a rigorous proof that uses only the axiom system. Unlike the standard
Deduction Theorem, there is no implicit external justification that we
have to trust in order to use it.

Mario Carneiro's natural deduction\index{natural deduction}
method also does not require any new metalogical
notions. It avoids the Deduction Theorem's metalogic by prefixing the
hypotheses and conclusion of every would-be inference with a universal
antecedent (``$\varphi \rightarrow$'') from the very start.

We think it is impressive and satisfying that we can do so much in a
practical sense without stepping outside of our Hilbert-style axiom system.
Of course our axiomatization, which is in the form of schemes,
contains a metalogic of its own that we exploit. But this metalogic
is relatively simple, and for our Deduction Theorem alternatives,
we primarily use just the direct substitution of expressions for
metavariables.

\begin{sloppy}
\section{Exploring the Set The\-o\-ry Data\-base}\label{exploring}
\end{sloppy}
% NOTE: All examples performed in this section are
% recorded wtih "set width 61" % on set.mm as of 2019-05-28
% commit c1e7849557661260f77cfdf0f97ac4354fbb4f4d.

At this point you may wish to study the \texttt{set.mm}\index{set theory
database (\texttt{set.mm})} file in more detail.  Pay particular
attention to the assumptions needed to define wffs\index{well-formed
formula (wff)} (which are not included above), the variable types
(\texttt{\$f}\index{\texttt{\$f} statement} statements), and the
definitions that are introduced.  Start with some simple theorems in
propositional calculus, making sure you understand in detail each step
of a proof.  Once you get past the first few proofs and become familiar
with the Metamath language, any part of the \texttt{set.mm} database
will be as easy to follow, step by step, as any other part---you won't
have to undergo a ``quantum leap'' in mathematical sophistication to be
able to follow a deep proof in set theory.

Next, you may want to explore how concepts such as natural numbers are
defined and described.  This is probably best done in conjunction with
standard set theory textbooks, which can help give you a higher-level
understanding.  The \texttt{set.mm} database provides references that will get
you started.  From there, you will be on your way towards a very deep,
rigorous understanding of abstract mathematics.

The Metamath\index{Metamath} program can help you peruse a Metamath data\-base,
wheth\-er you are trying to figure out how a certain step follows in a proof or
just have a general curiosity.  We will go through some examples of the
commands, using the \texttt{set.mm}\index{set theory database (\texttt{set.mm})}
database provided with the Metamath software.  These should help get you
started.  See Chapter~\ref{commands} for a more detailed description of
the commands.  Note that we have included the full spelling of all commands to
prevent ambiguity with future commands.  In practice you may type just the
characters needed to specify each command keyword\index{command keyword}
unambiguously, often just one or two characters per keyword, and you don't
need to type them in upper case.

First run the Metamath program as described earlier.  You should see the
\verb/MM>/ prompt.  Read in the \texttt{set.mm} file:\index{\texttt{read}
command}

\begin{verbatim}
MM> read set.mm
Reading source file "set.mm"... 34554442 bytes
34554442 bytes were read into the source buffer.
The source has 155711 statements; 2254 are $a and 32250 are $p.
No errors were found.  However, proofs were not checked.
Type VERIFY PROOF * if you want to check them.
\end{verbatim}

As with most examples in this book, what you will see
will be slightly different because we are continuously
improving our databases (including \texttt{set.mm}).

Let's check the database integrity.  This may take a minute or two to run if
your computer is slow.

\begin{verbatim}
MM> verify proof *
0 10%  20%  30%  40%  50%  60%  70%  80%  90% 100%
..................................................
All proofs in the database were verified in 2.84 s.
\end{verbatim}

No errors were reported, so every proof is correct.

You need to know the names (labels) of theorems before you can look at them.
Often just examining the database file(s) with a text editor is the best
approach.  In \texttt{set.mm} there are many detailed comments, especially near
the beginning, that can help guide you. The \texttt{search} command in the
Metamath program is also handy.  The \texttt{comments} qualifier will list the
statements whose associated comment (the one immediately before it) contain a
string you give it.  For example, if you are studying Enderton's {\em Elements
of Set Theory} \cite{Enderton}\index{Enderton, Herbert B.} you may want to see
the references to it in the database.  The search string \texttt{enderton} is not
case sensitive.  (This will not show you all the database theorems that are in
Enderton's book because there is usually only one citation for a given
theorem, which may appear in several textbooks.)\index{\texttt{search}
command}

\begin{verbatim}
MM> search * "enderton" / comments
12067 unineq $p "... Exercise 20 of [Enderton] p. 32 and ..."
12459 undif2 $p "...Corollary 6K of [Enderton] p. 144. (C..."
12953 df-tp $a "...s. Definition of [Enderton] p. 19. (Co..."
13689 unissb $p ".... Exercise 5 of [Enderton] p. 26 and ..."
\end{verbatim}
\begin{center}
(etc.)
\end{center}

Or you may want to see what theorems have something to do with
conjunction (logical {\sc and}).  The quotes around the search
string are optional when there's no ambiguity.\index{\texttt{search}
command}

\begin{verbatim}
MM> search * conjunction / comments
120 a1d $p "...be replaced with a conjunction ( ~ df-an )..."
662 df-bi $a "...viated form after conjunction is introdu..."
1319 wa $a "...ff definition to include conjunction ('and')."
1321 df-an $a "Define conjunction (logical 'and'). Defini..."
1420 imnan $p "...tion in terms of conjunction. (Contribu..."
\end{verbatim}
\begin{center}
(etc.)
\end{center}


Now we will start to look at some details.  Let's look at the first
axiom of propositional calculus
(we could use \texttt{sh st} to abbreviate
\texttt{show statement}).\index{\texttt{show statement} command}

\begin{verbatim}
MM> show statement ax-1/full
Statement 19 is located on line 881 of the file "set.mm".
"Axiom _Simp_.  Axiom A1 of [Margaris] p. 49.  One of the 3
axioms of propositional calculus.  The 3 axioms are also
        ...
19 ax-1 $a |- ( ph -> ( ps -> ph ) ) $.
Its mandatory hypotheses in RPN order are:
  wph $f wff ph $.
  wps $f wff ps $.
The statement and its hypotheses require the variables:  ph
      ps
The variables it contains are:  ph ps


Statement 49 is located on line 11182 of the file "set.mm".
Its statement number for HTML pages is 6.
"Axiom _Simp_.  Axiom A1 of [Margaris] p. 49.  One of the 3
axioms of propositional calculus.  The 3 axioms are also
given as Definition 2.1 of [Hamilton] p. 28.
...
49 ax-1 $a |- ( ph -> ( ps -> ph ) ) $.
Its mandatory hypotheses in RPN order are:
  wph $f wff ph $.
  wps $f wff ps $.
The statement and its hypotheses require the variables:
  ph ps
The variables it contains are:  ph ps
\end{verbatim}

Compare this to \texttt{ax-1} on p.~\pageref{ax1}.  You can see that the
symbol \texttt{ph} is the {\sc ascii} notation for $\varphi$, etc.  To
see the mathematical symbols for any expression you may typeset it in
\LaTeX\ (type \texttt{help tex} for instructions)\index{latex@{\LaTeX}}
or, easier, just use a text editor to look at the comments where symbols
are first introduced in \texttt{set.mm}.  The hypotheses \texttt{wph}
and \texttt{wps} required by \texttt{ax-1} mean that variables
\texttt{ph} and \texttt{ps} must be wffs.

Next we'll pick a simple theorem of propositional calculus, the Principle of
Identity, which is proved directly from the axioms.  We'll look at the
statement then its proof.\index{\texttt{show statement}
command}

\begin{verbatim}
MM> show statement id1/full
Statement 116 is located on line 11371 of the file "set.mm".
Its statement number for HTML pages is 22.
"Principle of identity.  Theorem *2.08 of [WhiteheadRussell]
p. 101.  This version is proved directly from the axioms for
demonstration purposes.
...
116 id1 $p |- ( ph -> ph ) $= ... $.
Its mandatory hypotheses in RPN order are:
  wph $f wff ph $.
Its optional hypotheses are:  wps wch wth wta wet
      wze wsi wrh wmu wla wka
The statement and its hypotheses require the variables:  ph
These additional variables are allowed in its proof:
      ps ch th ta et ze si rh mu la ka
The variables it contains are:  ph
\end{verbatim}

The optional variables\index{optional variable} \texttt{ps}, \texttt{ch}, etc.\ are
available for use in a proof of this statement if we wish, and were we to do
so we would make use of optional hypotheses \texttt{wps}, \texttt{wch}, etc.  (See
Section~\ref{dollaref} for the meaning of ``optional
hypothesis.''\index{optional hypothesis}) The reason these show up in the
statement display is that statement \texttt{id1} happens to be in their scope
(see Section~\ref{scoping} for the definition of ``scope''\index{scope}), but
in fact in propositional calculus we will never make use of optional
hypotheses or variables.  This becomes important after quantifiers are
introduced, where ``dummy'' variables\index{dummy variable} are often needed
in the middle of a proof.

Let's look at the proof of statement \texttt{id1}.  We'll use the
\texttt{show proof} command, which by default suppresses the
``non-essential'' steps that construct the wffs.\index{\texttt{show proof}
command}
We will display the proof in ``lemmon' format (a non-indented format
with explicit previous step number references) and renumber the
displayed steps:

\begin{verbatim}
MM> show proof id1 /lemmon/renumber
1 ax-1           $a |- ( ph -> ( ph -> ph ) )
2 ax-1           $a |- ( ph -> ( ( ph -> ph ) -> ph ) )
3 ax-2           $a |- ( ( ph -> ( ( ph -> ph ) -> ph ) ) ->
                     ( ( ph -> ( ph -> ph ) ) -> ( ph -> ph )
                                                          ) )
4 2,3 ax-mp      $a |- ( ( ph -> ( ph -> ph ) ) -> ( ph -> ph
                                                          ) )
5 1,4 ax-mp      $a |- ( ph -> ph )
\end{verbatim}

If you have read Section~\ref{trialrun}, you'll know how to interpret this
proof.  Step~2, for example, is an application of axiom \texttt{ax-1}.  This
proof is identical to the one in Hamilton's {\em Logic for Mathematicians}
\cite[p.~32]{Hamilton}\index{Hamilton, Alan G.}.

You may want to look at what
substitutions\index{substitution!variable}\index{variable substitution} are
made into \texttt{ax-1} to arrive at step~2.  The command to do this needs to
know the ``real'' step number, so we'll display the proof again without
the \texttt{renumber} qualifier.\index{\texttt{show proof}
command}

\begin{verbatim}
MM> show proof id1 /lemmon
 9 ax-1          $a |- ( ph -> ( ph -> ph ) )
20 ax-1          $a |- ( ph -> ( ( ph -> ph ) -> ph ) )
24 ax-2          $a |- ( ( ph -> ( ( ph -> ph ) -> ph ) ) ->
                     ( ( ph -> ( ph -> ph ) ) -> ( ph -> ph )
                                                          ) )
25 20,24 ax-mp   $a |- ( ( ph -> ( ph -> ph ) ) -> ( ph -> ph
                                                          ) )
26 9,25 ax-mp    $a |- ( ph -> ph )
\end{verbatim}

The ``real'' step number is 20.  Let's look at its details.

\begin{verbatim}
MM> show proof id1 /detailed_step 20
Proof step 20:  min=ax-1 $a |- ( ph -> ( ( ph -> ph ) -> ph )
  )
This step assigns source "ax-1" ($a) to target "min" ($e).
The source assertion requires the hypotheses "wph" ($f, step
18) and "wps" ($f, step 19).  The parent assertion of the
target hypothesis is "ax-mp" ($a, step 25).
The source assertion before substitution was:
    ax-1 $a |- ( ph -> ( ps -> ph ) )
The following substitutions were made to the source
assertion:
    Variable  Substituted with
     ph        ph
     ps        ( ph -> ph )
The target hypothesis before substitution was:
    min $e |- ph
The following substitution was made to the target hypothesis:
    Variable  Substituted with
     ph        ( ph -> ( ( ph -> ph ) -> ph ) )
\end{verbatim}

This shows the substitutions\index{substitution!variable}\index{variable
substitution} made to the variables in \texttt{ax-1}.  References are made to
steps 18 and 19 which are not shown in our proof display.  To see these steps,
you can display the proof with the \texttt{all} qualifier.

Let's look at a slightly more advanced proof of propositional calculus.  Note
that \verb+/\+ is the symbol for $\wedge$ (logical {\sc and}, also
called conjunction).\index{conjunction ($\wedge$)}
\index{logical {\sc and} ($\wedge$)}

\begin{verbatim}
MM> show statement prth/full
Statement 1791 is located on line 15503 of the file "set.mm".
Its statement number for HTML pages is 559.
"Conjoin antecedents and consequents of two premises.  This
is the closed theorem form of ~ anim12d .  Theorem *3.47 of
[WhiteheadRussell] p. 113.  It was proved by Leibniz,
and it evidently pleased him enough to call it
_praeclarum theorema_ (splendid theorem).
...
1791 prth $p |- ( ( ( ph -> ps ) /\ ( ch -> th ) ) -> ( ( ph
      /\ ch ) -> ( ps /\ th ) ) ) $= ... $.
Its mandatory hypotheses in RPN order are:
  wph $f wff ph $.
  wps $f wff ps $.
  wch $f wff ch $.
  wth $f wff th $.
Its optional hypotheses are:  wta wet wze wsi wrh wmu wla wka
The statement and its hypotheses require the variables:  ph
      ps ch th
These additional variables are allowed in its proof:  ta et
      ze si rh mu la ka
The variables it contains are:  ph ps ch th


MM> show proof prth /lemmon/renumber
1 simpl          $p |- ( ( ( ph -> ps ) /\ ( ch -> th ) ) ->
                                               ( ph -> ps ) )
2 simpr          $p |- ( ( ( ph -> ps ) /\ ( ch -> th ) ) ->
                                               ( ch -> th ) )
3 1,2 anim12d    $p |- ( ( ( ph -> ps ) /\ ( ch -> th ) ) ->
                           ( ( ph /\ ch ) -> ( ps /\ th ) ) )
\end{verbatim}

There are references to a lot of unfamiliar statements.  To see what they are,
you may type the following:

\begin{verbatim}
MM> show proof prth /statement_summary
Summary of statements used in the proof of "prth":

Statement simpl is located on line 14748 of the file
"set.mm".
"Elimination of a conjunct.  Theorem *3.26 (Simp) of
[WhiteheadRussell] p. 112. ..."
  simpl $p |- ( ( ph /\ ps ) -> ph ) $= ... $.

Statement simpr is located on line 14777 of the file
"set.mm".
"Elimination of a conjunct.  Theorem *3.27 (Simp) of
[WhiteheadRussell] ..."
  simpr $p |- ( ( ph /\ ps ) -> ps ) $= ... $.

Statement anim12d is located on line 15445 of the file
"set.mm".
"Conjoin antecedents and consequents in a deduction.
..."
  anim12d.1 $e |- ( ph -> ( ps -> ch ) ) $.
  anim12d.2 $e |- ( ph -> ( th -> ta ) ) $.
  anim12d $p |- ( ph -> ( ( ps /\ th ) -> ( ch /\ ta ) ) )
      $= ... $.
\end{verbatim}
\begin{center}
(etc.)
\end{center}

Of course you can look at each of these statements and their proofs, and
so on, back to the axioms of propositional calculus if you wish.

The \texttt{search} command is useful for finding statements when you
know all or part of their contents.  The following example finds all
statements containing \verb@ph -> ps@ followed by \verb@ch -> th@.  The
\verb@$*@ is a wildcard that matches anything; the \texttt{\$} before the
\verb$*$ prevents conflicts with math symbol token names.  The \verb@*@ after
\texttt{SEARCH} is also a wildcard that in this case means ``match any label.''
\index{\texttt{search} command}

% I'm omitting this one, since readers are unlikely to see it:
% 1096 bisymOLD $p |- ( ( ( ph -> ps ) -> ( ch -> th ) ) -> ( (
%   ( ps -> ph ) -> ( th -> ch ) ) -> ( ( ph <-> ps ) -> ( ch
%    <-> th ) ) ) )
\begin{verbatim}
MM> search * "ph -> ps $* ch -> th"
1791 prth $p |- ( ( ( ph -> ps ) /\ ( ch -> th ) ) -> ( ( ph
    /\ ch ) -> ( ps /\ th ) ) )
2455 pm3.48 $p |- ( ( ( ph -> ps ) /\ ( ch -> th ) ) -> ( (
    ph \/ ch ) -> ( ps \/ th ) ) )
117859 pm11.71 $p |- ( ( E. x ph /\ E. y ch ) -> ( ( A. x (
    ph -> ps ) /\ A. y ( ch -> th ) ) <-> A. x A. y ( ( ph /\
    ch ) -> ( ps /\ th ) ) ) )
\end{verbatim}

Three statements, \texttt{prth}, \texttt{pm3.48},
 and \texttt{pm11.71}, were found to match.

To see what axioms\index{axiom} and definitions\index{definition}
\texttt{prth} ultimately depends on for its proof, you can have the
program backtrack through the hierarchy\index{hierarchy} of theorems and
definitions.\index{\texttt{show trace{\char`\_}back} command}

\begin{verbatim}
MM> show trace_back prth /essential/axioms
Statement "prth" assumes the following axioms ($a
statements):
  ax-1 ax-2 ax-3 ax-mp df-bi df-an
\end{verbatim}

Note that the 3 axioms of propositional calculus and the modus ponens rule are
needed (as expected); in addition, there are a couple of definitions that are used
along the way.  Note that Metamath makes no distinction\index{axiom vs.\
definition} between axioms\index{axiom} and definitions\index{definition}.  In
\texttt{set.mm} they have been distinguished artificially by prefixing their
labels\index{labels in \texttt{set.mm}} with \texttt{ax-} and \texttt{df-}
respectively.  For example, \texttt{df-an} defines conjunction (logical {\sc
and}), which is represented by the symbol \verb+/\+.
Section~\ref{definitions} discusses the philosophy of definitions, and the
Metamath language takes a particularly simple, conservative approach by using
the \texttt{\$a}\index{\texttt{\$a} statement} statement for both axioms and
definitions.

You can also have the program compute how many steps a proof
has\index{proof length} if we were to follow it all the way back to
\texttt{\$a} statements.

\begin{verbatim}
MM> show trace_back prth /essential/count_steps
The statement's actual proof has 3 steps.  Backtracking, a
total of 79 different subtheorems are used.  The statement
and subtheorems have a total of 274 actual steps.  If
subtheorems used only once were eliminated, there would be a
total of 38 subtheorems, and the statement and subtheorems
would have a total of 185 steps.  The proof would have 28349
steps if fully expanded back to axiom references.  The
maximum path length is 38.  A longest path is:  prth <-
anim12d <- syl2and <- sylan2d <- ancomsd <- ancom <- pm3.22
<- pm3.21 <- pm3.2 <- ex <- sylbir <- biimpri <- bicomi <-
bicom1 <- bi2 <- dfbi1 <- impbii <- bi3 <- simprim <- impi <-
con1i <- nsyl2 <- mt3d <- con1d <- notnot1 <- con2i <- nsyl3
<- mt2d <- con2d <- notnot2 <- pm2.18d <- pm2.18 <- pm2.21 <-
pm2.21d <- a1d <- syl <- mpd <- a2i <- a2i.1 .
\end{verbatim}

This tells us that we would have to inspect 274 steps if we want to
verify the proof completely starting from the axioms.  A few more
statistics are also shown.  There are one or more paths back to axioms
that are the longest; this command ferrets out one of them and shows it
to you.  There may be a sense in which the longest path length is
related to how ``deep'' the theorem is.

We might also be curious about what proofs depend on the theorem
\texttt{prth}.  If it is never used later on, we could eliminate it as
redundant if it has no intrinsic interest by itself.\index{\texttt{show
usage} command}

% I decided to show the OLD values here.
\begin{verbatim}
MM> show usage prth
Statement "prth" is directly referenced in the proofs of 18
statements:
  mo3 moOLD 2mo 2moOLD euind reuind reuss2 reusv3i opelopabt
  wemaplem2 rexanre rlimcn2 o1of2 o1rlimmul 2sqlem6 spanuni
  heicant pm11.71
\end{verbatim}

Thus \texttt{prth} is directly used by 18 proofs.
We can use the \texttt{/recursive} qualifier to include indirect use:

\begin{verbatim}
MM> show usage prth /recursive
Statement "prth" directly or indirectly affects the proofs of
24214 statements:
  mo3 mo mo3OLD eu2 moOLD eu2OLD eu3OLD mo4f mo4 eu4 mopick
...
\end{verbatim}

\subsection{A Note on the ``Compact'' Proof Format}

The Metamath program will display proofs in a ``compact''\index{compact proof}
format whenever the proof is stored in compressed format in the database.  It
may be be slightly confusing unless you know how to interpret it.
For example,
if you display the complete proof of theorem \texttt{id1} it will start
off as follows:

\begin{verbatim}
MM> show proof id1 /lemmon/all
 1 wph           $f wff ph
 2 wph           $f wff ph
 3 wph           $f wff ph
 4 2,3 wi    @4: $a wff ( ph -> ph )
 5 1,4 wi    @5: $a wff ( ph -> ( ph -> ph ) )
 6 @4            $a wff ( ph -> ph )
\end{verbatim}

\begin{center}
{etc.}
\end{center}

Step 4 has a ``local label,''\index{local label} \texttt{@4}, assigned to it.
Later on, at step 6, the label \texttt{@1} is referenced instead of
displaying the explicit proof for that step.  This technique takes advantage
of the fact that steps in a proof often repeat, especially during the
construction of wffs.  The compact format reduces the number of steps in the
proof display and may be preferred by some people.

If you want to see the normal format with the ``true'' step numbers, you can
use the following workaround:\index{\texttt{save proof} command}

\begin{verbatim}
MM> save proof id1 /normal
The proof of "id1" has been reformatted and saved internally.
Remember to use WRITE SOURCE to save it permanently.
MM> show proof id1 /lemmon/all
 1 wph           $f wff ph
 2 wph           $f wff ph
 3 wph           $f wff ph
 4 2,3 wi        $a wff ( ph -> ph )
 5 1,4 wi        $a wff ( ph -> ( ph -> ph ) )
 6 wph           $f wff ph
 7 wph           $f wff ph
 8 6,7 wi        $a wff ( ph -> ph )
\end{verbatim}

\begin{center}
{etc.}
\end{center}

Note that the original 6 steps are now 8 steps.  However, the format is
now the same as that described in Chapter~\ref{using}.

\chapter{The Metamath Language}
\label{languagespec}

\begin{quote}
  {\em Thus mathematics may be defined as the subject in which we never know
what we are talking about, nor whether what we are saying is true.}
    \flushright\sc  Bertrand Russell\footnote{\cite[p.~84]{Russell2}.}\\
\end{quote}\index{Russell, Bertrand}

Probably the most striking feature of the Metamath language is its almost
complete absence of hard-wired syntax. Metamath\index{Metamath} does not
understand any mathematics or logic other than that needed to construct finite
sequences of symbols according to a small set of simple, built-in rules.  The
only rule it uses in a proof is the substitution of an expression (symbol
sequence) for a variable, subject to a simple constraint to prevent
bound-variable clashes.  The primitive notions built into Metamath involve the
simple manipulation of finite objects (symbols) that we as humans can easily
visualize and that computers can easily deal with.  They seem to be just
about the simplest notions possible that are required to do standard
mathematics.

This chapter serves as a reference manual for the Metamath\index{Metamath}
language. It covers the tedious technical details of the language, some of
which you may wish to skim in a first reading.  On the other hand, you should
pay close attention to the defined terms in {\bf boldface}; they have precise
meanings that are important to keep in mind for later understanding.  It may
be best to first become familiar with the examples in Chapter~\ref{using} to
gain some motivation for the language.

%% Uncomment this when uncommenting section {formalspec} below
If you have some knowledge of set theory, you may wish to study this
chapter in conjunction with the formal set-theoretical description of the
Metamath language in Appendix~\ref{formalspec}.

We will use the name ``Metamath''\index{Metamath} to mean either the Metamath
computer language or the Metamath software associated with the computer
language.  We will not distinguish these two when the context is clear.

The next section contains the complete specification of the Metamath
language.
It serves as an
authoritative reference and presents the syntax in enough detail to
write a parser\index{parsing Metamath} and proof verifier.  The
specification is terse and it is probably hard to learn the language
directly from it, but we include it here for those impatient people who
prefer to see everything up front before looking at verbose expository
material.  Later sections explain this material and provide examples.
We will repeat the definitions in those sections, and you may skip the
next section at first reading and proceed to Section~\ref{tut1}
(p.~\pageref{tut1}).

\section{Specification of the Metamath Language}\label{spec}
\index{Metamath!specification}

\begin{quote}
  {\em Sometimes one has to say difficult things, but one ought to say
them as simply as one knows how.}
    \flushright\sc  G. H. Hardy\footnote{As quoted in
    \cite{deMillo}, p.~273.}\\
\end{quote}\index{Hardy, G. H.}

\subsection{Preliminaries}\label{spec1}

% Space is technically a printable character, so we'll word things
% carefully so it's unambiguous.
A Metamath {\bf database}\index{database} is built up from a top-level source
file together with any source files that are brought in through file inclusion
commands (see below).  The only characters that are allowed to appear in a
Metamath source file are the 94 non-whitespace printable {\sc
ascii}\index{ascii@{\sc ascii}} characters, which are digits, upper and lower
case letters, and the following 32 special
characters\index{special characters}:\label{spec1chars}

\begin{verbatim}
! " # $ % & ' ( ) * + , - . / :
; < = > ? @ [ \ ] ^ _ ` { | } ~
\end{verbatim}
plus the following characters which are the ``white space'' characters:
space (a printable character),
tab, carriage return, line feed, and form feed.\label{whitespace}
We will use \texttt{typewriter}
font to display the printable characters.

A Metamath database consists of a sequence of three kinds of {\bf
tokens}\index{token} separated by {\bf white space}\index{white space}
(which is any sequence of one or more white space characters).  The set
of {\bf keyword}\index{keyword} tokens is \texttt{\$\char`\{},
\texttt{\$\char`\}}, \texttt{\$c}, \texttt{\$v}, \texttt{\$f},
\texttt{\$e}, \texttt{\$d}, \texttt{\$a}, \texttt{\$p}, \texttt{\$.},
\texttt{\$=}, \texttt{\$(}, \texttt{\$)}, \texttt{\$[}, and
\texttt{\$]}.  The last four are called {\bf auxiliary}\index{auxiliary
keyword} or preprocessing keywords.  A {\bf label}\index{label} token
consists of any combination of letters, digits, and the characters
hyphen, underscore, and period.  A {\bf math symbol}\index{math symbol}
token may consist of any combination of the 93 printable standard {\sc
ascii} characters other than space or \texttt{\$}~. All tokens are
case-sensitive.

\subsection{Preprocessing}

The token \texttt{\$(} begins a {\bf comment} and
\texttt{\$)} ends a comment.\index{\texttt{\$(}
and \texttt{\$)} auxiliary keywords}\index{comment}
Comments may contain any of
the 94 non-whitespace printable characters and white space,
except they may not contain the
2-character sequences \texttt{\$(} or \texttt{\$)} (comments do not nest).
Comments are ignored (treated
like white space) for the purpose of parsing, e.g.,
\texttt{\$( \$[ \$)} is a comment.
See p.~\pageref{mathcomments} for comment typesetting conventions; these
conventions may be ignored for the purpose of parsing.

A {\bf file inclusion command} consists of \texttt{\$[} followed by a file name
followed by \texttt{\$]}.\index{\texttt{\$[} and \texttt{\$]} auxiliary
keywords}\index{included file}\index{file inclusion}
It is only allowed in the outermost scope (i.e., not between
\texttt{\$\char`\{} and \texttt{\$\char`\}})
and must not be inside a statement (e.g., it may not occur
between the label of a \texttt{\$a} statement and its \texttt{\$.}).
The file name may not
contain a \texttt{\$} or white space.  The file must exist.
The case-sensitivity
of its name follows the conventions of the operating system.  The contents of
the file replace the inclusion command.
Included files may include other files.
Only the first reference to a given file is included; any later
references to the same file (whether in the top-level file or in included
files) cause the inclusion command to be ignored (treated like white space).
A verifier may assume that file names with different strings
refer to different files for the purpose of ignoring later references.
A file self-reference is ignored, as is any reference to the top-level file
(to avoid loops).
Included files may not include a \texttt{\$(} without a matching \texttt{\$)},
may not include a \texttt{\$[} without a matching \texttt{\$]}, and may
not include incomplete statements (e.g., a \texttt{\$a} without a matching
\texttt{\$.}).
It is currently unspecified if path references are relative to the process'
current directory or the file's containing directory, so databases should
avoid using pathname separators (e.g., ``/'') in file names.

Like all tokens, the \texttt{\$(}, \texttt{\$)}, \texttt{\$[}, and \texttt{\$]} keywords
must be surrounded by white space.

\subsection{Basic Syntax}

After preprocessing, a database will consist of a sequence of {\bf
statements}.
These are the scoping statements \texttt{\$\char`\{} and
\texttt{\$\char`\}}, along with the \texttt{\$c}, \texttt{\$v},
\texttt{\$f}, \texttt{\$e}, \texttt{\$d}, \texttt{\$a}, and \texttt{\$p}
statements.

A {\bf scoping statement}\index{scoping statement} consists only of its
keyword, \texttt{\$\char`\{} or \texttt{\$\char`\}}.
A \texttt{\$\char`\{} begins a {\bf
block}\index{block} and a matching \texttt{\$\char`\}} ends the block.
Every \texttt{\$\char`\{}
must have a matching \texttt{\$\char`\}}.
Defining it recursively, we say a block
contains a sequence of zero or more tokens other
than \texttt{\$\char`\{} and \texttt{\$\char`\}} and
possibly other blocks.  There is an {\bf outermost
block}\index{block!outermost} not bracketed by \texttt{\$\char`\{} \texttt{\$\char`\}}; the end
of the outermost block is the end of the database.

% LaTeX bug? can't do \bf\tt

A {\bf \$v} or {\bf \$c statement}\index{\texttt{\$v} statement}\index{\texttt{\$c}
statement} consists of the keyword token \texttt{\$v} or \texttt{\$c} respectively,
followed by one or more math symbols,
% The word "token" is used to distinguish "$." from the sentence-ending period.
followed by the \texttt{\$.}\ token.
These
statements {\bf declare}\index{declaration} the math symbols to be {\bf
variables}\index{variable!Metamath} or {\bf constants}\index{constant}
respectively. The same math symbol may not occur twice in a given \texttt{\$v} or
\texttt{\$c} statement.

%c%A math symbol becomes an {\bf active}\index{active math symbol}
%c%when declared and stays active until the end of the block in which it is
%c%declared.  A math symbol may not be declared a second time while it is active,
%c%but it may be declared again after it becomes inactive.

A math symbol becomes {\bf active}\index{active math symbol} when declared
and stays active until the end of the block in which it is declared.  A
variable may not be declared a second time while it is active, but it
may be declared again (as a variable, but not as a constant) after it
becomes inactive.  A constant must be declared in the outermost block and may
not be declared a second time.\index{redeclaration of symbols}

A {\bf \$f statement}\index{\texttt{\$f} statement} consists of a label,
followed by \texttt{\$f}, followed by its typecode (an active constant),
followed by an
active variable, followed by the \texttt{\$.}\ token.  A {\bf \$e
statement}\index{\texttt{\$e} statement} consists of a label, followed
by \texttt{\$e}, followed by its typecode (an active constant),
followed by zero or more
active math symbols, followed by the \texttt{\$.}\ token.  A {\bf
hypothesis}\index{hypothesis} is a \texttt{\$f} or \texttt{\$e}
statement.
The type declared by a \texttt{\$f} statement for a given label
is global even if the variable is not
(e.g., a database may not have \texttt{wff P} in one local scope
and \texttt{class P} in another).

A {\bf simple \$d statement}\index{\texttt{\$d} statement!simple}
consists of \texttt{\$d}, followed by two different active variables,
followed by the \texttt{\$.}\ token.  A {\bf compound \$d
statement}\index{\texttt{\$d} statement!compound} consists of
\texttt{\$d}, followed by three or more variables (all different),
followed by the \texttt{\$.}\ token.  The order of the variables in a
\texttt{\$d} statement is unimportant.  A compound \texttt{\$d}
statement is equivalent to a set of simple \texttt{\$d} statements, one
for each possible pair of variables occurring in the compound
\texttt{\$d} statement.  Henceforth in this specification we shall
assume all \texttt{\$d} statements are simple.  A \texttt{\$d} statement
is also called a {\bf disjoint} (or {\bf distinct}) {\bf variable
restriction}.\index{disjoint-variable restriction}

A {\bf \$a statement}\index{\texttt{\$a} statement} consists of a label,
followed by \texttt{\$a}, followed by its typecode (an active constant),
followed by
zero or more active math symbols, followed by the \texttt{\$.}\ token.  A {\bf
\$p statement}\index{\texttt{\$p} statement} consists of a label,
followed by \texttt{\$p}, followed by its typecode (an active constant),
followed by
zero or more active math symbols, followed by \texttt{\$=}, followed by
a sequence of labels, followed by the \texttt{\$.}\ token.  An {\bf
assertion}\index{assertion} is a \texttt{\$a} or \texttt{\$p} statement.

A \texttt{\$f}, \texttt{\$e}, or \texttt{\$d} statement is {\bf active}\index{active
statement} from the place it occurs until the end of the block it occurs in.
A \texttt{\$a} or \texttt{\$p} statement is {\bf active} from the place it occurs
through the end of the database.
There may not be two active \texttt{\$f} statements containing the same
variable.  Each variable in a \texttt{\$e}, \texttt{\$a}, or
\texttt{\$p} statement must exist in an active \texttt{\$f}
statement.\footnote{This requirement can greatly simplify the
unification algorithm (substitution calculation) required by proof
verification.}

%The label that begins each \texttt{\$f}, \texttt{\$e}, \texttt{\$a}, and
%\texttt{\$p} statement must be unique.
Each label token must be unique, and
no label token may match any math symbol
token.\label{namespace}\footnote{This
restriction was added on June 24, 2006.
It is not theoretically necessary but is imposed to make it easier to
write certain parsers.}

The set of {\bf mandatory variables}\index{mandatory variable} associated with
an assertion is the set of (zero or more) variables in the assertion and in any
active \texttt{\$e} statements.  The (possibly empty) set of {\bf mandatory
hypotheses}\index{mandatory hypothesis} is the set of all active \texttt{\$f}
statements containing mandatory variables, together with all active \texttt{\$e}
statements.
The set of {\bf mandatory {\bf \$d} statements}\index{mandatory
disjoint-variable restriction} associated with an assertion are those active
\texttt{\$d} statements whose variables are both among the assertion's
mandatory variables.

\subsection{Proof Verification}\label{spec4}

The sequence of labels between the \texttt{\$=} and \texttt{\$.}\ tokens
in a \texttt{\$p} statement is a {\bf proof}.\index{proof!Metamath} Each
label in a proof must be the label of an active statement other than the
\texttt{\$p} statement itself; thus a label must refer either to an
active hypothesis of the \texttt{\$p} statement or to an earlier
assertion.

An {\bf expression}\index{expression} is a sequence of math symbols. A {\bf
substitution map}\index{substitution map} associates a set of variables with a
set of expressions.  It is acceptable for a variable to be mapped to an
expression containing it.  A {\bf
substitution}\index{substitution!variable}\index{variable substitution} is the
simultaneous replacement of all variables in one or more expressions with the
expressions that the variables map to.

A proof is scanned in order of its label sequence.  If the label refers to an
active hypothesis, the expression in the hypothesis is pushed onto a
stack.\index{stack}\index{RPN stack}  If the label refers to an assertion, a
(unique) substitution must exist that, when made to the mandatory hypotheses
of the referenced assertion, causes them to match the topmost (i.e.\ most
recent) entries of the stack, in order of occurrence of the mandatory
hypotheses, with the topmost stack entry matching the last mandatory
hypothesis of the referenced assertion.  As many stack entries as there are
mandatory hypotheses are then popped from the stack.  The same substitution is
made to the referenced assertion, and the result is pushed onto the stack.
After the last label in the proof is processed, the stack must have a single
entry that matches the expression in the \texttt{\$p} statement containing the
proof.

%c%{\footnotesize\begin{quotation}\index{redeclaration of symbols}
%c%{{\em Comment.}\label{spec4comment} Whenever a math symbol token occurs in a
%c%{\texttt{\$c} or \texttt{\$v} statement, it is considered to designate a distinct new
%c%{symbol, even if the same token was previously declared (and is now inactive).
%c%{Thus a math token declared as a constant in two different blocks is considered
%c%{to designate two distinct constants (even though they have the same name).
%c%{The two constants will not match in a proof that references both blocks.
%c%{However, a proof referencing both blocks is acceptable as long as it doesn't
%c%{require that the constants match.  Similarly, a token declared to be a
%c%{constant for a referenced assertion will not match the same token declared to
%c%{be a variable for the \texttt{\$p} statement containing the proof.  In the case
%c%{of a token declared to be a variable for a referenced assertion, this is not
%c%{an issue since the variable can be substituted with whatever expression is
%c%{needed to achieve the required match.
%c%{\end{quotation}}
%c2%A proof may reference an assertion that contains or whose hypotheses contain a
%c2%constant that is not active for the \texttt{\$p} statement containing the proof.
%c2%However, the final result of the proof may not contain that constant. A proof
%c2%may also reference an assertion that contains or whose hypotheses contain a
%c2%variable that is not active for the \texttt{\$p} statement containing the proof.
%c2%That variable, of course, will be substituted with whatever expression is
%c2%needed to achieve the required match.

A proof may contain a \texttt{?}\ in place of a label to indicate an unknown step
(Section~\ref{unknown}).  A proof verifier may ignore any proof containing
\texttt{?}\ but should warn the user that the proof is incomplete.

A {\bf compressed proof}\index{compressed proof}\index{proof!compressed} is an
alternate proof notation described in Appen\-dix~\ref{compressed}; also see
references to ``compressed proof'' in the Index.  Compressed proofs are a
Metamath language extension which a complete proof verifier should be able to
parse and verify.

\subsubsection{Verifying Disjoint Variable Restrictions}

Each substitution made in a proof must be checked to verify that any
disjoint variable restrictions are satisfied, as follows.

If two variables replaced by a substitution exist in a mandatory \texttt{\$d}
statement\index{\texttt{\$d} statement} of the assertion referenced, the two
expressions resulting from the substitution must satisfy the following
conditions.  First, the two expressions must have no variables in common.
Second, each possible pair of variables, one from each expression, must exist
in an active \texttt{\$d} statement of the \texttt{\$p} statement containing the
proof.

\vskip 1ex

This ends the specification of the Metamath language;
see Appendix \ref{BNF} for its syntax in
Extended Backus--Naur Form (EBNF)\index{Extended Backus--Naur Form}\index{EBNF}.

\section{The Basic Keywords}\label{tut1}

Our expository material begins here.

Like most computer languages, Metamath\index{Metamath} takes its input from
one or more {\bf source files}\index{source file} which contain characters
expressed in the standard {\sc ascii} (American Standard Code for Information
Interchange)\index{ascii@{\sc ascii}} code for computers.  A source file
consists of a series of {\bf tokens}\index{token}, which are strings of
non-whitespace
printable characters (from the set of 94 shown on p.~\pageref{spec1chars})
separated by {\bf white space}\index{white space} (spaces, tabs, carriage
returns, line feeds, and form feeds). Any string consisting only of these
characters is treated the same as a single space.  The non-whitespace printable
characters\index{printable character} that Metamath recognizes are the 94
characters on standard {\sc ascii} keyboards.

Metamath has the ability to join several files together to form its
input (Section~\ref{include}).  We call the aggregate contents of all
the files after they have been joined together a {\bf
database}\index{database} to distinguish it from an individual source
file.  The tokens in a database consist of {\bf
keywords}\index{keyword}, which are built into the language, together
with two kinds of user-defined tokens called {\bf labels}\index{label}
and {\bf math symbols}\index{math symbol}.  (Often we will simply say
{\bf symbol}\index{symbol} instead of math symbol for brevity).  The set
of {\bf basic keywords}\index{basic keyword} is
\texttt{\$c}\index{\texttt{\$c} statement},
\texttt{\$v}\index{\texttt{\$v} statement},
\texttt{\$e}\index{\texttt{\$e} statement},
\texttt{\$f}\index{\texttt{\$f} statement},
\texttt{\$d}\index{\texttt{\$d} statement},
\texttt{\$a}\index{\texttt{\$a} statement},
\texttt{\$p}\index{\texttt{\$p} statement},
\texttt{\$=}\index{\texttt{\$=} keyword},
\texttt{\$.}\index{\texttt{\$.}\ keyword},
\texttt{\$\char`\{}\index{\texttt{\$\char`\{} and \texttt{\$\char`\}}
keywords}, and \texttt{\$\char`\}}.  This is the complete set of
syntactical elements of what we call the {\bf basic
language}\index{basic language} of Metamath, and with them you can
express all of the mathematics that were intended by the design of
Metamath.  You should make it a point to become very familiar with them.
Table~\ref{basickeywords} lists the basic keywords along with a brief
description of their functions.  For now, this description will give you
only a vague notion of what the keywords are for; later we will describe
the keywords in detail.


\begin{table}[htp] \caption{Summary of the basic Metamath
keywords} \label{basickeywords}
\begin{center}
\begin{tabular}{|p{4pc}|l|}
\hline
\em \centering Keyword&\em Description\\
\hline
\hline
\centering
   \texttt{\$c}&Constant symbol declaration\\
\hline
\centering
   \texttt{\$v}&Variable symbol declaration\\
\hline
\centering
   \texttt{\$d}&Disjoint variable restriction\\
\hline
\centering
   \texttt{\$f}&Variable-type (``floating'') hypothesis\\
\hline
\centering
   \texttt{\$e}&Logical (``essential'') hypothesis\\
\hline
\centering
   \texttt{\$a}&Axiomatic assertion\\
\hline
\centering
   \texttt{\$p}&Provable assertion\\
\hline
\centering
   \texttt{\$=}&Start of proof in \texttt{\$p} statement\\
\hline
\centering
   \texttt{\$.}&End of the above statement types\\
\hline
\centering
   \texttt{\$\char`\{}&Start of block\\
\hline
\centering
   \texttt{\$\char`\}}&End of block\\
\hline
\end{tabular}
\end{center}
\end{table}

%For LaTeX bug(?) where it puts tables on blank page instead of btwn text
%May have to adjust if text changes
%\newpage

There are some additional keywords, called {\bf auxiliary
keywords}\index{auxiliary keyword} that help make Metamath\index{Metamath}
more practical. These are part of the {\bf extended language}\index{extended
language}. They provide you with a means to put comments into a Metamath
source file\index{source file} and reference other source files.  We will
introduce these in later sections. Table~\ref{otherkeywords} summarizes them
so that you can recognize them now if you want to peruse some source
files while learning the basic keywords.


\begin{table}[htp] \caption{Auxiliary Metamath
keywords} \label{otherkeywords}
\begin{center}
\begin{tabular}{|p{4pc}|l|}
\hline
\em \centering Keyword&\em Description\\
\hline
\hline
\centering
   \texttt{\$(}&Start of comment\\
\hline
\centering
   \texttt{\$)}&End of comment\\
\hline
\centering
   \texttt{\$[}&Start of included source file name\\
\hline
\centering
   \texttt{\$]}&End of included source file name\\
\hline
\end{tabular}
\end{center}
\end{table}
\index{\texttt{\$(} and \texttt{\$)} auxiliary keywords}
\index{\texttt{\$[} and \texttt{\$]} auxiliary keywords}


Unlike those in some computer languages, the keywords\index{keyword} are short
two-character sequences rather than English-like words.  While this may make
them slightly more difficult to remember at first, their brevity allows
them to blend in with the mathematics being described, not
distract from it, like punctuation marks.


\subsection{User-Defined Tokens}\label{dollardollar}\index{token}

As you may have noticed, all keywords\index{keyword} begin with the \texttt{\$}
character.  This mundane monetary symbol is not ordinarily used in higher
mathematics (outside of grant proposals), so we have appropriated it to
distinguish the Metamath\index{Metamath} keywords from ordinary mathematical
symbols. The \texttt{\$} character is thus considered special and may not be
used as a character in a user-defined token.  All tokens and keywords are
case-sensitive; for example, \texttt{n} is considered to be a different character
from \texttt{N}.  Case-sensitivity makes the available {\sc ascii} character set
as rich as possible.

\subsubsection{Math Symbol Tokens}\index{token}

Math symbols\index{math symbol} are tokens used to represent the symbols
that appear in ordinary mathematical formulas.  They may consist of any
combination of the 93 non-whitespace printable {\sc ascii} characters other than
\texttt{\$}~. Some examples are \texttt{x}, \texttt{+}, \texttt{(},
\texttt{|-}, \verb$!%@?&$, and \texttt{bounded}.  For readability, it is
best to try to make these look as similar to actual mathematical symbols
as possible, within the constraints of the {\sc ascii} character set, in
order to make the resulting mathematical expressions more readable.

In the Metamath\index{Metamath} language, you express ordinary
mathematical formulas and statements as sequences of math symbols such
as \texttt{2 + 2 = 4} (five symbols, all constants).\footnote{To
eliminate ambiguity with other expressions, this is expressed in the set
theory database \texttt{set.mm} as \texttt{|- ( 2 + 2
 ) = 4 }, whose \LaTeX\ equivalent is $\vdash
(2+2)=4$.  The \,$\vdash$ means ``is a theorem'' and the
parentheses allow explicit associative grouping.}\index{turnstile
({$\,\vdash$})} They may even be English
sentences, as in \texttt{E is closed and bounded} (five symbols)---here
\texttt{E} would be a variable and the other four symbols constants.  In
principle, a Metamath database could be constructed to work with almost
any unambiguous English-language mathematical statement, but as a
practical matter the definitions needed to provide for all possible
syntax variations would be cumbersome and distracting and possibly have
subtle pitfalls accidentally built in.  We generally recommend that you
express mathematical statements with compact standard mathematical
symbols whenever possible and put their English-language descriptions in
comments.  Axioms\index{axiom} and definitions\index{definition}
(\texttt{\$a}\index{\texttt{\$a} statement} statements) are the only
places where Metamath will not detect an error, and doing this will help
reduce the number of definitions needed.

You are free to use any tokens\index{token} you like for math
symbols\index{math symbol}.  Appendix~\ref{ASCII} recommends token names to
use for symbols in set theory, and we suggest you adopt these in order to be
able to include the \texttt{set.mm} set theory database in your database.  For
printouts, you can convert the tokens in a database
to standard mathematical symbols with the \LaTeX\ typesetting program.  The
Metamath command \texttt{open tex} {\em filename}\index{\texttt{open tex} command}
produces output that can be read by \LaTeX.\index{latex@{\LaTeX}}
The correspondence
between tokens and the actual symbols is made by \texttt{latexdef}
statements inside a special database comment tagged
with \texttt{\$t}.\index{\texttt{\$t} comment}\index{typesetting comment}
  You can edit
this comment to change the definitions or add new ones.
Appendix~\ref{ASCII} describes how to do this in more detail.

% White space\index{white space} is normally used to separate math
% symbol\index{math symbol} tokens, but they may be juxtaposed without white
% space in \texttt{\$d}\index{\texttt{\$d} statement}, \texttt{\$e}\index{\texttt{\$e}
% statement}, \texttt{\$f}\index{\texttt{\$f} statement}, \texttt{\$a}\index{\texttt{\$a}
% statement}, and \texttt{\$p}\index{\texttt{\$p} statement} statements when no
% ambiguity will result.  Specifically, Metamath parses the math symbol sequence
% in one of these statements in the following manner:  when the math symbol
% sequence has been broken up into tokens\index{token} up to a given character,
% the next token is the longest string of characters that could constitute a
% math symbol that is active\index{active
% math symbol} at that point.  (See Section~\ref{scoping} for the
% definition of an active math symbol.)  For example, if \texttt{-}, \texttt{>}, and
% \texttt{->} are the only active math symbols, the juxtaposition \texttt{>-} will be
% interpreted as the two symbols \texttt{>} and \texttt{-}, whereas \texttt{->} will
% always be interpreted as that single symbol.\footnote{For better readability we
% recommend a white space between each token.  This also makes searching for a
% symbol easier to do with an editor.  Omission of optional white space is useful
% for reducing typing when assigning an expression to a temporary
% variable\index{temporary variable} with the \texttt{let variable} Metamath
% program command.}\index{\texttt{let variable} command}
%
% Keywords\index{keyword} may be placed next to math symbols without white
% space\index{white space} between them.\footnote{Again, we do not recommend
% this for readability.}
%
% The math symbols\index{math symbol} in \texttt{\$c}\index{\texttt{\$c} statement}
% and \texttt{\$v}\index{\texttt{\$v} statement} statements must always be separated
% by white space\index{white
% space}, for the obvious reason that these statements define the names
% of the symbols.
%
% Math symbols referred to in comments (see Section~\ref{comments}) must also be
% separated by white space.  This allows you to make comments about symbols that
% are not yet active\index{active
% math symbol}.  (The ``math mode'' feature of comments is also a quick and
% easy way to obtain word processing text with embedded mathematical symbols,
% independently of the main purpose of Metamath; the way to do this is described
% in Section~\ref{comments})

\subsubsection{Label Tokens}\index{token}\index{label}

Label tokens are used to identify Metamath\index{Metamath} statements for
later reference. Label tokens may contain only letters, digits, and the three
characters period, hyphen, and underscore:
\begin{verbatim}
. - _
\end{verbatim}

A label is {\bf declared}\index{label declaration} by placing it immediately
before the keyword of the statement it identifies.  For example, the label
\texttt{axiom.1} might be declared as follows:
\begin{verbatim}
axiom.1 $a |- x = x $.
\end{verbatim}

Each \texttt{\$e}\index{\texttt{\$e} statement},
\texttt{\$f}\index{\texttt{\$f} statement},
\texttt{\$a}\index{\texttt{\$a} statement}, and
\texttt{\$p}\index{\texttt{\$p} statement} statement in a database must
have a label declared for it.  No other statement types may have label
declarations.  Every label must be unique.

A label (and the statement it identifies) is {\bf referenced}\index{label
reference} by including the label between the \texttt{\$=}\index{\texttt{\$=}
keyword} and \texttt{\$.}\index{\texttt{\$.}\ keyword}\ keywords in a \texttt{\$p}
statement.  The sequence of labels\index{label sequence} between these two
keywords is called a {\bf proof}\index{proof}.  An example of a statement with
a proof that we will encounter later (Section~\ref{proof}) is
\begin{verbatim}
wnew $p wff ( s -> ( r -> p ) )
     $= ws wr wp w2 w2 $.
\end{verbatim}

You don't have to know what this means just yet, but you should know that the
label \texttt{wnew} is declared by this \texttt{\$p} statement and that the labels
\texttt{ws}, \texttt{wr}, \texttt{wp}, and \texttt{w2} are assumed to have been declared
earlier in the database and are referenced here.

\subsection{Constants and Variables}
\index{constant}
\index{variable}

An {\bf expression}\index{expression} is any sequence of math
symbols, possibly empty.

The basic Metamath\index{Metamath} language\index{basic language} has two
kinds of math symbols\index{math symbol}:  {\bf constants}\index{constant} and
{\bf variables}\index{variable}.  In a Metamath proof, a constant may not be
substituted with any expression.  A variable can be
substituted\index{substitution!variable}\index{variable substitution} with any
expression.  This sequence may include other variables and may even include
the variable being substituted.  This substitution takes place when proofs are
verified, and it will be described in Section~\ref{proof}.  The \texttt{\$f}
statement (described later in Section~\ref{dollaref}) is used to specify the
{\bf type} of a variable (i.e.\ what kind of
variable it is)\index{variable type}\index{type} and
give it a meaning typically
associated with a ``metavariable''\index{metavariable}\footnote{A metavariable
is a variable that ranges over the syntactical elements of the object language
being discussed; for example, one metavariable might represent a variable of
the object language and another metavariable might represent a formula in the
object language.} in ordinary mathematics; for example, a variable may be
specified to be a wff or well-formed formula (in logic), a set (in set
theory), or a non-negative integer (in number theory).

%\subsection{The \texttt{\$c} and \texttt{\$v} Declaration Statements}
\subsection{The \texttt{\$c} and \texttt{\$v} Declaration Statements}
\index{\texttt{\$c} statement}
\index{constant declaration}
\index{\texttt{\$v} statement}
\index{variable declaration}

Constants are introduced or {\bf declared}\index{constant declaration}
with \texttt{\$c}\index{\texttt{\$c} statement} statements, and
variables are declared\index{variable declaration} with
\texttt{\$v}\index{\texttt{\$v} statement} statements.  A {\bf simple}
declaration\index{simple declaration} statement introduces a single
constant or variable.  Its syntax is one of the following:
\begin{center}
  \texttt{\$c} {\em math-symbol} \texttt{\$.}\\
  \texttt{\$v} {\em math-symbol} \texttt{\$.}
\end{center}
The notation {\em math-symbol} means any math symbol token\index{token}.

Some examples of simple declaration statements are:
\begin{center}
  \texttt{\$c + \$.}\\
  \texttt{\$c -> \$.}\\
  \texttt{\$c ( \$.}\\
  \texttt{\$v x \$.}\\
  \texttt{\$v y2 \$.}
\end{center}

The characters in a math symbol\index{math symbol} being declared are
irrelevant to Meta\-math; for example, we could declare a right parenthesis to
be a variable,
\begin{center}
  \texttt{\$v ) \$.}\\
\end{center}
although this would be unconventional.

A {\bf compound} declaration\index{compound declaration} statement is a
shorthand for declaring several symbols at once.  Its syntax is one of the
following:
\begin{center}
  \texttt{\$c} {\em math-symbol}\ \,$\cdots$\ {\em math-symbol} \texttt{\$.}\\
  \texttt{\$v} {\em math-symbol}\ \,$\cdots$\ {\em math-symbol} \texttt{\$.}
\end{center}\index{\texttt{\$c} statement}
Here, the ellipsis (\ldots) means any number of {\em math-symbol}\,s.

An example of a compound declaration statement is:
\begin{center}
  \texttt{\$v x y mu \$.}\\
\end{center}
This is equivalent to the three simple declaration statements
\begin{center}
  \texttt{\$v x \$.}\\
  \texttt{\$v y \$.}\\
  \texttt{\$v mu \$.}\\
\end{center}
\index{\texttt{\$v} statement}

There are certain rules on where in the database math symbols may be declared,
what sections of the database are aware of them (i.e.\ where they are
``active''), and when they may be declared more than once.  These will be
discussed in Section~\ref{scoping} and specifically on
p.~\pageref{redeclaration}.

\subsection{The \texttt{\$d} Statement}\label{dollard}
\index{\texttt{\$d} statement}

The \texttt{\$d} statement is called a {\bf disjoint-variable restriction}.  The
syntax of the {\bf simple} version of this statement is
\begin{center}
  \texttt{\$d} {\em variable variable} \texttt{\$.}
\end{center}
where each {\em variable} is a previously declared variable and the two {\em
variable}\,s are different.  (More specifically, each  {\em variable} must be
an {\bf active} variable\index{active math symbol}, which means there must be
a previous \texttt{\$v} statement whose {\bf scope}\index{scope} includes the
\texttt{\$d} statement.  These terms will be defined when we discuss scoping
statements in Section~\ref{scoping}.)

In ordinary mathematics, formulas may arise that are true if the variables in
them are distinct\index{distinct variables}, but become false when those
variables are made identical. For example, the formula in logic $\exists x\,x
\neq y$, which means ``for a given $y$, there exists an $x$ that is not equal
to $y$,'' is true in most mathematical theories (namely all non-trivial
theories\index{non-trivial theory}, i.e.\ those that describe more than one
individual, such as arithmetic).  However, if we substitute $y$ with $x$, we
obtain $\exists x\,x \neq x$, which is always false, as it means ``there
exists something that is not equal to itself.''\footnote{If you are a
logician, you will recognize this as the improper substitution\index{proper
substitution}\index{substitution!proper} of a free variable\index{free
variable} with a bound variable\index{bound variable}.  Metamath makes no
inherent distinction between free and bound variables; instead, you let
Metamath know what substitutions are permissible by using \texttt{\$d} statements
in the right way in your axiom system.}\index{free vs.\ bound variable}  The
\texttt{\$d} statement allows you to specify a restriction that forbids the
substitution of one variable with another.  In
this case, we would use the statement
\begin{center}
  \texttt{\$d x y \$.}
\end{center}\index{\texttt{\$d} statement}
to specify this restriction.

The order in which the variables appear in a \texttt{\$d} statement is not
important.  We could also use
\begin{center}
  \texttt{\$d y x \$.}
\end{center}

The \texttt{\$d} statement is actually more general than this, as the
``disjoint''\index{disjoint variables} in its name suggests.  The full meaning
is that if any substitution is made to its two variables (during the
course of a proof that references a \texttt{\$a} or \texttt{\$p} statement
associated with the \texttt{\$d}), the two expressions that result from the
substitution must have no variables in common.  In addition, each possible
pair of variables, one from each expression, must be in a \texttt{\$d} statement
associated with the statement being proved.  (This requirement forces the
statement being proved to ``inherit'' the original disjoint variable
restriction.)

For example, suppose \texttt{u} is a variable.  If the restriction
\begin{center}
  \texttt{\$d A B \$.}
\end{center}
has been specified for a theorem referenced in a
proof, we may not substitute \texttt{A} with \mbox{\tt a + u} and
\texttt{B} with \mbox{\tt b + u} because these two symbol sequences have the
variable \texttt{u} in common.  Furthermore, if \texttt{a} and \texttt{b} are
variables, we may not substitute \texttt{A} with \texttt{a} and \texttt{B} with \texttt{b}
unless we have also specified \texttt{\$d a b} for the theorem being proved; in
other words, the \texttt{\$d} property associated with a pair of variables must
be effectively preserved after substitution.

The \texttt{\$d}\index{\texttt{\$d} statement} statement does {\em not} mean ``the
two variables may not be substituted with the same thing,'' as you might think
at first.  For example, substituting each of \texttt{A} and \texttt{B} in the above
example with identical symbol sequences consisting only of constants does not
cause a disjoint variable conflict, because two symbol sequences have no
variables in common (since they have no variables, period).  Similarly, a
conflict will not occur by substituting the two variables in a \texttt{\$d}
statement with the empty symbol sequence\index{empty substitution}.

The \texttt{\$d} statement does not have a direct counterpart in
ordinary mathematics, partly because the variables\index{variable} of
Metamath are not really the same as the variables\index{variable!in
ordinary mathematics} of ordinary mathematics but rather are
metavariables\index{metavariable} ranging over them (as well as over
other kinds of symbols and groups of symbols).  Depending on the
situation, we may informally interpret the \texttt{\$d} statement in
different ways.  Suppose, for example, that \texttt{x} and \texttt{y}
are variables ranging over numbers (more precisely, that \texttt{x} and
\texttt{y} are metavariables ranging over variables that range over
numbers), and that \texttt{ph} ($\varphi$) and \texttt{ps} ($\psi$) are
variables (more precisely, metavariables) ranging over formulas.  We can
make the following interpretations that correspond to the informal
language of ordinary mathematics:
\begin{quote}
\begin{tabbing}
\texttt{\$d x y \$.} means ``assume $x$ and $y$ are
distinct variables.''\\
\texttt{\$d x ph \$.} means ``assume $x$ does not
occur in $\varphi$.''\\
\texttt{\$d ph ps \$.} \=means ``assume $\varphi$ and
$\psi$ have no variables\\ \>in common.''
\end{tabbing}
\end{quote}\index{\texttt{\$d} statement}

\subsubsection{Compound \texttt{\$d} Statements}

The {\bf compound} version of the \texttt{\$d} statement is a shorthand for
specifying several variables whose substitutions must be pairwise disjoint.
Its syntax is:
\begin{center}
  \texttt{\$d} {\em variable}\ \,$\cdots$\ {\em variable} \texttt{\$.}
\end{center}\index{\texttt{\$d} statement}
Here, {\em variable} represents the token of a previously declared
variable (specifically, an active variable) and all {\em variable}\,s are
different.  The compound \texttt{\$d}
statement is internally broken up by Metamath into one simple \texttt{\$d}
statement for each possible pair of variables in the original \texttt{\$d}
statement.  For example,
\begin{center}
  \texttt{\$d w x y z \$.}
\end{center}
is equivalent to
\begin{center}
  \texttt{\$d w x \$.}\\
  \texttt{\$d w y \$.}\\
  \texttt{\$d w z \$.}\\
  \texttt{\$d x y \$.}\\
  \texttt{\$d x z \$.}\\
  \texttt{\$d y z \$.}
\end{center}

Two or more simple \texttt{\$d} statements specifying the same variable pair are
internally combined into a single \texttt{\$d} statement.  Thus the set of three
statements
\begin{center}
  \texttt{\$d x y \$.}
  \texttt{\$d x y \$.}
  \texttt{\$d y x \$.}
\end{center}
is equivalent to
\begin{center}
  \texttt{\$d x y \$.}
\end{center}

Similarly, compound \texttt{\$d} statements, after being internally broken up,
internally have their common variable pairs combined.  For example the
set of statements
\begin{center}
  \texttt{\$d x y A \$.}
  \texttt{\$d x y B \$.}
\end{center}
is equivalent to
\begin{center}
  \texttt{\$d x y \$.}
  \texttt{\$d x A \$.}
  \texttt{\$d y A \$.}
  \texttt{\$d x y \$.}
  \texttt{\$d x B \$.}
  \texttt{\$d y B \$.}
\end{center}
which is equivalent to
\begin{center}
  \texttt{\$d x y \$.}
  \texttt{\$d x A \$.}
  \texttt{\$d y A \$.}
  \texttt{\$d x B \$.}
  \texttt{\$d y B \$.}
\end{center}

Metamath\index{Metamath} automatically verifies that all \texttt{\$d}
restrictions are met whenever it verifies proofs.  \texttt{\$d} statements are
never referenced directly in proofs (this is why they do not have
labels\index{label}), but Metamath is always aware of which ones must be
satisfied (i.e.\ are active) and will notify you with an error message if any
violation occurs.

To illustrate how Metamath detects a missing \texttt{\$d}
statement, we will look at the following example from the
\texttt{set.mm} database.

\begin{verbatim}
$d x z $.  $d y z $.
$( Theorem to add distinct quantifier to atomic formula. $)
ax17eq $p |- ( x = y -> A. z x = y ) $=...
\end{verbatim}

This statement has the obvious requirement that $z$ must be
distinct\index{distinct variables} from $x$ in theorem \texttt{ax17eq} that
states $x=y \rightarrow \forall z \, x=y$ (well, obvious if you're a logician,
for otherwise we could conclude  $x=y \rightarrow \forall x \, x=y$, which is
false when the free variables $x$ and $y$ are equal).

Let's look at what happens if we edit the database to comment out this
requirement.

\begin{verbatim}
$( $d x z $. $) $d y z $.
$( Theorem to add distinct quantifier to atomic formula. $)
ax17eq $p |- ( x = y -> A. z x = y ) $=...
\end{verbatim}

When it tries to verify the proof, Metamath will tell you that \texttt{x} and
\texttt{z} must be disjoint, because one of its steps references an axiom or
theorem that has this requirement.

\begin{verbatim}
MM> verify proof ax17eq
ax17eq ?Error at statement 1918, label "ax17eq", type "$p":
      vz wal wi vx vy vz ax-13 vx vy weq vz vx ax-c16 vx vy
                                               ^^^^^
There is a disjoint variable ($d) violation at proof step 29.
Assertion "ax-c16" requires that variables "x" and "y" be
disjoint.  But "x" was substituted with "z" and "y" was
substituted with "x".  The assertion being proved, "ax17eq",
does not require that variables "z" and "x" be disjoint.
\end{verbatim}

We can see the substitutions into \texttt{ax-c16} with the following command.

\begin{verbatim}
MM> show proof ax17eq / detailed_step 29
Proof step 29:  pm2.61dd.2=ax-c16 $a |- ( A. z z = x -> ( x =
  y -> A. z x = y ) )
This step assigns source "ax-c16" ($a) to target "pm2.61dd.2"
($e).  The source assertion requires the hypotheses "wph"
($f, step 26), "vx" ($f, step 27), and "vy" ($f, step 28).
The parent assertion of the target hypothesis is "pm2.61dd"
($p, step 36).
The source assertion before substitution was:
    ax-c16 $a |- ( A. x x = y -> ( ph -> A. x ph ) )
The following substitutions were made to the source
assertion:
    Variable  Substituted with
     x         z
     y         x
     ph        x = y
The target hypothesis before substitution was:
    pm2.61dd.2 $e |- ( ph -> ch )
The following substitutions were made to the target
hypothesis:
    Variable  Substituted with
     ph        A. z z = x
     ch        ( x = y -> A. z x = y )
\end{verbatim}

The disjoint variable restrictions of \texttt{ax-c16} can be seen from the
\texttt{show state\-ment} command.  The line that begins ``\texttt{Its mandatory
dis\-joint var\-i\-able pairs are:}\ldots'' lists any \texttt{\$d} variable
pairs in brackets.

\begin{verbatim}
MM> show statement ax-c16/full
Statement 3033 is located on line 9338 of the file "set.mm".
"Axiom of Distinct Variables. ..."
  ax-c16 $a |- ( A. x x = y -> ( ph -> A. x ph ) ) $.
Its mandatory hypotheses in RPN order are:
  wph $f wff ph $.
  vx $f setvar x $.
  vy $f setvar y $.
Its mandatory disjoint variable pairs are:  <x,y>
The statement and its hypotheses require the variables:  x y
      ph
The variables it contains are:  x y ph
\end{verbatim}

Since Metamath will always detect when \texttt{\$d}\index{\texttt{\$d} statement}
statements are needed for a proof, you don't have to worry too much about
forgetting to put one in; it can always be added if you see the error message
above.  If you put in unnecessary \texttt{\$d} statements, the worst that could
happen is that your theorem might not be as general as it could be, and this
may limit its use later on.

On the other hand, when you introduce axioms (\texttt{\$a}\index{\texttt{\$a}
statement} statements), you must be very careful to properly specify the
necessary associated \texttt{\$d} statements since Metamath has no way of knowing
whether your axioms are correct.  For example, Metamath would have no idea
that \texttt{ax-c16}, which we are telling it is an axiom of logic, would lead to
contradictions if we omitted its associated \texttt{\$d} statement.

% This was previously a comment in footnote-sized type, but it can be
% hard to read this much text in a small size.
% As a result, it's been changed to normally-sized text.
\label{nodd}
You may wonder if it is possible to develop standard
mathematics in the Metamath language without the \texttt{\$d}\index{\texttt{\$d}
statement} statement, since it seems like a nuisance that complicates proof
verification. The \texttt{\$d} statement is not needed in certain subsets of
mathematics such as propositional calculus.  However, dummy
variables\index{dummy variable!eliminating} and their associated \texttt{\$d}
statements are impossible to avoid in proofs in standard first-order logic as
well as in the variant used in \texttt{set.mm}.  In fact, there is no upper bound to
the number of dummy variables that might be needed in a proof of a theorem of
first-order logic containing 3 or more variables, as shown by H.\
Andr\'{e}ka\index{Andr{\'{e}}ka, H.} \cite{Nemeti}.  A first-order system that
avoids them entirely is given in \cite{Megill}\index{Megill, Norman}; the
trick there is simply to embed harmlessly the necessary dummy variables into a
theorem being proved so that they aren't ``dummy'' anymore, then interpret the
resulting longer theorem so as to ignore the embedded dummy variables.  If
this interests you, the system in \texttt{set.mm} obtained from \texttt{ax-1}
through \texttt{ax-c14} in \texttt{set.mm}, and deleting \texttt{ax-c16} and \texttt{ax-5},
requires no \texttt{\$d} statements but is logically complete in the sense
described in \cite{Megill}.  This means it can prove any theorem of
first-order logic as long as we add to the theorem an antecedent that embeds
dummy and any other variables that must be distinct.  In a similar fashion,
axioms for set theory can be devised that
do not require distinct variable
provisos\index{Set theory without distinct variable provisos},
as explained at
\url{http://us.metamath.org/mpeuni/mmzfcnd.html}.
Together, these in principle allow all of
mathematics to be developed under Metamath without a \texttt{\$d} statement,
although the length of the resulting theorems will grow as more and
more dummy variables become required in their proofs.

\subsection{The \texttt{\$f}
and \texttt{\$e} Statements}\label{dollaref}
\index{\texttt{\$e} statement}
\index{\texttt{\$f} statement}
\index{floating hypothesis}
\index{essential hypothesis}
\index{variable-type hypothesis}
\index{logical hypothesis}
\index{hypothesis}

Metamath has two kinds of hypo\-theses, the \texttt{\$f}\index{\texttt{\$f}
statement} or {\bf variable-type} hypothesis and the \texttt{\$e} or {\bf logical}
hypo\-the\-sis.\index{\texttt{\$d} statement}\footnote{Strictly speaking, the
\texttt{\$d} statement is also a hypothesis, but it is never directly referenced
in a proof, so we call it a restriction rather than a hypothesis to lessen
confusion.  The checking for violations of \texttt{\$d} restrictions is automatic
and built into Metamath's proof-checking algorithm.} The letters \texttt{f} and
\texttt{e} stand for ``floating''\index{floating hypothesis} (roughly meaning
used only if relevant) and ``essential''\index{essential hypothesis} (meaning
always used) respectively, for reasons that will become apparent
when we discuss frames in
Section~\ref{frames} and scoping in Section~\ref{scoping}. The syntax of these
are as follows:
\begin{center}
  {\em label} \texttt{\$f} {\em typecode} {\em variable} \texttt{\$.}\\
  {\em label} \texttt{\$e} {\em typecode}
      {\em math-symbol}\ \,$\cdots$\ {\em math-symbol} \texttt{\$.}\\
\end{center}
\index{\texttt{\$e} statement}
\index{\texttt{\$f} statement}
A hypothesis must have a {\em label}\index{label}.  The expression in a
\texttt{\$e} hypothesis consists of a typecode (an active constant math symbol)
followed by a sequence
of zero or more math symbols. Each math symbol (including {\em constant}
and {\em variable}) must be a previously declared constant or variable.  (In
addition, each math symbol must be active, which will be covered when we
discuss scoping statements in Section~\ref{scoping}.)  You use a \texttt{\$f}
hypothesis to specify the
nature or {\bf type}\index{variable type}\index{type} of a variable (such as ``let $x$ be an
integer'') and use a \texttt{\$e} hypothesis to express a logical truth (such as
``assume $x$ is prime'') that must be established in order for an assertion
requiring it to also be true.

A variable must have its type specified in a \texttt{\$f} statement before
it may be used in a \texttt{\$e}, \texttt{\$a}, or \texttt{\$p}
statement.  There may be only one (active) \texttt{\$f} statement for a
given variable.  (``Active'' is defined in Section~\ref{scoping}.)

In ordinary mathematics, theorems\index{theorem} are often expressed in the
form ``Assume $P$; then $Q$,'' where $Q$ is a statement that you can derive
if you start with statement $P$.\index{free variable}\footnote{A stronger
version of a theorem like this would be the {\em single} formula $P\rightarrow
Q$ ($P$ implies $Q$) from which the weaker version above follows by the rule
of modus ponens in logic.  We are not discussing this stronger form here.  In
the weaker form, we are saying only that if we can {\em prove} $P$, then we can
{\em prove} $Q$.  In a logician's language, if $x$ is the only free variable
in $P$ and $Q$, the stronger form is equivalent to $\forall x ( P \rightarrow
Q)$ (for all $x$, $P$ implies $Q$), whereas the weaker form is equivalent to
$\forall x P \rightarrow \forall x Q$. The stronger form implies the weaker,
but not vice-versa.  To be precise, the weaker form of the theorem is more
properly called an ``inference'' rather than a theorem.}\index{inference}
In the
Metamath\index{Metamath} language, you would express mathematical statement
$P$ as a hypothesis (a \texttt{\$e} Metamath language statement in this case) and
statement $Q$ as a provable assertion (a \texttt{\$p}\index{\texttt{\$p} statement}
statement).

Some examples of hypotheses you might encounter in logic and set theory are
\begin{center}
  \texttt{stmt1 \$f wff P \$.}\\
  \texttt{stmt2 \$f setvar x \$.}\\
  \texttt{stmt3 \$e |- ( P -> Q ) \$.}
\end{center}
\index{\texttt{\$e} statement}
\index{\texttt{\$f} statement}
Informally, these would be read, ``Let $P$ be a well-formed-formula,'' ``Let
$x$ be an (individual) variable,'' and ``Assume we have proved $P \rightarrow
Q$.''  The turnstile symbol \,$\vdash$\index{turnstile ({$\,\vdash$})} is
commonly used in logic texts to mean ``a proof exists for.''

To summarize:
\begin{itemize}
\item A \texttt{\$f} hypothesis tells Metamath the type or kind of its variable.
It is analogous to a variable declaration in a computer language that
tells the compiler that a variable is an integer or a floating-point
number.
\item The \texttt{\$e} hypothesis corresponds to what you would usually call a
``hypothesis'' in ordinary mathematics.
\end{itemize}

Before an assertion\index{assertion} (\texttt{\$a} or \texttt{\$p} statement) can be
referenced in a proof, all of its associated \texttt{\$f} and \texttt{\$e} hypotheses
(i.e.\ those \texttt{\$e} hypotheses that are active) must be satisfied (i.e.
established by the proof).  The meaning of ``associated'' (which we will call
{\bf mandatory} in Section~\ref{frames}) will become clear when we discuss
scoping later.

Note that after any \texttt{\$f}, \texttt{\$e},
\texttt{\$a}, or \texttt{\$p} token there is a required
\textit{typecode}\index{typecode}.
The typecode is a constant used to enforce types of expressions.
This will become clearer once we learn more about
assertions (\texttt{\$a} and \texttt{\$p} statements).
An example may also clarify their purpose.
In the
\texttt{set.mm}\index{set theory database (\texttt{set.mm})}%
\index{Metamath Proof Explorer}
database,
the following typecodes are used:

\begin{itemize}
\item \texttt{wff} :
  Well-formed formula (wff) symbol
  (read: ``the following symbol sequence is a wff'').
% The *textual* typecode for turnstile is "|-", but when read it's a little
% confusing, so I intentionally display the mathematical symbol here instead
% (I think it's clearer in this context).
\item \texttt{$\vdash$} :
  Turnstile (read: ``the following symbol sequence is provable'' or
  ``a proof exists for'').
\item \texttt{setvar} :
  Individual set variable type (read: ``the following is an
  individual set variable'').
  Note that this is \textit{not} the type of an arbitrary set expression,
  instead, it is used to ensure that there is only a single symbol used
  after quantifiers like for-all ($\forall$) and there-exists ($\exists$).
\item \texttt{class} :
  An expression that is a syntactically valid class expression.
  All valid set expressions are also valid class expression, so expressions
  of sets normally have the \texttt{class} typecode.
  Use the \texttt{class} typecode,
  \textit{not} the \texttt{setvar} typecode,
  for the type of set expressions unless you are specifically identifying
  a single set variable.
\end{itemize}

\subsection{Assertions (\texttt{\$a} and \texttt{\$p} Statements)}
\index{\texttt{\$a} statement}
\index{\texttt{\$p} statement}\index{assertion}\index{axiomatic assertion}
\index{provable assertion}

There are two types of assertions, \texttt{\$a}\index{\texttt{\$a} statement}
statements ({\bf axiomatic assertions}) and \texttt{\$p} statements ({\bf
provable assertions}).  Their syntax is as follows:
\begin{center}
  {\em label} \texttt{\$a} {\em typecode} {\em math-symbol} \ldots
         {\em math-symbol} \texttt{\$.}\\
  {\em label} \texttt{\$p} {\em typecode} {\em math-symbol} \ldots
        {\em math-symbol} \texttt{\$=} {\em proof} \texttt{\$.}
\end{center}
\index{\texttt{\$a} statement}
\index{\texttt{\$p} statement}
\index{\texttt{\$=} keyword}
An assertion always requires a {\em label}\index{label}. The expression in an
assertion consists of a typecode (an active constant)
followed by a sequence of zero
or more math symbols.  Each math symbol, including any {\em constant}, must be a
previously declared constant or variable.  (In addition, each math symbol
must be active, which will be covered when we discuss scoping statements in
Section~\ref{scoping}.)

A \texttt{\$a} statement is usually a definition of syntax (for example, if $P$
and $Q$ are wffs then so is $(P\to Q)$), an axiom\index{axiom} of ordinary
mathematics (for example, $x=x$), or a definition\index{definition} of
ordinary mathematics (for example, $x\ne y$ means $\lnot x=y$). A \texttt{\$p}
statement is a claim that a certain combination of math symbols follows from
previous assertions and is accompanied by a proof that demonstrates it.

Assertions can also be referenced in (later) proofs in order to derive new
assertions from them. The label of an assertion is used to refer to it in a
proof. Section~\ref{proof} will describe the proof in detail.

Assertions also provide the primary means for communicating the mathematical
results in the database to people.  Proofs (when conveniently displayed)
communicate to people how the results were arrived at.

\subsubsection{The \texttt{\$a} Statement}
\index{\texttt{\$a} statement}

Axiomatic assertions (\texttt{\$a} statements) represent the starting points from
which other assertions (\texttt{\$p}\index{\texttt{\$p} statement} statements) are
derived.  Their most obvious use is for specifying ordinary mathematical
axioms\index{axiom}, but they are also used for two other purposes.

First, Metamath\index{Metamath} needs to know the syntax of symbol
sequences that constitute valid mathematical statements.  A Metamath
proof must be broken down into much more detail than ordinary
mathematical proofs that you may be used to thinking of (even the
``complete'' proofs of formal logic\index{formal logic}).  This is one
of the things that makes Metamath a general-purpose language,
independent of any system of logic or even syntax.  If you want to use a
substitution instance of an assertion as a step in a proof, you must
first prove that the substitution is syntactically correct (or if you
prefer, you must ``construct'' it), showing for example that the
expression you are substituting for a wff metavariable is a valid wff.
The \texttt{\$a}\index{\texttt{\$a} statement} statement is used to
specify those combinations of symbols that are considered syntactically
valid, such as the legal forms of wffs.

Second, \texttt{\$a} statements are used to specify what are ordinarily thought of
as definitions, i.e.\ new combinations of symbols that abbreviate other
combinations of symbols.  Metamath makes no distinction\index{axiom vs.\
definition} between axioms\index{axiom} and definitions\index{definition}.
Indeed, it has been argued that such distinction should not be made even in
ordinary mathematics; see Section~\ref{definitions}, which discusses the
philosophy of definitions.  Section~\ref{hierarchy} discusses some
technical requirements for definitions.  In \texttt{set.mm} we adopt the
convention of prefixing axiom labels with \texttt{ax-} and definition labels with
\texttt{df-}\index{label}.

The results that can be derived with the Metamath language are only as good as
the \texttt{\$a}\index{\texttt{\$a} statement} statements used as their starting
point.  We cannot stress this too strongly.  For example, Metamath will
not prevent you from specifying $x\neq x$ as an axiom of logic.  It is
essential that you scrutinize all \texttt{\$a} statements with great care.
Because they are a source of potential pitfalls, it is best not to add new
ones (usually new definitions) casually; rather you should carefully evaluate
each one's necessity and advantages.

Once you have in place all of the basic axioms\index{axiom} and
rules\index{rule} of a mathematical theory, the only \texttt{\$a} statements that
you will be adding will be what are ordinarily called definitions.  In
principle, definitions should be in some sense eliminable from the language of
a theory according to some convention (usually involving logical equivalence
or equality).  The most common convention is that any formula that was
syntactically valid but not provable before the definition was introduced will
not become provable after the definition is introduced.  In an ideal world,
definitions should not be present at all if one is to have absolute confidence
in a mathematical result.  However, they are necessary to make
mathematics practical, for otherwise the resulting formulas would be
extremely long and incomprehensible.  Since the nature of definitions (in the
most general sense) does not permit them to automatically be verified as
``proper,''\index{proper definition}\index{definition!proper} the judgment of
the mathematician is required to ensure it.  (In \texttt{set.mm} effort was made
to make almost all definitions directly eliminable and thus minimize the need
for such judgment.)

If you are not a mathematician, it may be best not to add or change any
\texttt{\$a}\index{\texttt{\$a} statement} statements but instead use
the mathematical language already provided in standard databases.  This
way Metamath will not allow you to make a mistake (i.e.\ prove a false
result).


\subsection{Frames}\label{frames}

We now introduce the concept of a collection of related Metamath statements
called a frame.  Every assertion (\texttt{\$a} or \texttt{\$p} statement) in the database has
an associated frame.

A {\bf frame}\index{frame} is a sequence of \texttt{\$d}, \texttt{\$f},
and \texttt{\$e} statements (zero or more of each) followed by one
\texttt{\$a} or \texttt{\$p} statement, subject to certain conditions we
will describe.  For simplicity we will assume that all math symbol
tokens used are declared at the beginning of the database with
\texttt{\$c} and \texttt{\$v} statements (which are not properly part of
a frame).  Also for simplicity we will assume there are only simple
\texttt{\$d} statements (those with only two variables) and imagine any
compound \texttt{\$d} statements (those with more than two variables) as
broken up into simple ones.

A frame groups together those hypotheses (and \texttt{\$d} statements) relevant
to an assertion (\texttt{\$a} or \texttt{\$p} statement).  The statements in a frame
may or may not be physically adjacent in a database; we will cover
this in our discussion of scoping statements
in Section~\ref{scoping}.

A frame has the following properties:
\begin{enumerate}
 \item The set of variables contained in its \texttt{\$f} statements must
be identical to the set of variables contained in its \texttt{\$e},
\texttt{\$a}, and/or \texttt{\$p} statements.  In other words, each
variable in a \texttt{\$e}, \texttt{\$a}, or \texttt{\$p} statement must
have an associated ``variable type'' defined for it in a \texttt{\$f}
statement.
  \item No two \texttt{\$f} statements may contain the same variable.
  \item Any \texttt{\$f} statement
must occur before a \texttt{\$e} statement in which its variable occurs.
\end{enumerate}

The first property determines the set of variables occurring in a frame.
These are the {\bf mandatory
variables}\index{mandatory variable} of the frame.  The second property
tells us there must be only one type specified for a variable.
The last property is not a theoretical requirement but it
makes parsing of the database easier.

For our examples, we assume our database has the following declarations:

\begin{verbatim}
$v P Q R $.
$c -> ( ) |- wff $.
\end{verbatim}

The following sequence of statements, describing the modus ponens inference
rule, is an example of a frame:

\begin{verbatim}
wp  $f wff P $.
wq  $f wff Q $.
maj $e |- ( P -> Q ) $.
min $e |- P $.
mp  $a |- Q $.
\end{verbatim}

The following sequence of statements is not a frame because \texttt{R} does not
occur in the \texttt{\$e}'s or the \texttt{\$a}:

\begin{verbatim}
wp  $f wff P $.
wq  $f wff Q $.
wr  $f wff R $.
maj $e |- ( P -> Q ) $.
min $e |- P $.
mp  $a |- Q $.
\end{verbatim}

The following sequence of statements is not a frame because \texttt{Q} does not
occur in a \texttt{\$f}:

\begin{verbatim}
wp  $f wff P $.
maj $e |- ( P -> Q ) $.
min $e |- P $.
mp  $a |- Q $.
\end{verbatim}

The following sequence of statements is not a frame because the \texttt{\$a} statement is
not the last one:

\begin{verbatim}
wp  $f wff P $.
wq  $f wff Q $.
maj $e |- ( P -> Q ) $.
mp  $a |- Q $.
min $e |- P $.
\end{verbatim}

Associated with a frame is a sequence of {\bf mandatory
hypotheses}\index{mandatory hypothesis}.  This is simply the set of all
\texttt{\$f} and \texttt{\$e} statements in the frame, in the order they
appear.  A frame can be referenced in a later proof using the label of
the \texttt{\$a} or \texttt{\$p} assertion statement, and the proof
makes an assignment to each mandatory hypothesis in the order in which
it appears.  This means the order of the hypotheses, once chosen, must
not be changed so as not to affect later proofs referencing the frame's
assertion statement.  (The Metamath proof verifier will, of course, flag
an error if a proof becomes incorrect by doing this.)  Since proofs make
use of ``Reverse Polish notation,'' described in Section~\ref{proof}, we
call this order the {\bf RPN order}\index{RPN order} of the hypotheses.

Note that \texttt{\$d} statements are not part of the set of mandatory
hypotheses, and their order doesn't matter (as long as they satisfy the
fourth property for a frame described above).  The \texttt{\$d}
statements specify restrictions on variables that must be satisfied (and
are checked by the proof verifier) when expressions are substituted for
them in a proof, and the \texttt{\$d} statements themselves are never
referenced directly in a proof.

A frame with a \texttt{\$p} (provable) statement requires a proof as part of the
\texttt{\$p} statement.  Sometimes in a proof we want to make use of temporary or
dummy variables\index{dummy variable} that do not occur in the \texttt{\$p}
statement or its mandatory hypotheses.  To accommodate this we define an {\bf
extended frame}\index{extended frame} as a frame together with zero or more
\texttt{\$d} and \texttt{\$f} statements that reference variables not among the
mandatory variables of the frame.  Any new variables referenced are called the
{\bf optional variables}\index{optional variable} of the extended frame. If a
\texttt{\$f} statement references an optional variable it is called an {\bf
optional hypothesis}\index{optional hypothesis}, and if one or both of the
variables in a \texttt{\$d} statement are optional variables it is called an {\bf
optional disjoint-variable restriction}\index{optional disjoint-variable
restriction}.  Properties 2 and 3 for a frame also apply to an extended
frame.

The concept of optional variables is not meaningful for frames with \texttt{\$a}
statements, since those statements have no proofs that might make use of them.
There is no restriction on including optional hypotheses in the extended frame
for a \texttt{\$a} statement, but they serve no purpose.

The following set of statements is an example of an extended frame, which
contains an optional variable \texttt{R} and an optional hypothesis \texttt{wr}.  In
this example, we suppose the rule of modus ponens is not an axiom but is
derived as a theorem from earlier statements (we omit its presumed proof).
Variable \texttt{R} may be used in its proof if desired (although this would
probably have no advantage in propositional calculus).  Note that the sequence
of mandatory hypotheses in RPN order is still \texttt{wp}, \texttt{wq}, \texttt{maj},
\texttt{min} (i.e.\ \texttt{wr} is omitted), and this sequence is still assumed
whenever the assertion \texttt{mp} is referenced in a subsequent proof.

\begin{verbatim}
wp  $f wff P $.
wq  $f wff Q $.
wr  $f wff R $.
maj $e |- ( P -> Q ) $.
min $e |- P $.
mp  $p |- Q $= ... $.
\end{verbatim}

Every frame is an extended frame, but not every extended frame is a frame, as
this example shows.  The underlying frame for an extended frame is
obtained by simply removing all statements containing optional variables.
Any proof referencing an assertion will ignore any extensions to its
frame, which means we may add or delete optional hypotheses at will without
affecting subsequent proofs.

The conceptually simplest way of organizing a Metamath database is as a
sequence of extended frames.  The scoping statements
\texttt{\$\char`\{}\index{\texttt{\$\char`\{} and \texttt{\$\char`\}}
keywords} and \texttt{\$\char`\}} can be used to delimit the start and
end of an extended frame, leading to the following possible structure for a
database.  \label{framelist}

\vskip 2ex
\setbox\startprefix=\hbox{\tt \ \ \ \ \ \ \ \ }
\setbox\contprefix=\hbox{}
\startm
\m{\mbox{(\texttt{\$v} {\em and} \texttt{\$c}\,{\em statements})}}
\endm
\startm
\m{\mbox{\texttt{\$\char`\{}}}
\endm
\startm
\m{\mbox{\texttt{\ \ } {\em extended frame}}}
\endm
\startm
\m{\mbox{\texttt{\$\char`\}}}}
\endm
\startm
\m{\mbox{\texttt{\$\char`\{}}}
\endm
\startm
\m{\mbox{\texttt{\ \ } {\em extended frame}}}
\endm
\startm
\m{\mbox{\texttt{\$\char`\}}}}
\endm
\startm
\m{\mbox{\texttt{\ \ \ \ \ \ \ \ \ }}\vdots}
\endm
\vskip 2ex

In practice, this structure is inconvenient because we have to repeat
any \texttt{\$f}, \texttt{\$e}, and \texttt{\$d} statements over and
over again rather than stating them once for use by several assertions.
The scoping statements, which we will discuss next, allow this to be
done.  In principle, any Metamath database can be converted to the above
format, and the above format is the most convenient to use when studying
a Metamath database as a formal system%
%% Uncomment this when uncommenting section {formalspec} below
   (Appendix \ref{formalspec})%
.
In fact, Metamath internally converts the database to the above format.
The command \texttt{show statement} in the Metamath program will show
you the contents of the frame for any \texttt{\$a} or \texttt{\$p}
statement, as well as its extension in the case of a \texttt{\$p}
statement.

%c%(provided that all ``local'' variables and constants with limited scope have
%c%unique names),

During our discussion of scoping statements, it may be helpful to
think in terms of the equivalent sequence of frames that will result when
the database is parsed.  Scoping (other than the limited
use above to delimit frames) is not a theoretical requirement for
Metamath but makes it more convenient.


\subsection{Scoping Statements (\texttt{\$\{} and \texttt{\$\}})}\label{scoping}
\index{\texttt{\$\char`\{} and \texttt{\$\char`\}} keywords}\index{scoping statement}

%c%Some Metamath statements may be needed only temporarily to
%c%serve a specific purpose, and after we're done with them we would like to
%c%disregard or ignore them.  For example, when we're finished using a variable,
%c%we might want to
%c%we might want to free up the token\index{token} used to name it so that the
%c%token can be used for other purposes later on, such as a different kind of
%c%variable or even a constant.  In the terminology of computer programming, we
%c%might want to let some symbol declarations be ``local'' rather than ``global.''
%c%\index{local symbol}\index{global symbol}

The {\bf scoping} statements, \texttt{\$\char`\{} ({\bf start of block}) and \texttt{\$\char`\}}
({\bf end of block})\index{block}, provide a means for controlling the portion
of a database over which certain statement types are recognized.  The
syntax of a scoping statement is very simple; it just consists of the
statement's keyword:
\begin{center}
\texttt{\$\char`\{}\\
\texttt{\$\char`\}}
\end{center}
\index{\texttt{\$\char`\{} and \texttt{\$\char`\}} keywords}

For example, consider the following database where we have stripped out
all tokens except the scoping statement keywords.  For the purpose of the
discussion, we have added subscripts to the scoping statements; these subscripts
do not appear in the actual database.
\[
 \mbox{\tt \ \$\char`\{}_1
 \mbox{\tt \ \$\char`\{}_2
 \mbox{\tt \ \$\char`\}}_2
 \mbox{\tt \ \$\char`\{}_3
 \mbox{\tt \ \$\char`\{}_4
 \mbox{\tt \ \$\char`\}}_4
 \mbox{\tt \ \$\char`\}}_3
 \mbox{\tt \ \$\char`\}}_1
\]
Each \texttt{\$\char`\{} statement in this example is said to be {\bf
matched} with the \texttt{\$\char`\}} statement that has the same
subscript.  Each pair of matched scoping statements defines a region of
the database called a {\bf block}.\index{block} Blocks can be {\bf
nested}\index{nested block} inside other blocks; in the example, the
block defined by $\mbox{\tt \$\char`\{}_4$ and $\mbox{\tt \$\char`\}}_4$
is nested inside the block defined by $\mbox{\tt \$\char`\{}_3$ and
$\mbox{\tt \$\char`\}}_3$ as well as inside the block defined by
$\mbox{\tt \$\char`\{}_1$ and $\mbox{\tt \$\char`\}}_1$.  In general, a
block may be empty, it may contain only non-scoping
statements,\footnote{Those statements other than \texttt{\$\char`\{} and
\texttt{\$\char`\}}.}\index{non-scoping statement} or it may contain any
mixture of other blocks and non-scoping statements.  (This is called a
``recursive'' definition\index{recursive definition} of a block.)

Associated with each block is a number called its {\bf nesting
level}\index{nesting level} that indicates how deeply the block is nested.
The nesting levels of the blocks in our example are as follows:
\[
  \underbrace{
    \mbox{\tt \ }
    \underbrace{
     \mbox{\tt \$\char`\{\ }
     \underbrace{
       \mbox{\tt \$\char`\{\ }
       \mbox{\tt \$\char`\}}
     }_{2}
     \mbox{\tt \ }
     \underbrace{
       \mbox{\tt \$\char`\{\ }
       \underbrace{
         \mbox{\tt \$\char`\{\ }
         \mbox{\tt \$\char`\}}
       }_{3}
       \mbox{\tt \ \$\char`\}}
     }_{2}
     \mbox{\tt \ \$\char`\}}
   }_{1}
   \mbox{\tt \ }
 }_{0}
\]
\index{\texttt{\$\char`\{} and \texttt{\$\char`\}} keywords}
The entire database is considered to be one big block (the {\bf outermost}
block) with a nesting level of 0.  The outermost block is {\em not} bracketed
by scoping statements.\footnote{The language was designed this way so that
several source files can be joined together more easily.}\index{outermost
block}

All non-scoping Metamath statements become recognized or {\bf
active}\index{active statement} at the place where they appear.\footnote{To
keep things slightly simpler, we do not bother to define the concept of
``active'' for the scoping statements.}  Certain of these statement types
become inactive at the end of the block in which they appear; these statement
types are:
\begin{center}
  \texttt{\$c}, \texttt{\$v}, \texttt{\$d}, \texttt{\$e}, and \texttt{\$f}.
%  \texttt{\$v}, \texttt{\$f}, \texttt{\$e}, and \texttt{\$d}.
\end{center}
\index{\texttt{\$c} statement}
\index{\texttt{\$d} statement}
\index{\texttt{\$e} statement}
\index{\texttt{\$f} statement}
\index{\texttt{\$v} statement}
The other statement types remain active forever (i.e.\ through the end of the
database); they are:
\begin{center}
  \texttt{\$a} and \texttt{\$p}.
%  \texttt{\$c}, \texttt{\$a}, and \texttt{\$p}.
\end{center}
\index{\texttt{\$a} statement}
\index{\texttt{\$p} statement}
Any statement (of these 7 types) located in the outermost
block\index{outermost block} will remain active through the end of the
database and thus are effectively ``global'' statements.\index{global
statement}

All \texttt{\$c} statements must be placed in the outermost block.  Since they are
therefore always global, they could be considered as belonging to both of the
above categories.

The {\bf scope}\index{scope} of a statement is the set of statements that
recognize it as active.

%c%The concept of ``active'' is also defined for math symbols\index{math
%c%symbol}.  Math symbols (constants\index{constant} and
%c%variables\index{variable}) become {\bf active}\index{active
%c%math symbol} in the \texttt{\$c}\index{\texttt{\$c}
%c%statement} and \texttt{\$v}\index{\texttt{\$v} statement} statements that
%c%declare them.  They become inactive when their declaration statements become
%c%inactive.

The concept of ``active'' is also defined for math symbols\index{math
symbol}.  Math symbols (constants\index{constant} and
variables\index{variable}) become {\bf active}\index{active math symbol}
in the \texttt{\$c}\index{\texttt{\$c} statement} and
\texttt{\$v}\index{\texttt{\$v} statement} statements that declare them.
A variable becomes inactive when its declaration statement becomes
inactive.  Because all \texttt{\$c} statements must be in the outermost
block, a constant will never become inactive after it is declared.

\subsubsection{Redeclaration of Math Symbols}
\index{redeclaration of symbols}\label{redeclaration}

%c%A math symbol may not be declared a second time while it is active, but it may
%c%be declared again after it becomes inactive.

A variable may not be declared a second time while it is active, but it may be
declared again after it becomes inactive.  This provides a convenient way to
introduce ``local'' variables,\index{local variable} i.e.\ temporary variables
for use in the frame of an assertion or in a proof without keeping them around
forever.  A previously declared variable may not be redeclared as a constant.

A constant may not be redeclared.  And, as mentioned above, constants must be
declared in the outermost block.

The reason variables may have limited scope but not constants is that an
assertion (\texttt{\$a} or \texttt{\$p} statement) remains available for use in
proofs through the end of the database.  Variables in an assertion's frame may
be substituted with whatever is needed in a proof step that references the
assertion, whereas constants remain fixed and may not be substituted with
anything.  The particular token used for a variable in an assertion's frame is
irrelevant when the assertion is referenced in a proof, and it doesn't matter
if that token is not available outside of the referenced assertion's frame.
Constants, however, must be globally fixed.

There is no theoretical
benefit for the feature allowing variables to be active for limited scopes
rather than global. It is just a convenience that allows them, for example, to
be locally grouped together with their corresponding \texttt{\$f} variable-type
declarations.

%c%If you declare a math symbol more than once, internally Metamath considers it a
%c%new distinct symbol, even though it has the same name.  If you are unaware of
%c%this, you may find that what you think are correct proofs are incorrectly
%c%rejected as invalid, because Metamath may tell you that a constant you
%c%previously declared does not match a newly declared math symbol with the same
%c%name.  For details on this subtle point, see the Comment on
%c%p.~\pageref{spec4comment}.  This is done purposely to allow temporary
%c%constants to be introduced while developing a subtheory, then allow their math
%c%symbol tokens to be reused later on; in general they will not refer to the
%c%same thing.  In practice, you would not ordinarily reuse the names of
%c%constants because it would tend to be confusing to the reader.  The reuse of
%c%names of variables, on the other hand, is something that is often useful to do
%c%(for example it is done frequently in \texttt{set.mm}).  Since variables in an
%c%assertion referenced in a proof can be substituted as needed to achieve a
%c%symbol match, this is not an issue.

% (This section covers a somewhat advanced topic you may want to skip
% at first reading.)
%
% Under certain circumstances, math symbol\index{math symbol}
% tokens\index{token} may be redeclared (i.e.\ the token
% may appear in more than
% one \texttt{\$c}\index{\texttt{\$c} statement} or \texttt{\$v}\index{\texttt{\$v}
% statement} statement).  You might want to do this say, to make temporary use
% of a variable name without having to worry about its affect elsewhere,
% somewhat analogous to declaring a local variable in a standard computer
% language.  Understanding what goes on when math symbol tokens are redeclared
% is a little tricky to understand at first, since it requires that we
% distinguish the token itself from the math symbol that it names.  It will help
% if we first take a peek at the internal workings of the
% Metamath\index{Metamath} program.
%
% Metamath reserves a memory location for each occurrence of a
% token\index{token} in a declaration statement (\texttt{\$c}\index{\texttt{\$c}
% statement} or \texttt{\$v}\index{\texttt{\$v} statement}).  If a given token appears
% in more than one declaration statement, it will refer to more than one memory
% locations.  A math symbol\index{math symbol} may be thought of as being one of
% these memory locations rather than as the token itself.  Only one of the
% memory locations associated with a given token may be active at any one time.
% The math symbol (memory location) that gets looked up when the token appears
% in a non-declaration statement is the one that happens to be active at that
% time.
%
% We now look at the rules for the redeclaration\index{redeclaration of symbols}
% of math symbol tokens.
% \begin{itemize}
% \item A math symbol token may not be declared twice in the
% same block.\footnote{While there is no theoretical reason for disallowing
% this, it was decided in the design of Metamath that allowing it would offer no
% advantage and might cause confusion.}
% \item An inactive math symbol may always be
% redeclared.
% \item  An active math symbol may be redeclared in a different (i.e.\
% inner) block\index{block} from the one it became active in.
% \end{itemize}
%
% When a math symbol token is redeclared, it conceptually refers to a different
% math symbol, just as it would be if it were called a different name.  In
% addition, the original math symbol that it referred to, if it was active,
% temporarily becomes inactive.  At the end of the block in which the
% redeclaration occurred, the new math symbol\index{math symbol} becomes
% inactive and the original symbol becomes active again.  This concept is
% illustrated in the following example, where the symbol \texttt{e} is
% ordinarily a constant (say Euler's constant, 2.71828...) but
% temporarily we want to use it as a ``local'' variable, say as a coefficient
% in the equation $a x^4 + b x^3 + c x^2 + d x + e$:
% \[
%   \mbox{\tt \$\char`\{\ \$c e \$.}
%   \underbrace{
%     \ \ldots\ %
%     \mbox{\tt \$\char`\{}\ \ldots\ %
%   }_{\mbox{\rm region A}}
%   \mbox{\tt \$v e \$.}
%   \underbrace{
%     \mbox{\ \ \ \ldots\ \ \ }
%   }_{\mbox{\rm region B}}
%   \mbox{\tt \$\char`\}}
%   \underbrace{
%     \mbox{\ \ \ \ldots\ \ \ }
%   }_{\mbox{\rm region C}}
%   \mbox{\tt \$\char`\}}
% \]
% \index{\texttt{\$\char`\{} and \texttt{\$\char`\}} keywords}
% In region A, the token \texttt{e} refers to a constant.  It is redeclared as a
% variable in region B, and any reference to it in this region will refer to this
% variable.  In region C, the redeclaration becomes inactive, and the original
% declaration becomes active again.  In region C, the token \texttt{x} refers to the
% original constant.
%
% As a practical matter, overuse of math symbol\index{math symbol}
% redeclarations\index{redeclaration of symbols} can be confusing (even though
% it is well-defined) and is best avoided when possible.  Here are some good
% general guidelines you can follow.  Usually, you should declare all
% constants\index{constant} in the outermost block\index{outermost block},
% especially if they are general-purpose (such as the token \verb$A.$, meaning
% $\forall$ or ``for all'').  This will make them ``globally'' active (although
% as in the example above local redeclarations will temporarily make them
% inactive.)  Most or all variables\index{variable}, on the other hand, could be
% declared in inner blocks, so that the token for them can be used later for a
% different type of variable or a constant.  (The names of the variables you
% choose are not used when you refer to an assertion\index{assertion} in a
% proof, whereas constants must match exactly.  A locally declared constant will
% not match a globally declared constant in a proof, even if they use the same
% token, because Metamath internally considers them to be different math
% symbols.)  To avoid confusion, you should generally avoid redeclaring active
% variables.  If you must redeclare them, do so at the beginning of a block.
% The temporary declaration of constants in inner blocks might be occasionally
% appropriate when you make use of a temporary definition to prove lemmas
% leading to a main result that does not make direct use of the definition.
% This way, you will not clutter up your database with a large number of
% seldom-used global constant symbols.  You might want to note that while
% inactive constants may not appear directly in an assertion (a \texttt{\$a}\index{\texttt{\$a}
% statement} or \texttt{\$p}\index{\texttt{\$p} statement}
% statement), they may be indirectly used in the proof of a \texttt{\$p} statement
% so long as they do not appear in the final math symbol sequence constructed by
% the proof.  In the end, you will have to use your best judgment, taking into
% account standard mathematical usage of the symbols as well as consideration
% for the reader of your work.
%
% \subsubsection{Reuse of Labels}\index{reuse of labels}\index{label}
%
% The \texttt{\$e}\index{\texttt{\$e} statement}, \texttt{\$f}\index{\texttt{\$f}
% statement}, \texttt{\$a}\index{\texttt{\$a} statement}, and
% \texttt{\$p}\index{\texttt{\$p}
% statement} statement types require labels, which allow them to be
% referenced later inside proofs.  A label is considered {\bf
% active}\index{active label} when the statement it is associated with is
% active.  The token\index{token} for a label may be reused
% (redeclared)\index{redeclaration of labels} provided that it is not being used
% for a currently active label.  (Unlike the tokens for math symbols, active
% label tokens may not be redeclared in an inner scope.)  Note that the labels
% of \texttt{\$a} and \texttt{\$p} statements can never be reused after these
% statements appear, because these statements remain active through the end of
% the database.
%
% You might find the reuse of labels a convenient way to have standard names for
% temporary hypotheses, such as \texttt{h1}, \texttt{h2}, etc.  This way you don't have
% to invent unique names for each of them, and in some cases it may be less
% confusing to the reader (although in other cases it might be more confusing, if
% the hypothesis is located far away from the assertion that uses
% it).\footnote{The current implementation requires that all labels, even
% inactive ones, be unique.}

\subsubsection{Frames Revisited}\index{frames and scoping statements}

Now that we have covered scoping, we will look at how an arbitrary
Metamath database can be converted to the simple sequence of extended
frames described on p.~\pageref{framelist}.  This is also how Metamath
stores the database internally when it reads in the database
source.\label{frameconvert} The method is simple.  First, we collect all
constant and variable (\texttt{\$c} and \texttt{\$v}) declarations in
the database, ignoring duplicate declarations of the same variable in
different scopes.  We then put our collected \texttt{\$c} and
\texttt{\$v} declarations at the beginning of the database, so that
their scope is the entire database.  Next, for each assertion in the
database, we determine its frame and extended frame.  The extended frame
is simply the \texttt{\$f}, \texttt{\$e}, and \texttt{\$d} statements
that are active.  The frame is the extended frame with all optional
hypotheses removed.

An equivalent way of saying this is that the extended frame of an assertion
is the collection of all \texttt{\$f}, \texttt{\$e}, and \texttt{\$d} statements
whose scope includes the assertion.
The \texttt{\$f} and \texttt{\$e} statements
occur in the order they appear
(order is irrelevant for \texttt{\$d} statements).

%c%, renaming any
%c%redeclared variables as needed so that all of them have unique names.  (The
%c%exact renaming convention is unimportant.  You might imagine renaming
%c%different declarations of math symbol \texttt{a} as \texttt{a\$1}, \texttt{a\$2}, etc.\
%c%which would prevent any conflicts since \texttt{\$} is not a legal character in a
%c%math symbol token.)

\section{The Anatomy of a Proof} \label{proof}
\index{proof!Metamath, description of}

Each provable assertion (\texttt{\$p}\index{\texttt{\$p} statement} statement) in a
database must include a {\bf proof}\index{proof}.  The proof is located
between the \texttt{\$=}\index{\texttt{\$=} keyword} and \texttt{\$.}\ keywords in the
\texttt{\$p} statement.

In the basic Metamath language\index{basic language}, a proof is a
sequence of statement labels.  This label sequence\index{label sequence}
serves as a set of instructions that the Metamath program uses to
construct a series of math symbol sequences.  The construction must
ultimately result in the math symbol sequence contained between the
\texttt{\$p}\index{\texttt{\$p} statement} and
\texttt{\$=}\index{\texttt{\$=} keyword} keywords of the \texttt{\$p}
statement.  Otherwise, the Metamath program will consider the proof
incorrect, and it will notify you with an appropriate error message when
you ask it to verify the proof.\footnote{To make the loading faster, the
Metamath program does not automatically verify proofs when you
\texttt{read} in a database unless you use the \texttt{/verify}
qualifier.  After a database has been read in, you may use the
\texttt{verify proof *} command to verify proofs.}\index{\texttt{verify
proof} command} Each label in a proof is said to {\bf
reference}\index{label reference} its corresponding statement.

Associated with any assertion\index{assertion} (\texttt{\$p} or
\texttt{\$a}\index{\texttt{\$a} statement} statement) is a set of
hypotheses (\texttt{\$f}\index{\texttt{\$f} statement} or
\texttt{\$e}\index{\texttt{\$e} statement} statements) that are active
with respect to that assertion.  Some are mandatory and the others are
optional.  You should review these concepts if necessary.

Each label\index{label} in a proof must be either the label of a
previous assertion (\texttt{\$a}\index{\texttt{\$a} statement} or
\texttt{\$p}\index{\texttt{\$p} statement} statement) or the label of an
active hypothesis (\texttt{\$e} or \texttt{\$f}\index{\texttt{\$f}
statement} statement) of the \texttt{\$p} statement containing the
proof.  Hypothesis labels may reference both the
mandatory\index{mandatory hypothesis} and the optional hypotheses of the
\texttt{\$p} statement.

The label sequence in a proof specifies a construction in {\bf reverse Polish
notation}\index{reverse Polish notation (RPN)} (RPN).  You may be familiar
with RPN if you have used older
Hewlett--Packard or similar hand-held calculators.
In the calculator analogy, a hypothesis label\index{hypothesis label} is like
a number and an assertion label\index{assertion label} is like an operation
(more precisely, an $n$-ary operation when the
assertion has $n$ \texttt{\$e}-hypotheses).
On an RPN calculator, an operation takes one or more previous numbers in an
input sequence, performs a calculation on them, and replaces those numbers and
itself with the result of the calculation.  For example, the input sequence
$2,3,+$ on an RPN calculator results in $5$, and the input sequence
$2,3,5,{\times},+$ results in $2,15,+$ which results in $17$.

Understanding how RPN is processed involves the concept of a {\bf
stack}\index{stack}\index{RPN stack}, which can be thought of as a set of
temporary memory locations that hold intermediate results.  When Metamath
encounters a hypothesis label it places or {\bf pushes}\index{push} the math
symbol sequence of the hypothesis onto the stack.  When Metamath encounters an
assertion label, it associates the most recent stack entries with the {\em
mandatory} hypotheses\index{mandatory hypothesis} of the assertion, in the
order where the most recent stack entry is associated with the last mandatory
hypothesis of the assertion.  It then determines what
substitutions\index{substitution!variable}\index{variable substitution} have
to be made into the variables of the assertion's mandatory hypotheses to make
them identical to the associated stack entries.  It then makes those same
substitutions into the assertion itself.  Finally, Metamath removes or {\bf
pops}\index{pop} the matched hypotheses from the stack and pushes the
substituted assertion onto the stack.

For the purpose of matching the mandatory hypothesis to the most recent stack
entries, whether a hypothesis is a \texttt{\$e} or \texttt{\$f} statement is
irrelevant.  The only important thing is that a set of
substitutions\footnote{In the Metamath spec (Section~\ref{spec}), we use the
singular term ``substitution'' to refer to the set of substitutions we talk
about here.} exist that allow a match (and if they don't, the proof verifier
will let you know with an error message).  The Metamath language is specified
in such a way that if a set of substitutions exists, it will be unique.
Specifically, the requirement that each variable have a type specified for it
with a \texttt{\$f} statement ensures the uniqueness.

We will illustrate this with an example.
Consider the following Metamath source file:
\begin{verbatim}
$c ( ) -> wff $.
$v p q r s $.
wp $f wff p $.
wq $f wff q $.
wr $f wff r $.
ws $f wff s $.
w2 $a wff ( p -> q ) $.
wnew $p wff ( s -> ( r -> p ) ) $= ws wr wp w2 w2 $.
\end{verbatim}
This Metamath source example shows the definition and ``proof'' (i.e.,
construction) of a well-formed formula (wff)\index{well-formed formula (wff)}
in propositional calculus.  (You may wish to type this example into a file to
experiment with the Metamath program.)  The first two statements declare
(introduce the names of) four constants and four variables.  The next four
statements specify the variable types, namely that
each variable is assumed to be a wff.  Statement \texttt{w2} defines (postulates)
a way to produce a new wff, \texttt{( p -> q )}, from two given wffs \texttt{p} and
\texttt{q}. The mandatory hypotheses of \texttt{w2} are \texttt{wp} and \texttt{wq}.
Statement \texttt{wnew} claims that \texttt{( s -> ( r -> p ) )} is a wff given
three wffs \texttt{s}, \texttt{r}, and \texttt{p}.  More precisely, \texttt{wnew} claims
that the sequence of ten symbols \texttt{wff ( s -> ( r -> p ) )} is provable from
previous assertions and the hypotheses of \texttt{wnew}.  Metamath does not know
or care what a wff is, and as far as it is concerned
the typecode \texttt{wff} is just an
arbitrary constant symbol in a math symbol sequence.  The mandatory hypotheses
of \texttt{wnew} are \texttt{wp}, \texttt{wr}, and \texttt{ws}; \texttt{wq} is an optional
hypothesis.  In our particular proof, the optional hypothesis is not
referenced, but in general, any combination of active (i.e.\ optional and
mandatory) hypotheses could be referenced.  The proof of statement \texttt{wnew}
is the sequence of five labels starting with \texttt{ws} (step~1) and ending with
\texttt{w2} (step~5).

When Metamath verifies the proof, it scans the proof from left to right.  We
will examine what happens at each step of the proof.  The stack starts off
empty.  At step 1, Metamath looks up label \texttt{ws} and determines that it is a
hypothesis, so it pushes the symbol sequence of statement \texttt{ws} onto the
stack:

\begin{center}\begin{tabular}{|l|l|}\hline
{Stack location} & {Contents} \\ \hline \hline
1 & \texttt{wff s} \\ \hline
\end{tabular}\end{center}

Metamath sees that the labels \texttt{wr} and \texttt{wp} in steps~2 and 3 are also
hypotheses, so it pushes them onto the stack.  After step~3, the stack looks
like
this:

\begin{center}\begin{tabular}{|l|l|}\hline
{Stack location} & {Contents} \\ \hline \hline
3 & \texttt{wff p} \\ \hline
2 & \texttt{wff r} \\ \hline
1 & \texttt{wff s} \\ \hline
\end{tabular}\end{center}

At step 4, Metamath sees that label \texttt{w2} is an assertion, so it must do
some processing.  First, it associates the mandatory hypotheses of \texttt{w2},
which are \texttt{wp} and \texttt{wq}, with stack locations~2 and 3, {\em in that
order}. Metamath determines that the only possible way
to make hypothesis \texttt{wp} match (become identical to) stack location~2 and
\texttt{wq} match stack location 3 is to substitute variable \texttt{p} with \texttt{r}
and \texttt{q} with \texttt{p}.  Metamath makes these substitutions into \texttt{w2} and
obtains the symbol sequence \texttt{wff ( r -> p )}.  It removes the hypotheses
from stack locations~2 and 3, then places the result into stack location~2:

\begin{center}\begin{tabular}{|l|l|}\hline
{Stack location} & {Contents} \\ \hline \hline
2 & \texttt{wff ( r -> p )} \\ \hline
1 & \texttt{wff s} \\ \hline
\end{tabular}\end{center}

At step 5, Metamath sees that label \texttt{w2} is an assertion, so it must again
do some processing.  First, it matches the mandatory hypotheses of \texttt{w2},
which are \texttt{wp} and \texttt{wq}, to stack locations 1 and 2.
Metamath determines that the only possible way to make the
hypotheses match is to substitute variable \texttt{p} with \texttt{s} and \texttt{q} with
\texttt{( r -> p )}.  Metamath makes these substitutions into \texttt{w2} and obtains
the symbol
sequence \texttt{wff ( s -> ( r -> p ) )}.  It removes stack
locations 1 and 2, then places the result into stack location~1:

\begin{center}\begin{tabular}{|l|l|}\hline
{Stack location} & {Contents} \\ \hline \hline
1 & \texttt{wff ( s -> ( r -> p ) )} \\ \hline
\end{tabular}\end{center}

After Metamath finishes processing the proof, it checks to see that the
stack contains exactly one element and that this element is
the same as the math symbol sequence in the
\texttt{\$p}\index{\texttt{\$p} statement} statement.  This is the case for our
proof of \texttt{wnew},
so we have proved \texttt{wnew} successfully.  If the result
differs, Metamath will notify you with an error message.  An error message
will also result if the stack contains more than one entry at the end of the
proof, or if the stack did not contain enough entries at any point in the
proof to match all of the mandatory hypotheses\index{mandatory hypothesis} of
an assertion.  Finally, Metamath will notify you with an error message if no
substitution is possible that will make a referenced assertion's hypothesis
match the
stack entries.  You may want to experiment with the different kinds of errors
that Metamath will detect by making some small changes in the proof of our
example.

Metamath's proof notation was designed primarily to express proofs in a
relatively compact manner, not for readability by humans.  Metamath can display
proofs in a number of different ways with the \texttt{show proof}\index{\texttt{show
proof} command} command.  The
\texttt{/lemmon} qualifier displays it in a format that is easier to read when the
proofs are short, and you saw examples of its use in Chapter~\ref{using}.  For
longer proofs, it is useful to see the tree structure of the proof.  A tree
structure is displayed when the \texttt{/lemmon} qualifier is omitted.  You will
probably find this display more convenient as you get used to it. The tree
display of the proof in our example looks like
this:\label{treeproof}\index{tree-style proof}\index{proof!tree-style}
\begin{verbatim}
1     wp=ws    $f wff s
2        wp=wr    $f wff r
3        wq=wp    $f wff p
4     wq=w2    $a wff ( r -> p )
5  wnew=w2  $a wff ( s -> ( r -> p ) )
\end{verbatim}
The number to the left of each line is the step number.  Following it is a
{\bf hypothesis association}\index{hypothesis association}, consisting of two
labels\index{label} separated by \texttt{=}.  To the left of the \texttt{=} (except
in the last step) is the label of a hypothesis of an assertion referenced
later in the proof; here, steps 1 and 4 are the hypothesis associations for
the assertion \texttt{w2} that is referenced in step 5.  A hypothesis association
is indented one level more than the assertion that uses it, so it is easy to
find the corresponding assertion by moving directly down until the indentation
level decreases to one less than where you started from.  To the right of each
\texttt{=} is the proof step label for that proof step.  The statement keyword of
the proof step label is listed next, followed by the content of the top of the
stack (the most recent stack entry) as it exists after that proof step is
processed.  With a little practice, you should have no trouble reading proofs
displayed in this format.

Metamath proofs include the syntax construction of a formula.
In standard mathematics, this kind of
construction is not considered a proper part of the proof at all, and it
certainly becomes rather boring after a while.
Therefore,
by default the \texttt{show proof}\index{\texttt{show proof}
command} command does not show the syntax construction.
Historically \texttt{show proof} command
\textit{did} show the syntax construction, and you needed to add the
\texttt{/essential} option to hide, them, but today
\texttt{/essential} is the default and you need to use
\texttt{/all} to see the syntax constructions.

When verifying a proof, Metamath will check that no mandatory
\texttt{\$d}\index{\texttt{\$d} statement}\index{mandatory \texttt{\$d}
statement} statement of an assertion referenced in a proof is violated
when substitutions\index{substitution!variable}\index{variable
substitution} are made to the variables in the assertion.  For details
see Section~\ref{spec4} or \ref{dollard}.

\subsection{The Concept of Unification} \label{unify}

During the course of verifying a proof, when Metamath\index{Metamath}
encounters an assertion label\index{assertion label}, it associates the
mandatory hypotheses\index{mandatory hypothesis} of the assertion with the top
entries of the RPN stack\index{stack}\index{RPN stack}.  Metamath then
determines what substitutions\index{substitution!variable}\index{variable
substitution} it must make to the variables in the assertion's mandatory
hypotheses in order for these hypotheses to become identical to their
corresponding stack entries.  This process is called {\bf
unification}\index{unification}.  (We also informally use the term
``unification'' to refer to a set of substitutions that results from the
process, as in ``two unifications are possible.'')  After the substitutions
are made, the hypotheses are said to be {\bf unified}.

If no such substitutions are possible, Metamath will consider the proof
incorrect and notify you with an error message.
% (deleted 3/10/07, per suggestion of Mel O'Cat:)
% The syntax of the
% Metamath language ensures that if a set of substitutions exists, it
% will be unique.

The general algorithm for unification described in the literature is
somewhat complex.
However, in the case of Metamath it is intentionally trivial.
Mandatory hypotheses must be
pushed on the proof stack in the order in which they appear.
In addition, each variable must have its type specified
with a \texttt{\$f} hypothesis before it is used
and that each \texttt{\$f} hypothesis
have the restricted syntax of a typecode (a constant) followed by a variable.
The typecode in the \texttt{\$f} hypothesis must match the first symbol of
the corresponding RPN stack entry (which will also be a constant), so
the only possible match for the variable in the \texttt{\$f} hypothesis is
the sequence of symbols in the stack entry after the initial constant.

In the Proof Assistant\index{Proof Assistant}, a more general unification
algorithm is used.  While a proof is being developed, sometimes not enough
information is available to determine a unique unification.  In this case
Metamath will ask you to pick the correct one.\index{ambiguous
unification}\index{unification!ambiguous}

\section{Extensions to the Metamath Language}\index{extended
language}

\subsection{Comments in the Metamath Language}\label{comments}
\index{markup notation}
\index{comments!markup notation}

The commenting feature allows you to annotate the contents of
a database.  Just as with most
computer languages, comments are ignored for the purpose of interpreting the
contents of the database. Comments effectively act as
additional white space\index{white
space} between tokens
when a database is parsed.

A comment may be placed at the beginning, end, or
between any two tokens\index{token} in a source file.

Comments have the following syntax:
\begin{center}
 \texttt{\$(} {\em text} \texttt{\$)}
\end{center}
Here,\index{\texttt{\$(} and \texttt{\$)} auxiliary
keywords}\index{comment} {\em text} is a string, possibly empty, of any
characters in Metamath's character set (p.~\pageref{spec1chars}), except
that the character strings \texttt{\$(} and \texttt{\$)} may not appear
in {\em text}.  Thus nested comments are not
permitted:\footnote{Computer languages have differing standards for
nested comments, and rather than picking one it was felt simplest not to
allow them at all, at least in the current version (0.177) of
Metamath\index{Metamath!limitations of version 0.177}.} Metamath will
complain if you give it
\begin{center}
 \texttt{\$( This is a \$( nested \$) comment.\ \$)}
\end{center}
To compensate for this non-nesting behavior, I often change all \texttt{\$}'s
to \texttt{@}'s in sections of Metamath code I wish to comment out.

The Metamath program supports a number of markup mechanisms and conventions
to generate good-looking results in \LaTeX\ and {\sc html},
as discussed below.
These markup features have to do only with how the comments are typeset,
and have no effect on how Metamath verifies the proofs in the database.
The improper
use of them may result in incorrectly typeset output, but no Metamath
error messages will result during the \texttt{read} and \texttt{verify
proof} commands.  (However, the \texttt{write
theorem\texttt{\char`\_}list} command
will check for markup errors as a side-effect of its
{\sc html} generation.)
Section~\ref{texout} has instructions for creating \LaTeX\ output, and
section~\ref{htmlout} has instructions for creating
{\sc html}\index{HTML} output.

\subsubsection{Headings}\label{commentheadings}

If the \texttt{\$(} is immediately followed by a new line
starting with a heading marker, it is a header.
This can start with:

\begin{itemize}
 \item[] \texttt{\#\#\#\#} - major part header
 \item[] \texttt{\#*\#*} - section header
 \item[] \texttt{=-=-} - subsection header
 \item[] \texttt{-.-.} - subsubsection header
\end{itemize}

The line following the marker line
will be used for the table of contents entry, after trimming spaces.
The next line should be another (closing) matching marker line.
Any text after that
but before the closing \texttt{\$}, such as an extended description of the
section, will be included on the \texttt{mmtheoremsNNN.html} page.

For more information, run
\texttt{help write theorem\char`\_list}.

\subsubsection{Math mode}
\label{mathcomments}
\index{\texttt{`} inside comments}
\index{\texttt{\char`\~} inside comments}
\index{math mode}

Inside of comments, a string of tokens\index{token} enclosed in
grave accents\index{grave accent (\texttt{`})} (\texttt{`}) will be converted
to standard mathematical symbols during
{\sc HTML}\index{HTML} or \LaTeX\ output
typesetting,\index{latex@{\LaTeX}} according to the information in the
special \texttt{\$t}\index{\texttt{\$t} comment}\index{typesetting
comment} comment in the database
(see section~\ref{tcomment} for information about the typesetting
comment, and Appendix~\ref{ASCII} to see examples of its results).

The first grave accent\index{grave accent (\texttt{`})} \texttt{`}
causes the output processor to enter {\bf math mode}\index{math mode}
and the second one exits it.
In this
mode, the characters following the \texttt{`} are interpreted as a
sequence of math symbol tokens separated by white space\index{white
space}.  The tokens are looked up in the \texttt{\$t}
comment\index{\texttt{\$t} comment}\index{typesetting comment} and if
found, they will be replaced by the standard mathematical symbols that
they correspond to before being placed in the typeset output file.  If
not found, the symbol will be output as is and a warning will be issued.
The tokens do not have to be active in the database, although a warning
will be issued if they are not declared with \texttt{\$c} or
\texttt{\$v} statements.

Two consecutive
grave accents \texttt{``} are treated as a single actual grave accent
(both inside and outside of math mode) and will not cause the output
processor to enter or exit math mode.

Here is an example of its use\index{Pierce's axiom}:
\begin{center}
\texttt{\$( Pierce's axiom, ` ( ( ph -> ps ) -> ph ) -> ph ` ,\\
         is not very intuitive. \$)}
\end{center}
becomes
\begin{center}
   \texttt{\$(} Pierce's axiom, $((\varphi \rightarrow \psi)\rightarrow
\varphi)\rightarrow \varphi$, is not very intuitive. \texttt{\$)}
\end{center}

Note that the math symbol tokens\index{token} must be surrounded by white
space\index{white space}.
%, since there is no context that allows ambiguity to be
%resolved, as is the case with math symbol sequences in some of the Metamath
%statements.
White space should also surround the \texttt{`}
delimiters.

The math mode feature also gives you a quick and easy way to generate
text containing mathematical symbols, independently of the intended
purpose of Metamath.\index{Metamath!using as a math editor} To do this,
simply create your text with grave accents surrounding your formulas,
after making sure that your math symbols are mapped to \LaTeX\ symbols
as described in Appendix~\ref{ASCII}.  It is easier if you start with a
database with predefined symbols such as \texttt{set.mm}.  Use your
grave-quoted math string to replace an existing comment, then typeset
the statement corresponding to that comment following the instructions
from the \texttt{help tex} command in the Metamath program.  You will
then probably want to edit the resulting file with a text editor to fine
tune it to your exact needs.

\subsubsection{Label Mode}\index{label mode}

Outside of math mode, a tilde\index{tilde (\texttt{\char`\~})} \verb/~/
indicates to Metamath's\index{Metamath} output processor that the
token\index{token} that follows (i.e.\ the characters up to the next
white space\index{white space}) represents a statement label or URL.
This formatting mode is called {\bf label mode}\index{label mode}.
If a literal tilde
is desired (outside of math mode) instead of label mode,
use two tildes in a row to represent it.

When generating a \LaTeX\ output file,
the following token will be formatted in \texttt{typewriter}
font, and the tilde removed, to make it stand out from the rest of the text.
This formatting will be applied to all characters after the
tilde up to the first white space\index{white space}.
Whether
or not the token is an actual statement label is not checked, and the
token does not have to have the correct syntax for a label; no error
messages will be produced.  The only effect of the label mode on the
output is that typewriter font will be used for the tokens that are
placed in the \LaTeX\ output file.

When generating {\sc html},
the tokens after the tilde {\em must} be a URL (either http: or https:)
or a valid label.
Error messages will be issued during that output if they aren't.
A hyperlink will be generated to that URL or label.

\subsubsection{Link to bibliographical reference}\index{citation}%
\index{link to bibliographical reference}

Bibliographical references are handled specially when generating
{\sc html} if formatted specially.
Text in the form \texttt{[}{\em author}\texttt{]}
is considered a link to a bibliographical reference.
See \texttt{help html} and \texttt{help write
bibliography} in the Metamath program for more
information.
% \index{\texttt{\char`\[}\ldots\texttt{]} inside comments}
See also Sections~\ref{tcomment} and \ref{wrbib}.

The \texttt{[}{\em author}\texttt{]} notation will also create an entry in
the bibliography cross-reference file generated by \texttt{write
bibliography} (Section~\ref{wrbib}) for {\sc HTML}.
For this to work properly, the
surrounding comment must be formatted as follows:
\begin{quote}
    {\em keyword} {\em label} {\em noise-word}
     \texttt{[}{\em author}\texttt{] p.} {\em number}
\end{quote}
for example
\begin{verbatim}
     Theorem 5.2 of [Monk] p. 223
\end{verbatim}
The {\em keyword} is not case sensitive and must be one of the following:
\begin{verbatim}
     theorem lemma definition compare proposition corollary
     axiom rule remark exercise problem notation example
     property figure postulate equation scheme chapter
\end{verbatim}
The optional {\em label} may consist of more than one
(non-{\em keyword} and non-{\em noise-word}) word.
The optional {\em noise-word} is one of:
\begin{verbatim}
     of in from on
\end{verbatim}
and is  ignored when the cross-reference file is created.  The
\texttt{write
biblio\-graphy} command will perform error checking to verify the
above format.\index{error checking}

\subsubsection{Parentheticals}\label{parentheticals}

The end of a comment may include one or more parenthicals, that is,
statements enclosed in parentheses.
The Metamath program looks for certain parentheticals and can issue
warnings based on them.
They are:

\begin{itemize}
 \item[] \texttt{(Contributed by }
   \textit{NAME}\texttt{,} \textit{DATE}\texttt{.)} -
   document the original contributor's name and the date it was created.
 \item[] \texttt{(Revised by }
   \textit{NAME}\texttt{,} \textit{DATE}\texttt{.)} -
   document the contributor's name and creation date
   that resulted in significant revision
   (not just an automated minimization or shortening).
 \item[] \texttt{(Proof shortened by }
   \textit{NAME}\texttt{,} \textit{DATE}\texttt{.)} -
   document the contributor's name and date that developed a significant
   shortening of the proof (not just an automated minimization).
 \item[] \texttt{(Proof modification is discouraged.)} -
   Note that this proof should normally not be modified.
 \item[] \texttt{(New usage is discouraged.)} -
   Note that this assertion should normally not be used.
\end{itemize}

The \textit{DATE} must be in form YYYY-MMM-DD, where MMM is the
English abbreviation of that month.

\subsubsection{Other markup}\label{othermarkup}
\index{markup notation}

There are other markup notations for generating good-looking results
beyond math mode and label mode:

\begin{itemize}
 \item[]
         \texttt{\char`\_} (underscore)\index{\texttt{\char`\_} inside comments} -
             Italicize text starting from
              {\em space}\texttt{\char`\_}{\em non-space} (i.e.\ \texttt{\char`\_}
              with a space before it and a non-space character after it) until
             the next
             {\em non-space}\texttt{\char`\_}{\em space}.  Normal
             punctuation (e.g.\ a trailing
             comma or period) is ignored when determining {\em space}.
 \item[]
         \texttt{\char`\_} (underscore) - {\em
         non-space}\texttt{\char`\_}{\em non-space-string}, where
          {\em non-space-string} is a string of non-space characters,
         will make {\em non-space-string} become a subscript.
 \item[]
         \texttt{<HTML>}...\texttt{</HTML>} - do not convert
         ``\texttt{<}'' and ``\texttt{>}''
         in the enclosed text when generating {\sc HTML},
         otherwise process markup normally. This allows direct insertion
         of {\sc html} commands.
 \item[]
       ``\texttt{\&}ref\texttt{;}'' - insert an {\sc HTML}
         character reference.
         This is how to insert arbitrary Unicode characters
         (such as accented characters).  Currently only directly supported
         when generating {\sc HTML}.
\end{itemize}

It is recommended that spaces surround any \texttt{\char`\~} and
\texttt{`} tokens in the comment and that a space follow the {\em label}
after a \texttt{\char`\~} token.  This will make global substitutions
to change labels and symbol names much easier and also eliminate any
future chance of ambiguity.  Spaces around these tokens are automatically
removed in the final output to conform with normal rules of punctuation;
for example, a space between a trailing \texttt{`} and a left parenthesis
will be removed.

A good way to become familiar with the markup notation is to look at
the extensive examples in the \texttt{set.mm} database.

\subsection{The Typesetting Comment (\texttt{\$t})}\label{tcomment}

The typesetting comment \texttt{\$t} in the input database file
provides the information necessary to produce good-looking results.
It provides \LaTeX\ and {\sc html}
definitions for math symbols,
as well supporting as some
customization of the generated web page.
If you add a new token to a database, you should also
update the \texttt{\$t} comment information if you want to eventually
create output in \LaTeX\ or {\sc HTML}.
See the
\texttt{set.mm}\index{set theory database (\texttt{set.mm})} database
file for an extensive example of a \texttt{\$t} comment illustrating
many of the features described below.

Programs that do not need to generate good-looking presentation results,
such as programs that only verify Metamath databases,
can completely ignore typesetting comments
and just treat them as normal comments.
Even the Metamath program only consults the
\texttt{\$t} comment information when it needs to generate typeset output
in \LaTeX\ or {\sc HTML}
(e.g., when you open a \LaTeX\ output file with the \texttt{open tex} command).

We will first discuss the syntax of typesetting comments, and then
briefly discuss how this can be used within the Metamath program.

\subsubsection{Typesetting Comment Syntax Overview}

The typesetting comment is identified by the token
\texttt{\$t}\index{\texttt{\$t} comment}\index{typesetting comment} in
the comment, and the typesetting comment ends at the matching
\texttt{\$)}:
\[
  \mbox{\tt \$(\ }
  \mbox{\tt \$t\ }
  \underbrace{
    \mbox{\tt \ \ \ \ \ \ \ \ \ \ \ }
    \cdots
    \mbox{\tt \ \ \ \ \ \ \ \ \ \ \ }
  }_{\mbox{Typesetting definitions go here}}
  \mbox{\tt \ \$)}
\]

There must be one or more white space characters, and only white space
characters, between the \texttt{\$(} that starts the comment
and the \texttt{\$t} symbol,
and the \texttt{\$t} must be followed by one
or more white space characters
(see section \ref{whitespace} for the definition of white space characters).
The typesetting comment continues until the comment end token \texttt{\$)}
(which must be preceded by one or more white space characters).

In version 0.177\index{Metamath!limitations of version 0.177} of the
Metamath program, there may be only one \texttt{\$t} comment in a
database.  This restriction may be lifted in the future to allow
many \texttt{\$t} comments in a database.

Between the \texttt{\$t} symbol (and its following white space) and the
comment end token \texttt{\$)} (and its preceding white space)
is a sequence of one or more typesetting definitions, where
each definition has the form
\textit{definition-type arg arg ... ;}.
Each of the zero or more \textit{arg} values
can be either a typesetting data or a keyword
(what keywords are allowed, and where, depends on the specific
\textit{definition-type}).
The \textit{definition-type}, and each argument \textit{arg},
are separated by one or more white space characters.
Every definition ends in an unquoted semicolon;
white space is not required before the terminating semicolon of a definition.
Each definition should start on a new line.\footnote{This
restriction of the current version of Metamath
(0.177)\index{Metamath!limitations of version 0.177} may be removed
in a future version, but you should do it anyway for readability.}

For example, this typesetting definition:
\begin{center}
 \verb$latexdef "C_" as "\subseteq";$
\end{center}
defines the token \verb$C_$ as the \LaTeX\ symbol $\subseteq$ (which means
``subset'').

Typesetting data is a sequence of one or more quoted strings
(if there is more than one, they are connected by \texttt{\char`\+}).
Often a single quoted string is used to provide data for a definition, using
either double (\texttt{\char`\"}) or single (\texttt{'}) quotation marks.
However,
{\em a quoted string (enclosed in quotation marks) may not include
line breaks.}
A quoted string
may include a quotation mark that matches the enclosing quotes by repeating
the quotation mark twice.  Here are some examples:

\begin{tabu}   { l l }
\textbf{Example} & \textbf{Meaning} \\
\texttt{\char`\"a\char`\"\char`\"b\char`\"} & \texttt{a\char`\"b} \\
\texttt{'c''d'} & \texttt{c'd} \\
\texttt{\char`\"e''f\char`\"} & \texttt{e''f} \\
\texttt{'g\char`\"\char`\"h'} & \texttt{g\char`\"\char`\"h} \\
\end{tabu}

Finally, a long quoted string
may be broken up into multiple quoted strings (considered, as a whole,
a single quoted string) and joined with \texttt{\char`\+}.
You can even use multiple lines as long as a
'+' is at the end of every line except the last one.
The \texttt{\char`\+} should be preceded and followed by at least one
white space character.
Thus, for example,
\begin{center}
 \texttt{\char`\"ab\char`\"\ \char`\+\ \char`\"cd\char`\"
    \ \char`\+\ \\ 'ef'}
\end{center}
is the same as
\begin{center}
 \texttt{\char`\"abcdef\char`\"}
\end{center}

{\sc c}-style comments \texttt{/*}\ldots\texttt{*/} are also supported.

In practice, whenever you add a new math token you will often want to add
typesetting definitions using
\texttt{latexdef}, \texttt{htmldef}, and
\texttt{althtmldef}, as described below.
That way, they will all be up to date.
Of course, whether or not you want to use all three definitions will
depend on how the database is intended to be used.

Below we discuss the different possible \textit{definition-kind} options.
We will show data surrounded by double quotes (in practice they can also use
single quotes and/or be a sequence joined by \texttt{+}s).
We will use specific names for the \textit{data} to make clear what
the data is used for, such as
{\em math-token} (for a Metamath math token,
{\em latex-string} (for string to be placed in a \LaTeX\ stream),
{\em {\sc html}-code} (for {\sc html} code),
and {\em filename} (for a filename).

\subsubsection{Typesetting Comment - \LaTeX}

The syntax for a \LaTeX\ definition is:
\begin{center}
 \texttt{latexdef "}{\em math-token}\texttt{" as "}{\em latex-string}\texttt{";}
\end{center}
\index{latex definitions@\LaTeX\ definitions}%
\index{\texttt{latexdef} statement}

The {\em token-string} and {\em latex-string} are the data
(character strings) for
the token and the \LaTeX\ definition of the token, respectively,

These \LaTeX\ definitions are used by the Metamath program
when it is asked to product \LaTeX output using
the \texttt{write tex} command.

\subsubsection{Typesetting Comment - {\sc html}}

The key kinds of {\sc HTML} definitions have the following syntax:

\vskip 1ex
    \texttt{htmldef "}{\em math-token}\texttt{" as "}{\em
    {\sc html}-code}\texttt{";}\index{\texttt{htmldef} statement}
                    \ \ \ \ \ \ldots

    \texttt{althtmldef "}{\em math-token}\texttt{" as "}{\em
{\sc html}-code}\texttt{";}\index{\texttt{althtmldef} statement}

                    \ \ \ \ \ \ldots

Note that in {\sc HTML} there are two possible definitions for math tokens.
This feature is useful when
an alternate representation of symbols is desired, for example one that
uses Unicode entities and another uses {\sc gif} images.

There are many other typesetting definitions that can control {\sc HTML}.
These include:

\vskip 1ex

    \texttt{htmldef "}{\em math-token}\texttt{" as "}{\em {\sc
    html}-code}\texttt{";}

    \texttt{htmltitle "}{\em {\sc html}-code}\texttt{";}%
\index{\texttt{htmltitle} statement}

    \texttt{htmlhome "}{\em {\sc html}-code}\texttt{";}%
\index{\texttt{htmlhome} statement}

    \texttt{htmlvarcolor "}{\em {\sc html}-code}\texttt{";}%
\index{\texttt{htmlvarcolor} statement}

    \texttt{htmlbibliography "}{\em filename}\texttt{";}%
\index{\texttt{htmlbibliography} statement}

\vskip 1ex

\noindent The \texttt{htmltitle} is the {\sc html} code for a common
title, such as ``Metamath Proof Explorer.''  The \texttt{htmlhome} is
code for a link back to the home page.  The \texttt{htmlvarcolor} is
code for a color key that appears at the bottom of each proof.  The file
specified by {\em filename} is an {\sc html} file that is assumed to
have a \texttt{<A NAME=}\ldots\texttt{>} tag for each bibiographic
reference in the database comments.  For example, if
\texttt{[Monk]}\index{\texttt{\char`\[}\ldots\texttt{]} inside comments}
occurs in the comment for a theorem, then \texttt{<A NAME='Monk'>} must
be present in the file; if not, a warning message is given.

Associated with
\texttt{althtmldef}
are the statements
\vskip 1ex

    \texttt{htmldir "}{\em
      directoryname}\texttt{";}\index{\texttt{htmldir} statement}

    \texttt{althtmldir "}{\em
     directoryname}\texttt{";}\index{\texttt{althtmldir} statement}

\vskip 1ex
\noindent giving the directories of the {\sc gif} and Unicode versions
respectively; their purpose is to provide cross-linking between the
two versions in the generated web pages.

When two different types of pages need to be produced from a single
database, such as the Hilbert Space Explorer that extends the Metamath
Proof Explorer, ``extended'' variables may be declared in the
\texttt{\$t} comment:
\vskip 1ex

    \texttt{exthtmltitle "}{\em {\sc html}-code}\texttt{";}%
\index{\texttt{exthtmltitle} statement}

    \texttt{exthtmlhome "}{\em {\sc html}-code}\texttt{";}%
\index{\texttt{exthtmlhome} statement}

    \texttt{exthtmlbibliography "}{\em filename}\texttt{";}%
\index{\texttt{exthtmlbibliography} statement}

\vskip 1ex
\noindent When these are declared, you also must declare
\vskip 1ex

    \texttt{exthtmllabel "}{\em label}\texttt{";}%
\index{\texttt{exthtmllabel} statement}

\vskip 1ex \noindent that identifies the database statement where the
``extended'' section of the database starts (in our example, where the
Hilbert Space Explorer starts).  During the generation of web pages for
that starting statement and the statements after it, the {\sc html} code
assigned to \texttt{exthtmltitle} and \texttt{exthtmlhome} is used
instead of that assigned to \texttt{htmltitle} and \texttt{htmlhome},
respectively.

\begin{sloppy}
\subsection{Additional Information Com\-ment (\texttt{\$j})} \label{jcomment}
\end{sloppy}

The additional information comment, aka the
\texttt{\$j}\index{\texttt{\$j} comment}\index{additional information comment}
comment,
provides a way to add additional structured information that can
be optionally parsed by systems.

The additional information comment is parsed the same way as the
typesetting comment (\texttt{\$t}) (see section \ref{tcomment}).
That is,
the additional information comment begins with the token
\texttt{\$j} within a comment,
and continues until the comment close \texttt{\$)}.
Within an additional information comment is a sequence of one or more
commands of the form \texttt{command arg arg ... ;}
where each of the zero or more \texttt{arg} values
can be either a quoted string or a keyword.
Note that every command ends in an unquoted semicolon.
If a verifier is parsing an additional information comment, but
doesn't recognize a particular command, it must skip the command
by finding the end of the command (an unquoted semicolon).

A database may have 0 or more additional information comments.
Note, however, that a verifier may ignore these comments entirely or only
process certain commands in an additional information comment.
The \texttt{mmj2} verifier supports many commands in additional information
comments.
We encourage systems that process additional information comments
to coordinate so that they will use the same command for the same effect.

Examples of additional information comments with various commands
(from the \texttt{set.mm} database) are:

\begin{itemize}
   \item Define the syntax and logical typecodes,
     and declare that our grammar is
     unambiguous (verifiable using the KLR parser, with compositing depth 5).
\begin{verbatim}
  $( $j
    syntax 'wff';
    syntax '|-' as 'wff';
    unambiguous 'klr 5';
  $)
\end{verbatim}

   \item Register $\lnot$ and $\rightarrow$ as primitive expressions
           (lacking definitions).
\begin{verbatim}
  $( $j primitive 'wn' 'wi'; $)
\end{verbatim}

   \item There is a special justification for \texttt{df-bi}.
\begin{verbatim}
  $( $j justification 'bijust' for 'df-bi'; $)
\end{verbatim}

   \item Register $\leftrightarrow$ as an equality for its type (wff).
\begin{verbatim}
  $( $j
    equality 'wb' from 'biid' 'bicomi' 'bitri';
    definition 'dfbi1' for 'wb';
  $)
\end{verbatim}

   \item Theorem \texttt{notbii} is the congruence law for negation.
\begin{verbatim}
  $( $j congruence 'notbii'; $)
\end{verbatim}

   \item Add \texttt{setvar} as a typecode.
\begin{verbatim}
  $( $j syntax 'setvar'; $)
\end{verbatim}

   \item Register $=$ as an equality for its type (\texttt{class}).
\begin{verbatim}
  $( $j equality 'wceq' from 'eqid' 'eqcomi' 'eqtri'; $)
\end{verbatim}

\end{itemize}


\subsection{Including Other Files in a Metamath Source File} \label{include}
\index{\texttt{\$[} and \texttt{\$]} auxiliary keywords}

The keywords \texttt{\$[} and \texttt{\$]} specify a file to be
included\index{included file}\index{file inclusion} at that point in a
Metamath\index{Metamath} source file\index{source file}.  The syntax for
including a file is as follows:
\begin{center}
\texttt{\$[} {\em file-name} \texttt{\$]}
\end{center}

The {\em file-name} should be a single token\index{token} with the same syntax
as a math symbol (i.e., all 93 non-whitespace
printable characters other than \texttt{\$} are
allowed, subject to the file-naming limitations of your operating system).
Comments may appear between the \texttt{\$[} and \texttt{\$]} keywords.  Included
files may include other files, which may in turn include other files, and so
on.

For example, suppose you want to use the set theory database as the starting
point for your own theory.  The first line in your file could be
\begin{center}
\texttt{\$[ set.mm \$]}
\end{center} All of the information (axioms, theorems,
etc.) in \texttt{set.mm} and any files that {\em it} includes will become
available for you to reference in your file. This can help make your work more
modular. A drawback to including files is that if you change the name of a
symbol or the label of a statement, you must also remember to update any
references in any file that includes it.


The naming conventions for included files are the same as those of your
operating system.\footnote{On the Macintosh, prior to Mac OS X,
 a colon is used to separate disk
and folder names from your file name.  For example, {\em volume}\texttt{:}{\em
file-name} refers to the root directory, {\em volume}\texttt{:}{\em
folder-name}\texttt{:}{\em file-name} refers to a folder in root, and {\em
volume}\texttt{:}{\em folder-name}\texttt{:}\ldots\texttt{:}{\em file-name} refers to a
deeper folder.  A simple {\em file-name} refers to a file in the folder from
which you launch the Metamath application.  Under Mac OS X and later,
the Metamath program is run under the Terminal application, which
conforms to Unix naming conventions.}\index{Macintosh file
names}\index{file names!Macintosh}\label{includef} For compatibility among
operating systems, you should keep the file names as simple as possible.  A
good convention to use is {\em file}\texttt{.mm} where {\em file} is eight
characters or less, in lower case.

There is no limit to the nesting depth of included files.  One thing that you
should be aware of is that if two included files themselves include a common
third file, only the {\em first} reference to this common file will be read
in.  This allows you to include two or more files that build on a common
starting file without having to worry about label and symbol conflicts that
would occur if the common file were read in more than once.  (In fact, if a
file includes itself, the self-reference will be ignored, although of course
it would not make any sense to do that.)  This feature also means, however,
that if you try to include a common file in several inner blocks, the result
might not be what you expect, since only the first reference will be replaced
with the included file (unlike the include statement in most other computer
languages).  Thus you would normally include common files only in the
outermost block\index{outermost block}.

\subsection{Compressed Proof Format}\label{compressed1}\index{compressed
proof}\index{proof!compressed}

The proof notation presented in Section~\ref{proof} is called a
{\bf normal proof}\index{normal proof}\index{proof!normal} and in principle is
sufficient to express any proof.  However, proofs often contain steps and
subproofs that are identical.  This is particularly true in typical
Metamath\index{Metamath} applications, because Metamath requires that the math
symbol sequence (usually containing a formula) at each step be separately
constructed, that is, built up piece by piece. As a result, a lot of
repetition often results.  The {\bf compressed proof} format allows Metamath
to take advantage of this redundancy to shorten proofs.

The specification for the compressed proof format is given in
Appen\-dix~\ref{compressed}.

Normally you need not concern yourself with the details of the compressed
proof format, since the Metamath program will allow you to convert from
the normal format to the compressed format with ease, and will also
automatically convert from the compressed format when proofs are displayed.
The overall structure of the compressed format is as follows:
\begin{center}
  \texttt{\$= ( } {\em label-list} \texttt{) } {\em compressed-proof\ }\ \texttt{\$.}
\end{center}
\index{\texttt{\$=} keyword}
The first \texttt{(} serves as a flag to Metamath that a compressed proof
follows.  The {\em label-list} includes all statements referred to by the
proof except the mandatory hypotheses\index{mandatory hypothesis}.  The {\em
compressed-proof} is a compact encoding of the proof, using upper-case
letters, and can be thought of as a large integer in base 26.  White
space\index{white space} inside a {\em compressed-proof} is
optional and is ignored.

It is important to note that the order of the mandatory hypotheses of
the statement being proved must not be changed if the compressed proof
format is used, otherwise the proof will become incorrect.  The reason
for this is that the mandatory hypotheses are not mentioned explicitly
in the compressed proof in order to make the compression more efficient.
If you wish to change the order of mandatory hypotheses, you must first
convert the proof back to normal format using the \texttt{save proof
{\em statement} /normal}\index{\texttt{save proof} command} command.
Later, you can go back to compressed format with \texttt{save proof {\em
statement} /compressed}.

During error checking with the \texttt{verify proof} command, an error
found in a compressed proof may point to a character in {\em
compressed-proof}, which may not be very meaningful to you.  In this
case, try to \texttt{save proof /normal} first, then do the
\texttt{verify proof} again.  In general, it is best to make sure a
proof is correct before saving it in compressed format, because severe
errors are less likely to be recoverable than in normal format.

\subsection{Specifying Unknown Proofs or Subproofs}\label{unknown}

In a proof under development, any step or subproof that is not yet known
may be represented with a single \texttt{?}.  For the purposes of
parsing the proof, the \texttt{?}\ \index{\texttt{]}@\texttt{?}\ inside
proofs} will push a single entry onto the RPN stack just as if it were a
hypothesis.  While developing a proof with the Proof
Assistant\index{Proof Assistant}, a partially developed proof may be
saved with the \texttt{save new{\char`\_}proof}\index{\texttt{save
new{\char`\_}proof} command} command, and \texttt{?}'s will be placed at
the appropriate places.

All \texttt{\$p}\index{\texttt{\$p} statement} statements must have
proofs, even if they are entirely unknown.  Before creating a proof with
the Proof Assistant, you should specify a completely unknown proof as
follows:
\begin{center}
  {\em label} \texttt{\$p} {\em statement} \texttt{\$= ?\ \$.}
\end{center}
\index{\texttt{\$=} keyword}
\index{\texttt{]}@\texttt{?}\ inside proofs}

The \texttt{verify proof}\index{\texttt{verify proof} command} command
will check the known portions of a partial proof for errors, but will
warn you that the statement has not been proved.

Note that partially developed proofs may be saved in compressed format
if desired.  In this case, you will see one or more \texttt{?}'s in the
{\em compressed-proof} part.\index{compressed
proof}\index{proof!compressed}

\section{Axioms vs.\ Definitions}\label{definitions}

The \textit{basic}
Metamath\index{Metamath} language and program
make no distinction\index{axiom vs.\
definition} between axioms\index{axiom} and
definitions.\index{definition} The \texttt{\$a}\index{\texttt{\$a}
statement} statement is used for both.  At first, this may seem
puzzling.  In the minds of many mathematicians, the distinction is
clear, even obvious, and hardly worth discussing.  A definition is
considered to be merely an abbreviation that can be replaced by the
expression for which it stands; although unless one actually does this,
to be precise then one should say that a theorem\index{theorem} is a
consequence of the axioms {\em and} the definitions that are used in the
formulation of the theorem \cite[p.~20]{Behnke}.\index{Behnke, H.}

\subsection{What is a Definition?}

What is a definition?  In its simplest form, a definition introduces a new
symbol and provides an unambiguous rule to transform an expression containing
the new symbol to one without it.  The concept of a ``proper
definition''\index{proper definition}\index{definition!proper} (as opposed to
a creative definition)\index{creative definition}\index{definition!creative}
that is usually agreed upon is (1) the definition should not strengthen the
language and (2) any symbols introduced by the definition should be eliminable
from the language \cite{Nemesszeghy}\index{Nemesszeghy, E. Z.}.  In other
words, they are mere typographical conveniences that do not belong to the
system and are theoretically superfluous.  This may seem obvious, but in fact
the nature of definitions can be subtle, sometimes requiring difficult
metatheorems to establish that they are not creative.

A more conservative stance was taken by logician S.
Le\'{s}niewski.\index{Le\'{s}niewski, S.}
\begin{quote}
Le\'{s}niewski
regards definitions as theses of the system.  In this respect they do
not differ either from the axioms or from theorems, i.e.\ from the
theses added to the system on the basis of the rule of substitution or
the rule of detachment [modus ponens].  Once definitions have been
accepted as theses of the system, it becomes necessary to consider them
as true propositions in the same sense in which axioms are true
\cite{Lejewski}.
\end{quote}\index{Lejewski, Czeslaw}

Let us look at some simple examples of definitions in propositional
calculus.  Consider the definition of logical {\sc or}
(disjunction):\index{disjunction ($\vee$)} ``$P\vee Q$ denotes $\neg P
\rightarrow Q$ (not $P$ implies $Q$).''  It is very easy to recognize a
statement making use of this definition, because it introduces the new
symbol $\vee$ that did not previously exist in the language.  It is easy
to see that no new theorems of the original language will result from
this definition.

Next, consider a definition that eliminates parentheses:  ``$P
\rightarrow Q\rightarrow R$ denotes $P\rightarrow (Q \rightarrow R)$.''
This is more subtle, because no new symbols are introduced.  The reason
this definition is considered proper is that no new symbol sequences
that are valid wffs (well-formed formulas)\index{well-formed formula
(wff)} in the original language will result from the definition, since
``$P \rightarrow Q\rightarrow R$'' is not a wff in the original
language.  Here, we implicitly make use of the fact that there is a
decision procedure that allows us to determine whether or not a symbol
sequence is a wff, and this fact allows us to use symbol sequences that
are not wffs to represent other things (such as wffs) by means of the
definition.  However, to justify the definition as not being creative we
need to prove that ``$P \rightarrow Q\rightarrow R$'' is in fact not a
wff in the original language, and this is more difficult than in the
case where we simply introduce a new symbol.

%Now let's take this reasoning to an extreme.  Propositional calculus is a
%decidable theory,\footnote{This means that a mechanical algorithm exists to
%determine whether or not a wff is a theorem.} so in principle we could make use
%of symbol sequences that are not theorems to represent other things (say, to
%encode actual theorems in a more compact way).  For example, let us extend the
%language by defining a wff ``$P$'' in the extended language as the theorem
%``$P\rightarrow P$''\footnote{This is one of the first theorems proved in the
%Metamath database \texttt{set.mm}.}\index{set
%theory database (\texttt{set.mm})} in the original language whenever ``$P$'' is
%not a theorem in the original language.  In the extended language, any wff
%``$Q$'' thus represents a theorem; to find out what theorem (in the original
%language) ``$Q$'' represents, we determine whether ``$Q$'' is a theorem in the
%original language (before the definition was introduced).  If so, we're done; if
%not, we replace ``$Q$'' by ``$Q\rightarrow Q$'' to eliminate the definition.
%This definition is therefore eliminable, and it does not ``strengthen'' the
%language because any wff that is not a theorem is not in the set of statements
%provable in the original language and thus is available for use by definitions.
%
%Of course, a definition such as this would render practically useless the
%communication of theorems of propositional calculus; but
%this is just a human shortcoming, since we can't always easily discern what is
%and is not a theorem by inspection.  In fact, the extended theory with this
%definition has no more and no less information than the original theory; it just
%expresses certain theorems of the form ``$P\rightarrow P$''
%in a more compact way.
%
%The point here is that what constitutes a proper definition is a matter of
%judgment about whether a symbol sequence can easily be recognized by a human
%as invalid in some sense (for example, not a wff); if so, the symbol sequence
%can be appropriated for use by a definition in order to make the extended
%language more compact.  Metamath\index{Metamath} lacks the ability to make this
%judgment, since as far as Metamath is concerned the definition of a wff, for
%example, is arbitrary.  You define for Metamath how wffs\index{well-formed
%formula (wff)} are constructed according to your own preferred style.  The
%concept of a wff may not even exist in a given formal system\index{formal
%system}.  Metamath treats all definitions as if they were new axioms, and it
%is up to the human mathematician to judge whether the definition is ``proper''
%'\index{proper definition}\index{definition!proper} in some agreed-upon way.

What constitutes a definition\index{definition} versus\index{axiom vs.\
definition} an axiom\index{axiom} is sometimes arbitrary in mathematical
literature.  For example, the connectives $\vee$ ({\sc or}), $\wedge$
({\sc and}), and $\leftrightarrow$ (equivalent to) in propositional
calculus are usually considered defined symbols that can be used as
abbreviations for expressions containing the ``primitive'' connectives
$\rightarrow$ and $\neg$.  This is the way we treat them in the standard
logic and set theory database \texttt{set.mm}\index{set theory database
(\texttt{set.mm})}.  However, the first three connectives can also be
considered ``primitive,'' and axiom systems have been devised that treat
all of them as such.  For example,
\cite[p.~35]{Goodstein}\index{Goodstein, R. L.} presents one with 15
axioms, some of which in fact coincide with what we have chosen to call
definitions in \texttt{set.mm}.  In certain subsets of classical
propositional calculus, such as the intuitionist
fragment\index{intuitionism}, it can be shown that one cannot make do
with just $\rightarrow$ and $\neg$ but must treat additional connectives
as primitive in order for the system to make sense.\footnote{Two nice
systems that make the transition from intuitionistic and other weak
fragments to classical logic just by adding axioms are given in
\cite{Robinsont}\index{Robinson, T. Thacher}.}

\subsection{The Approach to Definitions in \texttt{set.mm}}

In set theory, recursive definitions define a newly introduced symbol in
terms of itself.
The justification of recursive definitions, using
several ``recursion theorems,'' is usually one of the first
sophisticated proofs a student encounters when learning set theory, and
there is a significant amount of implicit metalogic behind a recursive
definition even though the definition itself is typically simple to
state.

Metamath itself has no built-in technical limitation that prevents
multiple-part recursive definitions in the traditional textbook style.
However, because the recursive definition requires advanced metalogic
to justify, eliminating a recursive definition is very difficult and
often not even shown in textbooks.

\subsubsection{Direct definitions instead of recursive definitions}

It is, however, possible to substitute one kind of complexity
for another.  We can eliminate the need for metalogical justification by
defining the operation directly with an explicit (but complicated)
expression, then deriving the recursive definition directly as a
theorem, using a recursion theorem ``in reverse.''
The elimination
of a direct definition is a matter of simple mechanical substitution.
We do this in
\texttt{set.mm}, as follows.

In \texttt{set.mm} our goal was to introduce almost all definitions in
the form of two expressions connected by either $\leftrightarrow$ or
$=$, where the thing being defined does not appear on the right hand
side.  Quine calls this form ``a genuine or direct definition'' \cite[p.
174]{Quine}\index{Quine, Willard Van Orman}, which makes the definitions
very easy to eliminate and the metalogic\index{metalogic} needed to
justify them as simple as possible.
Put another way, we had a goal of being able to
eliminate all definitions with direct mechanical substitution and to
verify easily the soundness of the definitions.

\subsubsection{Example of direct definitions}

We achieved this goal in almost all cases in \texttt{set.mm}.
Sometimes this makes the definitions more complex and less
intuitive.
For example, the traditional way to define addition of
natural numbers is to define an operation called {\em
successor}\index{successor} (which means ``plus one'' and is denoted by
``${\rm suc}$''), then define addition recursively\index{recursive
definition} with the two definitions $n + 0 = n$ and $m + {\rm suc}\,n =
{\rm suc} (m + n)$.  Although this definition seems simple and obvious,
the method to eliminate the definition is not obvious:  in the second
part of the definition, addition is defined in terms of itself.  By
eliminating the definition, we don't mean repeatedly applying it to
specific $m$ and $n$ but rather showing the explicit, closed-form
set-theoretical expression that $m + n$ represents, that will work for
any $m$ and $n$ and that does not have a $+$ sign on its right-hand
side.  For a recursive definition like this not to be circular
(creative), there are some hidden, underlying assumptions we must make,
for example that the natural numbers have a certain kind of order.

In \texttt{set.mm} we chose to start with the direct (though complex and
nonintuitive) definition then derive from it the standard recursive
definition.
For example, the closed-form definition used in \texttt{set.mm}
for the addition operation on ordinals\index{ordinal
addition}\index{addition!of ordinals} (of which natural numbers are a
subset) is

\setbox\startprefix=\hbox{\tt \ \ df-oadd\ \$a\ }
\setbox\contprefix=\hbox{\tt \ \ \ \ \ \ \ \ \ \ \ \ \ }
\startm
\m{\vdash}\m{+_o}\m{=}\m{(}\m{x}\m{\in}\m{{\rm On}}\m{,}\m{y}\m{\in}\m{{\rm
On}}\m{\mapsto}\m{(}\m{{\rm rec}}\m{(}\m{(}\m{z}\m{\in}\m{{\rm
V}}\m{\mapsto}\m{{\rm suc}}\m{z}\m{)}\m{,}\m{x}\m{)}\m{`}\m{y}\m{)}\m{)}
\endm
\noindent which depends on ${\rm rec}$.

\subsubsection{Recursion operators}

The above definition of \texttt{df-oadd} depends on the definition of
${\rm rec}$, a ``recursion operator''\index{recursion operator} with
the definition \texttt{df-rdg}:

\setbox\startprefix=\hbox{\tt \ \ df-rdg\ \$a\ }
\setbox\contprefix=\hbox{\tt \ \ \ \ \ \ \ \ \ \ \ \ }
\startm
\m{\vdash}\m{{\rm
rec}}\m{(}\m{F}\m{,}\m{I}\m{)}\m{=}\m{\mathrm{recs}}\m{(}\m{(}\m{g}\m{\in}\m{{\rm
V}}\m{\mapsto}\m{{\rm if}}\m{(}\m{g}\m{=}\m{\varnothing}\m{,}\m{I}\m{,}\m{{\rm
if}}\m{(}\m{{\rm Lim}}\m{{\rm dom}}\m{g}\m{,}\m{\bigcup}\m{{\rm
ran}}\m{g}\m{,}\m{(}\m{F}\m{`}\m{(}\m{g}\m{`}\m{\bigcup}\m{{\rm
dom}}\m{g}\m{)}\m{)}\m{)}\m{)}\m{)}\m{)}
\endm

\noindent which can be further broken down with definitions shown in
Section~\ref{setdefinitions}.

This definition of ${\rm rec}$
defines a recursive definition generator on ${\rm On}$ (the class of ordinal
numbers) with characteristic function $F$ and initial value $I$.
This operation allows us to define, with
compact direct definitions, functions that are usually defined in
textbooks with recursive definitions.
The price paid with our approach
is the complexity of our ${\rm rec}$ operation
(especially when {\tt df-recs} that it is built on is also eliminated).
But once we get past this hurdle, definitions that would otherwise be
recursive become relatively simple, as in for example {\tt oav}, from
which we prove the recursive textbook definition as theorems {\tt oa0}, {\tt
oasuc}, and {\tt oalim} (with the help of theorems {\tt rdg0}, {\tt rdgsuc},
and {\tt rdglim2a}).  We can also restrict the ${\rm rec}$ operation to
define otherwise recursive functions on the natural numbers $\omega$; see {\tt
fr0g} and {\tt frsuc}.  Our ${\rm rec}$ operation apparently does not appear
in published literature, although closely related is Definition 25.2 of
[Quine] p. 177, which he uses to ``turn...a recursion into a genuine or
direct definition" (p. 174).  Note that the ${\rm if}$ operations (see
{\tt df-if}) select cases based on whether the domain of $g$ is zero, a
successor, or a limit ordinal.

An important use of this definition ${\rm rec}$ is in the recursive sequence
generator {\tt df-seq} on the natural numbers (as a subset of the
complex infinite sequences such as the factorial function {\tt df-fac} and
integer powers {\tt df-exp}).

The definition of ${\rm rec}$ depends on ${\rm recs}$.
New direct usage of the more powerful (and more primitive) ${\rm recs}$
construct is discouraged, but it is available when needed.
This
defines a function $\mathrm{recs} ( F )$ on ${\rm On}$, the class of ordinal
numbers, by transfinite recursion given a rule $F$ which sets the next
value given all values so far.
Unlike {\tt df-rdg} which restricts the
update rule to use only the previous value, this version allows the
update rule to use all previous values, which is why it is described
as ``strong,'' although it is actually more primitive.  See {\tt
recsfnon} and {\tt recsval} for the primary contract of this definition.
It is defined as:

\setbox\startprefix=\hbox{\tt \ \ df-recs\ \$a\ }
\setbox\contprefix=\hbox{\tt \ \ \ \ \ \ \ \ \ \ \ \ \ }
\startm
\m{\vdash}\m{\mathrm{recs}}\m{(}\m{F}\m{)}\m{=}\m{\bigcup}\m{\{}\m{f}\m{|}\m{\exists}\m{x}\m{\in}\m{{\rm
On}}\m{(}\m{f}\m{{\rm
Fn}}\m{x}\m{\wedge}\m{\forall}\m{y}\m{\in}\m{x}\m{(}\m{f}\m{`}\m{y}\m{)}\m{=}\m{(}\m{F}\m{`}\m{(}\m{f}\m{\restriction}\m{y}\m{)}\m{)}\m{)}\m{\}}
\endm

\subsubsection{Closing comments on direct definitions}

From these direct definitions the simpler, more
intuitive recursive definition is derived as a set of theorems.\index{natural
number}\index{addition}\index{recursive definition}\index{ordinal addition}
The end result is the same, but we completely eliminate the rather complex
metalogic that justifies the recursive definition.

Recursive definitions are often considered more efficient and intuitive than
direct ones once the metalogic has been learned or possibly just accepted as
correct.  However, it was felt that direct definition in \texttt{set.mm}
maximizes rigor by minimizing metalogic.  It can be eliminated effortlessly,
something that is difficult to do with a recursive definition.

Again, Metamath itself has no built-in technical limitation that prevents
multiple-part recursive definitions in the traditional textbook style.
Instead, our goal is to eliminate all definitions with
direct mechanical substitution and to verify easily the soundness of
definitions.

\subsection{Adding Constraints on Definitions}

The basic Metamath language and the Metamath program do
not have any built-in constraints on definitions, since they are just
\$a statements.

However, nothing prevents a verification system from verifying
additional rules to impose further limitations on definitions.
For example, the \texttt{mmj2}\index{mmj2} program
supports various kinds of
additional information comments (see section \ref{jcomment}).
One of their uses is to optionally verify additional constraints,
including constraints to verify that definitions meet certain
requirements.
These additional checks are required by the
continuous integration (CI)\index{continuous integration (CI)}
checks of the
\texttt{set.mm}\index{set theory database (\texttt{set.mm})}%
\index{Metamath Proof Explorer}
database.
This approach enables us to optionally impose additional requirements
on definitions if we wish, without necessarily imposing those rules on
all databases or requiring all verification systems to implement them.
In addition, this allows us to impose specialized constraints tailored
to one database while not requiring other systems to implement
those specialized constraints.

We impose two constraints on the
\texttt{set.mm}\index{set theory database (\texttt{set.mm})}%
\index{Metamath Proof Explorer} database
via the \texttt{mmj2}\index{mmj2} program that are worth discussing here:
a parse check and a definition soundness check.

% On February 11, 2019 8:32:32 PM EST, saueran@oregonstate.edu wrote:
% The following addition to the end of set.mm is accepted by the mmj2
% parser and definition checker and the metamath verifier(at least it was
% when I checked, you should check it too), and creates a contradiction by
% proving the theorem |- ph.
% ${
% wleftp $a wff ( ( ph ) $.
% wbothp $a wff ( ph ) $.
% df-leftp $a |- ( ( ( ph ) <-> -. ph ) $.
% df-bothp $a |- ( ( ph ) <-> ph ) $.
% anything $p |- ph $=
%   ( wbothp wn wi wleftp df-leftp biimpi df-bothp mpbir mpbi simplim ax-mp)
%   ABZAMACZDZCZMOEZOCQAEZNDZRNAFGSHIOFJMNKLAHJ $.
% $}
%
% This particular problem is countered by enabling, within mmj2,
% SetParser,mmj.verify.LRParser

First,
we enable a parse check in \texttt{mmj2} (through its
\texttt{SetParser} command) that requires that all new definitions
must \textit{not} create an ambiguous parse for a KLR(5) parser.
This prevents some errors such as definitions with imbalanced parentheses.

Second, we run a definition soundness check specific to
\texttt{set.mm} or databases similar to it.
(through the \texttt{definitionCheck} macro).
Some \texttt{\$a} statements (including all ax-* statemnets)
are excluded from these checks, as they will
always fail this simple check,
but they are appropriate for most definitions.
This check imposes a set of additional rules:

\begin{enumerate}

\item New definitions must be introduced using $=$ or $\leftrightarrow$.

\item No \texttt{\$a} statement introduced before this one is allowed to use the
symbol being defined in this definition, and the definition is not
allowed to use itself (except once, in the definiendum).

\item Every variable in the definiens should not be distinct

\item Every dummy variable in the definiendum
are required to be distinct from each other and from variables in
the definiendum.
To determine this, the system will look for a "justification" theorem
in the database, and if it is not there, attempt to briefly prove
$( \varphi \rightarrow \forall x \varphi )$  for each dummy variable x.

\item Every dummy variable should be a set variable,
unless there is a justification theorem available.

\item Every dummy variable must be bound
(if the system cannot determine this a justification theorem must be
provided).

\end{enumerate}

\subsection{Summary of Approach to Definitions}

In short, when being rigorous it turns out that
definitions can be subtle, sometimes requiring difficult
metatheorems to establish that they are not creative.

Instead of building such complications into the Metamath language itself,
the basic Metmath language and program simply treat traditional
axioms and definitions as the same kind of \texttt{\$a} statement.
We have then built various tools to enable people to
verify additional conditions as their creators believe is appropriate
for those specific databases, without complicating the Metamath foundations.

\chapter{The Metamath Program}\label{commands}

This chapter provides a reference manual for the
Metamath program.\index{Metamath!commands}

Current instructions for obtaining and installing the Metamath program
can be found at the \url{http://metamath.org} web site.
For Windows, there is a pre-compiled version called
\texttt{metamath.exe}.  For Unix, Linux, and Mac OS X
(which we will refer to collectively as ``Unix''), the Metamath program
can be compiled from its source code with the command
\begin{verbatim}
gcc *.c -o metamath
\end{verbatim}
using the \texttt{gcc} {\sc c} compiler available on those systems.

In the command syntax descriptions below, fields enclosed in square
brackets [\ ] are optional.  File names may be optionally enclosed in
single or double quotes.  This is useful if the file name contains
spaces or
slashes (\texttt{/}), such as in Unix path names, \index{Unix file
names}\index{file names!Unix} that might be confused with Metamath
command qualifiers.\index{Unix file names}\index{file names!Unix}

\section{Invoking Metamath}

Unix, Linux, and Mac OS X
have a command-line interface called the {\em
bash shell}.  (In Mac OS X, select the Terminal application from
Applications/Utilities.)  To invoke Metamath from the bash shell prompt,
assuming that the Metamath program is in the current directory, type
\begin{verbatim}
bash$ ./metamath
\end{verbatim}

To invoke Metamath from a Windows DOS or Command Prompt, assuming that
the Metamath program is in the current directory (or in a directory
included in the Path system environment variable), type
\begin{verbatim}
C:\metamath>metamath
\end{verbatim}

To use command-line arguments at invocation, the command-line arguments
should be a list of Metamath commands, surrounded by quotes if they
contain spaces.  In Windows, the surrounding quotes must be double (not
single) quotes.  For example, to read the database file \texttt{set.mm},
verify all proofs, and exit the program, type (under Unix)
\begin{verbatim}
bash$ ./metamath 'read set.mm' 'verify proof *' exit
\end{verbatim}
Note that in Unix, any directory path with \texttt{/}'s must be
surrounded by quotes so Metamath will not interpret the \texttt{/} as a
command qualifier.  So if \texttt{set.mm} is in the \texttt{/tmp}
directory, use for the above example
\begin{verbatim}
bash$ ./metamath 'read "/tmp/set.mm"' 'verify proof *' exit
\end{verbatim}

For convenience, if the command-line has one argument and no spaces in
the argument, the command is implicitly assumed to be \texttt{read}.  In
this one special case, \texttt{/}'s are not interpreted as command
qualifiers, so you don't need quotes around a Unix file name.  Thus
\begin{verbatim}
bash$ ./metamath /tmp/set.mm
\end{verbatim}
and
\begin{verbatim}
bash$ ./metamath "read '/tmp/set.mm'"
\end{verbatim}
are equivalent.


\section{Controlling Metamath}

The Metamath program was first developed on a {\sc vax/vms} system, and
some aspects of its command line behavior reflect this heritage.
Hopefully you will find it reasonably user-friendly once you get used to
it.

Each command line is a sequence of English-like words separated by
spaces, as in \texttt{show settings}.  Command words are not case
sensitive, and only as many letters are needed as are necessary to
eliminate ambiguity; for example, \texttt{sh se} would work for the
command \texttt{show settings}.  In some cases arguments such as file
names, statement labels, or symbol names are required; these are
case-sensitive (although file names may not be on some operating
systems).

A command line is entered by typing it in then pressing the {\em return}
({\em enter}) key.  To find out what commands are available, type
\texttt{?} at the \texttt{MM>} prompt.  To find out the choices at any
point in a command, press {\em return} and you will be prompted for
them.  The default choice (the one selected if you just press {\em
return}) is shown in brackets (\texttt{<>}).

You may also type \texttt{?} in place of a command word to force
Metamath to tell you what the choices are.  The \texttt{?} method won't
work, though, if a non-keyword argument such as a file name is expected
at that point, because the program will think that \texttt{?} is the
value of the argument.

Some commands have one or more optional qualifiers which modify the
behavior of the command.  Qualifiers are preceded by a slash
(\texttt{/}), such as in \texttt{read set.mm / verify}.  Spaces are
optional around the \texttt{/}.  If you need to use a space or
slash in a command
argument, as in a Unix file name, put single or double quotes around the
command argument.

The \texttt{open log} command will save everything you see on the
screen and is useful to help you recover should something go wrong in a
proof, or if you want to document a bug.

If a command responds with more than a screenful, you will be
prompted to \texttt{<return> to continue, Q to quit, or S to scroll to
end}.  \texttt{Q} or \texttt{q} (not case-sensitive) will complete the
command internally but will suppress further output until the next
\texttt{MM>} prompt.  \texttt{s} will suppress further pausing until the
next \texttt{MM>} prompt.  After the first screen, you are also
presented with the choice of \texttt{b} to go back a screenful.  Note
that \texttt{b} may also be entered at the \texttt{MM>} prompt
immediately after a command to scroll back through the output of that
command.

A command line enclosed in quotes is executed by your operating system.
See Section~\ref{oscmd}.

{\em Warning:} Pressing {\sc ctrl-c} will abort the Metamath program
unconditionally.  This means any unsaved work will be lost.


\subsection{\texttt{exit} Command}\index{\texttt{exit} command}

Syntax:  \texttt{exit} [\texttt{/force}]

This command exits from Metamath.  If there have been changes to the
source with the \texttt{save proof} or \texttt{save new{\char`\_}proof}
commands, you will be given an opportunity to \texttt{write source} to
permanently save the changes.

In Proof Assistant\index{Proof Assistant} mode, the \texttt{exit} command will
return to the \verb/MM>/ prompt. If there were changes to the proof, you will
be given an opportunity to \texttt{save new{\char`\_}proof}.

The \texttt{quit} command is a synonym for \texttt{exit}.

Optional qualifier:
    \texttt{/force} - Do not prompt if changes were not saved.  This qualifier is
        useful in \texttt{submit} command files (Section~\ref{sbmt})
        to ensure predictable behavior.





\subsection{\texttt{open log} Command}\index{\texttt{open log} command}
Syntax:  \texttt{open log} {\em file-name}

This command will open a log file that will store everything you see on
the screen.  It is useful to help recovery from a mistake in a long Proof
Assistant session, or to document bugs.\index{Metamath!bugs}

The log file can be closed with \texttt{close log}.  It will automatically be
closed upon exiting Metamath.



\subsection{\texttt{close log} Command}\index{\texttt{close log} command}
Syntax:  \texttt{close log}

The \texttt{close log} command closes a log file if one is open.  See
also \texttt{open log}.




\subsection{\texttt{submit} Command}\index{\texttt{submit} command}\label{sbmt}
Syntax:  \texttt{submit} {\em filename}

This command causes further command lines to be taken from the specified
file.  Note that any line beginning with an exclamation point (\texttt{!}) is
treated as a comment (i.e.\ ignored).  Also note that the scrolling
of the screen output is continuous, so you may want to open a log file
(see \texttt{open log}) to record the results that fly by on the screen.
After the lines in the file are exhausted, Metamath returns to its
normal user interface mode.

The \texttt{submit} command can be called recursively (i.e. from inside
of a \texttt{submit} command file).


Optional command qualifier:

    \texttt{/silent} - suppresses the screen output but still
        records the output in a log file if one is open.


\subsection{\texttt{erase} Command}\index{\texttt{erase} command}
Syntax:  \texttt{erase}

This command will reset Metamath to its starting state, deleting any
data\-base that was \texttt{read} in.
 If there have been changes to the
source with the \texttt{save proof} or \texttt{save new{\char`\_}proof}
commands, you will be given an opportunity to \texttt{write source} to
permanently save the changes.



\subsection{\texttt{set echo} Command}\index{\texttt{set echo} command}
Syntax:  \texttt{set echo on} or \texttt{set echo off}

The \texttt{set echo on} command will cause command lines to be echoed with any
abbreviations expanded.  While learning the Metamath commands, this
feature will show you the exact command that your abbreviated input
corresponds to.



\subsection{\texttt{set scroll} Command}\index{\texttt{set scroll} command}
Syntax:  \texttt{set scroll prompted} or \texttt{set scroll continuous}

The Metamath command line interface starts off in the \texttt{prompted} mode,
which means that you will be prompted to continue or quit after each
full screen in a long listing.  In \texttt{continuous} mode, long listings will be
scrolled without pausing.

% LaTeX bug? (1) \texttt{\_} puts out different character than
% \texttt{{\char`\_}}
%  = \verb$_$  (2) \texttt{{\char`\_}} puts out garbage in \subsection
%  argument
\subsection{\texttt{set width} Command}\index{\texttt{set
width} command}
Syntax:  \texttt{set width} {\em number}

Metamath assumes the width of your screen is 79 characters (chosen
because the Command Prompt in Windows XP has a wrapping bug at column
80).  If your screen is wider or narrower, this command allows you to
change this default screen width.  A larger width is advantageous for
logging proofs to an output file to be printed on a wide printer.  A
smaller width may be necessary on some terminals; in this case, the
wrapping of the information messages may sometimes seem somewhat
unnatural, however.  In \LaTeX\index{latex@{\LaTeX}!characters per line},
there is normally a maximum of 61 characters per line with typewriter
font.  (The examples in this book were produced with 61 characters per
line.)

\subsection{\texttt{set height} Command}\index{\texttt{set
height} command}
Syntax:  \texttt{set height} {\em number}

Metamath assumes your screen height is 24 lines of characters.  If your
screen is taller or shorter, this command lets you to change the number
of lines at which the display pauses and prompts you to continue.

\subsection{\texttt{beep} Command}\index{\texttt{beep} command}

Syntax:  \texttt{beep}

This command will produce a beep.  By typing it ahead after a
long-running command has started, it will alert you that the command is
finished.  For convenience, \texttt{b} is an abbreviation for
\texttt{beep}.

Note:  If \texttt{b} is typed at the \texttt{MM>} prompt immediately
after the end of a multiple-page display paged with ``\texttt{Press
<return> for more}...'' prompts, then the \texttt{b} will back up to the
previous page rather than perform the \texttt{beep} command.
In that case you must type the unabbreviated \texttt{beep} form
of the command.

\subsection{\texttt{more} Command}\index{\texttt{more} command}

Syntax:  \texttt{more} {\em filename}

This command will display the contents of an {\sc ascii} file on your
screen.  (This command is provided for convenience but is not very
powerful.  See Section~\ref{oscmd} to invoke your operating system's
command to do this, such as the \texttt{more} command in Unix.)

\subsection{Operating System Commands}\index{operating system
command}\label{oscmd}

A line enclosed in single or double quotes will be executed by your
computer's operating system if it has a command line interface.  For
example, on a {\sc vax/vms} system,
\verb/MM> 'dir'/
will print disk directory contents.  Note that this feature will not
work on the Macintosh prior to Mac OS X, which does not have a command
line interface.

For your convenience, the trailing quote is optional.

\subsection{Size Limitations in Metamath}

In general, there are no fixed, predefined limits\index{Metamath!memory
limits} on how many labels, tokens\index{token}, statements, etc.\ that
you may have in a database file.  The Metamath program uses 32-bit
variables (64-bit on 64-bit CPUs) as indices for almost all internal
arrays, which are allocated dynamically as needed.



\section{Reading and Writing Files}

The following commands create new files:  the \texttt{open} commands;
the \texttt{write} commands; the \texttt{/html},
\texttt{/alt{\char`\_}html}, \texttt{/brief{\char`\_}html},
\texttt{/brief{\char`\_}alt{\char`\_}html} qualifiers of \texttt{show
statement}, and \texttt{midi}.  The following commands append to files
previously opened:  the \texttt{/tex} qualifier of \texttt{show proof}
and \texttt{show new{\char`\_}proof}; the \texttt{/tex} and
\texttt{/simple{\char`\_}tex} qualifiers of \texttt{show statement}; the
\texttt{close} commands; and all screen dialog between \texttt{open log}
and \texttt{close log}.

The commands that create new files will not overwrite an existing {\em
filename} but will rename the existing one to {\em
filename}\texttt{{\char`\~}1}.  An existing {\em
filename}\texttt{{\char`\~}1} is renamed {\em
filename}\texttt{{\char`\~}2}, etc.\ up to {\em
filename}\texttt{{\char`\~}9}.  An existing {\em
filename}\texttt{{\char`\~}9} is deleted.  This makes recovery from
mistakes easier but also will clutter up your directory, so occasionally
you may want to clean up (delete) these \texttt{{\char`\~}}$n$ files.


\subsection{\texttt{read} Command}\index{\texttt{read} command}
Syntax:  \texttt{read} {\em file-name} [\texttt{/verify}]

This command will read in a Metamath language source file and any included
files.  Normally it will be the first thing you do when entering Metamath.
Statement syntax is checked, but proof syntax is not checked.
Note that the file name may be enclosed in single or double quotes;
this is useful if the file name contains slashes, as might be the case
under Unix.

If you are getting an ``\texttt{?Expected VERIFY}'' error
when trying to read a Unix file name with slashes, you probably haven't
quoted it.\index{Unix file names}\index{file names!Unix}

If you are prompted for the file name (by pressing {\em return}
 after \texttt{read})
you should {\em not} put quotes around it, even if it is a Unix file name
with slashes.

Optional command qualifier:

    \texttt{/verify} - Verify all proofs as the database is read in.  This
         qualifier will slow down reading in the file.  See \texttt{verify
         proof} for more information on file error-checking.

See also \texttt{erase}.



\subsection{\texttt{write source} Command}\index{\texttt{write source} command}
Syntax:  \texttt{write source} {\em filename}
[\texttt{/rewrap}]
[\texttt{/split}]
% TeX doesn't handle this long line with tt text very well,
% so force a line break here.
[\texttt{/keep\_includes}] {\\}
[\texttt{/no\_versioning}]

This command will write the contents of a Metamath\index{database}
database into a file.\index{source file}

Optional command qualifiers:

\texttt{/rewrap} -
Reformats statements and comments according to the
convention used in the set.mm database.
It unwraps the
lines in the comment before each \texttt{\$a} and \texttt{\$p} statement,
then it
rewraps the line.  You should compare the output to the original
to make sure that the desired effect results; if not, go back to
the original source.  The wrapped line length honors the
\texttt{set width}
parameter currently in effect.  Note:  Text
enclosed in \texttt{<HTML>}...\texttt{</HTML>} tags is not modified by the
\texttt{/rewrap} qualifier.
Proofs themselves are not reformatted;
use \texttt{save proof * / compressed} to do that.
An isolated tilde (\~{}) is always kept on the same line as the following
symbol, so you can find all comment references to a symbol by
searching for \~{} followed by a space and that symbol
(this is useful for finding cross-references).
Incidentally, \texttt{save proof} also honors the \texttt{set width}
parameter currently in effect.

\texttt{/split} - Files included in the source using the expression
\$[ \textit{inclfile} \$] will be
written into separate files instead of being included in a single output
file.  The name of each separately written included file will be the
\textit{inclfile} argument of its inclusion command.

\texttt{/keep\_includes} - If a source file has includes but is written as a
single file by omitting \texttt{/split}, by default the included files will
be deleted (actually just renamed with a \char`\~1 suffix unless
\texttt{/no\_versioning} is specified) to prevent the possibly confusing
source duplication in both the output file and the included file.
The \texttt{/keep\_includes} qualifier will prevent this deletion.

\texttt{/no\_versioning} - Backup files suffixed with \char`\~1 are not created.


\section{Showing Status and Statements}



\subsection{\texttt{show settings} Command}\index{\texttt{show settings} command}
Syntax:  \texttt{show settings}

This command shows the state of various parameters.

\subsection{\texttt{show memory} Command}\index{\texttt{show memory} command}
Syntax:  \texttt{show memory}

This command shows the available memory left.  It is not meaningful
on most modern operating systems,
which have virtual memory.\index{Metamath!memory usage}


\subsection{\texttt{show labels} Command}\index{\texttt{show labels} command}
Syntax:  \texttt{show labels} {\em label-match} [\texttt{/all}]
   [\texttt{/linear}]

This command shows the labels of \texttt{\$a} and \texttt{\$p}
statements that match {\em label-match}.  A \verb$*$ in {label-match}
matches zero or more characters.  For example, \verb$*abc*def$ will match all
labels containing \verb$abc$ and ending with \verb$def$.

Optional command qualifiers:

   \texttt{/all} - Include matches for \texttt{\$e} and \texttt{\$f}
   statement labels.

   \texttt{/linear} - Display only one label per line.  This can be useful for
       building scripts in conjunction with the utilities under the
       \texttt{tools}\index{\texttt{tools} command} command.



\subsection{\texttt{show statement} Command}\index{\texttt{show statement} command}
Syntax:  \texttt{show statement} {\em label-match} [{\em qualifiers} (see below)]

This command provides information about a statement.  Only statements
that have labels (\texttt{\$f}\index{\texttt{\$f} statement},
\texttt{\$e}\index{\texttt{\$e} statement},
\texttt{\$a}\index{\texttt{\$a} statement}, and
\texttt{\$p}\index{\texttt{\$p} statement}) may be specified.
If {\em label-match}
contains wildcard (\verb$*$) characters, all matching statements will be
displayed in the order they occur in the database.

Optional qualifiers (only one qualifier at a time is allowed):

    \texttt{/comment} - This qualifier includes the comment that immediately
       precedes the statement.

    \texttt{/full} - Show complete information about each statement,
       and show all
       statements matching {\em label} (including \texttt{\$e}
       and \texttt{\$f} statements).

    \texttt{/tex} - This qualifier will write the statement information to the
       \LaTeX\ file previously opened with \texttt{open tex}.  See
       Section~\ref{texout}.

    \texttt{/simple{\char`\_}tex} - The same as \texttt{/tex}, except that
       \LaTeX\ macros are not used for formatting equations, allowing easier
       manual edits of the output for slide presentations, etc.

    \texttt{/html}\index{html generation@{\sc html} generation},
       \texttt{/alt{\char`\_}html}, \texttt{/brief{\char`\_}html},
       \texttt{/brief{\char`\_}alt{\char`\_}html} -
       These qualifiers invoke a special mode of
       \texttt{show statement} that
       creates a web page for the statement.  They may not be used with
       any other qualifier.  See Section~\ref{htmlout} or
       \texttt{help html} in the program.


\subsection{\texttt{search} Command}\index{\texttt{search} command}
Syntax:  search {\em label-match}
\texttt{"}{\em symbol-match}{\tt}" [\texttt{/all}] [\texttt{/comments}]
[\texttt{/join}]

This command searches all \texttt{\$a} and \texttt{\$p} statements
matching {\em label-match} for occurrences of {\em symbol-match}.  A
\verb@*@ in {\em label-match} matches any label character.  A \verb@$*@
in {\em symbol-match} matches any sequence of symbols.  The symbols in
{\em symbol-match} must be separated by white space.  The quotes
surrounding {\em symbol-match} may be single or double quotes.  For
example, \texttt{search b}\verb@* "-> $* ch"@ will list all statements
whose labels begin with \texttt{b} and contain the symbols \verb@->@ and
\texttt{ch} surrounding any symbol sequence (including no symbol
sequence).  The wildcards \texttt{?} and \texttt{\$?} are also available
to match individual characters in labels and symbols respectively; see
\texttt{help search} in the Metamath program for details on their usage.

Optional command qualifiers:

    \texttt{/all} - Also search \texttt{\$e} and \texttt{\$f} statements.

    \texttt{/comments} - Search the comment that immediately precedes each
        label-matched statement for {\em symbol-match}.  In this case
        {\em symbol-match} is an arbitrary, non-case-sensitive character
        string.  Quotes around {\em symbol-match} are optional if there
        is no ambiguity.

    \texttt{/join} - In the case of a \texttt{\$a} or \texttt{\$p} statement,
	prepend its \texttt{\$e}
	hypotheses for searching. The
	\texttt{/join} qualifier has no effect in \texttt{/comments} mode.

\section{Displaying and Verifying Proofs}


\subsection{\texttt{show proof} Command}\index{\texttt{show proof} command}
Syntax:  \texttt{show proof} {\em label-match} [{\em qualifiers} (see below)]

This command displays the proof of the specified
\texttt{\$p}\index{\texttt{\$p} statement} statement in various formats.
The {\em label-match} may contain wildcard (\verb@$*@) characters to match
multiple statements.  Without any qualifiers, only the logical steps
will be shown (i.e.\ syntax construction steps will be omitted), in an
indented format.

Most of the time, you will use
    \texttt{show proof} {\em label}
to see just the proof steps corresponding to logical inferences.

Optional command qualifiers:

    \texttt{/essential} - The proof tree is trimmed of all
        \texttt{\$f}\index{\texttt{\$f} statement} hypotheses before
        being displayed.  (This is the default, and it is redundant to
        specify it.)

    \texttt{/all} - the proof tree is not trimmed of all \texttt{\$f} hypotheses before
        being displayed.  \texttt{/essential} and \texttt{/all} are mutually exclusive.

    \texttt{/from{\char`\_}step} {\em step} - The display starts at the specified
        step.  If
        this qualifier is omitted, the display starts at the first step.

    \texttt{/to{\char`\_}step} {\em step} - The display ends at the specified
        step.  If this
        qualifier is omitted, the display ends at the last step.

    \texttt{/tree{\char`\_}depth} {\em number} - Only
         steps at less than the specified proof
        tree depth are displayed.  Sometimes useful for obtaining an overview of
        the proof.

    \texttt{/reverse} - The steps are displayed in reverse order.

    \texttt{/renumber} - When used with \texttt{/essential}, the steps are renumbered
        to correspond only to the essential steps.

    \texttt{/tex} - The proof is converted to \LaTeX\ and\index{latex@{\LaTeX}}
        stored in the file opened
        with \texttt{open tex}.  See Section~\ref{texout} or
        \texttt{help tex} in the program.

    \texttt{/lemmon} - The proof is displayed in a non-indented format known
        as Lemmon style, with explicit previous step number references.
        If this qualifier is omitted, steps are indented in a tree format.

    \texttt{/start{\char`\_}column} {\em number} - Overrides the default column
        (16)
        at which the formula display starts in a Lemmon-style display.  May be
        used only in conjunction with \texttt{/lemmon}.

    \texttt{/normal} - The proof is displayed in normal format suitable for
        inclusion in a Metamath source file.  May not be used with any other
        qualifier.

    \texttt{/compressed} - The proof is displayed in compressed format
        suitable for inclusion in a Metamath source file.  May not be used with
        any other qualifier.

    \texttt{/statement{\char`\_}summary} - Summarizes all statements (like a
        brief \texttt{show statement})
        used by the proof.  It may not be used with any other qualifier
        except \texttt{/essential}.

    \texttt{/detailed{\char`\_}step} {\em step} - Shows the details of what is
        happening at
        a specific proof step.  May not be used with any other qualifier.
        The {\em step} is the step number shown when displaying a
        proof without the \texttt{/renumber} qualifier.


\subsection{\texttt{show usage} Command}\index{\texttt{show usage} command}
Syntax:  \texttt{show usage} {\em label-match} [\texttt{/recursive}]

This command lists the statements whose proofs make direct reference to
the statement specified.

Optional command qualifier:

    \texttt{/recursive} - Also include statements whose proofs ultimately
        depend on the statement specified.



\subsection{\texttt{show trace\_back} Command}\index{\texttt{show
       trace{\char`\_}back} command}
Syntax:  \texttt{show trace{\char`\_}back} {\em label-match} [\texttt{/essential}] [\texttt{/axioms}]
    [\texttt{/tree}] [\texttt{/depth} {\em number}]

This command lists all statements that the proof of the \texttt{\$p}
statement(s) specified by {\em label-match} depends on.

Optional command qualifiers:

    \texttt{/essential} - Restrict the trace-back to \texttt{\$e}
        \index{\texttt{\$e} statement} hypotheses of proof trees.

    \texttt{/axioms} - List only the axioms that the proof ultimately depends on.

    \texttt{/tree} - Display the trace-back in an indented tree format.

    \texttt{/depth} {\em number} - Restrict the \texttt{/tree} trace-back to the
        specified indentation depth.

    \texttt{/count{\char`\_}steps} - Count the number of steps the proof has
       all the way back to axioms.  If \texttt{/essential} is specified,
       expansions of variable-type hypotheses (syntax constructions) are not counted.

\subsection{\texttt{verify proof} Command}\index{\texttt{verify proof} command}
Syntax:  \texttt{verify proof} {\em label-match} [\texttt{/syntax{\char`\_}only}]

This command verifies the proofs of the specified statements.  {\em
label-match} may contain wild card characters (\texttt{*}) to verify
more than one proof; for example \verb/*abc*def/ will match all labels
containing \texttt{abc} and ending with \texttt{def}.
The command \texttt{verify proof *} will verify all proofs in the database.

Optional command qualifier:

    \texttt{/syntax{\char`\_}only} - This qualifier will perform a check of syntax
        and RPN
        stack violations only.  It will not verify that the proof is
        correct.  This qualifier is useful for quickly determining which
        proofs are incomplete (i.e.\ are under development and have \texttt{?}'s
        in them).

{\em Note:} \texttt{read}, followed by \texttt{verify proof *}, will ensure
 the database is free
from errors in the Metamath language but will not check the markup notation
in comments.
You can also check the markup notation by running \texttt{verify markup *}
as discussed in Section~\ref{verifymarkup}; see also the discussion
on generating {\sc HTML} in Section~\ref{htmlout}.

\subsection{\texttt{verify markup} Command}\index{\texttt{verify markup} command}\label{verifymarkup}
Syntax:  \texttt{verify markup} {\em label-match}
[\texttt{/date{\char`\_}skip}]
[\texttt{/top{\char`\_}date{\char`\_}skip}] {\\}
[\texttt{/file{\char`\_}skip}]
[\texttt{/verbose}]

This command checks comment markup and other informal conventions we have
adopted.  It error-checks the latexdef, htmldef, and althtmldef statements
in the \texttt{\$t} statement of a Metamath source file.\index{error checking}
It error-checks any \texttt{`}...\texttt{`},
\texttt{\char`\~}~\textit{label},
and bibliographic markups in statement descriptions.
It checks that
\texttt{\$p} and \texttt{\$a} statements
have the same content when their labels start with
``ax'' and ``ax-'' respectively but are otherwise identical, for example
ax4 and ax-4.
It also verifies the date consistency of ``(Contributed by...),''
``(Revised by...),'' and ``(Proof shortened by...)'' tags in the comment
above each \texttt{\$a} and \texttt{\$p} statement.

Optional command qualifiers:

    \texttt{/date{\char`\_}skip} - This qualifier will
        skip date consistency checking,
        which is usually not required for databases other than
	\texttt{set.mm}.

    \texttt{/top{\char`\_}date{\char`\_}skip} - This qualifier will check date consistency except
        that the version date at the top of the database file will not
        be checked.  Only one of
        \texttt{/date{\char`\_}skip} and
        \texttt{/top{\char`\_}date{\char`\_}skip} may be
        specified.

    \texttt{/file{\char`\_}skip} - This qualifier will skip checks that require
        external files to be present, such as checking GIF existence and
        bibliographic links to mmset.html or equivalent.  It is useful
        for doing a quick check from a directory without these files.

    \texttt{/verbose} - Provides more information.  Currently it provides a list
        of axXXX vs. ax-XXX matches.

\subsection{\texttt{save proof} Command}\index{\texttt{save proof} command}
Syntax:  \texttt{save proof} {\em label-match} [\texttt{/normal}]
   [\texttt{/compressed}]

The \texttt{save proof} command will reformat a proof in one of two formats and
replace the existing proof in the source buffer\index{source
buffer}.  It is useful for
converting between proof formats.  Note that a proof will not be
permanently saved until a \texttt{write source} command is issued.

Optional command qualifiers:

    \texttt{/normal} - The proof is saved in the normal format (i.e., as a
        sequence
        of labels, which is the defined format of the basic Metamath
        language).\index{basic language}  This is the default format that
        is used if a qualifier
        is omitted.

    \texttt{/compressed} - The proof is saved in the compressed format which
        reduces storage requirements for a database.
        See Appendix~\ref{compressed}.




\section{Creating Proofs}\label{pfcommands}\index{Proof Assistant}

Before using the Proof Assistant, you must add a \texttt{\$p} to your
source file (using a text editor) containing the statement you want to
prove.  Its proof should consist of a single \texttt{?}, meaning
``unknown step.''  Example:
\begin{verbatim}
equid $p x = x $= ? $.
\end{verbatim}

To enter the Proof assistant, type \texttt{prove} {\em label}, e.g.
\texttt{prove equid}.  Metamath will respond with the \texttt{MM-PA>}
prompt.

Proofs are created working backwards from the statement being proved,
primarily using a series of \texttt{assign} commands.  A proof is
complete when all steps are assigned to statements and all steps are
unified and completely known.  During the creation of a proof, Metamath
will allow only operations that are legal based on what is known up to
that point.  For example, it will not allow an \texttt{assign} of a
statement that cannot be unified with the unknown proof step being
assigned.

{\em Important:}
The Proof Assistant is
{\em not} a tool to help you discover proofs.  It is just a tool to help
you add them to the database.  For a tutorial read
Section~\ref{frstprf}.
To practice using the Proof Assistant, you may
want to \texttt{prove} an existing theorem, then delete all steps with
\texttt{delete all}, then re-create it with the Proof Assistant while
looking at its proof display (before deletion).
You might want to figure out your first few proofs completely
and write them down by hand, before using the Proof Assistant, though
not everyone finds that effective.

{\em Important:}
The \texttt{undo} command if very helpful when entering a proof, because
it allows you to undo a previously-entered step.
In addition, we suggest that you
keep track of your work with a log file (\texttt{open
log}) and save it frequently (\texttt{save new{\char`\_}proof},
\texttt{write source}).
You can use \texttt{delete} to reverse an \texttt{assign}.
You can also do \texttt{delete floating{\char`\_}hypotheses}, then
\texttt{initialize all}, then \texttt{unify all /interactive} to
reinitialize bad unifications made accidentally or by bad
\texttt{assign}s.  You cannot reverse a \texttt{delete} except by
a relevant \texttt{undo} or using
\texttt{exit /force} then reentering the Proof Assistant to recover from
the last \texttt{save new{\char`\_}proof}.

The following commands are available in the Proof Assistant (at the
\texttt{MM-PA>} prompt) to help you create your proof.  See the
individual commands for more detail.

\begin{itemize}
\item[]
    \texttt{show new{\char`\_}proof} [\texttt{/all},...] - Displays the
        proof in progress.  You will use this command a lot; see \texttt{help
        show new{\char`\_}proof} to become familiar with its qualifiers.  The
        qualifiers \texttt{/unknown} and \texttt{/not{\char`\_}unified} are
        useful for seeing the work remaining to be done.  The combination
        \texttt{/all/unknown} is useful for identifying dummy variables that must be
        assigned, or attempts to use illegal syntax, when \texttt{improve all}
        is unable to complete the syntax constructions.  Unknown variables are
        shown as \texttt{\$1}, \texttt{\$2},...
\item[]
    \texttt{assign} {\em step} {\em label} - Assigns an unknown {\em step}
        number with the statement
        specified by {\em label}.
\item[]
    \texttt{let variable} {\em variable}
        \texttt{= "}{\em symbol sequence}\texttt{"}
          - Forces a symbol
        sequence to replace an unknown variable (such as \texttt{\$1}) in a proof.
        It is useful
        for helping difficult unifications, and it is necessary when you have
        dummy variables that eventually must be assigned a name.
\item[]
    \texttt{let step} {\em step} \texttt{= "}{\em symbol sequence}\texttt{"} -
          Forces a symbol sequence
        to replace the contents of a proof step, provided it can be
        unified with the existing step contents.  (I rarely use this.)
\item[]
    \texttt{unify step} {\em step} (or \texttt{unify all}) - Unifies
        the source and target of
        a step.  If you specify a specific step, you will be prompted
        to select among the unifications that are possible.  If you
        specify \texttt{all}, all steps with unique unifications, but only
        those steps, will be
        unified.  \texttt{unify all /interactive} goes through all non-unified
        steps.
\item[]
    \texttt{initialize} {\em step} (or \texttt{all}) - De-unifies the target and source of
        a step (or all steps), as well as the hypotheses of the source,
        and makes all variables in the source unknown.  Useful to recover from
        an \texttt{assign} or \texttt{let} mistake that
        resulted in incorrect unifications.
\item[]
    \texttt{delete} {\em step} (or \texttt{all} or \texttt{floating{\char`\_}hypotheses}) -
        Deletes the specified
        step(s).  \texttt{delete floating{\char`\_}hypotheses}, then \texttt{initialize all}, then
        \texttt{unify all /interactive} is useful for recovering from mistakes
        where incorrect unifications assigned wrong math symbol strings to
        variables.
\item[]
    \texttt{improve} {\em step} (or \texttt{all}) -
      Automatically creates a proof for steps (with no unknown variables)
      whose proof requires no statements with \texttt{\$e} hypotheses.  Useful
      for filling in proofs of \texttt{\$f} hypotheses.  The \texttt{/depth}
      qualifier will also try statements whose \texttt{\$e} hypotheses contain
      no new variables.  {\em Warning:} Save your work (with \texttt{save
      new{\char`\_}proof} then \texttt{write source}) before using
      \texttt{/depth = 2} or greater, since the search time grows
      exponentially and may never terminate in a reasonable time, and you
      cannot interrupt the search.  I have found that it is rare for
      \texttt{/depth = 3} or greater to be useful.
 \item[]
    \texttt{save new{\char`\_}proof} - Saves the proof in progress in the program's
        internal database buffer.  To save it permanently into the database file,
        use \texttt{write source} after
        \texttt{save new{\char`\_}proof}.  To revert to the last
        \texttt{save new{\char`\_}proof},
        \texttt{exit /force} from the Proof Assistant then re-enter the Proof
        Assistant.
 \item[]
    \texttt{match step} {\em step} (or \texttt{match all}) - Shows what
        statements are
        possibilities for the \texttt{assign} statement. (This command
        is not very
        useful in its present form and hopefully will be improved
        eventually.  In the meantime, use the \texttt{search} statement for
        candidates matching specific math token combinations.)
 \item[]
 \texttt{minimize{\char`\_}with}\index{\texttt{minimize{\char`\_}with} command}
% 3/10/07 Note: line-breaking the above results in duplicate index entries
     - After a proof is complete, this command will attempt
        to match other database theorems to the proof to see if the proof
        size can be reduced as a result.  See \texttt{help
        minimize{\char`\_}with} in the
        Metamath program for its usage.
 \item[]
 \texttt{undo}\index{\texttt{undo} command}
    - Undo the effect of a proof-changing command (all but the
      \texttt{show} and \texttt{save} commands above).
 \item[]
 \texttt{redo}\index{\texttt{redo} command}
    - Reverse the previous \texttt{undo}.
\end{itemize}

The following commands set parameters that may be relevant to your proof.
Consult the individual \texttt{help set}... commands.
\begin{itemize}
   \item[] \texttt{set unification{\char`\_}timeout}
 \item[]
    \texttt{set search{\char`\_}limit}
  \item[]
    \texttt{set empty{\char`\_}substitution} - note that default is \texttt{off}
\end{itemize}

Type \texttt{exit} to exit the \texttt{MM-PA>}
 prompt and get back to the \texttt{MM>} prompt.
Another \texttt{exit} will then get you out of Metamath.



\subsection{\texttt{prove} Command}\index{\texttt{prove} command}
Syntax:  \texttt{prove} {\em label}

This command will enter the Proof Assistant\index{Proof Assistant}, which will
allow you to create or edit the proof of the specified statement.
The command-line prompt will change from \texttt{MM>} to \texttt{MM-PA>}.

Note:  In the present version (0.177) of
Metamath\index{Metamath!limitations of version 0.177}, the Proof
Assistant does not verify that \texttt{\$d}\index{\texttt{\$d}
statement} restrictions are met as a proof is being built.  After you
have completed a proof, you should type \texttt{save new{\char`\_}proof}
followed by \texttt{verify proof} {\em label} (where {\em label} is the
statement you are proving with the \texttt{prove} command) to verify the
\texttt{\$d} restrictions.

See also: \texttt{exit}

\subsection{\texttt{set unification\_timeout} Command}\index{\texttt{set
unification{\char`\_}timeout} command}
Syntax:  \texttt{set unification{\char`\_}timeout} {\em number}

(This command is available outside the Proof Assistant but affects the
Proof Assistant\index{Proof Assistant} only.)

Sometimes the Proof Assistant will inform you that a unification
time-out occurred.  This may happen when you try to \texttt{unify}
formulas with many temporary variables\index{temporary variable}
(\texttt{\$1}, \texttt{\$2}, etc.), since the time to compute all possible
unifications may grow exponentially with the number of variables.  If
you want Metamath to try harder (and you're willing to wait longer) you
may increase this parameter.  \texttt{show settings} will show you the
current value.

\subsection{\texttt{set empty\_substitution} Command}\index{\texttt{set
empty{\char`\_}substitution} command}
% These long names can't break well in narrow mode, and even "sloppy"
% is not enough. Work around this by not demanding justification.
\begin{flushleft}
Syntax:  \texttt{set empty{\char`\_}substitution on} or \texttt{set
empty{\char`\_}substitution off}
\end{flushleft}

(This command is available outside the Proof Assistant but affects the
Proof Assistant\index{Proof Assistant} only.)

The Metamath language allows variables to be
substituted\index{substitution!variable}\index{variable substitution}
with empty symbol sequences\index{empty substitution}.  However, in many
formal systems\index{formal system} this will never happen in a valid
proof.  Allowing for this possibility increases the likelihood of
ambiguous unifications\index{ambiguous
unification}\index{unification!ambiguous} during proof creation.
The default is that
empty substitutions are not allowed; for formal systems requiring them,
you must \texttt{set empty{\char`\_}substitution on}.
(An example where this must be \texttt{on}
would be a system that implements a Deduction Rule and in
which deductions from empty assumption lists would be permissible.  The
MIU-system\index{MIU-system} described in Appendix~\ref{MIU} is another
example.)
Note that empty substitutions are
always permissible in proof verification (VERIFY PROOF...) outside the
Proof Assistant.  (See the MIU system in the Metamath book for an example
of a system needing empty substitutions; another example would be a
system that implements a Deduction Rule and in which deductions from
empty assumption lists would be permissible.)

It is better to leave this \texttt{off} when working with \texttt{set.mm}.
Note that this command does not affect the way proofs are verified with
the \texttt{verify proof} command.  Outside of the Proof Assistant,
substitution of empty sequences for math symbols is always allowed.

\subsection{\texttt{set search\_limit} Command}\index{\texttt{set
search{\char`\_}limit} command} Syntax:  \texttt{set search{\char`\_}limit} {\em
number}

(This command is available outside the Proof Assistant but affects the
Proof Assistant\index{Proof Assistant} only.)

This command sets a parameter that determines when the \texttt{improve} command
in Proof Assistant mode gives up.  If you want \texttt{improve} to search harder,
you may increase it.  The \texttt{show settings} command tells you its current
value.


\subsection{\texttt{show new\_proof} Command}\index{\texttt{show
new{\char`\_}proof} command}
Syntax:  \texttt{show new{\char`\_}proof} [{\em
qualifiers} (see below)]

This command (available only in Proof Assistant mode) displays the proof
in progress.  It is identical to the \texttt{show proof} command, except that
there is no statement argument (since it is the statement being proved) and
the following qualifiers are not available:

    \texttt{/statement{\char`\_}summary}

    \texttt{/detailed{\char`\_}step}

Also, the following additional qualifiers are available:

    \texttt{/unknown} - Shows only steps that have no statement assigned.

    \texttt{/not{\char`\_}unified} - Shows only steps that have not been unified.

Note that \texttt{/essential}, \texttt{/depth}, \texttt{/unknown}, and
\texttt{/not{\char`\_}unified} may be
used in any combination; each of them effectively filters out additional
steps from the proof display.

See also:  \texttt{show proof}






\subsection{\texttt{assign} Command}\index{\texttt{assign} command}
Syntax:   \texttt{assign} {\em step} {\em label} [\texttt{/no{\char`\_}unify}]

   and:   \texttt{assign first} {\em label}

   and:   \texttt{assign last} {\em label}


This command, available in the Proof Assistant only, assigns an unknown
step (one with \texttt{?} in the \texttt{show new{\char`\_}proof}
listing) with the statement specified by {\em label}.  The assignment
will not be allowed if the statement cannot be unified with the step.

If \texttt{last} is specified instead of {\em step} number, the last
step that is shown by \texttt{show new{\char`\_}proof /unknown} will be
used.  This can be useful for building a proof with a command file (see
\texttt{help submit}).  It also makes building proofs faster when you know
the assignment for the last step.

If \texttt{first} is specified instead of {\em step} number, the first
step that is shown by \texttt{show new{\char`\_}proof /unknown} will be
used.

If {\em step} is zero or negative, the -{\em step}th from last unknown
step, as shown by \texttt{show new{\char`\_}proof /unknown}, will be
used.  \texttt{assign -1} {\em label} will assign the penultimate
unknown step, \texttt{assign -2} {\em label} the antepenultimate, and
\texttt{assign 0} {\em label} is the same as \texttt{assign last} {\em
label}.

Optional command qualifier:

    \texttt{/no{\char`\_}unify} - do not prompt user to select a unification if there is
        more than one possibility.  This is useful for noninteractive
        command files.  Later, the user can \texttt{unify all /interactive}.
        (The assignment will still be automatically unified if there is only
        one possibility and will be refused if unification is not possible.)



\subsection{\texttt{match} Command}\index{\texttt{match} command}
Syntax:  \texttt{match step} {\em step} [\texttt{/max{\char`\_}essential{\char`\_}hyp}
{\em number}]

    and:  \texttt{match all} [\texttt{/essential}]
          [\texttt{/max{\char`\_}essential{\char`\_}hyp} {\em number}]

This command, available in the Proof Assistant only, shows what
statements can be unified with the specified step(s).  {\em Note:} In
its current form, this command is not very useful because of the large
number of matches it reports.
It may be enhanced in the future.  In the meantime, the \texttt{search}
command can often provide finer control over locating theorems of interest.

Optional command qualifiers:

    \texttt{/max{\char`\_}essential{\char`\_}hyp} {\em number} - filters out
        of the list any statements
        with more than the specified number of
        \texttt{\$e}\index{\texttt{\$e} statement} hypotheses.

    \texttt{/essential{\char`\_}only} - in the \texttt{match all} statement, only
        the steps that
        would be listed in the \texttt{show new{\char`\_}proof /essential} display are
        matched.



\subsection{\texttt{let} Command}\index{\texttt{let} command}
Syntax: \texttt{let variable} {\em variable} = \verb/"/{\em symbol-sequence}\verb/"/

  and: \texttt{let step} {\em step} = \verb/"/{\em symbol-sequence}\verb/"/

These commands, available in the Proof Assistant\index{Proof Assistant}
only, assign a temporary variable\index{temporary variable} or unknown
step with a specific symbol sequence.  They are useful in the middle of
creating a proof, when you know what should be in the proof step but the
unification algorithm doesn't yet have enough information to completely
specify the temporary variables.  A ``temporary variable'' is one that
has the form \texttt{\$}{\em nn} in the proof display, such as
\texttt{\$1}, \texttt{\$2}, etc.  The {\em symbol-sequence} may contain
other unknown variables if desired.  Examples:

    \verb/let variable $32 = "A = B"/

    \verb/let variable $32 = "A = $35"/

    \verb/let step 10 = '|- x = x'/

    \verb/let step -2 = "|- ( $7 = ph )"/

Any symbol sequence will be accepted for the \texttt{let variable}
command.  Only those symbol sequences that can be unified with the step
will be accepted for \texttt{let step}.

The \texttt{let} commands ``zap'' the proof with information that can
only be verified when the proof is built up further.  If you make an
error, the command sequence \texttt{delete
floating{\char`\_}hypotheses}, \texttt{initialize all}, and
\texttt{unify all /interactive} will undo a bad \texttt{let} assignment.

If {\em step} is zero or negative, the -{\em step}th from last unknown
step, as shown by \texttt{show new{\char`\_}proof /unknown}, will be
used.  The command \texttt{let step 0} = \verb/"/{\em
symbol-sequence}\verb/"/ will use the last unknown step, \texttt{let
step -1} = \verb/"/{\em symbol-sequence}\verb/"/ the penultimate, etc.
If {\em step} is positive, \texttt{let step} may be used to assign known
(in the sense of having previously been assigned a label with
\texttt{assign}) as well as unknown steps.

Either single or double quotes can surround the {\em symbol-sequence} as
long as they are different from any quotes inside a {\em
symbol-sequence}.  If {\em symbol-sequence} contains both kinds of
quotes, see the instructions at the end of \texttt{help let} in the
Metamath program.


\subsection{\texttt{unify} Command}\index{\texttt{unify} command}
Syntax:  \texttt{unify step} {\em step}

      and:   \texttt{unify all} [\texttt{/interactive}]

These commands, available in the Proof Assistant only, unify the source
and target of the specified step(s). If you specify a specific step, you
will be prompted to select among the unifications that are possible.  If
you specify \texttt{all}, only those steps with unique unifications will be
unified.

Optional command qualifier for \texttt{unify all}:

    \texttt{/interactive} - You will be prompted to select among the
        unifications
        that are possible for any steps that do not have unique
        unifications.  (Otherwise \texttt{unify all} will bypass these.)

See also \texttt{set unification{\char`\_}timeout}.  The default is
100000, but increasing it to 1000000 can help difficult cases.  Manually
assigning some or all of the unknown variables with the \texttt{let
variable} command also helps difficult cases.



\subsection{\texttt{initialize} Command}\index{\texttt{initialize} command}
Syntax:  \texttt{initialize step} {\em step}

    and: \texttt{initialize all}

These commands, available in the Proof Assistant\index{Proof Assistant}
only, ``de-unify'' the target and source of a step (or all steps), as
well as the hypotheses of the source, and makes all variables in the
source and the source's hypotheses unknown.  This command is useful to
help recover from incorrect unifications that resulted from an incorrect
\texttt{assign}, \texttt{let}, or unification choice.  Part or all of
the command sequence \texttt{delete floating{\char`\_}hypotheses},
\texttt{initialize all}, and \texttt{unify all /interactive} will recover
from incorrect unifications.

See also:  \texttt{unify} and \texttt{delete}



\subsection{\texttt{delete} Command}\index{\texttt{delete} command}
Syntax:  \texttt{delete step} {\em step}

   and:      \texttt{delete all} -- {\em Warning: dangerous!}

   and:      \texttt{delete floating{\char`\_}hypotheses}

These commands are available in the Proof Assistant only.  The
\texttt{delete step} command deletes the proof tree section that
branches off of the specified step and makes the step become unknown.
\texttt{delete all} is equivalent to \texttt{delete step} {\em step}
where {\em step} is the last step in the proof (i.e.\ the beginning of
the proof tree).

In most cases the \texttt{undo} command is the best way to undo
a previous step.
An alternative is to salvage your last \texttt{save
new{\char`\_}proof} by exiting and reentering the Proof Assistant.
For this to work, keep a log file open to record your work
and to do \texttt{save new{\char`\_}proof} frequently, especially before
\texttt{delete}.

\texttt{delete floating{\char`\_}hypotheses} will delete all sections of
the proof that branch off of \texttt{\$f}\index{\texttt{\$f} statement}
statements.  It is sometimes useful to do this before an
\texttt{initialize} command to recover from an error.  Note that once a
proof step with a \texttt{\$f} hypothesis as the target is completely
known, the \texttt{improve} command can usually fill in the proof for
that step.  Unlike the deletion of logical steps, \texttt{delete
floating{\char`\_}hypotheses} is a relatively safe command that is
usually easy to recover from.



\subsection{\texttt{improve} Command}\index{\texttt{improve} command}
\label{improve}
Syntax:  \texttt{improve} {\em step} [\texttt{/depth} {\em number}]
                                               [\texttt{/no{\char`\_}distinct}]

   and:   \texttt{improve first} [\texttt{/depth} {\em number}]
                                              [\texttt{/no{\char`\_}distinct}]

   and:   \texttt{improve last} [\texttt{/depth} {\em number}]
                                              [\texttt{/no{\char`\_}distinct}]

   and:   \texttt{improve all} [\texttt{/depth} {\em number}]
                                              [\texttt{/no{\char`\_}distinct}]

These commands, available in the Proof Assistant\index{Proof Assistant}
only, try to find proofs automatically for unknown steps whose symbol
sequences are completely known.  They are primarily useful for filling in
proofs of \texttt{\$f}\index{\texttt{\$f} statement} hypotheses.  The
search will be restricted to statements having no
\texttt{\$e}\index{\texttt{\$e} statement} hypotheses.

\begin{sloppypar} % narrow
Note:  If memory is limited, \texttt{improve all} on a large proof may
overflow memory.  If you use \texttt{set unification{\char`\_}timeout 1}
before \texttt{improve all}, there will usually be sufficient
improvement to easily recover and completely \texttt{improve} the proof
later on a larger computer.  Warning:  Once memory has overflowed, there
is no recovery.  If in doubt, save the intermediate proof (\texttt{save
new{\char`\_}proof} then \texttt{write source}) before \texttt{improve
all}.
\end{sloppypar}

If \texttt{last} is specified instead of {\em step} number, the last
step that is shown by \texttt{show new{\char`\_}proof /unknown} will be
used.

If \texttt{first} is specified instead of {\em step} number, the first
step that is shown by \texttt{show new{\char`\_}proof /unknown} will be
used.

If {\em step} is zero or negative, the -{\em step}th from last unknown
step, as shown by \texttt{show new{\char`\_}proof /unknown}, will be
used.  \texttt{improve -1} will use the penultimate
unknown step, \texttt{improve -2} {\em label} the antepenultimate, and
\texttt{improve 0} is the same as \texttt{improve last}.

Optional command qualifier:

    \texttt{/depth} {\em number} - This qualifier will cause the search
        to include
        statements with \texttt{\$e} hypotheses (but no new variables in
        the \texttt{\$e}
        hypotheses), provided that the backtracking has not exceeded the
        specified depth. {\em Warning:}  Try \texttt{/depth 1},
        then \texttt{2}, then \texttt{3}, etc.
        in sequence because of possible exponential blowups.  Save your
        work before trying \texttt{/depth} greater than \texttt{1}!

    \texttt{/no{\char`\_}distinct} - Skip trial statements that have
        \texttt{\$d}\index{\texttt{\$d} statement} requirements.
        This qualifier will prevent assignments that might violate \texttt{\$d}
        requirements but it also could miss possible legal assignments.

See also: \texttt{set search{\char`\_}limit}

\subsection{\texttt{save new\_proof} Command}\index{\texttt{save
new{\char`\_}proof} command}
Syntax:  \texttt{save new{\char`\_}proof} {\em label} [\texttt{/normal}]
   [\texttt{/compressed}]

The \texttt{save new{\char`\_}proof} command is available in the Proof
Assistant only.  It saves the proof in progress in the source
buffer\index{source buffer}.  \texttt{save new{\char`\_}proof} may be
used to save a completed proof, or it may be used to save a proof in
progress in order to work on it later.  If an incomplete proof is saved,
any user assignments with \texttt{let step} or \texttt{let variable}
will be lost, as will any ambiguous unifications\index{ambiguous
unification}\index{unification!ambiguous} that were resolved manually.
To help make recovery easier, it can be helpful to \texttt{improve all}
before \texttt{save new{\char`\_}proof} so that the incomplete proof
will have as much information as possible.

Note that the proof will not be permanently saved until a \texttt{write
source} command is issued.

Optional command qualifiers:

    \texttt{/normal} - The proof is saved in the normal format (i.e., as a
        sequence of labels, which is the defined format of the basic Metamath
        language).\index{basic language}  This is the default format that
        is used if a qualifier is omitted.

    \texttt{/compressed} - The proof is saved in the compressed format, which
        reduces storage requirements for a database.
        (See Appendix~\ref{compressed}.)


\section{Creating \LaTeX\ Output}\label{texout}\index{latex@{\LaTeX}}

You can generate \LaTeX\ output given the
information in a database.
The database must already include the necessary typesetting information
(see section \ref{tcomment} for how to provide this information).

The \texttt{show statement} and \texttt{show proof} commands each have a
special \texttt{/tex} command qualifier that produces \LaTeX\ output.
(The \texttt{show statement} command also has the
\texttt{/simple{\char`\_}tex} qualifier for output that is easier to
edit by hand.)  Before you can use them, you must open a \LaTeX\ file to
which to send their output.  A typical complete session will use this
sequence of Metamath commands:

\begin{verbatim}
read set.mm
open tex example.tex
show statement a1i /tex
show proof a1i /all/lemmon/renumber/tex
show statement uneq2 /tex
show proof uneq2 /all/lemmon/renumber/tex
close tex
\end{verbatim}

See Section~\ref{mathcomments} for information on comment markup and
Appendix~\ref{ASCII} for information on how math symbol translation is
specified.

To format and print the \LaTeX\ source, you will need the \LaTeX\
program, which is standard on most Linux installations and available for
Windows.  On Linux, in order to create a {\sc pdf} file, you will
typically type at the shell prompt
\begin{verbatim}
$ pdflatex example.tex
\end{verbatim}

\subsection{\texttt{open tex} Command}\index{\texttt{open tex} command}
Syntax:  \texttt{open tex} {\em file-name} [\texttt{/no{\char`\_}header}]

This command opens a file for writing \LaTeX\
source\index{latex@{\LaTeX}} and writes a \LaTeX\ header to the file.
\LaTeX\ source can be written with the \texttt{show proof}, \texttt{show
new{\char`\_}proof}, and \texttt{show statement} commands using the
\texttt{/tex} qualifier.

The mapping to \LaTeX\ symbols is defined in a special comment
containing a \texttt{\$t} token, described in Appendix~\ref{ASCII}.

There is an optional command qualifier:

    \texttt{/no{\char`\_}header} - This qualifier prevents a standard
        \LaTeX\ header and trailer
        from being included with the output \LaTeX\ code.


\subsection{\texttt{close tex} Command}\index{\texttt{close tex} command}
Syntax:  \texttt{close tex}

This command writes a trailer to any \LaTeX\ file\index{latex@{\LaTeX}}
that was opened with \texttt{open tex} (unless
\texttt{/no{\char`\_}header} was used with \texttt{open tex}) and closes
the \LaTeX\ file.


\section{Creating {\sc HTML} Output}\label{htmlout}

You can generate {\sc html} web pages given the
information in a database.
The database must already include the necessary typesetting information
(see section \ref{tcomment} for how to provide this information).
The ability to produce {\sc html} web pages was added in Metamath version
0.07.30.

To create an {\sc html} output file(s) for \texttt{\$a} or \texttt{\$p}
statement(s), use
\begin{quote}
    \texttt{show statement} {\em label-match} \texttt{/html}
\end{quote}
The output file will be named {\em label-match}\texttt{.html}
for each match.  When {\em
label-match} has wildcard (\texttt{*}) characters, all statements with
matching labels will have {\sc html} files produced for them.  Also,
when {\em label-match} has a wildcard (\texttt{*}) character, two additional
files, \texttt{mmdefinitions.html} and \texttt{mmascii.html} will be
produced.  To produce {\em only} these two additional files, you can use
\texttt{?*}, which will not match any statement label, in place of {\em
label-match}.

There are three other qualifiers for \texttt{show statement} that also
generate {\sc HTML} code.  These are \texttt{/alt{\char`\_}html},
\texttt{/brief{\char`\_}html}, and
\texttt{/brief{\char`\_}alt{\char`\_}html}, and are described in the
next section.

The command
\begin{quote}
    \texttt{show statement} {\em label-match} \texttt{/alt{\char`\_}html}
\end{quote}
does the same as \texttt{show statement} {\em label-match} \texttt{/html},
except that the {\sc html} code for the symbols is taken from
\texttt{althtmldef} statements instead of \texttt{htmldef} statements in
the \texttt{\$t} comment.

The command
\begin{verbatim}
show statement * /brief_html
\end{verbatim}
invokes a special mode that just produces definition and theorem lists
accompanied by their symbol strings, in a format suitable for copying and
pasting into another web page (such as the tutorial pages on the
Metamath web site).

Finally, the command
\begin{verbatim}
show statement * /brief_alt_html
\end{verbatim}
does the same as \texttt{show statement * / brief{\char`\_}html}
for the alternate {\sc html}
symbol representation.

A statement's comment can include a special notation that provides a
certain amount of control over the {\sc HTML} version of the comment.  See
Section~\ref{mathcomments} (p.~\pageref{mathcomments}) for the comment
markup features.

The \texttt{write theorem{\char`\_}list} and \texttt{write bibliography}
commands, which are described below, provide as a side effect complete
error checking for all of the features described in this
section.\index{error checking}

\subsection{\texttt{write theorem\_list}
Command}\index{\texttt{write theorem{\char`\_}list} command}
Syntax:  \texttt{write theorem{\char`\_}list}
[\texttt{/theorems{\char`\_}per{\char`\_}page} {\em number}]

This command writes a list of all of the \texttt{\$a} and \texttt{\$p}
statements in the database into a web page file
 called \texttt{mmtheorems.html}.
When additional files are needed, they are called
\texttt{mmtheorems2.html}, \texttt{mmtheorems3.html}, etc.

Optional command qualifier:

    \texttt{/theorems{\char`\_}per{\char`\_}page} {\em number} -
 This qualifier specifies the number of statements to
        write per web page.  The default is 100.

{\em Note:} In version 0.177\index{Metamath!limitations of version
0.177} of Metamath, the ``Nearby theorems'' links on the individual
web pages presuppose 100 theorems per page when linking to the theorem
list pages.  Therefore the \texttt{/theorems{\char`\_}per{\char`\_}page}
qualifier, if it specifies a number other than 100, will cause the
individual web pages to be out of sync and should not be used to
generate the main theorem list for the web site.  This may be
fixed in a future version.


\subsection{\texttt{write bibliography}\label{wrbib}
Command}\index{\texttt{write bibliography} command}
Syntax:  \texttt{write bibliography} {\em filename}

This command reads an existing {\sc html} bibliographic cross-reference
file, normally called \texttt{mmbiblio.html}, and updates it per the
bibliographic links in the database comments.  The file is updated
between the {\sc html} comment lines \texttt{<!--
{\char`\#}START{\char`\#} -->} and \texttt{<!-- {\char`\#}END{\char`\#}
-->}.  The original input file is renamed to {\em
filename}\texttt{{\char`\~}1}.

A bibliographic reference is indicated with the reference name
in brackets, such as  \texttt{Theorem 3.1 of
[Monk] p.\ 22}.
See Section~\ref{htmlout} (p.~\pageref{htmlout}) for
syntax details.


\subsection{\texttt{write recent\_additions}
Command}\index{\texttt{write recent{\char`\_}additions} command}
Syntax:  \texttt{write recent{\char`\_}additions} {\em filename}
[\texttt{/limit} {\em number}]

This command reads an existing ``Recent Additions'' {\sc html} file,
normally called \texttt{mmrecent.html}, and updates it with the
descriptions of the most recently added theorems to the database.
 The file is updated between
the {\sc html} comment lines \texttt{<!-- {\char`\#}START{\char`\#} -->}
and \texttt{<!-- {\char`\#}END{\char`\#} -->}.  The original input file
is renamed to {\em filename}\texttt{{\char`\~}1}.

Optional command qualifier:

    \texttt{/limit} {\em number} -
 This qualifier specifies the number of most recent theorems to
   write to the output file.  The default is 100.


\section{Text File Utilities}

\subsection{\texttt{tools} Command}\index{\texttt{tools} command}
Syntax:  \texttt{tools}

This command invokes an easy-to-use, general purpose utility for
manipulating the contents of {\sc ascii} text files.  Upon typing
\texttt{tools}, the command-line prompt will change to \texttt{TOOLS>}
until you type \texttt{exit}.  The \texttt{tools} commands can be used
to perform simple, global edits on an input/output file,
such as making a character string substitution on each line, adding a
string to each line, and so on.  A typical use of this utility is
to build a \texttt{submit} input file to perform a common operation on a
list of statements obtained from \texttt{show label} or \texttt{show
usage}.

The actions of most of the \texttt{tools} commands can also be
performed with equivalent (and more powerful) Unix shell commands, and
some users may find those more efficient.  But for Windows users or
users not comfortable with Unix, \texttt{tools} provides an
easy-to-learn alternative that is adequate for most of the
script-building tasks needed to use the Metamath program effectively.

\subsection{\texttt{help} Command (in \texttt{tools})}
Syntax:  \texttt{help}

The \texttt{help} command lists the commands available in the
\texttt{tools} utility, along with a brief description.  Each command,
in turn, has its own help, such as \texttt{help add}.  As with
Metamath's \texttt{MM>} prompt, a complete command can be entered at
once, or just the command word can be typed, causing the program to
prompt for each argument.

\vskip 1ex
\noindent Line-by-line editing commands:

  \texttt{add} - Add a specified string to each line in a file.

  \texttt{clean} - Trim spaces and tabs on each line in a file; convert
         characters.

  \texttt{delete} - Delete a section of each line in a file.

  \texttt{insert} - Insert a string at a specified column in each line of
        a file.

  \texttt{substitute} - Make a simple substitution on each line of the file.

  \texttt{tag} - Like \texttt{add}, but restricted to a range of lines.

  \texttt{swap} - Swap the two halves of each line in a file.

\vskip 1ex
\noindent Other file-processing commands:

  \texttt{break} - Break up (tokenize) a file into a list of tokens (one per
        line).

  \texttt{build} - Build a file with multiple tokens per line from a list.

  \texttt{count} - Count the occurrences in a file of a specified string.

  \texttt{number} - Create a list of numbers.

  \texttt{parallel} - Put two files in parallel.

  \texttt{reverse} - Reverse the order of the lines in a file.

  \texttt{right} - Right-justify lines in a file (useful before sorting
         numbers).

%  \texttt{tag} - Tag edit updates in a program for revision control.

  \texttt{sort} - Sort the lines in a file with key starting at
         specified string.

  \texttt{match} - Extract lines containing (or not) a specified string.

  \texttt{unduplicate} - Eliminate duplicate occurrences of lines in a file.

  \texttt{duplicate} - Extract first occurrence of any line occurring
         more than

   \ \ \    once in a file, discarding lines occurring exactly once.

  \texttt{unique} - Extract lines occurring exactly once in a file.

  \texttt{type} (10 lines) - Display the first few lines in a file.
                  Similar to Unix \texttt{head}.

  \texttt{copy} - Similar to Unix \texttt{cat} but safe (same input
         and output file allowed).

  \texttt{submit} - Run a script containing \texttt{tools} commands.

\vskip 1ex

\noindent Note:
  \texttt{unduplicate}, \texttt{duplicate}, and \texttt{unique} also
 sort the lines as a side effect.


\subsection{Using \texttt{tools} to Build Metamath \texttt{submit}
Scripts}

The \texttt{break} command is typically used to break up a series of
statement labels, such as the output of Metamath's \texttt{show usage},
into one label per line.  The other \texttt{tools} commands can then be
used to add strings before and after each statement label to specify
commands to be performed on the statement.  The \texttt{parallel}
command is useful when a statement label must be mentioned more than
once on a line.

Very often a \texttt{submit} script for Metamath will require multiple
command lines for each statement being processed.  For example, you may
want to enter the Proof Assistant, \texttt{minimize{\char`\_}with} your
latest theorem, \texttt{save} the new proof, and \texttt{exit} the Proof
Assistant.  To accomplish this, you can build a file with these four
commands for each statement on a single line, separating each command
with a designated character such as \texttt{@}.  Then at the end you can
\texttt{substitute} each \texttt{@} with \texttt{{\char`\\}n} to break
up the lines into individual command lines (see \texttt{help
substitute}).


\subsection{Example of a \texttt{tools} Session}

To give you a quick feel for the \texttt{tools} utility, we show a
simple session where we create a file \texttt{n.txt} with 3 lines, add
strings before and after each line, and display the lines on the screen.
You can experiment with the various commands to gain experience with the
\texttt{tools} utility.

\begin{verbatim}
MM> tools
Entering the Text Tools utilities.
Type HELP for help, EXIT to exit.
TOOLS> number
Output file <n.tmp>? n.txt
First number <1>?
Last number <10>? 3
Increment <1>?
TOOLS> add
Input/output file? n.txt
String to add to beginning of each line <>? This is line
String to add to end of each line <>? .
The file n.txt has 3 lines; 3 were changed.
First change is on line 1:
This is line 1.
TOOLS> type n.txt
This is line 1.
This is line 2.
This is line 3.
TOOLS> exit
Exiting the Text Tools.
Type EXIT again to exit Metamath.
MM>
\end{verbatim}



\appendix
\chapter{Sample Representations}
\label{ASCII}

This Appendix provides a sample of {\sc ASCII} representations,
their corresponding traditional mathematical symbols,
and a discussion of their meanings
in the \texttt{set.mm} database.
These are provided in order of appearance.
This is only a partial list, and new definitions are routinely added.
A complete list is available at \url{http://metamath.org}.

These {\sc ASCII} representations, along
with information on how to display them,
are defined in the \texttt{set.mm} database file inside
a special comment called a \texttt{\$t} {\em
comment}\index{\texttt{\$t} comment} or {\em typesetting
comment.}\index{typesetting comment}
A typesetting comment
is indicated by the appearance of the
two-character string \texttt{\$t} at the beginning of the comment.
For more information,
see Section~\ref{tcomment}, p.~\pageref{tcomment}.

In the following table the ``{\sc ASCII}'' column shows the {\sc ASCII}
representation,
``Symbol'' shows the mathematical symbolic display
that corresponds to that {\sc ASCII} representation, ``Labels'' shows
the key label(s) that define the representation, and
``Description'' provides a description about the symbol.
As usual, ``iff'' is short for ``if and only if.''\index{iff}
In most cases the ``{\sc ASCII}'' column only shows
the key token, but it will sometimes show a sequence of tokens
if that is necessary for clarity.

{\setlength{\extrarowsep}{4pt} % Keep rows from being too close together
\begin{longtabu}   { @{} c c l X }
\textbf{ASCII} & \textbf{Symbol} & \textbf{Labels} & \textbf{Description} \\
\endhead
\texttt{|-} & $\vdash$ & &
  ``It is provable that...'' \\
\texttt{ph} & $\varphi$ & \texttt{wph} &
  The wff (boolean) variable phi,
  conventionally the first wff variable. \\
\texttt{ps} & $\psi$ & \texttt{wps} &
  The wff (boolean) variable psi,
  conventionally the second wff variable. \\
\texttt{ch} & $\chi$ & \texttt{wch} &
  The wff (boolean) variable chi,
  conventionally the third wff variable. \\
\texttt{-.} & $\lnot$ & \texttt{wn} &
  Logical not. E.g., if $\varphi$ is true, then $\lnot \varphi$ is false. \\
\texttt{->} & $\rightarrow$ & \texttt{wi} &
  Implies, also known as material implication.
  In classical logic the expression $\varphi \rightarrow \psi$ is true
  if either $\varphi$ is false or $\psi$ is true (or both), that is,
  $\varphi \rightarrow \psi$ has the same meaning as
  $\lnot \varphi \lor \psi$ (as proven in theorem \texttt{imor}). \\
\texttt{<->} & $\leftrightarrow$ &
  \hyperref[df-bi]{\texttt{df-bi}} &
  Biconditional (aka is-equals for boolean values).
  $\varphi \leftrightarrow \psi$ is true iff
  $\varphi$ and $\psi$ have the same value. \\
\texttt{\char`\\/} & $\lor$ &
  \makecell[tl]{{\hyperref[df-or]{\texttt{df-or}}}, \\
	         \hyperref[df-3or]{\texttt{df-3or}}} &
  Disjunction (logical ``or''). $\varphi \lor \psi$ is true iff
  $\varphi$, $\psi$, or both are true. \\
\texttt{/\char`\\} & $\land$ &
  \makecell[tl]{{\hyperref[df-an]{\texttt{df-an}}}, \\
                 \hyperref[df-3an]{\texttt{df-3an}}} &
  Conjunction (logical ``and''). $\varphi \land \psi$ is true iff
  both $\varphi$ and $\psi$ are true. \\
\texttt{A.} & $\forall$ &
  \texttt{wal} &
  For all; the wff $\forall x \varphi$ is true iff
  $\varphi$ is true for all values of $x$. \\
\texttt{E.} & $\exists$ &
  \hyperref[df-ex]{\texttt{df-ex}} &
  There exists; the wff
  $\exists x \varphi$ is true iff
  there is at least one $x$ where $\varphi$ is true. \\
\texttt{[ y / x ]} & $[ y / x ]$ &
  \hyperref[df-sb]{\texttt{df-sb}} &
  The wff $[ y / x ] \varphi$ produces
  the result when $y$ is properly substituted for $x$ in $\varphi$
  ($y$ replaces $x$).
  % This is elsb4
  % ( [ x / y ] z e. y <-> z e. x )
  For example,
  $[ x / y ] z \in y$ is the same as $z \in x$. \\
\texttt{E!} & $\exists !$ &
  \hyperref[df-eu]{\texttt{df-eu}} &
  There exists exactly one;
  $\exists ! x \varphi$ is true iff
  there is at least one $x$ where $\varphi$ is true. \\
\texttt{\{ y | phi \}}  & $ \{ y | \varphi \}$ &
  \hyperref[df-clab]{\texttt{df-clab}} &
  The class of all sets where $\varphi$ is true. \\
\texttt{=} & $ = $ &
  \hyperref[df-cleq]{\texttt{df-cleq}} &
  Class equality; $A = B$ iff $A$ equals $B$. \\
\texttt{e.} & $ \in $ &
  \hyperref[df-clel]{\texttt{df-clel}} &
  Class membership; $A \in B$ if $A$ is a member of $B$. \\
\texttt{{\char`\_}V} & {\rm V} &
  \hyperref[df-v]{\texttt{df-v}} &
  Class of all sets (not itself a set). \\
\texttt{C\_} & $ \subseteq $ &
  \hyperref[df-ss]{\texttt{df-ss}} &
  Subclass (subset); $A \subseteq B$ is true iff
  $A$ is a subclass of $B$. \\
\texttt{u.} & $ \cup $ &
  \hyperref[df-un]{\texttt{df-un}} &
  $A \cup B$ is the union of classes $A$ and $B$. \\
\texttt{i^i} & $ \cap $ &
  \hyperref[df-in]{\texttt{df-in}} &
  $A \cap B$ is the intersection of classes $A$ and $B$. \\
\texttt{\char`\\} & $ \setminus $ &
  \hyperref[df-dif]{\texttt{df-dif}} &
  $A \setminus B$ (set difference)
  is the class of all sets in $A$ except for those in $B$. \\
\texttt{(/)} & $ \varnothing $ &
  \hyperref[df-nul]{\texttt{df-nul}} &
  $ \varnothing $ is the empty set (aka null set). \\
\texttt{\char`\~P} & $ \cal P $ &
  \hyperref[df-pw]{\texttt{df-pw}} &
  Power class. \\
\texttt{<.\ A , B >.} & $\langle A , B \rangle$ &
  \hyperref[df-op]{\texttt{df-op}} &
  The ordered pair $\langle A , B \rangle$. \\
\texttt{( F ` A )} & $ ( F ` A ) $ &
  \hyperref[df-fv]{\texttt{df-fv}} &
  The value of function $F$ when applied to $A$. \\
\texttt{\_i} & $ i $ &
  \texttt{df-i} &
  The square root of negative one. \\
\texttt{x.} & $ \cdot $ &
  \texttt{df-mul} &
  Complex number multiplication; $2~\cdot~3~=~6$. \\
\texttt{CC} & $ \mathbb{C} $ &
  \texttt{df-c} &
  The set of complex numbers. \\
\texttt{RR} & $ \mathbb{R} $ &
  \texttt{df-r} &
  The set of real numbers. \\
\end{longtabu}
} % end of extrarowsep

\chapter{Compressed Proofs}
\label{compressed}\index{compressed proof}\index{proof!compressed}

The proofs in the \texttt{set.mm} set theory database are stored in compressed
format for efficiency.  Normally you needn't concern yourself with the
compressed format, since you can display it with the usual proof display tools
in the Metamath program (\texttt{show proof}\ldots) or convert it to the normal
RPN proof format described in Section~\ref{proof} (with \texttt{save proof}
{\em label} \texttt{/normal}).  However for sake of completeness we describe the
format here and show how it maps to the normal RPN proof format.

A compressed proof, located between \texttt{\$=} and \texttt{\$.}\ keywords, consists
of a left parenthesis, a sequence of statement labels, a right parenthesis,
and a sequence of upper-case letters \texttt{A} through \texttt{Z} (with optional
white space between them).  White space must surround the parentheses
and the labels.  The left parenthesis tells Metamath that a
compressed proof follows.  (A normal RPN proof consists of just a sequence of
labels, and a parenthesis is not a legal character in a label.)

The sequence of upper-case letters corresponds to a sequence of integers
with the following mapping.  Each integer corresponds to a proof step as
described later.
\begin{center}
  \texttt{A} = 1 \\
  \texttt{B} = 2 \\
   \ldots \\
  \texttt{T} = 20 \\
  \texttt{UA} = 21 \\
  \texttt{UB} = 22 \\
   \ldots \\
  \texttt{UT} = 40 \\
  \texttt{VA} = 41 \\
  \texttt{VB} = 42 \\
   \ldots \\
  \texttt{YT} = 120 \\
  \texttt{UUA} = 121 \\
   \ldots \\
  \texttt{YYT} = 620 \\
  \texttt{UUUA} = 621 \\
   etc.
\end{center}

In other words, \texttt{A} through \texttt{T} represent the
least-significant digit in base 20, and \texttt{U} through \texttt{Y}
represent zero or more most-significant digits in base 5, where the
digits start counting at 1 instead of the usual 0. With this scheme, we
don't need white space between these ``numbers.''

(In the design of the compressed proof format, only upper-case letters,
as opposed to say all non-whitespace printable {\sc ascii} characters other than
%\texttt{\$}, was chosen to make the compressed proof a little less
%displeasing to the eye, at the expense of a typical 20\% compression
\texttt{\$}, were chosen so as not to collide with most text editor
searches, at the expense of a typical 20\% compression
loss.  The base 5/base 20 grouping, as opposed to say base 6/base 19,
was chosen by experimentally determining the grouping that resulted in
best typical compression.)

The letter \texttt{Z} identifies (tags) a proof step that is identical to one
that occurs later on in the proof; it helps shorten the proof by not requiring
that identical proof steps be proved over and over again (which happens often
when building wff's).  The \texttt{Z} is placed immediately after the
least-significant digit (letters \texttt{A} through \texttt{T}) that ends the integer
corresponding to the step to later be referenced.

The integers that the upper-case letters correspond to are mapped to labels as
follows.  If the statement being proved has $m$ mandatory hypotheses, integers
1 through $m$ correspond to the labels of these hypotheses in the order shown
by the \texttt{show statement ... / full} command, i.e., the RPN order\index{RPN
order} of the mandatory
hypotheses.  Integers $m+1$ through $m+n$ correspond to the labels enclosed in
the parentheses of the compressed proof, in the order that they appear, where
$n$ is the number of those labels.  Integers $m+n+1$ on up don't directly
correspond to statement labels but point to proof steps identified with the
letter \texttt{Z}, so that these proof steps can be referenced later in the
proof.  Integer $m+n+1$ corresponds to the first step tagged with a \texttt{Z},
$m+n+2$ to the second step tagged with a \texttt{Z}, etc.  When the compressed
proof is converted to a normal proof, the entire subproof of a step tagged
with \texttt{Z} replaces the reference to that step.

For efficiency, Metamath works with compressed proofs directly, without
converting them internally to normal proofs.  In addition to the usual
error-checking, an error message is given if (1) a label in the label list in
parentheses does not refer to a previous \texttt{\$p} or \texttt{\$a} statement or a
non-mandatory hypothesis of the statement being proved and (2) a proof step
tagged with \texttt{Z} is referenced before the step tagged with the \texttt{Z}.

Just as in a normal proof under development (Section~\ref{unknown}), any step
or subproof that is not yet known may be represented with a single \texttt{?}.
White space does not have to appear between the \texttt{?}\ and the upper-case
letters (or other \texttt{?}'s) representing the remainder of the proof.

% April 1, 2004 Appendix C has been added back in with corrections.
%
% May 20, 2003 Appendix C was removed for now because there was a problem found
% by Bob Solovay
%
% Also, removed earlier \ref{formalspec} 's (3 cases above)
%
% Bob Solovay wrote on 30 Nov 2002:
%%%%%%%%%%%%% (start of email comment )
%      3. My next set of comments concern appendix C. I read this before I
% read Chapter 4. So I first noted that the system as presented in the
% Appendix lacked a certain formal property that I thought desirable. I
% then came up with a revised formal system that had this property. Upon
% reading Chapter 4, I noticed that the revised system was closer to the
% treatment in Chapter 4 than the system in Appendix C.
%
%         First a very minor correction:
%
%         On page 142 line 2: The condition that V(e) != V(f) should only be
% required of e, f in T such that e != f.
%
%         Here is a natural property [transitivity] that one would like
% the formal system to have:
%
%         Let Gamma be a set of statements. Suppose that the statement Phi
% is provable from Gamma and that the statement Psi is provable from Gamma
% \cup {Phi}. Then Psi is provable from Gamma.
%
%         I shall present an example to show that this property does not
% hold for the formal systems of Appendix C:
%
%         I write the example in metamath style:
%
% $c A B C D E $.
% $v x y
%
% ${
% tx $f A x $.
% ty $f B y $.
% ax1 $a C x y $.
% $}
%
% ${
% tx $f A x $.
% ty $f B y $.
% ax2-h1 $e C x y $.
% ax2 $a D y $.
% $}
%
% ${
% ty $f B y $.
% ax3-h1 $e D y $.
% ax3 $a E y $.
% $}
%
% $(These three axioms are Gamma $)
%
% ${
% tx $f A x $.
% ty $f B y $.
% Phi $p D y $=
% tx ty tx ty ax1 ax2 $.
% $}
%
% ${
% ty $f B y $.
% Psi $p E y $=
% ty ty Phi ax3 $.
% $}
%
%
% I omit the formal proofs of the following claims. [I will be glad to
% supply them upon request.]
%
% 1) Psi is not provable from Gamma;
%
% 2) Psi is provable from Gamma + Phi.
%
% Here "provable" refers to the formalism of Appendix C.
%
% The trouble of course is that Psi is lacking the variable declaration
%
% $f Ax $.
%
% In the Metamath system there is no trouble proving Psi. I attach a
% metamath file that shows this and which has been checked by the
% metamath program.
%
% I next want to indicate how I think the treatment in Appendix C should
% be revised so as to conform more closely to the metamath system of the
% main text. The revised system *does* have the transitivity property.
%
% We want to give revised definitions of "statement" and
% "provable". [cf. sections C.2.4. and C.2.5] Our new definitions will
% use the definitions given in Appendix C. So we take the following
% tack. We refer to the original notions as o-statement and o-provable. And
% we refer to the notions we are defining as n-statement and n-provable.
%
%         A n-statement is an o-statement in which the only variables
% that appear in the T component are mandatory.
%
%         To any o-statement we can associate its reduct which is a
% n-statement by dropping all the elements of T or D which contain
% non-mandatory variables.
%
%         An n-statement gamma is n-provable if there is an o-statement
% gamma' which has gamma as its reduct andf such that gamma' is
% o-provable.
%
%         It seems to me [though I am not completely sure on this point]
% that n-provability corresponds to metamath provability as discussed
% say in Chapter 4.
%
%         Attached to this letter is the metamath proof of Phi and Psi
% from Gamma discussed above.
%
%         I am still brooding over the question of whether metamath
% correctly formalizes set-theory. No doubt I will have some questions
% re this after my thoughts become clearer.
%%%%%%%%%%%%%%%% (end of email comment)

%%%%%%%%%%%%%%%% (start of 2nd email comment from Bob Solovay 1-Apr-04)
%
%         I hope that Appendix C is the one that gives a "formal" treatment
% of Metamath. At any rate, thats the appendix I want to comment on.
%
%         I'm going to suggest two changes in the definition.
%
%         First change (in the definition of statement): Require that the
% sets D, T, and E be finite.
%
%         Probably things are fine as you give them. But in the applications
% to the main metamath system they will always be finite, and its useful in
% thinking about things [at least for me] to stick to the finite case.
%
%         Second change:
%
%         First let me give an approximate description. Remove the dummy
% variables from the statement. Instead, include them in the proof.
%
%         More formally: Require that T consists of type declarations only
% for mandatory variables. Require that all the pairs in D consist of
% mandatory variables.
%
%         At the start of a proof we are allowed to declare a finite number
% of dummy variables [provided that none of them appear in any of the
% statements in E \cup {A}. We have to supply type declarations for all the
% dummy variables. We are allowed to add new $d statements referring to
% either the mandatory or dummy variables. But we require that no new $d
% statement references only mandatory variables.
%
%         I find this way of doing things more conceptual than the treatment
% in Appendix C. But the change [which I will use implicitly in later
% letters about doing Peano] is mainly aesthetic. I definitely claim that my
% results on doing Peano all apply to Metamath as it is presented in your
% book.
%
%         --Bob
%
%%%%%%%%%%%%%%%% (end of 2nd email comment)

%%
%% When uncommenting the below, also uncomment references above to {formalspec}
%%
\chapter{Metamath's Formal System}\label{formalspec}\index{Metamath!as a formal
system}

\section{Introduction}

\begin{quote}
  {\em Perfection is when there is no longer anything more to take away.}
    \flushright\sc Antoine de
     Saint-Exupery\footnote{\cite[p.~3-25]{Campbell}.}\\
\end{quote}\index{de Saint-Exupery, Antoine}

This appendix describes the theory behind the Metamath language in an abstract
way intended for mathematicians.  Specifically, we construct two
set-theo\-ret\-i\-cal objects:  a ``formal system'' (roughly, a set of syntax
rules, axioms, and logical rules) and its ``universe'' (roughly, the set of
theorems derivable in the formal system).  The Metamath computer language
provides us with a way to describe specific formal systems and, with the aid of
a proof provided by the user, to verify that given theorems
belong to their universes.

To understand this appendix, you need a basic knowledge of informal set theory.
It should be sufficient to understand, for example, Ch.\ 1 of Munkres' {\em
Topology} \cite{Munkres}\index{Munkres, James R.} or the
introductory set theory chapter
in many textbooks that introduce abstract mathematics. (Note that there are
minor notational differences among authors; e.g.\ Munkres uses $\subset$ instead
of our $\subseteq$ for ``subset.''  We use ``included in'' to mean ``a subset
of,'' and ``belongs to'' or ``is contained in'' to mean ``is an element of.'')
What we call a ``formal'' description here, unlike earlier, is actually an
informal description in the ordinary language of mathematicians.  However we
provide sufficient detail so that a mathematician could easily formalize it,
even in the language of Metamath itself if desired.  To understand the logic
examples at the end of this appendix, familiarity with an introductory book on
mathematical logic would be helpful.

\section{The Formal Description}

\subsection[Preliminaries]{Preliminaries\protect\footnotemark}%
\footnotetext{This section is taken mostly verbatim
from Tarski \cite[p.~63]{Tarski1965}\index{Tarski, Alfred}.}

By $\omega$ we denote the set of all natural numbers (non-negative integers).
Each natural number $n$ is identified with the set of all smaller numbers: $n =
\{ m | m < n \}$.  The formula $m < n$ is thus equivalent to the condition: $m
\in n$ and $m,n \in \omega$. In particular, 0 is the number zero and at the
same time the empty set $\varnothing$, $1=\{0\}$, $2=\{0,1\}$, etc. ${}^B A$
denotes the set of all functions on $B$ to $A$ (i.e.\ with domain $B$ and range
included in $A$).  The members of ${}^\omega A$ are what are called {\em simple
infinite sequences},\index{simple infinite sequence}
with all {\em terms}\index{term} in $A$.  In case $n \in \omega$, the
members of ${}^n A$ are referred to as {\em finite $n$-termed
sequences},\index{finite $n$-termed
sequence} again
with terms in $A$.  The consecutive terms (function values) of a finite or
infinite sequence $f$ are denoted by $f_0, f_1, \ldots ,f_n,\ldots$.  Every
finite sequence $f \in \bigcup _{n \in \omega} {}^n A$ uniquely determines the
number $n$ such that $f \in {}^n A$; $n$ is called the {\em
length}\index{length of a sequence ({$"|\ "|$})} of $f$ and
is denoted by $|f|$.  $\langle a \rangle$ is the sequence $f$ with $|f|=1$ and
$f_0=a$; $\langle a,b \rangle$ is the sequence $f$ with $|f|=2$, $f_0=a$,
$f_1=b$; etc.  Given two finite sequences $f$ and $g$, we denote by $f\frown g$
their {\em concatenation},\index{concatenation} i.e., the
finite sequence $h$ determined by the
conditions:
\begin{eqnarray*}
& |h| = |f|+|g|;&  \\
& h_n = f_n & \mbox{\ for\ } n < |f|;  \\
& h_{|f|+n} = g_n & \mbox{\ for\ } n < |g|.
\end{eqnarray*}

\subsection{Constants, Variables, and Expressions}

A formal system has a set of {\em symbols}\index{symbol!in
a formal system} denoted
by $\mbox{\em SM}$.  A
precise set-theo\-ret\-i\-cal definition of this set is unimportant; a symbol
could be considered a primitive or atomic element if we wish.  We assume this
set is divided into two disjoint subsets:  a set $\mbox{\em CN}$ of {\em
constants}\index{constant!in a formal system} and a set $\mbox{\em VR}$ of
{\em variables}.\index{variable!in a formal system}  $\mbox{\em CN}$ and
$\mbox{\em VR}$ are each assumed to consist of countably many symbols which
may be arranged in finite or simple infinite sequences $c_0, c_1, \ldots$ and
$v_0, v_1, \ldots$ respectively, without repeating terms.  We will represent
arbitrary symbols by metavariables $\alpha$, $\beta$, etc.

{\footnotesize\begin{quotation}
{\em Comment.} The variables $v_0, v_1, \ldots$ of our formal system
correspond to what are usually considered ``metavariables'' in
descriptions of specific formal systems in the literature.  Typically,
when describing a specific formal system a book will postulate a set of
primitive objects called variables, then proceed to describe their
properties using metavariables that range over them, never mentioning
again the actual variables themselves.  Our formal system does not
mention these primitive variable objects at all but deals directly with
metavariables, as its primitive objects, from the start.  This is a
subtle but key distinction you should keep in mind, and it makes our
definition of ``formal system'' somewhat different from that typically
found in the literature.  (So, the $\alpha$, $\beta$, etc.\ above are
actually ``metametavariables'' when used to represent $v_0, v_1,
\ldots$.)
\end{quotation}}

Finite sequences all terms of which are symbols are called {\em
expressions}.\index{expression!in a formal system}  $\mbox{\em EX}$ is
the set of all expressions; thus
\begin{displaymath}
\mbox{\em EX} = \bigcup _{n \in \omega} {}^n \mbox{\em SM}.
\end{displaymath}

A {\em constant-prefixed expression}\index{constant-prefixed expression}
is an expression of non-zero length
whose first term is a constant.  We denote the set of all constant-prefixed
expressions by $\mbox{\em EX}_C = \{ e \in \mbox{\em EX} | ( |e| > 0 \wedge
e_0 \in \mbox{\em CN} ) \}$.

A {\em constant-variable pair}\index{constant-variable pair}
is an expression of length 2 whose first term
is a constant and whose second term is a variable.  We denote the set of all
constant-variable pairs by $\mbox{\em EX}_2 = \{ e \in \mbox{\em EX}_C | ( |e|
= 2 \wedge e_1 \in \mbox{\em VR} ) \}$.


{\footnotesize\begin{quotation}
{\em Relationship to Metamath.} In general, the set $\mbox{\em SM}$
corresponds to the set of declared math symbols in a Metamath database, the
set $\mbox{\em CN}$ to those declared with \texttt{\$c} statements, and the set
$\mbox{\em VR}$ to those declared with \texttt{\$v} statements.  Of course a
Metamath database can only have a finite number of math symbols, whereas
formal systems in general can have an infinite number, although the number of
Metamath math symbols available is in principle unlimited.

The set $\mbox{\em EX}_C$ corresponds to the set of permissible expressions
for \texttt{\$e}, \texttt{\$a}, and \texttt{\$p} statements.  The set $\mbox{\em EX}_2$
corresponds to the set of permissible expressions for \texttt{\$f} statements.
\end{quotation}}

We denote by ${\cal V}(e)$ the set of all variables in an expression $e \in
\mbox{\em EX}$, i.e.\ the set of all $\alpha \in \mbox{\em VR}$ such that
$\alpha = e_n$ for some $n < |e|$.  We also denote (with abuse of notation) by
${\cal V}(E)$ the set of all variables in a collection of expressions $E
\subseteq \mbox{\em EX}$, i.e.\ $\bigcup _{e \in E} {\cal V}(e)$.


\subsection{Substitution}

Given a function $F$ from $\mbox{\em VR}$ to
$\mbox{\em EX}$, we
denote by $\sigma_{F}$ or just $\sigma$ the function from $\mbox{\em EX}$ to
$\mbox{\em EX}$ defined recursively for nonempty sequences by
\begin{eqnarray*}
& \sigma(<\alpha>) = F(\alpha) & \mbox{for\ } \alpha \in \mbox{\em VR}; \\
& \sigma(<\alpha>) = <\alpha> & \mbox{for\ } \alpha \not\in \mbox{\em VR}; \\
& \sigma(g \frown h) = \sigma(g) \frown
    \sigma(h) & \mbox{for\ } g,h \in \mbox{\em EX}.
\end{eqnarray*}
We also define $\sigma(\varnothing)=\varnothing$.  We call $\sigma$ a {\em
simultaneous substitution}\index{substitution!variable}\index{variable
substitution} (or just {\em substitution}) with {\em substitution
map}\index{substitution map} $F$.

We also denote (with abuse of notation) by $\sigma(E)$ a substitution on a
collection of expressions $E \subseteq \mbox{\em EX}$, i.e.\ the set $\{
\sigma(e) | e \in E \}$.  The collection $\sigma(E)$ may of course contain
fewer expressions than $E$ because duplicate expressions could result from the
substitution.

\subsection{Statements}

We denote by $\mbox{\em DV}$ the set of all
unordered pairs $\{\alpha, \beta \} \subseteq \mbox{\em VR}$ such that $\alpha
\neq \beta$.  $\mbox{\em DV}$ stands for ``distinct variables.''

A {\em pre-statement}\index{pre-statement!in a formal system} is a
quadruple $\langle D,T,H,A \rangle$ such that
$D\subseteq \mbox{\em DV}$, $T\subseteq \mbox{\em EX}_2$, $H\subseteq
\mbox{\em EX}_C$ and $H$ is finite,
$A\in \mbox{\em EX}_C$, ${\cal V}(H\cup\{A\}) \subseteq
{\cal V}(T)$, and $\forall e,f\in T {\ } {\cal V}(e) \neq {\cal V}(f)$ (or
equivalently, $e_1 \ne f_1$) whenever $e \neq f$. The terms of the quadruple are called {\em
distinct-variable restrictions},\index{disjoint-variable restriction!in a
formal system} {\em variable-type hypotheses},\index{variable-type
hypothesis!in a formal system} {\em logical hypotheses},\index{logical
hypothesis!in a formal system} and the {\em assertion}\index{assertion!in a
formal system} respectively.  We denote by $T_M$ ({\em mandatory variable-type
hypotheses}\index{mandatory variable-type hypothesis!in a formal system}) the
subset of $T$ such that ${\cal V}(T_M) ={\cal V}(H \cup \{A\})$.  We denote by
$D_M=\{\{\alpha,\beta\}\in D|\{\alpha,\beta\}\subseteq {\cal V}(T_M)\}$ the
{\em mandatory distinct-variable restrictions}\index{mandatory
disjoint-variable restriction!in a formal system} of the pre-statement.
The set
of {\em mandatory hypotheses}\index{mandatory hypothesis!in a formal system}
is $T_M\cup H$.  We call the quadruple $\langle D_M,T_M,H,A \rangle$
the {\em reduct}\index{reduct!in a formal system} of
the pre-statement $\langle D,T,H,A \rangle$.

A {\em statement} is the reduct of some pre-statement\index{statement!in a
formal system}.  A statement is therefore a special kind of pre-statement;
in particular, a statement is the reduct of itself.

{\footnotesize\begin{quotation}
{\em Comment.}  $T$ is a set of expressions, each of length 2, that associate
a set of constants (``variable types'') with a set of variables.  The
condition ${\cal V}(H\cup\{A\}) \subseteq {\cal V}(T) $
means that each variable occurring in a statement's logical
hypotheses or assertion must have an associated variable-type hypothesis or
``type declaration,'' in  analogy to a computer programming language, where a
variable must be declared to be say, a string or an integer.  The requirement
that $\forall e,f\in T \, e_1 \ne f_1$ for $e\neq f$
means that each variable must be
associated with a unique constant designating its variable type; e.g., a
variable might be a ``wff'' or a ``set'' but not both.

Distinct-variable restrictions are used to specify what variable substitutions
are permissible to make for the statement to remain valid.  For example, in
the theorem scheme of set theory $\lnot\forall x\,x=y$ we may not substitute
the same variable for both $x$ and $y$.  On the other hand, the theorem scheme
$x=y\to y=x$ does not require that $x$ and $y$ be distinct, so we do not
require a distinct-variable restriction, although having one
would cause no harm other than making the scheme less general.

A mandatory variable-type hypothesis is one whose variable exists in a logical
hypothesis or the assertion.  A provable pre-statement
(defined below) may require
non-mandatory variable-type hypotheses that effectively introduce ``dummy''
variables for use in its proof.  Any number of dummy variables might
be required by a specific proof; indeed, it has been shown by H.\
Andr\'{e}ka\index{Andr{\'{e}}ka, H.} \cite{Nemeti} that there is no finite
upper bound to the number of dummy variables needed to prove an arbitrary
theorem in first-order logic (with equality) having a fixed number $n>2$ of
individual variables.  (See also the Comment on p.~\pageref{nodd}.)
For this reason we do not set a finite size bound on the collections $D$ and
$T$, although in an actual application (Metamath database) these will of
course be finite, increased to whatever size is necessary as more
proofs are added.
\end{quotation}}

{\footnotesize\begin{quotation}
{\em Relationship to Metamath.} A pre-statement of a formal system
corresponds to an extended frame in a Metamath database
(Section~\ref{frames}).  The collections $D$, $T$, and $H$ correspond
respectively to the \texttt{\$d}, \texttt{\$f}, and \texttt{\$e}
statement collections in an extended frame.  The expression $A$
corresponds to the \texttt{\$a} (or \texttt{\$p}) statement in an
extended frame.

A statement of a formal system corresponds to a frame in a Metamath
database.
\end{quotation}}

\subsection{Formal Systems}

A {\em formal system}\index{formal system} is a
triple $\langle \mbox{\em CN},\mbox{\em
VR},\Gamma\rangle$ where $\Gamma$ is a set of statements.  The members of
$\Gamma$ are called {\em axiomatic statements}.\index{axiomatic
statement!in a formal system}  Sometimes we will refer to a
formal system by just $\Gamma$ when $\mbox{\em CN}$ and $\mbox{\em VR}$ are
understood.

Given a formal system $\Gamma$, the {\em closure}\index{closure}\footnote{This
definition of closure incorporates a simplification due to
Josh Purinton.\index{Purinton, Josh}.} of a
pre-statement
$\langle D,T,H,A \rangle$ is the smallest set $C$ of expressions
such that:
%\begin{enumerate}
%  \item $T\cup H\subseteq C$; and
%  \item If for some axiomatic statement
%    $\langle D_M',T_M',H',A' \rangle \in \Gamma_A$, for
%    some $E \subseteq C$, some $F \subseteq C-T$ (where ``-'' denotes
%    set difference), and some substitution
%    $\sigma$ we have
%    \begin{enumerate}
%       \item $\sigma(T_M') = E$ (where, as above, the $M$ denotes the
%           mandatory variable-type hypotheses of $T^A$);
%       \item $\sigma(H') = F$;
%       \item for all $\{\alpha,\beta\}\in D^A$ and $\subseteq
%         {\cal V}(T_M')$, for all $\gamma\in {\cal V}(\sigma(\langle \alpha
%         \rangle))$, and for all $\delta\in  {\cal V}(\sigma(\langle \beta
%         \rangle))$, we have $\{\gamma, \delta\} \in D$;
%   \end{enumerate}
%   then $\sigma(A') \in C$.
%\end{enumerate}
\begin{list}{}{\itemsep 0.0pt}
  \item[1.] $T\cup H\subseteq C$; and
  \item[2.] If for some axiomatic statement
    $\langle D_M',T_M',H',A' \rangle \in
       \Gamma$ and for some substitution
    $\sigma$ we have
    \begin{enumerate}
       \item[a.] $\sigma(T_M' \cup H') \subseteq C$; and
       \item[b.] for all $\{\alpha,\beta\}\in D_M'$, for all $\gamma\in
         {\cal V}(\sigma(\langle \alpha
         \rangle))$, and for all $\delta\in  {\cal V}(\sigma(\langle \beta
         \rangle))$, we have $\{\gamma, \delta\} \in D$;
   \end{enumerate}
   then $\sigma(A') \in C$.
\end{list}
A pre-statement $\langle D,T,H,A
\rangle$ is {\em provable}\index{provable statement!in a formal
system} if $A\in C$ i.e.\ if its assertion belongs to its
closure.  A statement is {\em provable} if it is
the reduct of a provable pre-statement.
The {\em universe}\index{universe of a formal system}
of a formal system is
the collection of all of its provable statements.  Note that the
set of axiomatic statements $\Gamma$ in a formal system is a subset of its
universe.

{\footnotesize\begin{quotation}
{\em Comment.} The first condition in the definition of closure simply says
that the hypotheses of the pre-statement are in its closure.

Condition 2(a) says that a substitution exists that makes the
mandatory hypotheses of an axiomatic statement exactly match some members of
the closure.  This is what we explicitly demonstrate in a Metamath language
proof.

%Conditions 2(a) and 2(b) say that a substitution exists that makes the
%(mandatory) hypotheses of an axiomatic statement exactly match some members of
%the closure.  This is what we explicitly demonstrate with a Metamath language
%proof.
%
%The set of expressions $F$ in condition 2(b) excludes the variable-type
%hypotheses; this is done because non-mandatory variable-type hypotheses are
%effectively ``dropped'' as irrelevant whereas logical hypotheses must be
%retained to achieve a consistent logical system.

Condition 2(b) describes how distinct-variable restrictions in the axiomatic
statement must be met.  It means that after a substitution for two variables
that must be distinct, the resulting two expressions must either contain no
variables, or if they do, they may not have variables in common, and each pair
of any variables they do have, with one variable from each expression, must be
specified as distinct in the original statement.
\end{quotation}}

{\footnotesize\begin{quotation}
{\em Relationship to Metamath.} Axiomatic statements
 and provable statements in a formal
system correspond to the frames for \texttt{\$a} and \texttt{\$p} statements
respectively in a Metamath database.  The set of axiomatic statements is a
subset of the set of provable statements in a formal system, although in a
Metamath database a \texttt{\$a} statement is distinguished by not having a
proof.  A Metamath language proof for a \texttt{\$p} statement tells the computer
how to explicitly construct a series of members of the closure ultimately
leading to a demonstration that the assertion
being proved is in the closure.  The actual closure typically contains
an infinite number of expressions.  A formal system itself does not have
an explicit object called a ``proof'' but rather the existence of a proof
is implied indirectly by membership of an assertion in a provable
statement's closure.  We do this to make the formal system easier
to describe in the language of set theory.

We also note that once established as provable, a statement may be considered
to acquire the same status as an axiomatic statement, because if the set of
axiomatic statements is extended with a provable statement, the universe of
the formal system remains unchanged (provided that $\mbox{\em VR}$ is
infinite).
In practice, this means we can build a hierarchy of provable statements to
more efficiently establish additional provable statements.  This is
what we do in Metamath when we allow proofs to reference previous
\texttt{\$p} statements as well as previous \texttt{\$a} statements.
\end{quotation}}

\section{Examples of Formal Systems}

{\footnotesize\begin{quotation}
{\em Relationship to Metamath.} The examples in this section, except Example~2,
are for the most part exact equivalents of the development in the set
theory database \texttt{set.mm}.  You may want to compare Examples~1, 3, and 5
to Section~\ref{metaaxioms}, Example 4 to Sections~\ref{metadefprop} and
\ref{metadefpred}, and Example 6 to
Section~\ref{setdefinitions}.\label{exampleref}
\end{quotation}}

\subsection{Example~1---Propositional Calculus}\index{propositional calculus}

Classical propositional calculus can be described by the following formal
system.  We assume the set of variables is infinite.  Rather than denoting the
constants and variables by $c_0, c_1, \ldots$ and $v_0, v_1, \ldots$, for
readability we will instead use more conventional symbols, with the
understanding of course that they denote distinct primitive objects.
Also for readability we may omit commas between successive terms of a
sequence; thus $\langle \mbox{wff\ } \varphi\rangle$ denotes
$\langle \mbox{wff}, \varphi\rangle$.

Let
\begin{itemize}
  \item[] $\mbox{\em CN}=\{\mbox{wff}, \vdash, \to, \lnot, (,)\}$
  \item[] $\mbox{\em VR}=\{\varphi,\psi,\chi,\ldots\}$
  \item[] $T = \{\langle \mbox{wff\ } \varphi\rangle,
             \langle \mbox{wff\ } \psi\rangle,
             \langle \mbox{wff\ } \chi\rangle,\ldots\}$, i.e.\ those
             expressions of length 2 whose first member is $\mbox{\rm wff}$
             and whose second member belongs to $\mbox{\em VR}$.\footnote{For
convenience we let $T$ be an infinite set; the definition of a statement
permits this in principle.  Since a Metamath source file has a finite size, in
practice we must of course use appropriate finite subsets of this $T$,
specifically ones containing at least the mandatory variable-type
hypotheses.  Similarly, in the source file we introduce new variables as
required, with the understanding that a potentially infinite number of
them are available.}
\noindent Then $\Gamma$ consists of the axiomatic statements that
are the reducts of the following pre-statements:
    \begin{itemize}
      \item[] $\langle\varnothing,T,\varnothing,
               \langle \mbox{wff\ }(\varphi\to\psi)\rangle\rangle$
      \item[] $\langle\varnothing,T,\varnothing,
               \langle \mbox{wff\ }\lnot\varphi\rangle\rangle$
      \item[] $\langle\varnothing,T,\varnothing,
               \langle \vdash(\varphi\to(\psi\to\varphi))
               \rangle\rangle$
      \item[] $\langle\varnothing,T,
               \varnothing,
               \langle \vdash((\varphi\to(\psi\to\chi))\to
               ((\varphi\to\psi)\to(\varphi\to\chi)))
               \rangle\rangle$
      \item[] $\langle\varnothing,T,
               \varnothing,
               \langle \vdash((\lnot\varphi\to\lnot\psi)\to
               (\psi\to\varphi))\rangle\rangle$
      \item[] $\langle\varnothing,T,
               \{\langle\vdash(\varphi\to\psi)\rangle,
                 \langle\vdash\varphi\rangle\},
               \langle\vdash\psi\rangle\rangle$
    \end{itemize}
\end{itemize}

(For example, the reduct of $\langle\varnothing,T,\varnothing,
               \langle \mbox{wff\ }(\varphi\to\psi)\rangle\rangle$
is
\begin{itemize}
\item[] $\langle\varnothing,
\{\langle \mbox{wff\ } \varphi\rangle,
             \langle \mbox{wff\ } \psi\rangle\},
             \varnothing,
               \langle \mbox{wff\ }(\varphi\to\psi)\rangle\rangle$,
\end{itemize}
which is the first axiomatic statement.)

We call the members of $\mbox{\em VR}$ {\em wff variables} or (in the context
of first-order logic which we will describe shortly) {\em wff metavariables}.
Note that the symbols $\phi$, $\psi$, etc.\ denote actual specific members of
$\mbox{\em VR}$; they are not metavariables of our expository language (which
we denote with $\alpha$, $\beta$, etc.) but are instead (meta)constant symbols
(members of $\mbox{\em SM}$) from the point of view of our expository
language.  The equivalent system of propositional calculus described in
\cite{Tarski1965} also uses the symbols $\phi$, $\psi$, etc.\ to denote wff
metavariables, but in \cite{Tarski1965} unlike here those are metavariables of
the expository language and not primitive symbols of the formal system.

The first two statements define wffs: if $\varphi$ and $\psi$ are wffs, so is
$(\varphi \to \psi)$; if $\varphi$ is a wff, so is $\lnot\varphi$. The next
three are the axioms of propositional calculus: if $\varphi$ and $\psi$ are
wffs, then $\vdash (\varphi \to (\psi \to \varphi))$ is an (axiomatic)
theorem; etc. The
last is the rule of modus ponens: if $\varphi$ and $\psi$ are wffs, and
$\vdash (\varphi\to\psi)$ and $\vdash \varphi$ are theorems, then $\vdash
\psi$ is a theorem.

The correspondence to ordinary propositional calculus is as follows.  We
consider only provable statements of the form $\langle\varnothing,
T,\varnothing,A\rangle$ with $T$ defined as above.  The first term of the
assertion $A$ of any such statement is either ``wff'' or ``$\vdash$''.  A
statement for which the first term is ``wff'' is a {\em wff} of propositional
calculus, and one where the first term is ``$\vdash$'' is a {\em
theorem (scheme)} of propositional calculus.

The universe of this formal system also contains many other provable
statements.  Those with distinct-variable restrictions are irrelevant because
propositional calculus has no constraints on substitutions.  Those that have
logical hypotheses we call {\em inferences}\index{inference} when
the logical hypotheses are of the form
$\langle\vdash\rangle\frown w$ where $w$ is a wff (with the leading constant
term ``wff'' removed).  Inferences (other than the modus ponens rule) are not a
proper part of propositional calculus but are convenient to use when building a
hierarchy of provable statements.  A provable statement with a nonsense
hypothesis such as $\langle \to,\vdash,\lnot\rangle$, and this same expression
as its assertion, we consider irrelevant; no use can be made of it in
proving theorems, since there is no way to eliminate the nonsense hypothesis.

{\footnotesize\begin{quotation}
{\em Comment.} Our use of parentheses in the definition of a wff illustrates
how axiomatic statements should be carefully stated in a way that
ties in unambiguously with the substitutions allowed by the formal system.
There are many ways we could have defined wffs---for example, Polish
prefix notation would have allowed us to omit parentheses entirely, at
the expense of readability---but we must define them in a way that is
unambiguous.  For example, if we had omitted parentheses from the
definition of $(\varphi\to \psi)$, the wff $\lnot\varphi\to \psi$ could
be interpreted as either $\lnot(\varphi\to\psi)$ or $(\lnot\varphi\to\psi)$
and would have allowed us to prove nonsense.  Note that there is no
concept of operator binding precedence built into our formal system.
\end{quotation}}

\begin{sloppy}
\subsection{Example~2---Predicate Calculus with Equality}\index{predicate
calculus}
\end{sloppy}

Here we extend Example~1 to include predicate calculus with equality,
illustrating the use of distinct-variable restrictions.  This system is the
same as Tarski's system $\mathfrak{S}_2$ in \cite{Tarski1965} (except that the
axioms of propositional calculus are different but equivalent, and a redundant
axiom is omitted).  We extend $\mbox{\em CN}$ with the constants
$\{\mbox{var},\forall,=\}$.  We extend $\mbox{\em VR}$ with an infinite set of
{\em individual metavariables}\index{individual
metavariable} $\{x,y,z,\ldots\}$ and denote this subset
$\mbox{\em Vr}$.

We also join to $\mbox{\em CN}$ a possibly infinite set $\mbox{\em Pr}$ of {\em
predicates} $\{R,S,\ldots\}$.  We associate with $\mbox{\em Pr}$ a function
$\mbox{rnk}$ from $\mbox{\em Pr}$ to $\omega$, and for $\alpha\in \mbox{\em
Pr}$ we call $\mbox{rnk}(\alpha)$ the {\em rank} of the predicate $\alpha$,
which is simply the number of ``arguments'' that the predicate has.  (Most
applications of predicate calculus will have a finite number of predicates;
for example, set theory has the single two-argument or binary predicate $\in$,
which is usually written with its arguments surrounding the predicate symbol
rather than with the prefix notation we will use for the general case.)  As a
device to facilitate our discussion, we will let $\mbox{\em Vs}$ be any fixed
one-to-one function from $\omega$ to $\mbox{\em Vr}$; thus $\mbox{\em Vs}$ is
any simple infinite sequence of individual metavariables with no repeating
terms.

In this example we will not include the function symbols that are often part of
formalizations of predicate calculus.  Using metalogical arguments that are
beyond the scope of our discussion, it can be shown that our formalization is
equivalent when functions are introduced via appropriate definitions.

We extend the set $T$ defined in Example~1 with the expressions
$\{\langle \mbox{var\ } x\rangle,$ $ \langle \mbox{var\ } y\rangle, \langle
\mbox{var\ } z\rangle,\ldots\}$.  We extend the $\Gamma$ above
with the axiomatic statements that are the reducts of the following
pre-statements:
\begin{list}{}{\itemsep 0.0pt}
      \item[] $\langle\varnothing,T,\varnothing,
               \langle \mbox{wff\ }\forall x\,\varphi\rangle\rangle$
      \item[] $\langle\varnothing,T,\varnothing,
               \langle \mbox{wff\ }x=y\rangle\rangle$
      \item[] $\langle\varnothing,T,
               \{\langle\vdash\varphi\rangle\},
               \langle\vdash\forall x\,\varphi\rangle\rangle$
      \item[] $\langle\varnothing,T,\varnothing,
               \langle \vdash((\forall x(\varphi\to\psi)
                  \to(\forall x\,\varphi\to\forall x\,\psi))
               \rangle\rangle$
      \item[] $\langle\{\{x,\varphi\}\},T,\varnothing,
               \langle \vdash(\varphi\to\forall x\,\varphi)
               \rangle\rangle$
      \item[] $\langle\{\{x,y\}\},T,\varnothing,
               \langle \vdash\lnot\forall x\lnot x=y
               \rangle\rangle$
      \item[] $\langle\varnothing,T,\varnothing,
               \langle \vdash(x=z
                  \to(x=y\to z=y))
               \rangle\rangle$
      \item[] $\langle\varnothing,T,\varnothing,
               \langle \vdash(y=z
                  \to(x=y\to x=z))
               \rangle\rangle$
\end{list}
These are the axioms not involving predicate symbols. The first two statements
extend the definition of a wff.  The third is the rule of generalization.  The
fifth states, in effect, ``For a wff $\varphi$ and variable $x$,
$\vdash(\varphi\to\forall x\,\varphi)$, provided that $x$ does not occur in
$\varphi$.''  The sixth states ``For variables $x$ and $y$,
$\vdash\lnot\forall x\lnot x = y$, provided that $x$ and $y$ are distinct.''
(This proviso is not necessary but was included by Tarski to
weaken the axiom and still show that the system is logically complete.)

Finally, for each predicate symbol $\alpha\in \mbox{\em Pr}$, we add to
$\Gamma$ an axiomatic statement, extending the definition of wff,
that is the reduct of the following pre-statement:
\begin{displaymath}
    \langle\varnothing,T,\varnothing,
            \langle \mbox{wff},\alpha\rangle\
            \frown \mbox{\em Vs}\restriction\mbox{rnk}(\alpha)\rangle
\end{displaymath}
and for each $\alpha\in \mbox{\em Pr}$ and each $n < \mbox{rnk}(\alpha)$
we add to $\Gamma$ an equality axiom that is the reduct of the
following pre-statement:
\begin{eqnarray*}
    \lefteqn{\langle\varnothing,T,\varnothing,
            \langle
      \vdash,(,\mbox{\em Vs}_n,=,\mbox{\em Vs}_{\mbox{rnk}(\alpha)},\to,
            (,\alpha\rangle\frown \mbox{\em Vs}\restriction\mbox{rnk}(\alpha)} \\
  & & \frown
            \langle\to,\alpha\rangle\frown \mbox{\em Vs}\restriction n\frown
            \langle \mbox{\em Vs}_{\mbox{rnk}(\alpha)}\rangle \\
 & & \frown
            \mbox{\em Vs}\restriction(\mbox{rnk}(\alpha)\setminus(n+1))\frown
            \langle),)\rangle\rangle
\end{eqnarray*}
where $\restriction$ denotes function domain restriction and $\setminus$
denotes set difference.  Recall that a subscript on $\mbox{\em Vs}$
denotes one of its terms.  (In the above two axiom sets commas are placed
between successive terms of sequences to prevent ambiguity, and if you examine
them with care you will be able to distinguish those parentheses that denote
constant symbols from those of our expository language that delimit function
arguments.  Although it might have been better to use boldface for our
primitive symbols, unfortunately boldface was not available for all characters
on the \LaTeX\ system used to typeset this text.)  These seemingly forbidding
axioms can be understood by analogy to concatenation of substrings in a
computer language.  They are actually relatively simple for each specific case
and will become clearer by looking at the special case of a binary predicate
$\alpha = R$ where $\mbox{rnk}(R)=2$.  Letting $\mbox{\em Vs}$ be the sequence
$\langle x,y,z,\ldots\rangle$, the axioms we would add to $\Gamma$ for this
case would be the wff extension and two equality axioms that are the
reducts of the pre-statements:
\begin{list}{}{\itemsep 0.0pt}
      \item[] $\langle\varnothing,T,\varnothing,
               \langle \mbox{wff\ }R x y\rangle\rangle$
      \item[] $\langle\varnothing,T,\varnothing,
               \langle \vdash(x=z
                  \to(R x y \to R z y))
               \rangle\rangle$
      \item[] $\langle\varnothing,T,\varnothing,
               \langle \vdash(y=z
                  \to(R x y \to R x z))
               \rangle\rangle$
\end{list}
Study these carefully to see how the general axioms above evaluate to
them.  In practice, typically only a few special cases such as this would be
needed, and in any case the Metamath language will only permit us to describe
a finite number of predicates, as opposed to the infinite number permitted by
the formal system.  (If an infinite number should be needed for some reason,
we could not define the formal system directly in the Metamath language but
could instead define it metalogically under set theory as we
do in this appendix, and only the underlying set theory, with its single
binary predicate, would be defined directly in the Metamath language.)


{\footnotesize\begin{quotation}
{\em Comment.}  As we noted earlier, the specific variables denoted by the
symbols $x,y,z,\ldots\in \mbox{\em Vr}\subseteq \mbox{\em VR}\subseteq
\mbox{\em SM}$ in Example~2 are not the actual variables of ordinary predicate
calculus but should be thought of as metavariables ranging over them.  For
example, a distinct-variable restriction would be meaningless for actual
variables of ordinary predicate calculus since two different actual variables
are by definition distinct.  And when we talk about an arbitrary
representative $\alpha\in \mbox{\em Vr}$, $\alpha$ is a metavariable (in our
expository language) that ranges over metavariables (which are primitives of
our formal system) each of which ranges over the actual individual variables
of predicate calculus (which are never mentioned in our formal system).

The constant called ``var'' above is called \texttt{setvar} in the
\texttt{set.mm} database file, but it means the same thing.  I felt
that ``var'' is a more meaningful name in the context of predicate
calculus, whose use is not limited to set theory.  For consistency we
stick with the name ``var'' throughout this Appendix, even after set
theory is introduced.
\end{quotation}}

\subsection{Free Variables and Proper Substitution}\index{free variable}
\index{proper substitution}\index{substitution!proper}

Typical representations of mathematical axioms use concepts such
as ``free variable,'' ``bound variable,'' and ``proper substitution''
as primitive notions.
A free variable is a variable that
is not a parameter of any container expression.
A bound variable is the opposite of a free variable; it is a
a variable that has been bound in a container expression.
For example, in the expression $\forall x \varphi$ (for all $x$, $\varphi$
is true), the variable $x$
is bound within the for-all ($\forall$) expression.
It is possible to change one variable to another, and that process is called
``proper substitution.''
In most books, proper substitution has a somewhat complicated recursive
definition with multiple cases based on the occurrences of free and
bound variables.
You may consult
\cite[ch.\ 3--4]{Hamilton}\index{Hamilton, Alan G.} (as well as
many other texts) for more formal details about these terms.

Using these concepts as \texttt{primitives} creates complications
for computer implementations.

In the system of Example~2, there are no primitive notions of free variable
and proper substitution.  Tarski \cite{Tarski1965} shows that this system is
logically equivalent to the more typical textbook systems that do have these
primitive notions, if we introduce these notions with appropriate definitions
and metalogic.  We could also define axioms for such systems directly,
although the recursive definitions of free variable and proper substitution
would be messy and awkward to work with.  Instead, we mention two devices that
can be used in practice to mimic these notions.  (1) Instead of introducing
special notation to express (as a logical hypothesis) ``where $x$ is not free
in $\varphi$'' we can use the logical hypothesis $\vdash(\varphi\to\forall
x\,\varphi)$.\label{effectivelybound}\index{effectively
not free}\footnote{This is a slightly weaker requirement than ``where $x$ is
not free in $\varphi$.''  If we let $\varphi$ be $x=x$, we have the theorem
$(x=x\to\forall x\,x=x)$ which satisfies the hypothesis, even though $x$ is
free in $x=x$ .  In a case like this we say that $x$ is {\em effectively not
free}\index{effectively not free} in $x=x$, since $x=x$ is logically
equivalent to $\forall x\,x=x$ in which $x$ is bound.} (2) It can be shown
that the wff $((x=y\to\varphi)\wedge\exists x(x=y\wedge\varphi))$ (with the
usual definitions of $\wedge$ and $\exists$; see Example~4 below) is logically
equivalent to ``the wff that results from proper substitution of $y$ for $x$
in $\varphi$.''  This works whether or not $x$ and $y$ are distinct.

\subsection{Metalogical Completeness}\index{metalogical completeness}

In the system of Example~2, the
following are provable pre-statements (and their reducts are
provable statements):
\begin{eqnarray*}
      & \langle\{\{x,y\}\},T,\varnothing,
               \langle \vdash\lnot\forall x\lnot x=y
               \rangle\rangle & \\
     &  \langle\varnothing,T,\varnothing,
               \langle \vdash\lnot\forall x\lnot x=x
               \rangle\rangle &
\end{eqnarray*}
whereas the following pre-statement is not to my knowledge provable (but
in any case we will pretend it's not for sake of illustration):
\begin{eqnarray*}
     &  \langle\varnothing,T,\varnothing,
               \langle \vdash\lnot\forall x\lnot x=y
               \rangle\rangle &
\end{eqnarray*}
In other words, we can prove ``$\lnot\forall x\lnot x=y$ where $x$ and $y$ are
distinct'' and separately prove ``$\lnot\forall x\lnot x=x$'', but we can't
prove the combined general case ``$\lnot\forall x\lnot x=y$'' that has no
proviso.  Now this does not compromise logical completeness, because the
variables are really metavariables and the two provable cases together cover
all possible cases.  The third case can be considered a metatheorem whose
direct proof, using the system of Example~2, lies outside the capability of the
formal system.

Also, in the system of Example~2 the following pre-statement is not to my
knowledge provable (again, a conjecture that we will pretend to be the case):
\begin{eqnarray*}
     & \langle\varnothing,T,\varnothing,
               \langle \vdash(\forall x\, \varphi\to\varphi)
               \rangle\rangle &
\end{eqnarray*}
Instead, we can only prove specific cases of $\varphi$ involving individual
metavariables, and by induction on formula length, prove as a metatheorem
outside of our formal system the general statement above.  The details of this
proof are found in \cite{Kalish}.

There does, however, exist a system of predicate calculus in which all such
``simple metatheorems'' as those above can be proved directly, and we present
it in Example~3. A {\em simple metatheorem}\index{simple metatheorem}
is any statement of the formal
system of Example~2 where all distinct variable restrictions consist of either
two individual metavariables or an individual metavariable and a wff
metavariable, and which is provable by combining cases outside the system as
above.  A system is {\em metalogically complete}\index{metalogical
completeness} if all of its simple
metatheorems are (directly) provable statements. The precise definition of
``simple metatheorem'' and the proof of the ``metalogical completeness'' of
Example~3 is found in Remark 9.6 and Theorem 9.7 of \cite{Megill}.\index{Megill,
Norman}

\begin{sloppy}
\subsection{Example~3---Metalogically Complete Predicate
Calculus with
Equality}
\end{sloppy}

For simplicity we will assume there is one binary predicate $R$;
this system suffices for set theory, where the $R$ is of course the $\in$
predicate.  We label the axioms as they appear in \cite{Megill}.  This
system is logically equivalent to that of Example~2 (when the latter is
restricted to this single binary predicate) but is also metalogically
complete.\index{metalogical completeness}

Let
\begin{itemize}
  \item[] $\mbox{\em CN}=\{\mbox{wff}, \mbox{var}, \vdash, \to, \lnot, (,),\forall,=,R\}$.
  \item[] $\mbox{\em VR}=\{\varphi,\psi,\chi,\ldots\}\cup\{x,y,z,\ldots\}$.
  \item[] $T = \{\langle \mbox{wff\ } \varphi\rangle,
             \langle \mbox{wff\ } \psi\rangle,
             \langle \mbox{wff\ } \chi\rangle,\ldots\}\cup
       \{\langle \mbox{var\ } x\rangle, \langle \mbox{var\ } y\rangle, \langle
       \mbox{var\ }z\rangle,\ldots\}$.

\noindent Then
  $\Gamma$ consists of the reducts of the following pre-statements:
    \begin{itemize}
      \item[] $\langle\varnothing,T,\varnothing,
               \langle \mbox{wff\ }(\varphi\to\psi)\rangle\rangle$
      \item[] $\langle\varnothing,T,\varnothing,
               \langle \mbox{wff\ }\lnot\varphi\rangle\rangle$
      \item[] $\langle\varnothing,T,\varnothing,
               \langle \mbox{wff\ }\forall x\,\varphi\rangle\rangle$
      \item[] $\langle\varnothing,T,\varnothing,
               \langle \mbox{wff\ }x=y\rangle\rangle$
      \item[] $\langle\varnothing,T,\varnothing,
               \langle \mbox{wff\ }Rxy\rangle\rangle$
      \item[(C1$'$)] $\langle\varnothing,T,\varnothing,
               \langle \vdash(\varphi\to(\psi\to\varphi))
               \rangle\rangle$
      \item[(C2$'$)] $\langle\varnothing,T,
               \varnothing,
               \langle \vdash((\varphi\to(\psi\to\chi))\to
               ((\varphi\to\psi)\to(\varphi\to\chi)))
               \rangle\rangle$
      \item[(C3$'$)] $\langle\varnothing,T,
               \varnothing,
               \langle \vdash((\lnot\varphi\to\lnot\psi)\to
               (\psi\to\varphi))\rangle\rangle$
      \item[(C4$'$)] $\langle\varnothing,T,
               \varnothing,
               \langle \vdash(\forall x(\forall x\,\varphi\to\psi)\to
                 (\forall x\,\varphi\to\forall x\,\psi))\rangle\rangle$
      \item[(C5$'$)] $\langle\varnothing,T,
               \varnothing,
               \langle \vdash(\forall x\,\varphi\to\varphi)\rangle\rangle$
      \item[(C6$'$)] $\langle\varnothing,T,
               \varnothing,
               \langle \vdash(\forall x\forall y\,\varphi\to
                 \forall y\forall x\,\varphi)\rangle\rangle$
      \item[(C7$'$)] $\langle\varnothing,T,
               \varnothing,
               \langle \vdash(\lnot\varphi\to\forall x\lnot\forall x\,\varphi
                 )\rangle\rangle$
      \item[(C8$'$)] $\langle\varnothing,T,
               \varnothing,
               \langle \vdash(x=y\to(x=z\to y=z))\rangle\rangle$
      \item[(C9$'$)] $\langle\varnothing,T,
               \varnothing,
               \langle \vdash(\lnot\forall x\, x=y\to(\lnot\forall x\, x=z\to
                 (y=z\to\forall x\, y=z)))\rangle\rangle$
      \item[(C10$'$)] $\langle\varnothing,T,
               \varnothing,
               \langle \vdash(\forall x(x=y\to\forall x\,\varphi)\to
                 \varphi))\rangle\rangle$
      \item[(C11$'$)] $\langle\varnothing,T,
               \varnothing,
               \langle \vdash(\forall x\, x=y\to(\forall x\,\varphi
               \to\forall y\,\varphi))\rangle\rangle$
      \item[(C12$'$)] $\langle\varnothing,T,
               \varnothing,
               \langle \vdash(x=y\to(Rxz\to Ryz))\rangle\rangle$
      \item[(C13$'$)] $\langle\varnothing,T,
               \varnothing,
               \langle \vdash(x=y\to(Rzx\to Rzy))\rangle\rangle$
      \item[(C15$'$)] $\langle\varnothing,T,
               \varnothing,
               \langle \vdash(\lnot\forall x\, x=y\to(x=y\to(\varphi
                 \to\forall x(x=y\to\varphi))))\rangle\rangle$
      \item[(C16$'$)] $\langle\{\{x,y\}\},T,
               \varnothing,
               \langle \vdash(\forall x\, x=y\to(\varphi\to\forall x\,\varphi)
                 )\rangle\rangle$
      \item[(C5)] $\langle\{\{x,\varphi\}\},T,\varnothing,
               \langle \vdash(\varphi\to\forall x\,\varphi)
               \rangle\rangle$
      \item[(MP)] $\langle\varnothing,T,
               \{\langle\vdash(\varphi\to\psi)\rangle,
                 \langle\vdash\varphi\rangle\},
               \langle\vdash\psi\rangle\rangle$
      \item[(Gen)] $\langle\varnothing,T,
               \{\langle\vdash\varphi\rangle\},
               \langle\vdash\forall x\,\varphi\rangle\rangle$
    \end{itemize}
\end{itemize}

While it is known that these axioms are ``metalogically complete,'' it is
not known whether they are independent (i.e.\ none is
redundant) in the metalogical sense; specifically, whether any axiom (possibly
with additional non-mandatory distinct-variable restrictions, for use with any
dummy variables in its proof) is provable from the others.  Note that
metalogical independence is a weaker requirement than independence in the
usual logical sense.  Not all of the above axioms are logically independent:
for example, C9$'$ can be proved as a metatheorem from the others, outside the
formal system, by combining the possible cases of distinct variables.

\subsection{Example~4---Adding Definitions}\index{definition}
There are several ways to add definitions to a formal system.  Probably the
most proper way is to consider definitions not as part of the formal system at
all but rather as abbreviations that are part of the expository metalogic
outside the formal system.  For convenience, though, we may use the formal
system itself to incorporate definitions, adding them as axiomatic extensions
to the system.  This could be done by adding a constant representing the
concept ``is defined as'' along with axioms for it. But there is a nicer way,
at least in this writer's opinion, that introduces definitions as direct
extensions to the language rather than as extralogical primitive notions.  We
introduce additional logical connectives and provide axioms for them.  For
systems of logic such as Examples 1 through 3, the additional axioms must be
conservative in the sense that no wff of the original system that was not a
theorem (when the initial term ``wff'' is replaced by ``$\vdash$'' of course)
becomes a theorem of the extended system.  In this example we extend Example~3
(or 2) with standard abbreviations of logic.

We extend $\mbox{\em CN}$ of Example~3 with new constants $\{\leftrightarrow,
\wedge,\vee,\exists\}$, corresponding to logical equivalence,\index{logical
equivalence ($\leftrightarrow$)}\index{biconditional ($\leftrightarrow$)}
conjunction,\index{conjunction ($\wedge$)} disjunction,\index{disjunction
($\vee$)} and the existential quantifier.\index{existential quantifier
($\exists$)}  We extend $\Gamma$ with the axiomatic statements that are
the reducts of the following pre-statements:
\begin{list}{}{\itemsep 0.0pt}
      \item[] $\langle\varnothing,T,\varnothing,
               \langle \mbox{wff\ }(\varphi\leftrightarrow\psi)\rangle\rangle$
      \item[] $\langle\varnothing,T,\varnothing,
               \langle \mbox{wff\ }(\varphi\vee\psi)\rangle\rangle$
      \item[] $\langle\varnothing,T,\varnothing,
               \langle \mbox{wff\ }(\varphi\wedge\psi)\rangle\rangle$
      \item[] $\langle\varnothing,T,\varnothing,
               \langle \mbox{wff\ }\exists x\, \varphi\rangle\rangle$
  \item[] $\langle\varnothing,T,\varnothing,
     \langle\vdash ( ( \varphi \leftrightarrow \psi ) \to
     ( \varphi \to \psi ) )\rangle\rangle$
  \item[] $\langle\varnothing,T,\varnothing,
     \langle\vdash ((\varphi\leftrightarrow\psi)\to
    (\psi\to\varphi))\rangle\rangle$
  \item[] $\langle\varnothing,T,\varnothing,
     \langle\vdash ((\varphi\to\psi)\to(
     (\psi\to\varphi)\to(\varphi
     \leftrightarrow\psi)))\rangle\rangle$
  \item[] $\langle\varnothing,T,\varnothing,
     \langle\vdash (( \varphi \wedge \psi ) \leftrightarrow\neg ( \varphi
     \to \neg \psi )) \rangle\rangle$
  \item[] $\langle\varnothing,T,\varnothing,
     \langle\vdash (( \varphi \vee \psi ) \leftrightarrow (\neg \varphi
     \to \psi )) \rangle\rangle$
  \item[] $\langle\varnothing,T,\varnothing,
     \langle\vdash (\exists x \,\varphi\leftrightarrow
     \lnot \forall x \lnot \varphi)\rangle\rangle$
\end{list}
The first three logical axioms (statements containing ``$\vdash$'') introduce
and effectively define logical equivalence, ``$\leftrightarrow$''.  The last
three use ``$\leftrightarrow$'' to effectively mean ``is defined as.''

\subsection{Example~5---ZFC Set Theory}\index{ZFC set theory}

Here we add to the system of Example~4 the axioms of Zermelo--Fraenkel set
theory with Choice.  For convenience we make use of the
definitions in Example~4.

In the $\mbox{\em CN}$ of Example~4 (which extends Example~3), we replace the symbol $R$
with the symbol $\in$.
More explicitly, we remove from $\Gamma$ of Example~4 the three
axiomatic statements containing $R$ and replace them with the
reducts of the following:
\begin{list}{}{\itemsep 0.0pt}
      \item[] $\langle\varnothing,T,\varnothing,
               \langle \mbox{wff\ }x\in y\rangle\rangle$
      \item[] $\langle\varnothing,T,
               \varnothing,
               \langle \vdash(x=y\to(x\in z\to y\in z))\rangle\rangle$
      \item[] $\langle\varnothing,T,
               \varnothing,
               \langle \vdash(x=y\to(z\in x\to z\in y))\rangle\rangle$
\end{list}
Letting $D=\{\{\alpha,\beta\}\in \mbox{\em DV}\,|\alpha,\beta\in \mbox{\em
Vr}\}$ (in other words all individual variables must be distinct), we extend
$\Gamma$ with the ZFC axioms, called
\index{Axiom of Extensionality}
\index{Axiom of Replacement}
\index{Axiom of Union}
\index{Axiom of Power Sets}
\index{Axiom of Regularity}
\index{Axiom of Infinity}
\index{Axiom of Choice}
Extensionality, Replacement, Union, Power
Set, Regularity, Infinity, and Choice, that are the reducts of:
\begin{list}{}{\itemsep 0.0pt}
      \item[Ext] $\langle D,T,
               \varnothing,
               \langle\vdash (\forall x(x\in y\leftrightarrow x \in z)\to y
               =z) \rangle\rangle$
      \item[Rep] $\langle D,T,
               \varnothing,
               \langle\vdash\exists x ( \exists y \forall z (\varphi \to z = y
                        ) \to
                        \forall z ( z \in x \leftrightarrow \exists x ( x \in
                        y \wedge \forall y\,\varphi ) ) )\rangle\rangle$
      \item[Un] $\langle D,T,
               \varnothing,
               \langle\vdash \exists x \forall y ( \exists x ( y \in x \wedge
               x \in z ) \to y \in x ) \rangle\rangle$
      \item[Pow] $\langle D,T,
               \varnothing,
               \langle\vdash \exists x \forall y ( \forall x ( x \in y \to x
               \in z ) \to y \in x ) \rangle\rangle$
      \item[Reg] $\langle D,T,
               \varnothing,
               \langle\vdash (  x \in y \to
                 \exists x ( x \in y \wedge \forall z ( z \in x \to \lnot z
                \in y ) ) ) \rangle\rangle$
      \item[Inf] $\langle D,T,
               \varnothing,
               \langle\vdash \exists x(y\in x\wedge\forall y(y\in
               x\to
               \exists z(y \in z\wedge z\in x))) \rangle\rangle$
      \item[AC] $\langle D,T,
               \varnothing,
               \langle\vdash \exists x \forall y \forall z ( ( y \in z
               \wedge z \in w ) \to \exists w \forall y ( \exists w
              ( ( y \in z \wedge z \in w ) \wedge ( y \in w \wedge w \in x
              ) ) \leftrightarrow y = w ) ) \rangle\rangle$
\end{list}

\subsection{Example~6---Class Notation in Set Theory}\label{class}

A powerful device that makes set theory easier (and that we have
been using all along in our informal expository language) is {\em class
abstraction notation}.\index{class abstraction}\index{abstraction class}  The
definitions we introduce are rigorously justified
as conservative by Takeuti and Zaring \cite{Takeuti}\index{Takeuti, Gaisi} or
Quine \cite{Quine}\index{Quine, Willard Van Orman}.  The key idea is to
introduce the notation $\{x|\mbox{---}\}$ which means ``the class of all $x$
such that ---'' for abstraction classes and introduce (meta)variables that
range over them.  An abstraction class may or may not be a set, depending on
whether it exists (as a set).  A class that does not exist is
called a {\em proper class}.\index{proper class}\index{class!proper}

To illustrate the use of abstraction classes we will provide some examples
of definitions that make use of them:  the empty set, class union, and
unordered pair.  Many other such definitions can be found in the
Metamath set theory database,
\texttt{set.mm}.\index{set theory database (\texttt{set.mm})}

% We intentionally break up the sequence of math symbols here
% because otherwise the overlong line goes beyond the page in narrow mode.
We extend $\mbox{\em CN}$ of Example~5 with new symbols $\{$
$\mbox{class},$ $\{,$ $|,$ $\},$ $\varnothing,$ $\cup,$ $,$ $\}$
where the inner braces and last comma are
constant symbols. (As before,
our dual use of some mathematical symbols for both our expository
language and as primitives of the formal system should be clear from context.)

We extend $\mbox{\em VR}$ of Example~5 with a set of {\em class
variables}\index{class variable}
$\{A,B,C,\ldots\}$. We extend the $T$ of Example~5 with $\{\langle
\mbox{class\ } A\rangle, \langle \mbox{class\ }B\rangle, \langle \mbox{class\ }
C\rangle,\ldots\}$.

To
introduce our definitions,
we add to $\Gamma$ of Example~5 the axiomatic statements
that are the reducts of the following pre-statements:
\begin{list}{}{\itemsep 0.0pt}
      \item[] $\langle\varnothing,T,\varnothing,
               \langle \mbox{class\ }x\rangle\rangle$
      \item[] $\langle\varnothing,T,\varnothing,
               \langle \mbox{class\ }\{x|\varphi\}\rangle\rangle$
      \item[] $\langle\varnothing,T,\varnothing,
               \langle \mbox{wff\ }A=B\rangle\rangle$
      \item[] $\langle\varnothing,T,\varnothing,
               \langle \mbox{wff\ }A\in B\rangle\rangle$
      \item[Ab] $\langle\varnothing,T,\varnothing,
               \langle \vdash ( y \in \{ x |\varphi\} \leftrightarrow
                  ( ( x = y \to\varphi) \wedge \exists x ( x = y
                  \wedge\varphi) ))
               \rangle\rangle$
      \item[Eq] $\langle\{\{x,A\},\{x,B\}\},T,\varnothing,
               \langle \vdash ( A = B \leftrightarrow
               \forall x ( x \in A \leftrightarrow x \in B ) )
               \rangle\rangle$
      \item[El] $\langle\{\{x,A\},\{x,B\}\},T,\varnothing,
               \langle \vdash ( A \in B \leftrightarrow \exists x
               ( x = A \wedge x \in B ) )
               \rangle\rangle$
\end{list}
Here we say that an individual variable is a class; $\{x|\varphi\}$ is a
class; and we extend the definition of a wff to include class equality and
membership.  Axiom Ab defines membership of a variable in a class abstraction;
the right-hand side can be read as ``the wff that results from proper
substitution of $y$ for $x$ in $\varphi$.''\footnote{Note that this definition
makes unnecessary the introduction of a separate notation similar to
$\varphi(x|y)$ for proper substitution, although we may choose to do so to be
conventional.  Incidentally, $\varphi(x|y)$ as it stands would be ambiguous in
the formal systems of our examples, since we wouldn't know whether
$\lnot\varphi(x|y)$ meant $\lnot(\varphi(x|y))$ or $(\lnot\varphi)(x|y)$.
Instead, we would have to use an unambiguous variant such as $(\varphi\,
x|y)$.}  Axioms Eq and El extend the meaning of the existing equality and
membership connectives.  This is potentially dangerous and requires careful
justification.  For example, from Eq we can derive the Axiom of Extensionality
with predicate logic alone; thus in principle we should include the Axiom of
Extensionality as a logical hypothesis.  However we do not bother to do this
since we have already presupposed that axiom earlier. The distinct variable
restrictions should be read ``where $x$ does not occur in $A$ or $B$.''  We
typically do this when the right-hand side of a definition involves an
individual variable not in the expression being defined; it is done so that
the right-hand side remains independent of the particular ``dummy'' variable
we use.

We continue to add to $\Gamma$ the following definitions
(i.e. the reducts of the following pre-statements) for empty
set,\index{empty set} class union,\index{union} and unordered
pair.\index{unordered pair}  They should be self-explanatory.  Analogous to our
use of ``$\leftrightarrow$'' to define new wffs in Example~4, we use ``$=$''
to define new abstraction terms, and both may be read informally as ``is
defined as'' in this context.
\begin{list}{}{\itemsep 0.0pt}
      \item[] $\langle\varnothing,T,\varnothing,
               \langle \mbox{class\ }\varnothing\rangle\rangle$
      \item[] $\langle\varnothing,T,\varnothing,
               \langle \vdash \varnothing = \{ x | \lnot x = x \}
               \rangle\rangle$
      \item[] $\langle\varnothing,T,\varnothing,
               \langle \mbox{class\ }(A\cup B)\rangle\rangle$
      \item[] $\langle\{\{x,A\},\{x,B\}\},T,\varnothing,
               \langle \vdash ( A \cup B ) = \{ x | ( x \in A \vee x \in B ) \}
               \rangle\rangle$
      \item[] $\langle\varnothing,T,\varnothing,
               \langle \mbox{class\ }\{A,B\}\rangle\rangle$
      \item[] $\langle\{\{x,A\},\{x,B\}\},T,\varnothing,
               \langle \vdash \{ A , B \} = \{ x | ( x = A \vee x = B ) \}
               \rangle\rangle$
\end{list}

\section{Metamath as a Formal System}\label{theorymm}

This section presupposes a familiarity with the Metamath computer language.

Our theory describes formal systems and their universes.  The Metamath
language provides a way of representing these set-theoretical objects to
a computer.  A Metamath database, being a finite set of {\sc ascii}
characters, can usually describe only a subset of a formal system and
its universe, which are typically infinite.  However the database can
contain as large a finite subset of the formal system and its universe
as we wish.  (Of course a Metamath set theory database can, in
principle, indirectly describe an entire infinite formal system by
formalizing the expository language in this Appendix.)

For purpose of our discussion, we assume the Metamath database
is in the simple form described on p.~\pageref{framelist},
consisting of all constant and variable declarations at the beginning,
followed by a sequence of extended frames each
delimited by \texttt{\$\char`\{} and \texttt{\$\char`\}}.  Any Metamath database can
be converted to this form, as described on p.~\pageref{frameconvert}.

The math symbol tokens of a Metamath source file, which are declared
with \texttt{\$c} and \texttt{\$v} statements, are names we assign to
representatives of $\mbox{\em CN}$ and $\mbox{\em VR}$.  For
definiteness we could assume that the first math symbol declared as a
variable corresponds to $v_0$, the second to $v_1$, etc., although the
exact correspondence we choose is not important.

In the Metamath language, each \texttt{\$d}, \texttt{\$f}, and
 \texttt{\$e} source
statement in an extended frame (Section~\ref{frames})
corresponds respectively to a member of the
collections $D$, $T$, and $H$ in a formal system statement $\langle
D_M,T_M,H,A\rangle$.  The math symbol strings following these Metamath keywords
correspond to a variable pair (in the case of \texttt{\$d}) or an expression (for
the other two keywords). The math symbol string following a \texttt{\$a} source
statement corresponds to expression $A$ in an axiomatic statement of the
formal system; the one following a \texttt{\$p} source statement corresponds to
$A$ in a provable statement that is not axiomatic.  In other words, each
extended frame in a Metamath database corresponds to
a pre-statement of the formal system, and a frame corresponds to
a statement of the formal system.  (Don't confuse the two meanings of
``statement'' here.  A statement of the formal system corresponds to the
several statements in a Metamath database that may constitute a
frame.)

In order for the computer to verify that a formal system statement is
provable, each \texttt{\$p} source statement is accompanied by a proof.
However, the proof does not correspond to anything in the formal system
but is simply a way of communicating to the computer the information
needed for its verification.  The proof tells the computer {\em how to
construct} specific members of closure of the formal system
pre-statement corresponding to the extended frame of the \texttt{\$p}
statement.  The final result of the construction is the member of the
closure that matches the \texttt{\$p} statement.  The abstract formal
system, on the other hand, is concerned only with the {\em existence} of
members of the closure.

As mentioned on p.~\pageref{exampleref}, Examples 1 and 3--6 in the
previous Section parallel the development of logic and set theory in the
Metamath database
\texttt{set.mm}.\index{set theory database (\texttt{set.mm})} You may
find it instructive to compare them.


\chapter{The MIU System}
\label{MIU}
\index{formal system}
\index{MIU-system}

The following is a listing of the file \texttt{miu.mm}.  It is self-explanatory.

%%%%%%%%%%%%%%%%%%%%%%%%%%%%%%%%%%%%%%%%%%%%%%%%%%%%%%%%%%%%

\begin{verbatim}
$( The MIU-system:  A simple formal system $)

$( Note:  This formal system is unusual in that it allows
empty wffs.  To work with a proof, you must type
SET EMPTY_SUBSTITUTION ON before using the PROVE command.
By default, this is OFF in order to reduce the number of
ambiguous unification possibilities that have to be selected
during the construction of a proof.  $)

$(
Hofstadter's MIU-system is a simple example of a formal
system that illustrates some concepts of Metamath.  See
Douglas R. Hofstadter, _Goedel, Escher, Bach:  An Eternal
Golden Braid_ (Vintage Books, New York, 1979), pp. 33ff. for
a description of the MIU-system.

The system has 3 constant symbols, M, I, and U.  The sole
axiom of the system is MI. There are 4 rules:
     Rule I:  If you possess a string whose last letter is I,
     you can add on a U at the end.
     Rule II:  Suppose you have Mx.  Then you may add Mxx to
     your collection.
     Rule III:  If III occurs in one of the strings in your
     collection, you may make a new string with U in place
     of III.
     Rule IV:  If UU occurs inside one of your strings, you
     can drop it.
Unfortunately, Rules III and IV do not have unique results:
strings could have more than one occurrence of III or UU.
This requires that we introduce the concept of an "MIU
well-formed formula" or wff, which allows us to construct
unique symbol sequences to which Rules III and IV can be
applied.
$)

$( First, we declare the constant symbols of the language.
Note that we need two symbols to distinguish the assertion
that a sequence is a wff from the assertion that it is a
theorem; we have arbitrarily chosen "wff" and "|-". $)
      $c M I U |- wff $. $( Declare constants $)

$( Next, we declare some variables. $)
     $v x y $.

$( Throughout our theory, we shall assume that these
variables represent wffs. $)
 wx   $f wff x $.
 wy   $f wff y $.

$( Define MIU-wffs.  We allow the empty sequence to be a
wff. $)

$( The empty sequence is a wff. $)
 we   $a wff $.
$( "M" after any wff is a wff. $)
 wM   $a wff x M $.
$( "I" after any wff is a wff. $)
 wI   $a wff x I $.
$( "U" after any wff is a wff. $)
 wU   $a wff x U $.

$( Assert the axiom. $)
 ax   $a |- M I $.

$( Assert the rules. $)
 ${
   Ia   $e |- x I $.
$( Given any theorem ending with "I", it remains a theorem
if "U" is added after it.  (We distinguish the label I_
from the math symbol I to conform to the 24-Jun-2006
Metamath spec.) $)
   I_    $a |- x I U $.
 $}
 ${
IIa  $e |- M x $.
$( Given any theorem starting with "M", it remains a theorem
if the part after the "M" is added again after it. $)
   II   $a |- M x x $.
 $}
 ${
   IIIa $e |- x I I I y $.
$( Given any theorem with "III" in the middle, it remains a
theorem if the "III" is replaced with "U". $)
   III  $a |- x U y $.
 $}
 ${
   IVa  $e |- x U U y $.
$( Given any theorem with "UU" in the middle, it remains a
theorem if the "UU" is deleted. $)
   IV   $a |- x y $.
  $}

$( Now we prove the theorem MUIIU.  You may be interested in
comparing this proof with that of Hofstadter (pp. 35 - 36).
$)
 theorem1  $p |- M U I I U $=
      we wM wU wI we wI wU we wU wI wU we wM we wI wU we wM
      wI wI wI we wI wI we wI ax II II I_ III II IV $.
\end{verbatim}\index{well-formed formula (wff)}

The \texttt{show proof /lemmon/renumber} command
yields the following display.  It is very similar
to the one in \cite[pp.~35--36]{Hofstadter}.\index{Hofstadter, Douglas R.}

\begin{verbatim}
1 ax             $a |- M I
2 1 II           $a |- M I I
3 2 II           $a |- M I I I I
4 3 I_           $a |- M I I I I U
5 4 III          $a |- M U I U
6 5 II           $a |- M U I U U I U
7 6 IV           $a |- M U I I U
\end{verbatim}

We note that Hofstadter's ``MU-puzzle,'' which asks whether
MU is a theorem of the MIU-system, cannot be answered using
the system above because the MU-puzzle is a question {\em
about} the system.  To prove the answer to the MU-puzzle,
a much more elaborate system is needed, namely one that
models the MIU-system within set theory.  (Incidentally, the
answer to the MU-puzzle is no.)

\chapter{Metamath Language EBNF}%
\label{BNF}%
\index{Metamath Language EBNF}

The following is a formal description of the basic Metamath language syntax
(with compressed proofs and support for unknown proof steps).
It is defined using the
Extended Backus--Naur Form (EBNF)\index{Extended Backus--Naur Form}\index{EBNF}
notation from W3C\index{W3C}
\textit{Extensible Markup Language (XML) 1.0 (Fifth Edition)}
(W3C Recommendation 26 November 2008) at
\url{https://www.w3.org/TR/xml/#sec-notation}.

The \texttt{database}
rule is processed until the end of the file (\texttt{EOF}).
The rules eventually require reading whitespace-separated tokens.
A token has an upper-case definition (see below)
or is a string constant in a non-token (such as \texttt{'\$a'}).
We intend for this to be correct, but if there is a conflict the
rules of section \ref{spec} govern. That section also discusses
non-syntax restrictions not shown here
(e.g., that each new label token
defined in a \texttt{hypothesis-stmt} or \texttt{assert-stmt}
must be unique).

\begin{verbatim}
database ::= outermost-scope-stmt*

outermost-scope-stmt ::=
  include-stmt | constant-stmt | stmt

/* File inclusion command; process file as a database.
   Databases should NOT have a comment in the filename. */
include-stmt ::= '$[' filename '$]'

/* Constant symbols declaration. */
constant-stmt ::= '$c' constant+ '$.'

/* A normal statement can occur in any scope. */
stmt ::= block | variable-stmt | disjoint-stmt |
  hypothesis-stmt | assert-stmt

/* A block. You can have 0 statements in a block. */
block ::= '${' stmt* '$}'

/* Variable symbols declaration. */
variable-stmt ::= '$v' variable+ '$.'

/* Disjoint variables. Simple disjoint statements have
   2 variables, i.e., "variable*" is empty for them. */
disjoint-stmt ::= '$d' variable variable variable* '$.'

hypothesis-stmt ::= floating-stmt | essential-stmt

/* Floating (variable-type) hypothesis. */
floating-stmt ::= LABEL '$f' typecode variable '$.'

/* Essential (logical) hypothesis. */
essential-stmt ::= LABEL '$e' typecode MATH-SYMBOL* '$.'

assert-stmt ::= axiom-stmt | provable-stmt

/* Axiomatic assertion. */
axiom-stmt ::= LABEL '$a' typecode MATH-SYMBOL* '$.'

/* Provable assertion. */
provable-stmt ::= LABEL '$p' typecode MATH-SYMBOL*
  '$=' proof '$.'

/* A proof. Proofs may be interspersed by comments.
   If '?' is in a proof it's an "incomplete" proof. */
proof ::= uncompressed-proof | compressed-proof
uncompressed-proof ::= (LABEL | '?')+
compressed-proof ::= '(' LABEL* ')' COMPRESSED-PROOF-BLOCK+

typecode ::= constant

filename ::= MATH-SYMBOL /* No whitespace or '$' */
constant ::= MATH-SYMBOL
variable ::= MATH-SYMBOL
\end{verbatim}

\needspace{2\baselineskip}
A \texttt{frame} is a sequence of 0 or more
\texttt{disjoint-{\allowbreak}stmt} and
\texttt{hypotheses-{\allowbreak}stmt} statements
(possibly interleaved with other non-\texttt{assert-stmt} statements)
followed by one \texttt{assert-stmt}.

\needspace{3\baselineskip}
Here are the rules for lexical processing (tokenization) beyond
the constant tokens shown above.
By convention these tokenization rules have upper-case names.
Every token is read for the longest possible length.
Whitespace-separated tokens are read sequentially;
note that the separating whitespace and \texttt{\$(} ... \texttt{\$)}
comments are skipped.

If a token definition uses another token definition, the whole thing
is considered a single token.
A pattern that is only part of a full token has a name beginning
with an underscore (``\_'').
An implementation could tokenize many tokens as a
\texttt{PRINTABLE-SEQUENCE}
and then check if it meets the more specific rule shown here.

Comments do not nest, and both \texttt{\$(} and \texttt{\$)}
have to be surrounded
by at least one whitespace character (\texttt{\_WHITECHAR}).
Technically comments end without consuming the trailing
\texttt{\_WHITECHAR}, but the trailing
\texttt{\_WHITECHAR} gets ignored anyway so we ignore that detail here.
Metamath language processors
are not required to support \texttt{\$)} followed
immediately by a bare end-of-file, because the closing
comment symbol is supposed to be followed by a
\texttt{\_WHITECHAR} such as a newline.

\begin{verbatim}
PRINTABLE-SEQUENCE ::= _PRINTABLE-CHARACTER+

MATH-SYMBOL ::= (_PRINTABLE-CHARACTER - '$')+

/* ASCII non-whitespace printable characters */
_PRINTABLE-CHARACTER ::= [#x21-#x7e]

LABEL ::= ( _LETTER-OR-DIGIT | '.' | '-' | '_' )+

_LETTER-OR-DIGIT ::= [A-Za-z0-9]

COMPRESSED-PROOF-BLOCK ::= ([A-Z] | '?')+

/* Define whitespace between tokens. The -> SKIP
   means that when whitespace is seen, it is
   skipped and we simply read again. */
WHITESPACE ::= (_WHITECHAR+ | _COMMENT) -> SKIP

/* Comments. $( ... $) and do not nest. */
_COMMENT ::= '$(' (_WHITECHAR+ (PRINTABLE-SEQUENCE - '$)'))*
  _WHITECHAR+ '$)' _WHITECHAR

/* Whitespace: (' ' | '\t' | '\r' | '\n' | '\f') */
_WHITECHAR ::= [#x20#x09#x0d#x0a#x0c]
\end{verbatim}
% This EBNF was developed as a collaboration between
% David A. Wheeler\index{Wheeler, David A.},
% Mario Carneiro\index{Carneiro, Mario}, and
% Benoit Jubin\index{Jubin, Benoit}, inspired by a request
% (and a lot of initial work) by Benoit Jubin.
%
% \chapter{Disclaimer and Trademarks}
%
% Information in this document is subject to change without notice and does not
% represent a commitment on the part of Norman Megill.
% \vspace{2ex}
%
% \noindent Norman D. Megill makes no warranties, either express or implied,
% regarding the Metamath computer software package.
%
% \vspace{2ex}
%
% \noindent Any trademarks mentioned in this book are the property of
% their respective owners.  The name ``Metamath'' is a trademark of
% Norman Megill.
%
\cleardoublepage
\phantomsection  % fixes the link anchor
\addcontentsline{toc}{chapter}{\bibname}

\bibliography{metamath}
%% metamath.tex - Version of 2-Jun-2019
% If you change the date above, also change the "Printed date" below.
% SPDX-License-Identifier: CC0-1.0
%
%                              PUBLIC DOMAIN
%
% This file (specifically, the version of this file with the above date)
% has been released into the Public Domain per the
% Creative Commons CC0 1.0 Universal (CC0 1.0) Public Domain Dedication
% https://creativecommons.org/publicdomain/zero/1.0/
%
% The public domain release applies worldwide.  In case this is not
% legally possible, the right is granted to use the work for any purpose,
% without any conditions, unless such conditions are required by law.
%
% Several short, attributed quotations from copyrighted works
% appear in this file under the ``fair use'' provision of Section 107 of
% the United States Copyright Act (Title 17 of the {\em United States
% Code}).  The public-domain status of this file is not applicable to
% those quotations.
%
% Norman Megill - email: nm(at)alum(dot)mit(dot)edu
%
% David A. Wheeler also donates his improvements to this file to the
% public domain per the CC0.  He works at the Institute for Defense Analyses
% (IDA), but IDA has agreed that this Metamath work is outside its "lane"
% and is not a work by IDA.  This was specifically confirmed by
% Margaret E. Myers (Division Director of the Information Technology
% and Systems Division) on 2019-05-24 and by Ben Lindorf (General Counsel)
% on 2019-05-22.

% This file, 'metamath.tex', is self-contained with everything needed to
% generate the the PDF file 'metamath.pdf' (the _Metamath_ book) on
% standard LaTeX 2e installations.  The auxiliary files are embedded with
% "filecontents" commands.  To generate metamath.pdf file, run these
% commands under Linux or Cygwin in the directory that contains
% 'metamath.tex':
%
%   rm -f realref.sty metamath.bib
%   touch metamath.ind
%   pdflatex metamath
%   pdflatex metamath
%   bibtex metamath
%   makeindex metamath
%   pdflatex metamath
%   pdflatex metamath
%
% The warnings that occur in the initial runs of pdflatex can be ignored.
% For the final run,
%
%   egrep -i 'error|warn' metamath.log
%
% should show exactly these 5 warnings:
%
%   LaTeX Warning: File `realref.sty' already exists on the system.
%   LaTeX Warning: File `metamath.bib' already exists on the system.
%   LaTeX Font Warning: Font shape `OMS/cmtt/m/n' undefined
%   LaTeX Font Warning: Font shape `OMS/cmtt/bx/n' undefined
%   LaTeX Font Warning: Some font shapes were not available, defaults
%       substituted.
%
% Search for "Uncomment" below if you want to suppress hyperlink boxes
% in the PDF output file
%
% TYPOGRAPHICAL NOTES:
% * It is customary to use an en dash (--) to "connect" names of different
%   people (and to denote ranges), and use a hyphen (-) for a
%   single compound name. Examples of connected multiple people are
%   Zermelo--Fraenkel, Schr\"{o}der--Bernstein, Tarski--Grothendieck,
%   Hewlett--Packard, and Backus--Naur.  Examples of a single person with
%   a compound name include Levi-Civita, Mittag-Leffler, and Burali-Forti.
% * Use non-breaking spaces after page abbreviations, e.g.,
%   p.~\pageref{note2002}.
%
% --------------------------- Start of realref.sty -----------------------------
\begin{filecontents}{realref.sty}
% Save the following as realref.sty.
% You can then use it with \usepackage{realref}
%
% This has \pageref jumping to the page on which the ref appears,
% \ref jumping to the point of the anchor, and \sectionref
% jumping to the start of section.
%
% Author:  Anthony Williams
%          Software Engineer
%          Nortel Networks Optical Components Ltd
% Date:    9 Nov 2001 (posted to comp.text.tex)
%
% The following declaration was made by Anthony Williams on
% 24 Jul 2006 (private email to Norman Megill):
%
%   ``I hereby donate the code for realref.sty posted on the
%   comp.text.tex newsgroup on 9th November 2001, accessible from
%   http://groups.google.com/group/comp.text.tex/msg/5a0e1cc13ea7fbb2
%   to the public domain.''
%
\ProvidesPackage{realref}
\RequirePackage[plainpages=false,pdfpagelabels=true]{hyperref}
\def\realref@anchorname{}
\AtBeginDocument{%
% ensure every label is a possible hyperlink target
\let\realref@oldrefstepcounter\refstepcounter%
\DeclareRobustCommand{\refstepcounter}[1]{\realref@oldrefstepcounter{#1}
\edef\realref@anchorname{\string #1.\@currentlabel}%
}%
\let\realref@oldlabel\label%
\DeclareRobustCommand{\label}[1]{\realref@oldlabel{#1}\hypertarget{#1}{}%
\@bsphack\protected@write\@auxout{}{%
    \string\expandafter\gdef\protect\csname
    page@num.#1\string\endcsname{\thepage}%
    \string\expandafter\gdef\protect\csname
    ref@num.#1\string\endcsname{\@currentlabel}%
    \string\expandafter\gdef\protect\csname
    sectionref@name.#1\string\endcsname{\realref@anchorname}%
}\@esphack}%
\DeclareRobustCommand\pageref[1]{{\edef\a{\csname
            page@num.#1\endcsname}\expandafter\hyperlink{page.\a}{\a}}}%
\DeclareRobustCommand\ref[1]{{\edef\a{\csname
            ref@num.#1\endcsname}\hyperlink{#1}{\a}}}%
\DeclareRobustCommand\sectionref[1]{{\edef\a{\csname
            ref@num.#1\endcsname}\edef\b{\csname
            sectionref@name.#1\endcsname}\hyperlink{\b}{\a}}}%
}
\end{filecontents}
% ---------------------------- End of realref.sty ------------------------------

% --------------------------- Start of metamath.bib -----------------------------
\begin{filecontents}{metamath.bib}
@book{Albers, editor = "Donald J. Albers and G. L. Alexanderson",
  title = "Mathematical People",
  publisher = "Contemporary Books, Inc.",
  address = "Chicago",
  note = "[QA28.M37]",
  year = 1985 }
@book{Anderson, author = "Alan Ross Anderson and Nuel D. Belnap",
  title = "Entailment",
  publisher = "Princeton University Press",
  address = "Princeton",
  volume = 1,
  note = "[QA9.A634 1975 v.1]",
  year = 1975}
@book{Barrow, author = "John D. Barrow",
  title = "Theories of Everything:  The Quest for Ultimate Explanation",
  publisher = "Oxford University Press",
  address = "Oxford",
  note = "[Q175.B225]",
  year = 1991 }
@book{Behnke,
  editor = "H. Behnke and F. Backmann and K. Fladt and W. S{\"{u}}ss",
  title = "Fundamentals of Mathematics",
  volume = "I",
  publisher = "The MIT Press",
  address = "Cambridge, Massachusetts",
  note = "[QA37.2.B413]",
  year = 1974 }
@book{Bell, author = "J. L. Bell and M. Machover",
  title = "A Course in Mathematical Logic",
  publisher = "North-Holland",
  address = "Amsterdam",
  note = "[QA9.B3953]",
  year = 1977 }
@inproceedings{Blass, author = "Andrea Blass",
  title = "The Interaction Between Category Theory and Set Theory",
  pages = "5--29",
  booktitle = "Mathematical Applications of Category Theory (Proceedings
     of the Special Session on Mathematical Applications
     Category Theory, 89th Annual Meeting of the American Mathematical
     Society, held in Denver, Colorado January 5--9, 1983)",
  editor = "John Walter Gray",
  year = 1983,
  note = "[QA169.A47 1983]",
  publisher = "American Mathematical Society",
  address = "Providence, Rhode Island"}
@proceedings{Bledsoe, editor = "W. W. Bledsoe and D. W. Loveland",
  title = "Automated Theorem Proving:  After 25 Years (Proceedings
     of the Special Session on Automatic Theorem Proving,
     89th Annual Meeting of the American Mathematical
     Society, held in Denver, Colorado January 5--9, 1983)",
  year = 1983,
  note = "[QA76.9.A96.S64 1983]",
  publisher = "American Mathematical Society",
  address = "Providence, Rhode Island" }
@book{Boolos, author = "George S. Boolos and Richard C. Jeffrey",
  title = "Computability and Log\-ic",
  publisher = "Cambridge University Press",
  edition = "third",
  address = "Cambridge",
  note = "[QA9.59.B66 1989]",
  year = 1989 }
@book{Campbell, author = "John Campbell",
  title = "Programmer's Progress",
  publisher = "White Star Software",
  address = "Box 51623, Palo Alto, CA 94303",
  year = 1991 }
@article{DBLP:journals/corr/Carneiro14,
  author    = {Mario Carneiro},
  title     = {Conversion of {HOL} Light proofs into Metamath},
  journal   = {CoRR},
  volume    = {abs/1412.8091},
  year      = {2014},
  url       = {http://arxiv.org/abs/1412.8091},
  archivePrefix = {arXiv},
  eprint    = {1412.8091},
  timestamp = {Mon, 13 Aug 2018 16:47:05 +0200},
  biburl    = {https://dblp.org/rec/bib/journals/corr/Carneiro14},
  bibsource = {dblp computer science bibliography, https://dblp.org}
}
@article{CarneiroND,
  author    = {Mario Carneiro},
  title     = {Natural Deductions in the Metamath Proof Language},
  url       = {http://us.metamath.org/ocat/natded.pdf},
  year      = 2014
}
@inproceedings{Chou, author = "Shang-Ching Chou",
  title = "Proving Elementary Geometry Theorems Using {W}u's Algorithm",
  pages = "243--286",
  booktitle = "Automated Theorem Proving:  After 25 Years (Proceedings
     of the Special Session on Automatic Theorem Proving,
     89th Annual Meeting of the American Mathematical
     Society, held in Denver, Colorado January 5--9, 1983)",
  editor = "W. W. Bledsoe and D. W. Loveland",
  year = 1983,
  note = "[QA76.9.A96.S64 1983]",
  publisher = "American Mathematical Society",
  address = "Providence, Rhode Island" }
@book{Clemente, author = "Daniel Clemente Laboreo",
  title = "Introduction to natural deduction",
  year = 2014,
  url = "http://www.danielclemente.com/logica/dn.en.pdf" }
@incollection{Courant, author = "Richard Courant and Herbert Robbins",
  title = "Topology",
  pages = "573--590",
  booktitle = "The World of Mathematics, Volume One",
  editor = "James R. Newman",
  publisher = "Simon and Schuster",
  address = "New York",
  note = "[QA3.W67 1988]",
  year = 1956 }
@book{Curry, author = "Haskell B. Curry",
  title = "Foundations of Mathematical Logic",
  publisher = "Dover Publications, Inc.",
  address = "New York",
  note = "[QA9.C976 1977]",
  year = 1977 }
@book{Davis, author = "Philip J. Davis and Reuben Hersh",
  title = "The Mathematical Experience",
  publisher = "Birkh{\"{a}}user Boston",
  address = "Boston",
  note = "[QA8.4.D37 1982]",
  year = 1981 }
@incollection{deMillo,
  author = "Richard de Millo and Richard Lipton and Alan Perlis",
  title = "Social Processes and Proofs of Theorems and Programs",
  pages = "267--285",
  booktitle = "New Directions in the Philosophy of Mathematics",
  editor = "Thomas Tymoczko",
  publisher = "Birkh{\"{a}}user Boston, Inc.",
  address = "Boston",
  note = "[QA8.6.N48 1986]",
  year = 1986 }
@book{Edwards, author = "Robert E. Edwards",
  title = "A Formal Background to Mathematics",
  publisher = "Springer-Verlag",
  address = "New York",
  note = "[QA37.2.E38 v.1a]",
  year = 1979 }
@book{Enderton, author = "Herbert B. Enderton",
  title = "Elements of Set Theory",
  publisher = "Academic Press, Inc.",
  address = "San Diego",
  note = "[QA248.E5]",
  year = 1977 }
@book{Goodstein, author = "R. L. Goodstein",
  title = "Development of Mathematical Logic",
  publisher = "Springer-Verlag New York Inc.",
  address = "New York",
  note = "[QA9.G6554]",
  year = 1971 }
@book{Guillen, author = "Michael Guillen",
  title = "Bridges to Infinity",
  publisher = "Jeremy P. Tarcher, Inc.",
  address = "Los Angeles",
  note = "[QA93.G8]",
  year = 1983 }
@book{Hamilton, author = "Alan G. Hamilton",
  title = "Logic for Mathematicians",
  edition = "revised",
  publisher = "Cambridge University Press",
  address = "Cambridge",
  note = "[QA9.H298]",
  year = 1988 }
@unpublished{Harrison, author = "John Robert Harrison",
  title = "Metatheory and Reflection in Theorem Proving:
    A Survey and Critique",
  note = "Technical Report
    CRC-053.
    SRI Cambridge,
    Millers Yard, Cambridge, UK,
    1995.
    Available on the Web as
{\verb+http:+}\-{\verb+//www.cl.cam.ac.uk/users/jrh/papers/reflect.html+}"}
@TECHREPORT{Harrison-thesis,
        author          = "John Robert Harrison",
        title           = "Theorem Proving with the Real Numbers",
        institution   = "University of Cambridge Computer
                         Lab\-o\-ra\-to\-ry",
        address         = "New Museums Site, Pembroke Street, Cambridge,
                           CB2 3QG, UK",
        year            = 1996,
        number          = 408,
        type            = "Technical Report",
        note            = "Author's PhD thesis,
   available on the Web at
{\verb+http:+}\-{\verb+//www.cl.cam.ac.uk+}\-{\verb+/users+}\-{\verb+/jrh+}%
\-{\verb+/papers+}\-{\verb+/thesis.html+}"}
@book{Herrlich, author = "Horst Herrlich and George E. Strecker",
  title = "Category Theory:  An Introduction",
  publisher = "Allyn and Bacon Inc.",
  address = "Boston",
  note = "[QA169.H567]",
  year = 1973 }
@article{Hindley, author = "J. Roger Hindley and David Meredith",
  title = "Principal Type-Schemes and Condensed Detachment",
  journal = "The Journal of Symbolic Logic",
  volume = 55,
  year = 1990,
  note = "[QA.J87]",
  pages = "90--105" }
@book{Hofstadter, author = "Douglas R. Hofstadter",
  title = "G{\"{o}}del, Escher, Bach",
  publisher = "Basic Books, Inc.",
  address = "New York",
  note = "[QA9.H63 1980]",
  year = 1979 }
@article{Indrzejczak, author= "Andrzej Indrzejczak",
  title = "Natural Deduction, Hybrid Systems and Modal Logic",
  journal = "Trends in Logic",
  volume = 30,
  publisher = "Springer",
  year = 2010 }
@article{Kalish, author = "D. Kalish and R. Montague",
  title = "On {T}arski's Formalization of Predicate Logic with Identity",
  journal = "Archiv f{\"{u}}r Mathematische Logik und Grundlagenfor\-schung",
  volume = 7,
  year = 1965,
  note = "[QA.A673]",
  pages = "81--101" }
@article{Kalman, author = "J. A. Kalman",
  title = "Condensed Detachment as a Rule of Inference",
  journal = "Studia Logica",
  volume = 42,
  number = 4,
  year = 1983,
  note = "[B18.P6.S933]",
  pages = "443-451" }
@book{Kline, author = "Morris Kline",
  title = "Mathematical Thought from Ancient to Modern Times",
  publisher = "Oxford University Press",
  address = "New York",
  note = "[QA21.K516 1990 v.3]",
  year = 1972 }
@book{Klinel, author = "Morris Kline",
  title = "Mathematics, The Loss of Certainty",
  publisher = "Oxford University Press",
  address = "New York",
  note = "[QA21.K525]",
  year = 1980 }
@book{Kramer, author = "Edna E. Kramer",
  title = "The Nature and Growth of Modern Mathematics",
  publisher = "Princeton University Press",
  address = "Princeton, New Jersey",
  note = "[QA93.K89 1981]",
  year = 1981 }
@article{Knill, author = "Oliver Knill",
  title = "Some Fundamental Theorems in Mathematics",
  year = "2018",
  url = "https://arxiv.org/abs/1807.08416" }
@book{Landau, author = "Edmund Landau",
  title = "Foundations of Analysis",
  publisher = "Chelsea Publishing Company",
  address = "New York",
  edition = "second",
  note = "[QA241.L2541 1960]",
  year = 1960 }
@article{Leblanc, author = "Hugues Leblanc",
  title = "On {M}eyer and {L}ambert's Quantificational Calculus {FQ}",
  journal = "The Journal of Symbolic Logic",
  volume = 33,
  year = 1968,
  note = "[QA.J87]",
  pages = "275--280" }
@article{Lejewski, author = "Czeslaw Lejewski",
  title = "On Implicational Definitions",
  journal = "Studia Logica",
  volume = 8,
  year = 1958,
  note = "[B18.P6.S933]",
  pages = "189--208" }
@book{Levy, author = "Azriel Levy",
  title = "Basic Set Theory",
  publisher = "Dover Publications",
  address = "Mineola, NY",
  year = "2002"
}
@book{Margaris, author = "Angelo Margaris",
  title = "First Order Mathematical Logic",
  publisher = "Blaisdell Publishing Company",
  address = "Waltham, Massachusetts",
  note = "[QA9.M327]",
  year = 1967}
@book{Manin, author = "Yu I. Manin",
  title = "A Course in Mathematical Logic",
  publisher = "Springer-Verlag",
  address = "New York",
  note = "[QA9.M29613]",
  year = "1977" }
@article{Mathias, author = "Adrian R. D. Mathias",
  title = "A Term of Length 4,523,659,424,929",
  journal = "Synthese",
  volume = 133,
  year = 2002,
  note = "[Q.S993]",
  pages = "75--86" }
@article{Megill, author = "Norman D. Megill",
  title = "A Finitely Axiomatized Formalization of Predicate Calculus
     with Equality",
  journal = "Notre Dame Journal of Formal Logic",
  volume = 36,
  year = 1995,
  note = "[QA.N914]",
  pages = "435--453" }
@unpublished{Megillc, author = "Norman D. Megill",
  title = "A Shorter Equivalent of the Axiom of Choice",
  month = "June",
  note = "Unpublished",
  year = 1991 }
@article{MegillBunder, author = "Norman D. Megill and Martin W.
    Bunder",
  title = "Weaker {D}-Complete Logics",
  journal = "Journal of the IGPL",
  volume = 4,
  year = 1996,
  pages = "215--225",
  note = "Available on the Web at
{\verb+http:+}\-{\verb+//www.mpi-sb.mpg.de+}\-{\verb+/igpl+}%
\-{\verb+/Journal+}\-{\verb+/V4-2+}\-{\verb+/#Megill+}"}
}
@book{Mendelson, author = "Elliott Mendelson",
  title = "Introduction to Mathematical Logic",
  edition = "second",
  publisher = "D. Van Nostrand Company, Inc.",
  address = "New York",
  note = "[QA9.M537 1979]",
  year = 1979 }
@article{Meredith, author = "David Meredith",
  title = "In Memoriam {C}arew {A}rthur {M}eredith (1904-1976)",
  journal = "Notre Dame Journal of Formal Logic",
  volume = 18,
  year = 1977,
  note = "[QA.N914]",
  pages = "513--516" }
@article{CAMeredith, author = "C. A. Meredith",
  title = "Single Axioms for the Systems ({C},{N}), ({C},{O}) and ({A},{N})
      of the Two-Valued Propositional Calculus",
  journal = "The Journal of Computing Systems",
  volume = 3,
  year = 1953,
  pages = "155--164" }
@article{Monk, author = "J. Donald Monk",
  title = "Provability With Finitely Many Variables",
  journal = "The Journal of Symbolic Logic",
  volume = 27,
  year = 1971,
  note = "[QA.J87]",
  pages = "353--358" }
@article{Monks, author = "J. Donald Monk",
  title = "Substitutionless Predicate Logic With Identity",
  journal = "Archiv f{\"{u}}r Mathematische Logik und Grundlagenfor\-schung",
  volume = 7,
  year = 1965,
  pages = "103--121" }
  %% Took out this from above to prevent LaTeX underfull warning:
  % note = "[QA.A673]",
@book{Moore, author = "A. W. Moore",
  title = "The Infinite",
  publisher = "Routledge",
  address = "New York",
  note = "[BD411.M59]",
  year = 1989}
@book{Munkres, author = "James R. Munkres",
  title = "Topology: A First Course",
  publisher = "Prentice-Hall, Inc.",
  address = "Englewood Cliffs, New Jersey",
  note = "[QA611.M82]",
  year = 1975}
@article{Nemesszeghy, author = "E. Z. Nemesszeghy and E. A. Nemesszeghy",
  title = "On Strongly Creative Definitions:  A Reply to {V}. {F}. {R}ickey",
  journal = "Logique et Analyse (N.\ S.)",
  year = 1977,
  volume = 20,
  note = "[BC.L832]",
  pages = "111--115" }
@unpublished{Nemeti, author = "N{\'{e}}meti, I.",
  title = "Algebraizations of Quantifier Logics, an Overview",
  note = "Version 11.4, preprint, Mathematical Institute, Budapest,
    1994.  A shortened version without proofs appeared in
    ``Algebraizations of quantifier logics, an introductory overview,''
   {\em Studia Logica}, 50:485--569, 1991 [B18.P6.S933]"}
@article{Pavicic, author = "M. Pavi{\v{c}}i{\'{c}}",
  title = "A New Axiomatization of Unified Quantum Logic",
  journal = "International Journal of Theoretical Physics",
  year = 1992,
  volume = 31,
  note = "[QC.I626]",
  pages = "1753 --1766" }
@book{Penrose, author = "Roger Penrose",
  title = "The Emperor's New Mind",
  publisher = "Oxford University Press",
  address = "New York",
  note = "[Q335.P415]",
  year = 1989 }
@book{PetersonI, author = "Ivars Peterson",
  title = "The Mathematical Tourist",
  publisher = "W. H. Freeman and Company",
  address = "New York",
  note = "[QA93.P475]",
  year = 1988 }
@article{Peterson, author = "Jeremy George Peterson",
  title = "An automatic theorem prover for substitution and detachment systems",
  journal = "Notre Dame Journal of Formal Logic",
  volume = 19,
  year = 1978,
  note = "[QA.N914]",
  pages = "119--122" }
@book{Quine, author = "Willard Van Orman Quine",
  title = "Set Theory and Its Logic",
  edition = "revised",
  publisher = "The Belknap Press of Harvard University Press",
  address = "Cambridge, Massachusetts",
  note = "[QA248.Q7 1969]",
  year = 1969 }
@article{Robinson, author = "J. A. Robinson",
  title = "A Machine-Oriented Logic Based on the Resolution Principle",
  journal = "Journal of the Association for Computing Machinery",
  year = 1965,
  volume = 12,
  pages = "23--41" }
@article{RobinsonT, author = "T. Thacher Robinson",
  title = "Independence of Two Nice Sets of Axioms for the Propositional
    Calculus",
  journal = "The Journal of Symbolic Logic",
  volume = 33,
  year = 1968,
  note = "[QA.J87]",
  pages = "265--270" }
@book{Rucker, author = "Rudy Rucker",
  title = "Infinity and the Mind:  The Science and Philosophy of the
    Infinite",
  publisher = "Bantam Books, Inc.",
  address = "New York",
  note = "[QA9.R79 1982]",
  year = 1982 }
@book{Russell, author = "Bertrand Russell",
  title = "Mysticism and Logic, and Other Essays",
  publisher = "Barnes \& Noble Books",
  address = "Totowa, New Jersey",
  note = "[B1649.R963.M9 1981]",
  year = 1981 }
@article{Russell2, author = "Bertrand Russell",
  title = "Recent Work on the Principles of Mathematics",
  journal = "International Monthly",
  volume = 4,
  year = 1901,
  pages = "84"}
@article{Schmidt, author = "Eric Schmidt",
  title = "Reductions in Norman Megill's axiom system for complex numbers",
  url = "http://us.metamath.org/downloads/schmidt-cnaxioms.pdf",
  year = "2012" }
@book{Shoenfield, author = "Joseph R. Shoenfield",
  title = "Mathematical Logic",
  publisher = "Addison-Wesley Publishing Company, Inc.",
  address = "Reading, Massachusetts",
  year = 1967,
  note = "[QA9.S52]" }
@book{Smullyan, author = "Raymond M. Smullyan",
  title = "Theory of Formal Systems",
  publisher = "Princeton University Press",
  address = "Princeton, New Jersey",
  year = 1961,
  note = "[QA248.5.S55]" }
@book{Solow, author = "Daniel Solow",
  title = "How to Read and Do Proofs:  An Introduction to Mathematical
    Thought Process",
  publisher = "John Wiley \& Sons",
  address = "New York",
  year = 1982,
  note = "[QA9.S577]" }
@book{Stark, author = "Harold M. Stark",
  title = "An Introduction to Number Theory",
  publisher = "Markham Publishing Company",
  address = "Chicago",
  note = "[QA241.S72 1978]",
  year = 1970 }
@article{Swart, author = "E. R. Swart",
  title = "The Philosophical Implications of the Four-Color Problem",
  journal = "American Mathematical Monthly",
  year = 1980,
  volume = 87,
  month = "November",
  note = "[QA.A5125]",
  pages = "697--707" }
@book{Szpiro, author = "George G. Szpiro",
  title = "Poincar{\'{e}}'s Prize: The Hundred-Year Quest to Solve One
    of Math's Greatest Puzzles",
  publisher = "Penguin Books Ltd",
  address = "London",
  note = "[QA43.S985 2007]",
  year = 2007}
@book{Takeuti, author = "Gaisi Takeuti and Wilson M. Zaring",
  title = "Introduction to Axiomatic Set Theory",
  edition = "second",
  publisher = "Springer-Verlag New York Inc.",
  address = "New York",
  note = "[QA248.T136 1982]",
  year = 1982}
@inproceedings{Tarski, author = "Alfred Tarski",
  title = "What is Elementary Geometry",
  pages = "16--29",
  booktitle = "The Axiomatic Method, with Special Reference to Geometry and
     Physics (Proceedings of an International Symposium held at the University
     of California, Berkeley, December 26, 1957 --- January 4, 1958)",
  editor = "Leon Henkin and Patrick Suppes and Alfred Tarski",
  year = 1959,
  publisher = "North-Holland Publishing Company",
  address = "Amsterdam"}
@article{Tarski1965, author = "Alfred Tarski",
  title = "A Simplified Formalization of Predicate Logic with Identity",
  journal = "Archiv f{\"{u}}r Mathematische Logik und Grundlagenforschung",
  volume = 7,
  year = 1965,
  note = "[QA.A673]",
  pages = "61--79" }
@book{Tymoczko,
  title = "New Directions in the Philosophy of Mathematics",
  editor = "Thomas Tymoczko",
  publisher = "Birkh{\"{a}}user Boston, Inc.",
  address = "Boston",
  note = "[QA8.6.N48 1986]",
  year = 1986 }
@incollection{Wang,
  author = "Hao Wang",
  title = "Theory and Practice in Mathematics",
  pages = "129--152",
  booktitle = "New Directions in the Philosophy of Mathematics",
  editor = "Thomas Tymoczko",
  publisher = "Birkh{\"{a}}user Boston, Inc.",
  address = "Boston",
  note = "[QA8.6.N48 1986]",
  year = 1986 }
@manual{Webster,
  title = "Webster's New Collegiate Dictionary",
  organization = "G. \& C. Merriam Co.",
  address = "Springfield, Massachusetts",
  note = "[PE1628.W4M4 1977]",
  year = 1977 }
@manual{Whitehead, author = "Alfred North Whitehead",
  title = "An Introduction to Mathematics",
  year = 1911 }
@book{PM, author = "Alfred North Whitehead and Bertrand Russell",
  title = "Principia Mathematica",
  edition = "second",
  publisher = "Cambridge University Press",
  address = "Cambridge",
  year = "1927",
  note = "(3 vols.) [QA9.W592 1927]" }
@article{DBLP:journals/corr/Whalen16,
  author    = {Daniel Whalen},
  title     = {Holophrasm: a neural Automated Theorem Prover for higher-order logic},
  journal   = {CoRR},
  volume    = {abs/1608.02644},
  year      = {2016},
  url       = {http://arxiv.org/abs/1608.02644},
  archivePrefix = {arXiv},
  eprint    = {1608.02644},
  timestamp = {Mon, 13 Aug 2018 16:46:19 +0200},
  biburl    = {https://dblp.org/rec/bib/journals/corr/Whalen16},
  bibsource = {dblp computer science bibliography, https://dblp.org} }
@article{Wiedijk-revisited,
  author = {Freek Wiedijk},
  title = {The QED Manifesto Revisited},
  year = {2007},
  url = {http://mizar.org/trybulec65/8.pdf} }
@book{Wolfram,
  author = "Stephen Wolfram",
  title = "Mathematica:  A System for Doing Mathematics by Computer",
  edition = "second",
  publisher = "Addison-Wesley Publishing Co.",
  address = "Redwood City, California",
  note = "[QA76.95.W65 1991]",
  year = 1991 }
@book{Wos, author = "Larry Wos and Ross Overbeek and Ewing Lusk and Jim Boyle",
  title = "Automated Reasoning:  Introduction and Applications",
  edition = "second",
  publisher = "McGraw-Hill, Inc.",
  address = "New York",
  note = "[QA76.9.A96.A93 1992]",
  year = 1992 }

%
%
%[1] Church, Alonzo, Introduction to Mathematical Logic,
% Volume 1, Princeton University Press, Princeton, N. J., 1956.
%
%[2] Cohen, Paul J., Set Theory and the Continuum Hypothesis,
% W. A. Benjamin, Inc., Reading, Mass., 1966.
%
%[3] Hamilton, Alan G., Logic for Mathematicians, Cambridge
% University Press,
% Cambridge, 1988.

%[6] Kleene, Stephen Cole, Introduction to Metamathematics, D.  Van
% Nostrand Company, Inc., Princeton (1952).

%[13] Tarski, Alfred, "A simplified formalization of predicate
% logic with identity," Archiv fur Mathematische Logik und
% Grundlagenforschung, vol. 7 (1965), pp. 61-79.

%[14] Tarski, Alfred and Steven Givant, A Formalization of Set
% Theory Without Variables, American Mathematical Society Colloquium
% Publications, vol. 41, American Mathematical Society,
% Providence, R. I., 1987.

%[15] Zeman, J. J., Modal Logic, Oxford University Press, Oxford, 1973.
\end{filecontents}
% --------------------------- End of metamath.bib -----------------------------


%Book: Metamath
%Author:  Norman Megill Email:  nm at alum.mit.edu
%Author:  David A. Wheeler Email:  dwheeler at dwheeler.com

% A book template example
% http://www.stsci.edu/ftp/software/tex/bookstuff/book.template

\documentclass[leqno]{book} % LaTeX 2e. 10pt. Use [leqno,12pt] for 12pt
% hyperref 2002/05/27 v6.72r  (couldn't get pagebackref to work)
\usepackage[plainpages=false,pdfpagelabels=true]{hyperref}

\usepackage{needspace}     % Enable control over page breaks
\usepackage{breqn}         % automatic equation breaking
\usepackage{microtype}     % microtypography, reduces hyphenation

% Packages for flexible tables.  We need to be able to
% wrap text within a cell (with automatically-determined widths) AND
% split a table automatically across multiple pages.
% * "tabularx" wraps text in cells but only 1 page
% * "longtable" goes across pages but by itself is incompatible with tabularx
% * "ltxtable" combines longtable and tabularx, but table contents
%    must be in a separate file.
% * "ltablex" combines tabularx and longtable - must install specially
% * "booktabs" is recommended as a way to improve the look of tables,
%   but doesn't add these capabilities.
% * "tabu" much more capable and seems to be recommended. So use that.

\usepackage{makecell}      % Enable forced line splits within a table cell
% v4.13 needed for tabu: https://tex.stackexchange.com/questions/600724/dimension-too-large-after-recent-longtable-update
\usepackage{longtable}[=v4.13] % Enable multi-page tables  
\usepackage{tabu}          % Multi-page tables with wrapped text in a cell

% You can find more Tex packages using commands like:
% tlmgr search --file tabu.sty
% find /usr/share/texmf-dist/ -name '*tab*'
%
%%%%%%%%%%%%%%%%%%%%%%%%%%%%%%%%%%%%%%%%%%%%%%%%%%%%%%%%%%%%%%%%%%%%%%%%%%%%
% Uncomment the next 3 lines to suppress boxes and colors on the hyperlinks
%%%%%%%%%%%%%%%%%%%%%%%%%%%%%%%%%%%%%%%%%%%%%%%%%%%%%%%%%%%%%%%%%%%%%%%%%%%%
%\hypersetup{
%colorlinks,citecolor=black,filecolor=black,linkcolor=black,urlcolor=black
%}
%
\usepackage{realref}

% Restarting page numbers: try?
%   \printglossary
%   \cleardoublepage
%   \pagenumbering{arabic}
%   \setcounter{page}{1}    ???needed
%   \include{chap1}

% not used:
% \def\R2Lurl#1#2{\mbox{\href{#1}\texttt{#2}}}

\usepackage{amssymb}

% Version 1 of book: margins: t=.4, b=.2, ll=.4, rr=.55
% \usepackage{anysize}
% % \papersize{<height>}{<width>}
% % \marginsize{<left>}{<right>}{<top>}{<bottom>}
% \papersize{9in}{6in}
% % l/r 0.6124-0.6170 works t/b 0.2418-0.3411 = 192pp. 0.2926-03118=exact
% \marginsize{0.7147in}{0.5147in}{0.4012in}{0.2012in}

\usepackage{anysize}
% \papersize{<height>}{<width>}
% \marginsize{<left>}{<right>}{<top>}{<bottom>}
\papersize{9in}{6in}
% l/r 0.85in&0.6431-0.6539 works t/b ?-?
%\marginsize{0.85in}{0.6485in}{0.55in}{0.35in}
\marginsize{0.8in}{0.65in}{0.5in}{0.3in}

% \usepackage[papersize={3.6in,4.8in},hmargin=0.1in,vmargin={0.1in,0.1in}]{geometry}  % page geometry
\usepackage{special-settings}

\raggedbottom
\makeindex

\begin{document}
% Discourage page widows and orphans:
\clubpenalty=300
\widowpenalty=300

%%%%%%% load in AMS fonts %%%%%%% % LaTeX 2.09 - obsolete in LaTeX 2e
%\input{amssym.def}
%\input{amssym.tex}
%\input{c:/texmf/tex/plain/amsfonts/amssym.def}
%\input{c:/texmf/tex/plain/amsfonts/amssym.tex}

\bibliographystyle{plain}
\pagenumbering{roman}
\pagestyle{headings}

\thispagestyle{empty}

\hfill
\vfill

\begin{center}
{\LARGE\bf Metamath} \\
\vspace{1ex}
{\large A Computer Language for Mathematical Proofs} \\
\vspace{7ex}
{\large Norman Megill} \\
\vspace{7ex}
with extensive revisions by \\
\vspace{1ex}
{\large David A. Wheeler} \\
\vspace{7ex}
% Printed date. If changing the date below, also fix the date at the beginning.
2019-06-02
\end{center}

\vfill
\hfill

\newpage
\thispagestyle{empty}

\hfill
\vfill

\begin{center}
$\sim$\ {\sc Public Domain}\ $\sim$

\vspace{2ex}
This book (including its later revisions)
has been released into the Public Domain by Norman Megill per the
Creative Commons CC0 1.0 Universal (CC0 1.0) Public Domain Dedication.
David A. Wheeler has done the same.
This public domain release applies worldwide.  In case this is not
legally possible, the right is granted to use the work for any purpose,
without any conditions, unless such conditions are required by law.
See \url{https://creativecommons.org/publicdomain/zero/1.0/}.

\vspace{3ex}
Several short, attributed quotations from copyrighted works
appear in this book under the ``fair use'' provision of Section 107 of
the United States Copyright Act (Title 17 of the {\em United States
Code}).  The public-domain status of this book is not applicable to
those quotations.

\vspace{3ex}
Any trademarks used in this book are the property of their owners.

% QA76.9.L63.M??

% \vspace{1ex}
%
% \vspace{1ex}
% {\small Permission is granted to make and distribute verbatim copies of this
% book
% provided the copyright notice and this
% permission notice are preserved on all copies.}
%
% \vspace{1ex}
% {\small Permission is granted to copy and distribute modified versions of this
% book under the conditions for verbatim copying, provided that the
% entire
% resulting derived work is distributed under the terms of a permission
% notice
% identical to this one.}
%
% \vspace{1ex}
% {\small Permission is granted to copy and distribute translations of this
% book into another language, under the above conditions for modified
% versions,
% except that this permission notice may be stated in a translation
% approved by the
% author.}
%
% \vspace{1ex}
% %{\small   For a copy of the \LaTeX\ source files for this book, contact
% %the author.} \\
% \ \\
% \ \\

\vspace{7ex}
% ISBN: 1-4116-3724-0 \\
% ISBN: 978-1-4116-3724-5 \\
ISBN: 978-0-359-70223-7 \\
{\ } \\
Lulu Press \\
Morrisville, North Carolina\\
USA


\hfill
\vfill

Norman Megill\\ 93 Bridge St., Lexington, MA 02421 \\
E-mail address: \texttt{nm{\char`\@}alum.mit.edu} \\
\vspace{7ex}
David A. Wheeler \\
E-mail address: \texttt{dwheeler{\char`\@}dwheeler.com} \\
% See notes added at end of Preface for revision history. \\
% For current information on the Metamath software see \\
\vspace{7ex}
\url{http://metamath.org}
\end{center}

\hfill
\vfill

{\parindent0pt%
\footnotesize{%
Cover: Aleph null ($\aleph_0$) is the symbol for the
first infinite cardinal number, discovered by Georg Cantor in 1873.
We use a red aleph null (with dark outline and gold glow) as the Metamath logo.
Credit: Norman Megill (1994) and Giovanni Mascellani (2019),
public domain.%
\index{aleph null}%
\index{Metamath!logo}\index{Cantor, Georg}\index{Mascellani, Giovanni}}}

% \newpage
% \thispagestyle{empty}
%
% \hfill
% \vfill
%
% \begin{center}
% {\it To my son Robin Dwight Megill}
% \end{center}
%
% \vfill
% \hfill
%
% \newpage

\tableofcontents
%\listoftables

\chapter*{Preface}
\markboth{PREFACE}{PREFACE}
\addcontentsline{toc}{section}{Preface}


% (For current information, see the notes added at the
% end of this preface on p.~\pageref{note2002}.)

\subsubsection{Overview}

Metamath\index{Metamath} is a computer language and an associated computer
program for archiving, verifying, and studying mathematical proofs at a very
detailed level.  The Metamath language incorporates no mathematics per se but
treats all mathematical statements as mere sequences of symbols.  You provide
Metamath with certain special sequences (axioms) that tell it what rules
of inference are allowed.  Metamath is not limited to any specific field of
mathematics.  The Metamath language is simple and robust, with an
almost total absence of hard-wired syntax, and
we\footnote{Unless otherwise noted, the words
``I,'' ``me,'' and ``my'' refer to Norman Megill\index{Megill, Norman}, while
``we,'' ``us,'' and ``our'' refer to Norman Megill and
David A. Wheeler\index{Wheeler, David A.}.}
believe that it
provides about the simplest possible framework that allows essentially all of
mathematics to be expressed with absolute rigor.

% index test
%\newcommand{\nn}[1]{#1n}
%\index{aaa@bbb}
%\index{abc!def}
%\index{abd|see{qqq}}
%\index{abe|nn}
%\index{abf|emph}
%\index{abg|(}
%\index{abg|)}

Using the Metamath language, you can build formal or mathematical
systems\index{formal system}\footnote{A formal or mathematical system consists
of a collection of symbols (such as $2$, $4$, $+$ and $=$), syntax rules that
describe how symbols may be combined to form a legal expression (called a
well-formed formula or {\em wff}, pronounced ``whiff''), some starting wffs
called axioms, and inference rules that describe how theorems may be derived
(proved) from the axioms.  A theorem is a mathematical fact such as $2+2=4$.
Strictly speaking, even an obvious fact such as this must be proved from
axioms to be formally acceptable to a mathematician.}\index{theorem}
\index{axiom}\index{rule}\index{well-formed formula (wff)} that involve
inferences from axioms.  Although a database is provided
that includes a recommended set of axioms for standard mathematics, if you
wish you can supply your own symbols, syntax, axioms, rules, and definitions.

The name ``Metamath'' was chosen to suggest that the language provides a
means for {\em describing} mathematics rather than {\em being} the
mathematics itself.  Actually in some sense any mathematical language is
metamathematical.  Symbols written on paper, or stored in a computer,
are not mathematics itself but rather a way of expressing mathematics.
For example ``7'' and ``VII'' are symbols for denoting the number seven
in Arabic and Roman numerals; neither {\em is} the number seven.

If you are able to understand and write computer programs, you should be able
to follow abstract mathematics with the aid of Metamath.  Used in conjunction
with standard textbooks, Metamath can guide you step-by-step towards an
understanding of abstract mathematics from a very rigorous viewpoint, even if
you have no formal abstract mathematics background.  By using a single,
consistent notation to express proofs, once you grasp its basic concepts
Metamath provides you with the ability to immediately follow and dissect
proofs even in totally unfamiliar areas.

Of course, just being able follow a proof will not necessarily give you an
intuitive familiarity with mathematics.  Memorizing the rules of chess does not
give you the ability to appreciate the game of a master, and knowing how the
notes on a musical score map to piano keys does not give you the ability to
hear in your head how it would sound.  But each of these can be a first step.

Metamath allows you to explore proofs in the sense that you can see the
theorem referenced at any step expanded in as much detail as you want, right
down to the underlying axioms of logic and set theory (in the case of the set
theory database provided).  While Metamath will not replace the higher-level
understanding that can only be acquired through exercises and hard work, being
able to see how gaps in a proof are filled in can give you increased
confidence that can speed up the learning process and save you time when you
get stuck.

The Metamath language breaks down a mathematical proof into its tiniest
possible parts.  These can be pieced together, like interlocking
pieces in a puzzle, only in a way that produces correct and absolutely rigorous
mathematics.

The nature of Metamath\index{Metamath} enforces very precise mathematical
thinking, similar to that involved in writing a computer program.  A crucial
difference, though, is that once a proof is verified (by the Metamath program)
to be correct, it is definitely correct; it can never have a hidden
``bug.''\index{computer program bugs}  After getting used to the kind of rigor
and accuracy provided by Metamath, you might even be tempted to
adopt the attitude that a proof should never be considered correct until it
has been verified by a computer, just as you would not completely trust a
manual calculation until you have verified it on a
calculator.

My goal
for Metamath was a system for describing and verifying
mathematics that is completely universal yet conceptually as simple as
possible.  In approaching mathematics from an axiomatic, formal viewpoint, I
wanted Metamath to be able to handle almost any mathematical system, not
necessarily with ease, but at least in principle and hopefully in practice. I
wanted it to verify proofs with absolute rigor, and for this reason Metamath
is what might be thought of as a ``compile-only'' language rather than an
algorithmic or Turing-machine language (Pascal, C, Prolog, Mathematica,
etc.).  In other words, a database written in the Metamath
language doesn't ``do'' anything; it merely exhibits mathematical knowledge
and permits this knowledge to be verified as being correct.  A program in an
algorithmic language can potentially have hidden bugs\index{computer program
bugs} as well as possibly being hard to understand.  But each token in a
Metamath database must be consistent with the database's earlier
contents according to simple, fixed rules.
If a database is verified
to be correct,\footnote{This includes
verification that a sequential list of proof steps results in the specified
theorem.} then the mathematical content is correct if the
verifier is correct and the axioms are correct.
The verification program could be incorrect, but the verification algorithm
is relatively simple (making it unlikely to be implemented incorrectly
by the Metamath program),
and there are over a dozen Metamath database verifiers
written by different people in different programming languages
(so these different verifiers can act as multiple reviewers of a database).
The most-used Metamath database, the Metamath Proof Explorer
(aka \texttt{set.mm}\index{set theory database (\texttt{set.mm})}%
\index{Metamath Proof Explorer}),
is currently verified by four different Metamath verifiers written by
four different people in four different languages, including the
original Metamath program described in this book.
The only ``bugs'' that can exist are in the statement of the axioms,
for example if the axioms are inconsistent (a famous problem shown to be
unsolvable by G\"{o}del's incompleteness theorem\index{G\"{o}del's
incompleteness theorem}).
However, real mathematical systems have very few axioms, and these can
be carefully studied.
All of this provides extraordinarily high confidence that the verified database
is in fact correct.

The Metamath program
doesn't prove theorems automatically but is designed to verify proofs
that you supply to it.
The underlying Metamath language is completely general and has no built-in,
preconceived notions about your formal system\index{formal system}, its logic
or its syntax.
For constructing proofs, the Metamath program has a Proof Assistant\index{Proof
Assistant} which helps you fill in some of a proof step's details, shows you
what choices you have at any step, and verifies the proof as you build it; but
you are still expected to provide the proof.

There are many other programs that can process or generate information
in the Metamath language, and more continue to be written.
This is in part because the Metamath language itself is very simple
and intentionally easy to automatically process.
Some programs, such as \texttt{mmj2}\index{mmj2}, include a proof assistant
that can automate some steps beyond what the Metamath program can do.
Mario Carneiro has developed an algorithm for converting proofs from
the OpenTheory interchange format, which can be translated to and from
any of the HOL family of proof languages (HOL4, HOL Light, ProofPower,
and Isabelle), into the
Metamath language \cite{DBLP:journals/corr/Carneiro14}\index{Carneiro, Mario}.
Daniel Whalen has developed Holophrasm, which can automatically
prove many Metamath proofs using
machine learning\index{machine learning}\index{artificial intelligence}
approaches
(including multiple neural networks\index{neural networks})\cite{DBLP:journals/corr/Whalen16}\index{Whalen, Daniel}.
However,
a discussion of these other programs is beyond the scope of this book.

Like most computer languages, the Metamath\index{Metamath} language uses the
standard ({\sc ascii}) characters on a computer keyboard, so it cannot
directly represent many of the special symbols that mathematicians use.  A
useful feature of the Metamath program is its ability to convert its notation
into the \LaTeX\ typesetting language.\index{latex@{\LaTeX}}  This feature
lets you convert the {\sc ascii} tokens you've defined into standard
mathematical symbols, so you end up with symbols and formulas you are familiar
with instead of somewhat cryptic {\sc ascii} representations of them.
The Metamath program can also generate HTML\index{HTML}, making it easy
to view results on the web and to see related information by using
hypertext links.

Metamath is probably conceptually different from anything you've seen
before and some aspects may take some getting used to.  This book will
help you decide whether Metamath suits your specific needs.

\subsubsection{Setting Your Expectations}
It is important for you to understand what Metamath\index{Metamath} is and is
not.  As mentioned, the Metamath program
is {\em not} an automated theorem prover but
rather a proof verifier.  Developing a database can be tedious, hard work,
especially if you want to make the proofs as short as possible, but it becomes
easier as you build up a collection of useful theorems.  The purpose of
Metamath is simply to document existing mathematics in an absolutely rigorous,
computer-verifiable way, not to aid directly in the creation of new
mathematics.  It also is not a magic solution for learning abstract
mathematics, although it may be helpful to be able to actually see the implied
rigor behind what you are learning from textbooks, as well as providing hints
to work out proofs that you are stumped on.

As of this writing, a sizable set theory database has been developed to
provide a foundation for many fields of mathematics, but much more work would
be required to develop useful databases for specific fields.

Metamath\index{Metamath} ``knows no math;'' it just provides a framework in
which to express mathematics.  Its language is very small.  You can define two
kinds of symbols, constants\index{constant} and variables\index{variable}.
The only thing Metamath knows how to do is to substitute strings of symbols
for the variables\index{substitution!variable}\index{variable substitution} in
an expression based on instructions you provide it in a proof, subject to
certain constraints you specify for the variables.  Even the decimal
representation of a number is merely a string of certain constants (digits)
which together, in a specific context, correspond to whatever mathematical
object you choose to define for it; unlike other computer languages, there is
no actual number stored inside the computer.  In a proof, you in effect
instruct Metamath what symbol substitutions to make in previous axioms or
theorems and join a sequence of them together to result in the desired
theorem.  This kind of symbol manipulation captures the essence of mathematics
at a preaxiomatic level.

\subsubsection{Metamath and Mathematical Literature}

In advanced mathematical literature, proofs are usually presented in the form
of short outlines that often only an expert can follow.  This is partly out of
a desire for brevity, but it would also be unwise (even if it were practical)
to present proofs in complete formal detail, since the overall picture would
be lost.\index{formal proof}

A solution I envision\label{envision} that would allow mathematics to remain
acceptable to the expert, yet increase its accessibility to non-specialists,
consists of a combination of the traditional short, informal proof in print
accompanied by a complete formal proof stored in a computer database.  In an
analogy with a computer program, the informal proof is like a set of comments
that describe the overall reasoning and content of the proof, whereas the
computer database is like the actual program and provides a means for anyone,
even a non-expert, to follow the proof in as much detail as desired, exploring
it back through layers of theorems (like subroutines that call other
subroutines) all the way back to the axioms of the theory.  In addition, the
computer database would have the advantage of providing absolute assurance
that the proof is correct, since each step can be verified automatically.

There are several other approaches besides Metamath to a project such
as this.  Section~\ref{proofverifiers} discusses some of these.

To us, a noble goal would be a database with hundreds of thousands of
theorems and their computer-verifiable proofs, encompassing a significant
fraction of known mathematics and available for instant access.
These would be fully verified by multiple independently-implemented verifiers,
to provide extremely high confidence that the proofs are completely correct.
The database would allow people to investigate whatever details they were
interested in, so that they could confirm whatever portions they wished.
Whether or not Metamath is an appropriate choice remains to be seen, but in
principle we believe it is sufficient.

\subsubsection{Formalism}

Over the past fifty years, a group of French mathematicians working
collectively under the pseudonym of Bourbaki\index{Bourbaki, Nicolas} have
co-authored a series of monographs that attempt to rigorously and
consistently formalize large bodies of mathematics from foundations.  On the
one hand, certainly such an effort has its merits; on the other hand, the
Bourbaki project has been criticized for its ``scholasticism'' and
``hyperaxiomatics'' that hide the intuitive steps that lead to the results
\cite[p.~191]{Barrow}\index{Barrow, John D.}.

Metamath unabashedly carries this philosophy to its extreme and no doubt is
subject to the same kind of criticism.  Nonetheless I think that in
conjunction with conventional approaches to mathematics Metamath can serve a
useful purpose.  The Bourbaki approach is essentially pedagogic, requiring the
reader to become intimately familiar with each detail in a very large
hierarchy before he or she can proceed to the next step.  The difference with
Metamath is that the ``reader'' (user) knows that all details are contained in
its computer database, available as needed; it does not demand that the user
know everything but conveniently makes available those portions that are of
interest.  As the body of all mathematical knowledge grows larger and larger,
no one individual can have a thorough grasp of its entirety.  Metamath
can finalize and put to rest any questions about the validity of any part of it
and can make any part of it accessible, in principle, to a non-specialist.

\subsubsection{A Personal Note}
Why did I develop Metamath\index{Metamath}?  I enjoy abstract mathematics, but
I sometimes get lost in a barrage of definitions and start to lose confidence
that my proofs are correct.  Or I reach a point where I lose sight of how
anything I'm doing relates to the axioms that a theory is based on and am
sometimes suspicious that there may be some overlooked implicit axiom
accidentally introduced along the way (as happened historically with Euclidean
geometry\index{Euclidean geometry}, whose omission of Pasch's
axiom\index{Pasch's axiom} went unnoticed for 2000 years
\cite[p.~160]{Davis}!). I'm also somewhat lazy and wish to avoid the effort
involved in re-verifying the gaps in informal proofs ``left to the reader;'' I
prefer to figure them out just once and not have to go through the same
frustration a year from now when I've forgotten what I did.  Metamath provides
better recovery of my efforts than scraps of paper that I can't
decipher anymore.  But mostly I find very appealing the idea of rigorously
archiving mathematical knowledge in a computer database, providing precision,
certainty, and elimination of human error.

\subsubsection{Note on Bibliography and Index}

The Bibliography usually includes the Library of Congress classification
for a work to make it easier for you to find it in on a university
library shelf.  The Index has author references to pages where their works
are cited, even though the authors' names may not appear on those pages.

\subsubsection{Acknowledgments}

Acknowledgments are first due to my wife, Deborah (who passed away on
September 4, 1998), for critiquing the manu\-script but most of all for
her patience and support.  I also wish to thank Joe Wright, Richard
Becker, Clarke Evans, Buddha Buck, and Jeremy Henty for helpful
comments.  Any errors, omissions, and other shortcomings are of course
my responsibility.

\subsubsection{Note Added June 22, 2005}\label{note2002}

The original, unpublished version of this book was written in 1997 and
distributed via the web.  The present edition has been updated to
reflect the current Metamath program and databases, as well as more
current {\sc url}s for Internet sites.  Thanks to Josh
Purinton\index{Purinton, Josh}, One Hand
Clapping, Mel L.\ O'Cat, and Roy F. Longton for pointing out
typographical and other errors.  I have also benefitted from numerous
discussions with Raph Levien\index{Levien, Raph}, who has extended
Metamath's philosophy of rigor to result in his {\em
Ghilbert}\index{Ghilbert} proof language (\url{http://ghilbert.org}).

Robert (Bob) Solovay\index{Solovay, Robert} communicated a new result of
A.~R.~D.~Mathias on the system of Bourbaki, and the text has been
updated accordingly (p.~\pageref{bourbaki}).

Bob also pointed out a clarification of the literature regarding
category theory and inaccessible cardinals\index{category
theory}\index{cardinal, inaccessible} (p.~\pageref{categoryth}),
and a misleading statement was removed from the text.  Specifically,
contrary to a statement in previous editions, it is possible to express
``There is a proper class of inaccessible cardinals'' in the language of
ZFC.  This can be done as follows:  ``For every set $x$ there is an
inaccessible cardinal $\kappa$ such that $\kappa$ is not in $x$.''
Bob writes:\footnote{Private communication, Nov.~30, 2002.}
\begin{quotation}
     This axiom is how Grothendieck presents category theory.  To each
inaccessible cardinal $\kappa$ one associates a Grothendieck universe
\index{Grothendieck, Alexander} $U(\kappa)$.  $U(\kappa)$ consists of
those sets which lie in a transitive set of cardinality less than
$\kappa$.  Instead of the ``category of all groups,'' one works relative
to a universe [considering the category of groups of cardinality less
than $\kappa$].  Now the category whose objects are all categories
``relative to the universe $U(\kappa)$'' will be a category not
relative to this universe but to the next universe.

     All of the things category theorists like to do can be done in this
framework.  The only controversial point is whether the Grothen\-dieck
axiom is too strong for the needs of category theorists.  Mac Lane
\index{Mac Lane, Saunders} argues that ``one universe is enough'' and
Feferman\index{Feferman, Solomon} has argued that one can get by with
ordinary ZFC.  I don't find Feferman's arguments persuasive.  Mac Lane
may be right, but when I think about category theory I do it \`{a} la
Grothendieck.

        By the way Mizar\index{Mizar} adds the axiom ``there is a proper
class of inaccessibles'' precisely so as to do category theory.
\end{quotation}

The most current information on the Metamath program and databases can
always be found at \url{http://metamath.org}.


\subsubsection{Note Added June 24, 2006}\label{note2006}

The Metamath spec was restricted slightly to make parsers easier to
write.  See the footnote on p.~\pageref{namespace}.

%\subsubsection{Note Added July 24, 2006}\label{note2006b}
\subsubsection{Note Added March 10, 2007}\label{note2006b}

I am grateful to Anthony Williams\index{Williams, Anthony} for writing
the \LaTeX\ package called {\tt realref.sty} and contributing it to the
public domain.  This package allows the internal hyperlinks in a {\sc
pdf} file to anchor to specific page numbers instead of just section
titles, making the navigation of the {\sc pdf} file for this book much
more pleasant and ``logical.''

A typographical error found by Martin Kiselkov was corrected.
A confusing remark about unification was deleted per suggestion of
Mel O'Cat.

\subsubsection{Note Added May 27, 2009}\label{note2009}

Several typos found by Kim Sparre were corrected.  A note was added that
the Poincar\'{e} conjecture has been proved (p.~\pageref{poincare}).

\subsubsection{Note Added Nov. 17, 2014}\label{note2014}

The statement of the Schr\"{o}der--Bernstein theorem was corrected in
Section~\ref{trust}.  Thanks to Bob Solovay for pointing out the error.

\subsubsection{Note Added May 25, 2016}\label{note2016}

Thanks to Jerry James for correcting 16 typos.

\subsubsection{Note Added February 25, 2019}\label{note201902}

David A. Wheeler\index{Wheeler, David A.}
made a large number of improvements and updates,
in coordination with Norman Megill.
The predicate calculus axioms were renumbered, and the text makes
it clear that they are based on Tarski's system S2;
the one slight deviation in axiom ax-6 is explained and justified.
The real and complex number axioms were modified to be consistent with
\texttt{set.mm}\index{set theory database (\texttt{set.mm})}%
\index{Metamath Proof Explorer}.
Long-awaited specification changes ``1--8'' were made,
which clarified previously ambiguous points.
Some errors in the text involving \texttt{\$f} and
\texttt{\$d} statements were corrected (the spec was correct, but
the in-book explanations unintentionally contradicted the spec).
We now have a system for automatically generating narrow PDFs,
so that those with smartphones can have easy access to the current
version of this document.
A new section on deduction was added;
it discusses the standard deduction theorem,
the weak deduction theorem,
deduction style, and natural deduction.
Many minor corrections (too numerous to list here) were also made.

\subsubsection{Note Added March 7, 2019}\label{note201903}

This added a description of the Matamath language syntax in
Extended Backus--Naur Form (EBNF)\index{Extended Backus--Naur Form}\index{EBNF}
in Appendix \ref{BNF}, added a brief explanation about typecodes,
inserted more examples in the deduction section,
and added a variety of smaller improvements.

\subsubsection{Note Added April 7, 2019}\label{note201904}

This version clarified the proper substitution notation, improved the
discussion on the weak deduction theorem and natural deduction,
documented the \texttt{undo} command, updated the information on
\texttt{write source}, changed the typecode
from \texttt{set} to \texttt{setvar} to be consistent with the current
version of \texttt{set.mm}, added more documentation about comment markup
(e.g., documented how to create headings), and clarified the
differences between various assertion forms (in particular deduction form).

\subsubsection{Note Added June 2, 2019}\label{note201906}

This version fixes a large number of small issues reported by
Beno\^{i}t Jubin\index{Jubin, Beno\^{i}t}, such as editorial issues
and the need to document \texttt{verify markup} (thank you!).
This version also includes specific examples
of forms (deduction form, inference form, and closed form).
We call this version the ``second edition'';
the previous edition formally published in 2007 had a slightly different title
(\textit{Metamath: A Computer Language for Pure Mathematics}).

\chapter{Introduction}
\pagenumbering{arabic}

\begin{quotation}
  {\em {\em I.M.:}  No, no.  There's nothing subjective about it!  Everybody
knows what a proof is.  Just read some books, take courses from a competent
mathematician, and you'll catch on.

{\em Student:}  Are you sure?

{\em I.M.:}  Well---it is possible that you won't, if you don't have any
aptitude for it.  That can happen, too.

{\em Student:}  Then {\em you} decide what a proof is, and if I don't learn
to decide in the same way, you decide I don't have any aptitude.

{\em I.M.:}  If not me, then who?}
    \flushright\sc  ``The Ideal Mathematician''
    \index{Davis, Phillip J.}
    \footnote{\cite{Davis}, p.~40.}\\
\end{quotation}

Brilliant mathematicians have discovered almost
unimaginably profound results that rank among the crowning intellectual
achievements of mankind.  However, there is a sense in which modern abstract
mathematics is behind the times, stuck in an era before computers existed.
While no one disputes the remarkable results that have been achieved,
communicating these results in a precise way to the uninitiated is virtually
impossible.  To describe these results, a terse informal language is used which
despite its elegance is very difficult to learn.  This informal language is not
imprecise, far from it, but rather it often has omitted detail
and symbols with hidden context that are
implicitly understood by an expert but few others.  Extremely complex technical
meanings are associated with innocent-sounding English words such as
``compact'' and ``measurable'' that barely hint at what is actually being
said.  Anyone who does not keep the precise technical meaning constantly in
mind is bound to fail, and acquiring the ability to do this can be achieved
only through much practice and hard work.  Only the few who complete the
painful learning experience can join the small in-group of pure
mathematicians.  The informal language effectively cuts off the true nature of
their knowledge from most everyone else.

Metamath\index{Metamath} makes abstract mathematics more concrete.  It allows
a computer to keep track of the complexity associated with each word or symbol
with absolute rigor.  You can explore this complexity at your leisure, to
whatever degree you desire.  Whether or not you believe that concepts such as
infinity actually ``exist'' outside of the mind, Metamath lets you get to the
foundation for what's really being said.

Metamath also enables completely rigorous and thorough proof verification.
Its language is simple enough so that you
don't have to rely on the authority of experts but can verify the results
yourself, step by step.  If you want to attempt to derive your own results,
Metamath will not let you make a mistake in reasoning.
Even professional mathematicians make mistakes; Metamath makes it possible
to thoroughly verify that proofs are correct.

Metamath\index{Metamath} is a computer language and an associated computer
program for archiving, verifying, and studying mathematical proofs at a very
detailed level.
The Metamath language
describes formal\index{formal system} mathematical
systems and expresses proofs of theorems in those systems.  Such a language
is called a metalanguage\index{metalanguage} by mathematicians.
The Metamath program is a computer program that verifies
proofs expressed in the Metamath language.
The Metamath program does not have the built-in
ability to make logical inferences; it just makes a series of symbol
substitutions according to instructions given to it in a proof
and verifies that the result matches the expected theorem.  It makes logical
inferences based only on rules of logic that are contained in a set of
axioms\index{axiom}, or first principles, that you provide to it as the
starting point for proofs.

The complete specification of the Metamath language is only four pages long
(Section~\ref{spec}, p.~\pageref{spec}).  Its simplicity may at first make you
wonder how it can do much of anything at all.  But in fact the kinds of
symbol manipulations it performs are the ones that are implicitly done in all
mathematical systems at the lowest level.  You can learn it relatively quickly
and have complete confidence in any mathematical proof that it verifies.  On
the other hand, it is powerful and general enough so that virtually any
mathematical theory, from the most basic to the deeply abstract, can be
described with it.

Although in principle Metamath can be used with any
kind of mathematics, it is best suited for abstract or ``pure'' mathematics
that is mostly concerned with theorems and their proofs, as opposed to the
kind of mathematics that deals with the practical manipulation of numbers.
Examples of branches of pure mathematics are logic\index{logic},\footnote{Logic
is the study of statements that are universally true regardless of the objects
being described by the statements.  An example is the statement, ``if $P$
implies $Q$, then either $P$ is false or $Q$ is true.''} set theory\index{set
theory},\footnote{Set theory is the study of general-purpose mathematical objects called
``sets,'' and from it essentially all of mathematics can be derived.  For
example, numbers can be defined as specific sets, and their properties
can be explored using the tools of set theory.} number theory\index{number
theory},\footnote{Number theory deals with the properties of positive and
negative integers (whole numbers).} group theory\index{group
theory},\footnote{Group theory studies the properties of mathematical objects
called groups that obey a simple set of axioms and have properties of symmetry
that make them useful in many other fields.} abstract algebra\index{abstract
algebra},\footnote{Abstract algebra includes group theory and also studies
groups with additional properties that qualify them as ``rings'' and
``fields.''  The set of real numbers is a familiar example of a field.},
analysis\index{analysis} \index{real and complex numbers}\footnote{Analysis is
the study of real and complex numbers.} and
topology\index{topology}.\footnote{One area studied by topology are properties
that remain unchanged when geometrical objects undergo stretching
deformations; for example a doughnut and a coffee cup each have one hole (the
cup's hole is in its handle) and are thus considered topologically
equivalent.  In general, though, topology is the study of abstract
mathematical objects that obey a certain (surprisingly simple) set of axioms.
See, for example, Munkres \cite{Munkres}\index{Munkres, James R.}.} Even in
physics, Metamath could be applied to certain branches that make use of
abstract mathematics, such as quantum logic\index{quantum logic} (used to study
aspects of quantum mechanics\index{quantum mechanics}).

On the other hand, Metamath\index{Metamath} is less suited to applications
that deal primarily with intensive numeric computations.  Metamath does not
have any built-in representation of numbers\index{Metamath!representation of
numbers}; instead, a specific string of symbols (digits) must be syntactically
constructed as part of any proof in which an ordinary number is used.  For
this reason, numbers in Metamath are best limited to specific constants that
arise during the course of a theorem or its proof.  Numbers are only a tiny
part of the world of abstract mathematics.  The exclusion of built-in numbers
was a conscious decision to help achieve Metamath's simplicity, and there are
other software tools if you have different mathematical needs.
If you wish to quickly solve algebraic problems, the computer algebra
programs\index{computer algebra system} {\sc
macsyma}\index{macsyma@{\sc macsyma}}, Mathematica\index{Mathematica}, and
Maple\index{Maple} are specifically suited to handling numbers and
algebra efficiently.
If you wish to simply calculate numeric or matrix expressions easily,
tools such as Octave\index{Octave} may be a better choice.

After learning Metamath's basic statement types, any
tech\-ni\-cal\-ly ori\-ent\-ed person, mathematician or not, can
immediately trace
any theorem proved in the language as far back as he or she wants, all the way
to the axioms on which the theorem is based.  This ability suggests a
non-traditional way of learning about pure mathematics.  Used in conjunction
with traditional methods, Metamath could make pure mathematics accessible to
people who are not sufficiently skilled to figure out the implicit detail in
ordinary textbook proofs.  Once you learn the axioms of a theory, you can have
complete confidence that everything you need to understand a proof you are
studying is all there, at your beck and call, allowing you to focus in on any
proof step you don't understand in as much depth as you need, without worrying
about getting stuck on a step you can't figure out.\footnote{On the other
hand, writing proofs in the Metamath language is challenging, requiring
a degree of rigor far in excess of that normally taught to students.  In a
classroom setting, I doubt that writing Metamath proofs would ever replace
traditional homework exercises involving informal proofs, because the time
needed to work out the details would not allow a course to
cover much material.  For students who have trouble grasping the implied rigor
in traditional material, writing a few simple proofs in the Metamath language
might help clarify fuzzy thought processes.  Although somewhat difficult at
first, it eventually becomes fun to do, like solving a puzzle, because of the
instant feedback provided by the computer.}

Metamath\index{Metamath} is probably unlike anything you have
encountered before.  In this first chapter we will look at the philosophy and
use of computers in mathematics in order to better understand the motivation
behind Metamath.  The material in this chapter is not required in order to use
Metamath.  You may skip it if you are impatient, but I hope you will find it
educational and enjoyable.  If you want to start experimenting with the
Metamath program right away, proceed directly to Chapter~\ref{using}
(p.~\pageref{using}).  To
learn the Metamath language, skim Chapter~\ref{using} then proceed to
Chapter~\ref{languagespec} (p.~\pageref{languagespec}).

\section{Mathematics as a Computer Language}

\begin{quote}
  {\em The study of mathematics is apt to commence in
dis\-ap\-point\-ment.\ldots \\
We are told that by its aid the stars are weighted
and the billions of molecules in a drop of water are counted.  Yet, like the
ghost of Hamlet's father, this great science eludes the efforts of our mental
weapons to grasp it.}
  \flushright\sc  Alfred North Whitehead\footnote{\cite{Whitehead}, ch.\ 1.}\\
\end{quote}\index{Whitehead, Alfred North}

\subsection{Is Mathematics ``User-Friendly''?}

Suppose you have no formal training in abstract mathematics.  But popular
books you've read offer tempting glimpses of this world filled with profound
ideas that have stirred the human spirit.  You are not satisfied with the
informal, watered-down descriptions you've read but feel it is important to
grasp the underlying mathematics itself to understand its true meaning. It's
not practical to go back to school to learn it, though; you don't want to
dedicate years of your life to it.  There are many important things in life,
and you have to set priorities for what's important to you.  What would happen
if you tried to pursue it on your own, in your spare time?

After all, you were able to learn a computer programming language such as
Pascal on your own without too much difficulty, even though you had no formal
training in computers.  You don't claim to be an expert in software design,
but you can write a passable program when necessary to suit your needs.  Even
more important, you know that you can look at anyone else's Pascal program, no
matter how complex, and with enough patience figure out exactly how it works,
even though you are not a specialist.  Pascal allows you do anything that a
computer can do, at least in principle.  Thus you know you have the ability,
in principle, to follow anything that a computer program can do:  you just
have to break it down into small enough pieces.

Here's an imaginary scenario of what might happen if you na\-ive\-ly a\-dopted
this same view of abstract mathematics and tried to pick it up on your own, in
a period of time comparable to, say, learning a computer programming
language.

\subsubsection{A Non-Mathematician's Quest for Truth}

\begin{quote}
  {\em \ldots my daughters have been studying (chemistry) for several
se\-mes\-ters, think they have learned differential and integral calculus in
school, and yet even today don't know why $x\cdot y=y\cdot x$ is true.}
  \flushright\sc  Edmund Landau\footnote{\cite{Landau}, p.~vi.}\\
\end{quote}\index{Landau, Edmund}

\begin{quote}
  {\em Minus times minus is plus,\\
The reason for this we need not discuss.}
  \flushright\sc W.\ H.\ Auden\footnote{As quoted in \cite{Guillen}, p.~64.}\\
\end{quote}\index{Auden, W.\ H.}\index{Guillen, Michael}

We'll suppose you are a technically oriented professional, perhaps an engineer, a
computer programmer, or a physicist, but probably not a mathematician.  You
consider yourself reasonably intelligent.  You did well in school, learning a
variety of methods and techniques in practical mathematics such as calculus and
differential equations.  But rarely did your courses get into anything
resembling modern abstract mathematics, and proofs were something that appeared
only occasionally in your textbooks, a kind of necessary evil that was
supposed to convince you of a certain key result.  Most of your
homework consisted of exercises that gave you practice in the techniques, and
you were hardly ever asked to come up with a proof of your own.

You find yourself curious about advanced, abstract mathematics.  You are
driven by an inner conviction that it is important to understand and
appreciate some of the most profound knowledge discovered by mankind.  But it
seems very hard to learn, something that only certain gifted longhairs can
access and understand.  You are frustrated that it seems forever cut off from
you.

Eventually your curiosity drives you to do something about it.
You set for yourself a goal of ``really'' understanding mathematics:  not just
how to manipulate equations in algebra or calculus according to cookbook
rules, but rather to gain a deep understanding of where those rules come from.
In fact, you're not thinking about this kind of ordinary mathematics at all,
but about a much more abstract, ethereal realm of pure mathematics, where
famous results such as G\"{o}del's incompleteness theorem\index{G\"{o}del's
incompleteness theorem} and Cantor's different kinds of infinities
reside.

You have probably read a number of popular books, with titles like {\em
Infinity and the Mind} \cite{Rucker}\index{Rucker, Rudy}, on topics such as
these.  You found them inspiring but at the same time somewhat
unsatisfactory.  They gave you a general idea of what these results are about,
but if someone asked you to prove them, you wouldn't have the faintest idea of
where to begin.   Sure, you could give the same overall outline that you
learned from the popular books; and in a general sort of way, you do have an
understanding.  But deep down inside, you know that there is a rigor that is
missing, that probably there are many subtle steps and pitfalls along the way,
and ultimately it seems you have to place your trust in the experts in the
field.  You don't like this; you want to be able to verify these results for
yourself.

So where do you go next?  As a first step, you decide to look up some of the
original papers on the theorems you are curious about, or better, obtain some
standard textbooks in the field.  You look up a theorem you want to
understand.  Sure enough, it's there, but it's expressed with strange
terms and odd symbols that mean absolutely nothing to you.  It might as well be written in
a foreign language you've never seen before, whose symbols are totally alien.
You look at the proof, and you haven't the foggiest notion what each step
means, much less how one step follows from another.  Well, obviously you have
a lot to learn if you want to understand this stuff.

You feel that you could probably understand it by
going back to college for another three to six years and getting a math
degree.  But that does not fit in with your career and the other things in
your life and would serve no practical purpose.  You decide to seek a quicker
path.  You figure you'll just trace your way back to the beginning, step by
step, as you would do with a computer program, until you understand it.  But
you quickly find that this is not possible, since you can't even understand
enough to know what you have to trace back to.

Maybe a different approach is in order---maybe you should start at the
beginning and work your way up.  First, you read the introduction to the book
to find out what the prerequisites are.  In a similar fashion, you trace your
way back through two or three more books, finally arriving at one that seems
to start at a beginning:  it lists the axioms of arithmetic.  ``Aha!'' you
naively think, ``This must be the starting point, the source of all mathematical
knowledge.'' Or at least the starting point for mathematics dealing with
numbers; you have to start somewhere and have no idea what the starting point
for other mathematics would be.  But the word ``axioms'' looks promising.  So
you eagerly read along and work through some elementary exercises at the
beginning of the book.  You feel vaguely bothered:  these
don't seem like axioms at all, at least not in the sense that you want to
think of axioms.  Axioms imply a starting point from which everything else can
be built up, according to precise rules specified in the axiom system.  Even
though you can understand the first few proofs in an informal way,
and are able to do some of the
exercises, it's hard to pin down precisely what the
rules are.   Sure, each step seems to follow logically from the others, but
exactly what does that mean?  Is the ``logic'' just a matter of common sense,
something vague that we all understand but can never quite state precisely?

You've spent a number of years, off and on, programming computers, and you
know that in the case of computer languages there is no question of what the
rules are---they are precise and crystal clear.  If you follow them, your
program will work, and if you don't, it won't.  No matter how complex a
program, it can always be broken down into simpler and simpler pieces, until
you can ultimately identify the bits that are moved around to perform a
specific function.  Some programs might require a lot of perseverance to
accomplish this, but if you focus on a specific portion of it, you don't even
necessarily have to know how the rest of it works. Shouldn't there be an
analogy in mathematics?

You decide to apply the ultimate test:  you ask yourself how a computer could
verify or ensure that the steps in these proofs follow from one another.
Certainly mathematics must be at least as precisely defined as a computer
language, if not more so; after all, computer science itself is based on it.
If you can get a computer to verify these proofs, then you should also be
able, in principle, to understand them yourself in a very crystal clear,
precise way.

You're in for a surprise:  you can conceive of no way to convert the
proofs, which are in English, to a form that the computer can understand.
The proofs are filled with phrases such as ``assume there exists a unique
$x$\ldots'' and ``given any $y$, let $z$ be the number such that\ldots''  This
isn't the kind of logic you are used to in computer programming, where
everything, even arithmetic, reduces to Boolean ones and zeroes if you care to
break it down sufficiently.  Even though you think you understand the proofs,
there seems to be some kind of higher reasoning involved rather than precise
rules that define how you manipulate the symbols in the axioms.  Whatever it
is, it just isn't obvious how you would express it to a computer, and the more
you think about it, the more puzzled and confused you get, to the point where
you even wonder whether {\em you} really understand it.  There's a lot more to
these axioms of arithmetic than meets the eye.

Nobody ever talked about this in school in your applied math and engineering
courses.  You just learned the rules they gave you, not quite understanding
how or why they worked, sometimes vaguely suspicious or uncertain of them, and
through homework problems and osmosis learned how to present solutions that
satisfied the instructor and earned you an ``A.''  Rarely did you actually
``prove'' anything in a rigorous way, and the math majors who did do stuff
like that seemed to be in a different world.

Of course, there are computer algebra programs that can do mathematics, and
rather impressively.  They can instantly solve the integrals that you
struggled with in freshman calculus, and do much, much more.  But when you
look at these programs, what you see is a big collection of algorithms and
techniques that evolved and were added to over time, along with some basic
software that manipulates symbols.  Each algorithm that is built in is the
result of someone's theorem whose proof is omitted; you just have to trust the
person who proved it and the person who programmed it in and hope there are no
bugs.\index{computer program bugs}  Somehow this doesn't seem to be the
essence of mathematics.  Although computer algebra systems can generate
theorems with amazing speed, they can't actually prove a single one of them.

After some puzzlement, you revisit some popular books on what mathematics is
all about.  Somewhere you read that all of mathematics is actually derived
from something called ``set theory.''  This is a little confusing, because
nowhere in the book that presented the axioms of arithmetic was there any
mention of set theory, or if there was, it seemed to be just a tool that helps
you describe things better---the set of even numbers, that sort of thing.  If
set theory is the basis for all mathematics, then why are additional axioms
needed for arithmetic?

Something is wrong but you're not sure what.  One of your friends is a pure
mathematician.  He knows he is unable to communicate to you what he does for a
living and seems to have little interest in trying.  You do know that for him,
proofs are what mathematics is all about. You ask him what a proof is, and he
essentially tells you that, while of course it's based on logic, really it's
something you learn by doing it over and over until you pick it up.  He refers
you to a book, {\em How to Read and Do Proofs} \cite{Solow}.\index{Solow,
Daniel}  Although this book helps you understand traditional informal proofs,
there is still something missing you can't seem to pin down yet.

You ask your friend how you would go about having a computer verify a proof.
At first he seems puzzled by the question; why would you want to do that?
Then he says it's not something that would make any sense to do, but he's
heard that you'd have to break the proof down into thousands or even millions
of individual steps to do such a thing, because the reasoning involved is at
such a high level of abstraction.  He says that maybe it's something you could
do up to a point, but the computer would be completely impractical once you
get into any meaningful mathematics.  There, the only way you can verify a
proof is by hand, and you can only acquire the ability to do this by
specializing in the field for a couple of years in grad school.  Anyway, he
thinks it all has to do with set theory, although he has never taken a formal
course in set theory but just learned what he needed as he went along.

You are intrigued and amazed.  Apparently a mathematician can grasp as a
single concept something that would take a computer a thousand or a million
steps to verify, and have complete confidence in it.  Each one of these
thousand or million steps must be absolutely correct, or else the whole proof
is meaningless.  If you added a million numbers by hand, would you trust the
result?  How do you really know that all these steps are correct, that there
isn't some subtle pitfall in one of these million steps, like a bug in a
computer program?\index{computer program bugs}  After all, you've read that
famous mathematicians have occasionally made mistakes, and you certainly know
you've made your share on your math homework problems in school.

You recall the analogy with a computer program.  Sure, you can understand what
a large computer program such as a word processor does, as a single high-level
concept or a small set of such concepts, but your ability to understand it in
no way ensures that the program is correct and doesn't have hidden bugs.  Even
if you wrote the program yourself you can't really know this; most large
programs that you've written have had bugs that crop up at some later date, no
matter how careful you tried to be while writing them.

OK, so now it seems the reason you can't figure out how to make a
computer verify proofs is because each step really corresponds to a
million small steps.  Well, you say, a computer can do a million
calculations in a second, so maybe it's still practical to do.  Now the
puzzle becomes how to figure out what the million steps are that each
English-language step corresponds to.  Your mathematician friend hasn't
a clue, but suggests that maybe you would find the answer by studying
set theory.  Actually, your friend thinks you're a little off the wall
for even wondering such a thing.  For him, this is not what mathematics
is all about.

The subject of set theory keeps popping up, so you decide it's
time to look it up.

You decide to start off on a careful footing, so you start reading a couple of
very elementary books on set theory.  A lot of it seems pretty obvious, like
intersections, subsets, and Venn diagrams.  You thumb through one of the
books; nowhere is anything about axioms mentioned. The other book relegates to
an appendix a brief discussion that mentions a set of axioms called
``Zermelo--Fraenkel set theory''\index{Zermelo--Fraenkel set theory} and states
them in English.  You look at them and have no idea what they really mean or
what you can do with them.  The comments in this appendix say that the purpose
of mentioning them is to expose you to the idea, but imply that they are not
necessary for basic understanding and that they are really the subject matter
of advanced treatments where fine points such as a certain paradox (Russell's
paradox\index{Russell's paradox}\footnote{Russell's paradox assumes that there
exists a set $S$ that is a collection of all sets that don't contain
themselves.  Now, either $S$ contains itself or it doesn't.  If it contains
itself, it contradicts its definition.  But if it doesn't contain itself, it
also contradicts its definition.  Russell's paradox is resolved in ZF set
theory by denying that such a set $S$ exists.}) are resolved.  Wait a
minute---shouldn't the axioms be a starting point, not an ending point?  If
there are paradoxes that arise without the axioms, how do you know you won't
stumble across one accidentally when using the informal approach?

And nowhere do these books describe how ``all of mathematics can be
derived from set theory'' which by now you've heard a few times.

You find a more advanced book on set theory.  This one actually lists the
axioms of ZF set theory in plain English on page one.  {\em Now} you think
your quest has ended and you've finally found the source of all mathematical
knowledge; you just have to understand what it means.  Here, in one place, is
the basis for all of mathematics!  You stare at the axioms in awe, puzzle over
them, memorize them, hoping that if you just meditate on them long enough they
will become clear.  Of course, you haven't the slightest idea how the rest of
mathematics is ``derived'' from them; in particular, if these are the axioms
of mathematics, then why do arithmetic, group theory, and so on need their own
axioms?

You start reading this advanced book carefully, pondering the meaning of every
word, because by now you're really determined to get to the bottom of this.
The first thing the book does is explain how the axioms came about, which was
to resolve Russell's paradox.\index{Russell's paradox}  In fact that seems to
be the main purpose of their existence; that they supposedly can be used to
derive all of mathematics seems irrelevant and is not even mentioned.  Well,
you go on.  You hope the book will explain to you clearly, step by step, how
to derive things from the axioms.  After all, this is the starting point of
mathematics, like a book that explains the basics of a computer programming
language.  But something is missing.  You find you can't even understand the
first proof or do the first exercise.  Symbols such as $\exists$ and $\forall$
permeate the page without any mention of where they came from or how to
manipulate them; the author assumes you are totally familiar with them and
doesn't even tell you what they mean.  By now you know that $\exists$ means
``there exists'' and $\forall$ means ``for all,'' but shouldn't the rules for
manipulating these symbols be part of the axioms?  You still have no idea
how you could even describe the axioms to a computer.

Certainly there is something much different here from the technical
literature you're used to reading.  A computer language manual almost
always explains very clearly what all the symbols mean, precisely what
they do, and the rules used for combining them, and you work your way up
from there.

After glancing at four or five other such books, you come to the realization
that there is another whole field of study that you need just to get to the
point at which you can understand the axioms of set theory.  The field is
called ``logic.''  In fact, some of the books did recommend it as a
prerequisite, but it just didn't sink in.  You assumed logic was, well, just
logic, something that a person with common sense intuitively understood.  Why
waste your time reading boring treatises on symbolic logic, the manipulation
of 1's and 0's that computers do, when you already know that?  But this is a
different kind of logic, quite alien to you.  The subject of {\sc nand} and
{\sc nor} gates is not even touched upon or in any case has to do with only a
very small part of this field.

So your quest continues.  Skimming through the first couple of introductory
books, you get a general idea of what logic is about and what quantifiers
(``for all,'' ``there exists'') mean, but you find their examples somewhat
trivial and mildly annoying (``all dogs are animals,'' ``some animals are
dogs,'' and such).  But all you want to know is what the rules are for
manipulating the symbols so you can apply them to set theory.  Some formulas
describing the relationships among quantifiers ($\exists$ and $\forall$) are
listed in tables, along with some verbal reasoning to justify them.
Presumably, if you want to find out if a formula is correct, you go through
this same kind of mental reasoning process, possibly using images of dogs and
animals. Intuitively, the formulas seem to make sense.  But when you ask
yourself, ``What are the rules I need to get a computer to figure out whether
this formula is correct?'', you still don't know.  Certainly you don't ask the
computer to imagine dogs and animals.

You look at some more advanced logic books.  Many of them have an introductory
chapter summarizing set theory, which turns out to be a prerequisite.  You
need logic to understand set theory, but it seems you also need set theory to
understand logic!  These books jump right into proving rather advanced
theorems about logic, without offering the faintest clue about where the logic
came from that allows them to prove these theorems.

Luckily, you come across an elementary book of logic that, halfway through,
after the usual truth tables and metaphors, presents in a clear, precise way
what you've been looking for all along: the axioms!  They're divided into
propositional calculus (also called sentential logic) and predicate calculus
(also called first-order logic),\index{first-order logic} with rules so simple
and crystal clear that now you can finally program a computer to understand
them.  Indeed, they're no harder than learning how to play a game of chess.
As far as what you seem to need is concerned, the whole book could have been
written in five pages!

{\em Now} you think you've found the ultimate source of mathematical
truth.  So---the axioms of mathematics consist of these axioms of logic,
together with the axioms of ZF set theory. (By now you've also been able to
figure out how to translate the ZF axioms from English into the
actual symbols of logic which you can now manipulate according to
precise, easy-to-understand rules.)

Of course, you still don't understand how ``all of mathematics can be
derived from set theory,'' but maybe this will reveal itself in due
course.

You eagerly set out to program the axioms and rules into a computer and start
to look at the theorems you will have to prove as the logic is developed.  All
sorts of important theorems start popping up:  the deduction
theorem,\index{deduction theorem} the substitution theorem,\index{substitution
theorem} the completeness theorem of propositional calculus,\index{first-order
logic!completeness} the completeness theorem of predicate calculus.  Uh-oh,
there seems to be trouble.  They all get harder and harder, and not one of
them can be derived with the axioms and rules of logic you've just been
handed.  Instead, they all require ``metalogic'' for their proofs, a kind of
mixture of logic and set theory that allows you to prove things {\em about}
the axioms and theorems of logic rather than {\em with} them.

You plow ahead anyway.  A month later, you've spent much of your
free time getting the computer to verify proofs in propositional calculus.
You've programmed in the axioms, but you've also had to program in the
deduction theorem, the substitution theorem, and the completeness theorem of
propositional calculus, which by now you've resigned yourself to treating as
rather complex additional axioms, since they can't be proved from the axioms
you were given.  You can now get the computer to verify and even generate
complete, rigorous, formal proofs\index{formal proof}.  Never mind that they
may have 100,000 steps---at least now you can have complete, absolute
confidence in them.  Unfortunately, the only theorems you have proved are
pretty trivial and you can easily verify them in a few minutes with truth
tables, if not by inspection.

It looks like your mathematician friend was right.  Getting the computer to do
serious mathematics with this kind of rigor seems almost hopeless.  Even
worse, it seems that the further along you get, the more ``axioms'' you have
to add, as each new theorem seems to involve additional ``metamathematical''
reasoning that hasn't been formalized, and none of it can be derived from the
axioms of logic.  Not only do the proofs keep growing exponentially as you get
further along, but the program to verify them keeps getting bigger and bigger
as you program in more ``metatheorems.''\index{metatheorem}\footnote{A
metatheorem is usually a statement that is too general to be directly provable
in a theory.  For example, ``if $n_1$, $n_2$, and $n_3$ are integers, then
$n_1+n_2+n_3$ is an integer'' is a theorem of number theory.  But ``for any
integer $k > 1$, if $n_1, \ldots, n_k$ are integers, then $n_1+\ldots +n_k$ is
an integer'' is a metatheorem, in other words a family of theorems, one for
every $k$.  The reason it is not a theorem is that the general sum $n_1+\ldots
+n_k$ (as a function of $k$) is not an operation that can be defined directly
in number theory.} The bugs\index{computer program bugs} that have cropped up
so far have already made you start to lose faith in the rigor you seem to have
achieved, and you know it's just going to get worse as your program gets larger.

\subsection{Mathematics and the Non-Specialist}

\begin{quote}
  {\em A real proof is not checkable by a machine, or even by any mathematician
not privy to the gestalt, the mode of thought of the particular field of
mathematics in which the proof is located.}
  \flushright\sc  Davis and Hersh\index{Davis, Phillip J.}
  \footnote{\cite{Davis}, p.~354.}\\
\end{quote}

The bulk of abstract or theoretical mathematics is ordinarily outside
the reach of anyone but a few specialists in each field who have completed
the necessary difficult internship in order to enter its coterie.  The
typical intelligent layperson has no reasonable hope of understanding much of
it, nor even the specialist mathematician of understanding other fields.  It
is like a foreign language that has no dictionary to look up the translation;
the only way you can learn it is by living in the country for a few years.  It
is argued that the effort involved in learning a specialty is a necessary
process for acquiring a deep understanding.  Of course, this is almost certainly
true if one is to make significant contributions to a field; in particular,
``doing'' proofs is probably the most important part of a mathematician's
training.  But is it also necessary to deny outsiders access to it?  Is it
necessary that abstract mathematics be so hard for a layperson to grasp?

A computer normally is of no help whatsoever.  Most published proofs are
actually just series of hints written in an informal style that requires
considerable knowledge of the field to understand.  These are the ``real
proofs'' referred to by Davis and Hersh.\index{informal proof}  There is an
implicit understanding that, in principle, such a proof could be converted to
a complete formal proof\index{formal proof}.  However, it is said that no one
would ever attempt such a conversion, even if they could, because that would
presumably require millions of steps (Section~\ref{dream}).  Unfortunately the
informal style automatically excludes the understanding of the proof
by anyone who hasn't gone through the necessary apprenticeship. The
best that the intelligent layperson can do is to read popular books about deep
and famous results; while this can be helpful, it can also be misleading, and
the lack of detail usually leaves the reader with no ability whatsoever to
explore any aspect of the field being described.

The statements of theorems often use sophisticated notation that makes them
inaccessible to the non-specialist.  For a non-specialist who wants to achieve
a deeper understanding of a proof, the process of tracing definitions and
lemmas back through their hierarchy\index{hierarchy} quickly becomes confusing
and discouraging.  Textbooks are usually written to train mathematicians or to
communicate to people who are already mathematicians, and large gaps in proofs
are often left as exercises to the reader who is left at an impasse if he or
she becomes stuck.

I believe that eventually computers will enable non-specialists and even
intelligent laypersons to follow almost any mathematical proof in any field.
Metamath is an attempt in that direction.  If all of mathematics were as
easily accessible as a computer programming language, I could envision
computer programmers and hobbyists who otherwise lack mathematical
sophistication exploring and being amazed by the world of theorems and proofs
in obscure specialties, perhaps even coming up with results of their own.  A
tremendous advantage would be that anyone could experiment with conjectures in
any field---the computer would offer instant feedback as to whether
an inference step was correct.

Mathematicians sometimes have to put up with the annoyance of
cranks\index{cranks} who lack a fundamental understanding of mathematics but
insist that their ``proofs'' of, say, Fermat's Last Theorem\index{Fermat's
Last Theorem} be taken seriously.  I think part of the problem is that these
people are misled by informal mathematical language, treating it as if they
were reading ordinary expository English and failing to appreciate the
implicit underlying rigor.  Such cranks are rare in the field of computers,
because computer languages are much more explicit, and ultimately the proof is
in whether a computer program works or not.  With easily accessible
computer-based abstract mathematics, a mathematician could say to a crank,
``don't bother me until you've demonstrated your claim on the computer!''

% 22-May-04 nm
% Attempt to move De Millo quote so it doesn't separate from attribution
% CHANGE THIS NUMBER (AND ELIMINATE IF POSSIBLE) WHEN ABOVE TEXT CHANGES
\vspace{-0.5em}

\subsection{An Impossible Dream?}\label{dream}

\begin{quote}
  {\em Even quite basic theorems would demand almost unbelievably vast
  books to display their proofs.}
    \flushright\sc  Robert E. Edwards\footnote{\cite{Edwards}, p.~68.}\\
\end{quote}\index{Edwards, Robert E.}

\begin{quote}
  {\em Oh, of course no one ever really {\em does} it.  It would take
  forever!  You just show that you could do it, that's sufficient.}
    \flushright\sc  ``The Ideal Mathematician''
    \index{Davis, Phillip J.}\footnote{\cite{Davis},
p.~40.}\\
\end{quote}

\begin{quote}
  {\em There is a theorem in the primitive notation of set theory that
  corresponds to the arithmetic theorem `$1000+2000=3000$'.  The formula
  would be forbiddingly long\ldots even if [one] knows the definitions
  and is asked to simplify the long formula according to them, chances are
  he will make errors and arrive at some incorrect result.}
    \flushright\sc  Hao Wang\footnote{\cite{Wang}, p.~140.}\\
\end{quote}\index{Wang, Hao}

% 22-May-04 nm
% Attempt to move De Millo quote so it doesn't separate from attribution
% CHANGE THIS NUMBER (AND ELIMINATE IF POSSIBLE) WHEN ABOVE TEXT CHANGES
\vspace{-0.5em}

\begin{quote}
  {\em The {\em Principia Mathematica} was the crowning achievement of the
  formalists.  It was also the deathblow of the formalist view.\ldots
  {[Rus\-sell]} failed, in three enormous volumes, to get beyond the elementary
  facts of arithmetic.  He showed what can be done in principle and what
  cannot be done in practice.  If the mathematical process were really
  one of strict, logical progression, we would still be counting our
  fingers.\ldots
  One theoretician estimates\ldots that a demonstration of one of
  Ramanujan's conjectures assuming set theory and elementary analysis would
  take about two thousand pages; the length of a deduction from first principles
  is nearly in\-con\-ceiv\-a\-ble\ldots The probabilists argue that\ldots any
  very long proof can at best be viewed as only probably correct\ldots}
  \flushright\sc Richard de Millo et. al.\footnote{\cite{deMillo}, pp.~269,
  271.}\\
\end{quote}\index{de Millo, Richard}

A number of writers have conveyed the impression that the kind of absolute
rigor provided by Metamath\index{Metamath} is an impossible dream, suggesting
that a complete, formal verification\index{formal proof} of a typical theorem
would take millions of steps in untold volumes of books.  Even if it could be
done, the thinking sometimes goes, all meaning would be lost in such a
monstrous, tedious verification.\index{informal proof}\index{proof length}

These writers assume, however, that in order to achieve the kind of complete
formal verification they desire one must break down a proof into individual
primitive steps that make direct reference to the axioms.  This is
not necessary.  There is no reason not to make use of previously proved
theorems rather than proving them over and over.

Just as important, definitions\index{definition} can be introduced along
the way, allowing very complex formulas to be represented with few
symbols.  Not doing this can lead to absurdly long formulas.  For
example, the mere statement of
G\"{o}del's incompleteness theorem\index{G\"{o}del's
incompleteness theorem}, which can be expressed with a small number of
defined symbols, would require about 20,000 primitive symbols to express
it.\index{Boolos, George S.}\footnote{George S.\ Boolos, lecture at
Massachusetts Institute of Technology, spring 1990.} An extreme example
is Bourbaki's\label{bourbaki} language for set theory, which requires
4,523,659,424,929 symbols plus 1,179,618,517,981 disambiguatory links
(lines connecting symbol pairs, usually drawn below or above the
formula) to express the number
``one'' \cite{Mathias}.\index{Mathias, Adrian R. D.}\index{Bourbaki,
Nicolas}
% http://www.dpmms.cam.ac.uk/~ardm/

A hierarchy\index{hierarchy} of theorems and definitions permits an
exponential growth in the formula sizes and primitive proof steps to be
described with only a linear growth in the number of symbols used.  Of course,
this is how ordinary informal mathematics is normally done anyway, but with
Metamath\index{Metamath} it can be done with absolute rigor and precision.

\subsection{Beauty}


\begin{quote}
  {\em No one shall be able to drive us from the paradise that Cantor has
created for us.}
   \flushright\sc  David Hilbert\footnote{As quoted in \cite{Moore}, p.~131.}\\
\end{quote}\index{Hilbert, David}

\needspace{3\baselineskip}
\begin{quote}
  {\em Mathematics possesses not only truth, but some supreme beauty ---a
  beauty cold and austere, like that of a sculpture.}
    \flushright\sc  Bertrand
    Russell\footnote{\cite{Russell}.}\\
\end{quote}\index{Russell, Bertrand}

\begin{quote}
  {\em Euclid alone has looked on Beauty bare.}
  \flushright\sc Edna Millay\footnote{As quoted in \cite{Davis}, p.~150.}\\
\end{quote}\index{Millay, Edna}

For most people, abstract mathematics is distant, strange, and
incomprehensible.  Many popular books have tried to convey some of the sense
of beauty in famous theorems.  But even an intelligent layperson is left with
only a general idea of what a theorem is about and is hardly given the tools
needed to make use of it.  Traditionally, it is only after years of arduous
study that one can grasp the concepts needed for deep understanding.
Metamath\index{Metamath} allows you to approach the proof of the theorem from
a quite different perspective, peeling apart the formulas and definitions
layer by layer until an entirely different kind of understanding is achieved.
Every step of the proof is there, pieced together with absolute precision and
instantly available for inspection through a microscope with a magnification
as powerful as you desire.

A proof in itself can be considered an object of beauty.  Constructing an
elegant proof is an art.  Once a famous theorem has been proved, often
considerable effort is made to find simpler and more easily understood
proofs.  Creating and communicating elegant proofs is a major concern of
mathematicians.  Metamath is one way of providing a common language for
archiving and preserving this information.

The length of a proof can, to a certain extent, be considered an
objective measure of its ``beauty,'' since shorter proofs are usually
considered more elegant.  In the set theory database
\texttt{set.mm}\index{set theory database (\texttt{set.mm})}%
\index{Metamath Proof Explorer}
provided with Metamath, one goal was to make all proofs as short as possible.

\needspace{4\baselineskip}
\subsection{Simplicity}

\begin{quote}
  {\em God made man simple; man's complex problems are of his own
  devising.}
    \flushright\sc Eccles. 7:29\footnote{Jerusalem Bible.}\\
\end{quote}\index{Bible}

\needspace{3\baselineskip}
\begin{quote}
  {\em God made integers, all else is the work of man.}
    \flushright\sc Leopold Kronecker\footnote{{\em Jahresbericht
	der Deutschen Mathematiker-Vereinigung }, vol. 2, p. 19.}\\
\end{quote}\index{Kronecker, Leopold}

\needspace{3\baselineskip}
\begin{quote}
  {\em For what is clear and easily comprehended attracts; the
  complicated repels.}
    \flushright\sc David Hilbert\footnote{As quoted in \cite{deMillo},
p.~273.}\\
\end{quote}\index{Hilbert, David}

The Metamath\index{Metamath} language is simple and Spartan.  Metamath treats
all mathematical expressions as simple sequences of symbols, devoid of meaning.
The higher-level or ``metamathematical'' notions underlying Metamath are about
as simple as they could possibly be.  Each individual step in a proof involves
a single basic concept, the substitution of an expression for a variable, so
that in principle almost anyone, whether mathematician or not, can
completely understand how it was arrived at.

In one of its most basic applications, Metamath\index{Metamath} can be used to
develop the foundations of mathematics\index{foundations of mathematics} from
the very beginning.  This is done in the set theory database that is provided
with the Metamath package and is the subject matter
of Chapter~\ref{fol}. Any language (a metalanguage\index{metalanguage})
used to describe mathematics (an object language\index{object language}) must
have a mathematical content of its own, but it is desirable to keep this
content down to a bare minimum, namely that needed to make use of the
inference rules specified by the axioms.  With any metalanguage there is a
``chicken and egg'' problem somewhat like circular reasoning:  you must assume
the validity of the mathematics of the metalanguage in order to prove the
validity of the mathematics of the object language.  The mathematical content
of Metamath itself is quite limited.  Like the rules of a game of chess, the
essential concepts are simple enough so that virtually anyone should be able to
understand them (although that in itself will not let you play like
a master).  The symbols that Metamath manipulates do not in themselves
have any intrinsic meaning.  Your interpretation of the axioms that you supply
to Metamath is what gives them meaning.  Metamath is an attempt to strip down
mathematical thought to its bare essence and show you exactly how the symbols
are manipulated.

Philosophers and logicians, with various motivations, have often thought it
important to study ``weak'' fragments of logic\index{weak logic}
\cite{Anderson}\index{Anderson, Alan Ross} \cite{MegillBunder}\index{Megill,
Norman}\index{Bunder, Martin}, other unconventional systems of logic (such as
``modal'' logic\index{modal logic} \cite[ch.\ 27]{Boolos}\index{Boolos, George
S.}), and quantum logic\index{quantum logic} in physics
\cite{Pavicic}\index{Pavi{\v{c}}i{\'{c}}, M.}.  Metamath\index{Metamath}
provides a framework in which such systems can be expressed, with an absolute
precision that makes all underlying metamathematical assumptions rigorous and
crystal clear.

Some schools of philosophical thought, for example
intuitionism\index{intuitionism} and constructivism\index{constructivism},
demand that the notions underlying any mathematical system be as simple and
concrete as possible.  Metamath should meet the requirements of these
philosophies.  Metamath must be taught the symbols, axioms\index{axiom}, and
rules\index{rule} for a specific theory, from the skeptical (such as
intuitionism\index{intuitionism}\footnote{Intuitionism does not accept the law
of excluded middle (``either something is true or it is not true'').  See
\cite[p.~xi]{Tymoczko}\index{Tymoczko, Thomas} for discussion and references
on this topic.  Consider the theorem, ``There exist irrational numbers $a$ and
$b$ such that $a^b$ is rational.''  An intuitionist would reject the following
proof:  If $\sqrt{2}^{\sqrt{2}}$ is rational, we are done.  Otherwise, let
$a=\sqrt{2}^{\sqrt{2}}$ and $b=\sqrt{2}$. Then $a^b=2$, which is rational.})
to the bold (such as the axiom of choice in set theory\footnote{The axiom of
choice\index{Axiom of Choice} asserts that given any collection of pairwise
disjoint nonempty sets, there exists a set that has exactly one element in
common with each set of the collection.  It is used to prove many important
theorems in standard mathematics.  Some philosophers object to it because it
asserts the existence of a set without specifying what the set contains
\cite[p.~154]{Enderton}\index{Enderton, Herbert B.}.  In one foundation for
mathematics due to Quine\index{Quine, Willard Van Orman}, that has not been
otherwise shown to be inconsistent, the axiom of choice turns out to be false
\cite[p.~23]{Curry}\index{Curry, Haskell B.}.  The \texttt{show
trace{\char`\_}back} command of the Metamath program allows you to find out
whether the axiom of choice, or any other axiom, was assumed by a
proof.}\index{\texttt{show trace{\char`\_}back} command}).

The simplicity of the Metamath language lets the algorithm (computer program)
that verifies the validity of a Metamath proof be straightforward and
robust.  You can have confidence that the theorems it verifies really can be
derived from your axioms.

\subsection{Rigor}

\begin{quote}
  {\em Rigor became a goal with the Greeks\ldots But the efforts to
  pursue rigor to the utmost have led to an impasse in which there is
  no longer any agreement on what it really means.  Mathematics remains
  alive and vital, but only on a pragmatic basis.}
    \flushright\sc  Morris Kline\footnote{\cite{Kline}, p.~1209.}\\
\end{quote}\index{Kline, Morris}

Kline refers to a much deeper kind of rigor than that which we will discuss in
this section.  G\"{o}del's incompleteness theorem\index{G\"{o}del's
incompleteness theorem} showed that it is impossible to achieve absolute rigor
in standard mathematics because we can never prove that mathematics is
consistent (free from contradictions).\index{consistent theory}  If
mathematics is consistent, we will never know it, but must rely on faith.  If
mathematics is inconsistent, the best we can hope for is that some clever
future mathematician will discover the inconsistency.  In this case, the
axioms would probably be revised slightly to eliminate the inconsistency, as
was done in the case of Russell's paradox,\index{Russell's paradox} but the
bulk of mathematics would probably not be affected by such a discovery.
Russell's paradox, for example, did not affect most of the remarkable results
achieved by 19th-century and earlier mathematicians.  It mainly invalidated
some of Gottlob Frege's\index{Frege, Gottlob} work on the foundations of
mathematics in the late 1800's; in fact Frege's work inspired Russell's
discovery.  Despite the paradox, Frege's work contains important concepts that
have significantly influenced modern logic.  Kline's {\em Mathematics, The
Loss of Certainty} \cite{Klinel}\index{Kline, Morris} has an interesting
discussion of this topic.

What {\em can} be achieved with absolute certainty\index{certainty} is the
knowledge that if we assume the axioms are consistent and true, then the
results derived from them are true.  Part of the beauty of mathematics is that
it is the one area of human endeavor where absolute certainty can be achieved
in this sense.  A mathematical truth will remain such for eternity.  However,
our actual knowledge of whether a particular statement is a mathematical truth
is only as certain as the correctness of the proof that establishes it.  If
the proof of a statement is questionable or vague, we can't have absolute
confidence in the truth that the statement claims.

Let us look at some traditional ways of expressing proofs.

Except in the field of formal logic\index{formal logic}, almost all
traditional proofs in mathematics are really not proofs at all, but rather
proof outlines or hints as to how to go about constructing the proof.  Many
gaps\index{gaps in proofs} are left for the reader to fill in. There are
several reasons for this.  First, it is usually assumed in mathematical
literature that the person reading the proof is a mathematician familiar with
the specialty being described, and that the missing steps are obvious to such
a reader or at least that the reader is capable of filling them in.  This
attitude is fine for professional mathematicians in the specialty, but
unfortunately it often has the drawback of cutting off the rest of the world,
including mathematicians in other specialties, from understanding the proof.
We discussed one possible resolution to this on p.~\pageref{envision}.
Second, it is often assumed that a complete formal proof\index{formal proof}
would require countless millions of symbols (Section~\ref{dream}). This might
be true if the proof were to be expressed directly in terms of the axioms of
logic and set theory,\index{set theory} but it is usually not true if we allow
ourselves a hierarchy\index{hierarchy} of definitions and theorems to build
upon, using a notation that allows us to introduce new symbols, definitions,
and theorems in a precisely specified way.

Even in formal logic,\index{formal logic} formal proofs\index{formal proof}
that are considered complete still contain hidden or implicit information.
For example, a ``proof'' is usually defined as a sequence of
wffs,\index{well-formed formula (wff)}\footnote{A {\em wff} or well-formed
formula is a mathematical expression (string of symbols) constructed according
to some precise rules.  A formal mathematical system\index{formal system}
contains (1) the rules for constructing syntactically correct
wffs,\index{syntax rules} (2) a list of starting wffs called
axioms,\index{axiom} and (3) one or more rules prescribing how to derive new
wffs, called theorems\index{theorem}, from the axioms or previously derived
theorems.  An example of such a system is contained in
Metamath's\index{Metamath} set theory database, which defines a formal
system\index{formal system} from which all of standard mathematics can be
derived.  Section~\ref{startf} steps you through a complete example of a formal
system, and you may want to skim it now if you are unfamiliar with the
concept.} each of which is an axiom or follows from a rule applied to previous
wffs in the sequence.  The implicit part of the proof is the algorithm by
which a sequence of symbols is verified to be a valid wff, given the
definition of a wff.  The algorithm in this case is rather simple, but for a
computer to verify the proof,\index{automated proof verification} it must have
the algorithm built into its verification program.\footnote{It is possible, of
course, to specify wff construction syntax outside of the program itself
with a suitable input language (the Metamath language being an example), but
some proof-verification or theorem-proving programs lack the ability to extend
wff syntax in such a fashion.} If one deals exclusively with axioms and
elementary wffs, it is straightforward to implement such an algorithm.  But as
more and more definitions are added to the theory in order to make the
expression of wffs more compact, the algorithm becomes more and more
complicated.  A computer program that implements the algorithm becomes larger
and harder to understand as each definition is introduced, and thus more prone
to bugs.\index{computer program bugs}  The larger the program, the
more suspicious the mathematician may be about
the validity of its algorithms.  This is especially true because
computer programs are inherently hard to follow to begin with, and few people
enjoy verifying them manually in detail.

Metamath\index{Metamath} takes a different approach.  Metamath's ``knowledge''
is limited to the ability to substitute variables for expressions, subject to
some simple constraints.  Once the basic algorithm of Metamath is assumed to
be debugged, and perhaps independently confirmed, it
can be trusted once and for all.  The information that Metamath needs to
``understand'' mathematics is contained entirely in the body of knowledge
presented to Metamath.  Any errors in reasoning can only be errors in the
axioms or definitions contained in this body of knowledge.  As a
``constructive'' language\index{constructive language} Metamath has no
conditional branches or loops like the ones that make computer programs hard
to decipher; instead, the language can only build new sequences of symbols
from earlier sequences  of symbols.

The simplicity of the rules that underlie Metamath not only makes Metamath
easy to learn but also gives Metamath a great deal of flexibility. For
example, Metamath is not limited to describing standard first-order
logic\index{first-order logic}; higher-order logics\index{higher-order logic}
and fragments of logic\index{weak logic} can be described just as easily.
Metamath gives you the freedom to define whatever wff notation you prefer; it
has no built-in conception of the syntax of a wff.\index{well-formed formula
(wff)}  With suitable axioms and definitions, Metamath can even describe and
prove things about itself.\index{Metamath!self-description}  (John
Harrison\index{Harrison, John} discusses the ``reflection''
principle\index{reflection principle} involved in self-descriptive systems in
\cite{Harrison}.)

The flexibility of Metamath requires that its proofs specify a lot of detail,
much more than in an ordinary ``formal'' proof.\index{formal proof}  For
example, in an ordinary formal proof, a single step consists of displaying the
wff that constitutes that step.  In order for a computer program to verify
that the step is acceptable, it first must verify that the symbol sequence
being displayed is an acceptable wff.\index{automated proof verification} Most
proof verifiers have at least basic wff syntax built into their programs.
Metamath has no hard-wired knowledge of what constitutes a wff built into it;
instead every wff must be explicitly constructed based on rules defining wffs
that are present in a database.  Thus a single step in an ordinary formal
proof may be correspond to many steps in a Metamath proof. Despite the larger
number of steps, though, this does not mean that a Metamath proof must be
significantly larger than an ordinary formal proof. The reason is that since
we have constructed the wff from scratch, we know what the wff is, so there is
no reason to display it.  We only need to refer to a sequence of statements
that construct it.  In a sense, the display of the wff in an ordinary formal
proof is an implicit proof of its own validity as a wff; Metamath just makes
the proof explicit. (Section~\ref{proof} describes Metamath's proof notation.)

\section{Computers and Mathematicians}

\begin{quote}
  {\em The computer is important, but not to mathematics.}
    \flushright\sc  Paul Halmos\footnote{As quoted in \cite{Albers}, p.~121.}\\
\end{quote}\index{Halmos, Paul}

Pure mathematicians have traditionally been indifferent to computers, even to
the point of disdain.\index{computers and pure mathematics}  Computer science
itself is sometimes considered to fall in the mundane realm of ``applied''
mathematics, perhaps essential for the real world but intellectually unexciting
to those who seek the deepest truths in mathematics.  Perhaps a reason for this
attitude towards computers is that there is little or no computer software that
meets their needs, and there may be a general feeling that such software could
not even exist.  On the one hand, there are the practical computer algebra
systems, which can perform amazing symbolic manipulations in algebra and
calculus,\index{computer algebra system} yet can't prove the simplest
existence theorem, if the idea of a proof is present at all.  On the other
hand, there are specialized automated theorem provers that technically speaking
may generate correct proofs.\index{automated theorem proving}  But sometimes
their specialized input notation may be cryptic and their output perceived to
be long, inelegant, incomprehensible proofs.    The output
may be viewed with suspicion, since the program that generates it tends to be
very large, and its size increases the potential for bugs\index{computer
program bugs}.  Such a proof may be considered trustworthy only if
independently verified and ``understood'' by a human, but no one wants to
waste their time on such a boring, unrewarding chore.



\needspace{4\baselineskip}
\subsection{Trusting the Computer}

\begin{quote}
  {\em \ldots I continue to find the quasi-empirical interpretation of
  computer proofs to be the more plausible.\ldots Since not
  everything that claims to be a computer proof can be
  accepted as valid, what are the mathematical criteria for acceptable
  computer proofs?}
    \flushright\sc  Thomas Tymoczko\footnote{\cite{Tymoczko}, p.~245.}\\
\end{quote}\index{Tymoczko, Thomas}

In some cases, computers have been essential tools for proving famous
theorems.  But if a proof is so long and obscure that it can be verified in a
practical way only with a computer, it is vaguely felt to be suspicious.  For
example, proving the famous four-color theorem\index{four-color
theorem}\index{proof length} (``a map needs no more than four colors to
prevent any two adjacent countries from having the same color'') can presently
only be done with the aid of a very complex computer program which originally
required 1200 hours of computer time. There has been considerable debate about
whether such a proof can be trusted and whether such a proof is ``real''
mathematics \cite{Swart}\index{Swart, E. R.}.\index{trusting computers}

However, under normal circumstances even a skeptical mathematician would have a
great deal of confidence in the result of multiplying two numbers on a pocket
calculator, even though the precise details of what goes on are hidden from its
user.  Even the verification on a supercomputer that a huge number is prime is
trusted, especially if there is independent verification; no one bothers to
debate the philosophical significance of its ``proof,'' even though the actual
proof would be so large that it would be completely impractical to ever write
it down on paper.  It seems that if the algorithm used by the computer is
simple enough to be readily understood, then the computer can be trusted.

Metamath\index{Metamath} adopts this philosophy.  The simplicity of its
language makes it easy to learn, and because of its simplicity one can have
essentially absolute confidence that a proof is correct. All axioms, rules, and
definitions are available for inspection at any time because they are defined
by the user; there are no hidden or built-in rules that may be prone to subtle
bugs\index{computer program bugs}.  The basic algorithm at the heart of
Metamath is simple and fixed, and it can be assumed to be bug-free and robust
with a degree of confidence approaching certainty.
Independently written implementations of the Metamath verifier
can reduce any residual doubt on the part of a skeptic even further;
there are now over a dozen such implementations, written by many people.

\subsection{Trusting the Mathematician}\label{trust}

\begin{quote}
  {\em There is no Algebraist nor Mathematician so expert in his science, as
  to place entire confidence in any truth immediately upon his discovery of it,
  or regard it as any thing, but a mere probability.  Every time he runs over
  his proofs, his confidence encreases; but still more by the approbation of
  his friends; and is rais'd to its utmost perfection by the universal assent
  and applauses of the learned world.}
  \flushright\sc David Hume\footnote{{\em A Treatise of Human Nature}, as
  quoted in \cite{deMillo}, p.~267.}\\
\end{quote}\index{Hume, David}

\begin{quote}
  {\em Stanislaw Ulam estimates that mathematicians publish 200,000 theorems
  every year.  A number of these are subsequently contradicted or otherwise
  disallowed, others are thrown into doubt, and most are ignored.}
  \flushright\sc Richard de Millo et. al.\footnote{\cite{deMillo}, p.~269.}\\
\end{quote}\index{Ulam, Stanislaw}

Whether or not the computer can be trusted, humans  of course will occasionally
err. Only the most memorable proofs get independently verified, and of these
only a handful of truly great ones achieve the status of being ``known''
mathematical truths that are used without giving a second thought to their
correctness.

There are many famous examples of incorrect theorems and proofs in
mathematical literature.\index{errors in proofs}

\begin{itemize}
\item There have been thousands of purported proofs of Fermat's Last
Theorem\index{Fermat's Last Theorem} (``no integer solutions exist to $x^n +
y^n = z^n$ for $n > 2$''), by amateurs, cranks, and well-regarded
mathematicians \cite[p.~5]{Stark}\index{Stark, Harold M}.  Fermat wrote a note
in his copy of Bachet's {\em Diophantus} that he found ``a truly marvelous
proof of this theorem but this margin is too narrow to contain it''
\cite[p.~507]{Kramer}.  A recent, much publicized proof by Yoichi
Miyaoka\index{Miyaoka, Yoichi} was shown to be incorrect ({\em Science News},
April 9, 1988, p.~230).  The theorem was finally proved by Andrew
Wiles\index{Wiles, Andrew} ({\em Science News}, July 3, 1993, p.~5), but it
initially had some gaps and took over a year after its announcement to be
checked thoroughly by experts.  On Oct. 25, 1994, Wiles announced that the last
gap found in his proof had been filled in.
  \item In 1882, M. Pasch discovered that an axiom was omitted from Euclid's
formulation of geometry\index{Euclidean geometry}; without it, the proofs of
important theorems of Euclid are not valid.  Pasch's axiom\index{Pasch's
axiom} states that a line that intersects one side of a triangle must also
intersect another side, provided that it does not touch any of the triangle's
vertices.  The omission of Pasch's axiom went unnoticed for 2000
years \cite[p.~160]{Davis}, in spite of (one presumes) the thousands of
students, instructors, and mathematicians who studied Euclid.
  \item The first published proof of the famous Schr\"{o}der--Bernstein
theorem\index{Schr\"{o}der--Bernstein theorem} in set theory was incorrect
\cite[p.~148]{Enderton}\index{Enderton, Herbert B.}.  This theorem states
that if there exists a 1-to-1 function\footnote{A {\em set}\index{set} is any
collection of objects. A {\em function}\index{function} or {\em
mapping}\index{mapping} is a rule that assigns to each element of one set
(called the function's {\em domain}\index{domain}) an element from another
set.} from set $A$ into set $B$ and vice-versa, then sets $A$ and $B$ have
a 1-to-1 correspondence.  Although it sounds simple and obvious,
the standard proof is quite long and complex.
  \item In the early 1900's, Hilbert\index{Hilbert, David} published a
purported proof of the continuum hypothesis\index{continuum hypothesis}, which
was eventually established as unprovable by Cohen\index{Cohen, Paul} in 1963
\cite[p.~166]{Enderton}.  The continuum hypothesis states that no
infinity\index{infinity} (``transfinite cardinal number'')\index{cardinal,
transfinite} exists whose size (or ``cardinality''\index{cardinality}) is
between the size of the set of integers and the size of the set of real
numbers.  This hypothesis originated with German mathematician Georg
Cantor\index{Cantor, Georg} in the late 1800's, and his inability to prove it
is said to have contributed to mental illness that afflicted him in his later
years.
  \item An incorrect proof of the four-color theorem\index{four-color theorem}
was published by Kempe\index{Kempe, A. B.} in 1879
\cite[p.~582]{Courant}\index{Courant, Richard}; it stood for 11 years before
its flaw was discovered.  This theorem states that any map can be colored
using only four colors, so that no two adjacent countries have the same
color.  In 1976 the theorem was finally proved by the famous computer-assisted
proof of Haken, Appel, and Koch \cite{Swart}\index{Appel, K.}\index{Haken,
W.}\index{Koch, K.}.  Or at least it seems that way.  Mathematician
H.~S.~M.~Coxeter\index{Coxeter, H. S. M.} has doubts \cite[p.~58]{Davis}:  ``I
have a feeling that that is an untidy kind of use of the computers, and the more
you correspond with Haken and Appel, the more shaky you seem to be.''
  \item Many false ``proofs'' of the Poincar\'{e}
conjecture\index{Poincar\'{e} conjecture} have been proposed over the years.
This conjecture states that any object that mathematically behaves like a
three-dimensional sphere is a three-dimensional sphere topologically,
regardless of how it is distorted.  In March 1986, mathematicians Colin
Rourke\index{Rourke, Colin} and Eduardo R\^{e}go\index{R\^{e}go, Eduardo}
caused  a stir in the mathematical community by announcing that they had found
a proof; in November of that year the proof was found to be false \cite[p.
218]{PetersonI}.  It was finally proved in 2003 by Grigory Perelman
\label{poincare}\index{Szpiro, George}\index{Perelman, Grigory}\cite{Szpiro}.
 \end{itemize}

Many counterexamples to ``theorems'' in recent mathematical
literature related to Clifford algebras\index{Clifford algebras}
 have been found by Pertti
Lounesto (who passed away in 2002).\index{Lounesto, Pertti}
See the web page \url{http://mathforum.org/library/view/4933.html}.
% http://users.tkk.fi/~ppuska/mirror/Lounesto/counterexamples.htm

One of the purposes of Metamath\index{Metamath} is to allow proofs to be
expressed with absolute precision.  Developing a proof in the Metamath
language can be challenging, because Metamath will not permit even the
tiniest mistake.\index{errors in proofs}  But once the proof is created, its
correctness can be trusted immediately, without having to depend on the
process of peer review for confirmation.

\section{The Use of Computers in Mathematics}

\subsection{Computer Algebra Systems}

For the most part, you will find that Metamath\index{Metamath} is not a
practical tool for manipulating numbers.  (Even proving that $2 + 2 = 4$, if
you start with set theory, can be quite complex!)  Several commercial
mathematics packages are quite good at arithmetic, algebra, and calculus, and
as practical tools they are invaluable.\index{computer algebra system} But
they have no notion of proof, and cannot understand statements starting with
``there exists such and such...''.

Software packages such as Mathematica \cite{Wolfram}\index{Mathematica} do not
concern themselves with proofs but instead work directly with known results.
These packages primarily emphasize heuristic rules such as the substitution of
equals for equals to achieve simpler expressions or expressions in a different
form.  Starting with a rich collection of built-in rules and algorithms, users
can add to the collection by means of a powerful programming language.
However, results such as, say, the existence of a certain abstract object
without displaying the actual object cannot be expressed (directly) in their
languages.  The idea of a proof from a small set of axioms is absent.  Instead
this software simply assumes that each fact or rule you add to the built-in
collection of algorithms is valid.  One way to view the software is as a large
collection of axioms from which the software, with certain goals, attempts to
derive new theorems, for example equating a complex expression with a simpler
equivalent. But the terms ``theorem''\index{theorem} and
``proof,''\index{proof} for example, are not even mentioned in the index of
the user's manual for Mathematica.\index{Mathematica and proofs}  What is also
unsatisfactory from a philosophical point of view is that there is no way to
ensure the validity of the results other than by trusting the writer of each
application module or tediously checking each module by hand, similar to
checking a computer program for bugs.\index{computer program
bugs}\footnote{Two examples illustrate why the knowledge database of computer
algebra systems should sometimes be regarded with a certain caution.  If you
ask Mathematica (version 3.0) to \texttt{Solve[x\^{ }n + y\^{ }n == z\^{ }n , n]}
it will respond with \texttt{\{\{n-\char`\>-2\}, \{n-\char`\>-1\},
\{n-\char`\>1\}, \{n-\char`\>2\}\}}. In other words, Mathematica seems to
``know'' that Fermat's Last Theorem\index{Fermat's Last Theorem} is true!  (At
the time this version of Mathematica was released this fact was unknown.)  If
you ask Maple\index{Maple} to \texttt{solve(x\^{ }2 = 2\^{ }x)} then
\texttt{simplify(\{"\})}, it returns the solution set \texttt{\{2, 4\}}, apparently
unaware that $-0.7666647$\ldots is also a solution.} While of course extremely
valuable in applied mathematics, computer algebra systems tend to be of little
interest to the theoretical mathematician except as aids for exploring certain
specific problems.

Because of possible bugs, trusting the output of a computer algebra system for
use as theorems in a proof-verifier would defeat the latter's goal of rigor.
On the other hand, a fact such that a certain relatively large number is
prime, while easy for a computer algebra system to derive, might have a long,
tedious proof that could overwhelm a proof-verifier. One approach for linking
computer algebra systems to a proof-verifier while retaining the advantages of
both is to add a hypothesis to each such theorem indicating its source.  For
example, a constant {\sc maple} could indicate the theorem came from the Maple
package, and would mean ``assuming Maple is consistent, then\ldots''  This and
many other topics concerning the formalization of mathematics are discussed in
John Harrison's\index{Harrison, John} very interesting
PhD thesis~\cite{Harrison-thesis}.

\subsection{Automated Theorem Provers}\label{theoremprovers}

A mathematical theory is ``decidable''\index{decidable theory} if a mechanical
method or algorithm exists that is guaranteed to determine whether or not a
particular formula is a theorem.  Among the few theories that are decidable is
elementary geometry,\index{Euclidean geometry} as was shown by a classic
result of logician Alfred Tarski\index{Tarski, Alfred} in 1948
\cite{Tarski}.\footnote{Tarski's result actually applies to a subset of the
geometry discussed in elementary textbooks.  This subset includes most of what
would be considered elementary geometry but it is not powerful enough to
express, among other things, the notions of the circumference and area of a
circle.  Extending the theory in a way that includes notions such as these
makes the theory undecidable, as was also shown by Tarski.  Tarski's algorithm
is far too inefficient to implement practically on a computer.  A practical
algorithm for proving a smaller subset of geometry theorems (those not
involving concepts of ``order'' or ``continuity'') was discovered by Chinese
mathematician Wu Wen-ts\"{u}n in 1977 \cite{Chou}\index{Chou,
Shang-Ching}.}\index{Wen-ts{\"{u}}n, Wu}  But most theories, including
elementary arithmetic, are undecidable.  This fact contributes to keeping
mathematics alive and well, since many mathematicians believe
that they will never be
replaced by computers (if they believe Roger Penrose's argument that a
computer can never replace the brain \cite{Penrose}\index{Penrose, Roger}).
In fact,  elementary geometry is often considered a ``dead'' field
for the simple reason that it is decidable.

On the other hand, the undecidability of a theory does not mean that one cannot
use a computer to search for proofs, providing one is willing to give up if a
proof is not found after a reasonable amount of time.  The field of automated
theorem proving\index{automated theorem proving} specializes in pursuing such
computer searches.  Among the more successful results to date are those based
on an algorithm known as Robinson's resolution principle
\cite{Robinson}\index{Robinson's resolution principle}.

Automated theorem provers can be excellent tools for those willing to learn
how to use them.  But they are not widely used in mainstream pure
mathematics, even though they could probably be useful in many areas.  There
are several reasons for this.  Probably most important, the main goal in pure
mathematics is to arrive at results that are considered to be deep or
important; proving them is essential but secondary.  Usually, an automated
theorem prover cannot assist in this main goal, and by the time the main goal
is achieved, the mathematician may have already figured out the proof as a
by-product.  There is also a notational problem.  Mathematicians are used to
using very compact syntax where one or two symbols (heavily dependent on
context) can represent very complex concepts; this is part of the
hierarchy\index{hierarchy} they have built up to tackle difficult problems.  A
theorem prover on the other hand might require that a theorem be expressed in
``first-order logic,''\index{first-order logic} which is the logic on which
most of mathematics is ultimately based but which is not ordinarily used
directly because expressions can become very long.  Some automated theorem
provers are experimental programs, limited in their use to very specialized
areas, and the goal of many is simply research into the nature of automated
theorem proving itself.  Finally, much research remains to be done to enable
them to prove very deep theorems.  One significant result was a
computer proof by Larry Wos\index{Wos, Larry} and colleagues that every Robbins
algebra\index{Robbins algebra} is a Boolean  algebra\index{Boolean algebra}
({\em New York Times}, Dec. 10, 1996).\footnote{In 1933, E.~V.\
Huntington\index{Huntington, E. V.}
presented the following axiom system for
Boolean algebra with a unary operation $n$ and a binary operation $+$:
\begin{center}
    $x + y = y + x$ \\
    $(x + y) + z = x + (y + z)$ \\
    $n(n(x) + y) + n(n(x) + n(y)) = x$
\end{center}
Herbert Robbins\index{Robbins, Herbert}, a student of Huntington, conjectured
that the last equation can be replaced with a simpler one:
\begin{center}
    $n(n(x + y) + n(x + n(y))) = x$
\end{center}
Robbins and Huntington could not find a proof.  The problem was
later studied unsuccessfully by Tarski\index{Tarski, Alfred} and his
students, and it remained an unsolved problem until a
computer found the proof in 1996.  For more information on
the Robbins algebra problem see \cite{Wos}.}

How does Metamath\index{Metamath} relate to automated theorem provers?  A
theorem prover is primarily concerned with one theorem at a time (perhaps
tapping into a small database of known theorems) whereas Metamath is more like
a theorem archiving system, storing both the theorem and its proof in a
database for access and verification.  Metamath is one answer to ``what do you
do with the output of a theorem prover?''  and could be viewed as the
next step in the process.  Automated theorem provers could be useful tools for
helping develop its database.
Note that very long, automatically
generated proofs can make your database fat and ugly and cause Metamath's proof
verification to take a long time to run.  Unless you have a particularly good
program that generates very concise proofs, it might be best to consider the
use of automatically generated proofs as a quick-and-dirty approach, to be
manually rewritten at some later date.

The program {\sc otter}\index{otter@{\sc otter}}\footnote{\url{http://www.cs.unm.edu/\~mccune/otter/}.}, later succeeded by
prover9\index{prover9}\footnote{\url{https://www.cs.unm.edu/~mccune/mace4/}.},
have been historically influential.
The E prover\index{E prover}\footnote{\url{https://github.com/eprover/eprover}.}
is a maintained automated theorem prover
for full first-order logic with equality.
There are many other automated theorem provers as well.

If you want to combine automated theorem provers with Metamath
consider investigating
the book {\em Automated Reasoning:  Introduction and Applications}
\cite{Wos}\index{Wos, Larry}.  This book discusses
how to use {\sc otter} in a way that can
not only able to generate
relatively efficient proofs, it can even be instructed to search for
shorter proofs.  The effective use of {\sc otter} (and similar tools)
does require a certain
amount of experience, skill, and patience.  The axiom system used in the
\texttt{set.mm}\index{set theory database (\texttt{set.mm})} set theory
database can be expressed to {\sc otter} using a method described in
\cite{Megill}.\index{Megill, Norman}\footnote{To use those axioms with
{\sc otter}, they must be restated in a way that eliminates the need for
``dummy variables.''\index{dummy variable!eliminating} See the Comment
on p.~\pageref{nodd}.} When successful, this method tends to generate
short and clever proofs, but my experiments with it indicate that the
method will find proofs within a reasonable time only for relatively
easy theorems.  It is still fun to experiment with.

Reference \cite{Bledsoe}\index{Bledsoe, W. W.} surveys a number of approaches
people have explored in the field of automated theorem proving\index{automated
theorem proving}.

\subsection{Interactive Theorem Provers}\label{interactivetheoremprovers}

Finding proofs completely automatically is difficult, so there
are some interactive theorem provers that allow a human to guide the
computer to find a proof.
Examples include
HOL Light\index{HOL light}%
\footnote{\url{https://www.cl.cam.ac.uk/~jrh13/hol-light/}.},
Isabelle\index{Isabelle}%
\footnote{\url{http://www.cl.cam.ac.uk/Research/HVG/Isabelle}.},
{\sc hol}\index{hol@{\sc hol}}%
\footnote{\url{https://hol-theorem-prover.org/}.},
and
Coq\index{Coq}\footnote{\url{https://coq.inria.fr/}.}.

A major difference between most of these tools and Metamath is that the
``proofs'' are actually programs that guide the program to find a proof,
and not the proof itself.
For example, an Isabelle/HOL proof might apply a step
\texttt{apply (blast dest: rearrange reduction)}. The \texttt{blast}
instruction applies
an automatic tableux prover and returns if it found a sequence of proof
steps that work... but the sequence is not considered part of the proof.

A good overview of
higher-level proof verification languages (such as {\sc lcf}\index{lcf@{\sc
lcf}} and {\sc hol}\index{hol@{\sc hol}})
is given in \cite{Harrison}.  All of these languages are fundamentally
different from Metamath in that much of the mathematical foundational
knowledge is embedded in the underlying proof-verification program, rather
than placed directly in the database that is being verified.
These can have a steep learning curve for those without a mathematical
background.  For example, one usually must have a fair understanding of
mathematical logic in order to follow their proofs.

\subsection{Proof Verifiers}\label{proofverifiers}

A proof verifier is a program that doesn't generate proofs but instead
verifies proofs that you give it.  Many proof verifiers have limited built-in
automated proof capabilities, such as figuring out simple logical inferences
(while still being guided by a person who provides the overall proof).  Metamath
has no built-in automated proof capability other than the limited
capability of its Proof Assistant.

Proof-verification languages are not used as frequently as they might be.
Pure mathematicians are more concerned with producing new results, and such
detail and rigor would interfere with that goal.  The use of computers in pure
mathematics is primarily focused on automated theorem provers (not verifiers),
again with the ultimate goal of aiding the creation of new mathematics.
Automated theorem provers are usually concerned with attacking one theorem at
time rather than making a large, organized database easily available to the
user.  Metamath is one way to help close this gap.

By itself Metamath is a mostly a proof verifier.
This does not mean that other approaches can't be used; the difference
is that in Metamath, the results of various provers must be recorded
step-by-step so that they can be verified.

Another proof-verification language is Mizar,\index{Mizar} which can display
its proofs in the informal language that mathematicians are accustomed to.
Information on the Mizar language is available at \url{http://mizar.org}.

For the working mathematician, Mizar is an excellent tool for rigorously
documenting proofs. Mizar typesets its proofs in the informal English used by
mathematicians (and, while fine for them, are just as inscrutable by
laypersons!). A price paid for Mizar is a relatively steep learning curve of a
couple of weeks.  Several mathematicians are actively formalizing different
areas of mathematics using Mizar and publishing the proofs in a dedicated
journal. Unfortunately the task of formalizing mathematics is still looked
down upon to a certain extent since it doesn't involve the creation of ``new''
mathematics.

The closest system to Metamath is
the {\em Ghilbert}\index{Ghilbert} proof language (\url{http://ghilbert.org})
system developed by
Raph Levien\index{Levien, Raph}.
Ghilbert is a formal proof checker heavily inspired by Metamath.
Ghilbert statements are s-expressions (a la Lisp), which is easy
for computers to parse but many people find them hard to read.
There are a number of differences in their specific constructs, but
there is at least one tool to translate some Metamath materials into Ghilbert.
As of 2019 the Ghilbert community is smaller and less active than the
Metamath community.
That said, the Metamath and Ghilbert communities overlap, and fruitful
conversations between them have occurred many times over the years.

\subsection{Creating a Database of Formalized Mathematics}\label{mathdatabase}

Besides Metamath, there are several other ongoing projects with the goal of
formalizing mathematics into computer-verifiable databases.
Understanding some history will help.

The {\sc qed}\index{qed project@{\sc qed} project}%
\footnote{\url{http://www-unix.mcs.anl.gov/qed}.}
project arose in 1993 and its goals were outlined in the
{\sc qed} manifesto.
The {\sc qed} manifesto was
a proposal for a computer-based database of all mathematical knowledge,
strictly formalized and with all proofs having been checked automatically.
The project had a conference in 1994 and another in 1995;
there was also a ``twenty years of the {\sc qed} manifesto'' workshop
in 2014.
Its ideals are regularly reraised.

In a 2007 paper, Freek Wiedijk identified two reasons
for the failure of the {\sc qed} project as originally envisioned:%
\cite{Wiedijk-revisited}\index{Wiedijk, Freek}

\begin{itemize}
\item Very few people are working on formalization of mathematics. There is no compelling application for fully mechanized mathematics.
\item Formalized mathematics does not yet resemble traditional mathematics. This is partly due to the complexity of mathematical notation, and partly to the limitations of existing theorem provers and proof assistants.
\end{itemize}

But this did not end the dream of
formalizing mathematics into computer-verifiable databases.
The problems that led to the {\sc qed} manifesto are still with us,
even though the challenges were harder than originally considered.
What has happened instead is that various independent projects have
worked towards formalizing mathematics into computer-verifiable databases,
each simultaneously competing and cooperating with each other.

A concrete way to see this is
Freek Wiedijk's ``Formalizing 100 Theorems'' list%
\footnote{\url{http://www.cs.ru.nl/\%7Efreek/100/}.}
which shows the progress different systems have made on a challenge list
of 100 mathematical theorems.%
\footnote{ This is not the only list of ``interesting'' theorems.
Another interesting list was posted by Oliver Knill's list
\cite{Knill}\index{Knill, Oliver}.}
The top systems as of February 2019
(in order of the number of challenges completed) are
HOL Light, Isabelle, Metamath, Coq, and Mizar.

The Metamath 100%
\footnote{\url{http://us.metamath.org/mm\_100.html}}
page (maintained by David A. Wheeler\index{Wheeler, David A.})
shows the progress of Metamath (specifically its \texttt{set.mm} database)
against this challenge list maintained by Freek Wiedijk.
The Metamath \texttt{set.mm} database
has made a lot of progress over the years,
in part because working to prove those challenge theorems required
defining various terms and proving their properties as a prerequisite.
Here are just a few of the many statements that have been
formally proven with Metamath:

% The entries of this cause the narrow display to break poorly,
% since the short amount of text means LaTeX doesn't get a lot to work with
% and the itemize format gives it even *less* margin than usual.
% No one will mind if we make just this list flushleft, since this list
% will be internally consistent.
\begin{flushleft}
\begin{itemize}
\item 1. The Irrationality of the Square Root of 2
  (\texttt{sqr2irr}, by Norman Megill, 2001-08-20)
\item 2. The Fundamental Theorem of Algebra
  (\texttt{fta}, by Mario Carneiro, 2014-09-15)
\item 22. The Non-Denumerability of the Continuum
  (\texttt{ruc}, by Norman Megill, 2004-08-13)
\item 54. The Konigsberg Bridge Problem
  (\texttt{konigsberg}, by Mario Carneiro, 2015-04-16)
\item 83. The Friendship Theorem
  (\texttt{friendship}, by Alexander W. van der Vekens, 2018-10-09)
\end{itemize}
\end{flushleft}

We thank all of those who have developed at least one of the Metamath 100
proofs, and we particularly thank
Mario Carneiro\index{Carneiro, Mario}
who has contributed the most Metamath 100 proofs as of 2019.
The Metamath 100 page shows the list of all people who have contributed a
proof, and links to graphs and charts showing progress over time.
We encourage others to work on proving theorems not yet proven in Metamath,
since doing so improves the work as a whole.

Each of the math formalization systems (including Metamath)
has different strengths and weaknesses, depending on what you value.
Key aspects that differentiate Metamath from the other top systems are:

\begin{itemize}
\item Metamath is not tied to any particular set of axioms.
\item Metamath can show every step of every proof, no exceptions.
  Most other provers only assert that a proof can be found, and do not
  show every step. This also makes verification fast, because
  the system does not need to rediscover proof details.
\item The Metamath verifier has been re-implemented in many different
  programming languages, so verification can be done by multiple
  implementations.  In particular, the
  \texttt{set.mm}\index{set theory database (\texttt{set.mm})}%
  \index{Metamath Proof Explorer} database is verified by
  four different verifiers
  written in four different languages by four different authors.
  This greatly reduces the risk of accepting an invalid
  proof due to an error in the verifier.
\item Proofs stay proven.  In some systems, changes to the system's
  syntax or how a tactic works causes proofs to fail in later versions,
  causing older work to become essentially lost.
  Metamath's language is
  extremely small and fixed, so once a proof is added to a database,
  the database can be rechecked with later versions of the Metamath program
  and with other verifiers of Metamath databases.
  If an axiom or key definition needs to be changed, it is easy to
  manipulate the database as a whole to handle the change
  without touching the underlying verifier.
  Since re-verification of an entire database takes seconds, there
  is never a reason to delay complete verification.
  This aspect is especially compelling if your
  goal is to have a long-term database of proofs.
\item Licensing is generous.  The main Metamath databases are released to
  the public domain, and the main Metamath program is open source software
  under a standard, widely-used license.
\item Substitutions are easy to understand, even by those who are not
  professional mathematicians.
\end{itemize}

Of course, other systems may have advantages over Metamath
that are more compelling, depending on what you value.
In any case, we hope this helps you understand Metamath
within a wider context.

\subsection{In Summary}\label{computers-summary}

To summarize our discussions of computers and mathematics, computer algebra
systems can be viewed as theorem generators focusing on a narrow realm of
mathematics (numbers and their properties), automated theorem provers as proof
generators for specific theorems in a much broader realm covered by a built-in
formal system such as first-order logic, interactive theorem
provers require human guidance, proof verifiers verify proofs but
historically they have been
restricted to first-order logic.
Metamath, in contrast,
is a proof verifier and documenter whose realm is essentially unlimited.

\section{Mathematics and Metamath}

\subsection{Standard Mathematics}

There are a number of ways that Metamath\index{Metamath} can be used with
standard mathematics.  The most satisfying way philosophically is to start at
the very beginning, and develop the desired mathematics from the axioms of
logic and set theory.\index{set theory}  This is the approach taken in the
\texttt{set.mm}\index{set theory database (\texttt{set.mm})}%
\index{Metamath Proof Explorer}
database (also known as the Metamath Proof Explorer).
Among other things, this database builds up to the
axioms of real and complex numbers\index{analysis}\index{real and complex
numbers} (see Section~\ref{real}), and a standard development of analysis, for
example, could start at that point, using it as a basis.   Besides this
philosophical advantage, there are practical advantages to having all of the
tools of set theory available in the supporting infrastructure.

On the other hand, you may wish to start with the standard axioms of a
mathematical theory without going through the set theoretical proofs of those
axioms.  You will need mathematical logic to make inferences, but if you wish
you can simply introduce theorems\index{theorem} of logic as
``axioms''\index{axiom} wherever you need them, with the implicit assumption
that in principle they can be proved, if they are obvious to you.  If you
choose this approach, you will probably want to review the notation used in
\texttt{set.mm}\index{set theory database (\texttt{set.mm})} so that your own
notation will be consistent with it.

\subsection{Other Formal Systems}
\index{formal system}

Unlike some programs, Metamath\index{Metamath} is not limited to any specific
area of mathematics, nor committed to any particular mathematical philosophy
such as classical logic versus intuitionism, nor limited, say, to expressions
in first-order logic.  Although the database \texttt{set.mm}
describes standard logic and set theory, Meta\-math
is actually a general-purpose language for describing a wide variety of formal
systems.\index{formal system}  Non-standard systems such as modal
logic,\index{modal logic} intuitionist logic\index{intuitionism}, higher-order
logic\index{higher-order logic}, quantum logic\index{quantum logic}, and
category theory\index{category theory} can all be described with the Metamath
language.  You define the symbols you prefer and tell Metamath the axioms and
rules you want to start from, and Metamath will verify any inferences you make
from those axioms and rules.  A simple example of a non-standard formal system
is Hofstadter's\index{Hofstadter, Douglas R.} MIU system,\index{MIU-system}
whose Metamath description is presented in Appendix~\ref{MIU}.

This is not hypothetical.
The largest Metamath database is
\texttt{set.mm}\index{set theory database (\texttt{set.mm}}%
\index{Metamath Proof Explorer}), aka the Metamath Proof Explorer,
which uses the most common axioms for mathematical foundations
(specifically classical logic combined with Zermelo--Fraenkel
set theory\index{Zermelo--Fraenkel set theory} with the Axiom of Choice).
But other Metamath databases are available:

\begin{itemize}
\item The database
  \texttt{iset.mm}\index{intuitionistic logic database (\texttt{iset.mm})},
  aka the
  Intuitionistic Logic Explorer\index{Intuitionistic Logic Explorer},
  uses intuitionistic logic (a constructivist point of view)
  instead of classical logic.
\item The database
  \texttt{nf.mm}\index{New Foundations database (\texttt{nf.mm})},
  aka the
  New Foundations Explorer\index{New Foundations Explorer},
  constructs mathematics from scratch,
  starting from Quine's New Foundations (NF) set theory axioms.
\item The database
  \texttt{hol.mm}\index{Higher-order Logic database (\texttt{hol.mm})},
  aka the
  Higher-Order Logic (HOL) Explorer\index{Higher-Order Logic (HOL) Explorer},
  starts with HOL (also called simple type theory) and derives
  equivalents to ZFC axioms, connecting the two approaches.
\end{itemize}

Since the days of David Hilbert,\index{Hilbert, David} mathematicians have
been concerned with the fact that the metalanguage\index{metalanguage} used to
describe mathematics may be stronger than the mathematics being described.
Metamath\index{Metamath}'s underlying finitary\index{finitary proof},
constructive nature provides a good philosophical basis for studying even the
weakest logics.\index{weak logic}

The usual treatment of many non-standard formal systems\index{formal
system} uses model theory\index{model theory} or proof theory\index{proof
theory} to describe these systems; these theories, in turn, are based on
standard set theory.  In other words, a non-standard formal system is defined
as a set with certain properties, and standard set theory is used to derive
additional properties of this set.  The standard set theory database provided
with Metamath can be used for this purpose, and when used this way
the development of a special
axiom system for the non-standard formal system becomes unnecessary.  The
model- or proof-theoretic approach often allows you to prove much deeper
results with less effort.

Metamath supports both approaches.  You can define the non-standard
formal system directly, or define the non-standard formal system as
a set with certain properties, whichever you find most helpful.

%\section{Additional Remarks}

\subsection{Metamath and Its Philosophy}

Closely related to Metamath\index{Metamath} is a philosophy or way of looking
at mathematics. This philosophy is related to the formalist
philosophy\index{formalism} of Hilbert\index{Hilbert, David} and his followers
\cite[pp.~1203--1208]{Kline}\index{Kline, Morris}
\cite[p.~6]{Behnke}\index{Behnke, H.}. In this philosophy, mathematics is
viewed as nothing more than a set of rules that manipulate symbols, together
with the consequences of those rules.  While the mathematics being described
may be complex, the rules used to describe it (the
``metamathematics''\index{metamathematics}) should be as simple as possible.
In particular, proofs should be restricted to dealing with concrete objects
(the symbols we write on paper rather than the abstract concepts they
represent) in a constructive manner; these are called ``finitary''
proofs\index{finitary proof} \cite[pp.~2--3]{Shoenfield}\index{Shoenfield,
Joseph R.}.

Whether or not you find Metamath interesting or useful will in part depend on
the appeal you find in its philosophy, and this appeal will probably depend on
your particular goals with respect to mathematics.  For example, if you are a
pure mathematician at the forefront of discovering new mathematical knowledge,
you will probably find that the rigid formality of Metamath stifles your
creativity.  On the other hand, we would argue that once this knowledge is
discovered, there are advantages to documenting it in a standard format that
will make it accessible to others.  Sixty years from now, your field may be
dormant, and as Davis and Hersh put it, your ``writings would become less
translatable than those of the Maya'' \cite[p.~37]{Davis}\index{Davis, Phillip
J.}.


\subsection{A History of the Approach Behind Metamath}

Probably the one work that has had the most motivating influence on
Metamath\index{Metamath} is Whitehead and Russell's monumental {\em Principia
Mathematica} \cite{PM}\index{Whitehead, Alfred North}\index{Russell,
Bertrand}\index{principia mathematica@{\em Principia Mathematica}}, whose aim
was to deduce all of mathematics from a small number of primitive ideas, in a
very explicit way that in principle anyone could understand and follow.  While
this work was tremendously influential in its time, from a modern perspective
it suffers from several drawbacks.  Both its notation and its underlying
axioms are now considered dated and are no longer used.  From our point of
view, its development is not really as accessible as we would like to see; for
practical reasons, proofs become more and more sketchy as its mathematics
progresses, and working them out in fine detail requires a degree of
mathematical skill and patience that many people don't have.  There are
numerous small errors, which is understandable given the tedious, technical
nature of the proofs and the lack of a computer to verify the details.
However, even today {\em Principia Mathematica} stands out as the work closest
in spirit to Metamath.  It remains a mind-boggling work, and one can't help
but be amazed at seeing ``$1+1=2$'' finally appear on page 83 of Volume II
(Theorem *110.643).

The origin of the proof notation used by Metamath dates back to the 1950's,
when the logician C.~A.~Meredith expressed his proofs in a compact notation
called ``condensed detachment''\index{condensed detachment}
\cite{Hindley}\index{Hindley, J. Roger} \cite{Kalman}\index{Kalman, J. A.}
\cite{Meredith}\index{Meredith, C. A.} \cite{Peterson}\index{Peterson, Jeremy
George}.  This notation allows proofs to be communicated unambiguously by
merely referencing the axiom\index{axiom}, rule\index{rule}, or
theorem\index{theorem} used at each step, without explicitly indicating the
substitutions\index{substitution!variable}\index{variable substitution} that
have to be made to the variables in that axiom, rule, or theorem.  Ordinarily,
condensed detachment is more or less limited to propositional
calculus\index{propositional calculus}.  The concept has been extended to
first-order logic\index{first-order logic} in \cite{Megill}\index{Megill,
Norman}, making it is easy to write a small computer program to verify proofs
of simple first-order logic theorems.\index{condensed detachment!and
first-order logic}

A key concept behind the notation of condensed detachment is called
``unification,''\index{unification} which is an algorithm for determining what
substitutions\index{substitution!variable}\index{variable substitution} to
variables have to be made to make two expressions match each other.
Unification was first precisely defined by the logician J.~A.~Robinson, who
used it in the development of a powerful
theorem-proving technique called the ``resolution principle''
\cite{Robinson}\index{Robinson's resolution principle}. Metamath does not make
use of the resolution principle, which is intended for systems of first-order
logic.\index{first-order logic}  Metamath's use is not restricted to
first-order logic, and as we have mentioned it does not automatically discover
proofs.  However, unification is a key idea behind Metamath's proof
notation, and Metamath makes use of a very simple version of it
(Section~\ref{unify}).

\subsection{Metamath and First-Order Logic}

First-order logic\index{first-order logic} is the supporting structure
for standard mathematics.  On top of it is set theory, which contains
the axioms from which virtually all of mathematics can be derived---a
remarkable fact.\index{category
theory}\index{cardinal, inaccessible}\label{categoryth}\footnote{An exception seems
to be category theory.  There are several schools of thought on whether
category theory is derivable from set theory.  At a minimum, it appears
that an additional axiom is needed that asserts the existence of an
``inaccessible cardinal'' (a type of infinity so large that standard set
theory can't prove or deny that it exists).
%
%%%% (I took this out that was in previous editions:)
% But it is also argued that not just one but a ``proper class'' of them
% is needed, and the existence of proper classes is impossible in standard
% set theory.  (A proper class is a collection of sets so huge that no set
% can contain it as an element.  Proper classes can lead to
% inconsistencies such as ``Russell's paradox.''  The axioms of standard
% set theory are devised so as to deny the existence of proper classes.)
%
For more information, see
\cite[pp.~328--331]{Herrlich}\index{Herrlich, Horst} and
\cite{Blass}\index{Blass, Andrea}.}

One of the things that makes Metamath\index{Metamath} more practical for
first-order theories is a set of axioms for first-order logic designed
specifically with Metamath's approach in mind.  These are included in
the database \texttt{set.mm}\index{set theory database (\texttt{set.mm})}.
See Chapter~\ref{fol} for a detailed
description; the axioms are shown in Section~\ref{metaaxioms}.  While
logically equivalent to standard axiom systems, our axiom system breaks
up the standard axioms into smaller pieces such that from them, you can
directly derive what in other systems can only be derived as higher-level
``metatheorems.''\index{metatheorem}  In other words, it is more powerful than
the standard axioms from a metalogical point of view.  A rigorous
justification for this system and its ``metalogical
completeness''\index{metalogical completeness} is found in
\cite{Megill}\index{Megill, Norman}.  The system is closely related to a
system developed by Monk\index{Monk, J. Donald} and Tarski\index{Tarski,
Alfred} in 1965 \cite{Monks}.

For example, the formula $\exists x \, x = y $ (given $y$, there exists some
$x$ equal to it) is a theorem of logic,\footnote{Specifically, it is a theorem
of those systems of logic that assume non-empty domains.  It is not a theorem
of more general systems that include the empty domain\index{empty domain}, in
which nothing exists, period!  Such systems are called ``free
logics.''\index{free logic} For a discussion of these systems, see
\cite{Leblanc}\index{Leblanc, Hugues}.  Since our use for logic is as a basis
for set theory, which has a non-empty domain, it is more convenient (and more
traditional) to use a less general system.  An interesting curiosity is that,
using a free logic as a basis for Zermelo--Fraenkel set
theory\index{Zermelo--Fraenkel set theory} (with the redundant Axiom of the
Null Set omitted),\index{Axiom of the Null Set} we cannot even prove the
existence of a single set without assuming the axiom of infinity!\index{Axiom
of Infinity}} whether or not $x$ and $y$ are distinct variables\index{distinct
variables}.  In many systems of logic, we would have to prove two theorems to
arrive at this result.  First we would prove ``$\exists x \, x = x $,'' then
we would separately prove ``$\exists x \, x = y $, where $x$ and $y$ are
distinct variables.''  We would then combine these two special cases ``outside
of the system'' (i.e.\ in our heads) to be able to claim, ``$\exists x \, x =
y $, regardless of whether $x$ and $y$ are distinct.''  In other words, the
combination of the two special cases is a
metatheorem.  In the system of logic
used in Metamath's set theory\index{set theory database (\texttt{set.mm})}
database, the axioms of logic are broken down into small pieces that allow
them to be reassembled in such a way that theorems such as these can be proved
directly.

Breaking down the axioms in this way makes them look peculiar and not very
intuitive at first, but rest assured that they are correct and complete.  Their
correctness is ensured because they are theorem schemes of standard first-order
logic (which you can easily verify if you are a logician).  Their completeness
follows from the fact that we explicitly derive the standard axioms of
first-order logic as theorems.  Deriving the standard axioms is somewhat
tricky, but once we're there, we have at our disposal a system that is less
awkward to work with in formal proofs\index{formal proof}.  In technical terms
that logicians understand, we eliminate the cumbersome concepts of ``free
variable,''\index{free variable} ``bound variable,''\index{bound variable} and
``proper substitution''\index{proper substitution}\index{substitution!proper}
as primitive notions.  These concepts are present in our system but are
defined in terms of concepts expressed by the axioms and can be eliminated in
principle.  In standard systems, these concepts are really like additional,
implicit axioms\index{implicit axiom} that are somewhat complex and cannot be
eliminated.

The traditional approach to logic, wherein free variables and proper
substitution is defined, is also possible to do directly in the Metamath
language.  However, the notation tends to become awkward, and there are
disadvantages:  for example, extending the definition of a wff with a
definition is awkward, because the free variable and proper substitution
concepts have to have their definitions also extended.  Our choice of
axioms for \texttt{set.mm} is to a certain extent a matter of style, in
an attempt to achieve overall simplicity, but you should also be aware
that the traditional approach is possible as well if you should choose
it.

\chapter{Using the Metamath Program}
\label{using}

\section{Installation}

The way that you install Metamath\index{Metamath!installation} on your
computer system will vary for different computers.  Current
instructions are provided with the Metamath program download at
\url{http://metamath.org}.  In general, the installation is simple.
There is one file containing the Metamath program itself.  This file is
usually called \texttt{metamath} or \texttt{metamath.}{\em xxx} where
{\em xxx} is the convention (such as \texttt{exe}) for an executable
program on your operating system.  There are several additional files
containing samples of the Metamath language, all ending with
\texttt{.mm}.  The file \texttt{set.mm}\index{set theory database
(\texttt{set.mm})} contains logic and set theory and can be used as a
starting point for other areas of mathematics.

You will also need a text editor\index{text editor} capable of editing plain
{\sc ascii}\footnote{American Standard Code for Information Interchange.} text
in order to prepare your input files.\index{ascii@{\sc ascii}}  Most computers
have this capability built in.  Note that plain text is not necessarily the
default for some word processing programs\index{word processor}, especially if
they can handle different fonts; for example, with Microsoft Word\index{Word
(Microsoft)}, you must save the file in the format ``Text Only With Line
Breaks'' to get a plain text\index{plain text} file.\footnote{It is
recommended that all lines in a Metamath source file be 79 characters or less
in length for compatibility among different computer terminals.  When creating
a source file on an editor such as Word, select a monospaced
font\index{monospaced font} such as Courier\index{Courier font} or
Monaco\index{Monaco font} to make this easier to achieve.  Better yet,
just use a plain text editor such as Notepad.}

On some computer systems, Metamath does not have the capability to print
its output directly; instead, you send its output to a file (using the
\texttt{open} commands described later).  The way you print this output
file depends on your computer.\index{printers} Some computers have a
print command, whereas with others, you may have to read the file into
an editor and print it from there.

If you want to print your Metamath source files with typeset formulas
containing standard mathematical symbols, you will need the \LaTeX\
typesetting program\index{latex@{\LaTeX}}, which is widely and freely
available for most operating systems.  It runs natively on Unix and
Linux, and can be installed on Windows as part of the free Cygwin
package (\url{http://cygwin.com}).

You can also produce {\sc html}\footnote{HyperText Markup Language.}
web pages.  The {\tt help html} command in the Metamath program will
assist you with this feature.

\section{Your First Formal System}\label{start}
\subsection{From Nothing to Zero}\label{startf}

To give you a feel for what the Metamath\index{Metamath} language looks like,
we will take a look at a very simple example from formal number
theory\index{number theory}.  This example is taken from
Mendelson\index{Mendelson, Elliot} \cite[p. 123]{Mendelson}.\footnote{To keep
the example simple, we have changed the formalism slightly, and what we call
axioms\index{axiom} are strictly speaking theorems\index{theorem} in
\cite{Mendelson}.}  We will look at a small subset of this theory, namely that
part needed for the first number theory theorem proved in \cite{Mendelson}.

First we will look at a standard formal proof\index{formal proof} for the
example we have picked, then we will look at the Metamath version.  If you
have never been exposed to formal proofs, the notation may seem to be such
overkill to express such simple notions that you may wonder if you are missing
something.  You aren't.  The concepts involved are in fact very simple, and a
detailed breakdown in this fashion is necessary to express the proof in a way
that can be verified mechanically.  And as you will see, Metamath breaks the
proof down into even finer pieces so that the mechanical verification process
can be about as simple as possible.

Before we can introduce the axioms\index{axiom} of the theory, we must define
the syntax rules for forming legal expressions\index{syntax rules}
(combinations of symbols) with which those axioms can be used. The number 0 is
a {\bf term}\index{term}; and if $ t$ and $r$ are terms, so is $(t+r)$. Here,
$ t$ and $r$ are ``metavariables''\index{metavariable} ranging over terms; they
themselves do not appear as symbols in an actual term.  Some examples of
actual terms are $(0 + 0)$ and $((0+0)+0)$.  (Note that our theory describes
only the number zero and sums of zeroes.  Of course, not much can be done with
such a trivial theory, but remember that we have picked a very small subset of
complete number theory for our example.  The important thing for you to focus
on is our definitions that describe how symbols are combined to form valid
expressions, and not on the content or meaning of those expressions.) If $ t$
and $r$ are terms, an expression of the form $ t=r$ is a {\bf wff}
(well-formed formula)\index{well-formed formula (wff)}; and if $P$ and $Q$ are
wffs, so is $(P\rightarrow Q)$ (which means ``$P$ implies
$Q$''\index{implication ($\rightarrow$)} or ``if $P$ then $Q$'').
Here $P$ and $Q$ are metavariables ranging over wffs.  Examples of actual
wffs are $0=0$, $(0+0)=0$, $(0=0 \rightarrow (0+0)=0)$, and $(0=0\rightarrow
(0=0\rightarrow 0=(0+0)))$.  (Our notation makes use of more parentheses than
are customary, but the elimination of ambiguity this way simplifies our
example by avoiding the need to define operator precedence\index{operator
precedence}.)

The {\bf axioms}\index{axiom} of our theory are all wffs of the following
form, where $ t$, $r$, and $s$ are any terms:

%Latex p. 92
\renewcommand{\theequation}{A\arabic{equation}}

\begin{equation}
(t=r\rightarrow (t=s\rightarrow r=s))
\end{equation}
\begin{equation}
(t+0)=t
\end{equation}

Note that there are an infinite number of axioms since there are an infinite
number of possible terms.  A1 and A2 are properly called ``axiom
schemes,''\index{axiom scheme} but we will refer to them as ``axioms'' for
brevity.

An axiom is a {\bf theorem}; and if $P$ and $(P\rightarrow Q)$ are theorems
(where $P$ and $Q$ are wffs), then $Q$ is also a theorem.\index{theorem}  The
second part of this definition is called the modus ponens (MP) rule of
inference\index{inference rule}\index{modus ponens}.  It allows us to obtain
new theorems from old ones.

The {\bf proof}\index{proof} of a theorem is a sequence of one or more
theorems, each of which is either an axiom or the result of modus ponens
applied to two previous theorems in the sequence, and the last of which is the
theorem being proved.

The theorem we will prove for our example is very simple:  $ t=t$.  The proof of
our theorem follows.  Study it carefully until you feel sure you
understand it.\label{zeroproof}

% Use tabu so that lines will wrap automatically as needed.
\begin{tabu} { l X X }
1. & $(t+0)=t$ & (by axiom A2) \\
2. & $(t+0)=t$ & (by axiom A2) \\
3. & $((t+0)=t \rightarrow ((t+0)=t\rightarrow t=t))$ & (by axiom A1) \\
4. & $((t+0)=t\rightarrow t=t)$ & (by MP applied to steps 2 and 3) \\
5. & $t=t$ & (by MP applied to steps 1 and 4) \\
\end{tabu}

(You may wonder why step 1 is repeated twice.  This is not necessary in the
formal language we have defined, but in Metamath's ``reverse Polish
notation''\index{reverse Polish notation (RPN)} for proofs, a previous step
can be referred to only once.  The repetition of step~1 here will enable you
to see more clearly the correspondence of this proof with the
Metamath\index{Metamath} version on p.~\pageref{demoproof}.)

Our theorem is more properly called a ``theorem scheme,''\index{theorem
scheme} for it represents an infinite number of theorems, one for each
possible term $ t$.  Two examples of actual theorems would be $0=0$ and
$(0+0)=(0+0)$.  Rarely do we prove actual theorems, since by proving schemes
we can prove an infinite number of theorems in one fell swoop.  Similarly, our
proof should really be called a ``proof scheme.''\index{proof scheme}  To
obtain an actual proof, pick an actual term to use in place of $ t$, and
substitute it for $ t$ throughout the proof.

Let's discuss what we have done here.  The axioms\index{axiom} of our theory,
A1 and A2, are trivial and obvious.  Everyone knows that adding zero to
something doesn't change it, and also that if two things are equal to a third,
then they are equal to each other. In fact, stating the trivial and obvious is
a goal to strive for in any axiomatic system.  From trivial and obvious truths
that everyone agrees upon, we can prove results that are not so obvious yet
have absolute faith in them.  If we trust the axioms and the rules, we must,
by definition, trust the consequences of those axioms and rules, if logic is
to mean anything at all.

Our rule of inference\index{rule}, modus ponens\index{modus ponens}, is also
pretty obvious once you understand what it means.  If we prove a fact $P$, and
we also prove that $P$ implies $Q$, then $Q$ necessarily follows as a new
fact.  The rule provides us with a means for obtaining new facts (i.e.\
theorems\index{theorem}) from old ones.

The theorem that we have proved, $ t=t$, is so fundamental that you may wonder
why it isn't one of the axioms\index{axiom}.  In some axiom systems of
arithmetic, it {\em is} an axiom.  The choice of axioms in a theory is to some
extent arbitrary and even an art form, constrained only by the requirement
that any two equivalent axiom systems be able to derive each other as
theorems.  We could imagine that the inventor of our axiom system originally
included $ t=t$ as an axiom, then discovered that it could be derived as a
theorem from the other axioms.  Because of this, it was not necessary to
keep it as an axiom.  By eliminating it, the final set of axioms became
that much simpler.

Unless you have worked with formal proofs\index{formal proof} before, it
probably wasn't apparent to you that $ t=t$ could be derived from our two
axioms until you saw the proof. While you certainly believe that $ t=t$ is
true, you might not be able to convince an imaginary skeptic who believes only
in our two axioms until you produce the proof.  Formal proofs such as this are
hard to come up with when you first start working with them, but after you get
used to them they can become interesting and fun.  Once you understand the
idea behind formal proofs you will have grasped the fundamental principle that
underlies all of mathematics.  As the mathematics becomes more sophisticated,
its proofs become more challenging, but ultimately they all can be broken down
into individual steps as simple as the ones in our proof above.

Mendelson's\index{Mendelson, Elliot} book, from which our example was taken,
contains a number of detailed formal proofs such as these, and you may be
interested in looking it up.  The book is intended for mathematicians,
however, and most of it is rather advanced.  Popular literature describing
formal proofs\index{formal proof} include \cite[p.~296]{Rucker}\index{Rucker,
Rudy} and \cite[pp.~204--230]{Hofstadter}\index{Hofstadter, Douglas R.}.

\subsection{Converting It to Metamath}\label{convert}

Formal proofs\index{formal proof} such as the one in our example break down
logical reasoning into small, precise steps that leave little doubt that the
results follow from the axioms\index{axiom}.  You might think that this would
be the finest breakdown we can achieve in mathematics.  However, there is more
to the proof than meets the eye. Although our axioms were rather simple, a lot
of verbiage was needed before we could even state them:  we needed to define
``term,'' ``wff,'' and so on.  In addition, there are a number of implied
rules that we haven't even mentioned. For example, how do we know that step 3
of our proof follows from axiom A1? There is some hidden reasoning involved in
determining this.  Axiom A1 has two occurrences of the letter $ t$.  One of
the implied rules states that whatever we substitute for $ t$ must be a legal
term\index{term}.\footnote{Some authors make this implied rule explicit by
stating, ``only expressions of the above form are terms,'' after defining
``term.''}  The expression $ t+0$ is pretty obviously a legal term whenever $
t$ is, but suppose we wanted to substitute a huge term with thousands of
symbols?  Certainly a lot of work would be involved in determining that it
really is a term, but in ordinary formal proofs all of this work would be
considered a single ``step.''

To express our axiom system in the Metamath\index{Metamath} language, we must
describe this auxiliary information in addition to the axioms themselves.
Metamath does not know what a ``term'' or a ``wff''\index{well-formed formula
(wff)} is.  In Metamath, the specification of the ways in which we can combine
symbols to obtain terms and wffs are like little axioms in themselves.  These
auxiliary axioms are expressed in the same notation as the ``real''
axioms\index{axiom}, and Metamath does not distinguish between the two.  The
distinction is made by you, i.e.\ by the way in which you interpret the
notation you have chosen to express these two kinds of axioms.

The Metamath language breaks down mathematical proofs into tiny pieces, much
more so than in ordinary formal proofs\index{formal proof}.  If a single
step\index{proof step} involves the
substitution\index{substitution!variable}\index{variable substitution} of a
complex term for one of its variables, Metamath must see this single step
broken down into many small steps.  This fine-grained breakdown is what gives
Metamath generality and flexibility as it lets it not be limited to any
particular mathematical notation.

Metamath's proof notation is not, in itself, intended to be read by humans but
rather is in a compact format intended for a machine.  The Metamath program
will convert this notation to a form you can understand, using the \texttt{show
proof}\index{\texttt{show proof} command} command.  You can tell the program what
level of detail of the proof you want to look at.  You may want to look at
just the logical inference steps that correspond
to ordinary formal proof steps,
or you may want to see the fine-grained steps that prove that an expression is
a term.

Here, without further ado, is our example converted to the
Metamath\index{Metamath} language:\index{metavariable}\label{demo0}

\begin{verbatim}
$( Declare the constant symbols we will use $)
    $c 0 + = -> ( ) term wff |- $.
$( Declare the metavariables we will use $)
    $v t r s P Q $.
$( Specify properties of the metavariables $)
    tt $f term t $.
    tr $f term r $.
    ts $f term s $.
    wp $f wff P $.
    wq $f wff Q $.
$( Define "term" and "wff" $)
    tze $a term 0 $.
    tpl $a term ( t + r ) $.
    weq $a wff t = r $.
    wim $a wff ( P -> Q ) $.
$( State the axioms $)
    a1 $a |- ( t = r -> ( t = s -> r = s ) ) $.
    a2 $a |- ( t + 0 ) = t $.
$( Define the modus ponens inference rule $)
    ${
       min $e |- P $.
       maj $e |- ( P -> Q ) $.
       mp  $a |- Q $.
    $}
$( Prove a theorem $)
    th1 $p |- t = t $=
  $( Here is its proof: $)
       tt tze tpl tt weq tt tt weq tt a2 tt tze tpl
       tt weq tt tze tpl tt weq tt tt weq wim tt a2
       tt tze tpl tt tt a1 mp mp
     $.
\end{verbatim}\index{metavariable}

A ``database''\index{database} is a set of one or more {\sc ascii} source
files.  Here's a brief description of this Metamath\index{Metamath} database
(which consists of this single source file), so that you can understand in
general terms what is going on.  To understand the source file in detail, you
should read Chapter~\ref{languagespec}.

The database is a sequence of ``tokens,''\index{token} which are normally
separated by spaces or line breaks.  The only tokens that are built into
the Metamath language are those beginning with \texttt{\$}.  These tokens
are called ``keywords.''\index{keyword}  All other tokens are
user-defined, and their names are arbitrary.

As you might have guessed, the Metamath token \texttt{\$(}\index{\texttt{\$(} and
\texttt{\$)} auxiliary keywords} starts a comment and \texttt{\$)} ends a comment.

The Metamath tokens \texttt{\$c}\index{\texttt{\$c} statement},
\texttt{\$v}\index{\texttt{\$v} statement},
\texttt{\$e}\index{\texttt{\$e} statement},
\texttt{\$f}\index{\texttt{\$f} statement},
\texttt{\$a}\index{\texttt{\$a} statement}, and
\texttt{\$p}\index{\texttt{\$p} statement} specify ``statements'' that
end with \texttt{\$.}\,.\index{\texttt{\$.}\ keyword}

The Metamath tokens \texttt{\$c} and \texttt{\$v} each declare\index{constant
declaration}\index{variable declaration} a list of user-defined tokens, called
``math symbols,''\index{math symbol} that the database will reference later
on.  All of the math symbols they define you have seen earlier except the
turnstile symbol \texttt{|-} ($\vdash$)\index{turnstile ({$\,\vdash$})}, which is
commonly used by logicians to mean ``a proof exists for.''  For us
the turnstile is just a
convenient symbol that distinguishes expressions that are axioms\index{axiom}
or theorems\index{theorem} from expressions that are terms or wffs.

The \texttt{\$c} statement declares ``constants''\index{constant} and
the \texttt{\$v} statement declares
``variables''\index{variable}\index{constant declaration}\index{variable
declaration} (or more precisely, metavariables\index{metavariable}).  A
variable may be substituted\index{substitution!variable}\index{variable
substitution} with sequences of math symbols whereas a constant may not
be substituted with anything.

It may seem redundant to require both \texttt{\$c}\index{\texttt{\$c} statement} and
\texttt{\$v}\index{\texttt{\$v} statement} statements (since any math
symbol\index{math symbol} not specified with a \texttt{\$c} statement could be
presumed to be a variable), but this provides for better error checking and
also allows math symbols to be redeclared\index{redeclaration of symbols}
(Section~\ref{scoping}).

The token \texttt{\$f}\index{\texttt{\$f} statement} specifies a
statement called a ``variable-type hypothesis'' (also called a
``floating hypothesis'') and \texttt{\$e}\index{\texttt{\$e} statement}
specifies a ``logical hypothesis'' (also called an ``essential
hypothesis'').\index{hypothesis}\index{variable-type
hypothesis}\index{logical hypothesis}\index{floating
hypothesis}\index{essential hypothesis} The token
\texttt{\$a}\index{\texttt{\$a} statement} specifies an ``axiomatic
assertion,''\index{axiomatic assertion} and
\texttt{\$p}\index{\texttt{\$p} statement} specifies a ``provable
assertion.''\index{provable assertion} To the left of each occurrence of
these four tokens is a ``label''\index{label} that identifies the
hypothesis or assertion for later reference.  For example, the label of
the first axiomatic assertion is \texttt{tze}.  A \texttt{\$f} statement
must contain exactly two math symbols, a constant followed by a
variable.  The \texttt{\$e}, \texttt{\$a}, and \texttt{\$p} statements
each start with a constant followed by, in general, an arbitrary
sequence of math symbols.

Associated with each assertion\index{assertion} is a set of hypotheses
that must be satisfied in order for the assertion to be used in a proof.
These are called the ``mandatory hypotheses''\index{mandatory
hypothesis} of the assertion.  Among those hypotheses whose ``scope''
(described below) includes the assertion, \texttt{\$e} hypotheses are
always mandatory and \texttt{\$f}\index{\texttt{\$f} statement}
hypotheses are mandatory when they share their variable with the
assertion or its \texttt{\$e} hypotheses.  The exact rules for
determining which hypotheses are mandatory are described in detail in
Sections~\ref{frames} and \ref{scoping}.  For example, the mandatory
hypotheses of assertion \texttt{tpl} are \texttt{tt} and \texttt{tr},
whereas assertion \texttt{tze} has no mandatory hypotheses because it
contains no variables and has no \texttt{\$e}\index{\texttt{\$e}
statement} hypothesis.  Metamath's \texttt{show statement}
command\index{\texttt{show statement} command}, described in the next
section, will show you a statement's mandatory hypotheses.

Sometimes we need to make a hypothesis relevant to only certain
assertions.  The set of statements to which a hypothesis is relevant is
called its ``scope.''  The Metamath brackets,
\texttt{\$\char`\{}\index{\texttt{\$\char`\{} and \texttt{\$\char`\}}
keywords} and \texttt{\$\char`\}}, define a ``block''\index{block} that
delimits the scope of any hypothesis contained between them.  The
assertion \texttt{mp} has mandatory hypotheses \texttt{wp}, \texttt{wq},
\texttt{min}, and \texttt{maj}.  The only mandatory hypothesis of
\texttt{th1}, on the other hand, is \texttt{tt}, since \texttt{th1}
occurs outside of the block containing \texttt{min} and \texttt{maj}.

Note that \texttt{\$\char`\{} and \texttt{\$\char`\}} do not affect the
scope of assertions (\texttt{\$a} and \texttt{\$p}).  Assertions are always
available to be referenced by any later proof in the source file.

Each provable assertion (\texttt{\$p}\index{\texttt{\$p} statement}
statement) has two parts.  The first part is the
assertion\index{assertion} itself, which is a sequence of math
symbol\index{math symbol} tokens placed between the \texttt{\$p} token
and a \texttt{\$=}\index{\texttt{\$=} keyword} token.  The second part
is a ``proof,'' which is a list of label tokens placed between the
\texttt{\$=} token and the \texttt{\$.}\index{\texttt{\$.}\ keyword}\
token that ends the statement.\footnote{If you've looked at the
\texttt{set.mm} database, you may have noticed another notation used for
proofs.  The other notation is called ``compressed.''\index{compressed
proof}\index{proof!compressed} It reduces the amount of space needed to
store a proof in the database and is described in
Appendix~\ref{compressed}.  In the example above, we use
``normal''\index{normal proof}\index{proof!normal} notation.} The proof
acts as a series of instructions to the Metamath program, telling it how
to build up the sequence of math symbols contained in the assertion part of
the \texttt{\$p} statement, making use of the hypotheses of the
\texttt{\$p} statement and previous assertions.  The construction takes
place according to precise rules.  If the list of labels in the proof
causes these rules to be violated, or if the final sequence that results
does not match the assertion, the Metamath program will notify you with
an error message.

If you are familiar with reverse Polish notation (RPN), which is sometimes used
on pocket calculators, here in a nutshell is how a proof works.  Each
hypothesis label\index{hypothesis label} in the proof is pushed\index{push}
onto the RPN stack\index{stack}\index{RPN stack} as it is encountered. Each
assertion label\index{assertion label} pops\index{pop} off the stack as many
entries as the referenced assertion has mandatory hypotheses.  Variable
substitutions\index{substitution!variable}\index{variable substitution} are
computed which, when made to the referenced assertion's mandatory hypotheses,
cause these hypotheses to match the stack entries. These same substitutions
are then made to the variables in the referenced assertion itself, which is
then pushed onto the stack.  At the end of the proof, there should be one
stack entry, namely the assertion being proved.  This process is explained in
detail in Section~\ref{proof}.

Metamath's proof notation is not very readable for humans, but it allows the
proof to be stored compactly in a file.  The Metamath\index{Metamath} program
has proof display features that let you see what's going on in a more
readable way, as you will see in the next section.

The rules used in verifying a proof are not based on any built-in syntax of the
symbol sequence in an assertion\index{assertion} nor on any built-in meanings
attached to specific symbol names.  They are based strictly on symbol
matching:  constants\index{constant} must match themselves, and
variables\index{variable} may be replaced with anything that allows a match to
occur.  For example, instead of \texttt{term}, \texttt{0}, and \verb$|-$ we could
have just as well used \texttt{yellow}, \texttt{zero}, and \texttt{provable}, as long
as we did so consistently throughout the database.  Also, we could have used
\texttt{is provable} (two tokens) instead of \verb$|-$ (one token) throughout the
database.  In each of these cases, the proof would be exactly the same.  The
independence of proofs and notation means that you have a lot of flexibility to
change the notation you use without having to change any proofs.

\section{A Trial Run}\label{trialrun}

Now you are ready to try out the Metamath\index{Metamath} program.

On all computer systems, Metamath has a standard ``command line
interface'' (CLI)\index{command line interface (CLI)} that allows you to
interact with it.  You supply commands to the CLI by typing them on the
keyboard and pressing your keyboard's {\em return} key after each line
you enter.  The CLI is designed to be easy to use and has built-in help
features.

The first thing you should do is to use a text editor to create a file
called \texttt{demo0.mm} and type into it the Metamath source shown on
p.~\pageref{demo0}.  Actually, this file is included with your Metamath
software package, so check that first.  If you type it in, make sure
that you save it in the form of ``plain {\sc ascii} text with line
breaks.''  Most word processors will have this feature.

Next you must run the Metamath program.  Depending on your computer
system and how Metamath is installed, this could range from clicking the
mouse on the Metamath icon to typing \texttt{run metamath} to typing
simply \texttt{metamath}.  (Metamath's {\tt help invoke} command describes
alternate ways of invoking the Metamath program.)

When you first enter Metamath\index{Metamath}, it will be at the CLI, waiting
for your input. You will see something like the following on your screen:
\begin{verbatim}
Metamath - Version 0.177 27-Apr-2019
Type HELP for help, EXIT to exit.
MM>
\end{verbatim}
The \texttt{MM>} prompt means that Metamath is waiting for a command.
Command keywords\index{command keyword} are not case sensitive;
we will use lower-case commands in our examples.
The version number and its release date will probably be different on your
system from the one we show above.

The first thing that you need to do is to read in your
database:\index{\texttt{read} command}\footnote{If a directory path is
needed on Unix,\index{Unix file names}\index{file names!Unix} you should
enclose the path/file name in quotes to prevent Metamath from thinking
that the \texttt{/} in the path name is a command qualifier, e.g.,
\texttt{read \char`\"db/set.mm\char`\"}.  Quotes are optional when there
is no ambiguity.}
\begin{verbatim}
MM> read demo0.mm
\end{verbatim}
Remember to press the {\em return} key after entering this command.  If
you omit the file name, Metamath will prompt you for one.   The syntax for
specifying a Macintosh file name path is given in a footnote on
p.~\pageref{includef}.\index{Macintosh file names}\index{file
names!Macintosh}

If there are any syntax errors in the database, Metamath will let you know
when it reads in the file.  The one thing that Metamath does not check when
reading in a database is that all proofs are correct, because this would
slow it down too much.  It is a good idea to periodically verify the proofs in
a database you are making changes to.  To do this, use the following command
(and do it for your \texttt{demo0.mm} file now).  Note that the \texttt{*} is a
``wild card'' meaning all proofs in the file.\index{\texttt{verify proof} command}
\begin{verbatim}
MM> verify proof *
\end{verbatim}
Metamath will report any proofs that are incorrect.

It is often useful to save the information that the Metamath program displays
on the screen. You can save everything that happens on the screen by opening a
log file. You may want to do this before you read in a database so that you
can examine any errors later on.  To open a log file, type
\begin{verbatim}
MM> open log abc.log
\end{verbatim}
This will open a file called \texttt{abc.log}, and everything that appears on the
screen from this point on will be stored in this file.  The name of the log file
is arbitrary. To close the log file, type
\begin{verbatim}
MM> close log
\end{verbatim}

Several commands let you examine what's inside your database.
Section~\ref{exploring} has an overview of some useful ones.  The
\texttt{show labels} command lets you see what statement
labels\index{label} exist.  A \texttt{*} matches any combination of
characters, and \texttt{t*} refers to all labels starting with the
letter \texttt{t}.\index{\texttt{show labels} command} The \texttt{/all}
is a ``command qualifier''\index{command qualifier} that tells Metamath
to include labels of hypotheses.  (To see the syntax explained, type
\texttt{help show labels}.)  Type
\begin{verbatim}
MM> show labels t* /all
\end{verbatim}
Metamath will respond with
\begin{verbatim}
The statement number, label, and type are shown.
3 tt $f       4 tr $f       5 ts $f       8 tze $a
9 tpl $a      19 th1 $p
\end{verbatim}

You can use the \texttt{show statement} command to get information about a
particular statement.\index{\texttt{show statement} command}
For example, you can get information about the statement with label \texttt{mp}
by typing
\begin{verbatim}
MM> show statement mp /full
\end{verbatim}
Metamath will respond with
\begin{verbatim}
Statement 17 is located on line 43 of the file
"demo0.mm".
"Define the modus ponens inference rule"
17 mp $a |- Q $.
Its mandatory hypotheses in RPN order are:
  wp $f wff P $.
  wq $f wff Q $.
  min $e |- P $.
  maj $e |- ( P -> Q ) $.
The statement and its hypotheses require the
      variables:  Q P
The variables it contains are:  Q P
\end{verbatim}
The mandatory hypotheses\index{mandatory hypothesis} and their
order\index{RPN order} are
useful to know when you are trying to understand or debug a proof.

Now you are ready to look at what's really inside our proof.  First, here is
how to look at every step in the proof---not just the ones corresponding to an
ordinary formal proof\index{formal proof}, but also the ones that build up the
formulas that appear in each ordinary formal proof step.\index{\texttt{show
proof} command}
\begin{verbatim}
MM> show proof th1 /lemmon /all
\end{verbatim}

This will display the proof on the screen in the following format:
\begin{verbatim}
 1 tt            $f term t
 2 tze           $a term 0
 3 1,2 tpl       $a term ( t + 0 )
 4 tt            $f term t
 5 3,4 weq       $a wff ( t + 0 ) = t
 6 tt            $f term t
 7 tt            $f term t
 8 6,7 weq       $a wff t = t
 9 tt            $f term t
10 9 a2          $a |- ( t + 0 ) = t
11 tt            $f term t
12 tze           $a term 0
13 11,12 tpl     $a term ( t + 0 )
14 tt            $f term t
15 13,14 weq     $a wff ( t + 0 ) = t
16 tt            $f term t
17 tze           $a term 0
18 16,17 tpl     $a term ( t + 0 )
19 tt            $f term t
20 18,19 weq     $a wff ( t + 0 ) = t
21 tt            $f term t
22 tt            $f term t
23 21,22 weq     $a wff t = t
24 20,23 wim     $a wff ( ( t + 0 ) = t -> t = t )
25 tt            $f term t
26 25 a2         $a |- ( t + 0 ) = t
27 tt            $f term t
28 tze           $a term 0
29 27,28 tpl     $a term ( t + 0 )
30 tt            $f term t
31 tt            $f term t
32 29,30,31 a1   $a |- ( ( t + 0 ) = t -> ( ( t + 0 )
                                     = t -> t = t ) )
33 15,24,26,32 mp  $a |- ( ( t + 0 ) = t -> t = t )
34 5,8,10,33 mp  $a |- t = t
\end{verbatim}

The \texttt{/lemmon} command qualifier specifies what is known as a Lemmon-style
display\index{Lemmon-style proof}\index{proof!Lemmon-style}.  Omitting the
\texttt{/lemmon} qualifier results in a tree-style proof (see
p.~\pageref{treeproof} for an example) that is somewhat less explicit but
easier to follow once you get used to it.\index{tree-style
proof}\index{proof!tree-style}

The first number on each line is the step
number of the proof.  Any numbers that follow are step numbers assigned to the
hypotheses of the statement referenced by that step.  Next is the label of
the statement referenced by the step.  The statement type of the statement
referenced comes next, followed by the math symbol\index{math symbol} string
constructed by the proof up to that step.

The last step, 34, contains the statement that is being proved.

Looking at a small piece of the proof, notice that steps 3 and 4 have
established that
\texttt{( t + 0 )} and \texttt{t} are \texttt{term}\,s, and step 5 makes use of steps 3 and
4 to establish that \texttt{( t + 0 ) = t} is a \texttt{wff}.  Let Metamath
itself tell us in detail what is happening in step 5.  Note that the
``target hypothesis'' refers to where step 5 is eventually used, i.e., in step
34.
\begin{verbatim}
MM> show proof th1 /detailed_step 5
Proof step 5:  wp=weq $a wff ( t + 0 ) = t
This step assigns source "weq" ($a) to target "wp"
($f).  The source assertion requires the hypotheses
"tt" ($f, step 3) and "tr" ($f, step 4).  The parent
assertion of the target hypothesis is "mp" ($a,
step 34).
The source assertion before substitution was:
    weq $a wff t = r
The following substitutions were made to the source
assertion:
    Variable  Substituted with
     t         ( t + 0 )
     r         t
The target hypothesis before substitution was:
    wp $f wff P
The following substitution was made to the target
hypothesis:
    Variable  Substituted with
     P         ( t + 0 ) = t
\end{verbatim}

The full proof just shown is useful to understand what is going on in detail.
However, most of the time you will just be interested in
the ``essential'' or logical steps of a proof, i.e.\ those steps
that correspond to an
ordinary formal proof\index{formal proof}.  If you type
\begin{verbatim}
MM> show proof th1 /lemmon /renumber
\end{verbatim}
you will see\label{demoproof}
\begin{verbatim}
1 a2             $a |- ( t + 0 ) = t
2 a2             $a |- ( t + 0 ) = t
3 a1             $a |- ( ( t + 0 ) = t -> ( ( t + 0 )
                                     = t -> t = t ) )
4 2,3 mp         $a |- ( ( t + 0 ) = t -> t = t )
5 1,4 mp         $a |- t = t
\end{verbatim}
Compare this to the formal proof on p.~\pageref{zeroproof} and
notice the resemblance.
By default Metamath
does not show \texttt{\$f}\index{\texttt{\$f}
statement} hypotheses and everything branching off of them in the proof tree
when the proof is displayed; this makes the proof look more like an ordinary
mathematical proof, which does not normally incorporate the explicit
construction of expressions.
This is called the ``essential'' view
(at one time you had to add the
\texttt{/essential} qualifier in the \texttt{show proof}
command to get this view, but this is now the default).
You can could use the \texttt{/all} qualifier in the \texttt{show
proof} command to also show the explicit construction of expressions.
The \texttt{/renumber} qualifier means to renumber
the steps to correspond only to what is displayed.\index{\texttt{show proof}
command}

To exit Metamath, type\index{\texttt{exit} command}
\begin{verbatim}
MM> exit
\end{verbatim}

\subsection{Some Hints for Using the Command Line Interface}

We will conclude this quick introduction to Metamath\index{Metamath} with some
helpful hints on how to navigate your way through the commands.
\index{command line interface (CLI)}

When you type commands into Metamath's CLI, you only have to type as many
characters of a command keyword\index{command keyword} as are needed to make
it unambiguous.  If you type too few characters, Metamath will tell you what
the choices are.  In the case of the \texttt{read} command, only the \texttt{r} is
needed to specify it unambiguously, so you could have typed\index{\texttt{read}
command}
\begin{verbatim}
MM> r demo0.mm
\end{verbatim}
instead of
\begin{verbatim}
MM> read demo0.mm
\end{verbatim}
In our description, we always show the full command words.  When using the
Metamath CLI commands in a command file (to be read with the \texttt{submit}
command)\index{\texttt{submit} command}, it is good practice to use
the unabbreviated command to ensure your instructions will not become ambiguous
if more commands are added to the Metamath program in the future.

The command keywords\index{command
keyword} are not case sensitive; you may type either \texttt{read} or
\texttt{ReAd}.  File names may or may not be case sensitive, depending on your
computer's operating system.  Metamath label\index{label} and math
symbol\index{math symbol} tokens\index{token} are case-sensitive.

The \texttt{help} command\index{\texttt{help} command} will provide you
with a list of topics you can get help on.  You can then type
\texttt{help} {\em topic} to get help on that topic.

If you are uncertain of a command's spelling, just type as many characters
as you remember of the command.  If you have not typed enough characters to
specify it unambiguously, Metamath will tell you what choices you have.

\begin{verbatim}
MM> show s
         ^
?Ambiguous keyword - please specify SETTINGS,
STATEMENT, or SOURCE.
\end{verbatim}

If you don't know what argument to use as part of a command, type a
\texttt{?}\index{\texttt{]}@\texttt{?}\ in command lines}\ at the
argument position.  Metamath will tell you what it expected there.

\begin{verbatim}
MM> show ?
         ^
?Expected SETTINGS, LABELS, STATEMENT, SOURCE, PROOF,
MEMORY, TRACE_BACK, or USAGE.
\end{verbatim}

Finally, you may type just the first word or words of a command followed
by {\em return}.  Metamath will prompt you for the remaining part of the
command, showing you the choices at each step.  For example, instead of
typing \texttt{show statement th1 /full} you could interact in the
following manner:
\begin{verbatim}
MM> show
SETTINGS, LABELS, STATEMENT, SOURCE, PROOF,
MEMORY, TRACE_BACK, or USAGE <SETTINGS>? st
What is the statement label <th1>?
/ or nothing <nothing>? /
TEX, COMMENT_ONLY, or FULL <TEX>? f
/ or nothing <nothing>?
19 th1 $p |- t = t $= ... $.
\end{verbatim}
After each \texttt{?}\ in this mode, you must give Metamath the
information it requests.  Sometimes Metamath gives you a list of choices
with the default choice indicated by brackets \texttt{< > }. Pressing
{\em return} after the \texttt{?}\ will select the default choice.
Answering anything else will override the default.  Note that the
\texttt{/} in command qualifiers is considered a separate
token\index{token} by the parser, and this is why it is asked for
separately.

\section{Your First Proof}\label{frstprf}

Proofs are developed with the aid of the Proof Assistant\index{Proof
Assistant}.  We will now show you how the proof of theorem \texttt{th1}
was built.  So that you can repeat these steps, we will first have the
Proof Assistant erase the proof in Metamath's source buffer\index{source
buffer}, then reconstruct it.  (The source buffer is the place in memory
where Metamath stores the information in the database when it is
\texttt{read}\index{\texttt{read} command} in.  New or modified proofs
are kept in the source buffer until a \texttt{write source}
command\index{\texttt{write source} command} is issued.)  In practice, you
would place a \texttt{?}\index{\texttt{]}@\texttt{?}\ inside proofs}\
between \texttt{\$=}\index{\texttt{\$=} keyword} and
\texttt{\$.}\index{\texttt{\$.}\ keyword}\ in the database to indicate
to Metamath\index{Metamath} that the proof is unknown, and that would be
your starting point.  Whenever the \texttt{verify proof} command encounters
a proof with a \texttt{?}\ in place of a proof step, the statement is
identified as not proved.

When I first started creating Metamath proofs, I would write down
on a piece of paper the complete
formal proof\index{formal proof} as it would appear
in a \texttt{show proof} command\index{\texttt{show proof} command}; see
the display of \texttt{show proof th1 /lemmon /re\-num\-ber} above as an
example.  After you get used to using the Proof Assistant\index{Proof
Assistant} you may get to a point where you can ``see'' the proof in your mind
and let the Proof Assistant guide you in filling in the details, at least for
simpler proofs, but until you gain that experience you may find it very useful
to write down all the details in advance.
Otherwise you may waste a lot of time as you let it take you down a wrong path.
However, others do not find this approach as helpful.
For example, Thomas Brendan Leahy\index{Leahy, Thomas Brendan}
finds that it is more helpful to him to interactively
work backward from a machine-readable statement.
David A. Wheeler\index{Wheeler, David A.}
writes down a general approach, but develops the proof
interactively by switching between
working forwards (from hypotheses and facts likely to be useful) and
backwards (from the goal) until the forwards and backwards approaches meet.
In the end, use whatever approach works for you.

A proof is developed with the Proof Assistant by working backwards, starting
with the theorem\index{theorem} to be proved, and assigning each unknown step
with a theorem or hypothesis until no more unknown steps remain.  The Proof
Assistant will not let you make an assignment unless it can be ``unified''
with the unknown step.  This means that a
substitution\index{substitution!variable}\index{variable substitution} of
variables exists that will make the assignment match the unknown step.  On the
other hand, in the middle of a proof, when working backwards, often more than
one unification\index{unification} (set of substitutions) is possible, since
there is not enough information available at that point to uniquely establish
it.  In this case you can tell Metamath which unification to choose, or you
can continue to assign unknown steps until enough information is available to
make the unification unique.

We will assume you have entered Metamath and read in the database as described
above.  The following dialog shows how the proof was developed.  For more
details on what some of the commands do, refer to Section~\ref{pfcommands}.
\index{\texttt{prove} command}

\begin{verbatim}
MM> prove th1
Entering the Proof Assistant.  Type HELP for help, EXIT
to exit.  You will be working on the proof of statement th1:
  $p |- t = t
Note:  The proof you are starting with is already complete.
MM-PA>
\end{verbatim}

The \verb/MM-PA>/ prompt means we are inside the Proof
Assistant.\index{Proof Assistant} Most of the regular Metamath commands
(\texttt{show statement}, etc.) are still available if you need them.

\begin{verbatim}
MM-PA> delete all
The entire proof was deleted.
\end{verbatim}

We have deleted the whole proof so we can start from scratch.

\begin{verbatim}
MM-PA> show new_proof/lemmon/all
1 ?              $? |- t = t
\end{verbatim}

The \texttt{show new{\char`\_}proof} command\index{\texttt{show
new{\char`\_}proof} command} is like \texttt{show proof} except that we
don't specify a statement; instead, the proof we're working on is
displayed.

\begin{verbatim}
MM-PA> assign 1 mp
To undo the assignment, DELETE STEP 5 and INITIALIZE, UNIFY
if needed.
3   min=?  $? |- $2
4   maj=?  $? |- ( $2 -> t = t )
\end{verbatim}

The \texttt{assign} command\index{\texttt{assign} command} above means
``assign step 1 with the statement whose label is \texttt{mp}.''  Note
that step renumbering will constantly occur as you assign steps in the
middle of a proof; in general all steps from the step you assign until
the end of the proof will get moved up.  In this case, what used to be
step 1 is now step 5, because the (partial) proof now has five steps:
the four hypotheses of the \texttt{mp} statement and the \texttt{mp}
statement itself.  Let's look at all the steps in our partial proof:

\begin{verbatim}
MM-PA> show new_proof/lemmon/all
1 ?              $? wff $2
2 ?              $? wff t = t
3 ?              $? |- $2
4 ?              $? |- ( $2 -> t = t )
5 1,2,3,4 mp     $a |- t = t
\end{verbatim}

The symbol \texttt{\$2} is a temporary variable\index{temporary
variable} that represents a symbol sequence not yet known.  In the final
proof, all temporary variables will be eliminated.  The general format
for a temporary variable is \texttt{\$} followed by an integer.  Note
that \texttt{\$} is not a legal character in a math symbol (see
Section~\ref{dollardollar}, p.~\pageref{dollardollar}), so there will
never be a naming conflict between real symbols and temporary variables.

Unknown steps 1 and 2 are constructions of the two wffs used by the
modus ponens rule.  As you will see at the end of this section, the
Proof Assistant\index{Proof Assistant} can usually figure these steps
out by itself, and we will not have to worry about them.  Therefore from
here on we will display only the ``essential'' hypotheses, i.e.\ those
steps that correspond to traditional formal proofs\index{formal proof}.

\begin{verbatim}
MM-PA> show new_proof/lemmon
3 ?              $? |- $2
4 ?              $? |- ( $2 -> t = t )
5 3,4 mp         $a |- t = t
\end{verbatim}

Unknown steps 3 and 4 are the ones we must focus on.  They correspond to the
minor and major premises of the modus ponens rule.  We will assign them as
follows.  Notice that because of the step renumbering that takes place
after an assignment, it is advantageous to assign unknown steps in reverse
order, because earlier steps will not get renumbered.

\begin{verbatim}
MM-PA> assign 4 mp
To undo the assignment, DELETE STEP 8 and INITIALIZE, UNIFY
if needed.
3   min=?  $? |- $2
6     min=?  $? |- $4
7     maj=?  $? |- ( $4 -> ( $2 -> t = t ) )
\end{verbatim}

We are now going to describe an obscure feature that you will probably
never use but should be aware of.  The Metamath language allows empty
symbol sequences to be substituted for variables, but in most formal
systems this feature is never used.  One of the few examples where is it
used is the MIU-system\index{MIU-system} described in
Appendix~\ref{MIU}.  But such systems are rare, and by default this
feature is turned off in the Proof Assistant.  (It is always allowed for
{\tt verify proof}.)  Let us turn it on and see what
happens.\index{\texttt{set empty{\char`\_}substitution} command}

\begin{verbatim}
MM-PA> set empty_substitution on
Substitutions with empty symbol sequences is now allowed.
\end{verbatim}

With this feature enabled, more unifications will be
ambiguous\index{ambiguous unification}\index{unification!ambiguous} in
the middle of a proof, because
substitution\index{substitution!variable}\index{variable substitution}
of variables with empty symbol sequences will become an additional
possibility.  Let's see what happens when we make our next assignment.

\begin{verbatim}
MM-PA> assign 3 a2
There are 2 possible unifications.  Please select the correct
    one or Q if you want to UNIFY later.
Unify:  |- $6
 with:  |- ( $9 + 0 ) = $9
Unification #1 of 2 (weight = 7):
  Replace "$6" with "( + 0 ) ="
  Replace "$9" with ""
  Accept (A), reject (R), or quit (Q) <A>? r
\end{verbatim}

The first choice presented is the wrong one.  If we had selected it,
temporary variable \texttt{\$6} would have been assigned a truncated
wff, and temporary variable \texttt{\$9} would have been assigned an
empty sequence (which is not allowed in our system).  With this choice,
eventually we would reach a point where we would get stuck because
we would end up with steps impossible to prove.  (You may want to
try it.)  We typed \texttt{r} to reject the choice.

\begin{verbatim}
Unification #2 of 2 (weight = 21):
  Replace "$6" with "( $9 + 0 ) = $9"
  Accept (A), reject (R), or quit (Q) <A>? q
To undo the assignment, DELETE STEP 4 and INITIALIZE, UNIFY
if needed.
 7     min=?  $? |- $8
 8     maj=?  $? |- ( $8 -> ( $6 -> t = t ) )
\end{verbatim}

The second choice is correct, and normally we would type \texttt{a}
to accept it.  But instead we typed \texttt{q} to show what will happen:
it will leave the step with an unknown unification, which can be
seen as follows:

\begin{verbatim}
MM-PA> show new_proof/not_unified
 4   min    $a |- $6
        =a2  = |- ( $9 + 0 ) = $9
\end{verbatim}

Later we can unify this with the \texttt{unify}
\texttt{all/interactive} command.

The important point to remember is that occasionally you will be
presented with several unification choices while entering a proof, when
the program determines that there is not enough information yet to make
an unambiguous choice automatically (and this can happen even with
\texttt{set empty{\char`\_}substitution} turned off).  Usually it is
obvious by inspection which choice is correct, since incorrect ones will
tend to be meaningless fragments of wffs.  In addition, the correct
choice will usually be the first one presented, unlike our example
above.

Enough of this digression.  Let us go back to the default setting.

\begin{verbatim}
MM-PA> set empty_substitution off
The ability to substitute empty expressions for variables
has been turned off.  Note that this may make the Proof
Assistant too restrictive in some cases.
\end{verbatim}

If we delete the proof, start over, and get to the point where
we digressed above, there will no longer be an ambiguous unification.

\begin{verbatim}
MM-PA> assign 3 a2
To undo the assignment, DELETE STEP 4 and INITIALIZE, UNIFY
if needed.
 7     min=?  $? |- $4
 8     maj=?  $? |- ( $4 -> ( ( $5 + 0 ) = $5 -> t = t ) )
\end{verbatim}

Let us look at our proof so far, and continue.

\begin{verbatim}
MM-PA> show new_proof/lemmon
 4 a2            $a |- ( $5 + 0 ) = $5
 7 ?             $? |- $4
 8 ?             $? |- ( $4 -> ( ( $5 + 0 ) = $5 -> t = t ) )
 9 7,8 mp        $a |- ( ( $5 + 0 ) = $5 -> t = t )
10 4,9 mp        $a |- t = t
MM-PA> assign 8 a1
To undo the assignment, DELETE STEP 11 and INITIALIZE, UNIFY
if needed.
 7     min=?  $? |- ( t + 0 ) = t
MM-PA> assign 7 a2
To undo the assignment, DELETE STEP 8 and INITIALIZE, UNIFY
if needed.
MM-PA> show new_proof/lemmon
 4 a2            $a |- ( t + 0 ) = t
 8 a2            $a |- ( t + 0 ) = t
12 a1            $a |- ( ( t + 0 ) = t -> ( ( t + 0 ) = t ->
                                                    t = t ) )
13 8,12 mp       $a |- ( ( t + 0 ) = t -> t = t )
14 4,13 mp       $a |- t = t
\end{verbatim}

Now all temporary variables and unknown steps have been eliminated from the
``essential'' part of the proof.  When this is achieved, the Proof
Assistant\index{Proof Assistant} can usually figure out the rest of the proof
automatically.  (Note that the \texttt{improve} command can occasionally be
useful for filling in essential steps as well, but it only tries to make use
of statements that introduce no new variables in their hypotheses, which is
not the case for \texttt{mp}. Also it will not try to improve steps containing
temporary variables.)  Let's look at the complete proof, then run
the \texttt{improve} command, then look at it again.

\begin{verbatim}
MM-PA> show new_proof/lemmon/all
 1 ?             $? wff ( t + 0 ) = t
 2 ?             $? wff t = t
 3 ?             $? term t
 4 3 a2          $a |- ( t + 0 ) = t
 5 ?             $? wff ( t + 0 ) = t
 6 ?             $? wff ( ( t + 0 ) = t -> t = t )
 7 ?             $? term t
 8 7 a2          $a |- ( t + 0 ) = t
 9 ?             $? term ( t + 0 )
10 ?             $? term t
11 ?             $? term t
12 9,10,11 a1    $a |- ( ( t + 0 ) = t -> ( ( t + 0 ) = t ->
                                                    t = t ) )
13 5,6,8,12 mp   $a |- ( ( t + 0 ) = t -> t = t )
14 1,2,4,13 mp   $a |- t = t
\end{verbatim}

\begin{verbatim}
MM-PA> improve all
A proof of length 1 was found for step 11.
A proof of length 1 was found for step 10.
A proof of length 3 was found for step 9.
A proof of length 1 was found for step 7.
A proof of length 9 was found for step 6.
A proof of length 5 was found for step 5.
A proof of length 1 was found for step 3.
A proof of length 3 was found for step 2.
A proof of length 5 was found for step 1.
Steps 1 and above have been renumbered.
CONGRATULATIONS!  The proof is complete.  Use SAVE
NEW_PROOF to save it.  Note:  The Proof Assistant does
not detect $d violations.  After saving the proof, you
should verify it with VERIFY PROOF.
\end{verbatim}

The \texttt{save new{\char`\_}proof} command\index{\texttt{save
new{\char`\_}proof} command} will save the proof in the database.  Here
we will just display it in a form that can be clipped out of a log file
and inserted manually into the database source file with a text
editor.\index{normal proof}\index{proof!normal}

\begin{verbatim}
MM-PA> show new_proof/normal
---------Clip out the proof below this line:
      tt tze tpl tt weq tt tt weq tt a2 tt tze tpl tt weq
      tt tze tpl tt weq tt tt weq wim tt a2 tt tze tpl tt
      tt a1 mp mp $.
---------The proof of 'th1' to clip out ends above this line.
\end{verbatim}

There is another proof format called ``compressed''\index{compressed
proof}\index{proof!compressed} that you will see in databases.  It is
not important to understand how it is encoded but only to recognize it
when you see it.  Its only purpose is to reduce storage requirements for
large proofs.  A compressed proof can always be converted to a normal
one and vice-versa, and the Metamath \texttt{show proof}
commands\index{\texttt{show proof} command} work equally well with
compressed proofs.  The compressed proof format is described in
Appendix~\ref{compressed}.

\begin{verbatim}
MM-PA> show new_proof/compressed
---------Clip out the proof below this line:
      ( tze tpl weq a2 wim a1 mp ) ABCZADZAADZAEZJJKFLIA
      AGHH $.
---------The proof of 'th1' to clip out ends above this line.
\end{verbatim}

Now we will exit the Proof Assistant.  Since we made changes to the proof,
it will warn us that we have not saved it.  In this case, we don't care.

\begin{verbatim}
MM-PA> exit
Warning:  You have not saved changes to the proof.
Do you want to EXIT anyway (Y, N) <N>? y
Exiting the Proof Assistant.
Type EXIT again to exit Metamath.
\end{verbatim}

The Proof Assistant\index{Proof Assistant} has several other commands
that can help you while creating proofs.  See Section~\ref{pfcommands}
for a list of them.

A command that is often useful is \texttt{minimize{\char`\_}with
*/brief}, which tries to shorten the proof.  It can make the process
more efficient by letting you write a somewhat ``sloppy'' proof then
clean up some of the fine details of optimization for you (although it
can't perform miracles such as restructuring the overall proof).

\section{A Note About Editing a Data\-base File}

Once your source file contains proofs, there are some restrictions on
how you can edit it so that the proofs remain valid.  Pay particular
attention to these rules, since otherwise you can lose a lot of work.
It is a good idea to periodically verify all proofs with \texttt{verify
proof *} to ensure their integrity.

If your file contains only normal (as opposed to compressed) proofs, the
main rule is that you may not change the order of the mandatory
hypotheses\index{mandatory hypothesis} of any statement referenced in a
later proof.  For example, if you swap the order of the major and minor
premise in the modus ponens rule, all proofs making use of that rule
will become incorrect.  The \texttt{show statement}
command\index{\texttt{show statement} command} will show you the
mandatory hypotheses of a statement and their order.

If a statement has a compressed proof, you also must not change the
order of {\em its} mandatory hypotheses.  The compressed proof format
makes use of this information as part of the compression technique.
Note that swapping the names of two variables in a theorem will change
the order of its mandatory hypotheses.

The safest way to edit a statement, say \texttt{mytheorem}, is to
duplicate it then rename the original to \texttt{mytheoremOLD}
throughout the database.  Once the edited version is re-proved, all
statements referencing \texttt{mytheoremOLD} can be updated in the Proof
Assistant using \texttt{minimize{\char`\_}with
mytheorem
/allow{\char`\_}growth}.\index{\texttt{minimize{\char`\_}with} command}
% 3/10/07 Note: line-breaking the above results in duplicate index entries

\chapter{Abstract Mathematics Revealed}\label{fol}

\section{Logic and Set Theory}\label{logicandsettheory}

\begin{quote}
  {\em Set theory can be viewed as a form of exact theology.}
  \flushright\sc  Rudy Rucker\footnote{\cite{Barrow}, p.~31.}\\
\end{quote}\index{Rucker, Rudy}

Despite its seeming complexity, all of standard mathematics, no matter how
deep or abstract, can amazingly enough be derived from a relatively small set
of axioms\index{axiom} or first principles. The development of these axioms is
among the most impressive and important accomplishments of mathematics in the
20th century. Ultimately, these axioms can be broken down into a set of rules
for manipulating symbols that any technically oriented person can follow.

We will not spend much time trying to convey a deep, higher-level
understanding of the meaning of the axioms. This kind of understanding
requires some mathematical sophistication as well as an understanding of the
philosophy underlying the foundations of mathematics and typically develops
over time as you work with mathematics.  Our goal, instead, is to give you the
immediate ability to follow how theorems\index{theorem} are derived from the
axioms and from other theorems.  This will be similar to learning the syntax
of a computer language, which lets you follow the details in a program but
does not necessarily give you the ability to write non-trivial programs on
your own, an ability that comes with practice. For now don't be alarmed by
abstract-sounding names of the axioms; just focus on the rules for
manipulating the symbols, which follow the simple conventions of the
Metamath\index{Metamath} language.

The axioms that underlie all of standard mathematics consist of axioms of logic
and axioms of set theory. The axioms of logic are divided into two
subcategories, propositional calculus\index{propositional calculus} (sometimes
called sentential logic\index{sentential logic}) and predicate calculus
(sometimes called first-order logic\index{first-order logic}\index{quantifier
theory}\index{predicate calculus} or quantifier theory).  Propositional
calculus is a prerequisite for predicate calculus, and predicate calculus is a
prerequisite for set theory.  The version of set theory most commonly used is
Zermelo--Fraenkel set theory\index{Zermelo--Fraenkel set theory}\index{set theory}
with the axiom of choice,
often abbreviated as ZFC\index{ZFC}.

Here in a nutshell is what the axioms are all about in an informal way. The
connection between this description and symbols we will show you won't be
immediately apparent and in principle needn't ever be.  Our description just
tries to summarize what mathematicians think about when they work with the
axioms.

Logic is a set of rules that allow us determine truths given other truths.
Put another way,
logic is more or less the translation of what we would consider common sense
into a rigorous set of axioms.\index{axioms of logic}  Suppose $\varphi$,
$\psi$, and $\chi$ (the Greek letters phi, psi, and chi) represent statements
that are either true or false, and $x$ is a variable\index{variable!in predicate
calculus} ranging over some group of mathematical objects (sets, integers,
real numbers, etc.). In mathematics, a ``statement'' really means a formula,
and $\psi$ could be for example ``$x = 2$.''
Propositional calculus\index{propositional calculus}
allows us to use variables that are either true or false
and make deductions such as
``if $\varphi$ implies $\psi$ and $\psi$ implies $\chi$, then $\varphi$
implies $\chi$.''
Predicate calculus\index{predicate calculus}
extends propositional calculus by also allowing us
to discuss statements about objects (not just true and false values), including
statements about ``all'' or ``at least one'' object.
For example, predicate calculus allows to say,
``if $\varphi$ is true for all $x$, then $\varphi$ is true for some $x$.''
The logic used in \texttt{set.mm} is standard classical logic
(as opposed to other logic systems like intuitionistic logic).

Set theory\index{set theory} has to do with the manipulation of objects and
collections of objects, specifically the abstract, imaginary objects that
mathematics deals with, such as numbers. Everything that is claimed to exist
in mathematics is considered to be a set.  A set called the empty
set\index{empty set} contains nothing.  We represent the empty set by
$\varnothing$.  Many sets can be built up from the empty set.  There is a set
represented by $\{\varnothing\}$ that contains the empty set, another set
represented by $\{\varnothing,\{\varnothing\}\}$ that contains this set as
well as the empty set, another set represented by $\{\{\varnothing\}\}$ that
contains just the set that contains the empty set, and so on ad infinitum. All
mathematical objects, no matter how complex, are defined as being identical to
certain sets: the integer\index{integer} 0 is defined as the empty set, the
integer 1 is defined as $\{\varnothing\}$, the integer 2 is defined as
$\{\varnothing,\{\varnothing\}\}$.  (How these definitions were chosen doesn't
matter now, but the idea behind it is that these sets have the properties we
expect of integers once suitable operations are defined.)  Mathematical
operations, such as addition, are defined in terms of operations on
sets---their union\index{set union}, intersection\index{set intersection}, and
so on---operations you may have used in elementary school when you worked
with groups of apples and oranges.

With a leap of faith, the axioms also postulate the existence of infinite
sets\index{infinite set}, such as the set of all non-negative integers ($0, 1,
2,\ldots$, also called ``natural numbers''\index{natural number}).  This set
can't be represented with the brace notation\index{brace notation} we just
showed you, but requires a more complicated notation called ``class
abstraction.''\index{class abstraction}\index{abstraction class}  For
example, the infinite set $\{ x |
\mbox{``$x$ is a natural number''} \} $ means the ``set of all objects $x$
such that $x$ is a natural number'' i.e.\ the set of natural numbers; here,
``$x$ is a natural number'' is a rather complicated formula when broken down
into the primitive symbols.\label{expandom}\footnote{The statement ``$x$ is a
natural number'' is formally expressed as ``$x \in \omega$,'' where $\in$
(stylized epsilon) means ``is in'' or ``is an element of'' and $\omega$
(omega) means ``the set of natural numbers.''  When ``$x\in\omega$'' is
completely expanded in terms of the primitive symbols of set theory, the
result is  $\lnot$ $($ $\lnot$ $($ $\forall$ $z$ $($ $\lnot$ $\forall$ $w$ $($
$z$ $\in$ $w$ $\rightarrow$ $\lnot$ $w$ $\in$ $x$ $)$ $\rightarrow$ $z$ $\in$
$x$ $)$ $\rightarrow$ $($ $\forall$ $z$ $($ $\lnot$ $($ $\forall$ $w$ $($ $w$
$\in$ $z$ $\rightarrow$ $w$ $\in$ $x$ $)$ $\rightarrow$ $\forall$ $w$ $\lnot$
$w$ $\in$ $z$ $)$ $\rightarrow$ $\lnot$ $\forall$ $w$ $($ $w$ $\in$ $z$
$\rightarrow$ $\lnot$ $\forall$ $v$ $($ $v$ $\in$ $z$ $\rightarrow$ $\lnot$
$v$ $\in$ $w$ $)$ $)$ $)$ $\rightarrow$ $\lnot$ $\forall$ $z$ $\forall$ $w$
$($ $\lnot$ $($ $z$ $\in$ $x$ $\rightarrow$ $\lnot$ $w$ $\in$ $x$ $)$
$\rightarrow$ $($ $\lnot$ $z$ $\in$ $w$ $\rightarrow$ $($ $\lnot$ $z$ $=$ $w$
$\rightarrow$ $w$ $\in$ $z$ $)$ $)$ $)$ $)$ $)$ $\rightarrow$ $\lnot$
$\forall$ $y$ $($ $\lnot$ $($ $\lnot$ $($ $\forall$ $z$ $($ $\lnot$ $\forall$
$w$ $($ $z$ $\in$ $w$ $\rightarrow$ $\lnot$ $w$ $\in$ $y$ $)$ $\rightarrow$
$z$ $\in$ $y$ $)$ $\rightarrow$ $($ $\forall$ $z$ $($ $\lnot$ $($ $\forall$
$w$ $($ $w$ $\in$ $z$ $\rightarrow$ $w$ $\in$ $y$ $)$ $\rightarrow$ $\forall$
$w$ $\lnot$ $w$ $\in$ $z$ $)$ $\rightarrow$ $\lnot$ $\forall$ $w$ $($ $w$
$\in$ $z$ $\rightarrow$ $\lnot$ $\forall$ $v$ $($ $v$ $\in$ $z$ $\rightarrow$
$\lnot$ $v$ $\in$ $w$ $)$ $)$ $)$ $\rightarrow$ $\lnot$ $\forall$ $z$
$\forall$ $w$ $($ $\lnot$ $($ $z$ $\in$ $y$ $\rightarrow$ $\lnot$ $w$ $\in$
$y$ $)$ $\rightarrow$ $($ $\lnot$ $z$ $\in$ $w$ $\rightarrow$ $($ $\lnot$ $z$
$=$ $w$ $\rightarrow$ $w$ $\in$ $z$ $)$ $)$ $)$ $)$ $\rightarrow$ $($
$\forall$ $z$ $\lnot$ $z$ $\in$ $y$ $\rightarrow$ $\lnot$ $\forall$ $w$ $($
$\lnot$ $($ $w$ $\in$ $y$ $\rightarrow$ $\lnot$ $\forall$ $z$ $($ $w$ $\in$
$z$ $\rightarrow$ $\lnot$ $z$ $\in$ $y$ $)$ $)$ $\rightarrow$ $\lnot$ $($
$\lnot$ $\forall$ $z$ $($ $w$ $\in$ $z$ $\rightarrow$ $\lnot$ $z$ $\in$ $y$
$)$ $\rightarrow$ $w$ $\in$ $y$ $)$ $)$ $)$ $)$ $\rightarrow$ $x$ $\in$ $y$
$)$ $)$ $)$. Section~\ref{hierarchy} shows the hierarchy of definitions that
leads up to this expression.}\index{stylized epsilon ($\in$)}\index{omega
($\omega$)}  Actually, the primitive symbols don't even include the brace
notation.  The brace notation is a high-level definition, which you can find in
Section~\ref{hierarchy}.

Interestingly, the arithmetic of integers\index{integer} and
rationals\index{rational number} can be developed without appealing to the
existence of an infinite set, whereas the arithmetic of real
numbers\index{real number} requires it.

Each variable\index{variable!in set theory} in the axioms of set theory
represents an arbitrary set, and the axioms specify the legal kinds of things
you can do with these variables at a very primitive level.

Now, you may think that numbers and arithmetic are a lot more intuitive and
fundamental than sets and therefore should be the foundation of mathematics.
What is really the case is that you've dealt with numbers all your life and
are comfortable with a few rules for manipulating them such as addition and
multiplication.  Those rules only cover a small portion of what can be done
with numbers and only a very tiny fraction of the rest of mathematics.  If you
look at any elementary book on number theory, you will quickly become lost if
these are the only rules that you know.  Even though such books may present a
list of ``axioms''\index{axiom} for arithmetic, the ability to use the axioms
and to understand proofs of theorems\index{theorem} (facts) about numbers
requires an implicit mathematical talent that frustrates many people
from studying abstract mathematics.  The kind of mathematics that most people
know limits them to the practical, everyday usage of blindly manipulating
numbers and formulas, without any understanding of why those rules are correct
nor any ability to go any further.  For example, do you know why multiplying
two negative numbers yields a positive number?  Starting with set theory, you
will also start off blindly manipulating symbols according to the rules we give
you, but with the advantage that these rules will allow you, in principle, to
access {\em all} of mathematics, not just a tiny part of it.

Of course, concrete examples are often helpful in the learning process. For
example, you can verify that $2\cdot 3=3 \cdot 2$ by actually grouping
objects and can easily ``see'' how it generalizes to $x\cdot y = y\cdot x$,
even though you might not be able to rigorously prove it.  Similarly, in set
theory it can be helpful to understand how the axioms of set theory apply to
(and are correct for) small finite collections of objects.  You should be aware
that in set theory intuition can be misleading for infinite collections, and
rigorous proofs become more important.  For example, while $x\cdot y = y\cdot
x$ is correct for finite ordinals (which are the natural numbers), it is not
usually true for infinite ordinals.

\section{The Axioms for All of Mathematics}

In this section\index{axioms for mathematics}, we will show you the axioms
for all of standard mathematics (i.e.\ logic and set theory) as they are
traditionally presented.  The traditional presentation is useful for someone
with the mathematical experience needed to correctly manipulate high-level
abstract concepts.  For someone without this talent, knowing how to actually
make use of these axioms can be difficult.  The purpose of this section is to
allow you to see how the version of the axioms used in the standard
Metamath\index{Metamath} database \texttt{set.mm}\index{set
theory database (\texttt{set.mm})} relates to  the typical version
in textbooks, and also to give you an informal feel for them.

\subsection{Propositional Calculus}

Propositional calculus\index{propositional calculus} concerns itself with
statements that can be interpreted as either true or false.  Some examples of
statements (outside of mathematics) that are either true or false are ``It is
raining today'' and ``The United States has a female president.'' In
mathematics, as we mentioned, statements are really formulas.

In propositional calculus, we don't care what the statements are.  We also
treat a logical combination of statements, such as ``It is raining today and
the United States has a female president,'' no differently from a single
statement.  Statements and their combinations are called well-formed formulas
(wffs)\index{well-formed formula (wff)}.  We define wffs only in terms of
other wffs and don't define what a ``starting'' wff is.  As is common practice
in the literature, we use Greek letters to represent wffs.

Specifically, suppose $\varphi$ and $\psi$ are wffs.  Then the combinations
$\varphi\rightarrow\psi$ (``$\varphi$ implies $\psi$,'' also read ``if
$\varphi$ then $\psi$'')\index{implication ($\rightarrow$)} and $\lnot\varphi$
(``not $\varphi$'')\index{negation ($\lnot$)} are also wffs.

The three axioms of propositional calculus\index{axioms of propositional
calculus} are all wffs of the following form:\footnote{A remarkable result of
C.~A.~Meredith\index{Meredith, C. A.} squeezes these three axioms into the
single axiom $((((\varphi\rightarrow \psi)\rightarrow(\neg \chi\rightarrow\neg
\theta))\rightarrow \chi )\rightarrow \tau)\rightarrow((\tau\rightarrow
\varphi)\rightarrow(\theta\rightarrow \varphi))$ \cite{CAMeredith},
which is believed to be the shortest possible.}
\begin{center}
     $\varphi\rightarrow(\psi\rightarrow \varphi)$\\

     $(\varphi\rightarrow (\psi\rightarrow \chi))\rightarrow
((\varphi\rightarrow  \psi)\rightarrow (\varphi\rightarrow \chi))$\\

     $(\neg \varphi\rightarrow \neg\psi)\rightarrow (\psi\rightarrow
\varphi)$
\end{center}

These three axioms are widely used.
They are attributed to Jan {\L}ukasiewicz
(pronounced woo-kah-SHAY-vitch) and was popularized by Alonzo Church,
who called it system P2. (Thanks to Ted Ulrich for this information.)

There are an infinite number of axioms, one for each possible
wff\index{well-formed formula (wff)} of the above form.  (For this reason,
axioms such as the above are often called ``axiom schemes.''\index{axiom
scheme})  Each Greek letter in the axioms may be substituted with a more
complex wff to result in another axiom.  For example, substituting
$\neg(\varphi\rightarrow\chi)$ for $\varphi$ in the first axiom yields
$\neg(\varphi\rightarrow\chi)\rightarrow(\psi\rightarrow
\neg(\varphi\rightarrow\chi))$, which is still an axiom.

To deduce new true statements (theorems\index{theorem}) from the axioms, a
rule\index{rule} called ``modus ponens''\index{modus ponens} is used.  This
rule states that if the wff $\varphi$ is an axiom or a theorem, and the wff
$\varphi\rightarrow\psi$ is an axiom or a theorem, then the wff $\psi$ is also
a theorem\index{theorem}.

As a non-mathematical example of modus ponens, suppose we have proved (or
taken as an axiom) ``Bob is a man'' and separately have proved (or taken as
an axiom) ``If Bob is a man, then Bob is a human.''  Using the rule of modus
ponens, we can logically deduce, ``Bob is a human.''

From Metamath's\index{Metamath} point of view, the axioms and the rule of
modus ponens just define a mechanical means for deducing new true statements
from existing true statements, and that is the complete content of
propositional calculus as far as Metamath is concerned.  You can read a logic
textbook to gain a better understanding of their meaning, or you can just let
their meaning slowly become apparent to you after you use them for a while.

It is actually rather easy to check to see if a formula is a theorem of
propositional calculus.  Theorems of propositional calculus are also called
``tautologies.''\index{tautology}  The technique to check whether a formula is
a tautology is called the ``truth table method,''\index{truth table} and it
works like this.  A wff $\varphi\rightarrow\psi$ is false whenever $\varphi$ is true
and $\psi$ is false.  Otherwise it is true.  A wff $\lnot\varphi$ is false
whenever $\varphi$ is true and false otherwise. To verify a tautology such as
$\varphi\rightarrow(\psi\rightarrow \varphi)$, you break it down into sub-wffs and
construct a truth table that accounts for all possible combinations of true
and false assigned to the wff metavariables:
\begin{center}\begin{tabular}{|c|c|c|c|}\hline
\mbox{$\varphi$} & \mbox{$\psi$} & \mbox{$\psi\rightarrow\varphi$}
    & \mbox{$\varphi\rightarrow(\psi\rightarrow \varphi)$} \\ \hline \hline
              T   &  T    &      T       &        T    \\ \hline
              T   &  F    &      T       &        T    \\ \hline
              F   &  T    &      F       &        T    \\ \hline
              F   &  F    &      T       &        T    \\ \hline
\end{tabular}\end{center}
If all entries in the last column are true, the formula is a tautology.

Now, the truth table method doesn't tell you how to prove the tautology from
the axioms, but only that a proof exists.  Finding an actual proof (especially
one that is short and elegant) can be challenging.  Methods do exist for
automatically generating proofs in propositional calculus, but the proofs that
result can sometimes be very long.  In the Metamath \texttt{set.mm}\index{set
theory database (\texttt{set.mm})} database, most
or all proofs were created manually.

Section \ref{metadefprop} discusses various definitions
that make propositional calculus easier to use.
For example, we define:

\begin{itemize}
\item $\varphi \vee \psi$
  is true if either $\varphi$ or $\psi$ (or both) are true
  (this is disjunction\index{disjunction ($\vee$)}
  aka logical {\sc or}\index{logical {\sc or} ($\vee$)}).

\item $\varphi \wedge \psi$
  is true if both $\varphi$ and $\psi$ are true
  (this is conjunction\index{conjunction ($\wedge$)}
  aka logical {\sc and}\index{logical {\sc and} ($\wedge$)}).

\item $\varphi \leftrightarrow \psi$
  is true if $\varphi$ and $\psi$ have the same value, that is,
  they are both true or both false
  (this is the biconditional\index{biconditional ($\leftrightarrow$)}).
\end{itemize}

\subsection{Predicate Calculus}

Predicate calculus\index{predicate calculus} introduces the concept of
``individual variables,''\index{variable!in predicate calculus}\index{individual
variable} which
we will usually just call ``variables.''
These variables can represent something other than true or false (wffs),
and will always represent sets when we get to set theory.  There are also
three new symbols $\forall$\index{universal quantifier ($\forall$)},
$=$\index{equality ($=$)}, and $\in$\index{stylized epsilon ($\in$)},
read ``for all,'' ``equals,'' and ``is an element of''
respectively.  We will represent variables with the letters $x$, $y$, $z$, and
$w$, as is common practice in the literature.
For example, $\forall x \varphi$ means ``for all possible values of
$x$, $\varphi$ is true.''

In predicate calculus, we extend the definition of a wff\index{well-formed
formula (wff)}.  If $\varphi$ is a wff and $x$ and $y$ are variables, then
$\forall x \, \varphi$, $x=y$, and $x\in y$ are wffs. Note that these three new
types of wffs can be considered ``starting'' wffs from which we can build
other wffs with $\rightarrow$ and $\neg$ .  The concept of a starting wff was
absent in propositional calculus.  But starting wff or not, all we are really
concerned with is whether our wffs are correctly constructed according to
these mechanical rules.

A quick aside:
To prevent confusion, it might be best at this point to think of the variables
of Metamath\index{Metamath} as ``metavariables,''\index{metavariable} because
they are not quite the same as the variables we are introducing here.  A
(meta)variable in Metamath can be a wff or an individual variable, as well
as many other things; in general, it represents a kind of place holder for an
unspecified sequence of math symbols\index{math symbol}.

Unlike propositional calculus, no decision procedure\index{decision procedure}
analogous to the truth table method exists (nor theoretically can exist) that
will definitely determine whether a formula is a theorem of predicate
calculus.  Much of the work in the field of automated theorem
proving\index{automated theorem proving} has been dedicated to coming up with
clever heuristics for proving theorems of predicate calculus, but they can
never be guaranteed to work always.

Section \ref{metadefpred} discusses various definitions
that make predicate calculus easier to use.
For example, we define
$\exists x \varphi$ to mean
``there exists at least one possible value of $x$ where $\varphi$ is true.''

We now turn to looking at how predicate calculus can be formally
represented.

\subsubsection{Common Axioms}

There is a new rule of inference in predicate calculus:  if $\varphi$ is
an axiom or a theorem, then $\forall x \,\varphi$ is also a
theorem\index{theorem}.  This is called the rule of
``generalization.''\index{rule of generalization}
This is easily represented in Metamath.

In standard texts of logic, there are often two axioms of predicate
calculus\index{axioms of predicate calculus}:
\begin{center}
  $\forall x \,\varphi ( x ) \rightarrow \varphi ( y )$,
      where ``$y$ is properly substituted for $x$.''\\
  $\forall x ( \varphi \rightarrow \psi )\rightarrow ( \varphi \rightarrow
    \forall x\, \psi )$,
    where ``$x$ is not free in $\varphi$.''
\end{center}

Now at first glance, this seems simple:  just two axioms.  However,
conditional clauses are attached to each axiom describing requirements that
may seem puzzling to you.  In addition, the first axiom puts a variable symbol
in parentheses after each wff, seemingly violating our definition of a
wff\index{well-formed formula (wff)}; this is just an informal way of
referring to some arbitrary variable that may occur in the wff.  The
conditional clauses do, of course, have a precise meaning, but as it turns out
the precise meaning is somewhat complicated and awkward to formalize in a
way that a computer can handle easily.  Unlike propositional calculus, a
certain amount of mathematical sophistication and practice is needed to be
able to easily grasp and manipulate these concepts correctly.

Predicate calculus may be presented with or without axioms for
equality\index{axioms of equality}\index{equality ($=$)}. We will require the
axioms of equality as a prerequisite for the version of set theory we will
use.  The axioms for equality, when included, are often represented using these
two axioms:
\begin{center}
$x=x$\\ \ \\
$x=y\rightarrow (\varphi(x,x)\rightarrow\varphi(x,y))$ where ``$\varphi(x,y)$
   arises from $\varphi(x,x)$ by replacing some, but not necessarily all,
   free\index{free variable}
   occurrences of $x$ by $y$,\\ provided that $y$ is free for $x$
   in $\varphi(x,x)$.'' \end{center}
% (Mendelson p. 95)
The first equality axiom is simple, but again,
the condition on the second one is
somewhat awkward to implement on a computer.

\subsubsection{Tarski System S2}

Of course, we are not the first to notice the complications of these
predicate calculus axioms when being rigorous.

Well-known logician Alfred Tarski published in 1965
a system he called system S2\cite[p.~77]{Tarski1965}.
Tarski's system is \textit{exactly equivalent} to the traditional textbook
formalization, but (by clever use of equality axioms) it eliminates the
latter's primitive notions of ``proper substitution'' and ``free variable,''
replacing them with direct substitution and the notion of a variable
not occurring in a formula (which we express with distinct variable
constraints).

In advocating his system, Tarski wrote, ``The relatively complicated
character of [free variables and proper substitution] is a source
of certain inconveniences of both practical and theoretical nature;
this is clearly experienced both in teaching an elementary course of
mathematical logic and in formalizing the syntax of predicate logic for
some theoretical purposes''\cite[p.~61]{Tarski1965}\index{Tarski, Alfred}.

\subsubsection{Developing a Metamath Representation}

The standard textbook axioms of predicate calculus are somewhat
cumbersome to implement on a computer because of the complex notions of
``free variable''\index{free variable} and ``proper
substitution.''\index{proper substitution}\index{substitution!proper}
While it is possible to use the Metamath\index{Metamath} language to
implement these concepts, we have chosen not to implement them
as primitive constructs in the
\texttt{set.mm} set theory database.  Instead, we have eliminated them
within the axioms
by carefully crafting the axioms so as to avoid them,
building on Tarski's system S2.  This makes it
easy for a beginner to follow the steps in a proof without knowing any
advanced concepts other than the simple concept of
replacing\index{substitution!variable}\index{variable substitution}
variables with expressions.

In order to develop the concepts of free variable and proper
substitution from the axioms, we use an additional
Metamath statement type called ``disjoint variable
restriction''\index{disjoint variables} that we have not encountered
before.  In the context of the axioms, the statement \texttt{\$d} $ x\,
y$\index{\texttt{\$d} statement} simply means that $x$ and $y$ must be
distinct\index{distinct variables}, i.e.\ they may not be simultaneously
substituted\index{substitution!variable}\index{variable substitution}
with the same variable.  The statement \texttt{\$d} $ x\, \varphi$ means
variable $x$ must not occur in wff $\varphi$.  For the precise
definition of \texttt{\$d}, see Section~\ref{dollard}.

\subsubsection{Metamath representation}

The Metamath axiom system for predicate calculus
defined in set.mm uses Tarski's system S2.
As noted above, this has a different representation
than the traditional textbook formalization,
but it is \textit{exactly equivalent} to the textbook formalization,
and it is \textit{much} easier to work with.
This is reproduced as system S3 in Section 6 of
Megill's formalization \cite{Megill}\index{Megill, Norman}.

There is one exception, Tarski's axiom of existence,
which we label as axiom ax-6.
In the case of ax-6, Tarski's version is weaker because it includes a
distinct variable proviso. If we wish, we can also weaken our version
in this way and still have a metalogically complete system. Theorem
ax6 shows this by deriving, in the presence of the other axioms, our
ax-6 from Tarski's weaker version ax6v. However, we chose the stronger
version for our system because it is simpler to state and easier to use.

Tarski's system was designed for proving specific theorems rather than
more general theorem schemes. However, theorem schemes are much more
efficient than specific theorems for building a body of mathematical
knowledge, since they can be reused with different instances as
needed. While Tarski does derive some theorem schemes from his axioms,
their proofs require concepts that are ``outside'' of the system, such as
induction on formula length. The verification of such proofs is difficult
to automate in a proof verifier. (Specifically, Tarski treats the formulas
of his system as set-theoretical objects. In order to verify the proofs
of his theorem schemes, a proof verifier would need a significant amount
of set theory built into it.)

The Metamath axiom system for predicate calculus extends
Tarski's system to eliminate this difficulty. The additional
``auxilliary'' axiom
schemes (as we will call them in this section; see below) endow Tarski's
system with a nice property we call
metalogical completeness \cite[Remark 9.6]{Megill}\index{Megill, Norman}.
As a result, we can prove any theorem scheme
expressable in the ``simple metalogic'' of Tarski's system by using
only Metamath's direct substitution rule applied to the axiom system
(and no other metalogical or set-theoretical notions ``outside'' of the
system). Simple metalogic consists of schemes containing wff metavariables
(with no arguments) and/or set (also called ``individual'') metavariables,
accompanied by optional provisos each stating that two specified set
metavariables must be distinct or that a specified set metavariable may
not occur in a specified wff metavariable. Metamath's logic and set theory
axiom and rule schemes are all examples of simple metalogic. The schemes
of traditional predicate calculus with equality are examples which are
not simple metalogic, because they use wff metavariables with arguments
and have ``free for'' and ``not free in'' side conditions.

A rigorous justification for this system, using an older but
exactly equivalent set of axioms, can be
found in \cite{Megill}\index{Megill, Norman}.

This allows us to
take a different approach in the Metamath\index{Metamath} database
\texttt{set.mm}\index{set theory database (\texttt{set.mm})}.  We do not
directly use the primitive notions of ``free variable''\index{free variable}
and ``proper substitution''\index{proper
substitution}\index{substitution!proper} at all as primitive constructs.
Instead, we use a set
of axioms that are almost as simple to manipulate as those of
propositional calculus.  Our axiom system avoids complex primitive
notions by effectively embedding the complexity into the axioms
themselves.  As a result, we will end up with a larger number of axioms,
but they are ideally suited for a computer language such as Metamath.
(Section~\ref{metaaxioms} shows these axioms.)

We will not elaborate further
on the ``free variable'' and ``proper substitution''
concepts here.  You may consult
\cite[ch.\ 3--4]{Hamilton}\index{Hamilton, Alan G.} (as well as
many other books) for a precise explanation
of these concepts.  If you intend to do serious mathematical work, it is wise
to become familiar with the traditional textbook approach; even though the
concepts embedded in their axioms require a higher level of sophistication,
they can be more practical to deal with on an everyday, informal basis.  Even
if you are just developing Metamath proofs, familiarity with the traditional
approach can help you arrive at a proof outline much faster, which you can
then convert to the detail required by Metamath.

We do develop proper substitution rules later on, but in set.mm
they are defined as derived constructs; they are not primitives.

You should also note that our system of predicate calculus is specifically
tailored for set theory; thus there are only two specific predicates $=$ and
$\in$ and no functions\index{function!in predicate calculus}
or constants\index{constant!in predicate calculus} unlike more general systems.
We later add these.

\subsection{Set Theory}

Traditional Zermelo--Fraenkel set theory\index{Zermelo--Fraenkel set
theory}\index{set theory} with the Axiom of Choice
has 10 axioms, which can be expressed in the
language of predicate calculus.  In this section, we will list only the
names and brief English descriptions of these axioms, since we will give
you the precise formulas used by the Metamath\index{Metamath} set theory
database \texttt{set.mm} later on.

In the descriptions of the axioms, we assume that $x$, $y$, $z$, $w$, and $v$
represent sets.  These are the same as the variables\index{variable!in set
theory} in our predicate calculus system above, except that now we informally
think of the variables as ranging over sets.  Note that the terms
``object,''\index{object} ``set,''\index{set} ``element,''\index{element}
``collection,''\index{collection} and ``family''\index{family} are synonymous,
as are ``is an element of,'' ``is a member of,''\index{member} ``is contained
in,'' and ``belongs to.''  The different terms are used for convenience; for
example, ``a collection of sets'' is less confusing than ``a set of sets.''
A set $x$ is said to be a ``subset''\index{subset} of $y$ if every element of
$x$ is also an element of $y$; we also say $x$ is ``included in''
$y$.

The axioms are very general and apply to almost any conceivable mathematical
object, and this level of abstraction can be overwhelming at first.  To gain an
intuitive feel, it can be helpful to draw a picture illustrating the concept;
for example, a circle containing dots could represent a collection of sets,
and a smaller circle drawn inside the circle could represent a subset.
Overlapping circles can illustrate intersection and union.  Circles that
illustrate the concepts of set theory are frequently used in elementary
textbooks and are called Venn diagrams\index{Venn diagram}.\index{axioms of
set theory}

1. Axiom of Extensionality:  Two sets are identical if they contain the same
   elements.\index{Axiom of Extensionality}

2. Axiom of Pairing:  The set $\{ x , y \}$ exists.\index{Axiom of Pairing}

3. Axiom of Power Sets:  The power set of a set (the collection of all of
   its subsets) exists.  For example, the power set of $\{x,y\}$ is
   $\{\varnothing,\{x\},\{y\},\{x,y\}\}$ and it exists.\index{Axiom
of Power Sets}

4. Axiom of the Null Set:  The empty set $\varnothing$ exists.\index{Axiom of
the Null Set}

5. Axiom of Union:  The union of a set (the set containing the elements of
   its members) exists.  For example, the union of $\{\{x,y\},\{z\}\}$ is
 $\{x,y,z\}$ and
   it exists.\index{Axiom of Union}

6. Axiom of Regularity:  Roughly, no set can contain itself, nor can there
   be membership ``loops,'' such as a set being an
   element of one of its members.\index{Axiom of Regularity}

7. Axiom of Infinity:  An infinite set exists.  An example of an infinite
   set is the set of all
   integers.\index{Axiom of Infinity}

8. Axiom of Separation:  The set exists that is obtained by restricting $x$
   with some property.  For example, if the set of all integers exists,
   then the set of all even integers exists.\index{Axiom of Separation}

9. Axiom of Replacement:  The range of a function whose domain is restricted
   to the elements of a set $x$, is also a set.  For example, there
   is a function
   from integers (the function's domain) to their squares (its
   range).  If we
   restrict the domain to even integers, its range will become the set of
   squares of even integers, so this axiom asserts that the set of
    squares of even numbers exists.  Technical note:  In general, the
   ``function'' need not be a set but can be a proper class.
   \index{Axiom of Replacement}

10. Axiom of Choice:  Let $x$ be a set whose members are pairwise
  disjoint\index{disjoint sets} (i.e,
  whose members contain no elements in common).  Then there exists another
  set containing one element from each member of $x$.  For
  example, if $x$ is
  $\{\{y,z\},\{w,v\}\}$, where $y$, $z$, $w$, and $v$ are
  different sets, then a set such as $\{z,w\}$
  exists (but the axiom doesn't tell
  us which one).  (Actually the Axiom
  of Choice is redundant if the set $x$, as in this example, has a finite
  number of elements.)\index{Axiom of Choice}

The Axiom of Choice is usually considered an extension of ZF set theory rather
than a proper part of it.  It is sometimes considered philosophically
controversial because it specifies the existence of a set without specifying
what the set is. Constructive logics, including intuitionistic logic,
do not accept the axiom of choice.
Since there is some lingering controversy, we often prefer proofs that do
not use the axiom of choice (where there is a known alternative), and
in some cases we will use weaker axioms than the full axiom of choice.
That said, the axiom of choice is a powerful and widely-accepted tool,
so we do use it when needed.
ZF set theory that includes the Axiom of Choice is
called Zermelo--Fraenkel set theory with choice (ZFC\index{ZFC set theory}).

When expressed symbolically, the Axiom of Separation and the Axiom of
Replacement contain wff symbols and therefore each represent infinitely many
axioms, one for each possible wff. For this reason, they are often called
axiom schemes\index{axiom scheme}\index{well-formed formula (wff)}.

It turns out that the Axiom of the Null Set, the Axiom of Pairing, and the
Axiom of Separation can be derived from the other axioms and are therefore
unnecessary, although they tend to be included in standard texts for various
reasons (historical, philosophical, and possibly because some authors may not
know this).  In the Metamath\index{Metamath} set theory database, these
redundant axioms are derived from the other ones instead of truly
being considered axioms.
This is in keeping with our general goal of minimizing the number of
axioms we must depend on.

\subsection{Other Axioms}

Above we qualified the phrase ``all of mathematics'' with ``essentially.''
The main important missing piece is the ability to do category theory,
which requires huge sets (inaccessible cardinals) larger than those
postulated by the ZFC axioms. The Tarski--Grothendieck Axiom postulates
the existence of such sets.
Note that this is the same axiom used by Mizar for supporting
category theory.
The Tarski--Grothendieck axiom
can be viewed as a very strong replacement of the Axiom of Infinity,
the Axiom of Choice, and the Axiom of Power Sets.
The \texttt{set.mm} database includes this axiom; see the database
for details about it.
Again, we only use this axiom when we need to.
You are only likely to encounter or use this axiom if you are doing
category theory, since its use is highly specialized,
so we will not list the Tarsky-Grothendieck axiom
in the short list of axioms below.

Can there be even more axioms?
Of course.
G\"{o}del showed that no finite set of axioms or axiom schemes can completely
describe any consistent theory strong enough to include arithmetic.
But practically speaking, the ones above are the accepted foundation that
almost all mathematicians explicitly or implicitly base their work on.

\section{The Axioms in the Metamath Language}\label{metaaxioms}

Here we list the axioms as they appear in
\texttt{set.mm}\index{set theory database (\texttt{set.mm})} so you can
look them up there easily.  Incidentally, the \texttt{show statement
/tex} command\index{\texttt{show statement} command} was used to
typeset them.

%macros from show statement /tex
\newbox\mlinebox
\newbox\mtrialbox
\newbox\startprefix  % Prefix for first line of a formula
\newbox\contprefix  % Prefix for continuation line of a formula
\def\startm{  % Initialize formula line
  \setbox\mlinebox=\hbox{\unhcopy\startprefix}
}
\def\m#1{  % Add a symbol to the formula
  \setbox\mtrialbox=\hbox{\unhcopy\mlinebox $\,#1$}
  \ifdim\wd\mtrialbox>\hsize
    \box\mlinebox
    \setbox\mlinebox=\hbox{\unhcopy\contprefix $\,#1$}
  \else
    \setbox\mlinebox=\hbox{\unhbox\mtrialbox}
  \fi
}
\def\endm{  % Output the last line of a formula
  \box\mlinebox
}

% \SLASH for \ , \TOR for \/ (text OR), \TAND for /\ (text and)
% This embeds a following forced space to force the space.
\newcommand\SLASH{\char`\\~}
\newcommand\TOR{\char`\\/~}
\newcommand\TAND{/\char`\\~}
%
% Macro to output metamath raw text.
% This assumes \startprefix and \contprefix are set.
% NOTE: "\" is tricky to escape, use \SLASH, \TOR, and \TAND inside.
% Any use of "$ { ~ ^" must be escaped; ~ and ^ must be escaped specially.
% We escape { and } for consistency.
% For more about how this macro written, see:
% https://stackoverflow.com/questions/4073674/
% how-to-disable-indentation-in-particular-section-in-latex/4075706
% Use frenchspacing, or "e." will get an extra space after it.
\newlength\mystoreparindent
\newlength\mystorehangindent
\newenvironment{mmraw}{%
\setlength{\mystoreparindent}{\the\parindent}
\setlength{\mystorehangindent}{\the\hangindent}
\setlength{\parindent}{0pt} % TODO - we'll put in the \startprefix instead
\setlength{\hangindent}{\wd\the\contprefix}
\begin{flushleft}
\begin{frenchspacing}
\begin{tt}
{\unhcopy\startprefix}%
}{%
\end{tt}
\end{frenchspacing}
\end{flushleft}
\setlength{\parindent}{\mystoreparindent}
\setlength{\hangindent}{\mystorehangindent}
\vskip 1ex
}

\needspace{5\baselineskip}
\subsection{Propositional Calculus}\label{propcalc}\index{axioms of
propositional calculus}

\needspace{2\baselineskip}
Axiom of Simplification.\label{ax1}

\setbox\startprefix=\hbox{\tt \ \ ax-1\ \$a\ }
\setbox\contprefix=\hbox{\tt \ \ \ \ \ \ \ \ \ \ }
\startm
\m{\vdash}\m{(}\m{\varphi}\m{\rightarrow}\m{(}\m{\psi}\m{\rightarrow}\m{\varphi}\m{)}
\m{)}
\endm

\needspace{3\baselineskip}
\noindent Axiom of Distribution.

\setbox\startprefix=\hbox{\tt \ \ ax-2\ \$a\ }
\setbox\contprefix=\hbox{\tt \ \ \ \ \ \ \ \ \ \ }
\startm
\m{\vdash}\m{(}\m{(}\m{\varphi}\m{\rightarrow}\m{(}\m{\psi}\m{\rightarrow}\m{\chi}
\m{)}\m{)}\m{\rightarrow}\m{(}\m{(}\m{\varphi}\m{\rightarrow}\m{\psi}\m{)}\m{
\rightarrow}\m{(}\m{\varphi}\m{\rightarrow}\m{\chi}\m{)}\m{)}\m{)}
\endm

\needspace{2\baselineskip}
\noindent Axiom of Contraposition.

\setbox\startprefix=\hbox{\tt \ \ ax-3\ \$a\ }
\setbox\contprefix=\hbox{\tt \ \ \ \ \ \ \ \ \ \ }
\startm
\m{\vdash}\m{(}\m{(}\m{\lnot}\m{\varphi}\m{\rightarrow}\m{\lnot}\m{\psi}\m{)}\m{
\rightarrow}\m{(}\m{\psi}\m{\rightarrow}\m{\varphi}\m{)}\m{)}
\endm


\needspace{4\baselineskip}
\noindent Rule of Modus Ponens.\label{axmp}\index{modus ponens}

\setbox\startprefix=\hbox{\tt \ \ min\ \$e\ }
\setbox\contprefix=\hbox{\tt \ \ \ \ \ \ \ \ \ }
\startm
\m{\vdash}\m{\varphi}
\endm

\setbox\startprefix=\hbox{\tt \ \ maj\ \$e\ }
\setbox\contprefix=\hbox{\tt \ \ \ \ \ \ \ \ \ }
\startm
\m{\vdash}\m{(}\m{\varphi}\m{\rightarrow}\m{\psi}\m{)}
\endm

\setbox\startprefix=\hbox{\tt \ \ ax-mp\ \$a\ }
\setbox\contprefix=\hbox{\tt \ \ \ \ \ \ \ \ \ \ \ }
\startm
\m{\vdash}\m{\psi}
\endm


\needspace{7\baselineskip}
\subsection{Axioms of Predicate Calculus with Equality---Tarski's S2}\index{axioms of predicate calculus}

\needspace{3\baselineskip}
\noindent Rule of Generalization.\index{rule of generalization}

\setbox\startprefix=\hbox{\tt \ \ ax-g.1\ \$e\ }
\setbox\contprefix=\hbox{\tt \ \ \ \ \ \ \ \ \ \ \ \ }
\startm
\m{\vdash}\m{\varphi}
\endm

\setbox\startprefix=\hbox{\tt \ \ ax-gen\ \$a\ }
\setbox\contprefix=\hbox{\tt \ \ \ \ \ \ \ \ \ \ \ \ }
\startm
\m{\vdash}\m{\forall}\m{x}\m{\varphi}
\endm

\needspace{2\baselineskip}
\noindent Axiom of Quantified Implication.

\setbox\startprefix=\hbox{\tt \ \ ax-4\ \$a\ }
\setbox\contprefix=\hbox{\tt \ \ \ \ \ \ \ \ \ \ }
\startm
\m{\vdash}\m{(}\m{\forall}\m{x}\m{(}\m{\forall}\m{x}\m{\varphi}\m{\rightarrow}\m{
\psi}\m{)}\m{\rightarrow}\m{(}\m{\forall}\m{x}\m{\varphi}\m{\rightarrow}\m{
\forall}\m{x}\m{\psi}\m{)}\m{)}
\endm

\needspace{3\baselineskip}
\noindent Axiom of Distinctness.

% Aka: Add $d x ph $.
\setbox\startprefix=\hbox{\tt \ \ ax-5\ \$a\ }
\setbox\contprefix=\hbox{\tt \ \ \ \ \ \ \ \ \ \ }
\startm
\m{\vdash}\m{(}\m{\varphi}\m{\rightarrow}\m{\forall}\m{x}\m{\varphi}\m{)}\m{where}\m{ }\m{\$d}\m{ }\m{x}\m{ }\m{\varphi}\m{ }\m{(}\m{x}\m{ }\m{does}\m{ }\m{not}\m{ }\m{occur}\m{ }\m{in}\m{ }\m{\varphi}\m{)}
\endm

\needspace{2\baselineskip}
\noindent Axiom of Existence.

\setbox\startprefix=\hbox{\tt \ \ ax-6\ \$a\ }
\setbox\contprefix=\hbox{\tt \ \ \ \ \ \ \ \ \ \ }
\startm
\m{\vdash}\m{(}\m{\forall}\m{x}\m{(}\m{x}\m{=}\m{y}\m{\rightarrow}\m{\forall}
\m{x}\m{\varphi}\m{)}\m{\rightarrow}\m{\varphi}\m{)}
\endm

\needspace{2\baselineskip}
\noindent Axiom of Equality.

\setbox\startprefix=\hbox{\tt \ \ ax-7\ \$a\ }
\setbox\contprefix=\hbox{\tt \ \ \ \ \ \ \ \ \ \ }
\startm
\m{\vdash}\m{(}\m{x}\m{=}\m{y}\m{\rightarrow}\m{(}\m{x}\m{=}\m{z}\m{
\rightarrow}\m{y}\m{=}\m{z}\m{)}\m{)}
\endm

\needspace{2\baselineskip}
\noindent Axiom of Left Equality for Binary Predicate.

\setbox\startprefix=\hbox{\tt \ \ ax-8\ \$a\ }
\setbox\contprefix=\hbox{\tt \ \ \ \ \ \ \ \ \ \ \ }
\startm
\m{\vdash}\m{(}\m{x}\m{=}\m{y}\m{\rightarrow}\m{(}\m{x}\m{\in}\m{z}\m{
\rightarrow}\m{y}\m{\in}\m{z}\m{)}\m{)}
\endm

\needspace{2\baselineskip}
\noindent Axiom of Right Equality for Binary Predicate.

\setbox\startprefix=\hbox{\tt \ \ ax-9\ \$a\ }
\setbox\contprefix=\hbox{\tt \ \ \ \ \ \ \ \ \ \ \ }
\startm
\m{\vdash}\m{(}\m{x}\m{=}\m{y}\m{\rightarrow}\m{(}\m{z}\m{\in}\m{x}\m{
\rightarrow}\m{z}\m{\in}\m{y}\m{)}\m{)}
\endm


\needspace{4\baselineskip}
\subsection{Axioms of Predicate Calculus with Equality---Auxiliary}\index{axioms of predicate calculus - auxiliary}

\needspace{2\baselineskip}
\noindent Axiom of Quantified Negation.

\setbox\startprefix=\hbox{\tt \ \ ax-10\ \$a\ }
\setbox\contprefix=\hbox{\tt \ \ \ \ \ \ \ \ \ \ }
\startm
\m{\vdash}\m{(}\m{\lnot}\m{\forall}\m{x}\m{\lnot}\m{\forall}\m{x}\m{\varphi}\m{
\rightarrow}\m{\varphi}\m{)}
\endm

\needspace{2\baselineskip}
\noindent Axiom of Quantifier Commutation.

\setbox\startprefix=\hbox{\tt \ \ ax-11\ \$a\ }
\setbox\contprefix=\hbox{\tt \ \ \ \ \ \ \ \ \ \ }
\startm
\m{\vdash}\m{(}\m{\forall}\m{x}\m{\forall}\m{y}\m{\varphi}\m{\rightarrow}\m{
\forall}\m{y}\m{\forall}\m{x}\m{\varphi}\m{)}
\endm

\needspace{3\baselineskip}
\noindent Axiom of Substitution.

\setbox\startprefix=\hbox{\tt \ \ ax-12\ \$a\ }
\setbox\contprefix=\hbox{\tt \ \ \ \ \ \ \ \ \ \ \ }
\startm
\m{\vdash}\m{(}\m{\lnot}\m{\forall}\m{x}\m{\,x}\m{=}\m{y}\m{\rightarrow}\m{(}
\m{x}\m{=}\m{y}\m{\rightarrow}\m{(}\m{\varphi}\m{\rightarrow}\m{\forall}\m{x}\m{(}
\m{x}\m{=}\m{y}\m{\rightarrow}\m{\varphi}\m{)}\m{)}\m{)}\m{)}
\endm

\needspace{3\baselineskip}
\noindent Axiom of Quantified Equality.

\setbox\startprefix=\hbox{\tt \ \ ax-13\ \$a\ }
\setbox\contprefix=\hbox{\tt \ \ \ \ \ \ \ \ \ \ \ }
\startm
\m{\vdash}\m{(}\m{\lnot}\m{\forall}\m{z}\m{\,z}\m{=}\m{x}\m{\rightarrow}\m{(}
\m{\lnot}\m{\forall}\m{z}\m{\,z}\m{=}\m{y}\m{\rightarrow}\m{(}\m{x}\m{=}\m{y}
\m{\rightarrow}\m{\forall}\m{z}\m{\,x}\m{=}\m{y}\m{)}\m{)}\m{)}
\endm

% \noindent Axiom of Quantifier Substitution
%
% \setbox\startprefix=\hbox{\tt \ \ ax-c11n\ \$a\ }
% \setbox\contprefix=\hbox{\tt \ \ \ \ \ \ \ \ \ \ \ }
% \startm
% \m{\vdash}\m{(}\m{\forall}\m{x}\m{\,x}\m{=}\m{y}\m{\rightarrow}\m{(}\m{\forall}
% \m{x}\m{\varphi}\m{\rightarrow}\m{\forall}\m{y}\m{\varphi}\m{)}\m{)}
% \endm
%
% \noindent Axiom of Distinct Variables. (This axiom requires
% that two individual variables
% be distinct\index{\texttt{\$d} statement}\index{distinct
% variables}.)
%
% \setbox\startprefix=\hbox{\tt \ \ \ \ \ \ \ \ \$d\ }
% \setbox\contprefix=\hbox{\tt \ \ \ \ \ \ \ \ \ \ \ }
% \startm
% \m{x}\m{\,}\m{y}
% \endm
%
% \setbox\startprefix=\hbox{\tt \ \ ax-c16\ \$a\ }
% \setbox\contprefix=\hbox{\tt \ \ \ \ \ \ \ \ \ \ \ }
% \startm
% \m{\vdash}\m{(}\m{\forall}\m{x}\m{\,x}\m{=}\m{y}\m{\rightarrow}\m{(}\m{\varphi}\m{
% \rightarrow}\m{\forall}\m{x}\m{\varphi}\m{)}\m{)}
% \endm

% \noindent Axiom of Quantifier Introduction (2).  (This axiom requires
% that the individual variable not occur in the
% wff\index{\texttt{\$d} statement}\index{distinct variables}.)
%
% \setbox\startprefix=\hbox{\tt \ \ \ \ \ \ \ \ \$d\ }
% \setbox\contprefix=\hbox{\tt \ \ \ \ \ \ \ \ \ \ \ }
% \startm
% \m{x}\m{\,}\m{\varphi}
% \endm
% \setbox\startprefix=\hbox{\tt \ \ ax-5\ \$a\ }
% \setbox\contprefix=\hbox{\tt \ \ \ \ \ \ \ \ \ \ \ }
% \startm
% \m{\vdash}\m{(}\m{\varphi}\m{\rightarrow}\m{\forall}\m{x}\m{\varphi}\m{)}
% \endm

\subsection{Set Theory}\label{mmsettheoryaxioms}

In order to make the axioms of set theory\index{axioms of set theory} a little
more compact, there are several definitions from logic that we make use of
implicitly, namely, ``logical {\sc and},''\index{conjunction ($\wedge$)}
\index{logical {\sc and} ($\wedge$)} ``logical equivalence,''\index{logical
equivalence ($\leftrightarrow$)}\index{biconditional ($\leftrightarrow$)} and
``there exists.''\index{existential quantifier ($\exists$)}

\begin{center}\begin{tabular}{rcl}
  $( \varphi \wedge \psi )$ &\mbox{stands for}& $\neg ( \varphi
     \rightarrow \neg \psi )$\\
  $( \varphi \leftrightarrow \psi )$& \mbox{stands
     for}& $( ( \varphi \rightarrow \psi ) \wedge
     ( \psi \rightarrow \varphi ) )$\\
  $\exists x \,\varphi$ &\mbox{stands for}& $\neg \forall x \neg \varphi$
\end{tabular}\end{center}

In addition, the axioms of set theory require that all variables be
dis\-tinct,\index{distinct variables}\footnote{Set theory axioms can be
devised so that {\em no} variables are required to be distinct,
provided we replace \texttt{ax-c16} with an axiom stating that ``at
least two things exist,'' thus
making \texttt{ax-5} the only other axiom requiring the
\texttt{\$d} statement.  These axioms are unconventional and are not
presented here, but they can be found on the \url{http://metamath.org}
web site.  See also the Comment on
p.~\pageref{nodd}.}\index{\texttt{\$d} statement} thus we also assume:
\begin{center}
  \texttt{\$d }$x\,y\,z\,w$
\end{center}

\needspace{2\baselineskip}
\noindent Axiom of Extensionality.\index{Axiom of Extensionality}

\setbox\startprefix=\hbox{\tt \ \ ax-ext\ \$a\ }
\setbox\contprefix=\hbox{\tt \ \ \ \ \ \ \ \ \ \ \ \ }
\startm
\m{\vdash}\m{(}\m{\forall}\m{x}\m{(}\m{x}\m{\in}\m{y}\m{\leftrightarrow}\m{x}
\m{\in}\m{z}\m{)}\m{\rightarrow}\m{y}\m{=}\m{z}\m{)}
\endm

\needspace{3\baselineskip}
\noindent Axiom of Replacement.\index{Axiom of Replacement}

\setbox\startprefix=\hbox{\tt \ \ ax-rep\ \$a\ }
\setbox\contprefix=\hbox{\tt \ \ \ \ \ \ \ \ \ \ \ \ }
\startm
\m{\vdash}\m{(}\m{\forall}\m{w}\m{\exists}\m{y}\m{\forall}\m{z}\m{(}\m{%
\forall}\m{y}\m{\varphi}\m{\rightarrow}\m{z}\m{=}\m{y}\m{)}\m{\rightarrow}\m{%
\exists}\m{y}\m{\forall}\m{z}\m{(}\m{z}\m{\in}\m{y}\m{\leftrightarrow}\m{%
\exists}\m{w}\m{(}\m{w}\m{\in}\m{x}\m{\wedge}\m{\forall}\m{y}\m{\varphi}\m{)}%
\m{)}\m{)}
\endm

\needspace{2\baselineskip}
\noindent Axiom of Union.\index{Axiom of Union}

\setbox\startprefix=\hbox{\tt \ \ ax-un\ \$a\ }
\setbox\contprefix=\hbox{\tt \ \ \ \ \ \ \ \ \ \ \ }
\startm
\m{\vdash}\m{\exists}\m{x}\m{\forall}\m{y}\m{(}\m{\exists}\m{x}\m{(}\m{y}\m{
\in}\m{x}\m{\wedge}\m{x}\m{\in}\m{z}\m{)}\m{\rightarrow}\m{y}\m{\in}\m{x}\m{)}
\endm

\needspace{2\baselineskip}
\noindent Axiom of Power Sets.\index{Axiom of Power Sets}

\setbox\startprefix=\hbox{\tt \ \ ax-pow\ \$a\ }
\setbox\contprefix=\hbox{\tt \ \ \ \ \ \ \ \ \ \ \ \ }
\startm
\m{\vdash}\m{\exists}\m{x}\m{\forall}\m{y}\m{(}\m{\forall}\m{x}\m{(}\m{x}\m{
\in}\m{y}\m{\rightarrow}\m{x}\m{\in}\m{z}\m{)}\m{\rightarrow}\m{y}\m{\in}\m{x}
\m{)}
\endm

\needspace{3\baselineskip}
\noindent Axiom of Regularity.\index{Axiom of Regularity}

\setbox\startprefix=\hbox{\tt \ \ ax-reg\ \$a\ }
\setbox\contprefix=\hbox{\tt \ \ \ \ \ \ \ \ \ \ \ \ }
\startm
\m{\vdash}\m{(}\m{\exists}\m{x}\m{\,x}\m{\in}\m{y}\m{\rightarrow}\m{\exists}
\m{x}\m{(}\m{x}\m{\in}\m{y}\m{\wedge}\m{\forall}\m{z}\m{(}\m{z}\m{\in}\m{x}\m{
\rightarrow}\m{\lnot}\m{z}\m{\in}\m{y}\m{)}\m{)}\m{)}
\endm

\needspace{3\baselineskip}
\noindent Axiom of Infinity.\index{Axiom of Infinity}

\setbox\startprefix=\hbox{\tt \ \ ax-inf\ \$a\ }
\setbox\contprefix=\hbox{\tt \ \ \ \ \ \ \ \ \ \ \ \ \ \ \ }
\startm
\m{\vdash}\m{\exists}\m{x}\m{(}\m{y}\m{\in}\m{x}\m{\wedge}\m{\forall}\m{y}%
\m{(}\m{y}\m{\in}\m{x}\m{\rightarrow}\m{\exists}\m{z}\m{(}\m{y}\m{\in}\m{z}\m{%
\wedge}\m{z}\m{\in}\m{x}\m{)}\m{)}\m{)}
\endm

\needspace{4\baselineskip}
\noindent Axiom of Choice.\index{Axiom of Choice}

\setbox\startprefix=\hbox{\tt \ \ ax-ac\ \$a\ }
\setbox\contprefix=\hbox{\tt \ \ \ \ \ \ \ \ \ \ \ \ \ \ }
\startm
\m{\vdash}\m{\exists}\m{x}\m{\forall}\m{y}\m{\forall}\m{z}\m{(}\m{(}\m{y}\m{%
\in}\m{z}\m{\wedge}\m{z}\m{\in}\m{w}\m{)}\m{\rightarrow}\m{\exists}\m{w}\m{%
\forall}\m{y}\m{(}\m{\exists}\m{w}\m{(}\m{(}\m{y}\m{\in}\m{z}\m{\wedge}\m{z}%
\m{\in}\m{w}\m{)}\m{\wedge}\m{(}\m{y}\m{\in}\m{w}\m{\wedge}\m{w}\m{\in}\m{x}%
\m{)}\m{)}\m{\leftrightarrow}\m{y}\m{=}\m{w}\m{)}\m{)}
\endm

\subsection{That's It}

There you have it, the axioms for (essentially) all of mathematics!
Wonder at them and stare at them in awe.  Put a copy in your wallet, and
you will carry in your pocket the encoding for all theorems ever proved
and that ever will be proved, from the most mundane to the most
profound.

\section{A Hierarchy of Definitions}\label{hierarchy}

The axioms in the previous section in principle embody everything that can be
done within standard mathematics.  However, it is impractical to accomplish
very much by using them directly, for even simple concepts (from a human
perspective) can involve extremely long, incomprehensible formulas.
Mathematics is made practical by introducing definitions\index{definition}.
Definitions usually introduce new symbols, or at least new relationships among
existing symbols, to abbreviate more complex formulas.  An important
requirement for a definition is that there exist a straightforward
(algorithmic) method for eliminating the abbreviation by expanding it into the
more primitive symbol string that it represents.  Some
important definitions included in
the file \texttt{set.mm} are listed in this section for reference, and also to
give you a feel for why something like $\omega$\index{omega ($\omega$)} (the
set of natural numbers\index{natural number} 0, 1, 2,\ldots) becomes very
complicated when completely expanded into primitive symbols.

What is the motivation for definitions, aside from allowing complicated
expressions to be expressed more simply?  In the case of  $\omega$, one goal is
to provide a basis for the theory of natural numbers.\index{natural number}
Before set theory was invented, a set of axioms for arithmetic, called Peano's
postulates\index{Peano's postulates}, was devised and shown to have the
properties one expects for natural numbers.  Now anyone can postulate a
set of axioms, but if the axioms are inconsistent contradictions can be derived
from them.  Once a contradiction is derived, anything can be trivially
proved, including
all the facts of arithmetic and their negations.  To ensure that an
axiom system is at least as reliable as the axioms for set theory, we can
define sets and operations on those sets that satisfy the new axioms. In the
\texttt{set.mm} Metamath database, we prove that the elements of $\omega$ satisfy
Peano's postulates, and it's a long and hard journey to get there directly
from the axioms of set theory.  But the result is confidence in the
foundations of arithmetic.  And there is another advantage:  we now have all
the tools of set theory at our disposal for manipulating objects that obey the
axioms for arithmetic.

What are the criteria we use for definitions?  First, and of utmost importance,
the definition should not be {\em creative}\index{creative
definition}\index{definition!creative}, that
is it should not allow an expression that previously qualified as a wff but
was not provable, to become provable.   Second, the definition should be {\em
eliminable}\index{definition!eliminability}, that is, there should exist an
algorithmic method for converting any expression using the definition into
a logically equivalent expression that previously qualified as a wff.

In almost all cases below, definitions connect two expressions with either
$\leftrightarrow$ or $=$.  Eliminating\footnote{Here we mean the
elimination that a human might do in his or her head.  To eliminate them as
part of a Metamath proof we would invoke one of a number of
theorems that deal with transitivity of equivalence or equality; there are
many such examples in the proofs in \texttt{set.mm}.} such a definition is a
simple matter of substituting the expression on the left-hand side ({\em
definiendum}\index{definiendum} or thing being defined) with the equivalent,
more primitive expression on the right-hand side ({\em
definiens}\index{definiens} or definition).

Often a definition has variables on the right-hand side which do not appear on
the left-hand side; these are called {\em dummy variables}.\index{dummy
variable!in definitions}  In this case, any
allowable substitution (such as a new, distinct
variable) can be used when the definition is eliminated.  Dummy variables may
be used only if they are {\em effectively bound}\index{effectively bound
variable}, meaning that the definition will remain logically equivalent upon
any substitution of a dummy variable with any other {\em qualifying
expression}\index{qualifying expression}, i.e.\ any symbol string (such as
another variable) that
meets the restrictions on the dummy variable imposed by \texttt{\$d} and
\texttt{\$f} statements.  For example, we could define a constant $\perp$
(inverted tee, meaning logical ``false'') as $( \varphi \wedge \lnot \varphi
)$, i.e.\ ``phi and not phi.''  Here $\varphi$ is effectively bound because the
definition remains logically equivalent when we replace $\varphi$ with any
other wff.  (It is actually \texttt{df-fal}
in \texttt{set.mm}, which defines $\perp$.)

There are two cases where eliminating definitions is a little more
complex.  These cases are the definitions \texttt{df-bi} and
\texttt{df-cleq}.  The first stretches the concept of a definition a
little, as in effect it ``defines a definition;'' however, it meets our
requirements for a definition in that it is eliminable and does not
strengthen the language.  Theorem \texttt{bii} shows the substitution
needed to eliminate the $\leftrightarrow$\index{logical equivalence
($\leftrightarrow$)}\index{biconditional ($\leftrightarrow$)} symbol.

Definition \texttt{df-cleq}\index{equality ($=$)} extends the usage of
the equality symbol to include ``classes''\index{class} in set theory.  The
reason it is potentially problematic is that it can lead to statements which
do not follow from logic alone but presuppose the Axiom of
Extensionality\index{Axiom of Extensionality}, so we include this axiom
as a hypothesis for the definition.  We could have made \texttt{df-cleq} directly
eliminable by introducing a new equality symbol, but have chosen not to do so
in keeping with standard textbook practice.  Definitions such as \texttt{df-cleq}
that extend the meaning of existing symbols must be introduced carefully so
that they do not lead to contradictions.  Definition \texttt{df-clel} also
extends the meaning of an existing symbol ($\in$); while it doesn't strengthen
the language like \texttt{df-cleq}, this is not obvious and it must also be
subject to the same scrutiny.

Exercise:  Study how the wff $x\in\omega$, meaning ``$x$ is a natural
number,'' could be expanded in terms of primitive symbols, starting with the
definitions \texttt{df-clel} on p.~\pageref{dfclel} and \texttt{df-om} on
p.~\pageref{dfom} and working your way back.  Don't bother to work out the
details; just make sure that you understand how you could do it in principle.
The answer is shown in the footnote on p.~\pageref{expandom}.  If you
actually do work it out, you won't get exactly the same answer because we used
a few simplifications such as discarding occurrences of $\lnot\lnot$ (double
negation).

In the definitions below, we have placed the {\sc ascii} Metamath source
below each of the formulas to help you become familiar with the
notation in the database.  For simplicity, the necessary \texttt{\$f}
and \texttt{\$d} statements are not shown.  If you are in doubt, use the
\texttt{show statement}\index{\texttt{show statement} command} command
in the Metamath program to see the full statement.
A selection of this notation is summarized in Appendix~\ref{ASCII}.

To understand the motivation for these definitions, you should consult the
references indicated:  Takeuti and Zaring \cite{Takeuti}\index{Takeuti, Gaisi},
Quine \cite{Quine}\index{Quine, Willard Van Orman}, Bell and Machover
\cite{Bell}\index{Bell, J. L.}, and Enderton \cite{Enderton}\index{Enderton,
Herbert B.}.  Our list of definitions is provided more for reference than as a
learning aid.  However, by looking at a few of them you can gain a feel for
how the hierarchy is built up.  The definitions are a representative sample of
the many definitions
in \texttt{set.mm}, but they are complete with respect to the
theorem examples we will present in Section~\ref{sometheorems}.  Also, some are
slightly different from, but logically equivalent to, the ones in \texttt{set.mm}
(some of which have been revised over time to shorten them, for example).

\subsection{Definitions for Propositional Calculus}\label{metadefprop}

The symbols $\varphi$, $\psi$, and $\chi$ represent wffs.

Our first definition introduces the biconditional
connective\footnote{The term ``connective'' is informally used to mean a
symbol that is placed between two variables or adjacent to a variable,
whereas a mathematical ``constant'' usually indicates a symbol such as
the number 0 that may replace a variable or metavariable.  From
Metamath's point of view, there is no distinction between a connective
and a constant; both are constants in the Metamath
language.}\index{connective}\index{constant} (also called logical
equivalence)\index{logical equivalence
($\leftrightarrow$)}\index{biconditional ($\leftrightarrow$)}.  Unlike
most traditional developments, we have chosen not to have a separate
symbol such as ``Df.'' to mean ``is defined as.''  Instead, we will use
the biconditional connective for this purpose, as it lets us use
logic to manipulate definitions directly.  Here we state the properties
of the biconditional connective with a carefully crafted \texttt{\$a}
statement, which effectively uses the biconditional connective to define
itself.  The $\leftrightarrow$ symbol can be eliminated from a formula
using theorem \texttt{bii}, which is derived later.

\vskip 2ex
\noindent Define the biconditional connective.\label{df-bi}

\vskip 0.5ex
\setbox\startprefix=\hbox{\tt \ \ df-bi\ \$a\ }
\setbox\contprefix=\hbox{\tt \ \ \ \ \ \ \ \ \ \ \ }
\startm
\m{\vdash}\m{\lnot}\m{(}\m{(}\m{(}\m{\varphi}\m{\leftrightarrow}\m{\psi}\m{)}%
\m{\rightarrow}\m{\lnot}\m{(}\m{(}\m{\varphi}\m{\rightarrow}\m{\psi}\m{)}\m{%
\rightarrow}\m{\lnot}\m{(}\m{\psi}\m{\rightarrow}\m{\varphi}\m{)}\m{)}\m{)}\m{%
\rightarrow}\m{\lnot}\m{(}\m{\lnot}\m{(}\m{(}\m{\varphi}\m{\rightarrow}\m{%
\psi}\m{)}\m{\rightarrow}\m{\lnot}\m{(}\m{\psi}\m{\rightarrow}\m{\varphi}\m{)}%
\m{)}\m{\rightarrow}\m{(}\m{\varphi}\m{\leftrightarrow}\m{\psi}\m{)}\m{)}\m{)}
\endm
\begin{mmraw}%
|- -. ( ( ( ph <-> ps ) -> -. ( ( ph -> ps ) ->
-. ( ps -> ph ) ) ) -> -. ( -. ( ( ph -> ps ) -> -. (
ps -> ph ) ) -> ( ph <-> ps ) ) ) \$.
\end{mmraw}

\noindent This theorem relates the biconditional connective to primitive
connectives and can be used to eliminate the $\leftrightarrow$ symbol from any
wff.

\vskip 0.5ex
\setbox\startprefix=\hbox{\tt \ \ bii\ \$p\ }
\setbox\contprefix=\hbox{\tt \ \ \ \ \ \ \ \ \ }
\startm
\m{\vdash}\m{(}\m{(}\m{\varphi}\m{\leftrightarrow}\m{\psi}\m{)}\m{\leftrightarrow}
\m{\lnot}\m{(}\m{(}\m{\varphi}\m{\rightarrow}\m{\psi}\m{)}\m{\rightarrow}\m{\lnot}
\m{(}\m{\psi}\m{\rightarrow}\m{\varphi}\m{)}\m{)}\m{)}
\endm
\begin{mmraw}%
|- ( ( ph <-> ps ) <-> -. ( ( ph -> ps ) -> -. ( ps -> ph ) ) ) \$= ... \$.
\end{mmraw}

\noindent Define disjunction ({\sc or}).\index{disjunction ($\vee$)}%
\index{logical or (vee)@logical {\sc or} ($\vee$)}%
\index{df-or@\texttt{df-or}}\label{df-or}

\vskip 0.5ex
\setbox\startprefix=\hbox{\tt \ \ df-or\ \$a\ }
\setbox\contprefix=\hbox{\tt \ \ \ \ \ \ \ \ \ \ \ }
\startm
\m{\vdash}\m{(}\m{(}\m{\varphi}\m{\vee}\m{\psi}\m{)}\m{\leftrightarrow}\m{(}\m{
\lnot}\m{\varphi}\m{\rightarrow}\m{\psi}\m{)}\m{)}
\endm
\begin{mmraw}%
|- ( ( ph \TOR ps ) <-> ( -. ph -> ps ) ) \$.
\end{mmraw}

\noindent Define conjunction ({\sc and}).\index{conjunction ($\wedge$)}%
\index{logical {\sc and} ($\wedge$)}%
\index{df-an@\texttt{df-an}}\label{df-an}

\vskip 0.5ex
\setbox\startprefix=\hbox{\tt \ \ df-an\ \$a\ }
\setbox\contprefix=\hbox{\tt \ \ \ \ \ \ \ \ \ \ \ }
\startm
\m{\vdash}\m{(}\m{(}\m{\varphi}\m{\wedge}\m{\psi}\m{)}\m{\leftrightarrow}\m{\lnot}
\m{(}\m{\varphi}\m{\rightarrow}\m{\lnot}\m{\psi}\m{)}\m{)}
\endm
\begin{mmraw}%
|- ( ( ph \TAND ps ) <-> -. ( ph -> -. ps ) ) \$.
\end{mmraw}

\noindent Define disjunction ({\sc or}) of 3 wffs.%
\index{df-3or@\texttt{df-3or}}\label{df-3or}

\vskip 0.5ex
\setbox\startprefix=\hbox{\tt \ \ df-3or\ \$a\ }
\setbox\contprefix=\hbox{\tt \ \ \ \ \ \ \ \ \ \ \ \ }
\startm
\m{\vdash}\m{(}\m{(}\m{\varphi}\m{\vee}\m{\psi}\m{\vee}\m{\chi}\m{)}\m{
\leftrightarrow}\m{(}\m{(}\m{\varphi}\m{\vee}\m{\psi}\m{)}\m{\vee}\m{\chi}\m{)}
\m{)}
\endm
\begin{mmraw}%
|- ( ( ph \TOR ps \TOR ch ) <-> ( ( ph \TOR ps ) \TOR ch ) ) \$.
\end{mmraw}

\noindent Define conjunction ({\sc and}) of 3 wffs.%
\index{df-3an}\label{df-3an}

\vskip 0.5ex
\setbox\startprefix=\hbox{\tt \ \ df-3an\ \$a\ }
\setbox\contprefix=\hbox{\tt \ \ \ \ \ \ \ \ \ \ \ \ }
\startm
\m{\vdash}\m{(}\m{(}\m{\varphi}\m{\wedge}\m{\psi}\m{\wedge}\m{\chi}\m{)}\m{
\leftrightarrow}\m{(}\m{(}\m{\varphi}\m{\wedge}\m{\psi}\m{)}\m{\wedge}\m{\chi}
\m{)}\m{)}
\endm

\begin{mmraw}%
|- ( ( ph \TAND ps \TAND ch ) <-> ( ( ph \TAND ps ) \TAND ch ) ) \$.
\end{mmraw}

\subsection{Definitions for Predicate Calculus}\label{metadefpred}

The symbols $x$, $y$, and $z$ represent individual variables of predicate
calculus.  In this section, they are not necessarily distinct unless it is
explicitly
mentioned.

\vskip 2ex
\noindent Define existential quantification.
The expression $\exists x \varphi$ means
``there exists an $x$ where $\varphi$ is true.''\index{existential quantifier
($\exists$)}\label{df-ex}

\vskip 0.5ex
\setbox\startprefix=\hbox{\tt \ \ df-ex\ \$a\ }
\setbox\contprefix=\hbox{\tt \ \ \ \ \ \ \ \ \ \ \ }
\startm
\m{\vdash}\m{(}\m{\exists}\m{x}\m{\varphi}\m{\leftrightarrow}\m{\lnot}\m{\forall}
\m{x}\m{\lnot}\m{\varphi}\m{)}
\endm
\begin{mmraw}%
|- ( E. x ph <-> -. A. x -. ph ) \$.
\end{mmraw}

\noindent Define proper substitution.\index{proper
substitution}\index{substitution!proper}\label{df-sb}
In our notation, we use $[ y / x ] \varphi$ to mean ``the wff that
results when $y$ is properly substituted for $x$ in the wff
$\varphi$.''\footnote{
This can also be described
as substituting $x$ with $y$, $y$ properly replaces $x$, or
$x$ is properly replaced by $y$.}
% This is elsb4, though it currently says: ( [ x / y ] z e. y <-> z e. x )
For example,
$[ y / x ] z \in x$ is the same as $z \in y$.
One way to remember this notation is to notice that it looks like division
and recall that $( y / x ) \cdot x $ is $y$ (when $x \neq 0$).
The notation is different from the notation $\varphi ( x | y )$
that is sometimes used, because the latter notation is ambiguous for us:
for example, we don't know whether $\lnot \varphi ( x | y )$ is to be
interpreted as $\lnot ( \varphi ( x | y ) )$ or
$( \lnot \varphi ) ( x | y )$.\footnote{Because of the way
we initially defined wffs, this is the case
with any postfix connective\index{postfix connective} (one occurring after the
symbols being connected) or infix connective\index{infix connective} (one
occurring between the symbols being connected).  Metamath does not have a
built-in notion of operator binding strength that could eliminate the
ambiguity.  The initial parenthesis effectively provides a prefix
connective\index{prefix connective} to eliminate ambiguity.  Some conventions,
such as Polish notation\index{Polish notation} used in the 1930's and 1940's
by Polish logicians, use only prefix connectives and thus allow the total
elimination of parentheses, at the expense of readability.  In Metamath we
could actually redefine all notation to be Polish if we wanted to without
having to change any proofs!}  Other texts often use $\varphi(y)$ to indicate
our $[ y / x ] \varphi$, but this notation is even more ambiguous since there is
no explicit indication of what is being substituted.
Note that this
definition is valid even when
$x$ and $y$ are the same variable.  The first conjunct is a ``trick'' used to
achieve this property, making the definition look somewhat peculiar at
first.

\vskip 0.5ex
\setbox\startprefix=\hbox{\tt \ \ df-sb\ \$a\ }
\setbox\contprefix=\hbox{\tt \ \ \ \ \ \ \ \ \ \ \ }
\startm
\m{\vdash}\m{(}\m{[}\m{y}\m{/}\m{x}\m{]}\m{\varphi}\m{\leftrightarrow}\m{(}%
\m{(}\m{x}\m{=}\m{y}\m{\rightarrow}\m{\varphi}\m{)}\m{\wedge}\m{\exists}\m{x}%
\m{(}\m{x}\m{=}\m{y}\m{\wedge}\m{\varphi}\m{)}\m{)}\m{)}
\endm
\begin{mmraw}%
|- ( [ y / x ] ph <-> ( ( x = y -> ph ) \TAND E. x ( x = y \TAND ph ) ) ) \$.
\end{mmraw}


\noindent Define existential uniqueness\index{existential uniqueness
quantifier ($\exists "!$)} (``there exists exactly one'').  Note that $y$ is a
variable distinct from $x$ and not occurring in $\varphi$.\label{df-eu}

\vskip 0.5ex
\setbox\startprefix=\hbox{\tt \ \ df-eu\ \$a\ }
\setbox\contprefix=\hbox{\tt \ \ \ \ \ \ \ \ \ \ \ }
\startm
\m{\vdash}\m{(}\m{\exists}\m{{!}}\m{x}\m{\varphi}\m{\leftrightarrow}\m{\exists}
\m{y}\m{\forall}\m{x}\m{(}\m{\varphi}\m{\leftrightarrow}\m{x}\m{=}\m{y}\m{)}\m{)}
\endm

\begin{mmraw}%
|- ( E! x ph <-> E. y A. x ( ph <-> x = y ) ) \$.
\end{mmraw}

\subsection{Definitions for Set Theory}\label{setdefinitions}

The symbols $x$, $y$, $z$, and $w$ represent individual variables of
predicate calculus, which in set theory are understood to be sets.
However, using only the constructs shown so far would be very inconvenient.

To make set theory more practical, we introduce the notion of a ``class.''
A class\index{class} is either a set variable (such as $x$) or an
expression of the form $\{ x | \varphi\}$ (called an ``abstraction
class''\index{abstraction class}\index{class abstraction}).  Note that
sets (i.e.\ individual variables) always exist (this is a theorem of
logic, namely $\exists y \, y = x$ for any set $x$), whereas classes may
or may not exist (i.e.\ $\exists y \, y = A$ may or may not be true).
If a class does not exist it is called a ``proper class.''\index{proper
class}\index{class!proper} Definitions \texttt{df-clab},
\texttt{df-cleq}, and \texttt{df-clel} can be used to convert an
expression containing classes into one containing only set variables and
wff metavariables.

The symbols $A$, $B$, $C$, $D$, $ F$, $G$, and $R$ are metavariables that range
over classes.  A class metavariable $A$ may be eliminated from a wff by
replacing it with $\{ x|\varphi\}$ where neither $x$ nor $\varphi$ occur in
the wff.

The theory of classes can be shown to be an eliminable and conservative
extension of set theory. The \textbf{eliminability}
property shows that for every
formula in the extended language we can build a logically equivalent
formula in the basic language; so that even if the extended language
provides more ease to convey and formulate mathematical ideas for set
theory, its expressive power does not in fact strengthen the basic
language's expressive power.
The \textbf{conservation} property shows that for
every proof of a formula of the basic language in the extended system
we can build another proof of the same formula in the basic system;
so that, concerning theorems on sets only, the deductive powers of
the extended system and of the basic system are identical. Together,
these properties mean that the extended language can be treated as a
definitional extension that is \textbf{sound}.

A rigorous justification, which we will not give here, can be found in
Levy \cite[pp.~357-366]{Levy} supplementing his informal introduction to class
theory on pp.~7-17. Two other good treatments of class theory are provided
by Quine \cite[pp.~15-21]{Quine}\index{Quine, Willard Van Orman}
and also \cite[pp.~10-14]{Takeuti}\index{Takeuti, Gaisi}.
Quine's exposition (he calls them virtual classes)
is nicely written and very readable.

In the rest of this
section, individual variables are always assumed to be distinct from
each other unless otherwise indicated.  In addition, dummy variables on the
right-hand side of a definition do not occur in the class and wff
metavariables in the definition.

The definitions we present here are a partial but self-contained
collection selected from several hundred that appear in the current
\texttt{set.mm} database.  They are adequate for a basic development of
elementary set theory.

\vskip 2ex
\noindent Define the abstraction class.\index{abstraction class}\index{class
abstraction}\label{df-clab}  $x$ and $y$
need not be distinct.  Definition 2.1 of Quine, p.~16.  This definition may
seem puzzling since it is shorter than the expression being defined and does not
buy us anything in terms of brevity.  The reason we introduce this definition
is because it fits in neatly with the extension of the $\in$ connective
provided by \texttt{df-clel}.

\vskip 0.5ex
\setbox\startprefix=\hbox{\tt \ \ df-clab\ \$a\ }
\setbox\contprefix=\hbox{\tt \ \ \ \ \ \ \ \ \ \ \ \ \ }
\startm
\m{\vdash}\m{(}\m{x}\m{\in}\m{\{}\m{y}\m{|}\m{\varphi}\m{\}}\m{%
\leftrightarrow}\m{[}\m{x}\m{/}\m{y}\m{]}\m{\varphi}\m{)}
\endm
\begin{mmraw}%
|- ( x e. \{ y | ph \} <-> [ x / y ] ph ) \$.
\end{mmraw}

\noindent Define the equality connective between classes\index{class
equality}\label{df-cleq}.  See Quine or Chapter 4 of Takeuti and Zaring for its
justification and methods for eliminating it.  This is an example of a
somewhat ``dangerous'' definition, because it extends the use of the
existing equality symbol rather than introducing a new symbol, allowing
us to make statements in the original language that may not be true.
For example, it permits us to deduce $y = z \leftrightarrow \forall x (
x \in y \leftrightarrow x \in z )$ which is not a theorem of logic but
rather presupposes the Axiom of Extensionality,\index{Axiom of
Extensionality} which we include as a hypothesis so that we can know
when this axiom is assumed in a proof (with the \texttt{show
trace{\char`\_}back} command).  We could avoid the danger by introducing
another symbol, say $\eqcirc$, in place of $=$; this
would also have the advantage of making elimination of the definition
straightforward and would eliminate the need for Extensionality as a
hypothesis.  We would then also have the advantage of being able to
identify exactly where Extensionality truly comes into play.  One of our
theorems would be $x \eqcirc y \leftrightarrow x = y$
by invoking Extensionality.  However in keeping with standard practice
we retain the ``dangerous'' definition.

\vskip 0.5ex
\setbox\startprefix=\hbox{\tt \ \ df-cleq.1\ \$e\ }
\setbox\contprefix=\hbox{\tt \ \ \ \ \ \ \ \ \ \ \ \ \ \ \ }
\startm
\m{\vdash}\m{(}\m{\forall}\m{x}\m{(}\m{x}\m{\in}\m{y}\m{\leftrightarrow}\m{x}
\m{\in}\m{z}\m{)}\m{\rightarrow}\m{y}\m{=}\m{z}\m{)}
\endm
\setbox\startprefix=\hbox{\tt \ \ df-cleq\ \$a\ }
\setbox\contprefix=\hbox{\tt \ \ \ \ \ \ \ \ \ \ \ \ \ }
\startm
\m{\vdash}\m{(}\m{A}\m{=}\m{B}\m{\leftrightarrow}\m{\forall}\m{x}\m{(}\m{x}\m{
\in}\m{A}\m{\leftrightarrow}\m{x}\m{\in}\m{B}\m{)}\m{)}
\endm
% We need to reset the startprefix and contprefix.
\setbox\startprefix=\hbox{\tt \ \ df-cleq.1\ \$e\ }
\setbox\contprefix=\hbox{\tt \ \ \ \ \ \ \ \ \ \ \ \ \ \ \ }
\begin{mmraw}%
|- ( A. x ( x e. y <-> x e. z ) -> y = z ) \$.
\end{mmraw}
\setbox\startprefix=\hbox{\tt \ \ df-cleq\ \$a\ }
\setbox\contprefix=\hbox{\tt \ \ \ \ \ \ \ \ \ \ \ \ \ }
\begin{mmraw}%
|- ( A = B <-> A. x ( x e. A <-> x e. B ) ) \$.
\end{mmraw}

\noindent Define the membership connective between classes\index{class
membership}.  Theorem 6.3 of Quine, p.~41, which we adopt as a definition.
Note that it extends the use of the existing membership symbol, but unlike
{\tt df-cleq} it does not extend the set of valid wffs of logic when the class
metavariables are replaced with set variables.\label{dfclel}\label{df-clel}

\vskip 0.5ex
\setbox\startprefix=\hbox{\tt \ \ df-clel\ \$a\ }
\setbox\contprefix=\hbox{\tt \ \ \ \ \ \ \ \ \ \ \ \ \ }
\startm
\m{\vdash}\m{(}\m{A}\m{\in}\m{B}\m{\leftrightarrow}\m{\exists}\m{x}\m{(}\m{x}
\m{=}\m{A}\m{\wedge}\m{x}\m{\in}\m{B}\m{)}\m{)}
\endm
\begin{mmraw}%
|- ( A e. B <-> E. x ( x = A \TAND x e. B ) ) \$.?
\end{mmraw}

\noindent Define inequality.

\vskip 0.5ex
\setbox\startprefix=\hbox{\tt \ \ df-ne\ \$a\ }
\setbox\contprefix=\hbox{\tt \ \ \ \ \ \ \ \ \ \ \ }
\startm
\m{\vdash}\m{(}\m{A}\m{\ne}\m{B}\m{\leftrightarrow}\m{\lnot}\m{A}\m{=}\m{B}%
\m{)}
\endm
\begin{mmraw}%
|- ( A =/= B <-> -. A = B ) \$.
\end{mmraw}

\noindent Define restricted universal quantification.\index{universal
quantifier ($\forall$)!restricted}  Enderton, p.~22.

\vskip 0.5ex
\setbox\startprefix=\hbox{\tt \ \ df-ral\ \$a\ }
\setbox\contprefix=\hbox{\tt \ \ \ \ \ \ \ \ \ \ \ \ }
\startm
\m{\vdash}\m{(}\m{\forall}\m{x}\m{\in}\m{A}\m{\varphi}\m{\leftrightarrow}\m{%
\forall}\m{x}\m{(}\m{x}\m{\in}\m{A}\m{\rightarrow}\m{\varphi}\m{)}\m{)}
\endm
\begin{mmraw}%
|- ( A. x e. A ph <-> A. x ( x e. A -> ph ) ) \$.
\end{mmraw}

\noindent Define restricted existential quantification.\index{existential
quantifier ($\exists$)!restricted}  Enderton, p.~22.

\vskip 0.5ex
\setbox\startprefix=\hbox{\tt \ \ df-rex\ \$a\ }
\setbox\contprefix=\hbox{\tt \ \ \ \ \ \ \ \ \ \ \ \ }
\startm
\m{\vdash}\m{(}\m{\exists}\m{x}\m{\in}\m{A}\m{\varphi}\m{\leftrightarrow}\m{%
\exists}\m{x}\m{(}\m{x}\m{\in}\m{A}\m{\wedge}\m{\varphi}\m{)}\m{)}
\endm
\begin{mmraw}%
|- ( E. x e. A ph <-> E. x ( x e. A \TAND ph ) ) \$.
\end{mmraw}

\noindent Define the universal class\index{universal class ($V$)}.  Definition
5.20, p.~21, of Takeuti and Zaring.\label{df-v}

\vskip 0.5ex
\setbox\startprefix=\hbox{\tt \ \ df-v\ \$a\ }
\setbox\contprefix=\hbox{\tt \ \ \ \ \ \ \ \ \ \ }
\startm
\m{\vdash}\m{{\rm V}}\m{=}\m{\{}\m{x}\m{|}\m{x}\m{=}\m{x}\m{\}}
\endm
\begin{mmraw}%
|- {\char`\_}V = \{ x | x = x \} \$.
\end{mmraw}

\noindent Define the subclass\index{subclass}\index{subset} relationship
between two classes (called the subset relation if the classes are sets i.e.\
are not proper).  Definition 5.9 of Takeuti and Zaring, p.~17.\label{df-ss}

\vskip 0.5ex
\setbox\startprefix=\hbox{\tt \ \ df-ss\ \$a\ }
\setbox\contprefix=\hbox{\tt \ \ \ \ \ \ \ \ \ \ \ }
\startm
\m{\vdash}\m{(}\m{A}\m{\subseteq}\m{B}\m{\leftrightarrow}\m{\forall}\m{x}\m{(}
\m{x}\m{\in}\m{A}\m{\rightarrow}\m{x}\m{\in}\m{B}\m{)}\m{)}
\endm
\begin{mmraw}%
|- ( A C\_ B <-> A. x ( x e. A -> x e. B ) ) \$.
\end{mmraw}

\noindent Define the union\index{union} of two classes.  Definition 5.6 of Takeuti and Zaring,
p.~16.\label{df-un}

\vskip 0.5ex
\setbox\startprefix=\hbox{\tt \ \ df-un\ \$a\ }
\setbox\contprefix=\hbox{\tt \ \ \ \ \ \ \ \ \ \ \ }
\startm
\m{\vdash}\m{(}\m{A}\m{\cup}\m{B}\m{)}\m{=}\m{\{}\m{x}\m{|}\m{(}\m{x}\m{\in}
\m{A}\m{\vee}\m{x}\m{\in}\m{B}\m{)}\m{\}}
\endm
\begin{mmraw}%
( A u. B ) = \{ x | ( x e. A \TOR x e. B ) \} \$.
\end{mmraw}

\noindent Define the intersection\index{intersection} of two classes.  Definition 5.6 of
Takeuti and Zaring, p.~16.\label{df-in}

\vskip 0.5ex
\setbox\startprefix=\hbox{\tt \ \ df-in\ \$a\ }
\setbox\contprefix=\hbox{\tt \ \ \ \ \ \ \ \ \ \ \ }
\startm
\m{\vdash}\m{(}\m{A}\m{\cap}\m{B}\m{)}\m{=}\m{\{}\m{x}\m{|}\m{(}\m{x}\m{\in}
\m{A}\m{\wedge}\m{x}\m{\in}\m{B}\m{)}\m{\}}
\endm
% Caret ^ requires special treatment
\begin{mmraw}%
|- ( A i\^{}i B ) = \{ x | ( x e. A \TAND x e. B ) \} \$.
\end{mmraw}

\noindent Define class difference\index{class difference}\index{set difference}.
Definition 5.12 of Takeuti and Zaring, p.~20.  Several notations are used in
the literature; we chose the $\setminus$ convention instead of a minus sign to
reserve the latter for later use in, e.g., arithmetic.\label{df-dif}

\vskip 0.5ex
\setbox\startprefix=\hbox{\tt \ \ df-dif\ \$a\ }
\setbox\contprefix=\hbox{\tt \ \ \ \ \ \ \ \ \ \ \ \ }
\startm
\m{\vdash}\m{(}\m{A}\m{\setminus}\m{B}\m{)}\m{=}\m{\{}\m{x}\m{|}\m{(}\m{x}\m{
\in}\m{A}\m{\wedge}\m{\lnot}\m{x}\m{\in}\m{B}\m{)}\m{\}}
\endm
\begin{mmraw}%
( A \SLASH B ) = \{ x | ( x e. A \TAND -. x e. B ) \} \$.
\end{mmraw}

\noindent Define the empty or null set\index{empty set}\index{null set}.
Compare  Definition 5.14 of Takeuti and Zaring, p.~20.\label{df-nul}

\vskip 0.5ex
\setbox\startprefix=\hbox{\tt \ \ df-nul\ \$a\ }
\setbox\contprefix=\hbox{\tt \ \ \ \ \ \ \ \ \ \ }
\startm
\m{\vdash}\m{\varnothing}\m{=}\m{(}\m{{\rm V}}\m{\setminus}\m{{\rm V}}\m{)}
\endm
\begin{mmraw}%
|- (/) = ( {\char`\_}V \SLASH {\char`\_}V ) \$.
\end{mmraw}

\noindent Define power class\index{power set}\index{power class}.  Definition 5.10 of
Takeuti and Zaring, p.~17, but we also let it apply to proper classes.  (Note
that \verb$~P$ is the symbol for calligraphic P, the tilde
suggesting ``curly;'' see Appendix~\ref{ASCII}.)\label{df-pw}

\vskip 0.5ex
\setbox\startprefix=\hbox{\tt \ \ df-pw\ \$a\ }
\setbox\contprefix=\hbox{\tt \ \ \ \ \ \ \ \ \ \ \ }
\startm
\m{\vdash}\m{{\cal P}}\m{A}\m{=}\m{\{}\m{x}\m{|}\m{x}\m{\subseteq}\m{A}\m{\}}
\endm
% Special incantation required to put ~ into the text
\begin{mmraw}%
|- \char`\~P~A = \{ x | x C\_ A \} \$.
\end{mmraw}

\noindent Define the singleton of a class\index{singleton}.  Definition 7.1 of
Quine, p.~48.  It is well-defined for proper classes, although
it is not very meaningful in this case, where it evaluates to the empty
set.

\vskip 0.5ex
\setbox\startprefix=\hbox{\tt \ \ df-sn\ \$a\ }
\setbox\contprefix=\hbox{\tt \ \ \ \ \ \ \ \ \ \ \ }
\startm
\m{\vdash}\m{\{}\m{A}\m{\}}\m{=}\m{\{}\m{x}\m{|}\m{x}\m{=}\m{A}\m{\}}
\endm
\begin{mmraw}%
|- \{ A \} = \{ x | x = A \} \$.
\end{mmraw}%

\noindent Define an unordered pair of classes\index{unordered pair}\index{pair}.  Definition
7.1 of Quine, p.~48.

\vskip 0.5ex
\setbox\startprefix=\hbox{\tt \ \ df-pr\ \$a\ }
\setbox\contprefix=\hbox{\tt \ \ \ \ \ \ \ \ \ \ \ }
\startm
\m{\vdash}\m{\{}\m{A}\m{,}\m{B}\m{\}}\m{=}\m{(}\m{\{}\m{A}\m{\}}\m{\cup}\m{\{}
\m{B}\m{\}}\m{)}
\endm
\begin{mmraw}%
|- \{ A , B \} = ( \{ A \} u. \{ B \} ) \$.
\end{mmraw}

\noindent Define an unordered triple of classes\index{unordered triple}.  Definition of
Enderton, p.~19.

\vskip 0.5ex
\setbox\startprefix=\hbox{\tt \ \ df-tp\ \$a\ }
\setbox\contprefix=\hbox{\tt \ \ \ \ \ \ \ \ \ \ \ }
\startm
\m{\vdash}\m{\{}\m{A}\m{,}\m{B}\m{,}\m{C}\m{\}}\m{=}\m{(}\m{\{}\m{A}\m{,}\m{B}
\m{\}}\m{\cup}\m{\{}\m{C}\m{\}}\m{)}
\endm
\begin{mmraw}%
|- \{ A , B , C \} = ( \{ A , B \} u. \{ C \} ) \$.
\end{mmraw}%

\noindent Kuratowski's\index{Kuratowski, Kazimierz} ordered pair\index{ordered
pair} definition.  Definition 9.1 of Quine, p.~58. For proper classes it is
not meaningful but is well-defined for convenience.  (Note that \verb$<.$
stands for $\langle$ whereas \verb$<$ stands for $<$, and similarly for
\verb$>.$\,.)\label{df-op}

\vskip 0.5ex
\setbox\startprefix=\hbox{\tt \ \ df-op\ \$a\ }
\setbox\contprefix=\hbox{\tt \ \ \ \ \ \ \ \ \ \ \ }
\startm
\m{\vdash}\m{\langle}\m{A}\m{,}\m{B}\m{\rangle}\m{=}\m{\{}\m{\{}\m{A}\m{\}}
\m{,}\m{\{}\m{A}\m{,}\m{B}\m{\}}\m{\}}
\endm
\begin{mmraw}%
|- <. A , B >. = \{ \{ A \} , \{ A , B \} \} \$.
\end{mmraw}

\noindent Define the union of a class\index{union}.  Definition 5.5, p.~16,
of Takeuti and Zaring.

\vskip 0.5ex
\setbox\startprefix=\hbox{\tt \ \ df-uni\ \$a\ }
\setbox\contprefix=\hbox{\tt \ \ \ \ \ \ \ \ \ \ \ \ }
\startm
\m{\vdash}\m{\bigcup}\m{A}\m{=}\m{\{}\m{x}\m{|}\m{\exists}\m{y}\m{(}\m{x}\m{
\in}\m{y}\m{\wedge}\m{y}\m{\in}\m{A}\m{)}\m{\}}
\endm
\begin{mmraw}%
|- U. A = \{ x | E. y ( x e. y \TAND y e. A ) \} \$.
\end{mmraw}

\noindent Define the intersection\index{intersection} of a class.  Definition 7.35,
p.~44, of Takeuti and Zaring.

\vskip 0.5ex
\setbox\startprefix=\hbox{\tt \ \ df-int\ \$a\ }
\setbox\contprefix=\hbox{\tt \ \ \ \ \ \ \ \ \ \ \ \ }
\startm
\m{\vdash}\m{\bigcap}\m{A}\m{=}\m{\{}\m{x}\m{|}\m{\forall}\m{y}\m{(}\m{y}\m{
\in}\m{A}\m{\rightarrow}\m{x}\m{\in}\m{y}\m{)}\m{\}}
\endm
\begin{mmraw}%
|- |\^{}| A = \{ x | A. y ( y e. A -> x e. y ) \} \$.
\end{mmraw}

\noindent Define a transitive class\index{transitive class}\index{transitive
set}.  This should not be confused with a transitive relation, which is a different
concept.  Definition from p.~71 of Enderton, extended to classes.

\vskip 0.5ex
\setbox\startprefix=\hbox{\tt \ \ df-tr\ \$a\ }
\setbox\contprefix=\hbox{\tt \ \ \ \ \ \ \ \ \ \ \ }
\startm
\m{\vdash}\m{(}\m{\mbox{\rm Tr}}\m{A}\m{\leftrightarrow}\m{\bigcup}\m{A}\m{
\subseteq}\m{A}\m{)}
\endm
\begin{mmraw}%
|- ( Tr A <-> U. A C\_ A ) \$.
\end{mmraw}

\noindent Define a notation for a general binary relation\index{binary
relation}.  Definition 6.18, p.~29, of Takeuti and Zaring, generalized to
arbitrary classes.  This definition is well-defined, although not very
meaningful, when classes $A$ and/or $B$ are proper.\label{dfbr}  The lack of
parentheses (or any other connective) creates no ambiguity since we are defining
an atomic wff.

\vskip 0.5ex
\setbox\startprefix=\hbox{\tt \ \ df-br\ \$a\ }
\setbox\contprefix=\hbox{\tt \ \ \ \ \ \ \ \ \ \ \ }
\startm
\m{\vdash}\m{(}\m{A}\m{\,R}\m{\,B}\m{\leftrightarrow}\m{\langle}\m{A}\m{,}\m{B}
\m{\rangle}\m{\in}\m{R}\m{)}
\endm
\begin{mmraw}%
|- ( A R B <-> <. A , B >. e. R ) \$.
\end{mmraw}

\noindent Define an abstraction class of ordered pairs\index{abstraction
class!of ordered
pairs}.  A special case of Definition 4.16, p.~14, of Takeuti and Zaring.
Note that $ z $ must be distinct from $ x $ and $ y $,
and $ z $ must not occur in $\varphi$, but $ x $ and $ y $ may be
identical and may appear in $\varphi$.

\vskip 0.5ex
\setbox\startprefix=\hbox{\tt \ \ df-opab\ \$a\ }
\setbox\contprefix=\hbox{\tt \ \ \ \ \ \ \ \ \ \ \ \ \ }
\startm
\m{\vdash}\m{\{}\m{\langle}\m{x}\m{,}\m{y}\m{\rangle}\m{|}\m{\varphi}\m{\}}\m{=}
\m{\{}\m{z}\m{|}\m{\exists}\m{x}\m{\exists}\m{y}\m{(}\m{z}\m{=}\m{\langle}\m{x}
\m{,}\m{y}\m{\rangle}\m{\wedge}\m{\varphi}\m{)}\m{\}}
\endm

\begin{mmraw}%
|- \{ <. x , y >. | ph \} = \{ z | E. x E. y ( z =
<. x , y >. /\ ph ) \} \$.
\end{mmraw}

\noindent Define the epsilon relation\index{epsilon relation}.  Similar to Definition
6.22, p.~30, of Takeuti and Zaring.

\vskip 0.5ex
\setbox\startprefix=\hbox{\tt \ \ df-eprel\ \$a\ }
\setbox\contprefix=\hbox{\tt \ \ \ \ \ \ \ \ \ \ \ \ \ \ }
\startm
\m{\vdash}\m{{\rm E}}\m{=}\m{\{}\m{\langle}\m{x}\m{,}\m{y}\m{\rangle}\m{|}\m{x}\m{
\in}\m{y}\m{\}}
\endm
\begin{mmraw}%
|- \_E = \{ <. x , y >. | x e. y \} \$.
\end{mmraw}

\noindent Define a founded relation\index{founded relation}.  $R$ is a founded
relation on $A$ iff\index{iff} (if and only if) each nonempty subset of $A$
has an ``$R$-minimal element.''  Similar to Definition 6.21, p.~30, of
Takeuti and Zaring.

\vskip 0.5ex
\setbox\startprefix=\hbox{\tt \ \ df-fr\ \$a\ }
\setbox\contprefix=\hbox{\tt \ \ \ \ \ \ \ \ \ \ \ }
\startm
\m{\vdash}\m{(}\m{R}\m{\,\mbox{\rm Fr}}\m{\,A}\m{\leftrightarrow}\m{\forall}\m{x}
\m{(}\m{(}\m{x}\m{\subseteq}\m{A}\m{\wedge}\m{\lnot}\m{x}\m{=}\m{\varnothing}
\m{)}\m{\rightarrow}\m{\exists}\m{y}\m{(}\m{y}\m{\in}\m{x}\m{\wedge}\m{(}\m{x}
\m{\cap}\m{\{}\m{z}\m{|}\m{z}\m{\,R}\m{\,y}\m{\}}\m{)}\m{=}\m{\varnothing}\m{)}
\m{)}\m{)}
\endm
\begin{mmraw}%
|- ( R Fr A <-> A. x ( ( x C\_ A \TAND -. x = (/) ) ->
E. y ( y e. x \TAND ( x i\^{}i \{ z | z R y \} ) = (/) ) ) ) \$.
\end{mmraw}

\noindent Define a well-ordering\index{well-ordering}.  $R$ is a well-ordering of $A$ iff
it is founded on $A$ and the elements of $A$ are pairwise $R$-comparable.
Similar to Definition 6.24(2), p.~30, of Takeuti and Zaring.

\vskip 0.5ex
\setbox\startprefix=\hbox{\tt \ \ df-we\ \$a\ }
\setbox\contprefix=\hbox{\tt \ \ \ \ \ \ \ \ \ \ \ }
\startm
\m{\vdash}\m{(}\m{R}\m{\,\mbox{\rm We}}\m{\,A}\m{\leftrightarrow}\m{(}\m{R}\m{\,
\mbox{\rm Fr}}\m{\,A}\m{\wedge}\m{\forall}\m{x}\m{\forall}\m{y}\m{(}\m{(}\m{x}\m{
\in}\m{A}\m{\wedge}\m{y}\m{\in}\m{A}\m{)}\m{\rightarrow}\m{(}\m{x}\m{\,R}\m{\,y}
\m{\vee}\m{x}\m{=}\m{y}\m{\vee}\m{y}\m{\,R}\m{\,x}\m{)}\m{)}\m{)}\m{)}
\endm
\begin{mmraw}%
( R We A <-> ( R Fr A \TAND A. x A. y ( ( x e.
A \TAND y e. A ) -> ( x R y \TOR x = y \TOR y R x ) ) ) ) \$.
\end{mmraw}

\noindent Define the ordinal predicate\index{ordinal predicate}, which is true for a class
that is transitive and is well-ordered by the epsilon relation.  Similar to
definition on p.~468, Bell and Machover.

\vskip 0.5ex
\setbox\startprefix=\hbox{\tt \ \ df-ord\ \$a\ }
\setbox\contprefix=\hbox{\tt \ \ \ \ \ \ \ \ \ \ \ \ }
\startm
\m{\vdash}\m{(}\m{\mbox{\rm Ord}}\m{\,A}\m{\leftrightarrow}\m{(}
\m{\mbox{\rm Tr}}\m{\,A}\m{\wedge}\m{E}\m{\,\mbox{\rm We}}\m{\,A}\m{)}\m{)}
\endm
\begin{mmraw}%
|- ( Ord A <-> ( Tr A \TAND E We A ) ) \$.
\end{mmraw}

\noindent Define the class of all ordinal numbers\index{ordinal number}.  An ordinal number is
a set that satisfies the ordinal predicate.  Definition 7.11 of Takeuti and
Zaring, p.~38.

\vskip 0.5ex
\setbox\startprefix=\hbox{\tt \ \ df-on\ \$a\ }
\setbox\contprefix=\hbox{\tt \ \ \ \ \ \ \ \ \ \ \ }
\startm
\m{\vdash}\m{\,\mbox{\rm On}}\m{=}\m{\{}\m{x}\m{|}\m{\mbox{\rm Ord}}\m{\,x}
\m{\}}
\endm
\begin{mmraw}%
|- On = \{ x | Ord x \} \$.
\end{mmraw}

\noindent Define the limit ordinal predicate\index{limit ordinal}, which is true for a
non-empty ordinal that is not a successor (i.e.\ that is the union of itself).
Compare Bell and Machover, p.~471 and Exercise (1), p.~42 of Takeuti and
Zaring.

\vskip 0.5ex
\setbox\startprefix=\hbox{\tt \ \ df-lim\ \$a\ }
\setbox\contprefix=\hbox{\tt \ \ \ \ \ \ \ \ \ \ \ \ }
\startm
\m{\vdash}\m{(}\m{\mbox{\rm Lim}}\m{\,A}\m{\leftrightarrow}\m{(}\m{\mbox{
\rm Ord}}\m{\,A}\m{\wedge}\m{\lnot}\m{A}\m{=}\m{\varnothing}\m{\wedge}\m{A}
\m{=}\m{\bigcup}\m{A}\m{)}\m{)}
\endm
\begin{mmraw}%
|- ( Lim A <-> ( Ord A \TAND -. A = (/) \TAND A = U. A ) ) \$.
\end{mmraw}

\noindent Define the successor\index{successor} of a class.  Definition 7.22 of Takeuti
and Zaring, p.~41.  Our definition is a generalization to classes, although it
is meaningless when classes are proper.

\vskip 0.5ex
\setbox\startprefix=\hbox{\tt \ \ df-suc\ \$a\ }
\setbox\contprefix=\hbox{\tt \ \ \ \ \ \ \ \ \ \ \ \ }
\startm
\m{\vdash}\m{\,\mbox{\rm suc}}\m{\,A}\m{=}\m{(}\m{A}\m{\cup}\m{\{}\m{A}\m{\}}
\m{)}
\endm
\begin{mmraw}%
|- suc A = ( A u. \{ A \} ) \$.
\end{mmraw}

\noindent Define the class of natural numbers\index{natural number}\index{omega
($\omega$)}.  Compare Bell and Machover, p.~471.\label{dfom}

\vskip 0.5ex
\setbox\startprefix=\hbox{\tt \ \ df-om\ \$a\ }
\setbox\contprefix=\hbox{\tt \ \ \ \ \ \ \ \ \ \ \ }
\startm
\m{\vdash}\m{\omega}\m{=}\m{\{}\m{x}\m{|}\m{(}\m{\mbox{\rm Ord}}\m{\,x}\m{
\wedge}\m{\forall}\m{y}\m{(}\m{\mbox{\rm Lim}}\m{\,y}\m{\rightarrow}\m{x}\m{
\in}\m{y}\m{)}\m{)}\m{\}}
\endm
\begin{mmraw}%
|- om = \{ x | ( Ord x \TAND A. y ( Lim y -> x e. y ) ) \} \$.
\end{mmraw}

\noindent Define the Cartesian product (also called the
cross product)\index{Cartesian product}\index{cross product}
of two classes.  Definition 9.11 of Quine, p.~64.

\vskip 0.5ex
\setbox\startprefix=\hbox{\tt \ \ df-xp\ \$a\ }
\setbox\contprefix=\hbox{\tt \ \ \ \ \ \ \ \ \ \ \ }
\startm
\m{\vdash}\m{(}\m{A}\m{\times}\m{B}\m{)}\m{=}\m{\{}\m{\langle}\m{x}\m{,}\m{y}
\m{\rangle}\m{|}\m{(}\m{x}\m{\in}\m{A}\m{\wedge}\m{y}\m{\in}\m{B}\m{)}\m{\}}
\endm
\begin{mmraw}%
|- ( A X. B ) = \{ <. x , y >. | ( x e. A \TAND y e. B) \} \$.
\end{mmraw}

\noindent Define a relation\index{relation}.  Definition 6.4(1) of Takeuti and
Zaring, p.~23.

\vskip 0.5ex
\setbox\startprefix=\hbox{\tt \ \ df-rel\ \$a\ }
\setbox\contprefix=\hbox{\tt \ \ \ \ \ \ \ \ \ \ \ \ }
\startm
\m{\vdash}\m{(}\m{\mbox{\rm Rel}}\m{\,A}\m{\leftrightarrow}\m{A}\m{\subseteq}
\m{(}\m{{\rm V}}\m{\times}\m{{\rm V}}\m{)}\m{)}
\endm
\begin{mmraw}%
|- ( Rel A <-> A C\_ ( {\char`\_}V X. {\char`\_}V ) ) \$.
\end{mmraw}

\noindent Define the domain\index{domain} of a class.  Definition 6.5(1) of
Takeuti and Zaring, p.~24.

\vskip 0.5ex
\setbox\startprefix=\hbox{\tt \ \ df-dm\ \$a\ }
\setbox\contprefix=\hbox{\tt \ \ \ \ \ \ \ \ \ \ \ }
\startm
\m{\vdash}\m{\,\mbox{\rm dom}}\m{A}\m{=}\m{\{}\m{x}\m{|}\m{\exists}\m{y}\m{
\langle}\m{x}\m{,}\m{y}\m{\rangle}\m{\in}\m{A}\m{\}}
\endm
\begin{mmraw}%
|- dom A = \{ x | E. y <. x , y >. e. A \} \$.
\end{mmraw}

\noindent Define the range\index{range} of a class.  Definition 6.5(2) of
Takeuti and Zaring, p.~24.

\vskip 0.5ex
\setbox\startprefix=\hbox{\tt \ \ df-rn\ \$a\ }
\setbox\contprefix=\hbox{\tt \ \ \ \ \ \ \ \ \ \ \ }
\startm
\m{\vdash}\m{\,\mbox{\rm ran}}\m{A}\m{=}\m{\{}\m{y}\m{|}\m{\exists}\m{x}\m{
\langle}\m{x}\m{,}\m{y}\m{\rangle}\m{\in}\m{A}\m{\}}
\endm
\begin{mmraw}%
|- ran A = \{ y | E. x <. x , y >. e. A \} \$.
\end{mmraw}

\noindent Define the restriction\index{restriction} of a class.  Definition
6.6(1) of Takeuti and Zaring, p.~24.

\vskip 0.5ex
\setbox\startprefix=\hbox{\tt \ \ df-res\ \$a\ }
\setbox\contprefix=\hbox{\tt \ \ \ \ \ \ \ \ \ \ \ \ }
\startm
\m{\vdash}\m{(}\m{A}\m{\restriction}\m{B}\m{)}\m{=}\m{(}\m{A}\m{\cap}\m{(}\m{B}
\m{\times}\m{{\rm V}}\m{)}\m{)}
\endm
\begin{mmraw}%
|- ( A |` B ) = ( A i\^{}i ( B X. {\char`\_}V ) ) \$.
\end{mmraw}

\noindent Define the image\index{image} of a class.  Definition 6.6(2) of
Takeuti and Zaring, p.~24.

\vskip 0.5ex
\setbox\startprefix=\hbox{\tt \ \ df-ima\ \$a\ }
\setbox\contprefix=\hbox{\tt \ \ \ \ \ \ \ \ \ \ \ \ }
\startm
\m{\vdash}\m{(}\m{A}\m{``}\m{B}\m{)}\m{=}\m{\,\mbox{\rm ran}}\m{\,(}\m{A}\m{
\restriction}\m{B}\m{)}
\endm
\begin{mmraw}%
|- ( A " B ) = ran ( A |` B ) \$.
\end{mmraw}

\noindent Define the composition\index{composition} of two classes.  Definition 6.6(3) of
Takeuti and Zaring, p.~24.

\vskip 0.5ex
\setbox\startprefix=\hbox{\tt \ \ df-co\ \$a\ }
\setbox\contprefix=\hbox{\tt \ \ \ \ \ \ \ \ \ \ \ \ }
\startm
\m{\vdash}\m{(}\m{A}\m{\circ}\m{B}\m{)}\m{=}\m{\{}\m{\langle}\m{x}\m{,}\m{y}\m{
\rangle}\m{|}\m{\exists}\m{z}\m{(}\m{\langle}\m{x}\m{,}\m{z}\m{\rangle}\m{\in}
\m{B}\m{\wedge}\m{\langle}\m{z}\m{,}\m{y}\m{\rangle}\m{\in}\m{A}\m{)}\m{\}}
\endm
\begin{mmraw}%
|- ( A o. B ) = \{ <. x , y >. | E. z ( <. x , z
>. e. B \TAND <. z , y >. e. A ) \} \$.
\end{mmraw}

\noindent Define a function\index{function}.  Definition 6.4(4) of Takeuti and
Zaring, p.~24.

\vskip 0.5ex
\setbox\startprefix=\hbox{\tt \ \ df-fun\ \$a\ }
\setbox\contprefix=\hbox{\tt \ \ \ \ \ \ \ \ \ \ \ \ }
\startm
\m{\vdash}\m{(}\m{\mbox{\rm Fun}}\m{\,A}\m{\leftrightarrow}\m{(}
\m{\mbox{\rm Rel}}\m{\,A}\m{\wedge}
\m{\forall}\m{x}\m{\exists}\m{z}\m{\forall}\m{y}\m{(}
\m{\langle}\m{x}\m{,}\m{y}\m{\rangle}\m{\in}\m{A}\m{\rightarrow}\m{y}\m{=}\m{z}
\m{)}\m{)}\m{)}
\endm
\begin{mmraw}%
|- ( Fun A <-> ( Rel A /\ A. x E. z A. y ( <. x
   , y >. e. A -> y = z ) ) ) \$.
\end{mmraw}

\noindent Define a function with domain.  Definition 6.15(1) of Takeuti and
Zaring, p.~27.

\vskip 0.5ex
\setbox\startprefix=\hbox{\tt \ \ df-fn\ \$a\ }
\setbox\contprefix=\hbox{\tt \ \ \ \ \ \ \ \ \ \ \ }
\startm
\m{\vdash}\m{(}\m{A}\m{\,\mbox{\rm Fn}}\m{\,B}\m{\leftrightarrow}\m{(}
\m{\mbox{\rm Fun}}\m{\,A}\m{\wedge}\m{\mbox{\rm dom}}\m{\,A}\m{=}\m{B}\m{)}
\m{)}
\endm
\begin{mmraw}%
|- ( A Fn B <-> ( Fun A \TAND dom A = B ) ) \$.
\end{mmraw}

\noindent Define a function with domain and co-domain.  Definition 6.15(3)
of Takeuti and Zaring, p.~27.

\vskip 0.5ex
\setbox\startprefix=\hbox{\tt \ \ df-f\ \$a\ }
\setbox\contprefix=\hbox{\tt \ \ \ \ \ \ \ \ \ \ }
\startm
\m{\vdash}\m{(}\m{F}\m{:}\m{A}\m{\longrightarrow}\m{B}\m{
\leftrightarrow}\m{(}\m{F}\m{\,\mbox{\rm Fn}}\m{\,A}\m{\wedge}\m{
\mbox{\rm ran}}\m{\,F}\m{\subseteq}\m{B}\m{)}\m{)}
\endm
\begin{mmraw}%
|- ( F : A --> B <-> ( F Fn A \TAND ran F C\_ B ) ) \$.
\end{mmraw}

\noindent Define a one-to-one function\index{one-to-one function}.  Compare
Definition 6.15(5) of Takeuti and Zaring, p.~27.

\vskip 0.5ex
\setbox\startprefix=\hbox{\tt \ \ df-f1\ \$a\ }
\setbox\contprefix=\hbox{\tt \ \ \ \ \ \ \ \ \ \ \ }
\startm
\m{\vdash}\m{(}\m{F}\m{:}\m{A}\m{
\raisebox{.5ex}{${\textstyle{\:}_{\mbox{\footnotesize\rm
1\tt -\rm 1}}}\atop{\textstyle{
\longrightarrow}\atop{\textstyle{}^{\mbox{\footnotesize\rm {\ }}}}}$}
}\m{B}
\m{\leftrightarrow}\m{(}\m{F}\m{:}\m{A}\m{\longrightarrow}\m{B}
\m{\wedge}\m{\forall}\m{y}\m{\exists}\m{z}\m{\forall}\m{x}\m{(}\m{\langle}\m{x}
\m{,}\m{y}\m{\rangle}\m{\in}\m{F}\m{\rightarrow}\m{x}\m{=}\m{z}\m{)}\m{)}\m{)}
\endm
\begin{mmraw}%
|- ( F : A -1-1-> B <-> ( F : A --> B \TAND
   A. y E. z A. x ( <. x , y >. e. F -> x = z ) ) ) \$.
\end{mmraw}

\noindent Define an onto function\index{onto function}.  Definition 6.15(4) of Takeuti and
Zaring, p.~27.

\vskip 0.5ex
\setbox\startprefix=\hbox{\tt \ \ df-fo\ \$a\ }
\setbox\contprefix=\hbox{\tt \ \ \ \ \ \ \ \ \ \ \ }
\startm
\m{\vdash}\m{(}\m{F}\m{:}\m{A}\m{
\raisebox{.5ex}{${\textstyle{\:}_{\mbox{\footnotesize\rm
{\ }}}}\atop{\textstyle{
\longrightarrow}\atop{\textstyle{}^{\mbox{\footnotesize\rm onto}}}}$}
}\m{B}
\m{\leftrightarrow}\m{(}\m{F}\m{\,\mbox{\rm Fn}}\m{\,A}\m{\wedge}
\m{\mbox{\rm ran}}\m{\,F}\m{=}\m{B}\m{)}\m{)}
\endm
\begin{mmraw}%
|- ( F : A -onto-> B <-> ( F Fn A /\ ran F = B ) ) \$.
\end{mmraw}

\noindent Define a one-to-one, onto function.  Compare Definition 6.15(6) of
Takeuti and Zaring, p.~27.

\vskip 0.5ex
\setbox\startprefix=\hbox{\tt \ \ df-f1o\ \$a\ }
\setbox\contprefix=\hbox{\tt \ \ \ \ \ \ \ \ \ \ \ \ }
\startm
\m{\vdash}\m{(}\m{F}\m{:}\m{A}
\m{
\raisebox{.5ex}{${\textstyle{\:}_{\mbox{\footnotesize\rm
1\tt -\rm 1}}}\atop{\textstyle{
\longrightarrow}\atop{\textstyle{}^{\mbox{\footnotesize\rm onto}}}}$}
}
\m{B}
\m{\leftrightarrow}\m{(}\m{F}\m{:}\m{A}
\m{
\raisebox{.5ex}{${\textstyle{\:}_{\mbox{\footnotesize\rm
1\tt -\rm 1}}}\atop{\textstyle{
\longrightarrow}\atop{\textstyle{}^{\mbox{\footnotesize\rm {\ }}}}}$}
}
\m{B}\m{\wedge}\m{F}\m{:}\m{A}
\m{
\raisebox{.5ex}{${\textstyle{\:}_{\mbox{\footnotesize\rm
{\ }}}}\atop{\textstyle{
\longrightarrow}\atop{\textstyle{}^{\mbox{\footnotesize\rm onto}}}}$}
}
\m{B}\m{)}\m{)}
\endm
\begin{mmraw}%
|- ( F : A -1-1-onto-> B <-> ( F : A -1-1-> B? \TAND F : A -onto-> B ) ) \$.?
\end{mmraw}

\noindent Define the value of a function\index{function value}.  This
definition applies to any class and evaluates to the empty set when it is not
meaningful. Note that $ F`A$ means the same thing as the more familiar $ F(A)$
notation for a function's value at $A$.  The $ F`A$ notation is common in
formal set theory.\label{df-fv}

\vskip 0.5ex
\setbox\startprefix=\hbox{\tt \ \ df-fv\ \$a\ }
\setbox\contprefix=\hbox{\tt \ \ \ \ \ \ \ \ \ \ \ }
\startm
\m{\vdash}\m{(}\m{F}\m{`}\m{A}\m{)}\m{=}\m{\bigcup}\m{\{}\m{x}\m{|}\m{(}\m{F}%
\m{``}\m{\{}\m{A}\m{\}}\m{)}\m{=}\m{\{}\m{x}\m{\}}\m{\}}
\endm
\begin{mmraw}%
|- ( F ` A ) = U. \{ x | ( F " \{ A \} ) = \{ x \} \} \$.
\end{mmraw}

\noindent Define the result of an operation.\index{operation}  Here, $F$ is
     an operation on two
     values (such as $+$ for real numbers).   This is defined for proper
     classes $A$ and $B$ even though not meaningful in that case.  However,
     the definition can be meaningful when $F$ is a proper class.\label{dfopr}

\vskip 0.5ex
\setbox\startprefix=\hbox{\tt \ \ df-opr\ \$a\ }
\setbox\contprefix=\hbox{\tt \ \ \ \ \ \ \ \ \ \ \ \ }
\startm
\m{\vdash}\m{(}\m{A}\m{\,F}\m{\,B}\m{)}\m{=}\m{(}\m{F}\m{`}\m{\langle}\m{A}%
\m{,}\m{B}\m{\rangle}\m{)}
\endm
\begin{mmraw}%
|- ( A F B ) = ( F ` <. A , B >. ) \$.
\end{mmraw}

\section{Tricks of the Trade}\label{tricks}

In the \texttt{set.mm}\index{set theory database (\texttt{set.mm})} database our goal
was usually to conform to modern notation.  However in some cases the
relationship to standard textbook language may be obscured by several
unconventional devices we used to simplify the development and to take
advantage of the Metamath language.  In this section we will describe some
common conventions used in \texttt{set.mm}.

\begin{itemize}
\item
The turnstile\index{turnstile ({$\,\vdash$})} symbol, $\vdash$, meaning ``it
is provable that,'' is the first token of all assertions and hypotheses that
aren't syntax constructions.  This is a standard convention in logic.  (We
mentioned this earlier, but this symbol is bothersome to some people without a
logic background.  It has no deeper meaning but just provides us with a way to
distinguish syntax constructions from ordinary mathematical statements.)

\item
A hypothesis of the form

\vskip 1ex
\setbox\startprefix=\hbox{\tt \ \ \ \ \ \ \ \ \ \$e\ }
\setbox\contprefix=\hbox{\tt \ \ \ \ \ \ \ \ \ \ }
\startm
\m{\vdash}\m{(}\m{\varphi}\m{\rightarrow}\m{\forall}\m{x}\m{\varphi}\m{)}
\endm
\vskip 1ex

should be read ``assume variable $x$ is (effectively) not free in wff
$\varphi$.''\index{effectively not free}
Literally, this says ``assume it is provable that $\varphi \rightarrow \forall
x\, \varphi$.''  This device lets us avoid the complexities associated with
the standard treatment of free and bound variables.
%% Uncomment this when uncommenting section {formalspec} below
The footnote on p.~\pageref{effectivelybound} discusses this further.

\item
A statement of one of the forms

\vskip 1ex
\setbox\startprefix=\hbox{\tt \ \ \ \ \ \ \ \ \ \$a\ }
\setbox\contprefix=\hbox{\tt \ \ \ \ \ \ \ \ \ \ }
\startm
\m{\vdash}\m{(}\m{\lnot}\m{\forall}\m{x}\m{\,x}\m{=}\m{y}\m{\rightarrow}
\m{\ldots}\m{)}
\endm
\setbox\startprefix=\hbox{\tt \ \ \ \ \ \ \ \ \ \$p\ }
\setbox\contprefix=\hbox{\tt \ \ \ \ \ \ \ \ \ \ }
\startm
\m{\vdash}\m{(}\m{\lnot}\m{\forall}\m{x}\m{\,x}\m{=}\m{y}\m{\rightarrow}
\m{\ldots}\m{)}
\endm
\vskip 1ex

should be read ``if $x$ and $y$ are distinct variables, then...''  This
antecedent provides us with a technical device to avoid the need for the
\texttt{\$d} statement early in our development of predicate calculus,
permitting symbol manipulations to be as conceptually simple as those in
propositional calculus.  However, the \texttt{\$d} statement eventually
becomes a requirement, and after that this device is rarely used.

\item
The statement

\vskip 1ex
\setbox\startprefix=\hbox{\tt \ \ \ \ \ \ \ \ \ \$d\ }
\setbox\contprefix=\hbox{\tt \ \ \ \ \ \ \ \ \ \ }
\startm
\m{x}\m{\,y}
\endm
\vskip 1ex

should be read ``assume $x$ and $y$ are distinct variables.''

\item
The statement

\vskip 1ex
\setbox\startprefix=\hbox{\tt \ \ \ \ \ \ \ \ \ \$d\ }
\setbox\contprefix=\hbox{\tt \ \ \ \ \ \ \ \ \ \ }
\startm
\m{x}\m{\,\varphi}
\endm
\vskip 1ex

should be read ``assume $x$ does not occur in $\varphi$.''

\item
The statement

\vskip 1ex
\setbox\startprefix=\hbox{\tt \ \ \ \ \ \ \ \ \ \$d\ }
\setbox\contprefix=\hbox{\tt \ \ \ \ \ \ \ \ \ \ }
\startm
\m{x}\m{\,A}
\endm
\vskip 1ex

should be read ``assume variable $x$ does not occur in class $A$.''

\item
The restriction and hypothesis group

\vskip 1ex
\setbox\startprefix=\hbox{\tt \ \ \ \ \ \ \ \ \ \$d\ }
\setbox\contprefix=\hbox{\tt \ \ \ \ \ \ \ \ \ \ }
\startm
\m{x}\m{\,A}
\endm
\setbox\startprefix=\hbox{\tt \ \ \ \ \ \ \ \ \ \$d\ }
\setbox\contprefix=\hbox{\tt \ \ \ \ \ \ \ \ \ \ }
\startm
\m{x}\m{\,\psi}
\endm
\setbox\startprefix=\hbox{\tt \ \ \ \ \ \ \ \ \ \$e\ }
\setbox\contprefix=\hbox{\tt \ \ \ \ \ \ \ \ \ \ }
\startm
\m{\vdash}\m{(}\m{x}\m{=}\m{A}\m{\rightarrow}\m{(}\m{\varphi}\m{\leftrightarrow}
\m{\psi}\m{)}\m{)}
\endm
\vskip 1ex

is frequently used in place of explicit substitution, meaning ``assume
$\psi$ results from the proper substitution of $A$ for $x$ in
$\varphi$.''  Sometimes ``\texttt{\$e} $\vdash ( \psi \rightarrow
\forall x \, \psi )$'' is used instead of ``\texttt{\$d} $x\, \psi $,''
which requires only that $x$ be effectively not free in $\varphi$ but
not necessarily absent from it.  The use of implicit
substitution\index{substitution!implicit} is partly a matter of personal
style, although it may make proofs somewhat shorter than would be the
case with explicit substitution.

\item
The hypothesis


\vskip 1ex
\setbox\startprefix=\hbox{\tt \ \ \ \ \ \ \ \ \ \$e\ }
\setbox\contprefix=\hbox{\tt \ \ \ \ \ \ \ \ \ \ }
\startm
\m{\vdash}\m{A}\m{\in}\m{{\rm V}}
\endm
\vskip 1ex

should be read ``assume class $A$ is a set (i.e.\ exists).''
This is a convenient convention used by Quine.

\item
The restriction and hypothesis

\vskip 1ex
\setbox\startprefix=\hbox{\tt \ \ \ \ \ \ \ \ \ \$d\ }
\setbox\contprefix=\hbox{\tt \ \ \ \ \ \ \ \ \ \ }
\startm
\m{x}\m{\,y}
\endm
\setbox\startprefix=\hbox{\tt \ \ \ \ \ \ \ \ \ \$e\ }
\setbox\contprefix=\hbox{\tt \ \ \ \ \ \ \ \ \ \ }
\startm
\m{\vdash}\m{(}\m{y}\m{\in}\m{A}\m{\rightarrow}\m{\forall}\m{x}\m{\,y}
\m{\in}\m{A}\m{)}
\endm
\vskip 1ex

should be read ``assume variable $x$ is
(effectively) not free in class $A$.''

\end{itemize}

\section{A Theorem Sampler}\label{sometheorems}

In this section we list some of the more important theorems that are proved in
the \texttt{set.mm} database, and they illustrate the kinds of things that can be
done with Metamath.  While all of these facts are well-known results,
Metamath offers the advantage of easily allowing you to trace their
derivation back to axioms.  Our intent here is not to try to explain the
details or motivation; for this we refer you to the textbooks that are
mentioned in the descriptions.  (The \texttt{set.mm} file has bibliographic
references for the text references.)  Their proofs often embody important
concepts you may wish to explore with the Metamath program (see
Section~\ref{exploring}).  All the symbols that are used here are defined in
Section~\ref{hierarchy}.  For brevity we haven't included the \texttt{\$d}
restrictions or \texttt{\$f} hypotheses for these theorems; when you are
uncertain consult the \texttt{set.mm} database.

We start with \texttt{syl} (principle of the syllogism).
In \textit{Principia Mathematica}
Whitehead and Russell call this ``the principle of the
syllogism... because... the syllogism in Barbara is derived from them''
\cite[quote after Theorem *2.06 p.~101]{PM}.
Some authors call this law a ``hypothetical syllogism.''
As of 2019 \texttt{syl} is the most commonly referenced proven
assertion in the \texttt{set.mm} database.\footnote{
The Metamath program command \texttt{show usage}
shows the number of references.
On 2019-04-29 (commit 71cbbbdb387e) \texttt{syl} was directly referenced
10,819 times. The second most commonly referenced proven assertion
was \texttt{eqid}, which was directly referenced 7,738 times.
}

\vskip 2ex
\noindent Theorem syl (principle of the syllogism)\index{Syllogism}%
\index{\texttt{syl}}\label{syl}.

\vskip 0.5ex
\setbox\startprefix=\hbox{\tt \ \ syl.1\ \$e\ }
\setbox\contprefix=\hbox{\tt \ \ \ \ \ \ \ \ \ \ \ }
\startm
\m{\vdash}\m{(}\m{\varphi}\m{ \rightarrow }\m{\psi}\m{)}
\endm
\setbox\startprefix=\hbox{\tt \ \ syl.2\ \$e\ }
\setbox\contprefix=\hbox{\tt \ \ \ \ \ \ \ \ \ \ \ }
\startm
\m{\vdash}\m{(}\m{\psi}\m{ \rightarrow }\m{\chi}\m{)}
\endm
\setbox\startprefix=\hbox{\tt \ \ syl\ \$p\ }
\setbox\contprefix=\hbox{\tt \ \ \ \ \ \ \ \ \ }
\startm
\m{\vdash}\m{(}\m{\varphi}\m{ \rightarrow }\m{\chi}\m{)}
\endm
\vskip 2ex

The following theorem is not very deep but provides us with a notational device
that is frequently used.  It allows us to use the expression ``$A \in V$'' as
a compact way of saying that class $A$ exists, i.e.\ is a set.

\vskip 2ex
\noindent Two ways to say ``$A$ is a set'':  $A$ is a member of the universe
$V$ if and only if $A$ exists (i.e.\ there exists a set equal to $A$).
Theorem 6.9 of Quine, p. 43.

\vskip 0.5ex
\setbox\startprefix=\hbox{\tt \ \ isset\ \$p\ }
\setbox\contprefix=\hbox{\tt \ \ \ \ \ \ \ \ \ \ \ }
\startm
\m{\vdash}\m{(}\m{A}\m{\in}\m{{\rm V}}\m{\leftrightarrow}\m{\exists}\m{x}\m{\,x}\m{=}
\m{A}\m{)}
\endm
\vskip 1ex

Next we prove the axioms of standard ZF set theory that were missing from our
axiom system.  From our point of view they are theorems since they
can be derived from the other axioms.

\vskip 2ex
\noindent Axiom of Separation\index{Axiom of Separation}
(Aussonderung)\index{Aussonderung} proved from the other axioms of ZF set
theory.  Compare Exercise 4 of Takeuti and Zaring, p.~22.

\vskip 0.5ex
\setbox\startprefix=\hbox{\tt \ \ inex1.1\ \$e\ }
\setbox\contprefix=\hbox{\tt \ \ \ \ \ \ \ \ \ \ \ \ \ \ \ }
\startm
\m{\vdash}\m{A}\m{\in}\m{{\rm V}}
\endm
\setbox\startprefix=\hbox{\tt \ \ inex\ \$p\ }
\setbox\contprefix=\hbox{\tt \ \ \ \ \ \ \ \ \ \ \ \ \ }
\startm
\m{\vdash}\m{(}\m{A}\m{\cap}\m{B}\m{)}\m{\in}\m{{\rm V}}
\endm
\vskip 1ex

\noindent Axiom of the Null Set\index{Axiom of the Null Set} proved from the
other axioms of ZF set theory. Corollary 5.16 of Takeuti and Zaring, p.~20.

\vskip 0.5ex
\setbox\startprefix=\hbox{\tt \ \ 0ex\ \$p\ }
\setbox\contprefix=\hbox{\tt \ \ \ \ \ \ \ \ \ \ \ \ }
\startm
\m{\vdash}\m{\varnothing}\m{\in}\m{{\rm V}}
\endm
\vskip 1ex

\noindent The Axiom of Pairing\index{Axiom of Pairing} proved from the other
axioms of ZF set theory.  Theorem 7.13 of Quine, p.~51.
\vskip 0.5ex
\setbox\startprefix=\hbox{\tt \ \ prex\ \$p\ }
\setbox\contprefix=\hbox{\tt \ \ \ \ \ \ \ \ \ \ \ \ \ \ }
\startm
\m{\vdash}\m{\{}\m{A}\m{,}\m{B}\m{\}}\m{\in}\m{{\rm V}}
\endm
\vskip 2ex

Next we will list some famous or important theorems that are proved in
the \texttt{set.mm} database.  None of them except \texttt{omex}
require the Axiom of Infinity, as you can verify with the \texttt{show
trace{\char`\_}back} Metamath command.

\vskip 2ex
\noindent The resolution of Russell's paradox\index{Russell's paradox}.  There
exists no set containing the set of all sets which are not members of
themselves.  Proposition 4.14 of Takeuti and Zaring, p.~14.

\vskip 0.5ex
\setbox\startprefix=\hbox{\tt \ \ ru\ \$p\ }
\setbox\contprefix=\hbox{\tt \ \ \ \ \ \ \ \ }
\startm
\m{\vdash}\m{\lnot}\m{\exists}\m{x}\m{\,x}\m{=}\m{\{}\m{y}\m{|}\m{\lnot}\m{y}
\m{\in}\m{y}\m{\}}
\endm
\vskip 1ex

\noindent Cantor's theorem\index{Cantor's theorem}.  No set can be mapped onto
its power set.  Compare Theorem 6B(b) of Enderton, p.~132.

\vskip 0.5ex
\setbox\startprefix=\hbox{\tt \ \ canth.1\ \$e\ }
\setbox\contprefix=\hbox{\tt \ \ \ \ \ \ \ \ \ \ \ \ \ }
\startm
\m{\vdash}\m{A}\m{\in}\m{{\rm V}}
\endm
\setbox\startprefix=\hbox{\tt \ \ canth\ \$p\ }
\setbox\contprefix=\hbox{\tt \ \ \ \ \ \ \ \ \ \ \ }
\startm
\m{\vdash}\m{\lnot}\m{F}\m{:}\m{A}\m{\raisebox{.5ex}{${\textstyle{\:}_{
\mbox{\footnotesize\rm {\ }}}}\atop{\textstyle{\longrightarrow}\atop{
\textstyle{}^{\mbox{\footnotesize\rm onto}}}}$}}\m{{\cal P}}\m{A}
\endm
\vskip 1ex

\noindent The Burali-Forti paradox\index{Burali-Forti paradox}.  No set
contains all ordinal numbers. Enderton, p.~194.  (Burali-Forti was one person,
not two.)

\vskip 0.5ex
\setbox\startprefix=\hbox{\tt \ \ onprc\ \$p\ }
\setbox\contprefix=\hbox{\tt \ \ \ \ \ \ \ \ \ \ \ \ }
\startm
\m{\vdash}\m{\lnot}\m{\mbox{\rm On}}\m{\in}\m{{\rm V}}
\endm
\vskip 1ex

\noindent Peano's postulates\index{Peano's postulates} for arithmetic.
Proposition 7.30 of Takeuti and Zaring, pp.~42--43.  The objects being
described are the members of $\omega$ i.e.\ the natural numbers 0, 1,
2,\ldots.  The successor\index{successor} operation suc means ``plus
one.''  \texttt{peano1} says that 0 (which is defined as the empty set)
is a natural number.  \texttt{peano2} says that if $A$ is a natural
number, so is $A+1$.  \texttt{peano3} says that 0 is not the successor
of any natural number.  \texttt{peano4} says that two natural numbers
are equal if and only if their successors are equal.  \texttt{peano5} is
essentially the same as mathematical induction.

\vskip 1ex
\setbox\startprefix=\hbox{\tt \ \ peano1\ \$p\ }
\setbox\contprefix=\hbox{\tt \ \ \ \ \ \ \ \ \ \ \ \ }
\startm
\m{\vdash}\m{\varnothing}\m{\in}\m{\omega}
\endm
\vskip 1.5ex

\setbox\startprefix=\hbox{\tt \ \ peano2\ \$p\ }
\setbox\contprefix=\hbox{\tt \ \ \ \ \ \ \ \ \ \ \ \ }
\startm
\m{\vdash}\m{(}\m{A}\m{\in}\m{\omega}\m{\rightarrow}\m{{\rm suc}}\m{A}\m{\in}%
\m{\omega}\m{)}
\endm
\vskip 1.5ex

\setbox\startprefix=\hbox{\tt \ \ peano3\ \$p\ }
\setbox\contprefix=\hbox{\tt \ \ \ \ \ \ \ \ \ \ \ \ }
\startm
\m{\vdash}\m{(}\m{A}\m{\in}\m{\omega}\m{\rightarrow}\m{\lnot}\m{{\rm suc}}%
\m{A}\m{=}\m{\varnothing}\m{)}
\endm
\vskip 1.5ex

\setbox\startprefix=\hbox{\tt \ \ peano4\ \$p\ }
\setbox\contprefix=\hbox{\tt \ \ \ \ \ \ \ \ \ \ \ \ }
\startm
\m{\vdash}\m{(}\m{(}\m{A}\m{\in}\m{\omega}\m{\wedge}\m{B}\m{\in}\m{\omega}%
\m{)}\m{\rightarrow}\m{(}\m{{\rm suc}}\m{A}\m{=}\m{{\rm suc}}\m{B}\m{%
\leftrightarrow}\m{A}\m{=}\m{B}\m{)}\m{)}
\endm
\vskip 1.5ex

\setbox\startprefix=\hbox{\tt \ \ peano5\ \$p\ }
\setbox\contprefix=\hbox{\tt \ \ \ \ \ \ \ \ \ \ \ \ }
\startm
\m{\vdash}\m{(}\m{(}\m{\varnothing}\m{\in}\m{A}\m{\wedge}\m{\forall}\m{x}\m{%
\in}\m{\omega}\m{(}\m{x}\m{\in}\m{A}\m{\rightarrow}\m{{\rm suc}}\m{x}\m{\in}%
\m{A}\m{)}\m{)}\m{\rightarrow}\m{\omega}\m{\subseteq}\m{A}\m{)}
\endm
\vskip 1.5ex

\noindent Finite Induction (mathematical induction).\index{finite
induction}\index{mathematical induction} The first hypothesis is the
basis and the second is the induction hypothesis.  Theorem Schema 22 of
Suppes, p.~136.

\vskip 0.5ex
\setbox\startprefix=\hbox{\tt \ \ findes.1\ \$e\ }
\setbox\contprefix=\hbox{\tt \ \ \ \ \ \ \ \ \ \ \ \ \ \ }
\startm
\m{\vdash}\m{[}\m{\varnothing}\m{/}\m{x}\m{]}\m{\varphi}
\endm
\setbox\startprefix=\hbox{\tt \ \ findes.2\ \$e\ }
\setbox\contprefix=\hbox{\tt \ \ \ \ \ \ \ \ \ \ \ \ \ \ }
\startm
\m{\vdash}\m{(}\m{x}\m{\in}\m{\omega}\m{\rightarrow}\m{(}\m{\varphi}\m{%
\rightarrow}\m{[}\m{{\rm suc}}\m{x}\m{/}\m{x}\m{]}\m{\varphi}\m{)}\m{)}
\endm
\setbox\startprefix=\hbox{\tt \ \ findes\ \$p\ }
\setbox\contprefix=\hbox{\tt \ \ \ \ \ \ \ \ \ \ \ \ }
\startm
\m{\vdash}\m{(}\m{x}\m{\in}\m{\omega}\m{\rightarrow}\m{\varphi}\m{)}
\endm
\vskip 1ex

\noindent Transfinite Induction with explicit substitution.  The first
hypothesis is the basis, the second is the induction hypothesis for
successors, and the third is the induction hypothesis for limit
ordinals.  Theorem Schema 4 of Suppes, p. 197.

\vskip 0.5ex
\setbox\startprefix=\hbox{\tt \ \ tfindes.1\ \$e\ }
\setbox\contprefix=\hbox{\tt \ \ \ \ \ \ \ \ \ \ \ \ \ \ \ }
\startm
\m{\vdash}\m{[}\m{\varnothing}\m{/}\m{x}\m{]}\m{\varphi}
\endm
\setbox\startprefix=\hbox{\tt \ \ tfindes.2\ \$e\ }
\setbox\contprefix=\hbox{\tt \ \ \ \ \ \ \ \ \ \ \ \ \ \ \ }
\startm
\m{\vdash}\m{(}\m{x}\m{\in}\m{{\rm On}}\m{\rightarrow}\m{(}\m{\varphi}\m{%
\rightarrow}\m{[}\m{{\rm suc}}\m{x}\m{/}\m{x}\m{]}\m{\varphi}\m{)}\m{)}
\endm
\setbox\startprefix=\hbox{\tt \ \ tfindes.3\ \$e\ }
\setbox\contprefix=\hbox{\tt \ \ \ \ \ \ \ \ \ \ \ \ \ \ \ }
\startm
\m{\vdash}\m{(}\m{{\rm Lim}}\m{y}\m{\rightarrow}\m{(}\m{\forall}\m{x}\m{\in}%
\m{y}\m{\varphi}\m{\rightarrow}\m{[}\m{y}\m{/}\m{x}\m{]}\m{\varphi}\m{)}\m{)}
\endm
\setbox\startprefix=\hbox{\tt \ \ tfindes\ \$p\ }
\setbox\contprefix=\hbox{\tt \ \ \ \ \ \ \ \ \ \ \ \ \ }
\startm
\m{\vdash}\m{(}\m{x}\m{\in}\m{{\rm On}}\m{\rightarrow}\m{\varphi}\m{)}
\endm
\vskip 1ex

\noindent Principle of Transfinite Recursion.\index{transfinite
recursion} Theorem 7.41 of Takeuti and Zaring, p.~47.  Transfinite
recursion is the key theorem that allows arithmetic of ordinals to be
rigorously defined, and has many other important uses as well.
Hypotheses \texttt{tfr.1} and \texttt{tfr.2} specify a certain (proper)
class $ F$.  The complicated definition of $ F$ is not important in
itself; what is important is that there be such an $ F$ with the
required properties, and we show this by displaying $ F$ explicitly.
\texttt{tfr1} states that $ F$ is a function whose domain is the set of
ordinal numbers.  \texttt{tfr2} states that any value of $ F$ is
completely determined by its previous values and the values of an
auxiliary function, $G$.  \texttt{tfr3} states that $ F$ is unique,
i.e.\ it is the only function that satisfies \texttt{tfr1} and
\texttt{tfr2}.  Note that $ f$ is an individual variable like $x$ and
$y$; it is just a mnemonic to remind us that $A$ is a collection of
functions.

\vskip 0.5ex
\setbox\startprefix=\hbox{\tt \ \ tfr.1\ \$e\ }
\setbox\contprefix=\hbox{\tt \ \ \ \ \ \ \ \ \ \ \ }
\startm
\m{\vdash}\m{A}\m{=}\m{\{}\m{f}\m{|}\m{\exists}\m{x}\m{\in}\m{{\rm On}}\m{(}%
\m{f}\m{{\rm Fn}}\m{x}\m{\wedge}\m{\forall}\m{y}\m{\in}\m{x}\m{(}\m{f}\m{`}%
\m{y}\m{)}\m{=}\m{(}\m{G}\m{`}\m{(}\m{f}\m{\restriction}\m{y}\m{)}\m{)}\m{)}%
\m{\}}
\endm
\setbox\startprefix=\hbox{\tt \ \ tfr.2\ \$e\ }
\setbox\contprefix=\hbox{\tt \ \ \ \ \ \ \ \ \ \ \ }
\startm
\m{\vdash}\m{F}\m{=}\m{\bigcup}\m{A}
\endm
\setbox\startprefix=\hbox{\tt \ \ tfr1\ \$p\ }
\setbox\contprefix=\hbox{\tt \ \ \ \ \ \ \ \ \ \ }
\startm
\m{\vdash}\m{F}\m{{\rm Fn}}\m{{\rm On}}
\endm
\setbox\startprefix=\hbox{\tt \ \ tfr2\ \$p\ }
\setbox\contprefix=\hbox{\tt \ \ \ \ \ \ \ \ \ \ }
\startm
\m{\vdash}\m{(}\m{z}\m{\in}\m{{\rm On}}\m{\rightarrow}\m{(}\m{F}\m{`}\m{z}%
\m{)}\m{=}\m{(}\m{G}\m{`}\m{(}\m{F}\m{\restriction}\m{z}\m{)}\m{)}\m{)}
\endm
\setbox\startprefix=\hbox{\tt \ \ tfr3\ \$p\ }
\setbox\contprefix=\hbox{\tt \ \ \ \ \ \ \ \ \ \ }
\startm
\m{\vdash}\m{(}\m{(}\m{B}\m{{\rm Fn}}\m{{\rm On}}\m{\wedge}\m{\forall}\m{x}\m{%
\in}\m{{\rm On}}\m{(}\m{B}\m{`}\m{x}\m{)}\m{=}\m{(}\m{G}\m{`}\m{(}\m{B}\m{%
\restriction}\m{x}\m{)}\m{)}\m{)}\m{\rightarrow}\m{B}\m{=}\m{F}\m{)}
\endm
\vskip 1ex

\noindent The existence of omega (the class of natural numbers).\index{natural
number}\index{omega ($\omega$)}\index{Axiom of Infinity}  Axiom 7 of Takeuti
and Zaring, p.~43.  (This is the only theorem in this section requiring the
Axiom of Infinity.)

\vskip 0.5ex
\setbox\startprefix=\hbox{\tt \
\ omex\ \$p\ }
\setbox\contprefix=\hbox{\tt \ \ \ \ \ \ \ \ \ \ }
\startm
\m{\vdash}\m{\omega}\m{\in}\m{{\rm V}}
\endm
%\vskip 2ex


\section{Axioms for Real and Complex Numbers}\label{real}
\index{real and complex numbers!axioms for}

This section presents the axioms for real and complex numbers, along
with some commentary about them.  Analysis
textbooks implicitly or explicitly use these axioms or their equivalents
as their starting point.  In the database \texttt{set.mm}, we define real
and complex numbers as (rather complicated) specific sets and derive these
axioms as {\em theorems} from the axioms of ZF set theory, using a method
called Dedekind cuts.  We omit the details of this construction, which you can
follow if you wish using the \texttt{set.mm} database in conjunction with the
textbooks referenced therein.

Once we prove those theorems, we then restate these proven theorems as axioms.
This lets us easily identify which axioms are needed for a particular complex number proof, without the obfuscation of the set theory used to derive them.
As a result,
the construction is actually unimportant other
than to show that sets exist that satisfy the axioms, and thus that the axioms
are consistent if ZF set theory is consistent.  When working with real numbers
you can think of them as being the actual sets resulting
from the construction (for definiteness), or you can
think of them as otherwise unspecified sets that happen to satisfy the axioms.
The derivation is not easy, but the fact that it works is quite remarkable
and lends support to the idea that ZFC set theory is all we need to
provide a foundation for essentially all of mathematics.

\needspace{3\baselineskip}
\subsection{The Axioms for Real and Complex Numbers Themselves}\label{realactual}

For the axioms we are given (or postulate) 8 classes:  $\mathbb{C}$ (the
set of complex numbers), $\mathbb{R}$ (the set of real numbers, a subset
of $\mathbb{C}$), $0$ (zero), $1$ (one), $i$ (square root of
$-1$), $+$ (plus), $\cdot$ (times), and
$<_{\mathbb{R}}$ (less than for just the real numbers).
Subtraction and division are defined terms and are not part of the
axioms; for their definitions see \texttt{set.mm}.

Note that the notation $(A+B)$ (and similarly $(A\cdot B)$) specifies a class
called an {\em operation},\index{operation} and is the function value of the
class $+$ at ordered pair $\langle A,B \rangle$.  An operation is defined by
statement \texttt{df-opr} on p.~\pageref{dfopr}.
The notation $A <_{\mathbb{R}} B$ specifies a
wff called a {\em binary relation}\index{binary relation} and means $\langle A,B \rangle \in \,<_{\mathbb{R}}$, as defined by statement \texttt{df-br} on p.~\pageref{dfbr}.

Our set of 8 given classes is assumed to satisfy the following 22 axioms
(in the axioms listed below, $<$ really means $<_{\mathbb{R}}$).

\vskip 2ex

\noindent 1. The real numbers are a subset of the complex numbers.

%\vskip 0.5ex
\setbox\startprefix=\hbox{\tt \ \ ax-resscn\ \$p\ }
\setbox\contprefix=\hbox{\tt \ \ \ \ \ \ \ \ \ \ \ \ \ \ }
\startm
\m{\vdash}\m{\mathbb{R}}\m{\subseteq}\m{\mathbb{C}}
\endm
%\vskip 1ex

\noindent 2. One is a complex number.

%\vskip 0.5ex
\setbox\startprefix=\hbox{\tt \ \ ax-1cn\ \$p\ }
\setbox\contprefix=\hbox{\tt \ \ \ \ \ \ \ \ \ \ \ }
\startm
\m{\vdash}\m{1}\m{\in}\m{\mathbb{C}}
\endm
%\vskip 1ex

\noindent 3. The imaginary number $i$ is a complex number.

%\vskip 0.5ex
\setbox\startprefix=\hbox{\tt \ \ ax-icn\ \$p\ }
\setbox\contprefix=\hbox{\tt \ \ \ \ \ \ \ \ \ \ \ }
\startm
\m{\vdash}\m{i}\m{\in}\m{\mathbb{C}}
\endm
%\vskip 1ex

\noindent 4. Complex numbers are closed under addition.

%\vskip 0.5ex
\setbox\startprefix=\hbox{\tt \ \ ax-addcl\ \$p\ }
\setbox\contprefix=\hbox{\tt \ \ \ \ \ \ \ \ \ \ \ \ \ }
\startm
\m{\vdash}\m{(}\m{(}\m{A}\m{\in}\m{\mathbb{C}}\m{\wedge}\m{B}\m{\in}\m{\mathbb{C}}%
\m{)}\m{\rightarrow}\m{(}\m{A}\m{+}\m{B}\m{)}\m{\in}\m{\mathbb{C}}\m{)}
\endm
%\vskip 1ex

\noindent 5. Real numbers are closed under addition.

%\vskip 0.5ex
\setbox\startprefix=\hbox{\tt \ \ ax-addrcl\ \$p\ }
\setbox\contprefix=\hbox{\tt \ \ \ \ \ \ \ \ \ \ \ \ \ \ }
\startm
\m{\vdash}\m{(}\m{(}\m{A}\m{\in}\m{\mathbb{R}}\m{\wedge}\m{B}\m{\in}\m{\mathbb{R}}%
\m{)}\m{\rightarrow}\m{(}\m{A}\m{+}\m{B}\m{)}\m{\in}\m{\mathbb{R}}\m{)}
\endm
%\vskip 1ex

\noindent 6. Complex numbers are closed under multiplication.

%\vskip 0.5ex
\setbox\startprefix=\hbox{\tt \ \ ax-mulcl\ \$p\ }
\setbox\contprefix=\hbox{\tt \ \ \ \ \ \ \ \ \ \ \ \ \ }
\startm
\m{\vdash}\m{(}\m{(}\m{A}\m{\in}\m{\mathbb{C}}\m{\wedge}\m{B}\m{\in}\m{\mathbb{C}}%
\m{)}\m{\rightarrow}\m{(}\m{A}\m{\cdot}\m{B}\m{)}\m{\in}\m{\mathbb{C}}\m{)}
\endm
%\vskip 1ex

\noindent 7. Real numbers are closed under multiplication.

%\vskip 0.5ex
\setbox\startprefix=\hbox{\tt \ \ ax-mulrcl\ \$p\ }
\setbox\contprefix=\hbox{\tt \ \ \ \ \ \ \ \ \ \ \ \ \ \ }
\startm
\m{\vdash}\m{(}\m{(}\m{A}\m{\in}\m{\mathbb{R}}\m{\wedge}\m{B}\m{\in}\m{\mathbb{R}}%
\m{)}\m{\rightarrow}\m{(}\m{A}\m{\cdot}\m{B}\m{)}\m{\in}\m{\mathbb{R}}\m{)}
\endm
%\vskip 1ex

\noindent 8. Multiplication of complex numbers is commutative.

%\vskip 0.5ex
\setbox\startprefix=\hbox{\tt \ \ ax-mulcom\ \$p\ }
\setbox\contprefix=\hbox{\tt \ \ \ \ \ \ \ \ \ \ \ \ \ \ }
\startm
\m{\vdash}\m{(}\m{(}\m{A}\m{\in}\m{\mathbb{C}}\m{\wedge}\m{B}\m{\in}\m{\mathbb{C}}%
\m{)}\m{\rightarrow}\m{(}\m{A}\m{\cdot}\m{B}\m{)}\m{=}\m{(}\m{B}\m{\cdot}\m{A}%
\m{)}\m{)}
\endm
%\vskip 1ex

\noindent 9. Addition of complex numbers is associative.

%\vskip 0.5ex
\setbox\startprefix=\hbox{\tt \ \ ax-addass\ \$p\ }
\setbox\contprefix=\hbox{\tt \ \ \ \ \ \ \ \ \ \ \ \ \ \ }
\startm
\m{\vdash}\m{(}\m{(}\m{A}\m{\in}\m{\mathbb{C}}\m{\wedge}\m{B}\m{\in}\m{\mathbb{C}}%
\m{\wedge}\m{C}\m{\in}\m{\mathbb{C}}\m{)}\m{\rightarrow}\m{(}\m{(}\m{A}\m{+}%
\m{B}\m{)}\m{+}\m{C}\m{)}\m{=}\m{(}\m{A}\m{+}\m{(}\m{B}\m{+}\m{C}\m{)}\m{)}%
\m{)}
\endm
%\vskip 1ex

\noindent 10. Multiplication of complex numbers is associative.

%\vskip 0.5ex
\setbox\startprefix=\hbox{\tt \ \ ax-mulass\ \$p\ }
\setbox\contprefix=\hbox{\tt \ \ \ \ \ \ \ \ \ \ \ \ \ \ }
\startm
\m{\vdash}\m{(}\m{(}\m{A}\m{\in}\m{\mathbb{C}}\m{\wedge}\m{B}\m{\in}\m{\mathbb{C}}%
\m{\wedge}\m{C}\m{\in}\m{\mathbb{C}}\m{)}\m{\rightarrow}\m{(}\m{(}\m{A}\m{\cdot}%
\m{B}\m{)}\m{\cdot}\m{C}\m{)}\m{=}\m{(}\m{A}\m{\cdot}\m{(}\m{B}\m{\cdot}\m{C}%
\m{)}\m{)}\m{)}
\endm
%\vskip 1ex

\noindent 11. Multiplication distributes over addition for complex numbers.

%\vskip 0.5ex
\setbox\startprefix=\hbox{\tt \ \ ax-distr\ \$p\ }
\setbox\contprefix=\hbox{\tt \ \ \ \ \ \ \ \ \ \ \ \ \ }
\startm
\m{\vdash}\m{(}\m{(}\m{A}\m{\in}\m{\mathbb{C}}\m{\wedge}\m{B}\m{\in}\m{\mathbb{C}}%
\m{\wedge}\m{C}\m{\in}\m{\mathbb{C}}\m{)}\m{\rightarrow}\m{(}\m{A}\m{\cdot}\m{(}%
\m{B}\m{+}\m{C}\m{)}\m{)}\m{=}\m{(}\m{(}\m{A}\m{\cdot}\m{B}\m{)}\m{+}\m{(}%
\m{A}\m{\cdot}\m{C}\m{)}\m{)}\m{)}
\endm
%\vskip 1ex

\noindent 12. The square of $i$ equals $-1$ (expressed as $i$-squared plus 1 is
0).

%\vskip 0.5ex
\setbox\startprefix=\hbox{\tt \ \ ax-i2m1\ \$p\ }
\setbox\contprefix=\hbox{\tt \ \ \ \ \ \ \ \ \ \ \ \ }
\startm
\m{\vdash}\m{(}\m{(}\m{i}\m{\cdot}\m{i}\m{)}\m{+}\m{1}\m{)}\m{=}\m{0}
\endm
%\vskip 1ex

\noindent 13. One and zero are distinct.

%\vskip 0.5ex
\setbox\startprefix=\hbox{\tt \ \ ax-1ne0\ \$p\ }
\setbox\contprefix=\hbox{\tt \ \ \ \ \ \ \ \ \ \ \ \ }
\startm
\m{\vdash}\m{1}\m{\ne}\m{0}
\endm
%\vskip 1ex

\noindent 14. One is an identity element for real multiplication.

%\vskip 0.5ex
\setbox\startprefix=\hbox{\tt \ \ ax-1rid\ \$p\ }
\setbox\contprefix=\hbox{\tt \ \ \ \ \ \ \ \ \ \ \ }
\startm
\m{\vdash}\m{(}\m{A}\m{\in}\m{\mathbb{R}}\m{\rightarrow}\m{(}\m{A}\m{\cdot}\m{1}%
\m{)}\m{=}\m{A}\m{)}
\endm
%\vskip 1ex

\noindent 15. Every real number has a negative.

%\vskip 0.5ex
\setbox\startprefix=\hbox{\tt \ \ ax-rnegex\ \$p\ }
\setbox\contprefix=\hbox{\tt \ \ \ \ \ \ \ \ \ \ \ \ \ \ }
\startm
\m{\vdash}\m{(}\m{A}\m{\in}\m{\mathbb{R}}\m{\rightarrow}\m{\exists}\m{x}\m{\in}%
\m{\mathbb{R}}\m{(}\m{A}\m{+}\m{x}\m{)}\m{=}\m{0}\m{)}
\endm
%\vskip 1ex

\noindent 16. Every nonzero real number has a reciprocal.

%\vskip 0.5ex
\setbox\startprefix=\hbox{\tt \ \ ax-rrecex\ \$p\ }
\setbox\contprefix=\hbox{\tt \ \ \ \ \ \ \ \ \ \ \ \ \ \ }
\startm
\m{\vdash}\m{(}\m{A}\m{\in}\m{\mathbb{R}}\m{\rightarrow}\m{(}\m{A}\m{\ne}\m{0}%
\m{\rightarrow}\m{\exists}\m{x}\m{\in}\m{\mathbb{R}}\m{(}\m{A}\m{\cdot}%
\m{x}\m{)}\m{=}\m{1}\m{)}\m{)}
\endm
%\vskip 1ex

\noindent 17. A complex number can be expressed in terms of two reals.

%\vskip 0.5ex
\setbox\startprefix=\hbox{\tt \ \ ax-cnre\ \$p\ }
\setbox\contprefix=\hbox{\tt \ \ \ \ \ \ \ \ \ \ \ \ }
\startm
\m{\vdash}\m{(}\m{A}\m{\in}\m{\mathbb{C}}\m{\rightarrow}\m{\exists}\m{x}\m{\in}%
\m{\mathbb{R}}\m{\exists}\m{y}\m{\in}\m{\mathbb{R}}\m{A}\m{=}\m{(}\m{x}\m{+}\m{(}%
\m{y}\m{\cdot}\m{i}\m{)}\m{)}\m{)}
\endm
%\vskip 1ex

\noindent 18. Ordering on reals satisfies strict trichotomy.

%\vskip 0.5ex
\setbox\startprefix=\hbox{\tt \ \ ax-pre-lttri\ \$p\ }
\setbox\contprefix=\hbox{\tt \ \ \ \ \ \ \ \ \ \ \ \ \ }
\startm
\m{\vdash}\m{(}\m{(}\m{A}\m{\in}\m{\mathbb{R}}\m{\wedge}\m{B}\m{\in}\m{\mathbb{R}}%
\m{)}\m{\rightarrow}\m{(}\m{A}\m{<}\m{B}\m{\leftrightarrow}\m{\lnot}\m{(}\m{A}%
\m{=}\m{B}\m{\vee}\m{B}\m{<}\m{A}\m{)}\m{)}\m{)}
\endm
%\vskip 1ex

\noindent 19. Ordering on reals is transitive.

%\vskip 0.5ex
\setbox\startprefix=\hbox{\tt \ \ ax-pre-lttrn\ \$p\ }
\setbox\contprefix=\hbox{\tt \ \ \ \ \ \ \ \ \ \ \ \ \ }
\startm
\m{\vdash}\m{(}\m{(}\m{A}\m{\in}\m{\mathbb{R}}\m{\wedge}\m{B}\m{\in}\m{\mathbb{R}}%
\m{\wedge}\m{C}\m{\in}\m{\mathbb{R}}\m{)}\m{\rightarrow}\m{(}\m{(}\m{A}\m{<}%
\m{B}\m{\wedge}\m{B}\m{<}\m{C}\m{)}\m{\rightarrow}\m{A}\m{<}\m{C}\m{)}\m{)}
\endm
%\vskip 1ex

\noindent 20. Ordering on reals is preserved after addition to both sides.

%\vskip 0.5ex
\setbox\startprefix=\hbox{\tt \ \ ax-pre-ltadd\ \$p\ }
\setbox\contprefix=\hbox{\tt \ \ \ \ \ \ \ \ \ \ \ \ \ }
\startm
\m{\vdash}\m{(}\m{(}\m{A}\m{\in}\m{\mathbb{R}}\m{\wedge}\m{B}\m{\in}\m{\mathbb{R}}%
\m{\wedge}\m{C}\m{\in}\m{\mathbb{R}}\m{)}\m{\rightarrow}\m{(}\m{A}\m{<}\m{B}\m{%
\rightarrow}\m{(}\m{C}\m{+}\m{A}\m{)}\m{<}\m{(}\m{C}\m{+}\m{B}\m{)}\m{)}\m{)}
\endm
%\vskip 1ex

\noindent 21. The product of two positive reals is positive.

%\vskip 0.5ex
\setbox\startprefix=\hbox{\tt \ \ ax-pre-mulgt0\ \$p\ }
\setbox\contprefix=\hbox{\tt \ \ \ \ \ \ \ \ \ \ \ \ \ \ }
\startm
\m{\vdash}\m{(}\m{(}\m{A}\m{\in}\m{\mathbb{R}}\m{\wedge}\m{B}\m{\in}\m{\mathbb{R}}%
\m{)}\m{\rightarrow}\m{(}\m{(}\m{0}\m{<}\m{A}\m{\wedge}\m{0}%
\m{<}\m{B}\m{)}\m{\rightarrow}\m{0}\m{<}\m{(}\m{A}\m{\cdot}\m{B}\m{)}%
\m{)}\m{)}
\endm
%\vskip 1ex

\noindent 22. A non-empty, bounded-above set of reals has a supremum.

%\vskip 0.5ex
\setbox\startprefix=\hbox{\tt \ \ ax-pre-sup\ \$p\ }
\setbox\contprefix=\hbox{\tt \ \ \ \ \ \ \ \ \ \ \ }
\startm
\m{\vdash}\m{(}\m{(}\m{A}\m{\subseteq}\m{\mathbb{R}}\m{\wedge}\m{A}\m{\ne}\m{%
\varnothing}\m{\wedge}\m{\exists}\m{x}\m{\in}\m{\mathbb{R}}\m{\forall}\m{y}\m{%
\in}\m{A}\m{\,y}\m{<}\m{x}\m{)}\m{\rightarrow}\m{\exists}\m{x}\m{\in}\m{%
\mathbb{R}}\m{(}\m{\forall}\m{y}\m{\in}\m{A}\m{\lnot}\m{x}\m{<}\m{y}\m{\wedge}\m{%
\forall}\m{y}\m{\in}\m{\mathbb{R}}\m{(}\m{y}\m{<}\m{x}\m{\rightarrow}\m{\exists}%
\m{z}\m{\in}\m{A}\m{\,y}\m{<}\m{z}\m{)}\m{)}\m{)}
\endm

% NOTE: The \m{...} expressions above could be represented as
% $ \vdash ( ( A \subseteq \mathbb{R} \wedge A \ne \varnothing \wedge \exists x \in \mathbb{R} \forall y \in A \,y < x ) \rightarrow \exists x \in \mathbb{R} ( \forall y \in A \lnot x < y \wedge \forall y \in \mathbb{R} ( y < x \rightarrow \exists z \in A \,y < z ) ) ) $

\vskip 2ex

This completes the set of axioms for real and complex numbers.  You may
wish to look at how subtraction, division, and decimal numbers
are defined in \texttt{set.mm}, and for fun look at the proof of $2+
2 = 4$ (theorem \texttt{2p2e4} in \texttt{set.mm})
as discussed in section \ref{2p2e4}.

In \texttt{set.mm} we define the non-negative integers $\mathbb{N}$, the integers
$\mathbb{Z}$, and the rationals $\mathbb{Q}$ as subsets of $\mathbb{R}$.  This leads
to the nice inclusion $\mathbb{N} \subseteq \mathbb{Z} \subseteq \mathbb{Q} \subseteq
\mathbb{R} \subseteq \mathbb{C}$, giving us a uniform framework in which, for
example, a property such as commutativity of complex number addition
automatically applies to integers.  The natural numbers $\mathbb{N}$
are different from the set $\omega$ we defined earlier, but both satisfy
Peano's postulates.

\subsection{Complex Number Axioms in Analysis Texts}

Most analysis texts construct complex numbers as ordered pairs of reals,
leading to construction-dependent properties that satisfy these axioms
but are not stated in their pure form.  (This is also done in
\texttt{set.mm} but our axioms are extracted from that construction.)
Other texts will simply state that $\mathbb{R}$ is a ``complete ordered
subfield of $\mathbb{C}$,'' leading to redundant axioms when this phrase
is completely expanded out.  In fact I have not seen a text with the
axioms in the explicit form above.
None of these axioms is unique individually, but this carefully worked out
collection of axioms is the result of years of work
by the Metamath community.

\subsection{Eliminating Unnecessary Complex Number Axioms}

We once had more axioms for real and complex numbers, but over years of time
we (the Metamath community)
have found ways to eliminate them (by proving them from other axioms)
or weaken them (by making weaker claims without reducing what
can be proved).
In particular, here are statements that used to be complex number
axioms but have since been formally proven (with Metamath) to be redundant:

\begin{itemize}
\item
  $\mathbb{C} \in V$.
  At one time this was listed as a ``complex number axiom.''
  However, this is not properly speaking a complex number axiom,
  and in any case its proof uses axioms of set theory.
  Proven redundant by Mario Carneiro\index{Carneiro, Mario} on
  17-Nov-2014 (see \texttt{axcnex}).
\item
  $((A \in \mathbb{C} \land B \in \mathbb{C}$) $\rightarrow$
  $(A + B) = (B + A))$.
  Proved redundant by Eric Schmidt\index{Schmidt, Eric} on 19-Jun-2012,
  and formalized by Scott Fenton\index{Fenton, Scott} on 3-Jan-2013
  (see \texttt{addcom}).
\item
  $(A \in \mathbb{C} \rightarrow (A + 0) = A)$.
  Proved redundant by Eric Schmidt on 19-Jun-2012,
  and formalized by Scott Fenton on 3-Jan-2013
  (see \texttt{addid1}).
\item
  $(A \in \mathbb{C} \rightarrow \exists x \in \mathbb{C} (A + x) = 0)$.
  Proved redundant by Eric Schmidt and formalized on 21-May-2007
  (see \texttt{cnegex}).
\item
  $((A \in \mathbb{C} \land A \ne 0) \rightarrow \exists x \in \mathbb{C} (A \cdot x) = 1)$.
  Proved redundant by Eric Schmidt and formalized on 22-May-2007
  (see \texttt{recex}).
\item
  $0 \in \mathbb{R}$.
  Proved redundant by Eric Schmidt on 19-Feb-2005 and formalized 21-May-2007
  (see \texttt{0re}).
\end{itemize}

We could eliminate 0 as an axiomatic object by defining it as
$( ( i \cdot i ) + 1 )$
and replacing it with this expression throughout the axioms. If this
is done, axiom ax-i2m1 becomes redundant. However, the remaining axioms
would become longer and less intuitive.

Eric Schmidt's paper analyzing this axiom system \cite{Schmidt}
presented a proof that these remaining axioms,
with the possible exception of ax-mulcom, are independent of the others.
It is currently an open question if ax-mulcom is independent of the others.

\section{Two Plus Two Equals Four}\label{2p2e4}

Here is a proof that $2 + 2 = 4$, as proven in the theorem \texttt{2p2e4}
in the database \texttt{set.mm}.
This is a useful demonstration of what a Metamath proof can look like.
This proof may have more steps than you're used to, but each step is rigorously
proven all the way back to the axioms of logic and set theory.
This display was originally generated by the Metamath program
as an {\sc HTML} file.

In the table showing the proof ``Step'' is the sequential step number,
while its associated ``Expression'' is an expression that we have proved.
``Ref'' is the name of a theorem or axiom that justifies that expression,
and ``Hyp'' refers to previous steps (if any) that the theorem or axiom
needs so that we can use it.  Expressions are indented further than
the expressions that depend on them to show their interdependencies.

\begin{table}[!htbp]
\caption{Two plus two equals four}
\begin{tabular}{lllll}
\textbf{Step} & \textbf{Hyp} & \textbf{Ref} & \textbf{Expression} & \\
1  &       & df-2    & $ \; \; \vdash 2 = 1 + 1$  & \\
2  & 1     & oveq2i  & $ \; \vdash (2 + 2) = (2 + (1 + 1))$ & \\
3  &       & df-4    & $ \; \; \vdash 4 = (3 + 1)$ & \\
4  &       & df-3    & $ \; \; \; \vdash 3 = (2 + 1)$ & \\
5  & 4     & oveq1i  & $ \; \; \vdash (3 + 1) = ((2 + 1) + 1)$ & \\
6  &       & 2cn     & $ \; \; \; \vdash 2 \in \mathbb{C}$ & \\
7  &       & ax-1cn  & $ \; \; \; \vdash 1 \in \mathbb{C}$ & \\
8  & 6,7,7 & addassi & $ \; \; \vdash ((2 + 1) + 1) = (2 + (1 + 1))$ & \\
9  & 3,5,8 & 3eqtri  & $ \; \vdash 4 = (2 + (1 + 1))$ & \\
10 & 2,9   & eqtr4i  & $ \vdash (2 + 2) = 4$ & \\
\end{tabular}
\end{table}

Step 1 says that we can assert that $2 = 1 + 1$ because it is
justified by \texttt{df-2}.
What is \texttt{df-2}?
It is simply the definition of $2$, which in our system is defined as being
equal to $1 + 1$.  This shows how we can use definitions in proofs.

Look at Step 2 of the proof. In the Ref column, we see that it references
a previously proved theorem, \texttt{oveq2i}.
It turns out that
theorem \texttt{oveq2i} requires a
hypothesis, and in the Hyp column of Step 2 we indicate that Step 1 will
satisfy (match) this hypothesis.
If we looked at \texttt{oveq2i}
we would find that it proves that given some hypothesis
$A = B$, we can prove that $( C F A ) = ( C F B )$.
If we use \texttt{oveq2i} and apply step 1's result as the hypothesis,
that will mean that $A = 2$ and $B = ( 1 + 1 )$ within this use of
\texttt{oveq2i}.
We are free to select any value of $C$ and $F$ (subject to syntax constraints),
so we are free to select $C = 2$ and $F = +$,
producing our desired result,
$ (2 + 2) = (2 + (1 + 1))$.

Step 2 is an example of substitution.
In the end, every step in every proof uses only this one substitution rule.
All the rules of logic, and all the axioms, are expressed so that
they can be used via this one substitution rule.
So once you master substitution, you can master every Metamath proof,
no exceptions.

Each step is clear and can be immediately checked.
In the {\sc HTML} display you can even click on each reference to see why it is
justified, making it easy to see why the proof works.

\section{Deduction}\label{deduction}

Strictly speaking,
a deduction (also called an inference) is a kind of statement that needs
some hypotheses to be true in order for its conclusion to be true.
A theorem, on the other hand, has no hypotheses.
Informally we often call both of them theorems, but in this section we
will stick to the strict definitions.

It sometimes happens that we have proved a deduction of the form
$\varphi \Rightarrow \psi$\index{$\Rightarrow$}
(given hypothesis $\varphi$ we can prove $\psi$)
and we want to then prove a theorem of the form
$\varphi \rightarrow \psi$.

Converting a deduction (which uses a hypothesis) into a theorem
(which does not) is not as simple as you might think.
The deduction says, ``if we can prove $\varphi$ then we can prove $\psi$,''
which is in some sense weaker than saying
``$\varphi$ implies $\psi$.''
There is no axiom of logic that permits us to directly obtain the theorem
given the deduction.\footnote{
The conversion of a deduction to a theorem does not even hold in general
for quantum propositional calculus,
which is a weak subset of classical propositional calculus.
It has been shown that adding the Standard Deduction Theorem (discussed below)
to quantum propositional calculus turns it into classical
propositional calculus!
}

This is in contrast to going the other way.
If we have the theorem ($\varphi \rightarrow \psi$),
it is easy to recover the deduction
($\varphi \Rightarrow \psi$)
using modus ponens\index{modus ponens}
(\texttt{ax-mp}; see section \ref{axmp}).

In the following subsections we first discuss the standard deduction theorem
(the traditional but awkward way to convert deductions into theorems) and
the weak deduction theorem (a limited version of the standard deduction
theorem that is easier to use and was once widely used in
\texttt{set.mm}\index{set theory database (\texttt{set.mm})}).
In section \ref{deductionstyle} we discuss
deduction style, the newer approach we now recommend in most cases.
Deduction style uses ``deduction form,'' a form that
prefixes each hypothesis (other than definitions) and the
conclusion with a universal antecedent (``$\varphi \rightarrow$'').
Deduction style is widely used in \texttt{set.mm},
so it is useful to understand it and \textit{why} it is widely used.
Section \ref{naturaldeduction}
briefly discusses our approach for using natural deduction
within \texttt{set.mm},
as that approach is deeply related to deduction style.
We conclude with a summary of the strengths of
our approach, which we believe are compelling.

\subsection{The Standard Deduction Theorem}\label{standarddeductiontheorem}

It is possible to make use of information
contained in the deduction or its proof to assist us with the proof of
the related theorem.
In traditional logic books, there is a metatheorem called the
Deduction Theorem\index{Deduction Theorem}\index{Standard Deduction Theorem},
discovered independently by Herbrand and Tarski around 1930.
The Deduction Theorem, which we often call the Standard Deduction Theorem,
provides an algorithm for constructing a proof of a theorem from the
proof of its corresponding deduction. See, for example,
\cite[p.~56]{Margaris}\index{Margaris, Angelo}.
To construct a proof for a theorem, the
algorithm looks at each step in the proof of the original deduction and
rewrites the step with several steps wherein the hypothesis is eliminated
and becomes an antecedent.

In ordinary mathematics, no one actually carries out the algorithm,
because (in its most basic form) it involves an exponential explosion of
the number of proof steps as more hypotheses are eliminated. Instead,
the Standard Deduction Theorem is invoked simply to claim that it can
be done in principle, without actually doing it.
What's more, the algorithm is not as simple as it might first appear
when applying it rigorously.
There is a subtle restriction on the Standard Deduction Theorem
that must be taken into account involving the axiom of generalization
when working with predicate calculus (see the literature for more detail).

One of the goals of Metamath is to let you plainly see, with as few
underlying concepts as possible, how mathematics can be derived directly
from the axioms, and not indirectly according to some hidden rules
buried inside a program or understood only by logicians. If we added
the Standard Deduction Theorem to the language and proof verifier,
that would greatly complicate both and largely defeat Metamath's goal
of simplicity. In principle, we could show direct proofs by expanding
out the proof steps generated by the algorithm of the Standard Deduction
Theorem, but that is not feasible in practice because the number of proof
steps quickly becomes huge, even astronomical.
Since the algorithm of the Standard Deduction Theorem is driven by the proof,
we would have to go through that proof
all over again---starting from axioms---in order to obtain the theorem form.
In terms of proof length, there would be no savings over just
proving the theorem directly instead of first proving the deduction form.

\subsection{Weak Deduction Theorem}\label{weakdeductiontheorem}

We have developed
a more efficient method for proving a theorem from a deduction
that can be used instead of the Standard Deduction Theorem
in many (but not all) cases.
We call this more efficient method the
Weak Deduction Theorem\index{Weak Deduction Theorem}.\footnote{
There is also an unrelated ``Weak Deduction Theorem''
in the field of relevance logic, so to avoid confusion we could call
ours the ``Weak Deduction Theorem for Classical Logic.''}
Unlike the Standard Deduction Theorem, the Weak Deduction Theorem produces the
theorem directly from a special substitution instance of the deduction,
using a small, fixed number of steps roughly proportional to the length
of the final theorem.

If you come to a proof referencing the Weak Deduction Theorem
\texttt{dedth} (or one of its variants \texttt{dedthxx}),
here is how to follow the proof without getting into the details:
just click on the theorem referenced in the step
just before the reference to \texttt{dedth} and ignore everything else.
Theorem \texttt{dedth} simply turns a hypothesis into an antecedent
(i.e. the hypothesis followed by $\rightarrow$
is placed in front of the assertion, and the hypothesis
itself is eliminated) given certain conditions.

The Weak Deduction Theorem
eliminates a hypothesis $\varphi$, making it become an antecedent.
It does this by proving an expression
$ \varphi \rightarrow \psi $ given two hypotheses:
(1)
$ ( A = {\rm if} ( \varphi , A , B ) \rightarrow ( \varphi \leftrightarrow \chi ) ) $
and
(2) $\chi$.
Note that it requires that a proof exists for $\varphi$ when the class variable
$A$ is replaced with a specific class $B$. The hypothesis $\chi$
should be assigned to the inference.
You can see the details of the proof of the Weak Deduction Theorem
in theorem \texttt{dedth}.

The Weak Deduction Theorem
is probably easier to understand by studying proofs that make use of it.
For example, let's look at the proof of \texttt{renegcl}, which proves that
$ \vdash ( A \in \mathbb{R} \rightarrow - A \in \mathbb{R} )$:

\needspace{4\baselineskip}
\begin{longtabu} {l l l X}
\textbf{Step} & \textbf{Hyp} & \textbf{Ref} & \textbf{Expression} \\
  1 &  & negeq &
  $\vdash$ $($ $A$ $=$ ${\rm if}$ $($ $A$ $\in$ $\mathbb{R}$ $,$ $A$ $,$ $1$ $)$ $\rightarrow$
  $\textrm{-}$ $A$ $=$ $\textrm{-}$ ${\rm if}$ $($ $A$ $\in$ $\mathbb{R}$
  $,$ $A$ $,$ $1$ $)$ $)$ \\
 2 & 1 & eleq1d &
    $\vdash$ $($ $A$ $=$ ${\rm if}$ $($ $A$ $\in$ $\mathbb{R}$ $,$ $A$ $,$ $1$ $)$ $\rightarrow$ $($
    $\textrm{-}$ $A$ $\in$ $\mathbb{R}$ $\leftrightarrow$
    $\textrm{-}$ ${\rm if}$ $($ $A$ $\in$ $\mathbb{R}$ $,$ $A$ $,$ $1$ $)$ $\in$
    $\mathbb{R}$ $)$ $)$ \\
 3 &  & 1re & $\vdash 1 \in \mathbb{R}$ \\
 4 & 3 & elimel &
   $\vdash {\rm if} ( A \in \mathbb{R} , A , 1 ) \in \mathbb{R}$ \\
 5 & 4 & renegcli &
   $\vdash \textrm{-} {\rm if} ( A \in \mathbb{R} , A , 1 ) \in \mathbb{R}$ \\
 6 & 2,5 & dedth &
   $\vdash ( A \in \mathbb{R} \rightarrow \textrm{-} A \in \mathbb{R}$ ) \\
\end{longtabu}

The somewhat strange-looking steps in \texttt{renegcl} before step 5 are
technical stuff that makes this magic work, and they can be ignored
for a quick overview of the proof. To continue following the ``important''
part of the proof of \texttt{renegcl},
you can look at the reference to \texttt{renegcli} at step 5.

That said, let's briefly look at how
\texttt{renegcl} uses the
Weak Deduction Theorem (\texttt{dedth}) to do its job,
in case you want to do something similar or want understand it more deeply.
Let's work backwards in the proof of \texttt{renegcl}.
Step 6 applies \texttt{dedth} to produce our goal result
$ \vdash ( A \in \mathbb{R} \rightarrow\, - A \in \mathbb{R} )$.
This requires on the one hand the (substituted) deduction
\texttt{renegcli} in step 5.
By itself \texttt{renegcli} proves the deduction
$ \vdash A \in \mathbb{R} \Rightarrow\, \vdash - A \in \mathbb{R}$;
this is the deduction form we are trying to turn into theorem form,
and thus
\texttt{renegcli} has a separate hypothesis that must be fulfilled.
To fulfill the hypothesis of the invocation of
\texttt{renegcli} in step 5, it is eventually
reduced to the already proven theorem $1 \in \mathbb{R}$ in step 3.
Step 4 connects steps 3 and 5; step 4 invokes
\texttt{elimel}, a special case of \texttt{elimhyp} that eliminates
a membership hypothesis for the weak deduction theorem.
On the other hand, the equivalence of the conclusion of
\texttt{renegcl}
$( - A \in \mathbb{R} )$ and the substituted conclusion of
\texttt{renegcli} must be proven, which is done in steps 2 and 1.

The weak deduction theorem has limitations.
In particular, we must be able to prove a special case of the deduction's
hypothesis as a stand-alone theorem.
For example, we used $1 \in \mathbb{R}$ in step 3 of \texttt{renegcl}.

We used to use the weak deduction theorem
extensively within \texttt{set.mm}.
However, we now recommend applying ``deduction style''
instead in most cases, as deduction style is
often an easier and clearer approach.
Therefore, we will now describe deduction style.

\subsection{Deduction Style}\label{deductionstyle}

We now prefer to write assertions in ``deduction form''
instead of writing a proof that would require use of the standard or
weak deduction theorem.
We call this appraoch
``deduction style.''\index{deduction style}

It will be easier to explain this by first defining some terms:

\begin{itemize}
\item \textbf{closed form}\index{closed form}\index{forms!closed}:
A kind of assertion (theorem) with no hypotheses.
Typically its label has no special suffix.
An example is \texttt{unss}, which states:
$\vdash ( ( A \subseteq C \wedge B \subseteq C ) \leftrightarrow ( A \cup B )
\subseteq C )\label{eq:unss}$
\item \textbf{deduction form}\index{deduction form}\index{forms!deduction}:
A kind of assertion with one or more hypotheses
where the conclusion is an implication with
a wff variable as the antecedent (usually $\varphi$), and every hypothesis
(\$e statement)
is either (1) an implication with the same antecedent as the conclusion or
(2) a definition.
A definition
can be for a class variable (this is a class variable followed by ``='')
or a wff variable (this is a wff variable followed by $\leftrightarrow$);
class variable definitions are more common.
In practice, a proof
in deduction form will also contain many steps that are implications
where the antecedent is either that wff variable (normally $\varphi$)
or is
a conjunction (...$\land$...) including that wff variable ($\varphi$).
If an assertion is in deduction form, and other forms are also available,
then we suffix its label with ``d.''
An example is \texttt{unssd}, which states\footnote{
For brevity we show here (and in other places)
a $\&$\index{$\&$} between hypotheses\index{hypotheses}
and a $\Rightarrow$\index{$\Rightarrow$}\index{conclusion}
between the hypotheses and the conclusion.
This notation is technically not part of the Metamath language, but is
instead a convenient abbreviation to show both the hypotheses and conclusion.}:
$\vdash ( \varphi \rightarrow A \subseteq C )\quad\&\quad \vdash ( \varphi
    \rightarrow B \subseteq C )\quad\Rightarrow\quad \vdash ( \varphi
    \rightarrow ( A \cup B ) \subseteq C )\label{eq:unssd}$
\item \textbf{inference form}\index{inference form}\index{forms!inference}:
A kind of assertion with one or more hypotheses that is not in deduction form
(e.g., there is no common antecedent).
If an assertion is in inference form, and other forms are also available,
then we suffix its label with ``i.''
An example is \texttt{unssi}, which states:
$\vdash A \subseteq C\quad\&\quad \vdash B \subseteq C\quad\Rightarrow\quad
    \vdash ( A \cup B ) \subseteq C\label{eq:unssi}$
\end{itemize}

When using deduction style we express an assertion in deduction form.
This form prefixes each hypothesis (other than definitions) and the
conclusion with a universal antecedent (``$\varphi \rightarrow$'').
The antecedent (e.g., $\varphi$)
mimics the context handled in the deduction theorem, eliminating
the need to directly use the deduction theorem.

Once you have an assertion in deduction form, you can easily convert it
to inference form or closed form:

\begin{itemize}
\item To
prove some assertion Ti in inference form, given assertion Td in deduction
form, there is a simple mechanical process you can use. First take each
Ti hypothesis and insert a \texttt{T.} $\rightarrow$ prefix (``true implies'')
using \texttt{a1i}. You
can then use the existing assertion Td to prove the resulting conclusion
with a \texttt{T.} $\rightarrow$ prefix.
Finally, you can remove that prefix using \texttt{mptru},
resulting in the conclusion you wanted to prove.
\item To
prove some assertion T in closed form, given assertion Td in deduction
form, there is another simple mechanical process you can use. First,
select an expression that is the conjunction (...$\land$...) of all of the
consequents of every hypothesis of Td. Next, prove that this expression
implies each of the separate hypotheses of Td in turn by eliminating
conjuncts (there are a variety of proven assertions to do this, including
\texttt{simpl},
\texttt{simpr},
\texttt{3simpa},
\texttt{3simpb},
\texttt{3simpc},
\texttt{simp1},
\texttt{simp2},
and
\texttt{simp3}).
If the
expression has nested conjunctions, inner conjuncts can be broken out by
chaining the above theorems with \texttt{syl}
(see section \ref{syl}).\footnote{
There are actually many theorems
(labeled simp* such as \texttt{simp333}) that break out inner conjuncts in one
step, but rather than learning them you can just use the chaining we
just described to prove them, and then let the Metamath program command
\texttt{minimize{\char`\_}with}\index{\texttt{minimize{\char`\_}with} command}
figure out the right ones needed to collapse them.}
As your final step, you can then apply the already-proven assertion Td
(which is in deduction form), proving assertion T in closed form.
\end{itemize}

We can also easily convert any assertion T in closed form to its related
assertion Ti in inference form by applying
modus ponens\index{modus ponens} (see section \ref{axmp}).

The deduction form antecedent can also be used to represent the context
necessary to support natural deduction systems, so we will now
discuss natural deduction.

\subsection{Natural Deduction}\label{naturaldeduction}

Natural deduction\index{natural deduction}
(ND) systems, as such, were originally introduced in
1934 by two logicians working independently: Ja\'skowski and Gentzen. ND
systems are supposed to reconstruct, in a formally proper way, traditional
ways of mathematical reasoning (such as conditional proof, indirect proof,
and proof by cases). As reconstructions they were naturally influenced
by previous work, and many specific ND systems and notations have been
developed since their original work.

There are many ND variants, but
Indrzejczak \cite[p.~31-32]{Indrzejczak}\index{Indrzejczak, Andrzej}
suggests that any natural deductive system must satisfy at
least these three criteria:

\begin{itemize}
\item ``There are some means for entering assumptions into a proof and
also for eliminating them. Usually it requires some bookkeeping devices
for indicating the scope of an assumption, and showing that a part of
a proof depending on eliminated assumption is discharged.
\item There are no (or, at least, a very limited set of) axioms, because
their role is taken over by the set of primitive rules for introduction
and elimination of logical constants which means that elementary
inferences instead of formulae are taken as primitive.
\item (A genuine) ND system admits a lot of freedom in proof construction
and possibility of applying several proof search strategies, like
conditional proof, proof by cases, proof by reductio ad absurdum etc.''
\end{itemize}

The Metamath Proof Explorer (MPE) as defined in \texttt{set.mm}
is fundamentally a Hilbert-style system.
That is, MPE is based on a larger number of axioms (compared
to natural deduction systems), a very small set of rules of inference
(modus ponens), and the context is not changed by the rules of inference
in the middle of a proof. That said, MPE proofs can be developed using
the natural deduction (ND) approach as originally developed by Ja\'skowski
and Gentzen.

The most common and recommended approach for applying ND in MPE is to use
deduction form\index{deduction form}%
\index{forms!deduction}
and apply the MPE proven assertions that are equivalent to ND rules.
For example, MPE's \texttt{jca} is equivalent to ND rule $\land$-I
(and-insertion).
We maintain a list of equivalences that you may consult.
This approach for applying an ND approach within MPE relies on Metamath's
wff metavariables in an essential way, and is described in more detail
in the presentation ``Natural Deductions in the Metamath Proof Language''
by Mario Carneiro \cite{CarneiroND}\index{Carneiro, Mario}.

In this style many steps are an implication, whose antecedent mimics
the context ($\Gamma$) of most ND systems. To add an assumption, simply add
it to the implication antecedent (typically using
\texttt{simpr}),
and use that
new antecedent for all later claims in the same scope. If you wish to
use an assertion in an ND hypothesis scope that is outside the current
ND hypothesis scope, modify the assertion so that the ND hypothesis
assumption is added to its antecedent (typically using \texttt{adantr}). Most
proof steps will be proved using rules that have hypotheses and results
of the form $\varphi \rightarrow$ ...

An example may make this clearer.
Let's show theorem 5.5 of
\cite[p.~18]{Clemente}\index{Clemente Laboreo, Daniel}
along with a line by line translation using the usual
translation of natural deduction (ND) in the Metamath Proof Explorer
(MPE) notation (this is proof \texttt{ex-natded5.5}).
The proof's original goal was to prove
$\lnot \psi$ given two hypotheses,
$( \psi \rightarrow \chi )$ and $ \lnot \chi$.
We will translate these statements into MPE deduction form
by prefixing them all with $\varphi \rightarrow$.
As a result, in MPE the goal is stated as
$( \varphi \rightarrow \lnot \psi )$, and the two hypotheses are stated as
$( \varphi \rightarrow ( \psi \rightarrow \chi ) )$ and
$( \varphi \rightarrow \lnot \chi )$.

The following table shows the proof in Fitch natural deduction style
and its MPE equivalent.
The \textit{\#} column shows the original numbering,
\textit{MPE\#} shows the number in the equivalent MPE proof
(which we will show later),
\textit{ND Expression} shows the original proof claim in ND notation,
and \textit{MPE Translation} shows its translation into MPE
as discussed in this section.
The final columns show the rationale in ND and MPE respectively.

\needspace{4\baselineskip}
{\setlength{\extrarowsep}{4pt} % Keep rows from being too close together
\begin{longtabu}   { @{} c c X X X X }
\textbf{\#} & \textbf{MPE\#} & \textbf{ND Ex\-pres\-sion} &
\textbf{MPE Trans\-lation} & \textbf{ND Ration\-ale} &
\textbf{MPE Ra\-tio\-nale} \\
\endhead

1 & 2;3 &
$( \psi \rightarrow \chi )$ &
$( \varphi \rightarrow ( \psi \rightarrow \chi ) )$ &
Given &
\$e; \texttt{adantr} to put in ND hypothesis \\

2 & 5 &
$ \lnot \chi$ &
$( \varphi \rightarrow \lnot \chi )$ &
Given &
\$e; \texttt{adantr} to put in ND hypothesis \\

3 & 1 &
... $\vert$ $\psi$ &
$( \varphi \rightarrow \psi )$ &
ND hypothesis assumption &
\texttt{simpr} \\

4 & 4 &
... $\chi$ &
$( ( \varphi \land \psi ) \rightarrow \chi )$ &
$\rightarrow$\,E 1,3 &
\texttt{mpd} 1,3 \\

5 & 6 &
... $\lnot \chi$ &
$( ( \varphi \land \psi ) \rightarrow \lnot \chi )$ &
IT 2 &
\texttt{adantr} 5 \\

6 & 7 &
$\lnot \psi$ &
$( \varphi \rightarrow \lnot \psi )$ &
$\land$\,I 3,4,5 &
\texttt{pm2.65da} 4,6 \\

\end{longtabu}
}


The original used Latin letters; we have replaced them with Greek letters
to follow Metamath naming conventions and so that it is easier to follow
the Metamath translation. The Metamath line-for-line translation of
this natural deduction approach precedes every line with an antecedent
including $\varphi$ and uses the Metamath equivalents of the natural deduction
rules. To add an assumption, the antecedent is modified to include it
(typically by using \texttt{adantr};
\texttt{simpr} is useful when you want to
depend directly on the new assumption, as is shown here).

In Metamath we can represent the two given statements as these hypotheses:

\needspace{2\baselineskip}
\begin{itemize}
\item ex-natded5.5.1 $\vdash ( \varphi \rightarrow ( \psi \rightarrow \chi ) )$
\item ex-natded5.5.2 $\vdash ( \varphi \rightarrow \lnot \chi )$
\end{itemize}

\needspace{4\baselineskip}
Here is the proof in Metamath as a line-by-line translation:

\begin{longtabu}   { l l l X }
\textbf{Step} & \textbf{Hyp} & \textbf{Ref} & \textbf{Ex\-pres\-sion} \\
\endhead
1 & & simpr & $\vdash ( ( \varphi \land \psi ) \rightarrow \psi )$ \\
2 & & ex-natded5.5.1 &
  $\vdash ( \varphi \rightarrow ( \psi \rightarrow \chi ) )$ \\
3 & 2 & adantr &
 $\vdash ( ( \varphi \land \psi ) \rightarrow ( \psi \rightarrow \chi ) )$ \\
4 & 1, 3 & mpd &
 $\vdash ( ( \varphi \land \psi ) \rightarrow \chi ) $ \\
5 & & ex-natded5.5.2 &
 $\vdash ( \varphi \rightarrow \lnot \chi )$ \\
6 & 5 & adantr &
 $\vdash ( ( \varphi \land \psi ) \rightarrow \lnot \chi )$ \\
7 & 4, 6 & pm2.65da &
 $\vdash ( \varphi \rightarrow \lnot \psi )$ \\
\end{longtabu}

Only using specific natural deduction rules directly can lead to very
long proofs, for exactly the same reason that only using axioms directly
in Hilbert-style proofs can lead to very long proofs.
If the goal is short and clear proofs,
then it is better to reuse already-proven assertions
in deduction form than to start from scratch each time
and using only basic natural deduction rules.

\subsection{Strengths of Our Approach}

As far as we know there is nothing else in the literature like either the
weak deduction theorem or Mario Carneiro\index{Carneiro, Mario}'s
natural deduction method.
In order to
transform a hypothesis into an antecedent, the literature's standard
``Deduction Theorem''\index{Deduction Theorem}\index{Standard Deduction Theorem}
requires metalogic outside of the notions provided
by the axiom system. We instead generally prefer to use Mario Carneiro's
natural deduction method, then use the weak deduction theorem in cases
where that is difficult to apply, and only then use the full standard
deduction theorem as a last resort.

The weak deduction theorem\index{Weak Deduction Theorem}
does not require any additional metalogic
but converts an inference directly into a closed form theorem, with
a rigorous proof that uses only the axiom system. Unlike the standard
Deduction Theorem, there is no implicit external justification that we
have to trust in order to use it.

Mario Carneiro's natural deduction\index{natural deduction}
method also does not require any new metalogical
notions. It avoids the Deduction Theorem's metalogic by prefixing the
hypotheses and conclusion of every would-be inference with a universal
antecedent (``$\varphi \rightarrow$'') from the very start.

We think it is impressive and satisfying that we can do so much in a
practical sense without stepping outside of our Hilbert-style axiom system.
Of course our axiomatization, which is in the form of schemes,
contains a metalogic of its own that we exploit. But this metalogic
is relatively simple, and for our Deduction Theorem alternatives,
we primarily use just the direct substitution of expressions for
metavariables.

\begin{sloppy}
\section{Exploring the Set The\-o\-ry Data\-base}\label{exploring}
\end{sloppy}
% NOTE: All examples performed in this section are
% recorded wtih "set width 61" % on set.mm as of 2019-05-28
% commit c1e7849557661260f77cfdf0f97ac4354fbb4f4d.

At this point you may wish to study the \texttt{set.mm}\index{set theory
database (\texttt{set.mm})} file in more detail.  Pay particular
attention to the assumptions needed to define wffs\index{well-formed
formula (wff)} (which are not included above), the variable types
(\texttt{\$f}\index{\texttt{\$f} statement} statements), and the
definitions that are introduced.  Start with some simple theorems in
propositional calculus, making sure you understand in detail each step
of a proof.  Once you get past the first few proofs and become familiar
with the Metamath language, any part of the \texttt{set.mm} database
will be as easy to follow, step by step, as any other part---you won't
have to undergo a ``quantum leap'' in mathematical sophistication to be
able to follow a deep proof in set theory.

Next, you may want to explore how concepts such as natural numbers are
defined and described.  This is probably best done in conjunction with
standard set theory textbooks, which can help give you a higher-level
understanding.  The \texttt{set.mm} database provides references that will get
you started.  From there, you will be on your way towards a very deep,
rigorous understanding of abstract mathematics.

The Metamath\index{Metamath} program can help you peruse a Metamath data\-base,
wheth\-er you are trying to figure out how a certain step follows in a proof or
just have a general curiosity.  We will go through some examples of the
commands, using the \texttt{set.mm}\index{set theory database (\texttt{set.mm})}
database provided with the Metamath software.  These should help get you
started.  See Chapter~\ref{commands} for a more detailed description of
the commands.  Note that we have included the full spelling of all commands to
prevent ambiguity with future commands.  In practice you may type just the
characters needed to specify each command keyword\index{command keyword}
unambiguously, often just one or two characters per keyword, and you don't
need to type them in upper case.

First run the Metamath program as described earlier.  You should see the
\verb/MM>/ prompt.  Read in the \texttt{set.mm} file:\index{\texttt{read}
command}

\begin{verbatim}
MM> read set.mm
Reading source file "set.mm"... 34554442 bytes
34554442 bytes were read into the source buffer.
The source has 155711 statements; 2254 are $a and 32250 are $p.
No errors were found.  However, proofs were not checked.
Type VERIFY PROOF * if you want to check them.
\end{verbatim}

As with most examples in this book, what you will see
will be slightly different because we are continuously
improving our databases (including \texttt{set.mm}).

Let's check the database integrity.  This may take a minute or two to run if
your computer is slow.

\begin{verbatim}
MM> verify proof *
0 10%  20%  30%  40%  50%  60%  70%  80%  90% 100%
..................................................
All proofs in the database were verified in 2.84 s.
\end{verbatim}

No errors were reported, so every proof is correct.

You need to know the names (labels) of theorems before you can look at them.
Often just examining the database file(s) with a text editor is the best
approach.  In \texttt{set.mm} there are many detailed comments, especially near
the beginning, that can help guide you. The \texttt{search} command in the
Metamath program is also handy.  The \texttt{comments} qualifier will list the
statements whose associated comment (the one immediately before it) contain a
string you give it.  For example, if you are studying Enderton's {\em Elements
of Set Theory} \cite{Enderton}\index{Enderton, Herbert B.} you may want to see
the references to it in the database.  The search string \texttt{enderton} is not
case sensitive.  (This will not show you all the database theorems that are in
Enderton's book because there is usually only one citation for a given
theorem, which may appear in several textbooks.)\index{\texttt{search}
command}

\begin{verbatim}
MM> search * "enderton" / comments
12067 unineq $p "... Exercise 20 of [Enderton] p. 32 and ..."
12459 undif2 $p "...Corollary 6K of [Enderton] p. 144. (C..."
12953 df-tp $a "...s. Definition of [Enderton] p. 19. (Co..."
13689 unissb $p ".... Exercise 5 of [Enderton] p. 26 and ..."
\end{verbatim}
\begin{center}
(etc.)
\end{center}

Or you may want to see what theorems have something to do with
conjunction (logical {\sc and}).  The quotes around the search
string are optional when there's no ambiguity.\index{\texttt{search}
command}

\begin{verbatim}
MM> search * conjunction / comments
120 a1d $p "...be replaced with a conjunction ( ~ df-an )..."
662 df-bi $a "...viated form after conjunction is introdu..."
1319 wa $a "...ff definition to include conjunction ('and')."
1321 df-an $a "Define conjunction (logical 'and'). Defini..."
1420 imnan $p "...tion in terms of conjunction. (Contribu..."
\end{verbatim}
\begin{center}
(etc.)
\end{center}


Now we will start to look at some details.  Let's look at the first
axiom of propositional calculus
(we could use \texttt{sh st} to abbreviate
\texttt{show statement}).\index{\texttt{show statement} command}

\begin{verbatim}
MM> show statement ax-1/full
Statement 19 is located on line 881 of the file "set.mm".
"Axiom _Simp_.  Axiom A1 of [Margaris] p. 49.  One of the 3
axioms of propositional calculus.  The 3 axioms are also
        ...
19 ax-1 $a |- ( ph -> ( ps -> ph ) ) $.
Its mandatory hypotheses in RPN order are:
  wph $f wff ph $.
  wps $f wff ps $.
The statement and its hypotheses require the variables:  ph
      ps
The variables it contains are:  ph ps


Statement 49 is located on line 11182 of the file "set.mm".
Its statement number for HTML pages is 6.
"Axiom _Simp_.  Axiom A1 of [Margaris] p. 49.  One of the 3
axioms of propositional calculus.  The 3 axioms are also
given as Definition 2.1 of [Hamilton] p. 28.
...
49 ax-1 $a |- ( ph -> ( ps -> ph ) ) $.
Its mandatory hypotheses in RPN order are:
  wph $f wff ph $.
  wps $f wff ps $.
The statement and its hypotheses require the variables:
  ph ps
The variables it contains are:  ph ps
\end{verbatim}

Compare this to \texttt{ax-1} on p.~\pageref{ax1}.  You can see that the
symbol \texttt{ph} is the {\sc ascii} notation for $\varphi$, etc.  To
see the mathematical symbols for any expression you may typeset it in
\LaTeX\ (type \texttt{help tex} for instructions)\index{latex@{\LaTeX}}
or, easier, just use a text editor to look at the comments where symbols
are first introduced in \texttt{set.mm}.  The hypotheses \texttt{wph}
and \texttt{wps} required by \texttt{ax-1} mean that variables
\texttt{ph} and \texttt{ps} must be wffs.

Next we'll pick a simple theorem of propositional calculus, the Principle of
Identity, which is proved directly from the axioms.  We'll look at the
statement then its proof.\index{\texttt{show statement}
command}

\begin{verbatim}
MM> show statement id1/full
Statement 116 is located on line 11371 of the file "set.mm".
Its statement number for HTML pages is 22.
"Principle of identity.  Theorem *2.08 of [WhiteheadRussell]
p. 101.  This version is proved directly from the axioms for
demonstration purposes.
...
116 id1 $p |- ( ph -> ph ) $= ... $.
Its mandatory hypotheses in RPN order are:
  wph $f wff ph $.
Its optional hypotheses are:  wps wch wth wta wet
      wze wsi wrh wmu wla wka
The statement and its hypotheses require the variables:  ph
These additional variables are allowed in its proof:
      ps ch th ta et ze si rh mu la ka
The variables it contains are:  ph
\end{verbatim}

The optional variables\index{optional variable} \texttt{ps}, \texttt{ch}, etc.\ are
available for use in a proof of this statement if we wish, and were we to do
so we would make use of optional hypotheses \texttt{wps}, \texttt{wch}, etc.  (See
Section~\ref{dollaref} for the meaning of ``optional
hypothesis.''\index{optional hypothesis}) The reason these show up in the
statement display is that statement \texttt{id1} happens to be in their scope
(see Section~\ref{scoping} for the definition of ``scope''\index{scope}), but
in fact in propositional calculus we will never make use of optional
hypotheses or variables.  This becomes important after quantifiers are
introduced, where ``dummy'' variables\index{dummy variable} are often needed
in the middle of a proof.

Let's look at the proof of statement \texttt{id1}.  We'll use the
\texttt{show proof} command, which by default suppresses the
``non-essential'' steps that construct the wffs.\index{\texttt{show proof}
command}
We will display the proof in ``lemmon' format (a non-indented format
with explicit previous step number references) and renumber the
displayed steps:

\begin{verbatim}
MM> show proof id1 /lemmon/renumber
1 ax-1           $a |- ( ph -> ( ph -> ph ) )
2 ax-1           $a |- ( ph -> ( ( ph -> ph ) -> ph ) )
3 ax-2           $a |- ( ( ph -> ( ( ph -> ph ) -> ph ) ) ->
                     ( ( ph -> ( ph -> ph ) ) -> ( ph -> ph )
                                                          ) )
4 2,3 ax-mp      $a |- ( ( ph -> ( ph -> ph ) ) -> ( ph -> ph
                                                          ) )
5 1,4 ax-mp      $a |- ( ph -> ph )
\end{verbatim}

If you have read Section~\ref{trialrun}, you'll know how to interpret this
proof.  Step~2, for example, is an application of axiom \texttt{ax-1}.  This
proof is identical to the one in Hamilton's {\em Logic for Mathematicians}
\cite[p.~32]{Hamilton}\index{Hamilton, Alan G.}.

You may want to look at what
substitutions\index{substitution!variable}\index{variable substitution} are
made into \texttt{ax-1} to arrive at step~2.  The command to do this needs to
know the ``real'' step number, so we'll display the proof again without
the \texttt{renumber} qualifier.\index{\texttt{show proof}
command}

\begin{verbatim}
MM> show proof id1 /lemmon
 9 ax-1          $a |- ( ph -> ( ph -> ph ) )
20 ax-1          $a |- ( ph -> ( ( ph -> ph ) -> ph ) )
24 ax-2          $a |- ( ( ph -> ( ( ph -> ph ) -> ph ) ) ->
                     ( ( ph -> ( ph -> ph ) ) -> ( ph -> ph )
                                                          ) )
25 20,24 ax-mp   $a |- ( ( ph -> ( ph -> ph ) ) -> ( ph -> ph
                                                          ) )
26 9,25 ax-mp    $a |- ( ph -> ph )
\end{verbatim}

The ``real'' step number is 20.  Let's look at its details.

\begin{verbatim}
MM> show proof id1 /detailed_step 20
Proof step 20:  min=ax-1 $a |- ( ph -> ( ( ph -> ph ) -> ph )
  )
This step assigns source "ax-1" ($a) to target "min" ($e).
The source assertion requires the hypotheses "wph" ($f, step
18) and "wps" ($f, step 19).  The parent assertion of the
target hypothesis is "ax-mp" ($a, step 25).
The source assertion before substitution was:
    ax-1 $a |- ( ph -> ( ps -> ph ) )
The following substitutions were made to the source
assertion:
    Variable  Substituted with
     ph        ph
     ps        ( ph -> ph )
The target hypothesis before substitution was:
    min $e |- ph
The following substitution was made to the target hypothesis:
    Variable  Substituted with
     ph        ( ph -> ( ( ph -> ph ) -> ph ) )
\end{verbatim}

This shows the substitutions\index{substitution!variable}\index{variable
substitution} made to the variables in \texttt{ax-1}.  References are made to
steps 18 and 19 which are not shown in our proof display.  To see these steps,
you can display the proof with the \texttt{all} qualifier.

Let's look at a slightly more advanced proof of propositional calculus.  Note
that \verb+/\+ is the symbol for $\wedge$ (logical {\sc and}, also
called conjunction).\index{conjunction ($\wedge$)}
\index{logical {\sc and} ($\wedge$)}

\begin{verbatim}
MM> show statement prth/full
Statement 1791 is located on line 15503 of the file "set.mm".
Its statement number for HTML pages is 559.
"Conjoin antecedents and consequents of two premises.  This
is the closed theorem form of ~ anim12d .  Theorem *3.47 of
[WhiteheadRussell] p. 113.  It was proved by Leibniz,
and it evidently pleased him enough to call it
_praeclarum theorema_ (splendid theorem).
...
1791 prth $p |- ( ( ( ph -> ps ) /\ ( ch -> th ) ) -> ( ( ph
      /\ ch ) -> ( ps /\ th ) ) ) $= ... $.
Its mandatory hypotheses in RPN order are:
  wph $f wff ph $.
  wps $f wff ps $.
  wch $f wff ch $.
  wth $f wff th $.
Its optional hypotheses are:  wta wet wze wsi wrh wmu wla wka
The statement and its hypotheses require the variables:  ph
      ps ch th
These additional variables are allowed in its proof:  ta et
      ze si rh mu la ka
The variables it contains are:  ph ps ch th


MM> show proof prth /lemmon/renumber
1 simpl          $p |- ( ( ( ph -> ps ) /\ ( ch -> th ) ) ->
                                               ( ph -> ps ) )
2 simpr          $p |- ( ( ( ph -> ps ) /\ ( ch -> th ) ) ->
                                               ( ch -> th ) )
3 1,2 anim12d    $p |- ( ( ( ph -> ps ) /\ ( ch -> th ) ) ->
                           ( ( ph /\ ch ) -> ( ps /\ th ) ) )
\end{verbatim}

There are references to a lot of unfamiliar statements.  To see what they are,
you may type the following:

\begin{verbatim}
MM> show proof prth /statement_summary
Summary of statements used in the proof of "prth":

Statement simpl is located on line 14748 of the file
"set.mm".
"Elimination of a conjunct.  Theorem *3.26 (Simp) of
[WhiteheadRussell] p. 112. ..."
  simpl $p |- ( ( ph /\ ps ) -> ph ) $= ... $.

Statement simpr is located on line 14777 of the file
"set.mm".
"Elimination of a conjunct.  Theorem *3.27 (Simp) of
[WhiteheadRussell] ..."
  simpr $p |- ( ( ph /\ ps ) -> ps ) $= ... $.

Statement anim12d is located on line 15445 of the file
"set.mm".
"Conjoin antecedents and consequents in a deduction.
..."
  anim12d.1 $e |- ( ph -> ( ps -> ch ) ) $.
  anim12d.2 $e |- ( ph -> ( th -> ta ) ) $.
  anim12d $p |- ( ph -> ( ( ps /\ th ) -> ( ch /\ ta ) ) )
      $= ... $.
\end{verbatim}
\begin{center}
(etc.)
\end{center}

Of course you can look at each of these statements and their proofs, and
so on, back to the axioms of propositional calculus if you wish.

The \texttt{search} command is useful for finding statements when you
know all or part of their contents.  The following example finds all
statements containing \verb@ph -> ps@ followed by \verb@ch -> th@.  The
\verb@$*@ is a wildcard that matches anything; the \texttt{\$} before the
\verb$*$ prevents conflicts with math symbol token names.  The \verb@*@ after
\texttt{SEARCH} is also a wildcard that in this case means ``match any label.''
\index{\texttt{search} command}

% I'm omitting this one, since readers are unlikely to see it:
% 1096 bisymOLD $p |- ( ( ( ph -> ps ) -> ( ch -> th ) ) -> ( (
%   ( ps -> ph ) -> ( th -> ch ) ) -> ( ( ph <-> ps ) -> ( ch
%    <-> th ) ) ) )
\begin{verbatim}
MM> search * "ph -> ps $* ch -> th"
1791 prth $p |- ( ( ( ph -> ps ) /\ ( ch -> th ) ) -> ( ( ph
    /\ ch ) -> ( ps /\ th ) ) )
2455 pm3.48 $p |- ( ( ( ph -> ps ) /\ ( ch -> th ) ) -> ( (
    ph \/ ch ) -> ( ps \/ th ) ) )
117859 pm11.71 $p |- ( ( E. x ph /\ E. y ch ) -> ( ( A. x (
    ph -> ps ) /\ A. y ( ch -> th ) ) <-> A. x A. y ( ( ph /\
    ch ) -> ( ps /\ th ) ) ) )
\end{verbatim}

Three statements, \texttt{prth}, \texttt{pm3.48},
 and \texttt{pm11.71}, were found to match.

To see what axioms\index{axiom} and definitions\index{definition}
\texttt{prth} ultimately depends on for its proof, you can have the
program backtrack through the hierarchy\index{hierarchy} of theorems and
definitions.\index{\texttt{show trace{\char`\_}back} command}

\begin{verbatim}
MM> show trace_back prth /essential/axioms
Statement "prth" assumes the following axioms ($a
statements):
  ax-1 ax-2 ax-3 ax-mp df-bi df-an
\end{verbatim}

Note that the 3 axioms of propositional calculus and the modus ponens rule are
needed (as expected); in addition, there are a couple of definitions that are used
along the way.  Note that Metamath makes no distinction\index{axiom vs.\
definition} between axioms\index{axiom} and definitions\index{definition}.  In
\texttt{set.mm} they have been distinguished artificially by prefixing their
labels\index{labels in \texttt{set.mm}} with \texttt{ax-} and \texttt{df-}
respectively.  For example, \texttt{df-an} defines conjunction (logical {\sc
and}), which is represented by the symbol \verb+/\+.
Section~\ref{definitions} discusses the philosophy of definitions, and the
Metamath language takes a particularly simple, conservative approach by using
the \texttt{\$a}\index{\texttt{\$a} statement} statement for both axioms and
definitions.

You can also have the program compute how many steps a proof
has\index{proof length} if we were to follow it all the way back to
\texttt{\$a} statements.

\begin{verbatim}
MM> show trace_back prth /essential/count_steps
The statement's actual proof has 3 steps.  Backtracking, a
total of 79 different subtheorems are used.  The statement
and subtheorems have a total of 274 actual steps.  If
subtheorems used only once were eliminated, there would be a
total of 38 subtheorems, and the statement and subtheorems
would have a total of 185 steps.  The proof would have 28349
steps if fully expanded back to axiom references.  The
maximum path length is 38.  A longest path is:  prth <-
anim12d <- syl2and <- sylan2d <- ancomsd <- ancom <- pm3.22
<- pm3.21 <- pm3.2 <- ex <- sylbir <- biimpri <- bicomi <-
bicom1 <- bi2 <- dfbi1 <- impbii <- bi3 <- simprim <- impi <-
con1i <- nsyl2 <- mt3d <- con1d <- notnot1 <- con2i <- nsyl3
<- mt2d <- con2d <- notnot2 <- pm2.18d <- pm2.18 <- pm2.21 <-
pm2.21d <- a1d <- syl <- mpd <- a2i <- a2i.1 .
\end{verbatim}

This tells us that we would have to inspect 274 steps if we want to
verify the proof completely starting from the axioms.  A few more
statistics are also shown.  There are one or more paths back to axioms
that are the longest; this command ferrets out one of them and shows it
to you.  There may be a sense in which the longest path length is
related to how ``deep'' the theorem is.

We might also be curious about what proofs depend on the theorem
\texttt{prth}.  If it is never used later on, we could eliminate it as
redundant if it has no intrinsic interest by itself.\index{\texttt{show
usage} command}

% I decided to show the OLD values here.
\begin{verbatim}
MM> show usage prth
Statement "prth" is directly referenced in the proofs of 18
statements:
  mo3 moOLD 2mo 2moOLD euind reuind reuss2 reusv3i opelopabt
  wemaplem2 rexanre rlimcn2 o1of2 o1rlimmul 2sqlem6 spanuni
  heicant pm11.71
\end{verbatim}

Thus \texttt{prth} is directly used by 18 proofs.
We can use the \texttt{/recursive} qualifier to include indirect use:

\begin{verbatim}
MM> show usage prth /recursive
Statement "prth" directly or indirectly affects the proofs of
24214 statements:
  mo3 mo mo3OLD eu2 moOLD eu2OLD eu3OLD mo4f mo4 eu4 mopick
...
\end{verbatim}

\subsection{A Note on the ``Compact'' Proof Format}

The Metamath program will display proofs in a ``compact''\index{compact proof}
format whenever the proof is stored in compressed format in the database.  It
may be be slightly confusing unless you know how to interpret it.
For example,
if you display the complete proof of theorem \texttt{id1} it will start
off as follows:

\begin{verbatim}
MM> show proof id1 /lemmon/all
 1 wph           $f wff ph
 2 wph           $f wff ph
 3 wph           $f wff ph
 4 2,3 wi    @4: $a wff ( ph -> ph )
 5 1,4 wi    @5: $a wff ( ph -> ( ph -> ph ) )
 6 @4            $a wff ( ph -> ph )
\end{verbatim}

\begin{center}
{etc.}
\end{center}

Step 4 has a ``local label,''\index{local label} \texttt{@4}, assigned to it.
Later on, at step 6, the label \texttt{@1} is referenced instead of
displaying the explicit proof for that step.  This technique takes advantage
of the fact that steps in a proof often repeat, especially during the
construction of wffs.  The compact format reduces the number of steps in the
proof display and may be preferred by some people.

If you want to see the normal format with the ``true'' step numbers, you can
use the following workaround:\index{\texttt{save proof} command}

\begin{verbatim}
MM> save proof id1 /normal
The proof of "id1" has been reformatted and saved internally.
Remember to use WRITE SOURCE to save it permanently.
MM> show proof id1 /lemmon/all
 1 wph           $f wff ph
 2 wph           $f wff ph
 3 wph           $f wff ph
 4 2,3 wi        $a wff ( ph -> ph )
 5 1,4 wi        $a wff ( ph -> ( ph -> ph ) )
 6 wph           $f wff ph
 7 wph           $f wff ph
 8 6,7 wi        $a wff ( ph -> ph )
\end{verbatim}

\begin{center}
{etc.}
\end{center}

Note that the original 6 steps are now 8 steps.  However, the format is
now the same as that described in Chapter~\ref{using}.

\chapter{The Metamath Language}
\label{languagespec}

\begin{quote}
  {\em Thus mathematics may be defined as the subject in which we never know
what we are talking about, nor whether what we are saying is true.}
    \flushright\sc  Bertrand Russell\footnote{\cite[p.~84]{Russell2}.}\\
\end{quote}\index{Russell, Bertrand}

Probably the most striking feature of the Metamath language is its almost
complete absence of hard-wired syntax. Metamath\index{Metamath} does not
understand any mathematics or logic other than that needed to construct finite
sequences of symbols according to a small set of simple, built-in rules.  The
only rule it uses in a proof is the substitution of an expression (symbol
sequence) for a variable, subject to a simple constraint to prevent
bound-variable clashes.  The primitive notions built into Metamath involve the
simple manipulation of finite objects (symbols) that we as humans can easily
visualize and that computers can easily deal with.  They seem to be just
about the simplest notions possible that are required to do standard
mathematics.

This chapter serves as a reference manual for the Metamath\index{Metamath}
language. It covers the tedious technical details of the language, some of
which you may wish to skim in a first reading.  On the other hand, you should
pay close attention to the defined terms in {\bf boldface}; they have precise
meanings that are important to keep in mind for later understanding.  It may
be best to first become familiar with the examples in Chapter~\ref{using} to
gain some motivation for the language.

%% Uncomment this when uncommenting section {formalspec} below
If you have some knowledge of set theory, you may wish to study this
chapter in conjunction with the formal set-theoretical description of the
Metamath language in Appendix~\ref{formalspec}.

We will use the name ``Metamath''\index{Metamath} to mean either the Metamath
computer language or the Metamath software associated with the computer
language.  We will not distinguish these two when the context is clear.

The next section contains the complete specification of the Metamath
language.
It serves as an
authoritative reference and presents the syntax in enough detail to
write a parser\index{parsing Metamath} and proof verifier.  The
specification is terse and it is probably hard to learn the language
directly from it, but we include it here for those impatient people who
prefer to see everything up front before looking at verbose expository
material.  Later sections explain this material and provide examples.
We will repeat the definitions in those sections, and you may skip the
next section at first reading and proceed to Section~\ref{tut1}
(p.~\pageref{tut1}).

\section{Specification of the Metamath Language}\label{spec}
\index{Metamath!specification}

\begin{quote}
  {\em Sometimes one has to say difficult things, but one ought to say
them as simply as one knows how.}
    \flushright\sc  G. H. Hardy\footnote{As quoted in
    \cite{deMillo}, p.~273.}\\
\end{quote}\index{Hardy, G. H.}

\subsection{Preliminaries}\label{spec1}

% Space is technically a printable character, so we'll word things
% carefully so it's unambiguous.
A Metamath {\bf database}\index{database} is built up from a top-level source
file together with any source files that are brought in through file inclusion
commands (see below).  The only characters that are allowed to appear in a
Metamath source file are the 94 non-whitespace printable {\sc
ascii}\index{ascii@{\sc ascii}} characters, which are digits, upper and lower
case letters, and the following 32 special
characters\index{special characters}:\label{spec1chars}

\begin{verbatim}
! " # $ % & ' ( ) * + , - . / :
; < = > ? @ [ \ ] ^ _ ` { | } ~
\end{verbatim}
plus the following characters which are the ``white space'' characters:
space (a printable character),
tab, carriage return, line feed, and form feed.\label{whitespace}
We will use \texttt{typewriter}
font to display the printable characters.

A Metamath database consists of a sequence of three kinds of {\bf
tokens}\index{token} separated by {\bf white space}\index{white space}
(which is any sequence of one or more white space characters).  The set
of {\bf keyword}\index{keyword} tokens is \texttt{\$\char`\{},
\texttt{\$\char`\}}, \texttt{\$c}, \texttt{\$v}, \texttt{\$f},
\texttt{\$e}, \texttt{\$d}, \texttt{\$a}, \texttt{\$p}, \texttt{\$.},
\texttt{\$=}, \texttt{\$(}, \texttt{\$)}, \texttt{\$[}, and
\texttt{\$]}.  The last four are called {\bf auxiliary}\index{auxiliary
keyword} or preprocessing keywords.  A {\bf label}\index{label} token
consists of any combination of letters, digits, and the characters
hyphen, underscore, and period.  A {\bf math symbol}\index{math symbol}
token may consist of any combination of the 93 printable standard {\sc
ascii} characters other than space or \texttt{\$}~. All tokens are
case-sensitive.

\subsection{Preprocessing}

The token \texttt{\$(} begins a {\bf comment} and
\texttt{\$)} ends a comment.\index{\texttt{\$(}
and \texttt{\$)} auxiliary keywords}\index{comment}
Comments may contain any of
the 94 non-whitespace printable characters and white space,
except they may not contain the
2-character sequences \texttt{\$(} or \texttt{\$)} (comments do not nest).
Comments are ignored (treated
like white space) for the purpose of parsing, e.g.,
\texttt{\$( \$[ \$)} is a comment.
See p.~\pageref{mathcomments} for comment typesetting conventions; these
conventions may be ignored for the purpose of parsing.

A {\bf file inclusion command} consists of \texttt{\$[} followed by a file name
followed by \texttt{\$]}.\index{\texttt{\$[} and \texttt{\$]} auxiliary
keywords}\index{included file}\index{file inclusion}
It is only allowed in the outermost scope (i.e., not between
\texttt{\$\char`\{} and \texttt{\$\char`\}})
and must not be inside a statement (e.g., it may not occur
between the label of a \texttt{\$a} statement and its \texttt{\$.}).
The file name may not
contain a \texttt{\$} or white space.  The file must exist.
The case-sensitivity
of its name follows the conventions of the operating system.  The contents of
the file replace the inclusion command.
Included files may include other files.
Only the first reference to a given file is included; any later
references to the same file (whether in the top-level file or in included
files) cause the inclusion command to be ignored (treated like white space).
A verifier may assume that file names with different strings
refer to different files for the purpose of ignoring later references.
A file self-reference is ignored, as is any reference to the top-level file
(to avoid loops).
Included files may not include a \texttt{\$(} without a matching \texttt{\$)},
may not include a \texttt{\$[} without a matching \texttt{\$]}, and may
not include incomplete statements (e.g., a \texttt{\$a} without a matching
\texttt{\$.}).
It is currently unspecified if path references are relative to the process'
current directory or the file's containing directory, so databases should
avoid using pathname separators (e.g., ``/'') in file names.

Like all tokens, the \texttt{\$(}, \texttt{\$)}, \texttt{\$[}, and \texttt{\$]} keywords
must be surrounded by white space.

\subsection{Basic Syntax}

After preprocessing, a database will consist of a sequence of {\bf
statements}.
These are the scoping statements \texttt{\$\char`\{} and
\texttt{\$\char`\}}, along with the \texttt{\$c}, \texttt{\$v},
\texttt{\$f}, \texttt{\$e}, \texttt{\$d}, \texttt{\$a}, and \texttt{\$p}
statements.

A {\bf scoping statement}\index{scoping statement} consists only of its
keyword, \texttt{\$\char`\{} or \texttt{\$\char`\}}.
A \texttt{\$\char`\{} begins a {\bf
block}\index{block} and a matching \texttt{\$\char`\}} ends the block.
Every \texttt{\$\char`\{}
must have a matching \texttt{\$\char`\}}.
Defining it recursively, we say a block
contains a sequence of zero or more tokens other
than \texttt{\$\char`\{} and \texttt{\$\char`\}} and
possibly other blocks.  There is an {\bf outermost
block}\index{block!outermost} not bracketed by \texttt{\$\char`\{} \texttt{\$\char`\}}; the end
of the outermost block is the end of the database.

% LaTeX bug? can't do \bf\tt

A {\bf \$v} or {\bf \$c statement}\index{\texttt{\$v} statement}\index{\texttt{\$c}
statement} consists of the keyword token \texttt{\$v} or \texttt{\$c} respectively,
followed by one or more math symbols,
% The word "token" is used to distinguish "$." from the sentence-ending period.
followed by the \texttt{\$.}\ token.
These
statements {\bf declare}\index{declaration} the math symbols to be {\bf
variables}\index{variable!Metamath} or {\bf constants}\index{constant}
respectively. The same math symbol may not occur twice in a given \texttt{\$v} or
\texttt{\$c} statement.

%c%A math symbol becomes an {\bf active}\index{active math symbol}
%c%when declared and stays active until the end of the block in which it is
%c%declared.  A math symbol may not be declared a second time while it is active,
%c%but it may be declared again after it becomes inactive.

A math symbol becomes {\bf active}\index{active math symbol} when declared
and stays active until the end of the block in which it is declared.  A
variable may not be declared a second time while it is active, but it
may be declared again (as a variable, but not as a constant) after it
becomes inactive.  A constant must be declared in the outermost block and may
not be declared a second time.\index{redeclaration of symbols}

A {\bf \$f statement}\index{\texttt{\$f} statement} consists of a label,
followed by \texttt{\$f}, followed by its typecode (an active constant),
followed by an
active variable, followed by the \texttt{\$.}\ token.  A {\bf \$e
statement}\index{\texttt{\$e} statement} consists of a label, followed
by \texttt{\$e}, followed by its typecode (an active constant),
followed by zero or more
active math symbols, followed by the \texttt{\$.}\ token.  A {\bf
hypothesis}\index{hypothesis} is a \texttt{\$f} or \texttt{\$e}
statement.
The type declared by a \texttt{\$f} statement for a given label
is global even if the variable is not
(e.g., a database may not have \texttt{wff P} in one local scope
and \texttt{class P} in another).

A {\bf simple \$d statement}\index{\texttt{\$d} statement!simple}
consists of \texttt{\$d}, followed by two different active variables,
followed by the \texttt{\$.}\ token.  A {\bf compound \$d
statement}\index{\texttt{\$d} statement!compound} consists of
\texttt{\$d}, followed by three or more variables (all different),
followed by the \texttt{\$.}\ token.  The order of the variables in a
\texttt{\$d} statement is unimportant.  A compound \texttt{\$d}
statement is equivalent to a set of simple \texttt{\$d} statements, one
for each possible pair of variables occurring in the compound
\texttt{\$d} statement.  Henceforth in this specification we shall
assume all \texttt{\$d} statements are simple.  A \texttt{\$d} statement
is also called a {\bf disjoint} (or {\bf distinct}) {\bf variable
restriction}.\index{disjoint-variable restriction}

A {\bf \$a statement}\index{\texttt{\$a} statement} consists of a label,
followed by \texttt{\$a}, followed by its typecode (an active constant),
followed by
zero or more active math symbols, followed by the \texttt{\$.}\ token.  A {\bf
\$p statement}\index{\texttt{\$p} statement} consists of a label,
followed by \texttt{\$p}, followed by its typecode (an active constant),
followed by
zero or more active math symbols, followed by \texttt{\$=}, followed by
a sequence of labels, followed by the \texttt{\$.}\ token.  An {\bf
assertion}\index{assertion} is a \texttt{\$a} or \texttt{\$p} statement.

A \texttt{\$f}, \texttt{\$e}, or \texttt{\$d} statement is {\bf active}\index{active
statement} from the place it occurs until the end of the block it occurs in.
A \texttt{\$a} or \texttt{\$p} statement is {\bf active} from the place it occurs
through the end of the database.
There may not be two active \texttt{\$f} statements containing the same
variable.  Each variable in a \texttt{\$e}, \texttt{\$a}, or
\texttt{\$p} statement must exist in an active \texttt{\$f}
statement.\footnote{This requirement can greatly simplify the
unification algorithm (substitution calculation) required by proof
verification.}

%The label that begins each \texttt{\$f}, \texttt{\$e}, \texttt{\$a}, and
%\texttt{\$p} statement must be unique.
Each label token must be unique, and
no label token may match any math symbol
token.\label{namespace}\footnote{This
restriction was added on June 24, 2006.
It is not theoretically necessary but is imposed to make it easier to
write certain parsers.}

The set of {\bf mandatory variables}\index{mandatory variable} associated with
an assertion is the set of (zero or more) variables in the assertion and in any
active \texttt{\$e} statements.  The (possibly empty) set of {\bf mandatory
hypotheses}\index{mandatory hypothesis} is the set of all active \texttt{\$f}
statements containing mandatory variables, together with all active \texttt{\$e}
statements.
The set of {\bf mandatory {\bf \$d} statements}\index{mandatory
disjoint-variable restriction} associated with an assertion are those active
\texttt{\$d} statements whose variables are both among the assertion's
mandatory variables.

\subsection{Proof Verification}\label{spec4}

The sequence of labels between the \texttt{\$=} and \texttt{\$.}\ tokens
in a \texttt{\$p} statement is a {\bf proof}.\index{proof!Metamath} Each
label in a proof must be the label of an active statement other than the
\texttt{\$p} statement itself; thus a label must refer either to an
active hypothesis of the \texttt{\$p} statement or to an earlier
assertion.

An {\bf expression}\index{expression} is a sequence of math symbols. A {\bf
substitution map}\index{substitution map} associates a set of variables with a
set of expressions.  It is acceptable for a variable to be mapped to an
expression containing it.  A {\bf
substitution}\index{substitution!variable}\index{variable substitution} is the
simultaneous replacement of all variables in one or more expressions with the
expressions that the variables map to.

A proof is scanned in order of its label sequence.  If the label refers to an
active hypothesis, the expression in the hypothesis is pushed onto a
stack.\index{stack}\index{RPN stack}  If the label refers to an assertion, a
(unique) substitution must exist that, when made to the mandatory hypotheses
of the referenced assertion, causes them to match the topmost (i.e.\ most
recent) entries of the stack, in order of occurrence of the mandatory
hypotheses, with the topmost stack entry matching the last mandatory
hypothesis of the referenced assertion.  As many stack entries as there are
mandatory hypotheses are then popped from the stack.  The same substitution is
made to the referenced assertion, and the result is pushed onto the stack.
After the last label in the proof is processed, the stack must have a single
entry that matches the expression in the \texttt{\$p} statement containing the
proof.

%c%{\footnotesize\begin{quotation}\index{redeclaration of symbols}
%c%{{\em Comment.}\label{spec4comment} Whenever a math symbol token occurs in a
%c%{\texttt{\$c} or \texttt{\$v} statement, it is considered to designate a distinct new
%c%{symbol, even if the same token was previously declared (and is now inactive).
%c%{Thus a math token declared as a constant in two different blocks is considered
%c%{to designate two distinct constants (even though they have the same name).
%c%{The two constants will not match in a proof that references both blocks.
%c%{However, a proof referencing both blocks is acceptable as long as it doesn't
%c%{require that the constants match.  Similarly, a token declared to be a
%c%{constant for a referenced assertion will not match the same token declared to
%c%{be a variable for the \texttt{\$p} statement containing the proof.  In the case
%c%{of a token declared to be a variable for a referenced assertion, this is not
%c%{an issue since the variable can be substituted with whatever expression is
%c%{needed to achieve the required match.
%c%{\end{quotation}}
%c2%A proof may reference an assertion that contains or whose hypotheses contain a
%c2%constant that is not active for the \texttt{\$p} statement containing the proof.
%c2%However, the final result of the proof may not contain that constant. A proof
%c2%may also reference an assertion that contains or whose hypotheses contain a
%c2%variable that is not active for the \texttt{\$p} statement containing the proof.
%c2%That variable, of course, will be substituted with whatever expression is
%c2%needed to achieve the required match.

A proof may contain a \texttt{?}\ in place of a label to indicate an unknown step
(Section~\ref{unknown}).  A proof verifier may ignore any proof containing
\texttt{?}\ but should warn the user that the proof is incomplete.

A {\bf compressed proof}\index{compressed proof}\index{proof!compressed} is an
alternate proof notation described in Appen\-dix~\ref{compressed}; also see
references to ``compressed proof'' in the Index.  Compressed proofs are a
Metamath language extension which a complete proof verifier should be able to
parse and verify.

\subsubsection{Verifying Disjoint Variable Restrictions}

Each substitution made in a proof must be checked to verify that any
disjoint variable restrictions are satisfied, as follows.

If two variables replaced by a substitution exist in a mandatory \texttt{\$d}
statement\index{\texttt{\$d} statement} of the assertion referenced, the two
expressions resulting from the substitution must satisfy the following
conditions.  First, the two expressions must have no variables in common.
Second, each possible pair of variables, one from each expression, must exist
in an active \texttt{\$d} statement of the \texttt{\$p} statement containing the
proof.

\vskip 1ex

This ends the specification of the Metamath language;
see Appendix \ref{BNF} for its syntax in
Extended Backus--Naur Form (EBNF)\index{Extended Backus--Naur Form}\index{EBNF}.

\section{The Basic Keywords}\label{tut1}

Our expository material begins here.

Like most computer languages, Metamath\index{Metamath} takes its input from
one or more {\bf source files}\index{source file} which contain characters
expressed in the standard {\sc ascii} (American Standard Code for Information
Interchange)\index{ascii@{\sc ascii}} code for computers.  A source file
consists of a series of {\bf tokens}\index{token}, which are strings of
non-whitespace
printable characters (from the set of 94 shown on p.~\pageref{spec1chars})
separated by {\bf white space}\index{white space} (spaces, tabs, carriage
returns, line feeds, and form feeds). Any string consisting only of these
characters is treated the same as a single space.  The non-whitespace printable
characters\index{printable character} that Metamath recognizes are the 94
characters on standard {\sc ascii} keyboards.

Metamath has the ability to join several files together to form its
input (Section~\ref{include}).  We call the aggregate contents of all
the files after they have been joined together a {\bf
database}\index{database} to distinguish it from an individual source
file.  The tokens in a database consist of {\bf
keywords}\index{keyword}, which are built into the language, together
with two kinds of user-defined tokens called {\bf labels}\index{label}
and {\bf math symbols}\index{math symbol}.  (Often we will simply say
{\bf symbol}\index{symbol} instead of math symbol for brevity).  The set
of {\bf basic keywords}\index{basic keyword} is
\texttt{\$c}\index{\texttt{\$c} statement},
\texttt{\$v}\index{\texttt{\$v} statement},
\texttt{\$e}\index{\texttt{\$e} statement},
\texttt{\$f}\index{\texttt{\$f} statement},
\texttt{\$d}\index{\texttt{\$d} statement},
\texttt{\$a}\index{\texttt{\$a} statement},
\texttt{\$p}\index{\texttt{\$p} statement},
\texttt{\$=}\index{\texttt{\$=} keyword},
\texttt{\$.}\index{\texttt{\$.}\ keyword},
\texttt{\$\char`\{}\index{\texttt{\$\char`\{} and \texttt{\$\char`\}}
keywords}, and \texttt{\$\char`\}}.  This is the complete set of
syntactical elements of what we call the {\bf basic
language}\index{basic language} of Metamath, and with them you can
express all of the mathematics that were intended by the design of
Metamath.  You should make it a point to become very familiar with them.
Table~\ref{basickeywords} lists the basic keywords along with a brief
description of their functions.  For now, this description will give you
only a vague notion of what the keywords are for; later we will describe
the keywords in detail.


\begin{table}[htp] \caption{Summary of the basic Metamath
keywords} \label{basickeywords}
\begin{center}
\begin{tabular}{|p{4pc}|l|}
\hline
\em \centering Keyword&\em Description\\
\hline
\hline
\centering
   \texttt{\$c}&Constant symbol declaration\\
\hline
\centering
   \texttt{\$v}&Variable symbol declaration\\
\hline
\centering
   \texttt{\$d}&Disjoint variable restriction\\
\hline
\centering
   \texttt{\$f}&Variable-type (``floating'') hypothesis\\
\hline
\centering
   \texttt{\$e}&Logical (``essential'') hypothesis\\
\hline
\centering
   \texttt{\$a}&Axiomatic assertion\\
\hline
\centering
   \texttt{\$p}&Provable assertion\\
\hline
\centering
   \texttt{\$=}&Start of proof in \texttt{\$p} statement\\
\hline
\centering
   \texttt{\$.}&End of the above statement types\\
\hline
\centering
   \texttt{\$\char`\{}&Start of block\\
\hline
\centering
   \texttt{\$\char`\}}&End of block\\
\hline
\end{tabular}
\end{center}
\end{table}

%For LaTeX bug(?) where it puts tables on blank page instead of btwn text
%May have to adjust if text changes
%\newpage

There are some additional keywords, called {\bf auxiliary
keywords}\index{auxiliary keyword} that help make Metamath\index{Metamath}
more practical. These are part of the {\bf extended language}\index{extended
language}. They provide you with a means to put comments into a Metamath
source file\index{source file} and reference other source files.  We will
introduce these in later sections. Table~\ref{otherkeywords} summarizes them
so that you can recognize them now if you want to peruse some source
files while learning the basic keywords.


\begin{table}[htp] \caption{Auxiliary Metamath
keywords} \label{otherkeywords}
\begin{center}
\begin{tabular}{|p{4pc}|l|}
\hline
\em \centering Keyword&\em Description\\
\hline
\hline
\centering
   \texttt{\$(}&Start of comment\\
\hline
\centering
   \texttt{\$)}&End of comment\\
\hline
\centering
   \texttt{\$[}&Start of included source file name\\
\hline
\centering
   \texttt{\$]}&End of included source file name\\
\hline
\end{tabular}
\end{center}
\end{table}
\index{\texttt{\$(} and \texttt{\$)} auxiliary keywords}
\index{\texttt{\$[} and \texttt{\$]} auxiliary keywords}


Unlike those in some computer languages, the keywords\index{keyword} are short
two-character sequences rather than English-like words.  While this may make
them slightly more difficult to remember at first, their brevity allows
them to blend in with the mathematics being described, not
distract from it, like punctuation marks.


\subsection{User-Defined Tokens}\label{dollardollar}\index{token}

As you may have noticed, all keywords\index{keyword} begin with the \texttt{\$}
character.  This mundane monetary symbol is not ordinarily used in higher
mathematics (outside of grant proposals), so we have appropriated it to
distinguish the Metamath\index{Metamath} keywords from ordinary mathematical
symbols. The \texttt{\$} character is thus considered special and may not be
used as a character in a user-defined token.  All tokens and keywords are
case-sensitive; for example, \texttt{n} is considered to be a different character
from \texttt{N}.  Case-sensitivity makes the available {\sc ascii} character set
as rich as possible.

\subsubsection{Math Symbol Tokens}\index{token}

Math symbols\index{math symbol} are tokens used to represent the symbols
that appear in ordinary mathematical formulas.  They may consist of any
combination of the 93 non-whitespace printable {\sc ascii} characters other than
\texttt{\$}~. Some examples are \texttt{x}, \texttt{+}, \texttt{(},
\texttt{|-}, \verb$!%@?&$, and \texttt{bounded}.  For readability, it is
best to try to make these look as similar to actual mathematical symbols
as possible, within the constraints of the {\sc ascii} character set, in
order to make the resulting mathematical expressions more readable.

In the Metamath\index{Metamath} language, you express ordinary
mathematical formulas and statements as sequences of math symbols such
as \texttt{2 + 2 = 4} (five symbols, all constants).\footnote{To
eliminate ambiguity with other expressions, this is expressed in the set
theory database \texttt{set.mm} as \texttt{|- ( 2 + 2
 ) = 4 }, whose \LaTeX\ equivalent is $\vdash
(2+2)=4$.  The \,$\vdash$ means ``is a theorem'' and the
parentheses allow explicit associative grouping.}\index{turnstile
({$\,\vdash$})} They may even be English
sentences, as in \texttt{E is closed and bounded} (five symbols)---here
\texttt{E} would be a variable and the other four symbols constants.  In
principle, a Metamath database could be constructed to work with almost
any unambiguous English-language mathematical statement, but as a
practical matter the definitions needed to provide for all possible
syntax variations would be cumbersome and distracting and possibly have
subtle pitfalls accidentally built in.  We generally recommend that you
express mathematical statements with compact standard mathematical
symbols whenever possible and put their English-language descriptions in
comments.  Axioms\index{axiom} and definitions\index{definition}
(\texttt{\$a}\index{\texttt{\$a} statement} statements) are the only
places where Metamath will not detect an error, and doing this will help
reduce the number of definitions needed.

You are free to use any tokens\index{token} you like for math
symbols\index{math symbol}.  Appendix~\ref{ASCII} recommends token names to
use for symbols in set theory, and we suggest you adopt these in order to be
able to include the \texttt{set.mm} set theory database in your database.  For
printouts, you can convert the tokens in a database
to standard mathematical symbols with the \LaTeX\ typesetting program.  The
Metamath command \texttt{open tex} {\em filename}\index{\texttt{open tex} command}
produces output that can be read by \LaTeX.\index{latex@{\LaTeX}}
The correspondence
between tokens and the actual symbols is made by \texttt{latexdef}
statements inside a special database comment tagged
with \texttt{\$t}.\index{\texttt{\$t} comment}\index{typesetting comment}
  You can edit
this comment to change the definitions or add new ones.
Appendix~\ref{ASCII} describes how to do this in more detail.

% White space\index{white space} is normally used to separate math
% symbol\index{math symbol} tokens, but they may be juxtaposed without white
% space in \texttt{\$d}\index{\texttt{\$d} statement}, \texttt{\$e}\index{\texttt{\$e}
% statement}, \texttt{\$f}\index{\texttt{\$f} statement}, \texttt{\$a}\index{\texttt{\$a}
% statement}, and \texttt{\$p}\index{\texttt{\$p} statement} statements when no
% ambiguity will result.  Specifically, Metamath parses the math symbol sequence
% in one of these statements in the following manner:  when the math symbol
% sequence has been broken up into tokens\index{token} up to a given character,
% the next token is the longest string of characters that could constitute a
% math symbol that is active\index{active
% math symbol} at that point.  (See Section~\ref{scoping} for the
% definition of an active math symbol.)  For example, if \texttt{-}, \texttt{>}, and
% \texttt{->} are the only active math symbols, the juxtaposition \texttt{>-} will be
% interpreted as the two symbols \texttt{>} and \texttt{-}, whereas \texttt{->} will
% always be interpreted as that single symbol.\footnote{For better readability we
% recommend a white space between each token.  This also makes searching for a
% symbol easier to do with an editor.  Omission of optional white space is useful
% for reducing typing when assigning an expression to a temporary
% variable\index{temporary variable} with the \texttt{let variable} Metamath
% program command.}\index{\texttt{let variable} command}
%
% Keywords\index{keyword} may be placed next to math symbols without white
% space\index{white space} between them.\footnote{Again, we do not recommend
% this for readability.}
%
% The math symbols\index{math symbol} in \texttt{\$c}\index{\texttt{\$c} statement}
% and \texttt{\$v}\index{\texttt{\$v} statement} statements must always be separated
% by white space\index{white
% space}, for the obvious reason that these statements define the names
% of the symbols.
%
% Math symbols referred to in comments (see Section~\ref{comments}) must also be
% separated by white space.  This allows you to make comments about symbols that
% are not yet active\index{active
% math symbol}.  (The ``math mode'' feature of comments is also a quick and
% easy way to obtain word processing text with embedded mathematical symbols,
% independently of the main purpose of Metamath; the way to do this is described
% in Section~\ref{comments})

\subsubsection{Label Tokens}\index{token}\index{label}

Label tokens are used to identify Metamath\index{Metamath} statements for
later reference. Label tokens may contain only letters, digits, and the three
characters period, hyphen, and underscore:
\begin{verbatim}
. - _
\end{verbatim}

A label is {\bf declared}\index{label declaration} by placing it immediately
before the keyword of the statement it identifies.  For example, the label
\texttt{axiom.1} might be declared as follows:
\begin{verbatim}
axiom.1 $a |- x = x $.
\end{verbatim}

Each \texttt{\$e}\index{\texttt{\$e} statement},
\texttt{\$f}\index{\texttt{\$f} statement},
\texttt{\$a}\index{\texttt{\$a} statement}, and
\texttt{\$p}\index{\texttt{\$p} statement} statement in a database must
have a label declared for it.  No other statement types may have label
declarations.  Every label must be unique.

A label (and the statement it identifies) is {\bf referenced}\index{label
reference} by including the label between the \texttt{\$=}\index{\texttt{\$=}
keyword} and \texttt{\$.}\index{\texttt{\$.}\ keyword}\ keywords in a \texttt{\$p}
statement.  The sequence of labels\index{label sequence} between these two
keywords is called a {\bf proof}\index{proof}.  An example of a statement with
a proof that we will encounter later (Section~\ref{proof}) is
\begin{verbatim}
wnew $p wff ( s -> ( r -> p ) )
     $= ws wr wp w2 w2 $.
\end{verbatim}

You don't have to know what this means just yet, but you should know that the
label \texttt{wnew} is declared by this \texttt{\$p} statement and that the labels
\texttt{ws}, \texttt{wr}, \texttt{wp}, and \texttt{w2} are assumed to have been declared
earlier in the database and are referenced here.

\subsection{Constants and Variables}
\index{constant}
\index{variable}

An {\bf expression}\index{expression} is any sequence of math
symbols, possibly empty.

The basic Metamath\index{Metamath} language\index{basic language} has two
kinds of math symbols\index{math symbol}:  {\bf constants}\index{constant} and
{\bf variables}\index{variable}.  In a Metamath proof, a constant may not be
substituted with any expression.  A variable can be
substituted\index{substitution!variable}\index{variable substitution} with any
expression.  This sequence may include other variables and may even include
the variable being substituted.  This substitution takes place when proofs are
verified, and it will be described in Section~\ref{proof}.  The \texttt{\$f}
statement (described later in Section~\ref{dollaref}) is used to specify the
{\bf type} of a variable (i.e.\ what kind of
variable it is)\index{variable type}\index{type} and
give it a meaning typically
associated with a ``metavariable''\index{metavariable}\footnote{A metavariable
is a variable that ranges over the syntactical elements of the object language
being discussed; for example, one metavariable might represent a variable of
the object language and another metavariable might represent a formula in the
object language.} in ordinary mathematics; for example, a variable may be
specified to be a wff or well-formed formula (in logic), a set (in set
theory), or a non-negative integer (in number theory).

%\subsection{The \texttt{\$c} and \texttt{\$v} Declaration Statements}
\subsection{The \texttt{\$c} and \texttt{\$v} Declaration Statements}
\index{\texttt{\$c} statement}
\index{constant declaration}
\index{\texttt{\$v} statement}
\index{variable declaration}

Constants are introduced or {\bf declared}\index{constant declaration}
with \texttt{\$c}\index{\texttt{\$c} statement} statements, and
variables are declared\index{variable declaration} with
\texttt{\$v}\index{\texttt{\$v} statement} statements.  A {\bf simple}
declaration\index{simple declaration} statement introduces a single
constant or variable.  Its syntax is one of the following:
\begin{center}
  \texttt{\$c} {\em math-symbol} \texttt{\$.}\\
  \texttt{\$v} {\em math-symbol} \texttt{\$.}
\end{center}
The notation {\em math-symbol} means any math symbol token\index{token}.

Some examples of simple declaration statements are:
\begin{center}
  \texttt{\$c + \$.}\\
  \texttt{\$c -> \$.}\\
  \texttt{\$c ( \$.}\\
  \texttt{\$v x \$.}\\
  \texttt{\$v y2 \$.}
\end{center}

The characters in a math symbol\index{math symbol} being declared are
irrelevant to Meta\-math; for example, we could declare a right parenthesis to
be a variable,
\begin{center}
  \texttt{\$v ) \$.}\\
\end{center}
although this would be unconventional.

A {\bf compound} declaration\index{compound declaration} statement is a
shorthand for declaring several symbols at once.  Its syntax is one of the
following:
\begin{center}
  \texttt{\$c} {\em math-symbol}\ \,$\cdots$\ {\em math-symbol} \texttt{\$.}\\
  \texttt{\$v} {\em math-symbol}\ \,$\cdots$\ {\em math-symbol} \texttt{\$.}
\end{center}\index{\texttt{\$c} statement}
Here, the ellipsis (\ldots) means any number of {\em math-symbol}\,s.

An example of a compound declaration statement is:
\begin{center}
  \texttt{\$v x y mu \$.}\\
\end{center}
This is equivalent to the three simple declaration statements
\begin{center}
  \texttt{\$v x \$.}\\
  \texttt{\$v y \$.}\\
  \texttt{\$v mu \$.}\\
\end{center}
\index{\texttt{\$v} statement}

There are certain rules on where in the database math symbols may be declared,
what sections of the database are aware of them (i.e.\ where they are
``active''), and when they may be declared more than once.  These will be
discussed in Section~\ref{scoping} and specifically on
p.~\pageref{redeclaration}.

\subsection{The \texttt{\$d} Statement}\label{dollard}
\index{\texttt{\$d} statement}

The \texttt{\$d} statement is called a {\bf disjoint-variable restriction}.  The
syntax of the {\bf simple} version of this statement is
\begin{center}
  \texttt{\$d} {\em variable variable} \texttt{\$.}
\end{center}
where each {\em variable} is a previously declared variable and the two {\em
variable}\,s are different.  (More specifically, each  {\em variable} must be
an {\bf active} variable\index{active math symbol}, which means there must be
a previous \texttt{\$v} statement whose {\bf scope}\index{scope} includes the
\texttt{\$d} statement.  These terms will be defined when we discuss scoping
statements in Section~\ref{scoping}.)

In ordinary mathematics, formulas may arise that are true if the variables in
them are distinct\index{distinct variables}, but become false when those
variables are made identical. For example, the formula in logic $\exists x\,x
\neq y$, which means ``for a given $y$, there exists an $x$ that is not equal
to $y$,'' is true in most mathematical theories (namely all non-trivial
theories\index{non-trivial theory}, i.e.\ those that describe more than one
individual, such as arithmetic).  However, if we substitute $y$ with $x$, we
obtain $\exists x\,x \neq x$, which is always false, as it means ``there
exists something that is not equal to itself.''\footnote{If you are a
logician, you will recognize this as the improper substitution\index{proper
substitution}\index{substitution!proper} of a free variable\index{free
variable} with a bound variable\index{bound variable}.  Metamath makes no
inherent distinction between free and bound variables; instead, you let
Metamath know what substitutions are permissible by using \texttt{\$d} statements
in the right way in your axiom system.}\index{free vs.\ bound variable}  The
\texttt{\$d} statement allows you to specify a restriction that forbids the
substitution of one variable with another.  In
this case, we would use the statement
\begin{center}
  \texttt{\$d x y \$.}
\end{center}\index{\texttt{\$d} statement}
to specify this restriction.

The order in which the variables appear in a \texttt{\$d} statement is not
important.  We could also use
\begin{center}
  \texttt{\$d y x \$.}
\end{center}

The \texttt{\$d} statement is actually more general than this, as the
``disjoint''\index{disjoint variables} in its name suggests.  The full meaning
is that if any substitution is made to its two variables (during the
course of a proof that references a \texttt{\$a} or \texttt{\$p} statement
associated with the \texttt{\$d}), the two expressions that result from the
substitution must have no variables in common.  In addition, each possible
pair of variables, one from each expression, must be in a \texttt{\$d} statement
associated with the statement being proved.  (This requirement forces the
statement being proved to ``inherit'' the original disjoint variable
restriction.)

For example, suppose \texttt{u} is a variable.  If the restriction
\begin{center}
  \texttt{\$d A B \$.}
\end{center}
has been specified for a theorem referenced in a
proof, we may not substitute \texttt{A} with \mbox{\tt a + u} and
\texttt{B} with \mbox{\tt b + u} because these two symbol sequences have the
variable \texttt{u} in common.  Furthermore, if \texttt{a} and \texttt{b} are
variables, we may not substitute \texttt{A} with \texttt{a} and \texttt{B} with \texttt{b}
unless we have also specified \texttt{\$d a b} for the theorem being proved; in
other words, the \texttt{\$d} property associated with a pair of variables must
be effectively preserved after substitution.

The \texttt{\$d}\index{\texttt{\$d} statement} statement does {\em not} mean ``the
two variables may not be substituted with the same thing,'' as you might think
at first.  For example, substituting each of \texttt{A} and \texttt{B} in the above
example with identical symbol sequences consisting only of constants does not
cause a disjoint variable conflict, because two symbol sequences have no
variables in common (since they have no variables, period).  Similarly, a
conflict will not occur by substituting the two variables in a \texttt{\$d}
statement with the empty symbol sequence\index{empty substitution}.

The \texttt{\$d} statement does not have a direct counterpart in
ordinary mathematics, partly because the variables\index{variable} of
Metamath are not really the same as the variables\index{variable!in
ordinary mathematics} of ordinary mathematics but rather are
metavariables\index{metavariable} ranging over them (as well as over
other kinds of symbols and groups of symbols).  Depending on the
situation, we may informally interpret the \texttt{\$d} statement in
different ways.  Suppose, for example, that \texttt{x} and \texttt{y}
are variables ranging over numbers (more precisely, that \texttt{x} and
\texttt{y} are metavariables ranging over variables that range over
numbers), and that \texttt{ph} ($\varphi$) and \texttt{ps} ($\psi$) are
variables (more precisely, metavariables) ranging over formulas.  We can
make the following interpretations that correspond to the informal
language of ordinary mathematics:
\begin{quote}
\begin{tabbing}
\texttt{\$d x y \$.} means ``assume $x$ and $y$ are
distinct variables.''\\
\texttt{\$d x ph \$.} means ``assume $x$ does not
occur in $\varphi$.''\\
\texttt{\$d ph ps \$.} \=means ``assume $\varphi$ and
$\psi$ have no variables\\ \>in common.''
\end{tabbing}
\end{quote}\index{\texttt{\$d} statement}

\subsubsection{Compound \texttt{\$d} Statements}

The {\bf compound} version of the \texttt{\$d} statement is a shorthand for
specifying several variables whose substitutions must be pairwise disjoint.
Its syntax is:
\begin{center}
  \texttt{\$d} {\em variable}\ \,$\cdots$\ {\em variable} \texttt{\$.}
\end{center}\index{\texttt{\$d} statement}
Here, {\em variable} represents the token of a previously declared
variable (specifically, an active variable) and all {\em variable}\,s are
different.  The compound \texttt{\$d}
statement is internally broken up by Metamath into one simple \texttt{\$d}
statement for each possible pair of variables in the original \texttt{\$d}
statement.  For example,
\begin{center}
  \texttt{\$d w x y z \$.}
\end{center}
is equivalent to
\begin{center}
  \texttt{\$d w x \$.}\\
  \texttt{\$d w y \$.}\\
  \texttt{\$d w z \$.}\\
  \texttt{\$d x y \$.}\\
  \texttt{\$d x z \$.}\\
  \texttt{\$d y z \$.}
\end{center}

Two or more simple \texttt{\$d} statements specifying the same variable pair are
internally combined into a single \texttt{\$d} statement.  Thus the set of three
statements
\begin{center}
  \texttt{\$d x y \$.}
  \texttt{\$d x y \$.}
  \texttt{\$d y x \$.}
\end{center}
is equivalent to
\begin{center}
  \texttt{\$d x y \$.}
\end{center}

Similarly, compound \texttt{\$d} statements, after being internally broken up,
internally have their common variable pairs combined.  For example the
set of statements
\begin{center}
  \texttt{\$d x y A \$.}
  \texttt{\$d x y B \$.}
\end{center}
is equivalent to
\begin{center}
  \texttt{\$d x y \$.}
  \texttt{\$d x A \$.}
  \texttt{\$d y A \$.}
  \texttt{\$d x y \$.}
  \texttt{\$d x B \$.}
  \texttt{\$d y B \$.}
\end{center}
which is equivalent to
\begin{center}
  \texttt{\$d x y \$.}
  \texttt{\$d x A \$.}
  \texttt{\$d y A \$.}
  \texttt{\$d x B \$.}
  \texttt{\$d y B \$.}
\end{center}

Metamath\index{Metamath} automatically verifies that all \texttt{\$d}
restrictions are met whenever it verifies proofs.  \texttt{\$d} statements are
never referenced directly in proofs (this is why they do not have
labels\index{label}), but Metamath is always aware of which ones must be
satisfied (i.e.\ are active) and will notify you with an error message if any
violation occurs.

To illustrate how Metamath detects a missing \texttt{\$d}
statement, we will look at the following example from the
\texttt{set.mm} database.

\begin{verbatim}
$d x z $.  $d y z $.
$( Theorem to add distinct quantifier to atomic formula. $)
ax17eq $p |- ( x = y -> A. z x = y ) $=...
\end{verbatim}

This statement has the obvious requirement that $z$ must be
distinct\index{distinct variables} from $x$ in theorem \texttt{ax17eq} that
states $x=y \rightarrow \forall z \, x=y$ (well, obvious if you're a logician,
for otherwise we could conclude  $x=y \rightarrow \forall x \, x=y$, which is
false when the free variables $x$ and $y$ are equal).

Let's look at what happens if we edit the database to comment out this
requirement.

\begin{verbatim}
$( $d x z $. $) $d y z $.
$( Theorem to add distinct quantifier to atomic formula. $)
ax17eq $p |- ( x = y -> A. z x = y ) $=...
\end{verbatim}

When it tries to verify the proof, Metamath will tell you that \texttt{x} and
\texttt{z} must be disjoint, because one of its steps references an axiom or
theorem that has this requirement.

\begin{verbatim}
MM> verify proof ax17eq
ax17eq ?Error at statement 1918, label "ax17eq", type "$p":
      vz wal wi vx vy vz ax-13 vx vy weq vz vx ax-c16 vx vy
                                               ^^^^^
There is a disjoint variable ($d) violation at proof step 29.
Assertion "ax-c16" requires that variables "x" and "y" be
disjoint.  But "x" was substituted with "z" and "y" was
substituted with "x".  The assertion being proved, "ax17eq",
does not require that variables "z" and "x" be disjoint.
\end{verbatim}

We can see the substitutions into \texttt{ax-c16} with the following command.

\begin{verbatim}
MM> show proof ax17eq / detailed_step 29
Proof step 29:  pm2.61dd.2=ax-c16 $a |- ( A. z z = x -> ( x =
  y -> A. z x = y ) )
This step assigns source "ax-c16" ($a) to target "pm2.61dd.2"
($e).  The source assertion requires the hypotheses "wph"
($f, step 26), "vx" ($f, step 27), and "vy" ($f, step 28).
The parent assertion of the target hypothesis is "pm2.61dd"
($p, step 36).
The source assertion before substitution was:
    ax-c16 $a |- ( A. x x = y -> ( ph -> A. x ph ) )
The following substitutions were made to the source
assertion:
    Variable  Substituted with
     x         z
     y         x
     ph        x = y
The target hypothesis before substitution was:
    pm2.61dd.2 $e |- ( ph -> ch )
The following substitutions were made to the target
hypothesis:
    Variable  Substituted with
     ph        A. z z = x
     ch        ( x = y -> A. z x = y )
\end{verbatim}

The disjoint variable restrictions of \texttt{ax-c16} can be seen from the
\texttt{show state\-ment} command.  The line that begins ``\texttt{Its mandatory
dis\-joint var\-i\-able pairs are:}\ldots'' lists any \texttt{\$d} variable
pairs in brackets.

\begin{verbatim}
MM> show statement ax-c16/full
Statement 3033 is located on line 9338 of the file "set.mm".
"Axiom of Distinct Variables. ..."
  ax-c16 $a |- ( A. x x = y -> ( ph -> A. x ph ) ) $.
Its mandatory hypotheses in RPN order are:
  wph $f wff ph $.
  vx $f setvar x $.
  vy $f setvar y $.
Its mandatory disjoint variable pairs are:  <x,y>
The statement and its hypotheses require the variables:  x y
      ph
The variables it contains are:  x y ph
\end{verbatim}

Since Metamath will always detect when \texttt{\$d}\index{\texttt{\$d} statement}
statements are needed for a proof, you don't have to worry too much about
forgetting to put one in; it can always be added if you see the error message
above.  If you put in unnecessary \texttt{\$d} statements, the worst that could
happen is that your theorem might not be as general as it could be, and this
may limit its use later on.

On the other hand, when you introduce axioms (\texttt{\$a}\index{\texttt{\$a}
statement} statements), you must be very careful to properly specify the
necessary associated \texttt{\$d} statements since Metamath has no way of knowing
whether your axioms are correct.  For example, Metamath would have no idea
that \texttt{ax-c16}, which we are telling it is an axiom of logic, would lead to
contradictions if we omitted its associated \texttt{\$d} statement.

% This was previously a comment in footnote-sized type, but it can be
% hard to read this much text in a small size.
% As a result, it's been changed to normally-sized text.
\label{nodd}
You may wonder if it is possible to develop standard
mathematics in the Metamath language without the \texttt{\$d}\index{\texttt{\$d}
statement} statement, since it seems like a nuisance that complicates proof
verification. The \texttt{\$d} statement is not needed in certain subsets of
mathematics such as propositional calculus.  However, dummy
variables\index{dummy variable!eliminating} and their associated \texttt{\$d}
statements are impossible to avoid in proofs in standard first-order logic as
well as in the variant used in \texttt{set.mm}.  In fact, there is no upper bound to
the number of dummy variables that might be needed in a proof of a theorem of
first-order logic containing 3 or more variables, as shown by H.\
Andr\'{e}ka\index{Andr{\'{e}}ka, H.} \cite{Nemeti}.  A first-order system that
avoids them entirely is given in \cite{Megill}\index{Megill, Norman}; the
trick there is simply to embed harmlessly the necessary dummy variables into a
theorem being proved so that they aren't ``dummy'' anymore, then interpret the
resulting longer theorem so as to ignore the embedded dummy variables.  If
this interests you, the system in \texttt{set.mm} obtained from \texttt{ax-1}
through \texttt{ax-c14} in \texttt{set.mm}, and deleting \texttt{ax-c16} and \texttt{ax-5},
requires no \texttt{\$d} statements but is logically complete in the sense
described in \cite{Megill}.  This means it can prove any theorem of
first-order logic as long as we add to the theorem an antecedent that embeds
dummy and any other variables that must be distinct.  In a similar fashion,
axioms for set theory can be devised that
do not require distinct variable
provisos\index{Set theory without distinct variable provisos},
as explained at
\url{http://us.metamath.org/mpeuni/mmzfcnd.html}.
Together, these in principle allow all of
mathematics to be developed under Metamath without a \texttt{\$d} statement,
although the length of the resulting theorems will grow as more and
more dummy variables become required in their proofs.

\subsection{The \texttt{\$f}
and \texttt{\$e} Statements}\label{dollaref}
\index{\texttt{\$e} statement}
\index{\texttt{\$f} statement}
\index{floating hypothesis}
\index{essential hypothesis}
\index{variable-type hypothesis}
\index{logical hypothesis}
\index{hypothesis}

Metamath has two kinds of hypo\-theses, the \texttt{\$f}\index{\texttt{\$f}
statement} or {\bf variable-type} hypothesis and the \texttt{\$e} or {\bf logical}
hypo\-the\-sis.\index{\texttt{\$d} statement}\footnote{Strictly speaking, the
\texttt{\$d} statement is also a hypothesis, but it is never directly referenced
in a proof, so we call it a restriction rather than a hypothesis to lessen
confusion.  The checking for violations of \texttt{\$d} restrictions is automatic
and built into Metamath's proof-checking algorithm.} The letters \texttt{f} and
\texttt{e} stand for ``floating''\index{floating hypothesis} (roughly meaning
used only if relevant) and ``essential''\index{essential hypothesis} (meaning
always used) respectively, for reasons that will become apparent
when we discuss frames in
Section~\ref{frames} and scoping in Section~\ref{scoping}. The syntax of these
are as follows:
\begin{center}
  {\em label} \texttt{\$f} {\em typecode} {\em variable} \texttt{\$.}\\
  {\em label} \texttt{\$e} {\em typecode}
      {\em math-symbol}\ \,$\cdots$\ {\em math-symbol} \texttt{\$.}\\
\end{center}
\index{\texttt{\$e} statement}
\index{\texttt{\$f} statement}
A hypothesis must have a {\em label}\index{label}.  The expression in a
\texttt{\$e} hypothesis consists of a typecode (an active constant math symbol)
followed by a sequence
of zero or more math symbols. Each math symbol (including {\em constant}
and {\em variable}) must be a previously declared constant or variable.  (In
addition, each math symbol must be active, which will be covered when we
discuss scoping statements in Section~\ref{scoping}.)  You use a \texttt{\$f}
hypothesis to specify the
nature or {\bf type}\index{variable type}\index{type} of a variable (such as ``let $x$ be an
integer'') and use a \texttt{\$e} hypothesis to express a logical truth (such as
``assume $x$ is prime'') that must be established in order for an assertion
requiring it to also be true.

A variable must have its type specified in a \texttt{\$f} statement before
it may be used in a \texttt{\$e}, \texttt{\$a}, or \texttt{\$p}
statement.  There may be only one (active) \texttt{\$f} statement for a
given variable.  (``Active'' is defined in Section~\ref{scoping}.)

In ordinary mathematics, theorems\index{theorem} are often expressed in the
form ``Assume $P$; then $Q$,'' where $Q$ is a statement that you can derive
if you start with statement $P$.\index{free variable}\footnote{A stronger
version of a theorem like this would be the {\em single} formula $P\rightarrow
Q$ ($P$ implies $Q$) from which the weaker version above follows by the rule
of modus ponens in logic.  We are not discussing this stronger form here.  In
the weaker form, we are saying only that if we can {\em prove} $P$, then we can
{\em prove} $Q$.  In a logician's language, if $x$ is the only free variable
in $P$ and $Q$, the stronger form is equivalent to $\forall x ( P \rightarrow
Q)$ (for all $x$, $P$ implies $Q$), whereas the weaker form is equivalent to
$\forall x P \rightarrow \forall x Q$. The stronger form implies the weaker,
but not vice-versa.  To be precise, the weaker form of the theorem is more
properly called an ``inference'' rather than a theorem.}\index{inference}
In the
Metamath\index{Metamath} language, you would express mathematical statement
$P$ as a hypothesis (a \texttt{\$e} Metamath language statement in this case) and
statement $Q$ as a provable assertion (a \texttt{\$p}\index{\texttt{\$p} statement}
statement).

Some examples of hypotheses you might encounter in logic and set theory are
\begin{center}
  \texttt{stmt1 \$f wff P \$.}\\
  \texttt{stmt2 \$f setvar x \$.}\\
  \texttt{stmt3 \$e |- ( P -> Q ) \$.}
\end{center}
\index{\texttt{\$e} statement}
\index{\texttt{\$f} statement}
Informally, these would be read, ``Let $P$ be a well-formed-formula,'' ``Let
$x$ be an (individual) variable,'' and ``Assume we have proved $P \rightarrow
Q$.''  The turnstile symbol \,$\vdash$\index{turnstile ({$\,\vdash$})} is
commonly used in logic texts to mean ``a proof exists for.''

To summarize:
\begin{itemize}
\item A \texttt{\$f} hypothesis tells Metamath the type or kind of its variable.
It is analogous to a variable declaration in a computer language that
tells the compiler that a variable is an integer or a floating-point
number.
\item The \texttt{\$e} hypothesis corresponds to what you would usually call a
``hypothesis'' in ordinary mathematics.
\end{itemize}

Before an assertion\index{assertion} (\texttt{\$a} or \texttt{\$p} statement) can be
referenced in a proof, all of its associated \texttt{\$f} and \texttt{\$e} hypotheses
(i.e.\ those \texttt{\$e} hypotheses that are active) must be satisfied (i.e.
established by the proof).  The meaning of ``associated'' (which we will call
{\bf mandatory} in Section~\ref{frames}) will become clear when we discuss
scoping later.

Note that after any \texttt{\$f}, \texttt{\$e},
\texttt{\$a}, or \texttt{\$p} token there is a required
\textit{typecode}\index{typecode}.
The typecode is a constant used to enforce types of expressions.
This will become clearer once we learn more about
assertions (\texttt{\$a} and \texttt{\$p} statements).
An example may also clarify their purpose.
In the
\texttt{set.mm}\index{set theory database (\texttt{set.mm})}%
\index{Metamath Proof Explorer}
database,
the following typecodes are used:

\begin{itemize}
\item \texttt{wff} :
  Well-formed formula (wff) symbol
  (read: ``the following symbol sequence is a wff'').
% The *textual* typecode for turnstile is "|-", but when read it's a little
% confusing, so I intentionally display the mathematical symbol here instead
% (I think it's clearer in this context).
\item \texttt{$\vdash$} :
  Turnstile (read: ``the following symbol sequence is provable'' or
  ``a proof exists for'').
\item \texttt{setvar} :
  Individual set variable type (read: ``the following is an
  individual set variable'').
  Note that this is \textit{not} the type of an arbitrary set expression,
  instead, it is used to ensure that there is only a single symbol used
  after quantifiers like for-all ($\forall$) and there-exists ($\exists$).
\item \texttt{class} :
  An expression that is a syntactically valid class expression.
  All valid set expressions are also valid class expression, so expressions
  of sets normally have the \texttt{class} typecode.
  Use the \texttt{class} typecode,
  \textit{not} the \texttt{setvar} typecode,
  for the type of set expressions unless you are specifically identifying
  a single set variable.
\end{itemize}

\subsection{Assertions (\texttt{\$a} and \texttt{\$p} Statements)}
\index{\texttt{\$a} statement}
\index{\texttt{\$p} statement}\index{assertion}\index{axiomatic assertion}
\index{provable assertion}

There are two types of assertions, \texttt{\$a}\index{\texttt{\$a} statement}
statements ({\bf axiomatic assertions}) and \texttt{\$p} statements ({\bf
provable assertions}).  Their syntax is as follows:
\begin{center}
  {\em label} \texttt{\$a} {\em typecode} {\em math-symbol} \ldots
         {\em math-symbol} \texttt{\$.}\\
  {\em label} \texttt{\$p} {\em typecode} {\em math-symbol} \ldots
        {\em math-symbol} \texttt{\$=} {\em proof} \texttt{\$.}
\end{center}
\index{\texttt{\$a} statement}
\index{\texttt{\$p} statement}
\index{\texttt{\$=} keyword}
An assertion always requires a {\em label}\index{label}. The expression in an
assertion consists of a typecode (an active constant)
followed by a sequence of zero
or more math symbols.  Each math symbol, including any {\em constant}, must be a
previously declared constant or variable.  (In addition, each math symbol
must be active, which will be covered when we discuss scoping statements in
Section~\ref{scoping}.)

A \texttt{\$a} statement is usually a definition of syntax (for example, if $P$
and $Q$ are wffs then so is $(P\to Q)$), an axiom\index{axiom} of ordinary
mathematics (for example, $x=x$), or a definition\index{definition} of
ordinary mathematics (for example, $x\ne y$ means $\lnot x=y$). A \texttt{\$p}
statement is a claim that a certain combination of math symbols follows from
previous assertions and is accompanied by a proof that demonstrates it.

Assertions can also be referenced in (later) proofs in order to derive new
assertions from them. The label of an assertion is used to refer to it in a
proof. Section~\ref{proof} will describe the proof in detail.

Assertions also provide the primary means for communicating the mathematical
results in the database to people.  Proofs (when conveniently displayed)
communicate to people how the results were arrived at.

\subsubsection{The \texttt{\$a} Statement}
\index{\texttt{\$a} statement}

Axiomatic assertions (\texttt{\$a} statements) represent the starting points from
which other assertions (\texttt{\$p}\index{\texttt{\$p} statement} statements) are
derived.  Their most obvious use is for specifying ordinary mathematical
axioms\index{axiom}, but they are also used for two other purposes.

First, Metamath\index{Metamath} needs to know the syntax of symbol
sequences that constitute valid mathematical statements.  A Metamath
proof must be broken down into much more detail than ordinary
mathematical proofs that you may be used to thinking of (even the
``complete'' proofs of formal logic\index{formal logic}).  This is one
of the things that makes Metamath a general-purpose language,
independent of any system of logic or even syntax.  If you want to use a
substitution instance of an assertion as a step in a proof, you must
first prove that the substitution is syntactically correct (or if you
prefer, you must ``construct'' it), showing for example that the
expression you are substituting for a wff metavariable is a valid wff.
The \texttt{\$a}\index{\texttt{\$a} statement} statement is used to
specify those combinations of symbols that are considered syntactically
valid, such as the legal forms of wffs.

Second, \texttt{\$a} statements are used to specify what are ordinarily thought of
as definitions, i.e.\ new combinations of symbols that abbreviate other
combinations of symbols.  Metamath makes no distinction\index{axiom vs.\
definition} between axioms\index{axiom} and definitions\index{definition}.
Indeed, it has been argued that such distinction should not be made even in
ordinary mathematics; see Section~\ref{definitions}, which discusses the
philosophy of definitions.  Section~\ref{hierarchy} discusses some
technical requirements for definitions.  In \texttt{set.mm} we adopt the
convention of prefixing axiom labels with \texttt{ax-} and definition labels with
\texttt{df-}\index{label}.

The results that can be derived with the Metamath language are only as good as
the \texttt{\$a}\index{\texttt{\$a} statement} statements used as their starting
point.  We cannot stress this too strongly.  For example, Metamath will
not prevent you from specifying $x\neq x$ as an axiom of logic.  It is
essential that you scrutinize all \texttt{\$a} statements with great care.
Because they are a source of potential pitfalls, it is best not to add new
ones (usually new definitions) casually; rather you should carefully evaluate
each one's necessity and advantages.

Once you have in place all of the basic axioms\index{axiom} and
rules\index{rule} of a mathematical theory, the only \texttt{\$a} statements that
you will be adding will be what are ordinarily called definitions.  In
principle, definitions should be in some sense eliminable from the language of
a theory according to some convention (usually involving logical equivalence
or equality).  The most common convention is that any formula that was
syntactically valid but not provable before the definition was introduced will
not become provable after the definition is introduced.  In an ideal world,
definitions should not be present at all if one is to have absolute confidence
in a mathematical result.  However, they are necessary to make
mathematics practical, for otherwise the resulting formulas would be
extremely long and incomprehensible.  Since the nature of definitions (in the
most general sense) does not permit them to automatically be verified as
``proper,''\index{proper definition}\index{definition!proper} the judgment of
the mathematician is required to ensure it.  (In \texttt{set.mm} effort was made
to make almost all definitions directly eliminable and thus minimize the need
for such judgment.)

If you are not a mathematician, it may be best not to add or change any
\texttt{\$a}\index{\texttt{\$a} statement} statements but instead use
the mathematical language already provided in standard databases.  This
way Metamath will not allow you to make a mistake (i.e.\ prove a false
result).


\subsection{Frames}\label{frames}

We now introduce the concept of a collection of related Metamath statements
called a frame.  Every assertion (\texttt{\$a} or \texttt{\$p} statement) in the database has
an associated frame.

A {\bf frame}\index{frame} is a sequence of \texttt{\$d}, \texttt{\$f},
and \texttt{\$e} statements (zero or more of each) followed by one
\texttt{\$a} or \texttt{\$p} statement, subject to certain conditions we
will describe.  For simplicity we will assume that all math symbol
tokens used are declared at the beginning of the database with
\texttt{\$c} and \texttt{\$v} statements (which are not properly part of
a frame).  Also for simplicity we will assume there are only simple
\texttt{\$d} statements (those with only two variables) and imagine any
compound \texttt{\$d} statements (those with more than two variables) as
broken up into simple ones.

A frame groups together those hypotheses (and \texttt{\$d} statements) relevant
to an assertion (\texttt{\$a} or \texttt{\$p} statement).  The statements in a frame
may or may not be physically adjacent in a database; we will cover
this in our discussion of scoping statements
in Section~\ref{scoping}.

A frame has the following properties:
\begin{enumerate}
 \item The set of variables contained in its \texttt{\$f} statements must
be identical to the set of variables contained in its \texttt{\$e},
\texttt{\$a}, and/or \texttt{\$p} statements.  In other words, each
variable in a \texttt{\$e}, \texttt{\$a}, or \texttt{\$p} statement must
have an associated ``variable type'' defined for it in a \texttt{\$f}
statement.
  \item No two \texttt{\$f} statements may contain the same variable.
  \item Any \texttt{\$f} statement
must occur before a \texttt{\$e} statement in which its variable occurs.
\end{enumerate}

The first property determines the set of variables occurring in a frame.
These are the {\bf mandatory
variables}\index{mandatory variable} of the frame.  The second property
tells us there must be only one type specified for a variable.
The last property is not a theoretical requirement but it
makes parsing of the database easier.

For our examples, we assume our database has the following declarations:

\begin{verbatim}
$v P Q R $.
$c -> ( ) |- wff $.
\end{verbatim}

The following sequence of statements, describing the modus ponens inference
rule, is an example of a frame:

\begin{verbatim}
wp  $f wff P $.
wq  $f wff Q $.
maj $e |- ( P -> Q ) $.
min $e |- P $.
mp  $a |- Q $.
\end{verbatim}

The following sequence of statements is not a frame because \texttt{R} does not
occur in the \texttt{\$e}'s or the \texttt{\$a}:

\begin{verbatim}
wp  $f wff P $.
wq  $f wff Q $.
wr  $f wff R $.
maj $e |- ( P -> Q ) $.
min $e |- P $.
mp  $a |- Q $.
\end{verbatim}

The following sequence of statements is not a frame because \texttt{Q} does not
occur in a \texttt{\$f}:

\begin{verbatim}
wp  $f wff P $.
maj $e |- ( P -> Q ) $.
min $e |- P $.
mp  $a |- Q $.
\end{verbatim}

The following sequence of statements is not a frame because the \texttt{\$a} statement is
not the last one:

\begin{verbatim}
wp  $f wff P $.
wq  $f wff Q $.
maj $e |- ( P -> Q ) $.
mp  $a |- Q $.
min $e |- P $.
\end{verbatim}

Associated with a frame is a sequence of {\bf mandatory
hypotheses}\index{mandatory hypothesis}.  This is simply the set of all
\texttt{\$f} and \texttt{\$e} statements in the frame, in the order they
appear.  A frame can be referenced in a later proof using the label of
the \texttt{\$a} or \texttt{\$p} assertion statement, and the proof
makes an assignment to each mandatory hypothesis in the order in which
it appears.  This means the order of the hypotheses, once chosen, must
not be changed so as not to affect later proofs referencing the frame's
assertion statement.  (The Metamath proof verifier will, of course, flag
an error if a proof becomes incorrect by doing this.)  Since proofs make
use of ``Reverse Polish notation,'' described in Section~\ref{proof}, we
call this order the {\bf RPN order}\index{RPN order} of the hypotheses.

Note that \texttt{\$d} statements are not part of the set of mandatory
hypotheses, and their order doesn't matter (as long as they satisfy the
fourth property for a frame described above).  The \texttt{\$d}
statements specify restrictions on variables that must be satisfied (and
are checked by the proof verifier) when expressions are substituted for
them in a proof, and the \texttt{\$d} statements themselves are never
referenced directly in a proof.

A frame with a \texttt{\$p} (provable) statement requires a proof as part of the
\texttt{\$p} statement.  Sometimes in a proof we want to make use of temporary or
dummy variables\index{dummy variable} that do not occur in the \texttt{\$p}
statement or its mandatory hypotheses.  To accommodate this we define an {\bf
extended frame}\index{extended frame} as a frame together with zero or more
\texttt{\$d} and \texttt{\$f} statements that reference variables not among the
mandatory variables of the frame.  Any new variables referenced are called the
{\bf optional variables}\index{optional variable} of the extended frame. If a
\texttt{\$f} statement references an optional variable it is called an {\bf
optional hypothesis}\index{optional hypothesis}, and if one or both of the
variables in a \texttt{\$d} statement are optional variables it is called an {\bf
optional disjoint-variable restriction}\index{optional disjoint-variable
restriction}.  Properties 2 and 3 for a frame also apply to an extended
frame.

The concept of optional variables is not meaningful for frames with \texttt{\$a}
statements, since those statements have no proofs that might make use of them.
There is no restriction on including optional hypotheses in the extended frame
for a \texttt{\$a} statement, but they serve no purpose.

The following set of statements is an example of an extended frame, which
contains an optional variable \texttt{R} and an optional hypothesis \texttt{wr}.  In
this example, we suppose the rule of modus ponens is not an axiom but is
derived as a theorem from earlier statements (we omit its presumed proof).
Variable \texttt{R} may be used in its proof if desired (although this would
probably have no advantage in propositional calculus).  Note that the sequence
of mandatory hypotheses in RPN order is still \texttt{wp}, \texttt{wq}, \texttt{maj},
\texttt{min} (i.e.\ \texttt{wr} is omitted), and this sequence is still assumed
whenever the assertion \texttt{mp} is referenced in a subsequent proof.

\begin{verbatim}
wp  $f wff P $.
wq  $f wff Q $.
wr  $f wff R $.
maj $e |- ( P -> Q ) $.
min $e |- P $.
mp  $p |- Q $= ... $.
\end{verbatim}

Every frame is an extended frame, but not every extended frame is a frame, as
this example shows.  The underlying frame for an extended frame is
obtained by simply removing all statements containing optional variables.
Any proof referencing an assertion will ignore any extensions to its
frame, which means we may add or delete optional hypotheses at will without
affecting subsequent proofs.

The conceptually simplest way of organizing a Metamath database is as a
sequence of extended frames.  The scoping statements
\texttt{\$\char`\{}\index{\texttt{\$\char`\{} and \texttt{\$\char`\}}
keywords} and \texttt{\$\char`\}} can be used to delimit the start and
end of an extended frame, leading to the following possible structure for a
database.  \label{framelist}

\vskip 2ex
\setbox\startprefix=\hbox{\tt \ \ \ \ \ \ \ \ }
\setbox\contprefix=\hbox{}
\startm
\m{\mbox{(\texttt{\$v} {\em and} \texttt{\$c}\,{\em statements})}}
\endm
\startm
\m{\mbox{\texttt{\$\char`\{}}}
\endm
\startm
\m{\mbox{\texttt{\ \ } {\em extended frame}}}
\endm
\startm
\m{\mbox{\texttt{\$\char`\}}}}
\endm
\startm
\m{\mbox{\texttt{\$\char`\{}}}
\endm
\startm
\m{\mbox{\texttt{\ \ } {\em extended frame}}}
\endm
\startm
\m{\mbox{\texttt{\$\char`\}}}}
\endm
\startm
\m{\mbox{\texttt{\ \ \ \ \ \ \ \ \ }}\vdots}
\endm
\vskip 2ex

In practice, this structure is inconvenient because we have to repeat
any \texttt{\$f}, \texttt{\$e}, and \texttt{\$d} statements over and
over again rather than stating them once for use by several assertions.
The scoping statements, which we will discuss next, allow this to be
done.  In principle, any Metamath database can be converted to the above
format, and the above format is the most convenient to use when studying
a Metamath database as a formal system%
%% Uncomment this when uncommenting section {formalspec} below
   (Appendix \ref{formalspec})%
.
In fact, Metamath internally converts the database to the above format.
The command \texttt{show statement} in the Metamath program will show
you the contents of the frame for any \texttt{\$a} or \texttt{\$p}
statement, as well as its extension in the case of a \texttt{\$p}
statement.

%c%(provided that all ``local'' variables and constants with limited scope have
%c%unique names),

During our discussion of scoping statements, it may be helpful to
think in terms of the equivalent sequence of frames that will result when
the database is parsed.  Scoping (other than the limited
use above to delimit frames) is not a theoretical requirement for
Metamath but makes it more convenient.


\subsection{Scoping Statements (\texttt{\$\{} and \texttt{\$\}})}\label{scoping}
\index{\texttt{\$\char`\{} and \texttt{\$\char`\}} keywords}\index{scoping statement}

%c%Some Metamath statements may be needed only temporarily to
%c%serve a specific purpose, and after we're done with them we would like to
%c%disregard or ignore them.  For example, when we're finished using a variable,
%c%we might want to
%c%we might want to free up the token\index{token} used to name it so that the
%c%token can be used for other purposes later on, such as a different kind of
%c%variable or even a constant.  In the terminology of computer programming, we
%c%might want to let some symbol declarations be ``local'' rather than ``global.''
%c%\index{local symbol}\index{global symbol}

The {\bf scoping} statements, \texttt{\$\char`\{} ({\bf start of block}) and \texttt{\$\char`\}}
({\bf end of block})\index{block}, provide a means for controlling the portion
of a database over which certain statement types are recognized.  The
syntax of a scoping statement is very simple; it just consists of the
statement's keyword:
\begin{center}
\texttt{\$\char`\{}\\
\texttt{\$\char`\}}
\end{center}
\index{\texttt{\$\char`\{} and \texttt{\$\char`\}} keywords}

For example, consider the following database where we have stripped out
all tokens except the scoping statement keywords.  For the purpose of the
discussion, we have added subscripts to the scoping statements; these subscripts
do not appear in the actual database.
\[
 \mbox{\tt \ \$\char`\{}_1
 \mbox{\tt \ \$\char`\{}_2
 \mbox{\tt \ \$\char`\}}_2
 \mbox{\tt \ \$\char`\{}_3
 \mbox{\tt \ \$\char`\{}_4
 \mbox{\tt \ \$\char`\}}_4
 \mbox{\tt \ \$\char`\}}_3
 \mbox{\tt \ \$\char`\}}_1
\]
Each \texttt{\$\char`\{} statement in this example is said to be {\bf
matched} with the \texttt{\$\char`\}} statement that has the same
subscript.  Each pair of matched scoping statements defines a region of
the database called a {\bf block}.\index{block} Blocks can be {\bf
nested}\index{nested block} inside other blocks; in the example, the
block defined by $\mbox{\tt \$\char`\{}_4$ and $\mbox{\tt \$\char`\}}_4$
is nested inside the block defined by $\mbox{\tt \$\char`\{}_3$ and
$\mbox{\tt \$\char`\}}_3$ as well as inside the block defined by
$\mbox{\tt \$\char`\{}_1$ and $\mbox{\tt \$\char`\}}_1$.  In general, a
block may be empty, it may contain only non-scoping
statements,\footnote{Those statements other than \texttt{\$\char`\{} and
\texttt{\$\char`\}}.}\index{non-scoping statement} or it may contain any
mixture of other blocks and non-scoping statements.  (This is called a
``recursive'' definition\index{recursive definition} of a block.)

Associated with each block is a number called its {\bf nesting
level}\index{nesting level} that indicates how deeply the block is nested.
The nesting levels of the blocks in our example are as follows:
\[
  \underbrace{
    \mbox{\tt \ }
    \underbrace{
     \mbox{\tt \$\char`\{\ }
     \underbrace{
       \mbox{\tt \$\char`\{\ }
       \mbox{\tt \$\char`\}}
     }_{2}
     \mbox{\tt \ }
     \underbrace{
       \mbox{\tt \$\char`\{\ }
       \underbrace{
         \mbox{\tt \$\char`\{\ }
         \mbox{\tt \$\char`\}}
       }_{3}
       \mbox{\tt \ \$\char`\}}
     }_{2}
     \mbox{\tt \ \$\char`\}}
   }_{1}
   \mbox{\tt \ }
 }_{0}
\]
\index{\texttt{\$\char`\{} and \texttt{\$\char`\}} keywords}
The entire database is considered to be one big block (the {\bf outermost}
block) with a nesting level of 0.  The outermost block is {\em not} bracketed
by scoping statements.\footnote{The language was designed this way so that
several source files can be joined together more easily.}\index{outermost
block}

All non-scoping Metamath statements become recognized or {\bf
active}\index{active statement} at the place where they appear.\footnote{To
keep things slightly simpler, we do not bother to define the concept of
``active'' for the scoping statements.}  Certain of these statement types
become inactive at the end of the block in which they appear; these statement
types are:
\begin{center}
  \texttt{\$c}, \texttt{\$v}, \texttt{\$d}, \texttt{\$e}, and \texttt{\$f}.
%  \texttt{\$v}, \texttt{\$f}, \texttt{\$e}, and \texttt{\$d}.
\end{center}
\index{\texttt{\$c} statement}
\index{\texttt{\$d} statement}
\index{\texttt{\$e} statement}
\index{\texttt{\$f} statement}
\index{\texttt{\$v} statement}
The other statement types remain active forever (i.e.\ through the end of the
database); they are:
\begin{center}
  \texttt{\$a} and \texttt{\$p}.
%  \texttt{\$c}, \texttt{\$a}, and \texttt{\$p}.
\end{center}
\index{\texttt{\$a} statement}
\index{\texttt{\$p} statement}
Any statement (of these 7 types) located in the outermost
block\index{outermost block} will remain active through the end of the
database and thus are effectively ``global'' statements.\index{global
statement}

All \texttt{\$c} statements must be placed in the outermost block.  Since they are
therefore always global, they could be considered as belonging to both of the
above categories.

The {\bf scope}\index{scope} of a statement is the set of statements that
recognize it as active.

%c%The concept of ``active'' is also defined for math symbols\index{math
%c%symbol}.  Math symbols (constants\index{constant} and
%c%variables\index{variable}) become {\bf active}\index{active
%c%math symbol} in the \texttt{\$c}\index{\texttt{\$c}
%c%statement} and \texttt{\$v}\index{\texttt{\$v} statement} statements that
%c%declare them.  They become inactive when their declaration statements become
%c%inactive.

The concept of ``active'' is also defined for math symbols\index{math
symbol}.  Math symbols (constants\index{constant} and
variables\index{variable}) become {\bf active}\index{active math symbol}
in the \texttt{\$c}\index{\texttt{\$c} statement} and
\texttt{\$v}\index{\texttt{\$v} statement} statements that declare them.
A variable becomes inactive when its declaration statement becomes
inactive.  Because all \texttt{\$c} statements must be in the outermost
block, a constant will never become inactive after it is declared.

\subsubsection{Redeclaration of Math Symbols}
\index{redeclaration of symbols}\label{redeclaration}

%c%A math symbol may not be declared a second time while it is active, but it may
%c%be declared again after it becomes inactive.

A variable may not be declared a second time while it is active, but it may be
declared again after it becomes inactive.  This provides a convenient way to
introduce ``local'' variables,\index{local variable} i.e.\ temporary variables
for use in the frame of an assertion or in a proof without keeping them around
forever.  A previously declared variable may not be redeclared as a constant.

A constant may not be redeclared.  And, as mentioned above, constants must be
declared in the outermost block.

The reason variables may have limited scope but not constants is that an
assertion (\texttt{\$a} or \texttt{\$p} statement) remains available for use in
proofs through the end of the database.  Variables in an assertion's frame may
be substituted with whatever is needed in a proof step that references the
assertion, whereas constants remain fixed and may not be substituted with
anything.  The particular token used for a variable in an assertion's frame is
irrelevant when the assertion is referenced in a proof, and it doesn't matter
if that token is not available outside of the referenced assertion's frame.
Constants, however, must be globally fixed.

There is no theoretical
benefit for the feature allowing variables to be active for limited scopes
rather than global. It is just a convenience that allows them, for example, to
be locally grouped together with their corresponding \texttt{\$f} variable-type
declarations.

%c%If you declare a math symbol more than once, internally Metamath considers it a
%c%new distinct symbol, even though it has the same name.  If you are unaware of
%c%this, you may find that what you think are correct proofs are incorrectly
%c%rejected as invalid, because Metamath may tell you that a constant you
%c%previously declared does not match a newly declared math symbol with the same
%c%name.  For details on this subtle point, see the Comment on
%c%p.~\pageref{spec4comment}.  This is done purposely to allow temporary
%c%constants to be introduced while developing a subtheory, then allow their math
%c%symbol tokens to be reused later on; in general they will not refer to the
%c%same thing.  In practice, you would not ordinarily reuse the names of
%c%constants because it would tend to be confusing to the reader.  The reuse of
%c%names of variables, on the other hand, is something that is often useful to do
%c%(for example it is done frequently in \texttt{set.mm}).  Since variables in an
%c%assertion referenced in a proof can be substituted as needed to achieve a
%c%symbol match, this is not an issue.

% (This section covers a somewhat advanced topic you may want to skip
% at first reading.)
%
% Under certain circumstances, math symbol\index{math symbol}
% tokens\index{token} may be redeclared (i.e.\ the token
% may appear in more than
% one \texttt{\$c}\index{\texttt{\$c} statement} or \texttt{\$v}\index{\texttt{\$v}
% statement} statement).  You might want to do this say, to make temporary use
% of a variable name without having to worry about its affect elsewhere,
% somewhat analogous to declaring a local variable in a standard computer
% language.  Understanding what goes on when math symbol tokens are redeclared
% is a little tricky to understand at first, since it requires that we
% distinguish the token itself from the math symbol that it names.  It will help
% if we first take a peek at the internal workings of the
% Metamath\index{Metamath} program.
%
% Metamath reserves a memory location for each occurrence of a
% token\index{token} in a declaration statement (\texttt{\$c}\index{\texttt{\$c}
% statement} or \texttt{\$v}\index{\texttt{\$v} statement}).  If a given token appears
% in more than one declaration statement, it will refer to more than one memory
% locations.  A math symbol\index{math symbol} may be thought of as being one of
% these memory locations rather than as the token itself.  Only one of the
% memory locations associated with a given token may be active at any one time.
% The math symbol (memory location) that gets looked up when the token appears
% in a non-declaration statement is the one that happens to be active at that
% time.
%
% We now look at the rules for the redeclaration\index{redeclaration of symbols}
% of math symbol tokens.
% \begin{itemize}
% \item A math symbol token may not be declared twice in the
% same block.\footnote{While there is no theoretical reason for disallowing
% this, it was decided in the design of Metamath that allowing it would offer no
% advantage and might cause confusion.}
% \item An inactive math symbol may always be
% redeclared.
% \item  An active math symbol may be redeclared in a different (i.e.\
% inner) block\index{block} from the one it became active in.
% \end{itemize}
%
% When a math symbol token is redeclared, it conceptually refers to a different
% math symbol, just as it would be if it were called a different name.  In
% addition, the original math symbol that it referred to, if it was active,
% temporarily becomes inactive.  At the end of the block in which the
% redeclaration occurred, the new math symbol\index{math symbol} becomes
% inactive and the original symbol becomes active again.  This concept is
% illustrated in the following example, where the symbol \texttt{e} is
% ordinarily a constant (say Euler's constant, 2.71828...) but
% temporarily we want to use it as a ``local'' variable, say as a coefficient
% in the equation $a x^4 + b x^3 + c x^2 + d x + e$:
% \[
%   \mbox{\tt \$\char`\{\ \$c e \$.}
%   \underbrace{
%     \ \ldots\ %
%     \mbox{\tt \$\char`\{}\ \ldots\ %
%   }_{\mbox{\rm region A}}
%   \mbox{\tt \$v e \$.}
%   \underbrace{
%     \mbox{\ \ \ \ldots\ \ \ }
%   }_{\mbox{\rm region B}}
%   \mbox{\tt \$\char`\}}
%   \underbrace{
%     \mbox{\ \ \ \ldots\ \ \ }
%   }_{\mbox{\rm region C}}
%   \mbox{\tt \$\char`\}}
% \]
% \index{\texttt{\$\char`\{} and \texttt{\$\char`\}} keywords}
% In region A, the token \texttt{e} refers to a constant.  It is redeclared as a
% variable in region B, and any reference to it in this region will refer to this
% variable.  In region C, the redeclaration becomes inactive, and the original
% declaration becomes active again.  In region C, the token \texttt{x} refers to the
% original constant.
%
% As a practical matter, overuse of math symbol\index{math symbol}
% redeclarations\index{redeclaration of symbols} can be confusing (even though
% it is well-defined) and is best avoided when possible.  Here are some good
% general guidelines you can follow.  Usually, you should declare all
% constants\index{constant} in the outermost block\index{outermost block},
% especially if they are general-purpose (such as the token \verb$A.$, meaning
% $\forall$ or ``for all'').  This will make them ``globally'' active (although
% as in the example above local redeclarations will temporarily make them
% inactive.)  Most or all variables\index{variable}, on the other hand, could be
% declared in inner blocks, so that the token for them can be used later for a
% different type of variable or a constant.  (The names of the variables you
% choose are not used when you refer to an assertion\index{assertion} in a
% proof, whereas constants must match exactly.  A locally declared constant will
% not match a globally declared constant in a proof, even if they use the same
% token, because Metamath internally considers them to be different math
% symbols.)  To avoid confusion, you should generally avoid redeclaring active
% variables.  If you must redeclare them, do so at the beginning of a block.
% The temporary declaration of constants in inner blocks might be occasionally
% appropriate when you make use of a temporary definition to prove lemmas
% leading to a main result that does not make direct use of the definition.
% This way, you will not clutter up your database with a large number of
% seldom-used global constant symbols.  You might want to note that while
% inactive constants may not appear directly in an assertion (a \texttt{\$a}\index{\texttt{\$a}
% statement} or \texttt{\$p}\index{\texttt{\$p} statement}
% statement), they may be indirectly used in the proof of a \texttt{\$p} statement
% so long as they do not appear in the final math symbol sequence constructed by
% the proof.  In the end, you will have to use your best judgment, taking into
% account standard mathematical usage of the symbols as well as consideration
% for the reader of your work.
%
% \subsubsection{Reuse of Labels}\index{reuse of labels}\index{label}
%
% The \texttt{\$e}\index{\texttt{\$e} statement}, \texttt{\$f}\index{\texttt{\$f}
% statement}, \texttt{\$a}\index{\texttt{\$a} statement}, and
% \texttt{\$p}\index{\texttt{\$p}
% statement} statement types require labels, which allow them to be
% referenced later inside proofs.  A label is considered {\bf
% active}\index{active label} when the statement it is associated with is
% active.  The token\index{token} for a label may be reused
% (redeclared)\index{redeclaration of labels} provided that it is not being used
% for a currently active label.  (Unlike the tokens for math symbols, active
% label tokens may not be redeclared in an inner scope.)  Note that the labels
% of \texttt{\$a} and \texttt{\$p} statements can never be reused after these
% statements appear, because these statements remain active through the end of
% the database.
%
% You might find the reuse of labels a convenient way to have standard names for
% temporary hypotheses, such as \texttt{h1}, \texttt{h2}, etc.  This way you don't have
% to invent unique names for each of them, and in some cases it may be less
% confusing to the reader (although in other cases it might be more confusing, if
% the hypothesis is located far away from the assertion that uses
% it).\footnote{The current implementation requires that all labels, even
% inactive ones, be unique.}

\subsubsection{Frames Revisited}\index{frames and scoping statements}

Now that we have covered scoping, we will look at how an arbitrary
Metamath database can be converted to the simple sequence of extended
frames described on p.~\pageref{framelist}.  This is also how Metamath
stores the database internally when it reads in the database
source.\label{frameconvert} The method is simple.  First, we collect all
constant and variable (\texttt{\$c} and \texttt{\$v}) declarations in
the database, ignoring duplicate declarations of the same variable in
different scopes.  We then put our collected \texttt{\$c} and
\texttt{\$v} declarations at the beginning of the database, so that
their scope is the entire database.  Next, for each assertion in the
database, we determine its frame and extended frame.  The extended frame
is simply the \texttt{\$f}, \texttt{\$e}, and \texttt{\$d} statements
that are active.  The frame is the extended frame with all optional
hypotheses removed.

An equivalent way of saying this is that the extended frame of an assertion
is the collection of all \texttt{\$f}, \texttt{\$e}, and \texttt{\$d} statements
whose scope includes the assertion.
The \texttt{\$f} and \texttt{\$e} statements
occur in the order they appear
(order is irrelevant for \texttt{\$d} statements).

%c%, renaming any
%c%redeclared variables as needed so that all of them have unique names.  (The
%c%exact renaming convention is unimportant.  You might imagine renaming
%c%different declarations of math symbol \texttt{a} as \texttt{a\$1}, \texttt{a\$2}, etc.\
%c%which would prevent any conflicts since \texttt{\$} is not a legal character in a
%c%math symbol token.)

\section{The Anatomy of a Proof} \label{proof}
\index{proof!Metamath, description of}

Each provable assertion (\texttt{\$p}\index{\texttt{\$p} statement} statement) in a
database must include a {\bf proof}\index{proof}.  The proof is located
between the \texttt{\$=}\index{\texttt{\$=} keyword} and \texttt{\$.}\ keywords in the
\texttt{\$p} statement.

In the basic Metamath language\index{basic language}, a proof is a
sequence of statement labels.  This label sequence\index{label sequence}
serves as a set of instructions that the Metamath program uses to
construct a series of math symbol sequences.  The construction must
ultimately result in the math symbol sequence contained between the
\texttt{\$p}\index{\texttt{\$p} statement} and
\texttt{\$=}\index{\texttt{\$=} keyword} keywords of the \texttt{\$p}
statement.  Otherwise, the Metamath program will consider the proof
incorrect, and it will notify you with an appropriate error message when
you ask it to verify the proof.\footnote{To make the loading faster, the
Metamath program does not automatically verify proofs when you
\texttt{read} in a database unless you use the \texttt{/verify}
qualifier.  After a database has been read in, you may use the
\texttt{verify proof *} command to verify proofs.}\index{\texttt{verify
proof} command} Each label in a proof is said to {\bf
reference}\index{label reference} its corresponding statement.

Associated with any assertion\index{assertion} (\texttt{\$p} or
\texttt{\$a}\index{\texttt{\$a} statement} statement) is a set of
hypotheses (\texttt{\$f}\index{\texttt{\$f} statement} or
\texttt{\$e}\index{\texttt{\$e} statement} statements) that are active
with respect to that assertion.  Some are mandatory and the others are
optional.  You should review these concepts if necessary.

Each label\index{label} in a proof must be either the label of a
previous assertion (\texttt{\$a}\index{\texttt{\$a} statement} or
\texttt{\$p}\index{\texttt{\$p} statement} statement) or the label of an
active hypothesis (\texttt{\$e} or \texttt{\$f}\index{\texttt{\$f}
statement} statement) of the \texttt{\$p} statement containing the
proof.  Hypothesis labels may reference both the
mandatory\index{mandatory hypothesis} and the optional hypotheses of the
\texttt{\$p} statement.

The label sequence in a proof specifies a construction in {\bf reverse Polish
notation}\index{reverse Polish notation (RPN)} (RPN).  You may be familiar
with RPN if you have used older
Hewlett--Packard or similar hand-held calculators.
In the calculator analogy, a hypothesis label\index{hypothesis label} is like
a number and an assertion label\index{assertion label} is like an operation
(more precisely, an $n$-ary operation when the
assertion has $n$ \texttt{\$e}-hypotheses).
On an RPN calculator, an operation takes one or more previous numbers in an
input sequence, performs a calculation on them, and replaces those numbers and
itself with the result of the calculation.  For example, the input sequence
$2,3,+$ on an RPN calculator results in $5$, and the input sequence
$2,3,5,{\times},+$ results in $2,15,+$ which results in $17$.

Understanding how RPN is processed involves the concept of a {\bf
stack}\index{stack}\index{RPN stack}, which can be thought of as a set of
temporary memory locations that hold intermediate results.  When Metamath
encounters a hypothesis label it places or {\bf pushes}\index{push} the math
symbol sequence of the hypothesis onto the stack.  When Metamath encounters an
assertion label, it associates the most recent stack entries with the {\em
mandatory} hypotheses\index{mandatory hypothesis} of the assertion, in the
order where the most recent stack entry is associated with the last mandatory
hypothesis of the assertion.  It then determines what
substitutions\index{substitution!variable}\index{variable substitution} have
to be made into the variables of the assertion's mandatory hypotheses to make
them identical to the associated stack entries.  It then makes those same
substitutions into the assertion itself.  Finally, Metamath removes or {\bf
pops}\index{pop} the matched hypotheses from the stack and pushes the
substituted assertion onto the stack.

For the purpose of matching the mandatory hypothesis to the most recent stack
entries, whether a hypothesis is a \texttt{\$e} or \texttt{\$f} statement is
irrelevant.  The only important thing is that a set of
substitutions\footnote{In the Metamath spec (Section~\ref{spec}), we use the
singular term ``substitution'' to refer to the set of substitutions we talk
about here.} exist that allow a match (and if they don't, the proof verifier
will let you know with an error message).  The Metamath language is specified
in such a way that if a set of substitutions exists, it will be unique.
Specifically, the requirement that each variable have a type specified for it
with a \texttt{\$f} statement ensures the uniqueness.

We will illustrate this with an example.
Consider the following Metamath source file:
\begin{verbatim}
$c ( ) -> wff $.
$v p q r s $.
wp $f wff p $.
wq $f wff q $.
wr $f wff r $.
ws $f wff s $.
w2 $a wff ( p -> q ) $.
wnew $p wff ( s -> ( r -> p ) ) $= ws wr wp w2 w2 $.
\end{verbatim}
This Metamath source example shows the definition and ``proof'' (i.e.,
construction) of a well-formed formula (wff)\index{well-formed formula (wff)}
in propositional calculus.  (You may wish to type this example into a file to
experiment with the Metamath program.)  The first two statements declare
(introduce the names of) four constants and four variables.  The next four
statements specify the variable types, namely that
each variable is assumed to be a wff.  Statement \texttt{w2} defines (postulates)
a way to produce a new wff, \texttt{( p -> q )}, from two given wffs \texttt{p} and
\texttt{q}. The mandatory hypotheses of \texttt{w2} are \texttt{wp} and \texttt{wq}.
Statement \texttt{wnew} claims that \texttt{( s -> ( r -> p ) )} is a wff given
three wffs \texttt{s}, \texttt{r}, and \texttt{p}.  More precisely, \texttt{wnew} claims
that the sequence of ten symbols \texttt{wff ( s -> ( r -> p ) )} is provable from
previous assertions and the hypotheses of \texttt{wnew}.  Metamath does not know
or care what a wff is, and as far as it is concerned
the typecode \texttt{wff} is just an
arbitrary constant symbol in a math symbol sequence.  The mandatory hypotheses
of \texttt{wnew} are \texttt{wp}, \texttt{wr}, and \texttt{ws}; \texttt{wq} is an optional
hypothesis.  In our particular proof, the optional hypothesis is not
referenced, but in general, any combination of active (i.e.\ optional and
mandatory) hypotheses could be referenced.  The proof of statement \texttt{wnew}
is the sequence of five labels starting with \texttt{ws} (step~1) and ending with
\texttt{w2} (step~5).

When Metamath verifies the proof, it scans the proof from left to right.  We
will examine what happens at each step of the proof.  The stack starts off
empty.  At step 1, Metamath looks up label \texttt{ws} and determines that it is a
hypothesis, so it pushes the symbol sequence of statement \texttt{ws} onto the
stack:

\begin{center}\begin{tabular}{|l|l|}\hline
{Stack location} & {Contents} \\ \hline \hline
1 & \texttt{wff s} \\ \hline
\end{tabular}\end{center}

Metamath sees that the labels \texttt{wr} and \texttt{wp} in steps~2 and 3 are also
hypotheses, so it pushes them onto the stack.  After step~3, the stack looks
like
this:

\begin{center}\begin{tabular}{|l|l|}\hline
{Stack location} & {Contents} \\ \hline \hline
3 & \texttt{wff p} \\ \hline
2 & \texttt{wff r} \\ \hline
1 & \texttt{wff s} \\ \hline
\end{tabular}\end{center}

At step 4, Metamath sees that label \texttt{w2} is an assertion, so it must do
some processing.  First, it associates the mandatory hypotheses of \texttt{w2},
which are \texttt{wp} and \texttt{wq}, with stack locations~2 and 3, {\em in that
order}. Metamath determines that the only possible way
to make hypothesis \texttt{wp} match (become identical to) stack location~2 and
\texttt{wq} match stack location 3 is to substitute variable \texttt{p} with \texttt{r}
and \texttt{q} with \texttt{p}.  Metamath makes these substitutions into \texttt{w2} and
obtains the symbol sequence \texttt{wff ( r -> p )}.  It removes the hypotheses
from stack locations~2 and 3, then places the result into stack location~2:

\begin{center}\begin{tabular}{|l|l|}\hline
{Stack location} & {Contents} \\ \hline \hline
2 & \texttt{wff ( r -> p )} \\ \hline
1 & \texttt{wff s} \\ \hline
\end{tabular}\end{center}

At step 5, Metamath sees that label \texttt{w2} is an assertion, so it must again
do some processing.  First, it matches the mandatory hypotheses of \texttt{w2},
which are \texttt{wp} and \texttt{wq}, to stack locations 1 and 2.
Metamath determines that the only possible way to make the
hypotheses match is to substitute variable \texttt{p} with \texttt{s} and \texttt{q} with
\texttt{( r -> p )}.  Metamath makes these substitutions into \texttt{w2} and obtains
the symbol
sequence \texttt{wff ( s -> ( r -> p ) )}.  It removes stack
locations 1 and 2, then places the result into stack location~1:

\begin{center}\begin{tabular}{|l|l|}\hline
{Stack location} & {Contents} \\ \hline \hline
1 & \texttt{wff ( s -> ( r -> p ) )} \\ \hline
\end{tabular}\end{center}

After Metamath finishes processing the proof, it checks to see that the
stack contains exactly one element and that this element is
the same as the math symbol sequence in the
\texttt{\$p}\index{\texttt{\$p} statement} statement.  This is the case for our
proof of \texttt{wnew},
so we have proved \texttt{wnew} successfully.  If the result
differs, Metamath will notify you with an error message.  An error message
will also result if the stack contains more than one entry at the end of the
proof, or if the stack did not contain enough entries at any point in the
proof to match all of the mandatory hypotheses\index{mandatory hypothesis} of
an assertion.  Finally, Metamath will notify you with an error message if no
substitution is possible that will make a referenced assertion's hypothesis
match the
stack entries.  You may want to experiment with the different kinds of errors
that Metamath will detect by making some small changes in the proof of our
example.

Metamath's proof notation was designed primarily to express proofs in a
relatively compact manner, not for readability by humans.  Metamath can display
proofs in a number of different ways with the \texttt{show proof}\index{\texttt{show
proof} command} command.  The
\texttt{/lemmon} qualifier displays it in a format that is easier to read when the
proofs are short, and you saw examples of its use in Chapter~\ref{using}.  For
longer proofs, it is useful to see the tree structure of the proof.  A tree
structure is displayed when the \texttt{/lemmon} qualifier is omitted.  You will
probably find this display more convenient as you get used to it. The tree
display of the proof in our example looks like
this:\label{treeproof}\index{tree-style proof}\index{proof!tree-style}
\begin{verbatim}
1     wp=ws    $f wff s
2        wp=wr    $f wff r
3        wq=wp    $f wff p
4     wq=w2    $a wff ( r -> p )
5  wnew=w2  $a wff ( s -> ( r -> p ) )
\end{verbatim}
The number to the left of each line is the step number.  Following it is a
{\bf hypothesis association}\index{hypothesis association}, consisting of two
labels\index{label} separated by \texttt{=}.  To the left of the \texttt{=} (except
in the last step) is the label of a hypothesis of an assertion referenced
later in the proof; here, steps 1 and 4 are the hypothesis associations for
the assertion \texttt{w2} that is referenced in step 5.  A hypothesis association
is indented one level more than the assertion that uses it, so it is easy to
find the corresponding assertion by moving directly down until the indentation
level decreases to one less than where you started from.  To the right of each
\texttt{=} is the proof step label for that proof step.  The statement keyword of
the proof step label is listed next, followed by the content of the top of the
stack (the most recent stack entry) as it exists after that proof step is
processed.  With a little practice, you should have no trouble reading proofs
displayed in this format.

Metamath proofs include the syntax construction of a formula.
In standard mathematics, this kind of
construction is not considered a proper part of the proof at all, and it
certainly becomes rather boring after a while.
Therefore,
by default the \texttt{show proof}\index{\texttt{show proof}
command} command does not show the syntax construction.
Historically \texttt{show proof} command
\textit{did} show the syntax construction, and you needed to add the
\texttt{/essential} option to hide, them, but today
\texttt{/essential} is the default and you need to use
\texttt{/all} to see the syntax constructions.

When verifying a proof, Metamath will check that no mandatory
\texttt{\$d}\index{\texttt{\$d} statement}\index{mandatory \texttt{\$d}
statement} statement of an assertion referenced in a proof is violated
when substitutions\index{substitution!variable}\index{variable
substitution} are made to the variables in the assertion.  For details
see Section~\ref{spec4} or \ref{dollard}.

\subsection{The Concept of Unification} \label{unify}

During the course of verifying a proof, when Metamath\index{Metamath}
encounters an assertion label\index{assertion label}, it associates the
mandatory hypotheses\index{mandatory hypothesis} of the assertion with the top
entries of the RPN stack\index{stack}\index{RPN stack}.  Metamath then
determines what substitutions\index{substitution!variable}\index{variable
substitution} it must make to the variables in the assertion's mandatory
hypotheses in order for these hypotheses to become identical to their
corresponding stack entries.  This process is called {\bf
unification}\index{unification}.  (We also informally use the term
``unification'' to refer to a set of substitutions that results from the
process, as in ``two unifications are possible.'')  After the substitutions
are made, the hypotheses are said to be {\bf unified}.

If no such substitutions are possible, Metamath will consider the proof
incorrect and notify you with an error message.
% (deleted 3/10/07, per suggestion of Mel O'Cat:)
% The syntax of the
% Metamath language ensures that if a set of substitutions exists, it
% will be unique.

The general algorithm for unification described in the literature is
somewhat complex.
However, in the case of Metamath it is intentionally trivial.
Mandatory hypotheses must be
pushed on the proof stack in the order in which they appear.
In addition, each variable must have its type specified
with a \texttt{\$f} hypothesis before it is used
and that each \texttt{\$f} hypothesis
have the restricted syntax of a typecode (a constant) followed by a variable.
The typecode in the \texttt{\$f} hypothesis must match the first symbol of
the corresponding RPN stack entry (which will also be a constant), so
the only possible match for the variable in the \texttt{\$f} hypothesis is
the sequence of symbols in the stack entry after the initial constant.

In the Proof Assistant\index{Proof Assistant}, a more general unification
algorithm is used.  While a proof is being developed, sometimes not enough
information is available to determine a unique unification.  In this case
Metamath will ask you to pick the correct one.\index{ambiguous
unification}\index{unification!ambiguous}

\section{Extensions to the Metamath Language}\index{extended
language}

\subsection{Comments in the Metamath Language}\label{comments}
\index{markup notation}
\index{comments!markup notation}

The commenting feature allows you to annotate the contents of
a database.  Just as with most
computer languages, comments are ignored for the purpose of interpreting the
contents of the database. Comments effectively act as
additional white space\index{white
space} between tokens
when a database is parsed.

A comment may be placed at the beginning, end, or
between any two tokens\index{token} in a source file.

Comments have the following syntax:
\begin{center}
 \texttt{\$(} {\em text} \texttt{\$)}
\end{center}
Here,\index{\texttt{\$(} and \texttt{\$)} auxiliary
keywords}\index{comment} {\em text} is a string, possibly empty, of any
characters in Metamath's character set (p.~\pageref{spec1chars}), except
that the character strings \texttt{\$(} and \texttt{\$)} may not appear
in {\em text}.  Thus nested comments are not
permitted:\footnote{Computer languages have differing standards for
nested comments, and rather than picking one it was felt simplest not to
allow them at all, at least in the current version (0.177) of
Metamath\index{Metamath!limitations of version 0.177}.} Metamath will
complain if you give it
\begin{center}
 \texttt{\$( This is a \$( nested \$) comment.\ \$)}
\end{center}
To compensate for this non-nesting behavior, I often change all \texttt{\$}'s
to \texttt{@}'s in sections of Metamath code I wish to comment out.

The Metamath program supports a number of markup mechanisms and conventions
to generate good-looking results in \LaTeX\ and {\sc html},
as discussed below.
These markup features have to do only with how the comments are typeset,
and have no effect on how Metamath verifies the proofs in the database.
The improper
use of them may result in incorrectly typeset output, but no Metamath
error messages will result during the \texttt{read} and \texttt{verify
proof} commands.  (However, the \texttt{write
theorem\texttt{\char`\_}list} command
will check for markup errors as a side-effect of its
{\sc html} generation.)
Section~\ref{texout} has instructions for creating \LaTeX\ output, and
section~\ref{htmlout} has instructions for creating
{\sc html}\index{HTML} output.

\subsubsection{Headings}\label{commentheadings}

If the \texttt{\$(} is immediately followed by a new line
starting with a heading marker, it is a header.
This can start with:

\begin{itemize}
 \item[] \texttt{\#\#\#\#} - major part header
 \item[] \texttt{\#*\#*} - section header
 \item[] \texttt{=-=-} - subsection header
 \item[] \texttt{-.-.} - subsubsection header
\end{itemize}

The line following the marker line
will be used for the table of contents entry, after trimming spaces.
The next line should be another (closing) matching marker line.
Any text after that
but before the closing \texttt{\$}, such as an extended description of the
section, will be included on the \texttt{mmtheoremsNNN.html} page.

For more information, run
\texttt{help write theorem\char`\_list}.

\subsubsection{Math mode}
\label{mathcomments}
\index{\texttt{`} inside comments}
\index{\texttt{\char`\~} inside comments}
\index{math mode}

Inside of comments, a string of tokens\index{token} enclosed in
grave accents\index{grave accent (\texttt{`})} (\texttt{`}) will be converted
to standard mathematical symbols during
{\sc HTML}\index{HTML} or \LaTeX\ output
typesetting,\index{latex@{\LaTeX}} according to the information in the
special \texttt{\$t}\index{\texttt{\$t} comment}\index{typesetting
comment} comment in the database
(see section~\ref{tcomment} for information about the typesetting
comment, and Appendix~\ref{ASCII} to see examples of its results).

The first grave accent\index{grave accent (\texttt{`})} \texttt{`}
causes the output processor to enter {\bf math mode}\index{math mode}
and the second one exits it.
In this
mode, the characters following the \texttt{`} are interpreted as a
sequence of math symbol tokens separated by white space\index{white
space}.  The tokens are looked up in the \texttt{\$t}
comment\index{\texttt{\$t} comment}\index{typesetting comment} and if
found, they will be replaced by the standard mathematical symbols that
they correspond to before being placed in the typeset output file.  If
not found, the symbol will be output as is and a warning will be issued.
The tokens do not have to be active in the database, although a warning
will be issued if they are not declared with \texttt{\$c} or
\texttt{\$v} statements.

Two consecutive
grave accents \texttt{``} are treated as a single actual grave accent
(both inside and outside of math mode) and will not cause the output
processor to enter or exit math mode.

Here is an example of its use\index{Pierce's axiom}:
\begin{center}
\texttt{\$( Pierce's axiom, ` ( ( ph -> ps ) -> ph ) -> ph ` ,\\
         is not very intuitive. \$)}
\end{center}
becomes
\begin{center}
   \texttt{\$(} Pierce's axiom, $((\varphi \rightarrow \psi)\rightarrow
\varphi)\rightarrow \varphi$, is not very intuitive. \texttt{\$)}
\end{center}

Note that the math symbol tokens\index{token} must be surrounded by white
space\index{white space}.
%, since there is no context that allows ambiguity to be
%resolved, as is the case with math symbol sequences in some of the Metamath
%statements.
White space should also surround the \texttt{`}
delimiters.

The math mode feature also gives you a quick and easy way to generate
text containing mathematical symbols, independently of the intended
purpose of Metamath.\index{Metamath!using as a math editor} To do this,
simply create your text with grave accents surrounding your formulas,
after making sure that your math symbols are mapped to \LaTeX\ symbols
as described in Appendix~\ref{ASCII}.  It is easier if you start with a
database with predefined symbols such as \texttt{set.mm}.  Use your
grave-quoted math string to replace an existing comment, then typeset
the statement corresponding to that comment following the instructions
from the \texttt{help tex} command in the Metamath program.  You will
then probably want to edit the resulting file with a text editor to fine
tune it to your exact needs.

\subsubsection{Label Mode}\index{label mode}

Outside of math mode, a tilde\index{tilde (\texttt{\char`\~})} \verb/~/
indicates to Metamath's\index{Metamath} output processor that the
token\index{token} that follows (i.e.\ the characters up to the next
white space\index{white space}) represents a statement label or URL.
This formatting mode is called {\bf label mode}\index{label mode}.
If a literal tilde
is desired (outside of math mode) instead of label mode,
use two tildes in a row to represent it.

When generating a \LaTeX\ output file,
the following token will be formatted in \texttt{typewriter}
font, and the tilde removed, to make it stand out from the rest of the text.
This formatting will be applied to all characters after the
tilde up to the first white space\index{white space}.
Whether
or not the token is an actual statement label is not checked, and the
token does not have to have the correct syntax for a label; no error
messages will be produced.  The only effect of the label mode on the
output is that typewriter font will be used for the tokens that are
placed in the \LaTeX\ output file.

When generating {\sc html},
the tokens after the tilde {\em must} be a URL (either http: or https:)
or a valid label.
Error messages will be issued during that output if they aren't.
A hyperlink will be generated to that URL or label.

\subsubsection{Link to bibliographical reference}\index{citation}%
\index{link to bibliographical reference}

Bibliographical references are handled specially when generating
{\sc html} if formatted specially.
Text in the form \texttt{[}{\em author}\texttt{]}
is considered a link to a bibliographical reference.
See \texttt{help html} and \texttt{help write
bibliography} in the Metamath program for more
information.
% \index{\texttt{\char`\[}\ldots\texttt{]} inside comments}
See also Sections~\ref{tcomment} and \ref{wrbib}.

The \texttt{[}{\em author}\texttt{]} notation will also create an entry in
the bibliography cross-reference file generated by \texttt{write
bibliography} (Section~\ref{wrbib}) for {\sc HTML}.
For this to work properly, the
surrounding comment must be formatted as follows:
\begin{quote}
    {\em keyword} {\em label} {\em noise-word}
     \texttt{[}{\em author}\texttt{] p.} {\em number}
\end{quote}
for example
\begin{verbatim}
     Theorem 5.2 of [Monk] p. 223
\end{verbatim}
The {\em keyword} is not case sensitive and must be one of the following:
\begin{verbatim}
     theorem lemma definition compare proposition corollary
     axiom rule remark exercise problem notation example
     property figure postulate equation scheme chapter
\end{verbatim}
The optional {\em label} may consist of more than one
(non-{\em keyword} and non-{\em noise-word}) word.
The optional {\em noise-word} is one of:
\begin{verbatim}
     of in from on
\end{verbatim}
and is  ignored when the cross-reference file is created.  The
\texttt{write
biblio\-graphy} command will perform error checking to verify the
above format.\index{error checking}

\subsubsection{Parentheticals}\label{parentheticals}

The end of a comment may include one or more parenthicals, that is,
statements enclosed in parentheses.
The Metamath program looks for certain parentheticals and can issue
warnings based on them.
They are:

\begin{itemize}
 \item[] \texttt{(Contributed by }
   \textit{NAME}\texttt{,} \textit{DATE}\texttt{.)} -
   document the original contributor's name and the date it was created.
 \item[] \texttt{(Revised by }
   \textit{NAME}\texttt{,} \textit{DATE}\texttt{.)} -
   document the contributor's name and creation date
   that resulted in significant revision
   (not just an automated minimization or shortening).
 \item[] \texttt{(Proof shortened by }
   \textit{NAME}\texttt{,} \textit{DATE}\texttt{.)} -
   document the contributor's name and date that developed a significant
   shortening of the proof (not just an automated minimization).
 \item[] \texttt{(Proof modification is discouraged.)} -
   Note that this proof should normally not be modified.
 \item[] \texttt{(New usage is discouraged.)} -
   Note that this assertion should normally not be used.
\end{itemize}

The \textit{DATE} must be in form YYYY-MMM-DD, where MMM is the
English abbreviation of that month.

\subsubsection{Other markup}\label{othermarkup}
\index{markup notation}

There are other markup notations for generating good-looking results
beyond math mode and label mode:

\begin{itemize}
 \item[]
         \texttt{\char`\_} (underscore)\index{\texttt{\char`\_} inside comments} -
             Italicize text starting from
              {\em space}\texttt{\char`\_}{\em non-space} (i.e.\ \texttt{\char`\_}
              with a space before it and a non-space character after it) until
             the next
             {\em non-space}\texttt{\char`\_}{\em space}.  Normal
             punctuation (e.g.\ a trailing
             comma or period) is ignored when determining {\em space}.
 \item[]
         \texttt{\char`\_} (underscore) - {\em
         non-space}\texttt{\char`\_}{\em non-space-string}, where
          {\em non-space-string} is a string of non-space characters,
         will make {\em non-space-string} become a subscript.
 \item[]
         \texttt{<HTML>}...\texttt{</HTML>} - do not convert
         ``\texttt{<}'' and ``\texttt{>}''
         in the enclosed text when generating {\sc HTML},
         otherwise process markup normally. This allows direct insertion
         of {\sc html} commands.
 \item[]
       ``\texttt{\&}ref\texttt{;}'' - insert an {\sc HTML}
         character reference.
         This is how to insert arbitrary Unicode characters
         (such as accented characters).  Currently only directly supported
         when generating {\sc HTML}.
\end{itemize}

It is recommended that spaces surround any \texttt{\char`\~} and
\texttt{`} tokens in the comment and that a space follow the {\em label}
after a \texttt{\char`\~} token.  This will make global substitutions
to change labels and symbol names much easier and also eliminate any
future chance of ambiguity.  Spaces around these tokens are automatically
removed in the final output to conform with normal rules of punctuation;
for example, a space between a trailing \texttt{`} and a left parenthesis
will be removed.

A good way to become familiar with the markup notation is to look at
the extensive examples in the \texttt{set.mm} database.

\subsection{The Typesetting Comment (\texttt{\$t})}\label{tcomment}

The typesetting comment \texttt{\$t} in the input database file
provides the information necessary to produce good-looking results.
It provides \LaTeX\ and {\sc html}
definitions for math symbols,
as well supporting as some
customization of the generated web page.
If you add a new token to a database, you should also
update the \texttt{\$t} comment information if you want to eventually
create output in \LaTeX\ or {\sc HTML}.
See the
\texttt{set.mm}\index{set theory database (\texttt{set.mm})} database
file for an extensive example of a \texttt{\$t} comment illustrating
many of the features described below.

Programs that do not need to generate good-looking presentation results,
such as programs that only verify Metamath databases,
can completely ignore typesetting comments
and just treat them as normal comments.
Even the Metamath program only consults the
\texttt{\$t} comment information when it needs to generate typeset output
in \LaTeX\ or {\sc HTML}
(e.g., when you open a \LaTeX\ output file with the \texttt{open tex} command).

We will first discuss the syntax of typesetting comments, and then
briefly discuss how this can be used within the Metamath program.

\subsubsection{Typesetting Comment Syntax Overview}

The typesetting comment is identified by the token
\texttt{\$t}\index{\texttt{\$t} comment}\index{typesetting comment} in
the comment, and the typesetting comment ends at the matching
\texttt{\$)}:
\[
  \mbox{\tt \$(\ }
  \mbox{\tt \$t\ }
  \underbrace{
    \mbox{\tt \ \ \ \ \ \ \ \ \ \ \ }
    \cdots
    \mbox{\tt \ \ \ \ \ \ \ \ \ \ \ }
  }_{\mbox{Typesetting definitions go here}}
  \mbox{\tt \ \$)}
\]

There must be one or more white space characters, and only white space
characters, between the \texttt{\$(} that starts the comment
and the \texttt{\$t} symbol,
and the \texttt{\$t} must be followed by one
or more white space characters
(see section \ref{whitespace} for the definition of white space characters).
The typesetting comment continues until the comment end token \texttt{\$)}
(which must be preceded by one or more white space characters).

In version 0.177\index{Metamath!limitations of version 0.177} of the
Metamath program, there may be only one \texttt{\$t} comment in a
database.  This restriction may be lifted in the future to allow
many \texttt{\$t} comments in a database.

Between the \texttt{\$t} symbol (and its following white space) and the
comment end token \texttt{\$)} (and its preceding white space)
is a sequence of one or more typesetting definitions, where
each definition has the form
\textit{definition-type arg arg ... ;}.
Each of the zero or more \textit{arg} values
can be either a typesetting data or a keyword
(what keywords are allowed, and where, depends on the specific
\textit{definition-type}).
The \textit{definition-type}, and each argument \textit{arg},
are separated by one or more white space characters.
Every definition ends in an unquoted semicolon;
white space is not required before the terminating semicolon of a definition.
Each definition should start on a new line.\footnote{This
restriction of the current version of Metamath
(0.177)\index{Metamath!limitations of version 0.177} may be removed
in a future version, but you should do it anyway for readability.}

For example, this typesetting definition:
\begin{center}
 \verb$latexdef "C_" as "\subseteq";$
\end{center}
defines the token \verb$C_$ as the \LaTeX\ symbol $\subseteq$ (which means
``subset'').

Typesetting data is a sequence of one or more quoted strings
(if there is more than one, they are connected by \texttt{\char`\+}).
Often a single quoted string is used to provide data for a definition, using
either double (\texttt{\char`\"}) or single (\texttt{'}) quotation marks.
However,
{\em a quoted string (enclosed in quotation marks) may not include
line breaks.}
A quoted string
may include a quotation mark that matches the enclosing quotes by repeating
the quotation mark twice.  Here are some examples:

\begin{tabu}   { l l }
\textbf{Example} & \textbf{Meaning} \\
\texttt{\char`\"a\char`\"\char`\"b\char`\"} & \texttt{a\char`\"b} \\
\texttt{'c''d'} & \texttt{c'd} \\
\texttt{\char`\"e''f\char`\"} & \texttt{e''f} \\
\texttt{'g\char`\"\char`\"h'} & \texttt{g\char`\"\char`\"h} \\
\end{tabu}

Finally, a long quoted string
may be broken up into multiple quoted strings (considered, as a whole,
a single quoted string) and joined with \texttt{\char`\+}.
You can even use multiple lines as long as a
'+' is at the end of every line except the last one.
The \texttt{\char`\+} should be preceded and followed by at least one
white space character.
Thus, for example,
\begin{center}
 \texttt{\char`\"ab\char`\"\ \char`\+\ \char`\"cd\char`\"
    \ \char`\+\ \\ 'ef'}
\end{center}
is the same as
\begin{center}
 \texttt{\char`\"abcdef\char`\"}
\end{center}

{\sc c}-style comments \texttt{/*}\ldots\texttt{*/} are also supported.

In practice, whenever you add a new math token you will often want to add
typesetting definitions using
\texttt{latexdef}, \texttt{htmldef}, and
\texttt{althtmldef}, as described below.
That way, they will all be up to date.
Of course, whether or not you want to use all three definitions will
depend on how the database is intended to be used.

Below we discuss the different possible \textit{definition-kind} options.
We will show data surrounded by double quotes (in practice they can also use
single quotes and/or be a sequence joined by \texttt{+}s).
We will use specific names for the \textit{data} to make clear what
the data is used for, such as
{\em math-token} (for a Metamath math token,
{\em latex-string} (for string to be placed in a \LaTeX\ stream),
{\em {\sc html}-code} (for {\sc html} code),
and {\em filename} (for a filename).

\subsubsection{Typesetting Comment - \LaTeX}

The syntax for a \LaTeX\ definition is:
\begin{center}
 \texttt{latexdef "}{\em math-token}\texttt{" as "}{\em latex-string}\texttt{";}
\end{center}
\index{latex definitions@\LaTeX\ definitions}%
\index{\texttt{latexdef} statement}

The {\em token-string} and {\em latex-string} are the data
(character strings) for
the token and the \LaTeX\ definition of the token, respectively,

These \LaTeX\ definitions are used by the Metamath program
when it is asked to product \LaTeX output using
the \texttt{write tex} command.

\subsubsection{Typesetting Comment - {\sc html}}

The key kinds of {\sc HTML} definitions have the following syntax:

\vskip 1ex
    \texttt{htmldef "}{\em math-token}\texttt{" as "}{\em
    {\sc html}-code}\texttt{";}\index{\texttt{htmldef} statement}
                    \ \ \ \ \ \ldots

    \texttt{althtmldef "}{\em math-token}\texttt{" as "}{\em
{\sc html}-code}\texttt{";}\index{\texttt{althtmldef} statement}

                    \ \ \ \ \ \ldots

Note that in {\sc HTML} there are two possible definitions for math tokens.
This feature is useful when
an alternate representation of symbols is desired, for example one that
uses Unicode entities and another uses {\sc gif} images.

There are many other typesetting definitions that can control {\sc HTML}.
These include:

\vskip 1ex

    \texttt{htmldef "}{\em math-token}\texttt{" as "}{\em {\sc
    html}-code}\texttt{";}

    \texttt{htmltitle "}{\em {\sc html}-code}\texttt{";}%
\index{\texttt{htmltitle} statement}

    \texttt{htmlhome "}{\em {\sc html}-code}\texttt{";}%
\index{\texttt{htmlhome} statement}

    \texttt{htmlvarcolor "}{\em {\sc html}-code}\texttt{";}%
\index{\texttt{htmlvarcolor} statement}

    \texttt{htmlbibliography "}{\em filename}\texttt{";}%
\index{\texttt{htmlbibliography} statement}

\vskip 1ex

\noindent The \texttt{htmltitle} is the {\sc html} code for a common
title, such as ``Metamath Proof Explorer.''  The \texttt{htmlhome} is
code for a link back to the home page.  The \texttt{htmlvarcolor} is
code for a color key that appears at the bottom of each proof.  The file
specified by {\em filename} is an {\sc html} file that is assumed to
have a \texttt{<A NAME=}\ldots\texttt{>} tag for each bibiographic
reference in the database comments.  For example, if
\texttt{[Monk]}\index{\texttt{\char`\[}\ldots\texttt{]} inside comments}
occurs in the comment for a theorem, then \texttt{<A NAME='Monk'>} must
be present in the file; if not, a warning message is given.

Associated with
\texttt{althtmldef}
are the statements
\vskip 1ex

    \texttt{htmldir "}{\em
      directoryname}\texttt{";}\index{\texttt{htmldir} statement}

    \texttt{althtmldir "}{\em
     directoryname}\texttt{";}\index{\texttt{althtmldir} statement}

\vskip 1ex
\noindent giving the directories of the {\sc gif} and Unicode versions
respectively; their purpose is to provide cross-linking between the
two versions in the generated web pages.

When two different types of pages need to be produced from a single
database, such as the Hilbert Space Explorer that extends the Metamath
Proof Explorer, ``extended'' variables may be declared in the
\texttt{\$t} comment:
\vskip 1ex

    \texttt{exthtmltitle "}{\em {\sc html}-code}\texttt{";}%
\index{\texttt{exthtmltitle} statement}

    \texttt{exthtmlhome "}{\em {\sc html}-code}\texttt{";}%
\index{\texttt{exthtmlhome} statement}

    \texttt{exthtmlbibliography "}{\em filename}\texttt{";}%
\index{\texttt{exthtmlbibliography} statement}

\vskip 1ex
\noindent When these are declared, you also must declare
\vskip 1ex

    \texttt{exthtmllabel "}{\em label}\texttt{";}%
\index{\texttt{exthtmllabel} statement}

\vskip 1ex \noindent that identifies the database statement where the
``extended'' section of the database starts (in our example, where the
Hilbert Space Explorer starts).  During the generation of web pages for
that starting statement and the statements after it, the {\sc html} code
assigned to \texttt{exthtmltitle} and \texttt{exthtmlhome} is used
instead of that assigned to \texttt{htmltitle} and \texttt{htmlhome},
respectively.

\begin{sloppy}
\subsection{Additional Information Com\-ment (\texttt{\$j})} \label{jcomment}
\end{sloppy}

The additional information comment, aka the
\texttt{\$j}\index{\texttt{\$j} comment}\index{additional information comment}
comment,
provides a way to add additional structured information that can
be optionally parsed by systems.

The additional information comment is parsed the same way as the
typesetting comment (\texttt{\$t}) (see section \ref{tcomment}).
That is,
the additional information comment begins with the token
\texttt{\$j} within a comment,
and continues until the comment close \texttt{\$)}.
Within an additional information comment is a sequence of one or more
commands of the form \texttt{command arg arg ... ;}
where each of the zero or more \texttt{arg} values
can be either a quoted string or a keyword.
Note that every command ends in an unquoted semicolon.
If a verifier is parsing an additional information comment, but
doesn't recognize a particular command, it must skip the command
by finding the end of the command (an unquoted semicolon).

A database may have 0 or more additional information comments.
Note, however, that a verifier may ignore these comments entirely or only
process certain commands in an additional information comment.
The \texttt{mmj2} verifier supports many commands in additional information
comments.
We encourage systems that process additional information comments
to coordinate so that they will use the same command for the same effect.

Examples of additional information comments with various commands
(from the \texttt{set.mm} database) are:

\begin{itemize}
   \item Define the syntax and logical typecodes,
     and declare that our grammar is
     unambiguous (verifiable using the KLR parser, with compositing depth 5).
\begin{verbatim}
  $( $j
    syntax 'wff';
    syntax '|-' as 'wff';
    unambiguous 'klr 5';
  $)
\end{verbatim}

   \item Register $\lnot$ and $\rightarrow$ as primitive expressions
           (lacking definitions).
\begin{verbatim}
  $( $j primitive 'wn' 'wi'; $)
\end{verbatim}

   \item There is a special justification for \texttt{df-bi}.
\begin{verbatim}
  $( $j justification 'bijust' for 'df-bi'; $)
\end{verbatim}

   \item Register $\leftrightarrow$ as an equality for its type (wff).
\begin{verbatim}
  $( $j
    equality 'wb' from 'biid' 'bicomi' 'bitri';
    definition 'dfbi1' for 'wb';
  $)
\end{verbatim}

   \item Theorem \texttt{notbii} is the congruence law for negation.
\begin{verbatim}
  $( $j congruence 'notbii'; $)
\end{verbatim}

   \item Add \texttt{setvar} as a typecode.
\begin{verbatim}
  $( $j syntax 'setvar'; $)
\end{verbatim}

   \item Register $=$ as an equality for its type (\texttt{class}).
\begin{verbatim}
  $( $j equality 'wceq' from 'eqid' 'eqcomi' 'eqtri'; $)
\end{verbatim}

\end{itemize}


\subsection{Including Other Files in a Metamath Source File} \label{include}
\index{\texttt{\$[} and \texttt{\$]} auxiliary keywords}

The keywords \texttt{\$[} and \texttt{\$]} specify a file to be
included\index{included file}\index{file inclusion} at that point in a
Metamath\index{Metamath} source file\index{source file}.  The syntax for
including a file is as follows:
\begin{center}
\texttt{\$[} {\em file-name} \texttt{\$]}
\end{center}

The {\em file-name} should be a single token\index{token} with the same syntax
as a math symbol (i.e., all 93 non-whitespace
printable characters other than \texttt{\$} are
allowed, subject to the file-naming limitations of your operating system).
Comments may appear between the \texttt{\$[} and \texttt{\$]} keywords.  Included
files may include other files, which may in turn include other files, and so
on.

For example, suppose you want to use the set theory database as the starting
point for your own theory.  The first line in your file could be
\begin{center}
\texttt{\$[ set.mm \$]}
\end{center} All of the information (axioms, theorems,
etc.) in \texttt{set.mm} and any files that {\em it} includes will become
available for you to reference in your file. This can help make your work more
modular. A drawback to including files is that if you change the name of a
symbol or the label of a statement, you must also remember to update any
references in any file that includes it.


The naming conventions for included files are the same as those of your
operating system.\footnote{On the Macintosh, prior to Mac OS X,
 a colon is used to separate disk
and folder names from your file name.  For example, {\em volume}\texttt{:}{\em
file-name} refers to the root directory, {\em volume}\texttt{:}{\em
folder-name}\texttt{:}{\em file-name} refers to a folder in root, and {\em
volume}\texttt{:}{\em folder-name}\texttt{:}\ldots\texttt{:}{\em file-name} refers to a
deeper folder.  A simple {\em file-name} refers to a file in the folder from
which you launch the Metamath application.  Under Mac OS X and later,
the Metamath program is run under the Terminal application, which
conforms to Unix naming conventions.}\index{Macintosh file
names}\index{file names!Macintosh}\label{includef} For compatibility among
operating systems, you should keep the file names as simple as possible.  A
good convention to use is {\em file}\texttt{.mm} where {\em file} is eight
characters or less, in lower case.

There is no limit to the nesting depth of included files.  One thing that you
should be aware of is that if two included files themselves include a common
third file, only the {\em first} reference to this common file will be read
in.  This allows you to include two or more files that build on a common
starting file without having to worry about label and symbol conflicts that
would occur if the common file were read in more than once.  (In fact, if a
file includes itself, the self-reference will be ignored, although of course
it would not make any sense to do that.)  This feature also means, however,
that if you try to include a common file in several inner blocks, the result
might not be what you expect, since only the first reference will be replaced
with the included file (unlike the include statement in most other computer
languages).  Thus you would normally include common files only in the
outermost block\index{outermost block}.

\subsection{Compressed Proof Format}\label{compressed1}\index{compressed
proof}\index{proof!compressed}

The proof notation presented in Section~\ref{proof} is called a
{\bf normal proof}\index{normal proof}\index{proof!normal} and in principle is
sufficient to express any proof.  However, proofs often contain steps and
subproofs that are identical.  This is particularly true in typical
Metamath\index{Metamath} applications, because Metamath requires that the math
symbol sequence (usually containing a formula) at each step be separately
constructed, that is, built up piece by piece. As a result, a lot of
repetition often results.  The {\bf compressed proof} format allows Metamath
to take advantage of this redundancy to shorten proofs.

The specification for the compressed proof format is given in
Appen\-dix~\ref{compressed}.

Normally you need not concern yourself with the details of the compressed
proof format, since the Metamath program will allow you to convert from
the normal format to the compressed format with ease, and will also
automatically convert from the compressed format when proofs are displayed.
The overall structure of the compressed format is as follows:
\begin{center}
  \texttt{\$= ( } {\em label-list} \texttt{) } {\em compressed-proof\ }\ \texttt{\$.}
\end{center}
\index{\texttt{\$=} keyword}
The first \texttt{(} serves as a flag to Metamath that a compressed proof
follows.  The {\em label-list} includes all statements referred to by the
proof except the mandatory hypotheses\index{mandatory hypothesis}.  The {\em
compressed-proof} is a compact encoding of the proof, using upper-case
letters, and can be thought of as a large integer in base 26.  White
space\index{white space} inside a {\em compressed-proof} is
optional and is ignored.

It is important to note that the order of the mandatory hypotheses of
the statement being proved must not be changed if the compressed proof
format is used, otherwise the proof will become incorrect.  The reason
for this is that the mandatory hypotheses are not mentioned explicitly
in the compressed proof in order to make the compression more efficient.
If you wish to change the order of mandatory hypotheses, you must first
convert the proof back to normal format using the \texttt{save proof
{\em statement} /normal}\index{\texttt{save proof} command} command.
Later, you can go back to compressed format with \texttt{save proof {\em
statement} /compressed}.

During error checking with the \texttt{verify proof} command, an error
found in a compressed proof may point to a character in {\em
compressed-proof}, which may not be very meaningful to you.  In this
case, try to \texttt{save proof /normal} first, then do the
\texttt{verify proof} again.  In general, it is best to make sure a
proof is correct before saving it in compressed format, because severe
errors are less likely to be recoverable than in normal format.

\subsection{Specifying Unknown Proofs or Subproofs}\label{unknown}

In a proof under development, any step or subproof that is not yet known
may be represented with a single \texttt{?}.  For the purposes of
parsing the proof, the \texttt{?}\ \index{\texttt{]}@\texttt{?}\ inside
proofs} will push a single entry onto the RPN stack just as if it were a
hypothesis.  While developing a proof with the Proof
Assistant\index{Proof Assistant}, a partially developed proof may be
saved with the \texttt{save new{\char`\_}proof}\index{\texttt{save
new{\char`\_}proof} command} command, and \texttt{?}'s will be placed at
the appropriate places.

All \texttt{\$p}\index{\texttt{\$p} statement} statements must have
proofs, even if they are entirely unknown.  Before creating a proof with
the Proof Assistant, you should specify a completely unknown proof as
follows:
\begin{center}
  {\em label} \texttt{\$p} {\em statement} \texttt{\$= ?\ \$.}
\end{center}
\index{\texttt{\$=} keyword}
\index{\texttt{]}@\texttt{?}\ inside proofs}

The \texttt{verify proof}\index{\texttt{verify proof} command} command
will check the known portions of a partial proof for errors, but will
warn you that the statement has not been proved.

Note that partially developed proofs may be saved in compressed format
if desired.  In this case, you will see one or more \texttt{?}'s in the
{\em compressed-proof} part.\index{compressed
proof}\index{proof!compressed}

\section{Axioms vs.\ Definitions}\label{definitions}

The \textit{basic}
Metamath\index{Metamath} language and program
make no distinction\index{axiom vs.\
definition} between axioms\index{axiom} and
definitions.\index{definition} The \texttt{\$a}\index{\texttt{\$a}
statement} statement is used for both.  At first, this may seem
puzzling.  In the minds of many mathematicians, the distinction is
clear, even obvious, and hardly worth discussing.  A definition is
considered to be merely an abbreviation that can be replaced by the
expression for which it stands; although unless one actually does this,
to be precise then one should say that a theorem\index{theorem} is a
consequence of the axioms {\em and} the definitions that are used in the
formulation of the theorem \cite[p.~20]{Behnke}.\index{Behnke, H.}

\subsection{What is a Definition?}

What is a definition?  In its simplest form, a definition introduces a new
symbol and provides an unambiguous rule to transform an expression containing
the new symbol to one without it.  The concept of a ``proper
definition''\index{proper definition}\index{definition!proper} (as opposed to
a creative definition)\index{creative definition}\index{definition!creative}
that is usually agreed upon is (1) the definition should not strengthen the
language and (2) any symbols introduced by the definition should be eliminable
from the language \cite{Nemesszeghy}\index{Nemesszeghy, E. Z.}.  In other
words, they are mere typographical conveniences that do not belong to the
system and are theoretically superfluous.  This may seem obvious, but in fact
the nature of definitions can be subtle, sometimes requiring difficult
metatheorems to establish that they are not creative.

A more conservative stance was taken by logician S.
Le\'{s}niewski.\index{Le\'{s}niewski, S.}
\begin{quote}
Le\'{s}niewski
regards definitions as theses of the system.  In this respect they do
not differ either from the axioms or from theorems, i.e.\ from the
theses added to the system on the basis of the rule of substitution or
the rule of detachment [modus ponens].  Once definitions have been
accepted as theses of the system, it becomes necessary to consider them
as true propositions in the same sense in which axioms are true
\cite{Lejewski}.
\end{quote}\index{Lejewski, Czeslaw}

Let us look at some simple examples of definitions in propositional
calculus.  Consider the definition of logical {\sc or}
(disjunction):\index{disjunction ($\vee$)} ``$P\vee Q$ denotes $\neg P
\rightarrow Q$ (not $P$ implies $Q$).''  It is very easy to recognize a
statement making use of this definition, because it introduces the new
symbol $\vee$ that did not previously exist in the language.  It is easy
to see that no new theorems of the original language will result from
this definition.

Next, consider a definition that eliminates parentheses:  ``$P
\rightarrow Q\rightarrow R$ denotes $P\rightarrow (Q \rightarrow R)$.''
This is more subtle, because no new symbols are introduced.  The reason
this definition is considered proper is that no new symbol sequences
that are valid wffs (well-formed formulas)\index{well-formed formula
(wff)} in the original language will result from the definition, since
``$P \rightarrow Q\rightarrow R$'' is not a wff in the original
language.  Here, we implicitly make use of the fact that there is a
decision procedure that allows us to determine whether or not a symbol
sequence is a wff, and this fact allows us to use symbol sequences that
are not wffs to represent other things (such as wffs) by means of the
definition.  However, to justify the definition as not being creative we
need to prove that ``$P \rightarrow Q\rightarrow R$'' is in fact not a
wff in the original language, and this is more difficult than in the
case where we simply introduce a new symbol.

%Now let's take this reasoning to an extreme.  Propositional calculus is a
%decidable theory,\footnote{This means that a mechanical algorithm exists to
%determine whether or not a wff is a theorem.} so in principle we could make use
%of symbol sequences that are not theorems to represent other things (say, to
%encode actual theorems in a more compact way).  For example, let us extend the
%language by defining a wff ``$P$'' in the extended language as the theorem
%``$P\rightarrow P$''\footnote{This is one of the first theorems proved in the
%Metamath database \texttt{set.mm}.}\index{set
%theory database (\texttt{set.mm})} in the original language whenever ``$P$'' is
%not a theorem in the original language.  In the extended language, any wff
%``$Q$'' thus represents a theorem; to find out what theorem (in the original
%language) ``$Q$'' represents, we determine whether ``$Q$'' is a theorem in the
%original language (before the definition was introduced).  If so, we're done; if
%not, we replace ``$Q$'' by ``$Q\rightarrow Q$'' to eliminate the definition.
%This definition is therefore eliminable, and it does not ``strengthen'' the
%language because any wff that is not a theorem is not in the set of statements
%provable in the original language and thus is available for use by definitions.
%
%Of course, a definition such as this would render practically useless the
%communication of theorems of propositional calculus; but
%this is just a human shortcoming, since we can't always easily discern what is
%and is not a theorem by inspection.  In fact, the extended theory with this
%definition has no more and no less information than the original theory; it just
%expresses certain theorems of the form ``$P\rightarrow P$''
%in a more compact way.
%
%The point here is that what constitutes a proper definition is a matter of
%judgment about whether a symbol sequence can easily be recognized by a human
%as invalid in some sense (for example, not a wff); if so, the symbol sequence
%can be appropriated for use by a definition in order to make the extended
%language more compact.  Metamath\index{Metamath} lacks the ability to make this
%judgment, since as far as Metamath is concerned the definition of a wff, for
%example, is arbitrary.  You define for Metamath how wffs\index{well-formed
%formula (wff)} are constructed according to your own preferred style.  The
%concept of a wff may not even exist in a given formal system\index{formal
%system}.  Metamath treats all definitions as if they were new axioms, and it
%is up to the human mathematician to judge whether the definition is ``proper''
%'\index{proper definition}\index{definition!proper} in some agreed-upon way.

What constitutes a definition\index{definition} versus\index{axiom vs.\
definition} an axiom\index{axiom} is sometimes arbitrary in mathematical
literature.  For example, the connectives $\vee$ ({\sc or}), $\wedge$
({\sc and}), and $\leftrightarrow$ (equivalent to) in propositional
calculus are usually considered defined symbols that can be used as
abbreviations for expressions containing the ``primitive'' connectives
$\rightarrow$ and $\neg$.  This is the way we treat them in the standard
logic and set theory database \texttt{set.mm}\index{set theory database
(\texttt{set.mm})}.  However, the first three connectives can also be
considered ``primitive,'' and axiom systems have been devised that treat
all of them as such.  For example,
\cite[p.~35]{Goodstein}\index{Goodstein, R. L.} presents one with 15
axioms, some of which in fact coincide with what we have chosen to call
definitions in \texttt{set.mm}.  In certain subsets of classical
propositional calculus, such as the intuitionist
fragment\index{intuitionism}, it can be shown that one cannot make do
with just $\rightarrow$ and $\neg$ but must treat additional connectives
as primitive in order for the system to make sense.\footnote{Two nice
systems that make the transition from intuitionistic and other weak
fragments to classical logic just by adding axioms are given in
\cite{Robinsont}\index{Robinson, T. Thacher}.}

\subsection{The Approach to Definitions in \texttt{set.mm}}

In set theory, recursive definitions define a newly introduced symbol in
terms of itself.
The justification of recursive definitions, using
several ``recursion theorems,'' is usually one of the first
sophisticated proofs a student encounters when learning set theory, and
there is a significant amount of implicit metalogic behind a recursive
definition even though the definition itself is typically simple to
state.

Metamath itself has no built-in technical limitation that prevents
multiple-part recursive definitions in the traditional textbook style.
However, because the recursive definition requires advanced metalogic
to justify, eliminating a recursive definition is very difficult and
often not even shown in textbooks.

\subsubsection{Direct definitions instead of recursive definitions}

It is, however, possible to substitute one kind of complexity
for another.  We can eliminate the need for metalogical justification by
defining the operation directly with an explicit (but complicated)
expression, then deriving the recursive definition directly as a
theorem, using a recursion theorem ``in reverse.''
The elimination
of a direct definition is a matter of simple mechanical substitution.
We do this in
\texttt{set.mm}, as follows.

In \texttt{set.mm} our goal was to introduce almost all definitions in
the form of two expressions connected by either $\leftrightarrow$ or
$=$, where the thing being defined does not appear on the right hand
side.  Quine calls this form ``a genuine or direct definition'' \cite[p.
174]{Quine}\index{Quine, Willard Van Orman}, which makes the definitions
very easy to eliminate and the metalogic\index{metalogic} needed to
justify them as simple as possible.
Put another way, we had a goal of being able to
eliminate all definitions with direct mechanical substitution and to
verify easily the soundness of the definitions.

\subsubsection{Example of direct definitions}

We achieved this goal in almost all cases in \texttt{set.mm}.
Sometimes this makes the definitions more complex and less
intuitive.
For example, the traditional way to define addition of
natural numbers is to define an operation called {\em
successor}\index{successor} (which means ``plus one'' and is denoted by
``${\rm suc}$''), then define addition recursively\index{recursive
definition} with the two definitions $n + 0 = n$ and $m + {\rm suc}\,n =
{\rm suc} (m + n)$.  Although this definition seems simple and obvious,
the method to eliminate the definition is not obvious:  in the second
part of the definition, addition is defined in terms of itself.  By
eliminating the definition, we don't mean repeatedly applying it to
specific $m$ and $n$ but rather showing the explicit, closed-form
set-theoretical expression that $m + n$ represents, that will work for
any $m$ and $n$ and that does not have a $+$ sign on its right-hand
side.  For a recursive definition like this not to be circular
(creative), there are some hidden, underlying assumptions we must make,
for example that the natural numbers have a certain kind of order.

In \texttt{set.mm} we chose to start with the direct (though complex and
nonintuitive) definition then derive from it the standard recursive
definition.
For example, the closed-form definition used in \texttt{set.mm}
for the addition operation on ordinals\index{ordinal
addition}\index{addition!of ordinals} (of which natural numbers are a
subset) is

\setbox\startprefix=\hbox{\tt \ \ df-oadd\ \$a\ }
\setbox\contprefix=\hbox{\tt \ \ \ \ \ \ \ \ \ \ \ \ \ }
\startm
\m{\vdash}\m{+_o}\m{=}\m{(}\m{x}\m{\in}\m{{\rm On}}\m{,}\m{y}\m{\in}\m{{\rm
On}}\m{\mapsto}\m{(}\m{{\rm rec}}\m{(}\m{(}\m{z}\m{\in}\m{{\rm
V}}\m{\mapsto}\m{{\rm suc}}\m{z}\m{)}\m{,}\m{x}\m{)}\m{`}\m{y}\m{)}\m{)}
\endm
\noindent which depends on ${\rm rec}$.

\subsubsection{Recursion operators}

The above definition of \texttt{df-oadd} depends on the definition of
${\rm rec}$, a ``recursion operator''\index{recursion operator} with
the definition \texttt{df-rdg}:

\setbox\startprefix=\hbox{\tt \ \ df-rdg\ \$a\ }
\setbox\contprefix=\hbox{\tt \ \ \ \ \ \ \ \ \ \ \ \ }
\startm
\m{\vdash}\m{{\rm
rec}}\m{(}\m{F}\m{,}\m{I}\m{)}\m{=}\m{\mathrm{recs}}\m{(}\m{(}\m{g}\m{\in}\m{{\rm
V}}\m{\mapsto}\m{{\rm if}}\m{(}\m{g}\m{=}\m{\varnothing}\m{,}\m{I}\m{,}\m{{\rm
if}}\m{(}\m{{\rm Lim}}\m{{\rm dom}}\m{g}\m{,}\m{\bigcup}\m{{\rm
ran}}\m{g}\m{,}\m{(}\m{F}\m{`}\m{(}\m{g}\m{`}\m{\bigcup}\m{{\rm
dom}}\m{g}\m{)}\m{)}\m{)}\m{)}\m{)}\m{)}
\endm

\noindent which can be further broken down with definitions shown in
Section~\ref{setdefinitions}.

This definition of ${\rm rec}$
defines a recursive definition generator on ${\rm On}$ (the class of ordinal
numbers) with characteristic function $F$ and initial value $I$.
This operation allows us to define, with
compact direct definitions, functions that are usually defined in
textbooks with recursive definitions.
The price paid with our approach
is the complexity of our ${\rm rec}$ operation
(especially when {\tt df-recs} that it is built on is also eliminated).
But once we get past this hurdle, definitions that would otherwise be
recursive become relatively simple, as in for example {\tt oav}, from
which we prove the recursive textbook definition as theorems {\tt oa0}, {\tt
oasuc}, and {\tt oalim} (with the help of theorems {\tt rdg0}, {\tt rdgsuc},
and {\tt rdglim2a}).  We can also restrict the ${\rm rec}$ operation to
define otherwise recursive functions on the natural numbers $\omega$; see {\tt
fr0g} and {\tt frsuc}.  Our ${\rm rec}$ operation apparently does not appear
in published literature, although closely related is Definition 25.2 of
[Quine] p. 177, which he uses to ``turn...a recursion into a genuine or
direct definition" (p. 174).  Note that the ${\rm if}$ operations (see
{\tt df-if}) select cases based on whether the domain of $g$ is zero, a
successor, or a limit ordinal.

An important use of this definition ${\rm rec}$ is in the recursive sequence
generator {\tt df-seq} on the natural numbers (as a subset of the
complex infinite sequences such as the factorial function {\tt df-fac} and
integer powers {\tt df-exp}).

The definition of ${\rm rec}$ depends on ${\rm recs}$.
New direct usage of the more powerful (and more primitive) ${\rm recs}$
construct is discouraged, but it is available when needed.
This
defines a function $\mathrm{recs} ( F )$ on ${\rm On}$, the class of ordinal
numbers, by transfinite recursion given a rule $F$ which sets the next
value given all values so far.
Unlike {\tt df-rdg} which restricts the
update rule to use only the previous value, this version allows the
update rule to use all previous values, which is why it is described
as ``strong,'' although it is actually more primitive.  See {\tt
recsfnon} and {\tt recsval} for the primary contract of this definition.
It is defined as:

\setbox\startprefix=\hbox{\tt \ \ df-recs\ \$a\ }
\setbox\contprefix=\hbox{\tt \ \ \ \ \ \ \ \ \ \ \ \ \ }
\startm
\m{\vdash}\m{\mathrm{recs}}\m{(}\m{F}\m{)}\m{=}\m{\bigcup}\m{\{}\m{f}\m{|}\m{\exists}\m{x}\m{\in}\m{{\rm
On}}\m{(}\m{f}\m{{\rm
Fn}}\m{x}\m{\wedge}\m{\forall}\m{y}\m{\in}\m{x}\m{(}\m{f}\m{`}\m{y}\m{)}\m{=}\m{(}\m{F}\m{`}\m{(}\m{f}\m{\restriction}\m{y}\m{)}\m{)}\m{)}\m{\}}
\endm

\subsubsection{Closing comments on direct definitions}

From these direct definitions the simpler, more
intuitive recursive definition is derived as a set of theorems.\index{natural
number}\index{addition}\index{recursive definition}\index{ordinal addition}
The end result is the same, but we completely eliminate the rather complex
metalogic that justifies the recursive definition.

Recursive definitions are often considered more efficient and intuitive than
direct ones once the metalogic has been learned or possibly just accepted as
correct.  However, it was felt that direct definition in \texttt{set.mm}
maximizes rigor by minimizing metalogic.  It can be eliminated effortlessly,
something that is difficult to do with a recursive definition.

Again, Metamath itself has no built-in technical limitation that prevents
multiple-part recursive definitions in the traditional textbook style.
Instead, our goal is to eliminate all definitions with
direct mechanical substitution and to verify easily the soundness of
definitions.

\subsection{Adding Constraints on Definitions}

The basic Metamath language and the Metamath program do
not have any built-in constraints on definitions, since they are just
\$a statements.

However, nothing prevents a verification system from verifying
additional rules to impose further limitations on definitions.
For example, the \texttt{mmj2}\index{mmj2} program
supports various kinds of
additional information comments (see section \ref{jcomment}).
One of their uses is to optionally verify additional constraints,
including constraints to verify that definitions meet certain
requirements.
These additional checks are required by the
continuous integration (CI)\index{continuous integration (CI)}
checks of the
\texttt{set.mm}\index{set theory database (\texttt{set.mm})}%
\index{Metamath Proof Explorer}
database.
This approach enables us to optionally impose additional requirements
on definitions if we wish, without necessarily imposing those rules on
all databases or requiring all verification systems to implement them.
In addition, this allows us to impose specialized constraints tailored
to one database while not requiring other systems to implement
those specialized constraints.

We impose two constraints on the
\texttt{set.mm}\index{set theory database (\texttt{set.mm})}%
\index{Metamath Proof Explorer} database
via the \texttt{mmj2}\index{mmj2} program that are worth discussing here:
a parse check and a definition soundness check.

% On February 11, 2019 8:32:32 PM EST, saueran@oregonstate.edu wrote:
% The following addition to the end of set.mm is accepted by the mmj2
% parser and definition checker and the metamath verifier(at least it was
% when I checked, you should check it too), and creates a contradiction by
% proving the theorem |- ph.
% ${
% wleftp $a wff ( ( ph ) $.
% wbothp $a wff ( ph ) $.
% df-leftp $a |- ( ( ( ph ) <-> -. ph ) $.
% df-bothp $a |- ( ( ph ) <-> ph ) $.
% anything $p |- ph $=
%   ( wbothp wn wi wleftp df-leftp biimpi df-bothp mpbir mpbi simplim ax-mp)
%   ABZAMACZDZCZMOEZOCQAEZNDZRNAFGSHIOFJMNKLAHJ $.
% $}
%
% This particular problem is countered by enabling, within mmj2,
% SetParser,mmj.verify.LRParser

First,
we enable a parse check in \texttt{mmj2} (through its
\texttt{SetParser} command) that requires that all new definitions
must \textit{not} create an ambiguous parse for a KLR(5) parser.
This prevents some errors such as definitions with imbalanced parentheses.

Second, we run a definition soundness check specific to
\texttt{set.mm} or databases similar to it.
(through the \texttt{definitionCheck} macro).
Some \texttt{\$a} statements (including all ax-* statemnets)
are excluded from these checks, as they will
always fail this simple check,
but they are appropriate for most definitions.
This check imposes a set of additional rules:

\begin{enumerate}

\item New definitions must be introduced using $=$ or $\leftrightarrow$.

\item No \texttt{\$a} statement introduced before this one is allowed to use the
symbol being defined in this definition, and the definition is not
allowed to use itself (except once, in the definiendum).

\item Every variable in the definiens should not be distinct

\item Every dummy variable in the definiendum
are required to be distinct from each other and from variables in
the definiendum.
To determine this, the system will look for a "justification" theorem
in the database, and if it is not there, attempt to briefly prove
$( \varphi \rightarrow \forall x \varphi )$  for each dummy variable x.

\item Every dummy variable should be a set variable,
unless there is a justification theorem available.

\item Every dummy variable must be bound
(if the system cannot determine this a justification theorem must be
provided).

\end{enumerate}

\subsection{Summary of Approach to Definitions}

In short, when being rigorous it turns out that
definitions can be subtle, sometimes requiring difficult
metatheorems to establish that they are not creative.

Instead of building such complications into the Metamath language itself,
the basic Metmath language and program simply treat traditional
axioms and definitions as the same kind of \texttt{\$a} statement.
We have then built various tools to enable people to
verify additional conditions as their creators believe is appropriate
for those specific databases, without complicating the Metamath foundations.

\chapter{The Metamath Program}\label{commands}

This chapter provides a reference manual for the
Metamath program.\index{Metamath!commands}

Current instructions for obtaining and installing the Metamath program
can be found at the \url{http://metamath.org} web site.
For Windows, there is a pre-compiled version called
\texttt{metamath.exe}.  For Unix, Linux, and Mac OS X
(which we will refer to collectively as ``Unix''), the Metamath program
can be compiled from its source code with the command
\begin{verbatim}
gcc *.c -o metamath
\end{verbatim}
using the \texttt{gcc} {\sc c} compiler available on those systems.

In the command syntax descriptions below, fields enclosed in square
brackets [\ ] are optional.  File names may be optionally enclosed in
single or double quotes.  This is useful if the file name contains
spaces or
slashes (\texttt{/}), such as in Unix path names, \index{Unix file
names}\index{file names!Unix} that might be confused with Metamath
command qualifiers.\index{Unix file names}\index{file names!Unix}

\section{Invoking Metamath}

Unix, Linux, and Mac OS X
have a command-line interface called the {\em
bash shell}.  (In Mac OS X, select the Terminal application from
Applications/Utilities.)  To invoke Metamath from the bash shell prompt,
assuming that the Metamath program is in the current directory, type
\begin{verbatim}
bash$ ./metamath
\end{verbatim}

To invoke Metamath from a Windows DOS or Command Prompt, assuming that
the Metamath program is in the current directory (or in a directory
included in the Path system environment variable), type
\begin{verbatim}
C:\metamath>metamath
\end{verbatim}

To use command-line arguments at invocation, the command-line arguments
should be a list of Metamath commands, surrounded by quotes if they
contain spaces.  In Windows, the surrounding quotes must be double (not
single) quotes.  For example, to read the database file \texttt{set.mm},
verify all proofs, and exit the program, type (under Unix)
\begin{verbatim}
bash$ ./metamath 'read set.mm' 'verify proof *' exit
\end{verbatim}
Note that in Unix, any directory path with \texttt{/}'s must be
surrounded by quotes so Metamath will not interpret the \texttt{/} as a
command qualifier.  So if \texttt{set.mm} is in the \texttt{/tmp}
directory, use for the above example
\begin{verbatim}
bash$ ./metamath 'read "/tmp/set.mm"' 'verify proof *' exit
\end{verbatim}

For convenience, if the command-line has one argument and no spaces in
the argument, the command is implicitly assumed to be \texttt{read}.  In
this one special case, \texttt{/}'s are not interpreted as command
qualifiers, so you don't need quotes around a Unix file name.  Thus
\begin{verbatim}
bash$ ./metamath /tmp/set.mm
\end{verbatim}
and
\begin{verbatim}
bash$ ./metamath "read '/tmp/set.mm'"
\end{verbatim}
are equivalent.


\section{Controlling Metamath}

The Metamath program was first developed on a {\sc vax/vms} system, and
some aspects of its command line behavior reflect this heritage.
Hopefully you will find it reasonably user-friendly once you get used to
it.

Each command line is a sequence of English-like words separated by
spaces, as in \texttt{show settings}.  Command words are not case
sensitive, and only as many letters are needed as are necessary to
eliminate ambiguity; for example, \texttt{sh se} would work for the
command \texttt{show settings}.  In some cases arguments such as file
names, statement labels, or symbol names are required; these are
case-sensitive (although file names may not be on some operating
systems).

A command line is entered by typing it in then pressing the {\em return}
({\em enter}) key.  To find out what commands are available, type
\texttt{?} at the \texttt{MM>} prompt.  To find out the choices at any
point in a command, press {\em return} and you will be prompted for
them.  The default choice (the one selected if you just press {\em
return}) is shown in brackets (\texttt{<>}).

You may also type \texttt{?} in place of a command word to force
Metamath to tell you what the choices are.  The \texttt{?} method won't
work, though, if a non-keyword argument such as a file name is expected
at that point, because the program will think that \texttt{?} is the
value of the argument.

Some commands have one or more optional qualifiers which modify the
behavior of the command.  Qualifiers are preceded by a slash
(\texttt{/}), such as in \texttt{read set.mm / verify}.  Spaces are
optional around the \texttt{/}.  If you need to use a space or
slash in a command
argument, as in a Unix file name, put single or double quotes around the
command argument.

The \texttt{open log} command will save everything you see on the
screen and is useful to help you recover should something go wrong in a
proof, or if you want to document a bug.

If a command responds with more than a screenful, you will be
prompted to \texttt{<return> to continue, Q to quit, or S to scroll to
end}.  \texttt{Q} or \texttt{q} (not case-sensitive) will complete the
command internally but will suppress further output until the next
\texttt{MM>} prompt.  \texttt{s} will suppress further pausing until the
next \texttt{MM>} prompt.  After the first screen, you are also
presented with the choice of \texttt{b} to go back a screenful.  Note
that \texttt{b} may also be entered at the \texttt{MM>} prompt
immediately after a command to scroll back through the output of that
command.

A command line enclosed in quotes is executed by your operating system.
See Section~\ref{oscmd}.

{\em Warning:} Pressing {\sc ctrl-c} will abort the Metamath program
unconditionally.  This means any unsaved work will be lost.


\subsection{\texttt{exit} Command}\index{\texttt{exit} command}

Syntax:  \texttt{exit} [\texttt{/force}]

This command exits from Metamath.  If there have been changes to the
source with the \texttt{save proof} or \texttt{save new{\char`\_}proof}
commands, you will be given an opportunity to \texttt{write source} to
permanently save the changes.

In Proof Assistant\index{Proof Assistant} mode, the \texttt{exit} command will
return to the \verb/MM>/ prompt. If there were changes to the proof, you will
be given an opportunity to \texttt{save new{\char`\_}proof}.

The \texttt{quit} command is a synonym for \texttt{exit}.

Optional qualifier:
    \texttt{/force} - Do not prompt if changes were not saved.  This qualifier is
        useful in \texttt{submit} command files (Section~\ref{sbmt})
        to ensure predictable behavior.





\subsection{\texttt{open log} Command}\index{\texttt{open log} command}
Syntax:  \texttt{open log} {\em file-name}

This command will open a log file that will store everything you see on
the screen.  It is useful to help recovery from a mistake in a long Proof
Assistant session, or to document bugs.\index{Metamath!bugs}

The log file can be closed with \texttt{close log}.  It will automatically be
closed upon exiting Metamath.



\subsection{\texttt{close log} Command}\index{\texttt{close log} command}
Syntax:  \texttt{close log}

The \texttt{close log} command closes a log file if one is open.  See
also \texttt{open log}.




\subsection{\texttt{submit} Command}\index{\texttt{submit} command}\label{sbmt}
Syntax:  \texttt{submit} {\em filename}

This command causes further command lines to be taken from the specified
file.  Note that any line beginning with an exclamation point (\texttt{!}) is
treated as a comment (i.e.\ ignored).  Also note that the scrolling
of the screen output is continuous, so you may want to open a log file
(see \texttt{open log}) to record the results that fly by on the screen.
After the lines in the file are exhausted, Metamath returns to its
normal user interface mode.

The \texttt{submit} command can be called recursively (i.e. from inside
of a \texttt{submit} command file).


Optional command qualifier:

    \texttt{/silent} - suppresses the screen output but still
        records the output in a log file if one is open.


\subsection{\texttt{erase} Command}\index{\texttt{erase} command}
Syntax:  \texttt{erase}

This command will reset Metamath to its starting state, deleting any
data\-base that was \texttt{read} in.
 If there have been changes to the
source with the \texttt{save proof} or \texttt{save new{\char`\_}proof}
commands, you will be given an opportunity to \texttt{write source} to
permanently save the changes.



\subsection{\texttt{set echo} Command}\index{\texttt{set echo} command}
Syntax:  \texttt{set echo on} or \texttt{set echo off}

The \texttt{set echo on} command will cause command lines to be echoed with any
abbreviations expanded.  While learning the Metamath commands, this
feature will show you the exact command that your abbreviated input
corresponds to.



\subsection{\texttt{set scroll} Command}\index{\texttt{set scroll} command}
Syntax:  \texttt{set scroll prompted} or \texttt{set scroll continuous}

The Metamath command line interface starts off in the \texttt{prompted} mode,
which means that you will be prompted to continue or quit after each
full screen in a long listing.  In \texttt{continuous} mode, long listings will be
scrolled without pausing.

% LaTeX bug? (1) \texttt{\_} puts out different character than
% \texttt{{\char`\_}}
%  = \verb$_$  (2) \texttt{{\char`\_}} puts out garbage in \subsection
%  argument
\subsection{\texttt{set width} Command}\index{\texttt{set
width} command}
Syntax:  \texttt{set width} {\em number}

Metamath assumes the width of your screen is 79 characters (chosen
because the Command Prompt in Windows XP has a wrapping bug at column
80).  If your screen is wider or narrower, this command allows you to
change this default screen width.  A larger width is advantageous for
logging proofs to an output file to be printed on a wide printer.  A
smaller width may be necessary on some terminals; in this case, the
wrapping of the information messages may sometimes seem somewhat
unnatural, however.  In \LaTeX\index{latex@{\LaTeX}!characters per line},
there is normally a maximum of 61 characters per line with typewriter
font.  (The examples in this book were produced with 61 characters per
line.)

\subsection{\texttt{set height} Command}\index{\texttt{set
height} command}
Syntax:  \texttt{set height} {\em number}

Metamath assumes your screen height is 24 lines of characters.  If your
screen is taller or shorter, this command lets you to change the number
of lines at which the display pauses and prompts you to continue.

\subsection{\texttt{beep} Command}\index{\texttt{beep} command}

Syntax:  \texttt{beep}

This command will produce a beep.  By typing it ahead after a
long-running command has started, it will alert you that the command is
finished.  For convenience, \texttt{b} is an abbreviation for
\texttt{beep}.

Note:  If \texttt{b} is typed at the \texttt{MM>} prompt immediately
after the end of a multiple-page display paged with ``\texttt{Press
<return> for more}...'' prompts, then the \texttt{b} will back up to the
previous page rather than perform the \texttt{beep} command.
In that case you must type the unabbreviated \texttt{beep} form
of the command.

\subsection{\texttt{more} Command}\index{\texttt{more} command}

Syntax:  \texttt{more} {\em filename}

This command will display the contents of an {\sc ascii} file on your
screen.  (This command is provided for convenience but is not very
powerful.  See Section~\ref{oscmd} to invoke your operating system's
command to do this, such as the \texttt{more} command in Unix.)

\subsection{Operating System Commands}\index{operating system
command}\label{oscmd}

A line enclosed in single or double quotes will be executed by your
computer's operating system if it has a command line interface.  For
example, on a {\sc vax/vms} system,
\verb/MM> 'dir'/
will print disk directory contents.  Note that this feature will not
work on the Macintosh prior to Mac OS X, which does not have a command
line interface.

For your convenience, the trailing quote is optional.

\subsection{Size Limitations in Metamath}

In general, there are no fixed, predefined limits\index{Metamath!memory
limits} on how many labels, tokens\index{token}, statements, etc.\ that
you may have in a database file.  The Metamath program uses 32-bit
variables (64-bit on 64-bit CPUs) as indices for almost all internal
arrays, which are allocated dynamically as needed.



\section{Reading and Writing Files}

The following commands create new files:  the \texttt{open} commands;
the \texttt{write} commands; the \texttt{/html},
\texttt{/alt{\char`\_}html}, \texttt{/brief{\char`\_}html},
\texttt{/brief{\char`\_}alt{\char`\_}html} qualifiers of \texttt{show
statement}, and \texttt{midi}.  The following commands append to files
previously opened:  the \texttt{/tex} qualifier of \texttt{show proof}
and \texttt{show new{\char`\_}proof}; the \texttt{/tex} and
\texttt{/simple{\char`\_}tex} qualifiers of \texttt{show statement}; the
\texttt{close} commands; and all screen dialog between \texttt{open log}
and \texttt{close log}.

The commands that create new files will not overwrite an existing {\em
filename} but will rename the existing one to {\em
filename}\texttt{{\char`\~}1}.  An existing {\em
filename}\texttt{{\char`\~}1} is renamed {\em
filename}\texttt{{\char`\~}2}, etc.\ up to {\em
filename}\texttt{{\char`\~}9}.  An existing {\em
filename}\texttt{{\char`\~}9} is deleted.  This makes recovery from
mistakes easier but also will clutter up your directory, so occasionally
you may want to clean up (delete) these \texttt{{\char`\~}}$n$ files.


\subsection{\texttt{read} Command}\index{\texttt{read} command}
Syntax:  \texttt{read} {\em file-name} [\texttt{/verify}]

This command will read in a Metamath language source file and any included
files.  Normally it will be the first thing you do when entering Metamath.
Statement syntax is checked, but proof syntax is not checked.
Note that the file name may be enclosed in single or double quotes;
this is useful if the file name contains slashes, as might be the case
under Unix.

If you are getting an ``\texttt{?Expected VERIFY}'' error
when trying to read a Unix file name with slashes, you probably haven't
quoted it.\index{Unix file names}\index{file names!Unix}

If you are prompted for the file name (by pressing {\em return}
 after \texttt{read})
you should {\em not} put quotes around it, even if it is a Unix file name
with slashes.

Optional command qualifier:

    \texttt{/verify} - Verify all proofs as the database is read in.  This
         qualifier will slow down reading in the file.  See \texttt{verify
         proof} for more information on file error-checking.

See also \texttt{erase}.



\subsection{\texttt{write source} Command}\index{\texttt{write source} command}
Syntax:  \texttt{write source} {\em filename}
[\texttt{/rewrap}]
[\texttt{/split}]
% TeX doesn't handle this long line with tt text very well,
% so force a line break here.
[\texttt{/keep\_includes}] {\\}
[\texttt{/no\_versioning}]

This command will write the contents of a Metamath\index{database}
database into a file.\index{source file}

Optional command qualifiers:

\texttt{/rewrap} -
Reformats statements and comments according to the
convention used in the set.mm database.
It unwraps the
lines in the comment before each \texttt{\$a} and \texttt{\$p} statement,
then it
rewraps the line.  You should compare the output to the original
to make sure that the desired effect results; if not, go back to
the original source.  The wrapped line length honors the
\texttt{set width}
parameter currently in effect.  Note:  Text
enclosed in \texttt{<HTML>}...\texttt{</HTML>} tags is not modified by the
\texttt{/rewrap} qualifier.
Proofs themselves are not reformatted;
use \texttt{save proof * / compressed} to do that.
An isolated tilde (\~{}) is always kept on the same line as the following
symbol, so you can find all comment references to a symbol by
searching for \~{} followed by a space and that symbol
(this is useful for finding cross-references).
Incidentally, \texttt{save proof} also honors the \texttt{set width}
parameter currently in effect.

\texttt{/split} - Files included in the source using the expression
\$[ \textit{inclfile} \$] will be
written into separate files instead of being included in a single output
file.  The name of each separately written included file will be the
\textit{inclfile} argument of its inclusion command.

\texttt{/keep\_includes} - If a source file has includes but is written as a
single file by omitting \texttt{/split}, by default the included files will
be deleted (actually just renamed with a \char`\~1 suffix unless
\texttt{/no\_versioning} is specified) to prevent the possibly confusing
source duplication in both the output file and the included file.
The \texttt{/keep\_includes} qualifier will prevent this deletion.

\texttt{/no\_versioning} - Backup files suffixed with \char`\~1 are not created.


\section{Showing Status and Statements}



\subsection{\texttt{show settings} Command}\index{\texttt{show settings} command}
Syntax:  \texttt{show settings}

This command shows the state of various parameters.

\subsection{\texttt{show memory} Command}\index{\texttt{show memory} command}
Syntax:  \texttt{show memory}

This command shows the available memory left.  It is not meaningful
on most modern operating systems,
which have virtual memory.\index{Metamath!memory usage}


\subsection{\texttt{show labels} Command}\index{\texttt{show labels} command}
Syntax:  \texttt{show labels} {\em label-match} [\texttt{/all}]
   [\texttt{/linear}]

This command shows the labels of \texttt{\$a} and \texttt{\$p}
statements that match {\em label-match}.  A \verb$*$ in {label-match}
matches zero or more characters.  For example, \verb$*abc*def$ will match all
labels containing \verb$abc$ and ending with \verb$def$.

Optional command qualifiers:

   \texttt{/all} - Include matches for \texttt{\$e} and \texttt{\$f}
   statement labels.

   \texttt{/linear} - Display only one label per line.  This can be useful for
       building scripts in conjunction with the utilities under the
       \texttt{tools}\index{\texttt{tools} command} command.



\subsection{\texttt{show statement} Command}\index{\texttt{show statement} command}
Syntax:  \texttt{show statement} {\em label-match} [{\em qualifiers} (see below)]

This command provides information about a statement.  Only statements
that have labels (\texttt{\$f}\index{\texttt{\$f} statement},
\texttt{\$e}\index{\texttt{\$e} statement},
\texttt{\$a}\index{\texttt{\$a} statement}, and
\texttt{\$p}\index{\texttt{\$p} statement}) may be specified.
If {\em label-match}
contains wildcard (\verb$*$) characters, all matching statements will be
displayed in the order they occur in the database.

Optional qualifiers (only one qualifier at a time is allowed):

    \texttt{/comment} - This qualifier includes the comment that immediately
       precedes the statement.

    \texttt{/full} - Show complete information about each statement,
       and show all
       statements matching {\em label} (including \texttt{\$e}
       and \texttt{\$f} statements).

    \texttt{/tex} - This qualifier will write the statement information to the
       \LaTeX\ file previously opened with \texttt{open tex}.  See
       Section~\ref{texout}.

    \texttt{/simple{\char`\_}tex} - The same as \texttt{/tex}, except that
       \LaTeX\ macros are not used for formatting equations, allowing easier
       manual edits of the output for slide presentations, etc.

    \texttt{/html}\index{html generation@{\sc html} generation},
       \texttt{/alt{\char`\_}html}, \texttt{/brief{\char`\_}html},
       \texttt{/brief{\char`\_}alt{\char`\_}html} -
       These qualifiers invoke a special mode of
       \texttt{show statement} that
       creates a web page for the statement.  They may not be used with
       any other qualifier.  See Section~\ref{htmlout} or
       \texttt{help html} in the program.


\subsection{\texttt{search} Command}\index{\texttt{search} command}
Syntax:  search {\em label-match}
\texttt{"}{\em symbol-match}{\tt}" [\texttt{/all}] [\texttt{/comments}]
[\texttt{/join}]

This command searches all \texttt{\$a} and \texttt{\$p} statements
matching {\em label-match} for occurrences of {\em symbol-match}.  A
\verb@*@ in {\em label-match} matches any label character.  A \verb@$*@
in {\em symbol-match} matches any sequence of symbols.  The symbols in
{\em symbol-match} must be separated by white space.  The quotes
surrounding {\em symbol-match} may be single or double quotes.  For
example, \texttt{search b}\verb@* "-> $* ch"@ will list all statements
whose labels begin with \texttt{b} and contain the symbols \verb@->@ and
\texttt{ch} surrounding any symbol sequence (including no symbol
sequence).  The wildcards \texttt{?} and \texttt{\$?} are also available
to match individual characters in labels and symbols respectively; see
\texttt{help search} in the Metamath program for details on their usage.

Optional command qualifiers:

    \texttt{/all} - Also search \texttt{\$e} and \texttt{\$f} statements.

    \texttt{/comments} - Search the comment that immediately precedes each
        label-matched statement for {\em symbol-match}.  In this case
        {\em symbol-match} is an arbitrary, non-case-sensitive character
        string.  Quotes around {\em symbol-match} are optional if there
        is no ambiguity.

    \texttt{/join} - In the case of a \texttt{\$a} or \texttt{\$p} statement,
	prepend its \texttt{\$e}
	hypotheses for searching. The
	\texttt{/join} qualifier has no effect in \texttt{/comments} mode.

\section{Displaying and Verifying Proofs}


\subsection{\texttt{show proof} Command}\index{\texttt{show proof} command}
Syntax:  \texttt{show proof} {\em label-match} [{\em qualifiers} (see below)]

This command displays the proof of the specified
\texttt{\$p}\index{\texttt{\$p} statement} statement in various formats.
The {\em label-match} may contain wildcard (\verb@$*@) characters to match
multiple statements.  Without any qualifiers, only the logical steps
will be shown (i.e.\ syntax construction steps will be omitted), in an
indented format.

Most of the time, you will use
    \texttt{show proof} {\em label}
to see just the proof steps corresponding to logical inferences.

Optional command qualifiers:

    \texttt{/essential} - The proof tree is trimmed of all
        \texttt{\$f}\index{\texttt{\$f} statement} hypotheses before
        being displayed.  (This is the default, and it is redundant to
        specify it.)

    \texttt{/all} - the proof tree is not trimmed of all \texttt{\$f} hypotheses before
        being displayed.  \texttt{/essential} and \texttt{/all} are mutually exclusive.

    \texttt{/from{\char`\_}step} {\em step} - The display starts at the specified
        step.  If
        this qualifier is omitted, the display starts at the first step.

    \texttt{/to{\char`\_}step} {\em step} - The display ends at the specified
        step.  If this
        qualifier is omitted, the display ends at the last step.

    \texttt{/tree{\char`\_}depth} {\em number} - Only
         steps at less than the specified proof
        tree depth are displayed.  Sometimes useful for obtaining an overview of
        the proof.

    \texttt{/reverse} - The steps are displayed in reverse order.

    \texttt{/renumber} - When used with \texttt{/essential}, the steps are renumbered
        to correspond only to the essential steps.

    \texttt{/tex} - The proof is converted to \LaTeX\ and\index{latex@{\LaTeX}}
        stored in the file opened
        with \texttt{open tex}.  See Section~\ref{texout} or
        \texttt{help tex} in the program.

    \texttt{/lemmon} - The proof is displayed in a non-indented format known
        as Lemmon style, with explicit previous step number references.
        If this qualifier is omitted, steps are indented in a tree format.

    \texttt{/start{\char`\_}column} {\em number} - Overrides the default column
        (16)
        at which the formula display starts in a Lemmon-style display.  May be
        used only in conjunction with \texttt{/lemmon}.

    \texttt{/normal} - The proof is displayed in normal format suitable for
        inclusion in a Metamath source file.  May not be used with any other
        qualifier.

    \texttt{/compressed} - The proof is displayed in compressed format
        suitable for inclusion in a Metamath source file.  May not be used with
        any other qualifier.

    \texttt{/statement{\char`\_}summary} - Summarizes all statements (like a
        brief \texttt{show statement})
        used by the proof.  It may not be used with any other qualifier
        except \texttt{/essential}.

    \texttt{/detailed{\char`\_}step} {\em step} - Shows the details of what is
        happening at
        a specific proof step.  May not be used with any other qualifier.
        The {\em step} is the step number shown when displaying a
        proof without the \texttt{/renumber} qualifier.


\subsection{\texttt{show usage} Command}\index{\texttt{show usage} command}
Syntax:  \texttt{show usage} {\em label-match} [\texttt{/recursive}]

This command lists the statements whose proofs make direct reference to
the statement specified.

Optional command qualifier:

    \texttt{/recursive} - Also include statements whose proofs ultimately
        depend on the statement specified.



\subsection{\texttt{show trace\_back} Command}\index{\texttt{show
       trace{\char`\_}back} command}
Syntax:  \texttt{show trace{\char`\_}back} {\em label-match} [\texttt{/essential}] [\texttt{/axioms}]
    [\texttt{/tree}] [\texttt{/depth} {\em number}]

This command lists all statements that the proof of the \texttt{\$p}
statement(s) specified by {\em label-match} depends on.

Optional command qualifiers:

    \texttt{/essential} - Restrict the trace-back to \texttt{\$e}
        \index{\texttt{\$e} statement} hypotheses of proof trees.

    \texttt{/axioms} - List only the axioms that the proof ultimately depends on.

    \texttt{/tree} - Display the trace-back in an indented tree format.

    \texttt{/depth} {\em number} - Restrict the \texttt{/tree} trace-back to the
        specified indentation depth.

    \texttt{/count{\char`\_}steps} - Count the number of steps the proof has
       all the way back to axioms.  If \texttt{/essential} is specified,
       expansions of variable-type hypotheses (syntax constructions) are not counted.

\subsection{\texttt{verify proof} Command}\index{\texttt{verify proof} command}
Syntax:  \texttt{verify proof} {\em label-match} [\texttt{/syntax{\char`\_}only}]

This command verifies the proofs of the specified statements.  {\em
label-match} may contain wild card characters (\texttt{*}) to verify
more than one proof; for example \verb/*abc*def/ will match all labels
containing \texttt{abc} and ending with \texttt{def}.
The command \texttt{verify proof *} will verify all proofs in the database.

Optional command qualifier:

    \texttt{/syntax{\char`\_}only} - This qualifier will perform a check of syntax
        and RPN
        stack violations only.  It will not verify that the proof is
        correct.  This qualifier is useful for quickly determining which
        proofs are incomplete (i.e.\ are under development and have \texttt{?}'s
        in them).

{\em Note:} \texttt{read}, followed by \texttt{verify proof *}, will ensure
 the database is free
from errors in the Metamath language but will not check the markup notation
in comments.
You can also check the markup notation by running \texttt{verify markup *}
as discussed in Section~\ref{verifymarkup}; see also the discussion
on generating {\sc HTML} in Section~\ref{htmlout}.

\subsection{\texttt{verify markup} Command}\index{\texttt{verify markup} command}\label{verifymarkup}
Syntax:  \texttt{verify markup} {\em label-match}
[\texttt{/date{\char`\_}skip}]
[\texttt{/top{\char`\_}date{\char`\_}skip}] {\\}
[\texttt{/file{\char`\_}skip}]
[\texttt{/verbose}]

This command checks comment markup and other informal conventions we have
adopted.  It error-checks the latexdef, htmldef, and althtmldef statements
in the \texttt{\$t} statement of a Metamath source file.\index{error checking}
It error-checks any \texttt{`}...\texttt{`},
\texttt{\char`\~}~\textit{label},
and bibliographic markups in statement descriptions.
It checks that
\texttt{\$p} and \texttt{\$a} statements
have the same content when their labels start with
``ax'' and ``ax-'' respectively but are otherwise identical, for example
ax4 and ax-4.
It also verifies the date consistency of ``(Contributed by...),''
``(Revised by...),'' and ``(Proof shortened by...)'' tags in the comment
above each \texttt{\$a} and \texttt{\$p} statement.

Optional command qualifiers:

    \texttt{/date{\char`\_}skip} - This qualifier will
        skip date consistency checking,
        which is usually not required for databases other than
	\texttt{set.mm}.

    \texttt{/top{\char`\_}date{\char`\_}skip} - This qualifier will check date consistency except
        that the version date at the top of the database file will not
        be checked.  Only one of
        \texttt{/date{\char`\_}skip} and
        \texttt{/top{\char`\_}date{\char`\_}skip} may be
        specified.

    \texttt{/file{\char`\_}skip} - This qualifier will skip checks that require
        external files to be present, such as checking GIF existence and
        bibliographic links to mmset.html or equivalent.  It is useful
        for doing a quick check from a directory without these files.

    \texttt{/verbose} - Provides more information.  Currently it provides a list
        of axXXX vs. ax-XXX matches.

\subsection{\texttt{save proof} Command}\index{\texttt{save proof} command}
Syntax:  \texttt{save proof} {\em label-match} [\texttt{/normal}]
   [\texttt{/compressed}]

The \texttt{save proof} command will reformat a proof in one of two formats and
replace the existing proof in the source buffer\index{source
buffer}.  It is useful for
converting between proof formats.  Note that a proof will not be
permanently saved until a \texttt{write source} command is issued.

Optional command qualifiers:

    \texttt{/normal} - The proof is saved in the normal format (i.e., as a
        sequence
        of labels, which is the defined format of the basic Metamath
        language).\index{basic language}  This is the default format that
        is used if a qualifier
        is omitted.

    \texttt{/compressed} - The proof is saved in the compressed format which
        reduces storage requirements for a database.
        See Appendix~\ref{compressed}.




\section{Creating Proofs}\label{pfcommands}\index{Proof Assistant}

Before using the Proof Assistant, you must add a \texttt{\$p} to your
source file (using a text editor) containing the statement you want to
prove.  Its proof should consist of a single \texttt{?}, meaning
``unknown step.''  Example:
\begin{verbatim}
equid $p x = x $= ? $.
\end{verbatim}

To enter the Proof assistant, type \texttt{prove} {\em label}, e.g.
\texttt{prove equid}.  Metamath will respond with the \texttt{MM-PA>}
prompt.

Proofs are created working backwards from the statement being proved,
primarily using a series of \texttt{assign} commands.  A proof is
complete when all steps are assigned to statements and all steps are
unified and completely known.  During the creation of a proof, Metamath
will allow only operations that are legal based on what is known up to
that point.  For example, it will not allow an \texttt{assign} of a
statement that cannot be unified with the unknown proof step being
assigned.

{\em Important:}
The Proof Assistant is
{\em not} a tool to help you discover proofs.  It is just a tool to help
you add them to the database.  For a tutorial read
Section~\ref{frstprf}.
To practice using the Proof Assistant, you may
want to \texttt{prove} an existing theorem, then delete all steps with
\texttt{delete all}, then re-create it with the Proof Assistant while
looking at its proof display (before deletion).
You might want to figure out your first few proofs completely
and write them down by hand, before using the Proof Assistant, though
not everyone finds that effective.

{\em Important:}
The \texttt{undo} command if very helpful when entering a proof, because
it allows you to undo a previously-entered step.
In addition, we suggest that you
keep track of your work with a log file (\texttt{open
log}) and save it frequently (\texttt{save new{\char`\_}proof},
\texttt{write source}).
You can use \texttt{delete} to reverse an \texttt{assign}.
You can also do \texttt{delete floating{\char`\_}hypotheses}, then
\texttt{initialize all}, then \texttt{unify all /interactive} to
reinitialize bad unifications made accidentally or by bad
\texttt{assign}s.  You cannot reverse a \texttt{delete} except by
a relevant \texttt{undo} or using
\texttt{exit /force} then reentering the Proof Assistant to recover from
the last \texttt{save new{\char`\_}proof}.

The following commands are available in the Proof Assistant (at the
\texttt{MM-PA>} prompt) to help you create your proof.  See the
individual commands for more detail.

\begin{itemize}
\item[]
    \texttt{show new{\char`\_}proof} [\texttt{/all},...] - Displays the
        proof in progress.  You will use this command a lot; see \texttt{help
        show new{\char`\_}proof} to become familiar with its qualifiers.  The
        qualifiers \texttt{/unknown} and \texttt{/not{\char`\_}unified} are
        useful for seeing the work remaining to be done.  The combination
        \texttt{/all/unknown} is useful for identifying dummy variables that must be
        assigned, or attempts to use illegal syntax, when \texttt{improve all}
        is unable to complete the syntax constructions.  Unknown variables are
        shown as \texttt{\$1}, \texttt{\$2},...
\item[]
    \texttt{assign} {\em step} {\em label} - Assigns an unknown {\em step}
        number with the statement
        specified by {\em label}.
\item[]
    \texttt{let variable} {\em variable}
        \texttt{= "}{\em symbol sequence}\texttt{"}
          - Forces a symbol
        sequence to replace an unknown variable (such as \texttt{\$1}) in a proof.
        It is useful
        for helping difficult unifications, and it is necessary when you have
        dummy variables that eventually must be assigned a name.
\item[]
    \texttt{let step} {\em step} \texttt{= "}{\em symbol sequence}\texttt{"} -
          Forces a symbol sequence
        to replace the contents of a proof step, provided it can be
        unified with the existing step contents.  (I rarely use this.)
\item[]
    \texttt{unify step} {\em step} (or \texttt{unify all}) - Unifies
        the source and target of
        a step.  If you specify a specific step, you will be prompted
        to select among the unifications that are possible.  If you
        specify \texttt{all}, all steps with unique unifications, but only
        those steps, will be
        unified.  \texttt{unify all /interactive} goes through all non-unified
        steps.
\item[]
    \texttt{initialize} {\em step} (or \texttt{all}) - De-unifies the target and source of
        a step (or all steps), as well as the hypotheses of the source,
        and makes all variables in the source unknown.  Useful to recover from
        an \texttt{assign} or \texttt{let} mistake that
        resulted in incorrect unifications.
\item[]
    \texttt{delete} {\em step} (or \texttt{all} or \texttt{floating{\char`\_}hypotheses}) -
        Deletes the specified
        step(s).  \texttt{delete floating{\char`\_}hypotheses}, then \texttt{initialize all}, then
        \texttt{unify all /interactive} is useful for recovering from mistakes
        where incorrect unifications assigned wrong math symbol strings to
        variables.
\item[]
    \texttt{improve} {\em step} (or \texttt{all}) -
      Automatically creates a proof for steps (with no unknown variables)
      whose proof requires no statements with \texttt{\$e} hypotheses.  Useful
      for filling in proofs of \texttt{\$f} hypotheses.  The \texttt{/depth}
      qualifier will also try statements whose \texttt{\$e} hypotheses contain
      no new variables.  {\em Warning:} Save your work (with \texttt{save
      new{\char`\_}proof} then \texttt{write source}) before using
      \texttt{/depth = 2} or greater, since the search time grows
      exponentially and may never terminate in a reasonable time, and you
      cannot interrupt the search.  I have found that it is rare for
      \texttt{/depth = 3} or greater to be useful.
 \item[]
    \texttt{save new{\char`\_}proof} - Saves the proof in progress in the program's
        internal database buffer.  To save it permanently into the database file,
        use \texttt{write source} after
        \texttt{save new{\char`\_}proof}.  To revert to the last
        \texttt{save new{\char`\_}proof},
        \texttt{exit /force} from the Proof Assistant then re-enter the Proof
        Assistant.
 \item[]
    \texttt{match step} {\em step} (or \texttt{match all}) - Shows what
        statements are
        possibilities for the \texttt{assign} statement. (This command
        is not very
        useful in its present form and hopefully will be improved
        eventually.  In the meantime, use the \texttt{search} statement for
        candidates matching specific math token combinations.)
 \item[]
 \texttt{minimize{\char`\_}with}\index{\texttt{minimize{\char`\_}with} command}
% 3/10/07 Note: line-breaking the above results in duplicate index entries
     - After a proof is complete, this command will attempt
        to match other database theorems to the proof to see if the proof
        size can be reduced as a result.  See \texttt{help
        minimize{\char`\_}with} in the
        Metamath program for its usage.
 \item[]
 \texttt{undo}\index{\texttt{undo} command}
    - Undo the effect of a proof-changing command (all but the
      \texttt{show} and \texttt{save} commands above).
 \item[]
 \texttt{redo}\index{\texttt{redo} command}
    - Reverse the previous \texttt{undo}.
\end{itemize}

The following commands set parameters that may be relevant to your proof.
Consult the individual \texttt{help set}... commands.
\begin{itemize}
   \item[] \texttt{set unification{\char`\_}timeout}
 \item[]
    \texttt{set search{\char`\_}limit}
  \item[]
    \texttt{set empty{\char`\_}substitution} - note that default is \texttt{off}
\end{itemize}

Type \texttt{exit} to exit the \texttt{MM-PA>}
 prompt and get back to the \texttt{MM>} prompt.
Another \texttt{exit} will then get you out of Metamath.



\subsection{\texttt{prove} Command}\index{\texttt{prove} command}
Syntax:  \texttt{prove} {\em label}

This command will enter the Proof Assistant\index{Proof Assistant}, which will
allow you to create or edit the proof of the specified statement.
The command-line prompt will change from \texttt{MM>} to \texttt{MM-PA>}.

Note:  In the present version (0.177) of
Metamath\index{Metamath!limitations of version 0.177}, the Proof
Assistant does not verify that \texttt{\$d}\index{\texttt{\$d}
statement} restrictions are met as a proof is being built.  After you
have completed a proof, you should type \texttt{save new{\char`\_}proof}
followed by \texttt{verify proof} {\em label} (where {\em label} is the
statement you are proving with the \texttt{prove} command) to verify the
\texttt{\$d} restrictions.

See also: \texttt{exit}

\subsection{\texttt{set unification\_timeout} Command}\index{\texttt{set
unification{\char`\_}timeout} command}
Syntax:  \texttt{set unification{\char`\_}timeout} {\em number}

(This command is available outside the Proof Assistant but affects the
Proof Assistant\index{Proof Assistant} only.)

Sometimes the Proof Assistant will inform you that a unification
time-out occurred.  This may happen when you try to \texttt{unify}
formulas with many temporary variables\index{temporary variable}
(\texttt{\$1}, \texttt{\$2}, etc.), since the time to compute all possible
unifications may grow exponentially with the number of variables.  If
you want Metamath to try harder (and you're willing to wait longer) you
may increase this parameter.  \texttt{show settings} will show you the
current value.

\subsection{\texttt{set empty\_substitution} Command}\index{\texttt{set
empty{\char`\_}substitution} command}
% These long names can't break well in narrow mode, and even "sloppy"
% is not enough. Work around this by not demanding justification.
\begin{flushleft}
Syntax:  \texttt{set empty{\char`\_}substitution on} or \texttt{set
empty{\char`\_}substitution off}
\end{flushleft}

(This command is available outside the Proof Assistant but affects the
Proof Assistant\index{Proof Assistant} only.)

The Metamath language allows variables to be
substituted\index{substitution!variable}\index{variable substitution}
with empty symbol sequences\index{empty substitution}.  However, in many
formal systems\index{formal system} this will never happen in a valid
proof.  Allowing for this possibility increases the likelihood of
ambiguous unifications\index{ambiguous
unification}\index{unification!ambiguous} during proof creation.
The default is that
empty substitutions are not allowed; for formal systems requiring them,
you must \texttt{set empty{\char`\_}substitution on}.
(An example where this must be \texttt{on}
would be a system that implements a Deduction Rule and in
which deductions from empty assumption lists would be permissible.  The
MIU-system\index{MIU-system} described in Appendix~\ref{MIU} is another
example.)
Note that empty substitutions are
always permissible in proof verification (VERIFY PROOF...) outside the
Proof Assistant.  (See the MIU system in the Metamath book for an example
of a system needing empty substitutions; another example would be a
system that implements a Deduction Rule and in which deductions from
empty assumption lists would be permissible.)

It is better to leave this \texttt{off} when working with \texttt{set.mm}.
Note that this command does not affect the way proofs are verified with
the \texttt{verify proof} command.  Outside of the Proof Assistant,
substitution of empty sequences for math symbols is always allowed.

\subsection{\texttt{set search\_limit} Command}\index{\texttt{set
search{\char`\_}limit} command} Syntax:  \texttt{set search{\char`\_}limit} {\em
number}

(This command is available outside the Proof Assistant but affects the
Proof Assistant\index{Proof Assistant} only.)

This command sets a parameter that determines when the \texttt{improve} command
in Proof Assistant mode gives up.  If you want \texttt{improve} to search harder,
you may increase it.  The \texttt{show settings} command tells you its current
value.


\subsection{\texttt{show new\_proof} Command}\index{\texttt{show
new{\char`\_}proof} command}
Syntax:  \texttt{show new{\char`\_}proof} [{\em
qualifiers} (see below)]

This command (available only in Proof Assistant mode) displays the proof
in progress.  It is identical to the \texttt{show proof} command, except that
there is no statement argument (since it is the statement being proved) and
the following qualifiers are not available:

    \texttt{/statement{\char`\_}summary}

    \texttt{/detailed{\char`\_}step}

Also, the following additional qualifiers are available:

    \texttt{/unknown} - Shows only steps that have no statement assigned.

    \texttt{/not{\char`\_}unified} - Shows only steps that have not been unified.

Note that \texttt{/essential}, \texttt{/depth}, \texttt{/unknown}, and
\texttt{/not{\char`\_}unified} may be
used in any combination; each of them effectively filters out additional
steps from the proof display.

See also:  \texttt{show proof}






\subsection{\texttt{assign} Command}\index{\texttt{assign} command}
Syntax:   \texttt{assign} {\em step} {\em label} [\texttt{/no{\char`\_}unify}]

   and:   \texttt{assign first} {\em label}

   and:   \texttt{assign last} {\em label}


This command, available in the Proof Assistant only, assigns an unknown
step (one with \texttt{?} in the \texttt{show new{\char`\_}proof}
listing) with the statement specified by {\em label}.  The assignment
will not be allowed if the statement cannot be unified with the step.

If \texttt{last} is specified instead of {\em step} number, the last
step that is shown by \texttt{show new{\char`\_}proof /unknown} will be
used.  This can be useful for building a proof with a command file (see
\texttt{help submit}).  It also makes building proofs faster when you know
the assignment for the last step.

If \texttt{first} is specified instead of {\em step} number, the first
step that is shown by \texttt{show new{\char`\_}proof /unknown} will be
used.

If {\em step} is zero or negative, the -{\em step}th from last unknown
step, as shown by \texttt{show new{\char`\_}proof /unknown}, will be
used.  \texttt{assign -1} {\em label} will assign the penultimate
unknown step, \texttt{assign -2} {\em label} the antepenultimate, and
\texttt{assign 0} {\em label} is the same as \texttt{assign last} {\em
label}.

Optional command qualifier:

    \texttt{/no{\char`\_}unify} - do not prompt user to select a unification if there is
        more than one possibility.  This is useful for noninteractive
        command files.  Later, the user can \texttt{unify all /interactive}.
        (The assignment will still be automatically unified if there is only
        one possibility and will be refused if unification is not possible.)



\subsection{\texttt{match} Command}\index{\texttt{match} command}
Syntax:  \texttt{match step} {\em step} [\texttt{/max{\char`\_}essential{\char`\_}hyp}
{\em number}]

    and:  \texttt{match all} [\texttt{/essential}]
          [\texttt{/max{\char`\_}essential{\char`\_}hyp} {\em number}]

This command, available in the Proof Assistant only, shows what
statements can be unified with the specified step(s).  {\em Note:} In
its current form, this command is not very useful because of the large
number of matches it reports.
It may be enhanced in the future.  In the meantime, the \texttt{search}
command can often provide finer control over locating theorems of interest.

Optional command qualifiers:

    \texttt{/max{\char`\_}essential{\char`\_}hyp} {\em number} - filters out
        of the list any statements
        with more than the specified number of
        \texttt{\$e}\index{\texttt{\$e} statement} hypotheses.

    \texttt{/essential{\char`\_}only} - in the \texttt{match all} statement, only
        the steps that
        would be listed in the \texttt{show new{\char`\_}proof /essential} display are
        matched.



\subsection{\texttt{let} Command}\index{\texttt{let} command}
Syntax: \texttt{let variable} {\em variable} = \verb/"/{\em symbol-sequence}\verb/"/

  and: \texttt{let step} {\em step} = \verb/"/{\em symbol-sequence}\verb/"/

These commands, available in the Proof Assistant\index{Proof Assistant}
only, assign a temporary variable\index{temporary variable} or unknown
step with a specific symbol sequence.  They are useful in the middle of
creating a proof, when you know what should be in the proof step but the
unification algorithm doesn't yet have enough information to completely
specify the temporary variables.  A ``temporary variable'' is one that
has the form \texttt{\$}{\em nn} in the proof display, such as
\texttt{\$1}, \texttt{\$2}, etc.  The {\em symbol-sequence} may contain
other unknown variables if desired.  Examples:

    \verb/let variable $32 = "A = B"/

    \verb/let variable $32 = "A = $35"/

    \verb/let step 10 = '|- x = x'/

    \verb/let step -2 = "|- ( $7 = ph )"/

Any symbol sequence will be accepted for the \texttt{let variable}
command.  Only those symbol sequences that can be unified with the step
will be accepted for \texttt{let step}.

The \texttt{let} commands ``zap'' the proof with information that can
only be verified when the proof is built up further.  If you make an
error, the command sequence \texttt{delete
floating{\char`\_}hypotheses}, \texttt{initialize all}, and
\texttt{unify all /interactive} will undo a bad \texttt{let} assignment.

If {\em step} is zero or negative, the -{\em step}th from last unknown
step, as shown by \texttt{show new{\char`\_}proof /unknown}, will be
used.  The command \texttt{let step 0} = \verb/"/{\em
symbol-sequence}\verb/"/ will use the last unknown step, \texttt{let
step -1} = \verb/"/{\em symbol-sequence}\verb/"/ the penultimate, etc.
If {\em step} is positive, \texttt{let step} may be used to assign known
(in the sense of having previously been assigned a label with
\texttt{assign}) as well as unknown steps.

Either single or double quotes can surround the {\em symbol-sequence} as
long as they are different from any quotes inside a {\em
symbol-sequence}.  If {\em symbol-sequence} contains both kinds of
quotes, see the instructions at the end of \texttt{help let} in the
Metamath program.


\subsection{\texttt{unify} Command}\index{\texttt{unify} command}
Syntax:  \texttt{unify step} {\em step}

      and:   \texttt{unify all} [\texttt{/interactive}]

These commands, available in the Proof Assistant only, unify the source
and target of the specified step(s). If you specify a specific step, you
will be prompted to select among the unifications that are possible.  If
you specify \texttt{all}, only those steps with unique unifications will be
unified.

Optional command qualifier for \texttt{unify all}:

    \texttt{/interactive} - You will be prompted to select among the
        unifications
        that are possible for any steps that do not have unique
        unifications.  (Otherwise \texttt{unify all} will bypass these.)

See also \texttt{set unification{\char`\_}timeout}.  The default is
100000, but increasing it to 1000000 can help difficult cases.  Manually
assigning some or all of the unknown variables with the \texttt{let
variable} command also helps difficult cases.



\subsection{\texttt{initialize} Command}\index{\texttt{initialize} command}
Syntax:  \texttt{initialize step} {\em step}

    and: \texttt{initialize all}

These commands, available in the Proof Assistant\index{Proof Assistant}
only, ``de-unify'' the target and source of a step (or all steps), as
well as the hypotheses of the source, and makes all variables in the
source and the source's hypotheses unknown.  This command is useful to
help recover from incorrect unifications that resulted from an incorrect
\texttt{assign}, \texttt{let}, or unification choice.  Part or all of
the command sequence \texttt{delete floating{\char`\_}hypotheses},
\texttt{initialize all}, and \texttt{unify all /interactive} will recover
from incorrect unifications.

See also:  \texttt{unify} and \texttt{delete}



\subsection{\texttt{delete} Command}\index{\texttt{delete} command}
Syntax:  \texttt{delete step} {\em step}

   and:      \texttt{delete all} -- {\em Warning: dangerous!}

   and:      \texttt{delete floating{\char`\_}hypotheses}

These commands are available in the Proof Assistant only.  The
\texttt{delete step} command deletes the proof tree section that
branches off of the specified step and makes the step become unknown.
\texttt{delete all} is equivalent to \texttt{delete step} {\em step}
where {\em step} is the last step in the proof (i.e.\ the beginning of
the proof tree).

In most cases the \texttt{undo} command is the best way to undo
a previous step.
An alternative is to salvage your last \texttt{save
new{\char`\_}proof} by exiting and reentering the Proof Assistant.
For this to work, keep a log file open to record your work
and to do \texttt{save new{\char`\_}proof} frequently, especially before
\texttt{delete}.

\texttt{delete floating{\char`\_}hypotheses} will delete all sections of
the proof that branch off of \texttt{\$f}\index{\texttt{\$f} statement}
statements.  It is sometimes useful to do this before an
\texttt{initialize} command to recover from an error.  Note that once a
proof step with a \texttt{\$f} hypothesis as the target is completely
known, the \texttt{improve} command can usually fill in the proof for
that step.  Unlike the deletion of logical steps, \texttt{delete
floating{\char`\_}hypotheses} is a relatively safe command that is
usually easy to recover from.



\subsection{\texttt{improve} Command}\index{\texttt{improve} command}
\label{improve}
Syntax:  \texttt{improve} {\em step} [\texttt{/depth} {\em number}]
                                               [\texttt{/no{\char`\_}distinct}]

   and:   \texttt{improve first} [\texttt{/depth} {\em number}]
                                              [\texttt{/no{\char`\_}distinct}]

   and:   \texttt{improve last} [\texttt{/depth} {\em number}]
                                              [\texttt{/no{\char`\_}distinct}]

   and:   \texttt{improve all} [\texttt{/depth} {\em number}]
                                              [\texttt{/no{\char`\_}distinct}]

These commands, available in the Proof Assistant\index{Proof Assistant}
only, try to find proofs automatically for unknown steps whose symbol
sequences are completely known.  They are primarily useful for filling in
proofs of \texttt{\$f}\index{\texttt{\$f} statement} hypotheses.  The
search will be restricted to statements having no
\texttt{\$e}\index{\texttt{\$e} statement} hypotheses.

\begin{sloppypar} % narrow
Note:  If memory is limited, \texttt{improve all} on a large proof may
overflow memory.  If you use \texttt{set unification{\char`\_}timeout 1}
before \texttt{improve all}, there will usually be sufficient
improvement to easily recover and completely \texttt{improve} the proof
later on a larger computer.  Warning:  Once memory has overflowed, there
is no recovery.  If in doubt, save the intermediate proof (\texttt{save
new{\char`\_}proof} then \texttt{write source}) before \texttt{improve
all}.
\end{sloppypar}

If \texttt{last} is specified instead of {\em step} number, the last
step that is shown by \texttt{show new{\char`\_}proof /unknown} will be
used.

If \texttt{first} is specified instead of {\em step} number, the first
step that is shown by \texttt{show new{\char`\_}proof /unknown} will be
used.

If {\em step} is zero or negative, the -{\em step}th from last unknown
step, as shown by \texttt{show new{\char`\_}proof /unknown}, will be
used.  \texttt{improve -1} will use the penultimate
unknown step, \texttt{improve -2} {\em label} the antepenultimate, and
\texttt{improve 0} is the same as \texttt{improve last}.

Optional command qualifier:

    \texttt{/depth} {\em number} - This qualifier will cause the search
        to include
        statements with \texttt{\$e} hypotheses (but no new variables in
        the \texttt{\$e}
        hypotheses), provided that the backtracking has not exceeded the
        specified depth. {\em Warning:}  Try \texttt{/depth 1},
        then \texttt{2}, then \texttt{3}, etc.
        in sequence because of possible exponential blowups.  Save your
        work before trying \texttt{/depth} greater than \texttt{1}!

    \texttt{/no{\char`\_}distinct} - Skip trial statements that have
        \texttt{\$d}\index{\texttt{\$d} statement} requirements.
        This qualifier will prevent assignments that might violate \texttt{\$d}
        requirements but it also could miss possible legal assignments.

See also: \texttt{set search{\char`\_}limit}

\subsection{\texttt{save new\_proof} Command}\index{\texttt{save
new{\char`\_}proof} command}
Syntax:  \texttt{save new{\char`\_}proof} {\em label} [\texttt{/normal}]
   [\texttt{/compressed}]

The \texttt{save new{\char`\_}proof} command is available in the Proof
Assistant only.  It saves the proof in progress in the source
buffer\index{source buffer}.  \texttt{save new{\char`\_}proof} may be
used to save a completed proof, or it may be used to save a proof in
progress in order to work on it later.  If an incomplete proof is saved,
any user assignments with \texttt{let step} or \texttt{let variable}
will be lost, as will any ambiguous unifications\index{ambiguous
unification}\index{unification!ambiguous} that were resolved manually.
To help make recovery easier, it can be helpful to \texttt{improve all}
before \texttt{save new{\char`\_}proof} so that the incomplete proof
will have as much information as possible.

Note that the proof will not be permanently saved until a \texttt{write
source} command is issued.

Optional command qualifiers:

    \texttt{/normal} - The proof is saved in the normal format (i.e., as a
        sequence of labels, which is the defined format of the basic Metamath
        language).\index{basic language}  This is the default format that
        is used if a qualifier is omitted.

    \texttt{/compressed} - The proof is saved in the compressed format, which
        reduces storage requirements for a database.
        (See Appendix~\ref{compressed}.)


\section{Creating \LaTeX\ Output}\label{texout}\index{latex@{\LaTeX}}

You can generate \LaTeX\ output given the
information in a database.
The database must already include the necessary typesetting information
(see section \ref{tcomment} for how to provide this information).

The \texttt{show statement} and \texttt{show proof} commands each have a
special \texttt{/tex} command qualifier that produces \LaTeX\ output.
(The \texttt{show statement} command also has the
\texttt{/simple{\char`\_}tex} qualifier for output that is easier to
edit by hand.)  Before you can use them, you must open a \LaTeX\ file to
which to send their output.  A typical complete session will use this
sequence of Metamath commands:

\begin{verbatim}
read set.mm
open tex example.tex
show statement a1i /tex
show proof a1i /all/lemmon/renumber/tex
show statement uneq2 /tex
show proof uneq2 /all/lemmon/renumber/tex
close tex
\end{verbatim}

See Section~\ref{mathcomments} for information on comment markup and
Appendix~\ref{ASCII} for information on how math symbol translation is
specified.

To format and print the \LaTeX\ source, you will need the \LaTeX\
program, which is standard on most Linux installations and available for
Windows.  On Linux, in order to create a {\sc pdf} file, you will
typically type at the shell prompt
\begin{verbatim}
$ pdflatex example.tex
\end{verbatim}

\subsection{\texttt{open tex} Command}\index{\texttt{open tex} command}
Syntax:  \texttt{open tex} {\em file-name} [\texttt{/no{\char`\_}header}]

This command opens a file for writing \LaTeX\
source\index{latex@{\LaTeX}} and writes a \LaTeX\ header to the file.
\LaTeX\ source can be written with the \texttt{show proof}, \texttt{show
new{\char`\_}proof}, and \texttt{show statement} commands using the
\texttt{/tex} qualifier.

The mapping to \LaTeX\ symbols is defined in a special comment
containing a \texttt{\$t} token, described in Appendix~\ref{ASCII}.

There is an optional command qualifier:

    \texttt{/no{\char`\_}header} - This qualifier prevents a standard
        \LaTeX\ header and trailer
        from being included with the output \LaTeX\ code.


\subsection{\texttt{close tex} Command}\index{\texttt{close tex} command}
Syntax:  \texttt{close tex}

This command writes a trailer to any \LaTeX\ file\index{latex@{\LaTeX}}
that was opened with \texttt{open tex} (unless
\texttt{/no{\char`\_}header} was used with \texttt{open tex}) and closes
the \LaTeX\ file.


\section{Creating {\sc HTML} Output}\label{htmlout}

You can generate {\sc html} web pages given the
information in a database.
The database must already include the necessary typesetting information
(see section \ref{tcomment} for how to provide this information).
The ability to produce {\sc html} web pages was added in Metamath version
0.07.30.

To create an {\sc html} output file(s) for \texttt{\$a} or \texttt{\$p}
statement(s), use
\begin{quote}
    \texttt{show statement} {\em label-match} \texttt{/html}
\end{quote}
The output file will be named {\em label-match}\texttt{.html}
for each match.  When {\em
label-match} has wildcard (\texttt{*}) characters, all statements with
matching labels will have {\sc html} files produced for them.  Also,
when {\em label-match} has a wildcard (\texttt{*}) character, two additional
files, \texttt{mmdefinitions.html} and \texttt{mmascii.html} will be
produced.  To produce {\em only} these two additional files, you can use
\texttt{?*}, which will not match any statement label, in place of {\em
label-match}.

There are three other qualifiers for \texttt{show statement} that also
generate {\sc HTML} code.  These are \texttt{/alt{\char`\_}html},
\texttt{/brief{\char`\_}html}, and
\texttt{/brief{\char`\_}alt{\char`\_}html}, and are described in the
next section.

The command
\begin{quote}
    \texttt{show statement} {\em label-match} \texttt{/alt{\char`\_}html}
\end{quote}
does the same as \texttt{show statement} {\em label-match} \texttt{/html},
except that the {\sc html} code for the symbols is taken from
\texttt{althtmldef} statements instead of \texttt{htmldef} statements in
the \texttt{\$t} comment.

The command
\begin{verbatim}
show statement * /brief_html
\end{verbatim}
invokes a special mode that just produces definition and theorem lists
accompanied by their symbol strings, in a format suitable for copying and
pasting into another web page (such as the tutorial pages on the
Metamath web site).

Finally, the command
\begin{verbatim}
show statement * /brief_alt_html
\end{verbatim}
does the same as \texttt{show statement * / brief{\char`\_}html}
for the alternate {\sc html}
symbol representation.

A statement's comment can include a special notation that provides a
certain amount of control over the {\sc HTML} version of the comment.  See
Section~\ref{mathcomments} (p.~\pageref{mathcomments}) for the comment
markup features.

The \texttt{write theorem{\char`\_}list} and \texttt{write bibliography}
commands, which are described below, provide as a side effect complete
error checking for all of the features described in this
section.\index{error checking}

\subsection{\texttt{write theorem\_list}
Command}\index{\texttt{write theorem{\char`\_}list} command}
Syntax:  \texttt{write theorem{\char`\_}list}
[\texttt{/theorems{\char`\_}per{\char`\_}page} {\em number}]

This command writes a list of all of the \texttt{\$a} and \texttt{\$p}
statements in the database into a web page file
 called \texttt{mmtheorems.html}.
When additional files are needed, they are called
\texttt{mmtheorems2.html}, \texttt{mmtheorems3.html}, etc.

Optional command qualifier:

    \texttt{/theorems{\char`\_}per{\char`\_}page} {\em number} -
 This qualifier specifies the number of statements to
        write per web page.  The default is 100.

{\em Note:} In version 0.177\index{Metamath!limitations of version
0.177} of Metamath, the ``Nearby theorems'' links on the individual
web pages presuppose 100 theorems per page when linking to the theorem
list pages.  Therefore the \texttt{/theorems{\char`\_}per{\char`\_}page}
qualifier, if it specifies a number other than 100, will cause the
individual web pages to be out of sync and should not be used to
generate the main theorem list for the web site.  This may be
fixed in a future version.


\subsection{\texttt{write bibliography}\label{wrbib}
Command}\index{\texttt{write bibliography} command}
Syntax:  \texttt{write bibliography} {\em filename}

This command reads an existing {\sc html} bibliographic cross-reference
file, normally called \texttt{mmbiblio.html}, and updates it per the
bibliographic links in the database comments.  The file is updated
between the {\sc html} comment lines \texttt{<!--
{\char`\#}START{\char`\#} -->} and \texttt{<!-- {\char`\#}END{\char`\#}
-->}.  The original input file is renamed to {\em
filename}\texttt{{\char`\~}1}.

A bibliographic reference is indicated with the reference name
in brackets, such as  \texttt{Theorem 3.1 of
[Monk] p.\ 22}.
See Section~\ref{htmlout} (p.~\pageref{htmlout}) for
syntax details.


\subsection{\texttt{write recent\_additions}
Command}\index{\texttt{write recent{\char`\_}additions} command}
Syntax:  \texttt{write recent{\char`\_}additions} {\em filename}
[\texttt{/limit} {\em number}]

This command reads an existing ``Recent Additions'' {\sc html} file,
normally called \texttt{mmrecent.html}, and updates it with the
descriptions of the most recently added theorems to the database.
 The file is updated between
the {\sc html} comment lines \texttt{<!-- {\char`\#}START{\char`\#} -->}
and \texttt{<!-- {\char`\#}END{\char`\#} -->}.  The original input file
is renamed to {\em filename}\texttt{{\char`\~}1}.

Optional command qualifier:

    \texttt{/limit} {\em number} -
 This qualifier specifies the number of most recent theorems to
   write to the output file.  The default is 100.


\section{Text File Utilities}

\subsection{\texttt{tools} Command}\index{\texttt{tools} command}
Syntax:  \texttt{tools}

This command invokes an easy-to-use, general purpose utility for
manipulating the contents of {\sc ascii} text files.  Upon typing
\texttt{tools}, the command-line prompt will change to \texttt{TOOLS>}
until you type \texttt{exit}.  The \texttt{tools} commands can be used
to perform simple, global edits on an input/output file,
such as making a character string substitution on each line, adding a
string to each line, and so on.  A typical use of this utility is
to build a \texttt{submit} input file to perform a common operation on a
list of statements obtained from \texttt{show label} or \texttt{show
usage}.

The actions of most of the \texttt{tools} commands can also be
performed with equivalent (and more powerful) Unix shell commands, and
some users may find those more efficient.  But for Windows users or
users not comfortable with Unix, \texttt{tools} provides an
easy-to-learn alternative that is adequate for most of the
script-building tasks needed to use the Metamath program effectively.

\subsection{\texttt{help} Command (in \texttt{tools})}
Syntax:  \texttt{help}

The \texttt{help} command lists the commands available in the
\texttt{tools} utility, along with a brief description.  Each command,
in turn, has its own help, such as \texttt{help add}.  As with
Metamath's \texttt{MM>} prompt, a complete command can be entered at
once, or just the command word can be typed, causing the program to
prompt for each argument.

\vskip 1ex
\noindent Line-by-line editing commands:

  \texttt{add} - Add a specified string to each line in a file.

  \texttt{clean} - Trim spaces and tabs on each line in a file; convert
         characters.

  \texttt{delete} - Delete a section of each line in a file.

  \texttt{insert} - Insert a string at a specified column in each line of
        a file.

  \texttt{substitute} - Make a simple substitution on each line of the file.

  \texttt{tag} - Like \texttt{add}, but restricted to a range of lines.

  \texttt{swap} - Swap the two halves of each line in a file.

\vskip 1ex
\noindent Other file-processing commands:

  \texttt{break} - Break up (tokenize) a file into a list of tokens (one per
        line).

  \texttt{build} - Build a file with multiple tokens per line from a list.

  \texttt{count} - Count the occurrences in a file of a specified string.

  \texttt{number} - Create a list of numbers.

  \texttt{parallel} - Put two files in parallel.

  \texttt{reverse} - Reverse the order of the lines in a file.

  \texttt{right} - Right-justify lines in a file (useful before sorting
         numbers).

%  \texttt{tag} - Tag edit updates in a program for revision control.

  \texttt{sort} - Sort the lines in a file with key starting at
         specified string.

  \texttt{match} - Extract lines containing (or not) a specified string.

  \texttt{unduplicate} - Eliminate duplicate occurrences of lines in a file.

  \texttt{duplicate} - Extract first occurrence of any line occurring
         more than

   \ \ \    once in a file, discarding lines occurring exactly once.

  \texttt{unique} - Extract lines occurring exactly once in a file.

  \texttt{type} (10 lines) - Display the first few lines in a file.
                  Similar to Unix \texttt{head}.

  \texttt{copy} - Similar to Unix \texttt{cat} but safe (same input
         and output file allowed).

  \texttt{submit} - Run a script containing \texttt{tools} commands.

\vskip 1ex

\noindent Note:
  \texttt{unduplicate}, \texttt{duplicate}, and \texttt{unique} also
 sort the lines as a side effect.


\subsection{Using \texttt{tools} to Build Metamath \texttt{submit}
Scripts}

The \texttt{break} command is typically used to break up a series of
statement labels, such as the output of Metamath's \texttt{show usage},
into one label per line.  The other \texttt{tools} commands can then be
used to add strings before and after each statement label to specify
commands to be performed on the statement.  The \texttt{parallel}
command is useful when a statement label must be mentioned more than
once on a line.

Very often a \texttt{submit} script for Metamath will require multiple
command lines for each statement being processed.  For example, you may
want to enter the Proof Assistant, \texttt{minimize{\char`\_}with} your
latest theorem, \texttt{save} the new proof, and \texttt{exit} the Proof
Assistant.  To accomplish this, you can build a file with these four
commands for each statement on a single line, separating each command
with a designated character such as \texttt{@}.  Then at the end you can
\texttt{substitute} each \texttt{@} with \texttt{{\char`\\}n} to break
up the lines into individual command lines (see \texttt{help
substitute}).


\subsection{Example of a \texttt{tools} Session}

To give you a quick feel for the \texttt{tools} utility, we show a
simple session where we create a file \texttt{n.txt} with 3 lines, add
strings before and after each line, and display the lines on the screen.
You can experiment with the various commands to gain experience with the
\texttt{tools} utility.

\begin{verbatim}
MM> tools
Entering the Text Tools utilities.
Type HELP for help, EXIT to exit.
TOOLS> number
Output file <n.tmp>? n.txt
First number <1>?
Last number <10>? 3
Increment <1>?
TOOLS> add
Input/output file? n.txt
String to add to beginning of each line <>? This is line
String to add to end of each line <>? .
The file n.txt has 3 lines; 3 were changed.
First change is on line 1:
This is line 1.
TOOLS> type n.txt
This is line 1.
This is line 2.
This is line 3.
TOOLS> exit
Exiting the Text Tools.
Type EXIT again to exit Metamath.
MM>
\end{verbatim}



\appendix
\chapter{Sample Representations}
\label{ASCII}

This Appendix provides a sample of {\sc ASCII} representations,
their corresponding traditional mathematical symbols,
and a discussion of their meanings
in the \texttt{set.mm} database.
These are provided in order of appearance.
This is only a partial list, and new definitions are routinely added.
A complete list is available at \url{http://metamath.org}.

These {\sc ASCII} representations, along
with information on how to display them,
are defined in the \texttt{set.mm} database file inside
a special comment called a \texttt{\$t} {\em
comment}\index{\texttt{\$t} comment} or {\em typesetting
comment.}\index{typesetting comment}
A typesetting comment
is indicated by the appearance of the
two-character string \texttt{\$t} at the beginning of the comment.
For more information,
see Section~\ref{tcomment}, p.~\pageref{tcomment}.

In the following table the ``{\sc ASCII}'' column shows the {\sc ASCII}
representation,
``Symbol'' shows the mathematical symbolic display
that corresponds to that {\sc ASCII} representation, ``Labels'' shows
the key label(s) that define the representation, and
``Description'' provides a description about the symbol.
As usual, ``iff'' is short for ``if and only if.''\index{iff}
In most cases the ``{\sc ASCII}'' column only shows
the key token, but it will sometimes show a sequence of tokens
if that is necessary for clarity.

{\setlength{\extrarowsep}{4pt} % Keep rows from being too close together
\begin{longtabu}   { @{} c c l X }
\textbf{ASCII} & \textbf{Symbol} & \textbf{Labels} & \textbf{Description} \\
\endhead
\texttt{|-} & $\vdash$ & &
  ``It is provable that...'' \\
\texttt{ph} & $\varphi$ & \texttt{wph} &
  The wff (boolean) variable phi,
  conventionally the first wff variable. \\
\texttt{ps} & $\psi$ & \texttt{wps} &
  The wff (boolean) variable psi,
  conventionally the second wff variable. \\
\texttt{ch} & $\chi$ & \texttt{wch} &
  The wff (boolean) variable chi,
  conventionally the third wff variable. \\
\texttt{-.} & $\lnot$ & \texttt{wn} &
  Logical not. E.g., if $\varphi$ is true, then $\lnot \varphi$ is false. \\
\texttt{->} & $\rightarrow$ & \texttt{wi} &
  Implies, also known as material implication.
  In classical logic the expression $\varphi \rightarrow \psi$ is true
  if either $\varphi$ is false or $\psi$ is true (or both), that is,
  $\varphi \rightarrow \psi$ has the same meaning as
  $\lnot \varphi \lor \psi$ (as proven in theorem \texttt{imor}). \\
\texttt{<->} & $\leftrightarrow$ &
  \hyperref[df-bi]{\texttt{df-bi}} &
  Biconditional (aka is-equals for boolean values).
  $\varphi \leftrightarrow \psi$ is true iff
  $\varphi$ and $\psi$ have the same value. \\
\texttt{\char`\\/} & $\lor$ &
  \makecell[tl]{{\hyperref[df-or]{\texttt{df-or}}}, \\
	         \hyperref[df-3or]{\texttt{df-3or}}} &
  Disjunction (logical ``or''). $\varphi \lor \psi$ is true iff
  $\varphi$, $\psi$, or both are true. \\
\texttt{/\char`\\} & $\land$ &
  \makecell[tl]{{\hyperref[df-an]{\texttt{df-an}}}, \\
                 \hyperref[df-3an]{\texttt{df-3an}}} &
  Conjunction (logical ``and''). $\varphi \land \psi$ is true iff
  both $\varphi$ and $\psi$ are true. \\
\texttt{A.} & $\forall$ &
  \texttt{wal} &
  For all; the wff $\forall x \varphi$ is true iff
  $\varphi$ is true for all values of $x$. \\
\texttt{E.} & $\exists$ &
  \hyperref[df-ex]{\texttt{df-ex}} &
  There exists; the wff
  $\exists x \varphi$ is true iff
  there is at least one $x$ where $\varphi$ is true. \\
\texttt{[ y / x ]} & $[ y / x ]$ &
  \hyperref[df-sb]{\texttt{df-sb}} &
  The wff $[ y / x ] \varphi$ produces
  the result when $y$ is properly substituted for $x$ in $\varphi$
  ($y$ replaces $x$).
  % This is elsb4
  % ( [ x / y ] z e. y <-> z e. x )
  For example,
  $[ x / y ] z \in y$ is the same as $z \in x$. \\
\texttt{E!} & $\exists !$ &
  \hyperref[df-eu]{\texttt{df-eu}} &
  There exists exactly one;
  $\exists ! x \varphi$ is true iff
  there is at least one $x$ where $\varphi$ is true. \\
\texttt{\{ y | phi \}}  & $ \{ y | \varphi \}$ &
  \hyperref[df-clab]{\texttt{df-clab}} &
  The class of all sets where $\varphi$ is true. \\
\texttt{=} & $ = $ &
  \hyperref[df-cleq]{\texttt{df-cleq}} &
  Class equality; $A = B$ iff $A$ equals $B$. \\
\texttt{e.} & $ \in $ &
  \hyperref[df-clel]{\texttt{df-clel}} &
  Class membership; $A \in B$ if $A$ is a member of $B$. \\
\texttt{{\char`\_}V} & {\rm V} &
  \hyperref[df-v]{\texttt{df-v}} &
  Class of all sets (not itself a set). \\
\texttt{C\_} & $ \subseteq $ &
  \hyperref[df-ss]{\texttt{df-ss}} &
  Subclass (subset); $A \subseteq B$ is true iff
  $A$ is a subclass of $B$. \\
\texttt{u.} & $ \cup $ &
  \hyperref[df-un]{\texttt{df-un}} &
  $A \cup B$ is the union of classes $A$ and $B$. \\
\texttt{i^i} & $ \cap $ &
  \hyperref[df-in]{\texttt{df-in}} &
  $A \cap B$ is the intersection of classes $A$ and $B$. \\
\texttt{\char`\\} & $ \setminus $ &
  \hyperref[df-dif]{\texttt{df-dif}} &
  $A \setminus B$ (set difference)
  is the class of all sets in $A$ except for those in $B$. \\
\texttt{(/)} & $ \varnothing $ &
  \hyperref[df-nul]{\texttt{df-nul}} &
  $ \varnothing $ is the empty set (aka null set). \\
\texttt{\char`\~P} & $ \cal P $ &
  \hyperref[df-pw]{\texttt{df-pw}} &
  Power class. \\
\texttt{<.\ A , B >.} & $\langle A , B \rangle$ &
  \hyperref[df-op]{\texttt{df-op}} &
  The ordered pair $\langle A , B \rangle$. \\
\texttt{( F ` A )} & $ ( F ` A ) $ &
  \hyperref[df-fv]{\texttt{df-fv}} &
  The value of function $F$ when applied to $A$. \\
\texttt{\_i} & $ i $ &
  \texttt{df-i} &
  The square root of negative one. \\
\texttt{x.} & $ \cdot $ &
  \texttt{df-mul} &
  Complex number multiplication; $2~\cdot~3~=~6$. \\
\texttt{CC} & $ \mathbb{C} $ &
  \texttt{df-c} &
  The set of complex numbers. \\
\texttt{RR} & $ \mathbb{R} $ &
  \texttt{df-r} &
  The set of real numbers. \\
\end{longtabu}
} % end of extrarowsep

\chapter{Compressed Proofs}
\label{compressed}\index{compressed proof}\index{proof!compressed}

The proofs in the \texttt{set.mm} set theory database are stored in compressed
format for efficiency.  Normally you needn't concern yourself with the
compressed format, since you can display it with the usual proof display tools
in the Metamath program (\texttt{show proof}\ldots) or convert it to the normal
RPN proof format described in Section~\ref{proof} (with \texttt{save proof}
{\em label} \texttt{/normal}).  However for sake of completeness we describe the
format here and show how it maps to the normal RPN proof format.

A compressed proof, located between \texttt{\$=} and \texttt{\$.}\ keywords, consists
of a left parenthesis, a sequence of statement labels, a right parenthesis,
and a sequence of upper-case letters \texttt{A} through \texttt{Z} (with optional
white space between them).  White space must surround the parentheses
and the labels.  The left parenthesis tells Metamath that a
compressed proof follows.  (A normal RPN proof consists of just a sequence of
labels, and a parenthesis is not a legal character in a label.)

The sequence of upper-case letters corresponds to a sequence of integers
with the following mapping.  Each integer corresponds to a proof step as
described later.
\begin{center}
  \texttt{A} = 1 \\
  \texttt{B} = 2 \\
   \ldots \\
  \texttt{T} = 20 \\
  \texttt{UA} = 21 \\
  \texttt{UB} = 22 \\
   \ldots \\
  \texttt{UT} = 40 \\
  \texttt{VA} = 41 \\
  \texttt{VB} = 42 \\
   \ldots \\
  \texttt{YT} = 120 \\
  \texttt{UUA} = 121 \\
   \ldots \\
  \texttt{YYT} = 620 \\
  \texttt{UUUA} = 621 \\
   etc.
\end{center}

In other words, \texttt{A} through \texttt{T} represent the
least-significant digit in base 20, and \texttt{U} through \texttt{Y}
represent zero or more most-significant digits in base 5, where the
digits start counting at 1 instead of the usual 0. With this scheme, we
don't need white space between these ``numbers.''

(In the design of the compressed proof format, only upper-case letters,
as opposed to say all non-whitespace printable {\sc ascii} characters other than
%\texttt{\$}, was chosen to make the compressed proof a little less
%displeasing to the eye, at the expense of a typical 20\% compression
\texttt{\$}, were chosen so as not to collide with most text editor
searches, at the expense of a typical 20\% compression
loss.  The base 5/base 20 grouping, as opposed to say base 6/base 19,
was chosen by experimentally determining the grouping that resulted in
best typical compression.)

The letter \texttt{Z} identifies (tags) a proof step that is identical to one
that occurs later on in the proof; it helps shorten the proof by not requiring
that identical proof steps be proved over and over again (which happens often
when building wff's).  The \texttt{Z} is placed immediately after the
least-significant digit (letters \texttt{A} through \texttt{T}) that ends the integer
corresponding to the step to later be referenced.

The integers that the upper-case letters correspond to are mapped to labels as
follows.  If the statement being proved has $m$ mandatory hypotheses, integers
1 through $m$ correspond to the labels of these hypotheses in the order shown
by the \texttt{show statement ... / full} command, i.e., the RPN order\index{RPN
order} of the mandatory
hypotheses.  Integers $m+1$ through $m+n$ correspond to the labels enclosed in
the parentheses of the compressed proof, in the order that they appear, where
$n$ is the number of those labels.  Integers $m+n+1$ on up don't directly
correspond to statement labels but point to proof steps identified with the
letter \texttt{Z}, so that these proof steps can be referenced later in the
proof.  Integer $m+n+1$ corresponds to the first step tagged with a \texttt{Z},
$m+n+2$ to the second step tagged with a \texttt{Z}, etc.  When the compressed
proof is converted to a normal proof, the entire subproof of a step tagged
with \texttt{Z} replaces the reference to that step.

For efficiency, Metamath works with compressed proofs directly, without
converting them internally to normal proofs.  In addition to the usual
error-checking, an error message is given if (1) a label in the label list in
parentheses does not refer to a previous \texttt{\$p} or \texttt{\$a} statement or a
non-mandatory hypothesis of the statement being proved and (2) a proof step
tagged with \texttt{Z} is referenced before the step tagged with the \texttt{Z}.

Just as in a normal proof under development (Section~\ref{unknown}), any step
or subproof that is not yet known may be represented with a single \texttt{?}.
White space does not have to appear between the \texttt{?}\ and the upper-case
letters (or other \texttt{?}'s) representing the remainder of the proof.

% April 1, 2004 Appendix C has been added back in with corrections.
%
% May 20, 2003 Appendix C was removed for now because there was a problem found
% by Bob Solovay
%
% Also, removed earlier \ref{formalspec} 's (3 cases above)
%
% Bob Solovay wrote on 30 Nov 2002:
%%%%%%%%%%%%% (start of email comment )
%      3. My next set of comments concern appendix C. I read this before I
% read Chapter 4. So I first noted that the system as presented in the
% Appendix lacked a certain formal property that I thought desirable. I
% then came up with a revised formal system that had this property. Upon
% reading Chapter 4, I noticed that the revised system was closer to the
% treatment in Chapter 4 than the system in Appendix C.
%
%         First a very minor correction:
%
%         On page 142 line 2: The condition that V(e) != V(f) should only be
% required of e, f in T such that e != f.
%
%         Here is a natural property [transitivity] that one would like
% the formal system to have:
%
%         Let Gamma be a set of statements. Suppose that the statement Phi
% is provable from Gamma and that the statement Psi is provable from Gamma
% \cup {Phi}. Then Psi is provable from Gamma.
%
%         I shall present an example to show that this property does not
% hold for the formal systems of Appendix C:
%
%         I write the example in metamath style:
%
% $c A B C D E $.
% $v x y
%
% ${
% tx $f A x $.
% ty $f B y $.
% ax1 $a C x y $.
% $}
%
% ${
% tx $f A x $.
% ty $f B y $.
% ax2-h1 $e C x y $.
% ax2 $a D y $.
% $}
%
% ${
% ty $f B y $.
% ax3-h1 $e D y $.
% ax3 $a E y $.
% $}
%
% $(These three axioms are Gamma $)
%
% ${
% tx $f A x $.
% ty $f B y $.
% Phi $p D y $=
% tx ty tx ty ax1 ax2 $.
% $}
%
% ${
% ty $f B y $.
% Psi $p E y $=
% ty ty Phi ax3 $.
% $}
%
%
% I omit the formal proofs of the following claims. [I will be glad to
% supply them upon request.]
%
% 1) Psi is not provable from Gamma;
%
% 2) Psi is provable from Gamma + Phi.
%
% Here "provable" refers to the formalism of Appendix C.
%
% The trouble of course is that Psi is lacking the variable declaration
%
% $f Ax $.
%
% In the Metamath system there is no trouble proving Psi. I attach a
% metamath file that shows this and which has been checked by the
% metamath program.
%
% I next want to indicate how I think the treatment in Appendix C should
% be revised so as to conform more closely to the metamath system of the
% main text. The revised system *does* have the transitivity property.
%
% We want to give revised definitions of "statement" and
% "provable". [cf. sections C.2.4. and C.2.5] Our new definitions will
% use the definitions given in Appendix C. So we take the following
% tack. We refer to the original notions as o-statement and o-provable. And
% we refer to the notions we are defining as n-statement and n-provable.
%
%         A n-statement is an o-statement in which the only variables
% that appear in the T component are mandatory.
%
%         To any o-statement we can associate its reduct which is a
% n-statement by dropping all the elements of T or D which contain
% non-mandatory variables.
%
%         An n-statement gamma is n-provable if there is an o-statement
% gamma' which has gamma as its reduct andf such that gamma' is
% o-provable.
%
%         It seems to me [though I am not completely sure on this point]
% that n-provability corresponds to metamath provability as discussed
% say in Chapter 4.
%
%         Attached to this letter is the metamath proof of Phi and Psi
% from Gamma discussed above.
%
%         I am still brooding over the question of whether metamath
% correctly formalizes set-theory. No doubt I will have some questions
% re this after my thoughts become clearer.
%%%%%%%%%%%%%%%% (end of email comment)

%%%%%%%%%%%%%%%% (start of 2nd email comment from Bob Solovay 1-Apr-04)
%
%         I hope that Appendix C is the one that gives a "formal" treatment
% of Metamath. At any rate, thats the appendix I want to comment on.
%
%         I'm going to suggest two changes in the definition.
%
%         First change (in the definition of statement): Require that the
% sets D, T, and E be finite.
%
%         Probably things are fine as you give them. But in the applications
% to the main metamath system they will always be finite, and its useful in
% thinking about things [at least for me] to stick to the finite case.
%
%         Second change:
%
%         First let me give an approximate description. Remove the dummy
% variables from the statement. Instead, include them in the proof.
%
%         More formally: Require that T consists of type declarations only
% for mandatory variables. Require that all the pairs in D consist of
% mandatory variables.
%
%         At the start of a proof we are allowed to declare a finite number
% of dummy variables [provided that none of them appear in any of the
% statements in E \cup {A}. We have to supply type declarations for all the
% dummy variables. We are allowed to add new $d statements referring to
% either the mandatory or dummy variables. But we require that no new $d
% statement references only mandatory variables.
%
%         I find this way of doing things more conceptual than the treatment
% in Appendix C. But the change [which I will use implicitly in later
% letters about doing Peano] is mainly aesthetic. I definitely claim that my
% results on doing Peano all apply to Metamath as it is presented in your
% book.
%
%         --Bob
%
%%%%%%%%%%%%%%%% (end of 2nd email comment)

%%
%% When uncommenting the below, also uncomment references above to {formalspec}
%%
\chapter{Metamath's Formal System}\label{formalspec}\index{Metamath!as a formal
system}

\section{Introduction}

\begin{quote}
  {\em Perfection is when there is no longer anything more to take away.}
    \flushright\sc Antoine de
     Saint-Exupery\footnote{\cite[p.~3-25]{Campbell}.}\\
\end{quote}\index{de Saint-Exupery, Antoine}

This appendix describes the theory behind the Metamath language in an abstract
way intended for mathematicians.  Specifically, we construct two
set-theo\-ret\-i\-cal objects:  a ``formal system'' (roughly, a set of syntax
rules, axioms, and logical rules) and its ``universe'' (roughly, the set of
theorems derivable in the formal system).  The Metamath computer language
provides us with a way to describe specific formal systems and, with the aid of
a proof provided by the user, to verify that given theorems
belong to their universes.

To understand this appendix, you need a basic knowledge of informal set theory.
It should be sufficient to understand, for example, Ch.\ 1 of Munkres' {\em
Topology} \cite{Munkres}\index{Munkres, James R.} or the
introductory set theory chapter
in many textbooks that introduce abstract mathematics. (Note that there are
minor notational differences among authors; e.g.\ Munkres uses $\subset$ instead
of our $\subseteq$ for ``subset.''  We use ``included in'' to mean ``a subset
of,'' and ``belongs to'' or ``is contained in'' to mean ``is an element of.'')
What we call a ``formal'' description here, unlike earlier, is actually an
informal description in the ordinary language of mathematicians.  However we
provide sufficient detail so that a mathematician could easily formalize it,
even in the language of Metamath itself if desired.  To understand the logic
examples at the end of this appendix, familiarity with an introductory book on
mathematical logic would be helpful.

\section{The Formal Description}

\subsection[Preliminaries]{Preliminaries\protect\footnotemark}%
\footnotetext{This section is taken mostly verbatim
from Tarski \cite[p.~63]{Tarski1965}\index{Tarski, Alfred}.}

By $\omega$ we denote the set of all natural numbers (non-negative integers).
Each natural number $n$ is identified with the set of all smaller numbers: $n =
\{ m | m < n \}$.  The formula $m < n$ is thus equivalent to the condition: $m
\in n$ and $m,n \in \omega$. In particular, 0 is the number zero and at the
same time the empty set $\varnothing$, $1=\{0\}$, $2=\{0,1\}$, etc. ${}^B A$
denotes the set of all functions on $B$ to $A$ (i.e.\ with domain $B$ and range
included in $A$).  The members of ${}^\omega A$ are what are called {\em simple
infinite sequences},\index{simple infinite sequence}
with all {\em terms}\index{term} in $A$.  In case $n \in \omega$, the
members of ${}^n A$ are referred to as {\em finite $n$-termed
sequences},\index{finite $n$-termed
sequence} again
with terms in $A$.  The consecutive terms (function values) of a finite or
infinite sequence $f$ are denoted by $f_0, f_1, \ldots ,f_n,\ldots$.  Every
finite sequence $f \in \bigcup _{n \in \omega} {}^n A$ uniquely determines the
number $n$ such that $f \in {}^n A$; $n$ is called the {\em
length}\index{length of a sequence ({$"|\ "|$})} of $f$ and
is denoted by $|f|$.  $\langle a \rangle$ is the sequence $f$ with $|f|=1$ and
$f_0=a$; $\langle a,b \rangle$ is the sequence $f$ with $|f|=2$, $f_0=a$,
$f_1=b$; etc.  Given two finite sequences $f$ and $g$, we denote by $f\frown g$
their {\em concatenation},\index{concatenation} i.e., the
finite sequence $h$ determined by the
conditions:
\begin{eqnarray*}
& |h| = |f|+|g|;&  \\
& h_n = f_n & \mbox{\ for\ } n < |f|;  \\
& h_{|f|+n} = g_n & \mbox{\ for\ } n < |g|.
\end{eqnarray*}

\subsection{Constants, Variables, and Expressions}

A formal system has a set of {\em symbols}\index{symbol!in
a formal system} denoted
by $\mbox{\em SM}$.  A
precise set-theo\-ret\-i\-cal definition of this set is unimportant; a symbol
could be considered a primitive or atomic element if we wish.  We assume this
set is divided into two disjoint subsets:  a set $\mbox{\em CN}$ of {\em
constants}\index{constant!in a formal system} and a set $\mbox{\em VR}$ of
{\em variables}.\index{variable!in a formal system}  $\mbox{\em CN}$ and
$\mbox{\em VR}$ are each assumed to consist of countably many symbols which
may be arranged in finite or simple infinite sequences $c_0, c_1, \ldots$ and
$v_0, v_1, \ldots$ respectively, without repeating terms.  We will represent
arbitrary symbols by metavariables $\alpha$, $\beta$, etc.

{\footnotesize\begin{quotation}
{\em Comment.} The variables $v_0, v_1, \ldots$ of our formal system
correspond to what are usually considered ``metavariables'' in
descriptions of specific formal systems in the literature.  Typically,
when describing a specific formal system a book will postulate a set of
primitive objects called variables, then proceed to describe their
properties using metavariables that range over them, never mentioning
again the actual variables themselves.  Our formal system does not
mention these primitive variable objects at all but deals directly with
metavariables, as its primitive objects, from the start.  This is a
subtle but key distinction you should keep in mind, and it makes our
definition of ``formal system'' somewhat different from that typically
found in the literature.  (So, the $\alpha$, $\beta$, etc.\ above are
actually ``metametavariables'' when used to represent $v_0, v_1,
\ldots$.)
\end{quotation}}

Finite sequences all terms of which are symbols are called {\em
expressions}.\index{expression!in a formal system}  $\mbox{\em EX}$ is
the set of all expressions; thus
\begin{displaymath}
\mbox{\em EX} = \bigcup _{n \in \omega} {}^n \mbox{\em SM}.
\end{displaymath}

A {\em constant-prefixed expression}\index{constant-prefixed expression}
is an expression of non-zero length
whose first term is a constant.  We denote the set of all constant-prefixed
expressions by $\mbox{\em EX}_C = \{ e \in \mbox{\em EX} | ( |e| > 0 \wedge
e_0 \in \mbox{\em CN} ) \}$.

A {\em constant-variable pair}\index{constant-variable pair}
is an expression of length 2 whose first term
is a constant and whose second term is a variable.  We denote the set of all
constant-variable pairs by $\mbox{\em EX}_2 = \{ e \in \mbox{\em EX}_C | ( |e|
= 2 \wedge e_1 \in \mbox{\em VR} ) \}$.


{\footnotesize\begin{quotation}
{\em Relationship to Metamath.} In general, the set $\mbox{\em SM}$
corresponds to the set of declared math symbols in a Metamath database, the
set $\mbox{\em CN}$ to those declared with \texttt{\$c} statements, and the set
$\mbox{\em VR}$ to those declared with \texttt{\$v} statements.  Of course a
Metamath database can only have a finite number of math symbols, whereas
formal systems in general can have an infinite number, although the number of
Metamath math symbols available is in principle unlimited.

The set $\mbox{\em EX}_C$ corresponds to the set of permissible expressions
for \texttt{\$e}, \texttt{\$a}, and \texttt{\$p} statements.  The set $\mbox{\em EX}_2$
corresponds to the set of permissible expressions for \texttt{\$f} statements.
\end{quotation}}

We denote by ${\cal V}(e)$ the set of all variables in an expression $e \in
\mbox{\em EX}$, i.e.\ the set of all $\alpha \in \mbox{\em VR}$ such that
$\alpha = e_n$ for some $n < |e|$.  We also denote (with abuse of notation) by
${\cal V}(E)$ the set of all variables in a collection of expressions $E
\subseteq \mbox{\em EX}$, i.e.\ $\bigcup _{e \in E} {\cal V}(e)$.


\subsection{Substitution}

Given a function $F$ from $\mbox{\em VR}$ to
$\mbox{\em EX}$, we
denote by $\sigma_{F}$ or just $\sigma$ the function from $\mbox{\em EX}$ to
$\mbox{\em EX}$ defined recursively for nonempty sequences by
\begin{eqnarray*}
& \sigma(<\alpha>) = F(\alpha) & \mbox{for\ } \alpha \in \mbox{\em VR}; \\
& \sigma(<\alpha>) = <\alpha> & \mbox{for\ } \alpha \not\in \mbox{\em VR}; \\
& \sigma(g \frown h) = \sigma(g) \frown
    \sigma(h) & \mbox{for\ } g,h \in \mbox{\em EX}.
\end{eqnarray*}
We also define $\sigma(\varnothing)=\varnothing$.  We call $\sigma$ a {\em
simultaneous substitution}\index{substitution!variable}\index{variable
substitution} (or just {\em substitution}) with {\em substitution
map}\index{substitution map} $F$.

We also denote (with abuse of notation) by $\sigma(E)$ a substitution on a
collection of expressions $E \subseteq \mbox{\em EX}$, i.e.\ the set $\{
\sigma(e) | e \in E \}$.  The collection $\sigma(E)$ may of course contain
fewer expressions than $E$ because duplicate expressions could result from the
substitution.

\subsection{Statements}

We denote by $\mbox{\em DV}$ the set of all
unordered pairs $\{\alpha, \beta \} \subseteq \mbox{\em VR}$ such that $\alpha
\neq \beta$.  $\mbox{\em DV}$ stands for ``distinct variables.''

A {\em pre-statement}\index{pre-statement!in a formal system} is a
quadruple $\langle D,T,H,A \rangle$ such that
$D\subseteq \mbox{\em DV}$, $T\subseteq \mbox{\em EX}_2$, $H\subseteq
\mbox{\em EX}_C$ and $H$ is finite,
$A\in \mbox{\em EX}_C$, ${\cal V}(H\cup\{A\}) \subseteq
{\cal V}(T)$, and $\forall e,f\in T {\ } {\cal V}(e) \neq {\cal V}(f)$ (or
equivalently, $e_1 \ne f_1$) whenever $e \neq f$. The terms of the quadruple are called {\em
distinct-variable restrictions},\index{disjoint-variable restriction!in a
formal system} {\em variable-type hypotheses},\index{variable-type
hypothesis!in a formal system} {\em logical hypotheses},\index{logical
hypothesis!in a formal system} and the {\em assertion}\index{assertion!in a
formal system} respectively.  We denote by $T_M$ ({\em mandatory variable-type
hypotheses}\index{mandatory variable-type hypothesis!in a formal system}) the
subset of $T$ such that ${\cal V}(T_M) ={\cal V}(H \cup \{A\})$.  We denote by
$D_M=\{\{\alpha,\beta\}\in D|\{\alpha,\beta\}\subseteq {\cal V}(T_M)\}$ the
{\em mandatory distinct-variable restrictions}\index{mandatory
disjoint-variable restriction!in a formal system} of the pre-statement.
The set
of {\em mandatory hypotheses}\index{mandatory hypothesis!in a formal system}
is $T_M\cup H$.  We call the quadruple $\langle D_M,T_M,H,A \rangle$
the {\em reduct}\index{reduct!in a formal system} of
the pre-statement $\langle D,T,H,A \rangle$.

A {\em statement} is the reduct of some pre-statement\index{statement!in a
formal system}.  A statement is therefore a special kind of pre-statement;
in particular, a statement is the reduct of itself.

{\footnotesize\begin{quotation}
{\em Comment.}  $T$ is a set of expressions, each of length 2, that associate
a set of constants (``variable types'') with a set of variables.  The
condition ${\cal V}(H\cup\{A\}) \subseteq {\cal V}(T) $
means that each variable occurring in a statement's logical
hypotheses or assertion must have an associated variable-type hypothesis or
``type declaration,'' in  analogy to a computer programming language, where a
variable must be declared to be say, a string or an integer.  The requirement
that $\forall e,f\in T \, e_1 \ne f_1$ for $e\neq f$
means that each variable must be
associated with a unique constant designating its variable type; e.g., a
variable might be a ``wff'' or a ``set'' but not both.

Distinct-variable restrictions are used to specify what variable substitutions
are permissible to make for the statement to remain valid.  For example, in
the theorem scheme of set theory $\lnot\forall x\,x=y$ we may not substitute
the same variable for both $x$ and $y$.  On the other hand, the theorem scheme
$x=y\to y=x$ does not require that $x$ and $y$ be distinct, so we do not
require a distinct-variable restriction, although having one
would cause no harm other than making the scheme less general.

A mandatory variable-type hypothesis is one whose variable exists in a logical
hypothesis or the assertion.  A provable pre-statement
(defined below) may require
non-mandatory variable-type hypotheses that effectively introduce ``dummy''
variables for use in its proof.  Any number of dummy variables might
be required by a specific proof; indeed, it has been shown by H.\
Andr\'{e}ka\index{Andr{\'{e}}ka, H.} \cite{Nemeti} that there is no finite
upper bound to the number of dummy variables needed to prove an arbitrary
theorem in first-order logic (with equality) having a fixed number $n>2$ of
individual variables.  (See also the Comment on p.~\pageref{nodd}.)
For this reason we do not set a finite size bound on the collections $D$ and
$T$, although in an actual application (Metamath database) these will of
course be finite, increased to whatever size is necessary as more
proofs are added.
\end{quotation}}

{\footnotesize\begin{quotation}
{\em Relationship to Metamath.} A pre-statement of a formal system
corresponds to an extended frame in a Metamath database
(Section~\ref{frames}).  The collections $D$, $T$, and $H$ correspond
respectively to the \texttt{\$d}, \texttt{\$f}, and \texttt{\$e}
statement collections in an extended frame.  The expression $A$
corresponds to the \texttt{\$a} (or \texttt{\$p}) statement in an
extended frame.

A statement of a formal system corresponds to a frame in a Metamath
database.
\end{quotation}}

\subsection{Formal Systems}

A {\em formal system}\index{formal system} is a
triple $\langle \mbox{\em CN},\mbox{\em
VR},\Gamma\rangle$ where $\Gamma$ is a set of statements.  The members of
$\Gamma$ are called {\em axiomatic statements}.\index{axiomatic
statement!in a formal system}  Sometimes we will refer to a
formal system by just $\Gamma$ when $\mbox{\em CN}$ and $\mbox{\em VR}$ are
understood.

Given a formal system $\Gamma$, the {\em closure}\index{closure}\footnote{This
definition of closure incorporates a simplification due to
Josh Purinton.\index{Purinton, Josh}.} of a
pre-statement
$\langle D,T,H,A \rangle$ is the smallest set $C$ of expressions
such that:
%\begin{enumerate}
%  \item $T\cup H\subseteq C$; and
%  \item If for some axiomatic statement
%    $\langle D_M',T_M',H',A' \rangle \in \Gamma_A$, for
%    some $E \subseteq C$, some $F \subseteq C-T$ (where ``-'' denotes
%    set difference), and some substitution
%    $\sigma$ we have
%    \begin{enumerate}
%       \item $\sigma(T_M') = E$ (where, as above, the $M$ denotes the
%           mandatory variable-type hypotheses of $T^A$);
%       \item $\sigma(H') = F$;
%       \item for all $\{\alpha,\beta\}\in D^A$ and $\subseteq
%         {\cal V}(T_M')$, for all $\gamma\in {\cal V}(\sigma(\langle \alpha
%         \rangle))$, and for all $\delta\in  {\cal V}(\sigma(\langle \beta
%         \rangle))$, we have $\{\gamma, \delta\} \in D$;
%   \end{enumerate}
%   then $\sigma(A') \in C$.
%\end{enumerate}
\begin{list}{}{\itemsep 0.0pt}
  \item[1.] $T\cup H\subseteq C$; and
  \item[2.] If for some axiomatic statement
    $\langle D_M',T_M',H',A' \rangle \in
       \Gamma$ and for some substitution
    $\sigma$ we have
    \begin{enumerate}
       \item[a.] $\sigma(T_M' \cup H') \subseteq C$; and
       \item[b.] for all $\{\alpha,\beta\}\in D_M'$, for all $\gamma\in
         {\cal V}(\sigma(\langle \alpha
         \rangle))$, and for all $\delta\in  {\cal V}(\sigma(\langle \beta
         \rangle))$, we have $\{\gamma, \delta\} \in D$;
   \end{enumerate}
   then $\sigma(A') \in C$.
\end{list}
A pre-statement $\langle D,T,H,A
\rangle$ is {\em provable}\index{provable statement!in a formal
system} if $A\in C$ i.e.\ if its assertion belongs to its
closure.  A statement is {\em provable} if it is
the reduct of a provable pre-statement.
The {\em universe}\index{universe of a formal system}
of a formal system is
the collection of all of its provable statements.  Note that the
set of axiomatic statements $\Gamma$ in a formal system is a subset of its
universe.

{\footnotesize\begin{quotation}
{\em Comment.} The first condition in the definition of closure simply says
that the hypotheses of the pre-statement are in its closure.

Condition 2(a) says that a substitution exists that makes the
mandatory hypotheses of an axiomatic statement exactly match some members of
the closure.  This is what we explicitly demonstrate in a Metamath language
proof.

%Conditions 2(a) and 2(b) say that a substitution exists that makes the
%(mandatory) hypotheses of an axiomatic statement exactly match some members of
%the closure.  This is what we explicitly demonstrate with a Metamath language
%proof.
%
%The set of expressions $F$ in condition 2(b) excludes the variable-type
%hypotheses; this is done because non-mandatory variable-type hypotheses are
%effectively ``dropped'' as irrelevant whereas logical hypotheses must be
%retained to achieve a consistent logical system.

Condition 2(b) describes how distinct-variable restrictions in the axiomatic
statement must be met.  It means that after a substitution for two variables
that must be distinct, the resulting two expressions must either contain no
variables, or if they do, they may not have variables in common, and each pair
of any variables they do have, with one variable from each expression, must be
specified as distinct in the original statement.
\end{quotation}}

{\footnotesize\begin{quotation}
{\em Relationship to Metamath.} Axiomatic statements
 and provable statements in a formal
system correspond to the frames for \texttt{\$a} and \texttt{\$p} statements
respectively in a Metamath database.  The set of axiomatic statements is a
subset of the set of provable statements in a formal system, although in a
Metamath database a \texttt{\$a} statement is distinguished by not having a
proof.  A Metamath language proof for a \texttt{\$p} statement tells the computer
how to explicitly construct a series of members of the closure ultimately
leading to a demonstration that the assertion
being proved is in the closure.  The actual closure typically contains
an infinite number of expressions.  A formal system itself does not have
an explicit object called a ``proof'' but rather the existence of a proof
is implied indirectly by membership of an assertion in a provable
statement's closure.  We do this to make the formal system easier
to describe in the language of set theory.

We also note that once established as provable, a statement may be considered
to acquire the same status as an axiomatic statement, because if the set of
axiomatic statements is extended with a provable statement, the universe of
the formal system remains unchanged (provided that $\mbox{\em VR}$ is
infinite).
In practice, this means we can build a hierarchy of provable statements to
more efficiently establish additional provable statements.  This is
what we do in Metamath when we allow proofs to reference previous
\texttt{\$p} statements as well as previous \texttt{\$a} statements.
\end{quotation}}

\section{Examples of Formal Systems}

{\footnotesize\begin{quotation}
{\em Relationship to Metamath.} The examples in this section, except Example~2,
are for the most part exact equivalents of the development in the set
theory database \texttt{set.mm}.  You may want to compare Examples~1, 3, and 5
to Section~\ref{metaaxioms}, Example 4 to Sections~\ref{metadefprop} and
\ref{metadefpred}, and Example 6 to
Section~\ref{setdefinitions}.\label{exampleref}
\end{quotation}}

\subsection{Example~1---Propositional Calculus}\index{propositional calculus}

Classical propositional calculus can be described by the following formal
system.  We assume the set of variables is infinite.  Rather than denoting the
constants and variables by $c_0, c_1, \ldots$ and $v_0, v_1, \ldots$, for
readability we will instead use more conventional symbols, with the
understanding of course that they denote distinct primitive objects.
Also for readability we may omit commas between successive terms of a
sequence; thus $\langle \mbox{wff\ } \varphi\rangle$ denotes
$\langle \mbox{wff}, \varphi\rangle$.

Let
\begin{itemize}
  \item[] $\mbox{\em CN}=\{\mbox{wff}, \vdash, \to, \lnot, (,)\}$
  \item[] $\mbox{\em VR}=\{\varphi,\psi,\chi,\ldots\}$
  \item[] $T = \{\langle \mbox{wff\ } \varphi\rangle,
             \langle \mbox{wff\ } \psi\rangle,
             \langle \mbox{wff\ } \chi\rangle,\ldots\}$, i.e.\ those
             expressions of length 2 whose first member is $\mbox{\rm wff}$
             and whose second member belongs to $\mbox{\em VR}$.\footnote{For
convenience we let $T$ be an infinite set; the definition of a statement
permits this in principle.  Since a Metamath source file has a finite size, in
practice we must of course use appropriate finite subsets of this $T$,
specifically ones containing at least the mandatory variable-type
hypotheses.  Similarly, in the source file we introduce new variables as
required, with the understanding that a potentially infinite number of
them are available.}
\noindent Then $\Gamma$ consists of the axiomatic statements that
are the reducts of the following pre-statements:
    \begin{itemize}
      \item[] $\langle\varnothing,T,\varnothing,
               \langle \mbox{wff\ }(\varphi\to\psi)\rangle\rangle$
      \item[] $\langle\varnothing,T,\varnothing,
               \langle \mbox{wff\ }\lnot\varphi\rangle\rangle$
      \item[] $\langle\varnothing,T,\varnothing,
               \langle \vdash(\varphi\to(\psi\to\varphi))
               \rangle\rangle$
      \item[] $\langle\varnothing,T,
               \varnothing,
               \langle \vdash((\varphi\to(\psi\to\chi))\to
               ((\varphi\to\psi)\to(\varphi\to\chi)))
               \rangle\rangle$
      \item[] $\langle\varnothing,T,
               \varnothing,
               \langle \vdash((\lnot\varphi\to\lnot\psi)\to
               (\psi\to\varphi))\rangle\rangle$
      \item[] $\langle\varnothing,T,
               \{\langle\vdash(\varphi\to\psi)\rangle,
                 \langle\vdash\varphi\rangle\},
               \langle\vdash\psi\rangle\rangle$
    \end{itemize}
\end{itemize}

(For example, the reduct of $\langle\varnothing,T,\varnothing,
               \langle \mbox{wff\ }(\varphi\to\psi)\rangle\rangle$
is
\begin{itemize}
\item[] $\langle\varnothing,
\{\langle \mbox{wff\ } \varphi\rangle,
             \langle \mbox{wff\ } \psi\rangle\},
             \varnothing,
               \langle \mbox{wff\ }(\varphi\to\psi)\rangle\rangle$,
\end{itemize}
which is the first axiomatic statement.)

We call the members of $\mbox{\em VR}$ {\em wff variables} or (in the context
of first-order logic which we will describe shortly) {\em wff metavariables}.
Note that the symbols $\phi$, $\psi$, etc.\ denote actual specific members of
$\mbox{\em VR}$; they are not metavariables of our expository language (which
we denote with $\alpha$, $\beta$, etc.) but are instead (meta)constant symbols
(members of $\mbox{\em SM}$) from the point of view of our expository
language.  The equivalent system of propositional calculus described in
\cite{Tarski1965} also uses the symbols $\phi$, $\psi$, etc.\ to denote wff
metavariables, but in \cite{Tarski1965} unlike here those are metavariables of
the expository language and not primitive symbols of the formal system.

The first two statements define wffs: if $\varphi$ and $\psi$ are wffs, so is
$(\varphi \to \psi)$; if $\varphi$ is a wff, so is $\lnot\varphi$. The next
three are the axioms of propositional calculus: if $\varphi$ and $\psi$ are
wffs, then $\vdash (\varphi \to (\psi \to \varphi))$ is an (axiomatic)
theorem; etc. The
last is the rule of modus ponens: if $\varphi$ and $\psi$ are wffs, and
$\vdash (\varphi\to\psi)$ and $\vdash \varphi$ are theorems, then $\vdash
\psi$ is a theorem.

The correspondence to ordinary propositional calculus is as follows.  We
consider only provable statements of the form $\langle\varnothing,
T,\varnothing,A\rangle$ with $T$ defined as above.  The first term of the
assertion $A$ of any such statement is either ``wff'' or ``$\vdash$''.  A
statement for which the first term is ``wff'' is a {\em wff} of propositional
calculus, and one where the first term is ``$\vdash$'' is a {\em
theorem (scheme)} of propositional calculus.

The universe of this formal system also contains many other provable
statements.  Those with distinct-variable restrictions are irrelevant because
propositional calculus has no constraints on substitutions.  Those that have
logical hypotheses we call {\em inferences}\index{inference} when
the logical hypotheses are of the form
$\langle\vdash\rangle\frown w$ where $w$ is a wff (with the leading constant
term ``wff'' removed).  Inferences (other than the modus ponens rule) are not a
proper part of propositional calculus but are convenient to use when building a
hierarchy of provable statements.  A provable statement with a nonsense
hypothesis such as $\langle \to,\vdash,\lnot\rangle$, and this same expression
as its assertion, we consider irrelevant; no use can be made of it in
proving theorems, since there is no way to eliminate the nonsense hypothesis.

{\footnotesize\begin{quotation}
{\em Comment.} Our use of parentheses in the definition of a wff illustrates
how axiomatic statements should be carefully stated in a way that
ties in unambiguously with the substitutions allowed by the formal system.
There are many ways we could have defined wffs---for example, Polish
prefix notation would have allowed us to omit parentheses entirely, at
the expense of readability---but we must define them in a way that is
unambiguous.  For example, if we had omitted parentheses from the
definition of $(\varphi\to \psi)$, the wff $\lnot\varphi\to \psi$ could
be interpreted as either $\lnot(\varphi\to\psi)$ or $(\lnot\varphi\to\psi)$
and would have allowed us to prove nonsense.  Note that there is no
concept of operator binding precedence built into our formal system.
\end{quotation}}

\begin{sloppy}
\subsection{Example~2---Predicate Calculus with Equality}\index{predicate
calculus}
\end{sloppy}

Here we extend Example~1 to include predicate calculus with equality,
illustrating the use of distinct-variable restrictions.  This system is the
same as Tarski's system $\mathfrak{S}_2$ in \cite{Tarski1965} (except that the
axioms of propositional calculus are different but equivalent, and a redundant
axiom is omitted).  We extend $\mbox{\em CN}$ with the constants
$\{\mbox{var},\forall,=\}$.  We extend $\mbox{\em VR}$ with an infinite set of
{\em individual metavariables}\index{individual
metavariable} $\{x,y,z,\ldots\}$ and denote this subset
$\mbox{\em Vr}$.

We also join to $\mbox{\em CN}$ a possibly infinite set $\mbox{\em Pr}$ of {\em
predicates} $\{R,S,\ldots\}$.  We associate with $\mbox{\em Pr}$ a function
$\mbox{rnk}$ from $\mbox{\em Pr}$ to $\omega$, and for $\alpha\in \mbox{\em
Pr}$ we call $\mbox{rnk}(\alpha)$ the {\em rank} of the predicate $\alpha$,
which is simply the number of ``arguments'' that the predicate has.  (Most
applications of predicate calculus will have a finite number of predicates;
for example, set theory has the single two-argument or binary predicate $\in$,
which is usually written with its arguments surrounding the predicate symbol
rather than with the prefix notation we will use for the general case.)  As a
device to facilitate our discussion, we will let $\mbox{\em Vs}$ be any fixed
one-to-one function from $\omega$ to $\mbox{\em Vr}$; thus $\mbox{\em Vs}$ is
any simple infinite sequence of individual metavariables with no repeating
terms.

In this example we will not include the function symbols that are often part of
formalizations of predicate calculus.  Using metalogical arguments that are
beyond the scope of our discussion, it can be shown that our formalization is
equivalent when functions are introduced via appropriate definitions.

We extend the set $T$ defined in Example~1 with the expressions
$\{\langle \mbox{var\ } x\rangle,$ $ \langle \mbox{var\ } y\rangle, \langle
\mbox{var\ } z\rangle,\ldots\}$.  We extend the $\Gamma$ above
with the axiomatic statements that are the reducts of the following
pre-statements:
\begin{list}{}{\itemsep 0.0pt}
      \item[] $\langle\varnothing,T,\varnothing,
               \langle \mbox{wff\ }\forall x\,\varphi\rangle\rangle$
      \item[] $\langle\varnothing,T,\varnothing,
               \langle \mbox{wff\ }x=y\rangle\rangle$
      \item[] $\langle\varnothing,T,
               \{\langle\vdash\varphi\rangle\},
               \langle\vdash\forall x\,\varphi\rangle\rangle$
      \item[] $\langle\varnothing,T,\varnothing,
               \langle \vdash((\forall x(\varphi\to\psi)
                  \to(\forall x\,\varphi\to\forall x\,\psi))
               \rangle\rangle$
      \item[] $\langle\{\{x,\varphi\}\},T,\varnothing,
               \langle \vdash(\varphi\to\forall x\,\varphi)
               \rangle\rangle$
      \item[] $\langle\{\{x,y\}\},T,\varnothing,
               \langle \vdash\lnot\forall x\lnot x=y
               \rangle\rangle$
      \item[] $\langle\varnothing,T,\varnothing,
               \langle \vdash(x=z
                  \to(x=y\to z=y))
               \rangle\rangle$
      \item[] $\langle\varnothing,T,\varnothing,
               \langle \vdash(y=z
                  \to(x=y\to x=z))
               \rangle\rangle$
\end{list}
These are the axioms not involving predicate symbols. The first two statements
extend the definition of a wff.  The third is the rule of generalization.  The
fifth states, in effect, ``For a wff $\varphi$ and variable $x$,
$\vdash(\varphi\to\forall x\,\varphi)$, provided that $x$ does not occur in
$\varphi$.''  The sixth states ``For variables $x$ and $y$,
$\vdash\lnot\forall x\lnot x = y$, provided that $x$ and $y$ are distinct.''
(This proviso is not necessary but was included by Tarski to
weaken the axiom and still show that the system is logically complete.)

Finally, for each predicate symbol $\alpha\in \mbox{\em Pr}$, we add to
$\Gamma$ an axiomatic statement, extending the definition of wff,
that is the reduct of the following pre-statement:
\begin{displaymath}
    \langle\varnothing,T,\varnothing,
            \langle \mbox{wff},\alpha\rangle\
            \frown \mbox{\em Vs}\restriction\mbox{rnk}(\alpha)\rangle
\end{displaymath}
and for each $\alpha\in \mbox{\em Pr}$ and each $n < \mbox{rnk}(\alpha)$
we add to $\Gamma$ an equality axiom that is the reduct of the
following pre-statement:
\begin{eqnarray*}
    \lefteqn{\langle\varnothing,T,\varnothing,
            \langle
      \vdash,(,\mbox{\em Vs}_n,=,\mbox{\em Vs}_{\mbox{rnk}(\alpha)},\to,
            (,\alpha\rangle\frown \mbox{\em Vs}\restriction\mbox{rnk}(\alpha)} \\
  & & \frown
            \langle\to,\alpha\rangle\frown \mbox{\em Vs}\restriction n\frown
            \langle \mbox{\em Vs}_{\mbox{rnk}(\alpha)}\rangle \\
 & & \frown
            \mbox{\em Vs}\restriction(\mbox{rnk}(\alpha)\setminus(n+1))\frown
            \langle),)\rangle\rangle
\end{eqnarray*}
where $\restriction$ denotes function domain restriction and $\setminus$
denotes set difference.  Recall that a subscript on $\mbox{\em Vs}$
denotes one of its terms.  (In the above two axiom sets commas are placed
between successive terms of sequences to prevent ambiguity, and if you examine
them with care you will be able to distinguish those parentheses that denote
constant symbols from those of our expository language that delimit function
arguments.  Although it might have been better to use boldface for our
primitive symbols, unfortunately boldface was not available for all characters
on the \LaTeX\ system used to typeset this text.)  These seemingly forbidding
axioms can be understood by analogy to concatenation of substrings in a
computer language.  They are actually relatively simple for each specific case
and will become clearer by looking at the special case of a binary predicate
$\alpha = R$ where $\mbox{rnk}(R)=2$.  Letting $\mbox{\em Vs}$ be the sequence
$\langle x,y,z,\ldots\rangle$, the axioms we would add to $\Gamma$ for this
case would be the wff extension and two equality axioms that are the
reducts of the pre-statements:
\begin{list}{}{\itemsep 0.0pt}
      \item[] $\langle\varnothing,T,\varnothing,
               \langle \mbox{wff\ }R x y\rangle\rangle$
      \item[] $\langle\varnothing,T,\varnothing,
               \langle \vdash(x=z
                  \to(R x y \to R z y))
               \rangle\rangle$
      \item[] $\langle\varnothing,T,\varnothing,
               \langle \vdash(y=z
                  \to(R x y \to R x z))
               \rangle\rangle$
\end{list}
Study these carefully to see how the general axioms above evaluate to
them.  In practice, typically only a few special cases such as this would be
needed, and in any case the Metamath language will only permit us to describe
a finite number of predicates, as opposed to the infinite number permitted by
the formal system.  (If an infinite number should be needed for some reason,
we could not define the formal system directly in the Metamath language but
could instead define it metalogically under set theory as we
do in this appendix, and only the underlying set theory, with its single
binary predicate, would be defined directly in the Metamath language.)


{\footnotesize\begin{quotation}
{\em Comment.}  As we noted earlier, the specific variables denoted by the
symbols $x,y,z,\ldots\in \mbox{\em Vr}\subseteq \mbox{\em VR}\subseteq
\mbox{\em SM}$ in Example~2 are not the actual variables of ordinary predicate
calculus but should be thought of as metavariables ranging over them.  For
example, a distinct-variable restriction would be meaningless for actual
variables of ordinary predicate calculus since two different actual variables
are by definition distinct.  And when we talk about an arbitrary
representative $\alpha\in \mbox{\em Vr}$, $\alpha$ is a metavariable (in our
expository language) that ranges over metavariables (which are primitives of
our formal system) each of which ranges over the actual individual variables
of predicate calculus (which are never mentioned in our formal system).

The constant called ``var'' above is called \texttt{setvar} in the
\texttt{set.mm} database file, but it means the same thing.  I felt
that ``var'' is a more meaningful name in the context of predicate
calculus, whose use is not limited to set theory.  For consistency we
stick with the name ``var'' throughout this Appendix, even after set
theory is introduced.
\end{quotation}}

\subsection{Free Variables and Proper Substitution}\index{free variable}
\index{proper substitution}\index{substitution!proper}

Typical representations of mathematical axioms use concepts such
as ``free variable,'' ``bound variable,'' and ``proper substitution''
as primitive notions.
A free variable is a variable that
is not a parameter of any container expression.
A bound variable is the opposite of a free variable; it is a
a variable that has been bound in a container expression.
For example, in the expression $\forall x \varphi$ (for all $x$, $\varphi$
is true), the variable $x$
is bound within the for-all ($\forall$) expression.
It is possible to change one variable to another, and that process is called
``proper substitution.''
In most books, proper substitution has a somewhat complicated recursive
definition with multiple cases based on the occurrences of free and
bound variables.
You may consult
\cite[ch.\ 3--4]{Hamilton}\index{Hamilton, Alan G.} (as well as
many other texts) for more formal details about these terms.

Using these concepts as \texttt{primitives} creates complications
for computer implementations.

In the system of Example~2, there are no primitive notions of free variable
and proper substitution.  Tarski \cite{Tarski1965} shows that this system is
logically equivalent to the more typical textbook systems that do have these
primitive notions, if we introduce these notions with appropriate definitions
and metalogic.  We could also define axioms for such systems directly,
although the recursive definitions of free variable and proper substitution
would be messy and awkward to work with.  Instead, we mention two devices that
can be used in practice to mimic these notions.  (1) Instead of introducing
special notation to express (as a logical hypothesis) ``where $x$ is not free
in $\varphi$'' we can use the logical hypothesis $\vdash(\varphi\to\forall
x\,\varphi)$.\label{effectivelybound}\index{effectively
not free}\footnote{This is a slightly weaker requirement than ``where $x$ is
not free in $\varphi$.''  If we let $\varphi$ be $x=x$, we have the theorem
$(x=x\to\forall x\,x=x)$ which satisfies the hypothesis, even though $x$ is
free in $x=x$ .  In a case like this we say that $x$ is {\em effectively not
free}\index{effectively not free} in $x=x$, since $x=x$ is logically
equivalent to $\forall x\,x=x$ in which $x$ is bound.} (2) It can be shown
that the wff $((x=y\to\varphi)\wedge\exists x(x=y\wedge\varphi))$ (with the
usual definitions of $\wedge$ and $\exists$; see Example~4 below) is logically
equivalent to ``the wff that results from proper substitution of $y$ for $x$
in $\varphi$.''  This works whether or not $x$ and $y$ are distinct.

\subsection{Metalogical Completeness}\index{metalogical completeness}

In the system of Example~2, the
following are provable pre-statements (and their reducts are
provable statements):
\begin{eqnarray*}
      & \langle\{\{x,y\}\},T,\varnothing,
               \langle \vdash\lnot\forall x\lnot x=y
               \rangle\rangle & \\
     &  \langle\varnothing,T,\varnothing,
               \langle \vdash\lnot\forall x\lnot x=x
               \rangle\rangle &
\end{eqnarray*}
whereas the following pre-statement is not to my knowledge provable (but
in any case we will pretend it's not for sake of illustration):
\begin{eqnarray*}
     &  \langle\varnothing,T,\varnothing,
               \langle \vdash\lnot\forall x\lnot x=y
               \rangle\rangle &
\end{eqnarray*}
In other words, we can prove ``$\lnot\forall x\lnot x=y$ where $x$ and $y$ are
distinct'' and separately prove ``$\lnot\forall x\lnot x=x$'', but we can't
prove the combined general case ``$\lnot\forall x\lnot x=y$'' that has no
proviso.  Now this does not compromise logical completeness, because the
variables are really metavariables and the two provable cases together cover
all possible cases.  The third case can be considered a metatheorem whose
direct proof, using the system of Example~2, lies outside the capability of the
formal system.

Also, in the system of Example~2 the following pre-statement is not to my
knowledge provable (again, a conjecture that we will pretend to be the case):
\begin{eqnarray*}
     & \langle\varnothing,T,\varnothing,
               \langle \vdash(\forall x\, \varphi\to\varphi)
               \rangle\rangle &
\end{eqnarray*}
Instead, we can only prove specific cases of $\varphi$ involving individual
metavariables, and by induction on formula length, prove as a metatheorem
outside of our formal system the general statement above.  The details of this
proof are found in \cite{Kalish}.

There does, however, exist a system of predicate calculus in which all such
``simple metatheorems'' as those above can be proved directly, and we present
it in Example~3. A {\em simple metatheorem}\index{simple metatheorem}
is any statement of the formal
system of Example~2 where all distinct variable restrictions consist of either
two individual metavariables or an individual metavariable and a wff
metavariable, and which is provable by combining cases outside the system as
above.  A system is {\em metalogically complete}\index{metalogical
completeness} if all of its simple
metatheorems are (directly) provable statements. The precise definition of
``simple metatheorem'' and the proof of the ``metalogical completeness'' of
Example~3 is found in Remark 9.6 and Theorem 9.7 of \cite{Megill}.\index{Megill,
Norman}

\begin{sloppy}
\subsection{Example~3---Metalogically Complete Predicate
Calculus with
Equality}
\end{sloppy}

For simplicity we will assume there is one binary predicate $R$;
this system suffices for set theory, where the $R$ is of course the $\in$
predicate.  We label the axioms as they appear in \cite{Megill}.  This
system is logically equivalent to that of Example~2 (when the latter is
restricted to this single binary predicate) but is also metalogically
complete.\index{metalogical completeness}

Let
\begin{itemize}
  \item[] $\mbox{\em CN}=\{\mbox{wff}, \mbox{var}, \vdash, \to, \lnot, (,),\forall,=,R\}$.
  \item[] $\mbox{\em VR}=\{\varphi,\psi,\chi,\ldots\}\cup\{x,y,z,\ldots\}$.
  \item[] $T = \{\langle \mbox{wff\ } \varphi\rangle,
             \langle \mbox{wff\ } \psi\rangle,
             \langle \mbox{wff\ } \chi\rangle,\ldots\}\cup
       \{\langle \mbox{var\ } x\rangle, \langle \mbox{var\ } y\rangle, \langle
       \mbox{var\ }z\rangle,\ldots\}$.

\noindent Then
  $\Gamma$ consists of the reducts of the following pre-statements:
    \begin{itemize}
      \item[] $\langle\varnothing,T,\varnothing,
               \langle \mbox{wff\ }(\varphi\to\psi)\rangle\rangle$
      \item[] $\langle\varnothing,T,\varnothing,
               \langle \mbox{wff\ }\lnot\varphi\rangle\rangle$
      \item[] $\langle\varnothing,T,\varnothing,
               \langle \mbox{wff\ }\forall x\,\varphi\rangle\rangle$
      \item[] $\langle\varnothing,T,\varnothing,
               \langle \mbox{wff\ }x=y\rangle\rangle$
      \item[] $\langle\varnothing,T,\varnothing,
               \langle \mbox{wff\ }Rxy\rangle\rangle$
      \item[(C1$'$)] $\langle\varnothing,T,\varnothing,
               \langle \vdash(\varphi\to(\psi\to\varphi))
               \rangle\rangle$
      \item[(C2$'$)] $\langle\varnothing,T,
               \varnothing,
               \langle \vdash((\varphi\to(\psi\to\chi))\to
               ((\varphi\to\psi)\to(\varphi\to\chi)))
               \rangle\rangle$
      \item[(C3$'$)] $\langle\varnothing,T,
               \varnothing,
               \langle \vdash((\lnot\varphi\to\lnot\psi)\to
               (\psi\to\varphi))\rangle\rangle$
      \item[(C4$'$)] $\langle\varnothing,T,
               \varnothing,
               \langle \vdash(\forall x(\forall x\,\varphi\to\psi)\to
                 (\forall x\,\varphi\to\forall x\,\psi))\rangle\rangle$
      \item[(C5$'$)] $\langle\varnothing,T,
               \varnothing,
               \langle \vdash(\forall x\,\varphi\to\varphi)\rangle\rangle$
      \item[(C6$'$)] $\langle\varnothing,T,
               \varnothing,
               \langle \vdash(\forall x\forall y\,\varphi\to
                 \forall y\forall x\,\varphi)\rangle\rangle$
      \item[(C7$'$)] $\langle\varnothing,T,
               \varnothing,
               \langle \vdash(\lnot\varphi\to\forall x\lnot\forall x\,\varphi
                 )\rangle\rangle$
      \item[(C8$'$)] $\langle\varnothing,T,
               \varnothing,
               \langle \vdash(x=y\to(x=z\to y=z))\rangle\rangle$
      \item[(C9$'$)] $\langle\varnothing,T,
               \varnothing,
               \langle \vdash(\lnot\forall x\, x=y\to(\lnot\forall x\, x=z\to
                 (y=z\to\forall x\, y=z)))\rangle\rangle$
      \item[(C10$'$)] $\langle\varnothing,T,
               \varnothing,
               \langle \vdash(\forall x(x=y\to\forall x\,\varphi)\to
                 \varphi))\rangle\rangle$
      \item[(C11$'$)] $\langle\varnothing,T,
               \varnothing,
               \langle \vdash(\forall x\, x=y\to(\forall x\,\varphi
               \to\forall y\,\varphi))\rangle\rangle$
      \item[(C12$'$)] $\langle\varnothing,T,
               \varnothing,
               \langle \vdash(x=y\to(Rxz\to Ryz))\rangle\rangle$
      \item[(C13$'$)] $\langle\varnothing,T,
               \varnothing,
               \langle \vdash(x=y\to(Rzx\to Rzy))\rangle\rangle$
      \item[(C15$'$)] $\langle\varnothing,T,
               \varnothing,
               \langle \vdash(\lnot\forall x\, x=y\to(x=y\to(\varphi
                 \to\forall x(x=y\to\varphi))))\rangle\rangle$
      \item[(C16$'$)] $\langle\{\{x,y\}\},T,
               \varnothing,
               \langle \vdash(\forall x\, x=y\to(\varphi\to\forall x\,\varphi)
                 )\rangle\rangle$
      \item[(C5)] $\langle\{\{x,\varphi\}\},T,\varnothing,
               \langle \vdash(\varphi\to\forall x\,\varphi)
               \rangle\rangle$
      \item[(MP)] $\langle\varnothing,T,
               \{\langle\vdash(\varphi\to\psi)\rangle,
                 \langle\vdash\varphi\rangle\},
               \langle\vdash\psi\rangle\rangle$
      \item[(Gen)] $\langle\varnothing,T,
               \{\langle\vdash\varphi\rangle\},
               \langle\vdash\forall x\,\varphi\rangle\rangle$
    \end{itemize}
\end{itemize}

While it is known that these axioms are ``metalogically complete,'' it is
not known whether they are independent (i.e.\ none is
redundant) in the metalogical sense; specifically, whether any axiom (possibly
with additional non-mandatory distinct-variable restrictions, for use with any
dummy variables in its proof) is provable from the others.  Note that
metalogical independence is a weaker requirement than independence in the
usual logical sense.  Not all of the above axioms are logically independent:
for example, C9$'$ can be proved as a metatheorem from the others, outside the
formal system, by combining the possible cases of distinct variables.

\subsection{Example~4---Adding Definitions}\index{definition}
There are several ways to add definitions to a formal system.  Probably the
most proper way is to consider definitions not as part of the formal system at
all but rather as abbreviations that are part of the expository metalogic
outside the formal system.  For convenience, though, we may use the formal
system itself to incorporate definitions, adding them as axiomatic extensions
to the system.  This could be done by adding a constant representing the
concept ``is defined as'' along with axioms for it. But there is a nicer way,
at least in this writer's opinion, that introduces definitions as direct
extensions to the language rather than as extralogical primitive notions.  We
introduce additional logical connectives and provide axioms for them.  For
systems of logic such as Examples 1 through 3, the additional axioms must be
conservative in the sense that no wff of the original system that was not a
theorem (when the initial term ``wff'' is replaced by ``$\vdash$'' of course)
becomes a theorem of the extended system.  In this example we extend Example~3
(or 2) with standard abbreviations of logic.

We extend $\mbox{\em CN}$ of Example~3 with new constants $\{\leftrightarrow,
\wedge,\vee,\exists\}$, corresponding to logical equivalence,\index{logical
equivalence ($\leftrightarrow$)}\index{biconditional ($\leftrightarrow$)}
conjunction,\index{conjunction ($\wedge$)} disjunction,\index{disjunction
($\vee$)} and the existential quantifier.\index{existential quantifier
($\exists$)}  We extend $\Gamma$ with the axiomatic statements that are
the reducts of the following pre-statements:
\begin{list}{}{\itemsep 0.0pt}
      \item[] $\langle\varnothing,T,\varnothing,
               \langle \mbox{wff\ }(\varphi\leftrightarrow\psi)\rangle\rangle$
      \item[] $\langle\varnothing,T,\varnothing,
               \langle \mbox{wff\ }(\varphi\vee\psi)\rangle\rangle$
      \item[] $\langle\varnothing,T,\varnothing,
               \langle \mbox{wff\ }(\varphi\wedge\psi)\rangle\rangle$
      \item[] $\langle\varnothing,T,\varnothing,
               \langle \mbox{wff\ }\exists x\, \varphi\rangle\rangle$
  \item[] $\langle\varnothing,T,\varnothing,
     \langle\vdash ( ( \varphi \leftrightarrow \psi ) \to
     ( \varphi \to \psi ) )\rangle\rangle$
  \item[] $\langle\varnothing,T,\varnothing,
     \langle\vdash ((\varphi\leftrightarrow\psi)\to
    (\psi\to\varphi))\rangle\rangle$
  \item[] $\langle\varnothing,T,\varnothing,
     \langle\vdash ((\varphi\to\psi)\to(
     (\psi\to\varphi)\to(\varphi
     \leftrightarrow\psi)))\rangle\rangle$
  \item[] $\langle\varnothing,T,\varnothing,
     \langle\vdash (( \varphi \wedge \psi ) \leftrightarrow\neg ( \varphi
     \to \neg \psi )) \rangle\rangle$
  \item[] $\langle\varnothing,T,\varnothing,
     \langle\vdash (( \varphi \vee \psi ) \leftrightarrow (\neg \varphi
     \to \psi )) \rangle\rangle$
  \item[] $\langle\varnothing,T,\varnothing,
     \langle\vdash (\exists x \,\varphi\leftrightarrow
     \lnot \forall x \lnot \varphi)\rangle\rangle$
\end{list}
The first three logical axioms (statements containing ``$\vdash$'') introduce
and effectively define logical equivalence, ``$\leftrightarrow$''.  The last
three use ``$\leftrightarrow$'' to effectively mean ``is defined as.''

\subsection{Example~5---ZFC Set Theory}\index{ZFC set theory}

Here we add to the system of Example~4 the axioms of Zermelo--Fraenkel set
theory with Choice.  For convenience we make use of the
definitions in Example~4.

In the $\mbox{\em CN}$ of Example~4 (which extends Example~3), we replace the symbol $R$
with the symbol $\in$.
More explicitly, we remove from $\Gamma$ of Example~4 the three
axiomatic statements containing $R$ and replace them with the
reducts of the following:
\begin{list}{}{\itemsep 0.0pt}
      \item[] $\langle\varnothing,T,\varnothing,
               \langle \mbox{wff\ }x\in y\rangle\rangle$
      \item[] $\langle\varnothing,T,
               \varnothing,
               \langle \vdash(x=y\to(x\in z\to y\in z))\rangle\rangle$
      \item[] $\langle\varnothing,T,
               \varnothing,
               \langle \vdash(x=y\to(z\in x\to z\in y))\rangle\rangle$
\end{list}
Letting $D=\{\{\alpha,\beta\}\in \mbox{\em DV}\,|\alpha,\beta\in \mbox{\em
Vr}\}$ (in other words all individual variables must be distinct), we extend
$\Gamma$ with the ZFC axioms, called
\index{Axiom of Extensionality}
\index{Axiom of Replacement}
\index{Axiom of Union}
\index{Axiom of Power Sets}
\index{Axiom of Regularity}
\index{Axiom of Infinity}
\index{Axiom of Choice}
Extensionality, Replacement, Union, Power
Set, Regularity, Infinity, and Choice, that are the reducts of:
\begin{list}{}{\itemsep 0.0pt}
      \item[Ext] $\langle D,T,
               \varnothing,
               \langle\vdash (\forall x(x\in y\leftrightarrow x \in z)\to y
               =z) \rangle\rangle$
      \item[Rep] $\langle D,T,
               \varnothing,
               \langle\vdash\exists x ( \exists y \forall z (\varphi \to z = y
                        ) \to
                        \forall z ( z \in x \leftrightarrow \exists x ( x \in
                        y \wedge \forall y\,\varphi ) ) )\rangle\rangle$
      \item[Un] $\langle D,T,
               \varnothing,
               \langle\vdash \exists x \forall y ( \exists x ( y \in x \wedge
               x \in z ) \to y \in x ) \rangle\rangle$
      \item[Pow] $\langle D,T,
               \varnothing,
               \langle\vdash \exists x \forall y ( \forall x ( x \in y \to x
               \in z ) \to y \in x ) \rangle\rangle$
      \item[Reg] $\langle D,T,
               \varnothing,
               \langle\vdash (  x \in y \to
                 \exists x ( x \in y \wedge \forall z ( z \in x \to \lnot z
                \in y ) ) ) \rangle\rangle$
      \item[Inf] $\langle D,T,
               \varnothing,
               \langle\vdash \exists x(y\in x\wedge\forall y(y\in
               x\to
               \exists z(y \in z\wedge z\in x))) \rangle\rangle$
      \item[AC] $\langle D,T,
               \varnothing,
               \langle\vdash \exists x \forall y \forall z ( ( y \in z
               \wedge z \in w ) \to \exists w \forall y ( \exists w
              ( ( y \in z \wedge z \in w ) \wedge ( y \in w \wedge w \in x
              ) ) \leftrightarrow y = w ) ) \rangle\rangle$
\end{list}

\subsection{Example~6---Class Notation in Set Theory}\label{class}

A powerful device that makes set theory easier (and that we have
been using all along in our informal expository language) is {\em class
abstraction notation}.\index{class abstraction}\index{abstraction class}  The
definitions we introduce are rigorously justified
as conservative by Takeuti and Zaring \cite{Takeuti}\index{Takeuti, Gaisi} or
Quine \cite{Quine}\index{Quine, Willard Van Orman}.  The key idea is to
introduce the notation $\{x|\mbox{---}\}$ which means ``the class of all $x$
such that ---'' for abstraction classes and introduce (meta)variables that
range over them.  An abstraction class may or may not be a set, depending on
whether it exists (as a set).  A class that does not exist is
called a {\em proper class}.\index{proper class}\index{class!proper}

To illustrate the use of abstraction classes we will provide some examples
of definitions that make use of them:  the empty set, class union, and
unordered pair.  Many other such definitions can be found in the
Metamath set theory database,
\texttt{set.mm}.\index{set theory database (\texttt{set.mm})}

% We intentionally break up the sequence of math symbols here
% because otherwise the overlong line goes beyond the page in narrow mode.
We extend $\mbox{\em CN}$ of Example~5 with new symbols $\{$
$\mbox{class},$ $\{,$ $|,$ $\},$ $\varnothing,$ $\cup,$ $,$ $\}$
where the inner braces and last comma are
constant symbols. (As before,
our dual use of some mathematical symbols for both our expository
language and as primitives of the formal system should be clear from context.)

We extend $\mbox{\em VR}$ of Example~5 with a set of {\em class
variables}\index{class variable}
$\{A,B,C,\ldots\}$. We extend the $T$ of Example~5 with $\{\langle
\mbox{class\ } A\rangle, \langle \mbox{class\ }B\rangle, \langle \mbox{class\ }
C\rangle,\ldots\}$.

To
introduce our definitions,
we add to $\Gamma$ of Example~5 the axiomatic statements
that are the reducts of the following pre-statements:
\begin{list}{}{\itemsep 0.0pt}
      \item[] $\langle\varnothing,T,\varnothing,
               \langle \mbox{class\ }x\rangle\rangle$
      \item[] $\langle\varnothing,T,\varnothing,
               \langle \mbox{class\ }\{x|\varphi\}\rangle\rangle$
      \item[] $\langle\varnothing,T,\varnothing,
               \langle \mbox{wff\ }A=B\rangle\rangle$
      \item[] $\langle\varnothing,T,\varnothing,
               \langle \mbox{wff\ }A\in B\rangle\rangle$
      \item[Ab] $\langle\varnothing,T,\varnothing,
               \langle \vdash ( y \in \{ x |\varphi\} \leftrightarrow
                  ( ( x = y \to\varphi) \wedge \exists x ( x = y
                  \wedge\varphi) ))
               \rangle\rangle$
      \item[Eq] $\langle\{\{x,A\},\{x,B\}\},T,\varnothing,
               \langle \vdash ( A = B \leftrightarrow
               \forall x ( x \in A \leftrightarrow x \in B ) )
               \rangle\rangle$
      \item[El] $\langle\{\{x,A\},\{x,B\}\},T,\varnothing,
               \langle \vdash ( A \in B \leftrightarrow \exists x
               ( x = A \wedge x \in B ) )
               \rangle\rangle$
\end{list}
Here we say that an individual variable is a class; $\{x|\varphi\}$ is a
class; and we extend the definition of a wff to include class equality and
membership.  Axiom Ab defines membership of a variable in a class abstraction;
the right-hand side can be read as ``the wff that results from proper
substitution of $y$ for $x$ in $\varphi$.''\footnote{Note that this definition
makes unnecessary the introduction of a separate notation similar to
$\varphi(x|y)$ for proper substitution, although we may choose to do so to be
conventional.  Incidentally, $\varphi(x|y)$ as it stands would be ambiguous in
the formal systems of our examples, since we wouldn't know whether
$\lnot\varphi(x|y)$ meant $\lnot(\varphi(x|y))$ or $(\lnot\varphi)(x|y)$.
Instead, we would have to use an unambiguous variant such as $(\varphi\,
x|y)$.}  Axioms Eq and El extend the meaning of the existing equality and
membership connectives.  This is potentially dangerous and requires careful
justification.  For example, from Eq we can derive the Axiom of Extensionality
with predicate logic alone; thus in principle we should include the Axiom of
Extensionality as a logical hypothesis.  However we do not bother to do this
since we have already presupposed that axiom earlier. The distinct variable
restrictions should be read ``where $x$ does not occur in $A$ or $B$.''  We
typically do this when the right-hand side of a definition involves an
individual variable not in the expression being defined; it is done so that
the right-hand side remains independent of the particular ``dummy'' variable
we use.

We continue to add to $\Gamma$ the following definitions
(i.e. the reducts of the following pre-statements) for empty
set,\index{empty set} class union,\index{union} and unordered
pair.\index{unordered pair}  They should be self-explanatory.  Analogous to our
use of ``$\leftrightarrow$'' to define new wffs in Example~4, we use ``$=$''
to define new abstraction terms, and both may be read informally as ``is
defined as'' in this context.
\begin{list}{}{\itemsep 0.0pt}
      \item[] $\langle\varnothing,T,\varnothing,
               \langle \mbox{class\ }\varnothing\rangle\rangle$
      \item[] $\langle\varnothing,T,\varnothing,
               \langle \vdash \varnothing = \{ x | \lnot x = x \}
               \rangle\rangle$
      \item[] $\langle\varnothing,T,\varnothing,
               \langle \mbox{class\ }(A\cup B)\rangle\rangle$
      \item[] $\langle\{\{x,A\},\{x,B\}\},T,\varnothing,
               \langle \vdash ( A \cup B ) = \{ x | ( x \in A \vee x \in B ) \}
               \rangle\rangle$
      \item[] $\langle\varnothing,T,\varnothing,
               \langle \mbox{class\ }\{A,B\}\rangle\rangle$
      \item[] $\langle\{\{x,A\},\{x,B\}\},T,\varnothing,
               \langle \vdash \{ A , B \} = \{ x | ( x = A \vee x = B ) \}
               \rangle\rangle$
\end{list}

\section{Metamath as a Formal System}\label{theorymm}

This section presupposes a familiarity with the Metamath computer language.

Our theory describes formal systems and their universes.  The Metamath
language provides a way of representing these set-theoretical objects to
a computer.  A Metamath database, being a finite set of {\sc ascii}
characters, can usually describe only a subset of a formal system and
its universe, which are typically infinite.  However the database can
contain as large a finite subset of the formal system and its universe
as we wish.  (Of course a Metamath set theory database can, in
principle, indirectly describe an entire infinite formal system by
formalizing the expository language in this Appendix.)

For purpose of our discussion, we assume the Metamath database
is in the simple form described on p.~\pageref{framelist},
consisting of all constant and variable declarations at the beginning,
followed by a sequence of extended frames each
delimited by \texttt{\$\char`\{} and \texttt{\$\char`\}}.  Any Metamath database can
be converted to this form, as described on p.~\pageref{frameconvert}.

The math symbol tokens of a Metamath source file, which are declared
with \texttt{\$c} and \texttt{\$v} statements, are names we assign to
representatives of $\mbox{\em CN}$ and $\mbox{\em VR}$.  For
definiteness we could assume that the first math symbol declared as a
variable corresponds to $v_0$, the second to $v_1$, etc., although the
exact correspondence we choose is not important.

In the Metamath language, each \texttt{\$d}, \texttt{\$f}, and
 \texttt{\$e} source
statement in an extended frame (Section~\ref{frames})
corresponds respectively to a member of the
collections $D$, $T$, and $H$ in a formal system statement $\langle
D_M,T_M,H,A\rangle$.  The math symbol strings following these Metamath keywords
correspond to a variable pair (in the case of \texttt{\$d}) or an expression (for
the other two keywords). The math symbol string following a \texttt{\$a} source
statement corresponds to expression $A$ in an axiomatic statement of the
formal system; the one following a \texttt{\$p} source statement corresponds to
$A$ in a provable statement that is not axiomatic.  In other words, each
extended frame in a Metamath database corresponds to
a pre-statement of the formal system, and a frame corresponds to
a statement of the formal system.  (Don't confuse the two meanings of
``statement'' here.  A statement of the formal system corresponds to the
several statements in a Metamath database that may constitute a
frame.)

In order for the computer to verify that a formal system statement is
provable, each \texttt{\$p} source statement is accompanied by a proof.
However, the proof does not correspond to anything in the formal system
but is simply a way of communicating to the computer the information
needed for its verification.  The proof tells the computer {\em how to
construct} specific members of closure of the formal system
pre-statement corresponding to the extended frame of the \texttt{\$p}
statement.  The final result of the construction is the member of the
closure that matches the \texttt{\$p} statement.  The abstract formal
system, on the other hand, is concerned only with the {\em existence} of
members of the closure.

As mentioned on p.~\pageref{exampleref}, Examples 1 and 3--6 in the
previous Section parallel the development of logic and set theory in the
Metamath database
\texttt{set.mm}.\index{set theory database (\texttt{set.mm})} You may
find it instructive to compare them.


\chapter{The MIU System}
\label{MIU}
\index{formal system}
\index{MIU-system}

The following is a listing of the file \texttt{miu.mm}.  It is self-explanatory.

%%%%%%%%%%%%%%%%%%%%%%%%%%%%%%%%%%%%%%%%%%%%%%%%%%%%%%%%%%%%

\begin{verbatim}
$( The MIU-system:  A simple formal system $)

$( Note:  This formal system is unusual in that it allows
empty wffs.  To work with a proof, you must type
SET EMPTY_SUBSTITUTION ON before using the PROVE command.
By default, this is OFF in order to reduce the number of
ambiguous unification possibilities that have to be selected
during the construction of a proof.  $)

$(
Hofstadter's MIU-system is a simple example of a formal
system that illustrates some concepts of Metamath.  See
Douglas R. Hofstadter, _Goedel, Escher, Bach:  An Eternal
Golden Braid_ (Vintage Books, New York, 1979), pp. 33ff. for
a description of the MIU-system.

The system has 3 constant symbols, M, I, and U.  The sole
axiom of the system is MI. There are 4 rules:
     Rule I:  If you possess a string whose last letter is I,
     you can add on a U at the end.
     Rule II:  Suppose you have Mx.  Then you may add Mxx to
     your collection.
     Rule III:  If III occurs in one of the strings in your
     collection, you may make a new string with U in place
     of III.
     Rule IV:  If UU occurs inside one of your strings, you
     can drop it.
Unfortunately, Rules III and IV do not have unique results:
strings could have more than one occurrence of III or UU.
This requires that we introduce the concept of an "MIU
well-formed formula" or wff, which allows us to construct
unique symbol sequences to which Rules III and IV can be
applied.
$)

$( First, we declare the constant symbols of the language.
Note that we need two symbols to distinguish the assertion
that a sequence is a wff from the assertion that it is a
theorem; we have arbitrarily chosen "wff" and "|-". $)
      $c M I U |- wff $. $( Declare constants $)

$( Next, we declare some variables. $)
     $v x y $.

$( Throughout our theory, we shall assume that these
variables represent wffs. $)
 wx   $f wff x $.
 wy   $f wff y $.

$( Define MIU-wffs.  We allow the empty sequence to be a
wff. $)

$( The empty sequence is a wff. $)
 we   $a wff $.
$( "M" after any wff is a wff. $)
 wM   $a wff x M $.
$( "I" after any wff is a wff. $)
 wI   $a wff x I $.
$( "U" after any wff is a wff. $)
 wU   $a wff x U $.

$( Assert the axiom. $)
 ax   $a |- M I $.

$( Assert the rules. $)
 ${
   Ia   $e |- x I $.
$( Given any theorem ending with "I", it remains a theorem
if "U" is added after it.  (We distinguish the label I_
from the math symbol I to conform to the 24-Jun-2006
Metamath spec.) $)
   I_    $a |- x I U $.
 $}
 ${
IIa  $e |- M x $.
$( Given any theorem starting with "M", it remains a theorem
if the part after the "M" is added again after it. $)
   II   $a |- M x x $.
 $}
 ${
   IIIa $e |- x I I I y $.
$( Given any theorem with "III" in the middle, it remains a
theorem if the "III" is replaced with "U". $)
   III  $a |- x U y $.
 $}
 ${
   IVa  $e |- x U U y $.
$( Given any theorem with "UU" in the middle, it remains a
theorem if the "UU" is deleted. $)
   IV   $a |- x y $.
  $}

$( Now we prove the theorem MUIIU.  You may be interested in
comparing this proof with that of Hofstadter (pp. 35 - 36).
$)
 theorem1  $p |- M U I I U $=
      we wM wU wI we wI wU we wU wI wU we wM we wI wU we wM
      wI wI wI we wI wI we wI ax II II I_ III II IV $.
\end{verbatim}\index{well-formed formula (wff)}

The \texttt{show proof /lemmon/renumber} command
yields the following display.  It is very similar
to the one in \cite[pp.~35--36]{Hofstadter}.\index{Hofstadter, Douglas R.}

\begin{verbatim}
1 ax             $a |- M I
2 1 II           $a |- M I I
3 2 II           $a |- M I I I I
4 3 I_           $a |- M I I I I U
5 4 III          $a |- M U I U
6 5 II           $a |- M U I U U I U
7 6 IV           $a |- M U I I U
\end{verbatim}

We note that Hofstadter's ``MU-puzzle,'' which asks whether
MU is a theorem of the MIU-system, cannot be answered using
the system above because the MU-puzzle is a question {\em
about} the system.  To prove the answer to the MU-puzzle,
a much more elaborate system is needed, namely one that
models the MIU-system within set theory.  (Incidentally, the
answer to the MU-puzzle is no.)

\chapter{Metamath Language EBNF}%
\label{BNF}%
\index{Metamath Language EBNF}

The following is a formal description of the basic Metamath language syntax
(with compressed proofs and support for unknown proof steps).
It is defined using the
Extended Backus--Naur Form (EBNF)\index{Extended Backus--Naur Form}\index{EBNF}
notation from W3C\index{W3C}
\textit{Extensible Markup Language (XML) 1.0 (Fifth Edition)}
(W3C Recommendation 26 November 2008) at
\url{https://www.w3.org/TR/xml/#sec-notation}.

The \texttt{database}
rule is processed until the end of the file (\texttt{EOF}).
The rules eventually require reading whitespace-separated tokens.
A token has an upper-case definition (see below)
or is a string constant in a non-token (such as \texttt{'\$a'}).
We intend for this to be correct, but if there is a conflict the
rules of section \ref{spec} govern. That section also discusses
non-syntax restrictions not shown here
(e.g., that each new label token
defined in a \texttt{hypothesis-stmt} or \texttt{assert-stmt}
must be unique).

\begin{verbatim}
database ::= outermost-scope-stmt*

outermost-scope-stmt ::=
  include-stmt | constant-stmt | stmt

/* File inclusion command; process file as a database.
   Databases should NOT have a comment in the filename. */
include-stmt ::= '$[' filename '$]'

/* Constant symbols declaration. */
constant-stmt ::= '$c' constant+ '$.'

/* A normal statement can occur in any scope. */
stmt ::= block | variable-stmt | disjoint-stmt |
  hypothesis-stmt | assert-stmt

/* A block. You can have 0 statements in a block. */
block ::= '${' stmt* '$}'

/* Variable symbols declaration. */
variable-stmt ::= '$v' variable+ '$.'

/* Disjoint variables. Simple disjoint statements have
   2 variables, i.e., "variable*" is empty for them. */
disjoint-stmt ::= '$d' variable variable variable* '$.'

hypothesis-stmt ::= floating-stmt | essential-stmt

/* Floating (variable-type) hypothesis. */
floating-stmt ::= LABEL '$f' typecode variable '$.'

/* Essential (logical) hypothesis. */
essential-stmt ::= LABEL '$e' typecode MATH-SYMBOL* '$.'

assert-stmt ::= axiom-stmt | provable-stmt

/* Axiomatic assertion. */
axiom-stmt ::= LABEL '$a' typecode MATH-SYMBOL* '$.'

/* Provable assertion. */
provable-stmt ::= LABEL '$p' typecode MATH-SYMBOL*
  '$=' proof '$.'

/* A proof. Proofs may be interspersed by comments.
   If '?' is in a proof it's an "incomplete" proof. */
proof ::= uncompressed-proof | compressed-proof
uncompressed-proof ::= (LABEL | '?')+
compressed-proof ::= '(' LABEL* ')' COMPRESSED-PROOF-BLOCK+

typecode ::= constant

filename ::= MATH-SYMBOL /* No whitespace or '$' */
constant ::= MATH-SYMBOL
variable ::= MATH-SYMBOL
\end{verbatim}

\needspace{2\baselineskip}
A \texttt{frame} is a sequence of 0 or more
\texttt{disjoint-{\allowbreak}stmt} and
\texttt{hypotheses-{\allowbreak}stmt} statements
(possibly interleaved with other non-\texttt{assert-stmt} statements)
followed by one \texttt{assert-stmt}.

\needspace{3\baselineskip}
Here are the rules for lexical processing (tokenization) beyond
the constant tokens shown above.
By convention these tokenization rules have upper-case names.
Every token is read for the longest possible length.
Whitespace-separated tokens are read sequentially;
note that the separating whitespace and \texttt{\$(} ... \texttt{\$)}
comments are skipped.

If a token definition uses another token definition, the whole thing
is considered a single token.
A pattern that is only part of a full token has a name beginning
with an underscore (``\_'').
An implementation could tokenize many tokens as a
\texttt{PRINTABLE-SEQUENCE}
and then check if it meets the more specific rule shown here.

Comments do not nest, and both \texttt{\$(} and \texttt{\$)}
have to be surrounded
by at least one whitespace character (\texttt{\_WHITECHAR}).
Technically comments end without consuming the trailing
\texttt{\_WHITECHAR}, but the trailing
\texttt{\_WHITECHAR} gets ignored anyway so we ignore that detail here.
Metamath language processors
are not required to support \texttt{\$)} followed
immediately by a bare end-of-file, because the closing
comment symbol is supposed to be followed by a
\texttt{\_WHITECHAR} such as a newline.

\begin{verbatim}
PRINTABLE-SEQUENCE ::= _PRINTABLE-CHARACTER+

MATH-SYMBOL ::= (_PRINTABLE-CHARACTER - '$')+

/* ASCII non-whitespace printable characters */
_PRINTABLE-CHARACTER ::= [#x21-#x7e]

LABEL ::= ( _LETTER-OR-DIGIT | '.' | '-' | '_' )+

_LETTER-OR-DIGIT ::= [A-Za-z0-9]

COMPRESSED-PROOF-BLOCK ::= ([A-Z] | '?')+

/* Define whitespace between tokens. The -> SKIP
   means that when whitespace is seen, it is
   skipped and we simply read again. */
WHITESPACE ::= (_WHITECHAR+ | _COMMENT) -> SKIP

/* Comments. $( ... $) and do not nest. */
_COMMENT ::= '$(' (_WHITECHAR+ (PRINTABLE-SEQUENCE - '$)'))*
  _WHITECHAR+ '$)' _WHITECHAR

/* Whitespace: (' ' | '\t' | '\r' | '\n' | '\f') */
_WHITECHAR ::= [#x20#x09#x0d#x0a#x0c]
\end{verbatim}
% This EBNF was developed as a collaboration between
% David A. Wheeler\index{Wheeler, David A.},
% Mario Carneiro\index{Carneiro, Mario}, and
% Benoit Jubin\index{Jubin, Benoit}, inspired by a request
% (and a lot of initial work) by Benoit Jubin.
%
% \chapter{Disclaimer and Trademarks}
%
% Information in this document is subject to change without notice and does not
% represent a commitment on the part of Norman Megill.
% \vspace{2ex}
%
% \noindent Norman D. Megill makes no warranties, either express or implied,
% regarding the Metamath computer software package.
%
% \vspace{2ex}
%
% \noindent Any trademarks mentioned in this book are the property of
% their respective owners.  The name ``Metamath'' is a trademark of
% Norman Megill.
%
\cleardoublepage
\phantomsection  % fixes the link anchor
\addcontentsline{toc}{chapter}{\bibname}

\bibliography{metamath}
%\input{metamath.bbl}

\raggedright
\cleardoublepage
\phantomsection % fixes the link anchor
\addcontentsline{toc}{chapter}{\indexname}
%\printindex   ??
\input{metamath.ind}

\end{document}



\raggedright
\cleardoublepage
\phantomsection % fixes the link anchor
\addcontentsline{toc}{chapter}{\indexname}
%\printindex   ??
% metamath.tex - Version of 2-Jun-2019
% If you change the date above, also change the "Printed date" below.
% SPDX-License-Identifier: CC0-1.0
%
%                              PUBLIC DOMAIN
%
% This file (specifically, the version of this file with the above date)
% has been released into the Public Domain per the
% Creative Commons CC0 1.0 Universal (CC0 1.0) Public Domain Dedication
% https://creativecommons.org/publicdomain/zero/1.0/
%
% The public domain release applies worldwide.  In case this is not
% legally possible, the right is granted to use the work for any purpose,
% without any conditions, unless such conditions are required by law.
%
% Several short, attributed quotations from copyrighted works
% appear in this file under the ``fair use'' provision of Section 107 of
% the United States Copyright Act (Title 17 of the {\em United States
% Code}).  The public-domain status of this file is not applicable to
% those quotations.
%
% Norman Megill - email: nm(at)alum(dot)mit(dot)edu
%
% David A. Wheeler also donates his improvements to this file to the
% public domain per the CC0.  He works at the Institute for Defense Analyses
% (IDA), but IDA has agreed that this Metamath work is outside its "lane"
% and is not a work by IDA.  This was specifically confirmed by
% Margaret E. Myers (Division Director of the Information Technology
% and Systems Division) on 2019-05-24 and by Ben Lindorf (General Counsel)
% on 2019-05-22.

% This file, 'metamath.tex', is self-contained with everything needed to
% generate the the PDF file 'metamath.pdf' (the _Metamath_ book) on
% standard LaTeX 2e installations.  The auxiliary files are embedded with
% "filecontents" commands.  To generate metamath.pdf file, run these
% commands under Linux or Cygwin in the directory that contains
% 'metamath.tex':
%
%   rm -f realref.sty metamath.bib
%   touch metamath.ind
%   pdflatex metamath
%   pdflatex metamath
%   bibtex metamath
%   makeindex metamath
%   pdflatex metamath
%   pdflatex metamath
%
% The warnings that occur in the initial runs of pdflatex can be ignored.
% For the final run,
%
%   egrep -i 'error|warn' metamath.log
%
% should show exactly these 5 warnings:
%
%   LaTeX Warning: File `realref.sty' already exists on the system.
%   LaTeX Warning: File `metamath.bib' already exists on the system.
%   LaTeX Font Warning: Font shape `OMS/cmtt/m/n' undefined
%   LaTeX Font Warning: Font shape `OMS/cmtt/bx/n' undefined
%   LaTeX Font Warning: Some font shapes were not available, defaults
%       substituted.
%
% Search for "Uncomment" below if you want to suppress hyperlink boxes
% in the PDF output file
%
% TYPOGRAPHICAL NOTES:
% * It is customary to use an en dash (--) to "connect" names of different
%   people (and to denote ranges), and use a hyphen (-) for a
%   single compound name. Examples of connected multiple people are
%   Zermelo--Fraenkel, Schr\"{o}der--Bernstein, Tarski--Grothendieck,
%   Hewlett--Packard, and Backus--Naur.  Examples of a single person with
%   a compound name include Levi-Civita, Mittag-Leffler, and Burali-Forti.
% * Use non-breaking spaces after page abbreviations, e.g.,
%   p.~\pageref{note2002}.
%
% --------------------------- Start of realref.sty -----------------------------
\begin{filecontents}{realref.sty}
% Save the following as realref.sty.
% You can then use it with \usepackage{realref}
%
% This has \pageref jumping to the page on which the ref appears,
% \ref jumping to the point of the anchor, and \sectionref
% jumping to the start of section.
%
% Author:  Anthony Williams
%          Software Engineer
%          Nortel Networks Optical Components Ltd
% Date:    9 Nov 2001 (posted to comp.text.tex)
%
% The following declaration was made by Anthony Williams on
% 24 Jul 2006 (private email to Norman Megill):
%
%   ``I hereby donate the code for realref.sty posted on the
%   comp.text.tex newsgroup on 9th November 2001, accessible from
%   http://groups.google.com/group/comp.text.tex/msg/5a0e1cc13ea7fbb2
%   to the public domain.''
%
\ProvidesPackage{realref}
\RequirePackage[plainpages=false,pdfpagelabels=true]{hyperref}
\def\realref@anchorname{}
\AtBeginDocument{%
% ensure every label is a possible hyperlink target
\let\realref@oldrefstepcounter\refstepcounter%
\DeclareRobustCommand{\refstepcounter}[1]{\realref@oldrefstepcounter{#1}
\edef\realref@anchorname{\string #1.\@currentlabel}%
}%
\let\realref@oldlabel\label%
\DeclareRobustCommand{\label}[1]{\realref@oldlabel{#1}\hypertarget{#1}{}%
\@bsphack\protected@write\@auxout{}{%
    \string\expandafter\gdef\protect\csname
    page@num.#1\string\endcsname{\thepage}%
    \string\expandafter\gdef\protect\csname
    ref@num.#1\string\endcsname{\@currentlabel}%
    \string\expandafter\gdef\protect\csname
    sectionref@name.#1\string\endcsname{\realref@anchorname}%
}\@esphack}%
\DeclareRobustCommand\pageref[1]{{\edef\a{\csname
            page@num.#1\endcsname}\expandafter\hyperlink{page.\a}{\a}}}%
\DeclareRobustCommand\ref[1]{{\edef\a{\csname
            ref@num.#1\endcsname}\hyperlink{#1}{\a}}}%
\DeclareRobustCommand\sectionref[1]{{\edef\a{\csname
            ref@num.#1\endcsname}\edef\b{\csname
            sectionref@name.#1\endcsname}\hyperlink{\b}{\a}}}%
}
\end{filecontents}
% ---------------------------- End of realref.sty ------------------------------

% --------------------------- Start of metamath.bib -----------------------------
\begin{filecontents}{metamath.bib}
@book{Albers, editor = "Donald J. Albers and G. L. Alexanderson",
  title = "Mathematical People",
  publisher = "Contemporary Books, Inc.",
  address = "Chicago",
  note = "[QA28.M37]",
  year = 1985 }
@book{Anderson, author = "Alan Ross Anderson and Nuel D. Belnap",
  title = "Entailment",
  publisher = "Princeton University Press",
  address = "Princeton",
  volume = 1,
  note = "[QA9.A634 1975 v.1]",
  year = 1975}
@book{Barrow, author = "John D. Barrow",
  title = "Theories of Everything:  The Quest for Ultimate Explanation",
  publisher = "Oxford University Press",
  address = "Oxford",
  note = "[Q175.B225]",
  year = 1991 }
@book{Behnke,
  editor = "H. Behnke and F. Backmann and K. Fladt and W. S{\"{u}}ss",
  title = "Fundamentals of Mathematics",
  volume = "I",
  publisher = "The MIT Press",
  address = "Cambridge, Massachusetts",
  note = "[QA37.2.B413]",
  year = 1974 }
@book{Bell, author = "J. L. Bell and M. Machover",
  title = "A Course in Mathematical Logic",
  publisher = "North-Holland",
  address = "Amsterdam",
  note = "[QA9.B3953]",
  year = 1977 }
@inproceedings{Blass, author = "Andrea Blass",
  title = "The Interaction Between Category Theory and Set Theory",
  pages = "5--29",
  booktitle = "Mathematical Applications of Category Theory (Proceedings
     of the Special Session on Mathematical Applications
     Category Theory, 89th Annual Meeting of the American Mathematical
     Society, held in Denver, Colorado January 5--9, 1983)",
  editor = "John Walter Gray",
  year = 1983,
  note = "[QA169.A47 1983]",
  publisher = "American Mathematical Society",
  address = "Providence, Rhode Island"}
@proceedings{Bledsoe, editor = "W. W. Bledsoe and D. W. Loveland",
  title = "Automated Theorem Proving:  After 25 Years (Proceedings
     of the Special Session on Automatic Theorem Proving,
     89th Annual Meeting of the American Mathematical
     Society, held in Denver, Colorado January 5--9, 1983)",
  year = 1983,
  note = "[QA76.9.A96.S64 1983]",
  publisher = "American Mathematical Society",
  address = "Providence, Rhode Island" }
@book{Boolos, author = "George S. Boolos and Richard C. Jeffrey",
  title = "Computability and Log\-ic",
  publisher = "Cambridge University Press",
  edition = "third",
  address = "Cambridge",
  note = "[QA9.59.B66 1989]",
  year = 1989 }
@book{Campbell, author = "John Campbell",
  title = "Programmer's Progress",
  publisher = "White Star Software",
  address = "Box 51623, Palo Alto, CA 94303",
  year = 1991 }
@article{DBLP:journals/corr/Carneiro14,
  author    = {Mario Carneiro},
  title     = {Conversion of {HOL} Light proofs into Metamath},
  journal   = {CoRR},
  volume    = {abs/1412.8091},
  year      = {2014},
  url       = {http://arxiv.org/abs/1412.8091},
  archivePrefix = {arXiv},
  eprint    = {1412.8091},
  timestamp = {Mon, 13 Aug 2018 16:47:05 +0200},
  biburl    = {https://dblp.org/rec/bib/journals/corr/Carneiro14},
  bibsource = {dblp computer science bibliography, https://dblp.org}
}
@article{CarneiroND,
  author    = {Mario Carneiro},
  title     = {Natural Deductions in the Metamath Proof Language},
  url       = {http://us.metamath.org/ocat/natded.pdf},
  year      = 2014
}
@inproceedings{Chou, author = "Shang-Ching Chou",
  title = "Proving Elementary Geometry Theorems Using {W}u's Algorithm",
  pages = "243--286",
  booktitle = "Automated Theorem Proving:  After 25 Years (Proceedings
     of the Special Session on Automatic Theorem Proving,
     89th Annual Meeting of the American Mathematical
     Society, held in Denver, Colorado January 5--9, 1983)",
  editor = "W. W. Bledsoe and D. W. Loveland",
  year = 1983,
  note = "[QA76.9.A96.S64 1983]",
  publisher = "American Mathematical Society",
  address = "Providence, Rhode Island" }
@book{Clemente, author = "Daniel Clemente Laboreo",
  title = "Introduction to natural deduction",
  year = 2014,
  url = "http://www.danielclemente.com/logica/dn.en.pdf" }
@incollection{Courant, author = "Richard Courant and Herbert Robbins",
  title = "Topology",
  pages = "573--590",
  booktitle = "The World of Mathematics, Volume One",
  editor = "James R. Newman",
  publisher = "Simon and Schuster",
  address = "New York",
  note = "[QA3.W67 1988]",
  year = 1956 }
@book{Curry, author = "Haskell B. Curry",
  title = "Foundations of Mathematical Logic",
  publisher = "Dover Publications, Inc.",
  address = "New York",
  note = "[QA9.C976 1977]",
  year = 1977 }
@book{Davis, author = "Philip J. Davis and Reuben Hersh",
  title = "The Mathematical Experience",
  publisher = "Birkh{\"{a}}user Boston",
  address = "Boston",
  note = "[QA8.4.D37 1982]",
  year = 1981 }
@incollection{deMillo,
  author = "Richard de Millo and Richard Lipton and Alan Perlis",
  title = "Social Processes and Proofs of Theorems and Programs",
  pages = "267--285",
  booktitle = "New Directions in the Philosophy of Mathematics",
  editor = "Thomas Tymoczko",
  publisher = "Birkh{\"{a}}user Boston, Inc.",
  address = "Boston",
  note = "[QA8.6.N48 1986]",
  year = 1986 }
@book{Edwards, author = "Robert E. Edwards",
  title = "A Formal Background to Mathematics",
  publisher = "Springer-Verlag",
  address = "New York",
  note = "[QA37.2.E38 v.1a]",
  year = 1979 }
@book{Enderton, author = "Herbert B. Enderton",
  title = "Elements of Set Theory",
  publisher = "Academic Press, Inc.",
  address = "San Diego",
  note = "[QA248.E5]",
  year = 1977 }
@book{Goodstein, author = "R. L. Goodstein",
  title = "Development of Mathematical Logic",
  publisher = "Springer-Verlag New York Inc.",
  address = "New York",
  note = "[QA9.G6554]",
  year = 1971 }
@book{Guillen, author = "Michael Guillen",
  title = "Bridges to Infinity",
  publisher = "Jeremy P. Tarcher, Inc.",
  address = "Los Angeles",
  note = "[QA93.G8]",
  year = 1983 }
@book{Hamilton, author = "Alan G. Hamilton",
  title = "Logic for Mathematicians",
  edition = "revised",
  publisher = "Cambridge University Press",
  address = "Cambridge",
  note = "[QA9.H298]",
  year = 1988 }
@unpublished{Harrison, author = "John Robert Harrison",
  title = "Metatheory and Reflection in Theorem Proving:
    A Survey and Critique",
  note = "Technical Report
    CRC-053.
    SRI Cambridge,
    Millers Yard, Cambridge, UK,
    1995.
    Available on the Web as
{\verb+http:+}\-{\verb+//www.cl.cam.ac.uk/users/jrh/papers/reflect.html+}"}
@TECHREPORT{Harrison-thesis,
        author          = "John Robert Harrison",
        title           = "Theorem Proving with the Real Numbers",
        institution   = "University of Cambridge Computer
                         Lab\-o\-ra\-to\-ry",
        address         = "New Museums Site, Pembroke Street, Cambridge,
                           CB2 3QG, UK",
        year            = 1996,
        number          = 408,
        type            = "Technical Report",
        note            = "Author's PhD thesis,
   available on the Web at
{\verb+http:+}\-{\verb+//www.cl.cam.ac.uk+}\-{\verb+/users+}\-{\verb+/jrh+}%
\-{\verb+/papers+}\-{\verb+/thesis.html+}"}
@book{Herrlich, author = "Horst Herrlich and George E. Strecker",
  title = "Category Theory:  An Introduction",
  publisher = "Allyn and Bacon Inc.",
  address = "Boston",
  note = "[QA169.H567]",
  year = 1973 }
@article{Hindley, author = "J. Roger Hindley and David Meredith",
  title = "Principal Type-Schemes and Condensed Detachment",
  journal = "The Journal of Symbolic Logic",
  volume = 55,
  year = 1990,
  note = "[QA.J87]",
  pages = "90--105" }
@book{Hofstadter, author = "Douglas R. Hofstadter",
  title = "G{\"{o}}del, Escher, Bach",
  publisher = "Basic Books, Inc.",
  address = "New York",
  note = "[QA9.H63 1980]",
  year = 1979 }
@article{Indrzejczak, author= "Andrzej Indrzejczak",
  title = "Natural Deduction, Hybrid Systems and Modal Logic",
  journal = "Trends in Logic",
  volume = 30,
  publisher = "Springer",
  year = 2010 }
@article{Kalish, author = "D. Kalish and R. Montague",
  title = "On {T}arski's Formalization of Predicate Logic with Identity",
  journal = "Archiv f{\"{u}}r Mathematische Logik und Grundlagenfor\-schung",
  volume = 7,
  year = 1965,
  note = "[QA.A673]",
  pages = "81--101" }
@article{Kalman, author = "J. A. Kalman",
  title = "Condensed Detachment as a Rule of Inference",
  journal = "Studia Logica",
  volume = 42,
  number = 4,
  year = 1983,
  note = "[B18.P6.S933]",
  pages = "443-451" }
@book{Kline, author = "Morris Kline",
  title = "Mathematical Thought from Ancient to Modern Times",
  publisher = "Oxford University Press",
  address = "New York",
  note = "[QA21.K516 1990 v.3]",
  year = 1972 }
@book{Klinel, author = "Morris Kline",
  title = "Mathematics, The Loss of Certainty",
  publisher = "Oxford University Press",
  address = "New York",
  note = "[QA21.K525]",
  year = 1980 }
@book{Kramer, author = "Edna E. Kramer",
  title = "The Nature and Growth of Modern Mathematics",
  publisher = "Princeton University Press",
  address = "Princeton, New Jersey",
  note = "[QA93.K89 1981]",
  year = 1981 }
@article{Knill, author = "Oliver Knill",
  title = "Some Fundamental Theorems in Mathematics",
  year = "2018",
  url = "https://arxiv.org/abs/1807.08416" }
@book{Landau, author = "Edmund Landau",
  title = "Foundations of Analysis",
  publisher = "Chelsea Publishing Company",
  address = "New York",
  edition = "second",
  note = "[QA241.L2541 1960]",
  year = 1960 }
@article{Leblanc, author = "Hugues Leblanc",
  title = "On {M}eyer and {L}ambert's Quantificational Calculus {FQ}",
  journal = "The Journal of Symbolic Logic",
  volume = 33,
  year = 1968,
  note = "[QA.J87]",
  pages = "275--280" }
@article{Lejewski, author = "Czeslaw Lejewski",
  title = "On Implicational Definitions",
  journal = "Studia Logica",
  volume = 8,
  year = 1958,
  note = "[B18.P6.S933]",
  pages = "189--208" }
@book{Levy, author = "Azriel Levy",
  title = "Basic Set Theory",
  publisher = "Dover Publications",
  address = "Mineola, NY",
  year = "2002"
}
@book{Margaris, author = "Angelo Margaris",
  title = "First Order Mathematical Logic",
  publisher = "Blaisdell Publishing Company",
  address = "Waltham, Massachusetts",
  note = "[QA9.M327]",
  year = 1967}
@book{Manin, author = "Yu I. Manin",
  title = "A Course in Mathematical Logic",
  publisher = "Springer-Verlag",
  address = "New York",
  note = "[QA9.M29613]",
  year = "1977" }
@article{Mathias, author = "Adrian R. D. Mathias",
  title = "A Term of Length 4,523,659,424,929",
  journal = "Synthese",
  volume = 133,
  year = 2002,
  note = "[Q.S993]",
  pages = "75--86" }
@article{Megill, author = "Norman D. Megill",
  title = "A Finitely Axiomatized Formalization of Predicate Calculus
     with Equality",
  journal = "Notre Dame Journal of Formal Logic",
  volume = 36,
  year = 1995,
  note = "[QA.N914]",
  pages = "435--453" }
@unpublished{Megillc, author = "Norman D. Megill",
  title = "A Shorter Equivalent of the Axiom of Choice",
  month = "June",
  note = "Unpublished",
  year = 1991 }
@article{MegillBunder, author = "Norman D. Megill and Martin W.
    Bunder",
  title = "Weaker {D}-Complete Logics",
  journal = "Journal of the IGPL",
  volume = 4,
  year = 1996,
  pages = "215--225",
  note = "Available on the Web at
{\verb+http:+}\-{\verb+//www.mpi-sb.mpg.de+}\-{\verb+/igpl+}%
\-{\verb+/Journal+}\-{\verb+/V4-2+}\-{\verb+/#Megill+}"}
}
@book{Mendelson, author = "Elliott Mendelson",
  title = "Introduction to Mathematical Logic",
  edition = "second",
  publisher = "D. Van Nostrand Company, Inc.",
  address = "New York",
  note = "[QA9.M537 1979]",
  year = 1979 }
@article{Meredith, author = "David Meredith",
  title = "In Memoriam {C}arew {A}rthur {M}eredith (1904-1976)",
  journal = "Notre Dame Journal of Formal Logic",
  volume = 18,
  year = 1977,
  note = "[QA.N914]",
  pages = "513--516" }
@article{CAMeredith, author = "C. A. Meredith",
  title = "Single Axioms for the Systems ({C},{N}), ({C},{O}) and ({A},{N})
      of the Two-Valued Propositional Calculus",
  journal = "The Journal of Computing Systems",
  volume = 3,
  year = 1953,
  pages = "155--164" }
@article{Monk, author = "J. Donald Monk",
  title = "Provability With Finitely Many Variables",
  journal = "The Journal of Symbolic Logic",
  volume = 27,
  year = 1971,
  note = "[QA.J87]",
  pages = "353--358" }
@article{Monks, author = "J. Donald Monk",
  title = "Substitutionless Predicate Logic With Identity",
  journal = "Archiv f{\"{u}}r Mathematische Logik und Grundlagenfor\-schung",
  volume = 7,
  year = 1965,
  pages = "103--121" }
  %% Took out this from above to prevent LaTeX underfull warning:
  % note = "[QA.A673]",
@book{Moore, author = "A. W. Moore",
  title = "The Infinite",
  publisher = "Routledge",
  address = "New York",
  note = "[BD411.M59]",
  year = 1989}
@book{Munkres, author = "James R. Munkres",
  title = "Topology: A First Course",
  publisher = "Prentice-Hall, Inc.",
  address = "Englewood Cliffs, New Jersey",
  note = "[QA611.M82]",
  year = 1975}
@article{Nemesszeghy, author = "E. Z. Nemesszeghy and E. A. Nemesszeghy",
  title = "On Strongly Creative Definitions:  A Reply to {V}. {F}. {R}ickey",
  journal = "Logique et Analyse (N.\ S.)",
  year = 1977,
  volume = 20,
  note = "[BC.L832]",
  pages = "111--115" }
@unpublished{Nemeti, author = "N{\'{e}}meti, I.",
  title = "Algebraizations of Quantifier Logics, an Overview",
  note = "Version 11.4, preprint, Mathematical Institute, Budapest,
    1994.  A shortened version without proofs appeared in
    ``Algebraizations of quantifier logics, an introductory overview,''
   {\em Studia Logica}, 50:485--569, 1991 [B18.P6.S933]"}
@article{Pavicic, author = "M. Pavi{\v{c}}i{\'{c}}",
  title = "A New Axiomatization of Unified Quantum Logic",
  journal = "International Journal of Theoretical Physics",
  year = 1992,
  volume = 31,
  note = "[QC.I626]",
  pages = "1753 --1766" }
@book{Penrose, author = "Roger Penrose",
  title = "The Emperor's New Mind",
  publisher = "Oxford University Press",
  address = "New York",
  note = "[Q335.P415]",
  year = 1989 }
@book{PetersonI, author = "Ivars Peterson",
  title = "The Mathematical Tourist",
  publisher = "W. H. Freeman and Company",
  address = "New York",
  note = "[QA93.P475]",
  year = 1988 }
@article{Peterson, author = "Jeremy George Peterson",
  title = "An automatic theorem prover for substitution and detachment systems",
  journal = "Notre Dame Journal of Formal Logic",
  volume = 19,
  year = 1978,
  note = "[QA.N914]",
  pages = "119--122" }
@book{Quine, author = "Willard Van Orman Quine",
  title = "Set Theory and Its Logic",
  edition = "revised",
  publisher = "The Belknap Press of Harvard University Press",
  address = "Cambridge, Massachusetts",
  note = "[QA248.Q7 1969]",
  year = 1969 }
@article{Robinson, author = "J. A. Robinson",
  title = "A Machine-Oriented Logic Based on the Resolution Principle",
  journal = "Journal of the Association for Computing Machinery",
  year = 1965,
  volume = 12,
  pages = "23--41" }
@article{RobinsonT, author = "T. Thacher Robinson",
  title = "Independence of Two Nice Sets of Axioms for the Propositional
    Calculus",
  journal = "The Journal of Symbolic Logic",
  volume = 33,
  year = 1968,
  note = "[QA.J87]",
  pages = "265--270" }
@book{Rucker, author = "Rudy Rucker",
  title = "Infinity and the Mind:  The Science and Philosophy of the
    Infinite",
  publisher = "Bantam Books, Inc.",
  address = "New York",
  note = "[QA9.R79 1982]",
  year = 1982 }
@book{Russell, author = "Bertrand Russell",
  title = "Mysticism and Logic, and Other Essays",
  publisher = "Barnes \& Noble Books",
  address = "Totowa, New Jersey",
  note = "[B1649.R963.M9 1981]",
  year = 1981 }
@article{Russell2, author = "Bertrand Russell",
  title = "Recent Work on the Principles of Mathematics",
  journal = "International Monthly",
  volume = 4,
  year = 1901,
  pages = "84"}
@article{Schmidt, author = "Eric Schmidt",
  title = "Reductions in Norman Megill's axiom system for complex numbers",
  url = "http://us.metamath.org/downloads/schmidt-cnaxioms.pdf",
  year = "2012" }
@book{Shoenfield, author = "Joseph R. Shoenfield",
  title = "Mathematical Logic",
  publisher = "Addison-Wesley Publishing Company, Inc.",
  address = "Reading, Massachusetts",
  year = 1967,
  note = "[QA9.S52]" }
@book{Smullyan, author = "Raymond M. Smullyan",
  title = "Theory of Formal Systems",
  publisher = "Princeton University Press",
  address = "Princeton, New Jersey",
  year = 1961,
  note = "[QA248.5.S55]" }
@book{Solow, author = "Daniel Solow",
  title = "How to Read and Do Proofs:  An Introduction to Mathematical
    Thought Process",
  publisher = "John Wiley \& Sons",
  address = "New York",
  year = 1982,
  note = "[QA9.S577]" }
@book{Stark, author = "Harold M. Stark",
  title = "An Introduction to Number Theory",
  publisher = "Markham Publishing Company",
  address = "Chicago",
  note = "[QA241.S72 1978]",
  year = 1970 }
@article{Swart, author = "E. R. Swart",
  title = "The Philosophical Implications of the Four-Color Problem",
  journal = "American Mathematical Monthly",
  year = 1980,
  volume = 87,
  month = "November",
  note = "[QA.A5125]",
  pages = "697--707" }
@book{Szpiro, author = "George G. Szpiro",
  title = "Poincar{\'{e}}'s Prize: The Hundred-Year Quest to Solve One
    of Math's Greatest Puzzles",
  publisher = "Penguin Books Ltd",
  address = "London",
  note = "[QA43.S985 2007]",
  year = 2007}
@book{Takeuti, author = "Gaisi Takeuti and Wilson M. Zaring",
  title = "Introduction to Axiomatic Set Theory",
  edition = "second",
  publisher = "Springer-Verlag New York Inc.",
  address = "New York",
  note = "[QA248.T136 1982]",
  year = 1982}
@inproceedings{Tarski, author = "Alfred Tarski",
  title = "What is Elementary Geometry",
  pages = "16--29",
  booktitle = "The Axiomatic Method, with Special Reference to Geometry and
     Physics (Proceedings of an International Symposium held at the University
     of California, Berkeley, December 26, 1957 --- January 4, 1958)",
  editor = "Leon Henkin and Patrick Suppes and Alfred Tarski",
  year = 1959,
  publisher = "North-Holland Publishing Company",
  address = "Amsterdam"}
@article{Tarski1965, author = "Alfred Tarski",
  title = "A Simplified Formalization of Predicate Logic with Identity",
  journal = "Archiv f{\"{u}}r Mathematische Logik und Grundlagenforschung",
  volume = 7,
  year = 1965,
  note = "[QA.A673]",
  pages = "61--79" }
@book{Tymoczko,
  title = "New Directions in the Philosophy of Mathematics",
  editor = "Thomas Tymoczko",
  publisher = "Birkh{\"{a}}user Boston, Inc.",
  address = "Boston",
  note = "[QA8.6.N48 1986]",
  year = 1986 }
@incollection{Wang,
  author = "Hao Wang",
  title = "Theory and Practice in Mathematics",
  pages = "129--152",
  booktitle = "New Directions in the Philosophy of Mathematics",
  editor = "Thomas Tymoczko",
  publisher = "Birkh{\"{a}}user Boston, Inc.",
  address = "Boston",
  note = "[QA8.6.N48 1986]",
  year = 1986 }
@manual{Webster,
  title = "Webster's New Collegiate Dictionary",
  organization = "G. \& C. Merriam Co.",
  address = "Springfield, Massachusetts",
  note = "[PE1628.W4M4 1977]",
  year = 1977 }
@manual{Whitehead, author = "Alfred North Whitehead",
  title = "An Introduction to Mathematics",
  year = 1911 }
@book{PM, author = "Alfred North Whitehead and Bertrand Russell",
  title = "Principia Mathematica",
  edition = "second",
  publisher = "Cambridge University Press",
  address = "Cambridge",
  year = "1927",
  note = "(3 vols.) [QA9.W592 1927]" }
@article{DBLP:journals/corr/Whalen16,
  author    = {Daniel Whalen},
  title     = {Holophrasm: a neural Automated Theorem Prover for higher-order logic},
  journal   = {CoRR},
  volume    = {abs/1608.02644},
  year      = {2016},
  url       = {http://arxiv.org/abs/1608.02644},
  archivePrefix = {arXiv},
  eprint    = {1608.02644},
  timestamp = {Mon, 13 Aug 2018 16:46:19 +0200},
  biburl    = {https://dblp.org/rec/bib/journals/corr/Whalen16},
  bibsource = {dblp computer science bibliography, https://dblp.org} }
@article{Wiedijk-revisited,
  author = {Freek Wiedijk},
  title = {The QED Manifesto Revisited},
  year = {2007},
  url = {http://mizar.org/trybulec65/8.pdf} }
@book{Wolfram,
  author = "Stephen Wolfram",
  title = "Mathematica:  A System for Doing Mathematics by Computer",
  edition = "second",
  publisher = "Addison-Wesley Publishing Co.",
  address = "Redwood City, California",
  note = "[QA76.95.W65 1991]",
  year = 1991 }
@book{Wos, author = "Larry Wos and Ross Overbeek and Ewing Lusk and Jim Boyle",
  title = "Automated Reasoning:  Introduction and Applications",
  edition = "second",
  publisher = "McGraw-Hill, Inc.",
  address = "New York",
  note = "[QA76.9.A96.A93 1992]",
  year = 1992 }

%
%
%[1] Church, Alonzo, Introduction to Mathematical Logic,
% Volume 1, Princeton University Press, Princeton, N. J., 1956.
%
%[2] Cohen, Paul J., Set Theory and the Continuum Hypothesis,
% W. A. Benjamin, Inc., Reading, Mass., 1966.
%
%[3] Hamilton, Alan G., Logic for Mathematicians, Cambridge
% University Press,
% Cambridge, 1988.

%[6] Kleene, Stephen Cole, Introduction to Metamathematics, D.  Van
% Nostrand Company, Inc., Princeton (1952).

%[13] Tarski, Alfred, "A simplified formalization of predicate
% logic with identity," Archiv fur Mathematische Logik und
% Grundlagenforschung, vol. 7 (1965), pp. 61-79.

%[14] Tarski, Alfred and Steven Givant, A Formalization of Set
% Theory Without Variables, American Mathematical Society Colloquium
% Publications, vol. 41, American Mathematical Society,
% Providence, R. I., 1987.

%[15] Zeman, J. J., Modal Logic, Oxford University Press, Oxford, 1973.
\end{filecontents}
% --------------------------- End of metamath.bib -----------------------------


%Book: Metamath
%Author:  Norman Megill Email:  nm at alum.mit.edu
%Author:  David A. Wheeler Email:  dwheeler at dwheeler.com

% A book template example
% http://www.stsci.edu/ftp/software/tex/bookstuff/book.template

\documentclass[leqno]{book} % LaTeX 2e. 10pt. Use [leqno,12pt] for 12pt
% hyperref 2002/05/27 v6.72r  (couldn't get pagebackref to work)
\usepackage[plainpages=false,pdfpagelabels=true]{hyperref}

\usepackage{needspace}     % Enable control over page breaks
\usepackage{breqn}         % automatic equation breaking
\usepackage{microtype}     % microtypography, reduces hyphenation

% Packages for flexible tables.  We need to be able to
% wrap text within a cell (with automatically-determined widths) AND
% split a table automatically across multiple pages.
% * "tabularx" wraps text in cells but only 1 page
% * "longtable" goes across pages but by itself is incompatible with tabularx
% * "ltxtable" combines longtable and tabularx, but table contents
%    must be in a separate file.
% * "ltablex" combines tabularx and longtable - must install specially
% * "booktabs" is recommended as a way to improve the look of tables,
%   but doesn't add these capabilities.
% * "tabu" much more capable and seems to be recommended. So use that.

\usepackage{makecell}      % Enable forced line splits within a table cell
% v4.13 needed for tabu: https://tex.stackexchange.com/questions/600724/dimension-too-large-after-recent-longtable-update
\usepackage{longtable}[=v4.13] % Enable multi-page tables  
\usepackage{tabu}          % Multi-page tables with wrapped text in a cell

% You can find more Tex packages using commands like:
% tlmgr search --file tabu.sty
% find /usr/share/texmf-dist/ -name '*tab*'
%
%%%%%%%%%%%%%%%%%%%%%%%%%%%%%%%%%%%%%%%%%%%%%%%%%%%%%%%%%%%%%%%%%%%%%%%%%%%%
% Uncomment the next 3 lines to suppress boxes and colors on the hyperlinks
%%%%%%%%%%%%%%%%%%%%%%%%%%%%%%%%%%%%%%%%%%%%%%%%%%%%%%%%%%%%%%%%%%%%%%%%%%%%
%\hypersetup{
%colorlinks,citecolor=black,filecolor=black,linkcolor=black,urlcolor=black
%}
%
\usepackage{realref}

% Restarting page numbers: try?
%   \printglossary
%   \cleardoublepage
%   \pagenumbering{arabic}
%   \setcounter{page}{1}    ???needed
%   \include{chap1}

% not used:
% \def\R2Lurl#1#2{\mbox{\href{#1}\texttt{#2}}}

\usepackage{amssymb}

% Version 1 of book: margins: t=.4, b=.2, ll=.4, rr=.55
% \usepackage{anysize}
% % \papersize{<height>}{<width>}
% % \marginsize{<left>}{<right>}{<top>}{<bottom>}
% \papersize{9in}{6in}
% % l/r 0.6124-0.6170 works t/b 0.2418-0.3411 = 192pp. 0.2926-03118=exact
% \marginsize{0.7147in}{0.5147in}{0.4012in}{0.2012in}

\usepackage{anysize}
% \papersize{<height>}{<width>}
% \marginsize{<left>}{<right>}{<top>}{<bottom>}
\papersize{9in}{6in}
% l/r 0.85in&0.6431-0.6539 works t/b ?-?
%\marginsize{0.85in}{0.6485in}{0.55in}{0.35in}
\marginsize{0.8in}{0.65in}{0.5in}{0.3in}

% \usepackage[papersize={3.6in,4.8in},hmargin=0.1in,vmargin={0.1in,0.1in}]{geometry}  % page geometry
\usepackage{special-settings}

\raggedbottom
\makeindex

\begin{document}
% Discourage page widows and orphans:
\clubpenalty=300
\widowpenalty=300

%%%%%%% load in AMS fonts %%%%%%% % LaTeX 2.09 - obsolete in LaTeX 2e
%\input{amssym.def}
%\input{amssym.tex}
%\input{c:/texmf/tex/plain/amsfonts/amssym.def}
%\input{c:/texmf/tex/plain/amsfonts/amssym.tex}

\bibliographystyle{plain}
\pagenumbering{roman}
\pagestyle{headings}

\thispagestyle{empty}

\hfill
\vfill

\begin{center}
{\LARGE\bf Metamath} \\
\vspace{1ex}
{\large A Computer Language for Mathematical Proofs} \\
\vspace{7ex}
{\large Norman Megill} \\
\vspace{7ex}
with extensive revisions by \\
\vspace{1ex}
{\large David A. Wheeler} \\
\vspace{7ex}
% Printed date. If changing the date below, also fix the date at the beginning.
2019-06-02
\end{center}

\vfill
\hfill

\newpage
\thispagestyle{empty}

\hfill
\vfill

\begin{center}
$\sim$\ {\sc Public Domain}\ $\sim$

\vspace{2ex}
This book (including its later revisions)
has been released into the Public Domain by Norman Megill per the
Creative Commons CC0 1.0 Universal (CC0 1.0) Public Domain Dedication.
David A. Wheeler has done the same.
This public domain release applies worldwide.  In case this is not
legally possible, the right is granted to use the work for any purpose,
without any conditions, unless such conditions are required by law.
See \url{https://creativecommons.org/publicdomain/zero/1.0/}.

\vspace{3ex}
Several short, attributed quotations from copyrighted works
appear in this book under the ``fair use'' provision of Section 107 of
the United States Copyright Act (Title 17 of the {\em United States
Code}).  The public-domain status of this book is not applicable to
those quotations.

\vspace{3ex}
Any trademarks used in this book are the property of their owners.

% QA76.9.L63.M??

% \vspace{1ex}
%
% \vspace{1ex}
% {\small Permission is granted to make and distribute verbatim copies of this
% book
% provided the copyright notice and this
% permission notice are preserved on all copies.}
%
% \vspace{1ex}
% {\small Permission is granted to copy and distribute modified versions of this
% book under the conditions for verbatim copying, provided that the
% entire
% resulting derived work is distributed under the terms of a permission
% notice
% identical to this one.}
%
% \vspace{1ex}
% {\small Permission is granted to copy and distribute translations of this
% book into another language, under the above conditions for modified
% versions,
% except that this permission notice may be stated in a translation
% approved by the
% author.}
%
% \vspace{1ex}
% %{\small   For a copy of the \LaTeX\ source files for this book, contact
% %the author.} \\
% \ \\
% \ \\

\vspace{7ex}
% ISBN: 1-4116-3724-0 \\
% ISBN: 978-1-4116-3724-5 \\
ISBN: 978-0-359-70223-7 \\
{\ } \\
Lulu Press \\
Morrisville, North Carolina\\
USA


\hfill
\vfill

Norman Megill\\ 93 Bridge St., Lexington, MA 02421 \\
E-mail address: \texttt{nm{\char`\@}alum.mit.edu} \\
\vspace{7ex}
David A. Wheeler \\
E-mail address: \texttt{dwheeler{\char`\@}dwheeler.com} \\
% See notes added at end of Preface for revision history. \\
% For current information on the Metamath software see \\
\vspace{7ex}
\url{http://metamath.org}
\end{center}

\hfill
\vfill

{\parindent0pt%
\footnotesize{%
Cover: Aleph null ($\aleph_0$) is the symbol for the
first infinite cardinal number, discovered by Georg Cantor in 1873.
We use a red aleph null (with dark outline and gold glow) as the Metamath logo.
Credit: Norman Megill (1994) and Giovanni Mascellani (2019),
public domain.%
\index{aleph null}%
\index{Metamath!logo}\index{Cantor, Georg}\index{Mascellani, Giovanni}}}

% \newpage
% \thispagestyle{empty}
%
% \hfill
% \vfill
%
% \begin{center}
% {\it To my son Robin Dwight Megill}
% \end{center}
%
% \vfill
% \hfill
%
% \newpage

\tableofcontents
%\listoftables

\chapter*{Preface}
\markboth{PREFACE}{PREFACE}
\addcontentsline{toc}{section}{Preface}


% (For current information, see the notes added at the
% end of this preface on p.~\pageref{note2002}.)

\subsubsection{Overview}

Metamath\index{Metamath} is a computer language and an associated computer
program for archiving, verifying, and studying mathematical proofs at a very
detailed level.  The Metamath language incorporates no mathematics per se but
treats all mathematical statements as mere sequences of symbols.  You provide
Metamath with certain special sequences (axioms) that tell it what rules
of inference are allowed.  Metamath is not limited to any specific field of
mathematics.  The Metamath language is simple and robust, with an
almost total absence of hard-wired syntax, and
we\footnote{Unless otherwise noted, the words
``I,'' ``me,'' and ``my'' refer to Norman Megill\index{Megill, Norman}, while
``we,'' ``us,'' and ``our'' refer to Norman Megill and
David A. Wheeler\index{Wheeler, David A.}.}
believe that it
provides about the simplest possible framework that allows essentially all of
mathematics to be expressed with absolute rigor.

% index test
%\newcommand{\nn}[1]{#1n}
%\index{aaa@bbb}
%\index{abc!def}
%\index{abd|see{qqq}}
%\index{abe|nn}
%\index{abf|emph}
%\index{abg|(}
%\index{abg|)}

Using the Metamath language, you can build formal or mathematical
systems\index{formal system}\footnote{A formal or mathematical system consists
of a collection of symbols (such as $2$, $4$, $+$ and $=$), syntax rules that
describe how symbols may be combined to form a legal expression (called a
well-formed formula or {\em wff}, pronounced ``whiff''), some starting wffs
called axioms, and inference rules that describe how theorems may be derived
(proved) from the axioms.  A theorem is a mathematical fact such as $2+2=4$.
Strictly speaking, even an obvious fact such as this must be proved from
axioms to be formally acceptable to a mathematician.}\index{theorem}
\index{axiom}\index{rule}\index{well-formed formula (wff)} that involve
inferences from axioms.  Although a database is provided
that includes a recommended set of axioms for standard mathematics, if you
wish you can supply your own symbols, syntax, axioms, rules, and definitions.

The name ``Metamath'' was chosen to suggest that the language provides a
means for {\em describing} mathematics rather than {\em being} the
mathematics itself.  Actually in some sense any mathematical language is
metamathematical.  Symbols written on paper, or stored in a computer,
are not mathematics itself but rather a way of expressing mathematics.
For example ``7'' and ``VII'' are symbols for denoting the number seven
in Arabic and Roman numerals; neither {\em is} the number seven.

If you are able to understand and write computer programs, you should be able
to follow abstract mathematics with the aid of Metamath.  Used in conjunction
with standard textbooks, Metamath can guide you step-by-step towards an
understanding of abstract mathematics from a very rigorous viewpoint, even if
you have no formal abstract mathematics background.  By using a single,
consistent notation to express proofs, once you grasp its basic concepts
Metamath provides you with the ability to immediately follow and dissect
proofs even in totally unfamiliar areas.

Of course, just being able follow a proof will not necessarily give you an
intuitive familiarity with mathematics.  Memorizing the rules of chess does not
give you the ability to appreciate the game of a master, and knowing how the
notes on a musical score map to piano keys does not give you the ability to
hear in your head how it would sound.  But each of these can be a first step.

Metamath allows you to explore proofs in the sense that you can see the
theorem referenced at any step expanded in as much detail as you want, right
down to the underlying axioms of logic and set theory (in the case of the set
theory database provided).  While Metamath will not replace the higher-level
understanding that can only be acquired through exercises and hard work, being
able to see how gaps in a proof are filled in can give you increased
confidence that can speed up the learning process and save you time when you
get stuck.

The Metamath language breaks down a mathematical proof into its tiniest
possible parts.  These can be pieced together, like interlocking
pieces in a puzzle, only in a way that produces correct and absolutely rigorous
mathematics.

The nature of Metamath\index{Metamath} enforces very precise mathematical
thinking, similar to that involved in writing a computer program.  A crucial
difference, though, is that once a proof is verified (by the Metamath program)
to be correct, it is definitely correct; it can never have a hidden
``bug.''\index{computer program bugs}  After getting used to the kind of rigor
and accuracy provided by Metamath, you might even be tempted to
adopt the attitude that a proof should never be considered correct until it
has been verified by a computer, just as you would not completely trust a
manual calculation until you have verified it on a
calculator.

My goal
for Metamath was a system for describing and verifying
mathematics that is completely universal yet conceptually as simple as
possible.  In approaching mathematics from an axiomatic, formal viewpoint, I
wanted Metamath to be able to handle almost any mathematical system, not
necessarily with ease, but at least in principle and hopefully in practice. I
wanted it to verify proofs with absolute rigor, and for this reason Metamath
is what might be thought of as a ``compile-only'' language rather than an
algorithmic or Turing-machine language (Pascal, C, Prolog, Mathematica,
etc.).  In other words, a database written in the Metamath
language doesn't ``do'' anything; it merely exhibits mathematical knowledge
and permits this knowledge to be verified as being correct.  A program in an
algorithmic language can potentially have hidden bugs\index{computer program
bugs} as well as possibly being hard to understand.  But each token in a
Metamath database must be consistent with the database's earlier
contents according to simple, fixed rules.
If a database is verified
to be correct,\footnote{This includes
verification that a sequential list of proof steps results in the specified
theorem.} then the mathematical content is correct if the
verifier is correct and the axioms are correct.
The verification program could be incorrect, but the verification algorithm
is relatively simple (making it unlikely to be implemented incorrectly
by the Metamath program),
and there are over a dozen Metamath database verifiers
written by different people in different programming languages
(so these different verifiers can act as multiple reviewers of a database).
The most-used Metamath database, the Metamath Proof Explorer
(aka \texttt{set.mm}\index{set theory database (\texttt{set.mm})}%
\index{Metamath Proof Explorer}),
is currently verified by four different Metamath verifiers written by
four different people in four different languages, including the
original Metamath program described in this book.
The only ``bugs'' that can exist are in the statement of the axioms,
for example if the axioms are inconsistent (a famous problem shown to be
unsolvable by G\"{o}del's incompleteness theorem\index{G\"{o}del's
incompleteness theorem}).
However, real mathematical systems have very few axioms, and these can
be carefully studied.
All of this provides extraordinarily high confidence that the verified database
is in fact correct.

The Metamath program
doesn't prove theorems automatically but is designed to verify proofs
that you supply to it.
The underlying Metamath language is completely general and has no built-in,
preconceived notions about your formal system\index{formal system}, its logic
or its syntax.
For constructing proofs, the Metamath program has a Proof Assistant\index{Proof
Assistant} which helps you fill in some of a proof step's details, shows you
what choices you have at any step, and verifies the proof as you build it; but
you are still expected to provide the proof.

There are many other programs that can process or generate information
in the Metamath language, and more continue to be written.
This is in part because the Metamath language itself is very simple
and intentionally easy to automatically process.
Some programs, such as \texttt{mmj2}\index{mmj2}, include a proof assistant
that can automate some steps beyond what the Metamath program can do.
Mario Carneiro has developed an algorithm for converting proofs from
the OpenTheory interchange format, which can be translated to and from
any of the HOL family of proof languages (HOL4, HOL Light, ProofPower,
and Isabelle), into the
Metamath language \cite{DBLP:journals/corr/Carneiro14}\index{Carneiro, Mario}.
Daniel Whalen has developed Holophrasm, which can automatically
prove many Metamath proofs using
machine learning\index{machine learning}\index{artificial intelligence}
approaches
(including multiple neural networks\index{neural networks})\cite{DBLP:journals/corr/Whalen16}\index{Whalen, Daniel}.
However,
a discussion of these other programs is beyond the scope of this book.

Like most computer languages, the Metamath\index{Metamath} language uses the
standard ({\sc ascii}) characters on a computer keyboard, so it cannot
directly represent many of the special symbols that mathematicians use.  A
useful feature of the Metamath program is its ability to convert its notation
into the \LaTeX\ typesetting language.\index{latex@{\LaTeX}}  This feature
lets you convert the {\sc ascii} tokens you've defined into standard
mathematical symbols, so you end up with symbols and formulas you are familiar
with instead of somewhat cryptic {\sc ascii} representations of them.
The Metamath program can also generate HTML\index{HTML}, making it easy
to view results on the web and to see related information by using
hypertext links.

Metamath is probably conceptually different from anything you've seen
before and some aspects may take some getting used to.  This book will
help you decide whether Metamath suits your specific needs.

\subsubsection{Setting Your Expectations}
It is important for you to understand what Metamath\index{Metamath} is and is
not.  As mentioned, the Metamath program
is {\em not} an automated theorem prover but
rather a proof verifier.  Developing a database can be tedious, hard work,
especially if you want to make the proofs as short as possible, but it becomes
easier as you build up a collection of useful theorems.  The purpose of
Metamath is simply to document existing mathematics in an absolutely rigorous,
computer-verifiable way, not to aid directly in the creation of new
mathematics.  It also is not a magic solution for learning abstract
mathematics, although it may be helpful to be able to actually see the implied
rigor behind what you are learning from textbooks, as well as providing hints
to work out proofs that you are stumped on.

As of this writing, a sizable set theory database has been developed to
provide a foundation for many fields of mathematics, but much more work would
be required to develop useful databases for specific fields.

Metamath\index{Metamath} ``knows no math;'' it just provides a framework in
which to express mathematics.  Its language is very small.  You can define two
kinds of symbols, constants\index{constant} and variables\index{variable}.
The only thing Metamath knows how to do is to substitute strings of symbols
for the variables\index{substitution!variable}\index{variable substitution} in
an expression based on instructions you provide it in a proof, subject to
certain constraints you specify for the variables.  Even the decimal
representation of a number is merely a string of certain constants (digits)
which together, in a specific context, correspond to whatever mathematical
object you choose to define for it; unlike other computer languages, there is
no actual number stored inside the computer.  In a proof, you in effect
instruct Metamath what symbol substitutions to make in previous axioms or
theorems and join a sequence of them together to result in the desired
theorem.  This kind of symbol manipulation captures the essence of mathematics
at a preaxiomatic level.

\subsubsection{Metamath and Mathematical Literature}

In advanced mathematical literature, proofs are usually presented in the form
of short outlines that often only an expert can follow.  This is partly out of
a desire for brevity, but it would also be unwise (even if it were practical)
to present proofs in complete formal detail, since the overall picture would
be lost.\index{formal proof}

A solution I envision\label{envision} that would allow mathematics to remain
acceptable to the expert, yet increase its accessibility to non-specialists,
consists of a combination of the traditional short, informal proof in print
accompanied by a complete formal proof stored in a computer database.  In an
analogy with a computer program, the informal proof is like a set of comments
that describe the overall reasoning and content of the proof, whereas the
computer database is like the actual program and provides a means for anyone,
even a non-expert, to follow the proof in as much detail as desired, exploring
it back through layers of theorems (like subroutines that call other
subroutines) all the way back to the axioms of the theory.  In addition, the
computer database would have the advantage of providing absolute assurance
that the proof is correct, since each step can be verified automatically.

There are several other approaches besides Metamath to a project such
as this.  Section~\ref{proofverifiers} discusses some of these.

To us, a noble goal would be a database with hundreds of thousands of
theorems and their computer-verifiable proofs, encompassing a significant
fraction of known mathematics and available for instant access.
These would be fully verified by multiple independently-implemented verifiers,
to provide extremely high confidence that the proofs are completely correct.
The database would allow people to investigate whatever details they were
interested in, so that they could confirm whatever portions they wished.
Whether or not Metamath is an appropriate choice remains to be seen, but in
principle we believe it is sufficient.

\subsubsection{Formalism}

Over the past fifty years, a group of French mathematicians working
collectively under the pseudonym of Bourbaki\index{Bourbaki, Nicolas} have
co-authored a series of monographs that attempt to rigorously and
consistently formalize large bodies of mathematics from foundations.  On the
one hand, certainly such an effort has its merits; on the other hand, the
Bourbaki project has been criticized for its ``scholasticism'' and
``hyperaxiomatics'' that hide the intuitive steps that lead to the results
\cite[p.~191]{Barrow}\index{Barrow, John D.}.

Metamath unabashedly carries this philosophy to its extreme and no doubt is
subject to the same kind of criticism.  Nonetheless I think that in
conjunction with conventional approaches to mathematics Metamath can serve a
useful purpose.  The Bourbaki approach is essentially pedagogic, requiring the
reader to become intimately familiar with each detail in a very large
hierarchy before he or she can proceed to the next step.  The difference with
Metamath is that the ``reader'' (user) knows that all details are contained in
its computer database, available as needed; it does not demand that the user
know everything but conveniently makes available those portions that are of
interest.  As the body of all mathematical knowledge grows larger and larger,
no one individual can have a thorough grasp of its entirety.  Metamath
can finalize and put to rest any questions about the validity of any part of it
and can make any part of it accessible, in principle, to a non-specialist.

\subsubsection{A Personal Note}
Why did I develop Metamath\index{Metamath}?  I enjoy abstract mathematics, but
I sometimes get lost in a barrage of definitions and start to lose confidence
that my proofs are correct.  Or I reach a point where I lose sight of how
anything I'm doing relates to the axioms that a theory is based on and am
sometimes suspicious that there may be some overlooked implicit axiom
accidentally introduced along the way (as happened historically with Euclidean
geometry\index{Euclidean geometry}, whose omission of Pasch's
axiom\index{Pasch's axiom} went unnoticed for 2000 years
\cite[p.~160]{Davis}!). I'm also somewhat lazy and wish to avoid the effort
involved in re-verifying the gaps in informal proofs ``left to the reader;'' I
prefer to figure them out just once and not have to go through the same
frustration a year from now when I've forgotten what I did.  Metamath provides
better recovery of my efforts than scraps of paper that I can't
decipher anymore.  But mostly I find very appealing the idea of rigorously
archiving mathematical knowledge in a computer database, providing precision,
certainty, and elimination of human error.

\subsubsection{Note on Bibliography and Index}

The Bibliography usually includes the Library of Congress classification
for a work to make it easier for you to find it in on a university
library shelf.  The Index has author references to pages where their works
are cited, even though the authors' names may not appear on those pages.

\subsubsection{Acknowledgments}

Acknowledgments are first due to my wife, Deborah (who passed away on
September 4, 1998), for critiquing the manu\-script but most of all for
her patience and support.  I also wish to thank Joe Wright, Richard
Becker, Clarke Evans, Buddha Buck, and Jeremy Henty for helpful
comments.  Any errors, omissions, and other shortcomings are of course
my responsibility.

\subsubsection{Note Added June 22, 2005}\label{note2002}

The original, unpublished version of this book was written in 1997 and
distributed via the web.  The present edition has been updated to
reflect the current Metamath program and databases, as well as more
current {\sc url}s for Internet sites.  Thanks to Josh
Purinton\index{Purinton, Josh}, One Hand
Clapping, Mel L.\ O'Cat, and Roy F. Longton for pointing out
typographical and other errors.  I have also benefitted from numerous
discussions with Raph Levien\index{Levien, Raph}, who has extended
Metamath's philosophy of rigor to result in his {\em
Ghilbert}\index{Ghilbert} proof language (\url{http://ghilbert.org}).

Robert (Bob) Solovay\index{Solovay, Robert} communicated a new result of
A.~R.~D.~Mathias on the system of Bourbaki, and the text has been
updated accordingly (p.~\pageref{bourbaki}).

Bob also pointed out a clarification of the literature regarding
category theory and inaccessible cardinals\index{category
theory}\index{cardinal, inaccessible} (p.~\pageref{categoryth}),
and a misleading statement was removed from the text.  Specifically,
contrary to a statement in previous editions, it is possible to express
``There is a proper class of inaccessible cardinals'' in the language of
ZFC.  This can be done as follows:  ``For every set $x$ there is an
inaccessible cardinal $\kappa$ such that $\kappa$ is not in $x$.''
Bob writes:\footnote{Private communication, Nov.~30, 2002.}
\begin{quotation}
     This axiom is how Grothendieck presents category theory.  To each
inaccessible cardinal $\kappa$ one associates a Grothendieck universe
\index{Grothendieck, Alexander} $U(\kappa)$.  $U(\kappa)$ consists of
those sets which lie in a transitive set of cardinality less than
$\kappa$.  Instead of the ``category of all groups,'' one works relative
to a universe [considering the category of groups of cardinality less
than $\kappa$].  Now the category whose objects are all categories
``relative to the universe $U(\kappa)$'' will be a category not
relative to this universe but to the next universe.

     All of the things category theorists like to do can be done in this
framework.  The only controversial point is whether the Grothen\-dieck
axiom is too strong for the needs of category theorists.  Mac Lane
\index{Mac Lane, Saunders} argues that ``one universe is enough'' and
Feferman\index{Feferman, Solomon} has argued that one can get by with
ordinary ZFC.  I don't find Feferman's arguments persuasive.  Mac Lane
may be right, but when I think about category theory I do it \`{a} la
Grothendieck.

        By the way Mizar\index{Mizar} adds the axiom ``there is a proper
class of inaccessibles'' precisely so as to do category theory.
\end{quotation}

The most current information on the Metamath program and databases can
always be found at \url{http://metamath.org}.


\subsubsection{Note Added June 24, 2006}\label{note2006}

The Metamath spec was restricted slightly to make parsers easier to
write.  See the footnote on p.~\pageref{namespace}.

%\subsubsection{Note Added July 24, 2006}\label{note2006b}
\subsubsection{Note Added March 10, 2007}\label{note2006b}

I am grateful to Anthony Williams\index{Williams, Anthony} for writing
the \LaTeX\ package called {\tt realref.sty} and contributing it to the
public domain.  This package allows the internal hyperlinks in a {\sc
pdf} file to anchor to specific page numbers instead of just section
titles, making the navigation of the {\sc pdf} file for this book much
more pleasant and ``logical.''

A typographical error found by Martin Kiselkov was corrected.
A confusing remark about unification was deleted per suggestion of
Mel O'Cat.

\subsubsection{Note Added May 27, 2009}\label{note2009}

Several typos found by Kim Sparre were corrected.  A note was added that
the Poincar\'{e} conjecture has been proved (p.~\pageref{poincare}).

\subsubsection{Note Added Nov. 17, 2014}\label{note2014}

The statement of the Schr\"{o}der--Bernstein theorem was corrected in
Section~\ref{trust}.  Thanks to Bob Solovay for pointing out the error.

\subsubsection{Note Added May 25, 2016}\label{note2016}

Thanks to Jerry James for correcting 16 typos.

\subsubsection{Note Added February 25, 2019}\label{note201902}

David A. Wheeler\index{Wheeler, David A.}
made a large number of improvements and updates,
in coordination with Norman Megill.
The predicate calculus axioms were renumbered, and the text makes
it clear that they are based on Tarski's system S2;
the one slight deviation in axiom ax-6 is explained and justified.
The real and complex number axioms were modified to be consistent with
\texttt{set.mm}\index{set theory database (\texttt{set.mm})}%
\index{Metamath Proof Explorer}.
Long-awaited specification changes ``1--8'' were made,
which clarified previously ambiguous points.
Some errors in the text involving \texttt{\$f} and
\texttt{\$d} statements were corrected (the spec was correct, but
the in-book explanations unintentionally contradicted the spec).
We now have a system for automatically generating narrow PDFs,
so that those with smartphones can have easy access to the current
version of this document.
A new section on deduction was added;
it discusses the standard deduction theorem,
the weak deduction theorem,
deduction style, and natural deduction.
Many minor corrections (too numerous to list here) were also made.

\subsubsection{Note Added March 7, 2019}\label{note201903}

This added a description of the Matamath language syntax in
Extended Backus--Naur Form (EBNF)\index{Extended Backus--Naur Form}\index{EBNF}
in Appendix \ref{BNF}, added a brief explanation about typecodes,
inserted more examples in the deduction section,
and added a variety of smaller improvements.

\subsubsection{Note Added April 7, 2019}\label{note201904}

This version clarified the proper substitution notation, improved the
discussion on the weak deduction theorem and natural deduction,
documented the \texttt{undo} command, updated the information on
\texttt{write source}, changed the typecode
from \texttt{set} to \texttt{setvar} to be consistent with the current
version of \texttt{set.mm}, added more documentation about comment markup
(e.g., documented how to create headings), and clarified the
differences between various assertion forms (in particular deduction form).

\subsubsection{Note Added June 2, 2019}\label{note201906}

This version fixes a large number of small issues reported by
Beno\^{i}t Jubin\index{Jubin, Beno\^{i}t}, such as editorial issues
and the need to document \texttt{verify markup} (thank you!).
This version also includes specific examples
of forms (deduction form, inference form, and closed form).
We call this version the ``second edition'';
the previous edition formally published in 2007 had a slightly different title
(\textit{Metamath: A Computer Language for Pure Mathematics}).

\chapter{Introduction}
\pagenumbering{arabic}

\begin{quotation}
  {\em {\em I.M.:}  No, no.  There's nothing subjective about it!  Everybody
knows what a proof is.  Just read some books, take courses from a competent
mathematician, and you'll catch on.

{\em Student:}  Are you sure?

{\em I.M.:}  Well---it is possible that you won't, if you don't have any
aptitude for it.  That can happen, too.

{\em Student:}  Then {\em you} decide what a proof is, and if I don't learn
to decide in the same way, you decide I don't have any aptitude.

{\em I.M.:}  If not me, then who?}
    \flushright\sc  ``The Ideal Mathematician''
    \index{Davis, Phillip J.}
    \footnote{\cite{Davis}, p.~40.}\\
\end{quotation}

Brilliant mathematicians have discovered almost
unimaginably profound results that rank among the crowning intellectual
achievements of mankind.  However, there is a sense in which modern abstract
mathematics is behind the times, stuck in an era before computers existed.
While no one disputes the remarkable results that have been achieved,
communicating these results in a precise way to the uninitiated is virtually
impossible.  To describe these results, a terse informal language is used which
despite its elegance is very difficult to learn.  This informal language is not
imprecise, far from it, but rather it often has omitted detail
and symbols with hidden context that are
implicitly understood by an expert but few others.  Extremely complex technical
meanings are associated with innocent-sounding English words such as
``compact'' and ``measurable'' that barely hint at what is actually being
said.  Anyone who does not keep the precise technical meaning constantly in
mind is bound to fail, and acquiring the ability to do this can be achieved
only through much practice and hard work.  Only the few who complete the
painful learning experience can join the small in-group of pure
mathematicians.  The informal language effectively cuts off the true nature of
their knowledge from most everyone else.

Metamath\index{Metamath} makes abstract mathematics more concrete.  It allows
a computer to keep track of the complexity associated with each word or symbol
with absolute rigor.  You can explore this complexity at your leisure, to
whatever degree you desire.  Whether or not you believe that concepts such as
infinity actually ``exist'' outside of the mind, Metamath lets you get to the
foundation for what's really being said.

Metamath also enables completely rigorous and thorough proof verification.
Its language is simple enough so that you
don't have to rely on the authority of experts but can verify the results
yourself, step by step.  If you want to attempt to derive your own results,
Metamath will not let you make a mistake in reasoning.
Even professional mathematicians make mistakes; Metamath makes it possible
to thoroughly verify that proofs are correct.

Metamath\index{Metamath} is a computer language and an associated computer
program for archiving, verifying, and studying mathematical proofs at a very
detailed level.
The Metamath language
describes formal\index{formal system} mathematical
systems and expresses proofs of theorems in those systems.  Such a language
is called a metalanguage\index{metalanguage} by mathematicians.
The Metamath program is a computer program that verifies
proofs expressed in the Metamath language.
The Metamath program does not have the built-in
ability to make logical inferences; it just makes a series of symbol
substitutions according to instructions given to it in a proof
and verifies that the result matches the expected theorem.  It makes logical
inferences based only on rules of logic that are contained in a set of
axioms\index{axiom}, or first principles, that you provide to it as the
starting point for proofs.

The complete specification of the Metamath language is only four pages long
(Section~\ref{spec}, p.~\pageref{spec}).  Its simplicity may at first make you
wonder how it can do much of anything at all.  But in fact the kinds of
symbol manipulations it performs are the ones that are implicitly done in all
mathematical systems at the lowest level.  You can learn it relatively quickly
and have complete confidence in any mathematical proof that it verifies.  On
the other hand, it is powerful and general enough so that virtually any
mathematical theory, from the most basic to the deeply abstract, can be
described with it.

Although in principle Metamath can be used with any
kind of mathematics, it is best suited for abstract or ``pure'' mathematics
that is mostly concerned with theorems and their proofs, as opposed to the
kind of mathematics that deals with the practical manipulation of numbers.
Examples of branches of pure mathematics are logic\index{logic},\footnote{Logic
is the study of statements that are universally true regardless of the objects
being described by the statements.  An example is the statement, ``if $P$
implies $Q$, then either $P$ is false or $Q$ is true.''} set theory\index{set
theory},\footnote{Set theory is the study of general-purpose mathematical objects called
``sets,'' and from it essentially all of mathematics can be derived.  For
example, numbers can be defined as specific sets, and their properties
can be explored using the tools of set theory.} number theory\index{number
theory},\footnote{Number theory deals with the properties of positive and
negative integers (whole numbers).} group theory\index{group
theory},\footnote{Group theory studies the properties of mathematical objects
called groups that obey a simple set of axioms and have properties of symmetry
that make them useful in many other fields.} abstract algebra\index{abstract
algebra},\footnote{Abstract algebra includes group theory and also studies
groups with additional properties that qualify them as ``rings'' and
``fields.''  The set of real numbers is a familiar example of a field.},
analysis\index{analysis} \index{real and complex numbers}\footnote{Analysis is
the study of real and complex numbers.} and
topology\index{topology}.\footnote{One area studied by topology are properties
that remain unchanged when geometrical objects undergo stretching
deformations; for example a doughnut and a coffee cup each have one hole (the
cup's hole is in its handle) and are thus considered topologically
equivalent.  In general, though, topology is the study of abstract
mathematical objects that obey a certain (surprisingly simple) set of axioms.
See, for example, Munkres \cite{Munkres}\index{Munkres, James R.}.} Even in
physics, Metamath could be applied to certain branches that make use of
abstract mathematics, such as quantum logic\index{quantum logic} (used to study
aspects of quantum mechanics\index{quantum mechanics}).

On the other hand, Metamath\index{Metamath} is less suited to applications
that deal primarily with intensive numeric computations.  Metamath does not
have any built-in representation of numbers\index{Metamath!representation of
numbers}; instead, a specific string of symbols (digits) must be syntactically
constructed as part of any proof in which an ordinary number is used.  For
this reason, numbers in Metamath are best limited to specific constants that
arise during the course of a theorem or its proof.  Numbers are only a tiny
part of the world of abstract mathematics.  The exclusion of built-in numbers
was a conscious decision to help achieve Metamath's simplicity, and there are
other software tools if you have different mathematical needs.
If you wish to quickly solve algebraic problems, the computer algebra
programs\index{computer algebra system} {\sc
macsyma}\index{macsyma@{\sc macsyma}}, Mathematica\index{Mathematica}, and
Maple\index{Maple} are specifically suited to handling numbers and
algebra efficiently.
If you wish to simply calculate numeric or matrix expressions easily,
tools such as Octave\index{Octave} may be a better choice.

After learning Metamath's basic statement types, any
tech\-ni\-cal\-ly ori\-ent\-ed person, mathematician or not, can
immediately trace
any theorem proved in the language as far back as he or she wants, all the way
to the axioms on which the theorem is based.  This ability suggests a
non-traditional way of learning about pure mathematics.  Used in conjunction
with traditional methods, Metamath could make pure mathematics accessible to
people who are not sufficiently skilled to figure out the implicit detail in
ordinary textbook proofs.  Once you learn the axioms of a theory, you can have
complete confidence that everything you need to understand a proof you are
studying is all there, at your beck and call, allowing you to focus in on any
proof step you don't understand in as much depth as you need, without worrying
about getting stuck on a step you can't figure out.\footnote{On the other
hand, writing proofs in the Metamath language is challenging, requiring
a degree of rigor far in excess of that normally taught to students.  In a
classroom setting, I doubt that writing Metamath proofs would ever replace
traditional homework exercises involving informal proofs, because the time
needed to work out the details would not allow a course to
cover much material.  For students who have trouble grasping the implied rigor
in traditional material, writing a few simple proofs in the Metamath language
might help clarify fuzzy thought processes.  Although somewhat difficult at
first, it eventually becomes fun to do, like solving a puzzle, because of the
instant feedback provided by the computer.}

Metamath\index{Metamath} is probably unlike anything you have
encountered before.  In this first chapter we will look at the philosophy and
use of computers in mathematics in order to better understand the motivation
behind Metamath.  The material in this chapter is not required in order to use
Metamath.  You may skip it if you are impatient, but I hope you will find it
educational and enjoyable.  If you want to start experimenting with the
Metamath program right away, proceed directly to Chapter~\ref{using}
(p.~\pageref{using}).  To
learn the Metamath language, skim Chapter~\ref{using} then proceed to
Chapter~\ref{languagespec} (p.~\pageref{languagespec}).

\section{Mathematics as a Computer Language}

\begin{quote}
  {\em The study of mathematics is apt to commence in
dis\-ap\-point\-ment.\ldots \\
We are told that by its aid the stars are weighted
and the billions of molecules in a drop of water are counted.  Yet, like the
ghost of Hamlet's father, this great science eludes the efforts of our mental
weapons to grasp it.}
  \flushright\sc  Alfred North Whitehead\footnote{\cite{Whitehead}, ch.\ 1.}\\
\end{quote}\index{Whitehead, Alfred North}

\subsection{Is Mathematics ``User-Friendly''?}

Suppose you have no formal training in abstract mathematics.  But popular
books you've read offer tempting glimpses of this world filled with profound
ideas that have stirred the human spirit.  You are not satisfied with the
informal, watered-down descriptions you've read but feel it is important to
grasp the underlying mathematics itself to understand its true meaning. It's
not practical to go back to school to learn it, though; you don't want to
dedicate years of your life to it.  There are many important things in life,
and you have to set priorities for what's important to you.  What would happen
if you tried to pursue it on your own, in your spare time?

After all, you were able to learn a computer programming language such as
Pascal on your own without too much difficulty, even though you had no formal
training in computers.  You don't claim to be an expert in software design,
but you can write a passable program when necessary to suit your needs.  Even
more important, you know that you can look at anyone else's Pascal program, no
matter how complex, and with enough patience figure out exactly how it works,
even though you are not a specialist.  Pascal allows you do anything that a
computer can do, at least in principle.  Thus you know you have the ability,
in principle, to follow anything that a computer program can do:  you just
have to break it down into small enough pieces.

Here's an imaginary scenario of what might happen if you na\-ive\-ly a\-dopted
this same view of abstract mathematics and tried to pick it up on your own, in
a period of time comparable to, say, learning a computer programming
language.

\subsubsection{A Non-Mathematician's Quest for Truth}

\begin{quote}
  {\em \ldots my daughters have been studying (chemistry) for several
se\-mes\-ters, think they have learned differential and integral calculus in
school, and yet even today don't know why $x\cdot y=y\cdot x$ is true.}
  \flushright\sc  Edmund Landau\footnote{\cite{Landau}, p.~vi.}\\
\end{quote}\index{Landau, Edmund}

\begin{quote}
  {\em Minus times minus is plus,\\
The reason for this we need not discuss.}
  \flushright\sc W.\ H.\ Auden\footnote{As quoted in \cite{Guillen}, p.~64.}\\
\end{quote}\index{Auden, W.\ H.}\index{Guillen, Michael}

We'll suppose you are a technically oriented professional, perhaps an engineer, a
computer programmer, or a physicist, but probably not a mathematician.  You
consider yourself reasonably intelligent.  You did well in school, learning a
variety of methods and techniques in practical mathematics such as calculus and
differential equations.  But rarely did your courses get into anything
resembling modern abstract mathematics, and proofs were something that appeared
only occasionally in your textbooks, a kind of necessary evil that was
supposed to convince you of a certain key result.  Most of your
homework consisted of exercises that gave you practice in the techniques, and
you were hardly ever asked to come up with a proof of your own.

You find yourself curious about advanced, abstract mathematics.  You are
driven by an inner conviction that it is important to understand and
appreciate some of the most profound knowledge discovered by mankind.  But it
seems very hard to learn, something that only certain gifted longhairs can
access and understand.  You are frustrated that it seems forever cut off from
you.

Eventually your curiosity drives you to do something about it.
You set for yourself a goal of ``really'' understanding mathematics:  not just
how to manipulate equations in algebra or calculus according to cookbook
rules, but rather to gain a deep understanding of where those rules come from.
In fact, you're not thinking about this kind of ordinary mathematics at all,
but about a much more abstract, ethereal realm of pure mathematics, where
famous results such as G\"{o}del's incompleteness theorem\index{G\"{o}del's
incompleteness theorem} and Cantor's different kinds of infinities
reside.

You have probably read a number of popular books, with titles like {\em
Infinity and the Mind} \cite{Rucker}\index{Rucker, Rudy}, on topics such as
these.  You found them inspiring but at the same time somewhat
unsatisfactory.  They gave you a general idea of what these results are about,
but if someone asked you to prove them, you wouldn't have the faintest idea of
where to begin.   Sure, you could give the same overall outline that you
learned from the popular books; and in a general sort of way, you do have an
understanding.  But deep down inside, you know that there is a rigor that is
missing, that probably there are many subtle steps and pitfalls along the way,
and ultimately it seems you have to place your trust in the experts in the
field.  You don't like this; you want to be able to verify these results for
yourself.

So where do you go next?  As a first step, you decide to look up some of the
original papers on the theorems you are curious about, or better, obtain some
standard textbooks in the field.  You look up a theorem you want to
understand.  Sure enough, it's there, but it's expressed with strange
terms and odd symbols that mean absolutely nothing to you.  It might as well be written in
a foreign language you've never seen before, whose symbols are totally alien.
You look at the proof, and you haven't the foggiest notion what each step
means, much less how one step follows from another.  Well, obviously you have
a lot to learn if you want to understand this stuff.

You feel that you could probably understand it by
going back to college for another three to six years and getting a math
degree.  But that does not fit in with your career and the other things in
your life and would serve no practical purpose.  You decide to seek a quicker
path.  You figure you'll just trace your way back to the beginning, step by
step, as you would do with a computer program, until you understand it.  But
you quickly find that this is not possible, since you can't even understand
enough to know what you have to trace back to.

Maybe a different approach is in order---maybe you should start at the
beginning and work your way up.  First, you read the introduction to the book
to find out what the prerequisites are.  In a similar fashion, you trace your
way back through two or three more books, finally arriving at one that seems
to start at a beginning:  it lists the axioms of arithmetic.  ``Aha!'' you
naively think, ``This must be the starting point, the source of all mathematical
knowledge.'' Or at least the starting point for mathematics dealing with
numbers; you have to start somewhere and have no idea what the starting point
for other mathematics would be.  But the word ``axioms'' looks promising.  So
you eagerly read along and work through some elementary exercises at the
beginning of the book.  You feel vaguely bothered:  these
don't seem like axioms at all, at least not in the sense that you want to
think of axioms.  Axioms imply a starting point from which everything else can
be built up, according to precise rules specified in the axiom system.  Even
though you can understand the first few proofs in an informal way,
and are able to do some of the
exercises, it's hard to pin down precisely what the
rules are.   Sure, each step seems to follow logically from the others, but
exactly what does that mean?  Is the ``logic'' just a matter of common sense,
something vague that we all understand but can never quite state precisely?

You've spent a number of years, off and on, programming computers, and you
know that in the case of computer languages there is no question of what the
rules are---they are precise and crystal clear.  If you follow them, your
program will work, and if you don't, it won't.  No matter how complex a
program, it can always be broken down into simpler and simpler pieces, until
you can ultimately identify the bits that are moved around to perform a
specific function.  Some programs might require a lot of perseverance to
accomplish this, but if you focus on a specific portion of it, you don't even
necessarily have to know how the rest of it works. Shouldn't there be an
analogy in mathematics?

You decide to apply the ultimate test:  you ask yourself how a computer could
verify or ensure that the steps in these proofs follow from one another.
Certainly mathematics must be at least as precisely defined as a computer
language, if not more so; after all, computer science itself is based on it.
If you can get a computer to verify these proofs, then you should also be
able, in principle, to understand them yourself in a very crystal clear,
precise way.

You're in for a surprise:  you can conceive of no way to convert the
proofs, which are in English, to a form that the computer can understand.
The proofs are filled with phrases such as ``assume there exists a unique
$x$\ldots'' and ``given any $y$, let $z$ be the number such that\ldots''  This
isn't the kind of logic you are used to in computer programming, where
everything, even arithmetic, reduces to Boolean ones and zeroes if you care to
break it down sufficiently.  Even though you think you understand the proofs,
there seems to be some kind of higher reasoning involved rather than precise
rules that define how you manipulate the symbols in the axioms.  Whatever it
is, it just isn't obvious how you would express it to a computer, and the more
you think about it, the more puzzled and confused you get, to the point where
you even wonder whether {\em you} really understand it.  There's a lot more to
these axioms of arithmetic than meets the eye.

Nobody ever talked about this in school in your applied math and engineering
courses.  You just learned the rules they gave you, not quite understanding
how or why they worked, sometimes vaguely suspicious or uncertain of them, and
through homework problems and osmosis learned how to present solutions that
satisfied the instructor and earned you an ``A.''  Rarely did you actually
``prove'' anything in a rigorous way, and the math majors who did do stuff
like that seemed to be in a different world.

Of course, there are computer algebra programs that can do mathematics, and
rather impressively.  They can instantly solve the integrals that you
struggled with in freshman calculus, and do much, much more.  But when you
look at these programs, what you see is a big collection of algorithms and
techniques that evolved and were added to over time, along with some basic
software that manipulates symbols.  Each algorithm that is built in is the
result of someone's theorem whose proof is omitted; you just have to trust the
person who proved it and the person who programmed it in and hope there are no
bugs.\index{computer program bugs}  Somehow this doesn't seem to be the
essence of mathematics.  Although computer algebra systems can generate
theorems with amazing speed, they can't actually prove a single one of them.

After some puzzlement, you revisit some popular books on what mathematics is
all about.  Somewhere you read that all of mathematics is actually derived
from something called ``set theory.''  This is a little confusing, because
nowhere in the book that presented the axioms of arithmetic was there any
mention of set theory, or if there was, it seemed to be just a tool that helps
you describe things better---the set of even numbers, that sort of thing.  If
set theory is the basis for all mathematics, then why are additional axioms
needed for arithmetic?

Something is wrong but you're not sure what.  One of your friends is a pure
mathematician.  He knows he is unable to communicate to you what he does for a
living and seems to have little interest in trying.  You do know that for him,
proofs are what mathematics is all about. You ask him what a proof is, and he
essentially tells you that, while of course it's based on logic, really it's
something you learn by doing it over and over until you pick it up.  He refers
you to a book, {\em How to Read and Do Proofs} \cite{Solow}.\index{Solow,
Daniel}  Although this book helps you understand traditional informal proofs,
there is still something missing you can't seem to pin down yet.

You ask your friend how you would go about having a computer verify a proof.
At first he seems puzzled by the question; why would you want to do that?
Then he says it's not something that would make any sense to do, but he's
heard that you'd have to break the proof down into thousands or even millions
of individual steps to do such a thing, because the reasoning involved is at
such a high level of abstraction.  He says that maybe it's something you could
do up to a point, but the computer would be completely impractical once you
get into any meaningful mathematics.  There, the only way you can verify a
proof is by hand, and you can only acquire the ability to do this by
specializing in the field for a couple of years in grad school.  Anyway, he
thinks it all has to do with set theory, although he has never taken a formal
course in set theory but just learned what he needed as he went along.

You are intrigued and amazed.  Apparently a mathematician can grasp as a
single concept something that would take a computer a thousand or a million
steps to verify, and have complete confidence in it.  Each one of these
thousand or million steps must be absolutely correct, or else the whole proof
is meaningless.  If you added a million numbers by hand, would you trust the
result?  How do you really know that all these steps are correct, that there
isn't some subtle pitfall in one of these million steps, like a bug in a
computer program?\index{computer program bugs}  After all, you've read that
famous mathematicians have occasionally made mistakes, and you certainly know
you've made your share on your math homework problems in school.

You recall the analogy with a computer program.  Sure, you can understand what
a large computer program such as a word processor does, as a single high-level
concept or a small set of such concepts, but your ability to understand it in
no way ensures that the program is correct and doesn't have hidden bugs.  Even
if you wrote the program yourself you can't really know this; most large
programs that you've written have had bugs that crop up at some later date, no
matter how careful you tried to be while writing them.

OK, so now it seems the reason you can't figure out how to make a
computer verify proofs is because each step really corresponds to a
million small steps.  Well, you say, a computer can do a million
calculations in a second, so maybe it's still practical to do.  Now the
puzzle becomes how to figure out what the million steps are that each
English-language step corresponds to.  Your mathematician friend hasn't
a clue, but suggests that maybe you would find the answer by studying
set theory.  Actually, your friend thinks you're a little off the wall
for even wondering such a thing.  For him, this is not what mathematics
is all about.

The subject of set theory keeps popping up, so you decide it's
time to look it up.

You decide to start off on a careful footing, so you start reading a couple of
very elementary books on set theory.  A lot of it seems pretty obvious, like
intersections, subsets, and Venn diagrams.  You thumb through one of the
books; nowhere is anything about axioms mentioned. The other book relegates to
an appendix a brief discussion that mentions a set of axioms called
``Zermelo--Fraenkel set theory''\index{Zermelo--Fraenkel set theory} and states
them in English.  You look at them and have no idea what they really mean or
what you can do with them.  The comments in this appendix say that the purpose
of mentioning them is to expose you to the idea, but imply that they are not
necessary for basic understanding and that they are really the subject matter
of advanced treatments where fine points such as a certain paradox (Russell's
paradox\index{Russell's paradox}\footnote{Russell's paradox assumes that there
exists a set $S$ that is a collection of all sets that don't contain
themselves.  Now, either $S$ contains itself or it doesn't.  If it contains
itself, it contradicts its definition.  But if it doesn't contain itself, it
also contradicts its definition.  Russell's paradox is resolved in ZF set
theory by denying that such a set $S$ exists.}) are resolved.  Wait a
minute---shouldn't the axioms be a starting point, not an ending point?  If
there are paradoxes that arise without the axioms, how do you know you won't
stumble across one accidentally when using the informal approach?

And nowhere do these books describe how ``all of mathematics can be
derived from set theory'' which by now you've heard a few times.

You find a more advanced book on set theory.  This one actually lists the
axioms of ZF set theory in plain English on page one.  {\em Now} you think
your quest has ended and you've finally found the source of all mathematical
knowledge; you just have to understand what it means.  Here, in one place, is
the basis for all of mathematics!  You stare at the axioms in awe, puzzle over
them, memorize them, hoping that if you just meditate on them long enough they
will become clear.  Of course, you haven't the slightest idea how the rest of
mathematics is ``derived'' from them; in particular, if these are the axioms
of mathematics, then why do arithmetic, group theory, and so on need their own
axioms?

You start reading this advanced book carefully, pondering the meaning of every
word, because by now you're really determined to get to the bottom of this.
The first thing the book does is explain how the axioms came about, which was
to resolve Russell's paradox.\index{Russell's paradox}  In fact that seems to
be the main purpose of their existence; that they supposedly can be used to
derive all of mathematics seems irrelevant and is not even mentioned.  Well,
you go on.  You hope the book will explain to you clearly, step by step, how
to derive things from the axioms.  After all, this is the starting point of
mathematics, like a book that explains the basics of a computer programming
language.  But something is missing.  You find you can't even understand the
first proof or do the first exercise.  Symbols such as $\exists$ and $\forall$
permeate the page without any mention of where they came from or how to
manipulate them; the author assumes you are totally familiar with them and
doesn't even tell you what they mean.  By now you know that $\exists$ means
``there exists'' and $\forall$ means ``for all,'' but shouldn't the rules for
manipulating these symbols be part of the axioms?  You still have no idea
how you could even describe the axioms to a computer.

Certainly there is something much different here from the technical
literature you're used to reading.  A computer language manual almost
always explains very clearly what all the symbols mean, precisely what
they do, and the rules used for combining them, and you work your way up
from there.

After glancing at four or five other such books, you come to the realization
that there is another whole field of study that you need just to get to the
point at which you can understand the axioms of set theory.  The field is
called ``logic.''  In fact, some of the books did recommend it as a
prerequisite, but it just didn't sink in.  You assumed logic was, well, just
logic, something that a person with common sense intuitively understood.  Why
waste your time reading boring treatises on symbolic logic, the manipulation
of 1's and 0's that computers do, when you already know that?  But this is a
different kind of logic, quite alien to you.  The subject of {\sc nand} and
{\sc nor} gates is not even touched upon or in any case has to do with only a
very small part of this field.

So your quest continues.  Skimming through the first couple of introductory
books, you get a general idea of what logic is about and what quantifiers
(``for all,'' ``there exists'') mean, but you find their examples somewhat
trivial and mildly annoying (``all dogs are animals,'' ``some animals are
dogs,'' and such).  But all you want to know is what the rules are for
manipulating the symbols so you can apply them to set theory.  Some formulas
describing the relationships among quantifiers ($\exists$ and $\forall$) are
listed in tables, along with some verbal reasoning to justify them.
Presumably, if you want to find out if a formula is correct, you go through
this same kind of mental reasoning process, possibly using images of dogs and
animals. Intuitively, the formulas seem to make sense.  But when you ask
yourself, ``What are the rules I need to get a computer to figure out whether
this formula is correct?'', you still don't know.  Certainly you don't ask the
computer to imagine dogs and animals.

You look at some more advanced logic books.  Many of them have an introductory
chapter summarizing set theory, which turns out to be a prerequisite.  You
need logic to understand set theory, but it seems you also need set theory to
understand logic!  These books jump right into proving rather advanced
theorems about logic, without offering the faintest clue about where the logic
came from that allows them to prove these theorems.

Luckily, you come across an elementary book of logic that, halfway through,
after the usual truth tables and metaphors, presents in a clear, precise way
what you've been looking for all along: the axioms!  They're divided into
propositional calculus (also called sentential logic) and predicate calculus
(also called first-order logic),\index{first-order logic} with rules so simple
and crystal clear that now you can finally program a computer to understand
them.  Indeed, they're no harder than learning how to play a game of chess.
As far as what you seem to need is concerned, the whole book could have been
written in five pages!

{\em Now} you think you've found the ultimate source of mathematical
truth.  So---the axioms of mathematics consist of these axioms of logic,
together with the axioms of ZF set theory. (By now you've also been able to
figure out how to translate the ZF axioms from English into the
actual symbols of logic which you can now manipulate according to
precise, easy-to-understand rules.)

Of course, you still don't understand how ``all of mathematics can be
derived from set theory,'' but maybe this will reveal itself in due
course.

You eagerly set out to program the axioms and rules into a computer and start
to look at the theorems you will have to prove as the logic is developed.  All
sorts of important theorems start popping up:  the deduction
theorem,\index{deduction theorem} the substitution theorem,\index{substitution
theorem} the completeness theorem of propositional calculus,\index{first-order
logic!completeness} the completeness theorem of predicate calculus.  Uh-oh,
there seems to be trouble.  They all get harder and harder, and not one of
them can be derived with the axioms and rules of logic you've just been
handed.  Instead, they all require ``metalogic'' for their proofs, a kind of
mixture of logic and set theory that allows you to prove things {\em about}
the axioms and theorems of logic rather than {\em with} them.

You plow ahead anyway.  A month later, you've spent much of your
free time getting the computer to verify proofs in propositional calculus.
You've programmed in the axioms, but you've also had to program in the
deduction theorem, the substitution theorem, and the completeness theorem of
propositional calculus, which by now you've resigned yourself to treating as
rather complex additional axioms, since they can't be proved from the axioms
you were given.  You can now get the computer to verify and even generate
complete, rigorous, formal proofs\index{formal proof}.  Never mind that they
may have 100,000 steps---at least now you can have complete, absolute
confidence in them.  Unfortunately, the only theorems you have proved are
pretty trivial and you can easily verify them in a few minutes with truth
tables, if not by inspection.

It looks like your mathematician friend was right.  Getting the computer to do
serious mathematics with this kind of rigor seems almost hopeless.  Even
worse, it seems that the further along you get, the more ``axioms'' you have
to add, as each new theorem seems to involve additional ``metamathematical''
reasoning that hasn't been formalized, and none of it can be derived from the
axioms of logic.  Not only do the proofs keep growing exponentially as you get
further along, but the program to verify them keeps getting bigger and bigger
as you program in more ``metatheorems.''\index{metatheorem}\footnote{A
metatheorem is usually a statement that is too general to be directly provable
in a theory.  For example, ``if $n_1$, $n_2$, and $n_3$ are integers, then
$n_1+n_2+n_3$ is an integer'' is a theorem of number theory.  But ``for any
integer $k > 1$, if $n_1, \ldots, n_k$ are integers, then $n_1+\ldots +n_k$ is
an integer'' is a metatheorem, in other words a family of theorems, one for
every $k$.  The reason it is not a theorem is that the general sum $n_1+\ldots
+n_k$ (as a function of $k$) is not an operation that can be defined directly
in number theory.} The bugs\index{computer program bugs} that have cropped up
so far have already made you start to lose faith in the rigor you seem to have
achieved, and you know it's just going to get worse as your program gets larger.

\subsection{Mathematics and the Non-Specialist}

\begin{quote}
  {\em A real proof is not checkable by a machine, or even by any mathematician
not privy to the gestalt, the mode of thought of the particular field of
mathematics in which the proof is located.}
  \flushright\sc  Davis and Hersh\index{Davis, Phillip J.}
  \footnote{\cite{Davis}, p.~354.}\\
\end{quote}

The bulk of abstract or theoretical mathematics is ordinarily outside
the reach of anyone but a few specialists in each field who have completed
the necessary difficult internship in order to enter its coterie.  The
typical intelligent layperson has no reasonable hope of understanding much of
it, nor even the specialist mathematician of understanding other fields.  It
is like a foreign language that has no dictionary to look up the translation;
the only way you can learn it is by living in the country for a few years.  It
is argued that the effort involved in learning a specialty is a necessary
process for acquiring a deep understanding.  Of course, this is almost certainly
true if one is to make significant contributions to a field; in particular,
``doing'' proofs is probably the most important part of a mathematician's
training.  But is it also necessary to deny outsiders access to it?  Is it
necessary that abstract mathematics be so hard for a layperson to grasp?

A computer normally is of no help whatsoever.  Most published proofs are
actually just series of hints written in an informal style that requires
considerable knowledge of the field to understand.  These are the ``real
proofs'' referred to by Davis and Hersh.\index{informal proof}  There is an
implicit understanding that, in principle, such a proof could be converted to
a complete formal proof\index{formal proof}.  However, it is said that no one
would ever attempt such a conversion, even if they could, because that would
presumably require millions of steps (Section~\ref{dream}).  Unfortunately the
informal style automatically excludes the understanding of the proof
by anyone who hasn't gone through the necessary apprenticeship. The
best that the intelligent layperson can do is to read popular books about deep
and famous results; while this can be helpful, it can also be misleading, and
the lack of detail usually leaves the reader with no ability whatsoever to
explore any aspect of the field being described.

The statements of theorems often use sophisticated notation that makes them
inaccessible to the non-specialist.  For a non-specialist who wants to achieve
a deeper understanding of a proof, the process of tracing definitions and
lemmas back through their hierarchy\index{hierarchy} quickly becomes confusing
and discouraging.  Textbooks are usually written to train mathematicians or to
communicate to people who are already mathematicians, and large gaps in proofs
are often left as exercises to the reader who is left at an impasse if he or
she becomes stuck.

I believe that eventually computers will enable non-specialists and even
intelligent laypersons to follow almost any mathematical proof in any field.
Metamath is an attempt in that direction.  If all of mathematics were as
easily accessible as a computer programming language, I could envision
computer programmers and hobbyists who otherwise lack mathematical
sophistication exploring and being amazed by the world of theorems and proofs
in obscure specialties, perhaps even coming up with results of their own.  A
tremendous advantage would be that anyone could experiment with conjectures in
any field---the computer would offer instant feedback as to whether
an inference step was correct.

Mathematicians sometimes have to put up with the annoyance of
cranks\index{cranks} who lack a fundamental understanding of mathematics but
insist that their ``proofs'' of, say, Fermat's Last Theorem\index{Fermat's
Last Theorem} be taken seriously.  I think part of the problem is that these
people are misled by informal mathematical language, treating it as if they
were reading ordinary expository English and failing to appreciate the
implicit underlying rigor.  Such cranks are rare in the field of computers,
because computer languages are much more explicit, and ultimately the proof is
in whether a computer program works or not.  With easily accessible
computer-based abstract mathematics, a mathematician could say to a crank,
``don't bother me until you've demonstrated your claim on the computer!''

% 22-May-04 nm
% Attempt to move De Millo quote so it doesn't separate from attribution
% CHANGE THIS NUMBER (AND ELIMINATE IF POSSIBLE) WHEN ABOVE TEXT CHANGES
\vspace{-0.5em}

\subsection{An Impossible Dream?}\label{dream}

\begin{quote}
  {\em Even quite basic theorems would demand almost unbelievably vast
  books to display their proofs.}
    \flushright\sc  Robert E. Edwards\footnote{\cite{Edwards}, p.~68.}\\
\end{quote}\index{Edwards, Robert E.}

\begin{quote}
  {\em Oh, of course no one ever really {\em does} it.  It would take
  forever!  You just show that you could do it, that's sufficient.}
    \flushright\sc  ``The Ideal Mathematician''
    \index{Davis, Phillip J.}\footnote{\cite{Davis},
p.~40.}\\
\end{quote}

\begin{quote}
  {\em There is a theorem in the primitive notation of set theory that
  corresponds to the arithmetic theorem `$1000+2000=3000$'.  The formula
  would be forbiddingly long\ldots even if [one] knows the definitions
  and is asked to simplify the long formula according to them, chances are
  he will make errors and arrive at some incorrect result.}
    \flushright\sc  Hao Wang\footnote{\cite{Wang}, p.~140.}\\
\end{quote}\index{Wang, Hao}

% 22-May-04 nm
% Attempt to move De Millo quote so it doesn't separate from attribution
% CHANGE THIS NUMBER (AND ELIMINATE IF POSSIBLE) WHEN ABOVE TEXT CHANGES
\vspace{-0.5em}

\begin{quote}
  {\em The {\em Principia Mathematica} was the crowning achievement of the
  formalists.  It was also the deathblow of the formalist view.\ldots
  {[Rus\-sell]} failed, in three enormous volumes, to get beyond the elementary
  facts of arithmetic.  He showed what can be done in principle and what
  cannot be done in practice.  If the mathematical process were really
  one of strict, logical progression, we would still be counting our
  fingers.\ldots
  One theoretician estimates\ldots that a demonstration of one of
  Ramanujan's conjectures assuming set theory and elementary analysis would
  take about two thousand pages; the length of a deduction from first principles
  is nearly in\-con\-ceiv\-a\-ble\ldots The probabilists argue that\ldots any
  very long proof can at best be viewed as only probably correct\ldots}
  \flushright\sc Richard de Millo et. al.\footnote{\cite{deMillo}, pp.~269,
  271.}\\
\end{quote}\index{de Millo, Richard}

A number of writers have conveyed the impression that the kind of absolute
rigor provided by Metamath\index{Metamath} is an impossible dream, suggesting
that a complete, formal verification\index{formal proof} of a typical theorem
would take millions of steps in untold volumes of books.  Even if it could be
done, the thinking sometimes goes, all meaning would be lost in such a
monstrous, tedious verification.\index{informal proof}\index{proof length}

These writers assume, however, that in order to achieve the kind of complete
formal verification they desire one must break down a proof into individual
primitive steps that make direct reference to the axioms.  This is
not necessary.  There is no reason not to make use of previously proved
theorems rather than proving them over and over.

Just as important, definitions\index{definition} can be introduced along
the way, allowing very complex formulas to be represented with few
symbols.  Not doing this can lead to absurdly long formulas.  For
example, the mere statement of
G\"{o}del's incompleteness theorem\index{G\"{o}del's
incompleteness theorem}, which can be expressed with a small number of
defined symbols, would require about 20,000 primitive symbols to express
it.\index{Boolos, George S.}\footnote{George S.\ Boolos, lecture at
Massachusetts Institute of Technology, spring 1990.} An extreme example
is Bourbaki's\label{bourbaki} language for set theory, which requires
4,523,659,424,929 symbols plus 1,179,618,517,981 disambiguatory links
(lines connecting symbol pairs, usually drawn below or above the
formula) to express the number
``one'' \cite{Mathias}.\index{Mathias, Adrian R. D.}\index{Bourbaki,
Nicolas}
% http://www.dpmms.cam.ac.uk/~ardm/

A hierarchy\index{hierarchy} of theorems and definitions permits an
exponential growth in the formula sizes and primitive proof steps to be
described with only a linear growth in the number of symbols used.  Of course,
this is how ordinary informal mathematics is normally done anyway, but with
Metamath\index{Metamath} it can be done with absolute rigor and precision.

\subsection{Beauty}


\begin{quote}
  {\em No one shall be able to drive us from the paradise that Cantor has
created for us.}
   \flushright\sc  David Hilbert\footnote{As quoted in \cite{Moore}, p.~131.}\\
\end{quote}\index{Hilbert, David}

\needspace{3\baselineskip}
\begin{quote}
  {\em Mathematics possesses not only truth, but some supreme beauty ---a
  beauty cold and austere, like that of a sculpture.}
    \flushright\sc  Bertrand
    Russell\footnote{\cite{Russell}.}\\
\end{quote}\index{Russell, Bertrand}

\begin{quote}
  {\em Euclid alone has looked on Beauty bare.}
  \flushright\sc Edna Millay\footnote{As quoted in \cite{Davis}, p.~150.}\\
\end{quote}\index{Millay, Edna}

For most people, abstract mathematics is distant, strange, and
incomprehensible.  Many popular books have tried to convey some of the sense
of beauty in famous theorems.  But even an intelligent layperson is left with
only a general idea of what a theorem is about and is hardly given the tools
needed to make use of it.  Traditionally, it is only after years of arduous
study that one can grasp the concepts needed for deep understanding.
Metamath\index{Metamath} allows you to approach the proof of the theorem from
a quite different perspective, peeling apart the formulas and definitions
layer by layer until an entirely different kind of understanding is achieved.
Every step of the proof is there, pieced together with absolute precision and
instantly available for inspection through a microscope with a magnification
as powerful as you desire.

A proof in itself can be considered an object of beauty.  Constructing an
elegant proof is an art.  Once a famous theorem has been proved, often
considerable effort is made to find simpler and more easily understood
proofs.  Creating and communicating elegant proofs is a major concern of
mathematicians.  Metamath is one way of providing a common language for
archiving and preserving this information.

The length of a proof can, to a certain extent, be considered an
objective measure of its ``beauty,'' since shorter proofs are usually
considered more elegant.  In the set theory database
\texttt{set.mm}\index{set theory database (\texttt{set.mm})}%
\index{Metamath Proof Explorer}
provided with Metamath, one goal was to make all proofs as short as possible.

\needspace{4\baselineskip}
\subsection{Simplicity}

\begin{quote}
  {\em God made man simple; man's complex problems are of his own
  devising.}
    \flushright\sc Eccles. 7:29\footnote{Jerusalem Bible.}\\
\end{quote}\index{Bible}

\needspace{3\baselineskip}
\begin{quote}
  {\em God made integers, all else is the work of man.}
    \flushright\sc Leopold Kronecker\footnote{{\em Jahresbericht
	der Deutschen Mathematiker-Vereinigung }, vol. 2, p. 19.}\\
\end{quote}\index{Kronecker, Leopold}

\needspace{3\baselineskip}
\begin{quote}
  {\em For what is clear and easily comprehended attracts; the
  complicated repels.}
    \flushright\sc David Hilbert\footnote{As quoted in \cite{deMillo},
p.~273.}\\
\end{quote}\index{Hilbert, David}

The Metamath\index{Metamath} language is simple and Spartan.  Metamath treats
all mathematical expressions as simple sequences of symbols, devoid of meaning.
The higher-level or ``metamathematical'' notions underlying Metamath are about
as simple as they could possibly be.  Each individual step in a proof involves
a single basic concept, the substitution of an expression for a variable, so
that in principle almost anyone, whether mathematician or not, can
completely understand how it was arrived at.

In one of its most basic applications, Metamath\index{Metamath} can be used to
develop the foundations of mathematics\index{foundations of mathematics} from
the very beginning.  This is done in the set theory database that is provided
with the Metamath package and is the subject matter
of Chapter~\ref{fol}. Any language (a metalanguage\index{metalanguage})
used to describe mathematics (an object language\index{object language}) must
have a mathematical content of its own, but it is desirable to keep this
content down to a bare minimum, namely that needed to make use of the
inference rules specified by the axioms.  With any metalanguage there is a
``chicken and egg'' problem somewhat like circular reasoning:  you must assume
the validity of the mathematics of the metalanguage in order to prove the
validity of the mathematics of the object language.  The mathematical content
of Metamath itself is quite limited.  Like the rules of a game of chess, the
essential concepts are simple enough so that virtually anyone should be able to
understand them (although that in itself will not let you play like
a master).  The symbols that Metamath manipulates do not in themselves
have any intrinsic meaning.  Your interpretation of the axioms that you supply
to Metamath is what gives them meaning.  Metamath is an attempt to strip down
mathematical thought to its bare essence and show you exactly how the symbols
are manipulated.

Philosophers and logicians, with various motivations, have often thought it
important to study ``weak'' fragments of logic\index{weak logic}
\cite{Anderson}\index{Anderson, Alan Ross} \cite{MegillBunder}\index{Megill,
Norman}\index{Bunder, Martin}, other unconventional systems of logic (such as
``modal'' logic\index{modal logic} \cite[ch.\ 27]{Boolos}\index{Boolos, George
S.}), and quantum logic\index{quantum logic} in physics
\cite{Pavicic}\index{Pavi{\v{c}}i{\'{c}}, M.}.  Metamath\index{Metamath}
provides a framework in which such systems can be expressed, with an absolute
precision that makes all underlying metamathematical assumptions rigorous and
crystal clear.

Some schools of philosophical thought, for example
intuitionism\index{intuitionism} and constructivism\index{constructivism},
demand that the notions underlying any mathematical system be as simple and
concrete as possible.  Metamath should meet the requirements of these
philosophies.  Metamath must be taught the symbols, axioms\index{axiom}, and
rules\index{rule} for a specific theory, from the skeptical (such as
intuitionism\index{intuitionism}\footnote{Intuitionism does not accept the law
of excluded middle (``either something is true or it is not true'').  See
\cite[p.~xi]{Tymoczko}\index{Tymoczko, Thomas} for discussion and references
on this topic.  Consider the theorem, ``There exist irrational numbers $a$ and
$b$ such that $a^b$ is rational.''  An intuitionist would reject the following
proof:  If $\sqrt{2}^{\sqrt{2}}$ is rational, we are done.  Otherwise, let
$a=\sqrt{2}^{\sqrt{2}}$ and $b=\sqrt{2}$. Then $a^b=2$, which is rational.})
to the bold (such as the axiom of choice in set theory\footnote{The axiom of
choice\index{Axiom of Choice} asserts that given any collection of pairwise
disjoint nonempty sets, there exists a set that has exactly one element in
common with each set of the collection.  It is used to prove many important
theorems in standard mathematics.  Some philosophers object to it because it
asserts the existence of a set without specifying what the set contains
\cite[p.~154]{Enderton}\index{Enderton, Herbert B.}.  In one foundation for
mathematics due to Quine\index{Quine, Willard Van Orman}, that has not been
otherwise shown to be inconsistent, the axiom of choice turns out to be false
\cite[p.~23]{Curry}\index{Curry, Haskell B.}.  The \texttt{show
trace{\char`\_}back} command of the Metamath program allows you to find out
whether the axiom of choice, or any other axiom, was assumed by a
proof.}\index{\texttt{show trace{\char`\_}back} command}).

The simplicity of the Metamath language lets the algorithm (computer program)
that verifies the validity of a Metamath proof be straightforward and
robust.  You can have confidence that the theorems it verifies really can be
derived from your axioms.

\subsection{Rigor}

\begin{quote}
  {\em Rigor became a goal with the Greeks\ldots But the efforts to
  pursue rigor to the utmost have led to an impasse in which there is
  no longer any agreement on what it really means.  Mathematics remains
  alive and vital, but only on a pragmatic basis.}
    \flushright\sc  Morris Kline\footnote{\cite{Kline}, p.~1209.}\\
\end{quote}\index{Kline, Morris}

Kline refers to a much deeper kind of rigor than that which we will discuss in
this section.  G\"{o}del's incompleteness theorem\index{G\"{o}del's
incompleteness theorem} showed that it is impossible to achieve absolute rigor
in standard mathematics because we can never prove that mathematics is
consistent (free from contradictions).\index{consistent theory}  If
mathematics is consistent, we will never know it, but must rely on faith.  If
mathematics is inconsistent, the best we can hope for is that some clever
future mathematician will discover the inconsistency.  In this case, the
axioms would probably be revised slightly to eliminate the inconsistency, as
was done in the case of Russell's paradox,\index{Russell's paradox} but the
bulk of mathematics would probably not be affected by such a discovery.
Russell's paradox, for example, did not affect most of the remarkable results
achieved by 19th-century and earlier mathematicians.  It mainly invalidated
some of Gottlob Frege's\index{Frege, Gottlob} work on the foundations of
mathematics in the late 1800's; in fact Frege's work inspired Russell's
discovery.  Despite the paradox, Frege's work contains important concepts that
have significantly influenced modern logic.  Kline's {\em Mathematics, The
Loss of Certainty} \cite{Klinel}\index{Kline, Morris} has an interesting
discussion of this topic.

What {\em can} be achieved with absolute certainty\index{certainty} is the
knowledge that if we assume the axioms are consistent and true, then the
results derived from them are true.  Part of the beauty of mathematics is that
it is the one area of human endeavor where absolute certainty can be achieved
in this sense.  A mathematical truth will remain such for eternity.  However,
our actual knowledge of whether a particular statement is a mathematical truth
is only as certain as the correctness of the proof that establishes it.  If
the proof of a statement is questionable or vague, we can't have absolute
confidence in the truth that the statement claims.

Let us look at some traditional ways of expressing proofs.

Except in the field of formal logic\index{formal logic}, almost all
traditional proofs in mathematics are really not proofs at all, but rather
proof outlines or hints as to how to go about constructing the proof.  Many
gaps\index{gaps in proofs} are left for the reader to fill in. There are
several reasons for this.  First, it is usually assumed in mathematical
literature that the person reading the proof is a mathematician familiar with
the specialty being described, and that the missing steps are obvious to such
a reader or at least that the reader is capable of filling them in.  This
attitude is fine for professional mathematicians in the specialty, but
unfortunately it often has the drawback of cutting off the rest of the world,
including mathematicians in other specialties, from understanding the proof.
We discussed one possible resolution to this on p.~\pageref{envision}.
Second, it is often assumed that a complete formal proof\index{formal proof}
would require countless millions of symbols (Section~\ref{dream}). This might
be true if the proof were to be expressed directly in terms of the axioms of
logic and set theory,\index{set theory} but it is usually not true if we allow
ourselves a hierarchy\index{hierarchy} of definitions and theorems to build
upon, using a notation that allows us to introduce new symbols, definitions,
and theorems in a precisely specified way.

Even in formal logic,\index{formal logic} formal proofs\index{formal proof}
that are considered complete still contain hidden or implicit information.
For example, a ``proof'' is usually defined as a sequence of
wffs,\index{well-formed formula (wff)}\footnote{A {\em wff} or well-formed
formula is a mathematical expression (string of symbols) constructed according
to some precise rules.  A formal mathematical system\index{formal system}
contains (1) the rules for constructing syntactically correct
wffs,\index{syntax rules} (2) a list of starting wffs called
axioms,\index{axiom} and (3) one or more rules prescribing how to derive new
wffs, called theorems\index{theorem}, from the axioms or previously derived
theorems.  An example of such a system is contained in
Metamath's\index{Metamath} set theory database, which defines a formal
system\index{formal system} from which all of standard mathematics can be
derived.  Section~\ref{startf} steps you through a complete example of a formal
system, and you may want to skim it now if you are unfamiliar with the
concept.} each of which is an axiom or follows from a rule applied to previous
wffs in the sequence.  The implicit part of the proof is the algorithm by
which a sequence of symbols is verified to be a valid wff, given the
definition of a wff.  The algorithm in this case is rather simple, but for a
computer to verify the proof,\index{automated proof verification} it must have
the algorithm built into its verification program.\footnote{It is possible, of
course, to specify wff construction syntax outside of the program itself
with a suitable input language (the Metamath language being an example), but
some proof-verification or theorem-proving programs lack the ability to extend
wff syntax in such a fashion.} If one deals exclusively with axioms and
elementary wffs, it is straightforward to implement such an algorithm.  But as
more and more definitions are added to the theory in order to make the
expression of wffs more compact, the algorithm becomes more and more
complicated.  A computer program that implements the algorithm becomes larger
and harder to understand as each definition is introduced, and thus more prone
to bugs.\index{computer program bugs}  The larger the program, the
more suspicious the mathematician may be about
the validity of its algorithms.  This is especially true because
computer programs are inherently hard to follow to begin with, and few people
enjoy verifying them manually in detail.

Metamath\index{Metamath} takes a different approach.  Metamath's ``knowledge''
is limited to the ability to substitute variables for expressions, subject to
some simple constraints.  Once the basic algorithm of Metamath is assumed to
be debugged, and perhaps independently confirmed, it
can be trusted once and for all.  The information that Metamath needs to
``understand'' mathematics is contained entirely in the body of knowledge
presented to Metamath.  Any errors in reasoning can only be errors in the
axioms or definitions contained in this body of knowledge.  As a
``constructive'' language\index{constructive language} Metamath has no
conditional branches or loops like the ones that make computer programs hard
to decipher; instead, the language can only build new sequences of symbols
from earlier sequences  of symbols.

The simplicity of the rules that underlie Metamath not only makes Metamath
easy to learn but also gives Metamath a great deal of flexibility. For
example, Metamath is not limited to describing standard first-order
logic\index{first-order logic}; higher-order logics\index{higher-order logic}
and fragments of logic\index{weak logic} can be described just as easily.
Metamath gives you the freedom to define whatever wff notation you prefer; it
has no built-in conception of the syntax of a wff.\index{well-formed formula
(wff)}  With suitable axioms and definitions, Metamath can even describe and
prove things about itself.\index{Metamath!self-description}  (John
Harrison\index{Harrison, John} discusses the ``reflection''
principle\index{reflection principle} involved in self-descriptive systems in
\cite{Harrison}.)

The flexibility of Metamath requires that its proofs specify a lot of detail,
much more than in an ordinary ``formal'' proof.\index{formal proof}  For
example, in an ordinary formal proof, a single step consists of displaying the
wff that constitutes that step.  In order for a computer program to verify
that the step is acceptable, it first must verify that the symbol sequence
being displayed is an acceptable wff.\index{automated proof verification} Most
proof verifiers have at least basic wff syntax built into their programs.
Metamath has no hard-wired knowledge of what constitutes a wff built into it;
instead every wff must be explicitly constructed based on rules defining wffs
that are present in a database.  Thus a single step in an ordinary formal
proof may be correspond to many steps in a Metamath proof. Despite the larger
number of steps, though, this does not mean that a Metamath proof must be
significantly larger than an ordinary formal proof. The reason is that since
we have constructed the wff from scratch, we know what the wff is, so there is
no reason to display it.  We only need to refer to a sequence of statements
that construct it.  In a sense, the display of the wff in an ordinary formal
proof is an implicit proof of its own validity as a wff; Metamath just makes
the proof explicit. (Section~\ref{proof} describes Metamath's proof notation.)

\section{Computers and Mathematicians}

\begin{quote}
  {\em The computer is important, but not to mathematics.}
    \flushright\sc  Paul Halmos\footnote{As quoted in \cite{Albers}, p.~121.}\\
\end{quote}\index{Halmos, Paul}

Pure mathematicians have traditionally been indifferent to computers, even to
the point of disdain.\index{computers and pure mathematics}  Computer science
itself is sometimes considered to fall in the mundane realm of ``applied''
mathematics, perhaps essential for the real world but intellectually unexciting
to those who seek the deepest truths in mathematics.  Perhaps a reason for this
attitude towards computers is that there is little or no computer software that
meets their needs, and there may be a general feeling that such software could
not even exist.  On the one hand, there are the practical computer algebra
systems, which can perform amazing symbolic manipulations in algebra and
calculus,\index{computer algebra system} yet can't prove the simplest
existence theorem, if the idea of a proof is present at all.  On the other
hand, there are specialized automated theorem provers that technically speaking
may generate correct proofs.\index{automated theorem proving}  But sometimes
their specialized input notation may be cryptic and their output perceived to
be long, inelegant, incomprehensible proofs.    The output
may be viewed with suspicion, since the program that generates it tends to be
very large, and its size increases the potential for bugs\index{computer
program bugs}.  Such a proof may be considered trustworthy only if
independently verified and ``understood'' by a human, but no one wants to
waste their time on such a boring, unrewarding chore.



\needspace{4\baselineskip}
\subsection{Trusting the Computer}

\begin{quote}
  {\em \ldots I continue to find the quasi-empirical interpretation of
  computer proofs to be the more plausible.\ldots Since not
  everything that claims to be a computer proof can be
  accepted as valid, what are the mathematical criteria for acceptable
  computer proofs?}
    \flushright\sc  Thomas Tymoczko\footnote{\cite{Tymoczko}, p.~245.}\\
\end{quote}\index{Tymoczko, Thomas}

In some cases, computers have been essential tools for proving famous
theorems.  But if a proof is so long and obscure that it can be verified in a
practical way only with a computer, it is vaguely felt to be suspicious.  For
example, proving the famous four-color theorem\index{four-color
theorem}\index{proof length} (``a map needs no more than four colors to
prevent any two adjacent countries from having the same color'') can presently
only be done with the aid of a very complex computer program which originally
required 1200 hours of computer time. There has been considerable debate about
whether such a proof can be trusted and whether such a proof is ``real''
mathematics \cite{Swart}\index{Swart, E. R.}.\index{trusting computers}

However, under normal circumstances even a skeptical mathematician would have a
great deal of confidence in the result of multiplying two numbers on a pocket
calculator, even though the precise details of what goes on are hidden from its
user.  Even the verification on a supercomputer that a huge number is prime is
trusted, especially if there is independent verification; no one bothers to
debate the philosophical significance of its ``proof,'' even though the actual
proof would be so large that it would be completely impractical to ever write
it down on paper.  It seems that if the algorithm used by the computer is
simple enough to be readily understood, then the computer can be trusted.

Metamath\index{Metamath} adopts this philosophy.  The simplicity of its
language makes it easy to learn, and because of its simplicity one can have
essentially absolute confidence that a proof is correct. All axioms, rules, and
definitions are available for inspection at any time because they are defined
by the user; there are no hidden or built-in rules that may be prone to subtle
bugs\index{computer program bugs}.  The basic algorithm at the heart of
Metamath is simple and fixed, and it can be assumed to be bug-free and robust
with a degree of confidence approaching certainty.
Independently written implementations of the Metamath verifier
can reduce any residual doubt on the part of a skeptic even further;
there are now over a dozen such implementations, written by many people.

\subsection{Trusting the Mathematician}\label{trust}

\begin{quote}
  {\em There is no Algebraist nor Mathematician so expert in his science, as
  to place entire confidence in any truth immediately upon his discovery of it,
  or regard it as any thing, but a mere probability.  Every time he runs over
  his proofs, his confidence encreases; but still more by the approbation of
  his friends; and is rais'd to its utmost perfection by the universal assent
  and applauses of the learned world.}
  \flushright\sc David Hume\footnote{{\em A Treatise of Human Nature}, as
  quoted in \cite{deMillo}, p.~267.}\\
\end{quote}\index{Hume, David}

\begin{quote}
  {\em Stanislaw Ulam estimates that mathematicians publish 200,000 theorems
  every year.  A number of these are subsequently contradicted or otherwise
  disallowed, others are thrown into doubt, and most are ignored.}
  \flushright\sc Richard de Millo et. al.\footnote{\cite{deMillo}, p.~269.}\\
\end{quote}\index{Ulam, Stanislaw}

Whether or not the computer can be trusted, humans  of course will occasionally
err. Only the most memorable proofs get independently verified, and of these
only a handful of truly great ones achieve the status of being ``known''
mathematical truths that are used without giving a second thought to their
correctness.

There are many famous examples of incorrect theorems and proofs in
mathematical literature.\index{errors in proofs}

\begin{itemize}
\item There have been thousands of purported proofs of Fermat's Last
Theorem\index{Fermat's Last Theorem} (``no integer solutions exist to $x^n +
y^n = z^n$ for $n > 2$''), by amateurs, cranks, and well-regarded
mathematicians \cite[p.~5]{Stark}\index{Stark, Harold M}.  Fermat wrote a note
in his copy of Bachet's {\em Diophantus} that he found ``a truly marvelous
proof of this theorem but this margin is too narrow to contain it''
\cite[p.~507]{Kramer}.  A recent, much publicized proof by Yoichi
Miyaoka\index{Miyaoka, Yoichi} was shown to be incorrect ({\em Science News},
April 9, 1988, p.~230).  The theorem was finally proved by Andrew
Wiles\index{Wiles, Andrew} ({\em Science News}, July 3, 1993, p.~5), but it
initially had some gaps and took over a year after its announcement to be
checked thoroughly by experts.  On Oct. 25, 1994, Wiles announced that the last
gap found in his proof had been filled in.
  \item In 1882, M. Pasch discovered that an axiom was omitted from Euclid's
formulation of geometry\index{Euclidean geometry}; without it, the proofs of
important theorems of Euclid are not valid.  Pasch's axiom\index{Pasch's
axiom} states that a line that intersects one side of a triangle must also
intersect another side, provided that it does not touch any of the triangle's
vertices.  The omission of Pasch's axiom went unnoticed for 2000
years \cite[p.~160]{Davis}, in spite of (one presumes) the thousands of
students, instructors, and mathematicians who studied Euclid.
  \item The first published proof of the famous Schr\"{o}der--Bernstein
theorem\index{Schr\"{o}der--Bernstein theorem} in set theory was incorrect
\cite[p.~148]{Enderton}\index{Enderton, Herbert B.}.  This theorem states
that if there exists a 1-to-1 function\footnote{A {\em set}\index{set} is any
collection of objects. A {\em function}\index{function} or {\em
mapping}\index{mapping} is a rule that assigns to each element of one set
(called the function's {\em domain}\index{domain}) an element from another
set.} from set $A$ into set $B$ and vice-versa, then sets $A$ and $B$ have
a 1-to-1 correspondence.  Although it sounds simple and obvious,
the standard proof is quite long and complex.
  \item In the early 1900's, Hilbert\index{Hilbert, David} published a
purported proof of the continuum hypothesis\index{continuum hypothesis}, which
was eventually established as unprovable by Cohen\index{Cohen, Paul} in 1963
\cite[p.~166]{Enderton}.  The continuum hypothesis states that no
infinity\index{infinity} (``transfinite cardinal number'')\index{cardinal,
transfinite} exists whose size (or ``cardinality''\index{cardinality}) is
between the size of the set of integers and the size of the set of real
numbers.  This hypothesis originated with German mathematician Georg
Cantor\index{Cantor, Georg} in the late 1800's, and his inability to prove it
is said to have contributed to mental illness that afflicted him in his later
years.
  \item An incorrect proof of the four-color theorem\index{four-color theorem}
was published by Kempe\index{Kempe, A. B.} in 1879
\cite[p.~582]{Courant}\index{Courant, Richard}; it stood for 11 years before
its flaw was discovered.  This theorem states that any map can be colored
using only four colors, so that no two adjacent countries have the same
color.  In 1976 the theorem was finally proved by the famous computer-assisted
proof of Haken, Appel, and Koch \cite{Swart}\index{Appel, K.}\index{Haken,
W.}\index{Koch, K.}.  Or at least it seems that way.  Mathematician
H.~S.~M.~Coxeter\index{Coxeter, H. S. M.} has doubts \cite[p.~58]{Davis}:  ``I
have a feeling that that is an untidy kind of use of the computers, and the more
you correspond with Haken and Appel, the more shaky you seem to be.''
  \item Many false ``proofs'' of the Poincar\'{e}
conjecture\index{Poincar\'{e} conjecture} have been proposed over the years.
This conjecture states that any object that mathematically behaves like a
three-dimensional sphere is a three-dimensional sphere topologically,
regardless of how it is distorted.  In March 1986, mathematicians Colin
Rourke\index{Rourke, Colin} and Eduardo R\^{e}go\index{R\^{e}go, Eduardo}
caused  a stir in the mathematical community by announcing that they had found
a proof; in November of that year the proof was found to be false \cite[p.
218]{PetersonI}.  It was finally proved in 2003 by Grigory Perelman
\label{poincare}\index{Szpiro, George}\index{Perelman, Grigory}\cite{Szpiro}.
 \end{itemize}

Many counterexamples to ``theorems'' in recent mathematical
literature related to Clifford algebras\index{Clifford algebras}
 have been found by Pertti
Lounesto (who passed away in 2002).\index{Lounesto, Pertti}
See the web page \url{http://mathforum.org/library/view/4933.html}.
% http://users.tkk.fi/~ppuska/mirror/Lounesto/counterexamples.htm

One of the purposes of Metamath\index{Metamath} is to allow proofs to be
expressed with absolute precision.  Developing a proof in the Metamath
language can be challenging, because Metamath will not permit even the
tiniest mistake.\index{errors in proofs}  But once the proof is created, its
correctness can be trusted immediately, without having to depend on the
process of peer review for confirmation.

\section{The Use of Computers in Mathematics}

\subsection{Computer Algebra Systems}

For the most part, you will find that Metamath\index{Metamath} is not a
practical tool for manipulating numbers.  (Even proving that $2 + 2 = 4$, if
you start with set theory, can be quite complex!)  Several commercial
mathematics packages are quite good at arithmetic, algebra, and calculus, and
as practical tools they are invaluable.\index{computer algebra system} But
they have no notion of proof, and cannot understand statements starting with
``there exists such and such...''.

Software packages such as Mathematica \cite{Wolfram}\index{Mathematica} do not
concern themselves with proofs but instead work directly with known results.
These packages primarily emphasize heuristic rules such as the substitution of
equals for equals to achieve simpler expressions or expressions in a different
form.  Starting with a rich collection of built-in rules and algorithms, users
can add to the collection by means of a powerful programming language.
However, results such as, say, the existence of a certain abstract object
without displaying the actual object cannot be expressed (directly) in their
languages.  The idea of a proof from a small set of axioms is absent.  Instead
this software simply assumes that each fact or rule you add to the built-in
collection of algorithms is valid.  One way to view the software is as a large
collection of axioms from which the software, with certain goals, attempts to
derive new theorems, for example equating a complex expression with a simpler
equivalent. But the terms ``theorem''\index{theorem} and
``proof,''\index{proof} for example, are not even mentioned in the index of
the user's manual for Mathematica.\index{Mathematica and proofs}  What is also
unsatisfactory from a philosophical point of view is that there is no way to
ensure the validity of the results other than by trusting the writer of each
application module or tediously checking each module by hand, similar to
checking a computer program for bugs.\index{computer program
bugs}\footnote{Two examples illustrate why the knowledge database of computer
algebra systems should sometimes be regarded with a certain caution.  If you
ask Mathematica (version 3.0) to \texttt{Solve[x\^{ }n + y\^{ }n == z\^{ }n , n]}
it will respond with \texttt{\{\{n-\char`\>-2\}, \{n-\char`\>-1\},
\{n-\char`\>1\}, \{n-\char`\>2\}\}}. In other words, Mathematica seems to
``know'' that Fermat's Last Theorem\index{Fermat's Last Theorem} is true!  (At
the time this version of Mathematica was released this fact was unknown.)  If
you ask Maple\index{Maple} to \texttt{solve(x\^{ }2 = 2\^{ }x)} then
\texttt{simplify(\{"\})}, it returns the solution set \texttt{\{2, 4\}}, apparently
unaware that $-0.7666647$\ldots is also a solution.} While of course extremely
valuable in applied mathematics, computer algebra systems tend to be of little
interest to the theoretical mathematician except as aids for exploring certain
specific problems.

Because of possible bugs, trusting the output of a computer algebra system for
use as theorems in a proof-verifier would defeat the latter's goal of rigor.
On the other hand, a fact such that a certain relatively large number is
prime, while easy for a computer algebra system to derive, might have a long,
tedious proof that could overwhelm a proof-verifier. One approach for linking
computer algebra systems to a proof-verifier while retaining the advantages of
both is to add a hypothesis to each such theorem indicating its source.  For
example, a constant {\sc maple} could indicate the theorem came from the Maple
package, and would mean ``assuming Maple is consistent, then\ldots''  This and
many other topics concerning the formalization of mathematics are discussed in
John Harrison's\index{Harrison, John} very interesting
PhD thesis~\cite{Harrison-thesis}.

\subsection{Automated Theorem Provers}\label{theoremprovers}

A mathematical theory is ``decidable''\index{decidable theory} if a mechanical
method or algorithm exists that is guaranteed to determine whether or not a
particular formula is a theorem.  Among the few theories that are decidable is
elementary geometry,\index{Euclidean geometry} as was shown by a classic
result of logician Alfred Tarski\index{Tarski, Alfred} in 1948
\cite{Tarski}.\footnote{Tarski's result actually applies to a subset of the
geometry discussed in elementary textbooks.  This subset includes most of what
would be considered elementary geometry but it is not powerful enough to
express, among other things, the notions of the circumference and area of a
circle.  Extending the theory in a way that includes notions such as these
makes the theory undecidable, as was also shown by Tarski.  Tarski's algorithm
is far too inefficient to implement practically on a computer.  A practical
algorithm for proving a smaller subset of geometry theorems (those not
involving concepts of ``order'' or ``continuity'') was discovered by Chinese
mathematician Wu Wen-ts\"{u}n in 1977 \cite{Chou}\index{Chou,
Shang-Ching}.}\index{Wen-ts{\"{u}}n, Wu}  But most theories, including
elementary arithmetic, are undecidable.  This fact contributes to keeping
mathematics alive and well, since many mathematicians believe
that they will never be
replaced by computers (if they believe Roger Penrose's argument that a
computer can never replace the brain \cite{Penrose}\index{Penrose, Roger}).
In fact,  elementary geometry is often considered a ``dead'' field
for the simple reason that it is decidable.

On the other hand, the undecidability of a theory does not mean that one cannot
use a computer to search for proofs, providing one is willing to give up if a
proof is not found after a reasonable amount of time.  The field of automated
theorem proving\index{automated theorem proving} specializes in pursuing such
computer searches.  Among the more successful results to date are those based
on an algorithm known as Robinson's resolution principle
\cite{Robinson}\index{Robinson's resolution principle}.

Automated theorem provers can be excellent tools for those willing to learn
how to use them.  But they are not widely used in mainstream pure
mathematics, even though they could probably be useful in many areas.  There
are several reasons for this.  Probably most important, the main goal in pure
mathematics is to arrive at results that are considered to be deep or
important; proving them is essential but secondary.  Usually, an automated
theorem prover cannot assist in this main goal, and by the time the main goal
is achieved, the mathematician may have already figured out the proof as a
by-product.  There is also a notational problem.  Mathematicians are used to
using very compact syntax where one or two symbols (heavily dependent on
context) can represent very complex concepts; this is part of the
hierarchy\index{hierarchy} they have built up to tackle difficult problems.  A
theorem prover on the other hand might require that a theorem be expressed in
``first-order logic,''\index{first-order logic} which is the logic on which
most of mathematics is ultimately based but which is not ordinarily used
directly because expressions can become very long.  Some automated theorem
provers are experimental programs, limited in their use to very specialized
areas, and the goal of many is simply research into the nature of automated
theorem proving itself.  Finally, much research remains to be done to enable
them to prove very deep theorems.  One significant result was a
computer proof by Larry Wos\index{Wos, Larry} and colleagues that every Robbins
algebra\index{Robbins algebra} is a Boolean  algebra\index{Boolean algebra}
({\em New York Times}, Dec. 10, 1996).\footnote{In 1933, E.~V.\
Huntington\index{Huntington, E. V.}
presented the following axiom system for
Boolean algebra with a unary operation $n$ and a binary operation $+$:
\begin{center}
    $x + y = y + x$ \\
    $(x + y) + z = x + (y + z)$ \\
    $n(n(x) + y) + n(n(x) + n(y)) = x$
\end{center}
Herbert Robbins\index{Robbins, Herbert}, a student of Huntington, conjectured
that the last equation can be replaced with a simpler one:
\begin{center}
    $n(n(x + y) + n(x + n(y))) = x$
\end{center}
Robbins and Huntington could not find a proof.  The problem was
later studied unsuccessfully by Tarski\index{Tarski, Alfred} and his
students, and it remained an unsolved problem until a
computer found the proof in 1996.  For more information on
the Robbins algebra problem see \cite{Wos}.}

How does Metamath\index{Metamath} relate to automated theorem provers?  A
theorem prover is primarily concerned with one theorem at a time (perhaps
tapping into a small database of known theorems) whereas Metamath is more like
a theorem archiving system, storing both the theorem and its proof in a
database for access and verification.  Metamath is one answer to ``what do you
do with the output of a theorem prover?''  and could be viewed as the
next step in the process.  Automated theorem provers could be useful tools for
helping develop its database.
Note that very long, automatically
generated proofs can make your database fat and ugly and cause Metamath's proof
verification to take a long time to run.  Unless you have a particularly good
program that generates very concise proofs, it might be best to consider the
use of automatically generated proofs as a quick-and-dirty approach, to be
manually rewritten at some later date.

The program {\sc otter}\index{otter@{\sc otter}}\footnote{\url{http://www.cs.unm.edu/\~mccune/otter/}.}, later succeeded by
prover9\index{prover9}\footnote{\url{https://www.cs.unm.edu/~mccune/mace4/}.},
have been historically influential.
The E prover\index{E prover}\footnote{\url{https://github.com/eprover/eprover}.}
is a maintained automated theorem prover
for full first-order logic with equality.
There are many other automated theorem provers as well.

If you want to combine automated theorem provers with Metamath
consider investigating
the book {\em Automated Reasoning:  Introduction and Applications}
\cite{Wos}\index{Wos, Larry}.  This book discusses
how to use {\sc otter} in a way that can
not only able to generate
relatively efficient proofs, it can even be instructed to search for
shorter proofs.  The effective use of {\sc otter} (and similar tools)
does require a certain
amount of experience, skill, and patience.  The axiom system used in the
\texttt{set.mm}\index{set theory database (\texttt{set.mm})} set theory
database can be expressed to {\sc otter} using a method described in
\cite{Megill}.\index{Megill, Norman}\footnote{To use those axioms with
{\sc otter}, they must be restated in a way that eliminates the need for
``dummy variables.''\index{dummy variable!eliminating} See the Comment
on p.~\pageref{nodd}.} When successful, this method tends to generate
short and clever proofs, but my experiments with it indicate that the
method will find proofs within a reasonable time only for relatively
easy theorems.  It is still fun to experiment with.

Reference \cite{Bledsoe}\index{Bledsoe, W. W.} surveys a number of approaches
people have explored in the field of automated theorem proving\index{automated
theorem proving}.

\subsection{Interactive Theorem Provers}\label{interactivetheoremprovers}

Finding proofs completely automatically is difficult, so there
are some interactive theorem provers that allow a human to guide the
computer to find a proof.
Examples include
HOL Light\index{HOL light}%
\footnote{\url{https://www.cl.cam.ac.uk/~jrh13/hol-light/}.},
Isabelle\index{Isabelle}%
\footnote{\url{http://www.cl.cam.ac.uk/Research/HVG/Isabelle}.},
{\sc hol}\index{hol@{\sc hol}}%
\footnote{\url{https://hol-theorem-prover.org/}.},
and
Coq\index{Coq}\footnote{\url{https://coq.inria.fr/}.}.

A major difference between most of these tools and Metamath is that the
``proofs'' are actually programs that guide the program to find a proof,
and not the proof itself.
For example, an Isabelle/HOL proof might apply a step
\texttt{apply (blast dest: rearrange reduction)}. The \texttt{blast}
instruction applies
an automatic tableux prover and returns if it found a sequence of proof
steps that work... but the sequence is not considered part of the proof.

A good overview of
higher-level proof verification languages (such as {\sc lcf}\index{lcf@{\sc
lcf}} and {\sc hol}\index{hol@{\sc hol}})
is given in \cite{Harrison}.  All of these languages are fundamentally
different from Metamath in that much of the mathematical foundational
knowledge is embedded in the underlying proof-verification program, rather
than placed directly in the database that is being verified.
These can have a steep learning curve for those without a mathematical
background.  For example, one usually must have a fair understanding of
mathematical logic in order to follow their proofs.

\subsection{Proof Verifiers}\label{proofverifiers}

A proof verifier is a program that doesn't generate proofs but instead
verifies proofs that you give it.  Many proof verifiers have limited built-in
automated proof capabilities, such as figuring out simple logical inferences
(while still being guided by a person who provides the overall proof).  Metamath
has no built-in automated proof capability other than the limited
capability of its Proof Assistant.

Proof-verification languages are not used as frequently as they might be.
Pure mathematicians are more concerned with producing new results, and such
detail and rigor would interfere with that goal.  The use of computers in pure
mathematics is primarily focused on automated theorem provers (not verifiers),
again with the ultimate goal of aiding the creation of new mathematics.
Automated theorem provers are usually concerned with attacking one theorem at
time rather than making a large, organized database easily available to the
user.  Metamath is one way to help close this gap.

By itself Metamath is a mostly a proof verifier.
This does not mean that other approaches can't be used; the difference
is that in Metamath, the results of various provers must be recorded
step-by-step so that they can be verified.

Another proof-verification language is Mizar,\index{Mizar} which can display
its proofs in the informal language that mathematicians are accustomed to.
Information on the Mizar language is available at \url{http://mizar.org}.

For the working mathematician, Mizar is an excellent tool for rigorously
documenting proofs. Mizar typesets its proofs in the informal English used by
mathematicians (and, while fine for them, are just as inscrutable by
laypersons!). A price paid for Mizar is a relatively steep learning curve of a
couple of weeks.  Several mathematicians are actively formalizing different
areas of mathematics using Mizar and publishing the proofs in a dedicated
journal. Unfortunately the task of formalizing mathematics is still looked
down upon to a certain extent since it doesn't involve the creation of ``new''
mathematics.

The closest system to Metamath is
the {\em Ghilbert}\index{Ghilbert} proof language (\url{http://ghilbert.org})
system developed by
Raph Levien\index{Levien, Raph}.
Ghilbert is a formal proof checker heavily inspired by Metamath.
Ghilbert statements are s-expressions (a la Lisp), which is easy
for computers to parse but many people find them hard to read.
There are a number of differences in their specific constructs, but
there is at least one tool to translate some Metamath materials into Ghilbert.
As of 2019 the Ghilbert community is smaller and less active than the
Metamath community.
That said, the Metamath and Ghilbert communities overlap, and fruitful
conversations between them have occurred many times over the years.

\subsection{Creating a Database of Formalized Mathematics}\label{mathdatabase}

Besides Metamath, there are several other ongoing projects with the goal of
formalizing mathematics into computer-verifiable databases.
Understanding some history will help.

The {\sc qed}\index{qed project@{\sc qed} project}%
\footnote{\url{http://www-unix.mcs.anl.gov/qed}.}
project arose in 1993 and its goals were outlined in the
{\sc qed} manifesto.
The {\sc qed} manifesto was
a proposal for a computer-based database of all mathematical knowledge,
strictly formalized and with all proofs having been checked automatically.
The project had a conference in 1994 and another in 1995;
there was also a ``twenty years of the {\sc qed} manifesto'' workshop
in 2014.
Its ideals are regularly reraised.

In a 2007 paper, Freek Wiedijk identified two reasons
for the failure of the {\sc qed} project as originally envisioned:%
\cite{Wiedijk-revisited}\index{Wiedijk, Freek}

\begin{itemize}
\item Very few people are working on formalization of mathematics. There is no compelling application for fully mechanized mathematics.
\item Formalized mathematics does not yet resemble traditional mathematics. This is partly due to the complexity of mathematical notation, and partly to the limitations of existing theorem provers and proof assistants.
\end{itemize}

But this did not end the dream of
formalizing mathematics into computer-verifiable databases.
The problems that led to the {\sc qed} manifesto are still with us,
even though the challenges were harder than originally considered.
What has happened instead is that various independent projects have
worked towards formalizing mathematics into computer-verifiable databases,
each simultaneously competing and cooperating with each other.

A concrete way to see this is
Freek Wiedijk's ``Formalizing 100 Theorems'' list%
\footnote{\url{http://www.cs.ru.nl/\%7Efreek/100/}.}
which shows the progress different systems have made on a challenge list
of 100 mathematical theorems.%
\footnote{ This is not the only list of ``interesting'' theorems.
Another interesting list was posted by Oliver Knill's list
\cite{Knill}\index{Knill, Oliver}.}
The top systems as of February 2019
(in order of the number of challenges completed) are
HOL Light, Isabelle, Metamath, Coq, and Mizar.

The Metamath 100%
\footnote{\url{http://us.metamath.org/mm\_100.html}}
page (maintained by David A. Wheeler\index{Wheeler, David A.})
shows the progress of Metamath (specifically its \texttt{set.mm} database)
against this challenge list maintained by Freek Wiedijk.
The Metamath \texttt{set.mm} database
has made a lot of progress over the years,
in part because working to prove those challenge theorems required
defining various terms and proving their properties as a prerequisite.
Here are just a few of the many statements that have been
formally proven with Metamath:

% The entries of this cause the narrow display to break poorly,
% since the short amount of text means LaTeX doesn't get a lot to work with
% and the itemize format gives it even *less* margin than usual.
% No one will mind if we make just this list flushleft, since this list
% will be internally consistent.
\begin{flushleft}
\begin{itemize}
\item 1. The Irrationality of the Square Root of 2
  (\texttt{sqr2irr}, by Norman Megill, 2001-08-20)
\item 2. The Fundamental Theorem of Algebra
  (\texttt{fta}, by Mario Carneiro, 2014-09-15)
\item 22. The Non-Denumerability of the Continuum
  (\texttt{ruc}, by Norman Megill, 2004-08-13)
\item 54. The Konigsberg Bridge Problem
  (\texttt{konigsberg}, by Mario Carneiro, 2015-04-16)
\item 83. The Friendship Theorem
  (\texttt{friendship}, by Alexander W. van der Vekens, 2018-10-09)
\end{itemize}
\end{flushleft}

We thank all of those who have developed at least one of the Metamath 100
proofs, and we particularly thank
Mario Carneiro\index{Carneiro, Mario}
who has contributed the most Metamath 100 proofs as of 2019.
The Metamath 100 page shows the list of all people who have contributed a
proof, and links to graphs and charts showing progress over time.
We encourage others to work on proving theorems not yet proven in Metamath,
since doing so improves the work as a whole.

Each of the math formalization systems (including Metamath)
has different strengths and weaknesses, depending on what you value.
Key aspects that differentiate Metamath from the other top systems are:

\begin{itemize}
\item Metamath is not tied to any particular set of axioms.
\item Metamath can show every step of every proof, no exceptions.
  Most other provers only assert that a proof can be found, and do not
  show every step. This also makes verification fast, because
  the system does not need to rediscover proof details.
\item The Metamath verifier has been re-implemented in many different
  programming languages, so verification can be done by multiple
  implementations.  In particular, the
  \texttt{set.mm}\index{set theory database (\texttt{set.mm})}%
  \index{Metamath Proof Explorer} database is verified by
  four different verifiers
  written in four different languages by four different authors.
  This greatly reduces the risk of accepting an invalid
  proof due to an error in the verifier.
\item Proofs stay proven.  In some systems, changes to the system's
  syntax or how a tactic works causes proofs to fail in later versions,
  causing older work to become essentially lost.
  Metamath's language is
  extremely small and fixed, so once a proof is added to a database,
  the database can be rechecked with later versions of the Metamath program
  and with other verifiers of Metamath databases.
  If an axiom or key definition needs to be changed, it is easy to
  manipulate the database as a whole to handle the change
  without touching the underlying verifier.
  Since re-verification of an entire database takes seconds, there
  is never a reason to delay complete verification.
  This aspect is especially compelling if your
  goal is to have a long-term database of proofs.
\item Licensing is generous.  The main Metamath databases are released to
  the public domain, and the main Metamath program is open source software
  under a standard, widely-used license.
\item Substitutions are easy to understand, even by those who are not
  professional mathematicians.
\end{itemize}

Of course, other systems may have advantages over Metamath
that are more compelling, depending on what you value.
In any case, we hope this helps you understand Metamath
within a wider context.

\subsection{In Summary}\label{computers-summary}

To summarize our discussions of computers and mathematics, computer algebra
systems can be viewed as theorem generators focusing on a narrow realm of
mathematics (numbers and their properties), automated theorem provers as proof
generators for specific theorems in a much broader realm covered by a built-in
formal system such as first-order logic, interactive theorem
provers require human guidance, proof verifiers verify proofs but
historically they have been
restricted to first-order logic.
Metamath, in contrast,
is a proof verifier and documenter whose realm is essentially unlimited.

\section{Mathematics and Metamath}

\subsection{Standard Mathematics}

There are a number of ways that Metamath\index{Metamath} can be used with
standard mathematics.  The most satisfying way philosophically is to start at
the very beginning, and develop the desired mathematics from the axioms of
logic and set theory.\index{set theory}  This is the approach taken in the
\texttt{set.mm}\index{set theory database (\texttt{set.mm})}%
\index{Metamath Proof Explorer}
database (also known as the Metamath Proof Explorer).
Among other things, this database builds up to the
axioms of real and complex numbers\index{analysis}\index{real and complex
numbers} (see Section~\ref{real}), and a standard development of analysis, for
example, could start at that point, using it as a basis.   Besides this
philosophical advantage, there are practical advantages to having all of the
tools of set theory available in the supporting infrastructure.

On the other hand, you may wish to start with the standard axioms of a
mathematical theory without going through the set theoretical proofs of those
axioms.  You will need mathematical logic to make inferences, but if you wish
you can simply introduce theorems\index{theorem} of logic as
``axioms''\index{axiom} wherever you need them, with the implicit assumption
that in principle they can be proved, if they are obvious to you.  If you
choose this approach, you will probably want to review the notation used in
\texttt{set.mm}\index{set theory database (\texttt{set.mm})} so that your own
notation will be consistent with it.

\subsection{Other Formal Systems}
\index{formal system}

Unlike some programs, Metamath\index{Metamath} is not limited to any specific
area of mathematics, nor committed to any particular mathematical philosophy
such as classical logic versus intuitionism, nor limited, say, to expressions
in first-order logic.  Although the database \texttt{set.mm}
describes standard logic and set theory, Meta\-math
is actually a general-purpose language for describing a wide variety of formal
systems.\index{formal system}  Non-standard systems such as modal
logic,\index{modal logic} intuitionist logic\index{intuitionism}, higher-order
logic\index{higher-order logic}, quantum logic\index{quantum logic}, and
category theory\index{category theory} can all be described with the Metamath
language.  You define the symbols you prefer and tell Metamath the axioms and
rules you want to start from, and Metamath will verify any inferences you make
from those axioms and rules.  A simple example of a non-standard formal system
is Hofstadter's\index{Hofstadter, Douglas R.} MIU system,\index{MIU-system}
whose Metamath description is presented in Appendix~\ref{MIU}.

This is not hypothetical.
The largest Metamath database is
\texttt{set.mm}\index{set theory database (\texttt{set.mm}}%
\index{Metamath Proof Explorer}), aka the Metamath Proof Explorer,
which uses the most common axioms for mathematical foundations
(specifically classical logic combined with Zermelo--Fraenkel
set theory\index{Zermelo--Fraenkel set theory} with the Axiom of Choice).
But other Metamath databases are available:

\begin{itemize}
\item The database
  \texttt{iset.mm}\index{intuitionistic logic database (\texttt{iset.mm})},
  aka the
  Intuitionistic Logic Explorer\index{Intuitionistic Logic Explorer},
  uses intuitionistic logic (a constructivist point of view)
  instead of classical logic.
\item The database
  \texttt{nf.mm}\index{New Foundations database (\texttt{nf.mm})},
  aka the
  New Foundations Explorer\index{New Foundations Explorer},
  constructs mathematics from scratch,
  starting from Quine's New Foundations (NF) set theory axioms.
\item The database
  \texttt{hol.mm}\index{Higher-order Logic database (\texttt{hol.mm})},
  aka the
  Higher-Order Logic (HOL) Explorer\index{Higher-Order Logic (HOL) Explorer},
  starts with HOL (also called simple type theory) and derives
  equivalents to ZFC axioms, connecting the two approaches.
\end{itemize}

Since the days of David Hilbert,\index{Hilbert, David} mathematicians have
been concerned with the fact that the metalanguage\index{metalanguage} used to
describe mathematics may be stronger than the mathematics being described.
Metamath\index{Metamath}'s underlying finitary\index{finitary proof},
constructive nature provides a good philosophical basis for studying even the
weakest logics.\index{weak logic}

The usual treatment of many non-standard formal systems\index{formal
system} uses model theory\index{model theory} or proof theory\index{proof
theory} to describe these systems; these theories, in turn, are based on
standard set theory.  In other words, a non-standard formal system is defined
as a set with certain properties, and standard set theory is used to derive
additional properties of this set.  The standard set theory database provided
with Metamath can be used for this purpose, and when used this way
the development of a special
axiom system for the non-standard formal system becomes unnecessary.  The
model- or proof-theoretic approach often allows you to prove much deeper
results with less effort.

Metamath supports both approaches.  You can define the non-standard
formal system directly, or define the non-standard formal system as
a set with certain properties, whichever you find most helpful.

%\section{Additional Remarks}

\subsection{Metamath and Its Philosophy}

Closely related to Metamath\index{Metamath} is a philosophy or way of looking
at mathematics. This philosophy is related to the formalist
philosophy\index{formalism} of Hilbert\index{Hilbert, David} and his followers
\cite[pp.~1203--1208]{Kline}\index{Kline, Morris}
\cite[p.~6]{Behnke}\index{Behnke, H.}. In this philosophy, mathematics is
viewed as nothing more than a set of rules that manipulate symbols, together
with the consequences of those rules.  While the mathematics being described
may be complex, the rules used to describe it (the
``metamathematics''\index{metamathematics}) should be as simple as possible.
In particular, proofs should be restricted to dealing with concrete objects
(the symbols we write on paper rather than the abstract concepts they
represent) in a constructive manner; these are called ``finitary''
proofs\index{finitary proof} \cite[pp.~2--3]{Shoenfield}\index{Shoenfield,
Joseph R.}.

Whether or not you find Metamath interesting or useful will in part depend on
the appeal you find in its philosophy, and this appeal will probably depend on
your particular goals with respect to mathematics.  For example, if you are a
pure mathematician at the forefront of discovering new mathematical knowledge,
you will probably find that the rigid formality of Metamath stifles your
creativity.  On the other hand, we would argue that once this knowledge is
discovered, there are advantages to documenting it in a standard format that
will make it accessible to others.  Sixty years from now, your field may be
dormant, and as Davis and Hersh put it, your ``writings would become less
translatable than those of the Maya'' \cite[p.~37]{Davis}\index{Davis, Phillip
J.}.


\subsection{A History of the Approach Behind Metamath}

Probably the one work that has had the most motivating influence on
Metamath\index{Metamath} is Whitehead and Russell's monumental {\em Principia
Mathematica} \cite{PM}\index{Whitehead, Alfred North}\index{Russell,
Bertrand}\index{principia mathematica@{\em Principia Mathematica}}, whose aim
was to deduce all of mathematics from a small number of primitive ideas, in a
very explicit way that in principle anyone could understand and follow.  While
this work was tremendously influential in its time, from a modern perspective
it suffers from several drawbacks.  Both its notation and its underlying
axioms are now considered dated and are no longer used.  From our point of
view, its development is not really as accessible as we would like to see; for
practical reasons, proofs become more and more sketchy as its mathematics
progresses, and working them out in fine detail requires a degree of
mathematical skill and patience that many people don't have.  There are
numerous small errors, which is understandable given the tedious, technical
nature of the proofs and the lack of a computer to verify the details.
However, even today {\em Principia Mathematica} stands out as the work closest
in spirit to Metamath.  It remains a mind-boggling work, and one can't help
but be amazed at seeing ``$1+1=2$'' finally appear on page 83 of Volume II
(Theorem *110.643).

The origin of the proof notation used by Metamath dates back to the 1950's,
when the logician C.~A.~Meredith expressed his proofs in a compact notation
called ``condensed detachment''\index{condensed detachment}
\cite{Hindley}\index{Hindley, J. Roger} \cite{Kalman}\index{Kalman, J. A.}
\cite{Meredith}\index{Meredith, C. A.} \cite{Peterson}\index{Peterson, Jeremy
George}.  This notation allows proofs to be communicated unambiguously by
merely referencing the axiom\index{axiom}, rule\index{rule}, or
theorem\index{theorem} used at each step, without explicitly indicating the
substitutions\index{substitution!variable}\index{variable substitution} that
have to be made to the variables in that axiom, rule, or theorem.  Ordinarily,
condensed detachment is more or less limited to propositional
calculus\index{propositional calculus}.  The concept has been extended to
first-order logic\index{first-order logic} in \cite{Megill}\index{Megill,
Norman}, making it is easy to write a small computer program to verify proofs
of simple first-order logic theorems.\index{condensed detachment!and
first-order logic}

A key concept behind the notation of condensed detachment is called
``unification,''\index{unification} which is an algorithm for determining what
substitutions\index{substitution!variable}\index{variable substitution} to
variables have to be made to make two expressions match each other.
Unification was first precisely defined by the logician J.~A.~Robinson, who
used it in the development of a powerful
theorem-proving technique called the ``resolution principle''
\cite{Robinson}\index{Robinson's resolution principle}. Metamath does not make
use of the resolution principle, which is intended for systems of first-order
logic.\index{first-order logic}  Metamath's use is not restricted to
first-order logic, and as we have mentioned it does not automatically discover
proofs.  However, unification is a key idea behind Metamath's proof
notation, and Metamath makes use of a very simple version of it
(Section~\ref{unify}).

\subsection{Metamath and First-Order Logic}

First-order logic\index{first-order logic} is the supporting structure
for standard mathematics.  On top of it is set theory, which contains
the axioms from which virtually all of mathematics can be derived---a
remarkable fact.\index{category
theory}\index{cardinal, inaccessible}\label{categoryth}\footnote{An exception seems
to be category theory.  There are several schools of thought on whether
category theory is derivable from set theory.  At a minimum, it appears
that an additional axiom is needed that asserts the existence of an
``inaccessible cardinal'' (a type of infinity so large that standard set
theory can't prove or deny that it exists).
%
%%%% (I took this out that was in previous editions:)
% But it is also argued that not just one but a ``proper class'' of them
% is needed, and the existence of proper classes is impossible in standard
% set theory.  (A proper class is a collection of sets so huge that no set
% can contain it as an element.  Proper classes can lead to
% inconsistencies such as ``Russell's paradox.''  The axioms of standard
% set theory are devised so as to deny the existence of proper classes.)
%
For more information, see
\cite[pp.~328--331]{Herrlich}\index{Herrlich, Horst} and
\cite{Blass}\index{Blass, Andrea}.}

One of the things that makes Metamath\index{Metamath} more practical for
first-order theories is a set of axioms for first-order logic designed
specifically with Metamath's approach in mind.  These are included in
the database \texttt{set.mm}\index{set theory database (\texttt{set.mm})}.
See Chapter~\ref{fol} for a detailed
description; the axioms are shown in Section~\ref{metaaxioms}.  While
logically equivalent to standard axiom systems, our axiom system breaks
up the standard axioms into smaller pieces such that from them, you can
directly derive what in other systems can only be derived as higher-level
``metatheorems.''\index{metatheorem}  In other words, it is more powerful than
the standard axioms from a metalogical point of view.  A rigorous
justification for this system and its ``metalogical
completeness''\index{metalogical completeness} is found in
\cite{Megill}\index{Megill, Norman}.  The system is closely related to a
system developed by Monk\index{Monk, J. Donald} and Tarski\index{Tarski,
Alfred} in 1965 \cite{Monks}.

For example, the formula $\exists x \, x = y $ (given $y$, there exists some
$x$ equal to it) is a theorem of logic,\footnote{Specifically, it is a theorem
of those systems of logic that assume non-empty domains.  It is not a theorem
of more general systems that include the empty domain\index{empty domain}, in
which nothing exists, period!  Such systems are called ``free
logics.''\index{free logic} For a discussion of these systems, see
\cite{Leblanc}\index{Leblanc, Hugues}.  Since our use for logic is as a basis
for set theory, which has a non-empty domain, it is more convenient (and more
traditional) to use a less general system.  An interesting curiosity is that,
using a free logic as a basis for Zermelo--Fraenkel set
theory\index{Zermelo--Fraenkel set theory} (with the redundant Axiom of the
Null Set omitted),\index{Axiom of the Null Set} we cannot even prove the
existence of a single set without assuming the axiom of infinity!\index{Axiom
of Infinity}} whether or not $x$ and $y$ are distinct variables\index{distinct
variables}.  In many systems of logic, we would have to prove two theorems to
arrive at this result.  First we would prove ``$\exists x \, x = x $,'' then
we would separately prove ``$\exists x \, x = y $, where $x$ and $y$ are
distinct variables.''  We would then combine these two special cases ``outside
of the system'' (i.e.\ in our heads) to be able to claim, ``$\exists x \, x =
y $, regardless of whether $x$ and $y$ are distinct.''  In other words, the
combination of the two special cases is a
metatheorem.  In the system of logic
used in Metamath's set theory\index{set theory database (\texttt{set.mm})}
database, the axioms of logic are broken down into small pieces that allow
them to be reassembled in such a way that theorems such as these can be proved
directly.

Breaking down the axioms in this way makes them look peculiar and not very
intuitive at first, but rest assured that they are correct and complete.  Their
correctness is ensured because they are theorem schemes of standard first-order
logic (which you can easily verify if you are a logician).  Their completeness
follows from the fact that we explicitly derive the standard axioms of
first-order logic as theorems.  Deriving the standard axioms is somewhat
tricky, but once we're there, we have at our disposal a system that is less
awkward to work with in formal proofs\index{formal proof}.  In technical terms
that logicians understand, we eliminate the cumbersome concepts of ``free
variable,''\index{free variable} ``bound variable,''\index{bound variable} and
``proper substitution''\index{proper substitution}\index{substitution!proper}
as primitive notions.  These concepts are present in our system but are
defined in terms of concepts expressed by the axioms and can be eliminated in
principle.  In standard systems, these concepts are really like additional,
implicit axioms\index{implicit axiom} that are somewhat complex and cannot be
eliminated.

The traditional approach to logic, wherein free variables and proper
substitution is defined, is also possible to do directly in the Metamath
language.  However, the notation tends to become awkward, and there are
disadvantages:  for example, extending the definition of a wff with a
definition is awkward, because the free variable and proper substitution
concepts have to have their definitions also extended.  Our choice of
axioms for \texttt{set.mm} is to a certain extent a matter of style, in
an attempt to achieve overall simplicity, but you should also be aware
that the traditional approach is possible as well if you should choose
it.

\chapter{Using the Metamath Program}
\label{using}

\section{Installation}

The way that you install Metamath\index{Metamath!installation} on your
computer system will vary for different computers.  Current
instructions are provided with the Metamath program download at
\url{http://metamath.org}.  In general, the installation is simple.
There is one file containing the Metamath program itself.  This file is
usually called \texttt{metamath} or \texttt{metamath.}{\em xxx} where
{\em xxx} is the convention (such as \texttt{exe}) for an executable
program on your operating system.  There are several additional files
containing samples of the Metamath language, all ending with
\texttt{.mm}.  The file \texttt{set.mm}\index{set theory database
(\texttt{set.mm})} contains logic and set theory and can be used as a
starting point for other areas of mathematics.

You will also need a text editor\index{text editor} capable of editing plain
{\sc ascii}\footnote{American Standard Code for Information Interchange.} text
in order to prepare your input files.\index{ascii@{\sc ascii}}  Most computers
have this capability built in.  Note that plain text is not necessarily the
default for some word processing programs\index{word processor}, especially if
they can handle different fonts; for example, with Microsoft Word\index{Word
(Microsoft)}, you must save the file in the format ``Text Only With Line
Breaks'' to get a plain text\index{plain text} file.\footnote{It is
recommended that all lines in a Metamath source file be 79 characters or less
in length for compatibility among different computer terminals.  When creating
a source file on an editor such as Word, select a monospaced
font\index{monospaced font} such as Courier\index{Courier font} or
Monaco\index{Monaco font} to make this easier to achieve.  Better yet,
just use a plain text editor such as Notepad.}

On some computer systems, Metamath does not have the capability to print
its output directly; instead, you send its output to a file (using the
\texttt{open} commands described later).  The way you print this output
file depends on your computer.\index{printers} Some computers have a
print command, whereas with others, you may have to read the file into
an editor and print it from there.

If you want to print your Metamath source files with typeset formulas
containing standard mathematical symbols, you will need the \LaTeX\
typesetting program\index{latex@{\LaTeX}}, which is widely and freely
available for most operating systems.  It runs natively on Unix and
Linux, and can be installed on Windows as part of the free Cygwin
package (\url{http://cygwin.com}).

You can also produce {\sc html}\footnote{HyperText Markup Language.}
web pages.  The {\tt help html} command in the Metamath program will
assist you with this feature.

\section{Your First Formal System}\label{start}
\subsection{From Nothing to Zero}\label{startf}

To give you a feel for what the Metamath\index{Metamath} language looks like,
we will take a look at a very simple example from formal number
theory\index{number theory}.  This example is taken from
Mendelson\index{Mendelson, Elliot} \cite[p. 123]{Mendelson}.\footnote{To keep
the example simple, we have changed the formalism slightly, and what we call
axioms\index{axiom} are strictly speaking theorems\index{theorem} in
\cite{Mendelson}.}  We will look at a small subset of this theory, namely that
part needed for the first number theory theorem proved in \cite{Mendelson}.

First we will look at a standard formal proof\index{formal proof} for the
example we have picked, then we will look at the Metamath version.  If you
have never been exposed to formal proofs, the notation may seem to be such
overkill to express such simple notions that you may wonder if you are missing
something.  You aren't.  The concepts involved are in fact very simple, and a
detailed breakdown in this fashion is necessary to express the proof in a way
that can be verified mechanically.  And as you will see, Metamath breaks the
proof down into even finer pieces so that the mechanical verification process
can be about as simple as possible.

Before we can introduce the axioms\index{axiom} of the theory, we must define
the syntax rules for forming legal expressions\index{syntax rules}
(combinations of symbols) with which those axioms can be used. The number 0 is
a {\bf term}\index{term}; and if $ t$ and $r$ are terms, so is $(t+r)$. Here,
$ t$ and $r$ are ``metavariables''\index{metavariable} ranging over terms; they
themselves do not appear as symbols in an actual term.  Some examples of
actual terms are $(0 + 0)$ and $((0+0)+0)$.  (Note that our theory describes
only the number zero and sums of zeroes.  Of course, not much can be done with
such a trivial theory, but remember that we have picked a very small subset of
complete number theory for our example.  The important thing for you to focus
on is our definitions that describe how symbols are combined to form valid
expressions, and not on the content or meaning of those expressions.) If $ t$
and $r$ are terms, an expression of the form $ t=r$ is a {\bf wff}
(well-formed formula)\index{well-formed formula (wff)}; and if $P$ and $Q$ are
wffs, so is $(P\rightarrow Q)$ (which means ``$P$ implies
$Q$''\index{implication ($\rightarrow$)} or ``if $P$ then $Q$'').
Here $P$ and $Q$ are metavariables ranging over wffs.  Examples of actual
wffs are $0=0$, $(0+0)=0$, $(0=0 \rightarrow (0+0)=0)$, and $(0=0\rightarrow
(0=0\rightarrow 0=(0+0)))$.  (Our notation makes use of more parentheses than
are customary, but the elimination of ambiguity this way simplifies our
example by avoiding the need to define operator precedence\index{operator
precedence}.)

The {\bf axioms}\index{axiom} of our theory are all wffs of the following
form, where $ t$, $r$, and $s$ are any terms:

%Latex p. 92
\renewcommand{\theequation}{A\arabic{equation}}

\begin{equation}
(t=r\rightarrow (t=s\rightarrow r=s))
\end{equation}
\begin{equation}
(t+0)=t
\end{equation}

Note that there are an infinite number of axioms since there are an infinite
number of possible terms.  A1 and A2 are properly called ``axiom
schemes,''\index{axiom scheme} but we will refer to them as ``axioms'' for
brevity.

An axiom is a {\bf theorem}; and if $P$ and $(P\rightarrow Q)$ are theorems
(where $P$ and $Q$ are wffs), then $Q$ is also a theorem.\index{theorem}  The
second part of this definition is called the modus ponens (MP) rule of
inference\index{inference rule}\index{modus ponens}.  It allows us to obtain
new theorems from old ones.

The {\bf proof}\index{proof} of a theorem is a sequence of one or more
theorems, each of which is either an axiom or the result of modus ponens
applied to two previous theorems in the sequence, and the last of which is the
theorem being proved.

The theorem we will prove for our example is very simple:  $ t=t$.  The proof of
our theorem follows.  Study it carefully until you feel sure you
understand it.\label{zeroproof}

% Use tabu so that lines will wrap automatically as needed.
\begin{tabu} { l X X }
1. & $(t+0)=t$ & (by axiom A2) \\
2. & $(t+0)=t$ & (by axiom A2) \\
3. & $((t+0)=t \rightarrow ((t+0)=t\rightarrow t=t))$ & (by axiom A1) \\
4. & $((t+0)=t\rightarrow t=t)$ & (by MP applied to steps 2 and 3) \\
5. & $t=t$ & (by MP applied to steps 1 and 4) \\
\end{tabu}

(You may wonder why step 1 is repeated twice.  This is not necessary in the
formal language we have defined, but in Metamath's ``reverse Polish
notation''\index{reverse Polish notation (RPN)} for proofs, a previous step
can be referred to only once.  The repetition of step~1 here will enable you
to see more clearly the correspondence of this proof with the
Metamath\index{Metamath} version on p.~\pageref{demoproof}.)

Our theorem is more properly called a ``theorem scheme,''\index{theorem
scheme} for it represents an infinite number of theorems, one for each
possible term $ t$.  Two examples of actual theorems would be $0=0$ and
$(0+0)=(0+0)$.  Rarely do we prove actual theorems, since by proving schemes
we can prove an infinite number of theorems in one fell swoop.  Similarly, our
proof should really be called a ``proof scheme.''\index{proof scheme}  To
obtain an actual proof, pick an actual term to use in place of $ t$, and
substitute it for $ t$ throughout the proof.

Let's discuss what we have done here.  The axioms\index{axiom} of our theory,
A1 and A2, are trivial and obvious.  Everyone knows that adding zero to
something doesn't change it, and also that if two things are equal to a third,
then they are equal to each other. In fact, stating the trivial and obvious is
a goal to strive for in any axiomatic system.  From trivial and obvious truths
that everyone agrees upon, we can prove results that are not so obvious yet
have absolute faith in them.  If we trust the axioms and the rules, we must,
by definition, trust the consequences of those axioms and rules, if logic is
to mean anything at all.

Our rule of inference\index{rule}, modus ponens\index{modus ponens}, is also
pretty obvious once you understand what it means.  If we prove a fact $P$, and
we also prove that $P$ implies $Q$, then $Q$ necessarily follows as a new
fact.  The rule provides us with a means for obtaining new facts (i.e.\
theorems\index{theorem}) from old ones.

The theorem that we have proved, $ t=t$, is so fundamental that you may wonder
why it isn't one of the axioms\index{axiom}.  In some axiom systems of
arithmetic, it {\em is} an axiom.  The choice of axioms in a theory is to some
extent arbitrary and even an art form, constrained only by the requirement
that any two equivalent axiom systems be able to derive each other as
theorems.  We could imagine that the inventor of our axiom system originally
included $ t=t$ as an axiom, then discovered that it could be derived as a
theorem from the other axioms.  Because of this, it was not necessary to
keep it as an axiom.  By eliminating it, the final set of axioms became
that much simpler.

Unless you have worked with formal proofs\index{formal proof} before, it
probably wasn't apparent to you that $ t=t$ could be derived from our two
axioms until you saw the proof. While you certainly believe that $ t=t$ is
true, you might not be able to convince an imaginary skeptic who believes only
in our two axioms until you produce the proof.  Formal proofs such as this are
hard to come up with when you first start working with them, but after you get
used to them they can become interesting and fun.  Once you understand the
idea behind formal proofs you will have grasped the fundamental principle that
underlies all of mathematics.  As the mathematics becomes more sophisticated,
its proofs become more challenging, but ultimately they all can be broken down
into individual steps as simple as the ones in our proof above.

Mendelson's\index{Mendelson, Elliot} book, from which our example was taken,
contains a number of detailed formal proofs such as these, and you may be
interested in looking it up.  The book is intended for mathematicians,
however, and most of it is rather advanced.  Popular literature describing
formal proofs\index{formal proof} include \cite[p.~296]{Rucker}\index{Rucker,
Rudy} and \cite[pp.~204--230]{Hofstadter}\index{Hofstadter, Douglas R.}.

\subsection{Converting It to Metamath}\label{convert}

Formal proofs\index{formal proof} such as the one in our example break down
logical reasoning into small, precise steps that leave little doubt that the
results follow from the axioms\index{axiom}.  You might think that this would
be the finest breakdown we can achieve in mathematics.  However, there is more
to the proof than meets the eye. Although our axioms were rather simple, a lot
of verbiage was needed before we could even state them:  we needed to define
``term,'' ``wff,'' and so on.  In addition, there are a number of implied
rules that we haven't even mentioned. For example, how do we know that step 3
of our proof follows from axiom A1? There is some hidden reasoning involved in
determining this.  Axiom A1 has two occurrences of the letter $ t$.  One of
the implied rules states that whatever we substitute for $ t$ must be a legal
term\index{term}.\footnote{Some authors make this implied rule explicit by
stating, ``only expressions of the above form are terms,'' after defining
``term.''}  The expression $ t+0$ is pretty obviously a legal term whenever $
t$ is, but suppose we wanted to substitute a huge term with thousands of
symbols?  Certainly a lot of work would be involved in determining that it
really is a term, but in ordinary formal proofs all of this work would be
considered a single ``step.''

To express our axiom system in the Metamath\index{Metamath} language, we must
describe this auxiliary information in addition to the axioms themselves.
Metamath does not know what a ``term'' or a ``wff''\index{well-formed formula
(wff)} is.  In Metamath, the specification of the ways in which we can combine
symbols to obtain terms and wffs are like little axioms in themselves.  These
auxiliary axioms are expressed in the same notation as the ``real''
axioms\index{axiom}, and Metamath does not distinguish between the two.  The
distinction is made by you, i.e.\ by the way in which you interpret the
notation you have chosen to express these two kinds of axioms.

The Metamath language breaks down mathematical proofs into tiny pieces, much
more so than in ordinary formal proofs\index{formal proof}.  If a single
step\index{proof step} involves the
substitution\index{substitution!variable}\index{variable substitution} of a
complex term for one of its variables, Metamath must see this single step
broken down into many small steps.  This fine-grained breakdown is what gives
Metamath generality and flexibility as it lets it not be limited to any
particular mathematical notation.

Metamath's proof notation is not, in itself, intended to be read by humans but
rather is in a compact format intended for a machine.  The Metamath program
will convert this notation to a form you can understand, using the \texttt{show
proof}\index{\texttt{show proof} command} command.  You can tell the program what
level of detail of the proof you want to look at.  You may want to look at
just the logical inference steps that correspond
to ordinary formal proof steps,
or you may want to see the fine-grained steps that prove that an expression is
a term.

Here, without further ado, is our example converted to the
Metamath\index{Metamath} language:\index{metavariable}\label{demo0}

\begin{verbatim}
$( Declare the constant symbols we will use $)
    $c 0 + = -> ( ) term wff |- $.
$( Declare the metavariables we will use $)
    $v t r s P Q $.
$( Specify properties of the metavariables $)
    tt $f term t $.
    tr $f term r $.
    ts $f term s $.
    wp $f wff P $.
    wq $f wff Q $.
$( Define "term" and "wff" $)
    tze $a term 0 $.
    tpl $a term ( t + r ) $.
    weq $a wff t = r $.
    wim $a wff ( P -> Q ) $.
$( State the axioms $)
    a1 $a |- ( t = r -> ( t = s -> r = s ) ) $.
    a2 $a |- ( t + 0 ) = t $.
$( Define the modus ponens inference rule $)
    ${
       min $e |- P $.
       maj $e |- ( P -> Q ) $.
       mp  $a |- Q $.
    $}
$( Prove a theorem $)
    th1 $p |- t = t $=
  $( Here is its proof: $)
       tt tze tpl tt weq tt tt weq tt a2 tt tze tpl
       tt weq tt tze tpl tt weq tt tt weq wim tt a2
       tt tze tpl tt tt a1 mp mp
     $.
\end{verbatim}\index{metavariable}

A ``database''\index{database} is a set of one or more {\sc ascii} source
files.  Here's a brief description of this Metamath\index{Metamath} database
(which consists of this single source file), so that you can understand in
general terms what is going on.  To understand the source file in detail, you
should read Chapter~\ref{languagespec}.

The database is a sequence of ``tokens,''\index{token} which are normally
separated by spaces or line breaks.  The only tokens that are built into
the Metamath language are those beginning with \texttt{\$}.  These tokens
are called ``keywords.''\index{keyword}  All other tokens are
user-defined, and their names are arbitrary.

As you might have guessed, the Metamath token \texttt{\$(}\index{\texttt{\$(} and
\texttt{\$)} auxiliary keywords} starts a comment and \texttt{\$)} ends a comment.

The Metamath tokens \texttt{\$c}\index{\texttt{\$c} statement},
\texttt{\$v}\index{\texttt{\$v} statement},
\texttt{\$e}\index{\texttt{\$e} statement},
\texttt{\$f}\index{\texttt{\$f} statement},
\texttt{\$a}\index{\texttt{\$a} statement}, and
\texttt{\$p}\index{\texttt{\$p} statement} specify ``statements'' that
end with \texttt{\$.}\,.\index{\texttt{\$.}\ keyword}

The Metamath tokens \texttt{\$c} and \texttt{\$v} each declare\index{constant
declaration}\index{variable declaration} a list of user-defined tokens, called
``math symbols,''\index{math symbol} that the database will reference later
on.  All of the math symbols they define you have seen earlier except the
turnstile symbol \texttt{|-} ($\vdash$)\index{turnstile ({$\,\vdash$})}, which is
commonly used by logicians to mean ``a proof exists for.''  For us
the turnstile is just a
convenient symbol that distinguishes expressions that are axioms\index{axiom}
or theorems\index{theorem} from expressions that are terms or wffs.

The \texttt{\$c} statement declares ``constants''\index{constant} and
the \texttt{\$v} statement declares
``variables''\index{variable}\index{constant declaration}\index{variable
declaration} (or more precisely, metavariables\index{metavariable}).  A
variable may be substituted\index{substitution!variable}\index{variable
substitution} with sequences of math symbols whereas a constant may not
be substituted with anything.

It may seem redundant to require both \texttt{\$c}\index{\texttt{\$c} statement} and
\texttt{\$v}\index{\texttt{\$v} statement} statements (since any math
symbol\index{math symbol} not specified with a \texttt{\$c} statement could be
presumed to be a variable), but this provides for better error checking and
also allows math symbols to be redeclared\index{redeclaration of symbols}
(Section~\ref{scoping}).

The token \texttt{\$f}\index{\texttt{\$f} statement} specifies a
statement called a ``variable-type hypothesis'' (also called a
``floating hypothesis'') and \texttt{\$e}\index{\texttt{\$e} statement}
specifies a ``logical hypothesis'' (also called an ``essential
hypothesis'').\index{hypothesis}\index{variable-type
hypothesis}\index{logical hypothesis}\index{floating
hypothesis}\index{essential hypothesis} The token
\texttt{\$a}\index{\texttt{\$a} statement} specifies an ``axiomatic
assertion,''\index{axiomatic assertion} and
\texttt{\$p}\index{\texttt{\$p} statement} specifies a ``provable
assertion.''\index{provable assertion} To the left of each occurrence of
these four tokens is a ``label''\index{label} that identifies the
hypothesis or assertion for later reference.  For example, the label of
the first axiomatic assertion is \texttt{tze}.  A \texttt{\$f} statement
must contain exactly two math symbols, a constant followed by a
variable.  The \texttt{\$e}, \texttt{\$a}, and \texttt{\$p} statements
each start with a constant followed by, in general, an arbitrary
sequence of math symbols.

Associated with each assertion\index{assertion} is a set of hypotheses
that must be satisfied in order for the assertion to be used in a proof.
These are called the ``mandatory hypotheses''\index{mandatory
hypothesis} of the assertion.  Among those hypotheses whose ``scope''
(described below) includes the assertion, \texttt{\$e} hypotheses are
always mandatory and \texttt{\$f}\index{\texttt{\$f} statement}
hypotheses are mandatory when they share their variable with the
assertion or its \texttt{\$e} hypotheses.  The exact rules for
determining which hypotheses are mandatory are described in detail in
Sections~\ref{frames} and \ref{scoping}.  For example, the mandatory
hypotheses of assertion \texttt{tpl} are \texttt{tt} and \texttt{tr},
whereas assertion \texttt{tze} has no mandatory hypotheses because it
contains no variables and has no \texttt{\$e}\index{\texttt{\$e}
statement} hypothesis.  Metamath's \texttt{show statement}
command\index{\texttt{show statement} command}, described in the next
section, will show you a statement's mandatory hypotheses.

Sometimes we need to make a hypothesis relevant to only certain
assertions.  The set of statements to which a hypothesis is relevant is
called its ``scope.''  The Metamath brackets,
\texttt{\$\char`\{}\index{\texttt{\$\char`\{} and \texttt{\$\char`\}}
keywords} and \texttt{\$\char`\}}, define a ``block''\index{block} that
delimits the scope of any hypothesis contained between them.  The
assertion \texttt{mp} has mandatory hypotheses \texttt{wp}, \texttt{wq},
\texttt{min}, and \texttt{maj}.  The only mandatory hypothesis of
\texttt{th1}, on the other hand, is \texttt{tt}, since \texttt{th1}
occurs outside of the block containing \texttt{min} and \texttt{maj}.

Note that \texttt{\$\char`\{} and \texttt{\$\char`\}} do not affect the
scope of assertions (\texttt{\$a} and \texttt{\$p}).  Assertions are always
available to be referenced by any later proof in the source file.

Each provable assertion (\texttt{\$p}\index{\texttt{\$p} statement}
statement) has two parts.  The first part is the
assertion\index{assertion} itself, which is a sequence of math
symbol\index{math symbol} tokens placed between the \texttt{\$p} token
and a \texttt{\$=}\index{\texttt{\$=} keyword} token.  The second part
is a ``proof,'' which is a list of label tokens placed between the
\texttt{\$=} token and the \texttt{\$.}\index{\texttt{\$.}\ keyword}\
token that ends the statement.\footnote{If you've looked at the
\texttt{set.mm} database, you may have noticed another notation used for
proofs.  The other notation is called ``compressed.''\index{compressed
proof}\index{proof!compressed} It reduces the amount of space needed to
store a proof in the database and is described in
Appendix~\ref{compressed}.  In the example above, we use
``normal''\index{normal proof}\index{proof!normal} notation.} The proof
acts as a series of instructions to the Metamath program, telling it how
to build up the sequence of math symbols contained in the assertion part of
the \texttt{\$p} statement, making use of the hypotheses of the
\texttt{\$p} statement and previous assertions.  The construction takes
place according to precise rules.  If the list of labels in the proof
causes these rules to be violated, or if the final sequence that results
does not match the assertion, the Metamath program will notify you with
an error message.

If you are familiar with reverse Polish notation (RPN), which is sometimes used
on pocket calculators, here in a nutshell is how a proof works.  Each
hypothesis label\index{hypothesis label} in the proof is pushed\index{push}
onto the RPN stack\index{stack}\index{RPN stack} as it is encountered. Each
assertion label\index{assertion label} pops\index{pop} off the stack as many
entries as the referenced assertion has mandatory hypotheses.  Variable
substitutions\index{substitution!variable}\index{variable substitution} are
computed which, when made to the referenced assertion's mandatory hypotheses,
cause these hypotheses to match the stack entries. These same substitutions
are then made to the variables in the referenced assertion itself, which is
then pushed onto the stack.  At the end of the proof, there should be one
stack entry, namely the assertion being proved.  This process is explained in
detail in Section~\ref{proof}.

Metamath's proof notation is not very readable for humans, but it allows the
proof to be stored compactly in a file.  The Metamath\index{Metamath} program
has proof display features that let you see what's going on in a more
readable way, as you will see in the next section.

The rules used in verifying a proof are not based on any built-in syntax of the
symbol sequence in an assertion\index{assertion} nor on any built-in meanings
attached to specific symbol names.  They are based strictly on symbol
matching:  constants\index{constant} must match themselves, and
variables\index{variable} may be replaced with anything that allows a match to
occur.  For example, instead of \texttt{term}, \texttt{0}, and \verb$|-$ we could
have just as well used \texttt{yellow}, \texttt{zero}, and \texttt{provable}, as long
as we did so consistently throughout the database.  Also, we could have used
\texttt{is provable} (two tokens) instead of \verb$|-$ (one token) throughout the
database.  In each of these cases, the proof would be exactly the same.  The
independence of proofs and notation means that you have a lot of flexibility to
change the notation you use without having to change any proofs.

\section{A Trial Run}\label{trialrun}

Now you are ready to try out the Metamath\index{Metamath} program.

On all computer systems, Metamath has a standard ``command line
interface'' (CLI)\index{command line interface (CLI)} that allows you to
interact with it.  You supply commands to the CLI by typing them on the
keyboard and pressing your keyboard's {\em return} key after each line
you enter.  The CLI is designed to be easy to use and has built-in help
features.

The first thing you should do is to use a text editor to create a file
called \texttt{demo0.mm} and type into it the Metamath source shown on
p.~\pageref{demo0}.  Actually, this file is included with your Metamath
software package, so check that first.  If you type it in, make sure
that you save it in the form of ``plain {\sc ascii} text with line
breaks.''  Most word processors will have this feature.

Next you must run the Metamath program.  Depending on your computer
system and how Metamath is installed, this could range from clicking the
mouse on the Metamath icon to typing \texttt{run metamath} to typing
simply \texttt{metamath}.  (Metamath's {\tt help invoke} command describes
alternate ways of invoking the Metamath program.)

When you first enter Metamath\index{Metamath}, it will be at the CLI, waiting
for your input. You will see something like the following on your screen:
\begin{verbatim}
Metamath - Version 0.177 27-Apr-2019
Type HELP for help, EXIT to exit.
MM>
\end{verbatim}
The \texttt{MM>} prompt means that Metamath is waiting for a command.
Command keywords\index{command keyword} are not case sensitive;
we will use lower-case commands in our examples.
The version number and its release date will probably be different on your
system from the one we show above.

The first thing that you need to do is to read in your
database:\index{\texttt{read} command}\footnote{If a directory path is
needed on Unix,\index{Unix file names}\index{file names!Unix} you should
enclose the path/file name in quotes to prevent Metamath from thinking
that the \texttt{/} in the path name is a command qualifier, e.g.,
\texttt{read \char`\"db/set.mm\char`\"}.  Quotes are optional when there
is no ambiguity.}
\begin{verbatim}
MM> read demo0.mm
\end{verbatim}
Remember to press the {\em return} key after entering this command.  If
you omit the file name, Metamath will prompt you for one.   The syntax for
specifying a Macintosh file name path is given in a footnote on
p.~\pageref{includef}.\index{Macintosh file names}\index{file
names!Macintosh}

If there are any syntax errors in the database, Metamath will let you know
when it reads in the file.  The one thing that Metamath does not check when
reading in a database is that all proofs are correct, because this would
slow it down too much.  It is a good idea to periodically verify the proofs in
a database you are making changes to.  To do this, use the following command
(and do it for your \texttt{demo0.mm} file now).  Note that the \texttt{*} is a
``wild card'' meaning all proofs in the file.\index{\texttt{verify proof} command}
\begin{verbatim}
MM> verify proof *
\end{verbatim}
Metamath will report any proofs that are incorrect.

It is often useful to save the information that the Metamath program displays
on the screen. You can save everything that happens on the screen by opening a
log file. You may want to do this before you read in a database so that you
can examine any errors later on.  To open a log file, type
\begin{verbatim}
MM> open log abc.log
\end{verbatim}
This will open a file called \texttt{abc.log}, and everything that appears on the
screen from this point on will be stored in this file.  The name of the log file
is arbitrary. To close the log file, type
\begin{verbatim}
MM> close log
\end{verbatim}

Several commands let you examine what's inside your database.
Section~\ref{exploring} has an overview of some useful ones.  The
\texttt{show labels} command lets you see what statement
labels\index{label} exist.  A \texttt{*} matches any combination of
characters, and \texttt{t*} refers to all labels starting with the
letter \texttt{t}.\index{\texttt{show labels} command} The \texttt{/all}
is a ``command qualifier''\index{command qualifier} that tells Metamath
to include labels of hypotheses.  (To see the syntax explained, type
\texttt{help show labels}.)  Type
\begin{verbatim}
MM> show labels t* /all
\end{verbatim}
Metamath will respond with
\begin{verbatim}
The statement number, label, and type are shown.
3 tt $f       4 tr $f       5 ts $f       8 tze $a
9 tpl $a      19 th1 $p
\end{verbatim}

You can use the \texttt{show statement} command to get information about a
particular statement.\index{\texttt{show statement} command}
For example, you can get information about the statement with label \texttt{mp}
by typing
\begin{verbatim}
MM> show statement mp /full
\end{verbatim}
Metamath will respond with
\begin{verbatim}
Statement 17 is located on line 43 of the file
"demo0.mm".
"Define the modus ponens inference rule"
17 mp $a |- Q $.
Its mandatory hypotheses in RPN order are:
  wp $f wff P $.
  wq $f wff Q $.
  min $e |- P $.
  maj $e |- ( P -> Q ) $.
The statement and its hypotheses require the
      variables:  Q P
The variables it contains are:  Q P
\end{verbatim}
The mandatory hypotheses\index{mandatory hypothesis} and their
order\index{RPN order} are
useful to know when you are trying to understand or debug a proof.

Now you are ready to look at what's really inside our proof.  First, here is
how to look at every step in the proof---not just the ones corresponding to an
ordinary formal proof\index{formal proof}, but also the ones that build up the
formulas that appear in each ordinary formal proof step.\index{\texttt{show
proof} command}
\begin{verbatim}
MM> show proof th1 /lemmon /all
\end{verbatim}

This will display the proof on the screen in the following format:
\begin{verbatim}
 1 tt            $f term t
 2 tze           $a term 0
 3 1,2 tpl       $a term ( t + 0 )
 4 tt            $f term t
 5 3,4 weq       $a wff ( t + 0 ) = t
 6 tt            $f term t
 7 tt            $f term t
 8 6,7 weq       $a wff t = t
 9 tt            $f term t
10 9 a2          $a |- ( t + 0 ) = t
11 tt            $f term t
12 tze           $a term 0
13 11,12 tpl     $a term ( t + 0 )
14 tt            $f term t
15 13,14 weq     $a wff ( t + 0 ) = t
16 tt            $f term t
17 tze           $a term 0
18 16,17 tpl     $a term ( t + 0 )
19 tt            $f term t
20 18,19 weq     $a wff ( t + 0 ) = t
21 tt            $f term t
22 tt            $f term t
23 21,22 weq     $a wff t = t
24 20,23 wim     $a wff ( ( t + 0 ) = t -> t = t )
25 tt            $f term t
26 25 a2         $a |- ( t + 0 ) = t
27 tt            $f term t
28 tze           $a term 0
29 27,28 tpl     $a term ( t + 0 )
30 tt            $f term t
31 tt            $f term t
32 29,30,31 a1   $a |- ( ( t + 0 ) = t -> ( ( t + 0 )
                                     = t -> t = t ) )
33 15,24,26,32 mp  $a |- ( ( t + 0 ) = t -> t = t )
34 5,8,10,33 mp  $a |- t = t
\end{verbatim}

The \texttt{/lemmon} command qualifier specifies what is known as a Lemmon-style
display\index{Lemmon-style proof}\index{proof!Lemmon-style}.  Omitting the
\texttt{/lemmon} qualifier results in a tree-style proof (see
p.~\pageref{treeproof} for an example) that is somewhat less explicit but
easier to follow once you get used to it.\index{tree-style
proof}\index{proof!tree-style}

The first number on each line is the step
number of the proof.  Any numbers that follow are step numbers assigned to the
hypotheses of the statement referenced by that step.  Next is the label of
the statement referenced by the step.  The statement type of the statement
referenced comes next, followed by the math symbol\index{math symbol} string
constructed by the proof up to that step.

The last step, 34, contains the statement that is being proved.

Looking at a small piece of the proof, notice that steps 3 and 4 have
established that
\texttt{( t + 0 )} and \texttt{t} are \texttt{term}\,s, and step 5 makes use of steps 3 and
4 to establish that \texttt{( t + 0 ) = t} is a \texttt{wff}.  Let Metamath
itself tell us in detail what is happening in step 5.  Note that the
``target hypothesis'' refers to where step 5 is eventually used, i.e., in step
34.
\begin{verbatim}
MM> show proof th1 /detailed_step 5
Proof step 5:  wp=weq $a wff ( t + 0 ) = t
This step assigns source "weq" ($a) to target "wp"
($f).  The source assertion requires the hypotheses
"tt" ($f, step 3) and "tr" ($f, step 4).  The parent
assertion of the target hypothesis is "mp" ($a,
step 34).
The source assertion before substitution was:
    weq $a wff t = r
The following substitutions were made to the source
assertion:
    Variable  Substituted with
     t         ( t + 0 )
     r         t
The target hypothesis before substitution was:
    wp $f wff P
The following substitution was made to the target
hypothesis:
    Variable  Substituted with
     P         ( t + 0 ) = t
\end{verbatim}

The full proof just shown is useful to understand what is going on in detail.
However, most of the time you will just be interested in
the ``essential'' or logical steps of a proof, i.e.\ those steps
that correspond to an
ordinary formal proof\index{formal proof}.  If you type
\begin{verbatim}
MM> show proof th1 /lemmon /renumber
\end{verbatim}
you will see\label{demoproof}
\begin{verbatim}
1 a2             $a |- ( t + 0 ) = t
2 a2             $a |- ( t + 0 ) = t
3 a1             $a |- ( ( t + 0 ) = t -> ( ( t + 0 )
                                     = t -> t = t ) )
4 2,3 mp         $a |- ( ( t + 0 ) = t -> t = t )
5 1,4 mp         $a |- t = t
\end{verbatim}
Compare this to the formal proof on p.~\pageref{zeroproof} and
notice the resemblance.
By default Metamath
does not show \texttt{\$f}\index{\texttt{\$f}
statement} hypotheses and everything branching off of them in the proof tree
when the proof is displayed; this makes the proof look more like an ordinary
mathematical proof, which does not normally incorporate the explicit
construction of expressions.
This is called the ``essential'' view
(at one time you had to add the
\texttt{/essential} qualifier in the \texttt{show proof}
command to get this view, but this is now the default).
You can could use the \texttt{/all} qualifier in the \texttt{show
proof} command to also show the explicit construction of expressions.
The \texttt{/renumber} qualifier means to renumber
the steps to correspond only to what is displayed.\index{\texttt{show proof}
command}

To exit Metamath, type\index{\texttt{exit} command}
\begin{verbatim}
MM> exit
\end{verbatim}

\subsection{Some Hints for Using the Command Line Interface}

We will conclude this quick introduction to Metamath\index{Metamath} with some
helpful hints on how to navigate your way through the commands.
\index{command line interface (CLI)}

When you type commands into Metamath's CLI, you only have to type as many
characters of a command keyword\index{command keyword} as are needed to make
it unambiguous.  If you type too few characters, Metamath will tell you what
the choices are.  In the case of the \texttt{read} command, only the \texttt{r} is
needed to specify it unambiguously, so you could have typed\index{\texttt{read}
command}
\begin{verbatim}
MM> r demo0.mm
\end{verbatim}
instead of
\begin{verbatim}
MM> read demo0.mm
\end{verbatim}
In our description, we always show the full command words.  When using the
Metamath CLI commands in a command file (to be read with the \texttt{submit}
command)\index{\texttt{submit} command}, it is good practice to use
the unabbreviated command to ensure your instructions will not become ambiguous
if more commands are added to the Metamath program in the future.

The command keywords\index{command
keyword} are not case sensitive; you may type either \texttt{read} or
\texttt{ReAd}.  File names may or may not be case sensitive, depending on your
computer's operating system.  Metamath label\index{label} and math
symbol\index{math symbol} tokens\index{token} are case-sensitive.

The \texttt{help} command\index{\texttt{help} command} will provide you
with a list of topics you can get help on.  You can then type
\texttt{help} {\em topic} to get help on that topic.

If you are uncertain of a command's spelling, just type as many characters
as you remember of the command.  If you have not typed enough characters to
specify it unambiguously, Metamath will tell you what choices you have.

\begin{verbatim}
MM> show s
         ^
?Ambiguous keyword - please specify SETTINGS,
STATEMENT, or SOURCE.
\end{verbatim}

If you don't know what argument to use as part of a command, type a
\texttt{?}\index{\texttt{]}@\texttt{?}\ in command lines}\ at the
argument position.  Metamath will tell you what it expected there.

\begin{verbatim}
MM> show ?
         ^
?Expected SETTINGS, LABELS, STATEMENT, SOURCE, PROOF,
MEMORY, TRACE_BACK, or USAGE.
\end{verbatim}

Finally, you may type just the first word or words of a command followed
by {\em return}.  Metamath will prompt you for the remaining part of the
command, showing you the choices at each step.  For example, instead of
typing \texttt{show statement th1 /full} you could interact in the
following manner:
\begin{verbatim}
MM> show
SETTINGS, LABELS, STATEMENT, SOURCE, PROOF,
MEMORY, TRACE_BACK, or USAGE <SETTINGS>? st
What is the statement label <th1>?
/ or nothing <nothing>? /
TEX, COMMENT_ONLY, or FULL <TEX>? f
/ or nothing <nothing>?
19 th1 $p |- t = t $= ... $.
\end{verbatim}
After each \texttt{?}\ in this mode, you must give Metamath the
information it requests.  Sometimes Metamath gives you a list of choices
with the default choice indicated by brackets \texttt{< > }. Pressing
{\em return} after the \texttt{?}\ will select the default choice.
Answering anything else will override the default.  Note that the
\texttt{/} in command qualifiers is considered a separate
token\index{token} by the parser, and this is why it is asked for
separately.

\section{Your First Proof}\label{frstprf}

Proofs are developed with the aid of the Proof Assistant\index{Proof
Assistant}.  We will now show you how the proof of theorem \texttt{th1}
was built.  So that you can repeat these steps, we will first have the
Proof Assistant erase the proof in Metamath's source buffer\index{source
buffer}, then reconstruct it.  (The source buffer is the place in memory
where Metamath stores the information in the database when it is
\texttt{read}\index{\texttt{read} command} in.  New or modified proofs
are kept in the source buffer until a \texttt{write source}
command\index{\texttt{write source} command} is issued.)  In practice, you
would place a \texttt{?}\index{\texttt{]}@\texttt{?}\ inside proofs}\
between \texttt{\$=}\index{\texttt{\$=} keyword} and
\texttt{\$.}\index{\texttt{\$.}\ keyword}\ in the database to indicate
to Metamath\index{Metamath} that the proof is unknown, and that would be
your starting point.  Whenever the \texttt{verify proof} command encounters
a proof with a \texttt{?}\ in place of a proof step, the statement is
identified as not proved.

When I first started creating Metamath proofs, I would write down
on a piece of paper the complete
formal proof\index{formal proof} as it would appear
in a \texttt{show proof} command\index{\texttt{show proof} command}; see
the display of \texttt{show proof th1 /lemmon /re\-num\-ber} above as an
example.  After you get used to using the Proof Assistant\index{Proof
Assistant} you may get to a point where you can ``see'' the proof in your mind
and let the Proof Assistant guide you in filling in the details, at least for
simpler proofs, but until you gain that experience you may find it very useful
to write down all the details in advance.
Otherwise you may waste a lot of time as you let it take you down a wrong path.
However, others do not find this approach as helpful.
For example, Thomas Brendan Leahy\index{Leahy, Thomas Brendan}
finds that it is more helpful to him to interactively
work backward from a machine-readable statement.
David A. Wheeler\index{Wheeler, David A.}
writes down a general approach, but develops the proof
interactively by switching between
working forwards (from hypotheses and facts likely to be useful) and
backwards (from the goal) until the forwards and backwards approaches meet.
In the end, use whatever approach works for you.

A proof is developed with the Proof Assistant by working backwards, starting
with the theorem\index{theorem} to be proved, and assigning each unknown step
with a theorem or hypothesis until no more unknown steps remain.  The Proof
Assistant will not let you make an assignment unless it can be ``unified''
with the unknown step.  This means that a
substitution\index{substitution!variable}\index{variable substitution} of
variables exists that will make the assignment match the unknown step.  On the
other hand, in the middle of a proof, when working backwards, often more than
one unification\index{unification} (set of substitutions) is possible, since
there is not enough information available at that point to uniquely establish
it.  In this case you can tell Metamath which unification to choose, or you
can continue to assign unknown steps until enough information is available to
make the unification unique.

We will assume you have entered Metamath and read in the database as described
above.  The following dialog shows how the proof was developed.  For more
details on what some of the commands do, refer to Section~\ref{pfcommands}.
\index{\texttt{prove} command}

\begin{verbatim}
MM> prove th1
Entering the Proof Assistant.  Type HELP for help, EXIT
to exit.  You will be working on the proof of statement th1:
  $p |- t = t
Note:  The proof you are starting with is already complete.
MM-PA>
\end{verbatim}

The \verb/MM-PA>/ prompt means we are inside the Proof
Assistant.\index{Proof Assistant} Most of the regular Metamath commands
(\texttt{show statement}, etc.) are still available if you need them.

\begin{verbatim}
MM-PA> delete all
The entire proof was deleted.
\end{verbatim}

We have deleted the whole proof so we can start from scratch.

\begin{verbatim}
MM-PA> show new_proof/lemmon/all
1 ?              $? |- t = t
\end{verbatim}

The \texttt{show new{\char`\_}proof} command\index{\texttt{show
new{\char`\_}proof} command} is like \texttt{show proof} except that we
don't specify a statement; instead, the proof we're working on is
displayed.

\begin{verbatim}
MM-PA> assign 1 mp
To undo the assignment, DELETE STEP 5 and INITIALIZE, UNIFY
if needed.
3   min=?  $? |- $2
4   maj=?  $? |- ( $2 -> t = t )
\end{verbatim}

The \texttt{assign} command\index{\texttt{assign} command} above means
``assign step 1 with the statement whose label is \texttt{mp}.''  Note
that step renumbering will constantly occur as you assign steps in the
middle of a proof; in general all steps from the step you assign until
the end of the proof will get moved up.  In this case, what used to be
step 1 is now step 5, because the (partial) proof now has five steps:
the four hypotheses of the \texttt{mp} statement and the \texttt{mp}
statement itself.  Let's look at all the steps in our partial proof:

\begin{verbatim}
MM-PA> show new_proof/lemmon/all
1 ?              $? wff $2
2 ?              $? wff t = t
3 ?              $? |- $2
4 ?              $? |- ( $2 -> t = t )
5 1,2,3,4 mp     $a |- t = t
\end{verbatim}

The symbol \texttt{\$2} is a temporary variable\index{temporary
variable} that represents a symbol sequence not yet known.  In the final
proof, all temporary variables will be eliminated.  The general format
for a temporary variable is \texttt{\$} followed by an integer.  Note
that \texttt{\$} is not a legal character in a math symbol (see
Section~\ref{dollardollar}, p.~\pageref{dollardollar}), so there will
never be a naming conflict between real symbols and temporary variables.

Unknown steps 1 and 2 are constructions of the two wffs used by the
modus ponens rule.  As you will see at the end of this section, the
Proof Assistant\index{Proof Assistant} can usually figure these steps
out by itself, and we will not have to worry about them.  Therefore from
here on we will display only the ``essential'' hypotheses, i.e.\ those
steps that correspond to traditional formal proofs\index{formal proof}.

\begin{verbatim}
MM-PA> show new_proof/lemmon
3 ?              $? |- $2
4 ?              $? |- ( $2 -> t = t )
5 3,4 mp         $a |- t = t
\end{verbatim}

Unknown steps 3 and 4 are the ones we must focus on.  They correspond to the
minor and major premises of the modus ponens rule.  We will assign them as
follows.  Notice that because of the step renumbering that takes place
after an assignment, it is advantageous to assign unknown steps in reverse
order, because earlier steps will not get renumbered.

\begin{verbatim}
MM-PA> assign 4 mp
To undo the assignment, DELETE STEP 8 and INITIALIZE, UNIFY
if needed.
3   min=?  $? |- $2
6     min=?  $? |- $4
7     maj=?  $? |- ( $4 -> ( $2 -> t = t ) )
\end{verbatim}

We are now going to describe an obscure feature that you will probably
never use but should be aware of.  The Metamath language allows empty
symbol sequences to be substituted for variables, but in most formal
systems this feature is never used.  One of the few examples where is it
used is the MIU-system\index{MIU-system} described in
Appendix~\ref{MIU}.  But such systems are rare, and by default this
feature is turned off in the Proof Assistant.  (It is always allowed for
{\tt verify proof}.)  Let us turn it on and see what
happens.\index{\texttt{set empty{\char`\_}substitution} command}

\begin{verbatim}
MM-PA> set empty_substitution on
Substitutions with empty symbol sequences is now allowed.
\end{verbatim}

With this feature enabled, more unifications will be
ambiguous\index{ambiguous unification}\index{unification!ambiguous} in
the middle of a proof, because
substitution\index{substitution!variable}\index{variable substitution}
of variables with empty symbol sequences will become an additional
possibility.  Let's see what happens when we make our next assignment.

\begin{verbatim}
MM-PA> assign 3 a2
There are 2 possible unifications.  Please select the correct
    one or Q if you want to UNIFY later.
Unify:  |- $6
 with:  |- ( $9 + 0 ) = $9
Unification #1 of 2 (weight = 7):
  Replace "$6" with "( + 0 ) ="
  Replace "$9" with ""
  Accept (A), reject (R), or quit (Q) <A>? r
\end{verbatim}

The first choice presented is the wrong one.  If we had selected it,
temporary variable \texttt{\$6} would have been assigned a truncated
wff, and temporary variable \texttt{\$9} would have been assigned an
empty sequence (which is not allowed in our system).  With this choice,
eventually we would reach a point where we would get stuck because
we would end up with steps impossible to prove.  (You may want to
try it.)  We typed \texttt{r} to reject the choice.

\begin{verbatim}
Unification #2 of 2 (weight = 21):
  Replace "$6" with "( $9 + 0 ) = $9"
  Accept (A), reject (R), or quit (Q) <A>? q
To undo the assignment, DELETE STEP 4 and INITIALIZE, UNIFY
if needed.
 7     min=?  $? |- $8
 8     maj=?  $? |- ( $8 -> ( $6 -> t = t ) )
\end{verbatim}

The second choice is correct, and normally we would type \texttt{a}
to accept it.  But instead we typed \texttt{q} to show what will happen:
it will leave the step with an unknown unification, which can be
seen as follows:

\begin{verbatim}
MM-PA> show new_proof/not_unified
 4   min    $a |- $6
        =a2  = |- ( $9 + 0 ) = $9
\end{verbatim}

Later we can unify this with the \texttt{unify}
\texttt{all/interactive} command.

The important point to remember is that occasionally you will be
presented with several unification choices while entering a proof, when
the program determines that there is not enough information yet to make
an unambiguous choice automatically (and this can happen even with
\texttt{set empty{\char`\_}substitution} turned off).  Usually it is
obvious by inspection which choice is correct, since incorrect ones will
tend to be meaningless fragments of wffs.  In addition, the correct
choice will usually be the first one presented, unlike our example
above.

Enough of this digression.  Let us go back to the default setting.

\begin{verbatim}
MM-PA> set empty_substitution off
The ability to substitute empty expressions for variables
has been turned off.  Note that this may make the Proof
Assistant too restrictive in some cases.
\end{verbatim}

If we delete the proof, start over, and get to the point where
we digressed above, there will no longer be an ambiguous unification.

\begin{verbatim}
MM-PA> assign 3 a2
To undo the assignment, DELETE STEP 4 and INITIALIZE, UNIFY
if needed.
 7     min=?  $? |- $4
 8     maj=?  $? |- ( $4 -> ( ( $5 + 0 ) = $5 -> t = t ) )
\end{verbatim}

Let us look at our proof so far, and continue.

\begin{verbatim}
MM-PA> show new_proof/lemmon
 4 a2            $a |- ( $5 + 0 ) = $5
 7 ?             $? |- $4
 8 ?             $? |- ( $4 -> ( ( $5 + 0 ) = $5 -> t = t ) )
 9 7,8 mp        $a |- ( ( $5 + 0 ) = $5 -> t = t )
10 4,9 mp        $a |- t = t
MM-PA> assign 8 a1
To undo the assignment, DELETE STEP 11 and INITIALIZE, UNIFY
if needed.
 7     min=?  $? |- ( t + 0 ) = t
MM-PA> assign 7 a2
To undo the assignment, DELETE STEP 8 and INITIALIZE, UNIFY
if needed.
MM-PA> show new_proof/lemmon
 4 a2            $a |- ( t + 0 ) = t
 8 a2            $a |- ( t + 0 ) = t
12 a1            $a |- ( ( t + 0 ) = t -> ( ( t + 0 ) = t ->
                                                    t = t ) )
13 8,12 mp       $a |- ( ( t + 0 ) = t -> t = t )
14 4,13 mp       $a |- t = t
\end{verbatim}

Now all temporary variables and unknown steps have been eliminated from the
``essential'' part of the proof.  When this is achieved, the Proof
Assistant\index{Proof Assistant} can usually figure out the rest of the proof
automatically.  (Note that the \texttt{improve} command can occasionally be
useful for filling in essential steps as well, but it only tries to make use
of statements that introduce no new variables in their hypotheses, which is
not the case for \texttt{mp}. Also it will not try to improve steps containing
temporary variables.)  Let's look at the complete proof, then run
the \texttt{improve} command, then look at it again.

\begin{verbatim}
MM-PA> show new_proof/lemmon/all
 1 ?             $? wff ( t + 0 ) = t
 2 ?             $? wff t = t
 3 ?             $? term t
 4 3 a2          $a |- ( t + 0 ) = t
 5 ?             $? wff ( t + 0 ) = t
 6 ?             $? wff ( ( t + 0 ) = t -> t = t )
 7 ?             $? term t
 8 7 a2          $a |- ( t + 0 ) = t
 9 ?             $? term ( t + 0 )
10 ?             $? term t
11 ?             $? term t
12 9,10,11 a1    $a |- ( ( t + 0 ) = t -> ( ( t + 0 ) = t ->
                                                    t = t ) )
13 5,6,8,12 mp   $a |- ( ( t + 0 ) = t -> t = t )
14 1,2,4,13 mp   $a |- t = t
\end{verbatim}

\begin{verbatim}
MM-PA> improve all
A proof of length 1 was found for step 11.
A proof of length 1 was found for step 10.
A proof of length 3 was found for step 9.
A proof of length 1 was found for step 7.
A proof of length 9 was found for step 6.
A proof of length 5 was found for step 5.
A proof of length 1 was found for step 3.
A proof of length 3 was found for step 2.
A proof of length 5 was found for step 1.
Steps 1 and above have been renumbered.
CONGRATULATIONS!  The proof is complete.  Use SAVE
NEW_PROOF to save it.  Note:  The Proof Assistant does
not detect $d violations.  After saving the proof, you
should verify it with VERIFY PROOF.
\end{verbatim}

The \texttt{save new{\char`\_}proof} command\index{\texttt{save
new{\char`\_}proof} command} will save the proof in the database.  Here
we will just display it in a form that can be clipped out of a log file
and inserted manually into the database source file with a text
editor.\index{normal proof}\index{proof!normal}

\begin{verbatim}
MM-PA> show new_proof/normal
---------Clip out the proof below this line:
      tt tze tpl tt weq tt tt weq tt a2 tt tze tpl tt weq
      tt tze tpl tt weq tt tt weq wim tt a2 tt tze tpl tt
      tt a1 mp mp $.
---------The proof of 'th1' to clip out ends above this line.
\end{verbatim}

There is another proof format called ``compressed''\index{compressed
proof}\index{proof!compressed} that you will see in databases.  It is
not important to understand how it is encoded but only to recognize it
when you see it.  Its only purpose is to reduce storage requirements for
large proofs.  A compressed proof can always be converted to a normal
one and vice-versa, and the Metamath \texttt{show proof}
commands\index{\texttt{show proof} command} work equally well with
compressed proofs.  The compressed proof format is described in
Appendix~\ref{compressed}.

\begin{verbatim}
MM-PA> show new_proof/compressed
---------Clip out the proof below this line:
      ( tze tpl weq a2 wim a1 mp ) ABCZADZAADZAEZJJKFLIA
      AGHH $.
---------The proof of 'th1' to clip out ends above this line.
\end{verbatim}

Now we will exit the Proof Assistant.  Since we made changes to the proof,
it will warn us that we have not saved it.  In this case, we don't care.

\begin{verbatim}
MM-PA> exit
Warning:  You have not saved changes to the proof.
Do you want to EXIT anyway (Y, N) <N>? y
Exiting the Proof Assistant.
Type EXIT again to exit Metamath.
\end{verbatim}

The Proof Assistant\index{Proof Assistant} has several other commands
that can help you while creating proofs.  See Section~\ref{pfcommands}
for a list of them.

A command that is often useful is \texttt{minimize{\char`\_}with
*/brief}, which tries to shorten the proof.  It can make the process
more efficient by letting you write a somewhat ``sloppy'' proof then
clean up some of the fine details of optimization for you (although it
can't perform miracles such as restructuring the overall proof).

\section{A Note About Editing a Data\-base File}

Once your source file contains proofs, there are some restrictions on
how you can edit it so that the proofs remain valid.  Pay particular
attention to these rules, since otherwise you can lose a lot of work.
It is a good idea to periodically verify all proofs with \texttt{verify
proof *} to ensure their integrity.

If your file contains only normal (as opposed to compressed) proofs, the
main rule is that you may not change the order of the mandatory
hypotheses\index{mandatory hypothesis} of any statement referenced in a
later proof.  For example, if you swap the order of the major and minor
premise in the modus ponens rule, all proofs making use of that rule
will become incorrect.  The \texttt{show statement}
command\index{\texttt{show statement} command} will show you the
mandatory hypotheses of a statement and their order.

If a statement has a compressed proof, you also must not change the
order of {\em its} mandatory hypotheses.  The compressed proof format
makes use of this information as part of the compression technique.
Note that swapping the names of two variables in a theorem will change
the order of its mandatory hypotheses.

The safest way to edit a statement, say \texttt{mytheorem}, is to
duplicate it then rename the original to \texttt{mytheoremOLD}
throughout the database.  Once the edited version is re-proved, all
statements referencing \texttt{mytheoremOLD} can be updated in the Proof
Assistant using \texttt{minimize{\char`\_}with
mytheorem
/allow{\char`\_}growth}.\index{\texttt{minimize{\char`\_}with} command}
% 3/10/07 Note: line-breaking the above results in duplicate index entries

\chapter{Abstract Mathematics Revealed}\label{fol}

\section{Logic and Set Theory}\label{logicandsettheory}

\begin{quote}
  {\em Set theory can be viewed as a form of exact theology.}
  \flushright\sc  Rudy Rucker\footnote{\cite{Barrow}, p.~31.}\\
\end{quote}\index{Rucker, Rudy}

Despite its seeming complexity, all of standard mathematics, no matter how
deep or abstract, can amazingly enough be derived from a relatively small set
of axioms\index{axiom} or first principles. The development of these axioms is
among the most impressive and important accomplishments of mathematics in the
20th century. Ultimately, these axioms can be broken down into a set of rules
for manipulating symbols that any technically oriented person can follow.

We will not spend much time trying to convey a deep, higher-level
understanding of the meaning of the axioms. This kind of understanding
requires some mathematical sophistication as well as an understanding of the
philosophy underlying the foundations of mathematics and typically develops
over time as you work with mathematics.  Our goal, instead, is to give you the
immediate ability to follow how theorems\index{theorem} are derived from the
axioms and from other theorems.  This will be similar to learning the syntax
of a computer language, which lets you follow the details in a program but
does not necessarily give you the ability to write non-trivial programs on
your own, an ability that comes with practice. For now don't be alarmed by
abstract-sounding names of the axioms; just focus on the rules for
manipulating the symbols, which follow the simple conventions of the
Metamath\index{Metamath} language.

The axioms that underlie all of standard mathematics consist of axioms of logic
and axioms of set theory. The axioms of logic are divided into two
subcategories, propositional calculus\index{propositional calculus} (sometimes
called sentential logic\index{sentential logic}) and predicate calculus
(sometimes called first-order logic\index{first-order logic}\index{quantifier
theory}\index{predicate calculus} or quantifier theory).  Propositional
calculus is a prerequisite for predicate calculus, and predicate calculus is a
prerequisite for set theory.  The version of set theory most commonly used is
Zermelo--Fraenkel set theory\index{Zermelo--Fraenkel set theory}\index{set theory}
with the axiom of choice,
often abbreviated as ZFC\index{ZFC}.

Here in a nutshell is what the axioms are all about in an informal way. The
connection between this description and symbols we will show you won't be
immediately apparent and in principle needn't ever be.  Our description just
tries to summarize what mathematicians think about when they work with the
axioms.

Logic is a set of rules that allow us determine truths given other truths.
Put another way,
logic is more or less the translation of what we would consider common sense
into a rigorous set of axioms.\index{axioms of logic}  Suppose $\varphi$,
$\psi$, and $\chi$ (the Greek letters phi, psi, and chi) represent statements
that are either true or false, and $x$ is a variable\index{variable!in predicate
calculus} ranging over some group of mathematical objects (sets, integers,
real numbers, etc.). In mathematics, a ``statement'' really means a formula,
and $\psi$ could be for example ``$x = 2$.''
Propositional calculus\index{propositional calculus}
allows us to use variables that are either true or false
and make deductions such as
``if $\varphi$ implies $\psi$ and $\psi$ implies $\chi$, then $\varphi$
implies $\chi$.''
Predicate calculus\index{predicate calculus}
extends propositional calculus by also allowing us
to discuss statements about objects (not just true and false values), including
statements about ``all'' or ``at least one'' object.
For example, predicate calculus allows to say,
``if $\varphi$ is true for all $x$, then $\varphi$ is true for some $x$.''
The logic used in \texttt{set.mm} is standard classical logic
(as opposed to other logic systems like intuitionistic logic).

Set theory\index{set theory} has to do with the manipulation of objects and
collections of objects, specifically the abstract, imaginary objects that
mathematics deals with, such as numbers. Everything that is claimed to exist
in mathematics is considered to be a set.  A set called the empty
set\index{empty set} contains nothing.  We represent the empty set by
$\varnothing$.  Many sets can be built up from the empty set.  There is a set
represented by $\{\varnothing\}$ that contains the empty set, another set
represented by $\{\varnothing,\{\varnothing\}\}$ that contains this set as
well as the empty set, another set represented by $\{\{\varnothing\}\}$ that
contains just the set that contains the empty set, and so on ad infinitum. All
mathematical objects, no matter how complex, are defined as being identical to
certain sets: the integer\index{integer} 0 is defined as the empty set, the
integer 1 is defined as $\{\varnothing\}$, the integer 2 is defined as
$\{\varnothing,\{\varnothing\}\}$.  (How these definitions were chosen doesn't
matter now, but the idea behind it is that these sets have the properties we
expect of integers once suitable operations are defined.)  Mathematical
operations, such as addition, are defined in terms of operations on
sets---their union\index{set union}, intersection\index{set intersection}, and
so on---operations you may have used in elementary school when you worked
with groups of apples and oranges.

With a leap of faith, the axioms also postulate the existence of infinite
sets\index{infinite set}, such as the set of all non-negative integers ($0, 1,
2,\ldots$, also called ``natural numbers''\index{natural number}).  This set
can't be represented with the brace notation\index{brace notation} we just
showed you, but requires a more complicated notation called ``class
abstraction.''\index{class abstraction}\index{abstraction class}  For
example, the infinite set $\{ x |
\mbox{``$x$ is a natural number''} \} $ means the ``set of all objects $x$
such that $x$ is a natural number'' i.e.\ the set of natural numbers; here,
``$x$ is a natural number'' is a rather complicated formula when broken down
into the primitive symbols.\label{expandom}\footnote{The statement ``$x$ is a
natural number'' is formally expressed as ``$x \in \omega$,'' where $\in$
(stylized epsilon) means ``is in'' or ``is an element of'' and $\omega$
(omega) means ``the set of natural numbers.''  When ``$x\in\omega$'' is
completely expanded in terms of the primitive symbols of set theory, the
result is  $\lnot$ $($ $\lnot$ $($ $\forall$ $z$ $($ $\lnot$ $\forall$ $w$ $($
$z$ $\in$ $w$ $\rightarrow$ $\lnot$ $w$ $\in$ $x$ $)$ $\rightarrow$ $z$ $\in$
$x$ $)$ $\rightarrow$ $($ $\forall$ $z$ $($ $\lnot$ $($ $\forall$ $w$ $($ $w$
$\in$ $z$ $\rightarrow$ $w$ $\in$ $x$ $)$ $\rightarrow$ $\forall$ $w$ $\lnot$
$w$ $\in$ $z$ $)$ $\rightarrow$ $\lnot$ $\forall$ $w$ $($ $w$ $\in$ $z$
$\rightarrow$ $\lnot$ $\forall$ $v$ $($ $v$ $\in$ $z$ $\rightarrow$ $\lnot$
$v$ $\in$ $w$ $)$ $)$ $)$ $\rightarrow$ $\lnot$ $\forall$ $z$ $\forall$ $w$
$($ $\lnot$ $($ $z$ $\in$ $x$ $\rightarrow$ $\lnot$ $w$ $\in$ $x$ $)$
$\rightarrow$ $($ $\lnot$ $z$ $\in$ $w$ $\rightarrow$ $($ $\lnot$ $z$ $=$ $w$
$\rightarrow$ $w$ $\in$ $z$ $)$ $)$ $)$ $)$ $)$ $\rightarrow$ $\lnot$
$\forall$ $y$ $($ $\lnot$ $($ $\lnot$ $($ $\forall$ $z$ $($ $\lnot$ $\forall$
$w$ $($ $z$ $\in$ $w$ $\rightarrow$ $\lnot$ $w$ $\in$ $y$ $)$ $\rightarrow$
$z$ $\in$ $y$ $)$ $\rightarrow$ $($ $\forall$ $z$ $($ $\lnot$ $($ $\forall$
$w$ $($ $w$ $\in$ $z$ $\rightarrow$ $w$ $\in$ $y$ $)$ $\rightarrow$ $\forall$
$w$ $\lnot$ $w$ $\in$ $z$ $)$ $\rightarrow$ $\lnot$ $\forall$ $w$ $($ $w$
$\in$ $z$ $\rightarrow$ $\lnot$ $\forall$ $v$ $($ $v$ $\in$ $z$ $\rightarrow$
$\lnot$ $v$ $\in$ $w$ $)$ $)$ $)$ $\rightarrow$ $\lnot$ $\forall$ $z$
$\forall$ $w$ $($ $\lnot$ $($ $z$ $\in$ $y$ $\rightarrow$ $\lnot$ $w$ $\in$
$y$ $)$ $\rightarrow$ $($ $\lnot$ $z$ $\in$ $w$ $\rightarrow$ $($ $\lnot$ $z$
$=$ $w$ $\rightarrow$ $w$ $\in$ $z$ $)$ $)$ $)$ $)$ $\rightarrow$ $($
$\forall$ $z$ $\lnot$ $z$ $\in$ $y$ $\rightarrow$ $\lnot$ $\forall$ $w$ $($
$\lnot$ $($ $w$ $\in$ $y$ $\rightarrow$ $\lnot$ $\forall$ $z$ $($ $w$ $\in$
$z$ $\rightarrow$ $\lnot$ $z$ $\in$ $y$ $)$ $)$ $\rightarrow$ $\lnot$ $($
$\lnot$ $\forall$ $z$ $($ $w$ $\in$ $z$ $\rightarrow$ $\lnot$ $z$ $\in$ $y$
$)$ $\rightarrow$ $w$ $\in$ $y$ $)$ $)$ $)$ $)$ $\rightarrow$ $x$ $\in$ $y$
$)$ $)$ $)$. Section~\ref{hierarchy} shows the hierarchy of definitions that
leads up to this expression.}\index{stylized epsilon ($\in$)}\index{omega
($\omega$)}  Actually, the primitive symbols don't even include the brace
notation.  The brace notation is a high-level definition, which you can find in
Section~\ref{hierarchy}.

Interestingly, the arithmetic of integers\index{integer} and
rationals\index{rational number} can be developed without appealing to the
existence of an infinite set, whereas the arithmetic of real
numbers\index{real number} requires it.

Each variable\index{variable!in set theory} in the axioms of set theory
represents an arbitrary set, and the axioms specify the legal kinds of things
you can do with these variables at a very primitive level.

Now, you may think that numbers and arithmetic are a lot more intuitive and
fundamental than sets and therefore should be the foundation of mathematics.
What is really the case is that you've dealt with numbers all your life and
are comfortable with a few rules for manipulating them such as addition and
multiplication.  Those rules only cover a small portion of what can be done
with numbers and only a very tiny fraction of the rest of mathematics.  If you
look at any elementary book on number theory, you will quickly become lost if
these are the only rules that you know.  Even though such books may present a
list of ``axioms''\index{axiom} for arithmetic, the ability to use the axioms
and to understand proofs of theorems\index{theorem} (facts) about numbers
requires an implicit mathematical talent that frustrates many people
from studying abstract mathematics.  The kind of mathematics that most people
know limits them to the practical, everyday usage of blindly manipulating
numbers and formulas, without any understanding of why those rules are correct
nor any ability to go any further.  For example, do you know why multiplying
two negative numbers yields a positive number?  Starting with set theory, you
will also start off blindly manipulating symbols according to the rules we give
you, but with the advantage that these rules will allow you, in principle, to
access {\em all} of mathematics, not just a tiny part of it.

Of course, concrete examples are often helpful in the learning process. For
example, you can verify that $2\cdot 3=3 \cdot 2$ by actually grouping
objects and can easily ``see'' how it generalizes to $x\cdot y = y\cdot x$,
even though you might not be able to rigorously prove it.  Similarly, in set
theory it can be helpful to understand how the axioms of set theory apply to
(and are correct for) small finite collections of objects.  You should be aware
that in set theory intuition can be misleading for infinite collections, and
rigorous proofs become more important.  For example, while $x\cdot y = y\cdot
x$ is correct for finite ordinals (which are the natural numbers), it is not
usually true for infinite ordinals.

\section{The Axioms for All of Mathematics}

In this section\index{axioms for mathematics}, we will show you the axioms
for all of standard mathematics (i.e.\ logic and set theory) as they are
traditionally presented.  The traditional presentation is useful for someone
with the mathematical experience needed to correctly manipulate high-level
abstract concepts.  For someone without this talent, knowing how to actually
make use of these axioms can be difficult.  The purpose of this section is to
allow you to see how the version of the axioms used in the standard
Metamath\index{Metamath} database \texttt{set.mm}\index{set
theory database (\texttt{set.mm})} relates to  the typical version
in textbooks, and also to give you an informal feel for them.

\subsection{Propositional Calculus}

Propositional calculus\index{propositional calculus} concerns itself with
statements that can be interpreted as either true or false.  Some examples of
statements (outside of mathematics) that are either true or false are ``It is
raining today'' and ``The United States has a female president.'' In
mathematics, as we mentioned, statements are really formulas.

In propositional calculus, we don't care what the statements are.  We also
treat a logical combination of statements, such as ``It is raining today and
the United States has a female president,'' no differently from a single
statement.  Statements and their combinations are called well-formed formulas
(wffs)\index{well-formed formula (wff)}.  We define wffs only in terms of
other wffs and don't define what a ``starting'' wff is.  As is common practice
in the literature, we use Greek letters to represent wffs.

Specifically, suppose $\varphi$ and $\psi$ are wffs.  Then the combinations
$\varphi\rightarrow\psi$ (``$\varphi$ implies $\psi$,'' also read ``if
$\varphi$ then $\psi$'')\index{implication ($\rightarrow$)} and $\lnot\varphi$
(``not $\varphi$'')\index{negation ($\lnot$)} are also wffs.

The three axioms of propositional calculus\index{axioms of propositional
calculus} are all wffs of the following form:\footnote{A remarkable result of
C.~A.~Meredith\index{Meredith, C. A.} squeezes these three axioms into the
single axiom $((((\varphi\rightarrow \psi)\rightarrow(\neg \chi\rightarrow\neg
\theta))\rightarrow \chi )\rightarrow \tau)\rightarrow((\tau\rightarrow
\varphi)\rightarrow(\theta\rightarrow \varphi))$ \cite{CAMeredith},
which is believed to be the shortest possible.}
\begin{center}
     $\varphi\rightarrow(\psi\rightarrow \varphi)$\\

     $(\varphi\rightarrow (\psi\rightarrow \chi))\rightarrow
((\varphi\rightarrow  \psi)\rightarrow (\varphi\rightarrow \chi))$\\

     $(\neg \varphi\rightarrow \neg\psi)\rightarrow (\psi\rightarrow
\varphi)$
\end{center}

These three axioms are widely used.
They are attributed to Jan {\L}ukasiewicz
(pronounced woo-kah-SHAY-vitch) and was popularized by Alonzo Church,
who called it system P2. (Thanks to Ted Ulrich for this information.)

There are an infinite number of axioms, one for each possible
wff\index{well-formed formula (wff)} of the above form.  (For this reason,
axioms such as the above are often called ``axiom schemes.''\index{axiom
scheme})  Each Greek letter in the axioms may be substituted with a more
complex wff to result in another axiom.  For example, substituting
$\neg(\varphi\rightarrow\chi)$ for $\varphi$ in the first axiom yields
$\neg(\varphi\rightarrow\chi)\rightarrow(\psi\rightarrow
\neg(\varphi\rightarrow\chi))$, which is still an axiom.

To deduce new true statements (theorems\index{theorem}) from the axioms, a
rule\index{rule} called ``modus ponens''\index{modus ponens} is used.  This
rule states that if the wff $\varphi$ is an axiom or a theorem, and the wff
$\varphi\rightarrow\psi$ is an axiom or a theorem, then the wff $\psi$ is also
a theorem\index{theorem}.

As a non-mathematical example of modus ponens, suppose we have proved (or
taken as an axiom) ``Bob is a man'' and separately have proved (or taken as
an axiom) ``If Bob is a man, then Bob is a human.''  Using the rule of modus
ponens, we can logically deduce, ``Bob is a human.''

From Metamath's\index{Metamath} point of view, the axioms and the rule of
modus ponens just define a mechanical means for deducing new true statements
from existing true statements, and that is the complete content of
propositional calculus as far as Metamath is concerned.  You can read a logic
textbook to gain a better understanding of their meaning, or you can just let
their meaning slowly become apparent to you after you use them for a while.

It is actually rather easy to check to see if a formula is a theorem of
propositional calculus.  Theorems of propositional calculus are also called
``tautologies.''\index{tautology}  The technique to check whether a formula is
a tautology is called the ``truth table method,''\index{truth table} and it
works like this.  A wff $\varphi\rightarrow\psi$ is false whenever $\varphi$ is true
and $\psi$ is false.  Otherwise it is true.  A wff $\lnot\varphi$ is false
whenever $\varphi$ is true and false otherwise. To verify a tautology such as
$\varphi\rightarrow(\psi\rightarrow \varphi)$, you break it down into sub-wffs and
construct a truth table that accounts for all possible combinations of true
and false assigned to the wff metavariables:
\begin{center}\begin{tabular}{|c|c|c|c|}\hline
\mbox{$\varphi$} & \mbox{$\psi$} & \mbox{$\psi\rightarrow\varphi$}
    & \mbox{$\varphi\rightarrow(\psi\rightarrow \varphi)$} \\ \hline \hline
              T   &  T    &      T       &        T    \\ \hline
              T   &  F    &      T       &        T    \\ \hline
              F   &  T    &      F       &        T    \\ \hline
              F   &  F    &      T       &        T    \\ \hline
\end{tabular}\end{center}
If all entries in the last column are true, the formula is a tautology.

Now, the truth table method doesn't tell you how to prove the tautology from
the axioms, but only that a proof exists.  Finding an actual proof (especially
one that is short and elegant) can be challenging.  Methods do exist for
automatically generating proofs in propositional calculus, but the proofs that
result can sometimes be very long.  In the Metamath \texttt{set.mm}\index{set
theory database (\texttt{set.mm})} database, most
or all proofs were created manually.

Section \ref{metadefprop} discusses various definitions
that make propositional calculus easier to use.
For example, we define:

\begin{itemize}
\item $\varphi \vee \psi$
  is true if either $\varphi$ or $\psi$ (or both) are true
  (this is disjunction\index{disjunction ($\vee$)}
  aka logical {\sc or}\index{logical {\sc or} ($\vee$)}).

\item $\varphi \wedge \psi$
  is true if both $\varphi$ and $\psi$ are true
  (this is conjunction\index{conjunction ($\wedge$)}
  aka logical {\sc and}\index{logical {\sc and} ($\wedge$)}).

\item $\varphi \leftrightarrow \psi$
  is true if $\varphi$ and $\psi$ have the same value, that is,
  they are both true or both false
  (this is the biconditional\index{biconditional ($\leftrightarrow$)}).
\end{itemize}

\subsection{Predicate Calculus}

Predicate calculus\index{predicate calculus} introduces the concept of
``individual variables,''\index{variable!in predicate calculus}\index{individual
variable} which
we will usually just call ``variables.''
These variables can represent something other than true or false (wffs),
and will always represent sets when we get to set theory.  There are also
three new symbols $\forall$\index{universal quantifier ($\forall$)},
$=$\index{equality ($=$)}, and $\in$\index{stylized epsilon ($\in$)},
read ``for all,'' ``equals,'' and ``is an element of''
respectively.  We will represent variables with the letters $x$, $y$, $z$, and
$w$, as is common practice in the literature.
For example, $\forall x \varphi$ means ``for all possible values of
$x$, $\varphi$ is true.''

In predicate calculus, we extend the definition of a wff\index{well-formed
formula (wff)}.  If $\varphi$ is a wff and $x$ and $y$ are variables, then
$\forall x \, \varphi$, $x=y$, and $x\in y$ are wffs. Note that these three new
types of wffs can be considered ``starting'' wffs from which we can build
other wffs with $\rightarrow$ and $\neg$ .  The concept of a starting wff was
absent in propositional calculus.  But starting wff or not, all we are really
concerned with is whether our wffs are correctly constructed according to
these mechanical rules.

A quick aside:
To prevent confusion, it might be best at this point to think of the variables
of Metamath\index{Metamath} as ``metavariables,''\index{metavariable} because
they are not quite the same as the variables we are introducing here.  A
(meta)variable in Metamath can be a wff or an individual variable, as well
as many other things; in general, it represents a kind of place holder for an
unspecified sequence of math symbols\index{math symbol}.

Unlike propositional calculus, no decision procedure\index{decision procedure}
analogous to the truth table method exists (nor theoretically can exist) that
will definitely determine whether a formula is a theorem of predicate
calculus.  Much of the work in the field of automated theorem
proving\index{automated theorem proving} has been dedicated to coming up with
clever heuristics for proving theorems of predicate calculus, but they can
never be guaranteed to work always.

Section \ref{metadefpred} discusses various definitions
that make predicate calculus easier to use.
For example, we define
$\exists x \varphi$ to mean
``there exists at least one possible value of $x$ where $\varphi$ is true.''

We now turn to looking at how predicate calculus can be formally
represented.

\subsubsection{Common Axioms}

There is a new rule of inference in predicate calculus:  if $\varphi$ is
an axiom or a theorem, then $\forall x \,\varphi$ is also a
theorem\index{theorem}.  This is called the rule of
``generalization.''\index{rule of generalization}
This is easily represented in Metamath.

In standard texts of logic, there are often two axioms of predicate
calculus\index{axioms of predicate calculus}:
\begin{center}
  $\forall x \,\varphi ( x ) \rightarrow \varphi ( y )$,
      where ``$y$ is properly substituted for $x$.''\\
  $\forall x ( \varphi \rightarrow \psi )\rightarrow ( \varphi \rightarrow
    \forall x\, \psi )$,
    where ``$x$ is not free in $\varphi$.''
\end{center}

Now at first glance, this seems simple:  just two axioms.  However,
conditional clauses are attached to each axiom describing requirements that
may seem puzzling to you.  In addition, the first axiom puts a variable symbol
in parentheses after each wff, seemingly violating our definition of a
wff\index{well-formed formula (wff)}; this is just an informal way of
referring to some arbitrary variable that may occur in the wff.  The
conditional clauses do, of course, have a precise meaning, but as it turns out
the precise meaning is somewhat complicated and awkward to formalize in a
way that a computer can handle easily.  Unlike propositional calculus, a
certain amount of mathematical sophistication and practice is needed to be
able to easily grasp and manipulate these concepts correctly.

Predicate calculus may be presented with or without axioms for
equality\index{axioms of equality}\index{equality ($=$)}. We will require the
axioms of equality as a prerequisite for the version of set theory we will
use.  The axioms for equality, when included, are often represented using these
two axioms:
\begin{center}
$x=x$\\ \ \\
$x=y\rightarrow (\varphi(x,x)\rightarrow\varphi(x,y))$ where ``$\varphi(x,y)$
   arises from $\varphi(x,x)$ by replacing some, but not necessarily all,
   free\index{free variable}
   occurrences of $x$ by $y$,\\ provided that $y$ is free for $x$
   in $\varphi(x,x)$.'' \end{center}
% (Mendelson p. 95)
The first equality axiom is simple, but again,
the condition on the second one is
somewhat awkward to implement on a computer.

\subsubsection{Tarski System S2}

Of course, we are not the first to notice the complications of these
predicate calculus axioms when being rigorous.

Well-known logician Alfred Tarski published in 1965
a system he called system S2\cite[p.~77]{Tarski1965}.
Tarski's system is \textit{exactly equivalent} to the traditional textbook
formalization, but (by clever use of equality axioms) it eliminates the
latter's primitive notions of ``proper substitution'' and ``free variable,''
replacing them with direct substitution and the notion of a variable
not occurring in a formula (which we express with distinct variable
constraints).

In advocating his system, Tarski wrote, ``The relatively complicated
character of [free variables and proper substitution] is a source
of certain inconveniences of both practical and theoretical nature;
this is clearly experienced both in teaching an elementary course of
mathematical logic and in formalizing the syntax of predicate logic for
some theoretical purposes''\cite[p.~61]{Tarski1965}\index{Tarski, Alfred}.

\subsubsection{Developing a Metamath Representation}

The standard textbook axioms of predicate calculus are somewhat
cumbersome to implement on a computer because of the complex notions of
``free variable''\index{free variable} and ``proper
substitution.''\index{proper substitution}\index{substitution!proper}
While it is possible to use the Metamath\index{Metamath} language to
implement these concepts, we have chosen not to implement them
as primitive constructs in the
\texttt{set.mm} set theory database.  Instead, we have eliminated them
within the axioms
by carefully crafting the axioms so as to avoid them,
building on Tarski's system S2.  This makes it
easy for a beginner to follow the steps in a proof without knowing any
advanced concepts other than the simple concept of
replacing\index{substitution!variable}\index{variable substitution}
variables with expressions.

In order to develop the concepts of free variable and proper
substitution from the axioms, we use an additional
Metamath statement type called ``disjoint variable
restriction''\index{disjoint variables} that we have not encountered
before.  In the context of the axioms, the statement \texttt{\$d} $ x\,
y$\index{\texttt{\$d} statement} simply means that $x$ and $y$ must be
distinct\index{distinct variables}, i.e.\ they may not be simultaneously
substituted\index{substitution!variable}\index{variable substitution}
with the same variable.  The statement \texttt{\$d} $ x\, \varphi$ means
variable $x$ must not occur in wff $\varphi$.  For the precise
definition of \texttt{\$d}, see Section~\ref{dollard}.

\subsubsection{Metamath representation}

The Metamath axiom system for predicate calculus
defined in set.mm uses Tarski's system S2.
As noted above, this has a different representation
than the traditional textbook formalization,
but it is \textit{exactly equivalent} to the textbook formalization,
and it is \textit{much} easier to work with.
This is reproduced as system S3 in Section 6 of
Megill's formalization \cite{Megill}\index{Megill, Norman}.

There is one exception, Tarski's axiom of existence,
which we label as axiom ax-6.
In the case of ax-6, Tarski's version is weaker because it includes a
distinct variable proviso. If we wish, we can also weaken our version
in this way and still have a metalogically complete system. Theorem
ax6 shows this by deriving, in the presence of the other axioms, our
ax-6 from Tarski's weaker version ax6v. However, we chose the stronger
version for our system because it is simpler to state and easier to use.

Tarski's system was designed for proving specific theorems rather than
more general theorem schemes. However, theorem schemes are much more
efficient than specific theorems for building a body of mathematical
knowledge, since they can be reused with different instances as
needed. While Tarski does derive some theorem schemes from his axioms,
their proofs require concepts that are ``outside'' of the system, such as
induction on formula length. The verification of such proofs is difficult
to automate in a proof verifier. (Specifically, Tarski treats the formulas
of his system as set-theoretical objects. In order to verify the proofs
of his theorem schemes, a proof verifier would need a significant amount
of set theory built into it.)

The Metamath axiom system for predicate calculus extends
Tarski's system to eliminate this difficulty. The additional
``auxilliary'' axiom
schemes (as we will call them in this section; see below) endow Tarski's
system with a nice property we call
metalogical completeness \cite[Remark 9.6]{Megill}\index{Megill, Norman}.
As a result, we can prove any theorem scheme
expressable in the ``simple metalogic'' of Tarski's system by using
only Metamath's direct substitution rule applied to the axiom system
(and no other metalogical or set-theoretical notions ``outside'' of the
system). Simple metalogic consists of schemes containing wff metavariables
(with no arguments) and/or set (also called ``individual'') metavariables,
accompanied by optional provisos each stating that two specified set
metavariables must be distinct or that a specified set metavariable may
not occur in a specified wff metavariable. Metamath's logic and set theory
axiom and rule schemes are all examples of simple metalogic. The schemes
of traditional predicate calculus with equality are examples which are
not simple metalogic, because they use wff metavariables with arguments
and have ``free for'' and ``not free in'' side conditions.

A rigorous justification for this system, using an older but
exactly equivalent set of axioms, can be
found in \cite{Megill}\index{Megill, Norman}.

This allows us to
take a different approach in the Metamath\index{Metamath} database
\texttt{set.mm}\index{set theory database (\texttt{set.mm})}.  We do not
directly use the primitive notions of ``free variable''\index{free variable}
and ``proper substitution''\index{proper
substitution}\index{substitution!proper} at all as primitive constructs.
Instead, we use a set
of axioms that are almost as simple to manipulate as those of
propositional calculus.  Our axiom system avoids complex primitive
notions by effectively embedding the complexity into the axioms
themselves.  As a result, we will end up with a larger number of axioms,
but they are ideally suited for a computer language such as Metamath.
(Section~\ref{metaaxioms} shows these axioms.)

We will not elaborate further
on the ``free variable'' and ``proper substitution''
concepts here.  You may consult
\cite[ch.\ 3--4]{Hamilton}\index{Hamilton, Alan G.} (as well as
many other books) for a precise explanation
of these concepts.  If you intend to do serious mathematical work, it is wise
to become familiar with the traditional textbook approach; even though the
concepts embedded in their axioms require a higher level of sophistication,
they can be more practical to deal with on an everyday, informal basis.  Even
if you are just developing Metamath proofs, familiarity with the traditional
approach can help you arrive at a proof outline much faster, which you can
then convert to the detail required by Metamath.

We do develop proper substitution rules later on, but in set.mm
they are defined as derived constructs; they are not primitives.

You should also note that our system of predicate calculus is specifically
tailored for set theory; thus there are only two specific predicates $=$ and
$\in$ and no functions\index{function!in predicate calculus}
or constants\index{constant!in predicate calculus} unlike more general systems.
We later add these.

\subsection{Set Theory}

Traditional Zermelo--Fraenkel set theory\index{Zermelo--Fraenkel set
theory}\index{set theory} with the Axiom of Choice
has 10 axioms, which can be expressed in the
language of predicate calculus.  In this section, we will list only the
names and brief English descriptions of these axioms, since we will give
you the precise formulas used by the Metamath\index{Metamath} set theory
database \texttt{set.mm} later on.

In the descriptions of the axioms, we assume that $x$, $y$, $z$, $w$, and $v$
represent sets.  These are the same as the variables\index{variable!in set
theory} in our predicate calculus system above, except that now we informally
think of the variables as ranging over sets.  Note that the terms
``object,''\index{object} ``set,''\index{set} ``element,''\index{element}
``collection,''\index{collection} and ``family''\index{family} are synonymous,
as are ``is an element of,'' ``is a member of,''\index{member} ``is contained
in,'' and ``belongs to.''  The different terms are used for convenience; for
example, ``a collection of sets'' is less confusing than ``a set of sets.''
A set $x$ is said to be a ``subset''\index{subset} of $y$ if every element of
$x$ is also an element of $y$; we also say $x$ is ``included in''
$y$.

The axioms are very general and apply to almost any conceivable mathematical
object, and this level of abstraction can be overwhelming at first.  To gain an
intuitive feel, it can be helpful to draw a picture illustrating the concept;
for example, a circle containing dots could represent a collection of sets,
and a smaller circle drawn inside the circle could represent a subset.
Overlapping circles can illustrate intersection and union.  Circles that
illustrate the concepts of set theory are frequently used in elementary
textbooks and are called Venn diagrams\index{Venn diagram}.\index{axioms of
set theory}

1. Axiom of Extensionality:  Two sets are identical if they contain the same
   elements.\index{Axiom of Extensionality}

2. Axiom of Pairing:  The set $\{ x , y \}$ exists.\index{Axiom of Pairing}

3. Axiom of Power Sets:  The power set of a set (the collection of all of
   its subsets) exists.  For example, the power set of $\{x,y\}$ is
   $\{\varnothing,\{x\},\{y\},\{x,y\}\}$ and it exists.\index{Axiom
of Power Sets}

4. Axiom of the Null Set:  The empty set $\varnothing$ exists.\index{Axiom of
the Null Set}

5. Axiom of Union:  The union of a set (the set containing the elements of
   its members) exists.  For example, the union of $\{\{x,y\},\{z\}\}$ is
 $\{x,y,z\}$ and
   it exists.\index{Axiom of Union}

6. Axiom of Regularity:  Roughly, no set can contain itself, nor can there
   be membership ``loops,'' such as a set being an
   element of one of its members.\index{Axiom of Regularity}

7. Axiom of Infinity:  An infinite set exists.  An example of an infinite
   set is the set of all
   integers.\index{Axiom of Infinity}

8. Axiom of Separation:  The set exists that is obtained by restricting $x$
   with some property.  For example, if the set of all integers exists,
   then the set of all even integers exists.\index{Axiom of Separation}

9. Axiom of Replacement:  The range of a function whose domain is restricted
   to the elements of a set $x$, is also a set.  For example, there
   is a function
   from integers (the function's domain) to their squares (its
   range).  If we
   restrict the domain to even integers, its range will become the set of
   squares of even integers, so this axiom asserts that the set of
    squares of even numbers exists.  Technical note:  In general, the
   ``function'' need not be a set but can be a proper class.
   \index{Axiom of Replacement}

10. Axiom of Choice:  Let $x$ be a set whose members are pairwise
  disjoint\index{disjoint sets} (i.e,
  whose members contain no elements in common).  Then there exists another
  set containing one element from each member of $x$.  For
  example, if $x$ is
  $\{\{y,z\},\{w,v\}\}$, where $y$, $z$, $w$, and $v$ are
  different sets, then a set such as $\{z,w\}$
  exists (but the axiom doesn't tell
  us which one).  (Actually the Axiom
  of Choice is redundant if the set $x$, as in this example, has a finite
  number of elements.)\index{Axiom of Choice}

The Axiom of Choice is usually considered an extension of ZF set theory rather
than a proper part of it.  It is sometimes considered philosophically
controversial because it specifies the existence of a set without specifying
what the set is. Constructive logics, including intuitionistic logic,
do not accept the axiom of choice.
Since there is some lingering controversy, we often prefer proofs that do
not use the axiom of choice (where there is a known alternative), and
in some cases we will use weaker axioms than the full axiom of choice.
That said, the axiom of choice is a powerful and widely-accepted tool,
so we do use it when needed.
ZF set theory that includes the Axiom of Choice is
called Zermelo--Fraenkel set theory with choice (ZFC\index{ZFC set theory}).

When expressed symbolically, the Axiom of Separation and the Axiom of
Replacement contain wff symbols and therefore each represent infinitely many
axioms, one for each possible wff. For this reason, they are often called
axiom schemes\index{axiom scheme}\index{well-formed formula (wff)}.

It turns out that the Axiom of the Null Set, the Axiom of Pairing, and the
Axiom of Separation can be derived from the other axioms and are therefore
unnecessary, although they tend to be included in standard texts for various
reasons (historical, philosophical, and possibly because some authors may not
know this).  In the Metamath\index{Metamath} set theory database, these
redundant axioms are derived from the other ones instead of truly
being considered axioms.
This is in keeping with our general goal of minimizing the number of
axioms we must depend on.

\subsection{Other Axioms}

Above we qualified the phrase ``all of mathematics'' with ``essentially.''
The main important missing piece is the ability to do category theory,
which requires huge sets (inaccessible cardinals) larger than those
postulated by the ZFC axioms. The Tarski--Grothendieck Axiom postulates
the existence of such sets.
Note that this is the same axiom used by Mizar for supporting
category theory.
The Tarski--Grothendieck axiom
can be viewed as a very strong replacement of the Axiom of Infinity,
the Axiom of Choice, and the Axiom of Power Sets.
The \texttt{set.mm} database includes this axiom; see the database
for details about it.
Again, we only use this axiom when we need to.
You are only likely to encounter or use this axiom if you are doing
category theory, since its use is highly specialized,
so we will not list the Tarsky-Grothendieck axiom
in the short list of axioms below.

Can there be even more axioms?
Of course.
G\"{o}del showed that no finite set of axioms or axiom schemes can completely
describe any consistent theory strong enough to include arithmetic.
But practically speaking, the ones above are the accepted foundation that
almost all mathematicians explicitly or implicitly base their work on.

\section{The Axioms in the Metamath Language}\label{metaaxioms}

Here we list the axioms as they appear in
\texttt{set.mm}\index{set theory database (\texttt{set.mm})} so you can
look them up there easily.  Incidentally, the \texttt{show statement
/tex} command\index{\texttt{show statement} command} was used to
typeset them.

%macros from show statement /tex
\newbox\mlinebox
\newbox\mtrialbox
\newbox\startprefix  % Prefix for first line of a formula
\newbox\contprefix  % Prefix for continuation line of a formula
\def\startm{  % Initialize formula line
  \setbox\mlinebox=\hbox{\unhcopy\startprefix}
}
\def\m#1{  % Add a symbol to the formula
  \setbox\mtrialbox=\hbox{\unhcopy\mlinebox $\,#1$}
  \ifdim\wd\mtrialbox>\hsize
    \box\mlinebox
    \setbox\mlinebox=\hbox{\unhcopy\contprefix $\,#1$}
  \else
    \setbox\mlinebox=\hbox{\unhbox\mtrialbox}
  \fi
}
\def\endm{  % Output the last line of a formula
  \box\mlinebox
}

% \SLASH for \ , \TOR for \/ (text OR), \TAND for /\ (text and)
% This embeds a following forced space to force the space.
\newcommand\SLASH{\char`\\~}
\newcommand\TOR{\char`\\/~}
\newcommand\TAND{/\char`\\~}
%
% Macro to output metamath raw text.
% This assumes \startprefix and \contprefix are set.
% NOTE: "\" is tricky to escape, use \SLASH, \TOR, and \TAND inside.
% Any use of "$ { ~ ^" must be escaped; ~ and ^ must be escaped specially.
% We escape { and } for consistency.
% For more about how this macro written, see:
% https://stackoverflow.com/questions/4073674/
% how-to-disable-indentation-in-particular-section-in-latex/4075706
% Use frenchspacing, or "e." will get an extra space after it.
\newlength\mystoreparindent
\newlength\mystorehangindent
\newenvironment{mmraw}{%
\setlength{\mystoreparindent}{\the\parindent}
\setlength{\mystorehangindent}{\the\hangindent}
\setlength{\parindent}{0pt} % TODO - we'll put in the \startprefix instead
\setlength{\hangindent}{\wd\the\contprefix}
\begin{flushleft}
\begin{frenchspacing}
\begin{tt}
{\unhcopy\startprefix}%
}{%
\end{tt}
\end{frenchspacing}
\end{flushleft}
\setlength{\parindent}{\mystoreparindent}
\setlength{\hangindent}{\mystorehangindent}
\vskip 1ex
}

\needspace{5\baselineskip}
\subsection{Propositional Calculus}\label{propcalc}\index{axioms of
propositional calculus}

\needspace{2\baselineskip}
Axiom of Simplification.\label{ax1}

\setbox\startprefix=\hbox{\tt \ \ ax-1\ \$a\ }
\setbox\contprefix=\hbox{\tt \ \ \ \ \ \ \ \ \ \ }
\startm
\m{\vdash}\m{(}\m{\varphi}\m{\rightarrow}\m{(}\m{\psi}\m{\rightarrow}\m{\varphi}\m{)}
\m{)}
\endm

\needspace{3\baselineskip}
\noindent Axiom of Distribution.

\setbox\startprefix=\hbox{\tt \ \ ax-2\ \$a\ }
\setbox\contprefix=\hbox{\tt \ \ \ \ \ \ \ \ \ \ }
\startm
\m{\vdash}\m{(}\m{(}\m{\varphi}\m{\rightarrow}\m{(}\m{\psi}\m{\rightarrow}\m{\chi}
\m{)}\m{)}\m{\rightarrow}\m{(}\m{(}\m{\varphi}\m{\rightarrow}\m{\psi}\m{)}\m{
\rightarrow}\m{(}\m{\varphi}\m{\rightarrow}\m{\chi}\m{)}\m{)}\m{)}
\endm

\needspace{2\baselineskip}
\noindent Axiom of Contraposition.

\setbox\startprefix=\hbox{\tt \ \ ax-3\ \$a\ }
\setbox\contprefix=\hbox{\tt \ \ \ \ \ \ \ \ \ \ }
\startm
\m{\vdash}\m{(}\m{(}\m{\lnot}\m{\varphi}\m{\rightarrow}\m{\lnot}\m{\psi}\m{)}\m{
\rightarrow}\m{(}\m{\psi}\m{\rightarrow}\m{\varphi}\m{)}\m{)}
\endm


\needspace{4\baselineskip}
\noindent Rule of Modus Ponens.\label{axmp}\index{modus ponens}

\setbox\startprefix=\hbox{\tt \ \ min\ \$e\ }
\setbox\contprefix=\hbox{\tt \ \ \ \ \ \ \ \ \ }
\startm
\m{\vdash}\m{\varphi}
\endm

\setbox\startprefix=\hbox{\tt \ \ maj\ \$e\ }
\setbox\contprefix=\hbox{\tt \ \ \ \ \ \ \ \ \ }
\startm
\m{\vdash}\m{(}\m{\varphi}\m{\rightarrow}\m{\psi}\m{)}
\endm

\setbox\startprefix=\hbox{\tt \ \ ax-mp\ \$a\ }
\setbox\contprefix=\hbox{\tt \ \ \ \ \ \ \ \ \ \ \ }
\startm
\m{\vdash}\m{\psi}
\endm


\needspace{7\baselineskip}
\subsection{Axioms of Predicate Calculus with Equality---Tarski's S2}\index{axioms of predicate calculus}

\needspace{3\baselineskip}
\noindent Rule of Generalization.\index{rule of generalization}

\setbox\startprefix=\hbox{\tt \ \ ax-g.1\ \$e\ }
\setbox\contprefix=\hbox{\tt \ \ \ \ \ \ \ \ \ \ \ \ }
\startm
\m{\vdash}\m{\varphi}
\endm

\setbox\startprefix=\hbox{\tt \ \ ax-gen\ \$a\ }
\setbox\contprefix=\hbox{\tt \ \ \ \ \ \ \ \ \ \ \ \ }
\startm
\m{\vdash}\m{\forall}\m{x}\m{\varphi}
\endm

\needspace{2\baselineskip}
\noindent Axiom of Quantified Implication.

\setbox\startprefix=\hbox{\tt \ \ ax-4\ \$a\ }
\setbox\contprefix=\hbox{\tt \ \ \ \ \ \ \ \ \ \ }
\startm
\m{\vdash}\m{(}\m{\forall}\m{x}\m{(}\m{\forall}\m{x}\m{\varphi}\m{\rightarrow}\m{
\psi}\m{)}\m{\rightarrow}\m{(}\m{\forall}\m{x}\m{\varphi}\m{\rightarrow}\m{
\forall}\m{x}\m{\psi}\m{)}\m{)}
\endm

\needspace{3\baselineskip}
\noindent Axiom of Distinctness.

% Aka: Add $d x ph $.
\setbox\startprefix=\hbox{\tt \ \ ax-5\ \$a\ }
\setbox\contprefix=\hbox{\tt \ \ \ \ \ \ \ \ \ \ }
\startm
\m{\vdash}\m{(}\m{\varphi}\m{\rightarrow}\m{\forall}\m{x}\m{\varphi}\m{)}\m{where}\m{ }\m{\$d}\m{ }\m{x}\m{ }\m{\varphi}\m{ }\m{(}\m{x}\m{ }\m{does}\m{ }\m{not}\m{ }\m{occur}\m{ }\m{in}\m{ }\m{\varphi}\m{)}
\endm

\needspace{2\baselineskip}
\noindent Axiom of Existence.

\setbox\startprefix=\hbox{\tt \ \ ax-6\ \$a\ }
\setbox\contprefix=\hbox{\tt \ \ \ \ \ \ \ \ \ \ }
\startm
\m{\vdash}\m{(}\m{\forall}\m{x}\m{(}\m{x}\m{=}\m{y}\m{\rightarrow}\m{\forall}
\m{x}\m{\varphi}\m{)}\m{\rightarrow}\m{\varphi}\m{)}
\endm

\needspace{2\baselineskip}
\noindent Axiom of Equality.

\setbox\startprefix=\hbox{\tt \ \ ax-7\ \$a\ }
\setbox\contprefix=\hbox{\tt \ \ \ \ \ \ \ \ \ \ }
\startm
\m{\vdash}\m{(}\m{x}\m{=}\m{y}\m{\rightarrow}\m{(}\m{x}\m{=}\m{z}\m{
\rightarrow}\m{y}\m{=}\m{z}\m{)}\m{)}
\endm

\needspace{2\baselineskip}
\noindent Axiom of Left Equality for Binary Predicate.

\setbox\startprefix=\hbox{\tt \ \ ax-8\ \$a\ }
\setbox\contprefix=\hbox{\tt \ \ \ \ \ \ \ \ \ \ \ }
\startm
\m{\vdash}\m{(}\m{x}\m{=}\m{y}\m{\rightarrow}\m{(}\m{x}\m{\in}\m{z}\m{
\rightarrow}\m{y}\m{\in}\m{z}\m{)}\m{)}
\endm

\needspace{2\baselineskip}
\noindent Axiom of Right Equality for Binary Predicate.

\setbox\startprefix=\hbox{\tt \ \ ax-9\ \$a\ }
\setbox\contprefix=\hbox{\tt \ \ \ \ \ \ \ \ \ \ \ }
\startm
\m{\vdash}\m{(}\m{x}\m{=}\m{y}\m{\rightarrow}\m{(}\m{z}\m{\in}\m{x}\m{
\rightarrow}\m{z}\m{\in}\m{y}\m{)}\m{)}
\endm


\needspace{4\baselineskip}
\subsection{Axioms of Predicate Calculus with Equality---Auxiliary}\index{axioms of predicate calculus - auxiliary}

\needspace{2\baselineskip}
\noindent Axiom of Quantified Negation.

\setbox\startprefix=\hbox{\tt \ \ ax-10\ \$a\ }
\setbox\contprefix=\hbox{\tt \ \ \ \ \ \ \ \ \ \ }
\startm
\m{\vdash}\m{(}\m{\lnot}\m{\forall}\m{x}\m{\lnot}\m{\forall}\m{x}\m{\varphi}\m{
\rightarrow}\m{\varphi}\m{)}
\endm

\needspace{2\baselineskip}
\noindent Axiom of Quantifier Commutation.

\setbox\startprefix=\hbox{\tt \ \ ax-11\ \$a\ }
\setbox\contprefix=\hbox{\tt \ \ \ \ \ \ \ \ \ \ }
\startm
\m{\vdash}\m{(}\m{\forall}\m{x}\m{\forall}\m{y}\m{\varphi}\m{\rightarrow}\m{
\forall}\m{y}\m{\forall}\m{x}\m{\varphi}\m{)}
\endm

\needspace{3\baselineskip}
\noindent Axiom of Substitution.

\setbox\startprefix=\hbox{\tt \ \ ax-12\ \$a\ }
\setbox\contprefix=\hbox{\tt \ \ \ \ \ \ \ \ \ \ \ }
\startm
\m{\vdash}\m{(}\m{\lnot}\m{\forall}\m{x}\m{\,x}\m{=}\m{y}\m{\rightarrow}\m{(}
\m{x}\m{=}\m{y}\m{\rightarrow}\m{(}\m{\varphi}\m{\rightarrow}\m{\forall}\m{x}\m{(}
\m{x}\m{=}\m{y}\m{\rightarrow}\m{\varphi}\m{)}\m{)}\m{)}\m{)}
\endm

\needspace{3\baselineskip}
\noindent Axiom of Quantified Equality.

\setbox\startprefix=\hbox{\tt \ \ ax-13\ \$a\ }
\setbox\contprefix=\hbox{\tt \ \ \ \ \ \ \ \ \ \ \ }
\startm
\m{\vdash}\m{(}\m{\lnot}\m{\forall}\m{z}\m{\,z}\m{=}\m{x}\m{\rightarrow}\m{(}
\m{\lnot}\m{\forall}\m{z}\m{\,z}\m{=}\m{y}\m{\rightarrow}\m{(}\m{x}\m{=}\m{y}
\m{\rightarrow}\m{\forall}\m{z}\m{\,x}\m{=}\m{y}\m{)}\m{)}\m{)}
\endm

% \noindent Axiom of Quantifier Substitution
%
% \setbox\startprefix=\hbox{\tt \ \ ax-c11n\ \$a\ }
% \setbox\contprefix=\hbox{\tt \ \ \ \ \ \ \ \ \ \ \ }
% \startm
% \m{\vdash}\m{(}\m{\forall}\m{x}\m{\,x}\m{=}\m{y}\m{\rightarrow}\m{(}\m{\forall}
% \m{x}\m{\varphi}\m{\rightarrow}\m{\forall}\m{y}\m{\varphi}\m{)}\m{)}
% \endm
%
% \noindent Axiom of Distinct Variables. (This axiom requires
% that two individual variables
% be distinct\index{\texttt{\$d} statement}\index{distinct
% variables}.)
%
% \setbox\startprefix=\hbox{\tt \ \ \ \ \ \ \ \ \$d\ }
% \setbox\contprefix=\hbox{\tt \ \ \ \ \ \ \ \ \ \ \ }
% \startm
% \m{x}\m{\,}\m{y}
% \endm
%
% \setbox\startprefix=\hbox{\tt \ \ ax-c16\ \$a\ }
% \setbox\contprefix=\hbox{\tt \ \ \ \ \ \ \ \ \ \ \ }
% \startm
% \m{\vdash}\m{(}\m{\forall}\m{x}\m{\,x}\m{=}\m{y}\m{\rightarrow}\m{(}\m{\varphi}\m{
% \rightarrow}\m{\forall}\m{x}\m{\varphi}\m{)}\m{)}
% \endm

% \noindent Axiom of Quantifier Introduction (2).  (This axiom requires
% that the individual variable not occur in the
% wff\index{\texttt{\$d} statement}\index{distinct variables}.)
%
% \setbox\startprefix=\hbox{\tt \ \ \ \ \ \ \ \ \$d\ }
% \setbox\contprefix=\hbox{\tt \ \ \ \ \ \ \ \ \ \ \ }
% \startm
% \m{x}\m{\,}\m{\varphi}
% \endm
% \setbox\startprefix=\hbox{\tt \ \ ax-5\ \$a\ }
% \setbox\contprefix=\hbox{\tt \ \ \ \ \ \ \ \ \ \ \ }
% \startm
% \m{\vdash}\m{(}\m{\varphi}\m{\rightarrow}\m{\forall}\m{x}\m{\varphi}\m{)}
% \endm

\subsection{Set Theory}\label{mmsettheoryaxioms}

In order to make the axioms of set theory\index{axioms of set theory} a little
more compact, there are several definitions from logic that we make use of
implicitly, namely, ``logical {\sc and},''\index{conjunction ($\wedge$)}
\index{logical {\sc and} ($\wedge$)} ``logical equivalence,''\index{logical
equivalence ($\leftrightarrow$)}\index{biconditional ($\leftrightarrow$)} and
``there exists.''\index{existential quantifier ($\exists$)}

\begin{center}\begin{tabular}{rcl}
  $( \varphi \wedge \psi )$ &\mbox{stands for}& $\neg ( \varphi
     \rightarrow \neg \psi )$\\
  $( \varphi \leftrightarrow \psi )$& \mbox{stands
     for}& $( ( \varphi \rightarrow \psi ) \wedge
     ( \psi \rightarrow \varphi ) )$\\
  $\exists x \,\varphi$ &\mbox{stands for}& $\neg \forall x \neg \varphi$
\end{tabular}\end{center}

In addition, the axioms of set theory require that all variables be
dis\-tinct,\index{distinct variables}\footnote{Set theory axioms can be
devised so that {\em no} variables are required to be distinct,
provided we replace \texttt{ax-c16} with an axiom stating that ``at
least two things exist,'' thus
making \texttt{ax-5} the only other axiom requiring the
\texttt{\$d} statement.  These axioms are unconventional and are not
presented here, but they can be found on the \url{http://metamath.org}
web site.  See also the Comment on
p.~\pageref{nodd}.}\index{\texttt{\$d} statement} thus we also assume:
\begin{center}
  \texttt{\$d }$x\,y\,z\,w$
\end{center}

\needspace{2\baselineskip}
\noindent Axiom of Extensionality.\index{Axiom of Extensionality}

\setbox\startprefix=\hbox{\tt \ \ ax-ext\ \$a\ }
\setbox\contprefix=\hbox{\tt \ \ \ \ \ \ \ \ \ \ \ \ }
\startm
\m{\vdash}\m{(}\m{\forall}\m{x}\m{(}\m{x}\m{\in}\m{y}\m{\leftrightarrow}\m{x}
\m{\in}\m{z}\m{)}\m{\rightarrow}\m{y}\m{=}\m{z}\m{)}
\endm

\needspace{3\baselineskip}
\noindent Axiom of Replacement.\index{Axiom of Replacement}

\setbox\startprefix=\hbox{\tt \ \ ax-rep\ \$a\ }
\setbox\contprefix=\hbox{\tt \ \ \ \ \ \ \ \ \ \ \ \ }
\startm
\m{\vdash}\m{(}\m{\forall}\m{w}\m{\exists}\m{y}\m{\forall}\m{z}\m{(}\m{%
\forall}\m{y}\m{\varphi}\m{\rightarrow}\m{z}\m{=}\m{y}\m{)}\m{\rightarrow}\m{%
\exists}\m{y}\m{\forall}\m{z}\m{(}\m{z}\m{\in}\m{y}\m{\leftrightarrow}\m{%
\exists}\m{w}\m{(}\m{w}\m{\in}\m{x}\m{\wedge}\m{\forall}\m{y}\m{\varphi}\m{)}%
\m{)}\m{)}
\endm

\needspace{2\baselineskip}
\noindent Axiom of Union.\index{Axiom of Union}

\setbox\startprefix=\hbox{\tt \ \ ax-un\ \$a\ }
\setbox\contprefix=\hbox{\tt \ \ \ \ \ \ \ \ \ \ \ }
\startm
\m{\vdash}\m{\exists}\m{x}\m{\forall}\m{y}\m{(}\m{\exists}\m{x}\m{(}\m{y}\m{
\in}\m{x}\m{\wedge}\m{x}\m{\in}\m{z}\m{)}\m{\rightarrow}\m{y}\m{\in}\m{x}\m{)}
\endm

\needspace{2\baselineskip}
\noindent Axiom of Power Sets.\index{Axiom of Power Sets}

\setbox\startprefix=\hbox{\tt \ \ ax-pow\ \$a\ }
\setbox\contprefix=\hbox{\tt \ \ \ \ \ \ \ \ \ \ \ \ }
\startm
\m{\vdash}\m{\exists}\m{x}\m{\forall}\m{y}\m{(}\m{\forall}\m{x}\m{(}\m{x}\m{
\in}\m{y}\m{\rightarrow}\m{x}\m{\in}\m{z}\m{)}\m{\rightarrow}\m{y}\m{\in}\m{x}
\m{)}
\endm

\needspace{3\baselineskip}
\noindent Axiom of Regularity.\index{Axiom of Regularity}

\setbox\startprefix=\hbox{\tt \ \ ax-reg\ \$a\ }
\setbox\contprefix=\hbox{\tt \ \ \ \ \ \ \ \ \ \ \ \ }
\startm
\m{\vdash}\m{(}\m{\exists}\m{x}\m{\,x}\m{\in}\m{y}\m{\rightarrow}\m{\exists}
\m{x}\m{(}\m{x}\m{\in}\m{y}\m{\wedge}\m{\forall}\m{z}\m{(}\m{z}\m{\in}\m{x}\m{
\rightarrow}\m{\lnot}\m{z}\m{\in}\m{y}\m{)}\m{)}\m{)}
\endm

\needspace{3\baselineskip}
\noindent Axiom of Infinity.\index{Axiom of Infinity}

\setbox\startprefix=\hbox{\tt \ \ ax-inf\ \$a\ }
\setbox\contprefix=\hbox{\tt \ \ \ \ \ \ \ \ \ \ \ \ \ \ \ }
\startm
\m{\vdash}\m{\exists}\m{x}\m{(}\m{y}\m{\in}\m{x}\m{\wedge}\m{\forall}\m{y}%
\m{(}\m{y}\m{\in}\m{x}\m{\rightarrow}\m{\exists}\m{z}\m{(}\m{y}\m{\in}\m{z}\m{%
\wedge}\m{z}\m{\in}\m{x}\m{)}\m{)}\m{)}
\endm

\needspace{4\baselineskip}
\noindent Axiom of Choice.\index{Axiom of Choice}

\setbox\startprefix=\hbox{\tt \ \ ax-ac\ \$a\ }
\setbox\contprefix=\hbox{\tt \ \ \ \ \ \ \ \ \ \ \ \ \ \ }
\startm
\m{\vdash}\m{\exists}\m{x}\m{\forall}\m{y}\m{\forall}\m{z}\m{(}\m{(}\m{y}\m{%
\in}\m{z}\m{\wedge}\m{z}\m{\in}\m{w}\m{)}\m{\rightarrow}\m{\exists}\m{w}\m{%
\forall}\m{y}\m{(}\m{\exists}\m{w}\m{(}\m{(}\m{y}\m{\in}\m{z}\m{\wedge}\m{z}%
\m{\in}\m{w}\m{)}\m{\wedge}\m{(}\m{y}\m{\in}\m{w}\m{\wedge}\m{w}\m{\in}\m{x}%
\m{)}\m{)}\m{\leftrightarrow}\m{y}\m{=}\m{w}\m{)}\m{)}
\endm

\subsection{That's It}

There you have it, the axioms for (essentially) all of mathematics!
Wonder at them and stare at them in awe.  Put a copy in your wallet, and
you will carry in your pocket the encoding for all theorems ever proved
and that ever will be proved, from the most mundane to the most
profound.

\section{A Hierarchy of Definitions}\label{hierarchy}

The axioms in the previous section in principle embody everything that can be
done within standard mathematics.  However, it is impractical to accomplish
very much by using them directly, for even simple concepts (from a human
perspective) can involve extremely long, incomprehensible formulas.
Mathematics is made practical by introducing definitions\index{definition}.
Definitions usually introduce new symbols, or at least new relationships among
existing symbols, to abbreviate more complex formulas.  An important
requirement for a definition is that there exist a straightforward
(algorithmic) method for eliminating the abbreviation by expanding it into the
more primitive symbol string that it represents.  Some
important definitions included in
the file \texttt{set.mm} are listed in this section for reference, and also to
give you a feel for why something like $\omega$\index{omega ($\omega$)} (the
set of natural numbers\index{natural number} 0, 1, 2,\ldots) becomes very
complicated when completely expanded into primitive symbols.

What is the motivation for definitions, aside from allowing complicated
expressions to be expressed more simply?  In the case of  $\omega$, one goal is
to provide a basis for the theory of natural numbers.\index{natural number}
Before set theory was invented, a set of axioms for arithmetic, called Peano's
postulates\index{Peano's postulates}, was devised and shown to have the
properties one expects for natural numbers.  Now anyone can postulate a
set of axioms, but if the axioms are inconsistent contradictions can be derived
from them.  Once a contradiction is derived, anything can be trivially
proved, including
all the facts of arithmetic and their negations.  To ensure that an
axiom system is at least as reliable as the axioms for set theory, we can
define sets and operations on those sets that satisfy the new axioms. In the
\texttt{set.mm} Metamath database, we prove that the elements of $\omega$ satisfy
Peano's postulates, and it's a long and hard journey to get there directly
from the axioms of set theory.  But the result is confidence in the
foundations of arithmetic.  And there is another advantage:  we now have all
the tools of set theory at our disposal for manipulating objects that obey the
axioms for arithmetic.

What are the criteria we use for definitions?  First, and of utmost importance,
the definition should not be {\em creative}\index{creative
definition}\index{definition!creative}, that
is it should not allow an expression that previously qualified as a wff but
was not provable, to become provable.   Second, the definition should be {\em
eliminable}\index{definition!eliminability}, that is, there should exist an
algorithmic method for converting any expression using the definition into
a logically equivalent expression that previously qualified as a wff.

In almost all cases below, definitions connect two expressions with either
$\leftrightarrow$ or $=$.  Eliminating\footnote{Here we mean the
elimination that a human might do in his or her head.  To eliminate them as
part of a Metamath proof we would invoke one of a number of
theorems that deal with transitivity of equivalence or equality; there are
many such examples in the proofs in \texttt{set.mm}.} such a definition is a
simple matter of substituting the expression on the left-hand side ({\em
definiendum}\index{definiendum} or thing being defined) with the equivalent,
more primitive expression on the right-hand side ({\em
definiens}\index{definiens} or definition).

Often a definition has variables on the right-hand side which do not appear on
the left-hand side; these are called {\em dummy variables}.\index{dummy
variable!in definitions}  In this case, any
allowable substitution (such as a new, distinct
variable) can be used when the definition is eliminated.  Dummy variables may
be used only if they are {\em effectively bound}\index{effectively bound
variable}, meaning that the definition will remain logically equivalent upon
any substitution of a dummy variable with any other {\em qualifying
expression}\index{qualifying expression}, i.e.\ any symbol string (such as
another variable) that
meets the restrictions on the dummy variable imposed by \texttt{\$d} and
\texttt{\$f} statements.  For example, we could define a constant $\perp$
(inverted tee, meaning logical ``false'') as $( \varphi \wedge \lnot \varphi
)$, i.e.\ ``phi and not phi.''  Here $\varphi$ is effectively bound because the
definition remains logically equivalent when we replace $\varphi$ with any
other wff.  (It is actually \texttt{df-fal}
in \texttt{set.mm}, which defines $\perp$.)

There are two cases where eliminating definitions is a little more
complex.  These cases are the definitions \texttt{df-bi} and
\texttt{df-cleq}.  The first stretches the concept of a definition a
little, as in effect it ``defines a definition;'' however, it meets our
requirements for a definition in that it is eliminable and does not
strengthen the language.  Theorem \texttt{bii} shows the substitution
needed to eliminate the $\leftrightarrow$\index{logical equivalence
($\leftrightarrow$)}\index{biconditional ($\leftrightarrow$)} symbol.

Definition \texttt{df-cleq}\index{equality ($=$)} extends the usage of
the equality symbol to include ``classes''\index{class} in set theory.  The
reason it is potentially problematic is that it can lead to statements which
do not follow from logic alone but presuppose the Axiom of
Extensionality\index{Axiom of Extensionality}, so we include this axiom
as a hypothesis for the definition.  We could have made \texttt{df-cleq} directly
eliminable by introducing a new equality symbol, but have chosen not to do so
in keeping with standard textbook practice.  Definitions such as \texttt{df-cleq}
that extend the meaning of existing symbols must be introduced carefully so
that they do not lead to contradictions.  Definition \texttt{df-clel} also
extends the meaning of an existing symbol ($\in$); while it doesn't strengthen
the language like \texttt{df-cleq}, this is not obvious and it must also be
subject to the same scrutiny.

Exercise:  Study how the wff $x\in\omega$, meaning ``$x$ is a natural
number,'' could be expanded in terms of primitive symbols, starting with the
definitions \texttt{df-clel} on p.~\pageref{dfclel} and \texttt{df-om} on
p.~\pageref{dfom} and working your way back.  Don't bother to work out the
details; just make sure that you understand how you could do it in principle.
The answer is shown in the footnote on p.~\pageref{expandom}.  If you
actually do work it out, you won't get exactly the same answer because we used
a few simplifications such as discarding occurrences of $\lnot\lnot$ (double
negation).

In the definitions below, we have placed the {\sc ascii} Metamath source
below each of the formulas to help you become familiar with the
notation in the database.  For simplicity, the necessary \texttt{\$f}
and \texttt{\$d} statements are not shown.  If you are in doubt, use the
\texttt{show statement}\index{\texttt{show statement} command} command
in the Metamath program to see the full statement.
A selection of this notation is summarized in Appendix~\ref{ASCII}.

To understand the motivation for these definitions, you should consult the
references indicated:  Takeuti and Zaring \cite{Takeuti}\index{Takeuti, Gaisi},
Quine \cite{Quine}\index{Quine, Willard Van Orman}, Bell and Machover
\cite{Bell}\index{Bell, J. L.}, and Enderton \cite{Enderton}\index{Enderton,
Herbert B.}.  Our list of definitions is provided more for reference than as a
learning aid.  However, by looking at a few of them you can gain a feel for
how the hierarchy is built up.  The definitions are a representative sample of
the many definitions
in \texttt{set.mm}, but they are complete with respect to the
theorem examples we will present in Section~\ref{sometheorems}.  Also, some are
slightly different from, but logically equivalent to, the ones in \texttt{set.mm}
(some of which have been revised over time to shorten them, for example).

\subsection{Definitions for Propositional Calculus}\label{metadefprop}

The symbols $\varphi$, $\psi$, and $\chi$ represent wffs.

Our first definition introduces the biconditional
connective\footnote{The term ``connective'' is informally used to mean a
symbol that is placed between two variables or adjacent to a variable,
whereas a mathematical ``constant'' usually indicates a symbol such as
the number 0 that may replace a variable or metavariable.  From
Metamath's point of view, there is no distinction between a connective
and a constant; both are constants in the Metamath
language.}\index{connective}\index{constant} (also called logical
equivalence)\index{logical equivalence
($\leftrightarrow$)}\index{biconditional ($\leftrightarrow$)}.  Unlike
most traditional developments, we have chosen not to have a separate
symbol such as ``Df.'' to mean ``is defined as.''  Instead, we will use
the biconditional connective for this purpose, as it lets us use
logic to manipulate definitions directly.  Here we state the properties
of the biconditional connective with a carefully crafted \texttt{\$a}
statement, which effectively uses the biconditional connective to define
itself.  The $\leftrightarrow$ symbol can be eliminated from a formula
using theorem \texttt{bii}, which is derived later.

\vskip 2ex
\noindent Define the biconditional connective.\label{df-bi}

\vskip 0.5ex
\setbox\startprefix=\hbox{\tt \ \ df-bi\ \$a\ }
\setbox\contprefix=\hbox{\tt \ \ \ \ \ \ \ \ \ \ \ }
\startm
\m{\vdash}\m{\lnot}\m{(}\m{(}\m{(}\m{\varphi}\m{\leftrightarrow}\m{\psi}\m{)}%
\m{\rightarrow}\m{\lnot}\m{(}\m{(}\m{\varphi}\m{\rightarrow}\m{\psi}\m{)}\m{%
\rightarrow}\m{\lnot}\m{(}\m{\psi}\m{\rightarrow}\m{\varphi}\m{)}\m{)}\m{)}\m{%
\rightarrow}\m{\lnot}\m{(}\m{\lnot}\m{(}\m{(}\m{\varphi}\m{\rightarrow}\m{%
\psi}\m{)}\m{\rightarrow}\m{\lnot}\m{(}\m{\psi}\m{\rightarrow}\m{\varphi}\m{)}%
\m{)}\m{\rightarrow}\m{(}\m{\varphi}\m{\leftrightarrow}\m{\psi}\m{)}\m{)}\m{)}
\endm
\begin{mmraw}%
|- -. ( ( ( ph <-> ps ) -> -. ( ( ph -> ps ) ->
-. ( ps -> ph ) ) ) -> -. ( -. ( ( ph -> ps ) -> -. (
ps -> ph ) ) -> ( ph <-> ps ) ) ) \$.
\end{mmraw}

\noindent This theorem relates the biconditional connective to primitive
connectives and can be used to eliminate the $\leftrightarrow$ symbol from any
wff.

\vskip 0.5ex
\setbox\startprefix=\hbox{\tt \ \ bii\ \$p\ }
\setbox\contprefix=\hbox{\tt \ \ \ \ \ \ \ \ \ }
\startm
\m{\vdash}\m{(}\m{(}\m{\varphi}\m{\leftrightarrow}\m{\psi}\m{)}\m{\leftrightarrow}
\m{\lnot}\m{(}\m{(}\m{\varphi}\m{\rightarrow}\m{\psi}\m{)}\m{\rightarrow}\m{\lnot}
\m{(}\m{\psi}\m{\rightarrow}\m{\varphi}\m{)}\m{)}\m{)}
\endm
\begin{mmraw}%
|- ( ( ph <-> ps ) <-> -. ( ( ph -> ps ) -> -. ( ps -> ph ) ) ) \$= ... \$.
\end{mmraw}

\noindent Define disjunction ({\sc or}).\index{disjunction ($\vee$)}%
\index{logical or (vee)@logical {\sc or} ($\vee$)}%
\index{df-or@\texttt{df-or}}\label{df-or}

\vskip 0.5ex
\setbox\startprefix=\hbox{\tt \ \ df-or\ \$a\ }
\setbox\contprefix=\hbox{\tt \ \ \ \ \ \ \ \ \ \ \ }
\startm
\m{\vdash}\m{(}\m{(}\m{\varphi}\m{\vee}\m{\psi}\m{)}\m{\leftrightarrow}\m{(}\m{
\lnot}\m{\varphi}\m{\rightarrow}\m{\psi}\m{)}\m{)}
\endm
\begin{mmraw}%
|- ( ( ph \TOR ps ) <-> ( -. ph -> ps ) ) \$.
\end{mmraw}

\noindent Define conjunction ({\sc and}).\index{conjunction ($\wedge$)}%
\index{logical {\sc and} ($\wedge$)}%
\index{df-an@\texttt{df-an}}\label{df-an}

\vskip 0.5ex
\setbox\startprefix=\hbox{\tt \ \ df-an\ \$a\ }
\setbox\contprefix=\hbox{\tt \ \ \ \ \ \ \ \ \ \ \ }
\startm
\m{\vdash}\m{(}\m{(}\m{\varphi}\m{\wedge}\m{\psi}\m{)}\m{\leftrightarrow}\m{\lnot}
\m{(}\m{\varphi}\m{\rightarrow}\m{\lnot}\m{\psi}\m{)}\m{)}
\endm
\begin{mmraw}%
|- ( ( ph \TAND ps ) <-> -. ( ph -> -. ps ) ) \$.
\end{mmraw}

\noindent Define disjunction ({\sc or}) of 3 wffs.%
\index{df-3or@\texttt{df-3or}}\label{df-3or}

\vskip 0.5ex
\setbox\startprefix=\hbox{\tt \ \ df-3or\ \$a\ }
\setbox\contprefix=\hbox{\tt \ \ \ \ \ \ \ \ \ \ \ \ }
\startm
\m{\vdash}\m{(}\m{(}\m{\varphi}\m{\vee}\m{\psi}\m{\vee}\m{\chi}\m{)}\m{
\leftrightarrow}\m{(}\m{(}\m{\varphi}\m{\vee}\m{\psi}\m{)}\m{\vee}\m{\chi}\m{)}
\m{)}
\endm
\begin{mmraw}%
|- ( ( ph \TOR ps \TOR ch ) <-> ( ( ph \TOR ps ) \TOR ch ) ) \$.
\end{mmraw}

\noindent Define conjunction ({\sc and}) of 3 wffs.%
\index{df-3an}\label{df-3an}

\vskip 0.5ex
\setbox\startprefix=\hbox{\tt \ \ df-3an\ \$a\ }
\setbox\contprefix=\hbox{\tt \ \ \ \ \ \ \ \ \ \ \ \ }
\startm
\m{\vdash}\m{(}\m{(}\m{\varphi}\m{\wedge}\m{\psi}\m{\wedge}\m{\chi}\m{)}\m{
\leftrightarrow}\m{(}\m{(}\m{\varphi}\m{\wedge}\m{\psi}\m{)}\m{\wedge}\m{\chi}
\m{)}\m{)}
\endm

\begin{mmraw}%
|- ( ( ph \TAND ps \TAND ch ) <-> ( ( ph \TAND ps ) \TAND ch ) ) \$.
\end{mmraw}

\subsection{Definitions for Predicate Calculus}\label{metadefpred}

The symbols $x$, $y$, and $z$ represent individual variables of predicate
calculus.  In this section, they are not necessarily distinct unless it is
explicitly
mentioned.

\vskip 2ex
\noindent Define existential quantification.
The expression $\exists x \varphi$ means
``there exists an $x$ where $\varphi$ is true.''\index{existential quantifier
($\exists$)}\label{df-ex}

\vskip 0.5ex
\setbox\startprefix=\hbox{\tt \ \ df-ex\ \$a\ }
\setbox\contprefix=\hbox{\tt \ \ \ \ \ \ \ \ \ \ \ }
\startm
\m{\vdash}\m{(}\m{\exists}\m{x}\m{\varphi}\m{\leftrightarrow}\m{\lnot}\m{\forall}
\m{x}\m{\lnot}\m{\varphi}\m{)}
\endm
\begin{mmraw}%
|- ( E. x ph <-> -. A. x -. ph ) \$.
\end{mmraw}

\noindent Define proper substitution.\index{proper
substitution}\index{substitution!proper}\label{df-sb}
In our notation, we use $[ y / x ] \varphi$ to mean ``the wff that
results when $y$ is properly substituted for $x$ in the wff
$\varphi$.''\footnote{
This can also be described
as substituting $x$ with $y$, $y$ properly replaces $x$, or
$x$ is properly replaced by $y$.}
% This is elsb4, though it currently says: ( [ x / y ] z e. y <-> z e. x )
For example,
$[ y / x ] z \in x$ is the same as $z \in y$.
One way to remember this notation is to notice that it looks like division
and recall that $( y / x ) \cdot x $ is $y$ (when $x \neq 0$).
The notation is different from the notation $\varphi ( x | y )$
that is sometimes used, because the latter notation is ambiguous for us:
for example, we don't know whether $\lnot \varphi ( x | y )$ is to be
interpreted as $\lnot ( \varphi ( x | y ) )$ or
$( \lnot \varphi ) ( x | y )$.\footnote{Because of the way
we initially defined wffs, this is the case
with any postfix connective\index{postfix connective} (one occurring after the
symbols being connected) or infix connective\index{infix connective} (one
occurring between the symbols being connected).  Metamath does not have a
built-in notion of operator binding strength that could eliminate the
ambiguity.  The initial parenthesis effectively provides a prefix
connective\index{prefix connective} to eliminate ambiguity.  Some conventions,
such as Polish notation\index{Polish notation} used in the 1930's and 1940's
by Polish logicians, use only prefix connectives and thus allow the total
elimination of parentheses, at the expense of readability.  In Metamath we
could actually redefine all notation to be Polish if we wanted to without
having to change any proofs!}  Other texts often use $\varphi(y)$ to indicate
our $[ y / x ] \varphi$, but this notation is even more ambiguous since there is
no explicit indication of what is being substituted.
Note that this
definition is valid even when
$x$ and $y$ are the same variable.  The first conjunct is a ``trick'' used to
achieve this property, making the definition look somewhat peculiar at
first.

\vskip 0.5ex
\setbox\startprefix=\hbox{\tt \ \ df-sb\ \$a\ }
\setbox\contprefix=\hbox{\tt \ \ \ \ \ \ \ \ \ \ \ }
\startm
\m{\vdash}\m{(}\m{[}\m{y}\m{/}\m{x}\m{]}\m{\varphi}\m{\leftrightarrow}\m{(}%
\m{(}\m{x}\m{=}\m{y}\m{\rightarrow}\m{\varphi}\m{)}\m{\wedge}\m{\exists}\m{x}%
\m{(}\m{x}\m{=}\m{y}\m{\wedge}\m{\varphi}\m{)}\m{)}\m{)}
\endm
\begin{mmraw}%
|- ( [ y / x ] ph <-> ( ( x = y -> ph ) \TAND E. x ( x = y \TAND ph ) ) ) \$.
\end{mmraw}


\noindent Define existential uniqueness\index{existential uniqueness
quantifier ($\exists "!$)} (``there exists exactly one'').  Note that $y$ is a
variable distinct from $x$ and not occurring in $\varphi$.\label{df-eu}

\vskip 0.5ex
\setbox\startprefix=\hbox{\tt \ \ df-eu\ \$a\ }
\setbox\contprefix=\hbox{\tt \ \ \ \ \ \ \ \ \ \ \ }
\startm
\m{\vdash}\m{(}\m{\exists}\m{{!}}\m{x}\m{\varphi}\m{\leftrightarrow}\m{\exists}
\m{y}\m{\forall}\m{x}\m{(}\m{\varphi}\m{\leftrightarrow}\m{x}\m{=}\m{y}\m{)}\m{)}
\endm

\begin{mmraw}%
|- ( E! x ph <-> E. y A. x ( ph <-> x = y ) ) \$.
\end{mmraw}

\subsection{Definitions for Set Theory}\label{setdefinitions}

The symbols $x$, $y$, $z$, and $w$ represent individual variables of
predicate calculus, which in set theory are understood to be sets.
However, using only the constructs shown so far would be very inconvenient.

To make set theory more practical, we introduce the notion of a ``class.''
A class\index{class} is either a set variable (such as $x$) or an
expression of the form $\{ x | \varphi\}$ (called an ``abstraction
class''\index{abstraction class}\index{class abstraction}).  Note that
sets (i.e.\ individual variables) always exist (this is a theorem of
logic, namely $\exists y \, y = x$ for any set $x$), whereas classes may
or may not exist (i.e.\ $\exists y \, y = A$ may or may not be true).
If a class does not exist it is called a ``proper class.''\index{proper
class}\index{class!proper} Definitions \texttt{df-clab},
\texttt{df-cleq}, and \texttt{df-clel} can be used to convert an
expression containing classes into one containing only set variables and
wff metavariables.

The symbols $A$, $B$, $C$, $D$, $ F$, $G$, and $R$ are metavariables that range
over classes.  A class metavariable $A$ may be eliminated from a wff by
replacing it with $\{ x|\varphi\}$ where neither $x$ nor $\varphi$ occur in
the wff.

The theory of classes can be shown to be an eliminable and conservative
extension of set theory. The \textbf{eliminability}
property shows that for every
formula in the extended language we can build a logically equivalent
formula in the basic language; so that even if the extended language
provides more ease to convey and formulate mathematical ideas for set
theory, its expressive power does not in fact strengthen the basic
language's expressive power.
The \textbf{conservation} property shows that for
every proof of a formula of the basic language in the extended system
we can build another proof of the same formula in the basic system;
so that, concerning theorems on sets only, the deductive powers of
the extended system and of the basic system are identical. Together,
these properties mean that the extended language can be treated as a
definitional extension that is \textbf{sound}.

A rigorous justification, which we will not give here, can be found in
Levy \cite[pp.~357-366]{Levy} supplementing his informal introduction to class
theory on pp.~7-17. Two other good treatments of class theory are provided
by Quine \cite[pp.~15-21]{Quine}\index{Quine, Willard Van Orman}
and also \cite[pp.~10-14]{Takeuti}\index{Takeuti, Gaisi}.
Quine's exposition (he calls them virtual classes)
is nicely written and very readable.

In the rest of this
section, individual variables are always assumed to be distinct from
each other unless otherwise indicated.  In addition, dummy variables on the
right-hand side of a definition do not occur in the class and wff
metavariables in the definition.

The definitions we present here are a partial but self-contained
collection selected from several hundred that appear in the current
\texttt{set.mm} database.  They are adequate for a basic development of
elementary set theory.

\vskip 2ex
\noindent Define the abstraction class.\index{abstraction class}\index{class
abstraction}\label{df-clab}  $x$ and $y$
need not be distinct.  Definition 2.1 of Quine, p.~16.  This definition may
seem puzzling since it is shorter than the expression being defined and does not
buy us anything in terms of brevity.  The reason we introduce this definition
is because it fits in neatly with the extension of the $\in$ connective
provided by \texttt{df-clel}.

\vskip 0.5ex
\setbox\startprefix=\hbox{\tt \ \ df-clab\ \$a\ }
\setbox\contprefix=\hbox{\tt \ \ \ \ \ \ \ \ \ \ \ \ \ }
\startm
\m{\vdash}\m{(}\m{x}\m{\in}\m{\{}\m{y}\m{|}\m{\varphi}\m{\}}\m{%
\leftrightarrow}\m{[}\m{x}\m{/}\m{y}\m{]}\m{\varphi}\m{)}
\endm
\begin{mmraw}%
|- ( x e. \{ y | ph \} <-> [ x / y ] ph ) \$.
\end{mmraw}

\noindent Define the equality connective between classes\index{class
equality}\label{df-cleq}.  See Quine or Chapter 4 of Takeuti and Zaring for its
justification and methods for eliminating it.  This is an example of a
somewhat ``dangerous'' definition, because it extends the use of the
existing equality symbol rather than introducing a new symbol, allowing
us to make statements in the original language that may not be true.
For example, it permits us to deduce $y = z \leftrightarrow \forall x (
x \in y \leftrightarrow x \in z )$ which is not a theorem of logic but
rather presupposes the Axiom of Extensionality,\index{Axiom of
Extensionality} which we include as a hypothesis so that we can know
when this axiom is assumed in a proof (with the \texttt{show
trace{\char`\_}back} command).  We could avoid the danger by introducing
another symbol, say $\eqcirc$, in place of $=$; this
would also have the advantage of making elimination of the definition
straightforward and would eliminate the need for Extensionality as a
hypothesis.  We would then also have the advantage of being able to
identify exactly where Extensionality truly comes into play.  One of our
theorems would be $x \eqcirc y \leftrightarrow x = y$
by invoking Extensionality.  However in keeping with standard practice
we retain the ``dangerous'' definition.

\vskip 0.5ex
\setbox\startprefix=\hbox{\tt \ \ df-cleq.1\ \$e\ }
\setbox\contprefix=\hbox{\tt \ \ \ \ \ \ \ \ \ \ \ \ \ \ \ }
\startm
\m{\vdash}\m{(}\m{\forall}\m{x}\m{(}\m{x}\m{\in}\m{y}\m{\leftrightarrow}\m{x}
\m{\in}\m{z}\m{)}\m{\rightarrow}\m{y}\m{=}\m{z}\m{)}
\endm
\setbox\startprefix=\hbox{\tt \ \ df-cleq\ \$a\ }
\setbox\contprefix=\hbox{\tt \ \ \ \ \ \ \ \ \ \ \ \ \ }
\startm
\m{\vdash}\m{(}\m{A}\m{=}\m{B}\m{\leftrightarrow}\m{\forall}\m{x}\m{(}\m{x}\m{
\in}\m{A}\m{\leftrightarrow}\m{x}\m{\in}\m{B}\m{)}\m{)}
\endm
% We need to reset the startprefix and contprefix.
\setbox\startprefix=\hbox{\tt \ \ df-cleq.1\ \$e\ }
\setbox\contprefix=\hbox{\tt \ \ \ \ \ \ \ \ \ \ \ \ \ \ \ }
\begin{mmraw}%
|- ( A. x ( x e. y <-> x e. z ) -> y = z ) \$.
\end{mmraw}
\setbox\startprefix=\hbox{\tt \ \ df-cleq\ \$a\ }
\setbox\contprefix=\hbox{\tt \ \ \ \ \ \ \ \ \ \ \ \ \ }
\begin{mmraw}%
|- ( A = B <-> A. x ( x e. A <-> x e. B ) ) \$.
\end{mmraw}

\noindent Define the membership connective between classes\index{class
membership}.  Theorem 6.3 of Quine, p.~41, which we adopt as a definition.
Note that it extends the use of the existing membership symbol, but unlike
{\tt df-cleq} it does not extend the set of valid wffs of logic when the class
metavariables are replaced with set variables.\label{dfclel}\label{df-clel}

\vskip 0.5ex
\setbox\startprefix=\hbox{\tt \ \ df-clel\ \$a\ }
\setbox\contprefix=\hbox{\tt \ \ \ \ \ \ \ \ \ \ \ \ \ }
\startm
\m{\vdash}\m{(}\m{A}\m{\in}\m{B}\m{\leftrightarrow}\m{\exists}\m{x}\m{(}\m{x}
\m{=}\m{A}\m{\wedge}\m{x}\m{\in}\m{B}\m{)}\m{)}
\endm
\begin{mmraw}%
|- ( A e. B <-> E. x ( x = A \TAND x e. B ) ) \$.?
\end{mmraw}

\noindent Define inequality.

\vskip 0.5ex
\setbox\startprefix=\hbox{\tt \ \ df-ne\ \$a\ }
\setbox\contprefix=\hbox{\tt \ \ \ \ \ \ \ \ \ \ \ }
\startm
\m{\vdash}\m{(}\m{A}\m{\ne}\m{B}\m{\leftrightarrow}\m{\lnot}\m{A}\m{=}\m{B}%
\m{)}
\endm
\begin{mmraw}%
|- ( A =/= B <-> -. A = B ) \$.
\end{mmraw}

\noindent Define restricted universal quantification.\index{universal
quantifier ($\forall$)!restricted}  Enderton, p.~22.

\vskip 0.5ex
\setbox\startprefix=\hbox{\tt \ \ df-ral\ \$a\ }
\setbox\contprefix=\hbox{\tt \ \ \ \ \ \ \ \ \ \ \ \ }
\startm
\m{\vdash}\m{(}\m{\forall}\m{x}\m{\in}\m{A}\m{\varphi}\m{\leftrightarrow}\m{%
\forall}\m{x}\m{(}\m{x}\m{\in}\m{A}\m{\rightarrow}\m{\varphi}\m{)}\m{)}
\endm
\begin{mmraw}%
|- ( A. x e. A ph <-> A. x ( x e. A -> ph ) ) \$.
\end{mmraw}

\noindent Define restricted existential quantification.\index{existential
quantifier ($\exists$)!restricted}  Enderton, p.~22.

\vskip 0.5ex
\setbox\startprefix=\hbox{\tt \ \ df-rex\ \$a\ }
\setbox\contprefix=\hbox{\tt \ \ \ \ \ \ \ \ \ \ \ \ }
\startm
\m{\vdash}\m{(}\m{\exists}\m{x}\m{\in}\m{A}\m{\varphi}\m{\leftrightarrow}\m{%
\exists}\m{x}\m{(}\m{x}\m{\in}\m{A}\m{\wedge}\m{\varphi}\m{)}\m{)}
\endm
\begin{mmraw}%
|- ( E. x e. A ph <-> E. x ( x e. A \TAND ph ) ) \$.
\end{mmraw}

\noindent Define the universal class\index{universal class ($V$)}.  Definition
5.20, p.~21, of Takeuti and Zaring.\label{df-v}

\vskip 0.5ex
\setbox\startprefix=\hbox{\tt \ \ df-v\ \$a\ }
\setbox\contprefix=\hbox{\tt \ \ \ \ \ \ \ \ \ \ }
\startm
\m{\vdash}\m{{\rm V}}\m{=}\m{\{}\m{x}\m{|}\m{x}\m{=}\m{x}\m{\}}
\endm
\begin{mmraw}%
|- {\char`\_}V = \{ x | x = x \} \$.
\end{mmraw}

\noindent Define the subclass\index{subclass}\index{subset} relationship
between two classes (called the subset relation if the classes are sets i.e.\
are not proper).  Definition 5.9 of Takeuti and Zaring, p.~17.\label{df-ss}

\vskip 0.5ex
\setbox\startprefix=\hbox{\tt \ \ df-ss\ \$a\ }
\setbox\contprefix=\hbox{\tt \ \ \ \ \ \ \ \ \ \ \ }
\startm
\m{\vdash}\m{(}\m{A}\m{\subseteq}\m{B}\m{\leftrightarrow}\m{\forall}\m{x}\m{(}
\m{x}\m{\in}\m{A}\m{\rightarrow}\m{x}\m{\in}\m{B}\m{)}\m{)}
\endm
\begin{mmraw}%
|- ( A C\_ B <-> A. x ( x e. A -> x e. B ) ) \$.
\end{mmraw}

\noindent Define the union\index{union} of two classes.  Definition 5.6 of Takeuti and Zaring,
p.~16.\label{df-un}

\vskip 0.5ex
\setbox\startprefix=\hbox{\tt \ \ df-un\ \$a\ }
\setbox\contprefix=\hbox{\tt \ \ \ \ \ \ \ \ \ \ \ }
\startm
\m{\vdash}\m{(}\m{A}\m{\cup}\m{B}\m{)}\m{=}\m{\{}\m{x}\m{|}\m{(}\m{x}\m{\in}
\m{A}\m{\vee}\m{x}\m{\in}\m{B}\m{)}\m{\}}
\endm
\begin{mmraw}%
( A u. B ) = \{ x | ( x e. A \TOR x e. B ) \} \$.
\end{mmraw}

\noindent Define the intersection\index{intersection} of two classes.  Definition 5.6 of
Takeuti and Zaring, p.~16.\label{df-in}

\vskip 0.5ex
\setbox\startprefix=\hbox{\tt \ \ df-in\ \$a\ }
\setbox\contprefix=\hbox{\tt \ \ \ \ \ \ \ \ \ \ \ }
\startm
\m{\vdash}\m{(}\m{A}\m{\cap}\m{B}\m{)}\m{=}\m{\{}\m{x}\m{|}\m{(}\m{x}\m{\in}
\m{A}\m{\wedge}\m{x}\m{\in}\m{B}\m{)}\m{\}}
\endm
% Caret ^ requires special treatment
\begin{mmraw}%
|- ( A i\^{}i B ) = \{ x | ( x e. A \TAND x e. B ) \} \$.
\end{mmraw}

\noindent Define class difference\index{class difference}\index{set difference}.
Definition 5.12 of Takeuti and Zaring, p.~20.  Several notations are used in
the literature; we chose the $\setminus$ convention instead of a minus sign to
reserve the latter for later use in, e.g., arithmetic.\label{df-dif}

\vskip 0.5ex
\setbox\startprefix=\hbox{\tt \ \ df-dif\ \$a\ }
\setbox\contprefix=\hbox{\tt \ \ \ \ \ \ \ \ \ \ \ \ }
\startm
\m{\vdash}\m{(}\m{A}\m{\setminus}\m{B}\m{)}\m{=}\m{\{}\m{x}\m{|}\m{(}\m{x}\m{
\in}\m{A}\m{\wedge}\m{\lnot}\m{x}\m{\in}\m{B}\m{)}\m{\}}
\endm
\begin{mmraw}%
( A \SLASH B ) = \{ x | ( x e. A \TAND -. x e. B ) \} \$.
\end{mmraw}

\noindent Define the empty or null set\index{empty set}\index{null set}.
Compare  Definition 5.14 of Takeuti and Zaring, p.~20.\label{df-nul}

\vskip 0.5ex
\setbox\startprefix=\hbox{\tt \ \ df-nul\ \$a\ }
\setbox\contprefix=\hbox{\tt \ \ \ \ \ \ \ \ \ \ }
\startm
\m{\vdash}\m{\varnothing}\m{=}\m{(}\m{{\rm V}}\m{\setminus}\m{{\rm V}}\m{)}
\endm
\begin{mmraw}%
|- (/) = ( {\char`\_}V \SLASH {\char`\_}V ) \$.
\end{mmraw}

\noindent Define power class\index{power set}\index{power class}.  Definition 5.10 of
Takeuti and Zaring, p.~17, but we also let it apply to proper classes.  (Note
that \verb$~P$ is the symbol for calligraphic P, the tilde
suggesting ``curly;'' see Appendix~\ref{ASCII}.)\label{df-pw}

\vskip 0.5ex
\setbox\startprefix=\hbox{\tt \ \ df-pw\ \$a\ }
\setbox\contprefix=\hbox{\tt \ \ \ \ \ \ \ \ \ \ \ }
\startm
\m{\vdash}\m{{\cal P}}\m{A}\m{=}\m{\{}\m{x}\m{|}\m{x}\m{\subseteq}\m{A}\m{\}}
\endm
% Special incantation required to put ~ into the text
\begin{mmraw}%
|- \char`\~P~A = \{ x | x C\_ A \} \$.
\end{mmraw}

\noindent Define the singleton of a class\index{singleton}.  Definition 7.1 of
Quine, p.~48.  It is well-defined for proper classes, although
it is not very meaningful in this case, where it evaluates to the empty
set.

\vskip 0.5ex
\setbox\startprefix=\hbox{\tt \ \ df-sn\ \$a\ }
\setbox\contprefix=\hbox{\tt \ \ \ \ \ \ \ \ \ \ \ }
\startm
\m{\vdash}\m{\{}\m{A}\m{\}}\m{=}\m{\{}\m{x}\m{|}\m{x}\m{=}\m{A}\m{\}}
\endm
\begin{mmraw}%
|- \{ A \} = \{ x | x = A \} \$.
\end{mmraw}%

\noindent Define an unordered pair of classes\index{unordered pair}\index{pair}.  Definition
7.1 of Quine, p.~48.

\vskip 0.5ex
\setbox\startprefix=\hbox{\tt \ \ df-pr\ \$a\ }
\setbox\contprefix=\hbox{\tt \ \ \ \ \ \ \ \ \ \ \ }
\startm
\m{\vdash}\m{\{}\m{A}\m{,}\m{B}\m{\}}\m{=}\m{(}\m{\{}\m{A}\m{\}}\m{\cup}\m{\{}
\m{B}\m{\}}\m{)}
\endm
\begin{mmraw}%
|- \{ A , B \} = ( \{ A \} u. \{ B \} ) \$.
\end{mmraw}

\noindent Define an unordered triple of classes\index{unordered triple}.  Definition of
Enderton, p.~19.

\vskip 0.5ex
\setbox\startprefix=\hbox{\tt \ \ df-tp\ \$a\ }
\setbox\contprefix=\hbox{\tt \ \ \ \ \ \ \ \ \ \ \ }
\startm
\m{\vdash}\m{\{}\m{A}\m{,}\m{B}\m{,}\m{C}\m{\}}\m{=}\m{(}\m{\{}\m{A}\m{,}\m{B}
\m{\}}\m{\cup}\m{\{}\m{C}\m{\}}\m{)}
\endm
\begin{mmraw}%
|- \{ A , B , C \} = ( \{ A , B \} u. \{ C \} ) \$.
\end{mmraw}%

\noindent Kuratowski's\index{Kuratowski, Kazimierz} ordered pair\index{ordered
pair} definition.  Definition 9.1 of Quine, p.~58. For proper classes it is
not meaningful but is well-defined for convenience.  (Note that \verb$<.$
stands for $\langle$ whereas \verb$<$ stands for $<$, and similarly for
\verb$>.$\,.)\label{df-op}

\vskip 0.5ex
\setbox\startprefix=\hbox{\tt \ \ df-op\ \$a\ }
\setbox\contprefix=\hbox{\tt \ \ \ \ \ \ \ \ \ \ \ }
\startm
\m{\vdash}\m{\langle}\m{A}\m{,}\m{B}\m{\rangle}\m{=}\m{\{}\m{\{}\m{A}\m{\}}
\m{,}\m{\{}\m{A}\m{,}\m{B}\m{\}}\m{\}}
\endm
\begin{mmraw}%
|- <. A , B >. = \{ \{ A \} , \{ A , B \} \} \$.
\end{mmraw}

\noindent Define the union of a class\index{union}.  Definition 5.5, p.~16,
of Takeuti and Zaring.

\vskip 0.5ex
\setbox\startprefix=\hbox{\tt \ \ df-uni\ \$a\ }
\setbox\contprefix=\hbox{\tt \ \ \ \ \ \ \ \ \ \ \ \ }
\startm
\m{\vdash}\m{\bigcup}\m{A}\m{=}\m{\{}\m{x}\m{|}\m{\exists}\m{y}\m{(}\m{x}\m{
\in}\m{y}\m{\wedge}\m{y}\m{\in}\m{A}\m{)}\m{\}}
\endm
\begin{mmraw}%
|- U. A = \{ x | E. y ( x e. y \TAND y e. A ) \} \$.
\end{mmraw}

\noindent Define the intersection\index{intersection} of a class.  Definition 7.35,
p.~44, of Takeuti and Zaring.

\vskip 0.5ex
\setbox\startprefix=\hbox{\tt \ \ df-int\ \$a\ }
\setbox\contprefix=\hbox{\tt \ \ \ \ \ \ \ \ \ \ \ \ }
\startm
\m{\vdash}\m{\bigcap}\m{A}\m{=}\m{\{}\m{x}\m{|}\m{\forall}\m{y}\m{(}\m{y}\m{
\in}\m{A}\m{\rightarrow}\m{x}\m{\in}\m{y}\m{)}\m{\}}
\endm
\begin{mmraw}%
|- |\^{}| A = \{ x | A. y ( y e. A -> x e. y ) \} \$.
\end{mmraw}

\noindent Define a transitive class\index{transitive class}\index{transitive
set}.  This should not be confused with a transitive relation, which is a different
concept.  Definition from p.~71 of Enderton, extended to classes.

\vskip 0.5ex
\setbox\startprefix=\hbox{\tt \ \ df-tr\ \$a\ }
\setbox\contprefix=\hbox{\tt \ \ \ \ \ \ \ \ \ \ \ }
\startm
\m{\vdash}\m{(}\m{\mbox{\rm Tr}}\m{A}\m{\leftrightarrow}\m{\bigcup}\m{A}\m{
\subseteq}\m{A}\m{)}
\endm
\begin{mmraw}%
|- ( Tr A <-> U. A C\_ A ) \$.
\end{mmraw}

\noindent Define a notation for a general binary relation\index{binary
relation}.  Definition 6.18, p.~29, of Takeuti and Zaring, generalized to
arbitrary classes.  This definition is well-defined, although not very
meaningful, when classes $A$ and/or $B$ are proper.\label{dfbr}  The lack of
parentheses (or any other connective) creates no ambiguity since we are defining
an atomic wff.

\vskip 0.5ex
\setbox\startprefix=\hbox{\tt \ \ df-br\ \$a\ }
\setbox\contprefix=\hbox{\tt \ \ \ \ \ \ \ \ \ \ \ }
\startm
\m{\vdash}\m{(}\m{A}\m{\,R}\m{\,B}\m{\leftrightarrow}\m{\langle}\m{A}\m{,}\m{B}
\m{\rangle}\m{\in}\m{R}\m{)}
\endm
\begin{mmraw}%
|- ( A R B <-> <. A , B >. e. R ) \$.
\end{mmraw}

\noindent Define an abstraction class of ordered pairs\index{abstraction
class!of ordered
pairs}.  A special case of Definition 4.16, p.~14, of Takeuti and Zaring.
Note that $ z $ must be distinct from $ x $ and $ y $,
and $ z $ must not occur in $\varphi$, but $ x $ and $ y $ may be
identical and may appear in $\varphi$.

\vskip 0.5ex
\setbox\startprefix=\hbox{\tt \ \ df-opab\ \$a\ }
\setbox\contprefix=\hbox{\tt \ \ \ \ \ \ \ \ \ \ \ \ \ }
\startm
\m{\vdash}\m{\{}\m{\langle}\m{x}\m{,}\m{y}\m{\rangle}\m{|}\m{\varphi}\m{\}}\m{=}
\m{\{}\m{z}\m{|}\m{\exists}\m{x}\m{\exists}\m{y}\m{(}\m{z}\m{=}\m{\langle}\m{x}
\m{,}\m{y}\m{\rangle}\m{\wedge}\m{\varphi}\m{)}\m{\}}
\endm

\begin{mmraw}%
|- \{ <. x , y >. | ph \} = \{ z | E. x E. y ( z =
<. x , y >. /\ ph ) \} \$.
\end{mmraw}

\noindent Define the epsilon relation\index{epsilon relation}.  Similar to Definition
6.22, p.~30, of Takeuti and Zaring.

\vskip 0.5ex
\setbox\startprefix=\hbox{\tt \ \ df-eprel\ \$a\ }
\setbox\contprefix=\hbox{\tt \ \ \ \ \ \ \ \ \ \ \ \ \ \ }
\startm
\m{\vdash}\m{{\rm E}}\m{=}\m{\{}\m{\langle}\m{x}\m{,}\m{y}\m{\rangle}\m{|}\m{x}\m{
\in}\m{y}\m{\}}
\endm
\begin{mmraw}%
|- \_E = \{ <. x , y >. | x e. y \} \$.
\end{mmraw}

\noindent Define a founded relation\index{founded relation}.  $R$ is a founded
relation on $A$ iff\index{iff} (if and only if) each nonempty subset of $A$
has an ``$R$-minimal element.''  Similar to Definition 6.21, p.~30, of
Takeuti and Zaring.

\vskip 0.5ex
\setbox\startprefix=\hbox{\tt \ \ df-fr\ \$a\ }
\setbox\contprefix=\hbox{\tt \ \ \ \ \ \ \ \ \ \ \ }
\startm
\m{\vdash}\m{(}\m{R}\m{\,\mbox{\rm Fr}}\m{\,A}\m{\leftrightarrow}\m{\forall}\m{x}
\m{(}\m{(}\m{x}\m{\subseteq}\m{A}\m{\wedge}\m{\lnot}\m{x}\m{=}\m{\varnothing}
\m{)}\m{\rightarrow}\m{\exists}\m{y}\m{(}\m{y}\m{\in}\m{x}\m{\wedge}\m{(}\m{x}
\m{\cap}\m{\{}\m{z}\m{|}\m{z}\m{\,R}\m{\,y}\m{\}}\m{)}\m{=}\m{\varnothing}\m{)}
\m{)}\m{)}
\endm
\begin{mmraw}%
|- ( R Fr A <-> A. x ( ( x C\_ A \TAND -. x = (/) ) ->
E. y ( y e. x \TAND ( x i\^{}i \{ z | z R y \} ) = (/) ) ) ) \$.
\end{mmraw}

\noindent Define a well-ordering\index{well-ordering}.  $R$ is a well-ordering of $A$ iff
it is founded on $A$ and the elements of $A$ are pairwise $R$-comparable.
Similar to Definition 6.24(2), p.~30, of Takeuti and Zaring.

\vskip 0.5ex
\setbox\startprefix=\hbox{\tt \ \ df-we\ \$a\ }
\setbox\contprefix=\hbox{\tt \ \ \ \ \ \ \ \ \ \ \ }
\startm
\m{\vdash}\m{(}\m{R}\m{\,\mbox{\rm We}}\m{\,A}\m{\leftrightarrow}\m{(}\m{R}\m{\,
\mbox{\rm Fr}}\m{\,A}\m{\wedge}\m{\forall}\m{x}\m{\forall}\m{y}\m{(}\m{(}\m{x}\m{
\in}\m{A}\m{\wedge}\m{y}\m{\in}\m{A}\m{)}\m{\rightarrow}\m{(}\m{x}\m{\,R}\m{\,y}
\m{\vee}\m{x}\m{=}\m{y}\m{\vee}\m{y}\m{\,R}\m{\,x}\m{)}\m{)}\m{)}\m{)}
\endm
\begin{mmraw}%
( R We A <-> ( R Fr A \TAND A. x A. y ( ( x e.
A \TAND y e. A ) -> ( x R y \TOR x = y \TOR y R x ) ) ) ) \$.
\end{mmraw}

\noindent Define the ordinal predicate\index{ordinal predicate}, which is true for a class
that is transitive and is well-ordered by the epsilon relation.  Similar to
definition on p.~468, Bell and Machover.

\vskip 0.5ex
\setbox\startprefix=\hbox{\tt \ \ df-ord\ \$a\ }
\setbox\contprefix=\hbox{\tt \ \ \ \ \ \ \ \ \ \ \ \ }
\startm
\m{\vdash}\m{(}\m{\mbox{\rm Ord}}\m{\,A}\m{\leftrightarrow}\m{(}
\m{\mbox{\rm Tr}}\m{\,A}\m{\wedge}\m{E}\m{\,\mbox{\rm We}}\m{\,A}\m{)}\m{)}
\endm
\begin{mmraw}%
|- ( Ord A <-> ( Tr A \TAND E We A ) ) \$.
\end{mmraw}

\noindent Define the class of all ordinal numbers\index{ordinal number}.  An ordinal number is
a set that satisfies the ordinal predicate.  Definition 7.11 of Takeuti and
Zaring, p.~38.

\vskip 0.5ex
\setbox\startprefix=\hbox{\tt \ \ df-on\ \$a\ }
\setbox\contprefix=\hbox{\tt \ \ \ \ \ \ \ \ \ \ \ }
\startm
\m{\vdash}\m{\,\mbox{\rm On}}\m{=}\m{\{}\m{x}\m{|}\m{\mbox{\rm Ord}}\m{\,x}
\m{\}}
\endm
\begin{mmraw}%
|- On = \{ x | Ord x \} \$.
\end{mmraw}

\noindent Define the limit ordinal predicate\index{limit ordinal}, which is true for a
non-empty ordinal that is not a successor (i.e.\ that is the union of itself).
Compare Bell and Machover, p.~471 and Exercise (1), p.~42 of Takeuti and
Zaring.

\vskip 0.5ex
\setbox\startprefix=\hbox{\tt \ \ df-lim\ \$a\ }
\setbox\contprefix=\hbox{\tt \ \ \ \ \ \ \ \ \ \ \ \ }
\startm
\m{\vdash}\m{(}\m{\mbox{\rm Lim}}\m{\,A}\m{\leftrightarrow}\m{(}\m{\mbox{
\rm Ord}}\m{\,A}\m{\wedge}\m{\lnot}\m{A}\m{=}\m{\varnothing}\m{\wedge}\m{A}
\m{=}\m{\bigcup}\m{A}\m{)}\m{)}
\endm
\begin{mmraw}%
|- ( Lim A <-> ( Ord A \TAND -. A = (/) \TAND A = U. A ) ) \$.
\end{mmraw}

\noindent Define the successor\index{successor} of a class.  Definition 7.22 of Takeuti
and Zaring, p.~41.  Our definition is a generalization to classes, although it
is meaningless when classes are proper.

\vskip 0.5ex
\setbox\startprefix=\hbox{\tt \ \ df-suc\ \$a\ }
\setbox\contprefix=\hbox{\tt \ \ \ \ \ \ \ \ \ \ \ \ }
\startm
\m{\vdash}\m{\,\mbox{\rm suc}}\m{\,A}\m{=}\m{(}\m{A}\m{\cup}\m{\{}\m{A}\m{\}}
\m{)}
\endm
\begin{mmraw}%
|- suc A = ( A u. \{ A \} ) \$.
\end{mmraw}

\noindent Define the class of natural numbers\index{natural number}\index{omega
($\omega$)}.  Compare Bell and Machover, p.~471.\label{dfom}

\vskip 0.5ex
\setbox\startprefix=\hbox{\tt \ \ df-om\ \$a\ }
\setbox\contprefix=\hbox{\tt \ \ \ \ \ \ \ \ \ \ \ }
\startm
\m{\vdash}\m{\omega}\m{=}\m{\{}\m{x}\m{|}\m{(}\m{\mbox{\rm Ord}}\m{\,x}\m{
\wedge}\m{\forall}\m{y}\m{(}\m{\mbox{\rm Lim}}\m{\,y}\m{\rightarrow}\m{x}\m{
\in}\m{y}\m{)}\m{)}\m{\}}
\endm
\begin{mmraw}%
|- om = \{ x | ( Ord x \TAND A. y ( Lim y -> x e. y ) ) \} \$.
\end{mmraw}

\noindent Define the Cartesian product (also called the
cross product)\index{Cartesian product}\index{cross product}
of two classes.  Definition 9.11 of Quine, p.~64.

\vskip 0.5ex
\setbox\startprefix=\hbox{\tt \ \ df-xp\ \$a\ }
\setbox\contprefix=\hbox{\tt \ \ \ \ \ \ \ \ \ \ \ }
\startm
\m{\vdash}\m{(}\m{A}\m{\times}\m{B}\m{)}\m{=}\m{\{}\m{\langle}\m{x}\m{,}\m{y}
\m{\rangle}\m{|}\m{(}\m{x}\m{\in}\m{A}\m{\wedge}\m{y}\m{\in}\m{B}\m{)}\m{\}}
\endm
\begin{mmraw}%
|- ( A X. B ) = \{ <. x , y >. | ( x e. A \TAND y e. B) \} \$.
\end{mmraw}

\noindent Define a relation\index{relation}.  Definition 6.4(1) of Takeuti and
Zaring, p.~23.

\vskip 0.5ex
\setbox\startprefix=\hbox{\tt \ \ df-rel\ \$a\ }
\setbox\contprefix=\hbox{\tt \ \ \ \ \ \ \ \ \ \ \ \ }
\startm
\m{\vdash}\m{(}\m{\mbox{\rm Rel}}\m{\,A}\m{\leftrightarrow}\m{A}\m{\subseteq}
\m{(}\m{{\rm V}}\m{\times}\m{{\rm V}}\m{)}\m{)}
\endm
\begin{mmraw}%
|- ( Rel A <-> A C\_ ( {\char`\_}V X. {\char`\_}V ) ) \$.
\end{mmraw}

\noindent Define the domain\index{domain} of a class.  Definition 6.5(1) of
Takeuti and Zaring, p.~24.

\vskip 0.5ex
\setbox\startprefix=\hbox{\tt \ \ df-dm\ \$a\ }
\setbox\contprefix=\hbox{\tt \ \ \ \ \ \ \ \ \ \ \ }
\startm
\m{\vdash}\m{\,\mbox{\rm dom}}\m{A}\m{=}\m{\{}\m{x}\m{|}\m{\exists}\m{y}\m{
\langle}\m{x}\m{,}\m{y}\m{\rangle}\m{\in}\m{A}\m{\}}
\endm
\begin{mmraw}%
|- dom A = \{ x | E. y <. x , y >. e. A \} \$.
\end{mmraw}

\noindent Define the range\index{range} of a class.  Definition 6.5(2) of
Takeuti and Zaring, p.~24.

\vskip 0.5ex
\setbox\startprefix=\hbox{\tt \ \ df-rn\ \$a\ }
\setbox\contprefix=\hbox{\tt \ \ \ \ \ \ \ \ \ \ \ }
\startm
\m{\vdash}\m{\,\mbox{\rm ran}}\m{A}\m{=}\m{\{}\m{y}\m{|}\m{\exists}\m{x}\m{
\langle}\m{x}\m{,}\m{y}\m{\rangle}\m{\in}\m{A}\m{\}}
\endm
\begin{mmraw}%
|- ran A = \{ y | E. x <. x , y >. e. A \} \$.
\end{mmraw}

\noindent Define the restriction\index{restriction} of a class.  Definition
6.6(1) of Takeuti and Zaring, p.~24.

\vskip 0.5ex
\setbox\startprefix=\hbox{\tt \ \ df-res\ \$a\ }
\setbox\contprefix=\hbox{\tt \ \ \ \ \ \ \ \ \ \ \ \ }
\startm
\m{\vdash}\m{(}\m{A}\m{\restriction}\m{B}\m{)}\m{=}\m{(}\m{A}\m{\cap}\m{(}\m{B}
\m{\times}\m{{\rm V}}\m{)}\m{)}
\endm
\begin{mmraw}%
|- ( A |` B ) = ( A i\^{}i ( B X. {\char`\_}V ) ) \$.
\end{mmraw}

\noindent Define the image\index{image} of a class.  Definition 6.6(2) of
Takeuti and Zaring, p.~24.

\vskip 0.5ex
\setbox\startprefix=\hbox{\tt \ \ df-ima\ \$a\ }
\setbox\contprefix=\hbox{\tt \ \ \ \ \ \ \ \ \ \ \ \ }
\startm
\m{\vdash}\m{(}\m{A}\m{``}\m{B}\m{)}\m{=}\m{\,\mbox{\rm ran}}\m{\,(}\m{A}\m{
\restriction}\m{B}\m{)}
\endm
\begin{mmraw}%
|- ( A " B ) = ran ( A |` B ) \$.
\end{mmraw}

\noindent Define the composition\index{composition} of two classes.  Definition 6.6(3) of
Takeuti and Zaring, p.~24.

\vskip 0.5ex
\setbox\startprefix=\hbox{\tt \ \ df-co\ \$a\ }
\setbox\contprefix=\hbox{\tt \ \ \ \ \ \ \ \ \ \ \ \ }
\startm
\m{\vdash}\m{(}\m{A}\m{\circ}\m{B}\m{)}\m{=}\m{\{}\m{\langle}\m{x}\m{,}\m{y}\m{
\rangle}\m{|}\m{\exists}\m{z}\m{(}\m{\langle}\m{x}\m{,}\m{z}\m{\rangle}\m{\in}
\m{B}\m{\wedge}\m{\langle}\m{z}\m{,}\m{y}\m{\rangle}\m{\in}\m{A}\m{)}\m{\}}
\endm
\begin{mmraw}%
|- ( A o. B ) = \{ <. x , y >. | E. z ( <. x , z
>. e. B \TAND <. z , y >. e. A ) \} \$.
\end{mmraw}

\noindent Define a function\index{function}.  Definition 6.4(4) of Takeuti and
Zaring, p.~24.

\vskip 0.5ex
\setbox\startprefix=\hbox{\tt \ \ df-fun\ \$a\ }
\setbox\contprefix=\hbox{\tt \ \ \ \ \ \ \ \ \ \ \ \ }
\startm
\m{\vdash}\m{(}\m{\mbox{\rm Fun}}\m{\,A}\m{\leftrightarrow}\m{(}
\m{\mbox{\rm Rel}}\m{\,A}\m{\wedge}
\m{\forall}\m{x}\m{\exists}\m{z}\m{\forall}\m{y}\m{(}
\m{\langle}\m{x}\m{,}\m{y}\m{\rangle}\m{\in}\m{A}\m{\rightarrow}\m{y}\m{=}\m{z}
\m{)}\m{)}\m{)}
\endm
\begin{mmraw}%
|- ( Fun A <-> ( Rel A /\ A. x E. z A. y ( <. x
   , y >. e. A -> y = z ) ) ) \$.
\end{mmraw}

\noindent Define a function with domain.  Definition 6.15(1) of Takeuti and
Zaring, p.~27.

\vskip 0.5ex
\setbox\startprefix=\hbox{\tt \ \ df-fn\ \$a\ }
\setbox\contprefix=\hbox{\tt \ \ \ \ \ \ \ \ \ \ \ }
\startm
\m{\vdash}\m{(}\m{A}\m{\,\mbox{\rm Fn}}\m{\,B}\m{\leftrightarrow}\m{(}
\m{\mbox{\rm Fun}}\m{\,A}\m{\wedge}\m{\mbox{\rm dom}}\m{\,A}\m{=}\m{B}\m{)}
\m{)}
\endm
\begin{mmraw}%
|- ( A Fn B <-> ( Fun A \TAND dom A = B ) ) \$.
\end{mmraw}

\noindent Define a function with domain and co-domain.  Definition 6.15(3)
of Takeuti and Zaring, p.~27.

\vskip 0.5ex
\setbox\startprefix=\hbox{\tt \ \ df-f\ \$a\ }
\setbox\contprefix=\hbox{\tt \ \ \ \ \ \ \ \ \ \ }
\startm
\m{\vdash}\m{(}\m{F}\m{:}\m{A}\m{\longrightarrow}\m{B}\m{
\leftrightarrow}\m{(}\m{F}\m{\,\mbox{\rm Fn}}\m{\,A}\m{\wedge}\m{
\mbox{\rm ran}}\m{\,F}\m{\subseteq}\m{B}\m{)}\m{)}
\endm
\begin{mmraw}%
|- ( F : A --> B <-> ( F Fn A \TAND ran F C\_ B ) ) \$.
\end{mmraw}

\noindent Define a one-to-one function\index{one-to-one function}.  Compare
Definition 6.15(5) of Takeuti and Zaring, p.~27.

\vskip 0.5ex
\setbox\startprefix=\hbox{\tt \ \ df-f1\ \$a\ }
\setbox\contprefix=\hbox{\tt \ \ \ \ \ \ \ \ \ \ \ }
\startm
\m{\vdash}\m{(}\m{F}\m{:}\m{A}\m{
\raisebox{.5ex}{${\textstyle{\:}_{\mbox{\footnotesize\rm
1\tt -\rm 1}}}\atop{\textstyle{
\longrightarrow}\atop{\textstyle{}^{\mbox{\footnotesize\rm {\ }}}}}$}
}\m{B}
\m{\leftrightarrow}\m{(}\m{F}\m{:}\m{A}\m{\longrightarrow}\m{B}
\m{\wedge}\m{\forall}\m{y}\m{\exists}\m{z}\m{\forall}\m{x}\m{(}\m{\langle}\m{x}
\m{,}\m{y}\m{\rangle}\m{\in}\m{F}\m{\rightarrow}\m{x}\m{=}\m{z}\m{)}\m{)}\m{)}
\endm
\begin{mmraw}%
|- ( F : A -1-1-> B <-> ( F : A --> B \TAND
   A. y E. z A. x ( <. x , y >. e. F -> x = z ) ) ) \$.
\end{mmraw}

\noindent Define an onto function\index{onto function}.  Definition 6.15(4) of Takeuti and
Zaring, p.~27.

\vskip 0.5ex
\setbox\startprefix=\hbox{\tt \ \ df-fo\ \$a\ }
\setbox\contprefix=\hbox{\tt \ \ \ \ \ \ \ \ \ \ \ }
\startm
\m{\vdash}\m{(}\m{F}\m{:}\m{A}\m{
\raisebox{.5ex}{${\textstyle{\:}_{\mbox{\footnotesize\rm
{\ }}}}\atop{\textstyle{
\longrightarrow}\atop{\textstyle{}^{\mbox{\footnotesize\rm onto}}}}$}
}\m{B}
\m{\leftrightarrow}\m{(}\m{F}\m{\,\mbox{\rm Fn}}\m{\,A}\m{\wedge}
\m{\mbox{\rm ran}}\m{\,F}\m{=}\m{B}\m{)}\m{)}
\endm
\begin{mmraw}%
|- ( F : A -onto-> B <-> ( F Fn A /\ ran F = B ) ) \$.
\end{mmraw}

\noindent Define a one-to-one, onto function.  Compare Definition 6.15(6) of
Takeuti and Zaring, p.~27.

\vskip 0.5ex
\setbox\startprefix=\hbox{\tt \ \ df-f1o\ \$a\ }
\setbox\contprefix=\hbox{\tt \ \ \ \ \ \ \ \ \ \ \ \ }
\startm
\m{\vdash}\m{(}\m{F}\m{:}\m{A}
\m{
\raisebox{.5ex}{${\textstyle{\:}_{\mbox{\footnotesize\rm
1\tt -\rm 1}}}\atop{\textstyle{
\longrightarrow}\atop{\textstyle{}^{\mbox{\footnotesize\rm onto}}}}$}
}
\m{B}
\m{\leftrightarrow}\m{(}\m{F}\m{:}\m{A}
\m{
\raisebox{.5ex}{${\textstyle{\:}_{\mbox{\footnotesize\rm
1\tt -\rm 1}}}\atop{\textstyle{
\longrightarrow}\atop{\textstyle{}^{\mbox{\footnotesize\rm {\ }}}}}$}
}
\m{B}\m{\wedge}\m{F}\m{:}\m{A}
\m{
\raisebox{.5ex}{${\textstyle{\:}_{\mbox{\footnotesize\rm
{\ }}}}\atop{\textstyle{
\longrightarrow}\atop{\textstyle{}^{\mbox{\footnotesize\rm onto}}}}$}
}
\m{B}\m{)}\m{)}
\endm
\begin{mmraw}%
|- ( F : A -1-1-onto-> B <-> ( F : A -1-1-> B? \TAND F : A -onto-> B ) ) \$.?
\end{mmraw}

\noindent Define the value of a function\index{function value}.  This
definition applies to any class and evaluates to the empty set when it is not
meaningful. Note that $ F`A$ means the same thing as the more familiar $ F(A)$
notation for a function's value at $A$.  The $ F`A$ notation is common in
formal set theory.\label{df-fv}

\vskip 0.5ex
\setbox\startprefix=\hbox{\tt \ \ df-fv\ \$a\ }
\setbox\contprefix=\hbox{\tt \ \ \ \ \ \ \ \ \ \ \ }
\startm
\m{\vdash}\m{(}\m{F}\m{`}\m{A}\m{)}\m{=}\m{\bigcup}\m{\{}\m{x}\m{|}\m{(}\m{F}%
\m{``}\m{\{}\m{A}\m{\}}\m{)}\m{=}\m{\{}\m{x}\m{\}}\m{\}}
\endm
\begin{mmraw}%
|- ( F ` A ) = U. \{ x | ( F " \{ A \} ) = \{ x \} \} \$.
\end{mmraw}

\noindent Define the result of an operation.\index{operation}  Here, $F$ is
     an operation on two
     values (such as $+$ for real numbers).   This is defined for proper
     classes $A$ and $B$ even though not meaningful in that case.  However,
     the definition can be meaningful when $F$ is a proper class.\label{dfopr}

\vskip 0.5ex
\setbox\startprefix=\hbox{\tt \ \ df-opr\ \$a\ }
\setbox\contprefix=\hbox{\tt \ \ \ \ \ \ \ \ \ \ \ \ }
\startm
\m{\vdash}\m{(}\m{A}\m{\,F}\m{\,B}\m{)}\m{=}\m{(}\m{F}\m{`}\m{\langle}\m{A}%
\m{,}\m{B}\m{\rangle}\m{)}
\endm
\begin{mmraw}%
|- ( A F B ) = ( F ` <. A , B >. ) \$.
\end{mmraw}

\section{Tricks of the Trade}\label{tricks}

In the \texttt{set.mm}\index{set theory database (\texttt{set.mm})} database our goal
was usually to conform to modern notation.  However in some cases the
relationship to standard textbook language may be obscured by several
unconventional devices we used to simplify the development and to take
advantage of the Metamath language.  In this section we will describe some
common conventions used in \texttt{set.mm}.

\begin{itemize}
\item
The turnstile\index{turnstile ({$\,\vdash$})} symbol, $\vdash$, meaning ``it
is provable that,'' is the first token of all assertions and hypotheses that
aren't syntax constructions.  This is a standard convention in logic.  (We
mentioned this earlier, but this symbol is bothersome to some people without a
logic background.  It has no deeper meaning but just provides us with a way to
distinguish syntax constructions from ordinary mathematical statements.)

\item
A hypothesis of the form

\vskip 1ex
\setbox\startprefix=\hbox{\tt \ \ \ \ \ \ \ \ \ \$e\ }
\setbox\contprefix=\hbox{\tt \ \ \ \ \ \ \ \ \ \ }
\startm
\m{\vdash}\m{(}\m{\varphi}\m{\rightarrow}\m{\forall}\m{x}\m{\varphi}\m{)}
\endm
\vskip 1ex

should be read ``assume variable $x$ is (effectively) not free in wff
$\varphi$.''\index{effectively not free}
Literally, this says ``assume it is provable that $\varphi \rightarrow \forall
x\, \varphi$.''  This device lets us avoid the complexities associated with
the standard treatment of free and bound variables.
%% Uncomment this when uncommenting section {formalspec} below
The footnote on p.~\pageref{effectivelybound} discusses this further.

\item
A statement of one of the forms

\vskip 1ex
\setbox\startprefix=\hbox{\tt \ \ \ \ \ \ \ \ \ \$a\ }
\setbox\contprefix=\hbox{\tt \ \ \ \ \ \ \ \ \ \ }
\startm
\m{\vdash}\m{(}\m{\lnot}\m{\forall}\m{x}\m{\,x}\m{=}\m{y}\m{\rightarrow}
\m{\ldots}\m{)}
\endm
\setbox\startprefix=\hbox{\tt \ \ \ \ \ \ \ \ \ \$p\ }
\setbox\contprefix=\hbox{\tt \ \ \ \ \ \ \ \ \ \ }
\startm
\m{\vdash}\m{(}\m{\lnot}\m{\forall}\m{x}\m{\,x}\m{=}\m{y}\m{\rightarrow}
\m{\ldots}\m{)}
\endm
\vskip 1ex

should be read ``if $x$ and $y$ are distinct variables, then...''  This
antecedent provides us with a technical device to avoid the need for the
\texttt{\$d} statement early in our development of predicate calculus,
permitting symbol manipulations to be as conceptually simple as those in
propositional calculus.  However, the \texttt{\$d} statement eventually
becomes a requirement, and after that this device is rarely used.

\item
The statement

\vskip 1ex
\setbox\startprefix=\hbox{\tt \ \ \ \ \ \ \ \ \ \$d\ }
\setbox\contprefix=\hbox{\tt \ \ \ \ \ \ \ \ \ \ }
\startm
\m{x}\m{\,y}
\endm
\vskip 1ex

should be read ``assume $x$ and $y$ are distinct variables.''

\item
The statement

\vskip 1ex
\setbox\startprefix=\hbox{\tt \ \ \ \ \ \ \ \ \ \$d\ }
\setbox\contprefix=\hbox{\tt \ \ \ \ \ \ \ \ \ \ }
\startm
\m{x}\m{\,\varphi}
\endm
\vskip 1ex

should be read ``assume $x$ does not occur in $\varphi$.''

\item
The statement

\vskip 1ex
\setbox\startprefix=\hbox{\tt \ \ \ \ \ \ \ \ \ \$d\ }
\setbox\contprefix=\hbox{\tt \ \ \ \ \ \ \ \ \ \ }
\startm
\m{x}\m{\,A}
\endm
\vskip 1ex

should be read ``assume variable $x$ does not occur in class $A$.''

\item
The restriction and hypothesis group

\vskip 1ex
\setbox\startprefix=\hbox{\tt \ \ \ \ \ \ \ \ \ \$d\ }
\setbox\contprefix=\hbox{\tt \ \ \ \ \ \ \ \ \ \ }
\startm
\m{x}\m{\,A}
\endm
\setbox\startprefix=\hbox{\tt \ \ \ \ \ \ \ \ \ \$d\ }
\setbox\contprefix=\hbox{\tt \ \ \ \ \ \ \ \ \ \ }
\startm
\m{x}\m{\,\psi}
\endm
\setbox\startprefix=\hbox{\tt \ \ \ \ \ \ \ \ \ \$e\ }
\setbox\contprefix=\hbox{\tt \ \ \ \ \ \ \ \ \ \ }
\startm
\m{\vdash}\m{(}\m{x}\m{=}\m{A}\m{\rightarrow}\m{(}\m{\varphi}\m{\leftrightarrow}
\m{\psi}\m{)}\m{)}
\endm
\vskip 1ex

is frequently used in place of explicit substitution, meaning ``assume
$\psi$ results from the proper substitution of $A$ for $x$ in
$\varphi$.''  Sometimes ``\texttt{\$e} $\vdash ( \psi \rightarrow
\forall x \, \psi )$'' is used instead of ``\texttt{\$d} $x\, \psi $,''
which requires only that $x$ be effectively not free in $\varphi$ but
not necessarily absent from it.  The use of implicit
substitution\index{substitution!implicit} is partly a matter of personal
style, although it may make proofs somewhat shorter than would be the
case with explicit substitution.

\item
The hypothesis


\vskip 1ex
\setbox\startprefix=\hbox{\tt \ \ \ \ \ \ \ \ \ \$e\ }
\setbox\contprefix=\hbox{\tt \ \ \ \ \ \ \ \ \ \ }
\startm
\m{\vdash}\m{A}\m{\in}\m{{\rm V}}
\endm
\vskip 1ex

should be read ``assume class $A$ is a set (i.e.\ exists).''
This is a convenient convention used by Quine.

\item
The restriction and hypothesis

\vskip 1ex
\setbox\startprefix=\hbox{\tt \ \ \ \ \ \ \ \ \ \$d\ }
\setbox\contprefix=\hbox{\tt \ \ \ \ \ \ \ \ \ \ }
\startm
\m{x}\m{\,y}
\endm
\setbox\startprefix=\hbox{\tt \ \ \ \ \ \ \ \ \ \$e\ }
\setbox\contprefix=\hbox{\tt \ \ \ \ \ \ \ \ \ \ }
\startm
\m{\vdash}\m{(}\m{y}\m{\in}\m{A}\m{\rightarrow}\m{\forall}\m{x}\m{\,y}
\m{\in}\m{A}\m{)}
\endm
\vskip 1ex

should be read ``assume variable $x$ is
(effectively) not free in class $A$.''

\end{itemize}

\section{A Theorem Sampler}\label{sometheorems}

In this section we list some of the more important theorems that are proved in
the \texttt{set.mm} database, and they illustrate the kinds of things that can be
done with Metamath.  While all of these facts are well-known results,
Metamath offers the advantage of easily allowing you to trace their
derivation back to axioms.  Our intent here is not to try to explain the
details or motivation; for this we refer you to the textbooks that are
mentioned in the descriptions.  (The \texttt{set.mm} file has bibliographic
references for the text references.)  Their proofs often embody important
concepts you may wish to explore with the Metamath program (see
Section~\ref{exploring}).  All the symbols that are used here are defined in
Section~\ref{hierarchy}.  For brevity we haven't included the \texttt{\$d}
restrictions or \texttt{\$f} hypotheses for these theorems; when you are
uncertain consult the \texttt{set.mm} database.

We start with \texttt{syl} (principle of the syllogism).
In \textit{Principia Mathematica}
Whitehead and Russell call this ``the principle of the
syllogism... because... the syllogism in Barbara is derived from them''
\cite[quote after Theorem *2.06 p.~101]{PM}.
Some authors call this law a ``hypothetical syllogism.''
As of 2019 \texttt{syl} is the most commonly referenced proven
assertion in the \texttt{set.mm} database.\footnote{
The Metamath program command \texttt{show usage}
shows the number of references.
On 2019-04-29 (commit 71cbbbdb387e) \texttt{syl} was directly referenced
10,819 times. The second most commonly referenced proven assertion
was \texttt{eqid}, which was directly referenced 7,738 times.
}

\vskip 2ex
\noindent Theorem syl (principle of the syllogism)\index{Syllogism}%
\index{\texttt{syl}}\label{syl}.

\vskip 0.5ex
\setbox\startprefix=\hbox{\tt \ \ syl.1\ \$e\ }
\setbox\contprefix=\hbox{\tt \ \ \ \ \ \ \ \ \ \ \ }
\startm
\m{\vdash}\m{(}\m{\varphi}\m{ \rightarrow }\m{\psi}\m{)}
\endm
\setbox\startprefix=\hbox{\tt \ \ syl.2\ \$e\ }
\setbox\contprefix=\hbox{\tt \ \ \ \ \ \ \ \ \ \ \ }
\startm
\m{\vdash}\m{(}\m{\psi}\m{ \rightarrow }\m{\chi}\m{)}
\endm
\setbox\startprefix=\hbox{\tt \ \ syl\ \$p\ }
\setbox\contprefix=\hbox{\tt \ \ \ \ \ \ \ \ \ }
\startm
\m{\vdash}\m{(}\m{\varphi}\m{ \rightarrow }\m{\chi}\m{)}
\endm
\vskip 2ex

The following theorem is not very deep but provides us with a notational device
that is frequently used.  It allows us to use the expression ``$A \in V$'' as
a compact way of saying that class $A$ exists, i.e.\ is a set.

\vskip 2ex
\noindent Two ways to say ``$A$ is a set'':  $A$ is a member of the universe
$V$ if and only if $A$ exists (i.e.\ there exists a set equal to $A$).
Theorem 6.9 of Quine, p. 43.

\vskip 0.5ex
\setbox\startprefix=\hbox{\tt \ \ isset\ \$p\ }
\setbox\contprefix=\hbox{\tt \ \ \ \ \ \ \ \ \ \ \ }
\startm
\m{\vdash}\m{(}\m{A}\m{\in}\m{{\rm V}}\m{\leftrightarrow}\m{\exists}\m{x}\m{\,x}\m{=}
\m{A}\m{)}
\endm
\vskip 1ex

Next we prove the axioms of standard ZF set theory that were missing from our
axiom system.  From our point of view they are theorems since they
can be derived from the other axioms.

\vskip 2ex
\noindent Axiom of Separation\index{Axiom of Separation}
(Aussonderung)\index{Aussonderung} proved from the other axioms of ZF set
theory.  Compare Exercise 4 of Takeuti and Zaring, p.~22.

\vskip 0.5ex
\setbox\startprefix=\hbox{\tt \ \ inex1.1\ \$e\ }
\setbox\contprefix=\hbox{\tt \ \ \ \ \ \ \ \ \ \ \ \ \ \ \ }
\startm
\m{\vdash}\m{A}\m{\in}\m{{\rm V}}
\endm
\setbox\startprefix=\hbox{\tt \ \ inex\ \$p\ }
\setbox\contprefix=\hbox{\tt \ \ \ \ \ \ \ \ \ \ \ \ \ }
\startm
\m{\vdash}\m{(}\m{A}\m{\cap}\m{B}\m{)}\m{\in}\m{{\rm V}}
\endm
\vskip 1ex

\noindent Axiom of the Null Set\index{Axiom of the Null Set} proved from the
other axioms of ZF set theory. Corollary 5.16 of Takeuti and Zaring, p.~20.

\vskip 0.5ex
\setbox\startprefix=\hbox{\tt \ \ 0ex\ \$p\ }
\setbox\contprefix=\hbox{\tt \ \ \ \ \ \ \ \ \ \ \ \ }
\startm
\m{\vdash}\m{\varnothing}\m{\in}\m{{\rm V}}
\endm
\vskip 1ex

\noindent The Axiom of Pairing\index{Axiom of Pairing} proved from the other
axioms of ZF set theory.  Theorem 7.13 of Quine, p.~51.
\vskip 0.5ex
\setbox\startprefix=\hbox{\tt \ \ prex\ \$p\ }
\setbox\contprefix=\hbox{\tt \ \ \ \ \ \ \ \ \ \ \ \ \ \ }
\startm
\m{\vdash}\m{\{}\m{A}\m{,}\m{B}\m{\}}\m{\in}\m{{\rm V}}
\endm
\vskip 2ex

Next we will list some famous or important theorems that are proved in
the \texttt{set.mm} database.  None of them except \texttt{omex}
require the Axiom of Infinity, as you can verify with the \texttt{show
trace{\char`\_}back} Metamath command.

\vskip 2ex
\noindent The resolution of Russell's paradox\index{Russell's paradox}.  There
exists no set containing the set of all sets which are not members of
themselves.  Proposition 4.14 of Takeuti and Zaring, p.~14.

\vskip 0.5ex
\setbox\startprefix=\hbox{\tt \ \ ru\ \$p\ }
\setbox\contprefix=\hbox{\tt \ \ \ \ \ \ \ \ }
\startm
\m{\vdash}\m{\lnot}\m{\exists}\m{x}\m{\,x}\m{=}\m{\{}\m{y}\m{|}\m{\lnot}\m{y}
\m{\in}\m{y}\m{\}}
\endm
\vskip 1ex

\noindent Cantor's theorem\index{Cantor's theorem}.  No set can be mapped onto
its power set.  Compare Theorem 6B(b) of Enderton, p.~132.

\vskip 0.5ex
\setbox\startprefix=\hbox{\tt \ \ canth.1\ \$e\ }
\setbox\contprefix=\hbox{\tt \ \ \ \ \ \ \ \ \ \ \ \ \ }
\startm
\m{\vdash}\m{A}\m{\in}\m{{\rm V}}
\endm
\setbox\startprefix=\hbox{\tt \ \ canth\ \$p\ }
\setbox\contprefix=\hbox{\tt \ \ \ \ \ \ \ \ \ \ \ }
\startm
\m{\vdash}\m{\lnot}\m{F}\m{:}\m{A}\m{\raisebox{.5ex}{${\textstyle{\:}_{
\mbox{\footnotesize\rm {\ }}}}\atop{\textstyle{\longrightarrow}\atop{
\textstyle{}^{\mbox{\footnotesize\rm onto}}}}$}}\m{{\cal P}}\m{A}
\endm
\vskip 1ex

\noindent The Burali-Forti paradox\index{Burali-Forti paradox}.  No set
contains all ordinal numbers. Enderton, p.~194.  (Burali-Forti was one person,
not two.)

\vskip 0.5ex
\setbox\startprefix=\hbox{\tt \ \ onprc\ \$p\ }
\setbox\contprefix=\hbox{\tt \ \ \ \ \ \ \ \ \ \ \ \ }
\startm
\m{\vdash}\m{\lnot}\m{\mbox{\rm On}}\m{\in}\m{{\rm V}}
\endm
\vskip 1ex

\noindent Peano's postulates\index{Peano's postulates} for arithmetic.
Proposition 7.30 of Takeuti and Zaring, pp.~42--43.  The objects being
described are the members of $\omega$ i.e.\ the natural numbers 0, 1,
2,\ldots.  The successor\index{successor} operation suc means ``plus
one.''  \texttt{peano1} says that 0 (which is defined as the empty set)
is a natural number.  \texttt{peano2} says that if $A$ is a natural
number, so is $A+1$.  \texttt{peano3} says that 0 is not the successor
of any natural number.  \texttt{peano4} says that two natural numbers
are equal if and only if their successors are equal.  \texttt{peano5} is
essentially the same as mathematical induction.

\vskip 1ex
\setbox\startprefix=\hbox{\tt \ \ peano1\ \$p\ }
\setbox\contprefix=\hbox{\tt \ \ \ \ \ \ \ \ \ \ \ \ }
\startm
\m{\vdash}\m{\varnothing}\m{\in}\m{\omega}
\endm
\vskip 1.5ex

\setbox\startprefix=\hbox{\tt \ \ peano2\ \$p\ }
\setbox\contprefix=\hbox{\tt \ \ \ \ \ \ \ \ \ \ \ \ }
\startm
\m{\vdash}\m{(}\m{A}\m{\in}\m{\omega}\m{\rightarrow}\m{{\rm suc}}\m{A}\m{\in}%
\m{\omega}\m{)}
\endm
\vskip 1.5ex

\setbox\startprefix=\hbox{\tt \ \ peano3\ \$p\ }
\setbox\contprefix=\hbox{\tt \ \ \ \ \ \ \ \ \ \ \ \ }
\startm
\m{\vdash}\m{(}\m{A}\m{\in}\m{\omega}\m{\rightarrow}\m{\lnot}\m{{\rm suc}}%
\m{A}\m{=}\m{\varnothing}\m{)}
\endm
\vskip 1.5ex

\setbox\startprefix=\hbox{\tt \ \ peano4\ \$p\ }
\setbox\contprefix=\hbox{\tt \ \ \ \ \ \ \ \ \ \ \ \ }
\startm
\m{\vdash}\m{(}\m{(}\m{A}\m{\in}\m{\omega}\m{\wedge}\m{B}\m{\in}\m{\omega}%
\m{)}\m{\rightarrow}\m{(}\m{{\rm suc}}\m{A}\m{=}\m{{\rm suc}}\m{B}\m{%
\leftrightarrow}\m{A}\m{=}\m{B}\m{)}\m{)}
\endm
\vskip 1.5ex

\setbox\startprefix=\hbox{\tt \ \ peano5\ \$p\ }
\setbox\contprefix=\hbox{\tt \ \ \ \ \ \ \ \ \ \ \ \ }
\startm
\m{\vdash}\m{(}\m{(}\m{\varnothing}\m{\in}\m{A}\m{\wedge}\m{\forall}\m{x}\m{%
\in}\m{\omega}\m{(}\m{x}\m{\in}\m{A}\m{\rightarrow}\m{{\rm suc}}\m{x}\m{\in}%
\m{A}\m{)}\m{)}\m{\rightarrow}\m{\omega}\m{\subseteq}\m{A}\m{)}
\endm
\vskip 1.5ex

\noindent Finite Induction (mathematical induction).\index{finite
induction}\index{mathematical induction} The first hypothesis is the
basis and the second is the induction hypothesis.  Theorem Schema 22 of
Suppes, p.~136.

\vskip 0.5ex
\setbox\startprefix=\hbox{\tt \ \ findes.1\ \$e\ }
\setbox\contprefix=\hbox{\tt \ \ \ \ \ \ \ \ \ \ \ \ \ \ }
\startm
\m{\vdash}\m{[}\m{\varnothing}\m{/}\m{x}\m{]}\m{\varphi}
\endm
\setbox\startprefix=\hbox{\tt \ \ findes.2\ \$e\ }
\setbox\contprefix=\hbox{\tt \ \ \ \ \ \ \ \ \ \ \ \ \ \ }
\startm
\m{\vdash}\m{(}\m{x}\m{\in}\m{\omega}\m{\rightarrow}\m{(}\m{\varphi}\m{%
\rightarrow}\m{[}\m{{\rm suc}}\m{x}\m{/}\m{x}\m{]}\m{\varphi}\m{)}\m{)}
\endm
\setbox\startprefix=\hbox{\tt \ \ findes\ \$p\ }
\setbox\contprefix=\hbox{\tt \ \ \ \ \ \ \ \ \ \ \ \ }
\startm
\m{\vdash}\m{(}\m{x}\m{\in}\m{\omega}\m{\rightarrow}\m{\varphi}\m{)}
\endm
\vskip 1ex

\noindent Transfinite Induction with explicit substitution.  The first
hypothesis is the basis, the second is the induction hypothesis for
successors, and the third is the induction hypothesis for limit
ordinals.  Theorem Schema 4 of Suppes, p. 197.

\vskip 0.5ex
\setbox\startprefix=\hbox{\tt \ \ tfindes.1\ \$e\ }
\setbox\contprefix=\hbox{\tt \ \ \ \ \ \ \ \ \ \ \ \ \ \ \ }
\startm
\m{\vdash}\m{[}\m{\varnothing}\m{/}\m{x}\m{]}\m{\varphi}
\endm
\setbox\startprefix=\hbox{\tt \ \ tfindes.2\ \$e\ }
\setbox\contprefix=\hbox{\tt \ \ \ \ \ \ \ \ \ \ \ \ \ \ \ }
\startm
\m{\vdash}\m{(}\m{x}\m{\in}\m{{\rm On}}\m{\rightarrow}\m{(}\m{\varphi}\m{%
\rightarrow}\m{[}\m{{\rm suc}}\m{x}\m{/}\m{x}\m{]}\m{\varphi}\m{)}\m{)}
\endm
\setbox\startprefix=\hbox{\tt \ \ tfindes.3\ \$e\ }
\setbox\contprefix=\hbox{\tt \ \ \ \ \ \ \ \ \ \ \ \ \ \ \ }
\startm
\m{\vdash}\m{(}\m{{\rm Lim}}\m{y}\m{\rightarrow}\m{(}\m{\forall}\m{x}\m{\in}%
\m{y}\m{\varphi}\m{\rightarrow}\m{[}\m{y}\m{/}\m{x}\m{]}\m{\varphi}\m{)}\m{)}
\endm
\setbox\startprefix=\hbox{\tt \ \ tfindes\ \$p\ }
\setbox\contprefix=\hbox{\tt \ \ \ \ \ \ \ \ \ \ \ \ \ }
\startm
\m{\vdash}\m{(}\m{x}\m{\in}\m{{\rm On}}\m{\rightarrow}\m{\varphi}\m{)}
\endm
\vskip 1ex

\noindent Principle of Transfinite Recursion.\index{transfinite
recursion} Theorem 7.41 of Takeuti and Zaring, p.~47.  Transfinite
recursion is the key theorem that allows arithmetic of ordinals to be
rigorously defined, and has many other important uses as well.
Hypotheses \texttt{tfr.1} and \texttt{tfr.2} specify a certain (proper)
class $ F$.  The complicated definition of $ F$ is not important in
itself; what is important is that there be such an $ F$ with the
required properties, and we show this by displaying $ F$ explicitly.
\texttt{tfr1} states that $ F$ is a function whose domain is the set of
ordinal numbers.  \texttt{tfr2} states that any value of $ F$ is
completely determined by its previous values and the values of an
auxiliary function, $G$.  \texttt{tfr3} states that $ F$ is unique,
i.e.\ it is the only function that satisfies \texttt{tfr1} and
\texttt{tfr2}.  Note that $ f$ is an individual variable like $x$ and
$y$; it is just a mnemonic to remind us that $A$ is a collection of
functions.

\vskip 0.5ex
\setbox\startprefix=\hbox{\tt \ \ tfr.1\ \$e\ }
\setbox\contprefix=\hbox{\tt \ \ \ \ \ \ \ \ \ \ \ }
\startm
\m{\vdash}\m{A}\m{=}\m{\{}\m{f}\m{|}\m{\exists}\m{x}\m{\in}\m{{\rm On}}\m{(}%
\m{f}\m{{\rm Fn}}\m{x}\m{\wedge}\m{\forall}\m{y}\m{\in}\m{x}\m{(}\m{f}\m{`}%
\m{y}\m{)}\m{=}\m{(}\m{G}\m{`}\m{(}\m{f}\m{\restriction}\m{y}\m{)}\m{)}\m{)}%
\m{\}}
\endm
\setbox\startprefix=\hbox{\tt \ \ tfr.2\ \$e\ }
\setbox\contprefix=\hbox{\tt \ \ \ \ \ \ \ \ \ \ \ }
\startm
\m{\vdash}\m{F}\m{=}\m{\bigcup}\m{A}
\endm
\setbox\startprefix=\hbox{\tt \ \ tfr1\ \$p\ }
\setbox\contprefix=\hbox{\tt \ \ \ \ \ \ \ \ \ \ }
\startm
\m{\vdash}\m{F}\m{{\rm Fn}}\m{{\rm On}}
\endm
\setbox\startprefix=\hbox{\tt \ \ tfr2\ \$p\ }
\setbox\contprefix=\hbox{\tt \ \ \ \ \ \ \ \ \ \ }
\startm
\m{\vdash}\m{(}\m{z}\m{\in}\m{{\rm On}}\m{\rightarrow}\m{(}\m{F}\m{`}\m{z}%
\m{)}\m{=}\m{(}\m{G}\m{`}\m{(}\m{F}\m{\restriction}\m{z}\m{)}\m{)}\m{)}
\endm
\setbox\startprefix=\hbox{\tt \ \ tfr3\ \$p\ }
\setbox\contprefix=\hbox{\tt \ \ \ \ \ \ \ \ \ \ }
\startm
\m{\vdash}\m{(}\m{(}\m{B}\m{{\rm Fn}}\m{{\rm On}}\m{\wedge}\m{\forall}\m{x}\m{%
\in}\m{{\rm On}}\m{(}\m{B}\m{`}\m{x}\m{)}\m{=}\m{(}\m{G}\m{`}\m{(}\m{B}\m{%
\restriction}\m{x}\m{)}\m{)}\m{)}\m{\rightarrow}\m{B}\m{=}\m{F}\m{)}
\endm
\vskip 1ex

\noindent The existence of omega (the class of natural numbers).\index{natural
number}\index{omega ($\omega$)}\index{Axiom of Infinity}  Axiom 7 of Takeuti
and Zaring, p.~43.  (This is the only theorem in this section requiring the
Axiom of Infinity.)

\vskip 0.5ex
\setbox\startprefix=\hbox{\tt \
\ omex\ \$p\ }
\setbox\contprefix=\hbox{\tt \ \ \ \ \ \ \ \ \ \ }
\startm
\m{\vdash}\m{\omega}\m{\in}\m{{\rm V}}
\endm
%\vskip 2ex


\section{Axioms for Real and Complex Numbers}\label{real}
\index{real and complex numbers!axioms for}

This section presents the axioms for real and complex numbers, along
with some commentary about them.  Analysis
textbooks implicitly or explicitly use these axioms or their equivalents
as their starting point.  In the database \texttt{set.mm}, we define real
and complex numbers as (rather complicated) specific sets and derive these
axioms as {\em theorems} from the axioms of ZF set theory, using a method
called Dedekind cuts.  We omit the details of this construction, which you can
follow if you wish using the \texttt{set.mm} database in conjunction with the
textbooks referenced therein.

Once we prove those theorems, we then restate these proven theorems as axioms.
This lets us easily identify which axioms are needed for a particular complex number proof, without the obfuscation of the set theory used to derive them.
As a result,
the construction is actually unimportant other
than to show that sets exist that satisfy the axioms, and thus that the axioms
are consistent if ZF set theory is consistent.  When working with real numbers
you can think of them as being the actual sets resulting
from the construction (for definiteness), or you can
think of them as otherwise unspecified sets that happen to satisfy the axioms.
The derivation is not easy, but the fact that it works is quite remarkable
and lends support to the idea that ZFC set theory is all we need to
provide a foundation for essentially all of mathematics.

\needspace{3\baselineskip}
\subsection{The Axioms for Real and Complex Numbers Themselves}\label{realactual}

For the axioms we are given (or postulate) 8 classes:  $\mathbb{C}$ (the
set of complex numbers), $\mathbb{R}$ (the set of real numbers, a subset
of $\mathbb{C}$), $0$ (zero), $1$ (one), $i$ (square root of
$-1$), $+$ (plus), $\cdot$ (times), and
$<_{\mathbb{R}}$ (less than for just the real numbers).
Subtraction and division are defined terms and are not part of the
axioms; for their definitions see \texttt{set.mm}.

Note that the notation $(A+B)$ (and similarly $(A\cdot B)$) specifies a class
called an {\em operation},\index{operation} and is the function value of the
class $+$ at ordered pair $\langle A,B \rangle$.  An operation is defined by
statement \texttt{df-opr} on p.~\pageref{dfopr}.
The notation $A <_{\mathbb{R}} B$ specifies a
wff called a {\em binary relation}\index{binary relation} and means $\langle A,B \rangle \in \,<_{\mathbb{R}}$, as defined by statement \texttt{df-br} on p.~\pageref{dfbr}.

Our set of 8 given classes is assumed to satisfy the following 22 axioms
(in the axioms listed below, $<$ really means $<_{\mathbb{R}}$).

\vskip 2ex

\noindent 1. The real numbers are a subset of the complex numbers.

%\vskip 0.5ex
\setbox\startprefix=\hbox{\tt \ \ ax-resscn\ \$p\ }
\setbox\contprefix=\hbox{\tt \ \ \ \ \ \ \ \ \ \ \ \ \ \ }
\startm
\m{\vdash}\m{\mathbb{R}}\m{\subseteq}\m{\mathbb{C}}
\endm
%\vskip 1ex

\noindent 2. One is a complex number.

%\vskip 0.5ex
\setbox\startprefix=\hbox{\tt \ \ ax-1cn\ \$p\ }
\setbox\contprefix=\hbox{\tt \ \ \ \ \ \ \ \ \ \ \ }
\startm
\m{\vdash}\m{1}\m{\in}\m{\mathbb{C}}
\endm
%\vskip 1ex

\noindent 3. The imaginary number $i$ is a complex number.

%\vskip 0.5ex
\setbox\startprefix=\hbox{\tt \ \ ax-icn\ \$p\ }
\setbox\contprefix=\hbox{\tt \ \ \ \ \ \ \ \ \ \ \ }
\startm
\m{\vdash}\m{i}\m{\in}\m{\mathbb{C}}
\endm
%\vskip 1ex

\noindent 4. Complex numbers are closed under addition.

%\vskip 0.5ex
\setbox\startprefix=\hbox{\tt \ \ ax-addcl\ \$p\ }
\setbox\contprefix=\hbox{\tt \ \ \ \ \ \ \ \ \ \ \ \ \ }
\startm
\m{\vdash}\m{(}\m{(}\m{A}\m{\in}\m{\mathbb{C}}\m{\wedge}\m{B}\m{\in}\m{\mathbb{C}}%
\m{)}\m{\rightarrow}\m{(}\m{A}\m{+}\m{B}\m{)}\m{\in}\m{\mathbb{C}}\m{)}
\endm
%\vskip 1ex

\noindent 5. Real numbers are closed under addition.

%\vskip 0.5ex
\setbox\startprefix=\hbox{\tt \ \ ax-addrcl\ \$p\ }
\setbox\contprefix=\hbox{\tt \ \ \ \ \ \ \ \ \ \ \ \ \ \ }
\startm
\m{\vdash}\m{(}\m{(}\m{A}\m{\in}\m{\mathbb{R}}\m{\wedge}\m{B}\m{\in}\m{\mathbb{R}}%
\m{)}\m{\rightarrow}\m{(}\m{A}\m{+}\m{B}\m{)}\m{\in}\m{\mathbb{R}}\m{)}
\endm
%\vskip 1ex

\noindent 6. Complex numbers are closed under multiplication.

%\vskip 0.5ex
\setbox\startprefix=\hbox{\tt \ \ ax-mulcl\ \$p\ }
\setbox\contprefix=\hbox{\tt \ \ \ \ \ \ \ \ \ \ \ \ \ }
\startm
\m{\vdash}\m{(}\m{(}\m{A}\m{\in}\m{\mathbb{C}}\m{\wedge}\m{B}\m{\in}\m{\mathbb{C}}%
\m{)}\m{\rightarrow}\m{(}\m{A}\m{\cdot}\m{B}\m{)}\m{\in}\m{\mathbb{C}}\m{)}
\endm
%\vskip 1ex

\noindent 7. Real numbers are closed under multiplication.

%\vskip 0.5ex
\setbox\startprefix=\hbox{\tt \ \ ax-mulrcl\ \$p\ }
\setbox\contprefix=\hbox{\tt \ \ \ \ \ \ \ \ \ \ \ \ \ \ }
\startm
\m{\vdash}\m{(}\m{(}\m{A}\m{\in}\m{\mathbb{R}}\m{\wedge}\m{B}\m{\in}\m{\mathbb{R}}%
\m{)}\m{\rightarrow}\m{(}\m{A}\m{\cdot}\m{B}\m{)}\m{\in}\m{\mathbb{R}}\m{)}
\endm
%\vskip 1ex

\noindent 8. Multiplication of complex numbers is commutative.

%\vskip 0.5ex
\setbox\startprefix=\hbox{\tt \ \ ax-mulcom\ \$p\ }
\setbox\contprefix=\hbox{\tt \ \ \ \ \ \ \ \ \ \ \ \ \ \ }
\startm
\m{\vdash}\m{(}\m{(}\m{A}\m{\in}\m{\mathbb{C}}\m{\wedge}\m{B}\m{\in}\m{\mathbb{C}}%
\m{)}\m{\rightarrow}\m{(}\m{A}\m{\cdot}\m{B}\m{)}\m{=}\m{(}\m{B}\m{\cdot}\m{A}%
\m{)}\m{)}
\endm
%\vskip 1ex

\noindent 9. Addition of complex numbers is associative.

%\vskip 0.5ex
\setbox\startprefix=\hbox{\tt \ \ ax-addass\ \$p\ }
\setbox\contprefix=\hbox{\tt \ \ \ \ \ \ \ \ \ \ \ \ \ \ }
\startm
\m{\vdash}\m{(}\m{(}\m{A}\m{\in}\m{\mathbb{C}}\m{\wedge}\m{B}\m{\in}\m{\mathbb{C}}%
\m{\wedge}\m{C}\m{\in}\m{\mathbb{C}}\m{)}\m{\rightarrow}\m{(}\m{(}\m{A}\m{+}%
\m{B}\m{)}\m{+}\m{C}\m{)}\m{=}\m{(}\m{A}\m{+}\m{(}\m{B}\m{+}\m{C}\m{)}\m{)}%
\m{)}
\endm
%\vskip 1ex

\noindent 10. Multiplication of complex numbers is associative.

%\vskip 0.5ex
\setbox\startprefix=\hbox{\tt \ \ ax-mulass\ \$p\ }
\setbox\contprefix=\hbox{\tt \ \ \ \ \ \ \ \ \ \ \ \ \ \ }
\startm
\m{\vdash}\m{(}\m{(}\m{A}\m{\in}\m{\mathbb{C}}\m{\wedge}\m{B}\m{\in}\m{\mathbb{C}}%
\m{\wedge}\m{C}\m{\in}\m{\mathbb{C}}\m{)}\m{\rightarrow}\m{(}\m{(}\m{A}\m{\cdot}%
\m{B}\m{)}\m{\cdot}\m{C}\m{)}\m{=}\m{(}\m{A}\m{\cdot}\m{(}\m{B}\m{\cdot}\m{C}%
\m{)}\m{)}\m{)}
\endm
%\vskip 1ex

\noindent 11. Multiplication distributes over addition for complex numbers.

%\vskip 0.5ex
\setbox\startprefix=\hbox{\tt \ \ ax-distr\ \$p\ }
\setbox\contprefix=\hbox{\tt \ \ \ \ \ \ \ \ \ \ \ \ \ }
\startm
\m{\vdash}\m{(}\m{(}\m{A}\m{\in}\m{\mathbb{C}}\m{\wedge}\m{B}\m{\in}\m{\mathbb{C}}%
\m{\wedge}\m{C}\m{\in}\m{\mathbb{C}}\m{)}\m{\rightarrow}\m{(}\m{A}\m{\cdot}\m{(}%
\m{B}\m{+}\m{C}\m{)}\m{)}\m{=}\m{(}\m{(}\m{A}\m{\cdot}\m{B}\m{)}\m{+}\m{(}%
\m{A}\m{\cdot}\m{C}\m{)}\m{)}\m{)}
\endm
%\vskip 1ex

\noindent 12. The square of $i$ equals $-1$ (expressed as $i$-squared plus 1 is
0).

%\vskip 0.5ex
\setbox\startprefix=\hbox{\tt \ \ ax-i2m1\ \$p\ }
\setbox\contprefix=\hbox{\tt \ \ \ \ \ \ \ \ \ \ \ \ }
\startm
\m{\vdash}\m{(}\m{(}\m{i}\m{\cdot}\m{i}\m{)}\m{+}\m{1}\m{)}\m{=}\m{0}
\endm
%\vskip 1ex

\noindent 13. One and zero are distinct.

%\vskip 0.5ex
\setbox\startprefix=\hbox{\tt \ \ ax-1ne0\ \$p\ }
\setbox\contprefix=\hbox{\tt \ \ \ \ \ \ \ \ \ \ \ \ }
\startm
\m{\vdash}\m{1}\m{\ne}\m{0}
\endm
%\vskip 1ex

\noindent 14. One is an identity element for real multiplication.

%\vskip 0.5ex
\setbox\startprefix=\hbox{\tt \ \ ax-1rid\ \$p\ }
\setbox\contprefix=\hbox{\tt \ \ \ \ \ \ \ \ \ \ \ }
\startm
\m{\vdash}\m{(}\m{A}\m{\in}\m{\mathbb{R}}\m{\rightarrow}\m{(}\m{A}\m{\cdot}\m{1}%
\m{)}\m{=}\m{A}\m{)}
\endm
%\vskip 1ex

\noindent 15. Every real number has a negative.

%\vskip 0.5ex
\setbox\startprefix=\hbox{\tt \ \ ax-rnegex\ \$p\ }
\setbox\contprefix=\hbox{\tt \ \ \ \ \ \ \ \ \ \ \ \ \ \ }
\startm
\m{\vdash}\m{(}\m{A}\m{\in}\m{\mathbb{R}}\m{\rightarrow}\m{\exists}\m{x}\m{\in}%
\m{\mathbb{R}}\m{(}\m{A}\m{+}\m{x}\m{)}\m{=}\m{0}\m{)}
\endm
%\vskip 1ex

\noindent 16. Every nonzero real number has a reciprocal.

%\vskip 0.5ex
\setbox\startprefix=\hbox{\tt \ \ ax-rrecex\ \$p\ }
\setbox\contprefix=\hbox{\tt \ \ \ \ \ \ \ \ \ \ \ \ \ \ }
\startm
\m{\vdash}\m{(}\m{A}\m{\in}\m{\mathbb{R}}\m{\rightarrow}\m{(}\m{A}\m{\ne}\m{0}%
\m{\rightarrow}\m{\exists}\m{x}\m{\in}\m{\mathbb{R}}\m{(}\m{A}\m{\cdot}%
\m{x}\m{)}\m{=}\m{1}\m{)}\m{)}
\endm
%\vskip 1ex

\noindent 17. A complex number can be expressed in terms of two reals.

%\vskip 0.5ex
\setbox\startprefix=\hbox{\tt \ \ ax-cnre\ \$p\ }
\setbox\contprefix=\hbox{\tt \ \ \ \ \ \ \ \ \ \ \ \ }
\startm
\m{\vdash}\m{(}\m{A}\m{\in}\m{\mathbb{C}}\m{\rightarrow}\m{\exists}\m{x}\m{\in}%
\m{\mathbb{R}}\m{\exists}\m{y}\m{\in}\m{\mathbb{R}}\m{A}\m{=}\m{(}\m{x}\m{+}\m{(}%
\m{y}\m{\cdot}\m{i}\m{)}\m{)}\m{)}
\endm
%\vskip 1ex

\noindent 18. Ordering on reals satisfies strict trichotomy.

%\vskip 0.5ex
\setbox\startprefix=\hbox{\tt \ \ ax-pre-lttri\ \$p\ }
\setbox\contprefix=\hbox{\tt \ \ \ \ \ \ \ \ \ \ \ \ \ }
\startm
\m{\vdash}\m{(}\m{(}\m{A}\m{\in}\m{\mathbb{R}}\m{\wedge}\m{B}\m{\in}\m{\mathbb{R}}%
\m{)}\m{\rightarrow}\m{(}\m{A}\m{<}\m{B}\m{\leftrightarrow}\m{\lnot}\m{(}\m{A}%
\m{=}\m{B}\m{\vee}\m{B}\m{<}\m{A}\m{)}\m{)}\m{)}
\endm
%\vskip 1ex

\noindent 19. Ordering on reals is transitive.

%\vskip 0.5ex
\setbox\startprefix=\hbox{\tt \ \ ax-pre-lttrn\ \$p\ }
\setbox\contprefix=\hbox{\tt \ \ \ \ \ \ \ \ \ \ \ \ \ }
\startm
\m{\vdash}\m{(}\m{(}\m{A}\m{\in}\m{\mathbb{R}}\m{\wedge}\m{B}\m{\in}\m{\mathbb{R}}%
\m{\wedge}\m{C}\m{\in}\m{\mathbb{R}}\m{)}\m{\rightarrow}\m{(}\m{(}\m{A}\m{<}%
\m{B}\m{\wedge}\m{B}\m{<}\m{C}\m{)}\m{\rightarrow}\m{A}\m{<}\m{C}\m{)}\m{)}
\endm
%\vskip 1ex

\noindent 20. Ordering on reals is preserved after addition to both sides.

%\vskip 0.5ex
\setbox\startprefix=\hbox{\tt \ \ ax-pre-ltadd\ \$p\ }
\setbox\contprefix=\hbox{\tt \ \ \ \ \ \ \ \ \ \ \ \ \ }
\startm
\m{\vdash}\m{(}\m{(}\m{A}\m{\in}\m{\mathbb{R}}\m{\wedge}\m{B}\m{\in}\m{\mathbb{R}}%
\m{\wedge}\m{C}\m{\in}\m{\mathbb{R}}\m{)}\m{\rightarrow}\m{(}\m{A}\m{<}\m{B}\m{%
\rightarrow}\m{(}\m{C}\m{+}\m{A}\m{)}\m{<}\m{(}\m{C}\m{+}\m{B}\m{)}\m{)}\m{)}
\endm
%\vskip 1ex

\noindent 21. The product of two positive reals is positive.

%\vskip 0.5ex
\setbox\startprefix=\hbox{\tt \ \ ax-pre-mulgt0\ \$p\ }
\setbox\contprefix=\hbox{\tt \ \ \ \ \ \ \ \ \ \ \ \ \ \ }
\startm
\m{\vdash}\m{(}\m{(}\m{A}\m{\in}\m{\mathbb{R}}\m{\wedge}\m{B}\m{\in}\m{\mathbb{R}}%
\m{)}\m{\rightarrow}\m{(}\m{(}\m{0}\m{<}\m{A}\m{\wedge}\m{0}%
\m{<}\m{B}\m{)}\m{\rightarrow}\m{0}\m{<}\m{(}\m{A}\m{\cdot}\m{B}\m{)}%
\m{)}\m{)}
\endm
%\vskip 1ex

\noindent 22. A non-empty, bounded-above set of reals has a supremum.

%\vskip 0.5ex
\setbox\startprefix=\hbox{\tt \ \ ax-pre-sup\ \$p\ }
\setbox\contprefix=\hbox{\tt \ \ \ \ \ \ \ \ \ \ \ }
\startm
\m{\vdash}\m{(}\m{(}\m{A}\m{\subseteq}\m{\mathbb{R}}\m{\wedge}\m{A}\m{\ne}\m{%
\varnothing}\m{\wedge}\m{\exists}\m{x}\m{\in}\m{\mathbb{R}}\m{\forall}\m{y}\m{%
\in}\m{A}\m{\,y}\m{<}\m{x}\m{)}\m{\rightarrow}\m{\exists}\m{x}\m{\in}\m{%
\mathbb{R}}\m{(}\m{\forall}\m{y}\m{\in}\m{A}\m{\lnot}\m{x}\m{<}\m{y}\m{\wedge}\m{%
\forall}\m{y}\m{\in}\m{\mathbb{R}}\m{(}\m{y}\m{<}\m{x}\m{\rightarrow}\m{\exists}%
\m{z}\m{\in}\m{A}\m{\,y}\m{<}\m{z}\m{)}\m{)}\m{)}
\endm

% NOTE: The \m{...} expressions above could be represented as
% $ \vdash ( ( A \subseteq \mathbb{R} \wedge A \ne \varnothing \wedge \exists x \in \mathbb{R} \forall y \in A \,y < x ) \rightarrow \exists x \in \mathbb{R} ( \forall y \in A \lnot x < y \wedge \forall y \in \mathbb{R} ( y < x \rightarrow \exists z \in A \,y < z ) ) ) $

\vskip 2ex

This completes the set of axioms for real and complex numbers.  You may
wish to look at how subtraction, division, and decimal numbers
are defined in \texttt{set.mm}, and for fun look at the proof of $2+
2 = 4$ (theorem \texttt{2p2e4} in \texttt{set.mm})
as discussed in section \ref{2p2e4}.

In \texttt{set.mm} we define the non-negative integers $\mathbb{N}$, the integers
$\mathbb{Z}$, and the rationals $\mathbb{Q}$ as subsets of $\mathbb{R}$.  This leads
to the nice inclusion $\mathbb{N} \subseteq \mathbb{Z} \subseteq \mathbb{Q} \subseteq
\mathbb{R} \subseteq \mathbb{C}$, giving us a uniform framework in which, for
example, a property such as commutativity of complex number addition
automatically applies to integers.  The natural numbers $\mathbb{N}$
are different from the set $\omega$ we defined earlier, but both satisfy
Peano's postulates.

\subsection{Complex Number Axioms in Analysis Texts}

Most analysis texts construct complex numbers as ordered pairs of reals,
leading to construction-dependent properties that satisfy these axioms
but are not stated in their pure form.  (This is also done in
\texttt{set.mm} but our axioms are extracted from that construction.)
Other texts will simply state that $\mathbb{R}$ is a ``complete ordered
subfield of $\mathbb{C}$,'' leading to redundant axioms when this phrase
is completely expanded out.  In fact I have not seen a text with the
axioms in the explicit form above.
None of these axioms is unique individually, but this carefully worked out
collection of axioms is the result of years of work
by the Metamath community.

\subsection{Eliminating Unnecessary Complex Number Axioms}

We once had more axioms for real and complex numbers, but over years of time
we (the Metamath community)
have found ways to eliminate them (by proving them from other axioms)
or weaken them (by making weaker claims without reducing what
can be proved).
In particular, here are statements that used to be complex number
axioms but have since been formally proven (with Metamath) to be redundant:

\begin{itemize}
\item
  $\mathbb{C} \in V$.
  At one time this was listed as a ``complex number axiom.''
  However, this is not properly speaking a complex number axiom,
  and in any case its proof uses axioms of set theory.
  Proven redundant by Mario Carneiro\index{Carneiro, Mario} on
  17-Nov-2014 (see \texttt{axcnex}).
\item
  $((A \in \mathbb{C} \land B \in \mathbb{C}$) $\rightarrow$
  $(A + B) = (B + A))$.
  Proved redundant by Eric Schmidt\index{Schmidt, Eric} on 19-Jun-2012,
  and formalized by Scott Fenton\index{Fenton, Scott} on 3-Jan-2013
  (see \texttt{addcom}).
\item
  $(A \in \mathbb{C} \rightarrow (A + 0) = A)$.
  Proved redundant by Eric Schmidt on 19-Jun-2012,
  and formalized by Scott Fenton on 3-Jan-2013
  (see \texttt{addid1}).
\item
  $(A \in \mathbb{C} \rightarrow \exists x \in \mathbb{C} (A + x) = 0)$.
  Proved redundant by Eric Schmidt and formalized on 21-May-2007
  (see \texttt{cnegex}).
\item
  $((A \in \mathbb{C} \land A \ne 0) \rightarrow \exists x \in \mathbb{C} (A \cdot x) = 1)$.
  Proved redundant by Eric Schmidt and formalized on 22-May-2007
  (see \texttt{recex}).
\item
  $0 \in \mathbb{R}$.
  Proved redundant by Eric Schmidt on 19-Feb-2005 and formalized 21-May-2007
  (see \texttt{0re}).
\end{itemize}

We could eliminate 0 as an axiomatic object by defining it as
$( ( i \cdot i ) + 1 )$
and replacing it with this expression throughout the axioms. If this
is done, axiom ax-i2m1 becomes redundant. However, the remaining axioms
would become longer and less intuitive.

Eric Schmidt's paper analyzing this axiom system \cite{Schmidt}
presented a proof that these remaining axioms,
with the possible exception of ax-mulcom, are independent of the others.
It is currently an open question if ax-mulcom is independent of the others.

\section{Two Plus Two Equals Four}\label{2p2e4}

Here is a proof that $2 + 2 = 4$, as proven in the theorem \texttt{2p2e4}
in the database \texttt{set.mm}.
This is a useful demonstration of what a Metamath proof can look like.
This proof may have more steps than you're used to, but each step is rigorously
proven all the way back to the axioms of logic and set theory.
This display was originally generated by the Metamath program
as an {\sc HTML} file.

In the table showing the proof ``Step'' is the sequential step number,
while its associated ``Expression'' is an expression that we have proved.
``Ref'' is the name of a theorem or axiom that justifies that expression,
and ``Hyp'' refers to previous steps (if any) that the theorem or axiom
needs so that we can use it.  Expressions are indented further than
the expressions that depend on them to show their interdependencies.

\begin{table}[!htbp]
\caption{Two plus two equals four}
\begin{tabular}{lllll}
\textbf{Step} & \textbf{Hyp} & \textbf{Ref} & \textbf{Expression} & \\
1  &       & df-2    & $ \; \; \vdash 2 = 1 + 1$  & \\
2  & 1     & oveq2i  & $ \; \vdash (2 + 2) = (2 + (1 + 1))$ & \\
3  &       & df-4    & $ \; \; \vdash 4 = (3 + 1)$ & \\
4  &       & df-3    & $ \; \; \; \vdash 3 = (2 + 1)$ & \\
5  & 4     & oveq1i  & $ \; \; \vdash (3 + 1) = ((2 + 1) + 1)$ & \\
6  &       & 2cn     & $ \; \; \; \vdash 2 \in \mathbb{C}$ & \\
7  &       & ax-1cn  & $ \; \; \; \vdash 1 \in \mathbb{C}$ & \\
8  & 6,7,7 & addassi & $ \; \; \vdash ((2 + 1) + 1) = (2 + (1 + 1))$ & \\
9  & 3,5,8 & 3eqtri  & $ \; \vdash 4 = (2 + (1 + 1))$ & \\
10 & 2,9   & eqtr4i  & $ \vdash (2 + 2) = 4$ & \\
\end{tabular}
\end{table}

Step 1 says that we can assert that $2 = 1 + 1$ because it is
justified by \texttt{df-2}.
What is \texttt{df-2}?
It is simply the definition of $2$, which in our system is defined as being
equal to $1 + 1$.  This shows how we can use definitions in proofs.

Look at Step 2 of the proof. In the Ref column, we see that it references
a previously proved theorem, \texttt{oveq2i}.
It turns out that
theorem \texttt{oveq2i} requires a
hypothesis, and in the Hyp column of Step 2 we indicate that Step 1 will
satisfy (match) this hypothesis.
If we looked at \texttt{oveq2i}
we would find that it proves that given some hypothesis
$A = B$, we can prove that $( C F A ) = ( C F B )$.
If we use \texttt{oveq2i} and apply step 1's result as the hypothesis,
that will mean that $A = 2$ and $B = ( 1 + 1 )$ within this use of
\texttt{oveq2i}.
We are free to select any value of $C$ and $F$ (subject to syntax constraints),
so we are free to select $C = 2$ and $F = +$,
producing our desired result,
$ (2 + 2) = (2 + (1 + 1))$.

Step 2 is an example of substitution.
In the end, every step in every proof uses only this one substitution rule.
All the rules of logic, and all the axioms, are expressed so that
they can be used via this one substitution rule.
So once you master substitution, you can master every Metamath proof,
no exceptions.

Each step is clear and can be immediately checked.
In the {\sc HTML} display you can even click on each reference to see why it is
justified, making it easy to see why the proof works.

\section{Deduction}\label{deduction}

Strictly speaking,
a deduction (also called an inference) is a kind of statement that needs
some hypotheses to be true in order for its conclusion to be true.
A theorem, on the other hand, has no hypotheses.
Informally we often call both of them theorems, but in this section we
will stick to the strict definitions.

It sometimes happens that we have proved a deduction of the form
$\varphi \Rightarrow \psi$\index{$\Rightarrow$}
(given hypothesis $\varphi$ we can prove $\psi$)
and we want to then prove a theorem of the form
$\varphi \rightarrow \psi$.

Converting a deduction (which uses a hypothesis) into a theorem
(which does not) is not as simple as you might think.
The deduction says, ``if we can prove $\varphi$ then we can prove $\psi$,''
which is in some sense weaker than saying
``$\varphi$ implies $\psi$.''
There is no axiom of logic that permits us to directly obtain the theorem
given the deduction.\footnote{
The conversion of a deduction to a theorem does not even hold in general
for quantum propositional calculus,
which is a weak subset of classical propositional calculus.
It has been shown that adding the Standard Deduction Theorem (discussed below)
to quantum propositional calculus turns it into classical
propositional calculus!
}

This is in contrast to going the other way.
If we have the theorem ($\varphi \rightarrow \psi$),
it is easy to recover the deduction
($\varphi \Rightarrow \psi$)
using modus ponens\index{modus ponens}
(\texttt{ax-mp}; see section \ref{axmp}).

In the following subsections we first discuss the standard deduction theorem
(the traditional but awkward way to convert deductions into theorems) and
the weak deduction theorem (a limited version of the standard deduction
theorem that is easier to use and was once widely used in
\texttt{set.mm}\index{set theory database (\texttt{set.mm})}).
In section \ref{deductionstyle} we discuss
deduction style, the newer approach we now recommend in most cases.
Deduction style uses ``deduction form,'' a form that
prefixes each hypothesis (other than definitions) and the
conclusion with a universal antecedent (``$\varphi \rightarrow$'').
Deduction style is widely used in \texttt{set.mm},
so it is useful to understand it and \textit{why} it is widely used.
Section \ref{naturaldeduction}
briefly discusses our approach for using natural deduction
within \texttt{set.mm},
as that approach is deeply related to deduction style.
We conclude with a summary of the strengths of
our approach, which we believe are compelling.

\subsection{The Standard Deduction Theorem}\label{standarddeductiontheorem}

It is possible to make use of information
contained in the deduction or its proof to assist us with the proof of
the related theorem.
In traditional logic books, there is a metatheorem called the
Deduction Theorem\index{Deduction Theorem}\index{Standard Deduction Theorem},
discovered independently by Herbrand and Tarski around 1930.
The Deduction Theorem, which we often call the Standard Deduction Theorem,
provides an algorithm for constructing a proof of a theorem from the
proof of its corresponding deduction. See, for example,
\cite[p.~56]{Margaris}\index{Margaris, Angelo}.
To construct a proof for a theorem, the
algorithm looks at each step in the proof of the original deduction and
rewrites the step with several steps wherein the hypothesis is eliminated
and becomes an antecedent.

In ordinary mathematics, no one actually carries out the algorithm,
because (in its most basic form) it involves an exponential explosion of
the number of proof steps as more hypotheses are eliminated. Instead,
the Standard Deduction Theorem is invoked simply to claim that it can
be done in principle, without actually doing it.
What's more, the algorithm is not as simple as it might first appear
when applying it rigorously.
There is a subtle restriction on the Standard Deduction Theorem
that must be taken into account involving the axiom of generalization
when working with predicate calculus (see the literature for more detail).

One of the goals of Metamath is to let you plainly see, with as few
underlying concepts as possible, how mathematics can be derived directly
from the axioms, and not indirectly according to some hidden rules
buried inside a program or understood only by logicians. If we added
the Standard Deduction Theorem to the language and proof verifier,
that would greatly complicate both and largely defeat Metamath's goal
of simplicity. In principle, we could show direct proofs by expanding
out the proof steps generated by the algorithm of the Standard Deduction
Theorem, but that is not feasible in practice because the number of proof
steps quickly becomes huge, even astronomical.
Since the algorithm of the Standard Deduction Theorem is driven by the proof,
we would have to go through that proof
all over again---starting from axioms---in order to obtain the theorem form.
In terms of proof length, there would be no savings over just
proving the theorem directly instead of first proving the deduction form.

\subsection{Weak Deduction Theorem}\label{weakdeductiontheorem}

We have developed
a more efficient method for proving a theorem from a deduction
that can be used instead of the Standard Deduction Theorem
in many (but not all) cases.
We call this more efficient method the
Weak Deduction Theorem\index{Weak Deduction Theorem}.\footnote{
There is also an unrelated ``Weak Deduction Theorem''
in the field of relevance logic, so to avoid confusion we could call
ours the ``Weak Deduction Theorem for Classical Logic.''}
Unlike the Standard Deduction Theorem, the Weak Deduction Theorem produces the
theorem directly from a special substitution instance of the deduction,
using a small, fixed number of steps roughly proportional to the length
of the final theorem.

If you come to a proof referencing the Weak Deduction Theorem
\texttt{dedth} (or one of its variants \texttt{dedthxx}),
here is how to follow the proof without getting into the details:
just click on the theorem referenced in the step
just before the reference to \texttt{dedth} and ignore everything else.
Theorem \texttt{dedth} simply turns a hypothesis into an antecedent
(i.e. the hypothesis followed by $\rightarrow$
is placed in front of the assertion, and the hypothesis
itself is eliminated) given certain conditions.

The Weak Deduction Theorem
eliminates a hypothesis $\varphi$, making it become an antecedent.
It does this by proving an expression
$ \varphi \rightarrow \psi $ given two hypotheses:
(1)
$ ( A = {\rm if} ( \varphi , A , B ) \rightarrow ( \varphi \leftrightarrow \chi ) ) $
and
(2) $\chi$.
Note that it requires that a proof exists for $\varphi$ when the class variable
$A$ is replaced with a specific class $B$. The hypothesis $\chi$
should be assigned to the inference.
You can see the details of the proof of the Weak Deduction Theorem
in theorem \texttt{dedth}.

The Weak Deduction Theorem
is probably easier to understand by studying proofs that make use of it.
For example, let's look at the proof of \texttt{renegcl}, which proves that
$ \vdash ( A \in \mathbb{R} \rightarrow - A \in \mathbb{R} )$:

\needspace{4\baselineskip}
\begin{longtabu} {l l l X}
\textbf{Step} & \textbf{Hyp} & \textbf{Ref} & \textbf{Expression} \\
  1 &  & negeq &
  $\vdash$ $($ $A$ $=$ ${\rm if}$ $($ $A$ $\in$ $\mathbb{R}$ $,$ $A$ $,$ $1$ $)$ $\rightarrow$
  $\textrm{-}$ $A$ $=$ $\textrm{-}$ ${\rm if}$ $($ $A$ $\in$ $\mathbb{R}$
  $,$ $A$ $,$ $1$ $)$ $)$ \\
 2 & 1 & eleq1d &
    $\vdash$ $($ $A$ $=$ ${\rm if}$ $($ $A$ $\in$ $\mathbb{R}$ $,$ $A$ $,$ $1$ $)$ $\rightarrow$ $($
    $\textrm{-}$ $A$ $\in$ $\mathbb{R}$ $\leftrightarrow$
    $\textrm{-}$ ${\rm if}$ $($ $A$ $\in$ $\mathbb{R}$ $,$ $A$ $,$ $1$ $)$ $\in$
    $\mathbb{R}$ $)$ $)$ \\
 3 &  & 1re & $\vdash 1 \in \mathbb{R}$ \\
 4 & 3 & elimel &
   $\vdash {\rm if} ( A \in \mathbb{R} , A , 1 ) \in \mathbb{R}$ \\
 5 & 4 & renegcli &
   $\vdash \textrm{-} {\rm if} ( A \in \mathbb{R} , A , 1 ) \in \mathbb{R}$ \\
 6 & 2,5 & dedth &
   $\vdash ( A \in \mathbb{R} \rightarrow \textrm{-} A \in \mathbb{R}$ ) \\
\end{longtabu}

The somewhat strange-looking steps in \texttt{renegcl} before step 5 are
technical stuff that makes this magic work, and they can be ignored
for a quick overview of the proof. To continue following the ``important''
part of the proof of \texttt{renegcl},
you can look at the reference to \texttt{renegcli} at step 5.

That said, let's briefly look at how
\texttt{renegcl} uses the
Weak Deduction Theorem (\texttt{dedth}) to do its job,
in case you want to do something similar or want understand it more deeply.
Let's work backwards in the proof of \texttt{renegcl}.
Step 6 applies \texttt{dedth} to produce our goal result
$ \vdash ( A \in \mathbb{R} \rightarrow\, - A \in \mathbb{R} )$.
This requires on the one hand the (substituted) deduction
\texttt{renegcli} in step 5.
By itself \texttt{renegcli} proves the deduction
$ \vdash A \in \mathbb{R} \Rightarrow\, \vdash - A \in \mathbb{R}$;
this is the deduction form we are trying to turn into theorem form,
and thus
\texttt{renegcli} has a separate hypothesis that must be fulfilled.
To fulfill the hypothesis of the invocation of
\texttt{renegcli} in step 5, it is eventually
reduced to the already proven theorem $1 \in \mathbb{R}$ in step 3.
Step 4 connects steps 3 and 5; step 4 invokes
\texttt{elimel}, a special case of \texttt{elimhyp} that eliminates
a membership hypothesis for the weak deduction theorem.
On the other hand, the equivalence of the conclusion of
\texttt{renegcl}
$( - A \in \mathbb{R} )$ and the substituted conclusion of
\texttt{renegcli} must be proven, which is done in steps 2 and 1.

The weak deduction theorem has limitations.
In particular, we must be able to prove a special case of the deduction's
hypothesis as a stand-alone theorem.
For example, we used $1 \in \mathbb{R}$ in step 3 of \texttt{renegcl}.

We used to use the weak deduction theorem
extensively within \texttt{set.mm}.
However, we now recommend applying ``deduction style''
instead in most cases, as deduction style is
often an easier and clearer approach.
Therefore, we will now describe deduction style.

\subsection{Deduction Style}\label{deductionstyle}

We now prefer to write assertions in ``deduction form''
instead of writing a proof that would require use of the standard or
weak deduction theorem.
We call this appraoch
``deduction style.''\index{deduction style}

It will be easier to explain this by first defining some terms:

\begin{itemize}
\item \textbf{closed form}\index{closed form}\index{forms!closed}:
A kind of assertion (theorem) with no hypotheses.
Typically its label has no special suffix.
An example is \texttt{unss}, which states:
$\vdash ( ( A \subseteq C \wedge B \subseteq C ) \leftrightarrow ( A \cup B )
\subseteq C )\label{eq:unss}$
\item \textbf{deduction form}\index{deduction form}\index{forms!deduction}:
A kind of assertion with one or more hypotheses
where the conclusion is an implication with
a wff variable as the antecedent (usually $\varphi$), and every hypothesis
(\$e statement)
is either (1) an implication with the same antecedent as the conclusion or
(2) a definition.
A definition
can be for a class variable (this is a class variable followed by ``='')
or a wff variable (this is a wff variable followed by $\leftrightarrow$);
class variable definitions are more common.
In practice, a proof
in deduction form will also contain many steps that are implications
where the antecedent is either that wff variable (normally $\varphi$)
or is
a conjunction (...$\land$...) including that wff variable ($\varphi$).
If an assertion is in deduction form, and other forms are also available,
then we suffix its label with ``d.''
An example is \texttt{unssd}, which states\footnote{
For brevity we show here (and in other places)
a $\&$\index{$\&$} between hypotheses\index{hypotheses}
and a $\Rightarrow$\index{$\Rightarrow$}\index{conclusion}
between the hypotheses and the conclusion.
This notation is technically not part of the Metamath language, but is
instead a convenient abbreviation to show both the hypotheses and conclusion.}:
$\vdash ( \varphi \rightarrow A \subseteq C )\quad\&\quad \vdash ( \varphi
    \rightarrow B \subseteq C )\quad\Rightarrow\quad \vdash ( \varphi
    \rightarrow ( A \cup B ) \subseteq C )\label{eq:unssd}$
\item \textbf{inference form}\index{inference form}\index{forms!inference}:
A kind of assertion with one or more hypotheses that is not in deduction form
(e.g., there is no common antecedent).
If an assertion is in inference form, and other forms are also available,
then we suffix its label with ``i.''
An example is \texttt{unssi}, which states:
$\vdash A \subseteq C\quad\&\quad \vdash B \subseteq C\quad\Rightarrow\quad
    \vdash ( A \cup B ) \subseteq C\label{eq:unssi}$
\end{itemize}

When using deduction style we express an assertion in deduction form.
This form prefixes each hypothesis (other than definitions) and the
conclusion with a universal antecedent (``$\varphi \rightarrow$'').
The antecedent (e.g., $\varphi$)
mimics the context handled in the deduction theorem, eliminating
the need to directly use the deduction theorem.

Once you have an assertion in deduction form, you can easily convert it
to inference form or closed form:

\begin{itemize}
\item To
prove some assertion Ti in inference form, given assertion Td in deduction
form, there is a simple mechanical process you can use. First take each
Ti hypothesis and insert a \texttt{T.} $\rightarrow$ prefix (``true implies'')
using \texttt{a1i}. You
can then use the existing assertion Td to prove the resulting conclusion
with a \texttt{T.} $\rightarrow$ prefix.
Finally, you can remove that prefix using \texttt{mptru},
resulting in the conclusion you wanted to prove.
\item To
prove some assertion T in closed form, given assertion Td in deduction
form, there is another simple mechanical process you can use. First,
select an expression that is the conjunction (...$\land$...) of all of the
consequents of every hypothesis of Td. Next, prove that this expression
implies each of the separate hypotheses of Td in turn by eliminating
conjuncts (there are a variety of proven assertions to do this, including
\texttt{simpl},
\texttt{simpr},
\texttt{3simpa},
\texttt{3simpb},
\texttt{3simpc},
\texttt{simp1},
\texttt{simp2},
and
\texttt{simp3}).
If the
expression has nested conjunctions, inner conjuncts can be broken out by
chaining the above theorems with \texttt{syl}
(see section \ref{syl}).\footnote{
There are actually many theorems
(labeled simp* such as \texttt{simp333}) that break out inner conjuncts in one
step, but rather than learning them you can just use the chaining we
just described to prove them, and then let the Metamath program command
\texttt{minimize{\char`\_}with}\index{\texttt{minimize{\char`\_}with} command}
figure out the right ones needed to collapse them.}
As your final step, you can then apply the already-proven assertion Td
(which is in deduction form), proving assertion T in closed form.
\end{itemize}

We can also easily convert any assertion T in closed form to its related
assertion Ti in inference form by applying
modus ponens\index{modus ponens} (see section \ref{axmp}).

The deduction form antecedent can also be used to represent the context
necessary to support natural deduction systems, so we will now
discuss natural deduction.

\subsection{Natural Deduction}\label{naturaldeduction}

Natural deduction\index{natural deduction}
(ND) systems, as such, were originally introduced in
1934 by two logicians working independently: Ja\'skowski and Gentzen. ND
systems are supposed to reconstruct, in a formally proper way, traditional
ways of mathematical reasoning (such as conditional proof, indirect proof,
and proof by cases). As reconstructions they were naturally influenced
by previous work, and many specific ND systems and notations have been
developed since their original work.

There are many ND variants, but
Indrzejczak \cite[p.~31-32]{Indrzejczak}\index{Indrzejczak, Andrzej}
suggests that any natural deductive system must satisfy at
least these three criteria:

\begin{itemize}
\item ``There are some means for entering assumptions into a proof and
also for eliminating them. Usually it requires some bookkeeping devices
for indicating the scope of an assumption, and showing that a part of
a proof depending on eliminated assumption is discharged.
\item There are no (or, at least, a very limited set of) axioms, because
their role is taken over by the set of primitive rules for introduction
and elimination of logical constants which means that elementary
inferences instead of formulae are taken as primitive.
\item (A genuine) ND system admits a lot of freedom in proof construction
and possibility of applying several proof search strategies, like
conditional proof, proof by cases, proof by reductio ad absurdum etc.''
\end{itemize}

The Metamath Proof Explorer (MPE) as defined in \texttt{set.mm}
is fundamentally a Hilbert-style system.
That is, MPE is based on a larger number of axioms (compared
to natural deduction systems), a very small set of rules of inference
(modus ponens), and the context is not changed by the rules of inference
in the middle of a proof. That said, MPE proofs can be developed using
the natural deduction (ND) approach as originally developed by Ja\'skowski
and Gentzen.

The most common and recommended approach for applying ND in MPE is to use
deduction form\index{deduction form}%
\index{forms!deduction}
and apply the MPE proven assertions that are equivalent to ND rules.
For example, MPE's \texttt{jca} is equivalent to ND rule $\land$-I
(and-insertion).
We maintain a list of equivalences that you may consult.
This approach for applying an ND approach within MPE relies on Metamath's
wff metavariables in an essential way, and is described in more detail
in the presentation ``Natural Deductions in the Metamath Proof Language''
by Mario Carneiro \cite{CarneiroND}\index{Carneiro, Mario}.

In this style many steps are an implication, whose antecedent mimics
the context ($\Gamma$) of most ND systems. To add an assumption, simply add
it to the implication antecedent (typically using
\texttt{simpr}),
and use that
new antecedent for all later claims in the same scope. If you wish to
use an assertion in an ND hypothesis scope that is outside the current
ND hypothesis scope, modify the assertion so that the ND hypothesis
assumption is added to its antecedent (typically using \texttt{adantr}). Most
proof steps will be proved using rules that have hypotheses and results
of the form $\varphi \rightarrow$ ...

An example may make this clearer.
Let's show theorem 5.5 of
\cite[p.~18]{Clemente}\index{Clemente Laboreo, Daniel}
along with a line by line translation using the usual
translation of natural deduction (ND) in the Metamath Proof Explorer
(MPE) notation (this is proof \texttt{ex-natded5.5}).
The proof's original goal was to prove
$\lnot \psi$ given two hypotheses,
$( \psi \rightarrow \chi )$ and $ \lnot \chi$.
We will translate these statements into MPE deduction form
by prefixing them all with $\varphi \rightarrow$.
As a result, in MPE the goal is stated as
$( \varphi \rightarrow \lnot \psi )$, and the two hypotheses are stated as
$( \varphi \rightarrow ( \psi \rightarrow \chi ) )$ and
$( \varphi \rightarrow \lnot \chi )$.

The following table shows the proof in Fitch natural deduction style
and its MPE equivalent.
The \textit{\#} column shows the original numbering,
\textit{MPE\#} shows the number in the equivalent MPE proof
(which we will show later),
\textit{ND Expression} shows the original proof claim in ND notation,
and \textit{MPE Translation} shows its translation into MPE
as discussed in this section.
The final columns show the rationale in ND and MPE respectively.

\needspace{4\baselineskip}
{\setlength{\extrarowsep}{4pt} % Keep rows from being too close together
\begin{longtabu}   { @{} c c X X X X }
\textbf{\#} & \textbf{MPE\#} & \textbf{ND Ex\-pres\-sion} &
\textbf{MPE Trans\-lation} & \textbf{ND Ration\-ale} &
\textbf{MPE Ra\-tio\-nale} \\
\endhead

1 & 2;3 &
$( \psi \rightarrow \chi )$ &
$( \varphi \rightarrow ( \psi \rightarrow \chi ) )$ &
Given &
\$e; \texttt{adantr} to put in ND hypothesis \\

2 & 5 &
$ \lnot \chi$ &
$( \varphi \rightarrow \lnot \chi )$ &
Given &
\$e; \texttt{adantr} to put in ND hypothesis \\

3 & 1 &
... $\vert$ $\psi$ &
$( \varphi \rightarrow \psi )$ &
ND hypothesis assumption &
\texttt{simpr} \\

4 & 4 &
... $\chi$ &
$( ( \varphi \land \psi ) \rightarrow \chi )$ &
$\rightarrow$\,E 1,3 &
\texttt{mpd} 1,3 \\

5 & 6 &
... $\lnot \chi$ &
$( ( \varphi \land \psi ) \rightarrow \lnot \chi )$ &
IT 2 &
\texttt{adantr} 5 \\

6 & 7 &
$\lnot \psi$ &
$( \varphi \rightarrow \lnot \psi )$ &
$\land$\,I 3,4,5 &
\texttt{pm2.65da} 4,6 \\

\end{longtabu}
}


The original used Latin letters; we have replaced them with Greek letters
to follow Metamath naming conventions and so that it is easier to follow
the Metamath translation. The Metamath line-for-line translation of
this natural deduction approach precedes every line with an antecedent
including $\varphi$ and uses the Metamath equivalents of the natural deduction
rules. To add an assumption, the antecedent is modified to include it
(typically by using \texttt{adantr};
\texttt{simpr} is useful when you want to
depend directly on the new assumption, as is shown here).

In Metamath we can represent the two given statements as these hypotheses:

\needspace{2\baselineskip}
\begin{itemize}
\item ex-natded5.5.1 $\vdash ( \varphi \rightarrow ( \psi \rightarrow \chi ) )$
\item ex-natded5.5.2 $\vdash ( \varphi \rightarrow \lnot \chi )$
\end{itemize}

\needspace{4\baselineskip}
Here is the proof in Metamath as a line-by-line translation:

\begin{longtabu}   { l l l X }
\textbf{Step} & \textbf{Hyp} & \textbf{Ref} & \textbf{Ex\-pres\-sion} \\
\endhead
1 & & simpr & $\vdash ( ( \varphi \land \psi ) \rightarrow \psi )$ \\
2 & & ex-natded5.5.1 &
  $\vdash ( \varphi \rightarrow ( \psi \rightarrow \chi ) )$ \\
3 & 2 & adantr &
 $\vdash ( ( \varphi \land \psi ) \rightarrow ( \psi \rightarrow \chi ) )$ \\
4 & 1, 3 & mpd &
 $\vdash ( ( \varphi \land \psi ) \rightarrow \chi ) $ \\
5 & & ex-natded5.5.2 &
 $\vdash ( \varphi \rightarrow \lnot \chi )$ \\
6 & 5 & adantr &
 $\vdash ( ( \varphi \land \psi ) \rightarrow \lnot \chi )$ \\
7 & 4, 6 & pm2.65da &
 $\vdash ( \varphi \rightarrow \lnot \psi )$ \\
\end{longtabu}

Only using specific natural deduction rules directly can lead to very
long proofs, for exactly the same reason that only using axioms directly
in Hilbert-style proofs can lead to very long proofs.
If the goal is short and clear proofs,
then it is better to reuse already-proven assertions
in deduction form than to start from scratch each time
and using only basic natural deduction rules.

\subsection{Strengths of Our Approach}

As far as we know there is nothing else in the literature like either the
weak deduction theorem or Mario Carneiro\index{Carneiro, Mario}'s
natural deduction method.
In order to
transform a hypothesis into an antecedent, the literature's standard
``Deduction Theorem''\index{Deduction Theorem}\index{Standard Deduction Theorem}
requires metalogic outside of the notions provided
by the axiom system. We instead generally prefer to use Mario Carneiro's
natural deduction method, then use the weak deduction theorem in cases
where that is difficult to apply, and only then use the full standard
deduction theorem as a last resort.

The weak deduction theorem\index{Weak Deduction Theorem}
does not require any additional metalogic
but converts an inference directly into a closed form theorem, with
a rigorous proof that uses only the axiom system. Unlike the standard
Deduction Theorem, there is no implicit external justification that we
have to trust in order to use it.

Mario Carneiro's natural deduction\index{natural deduction}
method also does not require any new metalogical
notions. It avoids the Deduction Theorem's metalogic by prefixing the
hypotheses and conclusion of every would-be inference with a universal
antecedent (``$\varphi \rightarrow$'') from the very start.

We think it is impressive and satisfying that we can do so much in a
practical sense without stepping outside of our Hilbert-style axiom system.
Of course our axiomatization, which is in the form of schemes,
contains a metalogic of its own that we exploit. But this metalogic
is relatively simple, and for our Deduction Theorem alternatives,
we primarily use just the direct substitution of expressions for
metavariables.

\begin{sloppy}
\section{Exploring the Set The\-o\-ry Data\-base}\label{exploring}
\end{sloppy}
% NOTE: All examples performed in this section are
% recorded wtih "set width 61" % on set.mm as of 2019-05-28
% commit c1e7849557661260f77cfdf0f97ac4354fbb4f4d.

At this point you may wish to study the \texttt{set.mm}\index{set theory
database (\texttt{set.mm})} file in more detail.  Pay particular
attention to the assumptions needed to define wffs\index{well-formed
formula (wff)} (which are not included above), the variable types
(\texttt{\$f}\index{\texttt{\$f} statement} statements), and the
definitions that are introduced.  Start with some simple theorems in
propositional calculus, making sure you understand in detail each step
of a proof.  Once you get past the first few proofs and become familiar
with the Metamath language, any part of the \texttt{set.mm} database
will be as easy to follow, step by step, as any other part---you won't
have to undergo a ``quantum leap'' in mathematical sophistication to be
able to follow a deep proof in set theory.

Next, you may want to explore how concepts such as natural numbers are
defined and described.  This is probably best done in conjunction with
standard set theory textbooks, which can help give you a higher-level
understanding.  The \texttt{set.mm} database provides references that will get
you started.  From there, you will be on your way towards a very deep,
rigorous understanding of abstract mathematics.

The Metamath\index{Metamath} program can help you peruse a Metamath data\-base,
wheth\-er you are trying to figure out how a certain step follows in a proof or
just have a general curiosity.  We will go through some examples of the
commands, using the \texttt{set.mm}\index{set theory database (\texttt{set.mm})}
database provided with the Metamath software.  These should help get you
started.  See Chapter~\ref{commands} for a more detailed description of
the commands.  Note that we have included the full spelling of all commands to
prevent ambiguity with future commands.  In practice you may type just the
characters needed to specify each command keyword\index{command keyword}
unambiguously, often just one or two characters per keyword, and you don't
need to type them in upper case.

First run the Metamath program as described earlier.  You should see the
\verb/MM>/ prompt.  Read in the \texttt{set.mm} file:\index{\texttt{read}
command}

\begin{verbatim}
MM> read set.mm
Reading source file "set.mm"... 34554442 bytes
34554442 bytes were read into the source buffer.
The source has 155711 statements; 2254 are $a and 32250 are $p.
No errors were found.  However, proofs were not checked.
Type VERIFY PROOF * if you want to check them.
\end{verbatim}

As with most examples in this book, what you will see
will be slightly different because we are continuously
improving our databases (including \texttt{set.mm}).

Let's check the database integrity.  This may take a minute or two to run if
your computer is slow.

\begin{verbatim}
MM> verify proof *
0 10%  20%  30%  40%  50%  60%  70%  80%  90% 100%
..................................................
All proofs in the database were verified in 2.84 s.
\end{verbatim}

No errors were reported, so every proof is correct.

You need to know the names (labels) of theorems before you can look at them.
Often just examining the database file(s) with a text editor is the best
approach.  In \texttt{set.mm} there are many detailed comments, especially near
the beginning, that can help guide you. The \texttt{search} command in the
Metamath program is also handy.  The \texttt{comments} qualifier will list the
statements whose associated comment (the one immediately before it) contain a
string you give it.  For example, if you are studying Enderton's {\em Elements
of Set Theory} \cite{Enderton}\index{Enderton, Herbert B.} you may want to see
the references to it in the database.  The search string \texttt{enderton} is not
case sensitive.  (This will not show you all the database theorems that are in
Enderton's book because there is usually only one citation for a given
theorem, which may appear in several textbooks.)\index{\texttt{search}
command}

\begin{verbatim}
MM> search * "enderton" / comments
12067 unineq $p "... Exercise 20 of [Enderton] p. 32 and ..."
12459 undif2 $p "...Corollary 6K of [Enderton] p. 144. (C..."
12953 df-tp $a "...s. Definition of [Enderton] p. 19. (Co..."
13689 unissb $p ".... Exercise 5 of [Enderton] p. 26 and ..."
\end{verbatim}
\begin{center}
(etc.)
\end{center}

Or you may want to see what theorems have something to do with
conjunction (logical {\sc and}).  The quotes around the search
string are optional when there's no ambiguity.\index{\texttt{search}
command}

\begin{verbatim}
MM> search * conjunction / comments
120 a1d $p "...be replaced with a conjunction ( ~ df-an )..."
662 df-bi $a "...viated form after conjunction is introdu..."
1319 wa $a "...ff definition to include conjunction ('and')."
1321 df-an $a "Define conjunction (logical 'and'). Defini..."
1420 imnan $p "...tion in terms of conjunction. (Contribu..."
\end{verbatim}
\begin{center}
(etc.)
\end{center}


Now we will start to look at some details.  Let's look at the first
axiom of propositional calculus
(we could use \texttt{sh st} to abbreviate
\texttt{show statement}).\index{\texttt{show statement} command}

\begin{verbatim}
MM> show statement ax-1/full
Statement 19 is located on line 881 of the file "set.mm".
"Axiom _Simp_.  Axiom A1 of [Margaris] p. 49.  One of the 3
axioms of propositional calculus.  The 3 axioms are also
        ...
19 ax-1 $a |- ( ph -> ( ps -> ph ) ) $.
Its mandatory hypotheses in RPN order are:
  wph $f wff ph $.
  wps $f wff ps $.
The statement and its hypotheses require the variables:  ph
      ps
The variables it contains are:  ph ps


Statement 49 is located on line 11182 of the file "set.mm".
Its statement number for HTML pages is 6.
"Axiom _Simp_.  Axiom A1 of [Margaris] p. 49.  One of the 3
axioms of propositional calculus.  The 3 axioms are also
given as Definition 2.1 of [Hamilton] p. 28.
...
49 ax-1 $a |- ( ph -> ( ps -> ph ) ) $.
Its mandatory hypotheses in RPN order are:
  wph $f wff ph $.
  wps $f wff ps $.
The statement and its hypotheses require the variables:
  ph ps
The variables it contains are:  ph ps
\end{verbatim}

Compare this to \texttt{ax-1} on p.~\pageref{ax1}.  You can see that the
symbol \texttt{ph} is the {\sc ascii} notation for $\varphi$, etc.  To
see the mathematical symbols for any expression you may typeset it in
\LaTeX\ (type \texttt{help tex} for instructions)\index{latex@{\LaTeX}}
or, easier, just use a text editor to look at the comments where symbols
are first introduced in \texttt{set.mm}.  The hypotheses \texttt{wph}
and \texttt{wps} required by \texttt{ax-1} mean that variables
\texttt{ph} and \texttt{ps} must be wffs.

Next we'll pick a simple theorem of propositional calculus, the Principle of
Identity, which is proved directly from the axioms.  We'll look at the
statement then its proof.\index{\texttt{show statement}
command}

\begin{verbatim}
MM> show statement id1/full
Statement 116 is located on line 11371 of the file "set.mm".
Its statement number for HTML pages is 22.
"Principle of identity.  Theorem *2.08 of [WhiteheadRussell]
p. 101.  This version is proved directly from the axioms for
demonstration purposes.
...
116 id1 $p |- ( ph -> ph ) $= ... $.
Its mandatory hypotheses in RPN order are:
  wph $f wff ph $.
Its optional hypotheses are:  wps wch wth wta wet
      wze wsi wrh wmu wla wka
The statement and its hypotheses require the variables:  ph
These additional variables are allowed in its proof:
      ps ch th ta et ze si rh mu la ka
The variables it contains are:  ph
\end{verbatim}

The optional variables\index{optional variable} \texttt{ps}, \texttt{ch}, etc.\ are
available for use in a proof of this statement if we wish, and were we to do
so we would make use of optional hypotheses \texttt{wps}, \texttt{wch}, etc.  (See
Section~\ref{dollaref} for the meaning of ``optional
hypothesis.''\index{optional hypothesis}) The reason these show up in the
statement display is that statement \texttt{id1} happens to be in their scope
(see Section~\ref{scoping} for the definition of ``scope''\index{scope}), but
in fact in propositional calculus we will never make use of optional
hypotheses or variables.  This becomes important after quantifiers are
introduced, where ``dummy'' variables\index{dummy variable} are often needed
in the middle of a proof.

Let's look at the proof of statement \texttt{id1}.  We'll use the
\texttt{show proof} command, which by default suppresses the
``non-essential'' steps that construct the wffs.\index{\texttt{show proof}
command}
We will display the proof in ``lemmon' format (a non-indented format
with explicit previous step number references) and renumber the
displayed steps:

\begin{verbatim}
MM> show proof id1 /lemmon/renumber
1 ax-1           $a |- ( ph -> ( ph -> ph ) )
2 ax-1           $a |- ( ph -> ( ( ph -> ph ) -> ph ) )
3 ax-2           $a |- ( ( ph -> ( ( ph -> ph ) -> ph ) ) ->
                     ( ( ph -> ( ph -> ph ) ) -> ( ph -> ph )
                                                          ) )
4 2,3 ax-mp      $a |- ( ( ph -> ( ph -> ph ) ) -> ( ph -> ph
                                                          ) )
5 1,4 ax-mp      $a |- ( ph -> ph )
\end{verbatim}

If you have read Section~\ref{trialrun}, you'll know how to interpret this
proof.  Step~2, for example, is an application of axiom \texttt{ax-1}.  This
proof is identical to the one in Hamilton's {\em Logic for Mathematicians}
\cite[p.~32]{Hamilton}\index{Hamilton, Alan G.}.

You may want to look at what
substitutions\index{substitution!variable}\index{variable substitution} are
made into \texttt{ax-1} to arrive at step~2.  The command to do this needs to
know the ``real'' step number, so we'll display the proof again without
the \texttt{renumber} qualifier.\index{\texttt{show proof}
command}

\begin{verbatim}
MM> show proof id1 /lemmon
 9 ax-1          $a |- ( ph -> ( ph -> ph ) )
20 ax-1          $a |- ( ph -> ( ( ph -> ph ) -> ph ) )
24 ax-2          $a |- ( ( ph -> ( ( ph -> ph ) -> ph ) ) ->
                     ( ( ph -> ( ph -> ph ) ) -> ( ph -> ph )
                                                          ) )
25 20,24 ax-mp   $a |- ( ( ph -> ( ph -> ph ) ) -> ( ph -> ph
                                                          ) )
26 9,25 ax-mp    $a |- ( ph -> ph )
\end{verbatim}

The ``real'' step number is 20.  Let's look at its details.

\begin{verbatim}
MM> show proof id1 /detailed_step 20
Proof step 20:  min=ax-1 $a |- ( ph -> ( ( ph -> ph ) -> ph )
  )
This step assigns source "ax-1" ($a) to target "min" ($e).
The source assertion requires the hypotheses "wph" ($f, step
18) and "wps" ($f, step 19).  The parent assertion of the
target hypothesis is "ax-mp" ($a, step 25).
The source assertion before substitution was:
    ax-1 $a |- ( ph -> ( ps -> ph ) )
The following substitutions were made to the source
assertion:
    Variable  Substituted with
     ph        ph
     ps        ( ph -> ph )
The target hypothesis before substitution was:
    min $e |- ph
The following substitution was made to the target hypothesis:
    Variable  Substituted with
     ph        ( ph -> ( ( ph -> ph ) -> ph ) )
\end{verbatim}

This shows the substitutions\index{substitution!variable}\index{variable
substitution} made to the variables in \texttt{ax-1}.  References are made to
steps 18 and 19 which are not shown in our proof display.  To see these steps,
you can display the proof with the \texttt{all} qualifier.

Let's look at a slightly more advanced proof of propositional calculus.  Note
that \verb+/\+ is the symbol for $\wedge$ (logical {\sc and}, also
called conjunction).\index{conjunction ($\wedge$)}
\index{logical {\sc and} ($\wedge$)}

\begin{verbatim}
MM> show statement prth/full
Statement 1791 is located on line 15503 of the file "set.mm".
Its statement number for HTML pages is 559.
"Conjoin antecedents and consequents of two premises.  This
is the closed theorem form of ~ anim12d .  Theorem *3.47 of
[WhiteheadRussell] p. 113.  It was proved by Leibniz,
and it evidently pleased him enough to call it
_praeclarum theorema_ (splendid theorem).
...
1791 prth $p |- ( ( ( ph -> ps ) /\ ( ch -> th ) ) -> ( ( ph
      /\ ch ) -> ( ps /\ th ) ) ) $= ... $.
Its mandatory hypotheses in RPN order are:
  wph $f wff ph $.
  wps $f wff ps $.
  wch $f wff ch $.
  wth $f wff th $.
Its optional hypotheses are:  wta wet wze wsi wrh wmu wla wka
The statement and its hypotheses require the variables:  ph
      ps ch th
These additional variables are allowed in its proof:  ta et
      ze si rh mu la ka
The variables it contains are:  ph ps ch th


MM> show proof prth /lemmon/renumber
1 simpl          $p |- ( ( ( ph -> ps ) /\ ( ch -> th ) ) ->
                                               ( ph -> ps ) )
2 simpr          $p |- ( ( ( ph -> ps ) /\ ( ch -> th ) ) ->
                                               ( ch -> th ) )
3 1,2 anim12d    $p |- ( ( ( ph -> ps ) /\ ( ch -> th ) ) ->
                           ( ( ph /\ ch ) -> ( ps /\ th ) ) )
\end{verbatim}

There are references to a lot of unfamiliar statements.  To see what they are,
you may type the following:

\begin{verbatim}
MM> show proof prth /statement_summary
Summary of statements used in the proof of "prth":

Statement simpl is located on line 14748 of the file
"set.mm".
"Elimination of a conjunct.  Theorem *3.26 (Simp) of
[WhiteheadRussell] p. 112. ..."
  simpl $p |- ( ( ph /\ ps ) -> ph ) $= ... $.

Statement simpr is located on line 14777 of the file
"set.mm".
"Elimination of a conjunct.  Theorem *3.27 (Simp) of
[WhiteheadRussell] ..."
  simpr $p |- ( ( ph /\ ps ) -> ps ) $= ... $.

Statement anim12d is located on line 15445 of the file
"set.mm".
"Conjoin antecedents and consequents in a deduction.
..."
  anim12d.1 $e |- ( ph -> ( ps -> ch ) ) $.
  anim12d.2 $e |- ( ph -> ( th -> ta ) ) $.
  anim12d $p |- ( ph -> ( ( ps /\ th ) -> ( ch /\ ta ) ) )
      $= ... $.
\end{verbatim}
\begin{center}
(etc.)
\end{center}

Of course you can look at each of these statements and their proofs, and
so on, back to the axioms of propositional calculus if you wish.

The \texttt{search} command is useful for finding statements when you
know all or part of their contents.  The following example finds all
statements containing \verb@ph -> ps@ followed by \verb@ch -> th@.  The
\verb@$*@ is a wildcard that matches anything; the \texttt{\$} before the
\verb$*$ prevents conflicts with math symbol token names.  The \verb@*@ after
\texttt{SEARCH} is also a wildcard that in this case means ``match any label.''
\index{\texttt{search} command}

% I'm omitting this one, since readers are unlikely to see it:
% 1096 bisymOLD $p |- ( ( ( ph -> ps ) -> ( ch -> th ) ) -> ( (
%   ( ps -> ph ) -> ( th -> ch ) ) -> ( ( ph <-> ps ) -> ( ch
%    <-> th ) ) ) )
\begin{verbatim}
MM> search * "ph -> ps $* ch -> th"
1791 prth $p |- ( ( ( ph -> ps ) /\ ( ch -> th ) ) -> ( ( ph
    /\ ch ) -> ( ps /\ th ) ) )
2455 pm3.48 $p |- ( ( ( ph -> ps ) /\ ( ch -> th ) ) -> ( (
    ph \/ ch ) -> ( ps \/ th ) ) )
117859 pm11.71 $p |- ( ( E. x ph /\ E. y ch ) -> ( ( A. x (
    ph -> ps ) /\ A. y ( ch -> th ) ) <-> A. x A. y ( ( ph /\
    ch ) -> ( ps /\ th ) ) ) )
\end{verbatim}

Three statements, \texttt{prth}, \texttt{pm3.48},
 and \texttt{pm11.71}, were found to match.

To see what axioms\index{axiom} and definitions\index{definition}
\texttt{prth} ultimately depends on for its proof, you can have the
program backtrack through the hierarchy\index{hierarchy} of theorems and
definitions.\index{\texttt{show trace{\char`\_}back} command}

\begin{verbatim}
MM> show trace_back prth /essential/axioms
Statement "prth" assumes the following axioms ($a
statements):
  ax-1 ax-2 ax-3 ax-mp df-bi df-an
\end{verbatim}

Note that the 3 axioms of propositional calculus and the modus ponens rule are
needed (as expected); in addition, there are a couple of definitions that are used
along the way.  Note that Metamath makes no distinction\index{axiom vs.\
definition} between axioms\index{axiom} and definitions\index{definition}.  In
\texttt{set.mm} they have been distinguished artificially by prefixing their
labels\index{labels in \texttt{set.mm}} with \texttt{ax-} and \texttt{df-}
respectively.  For example, \texttt{df-an} defines conjunction (logical {\sc
and}), which is represented by the symbol \verb+/\+.
Section~\ref{definitions} discusses the philosophy of definitions, and the
Metamath language takes a particularly simple, conservative approach by using
the \texttt{\$a}\index{\texttt{\$a} statement} statement for both axioms and
definitions.

You can also have the program compute how many steps a proof
has\index{proof length} if we were to follow it all the way back to
\texttt{\$a} statements.

\begin{verbatim}
MM> show trace_back prth /essential/count_steps
The statement's actual proof has 3 steps.  Backtracking, a
total of 79 different subtheorems are used.  The statement
and subtheorems have a total of 274 actual steps.  If
subtheorems used only once were eliminated, there would be a
total of 38 subtheorems, and the statement and subtheorems
would have a total of 185 steps.  The proof would have 28349
steps if fully expanded back to axiom references.  The
maximum path length is 38.  A longest path is:  prth <-
anim12d <- syl2and <- sylan2d <- ancomsd <- ancom <- pm3.22
<- pm3.21 <- pm3.2 <- ex <- sylbir <- biimpri <- bicomi <-
bicom1 <- bi2 <- dfbi1 <- impbii <- bi3 <- simprim <- impi <-
con1i <- nsyl2 <- mt3d <- con1d <- notnot1 <- con2i <- nsyl3
<- mt2d <- con2d <- notnot2 <- pm2.18d <- pm2.18 <- pm2.21 <-
pm2.21d <- a1d <- syl <- mpd <- a2i <- a2i.1 .
\end{verbatim}

This tells us that we would have to inspect 274 steps if we want to
verify the proof completely starting from the axioms.  A few more
statistics are also shown.  There are one or more paths back to axioms
that are the longest; this command ferrets out one of them and shows it
to you.  There may be a sense in which the longest path length is
related to how ``deep'' the theorem is.

We might also be curious about what proofs depend on the theorem
\texttt{prth}.  If it is never used later on, we could eliminate it as
redundant if it has no intrinsic interest by itself.\index{\texttt{show
usage} command}

% I decided to show the OLD values here.
\begin{verbatim}
MM> show usage prth
Statement "prth" is directly referenced in the proofs of 18
statements:
  mo3 moOLD 2mo 2moOLD euind reuind reuss2 reusv3i opelopabt
  wemaplem2 rexanre rlimcn2 o1of2 o1rlimmul 2sqlem6 spanuni
  heicant pm11.71
\end{verbatim}

Thus \texttt{prth} is directly used by 18 proofs.
We can use the \texttt{/recursive} qualifier to include indirect use:

\begin{verbatim}
MM> show usage prth /recursive
Statement "prth" directly or indirectly affects the proofs of
24214 statements:
  mo3 mo mo3OLD eu2 moOLD eu2OLD eu3OLD mo4f mo4 eu4 mopick
...
\end{verbatim}

\subsection{A Note on the ``Compact'' Proof Format}

The Metamath program will display proofs in a ``compact''\index{compact proof}
format whenever the proof is stored in compressed format in the database.  It
may be be slightly confusing unless you know how to interpret it.
For example,
if you display the complete proof of theorem \texttt{id1} it will start
off as follows:

\begin{verbatim}
MM> show proof id1 /lemmon/all
 1 wph           $f wff ph
 2 wph           $f wff ph
 3 wph           $f wff ph
 4 2,3 wi    @4: $a wff ( ph -> ph )
 5 1,4 wi    @5: $a wff ( ph -> ( ph -> ph ) )
 6 @4            $a wff ( ph -> ph )
\end{verbatim}

\begin{center}
{etc.}
\end{center}

Step 4 has a ``local label,''\index{local label} \texttt{@4}, assigned to it.
Later on, at step 6, the label \texttt{@1} is referenced instead of
displaying the explicit proof for that step.  This technique takes advantage
of the fact that steps in a proof often repeat, especially during the
construction of wffs.  The compact format reduces the number of steps in the
proof display and may be preferred by some people.

If you want to see the normal format with the ``true'' step numbers, you can
use the following workaround:\index{\texttt{save proof} command}

\begin{verbatim}
MM> save proof id1 /normal
The proof of "id1" has been reformatted and saved internally.
Remember to use WRITE SOURCE to save it permanently.
MM> show proof id1 /lemmon/all
 1 wph           $f wff ph
 2 wph           $f wff ph
 3 wph           $f wff ph
 4 2,3 wi        $a wff ( ph -> ph )
 5 1,4 wi        $a wff ( ph -> ( ph -> ph ) )
 6 wph           $f wff ph
 7 wph           $f wff ph
 8 6,7 wi        $a wff ( ph -> ph )
\end{verbatim}

\begin{center}
{etc.}
\end{center}

Note that the original 6 steps are now 8 steps.  However, the format is
now the same as that described in Chapter~\ref{using}.

\chapter{The Metamath Language}
\label{languagespec}

\begin{quote}
  {\em Thus mathematics may be defined as the subject in which we never know
what we are talking about, nor whether what we are saying is true.}
    \flushright\sc  Bertrand Russell\footnote{\cite[p.~84]{Russell2}.}\\
\end{quote}\index{Russell, Bertrand}

Probably the most striking feature of the Metamath language is its almost
complete absence of hard-wired syntax. Metamath\index{Metamath} does not
understand any mathematics or logic other than that needed to construct finite
sequences of symbols according to a small set of simple, built-in rules.  The
only rule it uses in a proof is the substitution of an expression (symbol
sequence) for a variable, subject to a simple constraint to prevent
bound-variable clashes.  The primitive notions built into Metamath involve the
simple manipulation of finite objects (symbols) that we as humans can easily
visualize and that computers can easily deal with.  They seem to be just
about the simplest notions possible that are required to do standard
mathematics.

This chapter serves as a reference manual for the Metamath\index{Metamath}
language. It covers the tedious technical details of the language, some of
which you may wish to skim in a first reading.  On the other hand, you should
pay close attention to the defined terms in {\bf boldface}; they have precise
meanings that are important to keep in mind for later understanding.  It may
be best to first become familiar with the examples in Chapter~\ref{using} to
gain some motivation for the language.

%% Uncomment this when uncommenting section {formalspec} below
If you have some knowledge of set theory, you may wish to study this
chapter in conjunction with the formal set-theoretical description of the
Metamath language in Appendix~\ref{formalspec}.

We will use the name ``Metamath''\index{Metamath} to mean either the Metamath
computer language or the Metamath software associated with the computer
language.  We will not distinguish these two when the context is clear.

The next section contains the complete specification of the Metamath
language.
It serves as an
authoritative reference and presents the syntax in enough detail to
write a parser\index{parsing Metamath} and proof verifier.  The
specification is terse and it is probably hard to learn the language
directly from it, but we include it here for those impatient people who
prefer to see everything up front before looking at verbose expository
material.  Later sections explain this material and provide examples.
We will repeat the definitions in those sections, and you may skip the
next section at first reading and proceed to Section~\ref{tut1}
(p.~\pageref{tut1}).

\section{Specification of the Metamath Language}\label{spec}
\index{Metamath!specification}

\begin{quote}
  {\em Sometimes one has to say difficult things, but one ought to say
them as simply as one knows how.}
    \flushright\sc  G. H. Hardy\footnote{As quoted in
    \cite{deMillo}, p.~273.}\\
\end{quote}\index{Hardy, G. H.}

\subsection{Preliminaries}\label{spec1}

% Space is technically a printable character, so we'll word things
% carefully so it's unambiguous.
A Metamath {\bf database}\index{database} is built up from a top-level source
file together with any source files that are brought in through file inclusion
commands (see below).  The only characters that are allowed to appear in a
Metamath source file are the 94 non-whitespace printable {\sc
ascii}\index{ascii@{\sc ascii}} characters, which are digits, upper and lower
case letters, and the following 32 special
characters\index{special characters}:\label{spec1chars}

\begin{verbatim}
! " # $ % & ' ( ) * + , - . / :
; < = > ? @ [ \ ] ^ _ ` { | } ~
\end{verbatim}
plus the following characters which are the ``white space'' characters:
space (a printable character),
tab, carriage return, line feed, and form feed.\label{whitespace}
We will use \texttt{typewriter}
font to display the printable characters.

A Metamath database consists of a sequence of three kinds of {\bf
tokens}\index{token} separated by {\bf white space}\index{white space}
(which is any sequence of one or more white space characters).  The set
of {\bf keyword}\index{keyword} tokens is \texttt{\$\char`\{},
\texttt{\$\char`\}}, \texttt{\$c}, \texttt{\$v}, \texttt{\$f},
\texttt{\$e}, \texttt{\$d}, \texttt{\$a}, \texttt{\$p}, \texttt{\$.},
\texttt{\$=}, \texttt{\$(}, \texttt{\$)}, \texttt{\$[}, and
\texttt{\$]}.  The last four are called {\bf auxiliary}\index{auxiliary
keyword} or preprocessing keywords.  A {\bf label}\index{label} token
consists of any combination of letters, digits, and the characters
hyphen, underscore, and period.  A {\bf math symbol}\index{math symbol}
token may consist of any combination of the 93 printable standard {\sc
ascii} characters other than space or \texttt{\$}~. All tokens are
case-sensitive.

\subsection{Preprocessing}

The token \texttt{\$(} begins a {\bf comment} and
\texttt{\$)} ends a comment.\index{\texttt{\$(}
and \texttt{\$)} auxiliary keywords}\index{comment}
Comments may contain any of
the 94 non-whitespace printable characters and white space,
except they may not contain the
2-character sequences \texttt{\$(} or \texttt{\$)} (comments do not nest).
Comments are ignored (treated
like white space) for the purpose of parsing, e.g.,
\texttt{\$( \$[ \$)} is a comment.
See p.~\pageref{mathcomments} for comment typesetting conventions; these
conventions may be ignored for the purpose of parsing.

A {\bf file inclusion command} consists of \texttt{\$[} followed by a file name
followed by \texttt{\$]}.\index{\texttt{\$[} and \texttt{\$]} auxiliary
keywords}\index{included file}\index{file inclusion}
It is only allowed in the outermost scope (i.e., not between
\texttt{\$\char`\{} and \texttt{\$\char`\}})
and must not be inside a statement (e.g., it may not occur
between the label of a \texttt{\$a} statement and its \texttt{\$.}).
The file name may not
contain a \texttt{\$} or white space.  The file must exist.
The case-sensitivity
of its name follows the conventions of the operating system.  The contents of
the file replace the inclusion command.
Included files may include other files.
Only the first reference to a given file is included; any later
references to the same file (whether in the top-level file or in included
files) cause the inclusion command to be ignored (treated like white space).
A verifier may assume that file names with different strings
refer to different files for the purpose of ignoring later references.
A file self-reference is ignored, as is any reference to the top-level file
(to avoid loops).
Included files may not include a \texttt{\$(} without a matching \texttt{\$)},
may not include a \texttt{\$[} without a matching \texttt{\$]}, and may
not include incomplete statements (e.g., a \texttt{\$a} without a matching
\texttt{\$.}).
It is currently unspecified if path references are relative to the process'
current directory or the file's containing directory, so databases should
avoid using pathname separators (e.g., ``/'') in file names.

Like all tokens, the \texttt{\$(}, \texttt{\$)}, \texttt{\$[}, and \texttt{\$]} keywords
must be surrounded by white space.

\subsection{Basic Syntax}

After preprocessing, a database will consist of a sequence of {\bf
statements}.
These are the scoping statements \texttt{\$\char`\{} and
\texttt{\$\char`\}}, along with the \texttt{\$c}, \texttt{\$v},
\texttt{\$f}, \texttt{\$e}, \texttt{\$d}, \texttt{\$a}, and \texttt{\$p}
statements.

A {\bf scoping statement}\index{scoping statement} consists only of its
keyword, \texttt{\$\char`\{} or \texttt{\$\char`\}}.
A \texttt{\$\char`\{} begins a {\bf
block}\index{block} and a matching \texttt{\$\char`\}} ends the block.
Every \texttt{\$\char`\{}
must have a matching \texttt{\$\char`\}}.
Defining it recursively, we say a block
contains a sequence of zero or more tokens other
than \texttt{\$\char`\{} and \texttt{\$\char`\}} and
possibly other blocks.  There is an {\bf outermost
block}\index{block!outermost} not bracketed by \texttt{\$\char`\{} \texttt{\$\char`\}}; the end
of the outermost block is the end of the database.

% LaTeX bug? can't do \bf\tt

A {\bf \$v} or {\bf \$c statement}\index{\texttt{\$v} statement}\index{\texttt{\$c}
statement} consists of the keyword token \texttt{\$v} or \texttt{\$c} respectively,
followed by one or more math symbols,
% The word "token" is used to distinguish "$." from the sentence-ending period.
followed by the \texttt{\$.}\ token.
These
statements {\bf declare}\index{declaration} the math symbols to be {\bf
variables}\index{variable!Metamath} or {\bf constants}\index{constant}
respectively. The same math symbol may not occur twice in a given \texttt{\$v} or
\texttt{\$c} statement.

%c%A math symbol becomes an {\bf active}\index{active math symbol}
%c%when declared and stays active until the end of the block in which it is
%c%declared.  A math symbol may not be declared a second time while it is active,
%c%but it may be declared again after it becomes inactive.

A math symbol becomes {\bf active}\index{active math symbol} when declared
and stays active until the end of the block in which it is declared.  A
variable may not be declared a second time while it is active, but it
may be declared again (as a variable, but not as a constant) after it
becomes inactive.  A constant must be declared in the outermost block and may
not be declared a second time.\index{redeclaration of symbols}

A {\bf \$f statement}\index{\texttt{\$f} statement} consists of a label,
followed by \texttt{\$f}, followed by its typecode (an active constant),
followed by an
active variable, followed by the \texttt{\$.}\ token.  A {\bf \$e
statement}\index{\texttt{\$e} statement} consists of a label, followed
by \texttt{\$e}, followed by its typecode (an active constant),
followed by zero or more
active math symbols, followed by the \texttt{\$.}\ token.  A {\bf
hypothesis}\index{hypothesis} is a \texttt{\$f} or \texttt{\$e}
statement.
The type declared by a \texttt{\$f} statement for a given label
is global even if the variable is not
(e.g., a database may not have \texttt{wff P} in one local scope
and \texttt{class P} in another).

A {\bf simple \$d statement}\index{\texttt{\$d} statement!simple}
consists of \texttt{\$d}, followed by two different active variables,
followed by the \texttt{\$.}\ token.  A {\bf compound \$d
statement}\index{\texttt{\$d} statement!compound} consists of
\texttt{\$d}, followed by three or more variables (all different),
followed by the \texttt{\$.}\ token.  The order of the variables in a
\texttt{\$d} statement is unimportant.  A compound \texttt{\$d}
statement is equivalent to a set of simple \texttt{\$d} statements, one
for each possible pair of variables occurring in the compound
\texttt{\$d} statement.  Henceforth in this specification we shall
assume all \texttt{\$d} statements are simple.  A \texttt{\$d} statement
is also called a {\bf disjoint} (or {\bf distinct}) {\bf variable
restriction}.\index{disjoint-variable restriction}

A {\bf \$a statement}\index{\texttt{\$a} statement} consists of a label,
followed by \texttt{\$a}, followed by its typecode (an active constant),
followed by
zero or more active math symbols, followed by the \texttt{\$.}\ token.  A {\bf
\$p statement}\index{\texttt{\$p} statement} consists of a label,
followed by \texttt{\$p}, followed by its typecode (an active constant),
followed by
zero or more active math symbols, followed by \texttt{\$=}, followed by
a sequence of labels, followed by the \texttt{\$.}\ token.  An {\bf
assertion}\index{assertion} is a \texttt{\$a} or \texttt{\$p} statement.

A \texttt{\$f}, \texttt{\$e}, or \texttt{\$d} statement is {\bf active}\index{active
statement} from the place it occurs until the end of the block it occurs in.
A \texttt{\$a} or \texttt{\$p} statement is {\bf active} from the place it occurs
through the end of the database.
There may not be two active \texttt{\$f} statements containing the same
variable.  Each variable in a \texttt{\$e}, \texttt{\$a}, or
\texttt{\$p} statement must exist in an active \texttt{\$f}
statement.\footnote{This requirement can greatly simplify the
unification algorithm (substitution calculation) required by proof
verification.}

%The label that begins each \texttt{\$f}, \texttt{\$e}, \texttt{\$a}, and
%\texttt{\$p} statement must be unique.
Each label token must be unique, and
no label token may match any math symbol
token.\label{namespace}\footnote{This
restriction was added on June 24, 2006.
It is not theoretically necessary but is imposed to make it easier to
write certain parsers.}

The set of {\bf mandatory variables}\index{mandatory variable} associated with
an assertion is the set of (zero or more) variables in the assertion and in any
active \texttt{\$e} statements.  The (possibly empty) set of {\bf mandatory
hypotheses}\index{mandatory hypothesis} is the set of all active \texttt{\$f}
statements containing mandatory variables, together with all active \texttt{\$e}
statements.
The set of {\bf mandatory {\bf \$d} statements}\index{mandatory
disjoint-variable restriction} associated with an assertion are those active
\texttt{\$d} statements whose variables are both among the assertion's
mandatory variables.

\subsection{Proof Verification}\label{spec4}

The sequence of labels between the \texttt{\$=} and \texttt{\$.}\ tokens
in a \texttt{\$p} statement is a {\bf proof}.\index{proof!Metamath} Each
label in a proof must be the label of an active statement other than the
\texttt{\$p} statement itself; thus a label must refer either to an
active hypothesis of the \texttt{\$p} statement or to an earlier
assertion.

An {\bf expression}\index{expression} is a sequence of math symbols. A {\bf
substitution map}\index{substitution map} associates a set of variables with a
set of expressions.  It is acceptable for a variable to be mapped to an
expression containing it.  A {\bf
substitution}\index{substitution!variable}\index{variable substitution} is the
simultaneous replacement of all variables in one or more expressions with the
expressions that the variables map to.

A proof is scanned in order of its label sequence.  If the label refers to an
active hypothesis, the expression in the hypothesis is pushed onto a
stack.\index{stack}\index{RPN stack}  If the label refers to an assertion, a
(unique) substitution must exist that, when made to the mandatory hypotheses
of the referenced assertion, causes them to match the topmost (i.e.\ most
recent) entries of the stack, in order of occurrence of the mandatory
hypotheses, with the topmost stack entry matching the last mandatory
hypothesis of the referenced assertion.  As many stack entries as there are
mandatory hypotheses are then popped from the stack.  The same substitution is
made to the referenced assertion, and the result is pushed onto the stack.
After the last label in the proof is processed, the stack must have a single
entry that matches the expression in the \texttt{\$p} statement containing the
proof.

%c%{\footnotesize\begin{quotation}\index{redeclaration of symbols}
%c%{{\em Comment.}\label{spec4comment} Whenever a math symbol token occurs in a
%c%{\texttt{\$c} or \texttt{\$v} statement, it is considered to designate a distinct new
%c%{symbol, even if the same token was previously declared (and is now inactive).
%c%{Thus a math token declared as a constant in two different blocks is considered
%c%{to designate two distinct constants (even though they have the same name).
%c%{The two constants will not match in a proof that references both blocks.
%c%{However, a proof referencing both blocks is acceptable as long as it doesn't
%c%{require that the constants match.  Similarly, a token declared to be a
%c%{constant for a referenced assertion will not match the same token declared to
%c%{be a variable for the \texttt{\$p} statement containing the proof.  In the case
%c%{of a token declared to be a variable for a referenced assertion, this is not
%c%{an issue since the variable can be substituted with whatever expression is
%c%{needed to achieve the required match.
%c%{\end{quotation}}
%c2%A proof may reference an assertion that contains or whose hypotheses contain a
%c2%constant that is not active for the \texttt{\$p} statement containing the proof.
%c2%However, the final result of the proof may not contain that constant. A proof
%c2%may also reference an assertion that contains or whose hypotheses contain a
%c2%variable that is not active for the \texttt{\$p} statement containing the proof.
%c2%That variable, of course, will be substituted with whatever expression is
%c2%needed to achieve the required match.

A proof may contain a \texttt{?}\ in place of a label to indicate an unknown step
(Section~\ref{unknown}).  A proof verifier may ignore any proof containing
\texttt{?}\ but should warn the user that the proof is incomplete.

A {\bf compressed proof}\index{compressed proof}\index{proof!compressed} is an
alternate proof notation described in Appen\-dix~\ref{compressed}; also see
references to ``compressed proof'' in the Index.  Compressed proofs are a
Metamath language extension which a complete proof verifier should be able to
parse and verify.

\subsubsection{Verifying Disjoint Variable Restrictions}

Each substitution made in a proof must be checked to verify that any
disjoint variable restrictions are satisfied, as follows.

If two variables replaced by a substitution exist in a mandatory \texttt{\$d}
statement\index{\texttt{\$d} statement} of the assertion referenced, the two
expressions resulting from the substitution must satisfy the following
conditions.  First, the two expressions must have no variables in common.
Second, each possible pair of variables, one from each expression, must exist
in an active \texttt{\$d} statement of the \texttt{\$p} statement containing the
proof.

\vskip 1ex

This ends the specification of the Metamath language;
see Appendix \ref{BNF} for its syntax in
Extended Backus--Naur Form (EBNF)\index{Extended Backus--Naur Form}\index{EBNF}.

\section{The Basic Keywords}\label{tut1}

Our expository material begins here.

Like most computer languages, Metamath\index{Metamath} takes its input from
one or more {\bf source files}\index{source file} which contain characters
expressed in the standard {\sc ascii} (American Standard Code for Information
Interchange)\index{ascii@{\sc ascii}} code for computers.  A source file
consists of a series of {\bf tokens}\index{token}, which are strings of
non-whitespace
printable characters (from the set of 94 shown on p.~\pageref{spec1chars})
separated by {\bf white space}\index{white space} (spaces, tabs, carriage
returns, line feeds, and form feeds). Any string consisting only of these
characters is treated the same as a single space.  The non-whitespace printable
characters\index{printable character} that Metamath recognizes are the 94
characters on standard {\sc ascii} keyboards.

Metamath has the ability to join several files together to form its
input (Section~\ref{include}).  We call the aggregate contents of all
the files after they have been joined together a {\bf
database}\index{database} to distinguish it from an individual source
file.  The tokens in a database consist of {\bf
keywords}\index{keyword}, which are built into the language, together
with two kinds of user-defined tokens called {\bf labels}\index{label}
and {\bf math symbols}\index{math symbol}.  (Often we will simply say
{\bf symbol}\index{symbol} instead of math symbol for brevity).  The set
of {\bf basic keywords}\index{basic keyword} is
\texttt{\$c}\index{\texttt{\$c} statement},
\texttt{\$v}\index{\texttt{\$v} statement},
\texttt{\$e}\index{\texttt{\$e} statement},
\texttt{\$f}\index{\texttt{\$f} statement},
\texttt{\$d}\index{\texttt{\$d} statement},
\texttt{\$a}\index{\texttt{\$a} statement},
\texttt{\$p}\index{\texttt{\$p} statement},
\texttt{\$=}\index{\texttt{\$=} keyword},
\texttt{\$.}\index{\texttt{\$.}\ keyword},
\texttt{\$\char`\{}\index{\texttt{\$\char`\{} and \texttt{\$\char`\}}
keywords}, and \texttt{\$\char`\}}.  This is the complete set of
syntactical elements of what we call the {\bf basic
language}\index{basic language} of Metamath, and with them you can
express all of the mathematics that were intended by the design of
Metamath.  You should make it a point to become very familiar with them.
Table~\ref{basickeywords} lists the basic keywords along with a brief
description of their functions.  For now, this description will give you
only a vague notion of what the keywords are for; later we will describe
the keywords in detail.


\begin{table}[htp] \caption{Summary of the basic Metamath
keywords} \label{basickeywords}
\begin{center}
\begin{tabular}{|p{4pc}|l|}
\hline
\em \centering Keyword&\em Description\\
\hline
\hline
\centering
   \texttt{\$c}&Constant symbol declaration\\
\hline
\centering
   \texttt{\$v}&Variable symbol declaration\\
\hline
\centering
   \texttt{\$d}&Disjoint variable restriction\\
\hline
\centering
   \texttt{\$f}&Variable-type (``floating'') hypothesis\\
\hline
\centering
   \texttt{\$e}&Logical (``essential'') hypothesis\\
\hline
\centering
   \texttt{\$a}&Axiomatic assertion\\
\hline
\centering
   \texttt{\$p}&Provable assertion\\
\hline
\centering
   \texttt{\$=}&Start of proof in \texttt{\$p} statement\\
\hline
\centering
   \texttt{\$.}&End of the above statement types\\
\hline
\centering
   \texttt{\$\char`\{}&Start of block\\
\hline
\centering
   \texttt{\$\char`\}}&End of block\\
\hline
\end{tabular}
\end{center}
\end{table}

%For LaTeX bug(?) where it puts tables on blank page instead of btwn text
%May have to adjust if text changes
%\newpage

There are some additional keywords, called {\bf auxiliary
keywords}\index{auxiliary keyword} that help make Metamath\index{Metamath}
more practical. These are part of the {\bf extended language}\index{extended
language}. They provide you with a means to put comments into a Metamath
source file\index{source file} and reference other source files.  We will
introduce these in later sections. Table~\ref{otherkeywords} summarizes them
so that you can recognize them now if you want to peruse some source
files while learning the basic keywords.


\begin{table}[htp] \caption{Auxiliary Metamath
keywords} \label{otherkeywords}
\begin{center}
\begin{tabular}{|p{4pc}|l|}
\hline
\em \centering Keyword&\em Description\\
\hline
\hline
\centering
   \texttt{\$(}&Start of comment\\
\hline
\centering
   \texttt{\$)}&End of comment\\
\hline
\centering
   \texttt{\$[}&Start of included source file name\\
\hline
\centering
   \texttt{\$]}&End of included source file name\\
\hline
\end{tabular}
\end{center}
\end{table}
\index{\texttt{\$(} and \texttt{\$)} auxiliary keywords}
\index{\texttt{\$[} and \texttt{\$]} auxiliary keywords}


Unlike those in some computer languages, the keywords\index{keyword} are short
two-character sequences rather than English-like words.  While this may make
them slightly more difficult to remember at first, their brevity allows
them to blend in with the mathematics being described, not
distract from it, like punctuation marks.


\subsection{User-Defined Tokens}\label{dollardollar}\index{token}

As you may have noticed, all keywords\index{keyword} begin with the \texttt{\$}
character.  This mundane monetary symbol is not ordinarily used in higher
mathematics (outside of grant proposals), so we have appropriated it to
distinguish the Metamath\index{Metamath} keywords from ordinary mathematical
symbols. The \texttt{\$} character is thus considered special and may not be
used as a character in a user-defined token.  All tokens and keywords are
case-sensitive; for example, \texttt{n} is considered to be a different character
from \texttt{N}.  Case-sensitivity makes the available {\sc ascii} character set
as rich as possible.

\subsubsection{Math Symbol Tokens}\index{token}

Math symbols\index{math symbol} are tokens used to represent the symbols
that appear in ordinary mathematical formulas.  They may consist of any
combination of the 93 non-whitespace printable {\sc ascii} characters other than
\texttt{\$}~. Some examples are \texttt{x}, \texttt{+}, \texttt{(},
\texttt{|-}, \verb$!%@?&$, and \texttt{bounded}.  For readability, it is
best to try to make these look as similar to actual mathematical symbols
as possible, within the constraints of the {\sc ascii} character set, in
order to make the resulting mathematical expressions more readable.

In the Metamath\index{Metamath} language, you express ordinary
mathematical formulas and statements as sequences of math symbols such
as \texttt{2 + 2 = 4} (five symbols, all constants).\footnote{To
eliminate ambiguity with other expressions, this is expressed in the set
theory database \texttt{set.mm} as \texttt{|- ( 2 + 2
 ) = 4 }, whose \LaTeX\ equivalent is $\vdash
(2+2)=4$.  The \,$\vdash$ means ``is a theorem'' and the
parentheses allow explicit associative grouping.}\index{turnstile
({$\,\vdash$})} They may even be English
sentences, as in \texttt{E is closed and bounded} (five symbols)---here
\texttt{E} would be a variable and the other four symbols constants.  In
principle, a Metamath database could be constructed to work with almost
any unambiguous English-language mathematical statement, but as a
practical matter the definitions needed to provide for all possible
syntax variations would be cumbersome and distracting and possibly have
subtle pitfalls accidentally built in.  We generally recommend that you
express mathematical statements with compact standard mathematical
symbols whenever possible and put their English-language descriptions in
comments.  Axioms\index{axiom} and definitions\index{definition}
(\texttt{\$a}\index{\texttt{\$a} statement} statements) are the only
places where Metamath will not detect an error, and doing this will help
reduce the number of definitions needed.

You are free to use any tokens\index{token} you like for math
symbols\index{math symbol}.  Appendix~\ref{ASCII} recommends token names to
use for symbols in set theory, and we suggest you adopt these in order to be
able to include the \texttt{set.mm} set theory database in your database.  For
printouts, you can convert the tokens in a database
to standard mathematical symbols with the \LaTeX\ typesetting program.  The
Metamath command \texttt{open tex} {\em filename}\index{\texttt{open tex} command}
produces output that can be read by \LaTeX.\index{latex@{\LaTeX}}
The correspondence
between tokens and the actual symbols is made by \texttt{latexdef}
statements inside a special database comment tagged
with \texttt{\$t}.\index{\texttt{\$t} comment}\index{typesetting comment}
  You can edit
this comment to change the definitions or add new ones.
Appendix~\ref{ASCII} describes how to do this in more detail.

% White space\index{white space} is normally used to separate math
% symbol\index{math symbol} tokens, but they may be juxtaposed without white
% space in \texttt{\$d}\index{\texttt{\$d} statement}, \texttt{\$e}\index{\texttt{\$e}
% statement}, \texttt{\$f}\index{\texttt{\$f} statement}, \texttt{\$a}\index{\texttt{\$a}
% statement}, and \texttt{\$p}\index{\texttt{\$p} statement} statements when no
% ambiguity will result.  Specifically, Metamath parses the math symbol sequence
% in one of these statements in the following manner:  when the math symbol
% sequence has been broken up into tokens\index{token} up to a given character,
% the next token is the longest string of characters that could constitute a
% math symbol that is active\index{active
% math symbol} at that point.  (See Section~\ref{scoping} for the
% definition of an active math symbol.)  For example, if \texttt{-}, \texttt{>}, and
% \texttt{->} are the only active math symbols, the juxtaposition \texttt{>-} will be
% interpreted as the two symbols \texttt{>} and \texttt{-}, whereas \texttt{->} will
% always be interpreted as that single symbol.\footnote{For better readability we
% recommend a white space between each token.  This also makes searching for a
% symbol easier to do with an editor.  Omission of optional white space is useful
% for reducing typing when assigning an expression to a temporary
% variable\index{temporary variable} with the \texttt{let variable} Metamath
% program command.}\index{\texttt{let variable} command}
%
% Keywords\index{keyword} may be placed next to math symbols without white
% space\index{white space} between them.\footnote{Again, we do not recommend
% this for readability.}
%
% The math symbols\index{math symbol} in \texttt{\$c}\index{\texttt{\$c} statement}
% and \texttt{\$v}\index{\texttt{\$v} statement} statements must always be separated
% by white space\index{white
% space}, for the obvious reason that these statements define the names
% of the symbols.
%
% Math symbols referred to in comments (see Section~\ref{comments}) must also be
% separated by white space.  This allows you to make comments about symbols that
% are not yet active\index{active
% math symbol}.  (The ``math mode'' feature of comments is also a quick and
% easy way to obtain word processing text with embedded mathematical symbols,
% independently of the main purpose of Metamath; the way to do this is described
% in Section~\ref{comments})

\subsubsection{Label Tokens}\index{token}\index{label}

Label tokens are used to identify Metamath\index{Metamath} statements for
later reference. Label tokens may contain only letters, digits, and the three
characters period, hyphen, and underscore:
\begin{verbatim}
. - _
\end{verbatim}

A label is {\bf declared}\index{label declaration} by placing it immediately
before the keyword of the statement it identifies.  For example, the label
\texttt{axiom.1} might be declared as follows:
\begin{verbatim}
axiom.1 $a |- x = x $.
\end{verbatim}

Each \texttt{\$e}\index{\texttt{\$e} statement},
\texttt{\$f}\index{\texttt{\$f} statement},
\texttt{\$a}\index{\texttt{\$a} statement}, and
\texttt{\$p}\index{\texttt{\$p} statement} statement in a database must
have a label declared for it.  No other statement types may have label
declarations.  Every label must be unique.

A label (and the statement it identifies) is {\bf referenced}\index{label
reference} by including the label between the \texttt{\$=}\index{\texttt{\$=}
keyword} and \texttt{\$.}\index{\texttt{\$.}\ keyword}\ keywords in a \texttt{\$p}
statement.  The sequence of labels\index{label sequence} between these two
keywords is called a {\bf proof}\index{proof}.  An example of a statement with
a proof that we will encounter later (Section~\ref{proof}) is
\begin{verbatim}
wnew $p wff ( s -> ( r -> p ) )
     $= ws wr wp w2 w2 $.
\end{verbatim}

You don't have to know what this means just yet, but you should know that the
label \texttt{wnew} is declared by this \texttt{\$p} statement and that the labels
\texttt{ws}, \texttt{wr}, \texttt{wp}, and \texttt{w2} are assumed to have been declared
earlier in the database and are referenced here.

\subsection{Constants and Variables}
\index{constant}
\index{variable}

An {\bf expression}\index{expression} is any sequence of math
symbols, possibly empty.

The basic Metamath\index{Metamath} language\index{basic language} has two
kinds of math symbols\index{math symbol}:  {\bf constants}\index{constant} and
{\bf variables}\index{variable}.  In a Metamath proof, a constant may not be
substituted with any expression.  A variable can be
substituted\index{substitution!variable}\index{variable substitution} with any
expression.  This sequence may include other variables and may even include
the variable being substituted.  This substitution takes place when proofs are
verified, and it will be described in Section~\ref{proof}.  The \texttt{\$f}
statement (described later in Section~\ref{dollaref}) is used to specify the
{\bf type} of a variable (i.e.\ what kind of
variable it is)\index{variable type}\index{type} and
give it a meaning typically
associated with a ``metavariable''\index{metavariable}\footnote{A metavariable
is a variable that ranges over the syntactical elements of the object language
being discussed; for example, one metavariable might represent a variable of
the object language and another metavariable might represent a formula in the
object language.} in ordinary mathematics; for example, a variable may be
specified to be a wff or well-formed formula (in logic), a set (in set
theory), or a non-negative integer (in number theory).

%\subsection{The \texttt{\$c} and \texttt{\$v} Declaration Statements}
\subsection{The \texttt{\$c} and \texttt{\$v} Declaration Statements}
\index{\texttt{\$c} statement}
\index{constant declaration}
\index{\texttt{\$v} statement}
\index{variable declaration}

Constants are introduced or {\bf declared}\index{constant declaration}
with \texttt{\$c}\index{\texttt{\$c} statement} statements, and
variables are declared\index{variable declaration} with
\texttt{\$v}\index{\texttt{\$v} statement} statements.  A {\bf simple}
declaration\index{simple declaration} statement introduces a single
constant or variable.  Its syntax is one of the following:
\begin{center}
  \texttt{\$c} {\em math-symbol} \texttt{\$.}\\
  \texttt{\$v} {\em math-symbol} \texttt{\$.}
\end{center}
The notation {\em math-symbol} means any math symbol token\index{token}.

Some examples of simple declaration statements are:
\begin{center}
  \texttt{\$c + \$.}\\
  \texttt{\$c -> \$.}\\
  \texttt{\$c ( \$.}\\
  \texttt{\$v x \$.}\\
  \texttt{\$v y2 \$.}
\end{center}

The characters in a math symbol\index{math symbol} being declared are
irrelevant to Meta\-math; for example, we could declare a right parenthesis to
be a variable,
\begin{center}
  \texttt{\$v ) \$.}\\
\end{center}
although this would be unconventional.

A {\bf compound} declaration\index{compound declaration} statement is a
shorthand for declaring several symbols at once.  Its syntax is one of the
following:
\begin{center}
  \texttt{\$c} {\em math-symbol}\ \,$\cdots$\ {\em math-symbol} \texttt{\$.}\\
  \texttt{\$v} {\em math-symbol}\ \,$\cdots$\ {\em math-symbol} \texttt{\$.}
\end{center}\index{\texttt{\$c} statement}
Here, the ellipsis (\ldots) means any number of {\em math-symbol}\,s.

An example of a compound declaration statement is:
\begin{center}
  \texttt{\$v x y mu \$.}\\
\end{center}
This is equivalent to the three simple declaration statements
\begin{center}
  \texttt{\$v x \$.}\\
  \texttt{\$v y \$.}\\
  \texttt{\$v mu \$.}\\
\end{center}
\index{\texttt{\$v} statement}

There are certain rules on where in the database math symbols may be declared,
what sections of the database are aware of them (i.e.\ where they are
``active''), and when they may be declared more than once.  These will be
discussed in Section~\ref{scoping} and specifically on
p.~\pageref{redeclaration}.

\subsection{The \texttt{\$d} Statement}\label{dollard}
\index{\texttt{\$d} statement}

The \texttt{\$d} statement is called a {\bf disjoint-variable restriction}.  The
syntax of the {\bf simple} version of this statement is
\begin{center}
  \texttt{\$d} {\em variable variable} \texttt{\$.}
\end{center}
where each {\em variable} is a previously declared variable and the two {\em
variable}\,s are different.  (More specifically, each  {\em variable} must be
an {\bf active} variable\index{active math symbol}, which means there must be
a previous \texttt{\$v} statement whose {\bf scope}\index{scope} includes the
\texttt{\$d} statement.  These terms will be defined when we discuss scoping
statements in Section~\ref{scoping}.)

In ordinary mathematics, formulas may arise that are true if the variables in
them are distinct\index{distinct variables}, but become false when those
variables are made identical. For example, the formula in logic $\exists x\,x
\neq y$, which means ``for a given $y$, there exists an $x$ that is not equal
to $y$,'' is true in most mathematical theories (namely all non-trivial
theories\index{non-trivial theory}, i.e.\ those that describe more than one
individual, such as arithmetic).  However, if we substitute $y$ with $x$, we
obtain $\exists x\,x \neq x$, which is always false, as it means ``there
exists something that is not equal to itself.''\footnote{If you are a
logician, you will recognize this as the improper substitution\index{proper
substitution}\index{substitution!proper} of a free variable\index{free
variable} with a bound variable\index{bound variable}.  Metamath makes no
inherent distinction between free and bound variables; instead, you let
Metamath know what substitutions are permissible by using \texttt{\$d} statements
in the right way in your axiom system.}\index{free vs.\ bound variable}  The
\texttt{\$d} statement allows you to specify a restriction that forbids the
substitution of one variable with another.  In
this case, we would use the statement
\begin{center}
  \texttt{\$d x y \$.}
\end{center}\index{\texttt{\$d} statement}
to specify this restriction.

The order in which the variables appear in a \texttt{\$d} statement is not
important.  We could also use
\begin{center}
  \texttt{\$d y x \$.}
\end{center}

The \texttt{\$d} statement is actually more general than this, as the
``disjoint''\index{disjoint variables} in its name suggests.  The full meaning
is that if any substitution is made to its two variables (during the
course of a proof that references a \texttt{\$a} or \texttt{\$p} statement
associated with the \texttt{\$d}), the two expressions that result from the
substitution must have no variables in common.  In addition, each possible
pair of variables, one from each expression, must be in a \texttt{\$d} statement
associated with the statement being proved.  (This requirement forces the
statement being proved to ``inherit'' the original disjoint variable
restriction.)

For example, suppose \texttt{u} is a variable.  If the restriction
\begin{center}
  \texttt{\$d A B \$.}
\end{center}
has been specified for a theorem referenced in a
proof, we may not substitute \texttt{A} with \mbox{\tt a + u} and
\texttt{B} with \mbox{\tt b + u} because these two symbol sequences have the
variable \texttt{u} in common.  Furthermore, if \texttt{a} and \texttt{b} are
variables, we may not substitute \texttt{A} with \texttt{a} and \texttt{B} with \texttt{b}
unless we have also specified \texttt{\$d a b} for the theorem being proved; in
other words, the \texttt{\$d} property associated with a pair of variables must
be effectively preserved after substitution.

The \texttt{\$d}\index{\texttt{\$d} statement} statement does {\em not} mean ``the
two variables may not be substituted with the same thing,'' as you might think
at first.  For example, substituting each of \texttt{A} and \texttt{B} in the above
example with identical symbol sequences consisting only of constants does not
cause a disjoint variable conflict, because two symbol sequences have no
variables in common (since they have no variables, period).  Similarly, a
conflict will not occur by substituting the two variables in a \texttt{\$d}
statement with the empty symbol sequence\index{empty substitution}.

The \texttt{\$d} statement does not have a direct counterpart in
ordinary mathematics, partly because the variables\index{variable} of
Metamath are not really the same as the variables\index{variable!in
ordinary mathematics} of ordinary mathematics but rather are
metavariables\index{metavariable} ranging over them (as well as over
other kinds of symbols and groups of symbols).  Depending on the
situation, we may informally interpret the \texttt{\$d} statement in
different ways.  Suppose, for example, that \texttt{x} and \texttt{y}
are variables ranging over numbers (more precisely, that \texttt{x} and
\texttt{y} are metavariables ranging over variables that range over
numbers), and that \texttt{ph} ($\varphi$) and \texttt{ps} ($\psi$) are
variables (more precisely, metavariables) ranging over formulas.  We can
make the following interpretations that correspond to the informal
language of ordinary mathematics:
\begin{quote}
\begin{tabbing}
\texttt{\$d x y \$.} means ``assume $x$ and $y$ are
distinct variables.''\\
\texttt{\$d x ph \$.} means ``assume $x$ does not
occur in $\varphi$.''\\
\texttt{\$d ph ps \$.} \=means ``assume $\varphi$ and
$\psi$ have no variables\\ \>in common.''
\end{tabbing}
\end{quote}\index{\texttt{\$d} statement}

\subsubsection{Compound \texttt{\$d} Statements}

The {\bf compound} version of the \texttt{\$d} statement is a shorthand for
specifying several variables whose substitutions must be pairwise disjoint.
Its syntax is:
\begin{center}
  \texttt{\$d} {\em variable}\ \,$\cdots$\ {\em variable} \texttt{\$.}
\end{center}\index{\texttt{\$d} statement}
Here, {\em variable} represents the token of a previously declared
variable (specifically, an active variable) and all {\em variable}\,s are
different.  The compound \texttt{\$d}
statement is internally broken up by Metamath into one simple \texttt{\$d}
statement for each possible pair of variables in the original \texttt{\$d}
statement.  For example,
\begin{center}
  \texttt{\$d w x y z \$.}
\end{center}
is equivalent to
\begin{center}
  \texttt{\$d w x \$.}\\
  \texttt{\$d w y \$.}\\
  \texttt{\$d w z \$.}\\
  \texttt{\$d x y \$.}\\
  \texttt{\$d x z \$.}\\
  \texttt{\$d y z \$.}
\end{center}

Two or more simple \texttt{\$d} statements specifying the same variable pair are
internally combined into a single \texttt{\$d} statement.  Thus the set of three
statements
\begin{center}
  \texttt{\$d x y \$.}
  \texttt{\$d x y \$.}
  \texttt{\$d y x \$.}
\end{center}
is equivalent to
\begin{center}
  \texttt{\$d x y \$.}
\end{center}

Similarly, compound \texttt{\$d} statements, after being internally broken up,
internally have their common variable pairs combined.  For example the
set of statements
\begin{center}
  \texttt{\$d x y A \$.}
  \texttt{\$d x y B \$.}
\end{center}
is equivalent to
\begin{center}
  \texttt{\$d x y \$.}
  \texttt{\$d x A \$.}
  \texttt{\$d y A \$.}
  \texttt{\$d x y \$.}
  \texttt{\$d x B \$.}
  \texttt{\$d y B \$.}
\end{center}
which is equivalent to
\begin{center}
  \texttt{\$d x y \$.}
  \texttt{\$d x A \$.}
  \texttt{\$d y A \$.}
  \texttt{\$d x B \$.}
  \texttt{\$d y B \$.}
\end{center}

Metamath\index{Metamath} automatically verifies that all \texttt{\$d}
restrictions are met whenever it verifies proofs.  \texttt{\$d} statements are
never referenced directly in proofs (this is why they do not have
labels\index{label}), but Metamath is always aware of which ones must be
satisfied (i.e.\ are active) and will notify you with an error message if any
violation occurs.

To illustrate how Metamath detects a missing \texttt{\$d}
statement, we will look at the following example from the
\texttt{set.mm} database.

\begin{verbatim}
$d x z $.  $d y z $.
$( Theorem to add distinct quantifier to atomic formula. $)
ax17eq $p |- ( x = y -> A. z x = y ) $=...
\end{verbatim}

This statement has the obvious requirement that $z$ must be
distinct\index{distinct variables} from $x$ in theorem \texttt{ax17eq} that
states $x=y \rightarrow \forall z \, x=y$ (well, obvious if you're a logician,
for otherwise we could conclude  $x=y \rightarrow \forall x \, x=y$, which is
false when the free variables $x$ and $y$ are equal).

Let's look at what happens if we edit the database to comment out this
requirement.

\begin{verbatim}
$( $d x z $. $) $d y z $.
$( Theorem to add distinct quantifier to atomic formula. $)
ax17eq $p |- ( x = y -> A. z x = y ) $=...
\end{verbatim}

When it tries to verify the proof, Metamath will tell you that \texttt{x} and
\texttt{z} must be disjoint, because one of its steps references an axiom or
theorem that has this requirement.

\begin{verbatim}
MM> verify proof ax17eq
ax17eq ?Error at statement 1918, label "ax17eq", type "$p":
      vz wal wi vx vy vz ax-13 vx vy weq vz vx ax-c16 vx vy
                                               ^^^^^
There is a disjoint variable ($d) violation at proof step 29.
Assertion "ax-c16" requires that variables "x" and "y" be
disjoint.  But "x" was substituted with "z" and "y" was
substituted with "x".  The assertion being proved, "ax17eq",
does not require that variables "z" and "x" be disjoint.
\end{verbatim}

We can see the substitutions into \texttt{ax-c16} with the following command.

\begin{verbatim}
MM> show proof ax17eq / detailed_step 29
Proof step 29:  pm2.61dd.2=ax-c16 $a |- ( A. z z = x -> ( x =
  y -> A. z x = y ) )
This step assigns source "ax-c16" ($a) to target "pm2.61dd.2"
($e).  The source assertion requires the hypotheses "wph"
($f, step 26), "vx" ($f, step 27), and "vy" ($f, step 28).
The parent assertion of the target hypothesis is "pm2.61dd"
($p, step 36).
The source assertion before substitution was:
    ax-c16 $a |- ( A. x x = y -> ( ph -> A. x ph ) )
The following substitutions were made to the source
assertion:
    Variable  Substituted with
     x         z
     y         x
     ph        x = y
The target hypothesis before substitution was:
    pm2.61dd.2 $e |- ( ph -> ch )
The following substitutions were made to the target
hypothesis:
    Variable  Substituted with
     ph        A. z z = x
     ch        ( x = y -> A. z x = y )
\end{verbatim}

The disjoint variable restrictions of \texttt{ax-c16} can be seen from the
\texttt{show state\-ment} command.  The line that begins ``\texttt{Its mandatory
dis\-joint var\-i\-able pairs are:}\ldots'' lists any \texttt{\$d} variable
pairs in brackets.

\begin{verbatim}
MM> show statement ax-c16/full
Statement 3033 is located on line 9338 of the file "set.mm".
"Axiom of Distinct Variables. ..."
  ax-c16 $a |- ( A. x x = y -> ( ph -> A. x ph ) ) $.
Its mandatory hypotheses in RPN order are:
  wph $f wff ph $.
  vx $f setvar x $.
  vy $f setvar y $.
Its mandatory disjoint variable pairs are:  <x,y>
The statement and its hypotheses require the variables:  x y
      ph
The variables it contains are:  x y ph
\end{verbatim}

Since Metamath will always detect when \texttt{\$d}\index{\texttt{\$d} statement}
statements are needed for a proof, you don't have to worry too much about
forgetting to put one in; it can always be added if you see the error message
above.  If you put in unnecessary \texttt{\$d} statements, the worst that could
happen is that your theorem might not be as general as it could be, and this
may limit its use later on.

On the other hand, when you introduce axioms (\texttt{\$a}\index{\texttt{\$a}
statement} statements), you must be very careful to properly specify the
necessary associated \texttt{\$d} statements since Metamath has no way of knowing
whether your axioms are correct.  For example, Metamath would have no idea
that \texttt{ax-c16}, which we are telling it is an axiom of logic, would lead to
contradictions if we omitted its associated \texttt{\$d} statement.

% This was previously a comment in footnote-sized type, but it can be
% hard to read this much text in a small size.
% As a result, it's been changed to normally-sized text.
\label{nodd}
You may wonder if it is possible to develop standard
mathematics in the Metamath language without the \texttt{\$d}\index{\texttt{\$d}
statement} statement, since it seems like a nuisance that complicates proof
verification. The \texttt{\$d} statement is not needed in certain subsets of
mathematics such as propositional calculus.  However, dummy
variables\index{dummy variable!eliminating} and their associated \texttt{\$d}
statements are impossible to avoid in proofs in standard first-order logic as
well as in the variant used in \texttt{set.mm}.  In fact, there is no upper bound to
the number of dummy variables that might be needed in a proof of a theorem of
first-order logic containing 3 or more variables, as shown by H.\
Andr\'{e}ka\index{Andr{\'{e}}ka, H.} \cite{Nemeti}.  A first-order system that
avoids them entirely is given in \cite{Megill}\index{Megill, Norman}; the
trick there is simply to embed harmlessly the necessary dummy variables into a
theorem being proved so that they aren't ``dummy'' anymore, then interpret the
resulting longer theorem so as to ignore the embedded dummy variables.  If
this interests you, the system in \texttt{set.mm} obtained from \texttt{ax-1}
through \texttt{ax-c14} in \texttt{set.mm}, and deleting \texttt{ax-c16} and \texttt{ax-5},
requires no \texttt{\$d} statements but is logically complete in the sense
described in \cite{Megill}.  This means it can prove any theorem of
first-order logic as long as we add to the theorem an antecedent that embeds
dummy and any other variables that must be distinct.  In a similar fashion,
axioms for set theory can be devised that
do not require distinct variable
provisos\index{Set theory without distinct variable provisos},
as explained at
\url{http://us.metamath.org/mpeuni/mmzfcnd.html}.
Together, these in principle allow all of
mathematics to be developed under Metamath without a \texttt{\$d} statement,
although the length of the resulting theorems will grow as more and
more dummy variables become required in their proofs.

\subsection{The \texttt{\$f}
and \texttt{\$e} Statements}\label{dollaref}
\index{\texttt{\$e} statement}
\index{\texttt{\$f} statement}
\index{floating hypothesis}
\index{essential hypothesis}
\index{variable-type hypothesis}
\index{logical hypothesis}
\index{hypothesis}

Metamath has two kinds of hypo\-theses, the \texttt{\$f}\index{\texttt{\$f}
statement} or {\bf variable-type} hypothesis and the \texttt{\$e} or {\bf logical}
hypo\-the\-sis.\index{\texttt{\$d} statement}\footnote{Strictly speaking, the
\texttt{\$d} statement is also a hypothesis, but it is never directly referenced
in a proof, so we call it a restriction rather than a hypothesis to lessen
confusion.  The checking for violations of \texttt{\$d} restrictions is automatic
and built into Metamath's proof-checking algorithm.} The letters \texttt{f} and
\texttt{e} stand for ``floating''\index{floating hypothesis} (roughly meaning
used only if relevant) and ``essential''\index{essential hypothesis} (meaning
always used) respectively, for reasons that will become apparent
when we discuss frames in
Section~\ref{frames} and scoping in Section~\ref{scoping}. The syntax of these
are as follows:
\begin{center}
  {\em label} \texttt{\$f} {\em typecode} {\em variable} \texttt{\$.}\\
  {\em label} \texttt{\$e} {\em typecode}
      {\em math-symbol}\ \,$\cdots$\ {\em math-symbol} \texttt{\$.}\\
\end{center}
\index{\texttt{\$e} statement}
\index{\texttt{\$f} statement}
A hypothesis must have a {\em label}\index{label}.  The expression in a
\texttt{\$e} hypothesis consists of a typecode (an active constant math symbol)
followed by a sequence
of zero or more math symbols. Each math symbol (including {\em constant}
and {\em variable}) must be a previously declared constant or variable.  (In
addition, each math symbol must be active, which will be covered when we
discuss scoping statements in Section~\ref{scoping}.)  You use a \texttt{\$f}
hypothesis to specify the
nature or {\bf type}\index{variable type}\index{type} of a variable (such as ``let $x$ be an
integer'') and use a \texttt{\$e} hypothesis to express a logical truth (such as
``assume $x$ is prime'') that must be established in order for an assertion
requiring it to also be true.

A variable must have its type specified in a \texttt{\$f} statement before
it may be used in a \texttt{\$e}, \texttt{\$a}, or \texttt{\$p}
statement.  There may be only one (active) \texttt{\$f} statement for a
given variable.  (``Active'' is defined in Section~\ref{scoping}.)

In ordinary mathematics, theorems\index{theorem} are often expressed in the
form ``Assume $P$; then $Q$,'' where $Q$ is a statement that you can derive
if you start with statement $P$.\index{free variable}\footnote{A stronger
version of a theorem like this would be the {\em single} formula $P\rightarrow
Q$ ($P$ implies $Q$) from which the weaker version above follows by the rule
of modus ponens in logic.  We are not discussing this stronger form here.  In
the weaker form, we are saying only that if we can {\em prove} $P$, then we can
{\em prove} $Q$.  In a logician's language, if $x$ is the only free variable
in $P$ and $Q$, the stronger form is equivalent to $\forall x ( P \rightarrow
Q)$ (for all $x$, $P$ implies $Q$), whereas the weaker form is equivalent to
$\forall x P \rightarrow \forall x Q$. The stronger form implies the weaker,
but not vice-versa.  To be precise, the weaker form of the theorem is more
properly called an ``inference'' rather than a theorem.}\index{inference}
In the
Metamath\index{Metamath} language, you would express mathematical statement
$P$ as a hypothesis (a \texttt{\$e} Metamath language statement in this case) and
statement $Q$ as a provable assertion (a \texttt{\$p}\index{\texttt{\$p} statement}
statement).

Some examples of hypotheses you might encounter in logic and set theory are
\begin{center}
  \texttt{stmt1 \$f wff P \$.}\\
  \texttt{stmt2 \$f setvar x \$.}\\
  \texttt{stmt3 \$e |- ( P -> Q ) \$.}
\end{center}
\index{\texttt{\$e} statement}
\index{\texttt{\$f} statement}
Informally, these would be read, ``Let $P$ be a well-formed-formula,'' ``Let
$x$ be an (individual) variable,'' and ``Assume we have proved $P \rightarrow
Q$.''  The turnstile symbol \,$\vdash$\index{turnstile ({$\,\vdash$})} is
commonly used in logic texts to mean ``a proof exists for.''

To summarize:
\begin{itemize}
\item A \texttt{\$f} hypothesis tells Metamath the type or kind of its variable.
It is analogous to a variable declaration in a computer language that
tells the compiler that a variable is an integer or a floating-point
number.
\item The \texttt{\$e} hypothesis corresponds to what you would usually call a
``hypothesis'' in ordinary mathematics.
\end{itemize}

Before an assertion\index{assertion} (\texttt{\$a} or \texttt{\$p} statement) can be
referenced in a proof, all of its associated \texttt{\$f} and \texttt{\$e} hypotheses
(i.e.\ those \texttt{\$e} hypotheses that are active) must be satisfied (i.e.
established by the proof).  The meaning of ``associated'' (which we will call
{\bf mandatory} in Section~\ref{frames}) will become clear when we discuss
scoping later.

Note that after any \texttt{\$f}, \texttt{\$e},
\texttt{\$a}, or \texttt{\$p} token there is a required
\textit{typecode}\index{typecode}.
The typecode is a constant used to enforce types of expressions.
This will become clearer once we learn more about
assertions (\texttt{\$a} and \texttt{\$p} statements).
An example may also clarify their purpose.
In the
\texttt{set.mm}\index{set theory database (\texttt{set.mm})}%
\index{Metamath Proof Explorer}
database,
the following typecodes are used:

\begin{itemize}
\item \texttt{wff} :
  Well-formed formula (wff) symbol
  (read: ``the following symbol sequence is a wff'').
% The *textual* typecode for turnstile is "|-", but when read it's a little
% confusing, so I intentionally display the mathematical symbol here instead
% (I think it's clearer in this context).
\item \texttt{$\vdash$} :
  Turnstile (read: ``the following symbol sequence is provable'' or
  ``a proof exists for'').
\item \texttt{setvar} :
  Individual set variable type (read: ``the following is an
  individual set variable'').
  Note that this is \textit{not} the type of an arbitrary set expression,
  instead, it is used to ensure that there is only a single symbol used
  after quantifiers like for-all ($\forall$) and there-exists ($\exists$).
\item \texttt{class} :
  An expression that is a syntactically valid class expression.
  All valid set expressions are also valid class expression, so expressions
  of sets normally have the \texttt{class} typecode.
  Use the \texttt{class} typecode,
  \textit{not} the \texttt{setvar} typecode,
  for the type of set expressions unless you are specifically identifying
  a single set variable.
\end{itemize}

\subsection{Assertions (\texttt{\$a} and \texttt{\$p} Statements)}
\index{\texttt{\$a} statement}
\index{\texttt{\$p} statement}\index{assertion}\index{axiomatic assertion}
\index{provable assertion}

There are two types of assertions, \texttt{\$a}\index{\texttt{\$a} statement}
statements ({\bf axiomatic assertions}) and \texttt{\$p} statements ({\bf
provable assertions}).  Their syntax is as follows:
\begin{center}
  {\em label} \texttt{\$a} {\em typecode} {\em math-symbol} \ldots
         {\em math-symbol} \texttt{\$.}\\
  {\em label} \texttt{\$p} {\em typecode} {\em math-symbol} \ldots
        {\em math-symbol} \texttt{\$=} {\em proof} \texttt{\$.}
\end{center}
\index{\texttt{\$a} statement}
\index{\texttt{\$p} statement}
\index{\texttt{\$=} keyword}
An assertion always requires a {\em label}\index{label}. The expression in an
assertion consists of a typecode (an active constant)
followed by a sequence of zero
or more math symbols.  Each math symbol, including any {\em constant}, must be a
previously declared constant or variable.  (In addition, each math symbol
must be active, which will be covered when we discuss scoping statements in
Section~\ref{scoping}.)

A \texttt{\$a} statement is usually a definition of syntax (for example, if $P$
and $Q$ are wffs then so is $(P\to Q)$), an axiom\index{axiom} of ordinary
mathematics (for example, $x=x$), or a definition\index{definition} of
ordinary mathematics (for example, $x\ne y$ means $\lnot x=y$). A \texttt{\$p}
statement is a claim that a certain combination of math symbols follows from
previous assertions and is accompanied by a proof that demonstrates it.

Assertions can also be referenced in (later) proofs in order to derive new
assertions from them. The label of an assertion is used to refer to it in a
proof. Section~\ref{proof} will describe the proof in detail.

Assertions also provide the primary means for communicating the mathematical
results in the database to people.  Proofs (when conveniently displayed)
communicate to people how the results were arrived at.

\subsubsection{The \texttt{\$a} Statement}
\index{\texttt{\$a} statement}

Axiomatic assertions (\texttt{\$a} statements) represent the starting points from
which other assertions (\texttt{\$p}\index{\texttt{\$p} statement} statements) are
derived.  Their most obvious use is for specifying ordinary mathematical
axioms\index{axiom}, but they are also used for two other purposes.

First, Metamath\index{Metamath} needs to know the syntax of symbol
sequences that constitute valid mathematical statements.  A Metamath
proof must be broken down into much more detail than ordinary
mathematical proofs that you may be used to thinking of (even the
``complete'' proofs of formal logic\index{formal logic}).  This is one
of the things that makes Metamath a general-purpose language,
independent of any system of logic or even syntax.  If you want to use a
substitution instance of an assertion as a step in a proof, you must
first prove that the substitution is syntactically correct (or if you
prefer, you must ``construct'' it), showing for example that the
expression you are substituting for a wff metavariable is a valid wff.
The \texttt{\$a}\index{\texttt{\$a} statement} statement is used to
specify those combinations of symbols that are considered syntactically
valid, such as the legal forms of wffs.

Second, \texttt{\$a} statements are used to specify what are ordinarily thought of
as definitions, i.e.\ new combinations of symbols that abbreviate other
combinations of symbols.  Metamath makes no distinction\index{axiom vs.\
definition} between axioms\index{axiom} and definitions\index{definition}.
Indeed, it has been argued that such distinction should not be made even in
ordinary mathematics; see Section~\ref{definitions}, which discusses the
philosophy of definitions.  Section~\ref{hierarchy} discusses some
technical requirements for definitions.  In \texttt{set.mm} we adopt the
convention of prefixing axiom labels with \texttt{ax-} and definition labels with
\texttt{df-}\index{label}.

The results that can be derived with the Metamath language are only as good as
the \texttt{\$a}\index{\texttt{\$a} statement} statements used as their starting
point.  We cannot stress this too strongly.  For example, Metamath will
not prevent you from specifying $x\neq x$ as an axiom of logic.  It is
essential that you scrutinize all \texttt{\$a} statements with great care.
Because they are a source of potential pitfalls, it is best not to add new
ones (usually new definitions) casually; rather you should carefully evaluate
each one's necessity and advantages.

Once you have in place all of the basic axioms\index{axiom} and
rules\index{rule} of a mathematical theory, the only \texttt{\$a} statements that
you will be adding will be what are ordinarily called definitions.  In
principle, definitions should be in some sense eliminable from the language of
a theory according to some convention (usually involving logical equivalence
or equality).  The most common convention is that any formula that was
syntactically valid but not provable before the definition was introduced will
not become provable after the definition is introduced.  In an ideal world,
definitions should not be present at all if one is to have absolute confidence
in a mathematical result.  However, they are necessary to make
mathematics practical, for otherwise the resulting formulas would be
extremely long and incomprehensible.  Since the nature of definitions (in the
most general sense) does not permit them to automatically be verified as
``proper,''\index{proper definition}\index{definition!proper} the judgment of
the mathematician is required to ensure it.  (In \texttt{set.mm} effort was made
to make almost all definitions directly eliminable and thus minimize the need
for such judgment.)

If you are not a mathematician, it may be best not to add or change any
\texttt{\$a}\index{\texttt{\$a} statement} statements but instead use
the mathematical language already provided in standard databases.  This
way Metamath will not allow you to make a mistake (i.e.\ prove a false
result).


\subsection{Frames}\label{frames}

We now introduce the concept of a collection of related Metamath statements
called a frame.  Every assertion (\texttt{\$a} or \texttt{\$p} statement) in the database has
an associated frame.

A {\bf frame}\index{frame} is a sequence of \texttt{\$d}, \texttt{\$f},
and \texttt{\$e} statements (zero or more of each) followed by one
\texttt{\$a} or \texttt{\$p} statement, subject to certain conditions we
will describe.  For simplicity we will assume that all math symbol
tokens used are declared at the beginning of the database with
\texttt{\$c} and \texttt{\$v} statements (which are not properly part of
a frame).  Also for simplicity we will assume there are only simple
\texttt{\$d} statements (those with only two variables) and imagine any
compound \texttt{\$d} statements (those with more than two variables) as
broken up into simple ones.

A frame groups together those hypotheses (and \texttt{\$d} statements) relevant
to an assertion (\texttt{\$a} or \texttt{\$p} statement).  The statements in a frame
may or may not be physically adjacent in a database; we will cover
this in our discussion of scoping statements
in Section~\ref{scoping}.

A frame has the following properties:
\begin{enumerate}
 \item The set of variables contained in its \texttt{\$f} statements must
be identical to the set of variables contained in its \texttt{\$e},
\texttt{\$a}, and/or \texttt{\$p} statements.  In other words, each
variable in a \texttt{\$e}, \texttt{\$a}, or \texttt{\$p} statement must
have an associated ``variable type'' defined for it in a \texttt{\$f}
statement.
  \item No two \texttt{\$f} statements may contain the same variable.
  \item Any \texttt{\$f} statement
must occur before a \texttt{\$e} statement in which its variable occurs.
\end{enumerate}

The first property determines the set of variables occurring in a frame.
These are the {\bf mandatory
variables}\index{mandatory variable} of the frame.  The second property
tells us there must be only one type specified for a variable.
The last property is not a theoretical requirement but it
makes parsing of the database easier.

For our examples, we assume our database has the following declarations:

\begin{verbatim}
$v P Q R $.
$c -> ( ) |- wff $.
\end{verbatim}

The following sequence of statements, describing the modus ponens inference
rule, is an example of a frame:

\begin{verbatim}
wp  $f wff P $.
wq  $f wff Q $.
maj $e |- ( P -> Q ) $.
min $e |- P $.
mp  $a |- Q $.
\end{verbatim}

The following sequence of statements is not a frame because \texttt{R} does not
occur in the \texttt{\$e}'s or the \texttt{\$a}:

\begin{verbatim}
wp  $f wff P $.
wq  $f wff Q $.
wr  $f wff R $.
maj $e |- ( P -> Q ) $.
min $e |- P $.
mp  $a |- Q $.
\end{verbatim}

The following sequence of statements is not a frame because \texttt{Q} does not
occur in a \texttt{\$f}:

\begin{verbatim}
wp  $f wff P $.
maj $e |- ( P -> Q ) $.
min $e |- P $.
mp  $a |- Q $.
\end{verbatim}

The following sequence of statements is not a frame because the \texttt{\$a} statement is
not the last one:

\begin{verbatim}
wp  $f wff P $.
wq  $f wff Q $.
maj $e |- ( P -> Q ) $.
mp  $a |- Q $.
min $e |- P $.
\end{verbatim}

Associated with a frame is a sequence of {\bf mandatory
hypotheses}\index{mandatory hypothesis}.  This is simply the set of all
\texttt{\$f} and \texttt{\$e} statements in the frame, in the order they
appear.  A frame can be referenced in a later proof using the label of
the \texttt{\$a} or \texttt{\$p} assertion statement, and the proof
makes an assignment to each mandatory hypothesis in the order in which
it appears.  This means the order of the hypotheses, once chosen, must
not be changed so as not to affect later proofs referencing the frame's
assertion statement.  (The Metamath proof verifier will, of course, flag
an error if a proof becomes incorrect by doing this.)  Since proofs make
use of ``Reverse Polish notation,'' described in Section~\ref{proof}, we
call this order the {\bf RPN order}\index{RPN order} of the hypotheses.

Note that \texttt{\$d} statements are not part of the set of mandatory
hypotheses, and their order doesn't matter (as long as they satisfy the
fourth property for a frame described above).  The \texttt{\$d}
statements specify restrictions on variables that must be satisfied (and
are checked by the proof verifier) when expressions are substituted for
them in a proof, and the \texttt{\$d} statements themselves are never
referenced directly in a proof.

A frame with a \texttt{\$p} (provable) statement requires a proof as part of the
\texttt{\$p} statement.  Sometimes in a proof we want to make use of temporary or
dummy variables\index{dummy variable} that do not occur in the \texttt{\$p}
statement or its mandatory hypotheses.  To accommodate this we define an {\bf
extended frame}\index{extended frame} as a frame together with zero or more
\texttt{\$d} and \texttt{\$f} statements that reference variables not among the
mandatory variables of the frame.  Any new variables referenced are called the
{\bf optional variables}\index{optional variable} of the extended frame. If a
\texttt{\$f} statement references an optional variable it is called an {\bf
optional hypothesis}\index{optional hypothesis}, and if one or both of the
variables in a \texttt{\$d} statement are optional variables it is called an {\bf
optional disjoint-variable restriction}\index{optional disjoint-variable
restriction}.  Properties 2 and 3 for a frame also apply to an extended
frame.

The concept of optional variables is not meaningful for frames with \texttt{\$a}
statements, since those statements have no proofs that might make use of them.
There is no restriction on including optional hypotheses in the extended frame
for a \texttt{\$a} statement, but they serve no purpose.

The following set of statements is an example of an extended frame, which
contains an optional variable \texttt{R} and an optional hypothesis \texttt{wr}.  In
this example, we suppose the rule of modus ponens is not an axiom but is
derived as a theorem from earlier statements (we omit its presumed proof).
Variable \texttt{R} may be used in its proof if desired (although this would
probably have no advantage in propositional calculus).  Note that the sequence
of mandatory hypotheses in RPN order is still \texttt{wp}, \texttt{wq}, \texttt{maj},
\texttt{min} (i.e.\ \texttt{wr} is omitted), and this sequence is still assumed
whenever the assertion \texttt{mp} is referenced in a subsequent proof.

\begin{verbatim}
wp  $f wff P $.
wq  $f wff Q $.
wr  $f wff R $.
maj $e |- ( P -> Q ) $.
min $e |- P $.
mp  $p |- Q $= ... $.
\end{verbatim}

Every frame is an extended frame, but not every extended frame is a frame, as
this example shows.  The underlying frame for an extended frame is
obtained by simply removing all statements containing optional variables.
Any proof referencing an assertion will ignore any extensions to its
frame, which means we may add or delete optional hypotheses at will without
affecting subsequent proofs.

The conceptually simplest way of organizing a Metamath database is as a
sequence of extended frames.  The scoping statements
\texttt{\$\char`\{}\index{\texttt{\$\char`\{} and \texttt{\$\char`\}}
keywords} and \texttt{\$\char`\}} can be used to delimit the start and
end of an extended frame, leading to the following possible structure for a
database.  \label{framelist}

\vskip 2ex
\setbox\startprefix=\hbox{\tt \ \ \ \ \ \ \ \ }
\setbox\contprefix=\hbox{}
\startm
\m{\mbox{(\texttt{\$v} {\em and} \texttt{\$c}\,{\em statements})}}
\endm
\startm
\m{\mbox{\texttt{\$\char`\{}}}
\endm
\startm
\m{\mbox{\texttt{\ \ } {\em extended frame}}}
\endm
\startm
\m{\mbox{\texttt{\$\char`\}}}}
\endm
\startm
\m{\mbox{\texttt{\$\char`\{}}}
\endm
\startm
\m{\mbox{\texttt{\ \ } {\em extended frame}}}
\endm
\startm
\m{\mbox{\texttt{\$\char`\}}}}
\endm
\startm
\m{\mbox{\texttt{\ \ \ \ \ \ \ \ \ }}\vdots}
\endm
\vskip 2ex

In practice, this structure is inconvenient because we have to repeat
any \texttt{\$f}, \texttt{\$e}, and \texttt{\$d} statements over and
over again rather than stating them once for use by several assertions.
The scoping statements, which we will discuss next, allow this to be
done.  In principle, any Metamath database can be converted to the above
format, and the above format is the most convenient to use when studying
a Metamath database as a formal system%
%% Uncomment this when uncommenting section {formalspec} below
   (Appendix \ref{formalspec})%
.
In fact, Metamath internally converts the database to the above format.
The command \texttt{show statement} in the Metamath program will show
you the contents of the frame for any \texttt{\$a} or \texttt{\$p}
statement, as well as its extension in the case of a \texttt{\$p}
statement.

%c%(provided that all ``local'' variables and constants with limited scope have
%c%unique names),

During our discussion of scoping statements, it may be helpful to
think in terms of the equivalent sequence of frames that will result when
the database is parsed.  Scoping (other than the limited
use above to delimit frames) is not a theoretical requirement for
Metamath but makes it more convenient.


\subsection{Scoping Statements (\texttt{\$\{} and \texttt{\$\}})}\label{scoping}
\index{\texttt{\$\char`\{} and \texttt{\$\char`\}} keywords}\index{scoping statement}

%c%Some Metamath statements may be needed only temporarily to
%c%serve a specific purpose, and after we're done with them we would like to
%c%disregard or ignore them.  For example, when we're finished using a variable,
%c%we might want to
%c%we might want to free up the token\index{token} used to name it so that the
%c%token can be used for other purposes later on, such as a different kind of
%c%variable or even a constant.  In the terminology of computer programming, we
%c%might want to let some symbol declarations be ``local'' rather than ``global.''
%c%\index{local symbol}\index{global symbol}

The {\bf scoping} statements, \texttt{\$\char`\{} ({\bf start of block}) and \texttt{\$\char`\}}
({\bf end of block})\index{block}, provide a means for controlling the portion
of a database over which certain statement types are recognized.  The
syntax of a scoping statement is very simple; it just consists of the
statement's keyword:
\begin{center}
\texttt{\$\char`\{}\\
\texttt{\$\char`\}}
\end{center}
\index{\texttt{\$\char`\{} and \texttt{\$\char`\}} keywords}

For example, consider the following database where we have stripped out
all tokens except the scoping statement keywords.  For the purpose of the
discussion, we have added subscripts to the scoping statements; these subscripts
do not appear in the actual database.
\[
 \mbox{\tt \ \$\char`\{}_1
 \mbox{\tt \ \$\char`\{}_2
 \mbox{\tt \ \$\char`\}}_2
 \mbox{\tt \ \$\char`\{}_3
 \mbox{\tt \ \$\char`\{}_4
 \mbox{\tt \ \$\char`\}}_4
 \mbox{\tt \ \$\char`\}}_3
 \mbox{\tt \ \$\char`\}}_1
\]
Each \texttt{\$\char`\{} statement in this example is said to be {\bf
matched} with the \texttt{\$\char`\}} statement that has the same
subscript.  Each pair of matched scoping statements defines a region of
the database called a {\bf block}.\index{block} Blocks can be {\bf
nested}\index{nested block} inside other blocks; in the example, the
block defined by $\mbox{\tt \$\char`\{}_4$ and $\mbox{\tt \$\char`\}}_4$
is nested inside the block defined by $\mbox{\tt \$\char`\{}_3$ and
$\mbox{\tt \$\char`\}}_3$ as well as inside the block defined by
$\mbox{\tt \$\char`\{}_1$ and $\mbox{\tt \$\char`\}}_1$.  In general, a
block may be empty, it may contain only non-scoping
statements,\footnote{Those statements other than \texttt{\$\char`\{} and
\texttt{\$\char`\}}.}\index{non-scoping statement} or it may contain any
mixture of other blocks and non-scoping statements.  (This is called a
``recursive'' definition\index{recursive definition} of a block.)

Associated with each block is a number called its {\bf nesting
level}\index{nesting level} that indicates how deeply the block is nested.
The nesting levels of the blocks in our example are as follows:
\[
  \underbrace{
    \mbox{\tt \ }
    \underbrace{
     \mbox{\tt \$\char`\{\ }
     \underbrace{
       \mbox{\tt \$\char`\{\ }
       \mbox{\tt \$\char`\}}
     }_{2}
     \mbox{\tt \ }
     \underbrace{
       \mbox{\tt \$\char`\{\ }
       \underbrace{
         \mbox{\tt \$\char`\{\ }
         \mbox{\tt \$\char`\}}
       }_{3}
       \mbox{\tt \ \$\char`\}}
     }_{2}
     \mbox{\tt \ \$\char`\}}
   }_{1}
   \mbox{\tt \ }
 }_{0}
\]
\index{\texttt{\$\char`\{} and \texttt{\$\char`\}} keywords}
The entire database is considered to be one big block (the {\bf outermost}
block) with a nesting level of 0.  The outermost block is {\em not} bracketed
by scoping statements.\footnote{The language was designed this way so that
several source files can be joined together more easily.}\index{outermost
block}

All non-scoping Metamath statements become recognized or {\bf
active}\index{active statement} at the place where they appear.\footnote{To
keep things slightly simpler, we do not bother to define the concept of
``active'' for the scoping statements.}  Certain of these statement types
become inactive at the end of the block in which they appear; these statement
types are:
\begin{center}
  \texttt{\$c}, \texttt{\$v}, \texttt{\$d}, \texttt{\$e}, and \texttt{\$f}.
%  \texttt{\$v}, \texttt{\$f}, \texttt{\$e}, and \texttt{\$d}.
\end{center}
\index{\texttt{\$c} statement}
\index{\texttt{\$d} statement}
\index{\texttt{\$e} statement}
\index{\texttt{\$f} statement}
\index{\texttt{\$v} statement}
The other statement types remain active forever (i.e.\ through the end of the
database); they are:
\begin{center}
  \texttt{\$a} and \texttt{\$p}.
%  \texttt{\$c}, \texttt{\$a}, and \texttt{\$p}.
\end{center}
\index{\texttt{\$a} statement}
\index{\texttt{\$p} statement}
Any statement (of these 7 types) located in the outermost
block\index{outermost block} will remain active through the end of the
database and thus are effectively ``global'' statements.\index{global
statement}

All \texttt{\$c} statements must be placed in the outermost block.  Since they are
therefore always global, they could be considered as belonging to both of the
above categories.

The {\bf scope}\index{scope} of a statement is the set of statements that
recognize it as active.

%c%The concept of ``active'' is also defined for math symbols\index{math
%c%symbol}.  Math symbols (constants\index{constant} and
%c%variables\index{variable}) become {\bf active}\index{active
%c%math symbol} in the \texttt{\$c}\index{\texttt{\$c}
%c%statement} and \texttt{\$v}\index{\texttt{\$v} statement} statements that
%c%declare them.  They become inactive when their declaration statements become
%c%inactive.

The concept of ``active'' is also defined for math symbols\index{math
symbol}.  Math symbols (constants\index{constant} and
variables\index{variable}) become {\bf active}\index{active math symbol}
in the \texttt{\$c}\index{\texttt{\$c} statement} and
\texttt{\$v}\index{\texttt{\$v} statement} statements that declare them.
A variable becomes inactive when its declaration statement becomes
inactive.  Because all \texttt{\$c} statements must be in the outermost
block, a constant will never become inactive after it is declared.

\subsubsection{Redeclaration of Math Symbols}
\index{redeclaration of symbols}\label{redeclaration}

%c%A math symbol may not be declared a second time while it is active, but it may
%c%be declared again after it becomes inactive.

A variable may not be declared a second time while it is active, but it may be
declared again after it becomes inactive.  This provides a convenient way to
introduce ``local'' variables,\index{local variable} i.e.\ temporary variables
for use in the frame of an assertion or in a proof without keeping them around
forever.  A previously declared variable may not be redeclared as a constant.

A constant may not be redeclared.  And, as mentioned above, constants must be
declared in the outermost block.

The reason variables may have limited scope but not constants is that an
assertion (\texttt{\$a} or \texttt{\$p} statement) remains available for use in
proofs through the end of the database.  Variables in an assertion's frame may
be substituted with whatever is needed in a proof step that references the
assertion, whereas constants remain fixed and may not be substituted with
anything.  The particular token used for a variable in an assertion's frame is
irrelevant when the assertion is referenced in a proof, and it doesn't matter
if that token is not available outside of the referenced assertion's frame.
Constants, however, must be globally fixed.

There is no theoretical
benefit for the feature allowing variables to be active for limited scopes
rather than global. It is just a convenience that allows them, for example, to
be locally grouped together with their corresponding \texttt{\$f} variable-type
declarations.

%c%If you declare a math symbol more than once, internally Metamath considers it a
%c%new distinct symbol, even though it has the same name.  If you are unaware of
%c%this, you may find that what you think are correct proofs are incorrectly
%c%rejected as invalid, because Metamath may tell you that a constant you
%c%previously declared does not match a newly declared math symbol with the same
%c%name.  For details on this subtle point, see the Comment on
%c%p.~\pageref{spec4comment}.  This is done purposely to allow temporary
%c%constants to be introduced while developing a subtheory, then allow their math
%c%symbol tokens to be reused later on; in general they will not refer to the
%c%same thing.  In practice, you would not ordinarily reuse the names of
%c%constants because it would tend to be confusing to the reader.  The reuse of
%c%names of variables, on the other hand, is something that is often useful to do
%c%(for example it is done frequently in \texttt{set.mm}).  Since variables in an
%c%assertion referenced in a proof can be substituted as needed to achieve a
%c%symbol match, this is not an issue.

% (This section covers a somewhat advanced topic you may want to skip
% at first reading.)
%
% Under certain circumstances, math symbol\index{math symbol}
% tokens\index{token} may be redeclared (i.e.\ the token
% may appear in more than
% one \texttt{\$c}\index{\texttt{\$c} statement} or \texttt{\$v}\index{\texttt{\$v}
% statement} statement).  You might want to do this say, to make temporary use
% of a variable name without having to worry about its affect elsewhere,
% somewhat analogous to declaring a local variable in a standard computer
% language.  Understanding what goes on when math symbol tokens are redeclared
% is a little tricky to understand at first, since it requires that we
% distinguish the token itself from the math symbol that it names.  It will help
% if we first take a peek at the internal workings of the
% Metamath\index{Metamath} program.
%
% Metamath reserves a memory location for each occurrence of a
% token\index{token} in a declaration statement (\texttt{\$c}\index{\texttt{\$c}
% statement} or \texttt{\$v}\index{\texttt{\$v} statement}).  If a given token appears
% in more than one declaration statement, it will refer to more than one memory
% locations.  A math symbol\index{math symbol} may be thought of as being one of
% these memory locations rather than as the token itself.  Only one of the
% memory locations associated with a given token may be active at any one time.
% The math symbol (memory location) that gets looked up when the token appears
% in a non-declaration statement is the one that happens to be active at that
% time.
%
% We now look at the rules for the redeclaration\index{redeclaration of symbols}
% of math symbol tokens.
% \begin{itemize}
% \item A math symbol token may not be declared twice in the
% same block.\footnote{While there is no theoretical reason for disallowing
% this, it was decided in the design of Metamath that allowing it would offer no
% advantage and might cause confusion.}
% \item An inactive math symbol may always be
% redeclared.
% \item  An active math symbol may be redeclared in a different (i.e.\
% inner) block\index{block} from the one it became active in.
% \end{itemize}
%
% When a math symbol token is redeclared, it conceptually refers to a different
% math symbol, just as it would be if it were called a different name.  In
% addition, the original math symbol that it referred to, if it was active,
% temporarily becomes inactive.  At the end of the block in which the
% redeclaration occurred, the new math symbol\index{math symbol} becomes
% inactive and the original symbol becomes active again.  This concept is
% illustrated in the following example, where the symbol \texttt{e} is
% ordinarily a constant (say Euler's constant, 2.71828...) but
% temporarily we want to use it as a ``local'' variable, say as a coefficient
% in the equation $a x^4 + b x^3 + c x^2 + d x + e$:
% \[
%   \mbox{\tt \$\char`\{\ \$c e \$.}
%   \underbrace{
%     \ \ldots\ %
%     \mbox{\tt \$\char`\{}\ \ldots\ %
%   }_{\mbox{\rm region A}}
%   \mbox{\tt \$v e \$.}
%   \underbrace{
%     \mbox{\ \ \ \ldots\ \ \ }
%   }_{\mbox{\rm region B}}
%   \mbox{\tt \$\char`\}}
%   \underbrace{
%     \mbox{\ \ \ \ldots\ \ \ }
%   }_{\mbox{\rm region C}}
%   \mbox{\tt \$\char`\}}
% \]
% \index{\texttt{\$\char`\{} and \texttt{\$\char`\}} keywords}
% In region A, the token \texttt{e} refers to a constant.  It is redeclared as a
% variable in region B, and any reference to it in this region will refer to this
% variable.  In region C, the redeclaration becomes inactive, and the original
% declaration becomes active again.  In region C, the token \texttt{x} refers to the
% original constant.
%
% As a practical matter, overuse of math symbol\index{math symbol}
% redeclarations\index{redeclaration of symbols} can be confusing (even though
% it is well-defined) and is best avoided when possible.  Here are some good
% general guidelines you can follow.  Usually, you should declare all
% constants\index{constant} in the outermost block\index{outermost block},
% especially if they are general-purpose (such as the token \verb$A.$, meaning
% $\forall$ or ``for all'').  This will make them ``globally'' active (although
% as in the example above local redeclarations will temporarily make them
% inactive.)  Most or all variables\index{variable}, on the other hand, could be
% declared in inner blocks, so that the token for them can be used later for a
% different type of variable or a constant.  (The names of the variables you
% choose are not used when you refer to an assertion\index{assertion} in a
% proof, whereas constants must match exactly.  A locally declared constant will
% not match a globally declared constant in a proof, even if they use the same
% token, because Metamath internally considers them to be different math
% symbols.)  To avoid confusion, you should generally avoid redeclaring active
% variables.  If you must redeclare them, do so at the beginning of a block.
% The temporary declaration of constants in inner blocks might be occasionally
% appropriate when you make use of a temporary definition to prove lemmas
% leading to a main result that does not make direct use of the definition.
% This way, you will not clutter up your database with a large number of
% seldom-used global constant symbols.  You might want to note that while
% inactive constants may not appear directly in an assertion (a \texttt{\$a}\index{\texttt{\$a}
% statement} or \texttt{\$p}\index{\texttt{\$p} statement}
% statement), they may be indirectly used in the proof of a \texttt{\$p} statement
% so long as they do not appear in the final math symbol sequence constructed by
% the proof.  In the end, you will have to use your best judgment, taking into
% account standard mathematical usage of the symbols as well as consideration
% for the reader of your work.
%
% \subsubsection{Reuse of Labels}\index{reuse of labels}\index{label}
%
% The \texttt{\$e}\index{\texttt{\$e} statement}, \texttt{\$f}\index{\texttt{\$f}
% statement}, \texttt{\$a}\index{\texttt{\$a} statement}, and
% \texttt{\$p}\index{\texttt{\$p}
% statement} statement types require labels, which allow them to be
% referenced later inside proofs.  A label is considered {\bf
% active}\index{active label} when the statement it is associated with is
% active.  The token\index{token} for a label may be reused
% (redeclared)\index{redeclaration of labels} provided that it is not being used
% for a currently active label.  (Unlike the tokens for math symbols, active
% label tokens may not be redeclared in an inner scope.)  Note that the labels
% of \texttt{\$a} and \texttt{\$p} statements can never be reused after these
% statements appear, because these statements remain active through the end of
% the database.
%
% You might find the reuse of labels a convenient way to have standard names for
% temporary hypotheses, such as \texttt{h1}, \texttt{h2}, etc.  This way you don't have
% to invent unique names for each of them, and in some cases it may be less
% confusing to the reader (although in other cases it might be more confusing, if
% the hypothesis is located far away from the assertion that uses
% it).\footnote{The current implementation requires that all labels, even
% inactive ones, be unique.}

\subsubsection{Frames Revisited}\index{frames and scoping statements}

Now that we have covered scoping, we will look at how an arbitrary
Metamath database can be converted to the simple sequence of extended
frames described on p.~\pageref{framelist}.  This is also how Metamath
stores the database internally when it reads in the database
source.\label{frameconvert} The method is simple.  First, we collect all
constant and variable (\texttt{\$c} and \texttt{\$v}) declarations in
the database, ignoring duplicate declarations of the same variable in
different scopes.  We then put our collected \texttt{\$c} and
\texttt{\$v} declarations at the beginning of the database, so that
their scope is the entire database.  Next, for each assertion in the
database, we determine its frame and extended frame.  The extended frame
is simply the \texttt{\$f}, \texttt{\$e}, and \texttt{\$d} statements
that are active.  The frame is the extended frame with all optional
hypotheses removed.

An equivalent way of saying this is that the extended frame of an assertion
is the collection of all \texttt{\$f}, \texttt{\$e}, and \texttt{\$d} statements
whose scope includes the assertion.
The \texttt{\$f} and \texttt{\$e} statements
occur in the order they appear
(order is irrelevant for \texttt{\$d} statements).

%c%, renaming any
%c%redeclared variables as needed so that all of them have unique names.  (The
%c%exact renaming convention is unimportant.  You might imagine renaming
%c%different declarations of math symbol \texttt{a} as \texttt{a\$1}, \texttt{a\$2}, etc.\
%c%which would prevent any conflicts since \texttt{\$} is not a legal character in a
%c%math symbol token.)

\section{The Anatomy of a Proof} \label{proof}
\index{proof!Metamath, description of}

Each provable assertion (\texttt{\$p}\index{\texttt{\$p} statement} statement) in a
database must include a {\bf proof}\index{proof}.  The proof is located
between the \texttt{\$=}\index{\texttt{\$=} keyword} and \texttt{\$.}\ keywords in the
\texttt{\$p} statement.

In the basic Metamath language\index{basic language}, a proof is a
sequence of statement labels.  This label sequence\index{label sequence}
serves as a set of instructions that the Metamath program uses to
construct a series of math symbol sequences.  The construction must
ultimately result in the math symbol sequence contained between the
\texttt{\$p}\index{\texttt{\$p} statement} and
\texttt{\$=}\index{\texttt{\$=} keyword} keywords of the \texttt{\$p}
statement.  Otherwise, the Metamath program will consider the proof
incorrect, and it will notify you with an appropriate error message when
you ask it to verify the proof.\footnote{To make the loading faster, the
Metamath program does not automatically verify proofs when you
\texttt{read} in a database unless you use the \texttt{/verify}
qualifier.  After a database has been read in, you may use the
\texttt{verify proof *} command to verify proofs.}\index{\texttt{verify
proof} command} Each label in a proof is said to {\bf
reference}\index{label reference} its corresponding statement.

Associated with any assertion\index{assertion} (\texttt{\$p} or
\texttt{\$a}\index{\texttt{\$a} statement} statement) is a set of
hypotheses (\texttt{\$f}\index{\texttt{\$f} statement} or
\texttt{\$e}\index{\texttt{\$e} statement} statements) that are active
with respect to that assertion.  Some are mandatory and the others are
optional.  You should review these concepts if necessary.

Each label\index{label} in a proof must be either the label of a
previous assertion (\texttt{\$a}\index{\texttt{\$a} statement} or
\texttt{\$p}\index{\texttt{\$p} statement} statement) or the label of an
active hypothesis (\texttt{\$e} or \texttt{\$f}\index{\texttt{\$f}
statement} statement) of the \texttt{\$p} statement containing the
proof.  Hypothesis labels may reference both the
mandatory\index{mandatory hypothesis} and the optional hypotheses of the
\texttt{\$p} statement.

The label sequence in a proof specifies a construction in {\bf reverse Polish
notation}\index{reverse Polish notation (RPN)} (RPN).  You may be familiar
with RPN if you have used older
Hewlett--Packard or similar hand-held calculators.
In the calculator analogy, a hypothesis label\index{hypothesis label} is like
a number and an assertion label\index{assertion label} is like an operation
(more precisely, an $n$-ary operation when the
assertion has $n$ \texttt{\$e}-hypotheses).
On an RPN calculator, an operation takes one or more previous numbers in an
input sequence, performs a calculation on them, and replaces those numbers and
itself with the result of the calculation.  For example, the input sequence
$2,3,+$ on an RPN calculator results in $5$, and the input sequence
$2,3,5,{\times},+$ results in $2,15,+$ which results in $17$.

Understanding how RPN is processed involves the concept of a {\bf
stack}\index{stack}\index{RPN stack}, which can be thought of as a set of
temporary memory locations that hold intermediate results.  When Metamath
encounters a hypothesis label it places or {\bf pushes}\index{push} the math
symbol sequence of the hypothesis onto the stack.  When Metamath encounters an
assertion label, it associates the most recent stack entries with the {\em
mandatory} hypotheses\index{mandatory hypothesis} of the assertion, in the
order where the most recent stack entry is associated with the last mandatory
hypothesis of the assertion.  It then determines what
substitutions\index{substitution!variable}\index{variable substitution} have
to be made into the variables of the assertion's mandatory hypotheses to make
them identical to the associated stack entries.  It then makes those same
substitutions into the assertion itself.  Finally, Metamath removes or {\bf
pops}\index{pop} the matched hypotheses from the stack and pushes the
substituted assertion onto the stack.

For the purpose of matching the mandatory hypothesis to the most recent stack
entries, whether a hypothesis is a \texttt{\$e} or \texttt{\$f} statement is
irrelevant.  The only important thing is that a set of
substitutions\footnote{In the Metamath spec (Section~\ref{spec}), we use the
singular term ``substitution'' to refer to the set of substitutions we talk
about here.} exist that allow a match (and if they don't, the proof verifier
will let you know with an error message).  The Metamath language is specified
in such a way that if a set of substitutions exists, it will be unique.
Specifically, the requirement that each variable have a type specified for it
with a \texttt{\$f} statement ensures the uniqueness.

We will illustrate this with an example.
Consider the following Metamath source file:
\begin{verbatim}
$c ( ) -> wff $.
$v p q r s $.
wp $f wff p $.
wq $f wff q $.
wr $f wff r $.
ws $f wff s $.
w2 $a wff ( p -> q ) $.
wnew $p wff ( s -> ( r -> p ) ) $= ws wr wp w2 w2 $.
\end{verbatim}
This Metamath source example shows the definition and ``proof'' (i.e.,
construction) of a well-formed formula (wff)\index{well-formed formula (wff)}
in propositional calculus.  (You may wish to type this example into a file to
experiment with the Metamath program.)  The first two statements declare
(introduce the names of) four constants and four variables.  The next four
statements specify the variable types, namely that
each variable is assumed to be a wff.  Statement \texttt{w2} defines (postulates)
a way to produce a new wff, \texttt{( p -> q )}, from two given wffs \texttt{p} and
\texttt{q}. The mandatory hypotheses of \texttt{w2} are \texttt{wp} and \texttt{wq}.
Statement \texttt{wnew} claims that \texttt{( s -> ( r -> p ) )} is a wff given
three wffs \texttt{s}, \texttt{r}, and \texttt{p}.  More precisely, \texttt{wnew} claims
that the sequence of ten symbols \texttt{wff ( s -> ( r -> p ) )} is provable from
previous assertions and the hypotheses of \texttt{wnew}.  Metamath does not know
or care what a wff is, and as far as it is concerned
the typecode \texttt{wff} is just an
arbitrary constant symbol in a math symbol sequence.  The mandatory hypotheses
of \texttt{wnew} are \texttt{wp}, \texttt{wr}, and \texttt{ws}; \texttt{wq} is an optional
hypothesis.  In our particular proof, the optional hypothesis is not
referenced, but in general, any combination of active (i.e.\ optional and
mandatory) hypotheses could be referenced.  The proof of statement \texttt{wnew}
is the sequence of five labels starting with \texttt{ws} (step~1) and ending with
\texttt{w2} (step~5).

When Metamath verifies the proof, it scans the proof from left to right.  We
will examine what happens at each step of the proof.  The stack starts off
empty.  At step 1, Metamath looks up label \texttt{ws} and determines that it is a
hypothesis, so it pushes the symbol sequence of statement \texttt{ws} onto the
stack:

\begin{center}\begin{tabular}{|l|l|}\hline
{Stack location} & {Contents} \\ \hline \hline
1 & \texttt{wff s} \\ \hline
\end{tabular}\end{center}

Metamath sees that the labels \texttt{wr} and \texttt{wp} in steps~2 and 3 are also
hypotheses, so it pushes them onto the stack.  After step~3, the stack looks
like
this:

\begin{center}\begin{tabular}{|l|l|}\hline
{Stack location} & {Contents} \\ \hline \hline
3 & \texttt{wff p} \\ \hline
2 & \texttt{wff r} \\ \hline
1 & \texttt{wff s} \\ \hline
\end{tabular}\end{center}

At step 4, Metamath sees that label \texttt{w2} is an assertion, so it must do
some processing.  First, it associates the mandatory hypotheses of \texttt{w2},
which are \texttt{wp} and \texttt{wq}, with stack locations~2 and 3, {\em in that
order}. Metamath determines that the only possible way
to make hypothesis \texttt{wp} match (become identical to) stack location~2 and
\texttt{wq} match stack location 3 is to substitute variable \texttt{p} with \texttt{r}
and \texttt{q} with \texttt{p}.  Metamath makes these substitutions into \texttt{w2} and
obtains the symbol sequence \texttt{wff ( r -> p )}.  It removes the hypotheses
from stack locations~2 and 3, then places the result into stack location~2:

\begin{center}\begin{tabular}{|l|l|}\hline
{Stack location} & {Contents} \\ \hline \hline
2 & \texttt{wff ( r -> p )} \\ \hline
1 & \texttt{wff s} \\ \hline
\end{tabular}\end{center}

At step 5, Metamath sees that label \texttt{w2} is an assertion, so it must again
do some processing.  First, it matches the mandatory hypotheses of \texttt{w2},
which are \texttt{wp} and \texttt{wq}, to stack locations 1 and 2.
Metamath determines that the only possible way to make the
hypotheses match is to substitute variable \texttt{p} with \texttt{s} and \texttt{q} with
\texttt{( r -> p )}.  Metamath makes these substitutions into \texttt{w2} and obtains
the symbol
sequence \texttt{wff ( s -> ( r -> p ) )}.  It removes stack
locations 1 and 2, then places the result into stack location~1:

\begin{center}\begin{tabular}{|l|l|}\hline
{Stack location} & {Contents} \\ \hline \hline
1 & \texttt{wff ( s -> ( r -> p ) )} \\ \hline
\end{tabular}\end{center}

After Metamath finishes processing the proof, it checks to see that the
stack contains exactly one element and that this element is
the same as the math symbol sequence in the
\texttt{\$p}\index{\texttt{\$p} statement} statement.  This is the case for our
proof of \texttt{wnew},
so we have proved \texttt{wnew} successfully.  If the result
differs, Metamath will notify you with an error message.  An error message
will also result if the stack contains more than one entry at the end of the
proof, or if the stack did not contain enough entries at any point in the
proof to match all of the mandatory hypotheses\index{mandatory hypothesis} of
an assertion.  Finally, Metamath will notify you with an error message if no
substitution is possible that will make a referenced assertion's hypothesis
match the
stack entries.  You may want to experiment with the different kinds of errors
that Metamath will detect by making some small changes in the proof of our
example.

Metamath's proof notation was designed primarily to express proofs in a
relatively compact manner, not for readability by humans.  Metamath can display
proofs in a number of different ways with the \texttt{show proof}\index{\texttt{show
proof} command} command.  The
\texttt{/lemmon} qualifier displays it in a format that is easier to read when the
proofs are short, and you saw examples of its use in Chapter~\ref{using}.  For
longer proofs, it is useful to see the tree structure of the proof.  A tree
structure is displayed when the \texttt{/lemmon} qualifier is omitted.  You will
probably find this display more convenient as you get used to it. The tree
display of the proof in our example looks like
this:\label{treeproof}\index{tree-style proof}\index{proof!tree-style}
\begin{verbatim}
1     wp=ws    $f wff s
2        wp=wr    $f wff r
3        wq=wp    $f wff p
4     wq=w2    $a wff ( r -> p )
5  wnew=w2  $a wff ( s -> ( r -> p ) )
\end{verbatim}
The number to the left of each line is the step number.  Following it is a
{\bf hypothesis association}\index{hypothesis association}, consisting of two
labels\index{label} separated by \texttt{=}.  To the left of the \texttt{=} (except
in the last step) is the label of a hypothesis of an assertion referenced
later in the proof; here, steps 1 and 4 are the hypothesis associations for
the assertion \texttt{w2} that is referenced in step 5.  A hypothesis association
is indented one level more than the assertion that uses it, so it is easy to
find the corresponding assertion by moving directly down until the indentation
level decreases to one less than where you started from.  To the right of each
\texttt{=} is the proof step label for that proof step.  The statement keyword of
the proof step label is listed next, followed by the content of the top of the
stack (the most recent stack entry) as it exists after that proof step is
processed.  With a little practice, you should have no trouble reading proofs
displayed in this format.

Metamath proofs include the syntax construction of a formula.
In standard mathematics, this kind of
construction is not considered a proper part of the proof at all, and it
certainly becomes rather boring after a while.
Therefore,
by default the \texttt{show proof}\index{\texttt{show proof}
command} command does not show the syntax construction.
Historically \texttt{show proof} command
\textit{did} show the syntax construction, and you needed to add the
\texttt{/essential} option to hide, them, but today
\texttt{/essential} is the default and you need to use
\texttt{/all} to see the syntax constructions.

When verifying a proof, Metamath will check that no mandatory
\texttt{\$d}\index{\texttt{\$d} statement}\index{mandatory \texttt{\$d}
statement} statement of an assertion referenced in a proof is violated
when substitutions\index{substitution!variable}\index{variable
substitution} are made to the variables in the assertion.  For details
see Section~\ref{spec4} or \ref{dollard}.

\subsection{The Concept of Unification} \label{unify}

During the course of verifying a proof, when Metamath\index{Metamath}
encounters an assertion label\index{assertion label}, it associates the
mandatory hypotheses\index{mandatory hypothesis} of the assertion with the top
entries of the RPN stack\index{stack}\index{RPN stack}.  Metamath then
determines what substitutions\index{substitution!variable}\index{variable
substitution} it must make to the variables in the assertion's mandatory
hypotheses in order for these hypotheses to become identical to their
corresponding stack entries.  This process is called {\bf
unification}\index{unification}.  (We also informally use the term
``unification'' to refer to a set of substitutions that results from the
process, as in ``two unifications are possible.'')  After the substitutions
are made, the hypotheses are said to be {\bf unified}.

If no such substitutions are possible, Metamath will consider the proof
incorrect and notify you with an error message.
% (deleted 3/10/07, per suggestion of Mel O'Cat:)
% The syntax of the
% Metamath language ensures that if a set of substitutions exists, it
% will be unique.

The general algorithm for unification described in the literature is
somewhat complex.
However, in the case of Metamath it is intentionally trivial.
Mandatory hypotheses must be
pushed on the proof stack in the order in which they appear.
In addition, each variable must have its type specified
with a \texttt{\$f} hypothesis before it is used
and that each \texttt{\$f} hypothesis
have the restricted syntax of a typecode (a constant) followed by a variable.
The typecode in the \texttt{\$f} hypothesis must match the first symbol of
the corresponding RPN stack entry (which will also be a constant), so
the only possible match for the variable in the \texttt{\$f} hypothesis is
the sequence of symbols in the stack entry after the initial constant.

In the Proof Assistant\index{Proof Assistant}, a more general unification
algorithm is used.  While a proof is being developed, sometimes not enough
information is available to determine a unique unification.  In this case
Metamath will ask you to pick the correct one.\index{ambiguous
unification}\index{unification!ambiguous}

\section{Extensions to the Metamath Language}\index{extended
language}

\subsection{Comments in the Metamath Language}\label{comments}
\index{markup notation}
\index{comments!markup notation}

The commenting feature allows you to annotate the contents of
a database.  Just as with most
computer languages, comments are ignored for the purpose of interpreting the
contents of the database. Comments effectively act as
additional white space\index{white
space} between tokens
when a database is parsed.

A comment may be placed at the beginning, end, or
between any two tokens\index{token} in a source file.

Comments have the following syntax:
\begin{center}
 \texttt{\$(} {\em text} \texttt{\$)}
\end{center}
Here,\index{\texttt{\$(} and \texttt{\$)} auxiliary
keywords}\index{comment} {\em text} is a string, possibly empty, of any
characters in Metamath's character set (p.~\pageref{spec1chars}), except
that the character strings \texttt{\$(} and \texttt{\$)} may not appear
in {\em text}.  Thus nested comments are not
permitted:\footnote{Computer languages have differing standards for
nested comments, and rather than picking one it was felt simplest not to
allow them at all, at least in the current version (0.177) of
Metamath\index{Metamath!limitations of version 0.177}.} Metamath will
complain if you give it
\begin{center}
 \texttt{\$( This is a \$( nested \$) comment.\ \$)}
\end{center}
To compensate for this non-nesting behavior, I often change all \texttt{\$}'s
to \texttt{@}'s in sections of Metamath code I wish to comment out.

The Metamath program supports a number of markup mechanisms and conventions
to generate good-looking results in \LaTeX\ and {\sc html},
as discussed below.
These markup features have to do only with how the comments are typeset,
and have no effect on how Metamath verifies the proofs in the database.
The improper
use of them may result in incorrectly typeset output, but no Metamath
error messages will result during the \texttt{read} and \texttt{verify
proof} commands.  (However, the \texttt{write
theorem\texttt{\char`\_}list} command
will check for markup errors as a side-effect of its
{\sc html} generation.)
Section~\ref{texout} has instructions for creating \LaTeX\ output, and
section~\ref{htmlout} has instructions for creating
{\sc html}\index{HTML} output.

\subsubsection{Headings}\label{commentheadings}

If the \texttt{\$(} is immediately followed by a new line
starting with a heading marker, it is a header.
This can start with:

\begin{itemize}
 \item[] \texttt{\#\#\#\#} - major part header
 \item[] \texttt{\#*\#*} - section header
 \item[] \texttt{=-=-} - subsection header
 \item[] \texttt{-.-.} - subsubsection header
\end{itemize}

The line following the marker line
will be used for the table of contents entry, after trimming spaces.
The next line should be another (closing) matching marker line.
Any text after that
but before the closing \texttt{\$}, such as an extended description of the
section, will be included on the \texttt{mmtheoremsNNN.html} page.

For more information, run
\texttt{help write theorem\char`\_list}.

\subsubsection{Math mode}
\label{mathcomments}
\index{\texttt{`} inside comments}
\index{\texttt{\char`\~} inside comments}
\index{math mode}

Inside of comments, a string of tokens\index{token} enclosed in
grave accents\index{grave accent (\texttt{`})} (\texttt{`}) will be converted
to standard mathematical symbols during
{\sc HTML}\index{HTML} or \LaTeX\ output
typesetting,\index{latex@{\LaTeX}} according to the information in the
special \texttt{\$t}\index{\texttt{\$t} comment}\index{typesetting
comment} comment in the database
(see section~\ref{tcomment} for information about the typesetting
comment, and Appendix~\ref{ASCII} to see examples of its results).

The first grave accent\index{grave accent (\texttt{`})} \texttt{`}
causes the output processor to enter {\bf math mode}\index{math mode}
and the second one exits it.
In this
mode, the characters following the \texttt{`} are interpreted as a
sequence of math symbol tokens separated by white space\index{white
space}.  The tokens are looked up in the \texttt{\$t}
comment\index{\texttt{\$t} comment}\index{typesetting comment} and if
found, they will be replaced by the standard mathematical symbols that
they correspond to before being placed in the typeset output file.  If
not found, the symbol will be output as is and a warning will be issued.
The tokens do not have to be active in the database, although a warning
will be issued if they are not declared with \texttt{\$c} or
\texttt{\$v} statements.

Two consecutive
grave accents \texttt{``} are treated as a single actual grave accent
(both inside and outside of math mode) and will not cause the output
processor to enter or exit math mode.

Here is an example of its use\index{Pierce's axiom}:
\begin{center}
\texttt{\$( Pierce's axiom, ` ( ( ph -> ps ) -> ph ) -> ph ` ,\\
         is not very intuitive. \$)}
\end{center}
becomes
\begin{center}
   \texttt{\$(} Pierce's axiom, $((\varphi \rightarrow \psi)\rightarrow
\varphi)\rightarrow \varphi$, is not very intuitive. \texttt{\$)}
\end{center}

Note that the math symbol tokens\index{token} must be surrounded by white
space\index{white space}.
%, since there is no context that allows ambiguity to be
%resolved, as is the case with math symbol sequences in some of the Metamath
%statements.
White space should also surround the \texttt{`}
delimiters.

The math mode feature also gives you a quick and easy way to generate
text containing mathematical symbols, independently of the intended
purpose of Metamath.\index{Metamath!using as a math editor} To do this,
simply create your text with grave accents surrounding your formulas,
after making sure that your math symbols are mapped to \LaTeX\ symbols
as described in Appendix~\ref{ASCII}.  It is easier if you start with a
database with predefined symbols such as \texttt{set.mm}.  Use your
grave-quoted math string to replace an existing comment, then typeset
the statement corresponding to that comment following the instructions
from the \texttt{help tex} command in the Metamath program.  You will
then probably want to edit the resulting file with a text editor to fine
tune it to your exact needs.

\subsubsection{Label Mode}\index{label mode}

Outside of math mode, a tilde\index{tilde (\texttt{\char`\~})} \verb/~/
indicates to Metamath's\index{Metamath} output processor that the
token\index{token} that follows (i.e.\ the characters up to the next
white space\index{white space}) represents a statement label or URL.
This formatting mode is called {\bf label mode}\index{label mode}.
If a literal tilde
is desired (outside of math mode) instead of label mode,
use two tildes in a row to represent it.

When generating a \LaTeX\ output file,
the following token will be formatted in \texttt{typewriter}
font, and the tilde removed, to make it stand out from the rest of the text.
This formatting will be applied to all characters after the
tilde up to the first white space\index{white space}.
Whether
or not the token is an actual statement label is not checked, and the
token does not have to have the correct syntax for a label; no error
messages will be produced.  The only effect of the label mode on the
output is that typewriter font will be used for the tokens that are
placed in the \LaTeX\ output file.

When generating {\sc html},
the tokens after the tilde {\em must} be a URL (either http: or https:)
or a valid label.
Error messages will be issued during that output if they aren't.
A hyperlink will be generated to that URL or label.

\subsubsection{Link to bibliographical reference}\index{citation}%
\index{link to bibliographical reference}

Bibliographical references are handled specially when generating
{\sc html} if formatted specially.
Text in the form \texttt{[}{\em author}\texttt{]}
is considered a link to a bibliographical reference.
See \texttt{help html} and \texttt{help write
bibliography} in the Metamath program for more
information.
% \index{\texttt{\char`\[}\ldots\texttt{]} inside comments}
See also Sections~\ref{tcomment} and \ref{wrbib}.

The \texttt{[}{\em author}\texttt{]} notation will also create an entry in
the bibliography cross-reference file generated by \texttt{write
bibliography} (Section~\ref{wrbib}) for {\sc HTML}.
For this to work properly, the
surrounding comment must be formatted as follows:
\begin{quote}
    {\em keyword} {\em label} {\em noise-word}
     \texttt{[}{\em author}\texttt{] p.} {\em number}
\end{quote}
for example
\begin{verbatim}
     Theorem 5.2 of [Monk] p. 223
\end{verbatim}
The {\em keyword} is not case sensitive and must be one of the following:
\begin{verbatim}
     theorem lemma definition compare proposition corollary
     axiom rule remark exercise problem notation example
     property figure postulate equation scheme chapter
\end{verbatim}
The optional {\em label} may consist of more than one
(non-{\em keyword} and non-{\em noise-word}) word.
The optional {\em noise-word} is one of:
\begin{verbatim}
     of in from on
\end{verbatim}
and is  ignored when the cross-reference file is created.  The
\texttt{write
biblio\-graphy} command will perform error checking to verify the
above format.\index{error checking}

\subsubsection{Parentheticals}\label{parentheticals}

The end of a comment may include one or more parenthicals, that is,
statements enclosed in parentheses.
The Metamath program looks for certain parentheticals and can issue
warnings based on them.
They are:

\begin{itemize}
 \item[] \texttt{(Contributed by }
   \textit{NAME}\texttt{,} \textit{DATE}\texttt{.)} -
   document the original contributor's name and the date it was created.
 \item[] \texttt{(Revised by }
   \textit{NAME}\texttt{,} \textit{DATE}\texttt{.)} -
   document the contributor's name and creation date
   that resulted in significant revision
   (not just an automated minimization or shortening).
 \item[] \texttt{(Proof shortened by }
   \textit{NAME}\texttt{,} \textit{DATE}\texttt{.)} -
   document the contributor's name and date that developed a significant
   shortening of the proof (not just an automated minimization).
 \item[] \texttt{(Proof modification is discouraged.)} -
   Note that this proof should normally not be modified.
 \item[] \texttt{(New usage is discouraged.)} -
   Note that this assertion should normally not be used.
\end{itemize}

The \textit{DATE} must be in form YYYY-MMM-DD, where MMM is the
English abbreviation of that month.

\subsubsection{Other markup}\label{othermarkup}
\index{markup notation}

There are other markup notations for generating good-looking results
beyond math mode and label mode:

\begin{itemize}
 \item[]
         \texttt{\char`\_} (underscore)\index{\texttt{\char`\_} inside comments} -
             Italicize text starting from
              {\em space}\texttt{\char`\_}{\em non-space} (i.e.\ \texttt{\char`\_}
              with a space before it and a non-space character after it) until
             the next
             {\em non-space}\texttt{\char`\_}{\em space}.  Normal
             punctuation (e.g.\ a trailing
             comma or period) is ignored when determining {\em space}.
 \item[]
         \texttt{\char`\_} (underscore) - {\em
         non-space}\texttt{\char`\_}{\em non-space-string}, where
          {\em non-space-string} is a string of non-space characters,
         will make {\em non-space-string} become a subscript.
 \item[]
         \texttt{<HTML>}...\texttt{</HTML>} - do not convert
         ``\texttt{<}'' and ``\texttt{>}''
         in the enclosed text when generating {\sc HTML},
         otherwise process markup normally. This allows direct insertion
         of {\sc html} commands.
 \item[]
       ``\texttt{\&}ref\texttt{;}'' - insert an {\sc HTML}
         character reference.
         This is how to insert arbitrary Unicode characters
         (such as accented characters).  Currently only directly supported
         when generating {\sc HTML}.
\end{itemize}

It is recommended that spaces surround any \texttt{\char`\~} and
\texttt{`} tokens in the comment and that a space follow the {\em label}
after a \texttt{\char`\~} token.  This will make global substitutions
to change labels and symbol names much easier and also eliminate any
future chance of ambiguity.  Spaces around these tokens are automatically
removed in the final output to conform with normal rules of punctuation;
for example, a space between a trailing \texttt{`} and a left parenthesis
will be removed.

A good way to become familiar with the markup notation is to look at
the extensive examples in the \texttt{set.mm} database.

\subsection{The Typesetting Comment (\texttt{\$t})}\label{tcomment}

The typesetting comment \texttt{\$t} in the input database file
provides the information necessary to produce good-looking results.
It provides \LaTeX\ and {\sc html}
definitions for math symbols,
as well supporting as some
customization of the generated web page.
If you add a new token to a database, you should also
update the \texttt{\$t} comment information if you want to eventually
create output in \LaTeX\ or {\sc HTML}.
See the
\texttt{set.mm}\index{set theory database (\texttt{set.mm})} database
file for an extensive example of a \texttt{\$t} comment illustrating
many of the features described below.

Programs that do not need to generate good-looking presentation results,
such as programs that only verify Metamath databases,
can completely ignore typesetting comments
and just treat them as normal comments.
Even the Metamath program only consults the
\texttt{\$t} comment information when it needs to generate typeset output
in \LaTeX\ or {\sc HTML}
(e.g., when you open a \LaTeX\ output file with the \texttt{open tex} command).

We will first discuss the syntax of typesetting comments, and then
briefly discuss how this can be used within the Metamath program.

\subsubsection{Typesetting Comment Syntax Overview}

The typesetting comment is identified by the token
\texttt{\$t}\index{\texttt{\$t} comment}\index{typesetting comment} in
the comment, and the typesetting comment ends at the matching
\texttt{\$)}:
\[
  \mbox{\tt \$(\ }
  \mbox{\tt \$t\ }
  \underbrace{
    \mbox{\tt \ \ \ \ \ \ \ \ \ \ \ }
    \cdots
    \mbox{\tt \ \ \ \ \ \ \ \ \ \ \ }
  }_{\mbox{Typesetting definitions go here}}
  \mbox{\tt \ \$)}
\]

There must be one or more white space characters, and only white space
characters, between the \texttt{\$(} that starts the comment
and the \texttt{\$t} symbol,
and the \texttt{\$t} must be followed by one
or more white space characters
(see section \ref{whitespace} for the definition of white space characters).
The typesetting comment continues until the comment end token \texttt{\$)}
(which must be preceded by one or more white space characters).

In version 0.177\index{Metamath!limitations of version 0.177} of the
Metamath program, there may be only one \texttt{\$t} comment in a
database.  This restriction may be lifted in the future to allow
many \texttt{\$t} comments in a database.

Between the \texttt{\$t} symbol (and its following white space) and the
comment end token \texttt{\$)} (and its preceding white space)
is a sequence of one or more typesetting definitions, where
each definition has the form
\textit{definition-type arg arg ... ;}.
Each of the zero or more \textit{arg} values
can be either a typesetting data or a keyword
(what keywords are allowed, and where, depends on the specific
\textit{definition-type}).
The \textit{definition-type}, and each argument \textit{arg},
are separated by one or more white space characters.
Every definition ends in an unquoted semicolon;
white space is not required before the terminating semicolon of a definition.
Each definition should start on a new line.\footnote{This
restriction of the current version of Metamath
(0.177)\index{Metamath!limitations of version 0.177} may be removed
in a future version, but you should do it anyway for readability.}

For example, this typesetting definition:
\begin{center}
 \verb$latexdef "C_" as "\subseteq";$
\end{center}
defines the token \verb$C_$ as the \LaTeX\ symbol $\subseteq$ (which means
``subset'').

Typesetting data is a sequence of one or more quoted strings
(if there is more than one, they are connected by \texttt{\char`\+}).
Often a single quoted string is used to provide data for a definition, using
either double (\texttt{\char`\"}) or single (\texttt{'}) quotation marks.
However,
{\em a quoted string (enclosed in quotation marks) may not include
line breaks.}
A quoted string
may include a quotation mark that matches the enclosing quotes by repeating
the quotation mark twice.  Here are some examples:

\begin{tabu}   { l l }
\textbf{Example} & \textbf{Meaning} \\
\texttt{\char`\"a\char`\"\char`\"b\char`\"} & \texttt{a\char`\"b} \\
\texttt{'c''d'} & \texttt{c'd} \\
\texttt{\char`\"e''f\char`\"} & \texttt{e''f} \\
\texttt{'g\char`\"\char`\"h'} & \texttt{g\char`\"\char`\"h} \\
\end{tabu}

Finally, a long quoted string
may be broken up into multiple quoted strings (considered, as a whole,
a single quoted string) and joined with \texttt{\char`\+}.
You can even use multiple lines as long as a
'+' is at the end of every line except the last one.
The \texttt{\char`\+} should be preceded and followed by at least one
white space character.
Thus, for example,
\begin{center}
 \texttt{\char`\"ab\char`\"\ \char`\+\ \char`\"cd\char`\"
    \ \char`\+\ \\ 'ef'}
\end{center}
is the same as
\begin{center}
 \texttt{\char`\"abcdef\char`\"}
\end{center}

{\sc c}-style comments \texttt{/*}\ldots\texttt{*/} are also supported.

In practice, whenever you add a new math token you will often want to add
typesetting definitions using
\texttt{latexdef}, \texttt{htmldef}, and
\texttt{althtmldef}, as described below.
That way, they will all be up to date.
Of course, whether or not you want to use all three definitions will
depend on how the database is intended to be used.

Below we discuss the different possible \textit{definition-kind} options.
We will show data surrounded by double quotes (in practice they can also use
single quotes and/or be a sequence joined by \texttt{+}s).
We will use specific names for the \textit{data} to make clear what
the data is used for, such as
{\em math-token} (for a Metamath math token,
{\em latex-string} (for string to be placed in a \LaTeX\ stream),
{\em {\sc html}-code} (for {\sc html} code),
and {\em filename} (for a filename).

\subsubsection{Typesetting Comment - \LaTeX}

The syntax for a \LaTeX\ definition is:
\begin{center}
 \texttt{latexdef "}{\em math-token}\texttt{" as "}{\em latex-string}\texttt{";}
\end{center}
\index{latex definitions@\LaTeX\ definitions}%
\index{\texttt{latexdef} statement}

The {\em token-string} and {\em latex-string} are the data
(character strings) for
the token and the \LaTeX\ definition of the token, respectively,

These \LaTeX\ definitions are used by the Metamath program
when it is asked to product \LaTeX output using
the \texttt{write tex} command.

\subsubsection{Typesetting Comment - {\sc html}}

The key kinds of {\sc HTML} definitions have the following syntax:

\vskip 1ex
    \texttt{htmldef "}{\em math-token}\texttt{" as "}{\em
    {\sc html}-code}\texttt{";}\index{\texttt{htmldef} statement}
                    \ \ \ \ \ \ldots

    \texttt{althtmldef "}{\em math-token}\texttt{" as "}{\em
{\sc html}-code}\texttt{";}\index{\texttt{althtmldef} statement}

                    \ \ \ \ \ \ldots

Note that in {\sc HTML} there are two possible definitions for math tokens.
This feature is useful when
an alternate representation of symbols is desired, for example one that
uses Unicode entities and another uses {\sc gif} images.

There are many other typesetting definitions that can control {\sc HTML}.
These include:

\vskip 1ex

    \texttt{htmldef "}{\em math-token}\texttt{" as "}{\em {\sc
    html}-code}\texttt{";}

    \texttt{htmltitle "}{\em {\sc html}-code}\texttt{";}%
\index{\texttt{htmltitle} statement}

    \texttt{htmlhome "}{\em {\sc html}-code}\texttt{";}%
\index{\texttt{htmlhome} statement}

    \texttt{htmlvarcolor "}{\em {\sc html}-code}\texttt{";}%
\index{\texttt{htmlvarcolor} statement}

    \texttt{htmlbibliography "}{\em filename}\texttt{";}%
\index{\texttt{htmlbibliography} statement}

\vskip 1ex

\noindent The \texttt{htmltitle} is the {\sc html} code for a common
title, such as ``Metamath Proof Explorer.''  The \texttt{htmlhome} is
code for a link back to the home page.  The \texttt{htmlvarcolor} is
code for a color key that appears at the bottom of each proof.  The file
specified by {\em filename} is an {\sc html} file that is assumed to
have a \texttt{<A NAME=}\ldots\texttt{>} tag for each bibiographic
reference in the database comments.  For example, if
\texttt{[Monk]}\index{\texttt{\char`\[}\ldots\texttt{]} inside comments}
occurs in the comment for a theorem, then \texttt{<A NAME='Monk'>} must
be present in the file; if not, a warning message is given.

Associated with
\texttt{althtmldef}
are the statements
\vskip 1ex

    \texttt{htmldir "}{\em
      directoryname}\texttt{";}\index{\texttt{htmldir} statement}

    \texttt{althtmldir "}{\em
     directoryname}\texttt{";}\index{\texttt{althtmldir} statement}

\vskip 1ex
\noindent giving the directories of the {\sc gif} and Unicode versions
respectively; their purpose is to provide cross-linking between the
two versions in the generated web pages.

When two different types of pages need to be produced from a single
database, such as the Hilbert Space Explorer that extends the Metamath
Proof Explorer, ``extended'' variables may be declared in the
\texttt{\$t} comment:
\vskip 1ex

    \texttt{exthtmltitle "}{\em {\sc html}-code}\texttt{";}%
\index{\texttt{exthtmltitle} statement}

    \texttt{exthtmlhome "}{\em {\sc html}-code}\texttt{";}%
\index{\texttt{exthtmlhome} statement}

    \texttt{exthtmlbibliography "}{\em filename}\texttt{";}%
\index{\texttt{exthtmlbibliography} statement}

\vskip 1ex
\noindent When these are declared, you also must declare
\vskip 1ex

    \texttt{exthtmllabel "}{\em label}\texttt{";}%
\index{\texttt{exthtmllabel} statement}

\vskip 1ex \noindent that identifies the database statement where the
``extended'' section of the database starts (in our example, where the
Hilbert Space Explorer starts).  During the generation of web pages for
that starting statement and the statements after it, the {\sc html} code
assigned to \texttt{exthtmltitle} and \texttt{exthtmlhome} is used
instead of that assigned to \texttt{htmltitle} and \texttt{htmlhome},
respectively.

\begin{sloppy}
\subsection{Additional Information Com\-ment (\texttt{\$j})} \label{jcomment}
\end{sloppy}

The additional information comment, aka the
\texttt{\$j}\index{\texttt{\$j} comment}\index{additional information comment}
comment,
provides a way to add additional structured information that can
be optionally parsed by systems.

The additional information comment is parsed the same way as the
typesetting comment (\texttt{\$t}) (see section \ref{tcomment}).
That is,
the additional information comment begins with the token
\texttt{\$j} within a comment,
and continues until the comment close \texttt{\$)}.
Within an additional information comment is a sequence of one or more
commands of the form \texttt{command arg arg ... ;}
where each of the zero or more \texttt{arg} values
can be either a quoted string or a keyword.
Note that every command ends in an unquoted semicolon.
If a verifier is parsing an additional information comment, but
doesn't recognize a particular command, it must skip the command
by finding the end of the command (an unquoted semicolon).

A database may have 0 or more additional information comments.
Note, however, that a verifier may ignore these comments entirely or only
process certain commands in an additional information comment.
The \texttt{mmj2} verifier supports many commands in additional information
comments.
We encourage systems that process additional information comments
to coordinate so that they will use the same command for the same effect.

Examples of additional information comments with various commands
(from the \texttt{set.mm} database) are:

\begin{itemize}
   \item Define the syntax and logical typecodes,
     and declare that our grammar is
     unambiguous (verifiable using the KLR parser, with compositing depth 5).
\begin{verbatim}
  $( $j
    syntax 'wff';
    syntax '|-' as 'wff';
    unambiguous 'klr 5';
  $)
\end{verbatim}

   \item Register $\lnot$ and $\rightarrow$ as primitive expressions
           (lacking definitions).
\begin{verbatim}
  $( $j primitive 'wn' 'wi'; $)
\end{verbatim}

   \item There is a special justification for \texttt{df-bi}.
\begin{verbatim}
  $( $j justification 'bijust' for 'df-bi'; $)
\end{verbatim}

   \item Register $\leftrightarrow$ as an equality for its type (wff).
\begin{verbatim}
  $( $j
    equality 'wb' from 'biid' 'bicomi' 'bitri';
    definition 'dfbi1' for 'wb';
  $)
\end{verbatim}

   \item Theorem \texttt{notbii} is the congruence law for negation.
\begin{verbatim}
  $( $j congruence 'notbii'; $)
\end{verbatim}

   \item Add \texttt{setvar} as a typecode.
\begin{verbatim}
  $( $j syntax 'setvar'; $)
\end{verbatim}

   \item Register $=$ as an equality for its type (\texttt{class}).
\begin{verbatim}
  $( $j equality 'wceq' from 'eqid' 'eqcomi' 'eqtri'; $)
\end{verbatim}

\end{itemize}


\subsection{Including Other Files in a Metamath Source File} \label{include}
\index{\texttt{\$[} and \texttt{\$]} auxiliary keywords}

The keywords \texttt{\$[} and \texttt{\$]} specify a file to be
included\index{included file}\index{file inclusion} at that point in a
Metamath\index{Metamath} source file\index{source file}.  The syntax for
including a file is as follows:
\begin{center}
\texttt{\$[} {\em file-name} \texttt{\$]}
\end{center}

The {\em file-name} should be a single token\index{token} with the same syntax
as a math symbol (i.e., all 93 non-whitespace
printable characters other than \texttt{\$} are
allowed, subject to the file-naming limitations of your operating system).
Comments may appear between the \texttt{\$[} and \texttt{\$]} keywords.  Included
files may include other files, which may in turn include other files, and so
on.

For example, suppose you want to use the set theory database as the starting
point for your own theory.  The first line in your file could be
\begin{center}
\texttt{\$[ set.mm \$]}
\end{center} All of the information (axioms, theorems,
etc.) in \texttt{set.mm} and any files that {\em it} includes will become
available for you to reference in your file. This can help make your work more
modular. A drawback to including files is that if you change the name of a
symbol or the label of a statement, you must also remember to update any
references in any file that includes it.


The naming conventions for included files are the same as those of your
operating system.\footnote{On the Macintosh, prior to Mac OS X,
 a colon is used to separate disk
and folder names from your file name.  For example, {\em volume}\texttt{:}{\em
file-name} refers to the root directory, {\em volume}\texttt{:}{\em
folder-name}\texttt{:}{\em file-name} refers to a folder in root, and {\em
volume}\texttt{:}{\em folder-name}\texttt{:}\ldots\texttt{:}{\em file-name} refers to a
deeper folder.  A simple {\em file-name} refers to a file in the folder from
which you launch the Metamath application.  Under Mac OS X and later,
the Metamath program is run under the Terminal application, which
conforms to Unix naming conventions.}\index{Macintosh file
names}\index{file names!Macintosh}\label{includef} For compatibility among
operating systems, you should keep the file names as simple as possible.  A
good convention to use is {\em file}\texttt{.mm} where {\em file} is eight
characters or less, in lower case.

There is no limit to the nesting depth of included files.  One thing that you
should be aware of is that if two included files themselves include a common
third file, only the {\em first} reference to this common file will be read
in.  This allows you to include two or more files that build on a common
starting file without having to worry about label and symbol conflicts that
would occur if the common file were read in more than once.  (In fact, if a
file includes itself, the self-reference will be ignored, although of course
it would not make any sense to do that.)  This feature also means, however,
that if you try to include a common file in several inner blocks, the result
might not be what you expect, since only the first reference will be replaced
with the included file (unlike the include statement in most other computer
languages).  Thus you would normally include common files only in the
outermost block\index{outermost block}.

\subsection{Compressed Proof Format}\label{compressed1}\index{compressed
proof}\index{proof!compressed}

The proof notation presented in Section~\ref{proof} is called a
{\bf normal proof}\index{normal proof}\index{proof!normal} and in principle is
sufficient to express any proof.  However, proofs often contain steps and
subproofs that are identical.  This is particularly true in typical
Metamath\index{Metamath} applications, because Metamath requires that the math
symbol sequence (usually containing a formula) at each step be separately
constructed, that is, built up piece by piece. As a result, a lot of
repetition often results.  The {\bf compressed proof} format allows Metamath
to take advantage of this redundancy to shorten proofs.

The specification for the compressed proof format is given in
Appen\-dix~\ref{compressed}.

Normally you need not concern yourself with the details of the compressed
proof format, since the Metamath program will allow you to convert from
the normal format to the compressed format with ease, and will also
automatically convert from the compressed format when proofs are displayed.
The overall structure of the compressed format is as follows:
\begin{center}
  \texttt{\$= ( } {\em label-list} \texttt{) } {\em compressed-proof\ }\ \texttt{\$.}
\end{center}
\index{\texttt{\$=} keyword}
The first \texttt{(} serves as a flag to Metamath that a compressed proof
follows.  The {\em label-list} includes all statements referred to by the
proof except the mandatory hypotheses\index{mandatory hypothesis}.  The {\em
compressed-proof} is a compact encoding of the proof, using upper-case
letters, and can be thought of as a large integer in base 26.  White
space\index{white space} inside a {\em compressed-proof} is
optional and is ignored.

It is important to note that the order of the mandatory hypotheses of
the statement being proved must not be changed if the compressed proof
format is used, otherwise the proof will become incorrect.  The reason
for this is that the mandatory hypotheses are not mentioned explicitly
in the compressed proof in order to make the compression more efficient.
If you wish to change the order of mandatory hypotheses, you must first
convert the proof back to normal format using the \texttt{save proof
{\em statement} /normal}\index{\texttt{save proof} command} command.
Later, you can go back to compressed format with \texttt{save proof {\em
statement} /compressed}.

During error checking with the \texttt{verify proof} command, an error
found in a compressed proof may point to a character in {\em
compressed-proof}, which may not be very meaningful to you.  In this
case, try to \texttt{save proof /normal} first, then do the
\texttt{verify proof} again.  In general, it is best to make sure a
proof is correct before saving it in compressed format, because severe
errors are less likely to be recoverable than in normal format.

\subsection{Specifying Unknown Proofs or Subproofs}\label{unknown}

In a proof under development, any step or subproof that is not yet known
may be represented with a single \texttt{?}.  For the purposes of
parsing the proof, the \texttt{?}\ \index{\texttt{]}@\texttt{?}\ inside
proofs} will push a single entry onto the RPN stack just as if it were a
hypothesis.  While developing a proof with the Proof
Assistant\index{Proof Assistant}, a partially developed proof may be
saved with the \texttt{save new{\char`\_}proof}\index{\texttt{save
new{\char`\_}proof} command} command, and \texttt{?}'s will be placed at
the appropriate places.

All \texttt{\$p}\index{\texttt{\$p} statement} statements must have
proofs, even if they are entirely unknown.  Before creating a proof with
the Proof Assistant, you should specify a completely unknown proof as
follows:
\begin{center}
  {\em label} \texttt{\$p} {\em statement} \texttt{\$= ?\ \$.}
\end{center}
\index{\texttt{\$=} keyword}
\index{\texttt{]}@\texttt{?}\ inside proofs}

The \texttt{verify proof}\index{\texttt{verify proof} command} command
will check the known portions of a partial proof for errors, but will
warn you that the statement has not been proved.

Note that partially developed proofs may be saved in compressed format
if desired.  In this case, you will see one or more \texttt{?}'s in the
{\em compressed-proof} part.\index{compressed
proof}\index{proof!compressed}

\section{Axioms vs.\ Definitions}\label{definitions}

The \textit{basic}
Metamath\index{Metamath} language and program
make no distinction\index{axiom vs.\
definition} between axioms\index{axiom} and
definitions.\index{definition} The \texttt{\$a}\index{\texttt{\$a}
statement} statement is used for both.  At first, this may seem
puzzling.  In the minds of many mathematicians, the distinction is
clear, even obvious, and hardly worth discussing.  A definition is
considered to be merely an abbreviation that can be replaced by the
expression for which it stands; although unless one actually does this,
to be precise then one should say that a theorem\index{theorem} is a
consequence of the axioms {\em and} the definitions that are used in the
formulation of the theorem \cite[p.~20]{Behnke}.\index{Behnke, H.}

\subsection{What is a Definition?}

What is a definition?  In its simplest form, a definition introduces a new
symbol and provides an unambiguous rule to transform an expression containing
the new symbol to one without it.  The concept of a ``proper
definition''\index{proper definition}\index{definition!proper} (as opposed to
a creative definition)\index{creative definition}\index{definition!creative}
that is usually agreed upon is (1) the definition should not strengthen the
language and (2) any symbols introduced by the definition should be eliminable
from the language \cite{Nemesszeghy}\index{Nemesszeghy, E. Z.}.  In other
words, they are mere typographical conveniences that do not belong to the
system and are theoretically superfluous.  This may seem obvious, but in fact
the nature of definitions can be subtle, sometimes requiring difficult
metatheorems to establish that they are not creative.

A more conservative stance was taken by logician S.
Le\'{s}niewski.\index{Le\'{s}niewski, S.}
\begin{quote}
Le\'{s}niewski
regards definitions as theses of the system.  In this respect they do
not differ either from the axioms or from theorems, i.e.\ from the
theses added to the system on the basis of the rule of substitution or
the rule of detachment [modus ponens].  Once definitions have been
accepted as theses of the system, it becomes necessary to consider them
as true propositions in the same sense in which axioms are true
\cite{Lejewski}.
\end{quote}\index{Lejewski, Czeslaw}

Let us look at some simple examples of definitions in propositional
calculus.  Consider the definition of logical {\sc or}
(disjunction):\index{disjunction ($\vee$)} ``$P\vee Q$ denotes $\neg P
\rightarrow Q$ (not $P$ implies $Q$).''  It is very easy to recognize a
statement making use of this definition, because it introduces the new
symbol $\vee$ that did not previously exist in the language.  It is easy
to see that no new theorems of the original language will result from
this definition.

Next, consider a definition that eliminates parentheses:  ``$P
\rightarrow Q\rightarrow R$ denotes $P\rightarrow (Q \rightarrow R)$.''
This is more subtle, because no new symbols are introduced.  The reason
this definition is considered proper is that no new symbol sequences
that are valid wffs (well-formed formulas)\index{well-formed formula
(wff)} in the original language will result from the definition, since
``$P \rightarrow Q\rightarrow R$'' is not a wff in the original
language.  Here, we implicitly make use of the fact that there is a
decision procedure that allows us to determine whether or not a symbol
sequence is a wff, and this fact allows us to use symbol sequences that
are not wffs to represent other things (such as wffs) by means of the
definition.  However, to justify the definition as not being creative we
need to prove that ``$P \rightarrow Q\rightarrow R$'' is in fact not a
wff in the original language, and this is more difficult than in the
case where we simply introduce a new symbol.

%Now let's take this reasoning to an extreme.  Propositional calculus is a
%decidable theory,\footnote{This means that a mechanical algorithm exists to
%determine whether or not a wff is a theorem.} so in principle we could make use
%of symbol sequences that are not theorems to represent other things (say, to
%encode actual theorems in a more compact way).  For example, let us extend the
%language by defining a wff ``$P$'' in the extended language as the theorem
%``$P\rightarrow P$''\footnote{This is one of the first theorems proved in the
%Metamath database \texttt{set.mm}.}\index{set
%theory database (\texttt{set.mm})} in the original language whenever ``$P$'' is
%not a theorem in the original language.  In the extended language, any wff
%``$Q$'' thus represents a theorem; to find out what theorem (in the original
%language) ``$Q$'' represents, we determine whether ``$Q$'' is a theorem in the
%original language (before the definition was introduced).  If so, we're done; if
%not, we replace ``$Q$'' by ``$Q\rightarrow Q$'' to eliminate the definition.
%This definition is therefore eliminable, and it does not ``strengthen'' the
%language because any wff that is not a theorem is not in the set of statements
%provable in the original language and thus is available for use by definitions.
%
%Of course, a definition such as this would render practically useless the
%communication of theorems of propositional calculus; but
%this is just a human shortcoming, since we can't always easily discern what is
%and is not a theorem by inspection.  In fact, the extended theory with this
%definition has no more and no less information than the original theory; it just
%expresses certain theorems of the form ``$P\rightarrow P$''
%in a more compact way.
%
%The point here is that what constitutes a proper definition is a matter of
%judgment about whether a symbol sequence can easily be recognized by a human
%as invalid in some sense (for example, not a wff); if so, the symbol sequence
%can be appropriated for use by a definition in order to make the extended
%language more compact.  Metamath\index{Metamath} lacks the ability to make this
%judgment, since as far as Metamath is concerned the definition of a wff, for
%example, is arbitrary.  You define for Metamath how wffs\index{well-formed
%formula (wff)} are constructed according to your own preferred style.  The
%concept of a wff may not even exist in a given formal system\index{formal
%system}.  Metamath treats all definitions as if they were new axioms, and it
%is up to the human mathematician to judge whether the definition is ``proper''
%'\index{proper definition}\index{definition!proper} in some agreed-upon way.

What constitutes a definition\index{definition} versus\index{axiom vs.\
definition} an axiom\index{axiom} is sometimes arbitrary in mathematical
literature.  For example, the connectives $\vee$ ({\sc or}), $\wedge$
({\sc and}), and $\leftrightarrow$ (equivalent to) in propositional
calculus are usually considered defined symbols that can be used as
abbreviations for expressions containing the ``primitive'' connectives
$\rightarrow$ and $\neg$.  This is the way we treat them in the standard
logic and set theory database \texttt{set.mm}\index{set theory database
(\texttt{set.mm})}.  However, the first three connectives can also be
considered ``primitive,'' and axiom systems have been devised that treat
all of them as such.  For example,
\cite[p.~35]{Goodstein}\index{Goodstein, R. L.} presents one with 15
axioms, some of which in fact coincide with what we have chosen to call
definitions in \texttt{set.mm}.  In certain subsets of classical
propositional calculus, such as the intuitionist
fragment\index{intuitionism}, it can be shown that one cannot make do
with just $\rightarrow$ and $\neg$ but must treat additional connectives
as primitive in order for the system to make sense.\footnote{Two nice
systems that make the transition from intuitionistic and other weak
fragments to classical logic just by adding axioms are given in
\cite{Robinsont}\index{Robinson, T. Thacher}.}

\subsection{The Approach to Definitions in \texttt{set.mm}}

In set theory, recursive definitions define a newly introduced symbol in
terms of itself.
The justification of recursive definitions, using
several ``recursion theorems,'' is usually one of the first
sophisticated proofs a student encounters when learning set theory, and
there is a significant amount of implicit metalogic behind a recursive
definition even though the definition itself is typically simple to
state.

Metamath itself has no built-in technical limitation that prevents
multiple-part recursive definitions in the traditional textbook style.
However, because the recursive definition requires advanced metalogic
to justify, eliminating a recursive definition is very difficult and
often not even shown in textbooks.

\subsubsection{Direct definitions instead of recursive definitions}

It is, however, possible to substitute one kind of complexity
for another.  We can eliminate the need for metalogical justification by
defining the operation directly with an explicit (but complicated)
expression, then deriving the recursive definition directly as a
theorem, using a recursion theorem ``in reverse.''
The elimination
of a direct definition is a matter of simple mechanical substitution.
We do this in
\texttt{set.mm}, as follows.

In \texttt{set.mm} our goal was to introduce almost all definitions in
the form of two expressions connected by either $\leftrightarrow$ or
$=$, where the thing being defined does not appear on the right hand
side.  Quine calls this form ``a genuine or direct definition'' \cite[p.
174]{Quine}\index{Quine, Willard Van Orman}, which makes the definitions
very easy to eliminate and the metalogic\index{metalogic} needed to
justify them as simple as possible.
Put another way, we had a goal of being able to
eliminate all definitions with direct mechanical substitution and to
verify easily the soundness of the definitions.

\subsubsection{Example of direct definitions}

We achieved this goal in almost all cases in \texttt{set.mm}.
Sometimes this makes the definitions more complex and less
intuitive.
For example, the traditional way to define addition of
natural numbers is to define an operation called {\em
successor}\index{successor} (which means ``plus one'' and is denoted by
``${\rm suc}$''), then define addition recursively\index{recursive
definition} with the two definitions $n + 0 = n$ and $m + {\rm suc}\,n =
{\rm suc} (m + n)$.  Although this definition seems simple and obvious,
the method to eliminate the definition is not obvious:  in the second
part of the definition, addition is defined in terms of itself.  By
eliminating the definition, we don't mean repeatedly applying it to
specific $m$ and $n$ but rather showing the explicit, closed-form
set-theoretical expression that $m + n$ represents, that will work for
any $m$ and $n$ and that does not have a $+$ sign on its right-hand
side.  For a recursive definition like this not to be circular
(creative), there are some hidden, underlying assumptions we must make,
for example that the natural numbers have a certain kind of order.

In \texttt{set.mm} we chose to start with the direct (though complex and
nonintuitive) definition then derive from it the standard recursive
definition.
For example, the closed-form definition used in \texttt{set.mm}
for the addition operation on ordinals\index{ordinal
addition}\index{addition!of ordinals} (of which natural numbers are a
subset) is

\setbox\startprefix=\hbox{\tt \ \ df-oadd\ \$a\ }
\setbox\contprefix=\hbox{\tt \ \ \ \ \ \ \ \ \ \ \ \ \ }
\startm
\m{\vdash}\m{+_o}\m{=}\m{(}\m{x}\m{\in}\m{{\rm On}}\m{,}\m{y}\m{\in}\m{{\rm
On}}\m{\mapsto}\m{(}\m{{\rm rec}}\m{(}\m{(}\m{z}\m{\in}\m{{\rm
V}}\m{\mapsto}\m{{\rm suc}}\m{z}\m{)}\m{,}\m{x}\m{)}\m{`}\m{y}\m{)}\m{)}
\endm
\noindent which depends on ${\rm rec}$.

\subsubsection{Recursion operators}

The above definition of \texttt{df-oadd} depends on the definition of
${\rm rec}$, a ``recursion operator''\index{recursion operator} with
the definition \texttt{df-rdg}:

\setbox\startprefix=\hbox{\tt \ \ df-rdg\ \$a\ }
\setbox\contprefix=\hbox{\tt \ \ \ \ \ \ \ \ \ \ \ \ }
\startm
\m{\vdash}\m{{\rm
rec}}\m{(}\m{F}\m{,}\m{I}\m{)}\m{=}\m{\mathrm{recs}}\m{(}\m{(}\m{g}\m{\in}\m{{\rm
V}}\m{\mapsto}\m{{\rm if}}\m{(}\m{g}\m{=}\m{\varnothing}\m{,}\m{I}\m{,}\m{{\rm
if}}\m{(}\m{{\rm Lim}}\m{{\rm dom}}\m{g}\m{,}\m{\bigcup}\m{{\rm
ran}}\m{g}\m{,}\m{(}\m{F}\m{`}\m{(}\m{g}\m{`}\m{\bigcup}\m{{\rm
dom}}\m{g}\m{)}\m{)}\m{)}\m{)}\m{)}\m{)}
\endm

\noindent which can be further broken down with definitions shown in
Section~\ref{setdefinitions}.

This definition of ${\rm rec}$
defines a recursive definition generator on ${\rm On}$ (the class of ordinal
numbers) with characteristic function $F$ and initial value $I$.
This operation allows us to define, with
compact direct definitions, functions that are usually defined in
textbooks with recursive definitions.
The price paid with our approach
is the complexity of our ${\rm rec}$ operation
(especially when {\tt df-recs} that it is built on is also eliminated).
But once we get past this hurdle, definitions that would otherwise be
recursive become relatively simple, as in for example {\tt oav}, from
which we prove the recursive textbook definition as theorems {\tt oa0}, {\tt
oasuc}, and {\tt oalim} (with the help of theorems {\tt rdg0}, {\tt rdgsuc},
and {\tt rdglim2a}).  We can also restrict the ${\rm rec}$ operation to
define otherwise recursive functions on the natural numbers $\omega$; see {\tt
fr0g} and {\tt frsuc}.  Our ${\rm rec}$ operation apparently does not appear
in published literature, although closely related is Definition 25.2 of
[Quine] p. 177, which he uses to ``turn...a recursion into a genuine or
direct definition" (p. 174).  Note that the ${\rm if}$ operations (see
{\tt df-if}) select cases based on whether the domain of $g$ is zero, a
successor, or a limit ordinal.

An important use of this definition ${\rm rec}$ is in the recursive sequence
generator {\tt df-seq} on the natural numbers (as a subset of the
complex infinite sequences such as the factorial function {\tt df-fac} and
integer powers {\tt df-exp}).

The definition of ${\rm rec}$ depends on ${\rm recs}$.
New direct usage of the more powerful (and more primitive) ${\rm recs}$
construct is discouraged, but it is available when needed.
This
defines a function $\mathrm{recs} ( F )$ on ${\rm On}$, the class of ordinal
numbers, by transfinite recursion given a rule $F$ which sets the next
value given all values so far.
Unlike {\tt df-rdg} which restricts the
update rule to use only the previous value, this version allows the
update rule to use all previous values, which is why it is described
as ``strong,'' although it is actually more primitive.  See {\tt
recsfnon} and {\tt recsval} for the primary contract of this definition.
It is defined as:

\setbox\startprefix=\hbox{\tt \ \ df-recs\ \$a\ }
\setbox\contprefix=\hbox{\tt \ \ \ \ \ \ \ \ \ \ \ \ \ }
\startm
\m{\vdash}\m{\mathrm{recs}}\m{(}\m{F}\m{)}\m{=}\m{\bigcup}\m{\{}\m{f}\m{|}\m{\exists}\m{x}\m{\in}\m{{\rm
On}}\m{(}\m{f}\m{{\rm
Fn}}\m{x}\m{\wedge}\m{\forall}\m{y}\m{\in}\m{x}\m{(}\m{f}\m{`}\m{y}\m{)}\m{=}\m{(}\m{F}\m{`}\m{(}\m{f}\m{\restriction}\m{y}\m{)}\m{)}\m{)}\m{\}}
\endm

\subsubsection{Closing comments on direct definitions}

From these direct definitions the simpler, more
intuitive recursive definition is derived as a set of theorems.\index{natural
number}\index{addition}\index{recursive definition}\index{ordinal addition}
The end result is the same, but we completely eliminate the rather complex
metalogic that justifies the recursive definition.

Recursive definitions are often considered more efficient and intuitive than
direct ones once the metalogic has been learned or possibly just accepted as
correct.  However, it was felt that direct definition in \texttt{set.mm}
maximizes rigor by minimizing metalogic.  It can be eliminated effortlessly,
something that is difficult to do with a recursive definition.

Again, Metamath itself has no built-in technical limitation that prevents
multiple-part recursive definitions in the traditional textbook style.
Instead, our goal is to eliminate all definitions with
direct mechanical substitution and to verify easily the soundness of
definitions.

\subsection{Adding Constraints on Definitions}

The basic Metamath language and the Metamath program do
not have any built-in constraints on definitions, since they are just
\$a statements.

However, nothing prevents a verification system from verifying
additional rules to impose further limitations on definitions.
For example, the \texttt{mmj2}\index{mmj2} program
supports various kinds of
additional information comments (see section \ref{jcomment}).
One of their uses is to optionally verify additional constraints,
including constraints to verify that definitions meet certain
requirements.
These additional checks are required by the
continuous integration (CI)\index{continuous integration (CI)}
checks of the
\texttt{set.mm}\index{set theory database (\texttt{set.mm})}%
\index{Metamath Proof Explorer}
database.
This approach enables us to optionally impose additional requirements
on definitions if we wish, without necessarily imposing those rules on
all databases or requiring all verification systems to implement them.
In addition, this allows us to impose specialized constraints tailored
to one database while not requiring other systems to implement
those specialized constraints.

We impose two constraints on the
\texttt{set.mm}\index{set theory database (\texttt{set.mm})}%
\index{Metamath Proof Explorer} database
via the \texttt{mmj2}\index{mmj2} program that are worth discussing here:
a parse check and a definition soundness check.

% On February 11, 2019 8:32:32 PM EST, saueran@oregonstate.edu wrote:
% The following addition to the end of set.mm is accepted by the mmj2
% parser and definition checker and the metamath verifier(at least it was
% when I checked, you should check it too), and creates a contradiction by
% proving the theorem |- ph.
% ${
% wleftp $a wff ( ( ph ) $.
% wbothp $a wff ( ph ) $.
% df-leftp $a |- ( ( ( ph ) <-> -. ph ) $.
% df-bothp $a |- ( ( ph ) <-> ph ) $.
% anything $p |- ph $=
%   ( wbothp wn wi wleftp df-leftp biimpi df-bothp mpbir mpbi simplim ax-mp)
%   ABZAMACZDZCZMOEZOCQAEZNDZRNAFGSHIOFJMNKLAHJ $.
% $}
%
% This particular problem is countered by enabling, within mmj2,
% SetParser,mmj.verify.LRParser

First,
we enable a parse check in \texttt{mmj2} (through its
\texttt{SetParser} command) that requires that all new definitions
must \textit{not} create an ambiguous parse for a KLR(5) parser.
This prevents some errors such as definitions with imbalanced parentheses.

Second, we run a definition soundness check specific to
\texttt{set.mm} or databases similar to it.
(through the \texttt{definitionCheck} macro).
Some \texttt{\$a} statements (including all ax-* statemnets)
are excluded from these checks, as they will
always fail this simple check,
but they are appropriate for most definitions.
This check imposes a set of additional rules:

\begin{enumerate}

\item New definitions must be introduced using $=$ or $\leftrightarrow$.

\item No \texttt{\$a} statement introduced before this one is allowed to use the
symbol being defined in this definition, and the definition is not
allowed to use itself (except once, in the definiendum).

\item Every variable in the definiens should not be distinct

\item Every dummy variable in the definiendum
are required to be distinct from each other and from variables in
the definiendum.
To determine this, the system will look for a "justification" theorem
in the database, and if it is not there, attempt to briefly prove
$( \varphi \rightarrow \forall x \varphi )$  for each dummy variable x.

\item Every dummy variable should be a set variable,
unless there is a justification theorem available.

\item Every dummy variable must be bound
(if the system cannot determine this a justification theorem must be
provided).

\end{enumerate}

\subsection{Summary of Approach to Definitions}

In short, when being rigorous it turns out that
definitions can be subtle, sometimes requiring difficult
metatheorems to establish that they are not creative.

Instead of building such complications into the Metamath language itself,
the basic Metmath language and program simply treat traditional
axioms and definitions as the same kind of \texttt{\$a} statement.
We have then built various tools to enable people to
verify additional conditions as their creators believe is appropriate
for those specific databases, without complicating the Metamath foundations.

\chapter{The Metamath Program}\label{commands}

This chapter provides a reference manual for the
Metamath program.\index{Metamath!commands}

Current instructions for obtaining and installing the Metamath program
can be found at the \url{http://metamath.org} web site.
For Windows, there is a pre-compiled version called
\texttt{metamath.exe}.  For Unix, Linux, and Mac OS X
(which we will refer to collectively as ``Unix''), the Metamath program
can be compiled from its source code with the command
\begin{verbatim}
gcc *.c -o metamath
\end{verbatim}
using the \texttt{gcc} {\sc c} compiler available on those systems.

In the command syntax descriptions below, fields enclosed in square
brackets [\ ] are optional.  File names may be optionally enclosed in
single or double quotes.  This is useful if the file name contains
spaces or
slashes (\texttt{/}), such as in Unix path names, \index{Unix file
names}\index{file names!Unix} that might be confused with Metamath
command qualifiers.\index{Unix file names}\index{file names!Unix}

\section{Invoking Metamath}

Unix, Linux, and Mac OS X
have a command-line interface called the {\em
bash shell}.  (In Mac OS X, select the Terminal application from
Applications/Utilities.)  To invoke Metamath from the bash shell prompt,
assuming that the Metamath program is in the current directory, type
\begin{verbatim}
bash$ ./metamath
\end{verbatim}

To invoke Metamath from a Windows DOS or Command Prompt, assuming that
the Metamath program is in the current directory (or in a directory
included in the Path system environment variable), type
\begin{verbatim}
C:\metamath>metamath
\end{verbatim}

To use command-line arguments at invocation, the command-line arguments
should be a list of Metamath commands, surrounded by quotes if they
contain spaces.  In Windows, the surrounding quotes must be double (not
single) quotes.  For example, to read the database file \texttt{set.mm},
verify all proofs, and exit the program, type (under Unix)
\begin{verbatim}
bash$ ./metamath 'read set.mm' 'verify proof *' exit
\end{verbatim}
Note that in Unix, any directory path with \texttt{/}'s must be
surrounded by quotes so Metamath will not interpret the \texttt{/} as a
command qualifier.  So if \texttt{set.mm} is in the \texttt{/tmp}
directory, use for the above example
\begin{verbatim}
bash$ ./metamath 'read "/tmp/set.mm"' 'verify proof *' exit
\end{verbatim}

For convenience, if the command-line has one argument and no spaces in
the argument, the command is implicitly assumed to be \texttt{read}.  In
this one special case, \texttt{/}'s are not interpreted as command
qualifiers, so you don't need quotes around a Unix file name.  Thus
\begin{verbatim}
bash$ ./metamath /tmp/set.mm
\end{verbatim}
and
\begin{verbatim}
bash$ ./metamath "read '/tmp/set.mm'"
\end{verbatim}
are equivalent.


\section{Controlling Metamath}

The Metamath program was first developed on a {\sc vax/vms} system, and
some aspects of its command line behavior reflect this heritage.
Hopefully you will find it reasonably user-friendly once you get used to
it.

Each command line is a sequence of English-like words separated by
spaces, as in \texttt{show settings}.  Command words are not case
sensitive, and only as many letters are needed as are necessary to
eliminate ambiguity; for example, \texttt{sh se} would work for the
command \texttt{show settings}.  In some cases arguments such as file
names, statement labels, or symbol names are required; these are
case-sensitive (although file names may not be on some operating
systems).

A command line is entered by typing it in then pressing the {\em return}
({\em enter}) key.  To find out what commands are available, type
\texttt{?} at the \texttt{MM>} prompt.  To find out the choices at any
point in a command, press {\em return} and you will be prompted for
them.  The default choice (the one selected if you just press {\em
return}) is shown in brackets (\texttt{<>}).

You may also type \texttt{?} in place of a command word to force
Metamath to tell you what the choices are.  The \texttt{?} method won't
work, though, if a non-keyword argument such as a file name is expected
at that point, because the program will think that \texttt{?} is the
value of the argument.

Some commands have one or more optional qualifiers which modify the
behavior of the command.  Qualifiers are preceded by a slash
(\texttt{/}), such as in \texttt{read set.mm / verify}.  Spaces are
optional around the \texttt{/}.  If you need to use a space or
slash in a command
argument, as in a Unix file name, put single or double quotes around the
command argument.

The \texttt{open log} command will save everything you see on the
screen and is useful to help you recover should something go wrong in a
proof, or if you want to document a bug.

If a command responds with more than a screenful, you will be
prompted to \texttt{<return> to continue, Q to quit, or S to scroll to
end}.  \texttt{Q} or \texttt{q} (not case-sensitive) will complete the
command internally but will suppress further output until the next
\texttt{MM>} prompt.  \texttt{s} will suppress further pausing until the
next \texttt{MM>} prompt.  After the first screen, you are also
presented with the choice of \texttt{b} to go back a screenful.  Note
that \texttt{b} may also be entered at the \texttt{MM>} prompt
immediately after a command to scroll back through the output of that
command.

A command line enclosed in quotes is executed by your operating system.
See Section~\ref{oscmd}.

{\em Warning:} Pressing {\sc ctrl-c} will abort the Metamath program
unconditionally.  This means any unsaved work will be lost.


\subsection{\texttt{exit} Command}\index{\texttt{exit} command}

Syntax:  \texttt{exit} [\texttt{/force}]

This command exits from Metamath.  If there have been changes to the
source with the \texttt{save proof} or \texttt{save new{\char`\_}proof}
commands, you will be given an opportunity to \texttt{write source} to
permanently save the changes.

In Proof Assistant\index{Proof Assistant} mode, the \texttt{exit} command will
return to the \verb/MM>/ prompt. If there were changes to the proof, you will
be given an opportunity to \texttt{save new{\char`\_}proof}.

The \texttt{quit} command is a synonym for \texttt{exit}.

Optional qualifier:
    \texttt{/force} - Do not prompt if changes were not saved.  This qualifier is
        useful in \texttt{submit} command files (Section~\ref{sbmt})
        to ensure predictable behavior.





\subsection{\texttt{open log} Command}\index{\texttt{open log} command}
Syntax:  \texttt{open log} {\em file-name}

This command will open a log file that will store everything you see on
the screen.  It is useful to help recovery from a mistake in a long Proof
Assistant session, or to document bugs.\index{Metamath!bugs}

The log file can be closed with \texttt{close log}.  It will automatically be
closed upon exiting Metamath.



\subsection{\texttt{close log} Command}\index{\texttt{close log} command}
Syntax:  \texttt{close log}

The \texttt{close log} command closes a log file if one is open.  See
also \texttt{open log}.




\subsection{\texttt{submit} Command}\index{\texttt{submit} command}\label{sbmt}
Syntax:  \texttt{submit} {\em filename}

This command causes further command lines to be taken from the specified
file.  Note that any line beginning with an exclamation point (\texttt{!}) is
treated as a comment (i.e.\ ignored).  Also note that the scrolling
of the screen output is continuous, so you may want to open a log file
(see \texttt{open log}) to record the results that fly by on the screen.
After the lines in the file are exhausted, Metamath returns to its
normal user interface mode.

The \texttt{submit} command can be called recursively (i.e. from inside
of a \texttt{submit} command file).


Optional command qualifier:

    \texttt{/silent} - suppresses the screen output but still
        records the output in a log file if one is open.


\subsection{\texttt{erase} Command}\index{\texttt{erase} command}
Syntax:  \texttt{erase}

This command will reset Metamath to its starting state, deleting any
data\-base that was \texttt{read} in.
 If there have been changes to the
source with the \texttt{save proof} or \texttt{save new{\char`\_}proof}
commands, you will be given an opportunity to \texttt{write source} to
permanently save the changes.



\subsection{\texttt{set echo} Command}\index{\texttt{set echo} command}
Syntax:  \texttt{set echo on} or \texttt{set echo off}

The \texttt{set echo on} command will cause command lines to be echoed with any
abbreviations expanded.  While learning the Metamath commands, this
feature will show you the exact command that your abbreviated input
corresponds to.



\subsection{\texttt{set scroll} Command}\index{\texttt{set scroll} command}
Syntax:  \texttt{set scroll prompted} or \texttt{set scroll continuous}

The Metamath command line interface starts off in the \texttt{prompted} mode,
which means that you will be prompted to continue or quit after each
full screen in a long listing.  In \texttt{continuous} mode, long listings will be
scrolled without pausing.

% LaTeX bug? (1) \texttt{\_} puts out different character than
% \texttt{{\char`\_}}
%  = \verb$_$  (2) \texttt{{\char`\_}} puts out garbage in \subsection
%  argument
\subsection{\texttt{set width} Command}\index{\texttt{set
width} command}
Syntax:  \texttt{set width} {\em number}

Metamath assumes the width of your screen is 79 characters (chosen
because the Command Prompt in Windows XP has a wrapping bug at column
80).  If your screen is wider or narrower, this command allows you to
change this default screen width.  A larger width is advantageous for
logging proofs to an output file to be printed on a wide printer.  A
smaller width may be necessary on some terminals; in this case, the
wrapping of the information messages may sometimes seem somewhat
unnatural, however.  In \LaTeX\index{latex@{\LaTeX}!characters per line},
there is normally a maximum of 61 characters per line with typewriter
font.  (The examples in this book were produced with 61 characters per
line.)

\subsection{\texttt{set height} Command}\index{\texttt{set
height} command}
Syntax:  \texttt{set height} {\em number}

Metamath assumes your screen height is 24 lines of characters.  If your
screen is taller or shorter, this command lets you to change the number
of lines at which the display pauses and prompts you to continue.

\subsection{\texttt{beep} Command}\index{\texttt{beep} command}

Syntax:  \texttt{beep}

This command will produce a beep.  By typing it ahead after a
long-running command has started, it will alert you that the command is
finished.  For convenience, \texttt{b} is an abbreviation for
\texttt{beep}.

Note:  If \texttt{b} is typed at the \texttt{MM>} prompt immediately
after the end of a multiple-page display paged with ``\texttt{Press
<return> for more}...'' prompts, then the \texttt{b} will back up to the
previous page rather than perform the \texttt{beep} command.
In that case you must type the unabbreviated \texttt{beep} form
of the command.

\subsection{\texttt{more} Command}\index{\texttt{more} command}

Syntax:  \texttt{more} {\em filename}

This command will display the contents of an {\sc ascii} file on your
screen.  (This command is provided for convenience but is not very
powerful.  See Section~\ref{oscmd} to invoke your operating system's
command to do this, such as the \texttt{more} command in Unix.)

\subsection{Operating System Commands}\index{operating system
command}\label{oscmd}

A line enclosed in single or double quotes will be executed by your
computer's operating system if it has a command line interface.  For
example, on a {\sc vax/vms} system,
\verb/MM> 'dir'/
will print disk directory contents.  Note that this feature will not
work on the Macintosh prior to Mac OS X, which does not have a command
line interface.

For your convenience, the trailing quote is optional.

\subsection{Size Limitations in Metamath}

In general, there are no fixed, predefined limits\index{Metamath!memory
limits} on how many labels, tokens\index{token}, statements, etc.\ that
you may have in a database file.  The Metamath program uses 32-bit
variables (64-bit on 64-bit CPUs) as indices for almost all internal
arrays, which are allocated dynamically as needed.



\section{Reading and Writing Files}

The following commands create new files:  the \texttt{open} commands;
the \texttt{write} commands; the \texttt{/html},
\texttt{/alt{\char`\_}html}, \texttt{/brief{\char`\_}html},
\texttt{/brief{\char`\_}alt{\char`\_}html} qualifiers of \texttt{show
statement}, and \texttt{midi}.  The following commands append to files
previously opened:  the \texttt{/tex} qualifier of \texttt{show proof}
and \texttt{show new{\char`\_}proof}; the \texttt{/tex} and
\texttt{/simple{\char`\_}tex} qualifiers of \texttt{show statement}; the
\texttt{close} commands; and all screen dialog between \texttt{open log}
and \texttt{close log}.

The commands that create new files will not overwrite an existing {\em
filename} but will rename the existing one to {\em
filename}\texttt{{\char`\~}1}.  An existing {\em
filename}\texttt{{\char`\~}1} is renamed {\em
filename}\texttt{{\char`\~}2}, etc.\ up to {\em
filename}\texttt{{\char`\~}9}.  An existing {\em
filename}\texttt{{\char`\~}9} is deleted.  This makes recovery from
mistakes easier but also will clutter up your directory, so occasionally
you may want to clean up (delete) these \texttt{{\char`\~}}$n$ files.


\subsection{\texttt{read} Command}\index{\texttt{read} command}
Syntax:  \texttt{read} {\em file-name} [\texttt{/verify}]

This command will read in a Metamath language source file and any included
files.  Normally it will be the first thing you do when entering Metamath.
Statement syntax is checked, but proof syntax is not checked.
Note that the file name may be enclosed in single or double quotes;
this is useful if the file name contains slashes, as might be the case
under Unix.

If you are getting an ``\texttt{?Expected VERIFY}'' error
when trying to read a Unix file name with slashes, you probably haven't
quoted it.\index{Unix file names}\index{file names!Unix}

If you are prompted for the file name (by pressing {\em return}
 after \texttt{read})
you should {\em not} put quotes around it, even if it is a Unix file name
with slashes.

Optional command qualifier:

    \texttt{/verify} - Verify all proofs as the database is read in.  This
         qualifier will slow down reading in the file.  See \texttt{verify
         proof} for more information on file error-checking.

See also \texttt{erase}.



\subsection{\texttt{write source} Command}\index{\texttt{write source} command}
Syntax:  \texttt{write source} {\em filename}
[\texttt{/rewrap}]
[\texttt{/split}]
% TeX doesn't handle this long line with tt text very well,
% so force a line break here.
[\texttt{/keep\_includes}] {\\}
[\texttt{/no\_versioning}]

This command will write the contents of a Metamath\index{database}
database into a file.\index{source file}

Optional command qualifiers:

\texttt{/rewrap} -
Reformats statements and comments according to the
convention used in the set.mm database.
It unwraps the
lines in the comment before each \texttt{\$a} and \texttt{\$p} statement,
then it
rewraps the line.  You should compare the output to the original
to make sure that the desired effect results; if not, go back to
the original source.  The wrapped line length honors the
\texttt{set width}
parameter currently in effect.  Note:  Text
enclosed in \texttt{<HTML>}...\texttt{</HTML>} tags is not modified by the
\texttt{/rewrap} qualifier.
Proofs themselves are not reformatted;
use \texttt{save proof * / compressed} to do that.
An isolated tilde (\~{}) is always kept on the same line as the following
symbol, so you can find all comment references to a symbol by
searching for \~{} followed by a space and that symbol
(this is useful for finding cross-references).
Incidentally, \texttt{save proof} also honors the \texttt{set width}
parameter currently in effect.

\texttt{/split} - Files included in the source using the expression
\$[ \textit{inclfile} \$] will be
written into separate files instead of being included in a single output
file.  The name of each separately written included file will be the
\textit{inclfile} argument of its inclusion command.

\texttt{/keep\_includes} - If a source file has includes but is written as a
single file by omitting \texttt{/split}, by default the included files will
be deleted (actually just renamed with a \char`\~1 suffix unless
\texttt{/no\_versioning} is specified) to prevent the possibly confusing
source duplication in both the output file and the included file.
The \texttt{/keep\_includes} qualifier will prevent this deletion.

\texttt{/no\_versioning} - Backup files suffixed with \char`\~1 are not created.


\section{Showing Status and Statements}



\subsection{\texttt{show settings} Command}\index{\texttt{show settings} command}
Syntax:  \texttt{show settings}

This command shows the state of various parameters.

\subsection{\texttt{show memory} Command}\index{\texttt{show memory} command}
Syntax:  \texttt{show memory}

This command shows the available memory left.  It is not meaningful
on most modern operating systems,
which have virtual memory.\index{Metamath!memory usage}


\subsection{\texttt{show labels} Command}\index{\texttt{show labels} command}
Syntax:  \texttt{show labels} {\em label-match} [\texttt{/all}]
   [\texttt{/linear}]

This command shows the labels of \texttt{\$a} and \texttt{\$p}
statements that match {\em label-match}.  A \verb$*$ in {label-match}
matches zero or more characters.  For example, \verb$*abc*def$ will match all
labels containing \verb$abc$ and ending with \verb$def$.

Optional command qualifiers:

   \texttt{/all} - Include matches for \texttt{\$e} and \texttt{\$f}
   statement labels.

   \texttt{/linear} - Display only one label per line.  This can be useful for
       building scripts in conjunction with the utilities under the
       \texttt{tools}\index{\texttt{tools} command} command.



\subsection{\texttt{show statement} Command}\index{\texttt{show statement} command}
Syntax:  \texttt{show statement} {\em label-match} [{\em qualifiers} (see below)]

This command provides information about a statement.  Only statements
that have labels (\texttt{\$f}\index{\texttt{\$f} statement},
\texttt{\$e}\index{\texttt{\$e} statement},
\texttt{\$a}\index{\texttt{\$a} statement}, and
\texttt{\$p}\index{\texttt{\$p} statement}) may be specified.
If {\em label-match}
contains wildcard (\verb$*$) characters, all matching statements will be
displayed in the order they occur in the database.

Optional qualifiers (only one qualifier at a time is allowed):

    \texttt{/comment} - This qualifier includes the comment that immediately
       precedes the statement.

    \texttt{/full} - Show complete information about each statement,
       and show all
       statements matching {\em label} (including \texttt{\$e}
       and \texttt{\$f} statements).

    \texttt{/tex} - This qualifier will write the statement information to the
       \LaTeX\ file previously opened with \texttt{open tex}.  See
       Section~\ref{texout}.

    \texttt{/simple{\char`\_}tex} - The same as \texttt{/tex}, except that
       \LaTeX\ macros are not used for formatting equations, allowing easier
       manual edits of the output for slide presentations, etc.

    \texttt{/html}\index{html generation@{\sc html} generation},
       \texttt{/alt{\char`\_}html}, \texttt{/brief{\char`\_}html},
       \texttt{/brief{\char`\_}alt{\char`\_}html} -
       These qualifiers invoke a special mode of
       \texttt{show statement} that
       creates a web page for the statement.  They may not be used with
       any other qualifier.  See Section~\ref{htmlout} or
       \texttt{help html} in the program.


\subsection{\texttt{search} Command}\index{\texttt{search} command}
Syntax:  search {\em label-match}
\texttt{"}{\em symbol-match}{\tt}" [\texttt{/all}] [\texttt{/comments}]
[\texttt{/join}]

This command searches all \texttt{\$a} and \texttt{\$p} statements
matching {\em label-match} for occurrences of {\em symbol-match}.  A
\verb@*@ in {\em label-match} matches any label character.  A \verb@$*@
in {\em symbol-match} matches any sequence of symbols.  The symbols in
{\em symbol-match} must be separated by white space.  The quotes
surrounding {\em symbol-match} may be single or double quotes.  For
example, \texttt{search b}\verb@* "-> $* ch"@ will list all statements
whose labels begin with \texttt{b} and contain the symbols \verb@->@ and
\texttt{ch} surrounding any symbol sequence (including no symbol
sequence).  The wildcards \texttt{?} and \texttt{\$?} are also available
to match individual characters in labels and symbols respectively; see
\texttt{help search} in the Metamath program for details on their usage.

Optional command qualifiers:

    \texttt{/all} - Also search \texttt{\$e} and \texttt{\$f} statements.

    \texttt{/comments} - Search the comment that immediately precedes each
        label-matched statement for {\em symbol-match}.  In this case
        {\em symbol-match} is an arbitrary, non-case-sensitive character
        string.  Quotes around {\em symbol-match} are optional if there
        is no ambiguity.

    \texttt{/join} - In the case of a \texttt{\$a} or \texttt{\$p} statement,
	prepend its \texttt{\$e}
	hypotheses for searching. The
	\texttt{/join} qualifier has no effect in \texttt{/comments} mode.

\section{Displaying and Verifying Proofs}


\subsection{\texttt{show proof} Command}\index{\texttt{show proof} command}
Syntax:  \texttt{show proof} {\em label-match} [{\em qualifiers} (see below)]

This command displays the proof of the specified
\texttt{\$p}\index{\texttt{\$p} statement} statement in various formats.
The {\em label-match} may contain wildcard (\verb@$*@) characters to match
multiple statements.  Without any qualifiers, only the logical steps
will be shown (i.e.\ syntax construction steps will be omitted), in an
indented format.

Most of the time, you will use
    \texttt{show proof} {\em label}
to see just the proof steps corresponding to logical inferences.

Optional command qualifiers:

    \texttt{/essential} - The proof tree is trimmed of all
        \texttt{\$f}\index{\texttt{\$f} statement} hypotheses before
        being displayed.  (This is the default, and it is redundant to
        specify it.)

    \texttt{/all} - the proof tree is not trimmed of all \texttt{\$f} hypotheses before
        being displayed.  \texttt{/essential} and \texttt{/all} are mutually exclusive.

    \texttt{/from{\char`\_}step} {\em step} - The display starts at the specified
        step.  If
        this qualifier is omitted, the display starts at the first step.

    \texttt{/to{\char`\_}step} {\em step} - The display ends at the specified
        step.  If this
        qualifier is omitted, the display ends at the last step.

    \texttt{/tree{\char`\_}depth} {\em number} - Only
         steps at less than the specified proof
        tree depth are displayed.  Sometimes useful for obtaining an overview of
        the proof.

    \texttt{/reverse} - The steps are displayed in reverse order.

    \texttt{/renumber} - When used with \texttt{/essential}, the steps are renumbered
        to correspond only to the essential steps.

    \texttt{/tex} - The proof is converted to \LaTeX\ and\index{latex@{\LaTeX}}
        stored in the file opened
        with \texttt{open tex}.  See Section~\ref{texout} or
        \texttt{help tex} in the program.

    \texttt{/lemmon} - The proof is displayed in a non-indented format known
        as Lemmon style, with explicit previous step number references.
        If this qualifier is omitted, steps are indented in a tree format.

    \texttt{/start{\char`\_}column} {\em number} - Overrides the default column
        (16)
        at which the formula display starts in a Lemmon-style display.  May be
        used only in conjunction with \texttt{/lemmon}.

    \texttt{/normal} - The proof is displayed in normal format suitable for
        inclusion in a Metamath source file.  May not be used with any other
        qualifier.

    \texttt{/compressed} - The proof is displayed in compressed format
        suitable for inclusion in a Metamath source file.  May not be used with
        any other qualifier.

    \texttt{/statement{\char`\_}summary} - Summarizes all statements (like a
        brief \texttt{show statement})
        used by the proof.  It may not be used with any other qualifier
        except \texttt{/essential}.

    \texttt{/detailed{\char`\_}step} {\em step} - Shows the details of what is
        happening at
        a specific proof step.  May not be used with any other qualifier.
        The {\em step} is the step number shown when displaying a
        proof without the \texttt{/renumber} qualifier.


\subsection{\texttt{show usage} Command}\index{\texttt{show usage} command}
Syntax:  \texttt{show usage} {\em label-match} [\texttt{/recursive}]

This command lists the statements whose proofs make direct reference to
the statement specified.

Optional command qualifier:

    \texttt{/recursive} - Also include statements whose proofs ultimately
        depend on the statement specified.



\subsection{\texttt{show trace\_back} Command}\index{\texttt{show
       trace{\char`\_}back} command}
Syntax:  \texttt{show trace{\char`\_}back} {\em label-match} [\texttt{/essential}] [\texttt{/axioms}]
    [\texttt{/tree}] [\texttt{/depth} {\em number}]

This command lists all statements that the proof of the \texttt{\$p}
statement(s) specified by {\em label-match} depends on.

Optional command qualifiers:

    \texttt{/essential} - Restrict the trace-back to \texttt{\$e}
        \index{\texttt{\$e} statement} hypotheses of proof trees.

    \texttt{/axioms} - List only the axioms that the proof ultimately depends on.

    \texttt{/tree} - Display the trace-back in an indented tree format.

    \texttt{/depth} {\em number} - Restrict the \texttt{/tree} trace-back to the
        specified indentation depth.

    \texttt{/count{\char`\_}steps} - Count the number of steps the proof has
       all the way back to axioms.  If \texttt{/essential} is specified,
       expansions of variable-type hypotheses (syntax constructions) are not counted.

\subsection{\texttt{verify proof} Command}\index{\texttt{verify proof} command}
Syntax:  \texttt{verify proof} {\em label-match} [\texttt{/syntax{\char`\_}only}]

This command verifies the proofs of the specified statements.  {\em
label-match} may contain wild card characters (\texttt{*}) to verify
more than one proof; for example \verb/*abc*def/ will match all labels
containing \texttt{abc} and ending with \texttt{def}.
The command \texttt{verify proof *} will verify all proofs in the database.

Optional command qualifier:

    \texttt{/syntax{\char`\_}only} - This qualifier will perform a check of syntax
        and RPN
        stack violations only.  It will not verify that the proof is
        correct.  This qualifier is useful for quickly determining which
        proofs are incomplete (i.e.\ are under development and have \texttt{?}'s
        in them).

{\em Note:} \texttt{read}, followed by \texttt{verify proof *}, will ensure
 the database is free
from errors in the Metamath language but will not check the markup notation
in comments.
You can also check the markup notation by running \texttt{verify markup *}
as discussed in Section~\ref{verifymarkup}; see also the discussion
on generating {\sc HTML} in Section~\ref{htmlout}.

\subsection{\texttt{verify markup} Command}\index{\texttt{verify markup} command}\label{verifymarkup}
Syntax:  \texttt{verify markup} {\em label-match}
[\texttt{/date{\char`\_}skip}]
[\texttt{/top{\char`\_}date{\char`\_}skip}] {\\}
[\texttt{/file{\char`\_}skip}]
[\texttt{/verbose}]

This command checks comment markup and other informal conventions we have
adopted.  It error-checks the latexdef, htmldef, and althtmldef statements
in the \texttt{\$t} statement of a Metamath source file.\index{error checking}
It error-checks any \texttt{`}...\texttt{`},
\texttt{\char`\~}~\textit{label},
and bibliographic markups in statement descriptions.
It checks that
\texttt{\$p} and \texttt{\$a} statements
have the same content when their labels start with
``ax'' and ``ax-'' respectively but are otherwise identical, for example
ax4 and ax-4.
It also verifies the date consistency of ``(Contributed by...),''
``(Revised by...),'' and ``(Proof shortened by...)'' tags in the comment
above each \texttt{\$a} and \texttt{\$p} statement.

Optional command qualifiers:

    \texttt{/date{\char`\_}skip} - This qualifier will
        skip date consistency checking,
        which is usually not required for databases other than
	\texttt{set.mm}.

    \texttt{/top{\char`\_}date{\char`\_}skip} - This qualifier will check date consistency except
        that the version date at the top of the database file will not
        be checked.  Only one of
        \texttt{/date{\char`\_}skip} and
        \texttt{/top{\char`\_}date{\char`\_}skip} may be
        specified.

    \texttt{/file{\char`\_}skip} - This qualifier will skip checks that require
        external files to be present, such as checking GIF existence and
        bibliographic links to mmset.html or equivalent.  It is useful
        for doing a quick check from a directory without these files.

    \texttt{/verbose} - Provides more information.  Currently it provides a list
        of axXXX vs. ax-XXX matches.

\subsection{\texttt{save proof} Command}\index{\texttt{save proof} command}
Syntax:  \texttt{save proof} {\em label-match} [\texttt{/normal}]
   [\texttt{/compressed}]

The \texttt{save proof} command will reformat a proof in one of two formats and
replace the existing proof in the source buffer\index{source
buffer}.  It is useful for
converting between proof formats.  Note that a proof will not be
permanently saved until a \texttt{write source} command is issued.

Optional command qualifiers:

    \texttt{/normal} - The proof is saved in the normal format (i.e., as a
        sequence
        of labels, which is the defined format of the basic Metamath
        language).\index{basic language}  This is the default format that
        is used if a qualifier
        is omitted.

    \texttt{/compressed} - The proof is saved in the compressed format which
        reduces storage requirements for a database.
        See Appendix~\ref{compressed}.




\section{Creating Proofs}\label{pfcommands}\index{Proof Assistant}

Before using the Proof Assistant, you must add a \texttt{\$p} to your
source file (using a text editor) containing the statement you want to
prove.  Its proof should consist of a single \texttt{?}, meaning
``unknown step.''  Example:
\begin{verbatim}
equid $p x = x $= ? $.
\end{verbatim}

To enter the Proof assistant, type \texttt{prove} {\em label}, e.g.
\texttt{prove equid}.  Metamath will respond with the \texttt{MM-PA>}
prompt.

Proofs are created working backwards from the statement being proved,
primarily using a series of \texttt{assign} commands.  A proof is
complete when all steps are assigned to statements and all steps are
unified and completely known.  During the creation of a proof, Metamath
will allow only operations that are legal based on what is known up to
that point.  For example, it will not allow an \texttt{assign} of a
statement that cannot be unified with the unknown proof step being
assigned.

{\em Important:}
The Proof Assistant is
{\em not} a tool to help you discover proofs.  It is just a tool to help
you add them to the database.  For a tutorial read
Section~\ref{frstprf}.
To practice using the Proof Assistant, you may
want to \texttt{prove} an existing theorem, then delete all steps with
\texttt{delete all}, then re-create it with the Proof Assistant while
looking at its proof display (before deletion).
You might want to figure out your first few proofs completely
and write them down by hand, before using the Proof Assistant, though
not everyone finds that effective.

{\em Important:}
The \texttt{undo} command if very helpful when entering a proof, because
it allows you to undo a previously-entered step.
In addition, we suggest that you
keep track of your work with a log file (\texttt{open
log}) and save it frequently (\texttt{save new{\char`\_}proof},
\texttt{write source}).
You can use \texttt{delete} to reverse an \texttt{assign}.
You can also do \texttt{delete floating{\char`\_}hypotheses}, then
\texttt{initialize all}, then \texttt{unify all /interactive} to
reinitialize bad unifications made accidentally or by bad
\texttt{assign}s.  You cannot reverse a \texttt{delete} except by
a relevant \texttt{undo} or using
\texttt{exit /force} then reentering the Proof Assistant to recover from
the last \texttt{save new{\char`\_}proof}.

The following commands are available in the Proof Assistant (at the
\texttt{MM-PA>} prompt) to help you create your proof.  See the
individual commands for more detail.

\begin{itemize}
\item[]
    \texttt{show new{\char`\_}proof} [\texttt{/all},...] - Displays the
        proof in progress.  You will use this command a lot; see \texttt{help
        show new{\char`\_}proof} to become familiar with its qualifiers.  The
        qualifiers \texttt{/unknown} and \texttt{/not{\char`\_}unified} are
        useful for seeing the work remaining to be done.  The combination
        \texttt{/all/unknown} is useful for identifying dummy variables that must be
        assigned, or attempts to use illegal syntax, when \texttt{improve all}
        is unable to complete the syntax constructions.  Unknown variables are
        shown as \texttt{\$1}, \texttt{\$2},...
\item[]
    \texttt{assign} {\em step} {\em label} - Assigns an unknown {\em step}
        number with the statement
        specified by {\em label}.
\item[]
    \texttt{let variable} {\em variable}
        \texttt{= "}{\em symbol sequence}\texttt{"}
          - Forces a symbol
        sequence to replace an unknown variable (such as \texttt{\$1}) in a proof.
        It is useful
        for helping difficult unifications, and it is necessary when you have
        dummy variables that eventually must be assigned a name.
\item[]
    \texttt{let step} {\em step} \texttt{= "}{\em symbol sequence}\texttt{"} -
          Forces a symbol sequence
        to replace the contents of a proof step, provided it can be
        unified with the existing step contents.  (I rarely use this.)
\item[]
    \texttt{unify step} {\em step} (or \texttt{unify all}) - Unifies
        the source and target of
        a step.  If you specify a specific step, you will be prompted
        to select among the unifications that are possible.  If you
        specify \texttt{all}, all steps with unique unifications, but only
        those steps, will be
        unified.  \texttt{unify all /interactive} goes through all non-unified
        steps.
\item[]
    \texttt{initialize} {\em step} (or \texttt{all}) - De-unifies the target and source of
        a step (or all steps), as well as the hypotheses of the source,
        and makes all variables in the source unknown.  Useful to recover from
        an \texttt{assign} or \texttt{let} mistake that
        resulted in incorrect unifications.
\item[]
    \texttt{delete} {\em step} (or \texttt{all} or \texttt{floating{\char`\_}hypotheses}) -
        Deletes the specified
        step(s).  \texttt{delete floating{\char`\_}hypotheses}, then \texttt{initialize all}, then
        \texttt{unify all /interactive} is useful for recovering from mistakes
        where incorrect unifications assigned wrong math symbol strings to
        variables.
\item[]
    \texttt{improve} {\em step} (or \texttt{all}) -
      Automatically creates a proof for steps (with no unknown variables)
      whose proof requires no statements with \texttt{\$e} hypotheses.  Useful
      for filling in proofs of \texttt{\$f} hypotheses.  The \texttt{/depth}
      qualifier will also try statements whose \texttt{\$e} hypotheses contain
      no new variables.  {\em Warning:} Save your work (with \texttt{save
      new{\char`\_}proof} then \texttt{write source}) before using
      \texttt{/depth = 2} or greater, since the search time grows
      exponentially and may never terminate in a reasonable time, and you
      cannot interrupt the search.  I have found that it is rare for
      \texttt{/depth = 3} or greater to be useful.
 \item[]
    \texttt{save new{\char`\_}proof} - Saves the proof in progress in the program's
        internal database buffer.  To save it permanently into the database file,
        use \texttt{write source} after
        \texttt{save new{\char`\_}proof}.  To revert to the last
        \texttt{save new{\char`\_}proof},
        \texttt{exit /force} from the Proof Assistant then re-enter the Proof
        Assistant.
 \item[]
    \texttt{match step} {\em step} (or \texttt{match all}) - Shows what
        statements are
        possibilities for the \texttt{assign} statement. (This command
        is not very
        useful in its present form and hopefully will be improved
        eventually.  In the meantime, use the \texttt{search} statement for
        candidates matching specific math token combinations.)
 \item[]
 \texttt{minimize{\char`\_}with}\index{\texttt{minimize{\char`\_}with} command}
% 3/10/07 Note: line-breaking the above results in duplicate index entries
     - After a proof is complete, this command will attempt
        to match other database theorems to the proof to see if the proof
        size can be reduced as a result.  See \texttt{help
        minimize{\char`\_}with} in the
        Metamath program for its usage.
 \item[]
 \texttt{undo}\index{\texttt{undo} command}
    - Undo the effect of a proof-changing command (all but the
      \texttt{show} and \texttt{save} commands above).
 \item[]
 \texttt{redo}\index{\texttt{redo} command}
    - Reverse the previous \texttt{undo}.
\end{itemize}

The following commands set parameters that may be relevant to your proof.
Consult the individual \texttt{help set}... commands.
\begin{itemize}
   \item[] \texttt{set unification{\char`\_}timeout}
 \item[]
    \texttt{set search{\char`\_}limit}
  \item[]
    \texttt{set empty{\char`\_}substitution} - note that default is \texttt{off}
\end{itemize}

Type \texttt{exit} to exit the \texttt{MM-PA>}
 prompt and get back to the \texttt{MM>} prompt.
Another \texttt{exit} will then get you out of Metamath.



\subsection{\texttt{prove} Command}\index{\texttt{prove} command}
Syntax:  \texttt{prove} {\em label}

This command will enter the Proof Assistant\index{Proof Assistant}, which will
allow you to create or edit the proof of the specified statement.
The command-line prompt will change from \texttt{MM>} to \texttt{MM-PA>}.

Note:  In the present version (0.177) of
Metamath\index{Metamath!limitations of version 0.177}, the Proof
Assistant does not verify that \texttt{\$d}\index{\texttt{\$d}
statement} restrictions are met as a proof is being built.  After you
have completed a proof, you should type \texttt{save new{\char`\_}proof}
followed by \texttt{verify proof} {\em label} (where {\em label} is the
statement you are proving with the \texttt{prove} command) to verify the
\texttt{\$d} restrictions.

See also: \texttt{exit}

\subsection{\texttt{set unification\_timeout} Command}\index{\texttt{set
unification{\char`\_}timeout} command}
Syntax:  \texttt{set unification{\char`\_}timeout} {\em number}

(This command is available outside the Proof Assistant but affects the
Proof Assistant\index{Proof Assistant} only.)

Sometimes the Proof Assistant will inform you that a unification
time-out occurred.  This may happen when you try to \texttt{unify}
formulas with many temporary variables\index{temporary variable}
(\texttt{\$1}, \texttt{\$2}, etc.), since the time to compute all possible
unifications may grow exponentially with the number of variables.  If
you want Metamath to try harder (and you're willing to wait longer) you
may increase this parameter.  \texttt{show settings} will show you the
current value.

\subsection{\texttt{set empty\_substitution} Command}\index{\texttt{set
empty{\char`\_}substitution} command}
% These long names can't break well in narrow mode, and even "sloppy"
% is not enough. Work around this by not demanding justification.
\begin{flushleft}
Syntax:  \texttt{set empty{\char`\_}substitution on} or \texttt{set
empty{\char`\_}substitution off}
\end{flushleft}

(This command is available outside the Proof Assistant but affects the
Proof Assistant\index{Proof Assistant} only.)

The Metamath language allows variables to be
substituted\index{substitution!variable}\index{variable substitution}
with empty symbol sequences\index{empty substitution}.  However, in many
formal systems\index{formal system} this will never happen in a valid
proof.  Allowing for this possibility increases the likelihood of
ambiguous unifications\index{ambiguous
unification}\index{unification!ambiguous} during proof creation.
The default is that
empty substitutions are not allowed; for formal systems requiring them,
you must \texttt{set empty{\char`\_}substitution on}.
(An example where this must be \texttt{on}
would be a system that implements a Deduction Rule and in
which deductions from empty assumption lists would be permissible.  The
MIU-system\index{MIU-system} described in Appendix~\ref{MIU} is another
example.)
Note that empty substitutions are
always permissible in proof verification (VERIFY PROOF...) outside the
Proof Assistant.  (See the MIU system in the Metamath book for an example
of a system needing empty substitutions; another example would be a
system that implements a Deduction Rule and in which deductions from
empty assumption lists would be permissible.)

It is better to leave this \texttt{off} when working with \texttt{set.mm}.
Note that this command does not affect the way proofs are verified with
the \texttt{verify proof} command.  Outside of the Proof Assistant,
substitution of empty sequences for math symbols is always allowed.

\subsection{\texttt{set search\_limit} Command}\index{\texttt{set
search{\char`\_}limit} command} Syntax:  \texttt{set search{\char`\_}limit} {\em
number}

(This command is available outside the Proof Assistant but affects the
Proof Assistant\index{Proof Assistant} only.)

This command sets a parameter that determines when the \texttt{improve} command
in Proof Assistant mode gives up.  If you want \texttt{improve} to search harder,
you may increase it.  The \texttt{show settings} command tells you its current
value.


\subsection{\texttt{show new\_proof} Command}\index{\texttt{show
new{\char`\_}proof} command}
Syntax:  \texttt{show new{\char`\_}proof} [{\em
qualifiers} (see below)]

This command (available only in Proof Assistant mode) displays the proof
in progress.  It is identical to the \texttt{show proof} command, except that
there is no statement argument (since it is the statement being proved) and
the following qualifiers are not available:

    \texttt{/statement{\char`\_}summary}

    \texttt{/detailed{\char`\_}step}

Also, the following additional qualifiers are available:

    \texttt{/unknown} - Shows only steps that have no statement assigned.

    \texttt{/not{\char`\_}unified} - Shows only steps that have not been unified.

Note that \texttt{/essential}, \texttt{/depth}, \texttt{/unknown}, and
\texttt{/not{\char`\_}unified} may be
used in any combination; each of them effectively filters out additional
steps from the proof display.

See also:  \texttt{show proof}






\subsection{\texttt{assign} Command}\index{\texttt{assign} command}
Syntax:   \texttt{assign} {\em step} {\em label} [\texttt{/no{\char`\_}unify}]

   and:   \texttt{assign first} {\em label}

   and:   \texttt{assign last} {\em label}


This command, available in the Proof Assistant only, assigns an unknown
step (one with \texttt{?} in the \texttt{show new{\char`\_}proof}
listing) with the statement specified by {\em label}.  The assignment
will not be allowed if the statement cannot be unified with the step.

If \texttt{last} is specified instead of {\em step} number, the last
step that is shown by \texttt{show new{\char`\_}proof /unknown} will be
used.  This can be useful for building a proof with a command file (see
\texttt{help submit}).  It also makes building proofs faster when you know
the assignment for the last step.

If \texttt{first} is specified instead of {\em step} number, the first
step that is shown by \texttt{show new{\char`\_}proof /unknown} will be
used.

If {\em step} is zero or negative, the -{\em step}th from last unknown
step, as shown by \texttt{show new{\char`\_}proof /unknown}, will be
used.  \texttt{assign -1} {\em label} will assign the penultimate
unknown step, \texttt{assign -2} {\em label} the antepenultimate, and
\texttt{assign 0} {\em label} is the same as \texttt{assign last} {\em
label}.

Optional command qualifier:

    \texttt{/no{\char`\_}unify} - do not prompt user to select a unification if there is
        more than one possibility.  This is useful for noninteractive
        command files.  Later, the user can \texttt{unify all /interactive}.
        (The assignment will still be automatically unified if there is only
        one possibility and will be refused if unification is not possible.)



\subsection{\texttt{match} Command}\index{\texttt{match} command}
Syntax:  \texttt{match step} {\em step} [\texttt{/max{\char`\_}essential{\char`\_}hyp}
{\em number}]

    and:  \texttt{match all} [\texttt{/essential}]
          [\texttt{/max{\char`\_}essential{\char`\_}hyp} {\em number}]

This command, available in the Proof Assistant only, shows what
statements can be unified with the specified step(s).  {\em Note:} In
its current form, this command is not very useful because of the large
number of matches it reports.
It may be enhanced in the future.  In the meantime, the \texttt{search}
command can often provide finer control over locating theorems of interest.

Optional command qualifiers:

    \texttt{/max{\char`\_}essential{\char`\_}hyp} {\em number} - filters out
        of the list any statements
        with more than the specified number of
        \texttt{\$e}\index{\texttt{\$e} statement} hypotheses.

    \texttt{/essential{\char`\_}only} - in the \texttt{match all} statement, only
        the steps that
        would be listed in the \texttt{show new{\char`\_}proof /essential} display are
        matched.



\subsection{\texttt{let} Command}\index{\texttt{let} command}
Syntax: \texttt{let variable} {\em variable} = \verb/"/{\em symbol-sequence}\verb/"/

  and: \texttt{let step} {\em step} = \verb/"/{\em symbol-sequence}\verb/"/

These commands, available in the Proof Assistant\index{Proof Assistant}
only, assign a temporary variable\index{temporary variable} or unknown
step with a specific symbol sequence.  They are useful in the middle of
creating a proof, when you know what should be in the proof step but the
unification algorithm doesn't yet have enough information to completely
specify the temporary variables.  A ``temporary variable'' is one that
has the form \texttt{\$}{\em nn} in the proof display, such as
\texttt{\$1}, \texttt{\$2}, etc.  The {\em symbol-sequence} may contain
other unknown variables if desired.  Examples:

    \verb/let variable $32 = "A = B"/

    \verb/let variable $32 = "A = $35"/

    \verb/let step 10 = '|- x = x'/

    \verb/let step -2 = "|- ( $7 = ph )"/

Any symbol sequence will be accepted for the \texttt{let variable}
command.  Only those symbol sequences that can be unified with the step
will be accepted for \texttt{let step}.

The \texttt{let} commands ``zap'' the proof with information that can
only be verified when the proof is built up further.  If you make an
error, the command sequence \texttt{delete
floating{\char`\_}hypotheses}, \texttt{initialize all}, and
\texttt{unify all /interactive} will undo a bad \texttt{let} assignment.

If {\em step} is zero or negative, the -{\em step}th from last unknown
step, as shown by \texttt{show new{\char`\_}proof /unknown}, will be
used.  The command \texttt{let step 0} = \verb/"/{\em
symbol-sequence}\verb/"/ will use the last unknown step, \texttt{let
step -1} = \verb/"/{\em symbol-sequence}\verb/"/ the penultimate, etc.
If {\em step} is positive, \texttt{let step} may be used to assign known
(in the sense of having previously been assigned a label with
\texttt{assign}) as well as unknown steps.

Either single or double quotes can surround the {\em symbol-sequence} as
long as they are different from any quotes inside a {\em
symbol-sequence}.  If {\em symbol-sequence} contains both kinds of
quotes, see the instructions at the end of \texttt{help let} in the
Metamath program.


\subsection{\texttt{unify} Command}\index{\texttt{unify} command}
Syntax:  \texttt{unify step} {\em step}

      and:   \texttt{unify all} [\texttt{/interactive}]

These commands, available in the Proof Assistant only, unify the source
and target of the specified step(s). If you specify a specific step, you
will be prompted to select among the unifications that are possible.  If
you specify \texttt{all}, only those steps with unique unifications will be
unified.

Optional command qualifier for \texttt{unify all}:

    \texttt{/interactive} - You will be prompted to select among the
        unifications
        that are possible for any steps that do not have unique
        unifications.  (Otherwise \texttt{unify all} will bypass these.)

See also \texttt{set unification{\char`\_}timeout}.  The default is
100000, but increasing it to 1000000 can help difficult cases.  Manually
assigning some or all of the unknown variables with the \texttt{let
variable} command also helps difficult cases.



\subsection{\texttt{initialize} Command}\index{\texttt{initialize} command}
Syntax:  \texttt{initialize step} {\em step}

    and: \texttt{initialize all}

These commands, available in the Proof Assistant\index{Proof Assistant}
only, ``de-unify'' the target and source of a step (or all steps), as
well as the hypotheses of the source, and makes all variables in the
source and the source's hypotheses unknown.  This command is useful to
help recover from incorrect unifications that resulted from an incorrect
\texttt{assign}, \texttt{let}, or unification choice.  Part or all of
the command sequence \texttt{delete floating{\char`\_}hypotheses},
\texttt{initialize all}, and \texttt{unify all /interactive} will recover
from incorrect unifications.

See also:  \texttt{unify} and \texttt{delete}



\subsection{\texttt{delete} Command}\index{\texttt{delete} command}
Syntax:  \texttt{delete step} {\em step}

   and:      \texttt{delete all} -- {\em Warning: dangerous!}

   and:      \texttt{delete floating{\char`\_}hypotheses}

These commands are available in the Proof Assistant only.  The
\texttt{delete step} command deletes the proof tree section that
branches off of the specified step and makes the step become unknown.
\texttt{delete all} is equivalent to \texttt{delete step} {\em step}
where {\em step} is the last step in the proof (i.e.\ the beginning of
the proof tree).

In most cases the \texttt{undo} command is the best way to undo
a previous step.
An alternative is to salvage your last \texttt{save
new{\char`\_}proof} by exiting and reentering the Proof Assistant.
For this to work, keep a log file open to record your work
and to do \texttt{save new{\char`\_}proof} frequently, especially before
\texttt{delete}.

\texttt{delete floating{\char`\_}hypotheses} will delete all sections of
the proof that branch off of \texttt{\$f}\index{\texttt{\$f} statement}
statements.  It is sometimes useful to do this before an
\texttt{initialize} command to recover from an error.  Note that once a
proof step with a \texttt{\$f} hypothesis as the target is completely
known, the \texttt{improve} command can usually fill in the proof for
that step.  Unlike the deletion of logical steps, \texttt{delete
floating{\char`\_}hypotheses} is a relatively safe command that is
usually easy to recover from.



\subsection{\texttt{improve} Command}\index{\texttt{improve} command}
\label{improve}
Syntax:  \texttt{improve} {\em step} [\texttt{/depth} {\em number}]
                                               [\texttt{/no{\char`\_}distinct}]

   and:   \texttt{improve first} [\texttt{/depth} {\em number}]
                                              [\texttt{/no{\char`\_}distinct}]

   and:   \texttt{improve last} [\texttt{/depth} {\em number}]
                                              [\texttt{/no{\char`\_}distinct}]

   and:   \texttt{improve all} [\texttt{/depth} {\em number}]
                                              [\texttt{/no{\char`\_}distinct}]

These commands, available in the Proof Assistant\index{Proof Assistant}
only, try to find proofs automatically for unknown steps whose symbol
sequences are completely known.  They are primarily useful for filling in
proofs of \texttt{\$f}\index{\texttt{\$f} statement} hypotheses.  The
search will be restricted to statements having no
\texttt{\$e}\index{\texttt{\$e} statement} hypotheses.

\begin{sloppypar} % narrow
Note:  If memory is limited, \texttt{improve all} on a large proof may
overflow memory.  If you use \texttt{set unification{\char`\_}timeout 1}
before \texttt{improve all}, there will usually be sufficient
improvement to easily recover and completely \texttt{improve} the proof
later on a larger computer.  Warning:  Once memory has overflowed, there
is no recovery.  If in doubt, save the intermediate proof (\texttt{save
new{\char`\_}proof} then \texttt{write source}) before \texttt{improve
all}.
\end{sloppypar}

If \texttt{last} is specified instead of {\em step} number, the last
step that is shown by \texttt{show new{\char`\_}proof /unknown} will be
used.

If \texttt{first} is specified instead of {\em step} number, the first
step that is shown by \texttt{show new{\char`\_}proof /unknown} will be
used.

If {\em step} is zero or negative, the -{\em step}th from last unknown
step, as shown by \texttt{show new{\char`\_}proof /unknown}, will be
used.  \texttt{improve -1} will use the penultimate
unknown step, \texttt{improve -2} {\em label} the antepenultimate, and
\texttt{improve 0} is the same as \texttt{improve last}.

Optional command qualifier:

    \texttt{/depth} {\em number} - This qualifier will cause the search
        to include
        statements with \texttt{\$e} hypotheses (but no new variables in
        the \texttt{\$e}
        hypotheses), provided that the backtracking has not exceeded the
        specified depth. {\em Warning:}  Try \texttt{/depth 1},
        then \texttt{2}, then \texttt{3}, etc.
        in sequence because of possible exponential blowups.  Save your
        work before trying \texttt{/depth} greater than \texttt{1}!

    \texttt{/no{\char`\_}distinct} - Skip trial statements that have
        \texttt{\$d}\index{\texttt{\$d} statement} requirements.
        This qualifier will prevent assignments that might violate \texttt{\$d}
        requirements but it also could miss possible legal assignments.

See also: \texttt{set search{\char`\_}limit}

\subsection{\texttt{save new\_proof} Command}\index{\texttt{save
new{\char`\_}proof} command}
Syntax:  \texttt{save new{\char`\_}proof} {\em label} [\texttt{/normal}]
   [\texttt{/compressed}]

The \texttt{save new{\char`\_}proof} command is available in the Proof
Assistant only.  It saves the proof in progress in the source
buffer\index{source buffer}.  \texttt{save new{\char`\_}proof} may be
used to save a completed proof, or it may be used to save a proof in
progress in order to work on it later.  If an incomplete proof is saved,
any user assignments with \texttt{let step} or \texttt{let variable}
will be lost, as will any ambiguous unifications\index{ambiguous
unification}\index{unification!ambiguous} that were resolved manually.
To help make recovery easier, it can be helpful to \texttt{improve all}
before \texttt{save new{\char`\_}proof} so that the incomplete proof
will have as much information as possible.

Note that the proof will not be permanently saved until a \texttt{write
source} command is issued.

Optional command qualifiers:

    \texttt{/normal} - The proof is saved in the normal format (i.e., as a
        sequence of labels, which is the defined format of the basic Metamath
        language).\index{basic language}  This is the default format that
        is used if a qualifier is omitted.

    \texttt{/compressed} - The proof is saved in the compressed format, which
        reduces storage requirements for a database.
        (See Appendix~\ref{compressed}.)


\section{Creating \LaTeX\ Output}\label{texout}\index{latex@{\LaTeX}}

You can generate \LaTeX\ output given the
information in a database.
The database must already include the necessary typesetting information
(see section \ref{tcomment} for how to provide this information).

The \texttt{show statement} and \texttt{show proof} commands each have a
special \texttt{/tex} command qualifier that produces \LaTeX\ output.
(The \texttt{show statement} command also has the
\texttt{/simple{\char`\_}tex} qualifier for output that is easier to
edit by hand.)  Before you can use them, you must open a \LaTeX\ file to
which to send their output.  A typical complete session will use this
sequence of Metamath commands:

\begin{verbatim}
read set.mm
open tex example.tex
show statement a1i /tex
show proof a1i /all/lemmon/renumber/tex
show statement uneq2 /tex
show proof uneq2 /all/lemmon/renumber/tex
close tex
\end{verbatim}

See Section~\ref{mathcomments} for information on comment markup and
Appendix~\ref{ASCII} for information on how math symbol translation is
specified.

To format and print the \LaTeX\ source, you will need the \LaTeX\
program, which is standard on most Linux installations and available for
Windows.  On Linux, in order to create a {\sc pdf} file, you will
typically type at the shell prompt
\begin{verbatim}
$ pdflatex example.tex
\end{verbatim}

\subsection{\texttt{open tex} Command}\index{\texttt{open tex} command}
Syntax:  \texttt{open tex} {\em file-name} [\texttt{/no{\char`\_}header}]

This command opens a file for writing \LaTeX\
source\index{latex@{\LaTeX}} and writes a \LaTeX\ header to the file.
\LaTeX\ source can be written with the \texttt{show proof}, \texttt{show
new{\char`\_}proof}, and \texttt{show statement} commands using the
\texttt{/tex} qualifier.

The mapping to \LaTeX\ symbols is defined in a special comment
containing a \texttt{\$t} token, described in Appendix~\ref{ASCII}.

There is an optional command qualifier:

    \texttt{/no{\char`\_}header} - This qualifier prevents a standard
        \LaTeX\ header and trailer
        from being included with the output \LaTeX\ code.


\subsection{\texttt{close tex} Command}\index{\texttt{close tex} command}
Syntax:  \texttt{close tex}

This command writes a trailer to any \LaTeX\ file\index{latex@{\LaTeX}}
that was opened with \texttt{open tex} (unless
\texttt{/no{\char`\_}header} was used with \texttt{open tex}) and closes
the \LaTeX\ file.


\section{Creating {\sc HTML} Output}\label{htmlout}

You can generate {\sc html} web pages given the
information in a database.
The database must already include the necessary typesetting information
(see section \ref{tcomment} for how to provide this information).
The ability to produce {\sc html} web pages was added in Metamath version
0.07.30.

To create an {\sc html} output file(s) for \texttt{\$a} or \texttt{\$p}
statement(s), use
\begin{quote}
    \texttt{show statement} {\em label-match} \texttt{/html}
\end{quote}
The output file will be named {\em label-match}\texttt{.html}
for each match.  When {\em
label-match} has wildcard (\texttt{*}) characters, all statements with
matching labels will have {\sc html} files produced for them.  Also,
when {\em label-match} has a wildcard (\texttt{*}) character, two additional
files, \texttt{mmdefinitions.html} and \texttt{mmascii.html} will be
produced.  To produce {\em only} these two additional files, you can use
\texttt{?*}, which will not match any statement label, in place of {\em
label-match}.

There are three other qualifiers for \texttt{show statement} that also
generate {\sc HTML} code.  These are \texttt{/alt{\char`\_}html},
\texttt{/brief{\char`\_}html}, and
\texttt{/brief{\char`\_}alt{\char`\_}html}, and are described in the
next section.

The command
\begin{quote}
    \texttt{show statement} {\em label-match} \texttt{/alt{\char`\_}html}
\end{quote}
does the same as \texttt{show statement} {\em label-match} \texttt{/html},
except that the {\sc html} code for the symbols is taken from
\texttt{althtmldef} statements instead of \texttt{htmldef} statements in
the \texttt{\$t} comment.

The command
\begin{verbatim}
show statement * /brief_html
\end{verbatim}
invokes a special mode that just produces definition and theorem lists
accompanied by their symbol strings, in a format suitable for copying and
pasting into another web page (such as the tutorial pages on the
Metamath web site).

Finally, the command
\begin{verbatim}
show statement * /brief_alt_html
\end{verbatim}
does the same as \texttt{show statement * / brief{\char`\_}html}
for the alternate {\sc html}
symbol representation.

A statement's comment can include a special notation that provides a
certain amount of control over the {\sc HTML} version of the comment.  See
Section~\ref{mathcomments} (p.~\pageref{mathcomments}) for the comment
markup features.

The \texttt{write theorem{\char`\_}list} and \texttt{write bibliography}
commands, which are described below, provide as a side effect complete
error checking for all of the features described in this
section.\index{error checking}

\subsection{\texttt{write theorem\_list}
Command}\index{\texttt{write theorem{\char`\_}list} command}
Syntax:  \texttt{write theorem{\char`\_}list}
[\texttt{/theorems{\char`\_}per{\char`\_}page} {\em number}]

This command writes a list of all of the \texttt{\$a} and \texttt{\$p}
statements in the database into a web page file
 called \texttt{mmtheorems.html}.
When additional files are needed, they are called
\texttt{mmtheorems2.html}, \texttt{mmtheorems3.html}, etc.

Optional command qualifier:

    \texttt{/theorems{\char`\_}per{\char`\_}page} {\em number} -
 This qualifier specifies the number of statements to
        write per web page.  The default is 100.

{\em Note:} In version 0.177\index{Metamath!limitations of version
0.177} of Metamath, the ``Nearby theorems'' links on the individual
web pages presuppose 100 theorems per page when linking to the theorem
list pages.  Therefore the \texttt{/theorems{\char`\_}per{\char`\_}page}
qualifier, if it specifies a number other than 100, will cause the
individual web pages to be out of sync and should not be used to
generate the main theorem list for the web site.  This may be
fixed in a future version.


\subsection{\texttt{write bibliography}\label{wrbib}
Command}\index{\texttt{write bibliography} command}
Syntax:  \texttt{write bibliography} {\em filename}

This command reads an existing {\sc html} bibliographic cross-reference
file, normally called \texttt{mmbiblio.html}, and updates it per the
bibliographic links in the database comments.  The file is updated
between the {\sc html} comment lines \texttt{<!--
{\char`\#}START{\char`\#} -->} and \texttt{<!-- {\char`\#}END{\char`\#}
-->}.  The original input file is renamed to {\em
filename}\texttt{{\char`\~}1}.

A bibliographic reference is indicated with the reference name
in brackets, such as  \texttt{Theorem 3.1 of
[Monk] p.\ 22}.
See Section~\ref{htmlout} (p.~\pageref{htmlout}) for
syntax details.


\subsection{\texttt{write recent\_additions}
Command}\index{\texttt{write recent{\char`\_}additions} command}
Syntax:  \texttt{write recent{\char`\_}additions} {\em filename}
[\texttt{/limit} {\em number}]

This command reads an existing ``Recent Additions'' {\sc html} file,
normally called \texttt{mmrecent.html}, and updates it with the
descriptions of the most recently added theorems to the database.
 The file is updated between
the {\sc html} comment lines \texttt{<!-- {\char`\#}START{\char`\#} -->}
and \texttt{<!-- {\char`\#}END{\char`\#} -->}.  The original input file
is renamed to {\em filename}\texttt{{\char`\~}1}.

Optional command qualifier:

    \texttt{/limit} {\em number} -
 This qualifier specifies the number of most recent theorems to
   write to the output file.  The default is 100.


\section{Text File Utilities}

\subsection{\texttt{tools} Command}\index{\texttt{tools} command}
Syntax:  \texttt{tools}

This command invokes an easy-to-use, general purpose utility for
manipulating the contents of {\sc ascii} text files.  Upon typing
\texttt{tools}, the command-line prompt will change to \texttt{TOOLS>}
until you type \texttt{exit}.  The \texttt{tools} commands can be used
to perform simple, global edits on an input/output file,
such as making a character string substitution on each line, adding a
string to each line, and so on.  A typical use of this utility is
to build a \texttt{submit} input file to perform a common operation on a
list of statements obtained from \texttt{show label} or \texttt{show
usage}.

The actions of most of the \texttt{tools} commands can also be
performed with equivalent (and more powerful) Unix shell commands, and
some users may find those more efficient.  But for Windows users or
users not comfortable with Unix, \texttt{tools} provides an
easy-to-learn alternative that is adequate for most of the
script-building tasks needed to use the Metamath program effectively.

\subsection{\texttt{help} Command (in \texttt{tools})}
Syntax:  \texttt{help}

The \texttt{help} command lists the commands available in the
\texttt{tools} utility, along with a brief description.  Each command,
in turn, has its own help, such as \texttt{help add}.  As with
Metamath's \texttt{MM>} prompt, a complete command can be entered at
once, or just the command word can be typed, causing the program to
prompt for each argument.

\vskip 1ex
\noindent Line-by-line editing commands:

  \texttt{add} - Add a specified string to each line in a file.

  \texttt{clean} - Trim spaces and tabs on each line in a file; convert
         characters.

  \texttt{delete} - Delete a section of each line in a file.

  \texttt{insert} - Insert a string at a specified column in each line of
        a file.

  \texttt{substitute} - Make a simple substitution on each line of the file.

  \texttt{tag} - Like \texttt{add}, but restricted to a range of lines.

  \texttt{swap} - Swap the two halves of each line in a file.

\vskip 1ex
\noindent Other file-processing commands:

  \texttt{break} - Break up (tokenize) a file into a list of tokens (one per
        line).

  \texttt{build} - Build a file with multiple tokens per line from a list.

  \texttt{count} - Count the occurrences in a file of a specified string.

  \texttt{number} - Create a list of numbers.

  \texttt{parallel} - Put two files in parallel.

  \texttt{reverse} - Reverse the order of the lines in a file.

  \texttt{right} - Right-justify lines in a file (useful before sorting
         numbers).

%  \texttt{tag} - Tag edit updates in a program for revision control.

  \texttt{sort} - Sort the lines in a file with key starting at
         specified string.

  \texttt{match} - Extract lines containing (or not) a specified string.

  \texttt{unduplicate} - Eliminate duplicate occurrences of lines in a file.

  \texttt{duplicate} - Extract first occurrence of any line occurring
         more than

   \ \ \    once in a file, discarding lines occurring exactly once.

  \texttt{unique} - Extract lines occurring exactly once in a file.

  \texttt{type} (10 lines) - Display the first few lines in a file.
                  Similar to Unix \texttt{head}.

  \texttt{copy} - Similar to Unix \texttt{cat} but safe (same input
         and output file allowed).

  \texttt{submit} - Run a script containing \texttt{tools} commands.

\vskip 1ex

\noindent Note:
  \texttt{unduplicate}, \texttt{duplicate}, and \texttt{unique} also
 sort the lines as a side effect.


\subsection{Using \texttt{tools} to Build Metamath \texttt{submit}
Scripts}

The \texttt{break} command is typically used to break up a series of
statement labels, such as the output of Metamath's \texttt{show usage},
into one label per line.  The other \texttt{tools} commands can then be
used to add strings before and after each statement label to specify
commands to be performed on the statement.  The \texttt{parallel}
command is useful when a statement label must be mentioned more than
once on a line.

Very often a \texttt{submit} script for Metamath will require multiple
command lines for each statement being processed.  For example, you may
want to enter the Proof Assistant, \texttt{minimize{\char`\_}with} your
latest theorem, \texttt{save} the new proof, and \texttt{exit} the Proof
Assistant.  To accomplish this, you can build a file with these four
commands for each statement on a single line, separating each command
with a designated character such as \texttt{@}.  Then at the end you can
\texttt{substitute} each \texttt{@} with \texttt{{\char`\\}n} to break
up the lines into individual command lines (see \texttt{help
substitute}).


\subsection{Example of a \texttt{tools} Session}

To give you a quick feel for the \texttt{tools} utility, we show a
simple session where we create a file \texttt{n.txt} with 3 lines, add
strings before and after each line, and display the lines on the screen.
You can experiment with the various commands to gain experience with the
\texttt{tools} utility.

\begin{verbatim}
MM> tools
Entering the Text Tools utilities.
Type HELP for help, EXIT to exit.
TOOLS> number
Output file <n.tmp>? n.txt
First number <1>?
Last number <10>? 3
Increment <1>?
TOOLS> add
Input/output file? n.txt
String to add to beginning of each line <>? This is line
String to add to end of each line <>? .
The file n.txt has 3 lines; 3 were changed.
First change is on line 1:
This is line 1.
TOOLS> type n.txt
This is line 1.
This is line 2.
This is line 3.
TOOLS> exit
Exiting the Text Tools.
Type EXIT again to exit Metamath.
MM>
\end{verbatim}



\appendix
\chapter{Sample Representations}
\label{ASCII}

This Appendix provides a sample of {\sc ASCII} representations,
their corresponding traditional mathematical symbols,
and a discussion of their meanings
in the \texttt{set.mm} database.
These are provided in order of appearance.
This is only a partial list, and new definitions are routinely added.
A complete list is available at \url{http://metamath.org}.

These {\sc ASCII} representations, along
with information on how to display them,
are defined in the \texttt{set.mm} database file inside
a special comment called a \texttt{\$t} {\em
comment}\index{\texttt{\$t} comment} or {\em typesetting
comment.}\index{typesetting comment}
A typesetting comment
is indicated by the appearance of the
two-character string \texttt{\$t} at the beginning of the comment.
For more information,
see Section~\ref{tcomment}, p.~\pageref{tcomment}.

In the following table the ``{\sc ASCII}'' column shows the {\sc ASCII}
representation,
``Symbol'' shows the mathematical symbolic display
that corresponds to that {\sc ASCII} representation, ``Labels'' shows
the key label(s) that define the representation, and
``Description'' provides a description about the symbol.
As usual, ``iff'' is short for ``if and only if.''\index{iff}
In most cases the ``{\sc ASCII}'' column only shows
the key token, but it will sometimes show a sequence of tokens
if that is necessary for clarity.

{\setlength{\extrarowsep}{4pt} % Keep rows from being too close together
\begin{longtabu}   { @{} c c l X }
\textbf{ASCII} & \textbf{Symbol} & \textbf{Labels} & \textbf{Description} \\
\endhead
\texttt{|-} & $\vdash$ & &
  ``It is provable that...'' \\
\texttt{ph} & $\varphi$ & \texttt{wph} &
  The wff (boolean) variable phi,
  conventionally the first wff variable. \\
\texttt{ps} & $\psi$ & \texttt{wps} &
  The wff (boolean) variable psi,
  conventionally the second wff variable. \\
\texttt{ch} & $\chi$ & \texttt{wch} &
  The wff (boolean) variable chi,
  conventionally the third wff variable. \\
\texttt{-.} & $\lnot$ & \texttt{wn} &
  Logical not. E.g., if $\varphi$ is true, then $\lnot \varphi$ is false. \\
\texttt{->} & $\rightarrow$ & \texttt{wi} &
  Implies, also known as material implication.
  In classical logic the expression $\varphi \rightarrow \psi$ is true
  if either $\varphi$ is false or $\psi$ is true (or both), that is,
  $\varphi \rightarrow \psi$ has the same meaning as
  $\lnot \varphi \lor \psi$ (as proven in theorem \texttt{imor}). \\
\texttt{<->} & $\leftrightarrow$ &
  \hyperref[df-bi]{\texttt{df-bi}} &
  Biconditional (aka is-equals for boolean values).
  $\varphi \leftrightarrow \psi$ is true iff
  $\varphi$ and $\psi$ have the same value. \\
\texttt{\char`\\/} & $\lor$ &
  \makecell[tl]{{\hyperref[df-or]{\texttt{df-or}}}, \\
	         \hyperref[df-3or]{\texttt{df-3or}}} &
  Disjunction (logical ``or''). $\varphi \lor \psi$ is true iff
  $\varphi$, $\psi$, or both are true. \\
\texttt{/\char`\\} & $\land$ &
  \makecell[tl]{{\hyperref[df-an]{\texttt{df-an}}}, \\
                 \hyperref[df-3an]{\texttt{df-3an}}} &
  Conjunction (logical ``and''). $\varphi \land \psi$ is true iff
  both $\varphi$ and $\psi$ are true. \\
\texttt{A.} & $\forall$ &
  \texttt{wal} &
  For all; the wff $\forall x \varphi$ is true iff
  $\varphi$ is true for all values of $x$. \\
\texttt{E.} & $\exists$ &
  \hyperref[df-ex]{\texttt{df-ex}} &
  There exists; the wff
  $\exists x \varphi$ is true iff
  there is at least one $x$ where $\varphi$ is true. \\
\texttt{[ y / x ]} & $[ y / x ]$ &
  \hyperref[df-sb]{\texttt{df-sb}} &
  The wff $[ y / x ] \varphi$ produces
  the result when $y$ is properly substituted for $x$ in $\varphi$
  ($y$ replaces $x$).
  % This is elsb4
  % ( [ x / y ] z e. y <-> z e. x )
  For example,
  $[ x / y ] z \in y$ is the same as $z \in x$. \\
\texttt{E!} & $\exists !$ &
  \hyperref[df-eu]{\texttt{df-eu}} &
  There exists exactly one;
  $\exists ! x \varphi$ is true iff
  there is at least one $x$ where $\varphi$ is true. \\
\texttt{\{ y | phi \}}  & $ \{ y | \varphi \}$ &
  \hyperref[df-clab]{\texttt{df-clab}} &
  The class of all sets where $\varphi$ is true. \\
\texttt{=} & $ = $ &
  \hyperref[df-cleq]{\texttt{df-cleq}} &
  Class equality; $A = B$ iff $A$ equals $B$. \\
\texttt{e.} & $ \in $ &
  \hyperref[df-clel]{\texttt{df-clel}} &
  Class membership; $A \in B$ if $A$ is a member of $B$. \\
\texttt{{\char`\_}V} & {\rm V} &
  \hyperref[df-v]{\texttt{df-v}} &
  Class of all sets (not itself a set). \\
\texttt{C\_} & $ \subseteq $ &
  \hyperref[df-ss]{\texttt{df-ss}} &
  Subclass (subset); $A \subseteq B$ is true iff
  $A$ is a subclass of $B$. \\
\texttt{u.} & $ \cup $ &
  \hyperref[df-un]{\texttt{df-un}} &
  $A \cup B$ is the union of classes $A$ and $B$. \\
\texttt{i^i} & $ \cap $ &
  \hyperref[df-in]{\texttt{df-in}} &
  $A \cap B$ is the intersection of classes $A$ and $B$. \\
\texttt{\char`\\} & $ \setminus $ &
  \hyperref[df-dif]{\texttt{df-dif}} &
  $A \setminus B$ (set difference)
  is the class of all sets in $A$ except for those in $B$. \\
\texttt{(/)} & $ \varnothing $ &
  \hyperref[df-nul]{\texttt{df-nul}} &
  $ \varnothing $ is the empty set (aka null set). \\
\texttt{\char`\~P} & $ \cal P $ &
  \hyperref[df-pw]{\texttt{df-pw}} &
  Power class. \\
\texttt{<.\ A , B >.} & $\langle A , B \rangle$ &
  \hyperref[df-op]{\texttt{df-op}} &
  The ordered pair $\langle A , B \rangle$. \\
\texttt{( F ` A )} & $ ( F ` A ) $ &
  \hyperref[df-fv]{\texttt{df-fv}} &
  The value of function $F$ when applied to $A$. \\
\texttt{\_i} & $ i $ &
  \texttt{df-i} &
  The square root of negative one. \\
\texttt{x.} & $ \cdot $ &
  \texttt{df-mul} &
  Complex number multiplication; $2~\cdot~3~=~6$. \\
\texttt{CC} & $ \mathbb{C} $ &
  \texttt{df-c} &
  The set of complex numbers. \\
\texttt{RR} & $ \mathbb{R} $ &
  \texttt{df-r} &
  The set of real numbers. \\
\end{longtabu}
} % end of extrarowsep

\chapter{Compressed Proofs}
\label{compressed}\index{compressed proof}\index{proof!compressed}

The proofs in the \texttt{set.mm} set theory database are stored in compressed
format for efficiency.  Normally you needn't concern yourself with the
compressed format, since you can display it with the usual proof display tools
in the Metamath program (\texttt{show proof}\ldots) or convert it to the normal
RPN proof format described in Section~\ref{proof} (with \texttt{save proof}
{\em label} \texttt{/normal}).  However for sake of completeness we describe the
format here and show how it maps to the normal RPN proof format.

A compressed proof, located between \texttt{\$=} and \texttt{\$.}\ keywords, consists
of a left parenthesis, a sequence of statement labels, a right parenthesis,
and a sequence of upper-case letters \texttt{A} through \texttt{Z} (with optional
white space between them).  White space must surround the parentheses
and the labels.  The left parenthesis tells Metamath that a
compressed proof follows.  (A normal RPN proof consists of just a sequence of
labels, and a parenthesis is not a legal character in a label.)

The sequence of upper-case letters corresponds to a sequence of integers
with the following mapping.  Each integer corresponds to a proof step as
described later.
\begin{center}
  \texttt{A} = 1 \\
  \texttt{B} = 2 \\
   \ldots \\
  \texttt{T} = 20 \\
  \texttt{UA} = 21 \\
  \texttt{UB} = 22 \\
   \ldots \\
  \texttt{UT} = 40 \\
  \texttt{VA} = 41 \\
  \texttt{VB} = 42 \\
   \ldots \\
  \texttt{YT} = 120 \\
  \texttt{UUA} = 121 \\
   \ldots \\
  \texttt{YYT} = 620 \\
  \texttt{UUUA} = 621 \\
   etc.
\end{center}

In other words, \texttt{A} through \texttt{T} represent the
least-significant digit in base 20, and \texttt{U} through \texttt{Y}
represent zero or more most-significant digits in base 5, where the
digits start counting at 1 instead of the usual 0. With this scheme, we
don't need white space between these ``numbers.''

(In the design of the compressed proof format, only upper-case letters,
as opposed to say all non-whitespace printable {\sc ascii} characters other than
%\texttt{\$}, was chosen to make the compressed proof a little less
%displeasing to the eye, at the expense of a typical 20\% compression
\texttt{\$}, were chosen so as not to collide with most text editor
searches, at the expense of a typical 20\% compression
loss.  The base 5/base 20 grouping, as opposed to say base 6/base 19,
was chosen by experimentally determining the grouping that resulted in
best typical compression.)

The letter \texttt{Z} identifies (tags) a proof step that is identical to one
that occurs later on in the proof; it helps shorten the proof by not requiring
that identical proof steps be proved over and over again (which happens often
when building wff's).  The \texttt{Z} is placed immediately after the
least-significant digit (letters \texttt{A} through \texttt{T}) that ends the integer
corresponding to the step to later be referenced.

The integers that the upper-case letters correspond to are mapped to labels as
follows.  If the statement being proved has $m$ mandatory hypotheses, integers
1 through $m$ correspond to the labels of these hypotheses in the order shown
by the \texttt{show statement ... / full} command, i.e., the RPN order\index{RPN
order} of the mandatory
hypotheses.  Integers $m+1$ through $m+n$ correspond to the labels enclosed in
the parentheses of the compressed proof, in the order that they appear, where
$n$ is the number of those labels.  Integers $m+n+1$ on up don't directly
correspond to statement labels but point to proof steps identified with the
letter \texttt{Z}, so that these proof steps can be referenced later in the
proof.  Integer $m+n+1$ corresponds to the first step tagged with a \texttt{Z},
$m+n+2$ to the second step tagged with a \texttt{Z}, etc.  When the compressed
proof is converted to a normal proof, the entire subproof of a step tagged
with \texttt{Z} replaces the reference to that step.

For efficiency, Metamath works with compressed proofs directly, without
converting them internally to normal proofs.  In addition to the usual
error-checking, an error message is given if (1) a label in the label list in
parentheses does not refer to a previous \texttt{\$p} or \texttt{\$a} statement or a
non-mandatory hypothesis of the statement being proved and (2) a proof step
tagged with \texttt{Z} is referenced before the step tagged with the \texttt{Z}.

Just as in a normal proof under development (Section~\ref{unknown}), any step
or subproof that is not yet known may be represented with a single \texttt{?}.
White space does not have to appear between the \texttt{?}\ and the upper-case
letters (or other \texttt{?}'s) representing the remainder of the proof.

% April 1, 2004 Appendix C has been added back in with corrections.
%
% May 20, 2003 Appendix C was removed for now because there was a problem found
% by Bob Solovay
%
% Also, removed earlier \ref{formalspec} 's (3 cases above)
%
% Bob Solovay wrote on 30 Nov 2002:
%%%%%%%%%%%%% (start of email comment )
%      3. My next set of comments concern appendix C. I read this before I
% read Chapter 4. So I first noted that the system as presented in the
% Appendix lacked a certain formal property that I thought desirable. I
% then came up with a revised formal system that had this property. Upon
% reading Chapter 4, I noticed that the revised system was closer to the
% treatment in Chapter 4 than the system in Appendix C.
%
%         First a very minor correction:
%
%         On page 142 line 2: The condition that V(e) != V(f) should only be
% required of e, f in T such that e != f.
%
%         Here is a natural property [transitivity] that one would like
% the formal system to have:
%
%         Let Gamma be a set of statements. Suppose that the statement Phi
% is provable from Gamma and that the statement Psi is provable from Gamma
% \cup {Phi}. Then Psi is provable from Gamma.
%
%         I shall present an example to show that this property does not
% hold for the formal systems of Appendix C:
%
%         I write the example in metamath style:
%
% $c A B C D E $.
% $v x y
%
% ${
% tx $f A x $.
% ty $f B y $.
% ax1 $a C x y $.
% $}
%
% ${
% tx $f A x $.
% ty $f B y $.
% ax2-h1 $e C x y $.
% ax2 $a D y $.
% $}
%
% ${
% ty $f B y $.
% ax3-h1 $e D y $.
% ax3 $a E y $.
% $}
%
% $(These three axioms are Gamma $)
%
% ${
% tx $f A x $.
% ty $f B y $.
% Phi $p D y $=
% tx ty tx ty ax1 ax2 $.
% $}
%
% ${
% ty $f B y $.
% Psi $p E y $=
% ty ty Phi ax3 $.
% $}
%
%
% I omit the formal proofs of the following claims. [I will be glad to
% supply them upon request.]
%
% 1) Psi is not provable from Gamma;
%
% 2) Psi is provable from Gamma + Phi.
%
% Here "provable" refers to the formalism of Appendix C.
%
% The trouble of course is that Psi is lacking the variable declaration
%
% $f Ax $.
%
% In the Metamath system there is no trouble proving Psi. I attach a
% metamath file that shows this and which has been checked by the
% metamath program.
%
% I next want to indicate how I think the treatment in Appendix C should
% be revised so as to conform more closely to the metamath system of the
% main text. The revised system *does* have the transitivity property.
%
% We want to give revised definitions of "statement" and
% "provable". [cf. sections C.2.4. and C.2.5] Our new definitions will
% use the definitions given in Appendix C. So we take the following
% tack. We refer to the original notions as o-statement and o-provable. And
% we refer to the notions we are defining as n-statement and n-provable.
%
%         A n-statement is an o-statement in which the only variables
% that appear in the T component are mandatory.
%
%         To any o-statement we can associate its reduct which is a
% n-statement by dropping all the elements of T or D which contain
% non-mandatory variables.
%
%         An n-statement gamma is n-provable if there is an o-statement
% gamma' which has gamma as its reduct andf such that gamma' is
% o-provable.
%
%         It seems to me [though I am not completely sure on this point]
% that n-provability corresponds to metamath provability as discussed
% say in Chapter 4.
%
%         Attached to this letter is the metamath proof of Phi and Psi
% from Gamma discussed above.
%
%         I am still brooding over the question of whether metamath
% correctly formalizes set-theory. No doubt I will have some questions
% re this after my thoughts become clearer.
%%%%%%%%%%%%%%%% (end of email comment)

%%%%%%%%%%%%%%%% (start of 2nd email comment from Bob Solovay 1-Apr-04)
%
%         I hope that Appendix C is the one that gives a "formal" treatment
% of Metamath. At any rate, thats the appendix I want to comment on.
%
%         I'm going to suggest two changes in the definition.
%
%         First change (in the definition of statement): Require that the
% sets D, T, and E be finite.
%
%         Probably things are fine as you give them. But in the applications
% to the main metamath system they will always be finite, and its useful in
% thinking about things [at least for me] to stick to the finite case.
%
%         Second change:
%
%         First let me give an approximate description. Remove the dummy
% variables from the statement. Instead, include them in the proof.
%
%         More formally: Require that T consists of type declarations only
% for mandatory variables. Require that all the pairs in D consist of
% mandatory variables.
%
%         At the start of a proof we are allowed to declare a finite number
% of dummy variables [provided that none of them appear in any of the
% statements in E \cup {A}. We have to supply type declarations for all the
% dummy variables. We are allowed to add new $d statements referring to
% either the mandatory or dummy variables. But we require that no new $d
% statement references only mandatory variables.
%
%         I find this way of doing things more conceptual than the treatment
% in Appendix C. But the change [which I will use implicitly in later
% letters about doing Peano] is mainly aesthetic. I definitely claim that my
% results on doing Peano all apply to Metamath as it is presented in your
% book.
%
%         --Bob
%
%%%%%%%%%%%%%%%% (end of 2nd email comment)

%%
%% When uncommenting the below, also uncomment references above to {formalspec}
%%
\chapter{Metamath's Formal System}\label{formalspec}\index{Metamath!as a formal
system}

\section{Introduction}

\begin{quote}
  {\em Perfection is when there is no longer anything more to take away.}
    \flushright\sc Antoine de
     Saint-Exupery\footnote{\cite[p.~3-25]{Campbell}.}\\
\end{quote}\index{de Saint-Exupery, Antoine}

This appendix describes the theory behind the Metamath language in an abstract
way intended for mathematicians.  Specifically, we construct two
set-theo\-ret\-i\-cal objects:  a ``formal system'' (roughly, a set of syntax
rules, axioms, and logical rules) and its ``universe'' (roughly, the set of
theorems derivable in the formal system).  The Metamath computer language
provides us with a way to describe specific formal systems and, with the aid of
a proof provided by the user, to verify that given theorems
belong to their universes.

To understand this appendix, you need a basic knowledge of informal set theory.
It should be sufficient to understand, for example, Ch.\ 1 of Munkres' {\em
Topology} \cite{Munkres}\index{Munkres, James R.} or the
introductory set theory chapter
in many textbooks that introduce abstract mathematics. (Note that there are
minor notational differences among authors; e.g.\ Munkres uses $\subset$ instead
of our $\subseteq$ for ``subset.''  We use ``included in'' to mean ``a subset
of,'' and ``belongs to'' or ``is contained in'' to mean ``is an element of.'')
What we call a ``formal'' description here, unlike earlier, is actually an
informal description in the ordinary language of mathematicians.  However we
provide sufficient detail so that a mathematician could easily formalize it,
even in the language of Metamath itself if desired.  To understand the logic
examples at the end of this appendix, familiarity with an introductory book on
mathematical logic would be helpful.

\section{The Formal Description}

\subsection[Preliminaries]{Preliminaries\protect\footnotemark}%
\footnotetext{This section is taken mostly verbatim
from Tarski \cite[p.~63]{Tarski1965}\index{Tarski, Alfred}.}

By $\omega$ we denote the set of all natural numbers (non-negative integers).
Each natural number $n$ is identified with the set of all smaller numbers: $n =
\{ m | m < n \}$.  The formula $m < n$ is thus equivalent to the condition: $m
\in n$ and $m,n \in \omega$. In particular, 0 is the number zero and at the
same time the empty set $\varnothing$, $1=\{0\}$, $2=\{0,1\}$, etc. ${}^B A$
denotes the set of all functions on $B$ to $A$ (i.e.\ with domain $B$ and range
included in $A$).  The members of ${}^\omega A$ are what are called {\em simple
infinite sequences},\index{simple infinite sequence}
with all {\em terms}\index{term} in $A$.  In case $n \in \omega$, the
members of ${}^n A$ are referred to as {\em finite $n$-termed
sequences},\index{finite $n$-termed
sequence} again
with terms in $A$.  The consecutive terms (function values) of a finite or
infinite sequence $f$ are denoted by $f_0, f_1, \ldots ,f_n,\ldots$.  Every
finite sequence $f \in \bigcup _{n \in \omega} {}^n A$ uniquely determines the
number $n$ such that $f \in {}^n A$; $n$ is called the {\em
length}\index{length of a sequence ({$"|\ "|$})} of $f$ and
is denoted by $|f|$.  $\langle a \rangle$ is the sequence $f$ with $|f|=1$ and
$f_0=a$; $\langle a,b \rangle$ is the sequence $f$ with $|f|=2$, $f_0=a$,
$f_1=b$; etc.  Given two finite sequences $f$ and $g$, we denote by $f\frown g$
their {\em concatenation},\index{concatenation} i.e., the
finite sequence $h$ determined by the
conditions:
\begin{eqnarray*}
& |h| = |f|+|g|;&  \\
& h_n = f_n & \mbox{\ for\ } n < |f|;  \\
& h_{|f|+n} = g_n & \mbox{\ for\ } n < |g|.
\end{eqnarray*}

\subsection{Constants, Variables, and Expressions}

A formal system has a set of {\em symbols}\index{symbol!in
a formal system} denoted
by $\mbox{\em SM}$.  A
precise set-theo\-ret\-i\-cal definition of this set is unimportant; a symbol
could be considered a primitive or atomic element if we wish.  We assume this
set is divided into two disjoint subsets:  a set $\mbox{\em CN}$ of {\em
constants}\index{constant!in a formal system} and a set $\mbox{\em VR}$ of
{\em variables}.\index{variable!in a formal system}  $\mbox{\em CN}$ and
$\mbox{\em VR}$ are each assumed to consist of countably many symbols which
may be arranged in finite or simple infinite sequences $c_0, c_1, \ldots$ and
$v_0, v_1, \ldots$ respectively, without repeating terms.  We will represent
arbitrary symbols by metavariables $\alpha$, $\beta$, etc.

{\footnotesize\begin{quotation}
{\em Comment.} The variables $v_0, v_1, \ldots$ of our formal system
correspond to what are usually considered ``metavariables'' in
descriptions of specific formal systems in the literature.  Typically,
when describing a specific formal system a book will postulate a set of
primitive objects called variables, then proceed to describe their
properties using metavariables that range over them, never mentioning
again the actual variables themselves.  Our formal system does not
mention these primitive variable objects at all but deals directly with
metavariables, as its primitive objects, from the start.  This is a
subtle but key distinction you should keep in mind, and it makes our
definition of ``formal system'' somewhat different from that typically
found in the literature.  (So, the $\alpha$, $\beta$, etc.\ above are
actually ``metametavariables'' when used to represent $v_0, v_1,
\ldots$.)
\end{quotation}}

Finite sequences all terms of which are symbols are called {\em
expressions}.\index{expression!in a formal system}  $\mbox{\em EX}$ is
the set of all expressions; thus
\begin{displaymath}
\mbox{\em EX} = \bigcup _{n \in \omega} {}^n \mbox{\em SM}.
\end{displaymath}

A {\em constant-prefixed expression}\index{constant-prefixed expression}
is an expression of non-zero length
whose first term is a constant.  We denote the set of all constant-prefixed
expressions by $\mbox{\em EX}_C = \{ e \in \mbox{\em EX} | ( |e| > 0 \wedge
e_0 \in \mbox{\em CN} ) \}$.

A {\em constant-variable pair}\index{constant-variable pair}
is an expression of length 2 whose first term
is a constant and whose second term is a variable.  We denote the set of all
constant-variable pairs by $\mbox{\em EX}_2 = \{ e \in \mbox{\em EX}_C | ( |e|
= 2 \wedge e_1 \in \mbox{\em VR} ) \}$.


{\footnotesize\begin{quotation}
{\em Relationship to Metamath.} In general, the set $\mbox{\em SM}$
corresponds to the set of declared math symbols in a Metamath database, the
set $\mbox{\em CN}$ to those declared with \texttt{\$c} statements, and the set
$\mbox{\em VR}$ to those declared with \texttt{\$v} statements.  Of course a
Metamath database can only have a finite number of math symbols, whereas
formal systems in general can have an infinite number, although the number of
Metamath math symbols available is in principle unlimited.

The set $\mbox{\em EX}_C$ corresponds to the set of permissible expressions
for \texttt{\$e}, \texttt{\$a}, and \texttt{\$p} statements.  The set $\mbox{\em EX}_2$
corresponds to the set of permissible expressions for \texttt{\$f} statements.
\end{quotation}}

We denote by ${\cal V}(e)$ the set of all variables in an expression $e \in
\mbox{\em EX}$, i.e.\ the set of all $\alpha \in \mbox{\em VR}$ such that
$\alpha = e_n$ for some $n < |e|$.  We also denote (with abuse of notation) by
${\cal V}(E)$ the set of all variables in a collection of expressions $E
\subseteq \mbox{\em EX}$, i.e.\ $\bigcup _{e \in E} {\cal V}(e)$.


\subsection{Substitution}

Given a function $F$ from $\mbox{\em VR}$ to
$\mbox{\em EX}$, we
denote by $\sigma_{F}$ or just $\sigma$ the function from $\mbox{\em EX}$ to
$\mbox{\em EX}$ defined recursively for nonempty sequences by
\begin{eqnarray*}
& \sigma(<\alpha>) = F(\alpha) & \mbox{for\ } \alpha \in \mbox{\em VR}; \\
& \sigma(<\alpha>) = <\alpha> & \mbox{for\ } \alpha \not\in \mbox{\em VR}; \\
& \sigma(g \frown h) = \sigma(g) \frown
    \sigma(h) & \mbox{for\ } g,h \in \mbox{\em EX}.
\end{eqnarray*}
We also define $\sigma(\varnothing)=\varnothing$.  We call $\sigma$ a {\em
simultaneous substitution}\index{substitution!variable}\index{variable
substitution} (or just {\em substitution}) with {\em substitution
map}\index{substitution map} $F$.

We also denote (with abuse of notation) by $\sigma(E)$ a substitution on a
collection of expressions $E \subseteq \mbox{\em EX}$, i.e.\ the set $\{
\sigma(e) | e \in E \}$.  The collection $\sigma(E)$ may of course contain
fewer expressions than $E$ because duplicate expressions could result from the
substitution.

\subsection{Statements}

We denote by $\mbox{\em DV}$ the set of all
unordered pairs $\{\alpha, \beta \} \subseteq \mbox{\em VR}$ such that $\alpha
\neq \beta$.  $\mbox{\em DV}$ stands for ``distinct variables.''

A {\em pre-statement}\index{pre-statement!in a formal system} is a
quadruple $\langle D,T,H,A \rangle$ such that
$D\subseteq \mbox{\em DV}$, $T\subseteq \mbox{\em EX}_2$, $H\subseteq
\mbox{\em EX}_C$ and $H$ is finite,
$A\in \mbox{\em EX}_C$, ${\cal V}(H\cup\{A\}) \subseteq
{\cal V}(T)$, and $\forall e,f\in T {\ } {\cal V}(e) \neq {\cal V}(f)$ (or
equivalently, $e_1 \ne f_1$) whenever $e \neq f$. The terms of the quadruple are called {\em
distinct-variable restrictions},\index{disjoint-variable restriction!in a
formal system} {\em variable-type hypotheses},\index{variable-type
hypothesis!in a formal system} {\em logical hypotheses},\index{logical
hypothesis!in a formal system} and the {\em assertion}\index{assertion!in a
formal system} respectively.  We denote by $T_M$ ({\em mandatory variable-type
hypotheses}\index{mandatory variable-type hypothesis!in a formal system}) the
subset of $T$ such that ${\cal V}(T_M) ={\cal V}(H \cup \{A\})$.  We denote by
$D_M=\{\{\alpha,\beta\}\in D|\{\alpha,\beta\}\subseteq {\cal V}(T_M)\}$ the
{\em mandatory distinct-variable restrictions}\index{mandatory
disjoint-variable restriction!in a formal system} of the pre-statement.
The set
of {\em mandatory hypotheses}\index{mandatory hypothesis!in a formal system}
is $T_M\cup H$.  We call the quadruple $\langle D_M,T_M,H,A \rangle$
the {\em reduct}\index{reduct!in a formal system} of
the pre-statement $\langle D,T,H,A \rangle$.

A {\em statement} is the reduct of some pre-statement\index{statement!in a
formal system}.  A statement is therefore a special kind of pre-statement;
in particular, a statement is the reduct of itself.

{\footnotesize\begin{quotation}
{\em Comment.}  $T$ is a set of expressions, each of length 2, that associate
a set of constants (``variable types'') with a set of variables.  The
condition ${\cal V}(H\cup\{A\}) \subseteq {\cal V}(T) $
means that each variable occurring in a statement's logical
hypotheses or assertion must have an associated variable-type hypothesis or
``type declaration,'' in  analogy to a computer programming language, where a
variable must be declared to be say, a string or an integer.  The requirement
that $\forall e,f\in T \, e_1 \ne f_1$ for $e\neq f$
means that each variable must be
associated with a unique constant designating its variable type; e.g., a
variable might be a ``wff'' or a ``set'' but not both.

Distinct-variable restrictions are used to specify what variable substitutions
are permissible to make for the statement to remain valid.  For example, in
the theorem scheme of set theory $\lnot\forall x\,x=y$ we may not substitute
the same variable for both $x$ and $y$.  On the other hand, the theorem scheme
$x=y\to y=x$ does not require that $x$ and $y$ be distinct, so we do not
require a distinct-variable restriction, although having one
would cause no harm other than making the scheme less general.

A mandatory variable-type hypothesis is one whose variable exists in a logical
hypothesis or the assertion.  A provable pre-statement
(defined below) may require
non-mandatory variable-type hypotheses that effectively introduce ``dummy''
variables for use in its proof.  Any number of dummy variables might
be required by a specific proof; indeed, it has been shown by H.\
Andr\'{e}ka\index{Andr{\'{e}}ka, H.} \cite{Nemeti} that there is no finite
upper bound to the number of dummy variables needed to prove an arbitrary
theorem in first-order logic (with equality) having a fixed number $n>2$ of
individual variables.  (See also the Comment on p.~\pageref{nodd}.)
For this reason we do not set a finite size bound on the collections $D$ and
$T$, although in an actual application (Metamath database) these will of
course be finite, increased to whatever size is necessary as more
proofs are added.
\end{quotation}}

{\footnotesize\begin{quotation}
{\em Relationship to Metamath.} A pre-statement of a formal system
corresponds to an extended frame in a Metamath database
(Section~\ref{frames}).  The collections $D$, $T$, and $H$ correspond
respectively to the \texttt{\$d}, \texttt{\$f}, and \texttt{\$e}
statement collections in an extended frame.  The expression $A$
corresponds to the \texttt{\$a} (or \texttt{\$p}) statement in an
extended frame.

A statement of a formal system corresponds to a frame in a Metamath
database.
\end{quotation}}

\subsection{Formal Systems}

A {\em formal system}\index{formal system} is a
triple $\langle \mbox{\em CN},\mbox{\em
VR},\Gamma\rangle$ where $\Gamma$ is a set of statements.  The members of
$\Gamma$ are called {\em axiomatic statements}.\index{axiomatic
statement!in a formal system}  Sometimes we will refer to a
formal system by just $\Gamma$ when $\mbox{\em CN}$ and $\mbox{\em VR}$ are
understood.

Given a formal system $\Gamma$, the {\em closure}\index{closure}\footnote{This
definition of closure incorporates a simplification due to
Josh Purinton.\index{Purinton, Josh}.} of a
pre-statement
$\langle D,T,H,A \rangle$ is the smallest set $C$ of expressions
such that:
%\begin{enumerate}
%  \item $T\cup H\subseteq C$; and
%  \item If for some axiomatic statement
%    $\langle D_M',T_M',H',A' \rangle \in \Gamma_A$, for
%    some $E \subseteq C$, some $F \subseteq C-T$ (where ``-'' denotes
%    set difference), and some substitution
%    $\sigma$ we have
%    \begin{enumerate}
%       \item $\sigma(T_M') = E$ (where, as above, the $M$ denotes the
%           mandatory variable-type hypotheses of $T^A$);
%       \item $\sigma(H') = F$;
%       \item for all $\{\alpha,\beta\}\in D^A$ and $\subseteq
%         {\cal V}(T_M')$, for all $\gamma\in {\cal V}(\sigma(\langle \alpha
%         \rangle))$, and for all $\delta\in  {\cal V}(\sigma(\langle \beta
%         \rangle))$, we have $\{\gamma, \delta\} \in D$;
%   \end{enumerate}
%   then $\sigma(A') \in C$.
%\end{enumerate}
\begin{list}{}{\itemsep 0.0pt}
  \item[1.] $T\cup H\subseteq C$; and
  \item[2.] If for some axiomatic statement
    $\langle D_M',T_M',H',A' \rangle \in
       \Gamma$ and for some substitution
    $\sigma$ we have
    \begin{enumerate}
       \item[a.] $\sigma(T_M' \cup H') \subseteq C$; and
       \item[b.] for all $\{\alpha,\beta\}\in D_M'$, for all $\gamma\in
         {\cal V}(\sigma(\langle \alpha
         \rangle))$, and for all $\delta\in  {\cal V}(\sigma(\langle \beta
         \rangle))$, we have $\{\gamma, \delta\} \in D$;
   \end{enumerate}
   then $\sigma(A') \in C$.
\end{list}
A pre-statement $\langle D,T,H,A
\rangle$ is {\em provable}\index{provable statement!in a formal
system} if $A\in C$ i.e.\ if its assertion belongs to its
closure.  A statement is {\em provable} if it is
the reduct of a provable pre-statement.
The {\em universe}\index{universe of a formal system}
of a formal system is
the collection of all of its provable statements.  Note that the
set of axiomatic statements $\Gamma$ in a formal system is a subset of its
universe.

{\footnotesize\begin{quotation}
{\em Comment.} The first condition in the definition of closure simply says
that the hypotheses of the pre-statement are in its closure.

Condition 2(a) says that a substitution exists that makes the
mandatory hypotheses of an axiomatic statement exactly match some members of
the closure.  This is what we explicitly demonstrate in a Metamath language
proof.

%Conditions 2(a) and 2(b) say that a substitution exists that makes the
%(mandatory) hypotheses of an axiomatic statement exactly match some members of
%the closure.  This is what we explicitly demonstrate with a Metamath language
%proof.
%
%The set of expressions $F$ in condition 2(b) excludes the variable-type
%hypotheses; this is done because non-mandatory variable-type hypotheses are
%effectively ``dropped'' as irrelevant whereas logical hypotheses must be
%retained to achieve a consistent logical system.

Condition 2(b) describes how distinct-variable restrictions in the axiomatic
statement must be met.  It means that after a substitution for two variables
that must be distinct, the resulting two expressions must either contain no
variables, or if they do, they may not have variables in common, and each pair
of any variables they do have, with one variable from each expression, must be
specified as distinct in the original statement.
\end{quotation}}

{\footnotesize\begin{quotation}
{\em Relationship to Metamath.} Axiomatic statements
 and provable statements in a formal
system correspond to the frames for \texttt{\$a} and \texttt{\$p} statements
respectively in a Metamath database.  The set of axiomatic statements is a
subset of the set of provable statements in a formal system, although in a
Metamath database a \texttt{\$a} statement is distinguished by not having a
proof.  A Metamath language proof for a \texttt{\$p} statement tells the computer
how to explicitly construct a series of members of the closure ultimately
leading to a demonstration that the assertion
being proved is in the closure.  The actual closure typically contains
an infinite number of expressions.  A formal system itself does not have
an explicit object called a ``proof'' but rather the existence of a proof
is implied indirectly by membership of an assertion in a provable
statement's closure.  We do this to make the formal system easier
to describe in the language of set theory.

We also note that once established as provable, a statement may be considered
to acquire the same status as an axiomatic statement, because if the set of
axiomatic statements is extended with a provable statement, the universe of
the formal system remains unchanged (provided that $\mbox{\em VR}$ is
infinite).
In practice, this means we can build a hierarchy of provable statements to
more efficiently establish additional provable statements.  This is
what we do in Metamath when we allow proofs to reference previous
\texttt{\$p} statements as well as previous \texttt{\$a} statements.
\end{quotation}}

\section{Examples of Formal Systems}

{\footnotesize\begin{quotation}
{\em Relationship to Metamath.} The examples in this section, except Example~2,
are for the most part exact equivalents of the development in the set
theory database \texttt{set.mm}.  You may want to compare Examples~1, 3, and 5
to Section~\ref{metaaxioms}, Example 4 to Sections~\ref{metadefprop} and
\ref{metadefpred}, and Example 6 to
Section~\ref{setdefinitions}.\label{exampleref}
\end{quotation}}

\subsection{Example~1---Propositional Calculus}\index{propositional calculus}

Classical propositional calculus can be described by the following formal
system.  We assume the set of variables is infinite.  Rather than denoting the
constants and variables by $c_0, c_1, \ldots$ and $v_0, v_1, \ldots$, for
readability we will instead use more conventional symbols, with the
understanding of course that they denote distinct primitive objects.
Also for readability we may omit commas between successive terms of a
sequence; thus $\langle \mbox{wff\ } \varphi\rangle$ denotes
$\langle \mbox{wff}, \varphi\rangle$.

Let
\begin{itemize}
  \item[] $\mbox{\em CN}=\{\mbox{wff}, \vdash, \to, \lnot, (,)\}$
  \item[] $\mbox{\em VR}=\{\varphi,\psi,\chi,\ldots\}$
  \item[] $T = \{\langle \mbox{wff\ } \varphi\rangle,
             \langle \mbox{wff\ } \psi\rangle,
             \langle \mbox{wff\ } \chi\rangle,\ldots\}$, i.e.\ those
             expressions of length 2 whose first member is $\mbox{\rm wff}$
             and whose second member belongs to $\mbox{\em VR}$.\footnote{For
convenience we let $T$ be an infinite set; the definition of a statement
permits this in principle.  Since a Metamath source file has a finite size, in
practice we must of course use appropriate finite subsets of this $T$,
specifically ones containing at least the mandatory variable-type
hypotheses.  Similarly, in the source file we introduce new variables as
required, with the understanding that a potentially infinite number of
them are available.}
\noindent Then $\Gamma$ consists of the axiomatic statements that
are the reducts of the following pre-statements:
    \begin{itemize}
      \item[] $\langle\varnothing,T,\varnothing,
               \langle \mbox{wff\ }(\varphi\to\psi)\rangle\rangle$
      \item[] $\langle\varnothing,T,\varnothing,
               \langle \mbox{wff\ }\lnot\varphi\rangle\rangle$
      \item[] $\langle\varnothing,T,\varnothing,
               \langle \vdash(\varphi\to(\psi\to\varphi))
               \rangle\rangle$
      \item[] $\langle\varnothing,T,
               \varnothing,
               \langle \vdash((\varphi\to(\psi\to\chi))\to
               ((\varphi\to\psi)\to(\varphi\to\chi)))
               \rangle\rangle$
      \item[] $\langle\varnothing,T,
               \varnothing,
               \langle \vdash((\lnot\varphi\to\lnot\psi)\to
               (\psi\to\varphi))\rangle\rangle$
      \item[] $\langle\varnothing,T,
               \{\langle\vdash(\varphi\to\psi)\rangle,
                 \langle\vdash\varphi\rangle\},
               \langle\vdash\psi\rangle\rangle$
    \end{itemize}
\end{itemize}

(For example, the reduct of $\langle\varnothing,T,\varnothing,
               \langle \mbox{wff\ }(\varphi\to\psi)\rangle\rangle$
is
\begin{itemize}
\item[] $\langle\varnothing,
\{\langle \mbox{wff\ } \varphi\rangle,
             \langle \mbox{wff\ } \psi\rangle\},
             \varnothing,
               \langle \mbox{wff\ }(\varphi\to\psi)\rangle\rangle$,
\end{itemize}
which is the first axiomatic statement.)

We call the members of $\mbox{\em VR}$ {\em wff variables} or (in the context
of first-order logic which we will describe shortly) {\em wff metavariables}.
Note that the symbols $\phi$, $\psi$, etc.\ denote actual specific members of
$\mbox{\em VR}$; they are not metavariables of our expository language (which
we denote with $\alpha$, $\beta$, etc.) but are instead (meta)constant symbols
(members of $\mbox{\em SM}$) from the point of view of our expository
language.  The equivalent system of propositional calculus described in
\cite{Tarski1965} also uses the symbols $\phi$, $\psi$, etc.\ to denote wff
metavariables, but in \cite{Tarski1965} unlike here those are metavariables of
the expository language and not primitive symbols of the formal system.

The first two statements define wffs: if $\varphi$ and $\psi$ are wffs, so is
$(\varphi \to \psi)$; if $\varphi$ is a wff, so is $\lnot\varphi$. The next
three are the axioms of propositional calculus: if $\varphi$ and $\psi$ are
wffs, then $\vdash (\varphi \to (\psi \to \varphi))$ is an (axiomatic)
theorem; etc. The
last is the rule of modus ponens: if $\varphi$ and $\psi$ are wffs, and
$\vdash (\varphi\to\psi)$ and $\vdash \varphi$ are theorems, then $\vdash
\psi$ is a theorem.

The correspondence to ordinary propositional calculus is as follows.  We
consider only provable statements of the form $\langle\varnothing,
T,\varnothing,A\rangle$ with $T$ defined as above.  The first term of the
assertion $A$ of any such statement is either ``wff'' or ``$\vdash$''.  A
statement for which the first term is ``wff'' is a {\em wff} of propositional
calculus, and one where the first term is ``$\vdash$'' is a {\em
theorem (scheme)} of propositional calculus.

The universe of this formal system also contains many other provable
statements.  Those with distinct-variable restrictions are irrelevant because
propositional calculus has no constraints on substitutions.  Those that have
logical hypotheses we call {\em inferences}\index{inference} when
the logical hypotheses are of the form
$\langle\vdash\rangle\frown w$ where $w$ is a wff (with the leading constant
term ``wff'' removed).  Inferences (other than the modus ponens rule) are not a
proper part of propositional calculus but are convenient to use when building a
hierarchy of provable statements.  A provable statement with a nonsense
hypothesis such as $\langle \to,\vdash,\lnot\rangle$, and this same expression
as its assertion, we consider irrelevant; no use can be made of it in
proving theorems, since there is no way to eliminate the nonsense hypothesis.

{\footnotesize\begin{quotation}
{\em Comment.} Our use of parentheses in the definition of a wff illustrates
how axiomatic statements should be carefully stated in a way that
ties in unambiguously with the substitutions allowed by the formal system.
There are many ways we could have defined wffs---for example, Polish
prefix notation would have allowed us to omit parentheses entirely, at
the expense of readability---but we must define them in a way that is
unambiguous.  For example, if we had omitted parentheses from the
definition of $(\varphi\to \psi)$, the wff $\lnot\varphi\to \psi$ could
be interpreted as either $\lnot(\varphi\to\psi)$ or $(\lnot\varphi\to\psi)$
and would have allowed us to prove nonsense.  Note that there is no
concept of operator binding precedence built into our formal system.
\end{quotation}}

\begin{sloppy}
\subsection{Example~2---Predicate Calculus with Equality}\index{predicate
calculus}
\end{sloppy}

Here we extend Example~1 to include predicate calculus with equality,
illustrating the use of distinct-variable restrictions.  This system is the
same as Tarski's system $\mathfrak{S}_2$ in \cite{Tarski1965} (except that the
axioms of propositional calculus are different but equivalent, and a redundant
axiom is omitted).  We extend $\mbox{\em CN}$ with the constants
$\{\mbox{var},\forall,=\}$.  We extend $\mbox{\em VR}$ with an infinite set of
{\em individual metavariables}\index{individual
metavariable} $\{x,y,z,\ldots\}$ and denote this subset
$\mbox{\em Vr}$.

We also join to $\mbox{\em CN}$ a possibly infinite set $\mbox{\em Pr}$ of {\em
predicates} $\{R,S,\ldots\}$.  We associate with $\mbox{\em Pr}$ a function
$\mbox{rnk}$ from $\mbox{\em Pr}$ to $\omega$, and for $\alpha\in \mbox{\em
Pr}$ we call $\mbox{rnk}(\alpha)$ the {\em rank} of the predicate $\alpha$,
which is simply the number of ``arguments'' that the predicate has.  (Most
applications of predicate calculus will have a finite number of predicates;
for example, set theory has the single two-argument or binary predicate $\in$,
which is usually written with its arguments surrounding the predicate symbol
rather than with the prefix notation we will use for the general case.)  As a
device to facilitate our discussion, we will let $\mbox{\em Vs}$ be any fixed
one-to-one function from $\omega$ to $\mbox{\em Vr}$; thus $\mbox{\em Vs}$ is
any simple infinite sequence of individual metavariables with no repeating
terms.

In this example we will not include the function symbols that are often part of
formalizations of predicate calculus.  Using metalogical arguments that are
beyond the scope of our discussion, it can be shown that our formalization is
equivalent when functions are introduced via appropriate definitions.

We extend the set $T$ defined in Example~1 with the expressions
$\{\langle \mbox{var\ } x\rangle,$ $ \langle \mbox{var\ } y\rangle, \langle
\mbox{var\ } z\rangle,\ldots\}$.  We extend the $\Gamma$ above
with the axiomatic statements that are the reducts of the following
pre-statements:
\begin{list}{}{\itemsep 0.0pt}
      \item[] $\langle\varnothing,T,\varnothing,
               \langle \mbox{wff\ }\forall x\,\varphi\rangle\rangle$
      \item[] $\langle\varnothing,T,\varnothing,
               \langle \mbox{wff\ }x=y\rangle\rangle$
      \item[] $\langle\varnothing,T,
               \{\langle\vdash\varphi\rangle\},
               \langle\vdash\forall x\,\varphi\rangle\rangle$
      \item[] $\langle\varnothing,T,\varnothing,
               \langle \vdash((\forall x(\varphi\to\psi)
                  \to(\forall x\,\varphi\to\forall x\,\psi))
               \rangle\rangle$
      \item[] $\langle\{\{x,\varphi\}\},T,\varnothing,
               \langle \vdash(\varphi\to\forall x\,\varphi)
               \rangle\rangle$
      \item[] $\langle\{\{x,y\}\},T,\varnothing,
               \langle \vdash\lnot\forall x\lnot x=y
               \rangle\rangle$
      \item[] $\langle\varnothing,T,\varnothing,
               \langle \vdash(x=z
                  \to(x=y\to z=y))
               \rangle\rangle$
      \item[] $\langle\varnothing,T,\varnothing,
               \langle \vdash(y=z
                  \to(x=y\to x=z))
               \rangle\rangle$
\end{list}
These are the axioms not involving predicate symbols. The first two statements
extend the definition of a wff.  The third is the rule of generalization.  The
fifth states, in effect, ``For a wff $\varphi$ and variable $x$,
$\vdash(\varphi\to\forall x\,\varphi)$, provided that $x$ does not occur in
$\varphi$.''  The sixth states ``For variables $x$ and $y$,
$\vdash\lnot\forall x\lnot x = y$, provided that $x$ and $y$ are distinct.''
(This proviso is not necessary but was included by Tarski to
weaken the axiom and still show that the system is logically complete.)

Finally, for each predicate symbol $\alpha\in \mbox{\em Pr}$, we add to
$\Gamma$ an axiomatic statement, extending the definition of wff,
that is the reduct of the following pre-statement:
\begin{displaymath}
    \langle\varnothing,T,\varnothing,
            \langle \mbox{wff},\alpha\rangle\
            \frown \mbox{\em Vs}\restriction\mbox{rnk}(\alpha)\rangle
\end{displaymath}
and for each $\alpha\in \mbox{\em Pr}$ and each $n < \mbox{rnk}(\alpha)$
we add to $\Gamma$ an equality axiom that is the reduct of the
following pre-statement:
\begin{eqnarray*}
    \lefteqn{\langle\varnothing,T,\varnothing,
            \langle
      \vdash,(,\mbox{\em Vs}_n,=,\mbox{\em Vs}_{\mbox{rnk}(\alpha)},\to,
            (,\alpha\rangle\frown \mbox{\em Vs}\restriction\mbox{rnk}(\alpha)} \\
  & & \frown
            \langle\to,\alpha\rangle\frown \mbox{\em Vs}\restriction n\frown
            \langle \mbox{\em Vs}_{\mbox{rnk}(\alpha)}\rangle \\
 & & \frown
            \mbox{\em Vs}\restriction(\mbox{rnk}(\alpha)\setminus(n+1))\frown
            \langle),)\rangle\rangle
\end{eqnarray*}
where $\restriction$ denotes function domain restriction and $\setminus$
denotes set difference.  Recall that a subscript on $\mbox{\em Vs}$
denotes one of its terms.  (In the above two axiom sets commas are placed
between successive terms of sequences to prevent ambiguity, and if you examine
them with care you will be able to distinguish those parentheses that denote
constant symbols from those of our expository language that delimit function
arguments.  Although it might have been better to use boldface for our
primitive symbols, unfortunately boldface was not available for all characters
on the \LaTeX\ system used to typeset this text.)  These seemingly forbidding
axioms can be understood by analogy to concatenation of substrings in a
computer language.  They are actually relatively simple for each specific case
and will become clearer by looking at the special case of a binary predicate
$\alpha = R$ where $\mbox{rnk}(R)=2$.  Letting $\mbox{\em Vs}$ be the sequence
$\langle x,y,z,\ldots\rangle$, the axioms we would add to $\Gamma$ for this
case would be the wff extension and two equality axioms that are the
reducts of the pre-statements:
\begin{list}{}{\itemsep 0.0pt}
      \item[] $\langle\varnothing,T,\varnothing,
               \langle \mbox{wff\ }R x y\rangle\rangle$
      \item[] $\langle\varnothing,T,\varnothing,
               \langle \vdash(x=z
                  \to(R x y \to R z y))
               \rangle\rangle$
      \item[] $\langle\varnothing,T,\varnothing,
               \langle \vdash(y=z
                  \to(R x y \to R x z))
               \rangle\rangle$
\end{list}
Study these carefully to see how the general axioms above evaluate to
them.  In practice, typically only a few special cases such as this would be
needed, and in any case the Metamath language will only permit us to describe
a finite number of predicates, as opposed to the infinite number permitted by
the formal system.  (If an infinite number should be needed for some reason,
we could not define the formal system directly in the Metamath language but
could instead define it metalogically under set theory as we
do in this appendix, and only the underlying set theory, with its single
binary predicate, would be defined directly in the Metamath language.)


{\footnotesize\begin{quotation}
{\em Comment.}  As we noted earlier, the specific variables denoted by the
symbols $x,y,z,\ldots\in \mbox{\em Vr}\subseteq \mbox{\em VR}\subseteq
\mbox{\em SM}$ in Example~2 are not the actual variables of ordinary predicate
calculus but should be thought of as metavariables ranging over them.  For
example, a distinct-variable restriction would be meaningless for actual
variables of ordinary predicate calculus since two different actual variables
are by definition distinct.  And when we talk about an arbitrary
representative $\alpha\in \mbox{\em Vr}$, $\alpha$ is a metavariable (in our
expository language) that ranges over metavariables (which are primitives of
our formal system) each of which ranges over the actual individual variables
of predicate calculus (which are never mentioned in our formal system).

The constant called ``var'' above is called \texttt{setvar} in the
\texttt{set.mm} database file, but it means the same thing.  I felt
that ``var'' is a more meaningful name in the context of predicate
calculus, whose use is not limited to set theory.  For consistency we
stick with the name ``var'' throughout this Appendix, even after set
theory is introduced.
\end{quotation}}

\subsection{Free Variables and Proper Substitution}\index{free variable}
\index{proper substitution}\index{substitution!proper}

Typical representations of mathematical axioms use concepts such
as ``free variable,'' ``bound variable,'' and ``proper substitution''
as primitive notions.
A free variable is a variable that
is not a parameter of any container expression.
A bound variable is the opposite of a free variable; it is a
a variable that has been bound in a container expression.
For example, in the expression $\forall x \varphi$ (for all $x$, $\varphi$
is true), the variable $x$
is bound within the for-all ($\forall$) expression.
It is possible to change one variable to another, and that process is called
``proper substitution.''
In most books, proper substitution has a somewhat complicated recursive
definition with multiple cases based on the occurrences of free and
bound variables.
You may consult
\cite[ch.\ 3--4]{Hamilton}\index{Hamilton, Alan G.} (as well as
many other texts) for more formal details about these terms.

Using these concepts as \texttt{primitives} creates complications
for computer implementations.

In the system of Example~2, there are no primitive notions of free variable
and proper substitution.  Tarski \cite{Tarski1965} shows that this system is
logically equivalent to the more typical textbook systems that do have these
primitive notions, if we introduce these notions with appropriate definitions
and metalogic.  We could also define axioms for such systems directly,
although the recursive definitions of free variable and proper substitution
would be messy and awkward to work with.  Instead, we mention two devices that
can be used in practice to mimic these notions.  (1) Instead of introducing
special notation to express (as a logical hypothesis) ``where $x$ is not free
in $\varphi$'' we can use the logical hypothesis $\vdash(\varphi\to\forall
x\,\varphi)$.\label{effectivelybound}\index{effectively
not free}\footnote{This is a slightly weaker requirement than ``where $x$ is
not free in $\varphi$.''  If we let $\varphi$ be $x=x$, we have the theorem
$(x=x\to\forall x\,x=x)$ which satisfies the hypothesis, even though $x$ is
free in $x=x$ .  In a case like this we say that $x$ is {\em effectively not
free}\index{effectively not free} in $x=x$, since $x=x$ is logically
equivalent to $\forall x\,x=x$ in which $x$ is bound.} (2) It can be shown
that the wff $((x=y\to\varphi)\wedge\exists x(x=y\wedge\varphi))$ (with the
usual definitions of $\wedge$ and $\exists$; see Example~4 below) is logically
equivalent to ``the wff that results from proper substitution of $y$ for $x$
in $\varphi$.''  This works whether or not $x$ and $y$ are distinct.

\subsection{Metalogical Completeness}\index{metalogical completeness}

In the system of Example~2, the
following are provable pre-statements (and their reducts are
provable statements):
\begin{eqnarray*}
      & \langle\{\{x,y\}\},T,\varnothing,
               \langle \vdash\lnot\forall x\lnot x=y
               \rangle\rangle & \\
     &  \langle\varnothing,T,\varnothing,
               \langle \vdash\lnot\forall x\lnot x=x
               \rangle\rangle &
\end{eqnarray*}
whereas the following pre-statement is not to my knowledge provable (but
in any case we will pretend it's not for sake of illustration):
\begin{eqnarray*}
     &  \langle\varnothing,T,\varnothing,
               \langle \vdash\lnot\forall x\lnot x=y
               \rangle\rangle &
\end{eqnarray*}
In other words, we can prove ``$\lnot\forall x\lnot x=y$ where $x$ and $y$ are
distinct'' and separately prove ``$\lnot\forall x\lnot x=x$'', but we can't
prove the combined general case ``$\lnot\forall x\lnot x=y$'' that has no
proviso.  Now this does not compromise logical completeness, because the
variables are really metavariables and the two provable cases together cover
all possible cases.  The third case can be considered a metatheorem whose
direct proof, using the system of Example~2, lies outside the capability of the
formal system.

Also, in the system of Example~2 the following pre-statement is not to my
knowledge provable (again, a conjecture that we will pretend to be the case):
\begin{eqnarray*}
     & \langle\varnothing,T,\varnothing,
               \langle \vdash(\forall x\, \varphi\to\varphi)
               \rangle\rangle &
\end{eqnarray*}
Instead, we can only prove specific cases of $\varphi$ involving individual
metavariables, and by induction on formula length, prove as a metatheorem
outside of our formal system the general statement above.  The details of this
proof are found in \cite{Kalish}.

There does, however, exist a system of predicate calculus in which all such
``simple metatheorems'' as those above can be proved directly, and we present
it in Example~3. A {\em simple metatheorem}\index{simple metatheorem}
is any statement of the formal
system of Example~2 where all distinct variable restrictions consist of either
two individual metavariables or an individual metavariable and a wff
metavariable, and which is provable by combining cases outside the system as
above.  A system is {\em metalogically complete}\index{metalogical
completeness} if all of its simple
metatheorems are (directly) provable statements. The precise definition of
``simple metatheorem'' and the proof of the ``metalogical completeness'' of
Example~3 is found in Remark 9.6 and Theorem 9.7 of \cite{Megill}.\index{Megill,
Norman}

\begin{sloppy}
\subsection{Example~3---Metalogically Complete Predicate
Calculus with
Equality}
\end{sloppy}

For simplicity we will assume there is one binary predicate $R$;
this system suffices for set theory, where the $R$ is of course the $\in$
predicate.  We label the axioms as they appear in \cite{Megill}.  This
system is logically equivalent to that of Example~2 (when the latter is
restricted to this single binary predicate) but is also metalogically
complete.\index{metalogical completeness}

Let
\begin{itemize}
  \item[] $\mbox{\em CN}=\{\mbox{wff}, \mbox{var}, \vdash, \to, \lnot, (,),\forall,=,R\}$.
  \item[] $\mbox{\em VR}=\{\varphi,\psi,\chi,\ldots\}\cup\{x,y,z,\ldots\}$.
  \item[] $T = \{\langle \mbox{wff\ } \varphi\rangle,
             \langle \mbox{wff\ } \psi\rangle,
             \langle \mbox{wff\ } \chi\rangle,\ldots\}\cup
       \{\langle \mbox{var\ } x\rangle, \langle \mbox{var\ } y\rangle, \langle
       \mbox{var\ }z\rangle,\ldots\}$.

\noindent Then
  $\Gamma$ consists of the reducts of the following pre-statements:
    \begin{itemize}
      \item[] $\langle\varnothing,T,\varnothing,
               \langle \mbox{wff\ }(\varphi\to\psi)\rangle\rangle$
      \item[] $\langle\varnothing,T,\varnothing,
               \langle \mbox{wff\ }\lnot\varphi\rangle\rangle$
      \item[] $\langle\varnothing,T,\varnothing,
               \langle \mbox{wff\ }\forall x\,\varphi\rangle\rangle$
      \item[] $\langle\varnothing,T,\varnothing,
               \langle \mbox{wff\ }x=y\rangle\rangle$
      \item[] $\langle\varnothing,T,\varnothing,
               \langle \mbox{wff\ }Rxy\rangle\rangle$
      \item[(C1$'$)] $\langle\varnothing,T,\varnothing,
               \langle \vdash(\varphi\to(\psi\to\varphi))
               \rangle\rangle$
      \item[(C2$'$)] $\langle\varnothing,T,
               \varnothing,
               \langle \vdash((\varphi\to(\psi\to\chi))\to
               ((\varphi\to\psi)\to(\varphi\to\chi)))
               \rangle\rangle$
      \item[(C3$'$)] $\langle\varnothing,T,
               \varnothing,
               \langle \vdash((\lnot\varphi\to\lnot\psi)\to
               (\psi\to\varphi))\rangle\rangle$
      \item[(C4$'$)] $\langle\varnothing,T,
               \varnothing,
               \langle \vdash(\forall x(\forall x\,\varphi\to\psi)\to
                 (\forall x\,\varphi\to\forall x\,\psi))\rangle\rangle$
      \item[(C5$'$)] $\langle\varnothing,T,
               \varnothing,
               \langle \vdash(\forall x\,\varphi\to\varphi)\rangle\rangle$
      \item[(C6$'$)] $\langle\varnothing,T,
               \varnothing,
               \langle \vdash(\forall x\forall y\,\varphi\to
                 \forall y\forall x\,\varphi)\rangle\rangle$
      \item[(C7$'$)] $\langle\varnothing,T,
               \varnothing,
               \langle \vdash(\lnot\varphi\to\forall x\lnot\forall x\,\varphi
                 )\rangle\rangle$
      \item[(C8$'$)] $\langle\varnothing,T,
               \varnothing,
               \langle \vdash(x=y\to(x=z\to y=z))\rangle\rangle$
      \item[(C9$'$)] $\langle\varnothing,T,
               \varnothing,
               \langle \vdash(\lnot\forall x\, x=y\to(\lnot\forall x\, x=z\to
                 (y=z\to\forall x\, y=z)))\rangle\rangle$
      \item[(C10$'$)] $\langle\varnothing,T,
               \varnothing,
               \langle \vdash(\forall x(x=y\to\forall x\,\varphi)\to
                 \varphi))\rangle\rangle$
      \item[(C11$'$)] $\langle\varnothing,T,
               \varnothing,
               \langle \vdash(\forall x\, x=y\to(\forall x\,\varphi
               \to\forall y\,\varphi))\rangle\rangle$
      \item[(C12$'$)] $\langle\varnothing,T,
               \varnothing,
               \langle \vdash(x=y\to(Rxz\to Ryz))\rangle\rangle$
      \item[(C13$'$)] $\langle\varnothing,T,
               \varnothing,
               \langle \vdash(x=y\to(Rzx\to Rzy))\rangle\rangle$
      \item[(C15$'$)] $\langle\varnothing,T,
               \varnothing,
               \langle \vdash(\lnot\forall x\, x=y\to(x=y\to(\varphi
                 \to\forall x(x=y\to\varphi))))\rangle\rangle$
      \item[(C16$'$)] $\langle\{\{x,y\}\},T,
               \varnothing,
               \langle \vdash(\forall x\, x=y\to(\varphi\to\forall x\,\varphi)
                 )\rangle\rangle$
      \item[(C5)] $\langle\{\{x,\varphi\}\},T,\varnothing,
               \langle \vdash(\varphi\to\forall x\,\varphi)
               \rangle\rangle$
      \item[(MP)] $\langle\varnothing,T,
               \{\langle\vdash(\varphi\to\psi)\rangle,
                 \langle\vdash\varphi\rangle\},
               \langle\vdash\psi\rangle\rangle$
      \item[(Gen)] $\langle\varnothing,T,
               \{\langle\vdash\varphi\rangle\},
               \langle\vdash\forall x\,\varphi\rangle\rangle$
    \end{itemize}
\end{itemize}

While it is known that these axioms are ``metalogically complete,'' it is
not known whether they are independent (i.e.\ none is
redundant) in the metalogical sense; specifically, whether any axiom (possibly
with additional non-mandatory distinct-variable restrictions, for use with any
dummy variables in its proof) is provable from the others.  Note that
metalogical independence is a weaker requirement than independence in the
usual logical sense.  Not all of the above axioms are logically independent:
for example, C9$'$ can be proved as a metatheorem from the others, outside the
formal system, by combining the possible cases of distinct variables.

\subsection{Example~4---Adding Definitions}\index{definition}
There are several ways to add definitions to a formal system.  Probably the
most proper way is to consider definitions not as part of the formal system at
all but rather as abbreviations that are part of the expository metalogic
outside the formal system.  For convenience, though, we may use the formal
system itself to incorporate definitions, adding them as axiomatic extensions
to the system.  This could be done by adding a constant representing the
concept ``is defined as'' along with axioms for it. But there is a nicer way,
at least in this writer's opinion, that introduces definitions as direct
extensions to the language rather than as extralogical primitive notions.  We
introduce additional logical connectives and provide axioms for them.  For
systems of logic such as Examples 1 through 3, the additional axioms must be
conservative in the sense that no wff of the original system that was not a
theorem (when the initial term ``wff'' is replaced by ``$\vdash$'' of course)
becomes a theorem of the extended system.  In this example we extend Example~3
(or 2) with standard abbreviations of logic.

We extend $\mbox{\em CN}$ of Example~3 with new constants $\{\leftrightarrow,
\wedge,\vee,\exists\}$, corresponding to logical equivalence,\index{logical
equivalence ($\leftrightarrow$)}\index{biconditional ($\leftrightarrow$)}
conjunction,\index{conjunction ($\wedge$)} disjunction,\index{disjunction
($\vee$)} and the existential quantifier.\index{existential quantifier
($\exists$)}  We extend $\Gamma$ with the axiomatic statements that are
the reducts of the following pre-statements:
\begin{list}{}{\itemsep 0.0pt}
      \item[] $\langle\varnothing,T,\varnothing,
               \langle \mbox{wff\ }(\varphi\leftrightarrow\psi)\rangle\rangle$
      \item[] $\langle\varnothing,T,\varnothing,
               \langle \mbox{wff\ }(\varphi\vee\psi)\rangle\rangle$
      \item[] $\langle\varnothing,T,\varnothing,
               \langle \mbox{wff\ }(\varphi\wedge\psi)\rangle\rangle$
      \item[] $\langle\varnothing,T,\varnothing,
               \langle \mbox{wff\ }\exists x\, \varphi\rangle\rangle$
  \item[] $\langle\varnothing,T,\varnothing,
     \langle\vdash ( ( \varphi \leftrightarrow \psi ) \to
     ( \varphi \to \psi ) )\rangle\rangle$
  \item[] $\langle\varnothing,T,\varnothing,
     \langle\vdash ((\varphi\leftrightarrow\psi)\to
    (\psi\to\varphi))\rangle\rangle$
  \item[] $\langle\varnothing,T,\varnothing,
     \langle\vdash ((\varphi\to\psi)\to(
     (\psi\to\varphi)\to(\varphi
     \leftrightarrow\psi)))\rangle\rangle$
  \item[] $\langle\varnothing,T,\varnothing,
     \langle\vdash (( \varphi \wedge \psi ) \leftrightarrow\neg ( \varphi
     \to \neg \psi )) \rangle\rangle$
  \item[] $\langle\varnothing,T,\varnothing,
     \langle\vdash (( \varphi \vee \psi ) \leftrightarrow (\neg \varphi
     \to \psi )) \rangle\rangle$
  \item[] $\langle\varnothing,T,\varnothing,
     \langle\vdash (\exists x \,\varphi\leftrightarrow
     \lnot \forall x \lnot \varphi)\rangle\rangle$
\end{list}
The first three logical axioms (statements containing ``$\vdash$'') introduce
and effectively define logical equivalence, ``$\leftrightarrow$''.  The last
three use ``$\leftrightarrow$'' to effectively mean ``is defined as.''

\subsection{Example~5---ZFC Set Theory}\index{ZFC set theory}

Here we add to the system of Example~4 the axioms of Zermelo--Fraenkel set
theory with Choice.  For convenience we make use of the
definitions in Example~4.

In the $\mbox{\em CN}$ of Example~4 (which extends Example~3), we replace the symbol $R$
with the symbol $\in$.
More explicitly, we remove from $\Gamma$ of Example~4 the three
axiomatic statements containing $R$ and replace them with the
reducts of the following:
\begin{list}{}{\itemsep 0.0pt}
      \item[] $\langle\varnothing,T,\varnothing,
               \langle \mbox{wff\ }x\in y\rangle\rangle$
      \item[] $\langle\varnothing,T,
               \varnothing,
               \langle \vdash(x=y\to(x\in z\to y\in z))\rangle\rangle$
      \item[] $\langle\varnothing,T,
               \varnothing,
               \langle \vdash(x=y\to(z\in x\to z\in y))\rangle\rangle$
\end{list}
Letting $D=\{\{\alpha,\beta\}\in \mbox{\em DV}\,|\alpha,\beta\in \mbox{\em
Vr}\}$ (in other words all individual variables must be distinct), we extend
$\Gamma$ with the ZFC axioms, called
\index{Axiom of Extensionality}
\index{Axiom of Replacement}
\index{Axiom of Union}
\index{Axiom of Power Sets}
\index{Axiom of Regularity}
\index{Axiom of Infinity}
\index{Axiom of Choice}
Extensionality, Replacement, Union, Power
Set, Regularity, Infinity, and Choice, that are the reducts of:
\begin{list}{}{\itemsep 0.0pt}
      \item[Ext] $\langle D,T,
               \varnothing,
               \langle\vdash (\forall x(x\in y\leftrightarrow x \in z)\to y
               =z) \rangle\rangle$
      \item[Rep] $\langle D,T,
               \varnothing,
               \langle\vdash\exists x ( \exists y \forall z (\varphi \to z = y
                        ) \to
                        \forall z ( z \in x \leftrightarrow \exists x ( x \in
                        y \wedge \forall y\,\varphi ) ) )\rangle\rangle$
      \item[Un] $\langle D,T,
               \varnothing,
               \langle\vdash \exists x \forall y ( \exists x ( y \in x \wedge
               x \in z ) \to y \in x ) \rangle\rangle$
      \item[Pow] $\langle D,T,
               \varnothing,
               \langle\vdash \exists x \forall y ( \forall x ( x \in y \to x
               \in z ) \to y \in x ) \rangle\rangle$
      \item[Reg] $\langle D,T,
               \varnothing,
               \langle\vdash (  x \in y \to
                 \exists x ( x \in y \wedge \forall z ( z \in x \to \lnot z
                \in y ) ) ) \rangle\rangle$
      \item[Inf] $\langle D,T,
               \varnothing,
               \langle\vdash \exists x(y\in x\wedge\forall y(y\in
               x\to
               \exists z(y \in z\wedge z\in x))) \rangle\rangle$
      \item[AC] $\langle D,T,
               \varnothing,
               \langle\vdash \exists x \forall y \forall z ( ( y \in z
               \wedge z \in w ) \to \exists w \forall y ( \exists w
              ( ( y \in z \wedge z \in w ) \wedge ( y \in w \wedge w \in x
              ) ) \leftrightarrow y = w ) ) \rangle\rangle$
\end{list}

\subsection{Example~6---Class Notation in Set Theory}\label{class}

A powerful device that makes set theory easier (and that we have
been using all along in our informal expository language) is {\em class
abstraction notation}.\index{class abstraction}\index{abstraction class}  The
definitions we introduce are rigorously justified
as conservative by Takeuti and Zaring \cite{Takeuti}\index{Takeuti, Gaisi} or
Quine \cite{Quine}\index{Quine, Willard Van Orman}.  The key idea is to
introduce the notation $\{x|\mbox{---}\}$ which means ``the class of all $x$
such that ---'' for abstraction classes and introduce (meta)variables that
range over them.  An abstraction class may or may not be a set, depending on
whether it exists (as a set).  A class that does not exist is
called a {\em proper class}.\index{proper class}\index{class!proper}

To illustrate the use of abstraction classes we will provide some examples
of definitions that make use of them:  the empty set, class union, and
unordered pair.  Many other such definitions can be found in the
Metamath set theory database,
\texttt{set.mm}.\index{set theory database (\texttt{set.mm})}

% We intentionally break up the sequence of math symbols here
% because otherwise the overlong line goes beyond the page in narrow mode.
We extend $\mbox{\em CN}$ of Example~5 with new symbols $\{$
$\mbox{class},$ $\{,$ $|,$ $\},$ $\varnothing,$ $\cup,$ $,$ $\}$
where the inner braces and last comma are
constant symbols. (As before,
our dual use of some mathematical symbols for both our expository
language and as primitives of the formal system should be clear from context.)

We extend $\mbox{\em VR}$ of Example~5 with a set of {\em class
variables}\index{class variable}
$\{A,B,C,\ldots\}$. We extend the $T$ of Example~5 with $\{\langle
\mbox{class\ } A\rangle, \langle \mbox{class\ }B\rangle, \langle \mbox{class\ }
C\rangle,\ldots\}$.

To
introduce our definitions,
we add to $\Gamma$ of Example~5 the axiomatic statements
that are the reducts of the following pre-statements:
\begin{list}{}{\itemsep 0.0pt}
      \item[] $\langle\varnothing,T,\varnothing,
               \langle \mbox{class\ }x\rangle\rangle$
      \item[] $\langle\varnothing,T,\varnothing,
               \langle \mbox{class\ }\{x|\varphi\}\rangle\rangle$
      \item[] $\langle\varnothing,T,\varnothing,
               \langle \mbox{wff\ }A=B\rangle\rangle$
      \item[] $\langle\varnothing,T,\varnothing,
               \langle \mbox{wff\ }A\in B\rangle\rangle$
      \item[Ab] $\langle\varnothing,T,\varnothing,
               \langle \vdash ( y \in \{ x |\varphi\} \leftrightarrow
                  ( ( x = y \to\varphi) \wedge \exists x ( x = y
                  \wedge\varphi) ))
               \rangle\rangle$
      \item[Eq] $\langle\{\{x,A\},\{x,B\}\},T,\varnothing,
               \langle \vdash ( A = B \leftrightarrow
               \forall x ( x \in A \leftrightarrow x \in B ) )
               \rangle\rangle$
      \item[El] $\langle\{\{x,A\},\{x,B\}\},T,\varnothing,
               \langle \vdash ( A \in B \leftrightarrow \exists x
               ( x = A \wedge x \in B ) )
               \rangle\rangle$
\end{list}
Here we say that an individual variable is a class; $\{x|\varphi\}$ is a
class; and we extend the definition of a wff to include class equality and
membership.  Axiom Ab defines membership of a variable in a class abstraction;
the right-hand side can be read as ``the wff that results from proper
substitution of $y$ for $x$ in $\varphi$.''\footnote{Note that this definition
makes unnecessary the introduction of a separate notation similar to
$\varphi(x|y)$ for proper substitution, although we may choose to do so to be
conventional.  Incidentally, $\varphi(x|y)$ as it stands would be ambiguous in
the formal systems of our examples, since we wouldn't know whether
$\lnot\varphi(x|y)$ meant $\lnot(\varphi(x|y))$ or $(\lnot\varphi)(x|y)$.
Instead, we would have to use an unambiguous variant such as $(\varphi\,
x|y)$.}  Axioms Eq and El extend the meaning of the existing equality and
membership connectives.  This is potentially dangerous and requires careful
justification.  For example, from Eq we can derive the Axiom of Extensionality
with predicate logic alone; thus in principle we should include the Axiom of
Extensionality as a logical hypothesis.  However we do not bother to do this
since we have already presupposed that axiom earlier. The distinct variable
restrictions should be read ``where $x$ does not occur in $A$ or $B$.''  We
typically do this when the right-hand side of a definition involves an
individual variable not in the expression being defined; it is done so that
the right-hand side remains independent of the particular ``dummy'' variable
we use.

We continue to add to $\Gamma$ the following definitions
(i.e. the reducts of the following pre-statements) for empty
set,\index{empty set} class union,\index{union} and unordered
pair.\index{unordered pair}  They should be self-explanatory.  Analogous to our
use of ``$\leftrightarrow$'' to define new wffs in Example~4, we use ``$=$''
to define new abstraction terms, and both may be read informally as ``is
defined as'' in this context.
\begin{list}{}{\itemsep 0.0pt}
      \item[] $\langle\varnothing,T,\varnothing,
               \langle \mbox{class\ }\varnothing\rangle\rangle$
      \item[] $\langle\varnothing,T,\varnothing,
               \langle \vdash \varnothing = \{ x | \lnot x = x \}
               \rangle\rangle$
      \item[] $\langle\varnothing,T,\varnothing,
               \langle \mbox{class\ }(A\cup B)\rangle\rangle$
      \item[] $\langle\{\{x,A\},\{x,B\}\},T,\varnothing,
               \langle \vdash ( A \cup B ) = \{ x | ( x \in A \vee x \in B ) \}
               \rangle\rangle$
      \item[] $\langle\varnothing,T,\varnothing,
               \langle \mbox{class\ }\{A,B\}\rangle\rangle$
      \item[] $\langle\{\{x,A\},\{x,B\}\},T,\varnothing,
               \langle \vdash \{ A , B \} = \{ x | ( x = A \vee x = B ) \}
               \rangle\rangle$
\end{list}

\section{Metamath as a Formal System}\label{theorymm}

This section presupposes a familiarity with the Metamath computer language.

Our theory describes formal systems and their universes.  The Metamath
language provides a way of representing these set-theoretical objects to
a computer.  A Metamath database, being a finite set of {\sc ascii}
characters, can usually describe only a subset of a formal system and
its universe, which are typically infinite.  However the database can
contain as large a finite subset of the formal system and its universe
as we wish.  (Of course a Metamath set theory database can, in
principle, indirectly describe an entire infinite formal system by
formalizing the expository language in this Appendix.)

For purpose of our discussion, we assume the Metamath database
is in the simple form described on p.~\pageref{framelist},
consisting of all constant and variable declarations at the beginning,
followed by a sequence of extended frames each
delimited by \texttt{\$\char`\{} and \texttt{\$\char`\}}.  Any Metamath database can
be converted to this form, as described on p.~\pageref{frameconvert}.

The math symbol tokens of a Metamath source file, which are declared
with \texttt{\$c} and \texttt{\$v} statements, are names we assign to
representatives of $\mbox{\em CN}$ and $\mbox{\em VR}$.  For
definiteness we could assume that the first math symbol declared as a
variable corresponds to $v_0$, the second to $v_1$, etc., although the
exact correspondence we choose is not important.

In the Metamath language, each \texttt{\$d}, \texttt{\$f}, and
 \texttt{\$e} source
statement in an extended frame (Section~\ref{frames})
corresponds respectively to a member of the
collections $D$, $T$, and $H$ in a formal system statement $\langle
D_M,T_M,H,A\rangle$.  The math symbol strings following these Metamath keywords
correspond to a variable pair (in the case of \texttt{\$d}) or an expression (for
the other two keywords). The math symbol string following a \texttt{\$a} source
statement corresponds to expression $A$ in an axiomatic statement of the
formal system; the one following a \texttt{\$p} source statement corresponds to
$A$ in a provable statement that is not axiomatic.  In other words, each
extended frame in a Metamath database corresponds to
a pre-statement of the formal system, and a frame corresponds to
a statement of the formal system.  (Don't confuse the two meanings of
``statement'' here.  A statement of the formal system corresponds to the
several statements in a Metamath database that may constitute a
frame.)

In order for the computer to verify that a formal system statement is
provable, each \texttt{\$p} source statement is accompanied by a proof.
However, the proof does not correspond to anything in the formal system
but is simply a way of communicating to the computer the information
needed for its verification.  The proof tells the computer {\em how to
construct} specific members of closure of the formal system
pre-statement corresponding to the extended frame of the \texttt{\$p}
statement.  The final result of the construction is the member of the
closure that matches the \texttt{\$p} statement.  The abstract formal
system, on the other hand, is concerned only with the {\em existence} of
members of the closure.

As mentioned on p.~\pageref{exampleref}, Examples 1 and 3--6 in the
previous Section parallel the development of logic and set theory in the
Metamath database
\texttt{set.mm}.\index{set theory database (\texttt{set.mm})} You may
find it instructive to compare them.


\chapter{The MIU System}
\label{MIU}
\index{formal system}
\index{MIU-system}

The following is a listing of the file \texttt{miu.mm}.  It is self-explanatory.

%%%%%%%%%%%%%%%%%%%%%%%%%%%%%%%%%%%%%%%%%%%%%%%%%%%%%%%%%%%%

\begin{verbatim}
$( The MIU-system:  A simple formal system $)

$( Note:  This formal system is unusual in that it allows
empty wffs.  To work with a proof, you must type
SET EMPTY_SUBSTITUTION ON before using the PROVE command.
By default, this is OFF in order to reduce the number of
ambiguous unification possibilities that have to be selected
during the construction of a proof.  $)

$(
Hofstadter's MIU-system is a simple example of a formal
system that illustrates some concepts of Metamath.  See
Douglas R. Hofstadter, _Goedel, Escher, Bach:  An Eternal
Golden Braid_ (Vintage Books, New York, 1979), pp. 33ff. for
a description of the MIU-system.

The system has 3 constant symbols, M, I, and U.  The sole
axiom of the system is MI. There are 4 rules:
     Rule I:  If you possess a string whose last letter is I,
     you can add on a U at the end.
     Rule II:  Suppose you have Mx.  Then you may add Mxx to
     your collection.
     Rule III:  If III occurs in one of the strings in your
     collection, you may make a new string with U in place
     of III.
     Rule IV:  If UU occurs inside one of your strings, you
     can drop it.
Unfortunately, Rules III and IV do not have unique results:
strings could have more than one occurrence of III or UU.
This requires that we introduce the concept of an "MIU
well-formed formula" or wff, which allows us to construct
unique symbol sequences to which Rules III and IV can be
applied.
$)

$( First, we declare the constant symbols of the language.
Note that we need two symbols to distinguish the assertion
that a sequence is a wff from the assertion that it is a
theorem; we have arbitrarily chosen "wff" and "|-". $)
      $c M I U |- wff $. $( Declare constants $)

$( Next, we declare some variables. $)
     $v x y $.

$( Throughout our theory, we shall assume that these
variables represent wffs. $)
 wx   $f wff x $.
 wy   $f wff y $.

$( Define MIU-wffs.  We allow the empty sequence to be a
wff. $)

$( The empty sequence is a wff. $)
 we   $a wff $.
$( "M" after any wff is a wff. $)
 wM   $a wff x M $.
$( "I" after any wff is a wff. $)
 wI   $a wff x I $.
$( "U" after any wff is a wff. $)
 wU   $a wff x U $.

$( Assert the axiom. $)
 ax   $a |- M I $.

$( Assert the rules. $)
 ${
   Ia   $e |- x I $.
$( Given any theorem ending with "I", it remains a theorem
if "U" is added after it.  (We distinguish the label I_
from the math symbol I to conform to the 24-Jun-2006
Metamath spec.) $)
   I_    $a |- x I U $.
 $}
 ${
IIa  $e |- M x $.
$( Given any theorem starting with "M", it remains a theorem
if the part after the "M" is added again after it. $)
   II   $a |- M x x $.
 $}
 ${
   IIIa $e |- x I I I y $.
$( Given any theorem with "III" in the middle, it remains a
theorem if the "III" is replaced with "U". $)
   III  $a |- x U y $.
 $}
 ${
   IVa  $e |- x U U y $.
$( Given any theorem with "UU" in the middle, it remains a
theorem if the "UU" is deleted. $)
   IV   $a |- x y $.
  $}

$( Now we prove the theorem MUIIU.  You may be interested in
comparing this proof with that of Hofstadter (pp. 35 - 36).
$)
 theorem1  $p |- M U I I U $=
      we wM wU wI we wI wU we wU wI wU we wM we wI wU we wM
      wI wI wI we wI wI we wI ax II II I_ III II IV $.
\end{verbatim}\index{well-formed formula (wff)}

The \texttt{show proof /lemmon/renumber} command
yields the following display.  It is very similar
to the one in \cite[pp.~35--36]{Hofstadter}.\index{Hofstadter, Douglas R.}

\begin{verbatim}
1 ax             $a |- M I
2 1 II           $a |- M I I
3 2 II           $a |- M I I I I
4 3 I_           $a |- M I I I I U
5 4 III          $a |- M U I U
6 5 II           $a |- M U I U U I U
7 6 IV           $a |- M U I I U
\end{verbatim}

We note that Hofstadter's ``MU-puzzle,'' which asks whether
MU is a theorem of the MIU-system, cannot be answered using
the system above because the MU-puzzle is a question {\em
about} the system.  To prove the answer to the MU-puzzle,
a much more elaborate system is needed, namely one that
models the MIU-system within set theory.  (Incidentally, the
answer to the MU-puzzle is no.)

\chapter{Metamath Language EBNF}%
\label{BNF}%
\index{Metamath Language EBNF}

The following is a formal description of the basic Metamath language syntax
(with compressed proofs and support for unknown proof steps).
It is defined using the
Extended Backus--Naur Form (EBNF)\index{Extended Backus--Naur Form}\index{EBNF}
notation from W3C\index{W3C}
\textit{Extensible Markup Language (XML) 1.0 (Fifth Edition)}
(W3C Recommendation 26 November 2008) at
\url{https://www.w3.org/TR/xml/#sec-notation}.

The \texttt{database}
rule is processed until the end of the file (\texttt{EOF}).
The rules eventually require reading whitespace-separated tokens.
A token has an upper-case definition (see below)
or is a string constant in a non-token (such as \texttt{'\$a'}).
We intend for this to be correct, but if there is a conflict the
rules of section \ref{spec} govern. That section also discusses
non-syntax restrictions not shown here
(e.g., that each new label token
defined in a \texttt{hypothesis-stmt} or \texttt{assert-stmt}
must be unique).

\begin{verbatim}
database ::= outermost-scope-stmt*

outermost-scope-stmt ::=
  include-stmt | constant-stmt | stmt

/* File inclusion command; process file as a database.
   Databases should NOT have a comment in the filename. */
include-stmt ::= '$[' filename '$]'

/* Constant symbols declaration. */
constant-stmt ::= '$c' constant+ '$.'

/* A normal statement can occur in any scope. */
stmt ::= block | variable-stmt | disjoint-stmt |
  hypothesis-stmt | assert-stmt

/* A block. You can have 0 statements in a block. */
block ::= '${' stmt* '$}'

/* Variable symbols declaration. */
variable-stmt ::= '$v' variable+ '$.'

/* Disjoint variables. Simple disjoint statements have
   2 variables, i.e., "variable*" is empty for them. */
disjoint-stmt ::= '$d' variable variable variable* '$.'

hypothesis-stmt ::= floating-stmt | essential-stmt

/* Floating (variable-type) hypothesis. */
floating-stmt ::= LABEL '$f' typecode variable '$.'

/* Essential (logical) hypothesis. */
essential-stmt ::= LABEL '$e' typecode MATH-SYMBOL* '$.'

assert-stmt ::= axiom-stmt | provable-stmt

/* Axiomatic assertion. */
axiom-stmt ::= LABEL '$a' typecode MATH-SYMBOL* '$.'

/* Provable assertion. */
provable-stmt ::= LABEL '$p' typecode MATH-SYMBOL*
  '$=' proof '$.'

/* A proof. Proofs may be interspersed by comments.
   If '?' is in a proof it's an "incomplete" proof. */
proof ::= uncompressed-proof | compressed-proof
uncompressed-proof ::= (LABEL | '?')+
compressed-proof ::= '(' LABEL* ')' COMPRESSED-PROOF-BLOCK+

typecode ::= constant

filename ::= MATH-SYMBOL /* No whitespace or '$' */
constant ::= MATH-SYMBOL
variable ::= MATH-SYMBOL
\end{verbatim}

\needspace{2\baselineskip}
A \texttt{frame} is a sequence of 0 or more
\texttt{disjoint-{\allowbreak}stmt} and
\texttt{hypotheses-{\allowbreak}stmt} statements
(possibly interleaved with other non-\texttt{assert-stmt} statements)
followed by one \texttt{assert-stmt}.

\needspace{3\baselineskip}
Here are the rules for lexical processing (tokenization) beyond
the constant tokens shown above.
By convention these tokenization rules have upper-case names.
Every token is read for the longest possible length.
Whitespace-separated tokens are read sequentially;
note that the separating whitespace and \texttt{\$(} ... \texttt{\$)}
comments are skipped.

If a token definition uses another token definition, the whole thing
is considered a single token.
A pattern that is only part of a full token has a name beginning
with an underscore (``\_'').
An implementation could tokenize many tokens as a
\texttt{PRINTABLE-SEQUENCE}
and then check if it meets the more specific rule shown here.

Comments do not nest, and both \texttt{\$(} and \texttt{\$)}
have to be surrounded
by at least one whitespace character (\texttt{\_WHITECHAR}).
Technically comments end without consuming the trailing
\texttt{\_WHITECHAR}, but the trailing
\texttt{\_WHITECHAR} gets ignored anyway so we ignore that detail here.
Metamath language processors
are not required to support \texttt{\$)} followed
immediately by a bare end-of-file, because the closing
comment symbol is supposed to be followed by a
\texttt{\_WHITECHAR} such as a newline.

\begin{verbatim}
PRINTABLE-SEQUENCE ::= _PRINTABLE-CHARACTER+

MATH-SYMBOL ::= (_PRINTABLE-CHARACTER - '$')+

/* ASCII non-whitespace printable characters */
_PRINTABLE-CHARACTER ::= [#x21-#x7e]

LABEL ::= ( _LETTER-OR-DIGIT | '.' | '-' | '_' )+

_LETTER-OR-DIGIT ::= [A-Za-z0-9]

COMPRESSED-PROOF-BLOCK ::= ([A-Z] | '?')+

/* Define whitespace between tokens. The -> SKIP
   means that when whitespace is seen, it is
   skipped and we simply read again. */
WHITESPACE ::= (_WHITECHAR+ | _COMMENT) -> SKIP

/* Comments. $( ... $) and do not nest. */
_COMMENT ::= '$(' (_WHITECHAR+ (PRINTABLE-SEQUENCE - '$)'))*
  _WHITECHAR+ '$)' _WHITECHAR

/* Whitespace: (' ' | '\t' | '\r' | '\n' | '\f') */
_WHITECHAR ::= [#x20#x09#x0d#x0a#x0c]
\end{verbatim}
% This EBNF was developed as a collaboration between
% David A. Wheeler\index{Wheeler, David A.},
% Mario Carneiro\index{Carneiro, Mario}, and
% Benoit Jubin\index{Jubin, Benoit}, inspired by a request
% (and a lot of initial work) by Benoit Jubin.
%
% \chapter{Disclaimer and Trademarks}
%
% Information in this document is subject to change without notice and does not
% represent a commitment on the part of Norman Megill.
% \vspace{2ex}
%
% \noindent Norman D. Megill makes no warranties, either express or implied,
% regarding the Metamath computer software package.
%
% \vspace{2ex}
%
% \noindent Any trademarks mentioned in this book are the property of
% their respective owners.  The name ``Metamath'' is a trademark of
% Norman Megill.
%
\cleardoublepage
\phantomsection  % fixes the link anchor
\addcontentsline{toc}{chapter}{\bibname}

\bibliography{metamath}
%\input{metamath.bbl}

\raggedright
\cleardoublepage
\phantomsection % fixes the link anchor
\addcontentsline{toc}{chapter}{\indexname}
%\printindex   ??
\input{metamath.ind}

\end{document}



\end{document}



\end{document}



\end{document}

